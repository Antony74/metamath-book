% metamath.tex - Version of 17-Nov-2014
% SPDX-License-Identifier: CC0-1.0
%
%                              PUBLIC DOMAIN
%
% This file (specifically, the version of this file with the above date)
% has been released into the Public Domain per the
% Creative Commons CC0 1.0 Universal (CC0 1.0) Public Domain Dedication
% https://creativecommons.org/publicdomain/zero/1.0/
%
% The public domain release applies worldwide.  In case this is not
% legally possible, the right is granted to use the work for any purpose,
% without any conditions, unless such conditions are required by law.
%
% Several short, attributed quotations from copyrighted works
% appear in this file under the ``fair use'' provision of Section 107 of
% the United States Copyright Act (Title 17 of the {\em United States
% Code}).  The public-domain status of this file is not applicable to
% those quotations.
%
% Norman Megill - email: nm(at)alum(dot)mit(dot)edu
%
% David A. Wheeler also donates his improvements to this file to the
% public domain.

% This file, 'metamath.tex', is self-contained with everything needed to
% generate the the PDF file 'metamath.pdf' (the _Metamath_ book) on
% standard LaTeX 2e installations.  The auxiliary files are embedded with
% "filecontents" commands.  To generate metamath.pdf file, run these
% commands under Linux or Cygwin in the directory that contains
% 'metamath.tex':
%
%   rm -f realref.sty metamath.bib
%   touch metamath.ind
%   pdflatex metamath
%   pdflatex metamath
%   bibtex metamath
%   makeindex metamath
%   pdflatex metamath
%   pdflatex metamath
%
% The warnings that occur in the initial runs of pdflatex can be ignored.
% For the final run,
%
%   egrep -i 'error|warn' metamath.log
%
% should show exactly these 5 warnings:
%
%   LaTeX Warning: File `realref.sty' already exists on the system.
%   LaTeX Warning: File `metamath.bib' already exists on the system.
%   LaTeX Font Warning: Font shape `OMS/cmtt/m/n' undefined
%   LaTeX Font Warning: Font shape `OMS/cmtt/bx/n' undefined
%   LaTeX Font Warning: Some font shapes were not available, defaults
%       substituted.
%
% Search for "Uncomment" below if you want to suppress hyperlink boxes
% in the PDF output file
%

% --------------------------- Start of realref.sty -----------------------------
\begin{filecontents}{realref.sty}
% Save the following as realref.sty.
% You can then use it with \usepackage{realref}
%
% This has \pageref jumping to the page on which the ref appears,
% \ref jumping to the point of the anchor, and \sectionref
% jumping to the start of section.
%
% Author:  Anthony Williams
%          Software Engineer
%          Nortel Networks Optical Components Ltd
% Date:    9 Nov 2001 (posted to comp.text.tex)
%
% The following declaration was made by Anthony Williams on
% 24 Jul 2006 (private email to Norman Megill):
%
%   ``I hereby donate the code for realref.sty posted on the
%   comp.text.tex newsgroup on 9th November 2001, accessible from
%   http://groups.google.com/group/comp.text.tex/msg/5a0e1cc13ea7fbb2
%   to the public domain.''
%
\ProvidesPackage{realref}
\RequirePackage[plainpages=false,pdfpagelabels=true]{hyperref}
\def\realref@anchorname{}
\AtBeginDocument{%
% ensure every label is a possible hyperlink target
\let\realref@oldrefstepcounter\refstepcounter%
\DeclareRobustCommand{\refstepcounter}[1]{\realref@oldrefstepcounter{#1}
\edef\realref@anchorname{\string #1.\@currentlabel}%
}%
\let\realref@oldlabel\label%
\DeclareRobustCommand{\label}[1]{\realref@oldlabel{#1}\hypertarget{#1}{}%
\@bsphack\protected@write\@auxout{}{%
    \string\expandafter\gdef\protect\csname
    page@num.#1\string\endcsname{\thepage}%
    \string\expandafter\gdef\protect\csname
    ref@num.#1\string\endcsname{\@currentlabel}%
    \string\expandafter\gdef\protect\csname
    sectionref@name.#1\string\endcsname{\realref@anchorname}%
}\@esphack}%
\DeclareRobustCommand\pageref[1]{{\edef\a{\csname
            page@num.#1\endcsname}\expandafter\hyperlink{page.\a}{\a}}}%
\DeclareRobustCommand\ref[1]{{\edef\a{\csname
            ref@num.#1\endcsname}\hyperlink{#1}{\a}}}%
\DeclareRobustCommand\sectionref[1]{{\edef\a{\csname
            ref@num.#1\endcsname}\edef\b{\csname
            sectionref@name.#1\endcsname}\hyperlink{\b}{\a}}}%
}
\end{filecontents}
% ---------------------------- End of realref.sty ------------------------------

% --------------------------- Start of metamath.bib -----------------------------
\begin{filecontents}{metamath.bib}
@book{Albers, editor = "Donald J. Albers and G. L. Alexanderson",
  title = "Mathematical People",
  publisher = "Contemporary Books, Inc.",
  address = "Chicago",
  note = "[QA28.M37]",
  year = 1985 }
@book{Anderson, author = "Alan Ross Anderson and Nuel D. Belnap",
  title = "Entailment",
  publisher = "Princeton University Press",
  address = "Princeton",
  volume = 1,
  note = "[QA9.A634 1975 v.1]",
  year = 1975}
@book{Barrow, author = "John D. Barrow",
  title = "Theories of Everything:  The Quest for Ultimate Explanation",
  publisher = "Oxford University Press",
  address = "Oxford",
  note = "[Q175.B225]",
  year = 1991 }
@book{Behnke,
  editor = "H. Behnke and F. Backmann and K. Fladt and W. S{\"{u}}ss",
  title = "Fundamentals of Mathematics",
  volume = "I",
  publisher = "The MIT Press",
  address = "Cambridge, Massachusetts",
  note = "[QA37.2.B413]",
  year = 1974 }
@book{Bell, author = "J. L. Bell and M. Machover",
  title = "A Course in Mathematical Logic",
  publisher = "North-Holland",
  address = "Amsterdam",
  note = "[QA9.B3953]",
  year = 1977 }
@inproceedings{Blass, author = "Andrea Blass",
  title = "The Interaction Between Category Theory and Set Theory",
  pages = "5--29",
  booktitle = "Mathematical Applications of Category Theory (Proceedings
     of the Special Session on Mathematical Applications
     Category Theory, 89th Annual Meeting of the American Mathematical
     Society, held in Denver, Colorado January 5--9, 1983)",
  editor = "John Walter Gray",
  year = 1983,
  note = "[QA169.A47 1983]",
  publisher = "American Mathematical Society",
  address = "Providence, Rhode Island"}
@proceedings{Bledsoe, editor = "W. W. Bledsoe and D. W. Loveland",
  title = "Automated Theorem Proving:  After 25 Years (Proceedings
     of the Special Session on Automatic Theorem Proving,
     89th Annual Meeting of the American Mathematical
     Society, held in Denver, Colorado January 5--9, 1983)",
  year = 1983,
  note = "[QA76.9.A96.S64 1983]",
  publisher = "American Mathematical Society",
  address = "Providence, Rhode Island" }
@book{Boolos, author = "George S. Boolos and Richard C. Jeffrey",
  title = "Computability and Log\-ic",
  publisher = "Cambridge University Press",
  edition = "third",
  address = "Cambridge",
  note = "[QA9.59.B66 1989]",
  year = 1989 }
@book{Campbell, author = "John Campbell",
  title = "Programmer's Progress",
  publisher = "White Star Software",
  address = "Box 51623, Palo Alto, CA 94303",
  year = 1991 }
@article{DBLP:journals/corr/Carneiro14,
  author    = {Mario Carneiro},
  title     = {Conversion of {HOL} Light proofs into Metamath},
  journal   = {CoRR},
  volume    = {abs/1412.8091},
  year      = {2014},
  url       = {http://arxiv.org/abs/1412.8091},
  archivePrefix = {arXiv},
  eprint    = {1412.8091},
  timestamp = {Mon, 13 Aug 2018 16:47:05 +0200},
  biburl    = {https://dblp.org/rec/bib/journals/corr/Carneiro14},
  bibsource = {dblp computer science bibliography, https://dblp.org}
}
@inproceedings{Chou, author = "Shang-Ching Chou",
  title = "Proving Elementary Geometry Theorems Using {W}u's Algorithm",
  pages = "243--286",
  booktitle = "Automated Theorem Proving:  After 25 Years (Proceedings
     of the Special Session on Automatic Theorem Proving,
     89th Annual Meeting of the American Mathematical
     Society, held in Denver, Colorado January 5--9, 1983)",
  editor = "W. W. Bledsoe and D. W. Loveland",
  year = 1983,
  note = "[QA76.9.A96.S64 1983]",
  publisher = "American Mathematical Society",
  address = "Providence, Rhode Island" }
@incollection{Courant, author = "Richard Courant and Herbert Robbins",
  title = "Topology",
  pages = "573--590",
  booktitle = "The World of Mathematics, Volume One",
  editor = "James R. Newman",
  publisher = "Simon and Schuster",
  address = "New York",
  note = "[QA3.W67 1988]",
  year = 1956 }
@book{Curry, author = "Haskell B. Curry",
  title = "Foundations of Mathematical Logic",
  publisher = "Dover Publications, Inc.",
  address = "New York",
  note = "[QA9.C976 1977]",
  year = 1977 }
@book{Davis, author = "Philip J. Davis and Reuben Hersh",
  title = "The Mathematical Experience",
  publisher = "Birkh{\"{a}}user Boston",
  address = "Boston",
  note = "[QA8.4.D37 1982]",
  year = 1981 }
@incollection{deMillo,
  author = "Richard de Millo and Richard Lipton and Alan Perlis",
  title = "Social Processes and Proofs of Theorems and Programs",
  pages = "267--285",
  booktitle = "New Directions in the Philosophy of Mathematics",
  editor = "Thomas Tymoczko",
  publisher = "Birkh{\"{a}}user Boston, Inc.",
  address = "Boston",
  note = "[QA8.6.N48 1986]",
  year = 1986 }
@book{Edwards, author = "Robert E. Edwards",
  title = "A Formal Background to Mathematics",
  publisher = "Springer-Verlag",
  address = "New York",
  note = "[QA37.2.E38 v.1a]",
  year = 1979 }
@book{Enderton, author = "Herbert B. Enderton",
  title = "Elements of Set Theory",
  publisher = "Academic Press, Inc.",
  address = "San Diego",
  note = "[QA248.E5]",
  year = 1977 }
@book{Goodstein, author = "R. L. Goodstein",
  title = "Development of Mathematical Logic",
  publisher = "Springer-Verlag New York Inc.",
  address = "New York",
  note = "[QA9.G6554]",
  year = 1971 }
@book{Guillen, author = "Michael Guillen",
  title = "Bridges to Infinity",
  publisher = "Jeremy P. Tarcher, Inc.",
  address = "Los Angeles",
  note = "[QA93.G8]",
  year = 1983 }
@book{Hamilton, author = "Alan G. Hamilton",
  title = "Logic for Mathematicians",
  edition = "revised",
  publisher = "Cambridge University Press",
  address = "Cambridge",
  note = "[QA9.H298]",
  year = 1988 }
@unpublished{Harrison, author = "John Robert Harrison",
  title = "Metatheory and Reflection in Theorem Proving:
    A Survey and Critique",
  note = "Technical Report
    CRC-053.
    SRI Cambridge,
    Millers Yard, Cambridge, UK,
    1995.
    Available on the Web as
{\verb+http:+}\-{\verb+//www.cl.cam.ac.uk/users/jrh/papers/reflect.html+}"}
@TECHREPORT{Harrison-thesis,
        author          = "John Robert Harrison",
        title           = "Theorem Proving with the Real Numbers",
        institution   = "University of Cambridge Computer
                         Lab\-o\-ra\-to\-ry",
        address         = "New Museums Site, Pembroke Street, Cambridge,
                           CB2 3QG, UK",
        year            = 1996,
        number          = 408,
        type            = "Technical Report",
        note            = "Author's PhD thesis,
   available on the Web at
{\verb+http:+}\-{\verb+//www.cl.cam.ac.uk+}\-{\verb+/users+}\-{\verb+/jrh+}%
\-{\verb+/papers+}\-{\verb+/thesis.html+}"}
@book{Herrlich, author = "Horst Herrlich and George E. Strecker",
  title = "Category Theory:  An Introduction",
  publisher = "Allyn and Bacon Inc.",
  address = "Boston",
  note = "[QA169.H567]",
  year = 1973 }
@article{Hindley, author = "J. Roger Hindley and David Meredith",
  title = "Principal Type-Schemes and Condensed Detachment",
  journal = "The Journal of Symbolic Logic",
  volume = 55,
  year = 1990,
  note = "[QA.J87]",
  pages = "90--105" }
@book{Hofstadter, author = "Douglas R. Hofstadter",
  title = "G{\"{o}}del, Escher, Bach",
  publisher = "Basic Books, Inc.",
  address = "New York",
  note = "[QA9.H63 1980]",
  year = 1979 }
@article{Kalish, author = "D. Kalish and R. Montague",
  title = "On {T}arski's Formalization of Predicate Logic with Identity",
  journal = "Archiv f{\"{u}}r Mathematische Logik und Grundlagenfor\-schung",
  volume = 7,
  year = 1965,
  note = "[QA.A673]",
  pages = "81--101" }
@article{Kalman, author = "J. A. Kalman",
  title = "Condensed Detachment as a Rule of Inference",
  journal = "Studia Logica",
  volume = 42,
  number = 4,
  year = 1983,
  note = "[B18.P6.S933]",
  pages = "443-451" }
@book{Kline, author = "Morris Kline",
  title = "Mathematical Thought from Ancient to Modern Times",
  publisher = "Oxford University Press",
  address = "New York",
  note = "[QA21.K516 1990 v.3]",
  year = 1972 }
@book{Klinel, author = "Morris Kline",
  title = "Mathematics, The Loss of Certainty",
  publisher = "Oxford University Press",
  address = "New York",
  note = "[QA21.K525]",
  year = 1980 }
@book{Kramer, author = "Edna E. Kramer",
  title = "The Nature and Growth of Modern Mathematics",
  publisher = "Princeton University Press",
  address = "Princeton, New Jersey",
  note = "[QA93.K89 1981]",
  year = 1981 }
@book{Landau, author = "Edmund Landau",
  title = "Foundations of Analysis",
  publisher = "Chelsea Publishing Company",
  address = "New York",
  edition = "second",
  note = "[QA241.L2541 1960]",
  year = 1960 }
@article{Leblanc, author = "Hugues Leblanc",
  title = "On {M}eyer and {L}ambert's Quantificational Calculus {FQ}",
  journal = "The Journal of Symbolic Logic",
  volume = 33,
  year = 1968,
  note = "[QA.J87]",
  pages = "275--280" }
@article{Lejewski, author = "Czeslaw Lejewski",
  title = "On Implicational Definitions",
  journal = "Studia Logica",
  volume = 8,
  year = 1958,
  note = "[B18.P6.S933]",
  pages = "189--208" }
@book{Margaris, author = "Angelo Margaris",
  title = "First Order Mathematical Logic",
  publisher = "Blaisdell Publishing Company",
  address = "Waltham, Massachusetts",
  note = "[QA9.M327]",
  year = 1967}
@book{Manin, author = "Yu I. Manin",
  title = "A Course in Mathematical Logic",
  publisher = "Springer-Verlag",
  address = "New York",
  note = "[QA9.M29613]",
  year = "1977" }
@article{Mathias, author = "Adrian R. D. Mathias",
  title = "A Term of Length 4,523,659,424,929",
  journal = "Synthese",
  volume = 133,
  year = 2002,
  note = "[Q.S993]",
  pages = "75--86" }
@article{Megill, author = "Norman D. Megill",
  title = "A Finitely Axiomatized Formalization of Predicate Calculus
     with Equality",
  journal = "Notre Dame Journal of Formal Logic",
  volume = 36,
  year = 1995,
  note = "[QA.N914]",
  pages = "435--453" }
@unpublished{Megillc, author = "Norman D. Megill",
  title = "A Shorter Equivalent of the Axiom of Choice",
  month = "June",
  note = "Unpublished",
  year = 1991 }
@article{MegillBunder, author = "Norman D. Megill and Martin W.
    Bunder",
  title = "Weaker {D}-Complete Logics",
  journal = "Journal of the IGPL",
  volume = 4,
  year = 1996,
  pages = "215--225",
  note = "Available on the Web at
{\verb+http:+}\-{\verb+//www.mpi-sb.mpg.de+}\-{\verb+/igpl+}%
\-{\verb+/Journal+}\-{\verb+/V4-2+}\-{\verb+/#Megill+}"}
}
@book{Mendelson, author = "Elliott Mendelson",
  title = "Introduction to Mathematical Logic",
  edition = "second",
  publisher = "D. Van Nostrand Company, Inc.",
  address = "New York",
  note = "[QA9.M537 1979]",
  year = 1979 }
@article{Meredith, author = "David Meredith",
  title = "In Memoriam {C}arew {A}rthur {M}eredith (1904-1976)",
  journal = "Notre Dame Journal of Formal Logic",
  volume = 18,
  year = 1977,
  note = "[QA.N914]",
  pages = "513--516" }
@article{CAMeredith, author = "C. A. Meredith",
  title = "Single Axioms for the Systems ({C},{N}), ({C},{O}) and ({A},{N})
      of the Two-Valued Propositional Calculus",
  journal = "The Journal of Computing Systems",
  volume = 3,
  year = 1953,
  pages = "155--164" }
@article{Monk, author = "J. Donald Monk",
  title = "Provability With Finitely Many Variables",
  journal = "The Journal of Symbolic Logic",
  volume = 27,
  year = 1971,
  note = "[QA.J87]",
  pages = "353--358" }
@article{Monks, author = "J. Donald Monk",
  title = "Substitutionless Predicate Logic With Identity",
  journal = "Archiv f{\"{u}}r Mathematische Logik und Grundlagenfor\-schung",
  volume = 7,
  year = 1965,
  pages = "103--121" }
  %% Took out this from above to prevent LaTeX underfull warning:
  % note = "[QA.A673]",
@book{Moore, author = "A. W. Moore",
  title = "The Infinite",
  publisher = "Routledge",
  address = "New York",
  note = "[BD411.M59]",
  year = 1989}
@book{Munkres, author = "James R. Munkres",
  title = "Topology: A First Course",
  publisher = "Prentice-Hall, Inc.",
  address = "Englewood Cliffs, New Jersey",
  note = "[QA611.M82]",
  year = 1975}
@article{Nemesszeghy, author = "E. Z. Nemesszeghy and E. A. Nemesszeghy",
  title = "On Strongly Creative Definitions:  A Reply to {V}. {F}. {R}ickey",
  journal = "Logique et Analyse (N.\ S.)",
  year = 1977,
  volume = 20,
  note = "[BC.L832]",
  pages = "111--115" }
@unpublished{Nemeti, author = "N{\'{e}}meti, I.",
  title = "Algebraizations of Quantifier Logics, an Overview",
  note = "Version 11.4, preprint, Mathematical Institute, Budapest,
    1994.  A shortened version without proofs appeared in
    ``Algebraizations of quantifier logics, an introductory overview,''
   {\em Studia Logica}, 50:485--569, 1991 [B18.P6.S933]"}
@article{Pavicic, author = "M. Pavi{\v{c}}i{\'{c}}",
  title = "A New Axiomatization of Unified Quantum Logic",
  journal = "International Journal of Theoretical Physics",
  year = 1992,
  volume = 31,
  note = "[QC.I626]",
  pages = "1753 --1766" }
@book{Penrose, author = "Roger Penrose",
  title = "The Emperor's New Mind",
  publisher = "Oxford University Press",
  address = "New York",
  note = "[Q335.P415]",
  year = 1989 }
@book{PetersonI, author = "Ivars Peterson",
  title = "The Mathematical Tourist",
  publisher = "W. H. Freeman and Company",
  address = "New York",
  note = "[QA93.P475]",
  year = 1988 }
@article{Peterson, author = "Jeremy George Peterson",
  title = "An automatic theorem prover for substitution and detachment systems",
  journal = "Notre Dame Journal of Formal Logic",
  volume = 19,
  year = 1978,
  note = "[QA.N914]",
  pages = "119--122" }
@book{Quine, author = "Willard Van Orman Quine",
  title = "Set Theory and Its Logic",
  edition = "revised",
  publisher = "The Belknap Press of Harvard University Press",
  address = "Cambridge, Massachusetts",
  note = "[QA248.Q7 1969]",
  year = 1969 }
@article{Robinson, author = "J. A. Robinson",
  title = "A Machine-Oriented Logic Based on the Resolution Principle",
  journal = "Journal of the Association for Computing Machinery",
  year = 1965,
  volume = 12,
  pages = "23--41" }
@article{RobinsonT, author = "T. Thacher Robinson",
  title = "Independence of Two Nice Sets of Axioms for the Propositional
    Calculus",
  journal = "The Journal of Symbolic Logic",
  volume = 33,
  year = 1968,
  note = "[QA.J87]",
  pages = "265--270" }
@book{Rucker, author = "Rudy Rucker",
  title = "Infinity and the Mind:  The Science and Philosophy of the
    Infinite",
  publisher = "Bantam Books, Inc.",
  address = "New York",
  note = "[QA9.R79 1982]",
  year = 1982 }
@book{Russell, author = "Bertrand Russell",
  title = "Mysticism and Logic, and Other Essays",
  publisher = "Barnes \& Noble Books",
  address = "Totowa, New Jersey",
  note = "[B1649.R963.M9 1981]",
  year = 1981 }
@article{Russell2, author = "Bertrand Russell",
  title = "Recent Work on the Principles of Mathematics",
  journal = "International Monthly",
  volume = 4,
  year = 1901,
  pages = "84"}
@book{Shoenfield, author = "Joseph R. Shoenfield",
  title = "Mathematical Logic",
  publisher = "Addison-Wesley Publishing Company, Inc.",
  address = "Reading, Massachusetts",
  year = 1967,
  note = "[QA9.S52]" }
@book{Smullyan, author = "Raymond M. Smullyan",
  title = "Theory of Formal Systems",
  publisher = "Princeton University Press",
  address = "Princeton, New Jersey",
  year = 1961,
  note = "[QA248.5.S55]" }
@book{Solow, author = "Daniel Solow",
  title = "How to Read and Do Proofs:  An Introduction to Mathematical
    Thought Process",
  publisher = "John Wiley \& Sons",
  address = "New York",
  year = 1982,
  note = "[QA9.S577]" }
@book{Stark, author = "Harold M. Stark",
  title = "An Introduction to Number Theory",
  publisher = "Markham Publishing Company",
  address = "Chicago",
  note = "[QA241.S72 1978]",
  year = 1970 }
@article{Swart, author = "E. R. Swart",
  title = "The Philosophical Implications of the Four-Color Problem",
  journal = "American Mathematical Monthly",
  year = 1980,
  volume = 87,
  month = "November",
  note = "[QA.A5125]",
  pages = "697--707" }
@book{Szpiro, author = "George G. Szpiro",
  title = "Poincar{\'{e}}'s Prize: The Hundred-Year Quest to Solve One
    of Math's Greatest Puzzles",
  publisher = "Penguin Books Ltd",
  address = "London",
  note = "[QA43.S985 2007]",
  year = 2007}
@book{Takeuti, author = "Gaisi Takeuti and Wilson M. Zaring",
  title = "Introduction to Axiomatic Set Theory",
  edition = "second",
  publisher = "Springer-Verlag New York Inc.",
  address = "New York",
  note = "[QA248.T136 1982]",
  year = 1982}
@inproceedings{Tarski, author = "Alfred Tarski",
  title = "What is Elementary Geometry",
  pages = "16--29",
  booktitle = "The Axiomatic Method, with Special Reference to Geometry and
     Physics (Proceedings of an International Symposium held at the University
     of California, Berkeley, December 26, 1957 --- January 4, 1958)",
  editor = "Leon Henkin and Patrick Suppes and Alfred Tarski",
  year = 1959,
  publisher = "North-Holland Publishing Company",
  address = "Amsterdam"}
@article{Tarski1965, author = "Alfred Tarski",
  title = "A Simplified Formalization of Predicate Logic with Identity",
  journal = "Archiv f{\"{u}}r Mathematische Logik und Grundlagenforschung",
  volume = 7,
  year = 1965,
  note = "[QA.A673]",
  pages = "61--79" }
@book{Tymoczko,
  title = "New Directions in the Philosophy of Mathematics",
  editor = "Thomas Tymoczko",
  publisher = "Birkh{\"{a}}user Boston, Inc.",
  address = "Boston",
  note = "[QA8.6.N48 1986]",
  year = 1986 }
@incollection{Wang,
  author = "Hao Wang",
  title = "Theory and Practice in Mathematics",
  pages = "129--152",
  booktitle = "New Directions in the Philosophy of Mathematics",
  editor = "Thomas Tymoczko",
  publisher = "Birkh{\"{a}}user Boston, Inc.",
  address = "Boston",
  note = "[QA8.6.N48 1986]",
  year = 1986 }
@manual{Webster,
  title = "Webster's New Collegiate Dictionary",
  organization = "G. \& C. Merriam Co.",
  address = "Springfield, Massachusetts",
  note = "[PE1628.W4M4 1977]",
  year = 1977 }
@manual{Whitehead, author = "Alfred North Whitehead",
  title = "An Introduction to Mathematics",
  year = 1911 }
@book{PM, author = "Alfred North Whitehead and Bertrand Russell",
  title = "Principia Mathematica",
  edition = "second",
  publisher = "Cambridge University Press",
  address = "Cambridge",
  year = "1927",
  note = "(3 vols.) [QA9.W592 1927]" }
@article{DBLP:journals/corr/Whalen16,
  author    = {Daniel Whalen},
  title     = {Holophrasm: a neural Automated Theorem Prover for higher-order logic},
  journal   = {CoRR},
  volume    = {abs/1608.02644},
  year      = {2016},
  url       = {http://arxiv.org/abs/1608.02644},
  archivePrefix = {arXiv},
  eprint    = {1608.02644},
  timestamp = {Mon, 13 Aug 2018 16:46:19 +0200},
  biburl    = {https://dblp.org/rec/bib/journals/corr/Whalen16},
  bibsource = {dblp computer science bibliography, https://dblp.org} }
@article{Wiedijk-revisited,
  author = {Freek Wiedijk},
  title = {The QED Manifesto Revisited},
  year = {2007},
  url = {http://mizar.org/trybulec65/8.pdf} }
@book{Wolfram,
  author = "Stephen Wolfram",
  title = "Mathematica:  A System for Doing Mathematics by Computer",
  edition = "second",
  publisher = "Addison-Wesley Publishing Co.",
  address = "Redwood City, California",
  note = "[QA76.95.W65 1991]",
  year = 1991 }
@book{Wos, author = "Larry Wos and Ross Overbeek and Ewing Lusk and Jim Boyle",
  title = "Automated Reasoning:  Introduction and Applications",
  edition = "second",
  publisher = "McGraw-Hill, Inc.",
  address = "New York",
  note = "[QA76.9.A96.A93 1992]",
  year = 1992 }

%
%
%[1] Church, Alonzo, Introduction to Mathematical Logic,
% Volume 1, Princeton University Press, Princeton, N. J., 1956.
%
%[2] Cohen, Paul J., Set Theory and the Continuum Hypothesis,
% W. A. Benjamin, Inc., Reading, Mass., 1966.
%
%[3] Hamilton, Alan G., Logic for Mathematicians, Cambridge
% University Press,
% Cambridge, 1988.

%[6] Kleene, Stephen Cole, Introduction to Metamathematics, D.  Van
% Nostrand Company, Inc., Princeton (1952).

%[13] Tarski, Alfred, "A simplified formalization of predicate
% logic with identity," Archiv fur Mathematische Logik und
% Grundlagenforschung, vol. 7 (1965), pp. 61-79.

%[14] Tarski, Alfred and Steven Givant, A Formalization of Set
% Theory Without Variables, American Mathematical Society Colloquium
% Publications, vol. 41, American Mathematical Society,
% Providence, R. I., 1987.

%[15] Zeman, J. J., Modal Logic, Oxford University Press, Oxford, 1973.
\end{filecontents}
% --------------------------- End of metamath.bib -----------------------------


%Book: Metamath
%Author:  Norman Megill Email:  nm at alum.mit.edu
%Author:  David A. Wheeler Email:  dwheeler at dwheeler.com

% A book template example
% http://www.stsci.edu/ftp/software/tex/bookstuff/book.template

\documentclass[leqno]{book} % LaTeX 2e. 10pt. Use [leqno,12pt] for 12pt
% hyperref 2002/05/27 v6.72r  (couldn't get pagebackref to work)
\usepackage[plainpages=false,pdfpagelabels=true]{hyperref}
\usepackage{breqn}         % automatic equation breaking
\usepackage{microtype}     % microtypography, reduces hyphenation
\usepackage{tabularx}
%
%%%%%%%%%%%%%%%%%%%%%%%%%%%%%%%%%%%%%%%%%%%%%%%%%%%%%%%%%%%%%%%%%%%%%%%%%%%%
% Uncomment the next 3 lines to suppress boxes and colors on the hyperlinks
%%%%%%%%%%%%%%%%%%%%%%%%%%%%%%%%%%%%%%%%%%%%%%%%%%%%%%%%%%%%%%%%%%%%%%%%%%%%
%\hypersetup{
%colorlinks,citecolor=black,filecolor=black,linkcolor=black,urlcolor=black
%}
%
\usepackage{realref}

% Restarting page numbers: try?
%   \printglossary
%   \cleardoublepage
%   \pagenumbering{arabic}
%   \setcounter{page}{1}    ???needed
%   \include{chap1}

% not used:
% \def\R2Lurl#1#2{\mbox{\href{#1}\texttt{#2}}}

\usepackage{amssymb}

% Version 1 of book: margins: t=.4, b=.2, ll=.4, rr=.55
% \usepackage{anysize}
% % \papersize{<height>}{<width>}
% % \marginsize{<left>}{<right>}{<top>}{<bottom>}
% \papersize{9in}{6in}
% % l/r 0.6124-0.6170 works t/b 0.2418-0.3411 = 192pp. 0.2926-03118=exact
% \marginsize{0.7147in}{0.5147in}{0.4012in}{0.2012in}

\usepackage{anysize}
% \papersize{<height>}{<width>}
% \marginsize{<left>}{<right>}{<top>}{<bottom>}
\papersize{9in}{6in}
% l/r 0.85in&0.6431-0.6539 works t/b ?-?
%\marginsize{0.85in}{0.6485in}{0.55in}{0.35in}
\marginsize{0.8in}{0.65in}{0.5in}{0.3in}

% \usepackage[papersize={3.6in,4.8in},hmargin=0.1in,vmargin={0.1in,0.1in}]{geometry}  % page geometry
\usepackage{special-settings}

\raggedbottom
\makeindex

\begin{document}
% Discourage page widows and orphans:
\clubpenalty=300
\widowpenalty=300

%%%%%%% load in AMS fonts %%%%%%% % LaTeX 2.09 - obsolete in LaTeX 2e
%\input{amssym.def}
%\input{amssym.tex}
%\input{c:/texmf/tex/plain/amsfonts/amssym.def}
%\input{c:/texmf/tex/plain/amsfonts/amssym.tex}

\bibliographystyle{plain}
\pagenumbering{roman}
\pagestyle{headings}

\thispagestyle{empty}

\hfill
\vfill

\begin{center}
{\LARGE\bf Metamath} \\
\vspace{1ex}
{\large A Computer Language for Pure Mathematics} \\
\vspace{7ex}
{\large Norman Megill} \\
\vspace{7ex}
with improvements by \\
\vspace{7ex}
{\large David A. Wheeler}
\end{center}

\vfill
\hfill

\newpage
\thispagestyle{empty}

\hfill
\vfill

\begin{center}
$\sim$\ {\sc Public Domain}\ $\sim$

\vspace{2ex}
This book (including its later revisions)
has been released into the Public Domain by Norman Megill per the
Creative Commons CC0 1.0 Universal (CC0 1.0) Public Domain Dedication
(\url{https://creativecommons.org/publicdomain/zero/1.0/}).
David A. Wheeler has done the same.
The public domain release applies worldwide.  In case this is not
legally possible, the right is granted to use the work for any purpose,
without any conditions, unless such conditions are required by law.

\vspace{3ex}
Several short, attributed quotations from copyrighted works
appear in this book under the ``fair use'' provision of Section 107 of
the United States Copyright Act (Title 17 of the {\em United States
Code}).  The public-domain status of this book is not applicable to
those quotations.

\vspace{3ex}
Any trademarks used in this book are the property of their owners.

% QA76.9.L63.M??

% \vspace{1ex}
%
% \vspace{1ex}
% {\small Permission is granted to make and distribute verbatim copies of this
% book
% provided the copyright notice and this
% permission notice are preserved on all copies.}
%
% \vspace{1ex}
% {\small Permission is granted to copy and distribute modified versions of this
% book under the conditions for verbatim copying, provided that the
% entire
% resulting derived work is distributed under the terms of a permission
% notice
% identical to this one.}
%
% \vspace{1ex}
% {\small Permission is granted to copy and distribute translations of this
% book into another language, under the above conditions for modified
% versions,
% except that this permission notice may be stated in a translation
% approved by the
% author.}
%
% \vspace{1ex}
% %{\small   For a copy of the \LaTeX\ source files for this book, contact
% %the author.} \\
% \ \\
% \ \\

\vspace{7ex}
% ISBN: 1-4116-3724-0 \\
ISBN: 978-1-4116-3724-5 \\
{\ } \\
Lulu Press \\
Morrisville, North Carolina\\
USA


\hfill
\vfill

Norman Megill\\ 19 Locke Lane, Lexington, MA 02420 \\
E-mail address: \texttt{nm{\char`\@}alum.mit.edu} \\
\vspace{7ex}
David A. Wheeler \\
E-mail address: \texttt{dwheeler{\char`\@}dwheeler.com} \\
% See notes added at end of Preface for revision history. \\
% For current information on the Metamath software see \\
\vspace{7ex}
\url{http://metamath.org}
\end{center}

% \newpage
% \thispagestyle{empty}
%
% \hfill
% \vfill
%
% \begin{center}
% {\it To my son Robin Dwight Megill}
% \end{center}
%
% \vfill
% \hfill
%
% \newpage

\tableofcontents
%\listoftables

\chapter*{Preface}
\markboth{PREFACE}{PREFACE}
\addcontentsline{toc}{section}{Preface}


% (For current information, see the notes added at the
% end of this preface on p.~\pageref{note2002}.)

\subsubsection{Overview}

Metamath\index{Metamath} is a computer language and an associated computer
program for archiving, verifying, and studying mathematical proofs at a very
detailed level.  The Metamath language incorporates no mathematics per se but
treats all mathematical statements as mere sequences of symbols.  You provide
Metamath with certain special sequences (axioms) that tell it what rules
of inference are allowed.  Metamath is not limited to any specific field of
mathematics.  The Metamath language is simple and robust, with an
almost total absence of hard-wired syntax, and 
I\footnote{Unless otherwise noted, the words
``I,'' ``me,'' and ``my'' refer to Norman Megill, while
``we,'' ``us,'' and ``our'' refer to Norman Megill and David A. Wheeler.}
believe that it
provides about the simplest possible framework that allows essentially all of
mathematics to be expressed with absolute rigor.

% index test
%\newcommand{\nn}[1]{#1n}
%\index{aaa@bbb}
%\index{abc!def}
%\index{abd|see{qqq}}
%\index{abe|nn}
%\index{abf|emph}
%\index{abg|(}
%\index{abg|)}

Using the Metamath language, you can build formal or mathematical
systems\index{formal system}\footnote{A formal or mathematical system consists
of a collection of symbols (such as $2$, $4$, $+$ and $=$), syntax rules that
describe how symbols may be combined to form a legal expression (called a
well-formed formula or {\em wff}, pronounced ``whiff''), some starting wffs
called axioms, and inference rules that describe how theorems may be derived
(proved) from the axioms.  A theorem is a mathematical fact such as $2+2=4$.
Strictly speaking, even an obvious fact such as this must be proved from
axioms to be formally acceptable to a mathematician.}\index{theorem}
\index{axiom}\index{rule}\index{well-formed formula (wff)} that involve
inferences from axioms.  Although a database is provided
that includes a recommended set of axioms for standard mathematics, if you
wish you can supply your own symbols, syntax, axioms, rules, and definitions.

The name ``Metamath'' was chosen to suggest that the language provides a
means for {\em describing} mathematics rather than {\em being} the
mathematics itself.  Actually in some sense any mathematical language is
metamathematical.  Symbols written on paper, or stored in a computer,
are not mathematics itself but rather a way of expressing mathematics.
For example ``7'' and ``VII'' are symbols for denoting the number seven
in Arabic and Roman numerals; neither {\em is} the number seven.

If you are able to understand and write computer programs, you should be able
to follow abstract mathematics with the aid of Metamath.  Used in conjunction
with standard textbooks, Metamath can guide you step-by-step towards an
understanding of abstract mathematics from a very rigorous viewpoint, even if
you have no formal abstract mathematics background.  By using a single,
consistent notation to express proofs, once you grasp its basic concepts
Metamath provides you with the ability to immediately follow and dissect
proofs even in totally unfamiliar areas.

Of course, just being able follow a proof will not necessarily give you an
intuitive familiarity with mathematics.  Memorizing the rules of chess does not
give you the ability to appreciate the game of a master, and knowing how the
notes on a musical score map to piano keys does not give you the ability to
hear in your head how it would sound.  But each of these can be a first step.

Metamath allows you to explore proofs in the sense that you can see the
theorem referenced at any step expanded in as much detail as you want, right
down to the underlying axioms of logic and set theory (in the case of the set
theory database provided).  While Metamath will not replace the higher-level
understanding that can only be acquired through exercises and hard work, being
able to see how gaps in a proof are filled in can give you increased
confidence that can speed up the learning process and save you time when you
get stuck.

The Metamath language breaks down a mathematical proof into its tiniest
possible parts.  These can be pieced together, like interlocking
pieces in a puzzle, only in a way that produces correct and absolutely rigorous
mathematics.

The nature of Metamath\index{Metamath} enforces very precise mathematical
thinking, similar to that involved in writing a computer program.  A crucial
difference, though, is that once a proof is verified (by the Metamath program)
to be correct, it is definitely correct; it can never have a hidden
``bug.''\index{computer program bugs}  After getting used to the kind of rigor
and accuracy provided by Metamath, you might even be tempted to
adopt the attitude that a proof should never be considered correct until it
has been verified by a computer, just as you would not completely trust a
manual calculation until you have verified it on a
calculator.

My goal
for Metamath was a system for describing and verifying
mathematics that is completely universal yet conceptually as simple as
possible.  In approaching mathematics from an axiomatic, formal viewpoint, I
wanted Metamath to be able to handle almost any mathematical system, not
necessarily with ease, but at least in principle and hopefully in practice. I
wanted it to verify proofs with absolute rigor, and for this reason Metamath
is what might be thought of as a ``compile-only'' language rather than an
algorithmic or Turing-machine language (Pascal, C, Prolog, Mathematica,
etc.).  In other words, a database written in the Metamath
language doesn't ``do'' anything; it merely exhibits mathematical knowledge
and permits this knowledge to be verified as being correct.  A program in an
algorithmic language can potentially have hidden bugs\index{computer program
bugs} as well as possibly being hard to understand.  But each token in a
Metamath database must be consistent with the database's earlier
contents according to simple, fixed rules.
If a database is verified
to be correct,\footnote{This includes
verification that a sequential list of proof steps results in the specified
theorem.} then the mathematical content is correct if the
verifier is correct and the axioms are correct.
The verification program could be incorrect, but the verification algorithm
is relatively simple (making it unlikely to be implemented incorrectly
by the Metamath program),
and there are over a dozen Metamath database verifiers
written by different people in different programming languages
(so these different verifiers can act as multiple reviewers of a database).
The most-used Metamath database, the Metamath Proof Explorer
(aka \texttt{set.mm}\index{set theory database (\texttt{set.mm})}%
\index{Metamath Proof Explorer}),
is currently verified by four different Metamath verifiers written by
four different people in four different languages, including the
original Metamath program described in this book.
The only ``bugs'' that can exist are in the statement of the axioms,
for example if the axioms are inconsistent (a famous problem shown to be
unsolvable by G\"{o}del's incompleteness theorem\index{G\"{o}del's
incompleteness theorem}).
However, real mathematical systems have very few axioms, and these can
be carefully studied.
All of this provides extraordinarily high confidence that the verified database
is in fact correct.

The Metamath program
doesn't prove theorems automatically but is designed to verify proofs
that you supply to it.
The underlying Metamath language is completely general and has no built-in,
preconceived notions about your formal system\index{formal system}, its logic
or its syntax.
For constructing proofs, the Metamath program has a Proof Assistant\index{Proof
Assistant} which helps you fill in some of a proof step's details, shows you
what choices you have at any step, and verifies the proof as you build it; but
you are still expected to provide the proof.

There are many other programs that can process or generate information
in the Metamath language, and more continue to be written.
This is in part because the Metamath language itself is very simple
and intentionally easy to automatically process.
Some programs, such as \texttt{mmj2}\index{mmj2}, include a proof assistant
that can automate some steps beyond what the Metamath program can do.
Mario Carneiro has developed an algorithm for converting proofs from
the OpenTheory interchange format, which can be translated to and from
any of the HOL family of proof languages (HOL4, HOL Light, ProofPower,
and Isabelle), into the
Metamath language.\cite{DBLP:journals/corr/Carneiro14}\index{Carneiro, Mario}.
Daniel Whalen has developed Holophrasm, which can automatically
prove many Metamath proofs using
machine learning\index{machine learning}\index{artificial intelligence}
approaches
(including multiple neural networks\index{neural networks})\cite{DBLP:journals/corr/Whalen16}\index{Whalen, Daniel}
However,
a discussion of these other programs is beyond the scope of this book.

Like most computer languages, the Metamath\index{Metamath} language uses the
standard ({\sc ascii}) characters on a computer keyboard, so it cannot
directly represent many of the special symbols that mathematicians use.  A
useful feature of the Metamath program is its ability to convert its notation
into the \LaTeX\ typesetting language.\index{latex@{\LaTeX}}  This feature
lets you convert the {\sc ascii} tokens you've defined into standard
mathematical symbols, so you end up with symbols and formulas you are familiar
with instead of somewhat cryptic {\sc ascii} representations of them.
The Metamath program can also generate HTML\index{HTML}, making it easy
to view results on the web and to see related information by using
hypertext links.

Metamath is probably conceptually different from anything you've seen
before and some aspects may take some getting used to.  This book will
help you decide whether Metamath suits your specific needs.

\subsubsection{Setting Your Expectations}
It is important for you to understand what Metamath\index{Metamath} is and is
not.  As mentioned, the Metamath program
is {\em not} an automated theorem prover but
rather a proof verifier.  Developing a database can be tedious, hard work,
especially if you want to make the proofs as short as possible, but it becomes
easier as you build up a collection of useful theorems.  The purpose of
Metamath is simply to document existing mathematics in an absolutely rigorous,
computer-verifiable way, not to aid directly in the creation of new
mathematics.  It also is not a magic solution for learning abstract
mathematics, although it may be helpful to be able to actually see the implied
rigor behind what you are learning from textbooks, as well as providing hints
to work out proofs that you are stumped on.

As of this writing, a sizable set theory database has been developed to
provide a foundation for many fields of mathematics, but much more work would
be required to develop useful databases for specific fields.

Metamath\index{Metamath} ``knows no math;'' it just provides a framework in
which to express mathematics.  Its language is very small.  You can define two
kinds of symbols, constants\index{constant} and variables\index{variable}.
The only thing Metamath knows how to do is to substitute strings of symbols
for the variables\index{substitution!variable}\index{variable substitution} in
an expression based on instructions you provide it in a proof, subject to
certain constraints you specify for the variables.  Even the decimal
representation of a number is merely a string of certain constants (digits)
which together, in a specific context, correspond to whatever mathematical
object you choose to define for it; unlike other computer languages, there is
no actual number stored inside the computer.  In a proof, you in effect
instruct Metamath what symbol substitutions to make in previous axioms or
theorems and join a sequence of them together to result in the desired
theorem.  This kind of symbol manipulation captures the essence of mathematics
at a preaxiomatic level.

\subsubsection{Metamath and Mathematical Literature}

In advanced mathematical literature, proofs are usually presented in the form
of short outlines that often only an expert can follow.  This is partly out of
a desire for brevity, but it would also be unwise (even if it were practical)
to present proofs in complete formal detail, since the overall picture would
be lost.\index{formal proof}

A solution I envision\label{envision} that would allow mathematics to remain
acceptable to the expert, yet increase its accessibility to non-specialists,
consists of a combination of the traditional short, informal proof in print
accompanied by a complete formal proof stored in a computer database.  In an
analogy with a computer program, the informal proof is like a set of comments
that describe the overall reasoning and content of the proof, whereas the
computer database is like the actual program and provides a means for anyone,
even a non-expert, to follow the proof in as much detail as desired, exploring
it back through layers of theorems (like subroutines that call other
subroutines) all the way back to the axioms of the theory.  In addition, the
computer database would have the advantage of providing absolute assurance
that the proof is correct, since each step can be verified automatically.

There are several other approaches besides Metamath to a project such
as this.  Section~\ref{proofverifiers} discusses some of these.

To us, a noble goal would be a database with hundreds of thousands of
theorems and their computer-verifiable proofs, encompassing a significant
fraction of known mathematics and available for instant access.
These would be fully verified by multiple independently-implemented verifiers,
to provide extremely high confidence that the proofs are completely correct.
The database would allow people to investigate whatever details they were
interested in, so that they could confirm whatever portions they wished.
Whether or not Metamath is an appropriate choice remains to be seen, but in
principle we believe it is sufficient.

\subsubsection{Formalism}

Over the past fifty years, a group of French mathematicians working
collectively under the pseudonym of Bourbaki\index{Bourbaki, Nicolas} have
co-authored a series of monographs that attempt to rigorously and
consistently formalize large bodies of mathematics from foundations.  On the
one hand, certainly such an effort has its merits; on the other hand, the
Bourbaki project has been criticized for its ``scholasticism'' and
``hyperaxiomatics'' that hide the intuitive steps that lead to the results
\cite[p.~191]{Barrow}\index{Barrow, John D.}.

Metamath unabashedly carries this philosophy to its extreme and no doubt is
subject to the same kind of criticism.  Nonetheless I think that in
conjunction with conventional approaches to mathematics Metamath can serve a
useful purpose.  The Bourbaki approach is essentially pedagogic, requiring the
reader to become intimately familiar with each detail in a very large
hierarchy before he or she can proceed to the next step.  The difference with
Metamath is that the ``reader'' (user) knows that all details are contained in
its computer database, available as needed; it does not demand that the user
know everything but conveniently makes available those portions that are of
interest.  As the body of all mathematical knowledge grows larger and larger,
no one individual can have a thorough grasp of its entirety.  Metamath
can finalize and put to rest any questions about the validity of any part of it
and can make any part of it accessible, in principle, to a non-specialist.

\subsubsection{A Personal Note}
Why did I develop Metamath\index{Metamath}?  I enjoy abstract mathematics, but
I sometimes get lost in a barrage of definitions and start to lose confidence
that my proofs are correct.  Or I reach a point where I lose sight of how
anything I'm doing relates to the axioms that a theory is based on and am
sometimes suspicious that there may be some overlooked implicit axiom
accidentally introduced along the way (as happened historically with Euclidean
geometry\index{Euclidean geometry}, whose omission of Pasch's
axiom\index{Pasch's axiom} went unnoticed for 2000 years
\cite[p.~160]{Davis}!). I'm also somewhat lazy and wish to avoid the effort
involved in re-verifying the gaps in informal proofs ``left to the reader;'' I
prefer to figure them out just once and not have to go through the same
frustration a year from now when I've forgotten what I did.  Metamath provides
better recovery of my efforts than scraps of paper that I can't
decipher anymore.  But mostly I find very appealing the idea of rigorously
archiving mathematical knowledge in a computer database, providing precision,
certainty, and elimination of human error.

\subsubsection{Note on Bibliography and Index}

The Bibliography usually includes the Library of Congress classification
for a work to make it easier for you to find it in on a university
library shelf.  The Index has author references to pages where their works
are cited, even though the authors' names may not appear on those pages.

\subsubsection{Acknowledgments}

Acknowledgments are first due to my wife, Deborah (who passed away on
September 4, 1998), for critiquing the manu\-script but most of all for
her patience and support.  I also wish to thank Joe Wright, Richard
Becker, Clarke Evans, Buddha Buck, and Jeremy Henty for helpful
comments.  Any errors, omissions, and other shortcomings are of course
my responsibility.

\subsubsection{Note Added June 22, 2005}\label{note2002}

The original, unpublished version of this book was written in 1997 and
distributed via the web.  The present edition has been updated to
reflect the current Metamath program and databases, as well as more
current {\sc url}s for Internet sites.  Thanks to Josh
Purinton\index{Purinton, Josh}, One Hand
Clapping, Mel L.\ O'Cat, and Roy F. Longton for pointing out
typographical and other errors.  I have also benefitted from numerous
discussions with Raph Levien\index{Levien, Raph}, who has extended
Metamath's philosophy of rigor to result in his {\em
Ghilbert}\index{Ghilbert} proof language (\url{http://ghilbert.org}).

Robert (Bob) Solovay\index{Solovay, Robert} communicated a new result of
A.~R.~D.~Mathias on the system of Bourbaki, and the text has been
updated accordingly (p.~\pageref{bourbaki}).

Bob also pointed out a clarification of the literature regarding
category theory and inaccessible cardinals\index{category
theory}\index{cardinal, inaccessible} (p.~\pageref{categoryth}),
and a misleading statement was removed from the text.  Specifically,
contrary to a statement in previous editions, it is possible to express
``There is a proper class of inaccessible cardinals'' in the language of
ZFC.  This can be done as follows:  ``For every set $x$ there is an
inaccessible cardinal $\kappa$ such that $\kappa$ is not in $x$.''
Bob writes:\footnote{Private communication, Nov.~30, 2002.}
\begin{quotation}
     This axiom is how Grothendieck presents category theory.  To each
inaccessible cardinal $\kappa$ one associates a Grothendieck universe
\index{Grothendieck, Alexander} $U(\kappa)$.  $U(\kappa)$ consists of
those sets which lie in a transitive set of cardinality less than
$\kappa$.  Instead of the ``category of all groups,'' one works relative
to a universe [considering the category of groups of cardinality less
than $\kappa$].  Now the category whose objects are all categories
``relative'' to the universe $U(\kappa)$'' will be a category not
relative to this universe but to the next universe.

     All of the things category theorists like to do can be done in this
framework.  The only controversial point is whether the Grothen\-dieck
axiom is too strong for the needs of category theorists.  Mac Lane
\index{Mac Lane, Saunders} argues that ``one universe is enough'' and
Feferman\index{Feferman, Solomon} has argued that one can get by with
ordinary ZFC.  I don't find Feferman's arguments persuasive.  Mac Lane
may be right, but when I think about category theory I do it \`{a} la
Grothendieck.

        By the way Mizar\index{Mizar} adds the axiom ``there is a proper
class of inaccessibles'' precisely so as to do category theory.
\end{quotation}

The most current information on the Metamath program and databases can
always be found at \url{http://metamath.org}.


\subsubsection{Note Added June 24, 2006}\label{note2006}

The Metamath spec was restricted slightly to make parsers easier to
write.  See the footnote on p.~\pageref{namespace}.

%\subsubsection{Note Added July 24, 2006}\label{note2006b}
\subsubsection{Note Added March 10, 2007}\label{note2006b}

I am grateful to Anthony Williams\index{Williams, Anthony} for writing
the \LaTeX\ package called {\tt realref.sty} and contributing it to the
public domain.  This package allows the internal hyperlinks in a {\sc
pdf} file to anchor to specific page numbers instead of just section
titles, making the navigation of the {\sc pdf} file for this book much
more pleasant and ``logical.''

A typographical error found by Martin Kiselkov was corrected.
A confusing remark about unification was deleted per suggestion of
Mel O'Cat.

\subsubsection{Note Added May 27, 2009}\label{note2009}

Several typos found by Kim Sparre were corrected.  A note was added that
the Poincar\'{e} conjecture has been proved (p.~\pageref{poincare}).

\subsubsection{Note Added Nov. 17, 2014}\label{note2014}

The statement of the Schr\"{o}der-Bernstein theorem was corrected in
Section~\ref{trust}.  Thanks to Bob Solovay for pointing out the error.

\subsubsection{Note Added May 25, 2016}\label{note2016}

Thanks to Jerry James for correcting 16 typos.

\subsubsection{Note Added January 27, 2019}\label{note2019}

David A. Wheeler made a large number of improvements and updates,
in coordination with Norman Megill.
The predicate calculus axioms were renumbered, and the text makes
it clear that they are based on Tarski's system S2;
the one slight deviation in axiom ax-6 is explained and justified.
The real and complex number axioms were modified to be consistent with
\texttt{set.mm}\index{set theory database (\texttt{set.mm})}%
\index{Metamath Proof Explorer}.
Long-awaited specification changes ``1-8'' were made,
which clarified previously ambiguous points.
Some errors in the text involving \texttt{\$f} and
\texttt{\$d} statements were corrected (the spec was correct, but
the in-book explanations unintentionally contradicted the spec).
We now have a system for automatically generating narrow PDFs,
so that those with smartphones can have easy access to the current
version of this document.
Many minor corrections (too numerous to list here) were also made.

\chapter{Introduction}
\pagenumbering{arabic}

\begin{quotation}
  {\em {\em I.M.:}  No, no.  There's nothing subjective about it!  Everybody
knows what a proof is.  Just read some books, take courses from a competent
mathematician, and you'll catch on.

{\em Student:}  Are you sure?

{\em I.M.:}  Well---it is possible that you won't, if you don't have any
aptitude for it.  That can happen, too.

{\em Student:}  Then {\em you} decide what a proof is, and if I don't learn
to decide in the same way, you decide I don't have any aptitude.

{\em I.M.:}  If not me, then who?}
    \flushright\sc  ``The Ideal Mathematician''
    \index{Davis, Phillip J.}
    \footnote{\cite{Davis}, p.~40.}\\
\end{quotation}

In the past century, brilliant mathematicians have discovered almost
unimaginably profound results that rank among the crowning intellectual
achievements of mankind.  However, there is a sense in which modern abstract
mathematics is behind the times, stuck in an era before computers existed.
While no one disputes the remarkable results that have been achieved,
communicating these results in a precise way to the uninitiated is virtually
impossible.  To describe these results, a terse informal language is used which
despite its elegance is very difficult to learn.  This informal language is not
imprecise, far from it, but rather it often has omitted detail
and symbols with hidden context that are
implicitly understood by an expert but few others.  Extremely complex technical
meanings are associated with innocent-sounding English words such as
``compact'' and ``measurable'' that barely hint at what is actually being
said.  Anyone who does not keep the precise technical meaning constantly in
mind is bound to fail, and acquiring the ability to do this can be achieved
only through much practice and hard work.  Only the few who complete the
painful learning experience can join the small in-group of pure
mathematicians.  The informal language effectively cuts off the true nature of
their knowledge from most everyone else.

Metamath\index{Metamath} makes abstract mathematics more concrete.  It allows
a computer to keep track of the complexity associated with each word or symbol
with absolute rigor.  You can explore this complexity at your leisure, to
whatever degree you desire.  Whether or not you believe that concepts such as
infinity actually ``exist'' outside of the mind, Metamath lets you get to the
foundation for what's really being said.

Metamath also enables completely rigorous and thorough proof verification.
Its language is simple enough so that you
don't have to rely on the authority of experts but can verify the results
yourself, step by step.  If you want to attempt to derive your own results,
Metamath will not let you make a mistake in reasoning.
Even professional mathematicians make mistakes; Metamath makes it possible
to thoroughly verify that proofs are correct.

Metamath\index{Metamath} is a computer language and an associated computer
program for archiving, verifying, and studying mathematical proofs at a very
detailed level.
The Metamath language
describes formal\index{formal system} mathematical
systems and expresses proofs of theorems in those systems.  Such a language
is called a metalanguage\index{metalanguage} by mathematicians.
The Metamath program is a computer program that verifies
proofs expressed in the Metamath language.
The Metamath program does not have the built-in
ability to make logical inferences; it just makes a series of symbol
substitutions according to instructions given to it in a proof
and verifies that the result matches the expected theorem.  It makes logical
inferences based only on rules of logic that are contained in a set of
axioms\index{axiom}, or first principles, that you provide to it as the
starting point for proofs.

The complete specification of the Metamath language is only four pages long
(Section~\ref{spec}, p.~\pageref{spec}).  Its simplicity may at first make you
wonder how it can do much of anything at all.  But in fact the kinds of
symbol manipulations it performs are the ones that are implicitly done in all
mathematical systems at the lowest level.  You can learn it relatively quickly
and have complete confidence in any mathematical proof that it verifies.  On
the other hand, it is powerful and general enough so that virtually any
mathematical theory, from the most basic to the deeply abstract, can be
described with it.

Although in principle Metamath can be used with any
kind of mathematics, it is best suited for abstract or ``pure'' mathematics
that is mostly concerned with theorems and their proofs, as opposed to the
kind of mathematics that deals with the practical manipulation of numbers.
Examples of branches of pure mathematics are logic\index{logic},\footnote{Logic
is the study of statements that are universally true regardless of the objects
being described by the statements.  An example is the statement, ``if $P$
implies $Q$, then either $P$ is false or $Q$ is true.''} set theory\index{set
theory},\footnote{Set theory is the study of general-purpose mathematical objects called
``sets,'' and from it essentially all of mathematics can be derived.  For
example, numbers can be defined as specific sets, and their properties
can be explored using the tools of set theory.} number theory\index{number
theory},\footnote{Number theory deals with the properties of positive and
negative integers (whole numbers).} group theory\index{group
theory},\footnote{Group theory studies the properties of mathematical objects
called groups that obey a simple set of axioms and have properties of symmetry
that make them useful in many other fields.} abstract algebra\index{abstract
algebra},\footnote{Abstract algebra includes group theory and also studies
groups with additional properties that qualify them as ``rings'' and
``fields.''  The set of real numbers is a familiar example of a field.},
analysis\index{analysis} \index{real and complex numbers}\footnote{Analysis is
the study of real and complex numbers.} and
topology\index{topology}.\footnote{One area studied by topology are properties
that remain unchanged when geometrical objects undergo stretching
deformations; for example a doughnut and a coffee cup each have one hole (the
cup's hole is in its handle) and are thus considered topologically
equivalent.  In general, though, topology is the study of abstract
mathematical objects that obey a certain (surprisingly simple) set of axioms.
See, for example, Munkres \cite{Munkres}\index{Munkres, James R.}.} Even in
physics, Metamath could be applied to certain branches that make use of
abstract mathematics, such as quantum logic\index{quantum logic} (used to study
aspects of quantum mechanics\index{quantum mechanics}).

On the other hand, Metamath\index{Metamath} is less suited to applications
that deal primarily with intensive numeric computations.  Metamath does not
have any built-in representation of numbers\index{Metamath!representation of
numbers}; instead, a specific string of symbols (digits) must be syntactically
constructed as part of any proof in which an ordinary number is used.  For
this reason, numbers in Metamath are best limited to specific constants that
arise during the course of a theorem or its proof.  Numbers are only a tiny
part of the world of abstract mathematics.  The exclusion of built-in numbers
was a conscious decision to help achieve Metamath's simplicity, and there are
other software tools if you have different mathematical needs.
If you wish to quickly solve algebraic problems, the computer algebra
programs\index{computer algebra system} {\sc
macsyma}\index{macsyma@{\sc macsyma}}, Mathematica\index{Mathematica}, and
Maple\index{Maple} are specifically suited to handling numbers and
algebra efficiently.
If you wish to simply calculate numeric or matrix expressions easily,
tools such as Octave\index{Octave} may be a better choice.

After learning Metamath's basic statement types, any
tech\-ni\-cal\-ly ori\-ent\-ed person, mathematician or not, can
immediately trace
any theorem proved in the language as far back as he or she wants, all the way
to the axioms on which the theorem is based.  This ability suggests a
non-traditional way of learning about pure mathematics.  Used in conjunction
with traditional methods, Metamath could make pure mathematics accessible to
people who are not sufficiently skilled to figure out the implicit detail in
ordinary textbook proofs.  Once you learn the axioms of a theory, you can have
complete confidence that everything you need to understand a proof you are
studying is all there, at your beck and call, allowing you to focus in on any
proof step you don't understand in as much depth as you need, without worrying
about getting stuck on a step you can't figure out.\footnote{On the other
hand, writing proofs in the Metamath language is challenging, requiring
a degree of rigor far in excess of that normally taught to students.  In a
classroom setting, I doubt that writing Metamath proofs would ever replace
traditional homework exercises involving informal proofs, because the time
needed to work out the details would not allow a course to
cover much material.  For students who have trouble grasping the implied rigor
in traditional material, writing a few simple proofs in the Metamath language
might help clarify fuzzy thought processes.  Although somewhat difficult at
first, it eventually becomes fun to do, like solving a puzzle, because of the
instant feedback provided by the computer.}

Metamath\index{Metamath} is probably unlike anything you have
encountered before.  In this first chapter we will look at the philosophy and
use of computers in mathematics in order to better understand the motivation
behind Metamath.  The material in this chapter is not required in order to use
Metamath.  You may skip it if you are impatient, but I hope you will find it
educational and enjoyable.  If you want to start experimenting with the
Metamath program right away, proceed directly to Chapter~\ref{using}
(p.~\pageref{using}).  To
learn the Metamath language, skim Chapter~\ref{using} then proceed to
Chapter~\ref{languagespec} (p.~\pageref{languagespec}).

\section{Mathematics as a Computer Language}

\begin{quote}
  {\em The study of mathematics is apt to commence in
dis\-ap\-point\-ment.\ldots \\
We are told that by its aid the stars are weighted
and the billions of molecules in a drop of water are counted.  Yet, like the
ghost of Hamlet's father, this great science eludes the efforts of our mental
weapons to grasp it.}
  \flushright\sc  Alfred North Whitehead\footnote{\cite{Whitehead}, ch.\ 1.}\\
\end{quote}\index{Whitehead, Alfred North}

\subsection{Is Mathematics ``User-Friendly''?}

Suppose you have no formal training in abstract mathematics.  But popular
books you've read offer tempting glimpses of this world filled with profound
ideas that have stirred the human spirit.  You are not satisfied with the
informal, watered-down descriptions you've read but feel it is important to
grasp the underlying mathematics itself to understand its true meaning. It's
not practical to go back to school to learn it, though; you don't want to
dedicate years of your life to it.  There are many important things in life,
and you have to set priorities for what's important to you.  What would happen
if you tried to pursue it on your own, in your spare time?

After all, you were able to learn a computer programming language such as
Pascal on your own without too much difficulty, even though you had no formal
training in computers.  You don't claim to be an expert in software design,
but you can write a passable program when necessary to suit your needs.  Even
more important, you know that you can look at anyone else's Pascal program, no
matter how complex, and with enough patience figure out exactly how it works,
even though you are not a specialist.  Pascal allows you do anything that a
computer can do, at least in principle.  Thus you know you have the ability,
in principle, to follow anything that a computer program can do:  you just
have to break it down into small enough pieces.

Here's an imaginary scenario of what might happen if you na\-ive\-ly a\-dopted
this same view of abstract mathematics and tried to pick it up on your own, in
a period of time comparable to, say, learning a computer programming
language.

\subsubsection{A Non-Mathematician's Quest for Truth}

\begin{quote}
  {\em \ldots my daughters have been studying (chemistry) for several
se\-mes\-ters, think they have learned differential and integral calculus in
school, and yet even today don't know why $x\cdot y=y\cdot x$ is true.}
  \flushright\sc  Edmund Landau\footnote{\cite{Landau}, p.~vi.}\\
\end{quote}\index{Landau, Edmund}

\begin{quote}
  {\em Minus times minus is plus,\\
The reason for this we need not discuss.}
  \flushright\sc W.\ H.\ Auden\footnote{As quoted in \cite{Guillen}, p.~64.}\\
\end{quote}\index{Auden, W.\ H.}\index{Guillen, Michael}

We'll suppose you are a technically oriented professional, perhaps an engineer, a
computer programmer, or a physicist, but probably not a mathematician.  You
consider yourself reasonably intelligent.  You did well in school, learning a
variety of methods and techniques in practical mathematics such as calculus and
differential equations.  But rarely did your courses get into anything
resembling modern abstract mathematics, and proofs were something that appeared
only occasionally in your textbooks, a kind of necessary evil that was
supposed to convince you of a certain key result.  Most of your
homework consisted of exercises that gave you practice in the techniques, and
you were hardly ever asked to come up with a proof of your own.

You find yourself curious about advanced, abstract mathematics.  You are
driven by an inner conviction that it is important to understand and
appreciate some of the most profound knowledge discovered by mankind.  But it
seems very hard to learn, something that only certain gifted longhairs can
access and understand.  You are frustrated that it seems forever cut off from
you.

Eventually your curiosity drives you to do something about it.
You set for yourself a goal of ``really'' understanding mathematics:  not just
how to manipulate equations in algebra or calculus according to cookbook
rules, but rather to gain a deep understanding of where those rules come from.
In fact, you're not thinking about this kind of ordinary mathematics at all,
but about a much more abstract, ethereal realm of pure mathematics, where
famous results such as G\"{o}del's incompleteness theorem\index{G\"{o}del's
incompleteness theorem} and Cantor's different kinds of infinities
reside.

You have probably read a number of popular books, with titles like {\em
Infinity and the Mind} \cite{Rucker}\index{Rucker, Rudy}, on topics such as
these.  You found them inspiring but at the same time somewhat
unsatisfactory.  They gave you a general idea of what these results are about,
but if someone asked you to prove them, you wouldn't have the faintest idea of
where to begin.   Sure, you could give the same overall outline that you
learned from the popular books; and in a general sort of way, you do have an
understanding.  But deep down inside, you know that there is a rigor that is
missing, that probably there are many subtle steps and pitfalls along the way,
and ultimately it seems you have to place your trust in the experts in the
field.  You don't like this; you want to be able to verify these results for
yourself.

So where do you go next?  As a first step, you decide to look up some of the
original papers on the theorems you are curious about, or better, obtain some
standard textbooks in the field.  You look up a theorem you want to
understand.  Sure enough, it's there, but it's expressed with strange
terms and odd symbols that mean absolutely nothing to you.  It might as well be written in
a foreign language you've never seen before, whose symbols are totally alien.
You look at the proof, and you haven't the foggiest notion what each step
means, much less how one step follows from another.  Well, obviously you have
a lot to learn if you want to understand this stuff.

You feel that you could probably understand it by
going back to college for another three to six years and getting a math
degree.  But that does not fit in with your career and the other things in
your life and would serve no practical purpose.  You decide to seek a quicker
path.  You figure you'll just trace your way back to the beginning, step by
step, as you would do with a computer program, until you understand it.  But
you quickly find that this is not possible, since you can't even understand
enough to know what you have to trace back to.

Maybe a different approach is in order---maybe you should start at the
beginning and work your way up.  First, you read the introduction to the book
to find out what the prerequisites are.  In a similar fashion, you trace your
way back through two or three more books, finally arriving at one that seems
to start at a beginning:  it lists the axioms of arithmetic.  ``Aha!'' you
naively think, ``This must be the starting point, the source of all mathematical
knowledge.'' Or at least the starting point for mathematics dealing with
numbers; you have to start somewhere and have no idea what the starting point
for other mathematics would be.  But the word ``axioms'' looks promising.  So
you eagerly read along and work through some elementary exercises at the
beginning of the book.  You feel vaguely bothered:  these
don't seem like axioms at all, at least not in the sense that you want to
think of axioms.  Axioms imply a starting point from which everything else can
be built up, according to precise rules specified in the axiom system.  Even
though you can understand the first few proofs in an informal way,
and are able to do some of the
exercises, it's hard to pin down precisely what the
rules are.   Sure, each step seems to follow logically from the others, but
exactly what does that mean?  Is the ``logic'' just a matter of common sense,
something vague that we all understand but can never quite state precisely?

You've spent a number of years, off and on, programming computers, and you
know that in the case of computer languages there is no question of what the
rules are---they are precise and crystal clear.  If you follow them, your
program will work, and if you don't, it won't.  No matter how complex a
program, it can always be broken down into simpler and simpler pieces, until
you can ultimately identify the bits that are moved around to perform a
specific function.  Some programs might require a lot of perseverance to
accomplish this, but if you focus on a specific portion of it, you don't even
necessarily have to know how the rest of it works. Shouldn't there be an
analogy in mathematics?

You decide to apply the ultimate test:  you ask yourself how a computer could
verify or ensure that the steps in these proofs follow from one another.
Certainly mathematics must be at least as precisely defined as a computer
language, if not more so; after all, computer science itself is based on it.
If you can get a computer to verify these proofs, then you should also be
able, in principle, to understand them yourself in a very crystal clear,
precise way.

You're in for a surprise:  you can conceive of no way to convert the
proofs, which are in English, to a form that the computer can understand.
The proofs are filled with phrases such as ``assume there exists a unique
$x$\ldots'' and ``given any $y$, let $z$ be the number such that\ldots''  This
isn't the kind of logic you are used to in computer programming, where
everything, even arithmetic, reduces to Boolean ones and zeroes if you care to
break it down sufficiently.  Even though you think you understand the proofs,
there seems to be some kind of higher reasoning involved rather than precise
rules that define how you manipulate the symbols in the axioms.  Whatever it
is, it just isn't obvious how you would express it to a computer, and the more
you think about it, the more puzzled and confused you get, to the point where
you even wonder whether {\em you} really understand it.  There's a lot more to
these axioms of arithmetic than meets the eye.

Nobody ever talked about this in school in your applied math and engineering
courses.  You just learned the rules they gave you, not quite understanding
how or why they worked, sometimes vaguely suspicious or uncertain of them, and
through homework problems and osmosis learned how to present solutions that
satisfied the instructor and earned you an ``A.''  Rarely did you actually
``prove'' anything in a rigorous way, and the math majors who did do stuff
like that seemed to be in a different world.

Of course, there are computer algebra programs that can do mathematics, and
rather impressively.  They can instantly solve the integrals that you
struggled with in freshman calculus, and do much, much more.  But when you
look at these programs, what you see is a big collection of algorithms and
techniques that evolved and were added to over time, along with some basic
software that manipulates symbols.  Each algorithm that is built in is the
result of someone's theorem whose proof is omitted; you just have to trust the
person who proved it and the person who programmed it in and hope there are no
bugs.\index{computer program bugs}  Somehow this doesn't seem to be the
essence of mathematics.  Although computer algebra systems can generate
theorems with amazing speed, they can't actually prove a single one of them.

After some puzzlement, you revisit some popular books on what mathematics is
all about.  Somewhere you read that all of mathematics is actually derived
from something called ``set theory.''  This is a little confusing, because no
where in the book that presented the axioms of arithmetic was there any
mention of set theory, or if there was, it seemed to be just a tool that helps
you describe things better---the set of even numbers, that sort of thing.  If
set theory is the basis for all mathematics, then why are additional axioms
needed for arithmetic?

Something is wrong but you're not sure what.  One of your friends is a pure
mathematician.  He knows he is unable to communicate to you what he does for a
living and seems to have little interest in trying.  You do know that for him,
proofs are what mathematics is all about. You ask him what a proof is, and he
essentially tells you that, while of course it's based on logic, really it's
something you learn by doing it over and over until you pick it up.  He refers
you to a book, {\em How to Read and Do Proofs} \cite{Solow}.\index{Solow,
Daniel}  Although this book helps you understand traditional informal proofs,
there is still something missing you can't seem to pin down yet.

You ask your friend how you would go about having a computer verify a proof.
At first he seems puzzled by the question; why would you want to do that?
Then he says it's not something that would make any sense to do, but he's
heard that you'd have to break the proof down into thousands or even millions
of individual steps to do such a thing, because the reasoning involved is at
such a high level of abstraction.  He says that maybe it's something you could
do up to a point, but the computer would be completely impractical once you
get into any meaningful mathematics.  There, the only way you can verify a
proof is by hand, and you can only acquire the ability to do this by
specializing in the field for a couple of years in grad school.  Anyway, he
thinks it all has to do with set theory, although he has never taken a formal
course in set theory but just learned what he needed as he went along.

You are intrigued and amazed.  Apparently a mathematician can grasp as a
single concept something that would take a computer a thousand or a million
steps to verify, and have complete confidence in it.  Each one of these
thousand or million steps must be absolutely correct, or else the whole proof
is meaningless.  If you added a million numbers by hand, would you trust the
result?  How do you really know that all these steps are correct, that there
isn't some subtle pitfall in one of these million steps, like a bug in a
computer program?\index{computer program bugs}  After all, you've read that
famous mathematicians have occasionally made mistakes, and you certainly know
you've made your share on your math homework problems in school.

You recall the analogy with a computer program.  Sure, you can understand what
a large computer program such as a word processor does, as a single high-level
concept or a small set of such concepts, but your ability to understand it in
no way ensures that the program is correct and doesn't have hidden bugs.  Even
if you wrote the program yourself you can't really know this; most large
programs that you've written have had bugs that crop up at some later date, no
matter how careful you tried to be while writing them.

OK, so now it seems the reason you can't figure out how to make a
computer verify proofs is because each step really corresponds to a
million small steps.  Well, you say, a computer can do a million
calculations in a second, so maybe it's still practical to do.  Now the
puzzle becomes how to figure out what the million steps are that each
English-language step corresponds to.  Your mathematician friend hasn't
a clue, but suggests that maybe you would find the answer by studying
set theory.  Actually, your friend thinks you're a little off the wall
for even wondering such a thing.  For him, this is not what mathematics
is all about.

The subject of set theory keeps popping up, so you decide it's
time to look it up.

You decide to start off on a careful footing, so you start reading a couple of
very elementary books on set theory.  A lot of it seems pretty obvious, like
intersections, subsets, and Venn diagrams.  You thumb through one of the
books; nowhere is anything about axioms mentioned. The other book relegates to
an appendix a brief discussion that mentions a set of axioms called
``Zermelo-Fraenkel set theory''\index{Zermelo-Fraenkel set theory} and states
them in English.  You look at them and have no idea what they really mean or
what you can do with them.  The comments in this appendix say that the purpose
of mentioning them is to expose you to the idea, but imply that they are not
necessary for basic understanding and that they are really the subject matter
of advanced treatments where fine points such as a certain paradox (Russell's
paradox\index{Russell's paradox}\footnote{Russell's paradox assumes that there
exists a set $S$ that is a collection of all sets that don't contain
themselves.  Now, either $S$ contains itself or it doesn't.  If it contains
itself, it contradicts its definition.  But if it doesn't contain itself, it
also contradicts its definition.  Russell's paradox is resolved in ZF set
theory by denying that such a set $S$ exists.}) are resolved.  Wait a
minute---shouldn't the axioms be a starting point, not an ending point?  If
there are paradoxes that arise without the axioms, how do you know you won't
stumble across one accidentally when using the informal approach?

And nowhere do these books describe how ``all of mathematics can be
derived from set theory'' which by now you've heard a few times.

You find a more advanced book on set theory.  This one actually lists the
axioms of ZF set theory in plain English on page one.  {\em Now} you think
your quest has ended and you've finally found the source of all mathematical
knowledge; you just have to understand what it means.  Here, in one place, is
the basis for all of mathematics!  You stare at the axioms in awe, puzzle over
them, memorize them, hoping that if you just meditate on them long enough they
will become clear.  Of course, you haven't the slightest idea how the rest of
mathematics is ``derived'' from them; in particular, if these are the axioms
of mathematics, then why do arithmetic, group theory, and so on need their own
axioms?

You start reading this advanced book carefully, pondering the meaning of every
word, because by now you're really determined to get to the bottom of this.
The first thing the book does is explain how the axioms came about, which was
to resolve Russell's paradox.\index{Russell's paradox}  In fact that seems to
be the main purpose of their existence; that they supposedly can be used to
derive all of mathematics seems irrelevant and is not even mentioned.  Well,
you go on.  You hope the book will explain to you clearly, step by step, how
to derive things from the axioms.  After all, this is the starting point of
mathematics, like a book that explains the basics of a computer programming
language.  But something is missing.  You find you can't even understand the
first proof or do the first exercise.  Symbols such as $\exists$ and $\forall$
permeate the page without any mention of where they came from or how to
manipulate them; the author assumes you are totally familiar with them and
doesn't even tell you what they mean.  By now you know that $\exists$ means
``there exists'' and $\forall$ means ``for all,'' but shouldn't the rules for
manipulating these symbols be part of the axioms?  You still have no idea
how you could even describe the axioms to a computer.

Certainly there is something much different here from the technical
literature you're used to reading.  A computer language manual almost
always explains very clearly what all the symbols mean, precisely what
they do, and the rules used for combining them, and you work your way up
from there.

After glancing at four or five other such books, you come to the realization
that there is another whole field of study that you need just to get to the
point at which you can understand the axioms of set theory.  The field is
called ``logic.''  In fact, some of the books did recommend it as a
prerequisite, but it just didn't sink in.  You assumed logic was, well, just
logic, something that a person with common sense intuitively understood.  Why
waste your time reading boring treatises on symbolic logic, the manipulation
of 1's and 0's that computers do, when you already know that?  But this is a
different kind of logic, quite alien to you.  The subject of {\sc nand} and
{\sc nor} gates is not even touched upon or in any case has to do with only a
very small part of this field.

So your quest continues.  Skimming through the first couple of introductory
books, you get a general idea of what logic is about and what quantifiers
(``for all,'' ``there exists'') mean, but you find their examples somewhat
trivial and mildly annoying (``all dogs are animals,'' ``some animals are
dogs,'' and such).  But all you want to know is what the rules are for
manipulating the symbols so you can apply them to set theory.  Some formulas
describing the relationships among quantifiers ($\exists$ and $\forall$) are
listed in tables, along with some verbal reasoning to justify them.
Presumably, if you want to find out if a formula is correct, you go through
this same kind of mental reasoning process, possibly using images of dogs and
animals. Intuitively, the formulas seem to make sense.  But when you ask
yourself, ``What are the rules I need to get a computer to figure out whether
this formula is correct?'', you still don't know.  Certainly you don't ask the
computer to imagine dogs and animals.

You look at some more advanced logic books.  Many of them have an introductory
chapter summarizing set theory, which turns out to be a prerequisite.  You
need logic to understand set theory, but it seems you also need set theory to
understand logic!  These books jump right into proving rather advanced
theorems about logic, without offering the faintest clue about where the logic
came from that allows them to prove these theorems.

Luckily, you come across an elementary book of logic that, halfway through,
after the usual truth tables and metaphors, presents in a clear, precise way
what you've been looking for all along: the axioms!  They're divided into
propositional calculus (also called sentential logic) and predicate calculus
(also called first-order logic),\index{first-order logic} with rules so simple
and crystal clear that now you can finally program a computer to understand
them.  Indeed, they're no harder than learning how to play a game of chess.
As far as what you seem to need is concerned, the whole book could have been
written in five pages!

{\em Now} you think you've found the ultimate source of mathematical
truth.  So---the axioms of mathematics consist of these axioms of logic,
together with the axioms of ZF set theory. (By now you've also been able to
figure out how to translate the ZF axioms from English into the
actual symbols of logic which you can now manipulate according to
precise, easy-to-understand rules.)

Of course, you still don't understand how ``all of mathematics can be
derived from set theory,'' but maybe this will reveal itself in due
course.

You eagerly set out to program the axioms and rules into a computer and start
to look at the theorems you will have to prove as the logic is developed.  All
sorts of important theorems start popping up:  the deduction
theorem,\index{deduction theorem} the substitution theorem,\index{substitution
theorem} the completeness theorem of propositional calculus,\index{first-order
logic!completeness} the completeness theorem of predicate calculus.  Uh-oh,
there seems to be trouble.  They all get harder and harder, and not one of
them can be derived with the axioms and rules of logic you've just been
handed.  Instead, they all require ``metalogic'' for their proofs, a kind of
mixture of logic and set theory that allows you to prove things {\em about}
the axioms and theorems of logic rather than {\em with} them.

You plow ahead anyway.  A month later, you've spent much of your
free time getting the computer to verify proofs in propositional calculus.
You've programmed in the axioms, but you've also had to program in the
deduction theorem, the substitution theorem, and the completeness theorem of
propositional calculus, which by now you've resigned yourself to treating as
rather complex additional axioms, since they can't be proved from the axioms
you were given.  You can now get the computer to verify and even generate
complete, rigorous, formal proofs\index{formal proof}.  Never mind that they
may have 100,000 steps---at least now you can have complete, absolute
confidence in them.  Unfortunately, the only theorems you have proved are
pretty trivial and you can easily verify them in a few minutes with truth
tables, if not by inspection.

It looks like your mathematician friend was right.  Getting the computer to do
serious mathematics with this kind of rigor seems almost hopeless.  Even
worse, it seems that the further along you get, the more ``axioms'' you have
to add, as each new theorem seems to involve additional ``metamathematical''
reasoning that hasn't been formalized, and none of it can be derived from the
axioms of logic.  Not only do the proofs keep growing exponentially as you get
further along, but the program to verify them keeps getting bigger and bigger
as you program in more ``metatheorems.''\index{metatheorem}\footnote{A
metatheorem is usually a statement that is too general to be directly provable
in a theory.  For example, ``if $n_1$, $n_2$, and $n_3$ are integers, then
$n_1+n_2+n_3$ is an integer'' is a theorem of number theory.  But ``for any
integer $k > 1$, if $n_1, \ldots, n_k$ are integers, then $n_1+\ldots +n_k$ is
an integer'' is a metatheorem, in other words a family of theorems, one for
every $k$.  The reason it is not a theorem is that the general sum $n_1+\ldots
+n_k$ (as a function of $k$) is not an operation that can be defined directly
in number theory.} The bugs\index{computer program bugs} that have cropped up
so far have already made you start to lose faith in the rigor you seem to have
achieved, and you know it's just going to get worse as your program gets larger.

\begin{center}
***
\end{center}

\subsection{Mathematics and the Non-Specialist}

\begin{quote}
  {\em A real proof is not checkable by a machine, or even by any mathematician
not privy to the gestalt, the mode of thought of the particular field of
mathematics in which the proof is located.}
  \flushright\sc  Davis and Hersh\index{Davis, Phillip J.}
  \footnote{\cite{Davis}, p.~354.}\\
\end{quote}

The bulk of abstract or theoretical mathematics is ordinarily outside
the reach of anyone but a few specialists in each field who have completed
the necessary difficult internship in order to enter its coterie.  The
typical intelligent layperson has no reasonable hope of understanding much of
it, nor even the specialist mathematician of understanding other fields.  It
is like a foreign language that has no dictionary to look up the translation;
the only way you can learn it is by living in the country for a few years.  It
is argued that the effort involved in learning a specialty is a necessary
process for acquiring a deep understanding.  Of course, this is almost certainly
true if one is to make significant contributions to a field; in particular,
``doing'' proofs is probably the most important part of a mathematician's
training.  But is it also necessary to deny outsiders access to it?  Is it
necessary that abstract mathematics be so hard for a layperson to grasp?

A computer normally is of no help whatsoever.  Most published proofs are
actually just series of hints written in an informal style that requires
considerable knowledge of the field to understand.  These are the ``real
proofs'' referred to by Davis and Hersh.\index{informal proof}  There is an
implicit understanding that, in principle, such a proof could be converted to
a complete formal proof\index{formal proof}.  However, it is said that no one
would ever attempt such a conversion, even if they could, because that would
presumably require millions of steps (Section~\ref{dream}).  Unfortunately the
informal style automatically excludes the understanding of the proof
by anyone who hasn't gone through the necessary apprenticeship. The
best that the intelligent layperson can do is to read popular books about deep
and famous results; while this can be helpful, it can also be misleading, and
the lack of detail usually leaves the reader with no ability whatsoever to
explore any aspect of the field being described.

The statements of theorems often use sophisticated notation that makes them
inaccessible to the non-specialist.  For a non-specialist who wants to achieve
a deeper understanding of a proof, the process of tracing definitions and
lemmas back through their hierarchy\index{hierarchy} quickly becomes confusing
and discouraging.  Textbooks are usually written to train mathematicians or to
communicate to people who are already mathematicians, and large gaps in proofs
are often left as exercises to the reader who is left at an impasse if he or
she becomes stuck.

I believe that eventually computers will enable non-specialists and even
intelligent laypersons to follow almost any mathematical proof in any field.
Metamath is an attempt in that direction.  If all of mathematics were as
easily accessible as a computer programming language, I could envision
computer programmers and hobbyists who otherwise lack mathematical
sophistication exploring and being amazed by the world of theorems and proofs
in obscure specialties, perhaps even coming up with results of their own.  A
tremendous advantage would be that anyone could experiment with conjectures in
any field---the computer would offer instant feedback as to whether
an inference step was correct.

Mathematicians sometimes have to put up with the annoyance of
cranks\index{cranks} who lack a fundamental understanding of mathematics but
insist that their ``proofs'' of, say, Fermat's Last Theorem\index{Fermat's
Last Theorem} be taken seriously.  I think part of the problem is that these
people are misled by informal mathematical language, treating it as if they
were reading ordinary expository English and failing to appreciate the
implicit underlying rigor.  Such cranks are rare in the field of computers,
because computer languages are much more explicit, and ultimately the proof is
in whether a computer program works or not.  With easily accessible
computer-based abstract mathematics, a mathematician could say to a crank,
``don't bother me until you've demonstrated your claim on the computer!''

% 22-May-04 nm
% Attempt to move De Millo quote so it doesn't separate from attribution
% CHANGE THIS NUMBER (AND ELIMINATE IF POSSIBLE) WHEN ABOVE TEXT CHANGES
\vspace{-0.5em}

\subsection{An Impossible Dream?}\label{dream}

\begin{quote}
  {\em Even quite basic theorems would demand almost unbelievably vast
  books to display their proofs.}
    \flushright\sc  Robert E. Edwards\footnote{\cite{Edwards}, p.~68.}\\
\end{quote}\index{Edwards, Robert E.}

\begin{quote}
  {\em Oh, of course no one ever really {\em does} it.  It would take
  forever!  You just show that you could do it, that's sufficient.}
    \flushright\sc  ``The Ideal Mathematician''
    \index{Davis, Phillip J.}\footnote{\cite{Davis},
p.~40.}\\
\end{quote}

\begin{quote}
  {\em There is a theorem in the primitive notation of set theory that
  corresponds to the arithmetic theorem `$1000+2000=3000$'.  The formula
  would be forbiddingly long\ldots even if [one] knows the definitions
  and is asked to simplify the long formula according to them, chances are
  he will make errors and arrive at some incorrect result.}
    \flushright\sc  Hao Wang\footnote{\cite{Wang}, p.~140.}\\
\end{quote}\index{Wang, Hao}

% 22-May-04 nm
% Attempt to move De Millo quote so it doesn't separate from attribution
% CHANGE THIS NUMBER (AND ELIMINATE IF POSSIBLE) WHEN ABOVE TEXT CHANGES
\vspace{-0.5em}

\begin{quote}
  {\em The {\em Principia Mathematica} was the crowning achievement of the
  formalists.  It was also the deathblow of the formalist view.\ldots\\
  {[Rus\-sell]} failed, in three enormous volumes, to get beyond the elementary
  facts of arithmetic.  He showed what can be done in principle and what
  cannot be done in practice.  If the mathematical process were really
  one of strict, logical progression, we would still be counting our
  fingers.\ldots\\
  One theoretician estimates, for instance, that a demonstration of one of
  Ramanujan's conjectures assuming set theory and elementary analysis would
  take about two thousand pages; the length of a deduction from first principles
  is nearly in\-con\-ceiv\-a\-ble\ldots The probabilists argue that\ldots any
  very long proof can at best be viewed as only probably correct\ldots}
  \flushright\sc Richard de Millo et. al.\footnote{\cite{deMillo}, pp.~269,
  271.}\\
\end{quote}\index{de Millo, Richard}

A number of writers have conveyed the impression that the kind of absolute
rigor provided by Metamath\index{Metamath} is an impossible dream, suggesting
that a complete, formal verification\index{formal proof} of a typical theorem
would take millions of steps in untold volumes of books.  Even if it could be
done, the thinking sometimes goes, all meaning would be lost in such a
monstrous, tedious verification.\index{informal proof}\index{proof length}

These writers assume, however, that in order to achieve the kind of complete
formal verification they desire one must break down a proof into individual
primitive steps that make direct reference to the axioms.  This is
not necessary.  There is no reason not to make use of previously proved
theorems rather than proving them over and over.

Just as important, definitions\index{definition} can be introduced along
the way, allowing very complex formulas to be represented with few
symbols.  Not doing this can lead to absurdly long formulas.  For
example, G\"{o}del's incompleteness theorem\index{G\"{o}del's
incompleteness theorem}, which can be expressed with a small number of
defined symbols, would require about 20,000 primitive symbols to express
it.\index{Boolos, George S.}\footnote{George S.\ Boolos, lecture at
Massachusetts Institute of Technology, spring 1990.} An extreme example
is Bourbaki's\label{bourbaki} language for set theory, which requires
4,523,659,424,929 symbols plus 1,179,618,517,981 disambiguatory links
(lines connecting symbol pairs, usually drawn below or above the
formula) to express the number
``one'' \cite{Mathias}.\index{Mathias, Adrian R. D.}\index{Bourbaki,
Nicolas}
% http://www.dpmms.cam.ac.uk/~ardm/

A hierarchy\index{hierarchy} of theorems and definitions permits an
exponential growth in the formula sizes and primitive proof steps to be
described with only a linear growth in the number of symbols used.  Of course,
this is how ordinary informal mathematics is normally done anyway, but with
Metamath\index{Metamath} it can be done with absolute rigor and precision.

\subsection{Beauty}


\begin{quote}
  {\em No one shall be able to drive us from the paradise that Cantor has
created for us.}
   \flushright\sc  David Hilbert\footnote{As quoted in \cite{Moore}, p.~131.}\\
\end{quote}\index{Hilbert, David}

\begin{quote}
  {\em Mathematics possesses not only truth, but some supreme beauty ---a
  beauty cold and austere, like that of a sculpture.}
    \flushright\sc  Bertrand
    Russell\footnote{\cite{Russell}.}\\
\end{quote}\index{Russell, Bertrand}

\begin{quote}
  {\em Euclid alone has looked on Beauty bare.}
  \flushright\sc Edna Millay\footnote{As quoted in \cite{Davis}, p.~150.}\\
\end{quote}\index{Millay, Edna}

For most people, abstract mathematics is distant, strange, and
incomprehensible.  Many popular books have tried to convey some of the sense
of beauty in famous theorems.  But even an intelligent layperson is left with
only a general idea of what a theorem is about and is hardly given the tools
needed to make use of it.  Traditionally, it is only after years of arduous
study that one can grasp the concepts needed for deep understanding.
Metamath\index{Metamath} allows you to approach the proof of the theorem from
a quite different perspective, peeling apart the formulas and definitions
layer by layer until an entirely different kind of understanding is achieved.
Every step of the proof is there, pieced together with absolute precision and
instantly available for inspection through a microscope with a magnification
as powerful as you desire.

A proof in itself can be considered an object of beauty.  Constructing an
elegant proof is an art.  Once a famous theorem has been proved, often
considerable effort is made to find simpler and more easily understood
proofs.  Creating and communicating elegant proofs is a major concern of
mathematicians.  Metamath is one way of providing a common language for
archiving and preserving this information.

The length of a proof can, to a certain extent, be considered an
objective measure of its ``beauty,'' since shorter proofs are usually
considered more elegant.  In the set theory database
\texttt{set.mm}\index{set theory database (\texttt{set.mm})}%
\index{Metamath Proof Explorer}
provided with Metamath, one goal was to make all proofs as short as possible.

\subsection{Simplicity}

\begin{quote}
  {\em God made man simple; man's complex problems are of his own
  devising.}
    \flushright\sc Eccles. 7:29\footnote{Jerusalem Bible.}\\
\end{quote}\index{Bible}

\begin{quote}
  {\em God made integers, all else is the work of man.}
    \flushright\sc Leopold Kronecker\footnote{{\em Jahresbericht
	der Deutschen Mathematiker-Vereinigung }, vol. 2, p. 19.}\\
\end{quote}\index{Kronecker, Leopold}

\begin{quote}
  {\em For what is clear and easily comprehended attracts; the
  complicated repels.}
    \flushright\sc David Hilbert\footnote{As quoted in \cite{deMillo},
p.~273.}\\
\end{quote}\index{Hilbert, David}

The Metamath\index{Metamath} language is simple and Spartan.  Metamath treats
all mathematical expressions as simple sequences of symbols, devoid of meaning.
The higher-level or ``metamathematical'' notions underlying Metamath are about
as simple as they could possibly be.  Each individual step in a proof involves
a single basic concept, the substitution of an expression for a variable, so
that in principle almost anyone, whether mathematician or not, can
completely understand how it was arrived at.

In one of its most basic applications, Metamath\index{Metamath} can be used to
develop the foundations of mathematics\index{foundations of mathematics} from
the very beginning.  This is done in the set theory database that is provided
with the Metamath package and is the subject matter
of Chapter~\ref{fol}. Any language (a metalanguage\index{metalanguage})
used to describe mathematics (an object language\index{object language}) must
have a mathematical content of its own, but it is desirable to keep this
content down to a bare minimum, namely that needed to make use of the
inference rules specified by the axioms.  With any metalanguage there is a
``chicken and egg'' problem somewhat like circular reasoning:  you must assume
the validity of the mathematics of the metalanguage in order to prove the
validity of the mathematics of the object language.  The mathematical content
of Metamath itself is quite limited.  Like the rules of a game of chess, the
essential concepts are simple enough so that virtually anyone should be able to
understand them (although that in itself will not let you play like
a master).  The symbols that Metamath manipulates do not in themselves
have any intrinsic meaning.  Your interpretation of the axioms that you supply
to Metamath is what gives them meaning.  Metamath is an attempt to strip down
mathematical thought to its bare essence and show you exactly how the symbols
are manipulated.

Philosophers and logicians, with various motivations, have often thought it
important to study ``weak'' fragments of logic\index{weak logic}
\cite{Anderson}\index{Anderson, Alan Ross} \cite{MegillBunder}\index{Megill,
Norman}\index{Bunder, Martin}, other unconventional systems of logic (such as
``modal'' logic\index{modal logic} \cite[ch.\ 27]{Boolos}\index{Boolos, George
S.}), and quantum logic\index{quantum logic} in physics
\cite{Pavicic}\index{Pavi{\v{c}}i{\'{c}}, M.}.  Metamath\index{Metamath}
provides a framework in which such systems can be expressed, with an absolute
precision that makes all underlying metamathematical assumptions rigorous and
crystal clear.

Some schools of philosophical thought, for example
intuitionism\index{intuitionism} and constructivism\index{constructivism},
demand that the notions underlying any mathematical system be as simple and
concrete as possible.  Metamath should meet the requirements of these
philosophies.  Metamath must be taught the symbols, axioms\index{axiom}, and
rules\index{rule} for a specific theory, from the skeptical (such as
intuitionism\index{intuitionism}\footnote{Intuitionism does not accept the law
of excluded middle (``either something is true or it is not true'').  See
\cite[p.~xi]{Tymoczko}\index{Tymoczko, Thomas} for discussion and references
on this topic.  Consider the theorem, ``There exist irrational numbers $a$ and
$b$ such that $a^b$ is rational.''  An intuitionist would reject the following
proof:  If $\sqrt{2}^{\sqrt{2}}$ is rational, we are done.  Otherwise, let
$a=\sqrt{2}^{\sqrt{2}}$ and $b=\sqrt{2}$. Then $a^b=2$, which is rational.})
to the bold (such as the axiom of choice in set theory\footnote{The axiom of
choice\index{Axiom of Choice} asserts that given any collection of pairwise
disjoint nonempty sets, there exists a set that has exactly one element in
common with each set of the collection.  It is used to prove many important
theorems in standard mathematics.  Some philosophers object to it because it
asserts the existence of a set without specifying what the set contains
\cite[p.~154]{Enderton}\index{Enderton, Herbert B.}.  In one foundation for
mathematics due to Quine\index{Quine, Willard Van Orman}, that has not been
otherwise shown to be inconsistent, the axiom of choice turns out to be false
\cite[p.~23]{Curry}\index{Curry, Haskell B.}.  The \texttt{show
trace{\char`\_}back} command of the Metamath program allows you to find out
whether the axiom of choice, or any other axiom, was assumed by a
proof.}\index{\texttt{show trace{\char`\_}back} command}).

The simplicity of the Metamath language lets the algorithm (computer program)
that verifies the validity of a Metamath proof be straightforward and
robust.  You can have confidence that the theorems it verifies really can be
derived from your axioms.

\subsection{Rigor}

\begin{quote}
  {\em Rigor became a goal with the Greeks\ldots But the efforts to
  pursue rigor to the utmost have led to an impasse in which there is
  no longer any agreement on what it really means.  Mathematics remains
  alive and vital, but only on a pragmatic basis.}
    \flushright\sc  Morris Kline\footnote{\cite{Kline}, p.~1209.}\\
\end{quote}\index{Kline, Morris}

Kline refers to a much deeper kind of rigor than that which we will discuss in
this section.  G\"{o}del's incompleteness theorem\index{G\"{o}del's
incompleteness theorem} showed that it is impossible to achieve absolute rigor
in standard mathematics because we can never prove that mathematics is
consistent (free from contradictions).\index{consistent theory}  If
mathematics is consistent, we will never know it, but must rely on faith.  If
mathematics is inconsistent, the best we can hope for is that some clever
future mathematician will discover the inconsistency.  In this case, the
axioms would probably be revised slightly to eliminate the inconsistency, as
was done in the case of Russell's paradox,\index{Russell's paradox} but the
bulk of mathematics would probably not be affected by such a discovery.
Russell's paradox, for example, did not affect most of the remarkable results
achieved by 19th-century and earlier mathematicians.  It mainly invalidated
some of Gottlob Frege's\index{Frege, Gottlob} work on the foundations of
mathematics in the late 1800's; in fact Frege's work inspired Russell's
discovery.  Despite the paradox, Frege's work contains important concepts that
have significantly influenced modern logic.  Kline's {\em Mathematics, The
Loss of Certainty} \cite{Klinel}\index{Kline, Morris} has an interesting
discussion of this topic.

What {\em can} be achieved with absolute certainty\index{certainty} is the
knowledge that if we assume the axioms are consistent and true, then the
results derived from them are true.  Part of the beauty of mathematics is that
it is the one area of human endeavor where absolute certainty can be achieved
in this sense.  A mathematical truth will remain such for eternity.  However,
our actual knowledge of whether a particular statement is a mathematical truth
is only as certain as the correctness of the proof that establishes it.  If
the proof of a statement is questionable or vague, we can't have absolute
confidence in the truth that the statement claims.

Let us look at some traditional ways of expressing proofs.

Except in the field of formal logic\index{formal logic}, almost all
traditional proofs in mathematics are really not proofs at all, but rather
proof outlines or hints as to how to go about constructing the proof.  Many
gaps\index{gaps in proofs} are left for the reader to fill in. There are
several reasons for this.  First, it is usually assumed in mathematical
literature that the person reading the proof is a mathematician familiar with
the specialty being described, and that the missing steps are obvious to such
a reader or at least that the reader is capable of filling them in.  This
attitude is fine for professional mathematicians in the specialty, but
unfortunately it often has the drawback of cutting off the rest of the world,
including mathematicians in other specialties, from understanding the proof.
We discussed one possible resolution to this on p.~\pageref{envision}.
Second, it is often assumed that a complete formal proof\index{formal proof}
would require countless millions of symbols (Section~\ref{dream}). This might
be true if the proof were to be expressed directly in terms of the axioms of
logic and set theory,\index{set theory} but it is usually not true if we allow
ourselves a hierarchy\index{hierarchy} of definitions and theorems to build
upon, using a notation that allows us to introduce new symbols, definitions,
and theorems in a precisely specified way.

Even in formal logic,\index{formal logic} formal proofs\index{formal proof}
that are considered complete still contain hidden or implicit information.
For example, a ``proof'' is usually defined as a sequence of
wffs,\index{well-formed formula (wff)}\footnote{A {\em wff} or well-formed
formula is a mathematical expression (string of symbols) constructed according
to some precise rules.  A formal mathematical system\index{formal system}
contains (1) the rules for constructing syntactically correct
wffs,\index{syntax rules} (2) a list of starting wffs called
axioms,\index{axiom} and (3) one or more rules prescribing how to derive new
wffs, called theorems\index{theorem}, from the axioms or previously derived
theorems.  An example of such a system is contained in
Metamath's\index{Metamath} set theory database, which defines a formal
system\index{formal system} from which all of standard mathematics can be
derived.  Section~\ref{startf} steps you through a complete example of a formal
system, and you may want to skim it now if you are unfamiliar with the
concept.} each of which is an axiom or follows from a rule applied to previous
wffs in the sequence.  The implicit part of the proof is the algorithm by
which a sequence of symbols is verified to be a valid wff, given the
definition of a wff.  The algorithm in this case is rather simple, but for a
computer to verify the proof,\index{automated proof verification} it must have
the algorithm built into its verification program.\footnote{It is possible, of
course, to specify wff construction syntax outside of the program itself
with a suitable input language (the Metamath language being an example), but
some proof-verification or theorem-proving programs lack the ability to extend
wff syntax in such a fashion.} If one deals exclusively with axioms and
elementary wffs, it is straightforward to implement such an algorithm.  But as
more and more definitions are added to the theory in order to make the
expression of wffs more compact, the algorithm becomes more and more
complicated.  A computer program that implements the algorithm becomes larger
and harder to understand as each definition is introduced, and thus more prone
to bugs.\index{computer program bugs}  The larger the program, the
more suspicious the mathematician may be about
the validity of its algorithms.  This is especially true because
computer programs are inherently hard to follow to begin with, and few people
enjoy verifying them manually in detail.

Metamath\index{Metamath} takes a different approach.  Metamath's ``knowledge''
is limited to the ability to substitute variables for expressions, subject to
some simple constraints.  Once the basic algorithm of Metamath is assumed to
be debugged, and perhaps independently confirmed, it
can be trusted once and for all.  The information that Metamath needs to
``understand'' mathematics is contained entirely in the body of knowledge
presented to Metamath.  Any errors in reasoning can only be errors in the
axioms or definitions contained in this body of knowledge.  As a
``constructive'' language\index{constructive language} Metamath has no
conditional branches or loops like the ones that make computer programs hard
to decipher; instead, the language can only build new sequences of symbols
from earlier sequences  of symbols.

The simplicity of the rules that underlie Metamath not only makes Metamath
easy to learn but also gives Metamath a great deal of flexibility. For
example, Metamath is not limited to describing standard first-order
logic\index{first-order logic}; higher-order logics\index{higher-order logic}
and fragments of logic\index{weak logic} can be described just as easily.
Metamath gives you the freedom to define whatever wff notation you prefer; it
has no built-in conception of the syntax of a wff.\index{well-formed formula
(wff)}  With suitable axioms and definitions, Metamath can even describe and
prove things about itself.\index{Metamath!self-description}  (John
Harrison\index{Harrison, John} discusses the ``reflection''
principle\index{reflection principle} involved in self-descriptive systems in
\cite{Harrison}.)

The flexibility of Metamath requires that its proofs specify a lot of detail,
much more than in an ordinary ``formal'' proof.\index{formal proof}  For
example, in an ordinary formal proof, a single step consists of displaying the
wff that constitutes that step.  In order for a computer program to verify
that the step is acceptable, it first must verify that the symbol sequence
being displayed is an acceptable wff.\index{automated proof verification} Most
proof verifiers have at least basic wff syntax built into their programs.
Metamath has no hard-wired knowledge of what constitutes a wff built into it;
instead every wff must be explicitly constructed based on rules defining wffs
that are present in a database.  Thus a single step in an ordinary formal
proof may be correspond to many steps in a Metamath proof. Despite the larger
number of steps, though, this does not mean that a Metamath proof must be
significantly larger than an ordinary formal proof. The reason is that since
we have constructed the wff from scratch, we know what the wff is, so there is
no reason to display it.  We only need to refer to a sequence of statements
that construct it.  In a sense, the display of the wff in an ordinary formal
proof is an implicit proof of its own validity as a wff; Metamath just makes
the proof explicit. (Section~\ref{proof} describes Metamath's proof notation.)

\section{Computers and Mathematicians}

\begin{quote}
  {\em The computer is important, but not to mathematics.}
    \flushright\sc  Paul Halmos\footnote{As quoted in \cite{Albers}, p.~121.}\\
\end{quote}\index{Halmos, Paul}

Pure mathematicians have traditionally been indifferent to computers, even to
the point of disdain.\index{computers and pure mathematics}  Computer science
itself is sometimes considered to fall in the mundane realm of ``applied''
mathematics, perhaps essential for the real world but intellectually unexciting
to those who seek the deepest truths in mathematics.  Perhaps a reason for this
attitude towards computers is that there is little or no computer software that
meets their needs, and there may be a general feeling that such software could
not even exist.  On the one hand, there are the practical computer algebra
systems, which can perform amazing symbolic manipulations in algebra and
calculus,\index{computer algebra system} yet can't prove the simplest
existence theorem, if the idea of a proof is present at all.  On the other
hand, there are specialized automated theorem provers that technically speaking
may generate correct proofs.\index{automated theorem proving}  But sometimes
their specialized input notation may be cryptic and their output perceived to
be long, inelegant, incomprehensible proofs.    The output
may be viewed with suspicion, since the program that generates it tends to be
very large, and its size increases the potential for bugs\index{computer
program bugs}.  Such a proof may be considered trustworthy only if
independently verified and ``understood'' by a human, but no one wants to
waste their time on such a boring, unrewarding chore.



\subsection{Trusting the Computer}

\begin{quote}
  {\em \ldots I continue to find the quasi-empirical interpretation of
  computer proofs to be the more plausible.\ldots Since not
  everything that claims to be a computer proof can be
  accepted as valid, what are the mathematical criteria for acceptable
  computer proofs?}
    \flushright\sc  Thomas Tymoczko\footnote{\cite{Tymoczko}, p.~245.}\\
\end{quote}\index{Tymoczko, Thomas}

In some cases, computers have been essential tools for proving famous
theorems.  But if a proof is so long and obscure that it can be verified in a
practical way only with a computer, it is vaguely felt to be suspicious.  For
example, proving the famous four-color theorem\index{four-color
theorem}\index{proof length} (``a map needs no more than four colors to
prevent any two adjacent countries from having the same color'') can presently
only be done with the aid of a very complex computer program which originally
required 1200 hours of computer time. There has been considerable debate about
whether such a proof can be trusted and whether such a proof is ``real''
mathematics \cite{Swart}\index{Swart, E. R.}.\index{trusting computers}

However, under normal circumstances even a skeptical mathematician would a have
a great deal of confidence in the result of multiplying two numbers on a pocket
calculator, even though the precise details of what goes on are hidden from its
user.  Even the verification on a supercomputer that a huge number is prime is
trusted, especially if there is independent verification; no one bothers to
debate the philosophical significance of its ``proof,'' even though the actual
proof would be so large that it would be completely impractical to ever write
it down on paper.  It seems that if the algorithm used by the computer is
simple enough to be readily understood, then the computer can be trusted.

Metamath\index{Metamath} adopts this philosophy.  The simplicity of its
language makes it easy to learn, and because of its simplicity one can have
essentially absolute confidence that a proof is correct. All axioms, rules, and
definitions are available for inspection at any time because they are defined
by the user; there are no hidden or built-in rules that may be prone to subtle
bugs\index{computer program bugs}.  The basic algorithm at the heart of
Metamath is simple and fixed, and it can be assumed to be bug-free and robust
with a degree of confidence approaching certainty.
Independently written implementations of the Metamath verifier
can reduce any residual doubt on the part of a skeptic even further;
there are now over a dozen such implementations, written by many people.

\subsection{Trusting the Mathematician}\label{trust}

\begin{quote}
  {\em There is no Algebraist nor Mathematician so expert in his science, as
  to place entire confidence in any truth immediately upon his discovery of it,
  or regard it as any thing, but a mere probability.  Every time he runs over
  his proofs, his confidence encreases; but still more by the approbation of
  his friends; and is rais'd to its utmost perfection by the universal assent
  and applauses of the learned world.}
  \flushright\sc David Hume\footnote{{\em A Treatise of Human Nature}, as
  quoted in \cite{deMillo}, p.~267.}\\
\end{quote}\index{Hume, David}

\begin{quote}
  {\em Stanislaw Ulam estimates that mathematicians publish 200,000 theorems
  every year.  A number of these are subsequently contradicted or otherwise
  disallowed, others are thrown into doubt, and most are ignored.}
  \flushright\sc Richard de Millo et. al.\footnote{\cite{deMillo}, p.~269.}\\
\end{quote}\index{Ulam, Stanislaw}

Whether or not the computer can be trusted, humans  of course will occasionally
err. Only the most memorable proofs get independently verified, and of these
only a handful of truly great ones achieve the status of being ``known''
mathematical truths that are used without giving a second thought to their
correctness.

There are many famous examples of incorrect theorems and proofs in
mathematical literature.\index{errors in proofs}

\begin{itemize}
\item There have been thousands of purported proofs of Fermat's Last
Theorem\index{Fermat's Last Theorem} (``no integer solutions exist to $x^n +
y^n = z^n$ for $n > 2$''), by amateurs, cranks, and well-regarded
mathematicians \cite[p.~5]{Stark}\index{Stark, Harold M}.  Fermat wrote a note
in his copy of Bachet's {\em Diophantus} that he found ``a truly marvelous
proof of this theorem but this margin is too narrow to contain it''
\cite[p.~507]{Kramer}.  A recent, much publicized proof by Yoichi
Miyaoka\index{Miyaoka, Yoichi} was shown to be incorrect ({\em Science News},
April 9, 1988, p.~230).  The theorem was finally proved by Andrew
Wiles\index{Wiles, Andrew} ({\em Science News}, July 3, 1993, p.~5), but it
initially had some gaps and took over a year after its announcement to be
checked thoroughly by experts.  On Oct. 25, 1994, Wiles announced that the last
gap found in his proof had been filled in.
  \item In 1882, M. Pasch discovered that an axiom was omitted from Euclid's
formulation of geometry\index{Euclidean geometry}; without it, the proofs of
important theorems of Euclid are not valid.  Pasch's axiom\index{Pasch's
axiom} states that a line that intersects one side of a triangle must also
intersect another side, provided that it does not touch any of the triangle's
vertices.  The omission of Pasch's axiom went unnoticed for 2000
years \cite[p.~160]{Davis}, in spite of (one presumes) the thousands of
students, instructors, and mathematicians who studied Euclid.
  \item The first published proof of the famous Schr\"{o}der-Bernstein
theorem\index{Schr\"{o}der-Bernstein theorem} in set theory was incorrect
\cite[p.~148]{Enderton}\index{Enderton, Herbert B.}.  This theorem states
that if there exists a 1-to-1 function\footnote{A {\em set}\index{set} is any
collection of objects. A {\em function}\index{function} or {\em
mapping}\index{mapping} is a rule that assigns to each element of one set
(called the function's {\em domain}\index{domain}) an element from another
set.} from set $A$ into set $B$ and vice-versa, then sets $A$ and $B$ have
a 1-to-1 correspondence.  Although it sounds simple and obvious,
the standard proof is quite long and complex.
  \item In the early 1900's, Hilbert\index{Hilbert, David} published a
purported proof of the continuum hypothesis\index{continuum hypothesis}, which
was eventually established as unprovable by Cohen\index{Cohen, Paul} in 1963
\cite[p.~166]{Enderton}.  The continuum hypothesis states that no
infinity\index{infinity} (``transfinite cardinal number'')\index{cardinal,
transfinite} exists whose size (or ``cardinality''\index{cardinality}) is
between the size of the set of integers and the size of the set of real
numbers.  This hypothesis originated with German mathematician Georg
Cantor\index{Cantor, Georg} in the late 1800's, and his inability to prove it
is said to have contributed to mental illness that afflicted him in his later
years.
  \item An incorrect proof of the four-color theorem\index{four-color theorem}
was published by Kempe\index{Kempe, A. B.} in 1879
\cite[p.~582]{Courant}\index{Courant, Richard}; it stood for 11 years before
its flaw was discovered.  This theorem states that any map can be colored
using only four colors, so that no two adjacent countries have the same
color.  In 1976 the theorem was finally proved by the famous computer-assisted
proof of Haken, Appel, and Koch \cite{Swart}\index{Appel, K.}\index{Haken,
W.}\index{Koch, K.}.  Or at least it seems that way.  Mathematician
H.~S.~M.~Coxeter\index{Coxeter, H. S. M.} has doubts \cite[p.~58]{Davis}:  ``I
have a feeling that that is an untidy kind of use of the computers, and the more
you correspond with Haken and Appel, the more shaky you seem to be.''
  \item Many false ``proofs'' of the Poincar\'{e}
conjecture\index{Poincar\'{e} conjecture} have been proposed over the years.
This conjecture states that any object that mathematically behaves like a
three-dimensional sphere is a three-dimensional sphere topologically,
regardless of how it is distorted.  In March 1986, mathematicians Colin
Rourke\index{Rourke, Colin} and Eduardo R\^{e}go\index{R\^{e}go, Eduardo}
caused  a stir in the mathematical community by announcing that they had found
a proof; in November of that year the proof was found to be false \cite[p.
218]{PetersonI}.  It was finally proved in 2003 by Grigory Perelman
\label{poincare}\index{Szpiro, George}\index{Perelman, Grigory}\cite{Szpiro}.
 \end{itemize}

Many counterexamples to ``theorems'' in recent mathematical
literature related to Clifford algebras\index{Clifford algebras}
 have been found by Pertti
Lounesto (who passed away in 2002).\index{Lounesto, Pertti}
See the web page \url{http://mathforum.org/library/view/4933.html}.
% http://users.tkk.fi/~ppuska/mirror/Lounesto/counterexamples.htm

One of the purposes of Metamath\index{Metamath} is to allow proofs to be
expressed with absolute precision.  Developing a proof in the Metamath
language can be challenging, because Metamath will not permit even the
tiniest mistake.\index{errors in proofs}  But once the proof is created, its
correctness can be trusted immediately, without having to depend on the
process of peer review for confirmation.

\section{The Use of Computers in Mathematics}

\subsection{Computer Algebra Systems}

For the most part, you will find that Metamath\index{Metamath} is not a
practical tool for manipulating numbers.  (Even proving that $2 + 2 = 4$, if
you start with set theory, can be quite complex!)  Several commercial
mathematics packages are quite good at arithmetic, algebra, and calculus, and
as practical tools they are invaluable.\index{computer algebra system} But
they have no notion of proof, and cannot understand statements starting with
``there exists such and such...''.

Software packages such as Mathematica \cite{Wolfram}\index{Mathematica} do not
concern themselves with proofs but instead work directly with known results.
These packages primarily emphasize heuristic rules such as the substitution of
equals for equals to achieve simpler expressions or expressions in a different
form.  Starting with a rich collection of built-in rules and algorithms, users
can add to the collection by means of a powerful programming language.
However, results such as, say, the existence of a certain abstract object
without displaying the actual object cannot be expressed (directly) in their
languages.  The idea of a proof from a small set of axioms is absent.  Instead
this software simply assumes that each fact or rule you add to the built-in
collection of algorithms is valid.  One way to view the software is as a large
collection of axioms from which the software, with certain goals, attempts to
derive new theorems, for example equating a complex expression with a simpler
equivalent. But the terms ``theorem''\index{theorem} and
``proof,''\index{proof} for example, are not even mentioned in the index of
the user's manual for Mathematica.\index{Mathematica and proofs}  What is also
unsatisfactory from a philosophical point of view is that there is no way to
ensure the validity of the results other than by trusting the writer of each
application module or tediously checking each module by hand, similar to
checking a computer program for bugs.\index{computer program
bugs}\footnote{Two examples illustrate why the knowledge database of computer
algebra systems should sometimes be regarded with a certain caution.  If you
ask Mathematica (version 3.0) to \texttt{Solve[x\^{ }n + y\^{ }n == z\^{ }n , n]}
it will respond with \texttt{\{\{n-\char`\>-2\}, \{n-\char`\>-1\},
\{n-\char`\>1\}, \{n-\char`\>2\}\}}. In other words, Mathematica seems to
``know'' that Fermat's Last Theorem\index{Fermat's Last Theorem} is true!  (At
the time this version of Mathematica was released this fact was unknown.)  If
you ask Maple\index{Maple} to \texttt{solve(x\^{ }2 = 2\^{ }x)} then
\texttt{simplify(\{"\})}, it returns the solution set \texttt{\{2, 4\}}, apparently
unaware that 0.7666647\ldots is also a solution.} While of course extremely
valuable in applied mathematics, computer algebra systems tend to be of little
interest to the theoretical mathematician except as aids for exploring certain
specific problems.

Because of possible bugs, trusting the output of a computer algebra system for
use as theorems in a proof-verifier would defeat the latter's goal of rigor.
On the other hand, a fact such that a certain relatively large number is
prime, while easy for a computer algebra system to derive, might have a long,
tedious proof that could overwhelm a proof-verifier. One approach for linking
computer algebra systems to a proof-verifier while retaining the advantages of
both is to add a hypothesis to each such theorem indicating its source.  For
example, a constant {\sc maple} could indicate the theorem came from the Maple
package, and would mean ``assuming Maple is consistent, then\ldots''  This and
many other topics concerning the formalization of mathematics are discussed in
John Harrison's\index{Harrison, John} very interesting
PhD thesis~\cite{Harrison-thesis}.

\subsection{Automated Theorem Provers}\label{theoremprovers}

A mathematical theory is ``decidable''\index{decidable theory} if a mechanical
method or algorithm exists that is guaranteed to determine whether or not a
particular formula is a theorem.  Among the few theories that are decidable is
elementary geometry,\index{Euclidean geometry} as was shown by a classic
result of logician Alfred Tarski\index{Tarski, Alfred} in 1948
\cite{Tarski}.\footnote{Tarski's result actually applies to a subset of the
geometry discussed in elementary textbooks.  This subset includes most of what
would be considered elementary geometry but it is not powerful enough to
express, among other things, the notions of the circumference and area of a
circle.  Extending the theory in a way that includes notions such as these
makes the theory undecidable, as was also shown by Tarski.  Tarski's algorithm
is far too inefficient to implement practically on a computer.  A practical
algorithm for proving a smaller subset of geometry theorems (those not
involving concepts of ``order'' or ``continuity'') was discovered by Chinese
mathematician Wu Wen-ts\"{u}n in 1977 \cite{Chou}\index{Chou,
Shang-Ching}.}\index{Wen-ts{\"{u}}n, Wu}  But most theories, including
elementary arithmetic, are undecidable.  This fact contributes to keeping
mathematics alive and well, since many mathematicians believe
that they will never be
replaced by computers (if they believe Roger Penrose's argument that a
computer can never replace the brain \cite{Penrose}\index{Penrose, Roger}).
In fact,  elementary geometry is often considered a ``dead'' field
for the simple reason that it is decidable.

On the other hand, the undecidability of a theory does not mean that one cannot
use a computer to search for proofs, providing one is willing to give up if a
proof is not found after a reasonable amount of time.  The field of automated
theorem proving\index{automated theorem proving} specializes in pursuing such
computer searches.  Among the more successful results to date are those based
on an algorithm known as Robinson's resolution principle
\cite{Robinson}\index{Robinson's resolution principle}.

Automated theorem provers can be excellent tools for those willing to learn
how to use them.  But they are not widely used in mainstream pure
mathematics, even though they could probably be useful in many areas.  There
are several reasons for this.  Probably most important, the main goal in pure
mathematics is to arrive at results that are considered to be deep or
important; proving them is essential but secondary.  Usually, an automated
theorem prover cannot assist in this main goal, and by the time the main goal
is achieved, the mathematician may have already figured out the proof as a
by-product.  There is also a notational problem.  Mathematicians are used to
using very compact syntax where one or two symbols (heavily dependent on
context) can represent very complex concepts; this is part of the
hierarchy\index{hierarchy} they have built up to tackle difficult problems.  A
theorem prover on the other hand might require that a theorem be expressed in
``first-order logic,''\index{first-order logic} which is the logic on which
most of mathematics is ultimately based but which is not ordinarily used
directly because expressions can become very long.  Some automated theorem
provers are experimental programs, limited in their use to very specialized
areas, and the goal of many is simply research into the nature of automated
theorem proving itself.  Finally, much research remains to be done to enable
them to prove very deep theorems.  One significant result was a
computer proof by Larry Wos\index{Wos, Larry} and colleagues that every Robbins
algebra\index{Robbins algebra} is a Boolean  algebra\index{Boolean algebra}
({\em New York Times}, Dec. 10, 1996).\footnote{In 1933, E.~V.\
Huntington\index{Huntington, E. V.}
presented the following axiom system for
Boolean algebra with a unary operation $n$ and a binary operation $+$:
\begin{center}
    $x + y = y + x$ \\
    $(x + y) + z = x + (y + z)$ \\
    $n(n(x) + y) + n(n(x) + n(y)) = x$
\end{center}
Herbert Robbins\index{Robbins, Herbert}, a student of Huntington, conjectured
that the last equation can be replaced with a simpler one:
\begin{center}
    $n(n(x + y) + n(x + n(y))) = x$
\end{center}
Robbins and Huntington could not find a proof.  The problem was
later studied unsuccessfully by Tarski\index{Tarski, Alfred} and his
students, and it remained an unsolved problem until a
computer found the proof in 1996.  For more information on
the Robbins algebra problem see \cite{Wos}.}

How does Metamath\index{Metamath} relate to automated theorem provers?  A
theorem prover is primary concerned with one theorem at a time (perhaps
tapping into a small database of known theorems) whereas Metamath is more like
a theorem archiving system, storing both the theorem and its proof in a
database for access and verification.  Metamath is one answer to ``what do you
do with the output of a theorem prover?''  and could be viewed as the
next step in the process.  Automated theorem provers could be useful tools for
helping develop its database.
Note that very long, automatically
generated proofs can make your database fat and ugly and cause Metamath's proof
verification to take a long time to run.  Unless you have a particularly good
program that generates very concise proofs, it might be best to consider the
use of automatically generated proofs as a quick-and-dirty approach, to be
manually rewritten at some later date.

The program {\sc otter}\index{otter@{\sc otter}}\footnote{\url{http://www.cs.unm.edu/\~mccune/otter/}.}, later succeeded by
prover9\index{prover9}\footnote{\url{https://www.cs.unm.edu/~mccune/mace4/}.},
have been historically influential.
The E prover\index{E prover}\footnote{\url{https://github.com/eprover/eprover}.}
is a maintained automated theorem prover
for full first-order logic with equality.
There are many other automated theorem provers as well.

If you want to combine automated theorem provers with Metamath
consider investigating
the book {\em Automated Reasoning:  Introduction and Applications}
\cite{Wos}\index{Wos, Larry}.  This book discusses
how to use {\sc otter} in a way that can
not only able to generate
relatively efficient proofs, it can even be instructed to search for
shorter proofs.  The effective use of {\sc otter} (and similar tools)
does require a certain
amount of experience, skill, and patience.  The axiom system used in the
\texttt{set.mm}\index{set theory database (\texttt{set.mm})} set theory
database can be expressed to {\sc otter} using a method described in
\cite{Megill}.\index{Megill, Norman}\footnote{To use those axioms with
{\sc otter}, they must be restated in a way that eliminates the need for
``dummy variables.''\index{dummy variable!eliminating} See the Comment
on p.~\pageref{nodd}.} When successful, this method tends to generate
short and clever proofs, but my experiments with it indicate that the
method will find proofs within a reasonable time only for relatively
easy theorems.  It is still fun to experiment with.

Reference \cite{Bledsoe}\index{Bledsoe, W. W.} surveys a number of approaches
people have explored in the field of automated theorem proving\index{automated
theorem proving}.

\subsection{Interactive Theorem Provers}\label{interactivetheoremprovers}

Finding proofs completely automatically is difficult, so there
are some interactive theorem provers that allow a human to guide the
computer to find a proof.
Examples include
HOL Light\index{HOL light}%
\footnote{\url{https://www.cl.cam.ac.uk/~jrh13/hol-light/}.},
Isabelle\index{Isabelle}%
\footnote{\url{http://www.cl.cam.ac.uk/Research/HVG/Isabelle}.},
{\sc hol}\index{hol@{\sc hol}}%
\footnote{\url{https://hol-theorem-prover.org/}.},
and
Coq\index{Coq}\footnote{\url{https://coq.inria.fr/}.}.

A major difference between most of these tools and Metamath is that the
``proofs'' are actually programs that guide the program to find a proof,
and not the proof itself.
For example, an Isabelle/HOL proof might apply a step
\texttt{apply (blast dest: rearrange reduction)}. The \texttt{blast}
instruction applies
an automatic tableux prover and returns if it found a sequence of proof
steps that work... but the sequence is not considered part of the proof.

A good overview of
higher-level proof verification languages (such as {\sc lcf}\index{lcf@{\sc
lcf}} and {\sc hol};\index{hol@{\sc hol}})
is given in \cite{Harrison}.  All of these languages are fundamentally
different from Metamath in that much of the mathematical foundational
knowledge is embedded in the underlying proof-verification program, rather
than placed directly in the database that is being verified.
These can have a steep learning curve for those without a mathematical
background.  For example, one usually must have a fair understanding of
mathematical logic in order to follow their proofs.

\subsection{Proof Verifiers}\label{proofverifiers}

A proof verifier is a program that doesn't generate proofs but instead
verifies proofs that you give it.  Many proof verifiers have limited built-in
automated proof capabilities, such as figuring out simple logical inferences
(while still being guided by a person who provides the overall proof).  Metamath
has no built-in automated proof capability other than the limited
capability of its Proof Assistant.

Proof-verification languages are not used as frequently as they might be.
Pure mathematicians are more concerned with producing new results, and such
detail and rigor would interfere with that goal.  The use of computers in pure
mathematics is primarily focused on automated theorem provers (not verifiers),
again with the ultimate goal of aiding the creation of new mathematics.
Automated theorem provers are usually concerned with attacking one theorem at
time rather than making a large, organized database easily available to the
user.  Metamath is one way to help close this gap.

By itself Metamath is a mostly a proof verifier.
This does not mean that other approaches can't be used; the difference
is that in Metamath, the results of various provers must be recorded
step-by-step so that they can be verified.

Another proof-verification language is Mizar,\index{Mizar} which can display
its proofs in the informal language that mathematicians are accustomed to.
Information on the Mizar language is available at \url{http://mizar.org}.

For the working mathematician, Mizar is an excellent tool for rigorously
documenting proofs. Mizar typesets its proofs in the informal English used by
mathematicians (and, while fine for them, are just as inscrutable by
laypersons!). A price paid for Mizar is a relatively steep learning curve of a
couple of weeks.  Several mathematicians are actively formalizing different
areas of mathematics using Mizar and publishing the proofs in a dedicated
journal. Unfortunately the task of formalizing mathematics is still looked
down upon to a certain extent since it doesn't involve the creation of ``new''
mathematics.

\subsection{Creating a database of formalized mathematics}\label{mathdatabase}

Besides Metamath, there are several other on-going projects with the goal of
formalizing mathematics into computer-verifiable databases.
Understanding some history will help.

The {\sc qed}\index{qed project@{\sc qed} project}%
\footnote{\url{http://www-unix.mcs.anl.gov/qed}.}
project arose in 1993 and its goals were outlined in the
{\sc qed} manifesto.
The {\sc qed} manifesto was a
a proposal for a computer-based database of all mathematical knowledge,
trictly formalized and with all proofs having been checked automatically.
The project had a conference in 1994 and another in 1995;
there was also a ``twenty years of the {\sc qed} manifesto'' workshop
in 2014.
Its ideals are regularly reraised.

In a 2007 paper, Freek Wiedijk identified two reasons
for the failure of the {\sc qed} project as originally envisioned:%
\cite{Wiedijk-revisited}\index{Wiedijk, Freek}

\begin{itemize}
\item Very few people are working on formalization of mathematics. There is no compelling application for fully mechanized mathematics.
\item Formalized mathematics does not yet resemble traditional mathematics. This is partly due to the complexity of mathematical notation, and partly to the limitations of existing theorem provers and proof assistants.
\end{itemize}

But this did not end the dream of
formalizing mathematics into computer-verifiable databases.
The problems that led to the {\sc qed} manifesto are still with us,
even though the challenges were harder than originally considered.
What has happened instead is that various independent projects have
worked towards formalizing mathematics into computer-verifiable databases,
each simultaneously competing and cooperating with each other.

A concrete way to see this is
Freek Wiedijk's ``Formalizing 100 Theorems'' list%
\footnote{\url{http://www.cs.ru.nl/\%7Efreek/100/}.}
which shows the progress different systems have made on a challenge list
of 100 mathematical theorems.
The top systems as of 2019 (in terms of number of challenges completed) are
HOL Light, Isabelle, Coq, Mizar, and Metamath.

Each of these systems has different strengths and weaknesses, depending
on what you value.
Key aspects that differentiate Metamath from the other top systems are:

\begin{itemize}
\item Metamath is not tied to any particular set of axioms.
\item Metamath can shows every step of every proof, no exceptions.
  Most other provers only assert that a proof can be found, and do not
  show every step. This also makes verification fast, because
  the system does not need to rediscover proof details.
\item The Metamath verifier has been re-implemented in many different
  programming languages, so verification can (and is) be done by multiple
  implementations.  This greatly reduces the risk of accepting an invalid
  proof due to an error in the verifier.
\item Proofs stay proven.  In some systems, changes to the system's
  syntax or how a tactic works causes proofs to fail in later versions,
  causing older work to become essentially lost.
  Metamath's language is
  extremely small and fixed, so once a proof is added to a database,
  the database can be rechecked with later versions of the Metamath program
  and with other verifiers of Metamath databases.
  If an axiom or key definition needs to be changed, it is easy to
  manipulate the database as a whole to handle the change
  without touching the underlying verifier.
  Since re-verification of an entire database takes seconds, there
  is never a reason to delay complete verification.
  This aspect is especially compelling if your
  goal is to have a long-term database of proofs.
\item Licensing is generous.  The main Metamath databases are released to
  the public domain, and the main Metamath program is open source software
  under a standard, widely-used license.
\item Substitutions are easy to understand, even by those who are not
  professional mathematicians.
\end{itemize}

Of course, other systems may have advantages over Metamath
that are more compelling, depending on what you value.
In any case, we hope this helps you understand Metamath
within a wider context.

\subsection{In summary}\label{computers-summary}

To summarize our discussions of computers and mathematics, computer algebra
systems can be viewed as theorem generators focusing on a narrow realm of
mathematics (numbers and their properties), automated theorem provers as proof
generators for specific theorems in a much broader realm covered by a built-in
formal system such as first-order logic, interactive theorem
provers require human guidance, proof verifiers verify proofs but
historically they have been
restricted to first-order logic.
Metamath, in contrast,
is a proof verifier and documentor whose realm is essentially unlimited.

\section{Mathematics and Metamath}

\subsection{Standard Mathematics}

There are a number of ways that Metamath\index{Metamath} can be used with
standard mathematics.  The most satisfying way philosophically is to start at
the very beginning, and develop the desired mathematics from the axioms of
logic and set theory.\index{set theory}  This is the approach taken in the
\texttt{set.mm}\index{set theory database (\texttt{set.mm})}%
\index{Metamath Proof Explorer}
database (also known as the Metamath Proof Explorer).
Among other things, this database builds up to the
axioms of real and complex numbers\index{analysis}\index{real and complex
numbers} (see Section~\ref{real}), and a standard development of analysis, for
example, could start at that point, using it as a basis.   Besides this
philosophical advantage, there are practical advantages to having all of the
tools of set theory available in the supporting infrastructure.

On the other hand, you may wish to start with the standard axioms of a
mathematical theory without going through the set theoretical proofs of those
axioms.  You will need mathematical logic to make inferences, but if you wish
you can simply introduce theorems\index{theorem} of logic as
``axioms''\index{axiom} wherever you need them, with the implicit assumption
that in principle they can be proved, if they are obvious to you.  If you
choose this approach, you will probably want to review the notation used in
\texttt{set.mm}\index{set theory database (\texttt{set.mm})} so that your own
notation will be consistent with it.

\subsection{Other Formal Systems}
\index{formal system}

Unlike some programs, Metamath\index{Metamath} is not limited to any specific
area of mathematics, nor committed to any particular mathematical philosophy
such as classical logic versus intuitionism, nor limited, say, to expressions
in first-order logic.  Although the database \texttt{set.mm}
describes standard logic and set theory, Meta\-math
is actually a general-purpose language for describing a wide variety of formal
systems.\index{formal system}  Non-standard systems such as modal
logic,\index{modal logic} intuitionist logic\index{intuitionism}, higher-order
logic\index{higher-order logic}, quantum logic\index{quantum logic}, and
category theory\index{category theory} can all be described with the Metamath
language.  You define the symbols you prefer and tell Metamath the axioms and
rules you want to start from, and Metamath will verify any inferences you make
from those axioms and rules.  A simple example of a non-standard formal system
is Hofstadter's\index{Hofstadter, Douglas R.} MIU system,\index{MIU-system}
whose Metamath description is presented in Appendix~\ref{MIU}.

This is not hypothetical.
The largest Metamath database is
\texttt{set.mm}\index{set theory database (\texttt{set.mm})}%
\index{Metamath Proof Explorer}), aka the Metamath Proof Explorer,
which uses the most common axioms for mathematical foundations
(specifically classical logic combined with Zermelo-Fraenkel
set theory\index{Zermelo-Fraenkel set theory} with the Axiom of Choice).
But other Metamath databases are available:

\begin{itemize}
\item The database
  \texttt{iset.mm}\index{intuitionistic logic database (\texttt{iset.mm})},
  aka the
  Intuitionistic Logic Explorer\index{Intuitionistic Logic Explorer},
  uses intuitionistic logic (a constructivist point of view)
  instead of classical logic.
\item The database
  \texttt{nf.mm}\index{New Foundations database (\texttt{nf.mm})},
  aka the
  New Foundations Explorer\index{New Foundations Explorer},
  constructs mathematics from scratch,
  starting from Quine's New Foundations (NF) set theory axioms. 
\item The database
  \texttt{hol.mm}\index{Higher-order Logic database (\texttt{hol.mm})},
  aka the
  Higher-Order Logic (HOL) Explorer\index{Higher-Order Logic (HOL) Explorer},
  starts with HOL (also called simple type theory) and derives
  equivalents to ZFC axioms, connecting the two approaches. 
\end{itemize}

Since the days of David Hilbert,\index{Hilbert, David} mathematicians have
been concerned with the fact that the metalanguage\index{metalanguage} used to
describe mathematics may be stronger than the mathematics being described.
Metamath\index{Metamath}'s underlying finitary\index{finitary proof},
constructive nature provides a good philosophical basis for studying even the
weakest logics.\index{weak logic}

The usual treatment of many non-standard formal systems\index{formal
system} uses model theory\index{model theory} or proof theory\index{proof
theory} to describe these systems; these theories, in turn, are based on
standard set theory.  In other words, a non-standard formal system is defined
as a set with certain properties, and standard set theory is used to derive
additional properties of this set.  The standard set theory database provided
with Metamath can be used for this purpose, and when used this way
the development of a special
axiom system for the non-standard formal system becomes unnecessary.  The
model- or proof-theoretic approach often allows you to prove much deeper
results with less effort.

Metamath supports both approaches.  You can define the non-standard
formal system directly, or define the non-standard formal system as
a set with certain properties, whichever you find most helpful.

%\section{Additional Remarks}

\subsection{Metamath and Its Philosophy}

Closely related to Metamath\index{Metamath} is a philosophy or way of looking
at mathematics. This philosophy is related to the formalist
philosophy\index{formalism} of Hilbert\index{Hilbert, David} and his followers
\cite[pp.~1203--1208]{Kline}\index{Kline, Morris}
\cite[p.~6]{Behnke}\index{Behnke, H.}. In this philosophy, mathematics is
viewed as nothing more than a set of rules that manipulate symbols, together
with the consequences of those rules.  While the mathematics being described
may be complex, the rules used to describe it (the
``metamathematics''\index{metamathematics}) should be as simple as possible.
In particular, proofs should be restricted to dealing with concrete objects
(the symbols we write on paper rather than the abstract concepts they
represent) in a constructive manner; these are called ``finitary''
proofs\index{finitary proof} \cite[pp.~2--3]{Shoenfield}\index{Shoenfield,
Joseph R.}.

Whether or not you find Metamath interesting or useful will in part depend on
the appeal you find in its philosophy, and this appeal will probably depend on
your particular goals with respect to mathematics.  For example, if you are a
pure mathematician at the forefront of discovering new mathematical knowledge,
you will probably find that the rigid formality of Metamath stifles your
creativity.  On the other hand, we would argue that once this knowledge is
discovered, there are advantages to documenting it in a standard format that
will make it accessible to others.  Sixty years from now, your field may be
dormant, and as Davis and Hersh put it, your ``writings would become less
translatable than those of the Maya'' \cite[p.~37]{Davis}\index{Davis, Phillip
J.}.


\subsection{A History of the Approach Behind Metamath}

Probably the one work that has had the most motivating influence on
Metamath\index{Metamath} is Whitehead and Russell's monumental {\em Principia
Mathematica} \cite{PM}\index{Whitehead, Alfred North}\index{Russell,
Bertrand}\index{principia mathematica@{\em Principia Mathematica}}, whose aim
was to deduce all of mathematics from a small number of primitive ideas, in a
very explicit way that in principle anyone could understand and follow.  While
this work was tremendously influential in its time, from a modern perspective
it suffers from several drawbacks.  Both its notation and its underlying
axioms are now considered dated and are no longer used.  From our point of
view, its development is not really as accessible as we would like to see; for
practical reasons, proofs become more and more sketchy as its mathematics
progresses, and working them out in fine detail requires a degree of
mathematical skill and patience that many people don't have.  There are
numerous small errors, which is understandable given the tedious, technical
nature of the proofs and the lack of a computer to verify the details.
However, even today {\em Principia Mathematica} stands out as the work closest
in spirit to Metamath.  It remains a mind-boggling work, and one can't help
but be amazed at seeing ``$1+1=2$'' finally appear on page 83 of Volume II
(Theorem *110.643).

The origin of the proof notation used by Metamath dates back to the 1950's,
when the logician C.~A.~Meredith expressed his proofs in a compact notation
called ``condensed detachment''\index{condensed detachment}
\cite{Hindley}\index{Hindley, J. Roger} \cite{Kalman}\index{Kalman, J. A.}
\cite{Meredith}\index{Meredith, C. A.} \cite{Peterson}\index{Peterson, Jeremy
George}.  This notation allows proofs to be communicated unambiguously by
merely referencing the axiom\index{axiom}, rule\index{rule}, or
theorem\index{theorem} used at each step, without explicitly indicating the
substitutions\index{substitution!variable}\index{variable substitution} that
have to be made to the variables in that axiom, rule, or theorem.  Ordinarily,
condensed detachment is more or less limited to propositional
calculus\index{propositional calculus}.  The concept has been extended to
first-order logic\index{first-order logic} in \cite{Megill}\index{Megill,
Norman}, making it is easy to write a small computer program to verify proofs
of simple first-order logic theorems.\index{condensed detachment!and
first-order logic}

A key concept behind the notation of condensed detachment is called
``unification,''\index{unification} which is an algorithm for determining what
substitutions\index{substitution!variable}\index{variable substitution} to
variables have to be made to make two expressions match each other.
Unification was first precisely defined by the logician J.~A.~Robinson, who
used it in the development of a powerful
theorem-proving technique called the ``resolution principle''
\cite{Robinson}\index{Robinson's resolution principle}. Metamath does not make
use of the resolution principle, which is intended for systems of first-order
logic.\index{first-order logic}  Metamath's use is not restricted to
first-order logic, and as we have mentioned it does not automatically discover
proofs.  However, unification is a key idea behind Metamath's proof
notation, and Metamath makes use of a very simple version of it
(Section~\ref{unify}).

\subsection{Metamath and First-Order Logic}

First-order logic\index{first-order logic} is the supporting structure
for standard mathematics.  On top of it is set theory, which contains
the axioms from which virtually all of mathematics can be derived---a
remarkable fact.\index{category
theory}\index{cardinal, inaccessible}\label{categoryth}\footnote{An exception seems
to be category theory.  There are several schools of thought on whether
category theory is derivable from set theory.  At a minimum, it appears
that an additional axiom is needed that asserts the existence of an
``inaccessible cardinal'' (a type of infinity so large that standard set
theory can't prove or deny that it exists).
%
%%%% (I took this out that was in previous editions:)
% But it is also argued that not just one but a ``proper class'' of them
% is needed, and the existence of proper classes is impossible in standard
% set theory.  (A proper class is a collection of sets so huge that no set
% can contain it as an element.  Proper classes can lead to
% inconsistencies such as ``Russell's paradox.''  The axioms of standard
% set theory are devised so as to deny the existence of proper classes.)
%
For more information, see
\cite[pp.~328--331]{Herrlich}\index{Herrlich, Horst} and
\cite{Blass}\index{Blass, Andrea}.}

One of the things that makes Metamath\index{Metamath} more practical for
first-order theories is a set of axioms for first-order logic designed
specifically with Metamath's approach in mind.  These are included in a
the database \texttt{set.mm}\index{set theory database (\texttt{set.mm})}.
See Chapter~\ref{fol} for a detailed
description; the axioms are shown in Section~\ref{metaaxioms}.  While
logically equivalent to standard axiom systems, our axiom system breaks
up the standard axioms into smaller pieces such that from them, you can
directly derive what in other systems can only be derived as higher-level
``metatheorems.''\index{metatheorem}  In other words, it is more powerful than
the standard axioms from a metalogical point of view.  A rigorous
justification for this system and its ``metalogical
completeness''\index{metalogical completeness} is found in
\cite{Megill}\index{Megill, Norman}.  The system is closely related to a
system developed by Monk\index{Monk, J. Donald} and Tarski\index{Tarski,
Alfred} in 1965 \cite{Monks}.

For example, the formula $\exists x \, x = y $ (given $y$, there exists some
$x$ equal to it) is a theorem of logic,\footnote{Specifically, it is a theorem
of those systems of logic that assume non-empty domains.  It is not a theorem
of more general systems that include the empty domain\index{empty domain}, in
which nothing exists, period!  Such systems are called ``free
logics.''\index{free logic} For a discussion of these systems, see
\cite{Leblanc}\index{Leblanc, Hugues}.  Since our use for logic is as a basis
for set theory, which has a non-empty domain, it is more convenient (and more
traditional) to use a less general system.  An interesting curiosity is that,
using a free logic as a basis for Zermelo-Fraenkel set
theory\index{Zermelo-Fraenkel set theory} (with the redundant Axiom of the
Null Set omitted),\index{Axiom of the Null Set} we cannot even prove the
existence of a single set without assuming the axiom of infinity!\index{Axiom
of Infinity}} whether or not $x$ and $y$ are distinct variables\index{distinct
variables}.  In many systems of logic, we would have to prove two theorems to
arrive at this result.  First we would prove ``$\exists x \, x = x $,'' then
we would separately prove ``$\exists x \, x = y $, where $x$ and $y$ are
distinct variables.''  We would then combine these two special cases ``outside
of the system'' (i.e.\ in our heads) to be able to claim, ``$\exists x \, x =
y $, regardless of whether $x$ and $y$ are distinct.''  In other words, in the
combination of the two special cases is a
metatheorem.  In the system of logic
used in Metamath's set theory\index{set theory database (\texttt{set.mm})}
database, the axioms of logic are broken down into small pieces that allow
them to reassembled in such a way that theorems such as these can be proved
directly.

Breaking down the axioms in this way makes them look peculiar and not very
intuitive at first, but rest assured that they are correct and complete.  Their
correctness is ensured because they are theorem schemes of standard first-order
logic (which you can easily verify if you are a logician).  Their completeness
follows from the fact that we explicitly derive the standard axioms of
first-order logic as theorems.  Deriving the standard axioms is somewhat
tricky, but once we're there, we have at our disposal a system that is less
awkward to work with in formal proofs\index{formal proof}.  In technical terms
that logicians understand, we eliminate the cumbersome concepts of ``free
variable,''\index{free variable} ``bound variable,''\index{bound variable} and
``proper substitution''\index{proper substitution}\index{substitution!proper}
as primitive notions.  These concepts are present in our system but are
defined in terms of concepts expressed by the axioms and can be eliminated in
principle.  In standard systems, these concepts are really like additional,
implicit axioms\index{implicit axiom} that are somewhat complex and cannot be
eliminated.

The traditional approach to logic, wherein free variables and proper
substitution is defined, is also possible to do directly in the Metamath
language.  However, the notation tends to become awkward, and there are
disadvantages:  for example, extending the definition of a wff with a
definition is awkward, because the free variable and proper substitution
concepts have to have their definitions also extended.  Our choice of
axioms for \texttt{set.mm} is to a certain extent a matter of style, in
an attempt to achieve overall simplicity, but you should also be aware
that the traditional approach is possible as well if you should choose
it.

\chapter{Using the Metamath Program}
\label{using}

\section{Installation}

The way that you install Metamath\index{Metamath!installation} on your
computer system will vary for different computers.  Current
instructions are provided with the Metamath program download at
\url{http://metamath.org}.  In general, the installation is simple.
There is one file containing the Metamath program itself.  This file is
usually called \texttt{metamath} or \texttt{metamath.}{\em xxx} where
{\em xxx} is the convention (such as \texttt{exe}) for an executable
program on your operating system.  There are several additional files
containing samples of the Metamath language, all ending with
\texttt{.mm}.  The file \texttt{set.mm}\index{set theory database
(\texttt{set.mm})} contains logic and set theory and can be used as a
starting point for other areas of mathematics.

You will also need a text editor\index{text editor} capable of editing plain
{\sc ascii}\footnote{American Standard Code for Information Interchange.} text
in order to prepare your input files.\index{ascii@{\sc ascii}}  Most computers
have this capability built in.  Note that plain text is not necessarily the
default for some word processing programs\index{word processor}, especially if
they can handle different fonts; for example, with Microsoft Word\index{Word
(Microsoft)}, you must save the file in the format ``Text Only With Line
Breaks'' to get a plain text\index{plain text} file.\footnote{It is
recommended that all lines in a Metamath source file be 79 characters or less
in length for compatibility among different computer terminals.  When creating
a source file on an editor such as Word, select a monospaced
font\index{monospaced font} such as Courier\index{Courier font} or
Monaco\index{Monaco font} to make this easier to achieve.  Better yet,
just use a plain text editor such as Notepad.}

On some computer systems, Metamath does not have the capability to print
its output directly; instead, you send its output to a file (using the
\texttt{open} commands described later).  The way you print this output
file depends on your computer.\index{printers} Some computers have a
print command, whereas with others, you may have to read the file into
an editor and print it from there.

If you want to print your Metamath source files with typeset formulas
containing standard mathematical symbols, you will need the \LaTeX\
typesetting program\index{latex@{\LaTeX}}, which is widely and freely
available for most operating systems.  It runs natively on Unix and
Linux, and can be installed on Windows as part of the free Cygwin
package (\url{http://cygwin.com}).

You can also produce {\sc html}\footnote{HyperText Markup Language.}
web pages.  The {\tt help html} command in the Metamath program will
assist you with this feature.

\section{Your First Formal System}\label{start}
\subsection{From Nothing to Zero}\label{startf}

To give you a feel for what the Metamath\index{Metamath} language looks like,
we will take a look at a very simple example from formal number
theory\index{number theory}.  This example is taken from
Mendelson\index{Mendelson, Elliot} \cite[p. 123]{Mendelson}.\footnote{To keep
the example simple, we have changed the formalism slightly, and what we call
axioms\index{axiom} are strictly speaking theorems\index{theorem} in
\cite{Mendelson}.}  We will look at a small subset of this theory, namely that
part needed for the first number theory theorem proved in \cite{Mendelson}.

First we will look at a standard formal proof\index{formal proof} for the
example we have picked, then we will look at the Metamath version.  If you
have never been exposed to formal proofs, the notation may seem to be such
overkill to express such simple notions that you may wonder if you are missing
something.  You aren't.  The concepts involved are in fact very simple, and a
detailed breakdown in this fashion is necessary to express the proof in a way
that can be verified mechanically.  And as you will see, Metamath breaks the
proof down into even finer pieces so that the mechanical verification process
can be about as simple as possible.

Before we can introduce the axioms\index{axiom} of the theory, we must define
the syntax rules for forming legal expressions\index{syntax rules}
(combinations of symbols) with which those axioms can be used. The number 0 is
a {\bf term}\index{term}; and if $ t$ and $r$ are terms, so is $(t+r)$. Here,
$ t$ and $r$ are ``metavariables''\index{metavariable} ranging over terms; they
themselves do not appear as symbols in an actual term.  Some examples of
actual terms are $(0 + 0)$ and $((0+0)+0)$.  (Note that our theory describes
only the number zero and sums of zeroes.  Of course, not much can be done with
such a trivial theory, but remember that we have picked a very small subset of
complete number theory for our example.  The important thing for you to focus
on is our definitions that describe how symbols are combined to form valid
expressions, and not on the content or meaning of those expressions.) If $ t$
and $r$ are terms, an expression of the form $ t=r$ is a {\bf wff}
(well-formed formula)\index{well-formed formula (wff)}; and if $P$ and $Q$ are
wffs, so is $(P\rightarrow Q)$ (which means ``$P$ implies
$Q$''\index{implication ($\rightarrow$)} or ``if $P$ then $Q$'').
Here $P$ then $Q$ are metavariables ranging over wffs.  Examples of actual
wffs are $0=0$, $(0+0)=0$, $(0=0 \rightarrow (0+0)=0)$, and $(0=0\rightarrow
(0=0\rightarrow 0=(0+0)))$.  (Our notation makes use of more parentheses than
are customary, but the elimination of ambiguity this way simplifies our
example by avoiding the need to define operator precedence\index{operator
precedence}.)

The {\bf axioms}\index{axiom} of our theory are all wffs of the following
form, where $ t$, $r$, and $s$ are any terms:

%Latex p. 92
\renewcommand{\theequation}{A\arabic{equation}}

\begin{equation}
(t=r\rightarrow (t=s\rightarrow r=s))
\end{equation}
\begin{equation}
(t+0)=t
\end{equation}

Note that there are an infinite number of axioms since there are an infinite
number of possible terms.  A1 and A2 are properly called ``axiom
schemes,''\index{axiom scheme} but we will refer to them as ``axioms'' for
brevity.

An axiom is a {\bf theorem}; and if $P$ and $(P\rightarrow Q)$ are theorems
(where $P$ and $Q$ are wffs), then $Q$ is also a theorem.\index{theorem}  The
second part of this definition is called the modus ponens (MP) rule of
inference\index{inference rule}\index{modus ponens}.  It allows us to obtain
new theorems from old ones.

The {\bf proof}\index{proof} of a theorem is a sequence of one or more
theorems, each of which is either an axiom or the result of modus ponens
applied to two previous theorems in the sequence, and the last of which is the
theorem being proved.

The theorem we will prove for our example is very simple:  $ t=t$.  The proof of
our theorem follows.  Study it carefully until you feel sure you
understand it.\label{zeroproof}
\\[\baselineskip]
\begin{tabular}{lll}
1.&$(t+0)=t$&(by axiom A2)\\
2.&$(t+0)=t$&(by axiom A2)\\
3.&$((t+0)=t\rightarrow ((t+0)=t\rightarrow t=t))$&(by axiom A1)\\
4.&$((t+0)=t\rightarrow t=t)$&(by MP applied to\\
{}&{}&steps 2 and 3)\\
5.&$ t=t$&(by MP applied to\\
{}&{}&steps 1 and 4)
\end{tabular}
\\[\baselineskip]
(You may wonder why step 1 is repeated twice.  This is not necessary in the
formal language we have defined, but in Metamath's ``reverse Polish
notation''\index{reverse Polish notation (RPN)} for proofs, a previous step
can be referred to only once.  The repetition of step~1 here will enable you
to see more clearly the correspondence of this proof with the
Metamath\index{Metamath} version on p.~\pageref{demoproof}.)

Our theorem is more properly called a ``theorem scheme,''\index{theorem
scheme} for it represents an infinite number of theorems, one for each
possible term $ t$.  Two examples of actual theorems would be $0=0$ and
$(0+0)=(0+0)$.  Rarely do we prove actual theorems, since by proving schemes
we can prove an infinite number of theorems in one fell swoop.  Similarly, our
proof should really be called a ``proof scheme.''\index{proof scheme}  To
obtain an actual proof, pick an actual term to use in place of $ t$, and
substitute it for $ t$ throughout the proof.

Let's discuss what we have done here.  The axioms\index{axiom} of our theory,
A1 and A2, are trivial and obvious.  Everyone knows that adding zero to
something doesn't change it, and also that if two things are equal to a third,
then they are equal to each other. In fact, stating the trivial and obvious is
a goal to strive for in any axiomatic system.  From trivial and obvious truths
that everyone agrees upon, we can prove results that are not so obvious yet
have absolute faith in them.  If we trust the axioms and the rules, we must,
by definition, trust the consequences of those axioms and rules, if logic is
to mean anything at all.

Our rule of inference\index{rule}, modus ponens\index{modus ponens}, is also
pretty obvious once you understand what it means.  If we prove a fact $P$, and
we also prove that $P$ implies $Q$, then $Q$ necessarily follows as a new
fact.  The rule provides us with a means for obtaining new facts (i.e.\
theorems\index{theorem}) from old ones.

The theorem that we have proved, $ t=t$, is so fundamental that you may wonder
why it isn't one of the axioms\index{axiom}.  In some axiom systems of
arithmetic, it {\em is} an axiom.  The choice of axioms in a theory is to some
extent arbitrary and even an art form, constrained only by the requirement
that any two equivalent axiom systems be able to derive each other as
theorems.  We could imagine that the inventor of our axiom system originally
included $ t=t$ as an axiom, then discovered that it could be derived as a
theorem from the other axioms.  Because of this, it was not necessary to
keep it as an axiom.  By eliminating it, the final set of axioms became
that much simpler.

Unless you have worked with formal proofs\index{formal proof} before, it
probably wasn't apparent to you that $ t=t$ could be derived from our two
axioms until you saw the proof. While you certainly believe that $ t=t$ is
true, you might not be able to convince an imaginary skeptic who believes only
in our two axioms until you produce the proof.  Formal proofs such as this are
hard to come up with when you first start working with them, but after you get
used to them they can become interesting and fun.  Once you understand the
idea behind formal proofs you will have grasped the fundamental principle that
underlies all of mathematics.  As the mathematics becomes more sophisticated,
its proofs become more challenging, but ultimately they all can be broken down
into individual steps as simple as the ones in our proof above.

Mendelson's\index{Mendelson, Elliot} book, from which our example was taken,
contains a number of detailed formal proofs such as these, and you may be
interested in looking it up.  The book is intended for mathematicians,
however, and most of it is rather advanced.  Popular literature describing
formal proofs\index{formal proof} include \cite[p.~296]{Rucker}\index{Rucker,
Rudy} and \cite[pp.~204--230]{Hofstadter}\index{Hofstadter, Douglas R.}.

\subsection{Converting It to Metamath}\label{convert}

Formal proofs\index{formal proof} such as the one in our example break down
logical reasoning into small, precise steps that leave little doubt that the
results follow from the axioms\index{axiom}.  You might think that this would
be the finest breakdown we can achieve in mathematics.  However, there is more
to the proof than meets the eye. Although our axioms were rather simple, a lot
of verbiage was needed before we could even state them:  we needed to define
``term,'' ``wff,'' and so on.  In addition, there are a number of implied
rules that we haven't even mentioned. For example, how do we know that step 3
of our proof follows from axiom A1? There is some hidden reasoning involved in
determining this.  Axiom A1 has two occurrences of the letter $ t$.  One of
the implied rules states that whatever we substitute for $ t$ must be a legal
term\index{term}.\footnote{Some authors make this implied rule explicit by
stating, ``only expressions of the above form are terms,'' after defining
``term.''}  The expression $ t+0$ is pretty obviously a legal term whenever $
t$ is, but suppose we wanted to substitute a huge term with thousands of
symbols?  Certainly a lot of work would be involved in determining that it
really is a term, but in ordinary formal proofs all of this work would be
considered a single ``step.''

To express our axiom system in the Metamath\index{Metamath} language, we must
describe this auxiliary information in addition to the axioms themselves.
Metamath does not know what a ``term'' or a ``wff''\index{well-formed formula
(wff)} is.  In Metamath, the specification of the ways in which we can combine
symbols to obtain terms and wffs are like little axioms in themselves.  These
auxiliary axioms are expressed in the same notation as the ``real''
axioms\index{axiom}, and Metamath does not distinguish between the two.  The
distinction is made by you, i.e.\ by the way in which you interpret the
notation you have chosen to express these two kinds of axioms.

The Metamath language breaks down mathematical proofs into tiny pieces, much
more so than in ordinary formal proofs\index{formal proof}.  If a single
step\index{proof step} involves the
substitution\index{substitution!variable}\index{variable substitution} of a
complex term for one of its variables, Metamath must see this single step
broken down into many small steps.  This fine-grained breakdown is what gives
Metamath generality and flexibility as it lets it not be limited to any
particular mathematical notation.

Metamath's proof notation is not, in itself, intended to be read by humans but
rather is in a compact format intended for a machine.  The Metamath program
will convert this notation to a form you can understand, using the \texttt{show
proof}\index{\texttt{show proof} command} command.  You can tell the program what
level of detail of the proof you want to look at.  You may want to look at
just the logical inference steps that correspond
to ordinary formal proof steps,
or you may want to see the fine-grained steps that prove that an expression is
a term.

Here, without further ado, is our example converted to the
Metamath\index{Metamath} language:\index{metavariable}\label{demo0}

\begin{verbatim}

        $( Declare the constant symbols we will use $)
            $c 0 + = -> ( ) term wff |- $.
        $( Declare the metavariables we will use $)
            $v t r s P Q $.
        $( Specify properties of the metavariables $)
            tt $f term t $.
            tr $f term r $.
            ts $f term s $.
            wp $f wff P $.
            wq $f wff Q $.
        $( Define "term" and "wff" $)
            tze $a term 0 $.
            tpl $a term ( t + r ) $.
            weq $a wff t = r $.
            wim $a wff ( P -> Q ) $.
        $( State the axioms $)
            a1 $a |- ( t = r -> ( t = s -> r = s ) ) $.
            a2 $a |- ( t + 0 ) = t $.
        $( Define the modus ponens inference rule $)
            ${
               min $e |- P $.
               maj $e |- ( P -> Q ) $.
               mp  $a |- Q $.
            $}
        $( Prove a theorem $)
            th1 $p |- t = t $=
          $( Here is its proof: $)
               tt tze tpl tt weq tt tt weq tt a2 tt tze tpl
               tt weq tt tze tpl tt weq tt tt weq wim tt a2
               tt tze tpl tt tt a1 mp mp
             $.
\end{verbatim}\index{metavariable}

A ``database''\index{database} is a set of one or more {\sc ascii} source
files.  Here's a brief description of this Metamath\index{Metamath} database
(which consists of this single source file), so that you can understand in
general terms what is going on.  To understand the source file in detail, you
should read Chapter~\ref{languagespec}.

The database is a sequence of ``tokens,''\index{token} which are normally
separated by spaces or line breaks.  The only tokens that are built into
the Metamath language are those beginning with \texttt{\$}.  These tokens
are called ``keywords.''\index{keyword}  All other tokens are
user-defined, and their names are arbitrary.

As you might have guessed, the Metamath token \texttt{\$(}\index{\texttt{\$(} and
\texttt{\$)} auxiliary keywords} starts a comment and \texttt{\$)} ends a comment.

The Metamath tokens \texttt{\$c}\index{\texttt{\$c} statement},
\texttt{\$v}\index{\texttt{\$v} statement},
\texttt{\$e}\index{\texttt{\$e} statement},
\texttt{\$f}\index{\texttt{\$f} statement},
\texttt{\$a}\index{\texttt{\$a} statement}, and
\texttt{\$p}\index{\texttt{\$p} statement} specify ``statements'' that
end with \texttt{\$.}\,.\index{\texttt{\$.}\ keyword}

The Metamath tokens \texttt{\$c} and \texttt{\$v} each declare\index{constant
declaration}\index{variable declaration} a list of user-defined tokens, called
``math symbols,''\index{math symbol} that the database will reference later
on.  All of the math symbols they define you have seen earlier except the
turnstile symbol \texttt{|-} ($\vdash$)\index{turnstile ({$\,\vdash$})}, which is
commonly used by logicians to mean ``a proof exists for.''  For us
the turnstile is just a
convenient symbol that distinguishes expressions that are axioms\index{axiom}
or theorems\index{theorem} from expressions that are terms or wffs.

The \texttt{\$c} statement declares ``constants''\index{constant} and
the \texttt{\$v} statement declares
``variables''\index{variable}\index{constant declaration}\index{variable
declaration} (or more precisely, metavariables\index{metavariable}).  A
variable may be substituted\index{substitution!variable}\index{variable
substitution} with sequences of math symbols whereas a constant may not
be substituted with anything.

It may seem redundant to require both \texttt{\$c}\index{\texttt{\$c} statement} and
\texttt{\$v}\index{\texttt{\$v} statement} statements (since any math
symbol\index{math symbol} not specified with a \texttt{\$c} statement could be
presumed to be a variable), but this provides for better error checking and
also allows math symbols to be redeclared\index{redeclaration of symbols}
(Section~\ref{scoping}).

The token \texttt{\$f}\index{\texttt{\$f} statement} specifies a
statement called a ``variable-type hypothesis'' (also called a
``floating hypothesis'') and \texttt{\$e}\index{\texttt{\$e} statement}
specifies a ``logical hypothesis'' (also called an ``essential
hypothesis'').\index{hypothesis}\index{variable-type
hypothesis}\index{logical hypothesis}\index{floating
hypothesis}\index{essential hypothesis} The token
\texttt{\$a}\index{\texttt{\$a} statement} specifies an ``axiomatic
assertion,''\index{axiomatic assertion} and
\texttt{\$p}\index{\texttt{\$p} statement} specifies a ``provable
assertion.''\index{provable assertion} To the left of each occurrence of
these four tokens is a ``label''\index{label} that identifies the
hypothesis or assertion for later reference.  For example, the label of
the first axiomatic assertion is \texttt{tze}.  A \texttt{\$f} statement
must contain exactly two math symbols, a constant followed by a
variable.  The \texttt{\$e}, \texttt{\$a}, and \texttt{\$p} statements
each start with a constant followed by, in general, an arbitrary
sequence of math symbols.

Associated with each assertion\index{assertion} is a set of hypotheses
that must be satisfied in order for the assertion to be used in a proof.
These are called the ``mandatory hypotheses''\index{mandatory
hypothesis} of the assertion.  Among those hypotheses whose ``scope''
(described below) includes the assertion, \texttt{\$e} hypotheses are
always mandatory and \texttt{\$f}\index{\texttt{\$f} statement}
hypotheses are mandatory when they share their variable with the
assertion or its \texttt{\$e} hypotheses.  The exact rules for
determining which hypotheses are mandatory are described in detail in
Sections~\ref{frames} and \ref{scoping}.  For example, the mandatory
hypotheses of assertion \texttt{tpl} are \texttt{tt} and \texttt{tr},
whereas assertion \texttt{tze} has no mandatory hypotheses because it
contains no variables and has no \texttt{\$e}\index{\texttt{\$e}
statement} hypothesis.  Metamath's \texttt{show statement}
command\index{\texttt{show statement} command}, described in the next
section, will show you a statement's mandatory hypotheses.

Sometimes we need to make a hypothesis relevant to only certain
assertions.  The set of statements to which a hypothesis is relevant is
called its ``scope.''  The Metamath brackets,
\texttt{\$\char`\{}\index{\texttt{\$\char`\{} and \texttt{\$\char`\}}
keywords} and \texttt{\$\char`\}}, define a ``block''\index{block} that
delimits the scope of any hypothesis contained between them.  The
assertion \texttt{mp} has mandatory hypotheses \texttt{wp}, \texttt{wq},
\texttt{min}, and \texttt{maj}.  The only mandatory hypothesis of
\texttt{th1}, on the other hand, is \texttt{tt}, since \texttt{th1}
occurs outside of the block containing \texttt{min} and \texttt{maj}.

Note that \texttt{\$\char`\{} and \texttt{\$\char`\}} do not affect the
scope of assertions (\texttt{\$a} and \texttt{\$p}).  Assertions are always
available to be referenced by any later proof in the source file.

Each provable assertion (\texttt{\$p}\index{\texttt{\$p} statement}
statement) has two parts.  The first part is the
assertion\index{assertion} itself, which is a sequence of math
symbol\index{math symbol} tokens placed between the \texttt{\$p} token
and a \texttt{\$=}\index{\texttt{\$=} keyword} token.  The second part
is a ``proof,'' which is a list of label tokens placed between the
\texttt{\$=} token and the \texttt{\$.}\index{\texttt{\$.}\ keyword}\
token that ends the statement.\footnote{If you've looked at the
\texttt{set.mm} database, you may have noticed another notation used for
proofs.  The other notation is called ``compressed.''\index{compressed
proof}\index{proof!compressed} It reduces the amount of space needed to
store a proof in the database and is described in
Appendix~\ref{compressed}.  In the example above, we use
``normal''\index{normal proof}\index{proof!normal} notation.} The proof
acts as a series of instructions to the Metamath program, telling it how
to build up the sequence of math symbols contained in the assertion part of
the \texttt{\$p} statement, making use of the hypotheses of the
\texttt{\$p} statement and previous assertions.  The construction takes
place according to precise rules.  If the list of labels in the proof
causes these rules to be violated, or if the final sequence that results
does not match the assertion, the Metamath program will notify you with
an error message.

If you are familiar with reverse Polish notation (RPN), which is sometimes used
on pocket calculators, here in a nutshell is how a proof works.  Each
hypothesis label\index{hypothesis label} in the proof is pushed\index{push}
onto the RPN stack\index{stack}\index{RPN stack} as it is encountered. Each
assertion label\index{assertion label} pops\index{pop} off the stack as many
entries as the referenced assertion has mandatory hypotheses.  Variable
substitutions\index{substitution!variable}\index{variable substitution} are
computed which, when made to the referenced assertion's mandatory hypotheses,
cause these hypotheses to match the stack entries. These same substitutions
are then made to the variables in the referenced assertion itself, which is
then pushed onto the stack.  At the end of the proof, there should be one
stack entry, namely the assertion being proved.  This process is explained in
detail in Section~\ref{proof}.

Metamath's proof notation is not very readable for humans, but it allows the
proof to be stored compactly in a file.  The Metamath\index{Metamath} program
has proof display features that let you see what's going on in a more
readable way, as you will see in the next section.

The rules used in verifying a proof are not based on any built-in syntax of the
symbol sequence in an assertion\index{assertion} nor on any built-in meanings
attached to specific symbol names.  They are based strictly on symbol
matching:  constants\index{constant} must match themselves, and
variables\index{variable} may be replaced with anything that allows a match to
occur.  For example, instead of \texttt{term}, \texttt{0}, and \verb$|-$ we could
have just as well used \texttt{yellow}, \texttt{zero}, and \texttt{provable}, as long
as we did so consistently throughout the database.  Also, we could have used
\texttt{is provable} (two tokens) instead of \verb$|-$ (one token) throughout the
database.  In each of these cases, the proof would be exactly the same.  The
independence of proofs and notation means that you have a lot of flexibility to
change the notation you use without having to change any proofs.

\section{A Trial Run}\label{trialrun}

Now you are ready to try out the Metamath\index{Metamath} program.

On all computer systems, Metamath has a standard ``command line
interface'' (CLI)\index{command line interface (CLI)} that allows you to
interact with it.  You supply commands to the CLI by typing them on the
keyboard and pressing your keyboard's {\em return} key after each line
you enter.  The CLI is designed to be easy to use and has built-in help
features.

The first thing you should do is to use a text editor to create a file
called \texttt{demo0.mm} and type into it the Metamath source shown on
p.~\pageref{demo0}.  Actually, this file is included with your Metamath
software package, so check that first.  If you type it in, make sure
that you save it in the form of ``plain {\sc ascii} text with line
breaks.''  Most word processors will have this feature.

Next you must run the Metamath program.  Depending on your computer
system and how Metamath is installed, this could range from clicking the
mouse on the Metamath icon to typing \texttt{run metamath} to typing
simply \texttt{metamath}.  (Metamath's {\tt help invoke} command describes
alternate ways of invoking the Metamath program.)

When you first enter Metamath\index{Metamath}, it will be at the CLI, waiting
for your input. You will see something the following on your screen:
\begin{verbatim}
	Metamath - Version 0.172 25-Jan-2019
        Type HELP for help, EXIT to exit.
        MM>
\end{verbatim}
The \texttt{MM>} prompt means that Metamath is waiting for a command.
(The help message line suggests that commands should be typed in upper
case, but actually command keywords\index{command keyword} are not case
sensitive.  We will use lower case in our examples.)
The version number and its release date will probably be different on your
system from the one we show above.

The first thing that you need to do is to read in your
database:\index{\texttt{read} command}\footnote{If a directory path is
needed on Unix,\index{Unix file names}\index{file names!Unix} you should
enclose the path/file name in quotes to prevent Metamath from thinking
that the \texttt{/} in the path name is a command qualifier, e.g.,
\texttt{read \char`\"db/set.mm\char`\"}.  Quotes are optional when there
is no ambiguity.}
\begin{verbatim}
        MM> read demo0.mm
\end{verbatim}
Remember to press the {\em return} key after entering this command.  If
you omit the file name, Metamath will prompt you for one.   The syntax for
specifying a Macintosh file name path is given in a footnote on
p.~\pageref{includef}.\index{Macintosh file names}\index{file
names!Macintosh}

If there are any syntax errors in the database, Metamath will let you know
when it reads in the file.  The one thing that Metamath does not check when
reading in a database is that all proofs are correct, because this would
slow it down too much.  It is a good idea to periodically verify the proofs in
a database you are making changes to.  To do this, use the following command
(and do it for your \texttt{demo0.mm} file now).  Note that the \texttt{*} is a
``wild card'' meaning all proofs in the file.\index{\texttt{verify proof} command}
\begin{verbatim}
        MM> verify proof *
\end{verbatim}
Metamath will report any proofs that are incorrect.

It is often useful to save the information that the Metamath program displays
on the screen. You can save everything that happens on the screen by opening a
log file. You may want to do this before you read in a database so that you
can examine any errors later on.  To open a log file, type
\begin{verbatim}
        MM> open log abc.log
\end{verbatim}
This will open a file called \texttt{abc.log}, and everything that appears on the
screen from this point on will be stored in this file.  The name of the log file
is arbitrary. To close the log file, type
\begin{verbatim}
        MM> close log
\end{verbatim}

Several commands let you examine what's inside of your database.
Section~\ref{exploring} has an overview of some useful ones.  The
\texttt{show labels} command lets you see what statement
labels\index{label} exist.  A \texttt{*} matches any combination of
characters, and \texttt{t*} refers to all labels starting with the
letter \texttt{t}.\index{\texttt{show labels} command} The \texttt{/all}
is a ``command qualifier''\index{command qualifier} that tells Metamath
to include labels of hypotheses.  (To see the syntax explained, type
\texttt{help show labels}.)  Type
\begin{verbatim}
        MM> show labels t* /all
\end{verbatim}
Metamath will respond with
\begin{verbatim}
        The statement number, label, and type are shown.
        3 tt $f       4 tr $f       5 ts $f       8 tze $a
        9 tpl $a      19 th1 $p
\end{verbatim}

You can use the \texttt{show statement} command to get information about a
particular statement.\index{\texttt{show statement} command}
For example, you can get information about the statement with label \texttt{mp}
by typing
\begin{verbatim}
        MM> show statement mp /full
\end{verbatim}
Metamath will respond with
\begin{verbatim}
        Statement 17 is located on line 43 of the file
        "demo0.mm".
        "Define the modus ponens inference rule"
        17 mp $a |- Q $.
        Its mandatory hypotheses in RPN order are:
          wp $f wff P $.
          wq $f wff Q $.
          min $e |- P $.
          maj $e |- ( P -> Q ) $.
        The statement and its hypotheses require the
              variables:  Q P
        The variables it contains are:  Q P
\end{verbatim}
The mandatory hypotheses\index{mandatory hypothesis} and their
order\index{RPN order} are
useful to know when you are trying to understand or debug a proof.

Now you are ready to look at what's really inside of our proof.  First, here is
how to look at every step in the proof---not just the ones corresponding to an
ordinary formal proof\index{formal proof}, but also the ones that build up the
formulas that appear in each ordinary formal proof step.\index{\texttt{show
proof} command}
\begin{verbatim}
        MM> show proof th1 /lemmon /all
\end{verbatim}

This will display the proof on the screen in the following format:
\begin{verbatim}
         1 tt            $f term t
         2 tze           $a term 0
         3 1,2 tpl       $a term ( t + 0 )
         4 tt            $f term t
         5 3,4 weq       $a wff ( t + 0 ) = t
         6 tt            $f term t
         7 tt            $f term t
         8 6,7 weq       $a wff t = t
         9 tt            $f term t
        10 9 a2          $a |- ( t + 0 ) = t
        11 tt            $f term t
        12 tze           $a term 0
        13 11,12 tpl     $a term ( t + 0 )
        14 tt            $f term t
        15 13,14 weq     $a wff ( t + 0 ) = t
        16 tt            $f term t
        17 tze           $a term 0
        18 16,17 tpl     $a term ( t + 0 )
        19 tt            $f term t
        20 18,19 weq     $a wff ( t + 0 ) = t
        21 tt            $f term t
        22 tt            $f term t
        23 21,22 weq     $a wff t = t
        24 20,23 wim     $a wff ( ( t + 0 ) = t -> t = t )
        25 tt            $f term t
        26 25 a2         $a |- ( t + 0 ) = t
        27 tt            $f term t
        28 tze           $a term 0
        29 27,28 tpl     $a term ( t + 0 )
        30 tt            $f term t
        31 tt            $f term t
        32 29,30,31 a1   $a |- ( ( t + 0 ) = t -> ( ( t + 0 )
                                             = t -> t = t ) )
        33 15,24,26,32 mp  $a |- ( ( t + 0 ) = t -> t = t )
        34 5,8,10,33 mp  $a |- t = t
\end{verbatim}

The \texttt{/lemmon} command qualifier specifies what is known as a Lemmon-style
display\index{Lemmon-style proof}\index{proof!Lemmon-style}.  Omitting the
\texttt{/lemmon} qualifier results in a tree-style proof (see
p.~\pageref{treeproof} for an example) that is somewhat less explicit but
easier to follow once you get used to it.\index{tree-style
proof}\index{proof!tree-style}

The first number on each line is the step
number of the proof.  Any numbers that follow are step numbers assigned to the
hypotheses of the statement referenced by that step.  Next is the label of
the statement referenced by the step.  The statement type of the statement
referenced comes next, followed by the math symbol\index{math symbol} string
constructed by the proof up to that step.

The last step, 34, contains the statement that is being proved.

Looking at a small piece of the proof, notice that steps 3 and 4 have
established that
\texttt{( t + 0 )} and \texttt{t} are \texttt{term}\,s, and step 5 makes use of steps 3 and
4 to establish that \texttt{( t + 0 ) = t} is a \texttt{wff}.  Let us let Metamath
itself tell us in detail what is happening in step 5.  Note that the
``target hypothesis'' refers to where step 5 is eventually used, i.e., in step
34.
\begin{verbatim}
        MM> show proof th1 /detailed_step 5
        Proof step 5:  wp=weq $a wff ( t + 0 ) = t
        This step assigns source "weq" ($a) to target "wp"
        ($f).  The source assertion requires the hypotheses
        "tt" ($f, step 3) and "tr" ($f, step 4).  The parent
        assertion of the target hypothesis is "mp" ($a,
        step 34).
        The source assertion before substitution was:
            weq $a wff t = r
        The following substitutions were made to the source
        assertion:
            Variable  Substituted with
             t         ( t + 0 )
             r         t
        The target hypothesis before substitution was:
            wp $f wff P
        The following substitution was made to the target
        hypothesis:
            Variable  Substituted with
             P         ( t + 0 ) = t
\end{verbatim}

The full proof just shown is useful to understand what is going on in detail.
However, most of the time you will just be interested in
the ``essential'' or logical steps of a proof, i.e.\ those steps
that correspond to an
ordinary formal proof\index{formal proof}.  If you type
\begin{verbatim}
        MM> show proof th1 /essential /lemmon /renumber
\end{verbatim}
you will see\label{demoproof}
\begin{verbatim}
        1 a2             $a |- ( t + 0 ) = t
        2 a2             $a |- ( t + 0 ) = t
        3 a1             $a |- ( ( t + 0 ) = t -> ( ( t + 0 )
                                             = t -> t = t ) )
        4 2,3 mp         $a |- ( ( t + 0 ) = t -> t = t )
        5 1,4 mp         $a |- t = t
\end{verbatim}
Compare this to the formal proof on p.~\pageref{zeroproof} and
notice the resemblance.  The \texttt{/essential} qualifier in the \texttt{show
proof} command tells Metamath to discard all \texttt{\$f}\index{\texttt{\$f}
statement} hypotheses and everything branching off of them in the proof tree
when the proof is displayed; this makes the proof look more like an ordinary
mathematical proof, which does not normally incorporate the explicit
construction of expressions.  The \texttt{/renumber} qualifier means to renumber
the steps to correspond only to what is displayed.\index{\texttt{show proof}
command}

To exit Metamath, type\index{\texttt{exit} command}
\begin{verbatim}
        MM> exit
\end{verbatim}

\subsection{Some Hints for Using the Command Line Interface}

We will conclude this quick introduction to Metamath\index{Metamath} with some
helpful hints on how to navigate your way through the commands.
\index{command line interface (CLI)}

When you type commands into Metamath's CLI, you only have to type as many
characters of a command keyword\index{command keyword} as are needed to make
it unambiguous.  If you type too few characters, Metamath will tell you what
the choices are.  In the case of the \texttt{read} command, only the \texttt{r} is
needed to specify it unambiguously, so you could have typed\index{\texttt{read}
command}
\begin{verbatim}
        MM> r demo0.mm
\end{verbatim}
instead of
\begin{verbatim}
        MM> read demo0.mm
\end{verbatim}
In our description, we always show the full command words.  When using the
Metamath CLI commands in a command file (to be read with the \texttt{submit}
command)\index{\texttt{submit} command}, it is good practice to use
the unabbreviated command to ensure your instructions will not become ambiguous
if more commands are added to the Metamath program in the future.

The command keywords\index{command
keyword} are not case sensitive; you may type either \texttt{read} or
\texttt{ReAd}.  File names may or may not be case sensitive, depending on your
computer's operating system.  Metamath label\index{label} and math
symbol\index{math symbol} tokens\index{token} are case-sensitive.

The \texttt{help} command\index{\texttt{help} command} will provide you
with a list of topics you can get help on.  You can then type
\texttt{help} {\em topic} to get help on that topic.

If you are uncertain of a command's spelling, just type as many characters
as you remember of the command.  If you have not typed enough characters to
specify it unambiguously, Metamath will tell you what choices you have.

\begin{verbatim}
        MM> show s
                 ^
        ?Ambiguous keyword - please specify SETTINGS,
        STATEMENT, or SOURCE.
\end{verbatim}

If you don't what know argument to use as part of a command, type a
\texttt{?}\index{\texttt{]}@\texttt{?}\ in command lines}\ at the
argument position.  Metamath will tell you what it expected there.

\begin{verbatim}
        MM> show ?
                 ^
        ?Expected SETTINGS, LABELS, STATEMENT, SOURCE, PROOF,
        MEMORY, TRACE_BACK, or USAGE.
\end{verbatim}

Finally, you may type just the first word or words of a command followed
by {\em return}.  Metamath will prompt you for the remaining part of the
command, showing you the choices at each step.  For example, instead of
typing \texttt{show statement th1 /full} you could interact in the
following manner:
\begin{verbatim}
        MM> show
        SETTINGS, LABELS, STATEMENT, SOURCE, PROOF,
        MEMORY, TRACE_BACK, or USAGE <SETTINGS>? st
        What is the statement label <th1>?
        / or nothing <nothing>? /
        TEX, COMMENT_ONLY, or FULL <TEX>? f
        / or nothing <nothing>?
        19 th1 $p |- t = t $= ... $.
\end{verbatim}
After each \texttt{?}\ in this mode, you must give Metamath the
information it requests.  Sometimes Metamath gives you a list of choices
with the default choice indicated by brackets \texttt{< > }. Pressing
{\em return} after the \texttt{?}\ will select the default choice.
Answering anything else will override the default.  Note that the
\texttt{/} in command qualifiers is considered a separate
token\index{token} by the parser, and this is why it is asked for
separately.

\section{Your First Proof}\label{frstprf}

Proofs are developed with the aid of the Proof Assistant\index{Proof
Assistant}.  We will now show you how the proof of theorem \texttt{th1}
was built.  So that you can repeat these steps, we will first have the
Proof Assistant erase the proof in Metamath's source buffer\index{source
buffer}, then reconstruct it.  (The source buffer is the place in memory
where Metamath stores the information in the database when it is
\texttt{read}\index{\texttt{read} command} in.  New or modified proofs
are kept in the source buffer until a \texttt{write source}
command\index{\texttt{write source} command} is issued.)  In practice, you
would place a \texttt{?}\index{\texttt{]}@\texttt{?}\ inside proofs}\
between \texttt{\$=}\index{\texttt{\$=} keyword} and
\texttt{\$.}\index{\texttt{\$.}\ keyword}\ in the database to indicate
to Metamath\index{Metamath} that the proof is unknown, and that would be
your starting point.  Whenever the \texttt{verify proof} command encounters
a proof with a \texttt{?}\ in place of a proof step, the statement is
identified as not proved.

Before you start, you should write down on a piece of paper the complete
formal proof\index{formal proof} as it would appear with the \texttt{/essential}
qualifier in a \texttt{show proof} command\index{\texttt{show proof} command}; see
the display of \texttt{show proof th1 /essential /lemmon /re\-num\-ber} above as an
example.  After you get used to using the Proof Assistant\index{Proof
Assistant} you may get to a point where you can ``see'' the proof in your mind
and let the Proof Assistant guide you in filling in the details, at least for
simpler proofs, but until you gain that experience it is important to write
down all the details in advance.  Otherwise you may waste a lot of time as you
let it take you down a wrong path.

A proof is developed with the Proof Assistant by working backwards, starting
with the theorem\index{theorem} to be proved, and assigning each unknown step
with a theorem or hypothesis until no more unknown steps remain.  The Proof
Assistant will not let you make an assignment unless it can be ``unified''
with the unknown step.  This means that a
substitution\index{substitution!variable}\index{variable substitution} of
variables exists that will make the assignment match the unknown step.  On the
other hand, in the middle of a proof, when working backwards, often more than
one unification\index{unification} (set of substitutions) is possible, since
there is not enough information available at that point to uniquely establish
it.  In this case you can tell Metamath which unification to choose, or you
can continue to assign unknown steps until enough information is available to
make the unification unique.

We will assume you have entered Metamath and read in the database as described
above.  The following dialog shows how the proof was developed.  For more
details on what some of the commands do, refer to Section~\ref{pfcommands}.
\index{\texttt{prove} command}

\begin{verbatim}
MM> prove th1
Entering the Proof Assistant.  Type HELP for help, EXIT
to exit.  You will be working on the proof of statement th1:
  $p |- t = t
Note:  The proof you are starting with is already complete.
MM-PA>
\end{verbatim}

The \verb/MM-PA>/ prompt means we are inside of the Proof
Assistant.\index{Proof Assistant} Most of the regular Metamath commands
(\texttt{show statement}, etc.) are still available if you need them.

\begin{verbatim}
MM-PA> delete all
The entire proof was deleted.
\end{verbatim}

We have deleted the whole proof so we can start from scratch.

\begin{verbatim}
MM-PA> show new_proof/lemmon/all
1 ?              $? |- t = t
\end{verbatim}

The \texttt{show new{\char`\_}proof} command\index{\texttt{show
new{\char`\_}proof} command} is like \texttt{show proof} except that we
don't specify a statement; instead, the proof we're working on is
displayed.

\begin{verbatim}
MM-PA> assign 1 mp
To undo the assignment, DELETE STEP 5 and INITIALIZE, UNIFY
if needed.
3   min=?  $? |- $2
4   maj=?  $? |- ( $2 -> t = t )
\end{verbatim}

The \texttt{assign} command\index{\texttt{assign} command} above means
``assign step 1 with the statement whose label is \texttt{mp}.''  Note
that step renumbering will constantly occur as you assign steps in the
middle of a proof; in general all steps from the step you assign until
the end of the proof will get moved up.  In this case, what used to be
step 1 is now step 5, because the (partial) proof now has five steps:
the four hypotheses of the \texttt{mp} statement and the \texttt{mp}
statement itself.  Let's look at all the steps in our partial proof:

\begin{verbatim}
MM-PA> show new_proof/lemmon/all
1 ?              $? wff $2
2 ?              $? wff t = t
3 ?              $? |- $2
4 ?              $? |- ( $2 -> t = t )
5 1,2,3,4 mp     $a |- t = t
\end{verbatim}

The symbol \texttt{\$2} is a temporary variable\index{temporary
variable} that represents a symbol sequence not yet known.  In the final
proof, all temporary variables will be eliminated.  The general format
for a temporary variable is \texttt{\$} followed by an integer.  Note
that \texttt{\$} is not a legal character in a math symbol (see
Section~\ref{dollardollar}, p.~\pageref{dollardollar}), so there will
never be a naming conflict between real symbols and temporary variables.

Unknown steps 1 and 2 are constructions of the two wffs used by the
modus ponens rule.  As you will see at the end of this section, the
Proof Assistant\index{Proof Assistant} can usually figure these steps
out by itself, and we will not have to worry about them.  Therefore from
here on we will display only the ``essential'' hypotheses, i.e.\ those
steps that correspond to traditional formal proofs\index{formal proof}.

\begin{verbatim}
MM-PA> show new_proof/lemmon/essential
3 ?              $? |- $2
4 ?              $? |- ( $2 -> t = t )
5 3,4 mp         $a |- t = t
\end{verbatim}

Unknown steps 3 and 4 are the ones we must focus on.  They correspond to the
minor and major premises of the modus ponens rule.  We will assign them as
follows.  Notice that because of the step renumbering that takes place
after an assignment, it is advantageous to assign unknown steps in reverse
order, because earlier steps will not get renumbered.

\begin{verbatim}
MM-PA> assign 4 mp
To undo the assignment, DELETE STEP 8 and INITIALIZE, UNIFY
if needed.
3   min=?  $? |- $2
6     min=?  $? |- $4
7     maj=?  $? |- ( $4 -> ( $2 -> t = t ) )
\end{verbatim}

We are now going to describe an obscure feature that you will probably
never use but should be aware of.  The Metamath language allows empty
symbol sequences to be substituted for variables, but in most formal
systems this feature is never used.  One of the few examples where is it
used is the MIU-system\index{MIU-system} described in
Appendix~\ref{MIU}.  But such systems are rare, and by default this
feature is turned off in the Proof Assistant.  (It is always allowed for
{\tt verify proof}.)  Let us turn it on and see what
happens.\index{\texttt{set empty{\char`\_}substitution} command}

\begin{verbatim}
MM-PA> set empty_substitution on
Substitutions with empty symbol sequences is now allowed.
\end{verbatim}

With this feature enabled, more unifications will be
ambiguous\index{ambiguous unification}\index{unification!ambiguous} in
the middle of a proof, because
substitution\index{substitution!variable}\index{variable substitution}
of variables with empty symbol sequences will become an additional
possibility.  Let's see what happens when we make our next assignment.

\begin{verbatim}
MM-PA> assign 3 a2
There are 2 possible unifications.  Please select the correct
    one or Q if you want to UNIFY later.
Unify:  |- $6
 with:  |- ( $9 + 0 ) = $9
Unification #1 of 2 (weight = 7):
  Replace "$6" with "( + 0 ) ="
  Replace "$9" with ""
  Accept (A), reject (R), or quit (Q) <A>? r
\end{verbatim}

The first choice presented is the wrong one.  If we had selected it,
temporary variable \texttt{\$6} would have been assigned a truncated
wff, and temporary variable \texttt{\$9} would have been assigned an
empty sequence (which is not allowed in our system).  With this choice,
eventually we would reach a point where we would get stuck because
we would end up with steps impossible to prove.  (You may want to
try it.)  We typed \texttt{r} to reject the choice.

\begin{verbatim}
Unification #2 of 2 (weight = 21):
  Replace "$6" with "( $9 + 0 ) = $9"
  Accept (A), reject (R), or quit (Q) <A>? q
To undo the assignment, DELETE STEP 4 and INITIALIZE, UNIFY
if needed.
 7     min=?  $? |- $8
 8     maj=?  $? |- ( $8 -> ( $6 -> t = t ) )
\end{verbatim}

The second choice is correct, and normally we would type \texttt{a}
to accept it.  But instead we typed \texttt{q} to show what will happen:
it will leave the step with an unknown unification, which can be
seen as follows:

\begin{verbatim}
MM-PA> show new_proof/not_unified
 4   min    $a |- $6
        =a2  = |- ( $9 + 0 ) = $9
\end{verbatim}

Later we can unify this with the \texttt{unify all/interactive} command.

The important point to remember is that occasionally you will be
presented with several unification choices while entering a proof, when
the program determines that there is not enough information yet to make
an unambiguous choice automatically (and this can happen even with
\texttt{set empty{\char`\_}substitution} turned off).  Usually it is
obvious by inspection which choice is correct, since incorrect ones will
tend to be meaningless fragments of wffs.  In addition, the correct
choice will usually be the first one presented, unlike our example
above.

Enough of this digression.  Let us go back to the default setting.

\begin{verbatim}
MM-PA> set empty_substitution off
The ability to substitute empty expressions for variables
has been turned off.  Note that this may make the Proof
Assistant too restrictive in some cases.
\end{verbatim}

If we delete the proof, start over, and get to the point where
we digressed above, there will no longer be an ambiguous unification.

\begin{verbatim}
MM-PA> assign 3 a2
To undo the assignment, DELETE STEP 4 and INITIALIZE, UNIFY
if needed.
 7     min=?  $? |- $4
 8     maj=?  $? |- ( $4 -> ( ( $5 + 0 ) = $5 -> t = t ) )
\end{verbatim}

Let us look at our proof so far, and continue.

\begin{verbatim}
MM-PA> show new_proof/lemmon
 4 a2            $a |- ( $5 + 0 ) = $5
 7 ?             $? |- $4
 8 ?             $? |- ( $4 -> ( ( $5 + 0 ) = $5 -> t = t ) )
 9 7,8 mp        $a |- ( ( $5 + 0 ) = $5 -> t = t )
10 4,9 mp        $a |- t = t
MM-PA> assign 8 a1
To undo the assignment, DELETE STEP 11 and INITIALIZE, UNIFY
if needed.
 7     min=?  $? |- ( t + 0 ) = t
MM-PA> assign 7 a2
To undo the assignment, DELETE STEP 8 and INITIALIZE, UNIFY
if needed.
MM-PA> show new_proof/lemmon
 4 a2            $a |- ( t + 0 ) = t
 8 a2            $a |- ( t + 0 ) = t
12 a1            $a |- ( ( t + 0 ) = t -> ( ( t + 0 ) = t ->
                                                    t = t ) )
13 8,12 mp       $a |- ( ( t + 0 ) = t -> t = t )
14 4,13 mp       $a |- t = t
\end{verbatim}

Now all temporary variables and unknown steps have been eliminated from the
``essential'' part of the proof.  When this is achieved, the Proof
Assistant\index{Proof Assistant} can usually figure out the rest of the proof
automatically.  (Note that the \texttt{improve} command can occasionally be
useful for filling in essential steps as well, but it only tries to make use
of statements that introduce no new variables in their hypotheses, which is
not the case for \texttt{mp}. Also it will not try to improve steps containing
temporary variables.)  Let's look at the complete proof, then run
the \texttt{improve} command, then look at it again.

\begin{verbatim}
MM-PA> show new_proof/lemmon/all
 1 ?             $? wff ( t + 0 ) = t
 2 ?             $? wff t = t
 3 ?             $? term t
 4 3 a2          $a |- ( t + 0 ) = t
 5 ?             $? wff ( t + 0 ) = t
 6 ?             $? wff ( ( t + 0 ) = t -> t = t )
 7 ?             $? term t
 8 7 a2          $a |- ( t + 0 ) = t
 9 ?             $? term ( t + 0 )
10 ?             $? term t
11 ?             $? term t
12 9,10,11 a1    $a |- ( ( t + 0 ) = t -> ( ( t + 0 ) = t ->
                                                    t = t ) )
13 5,6,8,12 mp   $a |- ( ( t + 0 ) = t -> t = t )
14 1,2,4,13 mp   $a |- t = t
\end{verbatim}

\begin{verbatim}
MM-PA> improve all
A proof of length 1 was found for step 11.
A proof of length 1 was found for step 10.
A proof of length 3 was found for step 9.
A proof of length 1 was found for step 7.
A proof of length 9 was found for step 6.
A proof of length 5 was found for step 5.
A proof of length 1 was found for step 3.
A proof of length 3 was found for step 2.
A proof of length 5 was found for step 1.
Steps 1 and above have been renumbered.
CONGRATULATIONS!  The proof is complete.  Use SAVE
NEW_PROOF to save it.  Note:  The Proof Assistant does
not detect $d violations.  After saving the proof, you
should verify it with VERIFY PROOF.
\end{verbatim}

The \texttt{save new{\char`\_}proof} command\index{\texttt{save
new{\char`\_}proof} command} will save the proof in the database.  Here
we will just display it in a form that can be clipped out of a log file
and inserted manually into the database source file with a text
editor.\index{normal proof}\index{proof!normal}

\begin{verbatim}
MM-PA> show new_proof/normal
---------Clip out the proof below this line:
      tt tze tpl tt weq tt tt weq tt a2 tt tze tpl tt weq
      tt tze tpl tt weq tt tt weq wim tt a2 tt tze tpl tt
      tt a1 mp mp $.
---------The proof of 'th1' to clip out ends above this line.
\end{verbatim}

There is another proof format called ``compressed''\index{compressed
proof}\index{proof!compressed} that you will see in databases.  It is
not important to understand how it is encoded but only to recognize it
when you see it.  Its only purpose is to reduce storage requirements for
large proofs.  A compressed proof can always be converted to a normal
one and vice-versa, and the Metamath \texttt{show proof}
commands\index{\texttt{show proof} command} work equally well with
compressed proofs.  The compressed proof format is described in
Appendix~\ref{compressed}.

\begin{verbatim}
MM-PA> show new_proof/compressed
---------Clip out the proof below this line:
      ( tze tpl weq a2 wim a1 mp ) ABCZADZAADZAEZJJKFLIA
      AGHH $.
---------The proof of 'th1' to clip out ends above this line.
\end{verbatim}

Now we will exit the Proof Assistant.  Since we made changes to the proof,
it will warn us that we have not saved it.  In this case, we don't care.

\begin{verbatim}
MM-PA> exit
Warning:  You have not saved changes to the proof.
Do you want to EXIT anyway (Y, N) <N>? y
Exiting the Proof Assistant.
Type EXIT again to exit Metamath.
\end{verbatim}

The Proof Assistant\index{Proof Assistant} has several other commands
that can help you while creating proofs.  See Section~\ref{pfcommands}
for a list of them.

A command that is often useful is \texttt{minimize{\char`\_}with
*/brief}, which tries to shorten the proof.  It can make the process
more efficient by letting you write a somewhat ``sloppy'' proof then
clean up some of the fine details of optimization for you (although it
can't perform miracles such as restructuring the overall proof).

\section{A Note About Editing a Data\-base File}

Once your source file contains proofs, there are some restrictions on
how you can edit it so that the proofs remain valid.  Pay particular
attention to these rules, since otherwise you can lose a lot of work.
It is a good idea to periodically verify all proofs with \texttt{verify
proof *} to ensure their integrity.

If your file contains only normal (as opposed to compressed) proofs, the
main rule is that you may not change the order of the mandatory
hypotheses\index{mandatory hypothesis} of any statement referenced in a
later proof.  For example, if you swap the order of the major and minor
premise in the modus ponens rule, all proofs making use of that rule
will become incorrect.  The \texttt{show statement}
command\index{\texttt{show statement} command} will show you the
mandatory hypotheses of a statement and their order.

If a statement has a compressed proof, you also must not change the
order of {\em its} mandatory hypotheses.  The compressed proof format
makes use of this information as part of the compression technique.
Note that swapping the names of two variables in a theorem will change
the order of its mandatory hypotheses.

The safest way to edit a statement, say \texttt{mytheorem}, is to
duplicate it then rename the original to \texttt{mytheoremOLD}
throughout the database.  Once the edited version is re-proved, all
statements referencing \texttt{mytheoremOLD} can be updated in the Proof
Assistant using \texttt{minimize{\char`\_}with
mytheorem
/allow{\char`\_}growth}.\index{\texttt{minimize{\char`\_}with} command}
% 3/10/07 Note: line-breaking the above results in duplicate index entries

\chapter{Abstract Mathematics Revealed}\label{fol}

\section{Logic and Set Theory}\label{logicandsettheory}

\begin{quote}
  {\em Set theory can be viewed as a form of exact theology.}
  \flushright\sc  Rudy Rucker\footnote{\cite{Barrow}, p.~31.}\\
\end{quote}\index{Rucker, Rudy}

Despite its seeming complexity, all of standard mathematics, no matter how
deep or abstract, can amazingly enough be derived from a relatively small set
of axioms\index{axiom} or first principles. The development of these axioms is
among the most impressive and important accomplishments of mathematics in the
20th century. Ultimately, these axioms can be broken down into a set of rules
for manipulating symbols that any technically oriented person can follow.

We will not spend much time trying to convey a deep, higher-level
understanding of the meaning of the axioms. This kind of understanding
requires some mathematical sophistication as well as an understanding of the
philosophy underlying the foundations of mathematics and typically develops
over time as you work with mathematics.  Our goal, instead, is to give you the
immediate ability to follow how theorems\index{theorem} are derived from the
axioms and from other theorems.  This will be similar to learning the syntax
of a computer language, which lets you follow the details in a program but
does not necessarily give you the ability to write non-trivial programs on
your own, an ability that comes with practice. For now don't be alarmed by
abstract-sounding names of the axioms; just focus on the rules for
manipulating the symbols, which follow the simple conventions of the
Metamath\index{Metamath} language.

The axioms that underlie all of standard mathematics consist of axioms of logic
and axioms of set theory. The axioms of logic are divided into two
subcategories, propositional calculus\index{propositional calculus} (sometimes
called sentential logic\index{sentential logic}) and predicate calculus
(sometimes called first-order logic\index{first-order logic}\index{quantifier
theory}\index{predicate calculus} or quantifier theory).  Propositional
calculus is a prerequisite for predicate calculus, and predicate calculus is a
prerequisite for set theory.  The version of set theory most commonly used is
Zermelo-Fraenkel set theory.\index{Zermelo-Fraenkel set theory}\index{set
theory}

Here in a nutshell is what the axioms are all about in an informal way. The
connection between this description and symbols we will show you won't be
immediately apparent and in principle needn't ever be.  Our description just
tries to summarize what mathematicians think about when they work with the
axioms.

Logic is more or less the translation of what we would consider common sense
into a rigorous set of axioms.\index{axioms of logic}  Suppose $\varphi$,
$\psi$, and $\chi$ (the Greek letters phi, psi, and chi) represent statements
that are either true or false, and $x$ is a variable\index{variable!in predicate
calculus} ranging over some group of mathematical objects (sets, integers,
real numbers, etc.). In mathematics, a ``statement'' really means a formula,
and $\psi$ could be for example ``$x = 2$.'' Logic makes assertions such as
``if $\varphi$ implies $\psi$ and $\psi$ implies $\chi$, then $\varphi$
implies $\chi$'' (propositional calculus)\index{propositional calculus} and
``if $\varphi$ is true for all $x$, then $\varphi$ is true for some $x$''
(predicate calculus).\index{predicate calculus}

Set theory\index{set theory} has to do with the manipulation of objects and
collections of objects, specifically the abstract, imaginary objects that
mathematics deals with, such as numbers. Everything that is claimed to exist
in mathematics is considered to be a set.  A set called the empty
set\index{empty set} contains nothing.  We represent the empty set by
$\varnothing$.  Many sets can be built up from the empty set.  There is a set
represented by $\{\varnothing\}$ that contains the empty set, another set
represented by $\{\varnothing,\{\varnothing\}\}$ that contains this set as
well as the empty set, another set represented by $\{\{\varnothing\}\}$ that
contains just the set that contains the empty set, and so on ad infinitum. All
mathematical objects, no matter how complex, are defined as being identical to
certain sets: the integer\index{integer} 0 is defined as the empty set, the
integer 1 is defined as $\{\varnothing\}$, the integer 2 is defined as
$\{\varnothing,\{\varnothing\}\}$.  (How these definitions were chosen doesn't
matter now, but the idea behind it is that these sets have the properties we
expect of integers once suitable operations are defined.)  Mathematical
operations, such as addition, are defined in terms of operations on
sets---their union\index{set union}, intersection\index{set intersection}, and
so on---operations you may have used in elementary school when you worked
with groups of apples and oranges.

With a leap of faith, the axioms also postulate the existence of infinite
sets\index{infinite set}, such as the set of all non-negative integers ($0, 1,
2,\ldots$, also called ``natural numbers''\index{natural number}).  This set
can't be represented with the brace notation\index{brace notation} we just
showed you, but requires a more complicated notation called ``class
abstraction.''\index{class abstraction}\index{abstraction class}  For
example, the infinite set $\{ x |
\mbox{``$x$ is a natural number''} \} $ means the ``set of all objects $x$
such that $x$ is a natural number'' i.e.\ the set of natural numbers; here,
``$x$ is a natural number'' is a rather complicated formula when broken down
into the primitive symbols.\label{expandom}\footnote{The statement ``$x$ is a
natural number'' is formally expressed as ``$x \in \omega$,'' where $\in$
(stylized epsilon) means ``is in'' or ``is an element of'' and $\omega$
(omega) means ``the set of natural numbers.''  When ``$x\in\omega$'' is
completely expanded in terms of the primitive symbols of set theory, the
result is  $\lnot$ $($ $\lnot$ $($ $\forall$ $z$ $($ $\lnot$ $\forall$ $w$ $($
$z$ $\in$ $w$ $\rightarrow$ $\lnot$ $w$ $\in$ $x$ $)$ $\rightarrow$ $z$ $\in$
$x$ $)$ $\rightarrow$ $($ $\forall$ $z$ $($ $\lnot$ $($ $\forall$ $w$ $($ $w$
$\in$ $z$ $\rightarrow$ $w$ $\in$ $x$ $)$ $\rightarrow$ $\forall$ $w$ $\lnot$
$w$ $\in$ $z$ $)$ $\rightarrow$ $\lnot$ $\forall$ $w$ $($ $w$ $\in$ $z$
$\rightarrow$ $\lnot$ $\forall$ $v$ $($ $v$ $\in$ $z$ $\rightarrow$ $\lnot$
$v$ $\in$ $w$ $)$ $)$ $)$ $\rightarrow$ $\lnot$ $\forall$ $z$ $\forall$ $w$
$($ $\lnot$ $($ $z$ $\in$ $x$ $\rightarrow$ $\lnot$ $w$ $\in$ $x$ $)$
$\rightarrow$ $($ $\lnot$ $z$ $\in$ $w$ $\rightarrow$ $($ $\lnot$ $z$ $=$ $w$
$\rightarrow$ $w$ $\in$ $z$ $)$ $)$ $)$ $)$ $)$ $\rightarrow$ $\lnot$
$\forall$ $y$ $($ $\lnot$ $($ $\lnot$ $($ $\forall$ $z$ $($ $\lnot$ $\forall$
$w$ $($ $z$ $\in$ $w$ $\rightarrow$ $\lnot$ $w$ $\in$ $y$ $)$ $\rightarrow$
$z$ $\in$ $y$ $)$ $\rightarrow$ $($ $\forall$ $z$ $($ $\lnot$ $($ $\forall$
$w$ $($ $w$ $\in$ $z$ $\rightarrow$ $w$ $\in$ $y$ $)$ $\rightarrow$ $\forall$
$w$ $\lnot$ $w$ $\in$ $z$ $)$ $\rightarrow$ $\lnot$ $\forall$ $w$ $($ $w$
$\in$ $z$ $\rightarrow$ $\lnot$ $\forall$ $v$ $($ $v$ $\in$ $z$ $\rightarrow$
$\lnot$ $v$ $\in$ $w$ $)$ $)$ $)$ $\rightarrow$ $\lnot$ $\forall$ $z$
$\forall$ $w$ $($ $\lnot$ $($ $z$ $\in$ $y$ $\rightarrow$ $\lnot$ $w$ $\in$
$y$ $)$ $\rightarrow$ $($ $\lnot$ $z$ $\in$ $w$ $\rightarrow$ $($ $\lnot$ $z$
$=$ $w$ $\rightarrow$ $w$ $\in$ $z$ $)$ $)$ $)$ $)$ $\rightarrow$ $($
$\forall$ $z$ $\lnot$ $z$ $\in$ $y$ $\rightarrow$ $\lnot$ $\forall$ $w$ $($
$\lnot$ $($ $w$ $\in$ $y$ $\rightarrow$ $\lnot$ $\forall$ $z$ $($ $w$ $\in$
$z$ $\rightarrow$ $\lnot$ $z$ $\in$ $y$ $)$ $)$ $\rightarrow$ $\lnot$ $($
$\lnot$ $\forall$ $z$ $($ $w$ $\in$ $z$ $\rightarrow$ $\lnot$ $z$ $\in$ $y$
$)$ $\rightarrow$ $w$ $\in$ $y$ $)$ $)$ $)$ $)$ $\rightarrow$ $x$ $\in$ $y$
$)$ $)$ $)$. Section~\ref{hierarchy} shows the hierarchy of definitions that
leads up to this expression.}\index{stylized epsilon ($\in$)}\index{omega
($\omega$)}  Actually, the primitive symbols don't even include the brace
notation.  The brace notation is a high-level definition, which you can find in
Section~\ref{hierarchy}.

Interestingly, the arithmetic of integers\index{integer} and
rationals\index{rational number} can be developed without appealing to the
existence of an infinite set, whereas the arithmetic of real
numbers\index{real number} requires it.

Each variable\index{variable!in set theory} in the axioms of set theory
represents an arbitrary set, and the axioms specify the legal kinds of things
you can do with these variables at a very primitive level.

Now, you may think that numbers and arithmetic are a lot more intuitive and
fundamental than sets and therefore should be the foundation of mathematics.
What is really the case is that you've dealt with numbers all your life and
are comfortable with a few rules for manipulating them such as addition and
multiplication.  Those rules only cover a small portion of what can be done
with numbers and only a very tiny fraction of the rest of mathematics.  If you
look at any elementary book on number theory, you will quickly become lost if
these are the only rules that you know.  Even though such books may present a
list of ``axioms''\index{axiom} for arithmetic, the ability to use the axioms
and to understand proofs of theorems\index{theorem} (facts) about numbers
requires an implicit mathematical talent that frustrates many people
from studying abstract mathematics.  The kind of mathematics that most people
know limits them to the practical, everyday usage of blindly manipulating
numbers and formulas, without any understanding of why those rules are correct
nor any ability to go any further.  For example, do you know why multiplying
two negative numbers yields a positive number?  Starting with set theory, you
will also start off blindly manipulating symbols according to the rules we give
you, but with the advantage that these rules will allow you, in principle, to
access {\em all} of mathematics, not just a tiny part of it.

Of course, concrete examples are often helpful in the learning process. For
example, you can verify that $2\cdot 3=3 \cdot 2$ by actually grouping
objects and can easily ``see'' how it generalizes to $x\cdot y = y\cdot x$,
even though you might not be able to rigorously prove it.  Similarly, in set
theory it can be helpful to understand how the axioms of set theory apply to
(and are correct for) small finite collections of objects.  You should be aware
that in set theory intuition can be misleading for infinite collections, and
rigorous proofs become more important.  For example, while $x\cdot y = y\cdot
x$ is correct for finite ordinals (which are the natural numbers), it is not
usually true for infinite ordinals.

\section{The Axioms for All of Mathematics}

In this section\index{axioms for mathematics}, we will show you the axioms
for all of standard mathematics (i.e.\ logic and set theory) as they are
traditionally presented.  The traditional presentation is useful for someone
with the mathematical experience needed to correctly manipulate high-level
abstract concepts.  For someone without this talent, knowing how to actually
make use of these axioms can be difficult.  The purpose of this section is to
allow you to see how the version of the axioms used in the standard
Metamath\index{Metamath} database \texttt{set.mm}\index{set
theory database (\texttt{set.mm})} relates to  the typical version
in textbooks, and also to give you an informal feel for them.

\subsection{Propositional Calculus}

Propositional calculus\index{propositional calculus} concerns itself with
statements that can be interpreted as either true or false.  Some examples of
statements (outside of mathematics) that are either true or false are ``It is
raining today'' and ``The United States has a female president.'' In
mathematics, as we mentioned, statements are really formulas.

In propositional calculus, we don't care what the statements are.  We also
treat a logical combination of statements, such as ``It is raining today and
the United States has a female president,'' no differently from a single
statement.  Statements and their combinations are called well-formed formulas
(wffs)\index{well-formed formula (wff)}.  We define wffs only in terms of
other wffs and don't define what a ``starting'' wff is.  As is common practice
in the literature, we use Greek letters to represent wffs.

Specifically, suppose $\varphi$ and $\psi$ are wffs.  Then the combinations
$\varphi\rightarrow\psi$ (``$\varphi$ implies $\psi$,'' also read ``if
$\varphi$ then $\psi$'')\index{implication ($\rightarrow$)} and $\lnot\varphi$
(``not $\varphi$'')\index{negation ($\lnot$)} are also wffs.

The three axioms of propositional calculus\index{axioms of propositional
calculus} are all wffs of the following form:\footnote{A remarkable result of
C.~A.~Meredith\index{Meredith, C. A.} squeezes these three axioms into the
single axiom $((((\varphi\rightarrow \psi)\rightarrow(\neg \chi\rightarrow\neg
\theta))\rightarrow \chi )\rightarrow \tau)\rightarrow((\tau\rightarrow
\varphi)\rightarrow(\theta\rightarrow \varphi))$ \cite{CAMeredith},
which is believed to be the shortest possible.}
\begin{center}
     $\varphi\rightarrow(\psi\rightarrow \varphi)$\\

     $(\varphi\rightarrow (\psi\rightarrow \chi))\rightarrow
((\varphi\rightarrow  \psi)\rightarrow (\varphi\rightarrow \chi))$\\

     $(\neg \varphi\rightarrow \neg\psi)\rightarrow (\psi\rightarrow
\varphi)$
\end{center}

These three axioms are widely used.
They are attributed to Jan {\L}ukasiewicz
(pronounced woo-kah-SHAY-vitch) and was popularized by Alonzo Church,
who called it system P2. (Thanks to Ted Ulrich for this information.)

There are an infinite number of axioms, one for each possible
wff\index{well-formed formula (wff)} of the above form.  (For this reason,
axioms such as the above are often called ``axiom schemes.''\index{axiom
scheme})  Each Greek letter in the axioms may be substituted with a more
complex wff to result in another axiom.  For example, substituting
$\neg(\varphi\rightarrow\chi)$ for $\varphi$ in the first axiom yields
$\neg(\varphi\rightarrow\chi)\rightarrow(\psi\rightarrow
\neg(\varphi\rightarrow\chi))$, which is still an axiom.

To deduce new true statements (theorems\index{theorem}) from the axioms, a
rule\index{rule} called ``modus ponens''\index{modus ponens} is used.  This
rule states that if the wff $\varphi$ is an axiom or a theorem, and the wff
$\varphi\rightarrow\psi$ is an axiom or a theorem, then the wff $\psi$ is also
a theorem\index{theorem}.

As a non-mathematical example of modus ponens, suppose we have proved (or
taken as an axiom) ``Bob is a man'' and separately have proved (or taken as
an axiom) ``If Bob is a man, then Bob is a human.''  Using the rule of modus
ponens, we can logically deduce, ``Bob is a human.''

From Metamath's\index{Metamath} point of view, the axioms and the rule of
modus ponens just define a mechanical means for deducing new true statements
from existing true statements, and that is the complete content of
propositional calculus as far as Metamath is concerned.  You can read a logic
textbook to gain a better understanding of their meaning, or you can just let
their meaning slowly become apparent to you after you use them for a while.

It is actually rather easy to check to see if a formula is a theorem of
propositional calculus.  Theorems of propositional calculus are also called
``tautologies.''\index{tautology}  The technique to check whether a formula is
a tautology is called the ``truth table method,''\index{truth table} and it
works like this.  A wff $\varphi\rightarrow\psi$ is false whenever $\varphi$ is true
and $\psi$ is false.  Otherwise it is true.  A wff $\lnot\varphi$ is false
whenever $\varphi$ is true and false otherwise. To verify a tautology such as
$\varphi\rightarrow(\psi\rightarrow \varphi)$, you break it down into sub-wffs and
construct a truth table that accounts for all possible combinations of true
and false assigned to the wff metavariables:
\begin{center}\begin{tabular}{|c|c|c|c|}\hline
\mbox{$\varphi$} & \mbox{$\psi$} & \mbox{$\psi\rightarrow\varphi$}
    & \mbox{$\varphi\rightarrow(\psi\rightarrow \varphi)$} \\ \hline \hline
              T   &  T    &      T       &        T    \\ \hline
              T   &  F    &      T       &        T    \\ \hline
              F   &  T    &      F       &        T    \\ \hline
              F   &  F    &      T       &        T    \\ \hline
\end{tabular}\end{center}
If all entries in the last column are true, the formula is a tautology.

Now, the truth table method doesn't tell you how to prove the tautology from
the axioms, but only that a proof exists.  Finding an actual proof (especially
one that is short and elegant) can be challenging.  Methods do exist for
automatically generating proofs in propositional calculus, but the proofs that
result can sometimes be very long.  In the Metamath \texttt{set.mm}\index{set
theory database (\texttt{set.mm})} database, most
or all proofs were created manually.

\subsection{Predicate Calculus}

Predicate calculus\index{predicate calculus} introduces the concept of
``individual variables,''\index{variable!in predicate calculus}\index{individual
variable} which
we will usually just call ``variables.''
These variables can represent something other than true or false (wffs),
and will always represent sets when we get to set theory.  There are also
three new symbols $\forall$\index{universal quantifier ($\forall$)},
$=$\index{equality ($=$)}, and $\in$\index{stylized epsilon ($\in$)},
read ``for all,'' ``equals,'' and ``is an element of''
respectively.  We will represent variables with the letters $x$, $y$, $z$, and
$w$, as is common practice in the literature.
For example, $\forall x \varphi$ means ``for all possible values of
$x$, $\varphi$ is true.''

In predicate calculus, we extend the definition of a wff\index{well-formed
formula (wff)}.  If $\varphi$ is a wff and $x$ and $y$ are variables, then
$\forall x \, \varphi$, $x=y$, and $x\in y$ are wffs. Note that these three new
types of wffs can be considered ``starting'' wffs from which we can build
other wffs with $\rightarrow$ and $\neg$ .  The concept of a starting wff was
absent in propositional calculus.  But starting wff or not, all we are really
concerned with is whether our wffs are correctly constructed according to
these mechanical rules.

A quick aside:
To prevent confusion, it might be best at this point to think of the variables
of Metamath\index{Metamath} as ``metavariables,''\index{metavariable} because
they are not quite the same as the variables we are introducing here.  A
(meta)variable in Metamath can be a wff or an individual variable, as well
as many other things; in general, it represents a kind of place holder for an
unspecified sequence of math symbols\index{math symbol}.

Unlike propositional calculus, no decision procedure\index{decision procedure}
analogous to the truth table method exists (nor theoretically can exist) that
will definitely determine whether a formula is a theorem of predicate
calculus.  Much of the work in the field of automated theorem
proving\index{automated theorem proving} has been dedicated to coming up with
clever heuristics for proving theorems of predicate calculus, but they can
never be guaranteed to work always.

We now turn to looking at how predicate calculus can be formally
represented.

\subsection{Common Axioms}

There is a new rule of inference in predicate calculus:  if $\varphi$ is
an axiom or a theorem, then $\forall x \,\varphi$ is also a
theorem\index{theorem}.  This is called the rule of
``generalization.''\index{rule of generalization}
This is easily represented in Metamath.

In standard texts of logic, there are often two axioms of predicate
calculus\index{axioms of predicate calculus}:
\begin{center}
  $\forall x \,\varphi ( x ) \rightarrow \varphi ( y )$,
      where ``$y$ is properly substituted for $x$.''\\
  $\forall x ( \varphi \rightarrow \psi )\rightarrow ( \varphi \rightarrow
    \forall x\, \psi )$,
    where ``$x$ is not free in $\varphi$.''
\end{center}

Now at first glance, this seems simple:  just two axioms.  However,
conditional clauses are attached to each axiom describing requirements that
may seem puzzling to you.  In addition, the first axiom puts a variable symbol
in parentheses after each wff, seemingly violating our definition of a
wff\index{well-formed formula (wff)}; this is just an informal way of
referring to some arbitrary variable that may occur in the wff.  The
conditional clauses do, of course, have a precise meaning, but as it turns out
the precise meaning is somewhat complicated and awkward to formalize in a
way that a computer can handle easily.  Unlike propositional calculus, a
certain amount of mathematical sophistication and practice is needed to be
able to easily grasp and manipulate these concepts correctly.

Predicate calculus may be presented with or without axioms for
equality\index{axioms of equality}\index{equality ($=$)}. We will require the
axioms of equality as a prerequisite for the version of set theory we will
use.  The axioms for equality, when included, are often represented using these
two axioms:
\begin{center}
$x=x$\\ \ \\
$x=y\rightarrow (\varphi(x,x)\rightarrow\varphi(x,y))$ where ``$\varphi(x,y)$
   arises from $\varphi(x,x)$ by replacing some, but not necessarily all,
   free\index{free variable}
   occurrences of $x$ by $y$,\\ provided that $y$ is free for $x$
   in $\phi(x,x)$.'' \end{center}
% (Mendelson p. 95)
The first equality axiom is simple, but again,
the condition on the second one is
somewhat awkward to implement on a computer.

\subsection{Tarski System S2}

Of course, we are not the first to notice the complications of these
predicate calculus axioms when being rigorous.

Well-known logician Alfred Tarski published in 1965
a system he called system S2\cite[p.~77]{Tarski1965}.
Tarski's system is \textit{exactly equivalent} to the traditional textbook
formalization, but (by clever use of equality axioms) it eliminates the
latter's primitive notions of ``proper substitution'' and ``free variable,''
replacing them with direct substitution and the notion of a variable
not occurring in a formula (which we express with distinct variable
constraints).

In advocating his system, Tarski wrote, ``The relatively complicated
character of [free variables and proper substitution] is a source
of certain inconveniences of both practical and theoretical nature;
this is clearly experienced both in teaching an elementary course of
mathematical logic and in formalizing the syntax of predicate logic for
some theoretical purposes''\cite[p.~61]{Tarski1965}\index{Tarski, Alfred}.

\subsection{Developing a Metamath Representation}

The standard textbook axioms of predicate calculus are somewhat
cumbersome to implement on a computer because of the complex notions of
``free variable''\index{free variable} and ``proper
substitution.''\index{proper substitution}\index{substitution!proper}
While it is possible to use the Metamath\index{Metamath} language to
implement these concepts, we have chosen not to implement them
as primitive constructs in the
\texttt{set.mm} set theory database.  Instead, we have eliminated them
within the axioms
by carefully crafting the axioms so as to avoid them,
building on Tarski's system S2.  This makes it
easy for a beginner to follow the steps in a proof without knowing any
advanced concepts other than the simple concept of
replacing\index{substitution!variable}\index{variable substitution}
variables with expressions.

In order to develop the concepts of free variable and proper
substitution from the axioms, we use an additional
Metamath statement type called ``disjoint variable
restriction''\index{disjoint variables} that we have not encountered
before.  In the context of the axioms, the statement \texttt{\$d} $ x\,
y$\index{\texttt{\$d} statement} simply means that $x$ and $y$ must be
distinct\index{distinct variables}, i.e.\ they may not be simultaneously
substituted\index{substitution!variable}\index{variable substitution}
with the same variable.  The statement \texttt{\$d} $ x\, \varphi$ means
variable $x$ must not occur in wff $\varphi$.  For the precise
definition of \texttt{\$d}, see Section~\ref{dollard}.

\subsection{Metamath representation}

The Metamath axiom system for predicate calculus
defined in set.mm uses Tarski's system S2.
As noted above, this has a different representation
than the traditional textbook formalization,
but it is \textit{exactly equivalent} to the textbook formalization,
and it is \textit{much} easier to work with.
This is reproduced as system S3 in Section 6 of
Megill's formalization\cite{Megill}\index{Megill, Norman}.

There is one exception, Tarski's axiom of existence,
which we label as axiom ax-6.
In the case of ax-6, Tarski's version is weaker because it includes a
distinct variable proviso. If we wish, we can also weaken our version
in this way and still have a metalogically complete system. Theorem
ax6 shows this by deriving, in the presence of the other axioms, our
ax-6 from Tarski's weaker version ax6v. However, we chose the stronger
version for our system because it is simpler to state and easier to use.

Tarski's system was designed for proving specific theorems rather than
more general theorem schemes. However, theorem schemes are much more
efficient than specific theorems for building a body of mathematical
knowledge, since they can be reused with different instances as
needed. While Tarski does derive some theorem schemes from his axioms,
their proofs require concepts that are ``outside'' of the system, such as
induction on formula length. The verification of such proofs is difficult
to automate in a proof verifier. (Specifically, Tarski treats the formulas
of his system as set-theoretical objects. In order to verify the proofs
of his theorem schemes, a proof verifier would need a significant amount
of set theory built into it.)

The Metamath axiom system for predicate calculus extends
Tarski's system to eliminate this difficulty. The additional
``auxilliary'' axiom
schemes (as we will call them in this section; see below) endow Tarski's
system with a nice property we call
metalogical completeness\cite[Remark 9.6]{Megill}\index{Megill, Norman}.
As a result, we can prove any theorem scheme
expressable in the ``simple metalogic'' of Tarski's system by using
only Metamath's direct substitution rule applied to the axiom system
(and no other metalogical or set-theoretical notions ``outside'' of the
system). Simple metalogic consists of schemes containing wff metavariables
(with no arguments) and/or set (also called ``individual'') metavariables,
accompanied by optional provisos each stating that two specified set
metavariables must be distinct or that a specified set metavariable may
not occur in a specified wff metavariable. Metamath's logic and set theory
axiom and rule schemes are all examples of simple metalogic. The schemes
of traditional predicate calculus with equality are examples which are
not simple metalogic, because they use wff metavariables with arguments
and have ``free for'' and ``not free in'' side conditions.

A rigorous justification for this system, using an older but
exactly equivalent set of axioms, can be
found in \cite{Megill}\index{Megill, Norman}.

This allows us to
take a different approach in the Metamath\index{Metamath} database
\texttt{set.mm}\index{set theory database (\texttt{set.mm})}.  We do not
directly use the primitive notions of ``free variable''\index{free variable}
and ``proper substitution''\index{proper
substitution}\index{substitution!proper} at all as primitive constructs.
Instead, we use a set
of axioms that are almost as simple to manipulate as those of
propositional calculus.  Our axiom system avoids complex primitive
notions by effectively embedding the complexity into the axioms
themselves.  As a result, we will end up with a larger number of axioms,
but they are ideally suited for a computer language such as Metamath.
(Section~\ref{metaaxioms} shows these axioms.)

We will not elaborate further
on the ``free variable'' and ``proper substitution''
concepts here.  You may consult
\cite[ch.\ 3--4]{Hamilton}\index{Hamilton, Alan G.} (as well as
many other books) for a precise explanation
of these concepts.  If you intend to do serious mathematical work, it is wise
to become familiar with the traditional textbook approach; even though the
concepts embedded in their axioms require a higher level of sophistication,
they can be more practical to deal with on an everyday, informal basis.  Even
if you are just developing Metamath proofs, familiarity with the traditional
approach can help you arrive at a proof outline much faster, which you can
then convert to the detail required by Metamath.

We do develop proper substitution rules later on, but in set.mm
they are defined as derived constructs; they are not primitives.

You should also note that our system of predicate calculus is specifically
tailored for set theory; thus there are only two specific predicates $=$ and
$\in$ and no functions\index{function!in predicate calculus}
or constants\index{constant!in predicate calculus} unlike more general systems.
We later add these.

\subsection{Set Theory}

Traditional Zermelo-Fraenkel set theory\index{Zermelo-Fraenkel set
theory}\index{set theory} has 10 axioms, which can be expressed in the
language of predicate calculus.  In this section, we will list only the
names and brief English descriptions of these axioms, since we will give
you the precise formulas used by the Metamath\index{Metamath} set theory
database \texttt{set.mm} later on.

In the descriptions of the axioms, we assume that $x$, $y$, $z$, $w$, and $v$
represent sets.  These are the same as the variables\index{variable!in set
theory} in our predicate calculus system above, except that now we informally
think of the variables as ranging over sets.  Note that the terms
``object,''\index{object} ``set,''\index{set} ``element,''\index{element}
``collection,''\index{collection} and ``family''\index{family} are synonymous,
as are ``is an element of,'' ``is a member of,''\index{member} ``is contained
in,'' and ``belongs to.''  The different terms are used for convenience; for
example, ``a collection of sets'' is less confusing than ``a set of sets.''
A set $x$ is said to be a ``subset''\index{subset} of $y$ if every element of
$x$ is also an element of $y$; we also say $x$ is ``included in''
$y$.

The axioms are very general and apply to almost any conceivable mathematical
object, and this level of abstraction can be overwhelming at first.  To gain an
intuitive feel, it can be helpful to draw a picture illustrating the concept;
for example, a circle containing dots could represent a collection of sets,
and a smaller circle drawn inside of the circle could represent a subset.
Overlapping circles can illustrate intersection and union.  Circles that
illustrate the concepts of set theory are frequently used in elementary
textbooks and are called Venn diagrams\index{Venn diagram}.\index{axioms of
set theory}

1. Axiom of Extensionality:  Two sets are identical if they contain the same
   elements.\index{Axiom of Extensionality}

2. Axiom of Pairing:  The set $\{ x , y \}$ exists.\index{Axiom of Pairing}

3. Axiom of Power Sets:  The power set of a set (the collection of all of
   its subsets) exists.  For example, the power set of $\{x,y\}$ is
   $\{\varnothing,\{x\},\{y\},\{x,y\}\}$ and it exists.\index{Axiom
of Power Sets}

4. Axiom of the Null Set:  The empty set $\varnothing$ exists.\index{Axiom of
the Null Set}

5. Axiom of Union:  The union of a set (the set containing the elements of
   its members) exists.  For example, the union of $\{\{x,y\},\{z\}\}$ is
 $\{x,y,z\}$ and
   it exists.\index{Axiom of Union}

6. Axiom of Regularity:  Roughly, no set can contain itself, nor can there
   be membership ``loops,'' such as a set being an
   element of one of its members.\index{Axiom of Regularity}

7. Axiom of Infinity:  An infinite set exists.  An example of an infinite
   set is the set of all
   integers.\index{Axiom of Infinity}

8. Axiom of Separation:  The set exists that is obtained by restricting $x$
   with some property.  For example, if the set of all integers exists,
   then the set of all even integers exists.\index{Axiom of Separation}

9. Axiom of Replacement:  The range of a function whose domain is restricted
   to the elements of a set $x$, is also a set.  For example, there
   is a function
   from integers (the function's domain) to their squares (its
   range).  If we
   restrict the domain to even integers, its range will become the set of
   squares of even integers, so this axiom asserts that the set of
    squares of even numbers exists.  Technical note:  In general, the
   ``function'' need not be a set but can be a proper class.
   \index{Axiom of Replacement}

10. Axiom of Choice:  Let $x$ be a set whose members are pairwise
  disjoint\index{disjoint sets} (i.e,
  whose members contain no elements in common).  Then there exists another
  set containing one element from each member of $x$.  For
  example, if $x$ is
  $\{\{y,z\},\{w,v\}\}$, where $y$, $z$, $w$, and $v$ are
  different sets, then a set such as $\{z,w\}$
  exists (but the axiom doesn't tell
  us which one).  (Actually the Axiom
  of Choice is redundant if the set $x$, as in this example, has a finite
  number of elements.)\index{Axiom of Choice}

The Axiom of Choice is usually considered an extension of ZF set theory rather
than a proper part of it.  It is sometimes considered philosophically
controversial because it specifies the existence of a set without specifying
what the set is.  ZF set theory that includes the Axiom of Choice is
called ZFC.\index{ZFC set theory}

When expressed symbolically, the Axiom of Separation and the Axiom of
Replacement contain wff symbols and therefore each represent infinitely many
axioms, one for each possible wff. For this reason, they are often called
axiom schemes\index{axiom scheme}\index{well-formed formula (wff)}.

It turns out that the Axiom of the Null Set, the Axiom of Pairing, and the
Axiom of Separation can be derived from the other axioms and are therefore
unnecessary, although they tend to be included in standard texts for various
reasons (historical, philosophical, and possibly because some authors may not
know this).  In the Metamath\index{Metamath} set theory database, these
redundant axioms are derived from the other ones.

\section{The Axioms in the Metamath Language}\label{metaaxioms}

Here we list the axioms as they appear in
\texttt{set.mm}\index{set theory database (\texttt{set.mm})} so you can
look them up there easily.  Incidentally, the \texttt{show statement
/tex} command\index{\texttt{show statement} command} was used to
typeset them.

%macros from show statement /tex
\newbox\mlinebox
\newbox\mtrialbox
\newbox\startprefix  % Prefix for first line of a formula
\newbox\contprefix  % Prefix for continuation line of a formula
\def\startm{  % Initialize formula line
  \setbox\mlinebox=\hbox{\unhcopy\startprefix}
}
\def\m#1{  % Add a symbol to the formula
  \setbox\mtrialbox=\hbox{\unhcopy\mlinebox $\,#1$}
  \ifdim\wd\mtrialbox>\hsize
    \box\mlinebox
    \setbox\mlinebox=\hbox{\unhcopy\contprefix $\,#1$}
  \else
    \setbox\mlinebox=\hbox{\unhbox\mtrialbox}
  \fi
}
\def\endm{  % Output the last line of a formula
  \box\mlinebox
}

\subsection{Propositional Calculus}\label{propcalc}\index{axioms of
propositional calculus}

Axiom of Simplification.\label{ax1}

\setbox\startprefix=\hbox{\tt \ \ ax-1\ \$a\ }
\setbox\contprefix=\hbox{\tt \ \ \ \ \ \ \ \ \ \ }
\startm
\m{\vdash}\m{(}\m{\varphi}\m{\rightarrow}\m{(}\m{\psi}\m{\rightarrow}\m{\varphi}\m{)}
\m{)}
\endm

\noindent Axiom of Distribution.

\setbox\startprefix=\hbox{\tt \ \ ax-2\ \$a\ }
\setbox\contprefix=\hbox{\tt \ \ \ \ \ \ \ \ \ \ }
\startm
\m{\vdash}\m{(}\m{(}\m{\varphi}\m{\rightarrow}\m{(}\m{\psi}\m{\rightarrow}\m{\chi}
\m{)}\m{)}\m{\rightarrow}\m{(}\m{(}\m{\varphi}\m{\rightarrow}\m{\psi}\m{)}\m{
\rightarrow}\m{(}\m{\varphi}\m{\rightarrow}\m{\chi}\m{)}\m{)}\m{)}
\endm

\noindent Axiom of Contraposition.

\setbox\startprefix=\hbox{\tt \ \ ax-3\ \$a\ }
\setbox\contprefix=\hbox{\tt \ \ \ \ \ \ \ \ \ \ }
\startm
\m{\vdash}\m{(}\m{(}\m{\lnot}\m{\varphi}\m{\rightarrow}\m{\lnot}\m{\psi}\m{)}\m{
\rightarrow}\m{(}\m{\psi}\m{\rightarrow}\m{\varphi}\m{)}\m{)}
\endm


\noindent Rule of Modus Ponens.\index{modus ponens}

\setbox\startprefix=\hbox{\tt \ \ maj\ \$e\ }
\setbox\contprefix=\hbox{\tt \ \ \ \ \ \ \ \ \ }
\startm
\m{\vdash}\m{(}\m{\varphi}\m{\rightarrow}\m{\psi}\m{)}
\endm

\setbox\startprefix=\hbox{\tt \ \ min\ \$e\ }
\setbox\contprefix=\hbox{\tt \ \ \ \ \ \ \ \ \ }
\startm
\m{\vdash}\m{\varphi}
\endm

\setbox\startprefix=\hbox{\tt \ \ ax-mp\ \$a\ }
\setbox\contprefix=\hbox{\tt \ \ \ \ \ \ \ \ \ \ \ }
\startm
\m{\vdash}\m{\psi}
\endm


\subsection{Axioms of Predicate Calculus with equality - Tarski's S2}\index{axioms of predicate calculus}

\noindent Rule of Generalization.\index{rule of generalization}

\setbox\startprefix=\hbox{\tt \ \ ax-g.1\ \$e\ }
\setbox\contprefix=\hbox{\tt \ \ \ \ \ \ \ \ \ \ \ \ }
\startm
\m{\vdash}\m{\varphi}
\endm

\setbox\startprefix=\hbox{\tt \ \ ax-gen\ \$a\ }
\setbox\contprefix=\hbox{\tt \ \ \ \ \ \ \ \ \ \ \ \ }
\startm
\m{\vdash}\m{\forall}\m{x}\m{\varphi}
\endm

\noindent Axiom of Quantified Implication.

\setbox\startprefix=\hbox{\tt \ \ ax-4\ \$a\ }
\setbox\contprefix=\hbox{\tt \ \ \ \ \ \ \ \ \ \ }
\startm
\m{\vdash}\m{(}\m{\forall}\m{x}\m{(}\m{\forall}\m{x}\m{\varphi}\m{\rightarrow}\m{
\psi}\m{)}\m{\rightarrow}\m{(}\m{\forall}\m{x}\m{\varphi}\m{\rightarrow}\m{
\forall}\m{x}\m{\psi}\m{)}\m{)}
\endm

\noindent Axiom of Distinctness.

% Aka: Add $d x ph $.
\setbox\startprefix=\hbox{\tt \ \ ax-5\ \$a\ }
\setbox\contprefix=\hbox{\tt \ \ \ \ \ \ \ \ \ \ }
\startm
\m{\vdash}\m{(}\m{\varphi}\m{\rightarrow}\m{\forall}\m{x}\m{\varphi}\m{)}\m{where}\m{ }\m{\$d}\m{ }\m{x}\m{ }\m{\varphi}\m{ }\m{(}\m{x}\m{ }\m{does}\m{ }\m{not}\m{ }\m{occur}\m{ }\m{in}\m{ }\m{\varphi}\m{)}
\endm

\noindent Axiom of Existence.

\setbox\startprefix=\hbox{\tt \ \ ax-6\ \$a\ }
\setbox\contprefix=\hbox{\tt \ \ \ \ \ \ \ \ \ \ }
\startm
\m{\vdash}\m{(}\m{\forall}\m{x}\m{(}\m{x}\m{=}\m{y}\m{\rightarrow}\m{\forall}
\m{x}\m{\varphi}\m{)}\m{\rightarrow}\m{\varphi}\m{)}
\endm

\noindent Axiom of Equality.

\setbox\startprefix=\hbox{\tt \ \ ax-7\ \$a\ }
\setbox\contprefix=\hbox{\tt \ \ \ \ \ \ \ \ \ \ }
\startm
\m{\vdash}\m{(}\m{x}\m{=}\m{y}\m{\rightarrow}\m{(}\m{x}\m{=}\m{z}\m{
\rightarrow}\m{y}\m{=}\m{z}\m{)}\m{)}
\endm

\noindent Axiom of Left Equality for Binary Predicate.

\setbox\startprefix=\hbox{\tt \ \ ax-8\ \$a\ }
\setbox\contprefix=\hbox{\tt \ \ \ \ \ \ \ \ \ \ \ }
\startm
\m{\vdash}\m{(}\m{x}\m{=}\m{y}\m{\rightarrow}\m{(}\m{x}\m{\in}\m{z}\m{
\rightarrow}\m{y}\m{\in}\m{z}\m{)}\m{)}
\endm

\noindent Axiom of Right Equality for Binary Predicate.

\setbox\startprefix=\hbox{\tt \ \ ax-9\ \$a\ }
\setbox\contprefix=\hbox{\tt \ \ \ \ \ \ \ \ \ \ \ }
\startm
\m{\vdash}\m{(}\m{x}\m{=}\m{y}\m{\rightarrow}\m{(}\m{z}\m{\in}\m{x}\m{
\rightarrow}\m{z}\m{\in}\m{y}\m{)}\m{)}
\endm


\subsection{Axioms of Predicate Calculus with Equality - Auxiliary}\index{axioms of predicate calculus - auxiliary}

\noindent Axiom of Quantified Negation.

\setbox\startprefix=\hbox{\tt \ \ ax-10\ \$a\ }
\setbox\contprefix=\hbox{\tt \ \ \ \ \ \ \ \ \ \ }
\startm
\m{\vdash}\m{(}\m{\lnot}\m{\forall}\m{x}\m{\lnot}\m{\forall}\m{x}\m{\varphi}\m{
\rightarrow}\m{\varphi}\m{)}
\endm

\noindent Axiom of Quantifier Commutation.

\setbox\startprefix=\hbox{\tt \ \ ax-11\ \$a\ }
\setbox\contprefix=\hbox{\tt \ \ \ \ \ \ \ \ \ \ }
\startm
\m{\vdash}\m{(}\m{\forall}\m{x}\m{\forall}\m{y}\m{\varphi}\m{\rightarrow}\m{
\forall}\m{y}\m{\forall}\m{x}\m{\varphi}\m{)}
\endm


\noindent Axiom of Substitution.

\setbox\startprefix=\hbox{\tt \ \ ax-12\ \$a\ }
\setbox\contprefix=\hbox{\tt \ \ \ \ \ \ \ \ \ \ \ }
\startm
\m{\vdash}\m{(}\m{\lnot}\m{\forall}\m{x}\m{\,x}\m{=}\m{y}\m{\rightarrow}\m{(}
\m{x}\m{=}\m{y}\m{\rightarrow}\m{(}\m{\varphi}\m{\rightarrow}\m{\forall}\m{x}\m{(}
\m{x}\m{=}\m{y}\m{\rightarrow}\m{\varphi}\m{)}\m{)}\m{)}\m{)}
\endm

\noindent Axiom of Quantified Equality.

\setbox\startprefix=\hbox{\tt \ \ ax-13\ \$a\ }
\setbox\contprefix=\hbox{\tt \ \ \ \ \ \ \ \ \ \ \ }
\startm
\m{\vdash}\m{(}\m{\lnot}\m{\forall}\m{z}\m{\,z}\m{=}\m{x}\m{\rightarrow}\m{(}
\m{\lnot}\m{\forall}\m{z}\m{\,z}\m{=}\m{y}\m{\rightarrow}\m{(}\m{x}\m{=}\m{y}
\m{\rightarrow}\m{\forall}\m{z}\m{\,x}\m{=}\m{y}\m{)}\m{)}\m{)}
\endm

% \noindent Axiom of Quantifier Substitution
%
% \setbox\startprefix=\hbox{\tt \ \ ax-c11n\ \$a\ }
% \setbox\contprefix=\hbox{\tt \ \ \ \ \ \ \ \ \ \ \ }
% \startm
% \m{\vdash}\m{(}\m{\forall}\m{x}\m{\,x}\m{=}\m{y}\m{\rightarrow}\m{(}\m{\forall}
% \m{x}\m{\varphi}\m{\rightarrow}\m{\forall}\m{y}\m{\varphi}\m{)}\m{)}
% \endm
%
% \noindent Axiom of Distinct Variables. (This axiom requires
% that two individual variables
% be distinct\index{\texttt{\$d} statement}\index{distinct
% variables}.)
%
% \setbox\startprefix=\hbox{\tt \ \ \ \ \ \ \ \ \$d\ }
% \setbox\contprefix=\hbox{\tt \ \ \ \ \ \ \ \ \ \ \ }
% \startm
% \m{x}\m{\,}\m{y}
% \endm
%
% \setbox\startprefix=\hbox{\tt \ \ ax-c16\ \$a\ }
% \setbox\contprefix=\hbox{\tt \ \ \ \ \ \ \ \ \ \ \ }
% \startm
% \m{\vdash}\m{(}\m{\forall}\m{x}\m{\,x}\m{=}\m{y}\m{\rightarrow}\m{(}\m{\varphi}\m{
% \rightarrow}\m{\forall}\m{x}\m{\varphi}\m{)}\m{)}
% \endm

% \noindent Axiom of Quantifier Introduction (2).  (This axiom requires
% that the individual variable not occur in the
% wff\index{\texttt{\$d} statement}\index{distinct variables}.)
%
% \setbox\startprefix=\hbox{\tt \ \ \ \ \ \ \ \ \$d\ }
% \setbox\contprefix=\hbox{\tt \ \ \ \ \ \ \ \ \ \ \ }
% \startm
% \m{x}\m{\,}\m{\varphi}
% \endm
% \setbox\startprefix=\hbox{\tt \ \ ax-5\ \$a\ }
% \setbox\contprefix=\hbox{\tt \ \ \ \ \ \ \ \ \ \ \ }
% \startm
% \m{\vdash}\m{(}\m{\varphi}\m{\rightarrow}\m{\forall}\m{x}\m{\varphi}\m{)}
% \endm

\subsection{Set Theory}\label{mmsettheoryaxioms}

In order to make the axioms of set theory\index{axioms of set theory} a little
more compact, there are several definitions from logic that we make use of
implicitly, namely, ``logical {\sc and},''\index{conjunction ($\wedge$)}
\index{logical {\sc and} ($\wedge$)} ``logical equivalence,''\index{logical
equivalence ($\leftrightarrow$)}\index{biconditional ($\leftrightarrow$)} and
``there exists.''\index{existential quantifier ($\exists$)}

\begin{center}\begin{tabular}{rcl}
  $( \varphi \wedge \psi )$ &\mbox{stands for}& $\neg ( \varphi
     \rightarrow \neg \psi )$\\
  $( \varphi \leftrightarrow \psi )$& \mbox{stands
     for}& $( ( \varphi \rightarrow \psi ) \wedge
     ( \psi \rightarrow \varphi ) )$\\
  $\exists x \,\varphi$ &\mbox{stands for}& $\neg \forall x \neg \varphi$
\end{tabular}\end{center}

In addition, the axioms of set theory require that all variables be
dis\-tinct,\index{distinct variables}\footnote{Set theory axioms can be
devised so that {\em no} variables are required to be distinct,
provided we replace \texttt{ax-c16} with an axiom stating that ``at
least two things exist,'' thus
making \texttt{ax-5} the only other axiom requiring the
\texttt{\$d} statement.  These axioms are unconventional and are not
presented here, but they can be found on the \url{http://metamath.org}
web site.  See also the Comment on
p.~\pageref{nodd}.}\index{\texttt{\$d} statement} thus we also assume:
\begin{center}
  \texttt{\$d }$x\,y\,z\,w$
\end{center}

\noindent Axiom of Extensionality.\index{Axiom of Extensionality}

\setbox\startprefix=\hbox{\tt \ \ ax-ext\ \$a\ }
\setbox\contprefix=\hbox{\tt \ \ \ \ \ \ \ \ \ \ \ \ }
\startm
\m{\vdash}\m{(}\m{\forall}\m{x}\m{(}\m{x}\m{\in}\m{y}\m{\leftrightarrow}\m{x}
\m{\in}\m{z}\m{)}\m{\rightarrow}\m{y}\m{=}\m{z}\m{)}
\endm

\noindent Axiom of Replacement.\index{Axiom of Replacement}

\setbox\startprefix=\hbox{\tt \ \ ax-rep\ \$a\ }
\setbox\contprefix=\hbox{\tt \ \ \ \ \ \ \ \ \ \ \ \ }
\startm
\m{\vdash}\m{(}\m{\forall}\m{w}\m{\exists}\m{y}\m{\forall}\m{z}\m{(}\m{%
\forall}\m{y}\m{\varphi}\m{\rightarrow}\m{z}\m{=}\m{y}\m{)}\m{\rightarrow}\m{%
\exists}\m{y}\m{\forall}\m{z}\m{(}\m{z}\m{\in}\m{y}\m{\leftrightarrow}\m{%
\exists}\m{w}\m{(}\m{w}\m{\in}\m{x}\m{\wedge}\m{\forall}\m{y}\m{\varphi}\m{)}%
\m{)}\m{)}
\endm

\noindent Axiom of Union.\index{Axiom of Union}

\setbox\startprefix=\hbox{\tt \ \ ax-un\ \$a\ }
\setbox\contprefix=\hbox{\tt \ \ \ \ \ \ \ \ \ \ \ }
\startm
\m{\vdash}\m{\exists}\m{x}\m{\forall}\m{y}\m{(}\m{\exists}\m{x}\m{(}\m{y}\m{
\in}\m{x}\m{\wedge}\m{x}\m{\in}\m{z}\m{)}\m{\rightarrow}\m{y}\m{\in}\m{x}\m{)}
\endm

\noindent Axiom of Power Sets.\index{Axiom of Power Sets}

\setbox\startprefix=\hbox{\tt \ \ ax-pow\ \$a\ }
\setbox\contprefix=\hbox{\tt \ \ \ \ \ \ \ \ \ \ \ \ }
\startm
\m{\vdash}\m{\exists}\m{x}\m{\forall}\m{y}\m{(}\m{\forall}\m{x}\m{(}\m{x}\m{
\in}\m{y}\m{\rightarrow}\m{x}\m{\in}\m{z}\m{)}\m{\rightarrow}\m{y}\m{\in}\m{x}
\m{)}
\endm

\noindent Axiom of Regularity.\index{Axiom of Regularity}

\setbox\startprefix=\hbox{\tt \ \ ax-reg\ \$a\ }
\setbox\contprefix=\hbox{\tt \ \ \ \ \ \ \ \ \ \ \ \ }
\startm
\m{\vdash}\m{(}\m{\exists}\m{x}\m{\,x}\m{\in}\m{y}\m{\rightarrow}\m{\exists}
\m{x}\m{(}\m{x}\m{\in}\m{y}\m{\wedge}\m{\forall}\m{z}\m{(}\m{z}\m{\in}\m{x}\m{
\rightarrow}\m{\lnot}\m{z}\m{\in}\m{y}\m{)}\m{)}\m{)}
\endm

\noindent Axiom of Infinity.\index{Axiom of Infinity}

\setbox\startprefix=\hbox{\tt \ \ ax-inf\ \$a\ }
\setbox\contprefix=\hbox{\tt \ \ \ \ \ \ \ \ \ \ \ \ \ \ \ }
\startm
\m{\vdash}\m{\exists}\m{x}\m{(}\m{y}\m{\in}\m{x}\m{\wedge}\m{\forall}\m{y}%
\m{(}\m{y}\m{\in}\m{x}\m{\rightarrow}\m{\exists}\m{z}\m{(}\m{y}\m{\in}\m{z}\m{%
\wedge}\m{z}\m{\in}\m{x}\m{)}\m{)}\m{)}
\endm

\noindent Axiom of Choice.\index{Axiom of Choice}

\setbox\startprefix=\hbox{\tt \ \ ax-ac\ \$a\ }
\setbox\contprefix=\hbox{\tt \ \ \ \ \ \ \ \ \ \ \ \ \ \ }
\startm
\m{\vdash}\m{\exists}\m{x}\m{\forall}\m{y}\m{\forall}\m{z}\m{(}\m{(}\m{y}\m{%
\in}\m{z}\m{\wedge}\m{z}\m{\in}\m{w}\m{)}\m{\rightarrow}\m{\exists}\m{w}\m{%
\forall}\m{y}\m{(}\m{\exists}\m{w}\m{(}\m{(}\m{y}\m{\in}\m{z}\m{\wedge}\m{z}%
\m{\in}\m{w}\m{)}\m{\wedge}\m{(}\m{y}\m{\in}\m{w}\m{\wedge}\m{w}\m{\in}\m{x}%
\m{)}\m{)}\m{\leftrightarrow}\m{y}\m{=}\m{w}\m{)}\m{)}
\endm

\subsection{That's It}

There you have it, the axioms for (essentially) all of mathematics!
Wonder at them and stare at them in awe.  Put a copy in your wallet, and
you will carry in your pocket the encoding for all theorems ever proved
and that ever will be proved, from the most mundane to the most
profound.

\section{A Hierarchy of Definitions}\label{hierarchy}

The axioms in the previous section in principle embody everything that can be
done within standard mathematics.  However, it is impractical to accomplish
very much by using them directly, for even simple concepts (from a human
perspective) can involve extremely long, incomprehensible formulas.
Mathematics is made practical by introducing definitions\index{definition}.
Definitions usually introduce new symbols, or at least new relationships among
existing symbols, to abbreviate more complex formulas.  An important
requirement for a definition is that there exist a straightforward
(algorithmic) method for eliminating the abbreviation by expanding it into the
more primitive symbol string that it represents.  Some
important definitions included in
the file \texttt{set.mm} are listed in this section for reference, and also to
give you a feel for why something like $\omega$\index{omega ($\omega$)} (the
set of natural numbers\index{natural number} 0, 1, 2,\ldots) becomes very
complicated when completely expanded into primitive symbols.

What is the motivation for definitions, aside from allowing complicated
expressions to be expressed more simply?  In the case of  $\omega$, one goal is
to provide a basis for the theory of natural numbers.\index{natural number}
Before set theory was invented, a set of axioms for arithmetic, called Peano's
postulates\index{Peano's postulates}, was devised and shown to have the
properties one expects for natural numbers.  Now anyone can postulate a
set of axioms, but if the axioms are inconsistent contradictions can be derived
from them.  Once a contradiction is derived, anything can be trivially
proved, including
all the facts of arithmetic and their negations.  To ensure that an
axiom system is at least as reliable as the axioms for set theory, we can
define sets and operations on those sets that satisfy the new axioms. In the
\texttt{set.mm} Metamath database, we prove that the elements of $\omega$ satisfy
Peano's postulates, and it's a long and hard journey to get there directly
from the axioms of set theory.  But the result is confidence in the
foundations of arithmetic.  And there is another advantage:  we now have all
the tools of set theory at our disposal for manipulating objects that obey the
axioms for arithmetic.

What are the criteria we use for definitions?  First, and of utmost importance,
the definition should not be {\em creative}\index{creative
definition}\index{definition!creative}, that
is it should not allow an expression that previously qualified as a wff but
was not provable, to become provable.   Second, the definition should be {\em
eliminable}\index{definition!eliminability}, that is there should exist an
algorithmic method for proving any expression using the definition into
a logically equivalent expression that previously qualified as a wff.

In almost all cases below, definitions connect two expressions with either
$\leftrightarrow$ or $=$.  Eliminating\footnote{Here we mean the
elimination that a human might do in his or her head.  To eliminate them as
part of a Metamath proof we would invoke one of a number of
theorems that deal with transitivity of equivalence or equality; there are
many such examples in the proofs in \texttt{set.mm}.} such a definition is a
simple matter of substituting the expression on the left-hand side ({\em
definiendum}\index{definiendum} or thing being defined) with the equivalent,
more primitive expression on the right-hand side ({\em
definiens}\index{definiens} or definition).

Often a definition has variables on the right-hand side which do not appear on
the left-hand side; these are called {\em dummy variables}.\index{dummy
variable!in definitions}  In this case, any
allowable substitution (such as a new, distinct
variable) can be used when the definition is eliminated.  Dummy variables may
be used only if they are {\em effectively bound}\index{effectively bound
variable}, meaning that the definition will remain logically equivalent upon
any substitution of a dummy variable with any other {\em qualifying
expression}\index{qualifying expression}, i.e.\ any symbol string (such as
another variable) that
meets the restrictions on the dummy variable imposed by \texttt{\$d} and
\texttt{\$f} statements.  For example, we could define a constant $\perp$
(inverted tee, meaning logical ``false'') as $( \varphi \wedge \lnot \varphi
)$, i.e.\ ``phi and not phi.''  Here $\varphi$ is effectively bound because the
definition remains logically equivalent when we replace $\varphi$ with any
other wff.  (We do not define $\perp$ in \texttt{set.mm}.)

There are two cases where eliminating definitions is a little more
complex.  These cases are the definitions \texttt{df-bi} and
\texttt{df-cleq}.  The first stretches the concept of a definition a
little, as in effect it ``defines a definition;'' however, it meets our
requirements for a definition in that it is eliminable and does not
strengthen the language.  Theorem \texttt{bii} shows the substitution
needed to eliminate the $\leftrightarrow$\index{logical equivalence
($\leftrightarrow$)}\index{biconditional ($\leftrightarrow$)} symbol.

Definition \texttt{df-cleq}\index{equality ($=$)} extends the usage of
the equality symbol to include ``classes''\index{class} in set theory.  The
reason it is potentially problematic is that it can lead to statements which
do not follow from logic alone but presuppose the Axiom of
Extensionality\index{Axiom of Extensionality}, so we include this axiom
as a hypothesis for the definition.  We could have made \texttt{df-cleq} directly
eliminable by introducing a new equality symbol, but have chosen not to do so
in keeping with standard textbook practice.  Definitions such as \texttt{df-cleq}
that extend the meaning of existing symbols must be introduced carefully so
that they do not lead to contradictions.  Definition \texttt{df-clel} also
extends the meaning of an existing symbol ($\in$); while it doesn't strengthen
the language like \texttt{df-cleq}, this is not obvious and it must also be
subject to the same scrutiny.

Exercise:  Study how the wff $x\in\omega$, meaning ``$x$ is a natural
number,'' could be expanded in terms of primitive symbols, starting with the
definitions \texttt{df-clel} on p.~\pageref{dfclel} and \texttt{df-om} on
p.~\pageref{dfom} and working your way back.  Don't bother to work out the
details; just make sure that you understand how you could do it in principle.
The answer is shown in the footnote on p.~\pageref{expandom}.  If you
actually do work it out, you won't get exactly the same answer because we used
a few simplifications such as discarding occurrences of $\lnot\lnot$ (double
negation).

In the definitions below, we have placed the {\sc ascii} Metamath source
below each of the formulas to help you to become familiar with the
notation in the database.  For simplicity, the necessary \texttt{\$f}
and \texttt{\$d} statements are not shown.  If you are in doubt, use the
\texttt{show statement}\index{\texttt{show statement} command} command
in the Metamath program to see the full statement.

To understand the motivation for these definitions, you should consult the
references indicated:  Takeuti and Zaring \cite{Takeuti}\index{Takeuti, Gaisi},
Quine \cite{Quine}\index{Quine, Willard Van Orman}, Bell and Machover
\cite{Bell}\index{Bell, J. L.}, and Enderton \cite{Enderton}\index{Enderton,
Herbert B.}.  Our list of definitions is provided more for reference than as a
learning aid.  However, by looking at a few of them you can gain a feel for
how the hierarchy is built up.  The definitions are a representative sample of
the 130 or so in \texttt{set.mm}, but they are complete with respect to the
theorem examples we will present in Section~\ref{sometheorems}.  Also, some are
slightly different from, but logically equivalent to, the ones in \texttt{set.mm}
(some of which have been revised over time to shorten them, for example).

\subsection{Definitions for Propositional Calculus}\label{metadefprop}

The symbols $\varphi$, $\psi$, and $\chi$ represent wffs.

Our first definition introduces the biconditional
connective\footnote{The term ``connective'' is informally used to mean a
symbol that is placed between two variables or adjacent to a variable,
whereas a mathematical ``constant'' usually indicates a symbol such as
the number 0 that may replace a variable or metavariable.  From
Metamath's point of view, there is no distinction between a connective
and a constant; both are constants in the Metamath
language.}\index{connective}\index{constant} (also called logical
equivalence)\index{logical equivalence
($\leftrightarrow$)}\index{biconditional ($\leftrightarrow$)}.  Unlike
most traditional developments, we have chosen not to have a separate
symbol such as ``Df.'' to mean ``is defined as.''  Instead, we will use
the biconditional connective for this purpose, as it lets us use
logic to manipulate definitions directly.  Here we state the properties
of the biconditional connective with a carefully crafted \texttt{\$a}
statement, which effectively uses the biconditional connective to define
itself.  The $\leftrightarrow$ symbol can be eliminated from a formula
using theorem \texttt{bii}, which is derived later.

\vskip 2ex
\noindent Define the biconditional connective.

\vskip 0.5ex
\setbox\startprefix=\hbox{\tt \ \ df-bi\ \$a\ }
\setbox\contprefix=\hbox{\tt \ \ \ \ \ \ \ \ \ \ \ }
\startm
\m{\vdash}\m{\lnot}\m{(}\m{(}\m{(}\m{\varphi}\m{\leftrightarrow}\m{\psi}\m{)}%
\m{\rightarrow}\m{\lnot}\m{(}\m{(}\m{\varphi}\m{\rightarrow}\m{\psi}\m{)}\m{%
\rightarrow}\m{\lnot}\m{(}\m{\psi}\m{\rightarrow}\m{\varphi}\m{)}\m{)}\m{)}\m{%
\rightarrow}\m{\lnot}\m{(}\m{\lnot}\m{(}\m{(}\m{\varphi}\m{\rightarrow}\m{%
\psi}\m{)}\m{\rightarrow}\m{\lnot}\m{(}\m{\psi}\m{\rightarrow}\m{\varphi}\m{)}%
\m{)}\m{\rightarrow}\m{(}\m{\varphi}\m{\leftrightarrow}\m{\psi}\m{)}\m{)}\m{)}
\endm

\noindent\verb?  df-bi $a |- -. ( ( ( ph <-> ps ) -> -. ( ( ph -> ps ) ->?

\noindent\verb?      -. ( ps -> ph ) ) ) -> -. ( -. ( ( ph -> ps ) -> -. (?

\noindent\verb?      ps -> ph ) ) -> ( ph <-> ps ) ) ) $.?
\vskip 1ex

\noindent This theorem relates the biconditional connective to primitive
connectives and can be used to eliminate the $\leftrightarrow$ symbol from any
wff.

\vskip 0.5ex
\setbox\startprefix=\hbox{\tt \ \ bii\ \$p\ }
\setbox\contprefix=\hbox{\tt \ \ \ \ \ \ \ \ \ }
\startm
\m{\vdash}\m{(}\m{(}\m{\varphi}\m{\leftrightarrow}\m{\psi}\m{)}\m{\leftrightarrow}
\m{\lnot}\m{(}\m{(}\m{\varphi}\m{\rightarrow}\m{\psi}\m{)}\m{\rightarrow}\m{\lnot}
\m{(}\m{\psi}\m{\rightarrow}\m{\varphi}\m{)}\m{)}\m{)}
\endm

\noindent\verb?  bii $p |- ( ( ph <-> ps ) <-> -. ( ( ph -> ps ) -> -. ( ps?

\noindent\verb?      -> ph ) ) ) $= ... $.?
\vskip 1ex


\noindent Define disjunction ({\sc or}).\index{disjunction ($\vee$)}
\index{logical {\sc or} ($\vee$)}

\vskip 0.5ex
\setbox\startprefix=\hbox{\tt \ \ df-or\ \$a\ }
\setbox\contprefix=\hbox{\tt \ \ \ \ \ \ \ \ \ \ \ }
\startm
\m{\vdash}\m{(}\m{(}\m{\varphi}\m{\vee}\m{\psi}\m{)}\m{\leftrightarrow}\m{(}\m{
\lnot}\m{\varphi}\m{\rightarrow}\m{\psi}\m{)}\m{)}
\endm

\noindent\verb?  df-or $a |- ( ( ph \/ ps ) <-> ( -. ph -> ps ) ) $.?
\vskip 1ex

\noindent Define conjunction ({\sc and}).\index{conjunction ($\wedge$)}
\index{logical {\sc and} ($\wedge$)}

\vskip 0.5ex
\setbox\startprefix=\hbox{\tt \ \ df-an\ \$a\ }
\setbox\contprefix=\hbox{\tt \ \ \ \ \ \ \ \ \ \ \ }
\startm
\m{\vdash}\m{(}\m{(}\m{\varphi}\m{\wedge}\m{\psi}\m{)}\m{\leftrightarrow}\m{\lnot}
\m{(}\m{\varphi}\m{\rightarrow}\m{\lnot}\m{\psi}\m{)}\m{)}
\endm

\noindent\verb?  df-an $a |- ( ( ph /\ ps ) <-> -. ( ph -> -. ps ) ) $.?
\vskip 1ex

\noindent Define disjunction ({\sc or}) of 3 wffs.

\vskip 0.5ex
\setbox\startprefix=\hbox{\tt \ \ df-3or\ \$a\ }
\setbox\contprefix=\hbox{\tt \ \ \ \ \ \ \ \ \ \ \ \ }
\startm
\m{\vdash}\m{(}\m{(}\m{\varphi}\m{\vee}\m{\psi}\m{\vee}\m{\chi}\m{)}\m{
\leftrightarrow}\m{(}\m{(}\m{\varphi}\m{\vee}\m{\psi}\m{)}\m{\vee}\m{\chi}\m{)}
\m{)}
\endm

\noindent\verb?  df-3or $a |- ( ( ph \/ ps \/ ch ) <-> ( ( ph \/ ps ) \/ ch?

\noindent\verb?      ) ) $.?
\vskip 1ex

\noindent Define conjunction ({\sc and}) of 3 wffs.

\vskip 0.5ex
\setbox\startprefix=\hbox{\tt \ \ df-3an\ \$a\ }
\setbox\contprefix=\hbox{\tt \ \ \ \ \ \ \ \ \ \ \ \ }
\startm
\m{\vdash}\m{(}\m{(}\m{\varphi}\m{\wedge}\m{\psi}\m{\wedge}\m{\chi}\m{)}\m{
\leftrightarrow}\m{(}\m{(}\m{\varphi}\m{\wedge}\m{\psi}\m{)}\m{\wedge}\m{\chi}
\m{)}\m{)}
\endm

\noindent\verb?  df-3an $a |- ( ( ph /\ ps /\ ch ) <-> ( ( ph /\ ps ) /\ ch?

\noindent\verb?      ) ) $.?
%\vskip 1ex

\subsection{Definitions for Predicate Calculus}\label{metadefpred}

The symbols $x$, $y$, and $z$ represent individual variables of predicate
calculus.  In this section, they are not necessarily distinct unless it is
explicitly
mentioned.

\vskip 2ex
\noindent Define existential quantification.\index{existential quantifier
($\exists$)}

\vskip 0.5ex
\setbox\startprefix=\hbox{\tt \ \ df-ex\ \$a\ }
\setbox\contprefix=\hbox{\tt \ \ \ \ \ \ \ \ \ \ \ }
\startm
\m{\vdash}\m{(}\m{\exists}\m{x}\m{\varphi}\m{\leftrightarrow}\m{\lnot}\m{\forall}
\m{x}\m{\lnot}\m{\varphi}\m{)}
\endm

\noindent\verb?  df-ex $a |- ( E. x ph <-> -. A. x -. ph ) $.?
\vskip 1ex

\noindent Define proper substitution.\index{proper
substitution}\index{substitution!proper}  Note that this
definition is valid even when
$x$ and $y$ are the same variable.  The first conjunct is a ``trick'' used to
achieve this property, making the definition look somewhat peculiar at
first.   For our notation, we use $[ y / x ] \varphi$ to mean ``the wff that
results when $y$ is properly substituted with $x$ in the wff $\varphi$.''  The
notation is different from the notation $\varphi (
x | y )$ that is sometimes used, because the latter notation is ambiguous
for us:
for example, we don't know whether $\lnot \varphi ( x | y )$ is to be
interpreted as $\lnot ( \varphi x | y )$ or $( \lnot \varphi x | y
)$.\footnote{Because of the way we initially defined wffs, this is the case
with any postfix connective\index{postfix connective} (one occurring after the
symbols being connected) or infix connective\index{infix connective} (one
occurring between the symbols being connected).  Metamath does not have a
built-in notion of operator binding strength that could eliminate the
ambiguity.  The initial parenthesis effectively provides a prefix
connective\index{prefix connective} to eliminate ambiguity.  Some conventions,
such as Polish notation\index{Polish notation} used in the 1930's and 1940's
by Polish logicians, use only prefix connectives and thus allow the total
elimination of parentheses, at the expense of readability.  In Metamath we
could actually redefine all notation to be Polish if we wanted to without
having to change any proofs!}  Other texts often use $\varphi(y)$ to indicate
our $[ y / x ] \varphi$, but this notation is even more ambiguous since there is
no explicit indication of what is being substituted.

\vskip 0.5ex
\setbox\startprefix=\hbox{\tt \ \ df-sb\ \$a\ }
\setbox\contprefix=\hbox{\tt \ \ \ \ \ \ \ \ \ \ \ }
\startm
\m{\vdash}\m{(}\m{[}\m{y}\m{/}\m{x}\m{]}\m{\varphi}\m{\leftrightarrow}\m{(}%
\m{(}\m{x}\m{=}\m{y}\m{\rightarrow}\m{\varphi}\m{)}\m{\wedge}\m{\exists}\m{x}%
\m{(}\m{x}\m{=}\m{y}\m{\wedge}\m{\varphi}\m{)}\m{)}\m{)}
\endm

\noindent\verb?  df-sb $a |- ( [ y / x ] ph <-> ( ( x = y -> ph ) /\ E. x (?

\noindent\verb?      x = y /\ ph ) ) ) $.?
\vskip 1ex


\noindent Define existential uniqueness\index{existential uniqueness
quantifier ($\exists "!$)} (``there exists exactly one'').  Note that $y$ is a
variable distinct from $x$ and not occurring in $\varphi$.

\vskip 0.5ex
\setbox\startprefix=\hbox{\tt \ \ df-eu\ \$a\ }
\setbox\contprefix=\hbox{\tt \ \ \ \ \ \ \ \ \ \ \ }
\startm
\m{\vdash}\m{(}\m{\exists}\m{{!}}\m{x}\m{\varphi}\m{\leftrightarrow}\m{\exists}
\m{y}\m{\forall}\m{x}\m{(}\m{\varphi}\m{\leftrightarrow}\m{x}\m{=}\m{y}\m{)}\m{)}
\endm

\noindent\verb?  df-eu $a |- ( E! x ph <-> E. y A. x ( ph <-> x = y ) ) $.?
%\vskip 1ex

\subsection{Definitions for Set Theory}\label{setdefinitions}

The symbols $x$, $y$, $z$, and $w$ represent individual variables of
predicate calculus, which in set theory are understood to be sets.  To
make set theory more practical, we introduce the notion of a ``class.''
A class\index{class} is either a set variable (such as $x$) or an
expression of the form $\{ x | \varphi\}$ (called an ``abstraction
class''\index{abstraction class}\index{class abstraction}).  Note that
sets (i.e.\ individual variables) always exist (this is a theorem of
logic, namely $\exists y \, y = x$ for any set $x$), whereas classes may
or may not exist (i.e.\ $\exists y \, y = A$ may or may not be true).
If a class does not exist it is called a ``proper class.''\index{proper
class}\index{class!proper} Definitions \texttt{df-clab},
\texttt{df-cleq}, and \texttt{df-clel} can be used to convert an
expression containing classes into one containing only set variables and
wff metavariables.

The symbols $A$, $B$, $C$, $D$, $ F$, $G$, and $R$ are metavariables that range
over classes.  A class metavariable $A$ may be eliminated from a wff by
replacing it with $\{ x|\varphi\}$ where neither $x$ nor $\varphi$ occur in
the wff.

In this section, individual variables are always assumed to be distinct from
each other unless otherwise indicated.  In addition, dummy variables on the
right-hand side of a definition do not occur in the class and wff
metavariables in the definition.

The definitions we present here are a partial but self-contained
collection selected from several hundred that appear in the current
\texttt{set.mm} database.  They are adequate for a basic development of
elementary set theory.

\vskip 2ex
\noindent Define the abstraction class.\index{abstraction class}\index{class
abstraction}  $x$ and $y$
need not be distinct.  Definition 2.1 of Quine, p.~16.  This definition may
seem puzzling since it is shorter than the expression being defined and does not
buy us anything in terms of brevity.  The reason we introduce this definition
is because it fits in neatly with the extension of the $\in$ connective
provided by \texttt{df-clel}.

\vskip 0.5ex
\setbox\startprefix=\hbox{\tt \ \ df-clab\ \$a\ }
\setbox\contprefix=\hbox{\tt \ \ \ \ \ \ \ \ \ \ \ \ \ }
\startm
\m{\vdash}\m{(}\m{x}\m{\in}\m{\{}\m{y}\m{|}\m{\varphi}\m{\}}\m{%
\leftrightarrow}\m{[}\m{x}\m{/}\m{y}\m{]}\m{\varphi}\m{)}
\endm

\noindent\verb?  df-clab $a |- ( x e. { y | ph } <-> [ x / y ] ph ) $.?
\vskip 1ex

\noindent Define the equality connective between classes\index{class
equality}.  See Quine or Chapter 4 of Takeuti and Zaring for its
justification and methods for eliminating it.  This is an example of a
somewhat ``dangerous'' definition, because it extends the use of the
existing equality symbol rather than introducing a new symbol, allowing
us to make statements in the original language that may not be true.
For example, it permits us to deduce $y = z \leftrightarrow \forall x (
x \in y \leftrightarrow x \in z )$ which is not a theorem of logic but
rather presupposes the Axiom of Extensionality,\index{Axiom of
Extensionality} which we include as a hypothesis so that we can know
when this axiom is assumed in a proof (with the \texttt{show
trace{\char`\_}back} command).  We could avoid the danger by introducing
another symbol, say $\eqcirc$, in place of $=$; this
would also have the advantage of making elimination of the definition
straightforward and would eliminate the need for Extensionality as a
hypothesis.  We would then also have the advantage of being able to
identify exactly where Extensionality truly comes into play.  One of our
theorems would be $x \eqcirc y \leftrightarrow x = y$
by invoking Extensionality.  However in keeping with standard practice
we retain the ``dangerous'' definition.

\vskip 0.5ex
\setbox\startprefix=\hbox{\tt \ \ df-cleq.1\ \$e\ }
\setbox\contprefix=\hbox{\tt \ \ \ \ \ \ \ \ \ \ \ \ \ \ \ }
\startm
\m{\vdash}\m{(}\m{\forall}\m{x}\m{(}\m{x}\m{\in}\m{y}\m{\leftrightarrow}\m{x}
\m{\in}\m{z}\m{)}\m{\rightarrow}\m{y}\m{=}\m{z}\m{)}
\endm
\setbox\startprefix=\hbox{\tt \ \ df-cleq\ \$a\ }
\setbox\contprefix=\hbox{\tt \ \ \ \ \ \ \ \ \ \ \ \ \ }
\startm
\m{\vdash}\m{(}\m{A}\m{=}\m{B}\m{\leftrightarrow}\m{\forall}\m{x}\m{(}\m{x}\m{
\in}\m{A}\m{\leftrightarrow}\m{x}\m{\in}\m{B}\m{)}\m{)}
\endm

\vskip 1ex
\noindent\verb?  df-cleq.1 $e |- ( A. x ( x e. y <-> x e. z ) -> y = z ) $.?

\noindent\verb?  df-cleq $a |- ( A = B <-> A. x ( x e. A <-> x e. B ) ) $.?
\vskip 1ex

\noindent Define the membership connective between classes\index{class
membership}.  Theorem 6.3 of Quine, p.~41, which we adopt as a definition.
Note that it extends the use of the existing membership symbol, but unlike
{\tt df-cleq} it does not extend the set of valid wffs of logic when the class
metavariables are replaced with set variables.\label{dfclel}

\vskip 0.5ex
\setbox\startprefix=\hbox{\tt \ \ df-clel\ \$a\ }
\setbox\contprefix=\hbox{\tt \ \ \ \ \ \ \ \ \ \ \ \ \ }
\startm
\m{\vdash}\m{(}\m{A}\m{\in}\m{B}\m{\leftrightarrow}\m{\exists}\m{x}\m{(}\m{x}
\m{=}\m{A}\m{\wedge}\m{x}\m{\in}\m{B}\m{)}\m{)}
\endm

\noindent\verb?  df-clel $a |- ( A e. B <-> E. x ( x = A /\ x e. B ) ) $.?
\vskip 1ex

\noindent Define inequality.

\vskip 0.5ex
\setbox\startprefix=\hbox{\tt \ \ df-ne\ \$a\ }
\setbox\contprefix=\hbox{\tt \ \ \ \ \ \ \ \ \ \ \ }
\startm
\m{\vdash}\m{(}\m{A}\m{\ne}\m{B}\m{\leftrightarrow}\m{\lnot}\m{A}\m{=}\m{B}%
\m{)}
\endm

\noindent\verb?  df-ne $a |- ( A =/= B <-> -. A = B ) $.?
\vskip 1ex

\noindent Define restricted universal quantification.\index{universal
quantifier ($\forall$)!restricted}  Enderton, p.~22.

\vskip 0.5ex
\setbox\startprefix=\hbox{\tt \ \ df-ral\ \$a\ }
\setbox\contprefix=\hbox{\tt \ \ \ \ \ \ \ \ \ \ \ \ }
\startm
\m{\vdash}\m{(}\m{\forall}\m{x}\m{\in}\m{A}\m{\varphi}\m{\leftrightarrow}\m{%
\forall}\m{x}\m{(}\m{x}\m{\in}\m{A}\m{\rightarrow}\m{\varphi}\m{)}\m{)}
\endm

\noindent\verb?  df-ral $a |- ( A. x e. A ph <-> A. x ( x e. A -> ph ) ) $.?
\vskip 1ex

\noindent Define restricted existential quantification.\index{existential
quantifier ($\exists$)!restricted}  Enderton, p.~22.

\vskip 0.5ex
\setbox\startprefix=\hbox{\tt \ \ df-rex\ \$a\ }
\setbox\contprefix=\hbox{\tt \ \ \ \ \ \ \ \ \ \ \ \ }
\startm
\m{\vdash}\m{(}\m{\exists}\m{x}\m{\in}\m{A}\m{\varphi}\m{\leftrightarrow}\m{%
\exists}\m{x}\m{(}\m{x}\m{\in}\m{A}\m{\wedge}\m{\varphi}\m{)}\m{)}
\endm

\noindent\verb?  df-rex $a |- ( E. x e. A ph <-> E. x ( x e. A /\ ph ) ) $.?
\vskip 1ex

\noindent Define the universal class\index{universal class ($V$)}.  Definition
5.20, p.~21, of Takeuti and Zaring.

\vskip 0.5ex
\setbox\startprefix=\hbox{\tt \ \ df-v\ \$a\ }
\setbox\contprefix=\hbox{\tt \ \ \ \ \ \ \ \ \ \ }
\startm
\m{\vdash}\m{V}\m{=}\m{\{}\m{x}\m{|}\m{x}\m{=}\m{x}\m{\}}
\endm

\noindent\verb?  df-v $a |- V = { x | x = x } $.?
\vskip 1ex

\noindent Define the subclass\index{subclass}\index{subset} relationship
between two classes (called the subset relation if the classes are sets i.e.\
are not proper).  Definition 5.9 of Takeuti and Zaring, p.~17.

\vskip 0.5ex
\setbox\startprefix=\hbox{\tt \ \ df-ss\ \$a\ }
\setbox\contprefix=\hbox{\tt \ \ \ \ \ \ \ \ \ \ \ }
\startm
\m{\vdash}\m{(}\m{A}\m{\subseteq}\m{B}\m{\leftrightarrow}\m{\forall}\m{x}\m{(}
\m{x}\m{\in}\m{A}\m{\rightarrow}\m{x}\m{\in}\m{B}\m{)}\m{)}
\endm

\noindent\verb?  df-ss $a |- ( A (_ B <-> A. x ( x e. A -> x e. B ) ) $.?
\vskip 1ex

\noindent Define the union\index{union} of two classes.  Definition 5.6 of Takeuti and Zaring,
p.~16.

\vskip 0.5ex
\setbox\startprefix=\hbox{\tt \ \ df-un\ \$a\ }
\setbox\contprefix=\hbox{\tt \ \ \ \ \ \ \ \ \ \ \ }
\startm
\m{\vdash}\m{(}\m{A}\m{\cup}\m{B}\m{)}\m{=}\m{\{}\m{x}\m{|}\m{(}\m{x}\m{\in}
\m{A}\m{\vee}\m{x}\m{\in}\m{B}\m{)}\m{\}}
\endm

\noindent\verb?  df-un $a |- ( A u. B ) = { x | ( x e. A \/ x e. B ) } $.?
\vskip 1ex

\noindent Define the intersection\index{intersection} of two classes.  Definition 5.6 of
Takeuti and Zaring, p.~16.

\vskip 0.5ex
\setbox\startprefix=\hbox{\tt \ \ df-in\ \$a\ }
\setbox\contprefix=\hbox{\tt \ \ \ \ \ \ \ \ \ \ \ }
\startm
\m{\vdash}\m{(}\m{A}\m{\cap}\m{B}\m{)}\m{=}\m{\{}\m{x}\m{|}\m{(}\m{x}\m{\in}
\m{A}\m{\wedge}\m{x}\m{\in}\m{B}\m{)}\m{\}}
\endm

\noindent\verb?  df-in $a |- ( A i^i B ) = { x | ( x e. A /\ x e. B ) } $.?
\vskip 1ex

\noindent Define class difference\index{class difference}\index{set difference}.
Definition 5.12 of Takeuti and Zaring, p.~20.  Several notations are used in
the literature; we chose the $\setminus$ convention instead of a minus sign to
reserve the latter for later use in, e.g., arithmetic.

\vskip 0.5ex
\setbox\startprefix=\hbox{\tt \ \ df-dif\ \$a\ }
\setbox\contprefix=\hbox{\tt \ \ \ \ \ \ \ \ \ \ \ \ }
\startm
\m{\vdash}\m{(}\m{A}\m{\setminus}\m{B}\m{)}\m{=}\m{\{}\m{x}\m{|}\m{(}\m{x}\m{
\in}\m{A}\m{\wedge}\m{\lnot}\m{x}\m{\in}\m{B}\m{)}\m{\}}
\endm

\noindent\verb?  df-dif $a |- ( A \ B ) = { x | ( x e. A /\ -. x e. B ) } $.?
\vskip 1ex

\noindent Define the empty or null set\index{empty set}\index{null set}.
Compare  Definition 5.14 of Takeuti and Zaring, p.~20.

\vskip 0.5ex
\setbox\startprefix=\hbox{\tt \ \ df-nul\ \$a\ }
\setbox\contprefix=\hbox{\tt \ \ \ \ \ \ \ \ \ \ }
\startm
\m{\vdash}\m{\varnothing}\m{=}\m{(}\m{V}\m{\setminus}\m{V}\m{)}
\endm

\noindent\verb?  df-nul $a |- (/) = ( V \ V ) $.?
\vskip 1ex

\noindent Define power class\index{power set}\index{power class}.  Definition 5.10 of
Takeuti and Zaring, p.~17, but we also let it apply to proper classes.  (Note
that \verb$P~$ is the symbol for calligraphic P, the tilde
suggesting ``curly;'' see Appendix~\ref{ASCII}.)

\vskip 0.5ex
\setbox\startprefix=\hbox{\tt \ \ df-pw\ \$a\ }
\setbox\contprefix=\hbox{\tt \ \ \ \ \ \ \ \ \ \ \ }
\startm
\m{\vdash}\m{{\cal P}}\m{A}\m{=}\m{\{}\m{x}\m{|}\m{x}\m{\subseteq}\m{A}\m{\}}
\endm

\noindent\verb?  df-pw $a |- P~ A = { x | x (_ A } $.?
\vskip 1ex

\noindent Define the singleton of a class\index{singleton}.  Definition 7.1 of
Quine, p.~48.  It is well-defined for proper classes, although
it is not very meaningful in this case, where it evaluates to the empty
set.

\vskip 0.5ex
\setbox\startprefix=\hbox{\tt \ \ df-sn\ \$a\ }
\setbox\contprefix=\hbox{\tt \ \ \ \ \ \ \ \ \ \ \ }
\startm
\m{\vdash}\m{\{}\m{A}\m{\}}\m{=}\m{\{}\m{x}\m{|}\m{x}\m{=}\m{A}\m{\}}
\endm

\noindent\verb?  df-sn $a |- { A } = { x | x = A } $.?
\vskip 1ex

\noindent Define an unordered pair of classes\index{unordered pair}\index{pair}.  Definition
7.1 of Quine, p.~48.

\vskip 0.5ex
\setbox\startprefix=\hbox{\tt \ \ df-pr\ \$a\ }
\setbox\contprefix=\hbox{\tt \ \ \ \ \ \ \ \ \ \ \ }
\startm
\m{\vdash}\m{\{}\m{A}\m{,}\m{B}\m{\}}\m{=}\m{(}\m{\{}\m{A}\m{\}}\m{\cup}\m{\{}
\m{B}\m{\}}\m{)}
\endm

\noindent\verb?  df-pr $a |- { A , B } = ( { A } u. { B } ) $.?
\vskip 1ex

\noindent Define an unordered triple of classes\index{unordered triple}.  Definition of
Enderton, p.~19.

\vskip 0.5ex
\setbox\startprefix=\hbox{\tt \ \ df-tp\ \$a\ }
\setbox\contprefix=\hbox{\tt \ \ \ \ \ \ \ \ \ \ \ }
\startm
\m{\vdash}\m{\{}\m{A}\m{,}\m{B}\m{,}\m{C}\m{\}}\m{=}\m{(}\m{\{}\m{A}\m{,}\m{B}
\m{\}}\m{\cup}\m{\{}\m{C}\m{\}}\m{)}
\endm

\noindent\verb?  df-tp $a |- { A , B , C } = ( { A , B } u. { C } ) $.?
\vskip 1ex

\noindent Kuratowski's\index{Kuratowski, Kazimierz} ordered pair\index{ordered
pair} definition.  Definition 9.1 of Quine, p.~58. For proper classes it is
not meaningful but is well-defined for convenience.  (Note that \verb$<.$
stands for $\langle$ whereas \verb$<$ stands for $<$, and similarly for
\verb$>.$\,.)

\vskip 0.5ex
\setbox\startprefix=\hbox{\tt \ \ df-op\ \$a\ }
\setbox\contprefix=\hbox{\tt \ \ \ \ \ \ \ \ \ \ \ }
\startm
\m{\vdash}\m{\langle}\m{A}\m{,}\m{B}\m{\rangle}\m{=}\m{\{}\m{\{}\m{A}\m{\}}
\m{,}\m{\{}\m{A}\m{,}\m{B}\m{\}}\m{\}}
\endm

\noindent\verb?  df-op $a |- <. A , B >. = { { A } , { A , B } } $.?
\vskip 1ex

\noindent Define the union of a class\index{union}.  Definition 5.5, p.~16,
of Takeuti and Zaring.

\vskip 0.5ex
\setbox\startprefix=\hbox{\tt \ \ df-uni\ \$a\ }
\setbox\contprefix=\hbox{\tt \ \ \ \ \ \ \ \ \ \ \ \ }
\startm
\m{\vdash}\m{\bigcup}\m{A}\m{=}\m{\{}\m{x}\m{|}\m{\exists}\m{y}\m{(}\m{x}\m{
\in}\m{y}\m{\wedge}\m{y}\m{\in}\m{A}\m{)}\m{\}}
\endm

\noindent\verb?  df-uni $a |- U. A = { x | E. y ( x e. y /\ y e. A ) } $.?
\vskip 1ex

\noindent Define the intersection\index{intersection} of a class.  Definition 7.35,
p.~44, of Takeuti and Zaring.

\vskip 0.5ex
\setbox\startprefix=\hbox{\tt \ \ df-int\ \$a\ }
\setbox\contprefix=\hbox{\tt \ \ \ \ \ \ \ \ \ \ \ \ }
\startm
\m{\vdash}\m{\bigcap}\m{A}\m{=}\m{\{}\m{x}\m{|}\m{\forall}\m{y}\m{(}\m{y}\m{
\in}\m{A}\m{\rightarrow}\m{x}\m{\in}\m{y}\m{)}\m{\}}
\endm

\noindent\verb?  df-int $a |- |^| A = { x | A. y ( y e. A -> x e. y ) } $.?
\vskip 1ex

\noindent Define a transitive class\index{transitive class}\index{transitive
set}.  This should not be confused with a transitive relation, which is a different
concept.  Definition from p.~71 of Enderton, extended to classes.

\vskip 0.5ex
\setbox\startprefix=\hbox{\tt \ \ df-tr\ \$a\ }
\setbox\contprefix=\hbox{\tt \ \ \ \ \ \ \ \ \ \ \ }
\startm
\m{\vdash}\m{(}\m{\mbox{\rm Tr}}\m{A}\m{\leftrightarrow}\m{\bigcup}\m{A}\m{
\subseteq}\m{A}\m{)}
\endm

\noindent\verb?  df-tr $a |- ( Tr A <-> U. A (_ A ) $.?
\vskip 1ex

\noindent Define a notation for a general binary relation\index{binary
relation}.  Definition 6.18, p.~29, of Takeuti and Zaring, generalized to
arbitrary classes.  This definition is well-defined, although not very
meaningful, when classes $A$ and/or $B$ are proper.\label{dfbr}  The lack of
parentheses (or any other connective) is not ambiguous since we are defining
an atomic wff.

\vskip 0.5ex
\setbox\startprefix=\hbox{\tt \ \ df-br\ \$a\ }
\setbox\contprefix=\hbox{\tt \ \ \ \ \ \ \ \ \ \ \ }
\startm
\m{\vdash}\m{(}\m{A}\m{\,R}\m{\,B}\m{\leftrightarrow}\m{\langle}\m{A}\m{,}\m{B}
\m{\rangle}\m{\in}\m{R}\m{)}
\endm

\noindent\verb?  df-br $a |- ( A R B <-> <. A , B >. e. R ) $.?
\vskip 1ex

\noindent Define an abstraction class of ordered pairs\index{abstraction
class!of ordered
pairs}.  A special case of Definition 4.16, p.~14, of Takeuti and Zaring.  Note
that $x$, $y$, and $z$ must be distinct but that $x$ and $y$ may occur in $\varphi$.

\vskip 0.5ex
\setbox\startprefix=\hbox{\tt \ \ df-opab\ \$a\ }
\setbox\contprefix=\hbox{\tt \ \ \ \ \ \ \ \ \ \ \ \ \ }
\startm
\m{\vdash}\m{\{}\m{\langle}\m{x}\m{,}\m{y}\m{\rangle}\m{|}\m{\varphi}\m{\}}\m{=}
\m{\{}\m{z}\m{|}\m{\exists}\m{x}\m{\exists}\m{y}\m{(}\m{z}\m{=}\m{\langle}\m{x}
\m{,}\m{y}\m{\rangle}\m{\wedge}\m{\varphi}\m{)}\m{\}}
\endm

\noindent\verb?  df-opab $a |- { <. x , y >. | ph } = { z | E. x E. y ( z =?

\noindent\verb?      <. x , y >. /\ ph ) } $.?
\vskip 1ex

\noindent Define the epsilon relation\index{epsilon relation}.  Similar to Definition
6.22, p.~30, of Takeuti and Zaring.

\vskip 0.5ex
\setbox\startprefix=\hbox{\tt \ \ df-eprel\ \$a\ }
\setbox\contprefix=\hbox{\tt \ \ \ \ \ \ \ \ \ \ \ \ \ \ }
\startm
\m{\vdash}\m{E}\m{=}\m{\{}\m{\langle}\m{x}\m{,}\m{y}\m{\rangle}\m{|}\m{x}\m{
\in}\m{y}\m{\}}
\endm

\noindent\verb?  df-eprel $a |- E = { <. x , y >. | x e. y } $.?
\vskip 1ex

\noindent Define a founded relation\index{founded relation}.  $R$ is a founded
relation on $A$ iff\index{iff} (if and only if) each nonempty subset of $A$
has an ``$R$-minimal element.''  Similar to Definition 6.21, p.~30, of
Takeuti and Zaring.

\vskip 0.5ex
\setbox\startprefix=\hbox{\tt \ \ df-fr\ \$a\ }
\setbox\contprefix=\hbox{\tt \ \ \ \ \ \ \ \ \ \ \ }
\startm
\m{\vdash}\m{(}\m{R}\m{\,\mbox{\rm Fr}}\m{\,A}\m{\leftrightarrow}\m{\forall}\m{x}
\m{(}\m{(}\m{x}\m{\subseteq}\m{A}\m{\wedge}\m{\lnot}\m{x}\m{=}\m{\varnothing}
\m{)}\m{\rightarrow}\m{\exists}\m{y}\m{(}\m{y}\m{\in}\m{x}\m{\wedge}\m{(}\m{x}
\m{\cap}\m{\{}\m{z}\m{|}\m{z}\m{\,R}\m{\,y}\m{\}}\m{)}\m{=}\m{\varnothing}\m{)}
\m{)}\m{)}
\endm

\noindent\verb?  df-fr $a |- ( R Fr A <-> A. x ( ( x (_ A /\ -. x = (/) ) ->?

\noindent\verb?      E. y ( y e. x /\ ( x i^i { z | z R y } ) = (/) ) ) ) $.?
\vskip 1ex

\noindent Define a well-ordering\index{well-ordering}.  $R$ is a well-ordering of $A$ iff
it is founded on $A$ and the elements of $A$ are pairwise $R$-comparable.
Similar to Definition 6.24(2), p.~30, of Takeuti and Zaring.

\vskip 0.5ex
\setbox\startprefix=\hbox{\tt \ \ df-we\ \$a\ }
\setbox\contprefix=\hbox{\tt \ \ \ \ \ \ \ \ \ \ \ }
\startm
\m{\vdash}\m{(}\m{R}\m{\,\mbox{\rm We}}\m{\,A}\m{\leftrightarrow}\m{(}\m{R}\m{\,
\mbox{\rm Fr}}\m{\,A}\m{\wedge}\m{\forall}\m{x}\m{\forall}\m{y}\m{(}\m{(}\m{x}\m{
\in}\m{A}\m{\wedge}\m{y}\m{\in}\m{A}\m{)}\m{\rightarrow}\m{(}\m{x}\m{\,R}\m{\,y}
\m{\vee}\m{x}\m{=}\m{y}\m{\vee}\m{y}\m{\,R}\m{\,x}\m{)}\m{)}\m{)}\m{)}
\endm

\noindent\verb?  df-we $a |- ( R We A <-> ( R Fr A /\ A. x A. y ( ( x e.?

\noindent\verb?      A /\ y e. A ) -> ( x R y \/ x = y \/ y R x ) ) ) ) $.?
\vskip 1ex

\noindent Define the ordinal predicate\index{ordinal predicate}, which is true for a class
that is transitive and is well-ordered by the epsilon relation.  Similar to
definition on p.~468, Bell and Machover.

\vskip 0.5ex
\setbox\startprefix=\hbox{\tt \ \ df-ord\ \$a\ }
\setbox\contprefix=\hbox{\tt \ \ \ \ \ \ \ \ \ \ \ \ }
\startm
\m{\vdash}\m{(}\m{\mbox{\rm Ord}}\m{\,A}\m{\leftrightarrow}\m{(}
\m{\mbox{\rm Tr}}\m{\,A}\m{\wedge}\m{E}\m{\,\mbox{\rm We}}\m{\,A}\m{)}\m{)}
\endm

\noindent\verb?  df-ord $a |- ( Ord A <-> ( Tr A /\ E We A ) ) $.?
\vskip 1ex

\noindent Define class of all ordinal numbers\index{ordinal number}.  An ordinal number is
a set that satisfies the ordinal predicate.  Definition 7.11 of Takeuti and
Zaring, p.~38.

\vskip 0.5ex
\setbox\startprefix=\hbox{\tt \ \ df-on\ \$a\ }
\setbox\contprefix=\hbox{\tt \ \ \ \ \ \ \ \ \ \ \ }
\startm
\m{\vdash}\m{\,\mbox{\rm On}}\m{=}\m{\{}\m{x}\m{|}\m{\mbox{\rm Ord}}\m{\,x}
\m{\}}
\endm

\noindent\verb?  df-on $a |- On = { x | Ord x } $.?
\vskip 1ex

\noindent Define the limit ordinal predicate\index{limit ordinal}, which is true for a
non-empty ordinal that is not a successor (i.e.\ that is the union of itself).
Compare Bell and Machover, p.~471 and Exercise (1), p.~42 of Takeuti and
Zaring.

\vskip 0.5ex
\setbox\startprefix=\hbox{\tt \ \ df-lim\ \$a\ }
\setbox\contprefix=\hbox{\tt \ \ \ \ \ \ \ \ \ \ \ \ }
\startm
\m{\vdash}\m{(}\m{\mbox{\rm Lim}}\m{\,A}\m{\leftrightarrow}\m{(}\m{\mbox{
\rm Ord}}\m{\,A}\m{\wedge}\m{\lnot}\m{A}\m{=}\m{\varnothing}\m{\wedge}\m{A}
\m{=}\m{\bigcup}\m{A}\m{)}\m{)}
\endm

\noindent\verb?  df-lim $a |- ( Lim A <-> ( Ord A /\ -. A = (/) /\ A = U.?

\noindent\verb?      A ) ) $.?
\vskip 1ex

\noindent Define the successor\index{successor} of a class.  Definition 7.22 of Takeuti
and Zaring, p.~41.  Our definition is a generalization to classes, although it
is meaningless when classes are proper.

\vskip 0.5ex
\setbox\startprefix=\hbox{\tt \ \ df-suc\ \$a\ }
\setbox\contprefix=\hbox{\tt \ \ \ \ \ \ \ \ \ \ \ \ }
\startm
\m{\vdash}\m{\,\mbox{\rm suc}}\m{\,A}\m{=}\m{(}\m{A}\m{\cup}\m{\{}\m{A}\m{\}}
\m{)}
\endm

\noindent\verb?  df-suc $a |- suc A = ( A u. { A } ) $.?
\vskip 1ex

\noindent Define the class of natural numbers\index{natural number}\index{omega
($\omega$)}.  Compare Bell and Machover, p.~471.\label{dfom}

\vskip 0.5ex
\setbox\startprefix=\hbox{\tt \ \ df-om\ \$a\ }
\setbox\contprefix=\hbox{\tt \ \ \ \ \ \ \ \ \ \ \ }
\startm
\m{\vdash}\m{\omega}\m{=}\m{\{}\m{x}\m{|}\m{(}\m{\mbox{\rm Ord}}\m{\,x}\m{
\wedge}\m{\forall}\m{y}\m{(}\m{\mbox{\rm Lim}}\m{\,y}\m{\rightarrow}\m{x}\m{
\in}\m{y}\m{)}\m{)}\m{\}}
\endm

\noindent\verb?  df-om $a |- om = { x | ( Ord x /\ A. y ( Lim y -> x e. y )?

\noindent\verb?      ) } $.?
\vskip 1ex

\noindent Define the cross product\index{cross product} of two classes.  Definition 9.11
of Quine, p.~64.

\vskip 0.5ex
\setbox\startprefix=\hbox{\tt \ \ df-xp\ \$a\ }
\setbox\contprefix=\hbox{\tt \ \ \ \ \ \ \ \ \ \ \ }
\startm
\m{\vdash}\m{(}\m{A}\m{\times}\m{B}\m{)}\m{=}\m{\{}\m{\langle}\m{x}\m{,}\m{y}
\m{\rangle}\m{|}\m{(}\m{x}\m{\in}\m{A}\m{\wedge}\m{y}\m{\in}\m{B}\m{)}\m{\}}
\endm

\noindent\verb?  df-xp $a |- ( A X. B ) = { <. x , y >. | ( x e. A /\ y e. B?

\noindent\verb?      ) } $.?
\vskip 1ex

\noindent Define the domain\index{domain} of a class.  Definition 6.5(1) of
Takeuti and Zaring, p.~24.

\vskip 0.5ex
\setbox\startprefix=\hbox{\tt \ \ df-dm\ \$a\ }
\setbox\contprefix=\hbox{\tt \ \ \ \ \ \ \ \ \ \ \ }
\startm
\m{\vdash}\m{\,\mbox{\rm dom}}\m{A}\m{=}\m{\{}\m{x}\m{|}\m{\exists}\m{y}\m{
\langle}\m{x}\m{,}\m{y}\m{\rangle}\m{\in}\m{A}\m{\}}
\endm

\noindent\verb?  df-dm $a |- dom A = { x | E. y <. x , y >. e. A } $.?
\vskip 1ex

\noindent Define the range\index{range} of a class.  Definition 6.5(2) of
Takeuti and Zaring, p.~24.

\vskip 0.5ex
\setbox\startprefix=\hbox{\tt \ \ df-rn\ \$a\ }
\setbox\contprefix=\hbox{\tt \ \ \ \ \ \ \ \ \ \ \ }
\startm
\m{\vdash}\m{\,\mbox{\rm ran}}\m{A}\m{=}\m{\{}\m{y}\m{|}\m{\exists}\m{x}\m{
\langle}\m{x}\m{,}\m{y}\m{\rangle}\m{\in}\m{A}\m{\}}
\endm

\noindent\verb?  df-rn $a |- ran A = { y | E. x <. x , y >. e. A } $.?
\vskip 1ex

\noindent Define the restriction\index{restriction} of a class.  Definition
6.6(1) of Takeuti and Zaring, p.~24.

\vskip 0.5ex
\setbox\startprefix=\hbox{\tt \ \ df-res\ \$a\ }
\setbox\contprefix=\hbox{\tt \ \ \ \ \ \ \ \ \ \ \ \ }
\startm
\m{\vdash}\m{(}\m{A}\m{\restriction}\m{B}\m{)}\m{=}\m{(}\m{A}\m{\cap}\m{(}\m{B}
\m{\times}\m{V}\m{)}\m{)}
\endm

\noindent\verb?  df-res $a |- ( A |` B ) = ( A i^i ( B X. V ) ) $.?
\vskip 1ex

\noindent Define the image\index{image} of a class.  Definition 6.6(2) of
Takeuti and Zaring, p.~24.

\vskip 0.5ex
\setbox\startprefix=\hbox{\tt \ \ df-ima\ \$a\ }
\setbox\contprefix=\hbox{\tt \ \ \ \ \ \ \ \ \ \ \ \ }
\startm
\m{\vdash}\m{(}\m{A}\m{``}\m{B}\m{)}\m{=}\m{\,\mbox{\rm ran}}\m{\,(}\m{A}\m{
\restriction}\m{B}\m{)}
\endm

\noindent\verb?  df-ima $a |- ( A " B ) = ran ( A |` B ) $.?
\vskip 1ex

\noindent Define the composition\index{composition} of two classes.  Definition 6.6(3) of
Takeuti and Zaring, p.~24.

\vskip 0.5ex
\setbox\startprefix=\hbox{\tt \ \ df-co\ \$a\ }
\setbox\contprefix=\hbox{\tt \ \ \ \ \ \ \ \ \ \ \ \ }
\startm
\m{\vdash}\m{(}\m{A}\m{\circ}\m{B}\m{)}\m{=}\m{\{}\m{\langle}\m{x}\m{,}\m{y}\m{
\rangle}\m{|}\m{\exists}\m{z}\m{(}\m{\langle}\m{x}\m{,}\m{z}\m{\rangle}\m{\in}
\m{B}\m{\wedge}\m{\langle}\m{z}\m{,}\m{y}\m{\rangle}\m{\in}\m{A}\m{)}\m{\}}
\endm

\noindent\verb?  df-co $a |- ( A o. B ) = { <. x , y >. | E. z ( <. x , z?

\noindent\verb?      >. e. B /\ <. z , y >. e. A ) } $.?
\vskip 1ex

\noindent Define a relation\index{relation}.  Definition 6.4(1) of Takeuti and
Zaring, p.~23.

\vskip 0.5ex
\setbox\startprefix=\hbox{\tt \ \ df-rel\ \$a\ }
\setbox\contprefix=\hbox{\tt \ \ \ \ \ \ \ \ \ \ \ \ }
\startm
\m{\vdash}\m{(}\m{\mbox{\rm Rel}}\m{\,A}\m{\leftrightarrow}\m{A}\m{\subseteq}
\m{(}\m{V}\m{\times}\m{V}\m{)}\m{)}
\endm

\noindent\verb?  df-rel $a |- ( Rel A <-> A (_ ( V X. V ) ) $.?
\vskip 1ex

\noindent Define a function\index{function}.  Definition 6.4(4) of Takeuti and
Zaring, p.~24.

\vskip 0.5ex
\setbox\startprefix=\hbox{\tt \ \ df-fun\ \$a\ }
\setbox\contprefix=\hbox{\tt \ \ \ \ \ \ \ \ \ \ \ \ }
\startm
\m{\vdash}\m{(}\m{\mbox{\rm Fun}}\m{\,A}\m{\leftrightarrow}\m{(}
\m{\mbox{\rm Rel}}\m{\,A}\m{\wedge}
\m{\forall}\m{x}\m{\exists}\m{z}\m{\forall}\m{y}\m{(}
\m{\langle}\m{x}\m{,}\m{y}\m{\rangle}\m{\in}\m{A}\m{\rightarrow}\m{y}\m{=}\m{z}
\m{)}\m{)}\m{)}
\endm

\noindent\verb?  df-fun $a |- ( Fun A <-> ( Rel A /\ A. x E. z A. y ( <. x?

\noindent\verb?      , y >. e. A -> y = z ) ) ) $.?
\vskip 1ex

\noindent Define a function with domain.  Definition 6.15(1) of Takeuti and
Zaring, p.~27.

\vskip 0.5ex
\setbox\startprefix=\hbox{\tt \ \ df-fn\ \$a\ }
\setbox\contprefix=\hbox{\tt \ \ \ \ \ \ \ \ \ \ \ }
\startm
\m{\vdash}\m{(}\m{A}\m{\,\mbox{\rm Fn}}\m{\,B}\m{\leftrightarrow}\m{(}
\m{\mbox{\rm Fun}}\m{\,A}\m{\wedge}\m{\mbox{\rm dom}}\m{\,A}\m{=}\m{B}\m{)}
\m{)}
\endm

\noindent\verb?  df-fn $a |- ( A Fn B <-> ( Fun A /\ dom A = B ) ) $.?
\vskip 1ex

\noindent Define a function with domain and co-domain.  Definition 6.15(3)
of Takeuti and Zaring, p.~27.

\vskip 0.5ex
\setbox\startprefix=\hbox{\tt \ \ df-f\ \$a\ }
\setbox\contprefix=\hbox{\tt \ \ \ \ \ \ \ \ \ \ }
\startm
\m{\vdash}\m{(}\m{F}\m{:}\m{A}\m{\longrightarrow}\m{B}\m{
\leftrightarrow}\m{(}\m{F}\m{\,\mbox{\rm Fn}}\m{\,A}\m{\wedge}\m{
\mbox{\rm ran}}\m{\,F}\m{\subseteq}\m{B}\m{)}\m{)}
\endm

\noindent\verb?  df-f $a |- ( F : A --> B <-> ( F Fn A /\ ran F (_ B ) ) $.?
\vskip 1ex

\noindent Define a one-to-one function\index{one-to-one function}.  Compare
Definition 6.15(5) of Takeuti and Zaring, p.~27.

\vskip 0.5ex
\setbox\startprefix=\hbox{\tt \ \ df-f1\ \$a\ }
\setbox\contprefix=\hbox{\tt \ \ \ \ \ \ \ \ \ \ \ }
\startm
\m{\vdash}\m{(}\m{F}\m{:}\m{A}\m{
\raisebox{.5ex}{${\textstyle{\:}_{\mbox{\footnotesize\rm
1\tt -\rm 1}}}\atop{\textstyle{
\longrightarrow}\atop{\textstyle{}^{\mbox{\footnotesize\rm {\ }}}}}$}
}\m{B}
\m{\leftrightarrow}\m{(}\m{F}\m{:}\m{A}\m{\longrightarrow}\m{B}
\m{\wedge}\m{\forall}\m{y}\m{\exists}\m{z}\m{\forall}\m{x}\m{(}\m{\langle}\m{x}
\m{,}\m{y}\m{\rangle}\m{\in}\m{F}\m{\rightarrow}\m{x}\m{=}\m{z}\m{)}\m{)}\m{)}
\endm

\noindent\verb?  df-f1 $a |- ( F : A -1-1-> B <-> ( F : A --> B /\?

\noindent\verb?      A. y E. z A. x ( <. x , y >. e. F -> x = z ) ) ) $.?
\vskip 1ex

\noindent Define an onto function\index{onto function}.  Definition 6.15(4) of Takeuti and
Zaring, p.~27.

\vskip 0.5ex
\setbox\startprefix=\hbox{\tt \ \ df-fo\ \$a\ }
\setbox\contprefix=\hbox{\tt \ \ \ \ \ \ \ \ \ \ \ }
\startm
\m{\vdash}\m{(}\m{F}\m{:}\m{A}\m{
\raisebox{.5ex}{${\textstyle{\:}_{\mbox{\footnotesize\rm
{\ }}}}\atop{\textstyle{
\longrightarrow}\atop{\textstyle{}^{\mbox{\footnotesize\rm onto}}}}$}
}\m{B}
\m{\leftrightarrow}\m{(}\m{F}\m{\,\mbox{\rm Fn}}\m{\,A}\m{\wedge}
\m{\mbox{\rm ran}}\m{\,F}\m{=}\m{B}\m{)}\m{)}
\endm

\noindent\verb?  df-fo $a |- ( F : A -onto-> B <-> ( F Fn A /\ ran F?

\noindent\verb?      = B ) ) $.?
\vskip 1ex

\noindent Define a one-to-one, onto function.  Compare Definition 6.15(6) of
Takeuti and Zaring, p.~27.

\vskip 0.5ex
\setbox\startprefix=\hbox{\tt \ \ df-f1o\ \$a\ }
\setbox\contprefix=\hbox{\tt \ \ \ \ \ \ \ \ \ \ \ \ }
\startm
\m{\vdash}\m{(}\m{F}\m{:}\m{A}
\m{
\raisebox{.5ex}{${\textstyle{\:}_{\mbox{\footnotesize\rm
1\tt -\rm 1}}}\atop{\textstyle{
\longrightarrow}\atop{\textstyle{}^{\mbox{\footnotesize\rm onto}}}}$}
}
\m{B}
\m{\leftrightarrow}\m{(}\m{F}\m{:}\m{A}
\m{
\raisebox{.5ex}{${\textstyle{\:}_{\mbox{\footnotesize\rm
1\tt -\rm 1}}}\atop{\textstyle{
\longrightarrow}\atop{\textstyle{}^{\mbox{\footnotesize\rm {\ }}}}}$}
}
\m{B}\m{\wedge}\m{F}\m{:}\m{A}
\m{
\raisebox{.5ex}{${\textstyle{\:}_{\mbox{\footnotesize\rm
{\ }}}}\atop{\textstyle{
\longrightarrow}\atop{\textstyle{}^{\mbox{\footnotesize\rm onto}}}}$}
}
\m{B}\m{)}\m{)}
\endm

\noindent\verb?  df-f1o $a |- ( F : A -1-1-onto-> B <-> ( F : A -1-1-> B?

\noindent\verb?      /\ F : A -onto-> B ) ) $.?
\vskip 1ex

\noindent Define the value of a function\index{function value}.  This
definition applies to any class and evaluates to the empty set when it is not
meaningful. Note that $ F`A$ means the same thing as the more familiar $ F(A)$
notation for a function's value at $A$.  The $ F`A$ notation is common in
formal set theory.

\vskip 0.5ex
\setbox\startprefix=\hbox{\tt \ \ df-fv\ \$a\ }
\setbox\contprefix=\hbox{\tt \ \ \ \ \ \ \ \ \ \ \ }
\startm
\m{\vdash}\m{(}\m{F}\m{`}\m{A}\m{)}\m{=}\m{\bigcup}\m{\{}\m{x}\m{|}\m{(}\m{F}%
\m{``}\m{\{}\m{A}\m{\}}\m{)}\m{=}\m{\{}\m{x}\m{\}}\m{\}}
\endm

\noindent\verb?  df-fv $a |- ( F ` A ) = U. { x | ( F " { A } ) = { x } } $.?
\vskip 1ex

\noindent Define the result of an operation.\index{operation}  Here, $F$ is
     an operation on two
     values (such as $+$ for real numbers).   This is defined for proper
     classes $A$ and $B$ even though not meaningful in that case.  However,
     the definition can be meaningful when $F$ is a proper class.\label{dfopr}

\vskip 0.5ex
\setbox\startprefix=\hbox{\tt \ \ df-opr\ \$a\ }
\setbox\contprefix=\hbox{\tt \ \ \ \ \ \ \ \ \ \ \ \ }
\startm
\m{\vdash}\m{(}\m{A}\m{\,F}\m{\,B}\m{)}\m{=}\m{(}\m{F}\m{`}\m{\langle}\m{A}%
\m{,}\m{B}\m{\rangle}\m{)}
\endm

\noindent\verb?  df-opr $a |- ( A F B ) = ( F ` <. A , B >. ) $.?
%\vskip 1ex

\section{Tricks of the Trade}\label{tricks}

In the \texttt{set.mm}\index{set theory database (\texttt{set.mm})} database our goal
was usually to conform to modern notation.  However in some cases the
relationship to standard textbook language may be obscured by several
unconventional devices we used to simplify the development and to take
advantage of the Metamath language.  In this section we will describe some
common conventions used in \texttt{set.mm}.

\begin{itemize}
\item
The turnstile\index{turnstile ({$\,\vdash$})} symbol, $\vdash$, meaning ``it
is provable that,'' is the first token of all assertions and hypotheses that
aren't syntax constructions.  This is a standard convention in logic.  (We
mentioned this earlier, but this symbol is bothersome to some people without a
logic background.  It has no deeper meaning but just provides us with a way to
distinguish syntax constructions from ordinary mathematical statements.)

\item
A hypothesis of the form

\vskip 1ex
\setbox\startprefix=\hbox{\tt \ \ \ \ \ \ \ \ \ \$e\ }
\setbox\contprefix=\hbox{\tt \ \ \ \ \ \ \ \ \ \ }
\startm
\m{\vdash}\m{(}\m{\varphi}\m{\rightarrow}\m{\forall}\m{x}\m{\varphi}\m{)}
\endm
\vskip 1ex

should be read ``assume variable $x$ is (effectively) not free in wff
$\varphi$.''\index{effectively not free}
Literally, this says ``assume it is provable that $\varphi \rightarrow \forall
x\, \varphi$.''  This device lets us avoid the complexities associated with
the standard treatment of free and bound variables.
%% Uncomment this when uncommenting section {formalspec} below
The footnote on p.~\pageref{effectivelybound} discusses this further.

\item
A statement of one of the forms

\vskip 1ex
\setbox\startprefix=\hbox{\tt \ \ \ \ \ \ \ \ \ \$a\ }
\setbox\contprefix=\hbox{\tt \ \ \ \ \ \ \ \ \ \ }
\startm
\m{\vdash}\m{(}\m{\lnot}\m{\forall}\m{x}\m{\,x}\m{=}\m{y}\m{\rightarrow}
\m{\ldots}\m{)}
\endm
\setbox\startprefix=\hbox{\tt \ \ \ \ \ \ \ \ \ \$p\ }
\setbox\contprefix=\hbox{\tt \ \ \ \ \ \ \ \ \ \ }
\startm
\m{\vdash}\m{(}\m{\lnot}\m{\forall}\m{x}\m{\,x}\m{=}\m{y}\m{\rightarrow}
\m{\ldots}\m{)}
\endm
\vskip 1ex

should be read ``if $x$ and $y$ are distinct variables, then...''  This
antecedent provides us with a technical device to avoid the need for the
\texttt{\$d} statement early in our development of predicate calculus,
permitting symbol manipulations to be as conceptually simple as those in
propositional calculus.  However, the \texttt{\$d} statement eventually
becomes a requirement, and after that this device is rarely used.

\item
The statement

\vskip 1ex
\setbox\startprefix=\hbox{\tt \ \ \ \ \ \ \ \ \ \$d\ }
\setbox\contprefix=\hbox{\tt \ \ \ \ \ \ \ \ \ \ }
\startm
\m{x}\m{\,y}
\endm
\vskip 1ex

should be read ``assume $x$ and $y$ are distinct variables.''

\item
The statement

\vskip 1ex
\setbox\startprefix=\hbox{\tt \ \ \ \ \ \ \ \ \ \$d\ }
\setbox\contprefix=\hbox{\tt \ \ \ \ \ \ \ \ \ \ }
\startm
\m{x}\m{\,\varphi}
\endm
\vskip 1ex

should be read ``assume $x$ does not occur in $\varphi$.''

\item
The statement

\vskip 1ex
\setbox\startprefix=\hbox{\tt \ \ \ \ \ \ \ \ \ \$d\ }
\setbox\contprefix=\hbox{\tt \ \ \ \ \ \ \ \ \ \ }
\startm
\m{x}\m{\,A}
\endm
\vskip 1ex

should be read ``assume variable $x$ does not occur in class $A$.''

\item
The restriction and hypothesis group

\vskip 1ex
\setbox\startprefix=\hbox{\tt \ \ \ \ \ \ \ \ \ \$d\ }
\setbox\contprefix=\hbox{\tt \ \ \ \ \ \ \ \ \ \ }
\startm
\m{x}\m{\,A}
\endm
\setbox\startprefix=\hbox{\tt \ \ \ \ \ \ \ \ \ \$d\ }
\setbox\contprefix=\hbox{\tt \ \ \ \ \ \ \ \ \ \ }
\startm
\m{x}\m{\,\psi}
\endm
\setbox\startprefix=\hbox{\tt \ \ \ \ \ \ \ \ \ \$e\ }
\setbox\contprefix=\hbox{\tt \ \ \ \ \ \ \ \ \ \ }
\startm
\m{\vdash}\m{(}\m{x}\m{=}\m{A}\m{\rightarrow}\m{(}\m{\varphi}\m{\leftrightarrow}
\m{\psi}\m{)}\m{)}
\endm
\vskip 1ex

is frequently used in place of explicit substitution, meaning ``assume
$\psi$ results from the proper substitution of $A$ for $x$ in
$\varphi$.''  Sometimes ``\texttt{\$e} $\vdash ( \psi \rightarrow
\forall x \, \psi )$'' is used instead of ``\texttt{\$d} $x\, \psi $,''
which requires only that $x$ be effectively not free in $\varphi$ but
not necessarily absent from it.  The use of implicit
substitution\index{substitution!implicit} is partly a matter of personal
style, although it may make proofs somewhat shorter than would be the
case with explicit substitution.

\item
The hypothesis


\vskip 1ex
\setbox\startprefix=\hbox{\tt \ \ \ \ \ \ \ \ \ \$e\ }
\setbox\contprefix=\hbox{\tt \ \ \ \ \ \ \ \ \ \ }
\startm
\m{\vdash}\m{A}\m{\in}\m{V}
\endm
\vskip 1ex

should be read ``assume class $A$ is a set (i.e.\ exists).''
This is a convenient convention used by Quine.

\item
The restriction and hypothesis

\vskip 1ex
\setbox\startprefix=\hbox{\tt \ \ \ \ \ \ \ \ \ \$d\ }
\setbox\contprefix=\hbox{\tt \ \ \ \ \ \ \ \ \ \ }
\startm
\m{x}\m{\,y}
\endm
\setbox\startprefix=\hbox{\tt \ \ \ \ \ \ \ \ \ \$e\ }
\setbox\contprefix=\hbox{\tt \ \ \ \ \ \ \ \ \ \ }
\startm
\m{\vdash}\m{(}\m{y}\m{\in}\m{A}\m{\rightarrow}\m{\forall}\m{x}\m{\,y}
\m{\in}\m{A}\m{)}
\endm
\vskip 1ex

should be read ``assume variable $x$ is
(effectively) not free in class $A$.''

\end{itemize}

\section{A Theorem Sampler}\label{sometheorems}

In this section we list some of the more important theorems that are proved in
the \texttt{set.mm} database, and they illustrate the kinds of things that can be
done with Metamath.  While all of these facts are well-known results in set
theory, Metamath offers the advantage of easily allowing you to trace their
derivation back to axioms.  Our intent here is not to try to explain the
details or motivation; for this we refer you to the textbooks that are
mentioned in the descriptions.  (The \texttt{set.mm} file has bibliographic
references for the text references.)  Their proofs often embody important
concepts you may wish to explore with the Metamath program (see
Section~\ref{exploring}).  All the symbols that are used here are defined in
Section~\ref{hierarchy}.  For brevity we haven't included the \texttt{\$d}
restrictions or \texttt{\$f} hypotheses for these theorems; when you are
uncertain consult the \texttt{set.mm} database.

Our first theorem is not very deep but provides us with a notational device
that is frequently used.  It allows us to use the expression ``$A \in V$'' as
a compact way of saying that class $A$ exists, i.e.\ is a set.

\vskip 2ex
\noindent Two ways to say ``$A$ is a set'':  $A$ is a member of the universe
$V$ if and only if $A$ exists (i.e.\ there exists a set equal to $A$).
Theorem 6.9 of Quine, p. 43.

\vskip 0.5ex
\setbox\startprefix=\hbox{\tt \ \ isset\ \$p\ }
\setbox\contprefix=\hbox{\tt \ \ \ \ \ \ \ \ \ \ \ }
\startm
\m{\vdash}\m{(}\m{A}\m{\in}\m{V}\m{\leftrightarrow}\m{\exists}\m{x}\m{\,x}\m{=}
\m{A}\m{)}
\endm
\vskip 1ex

Next we prove the axioms of standard ZF set theory that were missing from our
axiom system.  From our point of view they are theorems since they
can be derived from the other axioms.

\vskip 2ex
\noindent Axiom of Separation\index{Axiom of Separation}
(Aussonderung)\index{Aussonderung} proved from the other axioms of ZF set
theory.  Compare Exercise 4 of Takeuti and Zaring, p.~22.

\vskip 0.5ex
\setbox\startprefix=\hbox{\tt \ \ inex1.1\ \$e\ }
\setbox\contprefix=\hbox{\tt \ \ \ \ \ \ \ \ \ \ \ \ \ \ \ }
\startm
\m{\vdash}\m{A}\m{\in}\m{V}
\endm
\setbox\startprefix=\hbox{\tt \ \ inex\ \$p\ }
\setbox\contprefix=\hbox{\tt \ \ \ \ \ \ \ \ \ \ \ \ \ }
\startm
\m{\vdash}\m{(}\m{A}\m{\cap}\m{B}\m{)}\m{\in}\m{V}
\endm
\vskip 1ex

\noindent Axiom of the Null Set\index{Axiom of the Null Set} proved from the
other axioms of ZF set theory. Corollary 5.16 of Takeuti and Zaring, p.~20.

\vskip 0.5ex
\setbox\startprefix=\hbox{\tt \ \ 0ex\ \$p\ }
\setbox\contprefix=\hbox{\tt \ \ \ \ \ \ \ \ \ \ \ \ }
\startm
\m{\vdash}\m{\varnothing}\m{\in}\m{V}
\endm
\vskip 1ex

\noindent The Axiom of Pairing\index{Axiom of Pairing} proved from the other
axioms of ZF set theory.  Theorem 7.13 of Quine, p.~51.
\vskip 0.5ex
\setbox\startprefix=\hbox{\tt \ \ prex\ \$p\ }
\setbox\contprefix=\hbox{\tt \ \ \ \ \ \ \ \ \ \ \ \ \ \ }
\startm
\m{\vdash}\m{\{}\m{A}\m{,}\m{B}\m{\}}\m{\in}\m{V}
\endm
\vskip 2ex

Next we will list some famous or important theorems that are proved in
the \texttt{set.mm} database.  None of them except \texttt{omex}
require the Axiom of Infinity, as you can verify with the \texttt{show
trace{\char`\_}back} Metamath command.

\vskip 2ex
\noindent The resolution of Russell's paradox\index{Russell's paradox}.  There
exists no set containing the set of all sets which are not members of
themselves.  Proposition 4.14 of Takeuti and Zaring, p.~14.

\vskip 0.5ex
\setbox\startprefix=\hbox{\tt \ \ ru\ \$p\ }
\setbox\contprefix=\hbox{\tt \ \ \ \ \ \ \ \ }
\startm
\m{\vdash}\m{\lnot}\m{\exists}\m{x}\m{\,x}\m{=}\m{\{}\m{y}\m{|}\m{\lnot}\m{y}
\m{\in}\m{y}\m{\}}
\endm
\vskip 1ex

\noindent Cantor's theorem\index{Cantor's theorem}.  No set can be mapped onto
its power set.  Compare Theorem 6B(b) of Enderton, p.~132.

\vskip 0.5ex
\setbox\startprefix=\hbox{\tt \ \ canth.1\ \$e\ }
\setbox\contprefix=\hbox{\tt \ \ \ \ \ \ \ \ \ \ \ \ \ }
\startm
\m{\vdash}\m{A}\m{\in}\m{V}
\endm
\setbox\startprefix=\hbox{\tt \ \ canth\ \$p\ }
\setbox\contprefix=\hbox{\tt \ \ \ \ \ \ \ \ \ \ \ }
\startm
\m{\vdash}\m{\lnot}\m{F}\m{:}\m{A}\m{\raisebox{.5ex}{${\textstyle{\:}_{
\mbox{\footnotesize\rm {\ }}}}\atop{\textstyle{\longrightarrow}\atop{
\textstyle{}^{\mbox{\footnotesize\rm onto}}}}$}}\m{{\cal P}}\m{A}
\endm
\vskip 1ex

\noindent The Burali-Forti paradox\index{Burali-Forti paradox}.  No set
contains all ordinal numbers. Enderton, p.~194.  (Burali-Forti was one person,
not two.)

\vskip 0.5ex
\setbox\startprefix=\hbox{\tt \ \ onprc\ \$p\ }
\setbox\contprefix=\hbox{\tt \ \ \ \ \ \ \ \ \ \ \ \ }
\startm
\m{\vdash}\m{\lnot}\m{\mbox{\rm On}}\m{\in}\m{V}
\endm
\vskip 1ex

\noindent Peano's postulates\index{Peano's postulates} for arithmetic.
Proposition 7.30 of Takeuti and Zaring, pp.~42--43.  The objects being
described are the members of $\omega$ i.e.\ the natural numbers 0, 1,
2,\ldots.  The successor\index{successor} operation suc means ``plus
one.''  \texttt{peano1} says that 0 (which is defined as the empty set)
is a natural number.  \texttt{peano2} says that if $A$ is a natural
number, so is $A+1$.  \texttt{peano3} says that 0 is not the successor
of any natural number.  \texttt{peano4} says that two natural numbers
are equal if and only if their successors are equal.  \texttt{peano5} is
essentially the same as mathematical induction.

\vskip 1ex
\setbox\startprefix=\hbox{\tt \ \ peano1\ \$p\ }
\setbox\contprefix=\hbox{\tt \ \ \ \ \ \ \ \ \ \ \ \ }
\startm
\m{\vdash}\m{\varnothing}\m{\in}\m{\omega}
\endm
\vskip 1.5ex

\setbox\startprefix=\hbox{\tt \ \ peano2\ \$p\ }
\setbox\contprefix=\hbox{\tt \ \ \ \ \ \ \ \ \ \ \ \ }
\startm
\m{\vdash}\m{(}\m{A}\m{\in}\m{\omega}\m{\rightarrow}\m{{\rm suc}}\m{A}\m{\in}%
\m{\omega}\m{)}
\endm
\vskip 1.5ex

\setbox\startprefix=\hbox{\tt \ \ peano3\ \$p\ }
\setbox\contprefix=\hbox{\tt \ \ \ \ \ \ \ \ \ \ \ \ }
\startm
\m{\vdash}\m{(}\m{A}\m{\in}\m{\omega}\m{\rightarrow}\m{\lnot}\m{{\rm suc}}%
\m{A}\m{=}\m{\varnothing}\m{)}
\endm
\vskip 1.5ex

\setbox\startprefix=\hbox{\tt \ \ peano4\ \$p\ }
\setbox\contprefix=\hbox{\tt \ \ \ \ \ \ \ \ \ \ \ \ }
\startm
\m{\vdash}\m{(}\m{(}\m{A}\m{\in}\m{\omega}\m{\wedge}\m{B}\m{\in}\m{\omega}%
\m{)}\m{\rightarrow}\m{(}\m{{\rm suc}}\m{A}\m{=}\m{{\rm suc}}\m{B}\m{%
\leftrightarrow}\m{A}\m{=}\m{B}\m{)}\m{)}
\endm
\vskip 1.5ex

\setbox\startprefix=\hbox{\tt \ \ peano5\ \$p\ }
\setbox\contprefix=\hbox{\tt \ \ \ \ \ \ \ \ \ \ \ \ }
\startm
\m{\vdash}\m{(}\m{(}\m{\varnothing}\m{\in}\m{A}\m{\wedge}\m{\forall}\m{x}\m{%
\in}\m{\omega}\m{(}\m{x}\m{\in}\m{A}\m{\rightarrow}\m{{\rm suc}}\m{x}\m{\in}%
\m{A}\m{)}\m{)}\m{\rightarrow}\m{\omega}\m{\subseteq}\m{A}\m{)}
\endm
\vskip 1.5ex

\noindent Finite Induction (mathematical induction).\index{finite
induction}\index{mathematical induction} The first hypothesis is the
basis and the second is the induction hypothesis.  Theorem Schema 22 of
Suppes, p.~136.

\vskip 0.5ex
\setbox\startprefix=\hbox{\tt \ \ findes.1\ \$e\ }
\setbox\contprefix=\hbox{\tt \ \ \ \ \ \ \ \ \ \ \ \ \ \ }
\startm
\m{\vdash}\m{[}\m{\varnothing}\m{/}\m{x}\m{]}\m{\varphi}
\endm
\setbox\startprefix=\hbox{\tt \ \ findes.2\ \$e\ }
\setbox\contprefix=\hbox{\tt \ \ \ \ \ \ \ \ \ \ \ \ \ \ }
\startm
\m{\vdash}\m{(}\m{x}\m{\in}\m{\omega}\m{\rightarrow}\m{(}\m{\varphi}\m{%
\rightarrow}\m{[}\m{{\rm suc}}\m{x}\m{/}\m{x}\m{]}\m{\varphi}\m{)}\m{)}
\endm
\setbox\startprefix=\hbox{\tt \ \ findes\ \$p\ }
\setbox\contprefix=\hbox{\tt \ \ \ \ \ \ \ \ \ \ \ \ }
\startm
\m{\vdash}\m{(}\m{x}\m{\in}\m{\omega}\m{\rightarrow}\m{\varphi}\m{)}
\endm
\vskip 1ex

\noindent Transfinite Induction with explicit substitution.  The first
hypothesis is the basis, the second is the induction hypothesis for
successors, and the third is the induction hypothesis for limit
ordinals.  Theorem Schema 4 of Suppes, p. 197.

\vskip 0.5ex
\setbox\startprefix=\hbox{\tt \ \ tfindes.1\ \$e\ }
\setbox\contprefix=\hbox{\tt \ \ \ \ \ \ \ \ \ \ \ \ \ \ \ }
\startm
\m{\vdash}\m{[}\m{\varnothing}\m{/}\m{x}\m{]}\m{\varphi}
\endm
\setbox\startprefix=\hbox{\tt \ \ tfindes.2\ \$e\ }
\setbox\contprefix=\hbox{\tt \ \ \ \ \ \ \ \ \ \ \ \ \ \ \ }
\startm
\m{\vdash}\m{(}\m{x}\m{\in}\m{{\rm On}}\m{\rightarrow}\m{(}\m{\varphi}\m{%
\rightarrow}\m{[}\m{{\rm suc}}\m{x}\m{/}\m{x}\m{]}\m{\varphi}\m{)}\m{)}
\endm
\setbox\startprefix=\hbox{\tt \ \ tfindes.3\ \$e\ }
\setbox\contprefix=\hbox{\tt \ \ \ \ \ \ \ \ \ \ \ \ \ \ \ }
\startm
\m{\vdash}\m{(}\m{{\rm Lim}}\m{y}\m{\rightarrow}\m{(}\m{\forall}\m{x}\m{\in}%
\m{y}\m{\varphi}\m{\rightarrow}\m{[}\m{y}\m{/}\m{x}\m{]}\m{\varphi}\m{)}\m{)}
\endm
\setbox\startprefix=\hbox{\tt \ \ tfindes\ \$p\ }
\setbox\contprefix=\hbox{\tt \ \ \ \ \ \ \ \ \ \ \ \ \ }
\startm
\m{\vdash}\m{(}\m{x}\m{\in}\m{{\rm On}}\m{\rightarrow}\m{\varphi}\m{)}
\endm
\vskip 1ex

\noindent Principle of Transfinite Recursion.\index{transfinite
recursion} Theorem 7.41 of Takeuti and Zaring, p.~47.  Transfinite
recursion is the key theorem that allows arithmetic of ordinals to be
rigorously defined, and has many other important uses as well.
Hypotheses \texttt{tfr.1} and \texttt{tfr.2} specify a certain (proper)
class $ F$.  The complicated definition of $ F$ is not important in
itself; what is important is that there be such an $ F$ with the
required properties, and we show this by displaying $ F$ explicitly.
\texttt{tfr1} states that $ F$ is a function whose domain is the set of
ordinal numbers.  \texttt{tfr2} states that any value of $ F$ is
completely determined by its previous values and the values of an
auxiliary function, $G$.  \texttt{tfr3} states that $ F$ is unique,
i.e.\ it is the only function that satisfies \texttt{tfr1} and
\texttt{tfr2}.  Note that $ f$ is an individual variable like $x$ and
$y$; it is just a mnemonic to remind us that $A$ is a collection of
functions.

\vskip 0.5ex
\setbox\startprefix=\hbox{\tt \ \ tfr.1\ \$e\ }
\setbox\contprefix=\hbox{\tt \ \ \ \ \ \ \ \ \ \ \ }
\startm
\m{\vdash}\m{A}\m{=}\m{\{}\m{f}\m{|}\m{\exists}\m{x}\m{\in}\m{{\rm On}}\m{(}%
\m{f}\m{{\rm Fn}}\m{x}\m{\wedge}\m{\forall}\m{y}\m{\in}\m{x}\m{(}\m{f}\m{`}%
\m{y}\m{)}\m{=}\m{(}\m{G}\m{`}\m{(}\m{f}\m{\restriction}\m{y}\m{)}\m{)}\m{)}%
\m{\}}
\endm
\setbox\startprefix=\hbox{\tt \ \ tfr.2\ \$e\ }
\setbox\contprefix=\hbox{\tt \ \ \ \ \ \ \ \ \ \ \ }
\startm
\m{\vdash}\m{F}\m{=}\m{\bigcup}\m{A}
\endm
\setbox\startprefix=\hbox{\tt \ \ tfr1\ \$p\ }
\setbox\contprefix=\hbox{\tt \ \ \ \ \ \ \ \ \ \ }
\startm
\m{\vdash}\m{F}\m{{\rm Fn}}\m{{\rm On}}
\endm
\setbox\startprefix=\hbox{\tt \ \ tfr2\ \$p\ }
\setbox\contprefix=\hbox{\tt \ \ \ \ \ \ \ \ \ \ }
\startm
\m{\vdash}\m{(}\m{z}\m{\in}\m{{\rm On}}\m{\rightarrow}\m{(}\m{F}\m{`}\m{z}%
\m{)}\m{=}\m{(}\m{G}\m{`}\m{(}\m{F}\m{\restriction}\m{z}\m{)}\m{)}\m{)}
\endm
\setbox\startprefix=\hbox{\tt \ \ tfr3\ \$p\ }
\setbox\contprefix=\hbox{\tt \ \ \ \ \ \ \ \ \ \ }
\startm
\m{\vdash}\m{(}\m{(}\m{B}\m{{\rm Fn}}\m{{\rm On}}\m{\wedge}\m{\forall}\m{x}\m{%
\in}\m{{\rm On}}\m{(}\m{B}\m{`}\m{x}\m{)}\m{=}\m{(}\m{G}\m{`}\m{(}\m{B}\m{%
\restriction}\m{x}\m{)}\m{)}\m{)}\m{\rightarrow}\m{B}\m{=}\m{F}\m{)}
\endm
\vskip 1ex

\noindent The existence of omega (the class of natural numbers).\index{natural
number}\index{omega ($\omega$)}\index{Axiom of Infinity}  Axiom 7 of Takeuti
and Zaring, p.~43.  (This is the only theorem in this section requiring the
Axiom of Infinity.)

\vskip 0.5ex
\setbox\startprefix=\hbox{\tt \
\ omex\ \$p\ }
\setbox\contprefix=\hbox{\tt \ \ \ \ \ \ \ \ \ \ }
\startm
\m{\vdash}\m{\omega}\m{\in}\m{V}
\endm
%\vskip 2ex


\section{Axioms for Real and Complex Numbers}\label{real}
\index{real and complex numbers!axioms for}

This section presents the axioms for real and complex numbers.  Analysis
textbooks implicitly or explicitly use these axioms or their equivalents
as their starting point.  In the database \texttt{set.mm}, we define real
and complex numbers as (rather complicated) specific sets and derive these
axioms as {\em theorems} from the axioms of ZF set theory, using a method
called Dedekind cuts.  We omit the details of this construction, which you can
follow if you wish using the \texttt{set.mm} database in conjunction with the
textbooks referenced therein.

Once we prove those theorems, we then restate these proven theorems as axioms.
This lets us easily identify which axioms are needed for a particular complex number proof, without the obfuscation of the set theory used to derive them.
As a result,
the construction is actually unimportant other
than to show that sets exist that satisfy the axioms, and thus that the axioms
are consistent if ZF set theory is consistent.  When working with real numbers
you can think of them as being the actual sets resulting
from the construction (for definiteness), or you can
think of them as otherwise unspecified sets that happen to satisfy the axioms.
The derivation is not easy, but the fact that it works is quite remarkable
and lends support to the idea that ZFC set theory is all we need to
provide a foundation for essentially all of mathematics.

For the axioms we are given (or postulate) 8 classes:  $\mathbb{C}$ (the
set of complex numbers), $\mathbb{R}$ (the set of real numbers, a subset
of $\mathbb{C}$), $0$ (zero), $1$ (one), $i$ (square root of
$-1$), $+$ (plus), $\cdot$ (times), and
$<_{\mathbb{R}}$ (less than for just the real numbers).
Subtraction and division are defined terms and are not part of the
axioms; for their definitions see \texttt{set.mm}.

Note that the notation $(A+B)$ (and similarly $(A\cdot B)$) specifies a class
called an {\em operation},\index{operation} and is the function value of the
class $+$ at ordered pair $\langle A,B \rangle$.  An operation is defined by
statement \texttt{df-opr} on p.~\pageref{dfopr}.
The notation $A <_{\mathbb{R}} B$ specifies a
wff called a {\em binary relation}\index{binary relation} and means $\langle A,B \rangle \in \,<_{\mathbb{R}}$, as defined by statement \texttt{df-br} on p.~\pageref{dfbr}.

Our set of 8 given classes is assumed to satisfy the following 22 axioms
(in the axioms listed below, $<$ really means $<_{\mathbb{R}}$).

\vskip 2ex

\noindent 1. The real numbers are a subset of the complex numbers.

%\vskip 0.5ex
\setbox\startprefix=\hbox{\tt \ \ ax-resscn\ \$p\ }
\setbox\contprefix=\hbox{\tt \ \ \ \ \ \ \ \ \ \ \ \ \ \ }
\startm
\m{\vdash}\m{\mathbb{R}}\m{\subseteq}\m{\mathbb{C}}
\endm
%\vskip 1ex

\noindent 2. One is a complex number.

%\vskip 0.5ex
\setbox\startprefix=\hbox{\tt \ \ ax-1cn\ \$p\ }
\setbox\contprefix=\hbox{\tt \ \ \ \ \ \ \ \ \ \ \ }
\startm
\m{\vdash}\m{1}\m{\in}\m{\mathbb{C}}
\endm
%\vskip 1ex

\noindent 3. The imaginary number $i$ is a complex number.

%\vskip 0.5ex
\setbox\startprefix=\hbox{\tt \ \ ax-icn\ \$p\ }
\setbox\contprefix=\hbox{\tt \ \ \ \ \ \ \ \ \ \ \ }
\startm
\m{\vdash}\m{i}\m{\in}\m{\mathbb{C}}
\endm
%\vskip 1ex

\noindent 4. Complex numbers are closed under addition.

%\vskip 0.5ex
\setbox\startprefix=\hbox{\tt \ \ ax-addcl\ \$p\ }
\setbox\contprefix=\hbox{\tt \ \ \ \ \ \ \ \ \ \ \ \ \ }
\startm
\m{\vdash}\m{(}\m{(}\m{A}\m{\in}\m{\mathbb{C}}\m{\wedge}\m{B}\m{\in}\m{\mathbb{C}}%
\m{)}\m{\rightarrow}\m{(}\m{A}\m{+}\m{B}\m{)}\m{\in}\m{\mathbb{C}}\m{)}
\endm
%\vskip 1ex

\noindent 5. Real numbers are closed under addition.

%\vskip 0.5ex
\setbox\startprefix=\hbox{\tt \ \ ax-addrcl\ \$p\ }
\setbox\contprefix=\hbox{\tt \ \ \ \ \ \ \ \ \ \ \ \ \ \ }
\startm
\m{\vdash}\m{(}\m{(}\m{A}\m{\in}\m{\mathbb{R}}\m{\wedge}\m{B}\m{\in}\m{\mathbb{R}}%
\m{)}\m{\rightarrow}\m{(}\m{A}\m{+}\m{B}\m{)}\m{\in}\m{\mathbb{R}}\m{)}
\endm
%\vskip 1ex

\noindent 6. Complex numbers are closed under multiplication.

%\vskip 0.5ex
\setbox\startprefix=\hbox{\tt \ \ ax-mulcl\ \$p\ }
\setbox\contprefix=\hbox{\tt \ \ \ \ \ \ \ \ \ \ \ \ \ }
\startm
\m{\vdash}\m{(}\m{(}\m{A}\m{\in}\m{\mathbb{C}}\m{\wedge}\m{B}\m{\in}\m{\mathbb{C}}%
\m{)}\m{\rightarrow}\m{(}\m{A}\m{\cdot}\m{B}\m{)}\m{\in}\m{\mathbb{C}}\m{)}
\endm
%\vskip 1ex

\noindent 7. Real numbers are closed under multiplication.

%\vskip 0.5ex
\setbox\startprefix=\hbox{\tt \ \ ax-mulrcl\ \$p\ }
\setbox\contprefix=\hbox{\tt \ \ \ \ \ \ \ \ \ \ \ \ \ \ }
\startm
\m{\vdash}\m{(}\m{(}\m{A}\m{\in}\m{\mathbb{R}}\m{\wedge}\m{B}\m{\in}\m{\mathbb{R}}%
\m{)}\m{\rightarrow}\m{(}\m{A}\m{\cdot}\m{B}\m{)}\m{\in}\m{\mathbb{R}}\m{)}
\endm
%\vskip 1ex

\noindent 8. Multiplication of complex numbers is commutative.

%\vskip 0.5ex
\setbox\startprefix=\hbox{\tt \ \ ax-mulcom\ \$p\ }
\setbox\contprefix=\hbox{\tt \ \ \ \ \ \ \ \ \ \ \ \ \ \ }
\startm
\m{\vdash}\m{(}\m{(}\m{A}\m{\in}\m{\mathbb{C}}\m{\wedge}\m{B}\m{\in}\m{\mathbb{C}}%
\m{)}\m{\rightarrow}\m{(}\m{A}\m{\cdot}\m{B}\m{)}\m{=}\m{(}\m{B}\m{\cdot}\m{A}%
\m{)}\m{)}
\endm
%\vskip 1ex

\noindent 9. Addition of complex numbers is associative.

%\vskip 0.5ex
\setbox\startprefix=\hbox{\tt \ \ ax-addass\ \$p\ }
\setbox\contprefix=\hbox{\tt \ \ \ \ \ \ \ \ \ \ \ \ \ \ }
\startm
\m{\vdash}\m{(}\m{(}\m{A}\m{\in}\m{\mathbb{C}}\m{\wedge}\m{B}\m{\in}\m{\mathbb{C}}%
\m{\wedge}\m{C}\m{\in}\m{\mathbb{C}}\m{)}\m{\rightarrow}\m{(}\m{(}\m{A}\m{+}%
\m{B}\m{)}\m{+}\m{C}\m{)}\m{=}\m{(}\m{A}\m{+}\m{(}\m{B}\m{+}\m{C}\m{)}\m{)}%
\m{)}
\endm
%\vskip 1ex

\noindent 10. Multiplication of complex numbers is associative.

%\vskip 0.5ex
\setbox\startprefix=\hbox{\tt \ \ ax-mulass\ \$p\ }
\setbox\contprefix=\hbox{\tt \ \ \ \ \ \ \ \ \ \ \ \ \ \ }
\startm
\m{\vdash}\m{(}\m{(}\m{A}\m{\in}\m{\mathbb{C}}\m{\wedge}\m{B}\m{\in}\m{\mathbb{C}}%
\m{\wedge}\m{C}\m{\in}\m{\mathbb{C}}\m{)}\m{\rightarrow}\m{(}\m{(}\m{A}\m{\cdot}%
\m{B}\m{)}\m{\cdot}\m{C}\m{)}\m{=}\m{(}\m{A}\m{\cdot}\m{(}\m{B}\m{\cdot}\m{C}%
\m{)}\m{)}\m{)}
\endm
%\vskip 1ex

\noindent 11. Multiplication distributes over addition for complex numbers.

%\vskip 0.5ex
\setbox\startprefix=\hbox{\tt \ \ ax-distr\ \$p\ }
\setbox\contprefix=\hbox{\tt \ \ \ \ \ \ \ \ \ \ \ \ \ }
\startm
\m{\vdash}\m{(}\m{(}\m{A}\m{\in}\m{\mathbb{C}}\m{\wedge}\m{B}\m{\in}\m{\mathbb{C}}%
\m{\wedge}\m{C}\m{\in}\m{\mathbb{C}}\m{)}\m{\rightarrow}\m{(}\m{A}\m{\cdot}\m{(}%
\m{B}\m{+}\m{C}\m{)}\m{)}\m{=}\m{(}\m{(}\m{A}\m{\cdot}\m{B}\m{)}\m{+}\m{(}%
\m{A}\m{\cdot}\m{C}\m{)}\m{)}\m{)}
\endm
%\vskip 1ex

\noindent 12. The square of $i$ equals $-1$ (expressed as $i$-squared plus 1 is
0).

%\vskip 0.5ex
\setbox\startprefix=\hbox{\tt \ \ ax-i2m1\ \$p\ }
\setbox\contprefix=\hbox{\tt \ \ \ \ \ \ \ \ \ \ \ \ }
\startm
\m{\vdash}\m{(}\m{(}\m{i}\m{\cdot}\m{i}\m{)}\m{+}\m{1}\m{)}\m{=}\m{0}
\endm
%\vskip 1ex

\noindent 13. One and zero are distinct.

%\vskip 0.5ex
\setbox\startprefix=\hbox{\tt \ \ ax-1ne0\ \$p\ }
\setbox\contprefix=\hbox{\tt \ \ \ \ \ \ \ \ \ \ \ \ }
\startm
\m{\vdash}\m{1}\m{\ne}\m{0}
\endm
%\vskip 1ex

\noindent 14. One is an identity element for real multiplication.

%\vskip 0.5ex
\setbox\startprefix=\hbox{\tt \ \ ax-1rid\ \$p\ }
\setbox\contprefix=\hbox{\tt \ \ \ \ \ \ \ \ \ \ \ }
\startm
\m{\vdash}\m{(}\m{A}\m{\in}\m{\mathbb{R}}\m{\rightarrow}\m{(}\m{A}\m{\cdot}\m{1}%
\m{)}\m{=}\m{A}\m{)}
\endm
%\vskip 1ex

\noindent 15. Every real number has a negative.

%\vskip 0.5ex
\setbox\startprefix=\hbox{\tt \ \ ax-rnegex\ \$p\ }
\setbox\contprefix=\hbox{\tt \ \ \ \ \ \ \ \ \ \ \ \ \ \ }
\startm
\m{\vdash}\m{(}\m{A}\m{\in}\m{\mathbb{R}}\m{\rightarrow}\m{\exists}\m{x}\m{\in}%
\m{\mathbb{R}}\m{(}\m{A}\m{+}\m{x}\m{)}\m{=}\m{0}\m{)}
\endm
%\vskip 1ex

\noindent 16. Every nonzero real number has a reciprocal.

%\vskip 0.5ex
\setbox\startprefix=\hbox{\tt \ \ ax-rrecex\ \$p\ }
\setbox\contprefix=\hbox{\tt \ \ \ \ \ \ \ \ \ \ \ \ \ \ }
\startm
\m{\vdash}\m{(}\m{A}\m{\in}\m{\mathbb{R}}\m{\rightarrow}\m{(}\m{A}\m{\ne}\m{0}%
\m{\rightarrow}\m{\exists}\m{x}\m{\in}\m{\mathbb{R}}\m{(}\m{A}\m{\cdot}%
\m{x}\m{)}\m{=}\m{1}\m{)}\m{)}
\endm
%\vskip 1ex

\noindent 17. A complex number can be expressed in terms of two reals.

%\vskip 0.5ex
\setbox\startprefix=\hbox{\tt \ \ ax-cnre\ \$p\ }
\setbox\contprefix=\hbox{\tt \ \ \ \ \ \ \ \ \ \ \ \ }
\startm
\m{\vdash}\m{(}\m{A}\m{\in}\m{\mathbb{C}}\m{\rightarrow}\m{\exists}\m{x}\m{\in}%
\m{\mathbb{R}}\m{\exists}\m{y}\m{\in}\m{\mathbb{R}}\m{A}\m{=}\m{(}\m{x}\m{+}\m{(}%
\m{y}\m{\cdot}\m{i}\m{)}\m{)}\m{)}
\endm
%\vskip 1ex

\noindent 18. Ordering on reals satisfies strict trichotomy.

%\vskip 0.5ex
\setbox\startprefix=\hbox{\tt \ \ ax-pre-lttri\ \$p\ }
\setbox\contprefix=\hbox{\tt \ \ \ \ \ \ \ \ \ \ \ \ \ }
\startm
\m{\vdash}\m{(}\m{(}\m{A}\m{\in}\m{\mathbb{R}}\m{\wedge}\m{B}\m{\in}\m{\mathbb{R}}%
\m{)}\m{\rightarrow}\m{(}\m{A}\m{<}\m{B}\m{\leftrightarrow}\m{\lnot}\m{(}\m{A}%
\m{=}\m{B}\m{\vee}\m{B}\m{<}\m{A}\m{)}\m{)}\m{)}
\endm
%\vskip 1ex

\noindent 19. Ordering on reals is transitive.

%\vskip 0.5ex
\setbox\startprefix=\hbox{\tt \ \ ax-pre-lttrn\ \$p\ }
\setbox\contprefix=\hbox{\tt \ \ \ \ \ \ \ \ \ \ \ \ \ }
\startm
\m{\vdash}\m{(}\m{(}\m{A}\m{\in}\m{\mathbb{R}}\m{\wedge}\m{B}\m{\in}\m{\mathbb{R}}%
\m{\wedge}\m{C}\m{\in}\m{\mathbb{R}}\m{)}\m{\rightarrow}\m{(}\m{(}\m{A}\m{<}%
\m{B}\m{\wedge}\m{B}\m{<}\m{C}\m{)}\m{\rightarrow}\m{A}\m{<}\m{C}\m{)}\m{)}
\endm
%\vskip 1ex

\noindent 20. Ordering on reals is preserved after addition to both sides.

%\vskip 0.5ex
\setbox\startprefix=\hbox{\tt \ \ ax-pre-ltadd\ \$p\ }
\setbox\contprefix=\hbox{\tt \ \ \ \ \ \ \ \ \ \ \ \ \ }
\startm
\m{\vdash}\m{(}\m{(}\m{A}\m{\in}\m{\mathbb{R}}\m{\wedge}\m{B}\m{\in}\m{\mathbb{R}}%
\m{\wedge}\m{C}\m{\in}\m{\mathbb{R}}\m{)}\m{\rightarrow}\m{(}\m{A}\m{<}\m{B}\m{%
\rightarrow}\m{(}\m{C}\m{+}\m{A}\m{)}\m{<}\m{(}\m{C}\m{+}\m{B}\m{)}\m{)}\m{)}
\endm
%\vskip 1ex

\noindent 21. The product of two positive reals is positive.

%\vskip 0.5ex
\setbox\startprefix=\hbox{\tt \ \ ax-pre-mulgt0\ \$p\ }
\setbox\contprefix=\hbox{\tt \ \ \ \ \ \ \ \ \ \ \ \ \ \ }
\startm
\m{\vdash}\m{(}\m{(}\m{A}\m{\in}\m{\mathbb{R}}\m{\wedge}\m{B}\m{\in}\m{\mathbb{R}}%
\m{)}\m{\rightarrow}\m{(}\m{(}\m{0}\m{<}\m{A}\m{\wedge}\m{0}%
\m{<}\m{B}\m{)}\m{\rightarrow}\m{0}\m{<}\m{(}\m{A}\m{\cdot}\m{B}\m{)}%
\m{)}\m{)}
\endm
%\vskip 1ex

\noindent 22. A non-empty, bounded-above set of reals has a supremum.

%\vskip 0.5ex
\setbox\startprefix=\hbox{\tt \ \ ax-pre-sup\ \$p\ }
\setbox\contprefix=\hbox{\tt \ \ \ \ \ \ \ \ \ \ \ }
\startm
\m{\vdash}\m{(}\m{(}\m{A}\m{\subseteq}\m{\mathbb{R}}\m{\wedge}\m{A}\m{\ne}\m{%
\varnothing}\m{\wedge}\m{\exists}\m{x}\m{\in}\m{\mathbb{R}}\m{\forall}\m{y}\m{%
\in}\m{A}\m{\,y}\m{<}\m{x}\m{)}\m{\rightarrow}\m{\exists}\m{x}\m{\in}\m{%
\mathbb{R}}\m{(}\m{\forall}\m{y}\m{\in}\m{A}\m{\lnot}\m{x}\m{<}\m{y}\m{\wedge}\m{%
\forall}\m{y}\m{\in}\m{\mathbb{R}}\m{(}\m{y}\m{<}\m{x}\m{\rightarrow}\m{\exists}%
\m{z}\m{\in}\m{A}\m{\,y}\m{<}\m{z}\m{)}\m{)}\m{)}
\endm

% NOTE: The \m{...} expressions above could be represented as
% $ \vdash ( ( A \subseteq \mathbb{R} \wedge A \ne \varnothing \wedge \exists x \in \mathbb{R} \forall y \in A \,y < x ) \rightarrow \exists x \in \mathbb{R} ( \forall y \in A \lnot x < y \wedge \forall y \in \mathbb{R} ( y < x \rightarrow \exists z \in A \,y < z ) ) ) $

\vskip 2ex

This completes the set of axioms for real and complex numbers.  You may
wish to look at how subtraction, division, and decimal numbers
are defined in \texttt{set.mm}, and for fun look at the proof of $2+
2 = 4$ (theorem \texttt{2p2e4} in \texttt{set.mm})
as discussed in section \ref{2p2e4}.

In \texttt{set.mm} we define the non-negative integers $\mathbb{N}$, the integers
$\mathbb{Z}$, and the rationals $\mathbb{Q}$ as subsets of $\mathbb{R}$.  This leads
to the nice inclusion $\mathbb{N} \subseteq \mathbb{Z} \subseteq \mathbb{Q} \subseteq
\mathbb{R} \subseteq \mathbb{C}$, giving us a uniform framework in which, for
example, a property such as commutativity of complex number addition
automatically applies to integers.  The natural numbers $\mathbb{N}$
are different from the set $\omega$ we defined earlier, but both satisfy
Peano's postulates.

\subsubsection{Complex Number Axioms in Analysis Texts}

Most analysis texts construct complex numbers as ordered pairs of reals,
leading to construction-dependent properties that satisfy these axioms
but are not stated in their pure form.  (This is also done in
\texttt{set.mm} but our axioms are extracted from that construction.)
Other texts will simply state that $\mathbb{R}$ is a ``complete ordered
subfield of $\mathbb{C}$,'' leading to redundant axioms when this phrase
is completely expanded out.  In fact I have not seen a text with the
axioms in the explicit form above.
None these axioms is unique individually, but this carefully worked out
collection of axioms is the result of years of work
by the Metamath community.

We once had more axioms for real and complex numbers, but over time
we have found ways to eliminate them (by proving them from other axioms),
and in some cases we have found ways to weaken them.
we could eliminate 0 as an axiomatic object by defining it as
$( ( i \cdot i ) + 1 )$
and replacing it with this expression throughout the axioms. If this
is done, axiom ax-i2m1 becomes redundant. However, the remaining axioms
would become longer and less intuitive.

\section{Two Plus Two Equals Four}\label{2p2e4}

Here is a proof that $2 + 2 = 4$, as proven in the theorem \texttt{2p2e4}
in the database \texttt{set.mm}.
This is a useful demonstration of what a Metamath proof can look like.
This proof may have more steps than you're used to, but each step is rigorously
proven all the way back to the axioms of logic and set theory.
This display was originally generated by the Metamath program
as an HTML file.

In the table showing the proof ``Step'' is the sequential step number,
while its associated ``Expression'' is an expression that we have proved.
``Ref'' is the name of a theorem or axiom that justifies that expression,
and ``Hyp'' refers to previous steps (if any) that the theorem or axiom
needs so that we can use it.  Expressions are indented further than
the expressions that depend on them to show their interdependencies.

\begin{table}[!htbp]
\caption{Two plus two equals four}
\begin{tabular}{lllll}
\textbf{Step} & \textbf{Hyp} & \textbf{Ref} & \textbf{Expression} & \\
1  &       & df-2    & $ \; \; \vdash 2 = 1 + 1$  & \\
2  & 1     & oveq2i  & $ \; \vdash (2 + 2) = (2 + (1 + 1))$ & \\
3  &       & df-4    & $ \; \; \vdash 4 = (3 + 1)$ & \\
4  &       & df-3    & $ \; \; \; \vdash 3 = (2 + 1)$ & \\
5  & 4     & oveq1i  & $ \; \; \vdash (3 + 1) = ((2 + 1) + 1)$ & \\
6  &       & 2cn     & $ \; \; \; \vdash 2 \in \mathbb{C}$ & \\
7  &       & ax-1cn  & $ \; \; \; \vdash 1 \in \mathbb{C}$ & \\
8  & 6,7,7 & addassi & $ \; \; \vdash ((2 + 1) + 1) = (2 + (1 + 1))$ & \\
9  & 3,5,8 & 3eqtri  & $ \; \vdash 4 = (2 + (1 + 1))$ & \\
10 & 2,9   & eqtr4i  & $ \vdash (2 + 2) = 4$ & \\
\end{tabular}
\end{table}

Step 1 says that we can assert that $2 = 1 + 1$ because it is
justified by \texttt{df-2}.
What is \texttt{df-2}?
It is simply the definition of $2$, which in our system is defined as being
equal to $1 + 1$.  This shows how we can use definitions in proofs.

Look at Step 2 of the proof. In the Ref column, we see that it references
a previously proved theorem, \texttt{oveq2i}.
It turns out that
theorem \texttt{oveq2i} requires a
hypothesis, and in the Hyp column of Step 2 we indicate that Step 1 will
satisfy (match) this hypothesis.
If we looked at \texttt{oveq2i}
we would find that it proves that given some hypothesis
$A = B$, we can prove that $( C F A ) = ( C F B )$.
If we use \texttt{oveq2i} and apply step 1's result as the hypothesis,
that will mean that $A = 2$ and $B = ( 1 + 1 )$ within this use of
\texttt{oveq2i}.
We are free to select any value of $C$ and $F$ (subject to syntax constraints),
so we are free to select $C = 2$ and $F = +$,
producing our desired result,
$ (2 + 2) = (2 + (1 + 1))$.

Step 2 is an example of substitution.
In the end, every step in every proof uses only this one substitution rule.
All the rules of logic, and all the axioms, are expressed so that
they can be used via this one substitution rule.
So once you master substitution, you can master every Metamath proof,
no exceptions.

Each step is clear and can be immediately checked.
In the HTML display you can even click on each reference to see why it is
justified, making it easy to see why the proof works.

\section{Exploring the Set Theory Database}\label{exploring}

At this point you may wish to study the \texttt{set.mm}\index{set theory
database (\texttt{set.mm})} file in more detail.  Pay particular
attention to the assumptions needed to define wffs\index{well-formed
formula (wff)} (which are not included above), the variable types
(\texttt{\$f}\index{\texttt{\$f} statement} statements), and the
definitions that are introduced.  Start with some simple theorems in
propositional calculus, making sure you understand in detail each step
of a proof.  Once you get past the first few proofs and become familiar
with the Metamath language, any part of the \texttt{set.mm} database
will be as easy to follow, step by step, as any other part---you won't
have to undergo a ``quantum leap'' in mathematical sophistication to be
able to follow a deep proof in set theory.

Next, you may want to explore how concepts such as natural numbers are
defined and described.  This is probably best done in conjunction with
standard set theory textbooks, which can help give you a higher-level
understanding.  The \texttt{set.mm} database provides references that will get
you started.  From there, you will be on your way towards a very deep,
rigorous understanding of abstract mathematics.

The Metamath\index{Metamath} program can help you peruse a Metamath data\-base,
wheth\-er you are trying to figure out how a certain step follows in a proof or
just have a general curiosity.  We will go through some examples of the
commands, using the \texttt{set.mm}\index{set theory database (\texttt{set.mm})}
database provided with the Metamath software.  These should help get you
started.  See Chapter~\ref{commands} for a more detailed description of
the commands.  Note that we have included the full spelling of all commands to
prevent ambiguity with future commands.  In practice you may type just the
characters needed to specify each command keyword\index{command keyword}
unambiguously, often just one or two characters per keyword, and you don't
need to type them in upper
case.

First run the Metamath program as described earlier.  You should see the
\verb/MM>/ prompt.  Read in the \texttt{set.mm} file:\index{\texttt{read}
command}

\begin{verbatim}
MM> read set.mm
Reading source file "set.mm"...
73689 lines (3543983 characters) were read from "set.mm".
The source has 21443 statements; 417 are $a and 5989 are $p.
No errors were found.  However, proofs were not checked.
Type VERIFY PROOF * if you want to check them.
\end{verbatim}

Let's check the database integrity.  This may take a minute or two to run if
your computer is slow.

\begin{verbatim}
MM> verify proof *
0 10%  20%  30%  40%  50%  60%  70%  80%  90% 100%
..................................................
All proofs in the database were verified in 2.84 s.
\end{verbatim}

No errors were reported, so every proof is correct.

You need to know the names (labels) of theorems before you can look at them.
Often just examining the database file(s) with a text editor is the best
approach.  In \texttt{set.mm} there are many detailed comments, especially near
the beginning, that can help guide you. The \texttt{search} command in the
Metamath program is also handy.  The \texttt{comments} qualifier will list the
statements whose associated comment (the one immediately before it) contain a
string you give it.  For example, if you are studying Enderton's {\em Elements
of Set Theory} \cite{Enderton}\index{Enderton, Herbert B.} you may want to see
the references to it in the database.  The search string \texttt{enderton} is not
case sensitive.  (This will not show you all the database theorems that are in
Enderton's book because there is usually only one citation for a given
theorem, which may appear in several textbooks.)\index{\texttt{search}
command}

\begin{verbatim}
MM> search * "enderton" / comments
3332 df-ral $a "...niversal quantification. Enderton, p. 22."
3333 df-rex $a "...stential quantification. Enderton, p. 22."
4828 df-tp $a "...of classes. Definition of Enderton, p. 19."
5200 ssuniss $p "...nd union. Exercise 5 of Enderton, p. 26."
5217 opeluu $p "...air belongs. Lemma 3D of Enderton, p. 41."
\end{verbatim}
\begin{center}
(etc.)
\end{center}

Or you may want to see what theorems have something to do with
conjunction (logical {\sc and}).  The quotes around the search
string are optional when there's no ambiguity.\index{\texttt{search}
command}

\begin{verbatim}
MM> search * conjunction / comments
634 wa $a "...wff definition to include conjunction ('and')."
636 df-an $a "Define conjunction ('and')."
654 iman $p "Express implication in terms of conjunction."
655 annim $p "Express conjunction in terms of implication."
687 pm3.2 $p "... antecedents with conjunction. Theorem *..."
751 anor $p "Conjunction in terms of disjunction (de Morg..."
\end{verbatim}
\begin{center}
(etc.)
\end{center}


Now we will start to look at some details.  Let's look at the first
axiom of propositional calculus.\index{\texttt{show statement}
command}

\begin{verbatim}
MM> show statement ax-1/full
MM> sh st ax-1/full
Statement 19 is located on line 881 of the file "set.mm".
"Axiom _Simp_.  Axiom A1 of [Margaris] p. 49.  One of the 3
axioms of propositional calculus.  The 3 axioms are also
        ...
19 ax-1 $a |- ( ph -> ( ps -> ph ) ) $.
Its mandatory hypotheses in RPN order are:
  wph $f wff ph $.
  wps $f wff ps $.
The statement and its hypotheses require the variables:  ph
      ps
The variables it contains are:  ph ps
\end{verbatim}

Compare this to \texttt{ax-1} on p.~\pageref{ax1}.  You can see that the
symbol \texttt{ph} is the {\sc ascii} notation for $\varphi$, etc.  To
see the mathematical symbols for any expression you may typeset it in
\LaTeX\ (type \texttt{help tex} for instructions)\index{latex@{\LaTeX}}
or, easier, just use a text editor to look at the comments where symbols
are first introduced in \texttt{set.mm}.  The hypotheses \texttt{wph}
and \texttt{wps} required by \texttt{ax-1} mean that variables
\texttt{ph} and \texttt{ps} must be wffs.

Next we'll pick a simple theorem of propositional calculus, the Principle of
Identity, which is proved directly from the axioms.  We'll look at the
statement then its proof.\index{\texttt{show statement}
command}

\begin{verbatim}
MM> show statement id1/full
Statement 36 is located on line 968 of the file "set.mm".
"Principle of identity.  Theorem *2.08 of Whitehead and
Russell.  This version is proved directly from the axioms
for demonstration purposes."
  id1 $p |- ( ph -> ph ) $= ... $.
Its mandatory hypotheses in RPN order are:
  wph $f wff ph $.
Its optional hypotheses are:  wps wch wth wet
The statement and its hypotheses require the variables:  ph
These additional variables are allowed in its proof:  ps ch
      th et
The variables it contains are:  ph
\end{verbatim}

The optional variables\index{optional variable} \texttt{ps}, \texttt{ch}, etc.\ are
available for use in a proof of this statement if we wish, and were we to do
so we would make use of optional hypotheses \texttt{wps}, \texttt{wch}, etc.  (See
Section~\ref{dollaref} for the meaning of ``optional
hypothesis.''\index{optional hypothesis}) The reason these show up in the
statement display is that statement \texttt{id1} happens to be in their scope
(see Section~\ref{scoping} for the definition of ``scope''\index{scope}), but
in fact in propositional calculus we will never make use of optional
hypotheses or variables.  This becomes important after quantifiers are
introduced, where ``dummy'' variables\index{dummy variable} are often needed
in the middle of a proof.

Let's look at the proof of statement \texttt{id1}.  We'll suppress the
``non-essential'' steps that construct the wffs.\index{\texttt{show proof}
command}

\begin{verbatim}
MM> show proof id1/essential/lemmon/renumber
1 ax-2           $a |- ( ( ph -> ( ( ph -> ph ) -> ph ) ) ->
                     ( ( ph -> ( ph -> ph ) ) -> ( ph -> ph )
                                                          ) )
2 ax-1           $a |- ( ph -> ( ( ph -> ph ) -> ph ) )
3 1,2 ax-mp      $a |- ( ( ph -> ( ph -> ph ) ) -> ( ph -> ph
                                                          ) )
4 ax-1           $a |- ( ph -> ( ph -> ph ) )
5 3,4 ax-mp      $a |- ( ph -> ph )
\end{verbatim}

If you have read Section~\ref{trialrun}, you'll know how to interpret this
proof.  Step~2, for example, is an application of axiom \texttt{ax-1}.  This
proof is identical to the one in Hamilton's {\em Logic for Mathematicians}
\cite[p.~32]{Hamilton}\index{Hamilton, Alan G.}.

You may want to look at what
substitutions\index{substitution!variable}\index{variable substitution} are
made into \texttt{ax-1} to arrive at step~2.  The command to do this needs to
know the ``real'' step number, so we'll display the proof again without
the \texttt{renumber} qualifier.\index{\texttt{show proof}
command}

\begin{verbatim}
MM> show proof id1/lemmon/essential
18 ax-2          $a |- ( ( ph -> ( ( ph -> ph ) -> ph ) ) ->
                     ( ( ph -> ( ph -> ph ) ) -> ( ph -> ph )
                                                          ) )
21 ax-1          $a |- ( ph -> ( ( ph -> ph ) -> ph ) )
22 18,21 ax-mp   $a |- ( ( ph -> ( ph -> ph ) ) -> ( ph -> ph
                                                          ) )
25 ax-1          $a |- ( ph -> ( ph -> ph ) )
26 22,25 ax-mp   $a |- ( ph -> ph )
\end{verbatim}

The ``real'' step number is 21.  Let's look at its details.

\begin{verbatim}
MM> show proof id1 /detailed_step 21
Proof step 21:  min=ax-1 $a |- ( ph -> ( ( ph -> ph ) -> ph )
  )
This step assigns source "ax-1" ($a) to target "min" ($e).
The source assertion requires the hypotheses "wph" ($f, step
19) and "wps" ($f, step 20).  The parent assertion of the
target hypothesis is "ax-mp" ($a, step 22).
The source assertion before substitution was:
    ax-1 $a |- ( ph -> ( ps -> ph ) )
The following substitutions were made to the source
assertion:
    Variable  Substituted with
     ph        ph
     ps        ( ph -> ph )
The target hypothesis before substitution was:
    min $e |- ph
The following substitution was made to the target hypothesis:
    Variable  Substituted with
     ph        ( ph -> ( ( ph -> ph ) -> ph ) )
\end{verbatim}

This shows the substitutions\index{substitution!variable}\index{variable
substitution} made to the variables in \texttt{ax-1}.  References are made to
steps 19 and 20 which are not shown in our proof display.  To see these steps,
you can display the proof without the \texttt{essential} qualifier.

Let's look at a slightly more advanced proof of propositional calculus.  Note
that \verb+/\+ is the symbol for $\wedge$ (logical {\sc and}, also
called conjunction).\index{conjunction ($\wedge$)}
\index{logical {\sc and} ($\wedge$)}

\begin{verbatim}
MM> show statement prth/full
Statement 1521 is located on line 4730 of the file "set.mm".
"Theorem *3.47 of Whitehead and Russell, called 'praeclarum
theorema' by Leibniz."
  prth $p |- ( ( ( ph -> ps ) /\ ( ch -> th ) ) -> ( ( ph /\
      ch ) -> ( ps /\ th ) ) ) $= ... $.
Its mandatory hypotheses in RPN order are:
  wph $f wff ph $.
  wps $f wff ps $.
  wch $f wff ch $.
  wth $f wff th $.
Its optional hypotheses are:  wet
The statement and its hypotheses require the variables:  ph
      ps ch th
These additional variables are allowed in its proof:  et
The variables it contains are:  ph ps ch th

MM> show proof prth/essential/lemmon/renumber
1 pm3.2          $p |- ( ps -> ( th -> ( ps /\ th ) ) )
2 1 syl3dt       $p |- ( ps -> ( ( ch -> th ) -> ( ch -> ( ps
                                                /\ th ) ) ) )
3 2 syl3         $p |- ( ( ph -> ps ) -> ( ph -> ( ( ch -> th
                            ) -> ( ch -> ( ps /\ th ) ) ) ) )
4 3 com23        $p |- ( ( ph -> ps ) -> ( ( ch -> th ) -> (
                           ph -> ( ch -> ( ps /\ th ) ) ) ) )
5 4 impa         $p |- ( ( ( ph -> ps ) /\ ( ch -> th ) ) ->
                           ( ph -> ( ch -> ( ps /\ th ) ) ) )
6 impexp         $p |- ( ( ( ph /\ ch ) -> ( ps /\ th ) ) <->
                           ( ph -> ( ch -> ( ps /\ th ) ) ) )
7 5,6 sylibr     $p |- ( ( ( ph -> ps ) /\ ( ch -> th ) ) ->
                           ( ( ph /\ ch ) -> ( ps /\ th ) ) )
\end{verbatim}

There are references to a lot of unfamiliar statements.  To see what they are,
you may type the following:

\begin{verbatim}
MM> show proof prth/statement_summary/essential
Summary of statements used in the proof of "prth":

Statement "syl3" is located on line 355 of the file "set.mm".
"Inference adding common antecedents in an implication."
  syl3.1 $e |- ( ph -> ps ) $.
  syl3 $p |- ( ( ch -> ph ) -> ( ch -> ps ) ) $= ... $.

Statement "syl3dt" is located on line 438 of the file
"set.mm".
"Deduction adding nested antecedents."
  syl3dt.1 $e |- ( ph -> ( ps -> ch ) ) $.
  syl3dt $p |- ( ph -> ( ( th -> ps ) -> ( th -> ch ) ) ) $=
      ... $.
\end{verbatim}
\begin{center}
(etc.)
\end{center}

Of course you can look at each of these statements and their proofs, and
so on, back to the axioms of propositional calculus if you wish.

The \texttt{search} command is useful for finding statements when you
know all or part of their contents.  The following example finds all
statements containing \verb@ph -> ps@ followed by \verb@ch -> th@.  The
\verb@$*@ is a wildcard that matches anything; the \texttt{\$} before the
\verb$*$ prevents conflicts with math symbol token names.  The \verb@*@ after
\texttt{SEARCH} is also a wildcard that in this case means ``match any label.''
\index{\texttt{search} command}

\begin{verbatim}
MM> SEARCH * "ph -> ps $* ch -> th"
1521 prth $p |- ( ( ( ph -> ps ) /\ ( ch -> th ) ) -> ( ( ph
    /\ ch ) -> ( ps /\ th ) ) )
1522 pm3.48 $p |- ( ( ( ph -> ps ) /\ ( ch -> th ) ) -> ( (
    ph \/ ch ) -> ( ps \/ th ) ) )
1739 elimant $p |- ( ( ( ph -> ps ) /\ ( ( ps -> ch ) -> ( ph
    -> th ) ) ) -> ( ph -> ( ch -> th ) ) )
\end{verbatim}

Three statements, \texttt{prth}, \texttt{pm3.48},
 and \texttt{elimant}, were found to match.

To see what axioms\index{axiom} and definitions\index{definition}
\texttt{prth} ultimately depends on for its proof, you can have the
program backtrack through the hierarchy\index{hierarchy} of theorems and
definitions.\index{\texttt{show trace{\char`\_}back} command}

\begin{verbatim}
MM> show trace_back prth /essential/axioms
Statement "prth" assumes the following axioms ($a
statements):
  ax-1 ax-2 ax-3 ax-mp df-bi df-an
\end{verbatim}

Note that the 3 axioms of propositional calculus and the modus ponens rule are
needed (as expected); in addition, there are a couple of definitions that are used
along the way.  Note that Metamath makes no distinction\index{axiom vs.\
definition} between axioms\index{axiom} and definitions\index{definition}.  In
\texttt{set.mm} they have been distinguished artificially by prefixing their
labels\index{labels in \texttt{set.mm}} with \texttt{ax-} and \texttt{df-}
respectively.  For example, \texttt{df-an} defines conjunction (logical {\sc
and}), which is represented by the symbol \verb+/\+.
Section~\ref{definitions} discusses the philosophy of definitions, and the
Metamath language takes a particularly simple, conservative approach by using
the \texttt{\$a}\index{\texttt{\$a} statement} statement for both axioms and
definitions.

You can also have the program compute how many steps a proof
has\index{proof length} if we were to follow it all the way back to
\texttt{\$a} statements.

\begin{verbatim}
MM> show trace_back prth /essential/count_steps
The statement's actual proof has 5 steps.  Backtracking, a
total of 55 different subtheorems are used.  The statement
and subtheorems have a total of 196 actual steps.  If
subtheorems used only once were eliminated, there would be a
total of 25 subtheorems, and the statement and subtheorems
would have a total of 143 steps.  The maximum path length is
20.  A longest path is:  prth <- imp4b <- imp4a <- impexp <-
imbi1i <- impbi <- bi3 <- expi <- expt <- pm3.2im <- con2d <-
con2 <- nega <- pm2.18 <- pm2.43i <- pm2.43 <- pm2.27 <-
com12 <- syl <- a1i <- a1i.1 .
\end{verbatim}

This tells us that we would have to inspect 196 steps if we want to
verify the proof completely starting from the axioms.  A few more
statistics are also shown.  There are one or more paths back to axioms
that are the longest; this command ferrets out one of them and shows it
to you.  There may be a sense in which the longest path length is
related to how ``deep'' the theorem is.

Finally, we might be curious about what proofs depend on the theorem
\texttt{prth}.  If it is never used later on, we could eliminate it as
redundant if it has no intrinsic interest by itself.\index{\texttt{show
usage} command}

\begin{verbatim}
MM> show usage prth
Statement "prth" is directly referenced in the proofs of 4
statements:
  anim12d mo 2mo ssxp tfrlem5 climunii climadd
\end{verbatim}

Thus \texttt{prth} is used by 7 proofs, and indirectly by many more that
make use of {\em those} proofs, and so on.  (The \texttt{/recursive}
qualifier gives you all of them.)

\subsection{A Note on ``Compact'' Proof Format}

The present version of Metamath\index{Metamath!limitations of
version 0.07.30} will display proofs in a ``compact''\index{compact proof}
format whenever the proof is stored in compressed format in the database.  It
may be be slightly confusing unless you know how to interpret it.  (A
future version may eliminate this format from displays.)  For example,
if you display the complete proof of theorem \texttt{id1} it will start
off as follows:

\begin{verbatim}
MM> show proof id1 /lemmon/all
 1 wph           $f wff ph
 2 wph           $f wff ph
 3 wph           $f wff ph
 4 2,3 wi    @1: $a wff ( ph -> ph )
 5 1,4 wi    @2: $a wff ( ph -> ( ph -> ph ) )
 6 @1            $a wff ( ph -> ph )
\end{verbatim}

\begin{center}
{etc.}
\end{center}

Step 4 has a ``local label,''\index{local label} \texttt{@1}, assigned to it.
Later on, at step 6, the label \texttt{@1} is referenced instead of
displaying the explicit proof for that step.  This technique takes advantage
of the fact that steps in a proof often repeat, especially during the
construction of wffs.  The compact format reduces the number of steps in the
proof display and may actually be preferred by some people, which is one
reason that it is still displayed by the program.

If you want to see the normal format with the ``true'' step numbers, you can
use the following workaround:\index{\texttt{save proof} command}

\begin{verbatim}
MM> save proof id1 /normal
The proof of "id1" has been reformatted and saved internally.
Remember to use WRITE SOURCE to save it permanently.
MM> show proof id1 /lemmon/all
 1 wph           $f wff ph
 2 wph           $f wff ph
 3 wph           $f wff ph
 4 2,3 wi        $a wff ( ph -> ph )
 5 1,4 wi        $a wff ( ph -> ( ph -> ph ) )
 6 wph           $f wff ph
 7 wph           $f wff ph
 8 6,7 wi        $a wff ( ph -> ph )
\end{verbatim}

\begin{center}
{etc.}
\end{center}

Note that the original 6 steps are now 8 steps.  However, the format is
now the same as that described in Chapter~\ref{using}.

\chapter{The Metamath Language}
\label{languagespec}

\begin{quote}
  {\em Thus mathematics may be defined as the subject in which we never know
what we are talking about, nor whether what we are saying is true.}
    \flushright\sc  Bertrand Russell\footnote{\cite[p.~84]{Russell2}.}\\
\end{quote}\index{Russell, Bertrand}

Probably the most striking feature of the Metamath language is its almost
complete absence of hard-wired syntax. Metamath\index{Metamath} does not
understand any mathematics or logic other than that needed to construct finite
sequences of symbols according to a small set of simple, built-in rules.  The
only rule it uses in a proof is the substitution of an expression (symbol
sequence) for a variable, subject to a simple constraint to prevent
bound-variable clashes.  The primitive notions built into Metamath involve the
simple manipulation of finite objects (symbols) that we as humans can easily
visualize and that computers can easily deal with.  They seem to be just
about the simplest notions possible that are required to do standard
mathematics.

This chapter serves as a reference manual for the Metamath\index{Metamath}
language. It covers the tedious technical details of the language, some of
which you may wish to skim in a first reading.  On the other hand, you should
pay close attention to the defined terms in {\bf boldface}; they have precise
meanings that are important to keep in mind for later understanding.  It may
be best to first become familiar with the examples in Chapter~\ref{using} to
gain some motivation for the language.

%% Uncomment this when uncommenting section {formalspec} below
If you have some knowledge of set theory, you may wish to study this
chapter in conjunction with the formal set-theoretical description of the
Metamath language in Appendix~\ref{formalspec}.

We will use the name ``Metamath''\index{Metamath} to mean either the Metamath
computer language or the Metamath software associated with the computer
language.  We will not distinguish these two when the context is clear.

The next section contains the complete specification of the Metamath
language.
It serves as an
authoritative reference and presents the syntax in enough detail to
write a parser\index{parsing Metamath} and proof verifier.  The
specification is terse and it is probably hard to learn the language
directly from it, but we include it here for those impatient people who
prefer to see everything up front before looking at verbose expository
material.  Later sections explain this material and provide examples.
We will repeat the definitions in those sections, and you may skip the
next section at first reading and proceed to Section~\ref{tut1}
(p.~\pageref{tut1}).

\section{Specification of the Metamath Language}\label{spec}
\index{Metamath!specification}

\begin{quote}
  {\em Sometimes one has to say difficult things, but one ought to say
them as simply as one knows how.}
    \flushright\sc  G. H. Hardy\footnote{As quoted in
    \cite{deMillo}, p.~273.}\\
\end{quote}\index{Hardy, G. H.}

\subsection{Preliminaries}\label{spec1}

A Metamath {\bf database}\index{database} is built up from a top-level source
file together with any source files that are brought in through file inclusion
commands (see below).  The only characters that are allowed to appear in a
Metamath source file are the 94 printable {\sc
ascii}\index{ascii@{\sc ascii}} characters, which are digits, upper and lower
case letters, and the following 32 special characters\index{special characters}
\label{spec1chars}

\begin{verbatim}
        ` ~ ! @ # $ % ^ & * ( ) - _ = +
        [ ] { } ; : ' " , . < > / ? \ |
\end{verbatim}
plus the following non-printable (white space) characters: space, tab,
carriage return, line feed, and form feed.  We will use \texttt{typewriter}
font to display the printable characters.

A Metamath database consists of a sequence of three kinds of {\bf
tokens}\index{token} separated by {\bf white space}\index{white space}
(which is any sequence of one or more white space characters).  The set
of {\bf keyword}\index{keyword} tokens is \texttt{\$\char`\{},
\texttt{\$\char`\}}, \texttt{\$c}, \texttt{\$v}, \texttt{\$f},
\texttt{\$e}, \texttt{\$d}, \texttt{\$a}, \texttt{\$p}, \texttt{\$.},
\texttt{\$=}, \texttt{\$(}, \texttt{\$)}, \texttt{\$[}, and
\texttt{\$]}.  The last four are called {\bf auxiliary}\index{auxiliary
keyword} or preprocessing keywords.  A {\bf label}\index{label} token
consists of any combination of letters, digits, and the characters
hyphen, underscore, and period.  A {\bf math symbol}\index{math symbol}
token may consist of any combination of the 93 printable standard {\sc
ascii} characters other than \texttt{\$}~. All tokens are
case-sensitive.

\subsection{Preprocessing}

\texttt{\$(} begins a {\bf comment} and \texttt{\$)} ends a comment.\index{\texttt{\$(}
and \texttt{\$)} auxiliary keywords}\index{comment}  Comments may contain any of
the 94 printable characters and white space, except they may not contain the
2-character sequences \texttt{\$(} or \texttt{\$)} (comments do not nest).  Comments are ignored (treated
like white space) for the purpose of parsing, e.g.,
\texttt{\$( \$[ \$)} is a comment.
See p.~\pageref{mathcomments} for comment typesetting conventions; these
conventions may be ignored for the purpose of parsing.

A {\bf file inclusion command} consists of \texttt{\$[} followed by a file name
followed by \texttt{\$]}.\index{\texttt{\$[} and \texttt{\$]} auxiliary
keywords}\index{included file}\index{file inclusion}
It is only allowed in the outermost scope (i.e., not between 
\texttt{\$\char`\{} and \texttt{\$\char`\}})
and must not be inside of a statement (e.g., it may not occur
between the label of a \texttt{\$a} statement and its \texttt{\$.}).
The file name may not
contain a \texttt{\$} or white space.  The file must exist.
The case-sensitivity
of its name follows the conventions of the operating system.  The contents of
the file replace the inclusion command.
Included files may include other files.
Only the first reference to a given file is included; any later
references to the same file (whether in the top-level file or in included
files) cause the inclusion command to be ignored (treated like white space).
A verifier may assume that file names with different strings
refer to different files for the purpose of ignoring later references.
A file self-reference is ignored, as is any reference to the top-level file.
Included files may not include a \texttt{\$(} without a matching \texttt{\$)},
may not include a \texttt{\$[} without a matching \texttt{\$]}, and may
not include incomplete statements (e.g., a \texttt{\$a} without a matching
\texttt{\$.}).
It is currently unspecified if path references are relative to the process'
current directory or the file's containing directory, so databases should
avoid using pathname separators (e.g., ``/'') in file names.

Like all tokens, the \texttt{\$(}, \texttt{\$)}, \texttt{\$[}, and \texttt{\$]} keywords
must be surrounded by white space.

\subsection{Basic Syntax}

After preprocessing, a database will consist of a sequence of {\bf
statements}.
These are the scoping statements \texttt{\$\char`\{} and
\texttt{\$\char`\}}, along with the \texttt{\$c}, \texttt{\$v},
\texttt{\$f}, \texttt{\$e}, \texttt{\$d}, \texttt{\$a}, and \texttt{\$p}
statements.

A {\bf scoping statement}\index{scoping statement} consists only of its
keyword, \texttt{\$\char`\{} or \texttt{\$\char`\}}. \texttt{\$\char`\{} begins a {\bf
block}\index{block} and a matching \texttt{\$\char`\}} ends the block. Every \texttt{\$\char`\{}
must have a matching \texttt{\$\char`\}}. Defining it recursively, we say a block
contains a sequence of zero or more tokens other
than \texttt{\$\char`\{} and \texttt{\$\char`\}} and
possibly other blocks.  There is an {\bf outermost
block}\index{block!outermost} not bracketed by \texttt{\$\char`\{} \texttt{\$\char`\}}; the end
of the outermost block is the end of the database.

% LaTeX bug? can't do \bf\tt

A {\bf \$v} or {\bf \$c statement}\index{\texttt{\$v} statement}\index{\texttt{\$c}
statement} consists of the keyword token \texttt{\$v} or \texttt{\$c} respectively,
followed by one or more math symbols, followed by \texttt{\$.}\,.  These
statements {\bf declare}\index{declaration} the math symbols to be {\bf
variables}\index{variable!Metamath} or {\bf constants}\index{constant}
respectively. The same math symbol may not occur twice in a given \texttt{\$v} or
\texttt{\$c} statement.

%c%A math symbol becomes an {\bf active}\index{active math symbol}
%c%when declared and stays active until the end of the block in which it is
%c%declared.  A math symbol may not be declared a second time while it is active,
%c%but it may be declared again after it becomes inactive.

A math symbol becomes {\bf active}\index{active math symbol} when declared
and stays active until the end of the block in which it is declared.  A
variable may not be declared a second time while it is active, but it
may be declared again (as a variable, but not as a constant) after it
becomes inactive.  A constant must be declared in the outermost block and may
not be declared a second time.\footnote{The rules for redeclaration may
become more general in the future;
see p.~\pageref{redeclarationf}.}\index{redeclaration of symbols}

A {\bf \$f statement}\index{\texttt{\$f} statement} consists of a label,
followed by \texttt{\$f}, followed by an active constant (its type), followed by an
active variable, followed by \texttt{\$.}\,.  A {\bf \$e
statement}\index{\texttt{\$e} statement} consists of a label, followed
by \texttt{\$e}, followed by an active constant, followed by zero or more
active math symbols, followed by \texttt{\$.}\,.  A {\bf
hypothesis}\index{hypothesis} is a \texttt{\$f} or \texttt{\$e}
statement.
The type declared by a \texttt{\$f} statement for a given label
is global even if the variable is not
(e.g., a database may not have \texttt{wff P} in one local scope
nd \texttt{class P} in another).
A \texttt{\$e} statement may not occur in the outermost scope.

A {\bf simple \$d statement}\index{\texttt{\$d} statement!simple}
consists of \texttt{\$d}, followed by two different active variables,
followed by \texttt{\$.}\,.  A {\bf compound \$d
statement}\index{\texttt{\$d} statement!compound} consists of
\texttt{\$d}, followed by three or more variables (all different),
followed by \texttt{\$.}\,.  The order of the variables in a
\texttt{\$d} statement is unimportant.  A compound \texttt{\$d}
statement is equivalent to a set of simple \texttt{\$d} statements, one
for each possible pair of variables occurring in the compound
\texttt{\$d} statement.  Henceforth in this specification we shall
assume all \texttt{\$d} statements are simple.  A \texttt{\$d} statement
is also called a {\bf disjoint} (or {\bf distinct}) {\bf variable
restriction}.\index{disjoint-variable restriction}

A {\bf \$a statement}\index{\texttt{\$a} statement} consists of a label,
followed by \texttt{\$a}, followed by an active constant, followed by
zero or more active math symbols, followed by \texttt{\$.}\,.  A {\bf
\$p statement}\index{\texttt{\$p} statement} consists of a label,
followed by \texttt{\$p}, followed by an active constant, followed by
zero or more active math symbols, followed by \texttt{\$=}, followed by
a sequence of labels, followed by \texttt{\$.}\,.  An {\bf
assertion}\index{assertion} is a \texttt{\$a} or \texttt{\$p} statement.

A \texttt{\$f}, \texttt{\$e}, or \texttt{\$d} statement is {\bf active}\index{active
statement} from the place it occurs until the end of the block it occurs in.
A \texttt{\$a} or \texttt{\$p} statement is {\bf active} from the place it occurs
through the end of the database.
There may not be two active \texttt{\$f} statements containing the same
variable.  Each variable in a \texttt{\$e}, \texttt{\$a}, or
\texttt{\$p} statement must exist in an active \texttt{\$f}
statement.\footnote{This requirement can greatly simplify the
unification algorithm (substitution calculation) required by proof
verification.}

%The label that begins each \texttt{\$f}, \texttt{\$e}, \texttt{\$a}, and
%\texttt{\$p} statement must be unique.
Each label token must be unique, and
no label token may match any math symbol
token.\label{namespace}\footnote{This
restriction was added on June 24, 2006.
It is not theoretically necessary but is imposed to make it easier to
write certain parsers.}

The set of {\bf mandatory variables}\index{mandatory variable} associated with
an assertion is the set of (zero or more) variables in the assertion and in any
active \texttt{\$e} statements.  The (possibly empty) set of {\bf mandatory
hypotheses}\index{mandatory hypothesis} is the set of all active \texttt{\$f}
statements containing mandatory variables, together with all active \texttt{\$e}
statements.
The set of {\bf mandatory {\bf \$d} statements}\index{mandatory
disjoint-variable restriction} associated with an assertion are those active
\texttt{\$d} statements whose variables are both among the assertion's
mandatory variables.

\subsection{Proof Verification}\label{spec4}

The sequence of labels between the \texttt{\$=} and \texttt{\$.}\ tokens
in a \texttt{\$p} statement is a {\bf proof}.\index{proof!Metamath} Each
label in a proof must be the label of an active statement other than the
\texttt{\$p} statement itself; thus a label must refer either to an
active hypothesis of the \texttt{\$p} statement or to an earlier
assertion.

An {\bf expression}\index{expression} is a sequence of math symbols. A {\bf
substitution map}\index{substitution map} associates a set of variables with a
set of expressions.  It is acceptable for a variable to be mapped to an
expression containing it.  A {\bf
substitution}\index{substitution!variable}\index{variable substitution} is the
simultaneous replacement of all variables in one or more expressions with the
expressions that the variables map to.

A proof is scanned in order of its label sequence.  If the label refers to an
active hypothesis, the expression in the hypothesis is pushed onto a
stack.\index{stack}\index{RPN stack}  If the label refers to an assertion, a
(unique) substitution must exist that, when made to the mandatory hypotheses
of the referenced assertion, causes them to match the topmost (i.e.\ most
recent) entries of the stack, in order of occurrence of the mandatory
hypotheses, with the topmost stack entry matching the last mandatory
hypothesis of the referenced assertion.  As many stack entries as there are
mandatory hypotheses are then popped from the stack.  The same substitution is
made to the referenced assertion, and the result is pushed onto the stack.
After the last label in the proof is processed, the stack must have a single
entry that matches the expression in the \texttt{\$p} statement containing the
proof.

%c%{\footnotesize\begin{quotation}\index{redeclaration of symbols}
%c%{{\em Comment.}\label{spec4comment} Whenever a math symbol token occurs in a
%c%{\texttt{\$c} or \texttt{\$v} statement, it is considered to designate a distinct new
%c%{symbol, even if the same token was previously declared (and is now inactive).
%c%{Thus a math token declared as a constant in two different blocks is considered
%c%{to designate two distinct constants (even though they have the same name).
%c%{The two constants will not match in a proof that references both blocks.
%c%{However, a proof referencing both blocks is acceptable as long as it doesn't
%c%{require that the constants match.  Similarly, a token declared to be a
%c%{constant for a referenced assertion will not match the same token declared to
%c%{be a variable for the \texttt{\$p} statement containing the proof.  In the case
%c%{of a token declared to be a variable for a referenced assertion, this is not
%c%{an issue since the variable can be substituted with whatever expression is
%c%{needed to achieve the required match.
%c%{\end{quotation}}
%c2%A proof may reference an assertion that contains or whose hypotheses contain a
%c2%constant that is not active for the \texttt{\$p} statement containing the proof.
%c2%However, the final result of the proof may not contain that constant. A proof
%c2%may also reference an assertion that contains or whose hypotheses contain a
%c2%variable that is not active for the \texttt{\$p} statement containing the proof.
%c2%That variable, of course, will be substituted with whatever expression is
%c2%needed to achieve the required match.

A proof may contain a \texttt{?}\ in place of a label to indicate an unknown step
(Section~\ref{unknown}).  A proof verifier may ignore any proof containing
\texttt{?}\ but should warn the user that the proof is incomplete.

A {\bf compressed proof}\index{compressed proof}\index{proof!compressed} is an
alternate proof notation described in Appen\-dix~\ref{compressed}; also see
references to ``compressed proof'' in the Index.  Compressed proofs are a
Metamath language extension which a complete proof verifier should be able to
parse and verify.

\subsubsection{Verifying Disjoint Variable Restrictions}

Each substitution made in a proof must be checked to verify that any
disjoint variable restrictions are satisfied, as follows.

If two variables replaced by a substitution exist in a mandatory \texttt{\$d}
statement\index{\texttt{\$d} statement} of the assertion referenced, the two
expressions resulting from the substitution must satisfy the following
conditions.  First, the two expressions must have no variables in common.
Second, each possible pair of variables, one from each expression, must exist
in an active \texttt{\$d} statement of the \texttt{\$p} statement containing the
proof.

\vskip 1ex

This ends the specification of the Metamath language.


\section{The Basic Keywords}\label{tut1}

Our expository material begins here.

Like most computer languages, Metamath\index{Metamath} takes its input from
one or more {\bf source files}\index{source file} which contain characters
expressed in the standard {\sc ascii} (American Standard Code for Information
Interchange)\index{ascii@{\sc ascii}} code for computers.  A source file
consists of a series of {\bf tokens}\index{token}, which are strings of
printable characters (from the set of 94 shown on p.~\pageref{spec1chars})
separated by {\bf white space}\index{white space} (spaces, tabs, carriage
returns, line feeds, and form feeds). Any string consisting only of these
characters is treated the same as a single space.  The printable
characters\index{printable character} that Metamath recognizes are the 94
characters on standard {\sc ascii} keyboards.

Metamath has the ability to join several files together to form its
input (Section~\ref{include}).  We call the aggregate contents of all
the files after they have been joined together a {\bf
database}\index{database} to distinguish it from an individual source
file.  The tokens in a database consist of {\bf
keywords}\index{keyword}, which are built into the language, together
with two kinds of user-defined tokens called {\bf labels}\index{label}
and {\bf math symbols}\index{math symbol}.  (Often we will simply say
{\bf symbol}\index{symbol} instead of math symbol for brevity).  The set
of {\bf basic keywords}\index{basic keyword} is
\texttt{\$c}\index{\texttt{\$c} statement},
\texttt{\$v}\index{\texttt{\$v} statement},
\texttt{\$e}\index{\texttt{\$e} statement},
\texttt{\$f}\index{\texttt{\$f} statement},
\texttt{\$d}\index{\texttt{\$d} statement},
\texttt{\$a}\index{\texttt{\$a} statement},
\texttt{\$p}\index{\texttt{\$p} statement},
\texttt{\$=}\index{\texttt{\$=} keyword},
\texttt{\$.}\index{\texttt{\$.}\ keyword},
\texttt{\$\char`\{}\index{\texttt{\$\char`\{} and \texttt{\$\char`\}}
keywords}, and \texttt{\$\char`\}}.  This is the complete set of
syntactical elements of what we call the {\bf basic
language}\index{basic language} of Metamath, and with them you can
express all of the mathematics that were intended by the design of
Metamath.  You should make it a point to become very familiar with them.
Table~\ref{basickeywords} lists the basic keywords along with a brief
description of their functions.  For now, this description will give you
only a vague notion of what the keywords are for; later we will describe
the keywords in detail.


\begin{table}[htp] \caption{Summary of the basic Metamath
keywords} \label{basickeywords}
\begin{center}
\begin{tabular}{|p{4pc}|l|}
\hline
\em \centering Keyword&\em Description\\
\hline
\hline
\centering
   \texttt{\$c}&Constant symbol declaration\\
\hline
\centering
   \texttt{\$v}&Variable symbol declaration\\
\hline
\centering
   \texttt{\$d}&Disjoint variable restriction\\
\hline
\centering
   \texttt{\$f}&Variable-type (``floating'') hypothesis\\
\hline
\centering
   \texttt{\$e}&Logical (``essential'') hypothesis\\
\hline
\centering
   \texttt{\$a}&Axiomatic assertion\\
\hline
\centering
   \texttt{\$p}&Provable assertion\\
\hline
\centering
   \texttt{\$=}&Start of proof in \texttt{\$p} statement\\
\hline
\centering
   \texttt{\$.}&End of the above statement types\\
\hline
\centering
   \texttt{\$\char`\{}&Start of block\\
\hline
\centering
   \texttt{\$\char`\}}&End of block\\
\hline
\end{tabular}
\end{center}
\end{table}

%For LaTeX bug(?) where it puts tables on blank page instead of btwn text
%May have to adjust if text changes
%\newpage

There are some additional keywords, called {\bf auxiliary
keywords}\index{auxiliary keyword} that help make Metamath\index{Metamath}
more practical. These are part of the {\bf extended language}\index{extended
language}. They provide you with a means to put comments into a Metamath
source file\index{source file} and reference other source files.  We will
introduce these in later sections. Table~\ref{otherkeywords} summarizes them
so that you can recognize them now if you want to peruse some source
files while learning the basic keywords.


\begin{table}[htp] \caption{Auxiliary Metamath
keywords} \label{otherkeywords}
\begin{center}
\begin{tabular}{|p{4pc}|l|}
\hline
\em \centering Keyword&\em Description\\
\hline
\hline
\centering
   \texttt{\$(}&Start of comment\\
\hline
\centering
   \texttt{\$)}&End of comment\\
\hline
\centering
   \texttt{\$[}&Start of included source file name\\
\hline
\centering
   \texttt{\$]}&End of included source file name\\
\hline
\end{tabular}
\end{center}
\end{table}
\index{\texttt{\$(} and \texttt{\$)} auxiliary keywords}
\index{\texttt{\$[} and \texttt{\$]} auxiliary keywords}


Unlike those in some computer languages, the keywords\index{keyword} are short
two-character sequences rather than English-like words.  While this may make
them slightly more difficult to remember at first, their brevity allows
them to blend in with the mathematics being described, not
distract from it, like punctuation marks.


\subsection{User-Defined Tokens}\label{dollardollar}\index{token}

As you may have noticed, all keywords\index{keyword} begin with the \texttt{\$}
character.  This mundane monetary symbol is not ordinarily used in higher
mathematics (outside of grant proposals), so we have appropriated it to
distinguish the Metamath\index{Metamath} keywords from ordinary mathematical
symbols. The \texttt{\$} character is thus considered special and may not be
used as a character in a user-defined token.  All tokens and keywords are
case-sensitive; for example, \texttt{n} is considered to be a different character
from \texttt{N}.  Case-sensitivity makes the available {\sc ascii} character set
as rich as possible.

\subsubsection{Math Symbol Tokens}\index{token}

Math symbols\index{math symbol} are tokens used to represent the symbols
that appear in ordinary mathematical formulas.  They may consist of any
combination of the 93 printable {\sc ascii} characters other than
\texttt{\$}~. Some examples are \texttt{x}, \texttt{+}, \texttt{(},
\texttt{|-}, \verb$!%@?&$, and \texttt{bounded}.  For readability, it is
best to try to make these look as similar to actual mathematical symbols
as possible, within the constraints of the {\sc ascii} character set, in
order to make the resulting mathematical expressions more readable.

In the Metamath\index{Metamath} language, you express ordinary
mathematical formulas and statements as sequences of math symbols such
as \texttt{2 + 2 = 4} (five symbols, all constants).\footnote{To
eliminate ambiguity with other expressions, this is expressed in the set
theory database \texttt{set.mm} as \texttt{|- ( 2 + 2
 ) = 4 }, whose \LaTeX\ equivalent is $\vdash
(2+2)=4$.  The \,$\vdash$ means ``is a theorem'' and the
parentheses allow explicit associative grouping.}\index{turnstile
({$\,\vdash$})} They may even be English
sentences, as in \texttt{E is closed and bounded} (five symbols)---here
\texttt{E} would be a variable and the other four symbols constants.  In
principle, a Metamath database could be constructed to work with almost
any unambiguous English-language mathematical statement, but as a
practical matter the definitions needed to provide for all possible
syntax variations would be cumbersome and distracting and possibly have
subtle pitfalls accidentally built in.  We generally recommend that you
express mathematical statements with compact standard mathematical
symbols whenever possible and put their English-language descriptions in
comments.  Axioms\index{axiom} and definitions\index{definition}
(\texttt{\$a}\index{\texttt{\$a} statement} statements) are the only
places where Metamath will not detect an error, and doing this will help
reduce the number of definitions needed.

You are free to use any tokens\index{token} you like for math
symbols\index{math symbol}.  Appendix~\ref{ASCII} recommends token names to
use for symbols in set theory, and we suggest you adopt these in order to be
able to include the \texttt{set.mm} set theory database in your database.  For
printouts, you can convert the tokens in a database
to standard mathematical symbols with the \LaTeX\ typesetting program.  The
Metamath command \texttt{open tex} {\em filename}\index{\texttt{open tex} command}
produces output that can be read by \LaTeX.\index{latex@{\LaTeX}}
The correspondence
between tokens and the actual symbols is made by \texttt{latexdef}
statements inside of a special database comment tagged
with \texttt{\$t}.\index{\texttt{\$t} comment}\index{typesetting comment}
  You can edit
this comment to change the definitions or add new ones.
Appendix~\ref{ASCII} describes how to do this in more detail.

% White space\index{white space} is normally used to separate math
% symbol\index{math symbol} tokens, but they may be juxtaposed without white
% space in \texttt{\$d}\index{\texttt{\$d} statement}, \texttt{\$e}\index{\texttt{\$e}
% statement}, \texttt{\$f}\index{\texttt{\$f} statement}, \texttt{\$a}\index{\texttt{\$a}
% statement}, and \texttt{\$p}\index{\texttt{\$p} statement} statements when no
% ambiguity will result.  Specifically, Metamath parses the math symbol sequence
% in one of these statements in the following manner:  when the math symbol
% sequence has been broken up into tokens\index{token} up to a given character,
% the next token is the longest string of characters that could constitute a
% math symbol that is active\index{active
% math symbol} at that point.  (See Section~\ref{scoping} for the
% definition of an active math symbol.)  For example, if \texttt{-}, \texttt{>}, and
% \texttt{->} are the only active math symbols, the juxtaposition \texttt{>-} will be
% interpreted as the two symbols \texttt{>} and \texttt{-}, whereas \texttt{->} will
% always be interpreted as that single symbol.\footnote{For better readability we
% recommend a white space between each token.  This also makes searching for a
% symbol easier to do with an editor.  Omission of optional white space is useful
% for reducing typing when assigning an expression to a temporary
% variable\index{temporary variable} with the \texttt{let variable} Metamath
% program command.}\index{\texttt{let variable} command}
%
% Keywords\index{keyword} may be placed next to math symbols without white
% space\index{white space} between them.\footnote{Again, we do not recommend
% this for readability.}
%
% The math symbols\index{math symbol} in \texttt{\$c}\index{\texttt{\$c} statement}
% and \texttt{\$v}\index{\texttt{\$v} statement} statements must always be separated
% by white space\index{white
% space}, for the obvious reason that these statements define the names
% of the symbols.
%
% Math symbols referred to in comments (see Section~\ref{comments}) must also be
% separated by white space.  This allows you to make comments about symbols that
% are not yet active\index{active
% math symbol}.  (The ``math mode'' feature of comments is also a quick and
% easy way to obtain word processing text with embedded mathematical symbols,
% independently of the main purpose of Metamath; the way to do this is described
% in Section~\ref{comments})

\subsubsection{Label Tokens}\index{token}\index{label}

Label tokens are used to identify Metamath\index{Metamath} statements for
later reference. Label tokens may contain only letters, digits, and the three
characters period, hyphen, and underscore:\begin{verbatim}
        . - _
\end{verbatim}

A label is {\bf declared}\index{label declaration} by placing it immediately
before the keyword of the statement it identifies.  For example, the label
\texttt{axiom.1} might be declared as follows:
\begin{verbatim}
        axiom.1 $a |- x = x $.
\end{verbatim}

Each \texttt{\$e}\index{\texttt{\$e} statement},
\texttt{\$f}\index{\texttt{\$f} statement},
\texttt{\$a}\index{\texttt{\$a} statement}, and
\texttt{\$p}\index{\texttt{\$p} statement} statement in a database must
have a label declared for it.  No other statement types may have label
declarations.  Every label must be unique.

A label (and the statement it identifies) is {\bf referenced}\index{label
reference} by including the label between the \texttt{\$=}\index{\texttt{\$=}
keyword} and \texttt{\$.}\index{\texttt{\$.}\ keyword}\ keywords in a \texttt{\$p}
statement.  The sequence of labels\index{label sequence} between these two
keywords is called a {\bf proof}\index{proof}.  An example of a statement with
a proof that we will encounter later (Section~\ref{proof}) is
\begin{verbatim}
        wnew $p wff ( s -> ( r -> p ) )
             $= ws wr wp w2 w2 $.
\end{verbatim}

You don't have to know what this means just yet, but you should know that the
label \texttt{wnew} is declared by this \texttt{\$p} statement and that the labels
\texttt{ws}, \texttt{wr}, \texttt{wp}, and \texttt{w2} are assumed to have been declared
earlier in the database and are referenced here.

\subsection{Constants and Variables}
\index{constant}
\index{variable}

An {\bf expression}\index{expression} is any sequence of math
symbols, possibly empty.

The basic Metamath\index{Metamath} language\index{basic language} has two
kinds of math symbols\index{math symbol}:  {\bf constants}\index{constant} and
{\bf variables}\index{variable}.  In a Metamath proof, a constant may not be
substituted with any expression.  A variable can be
substituted\index{substitution!variable}\index{variable substitution} with any
expression.  This sequence may include other variables and may even include
the variable being substituted.  This substitution takes place when proofs are
verified, and it will be described in Section~\ref{proof}.  The \texttt{\$f}
statement (described later in Section~\ref{dollaref}) is used to specify the
{\bf type} of a variable (i.e.\ what kind of
variable it is)\index{variable type}\index{type} and
give it a meaning typically
associated with a ``metavariable''\index{metavariable}\footnote{A metavariable
is a variable that ranges over the syntactical elements of the object language
being discussed; for example, one metavariable might represent a variable of
the object language and another metavariable might represent a formula in the
object language.} in ordinary mathematics; for example, a variable may be
specified to be a wff or well-formed formula (in logic), a set (in set
theory), or a non-negative integer (in number theory).

%\subsection{The \texttt{\$c} and \texttt{\$v} Declaration Statements}
\subsection{The \texttt{\$c} and \texttt{\$v} Declaration Statements}
\index{\texttt{\$c} statement}
\index{constant declaration}
\index{\texttt{\$v} statement}
\index{variable declaration}

Constants are introduced or {\bf declared}\index{constant declaration}
with \texttt{\$c}\index{\texttt{\$c} statement} statements, and
variables are declared\index{variable declaration} with
\texttt{\$v}\index{\texttt{\$v} statement} statements.  A {\bf simple}
declaration\index{simple declaration} statement introduces a single
constant or variable.  Its syntax is one of the following:
\begin{center}
  \texttt{\$c} {\em math-symbol} \texttt{\$.}\\
  \texttt{\$v} {\em math-symbol} \texttt{\$.}
\end{center}
The notation {\em math-symbol} means any math symbol token\index{token}.

Some examples of simple declaration statements are:
\begin{center}
  \texttt{\$c + \$.}\\
  \texttt{\$c -> \$.}\\
  \texttt{\$c ( \$.}\\
  \texttt{\$v x \$.}\\
  \texttt{\$v y2 \$.}
\end{center}

The characters in a math symbol\index{math symbol} being declared are
irrelevant to Meta\-math; for example, we could declare a right parenthesis to
be a variable,
\begin{center}
  \texttt{\$v ) \$.}\\
\end{center}
although this would be unconventional.

A {\bf compound} declaration\index{compound declaration} statement is a
shorthand for declaring several symbols at once.  Its syntax is one of the
following:
\begin{center}
  \texttt{\$c} {\em math-symbol}\ \,$\cdots$\ {\em math-symbol} \texttt{\$.}\\
  \texttt{\$v} {\em math-symbol}\ \,$\cdots$\ {\em math-symbol} \texttt{\$.}
\end{center}\index{\texttt{\$c} statement}
Here, the ellipsis (\ldots) means any number of {\em math-symbol}\,s.

An example of a compound declaration statement is:
\begin{center}
  \texttt{\$v x y mu \$.}\\
\end{center}
This is equivalent to the three simple declaration statements
\begin{center}
  \texttt{\$v x \$.}\\
  \texttt{\$v y \$.}\\
  \texttt{\$v mu \$.}\\
\end{center}
\index{\texttt{\$v} statement}

There are certain rules on where in the database math symbols may be declared,
what sections of the database are aware of them (i.e.\ where they are
``active''), and when they may be declared more than once.  These will be
discussed in Section~\ref{scoping} and specifically on
p.~\pageref{redeclaration}.

\subsection{The \texttt{\$d} Statement}\label{dollard}
\index{\texttt{\$d} statement}

The \texttt{\$d} statement is called a {\bf disjoint-variable restriction}.  The
syntax of the {\bf simple} version of this statement is
\begin{center}
  \texttt{\$d} {\em variable variable} \texttt{\$.}
\end{center}
where each {\em variable} is a previously declared variable and the two {\em
variable}\,s are different.  (More specifically, each  {\em variable} must be
an {\bf active} variable\index{active math symbol}, which means there must be
a previous \texttt{\$v} statement whose {\bf scope}\index{scope} includes the
\texttt{\$d} statement.  These terms will be defined when we discuss scoping
statements in Section~\ref{scoping}.)

In ordinary mathematics, formulas may arise that are true if the variables in
them are distinct\index{distinct variables}, but become false when those
variables are made identical. For example, the formula in logic $\exists x\,x
\neq y$, which means ``for a given $y$, there exists an $x$ that is not equal
to $y$,'' is true in most mathematical theories (namely all non-trivial
theories\index{non-trivial theory}, i.e.\ those that describe more than one
individual, such as arithmetic).  However, if we substitute $y$ with $x$, we
obtain $\exists x\,x \neq x$, which is always false, as it means ``there
exists something that is not equal to itself.''\footnote{If you are a
logician, you will recognize this as the improper substitution\index{proper
substitution}\index{substitution!proper} of a free variable\index{free
variable} with a bound variable\index{bound variable}.  Metamath makes no
inherent distinction between free and bound variables; instead, you let
Metamath know what substitutions are permissible by using \texttt{\$d} statements
in the right way in your axiom system.}\index{free vs.\ bound variable}  The
\texttt{\$d} statement allows you to specify a restriction that forbids the
substitution of one variable with another.  In
this case, we would use the statement
\begin{center}
  \texttt{\$d x y \$.}
\end{center}\index{\texttt{\$d} statement}
to specify this restriction.

The order in which the variables appear in a \texttt{\$d} statement is not
important.  We could also use
\begin{center}
  \texttt{\$d y x \$.}
\end{center}

The \texttt{\$d} statement is actually more general than this, as the
``disjoint''\index{disjoint variables} in its name suggests.  The full meaning
is that if any substitution is made to its two variables (during the
course of a proof that references a \texttt{\$a} or \texttt{\$p} statement
associated with the \texttt{\$d}), the two expressions that result from the
substitution must have no variables in common.  In addition, each possible
pair of variables, one from each expression, must be in a \texttt{\$d} statement
associated with the statement being proved.  (This requirement forces the
statement being proved to ``inherit'' the original disjoint variable
restriction.)

For example, suppose \texttt{u} is a variable.  If the restriction
\begin{center}
  \texttt{\$d A B \$.}
\end{center}
has been specified for a theorem referenced in a
proof, we may not substitute \texttt{A} with \mbox{\tt a + u} and
\texttt{B} with \mbox{\tt b + u} because these two symbol sequences have the
variable \texttt{u} in common.  Furthermore, if \texttt{a} and \texttt{b} are
variables, we may not substitute \texttt{A} with \texttt{a} and \texttt{B} with \texttt{b}
unless we have also specified \texttt{\$d a b} for the theorem being proved; in
other words, the \texttt{\$d} property associated with a pair of variables must
be effectively preserved after substitution.

The \texttt{\$d}\index{\texttt{\$d} statement} statement does {\em not} mean ``the
two variables may not be substituted with the same thing,'' as you might think
at first.  For example, substituting each of \texttt{A} and \texttt{B} in the above
example with identical symbol sequences consisting only of constants does not
cause a disjoint variable conflict, because two symbol sequences have no
variables in common (since they have no variables, period).  Similarly, a
conflict will not occur by substituting the two variables in a \texttt{\$d}
statement with the empty symbol sequence\index{empty substitution}.

The \texttt{\$d} statement does not have a direct counterpart in
ordinary mathematics, partly because the variables\index{variable} of
Metamath are not really the same as the variables\index{variable!in
ordinary mathematics} of ordinary mathematics but rather are
metavariables\index{metavariable} ranging over them (as well as over
other kinds of symbols and groups of symbols).  Depending on the
situation, we may informally interpret the \texttt{\$d} statement in
different ways.  Suppose, for example, that \texttt{x} and \texttt{y}
are variables ranging over numbers (more precisely, that \texttt{x} and
\texttt{y} are metavariables ranging over variables that range over
numbers), and that \texttt{ph} ($\varphi$) and \texttt{ps} ($\psi$) are
variables (more precisely, metavariables) ranging over formulas.  We can
make the following interpretations that correspond to the informal
language of ordinary mathematics:
\begin{quote}
\begin{tabbing}
\texttt{\$d x y \$.} means ``assume $x$ and $y$ are
distinct variables.''\\
\texttt{\$d x ph \$.} means ``assume $x$ does not
occur in $\varphi$.''\\
\texttt{\$d ph ps \$.} \=means ``assume $\varphi$ and
$\psi$ have no variables\\ \>in common.''
\end{tabbing}
\end{quote}\index{\texttt{\$d} statement}

\subsubsection{Compound \texttt{\$d} Statements}

The {\bf compound} version of the \texttt{\$d} statement is a shorthand for
specifying several variables whose substitutions must be pairwise disjoint.
Its syntax is:
\begin{center}
  \texttt{\$d} {\em variable}\ \,$\cdots$\ {\em variable} \texttt{\$.}
\end{center}\index{\texttt{\$d} statement}
Here, {\em variable} represents the token of a previously declared
variable (specifically, an active variable) and all {\em variable}\,s are
different.  The compound \texttt{\$d}
statement is internally broken up by Metamath into one simple \texttt{\$d}
statement for each possible pair of variables in the original \texttt{\$d}
statement.  For example,
\begin{center}
  \texttt{\$d w x y z \$.}
\end{center}
is equivalent to
\begin{center}
  \texttt{\$d w x \$.}\\
  \texttt{\$d w y \$.}\\
  \texttt{\$d w z \$.}\\
  \texttt{\$d x y \$.}\\
  \texttt{\$d x z \$.}\\
  \texttt{\$d y z \$.}
\end{center}

Two or more simple \texttt{\$d} statements specifying the same variable pair are
internally combined into a single \texttt{\$d} statement.  Thus the set of three
statements
\begin{center}
  \texttt{\$d x y \$.}
  \texttt{\$d x y \$.}
  \texttt{\$d y x \$.}
\end{center}
is equivalent to
\begin{center}
  \texttt{\$d x y \$.}
\end{center}

Similarly, compound \texttt{\$d} statements, after being internally broken up,
internally have their common variable pairs combined.  For example the
set of statements
\begin{center}
  \texttt{\$d x y A \$.}
  \texttt{\$d x y B \$.}
\end{center}
is equivalent to
\begin{center}
  \texttt{\$d x y \$.}
  \texttt{\$d x A \$.}
  \texttt{\$d y A \$.}
  \texttt{\$d x y \$.}
  \texttt{\$d x B \$.}
  \texttt{\$d y B \$.}
\end{center}
which is equivalent to
\begin{center}
  \texttt{\$d x y \$.}
  \texttt{\$d x A \$.}
  \texttt{\$d y A \$.}
  \texttt{\$d x B \$.}
  \texttt{\$d y B \$.}
\end{center}

Metamath\index{Metamath} automatically verifies that all \texttt{\$d}
restrictions are met whenever it verifies proofs.  \texttt{\$d} statements are
never referenced directly in proofs (this is why they do not have
labels\index{label}), but Metamath is always aware of which ones must be
satisfied (i.e.\ are active) and will notify you with an error message if any
violation occurs.

To illustrate how Metamath detects a missing \texttt{\$d}
statement, we will look at the following example from the
\texttt{set.mm} database.

\begin{verbatim}
$d x z $.  $d y z $.
$( Theorem to add distinct quantifier to atomic formula. $)
ax17eq $p |- ( x = y -> A. z x = y ) $=...
\end{verbatim}

This statement has the obvious requirement that $z$ must be
distinct\index{distinct variables} from $x$ in theorem \texttt{ax17eq} that
states $x=y \rightarrow \forall z \, x=y$ (well, obvious if you're a logician,
for otherwise we could conclude  $x=y \rightarrow \forall x \, x=y$, which is
false when free variables $x$ and $y$ are equal.).

Let's look at what happens if we edit the database to comment out this
requirement.

\begin{verbatim}
$( $d x z $. $) $d y z $.
$( Theorem to add distinct quantifier to atomic formula. $)
ax17eq $p |- ( x = y -> A. z x = y ) $=...
\end{verbatim}

When it tries to verify the proof, Metamath will tell you that \texttt{x} and
\texttt{z} must be disjoint, because one of its steps references an axiom or
theorem that has this requirement.

\begin{verbatim}
MM> verify proof ax17eq
ax17eq ?Error at statement 1918, label "ax17eq", type "$p":
      vz wal wi vx vy vz ax-13 vx vy weq vz vx ax-c16 vx vy
                                               ^^^^^
There is a disjoint variable ($d) violation at proof step 29.
Assertion "ax-c16" requires that variables "x" and "y" be
disjoint.  But "x" was substituted with "z" and "y" was
substituted with "x".  The assertion being proved, "ax17eq",
does not require that variables "z" and "x" be disjoint.
\end{verbatim}

We can see the substitutions into \texttt{ax-c16} with the following command.

\begin{verbatim}
MM> show proof ax17eq / detailed_step 29
Proof step 29:  pm2.61dd.2=ax-c16 $a |- ( A. z z = x -> ( x =
  y -> A. z x = y ) )
This step assigns source "ax-c16" ($a) to target "pm2.61dd.2"
($e).  The source assertion requires the hypotheses "wph"
($f, step 26), "vx" ($f, step 27), and "vy" ($f, step 28).
The parent assertion of the target hypothesis is "pm2.61dd"
($p, step 36).
The source assertion before substitution was:
    ax-c16 $a |- ( A. x x = y -> ( ph -> A. x ph ) )
The following substitutions were made to the source
assertion:
    Variable  Substituted with
     x         z
     y         x
     ph        x = y
The target hypothesis before substitution was:
    pm2.61dd.2 $e |- ( ph -> ch )
The following substitutions were made to the target
hypothesis:
    Variable  Substituted with
     ph        A. z z = x
     ch        ( x = y -> A. z x = y )
\end{verbatim}

The disjoint variable restrictions of \texttt{ax-c16} can be seen from the
\texttt{show state\-ment} command.  The line that begins ``\texttt{Its mandatory
dis\-joint var\-i\-able pairs are:}\ldots'' lists any \texttt{\$d} variable
pairs in brackets.

\begin{verbatim}
MM> show statement ax-c16/full
Statement 3033 is located on line 9338 of the file "set.mm".
"Axiom of Distinct Variables. ..."
  ax-c16 $a |- ( A. x x = y -> ( ph -> A. x ph ) ) $.
Its mandatory hypotheses in RPN order are:
  wph $f wff ph $.
  vx $f set x $.
  vy $f set y $.
Its mandatory disjoint variable pairs are:  <x,y>
The statement and its hypotheses require the variables:  x y
      ph
The variables it contains are:  x y ph
\end{verbatim}

Since Metamath will always detect when \texttt{\$d}\index{\texttt{\$d} statement}
statements are needed for a proof, you don't have to worry too much about
forgetting to put one in; it can always be added if you see the error message
above.  If you put in unnecessary \texttt{\$d} statements, the worst that will
happen is that your theorem might not be as general as it could be, and this
may limit its use later on.

On the other hand, when you introduce axioms (\texttt{\$a}\index{\texttt{\$a}
statement} statements), you must be very careful to properly specify the
necessary associated \texttt{\$d} statements since Metamath has no way of knowing
whether your axioms are correct.  For example, Metamath would have no idea
that \texttt{ax-c16}, which we are telling it is an axiom of logic, would lead to
contradictions if we omitted its associated \texttt{\$d} statement.

{\footnotesize\begin{quotation}\label{nodd}
{\em Comment.} You may wonder if it is possible to develop standard
mathematics in the Metamath language without the \texttt{\$d}\index{\texttt{\$d}
statement} statement, since it seems like a nuisance that complicates proof
verification. The \texttt{\$d} statement is not needed in certain subsets of
mathematics such as propositional calculus.  However, dummy
variables\index{dummy variable!eliminating} and their associated \texttt{\$d}
statements are impossible to avoid in proofs in standard first-order logic as
well as in the variant used in \texttt{set.mm}.  In fact, there is no upper bound to
the number of dummy variables that might be needed in a proof of a theorem of
first-order logic containing 3 or more variables, as shown by H.\
Andr\'{e}ka\index{Andr{\'{e}}ka, H.} \cite{Nemeti}.  A first-order system that
avoids them entirely is given in \cite{Megill}\index{Megill, Norman}; the
trick there is simply to embed harmlessly the necessary dummy variables into a
theorem being proved so that they aren't ``dummy'' anymore, then interpret the
resulting longer theorem so as to ignore the embedded dummy variables.  If
this interests you, the system in \texttt{set.mm} obtained from \texttt{ax-1}
through \texttt{ax-c14} in \texttt{set.mm}, and deleting \texttt{ax-c16} and \texttt{ax-5},
requires no \texttt{\$d} statements but is logically complete in the sense
described in \cite{Megill}.  This means it can prove any theorem of
first-order logic as long as we add to the theorem an antecedent that embeds
dummy and any other variables that must be distinct.  In a similar fashion,
axioms for set theory can be devised that do not require distinct variables
(contact me if interested).  Together, these in principle allow all of
mathematics to be developed under Metamath without a \texttt{\$d} statement,
although the length of the resulting theorems will grow as more and
more dummy variables become required in their proofs.
\end{quotation}}

\subsection{The \texttt{\$f}
and \texttt{\$e} Statements}\label{dollaref}
\index{\texttt{\$e} statement}
\index{\texttt{\$f} statement}
\index{floating hypothesis}
\index{essential hypothesis}
\index{variable-type hypothesis}
\index{logical hypothesis}
\index{hypothesis}

Metamath has two kinds of hypo\-theses, the \texttt{\$f}\index{\texttt{\$f}
statement} or {\bf variable-type} hypothesis and the \texttt{\$e} or {\bf logical}
hypo\-the\-sis.\index{\texttt{\$d} statement}\footnote{Strictly speaking, the
\texttt{\$d} statement is also a hypothesis, but it is never directly referenced
in a proof, so we call it a restriction rather than a hypothesis to lessen
confusion.  The checking for violations of \texttt{\$d} restrictions is automatic
and built into Metamath's proof-checking algorithm.} The letters \texttt{f} and
\texttt{e} stand for ``floating''\index{floating hypothesis} (roughly meaning
used only if relevant) and ``essential''\index{essential hypothesis} (meaning
always used) respectively, for reasons that will become apparent
when we discuss frames in
Section~\ref{frames} and scoping in Section~\ref{scoping}. The syntax of these
are as follows:
\begin{center}
  {\em label} \texttt{\$f} {\em constant} {\em variable} \texttt{\$.}\\
  {\em label} \texttt{\$e} {\em constant}
      {\em math-symbol}\ \,$\cdots$\ {\em math-symbol} \texttt{\$.}\\
\end{center}
\index{\texttt{\$e} statement}
\index{\texttt{\$f} statement}
A hypothesis must have a {\em label}\index{label}.  The expression in a
\texttt{\$e} hypothesis consists of a constant math symbol followed by a sequence
of zero or more math symbols. Each math symbol (including {\em constant}
and {\em variable}) must be a previously declared constant or variable.  (In
addition, each math symbol must be active, which will be covered when we
discuss scoping statements in Section~\ref{scoping}.)  You use a \texttt{\$f}
hypothesis to specify the
nature or {\bf type}\index{variable type}\index{type} of a variable (such as ``let $x$ be an
integer'') and use a \texttt{\$e} hypothesis to express a logical truth (such as
``assume $x$ is prime'') that must be established in order for an assertion
requiring it to also be true.

A variable must have its type specified in a \texttt{\$f} statement before
it may be used in a \texttt{\$e}, \texttt{\$a}, or \texttt{\$p}
statement.  There may be only one (active) \texttt{\$f} statement for a
given variable.  (``Active'' is defined in Section~\ref{scoping}.)

In ordinary mathematics, theorems\index{theorem} are often expressed in the
form ``Assume $P$; then $Q$,'' where $Q$ is a statement that you can derive
if you start with statement $P$.\index{free variable}\footnote{A stronger
version of a theorem like this would be the {\em single} formula $P\rightarrow
Q$ ($P$ implies $Q$) from which the weaker version above follows by the rule
of modus ponens in logic.  We are not discussing this stronger form here.  In
the weaker form, we are saying only that if we can {\em prove} $P$, then we can
{\em prove} $Q$.  In a logician's language, if $x$ is the only free variable
in $P$ and $Q$, the stronger form is equivalent to $\forall x ( P \rightarrow
Q)$ (for all $x$, $P$ implies $Q$), whereas the weaker form is equivalent to
$\forall x P \rightarrow \forall x Q$. The stronger form implies the weaker,
but not vice-versa.  To be precise, the weaker form of the theorem is more
properly called an ``inference'' rather than a theorem.}\index{inference}
In the
Metamath\index{Metamath} language, you would express mathematical statement
$P$ as a hypothesis (a \texttt{\$e} Metamath language statement in this case) and
statement $Q$ as a provable assertion (a \texttt{\$p}\index{\texttt{\$p} statement}
statement).

Some examples of hypotheses you might encounter in logic and set theory are
\begin{center}
  \texttt{stmt1 \$f wff P \$.}\\
  \texttt{stmt2 \$f set x \$.}\\
  \texttt{stmt3 \$e |- ( P -> Q ) \$.}
\end{center}
\index{\texttt{\$e} statement}
\index{\texttt{\$f} statement}
Informally, these would be read, ``Let $P$ be a well-formed-formula,'' ``Let
$x$ be an (individual) variable,'' and ``Assume we have proved $P \rightarrow
Q$.''  The turnstile symbol \,$\vdash$\index{turnstile ({$\,\vdash$})} is
commonly used in logic texts to mean ``a proof exists for.''

To summarize:
\begin{itemize}
\item A \texttt{\$f} hypothesis tells Metamath the type or kind of its variable.
It is analogous to a variable declaration in a computer language that
tells the compiler that a variable is an integer or a floating-point
number.
\item The \texttt{\$e} hypothesis corresponds to what you would usually call a
``hypothesis'' in ordinary mathematics.
\end{itemize}

Before an assertion\index{assertion} (\texttt{\$a} or \texttt{\$p} statement) can be
referenced in a proof, all of its associated \texttt{\$f} and \texttt{\$e} hypotheses
(i.e.\ those \texttt{\$e} hypotheses that are active) must be satisfied (i.e.
established by the proof).  The meaning of ``associated'' (which we will call
{\bf mandatory} in Section~\ref{frames}) will become clear when we discuss
scoping later.

\subsection{Assertions (\texttt{\$a} and \texttt{\$p} Statements)}
\index{\texttt{\$a} statement}
\index{\texttt{\$p} statement}\index{assertion}\index{axiomatic assertion}
\index{provable assertion}

There are two types of assertions, \texttt{\$a}\index{\texttt{\$a} statement}
statements ({\bf axiomatic assertions}) and \texttt{\$p} statements ({\bf
provable assertions}).  Their syntax is as follows:
\begin{center}
  {\em label} \texttt{\$a} {\em constant} {\em math-symbol} \ldots
         {\em math-symbol} \texttt{\$.}\\
  {\em label} \texttt{\$p} {\em constant} {\em math-symbol} \ldots
        {\em math-symbol} \texttt{\$=} {\em proof} \texttt{\$.}
\end{center}
\index{\texttt{\$a} statement}
\index{\texttt{\$p} statement}
\index{\texttt{\$=} keyword}
An assertion always requires a {\em label}\index{label}. The expression in an
assertion consists of a constant followed by a sequence of zero
or more math symbols.  Each math symbol, including {\em constant}, must be a
previously declared constant or variable.  (In addition, each math symbol
must be active, which will be covered when we discuss scoping statements in
Section~\ref{scoping}.)

A \texttt{\$a} statement is usually a definition of syntax (for example, if $P$
and $Q$ are wffs then so is $(P\to Q)$), an axiom\index{axiom} of ordinary
mathematics (for example, $x=x$), or a definition\index{definition} of
ordinary mathematics (for example, $x\ne y$ means $\lnot x=y$). A \texttt{\$p}
statement is a claim that a certain combination of math symbols follows from
previous assertions and is accompanied by a proof that demonstrates it.

Assertions can also be referenced in (later) proofs in order to derive new
assertions from them. The label of an assertion is used to refer to it in a
proof. Section~\ref{proof} will describe the proof in detail.

Assertions also provide the primary means for communicating the mathematical
results in the database to people.  Proofs (when conveniently displayed)
communicate to people how the results were arrived at.

\subsubsection{The \texttt{\$a} Statement}
\index{\texttt{\$a} statement}

Axiomatic assertions (\texttt{\$a} statements) represent the starting points from
which other assertions (\texttt{\$p}\index{\texttt{\$p} statement} statements) are
derived.  Their most obvious use is for specifying ordinary mathematical
axioms\index{axiom}, but they are also used for two other purposes.

First, Metamath\index{Metamath} needs to know the syntax of symbol
sequences that constitute valid mathematical statements.  A Metamath
proof must be broken down into much more detail than ordinary
mathematical proofs that you may be used to thinking of (even the
``complete'' proofs of formal logic\index{formal logic}).  This is one
of the things that makes Metamath a general-purpose language,
independent of any system of logic or even syntax.  If you want to use a
substitution instance of an assertion as a step in a proof, you must
first prove that the substitution is syntactically correct (or if you
prefer, you must ``construct'' it), showing for example that the
expression you are substituting for a wff metavariable is a valid wff.
The \texttt{\$a}\index{\texttt{\$a} statement} statement is used to
specify those combinations of symbols that are considered syntactically
valid, such as the legal forms of wffs.

Second, \texttt{\$a} statements are used to specify what are ordinarily thought of
as definitions, i.e.\ new combinations of symbols that abbreviate other
combinations of symbols.  Metamath makes no distinction\index{axiom vs.\
definition} between axioms\index{axiom} and definitions\index{definition}.
Indeed, it has been argued that such distinction should not be made even in
ordinary mathematics; see Section~\ref{definitions}, which discusses the
philosophy of definitions.  Section~\ref{hierarchy} discusses some
technical requirements for definitions.  In \texttt{set.mm} we adopt the
convention of prefixing axiom labels with \texttt{ax-} and definition labels with
\texttt{df-}\index{label}.

The results that can be derived with the Metamath language are only as good as
the \texttt{\$a}\index{\texttt{\$a} statement} statements used as their starting
point.  We cannot stress this too strongly.  For example, Metamath will
not prevent you from specifying $x\neq x$ as an axiom of logic.  It is
essential that you scrutinize all \texttt{\$a} statements with great care.
Because they are a source of potential pitfalls, it is best not to add new
ones (usually new definitions) casually; rather you should carefully evaluate
each one's necessity and advantages.

Once you have in place all of the basic axioms\index{axiom} and
rules\index{rule} of a mathematical theory, the only \texttt{\$a} statements that
you will be adding will be what are ordinarily called definitions.  In
principle, definitions should be in some sense eliminable from the language of
a theory according to some convention (usually involving logical equivalence
or equality).  The most common convention is that any formula that was
syntactically valid but not provable before the definition was introduced will
not become provable after the definition is introduced.  In an ideal world,
definitions should not be present at all if one is to have absolute confidence
in a mathematical result.  However, they are necessary to make
mathematics practical, for otherwise the resulting formulas would be
extremely long and incomprehensible.  Since the nature of definitions (in the
most general sense) does not permit them to automatically be verified as
``proper,''\index{proper definition}\index{definition!proper} the judgment of
the mathematician is required to ensure it.  (In \texttt{set.mm} effort was made
to make almost all definitions directly eliminable and thus minimize the need
for such judgment.)

If you are not a mathematician, it may be best not to add or change any
\texttt{\$a}\index{\texttt{\$a} statement} statements but instead use
the mathematical language already provided in standard databases.  This
way Metamath will not allow you to make a mistake (i.e.\ prove a false
result).


\subsection{Frames}\label{frames}

We now introduce the concept of a collection of related Metamath statements
called a frame.  Every assertion (\texttt{\$a} or \texttt{\$p} statement) in the database has
an associated frame.

A {\bf frame}\index{frame} is a sequence of \texttt{\$d}, \texttt{\$f},
and \texttt{\$e} statements (zero or more of each) followed by one
\texttt{\$a} or \texttt{\$p} statement, subject to certain conditions we
will describe.  For simplicity we will assume that all math symbol
tokens used are declared at the beginning of the database with
\texttt{\$c} and \texttt{\$v} statements (which are not properly part of
a frame).  Also for simplicity we will assume there are only simple
\texttt{\$d} statements (those with only two variables) and imagine any
compound \texttt{\$d} statements (those with more than two variables) as
broken up into simple ones.

A frame groups together those hypotheses (and \texttt{\$d} statements) relevant
to an assertion (\texttt{\$a} or \texttt{\$p} statement).  The statements in a frame
may or may not be physically adjacent in a database; we will cover
this in our discussion of scoping statements
in Section~\ref{scoping}.

A frame has the following properties:
\begin{enumerate}
 \item The set of variables contained in its \texttt{\$f} statements must
be identical to the set of variables contained in its \texttt{\$e},
\texttt{\$a}, and/or \texttt{\$p} statements.  In other words, each
variable in a \texttt{\$e}, \texttt{\$a}, or \texttt{\$p} statement must
have an associated ``variable type'' defined for it in a \texttt{\$f}
statement.
  \item No two \texttt{\$f} statements may contain the same variable.
  \item Any \texttt{\$f} statement
must occur before a \texttt{\$e} statement in which its variable occurs.
\end{enumerate}

The first property determines the set of variables occurring in a frame.
These are the {\bf mandatory
variables}\index{mandatory variable} of the frame.  The second property
tells us there must be only one type specified for a variable.  The
third property determines which \texttt{\$d} statements belong to the
frame.  The last property is not a theoretical requirement but it
makes parsing of the database easier.

For our examples, we assume our database has the following declarations:

\begin{verbatim}
        $v P Q R $.
        $c -> ( ) |- wff $.
\end{verbatim}

The following sequence of statements, describing the modus ponens inference
rule, is an example of a frame:

\begin{verbatim}
        wp  $f wff P $.
        wq  $f wff Q $.
        maj $e |- ( P -> Q ) $.
        min $e |- P $.
        mp  $a |- Q $.
\end{verbatim}

The following sequence of statements is not a frame because \texttt{R} does not
occur in the \texttt{\$e}'s or the \texttt{\$a}:

\begin{verbatim}
        wp  $f wff P $.
        wq  $f wff Q $.
        wr  $f wff R $.
        maj $e |- ( P -> Q ) $.
        min $e |- P $.
        mp  $a |- Q $.
\end{verbatim}

The following sequence of statements is not a frame because \texttt{Q} does not
occur in a \texttt{\$f}:

\begin{verbatim}
        wp  $f wff P $.
        maj $e |- ( P -> Q ) $.
        min $e |- P $.
        mp  $a |- Q $.
\end{verbatim}

The following sequence of statements is not a frame because the \texttt{\$a} statement is
not the last one:

\begin{verbatim}
        wp  $f wff P $.
        wq  $f wff Q $.
        maj $e |- ( P -> Q ) $.
        mp  $a |- Q $.
        min $e |- P $.
\end{verbatim}

Associated with a frame is a sequence of {\bf mandatory
hypotheses}\index{mandatory hypothesis}.  This is simply the set of all
\texttt{\$f} and \texttt{\$e} statements in the frame, in the order they
appear.  A frame can be referenced in a later proof using the label of
the \texttt{\$a} or \texttt{\$p} assertion statement, and the proof
makes an assignment to each mandatory hypothesis in the order in which
it appears.  This means the order of the hypotheses, once chosen, must
not be changed so as not to affect later proofs referencing the frame's
assertion statement.  (The Metamath proof verifier will, of course, flag
an error if a proof becomes incorrect by doing this.)  Since proofs make
use of ``Reverse Polish notation,'' described in Section~\ref{proof}, we
call this order the {\bf RPN order}\index{RPN order} of the hypotheses.

Note that \texttt{\$d} statements are not part of the set of mandatory
hypotheses, and their order doesn't matter (as long as they satisfy the
fourth property for a frame described above).  The \texttt{\$d}
statements specify restrictions on variables that must be satisfied (and
are checked by the proof verifier) when expressions are substituted for
them in a proof, and the \texttt{\$d} statements themselves are never
referenced directly in a proof.

A frame with a \texttt{\$p} (provable) statement requires a proof as part of the
\texttt{\$p} statement.  Sometimes in a proof we want to make use of temporary or
dummy variables\index{dummy variable} that do not occur in the \texttt{\$p}
statement or its mandatory hypotheses.  To accommodate this we define an {\bf
extended frame}\index{extended frame} as a frame together with zero or more
\texttt{\$d} and \texttt{\$f} statements that reference variables not among the
mandatory variables of the frame.  Any new variables referenced are called the
{\bf optional variables}\index{optional variable} of the extended frame. If a
\texttt{\$f} statement references an optional variable it is called an {\bf
optional hypothesis}\index{optional hypothesis}, and if one or both of the
variables in a \texttt{\$d} statement are optional variables it is called an {\bf
optional disjoint-variable restriction}\index{optional disjoint-variable
restriction}.  Properties 2, 3, and 4 for a frame also apply to an extended
frame.

The concept of optional variables is not meaningful for frames with \texttt{\$a}
statements, since those statements have no proofs that might make use of them.
There is no restriction on including optional hypotheses in the extended frame
for a \texttt{\$a} statement, but they serve no purpose.

The following set of statements is an example of an extended frame, which
contains an optional variable \texttt{R} and an optional hypothesis \texttt{wr}.  In
this example, we suppose the rule of modus ponens is not an axiom but is
derived as a theorem from earlier statements (we omit its presumed proof).
Variable \texttt{R} may be used in its proof if desired (although this would
probably have no advantage in propositional calculus).  Note that the sequence
of mandatory hypotheses in RPN order is still \texttt{wp}, \texttt{wq}, \texttt{maj},
\texttt{min} (i.e.\ \texttt{wr} is omitted), and this sequence is still assumed
whenever assertion \texttt{mp} is referenced in a subsequent proof.

\begin{verbatim}
        wp  $f wff P $.
        wq  $f wff Q $.
        wr  $f wff R $.
        maj $e |- ( P -> Q ) $.
        min $e |- P $.
        mp  $p |- Q $= ... $.
\end{verbatim}

Every frame is an extended frame, but not every extended frame is a frame, as
this example shows.  The underlying frame for an extended frame is
obtained by simply removing all statements containing optional variables.
Any proof referencing an assertion will ignore any extensions to its
frame, which means we may add or delete optional hypotheses at will without
affecting subsequent proofs.

The conceptually simplest way of organizing a Metamath database is as a
sequence of extended frames.  The scoping statements
\texttt{\$\char`\{}\index{\texttt{\$\char`\{} and \texttt{\$\char`\}}
keywords} and \texttt{\$\char`\}} can be used to delimit the start and
end of a frame, leading to the following possible structure for a
database.  \label{framelist}

\vskip 2ex
\setbox\startprefix=\hbox{\tt \ \ \ \ \ \ \ \ }
\setbox\contprefix=\hbox{}
\startm
\m{\mbox{(\texttt{\$v} {\em and} \texttt{\$c}\,{\em statements})}}
\endm
\startm
\m{\mbox{\texttt{\$\char`\{}}}
\endm
\startm
\m{\mbox{\texttt{\ \ } {\em extended frame}}}
\endm
\startm
\m{\mbox{\texttt{\$\char`\}}}}
\endm
\startm
\m{\mbox{\texttt{\$\char`\{}}}
\endm
\startm
\m{\mbox{\texttt{\ \ } {\em extended frame}}}
\endm
\startm
\m{\mbox{\texttt{\$\char`\}}}}
\endm
\startm
\m{\mbox{\texttt{\ \ \ \ \ \ \ \ \ }}\vdots}
\endm
\vskip 2ex

In practice, this structure is inconvenient because we have to repeat
any \texttt{\$f}, \texttt{\$e}, and \texttt{\$d} statements over and
over again rather than stating them once for use by several assertions.
The scoping statements, which we will discuss next, allow this to be
done.  In principle, any Metamath database can be converted to the above
format, and the above format is the most convenient to use when studying
a Metamath database as a formal system%
%% Uncomment this when uncommenting section {formalspec} below
   (Appendix \ref{formalspec})%
.
In fact, Metamath internally converts the database to the above format.
The command \texttt{show statement} in the Metamath program will show
you the contents of the frame for any \texttt{\$a} or \texttt{\$p}
statement, as well as its extension in the case of a \texttt{\$p}
statement.

%c%(provided that all ``local'' variables and constants with limited scope have
%c%unique names),

During our discussion of scoping statements, it may be helpful to
think in terms of the equivalent sequence of frames that will result when
the database is parsed.  Scoping (other than the limited
use above to delimit frames) is not a theoretical requirement for
Metamath but makes it more convenient.


\subsection{Scoping Statements (\texttt{\$\{} and \texttt{\$\}})}\label{scoping}
\index{\texttt{\$\char`\{} and \texttt{\$\char`\}} keywords}\index{scoping statement}

%c%Some Metamath statements may be needed only temporarily to
%c%serve a specific purpose, and after we're done with them we would like to
%c%disregard or ignore them.  For example, when we're finished using a variable,
%c%we might want to
%c%we might want to free up the token\index{token} used to name it so that the
%c%token can be used for other purposes later on, such as a different kind of
%c%variable or even a constant.  In the terminology of computer programming, we
%c%might want to let some symbol declarations be ``local'' rather than ``global.''
%c%\index{local symbol}\index{global symbol}

The {\bf scoping} statements, \texttt{\$\char`\{} ({\bf start of block}) and \texttt{\$\char`\}}
({\bf end of block})\index{block}, provide a means for controlling the portion
of a database over which certain statement types are recognized.  The
syntax of a scoping statement is very simple; it just consists of the
statement's keyword:
\begin{center}
\texttt{\$\char`\{}\\
\texttt{\$\char`\}}
\end{center}
\index{\texttt{\$\char`\{} and \texttt{\$\char`\}} keywords}

For example, consider the following database where we have stripped out
all tokens except the scoping statement keywords.  For the purpose of the
discussion, we have added subscripts to the scoping statements; these subscripts
do not appear in the actual database.
\[
 \mbox{\tt \ \$\char`\{}_1
 \mbox{\tt \ \$\char`\{}_2
 \mbox{\tt \ \$\char`\}}_2
 \mbox{\tt \ \$\char`\{}_3
 \mbox{\tt \ \$\char`\{}_4
 \mbox{\tt \ \$\char`\}}_4
 \mbox{\tt \ \$\char`\}}_3
 \mbox{\tt \ \$\char`\}}_1
\]
Each \texttt{\$\char`\{} statement in this example is said to be {\bf
matched} with the \texttt{\$\char`\}} statement that has the same
subscript.  Each pair of matched scoping statements defines a region of
the database called a {\bf block}.\index{block} Blocks can be {\bf
nested}\index{nested block} inside of other blocks; in the example, the
block defined by $\mbox{\tt \$\char`\{}_4$ and $\mbox{\tt \$\char`\}}_4$
is nested inside the block defined by $\mbox{\tt \$\char`\{}_3$ and
$\mbox{\tt \$\char`\}}_3$ as well as inside the block defined by
$\mbox{\tt \$\char`\{}_1$ and $\mbox{\tt \$\char`\}}_1$.  In general, a
block may be empty, it may contain only non-scoping
statements,\footnote{Those statements other than \texttt{\$\char`\{} and
\texttt{\$\char`\}}.}\index{non-scoping statement} or it may contain any
mixture of other blocks and non-scoping statements.  (This is called a
``recursive'' definition\index{recursive definition} of a block.)

Associated with each block is a number called its {\bf nesting
level}\index{nesting level} that indicates how deeply the block is nested.
The nesting levels of the blocks in our example are as follows:
\[
  \underbrace{
    \mbox{\tt \ }
    \underbrace{
     \mbox{\tt \$\char`\{\ }
     \underbrace{
       \mbox{\tt \$\char`\{\ }
       \mbox{\tt \$\char`\}}
     }_{2}
     \mbox{\tt \ }
     \underbrace{
       \mbox{\tt \$\char`\{\ }
       \underbrace{
         \mbox{\tt \$\char`\{\ }
         \mbox{\tt \$\char`\}}
       }_{3}
       \mbox{\tt \ \$\char`\}}
     }_{2}
     \mbox{\tt \ \$\char`\}}
   }_{1}
   \mbox{\tt \ }
 }_{0}
\]
\index{\texttt{\$\char`\{} and \texttt{\$\char`\}} keywords}
The entire database is considered to be one big block (the {\bf outermost}
block) with a nesting level of 0.  The outermost block is {\em not} bracketed
by scoping statements.\footnote{The language was designed this way so that
several source files can be joined together more easily.}\index{outermost
block}

All non-scoping Metamath statements become recognized or {\bf
active}\index{active statement} at the place where they appear.\footnote{To
keep things slightly simpler, we do not bother to define the concept of
``active'' for the scoping statements.}  Certain of these statement types
become inactive at the end of the block in which they appear; these statement
types are:
\begin{center}
  \texttt{\$c}, \texttt{\$v}, \texttt{\$d}, \texttt{\$e}, and \texttt{\$f}.
%  \texttt{\$v}, \texttt{\$f}, \texttt{\$e}, and \texttt{\$d}.
\end{center}
\index{\texttt{\$c} statement}
\index{\texttt{\$d} statement}
\index{\texttt{\$e} statement}
\index{\texttt{\$f} statement}
\index{\texttt{\$v} statement}
The other statement types remain active forever (i.e.\ through the end of the
database); they are:
\begin{center}
  \texttt{\$a} and \texttt{\$p}.
%  \texttt{\$c}, \texttt{\$a}, and \texttt{\$p}.
\end{center}
\index{\texttt{\$a} statement}
\index{\texttt{\$p} statement}
Any statement (of these 7 types) located in the outermost
block\index{outermost block} will remain active through the end of the
database and thus are effectively ``global'' statements.\index{global
statement}

All \texttt{\$c} statements must be placed in the outermost block.  Since they are
therefore always global, they could be considered as belonging to both of the
above categories.

The {\bf scope}\index{scope} of a statement is the set of statements that
recognize it as active.

%c%The concept of ``active'' is also defined for math symbols\index{math
%c%symbol}.  Math symbols (constants\index{constant} and
%c%variables\index{variable}) become {\bf active}\index{active
%c%math symbol} in the \texttt{\$c}\index{\texttt{\$c}
%c%statement} and \texttt{\$v}\index{\texttt{\$v} statement} statements that
%c%declare them.  They become inactive when their declaration statements become
%c%inactive.

The concept of ``active'' is also defined for math symbols\index{math
symbol}.  Math symbols (constants\index{constant} and
variables\index{variable}) become {\bf active}\index{active math symbol}
in the \texttt{\$c}\index{\texttt{\$c} statement} and
\texttt{\$v}\index{\texttt{\$v} statement} statements that declare them.
A variable becomes inactive when its declaration statement becomes
inactive.  Because all \texttt{\$c} statements must be in the outermost
block, a constant will never become inactive after it is declared.

\subsubsection{Redeclaration of Math Symbols}
\index{redeclaration of symbols}\label{redeclaration}

%c%A math symbol may not be declared a second time while it is active, but it may
%c%be declared again after it becomes inactive.

A variable may not be declared a second time while it is active, but it may be
declared again after it becomes inactive.  This provides a convenient way to
introduce ``local'' variables,\index{local variable} i.e.\ temporary variables
for use in the frame of an assertion or in a proof without keeping them around
forever.  A previously declared variable may not be redeclared as a constant.

A constant may not be redeclared.  And, as mentioned above, constants must be
declared in the outermost block.

The reason variables may have limited scope but not constants is that an
assertion (\texttt{\$a} or \texttt{\$p} statement) remains available for use in
proofs through the end of the database.  Variables in an assertion's frame may
be substituted with whatever is needed in a proof step that references the
assertion, whereas constants remain fixed and may not be substituted with
anything.  The particular token used for a variable in an assertion's frame is
irrelevant when the assertion is referenced in a proof, and it doesn't matter
if that token is not available outside of the referenced assertion's frame.
Constants, however, must be globally fixed.

In the present version of the Metamath language, there is no theoretical
benefit for the feature allowing variables to be active for limited scopes
rather than global. It is just a convenience that allows them, for example, to
be locally grouped together with their corresponding \texttt{\$f} variable-type
declarations.\footnote{A future version\label{redeclarationf} of the Metamath
language may extend the language to include true local variables as well as
local constants.  I have not decided upon the best way to do this or even if
it should be done at all; the main issue is whether the benefits offset
greater confusion in learning the language.  In any case, the rules for
the present language were chosen to let existing databases be compatible with
any future language extensions.}

%c%If you declare a math symbol more than once, internally Metamath considers it a
%c%new distinct symbol, even though it has the same name.  If you are unaware of
%c%this, you may find that what you think are correct proofs are incorrectly
%c%rejected as invalid, because Metamath may tell you that a constant you
%c%previously declared does not match a newly declared math symbol with the same
%c%name.  For details on this subtle point, see the Comment on
%c%p.~\pageref{spec4comment}.  This is done purposely to allow temporary
%c%constants to be introduced while developing a subtheory, then allow their math
%c%symbol tokens to be reused later on; in general they will not refer to the
%c%same thing.  In practice, you would not ordinarily reuse the names of
%c%constants because it would tend to be confusing to the reader.  The reuse of
%c%names of variables, on the other hand, is something that is often useful to do
%c%(for example it is done frequently in \texttt{set.mm}).  Since variables in an
%c%assertion referenced in a proof can be substituted as needed to achieve a
%c%symbol match, this is not an issue.

% (This section covers a somewhat advanced topic you may want to skip
% at first reading.)
%
% Under certain circumstances, math symbol\index{math symbol}
% tokens\index{token} may be redeclared (i.e.\ the token
% may appear in more than
% one \texttt{\$c}\index{\texttt{\$c} statement} or \texttt{\$v}\index{\texttt{\$v}
% statement} statement).  You might want to do this say, to make temporary use
% of a variable name without having to worry about its affect elsewhere,
% somewhat analogous to declaring a local variable in a standard computer
% language.  Understanding what goes on when math symbol tokens are redeclared
% is a little tricky to understand at first, since it requires that we
% distinguish the token itself from the math symbol that it names.  It will help
% if we first take a peek at the internal workings of the
% Metamath\index{Metamath} program.
%
% Metamath reserves a memory location for each occurrence of a
% token\index{token} in a declaration statement (\texttt{\$c}\index{\texttt{\$c}
% statement} or \texttt{\$v}\index{\texttt{\$v} statement}).  If a given token appears
% in more than one declaration statement, it will refer to more than one memory
% locations.  A math symbol\index{math symbol} may be thought of as being one of
% these memory locations rather than as the token itself.  Only one of the
% memory locations associated with a given token may be active at any one time.
% The math symbol (memory location) that gets looked up when the token appears
% in a non-declaration statement is the one that happens to be active at that
% time.
%
% We now look at the rules for the redeclaration\index{redeclaration of symbols}
% of math symbol tokens.
% \begin{itemize}
% \item A math symbol token may not be declared twice in the
% same block.\footnote{While there is no theoretical reason for disallowing
% this, it was decided in the design of Metamath that allowing it would offer no
% advantage and might cause confusion.}
% \item An inactive math symbol may always be
% redeclared.
% \item  An active math symbol may be redeclared in a different (i.e.\
% inner) block\index{block} from the one it became active in.
% \end{itemize}
%
% When a math symbol token is redeclared, it conceptually refers to a different
% math symbol, just as it would be if it were called a different name.  In
% addition, the original math symbol that it referred to, if it was active,
% temporarily becomes inactive.  At the end of the block in which the
% redeclaration occurred, the new math symbol\index{math symbol} becomes
% inactive and the original symbol becomes active again.  This concept is
% illustrated in the following example, where the symbol \texttt{e} is
% ordinarily a constant (say Euler's constant, 2.71828...) but
% temporarily we want to use it as a ``local'' variable, say as a coefficient
% in the equation $a x^4 + b x^3 + c x^2 + d x + e$:
% \[
%   \mbox{\tt \$\char`\{\ \$c e \$.}
%   \underbrace{
%     \ \ldots\ %
%     \mbox{\tt \$\char`\{}\ \ldots\ %
%   }_{\mbox{\rm region A}}
%   \mbox{\tt \$v e \$.}
%   \underbrace{
%     \mbox{\ \ \ \ldots\ \ \ }
%   }_{\mbox{\rm region B}}
%   \mbox{\tt \$\char`\}}
%   \underbrace{
%     \mbox{\ \ \ \ldots\ \ \ }
%   }_{\mbox{\rm region C}}
%   \mbox{\tt \$\char`\}}
% \]
% \index{\texttt{\$\char`\{} and \texttt{\$\char`\}} keywords}
% In region A, the token \texttt{e} refers to a constant.  It is redeclared as a
% variable in region B, and any reference to it in this region will refer to this
% variable.  In region C, the redeclaration becomes inactive, and the original
% declaration becomes active again.  In region C, the token \texttt{x} refers to the
% original constant.
%
% As a practical matter, overuse of math symbol\index{math symbol}
% redeclarations\index{redeclaration of symbols} can be confusing (even though
% it is well-defined) and is best avoided when possible.  Here are some good
% general guidelines you can follow.  Usually, you should declare all
% constants\index{constant} in the outermost block\index{outermost block},
% especially if they are general-purpose (such as the token \verb$A.$, meaning
% $\forall$ or ``for all'').  This will make them ``globally'' active (although
% as in the example above local redeclarations will temporarily make them
% inactive.)  Most or all variables\index{variable}, on the other hand, could be
% declared in inner blocks, so that the token for them can be used later for a
% different type of variable or a constant.  (The names of the variables you
% choose are not used when you refer to an assertion\index{assertion} in a
% proof, whereas constants must match exactly.  A locally declared constant will
% not match a globally declared constant in a proof, even if they use the same
% token, because Metamath internally considers them to be different math
% symbols.)  To avoid confusion, you should generally avoid redeclaring active
% variables.  If you must redeclare them, do so at the beginning of a block.
% The temporary declaration of constants in inner blocks might be occasionally
% appropriate when you make use of a temporary definition to prove lemmas
% leading to a main result that does not make direct use of the definition.
% This way, you will not clutter up your database with a large number of
% seldom-used global constant symbols.  You might want to note that while
% inactive constants may not appear directly in an assertion (a \texttt{\$a}\index{\texttt{\$a}
% statement} or \texttt{\$p}\index{\texttt{\$p} statement}
% statement), they may be indirectly used in the proof of a \texttt{\$p} statement
% so long as they do not appear in the final math symbol sequence constructed by
% the proof.  In the end, you will have to use your best judgment, taking into
% account standard mathematical usage of the symbols as well as consideration
% for the reader of your work.
%
% \subsubsection{Reuse of Labels}\index{reuse of labels}\index{label}
%
% The \texttt{\$e}\index{\texttt{\$e} statement}, \texttt{\$f}\index{\texttt{\$f}
% statement}, \texttt{\$a}\index{\texttt{\$a} statement}, and
% \texttt{\$p}\index{\texttt{\$p}
% statement} statement types require labels, which allow them to be
% referenced later inside of proofs.  A label is considered {\bf
% active}\index{active label} when the statement it is associated with is
% active.  The token\index{token} for a label may be reused
% (redeclared)\index{redeclaration of labels} provided that it is not being used
% for a currently active label.  (Unlike the tokens for math symbols, active
% label tokens may not be redeclared in an inner scope.)  Note that the labels
% of \texttt{\$a} and \texttt{\$p} statements can never be reused after these
% statements appear, because these statements remain active through the end of
% the database.
%
% You might find the reuse of labels a convenient way to have standard names for
% temporary hypotheses, such as \texttt{h1}, \texttt{h2}, etc.  This way you don't have
% to invent unique names for each of them, and in some cases it may be less
% confusing to the reader (although in other cases it might be more confusing, if
% the hypothesis is located far away from the assertion that uses
% it).\footnote{The current implementation requires that all labels, even
% inactive ones, be unique.}

\subsubsection{Frames Revisited}\index{frames and scoping statements}

Now that we have covered scoping, we will look at how an arbitrary
Metamath database can be converted to the simple sequence of extended
frames described on p.~\pageref{framelist}.  This is also how Metamath
stores the database internally when it reads in the database
source.\label{frameconvert} The method is simple.  First, we collect all
constant and variable (\texttt{\$c} and \texttt{\$v}) declarations in
the database, ignoring duplicate declarations of the same variable in
different scopes.  We then put our collected \texttt{\$c} and
\texttt{\$v} declarations at the beginning of the database, so that
their scope is the entire database.  Next, for each assertion in the
database, we determine its frame and extended frame.  The extended frame
is simply the \texttt{\$f}, \texttt{\$e}, and \texttt{\$d} statements
that are active.  The frame is the extended frame with all optional
hypotheses removed.

An equivalent way of saying this is that the extended frame of an assertion
is the collection of all \texttt{\$f}, \texttt{\$e}, and \texttt{\$d} statements
whose scope includes the assertion.
The \texttt{\$f} and \texttt{\$e} statements
occur in the order they appear
(order is irrelevant for \texttt{\$d} statements).

%c%, renaming any
%c%redeclared variables as needed so that all of them have unique names.  (The
%c%exact renaming convention is unimportant.  You might imagine renaming
%c%different declarations of math symbol \texttt{a} as \texttt{a\$1}, \texttt{a\$2}, etc.\
%c%which would prevent any conflicts since \texttt{\$} is not a legal character in a
%c%math symbol token.)

\section{The Anatomy of a Proof} \label{proof}
\index{proof!Metamath, description of}

Each provable assertion (\texttt{\$p}\index{\texttt{\$p} statement} statement) in a
database must include a {\bf proof}\index{proof}.  The proof is located
between the \texttt{\$=}\index{\texttt{\$=} keyword} and \texttt{\$.}\ keywords in the
\texttt{\$p} statement.

In the basic Metamath language\index{basic language}, a proof is a
sequence of statement labels.  This label sequence\index{label sequence}
serves as a set of instructions that the Metamath program uses to
construct a series of math symbol sequences.  The construction must
ultimately result in the math symbol sequence contained between the
\texttt{\$p}\index{\texttt{\$p} statement} and
\texttt{\$=}\index{\texttt{\$=} keyword} keywords of the \texttt{\$p}
statement.  Otherwise, the Metamath program will consider the proof
incorrect, and it will notify you with an appropriate error message when
you ask it to verify the proof.\footnote{To make the loading faster, the
Metamath program does not automatically verify proofs when you
\texttt{read} in a database unless you use the \texttt{/verify}
qualifier.  After a database has been read in, you may use the
\texttt{verify proof *} command to verify proofs.}\index{\texttt{verify
proof} command} Each label in a proof is said to {\bf
reference}\index{label reference} its corresponding statement.

Associated with any assertion\index{assertion} (\texttt{\$p} or
\texttt{\$a}\index{\texttt{\$a} statement} statement) is a set of
hypotheses (\texttt{\$f}\index{\texttt{\$f} statement} or
\texttt{\$e}\index{\texttt{\$e} statement} statements) that are active
with respect to that assertion.  Some are mandatory and the others are
optional.  You should review these concepts if necessary.

Each label\index{label} in a proof must be either the label of a
previous assertion (\texttt{\$a}\index{\texttt{\$a} statement} or
\texttt{\$p}\index{\texttt{\$p} statement} statement) or the label of an
active hypothesis (\texttt{\$e} or \texttt{\$f}\index{\texttt{\$f}
statement} statement) of the \texttt{\$p} statement containing the
proof.  Hypothesis labels may reference both the
mandatory\index{mandatory hypothesis} and the optional hypotheses of the
\texttt{\$p} statement.

The label sequence in a proof specifies a construction in {\bf reverse Polish
notation}\index{reverse Polish notation (RPN)} (RPN).  You may be familiar
with RPN if you have used Hewlett-Packard or similar hand-held calculators.
In the calculator analogy, a hypothesis label\index{hypothesis label} is like
a number and an assertion label\index{assertion label} is like an operation.
On an RPN calculator, an operation takes one or more previous numbers in an
input sequence, performs a calculation on them, and replaces those numbers and
itself with the result of the calculation.  For example, the input sequence
$2,2,+$ on an RPN calculator results in $4$, and the input sequence
$1,2,2,+,+$ results in $1,4,+$ which results in $5$.

Understanding how RPN is processed involves the concept of a {\bf
stack}\index{stack}\index{RPN stack}, which can be thought of as a set of
temporary memory locations that hold intermediate results.  When Metamath
encounters a hypothesis label it places or {\bf pushes}\index{push} the math
symbol sequence of the hypothesis onto the stack.  When Metamath encounters an
assertion label, it associates the most recent stack entries with the {\em
mandatory} hypotheses\index{mandatory hypothesis} of the assertion, in the
order where the most recent stack entry is associated with the last mandatory
hypothesis of the assertion.  It then determines what
substitutions\index{substitution!variable}\index{variable substitution} have
to be made into the variables of the assertion's mandatory hypotheses to make
them identical to the associated stack entries.  It then makes those same
substitutions into the assertion itself.  Finally, Metamath removes or {\bf
pops}\index{pop} the matched hypotheses from the stack and pushes the
substituted assertion onto the stack.

For the purpose of matching the mandatory hypothesis to the most recent stack
entries, whether a hypothesis is a \texttt{\$e} or \texttt{\$f} statement is
irrelevant.  The only important thing is that a set of
substitutions\footnote{In the Metamath spec (Section~\ref{spec}), we use the
singular term ``substitution'' to refer to the set of substitutions we talk
about here.} exist that allow a match (and if they don't, the proof verifier
will let you know with an error message).  The Metamath language is specified
in such a way that if a set of substitutions exists, it will be unique.
Specifically, the requirement that each variable have a type specified for it
with a \texttt{\$f} statement ensures the uniqueness.

We will illustrate this with an example.
Consider the following Metamath source file:
\begin{verbatim}
        $c ( ) -> wff $.
        $v p q r s $.
        wp $f wff p $.
        wq $f wff q $.
        wr $f wff r $.
        ws $f wff s $.
        w2 $a wff ( p -> q ) $.
        wnew $p wff ( s -> ( r -> p ) ) $= ws wr wp w2 w2 $.
\end{verbatim}
This Metamath source example shows the definition and ``proof'' (i.e.,
construction) of a well-formed formula (wff)\index{well-formed formula (wff)}
in propositional calculus.  (You may wish to type this example into a file to
experiment with the Metamath program.)  The first two statements declare
(introduce the names of) four constants and four variables.  The next four
statements specify the variable types, namely that
each variable is assumed to be a wff.  Statement \texttt{w2} defines (postulates)
a way to produce a new wff, \texttt{( p -> q )}, from two given wffs \texttt{p} and
\texttt{q}. The mandatory hypotheses of \texttt{w2} are \texttt{wp} and \texttt{wq}.
Statement \texttt{wnew} claims that \texttt{( s -> ( r -> p ) )} is a wff given
three wffs \texttt{s}, \texttt{r}, and \texttt{p}.  More precisely, \texttt{wnew} claims
that the sequence of ten symbols \texttt{wff ( s -> ( r -> p ) )} is provable from
previous assertions and the hypotheses of \texttt{wnew}.  Metamath does not know
or care what a wff is, and as far as it is concerned \texttt{wff} is just an
arbitrary constant symbol in a math symbol sequence.  The mandatory hypotheses
of \texttt{wnew} are \texttt{wp}, \texttt{wr}, and \texttt{ws}; \texttt{wq} is an optional
hypothesis.  In our particular proof, the optional hypothesis is not
referenced, but in general, any combination of active (i.e.\ optional and
mandatory) hypotheses could be referenced.  The proof of statement \texttt{wnew}
is the sequence of five labels starting with \texttt{ws} (step~1) and ending with
\texttt{w2} (step~5).

When Metamath verifies the proof, it scans the proof from left to right.  We
will examine what happens at each step of the proof.  The stack starts off
empty.  At step 1, Metamath looks up label \texttt{ws} and determines that it is a
hypothesis, so it pushes the symbol sequence of statement \texttt{ws} onto the
stack:

\begin{center}\begin{tabular}{|l|l|}\hline
{Stack location} & {Contents} \\ \hline \hline
1 & \texttt{wff s} \\ \hline
\end{tabular}\end{center}

Metamath sees that the labels \texttt{wr} and \texttt{wp} in steps~2 and 3 are also
hypotheses, so it pushes them onto the stack.  After step~3, the stack looks
like
this:

\begin{center}\begin{tabular}{|l|l|}\hline
{Stack location} & {Contents} \\ \hline \hline
3 & \texttt{wff p} \\ \hline
2 & \texttt{wff r} \\ \hline
1 & \texttt{wff s} \\ \hline
\end{tabular}\end{center}

At step 4, Metamath sees that label \texttt{w2} is an assertion, so it must do
some processing.  First, it associates the mandatory hypotheses of \texttt{w2},
which are \texttt{wp} and \texttt{wq}, with stack locations~2 and 3, {\em in that
order}. Metamath determines that the only possible way
to make hypothesis \texttt{wp} match (become identical to) stack location~2 and
\texttt{wq} match stack location 3 is to substitute variable \texttt{p} with \texttt{r}
and \texttt{q} with \texttt{p}.  Metamath makes these substitutions into \texttt{w2} and
obtains the symbol sequence \texttt{wff ( r -> p )}.  It removes the hypotheses
from stack locations~2 and 3, then places the result into stack location~2:

\begin{center}\begin{tabular}{|l|l|}\hline
{Stack location} & {Contents} \\ \hline \hline
2 & \texttt{wff ( r -> p )} \\ \hline
1 & \texttt{wff s} \\ \hline
\end{tabular}\end{center}

At step 5, Metamath sees that label \texttt{w2} is an assertion, so it must again
do some processing.  First, it matches the mandatory hypotheses of \texttt{w2},
which are \texttt{wp} and \texttt{wq}, to stack locations 1 and 2.
Metamath determines that the only possible way to make the
hypotheses match is to substitute variable \texttt{p} with \texttt{s} and \texttt{q} with
\texttt{( r -> p )}.  Metamath makes these substitutions into \texttt{w2} and obtains
the symbol
sequence \texttt{wff ( s -> ( r -> p ) )}.  It removes stack
locations 1 and 2, then places the result into stack location~1:

\begin{center}\begin{tabular}{|l|l|}\hline
{Stack location} & {Contents} \\ \hline \hline
1 & \texttt{wff ( s -> ( r -> p ) )} \\ \hline
\end{tabular}\end{center}

After Metamath finishes processing the proof, it checks to see that the
contents of stack location 1 is the same as the math symbol sequence in the
\texttt{\$p}\index{\texttt{\$p} statement} statement.  This is the case for our
proof of \texttt{wnew}, so we have proved \texttt{wnew} successfully.  If the result
differs, Metamath will notify you with an error message.  An error message
will also result if the stack contains more than one entry at the end of the
proof, or if the stack did not contain enough entries at any point in the
proof to match all of the mandatory hypotheses\index{mandatory hypothesis} of
an assertion.  Finally, Metamath will notify you with an error message if no
substitution is possible that will make a referenced assertion's hypothesis
match the
stack entries.  You may want to experiment with the different kinds of errors
that Metamath will detect by making some small changes in the proof of our
example.

Metamath's proof notation was designed primarily to express proofs in a
relatively compact manner, not for readability by humans.  Metamath can display
proofs in a number of different ways with the \texttt{show proof}\index{\texttt{show
proof} command} command.  The
\texttt{/lemmon} qualifier displays it in a format that is easier to read when the
proofs are short, and you saw examples of its use in Chapter~\ref{using}.  For
longer proofs, it is useful to see the tree structure of the proof.  A tree
structure is displayed when the \texttt{/lemmon} qualifier is omitted.  You will
probably find this display more convenient as you get used to it. The tree
display of the proof in our example looks like
this:\label{treeproof}\index{tree-style proof}\index{proof!tree-style}
\begin{verbatim}
        1     wp=ws    $f wff s
        2        wp=wr    $f wff r
        3        wq=wp    $f wff p
        4     wq=w2    $a wff ( r -> p )
        5  wnew=w2  $a wff ( s -> ( r -> p ) )
\end{verbatim}
The number to the left of each line is the step number.  Following it is a
{\bf hypothesis association}\index{hypothesis association}, consisting of two
labels\index{label} separated by \texttt{=}.  To the left of the \texttt{=} (except
in the last step) is the label of a hypothesis of an assertion referenced
later in the proof; here, steps 1 and 4 are the hypothesis associations for
the assertion \texttt{w2} that is referenced in step 5.  A hypothesis association
is indented one level more than the assertion that uses it, so it is easy to
find the corresponding assertion by moving directly down until the indentation
level decreases to one less than where you started from.  To the right of each
\texttt{=} is the proof step label for that proof step.  The statement keyword of
the proof step label is listed next, followed by the content of the top of the
stack (the most recent stack entry) as it exists after that proof step is
processed.  With a little practice, you should have no trouble reading proofs
displayed in this format.

Our simple example shows the syntax construction of a formula that might be
used as part of the construction of a ``real'' proof step that deduces
theorems from other theorems.  In standard mathematics, this kind of
construction is not considered a proper part of the proof at all, and it
certainly becomes rather boring after a while.  To filter out syntax
constructions in the proof display, the \texttt{show proof}\index{\texttt{show proof}
command} command has the qualifier \texttt{/essential} which you will probably
use quite often in order to see just the ``real'' steps of
the proof.

When verifying a proof, Metamath will check that no mandatory
\texttt{\$d}\index{\texttt{\$d} statement}\index{mandatory \texttt{\$d}
statement} statement of an assertion referenced in a proof is violated
when substitutions\index{substitution!variable}\index{variable
substitution} are made to the variables in the assertion.  For details
see Section~\ref{spec4} or \ref{dollard}.

\subsection{The Concept of Unification} \label{unify}

During the course of verifying a proof, when Metamath\index{Metamath}
encounters an assertion label\index{assertion label}, it associates the
mandatory hypotheses\index{mandatory hypothesis} of the assertion with the top
entries of the RPN stack\index{stack}\index{RPN stack}.  Metamath then
determines what substitutions\index{substitution!variable}\index{variable
substitution} it must make to the variables in the assertion's mandatory
hypotheses in order for these hypotheses to become identical to their
corresponding stack entries.  This process is called {\bf
unification}\index{unification}.  (We also informally use the term
``unification'' to refer to a set of substitutions that results from the
process, as in ``two unifications are possible.'')  After the substitutions
are made, the hypotheses are said to be {\bf unified}.

If no such substitutions are possible, Metamath will consider the proof
incorrect and notify you with an error message.
% (deleted 3/10/07, per suggestion of Mel O'Cat:)
% The syntax of the
% Metamath language ensures that if a set of substitutions exists, it
% will be unique.

The general algorithm for unification described in the literature is
somewhat complex.
However, in the case of Metamath it is intentionally trivial.
Mandatory hypotheses must be
pushed on the proof stack in the order in which they appear.
In addition, each variable must have its type specified
with a \texttt{\$f} hypothesis before it is used
and that each \texttt{\$f} hypothesis
have the restricted syntax of a constant followed by a variable.
The constant in the \texttt{\$f} hypothesis must match the first symbol of
the corresponding RPN stack entry (which will also be a constant), so
the only possible match for the variable in the \texttt{\$f} hypothesis is
the sequence of symbols in the stack entry after the initial constant.

In the Proof Assistant\index{Proof Assistant}, a more general unification
algorithm is used.  While a proof is being developed, sometimes not enough
information is available to determine a unique unification.  In this case
Metamath will ask you to pick the correct one.\index{ambiguous
unification}\index{unification!ambiguous}

\section{Extensions to the Metamath Language}\index{extended
language}

\subsection{Comments in the Metamath Language}\label{comments}

The commenting feature allows you to annotate the contents of
a database.  Just as with most
computer languages, comments are ignored for the purpose of interpreting the
contents of the database. Comments effectively act as
additional white space\index{white
space} between tokens
when a database is parsed.

A comment may be placed between any two tokens\index{token} in a source
file.

Comments have the following syntax:
\begin{center}
 \texttt{\$(} {\em text} \texttt{\$)}
\end{center}
Here,\index{\texttt{\$(} and \texttt{\$)} auxiliary
keywords}\index{comment} {\em text} is a string, possibly empty, of any
characters in Metamath's character set (p.~\pageref{spec1chars}), except
that the character strings \texttt{\$(} and \texttt{\$)} may not appear
in {\em text}.  Thus nested comments are not
permitted:\footnote{Computer languages have differing standards for
nested comments, and rather than picking one it was felt simplest not to
allow them at all, at least in the current version of
Metamath\index{Metamath!limitations of version 0.07.30}.} Metamath will
complain if you give it
\begin{center}
 \texttt{\$( This is a \$( nested \$) comment.\ \$)}
\end{center}
(To compensate for this behavior, I often change all \texttt{\$}'s to
\texttt{@}'s in sections of Metamath code I wish to comment out.)

\subsubsection{Math Symbols and Labels Inside Comments}
\label{mathcomments}
\index{\texttt{`} inside comments}
\index{\texttt{\char`\~} inside comments}
\index{markup notation}
\index{comments!markup notation}

Inside of comments, a string of tokens\index{token} enclosed in grave
accents\index{grave accent (\texttt{`})} (\texttt{`}) will be converted
to standard mathematical symbols during \LaTeX\ output
typesetting,\index{latex@{\LaTeX}} according to the information in the
special \texttt{\$t}\index{\texttt{\$t} comment}\index{typesetting
comment} comment in the database, as described in Appendix~\ref{ASCII}.
The first \texttt{`} causes the output processor to enter {\bf math
mode}\index{math mode} and the second one exits it.  Two consecutive
grave accents \texttt{``} are treated as a single actual grave accent
(both inside and outside of math mode) and will not cause the output
processor to enter or exit math mode.

Outside of math mode, any token preceded by a tilde\index{tilde
(\texttt{\char`\~})} (\verb/~/) will be formatted in \texttt{typewriter}
font, and the tilde removed, to make them stand out from the rest of the
text.  This formatting will be applied from all characters after the
tilde up to the first white space\index{white space}.  This formatting
mode is called {\bf label mode}\index{label mode}.  If a literal tilde
is desired (outside of math mode), use two tildes in a row to represent
it.

These markup features have to do only with how the comments are typeset,
and have no effect on how Metamath verifies the database.  The improper
use of them may result in incorrectly typeset output, but no Metamath
error messages will result during the \texttt{read} and \texttt{verify
proof} commands.  (However, the \texttt{write
theorem\texttt{\char`\_}list} command, added in version 0.07.30 of the
Metamath program, will check for markup errors as a side-effect of its
{\sc html} generation.)

Section~\ref{texout} has instructions for creating \LaTeX\ output.

\subsubsection{Math Symbols In Comments}\index{math symbol}\index{math mode}

The grave accent\index{grave accent (\texttt{`})} \texttt{`} tells
Metamath\index{Metamath} to switch a comment to math mode.  In this
mode, the characters following the \texttt{`} are interpreted as a
sequence of math symbol tokens separated by white space\index{white
space}.  The tokens are looked up in the \texttt{\$t}
comment\index{\texttt{\$t} comment}\index{typesetting comment} and if
found, they will be replaced by the standard mathematical symbols that
they correspond to before being placed in the typeset output file.  If
not found, the symbol will be output as is and a warning will be issued.
The tokens do not have to be active in the database, although a warning
will be issued if they are not declared with \texttt{\$c} or
\texttt{\$v} statements.

The comment will stay in math mode until a second \texttt{`} is found or the end
of the comment is reached.  Here is an example of its use\index{Pierce's
axiom}:
\begin{center}
\texttt{\$( Pierce's axiom, ` ( ( ph -> ps ) -> ph ) -> ph ` ,\\
         is not very intuitive. \$)}
\end{center}
becomes
\begin{center}
   \texttt{\$(} Pierce's axiom, $((\varphi \rightarrow \psi)\rightarrow
\varphi)\rightarrow \varphi$, is not very intuitive. \texttt{\$)}
\end{center}

Note that the math symbol tokens\index{token} must be surrounded by white
space\index{white space}.
%, since there is no context that allows ambiguity to be
%resolved, as is the case with math symbol sequences in some of the Metamath
%statements.
White space should also surround the \texttt{`}
delimiters.

The math mode feature also gives you a quick and easy way to generate
text containing mathematical symbols, independently of the intended
purpose of Metamath.\index{Metamath!using as a math editor} To do this,
simply create your text with grave accents surrounding your formulas,
after making sure that your math symbols are mapped to \LaTeX\ symbols
as described in Appendix~\ref{ASCII}.  It is easier if you start with a
database with predefined symbols such as \texttt{set.mm}.  Use your
grave-quoted math string to replace an existing comment, then typeset
the statement corresponding to that comment following the instructions
from the \texttt{help tex} command in the Metamath program.  You will
then probably want to edit the resulting file with a text editor to fine
tune it to your exact needs.

\subsubsection{Label References in Comments}\index{label mode}

Outside of math mode, a tilde\index{tilde (\texttt{\char`\~})} \verb/~/
indicates to Metamath's\index{Metamath} output processor that the
token\index{token} that follows (i.e.\ the characters up to the next
white space\index{white space}) represents a statement label.  Whether
or not the token is an actual statement label is not checked, and the
token does not have to have the correct syntax for a label; no error
messages will be produced.  The only effect of the label mode on the
output is that typewriter font will be used for the tokens that are
placed in the \LaTeX\ output file.

(On the other hand, the tokens after the tilde {\em must} be actual
labels for correct output of {\sc html}, described in the next section,
and error messages will be issued during that output if they aren't.)

\subsection{Comment Markup Notation for HTML}\label{htmlmkup}
\index{\texttt{`} inside comments}
\index{\texttt{\char`\~} inside comments}
\index{markup notation}
\index{comments!markup notation}

The automated generation of {\sc html}\index{html generation@{\sc html}
generation} web pages
(Section~\ref{htmlout}) was added in version 0.07.30 of the Metamath program.
The comment markup notation for \LaTeX\ also applies to {\sc html}, and
some additional features specifically for {\sc html} were added.
These markup features (including those reused from \LaTeX) are the
following.
\begin{itemize}
 \item[]
         \texttt{`} {\em math-symbol} {\em math-symbol} ... \texttt{`}
             (math symbols enclosed in grave accents i.e.\ backticks) -
             Use graphical {\em math-symbol}s
             in the {\sc html} output.
             Inside of a {\em math-symbol}, \texttt{``} means literal \texttt{`}.
 \item[]
         \texttt{\char`\~} {\em label} (tilde followed by a label)
             - Use  a hyperlink in the {\sc html} output
             that links to the web page for statement {\em label}.
             Exception:  if {\em label} begins with \texttt{http://},
             it is assumed to be
             a {\sc url} (which is used as-is, except that
             a \texttt{\char`\~} in the {\sc url} should
             be specified with \texttt{\char`\~}\texttt{\char`\~}).
             Only \texttt{\$a} and \texttt{\$p}
             statement labels may be used, since web pages for
             \texttt{\$e} and \texttt{\$f} statements are not generated.
 \item[]
         \texttt{[}{\em author}\texttt{]} - Link to a bibliographical reference. See
             \texttt{help html} and \texttt{help write
             bibliography} in the Metamath program for more
            information.\index{\texttt{\char`\[}\ldots\texttt{]} inside comments}
            See also Sections~\ref{tcomment} and \ref{wrbib}.
 \item[]
         \texttt{\$t} - Flag the comment as the special
             one\index{\texttt{\$t} comment}\index{typesetting comment}
             containing \LaTeX\ and/or {\sc html} typesetting
             definitions.  See Section~\ref{tcomment} and Appendix~\ref{ASCII}.
 \item[]
         \texttt{\char`\_} (underscore)\index{\texttt{\char`\_} inside comments} -
             Italicize text starting from
              {\em space}\texttt{\char`\_}{\em non-space} (i.e.\ \texttt{\char`\_}
              with a space before it and a non-space character after it) until
             the next
             {\em non-space}\texttt{\char`\_}{\em space}.  Normal
             punctuation (e.g.\ a trailing
             comma or period) is ignored when determining {\em space}.
 \item[]
         \texttt{\char`\_} (underscore) - {\em
         non-space}\texttt{\char`\_}{\em non-space-string}, where
          {\em non-space-string} is a string of non-space characters,
         will make {\em non-space-string} become a subscript.
\end{itemize}
It is recommended that spaces surround any \texttt{\char`\~} and \texttt{`} tokens
in the comment and that a space follow the {\em label} after a \texttt{\char`\~} token.
This will make global substitutions to change labels and symbol names much easier
and also eliminate any future chance of ambiguity.  Spaces
around these tokens are automatically removed in
the final output to conform with normal rules of punctuation; for example, a space
between a trailing \texttt{`} and a left parenthesis will be removed.

The \texttt{[}{\em author}\texttt{]} notation will also create an entry in
the bibliography cross-reference file generated by \texttt{write
bibliography} (Section~\ref{wrbib}).  For this to work properly, the
surrounding comment must be formatted as follows:
\begin{quote}
    {\em keyword} {\em label} {\em noise-word}
     \texttt{[}{\em author}\texttt{] p.} {\em number}
\end{quote}
for example
\begin{verbatim}
     Theorem 5.2 of [Monk] p. 223
\end{verbatim}
The {\em keyword} is not case sensitive and must be one of the following:
\begin{verbatim}
     theorem lemma definition compare proposition corollary
     axiom rule remark exercise problem notation example
     property figure postulate equation scheme chapter
\end{verbatim}
The optional {\em label} may consist of more than one
(non-{\em keyword} and non-{\em noise-word}) word.
The optional {\em noise-word} is one of:
\begin{verbatim}
     of in from on
\end{verbatim}
and is  ignored when the cross-reference file is created.  The
\texttt{write
biblio\-graphy} command will perform error checking to verify the
above format.\index{error checking}

A good way to become familiar with the markup notation is to look at
the extensive examples in the \texttt{set.mm} database.


\subsection{Including Other Files in a Metamath Source File} \label{include}
\index{\texttt{\$[} and \texttt{\$]} auxiliary keywords}

The keywords \texttt{\$[} and \texttt{\$]} specify a file to be
included\index{included file}\index{file inclusion} at that point in a
Metamath\index{Metamath} source file\index{source file}.  The syntax for
including a file is as follows:
\begin{center}
\texttt{\$[} {\em file-name} \texttt{\$]}
\end{center}

The {\em file-name} should be a single token\index{token} with the same syntax
as a math symbol (i.e., all 93 printable characters other than \texttt{\$} are
allowed, subject to the file-naming limitations of your operating system).
Comments may appear between the \texttt{\$[} and \texttt{\$]} keywords.  Included
files may include other files, which may in turn include other files, and so
on.

For example, suppose you want to use the set theory database as the starting
point for your own theory.  The first line in your file could be
\begin{center}
\texttt{\$[ set.mm \$]}
\end{center} All of the information (axioms, theorems,
etc.) in \texttt{set.mm} and any files that {\em it} includes will become
available for you to reference in your file. This can help make your work more
modular. A drawback to including files is that if you change the name of a
symbol or the label of a statement, you must also remember to update any
references in any file that includes it.


The naming conventions for included files are the same as those of your
operating system.\footnote{On the Macintosh, prior to Mac OS X,
 a colon is used to separate disk
and folder names from your file name.  For example, {\em volume}\texttt{:}{\em
file-name} refers to the root directory, {\em volume}\texttt{:}{\em
folder-name}\texttt{:}{\em file-name} refers to a folder in root, and {\em
volume}\texttt{:}{\em folder-name}\texttt{:}\ldots\texttt{:}{\em file-name} refers to a
deeper folder.  A simple {\em file-name} refers to a file in the folder from
which you launch the Metamath application.  Under Mac OS X and later,
the Metamath program is run under the Terminal application, which
conforms to Unix naming conventions.}\index{Macintosh file
names}\index{file names!Macintosh}\label{includef} For compatibility among
operating systems, you should keep the file names as simple as possible.  A
good convention to use is {\em file}\texttt{.mm} where {\em file} is eight
characters or less, in lower case.

There is no limit to the nesting depth of included files.  One thing that you
should be aware of is that if two included files themselves include a common
third file, only the {\em first} reference to this common file will be read
in.  This allows you to include two or more files that build on a common
starting file without having to worry about label and symbol conflicts that
would occur if the common file were read in more than once.  (In fact, if a
file includes itself, the self reference will be ignored, although of course
it would not make any sense to do that.)  This feature also means, however,
that if you try to include a common file in several inner blocks, the result
might not be what you expect, since only the first reference will be replaced
with the included file (unlike the include statement in most other computer
languages).  Thus you would normally include common files only in the
outermost block\index{outermost block}.

\subsection{Compressed Proof Format}\label{compressed1}\index{compressed
proof}\index{proof!compressed}

The proof notation presented in Section~\ref{proof} is called a
{\bf normal proof}\index{normal proof}\index{proof!normal} and in principle is
sufficient to express any proof.  However, proofs often contain steps and
subproofs that are identical.  This is particularly true in typical
Metamath\index{Metamath} applications, because Metamath requires that the math
symbol sequence (usually containing a formula) at each step be separately
constructed, that is, built up piece by piece. As a result, a lot of
repetition often results.  The {\bf compressed proof} format allows Metamath
to take advantage of this redundancy to shorten proofs.

The specification for the compressed proof format is given in
Appen\-dix~\ref{compressed}.

Normally you need not concern yourself with the details of the compressed
proof format, since the Metamath program will allow you to convert from
the normal format to the compressed format with ease, and will also
automatically convert from the compressed format when proofs are displayed.
The overall structure of the compressed format is as follows:
\begin{center}
  \texttt{\$= ( } {\em label-list} \texttt{) } {\em compressed-proof\ }\ \texttt{\$.}
\end{center}
\index{\texttt{\$=} keyword}
The first \texttt{(} serves as a flag to Metamath that a compressed proof
follows.  The {\em label-list} includes all statements referred to by the
proof except the mandatory hypotheses\index{mandatory hypothesis}.  The {\em
compressed-proof} is a compact encoding of the proof, using upper case
letters, and can be thought of as a large integer in base 26.  White
space\index{white space} inside of {\em compressed-proof} is
optional and is ignored.

It is important to note that the order of the mandatory hypotheses of
the statement being proved must not be changed if the compressed proof
format is used, otherwise the proof will become incorrect.  The reason
for this is that the mandatory hypotheses are not mentioned explicitly
in the compressed proof in order to make the compression more efficient.
If you wish to change the order of mandatory hypotheses, you must first
convert the proof back to normal format using the \texttt{save proof
{\em statement} /normal}\index{\texttt{save proof} command} command.
Later, you can go back to compressed format with \texttt{save proof {\em
statement} /compressed}.

During error checking with the \texttt{verify proof} command, an error
found in a compressed proof may point to a character in {\em
compressed-proof}, which may not be very meaningful to you.  In this
case, try to \texttt{save proof /normal} first, then do the
\texttt{verify proof} again.  In general, it is best to make sure a
proof is correct before saving it in compressed format, because severe
errors are less likely to be recoverable than in normal format.

\subsection{Specifying Unknown Proofs or Subproofs}\label{unknown}

In a proof under development, any step or subproof that is not yet known
may be represented with a single \texttt{?}.  For the purposes of
parsing the proof, the \texttt{?}\ \index{\texttt{]}@\texttt{?}\ inside
proofs} will push a single entry onto the RPN stack just as if it were a
hypothesis.  While developing a proof with the Proof
Assistant\index{Proof Assistant}, a partially developed proof may be
saved with the \texttt{save new{\char`\_}proof}\index{\texttt{save
new{\char`\_}proof} command} command, and \texttt{?}'s will be placed at
the appropriate places.

All \texttt{\$p}\index{\texttt{\$p} statement} statements must have
proofs, even if they are entirely unknown.  Before creating a proof with
the Proof Assistant, you should specify a completely unknown proof as
follows:
\begin{center}
  {\em label} \texttt{\$p} {\em statement} \texttt{\$= ?\ \$.}
\end{center}
\index{\texttt{\$=} keyword}
\index{\texttt{]}@\texttt{?}\ inside proofs}

The \texttt{verify proof}\index{\texttt{verify proof} command} command
will check the known portions of a partial proof for errors, but will
warn you that the statement has not been proved.

Note that partially developed proofs may be saved in compressed format
if desired.  In this case, you will see one or more \texttt{?}'s in the
{\em compressed-proof} part.\index{compressed
proof}\index{proof!compressed}

\section{Appendix:  Axioms vs.\ Definitions}\label{definitions}

Metamath\index{Metamath} makes no distinction\index{axiom vs.\
definition} between axioms\index{axiom} and
definitions.\index{definition} The \texttt{\$a}\index{\texttt{\$a}
statement} statement is used for both.  At first, this may seem
puzzling.  In the minds of many mathematicians, the distinction is
clear, even obvious, and hardly worth discussing.  A definition is
considered to be merely an abbreviation that can be replaced by the
expression for which it stands; although unless one actually does this,
to be precise then one should say that a theorem\index{theorem} is a
consequence of the axioms {\em and} the definitions that are used in the
formulation of the theorem \cite[p.~20]{Behnke}.\index{Behnke, H.}

What is a definition?  In its simplest form, a definition introduces a new
symbol and provides an unambiguous rule to transform an expression containing
the new symbol to one without it.  The concept of a ``proper
definition''\index{proper definition}\index{definition!proper} (as opposed to
a creative definition)\index{creative definition}\index{definition!creative}
that is usually agreed upon is (1) the definition should not strengthen the
language and (2) any symbols introduced by the definition should be eliminable
from the language \cite{Nemesszeghy}\index{Nemesszeghy, E. Z.}.  In other
words, they are mere typographical conveniences that do not belong to the
system and are theoretically superfluous.  This may seem obvious, but in fact
the nature of definitions can be subtle, sometimes requiring difficult
metatheorems to establish that they are not creative.

A more conservative stance was taken by logician S.
Le\'{s}niewski.\index{Le\'{s}niewski, S.}
\begin{quote}
Le\'{s}niewski
regards definitions as theses of the system.  In this respect they do
not differ either from the axioms or from theorems, i.e.\ from the
theses added to the system on the basis of the rule of substitution or
the rule of detachment [modus ponens].  Once definitions have been
accepted as theses of the system, it becomes necessary to consider them
as true propositions in the same sense in which axioms are true
\cite{Lejewski}.
\end{quote}\index{Lejewski, Czeslaw}

Let us look at some simple examples of definitions in propositional
calculus.  Consider the definition of logical {\sc or}
(disjunction):\index{disjunction ($\vee$)} ``$P\vee Q$ denotes $\neg P
\rightarrow Q$ (not $P$ implies $Q$).''  It is very easy to recognize a
statement making use of this definition, because it introduces the new
symbol $\vee$ that did not previously exist in the language.  It is easy
to see that no new theorems of the original language will result from
this definition.

Next, consider a definition that eliminates parentheses:  ``$P
\rightarrow Q\rightarrow R$ denotes $P\rightarrow (Q \rightarrow R)$.''
This is more subtle, because no new symbols are introduced.  The reason
this definition is considered proper is that no new symbol sequences
that are valid wffs (well-formed formulas)\index{well-formed formula
(wff)} in the original language will result from the definition, since
``$P \rightarrow Q\rightarrow R$'' is not a wff in the original
language.  Here, we implicitly make use of the fact that there is a
decision procedure that allows us to determine whether or not a symbol
sequence is a wff, and this fact allows us to use symbol sequences that
are not wffs to represent other things (such as wffs) by means of the
definition.  However, to justify the definition as not being creative we
need to prove that ``$P \rightarrow Q\rightarrow R$'' is in fact not a
wff in the original language, and this is more difficult than in the
case where we simply introduce a new symbol.

%Now let's take this reasoning to an extreme.  Propositional calculus is a
%decidable theory,\footnote{This means that a mechanical algorithm exists to
%determine whether or not a wff is a theorem.} so in principle we could make use
%of symbol sequences that are not theorems to represent other things (say, to
%encode actual theorems in a more compact way).  For example, let us extend the
%language by defining a wff ``$P$'' in the extended language as the theorem
%``$P\rightarrow P$''\footnote{This is one of the first theorems proved in the
%Metamath database \texttt{set.mm}.}\index{set
%theory database (\texttt{set.mm})} in the original language whenever ``$P$'' is
%not a theorem in the original language.  In the extended language, any wff
%``$Q$'' thus represents a theorem; to find out what theorem (in the original
%language) ``$Q$'' represents, we determine whether ``$Q$'' is a theorem in the
%original language (before the definition was introduced).  If so, we're done; if
%not, we replace ``$Q$'' by ``$Q\rightarrow Q$'' to eliminate the definition.
%This definition is therefore eliminable, and it does not ``strengthen'' the
%language because any wff that is not a theorem is not in the set of statements
%provable in the original language and thus is available for use by definitions.
%
%Of course, a definition such as this would render practically useless the
%communication of theorems of propositional calculus; but
%this is just a human shortcoming, since we can't always easily discern what is
%and is not a theorem by inspection.  In fact, the extended theory with this
%definition has no more and no less information than the original theory; it just
%expresses certain theorems of the form ``$P\rightarrow P$''
%in a more compact way.
%
%The point here is that what constitutes a proper definition is a matter of
%judgment about whether a symbol sequence can easily be recognized by a human
%as invalid in some sense (for example, not a wff); if so, the symbol sequence
%can be appropriated for use by a definition in order to make the extended
%language more compact.  Metamath\index{Metamath} lacks the ability to make this
%judgment, since as far as Metamath is concerned the definition of a wff, for
%example, is arbitrary.  You define for Metamath how wffs\index{well-formed
%formula (wff)} are constructed according to your own preferred style.  The
%concept of a wff may not even exist in a given formal system\index{formal
%system}.  Metamath treats all definitions as if they were new axioms, and it
%is up to the human mathematician to judge whether the definition is ``proper''
%'\index{proper definition}\index{definition!proper} in some agreed-upon way.

What constitutes a definition\index{definition} versus\index{axiom vs.\
definition} an axiom\index{axiom} is sometimes arbitrary in mathematical
literature.  For example, the connectives $\vee$ ({\sc or}), $\wedge$
({\sc and}), and $\leftrightarrow$ (equivalent to) in propositional
calculus are usually considered defined symbols that can be used as
abbreviations for expressions containing the ``primitive'' connectives
$\rightarrow$ and $\neg$.  This is the way we treat them in the standard
logic and set theory database \texttt{set.mm}\index{set theory database
(\texttt{set.mm})}.  However, the first three connectives can also be
considered ``primitive,'' and axiom systems have been devised that treat
all of them as such.  For example,
\cite[p.~35]{Goodstein}\index{Goodstein, R. L.} presents one with 15
axioms, some of which in fact coincide with what we have chosen to call
definitions in \texttt{set.mm}.  In certain subsets of classical
propositional calculus, such as the intuitionist
fragment\index{intuitionism}, it can be shown that one cannot make do
with just $\rightarrow$ and $\neg$ but must treat additional connectives
as primitive in order for the system to make sense.\footnote{Two nice
systems that make the transition from intuitionistic and other weak
fragments to classical logic just by adding axioms are given in
\cite{Robinsont}\index{Robinson, T. Thacher}.}

In set theory, recursive definitions define a newly introduced symbol in
terms of itself.  The justification of recursive definitions, using
several ``recursion theorems,'' is usually one of the first
sophisticated proofs a student encounters when learning set theory, and
there is a significant amount of implicit metalogic behind a recursive
definition even though the definition itself is typically simple to
state.  It is, however, possible to substitute one kind of complexity
for another.  We can eliminate the need for metalogical justification by
defining the operation directly with an explicit (but complicated)
expression, then deriving the recursive definition directly as a
theorem, using a recursion theorem ``in reverse.''  We do this in
\texttt{set.mm}, as follows.

In \texttt{set.mm} our goal was to introduce almost all definitions in
the form of two expressions connected by either $\leftrightarrow$ or
$=$, where the thing being defined does not appear on the right hand
side.  Quine calls this form ``a genuine or direct definition'' \cite[p.
174]{Quine}\index{Quine, Willard Van Orman}, which makes the definitions
very easy to eliminate and the metalogic\index{metalogic} needed to
justify them as simple as possible.  We achieved this goal in almost all
cases.  Sometimes this makes the definitions more complex and less
intuitive.  For example, the traditional way to define addition of
natural numbers is to define an operation called {\em
successor}\index{successor} (which means ``plus one'' and is denoted by
``${\rm suc}$''), then define addition recursively\index{recursive
definition} with the two definitions $n + 0 = n$ and $m + {\rm suc}\,n =
{\rm suc} (m + n)$.  Although this definition seems simple and obvious,
the method to eliminate the definition is not obvious:  in the second
part of the definition, addition is defined in terms of itself.  By
eliminating the definition, we don't mean repeatedly applying it to
specific $m$ and $n$ but rather showing the explicit, closed-form
set-theoretical expression that $m + n$ represents, that will work for
any $m$ and $n$ and that does not have a $+$ sign on its right-hand
side.  For a recursive definition like this not to be circular
(creative), there are some hidden, underlying assumptions we must make,
for example that the natural numbers have a certain kind of order.  In
\texttt{set.mm} we chose to start with the direct (though complex and
nonintuitive) definition then derive from it the standard recursive
definition.\footnote{The closed-form definition used in \texttt{set.mm}
for the addition operation on ordinals\index{ordinal
addition}\index{addition!of ordinals} (of which natural numbers are a
subset) is

\setbox\startprefix=\hbox{\tt \ \ df-oadd\ \$a\ }
\setbox\contprefix=\hbox{\tt \ \ \ \ \ \ \ \ \ \ \ \ \ }
\startm
\m{\vdash}\m{+_o}\m{=}\m{\{}\m{\langle}\m{\langle}\m{x}\m{,}\m{y}\m{\rangle}%
\m{,}\m{z}\m{\rangle}\m{|}\m{(}\m{(}\m{x}\m{\in}\m{{\rm On}}\m{\wedge}\m{y}\m{%
\in}\m{{\rm On}}\m{)}\m{\wedge}\m{z}\m{=}\m{(}\m{{\rm rec}}\m{(}\m{\{}\m{%
\langle}\m{w}\m{,}\m{v}\m{\rangle}\m{|}\m{v}\m{=}\m{{\rm suc}}\m{w}\m{\}}\m{,}%
\m{x}\m{)}\m{`}\m{y}\m{)}\m{)}\m{\}}
\endm

\noindent Here, the abstraction class of nested ordered pairs\index{abstraction
class!of nested ordered pairs} is defined by \texttt{df-oprab} in \texttt{set.mm},
and ${\rm rec}$ is a ``recursion operator''\index{recursion operator} with
the definition

\setbox\startprefix=\hbox{\tt \ \ df-rfg\ \$a\ }
\setbox\contprefix=\hbox{\tt \ \ \ \ \ \ \ \ \ \ \ \ }
\startm
\m{\vdash}\m{{\rm rec}}\m{(}\m{F}\m{,}\m{A}\m{)}\m{=}\m{\bigcup}\m{\{}\m{f}%
\m{|}\m{\exists}\m{x}\m{(}\m{x}\m{\in}\m{{\rm On}}\m{\wedge}\m{(}\m{f}\m{{%
\rm Fn}}\m{x}\m{\wedge}\m{\forall}\m{y}\m{(}\m{y}\m{\in}\m{x}\m{\rightarrow}%
\m{(}\m{f}\m{`}\m{y}\m{)}\m{=}\m{(}\m{\{}\m{\langle}\m{g}\m{,}\m{z}\m{\rangle}%
\m{|}\m{(}\m{(}\m{g}\m{=}\m{\varnothing}\m{\wedge}\m{z}\m{=}\m{A}\m{)}\m{\vee}%
\m{(}\m{\lnot}\m{(}\m{g}\m{=}\m{\varnothing}\m{\vee}\m{{\rm Lim}}\m{{\rm dom}}%
\m{g}\m{)}\m{\wedge}\m{z}\m{=}\m{(}\m{F}\m{`}\m{(}\m{g}\m{`}\m{\bigcup}\m{{%
\rm dom}}\m{g}\m{)}\m{)}\m{)}\m{\vee}\m{(}\m{{\rm Lim}}\m{{\rm dom}}\m{g}\m{%
\wedge}\m{z}\m{=}\m{\bigcup}\m{{\rm ran}}\m{g}\m{)}\m{)}\m{\}}\m{`}\m{(}\m{f}%
\m{\restriction}\m{y}\m{)}\m{)}\m{)}\m{)}\m{)}\m{\}}
\endm

\noindent which can be further broken down with definitions shown in
Section~\ref{setdefinitions}.  You may be surprised at the complexity of what
seems like such a simple notion.  From these definitions the simpler, more
intuitive recursive definition is derived as a set of theorems.}\index{natural
number}\index{addition}\index{recursive definition}\index{ordinal addition}
The end result is the same, but we completely eliminate the rather complex
metalogic that justifies the recursive definition.  (For a mathematician,
recursive definitions are more efficient and intuitive than
direct ones once the metalogic has been learned or possibly just accepted as
correct.  However, it was felt that direct definition in \texttt{set.mm}
maximizes rigor by minimizing metalogic.  It can be eliminated effortlessly,
something difficult to do with a recursive definition.)


\chapter{The Metamath Program}\label{commands}

This chapter provides a reference manual for the
Metamath program.\index{Metamath!commands}

Current instructions for obtaining and installing the Metamath program
can be found at the \url{http://metamath.org} web site.
For Windows, there is a pre-compiled version called
\texttt{metamath.exe}.  For Unix, Linux, and Mac OS X
(which we will refer to collectively as ``Unix''), the Metamath program
can be compiled from its source code with the command
\begin{verbatim}
  gcc *.c -o metamath
\end{verbatim}
using the \texttt{gcc} {\sc c} compiler available on those systems.

In the command syntax descriptions below, fields enclosed in square
brackets [\ ] are optional.  File names may be optionally enclosed in
single or double quotes.  This is useful if the file name contains
slashes (\texttt{/}), such as in Unix path names, \index{Unix file
names}\index{file names!Unix} that might be confused with Metamath
command qualifiers.\index{Unix file names}\index{file names!Unix}

\section{Invoking Metamath}

Unix, Linux, and Mac OS X
have a command-line interface called the {\em
bash shell}.  (In Mac OS X, select the Terminal application from
Applications/Utilities.)  To invoke Metamath from the bash shell prompt,
assuming that the Metamath program is in the current directory, type
\begin{verbatim}
  bash$ ./metamath
\end{verbatim}

To invoke Metamath from a Windows DOS or Command Prompt, assuming that
the Metamath program is in the current directory (or in a directory
included in the Path system environment variable), type
\begin{verbatim}
  C:\metamath>metamath
\end{verbatim}

To use command-line arguments at invocation, the command-line arguments
should be a list of Metamath commands, surrounded by quotes if they
contain spaces.  In Windows, the surrounding quotes must be double (not
single) quotes.  For example, to read the database file \texttt{set.mm},
verify all proofs, and exit the program, type (under Unix)
\begin{verbatim}
  bash$ ./metamath 'read set.mm' 'verify proof *' exit
\end{verbatim}
Note that in Unix, any directory path with \texttt{/}'s must be
surrounded by quotes so Metamath will not interpret the \texttt{/} as a
command qualifier.  So if \texttt{set.mm} is in the \texttt{/tmp}
directory, use for the above example
\begin{verbatim}
  bash$ ./metamath 'read "/tmp/set.mm"' 'verify proof *' exit
\end{verbatim}

For convenience, if the command-line has one argument and no spaces in
the argument, the command is implicitly assumed to be \texttt{read}.  In
this one special case, \texttt{/}'s are not interpreted as command
qualifiers, so you don't need quotes around a Unix file name.  Thus
\begin{verbatim}
  bash$ ./metamath /tmp/set.mm
\end{verbatim}
and
\begin{verbatim}
  bash$ ./metamath "read '/tmp/set.mm'"
\end{verbatim}
are equivalent.


\section{Controlling Metamath}

The Metamath program was first developed on a {\sc vax/vms} system, and
some aspects of its command line behavior reflect this heritage.
Hopefully you will find it reasonably user-friendly once you get used to
it.

Each command line is a sequence of English-like words separated by
spaces, as in \texttt{show settings}.  Command words are not case
sensitive, and only as many letters are needed as are necessary to
eliminate ambiguity; for example, \texttt{sh se} would work for the
command \texttt{show settings}.  In some cases arguments such as file
names, statement labels, or symbol names are required; these are
case-sensitive (although file names may not be on some operating
systems).

A command line is entered by typing it in then pressing the {\em return}
({\em enter}) key.  To find out what commands are available, type
\texttt{?} at the \texttt{MM>} prompt.  To find out the choices at any
point in a command, press {\em return} and you will be prompted for
them.  The default choice (the one selected if you just press {\em
return}) is shown in brackets (\texttt{<>}).

You may also type \texttt{?} in place of a command word to force
Metamath to tell you what the choices are.  The \texttt{?} method won't
work, though, if a non-keyword argument such as a file name is expected
at that point, because the program will think that \texttt{?} is the
value of the argument.

Some commands have one or more optional qualifiers which modify the
behavior of the command.  Qualifiers are preceded by a slash
(\texttt{/}), such as in \texttt{read set.mm / verify}.  Spaces are
optional around the \texttt{/}.  If you need to use a slash in a command
argument, as in a Unix file name, put single or double quotes around the
command argument.

The \texttt{open log} command will save everything you see on the
screen and is useful to help you recover should something go wrong in a
proof, or if you want to document a bug.

If a command responds with more than a screenful, you will be
prompted to \texttt{<return> to continue, Q to quit, or S to scroll to
end}.  \texttt{Q} or \texttt{q} (not case-sensitive) will complete the
command internally but will suppress further output until the next
\texttt{MM>} prompt.  \texttt{s} will suppress further pausing until the
next \texttt{MM>} prompt.  After the first screen, you are also
presented with the choice of \texttt{b} to go back a screenful.  Note
that \texttt{b} may also be entered at the \texttt{MM>} prompt
immediately after a command to scroll back through the output of that
command.

A command line enclosed in quotes is executed by your operating system.
See Section~\ref{oscmd}.

{\em Warning:} Pressing {\sc ctrl-c} will abort the Metamath program
unconditionally.  This means any unsaved work will be lost.


\subsection{\texttt{exit} Command}\index{\texttt{exit} command}

Syntax:  \texttt{exit} [\texttt{/force}]

This command exits from Metamath.  If there have been changes to the
source with the \texttt{save proof} or \texttt{save new{\char`\_}proof}
commands, you will be given an opportunity to \texttt{write source} to
permanently save the changes.

In Proof Assistant\index{Proof Assistant} mode, the \texttt{exit} command will
return to the \verb/MM>/ prompt. If there were changes to the proof, you will
be given an opportunity to \texttt{save new{\char`\_}proof}.

The \texttt{quit} command is a synonym for \texttt{exit}.

Optional qualifier:
    \texttt{/force} - Do not prompt if changes were not saved.  This qualifier is
        useful in \texttt{submit} command files (Section~\ref{sbmt})
        to ensure predictable behavior.





\subsection{\texttt{open log} Command}\index{\texttt{open log} command}
Syntax:  \texttt{open log} {\em file-name}

This command will open a log file that will store everything you see on
the screen.  It is useful to help recovery from a mistake in a long Proof
Assistant session, or to document bugs.\index{Metamath!bugs}

The log file can be closed with \texttt{close log}.  It will automatically be
closed upon exiting Metamath.



\subsection{\texttt{close log} Command}\index{\texttt{close log} command}
Syntax:  \texttt{close log}

The \texttt{close log} command closes a log file if one is open.  See
also \texttt{open log}.




\subsection{\texttt{submit} Command}\index{\texttt{submit} command}\label{sbmt}
Syntax:  \texttt{submit} {\em filename}

This command causes further command lines to be taken from the specified
file.  Note that any line beginning with an exclamation point (\texttt{!}) is
treated as a comment (i.e.\ ignored).  Also note that the scrolling
of the screen output is continuous, so you may want to open a log file
(see \texttt{open log}) to record the results that fly by on the screen.
After the lines in the file are exhausted, Metamath returns to its
normal user interface mode.

Currently, the \texttt{submit} command is not recursive.  In other words,
\texttt{submit} commands are not allowed inside of a command file.


\subsection{\texttt{erase} Command}\index{\texttt{erase} command}
Syntax:  \texttt{erase}

This command will reset Metamath to its starting state, deleting any
data\-base that was \texttt{read} in.
 If there have been changes to the
source with the \texttt{save proof} or \texttt{save new{\char`\_}proof}
commands, you will be given an opportunity to \texttt{write source} to
permanently save the changes.



\subsection{\texttt{set echo} Command}\index{\texttt{set echo} command}
Syntax:  \texttt{set echo on} or \texttt{set echo off}

The \texttt{set echo on} command will cause command lines to be echoed with any
abbreviations expanded.  While learning the Metamath commands, this
feature will show you the exact command that your abbreviated input
corresponds to.



\subsection{\texttt{set scroll} Command}\index{\texttt{set scroll} command}
Syntax:  \texttt{set scroll prompted} or \texttt{set scroll continuous}

The Metamath command line interface starts off in the \texttt{prompted} mode,
which means that you will be prompted to continue or quit after each
full screen in a long listing.  In \texttt{continuous} mode, long listings will be
scrolled without pausing.

% LaTeX bug? (1) \texttt{\_} puts out different character than
% \texttt{{\char`\_}}
%  = \verb$_$  (2) \texttt{{\char`\_}} puts out garbage in \subsection
%  argument
\subsection{\texttt{set width} Command}\index{\texttt{set
width} command}
Syntax:  \texttt{set width} {\em number}

Metamath assumes the width of your screen is 79 characters (chosen
because the Command Prompt in Windows XP has a wrapping bug at column
80).  If your screen is wider or narrower, this command allows you to
change this default screen width.  A larger width is advantageous for
logging proofs to an output file to be printed on a wide printer.  A
smaller width may be necessary on some terminals; in this case, the
wrapping of the information messages may sometimes seem somewhat
unnatural, however.  In \LaTeX\index{latex@{\LaTeX}!characters per line},
there is normally a maximum of 61 characters per line with typewriter
font.  (The examples in this book were produced with 61 characters per
line.)

\subsection{\texttt{set height} Command}\index{\texttt{set
height} command}
Syntax:  \texttt{set height} {\em number}

Metamath assumes your screen height is 24 lines of characters.  If your
screen is taller or shorter, this command lets you to change the number
of lines at which the display pauses and prompts you to continue.

\subsection{\texttt{beep} Command}\index{\texttt{beep} command}

Syntax:  \texttt{beep}

This command will produce a beep.  By typing it ahead after a
long-running command has started, it will alert you that the command is
finished.  For convenience, \texttt{b} is an abbreviation for
\texttt{beep}.

Note:  If \texttt{b} is typed at the \texttt{MM>} prompt immediately
after the end of a multiple-page display paged with ``\texttt{Press
<return> for more}...'' prompts, then the \texttt{b} will back up to the
previous page rather than perform the \texttt{beep} command.
In that case you must type the unabbreviated \texttt{beep} form
of the command.

\subsection{\texttt{more} Command}\index{\texttt{more} command}

Syntax:  \texttt{more} {\em filename}

This command will display the contents of an {\sc ascii} file on your
screen.  (This command is provided for convenience but is not very
powerful.  See Section~\ref{oscmd} to invoke your operating system's
command to do this, such as the \texttt{more} command in Unix.)

\subsection{Operating System Commands}\index{operating system
command}\label{oscmd}

A line enclosed in single or double quotes will be executed by your
computer's operating system if it has a command line interface.  For
example, on a {\sc vax/vms} system,
\verb/MM> 'dir'/
will print disk directory contents.  Note that this feature will not
work on the Macintosh prior to Mac OS X, which does not have a command
line interface.

For your convenience, the trailing quote is optional.

\subsection{Size Limitations in Metamath}

In general, there are no fixed, predefined limits\index{Metamath!memory
limits} on how many labels, tokens\index{token}, statements, etc.\ that
you may have in a database file.  The Metamath program uses 32-bit
variables (64-bit on 64-bit CPUs) as indices for almost all internal
arrays, which are allocated dynamically as needed.



\section{Reading and Writing Files}

The following commands create new files:  the \texttt{open} commands;
the \texttt{write} commands; the \texttt{/html},
\texttt{/alt{\char`\_}html}, \texttt{/brief{\char`\_}html},
\texttt{/brief{\char`\_}alt{\char`\_}html} qualifiers of \texttt{show
statement}, and \texttt{midi}.  The following commands append to files
previously opened:  the \texttt{/tex} qualifier of \texttt{show proof}
and \texttt{show new{\char`\_}proof}; the \texttt{/tex} and
\texttt{/simple{\char`\_}tex} qualifiers of \texttt{show statement}; the
\texttt{close} commands; and all screen dialog between \texttt{open log}
and \texttt{close log}.

The commands that create new files will not overwrite an existing {\em
filename} but will rename the existing one to {\em
filename}\texttt{{\char`\~}1}.  An existing {\em
filename}\texttt{{\char`\~}1} is renamed {\em
filename}\texttt{{\char`\~}2}, etc.\ up to {\em
filename}\texttt{{\char`\~}9}.  An existing {\em
filename}\texttt{{\char`\~}9} is deleted.  This makes recovery from
mistakes easier but also will clutter up your directory, so occasionally
you may want to clean up (delete) these \texttt{{\char`\~}}$n$ files.


\subsection{\texttt{read} Command}\index{\texttt{read} command}
Syntax:  \texttt{read} {\em file-name} [\texttt{/verify}]

This command will read in a Metamath language source file and any included
files.  Normally it will be the first thing you do when entering Metamath.
Statement syntax is checked, but proof syntax is not checked.
Note that the file name may be enclosed in single or double quotes;
this is useful if the file name contains slashes, as might be the case
under Unix.

If you are getting an ``\texttt{?Expected VERIFY}'' error
when trying to read a Unix file name with slashes, you probably haven't
quoted it.\index{Unix file names}\index{file names!Unix}

If you are prompted for the file name (by pressing {\em return}
 after \texttt{read})
you should {\em not} put quotes around it, even if it is a Unix file name
with slashes.

Optional command qualifier:

    \texttt{/verify} - Verify all proofs as the database is read in.  This
         qualifier will slow down reading in the file.  See \texttt{verify
         proof} for more information on file error-checking.

See also \texttt{erase}.



\subsection{\texttt{write source} Command}\index{\texttt{write source} command}
Syntax:  \texttt{write source} {\em filename} [\texttt{/clean}]

This command will write the contents of a Metamath\index{database}
database into a file.\index{source file} Note:  The present version of
Metamath\index{Metamath!limitations of version 0.07.30} will not
split the database into its constituent source files included with
\texttt{\$[}\index{\texttt{\$[} and \texttt{\$]} auxiliary keywords} and
\texttt{\$]} keywords.  A future version is planned to properly separate
all constituent files.

Optional command qualifier (primarily intended to assist web site updates):

    \texttt{/clean} - Suppresses (deletes) the output of any theorem that
        has been flagged
        with a question mark (\texttt{?}) placed in or in place of the date
       comment field at the
        end of its proof, for example ``\texttt{\$( [?31-Oct-00] \$)}.''
        This lets
        you strip out proofs under development so that a ``clean''
        version of the database can be generated for official release.
        {\em Note:}  Currently, hypotheses are not stripped, only \texttt{\$p}
        statements.
        Spurious date comment fields of the suppressed theorems may also
        remain.  Be careful to use a different name for the \texttt{/clean}
        version so that your work in progress won't be destroyed.


\section{Showing Status and Statements}



\subsection{\texttt{show settings} Command}\index{\texttt{show settings} command}
Syntax:  \texttt{show settings}

This command shows the state of various parameters.

\subsection{\texttt{show memory} Command}\index{\texttt{show memory} command}
Syntax:  \texttt{show memory}

This command shows the available memory left.  It is not meaningful
on most modern operating systems,
which have virtual memory.\index{Metamath!memory usage}


\subsection{\texttt{show labels} Command}\index{\texttt{show labels} command}
Syntax:  \texttt{show labels} {\em label-match} [\texttt{/all}]
   [\texttt{/linear}]

This command shows the labels of \texttt{\$a} and \texttt{\$p}
statements that match {\em label-match}.  A \verb$*$ in {label-match}
matches zero or more characters.  For example, \verb$*abc*def$ will match all
labels containing \verb$abc$ and ending with \verb$def$.

Optional command qualifier:

   \texttt{/all} - Include matches for \texttt{\$e} and \texttt{\$f}
   statement labels.

   \texttt{/linear} - Display only one label per line.  This can be useful for
       building scripts in conjunction with the utilities under the
       \texttt{tools}\index{\texttt{tools} command} command.



\subsection{\texttt{show statement} Command}\index{\texttt{show statement} command}
Syntax:  \texttt{show statement} {\em label} [{\em qualifiers} (see below)]

This command provides information about a statement.  Only statements
that have labels (\texttt{\$f}\index{\texttt{\$f} statement},
\texttt{\$e}\index{\texttt{\$e} statement},
\texttt{\$a}\index{\texttt{\$a} statement}, and
\texttt{\$p}\index{\texttt{\$p} statement}) may be specified.
If {\em label}
contains wildcard (\verb$*$) characters, all matching statements will be
displayed in the order they occur in the database.

Optional qualifiers (only one qualifier at a time is allowed):

    \texttt{/comment} - This qualifier includes the comment that immediately
       precedes the statement.

    \texttt{/full} - Show complete information about each statement,
       and show all
       statements matching {\em label} (including \texttt{\$e}
       and \texttt{\$f} statements).

    \texttt{/tex} - This qualifier will write the statement information to the
       \LaTeX\ file previously opened with \texttt{open tex}.  See
       Section~\ref{texout}.

    \texttt{/simple{\char`\_}tex} - The same as \texttt{/tex}, except that
       \LaTeX\ macros are not used for formatting equations, allowing easier
       manual edits of the output for slide presentations, etc.

    \texttt{/html}\index{html generation@{\sc html} generation},
       \texttt{/alt{\char`\_}html}, \texttt{/brief{\char`\_}html},
       \texttt{/brief{\char`\_}alt{\char`\_}html} -
       These qualifiers invoke a special mode of
       \texttt{show statement} that
       creates a web page for the statement.  They may not be used with
       any other qualifier.  See Section~\ref{htmlout} or
       \texttt{help html} in the program.


\subsection{\texttt{search} Command}\index{\texttt{search} command}
Syntax:  search {\em label-match}
\texttt{"}{\em symbol-match}{\tt}" [\texttt{/ all}] [\texttt{/ comments}]

This command searches all \texttt{\$a} and \texttt{\$p} statements
matching {\em label-match} for occurrences of {\em symbol-match}.  A
\verb@*@ in {\em label-match} matches any label character.  A \verb@$*@
in {\em symbol-match} matches any sequence of symbols.  The symbols in
{\em symbol-match} must be separated by white space.  The quotes
surrounding {\em symbol-match} may be single or double quotes.  For
example, \texttt{search b}\verb@* "-> $* ch"@ will list all statements
whose labels begin with \texttt{b} and contain the symbols \verb@->@ and
\texttt{ch} surrounding any symbol sequence (including no symbol
sequence).  The wildcards \texttt{?} and \texttt{\$?} are also available
to match individual characters in labels and symbols respectively; see
\texttt{help search} in the Metamath program for details on their usage.

Optional command qualifiers:

    \texttt{/ all} - Also search \texttt{\$e} and \texttt{\$f} statements.

    \texttt{/ comments} - Search the comment that immediately precedes each
        label-matched statement for {\em symbol-match}.  In this case
        {\em symbol-match} is an arbitrary, non-case-sensitive character
        string.  Quotes around {\em symbol-match} are optional if there
        is no ambiguity.



\section{Displaying and Verifying Proofs}


\subsection{\texttt{show proof} Command}\index{\texttt{show proof} command}
Syntax:  \texttt{show proof} {\em label} [{\em qualifiers} (see below)]

This command displays the proof of the specified
\texttt{\$p}\index{\texttt{\$p} statement} statement in various formats.
The {\em label} may contain wildcard (\verb@$*@) characters to match
multiple statements.  Without any qualifiers, only the logical steps
will be shown (i.e.\ syntax construction steps will be omitted), in an
indented format.

Most of the time, you will use
    \texttt{show proof} {\em label}
to see just the proof steps corresponding to logical inferences.

Optional command qualifiers:

    \texttt{/essential} - The proof tree is trimmed of all
        \texttt{\$f}\index{\texttt{\$f} statement} hypotheses before
        being displayed.  (This is the default, and it is redundant to
        specify it.)

    \texttt{/all} - the proof tree is not trimmed of all \texttt{\$f} hypotheses before
        being displayed.  \texttt{/essential} and \texttt{/all} are mutually exclusive.

    \texttt{/from{\char`\_}step} {\em step} - The display starts at the specified
        step.  If
        this qualifier is omitted, the display starts at the first step.

    \texttt{/to{\char`\_}step} {\em step} - The display ends at the specified
        step.  If this
        qualifier is omitted, the display ends at the last step.

    \texttt{/tree{\char`\_}depth} {\em number} - Only
         steps at less than the specified proof
        tree depth are displayed.  Sometimes useful for obtaining an overview of
        the proof.

    \texttt{/reverse} - The steps are displayed in reverse order.

    \texttt{/renumber} - When used with \texttt{/essential}, the steps are renumbered
        to correspond only to the essential steps.

    \texttt{/tex} - The proof is converted to \LaTeX\ and\index{latex@{\LaTeX}}
        stored in the file opened
        with \texttt{open tex}.  See Section~\ref{texout} or
        \texttt{help tex} in the program.

    \texttt{/lemmon} - The proof is displayed in a non-indented format known
        as Lemmon style, with explicit previous step number references.
        If this qualifier is omitted, steps are indented in a tree format.

    \texttt{/start{\char`\_}column} {\em number} - Overrides the default column
        at which the formula display starts in a Lemmon-style display.  May be
        used only in conjunction with \texttt{/lemmon}.

    \texttt{/normal} - The proof is displayed in normal format suitable for
        inclusion in a Metamath source file.  May not be used with any other
        qualifier.

    \texttt{/compressed} - The proof is displayed in compressed format
        suitable for inclusion in a Metamath source file.  May not be used with
        any other qualifier.

    \texttt{/statement{\char`\_}summary} - Summarizes all statements (like a
        brief \texttt{show statement})
        used by the proof.  It may not be used with any other qualifier
        except \texttt{/essential}.

    \texttt{/detailed{\char`\_}step} {\em step} - Shows the details of what is
        happening at
        a specific proof step.  May not be used with any other qualifier.
        The {\em step} is the step number shown when displaying a
        proof without the \texttt{/renumber} qualifier.


\subsection{\texttt{show usage} Command}\index{\texttt{show usage} command}
Syntax:  \texttt{show usage} {\em label} [\texttt{/recursive}]

This command lists the statements whose proofs make direct reference to
the statement specified.

Optional command qualifier:

    \texttt{/recursive} - Also include statements whose proofs ultimately
        depend on the statement specified.



\subsection{\texttt{show trace\_back} Command}\index{\texttt{show
       trace{\char`\_}back} command}
Syntax:  \texttt{show trace{\char`\_}back} [\texttt{/essential}] [\texttt{/axioms}]
    [\texttt{/tree}] [\texttt{/depth} {\em number}]

This command lists all statements that the proof of the specified
\texttt{\$p}\index{\texttt{\$p} statement} statement depends on.

Optional command qualifiers:

    \texttt{/essential} - Restrict the trace-back to \texttt{\$e}
        \index{\texttt{\$e} statement} hypotheses of proof trees.

    \texttt{/axioms} - List only the axioms that the proof ultimately depends on.

    \texttt{/tree} - Display the trace-back in an indented tree format.

    \texttt{/depth} {\em number} - Restrict the \texttt{/tree} trace-back to the
        specified indentation depth.

    \texttt{/count{\char`\_}steps} - Counts the number of steps the proof has
       all the way back to axioms.  If \texttt{/essential} is specified,
       expansions of variable-type hypotheses (syntax constructions) are not counted.

\subsection{\texttt{verify proof} Command}\index{\texttt{verify proof} command}
Syntax:  \texttt{verify proof} {\em label-match} [\texttt{/syntax{\char`\_}only}]

This command verifies the proofs of the specified statements.  {\em
label-match} may contain wild card characters (\texttt{*}) to verify
more than one proof; for example \verb/*abc*def/ will match all labels
containing \texttt{abc} and ending with \texttt{def}.  \texttt{verify
proof *} will verify all proofs in the database.

Optional command qualifier:

    \texttt{/syntax{\char`\_}only} - This qualifier will perform a check of syntax
        and RPN
        stack violations only.  It will not verify that the proof is
        correct.  This qualifier is useful for quickly determining which
        proofs are incomplete (i.e.\ are under development and have \texttt{?}'s
        in them).

{\em Note:} \texttt{read}, followed by \texttt{verify proof *}, will ensure
 the database is free
from errors in the Metamath language but will not check the markup notation
in comments; for that
see Section~\ref{htmlout}.

\subsection{\texttt{save proof} Command}\index{\texttt{save proof} command}
Syntax:  \texttt{save proof} {\em label} [\texttt{/normal}]
   [\texttt{/compressed}]

The \texttt{save proof} command will reformat a proof in one of two formats and
replace the existing proof in the source buffer\index{source
buffer}.  It is useful for
converting between proof formats.  Note that a proof will not be
permanently saved until a \texttt{write source} command is issued.

Optional command qualifiers:

    \texttt{/normal} - The proof is saved in the normal format (i.e., as a
        sequence
        of labels, which is the defined format of the basic Metamath
        language).\index{basic language}  This is the default format that
        is used if a qualifier
        is omitted.

    \texttt{/compressed} - The proof is saved in the compressed format which
        reduces storage requirements for a database.
        See Appendix~\ref{compressed}.




\section{Creating Proofs}\label{pfcommands}\index{Proof Assistant}

Before using the Proof Assistant, you must add a \texttt{\$p} to your
source file (using a text editor) containing the statement you want to
prove.  Its proof should consist of a single \texttt{?}, meaning
``unknown step.''  Example:
\begin{verbatim}
    equid $p x = x $= ? $.
\end{verbatim}

To enter the Proof assistant, type \texttt{prove} {\em label}, e.g.
\texttt{prove equid}.  Metamath will respond with the \texttt{MM-PA>}
prompt.

Proofs are created working backwards from the statement being proved,
primarily using a series of \texttt{assign} commands.  A proof is
complete when all steps are assigned to statements and all steps are
unified and completely known.  During the creation of a proof, Metamath
will allow only operations that are legal based on what is known up to
that point.  For example, it will not allow an \texttt{assign} of a
statement that cannot be unified with the unknown proof step being
assigned.

{\em Important:} You should figure out your first few proofs completely
and write them down by hand, before using the Proof Assistant.
Otherwise you will become extremely frustrated.  The Proof Assistant is
{\em not} a tool to help you discover proofs.  It is just a tool to help
you add them to the database.  For a tutorial read
Section~\ref{frstprf}.  To practice using the Proof Assistant, you may
want to \texttt{prove} an existing theorem, then delete all steps with
\texttt{delete all}, then re-create it with the Proof Assistant while
looking at its proof display (before deletion).

{\em Important:} Keep track of your work with a log file (\texttt{open
log}) and save it frequently (\texttt{save new{\char`\_}proof},
\texttt{write source}), because currently there is no undo command!
Hopefully there will be an undo command in a future version.
However, you can use \texttt{delete} to reverse an \texttt{assign}, and
you can do \texttt{delete floating{\char`\_}hypotheses}, then
\texttt{initialize all}, then \texttt{unify all /interactive} to
reinitialize bad unifications made accidentally or by bad
\texttt{assign}s.  You cannot reverse a \texttt{delete} except by
\texttt{exit /force} then reentering the Proof Assistant to recover from
the last \texttt{save new{\char`\_}proof}.

The following commands are available in the Proof Assistant (at the
\texttt{MM-PA>} prompt) to help you create your proof.  See the
individual commands for more detail.

\begin{itemize}
\item[]
    \texttt{show new{\char`\_}proof} [\texttt{/all},...] - Displays the
        proof in progress.  You will use this command a lot; see \texttt{help
        show new{\char`\_}proof} to become familiar with its qualifiers.  The
        qualifiers \texttt{/unknown} and \texttt{/not{\char`\_}unified} are
        useful for seeing the work remaining to be done.  The combination
        \texttt{/all/unknown} is useful for identifying dummy variables that must be
        assigned, or attempts to use illegal syntax, when \texttt{improve all}
        is unable to complete the syntax constructions.  Unknown variables are
        shown as \texttt{\$1}, \texttt{\$2},...
\item[]
    \texttt{assign} {\em step} {\em label} - Assigns an unknown {\em step}
        number with the statement
        specified by {\em label}.
\item[]
    \texttt{let variable} {\em variable}
        \texttt{= "}{\em symbol sequence}\texttt{"}
          - Forces a symbol
        sequence to replace an unknown variable (such as \texttt{\$1}) in a proof.
        It is useful
        for helping difficult unifications, and it is necessary when you have
        dummy variables that eventually must be assigned a name.
\item[]
    \texttt{let step} {\em step} \texttt{= "}{\em symbol sequence}\texttt{"} -
          Forces a symbol sequence
        to replace the contents of a proof step, provided it can be
        unified with the existing step contents.  (I rarely use this.)
\item[]
    \texttt{unify step} {\em step} (or \texttt{unify all}) - Unifies
        the source and target of
        a step.  If you specify a specific step, you will be prompted
        to select among the unifications that are possible.  If you
        specify \texttt{all}, all steps with unique unifications, but only
        those steps, will be
        unified.  \texttt{unify all /interactive} goes through all non-unified
        steps.
\item[]
    \texttt{initialize} {\em step} (or \texttt{all}) - De-unifies the target and source of
        a step (or all steps), as well as the hypotheses of the source,
        and makes all variables in the source unknown.  Useful to recover from
        an \texttt{assign} or \texttt{let} mistake that
        resulted in incorrect unifications.
\item[]
    \texttt{delete} {\em step} (or \texttt{all} or \texttt{floating{\char`\_}hypotheses}) -
        Deletes the specified
        step(s).  \texttt{delete floating{\char`\_}hypotheses}, then \texttt{initialize all}, then
        \texttt{unify all /interactive} is useful for recovering from mistakes
        where incorrect unifications assigned wrong math symbol strings to
        variables.
\item[]
    \texttt{improve} {\em step} (or \texttt{all}) -
      Automatically creates a proof for steps (with no unknown variables)
      whose proof requires no statements with \texttt{\$e} hypotheses.  Useful
      for filling in proofs of \texttt{\$f} hypotheses.  The \texttt{/depth}
      qualifier will also try statements whose \texttt{\$e} hypotheses contain
      no new variables.  {\em Warning:} Save your work (with \texttt{save
      new{\char`\_}proof} then \texttt{write source}) before using
      \texttt{/depth = 2} or greater, since the search time grows
      exponentially and may never terminate in a reasonable time, and you
      cannot interrupt the search.  I have found that it is rare for
      \texttt{/depth = 3} or greater to be useful.
 \item[]
    \texttt{save new{\char`\_}proof} - Saves the proof in progress in the program's
        internal database buffer.  To save it permanently into the database file,
        use \texttt{write source} after
        \texttt{save new{\char`\_}proof}.  To revert to the last
        \texttt{save new{\char`\_}proof},
        \texttt{exit /force} from the Proof Assistant then re-enter the Proof
        Assistant.
 \item[]
    \texttt{match step} {\em step} (or \texttt{match all}) - Shows what
        statements are
        possibilities for the \texttt{assign} statement. (This command
        is not very
        useful in its present form and hopefully will be improved
        eventually.  In the meantime, use the \texttt{search} statement for
        candidates matching specific math token combinations.)
 \item[]
 \texttt{minimize{\char`\_}with}\index{\texttt{minimize{\char`\_}with} command}
% 3/10/07 Note: line-breaking the above results in duplicate index entries
     - After a proof is complete, this command will attempt
        to match other database theorems to the proof to see if the proof
        size can be reduced as a result.  See \texttt{help
        minimize{\char`\_}with} in the
        Metamath program for its usage.
\end{itemize}

The following commands set parameters that may be relevant to your proof.
Consult the individual \texttt{help set}... commands.
\begin{itemize}
   \item[] \texttt{set unification{\char`\_}timeout}
 \item[]
    \texttt{set search{\char`\_}limit}
  \item[]
    \texttt{set empty{\char`\_}substitution} - note that default is \texttt{off}
\end{itemize}

Type \texttt{exit} to exit the \texttt{MM-PA>}
 prompt and get back to the \texttt{MM>} prompt.
Another \texttt{exit} will then get you out of Metamath.



\subsection{\texttt{prove} Command}\index{\texttt{prove} command}
Syntax:  \texttt{prove} {\em label}

This command will enter the Proof Assistant\index{Proof Assistant}, which will
allow you to create or edit the proof of the specified statement.
The command-line prompt will change from \texttt{MM>} to \texttt{MM-PA>}.

Note:  In the present version of
Metamath\index{Metamath!limitations of version 0.07.30}, the Proof
Assistant does not verify that \texttt{\$d}\index{\texttt{\$d}
statement} restrictions are met as a proof is being built.  After you
have completed a proof, you should type \texttt{save new{\char`\_}proof}
followed by \texttt{verify proof} {\em label} (where {\em label} is the
statement you are proving with the \texttt{prove} command) to verify the
\texttt{\$d} restrictions.

See also: \texttt{exit}

\subsection{\texttt{set unification\_timeout} Command}\index{\texttt{set
unification{\char`\_}timeout} command}
Syntax:  \texttt{set unification{\char`\_}timeout} {\em number}

(This command is available outside the Proof Assistant but affects the
Proof Assistant\index{Proof Assistant} only.)

Sometimes the Proof Assistant will inform you that a unification
time-out occurred.  This may happen when you try to \texttt{unify}
formulas with many temporary variables\index{temporary variable}
(\texttt{\$1}, \texttt{\$2}, etc.), since the time to compute all possible
unifications may grow exponentially with the number of variables.  If
you want Metamath to try harder (and you're willing to wait longer) you
may increase this parameter.  \texttt{show settings} will show you the
current value.



\subsection{\texttt{set empty\_substitution} Command}\index{\texttt{set
empty{\char`\_}substitution} command}
Syntax:  \texttt{set empty{\char`\_}substitution on} or \texttt{set
empty{\char`\_}substitution off}

(This command is available outside the Proof Assistant but affects the
Proof Assistant\index{Proof Assistant} only.)

The Metamath language allows variables to be
substituted\index{substitution!variable}\index{variable substitution}
with empty symbol sequences\index{empty substitution}.  However, in many
formal systems\index{formal system} this will never happen in a valid
proof.  Allowing for this possibility increases the likelihood of
ambiguous unifications\index{ambiguous
unification}\index{unification!ambiguous} during proof creation, and you
may want to \texttt{set empty{\char`\_}substitution off} to help make
the process more efficient.  With this mode set, you may not be able to
create some proofs in formal systems that allow empty substitutions.
(An example would be a system that implements a Deduction Rule and in
which deductions from empty assumption lists would be permissible.  The
MIU-system\index{MIU-system} described in Appendix~\ref{MIU} is another
example.)

Note that this command does not affect the way proofs are verified with
the \texttt{verify proof} command.  Outside of the Proof Assistant,
substitution of empty sequences for math symbols is always allowed.



\subsection{\texttt{set search\_limit} Command}\index{\texttt{set
search{\char`\_}limit} command} Syntax:  \texttt{set search{\char`\_}limit} {\em
number}

(This command is available outside the Proof Assistant but affects the
Proof Assistant\index{Proof Assistant} only.)

This command sets a parameter that determines when the \texttt{improve} command
in Proof Assistant mode gives up.  If you want \texttt{improve} to search harder,
you may increase it.  The \texttt{show settings} command tells you its current
value.


\subsection{\texttt{show new\_proof} Command}\index{\texttt{show
new{\char`\_}proof} command}
Syntax:  \texttt{show new{\char`\_}proof} [{\em
qualifiers} (see below)]

This command (available only in Proof Assistant mode) displays the proof
in progress.  It is identical to the \texttt{show proof} command, except that
there is no statement argument (since it is the statement being proved) and
the following qualifiers are not available:

    \texttt{/statement{\char`\_}summary}

    \texttt{/detailed{\char`\_}step}

Also, the following additional qualifiers are available:

    \texttt{/unknown} - Shows only steps that have no statement assigned.

    \texttt{/not{\char`\_}unified} - Shows only steps that have not been unified.

Note that \texttt{/essential}, \texttt{/depth}, \texttt{/unknown}, and
\texttt{/not{\char`\_}unified} may be
used in any combination; each of them effectively filters out additional
steps from the proof display.

See also:  \texttt{show proof}






\subsection{\texttt{assign} Command}\index{\texttt{assign} command}
Syntax:   \texttt{assign} {\em step} {\em label}

   and:   \texttt{assign first} {\em label}

   and:   \texttt{assign last} {\em label}


This command, available in the Proof Assistant only, assigns an unknown
step (one with \texttt{?} in the \texttt{show new{\char`\_}proof}
listing) with the statement specified by {\em label}.  The assignment
will not be allowed if the statement cannot be unified with the step.

If \texttt{last} is specified instead of {\em step} number, the last
step that is shown by \texttt{show new{\char`\_}proof /unknown} will be
used.  This can be useful for building a proof with a command file (see
\texttt{help submit}).  It also makes building proofs faster when you know
the assignment for the last step.

If \texttt{first} is specified instead of {\em step} number, the first
step that is shown by \texttt{show new{\char`\_}proof /unknown} will be
used.

If {\em step} is zero or negative, the -{\em step}th from last unknown
step, as shown by \texttt{show new{\char`\_}proof /unknown}, will be
used.  \texttt{assign -1} {\em label} will assign the penultimate
unknown step, \texttt{assign -2} {\em label} the antepenultimate, and
\texttt{assign 0} {\em label} is the same as \texttt{assign last} {\em
label}.

Optional command qualifier:

    \texttt{/no{\char`\_}unify} - do not prompt user to select a unification if there is
        more than one possibility.  This is useful for noninteractive
        command files.  Later, the user can \texttt{unify all /interactive}.
        (The assignment will still be automatically unified if there is only
        one possibility and will be refused if unification is not possible.)



\subsection{\texttt{match} Command}\index{\texttt{match} command}
Syntax:  \texttt{match step} {\em step} [\texttt{/max{\char`\_}essential{\char`\_}hyp}
{\em number}]

    and:  \texttt{match all} [\texttt{/essential}]
          [\texttt{/max{\char`\_}essential{\char`\_}hyp} {\em number}]

This command, available in the Proof Assistant only, shows what
statements can be unified with the specified step(s).  {\em Note:} In
its current form, this command is not very useful because of the large
number of matches it reports.  I rarely use it.
It may be enhanced in the future.

Optional command qualifiers:

    \texttt{/max{\char`\_}essential{\char`\_}hyp} {\em number} - filters out
        of the list any statements
        with more than the specified number of
        \texttt{\$e}\index{\texttt{\$e} statement} hypotheses

    \texttt{/essential{\char`\_}only} - in the \texttt{match all} statement, only
        the steps that
        would be listed in the \texttt{show new{\char`\_}proof /essential} display are
        matched.



\subsection{\texttt{let} Command}\index{\texttt{let} command}
Syntax: \texttt{let variable} {\em variable} = \verb/"/{\em symbol-sequence}\verb/"/

  and: \texttt{let step} {\em step} = \verb/"/{\em symbol-sequence}\verb/"/

These commands, available in the Proof Assistant\index{Proof Assistant}
only, assign a temporary variable\index{temporary variable} or unknown
step with a specific symbol sequence.  They are useful in the middle of
creating a proof, when you know what should be in the proof step but the
unification algorithm doesn't yet have enough information to completely
specify the temporary variables.  A ``temporary variable'' is one that
has the form \texttt{\$}{\em nn} in the proof display, such as
\texttt{\$1}, \texttt{\$2}, etc.  The {\em symbol-sequence} may contain
other unknown variables if desired.  Examples:

    \verb/let variable $32 = "A = B"/

    \verb/let variable $32 = "A = $35"/

    \verb/let step 10 = '|- x = x'/

    \verb/let step -2 = "|- ( $7 = ph )"/

Any symbol sequence will be accepted for the \texttt{let variable}
command.  Only those symbol sequences that can be unified with the step
will be accepted for \texttt{let step}.

The \texttt{let} commands ``zap'' the proof with information that can
only be verified when the proof is built up further.  If you make an
error, the command sequence \texttt{delete
floating{\char`\_}hypotheses}, \texttt{initialize all}, and
\texttt{unify all /interactive} will undo a bad \texttt{let} assignment.

If {\em step} is zero or negative, the -{\em step}th from last unknown
step, as shown by \texttt{show new{\char`\_}proof /unknown}, will be
used.  The command \texttt{let step 0} = \verb/"/{\em
symbol-sequence}\verb/"/ will use the last unknown step, \texttt{let
step -1} = \verb/"/{\em symbol-sequence}\verb/"/ the penultimate, etc.
If {\em step} is positive, \texttt{let step} may be used to assign known
(in the sense of having previously been assigned a label with
\texttt{assign}) as well as unknown steps.

Either single or double quotes can surround the {\em symbol-sequence} as
long as they are different from any quotes inside of {\em
symbol-sequence}.  If {\em symbol-sequence} contains both kinds of
quotes, see the instructions at the end of \texttt{help let} in the
Metamath program.


\subsection{\texttt{unify} Command}\index{\texttt{unify} command}
Syntax:  \texttt{unify step} {\em step}

      and:   \texttt{unify all} [\texttt{/interactive}]

These commands, available in the Proof Assistant only, unify the source
and target of the specified step(s). If you specify a specific step, you
will be prompted to select among the unifications that are possible.  If
you specify \texttt{all}, only those steps with unique unifications will be
unified.

Optional command qualifier for \texttt{unify all}:

    \texttt{/interactive} - You will be prompted to select among the
        unifications
        that are possible for any steps that do not have unique
        unifications.  (Otherwise \texttt{unify all} will bypass these.)

See also \texttt{set unification{\char`\_}timeout}.  The default is
100000, but increasing it to 1000000 can help difficult cases.  Manually
assigning some or all of the unknown variables with the \texttt{let
variable} command also helps difficult cases.



\subsection{\texttt{initialize} Command}\index{\texttt{initialize} command}
Syntax:  \texttt{initialize step} {\em step}

    and: \texttt{initialize all}

These commands, available in the Proof Assistant\index{Proof Assistant}
only, ``de-unify'' the target and source of a step (or all steps), as
well as the hypotheses of the source, and makes all variables in the
source and the source's hypotheses unknown.  This command is useful to
help recover from incorrect unifications that resulted from an incorrect
\texttt{assign}, \texttt{let}, or unification choice.  Part or all of
the command sequence \texttt{delete floating{\char`\_}hypotheses},
\texttt{initialize all}, and \texttt{unify all /interactive} will recover
from incorrect unifications.

See also:  \texttt{unify} and \texttt{delete}



\subsection{\texttt{delete} Command}\index{\texttt{delete} command}
Syntax:  \texttt{delete step} {\em step}

   and:      \texttt{delete all} -- {\em Warning: dangerous!}

   and:      \texttt{delete floating{\char`\_}hypotheses}

These commands are available in the Proof Assistant only.  The
\texttt{delete step} command deletes the proof tree section that
branches off of the specified step and makes the step become unknown.
\texttt{delete all} is equivalent to \texttt{delete step} {\em step}
where {\em step} is the last step in the proof (i.e.\ the beginning of
the proof tree).

There is currently no undo command, and you cannot reverse a
\texttt{delete}.  The best you can do is salvage your last \texttt{save
new{\char`\_}proof} by exiting and reentering the Proof Assistant.  For
this reason, it is important to keep a log file open to record your work
and to do \texttt{save new{\char`\_}proof} frequently, especially before
\texttt{delete}.

\texttt{delete floating{\char`\_}hypotheses} will delete all sections of
the proof that branch off of \texttt{\$f}\index{\texttt{\$f} statement}
statements.  It is sometimes useful to do this before an
\texttt{initialize} command to recover from an error.  Note that once a
proof step with a \texttt{\$f} hypothesis as the target is completely
known, the \texttt{improve} command can usually fill in the proof for
that step.  Unlike the deletion of logical steps, \texttt{delete
floating{\char`\_}hypotheses} is a relatively safe command that is
usually easy to recover from.



\subsection{\texttt{improve} Command}\index{\texttt{improve} command}
\label{improve}
Syntax:  \texttt{improve} {\em step} [\texttt{/depth} {\em number}]
                                               [\texttt{/no{\char`\_}distinct}]

   and:   \texttt{improve first} [\texttt{/depth} {\em number}]
                                              [\texttt{/no{\char`\_}distinct}]

   and:   \texttt{improve last} [\texttt{/depth} {\em number}]
                                              [\texttt{/no{\char`\_}distinct}]

   and:   \texttt{improve all} [\texttt{/depth} {\em number}]
                                              [\texttt{/no{\char`\_}distinct}]

These commands, available in the Proof Assistant\index{Proof Assistant}
only, try to find proofs automatically for unknown steps whose symbol
sequences are completely known.  They are primarily useful for filling in
proofs of \texttt{\$f}\index{\texttt{\$f} statement} hypotheses.  The
search will be restricted to statements having no
\texttt{\$e}\index{\texttt{\$e} statement} hypotheses.

Note:  If memory is limited, \texttt{improve all} on a large proof may
overflow memory.  If you use \texttt{set unification{\char`\_}timeout 1}
before \texttt{improve all}, there will usually be sufficient
improvement to easily recover and completely \texttt{improve} the proof
later on a larger computer.  Warning:  Once memory has overflowed, there
is no recovery.  If in doubt, save the intermediate proof (\texttt{save
new{\char`\_}proof} then \texttt{write source}) before \texttt{improve
all}.

If \texttt{last} is specified instead of {\em step} number, the last
step that is shown by \texttt{show new{\char`\_}proof /unknown} will be
used.

If \texttt{first} is specified instead of {\em step} number, the first
step that is shown by \texttt{show new{\char`\_}proof /unknown} will be
used.

If {\em step} is zero or negative, the -{\em step}th from last unknown
step, as shown by \texttt{show new{\char`\_}proof /unknown}, will be
used.  \texttt{improve -1} will use the penultimate
unknown step, \texttt{improve -2} {\em label} the antepenultimate, and
\texttt{improve 0} is the same as \texttt{improve last}.

Optional command qualifier:

    \texttt{/depth} {\em number} - This qualifier will cause the search
        to include
        statements with \texttt{\$e} hypotheses (but no new variables in
        the \texttt{\$e}
        hypotheses), provided that the backtracking has not exceeded the
        specified depth. {\em Warning:}  Try \texttt{/depth 1},
        then \texttt{2}, then \texttt{3}, etc.
        in sequence because of possible exponential blowups.  Save your
        work before trying \texttt{/depth} greater than \texttt{1}!
    \texttt{/no{\char`\_}distinct} - Skip trial statements that have
        \texttt{\$d}\index{\texttt{\$d} statement} requirements.
        This qualifier will prevent assignments that might violate \texttt{\$d}
        requirements but it also could miss possible legal assignments.

See also: \texttt{set search{\char`\_}limit}

\subsection{\texttt{save new\_proof} Command}\index{\texttt{save
new{\char`\_}proof} command}
Syntax:  \texttt{save new{\char`\_}proof} {\em label} [\texttt{/normal}]
   [\texttt{/compressed}]

The \texttt{save new{\char`\_}proof} command is available in the Proof
Assistant only.  It saves the proof in progress in the source
buffer\index{source buffer}.  \texttt{save new{\char`\_}proof} may be
used to save a completed proof, or it may be used to save a proof in
progress in order to work on it later.  If an incomplete proof is saved,
any user assignments with \texttt{let step} or \texttt{let variable}
will be lost, as will any ambiguous unifications\index{ambiguous
unification}\index{unification!ambiguous} that were resolved manually.
To help make recovery easier, it can be helpful to \texttt{improve all}
before \texttt{save new{\char`\_}proof} so that the incomplete proof
will have as much information as possible.

Note that the proof will not be permanently saved until a \texttt{write
source} command is issued.

Optional command qualifiers:

    \texttt{/normal} - The proof is saved in the normal format (i.e., as a
        sequence of labels, which is the defined format of the basic Metamath
        language).\index{basic language}  This is the default format that
        is used if a qualifier is omitted.

    \texttt{/compressed} - The proof is saved in the compressed format, which
        reduces storage requirements for a database.
        (See Appendix~\ref{compressed}.)



\section{Creating \LaTeX\ Output}\label{texout}\index{latex@{\LaTeX}}

The \texttt{show statement} and \texttt{show proof} commands each have a
special \texttt{/tex} command qualifier that produces \LaTeX\ output.
(The \texttt{show statement} command also has the
\texttt{/simple{\char`\_}tex} qualifier for output that is easier to
edit by hand.)  Before you can use them, you must open a \LaTeX\ file to
which to send their output.  A typical complete session will use this
sequence of Metamath commands:

\begin{verbatim}
    read set.mm
    open tex example.tex
    show statement a1i /tex
    show proof a1i /lemmon/renumber/tex
    show statement uneq2 /tex
    show proof uneq2 /lemmon/renumber/tex
    close tex
\end{verbatim}

See Section~\ref{mathcomments} for information on comment markup and
Appendix~\ref{ASCII} for information on how math symbol translation is
specified.

To format and print the \LaTeX\ source, you will need the \LaTeX\
program, which is standard on most Linux installations and available for
Windows.  On Linux, in order to create a {\sc pdf} file, you will
typically type at the shell prompt
\begin{verbatim}
    $ pdflatex example.tex
\end{verbatim}

\subsection{\texttt{open tex} Command}\index{\texttt{open tex} command}
Syntax:  \texttt{open tex} {\em file-name} [\texttt{/no{\char`\_}header}]

This command opens a file for writing \LaTeX\
source\index{latex@{\LaTeX}} and writes a \LaTeX\ header to the file.
\LaTeX\ source can be written with the \texttt{show proof}, \texttt{show
new{\char`\_}proof}, and \texttt{show statement} commands using the
\texttt{/tex} qualifier.

The mapping to \LaTeX\ symbols is defined in a special comment
containing a \texttt{\$t} token, described in Appendix~\ref{ASCII}.


Optional command qualifier:

    \texttt{/no{\char`\_}header} - This qualifier prevents a standard
        \LaTeX\ header and trailer
        from being included with the output \LaTeX\ code.


\subsection{\texttt{close tex} Command}\index{\texttt{close tex} command}
Syntax:  \texttt{close tex}

This command writes a trailer to any \LaTeX\ file\index{latex@{\LaTeX}}
that was opened with \texttt{open tex} (unless
\texttt{/no{\char`\_}header} was used with \texttt{open tex}) and closes
the \LaTeX\ file.


\section{Creating HTML Output}\label{htmlout}

The ability to produce {\sc html} web pages was added in Metamath version
0.07.30.

To create an {\sc html} output file for a \texttt{\$a} or \texttt{\$p}
statement, use
\begin{quote}
    \texttt{show statement} {\em label} \texttt{/html}
\end{quote}
The output file will be named {\em label}\texttt{.html}.  When {\em
label} has wildcard (\texttt{*}) characters, all statements with
matching labels will have {\sc html} files produced for them.  Also,
when {\em label} has a wildcard (\texttt{*}) character, two additional
files, \texttt{mmdefinitions.html} and \texttt{mmascii.html} will be
produced.  To produce {\em only} these two additional files, you can use
\texttt{?*}, which will not match any statement label, in place of {\em
label}.

There are three other qualifiers for \texttt{show statement} that also
generate {\sc HTML} code.  These are \texttt{/alt{\char`\_}html},
\texttt{/brief{\char`\_}html}, and
\texttt{/brief{\char`\_}alt{\char`\_}html}, and are described in the
next section.

A statement's comment can include a special notation that provides a
certain amount of control over the {\sc HTML} version of the comment.  See
Section~\ref{mathcomments} (p.~\pageref{htmlmkup}) for the comment
markup features.

The \texttt{write theorem{\char`\_}list} and \texttt{write bibliography}
commands, which are described below, provide as a side effect complete
error checking for all of the features described in this section.
(Currently there is no separate command to check for these
errors.)\index{error checking}

\subsection{The Typesetting Comment (\texttt{\$t})}\label{tcomment}

The {\sc html} definitions for math symbols, as well as some
customization of the generated web page, are specified by statements in
a special typesetting comment in the input database file.  The
typesetting comment is identified by the token
\texttt{\$t}\index{\texttt{\$t} comment}\index{typesetting comment} in
the comment, and the typesetting statements run until the next
\texttt{\$)}:
\[
  \mbox{\tt \$(\ }
  \mbox{\tt \$t\ }
  \underbrace{
    \mbox{\tt \ \ \ \ \ \ \ \ \ \ \ }
    \cdots
    \mbox{\tt \ \ \ \ \ \ \ \ \ \ \ }
  }_{\mbox{{\sc html} definitions go here}}
  \mbox{\tt \ \$)}
\]

In version 0.07.30\index{Metamath!limitations of version 0.07.30} of the
Metamath program, there may be only one \texttt{\$t} comment in a
database.  See the
\texttt{set.mm}\index{set theory database (\texttt{set.mm})} database
file for an extensive example of a \texttt{\$t} comment illustrating all
of the features described below.  In the {\sc html} definition section,
{\sc c}-style comments \texttt{/*}\ldots\texttt{*/} are recognized.  The
main {\sc html} specification statements are:

\vskip 1ex
    \texttt{htmldef "}{\em math-token}\texttt{" as "}{\em
    {\sc html}-code}\texttt{" ;}\index{\texttt{htmldef} statement}

                    \ \ \ \ \ \ldots

    \texttt{htmldef "}{\em math-token}\texttt{" as "}{\em {\sc
    html}-code}\texttt{" ;}

    \texttt{htmltitle "}{\em {\sc html}-code}\texttt{"
    ;}\index{\texttt{htmltitle} statement}

    \texttt{htmlhome "}{\em {\sc html}-code}\texttt{"
    ;}\index{\texttt{htmlhome} statement}

    \texttt{htmlvarcolors "}{\em {\sc html}-code}\texttt{"
    ;}\index{\texttt{htmlvarcolors} statement}

    \texttt{htmlbibliography "}{\em filename}\texttt{"
    ;}\index{\texttt{htmlbibliography} statement}

\vskip 1ex

\noindent The \texttt{htmltitle} is the {\sc html} code for a common
title, such as ``Metamath Proof Explorer.''  The \texttt{htmlhome} is
code for a link back to the home page.  The \texttt{htmlvarcolors} is
code for a color key that appears at the bottom of each proof.  The file
specified by {\em filename} is an {\sc html} file that is assumed to
have a \texttt{<A NAME=}\ldots\texttt{>} tag for each bibiographic
reference in the database comments.  For example, if
\texttt{[Monk]}\index{\texttt{\char`\[}\ldots\texttt{]} inside comments}
occurs in the comment for a theorem, then \texttt{<A NAME='Monk'>} must
be present in the file; if not, a warning message is given.

Single or double quotes surround the {\em {\sc html}-code} strings and
the {\em filename} string.  Strings too long for a line may be broken up
as descibed for the \texttt{latexdef} statement in Appendix~\ref{ASCII}.
That Appendix also describes how to handle strings containing quote
characters.

The \texttt{\$t} comment may also contain \LaTeX\ definitions (with
\texttt{latexdef} statements---see Appendix~\ref{ASCII}) that are
ignored for {\sc html} output.

Several other {\sc html}-related qualifiers exist for the
\texttt{show statement} command.  The command
\begin{quote}
    \texttt{show statement} {\em label} \texttt{/alt{\char`\_}html}
\end{quote}
does the same as \texttt{show statement} {\em label} \texttt{/html},
except that the {\sc html} code for the symbols is taken from
\texttt{althtmldef} statements instead of \texttt{htmldef} statements in
the \texttt{\$t} comment.

\vskip 1ex
    \texttt{althtmldef "}{\em math-token}\texttt{" as "}{\em
{\sc html}-code}\texttt{" ;}\index{\texttt{althtmldef} statement}

                    \ \ \ \ \ \ldots

    \texttt{althtmldef "}{\em math-token}\texttt{" as "}{\em
{\sc html}-code}\texttt{" ;}

\vskip 1ex
This feature is useful when
an alternate representation of symbols is desired, for example one that
uses Unicode entities instead of {\sc gif} images.  Associated with
\texttt{althtmldef}
are the statements
\vskip 1ex

    \texttt{htmldir "}{\em
      directoryname}\texttt{" ;}\index{\texttt{htmldir} statement}

    \texttt{althtmldir "}{\em
     directoryname}\texttt{" ;}\index{\texttt{althtmldir} statement}

\vskip 1ex
\noindent giving the directories of the {\sc gif} and Unicode versions
respectively; their purpose is to provide cross-linking between the
two versions in the generated web pages.

The command
\begin{verbatim}
     show statement * /brief_html
\end{verbatim}
invokes a special mode that just produces definition and theorem lists
accompanied by their symbol strings, in a format suitable for copying and
pasting into another web page (such as the tutorial pages on the
Metamath web site).

Finally, the command
\begin{verbatim}
     show statement * /brief_alt_html
\end{verbatim}
does the same as \texttt{show statement * / brief{\char`\_}html}
for the alternate {\sc html}
symbol representation.

When two different types of pages need to be produced from a single
database, such as the Hilbert Space Explorer that extends the Metamath
Proof Explorer, ``extended'' variables may be declared in the
\texttt{\$t} comment:
\vskip 1ex

    \texttt{exthtmltitle "}{\em {\sc html}-code}\texttt{"
    ;}\index{\texttt{exthtmltitle} statement}

    \texttt{exthtmlhome "}{\em {\sc html}-code}\texttt{"
    ;}\index{\texttt{exthtmlhome} statement}

    \texttt{exthtmlbibliography "}{\em filename}\texttt{"
    ;}\index{\texttt{exthtmlbibliography} statement}

\vskip 1ex
\noindent When these are declared, you also must declare
\vskip 1ex

    \texttt{exthtmllabel "}{\em label}\texttt{"
    ;}\index{\texttt{exthtmllabel} statement}

\vskip 1ex \noindent that identifies the database statement where the
``extended'' section of the database starts (in our example, where the
Hilbert Space Explorer starts).  During the generation of web pages for
that starting statement and the statements after it, the {\sc html} code
assigned to \texttt{exthtmltitle} and \texttt{exthtmlhome} is used
instead of that assigned to \texttt{htmltitle} and \texttt{htmlhome},
respectively.

If you want to become familiar with these features, you should study
them in conjunction with the
\texttt{set.mm}\index{set theory database (\texttt{set.mm})}
database example, in order
to understand the details that aren't precisely specified above, such as
exactly what the {\sc html} code snippets should look like.



\subsection{\texttt{write theorem\_list}
Command}\index{\texttt{write theorem{\char`\_}list} command}
Syntax:  \texttt{write theorem{\char`\_}list}
[\texttt{/theorems{\char`\_}per{\char`\_}page} {\em number}]

This command writes a list of all of the \texttt{\$a} and \texttt{\$p}
statements in the database into a web page file
 called \texttt{mmtheorems.html}.
When additional files are needed, they are called
\texttt{mmtheorems2.html}, \texttt{mmtheorems3.html}, etc.

Optional command qualifier:

    \texttt{/theorems{\char`\_}per{\char`\_}page} {\em number} -
 This qualifier specifies the number of statements to
        write per web page.  The default is 100.

{\em Note:} In version 0.07.30\index{Metamath!limitations of version
0.07.30} of Metamath, the ``Related Theorems'' links on the individual
web pages presuppose 100 theorems per page when linking to the theorem
list pages.  Therefore the \texttt{/theorems{\char`\_}per{\char`\_}page}
qualifier, if it specifies a number other than 100, will cause the
individual web pages to be out of sync and should not be used to
generate the main theorem list for the web site.  This is expected to be
fixed in a future version.


\subsection{\texttt{write bibliography}\label{wrbib}
Command}\index{\texttt{write bibliography} command}
Syntax:  \texttt{write bibliography} {\em filename}

This command reads an existing {\sc html} bibliographic cross-reference
file, normally called \texttt{mmbiblio.html}, and updates it per the
bibliographic links in the database comments.  The file is updated
between the {\sc html} comment lines \texttt{<!--
{\char`\#}START{\char`\#} -->} and \texttt{<!-- {\char`\#}END{\char`\#}
-->}.  The original input file is renamed to {\em
filename}\texttt{{\char`\~}1}.

A bibliographic reference is indicated with the reference name
in brackets, such as  \texttt{Theorem 3.1 of
[Monk] p.\ 22}.
See Section~\ref{htmlmkup} (p.~\pageref{htmlmkup}) for
syntax details.


\subsection{\texttt{write recent\_additions}
Command}\index{\texttt{write recent{\char`\_}additions} command}
Syntax:  \texttt{write recent{\char`\_}additions} {\em filename}
[\texttt{/limit} {\em number}]

This command reads an existing ``Recent Additions'' {\sc html} file,
normally called \texttt{mmrecent.html}, and updates it with the
descriptions of the most recently added theorems to the database.
 The file is updated between
the {\sc html} comment lines \texttt{<!-- {\char`\#}START{\char`\#} -->}
and \texttt{<!-- {\char`\#}END{\char`\#} -->}.  The original input file
is renamed to {\em filename}\texttt{{\char`\~}1}.

Optional command qualifier:

    \texttt{/limit} {\em number} -
 This qualifier specifies the number of most recent theorems to
   write to the output file.  The default is 100.


\section{Text File Utilities}

\subsection{\texttt{tools} Command}\index{\texttt{tools} command}
Syntax:  \texttt{tools}

This command invokes an easy-to-use, general purpose utility for
manipulating the contents of {\sc ascii} text files.  Upon typing
\texttt{tools}, the command-line prompt will change to \texttt{TOOLS>}
until you type \texttt{exit}.  The \texttt{tools} commands can be used
to perform simple, global edits on an input/output file,
such as making a character string substitution on each line, adding a
string to each line, and so on.  A typical use of this utility is
to build a \texttt{submit} input file to perform a common operation on a
list of statements obtained from \texttt{show label} or \texttt{show
usage}.

The actions of most of the \texttt{tools} commands can also be
performed with equivalent (and more powerful) Unix shell commands, and
some users may find those more efficient.  But for Windows users or
users not comfortable with Unix, \texttt{tools} provides an
easy-to-learn alternative that is adequate for most of the
script-building tasks needed to use the Metamath program effectively.

\subsection{\texttt{help} Command (in \texttt{tools})}
Syntax:  \texttt{help}

The \texttt{help} command lists the commands available in the
\texttt{tools} utility, along with a brief description.  Each command,
in turn, has its own help, such as \texttt{help add}.  As with
Metamath's \texttt{MM>} prompt, a complete command can be entered at
once, or just the command word can be typed, causing the program to
prompt for each argument.

\vskip 1ex
\noindent Line-by-line editing commands:

  \texttt{add} - Add a specified string to each line in a file.

  \texttt{clean} - Trim spaces and tabs on each line in a file; convert
         characters.

  \texttt{delete} - Delete a section of each line in a file.

  \texttt{insert} - Insert a string at a specified column in each line of
        a file.

  \texttt{substitute} - Make a simple substitution on each line of the file.

  \texttt{tag} - Like \texttt{add}, but restricted to a range of lines.

  \texttt{swap} - Swap the two halves of each line in a file.

\vskip 1ex
\noindent Other file processing commands:

  \texttt{break} - Break up (parse) a file into a list of tokens (one per
        line).

  \texttt{build} - Build a file with multiple tokens per line from a list.

  \texttt{count} - Count the occurrences in a file of a specified string.

  \texttt{number} - Create a list of numbers.

  \texttt{parallel} - Put two files in parallel.

  \texttt{reverse} - Reverse the order of the lines in a file.

  \texttt{right} - Right-justify lines in a file (useful before sorting
         numbers).

%  \texttt{tag} - Tag edit updates in a program for revision control.

  \texttt{sort} - Sort the lines in a file with key starting at
         specified string.

  \texttt{match} - Extract lines containing (or not) a specified string.

  \texttt{unduplicate} - Eliminate duplicate occurrences of lines in a file.

  \texttt{duplicate} - Extract first occurrence of any line occurring
         more than

   \ \ \    once in a file, discarding lines occurring exactly once.

  \texttt{unique} - Extract lines occurring exactly once in a file.

  \texttt{type} (10 lines) - Display the first few lines in a file.
                  Similar to Unix \texttt{head}.

  \texttt{copy} - Similar to Unix \texttt{cat} but safe (same input
         and output file allowed).

  \texttt{submit} - Run a script containing \texttt{tools} commands.

\vskip 1ex

\noindent Note:
  \texttt{unduplicate}, \texttt{duplicate}, and \texttt{unique} also
 sort the lines as a side effect.


\subsection{Using \texttt{tools} to Build Metamath \texttt{submit}
Scripts}

The \texttt{break} command is typically used to break up a series of
statement labels, such as the output of Metamath's \texttt{show usage},
into one label per line.  The other \texttt{tools} commands can then be
used to add strings before and after each statement label to specify
commands to be performed on the statement.  The \texttt{parallel}
command is useful when a statement label must be mentioned more than
once on a line.

Very often a \texttt{submit} script for Metamath will require multiple
command lines for each statement being processed.  For example, you may
want to enter the Proof Assistant, \texttt{minimize{\char`\_}with} your
latest theorem, \texttt{save} the new proof, and \texttt{exit} the Proof
Assistant.  To accomplish this, you can build a file with these four
commands for each statement on a single line, separating each command
with a designated character such as \texttt{@}.  Then at the end you can
\texttt{substitute} each \texttt{@} with \texttt{{\char`\\}n} to break
up the lines into individual command lines (see \texttt{help
substitute}).


\subsection{Example of a \texttt{tools} Session}

To give you a quick feel for the \texttt{tools} utility, we show a
simple session where we create a file \texttt{n.txt} with 3 lines, add
strings before and after each line, and display the lines on the screen.
You can experiment with the various commands to gain experience with the
\texttt{tools} utility.

\begin{verbatim}
    MM> tools
    Entering the Text Tools utilities.
    Type HELP for help, EXIT to exit.
    TOOLS> number
    Output file <n.tmp>? n.txt
    First number <1>?
    Last number <10>? 3
    Increment <1>?
    TOOLS> add
    Input/output file? n.txt
    String to add to beginning of each line <>? This is line
    String to add to end of each line <>? .
    The file n.txt has 3 lines; 3 were changed.
    First change is on line 1:
    This is line 1.
    TOOLS> type n.txt
    This is line 1.
    This is line 2.
    This is line 3.
    TOOLS> exit
    Exiting the Text Tools.
    Type EXIT again to exit Metamath.
    MM>
\end{verbatim}



\appendix
\chapter{Math Symbol Tokens for Set Theory}
\label{ASCII}

This Appendix lists the tokens (math symbols) used for basic set theory
development (up to complex numbers) in the \texttt{set.mm} database, in
order of appearance.  Next to each token we show the mathematical symbol
that corresponds to it.  The
\texttt{set.mm}\index{set theory database (\texttt{set.mm})} file has an
explanation of the meaning of each symbol.  (New definitions are added
to \texttt{set.mm} over time.  Consult the latest version at
\url{http://metamath.org} for the most recent additions.)

These symbols are defined in the \texttt{set.mm} database file inside of
a special comment, which is indicated by the appearance of the
two-character string \texttt{\$t} at the beginning of the comment
(see Section~\ref{tcomment}, p.~\pageref{tcomment}).  This
special comment is called a \texttt{\$t} {\em
comment}\index{\texttt{\$t} comment} or {\em typesetting
comment.}\index{typesetting comment}

The definitions in the \texttt{\$t} comment are referenced
by the \texttt{write tex} command to produce \LaTeX\ output.  If you add a new
token to the set theory database (or your own database), you should also
update the \texttt{\$t} comment if you want to create a \LaTeX\
output file.  The \texttt{\$t} comment is not needed for normal
operation of the Metamath program, but is referenced only when you open a
\LaTeX\ output file with the \texttt{open tex} command.  The
\texttt{\$t} comment consists of a series of \LaTeX\ definitions with
the following syntax:
\begin{center}
 \texttt{latexdef} {\em token-string} \texttt{as} {\em latex-string}
 \texttt{;}
\end{center}
\index{latex definitions@\LaTeX\ definitions}%
\index{\texttt{latexdef} statement}
The fields are separated by white space (blanks, carriage returns,
etc.), although white space is not needed before the \texttt{;}
terminator.  Each definition should start on a new line.\footnote{This
restriction of the current version of Metamath
(0.07.30)\index{Metamath!limitations of version 0.07.30} may be removed
in a future version, but you should do it anyway for readability.} For
example,
\begin{center}
 \verb$latexdef "(_" as "\subseteq";$
\end{center}
defines the token \verb$(_$ as the \LaTeX\ symbol $\subseteq$ (which means
``subset'').

The {\em token-string} and {\em latex-string} are the character strings for
the token and the \LaTeX\ definition of the token, respectively, enclosed in
either double (\texttt{\char`\"}) or single (\texttt{'}) quotation
marks.  {\em The string enclosed in quotation marks may not include
line breaks.}  A {\em token-string} or {\em latex-string}
may include a quotation mark that matches the enclosing quotes by repeating
the quotation mark twice; for example the {\em token-string}\,s
\begin{center}
 \texttt{\char`\"a\char`\"\char`\"b\char`\"}\\
 \texttt{'c''d'}\\
 \texttt{\char`\"e''f\char`\"}\\
 \texttt{'g\char`\"\char`\"h'}
\end{center}
specify the tokens \texttt{a\char`\"b}, \texttt{c'd}, \texttt{e''f}, and
\texttt{g\char`\"\char`\"h} respectively.  Finally, a long {\em
latex-string} may be broken up into multiple quote-enclosed strings
joined by \texttt{\char`\+} in order to fit them on several lines; thus
\begin{center}
 \texttt{\char`\"ab\char`\"\ \char`\+\ \char`\"cd\char`\"
    \ \char`\+\ 'ef'}
\end{center}
is the same as
\begin{center}
 \texttt{\char`\"abcdef\char`\"}
\end{center}

The \texttt{\$t} comment may also contain \texttt{htmldef}
statements,\index{\texttt{htmldef} statement} \texttt{althtmldef}
statements,\index{\texttt{althtmldef} statement} and some other types of
statements related to the generation of {\sc html} pages for the
Metamath web site.\index{html generation@{\sc html} generation} The
syntax for these is not yet final, and for current information you
should consult \texttt{help html} in the Metamath program.  Importantly,
whenever you add a \texttt{latexdef} statement, you should also add a
new \texttt{htmldef} statement and a new \texttt{althtmldef} statement
to keep all symbol definitions up to date.

%\begin{quote}
%  {\em
%As you crack your eyes one morning your reason is assaulted by
%a strange sight.  Over your head, humming quietly, there floats
%a monitor, an ethereal otherworldly screen on which is written a
%curious message.  ``I am the Screen of ultimate Truth.  I am bulging
%with information and ask nothing better than to be allowed to impart
%it.''
%}
%  \flushright\sc  John Baez\footnote{John Baez ({\em baez @ guitar.ucr.edu}),
%Internet newsgroup {\em sci.math}, 1995}\\
%\end{quote}\index{Baez, John}



%\input{metamaths.tex}
% Start of metamaths.tex
\newpage
{\samepage
\begin{center}
\begin{tabular}
{|l|lr||l|lr||l|lr|}
\hline
{\small Token}&\multicolumn{2}{l||}{\small Symbol}&
{\small Token}&\multicolumn{2}{l||}{\small Symbol}&
{\small Token}&\multicolumn{2}{l|}{\small Symbol}\\
 \hline \hline
\verb$($&\multicolumn{2}{l||}{$($}&
\verb$B$&\multicolumn{2}{l||}{$B$}&
\verb$dom$&\multicolumn{2}{l|}{${\rm dom}$}\\ \hline
\verb$)$&\multicolumn{2}{l||}{$)$}&
\verb$C$&\multicolumn{2}{l||}{$C$}&
\verb$ran$&\multicolumn{2}{l|}{${\rm ran}$}\\ \hline
\verb$->$&\multicolumn{2}{l||}{$\rightarrow$}&
\verb$D$&\multicolumn{2}{l||}{$D$}&
\verb$|`$&\multicolumn{2}{l|}{$\restriction$}\\ \hline
\verb$-.$&\multicolumn{2}{l||}{$\lnot$}&
\verb$R$&\multicolumn{2}{l||}{$R$}&
\verb$"$&\multicolumn{2}{l|}{$``$}\\ \hline
\verb$wff$&\multicolumn{2}{l||}{${\rm wff}$}&
\verb$S$&\multicolumn{2}{l||}{$S$}&
\verb$o.$&\multicolumn{2}{l|}{$\circ$}\\ \hline
\verb$|-$&\multicolumn{2}{l||}{$\vdash$}&
\verb$=/=$&\multicolumn{2}{l||}{$\ne$}&
\verb$Rel$&\multicolumn{2}{l|}{${\rm Rel}$}\\ \hline
\verb$ph$&\multicolumn{2}{l||}{$\varphi$}&
\verb$e/$&\multicolumn{2}{l||}{$\notin$}&
\verb/Fun/&\multicolumn{2}{l|}{${\rm Fun}$}\\ \hline
\verb$ps$&\multicolumn{2}{l||}{$\psi$}&
\verb$V$&\multicolumn{2}{l||}{$V$}&
\verb/Fn/&\multicolumn{2}{l|}{${\rm Fn}$}\\ \hline
\verb$ch$&\multicolumn{2}{l||}{$\chi$}&
\verb/F/&\multicolumn{2}{l||}{$ F$}&
\verb$:$&\multicolumn{2}{l|}{$:$}\\ \hline
\verb/th/&\multicolumn{2}{l||}{$\theta$}&
\verb$G$&\multicolumn{2}{l||}{$G$}&
\verb$-->$&\multicolumn{2}{l|}{$\longrightarrow$}\\ \hline
\verb/ta/&\multicolumn{2}{l||}{$\tau$}&
\verb$(_$&\multicolumn{2}{l||}{$\subseteq$}&
\verb$-1-1->$&\multicolumn{2}{l|}{$
  \raisebox{.5ex}{${\textstyle{\:}_{\mbox{\footnotesize\rm 1
  \tt -\rm 1}}}\atop{\textstyle{
  \longrightarrow}\atop{\textstyle{}^{\mbox{\footnotesize\rm {\ }}}}}$}
  $}\\ \hline
\verb$<->$&\multicolumn{2}{l||}{$\leftrightarrow$}&
\verb$(.$&\multicolumn{2}{l||}{$\subset$}&
\verb$-onto->$&\multicolumn{2}{l|}{$
  \raisebox{.5ex}{${\textstyle{\:}_{\mbox{\footnotesize\rm {\ }}}}
  \atop{\textstyle{
  \longrightarrow}\atop{\textstyle{}^{\mbox{\footnotesize\rm onto}}}}$}
  $}\\ \hline
\verb$\/$&\multicolumn{2}{l||}{$\vee$}&
\verb$\$&\multicolumn{2}{l||}{$\setminus$}&
\verb$-1-1-onto->$&\multicolumn{2}{l|}{$
  \raisebox{.5ex}{${\textstyle{\:}_{\mbox{\footnotesize\rm 1
  \tt -\rm 1}}}\atop{\textstyle{
  \longrightarrow}\atop{\textstyle{}^{\mbox{\footnotesize\rm onto}}}}$}
  $}\\ \hline
\verb$/\$&\multicolumn{2}{l||}{$\wedge$}&
\verb$u.$&\multicolumn{2}{l||}{$\cup$}&
\verb$`$&\multicolumn{2}{l|}{$`$}\\ \hline
\verb$et$&\multicolumn{2}{l||}{$\eta$}&
\verb$i^i$&\multicolumn{2}{l||}{$\cap$}&
\verb/h/&\multicolumn{2}{l|}{$ h$}\\ \hline
\verb$ze$&\multicolumn{2}{l||}{$\zeta$}&
\verb$(/)$&\multicolumn{2}{l||}{$\varnothing$}&
\verb/H/&\multicolumn{2}{l|}{$ H$}\\ \hline
\verb$A.$&\multicolumn{2}{l||}{$\forall$}&
\verb$P~$&\multicolumn{2}{l||}{${\cal P}$}&
\verb$rec$&\multicolumn{2}{l|}{${\rm rec}$}\\ \hline
\verb$set$&\multicolumn{2}{l||}{${\rm set}$}&
\verb$,$&\multicolumn{2}{l||}{$,$}&
\verb$+_o$&\multicolumn{2}{l|}{$+_o$}\\ \hline
\verb$x$&\multicolumn{2}{l||}{$x$}&
\verb$<.$&\multicolumn{2}{l||}{$\langle$}&
\verb$._o$&\multicolumn{2}{l|}{$\cdot_o$}\\ \hline
\verb$y$&\multicolumn{2}{l||}{$y$}&
\verb$>.$&\multicolumn{2}{l||}{$\rangle$}&
\verb$^_o$&\multicolumn{2}{l|}{$\hat{\ }_o$}\\ \hline
\verb$z$&\multicolumn{2}{l||}{$z$}&
\verb$U.$&\multicolumn{2}{l||}{$\bigcup$}&
\verb$1_o$&\multicolumn{2}{l|}{$1_o$}\\ \hline
\verb$w$&\multicolumn{2}{l||}{$w$}&
\verb$|^|$&\multicolumn{2}{l||}{$\bigcap$}&
\verb$Er$&\multicolumn{2}{l|}{${\rm Er}$}\\ \hline
\verb$v$&\multicolumn{2}{l||}{$v$}&
\verb/Tr/&\multicolumn{2}{l||}{${\rm Tr}$}&
\verb$[$&\multicolumn{2}{l|}{$[$}\\ \hline
\verb$E.$&\multicolumn{2}{l||}{$\exists$}&
\verb$E$&\multicolumn{2}{l||}{$E$}&
\verb$]$&\multicolumn{2}{l|}{$]$}\\ \hline
\verb$=$&\multicolumn{2}{l||}{$=$}&
\verb$I$&\multicolumn{2}{l||}{$I$}&
\verb$/.$&\multicolumn{2}{l|}{$\diagup$}\\ \hline
\verb$e.$&\multicolumn{2}{l||}{$\in$}&
\verb$Po$&\multicolumn{2}{l||}{${\rm Po}$}&
\verb$Q$&\multicolumn{2}{l|}{$Q$}\\ \hline
\verb$|$&\multicolumn{2}{l||}{$|$}&
\verb$Or$&\multicolumn{2}{l||}{${\rm Or}$}&
\verb/t/&\multicolumn{2}{l|}{$ t$}\\ \hline
\verb$u$&\multicolumn{2}{l||}{$u$}&
\verb/Fr/&\multicolumn{2}{l||}{${\rm Fr}$}&
\verb$s$&\multicolumn{2}{l|}{$s$}\\ \hline
\verb/f/&\multicolumn{2}{l||}{$ f$}&
\verb$We$&\multicolumn{2}{l||}{${\rm We}$}&
\verb$r$&\multicolumn{2}{l|}{$r$}\\ \hline
\verb$g$&\multicolumn{2}{l||}{$g$}&
\verb$Ord$&\multicolumn{2}{l||}{${\rm Ord}$}&
\verb$a$&\multicolumn{2}{l|}{$a$}\\ \hline
\verb$E!$&\multicolumn{2}{l||}{$\exists{!}$}&
\verb$On$&\multicolumn{2}{l||}{${\rm On}$}&
\verb$b$&\multicolumn{2}{l|}{$b$}\\ \hline
\verb$E*$&\multicolumn{2}{l||}{$\exists^\ast$}&
\verb$Lim$&\multicolumn{2}{l||}{${\rm Lim}$}&
\verb$c$&\multicolumn{2}{l|}{$c$}\\ \hline
\verb${$&\multicolumn{2}{l||}{$\{$}&
\verb$suc$&\multicolumn{2}{l||}{${\rm suc}$}&
\verb$d$&\multicolumn{2}{l|}{$d$}\\ \hline
\verb$}$&\multicolumn{2}{l||}{$\}$}&
\verb$om$&\multicolumn{2}{l||}{$\omega$}&
\verb$j$&\multicolumn{2}{l|}{$j$}\\ \hline
\verb$class$&\multicolumn{2}{l||}{${\rm class}$}&
\verb$X.$&\multicolumn{2}{l||}{$\times$}&
\verb$k$&\multicolumn{2}{l|}{$k$}\\ \hline
\verb$A$&\multicolumn{2}{l||}{$A$}&
\verb$`'$&\multicolumn{2}{l||}{$\breve{\ }$}&
\verb$m$&\multicolumn{2}{l|}{$m$}\\ \hline
\end{tabular}
\end{center}
}% End of \samepage
\newpage
{\samepage
\begin{center}
\begin{tabular}
{|l|lr||l|lr||l|lr|}
\hline
{\small Token}&\multicolumn{2}{l||}{\small Symbol}&
{\small Token}&\multicolumn{2}{l||}{\small Symbol}&
{\small Token}&\multicolumn{2}{l|}{\small Symbol}\\
 \hline \hline
\verb$n$&\multicolumn{2}{l||}{$n$}&
\verb$+R$&\multicolumn{2}{l||}{$+_{\cal R}$}&
 & &\\ \hline
\verb$q$&\multicolumn{2}{l||}{$q$}&
\verb$.R$&\multicolumn{2}{l||}{$\cdot_{\cal R}$}&
 & &\\ \hline
\verb$J$&\multicolumn{2}{l||}{$J$}&
\verb$<R$&\multicolumn{2}{l||}{$<_{\cal R}$}&
 & &\\ \hline
\verb$K$&\multicolumn{2}{l||}{$K$}&
\verb$CC$&\multicolumn{2}{l||}{$\mathbb{C}$}&
 & &\\ \hline
\verb$L$&\multicolumn{2}{l||}{$L$}&
\verb$RR$&\multicolumn{2}{l||}{$\mathbb{R}$}&
 & &\\ \hline
\verb$M$&\multicolumn{2}{l||}{$M$}&
\verb$0$&\multicolumn{2}{l||}{$0$}&
 & &\\ \hline
\verb$N$&\multicolumn{2}{l||}{$N$}&
\verb$1$&\multicolumn{2}{l||}{$1$}&
 & &\\ \hline
\verb$W$&\multicolumn{2}{l||}{$W$}&
\verb$i$&\multicolumn{2}{l||}{$i$}&
 & &\\ \hline
\verb$X$&\multicolumn{2}{l||}{$X$}&
\verb$+$&\multicolumn{2}{l||}{$+$}&
 & &\\ \hline
\verb$Y$&\multicolumn{2}{l||}{$Y$}&
\verb$-$&\multicolumn{2}{l||}{$-$}&
 & &\\ \hline
\verb$Z$&\multicolumn{2}{l||}{$Z$}&
\verb$x.$&\multicolumn{2}{l||}{$\cdot$}&
 & &\\ \hline
\verb$N.$&\multicolumn{2}{l||}{${\cal N}$}&
\verb$<$&\multicolumn{2}{l||}{$<$}&
 & &\\ \hline
\verb$+N$&\multicolumn{2}{l||}{$+_{\cal N}$}&
\verb$/$&\multicolumn{2}{l||}{$/$}&
 & &\\ \hline
\verb$.N$&\multicolumn{2}{l||}{$\cdot_{\cal N}$}&
\verb$>$&\multicolumn{2}{l||}{$>$}&
 & &\\ \hline
\verb$<N$&\multicolumn{2}{l||}{$<_{\cal N}$}&
\verb$<_$&\multicolumn{2}{l||}{$\le$}&
 & &\\ \hline
\verb$+pQ$&\multicolumn{2}{l||}{$+_{p{\cal Q}}$}&
\verb$>_$&\multicolumn{2}{l||}{$\ge$}&
 & &\\ \hline
\verb$.pQ$&\multicolumn{2}{l||}{$\cdot_{p{\cal Q}}$}&
\verb$NN$&\multicolumn{2}{l||}{$\mathbb{N}$}&
 & &\\ \hline
\verb$~Q$&\multicolumn{2}{l||}{$\sim_{\cal Q}$}&
\verb$ZZ$&\multicolumn{2}{l||}{$\mathbb{Z}$}&
 & &\\ \hline
\verb$Q.$&\multicolumn{2}{l||}{${\cal Q}$}&
\verb$QQ$&\multicolumn{2}{l||}{$\mathbb{Q}$}&
 & &\\ \hline
\verb$1Q$&\multicolumn{2}{l||}{$1_{\cal Q}$}&
\verb$2$&\multicolumn{2}{l||}{$2$}&
 & &\\ \hline
\verb$+Q$&\multicolumn{2}{l||}{$+_{\cal Q}$}&
\verb$3$&\multicolumn{2}{l||}{$3$}&
 & &\\ \hline
\verb$.Q$&\multicolumn{2}{l||}{$\cdot_{\cal Q}$}&
\verb$4$&\multicolumn{2}{l||}{$4$}&
 & &\\ \hline
\verb$*Q$&\multicolumn{2}{l||}{$\ast_{\cal Q}$}&
\verb$5$&\multicolumn{2}{l||}{$5$}&
 & &\\ \hline
\verb$<Q$&\multicolumn{2}{l||}{$<_{\cal Q}$}&
\verb$6$&\multicolumn{2}{l||}{$6$}&
 & &\\ \hline
\verb$P.$&\multicolumn{2}{l||}{${\cal P}$}&
\verb$7$&\multicolumn{2}{l||}{$7$}&
 & &\\ \hline
\verb$1P$&\multicolumn{2}{l||}{$1_{\cal P}$}&
\verb$8$&\multicolumn{2}{l||}{$8$}&
 & &\\ \hline
\verb$+P$&\multicolumn{2}{l||}{$+_{\cal P}$}&
\verb$9$&\multicolumn{2}{l||}{$9$}&
 & &\\ \hline
\verb$.P$&\multicolumn{2}{l||}{$\cdot_{\cal P}$}&
\verb$ $&\multicolumn{2}{l||}{${}$}&  % was .
 & &\\ \hline
\verb$<P$&\multicolumn{2}{l||}{$<_{\cal P}$}&
\verb$ $&\multicolumn{2}{l||}{${}$}&  % was .
 & &\\ \hline
\verb$+pR$&\multicolumn{2}{l||}{$+_{p{\cal R}}$}&
\verb$ $&\multicolumn{2}{l||}{${}$}&   % was digit
 & &\\ \hline
\verb$.pR$&\multicolumn{2}{l||}{$._{p{\cal R}}$}&
\verb$ $&\multicolumn{2}{l||}{${}$}&  % was digit-string
 & &\\ \hline
\verb$~R$&\multicolumn{2}{l||}{$\sim_{\cal R}$}&
 & &&
 & &\\ \hline
\verb$R.$&\multicolumn{2}{l||}{${\cal R}$}&
 & &&
 & &\\ \hline
\verb$0R$&\multicolumn{2}{l||}{$0_{\cal R}$}&
 & &&
 & &\\ \hline
\verb$1R$&\multicolumn{2}{l||}{$1_{\cal R}$}&
 & &&
 & &\\ \hline
\verb$-1R$&\multicolumn{2}{l||}{$-1_{\cal R}$}&
 & &&
 & &\\ \hline
\end{tabular}
\end{center}
}% End of \samepage
% End of metamathl.tex

SAMPLE OF PROPOSED CHANGE:

This Appendix provides a sample of key tokens (math symbols)
used for basic set theory development in the \texttt{set.mm} database, in
order of appearance.
For each token we show the displayed mathematical symbol
that corresponds to it, the label of its (primary) definition,
and in some cases a description about it.
This is only a partial list, and new definitions are routinely added.
A complete list is available at \url{http://metamath.org}.

% NOTE: Each row is on its own line, making for long rows, so we can
% easily sort later if we want to sort by token.
% NOTE: Not put in a (floating) table, because this may be >1 page.
\noindent\begin{tabularx}{\textwidth}{@{} l l l X}
\textbf{Token} & \textbf{Symbol} & \textbf{Label} & \textbf{Description} \\
\texttt{-.} & $\lnot$ & \texttt{wn} & Logical not. E.g., if $\varphi$ is true, then $\lnot \varphi$ is false. \\
\texttt{->} & $\rightarrow$ & \texttt{wi} & Implies, also known as material implication. The expression $\varphi \rightarrow \psi$ is true if either $\varphi$ is false or $\psi$ is true (or both), that is, in classical logic  $\varphi \rightarrow \psi$ has the same meaning as $\lnot \varphi \lor \psi$ (as proven in \texttt{imor}). \\
\texttt{<->} & $\leftrightarrow$ & \texttt{df-bi} & Biconditional (aka is-equals for boolean values).  $\varphi \leftrightarrow \psi$ is true if and only if $\varphi$ and $\psi$ have the same value. \\
\texttt{\char`\\/} & $\lor$ & \texttt{df-or} & Disjunction (logical ``or''). $\varphi \lor \psi$ is true if and only if $\varphi$, $\psi$, or both are true. \\
\end{tabularx}

\chapter{Compressed Proofs}
\label{compressed}\index{compressed proof}\index{proof!compressed}

The proofs in the \texttt{set.mm} set theory database are stored in compressed
format for efficiency.  Normally you needn't concern yourself with the
compressed format, since you can display it with the usual proof display tools
in the Metamath program (\texttt{show proof}\ldots) or convert it to the normal
RPN proof format described in Section~\ref{proof} (with \texttt{save proof}
{\em label} \texttt{/normal}).  However for sake of completeness we describe the
format here and show how it maps to the normal RPN proof format.

A compressed proof, located between \texttt{\$=} and \texttt{\$.} keywords, consists
of a left parenthesis, a sequence of statement labels, a right parenthesis,
and a sequence of upper-case letters \texttt{A} through \texttt{Z} (with optional
white space between them).  White space must surround the parentheses
and the labels.  The left parenthesis tells Metamath that a
compressed proof follows.  (A normal RPN proof consists of just a sequence of
labels, and a parenthesis is not a legal character in a label.)

The sequence of upper-case letters corresponds to a sequence of integers
with the following mapping.  Each integer corresponds to a proof step as
described later.
\begin{center}
  \texttt{A} = 1 \\
  \texttt{B} = 2 \\
   \ldots \\
  \texttt{T} = 20 \\
  \texttt{UA} = 21 \\
  \texttt{UB} = 22 \\
   \ldots \\
  \texttt{UT} = 40 \\
  \texttt{VA} = 41 \\
  \texttt{VB} = 42 \\
   \ldots \\
  \texttt{YT} = 120 \\
  \texttt{UUA} = 121 \\
   \ldots \\
  \texttt{YYT} = 620 \\
  \texttt{UUUA} = 621 \\
   etc.
\end{center}

In other words, \texttt{A} through \texttt{T} represent the
least-significant digit in base 20, and \texttt{U} through \texttt{Y}
represent zero or more most-significant digits in base 5, where the
digits start counting at 1 instead of the usual 0. With this scheme, we
don't need white space between these ``numbers.''

(In the design of the compressed proof format, only upper case letters,
as opposed to say all printable {\sc ascii} characters other than
%\texttt{\$}, was chosen to make the compressed proof a little less
%displeasing to the eye, at the expense of a typical 20\% compression
\texttt{\$}, were chosen so as not to collide with most text editor
searches, at the expense of a typical 20\% compression
loss.  The base 5/base 20 grouping, as opposed to say base 6/base 19,
was chosen by experimentally determining the grouping that resulted in
best typical compression.)

The letter \texttt{Z} identifies (tags) a proof step that is identical to one
that occurs later on in the proof; it helps shorten the proof by not requiring
that identical proof steps be proved over and over again (which happens often
when building wff's).  The \texttt{Z} is placed immediately after the
least-significant digit (letters \texttt{A} through \texttt{T}) that ends the integer
corresponding to the step to later be referenced.

The integers that the upper-case letters correspond to are mapped to labels as
follows.  If the statement being proved has $m$ mandatory hypotheses, integers
1 through $m$ correspond to the labels of these hypotheses in the order shown
by the \texttt{show statement ... / full} command, i.e., the RPN order\index{RPN
order} of the mandatory
hypotheses.  Integers $m+1$ through $m+n$ correspond to the labels enclosed in
the parentheses of the compressed proof, in the order that they appear, where
$n$ is the number of those labels.  Integers $m+n+1$ on up don't directly
correspond to statement labels but point to proof steps identified with the
letter \texttt{Z}, so that these proof steps can be referenced later in the
proof.  Integer $m+n+1$ corresponds to the first step tagged with a \texttt{Z},
$m+n+2$ to the second step tagged with a \texttt{Z}, etc.  When the compressed
proof is converted to a normal proof, the entire subproof of a step tagged
with \texttt{Z} replaces the reference to that step.

For efficiency, Metamath works with compressed proofs directly, without
converting them internally to normal proofs.  In addition to the usual
error-checking, an error message is given if (1) a label in the label list in
parentheses does not refer to a previous \texttt{\$p} or \texttt{\$a} statement or a
non-mandatory hypothesis of the statement being proved and (2) a proof step
tagged with \texttt{Z} is referenced before the step tagged with the \texttt{Z}.

Just as in a normal proof under development (Section~\ref{unknown}), any step
or subproof that is not yet known may be represented with a single \texttt{?}.
White space does not have to appear between the \texttt{?}\ and the upper-case
letters (or other \texttt{?}'s) representing the remainder of the proof.

% April 1, 2004 Appendix C has been added back in with corrections.
%
% May 20, 2003 Appendix C was removed for now because there was a problem found
% by Bob Solovay
%
% Also, removed earlier \ref{formalspec} 's (3 cases above)
%
% Bob Solovay wrote on 30 Nov 2002:
%%%%%%%%%%%%% (start of email comment )
%      3. My next set of comments concern appendix C. I read this before I
% read Chapter 4. So I first noted that the system as presented in the
% Appendix lacked a certain formal property that I thought desirable. I
% then came up with a revised formal system that had this property. Upon
% reading Chapter 4, I noticed that the revised system was closer to the
% treatment in Chapter 4 than the system in Appendix C.
%
%         First a very minor correction:
%
%         On page 142 line 2: The condition that V(e) != V(f) should only be
% required of e, f in T such that e != f.
%
%         Here is a natural property [transitivity] that one would like
% the formal system to have:
%
%         Let Gamma be a set of statements. Suppose that the statement Phi
% is provable from Gamma and that the statement Psi is provable from Gamma
% \cup {Phi}. Then Psi is provable from Gamma.
%
%         I shall present an example to show that this property does not
% hold for the formal systems of Appendix C:
%
%         I write the example in metamath style:
%
% $c A B C D E $.
% $v x y
%
% ${
% tx $f A x $.
% ty $f B y $.
% ax1 $a C x y $.
% $}
%
% ${
% tx $f A x $.
% ty $f B y $.
% ax2-h1 $e C x y $.
% ax2 $a D y $.
% $}
%
% ${
% ty $f B y $.
% ax3-h1 $e D y $.
% ax3 $a E y $.
% $}
%
% $(These three axioms are Gamma $)
%
% ${
% tx $f A x $.
% ty $f B y $.
% Phi $p D y $=
% tx ty tx ty ax1 ax2 $.
% $}
%
% ${
% ty $f B y $.
% Psi $p E y $=
% ty ty Phi ax3 $.
% $}
%
%
% I omit the formal proofs of the following claims. [I will be glad to
% supply them upon request.]
%
% 1) Psi is not provable from Gamma;
%
% 2) Psi is provable from Gamma + Phi.
%
% Here "provable" refers to the formalism of Appendix C.
%
% The trouble of course is that Psi is lacking the variable declaration
%
% $f Ax $.
%
% In the Metamath system there is no trouble proving Psi. I attach a
% metamath file that shows this and which has been checked by the
% metamath program.
%
% I next want to indicate how I think the treatment in Appendix C should
% be revised so as to conform more closely to the metamath system of the
% main text. The revised system *does* have the transitivity property.
%
% We want to give revised definitions of "statement" and
% "provable". [cf. sections C.2.4. and C.2.5] Our new definitions will
% use the definitions given in Appendix C. So we take the following
% tack. We refer to the original notions as o-statement and o-provable. And
% we refer to the notions we are defining as n-statement and n-provable.
%
%         A n-statement is an o-statement in which the only variables
% that appear in the T component are mandatory.
%
%         To any o-statement we can associate its reduct which is a
% n-statement by dropping all the elements of T or D which contain
% non-mandatory variables.
%
%         An n-statement gamma is n-provable if there is an o-statement
% gamma' which has gamma as its reduct andf such that gamma' is
% o-provable.
%
%         It seems to me [though I am not completely sure on this point]
% that n-provability corresponds to metamath provability as discussed
% say in Chapter 4.
%
%         Attached to this letter is the metamath proof of Phi and Psi
% from Gamma discussed above.
%
%         I am still brooding over the question of whether metamath
% correctly formalizes set-theory. No doubt I will have some questions
% re this after my thoughts become clearer.
%%%%%%%%%%%%%%%% (end of email comment)

%%%%%%%%%%%%%%%% (start of 2nd email comment from Bob Solovay 1-Apr-04)
%
%         I hope that Appendix C is the one that gives a "formal" treatment
% of Metamath. At any rate, thats the appendix I want to comment on.
%
%         I'm going to suggest two changes in the definition.
%
%         First change (in the definition of statement): Require that the
% sets D, T, and E be finite.
%
%         Probably things are fine as you give them. But in the applications
% to the main metamath system they will always be finite, and its useful in
% thinking about things [at least for me] to stick to the finite case.
%
%         Second change:
%
%         First let me give an approximate description. Remove the dummy
% variables from the statement. Instead, include them in the proof.
%
%         More formally: Require that T consists of type declarations only
% for mandatory variables. Require that all the pairs in D consist of
% mandatory variables.
%
%         At the start of a proof we are allowed to declare a finite number
% of dummy variables [provided that none of them appear in any of the
% statements in E \cup {A}. We have to supply type declarations for all the
% dummy variables. We are allowed to add new $d statements referring to
% either the mandatory or dummy variables. But we require that no new $d
% statement references only mandatory variables.
%
%         I find this way of doing things more conceptual than the treatment
% in Appendix C. But the change [which I will use implicitly in later
% letters about doing Peano] is mainly aesthetic. I definitely claim that my
% results on doing Peano all apply to Metamath as it is presented in your
% book.
%
%         --Bob
%
%%%%%%%%%%%%%%%% (end of 2nd email comment)

%%
%% When uncommenting the below, also uncomment references above to {formalspec}
%%
\chapter{Metamath's Formal System}\label{formalspec}\index{Metamath!as a formal
system}

\section{Introduction}

\begin{quote}
  {\em Perfection is when there is no longer anything more to take away.}
    \flushright\sc Antoine de
     Saint-Exupery\footnote{\cite[p.~3-25]{Campbell}.}\\
\end{quote}\index{de Saint-Exupery, Antoine}

This appendix describes the theory behind the Metamath language in an abstract
way intended for mathematicians.  Specifically, we construct two
set-theo\-ret\-i\-cal objects:  a ``formal system'' (roughly, a set of syntax
rules, axioms, and logical rules) and its ``universe'' (roughly, the set of
theorems derivable in the formal system).  The Metamath computer language
provides us with a way to describe specific formal systems and, with the aid of
a proof provided by the user, to verify that given theorems
belong to their universes.

To understand this appendix, you need a basic knowledge of informal set theory.
It should be sufficient to understand, for example, Ch.\ 1 of Munkres' {\em
Topology} \cite{Munkres}\index{Munkres, James R.} or the
introductory set theory chapter
in many textbooks that introduce abstract mathematics. (Note that there are
minor notational differences among authors; e.g.\ Munkres uses $\subset$ instead
of our $\subseteq$ for ``subset.''  We use ``included in'' to mean ``a subset
of,'' and ``belongs to'' or ``is contained in'' to mean ``is an element of.'')
What we call a ``formal'' description here, unlike earlier, is actually an
informal description in the ordinary language of mathematicians.  However we
provide sufficient detail so that a mathematician could easily formalize it,
even in the language of Metamath itself if desired.  To understand the logic
examples at the end of this appendix, familiarity with an introductory book on
mathematical logic would be helpful.

\section{The Formal Description}

\subsection[Preliminaries]{Preliminaries\protect\footnotemark}%
\footnotetext{This section is taken mostly verbatim
from Tarski \cite[p.~63]{Tarski1965}\index{Tarski, Alfred}.}

By $\omega$ we denote the set of all natural numbers (non-negative integers).
Each natural number $n$ is identified with the set of all smaller numbers: $n =
\{ m | m < n \}$.  The formula $m < n$ is thus equivalent to the condition: $m
\in n$ and $m,n \in \omega$. In particular, 0 is the number zero and at the
same time the empty set $\varnothing$, $1=\{0\}$, $2=\{0,1\}$, etc. ${}^B A$
denotes the set of all functions on $B$ to $A$ (i.e.\ with domain $B$ and range
included in $A$).  The members of ${}^\omega A$ are what are called {\em simple
infinite sequences},\index{simple infinite sequence}
with all {\em terms}\index{term} in $A$.  In case $n \in \omega$, the
members of ${}^n A$ are referred to as {\em finite $n$-termed
sequences},\index{finite $n$-termed
sequence} again
with terms in $A$.  The consecutive terms (function values) of a finite or
infinite sequence $f$ are denoted by $f_0, f_1, \ldots ,f_n,\ldots$.  Every
finite sequence $f \in \bigcup _{n \in \omega} {}^n A$ uniquely determines the
number $n$ such that $f \in {}^n A$; $n$ is called the {\em
length}\index{length of a sequence ({$"|\ "|$})} of $f$ and
is denoted by $|f|$.  $\langle a \rangle$ is the sequence $f$ with $|f|=1$ and
$f_0=a$; $\langle a,b \rangle$ is the sequence $f$ with $|f|=2$, $f_0=a$,
$f_1=b$; etc.  Given two finite sequences $f$ and $g$, we denote by $f\frown g$
their {\em concatenation},\index{concatenation} i.e., the
finite sequence $h$ determined by the
conditions:
\begin{eqnarray*}
& |h| = |f|+|g|;&  \\
& h_n = f_n & \mbox{\ for\ } n < |f|;  \\
& h_{|f|+n} = g_n & \mbox{\ for\ } n < |g|.
\end{eqnarray*}

\subsection{Constants, Variables, and Expressions}

A formal system has a set of {\em symbols}\index{symbol!in
a formal system} denoted
by $\mbox{\em SM}$.  A
precise set-theo\-ret\-i\-cal definition of this set is unimportant; a symbol
could be considered a primitive or atomic element if we wish.  We assume this
set is divided into two disjoint subsets:  a set $\mbox{\em CN}$ of {\em
constants}\index{constant!in a formal system} and a set $\mbox{\em VR}$ of
{\em variables}.\index{variable!in a formal system}  $\mbox{\em CN}$ and
$\mbox{\em VR}$ are each assumed to consist of countably many symbols which
may be arranged in finite or simple infinite sequences $c_0, c_1, \ldots$ and
$v_0, v_1, \ldots$ respectively, without repeating terms.  We will represent
arbitrary symbols by metavariables $\alpha$, $\beta$, etc.

{\footnotesize\begin{quotation}
{\em Comment.} The variables $v_0, v_1, \ldots$ of our formal system
correspond to what are usually considered ``metavariables'' in
descriptions of specific formal systems in the literature.  Typically,
when describing a specific formal system a book will postulate a set of
primitive objects called variables, then proceed to describe their
properties using metavariables that range over them, never mentioning
again the actual variables themselves.  Our formal system does not
mention these primitive variable objects at all but deals directly with
metavariables, as its primitive objects, from the start.  This is a
subtle but key distinction you should keep in mind, and it makes our
definition of ``formal system'' somewhat different from that typically
found in the literature.  (So, the $\alpha$, $\beta$, etc.\ above are
actually ``metametavariables'' when used to represent $v_0, v_1,
\ldots$.)
\end{quotation}}

Finite sequences all terms of which are symbols are called {\em
expressions}.\index{expression!in a formal system}  $\mbox{\em EX}$ is
the set of all expressions; thus
\begin{displaymath}
\mbox{\em EX} = \bigcup _{n \in \omega} {}^n \mbox{\em SM}.
\end{displaymath}

A {\em constant-prefixed expression}\index{constant-prefixed expression}
is an expression of non-zero length
whose first term is a constant.  We denote the set of all constant-prefixed
expressions by $\mbox{\em EX}_C = \{ e \in \mbox{\em EX} | ( |e| > 0 \wedge
e_0 \in \mbox{\em CN} ) \}$.

A {\em constant-variable pair}\index{constant-variable pair}
is an expression of length 2 whose first term
is a constant and whose second term is a variable.  We denote the set of all
constant-variable pairs by $\mbox{\em EX}_2 = \{ e \in \mbox{\em EX}_C | ( |e|
= 2 \wedge e_1 \in \mbox{\em VR} ) \}$.


{\footnotesize\begin{quotation}
{\em Relationship to Metamath.} In general, the set $\mbox{\em SM}$
corresponds to the set of declared math symbols in a Metamath database, the
set $\mbox{\em CN}$ to those declared with \texttt{\$c} statements, and the set
$\mbox{\em VR}$ to those declared with \texttt{\$v} statements.  Of course a
Metamath database can only have a finite number of math symbols, whereas
formal systems in general can have an infinite number, although the number of
Metamath math symbols available is in principle unlimited.

The set $\mbox{\em EX}_C$ corresponds to the set of permissible expressions
for \texttt{\$e}, \texttt{\$a}, and \texttt{\$p} statements.  The set $\mbox{\em EX}_2$
corresponds to the set of permissible expressions for \texttt{\$f} statements.
\end{quotation}}

We denote by ${\cal V}(e)$ the set of all variables in an expression $e \in
\mbox{\em EX}$, i.e.\ the set of all $\alpha \in \mbox{\em VR}$ such that
$\alpha = e_n$ for some $n < |e|$.  We also denote (with abuse of notation) by
${\cal V}(E)$ the set of all variables in a collection of expressions $E
\subseteq \mbox{\em EX}$, i.e.\ $\bigcup _{e \in E} {\cal V}(e)$.


\subsection{Substitution}

Given a function $F$ from $\mbox{\em VR}$ to
$\mbox{\em EX}$, we
denote by $\sigma_{F}$ or just $\sigma$ the function from $\mbox{\em EX}$ to
$\mbox{\em EX}$ defined recursively for nonempty sequences by
\begin{eqnarray*}
& \sigma(<\alpha>) = F(\alpha) & \mbox{for\ } \alpha \in \mbox{\em VR}; \\
& \sigma(<\alpha>) = <\alpha> & \mbox{for\ } \alpha \not\in \mbox{\em VR}; \\
& \sigma(g \frown h) = \sigma(g) \frown
    \sigma(h) & \mbox{for\ } g,h \in \mbox{\em EX}.
\end{eqnarray*}
We also define $\sigma(\varnothing)=\varnothing$.  We call $\sigma$ a {\em
simultaneous substitution}\index{substitution!variable}\index{variable
substitution} (or just {\em substitution}) with {\em substitution
map}\index{substitution map} $F$.

We also denote (with abuse of notation) by $\sigma(E)$ a substitution on a
collection of expressions $E \subseteq \mbox{\em EX}$, i.e.\ the set $\{
\sigma(e) | e \in E \}$.  The collection $\sigma(E)$ may of course contain
fewer expressions than $E$ because duplicate expressions could result from the
substitution.

\subsection{Statements}

We denote by $\mbox{\em DV}$ the set of all
unordered pairs $\{\alpha, \beta \} \subseteq \mbox{\em VR}$ such that $\alpha
\neq \beta$.  $\mbox{\em DV}$ stands for ``distinct variables.''

A {\em pre-statement}\index{pre-statement!in a formal system} is a
quadruple $\langle D,T,H,A \rangle$ such that
$D\subseteq \mbox{\em DV}$, $T\subseteq \mbox{\em EX}_2$, $H\subseteq
\mbox{\em EX}_C$ and $H$ is finite,
$A\in \mbox{\em EX}_C$, ${\cal V}(H\cup\{A\}) \subseteq
{\cal V}(T)$, and $\forall e,f\in T {\ } {\cal V}(e) \neq {\cal V}(f)$ (or
equivalently, $e_1 \ne f_1$) whenever $e \neq f$. The terms of the quadruple are called {\em
distinct-variable restrictions},\index{disjoint-variable restriction!in a
formal system} {\em variable-type hypotheses},\index{variable-type
hypothesis!in a formal system} {\em logical hypotheses},\index{logical
hypothesis!in a formal system} and the {\em assertion}\index{assertion!in a
formal system} respectively.  We denote by $T_M$ ({\em mandatory variable-type
hypotheses}\index{mandatory variable-type hypothesis!in a formal system}) the
subset of $T$ such that ${\cal V}(T_M) ={\cal V}(H \cup \{A\})$.  We denote by
$D_M=\{\{\alpha,\beta\}\in D|\{\alpha,\beta\}\subseteq {\cal V}(T_M)\}$ the
{\em mandatory distinct-variable restrictions}\index{mandatory
disjoint-variable restriction!in a formal system} of the pre-statement.
The set
of {\em mandatory hypotheses}\index{mandatory hypothesis!in a formal system}
is $T_M\cup H$.  We call the quadruple $\langle D_M,T_M,H,A \rangle$
the {\em reduct}\index{reduct!in a formal system} of
the pre-statement $\langle D,T,H,A \rangle$.

A {\em statement} is the reduct of some pre-statement\index{statement!in a
formal system}.  A statement is therefore a special kind of pre-statement;
in particular, a statement is the reduct of itself.

{\footnotesize\begin{quotation}
{\em Comment.}  $T$ is a set of expressions, each of length 2, that associate
a set of constants (``variable types'') with a set of variables.  The
condition ${\cal V}(H\cup\{A\}) \subseteq {\cal V}(T) $
means that each variable occurring in a statement's logical
hypotheses or assertion must have an associated variable-type hypothesis or
``type declaration,'' in  analogy to a computer programming language, where a
variable must be declared to be say, a string or an integer.  The requirement
that $\forall e,f\in T \, e_1 \ne f_1$ for $e\neq f$
means that each variable must be
associated with a unique constant designating its variable type; e.g., a
variable might be a ``wff'' or a ``set'' but not both.

Distinct-variable restrictions are used to specify what variable substitutions
are permissible to make for the statement to remain valid.  For example, in
the theorem scheme of set theory $\lnot\forall x\,x=y$ we may not substitute
the same variable for both $x$ and $y$.  On the other hand, the theorem scheme
$x=y\to y=x$ does not require that $x$ and $y$ be distinct, so we do not
require a distinct-variable restriction, although having one
would cause no harm other than making the scheme less general.

A mandatory variable-type hypothesis is one whose variable exists in a logical
hypothesis or the assertion.  A provable pre-statement
(defined below) may require
non-mandatory variable-type hypotheses that effectively introduce ``dummy''
variables for use in its proof.  Any number of dummy variables might
be required by a specific proof; indeed, it has been shown by H.\
Andr\'{e}ka\index{Andr{\'{e}}ka, H.} \cite{Nemeti} that there is no finite
upper bound to the number of dummy variables needed to prove an arbitrary
theorem in first-order logic (with equality) having a fixed number $n>2$ of
individual variables.  (See also the Comment on p.~\pageref{nodd}.)
For this reason we do not set a finite size bound on the collections $D$ and
$T$, although in an actual application (Metamath database) these will of
course be finite, increased to whatever size is necessary as more
proofs are added.
\end{quotation}}

{\footnotesize\begin{quotation}
{\em Relationship to Metamath.} A pre-statement of a formal system
corresponds to an extended frame in a Metamath database
(Section~\ref{frames}).  The collections $D$, $T$, and $H$ correspond
respectively to the \texttt{\$d}, \texttt{\$f}, and \texttt{\$e}
statement collections in an extended frame.  The expression $A$
corresponds to the \texttt{\$a} (or \texttt{\$p}) statement in an
extended frame.

A statement of a formal system corresponds to a frame in a Metamath
database.
\end{quotation}}

\subsection{Formal Systems}

A {\em formal system}\index{formal system} is a
triple $\langle \mbox{\em CN},\mbox{\em
VR},\Gamma\rangle$ where $\Gamma$ is a set of statements.  The members of
$\Gamma$ are called {\em axiomatic statements}.\index{axiomatic
statement!in a formal system}  Sometimes we will refer to a
formal system by just $\Gamma$ when $\mbox{\em CN}$ and $\mbox{\em VR}$ are
understood.

Given a formal system $\Gamma$, the {\em closure}\index{closure}\footnote{This
definition of closure incorporates a simplification due to
Josh Purinton.\index{Purinton, Josh}.} of a
pre-statement
$\langle D,T,H,A \rangle$ is the smallest set $C$ of expressions
such that:
%\begin{enumerate}
%  \item $T\cup H\subseteq C$; and
%  \item If for some axiomatic statement
%    $\langle D_M',T_M',H',A' \rangle \in \Gamma_A$, for
%    some $E \subseteq C$, some $F \subseteq C-T$ (where ``-'' denotes
%    set difference), and some substitution
%    $\sigma$ we have
%    \begin{enumerate}
%       \item $\sigma(T_M') = E$ (where, as above, the $M$ denotes the
%           mandatory variable-type hypotheses of $T^A$);
%       \item $\sigma(H') = F$;
%       \item for all $\{\alpha,\beta\}\in D^A$ and $\subseteq
%         {\cal V}(T_M')$, for all $\gamma\in {\cal V}(\sigma(\langle \alpha
%         \rangle))$, and for all $\delta\in  {\cal V}(\sigma(\langle \beta
%         \rangle))$, we have $\{\gamma, \delta\} \in D$;
%   \end{enumerate}
%   then $\sigma(A') \in C$.
%\end{enumerate}
\begin{list}{}{\itemsep 0.0pt}
  \item[1.] $T\cup H\subseteq C$; and
  \item[2.] If for some axiomatic statement
    $\langle D_M',T_M',H',A' \rangle \in
       \Gamma$ and for some substitution
    $\sigma$ we have
    \begin{enumerate}
       \item[a.] $\sigma(T_M' \cup H') \subseteq C$; and
       \item[b.] for all $\{\alpha,\beta\}\in D_M'$, for all $\gamma\in
         {\cal V}(\sigma(\langle \alpha
         \rangle))$, and for all $\delta\in  {\cal V}(\sigma(\langle \beta
         \rangle))$, we have $\{\gamma, \delta\} \in D$;
   \end{enumerate}
   then $\sigma(A') \in C$.
\end{list}
A pre-statement $\langle D,T,H,A
\rangle$ is {\em provable}\index{provable statement!in a formal
system} if $A\in C$ i.e.\ if its assertion belongs to its
closure.  A statement is {\em provable} if it is
the reduct of a provable pre-statement.
The {\em universe}\index{universe of a formal system}
of a formal system is
the collection of all of its provable statements.  Note that the
set of axiomatic statements $\Gamma$ in a formal system is a subset of its
universe.

{\footnotesize\begin{quotation}
{\em Comment.} The first condition in the definition of closure simply says
that the hypotheses of the pre-statement are in its closure.

Condition 2(a) says that a substitution exists that makes the
mandatory hypotheses of an axiomatic statement exactly match some members of
the closure.  This is what we explicitly demonstrate in a Metamath language
proof.

%Conditions 2(a) and 2(b) say that a substitution exists that makes the
%(mandatory) hypotheses of an axiomatic statement exactly match some members of
%the closure.  This is what we explicitly demonstrate with a Metamath language
%proof.
%
%The set of expressions $F$ in condition 2(b) excludes the variable-type
%hypotheses; this is done because non-mandatory variable-type hypotheses are
%effectively ``dropped'' as irrelevant whereas logical hypotheses must be
%retained to achieve a consistent logical system.

Condition 2(b) describes how distinct-variable restrictions in the axiomatic
statement must be met.  It means that after a substitution for two variables
that must be distinct, the resulting two expressions must either contain no
variables, or if they do, they may not have variables in common, and each pair
of any variables they do have, with one variable from each expression, must be
specified as distinct in the original statement.
\end{quotation}}

{\footnotesize\begin{quotation}
{\em Relationship to Metamath.} Axiomatic statements
 and provable statements in a formal
system correspond to the frames for \texttt{\$a} and \texttt{\$p} statements
respectively in a Metamath database.  The set of axiomatic statements are a
subset of the set of provable statements in a formal system, although in a
Metamath database a \texttt{\$a} statement is distinguished by not having a
proof.  A Metamath language proof for a \texttt{\$p} statement tells the computer
how to explicitly construct a series of members of the closure ultimately
leading to a demonstration that the assertion
being proved is in the closure.  The actual closure typically contains
an infinite number of expressions.  A formal system itself does not have
an explicit object called a ``proof'' but rather the existence of a proof
is implied indirectly by membership of an assertion in a provable
statement's closure.  We do this to make the formal system easier
to describe in the language of set theory.

We also note that once established as provable, a statement may be considered
to acquire the same status as an axiomatic statement, because if the set of
axiomatic statements is extended with a provable statement, the universe of
the formal system remains unchanged (provided that $\mbox{\em VR}$ is
infinite).
In practice, this means we can build a hierarchy of provable statements to
more efficiently establish additional provable statements.  This is
what we do in Metamath when we allow proofs to reference previous
\texttt{\$p} statements as well as previous \texttt{\$a} statements.
\end{quotation}}

\section{Examples of Formal Systems}

{\footnotesize\begin{quotation}
{\em Relationship to Metamath.} The examples in this section, except Example~2,
are for the most part exact equivalents of the development in the set
theory database \texttt{set.mm}.  You may want to compare Examples~1, 3, and 5
to Section~\ref{metaaxioms}, Example 4 to Sections~\ref{metadefprop} and
\ref{metadefpred}, and Example 6 to
Section~\ref{setdefinitions}.\label{exampleref}
\end{quotation}}

\subsection{Example~1---Propositional Calculus}\index{propositional calculus}

Classical propositional calculus can be described by the following formal
system.  We assume the set of variables is infinite.  Rather than denoting the
constants and variables by $c_0, c_1, \ldots$ and $v_0, v_1, \ldots$, for
readability we will instead use more conventional symbols, with the
understanding of course that they denote distinct primitive objects.
Also for readability we may omit commas between successive terms of a
sequence; thus $\langle \mbox{wff\ } \varphi\rangle$ denotes
$\langle \mbox{wff}, \varphi\rangle$.

Let
\begin{itemize}
  \item[] $\mbox{\em CN}=\{\mbox{wff}, \vdash, \to, \lnot, (,)\}$
  \item[] $\mbox{\em VR}=\{\varphi,\psi,\chi,\ldots\}$
  \item[] $T = \{\langle \mbox{wff\ } \varphi\rangle,
             \langle \mbox{wff\ } \psi\rangle,
             \langle \mbox{wff\ } \chi\rangle,\ldots\}$, i.e.\ those
             expressions of length 2 whose first member is $\mbox{\rm wff}$
             and whose second member belongs to $\mbox{\em VR}$.\footnote{For
convenience we let $T$ be an infinite set; the definition of a statement
permits this in principle.  Since a Metamath source file has a finite size, in
practice we must of course use appropriate finite subsets of this $T$,
specifically ones containing at least the mandatory variable-type
hypotheses.  Similarly, in the source file we introduce new variables as
required, with the understanding that a potentially infinite number of
them are available.}
\noindent Then $\Gamma$ consists of the axiomatic statements that
are the reducts of the following pre-statements:
    \begin{itemize}
      \item[] $\langle\varnothing,T,\varnothing,
               \langle \mbox{wff\ }(\varphi\to\psi)\rangle\rangle$
      \item[] $\langle\varnothing,T,\varnothing,
               \langle \mbox{wff\ }\lnot\varphi\rangle\rangle$
      \item[] $\langle\varnothing,T,\varnothing,
               \langle \vdash(\varphi\to(\psi\to\varphi))
               \rangle\rangle$
      \item[] $\langle\varnothing,T,
               \varnothing,
               \langle \vdash((\varphi\to(\psi\to\chi))\to
               ((\varphi\to\psi)\to(\varphi\to\chi)))
               \rangle\rangle$
      \item[] $\langle\varnothing,T,
               \varnothing,
               \langle \vdash((\lnot\varphi\to\lnot\psi)\to
               (\psi\to\varphi))\rangle\rangle$
      \item[] $\langle\varnothing,T,
               \{\langle\vdash(\varphi\to\psi)\rangle,
                 \langle\vdash\varphi\rangle\},
               \langle\vdash\psi\rangle\rangle$
    \end{itemize}
\end{itemize}

(For example, the reduct of $\langle\varnothing,T,\varnothing,
               \langle \mbox{wff\ }(\varphi\to\psi)\rangle\rangle$
is
\begin{itemize}
\item[] $\langle\varnothing,
\{\langle \mbox{wff\ } \varphi\rangle,
             \langle \mbox{wff\ } \psi\rangle\},
             \varnothing,
               \langle \mbox{wff\ }(\varphi\to\psi)\rangle\rangle$,
\end{itemize}
which is the first axiomatic statement.)

We call the members of $\mbox{\em VR}$ {\em wff variables} or (in the context
of first-order logic which we will describe shortly) {\em wff metavariables}.
Note that the symbols $\phi$, $\psi$, etc.\ denote actual specific members of
$\mbox{\em VR}$; they are not metavariables of our expository language (which
we denote with $\alpha$, $\beta$, etc.) but are instead (meta)constant symbols
(members of $\mbox{\em SM}$) from the point of view of our expository
language.  The equivalent system of propositional calculus described in
\cite{Tarski1965} also uses the symbols $\phi$, $\psi$, etc.\ to denote wff
metavariables, but in \cite{Tarski1965} unlike here those are metavariables of
the expository language and not primitive symbols of the formal system.

The first two statements define wffs: if $\varphi$ and $\psi$ are wffs, so is
$(\varphi \to \psi)$; if $\varphi$ is a wff, so is $\lnot\varphi$. The next
three are the axioms of propositional calculus: if $\varphi$ and $\psi$ are
wffs, then $\vdash (\varphi \to (\psi \to \varphi))$ is an (axiomatic)
theorem; etc. The
last is the rule of modus ponens: if $\varphi$ and $\psi$ are wffs, and
$\vdash (\varphi\to\psi)$ and $\vdash \varphi$ are theorems, then $\vdash
\psi$ is a theorem.

The correspondence to ordinary propositional calculus is as follows.  We
consider only provable statements of the form $\langle\varnothing,
T,\varnothing,A\rangle$ with $T$ defined as above.  The first term of the
assertion $A$ of any such statement is either ``wff'' or ``$\vdash$''.  A
statement for which the first term is ``wff'' is a {\em wff} of propositional
calculus, and one where the first term is ``$\vdash$'' is a {\em
theorem (scheme)} of propositional calculus.

The universe of this formal system also contains many other provable
statements.  Those with distinct-variable restrictions are irrelevant because
propositional calculus has no constraints on substitutions.  Those that have
logical hypotheses we call {\em inferences}\index{inference} when
the logical hypotheses are of the form
$\langle\vdash\rangle\frown w$ where $w$ is a wff (with the leading constant
term ``wff'' removed).  Inferences (other than the modus ponens rule) are not a
proper part of propositional calculus but are convenient to use when building a
hierarchy of provable statements.  A provable statement with a nonsense
hypothesis such as $\langle \to,\vdash,\lnot\rangle$, and this same expression
as its assertion, we consider irrelevant; no use can be made of it in
proving theorems, since there is no way to eliminate the nonsense hypothesis.

{\footnotesize\begin{quotation}
{\em Comment.} Our use of parentheses in the definition of a wff illustrates
how axiomatic statements should be carefully stated in a way that
ties in unambiguously with the substitutions allowed by the formal system.
There are many ways we could have defined wffs---for example, Polish
prefix notation would have allowed us to omit parentheses entirely, at
the expense of readability---but we must define them in a way that is
unambiguous.  For example, if we had omitted parentheses from the
definition of $(\varphi\to \psi)$, the wff $\lnot\varphi\to \psi$ could
be interpreted as either $\lnot(\varphi\to\psi)$ or $(\lnot\varphi\to\psi)$
and would have allowed us to prove nonsense.  Note that there is no
concept of operator binding precedence built into our formal system.
\end{quotation}}

\subsection{Example~2---Predicate Calculus with Equality}\index{predicate
calculus}

Here we extend Example~1 to include predicate calculus with equality,
illustrating the use of distinct-variable restrictions.  This system is the
same as Tarski's system $\mathfrak{S}_2$ in \cite{Tarski1965} (except that the
axioms of propositional calculus are different but equivalent, and a redundant
axiom is omitted).  We extend $\mbox{\em CN}$ with the constants
$\{\mbox{var},\forall,=\}$.  We extend $\mbox{\em VR}$ with an infinite set of
{\em individual metavariables}\index{individual
metavariable} $\{x,y,z,\ldots\}$ and denote this subset
$\mbox{\em Vr}$.

We also join to $\mbox{\em CN}$ a possibly infinite set $\mbox{\em Pr}$ of {\em
predicates} $\{R,S,\ldots\}$.  We associate with $\mbox{\em Pr}$ a function
$\mbox{rnk}$ from $\mbox{\em Pr}$ to $\omega$, and for $\alpha\in \mbox{\em
Pr}$ we call $\mbox{rnk}(\alpha)$ the {\em rank} of the predicate $\alpha$,
which is simply the number of ``arguments'' that the predicate has.  (Most
applications of predicate calculus will have a finite number of predicates;
for example, set theory has the single two-argument or binary predicate $\in$,
which is usually written with its arguments surrounding the predicate symbol
rather than with the prefix notation we will use for the general case.)  As a
device to facilitate our discussion, we will let $\mbox{\em Vs}$ be any fixed
one-to-one function from $\omega$ to $\mbox{\em Vr}$; thus $\mbox{\em Vs}$ is
any simple infinite sequence of individual metavariables with no repeating
terms.

In this example we will not include the function symbols that are often part of
formalizations of predicate calculus.  Using metalogical arguments that are
beyond the scope of our discussion, it can be shown that our formalization is
equivalent when functions are introduced via appropriate definitions.

We extend the set $T$ defined in Example~1 with the expressions
$\{\langle \mbox{var\ } x\rangle,$ $ \langle \mbox{var\ } y\rangle, \langle
\mbox{var\ } z\rangle,\ldots\}$.  We extend the $\Gamma$ above
with the axiomatic statements that are the reducts of the following
pre-statements:
\begin{list}{}{\itemsep 0.0pt}
      \item[] $\langle\varnothing,T,\varnothing,
               \langle \mbox{wff\ }\forall x\,\varphi\rangle\rangle$
      \item[] $\langle\varnothing,T,\varnothing,
               \langle \mbox{wff\ }x=y\rangle\rangle$
      \item[] $\langle\varnothing,T,
               \{\langle\vdash\varphi\rangle\},
               \langle\vdash\forall x\,\varphi\rangle\rangle$
      \item[] $\langle\varnothing,T,\varnothing,
               \langle \vdash((\forall x(\varphi\to\psi)
                  \to(\forall x\,\varphi\to\forall x\,\psi))
               \rangle\rangle$
      \item[] $\langle\{\{x,\varphi\}\},T,\varnothing,
               \langle \vdash(\varphi\to\forall x\,\varphi)
               \rangle\rangle$
      \item[] $\langle\{\{x,y\}\},T,\varnothing,
               \langle \vdash\lnot\forall x\lnot x=y
               \rangle\rangle$
      \item[] $\langle\varnothing,T,\varnothing,
               \langle \vdash(x=z
                  \to(x=y\to z=y))
               \rangle\rangle$
      \item[] $\langle\varnothing,T,\varnothing,
               \langle \vdash(y=z
                  \to(x=y\to x=z))
               \rangle\rangle$
\end{list}
These are the axioms not involving predicate symbols. The first two statements
extend the definition of a wff.  The third is the rule of generalization.  The
fifth states, in effect, ``For a wff $\varphi$ and variable $x$,
$\vdash(\varphi\to\forall x\,\varphi)$, provided that $x$ does not occur in
$\varphi$.''  The sixth states ``For variables $x$ and $y$,
$\vdash\lnot\forall x\lnot x = y$, provided that $x$ and $y$ are distinct.''
(This proviso is not necessary but was included by Tarski to
weaken the axiom and still show that the system is logically complete.)

Finally, for each predicate symbol $\alpha\in \mbox{\em Pr}$, we add to
$\Gamma$ an axiomatic statement, extending the definition of wff,
that is the reduct of the following pre-statement:
\begin{displaymath}
    \langle\varnothing,T,\varnothing,
            \langle \mbox{wff},\alpha\rangle\
            \frown \mbox{\em Vs}\restriction\mbox{rnk}(\alpha)\rangle
\end{displaymath}
and for each $\alpha\in \mbox{\em Pr}$ and each $n < \mbox{rnk}(\alpha)$
we add to $\Gamma$ an equality axiom that is the reduct of the
following pre-statement:
\begin{eqnarray*}
    \lefteqn{\langle\varnothing,T,\varnothing,
            \langle
      \vdash,(,\mbox{\em Vs}_n,=,\mbox{\em Vs}_{\mbox{rnk}(\alpha)},\to,
            (,\alpha\rangle\frown \mbox{\em Vs}\restriction\mbox{rnk}(\alpha)} \\
  & & \frown
            \langle\to,\alpha\rangle\frown \mbox{\em Vs}\restriction n\frown
            \langle \mbox{\em Vs}_{\mbox{rnk}(\alpha)}\rangle \\
 & & \frown
            \mbox{\em Vs}\restriction(\mbox{rnk}(\alpha)\setminus(n+1))\frown
            \langle),)\rangle\rangle
\end{eqnarray*}
where $\restriction$ denotes function domain restriction and $\setminus$
denotes set difference.  Recall that a subscript on $\mbox{\em Vs}$
denotes one of its terms.  (In the above two axiom sets commas are placed
between successive terms of sequences to prevent ambiguity, and if you examine
them with care you will be able to distinguish those parentheses that denote
constant symbols from those of our expository language that delimit function
arguments.  Although it might have been better to use boldface for our
primitive symbols, unfortunately boldface was not available for all characters
on the \LaTeX\ system used to typeset this text.)  These seemingly forbidding
axioms can be understood by analogy to concatenation of substrings in a
computer language.  They are actually relatively simple for each specific case
and will become clearer by looking at the special case of a binary predicate
$\alpha = R$ where $\mbox{rnk}(R)=2$.  Letting $\mbox{\em Vs}$ be the sequence
$\langle x,y,z,\ldots\rangle$, the axioms we would add to $\Gamma$ for this
case would be the wff extension and two equality axioms that are the
reducts of the pre-statements:
\begin{list}{}{\itemsep 0.0pt}
      \item[] $\langle\varnothing,T,\varnothing,
               \langle \mbox{wff\ }R x y\rangle\rangle$
      \item[] $\langle\varnothing,T,\varnothing,
               \langle \vdash(x=z
                  \to(R x y \to R z y))
               \rangle\rangle$
      \item[] $\langle\varnothing,T,\varnothing,
               \langle \vdash(y=z
                  \to(R x y \to R x z))
               \rangle\rangle$
\end{list}
Study these carefully to see how the general axioms above evaluate to
them.  In practice, typically only a few special cases such as this would be
needed, and in any case the Metamath language will only permit us to describe
a finite number of predicates, as opposed to the infinite number permitted by
the formal system.  (If an infinite number should be needed for some reason,
we could not define the formal system directly in the Metamath language but
could instead define it metalogically under set theory as we
do in this appendix, and only the underlying set theory, with its single
binary predicate, would be defined directly in the Metamath language.)


{\footnotesize\begin{quotation}
{\em Comment.}  As we noted earlier, the specific variables denoted by the
symbols $x,y,z,\ldots\in \mbox{\em Vr}\subseteq \mbox{\em VR}\subseteq
\mbox{\em SM}$ in Example~2 are not the actual variables of ordinary predicate
calculus but should be thought of as metavariables ranging over them.  For
example, a distinct-variable restriction would be meaningless for actual
variables of ordinary predicate calculus since two different actual variables
are by definition distinct.  And when we talk about an arbitrary
representative $\alpha\in \mbox{\em Vr}$, $\alpha$ is a metavariable (in our
expository language) that ranges over metavariables (which are primitives of
our formal system) each of which ranges over the actual individual variables
of predicate calculus (which are never mentioned in our formal system).

The constant called ``var'' above is called \texttt{set} in the
\texttt{set.mm} database file, but it means the same thing.  I felt
that ``var'' is a more meaningful name in the context of predicate
calculus, whose use is not limited to set theory.  For consistency we
stick with the name ``var'' throughout this Appendix, even after set
theory is introduced.
\end{quotation}}

\subsection{Free Variables and Proper Substitution}\index{free variable}
\index{proper substitution}\index{substitution!proper}

In the system of Example~2, there are no primitive notions of free variable
and proper substitution.  Tarski \cite{Tarski1965} shows that this system is
logically equivalent to the more typical textbook systems that do have these
primitive notions, if we introduce these notions with appropriate definitions
and metalogic.  We could also define axioms for such systems directly,
although the recursive definitions of free variable and proper substitution
would be messy and awkward to work with.  Instead, we mention two devices that
can be used in practice to mimic these notions.  (1) Instead of introducing
special notation to express (as a logical hypothesis) ``where $x$ is not free
in $\varphi$'' we can use the logical hypothesis $\vdash(\varphi\to\forall
x\,\varphi)$.\label{effectivelybound}\index{effectively
not free}\footnote{This is a slightly weaker requirement than ``where $x$ is
not free in $\varphi$.''  If we let $\varphi$ be $x=x$, we have the theorem
$(x=x\to\forall x\,x=x)$ which satisfies the hypothesis, even though $x$ is
free in $x=x$ .  In a case like this we say that $x$ is {\em effectively not
free}\index{effectively not free} in $x=x$, since $x=x$ is logically
equivalent to $\forall x\,x=x$ in which $x$ is bound.} (2) It can be shown
that the wff $((x=y\to\varphi)\wedge\exists x(x=y\wedge\varphi))$ (with the
usual definitions of $\wedge$ and $\exists$; see Example~4 below) is logically
equivalent to ``the wff that results from proper substitution of $y$ for $x$
in $\varphi$.''  This works whether or not $x$ and $y$ are distinct.

\subsection{Metalogical Completeness}\index{metalogical completeness}

In the system of Example~2, the
following are provable pre-statements (and their reducts are
provable statements):
\begin{eqnarray*}
      & \langle\{\{x,y\}\},T,\varnothing,
               \langle \vdash\lnot\forall x\lnot x=y
               \rangle\rangle & \\
     &  \langle\varnothing,T,\varnothing,
               \langle \vdash\lnot\forall x\lnot x=x
               \rangle\rangle &
\end{eqnarray*}
whereas the following pre-statement is not to my knowledge provable (but
in any case we will pretend it's not for sake of illustration):
\begin{eqnarray*}
     &  \langle\varnothing,T,\varnothing,
               \langle \vdash\lnot\forall x\lnot x=y
               \rangle\rangle &
\end{eqnarray*}
In other words, we can prove ``$\lnot\forall x\lnot x=y$ where $x$ and $y$ are
distinct'' and separately prove ``$\lnot\forall x\lnot x=x$'', but we can't
prove the combined general case ``$\lnot\forall x\lnot x=y$'' that has no
proviso.  Now this does not compromise logical completeness, because the
variables are really metavariables and the two provable cases together cover
all possible cases.  The third case can be considered a metatheorem whose
direct proof, using the system of Example~2, lies outside the capability of the
formal system.

Also, in the system of Example~2 the following pre-statement is not to my
knowledge provable (again, a conjecture that we will pretend to be the case):
\begin{eqnarray*}
     & \langle\varnothing,T,\varnothing,
               \langle \vdash(\forall x\, \varphi\to\varphi)
               \rangle\rangle &
\end{eqnarray*}
Instead, we can only prove specific cases of $\varphi$ involving individual
metavariables, and by induction on formula length, prove as a metatheorem
outside of our formal system the general statement above.  The details of this
proof are found in \cite{Kalish}.

There does, however, exist a system of predicate calculus in which all such
``simple metatheorems'' as those above can be proved directly, and we present
it in Example~3. A {\em simple metatheorem}\index{simple metatheorem}
is any statement of the formal
system of Example~2 where all distinct variable restrictions consist of either
two individual metavariables or an individual metavariable and a wff
metavariable, and which is provable by combining cases outside the system as
above.  A system is {\em metalogically complete}\index{metalogical
completeness} if all of its simple
metatheorems are (directly) provable statements. The precise definition of
``simple metatheorem'' and the proof of the ``metalogical completeness'' of
Example~3 is found in Remark 9.6 and Theorem 9.7 of \cite{Megill}.\index{Megill,
Norman}

\subsection{Example~3---Metalogically Complete Predicate
Calculus with
Equality}

For simplicity we will assume there is one binary predicate $R$;
this system suffices for set theory, where the $R$ is of course the $\in$
predicate.  We label the axioms as they appear in \cite{Megill}.  This
system is logically equivalent to that of Example~2 (when the latter is
restricted to this single binary predicate) but is also metalogically
complete.\index{metalogical completeness}

Let
\begin{itemize}
  \item[] $\mbox{\em CN}=\{\mbox{wff}, \mbox{var}, \vdash, \to, \lnot, (,),\forall,=,R\}$.
  \item[] $\mbox{\em VR}=\{\varphi,\psi,\chi,\ldots\}\cup\{x,y,z,\ldots\}$.
  \item[] $T = \{\langle \mbox{wff\ } \varphi\rangle,
             \langle \mbox{wff\ } \psi\rangle,
             \langle \mbox{wff\ } \chi\rangle,\ldots\}\cup
       \{\langle \mbox{var\ } x\rangle, \langle \mbox{var\ } y\rangle, \langle
       \mbox{var\ }z\rangle,\ldots\}$.

\noindent Then
  $\Gamma$ consists of the reducts of the following pre-statements:
    \begin{itemize}
      \item[] $\langle\varnothing,T,\varnothing,
               \langle \mbox{wff\ }(\varphi\to\psi)\rangle\rangle$
      \item[] $\langle\varnothing,T,\varnothing,
               \langle \mbox{wff\ }\lnot\varphi\rangle\rangle$
      \item[] $\langle\varnothing,T,\varnothing,
               \langle \mbox{wff\ }\forall x\,\varphi\rangle\rangle$
      \item[] $\langle\varnothing,T,\varnothing,
               \langle \mbox{wff\ }x=y\rangle\rangle$
      \item[] $\langle\varnothing,T,\varnothing,
               \langle \mbox{wff\ }Rxy\rangle\rangle$
      \item[(C1$'$)] $\langle\varnothing,T,\varnothing,
               \langle \vdash(\varphi\to(\psi\to\varphi))
               \rangle\rangle$
      \item[(C2$'$)] $\langle\varnothing,T,
               \varnothing,
               \langle \vdash((\varphi\to(\psi\to\chi))\to
               ((\varphi\to\psi)\to(\varphi\to\chi)))
               \rangle\rangle$
      \item[(C3$'$)] $\langle\varnothing,T,
               \varnothing,
               \langle \vdash((\lnot\varphi\to\lnot\psi)\to
               (\psi\to\varphi))\rangle\rangle$
      \item[(C4$'$)] $\langle\varnothing,T,
               \varnothing,
               \langle \vdash(\forall x(\forall x\,\varphi\to\psi)\to
                 (\forall x\,\varphi\to\forall x\,\psi))\rangle\rangle$
      \item[(C5$'$)] $\langle\varnothing,T,
               \varnothing,
               \langle \vdash(\forall x\,\varphi\to\varphi)\rangle\rangle$
      \item[(C6$'$)] $\langle\varnothing,T,
               \varnothing,
               \langle \vdash(\forall x\forall y\,\varphi\to
                 \forall y\forall x\,\varphi)\rangle\rangle$
      \item[(C7$'$)] $\langle\varnothing,T,
               \varnothing,
               \langle \vdash(\lnot\varphi\to\forall x\lnot\forall x\,\varphi
                 )\rangle\rangle$
      \item[(C8$'$)] $\langle\varnothing,T,
               \varnothing,
               \langle \vdash(x=y\to(x=z\to y=z))\rangle\rangle$
      \item[(C9$'$)] $\langle\varnothing,T,
               \varnothing,
               \langle \vdash(\lnot\forall x\, x=y\to(\lnot\forall x\, x=z\to
                 (y=z\to\forall x\, y=z)))\rangle\rangle$
      \item[(C10$'$)] $\langle\varnothing,T,
               \varnothing,
               \langle \vdash(\forall x(x=y\to\forall x\,\varphi)\to
                 \varphi))\rangle\rangle$
      \item[(C11$'$)] $\langle\varnothing,T,
               \varnothing,
               \langle \vdash(\forall x\, x=y\to(\forall x\,\varphi
               \to\forall y\,\varphi))\rangle\rangle$
      \item[(C12$'$)] $\langle\varnothing,T,
               \varnothing,
               \langle \vdash(x=y\to(Rxz\to Ryz))\rangle\rangle$
      \item[(C13$'$)] $\langle\varnothing,T,
               \varnothing,
               \langle \vdash(x=y\to(Rzx\to Rzy))\rangle\rangle$
      \item[(C15$'$)] $\langle\varnothing,T,
               \varnothing,
               \langle \vdash(\lnot\forall x\, x=y\to(x=y\to(\varphi
                 \to\forall x(x=y\to\varphi))))\rangle\rangle$
      \item[(C16$'$)] $\langle\{\{x,y\}\},T,
               \varnothing,
               \langle \vdash(\forall x\, x=y\to(\varphi\to\forall x\,\varphi)
                 )\rangle\rangle$
      \item[(C5)] $\langle\{\{x,\varphi\}\},T,\varnothing,
               \langle \vdash(\varphi\to\forall x\,\varphi)
               \rangle\rangle$
      \item[(MP)] $\langle\varnothing,T,
               \{\langle\vdash(\varphi\to\psi)\rangle,
                 \langle\vdash\varphi\rangle\},
               \langle\vdash\psi\rangle\rangle$
      \item[(Gen)] $\langle\varnothing,T,
               \{\langle\vdash\varphi\rangle\},
               \langle\vdash\forall x\,\varphi\rangle\rangle$
    \end{itemize}
\end{itemize}

While it is known that these axioms are ``metalogically complete,'' it is
not known whether they are independent (i.e.\ none is
redundant) in the metalogical sense; specifically, whether any axiom (possibly
with additional non-mandatory distinct-variable restrictions, for use with any
dummy variables in its proof) is provable from the others.  Note that
metalogical independence is a stronger requirement than independence in the
usual logical sense.  Not all of the above axioms are logically independent:
for example, C9$'$ can be proved as a metatheorem from the others, outside the
formal system, by combining the possible cases of distinct variables.

\subsection{Example~4---Adding Definitions}\index{definition}
There are several ways to add definitions to a formal system.  Probably the
most proper way is to consider definitions not as part of the formal system at
all but rather as abbreviations that are part of the expository metalogic
outside the formal system.  For convenience, though, we may use the formal
system itself to incorporate definitions, adding them as axiomatic extensions
to the system.  This could be done by adding a constant representing the
concept ``is defined as'' along with axioms for it. But there is a nicer way,
at least in this writer's opinion, that introduces definitions as direct
extensions to the language rather than as extralogical primitive notions.  We
introduce additional logical connectives and provide axioms for them.  For
systems of logic such as Examples 1 through 3, the additional axioms must be
conservative in the sense that no wff of the original system that was not a
theorem (when the initial term ``wff'' is replaced by ``$\vdash$'' of course)
becomes a theorem of the extended system.  In this example we extend Example~3
(or 2) with standard abbreviations of logic.

We extend $\mbox{\em CN}$ of Example~3 with new constants $\{\leftrightarrow,
\wedge,\vee,\exists\}$, corresponding to logical equivalence,\index{logical
equivalence ($\leftrightarrow$)}\index{biconditional ($\leftrightarrow$)}
conjunction,\index{conjunction ($\wedge$)} disjunction,\index{disjunction
($\vee$)} and the existential quantifier.\index{existential quantifier
($\exists$)}  We extend $\Gamma$ with the axiomatic statements that are
the reducts of the following pre-statements:
\begin{list}{}{\itemsep 0.0pt}
      \item[] $\langle\varnothing,T,\varnothing,
               \langle \mbox{wff\ }(\varphi\leftrightarrow\psi)\rangle\rangle$
      \item[] $\langle\varnothing,T,\varnothing,
               \langle \mbox{wff\ }(\varphi\vee\psi)\rangle\rangle$
      \item[] $\langle\varnothing,T,\varnothing,
               \langle \mbox{wff\ }(\varphi\wedge\psi)\rangle\rangle$
      \item[] $\langle\varnothing,T,\varnothing,
               \langle \mbox{wff\ }\exists x\, \varphi\rangle\rangle$
  \item[] $\langle\varnothing,T,\varnothing,
     \langle\vdash ( ( \varphi \leftrightarrow \psi ) \to
     ( \varphi \to \psi ) )\rangle\rangle$
  \item[] $\langle\varnothing,T,\varnothing,
     \langle\vdash ((\varphi\leftrightarrow\psi)\to
    (\psi\to\varphi))\rangle\rangle$
  \item[] $\langle\varnothing,T,\varnothing,
     \langle\vdash ((\varphi\to\psi)\to(
     (\psi\to\varphi)\to(\varphi
     \leftrightarrow\psi)))\rangle\rangle$
  \item[] $\langle\varnothing,T,\varnothing,
     \langle\vdash (( \varphi \wedge \psi ) \leftrightarrow\neg ( \varphi
     \to \neg \psi )) \rangle\rangle$
  \item[] $\langle\varnothing,T,\varnothing,
     \langle\vdash (( \varphi \vee \psi ) \leftrightarrow (\neg \varphi
     \to \psi )) \rangle\rangle$
  \item[] $\langle\varnothing,T,\varnothing,
     \langle\vdash (\exists x \,\varphi\leftrightarrow
     \lnot \forall x \lnot \varphi)\rangle\rangle$
\end{list}
The first three logical axioms (statements containing ``$\vdash$'') introduce
and effectively define logical equivalence, ``$\leftrightarrow$''.  The last
three use ``$\leftrightarrow$'' to effectively mean ``is defined as.''

\subsection{Example~5---ZFC Set Theory}\index{ZFC set theory}

Here we add to the system of Example~4 the axioms of Zermelo-Fraenkel set
theory with Choice.  For convenience we make use of the
definitions in Example~4.

In the $\mbox{\em CN}$ of Example~4 (which extends Example~3), we replace the symbol $R$
with the symbol $\in$.  We remove from $\Gamma$ of Example~4 the three
axiomatic statements containing $R$ and replace them with the
reducts of the following:
\begin{list}{}{\itemsep 0.0pt}
      \item[] $\langle\varnothing,T,\varnothing,
               \langle \mbox{wff\ }x\in y\rangle\rangle$
      \item[] $\langle\varnothing,T,
               \varnothing,
               \langle \vdash(x=y\to(x\in z\to y\in z))\rangle\rangle$
      \item[] $\langle\varnothing,T,
               \varnothing,
               \langle \vdash(x=y\to(z\in x\to z\in y))\rangle\rangle$
\end{list}
Letting $D=\{\{\alpha,\beta\}\in \mbox{\em DV}\,|\alpha,\beta\in \mbox{\em
Vr}\}$ (in other words all individual variables must be distinct), we extend
$\Gamma$ with the ZFC axioms, called
\index{Axiom of Extensionality}
\index{Axiom of Replacement}
\index{Axiom of Union}
\index{Axiom of Power Sets}
\index{Axiom of Regularity}
\index{Axiom of Infinity}
\index{Axiom of Choice}
Extensionality, Replacement, Union, Power
Set, Regularity, Infinity, and Choice, that are the reducts of:
\begin{list}{}{\itemsep 0.0pt}
      \item[Ext] $\langle D,T,
               \varnothing,
               \langle\vdash (\forall x(x\in y\leftrightarrow x \in z)\to y
               =z) \rangle\rangle$
      \item[Rep] $\langle D,T,
               \varnothing,
               \langle\vdash\exists x ( \exists y \forall z (\varphi \to z = y
                        ) \to
                        \forall z ( z \in x \leftrightarrow \exists x ( x \in
                        y \wedge \forall y\,\varphi ) ) )\rangle\rangle$
      \item[Un] $\langle D,T,
               \varnothing,
               \langle\vdash \exists x \forall y ( \exists x ( y \in x \wedge
               x \in z ) \to y \in x ) \rangle\rangle$
      \item[Pow] $\langle D,T,
               \varnothing,
               \langle\vdash \exists x \forall y ( \forall x ( x \in y \to x
               \in z ) \to y \in x ) \rangle\rangle$
      \item[Reg] $\langle D,T,
               \varnothing,
               \langle\vdash (  x \in y \to
                 \exists x ( x \in y \wedge \forall z ( z \in x \to \lnot z
                \in y ) ) ) \rangle\rangle$
      \item[Inf] $\langle D,T,
               \varnothing,
               \langle\vdash \exists x(y\in x\wedge\forall y(y\in
               x\to
               \exists z(y \in z\wedge z\in x))) \rangle\rangle$
      \item[AC] $\langle D,T,
               \varnothing,
               \langle\vdash \exists x \forall y \forall z ( ( y \in z
               \wedge z \in w ) \to \exists w \forall y ( \exists w
              ( ( y \in z \wedge z \in w ) \wedge ( y \in w \wedge w \in x
              ) ) \leftrightarrow y = w ) ) \rangle\rangle$
\end{list}

\subsection{Example~6---Class Notation in Set Theory}\label{class}

A powerful device that makes set theory easier (and that we have
been using all along in our informal expository language) is {\em class
abstraction notation}.\index{class abstraction}\index{abstraction class}  The
definitions we introduce are rigorously justified
as conservative by Takeuti and Zaring \cite{Takeuti}\index{Takeuti, Gaisi} or
Quine \cite{Quine}\index{Quine, Willard Van Orman}.  The key idea is to
introduce the notation $\{x|\mbox{---}\}$ which means ``the class of all $x$
such that ---'' for abstraction classes and introduce (meta)variables that
range over them.  An abstraction class may or may not be a set, depending on
whether it exists (as a set).  A class that does not exist is
called a {\em proper class}.\index{proper class}\index{class!proper}

To illustrate the use of abstraction classes we will provide some examples
of definitions that make use of them:  the empty set, class union, and
unordered pair.  Many other such definitions can be found in the
Metamath set theory database,
\texttt{set.mm}.\index{set theory database (\texttt{set.mm})}

We extend $\mbox{\em CN}$ of Example~5 with new symbols $\{\mbox{class}, \{,
|, \}, \varnothing, \cup, , \}$ where the inner braces and last comma are
constant symbols. (As before,
our dual use of some mathematical symbols for both our expository
language and as primitives of the formal system should be clear from context.)

We extend $\mbox{\em VR}$ of Example~5 with a set of {\em class
variables}\index{class variable}
$\{A,B,C,\ldots\}$. We extend the $T$ of Example~5 with $\{\langle
\mbox{class\ } A\rangle, \langle \mbox{class\ }B\rangle, \langle \mbox{class\ }
C\rangle,\ldots\}$.

To
introduce our definitions,
we add to $\Gamma$ of Example~5 the axiomatic statements
that are the reducts of the following pre-statements:
\begin{list}{}{\itemsep 0.0pt}
      \item[] $\langle\varnothing,T,\varnothing,
               \langle \mbox{class\ }x\rangle\rangle$
      \item[] $\langle\varnothing,T,\varnothing,
               \langle \mbox{class\ }\{x|\varphi\}\rangle\rangle$
      \item[] $\langle\varnothing,T,\varnothing,
               \langle \mbox{wff\ }A=B\rangle\rangle$
      \item[] $\langle\varnothing,T,\varnothing,
               \langle \mbox{wff\ }A\in B\rangle\rangle$
      \item[Ab] $\langle\varnothing,T,\varnothing,
               \langle \vdash ( y \in \{ x |\varphi\} \leftrightarrow
                  ( ( x = y \to\varphi) \wedge \exists x ( x = y
                  \wedge\varphi) ))
               \rangle\rangle$
      \item[Eq] $\langle\{\{x,A\},\{x,B\}\},T,\varnothing,
               \langle \vdash ( A = B \leftrightarrow
               \forall x ( x \in A \leftrightarrow x \in B ) )
               \rangle\rangle$
      \item[El] $\langle\{\{x,A\},\{x,B\}\},T,\varnothing,
               \langle \vdash ( A \in B \leftrightarrow \exists x
               ( x = A \wedge x \in B ) )
               \rangle\rangle$
\end{list}
Here we say that an individual variable is a class; $\{x|\varphi\}$ is a
class; and we extend the definition of a wff to include class equality and
membership.  Axiom Ab defines membership of a variable in a class abstraction;
the right-hand side can be read as ``the wff that results from proper
substitution of $y$ for $x$ in $\varphi$.''\footnote{Note that this definition
makes unnecessary the introduction of a separate notation similar to
$\varphi(x|y)$ for proper substitution, although we may choose to do so to be
conventional.  Incidentally, $\varphi(x|y)$ as it stands would be ambiguous in
the formal systems of our examples, since we wouldn't know whether
$\lnot\varphi(x|y)$ meant $\lnot(\varphi(x|y))$ or $(\lnot\varphi)(x|y)$.
Instead, we would have to use an unambiguous variant such as $(\varphi\,
x|y)$.}  Axioms Eq and El extend the meaning of the existing equality and
membership connectives.  This is potentially dangerous and requires careful
justification.  For example, from Eq we can derive the Axiom of Extensionality
with predicate logic alone; thus in principle we should include the Axiom of
Extensionality as a logical hypothesis.  However we do not bother to do this
since we have already presupposed that axiom earlier. The distinct variable
restrictions should be read ``where $x$ does not occur in $A$ or $B$.''  We
typically do this when the right-hand side of a definition involves an
individual variable not in the expression being defined; it is done so that
the right-hand side remains independent of the particular ``dummy'' variable
we use.

We continue to add to $\Gamma$ the following definitions
(i.e. the reducts of the following pre-statements) for empty
set,\index{empty set} class union,\index{union} and unordered
pair.\index{unordered pair}  They should be self-explanatory.  Analogous to our
use of ``$\leftrightarrow$'' to define new wffs in Example~4, we use ``$=$''
to define new abstraction terms, and both may be read informally as ``is
defined as'' in this context.
\begin{list}{}{\itemsep 0.0pt}
      \item[] $\langle\varnothing,T,\varnothing,
               \langle \mbox{class\ }\varnothing\rangle\rangle$
      \item[] $\langle\varnothing,T,\varnothing,
               \langle \vdash \varnothing = \{ x | \lnot x = x \}
               \rangle\rangle$
      \item[] $\langle\varnothing,T,\varnothing,
               \langle \mbox{class\ }(A\cup B)\rangle\rangle$
      \item[] $\langle\{\{x,A\},\{x,B\}\},T,\varnothing,
               \langle \vdash ( A \cup B ) = \{ x | ( x \in A \vee x \in B ) \}
               \rangle\rangle$
      \item[] $\langle\varnothing,T,\varnothing,
               \langle \mbox{class\ }\{A,B\}\rangle\rangle$
      \item[] $\langle\{\{x,A\},\{x,B\}\},T,\varnothing,
               \langle \vdash \{ A , B \} = \{ x | ( x = A \vee x = B ) \}
               \rangle\rangle$
\end{list}

\section{Metamath as a Formal System}\label{theorymm}

This section presupposes a familiarity with the Metamath computer language.

Our theory describes formal systems and their universes.  The Metamath
language provides a way of representing these set-theoretical objects to
a computer.  A Metamath database, being a finite set of {\sc ascii}
characters, can usually describe only a subset of a formal system and
its universe, which are typically infinite.  However the database can
contain as large a finite subset of the formal system and its universe
as we wish.  (Of course a Metamath set theory database can, in
principle, indirectly describe an entire infinite formal system by
formalizing the expository language in this Appendix.)

For purpose of our discussion, we assume the Metamath database
is in the simple form described on p.~\pageref{framelist},
consisting of all constant and variable declarations at the beginning,
followed by a sequence of extended frames each
delimited by \texttt{\$\char`\{} and \texttt{\$\char`\}}.  Any Metamath database can
be converted to this form, as described on p.~\pageref{frameconvert}.

The math symbol tokens of a Metamath source file, which are declared
with \texttt{\$c} and \texttt{\$v} statements, are names we assign to
representatives of $\mbox{\em CN}$ and $\mbox{\em VR}$.  For
definiteness we could assume that the first math symbol declared as a
variable corresponds to $v_0$, the second to $v_1$, etc., although the
exact correspondence we choose is not important.

In the Metamath language, each \texttt{\$d}, \texttt{\$f}, and
 \texttt{\$e} source
statement in an extended frame (Section~\ref{frames})
corresponds respectively to a member of the
collections $D$, $T$, and $H$ in a formal system statement $\langle
D_M,T_M,H,A\rangle$.  The math symbol strings following these Metamath keywords
correspond to a variable pair (in the case of \texttt{\$d}) or an expression (for
the other two keywords). The math symbol string following a \texttt{\$a} source
statement corresponds to expression $A$ in an axiomatic statement of the
formal system; the one following a \texttt{\$p} source statement corresponds to
$A$ in a provable statement that is not axiomatic.  In other words, each
extended frame in a Metamath database corresponds to
a pre-statement of the formal system, and a frame corresponds to
a statement of the formal system.  (Don't confuse the two meanings of
``statement'' here.  A statement of the formal system corresponds to the
several statements in a Metamath database that may constitute a
frame.)

In order for the computer to verify that a formal system statement is
provable, each \texttt{\$p} source statement is accompanied by a proof.
However, the proof does not correspond to anything in the formal system
but is simply a way of communicating to the computer the information
needed for its verification.  The proof tells the computer {\em how to
construct} specific members of closure of the formal system
pre-statement corresponding to the extended frame of the \texttt{\$p}
statement.  The final result of the construction is the member of the
closure that matches the \texttt{\$p} statement.  The abstract formal
system, on the other hand, is concerned only with the {\em existence} of
members of the closure.

As mentioned on p.~\pageref{exampleref}, Examples 1 and 3--6 in the
previous Section parallel the development of logic and set theory in the
Metamath database
\texttt{set.mm}.\index{set theory database (\texttt{set.mm})} You may
find it instructive to compare them.


\chapter{The MIU System}
\label{MIU}
\index{formal system}
\index{MIU-system}

The following is a listing of the file \texttt{miu.mm}.  It is self-explanatory.

%%%%%%%%%%%%%%%%%%%%%%%%%%%%%%%%%%%%%%%%%%%%%%%%%%%%%%%%%%%%

\begin{verbatim}
$( The MIU-system:  A simple formal system $)

$( Note:  This formal system is unusual in that it allows
empty wffs.  To work with a proof, you must type
SET EMPTY_SUBSTITUTION ON before using the PROVE command.
By default, this is OFF in order to reduce the number of
ambiguous unification possibilities that have to be selected
during the construction of a proof.  $)

$(
Hofstadter's MIU-system is a simple example of a formal
system that illustrates some concepts of Metamath.  See
Douglas R. Hofstadter, _Goedel, Escher, Bach:  An Eternal
Golden Braid_ (Vintage Books, New York, 1979), pp. 33ff. for
a description of the MIU-system.

The system has 3 constant symbols, M, I, and U.  The sole
axiom of the system is MI. There are 4 rules:
     Rule I:  If you possess a string whose last letter is I,
     you can add on a U at the end.
     Rule II:  Suppose you have Mx.  Then you may add Mxx to
     your collection.
     Rule III:  If III occurs in one of the strings in your
     collection, you may make a new string with U in place
     of III.
     Rule IV:  If UU occurs inside one of your strings, you
     can drop it.
Unfortunately, Rules III and IV do not have unique results:
strings could have more than one occurrence of III or UU.
This requires that we introduce the concept of an "MIU
well-formed formula" or wff, which allows us to construct
unique symbol sequences to which Rules III and IV can be
applied.
$)

$( First, we declare the constant symbols of the language.
Note that we need two symbols to distinguish the assertion
that a sequence is a wff from the assertion that it is a
theorem; we have arbitrarily chosen "wff" and "|-". $)
      $c M I U |- wff $. $( Declare constants $)

$( Next, we declare some variables. $)
     $v x y $.

$( Throughout our theory, we shall assume that these
variables represent wffs. $)
 wx   $f wff x $.
 wy   $f wff y $.

$( Define MIU-wffs.  We allow the empty sequence to be a
wff. $)

$( The empty sequence is a wff. $)
 we   $a wff $.
$( "M" after any wff is a wff. $)
 wM   $a wff x M $.
$( "I" after any wff is a wff. $)
 wI   $a wff x I $.
$( "U" after any wff is a wff. $)
 wU   $a wff x U $.

$( Assert the axiom. $)
 ax   $a |- M I $.

$( Assert the rules. $)
 ${
   Ia   $e |- x I $.
$( Given any theorem ending with "I", it remains a theorem
if "U" is added after it.  (We distinguish the label I_
from the math symbol I to conform to the 24-Jun-2006
Metamath spec.) $)
   I_    $a |- x I U $.
 $}
 ${
IIa  $e |- M x $.
$( Given any theorem starting with "M", it remains a theorem
if the part after the "M" is added again after it. $)
   II   $a |- M x x $.
 $}
 ${
   IIIa $e |- x I I I y $.
$( Given any theorem with "III" in the middle, it remains a
theorem if the "III" is replaced with "U". $)
   III  $a |- x U y $.
 $}
 ${
   IVa  $e |- x U U y $.
$( Given any theorem with "UU" in the middle, it remains a
theorem if the "UU" is deleted. $)
   IV   $a |- x y $.
  $}

$( Now we prove the theorem MUIIU.  You may be interested in
comparing this proof with that of Hofstadter (pp. 35 - 36).
$)
 theorem1  $p |- M U I I U $=
      we wM wU wI we wI wU we wU wI wU we wM we wI wU we wM
      wI wI wI we wI wI we wI ax II II I_ III II IV $.
\end{verbatim}\index{well-formed formula (wff)}

The \texttt{show proof /essential/lemmon/renumber} command
yields the following display.  It is very similar
to the one in \cite[pp.~35--36]{Hofstadter}.\index{Hofstadter, Douglas R.}

\begin{verbatim}
        1 ax             $a |- M I
        2 1 II           $a |- M I I
        3 2 II           $a |- M I I I I
        4 3 I_           $a |- M I I I I U
        5 4 III          $a |- M U I U
        6 5 II           $a |- M U I U U I U
        7 6 IV           $a |- M U I I U
\end{verbatim}

We note that Hofstadter's ``MU-puzzle,'' which asks whether
MU is a theorem of the MIU-system, cannot be answered using
the system above because the MU-puzzle is a question {\em
about} the system.  To prove the answer to the MU-puzzle,
a much more elaborate system is needed, namely one that
models the MIU-system within set theory.  (Incidentally, the
answer to the MU-puzzle is no.)

%
% \chapter{Disclaimer and Trademarks}
%
% Information in this document is subject to change without notice and does not
% represent a commitment on the part of Norman Megill.
% \vspace{2ex}
%
% \noindent Norman D. Megill makes no warranties, either express or implied,
% regarding the Metamath computer software package.
%
% \vspace{2ex}
%
% \noindent Any trademarks mentioned in this book are the property of
% their respective owners.  The name ``Metamath'' is a trademark of
% Norman Megill.
%

\cleardoublepage
\phantomsection  % fixes the link anchor
\addcontentsline{toc}{chapter}{\bibname}

\bibliography{metamath}
%% metamath.tex - Version of 24-May-2019
% If you change the date above, also change the "Printed date" below.
% SPDX-License-Identifier: CC0-1.0
%
%                              PUBLIC DOMAIN
%
% This file (specifically, the version of this file with the above date)
% has been released into the Public Domain per the
% Creative Commons CC0 1.0 Universal (CC0 1.0) Public Domain Dedication
% https://creativecommons.org/publicdomain/zero/1.0/
%
% The public domain release applies worldwide.  In case this is not
% legally possible, the right is granted to use the work for any purpose,
% without any conditions, unless such conditions are required by law.
%
% Several short, attributed quotations from copyrighted works
% appear in this file under the ``fair use'' provision of Section 107 of
% the United States Copyright Act (Title 17 of the {\em United States
% Code}).  The public-domain status of this file is not applicable to
% those quotations.
%
% Norman Megill - email: nm(at)alum(dot)mit(dot)edu
%
% David A. Wheeler also donates his improvements to this file to the
% public domain per the CC0.  He works at the Institute for Defense Analyses
% (IDA), but IDA has agreed that this Metamath work is outside its "lane"
% and is not a work by IDA.  This was specifically confirmed by
% Margaret E. Myers (Division Director of the Information Technology
% and Systems Division) on 2019-05-24 and by Ben Lindorf (General Counsel)
% on 2019-05-22.

% This file, 'metamath.tex', is self-contained with everything needed to
% generate the the PDF file 'metamath.pdf' (the _Metamath_ book) on
% standard LaTeX 2e installations.  The auxiliary files are embedded with
% "filecontents" commands.  To generate metamath.pdf file, run these
% commands under Linux or Cygwin in the directory that contains
% 'metamath.tex':
%
%   rm -f realref.sty metamath.bib
%   touch metamath.ind
%   pdflatex metamath
%   pdflatex metamath
%   bibtex metamath
%   makeindex metamath
%   pdflatex metamath
%   pdflatex metamath
%
% The warnings that occur in the initial runs of pdflatex can be ignored.
% For the final run,
%
%   egrep -i 'error|warn' metamath.log
%
% should show exactly these 5 warnings:
%
%   LaTeX Warning: File `realref.sty' already exists on the system.
%   LaTeX Warning: File `metamath.bib' already exists on the system.
%   LaTeX Font Warning: Font shape `OMS/cmtt/m/n' undefined
%   LaTeX Font Warning: Font shape `OMS/cmtt/bx/n' undefined
%   LaTeX Font Warning: Some font shapes were not available, defaults
%       substituted.
%
% Search for "Uncomment" below if you want to suppress hyperlink boxes
% in the PDF output file
%
% TYPOGRAPHICAL NOTES:
% * It is customary to use an en dash (--) to "connect" names of different
%   people (and to denote ranges), and use a hyphen (-) for a
%   single compound name. Examples of connected multiple people are
%   Zermelo--Fraenkel, Schr\"{o}der--Bernstein, Tarski--Grothendieck,
%   Hewlett--Packard, and Backus--Naur.  Examples of a single person with
%   a compound name include Levi-Civita, Mittag-Leffler, and Burali-Forti.
% * Use non-breaking spaces after page abbreviations, e.g.,
%   p.~\pageref{note2002}.
%
% --------------------------- Start of realref.sty -----------------------------
\begin{filecontents}{realref.sty}
% Save the following as realref.sty.
% You can then use it with \usepackage{realref}
%
% This has \pageref jumping to the page on which the ref appears,
% \ref jumping to the point of the anchor, and \sectionref
% jumping to the start of section.
%
% Author:  Anthony Williams
%          Software Engineer
%          Nortel Networks Optical Components Ltd
% Date:    9 Nov 2001 (posted to comp.text.tex)
%
% The following declaration was made by Anthony Williams on
% 24 Jul 2006 (private email to Norman Megill):
%
%   ``I hereby donate the code for realref.sty posted on the
%   comp.text.tex newsgroup on 9th November 2001, accessible from
%   http://groups.google.com/group/comp.text.tex/msg/5a0e1cc13ea7fbb2
%   to the public domain.''
%
\ProvidesPackage{realref}
\RequirePackage[plainpages=false,pdfpagelabels=true]{hyperref}
\def\realref@anchorname{}
\AtBeginDocument{%
% ensure every label is a possible hyperlink target
\let\realref@oldrefstepcounter\refstepcounter%
\DeclareRobustCommand{\refstepcounter}[1]{\realref@oldrefstepcounter{#1}
\edef\realref@anchorname{\string #1.\@currentlabel}%
}%
\let\realref@oldlabel\label%
\DeclareRobustCommand{\label}[1]{\realref@oldlabel{#1}\hypertarget{#1}{}%
\@bsphack\protected@write\@auxout{}{%
    \string\expandafter\gdef\protect\csname
    page@num.#1\string\endcsname{\thepage}%
    \string\expandafter\gdef\protect\csname
    ref@num.#1\string\endcsname{\@currentlabel}%
    \string\expandafter\gdef\protect\csname
    sectionref@name.#1\string\endcsname{\realref@anchorname}%
}\@esphack}%
\DeclareRobustCommand\pageref[1]{{\edef\a{\csname
            page@num.#1\endcsname}\expandafter\hyperlink{page.\a}{\a}}}%
\DeclareRobustCommand\ref[1]{{\edef\a{\csname
            ref@num.#1\endcsname}\hyperlink{#1}{\a}}}%
\DeclareRobustCommand\sectionref[1]{{\edef\a{\csname
            ref@num.#1\endcsname}\edef\b{\csname
            sectionref@name.#1\endcsname}\hyperlink{\b}{\a}}}%
}
\end{filecontents}
% ---------------------------- End of realref.sty ------------------------------

% --------------------------- Start of metamath.bib -----------------------------
\begin{filecontents}{metamath.bib}
@book{Albers, editor = "Donald J. Albers and G. L. Alexanderson",
  title = "Mathematical People",
  publisher = "Contemporary Books, Inc.",
  address = "Chicago",
  note = "[QA28.M37]",
  year = 1985 }
@book{Anderson, author = "Alan Ross Anderson and Nuel D. Belnap",
  title = "Entailment",
  publisher = "Princeton University Press",
  address = "Princeton",
  volume = 1,
  note = "[QA9.A634 1975 v.1]",
  year = 1975}
@book{Barrow, author = "John D. Barrow",
  title = "Theories of Everything:  The Quest for Ultimate Explanation",
  publisher = "Oxford University Press",
  address = "Oxford",
  note = "[Q175.B225]",
  year = 1991 }
@book{Behnke,
  editor = "H. Behnke and F. Backmann and K. Fladt and W. S{\"{u}}ss",
  title = "Fundamentals of Mathematics",
  volume = "I",
  publisher = "The MIT Press",
  address = "Cambridge, Massachusetts",
  note = "[QA37.2.B413]",
  year = 1974 }
@book{Bell, author = "J. L. Bell and M. Machover",
  title = "A Course in Mathematical Logic",
  publisher = "North-Holland",
  address = "Amsterdam",
  note = "[QA9.B3953]",
  year = 1977 }
@inproceedings{Blass, author = "Andrea Blass",
  title = "The Interaction Between Category Theory and Set Theory",
  pages = "5--29",
  booktitle = "Mathematical Applications of Category Theory (Proceedings
     of the Special Session on Mathematical Applications
     Category Theory, 89th Annual Meeting of the American Mathematical
     Society, held in Denver, Colorado January 5--9, 1983)",
  editor = "John Walter Gray",
  year = 1983,
  note = "[QA169.A47 1983]",
  publisher = "American Mathematical Society",
  address = "Providence, Rhode Island"}
@proceedings{Bledsoe, editor = "W. W. Bledsoe and D. W. Loveland",
  title = "Automated Theorem Proving:  After 25 Years (Proceedings
     of the Special Session on Automatic Theorem Proving,
     89th Annual Meeting of the American Mathematical
     Society, held in Denver, Colorado January 5--9, 1983)",
  year = 1983,
  note = "[QA76.9.A96.S64 1983]",
  publisher = "American Mathematical Society",
  address = "Providence, Rhode Island" }
@book{Boolos, author = "George S. Boolos and Richard C. Jeffrey",
  title = "Computability and Log\-ic",
  publisher = "Cambridge University Press",
  edition = "third",
  address = "Cambridge",
  note = "[QA9.59.B66 1989]",
  year = 1989 }
@book{Campbell, author = "John Campbell",
  title = "Programmer's Progress",
  publisher = "White Star Software",
  address = "Box 51623, Palo Alto, CA 94303",
  year = 1991 }
@article{DBLP:journals/corr/Carneiro14,
  author    = {Mario Carneiro},
  title     = {Conversion of {HOL} Light proofs into Metamath},
  journal   = {CoRR},
  volume    = {abs/1412.8091},
  year      = {2014},
  url       = {http://arxiv.org/abs/1412.8091},
  archivePrefix = {arXiv},
  eprint    = {1412.8091},
  timestamp = {Mon, 13 Aug 2018 16:47:05 +0200},
  biburl    = {https://dblp.org/rec/bib/journals/corr/Carneiro14},
  bibsource = {dblp computer science bibliography, https://dblp.org}
}
@article{CarneiroND,
  author    = {Mario Carneiro},
  title     = {Natural Deductions in the Metamath Proof Language},
  url       = {http://us.metamath.org/ocat/natded.pdf},
  year      = 2014
}
@inproceedings{Chou, author = "Shang-Ching Chou",
  title = "Proving Elementary Geometry Theorems Using {W}u's Algorithm",
  pages = "243--286",
  booktitle = "Automated Theorem Proving:  After 25 Years (Proceedings
     of the Special Session on Automatic Theorem Proving,
     89th Annual Meeting of the American Mathematical
     Society, held in Denver, Colorado January 5--9, 1983)",
  editor = "W. W. Bledsoe and D. W. Loveland",
  year = 1983,
  note = "[QA76.9.A96.S64 1983]",
  publisher = "American Mathematical Society",
  address = "Providence, Rhode Island" }
@book{Clemente, author = "Daniel Clemente Laboreo",
  title = "Introduction to natural deduction",
  year = 2014,
  url = "http://www.danielclemente.com/logica/dn.en.pdf" }
@incollection{Courant, author = "Richard Courant and Herbert Robbins",
  title = "Topology",
  pages = "573--590",
  booktitle = "The World of Mathematics, Volume One",
  editor = "James R. Newman",
  publisher = "Simon and Schuster",
  address = "New York",
  note = "[QA3.W67 1988]",
  year = 1956 }
@book{Curry, author = "Haskell B. Curry",
  title = "Foundations of Mathematical Logic",
  publisher = "Dover Publications, Inc.",
  address = "New York",
  note = "[QA9.C976 1977]",
  year = 1977 }
@book{Davis, author = "Philip J. Davis and Reuben Hersh",
  title = "The Mathematical Experience",
  publisher = "Birkh{\"{a}}user Boston",
  address = "Boston",
  note = "[QA8.4.D37 1982]",
  year = 1981 }
@incollection{deMillo,
  author = "Richard de Millo and Richard Lipton and Alan Perlis",
  title = "Social Processes and Proofs of Theorems and Programs",
  pages = "267--285",
  booktitle = "New Directions in the Philosophy of Mathematics",
  editor = "Thomas Tymoczko",
  publisher = "Birkh{\"{a}}user Boston, Inc.",
  address = "Boston",
  note = "[QA8.6.N48 1986]",
  year = 1986 }
@book{Edwards, author = "Robert E. Edwards",
  title = "A Formal Background to Mathematics",
  publisher = "Springer-Verlag",
  address = "New York",
  note = "[QA37.2.E38 v.1a]",
  year = 1979 }
@book{Enderton, author = "Herbert B. Enderton",
  title = "Elements of Set Theory",
  publisher = "Academic Press, Inc.",
  address = "San Diego",
  note = "[QA248.E5]",
  year = 1977 }
@book{Goodstein, author = "R. L. Goodstein",
  title = "Development of Mathematical Logic",
  publisher = "Springer-Verlag New York Inc.",
  address = "New York",
  note = "[QA9.G6554]",
  year = 1971 }
@book{Guillen, author = "Michael Guillen",
  title = "Bridges to Infinity",
  publisher = "Jeremy P. Tarcher, Inc.",
  address = "Los Angeles",
  note = "[QA93.G8]",
  year = 1983 }
@book{Hamilton, author = "Alan G. Hamilton",
  title = "Logic for Mathematicians",
  edition = "revised",
  publisher = "Cambridge University Press",
  address = "Cambridge",
  note = "[QA9.H298]",
  year = 1988 }
@unpublished{Harrison, author = "John Robert Harrison",
  title = "Metatheory and Reflection in Theorem Proving:
    A Survey and Critique",
  note = "Technical Report
    CRC-053.
    SRI Cambridge,
    Millers Yard, Cambridge, UK,
    1995.
    Available on the Web as
{\verb+http:+}\-{\verb+//www.cl.cam.ac.uk/users/jrh/papers/reflect.html+}"}
@TECHREPORT{Harrison-thesis,
        author          = "John Robert Harrison",
        title           = "Theorem Proving with the Real Numbers",
        institution   = "University of Cambridge Computer
                         Lab\-o\-ra\-to\-ry",
        address         = "New Museums Site, Pembroke Street, Cambridge,
                           CB2 3QG, UK",
        year            = 1996,
        number          = 408,
        type            = "Technical Report",
        note            = "Author's PhD thesis,
   available on the Web at
{\verb+http:+}\-{\verb+//www.cl.cam.ac.uk+}\-{\verb+/users+}\-{\verb+/jrh+}%
\-{\verb+/papers+}\-{\verb+/thesis.html+}"}
@book{Herrlich, author = "Horst Herrlich and George E. Strecker",
  title = "Category Theory:  An Introduction",
  publisher = "Allyn and Bacon Inc.",
  address = "Boston",
  note = "[QA169.H567]",
  year = 1973 }
@article{Hindley, author = "J. Roger Hindley and David Meredith",
  title = "Principal Type-Schemes and Condensed Detachment",
  journal = "The Journal of Symbolic Logic",
  volume = 55,
  year = 1990,
  note = "[QA.J87]",
  pages = "90--105" }
@book{Hofstadter, author = "Douglas R. Hofstadter",
  title = "G{\"{o}}del, Escher, Bach",
  publisher = "Basic Books, Inc.",
  address = "New York",
  note = "[QA9.H63 1980]",
  year = 1979 }
@article{Indrzejczak, author= "Andrzej Indrzejczak",
  title = "Natural Deduction, Hybrid Systems and Modal Logic",
  journal = "Trends in Logic",
  volume = 30,
  publisher = "Springer",
  year = 2010 }
@article{Kalish, author = "D. Kalish and R. Montague",
  title = "On {T}arski's Formalization of Predicate Logic with Identity",
  journal = "Archiv f{\"{u}}r Mathematische Logik und Grundlagenfor\-schung",
  volume = 7,
  year = 1965,
  note = "[QA.A673]",
  pages = "81--101" }
@article{Kalman, author = "J. A. Kalman",
  title = "Condensed Detachment as a Rule of Inference",
  journal = "Studia Logica",
  volume = 42,
  number = 4,
  year = 1983,
  note = "[B18.P6.S933]",
  pages = "443-451" }
@book{Kline, author = "Morris Kline",
  title = "Mathematical Thought from Ancient to Modern Times",
  publisher = "Oxford University Press",
  address = "New York",
  note = "[QA21.K516 1990 v.3]",
  year = 1972 }
@book{Klinel, author = "Morris Kline",
  title = "Mathematics, The Loss of Certainty",
  publisher = "Oxford University Press",
  address = "New York",
  note = "[QA21.K525]",
  year = 1980 }
@book{Kramer, author = "Edna E. Kramer",
  title = "The Nature and Growth of Modern Mathematics",
  publisher = "Princeton University Press",
  address = "Princeton, New Jersey",
  note = "[QA93.K89 1981]",
  year = 1981 }
@article{Knill, author = "Oliver Knill",
  title = "Some Fundamental Theorems in Mathematics",
  year = "2018",
  url = "https://arxiv.org/abs/1807.08416" }
@book{Landau, author = "Edmund Landau",
  title = "Foundations of Analysis",
  publisher = "Chelsea Publishing Company",
  address = "New York",
  edition = "second",
  note = "[QA241.L2541 1960]",
  year = 1960 }
@article{Leblanc, author = "Hugues Leblanc",
  title = "On {M}eyer and {L}ambert's Quantificational Calculus {FQ}",
  journal = "The Journal of Symbolic Logic",
  volume = 33,
  year = 1968,
  note = "[QA.J87]",
  pages = "275--280" }
@article{Lejewski, author = "Czeslaw Lejewski",
  title = "On Implicational Definitions",
  journal = "Studia Logica",
  volume = 8,
  year = 1958,
  note = "[B18.P6.S933]",
  pages = "189--208" }
@book{Levy, author = "Azriel Levy",
  title = "Basic Set Theory",
  publisher = "Dover Publications",
  address = "Mineola, NY",
  year = "2002"
}
@book{Margaris, author = "Angelo Margaris",
  title = "First Order Mathematical Logic",
  publisher = "Blaisdell Publishing Company",
  address = "Waltham, Massachusetts",
  note = "[QA9.M327]",
  year = 1967}
@book{Manin, author = "Yu I. Manin",
  title = "A Course in Mathematical Logic",
  publisher = "Springer-Verlag",
  address = "New York",
  note = "[QA9.M29613]",
  year = "1977" }
@article{Mathias, author = "Adrian R. D. Mathias",
  title = "A Term of Length 4,523,659,424,929",
  journal = "Synthese",
  volume = 133,
  year = 2002,
  note = "[Q.S993]",
  pages = "75--86" }
@article{Megill, author = "Norman D. Megill",
  title = "A Finitely Axiomatized Formalization of Predicate Calculus
     with Equality",
  journal = "Notre Dame Journal of Formal Logic",
  volume = 36,
  year = 1995,
  note = "[QA.N914]",
  pages = "435--453" }
@unpublished{Megillc, author = "Norman D. Megill",
  title = "A Shorter Equivalent of the Axiom of Choice",
  month = "June",
  note = "Unpublished",
  year = 1991 }
@article{MegillBunder, author = "Norman D. Megill and Martin W.
    Bunder",
  title = "Weaker {D}-Complete Logics",
  journal = "Journal of the IGPL",
  volume = 4,
  year = 1996,
  pages = "215--225",
  note = "Available on the Web at
{\verb+http:+}\-{\verb+//www.mpi-sb.mpg.de+}\-{\verb+/igpl+}%
\-{\verb+/Journal+}\-{\verb+/V4-2+}\-{\verb+/#Megill+}"}
}
@book{Mendelson, author = "Elliott Mendelson",
  title = "Introduction to Mathematical Logic",
  edition = "second",
  publisher = "D. Van Nostrand Company, Inc.",
  address = "New York",
  note = "[QA9.M537 1979]",
  year = 1979 }
@article{Meredith, author = "David Meredith",
  title = "In Memoriam {C}arew {A}rthur {M}eredith (1904-1976)",
  journal = "Notre Dame Journal of Formal Logic",
  volume = 18,
  year = 1977,
  note = "[QA.N914]",
  pages = "513--516" }
@article{CAMeredith, author = "C. A. Meredith",
  title = "Single Axioms for the Systems ({C},{N}), ({C},{O}) and ({A},{N})
      of the Two-Valued Propositional Calculus",
  journal = "The Journal of Computing Systems",
  volume = 3,
  year = 1953,
  pages = "155--164" }
@article{Monk, author = "J. Donald Monk",
  title = "Provability With Finitely Many Variables",
  journal = "The Journal of Symbolic Logic",
  volume = 27,
  year = 1971,
  note = "[QA.J87]",
  pages = "353--358" }
@article{Monks, author = "J. Donald Monk",
  title = "Substitutionless Predicate Logic With Identity",
  journal = "Archiv f{\"{u}}r Mathematische Logik und Grundlagenfor\-schung",
  volume = 7,
  year = 1965,
  pages = "103--121" }
  %% Took out this from above to prevent LaTeX underfull warning:
  % note = "[QA.A673]",
@book{Moore, author = "A. W. Moore",
  title = "The Infinite",
  publisher = "Routledge",
  address = "New York",
  note = "[BD411.M59]",
  year = 1989}
@book{Munkres, author = "James R. Munkres",
  title = "Topology: A First Course",
  publisher = "Prentice-Hall, Inc.",
  address = "Englewood Cliffs, New Jersey",
  note = "[QA611.M82]",
  year = 1975}
@article{Nemesszeghy, author = "E. Z. Nemesszeghy and E. A. Nemesszeghy",
  title = "On Strongly Creative Definitions:  A Reply to {V}. {F}. {R}ickey",
  journal = "Logique et Analyse (N.\ S.)",
  year = 1977,
  volume = 20,
  note = "[BC.L832]",
  pages = "111--115" }
@unpublished{Nemeti, author = "N{\'{e}}meti, I.",
  title = "Algebraizations of Quantifier Logics, an Overview",
  note = "Version 11.4, preprint, Mathematical Institute, Budapest,
    1994.  A shortened version without proofs appeared in
    ``Algebraizations of quantifier logics, an introductory overview,''
   {\em Studia Logica}, 50:485--569, 1991 [B18.P6.S933]"}
@article{Pavicic, author = "M. Pavi{\v{c}}i{\'{c}}",
  title = "A New Axiomatization of Unified Quantum Logic",
  journal = "International Journal of Theoretical Physics",
  year = 1992,
  volume = 31,
  note = "[QC.I626]",
  pages = "1753 --1766" }
@book{Penrose, author = "Roger Penrose",
  title = "The Emperor's New Mind",
  publisher = "Oxford University Press",
  address = "New York",
  note = "[Q335.P415]",
  year = 1989 }
@book{PetersonI, author = "Ivars Peterson",
  title = "The Mathematical Tourist",
  publisher = "W. H. Freeman and Company",
  address = "New York",
  note = "[QA93.P475]",
  year = 1988 }
@article{Peterson, author = "Jeremy George Peterson",
  title = "An automatic theorem prover for substitution and detachment systems",
  journal = "Notre Dame Journal of Formal Logic",
  volume = 19,
  year = 1978,
  note = "[QA.N914]",
  pages = "119--122" }
@book{Quine, author = "Willard Van Orman Quine",
  title = "Set Theory and Its Logic",
  edition = "revised",
  publisher = "The Belknap Press of Harvard University Press",
  address = "Cambridge, Massachusetts",
  note = "[QA248.Q7 1969]",
  year = 1969 }
@article{Robinson, author = "J. A. Robinson",
  title = "A Machine-Oriented Logic Based on the Resolution Principle",
  journal = "Journal of the Association for Computing Machinery",
  year = 1965,
  volume = 12,
  pages = "23--41" }
@article{RobinsonT, author = "T. Thacher Robinson",
  title = "Independence of Two Nice Sets of Axioms for the Propositional
    Calculus",
  journal = "The Journal of Symbolic Logic",
  volume = 33,
  year = 1968,
  note = "[QA.J87]",
  pages = "265--270" }
@book{Rucker, author = "Rudy Rucker",
  title = "Infinity and the Mind:  The Science and Philosophy of the
    Infinite",
  publisher = "Bantam Books, Inc.",
  address = "New York",
  note = "[QA9.R79 1982]",
  year = 1982 }
@book{Russell, author = "Bertrand Russell",
  title = "Mysticism and Logic, and Other Essays",
  publisher = "Barnes \& Noble Books",
  address = "Totowa, New Jersey",
  note = "[B1649.R963.M9 1981]",
  year = 1981 }
@article{Russell2, author = "Bertrand Russell",
  title = "Recent Work on the Principles of Mathematics",
  journal = "International Monthly",
  volume = 4,
  year = 1901,
  pages = "84"}
@article{Schmidt, author = "Eric Schmidt",
  title = "Reductions in Norman Megill's axiom system for complex numbers",
  url = "http://us.metamath.org/downloads/schmidt-cnaxioms.pdf",
  year = "2012" }
@book{Shoenfield, author = "Joseph R. Shoenfield",
  title = "Mathematical Logic",
  publisher = "Addison-Wesley Publishing Company, Inc.",
  address = "Reading, Massachusetts",
  year = 1967,
  note = "[QA9.S52]" }
@book{Smullyan, author = "Raymond M. Smullyan",
  title = "Theory of Formal Systems",
  publisher = "Princeton University Press",
  address = "Princeton, New Jersey",
  year = 1961,
  note = "[QA248.5.S55]" }
@book{Solow, author = "Daniel Solow",
  title = "How to Read and Do Proofs:  An Introduction to Mathematical
    Thought Process",
  publisher = "John Wiley \& Sons",
  address = "New York",
  year = 1982,
  note = "[QA9.S577]" }
@book{Stark, author = "Harold M. Stark",
  title = "An Introduction to Number Theory",
  publisher = "Markham Publishing Company",
  address = "Chicago",
  note = "[QA241.S72 1978]",
  year = 1970 }
@article{Swart, author = "E. R. Swart",
  title = "The Philosophical Implications of the Four-Color Problem",
  journal = "American Mathematical Monthly",
  year = 1980,
  volume = 87,
  month = "November",
  note = "[QA.A5125]",
  pages = "697--707" }
@book{Szpiro, author = "George G. Szpiro",
  title = "Poincar{\'{e}}'s Prize: The Hundred-Year Quest to Solve One
    of Math's Greatest Puzzles",
  publisher = "Penguin Books Ltd",
  address = "London",
  note = "[QA43.S985 2007]",
  year = 2007}
@book{Takeuti, author = "Gaisi Takeuti and Wilson M. Zaring",
  title = "Introduction to Axiomatic Set Theory",
  edition = "second",
  publisher = "Springer-Verlag New York Inc.",
  address = "New York",
  note = "[QA248.T136 1982]",
  year = 1982}
@inproceedings{Tarski, author = "Alfred Tarski",
  title = "What is Elementary Geometry",
  pages = "16--29",
  booktitle = "The Axiomatic Method, with Special Reference to Geometry and
     Physics (Proceedings of an International Symposium held at the University
     of California, Berkeley, December 26, 1957 --- January 4, 1958)",
  editor = "Leon Henkin and Patrick Suppes and Alfred Tarski",
  year = 1959,
  publisher = "North-Holland Publishing Company",
  address = "Amsterdam"}
@article{Tarski1965, author = "Alfred Tarski",
  title = "A Simplified Formalization of Predicate Logic with Identity",
  journal = "Archiv f{\"{u}}r Mathematische Logik und Grundlagenforschung",
  volume = 7,
  year = 1965,
  note = "[QA.A673]",
  pages = "61--79" }
@book{Tymoczko,
  title = "New Directions in the Philosophy of Mathematics",
  editor = "Thomas Tymoczko",
  publisher = "Birkh{\"{a}}user Boston, Inc.",
  address = "Boston",
  note = "[QA8.6.N48 1986]",
  year = 1986 }
@incollection{Wang,
  author = "Hao Wang",
  title = "Theory and Practice in Mathematics",
  pages = "129--152",
  booktitle = "New Directions in the Philosophy of Mathematics",
  editor = "Thomas Tymoczko",
  publisher = "Birkh{\"{a}}user Boston, Inc.",
  address = "Boston",
  note = "[QA8.6.N48 1986]",
  year = 1986 }
@manual{Webster,
  title = "Webster's New Collegiate Dictionary",
  organization = "G. \& C. Merriam Co.",
  address = "Springfield, Massachusetts",
  note = "[PE1628.W4M4 1977]",
  year = 1977 }
@manual{Whitehead, author = "Alfred North Whitehead",
  title = "An Introduction to Mathematics",
  year = 1911 }
@book{PM, author = "Alfred North Whitehead and Bertrand Russell",
  title = "Principia Mathematica",
  edition = "second",
  publisher = "Cambridge University Press",
  address = "Cambridge",
  year = "1927",
  note = "(3 vols.) [QA9.W592 1927]" }
@article{DBLP:journals/corr/Whalen16,
  author    = {Daniel Whalen},
  title     = {Holophrasm: a neural Automated Theorem Prover for higher-order logic},
  journal   = {CoRR},
  volume    = {abs/1608.02644},
  year      = {2016},
  url       = {http://arxiv.org/abs/1608.02644},
  archivePrefix = {arXiv},
  eprint    = {1608.02644},
  timestamp = {Mon, 13 Aug 2018 16:46:19 +0200},
  biburl    = {https://dblp.org/rec/bib/journals/corr/Whalen16},
  bibsource = {dblp computer science bibliography, https://dblp.org} }
@article{Wiedijk-revisited,
  author = {Freek Wiedijk},
  title = {The QED Manifesto Revisited},
  year = {2007},
  url = {http://mizar.org/trybulec65/8.pdf} }
@book{Wolfram,
  author = "Stephen Wolfram",
  title = "Mathematica:  A System for Doing Mathematics by Computer",
  edition = "second",
  publisher = "Addison-Wesley Publishing Co.",
  address = "Redwood City, California",
  note = "[QA76.95.W65 1991]",
  year = 1991 }
@book{Wos, author = "Larry Wos and Ross Overbeek and Ewing Lusk and Jim Boyle",
  title = "Automated Reasoning:  Introduction and Applications",
  edition = "second",
  publisher = "McGraw-Hill, Inc.",
  address = "New York",
  note = "[QA76.9.A96.A93 1992]",
  year = 1992 }

%
%
%[1] Church, Alonzo, Introduction to Mathematical Logic,
% Volume 1, Princeton University Press, Princeton, N. J., 1956.
%
%[2] Cohen, Paul J., Set Theory and the Continuum Hypothesis,
% W. A. Benjamin, Inc., Reading, Mass., 1966.
%
%[3] Hamilton, Alan G., Logic for Mathematicians, Cambridge
% University Press,
% Cambridge, 1988.

%[6] Kleene, Stephen Cole, Introduction to Metamathematics, D.  Van
% Nostrand Company, Inc., Princeton (1952).

%[13] Tarski, Alfred, "A simplified formalization of predicate
% logic with identity," Archiv fur Mathematische Logik und
% Grundlagenforschung, vol. 7 (1965), pp. 61-79.

%[14] Tarski, Alfred and Steven Givant, A Formalization of Set
% Theory Without Variables, American Mathematical Society Colloquium
% Publications, vol. 41, American Mathematical Society,
% Providence, R. I., 1987.

%[15] Zeman, J. J., Modal Logic, Oxford University Press, Oxford, 1973.
\end{filecontents}
% --------------------------- End of metamath.bib -----------------------------


%Book: Metamath
%Author:  Norman Megill Email:  nm at alum.mit.edu
%Author:  David A. Wheeler Email:  dwheeler at dwheeler.com

% A book template example
% http://www.stsci.edu/ftp/software/tex/bookstuff/book.template

\documentclass[leqno]{book} % LaTeX 2e. 10pt. Use [leqno,12pt] for 12pt
% hyperref 2002/05/27 v6.72r  (couldn't get pagebackref to work)
\usepackage[plainpages=false,pdfpagelabels=true]{hyperref}

\usepackage{needspace}     % Enable control over page breaks
\usepackage{breqn}         % automatic equation breaking
\usepackage{microtype}     % microtypography, reduces hyphenation

% Packages for flexible tables.  We need to be able to
% wrap text within a cell (with automatically-determined widths) AND
% split a table automatically across multiple pages.
% * "tabularx" wraps text in cells but only 1 page
% * "longtable" goes across pages but by itself is incompatible with tabularx
% * "ltxtable" combines longtable and tabularx, but table contents
%    must be in a separate file.
% * "ltablex" combines tabularx and longtable - must install specially
% * "booktabs" is recommended as a way to improve the look of tables,
%   but doesn't add these capabilities.
% * "tabu" much more capable and seems to be recommended. So use that.

\usepackage{makecell}      % Enable forced line splits within a table cell
\usepackage{longtable}     % Enable multi-page tables
\usepackage{tabu}          % Multi-page tables with wrapped text in a cell

% You can find more Tex packages using commands like:
% tlmgr search --file tabu.sty
% find /usr/share/texmf-dist/ -name '*tab*'
%
%%%%%%%%%%%%%%%%%%%%%%%%%%%%%%%%%%%%%%%%%%%%%%%%%%%%%%%%%%%%%%%%%%%%%%%%%%%%
% Uncomment the next 3 lines to suppress boxes and colors on the hyperlinks
%%%%%%%%%%%%%%%%%%%%%%%%%%%%%%%%%%%%%%%%%%%%%%%%%%%%%%%%%%%%%%%%%%%%%%%%%%%%
%\hypersetup{
%colorlinks,citecolor=black,filecolor=black,linkcolor=black,urlcolor=black
%}
%
\usepackage{realref}

% Restarting page numbers: try?
%   \printglossary
%   \cleardoublepage
%   \pagenumbering{arabic}
%   \setcounter{page}{1}    ???needed
%   \include{chap1}

% not used:
% \def\R2Lurl#1#2{\mbox{\href{#1}\texttt{#2}}}

\usepackage{amssymb}

% Version 1 of book: margins: t=.4, b=.2, ll=.4, rr=.55
% \usepackage{anysize}
% % \papersize{<height>}{<width>}
% % \marginsize{<left>}{<right>}{<top>}{<bottom>}
% \papersize{9in}{6in}
% % l/r 0.6124-0.6170 works t/b 0.2418-0.3411 = 192pp. 0.2926-03118=exact
% \marginsize{0.7147in}{0.5147in}{0.4012in}{0.2012in}

\usepackage{anysize}
% \papersize{<height>}{<width>}
% \marginsize{<left>}{<right>}{<top>}{<bottom>}
\papersize{9in}{6in}
% l/r 0.85in&0.6431-0.6539 works t/b ?-?
%\marginsize{0.85in}{0.6485in}{0.55in}{0.35in}
\marginsize{0.8in}{0.65in}{0.5in}{0.3in}

% \usepackage[papersize={3.6in,4.8in},hmargin=0.1in,vmargin={0.1in,0.1in}]{geometry}  % page geometry
\usepackage{special-settings}

\raggedbottom
\makeindex

\begin{document}
% Discourage page widows and orphans:
\clubpenalty=300
\widowpenalty=300

%%%%%%% load in AMS fonts %%%%%%% % LaTeX 2.09 - obsolete in LaTeX 2e
%\input{amssym.def}
%\input{amssym.tex}
%\input{c:/texmf/tex/plain/amsfonts/amssym.def}
%\input{c:/texmf/tex/plain/amsfonts/amssym.tex}

\bibliographystyle{plain}
\pagenumbering{roman}
\pagestyle{headings}

\thispagestyle{empty}

\hfill
\vfill

\begin{center}
{\LARGE\bf Metamath} \\
\vspace{1ex}
{\large A Computer Language for Mathematical Proofs} \\
\vspace{7ex}
{\large Norman Megill} \\
\vspace{7ex}
with extensive revisions by \\
\vspace{1ex}
{\large David A. Wheeler} \\
\vspace{7ex}
% Printed date. If changing the date below, also fix the date at the beginning.
2019-05-24
\end{center}

\vfill
\hfill

\newpage
\thispagestyle{empty}

\hfill
\vfill

\begin{center}
$\sim$\ {\sc Public Domain}\ $\sim$

\vspace{2ex}
This book (including its later revisions)
has been released into the Public Domain by Norman Megill per the
Creative Commons CC0 1.0 Universal (CC0 1.0) Public Domain Dedication
(\url{https://creativecommons.org/publicdomain/zero/1.0/}).
David A. Wheeler has done the same.
The public domain release applies worldwide.  In case this is not
legally possible, the right is granted to use the work for any purpose,
without any conditions, unless such conditions are required by law.

\vspace{3ex}
Several short, attributed quotations from copyrighted works
appear in this book under the ``fair use'' provision of Section 107 of
the United States Copyright Act (Title 17 of the {\em United States
Code}).  The public-domain status of this book is not applicable to
those quotations.

\vspace{3ex}
Any trademarks used in this book are the property of their owners.

% QA76.9.L63.M??

% \vspace{1ex}
%
% \vspace{1ex}
% {\small Permission is granted to make and distribute verbatim copies of this
% book
% provided the copyright notice and this
% permission notice are preserved on all copies.}
%
% \vspace{1ex}
% {\small Permission is granted to copy and distribute modified versions of this
% book under the conditions for verbatim copying, provided that the
% entire
% resulting derived work is distributed under the terms of a permission
% notice
% identical to this one.}
%
% \vspace{1ex}
% {\small Permission is granted to copy and distribute translations of this
% book into another language, under the above conditions for modified
% versions,
% except that this permission notice may be stated in a translation
% approved by the
% author.}
%
% \vspace{1ex}
% %{\small   For a copy of the \LaTeX\ source files for this book, contact
% %the author.} \\
% \ \\
% \ \\

\vspace{7ex}
% ISBN: 1-4116-3724-0 \\
ISBN: 978-1-4116-3724-5 \\
{\ } \\
Lulu Press \\
Morrisville, North Carolina\\
USA


\hfill
\vfill

Norman Megill\\ 93 Bridge St., Lexington, MA 02421 \\
E-mail address: \texttt{nm{\char`\@}alum.mit.edu} \\
\vspace{7ex}
David A. Wheeler \\
E-mail address: \texttt{dwheeler{\char`\@}dwheeler.com} \\
% See notes added at end of Preface for revision history. \\
% For current information on the Metamath software see \\
\vspace{7ex}
\url{http://metamath.org}
\end{center}

% \newpage
% \thispagestyle{empty}
%
% \hfill
% \vfill
%
% \begin{center}
% {\it To my son Robin Dwight Megill}
% \end{center}
%
% \vfill
% \hfill
%
% \newpage

\tableofcontents
%\listoftables

\chapter*{Preface}
\markboth{PREFACE}{PREFACE}
\addcontentsline{toc}{section}{Preface}


% (For current information, see the notes added at the
% end of this preface on p.~\pageref{note2002}.)

\subsubsection{Overview}

Metamath\index{Metamath} is a computer language and an associated computer
program for archiving, verifying, and studying mathematical proofs at a very
detailed level.  The Metamath language incorporates no mathematics per se but
treats all mathematical statements as mere sequences of symbols.  You provide
Metamath with certain special sequences (axioms) that tell it what rules
of inference are allowed.  Metamath is not limited to any specific field of
mathematics.  The Metamath language is simple and robust, with an
almost total absence of hard-wired syntax, and
I\footnote{Unless otherwise noted, the words
``I,'' ``me,'' and ``my'' refer to Norman Megill\index{Megill, Norman}, while
``we,'' ``us,'' and ``our'' refer to Norman Megill and
David A. Wheeler\index{Wheeler, David A.}.}
believe that it
provides about the simplest possible framework that allows essentially all of
mathematics to be expressed with absolute rigor.

% index test
%\newcommand{\nn}[1]{#1n}
%\index{aaa@bbb}
%\index{abc!def}
%\index{abd|see{qqq}}
%\index{abe|nn}
%\index{abf|emph}
%\index{abg|(}
%\index{abg|)}

Using the Metamath language, you can build formal or mathematical
systems\index{formal system}\footnote{A formal or mathematical system consists
of a collection of symbols (such as $2$, $4$, $+$ and $=$), syntax rules that
describe how symbols may be combined to form a legal expression (called a
well-formed formula or {\em wff}, pronounced ``whiff''), some starting wffs
called axioms, and inference rules that describe how theorems may be derived
(proved) from the axioms.  A theorem is a mathematical fact such as $2+2=4$.
Strictly speaking, even an obvious fact such as this must be proved from
axioms to be formally acceptable to a mathematician.}\index{theorem}
\index{axiom}\index{rule}\index{well-formed formula (wff)} that involve
inferences from axioms.  Although a database is provided
that includes a recommended set of axioms for standard mathematics, if you
wish you can supply your own symbols, syntax, axioms, rules, and definitions.

The name ``Metamath'' was chosen to suggest that the language provides a
means for {\em describing} mathematics rather than {\em being} the
mathematics itself.  Actually in some sense any mathematical language is
metamathematical.  Symbols written on paper, or stored in a computer,
are not mathematics itself but rather a way of expressing mathematics.
For example ``7'' and ``VII'' are symbols for denoting the number seven
in Arabic and Roman numerals; neither {\em is} the number seven.

If you are able to understand and write computer programs, you should be able
to follow abstract mathematics with the aid of Metamath.  Used in conjunction
with standard textbooks, Metamath can guide you step-by-step towards an
understanding of abstract mathematics from a very rigorous viewpoint, even if
you have no formal abstract mathematics background.  By using a single,
consistent notation to express proofs, once you grasp its basic concepts
Metamath provides you with the ability to immediately follow and dissect
proofs even in totally unfamiliar areas.

Of course, just being able follow a proof will not necessarily give you an
intuitive familiarity with mathematics.  Memorizing the rules of chess does not
give you the ability to appreciate the game of a master, and knowing how the
notes on a musical score map to piano keys does not give you the ability to
hear in your head how it would sound.  But each of these can be a first step.

Metamath allows you to explore proofs in the sense that you can see the
theorem referenced at any step expanded in as much detail as you want, right
down to the underlying axioms of logic and set theory (in the case of the set
theory database provided).  While Metamath will not replace the higher-level
understanding that can only be acquired through exercises and hard work, being
able to see how gaps in a proof are filled in can give you increased
confidence that can speed up the learning process and save you time when you
get stuck.

The Metamath language breaks down a mathematical proof into its tiniest
possible parts.  These can be pieced together, like interlocking
pieces in a puzzle, only in a way that produces correct and absolutely rigorous
mathematics.

The nature of Metamath\index{Metamath} enforces very precise mathematical
thinking, similar to that involved in writing a computer program.  A crucial
difference, though, is that once a proof is verified (by the Metamath program)
to be correct, it is definitely correct; it can never have a hidden
``bug.''\index{computer program bugs}  After getting used to the kind of rigor
and accuracy provided by Metamath, you might even be tempted to
adopt the attitude that a proof should never be considered correct until it
has been verified by a computer, just as you would not completely trust a
manual calculation until you have verified it on a
calculator.

My goal
for Metamath was a system for describing and verifying
mathematics that is completely universal yet conceptually as simple as
possible.  In approaching mathematics from an axiomatic, formal viewpoint, I
wanted Metamath to be able to handle almost any mathematical system, not
necessarily with ease, but at least in principle and hopefully in practice. I
wanted it to verify proofs with absolute rigor, and for this reason Metamath
is what might be thought of as a ``compile-only'' language rather than an
algorithmic or Turing-machine language (Pascal, C, Prolog, Mathematica,
etc.).  In other words, a database written in the Metamath
language doesn't ``do'' anything; it merely exhibits mathematical knowledge
and permits this knowledge to be verified as being correct.  A program in an
algorithmic language can potentially have hidden bugs\index{computer program
bugs} as well as possibly being hard to understand.  But each token in a
Metamath database must be consistent with the database's earlier
contents according to simple, fixed rules.
If a database is verified
to be correct,\footnote{This includes
verification that a sequential list of proof steps results in the specified
theorem.} then the mathematical content is correct if the
verifier is correct and the axioms are correct.
The verification program could be incorrect, but the verification algorithm
is relatively simple (making it unlikely to be implemented incorrectly
by the Metamath program),
and there are over a dozen Metamath database verifiers
written by different people in different programming languages
(so these different verifiers can act as multiple reviewers of a database).
The most-used Metamath database, the Metamath Proof Explorer
(aka \texttt{set.mm}\index{set theory database (\texttt{set.mm})}%
\index{Metamath Proof Explorer}),
is currently verified by four different Metamath verifiers written by
four different people in four different languages, including the
original Metamath program described in this book.
The only ``bugs'' that can exist are in the statement of the axioms,
for example if the axioms are inconsistent (a famous problem shown to be
unsolvable by G\"{o}del's incompleteness theorem\index{G\"{o}del's
incompleteness theorem}).
However, real mathematical systems have very few axioms, and these can
be carefully studied.
All of this provides extraordinarily high confidence that the verified database
is in fact correct.

The Metamath program
doesn't prove theorems automatically but is designed to verify proofs
that you supply to it.
The underlying Metamath language is completely general and has no built-in,
preconceived notions about your formal system\index{formal system}, its logic
or its syntax.
For constructing proofs, the Metamath program has a Proof Assistant\index{Proof
Assistant} which helps you fill in some of a proof step's details, shows you
what choices you have at any step, and verifies the proof as you build it; but
you are still expected to provide the proof.

There are many other programs that can process or generate information
in the Metamath language, and more continue to be written.
This is in part because the Metamath language itself is very simple
and intentionally easy to automatically process.
Some programs, such as \texttt{mmj2}\index{mmj2}, include a proof assistant
that can automate some steps beyond what the Metamath program can do.
Mario Carneiro has developed an algorithm for converting proofs from
the OpenTheory interchange format, which can be translated to and from
any of the HOL family of proof languages (HOL4, HOL Light, ProofPower,
and Isabelle), into the
Metamath language.\cite{DBLP:journals/corr/Carneiro14}\index{Carneiro, Mario}.
Daniel Whalen has developed Holophrasm, which can automatically
prove many Metamath proofs using
machine learning\index{machine learning}\index{artificial intelligence}
approaches
(including multiple neural networks\index{neural networks})\cite{DBLP:journals/corr/Whalen16}\index{Whalen, Daniel}.
However,
a discussion of these other programs is beyond the scope of this book.

Like most computer languages, the Metamath\index{Metamath} language uses the
standard ({\sc ascii}) characters on a computer keyboard, so it cannot
directly represent many of the special symbols that mathematicians use.  A
useful feature of the Metamath program is its ability to convert its notation
into the \LaTeX\ typesetting language.\index{latex@{\LaTeX}}  This feature
lets you convert the {\sc ascii} tokens you've defined into standard
mathematical symbols, so you end up with symbols and formulas you are familiar
with instead of somewhat cryptic {\sc ascii} representations of them.
The Metamath program can also generate HTML\index{HTML}, making it easy
to view results on the web and to see related information by using
hypertext links.

Metamath is probably conceptually different from anything you've seen
before and some aspects may take some getting used to.  This book will
help you decide whether Metamath suits your specific needs.

\subsubsection{Setting Your Expectations}
It is important for you to understand what Metamath\index{Metamath} is and is
not.  As mentioned, the Metamath program
is {\em not} an automated theorem prover but
rather a proof verifier.  Developing a database can be tedious, hard work,
especially if you want to make the proofs as short as possible, but it becomes
easier as you build up a collection of useful theorems.  The purpose of
Metamath is simply to document existing mathematics in an absolutely rigorous,
computer-verifiable way, not to aid directly in the creation of new
mathematics.  It also is not a magic solution for learning abstract
mathematics, although it may be helpful to be able to actually see the implied
rigor behind what you are learning from textbooks, as well as providing hints
to work out proofs that you are stumped on.

As of this writing, a sizable set theory database has been developed to
provide a foundation for many fields of mathematics, but much more work would
be required to develop useful databases for specific fields.

Metamath\index{Metamath} ``knows no math;'' it just provides a framework in
which to express mathematics.  Its language is very small.  You can define two
kinds of symbols, constants\index{constant} and variables\index{variable}.
The only thing Metamath knows how to do is to substitute strings of symbols
for the variables\index{substitution!variable}\index{variable substitution} in
an expression based on instructions you provide it in a proof, subject to
certain constraints you specify for the variables.  Even the decimal
representation of a number is merely a string of certain constants (digits)
which together, in a specific context, correspond to whatever mathematical
object you choose to define for it; unlike other computer languages, there is
no actual number stored inside the computer.  In a proof, you in effect
instruct Metamath what symbol substitutions to make in previous axioms or
theorems and join a sequence of them together to result in the desired
theorem.  This kind of symbol manipulation captures the essence of mathematics
at a preaxiomatic level.

\subsubsection{Metamath and Mathematical Literature}

In advanced mathematical literature, proofs are usually presented in the form
of short outlines that often only an expert can follow.  This is partly out of
a desire for brevity, but it would also be unwise (even if it were practical)
to present proofs in complete formal detail, since the overall picture would
be lost.\index{formal proof}

A solution I envision\label{envision} that would allow mathematics to remain
acceptable to the expert, yet increase its accessibility to non-specialists,
consists of a combination of the traditional short, informal proof in print
accompanied by a complete formal proof stored in a computer database.  In an
analogy with a computer program, the informal proof is like a set of comments
that describe the overall reasoning and content of the proof, whereas the
computer database is like the actual program and provides a means for anyone,
even a non-expert, to follow the proof in as much detail as desired, exploring
it back through layers of theorems (like subroutines that call other
subroutines) all the way back to the axioms of the theory.  In addition, the
computer database would have the advantage of providing absolute assurance
that the proof is correct, since each step can be verified automatically.

There are several other approaches besides Metamath to a project such
as this.  Section~\ref{proofverifiers} discusses some of these.

To us, a noble goal would be a database with hundreds of thousands of
theorems and their computer-verifiable proofs, encompassing a significant
fraction of known mathematics and available for instant access.
These would be fully verified by multiple independently-implemented verifiers,
to provide extremely high confidence that the proofs are completely correct.
The database would allow people to investigate whatever details they were
interested in, so that they could confirm whatever portions they wished.
Whether or not Metamath is an appropriate choice remains to be seen, but in
principle we believe it is sufficient.

\subsubsection{Formalism}

Over the past fifty years, a group of French mathematicians working
collectively under the pseudonym of Bourbaki\index{Bourbaki, Nicolas} have
co-authored a series of monographs that attempt to rigorously and
consistently formalize large bodies of mathematics from foundations.  On the
one hand, certainly such an effort has its merits; on the other hand, the
Bourbaki project has been criticized for its ``scholasticism'' and
``hyperaxiomatics'' that hide the intuitive steps that lead to the results
\cite[p.~191]{Barrow}\index{Barrow, John D.}.

Metamath unabashedly carries this philosophy to its extreme and no doubt is
subject to the same kind of criticism.  Nonetheless I think that in
conjunction with conventional approaches to mathematics Metamath can serve a
useful purpose.  The Bourbaki approach is essentially pedagogic, requiring the
reader to become intimately familiar with each detail in a very large
hierarchy before he or she can proceed to the next step.  The difference with
Metamath is that the ``reader'' (user) knows that all details are contained in
its computer database, available as needed; it does not demand that the user
know everything but conveniently makes available those portions that are of
interest.  As the body of all mathematical knowledge grows larger and larger,
no one individual can have a thorough grasp of its entirety.  Metamath
can finalize and put to rest any questions about the validity of any part of it
and can make any part of it accessible, in principle, to a non-specialist.

\subsubsection{A Personal Note}
Why did I develop Metamath\index{Metamath}?  I enjoy abstract mathematics, but
I sometimes get lost in a barrage of definitions and start to lose confidence
that my proofs are correct.  Or I reach a point where I lose sight of how
anything I'm doing relates to the axioms that a theory is based on and am
sometimes suspicious that there may be some overlooked implicit axiom
accidentally introduced along the way (as happened historically with Euclidean
geometry\index{Euclidean geometry}, whose omission of Pasch's
axiom\index{Pasch's axiom} went unnoticed for 2000 years
\cite[p.~160]{Davis}!). I'm also somewhat lazy and wish to avoid the effort
involved in re-verifying the gaps in informal proofs ``left to the reader;'' I
prefer to figure them out just once and not have to go through the same
frustration a year from now when I've forgotten what I did.  Metamath provides
better recovery of my efforts than scraps of paper that I can't
decipher anymore.  But mostly I find very appealing the idea of rigorously
archiving mathematical knowledge in a computer database, providing precision,
certainty, and elimination of human error.

\subsubsection{Note on Bibliography and Index}

The Bibliography usually includes the Library of Congress classification
for a work to make it easier for you to find it in on a university
library shelf.  The Index has author references to pages where their works
are cited, even though the authors' names may not appear on those pages.

\subsubsection{Acknowledgments}

Acknowledgments are first due to my wife, Deborah (who passed away on
September 4, 1998), for critiquing the manu\-script but most of all for
her patience and support.  I also wish to thank Joe Wright, Richard
Becker, Clarke Evans, Buddha Buck, and Jeremy Henty for helpful
comments.  Any errors, omissions, and other shortcomings are of course
my responsibility.

\subsubsection{Note Added June 22, 2005}\label{note2002}

The original, unpublished version of this book was written in 1997 and
distributed via the web.  The present edition has been updated to
reflect the current Metamath program and databases, as well as more
current {\sc url}s for Internet sites.  Thanks to Josh
Purinton\index{Purinton, Josh}, One Hand
Clapping, Mel L.\ O'Cat, and Roy F. Longton for pointing out
typographical and other errors.  I have also benefitted from numerous
discussions with Raph Levien\index{Levien, Raph}, who has extended
Metamath's philosophy of rigor to result in his {\em
Ghilbert}\index{Ghilbert} proof language (\url{http://ghilbert.org}).

Robert (Bob) Solovay\index{Solovay, Robert} communicated a new result of
A.~R.~D.~Mathias on the system of Bourbaki, and the text has been
updated accordingly (p.~\pageref{bourbaki}).

Bob also pointed out a clarification of the literature regarding
category theory and inaccessible cardinals\index{category
theory}\index{cardinal, inaccessible} (p.~\pageref{categoryth}),
and a misleading statement was removed from the text.  Specifically,
contrary to a statement in previous editions, it is possible to express
``There is a proper class of inaccessible cardinals'' in the language of
ZFC.  This can be done as follows:  ``For every set $x$ there is an
inaccessible cardinal $\kappa$ such that $\kappa$ is not in $x$.''
Bob writes:\footnote{Private communication, Nov.~30, 2002.}
\begin{quotation}
     This axiom is how Grothendieck presents category theory.  To each
inaccessible cardinal $\kappa$ one associates a Grothendieck universe
\index{Grothendieck, Alexander} $U(\kappa)$.  $U(\kappa)$ consists of
those sets which lie in a transitive set of cardinality less than
$\kappa$.  Instead of the ``category of all groups,'' one works relative
to a universe [considering the category of groups of cardinality less
than $\kappa$].  Now the category whose objects are all categories
``relative to the universe $U(\kappa)$'' will be a category not
relative to this universe but to the next universe.

     All of the things category theorists like to do can be done in this
framework.  The only controversial point is whether the Grothen\-dieck
axiom is too strong for the needs of category theorists.  Mac Lane
\index{Mac Lane, Saunders} argues that ``one universe is enough'' and
Feferman\index{Feferman, Solomon} has argued that one can get by with
ordinary ZFC.  I don't find Feferman's arguments persuasive.  Mac Lane
may be right, but when I think about category theory I do it \`{a} la
Grothendieck.

        By the way Mizar\index{Mizar} adds the axiom ``there is a proper
class of inaccessibles'' precisely so as to do category theory.
\end{quotation}

The most current information on the Metamath program and databases can
always be found at \url{http://metamath.org}.


\subsubsection{Note Added June 24, 2006}\label{note2006}

The Metamath spec was restricted slightly to make parsers easier to
write.  See the footnote on p.~\pageref{namespace}.

%\subsubsection{Note Added July 24, 2006}\label{note2006b}
\subsubsection{Note Added March 10, 2007}\label{note2006b}

I am grateful to Anthony Williams\index{Williams, Anthony} for writing
the \LaTeX\ package called {\tt realref.sty} and contributing it to the
public domain.  This package allows the internal hyperlinks in a {\sc
pdf} file to anchor to specific page numbers instead of just section
titles, making the navigation of the {\sc pdf} file for this book much
more pleasant and ``logical.''

A typographical error found by Martin Kiselkov was corrected.
A confusing remark about unification was deleted per suggestion of
Mel O'Cat.

\subsubsection{Note Added May 27, 2009}\label{note2009}

Several typos found by Kim Sparre were corrected.  A note was added that
the Poincar\'{e} conjecture has been proved (p.~\pageref{poincare}).

\subsubsection{Note Added Nov. 17, 2014}\label{note2014}

The statement of the Schr\"{o}der--Bernstein theorem was corrected in
Section~\ref{trust}.  Thanks to Bob Solovay for pointing out the error.

\subsubsection{Note Added May 25, 2016}\label{note2016}

Thanks to Jerry James for correcting 16 typos.

\subsubsection{Note Added February 25, 2019}\label{note201902}

David A. Wheeler\index{Wheeler, David A.}
made a large number of improvements and updates,
in coordination with Norman Megill.
The predicate calculus axioms were renumbered, and the text makes
it clear that they are based on Tarski's system S2;
the one slight deviation in axiom ax-6 is explained and justified.
The real and complex number axioms were modified to be consistent with
\texttt{set.mm}\index{set theory database (\texttt{set.mm})}%
\index{Metamath Proof Explorer}.
Long-awaited specification changes ``1--8'' were made,
which clarified previously ambiguous points.
Some errors in the text involving \texttt{\$f} and
\texttt{\$d} statements were corrected (the spec was correct, but
the in-book explanations unintentionally contradicted the spec).
We now have a system for automatically generating narrow PDFs,
so that those with smartphones can have easy access to the current
version of this document.
A new section on deduction was added;
it discusses the standard deduction theorem,
the weak deduction theorem,
deduction style, and natural deduction.
Many minor corrections (too numerous to list here) were also made.

\subsubsection{Note Added March 7, 2019}\label{note201903}

This added a description of the Matamath language syntax in
Extended Backus--Naur Form (EBNF)\index{Extended Backus--Naur Form}\index{EBNF}
in Appendix \ref{BNF}, added a brief explanation about typecodes,
inserted more examples in the deduction section,
and added a variety of smaller improvements.

\subsubsection{Note Added April 7, 2019}\label{note201904}

This version clarified the proper substitution notation, improved the
discussion on the weak deduction theorem and natural deduction,
documented the \texttt{undo} command, updated the information on
\texttt{write source}, changed the typecode
from \texttt{set} to \texttt{setvar} to be consistent with the current
version of \texttt{set.mm}, added more documentation about comment markup
(e.g., documented how to create headings), and clarified the
differences between various assertion forms (in particular deduction form).

\subsubsection{Note Added May 20, 2019}\label{note201905}

This version fixes various small issues and added specific examples
of forms (deduction form, inference form, and closed form).

\chapter{Introduction}
\pagenumbering{arabic}

\begin{quotation}
  {\em {\em I.M.:}  No, no.  There's nothing subjective about it!  Everybody
knows what a proof is.  Just read some books, take courses from a competent
mathematician, and you'll catch on.

{\em Student:}  Are you sure?

{\em I.M.:}  Well---it is possible that you won't, if you don't have any
aptitude for it.  That can happen, too.

{\em Student:}  Then {\em you} decide what a proof is, and if I don't learn
to decide in the same way, you decide I don't have any aptitude.

{\em I.M.:}  If not me, then who?}
    \flushright\sc  ``The Ideal Mathematician''
    \index{Davis, Phillip J.}
    \footnote{\cite{Davis}, p.~40.}\\
\end{quotation}

In the past century, brilliant mathematicians have discovered almost
unimaginably profound results that rank among the crowning intellectual
achievements of mankind.  However, there is a sense in which modern abstract
mathematics is behind the times, stuck in an era before computers existed.
While no one disputes the remarkable results that have been achieved,
communicating these results in a precise way to the uninitiated is virtually
impossible.  To describe these results, a terse informal language is used which
despite its elegance is very difficult to learn.  This informal language is not
imprecise, far from it, but rather it often has omitted detail
and symbols with hidden context that are
implicitly understood by an expert but few others.  Extremely complex technical
meanings are associated with innocent-sounding English words such as
``compact'' and ``measurable'' that barely hint at what is actually being
said.  Anyone who does not keep the precise technical meaning constantly in
mind is bound to fail, and acquiring the ability to do this can be achieved
only through much practice and hard work.  Only the few who complete the
painful learning experience can join the small in-group of pure
mathematicians.  The informal language effectively cuts off the true nature of
their knowledge from most everyone else.

Metamath\index{Metamath} makes abstract mathematics more concrete.  It allows
a computer to keep track of the complexity associated with each word or symbol
with absolute rigor.  You can explore this complexity at your leisure, to
whatever degree you desire.  Whether or not you believe that concepts such as
infinity actually ``exist'' outside of the mind, Metamath lets you get to the
foundation for what's really being said.

Metamath also enables completely rigorous and thorough proof verification.
Its language is simple enough so that you
don't have to rely on the authority of experts but can verify the results
yourself, step by step.  If you want to attempt to derive your own results,
Metamath will not let you make a mistake in reasoning.
Even professional mathematicians make mistakes; Metamath makes it possible
to thoroughly verify that proofs are correct.

Metamath\index{Metamath} is a computer language and an associated computer
program for archiving, verifying, and studying mathematical proofs at a very
detailed level.
The Metamath language
describes formal\index{formal system} mathematical
systems and expresses proofs of theorems in those systems.  Such a language
is called a metalanguage\index{metalanguage} by mathematicians.
The Metamath program is a computer program that verifies
proofs expressed in the Metamath language.
The Metamath program does not have the built-in
ability to make logical inferences; it just makes a series of symbol
substitutions according to instructions given to it in a proof
and verifies that the result matches the expected theorem.  It makes logical
inferences based only on rules of logic that are contained in a set of
axioms\index{axiom}, or first principles, that you provide to it as the
starting point for proofs.

The complete specification of the Metamath language is only four pages long
(Section~\ref{spec}, p.~\pageref{spec}).  Its simplicity may at first make you
wonder how it can do much of anything at all.  But in fact the kinds of
symbol manipulations it performs are the ones that are implicitly done in all
mathematical systems at the lowest level.  You can learn it relatively quickly
and have complete confidence in any mathematical proof that it verifies.  On
the other hand, it is powerful and general enough so that virtually any
mathematical theory, from the most basic to the deeply abstract, can be
described with it.

Although in principle Metamath can be used with any
kind of mathematics, it is best suited for abstract or ``pure'' mathematics
that is mostly concerned with theorems and their proofs, as opposed to the
kind of mathematics that deals with the practical manipulation of numbers.
Examples of branches of pure mathematics are logic\index{logic},\footnote{Logic
is the study of statements that are universally true regardless of the objects
being described by the statements.  An example is the statement, ``if $P$
implies $Q$, then either $P$ is false or $Q$ is true.''} set theory\index{set
theory},\footnote{Set theory is the study of general-purpose mathematical objects called
``sets,'' and from it essentially all of mathematics can be derived.  For
example, numbers can be defined as specific sets, and their properties
can be explored using the tools of set theory.} number theory\index{number
theory},\footnote{Number theory deals with the properties of positive and
negative integers (whole numbers).} group theory\index{group
theory},\footnote{Group theory studies the properties of mathematical objects
called groups that obey a simple set of axioms and have properties of symmetry
that make them useful in many other fields.} abstract algebra\index{abstract
algebra},\footnote{Abstract algebra includes group theory and also studies
groups with additional properties that qualify them as ``rings'' and
``fields.''  The set of real numbers is a familiar example of a field.},
analysis\index{analysis} \index{real and complex numbers}\footnote{Analysis is
the study of real and complex numbers.} and
topology\index{topology}.\footnote{One area studied by topology are properties
that remain unchanged when geometrical objects undergo stretching
deformations; for example a doughnut and a coffee cup each have one hole (the
cup's hole is in its handle) and are thus considered topologically
equivalent.  In general, though, topology is the study of abstract
mathematical objects that obey a certain (surprisingly simple) set of axioms.
See, for example, Munkres \cite{Munkres}\index{Munkres, James R.}.} Even in
physics, Metamath could be applied to certain branches that make use of
abstract mathematics, such as quantum logic\index{quantum logic} (used to study
aspects of quantum mechanics\index{quantum mechanics}).

On the other hand, Metamath\index{Metamath} is less suited to applications
that deal primarily with intensive numeric computations.  Metamath does not
have any built-in representation of numbers\index{Metamath!representation of
numbers}; instead, a specific string of symbols (digits) must be syntactically
constructed as part of any proof in which an ordinary number is used.  For
this reason, numbers in Metamath are best limited to specific constants that
arise during the course of a theorem or its proof.  Numbers are only a tiny
part of the world of abstract mathematics.  The exclusion of built-in numbers
was a conscious decision to help achieve Metamath's simplicity, and there are
other software tools if you have different mathematical needs.
If you wish to quickly solve algebraic problems, the computer algebra
programs\index{computer algebra system} {\sc
macsyma}\index{macsyma@{\sc macsyma}}, Mathematica\index{Mathematica}, and
Maple\index{Maple} are specifically suited to handling numbers and
algebra efficiently.
If you wish to simply calculate numeric or matrix expressions easily,
tools such as Octave\index{Octave} may be a better choice.

After learning Metamath's basic statement types, any
tech\-ni\-cal\-ly ori\-ent\-ed person, mathematician or not, can
immediately trace
any theorem proved in the language as far back as he or she wants, all the way
to the axioms on which the theorem is based.  This ability suggests a
non-traditional way of learning about pure mathematics.  Used in conjunction
with traditional methods, Metamath could make pure mathematics accessible to
people who are not sufficiently skilled to figure out the implicit detail in
ordinary textbook proofs.  Once you learn the axioms of a theory, you can have
complete confidence that everything you need to understand a proof you are
studying is all there, at your beck and call, allowing you to focus in on any
proof step you don't understand in as much depth as you need, without worrying
about getting stuck on a step you can't figure out.\footnote{On the other
hand, writing proofs in the Metamath language is challenging, requiring
a degree of rigor far in excess of that normally taught to students.  In a
classroom setting, I doubt that writing Metamath proofs would ever replace
traditional homework exercises involving informal proofs, because the time
needed to work out the details would not allow a course to
cover much material.  For students who have trouble grasping the implied rigor
in traditional material, writing a few simple proofs in the Metamath language
might help clarify fuzzy thought processes.  Although somewhat difficult at
first, it eventually becomes fun to do, like solving a puzzle, because of the
instant feedback provided by the computer.}

Metamath\index{Metamath} is probably unlike anything you have
encountered before.  In this first chapter we will look at the philosophy and
use of computers in mathematics in order to better understand the motivation
behind Metamath.  The material in this chapter is not required in order to use
Metamath.  You may skip it if you are impatient, but I hope you will find it
educational and enjoyable.  If you want to start experimenting with the
Metamath program right away, proceed directly to Chapter~\ref{using}
(p.~\pageref{using}).  To
learn the Metamath language, skim Chapter~\ref{using} then proceed to
Chapter~\ref{languagespec} (p.~\pageref{languagespec}).

\section{Mathematics as a Computer Language}

\begin{quote}
  {\em The study of mathematics is apt to commence in
dis\-ap\-point\-ment.\ldots \\
We are told that by its aid the stars are weighted
and the billions of molecules in a drop of water are counted.  Yet, like the
ghost of Hamlet's father, this great science eludes the efforts of our mental
weapons to grasp it.}
  \flushright\sc  Alfred North Whitehead\footnote{\cite{Whitehead}, ch.\ 1.}\\
\end{quote}\index{Whitehead, Alfred North}

\subsection{Is Mathematics ``User-Friendly''?}

Suppose you have no formal training in abstract mathematics.  But popular
books you've read offer tempting glimpses of this world filled with profound
ideas that have stirred the human spirit.  You are not satisfied with the
informal, watered-down descriptions you've read but feel it is important to
grasp the underlying mathematics itself to understand its true meaning. It's
not practical to go back to school to learn it, though; you don't want to
dedicate years of your life to it.  There are many important things in life,
and you have to set priorities for what's important to you.  What would happen
if you tried to pursue it on your own, in your spare time?

After all, you were able to learn a computer programming language such as
Pascal on your own without too much difficulty, even though you had no formal
training in computers.  You don't claim to be an expert in software design,
but you can write a passable program when necessary to suit your needs.  Even
more important, you know that you can look at anyone else's Pascal program, no
matter how complex, and with enough patience figure out exactly how it works,
even though you are not a specialist.  Pascal allows you do anything that a
computer can do, at least in principle.  Thus you know you have the ability,
in principle, to follow anything that a computer program can do:  you just
have to break it down into small enough pieces.

Here's an imaginary scenario of what might happen if you na\-ive\-ly a\-dopted
this same view of abstract mathematics and tried to pick it up on your own, in
a period of time comparable to, say, learning a computer programming
language.

\subsubsection{A Non-Mathematician's Quest for Truth}

\begin{quote}
  {\em \ldots my daughters have been studying (chemistry) for several
se\-mes\-ters, think they have learned differential and integral calculus in
school, and yet even today don't know why $x\cdot y=y\cdot x$ is true.}
  \flushright\sc  Edmund Landau\footnote{\cite{Landau}, p.~vi.}\\
\end{quote}\index{Landau, Edmund}

\begin{quote}
  {\em Minus times minus is plus,\\
The reason for this we need not discuss.}
  \flushright\sc W.\ H.\ Auden\footnote{As quoted in \cite{Guillen}, p.~64.}\\
\end{quote}\index{Auden, W.\ H.}\index{Guillen, Michael}

We'll suppose you are a technically oriented professional, perhaps an engineer, a
computer programmer, or a physicist, but probably not a mathematician.  You
consider yourself reasonably intelligent.  You did well in school, learning a
variety of methods and techniques in practical mathematics such as calculus and
differential equations.  But rarely did your courses get into anything
resembling modern abstract mathematics, and proofs were something that appeared
only occasionally in your textbooks, a kind of necessary evil that was
supposed to convince you of a certain key result.  Most of your
homework consisted of exercises that gave you practice in the techniques, and
you were hardly ever asked to come up with a proof of your own.

You find yourself curious about advanced, abstract mathematics.  You are
driven by an inner conviction that it is important to understand and
appreciate some of the most profound knowledge discovered by mankind.  But it
seems very hard to learn, something that only certain gifted longhairs can
access and understand.  You are frustrated that it seems forever cut off from
you.

Eventually your curiosity drives you to do something about it.
You set for yourself a goal of ``really'' understanding mathematics:  not just
how to manipulate equations in algebra or calculus according to cookbook
rules, but rather to gain a deep understanding of where those rules come from.
In fact, you're not thinking about this kind of ordinary mathematics at all,
but about a much more abstract, ethereal realm of pure mathematics, where
famous results such as G\"{o}del's incompleteness theorem\index{G\"{o}del's
incompleteness theorem} and Cantor's different kinds of infinities
reside.

You have probably read a number of popular books, with titles like {\em
Infinity and the Mind} \cite{Rucker}\index{Rucker, Rudy}, on topics such as
these.  You found them inspiring but at the same time somewhat
unsatisfactory.  They gave you a general idea of what these results are about,
but if someone asked you to prove them, you wouldn't have the faintest idea of
where to begin.   Sure, you could give the same overall outline that you
learned from the popular books; and in a general sort of way, you do have an
understanding.  But deep down inside, you know that there is a rigor that is
missing, that probably there are many subtle steps and pitfalls along the way,
and ultimately it seems you have to place your trust in the experts in the
field.  You don't like this; you want to be able to verify these results for
yourself.

So where do you go next?  As a first step, you decide to look up some of the
original papers on the theorems you are curious about, or better, obtain some
standard textbooks in the field.  You look up a theorem you want to
understand.  Sure enough, it's there, but it's expressed with strange
terms and odd symbols that mean absolutely nothing to you.  It might as well be written in
a foreign language you've never seen before, whose symbols are totally alien.
You look at the proof, and you haven't the foggiest notion what each step
means, much less how one step follows from another.  Well, obviously you have
a lot to learn if you want to understand this stuff.

You feel that you could probably understand it by
going back to college for another three to six years and getting a math
degree.  But that does not fit in with your career and the other things in
your life and would serve no practical purpose.  You decide to seek a quicker
path.  You figure you'll just trace your way back to the beginning, step by
step, as you would do with a computer program, until you understand it.  But
you quickly find that this is not possible, since you can't even understand
enough to know what you have to trace back to.

Maybe a different approach is in order---maybe you should start at the
beginning and work your way up.  First, you read the introduction to the book
to find out what the prerequisites are.  In a similar fashion, you trace your
way back through two or three more books, finally arriving at one that seems
to start at a beginning:  it lists the axioms of arithmetic.  ``Aha!'' you
naively think, ``This must be the starting point, the source of all mathematical
knowledge.'' Or at least the starting point for mathematics dealing with
numbers; you have to start somewhere and have no idea what the starting point
for other mathematics would be.  But the word ``axioms'' looks promising.  So
you eagerly read along and work through some elementary exercises at the
beginning of the book.  You feel vaguely bothered:  these
don't seem like axioms at all, at least not in the sense that you want to
think of axioms.  Axioms imply a starting point from which everything else can
be built up, according to precise rules specified in the axiom system.  Even
though you can understand the first few proofs in an informal way,
and are able to do some of the
exercises, it's hard to pin down precisely what the
rules are.   Sure, each step seems to follow logically from the others, but
exactly what does that mean?  Is the ``logic'' just a matter of common sense,
something vague that we all understand but can never quite state precisely?

You've spent a number of years, off and on, programming computers, and you
know that in the case of computer languages there is no question of what the
rules are---they are precise and crystal clear.  If you follow them, your
program will work, and if you don't, it won't.  No matter how complex a
program, it can always be broken down into simpler and simpler pieces, until
you can ultimately identify the bits that are moved around to perform a
specific function.  Some programs might require a lot of perseverance to
accomplish this, but if you focus on a specific portion of it, you don't even
necessarily have to know how the rest of it works. Shouldn't there be an
analogy in mathematics?

You decide to apply the ultimate test:  you ask yourself how a computer could
verify or ensure that the steps in these proofs follow from one another.
Certainly mathematics must be at least as precisely defined as a computer
language, if not more so; after all, computer science itself is based on it.
If you can get a computer to verify these proofs, then you should also be
able, in principle, to understand them yourself in a very crystal clear,
precise way.

You're in for a surprise:  you can conceive of no way to convert the
proofs, which are in English, to a form that the computer can understand.
The proofs are filled with phrases such as ``assume there exists a unique
$x$\ldots'' and ``given any $y$, let $z$ be the number such that\ldots''  This
isn't the kind of logic you are used to in computer programming, where
everything, even arithmetic, reduces to Boolean ones and zeroes if you care to
break it down sufficiently.  Even though you think you understand the proofs,
there seems to be some kind of higher reasoning involved rather than precise
rules that define how you manipulate the symbols in the axioms.  Whatever it
is, it just isn't obvious how you would express it to a computer, and the more
you think about it, the more puzzled and confused you get, to the point where
you even wonder whether {\em you} really understand it.  There's a lot more to
these axioms of arithmetic than meets the eye.

Nobody ever talked about this in school in your applied math and engineering
courses.  You just learned the rules they gave you, not quite understanding
how or why they worked, sometimes vaguely suspicious or uncertain of them, and
through homework problems and osmosis learned how to present solutions that
satisfied the instructor and earned you an ``A.''  Rarely did you actually
``prove'' anything in a rigorous way, and the math majors who did do stuff
like that seemed to be in a different world.

Of course, there are computer algebra programs that can do mathematics, and
rather impressively.  They can instantly solve the integrals that you
struggled with in freshman calculus, and do much, much more.  But when you
look at these programs, what you see is a big collection of algorithms and
techniques that evolved and were added to over time, along with some basic
software that manipulates symbols.  Each algorithm that is built in is the
result of someone's theorem whose proof is omitted; you just have to trust the
person who proved it and the person who programmed it in and hope there are no
bugs.\index{computer program bugs}  Somehow this doesn't seem to be the
essence of mathematics.  Although computer algebra systems can generate
theorems with amazing speed, they can't actually prove a single one of them.

After some puzzlement, you revisit some popular books on what mathematics is
all about.  Somewhere you read that all of mathematics is actually derived
from something called ``set theory.''  This is a little confusing, because
nowhere in the book that presented the axioms of arithmetic was there any
mention of set theory, or if there was, it seemed to be just a tool that helps
you describe things better---the set of even numbers, that sort of thing.  If
set theory is the basis for all mathematics, then why are additional axioms
needed for arithmetic?

Something is wrong but you're not sure what.  One of your friends is a pure
mathematician.  He knows he is unable to communicate to you what he does for a
living and seems to have little interest in trying.  You do know that for him,
proofs are what mathematics is all about. You ask him what a proof is, and he
essentially tells you that, while of course it's based on logic, really it's
something you learn by doing it over and over until you pick it up.  He refers
you to a book, {\em How to Read and Do Proofs} \cite{Solow}.\index{Solow,
Daniel}  Although this book helps you understand traditional informal proofs,
there is still something missing you can't seem to pin down yet.

You ask your friend how you would go about having a computer verify a proof.
At first he seems puzzled by the question; why would you want to do that?
Then he says it's not something that would make any sense to do, but he's
heard that you'd have to break the proof down into thousands or even millions
of individual steps to do such a thing, because the reasoning involved is at
such a high level of abstraction.  He says that maybe it's something you could
do up to a point, but the computer would be completely impractical once you
get into any meaningful mathematics.  There, the only way you can verify a
proof is by hand, and you can only acquire the ability to do this by
specializing in the field for a couple of years in grad school.  Anyway, he
thinks it all has to do with set theory, although he has never taken a formal
course in set theory but just learned what he needed as he went along.

You are intrigued and amazed.  Apparently a mathematician can grasp as a
single concept something that would take a computer a thousand or a million
steps to verify, and have complete confidence in it.  Each one of these
thousand or million steps must be absolutely correct, or else the whole proof
is meaningless.  If you added a million numbers by hand, would you trust the
result?  How do you really know that all these steps are correct, that there
isn't some subtle pitfall in one of these million steps, like a bug in a
computer program?\index{computer program bugs}  After all, you've read that
famous mathematicians have occasionally made mistakes, and you certainly know
you've made your share on your math homework problems in school.

You recall the analogy with a computer program.  Sure, you can understand what
a large computer program such as a word processor does, as a single high-level
concept or a small set of such concepts, but your ability to understand it in
no way ensures that the program is correct and doesn't have hidden bugs.  Even
if you wrote the program yourself you can't really know this; most large
programs that you've written have had bugs that crop up at some later date, no
matter how careful you tried to be while writing them.

OK, so now it seems the reason you can't figure out how to make a
computer verify proofs is because each step really corresponds to a
million small steps.  Well, you say, a computer can do a million
calculations in a second, so maybe it's still practical to do.  Now the
puzzle becomes how to figure out what the million steps are that each
English-language step corresponds to.  Your mathematician friend hasn't
a clue, but suggests that maybe you would find the answer by studying
set theory.  Actually, your friend thinks you're a little off the wall
for even wondering such a thing.  For him, this is not what mathematics
is all about.

The subject of set theory keeps popping up, so you decide it's
time to look it up.

You decide to start off on a careful footing, so you start reading a couple of
very elementary books on set theory.  A lot of it seems pretty obvious, like
intersections, subsets, and Venn diagrams.  You thumb through one of the
books; nowhere is anything about axioms mentioned. The other book relegates to
an appendix a brief discussion that mentions a set of axioms called
``Zermelo--Fraenkel set theory''\index{Zermelo--Fraenkel set theory} and states
them in English.  You look at them and have no idea what they really mean or
what you can do with them.  The comments in this appendix say that the purpose
of mentioning them is to expose you to the idea, but imply that they are not
necessary for basic understanding and that they are really the subject matter
of advanced treatments where fine points such as a certain paradox (Russell's
paradox\index{Russell's paradox}\footnote{Russell's paradox assumes that there
exists a set $S$ that is a collection of all sets that don't contain
themselves.  Now, either $S$ contains itself or it doesn't.  If it contains
itself, it contradicts its definition.  But if it doesn't contain itself, it
also contradicts its definition.  Russell's paradox is resolved in ZF set
theory by denying that such a set $S$ exists.}) are resolved.  Wait a
minute---shouldn't the axioms be a starting point, not an ending point?  If
there are paradoxes that arise without the axioms, how do you know you won't
stumble across one accidentally when using the informal approach?

And nowhere do these books describe how ``all of mathematics can be
derived from set theory'' which by now you've heard a few times.

You find a more advanced book on set theory.  This one actually lists the
axioms of ZF set theory in plain English on page one.  {\em Now} you think
your quest has ended and you've finally found the source of all mathematical
knowledge; you just have to understand what it means.  Here, in one place, is
the basis for all of mathematics!  You stare at the axioms in awe, puzzle over
them, memorize them, hoping that if you just meditate on them long enough they
will become clear.  Of course, you haven't the slightest idea how the rest of
mathematics is ``derived'' from them; in particular, if these are the axioms
of mathematics, then why do arithmetic, group theory, and so on need their own
axioms?

You start reading this advanced book carefully, pondering the meaning of every
word, because by now you're really determined to get to the bottom of this.
The first thing the book does is explain how the axioms came about, which was
to resolve Russell's paradox.\index{Russell's paradox}  In fact that seems to
be the main purpose of their existence; that they supposedly can be used to
derive all of mathematics seems irrelevant and is not even mentioned.  Well,
you go on.  You hope the book will explain to you clearly, step by step, how
to derive things from the axioms.  After all, this is the starting point of
mathematics, like a book that explains the basics of a computer programming
language.  But something is missing.  You find you can't even understand the
first proof or do the first exercise.  Symbols such as $\exists$ and $\forall$
permeate the page without any mention of where they came from or how to
manipulate them; the author assumes you are totally familiar with them and
doesn't even tell you what they mean.  By now you know that $\exists$ means
``there exists'' and $\forall$ means ``for all,'' but shouldn't the rules for
manipulating these symbols be part of the axioms?  You still have no idea
how you could even describe the axioms to a computer.

Certainly there is something much different here from the technical
literature you're used to reading.  A computer language manual almost
always explains very clearly what all the symbols mean, precisely what
they do, and the rules used for combining them, and you work your way up
from there.

After glancing at four or five other such books, you come to the realization
that there is another whole field of study that you need just to get to the
point at which you can understand the axioms of set theory.  The field is
called ``logic.''  In fact, some of the books did recommend it as a
prerequisite, but it just didn't sink in.  You assumed logic was, well, just
logic, something that a person with common sense intuitively understood.  Why
waste your time reading boring treatises on symbolic logic, the manipulation
of 1's and 0's that computers do, when you already know that?  But this is a
different kind of logic, quite alien to you.  The subject of {\sc nand} and
{\sc nor} gates is not even touched upon or in any case has to do with only a
very small part of this field.

So your quest continues.  Skimming through the first couple of introductory
books, you get a general idea of what logic is about and what quantifiers
(``for all,'' ``there exists'') mean, but you find their examples somewhat
trivial and mildly annoying (``all dogs are animals,'' ``some animals are
dogs,'' and such).  But all you want to know is what the rules are for
manipulating the symbols so you can apply them to set theory.  Some formulas
describing the relationships among quantifiers ($\exists$ and $\forall$) are
listed in tables, along with some verbal reasoning to justify them.
Presumably, if you want to find out if a formula is correct, you go through
this same kind of mental reasoning process, possibly using images of dogs and
animals. Intuitively, the formulas seem to make sense.  But when you ask
yourself, ``What are the rules I need to get a computer to figure out whether
this formula is correct?'', you still don't know.  Certainly you don't ask the
computer to imagine dogs and animals.

You look at some more advanced logic books.  Many of them have an introductory
chapter summarizing set theory, which turns out to be a prerequisite.  You
need logic to understand set theory, but it seems you also need set theory to
understand logic!  These books jump right into proving rather advanced
theorems about logic, without offering the faintest clue about where the logic
came from that allows them to prove these theorems.

Luckily, you come across an elementary book of logic that, halfway through,
after the usual truth tables and metaphors, presents in a clear, precise way
what you've been looking for all along: the axioms!  They're divided into
propositional calculus (also called sentential logic) and predicate calculus
(also called first-order logic),\index{first-order logic} with rules so simple
and crystal clear that now you can finally program a computer to understand
them.  Indeed, they're no harder than learning how to play a game of chess.
As far as what you seem to need is concerned, the whole book could have been
written in five pages!

{\em Now} you think you've found the ultimate source of mathematical
truth.  So---the axioms of mathematics consist of these axioms of logic,
together with the axioms of ZF set theory. (By now you've also been able to
figure out how to translate the ZF axioms from English into the
actual symbols of logic which you can now manipulate according to
precise, easy-to-understand rules.)

Of course, you still don't understand how ``all of mathematics can be
derived from set theory,'' but maybe this will reveal itself in due
course.

You eagerly set out to program the axioms and rules into a computer and start
to look at the theorems you will have to prove as the logic is developed.  All
sorts of important theorems start popping up:  the deduction
theorem,\index{deduction theorem} the substitution theorem,\index{substitution
theorem} the completeness theorem of propositional calculus,\index{first-order
logic!completeness} the completeness theorem of predicate calculus.  Uh-oh,
there seems to be trouble.  They all get harder and harder, and not one of
them can be derived with the axioms and rules of logic you've just been
handed.  Instead, they all require ``metalogic'' for their proofs, a kind of
mixture of logic and set theory that allows you to prove things {\em about}
the axioms and theorems of logic rather than {\em with} them.

You plow ahead anyway.  A month later, you've spent much of your
free time getting the computer to verify proofs in propositional calculus.
You've programmed in the axioms, but you've also had to program in the
deduction theorem, the substitution theorem, and the completeness theorem of
propositional calculus, which by now you've resigned yourself to treating as
rather complex additional axioms, since they can't be proved from the axioms
you were given.  You can now get the computer to verify and even generate
complete, rigorous, formal proofs\index{formal proof}.  Never mind that they
may have 100,000 steps---at least now you can have complete, absolute
confidence in them.  Unfortunately, the only theorems you have proved are
pretty trivial and you can easily verify them in a few minutes with truth
tables, if not by inspection.

It looks like your mathematician friend was right.  Getting the computer to do
serious mathematics with this kind of rigor seems almost hopeless.  Even
worse, it seems that the further along you get, the more ``axioms'' you have
to add, as each new theorem seems to involve additional ``metamathematical''
reasoning that hasn't been formalized, and none of it can be derived from the
axioms of logic.  Not only do the proofs keep growing exponentially as you get
further along, but the program to verify them keeps getting bigger and bigger
as you program in more ``metatheorems.''\index{metatheorem}\footnote{A
metatheorem is usually a statement that is too general to be directly provable
in a theory.  For example, ``if $n_1$, $n_2$, and $n_3$ are integers, then
$n_1+n_2+n_3$ is an integer'' is a theorem of number theory.  But ``for any
integer $k > 1$, if $n_1, \ldots, n_k$ are integers, then $n_1+\ldots +n_k$ is
an integer'' is a metatheorem, in other words a family of theorems, one for
every $k$.  The reason it is not a theorem is that the general sum $n_1+\ldots
+n_k$ (as a function of $k$) is not an operation that can be defined directly
in number theory.} The bugs\index{computer program bugs} that have cropped up
so far have already made you start to lose faith in the rigor you seem to have
achieved, and you know it's just going to get worse as your program gets larger.

\subsection{Mathematics and the Non-Specialist}

\begin{quote}
  {\em A real proof is not checkable by a machine, or even by any mathematician
not privy to the gestalt, the mode of thought of the particular field of
mathematics in which the proof is located.}
  \flushright\sc  Davis and Hersh\index{Davis, Phillip J.}
  \footnote{\cite{Davis}, p.~354.}\\
\end{quote}

The bulk of abstract or theoretical mathematics is ordinarily outside
the reach of anyone but a few specialists in each field who have completed
the necessary difficult internship in order to enter its coterie.  The
typical intelligent layperson has no reasonable hope of understanding much of
it, nor even the specialist mathematician of understanding other fields.  It
is like a foreign language that has no dictionary to look up the translation;
the only way you can learn it is by living in the country for a few years.  It
is argued that the effort involved in learning a specialty is a necessary
process for acquiring a deep understanding.  Of course, this is almost certainly
true if one is to make significant contributions to a field; in particular,
``doing'' proofs is probably the most important part of a mathematician's
training.  But is it also necessary to deny outsiders access to it?  Is it
necessary that abstract mathematics be so hard for a layperson to grasp?

A computer normally is of no help whatsoever.  Most published proofs are
actually just series of hints written in an informal style that requires
considerable knowledge of the field to understand.  These are the ``real
proofs'' referred to by Davis and Hersh.\index{informal proof}  There is an
implicit understanding that, in principle, such a proof could be converted to
a complete formal proof\index{formal proof}.  However, it is said that no one
would ever attempt such a conversion, even if they could, because that would
presumably require millions of steps (Section~\ref{dream}).  Unfortunately the
informal style automatically excludes the understanding of the proof
by anyone who hasn't gone through the necessary apprenticeship. The
best that the intelligent layperson can do is to read popular books about deep
and famous results; while this can be helpful, it can also be misleading, and
the lack of detail usually leaves the reader with no ability whatsoever to
explore any aspect of the field being described.

The statements of theorems often use sophisticated notation that makes them
inaccessible to the non-specialist.  For a non-specialist who wants to achieve
a deeper understanding of a proof, the process of tracing definitions and
lemmas back through their hierarchy\index{hierarchy} quickly becomes confusing
and discouraging.  Textbooks are usually written to train mathematicians or to
communicate to people who are already mathematicians, and large gaps in proofs
are often left as exercises to the reader who is left at an impasse if he or
she becomes stuck.

I believe that eventually computers will enable non-specialists and even
intelligent laypersons to follow almost any mathematical proof in any field.
Metamath is an attempt in that direction.  If all of mathematics were as
easily accessible as a computer programming language, I could envision
computer programmers and hobbyists who otherwise lack mathematical
sophistication exploring and being amazed by the world of theorems and proofs
in obscure specialties, perhaps even coming up with results of their own.  A
tremendous advantage would be that anyone could experiment with conjectures in
any field---the computer would offer instant feedback as to whether
an inference step was correct.

Mathematicians sometimes have to put up with the annoyance of
cranks\index{cranks} who lack a fundamental understanding of mathematics but
insist that their ``proofs'' of, say, Fermat's Last Theorem\index{Fermat's
Last Theorem} be taken seriously.  I think part of the problem is that these
people are misled by informal mathematical language, treating it as if they
were reading ordinary expository English and failing to appreciate the
implicit underlying rigor.  Such cranks are rare in the field of computers,
because computer languages are much more explicit, and ultimately the proof is
in whether a computer program works or not.  With easily accessible
computer-based abstract mathematics, a mathematician could say to a crank,
``don't bother me until you've demonstrated your claim on the computer!''

% 22-May-04 nm
% Attempt to move De Millo quote so it doesn't separate from attribution
% CHANGE THIS NUMBER (AND ELIMINATE IF POSSIBLE) WHEN ABOVE TEXT CHANGES
\vspace{-0.5em}

\subsection{An Impossible Dream?}\label{dream}

\begin{quote}
  {\em Even quite basic theorems would demand almost unbelievably vast
  books to display their proofs.}
    \flushright\sc  Robert E. Edwards\footnote{\cite{Edwards}, p.~68.}\\
\end{quote}\index{Edwards, Robert E.}

\begin{quote}
  {\em Oh, of course no one ever really {\em does} it.  It would take
  forever!  You just show that you could do it, that's sufficient.}
    \flushright\sc  ``The Ideal Mathematician''
    \index{Davis, Phillip J.}\footnote{\cite{Davis},
p.~40.}\\
\end{quote}

\begin{quote}
  {\em There is a theorem in the primitive notation of set theory that
  corresponds to the arithmetic theorem `$1000+2000=3000$'.  The formula
  would be forbiddingly long\ldots even if [one] knows the definitions
  and is asked to simplify the long formula according to them, chances are
  he will make errors and arrive at some incorrect result.}
    \flushright\sc  Hao Wang\footnote{\cite{Wang}, p.~140.}\\
\end{quote}\index{Wang, Hao}

% 22-May-04 nm
% Attempt to move De Millo quote so it doesn't separate from attribution
% CHANGE THIS NUMBER (AND ELIMINATE IF POSSIBLE) WHEN ABOVE TEXT CHANGES
\vspace{-0.5em}

\begin{quote}
  {\em The {\em Principia Mathematica} was the crowning achievement of the
  formalists.  It was also the deathblow of the formalist view.\ldots
  {[Rus\-sell]} failed, in three enormous volumes, to get beyond the elementary
  facts of arithmetic.  He showed what can be done in principle and what
  cannot be done in practice.  If the mathematical process were really
  one of strict, logical progression, we would still be counting our
  fingers.\ldots
  One theoretician estimates\ldots that a demonstration of one of
  Ramanujan's conjectures assuming set theory and elementary analysis would
  take about two thousand pages; the length of a deduction from first principles
  is nearly in\-con\-ceiv\-a\-ble\ldots The probabilists argue that\ldots any
  very long proof can at best be viewed as only probably correct\ldots}
  \flushright\sc Richard de Millo et. al.\footnote{\cite{deMillo}, pp.~269,
  271.}\\
\end{quote}\index{de Millo, Richard}

A number of writers have conveyed the impression that the kind of absolute
rigor provided by Metamath\index{Metamath} is an impossible dream, suggesting
that a complete, formal verification\index{formal proof} of a typical theorem
would take millions of steps in untold volumes of books.  Even if it could be
done, the thinking sometimes goes, all meaning would be lost in such a
monstrous, tedious verification.\index{informal proof}\index{proof length}

These writers assume, however, that in order to achieve the kind of complete
formal verification they desire one must break down a proof into individual
primitive steps that make direct reference to the axioms.  This is
not necessary.  There is no reason not to make use of previously proved
theorems rather than proving them over and over.

Just as important, definitions\index{definition} can be introduced along
the way, allowing very complex formulas to be represented with few
symbols.  Not doing this can lead to absurdly long formulas.  For
example, the mere statement of
G\"{o}del's incompleteness theorem\index{G\"{o}del's
incompleteness theorem}, which can be expressed with a small number of
defined symbols, would require about 20,000 primitive symbols to express
it.\index{Boolos, George S.}\footnote{George S.\ Boolos, lecture at
Massachusetts Institute of Technology, spring 1990.} An extreme example
is Bourbaki's\label{bourbaki} language for set theory, which requires
4,523,659,424,929 symbols plus 1,179,618,517,981 disambiguatory links
(lines connecting symbol pairs, usually drawn below or above the
formula) to express the number
``one'' \cite{Mathias}.\index{Mathias, Adrian R. D.}\index{Bourbaki,
Nicolas}
% http://www.dpmms.cam.ac.uk/~ardm/

A hierarchy\index{hierarchy} of theorems and definitions permits an
exponential growth in the formula sizes and primitive proof steps to be
described with only a linear growth in the number of symbols used.  Of course,
this is how ordinary informal mathematics is normally done anyway, but with
Metamath\index{Metamath} it can be done with absolute rigor and precision.

\subsection{Beauty}


\begin{quote}
  {\em No one shall be able to drive us from the paradise that Cantor has
created for us.}
   \flushright\sc  David Hilbert\footnote{As quoted in \cite{Moore}, p.~131.}\\
\end{quote}\index{Hilbert, David}

\needspace{3\baselineskip}
\begin{quote}
  {\em Mathematics possesses not only truth, but some supreme beauty ---a
  beauty cold and austere, like that of a sculpture.}
    \flushright\sc  Bertrand
    Russell\footnote{\cite{Russell}.}\\
\end{quote}\index{Russell, Bertrand}

\begin{quote}
  {\em Euclid alone has looked on Beauty bare.}
  \flushright\sc Edna Millay\footnote{As quoted in \cite{Davis}, p.~150.}\\
\end{quote}\index{Millay, Edna}

For most people, abstract mathematics is distant, strange, and
incomprehensible.  Many popular books have tried to convey some of the sense
of beauty in famous theorems.  But even an intelligent layperson is left with
only a general idea of what a theorem is about and is hardly given the tools
needed to make use of it.  Traditionally, it is only after years of arduous
study that one can grasp the concepts needed for deep understanding.
Metamath\index{Metamath} allows you to approach the proof of the theorem from
a quite different perspective, peeling apart the formulas and definitions
layer by layer until an entirely different kind of understanding is achieved.
Every step of the proof is there, pieced together with absolute precision and
instantly available for inspection through a microscope with a magnification
as powerful as you desire.

A proof in itself can be considered an object of beauty.  Constructing an
elegant proof is an art.  Once a famous theorem has been proved, often
considerable effort is made to find simpler and more easily understood
proofs.  Creating and communicating elegant proofs is a major concern of
mathematicians.  Metamath is one way of providing a common language for
archiving and preserving this information.

The length of a proof can, to a certain extent, be considered an
objective measure of its ``beauty,'' since shorter proofs are usually
considered more elegant.  In the set theory database
\texttt{set.mm}\index{set theory database (\texttt{set.mm})}%
\index{Metamath Proof Explorer}
provided with Metamath, one goal was to make all proofs as short as possible.

\needspace{4\baselineskip}
\subsection{Simplicity}

\begin{quote}
  {\em God made man simple; man's complex problems are of his own
  devising.}
    \flushright\sc Eccles. 7:29\footnote{Jerusalem Bible.}\\
\end{quote}\index{Bible}

\needspace{3\baselineskip}
\begin{quote}
  {\em God made integers, all else is the work of man.}
    \flushright\sc Leopold Kronecker\footnote{{\em Jahresbericht
	der Deutschen Mathematiker-Vereinigung }, vol. 2, p. 19.}\\
\end{quote}\index{Kronecker, Leopold}

\needspace{3\baselineskip}
\begin{quote}
  {\em For what is clear and easily comprehended attracts; the
  complicated repels.}
    \flushright\sc David Hilbert\footnote{As quoted in \cite{deMillo},
p.~273.}\\
\end{quote}\index{Hilbert, David}

The Metamath\index{Metamath} language is simple and Spartan.  Metamath treats
all mathematical expressions as simple sequences of symbols, devoid of meaning.
The higher-level or ``metamathematical'' notions underlying Metamath are about
as simple as they could possibly be.  Each individual step in a proof involves
a single basic concept, the substitution of an expression for a variable, so
that in principle almost anyone, whether mathematician or not, can
completely understand how it was arrived at.

In one of its most basic applications, Metamath\index{Metamath} can be used to
develop the foundations of mathematics\index{foundations of mathematics} from
the very beginning.  This is done in the set theory database that is provided
with the Metamath package and is the subject matter
of Chapter~\ref{fol}. Any language (a metalanguage\index{metalanguage})
used to describe mathematics (an object language\index{object language}) must
have a mathematical content of its own, but it is desirable to keep this
content down to a bare minimum, namely that needed to make use of the
inference rules specified by the axioms.  With any metalanguage there is a
``chicken and egg'' problem somewhat like circular reasoning:  you must assume
the validity of the mathematics of the metalanguage in order to prove the
validity of the mathematics of the object language.  The mathematical content
of Metamath itself is quite limited.  Like the rules of a game of chess, the
essential concepts are simple enough so that virtually anyone should be able to
understand them (although that in itself will not let you play like
a master).  The symbols that Metamath manipulates do not in themselves
have any intrinsic meaning.  Your interpretation of the axioms that you supply
to Metamath is what gives them meaning.  Metamath is an attempt to strip down
mathematical thought to its bare essence and show you exactly how the symbols
are manipulated.

Philosophers and logicians, with various motivations, have often thought it
important to study ``weak'' fragments of logic\index{weak logic}
\cite{Anderson}\index{Anderson, Alan Ross} \cite{MegillBunder}\index{Megill,
Norman}\index{Bunder, Martin}, other unconventional systems of logic (such as
``modal'' logic\index{modal logic} \cite[ch.\ 27]{Boolos}\index{Boolos, George
S.}), and quantum logic\index{quantum logic} in physics
\cite{Pavicic}\index{Pavi{\v{c}}i{\'{c}}, M.}.  Metamath\index{Metamath}
provides a framework in which such systems can be expressed, with an absolute
precision that makes all underlying metamathematical assumptions rigorous and
crystal clear.

Some schools of philosophical thought, for example
intuitionism\index{intuitionism} and constructivism\index{constructivism},
demand that the notions underlying any mathematical system be as simple and
concrete as possible.  Metamath should meet the requirements of these
philosophies.  Metamath must be taught the symbols, axioms\index{axiom}, and
rules\index{rule} for a specific theory, from the skeptical (such as
intuitionism\index{intuitionism}\footnote{Intuitionism does not accept the law
of excluded middle (``either something is true or it is not true'').  See
\cite[p.~xi]{Tymoczko}\index{Tymoczko, Thomas} for discussion and references
on this topic.  Consider the theorem, ``There exist irrational numbers $a$ and
$b$ such that $a^b$ is rational.''  An intuitionist would reject the following
proof:  If $\sqrt{2}^{\sqrt{2}}$ is rational, we are done.  Otherwise, let
$a=\sqrt{2}^{\sqrt{2}}$ and $b=\sqrt{2}$. Then $a^b=2$, which is rational.})
to the bold (such as the axiom of choice in set theory\footnote{The axiom of
choice\index{Axiom of Choice} asserts that given any collection of pairwise
disjoint nonempty sets, there exists a set that has exactly one element in
common with each set of the collection.  It is used to prove many important
theorems in standard mathematics.  Some philosophers object to it because it
asserts the existence of a set without specifying what the set contains
\cite[p.~154]{Enderton}\index{Enderton, Herbert B.}.  In one foundation for
mathematics due to Quine\index{Quine, Willard Van Orman}, that has not been
otherwise shown to be inconsistent, the axiom of choice turns out to be false
\cite[p.~23]{Curry}\index{Curry, Haskell B.}.  The \texttt{show
trace{\char`\_}back} command of the Metamath program allows you to find out
whether the axiom of choice, or any other axiom, was assumed by a
proof.}\index{\texttt{show trace{\char`\_}back} command}).

The simplicity of the Metamath language lets the algorithm (computer program)
that verifies the validity of a Metamath proof be straightforward and
robust.  You can have confidence that the theorems it verifies really can be
derived from your axioms.

\subsection{Rigor}

\begin{quote}
  {\em Rigor became a goal with the Greeks\ldots But the efforts to
  pursue rigor to the utmost have led to an impasse in which there is
  no longer any agreement on what it really means.  Mathematics remains
  alive and vital, but only on a pragmatic basis.}
    \flushright\sc  Morris Kline\footnote{\cite{Kline}, p.~1209.}\\
\end{quote}\index{Kline, Morris}

Kline refers to a much deeper kind of rigor than that which we will discuss in
this section.  G\"{o}del's incompleteness theorem\index{G\"{o}del's
incompleteness theorem} showed that it is impossible to achieve absolute rigor
in standard mathematics because we can never prove that mathematics is
consistent (free from contradictions).\index{consistent theory}  If
mathematics is consistent, we will never know it, but must rely on faith.  If
mathematics is inconsistent, the best we can hope for is that some clever
future mathematician will discover the inconsistency.  In this case, the
axioms would probably be revised slightly to eliminate the inconsistency, as
was done in the case of Russell's paradox,\index{Russell's paradox} but the
bulk of mathematics would probably not be affected by such a discovery.
Russell's paradox, for example, did not affect most of the remarkable results
achieved by 19th-century and earlier mathematicians.  It mainly invalidated
some of Gottlob Frege's\index{Frege, Gottlob} work on the foundations of
mathematics in the late 1800's; in fact Frege's work inspired Russell's
discovery.  Despite the paradox, Frege's work contains important concepts that
have significantly influenced modern logic.  Kline's {\em Mathematics, The
Loss of Certainty} \cite{Klinel}\index{Kline, Morris} has an interesting
discussion of this topic.

What {\em can} be achieved with absolute certainty\index{certainty} is the
knowledge that if we assume the axioms are consistent and true, then the
results derived from them are true.  Part of the beauty of mathematics is that
it is the one area of human endeavor where absolute certainty can be achieved
in this sense.  A mathematical truth will remain such for eternity.  However,
our actual knowledge of whether a particular statement is a mathematical truth
is only as certain as the correctness of the proof that establishes it.  If
the proof of a statement is questionable or vague, we can't have absolute
confidence in the truth that the statement claims.

Let us look at some traditional ways of expressing proofs.

Except in the field of formal logic\index{formal logic}, almost all
traditional proofs in mathematics are really not proofs at all, but rather
proof outlines or hints as to how to go about constructing the proof.  Many
gaps\index{gaps in proofs} are left for the reader to fill in. There are
several reasons for this.  First, it is usually assumed in mathematical
literature that the person reading the proof is a mathematician familiar with
the specialty being described, and that the missing steps are obvious to such
a reader or at least that the reader is capable of filling them in.  This
attitude is fine for professional mathematicians in the specialty, but
unfortunately it often has the drawback of cutting off the rest of the world,
including mathematicians in other specialties, from understanding the proof.
We discussed one possible resolution to this on p.~\pageref{envision}.
Second, it is often assumed that a complete formal proof\index{formal proof}
would require countless millions of symbols (Section~\ref{dream}). This might
be true if the proof were to be expressed directly in terms of the axioms of
logic and set theory,\index{set theory} but it is usually not true if we allow
ourselves a hierarchy\index{hierarchy} of definitions and theorems to build
upon, using a notation that allows us to introduce new symbols, definitions,
and theorems in a precisely specified way.

Even in formal logic,\index{formal logic} formal proofs\index{formal proof}
that are considered complete still contain hidden or implicit information.
For example, a ``proof'' is usually defined as a sequence of
wffs,\index{well-formed formula (wff)}\footnote{A {\em wff} or well-formed
formula is a mathematical expression (string of symbols) constructed according
to some precise rules.  A formal mathematical system\index{formal system}
contains (1) the rules for constructing syntactically correct
wffs,\index{syntax rules} (2) a list of starting wffs called
axioms,\index{axiom} and (3) one or more rules prescribing how to derive new
wffs, called theorems\index{theorem}, from the axioms or previously derived
theorems.  An example of such a system is contained in
Metamath's\index{Metamath} set theory database, which defines a formal
system\index{formal system} from which all of standard mathematics can be
derived.  Section~\ref{startf} steps you through a complete example of a formal
system, and you may want to skim it now if you are unfamiliar with the
concept.} each of which is an axiom or follows from a rule applied to previous
wffs in the sequence.  The implicit part of the proof is the algorithm by
which a sequence of symbols is verified to be a valid wff, given the
definition of a wff.  The algorithm in this case is rather simple, but for a
computer to verify the proof,\index{automated proof verification} it must have
the algorithm built into its verification program.\footnote{It is possible, of
course, to specify wff construction syntax outside of the program itself
with a suitable input language (the Metamath language being an example), but
some proof-verification or theorem-proving programs lack the ability to extend
wff syntax in such a fashion.} If one deals exclusively with axioms and
elementary wffs, it is straightforward to implement such an algorithm.  But as
more and more definitions are added to the theory in order to make the
expression of wffs more compact, the algorithm becomes more and more
complicated.  A computer program that implements the algorithm becomes larger
and harder to understand as each definition is introduced, and thus more prone
to bugs.\index{computer program bugs}  The larger the program, the
more suspicious the mathematician may be about
the validity of its algorithms.  This is especially true because
computer programs are inherently hard to follow to begin with, and few people
enjoy verifying them manually in detail.

Metamath\index{Metamath} takes a different approach.  Metamath's ``knowledge''
is limited to the ability to substitute variables for expressions, subject to
some simple constraints.  Once the basic algorithm of Metamath is assumed to
be debugged, and perhaps independently confirmed, it
can be trusted once and for all.  The information that Metamath needs to
``understand'' mathematics is contained entirely in the body of knowledge
presented to Metamath.  Any errors in reasoning can only be errors in the
axioms or definitions contained in this body of knowledge.  As a
``constructive'' language\index{constructive language} Metamath has no
conditional branches or loops like the ones that make computer programs hard
to decipher; instead, the language can only build new sequences of symbols
from earlier sequences  of symbols.

The simplicity of the rules that underlie Metamath not only makes Metamath
easy to learn but also gives Metamath a great deal of flexibility. For
example, Metamath is not limited to describing standard first-order
logic\index{first-order logic}; higher-order logics\index{higher-order logic}
and fragments of logic\index{weak logic} can be described just as easily.
Metamath gives you the freedom to define whatever wff notation you prefer; it
has no built-in conception of the syntax of a wff.\index{well-formed formula
(wff)}  With suitable axioms and definitions, Metamath can even describe and
prove things about itself.\index{Metamath!self-description}  (John
Harrison\index{Harrison, John} discusses the ``reflection''
principle\index{reflection principle} involved in self-descriptive systems in
\cite{Harrison}.)

The flexibility of Metamath requires that its proofs specify a lot of detail,
much more than in an ordinary ``formal'' proof.\index{formal proof}  For
example, in an ordinary formal proof, a single step consists of displaying the
wff that constitutes that step.  In order for a computer program to verify
that the step is acceptable, it first must verify that the symbol sequence
being displayed is an acceptable wff.\index{automated proof verification} Most
proof verifiers have at least basic wff syntax built into their programs.
Metamath has no hard-wired knowledge of what constitutes a wff built into it;
instead every wff must be explicitly constructed based on rules defining wffs
that are present in a database.  Thus a single step in an ordinary formal
proof may be correspond to many steps in a Metamath proof. Despite the larger
number of steps, though, this does not mean that a Metamath proof must be
significantly larger than an ordinary formal proof. The reason is that since
we have constructed the wff from scratch, we know what the wff is, so there is
no reason to display it.  We only need to refer to a sequence of statements
that construct it.  In a sense, the display of the wff in an ordinary formal
proof is an implicit proof of its own validity as a wff; Metamath just makes
the proof explicit. (Section~\ref{proof} describes Metamath's proof notation.)

\section{Computers and Mathematicians}

\begin{quote}
  {\em The computer is important, but not to mathematics.}
    \flushright\sc  Paul Halmos\footnote{As quoted in \cite{Albers}, p.~121.}\\
\end{quote}\index{Halmos, Paul}

Pure mathematicians have traditionally been indifferent to computers, even to
the point of disdain.\index{computers and pure mathematics}  Computer science
itself is sometimes considered to fall in the mundane realm of ``applied''
mathematics, perhaps essential for the real world but intellectually unexciting
to those who seek the deepest truths in mathematics.  Perhaps a reason for this
attitude towards computers is that there is little or no computer software that
meets their needs, and there may be a general feeling that such software could
not even exist.  On the one hand, there are the practical computer algebra
systems, which can perform amazing symbolic manipulations in algebra and
calculus,\index{computer algebra system} yet can't prove the simplest
existence theorem, if the idea of a proof is present at all.  On the other
hand, there are specialized automated theorem provers that technically speaking
may generate correct proofs.\index{automated theorem proving}  But sometimes
their specialized input notation may be cryptic and their output perceived to
be long, inelegant, incomprehensible proofs.    The output
may be viewed with suspicion, since the program that generates it tends to be
very large, and its size increases the potential for bugs\index{computer
program bugs}.  Such a proof may be considered trustworthy only if
independently verified and ``understood'' by a human, but no one wants to
waste their time on such a boring, unrewarding chore.



\needspace{4\baselineskip}
\subsection{Trusting the Computer}

\begin{quote}
  {\em \ldots I continue to find the quasi-empirical interpretation of
  computer proofs to be the more plausible.\ldots Since not
  everything that claims to be a computer proof can be
  accepted as valid, what are the mathematical criteria for acceptable
  computer proofs?}
    \flushright\sc  Thomas Tymoczko\footnote{\cite{Tymoczko}, p.~245.}\\
\end{quote}\index{Tymoczko, Thomas}

In some cases, computers have been essential tools for proving famous
theorems.  But if a proof is so long and obscure that it can be verified in a
practical way only with a computer, it is vaguely felt to be suspicious.  For
example, proving the famous four-color theorem\index{four-color
theorem}\index{proof length} (``a map needs no more than four colors to
prevent any two adjacent countries from having the same color'') can presently
only be done with the aid of a very complex computer program which originally
required 1200 hours of computer time. There has been considerable debate about
whether such a proof can be trusted and whether such a proof is ``real''
mathematics \cite{Swart}\index{Swart, E. R.}.\index{trusting computers}

However, under normal circumstances even a skeptical mathematician would have a
great deal of confidence in the result of multiplying two numbers on a pocket
calculator, even though the precise details of what goes on are hidden from its
user.  Even the verification on a supercomputer that a huge number is prime is
trusted, especially if there is independent verification; no one bothers to
debate the philosophical significance of its ``proof,'' even though the actual
proof would be so large that it would be completely impractical to ever write
it down on paper.  It seems that if the algorithm used by the computer is
simple enough to be readily understood, then the computer can be trusted.

Metamath\index{Metamath} adopts this philosophy.  The simplicity of its
language makes it easy to learn, and because of its simplicity one can have
essentially absolute confidence that a proof is correct. All axioms, rules, and
definitions are available for inspection at any time because they are defined
by the user; there are no hidden or built-in rules that may be prone to subtle
bugs\index{computer program bugs}.  The basic algorithm at the heart of
Metamath is simple and fixed, and it can be assumed to be bug-free and robust
with a degree of confidence approaching certainty.
Independently written implementations of the Metamath verifier
can reduce any residual doubt on the part of a skeptic even further;
there are now over a dozen such implementations, written by many people.

\subsection{Trusting the Mathematician}\label{trust}

\begin{quote}
  {\em There is no Algebraist nor Mathematician so expert in his science, as
  to place entire confidence in any truth immediately upon his discovery of it,
  or regard it as any thing, but a mere probability.  Every time he runs over
  his proofs, his confidence encreases; but still more by the approbation of
  his friends; and is rais'd to its utmost perfection by the universal assent
  and applauses of the learned world.}
  \flushright\sc David Hume\footnote{{\em A Treatise of Human Nature}, as
  quoted in \cite{deMillo}, p.~267.}\\
\end{quote}\index{Hume, David}

\begin{quote}
  {\em Stanislaw Ulam estimates that mathematicians publish 200,000 theorems
  every year.  A number of these are subsequently contradicted or otherwise
  disallowed, others are thrown into doubt, and most are ignored.}
  \flushright\sc Richard de Millo et. al.\footnote{\cite{deMillo}, p.~269.}\\
\end{quote}\index{Ulam, Stanislaw}

Whether or not the computer can be trusted, humans  of course will occasionally
err. Only the most memorable proofs get independently verified, and of these
only a handful of truly great ones achieve the status of being ``known''
mathematical truths that are used without giving a second thought to their
correctness.

There are many famous examples of incorrect theorems and proofs in
mathematical literature.\index{errors in proofs}

\begin{itemize}
\item There have been thousands of purported proofs of Fermat's Last
Theorem\index{Fermat's Last Theorem} (``no integer solutions exist to $x^n +
y^n = z^n$ for $n > 2$''), by amateurs, cranks, and well-regarded
mathematicians \cite[p.~5]{Stark}\index{Stark, Harold M}.  Fermat wrote a note
in his copy of Bachet's {\em Diophantus} that he found ``a truly marvelous
proof of this theorem but this margin is too narrow to contain it''
\cite[p.~507]{Kramer}.  A recent, much publicized proof by Yoichi
Miyaoka\index{Miyaoka, Yoichi} was shown to be incorrect ({\em Science News},
April 9, 1988, p.~230).  The theorem was finally proved by Andrew
Wiles\index{Wiles, Andrew} ({\em Science News}, July 3, 1993, p.~5), but it
initially had some gaps and took over a year after its announcement to be
checked thoroughly by experts.  On Oct. 25, 1994, Wiles announced that the last
gap found in his proof had been filled in.
  \item In 1882, M. Pasch discovered that an axiom was omitted from Euclid's
formulation of geometry\index{Euclidean geometry}; without it, the proofs of
important theorems of Euclid are not valid.  Pasch's axiom\index{Pasch's
axiom} states that a line that intersects one side of a triangle must also
intersect another side, provided that it does not touch any of the triangle's
vertices.  The omission of Pasch's axiom went unnoticed for 2000
years \cite[p.~160]{Davis}, in spite of (one presumes) the thousands of
students, instructors, and mathematicians who studied Euclid.
  \item The first published proof of the famous Schr\"{o}der--Bernstein
theorem\index{Schr\"{o}der--Bernstein theorem} in set theory was incorrect
\cite[p.~148]{Enderton}\index{Enderton, Herbert B.}.  This theorem states
that if there exists a 1-to-1 function\footnote{A {\em set}\index{set} is any
collection of objects. A {\em function}\index{function} or {\em
mapping}\index{mapping} is a rule that assigns to each element of one set
(called the function's {\em domain}\index{domain}) an element from another
set.} from set $A$ into set $B$ and vice-versa, then sets $A$ and $B$ have
a 1-to-1 correspondence.  Although it sounds simple and obvious,
the standard proof is quite long and complex.
  \item In the early 1900's, Hilbert\index{Hilbert, David} published a
purported proof of the continuum hypothesis\index{continuum hypothesis}, which
was eventually established as unprovable by Cohen\index{Cohen, Paul} in 1963
\cite[p.~166]{Enderton}.  The continuum hypothesis states that no
infinity\index{infinity} (``transfinite cardinal number'')\index{cardinal,
transfinite} exists whose size (or ``cardinality''\index{cardinality}) is
between the size of the set of integers and the size of the set of real
numbers.  This hypothesis originated with German mathematician Georg
Cantor\index{Cantor, Georg} in the late 1800's, and his inability to prove it
is said to have contributed to mental illness that afflicted him in his later
years.
  \item An incorrect proof of the four-color theorem\index{four-color theorem}
was published by Kempe\index{Kempe, A. B.} in 1879
\cite[p.~582]{Courant}\index{Courant, Richard}; it stood for 11 years before
its flaw was discovered.  This theorem states that any map can be colored
using only four colors, so that no two adjacent countries have the same
color.  In 1976 the theorem was finally proved by the famous computer-assisted
proof of Haken, Appel, and Koch \cite{Swart}\index{Appel, K.}\index{Haken,
W.}\index{Koch, K.}.  Or at least it seems that way.  Mathematician
H.~S.~M.~Coxeter\index{Coxeter, H. S. M.} has doubts \cite[p.~58]{Davis}:  ``I
have a feeling that that is an untidy kind of use of the computers, and the more
you correspond with Haken and Appel, the more shaky you seem to be.''
  \item Many false ``proofs'' of the Poincar\'{e}
conjecture\index{Poincar\'{e} conjecture} have been proposed over the years.
This conjecture states that any object that mathematically behaves like a
three-dimensional sphere is a three-dimensional sphere topologically,
regardless of how it is distorted.  In March 1986, mathematicians Colin
Rourke\index{Rourke, Colin} and Eduardo R\^{e}go\index{R\^{e}go, Eduardo}
caused  a stir in the mathematical community by announcing that they had found
a proof; in November of that year the proof was found to be false \cite[p.
218]{PetersonI}.  It was finally proved in 2003 by Grigory Perelman
\label{poincare}\index{Szpiro, George}\index{Perelman, Grigory}\cite{Szpiro}.
 \end{itemize}

Many counterexamples to ``theorems'' in recent mathematical
literature related to Clifford algebras\index{Clifford algebras}
 have been found by Pertti
Lounesto (who passed away in 2002).\index{Lounesto, Pertti}
See the web page \url{http://mathforum.org/library/view/4933.html}.
% http://users.tkk.fi/~ppuska/mirror/Lounesto/counterexamples.htm

One of the purposes of Metamath\index{Metamath} is to allow proofs to be
expressed with absolute precision.  Developing a proof in the Metamath
language can be challenging, because Metamath will not permit even the
tiniest mistake.\index{errors in proofs}  But once the proof is created, its
correctness can be trusted immediately, without having to depend on the
process of peer review for confirmation.

\section{The Use of Computers in Mathematics}

\subsection{Computer Algebra Systems}

For the most part, you will find that Metamath\index{Metamath} is not a
practical tool for manipulating numbers.  (Even proving that $2 + 2 = 4$, if
you start with set theory, can be quite complex!)  Several commercial
mathematics packages are quite good at arithmetic, algebra, and calculus, and
as practical tools they are invaluable.\index{computer algebra system} But
they have no notion of proof, and cannot understand statements starting with
``there exists such and such...''.

Software packages such as Mathematica \cite{Wolfram}\index{Mathematica} do not
concern themselves with proofs but instead work directly with known results.
These packages primarily emphasize heuristic rules such as the substitution of
equals for equals to achieve simpler expressions or expressions in a different
form.  Starting with a rich collection of built-in rules and algorithms, users
can add to the collection by means of a powerful programming language.
However, results such as, say, the existence of a certain abstract object
without displaying the actual object cannot be expressed (directly) in their
languages.  The idea of a proof from a small set of axioms is absent.  Instead
this software simply assumes that each fact or rule you add to the built-in
collection of algorithms is valid.  One way to view the software is as a large
collection of axioms from which the software, with certain goals, attempts to
derive new theorems, for example equating a complex expression with a simpler
equivalent. But the terms ``theorem''\index{theorem} and
``proof,''\index{proof} for example, are not even mentioned in the index of
the user's manual for Mathematica.\index{Mathematica and proofs}  What is also
unsatisfactory from a philosophical point of view is that there is no way to
ensure the validity of the results other than by trusting the writer of each
application module or tediously checking each module by hand, similar to
checking a computer program for bugs.\index{computer program
bugs}\footnote{Two examples illustrate why the knowledge database of computer
algebra systems should sometimes be regarded with a certain caution.  If you
ask Mathematica (version 3.0) to \texttt{Solve[x\^{ }n + y\^{ }n == z\^{ }n , n]}
it will respond with \texttt{\{\{n-\char`\>-2\}, \{n-\char`\>-1\},
\{n-\char`\>1\}, \{n-\char`\>2\}\}}. In other words, Mathematica seems to
``know'' that Fermat's Last Theorem\index{Fermat's Last Theorem} is true!  (At
the time this version of Mathematica was released this fact was unknown.)  If
you ask Maple\index{Maple} to \texttt{solve(x\^{ }2 = 2\^{ }x)} then
\texttt{simplify(\{"\})}, it returns the solution set \texttt{\{2, 4\}}, apparently
unaware that 0.7666647\ldots is also a solution.} While of course extremely
valuable in applied mathematics, computer algebra systems tend to be of little
interest to the theoretical mathematician except as aids for exploring certain
specific problems.

Because of possible bugs, trusting the output of a computer algebra system for
use as theorems in a proof-verifier would defeat the latter's goal of rigor.
On the other hand, a fact such that a certain relatively large number is
prime, while easy for a computer algebra system to derive, might have a long,
tedious proof that could overwhelm a proof-verifier. One approach for linking
computer algebra systems to a proof-verifier while retaining the advantages of
both is to add a hypothesis to each such theorem indicating its source.  For
example, a constant {\sc maple} could indicate the theorem came from the Maple
package, and would mean ``assuming Maple is consistent, then\ldots''  This and
many other topics concerning the formalization of mathematics are discussed in
John Harrison's\index{Harrison, John} very interesting
PhD thesis~\cite{Harrison-thesis}.

\subsection{Automated Theorem Provers}\label{theoremprovers}

A mathematical theory is ``decidable''\index{decidable theory} if a mechanical
method or algorithm exists that is guaranteed to determine whether or not a
particular formula is a theorem.  Among the few theories that are decidable is
elementary geometry,\index{Euclidean geometry} as was shown by a classic
result of logician Alfred Tarski\index{Tarski, Alfred} in 1948
\cite{Tarski}.\footnote{Tarski's result actually applies to a subset of the
geometry discussed in elementary textbooks.  This subset includes most of what
would be considered elementary geometry but it is not powerful enough to
express, among other things, the notions of the circumference and area of a
circle.  Extending the theory in a way that includes notions such as these
makes the theory undecidable, as was also shown by Tarski.  Tarski's algorithm
is far too inefficient to implement practically on a computer.  A practical
algorithm for proving a smaller subset of geometry theorems (those not
involving concepts of ``order'' or ``continuity'') was discovered by Chinese
mathematician Wu Wen-ts\"{u}n in 1977 \cite{Chou}\index{Chou,
Shang-Ching}.}\index{Wen-ts{\"{u}}n, Wu}  But most theories, including
elementary arithmetic, are undecidable.  This fact contributes to keeping
mathematics alive and well, since many mathematicians believe
that they will never be
replaced by computers (if they believe Roger Penrose's argument that a
computer can never replace the brain \cite{Penrose}\index{Penrose, Roger}).
In fact,  elementary geometry is often considered a ``dead'' field
for the simple reason that it is decidable.

On the other hand, the undecidability of a theory does not mean that one cannot
use a computer to search for proofs, providing one is willing to give up if a
proof is not found after a reasonable amount of time.  The field of automated
theorem proving\index{automated theorem proving} specializes in pursuing such
computer searches.  Among the more successful results to date are those based
on an algorithm known as Robinson's resolution principle
\cite{Robinson}\index{Robinson's resolution principle}.

Automated theorem provers can be excellent tools for those willing to learn
how to use them.  But they are not widely used in mainstream pure
mathematics, even though they could probably be useful in many areas.  There
are several reasons for this.  Probably most important, the main goal in pure
mathematics is to arrive at results that are considered to be deep or
important; proving them is essential but secondary.  Usually, an automated
theorem prover cannot assist in this main goal, and by the time the main goal
is achieved, the mathematician may have already figured out the proof as a
by-product.  There is also a notational problem.  Mathematicians are used to
using very compact syntax where one or two symbols (heavily dependent on
context) can represent very complex concepts; this is part of the
hierarchy\index{hierarchy} they have built up to tackle difficult problems.  A
theorem prover on the other hand might require that a theorem be expressed in
``first-order logic,''\index{first-order logic} which is the logic on which
most of mathematics is ultimately based but which is not ordinarily used
directly because expressions can become very long.  Some automated theorem
provers are experimental programs, limited in their use to very specialized
areas, and the goal of many is simply research into the nature of automated
theorem proving itself.  Finally, much research remains to be done to enable
them to prove very deep theorems.  One significant result was a
computer proof by Larry Wos\index{Wos, Larry} and colleagues that every Robbins
algebra\index{Robbins algebra} is a Boolean  algebra\index{Boolean algebra}
({\em New York Times}, Dec. 10, 1996).\footnote{In 1933, E.~V.\
Huntington\index{Huntington, E. V.}
presented the following axiom system for
Boolean algebra with a unary operation $n$ and a binary operation $+$:
\begin{center}
    $x + y = y + x$ \\
    $(x + y) + z = x + (y + z)$ \\
    $n(n(x) + y) + n(n(x) + n(y)) = x$
\end{center}
Herbert Robbins\index{Robbins, Herbert}, a student of Huntington, conjectured
that the last equation can be replaced with a simpler one:
\begin{center}
    $n(n(x + y) + n(x + n(y))) = x$
\end{center}
Robbins and Huntington could not find a proof.  The problem was
later studied unsuccessfully by Tarski\index{Tarski, Alfred} and his
students, and it remained an unsolved problem until a
computer found the proof in 1996.  For more information on
the Robbins algebra problem see \cite{Wos}.}

How does Metamath\index{Metamath} relate to automated theorem provers?  A
theorem prover is primarily concerned with one theorem at a time (perhaps
tapping into a small database of known theorems) whereas Metamath is more like
a theorem archiving system, storing both the theorem and its proof in a
database for access and verification.  Metamath is one answer to ``what do you
do with the output of a theorem prover?''  and could be viewed as the
next step in the process.  Automated theorem provers could be useful tools for
helping develop its database.
Note that very long, automatically
generated proofs can make your database fat and ugly and cause Metamath's proof
verification to take a long time to run.  Unless you have a particularly good
program that generates very concise proofs, it might be best to consider the
use of automatically generated proofs as a quick-and-dirty approach, to be
manually rewritten at some later date.

The program {\sc otter}\index{otter@{\sc otter}}\footnote{\url{http://www.cs.unm.edu/\~mccune/otter/}.}, later succeeded by
prover9\index{prover9}\footnote{\url{https://www.cs.unm.edu/~mccune/mace4/}.},
have been historically influential.
The E prover\index{E prover}\footnote{\url{https://github.com/eprover/eprover}.}
is a maintained automated theorem prover
for full first-order logic with equality.
There are many other automated theorem provers as well.

If you want to combine automated theorem provers with Metamath
consider investigating
the book {\em Automated Reasoning:  Introduction and Applications}
\cite{Wos}\index{Wos, Larry}.  This book discusses
how to use {\sc otter} in a way that can
not only able to generate
relatively efficient proofs, it can even be instructed to search for
shorter proofs.  The effective use of {\sc otter} (and similar tools)
does require a certain
amount of experience, skill, and patience.  The axiom system used in the
\texttt{set.mm}\index{set theory database (\texttt{set.mm})} set theory
database can be expressed to {\sc otter} using a method described in
\cite{Megill}.\index{Megill, Norman}\footnote{To use those axioms with
{\sc otter}, they must be restated in a way that eliminates the need for
``dummy variables.''\index{dummy variable!eliminating} See the Comment
on p.~\pageref{nodd}.} When successful, this method tends to generate
short and clever proofs, but my experiments with it indicate that the
method will find proofs within a reasonable time only for relatively
easy theorems.  It is still fun to experiment with.

Reference \cite{Bledsoe}\index{Bledsoe, W. W.} surveys a number of approaches
people have explored in the field of automated theorem proving\index{automated
theorem proving}.

\subsection{Interactive Theorem Provers}\label{interactivetheoremprovers}

Finding proofs completely automatically is difficult, so there
are some interactive theorem provers that allow a human to guide the
computer to find a proof.
Examples include
HOL Light\index{HOL light}%
\footnote{\url{https://www.cl.cam.ac.uk/~jrh13/hol-light/}.},
Isabelle\index{Isabelle}%
\footnote{\url{http://www.cl.cam.ac.uk/Research/HVG/Isabelle}.},
{\sc hol}\index{hol@{\sc hol}}%
\footnote{\url{https://hol-theorem-prover.org/}.},
and
Coq\index{Coq}\footnote{\url{https://coq.inria.fr/}.}.

A major difference between most of these tools and Metamath is that the
``proofs'' are actually programs that guide the program to find a proof,
and not the proof itself.
For example, an Isabelle/HOL proof might apply a step
\texttt{apply (blast dest: rearrange reduction)}. The \texttt{blast}
instruction applies
an automatic tableux prover and returns if it found a sequence of proof
steps that work... but the sequence is not considered part of the proof.

A good overview of
higher-level proof verification languages (such as {\sc lcf}\index{lcf@{\sc
lcf}} and {\sc hol}\index{hol@{\sc hol}})
is given in \cite{Harrison}.  All of these languages are fundamentally
different from Metamath in that much of the mathematical foundational
knowledge is embedded in the underlying proof-verification program, rather
than placed directly in the database that is being verified.
These can have a steep learning curve for those without a mathematical
background.  For example, one usually must have a fair understanding of
mathematical logic in order to follow their proofs.

\subsection{Proof Verifiers}\label{proofverifiers}

A proof verifier is a program that doesn't generate proofs but instead
verifies proofs that you give it.  Many proof verifiers have limited built-in
automated proof capabilities, such as figuring out simple logical inferences
(while still being guided by a person who provides the overall proof).  Metamath
has no built-in automated proof capability other than the limited
capability of its Proof Assistant.

Proof-verification languages are not used as frequently as they might be.
Pure mathematicians are more concerned with producing new results, and such
detail and rigor would interfere with that goal.  The use of computers in pure
mathematics is primarily focused on automated theorem provers (not verifiers),
again with the ultimate goal of aiding the creation of new mathematics.
Automated theorem provers are usually concerned with attacking one theorem at
time rather than making a large, organized database easily available to the
user.  Metamath is one way to help close this gap.

By itself Metamath is a mostly a proof verifier.
This does not mean that other approaches can't be used; the difference
is that in Metamath, the results of various provers must be recorded
step-by-step so that they can be verified.

Another proof-verification language is Mizar,\index{Mizar} which can display
its proofs in the informal language that mathematicians are accustomed to.
Information on the Mizar language is available at \url{http://mizar.org}.

For the working mathematician, Mizar is an excellent tool for rigorously
documenting proofs. Mizar typesets its proofs in the informal English used by
mathematicians (and, while fine for them, are just as inscrutable by
laypersons!). A price paid for Mizar is a relatively steep learning curve of a
couple of weeks.  Several mathematicians are actively formalizing different
areas of mathematics using Mizar and publishing the proofs in a dedicated
journal. Unfortunately the task of formalizing mathematics is still looked
down upon to a certain extent since it doesn't involve the creation of ``new''
mathematics.

The closest system to Metamath is
the {\em Ghilbert}\index{Ghilbert} proof language (\url{http://ghilbert.org})
system developed by
Raph Levien\index{Levien, Raph}.
Ghilbert is a formal proof checker heavily inspired by Metamath.
Ghilbert statements are s-expressions (a la Lisp), which is easy
for computers to parse but many people find them hard to read.
There are a number of differences in their specific constructs, but
there is at least one tool to translate some Metamath materials into Ghilbert.
As of 2019 the Ghilbert community is smaller and less active than the
Metamath community.
That said, the Metamath and Ghilbert communities overlap, and fruitful
conversations between them have occurred many times over the years.

\subsection{Creating a Database of Formalized Mathematics}\label{mathdatabase}

Besides Metamath, there are several other ongoing projects with the goal of
formalizing mathematics into computer-verifiable databases.
Understanding some history will help.

The {\sc qed}\index{qed project@{\sc qed} project}%
\footnote{\url{http://www-unix.mcs.anl.gov/qed}.}
project arose in 1993 and its goals were outlined in the
{\sc qed} manifesto.
The {\sc qed} manifesto was
a proposal for a computer-based database of all mathematical knowledge,
strictly formalized and with all proofs having been checked automatically.
The project had a conference in 1994 and another in 1995;
there was also a ``twenty years of the {\sc qed} manifesto'' workshop
in 2014.
Its ideals are regularly reraised.

In a 2007 paper, Freek Wiedijk identified two reasons
for the failure of the {\sc qed} project as originally envisioned:%
\cite{Wiedijk-revisited}\index{Wiedijk, Freek}

\begin{itemize}
\item Very few people are working on formalization of mathematics. There is no compelling application for fully mechanized mathematics.
\item Formalized mathematics does not yet resemble traditional mathematics. This is partly due to the complexity of mathematical notation, and partly to the limitations of existing theorem provers and proof assistants.
\end{itemize}

But this did not end the dream of
formalizing mathematics into computer-verifiable databases.
The problems that led to the {\sc qed} manifesto are still with us,
even though the challenges were harder than originally considered.
What has happened instead is that various independent projects have
worked towards formalizing mathematics into computer-verifiable databases,
each simultaneously competing and cooperating with each other.

A concrete way to see this is
Freek Wiedijk's ``Formalizing 100 Theorems'' list%
\footnote{\url{http://www.cs.ru.nl/\%7Efreek/100/}.}
which shows the progress different systems have made on a challenge list
of 100 mathematical theorems.%
\footnote{ This is not the only list of ``interesting'' theorems.
Another interesting list was posted by Oliver Knill's list
\cite{Knill}\index{Knill, Oliver}.}
The top systems as of February 2019
(in order of the number of challenges completed) are
HOL Light, Isabelle, Metamath, Coq, and Mizar.

The Metamath 100%
\footnote{\url{http://us.metamath.org/mm\_100.html}}
page (maintained by David A. Wheeler\index{Wheeler, David A.})
shows the progress of Metamath (specifically its \texttt{set.mm} database)
against this challenge list maintained by Freek Wiedijk.
The Metamath \texttt{set.mm} database
has made a lot of progress over the years,
in part because working to prove those challenge theorems required
defining various terms and proving their properties as a prerequisite.
Here are just a few of the many statements that have been
formally proven with Metamath:

% The entries of this cause the narrow display to break poorly,
% since the short amount of text means LaTeX doesn't get a lot to work with
% and the itemize format gives it even *less* margin than usual.
% No one will mind if we make just this list flushleft, since this list
% will be internally consistent.
\begin{flushleft}
\begin{itemize}
\item 1. The Irrationality of the Square Root of 2
  (\texttt{sqr2irr}, by Norman Megill, 2001-08-20)
\item 2. The Fundamental Theorem of Algebra
  (\texttt{fta}, by Mario Carneiro, 2014-09-15)
\item 22. The Non-Denumerability of the Continuum
  (\texttt{ruc}, by Norman Megill, 2004-08-13)
\item 54. The Konigsberg Bridge Problem
  (\texttt{konigsberg}, by Mario Carneiro, 2015-04-16)
\item 83. The Friendship Theorem
  (\texttt{friendship}, by Alexander W. van der Vekens, 2018-10-09)
\end{itemize}
\end{flushleft}

We thank all of those who have developed at least one of the Metamath 100
proofs, and we particularly thank
Mario Carneiro\index{Carneiro, Mario}
who has contributed the most Metamath 100 proofs as of 2019.
The Metamath 100 page shows the list of all people who have contributed a
proof, and links to graphs and charts showing progress over time.
We encourage others to work on proving theorems not yet proven in Metamath,
since doing so improves the work as a whole.

Each of the math formalization systems (including Metamath)
has different strengths and weaknesses, depending on what you value.
Key aspects that differentiate Metamath from the other top systems are:

\begin{itemize}
\item Metamath is not tied to any particular set of axioms.
\item Metamath can show every step of every proof, no exceptions.
  Most other provers only assert that a proof can be found, and do not
  show every step. This also makes verification fast, because
  the system does not need to rediscover proof details.
\item The Metamath verifier has been re-implemented in many different
  programming languages, so verification can be done by multiple
  implementations.  In particular, the
  \texttt{set.mm}\index{set theory database (\texttt{set.mm})}%
  \index{Metamath Proof Explorer} database is verified by
  four different verifiers
  written in four different languages by four different authors.
  This greatly reduces the risk of accepting an invalid
  proof due to an error in the verifier.
\item Proofs stay proven.  In some systems, changes to the system's
  syntax or how a tactic works causes proofs to fail in later versions,
  causing older work to become essentially lost.
  Metamath's language is
  extremely small and fixed, so once a proof is added to a database,
  the database can be rechecked with later versions of the Metamath program
  and with other verifiers of Metamath databases.
  If an axiom or key definition needs to be changed, it is easy to
  manipulate the database as a whole to handle the change
  without touching the underlying verifier.
  Since re-verification of an entire database takes seconds, there
  is never a reason to delay complete verification.
  This aspect is especially compelling if your
  goal is to have a long-term database of proofs.
\item Licensing is generous.  The main Metamath databases are released to
  the public domain, and the main Metamath program is open source software
  under a standard, widely-used license.
\item Substitutions are easy to understand, even by those who are not
  professional mathematicians.
\end{itemize}

Of course, other systems may have advantages over Metamath
that are more compelling, depending on what you value.
In any case, we hope this helps you understand Metamath
within a wider context.

\subsection{In Summary}\label{computers-summary}

To summarize our discussions of computers and mathematics, computer algebra
systems can be viewed as theorem generators focusing on a narrow realm of
mathematics (numbers and their properties), automated theorem provers as proof
generators for specific theorems in a much broader realm covered by a built-in
formal system such as first-order logic, interactive theorem
provers require human guidance, proof verifiers verify proofs but
historically they have been
restricted to first-order logic.
Metamath, in contrast,
is a proof verifier and documenter whose realm is essentially unlimited.

\section{Mathematics and Metamath}

\subsection{Standard Mathematics}

There are a number of ways that Metamath\index{Metamath} can be used with
standard mathematics.  The most satisfying way philosophically is to start at
the very beginning, and develop the desired mathematics from the axioms of
logic and set theory.\index{set theory}  This is the approach taken in the
\texttt{set.mm}\index{set theory database (\texttt{set.mm})}%
\index{Metamath Proof Explorer}
database (also known as the Metamath Proof Explorer).
Among other things, this database builds up to the
axioms of real and complex numbers\index{analysis}\index{real and complex
numbers} (see Section~\ref{real}), and a standard development of analysis, for
example, could start at that point, using it as a basis.   Besides this
philosophical advantage, there are practical advantages to having all of the
tools of set theory available in the supporting infrastructure.

On the other hand, you may wish to start with the standard axioms of a
mathematical theory without going through the set theoretical proofs of those
axioms.  You will need mathematical logic to make inferences, but if you wish
you can simply introduce theorems\index{theorem} of logic as
``axioms''\index{axiom} wherever you need them, with the implicit assumption
that in principle they can be proved, if they are obvious to you.  If you
choose this approach, you will probably want to review the notation used in
\texttt{set.mm}\index{set theory database (\texttt{set.mm})} so that your own
notation will be consistent with it.

\subsection{Other Formal Systems}
\index{formal system}

Unlike some programs, Metamath\index{Metamath} is not limited to any specific
area of mathematics, nor committed to any particular mathematical philosophy
such as classical logic versus intuitionism, nor limited, say, to expressions
in first-order logic.  Although the database \texttt{set.mm}
describes standard logic and set theory, Meta\-math
is actually a general-purpose language for describing a wide variety of formal
systems.\index{formal system}  Non-standard systems such as modal
logic,\index{modal logic} intuitionist logic\index{intuitionism}, higher-order
logic\index{higher-order logic}, quantum logic\index{quantum logic}, and
category theory\index{category theory} can all be described with the Metamath
language.  You define the symbols you prefer and tell Metamath the axioms and
rules you want to start from, and Metamath will verify any inferences you make
from those axioms and rules.  A simple example of a non-standard formal system
is Hofstadter's\index{Hofstadter, Douglas R.} MIU system,\index{MIU-system}
whose Metamath description is presented in Appendix~\ref{MIU}.

This is not hypothetical.
The largest Metamath database is
\texttt{set.mm}\index{set theory database (\texttt{set.mm}}%
\index{Metamath Proof Explorer}), aka the Metamath Proof Explorer,
which uses the most common axioms for mathematical foundations
(specifically classical logic combined with Zermelo--Fraenkel
set theory\index{Zermelo--Fraenkel set theory} with the Axiom of Choice).
But other Metamath databases are available:

\begin{itemize}
\item The database
  \texttt{iset.mm}\index{intuitionistic logic database (\texttt{iset.mm})},
  aka the
  Intuitionistic Logic Explorer\index{Intuitionistic Logic Explorer},
  uses intuitionistic logic (a constructivist point of view)
  instead of classical logic.
\item The database
  \texttt{nf.mm}\index{New Foundations database (\texttt{nf.mm})},
  aka the
  New Foundations Explorer\index{New Foundations Explorer},
  constructs mathematics from scratch,
  starting from Quine's New Foundations (NF) set theory axioms.
\item The database
  \texttt{hol.mm}\index{Higher-order Logic database (\texttt{hol.mm})},
  aka the
  Higher-Order Logic (HOL) Explorer\index{Higher-Order Logic (HOL) Explorer},
  starts with HOL (also called simple type theory) and derives
  equivalents to ZFC axioms, connecting the two approaches.
\end{itemize}

Since the days of David Hilbert,\index{Hilbert, David} mathematicians have
been concerned with the fact that the metalanguage\index{metalanguage} used to
describe mathematics may be stronger than the mathematics being described.
Metamath\index{Metamath}'s underlying finitary\index{finitary proof},
constructive nature provides a good philosophical basis for studying even the
weakest logics.\index{weak logic}

The usual treatment of many non-standard formal systems\index{formal
system} uses model theory\index{model theory} or proof theory\index{proof
theory} to describe these systems; these theories, in turn, are based on
standard set theory.  In other words, a non-standard formal system is defined
as a set with certain properties, and standard set theory is used to derive
additional properties of this set.  The standard set theory database provided
with Metamath can be used for this purpose, and when used this way
the development of a special
axiom system for the non-standard formal system becomes unnecessary.  The
model- or proof-theoretic approach often allows you to prove much deeper
results with less effort.

Metamath supports both approaches.  You can define the non-standard
formal system directly, or define the non-standard formal system as
a set with certain properties, whichever you find most helpful.

%\section{Additional Remarks}

\subsection{Metamath and Its Philosophy}

Closely related to Metamath\index{Metamath} is a philosophy or way of looking
at mathematics. This philosophy is related to the formalist
philosophy\index{formalism} of Hilbert\index{Hilbert, David} and his followers
\cite[pp.~1203--1208]{Kline}\index{Kline, Morris}
\cite[p.~6]{Behnke}\index{Behnke, H.}. In this philosophy, mathematics is
viewed as nothing more than a set of rules that manipulate symbols, together
with the consequences of those rules.  While the mathematics being described
may be complex, the rules used to describe it (the
``metamathematics''\index{metamathematics}) should be as simple as possible.
In particular, proofs should be restricted to dealing with concrete objects
(the symbols we write on paper rather than the abstract concepts they
represent) in a constructive manner; these are called ``finitary''
proofs\index{finitary proof} \cite[pp.~2--3]{Shoenfield}\index{Shoenfield,
Joseph R.}.

Whether or not you find Metamath interesting or useful will in part depend on
the appeal you find in its philosophy, and this appeal will probably depend on
your particular goals with respect to mathematics.  For example, if you are a
pure mathematician at the forefront of discovering new mathematical knowledge,
you will probably find that the rigid formality of Metamath stifles your
creativity.  On the other hand, we would argue that once this knowledge is
discovered, there are advantages to documenting it in a standard format that
will make it accessible to others.  Sixty years from now, your field may be
dormant, and as Davis and Hersh put it, your ``writings would become less
translatable than those of the Maya'' \cite[p.~37]{Davis}\index{Davis, Phillip
J.}.


\subsection{A History of the Approach Behind Metamath}

Probably the one work that has had the most motivating influence on
Metamath\index{Metamath} is Whitehead and Russell's monumental {\em Principia
Mathematica} \cite{PM}\index{Whitehead, Alfred North}\index{Russell,
Bertrand}\index{principia mathematica@{\em Principia Mathematica}}, whose aim
was to deduce all of mathematics from a small number of primitive ideas, in a
very explicit way that in principle anyone could understand and follow.  While
this work was tremendously influential in its time, from a modern perspective
it suffers from several drawbacks.  Both its notation and its underlying
axioms are now considered dated and are no longer used.  From our point of
view, its development is not really as accessible as we would like to see; for
practical reasons, proofs become more and more sketchy as its mathematics
progresses, and working them out in fine detail requires a degree of
mathematical skill and patience that many people don't have.  There are
numerous small errors, which is understandable given the tedious, technical
nature of the proofs and the lack of a computer to verify the details.
However, even today {\em Principia Mathematica} stands out as the work closest
in spirit to Metamath.  It remains a mind-boggling work, and one can't help
but be amazed at seeing ``$1+1=2$'' finally appear on page 83 of Volume II
(Theorem *110.643).

The origin of the proof notation used by Metamath dates back to the 1950's,
when the logician C.~A.~Meredith expressed his proofs in a compact notation
called ``condensed detachment''\index{condensed detachment}
\cite{Hindley}\index{Hindley, J. Roger} \cite{Kalman}\index{Kalman, J. A.}
\cite{Meredith}\index{Meredith, C. A.} \cite{Peterson}\index{Peterson, Jeremy
George}.  This notation allows proofs to be communicated unambiguously by
merely referencing the axiom\index{axiom}, rule\index{rule}, or
theorem\index{theorem} used at each step, without explicitly indicating the
substitutions\index{substitution!variable}\index{variable substitution} that
have to be made to the variables in that axiom, rule, or theorem.  Ordinarily,
condensed detachment is more or less limited to propositional
calculus\index{propositional calculus}.  The concept has been extended to
first-order logic\index{first-order logic} in \cite{Megill}\index{Megill,
Norman}, making it is easy to write a small computer program to verify proofs
of simple first-order logic theorems.\index{condensed detachment!and
first-order logic}

A key concept behind the notation of condensed detachment is called
``unification,''\index{unification} which is an algorithm for determining what
substitutions\index{substitution!variable}\index{variable substitution} to
variables have to be made to make two expressions match each other.
Unification was first precisely defined by the logician J.~A.~Robinson, who
used it in the development of a powerful
theorem-proving technique called the ``resolution principle''
\cite{Robinson}\index{Robinson's resolution principle}. Metamath does not make
use of the resolution principle, which is intended for systems of first-order
logic.\index{first-order logic}  Metamath's use is not restricted to
first-order logic, and as we have mentioned it does not automatically discover
proofs.  However, unification is a key idea behind Metamath's proof
notation, and Metamath makes use of a very simple version of it
(Section~\ref{unify}).

\subsection{Metamath and First-Order Logic}

First-order logic\index{first-order logic} is the supporting structure
for standard mathematics.  On top of it is set theory, which contains
the axioms from which virtually all of mathematics can be derived---a
remarkable fact.\index{category
theory}\index{cardinal, inaccessible}\label{categoryth}\footnote{An exception seems
to be category theory.  There are several schools of thought on whether
category theory is derivable from set theory.  At a minimum, it appears
that an additional axiom is needed that asserts the existence of an
``inaccessible cardinal'' (a type of infinity so large that standard set
theory can't prove or deny that it exists).
%
%%%% (I took this out that was in previous editions:)
% But it is also argued that not just one but a ``proper class'' of them
% is needed, and the existence of proper classes is impossible in standard
% set theory.  (A proper class is a collection of sets so huge that no set
% can contain it as an element.  Proper classes can lead to
% inconsistencies such as ``Russell's paradox.''  The axioms of standard
% set theory are devised so as to deny the existence of proper classes.)
%
For more information, see
\cite[pp.~328--331]{Herrlich}\index{Herrlich, Horst} and
\cite{Blass}\index{Blass, Andrea}.}

One of the things that makes Metamath\index{Metamath} more practical for
first-order theories is a set of axioms for first-order logic designed
specifically with Metamath's approach in mind.  These are included in
the database \texttt{set.mm}\index{set theory database (\texttt{set.mm})}.
See Chapter~\ref{fol} for a detailed
description; the axioms are shown in Section~\ref{metaaxioms}.  While
logically equivalent to standard axiom systems, our axiom system breaks
up the standard axioms into smaller pieces such that from them, you can
directly derive what in other systems can only be derived as higher-level
``metatheorems.''\index{metatheorem}  In other words, it is more powerful than
the standard axioms from a metalogical point of view.  A rigorous
justification for this system and its ``metalogical
completeness''\index{metalogical completeness} is found in
\cite{Megill}\index{Megill, Norman}.  The system is closely related to a
system developed by Monk\index{Monk, J. Donald} and Tarski\index{Tarski,
Alfred} in 1965 \cite{Monks}.

For example, the formula $\exists x \, x = y $ (given $y$, there exists some
$x$ equal to it) is a theorem of logic,\footnote{Specifically, it is a theorem
of those systems of logic that assume non-empty domains.  It is not a theorem
of more general systems that include the empty domain\index{empty domain}, in
which nothing exists, period!  Such systems are called ``free
logics.''\index{free logic} For a discussion of these systems, see
\cite{Leblanc}\index{Leblanc, Hugues}.  Since our use for logic is as a basis
for set theory, which has a non-empty domain, it is more convenient (and more
traditional) to use a less general system.  An interesting curiosity is that,
using a free logic as a basis for Zermelo--Fraenkel set
theory\index{Zermelo--Fraenkel set theory} (with the redundant Axiom of the
Null Set omitted),\index{Axiom of the Null Set} we cannot even prove the
existence of a single set without assuming the axiom of infinity!\index{Axiom
of Infinity}} whether or not $x$ and $y$ are distinct variables\index{distinct
variables}.  In many systems of logic, we would have to prove two theorems to
arrive at this result.  First we would prove ``$\exists x \, x = x $,'' then
we would separately prove ``$\exists x \, x = y $, where $x$ and $y$ are
distinct variables.''  We would then combine these two special cases ``outside
of the system'' (i.e.\ in our heads) to be able to claim, ``$\exists x \, x =
y $, regardless of whether $x$ and $y$ are distinct.''  In other words, the
combination of the two special cases is a
metatheorem.  In the system of logic
used in Metamath's set theory\index{set theory database (\texttt{set.mm})}
database, the axioms of logic are broken down into small pieces that allow
them to be reassembled in such a way that theorems such as these can be proved
directly.

Breaking down the axioms in this way makes them look peculiar and not very
intuitive at first, but rest assured that they are correct and complete.  Their
correctness is ensured because they are theorem schemes of standard first-order
logic (which you can easily verify if you are a logician).  Their completeness
follows from the fact that we explicitly derive the standard axioms of
first-order logic as theorems.  Deriving the standard axioms is somewhat
tricky, but once we're there, we have at our disposal a system that is less
awkward to work with in formal proofs\index{formal proof}.  In technical terms
that logicians understand, we eliminate the cumbersome concepts of ``free
variable,''\index{free variable} ``bound variable,''\index{bound variable} and
``proper substitution''\index{proper substitution}\index{substitution!proper}
as primitive notions.  These concepts are present in our system but are
defined in terms of concepts expressed by the axioms and can be eliminated in
principle.  In standard systems, these concepts are really like additional,
implicit axioms\index{implicit axiom} that are somewhat complex and cannot be
eliminated.

The traditional approach to logic, wherein free variables and proper
substitution is defined, is also possible to do directly in the Metamath
language.  However, the notation tends to become awkward, and there are
disadvantages:  for example, extending the definition of a wff with a
definition is awkward, because the free variable and proper substitution
concepts have to have their definitions also extended.  Our choice of
axioms for \texttt{set.mm} is to a certain extent a matter of style, in
an attempt to achieve overall simplicity, but you should also be aware
that the traditional approach is possible as well if you should choose
it.

\chapter{Using the Metamath Program}
\label{using}

\section{Installation}

The way that you install Metamath\index{Metamath!installation} on your
computer system will vary for different computers.  Current
instructions are provided with the Metamath program download at
\url{http://metamath.org}.  In general, the installation is simple.
There is one file containing the Metamath program itself.  This file is
usually called \texttt{metamath} or \texttt{metamath.}{\em xxx} where
{\em xxx} is the convention (such as \texttt{exe}) for an executable
program on your operating system.  There are several additional files
containing samples of the Metamath language, all ending with
\texttt{.mm}.  The file \texttt{set.mm}\index{set theory database
(\texttt{set.mm})} contains logic and set theory and can be used as a
starting point for other areas of mathematics.

You will also need a text editor\index{text editor} capable of editing plain
{\sc ascii}\footnote{American Standard Code for Information Interchange.} text
in order to prepare your input files.\index{ascii@{\sc ascii}}  Most computers
have this capability built in.  Note that plain text is not necessarily the
default for some word processing programs\index{word processor}, especially if
they can handle different fonts; for example, with Microsoft Word\index{Word
(Microsoft)}, you must save the file in the format ``Text Only With Line
Breaks'' to get a plain text\index{plain text} file.\footnote{It is
recommended that all lines in a Metamath source file be 79 characters or less
in length for compatibility among different computer terminals.  When creating
a source file on an editor such as Word, select a monospaced
font\index{monospaced font} such as Courier\index{Courier font} or
Monaco\index{Monaco font} to make this easier to achieve.  Better yet,
just use a plain text editor such as Notepad.}

On some computer systems, Metamath does not have the capability to print
its output directly; instead, you send its output to a file (using the
\texttt{open} commands described later).  The way you print this output
file depends on your computer.\index{printers} Some computers have a
print command, whereas with others, you may have to read the file into
an editor and print it from there.

If you want to print your Metamath source files with typeset formulas
containing standard mathematical symbols, you will need the \LaTeX\
typesetting program\index{latex@{\LaTeX}}, which is widely and freely
available for most operating systems.  It runs natively on Unix and
Linux, and can be installed on Windows as part of the free Cygwin
package (\url{http://cygwin.com}).

You can also produce {\sc html}\footnote{HyperText Markup Language.}
web pages.  The {\tt help html} command in the Metamath program will
assist you with this feature.

\section{Your First Formal System}\label{start}
\subsection{From Nothing to Zero}\label{startf}

To give you a feel for what the Metamath\index{Metamath} language looks like,
we will take a look at a very simple example from formal number
theory\index{number theory}.  This example is taken from
Mendelson\index{Mendelson, Elliot} \cite[p. 123]{Mendelson}.\footnote{To keep
the example simple, we have changed the formalism slightly, and what we call
axioms\index{axiom} are strictly speaking theorems\index{theorem} in
\cite{Mendelson}.}  We will look at a small subset of this theory, namely that
part needed for the first number theory theorem proved in \cite{Mendelson}.

First we will look at a standard formal proof\index{formal proof} for the
example we have picked, then we will look at the Metamath version.  If you
have never been exposed to formal proofs, the notation may seem to be such
overkill to express such simple notions that you may wonder if you are missing
something.  You aren't.  The concepts involved are in fact very simple, and a
detailed breakdown in this fashion is necessary to express the proof in a way
that can be verified mechanically.  And as you will see, Metamath breaks the
proof down into even finer pieces so that the mechanical verification process
can be about as simple as possible.

Before we can introduce the axioms\index{axiom} of the theory, we must define
the syntax rules for forming legal expressions\index{syntax rules}
(combinations of symbols) with which those axioms can be used. The number 0 is
a {\bf term}\index{term}; and if $ t$ and $r$ are terms, so is $(t+r)$. Here,
$ t$ and $r$ are ``metavariables''\index{metavariable} ranging over terms; they
themselves do not appear as symbols in an actual term.  Some examples of
actual terms are $(0 + 0)$ and $((0+0)+0)$.  (Note that our theory describes
only the number zero and sums of zeroes.  Of course, not much can be done with
such a trivial theory, but remember that we have picked a very small subset of
complete number theory for our example.  The important thing for you to focus
on is our definitions that describe how symbols are combined to form valid
expressions, and not on the content or meaning of those expressions.) If $ t$
and $r$ are terms, an expression of the form $ t=r$ is a {\bf wff}
(well-formed formula)\index{well-formed formula (wff)}; and if $P$ and $Q$ are
wffs, so is $(P\rightarrow Q)$ (which means ``$P$ implies
$Q$''\index{implication ($\rightarrow$)} or ``if $P$ then $Q$'').
Here $P$ and $Q$ are metavariables ranging over wffs.  Examples of actual
wffs are $0=0$, $(0+0)=0$, $(0=0 \rightarrow (0+0)=0)$, and $(0=0\rightarrow
(0=0\rightarrow 0=(0+0)))$.  (Our notation makes use of more parentheses than
are customary, but the elimination of ambiguity this way simplifies our
example by avoiding the need to define operator precedence\index{operator
precedence}.)

The {\bf axioms}\index{axiom} of our theory are all wffs of the following
form, where $ t$, $r$, and $s$ are any terms:

%Latex p. 92
\renewcommand{\theequation}{A\arabic{equation}}

\begin{equation}
(t=r\rightarrow (t=s\rightarrow r=s))
\end{equation}
\begin{equation}
(t+0)=t
\end{equation}

Note that there are an infinite number of axioms since there are an infinite
number of possible terms.  A1 and A2 are properly called ``axiom
schemes,''\index{axiom scheme} but we will refer to them as ``axioms'' for
brevity.

An axiom is a {\bf theorem}; and if $P$ and $(P\rightarrow Q)$ are theorems
(where $P$ and $Q$ are wffs), then $Q$ is also a theorem.\index{theorem}  The
second part of this definition is called the modus ponens (MP) rule of
inference\index{inference rule}\index{modus ponens}.  It allows us to obtain
new theorems from old ones.

The {\bf proof}\index{proof} of a theorem is a sequence of one or more
theorems, each of which is either an axiom or the result of modus ponens
applied to two previous theorems in the sequence, and the last of which is the
theorem being proved.

The theorem we will prove for our example is very simple:  $ t=t$.  The proof of
our theorem follows.  Study it carefully until you feel sure you
understand it.\label{zeroproof}

% Use tabu so that lines will wrap automatically as needed.
\begin{tabu} { l X X }
1. & $(t+0)=t$ & (by axiom A2) \\
2. & $(t+0)=t$ & (by axiom A2) \\
3. & $((t+0)=t \rightarrow ((t+0)=t\rightarrow t=t))$ & (by axiom A1) \\
4. & $((t+0)=t\rightarrow t=t)$ & (by MP applied to steps 2 and 3) \\
5. & $t=t$ & (by MP applied to steps 1 and 4) \\
\end{tabu}

(You may wonder why step 1 is repeated twice.  This is not necessary in the
formal language we have defined, but in Metamath's ``reverse Polish
notation''\index{reverse Polish notation (RPN)} for proofs, a previous step
can be referred to only once.  The repetition of step~1 here will enable you
to see more clearly the correspondence of this proof with the
Metamath\index{Metamath} version on p.~\pageref{demoproof}.)

Our theorem is more properly called a ``theorem scheme,''\index{theorem
scheme} for it represents an infinite number of theorems, one for each
possible term $ t$.  Two examples of actual theorems would be $0=0$ and
$(0+0)=(0+0)$.  Rarely do we prove actual theorems, since by proving schemes
we can prove an infinite number of theorems in one fell swoop.  Similarly, our
proof should really be called a ``proof scheme.''\index{proof scheme}  To
obtain an actual proof, pick an actual term to use in place of $ t$, and
substitute it for $ t$ throughout the proof.

Let's discuss what we have done here.  The axioms\index{axiom} of our theory,
A1 and A2, are trivial and obvious.  Everyone knows that adding zero to
something doesn't change it, and also that if two things are equal to a third,
then they are equal to each other. In fact, stating the trivial and obvious is
a goal to strive for in any axiomatic system.  From trivial and obvious truths
that everyone agrees upon, we can prove results that are not so obvious yet
have absolute faith in them.  If we trust the axioms and the rules, we must,
by definition, trust the consequences of those axioms and rules, if logic is
to mean anything at all.

Our rule of inference\index{rule}, modus ponens\index{modus ponens}, is also
pretty obvious once you understand what it means.  If we prove a fact $P$, and
we also prove that $P$ implies $Q$, then $Q$ necessarily follows as a new
fact.  The rule provides us with a means for obtaining new facts (i.e.\
theorems\index{theorem}) from old ones.

The theorem that we have proved, $ t=t$, is so fundamental that you may wonder
why it isn't one of the axioms\index{axiom}.  In some axiom systems of
arithmetic, it {\em is} an axiom.  The choice of axioms in a theory is to some
extent arbitrary and even an art form, constrained only by the requirement
that any two equivalent axiom systems be able to derive each other as
theorems.  We could imagine that the inventor of our axiom system originally
included $ t=t$ as an axiom, then discovered that it could be derived as a
theorem from the other axioms.  Because of this, it was not necessary to
keep it as an axiom.  By eliminating it, the final set of axioms became
that much simpler.

Unless you have worked with formal proofs\index{formal proof} before, it
probably wasn't apparent to you that $ t=t$ could be derived from our two
axioms until you saw the proof. While you certainly believe that $ t=t$ is
true, you might not be able to convince an imaginary skeptic who believes only
in our two axioms until you produce the proof.  Formal proofs such as this are
hard to come up with when you first start working with them, but after you get
used to them they can become interesting and fun.  Once you understand the
idea behind formal proofs you will have grasped the fundamental principle that
underlies all of mathematics.  As the mathematics becomes more sophisticated,
its proofs become more challenging, but ultimately they all can be broken down
into individual steps as simple as the ones in our proof above.

Mendelson's\index{Mendelson, Elliot} book, from which our example was taken,
contains a number of detailed formal proofs such as these, and you may be
interested in looking it up.  The book is intended for mathematicians,
however, and most of it is rather advanced.  Popular literature describing
formal proofs\index{formal proof} include \cite[p.~296]{Rucker}\index{Rucker,
Rudy} and \cite[pp.~204--230]{Hofstadter}\index{Hofstadter, Douglas R.}.

\subsection{Converting It to Metamath}\label{convert}

Formal proofs\index{formal proof} such as the one in our example break down
logical reasoning into small, precise steps that leave little doubt that the
results follow from the axioms\index{axiom}.  You might think that this would
be the finest breakdown we can achieve in mathematics.  However, there is more
to the proof than meets the eye. Although our axioms were rather simple, a lot
of verbiage was needed before we could even state them:  we needed to define
``term,'' ``wff,'' and so on.  In addition, there are a number of implied
rules that we haven't even mentioned. For example, how do we know that step 3
of our proof follows from axiom A1? There is some hidden reasoning involved in
determining this.  Axiom A1 has two occurrences of the letter $ t$.  One of
the implied rules states that whatever we substitute for $ t$ must be a legal
term\index{term}.\footnote{Some authors make this implied rule explicit by
stating, ``only expressions of the above form are terms,'' after defining
``term.''}  The expression $ t+0$ is pretty obviously a legal term whenever $
t$ is, but suppose we wanted to substitute a huge term with thousands of
symbols?  Certainly a lot of work would be involved in determining that it
really is a term, but in ordinary formal proofs all of this work would be
considered a single ``step.''

To express our axiom system in the Metamath\index{Metamath} language, we must
describe this auxiliary information in addition to the axioms themselves.
Metamath does not know what a ``term'' or a ``wff''\index{well-formed formula
(wff)} is.  In Metamath, the specification of the ways in which we can combine
symbols to obtain terms and wffs are like little axioms in themselves.  These
auxiliary axioms are expressed in the same notation as the ``real''
axioms\index{axiom}, and Metamath does not distinguish between the two.  The
distinction is made by you, i.e.\ by the way in which you interpret the
notation you have chosen to express these two kinds of axioms.

The Metamath language breaks down mathematical proofs into tiny pieces, much
more so than in ordinary formal proofs\index{formal proof}.  If a single
step\index{proof step} involves the
substitution\index{substitution!variable}\index{variable substitution} of a
complex term for one of its variables, Metamath must see this single step
broken down into many small steps.  This fine-grained breakdown is what gives
Metamath generality and flexibility as it lets it not be limited to any
particular mathematical notation.

Metamath's proof notation is not, in itself, intended to be read by humans but
rather is in a compact format intended for a machine.  The Metamath program
will convert this notation to a form you can understand, using the \texttt{show
proof}\index{\texttt{show proof} command} command.  You can tell the program what
level of detail of the proof you want to look at.  You may want to look at
just the logical inference steps that correspond
to ordinary formal proof steps,
or you may want to see the fine-grained steps that prove that an expression is
a term.

Here, without further ado, is our example converted to the
Metamath\index{Metamath} language:\index{metavariable}\label{demo0}

\begin{verbatim}
$( Declare the constant symbols we will use $)
    $c 0 + = -> ( ) term wff |- $.
$( Declare the metavariables we will use $)
    $v t r s P Q $.
$( Specify properties of the metavariables $)
    tt $f term t $.
    tr $f term r $.
    ts $f term s $.
    wp $f wff P $.
    wq $f wff Q $.
$( Define "term" and "wff" $)
    tze $a term 0 $.
    tpl $a term ( t + r ) $.
    weq $a wff t = r $.
    wim $a wff ( P -> Q ) $.
$( State the axioms $)
    a1 $a |- ( t = r -> ( t = s -> r = s ) ) $.
    a2 $a |- ( t + 0 ) = t $.
$( Define the modus ponens inference rule $)
    ${
       min $e |- P $.
       maj $e |- ( P -> Q ) $.
       mp  $a |- Q $.
    $}
$( Prove a theorem $)
    th1 $p |- t = t $=
  $( Here is its proof: $)
       tt tze tpl tt weq tt tt weq tt a2 tt tze tpl
       tt weq tt tze tpl tt weq tt tt weq wim tt a2
       tt tze tpl tt tt a1 mp mp
     $.
\end{verbatim}\index{metavariable}

A ``database''\index{database} is a set of one or more {\sc ascii} source
files.  Here's a brief description of this Metamath\index{Metamath} database
(which consists of this single source file), so that you can understand in
general terms what is going on.  To understand the source file in detail, you
should read Chapter~\ref{languagespec}.

The database is a sequence of ``tokens,''\index{token} which are normally
separated by spaces or line breaks.  The only tokens that are built into
the Metamath language are those beginning with \texttt{\$}.  These tokens
are called ``keywords.''\index{keyword}  All other tokens are
user-defined, and their names are arbitrary.

As you might have guessed, the Metamath token \texttt{\$(}\index{\texttt{\$(} and
\texttt{\$)} auxiliary keywords} starts a comment and \texttt{\$)} ends a comment.

The Metamath tokens \texttt{\$c}\index{\texttt{\$c} statement},
\texttt{\$v}\index{\texttt{\$v} statement},
\texttt{\$e}\index{\texttt{\$e} statement},
\texttt{\$f}\index{\texttt{\$f} statement},
\texttt{\$a}\index{\texttt{\$a} statement}, and
\texttt{\$p}\index{\texttt{\$p} statement} specify ``statements'' that
end with \texttt{\$.}\,.\index{\texttt{\$.}\ keyword}

The Metamath tokens \texttt{\$c} and \texttt{\$v} each declare\index{constant
declaration}\index{variable declaration} a list of user-defined tokens, called
``math symbols,''\index{math symbol} that the database will reference later
on.  All of the math symbols they define you have seen earlier except the
turnstile symbol \texttt{|-} ($\vdash$)\index{turnstile ({$\,\vdash$})}, which is
commonly used by logicians to mean ``a proof exists for.''  For us
the turnstile is just a
convenient symbol that distinguishes expressions that are axioms\index{axiom}
or theorems\index{theorem} from expressions that are terms or wffs.

The \texttt{\$c} statement declares ``constants''\index{constant} and
the \texttt{\$v} statement declares
``variables''\index{variable}\index{constant declaration}\index{variable
declaration} (or more precisely, metavariables\index{metavariable}).  A
variable may be substituted\index{substitution!variable}\index{variable
substitution} with sequences of math symbols whereas a constant may not
be substituted with anything.

It may seem redundant to require both \texttt{\$c}\index{\texttt{\$c} statement} and
\texttt{\$v}\index{\texttt{\$v} statement} statements (since any math
symbol\index{math symbol} not specified with a \texttt{\$c} statement could be
presumed to be a variable), but this provides for better error checking and
also allows math symbols to be redeclared\index{redeclaration of symbols}
(Section~\ref{scoping}).

The token \texttt{\$f}\index{\texttt{\$f} statement} specifies a
statement called a ``variable-type hypothesis'' (also called a
``floating hypothesis'') and \texttt{\$e}\index{\texttt{\$e} statement}
specifies a ``logical hypothesis'' (also called an ``essential
hypothesis'').\index{hypothesis}\index{variable-type
hypothesis}\index{logical hypothesis}\index{floating
hypothesis}\index{essential hypothesis} The token
\texttt{\$a}\index{\texttt{\$a} statement} specifies an ``axiomatic
assertion,''\index{axiomatic assertion} and
\texttt{\$p}\index{\texttt{\$p} statement} specifies a ``provable
assertion.''\index{provable assertion} To the left of each occurrence of
these four tokens is a ``label''\index{label} that identifies the
hypothesis or assertion for later reference.  For example, the label of
the first axiomatic assertion is \texttt{tze}.  A \texttt{\$f} statement
must contain exactly two math symbols, a constant followed by a
variable.  The \texttt{\$e}, \texttt{\$a}, and \texttt{\$p} statements
each start with a constant followed by, in general, an arbitrary
sequence of math symbols.

Associated with each assertion\index{assertion} is a set of hypotheses
that must be satisfied in order for the assertion to be used in a proof.
These are called the ``mandatory hypotheses''\index{mandatory
hypothesis} of the assertion.  Among those hypotheses whose ``scope''
(described below) includes the assertion, \texttt{\$e} hypotheses are
always mandatory and \texttt{\$f}\index{\texttt{\$f} statement}
hypotheses are mandatory when they share their variable with the
assertion or its \texttt{\$e} hypotheses.  The exact rules for
determining which hypotheses are mandatory are described in detail in
Sections~\ref{frames} and \ref{scoping}.  For example, the mandatory
hypotheses of assertion \texttt{tpl} are \texttt{tt} and \texttt{tr},
whereas assertion \texttt{tze} has no mandatory hypotheses because it
contains no variables and has no \texttt{\$e}\index{\texttt{\$e}
statement} hypothesis.  Metamath's \texttt{show statement}
command\index{\texttt{show statement} command}, described in the next
section, will show you a statement's mandatory hypotheses.

Sometimes we need to make a hypothesis relevant to only certain
assertions.  The set of statements to which a hypothesis is relevant is
called its ``scope.''  The Metamath brackets,
\texttt{\$\char`\{}\index{\texttt{\$\char`\{} and \texttt{\$\char`\}}
keywords} and \texttt{\$\char`\}}, define a ``block''\index{block} that
delimits the scope of any hypothesis contained between them.  The
assertion \texttt{mp} has mandatory hypotheses \texttt{wp}, \texttt{wq},
\texttt{min}, and \texttt{maj}.  The only mandatory hypothesis of
\texttt{th1}, on the other hand, is \texttt{tt}, since \texttt{th1}
occurs outside of the block containing \texttt{min} and \texttt{maj}.

Note that \texttt{\$\char`\{} and \texttt{\$\char`\}} do not affect the
scope of assertions (\texttt{\$a} and \texttt{\$p}).  Assertions are always
available to be referenced by any later proof in the source file.

Each provable assertion (\texttt{\$p}\index{\texttt{\$p} statement}
statement) has two parts.  The first part is the
assertion\index{assertion} itself, which is a sequence of math
symbol\index{math symbol} tokens placed between the \texttt{\$p} token
and a \texttt{\$=}\index{\texttt{\$=} keyword} token.  The second part
is a ``proof,'' which is a list of label tokens placed between the
\texttt{\$=} token and the \texttt{\$.}\index{\texttt{\$.}\ keyword}\
token that ends the statement.\footnote{If you've looked at the
\texttt{set.mm} database, you may have noticed another notation used for
proofs.  The other notation is called ``compressed.''\index{compressed
proof}\index{proof!compressed} It reduces the amount of space needed to
store a proof in the database and is described in
Appendix~\ref{compressed}.  In the example above, we use
``normal''\index{normal proof}\index{proof!normal} notation.} The proof
acts as a series of instructions to the Metamath program, telling it how
to build up the sequence of math symbols contained in the assertion part of
the \texttt{\$p} statement, making use of the hypotheses of the
\texttt{\$p} statement and previous assertions.  The construction takes
place according to precise rules.  If the list of labels in the proof
causes these rules to be violated, or if the final sequence that results
does not match the assertion, the Metamath program will notify you with
an error message.

If you are familiar with reverse Polish notation (RPN), which is sometimes used
on pocket calculators, here in a nutshell is how a proof works.  Each
hypothesis label\index{hypothesis label} in the proof is pushed\index{push}
onto the RPN stack\index{stack}\index{RPN stack} as it is encountered. Each
assertion label\index{assertion label} pops\index{pop} off the stack as many
entries as the referenced assertion has mandatory hypotheses.  Variable
substitutions\index{substitution!variable}\index{variable substitution} are
computed which, when made to the referenced assertion's mandatory hypotheses,
cause these hypotheses to match the stack entries. These same substitutions
are then made to the variables in the referenced assertion itself, which is
then pushed onto the stack.  At the end of the proof, there should be one
stack entry, namely the assertion being proved.  This process is explained in
detail in Section~\ref{proof}.

Metamath's proof notation is not very readable for humans, but it allows the
proof to be stored compactly in a file.  The Metamath\index{Metamath} program
has proof display features that let you see what's going on in a more
readable way, as you will see in the next section.

The rules used in verifying a proof are not based on any built-in syntax of the
symbol sequence in an assertion\index{assertion} nor on any built-in meanings
attached to specific symbol names.  They are based strictly on symbol
matching:  constants\index{constant} must match themselves, and
variables\index{variable} may be replaced with anything that allows a match to
occur.  For example, instead of \texttt{term}, \texttt{0}, and \verb$|-$ we could
have just as well used \texttt{yellow}, \texttt{zero}, and \texttt{provable}, as long
as we did so consistently throughout the database.  Also, we could have used
\texttt{is provable} (two tokens) instead of \verb$|-$ (one token) throughout the
database.  In each of these cases, the proof would be exactly the same.  The
independence of proofs and notation means that you have a lot of flexibility to
change the notation you use without having to change any proofs.

\section{A Trial Run}\label{trialrun}

Now you are ready to try out the Metamath\index{Metamath} program.

On all computer systems, Metamath has a standard ``command line
interface'' (CLI)\index{command line interface (CLI)} that allows you to
interact with it.  You supply commands to the CLI by typing them on the
keyboard and pressing your keyboard's {\em return} key after each line
you enter.  The CLI is designed to be easy to use and has built-in help
features.

The first thing you should do is to use a text editor to create a file
called \texttt{demo0.mm} and type into it the Metamath source shown on
p.~\pageref{demo0}.  Actually, this file is included with your Metamath
software package, so check that first.  If you type it in, make sure
that you save it in the form of ``plain {\sc ascii} text with line
breaks.''  Most word processors will have this feature.

Next you must run the Metamath program.  Depending on your computer
system and how Metamath is installed, this could range from clicking the
mouse on the Metamath icon to typing \texttt{run metamath} to typing
simply \texttt{metamath}.  (Metamath's {\tt help invoke} command describes
alternate ways of invoking the Metamath program.)

When you first enter Metamath\index{Metamath}, it will be at the CLI, waiting
for your input. You will see something like the following on your screen:
\begin{verbatim}
Metamath - Version 0.172 25-Jan-2019
Type HELP for help, EXIT to exit.
MM>
\end{verbatim}
The \texttt{MM>} prompt means that Metamath is waiting for a command.
Command keywords\index{command keyword} are not case sensitive;
we will use lower-case commands in our examples.
The version number and its release date will probably be different on your
system from the one we show above.

The first thing that you need to do is to read in your
database:\index{\texttt{read} command}\footnote{If a directory path is
needed on Unix,\index{Unix file names}\index{file names!Unix} you should
enclose the path/file name in quotes to prevent Metamath from thinking
that the \texttt{/} in the path name is a command qualifier, e.g.,
\texttt{read \char`\"db/set.mm\char`\"}.  Quotes are optional when there
is no ambiguity.}
\begin{verbatim}
MM> read demo0.mm
\end{verbatim}
Remember to press the {\em return} key after entering this command.  If
you omit the file name, Metamath will prompt you for one.   The syntax for
specifying a Macintosh file name path is given in a footnote on
p.~\pageref{includef}.\index{Macintosh file names}\index{file
names!Macintosh}

If there are any syntax errors in the database, Metamath will let you know
when it reads in the file.  The one thing that Metamath does not check when
reading in a database is that all proofs are correct, because this would
slow it down too much.  It is a good idea to periodically verify the proofs in
a database you are making changes to.  To do this, use the following command
(and do it for your \texttt{demo0.mm} file now).  Note that the \texttt{*} is a
``wild card'' meaning all proofs in the file.\index{\texttt{verify proof} command}
\begin{verbatim}
MM> verify proof *
\end{verbatim}
Metamath will report any proofs that are incorrect.

It is often useful to save the information that the Metamath program displays
on the screen. You can save everything that happens on the screen by opening a
log file. You may want to do this before you read in a database so that you
can examine any errors later on.  To open a log file, type
\begin{verbatim}
MM> open log abc.log
\end{verbatim}
This will open a file called \texttt{abc.log}, and everything that appears on the
screen from this point on will be stored in this file.  The name of the log file
is arbitrary. To close the log file, type
\begin{verbatim}
MM> close log
\end{verbatim}

Several commands let you examine what's inside your database.
Section~\ref{exploring} has an overview of some useful ones.  The
\texttt{show labels} command lets you see what statement
labels\index{label} exist.  A \texttt{*} matches any combination of
characters, and \texttt{t*} refers to all labels starting with the
letter \texttt{t}.\index{\texttt{show labels} command} The \texttt{/all}
is a ``command qualifier''\index{command qualifier} that tells Metamath
to include labels of hypotheses.  (To see the syntax explained, type
\texttt{help show labels}.)  Type
\begin{verbatim}
MM> show labels t* /all
\end{verbatim}
Metamath will respond with
\begin{verbatim}
The statement number, label, and type are shown.
3 tt $f       4 tr $f       5 ts $f       8 tze $a
9 tpl $a      19 th1 $p
\end{verbatim}

You can use the \texttt{show statement} command to get information about a
particular statement.\index{\texttt{show statement} command}
For example, you can get information about the statement with label \texttt{mp}
by typing
\begin{verbatim}
MM> show statement mp /full
\end{verbatim}
Metamath will respond with
\begin{verbatim}
Statement 17 is located on line 43 of the file
"demo0.mm".
"Define the modus ponens inference rule"
17 mp $a |- Q $.
Its mandatory hypotheses in RPN order are:
  wp $f wff P $.
  wq $f wff Q $.
  min $e |- P $.
  maj $e |- ( P -> Q ) $.
The statement and its hypotheses require the
      variables:  Q P
The variables it contains are:  Q P
\end{verbatim}
The mandatory hypotheses\index{mandatory hypothesis} and their
order\index{RPN order} are
useful to know when you are trying to understand or debug a proof.

Now you are ready to look at what's really inside our proof.  First, here is
how to look at every step in the proof---not just the ones corresponding to an
ordinary formal proof\index{formal proof}, but also the ones that build up the
formulas that appear in each ordinary formal proof step.\index{\texttt{show
proof} command}
\begin{verbatim}
MM> show proof th1 /lemmon /all
\end{verbatim}

This will display the proof on the screen in the following format:
\begin{verbatim}
 1 tt            $f term t
 2 tze           $a term 0
 3 1,2 tpl       $a term ( t + 0 )
 4 tt            $f term t
 5 3,4 weq       $a wff ( t + 0 ) = t
 6 tt            $f term t
 7 tt            $f term t
 8 6,7 weq       $a wff t = t
 9 tt            $f term t
10 9 a2          $a |- ( t + 0 ) = t
11 tt            $f term t
12 tze           $a term 0
13 11,12 tpl     $a term ( t + 0 )
14 tt            $f term t
15 13,14 weq     $a wff ( t + 0 ) = t
16 tt            $f term t
17 tze           $a term 0
18 16,17 tpl     $a term ( t + 0 )
19 tt            $f term t
20 18,19 weq     $a wff ( t + 0 ) = t
21 tt            $f term t
22 tt            $f term t
23 21,22 weq     $a wff t = t
24 20,23 wim     $a wff ( ( t + 0 ) = t -> t = t )
25 tt            $f term t
26 25 a2         $a |- ( t + 0 ) = t
27 tt            $f term t
28 tze           $a term 0
29 27,28 tpl     $a term ( t + 0 )
30 tt            $f term t
31 tt            $f term t
32 29,30,31 a1   $a |- ( ( t + 0 ) = t -> ( ( t + 0 )
                                     = t -> t = t ) )
33 15,24,26,32 mp  $a |- ( ( t + 0 ) = t -> t = t )
34 5,8,10,33 mp  $a |- t = t
\end{verbatim}

The \texttt{/lemmon} command qualifier specifies what is known as a Lemmon-style
display\index{Lemmon-style proof}\index{proof!Lemmon-style}.  Omitting the
\texttt{/lemmon} qualifier results in a tree-style proof (see
p.~\pageref{treeproof} for an example) that is somewhat less explicit but
easier to follow once you get used to it.\index{tree-style
proof}\index{proof!tree-style}

The first number on each line is the step
number of the proof.  Any numbers that follow are step numbers assigned to the
hypotheses of the statement referenced by that step.  Next is the label of
the statement referenced by the step.  The statement type of the statement
referenced comes next, followed by the math symbol\index{math symbol} string
constructed by the proof up to that step.

The last step, 34, contains the statement that is being proved.

Looking at a small piece of the proof, notice that steps 3 and 4 have
established that
\texttt{( t + 0 )} and \texttt{t} are \texttt{term}\,s, and step 5 makes use of steps 3 and
4 to establish that \texttt{( t + 0 ) = t} is a \texttt{wff}.  Let Metamath
itself tell us in detail what is happening in step 5.  Note that the
``target hypothesis'' refers to where step 5 is eventually used, i.e., in step
34.
\begin{verbatim}
MM> show proof th1 /detailed_step 5
Proof step 5:  wp=weq $a wff ( t + 0 ) = t
This step assigns source "weq" ($a) to target "wp"
($f).  The source assertion requires the hypotheses
"tt" ($f, step 3) and "tr" ($f, step 4).  The parent
assertion of the target hypothesis is "mp" ($a,
step 34).
The source assertion before substitution was:
    weq $a wff t = r
The following substitutions were made to the source
assertion:
    Variable  Substituted with
     t         ( t + 0 )
     r         t
The target hypothesis before substitution was:
    wp $f wff P
The following substitution was made to the target
hypothesis:
    Variable  Substituted with
     P         ( t + 0 ) = t
\end{verbatim}

The full proof just shown is useful to understand what is going on in detail.
However, most of the time you will just be interested in
the ``essential'' or logical steps of a proof, i.e.\ those steps
that correspond to an
ordinary formal proof\index{formal proof}.  If you type
\begin{verbatim}
MM> show proof th1 /lemmon /renumber
\end{verbatim}
you will see\label{demoproof}
\begin{verbatim}
1 a2             $a |- ( t + 0 ) = t
2 a2             $a |- ( t + 0 ) = t
3 a1             $a |- ( ( t + 0 ) = t -> ( ( t + 0 )
                                     = t -> t = t ) )
4 2,3 mp         $a |- ( ( t + 0 ) = t -> t = t )
5 1,4 mp         $a |- t = t
\end{verbatim}
Compare this to the formal proof on p.~\pageref{zeroproof} and
notice the resemblance.
By default Metamath
does not show \texttt{\$f}\index{\texttt{\$f}
statement} hypotheses and everything branching off of them in the proof tree
when the proof is displayed; this makes the proof look more like an ordinary
mathematical proof, which does not normally incorporate the explicit
construction of expressions.
This is called the ``essential'' view
(at one time you had to add the
\texttt{/essential} qualifier in the \texttt{show proof}
command to get this view, but this is now the default).
You can could use the \texttt{/all} qualifier in the \texttt{show
proof} command to also show the explicit construction of expressions.
The \texttt{/renumber} qualifier means to renumber
the steps to correspond only to what is displayed.\index{\texttt{show proof}
command}

To exit Metamath, type\index{\texttt{exit} command}
\begin{verbatim}
MM> exit
\end{verbatim}

\subsection{Some Hints for Using the Command Line Interface}

We will conclude this quick introduction to Metamath\index{Metamath} with some
helpful hints on how to navigate your way through the commands.
\index{command line interface (CLI)}

When you type commands into Metamath's CLI, you only have to type as many
characters of a command keyword\index{command keyword} as are needed to make
it unambiguous.  If you type too few characters, Metamath will tell you what
the choices are.  In the case of the \texttt{read} command, only the \texttt{r} is
needed to specify it unambiguously, so you could have typed\index{\texttt{read}
command}
\begin{verbatim}
MM> r demo0.mm
\end{verbatim}
instead of
\begin{verbatim}
MM> read demo0.mm
\end{verbatim}
In our description, we always show the full command words.  When using the
Metamath CLI commands in a command file (to be read with the \texttt{submit}
command)\index{\texttt{submit} command}, it is good practice to use
the unabbreviated command to ensure your instructions will not become ambiguous
if more commands are added to the Metamath program in the future.

The command keywords\index{command
keyword} are not case sensitive; you may type either \texttt{read} or
\texttt{ReAd}.  File names may or may not be case sensitive, depending on your
computer's operating system.  Metamath label\index{label} and math
symbol\index{math symbol} tokens\index{token} are case-sensitive.

The \texttt{help} command\index{\texttt{help} command} will provide you
with a list of topics you can get help on.  You can then type
\texttt{help} {\em topic} to get help on that topic.

If you are uncertain of a command's spelling, just type as many characters
as you remember of the command.  If you have not typed enough characters to
specify it unambiguously, Metamath will tell you what choices you have.

\begin{verbatim}
MM> show s
         ^
?Ambiguous keyword - please specify SETTINGS,
STATEMENT, or SOURCE.
\end{verbatim}

If you don't know what argument to use as part of a command, type a
\texttt{?}\index{\texttt{]}@\texttt{?}\ in command lines}\ at the
argument position.  Metamath will tell you what it expected there.

\begin{verbatim}
MM> show ?
         ^
?Expected SETTINGS, LABELS, STATEMENT, SOURCE, PROOF,
MEMORY, TRACE_BACK, or USAGE.
\end{verbatim}

Finally, you may type just the first word or words of a command followed
by {\em return}.  Metamath will prompt you for the remaining part of the
command, showing you the choices at each step.  For example, instead of
typing \texttt{show statement th1 /full} you could interact in the
following manner:
\begin{verbatim}
MM> show
SETTINGS, LABELS, STATEMENT, SOURCE, PROOF,
MEMORY, TRACE_BACK, or USAGE <SETTINGS>? st
What is the statement label <th1>?
/ or nothing <nothing>? /
TEX, COMMENT_ONLY, or FULL <TEX>? f
/ or nothing <nothing>?
19 th1 $p |- t = t $= ... $.
\end{verbatim}
After each \texttt{?}\ in this mode, you must give Metamath the
information it requests.  Sometimes Metamath gives you a list of choices
with the default choice indicated by brackets \texttt{< > }. Pressing
{\em return} after the \texttt{?}\ will select the default choice.
Answering anything else will override the default.  Note that the
\texttt{/} in command qualifiers is considered a separate
token\index{token} by the parser, and this is why it is asked for
separately.

\section{Your First Proof}\label{frstprf}

Proofs are developed with the aid of the Proof Assistant\index{Proof
Assistant}.  We will now show you how the proof of theorem \texttt{th1}
was built.  So that you can repeat these steps, we will first have the
Proof Assistant erase the proof in Metamath's source buffer\index{source
buffer}, then reconstruct it.  (The source buffer is the place in memory
where Metamath stores the information in the database when it is
\texttt{read}\index{\texttt{read} command} in.  New or modified proofs
are kept in the source buffer until a \texttt{write source}
command\index{\texttt{write source} command} is issued.)  In practice, you
would place a \texttt{?}\index{\texttt{]}@\texttt{?}\ inside proofs}\
between \texttt{\$=}\index{\texttt{\$=} keyword} and
\texttt{\$.}\index{\texttt{\$.}\ keyword}\ in the database to indicate
to Metamath\index{Metamath} that the proof is unknown, and that would be
your starting point.  Whenever the \texttt{verify proof} command encounters
a proof with a \texttt{?}\ in place of a proof step, the statement is
identified as not proved.

When I first started creating Metamath proofs, I would write down
on a piece of paper the complete
formal proof\index{formal proof} as it would appear
in a \texttt{show proof} command\index{\texttt{show proof} command}; see
the display of \texttt{show proof th1 /lemmon /re\-num\-ber} above as an
example.  After you get used to using the Proof Assistant\index{Proof
Assistant} you may get to a point where you can ``see'' the proof in your mind
and let the Proof Assistant guide you in filling in the details, at least for
simpler proofs, but until you gain that experience you may find it very useful
to write down all the details in advance.
Otherwise you may waste a lot of time as you let it take you down a wrong path.
However, others do not find this approach as helpful.
For example, Thomas Brendan Leahy\index{Leahy, Thomas Brendan}
finds that it is more helpful to him to interactively
work backward from a machine-readable statement.
David A. Wheeler\index{Wheeler, David A.}
writes down a general approach, but develops the proof
interactively by switching between
working forwards (from hypotheses and facts likely to be useful) and
backwards (from the goal) until the forwards and backwards approaches meet.
In the end, use whatever approach works for you.

A proof is developed with the Proof Assistant by working backwards, starting
with the theorem\index{theorem} to be proved, and assigning each unknown step
with a theorem or hypothesis until no more unknown steps remain.  The Proof
Assistant will not let you make an assignment unless it can be ``unified''
with the unknown step.  This means that a
substitution\index{substitution!variable}\index{variable substitution} of
variables exists that will make the assignment match the unknown step.  On the
other hand, in the middle of a proof, when working backwards, often more than
one unification\index{unification} (set of substitutions) is possible, since
there is not enough information available at that point to uniquely establish
it.  In this case you can tell Metamath which unification to choose, or you
can continue to assign unknown steps until enough information is available to
make the unification unique.

We will assume you have entered Metamath and read in the database as described
above.  The following dialog shows how the proof was developed.  For more
details on what some of the commands do, refer to Section~\ref{pfcommands}.
\index{\texttt{prove} command}

\begin{verbatim}
MM> prove th1
Entering the Proof Assistant.  Type HELP for help, EXIT
to exit.  You will be working on the proof of statement th1:
  $p |- t = t
Note:  The proof you are starting with is already complete.
MM-PA>
\end{verbatim}

The \verb/MM-PA>/ prompt means we are inside the Proof
Assistant.\index{Proof Assistant} Most of the regular Metamath commands
(\texttt{show statement}, etc.) are still available if you need them.

\begin{verbatim}
MM-PA> delete all
The entire proof was deleted.
\end{verbatim}

We have deleted the whole proof so we can start from scratch.

\begin{verbatim}
MM-PA> show new_proof/lemmon/all
1 ?              $? |- t = t
\end{verbatim}

The \texttt{show new{\char`\_}proof} command\index{\texttt{show
new{\char`\_}proof} command} is like \texttt{show proof} except that we
don't specify a statement; instead, the proof we're working on is
displayed.

\begin{verbatim}
MM-PA> assign 1 mp
To undo the assignment, DELETE STEP 5 and INITIALIZE, UNIFY
if needed.
3   min=?  $? |- $2
4   maj=?  $? |- ( $2 -> t = t )
\end{verbatim}

The \texttt{assign} command\index{\texttt{assign} command} above means
``assign step 1 with the statement whose label is \texttt{mp}.''  Note
that step renumbering will constantly occur as you assign steps in the
middle of a proof; in general all steps from the step you assign until
the end of the proof will get moved up.  In this case, what used to be
step 1 is now step 5, because the (partial) proof now has five steps:
the four hypotheses of the \texttt{mp} statement and the \texttt{mp}
statement itself.  Let's look at all the steps in our partial proof:

\begin{verbatim}
MM-PA> show new_proof/lemmon/all
1 ?              $? wff $2
2 ?              $? wff t = t
3 ?              $? |- $2
4 ?              $? |- ( $2 -> t = t )
5 1,2,3,4 mp     $a |- t = t
\end{verbatim}

The symbol \texttt{\$2} is a temporary variable\index{temporary
variable} that represents a symbol sequence not yet known.  In the final
proof, all temporary variables will be eliminated.  The general format
for a temporary variable is \texttt{\$} followed by an integer.  Note
that \texttt{\$} is not a legal character in a math symbol (see
Section~\ref{dollardollar}, p.~\pageref{dollardollar}), so there will
never be a naming conflict between real symbols and temporary variables.

Unknown steps 1 and 2 are constructions of the two wffs used by the
modus ponens rule.  As you will see at the end of this section, the
Proof Assistant\index{Proof Assistant} can usually figure these steps
out by itself, and we will not have to worry about them.  Therefore from
here on we will display only the ``essential'' hypotheses, i.e.\ those
steps that correspond to traditional formal proofs\index{formal proof}.

\begin{verbatim}
MM-PA> show new_proof/lemmon
3 ?              $? |- $2
4 ?              $? |- ( $2 -> t = t )
5 3,4 mp         $a |- t = t
\end{verbatim}

Unknown steps 3 and 4 are the ones we must focus on.  They correspond to the
minor and major premises of the modus ponens rule.  We will assign them as
follows.  Notice that because of the step renumbering that takes place
after an assignment, it is advantageous to assign unknown steps in reverse
order, because earlier steps will not get renumbered.

\begin{verbatim}
MM-PA> assign 4 mp
To undo the assignment, DELETE STEP 8 and INITIALIZE, UNIFY
if needed.
3   min=?  $? |- $2
6     min=?  $? |- $4
7     maj=?  $? |- ( $4 -> ( $2 -> t = t ) )
\end{verbatim}

We are now going to describe an obscure feature that you will probably
never use but should be aware of.  The Metamath language allows empty
symbol sequences to be substituted for variables, but in most formal
systems this feature is never used.  One of the few examples where is it
used is the MIU-system\index{MIU-system} described in
Appendix~\ref{MIU}.  But such systems are rare, and by default this
feature is turned off in the Proof Assistant.  (It is always allowed for
{\tt verify proof}.)  Let us turn it on and see what
happens.\index{\texttt{set empty{\char`\_}substitution} command}

\begin{verbatim}
MM-PA> set empty_substitution on
Substitutions with empty symbol sequences is now allowed.
\end{verbatim}

With this feature enabled, more unifications will be
ambiguous\index{ambiguous unification}\index{unification!ambiguous} in
the middle of a proof, because
substitution\index{substitution!variable}\index{variable substitution}
of variables with empty symbol sequences will become an additional
possibility.  Let's see what happens when we make our next assignment.

\begin{verbatim}
MM-PA> assign 3 a2
There are 2 possible unifications.  Please select the correct
    one or Q if you want to UNIFY later.
Unify:  |- $6
 with:  |- ( $9 + 0 ) = $9
Unification #1 of 2 (weight = 7):
  Replace "$6" with "( + 0 ) ="
  Replace "$9" with ""
  Accept (A), reject (R), or quit (Q) <A>? r
\end{verbatim}

The first choice presented is the wrong one.  If we had selected it,
temporary variable \texttt{\$6} would have been assigned a truncated
wff, and temporary variable \texttt{\$9} would have been assigned an
empty sequence (which is not allowed in our system).  With this choice,
eventually we would reach a point where we would get stuck because
we would end up with steps impossible to prove.  (You may want to
try it.)  We typed \texttt{r} to reject the choice.

\begin{verbatim}
Unification #2 of 2 (weight = 21):
  Replace "$6" with "( $9 + 0 ) = $9"
  Accept (A), reject (R), or quit (Q) <A>? q
To undo the assignment, DELETE STEP 4 and INITIALIZE, UNIFY
if needed.
 7     min=?  $? |- $8
 8     maj=?  $? |- ( $8 -> ( $6 -> t = t ) )
\end{verbatim}

The second choice is correct, and normally we would type \texttt{a}
to accept it.  But instead we typed \texttt{q} to show what will happen:
it will leave the step with an unknown unification, which can be
seen as follows:

\begin{verbatim}
MM-PA> show new_proof/not_unified
 4   min    $a |- $6
        =a2  = |- ( $9 + 0 ) = $9
\end{verbatim}

Later we can unify this with the \texttt{unify}
\texttt{all/interactive} command.

The important point to remember is that occasionally you will be
presented with several unification choices while entering a proof, when
the program determines that there is not enough information yet to make
an unambiguous choice automatically (and this can happen even with
\texttt{set empty{\char`\_}substitution} turned off).  Usually it is
obvious by inspection which choice is correct, since incorrect ones will
tend to be meaningless fragments of wffs.  In addition, the correct
choice will usually be the first one presented, unlike our example
above.

Enough of this digression.  Let us go back to the default setting.

\begin{verbatim}
MM-PA> set empty_substitution off
The ability to substitute empty expressions for variables
has been turned off.  Note that this may make the Proof
Assistant too restrictive in some cases.
\end{verbatim}

If we delete the proof, start over, and get to the point where
we digressed above, there will no longer be an ambiguous unification.

\begin{verbatim}
MM-PA> assign 3 a2
To undo the assignment, DELETE STEP 4 and INITIALIZE, UNIFY
if needed.
 7     min=?  $? |- $4
 8     maj=?  $? |- ( $4 -> ( ( $5 + 0 ) = $5 -> t = t ) )
\end{verbatim}

Let us look at our proof so far, and continue.

\begin{verbatim}
MM-PA> show new_proof/lemmon
 4 a2            $a |- ( $5 + 0 ) = $5
 7 ?             $? |- $4
 8 ?             $? |- ( $4 -> ( ( $5 + 0 ) = $5 -> t = t ) )
 9 7,8 mp        $a |- ( ( $5 + 0 ) = $5 -> t = t )
10 4,9 mp        $a |- t = t
MM-PA> assign 8 a1
To undo the assignment, DELETE STEP 11 and INITIALIZE, UNIFY
if needed.
 7     min=?  $? |- ( t + 0 ) = t
MM-PA> assign 7 a2
To undo the assignment, DELETE STEP 8 and INITIALIZE, UNIFY
if needed.
MM-PA> show new_proof/lemmon
 4 a2            $a |- ( t + 0 ) = t
 8 a2            $a |- ( t + 0 ) = t
12 a1            $a |- ( ( t + 0 ) = t -> ( ( t + 0 ) = t ->
                                                    t = t ) )
13 8,12 mp       $a |- ( ( t + 0 ) = t -> t = t )
14 4,13 mp       $a |- t = t
\end{verbatim}

Now all temporary variables and unknown steps have been eliminated from the
``essential'' part of the proof.  When this is achieved, the Proof
Assistant\index{Proof Assistant} can usually figure out the rest of the proof
automatically.  (Note that the \texttt{improve} command can occasionally be
useful for filling in essential steps as well, but it only tries to make use
of statements that introduce no new variables in their hypotheses, which is
not the case for \texttt{mp}. Also it will not try to improve steps containing
temporary variables.)  Let's look at the complete proof, then run
the \texttt{improve} command, then look at it again.

\begin{verbatim}
MM-PA> show new_proof/lemmon/all
 1 ?             $? wff ( t + 0 ) = t
 2 ?             $? wff t = t
 3 ?             $? term t
 4 3 a2          $a |- ( t + 0 ) = t
 5 ?             $? wff ( t + 0 ) = t
 6 ?             $? wff ( ( t + 0 ) = t -> t = t )
 7 ?             $? term t
 8 7 a2          $a |- ( t + 0 ) = t
 9 ?             $? term ( t + 0 )
10 ?             $? term t
11 ?             $? term t
12 9,10,11 a1    $a |- ( ( t + 0 ) = t -> ( ( t + 0 ) = t ->
                                                    t = t ) )
13 5,6,8,12 mp   $a |- ( ( t + 0 ) = t -> t = t )
14 1,2,4,13 mp   $a |- t = t
\end{verbatim}

\begin{verbatim}
MM-PA> improve all
A proof of length 1 was found for step 11.
A proof of length 1 was found for step 10.
A proof of length 3 was found for step 9.
A proof of length 1 was found for step 7.
A proof of length 9 was found for step 6.
A proof of length 5 was found for step 5.
A proof of length 1 was found for step 3.
A proof of length 3 was found for step 2.
A proof of length 5 was found for step 1.
Steps 1 and above have been renumbered.
CONGRATULATIONS!  The proof is complete.  Use SAVE
NEW_PROOF to save it.  Note:  The Proof Assistant does
not detect $d violations.  After saving the proof, you
should verify it with VERIFY PROOF.
\end{verbatim}

The \texttt{save new{\char`\_}proof} command\index{\texttt{save
new{\char`\_}proof} command} will save the proof in the database.  Here
we will just display it in a form that can be clipped out of a log file
and inserted manually into the database source file with a text
editor.\index{normal proof}\index{proof!normal}

\begin{verbatim}
MM-PA> show new_proof/normal
---------Clip out the proof below this line:
      tt tze tpl tt weq tt tt weq tt a2 tt tze tpl tt weq
      tt tze tpl tt weq tt tt weq wim tt a2 tt tze tpl tt
      tt a1 mp mp $.
---------The proof of 'th1' to clip out ends above this line.
\end{verbatim}

There is another proof format called ``compressed''\index{compressed
proof}\index{proof!compressed} that you will see in databases.  It is
not important to understand how it is encoded but only to recognize it
when you see it.  Its only purpose is to reduce storage requirements for
large proofs.  A compressed proof can always be converted to a normal
one and vice-versa, and the Metamath \texttt{show proof}
commands\index{\texttt{show proof} command} work equally well with
compressed proofs.  The compressed proof format is described in
Appendix~\ref{compressed}.

\begin{verbatim}
MM-PA> show new_proof/compressed
---------Clip out the proof below this line:
      ( tze tpl weq a2 wim a1 mp ) ABCZADZAADZAEZJJKFLIA
      AGHH $.
---------The proof of 'th1' to clip out ends above this line.
\end{verbatim}

Now we will exit the Proof Assistant.  Since we made changes to the proof,
it will warn us that we have not saved it.  In this case, we don't care.

\begin{verbatim}
MM-PA> exit
Warning:  You have not saved changes to the proof.
Do you want to EXIT anyway (Y, N) <N>? y
Exiting the Proof Assistant.
Type EXIT again to exit Metamath.
\end{verbatim}

The Proof Assistant\index{Proof Assistant} has several other commands
that can help you while creating proofs.  See Section~\ref{pfcommands}
for a list of them.

A command that is often useful is \texttt{minimize{\char`\_}with
*/brief}, which tries to shorten the proof.  It can make the process
more efficient by letting you write a somewhat ``sloppy'' proof then
clean up some of the fine details of optimization for you (although it
can't perform miracles such as restructuring the overall proof).

\section{A Note About Editing a Data\-base File}

Once your source file contains proofs, there are some restrictions on
how you can edit it so that the proofs remain valid.  Pay particular
attention to these rules, since otherwise you can lose a lot of work.
It is a good idea to periodically verify all proofs with \texttt{verify
proof *} to ensure their integrity.

If your file contains only normal (as opposed to compressed) proofs, the
main rule is that you may not change the order of the mandatory
hypotheses\index{mandatory hypothesis} of any statement referenced in a
later proof.  For example, if you swap the order of the major and minor
premise in the modus ponens rule, all proofs making use of that rule
will become incorrect.  The \texttt{show statement}
command\index{\texttt{show statement} command} will show you the
mandatory hypotheses of a statement and their order.

If a statement has a compressed proof, you also must not change the
order of {\em its} mandatory hypotheses.  The compressed proof format
makes use of this information as part of the compression technique.
Note that swapping the names of two variables in a theorem will change
the order of its mandatory hypotheses.

The safest way to edit a statement, say \texttt{mytheorem}, is to
duplicate it then rename the original to \texttt{mytheoremOLD}
throughout the database.  Once the edited version is re-proved, all
statements referencing \texttt{mytheoremOLD} can be updated in the Proof
Assistant using \texttt{minimize{\char`\_}with
mytheorem
/allow{\char`\_}growth}.\index{\texttt{minimize{\char`\_}with} command}
% 3/10/07 Note: line-breaking the above results in duplicate index entries

\chapter{Abstract Mathematics Revealed}\label{fol}

\section{Logic and Set Theory}\label{logicandsettheory}

\begin{quote}
  {\em Set theory can be viewed as a form of exact theology.}
  \flushright\sc  Rudy Rucker\footnote{\cite{Barrow}, p.~31.}\\
\end{quote}\index{Rucker, Rudy}

Despite its seeming complexity, all of standard mathematics, no matter how
deep or abstract, can amazingly enough be derived from a relatively small set
of axioms\index{axiom} or first principles. The development of these axioms is
among the most impressive and important accomplishments of mathematics in the
20th century. Ultimately, these axioms can be broken down into a set of rules
for manipulating symbols that any technically oriented person can follow.

We will not spend much time trying to convey a deep, higher-level
understanding of the meaning of the axioms. This kind of understanding
requires some mathematical sophistication as well as an understanding of the
philosophy underlying the foundations of mathematics and typically develops
over time as you work with mathematics.  Our goal, instead, is to give you the
immediate ability to follow how theorems\index{theorem} are derived from the
axioms and from other theorems.  This will be similar to learning the syntax
of a computer language, which lets you follow the details in a program but
does not necessarily give you the ability to write non-trivial programs on
your own, an ability that comes with practice. For now don't be alarmed by
abstract-sounding names of the axioms; just focus on the rules for
manipulating the symbols, which follow the simple conventions of the
Metamath\index{Metamath} language.

The axioms that underlie all of standard mathematics consist of axioms of logic
and axioms of set theory. The axioms of logic are divided into two
subcategories, propositional calculus\index{propositional calculus} (sometimes
called sentential logic\index{sentential logic}) and predicate calculus
(sometimes called first-order logic\index{first-order logic}\index{quantifier
theory}\index{predicate calculus} or quantifier theory).  Propositional
calculus is a prerequisite for predicate calculus, and predicate calculus is a
prerequisite for set theory.  The version of set theory most commonly used is
Zermelo--Fraenkel set theory\index{Zermelo--Fraenkel set theory}\index{set theory}
with the axiom of choice,
often abbreviated as ZFC\index{ZFC}.

Here in a nutshell is what the axioms are all about in an informal way. The
connection between this description and symbols we will show you won't be
immediately apparent and in principle needn't ever be.  Our description just
tries to summarize what mathematicians think about when they work with the
axioms.

Logic is a set of rules that allow us determine truths given other truths.
Put another way,
logic is more or less the translation of what we would consider common sense
into a rigorous set of axioms.\index{axioms of logic}  Suppose $\varphi$,
$\psi$, and $\chi$ (the Greek letters phi, psi, and chi) represent statements
that are either true or false, and $x$ is a variable\index{variable!in predicate
calculus} ranging over some group of mathematical objects (sets, integers,
real numbers, etc.). In mathematics, a ``statement'' really means a formula,
and $\psi$ could be for example ``$x = 2$.''
Propositional calculus\index{propositional calculus}
allows us to use variables that are either true or false
and make deductions such as
``if $\varphi$ implies $\psi$ and $\psi$ implies $\chi$, then $\varphi$
implies $\chi$.''
Predicate calculus\index{predicate calculus}
extends propositional calculus by also allowing us
to discuss statements about objects (not just true and false values), including
statements about ``all'' or ``at least one'' object.
For example, predicate calculus allows to say,
``if $\varphi$ is true for all $x$, then $\varphi$ is true for some $x$.''
The logic used in \texttt{set.mm} is standard classical logic
(as opposed to other logic systems like intuitionistic logic).

Set theory\index{set theory} has to do with the manipulation of objects and
collections of objects, specifically the abstract, imaginary objects that
mathematics deals with, such as numbers. Everything that is claimed to exist
in mathematics is considered to be a set.  A set called the empty
set\index{empty set} contains nothing.  We represent the empty set by
$\varnothing$.  Many sets can be built up from the empty set.  There is a set
represented by $\{\varnothing\}$ that contains the empty set, another set
represented by $\{\varnothing,\{\varnothing\}\}$ that contains this set as
well as the empty set, another set represented by $\{\{\varnothing\}\}$ that
contains just the set that contains the empty set, and so on ad infinitum. All
mathematical objects, no matter how complex, are defined as being identical to
certain sets: the integer\index{integer} 0 is defined as the empty set, the
integer 1 is defined as $\{\varnothing\}$, the integer 2 is defined as
$\{\varnothing,\{\varnothing\}\}$.  (How these definitions were chosen doesn't
matter now, but the idea behind it is that these sets have the properties we
expect of integers once suitable operations are defined.)  Mathematical
operations, such as addition, are defined in terms of operations on
sets---their union\index{set union}, intersection\index{set intersection}, and
so on---operations you may have used in elementary school when you worked
with groups of apples and oranges.

With a leap of faith, the axioms also postulate the existence of infinite
sets\index{infinite set}, such as the set of all non-negative integers ($0, 1,
2,\ldots$, also called ``natural numbers''\index{natural number}).  This set
can't be represented with the brace notation\index{brace notation} we just
showed you, but requires a more complicated notation called ``class
abstraction.''\index{class abstraction}\index{abstraction class}  For
example, the infinite set $\{ x |
\mbox{``$x$ is a natural number''} \} $ means the ``set of all objects $x$
such that $x$ is a natural number'' i.e.\ the set of natural numbers; here,
``$x$ is a natural number'' is a rather complicated formula when broken down
into the primitive symbols.\label{expandom}\footnote{The statement ``$x$ is a
natural number'' is formally expressed as ``$x \in \omega$,'' where $\in$
(stylized epsilon) means ``is in'' or ``is an element of'' and $\omega$
(omega) means ``the set of natural numbers.''  When ``$x\in\omega$'' is
completely expanded in terms of the primitive symbols of set theory, the
result is  $\lnot$ $($ $\lnot$ $($ $\forall$ $z$ $($ $\lnot$ $\forall$ $w$ $($
$z$ $\in$ $w$ $\rightarrow$ $\lnot$ $w$ $\in$ $x$ $)$ $\rightarrow$ $z$ $\in$
$x$ $)$ $\rightarrow$ $($ $\forall$ $z$ $($ $\lnot$ $($ $\forall$ $w$ $($ $w$
$\in$ $z$ $\rightarrow$ $w$ $\in$ $x$ $)$ $\rightarrow$ $\forall$ $w$ $\lnot$
$w$ $\in$ $z$ $)$ $\rightarrow$ $\lnot$ $\forall$ $w$ $($ $w$ $\in$ $z$
$\rightarrow$ $\lnot$ $\forall$ $v$ $($ $v$ $\in$ $z$ $\rightarrow$ $\lnot$
$v$ $\in$ $w$ $)$ $)$ $)$ $\rightarrow$ $\lnot$ $\forall$ $z$ $\forall$ $w$
$($ $\lnot$ $($ $z$ $\in$ $x$ $\rightarrow$ $\lnot$ $w$ $\in$ $x$ $)$
$\rightarrow$ $($ $\lnot$ $z$ $\in$ $w$ $\rightarrow$ $($ $\lnot$ $z$ $=$ $w$
$\rightarrow$ $w$ $\in$ $z$ $)$ $)$ $)$ $)$ $)$ $\rightarrow$ $\lnot$
$\forall$ $y$ $($ $\lnot$ $($ $\lnot$ $($ $\forall$ $z$ $($ $\lnot$ $\forall$
$w$ $($ $z$ $\in$ $w$ $\rightarrow$ $\lnot$ $w$ $\in$ $y$ $)$ $\rightarrow$
$z$ $\in$ $y$ $)$ $\rightarrow$ $($ $\forall$ $z$ $($ $\lnot$ $($ $\forall$
$w$ $($ $w$ $\in$ $z$ $\rightarrow$ $w$ $\in$ $y$ $)$ $\rightarrow$ $\forall$
$w$ $\lnot$ $w$ $\in$ $z$ $)$ $\rightarrow$ $\lnot$ $\forall$ $w$ $($ $w$
$\in$ $z$ $\rightarrow$ $\lnot$ $\forall$ $v$ $($ $v$ $\in$ $z$ $\rightarrow$
$\lnot$ $v$ $\in$ $w$ $)$ $)$ $)$ $\rightarrow$ $\lnot$ $\forall$ $z$
$\forall$ $w$ $($ $\lnot$ $($ $z$ $\in$ $y$ $\rightarrow$ $\lnot$ $w$ $\in$
$y$ $)$ $\rightarrow$ $($ $\lnot$ $z$ $\in$ $w$ $\rightarrow$ $($ $\lnot$ $z$
$=$ $w$ $\rightarrow$ $w$ $\in$ $z$ $)$ $)$ $)$ $)$ $\rightarrow$ $($
$\forall$ $z$ $\lnot$ $z$ $\in$ $y$ $\rightarrow$ $\lnot$ $\forall$ $w$ $($
$\lnot$ $($ $w$ $\in$ $y$ $\rightarrow$ $\lnot$ $\forall$ $z$ $($ $w$ $\in$
$z$ $\rightarrow$ $\lnot$ $z$ $\in$ $y$ $)$ $)$ $\rightarrow$ $\lnot$ $($
$\lnot$ $\forall$ $z$ $($ $w$ $\in$ $z$ $\rightarrow$ $\lnot$ $z$ $\in$ $y$
$)$ $\rightarrow$ $w$ $\in$ $y$ $)$ $)$ $)$ $)$ $\rightarrow$ $x$ $\in$ $y$
$)$ $)$ $)$. Section~\ref{hierarchy} shows the hierarchy of definitions that
leads up to this expression.}\index{stylized epsilon ($\in$)}\index{omega
($\omega$)}  Actually, the primitive symbols don't even include the brace
notation.  The brace notation is a high-level definition, which you can find in
Section~\ref{hierarchy}.

Interestingly, the arithmetic of integers\index{integer} and
rationals\index{rational number} can be developed without appealing to the
existence of an infinite set, whereas the arithmetic of real
numbers\index{real number} requires it.

Each variable\index{variable!in set theory} in the axioms of set theory
represents an arbitrary set, and the axioms specify the legal kinds of things
you can do with these variables at a very primitive level.

Now, you may think that numbers and arithmetic are a lot more intuitive and
fundamental than sets and therefore should be the foundation of mathematics.
What is really the case is that you've dealt with numbers all your life and
are comfortable with a few rules for manipulating them such as addition and
multiplication.  Those rules only cover a small portion of what can be done
with numbers and only a very tiny fraction of the rest of mathematics.  If you
look at any elementary book on number theory, you will quickly become lost if
these are the only rules that you know.  Even though such books may present a
list of ``axioms''\index{axiom} for arithmetic, the ability to use the axioms
and to understand proofs of theorems\index{theorem} (facts) about numbers
requires an implicit mathematical talent that frustrates many people
from studying abstract mathematics.  The kind of mathematics that most people
know limits them to the practical, everyday usage of blindly manipulating
numbers and formulas, without any understanding of why those rules are correct
nor any ability to go any further.  For example, do you know why multiplying
two negative numbers yields a positive number?  Starting with set theory, you
will also start off blindly manipulating symbols according to the rules we give
you, but with the advantage that these rules will allow you, in principle, to
access {\em all} of mathematics, not just a tiny part of it.

Of course, concrete examples are often helpful in the learning process. For
example, you can verify that $2\cdot 3=3 \cdot 2$ by actually grouping
objects and can easily ``see'' how it generalizes to $x\cdot y = y\cdot x$,
even though you might not be able to rigorously prove it.  Similarly, in set
theory it can be helpful to understand how the axioms of set theory apply to
(and are correct for) small finite collections of objects.  You should be aware
that in set theory intuition can be misleading for infinite collections, and
rigorous proofs become more important.  For example, while $x\cdot y = y\cdot
x$ is correct for finite ordinals (which are the natural numbers), it is not
usually true for infinite ordinals.

\section{The Axioms for All of Mathematics}

In this section\index{axioms for mathematics}, we will show you the axioms
for all of standard mathematics (i.e.\ logic and set theory) as they are
traditionally presented.  The traditional presentation is useful for someone
with the mathematical experience needed to correctly manipulate high-level
abstract concepts.  For someone without this talent, knowing how to actually
make use of these axioms can be difficult.  The purpose of this section is to
allow you to see how the version of the axioms used in the standard
Metamath\index{Metamath} database \texttt{set.mm}\index{set
theory database (\texttt{set.mm})} relates to  the typical version
in textbooks, and also to give you an informal feel for them.

\subsection{Propositional Calculus}

Propositional calculus\index{propositional calculus} concerns itself with
statements that can be interpreted as either true or false.  Some examples of
statements (outside of mathematics) that are either true or false are ``It is
raining today'' and ``The United States has a female president.'' In
mathematics, as we mentioned, statements are really formulas.

In propositional calculus, we don't care what the statements are.  We also
treat a logical combination of statements, such as ``It is raining today and
the United States has a female president,'' no differently from a single
statement.  Statements and their combinations are called well-formed formulas
(wffs)\index{well-formed formula (wff)}.  We define wffs only in terms of
other wffs and don't define what a ``starting'' wff is.  As is common practice
in the literature, we use Greek letters to represent wffs.

Specifically, suppose $\varphi$ and $\psi$ are wffs.  Then the combinations
$\varphi\rightarrow\psi$ (``$\varphi$ implies $\psi$,'' also read ``if
$\varphi$ then $\psi$'')\index{implication ($\rightarrow$)} and $\lnot\varphi$
(``not $\varphi$'')\index{negation ($\lnot$)} are also wffs.

The three axioms of propositional calculus\index{axioms of propositional
calculus} are all wffs of the following form:\footnote{A remarkable result of
C.~A.~Meredith\index{Meredith, C. A.} squeezes these three axioms into the
single axiom $((((\varphi\rightarrow \psi)\rightarrow(\neg \chi\rightarrow\neg
\theta))\rightarrow \chi )\rightarrow \tau)\rightarrow((\tau\rightarrow
\varphi)\rightarrow(\theta\rightarrow \varphi))$ \cite{CAMeredith},
which is believed to be the shortest possible.}
\begin{center}
     $\varphi\rightarrow(\psi\rightarrow \varphi)$\\

     $(\varphi\rightarrow (\psi\rightarrow \chi))\rightarrow
((\varphi\rightarrow  \psi)\rightarrow (\varphi\rightarrow \chi))$\\

     $(\neg \varphi\rightarrow \neg\psi)\rightarrow (\psi\rightarrow
\varphi)$
\end{center}

These three axioms are widely used.
They are attributed to Jan {\L}ukasiewicz
(pronounced woo-kah-SHAY-vitch) and was popularized by Alonzo Church,
who called it system P2. (Thanks to Ted Ulrich for this information.)

There are an infinite number of axioms, one for each possible
wff\index{well-formed formula (wff)} of the above form.  (For this reason,
axioms such as the above are often called ``axiom schemes.''\index{axiom
scheme})  Each Greek letter in the axioms may be substituted with a more
complex wff to result in another axiom.  For example, substituting
$\neg(\varphi\rightarrow\chi)$ for $\varphi$ in the first axiom yields
$\neg(\varphi\rightarrow\chi)\rightarrow(\psi\rightarrow
\neg(\varphi\rightarrow\chi))$, which is still an axiom.

To deduce new true statements (theorems\index{theorem}) from the axioms, a
rule\index{rule} called ``modus ponens''\index{modus ponens} is used.  This
rule states that if the wff $\varphi$ is an axiom or a theorem, and the wff
$\varphi\rightarrow\psi$ is an axiom or a theorem, then the wff $\psi$ is also
a theorem\index{theorem}.

As a non-mathematical example of modus ponens, suppose we have proved (or
taken as an axiom) ``Bob is a man'' and separately have proved (or taken as
an axiom) ``If Bob is a man, then Bob is a human.''  Using the rule of modus
ponens, we can logically deduce, ``Bob is a human.''

From Metamath's\index{Metamath} point of view, the axioms and the rule of
modus ponens just define a mechanical means for deducing new true statements
from existing true statements, and that is the complete content of
propositional calculus as far as Metamath is concerned.  You can read a logic
textbook to gain a better understanding of their meaning, or you can just let
their meaning slowly become apparent to you after you use them for a while.

It is actually rather easy to check to see if a formula is a theorem of
propositional calculus.  Theorems of propositional calculus are also called
``tautologies.''\index{tautology}  The technique to check whether a formula is
a tautology is called the ``truth table method,''\index{truth table} and it
works like this.  A wff $\varphi\rightarrow\psi$ is false whenever $\varphi$ is true
and $\psi$ is false.  Otherwise it is true.  A wff $\lnot\varphi$ is false
whenever $\varphi$ is true and false otherwise. To verify a tautology such as
$\varphi\rightarrow(\psi\rightarrow \varphi)$, you break it down into sub-wffs and
construct a truth table that accounts for all possible combinations of true
and false assigned to the wff metavariables:
\begin{center}\begin{tabular}{|c|c|c|c|}\hline
\mbox{$\varphi$} & \mbox{$\psi$} & \mbox{$\psi\rightarrow\varphi$}
    & \mbox{$\varphi\rightarrow(\psi\rightarrow \varphi)$} \\ \hline \hline
              T   &  T    &      T       &        T    \\ \hline
              T   &  F    &      T       &        T    \\ \hline
              F   &  T    &      F       &        T    \\ \hline
              F   &  F    &      T       &        T    \\ \hline
\end{tabular}\end{center}
If all entries in the last column are true, the formula is a tautology.

Now, the truth table method doesn't tell you how to prove the tautology from
the axioms, but only that a proof exists.  Finding an actual proof (especially
one that is short and elegant) can be challenging.  Methods do exist for
automatically generating proofs in propositional calculus, but the proofs that
result can sometimes be very long.  In the Metamath \texttt{set.mm}\index{set
theory database (\texttt{set.mm})} database, most
or all proofs were created manually.

Section \ref{metadefprop} discusses various definitions
that make propositional calculus easier to use.
For example, we define:

\begin{itemize}
\item $\varphi \vee \psi$
  is true if either $\varphi$ or $\psi$ (or both) are true
  (this is disjunction\index{disjunction ($\vee$)}
  aka logical {\sc or}\index{logical {\sc or} ($\vee$)}).

\item $\varphi \wedge \psi$
  is true if both $\varphi$ and $\psi$ are true
  (this is conjunction\index{conjunction ($\wedge$)}
  aka logical {\sc and}\index{logical {\sc and} ($\wedge$)}).

\item $\varphi \leftrightarrow \psi$
  is true if $\varphi$ and $\psi$ have the same value, that is,
  they are both true or both false
  (this is the biconditional\index{biconditional ($\leftrightarrow$)}).
\end{itemize}

\subsection{Predicate Calculus}

Predicate calculus\index{predicate calculus} introduces the concept of
``individual variables,''\index{variable!in predicate calculus}\index{individual
variable} which
we will usually just call ``variables.''
These variables can represent something other than true or false (wffs),
and will always represent sets when we get to set theory.  There are also
three new symbols $\forall$\index{universal quantifier ($\forall$)},
$=$\index{equality ($=$)}, and $\in$\index{stylized epsilon ($\in$)},
read ``for all,'' ``equals,'' and ``is an element of''
respectively.  We will represent variables with the letters $x$, $y$, $z$, and
$w$, as is common practice in the literature.
For example, $\forall x \varphi$ means ``for all possible values of
$x$, $\varphi$ is true.''

In predicate calculus, we extend the definition of a wff\index{well-formed
formula (wff)}.  If $\varphi$ is a wff and $x$ and $y$ are variables, then
$\forall x \, \varphi$, $x=y$, and $x\in y$ are wffs. Note that these three new
types of wffs can be considered ``starting'' wffs from which we can build
other wffs with $\rightarrow$ and $\neg$ .  The concept of a starting wff was
absent in propositional calculus.  But starting wff or not, all we are really
concerned with is whether our wffs are correctly constructed according to
these mechanical rules.

A quick aside:
To prevent confusion, it might be best at this point to think of the variables
of Metamath\index{Metamath} as ``metavariables,''\index{metavariable} because
they are not quite the same as the variables we are introducing here.  A
(meta)variable in Metamath can be a wff or an individual variable, as well
as many other things; in general, it represents a kind of place holder for an
unspecified sequence of math symbols\index{math symbol}.

Unlike propositional calculus, no decision procedure\index{decision procedure}
analogous to the truth table method exists (nor theoretically can exist) that
will definitely determine whether a formula is a theorem of predicate
calculus.  Much of the work in the field of automated theorem
proving\index{automated theorem proving} has been dedicated to coming up with
clever heuristics for proving theorems of predicate calculus, but they can
never be guaranteed to work always.

Section \ref{metadefpred} discusses various definitions
that make predicate calculus easier to use.
For example, we define
$\exists x \varphi$ to mean
``there exists at least one possible value of $x$ where $\varphi$ is true.''

We now turn to looking at how predicate calculus can be formally
represented.

\subsubsection{Common Axioms}

There is a new rule of inference in predicate calculus:  if $\varphi$ is
an axiom or a theorem, then $\forall x \,\varphi$ is also a
theorem\index{theorem}.  This is called the rule of
``generalization.''\index{rule of generalization}
This is easily represented in Metamath.

In standard texts of logic, there are often two axioms of predicate
calculus\index{axioms of predicate calculus}:
\begin{center}
  $\forall x \,\varphi ( x ) \rightarrow \varphi ( y )$,
      where ``$y$ is properly substituted for $x$.''\\
  $\forall x ( \varphi \rightarrow \psi )\rightarrow ( \varphi \rightarrow
    \forall x\, \psi )$,
    where ``$x$ is not free in $\varphi$.''
\end{center}

Now at first glance, this seems simple:  just two axioms.  However,
conditional clauses are attached to each axiom describing requirements that
may seem puzzling to you.  In addition, the first axiom puts a variable symbol
in parentheses after each wff, seemingly violating our definition of a
wff\index{well-formed formula (wff)}; this is just an informal way of
referring to some arbitrary variable that may occur in the wff.  The
conditional clauses do, of course, have a precise meaning, but as it turns out
the precise meaning is somewhat complicated and awkward to formalize in a
way that a computer can handle easily.  Unlike propositional calculus, a
certain amount of mathematical sophistication and practice is needed to be
able to easily grasp and manipulate these concepts correctly.

Predicate calculus may be presented with or without axioms for
equality\index{axioms of equality}\index{equality ($=$)}. We will require the
axioms of equality as a prerequisite for the version of set theory we will
use.  The axioms for equality, when included, are often represented using these
two axioms:
\begin{center}
$x=x$\\ \ \\
$x=y\rightarrow (\varphi(x,x)\rightarrow\varphi(x,y))$ where ``$\varphi(x,y)$
   arises from $\varphi(x,x)$ by replacing some, but not necessarily all,
   free\index{free variable}
   occurrences of $x$ by $y$,\\ provided that $y$ is free for $x$
   in $\varphi(x,x)$.'' \end{center}
% (Mendelson p. 95)
The first equality axiom is simple, but again,
the condition on the second one is
somewhat awkward to implement on a computer.

\subsubsection{Tarski System S2}

Of course, we are not the first to notice the complications of these
predicate calculus axioms when being rigorous.

Well-known logician Alfred Tarski published in 1965
a system he called system S2\cite[p.~77]{Tarski1965}.
Tarski's system is \textit{exactly equivalent} to the traditional textbook
formalization, but (by clever use of equality axioms) it eliminates the
latter's primitive notions of ``proper substitution'' and ``free variable,''
replacing them with direct substitution and the notion of a variable
not occurring in a formula (which we express with distinct variable
constraints).

In advocating his system, Tarski wrote, ``The relatively complicated
character of [free variables and proper substitution] is a source
of certain inconveniences of both practical and theoretical nature;
this is clearly experienced both in teaching an elementary course of
mathematical logic and in formalizing the syntax of predicate logic for
some theoretical purposes''\cite[p.~61]{Tarski1965}\index{Tarski, Alfred}.

\subsubsection{Developing a Metamath Representation}

The standard textbook axioms of predicate calculus are somewhat
cumbersome to implement on a computer because of the complex notions of
``free variable''\index{free variable} and ``proper
substitution.''\index{proper substitution}\index{substitution!proper}
While it is possible to use the Metamath\index{Metamath} language to
implement these concepts, we have chosen not to implement them
as primitive constructs in the
\texttt{set.mm} set theory database.  Instead, we have eliminated them
within the axioms
by carefully crafting the axioms so as to avoid them,
building on Tarski's system S2.  This makes it
easy for a beginner to follow the steps in a proof without knowing any
advanced concepts other than the simple concept of
replacing\index{substitution!variable}\index{variable substitution}
variables with expressions.

In order to develop the concepts of free variable and proper
substitution from the axioms, we use an additional
Metamath statement type called ``disjoint variable
restriction''\index{disjoint variables} that we have not encountered
before.  In the context of the axioms, the statement \texttt{\$d} $ x\,
y$\index{\texttt{\$d} statement} simply means that $x$ and $y$ must be
distinct\index{distinct variables}, i.e.\ they may not be simultaneously
substituted\index{substitution!variable}\index{variable substitution}
with the same variable.  The statement \texttt{\$d} $ x\, \varphi$ means
variable $x$ must not occur in wff $\varphi$.  For the precise
definition of \texttt{\$d}, see Section~\ref{dollard}.

\subsubsection{Metamath representation}

The Metamath axiom system for predicate calculus
defined in set.mm uses Tarski's system S2.
As noted above, this has a different representation
than the traditional textbook formalization,
but it is \textit{exactly equivalent} to the textbook formalization,
and it is \textit{much} easier to work with.
This is reproduced as system S3 in Section 6 of
Megill's formalization \cite{Megill}\index{Megill, Norman}.

There is one exception, Tarski's axiom of existence,
which we label as axiom ax-6.
In the case of ax-6, Tarski's version is weaker because it includes a
distinct variable proviso. If we wish, we can also weaken our version
in this way and still have a metalogically complete system. Theorem
ax6 shows this by deriving, in the presence of the other axioms, our
ax-6 from Tarski's weaker version ax6v. However, we chose the stronger
version for our system because it is simpler to state and easier to use.

Tarski's system was designed for proving specific theorems rather than
more general theorem schemes. However, theorem schemes are much more
efficient than specific theorems for building a body of mathematical
knowledge, since they can be reused with different instances as
needed. While Tarski does derive some theorem schemes from his axioms,
their proofs require concepts that are ``outside'' of the system, such as
induction on formula length. The verification of such proofs is difficult
to automate in a proof verifier. (Specifically, Tarski treats the formulas
of his system as set-theoretical objects. In order to verify the proofs
of his theorem schemes, a proof verifier would need a significant amount
of set theory built into it.)

The Metamath axiom system for predicate calculus extends
Tarski's system to eliminate this difficulty. The additional
``auxilliary'' axiom
schemes (as we will call them in this section; see below) endow Tarski's
system with a nice property we call
metalogical completeness \cite[Remark 9.6]{Megill}\index{Megill, Norman}.
As a result, we can prove any theorem scheme
expressable in the ``simple metalogic'' of Tarski's system by using
only Metamath's direct substitution rule applied to the axiom system
(and no other metalogical or set-theoretical notions ``outside'' of the
system). Simple metalogic consists of schemes containing wff metavariables
(with no arguments) and/or set (also called ``individual'') metavariables,
accompanied by optional provisos each stating that two specified set
metavariables must be distinct or that a specified set metavariable may
not occur in a specified wff metavariable. Metamath's logic and set theory
axiom and rule schemes are all examples of simple metalogic. The schemes
of traditional predicate calculus with equality are examples which are
not simple metalogic, because they use wff metavariables with arguments
and have ``free for'' and ``not free in'' side conditions.

A rigorous justification for this system, using an older but
exactly equivalent set of axioms, can be
found in \cite{Megill}\index{Megill, Norman}.

This allows us to
take a different approach in the Metamath\index{Metamath} database
\texttt{set.mm}\index{set theory database (\texttt{set.mm})}.  We do not
directly use the primitive notions of ``free variable''\index{free variable}
and ``proper substitution''\index{proper
substitution}\index{substitution!proper} at all as primitive constructs.
Instead, we use a set
of axioms that are almost as simple to manipulate as those of
propositional calculus.  Our axiom system avoids complex primitive
notions by effectively embedding the complexity into the axioms
themselves.  As a result, we will end up with a larger number of axioms,
but they are ideally suited for a computer language such as Metamath.
(Section~\ref{metaaxioms} shows these axioms.)

We will not elaborate further
on the ``free variable'' and ``proper substitution''
concepts here.  You may consult
\cite[ch.\ 3--4]{Hamilton}\index{Hamilton, Alan G.} (as well as
many other books) for a precise explanation
of these concepts.  If you intend to do serious mathematical work, it is wise
to become familiar with the traditional textbook approach; even though the
concepts embedded in their axioms require a higher level of sophistication,
they can be more practical to deal with on an everyday, informal basis.  Even
if you are just developing Metamath proofs, familiarity with the traditional
approach can help you arrive at a proof outline much faster, which you can
then convert to the detail required by Metamath.

We do develop proper substitution rules later on, but in set.mm
they are defined as derived constructs; they are not primitives.

You should also note that our system of predicate calculus is specifically
tailored for set theory; thus there are only two specific predicates $=$ and
$\in$ and no functions\index{function!in predicate calculus}
or constants\index{constant!in predicate calculus} unlike more general systems.
We later add these.

\subsection{Set Theory}

Traditional Zermelo--Fraenkel set theory\index{Zermelo--Fraenkel set
theory}\index{set theory} with the Axiom of Choice
has 10 axioms, which can be expressed in the
language of predicate calculus.  In this section, we will list only the
names and brief English descriptions of these axioms, since we will give
you the precise formulas used by the Metamath\index{Metamath} set theory
database \texttt{set.mm} later on.

In the descriptions of the axioms, we assume that $x$, $y$, $z$, $w$, and $v$
represent sets.  These are the same as the variables\index{variable!in set
theory} in our predicate calculus system above, except that now we informally
think of the variables as ranging over sets.  Note that the terms
``object,''\index{object} ``set,''\index{set} ``element,''\index{element}
``collection,''\index{collection} and ``family''\index{family} are synonymous,
as are ``is an element of,'' ``is a member of,''\index{member} ``is contained
in,'' and ``belongs to.''  The different terms are used for convenience; for
example, ``a collection of sets'' is less confusing than ``a set of sets.''
A set $x$ is said to be a ``subset''\index{subset} of $y$ if every element of
$x$ is also an element of $y$; we also say $x$ is ``included in''
$y$.

The axioms are very general and apply to almost any conceivable mathematical
object, and this level of abstraction can be overwhelming at first.  To gain an
intuitive feel, it can be helpful to draw a picture illustrating the concept;
for example, a circle containing dots could represent a collection of sets,
and a smaller circle drawn inside the circle could represent a subset.
Overlapping circles can illustrate intersection and union.  Circles that
illustrate the concepts of set theory are frequently used in elementary
textbooks and are called Venn diagrams\index{Venn diagram}.\index{axioms of
set theory}

1. Axiom of Extensionality:  Two sets are identical if they contain the same
   elements.\index{Axiom of Extensionality}

2. Axiom of Pairing:  The set $\{ x , y \}$ exists.\index{Axiom of Pairing}

3. Axiom of Power Sets:  The power set of a set (the collection of all of
   its subsets) exists.  For example, the power set of $\{x,y\}$ is
   $\{\varnothing,\{x\},\{y\},\{x,y\}\}$ and it exists.\index{Axiom
of Power Sets}

4. Axiom of the Null Set:  The empty set $\varnothing$ exists.\index{Axiom of
the Null Set}

5. Axiom of Union:  The union of a set (the set containing the elements of
   its members) exists.  For example, the union of $\{\{x,y\},\{z\}\}$ is
 $\{x,y,z\}$ and
   it exists.\index{Axiom of Union}

6. Axiom of Regularity:  Roughly, no set can contain itself, nor can there
   be membership ``loops,'' such as a set being an
   element of one of its members.\index{Axiom of Regularity}

7. Axiom of Infinity:  An infinite set exists.  An example of an infinite
   set is the set of all
   integers.\index{Axiom of Infinity}

8. Axiom of Separation:  The set exists that is obtained by restricting $x$
   with some property.  For example, if the set of all integers exists,
   then the set of all even integers exists.\index{Axiom of Separation}

9. Axiom of Replacement:  The range of a function whose domain is restricted
   to the elements of a set $x$, is also a set.  For example, there
   is a function
   from integers (the function's domain) to their squares (its
   range).  If we
   restrict the domain to even integers, its range will become the set of
   squares of even integers, so this axiom asserts that the set of
    squares of even numbers exists.  Technical note:  In general, the
   ``function'' need not be a set but can be a proper class.
   \index{Axiom of Replacement}

10. Axiom of Choice:  Let $x$ be a set whose members are pairwise
  disjoint\index{disjoint sets} (i.e,
  whose members contain no elements in common).  Then there exists another
  set containing one element from each member of $x$.  For
  example, if $x$ is
  $\{\{y,z\},\{w,v\}\}$, where $y$, $z$, $w$, and $v$ are
  different sets, then a set such as $\{z,w\}$
  exists (but the axiom doesn't tell
  us which one).  (Actually the Axiom
  of Choice is redundant if the set $x$, as in this example, has a finite
  number of elements.)\index{Axiom of Choice}

The Axiom of Choice is usually considered an extension of ZF set theory rather
than a proper part of it.  It is sometimes considered philosophically
controversial because it specifies the existence of a set without specifying
what the set is. Constructive logics, including intuitionistic logic,
do not accept the axiom of choice.
Since there is some lingering controversy, we often prefer proofs that do
not use the axiom of choice (where there is a known alternative), and
in some cases we will use weaker axioms than the full axiom of choice.
That said, the axiom of choice is a powerful and widely-accepted tool,
so we do use it when needed.
ZF set theory that includes the Axiom of Choice is
called Zermelo--Fraenkel set theory with choice (ZFC\index{ZFC set theory}).

When expressed symbolically, the Axiom of Separation and the Axiom of
Replacement contain wff symbols and therefore each represent infinitely many
axioms, one for each possible wff. For this reason, they are often called
axiom schemes\index{axiom scheme}\index{well-formed formula (wff)}.

It turns out that the Axiom of the Null Set, the Axiom of Pairing, and the
Axiom of Separation can be derived from the other axioms and are therefore
unnecessary, although they tend to be included in standard texts for various
reasons (historical, philosophical, and possibly because some authors may not
know this).  In the Metamath\index{Metamath} set theory database, these
redundant axioms are derived from the other ones instead of truly
being considered axioms.
This is in keeping with our general goal of minimizing the number of
axioms we must depend on.

\subsection{Other Axioms}

Above we qualified the phrase "all of mathematics" with "essentially."
The main important missing piece is the ability to do category theory,
which requires huge sets (inaccessible cardinals) larger than those
postulated by the ZFC axioms. The Tarski--Grothendieck Axiom postulates
the existence of such sets.
Note that this is the same axiom used by Mizar for supporting
category theory.
The Tarski--Grothendieck axiom
can be viewed as a very strong replacement of the Axiom of Infinity,
the Axiom of Choice, and the Axiom of Power Sets.
The \texttt{set.mm} database includes this axiom; see the database
for details about it.
Again, we only use this axiom when we need to.
You are only likely to encounter or use this axiom if you are doing
category theory, since its use is highly specialized,
so we will not list the Tarsky-Grothendieck axiom
in the short list of axioms below.

Can there be even more axioms?
Of course.
G\"{o}del showed that no finite set of axioms or axiom schemes can completely
describe any consistent theory strong enough to include arithmetic.
But practically speaking, the ones above are the accepted foundation that
almost all mathematicians explicitly or implicitly base their work on.

\section{The Axioms in the Metamath Language}\label{metaaxioms}

Here we list the axioms as they appear in
\texttt{set.mm}\index{set theory database (\texttt{set.mm})} so you can
look them up there easily.  Incidentally, the \texttt{show statement
/tex} command\index{\texttt{show statement} command} was used to
typeset them.

%macros from show statement /tex
\newbox\mlinebox
\newbox\mtrialbox
\newbox\startprefix  % Prefix for first line of a formula
\newbox\contprefix  % Prefix for continuation line of a formula
\def\startm{  % Initialize formula line
  \setbox\mlinebox=\hbox{\unhcopy\startprefix}
}
\def\m#1{  % Add a symbol to the formula
  \setbox\mtrialbox=\hbox{\unhcopy\mlinebox $\,#1$}
  \ifdim\wd\mtrialbox>\hsize
    \box\mlinebox
    \setbox\mlinebox=\hbox{\unhcopy\contprefix $\,#1$}
  \else
    \setbox\mlinebox=\hbox{\unhbox\mtrialbox}
  \fi
}
\def\endm{  % Output the last line of a formula
  \box\mlinebox
}

% \SLASH for \ , \TOR for \/ (text OR), \TAND for /\ (text and)
% This embeds a following forced space to force the space.
\newcommand\SLASH{\char`\\~}
\newcommand\TOR{\char`\\/~}
\newcommand\TAND{/\char`\\~}
%
% Macro to output metamath raw text.
% This assumes \startprefix and \contprefix are set.
% NOTE: "\" is tricky to escape, use \SLASH, \TOR, and \TAND inside.
% Any use of "$ { ~ ^" must be escaped; ~ and ^ must be escaped specially.
% We escape { and } for consistency.
% For more about how this macro written, see:
% https://stackoverflow.com/questions/4073674/
% how-to-disable-indentation-in-particular-section-in-latex/4075706
% Use frenchspacing, or "e." will get an extra space after it.
\newlength\mystoreparindent
\newlength\mystorehangindent
\newenvironment{mmraw}{%
\setlength{\mystoreparindent}{\the\parindent}
\setlength{\mystorehangindent}{\the\hangindent}
\setlength{\parindent}{0pt} % TODO - we'll put in the \startprefix instead
\setlength{\hangindent}{\wd\the\contprefix}
\begin{flushleft}
\begin{frenchspacing}
\begin{tt}
{\unhcopy\startprefix}%
}{%
\end{tt}
\end{frenchspacing}
\end{flushleft}
\setlength{\parindent}{\mystoreparindent}
\setlength{\hangindent}{\mystorehangindent}
\vskip 1ex
}

\needspace{5\baselineskip}
\subsection{Propositional Calculus}\label{propcalc}\index{axioms of
propositional calculus}

\needspace{2\baselineskip}
Axiom of Simplification.\label{ax1}

\setbox\startprefix=\hbox{\tt \ \ ax-1\ \$a\ }
\setbox\contprefix=\hbox{\tt \ \ \ \ \ \ \ \ \ \ }
\startm
\m{\vdash}\m{(}\m{\varphi}\m{\rightarrow}\m{(}\m{\psi}\m{\rightarrow}\m{\varphi}\m{)}
\m{)}
\endm

\needspace{3\baselineskip}
\noindent Axiom of Distribution.

\setbox\startprefix=\hbox{\tt \ \ ax-2\ \$a\ }
\setbox\contprefix=\hbox{\tt \ \ \ \ \ \ \ \ \ \ }
\startm
\m{\vdash}\m{(}\m{(}\m{\varphi}\m{\rightarrow}\m{(}\m{\psi}\m{\rightarrow}\m{\chi}
\m{)}\m{)}\m{\rightarrow}\m{(}\m{(}\m{\varphi}\m{\rightarrow}\m{\psi}\m{)}\m{
\rightarrow}\m{(}\m{\varphi}\m{\rightarrow}\m{\chi}\m{)}\m{)}\m{)}
\endm

\needspace{2\baselineskip}
\noindent Axiom of Contraposition.

\setbox\startprefix=\hbox{\tt \ \ ax-3\ \$a\ }
\setbox\contprefix=\hbox{\tt \ \ \ \ \ \ \ \ \ \ }
\startm
\m{\vdash}\m{(}\m{(}\m{\lnot}\m{\varphi}\m{\rightarrow}\m{\lnot}\m{\psi}\m{)}\m{
\rightarrow}\m{(}\m{\psi}\m{\rightarrow}\m{\varphi}\m{)}\m{)}
\endm


\needspace{4\baselineskip}
\noindent Rule of Modus Ponens.\label{axmp}\index{modus ponens}

\setbox\startprefix=\hbox{\tt \ \ maj\ \$e\ }
\setbox\contprefix=\hbox{\tt \ \ \ \ \ \ \ \ \ }
\startm
\m{\vdash}\m{(}\m{\varphi}\m{\rightarrow}\m{\psi}\m{)}
\endm

\setbox\startprefix=\hbox{\tt \ \ min\ \$e\ }
\setbox\contprefix=\hbox{\tt \ \ \ \ \ \ \ \ \ }
\startm
\m{\vdash}\m{\varphi}
\endm

\setbox\startprefix=\hbox{\tt \ \ ax-mp\ \$a\ }
\setbox\contprefix=\hbox{\tt \ \ \ \ \ \ \ \ \ \ \ }
\startm
\m{\vdash}\m{\psi}
\endm


\needspace{7\baselineskip}
\subsection{Axioms of Predicate Calculus with Equality---Tarski's S2}\index{axioms of predicate calculus}

\needspace{3\baselineskip}
\noindent Rule of Generalization.\index{rule of generalization}

\setbox\startprefix=\hbox{\tt \ \ ax-g.1\ \$e\ }
\setbox\contprefix=\hbox{\tt \ \ \ \ \ \ \ \ \ \ \ \ }
\startm
\m{\vdash}\m{\varphi}
\endm

\setbox\startprefix=\hbox{\tt \ \ ax-gen\ \$a\ }
\setbox\contprefix=\hbox{\tt \ \ \ \ \ \ \ \ \ \ \ \ }
\startm
\m{\vdash}\m{\forall}\m{x}\m{\varphi}
\endm

\needspace{2\baselineskip}
\noindent Axiom of Quantified Implication.

\setbox\startprefix=\hbox{\tt \ \ ax-4\ \$a\ }
\setbox\contprefix=\hbox{\tt \ \ \ \ \ \ \ \ \ \ }
\startm
\m{\vdash}\m{(}\m{\forall}\m{x}\m{(}\m{\forall}\m{x}\m{\varphi}\m{\rightarrow}\m{
\psi}\m{)}\m{\rightarrow}\m{(}\m{\forall}\m{x}\m{\varphi}\m{\rightarrow}\m{
\forall}\m{x}\m{\psi}\m{)}\m{)}
\endm

\needspace{3\baselineskip}
\noindent Axiom of Distinctness.

% Aka: Add $d x ph $.
\setbox\startprefix=\hbox{\tt \ \ ax-5\ \$a\ }
\setbox\contprefix=\hbox{\tt \ \ \ \ \ \ \ \ \ \ }
\startm
\m{\vdash}\m{(}\m{\varphi}\m{\rightarrow}\m{\forall}\m{x}\m{\varphi}\m{)}\m{where}\m{ }\m{\$d}\m{ }\m{x}\m{ }\m{\varphi}\m{ }\m{(}\m{x}\m{ }\m{does}\m{ }\m{not}\m{ }\m{occur}\m{ }\m{in}\m{ }\m{\varphi}\m{)}
\endm

\needspace{2\baselineskip}
\noindent Axiom of Existence.

\setbox\startprefix=\hbox{\tt \ \ ax-6\ \$a\ }
\setbox\contprefix=\hbox{\tt \ \ \ \ \ \ \ \ \ \ }
\startm
\m{\vdash}\m{(}\m{\forall}\m{x}\m{(}\m{x}\m{=}\m{y}\m{\rightarrow}\m{\forall}
\m{x}\m{\varphi}\m{)}\m{\rightarrow}\m{\varphi}\m{)}
\endm

\needspace{2\baselineskip}
\noindent Axiom of Equality.

\setbox\startprefix=\hbox{\tt \ \ ax-7\ \$a\ }
\setbox\contprefix=\hbox{\tt \ \ \ \ \ \ \ \ \ \ }
\startm
\m{\vdash}\m{(}\m{x}\m{=}\m{y}\m{\rightarrow}\m{(}\m{x}\m{=}\m{z}\m{
\rightarrow}\m{y}\m{=}\m{z}\m{)}\m{)}
\endm

\needspace{2\baselineskip}
\noindent Axiom of Left Equality for Binary Predicate.

\setbox\startprefix=\hbox{\tt \ \ ax-8\ \$a\ }
\setbox\contprefix=\hbox{\tt \ \ \ \ \ \ \ \ \ \ \ }
\startm
\m{\vdash}\m{(}\m{x}\m{=}\m{y}\m{\rightarrow}\m{(}\m{x}\m{\in}\m{z}\m{
\rightarrow}\m{y}\m{\in}\m{z}\m{)}\m{)}
\endm

\needspace{2\baselineskip}
\noindent Axiom of Right Equality for Binary Predicate.

\setbox\startprefix=\hbox{\tt \ \ ax-9\ \$a\ }
\setbox\contprefix=\hbox{\tt \ \ \ \ \ \ \ \ \ \ \ }
\startm
\m{\vdash}\m{(}\m{x}\m{=}\m{y}\m{\rightarrow}\m{(}\m{z}\m{\in}\m{x}\m{
\rightarrow}\m{z}\m{\in}\m{y}\m{)}\m{)}
\endm


\needspace{4\baselineskip}
\subsection{Axioms of Predicate Calculus with Equality---Auxiliary}\index{axioms of predicate calculus - auxiliary}

\needspace{2\baselineskip}
\noindent Axiom of Quantified Negation.

\setbox\startprefix=\hbox{\tt \ \ ax-10\ \$a\ }
\setbox\contprefix=\hbox{\tt \ \ \ \ \ \ \ \ \ \ }
\startm
\m{\vdash}\m{(}\m{\lnot}\m{\forall}\m{x}\m{\lnot}\m{\forall}\m{x}\m{\varphi}\m{
\rightarrow}\m{\varphi}\m{)}
\endm

\needspace{2\baselineskip}
\noindent Axiom of Quantifier Commutation.

\setbox\startprefix=\hbox{\tt \ \ ax-11\ \$a\ }
\setbox\contprefix=\hbox{\tt \ \ \ \ \ \ \ \ \ \ }
\startm
\m{\vdash}\m{(}\m{\forall}\m{x}\m{\forall}\m{y}\m{\varphi}\m{\rightarrow}\m{
\forall}\m{y}\m{\forall}\m{x}\m{\varphi}\m{)}
\endm

\needspace{3\baselineskip}
\noindent Axiom of Substitution.

\setbox\startprefix=\hbox{\tt \ \ ax-12\ \$a\ }
\setbox\contprefix=\hbox{\tt \ \ \ \ \ \ \ \ \ \ \ }
\startm
\m{\vdash}\m{(}\m{\lnot}\m{\forall}\m{x}\m{\,x}\m{=}\m{y}\m{\rightarrow}\m{(}
\m{x}\m{=}\m{y}\m{\rightarrow}\m{(}\m{\varphi}\m{\rightarrow}\m{\forall}\m{x}\m{(}
\m{x}\m{=}\m{y}\m{\rightarrow}\m{\varphi}\m{)}\m{)}\m{)}\m{)}
\endm

\needspace{3\baselineskip}
\noindent Axiom of Quantified Equality.

\setbox\startprefix=\hbox{\tt \ \ ax-13\ \$a\ }
\setbox\contprefix=\hbox{\tt \ \ \ \ \ \ \ \ \ \ \ }
\startm
\m{\vdash}\m{(}\m{\lnot}\m{\forall}\m{z}\m{\,z}\m{=}\m{x}\m{\rightarrow}\m{(}
\m{\lnot}\m{\forall}\m{z}\m{\,z}\m{=}\m{y}\m{\rightarrow}\m{(}\m{x}\m{=}\m{y}
\m{\rightarrow}\m{\forall}\m{z}\m{\,x}\m{=}\m{y}\m{)}\m{)}\m{)}
\endm

% \noindent Axiom of Quantifier Substitution
%
% \setbox\startprefix=\hbox{\tt \ \ ax-c11n\ \$a\ }
% \setbox\contprefix=\hbox{\tt \ \ \ \ \ \ \ \ \ \ \ }
% \startm
% \m{\vdash}\m{(}\m{\forall}\m{x}\m{\,x}\m{=}\m{y}\m{\rightarrow}\m{(}\m{\forall}
% \m{x}\m{\varphi}\m{\rightarrow}\m{\forall}\m{y}\m{\varphi}\m{)}\m{)}
% \endm
%
% \noindent Axiom of Distinct Variables. (This axiom requires
% that two individual variables
% be distinct\index{\texttt{\$d} statement}\index{distinct
% variables}.)
%
% \setbox\startprefix=\hbox{\tt \ \ \ \ \ \ \ \ \$d\ }
% \setbox\contprefix=\hbox{\tt \ \ \ \ \ \ \ \ \ \ \ }
% \startm
% \m{x}\m{\,}\m{y}
% \endm
%
% \setbox\startprefix=\hbox{\tt \ \ ax-c16\ \$a\ }
% \setbox\contprefix=\hbox{\tt \ \ \ \ \ \ \ \ \ \ \ }
% \startm
% \m{\vdash}\m{(}\m{\forall}\m{x}\m{\,x}\m{=}\m{y}\m{\rightarrow}\m{(}\m{\varphi}\m{
% \rightarrow}\m{\forall}\m{x}\m{\varphi}\m{)}\m{)}
% \endm

% \noindent Axiom of Quantifier Introduction (2).  (This axiom requires
% that the individual variable not occur in the
% wff\index{\texttt{\$d} statement}\index{distinct variables}.)
%
% \setbox\startprefix=\hbox{\tt \ \ \ \ \ \ \ \ \$d\ }
% \setbox\contprefix=\hbox{\tt \ \ \ \ \ \ \ \ \ \ \ }
% \startm
% \m{x}\m{\,}\m{\varphi}
% \endm
% \setbox\startprefix=\hbox{\tt \ \ ax-5\ \$a\ }
% \setbox\contprefix=\hbox{\tt \ \ \ \ \ \ \ \ \ \ \ }
% \startm
% \m{\vdash}\m{(}\m{\varphi}\m{\rightarrow}\m{\forall}\m{x}\m{\varphi}\m{)}
% \endm

\subsection{Set Theory}\label{mmsettheoryaxioms}

In order to make the axioms of set theory\index{axioms of set theory} a little
more compact, there are several definitions from logic that we make use of
implicitly, namely, ``logical {\sc and},''\index{conjunction ($\wedge$)}
\index{logical {\sc and} ($\wedge$)} ``logical equivalence,''\index{logical
equivalence ($\leftrightarrow$)}\index{biconditional ($\leftrightarrow$)} and
``there exists.''\index{existential quantifier ($\exists$)}

\begin{center}\begin{tabular}{rcl}
  $( \varphi \wedge \psi )$ &\mbox{stands for}& $\neg ( \varphi
     \rightarrow \neg \psi )$\\
  $( \varphi \leftrightarrow \psi )$& \mbox{stands
     for}& $( ( \varphi \rightarrow \psi ) \wedge
     ( \psi \rightarrow \varphi ) )$\\
  $\exists x \,\varphi$ &\mbox{stands for}& $\neg \forall x \neg \varphi$
\end{tabular}\end{center}

In addition, the axioms of set theory require that all variables be
dis\-tinct,\index{distinct variables}\footnote{Set theory axioms can be
devised so that {\em no} variables are required to be distinct,
provided we replace \texttt{ax-c16} with an axiom stating that ``at
least two things exist,'' thus
making \texttt{ax-5} the only other axiom requiring the
\texttt{\$d} statement.  These axioms are unconventional and are not
presented here, but they can be found on the \url{http://metamath.org}
web site.  See also the Comment on
p.~\pageref{nodd}.}\index{\texttt{\$d} statement} thus we also assume:
\begin{center}
  \texttt{\$d }$x\,y\,z\,w$
\end{center}

\needspace{2\baselineskip}
\noindent Axiom of Extensionality.\index{Axiom of Extensionality}

\setbox\startprefix=\hbox{\tt \ \ ax-ext\ \$a\ }
\setbox\contprefix=\hbox{\tt \ \ \ \ \ \ \ \ \ \ \ \ }
\startm
\m{\vdash}\m{(}\m{\forall}\m{x}\m{(}\m{x}\m{\in}\m{y}\m{\leftrightarrow}\m{x}
\m{\in}\m{z}\m{)}\m{\rightarrow}\m{y}\m{=}\m{z}\m{)}
\endm

\needspace{3\baselineskip}
\noindent Axiom of Replacement.\index{Axiom of Replacement}

\setbox\startprefix=\hbox{\tt \ \ ax-rep\ \$a\ }
\setbox\contprefix=\hbox{\tt \ \ \ \ \ \ \ \ \ \ \ \ }
\startm
\m{\vdash}\m{(}\m{\forall}\m{w}\m{\exists}\m{y}\m{\forall}\m{z}\m{(}\m{%
\forall}\m{y}\m{\varphi}\m{\rightarrow}\m{z}\m{=}\m{y}\m{)}\m{\rightarrow}\m{%
\exists}\m{y}\m{\forall}\m{z}\m{(}\m{z}\m{\in}\m{y}\m{\leftrightarrow}\m{%
\exists}\m{w}\m{(}\m{w}\m{\in}\m{x}\m{\wedge}\m{\forall}\m{y}\m{\varphi}\m{)}%
\m{)}\m{)}
\endm

\needspace{2\baselineskip}
\noindent Axiom of Union.\index{Axiom of Union}

\setbox\startprefix=\hbox{\tt \ \ ax-un\ \$a\ }
\setbox\contprefix=\hbox{\tt \ \ \ \ \ \ \ \ \ \ \ }
\startm
\m{\vdash}\m{\exists}\m{x}\m{\forall}\m{y}\m{(}\m{\exists}\m{x}\m{(}\m{y}\m{
\in}\m{x}\m{\wedge}\m{x}\m{\in}\m{z}\m{)}\m{\rightarrow}\m{y}\m{\in}\m{x}\m{)}
\endm

\needspace{2\baselineskip}
\noindent Axiom of Power Sets.\index{Axiom of Power Sets}

\setbox\startprefix=\hbox{\tt \ \ ax-pow\ \$a\ }
\setbox\contprefix=\hbox{\tt \ \ \ \ \ \ \ \ \ \ \ \ }
\startm
\m{\vdash}\m{\exists}\m{x}\m{\forall}\m{y}\m{(}\m{\forall}\m{x}\m{(}\m{x}\m{
\in}\m{y}\m{\rightarrow}\m{x}\m{\in}\m{z}\m{)}\m{\rightarrow}\m{y}\m{\in}\m{x}
\m{)}
\endm

\needspace{3\baselineskip}
\noindent Axiom of Regularity.\index{Axiom of Regularity}

\setbox\startprefix=\hbox{\tt \ \ ax-reg\ \$a\ }
\setbox\contprefix=\hbox{\tt \ \ \ \ \ \ \ \ \ \ \ \ }
\startm
\m{\vdash}\m{(}\m{\exists}\m{x}\m{\,x}\m{\in}\m{y}\m{\rightarrow}\m{\exists}
\m{x}\m{(}\m{x}\m{\in}\m{y}\m{\wedge}\m{\forall}\m{z}\m{(}\m{z}\m{\in}\m{x}\m{
\rightarrow}\m{\lnot}\m{z}\m{\in}\m{y}\m{)}\m{)}\m{)}
\endm

\needspace{3\baselineskip}
\noindent Axiom of Infinity.\index{Axiom of Infinity}

\setbox\startprefix=\hbox{\tt \ \ ax-inf\ \$a\ }
\setbox\contprefix=\hbox{\tt \ \ \ \ \ \ \ \ \ \ \ \ \ \ \ }
\startm
\m{\vdash}\m{\exists}\m{x}\m{(}\m{y}\m{\in}\m{x}\m{\wedge}\m{\forall}\m{y}%
\m{(}\m{y}\m{\in}\m{x}\m{\rightarrow}\m{\exists}\m{z}\m{(}\m{y}\m{\in}\m{z}\m{%
\wedge}\m{z}\m{\in}\m{x}\m{)}\m{)}\m{)}
\endm

\needspace{4\baselineskip}
\noindent Axiom of Choice.\index{Axiom of Choice}

\setbox\startprefix=\hbox{\tt \ \ ax-ac\ \$a\ }
\setbox\contprefix=\hbox{\tt \ \ \ \ \ \ \ \ \ \ \ \ \ \ }
\startm
\m{\vdash}\m{\exists}\m{x}\m{\forall}\m{y}\m{\forall}\m{z}\m{(}\m{(}\m{y}\m{%
\in}\m{z}\m{\wedge}\m{z}\m{\in}\m{w}\m{)}\m{\rightarrow}\m{\exists}\m{w}\m{%
\forall}\m{y}\m{(}\m{\exists}\m{w}\m{(}\m{(}\m{y}\m{\in}\m{z}\m{\wedge}\m{z}%
\m{\in}\m{w}\m{)}\m{\wedge}\m{(}\m{y}\m{\in}\m{w}\m{\wedge}\m{w}\m{\in}\m{x}%
\m{)}\m{)}\m{\leftrightarrow}\m{y}\m{=}\m{w}\m{)}\m{)}
\endm

\subsection{That's It}

There you have it, the axioms for (essentially) all of mathematics!
Wonder at them and stare at them in awe.  Put a copy in your wallet, and
you will carry in your pocket the encoding for all theorems ever proved
and that ever will be proved, from the most mundane to the most
profound.

\section{A Hierarchy of Definitions}\label{hierarchy}

The axioms in the previous section in principle embody everything that can be
done within standard mathematics.  However, it is impractical to accomplish
very much by using them directly, for even simple concepts (from a human
perspective) can involve extremely long, incomprehensible formulas.
Mathematics is made practical by introducing definitions\index{definition}.
Definitions usually introduce new symbols, or at least new relationships among
existing symbols, to abbreviate more complex formulas.  An important
requirement for a definition is that there exist a straightforward
(algorithmic) method for eliminating the abbreviation by expanding it into the
more primitive symbol string that it represents.  Some
important definitions included in
the file \texttt{set.mm} are listed in this section for reference, and also to
give you a feel for why something like $\omega$\index{omega ($\omega$)} (the
set of natural numbers\index{natural number} 0, 1, 2,\ldots) becomes very
complicated when completely expanded into primitive symbols.

What is the motivation for definitions, aside from allowing complicated
expressions to be expressed more simply?  In the case of  $\omega$, one goal is
to provide a basis for the theory of natural numbers.\index{natural number}
Before set theory was invented, a set of axioms for arithmetic, called Peano's
postulates\index{Peano's postulates}, was devised and shown to have the
properties one expects for natural numbers.  Now anyone can postulate a
set of axioms, but if the axioms are inconsistent contradictions can be derived
from them.  Once a contradiction is derived, anything can be trivially
proved, including
all the facts of arithmetic and their negations.  To ensure that an
axiom system is at least as reliable as the axioms for set theory, we can
define sets and operations on those sets that satisfy the new axioms. In the
\texttt{set.mm} Metamath database, we prove that the elements of $\omega$ satisfy
Peano's postulates, and it's a long and hard journey to get there directly
from the axioms of set theory.  But the result is confidence in the
foundations of arithmetic.  And there is another advantage:  we now have all
the tools of set theory at our disposal for manipulating objects that obey the
axioms for arithmetic.

What are the criteria we use for definitions?  First, and of utmost importance,
the definition should not be {\em creative}\index{creative
definition}\index{definition!creative}, that
is it should not allow an expression that previously qualified as a wff but
was not provable, to become provable.   Second, the definition should be {\em
eliminable}\index{definition!eliminability}, that is, there should exist an
algorithmic method for converting any expression using the definition into
a logically equivalent expression that previously qualified as a wff.

In almost all cases below, definitions connect two expressions with either
$\leftrightarrow$ or $=$.  Eliminating\footnote{Here we mean the
elimination that a human might do in his or her head.  To eliminate them as
part of a Metamath proof we would invoke one of a number of
theorems that deal with transitivity of equivalence or equality; there are
many such examples in the proofs in \texttt{set.mm}.} such a definition is a
simple matter of substituting the expression on the left-hand side ({\em
definiendum}\index{definiendum} or thing being defined) with the equivalent,
more primitive expression on the right-hand side ({\em
definiens}\index{definiens} or definition).

Often a definition has variables on the right-hand side which do not appear on
the left-hand side; these are called {\em dummy variables}.\index{dummy
variable!in definitions}  In this case, any
allowable substitution (such as a new, distinct
variable) can be used when the definition is eliminated.  Dummy variables may
be used only if they are {\em effectively bound}\index{effectively bound
variable}, meaning that the definition will remain logically equivalent upon
any substitution of a dummy variable with any other {\em qualifying
expression}\index{qualifying expression}, i.e.\ any symbol string (such as
another variable) that
meets the restrictions on the dummy variable imposed by \texttt{\$d} and
\texttt{\$f} statements.  For example, we could define a constant $\perp$
(inverted tee, meaning logical ``false'') as $( \varphi \wedge \lnot \varphi
)$, i.e.\ ``phi and not phi.''  Here $\varphi$ is effectively bound because the
definition remains logically equivalent when we replace $\varphi$ with any
other wff.  (It is actually \texttt{df-fal}
in \texttt{set.mm}, which defines $\perp$.)

There are two cases where eliminating definitions is a little more
complex.  These cases are the definitions \texttt{df-bi} and
\texttt{df-cleq}.  The first stretches the concept of a definition a
little, as in effect it ``defines a definition;'' however, it meets our
requirements for a definition in that it is eliminable and does not
strengthen the language.  Theorem \texttt{bii} shows the substitution
needed to eliminate the $\leftrightarrow$\index{logical equivalence
($\leftrightarrow$)}\index{biconditional ($\leftrightarrow$)} symbol.

Definition \texttt{df-cleq}\index{equality ($=$)} extends the usage of
the equality symbol to include ``classes''\index{class} in set theory.  The
reason it is potentially problematic is that it can lead to statements which
do not follow from logic alone but presuppose the Axiom of
Extensionality\index{Axiom of Extensionality}, so we include this axiom
as a hypothesis for the definition.  We could have made \texttt{df-cleq} directly
eliminable by introducing a new equality symbol, but have chosen not to do so
in keeping with standard textbook practice.  Definitions such as \texttt{df-cleq}
that extend the meaning of existing symbols must be introduced carefully so
that they do not lead to contradictions.  Definition \texttt{df-clel} also
extends the meaning of an existing symbol ($\in$); while it doesn't strengthen
the language like \texttt{df-cleq}, this is not obvious and it must also be
subject to the same scrutiny.

Exercise:  Study how the wff $x\in\omega$, meaning ``$x$ is a natural
number,'' could be expanded in terms of primitive symbols, starting with the
definitions \texttt{df-clel} on p.~\pageref{dfclel} and \texttt{df-om} on
p.~\pageref{dfom} and working your way back.  Don't bother to work out the
details; just make sure that you understand how you could do it in principle.
The answer is shown in the footnote on p.~\pageref{expandom}.  If you
actually do work it out, you won't get exactly the same answer because we used
a few simplifications such as discarding occurrences of $\lnot\lnot$ (double
negation).

In the definitions below, we have placed the {\sc ascii} Metamath source
below each of the formulas to help you become familiar with the
notation in the database.  For simplicity, the necessary \texttt{\$f}
and \texttt{\$d} statements are not shown.  If you are in doubt, use the
\texttt{show statement}\index{\texttt{show statement} command} command
in the Metamath program to see the full statement.
A selection of this notation is summarized in Appendix~\ref{ASCII}.

To understand the motivation for these definitions, you should consult the
references indicated:  Takeuti and Zaring \cite{Takeuti}\index{Takeuti, Gaisi},
Quine \cite{Quine}\index{Quine, Willard Van Orman}, Bell and Machover
\cite{Bell}\index{Bell, J. L.}, and Enderton \cite{Enderton}\index{Enderton,
Herbert B.}.  Our list of definitions is provided more for reference than as a
learning aid.  However, by looking at a few of them you can gain a feel for
how the hierarchy is built up.  The definitions are a representative sample of
the 130 or so in \texttt{set.mm}, but they are complete with respect to the
theorem examples we will present in Section~\ref{sometheorems}.  Also, some are
slightly different from, but logically equivalent to, the ones in \texttt{set.mm}
(some of which have been revised over time to shorten them, for example).

\subsection{Definitions for Propositional Calculus}\label{metadefprop}

The symbols $\varphi$, $\psi$, and $\chi$ represent wffs.

Our first definition introduces the biconditional
connective\footnote{The term ``connective'' is informally used to mean a
symbol that is placed between two variables or adjacent to a variable,
whereas a mathematical ``constant'' usually indicates a symbol such as
the number 0 that may replace a variable or metavariable.  From
Metamath's point of view, there is no distinction between a connective
and a constant; both are constants in the Metamath
language.}\index{connective}\index{constant} (also called logical
equivalence)\index{logical equivalence
($\leftrightarrow$)}\index{biconditional ($\leftrightarrow$)}.  Unlike
most traditional developments, we have chosen not to have a separate
symbol such as ``Df.'' to mean ``is defined as.''  Instead, we will use
the biconditional connective for this purpose, as it lets us use
logic to manipulate definitions directly.  Here we state the properties
of the biconditional connective with a carefully crafted \texttt{\$a}
statement, which effectively uses the biconditional connective to define
itself.  The $\leftrightarrow$ symbol can be eliminated from a formula
using theorem \texttt{bii}, which is derived later.

\vskip 2ex
\noindent Define the biconditional connective.\label{df-bi}

\vskip 0.5ex
\setbox\startprefix=\hbox{\tt \ \ df-bi\ \$a\ }
\setbox\contprefix=\hbox{\tt \ \ \ \ \ \ \ \ \ \ \ }
\startm
\m{\vdash}\m{\lnot}\m{(}\m{(}\m{(}\m{\varphi}\m{\leftrightarrow}\m{\psi}\m{)}%
\m{\rightarrow}\m{\lnot}\m{(}\m{(}\m{\varphi}\m{\rightarrow}\m{\psi}\m{)}\m{%
\rightarrow}\m{\lnot}\m{(}\m{\psi}\m{\rightarrow}\m{\varphi}\m{)}\m{)}\m{)}\m{%
\rightarrow}\m{\lnot}\m{(}\m{\lnot}\m{(}\m{(}\m{\varphi}\m{\rightarrow}\m{%
\psi}\m{)}\m{\rightarrow}\m{\lnot}\m{(}\m{\psi}\m{\rightarrow}\m{\varphi}\m{)}%
\m{)}\m{\rightarrow}\m{(}\m{\varphi}\m{\leftrightarrow}\m{\psi}\m{)}\m{)}\m{)}
\endm
\begin{mmraw}%
|- -. ( ( ( ph <-> ps ) -> -. ( ( ph -> ps ) ->
-. ( ps -> ph ) ) ) -> -. ( -. ( ( ph -> ps ) -> -. (
ps -> ph ) ) -> ( ph <-> ps ) ) ) \$.
\end{mmraw}

\noindent This theorem relates the biconditional connective to primitive
connectives and can be used to eliminate the $\leftrightarrow$ symbol from any
wff.

\vskip 0.5ex
\setbox\startprefix=\hbox{\tt \ \ bii\ \$p\ }
\setbox\contprefix=\hbox{\tt \ \ \ \ \ \ \ \ \ }
\startm
\m{\vdash}\m{(}\m{(}\m{\varphi}\m{\leftrightarrow}\m{\psi}\m{)}\m{\leftrightarrow}
\m{\lnot}\m{(}\m{(}\m{\varphi}\m{\rightarrow}\m{\psi}\m{)}\m{\rightarrow}\m{\lnot}
\m{(}\m{\psi}\m{\rightarrow}\m{\varphi}\m{)}\m{)}\m{)}
\endm
\begin{mmraw}%
|- ( ( ph <-> ps ) <-> -. ( ( ph -> ps ) -> -. ( ps -> ph ) ) ) \$= ... \$.
\end{mmraw}

\noindent Define disjunction ({\sc or}).\index{disjunction ($\vee$)}%
\index{logical or (vee)@logical {\sc or} ($\vee$)}%
\index{df-or@\texttt{df-or}}\label{df-or}

\vskip 0.5ex
\setbox\startprefix=\hbox{\tt \ \ df-or\ \$a\ }
\setbox\contprefix=\hbox{\tt \ \ \ \ \ \ \ \ \ \ \ }
\startm
\m{\vdash}\m{(}\m{(}\m{\varphi}\m{\vee}\m{\psi}\m{)}\m{\leftrightarrow}\m{(}\m{
\lnot}\m{\varphi}\m{\rightarrow}\m{\psi}\m{)}\m{)}
\endm
\begin{mmraw}%
|- ( ( ph \TOR ps ) <-> ( -. ph -> ps ) ) \$.
\end{mmraw}

\noindent Define conjunction ({\sc and}).\index{conjunction ($\wedge$)}%
\index{logical {\sc and} ($\wedge$)}%
\index{df-an@\texttt{df-an}}\label{df-an}

\vskip 0.5ex
\setbox\startprefix=\hbox{\tt \ \ df-an\ \$a\ }
\setbox\contprefix=\hbox{\tt \ \ \ \ \ \ \ \ \ \ \ }
\startm
\m{\vdash}\m{(}\m{(}\m{\varphi}\m{\wedge}\m{\psi}\m{)}\m{\leftrightarrow}\m{\lnot}
\m{(}\m{\varphi}\m{\rightarrow}\m{\lnot}\m{\psi}\m{)}\m{)}
\endm
\begin{mmraw}%
|- ( ( ph \TAND ps ) <-> -. ( ph -> -. ps ) ) \$.
\end{mmraw}

\noindent Define disjunction ({\sc or}) of 3 wffs.%
\index{df-3or@\texttt{df-3or}}\label{df-3or}

\vskip 0.5ex
\setbox\startprefix=\hbox{\tt \ \ df-3or\ \$a\ }
\setbox\contprefix=\hbox{\tt \ \ \ \ \ \ \ \ \ \ \ \ }
\startm
\m{\vdash}\m{(}\m{(}\m{\varphi}\m{\vee}\m{\psi}\m{\vee}\m{\chi}\m{)}\m{
\leftrightarrow}\m{(}\m{(}\m{\varphi}\m{\vee}\m{\psi}\m{)}\m{\vee}\m{\chi}\m{)}
\m{)}
\endm
\begin{mmraw}%
|- ( ( ph \TOR ps \TOR ch ) <-> ( ( ph \TOR ps ) \TOR ch ) ) \$.
\end{mmraw}

\noindent Define conjunction ({\sc and}) of 3 wffs.%
\index{df-3an}\label{df-3an}

\vskip 0.5ex
\setbox\startprefix=\hbox{\tt \ \ df-3an\ \$a\ }
\setbox\contprefix=\hbox{\tt \ \ \ \ \ \ \ \ \ \ \ \ }
\startm
\m{\vdash}\m{(}\m{(}\m{\varphi}\m{\wedge}\m{\psi}\m{\wedge}\m{\chi}\m{)}\m{
\leftrightarrow}\m{(}\m{(}\m{\varphi}\m{\wedge}\m{\psi}\m{)}\m{\wedge}\m{\chi}
\m{)}\m{)}
\endm

\begin{mmraw}%
|- ( ( ph \TAND ps \TAND ch ) <-> ( ( ph \TAND ps ) \TAND ch ) ) \$.
\end{mmraw}

\subsection{Definitions for Predicate Calculus}\label{metadefpred}

The symbols $x$, $y$, and $z$ represent individual variables of predicate
calculus.  In this section, they are not necessarily distinct unless it is
explicitly
mentioned.

\vskip 2ex
\noindent Define existential quantification.
The expression $\exists x \varphi$ means
``there exists an $x$ where $\varphi$ is true.''\index{existential quantifier
($\exists$)}\label{df-ex}

\vskip 0.5ex
\setbox\startprefix=\hbox{\tt \ \ df-ex\ \$a\ }
\setbox\contprefix=\hbox{\tt \ \ \ \ \ \ \ \ \ \ \ }
\startm
\m{\vdash}\m{(}\m{\exists}\m{x}\m{\varphi}\m{\leftrightarrow}\m{\lnot}\m{\forall}
\m{x}\m{\lnot}\m{\varphi}\m{)}
\endm
\begin{mmraw}%
|- ( E. x ph <-> -. A. x -. ph ) \$.
\end{mmraw}

\noindent Define proper substitution.\index{proper
substitution}\index{substitution!proper}\label{df-sb}
In our notation, we use $[ y / x ] \varphi$ to mean ``the wff that
results when $y$ is properly substituted for $x$ in the wff
$\varphi$.''\footnote{
This can also be described
as substituting $x$ with $y$, $y$ properly replaces $x$, or
$x$ is properly replaced by $y$.}
% This is elsb4, though it currently says: ( [ x / y ] z e. y <-> z e. x )
For example,
$[ y / x ] z \in x$ is the same as $z \in y$.
One way to remember this notation is to notice that it looks like division
and recall that $( y / x ) \cdot x $ is $y$ (when $x \neq 0$).
The notation is different from the notation $\varphi ( x | y )$
that is sometimes used, because the latter notation is ambiguous for us:
for example, we don't know whether $\lnot \varphi ( x | y )$ is to be
interpreted as $\lnot ( \varphi ( x | y ) )$ or
$( \lnot \varphi ) ( x | y )$.\footnote{Because of the way
we initially defined wffs, this is the case
with any postfix connective\index{postfix connective} (one occurring after the
symbols being connected) or infix connective\index{infix connective} (one
occurring between the symbols being connected).  Metamath does not have a
built-in notion of operator binding strength that could eliminate the
ambiguity.  The initial parenthesis effectively provides a prefix
connective\index{prefix connective} to eliminate ambiguity.  Some conventions,
such as Polish notation\index{Polish notation} used in the 1930's and 1940's
by Polish logicians, use only prefix connectives and thus allow the total
elimination of parentheses, at the expense of readability.  In Metamath we
could actually redefine all notation to be Polish if we wanted to without
having to change any proofs!}  Other texts often use $\varphi(y)$ to indicate
our $[ y / x ] \varphi$, but this notation is even more ambiguous since there is
no explicit indication of what is being substituted.
Note that this
definition is valid even when
$x$ and $y$ are the same variable.  The first conjunct is a ``trick'' used to
achieve this property, making the definition look somewhat peculiar at
first.

\vskip 0.5ex
\setbox\startprefix=\hbox{\tt \ \ df-sb\ \$a\ }
\setbox\contprefix=\hbox{\tt \ \ \ \ \ \ \ \ \ \ \ }
\startm
\m{\vdash}\m{(}\m{[}\m{y}\m{/}\m{x}\m{]}\m{\varphi}\m{\leftrightarrow}\m{(}%
\m{(}\m{x}\m{=}\m{y}\m{\rightarrow}\m{\varphi}\m{)}\m{\wedge}\m{\exists}\m{x}%
\m{(}\m{x}\m{=}\m{y}\m{\wedge}\m{\varphi}\m{)}\m{)}\m{)}
\endm
\begin{mmraw}%
|- ( [ y / x ] ph <-> ( ( x = y -> ph ) \TAND E. x ( x = y \TAND ph ) ) ) \$.
\end{mmraw}


\noindent Define existential uniqueness\index{existential uniqueness
quantifier ($\exists "!$)} (``there exists exactly one'').  Note that $y$ is a
variable distinct from $x$ and not occurring in $\varphi$.\label{df-eu}

\vskip 0.5ex
\setbox\startprefix=\hbox{\tt \ \ df-eu\ \$a\ }
\setbox\contprefix=\hbox{\tt \ \ \ \ \ \ \ \ \ \ \ }
\startm
\m{\vdash}\m{(}\m{\exists}\m{{!}}\m{x}\m{\varphi}\m{\leftrightarrow}\m{\exists}
\m{y}\m{\forall}\m{x}\m{(}\m{\varphi}\m{\leftrightarrow}\m{x}\m{=}\m{y}\m{)}\m{)}
\endm

\begin{mmraw}%
|- ( E! x ph <-> E. y A. x ( ph <-> x = y ) ) \$.
\end{mmraw}

\subsection{Definitions for Set Theory}\label{setdefinitions}

The symbols $x$, $y$, $z$, and $w$ represent individual variables of
predicate calculus, which in set theory are understood to be sets.
However, using only the constructs shown so far would be very inconvenient.

To make set theory more practical, we introduce the notion of a ``class.''
A class\index{class} is either a set variable (such as $x$) or an
expression of the form $\{ x | \varphi\}$ (called an ``abstraction
class''\index{abstraction class}\index{class abstraction}).  Note that
sets (i.e.\ individual variables) always exist (this is a theorem of
logic, namely $\exists y \, y = x$ for any set $x$), whereas classes may
or may not exist (i.e.\ $\exists y \, y = A$ may or may not be true).
If a class does not exist it is called a ``proper class.''\index{proper
class}\index{class!proper} Definitions \texttt{df-clab},
\texttt{df-cleq}, and \texttt{df-clel} can be used to convert an
expression containing classes into one containing only set variables and
wff metavariables.

The symbols $A$, $B$, $C$, $D$, $ F$, $G$, and $R$ are metavariables that range
over classes.  A class metavariable $A$ may be eliminated from a wff by
replacing it with $\{ x|\varphi\}$ where neither $x$ nor $\varphi$ occur in
the wff.

The theory of classes can be shown to be an eliminable and conservative
extension of set theory. The \textbf{eliminability}
property shows that for every
formula in the extended language we can build a logically equivalent
formula in the basic language; so that even if the extended language
provides more ease to convey and formulate mathematical ideas for set
theory, its expressive power does not in fact strengthen the basic
language's expressive power.
The \textbf{conservation} property shows that for
every proof of a formula of the basic language in the extended system
we can build another proof of the same formula in the basic system;
so that, concerning theorems on sets only, the deductive powers of
the extended system and of the basic system are identical. Together,
these properties mean that the extended language can be treated as a
definitional extension that is \textbf{sound}.

A rigorous justification, which we will not give here, can be found in
Levy \cite[pp.~357-366]{Levy} supplementing his informal introduction to class
theory on pp.~7-17. Two other good treatments of class theory are provided
by Quine \cite[pp.~15-21]{Quine}\index{Quine, Willard Van Orman}
and also \cite[pp.~10-14]{Takeuti}\index{Takeuti, Gaisi}.
Quine's exposition (he calls them virtual classes)
is nicely written and very readable.

In the rest of this
section, individual variables are always assumed to be distinct from
each other unless otherwise indicated.  In addition, dummy variables on the
right-hand side of a definition do not occur in the class and wff
metavariables in the definition.

The definitions we present here are a partial but self-contained
collection selected from several hundred that appear in the current
\texttt{set.mm} database.  They are adequate for a basic development of
elementary set theory.

\vskip 2ex
\noindent Define the abstraction class.\index{abstraction class}\index{class
abstraction}\label{df-clab}  $x$ and $y$
need not be distinct.  Definition 2.1 of Quine, p.~16.  This definition may
seem puzzling since it is shorter than the expression being defined and does not
buy us anything in terms of brevity.  The reason we introduce this definition
is because it fits in neatly with the extension of the $\in$ connective
provided by \texttt{df-clel}.

\vskip 0.5ex
\setbox\startprefix=\hbox{\tt \ \ df-clab\ \$a\ }
\setbox\contprefix=\hbox{\tt \ \ \ \ \ \ \ \ \ \ \ \ \ }
\startm
\m{\vdash}\m{(}\m{x}\m{\in}\m{\{}\m{y}\m{|}\m{\varphi}\m{\}}\m{%
\leftrightarrow}\m{[}\m{x}\m{/}\m{y}\m{]}\m{\varphi}\m{)}
\endm
\begin{mmraw}%
|- ( x e. \{ y | ph \} <-> [ x / y ] ph ) \$.
\end{mmraw}

\noindent Define the equality connective between classes\index{class
equality}\label{df-cleq}.  See Quine or Chapter 4 of Takeuti and Zaring for its
justification and methods for eliminating it.  This is an example of a
somewhat ``dangerous'' definition, because it extends the use of the
existing equality symbol rather than introducing a new symbol, allowing
us to make statements in the original language that may not be true.
For example, it permits us to deduce $y = z \leftrightarrow \forall x (
x \in y \leftrightarrow x \in z )$ which is not a theorem of logic but
rather presupposes the Axiom of Extensionality,\index{Axiom of
Extensionality} which we include as a hypothesis so that we can know
when this axiom is assumed in a proof (with the \texttt{show
trace{\char`\_}back} command).  We could avoid the danger by introducing
another symbol, say $\eqcirc$, in place of $=$; this
would also have the advantage of making elimination of the definition
straightforward and would eliminate the need for Extensionality as a
hypothesis.  We would then also have the advantage of being able to
identify exactly where Extensionality truly comes into play.  One of our
theorems would be $x \eqcirc y \leftrightarrow x = y$
by invoking Extensionality.  However in keeping with standard practice
we retain the ``dangerous'' definition.

\vskip 0.5ex
\setbox\startprefix=\hbox{\tt \ \ df-cleq.1\ \$e\ }
\setbox\contprefix=\hbox{\tt \ \ \ \ \ \ \ \ \ \ \ \ \ \ \ }
\startm
\m{\vdash}\m{(}\m{\forall}\m{x}\m{(}\m{x}\m{\in}\m{y}\m{\leftrightarrow}\m{x}
\m{\in}\m{z}\m{)}\m{\rightarrow}\m{y}\m{=}\m{z}\m{)}
\endm
\setbox\startprefix=\hbox{\tt \ \ df-cleq\ \$a\ }
\setbox\contprefix=\hbox{\tt \ \ \ \ \ \ \ \ \ \ \ \ \ }
\startm
\m{\vdash}\m{(}\m{A}\m{=}\m{B}\m{\leftrightarrow}\m{\forall}\m{x}\m{(}\m{x}\m{
\in}\m{A}\m{\leftrightarrow}\m{x}\m{\in}\m{B}\m{)}\m{)}
\endm
% We need to reset the startprefix and contprefix.
\setbox\startprefix=\hbox{\tt \ \ df-cleq.1\ \$e\ }
\setbox\contprefix=\hbox{\tt \ \ \ \ \ \ \ \ \ \ \ \ \ \ \ }
\begin{mmraw}%
|- ( A. x ( x e. y <-> x e. z ) -> y = z ) \$.
\end{mmraw}
\setbox\startprefix=\hbox{\tt \ \ df-cleq\ \$a\ }
\setbox\contprefix=\hbox{\tt \ \ \ \ \ \ \ \ \ \ \ \ \ }
\begin{mmraw}%
|- ( A = B <-> A. x ( x e. A <-> x e. B ) ) \$.
\end{mmraw}

\noindent Define the membership connective between classes\index{class
membership}.  Theorem 6.3 of Quine, p.~41, which we adopt as a definition.
Note that it extends the use of the existing membership symbol, but unlike
{\tt df-cleq} it does not extend the set of valid wffs of logic when the class
metavariables are replaced with set variables.\label{dfclel}\label{df-clel}

\vskip 0.5ex
\setbox\startprefix=\hbox{\tt \ \ df-clel\ \$a\ }
\setbox\contprefix=\hbox{\tt \ \ \ \ \ \ \ \ \ \ \ \ \ }
\startm
\m{\vdash}\m{(}\m{A}\m{\in}\m{B}\m{\leftrightarrow}\m{\exists}\m{x}\m{(}\m{x}
\m{=}\m{A}\m{\wedge}\m{x}\m{\in}\m{B}\m{)}\m{)}
\endm
\begin{mmraw}%
|- ( A e. B <-> E. x ( x = A \TAND x e. B ) ) \$.?
\end{mmraw}

\noindent Define inequality.

\vskip 0.5ex
\setbox\startprefix=\hbox{\tt \ \ df-ne\ \$a\ }
\setbox\contprefix=\hbox{\tt \ \ \ \ \ \ \ \ \ \ \ }
\startm
\m{\vdash}\m{(}\m{A}\m{\ne}\m{B}\m{\leftrightarrow}\m{\lnot}\m{A}\m{=}\m{B}%
\m{)}
\endm
\begin{mmraw}%
|- ( A =/= B <-> -. A = B ) \$.
\end{mmraw}

\noindent Define restricted universal quantification.\index{universal
quantifier ($\forall$)!restricted}  Enderton, p.~22.

\vskip 0.5ex
\setbox\startprefix=\hbox{\tt \ \ df-ral\ \$a\ }
\setbox\contprefix=\hbox{\tt \ \ \ \ \ \ \ \ \ \ \ \ }
\startm
\m{\vdash}\m{(}\m{\forall}\m{x}\m{\in}\m{A}\m{\varphi}\m{\leftrightarrow}\m{%
\forall}\m{x}\m{(}\m{x}\m{\in}\m{A}\m{\rightarrow}\m{\varphi}\m{)}\m{)}
\endm
\begin{mmraw}%
|- ( A. x e. A ph <-> A. x ( x e. A -> ph ) ) \$.
\end{mmraw}

\noindent Define restricted existential quantification.\index{existential
quantifier ($\exists$)!restricted}  Enderton, p.~22.

\vskip 0.5ex
\setbox\startprefix=\hbox{\tt \ \ df-rex\ \$a\ }
\setbox\contprefix=\hbox{\tt \ \ \ \ \ \ \ \ \ \ \ \ }
\startm
\m{\vdash}\m{(}\m{\exists}\m{x}\m{\in}\m{A}\m{\varphi}\m{\leftrightarrow}\m{%
\exists}\m{x}\m{(}\m{x}\m{\in}\m{A}\m{\wedge}\m{\varphi}\m{)}\m{)}
\endm
\begin{mmraw}%
|- ( E. x e. A ph <-> E. x ( x e. A \TAND ph ) ) \$.
\end{mmraw}

\noindent Define the universal class\index{universal class ($V$)}.  Definition
5.20, p.~21, of Takeuti and Zaring.\label{df-v}

\vskip 0.5ex
\setbox\startprefix=\hbox{\tt \ \ df-v\ \$a\ }
\setbox\contprefix=\hbox{\tt \ \ \ \ \ \ \ \ \ \ }
\startm
\m{\vdash}\m{{\rm V}}\m{=}\m{\{}\m{x}\m{|}\m{x}\m{=}\m{x}\m{\}}
\endm
\begin{mmraw}%
|- {\char`\_}V = \{ x | x = x \} \$.
\end{mmraw}

\noindent Define the subclass\index{subclass}\index{subset} relationship
between two classes (called the subset relation if the classes are sets i.e.\
are not proper).  Definition 5.9 of Takeuti and Zaring, p.~17.\label{df-ss}

\vskip 0.5ex
\setbox\startprefix=\hbox{\tt \ \ df-ss\ \$a\ }
\setbox\contprefix=\hbox{\tt \ \ \ \ \ \ \ \ \ \ \ }
\startm
\m{\vdash}\m{(}\m{A}\m{\subseteq}\m{B}\m{\leftrightarrow}\m{\forall}\m{x}\m{(}
\m{x}\m{\in}\m{A}\m{\rightarrow}\m{x}\m{\in}\m{B}\m{)}\m{)}
\endm
\begin{mmraw}%
|- ( A C\_ B <-> A. x ( x e. A -> x e. B ) ) \$.
\end{mmraw}

\noindent Define the union\index{union} of two classes.  Definition 5.6 of Takeuti and Zaring,
p.~16.\label{df-un}

\vskip 0.5ex
\setbox\startprefix=\hbox{\tt \ \ df-un\ \$a\ }
\setbox\contprefix=\hbox{\tt \ \ \ \ \ \ \ \ \ \ \ }
\startm
\m{\vdash}\m{(}\m{A}\m{\cup}\m{B}\m{)}\m{=}\m{\{}\m{x}\m{|}\m{(}\m{x}\m{\in}
\m{A}\m{\vee}\m{x}\m{\in}\m{B}\m{)}\m{\}}
\endm
\begin{mmraw}%
( A u. B ) = \{ x | ( x e. A \TOR x e. B ) \} \$.
\end{mmraw}

\noindent Define the intersection\index{intersection} of two classes.  Definition 5.6 of
Takeuti and Zaring, p.~16.\label{df-in}

\vskip 0.5ex
\setbox\startprefix=\hbox{\tt \ \ df-in\ \$a\ }
\setbox\contprefix=\hbox{\tt \ \ \ \ \ \ \ \ \ \ \ }
\startm
\m{\vdash}\m{(}\m{A}\m{\cap}\m{B}\m{)}\m{=}\m{\{}\m{x}\m{|}\m{(}\m{x}\m{\in}
\m{A}\m{\wedge}\m{x}\m{\in}\m{B}\m{)}\m{\}}
\endm
% Caret ^ requires special treatment
\begin{mmraw}%
|- ( A i\^{}i B ) = \{ x | ( x e. A \TAND x e. B ) \} \$.
\end{mmraw}

\noindent Define class difference\index{class difference}\index{set difference}.
Definition 5.12 of Takeuti and Zaring, p.~20.  Several notations are used in
the literature; we chose the $\setminus$ convention instead of a minus sign to
reserve the latter for later use in, e.g., arithmetic.\label{df-dif}

\vskip 0.5ex
\setbox\startprefix=\hbox{\tt \ \ df-dif\ \$a\ }
\setbox\contprefix=\hbox{\tt \ \ \ \ \ \ \ \ \ \ \ \ }
\startm
\m{\vdash}\m{(}\m{A}\m{\setminus}\m{B}\m{)}\m{=}\m{\{}\m{x}\m{|}\m{(}\m{x}\m{
\in}\m{A}\m{\wedge}\m{\lnot}\m{x}\m{\in}\m{B}\m{)}\m{\}}
\endm
\begin{mmraw}%
( A \SLASH B ) = \{ x | ( x e. A \TAND -. x e. B ) \} \$.
\end{mmraw}

\noindent Define the empty or null set\index{empty set}\index{null set}.
Compare  Definition 5.14 of Takeuti and Zaring, p.~20.\label{df-nul}

\vskip 0.5ex
\setbox\startprefix=\hbox{\tt \ \ df-nul\ \$a\ }
\setbox\contprefix=\hbox{\tt \ \ \ \ \ \ \ \ \ \ }
\startm
\m{\vdash}\m{\varnothing}\m{=}\m{(}\m{{\rm V}}\m{\setminus}\m{{\rm V}}\m{)}
\endm
\begin{mmraw}%
|- (/) = ( {\char`\_}V \SLASH {\char`\_}V ) \$.
\end{mmraw}

\noindent Define power class\index{power set}\index{power class}.  Definition 5.10 of
Takeuti and Zaring, p.~17, but we also let it apply to proper classes.  (Note
that \verb$~P$ is the symbol for calligraphic P, the tilde
suggesting ``curly;'' see Appendix~\ref{ASCII}.)\label{df-pw}

\vskip 0.5ex
\setbox\startprefix=\hbox{\tt \ \ df-pw\ \$a\ }
\setbox\contprefix=\hbox{\tt \ \ \ \ \ \ \ \ \ \ \ }
\startm
\m{\vdash}\m{{\cal P}}\m{A}\m{=}\m{\{}\m{x}\m{|}\m{x}\m{\subseteq}\m{A}\m{\}}
\endm
% Special incantation required to put ~ into the text
\begin{mmraw}%
|- \char`\~P~A = \{ x | x C\_ A \} \$.
\end{mmraw}

\noindent Define the singleton of a class\index{singleton}.  Definition 7.1 of
Quine, p.~48.  It is well-defined for proper classes, although
it is not very meaningful in this case, where it evaluates to the empty
set.

\vskip 0.5ex
\setbox\startprefix=\hbox{\tt \ \ df-sn\ \$a\ }
\setbox\contprefix=\hbox{\tt \ \ \ \ \ \ \ \ \ \ \ }
\startm
\m{\vdash}\m{\{}\m{A}\m{\}}\m{=}\m{\{}\m{x}\m{|}\m{x}\m{=}\m{A}\m{\}}
\endm
\begin{mmraw}%
|- \{ A \} = \{ x | x = A \} \$.
\end{mmraw}%

\noindent Define an unordered pair of classes\index{unordered pair}\index{pair}.  Definition
7.1 of Quine, p.~48.

\vskip 0.5ex
\setbox\startprefix=\hbox{\tt \ \ df-pr\ \$a\ }
\setbox\contprefix=\hbox{\tt \ \ \ \ \ \ \ \ \ \ \ }
\startm
\m{\vdash}\m{\{}\m{A}\m{,}\m{B}\m{\}}\m{=}\m{(}\m{\{}\m{A}\m{\}}\m{\cup}\m{\{}
\m{B}\m{\}}\m{)}
\endm
\begin{mmraw}%
|- \{ A , B \} = ( \{ A \} u. \{ B \} ) \$.
\end{mmraw}

\noindent Define an unordered triple of classes\index{unordered triple}.  Definition of
Enderton, p.~19.

\vskip 0.5ex
\setbox\startprefix=\hbox{\tt \ \ df-tp\ \$a\ }
\setbox\contprefix=\hbox{\tt \ \ \ \ \ \ \ \ \ \ \ }
\startm
\m{\vdash}\m{\{}\m{A}\m{,}\m{B}\m{,}\m{C}\m{\}}\m{=}\m{(}\m{\{}\m{A}\m{,}\m{B}
\m{\}}\m{\cup}\m{\{}\m{C}\m{\}}\m{)}
\endm
\begin{mmraw}%
|- \{ A , B , C \} = ( \{ A , B \} u. \{ C \} ) \$.
\end{mmraw}%

\noindent Kuratowski's\index{Kuratowski, Kazimierz} ordered pair\index{ordered
pair} definition.  Definition 9.1 of Quine, p.~58. For proper classes it is
not meaningful but is well-defined for convenience.  (Note that \verb$<.$
stands for $\langle$ whereas \verb$<$ stands for $<$, and similarly for
\verb$>.$\,.)\label{df-op}

\vskip 0.5ex
\setbox\startprefix=\hbox{\tt \ \ df-op\ \$a\ }
\setbox\contprefix=\hbox{\tt \ \ \ \ \ \ \ \ \ \ \ }
\startm
\m{\vdash}\m{\langle}\m{A}\m{,}\m{B}\m{\rangle}\m{=}\m{\{}\m{\{}\m{A}\m{\}}
\m{,}\m{\{}\m{A}\m{,}\m{B}\m{\}}\m{\}}
\endm
\begin{mmraw}%
|- <. A , B >. = \{ \{ A \} , \{ A , B \} \} \$.
\end{mmraw}

\noindent Define the union of a class\index{union}.  Definition 5.5, p.~16,
of Takeuti and Zaring.

\vskip 0.5ex
\setbox\startprefix=\hbox{\tt \ \ df-uni\ \$a\ }
\setbox\contprefix=\hbox{\tt \ \ \ \ \ \ \ \ \ \ \ \ }
\startm
\m{\vdash}\m{\bigcup}\m{A}\m{=}\m{\{}\m{x}\m{|}\m{\exists}\m{y}\m{(}\m{x}\m{
\in}\m{y}\m{\wedge}\m{y}\m{\in}\m{A}\m{)}\m{\}}
\endm
\begin{mmraw}%
|- U. A = \{ x | E. y ( x e. y \TAND y e. A ) \} \$.
\end{mmraw}

\noindent Define the intersection\index{intersection} of a class.  Definition 7.35,
p.~44, of Takeuti and Zaring.

\vskip 0.5ex
\setbox\startprefix=\hbox{\tt \ \ df-int\ \$a\ }
\setbox\contprefix=\hbox{\tt \ \ \ \ \ \ \ \ \ \ \ \ }
\startm
\m{\vdash}\m{\bigcap}\m{A}\m{=}\m{\{}\m{x}\m{|}\m{\forall}\m{y}\m{(}\m{y}\m{
\in}\m{A}\m{\rightarrow}\m{x}\m{\in}\m{y}\m{)}\m{\}}
\endm
\begin{mmraw}%
|- |\^{}| A = \{ x | A. y ( y e. A -> x e. y ) \} \$.
\end{mmraw}

\noindent Define a transitive class\index{transitive class}\index{transitive
set}.  This should not be confused with a transitive relation, which is a different
concept.  Definition from p.~71 of Enderton, extended to classes.

\vskip 0.5ex
\setbox\startprefix=\hbox{\tt \ \ df-tr\ \$a\ }
\setbox\contprefix=\hbox{\tt \ \ \ \ \ \ \ \ \ \ \ }
\startm
\m{\vdash}\m{(}\m{\mbox{\rm Tr}}\m{A}\m{\leftrightarrow}\m{\bigcup}\m{A}\m{
\subseteq}\m{A}\m{)}
\endm
\begin{mmraw}%
|- ( Tr A <-> U. A C\_ A ) \$.
\end{mmraw}

\noindent Define a notation for a general binary relation\index{binary
relation}.  Definition 6.18, p.~29, of Takeuti and Zaring, generalized to
arbitrary classes.  This definition is well-defined, although not very
meaningful, when classes $A$ and/or $B$ are proper.\label{dfbr}  The lack of
parentheses (or any other connective) creates no ambiguity since we are defining
an atomic wff.

\vskip 0.5ex
\setbox\startprefix=\hbox{\tt \ \ df-br\ \$a\ }
\setbox\contprefix=\hbox{\tt \ \ \ \ \ \ \ \ \ \ \ }
\startm
\m{\vdash}\m{(}\m{A}\m{\,R}\m{\,B}\m{\leftrightarrow}\m{\langle}\m{A}\m{,}\m{B}
\m{\rangle}\m{\in}\m{R}\m{)}
\endm
\begin{mmraw}%
|- ( A R B <-> <. A , B >. e. R ) \$.
\end{mmraw}

\noindent Define an abstraction class of ordered pairs\index{abstraction
class!of ordered
pairs}.  A special case of Definition 4.16, p.~14, of Takeuti and Zaring.
Note that $ z $ must be distinct from $ x $ and $ y $,
and $ z $ must not occur in $\varphi$, but $ x $ and $ y $ may be
identical and may appear in $\varphi$.

\vskip 0.5ex
\setbox\startprefix=\hbox{\tt \ \ df-opab\ \$a\ }
\setbox\contprefix=\hbox{\tt \ \ \ \ \ \ \ \ \ \ \ \ \ }
\startm
\m{\vdash}\m{\{}\m{\langle}\m{x}\m{,}\m{y}\m{\rangle}\m{|}\m{\varphi}\m{\}}\m{=}
\m{\{}\m{z}\m{|}\m{\exists}\m{x}\m{\exists}\m{y}\m{(}\m{z}\m{=}\m{\langle}\m{x}
\m{,}\m{y}\m{\rangle}\m{\wedge}\m{\varphi}\m{)}\m{\}}
\endm

\begin{mmraw}%
|- \{ <. x , y >. | ph \} = \{ z | E. x E. y ( z =
<. x , y >. /\ ph ) \} \$.
\end{mmraw}

\noindent Define the epsilon relation\index{epsilon relation}.  Similar to Definition
6.22, p.~30, of Takeuti and Zaring.

\vskip 0.5ex
\setbox\startprefix=\hbox{\tt \ \ df-eprel\ \$a\ }
\setbox\contprefix=\hbox{\tt \ \ \ \ \ \ \ \ \ \ \ \ \ \ }
\startm
\m{\vdash}\m{{\rm E}}\m{=}\m{\{}\m{\langle}\m{x}\m{,}\m{y}\m{\rangle}\m{|}\m{x}\m{
\in}\m{y}\m{\}}
\endm
\begin{mmraw}%
|- \_E = \{ <. x , y >. | x e. y \} \$.
\end{mmraw}

\noindent Define a founded relation\index{founded relation}.  $R$ is a founded
relation on $A$ iff\index{iff} (if and only if) each nonempty subset of $A$
has an ``$R$-minimal element.''  Similar to Definition 6.21, p.~30, of
Takeuti and Zaring.

\vskip 0.5ex
\setbox\startprefix=\hbox{\tt \ \ df-fr\ \$a\ }
\setbox\contprefix=\hbox{\tt \ \ \ \ \ \ \ \ \ \ \ }
\startm
\m{\vdash}\m{(}\m{R}\m{\,\mbox{\rm Fr}}\m{\,A}\m{\leftrightarrow}\m{\forall}\m{x}
\m{(}\m{(}\m{x}\m{\subseteq}\m{A}\m{\wedge}\m{\lnot}\m{x}\m{=}\m{\varnothing}
\m{)}\m{\rightarrow}\m{\exists}\m{y}\m{(}\m{y}\m{\in}\m{x}\m{\wedge}\m{(}\m{x}
\m{\cap}\m{\{}\m{z}\m{|}\m{z}\m{\,R}\m{\,y}\m{\}}\m{)}\m{=}\m{\varnothing}\m{)}
\m{)}\m{)}
\endm
\begin{mmraw}%
|- ( R Fr A <-> A. x ( ( x C\_ A \TAND -. x = (/) ) ->
E. y ( y e. x \TAND ( x i\^{}i \{ z | z R y \} ) = (/) ) ) ) \$.
\end{mmraw}

\noindent Define a well-ordering\index{well-ordering}.  $R$ is a well-ordering of $A$ iff
it is founded on $A$ and the elements of $A$ are pairwise $R$-comparable.
Similar to Definition 6.24(2), p.~30, of Takeuti and Zaring.

\vskip 0.5ex
\setbox\startprefix=\hbox{\tt \ \ df-we\ \$a\ }
\setbox\contprefix=\hbox{\tt \ \ \ \ \ \ \ \ \ \ \ }
\startm
\m{\vdash}\m{(}\m{R}\m{\,\mbox{\rm We}}\m{\,A}\m{\leftrightarrow}\m{(}\m{R}\m{\,
\mbox{\rm Fr}}\m{\,A}\m{\wedge}\m{\forall}\m{x}\m{\forall}\m{y}\m{(}\m{(}\m{x}\m{
\in}\m{A}\m{\wedge}\m{y}\m{\in}\m{A}\m{)}\m{\rightarrow}\m{(}\m{x}\m{\,R}\m{\,y}
\m{\vee}\m{x}\m{=}\m{y}\m{\vee}\m{y}\m{\,R}\m{\,x}\m{)}\m{)}\m{)}\m{)}
\endm
\begin{mmraw}%
( R We A <-> ( R Fr A \TAND A. x A. y ( ( x e.
A \TAND y e. A ) -> ( x R y \TOR x = y \TOR y R x ) ) ) ) \$.
\end{mmraw}

\noindent Define the ordinal predicate\index{ordinal predicate}, which is true for a class
that is transitive and is well-ordered by the epsilon relation.  Similar to
definition on p.~468, Bell and Machover.

\vskip 0.5ex
\setbox\startprefix=\hbox{\tt \ \ df-ord\ \$a\ }
\setbox\contprefix=\hbox{\tt \ \ \ \ \ \ \ \ \ \ \ \ }
\startm
\m{\vdash}\m{(}\m{\mbox{\rm Ord}}\m{\,A}\m{\leftrightarrow}\m{(}
\m{\mbox{\rm Tr}}\m{\,A}\m{\wedge}\m{E}\m{\,\mbox{\rm We}}\m{\,A}\m{)}\m{)}
\endm
\begin{mmraw}%
|- ( Ord A <-> ( Tr A \TAND E We A ) ) \$.
\end{mmraw}

\noindent Define the class of all ordinal numbers\index{ordinal number}.  An ordinal number is
a set that satisfies the ordinal predicate.  Definition 7.11 of Takeuti and
Zaring, p.~38.

\vskip 0.5ex
\setbox\startprefix=\hbox{\tt \ \ df-on\ \$a\ }
\setbox\contprefix=\hbox{\tt \ \ \ \ \ \ \ \ \ \ \ }
\startm
\m{\vdash}\m{\,\mbox{\rm On}}\m{=}\m{\{}\m{x}\m{|}\m{\mbox{\rm Ord}}\m{\,x}
\m{\}}
\endm
\begin{mmraw}%
|- On = \{ x | Ord x \} \$.
\end{mmraw}

\noindent Define the limit ordinal predicate\index{limit ordinal}, which is true for a
non-empty ordinal that is not a successor (i.e.\ that is the union of itself).
Compare Bell and Machover, p.~471 and Exercise (1), p.~42 of Takeuti and
Zaring.

\vskip 0.5ex
\setbox\startprefix=\hbox{\tt \ \ df-lim\ \$a\ }
\setbox\contprefix=\hbox{\tt \ \ \ \ \ \ \ \ \ \ \ \ }
\startm
\m{\vdash}\m{(}\m{\mbox{\rm Lim}}\m{\,A}\m{\leftrightarrow}\m{(}\m{\mbox{
\rm Ord}}\m{\,A}\m{\wedge}\m{\lnot}\m{A}\m{=}\m{\varnothing}\m{\wedge}\m{A}
\m{=}\m{\bigcup}\m{A}\m{)}\m{)}
\endm
\begin{mmraw}%
|- ( Lim A <-> ( Ord A \TAND -. A = (/) \TAND A = U. A ) ) \$.
\end{mmraw}

\noindent Define the successor\index{successor} of a class.  Definition 7.22 of Takeuti
and Zaring, p.~41.  Our definition is a generalization to classes, although it
is meaningless when classes are proper.

\vskip 0.5ex
\setbox\startprefix=\hbox{\tt \ \ df-suc\ \$a\ }
\setbox\contprefix=\hbox{\tt \ \ \ \ \ \ \ \ \ \ \ \ }
\startm
\m{\vdash}\m{\,\mbox{\rm suc}}\m{\,A}\m{=}\m{(}\m{A}\m{\cup}\m{\{}\m{A}\m{\}}
\m{)}
\endm
\begin{mmraw}%
|- suc A = ( A u. \{ A \} ) \$.
\end{mmraw}

\noindent Define the class of natural numbers\index{natural number}\index{omega
($\omega$)}.  Compare Bell and Machover, p.~471.\label{dfom}

\vskip 0.5ex
\setbox\startprefix=\hbox{\tt \ \ df-om\ \$a\ }
\setbox\contprefix=\hbox{\tt \ \ \ \ \ \ \ \ \ \ \ }
\startm
\m{\vdash}\m{\omega}\m{=}\m{\{}\m{x}\m{|}\m{(}\m{\mbox{\rm Ord}}\m{\,x}\m{
\wedge}\m{\forall}\m{y}\m{(}\m{\mbox{\rm Lim}}\m{\,y}\m{\rightarrow}\m{x}\m{
\in}\m{y}\m{)}\m{)}\m{\}}
\endm
\begin{mmraw}%
|- om = \{ x | ( Ord x \TAND A. y ( Lim y -> x e. y ) ) \} \$.
\end{mmraw}

\noindent Define the Cartesian product (also called the
cross product)\index{Cartesian product}\index{cross product}
of two classes.  Definition 9.11 of Quine, p.~64.

\vskip 0.5ex
\setbox\startprefix=\hbox{\tt \ \ df-xp\ \$a\ }
\setbox\contprefix=\hbox{\tt \ \ \ \ \ \ \ \ \ \ \ }
\startm
\m{\vdash}\m{(}\m{A}\m{\times}\m{B}\m{)}\m{=}\m{\{}\m{\langle}\m{x}\m{,}\m{y}
\m{\rangle}\m{|}\m{(}\m{x}\m{\in}\m{A}\m{\wedge}\m{y}\m{\in}\m{B}\m{)}\m{\}}
\endm
\begin{mmraw}%
|- ( A X. B ) = \{ <. x , y >. | ( x e. A \TAND y e. B) \} \$.
\end{mmraw}

\noindent Define a relation\index{relation}.  Definition 6.4(1) of Takeuti and
Zaring, p.~23.

\vskip 0.5ex
\setbox\startprefix=\hbox{\tt \ \ df-rel\ \$a\ }
\setbox\contprefix=\hbox{\tt \ \ \ \ \ \ \ \ \ \ \ \ }
\startm
\m{\vdash}\m{(}\m{\mbox{\rm Rel}}\m{\,A}\m{\leftrightarrow}\m{A}\m{\subseteq}
\m{(}\m{{\rm V}}\m{\times}\m{{\rm V}}\m{)}\m{)}
\endm
\begin{mmraw}%
|- ( Rel A <-> A C\_ ( {\char`\_}V X. {\char`\_}V ) ) \$.
\end{mmraw}

\noindent Define the domain\index{domain} of a class.  Definition 6.5(1) of
Takeuti and Zaring, p.~24.

\vskip 0.5ex
\setbox\startprefix=\hbox{\tt \ \ df-dm\ \$a\ }
\setbox\contprefix=\hbox{\tt \ \ \ \ \ \ \ \ \ \ \ }
\startm
\m{\vdash}\m{\,\mbox{\rm dom}}\m{A}\m{=}\m{\{}\m{x}\m{|}\m{\exists}\m{y}\m{
\langle}\m{x}\m{,}\m{y}\m{\rangle}\m{\in}\m{A}\m{\}}
\endm
\begin{mmraw}%
|- dom A = \{ x | E. y <. x , y >. e. A \} \$.
\end{mmraw}

\noindent Define the range\index{range} of a class.  Definition 6.5(2) of
Takeuti and Zaring, p.~24.

\vskip 0.5ex
\setbox\startprefix=\hbox{\tt \ \ df-rn\ \$a\ }
\setbox\contprefix=\hbox{\tt \ \ \ \ \ \ \ \ \ \ \ }
\startm
\m{\vdash}\m{\,\mbox{\rm ran}}\m{A}\m{=}\m{\{}\m{y}\m{|}\m{\exists}\m{x}\m{
\langle}\m{x}\m{,}\m{y}\m{\rangle}\m{\in}\m{A}\m{\}}
\endm
\begin{mmraw}%
|- ran A = \{ y | E. x <. x , y >. e. A \} \$.
\end{mmraw}

\noindent Define the restriction\index{restriction} of a class.  Definition
6.6(1) of Takeuti and Zaring, p.~24.

\vskip 0.5ex
\setbox\startprefix=\hbox{\tt \ \ df-res\ \$a\ }
\setbox\contprefix=\hbox{\tt \ \ \ \ \ \ \ \ \ \ \ \ }
\startm
\m{\vdash}\m{(}\m{A}\m{\restriction}\m{B}\m{)}\m{=}\m{(}\m{A}\m{\cap}\m{(}\m{B}
\m{\times}\m{{\rm V}}\m{)}\m{)}
\endm
\begin{mmraw}%
|- ( A |` B ) = ( A i\^{}i ( B X. {\char`\_}V ) ) \$.
\end{mmraw}

\noindent Define the image\index{image} of a class.  Definition 6.6(2) of
Takeuti and Zaring, p.~24.

\vskip 0.5ex
\setbox\startprefix=\hbox{\tt \ \ df-ima\ \$a\ }
\setbox\contprefix=\hbox{\tt \ \ \ \ \ \ \ \ \ \ \ \ }
\startm
\m{\vdash}\m{(}\m{A}\m{``}\m{B}\m{)}\m{=}\m{\,\mbox{\rm ran}}\m{\,(}\m{A}\m{
\restriction}\m{B}\m{)}
\endm
\begin{mmraw}%
|- ( A " B ) = ran ( A |` B ) \$.
\end{mmraw}

\noindent Define the composition\index{composition} of two classes.  Definition 6.6(3) of
Takeuti and Zaring, p.~24.

\vskip 0.5ex
\setbox\startprefix=\hbox{\tt \ \ df-co\ \$a\ }
\setbox\contprefix=\hbox{\tt \ \ \ \ \ \ \ \ \ \ \ \ }
\startm
\m{\vdash}\m{(}\m{A}\m{\circ}\m{B}\m{)}\m{=}\m{\{}\m{\langle}\m{x}\m{,}\m{y}\m{
\rangle}\m{|}\m{\exists}\m{z}\m{(}\m{\langle}\m{x}\m{,}\m{z}\m{\rangle}\m{\in}
\m{B}\m{\wedge}\m{\langle}\m{z}\m{,}\m{y}\m{\rangle}\m{\in}\m{A}\m{)}\m{\}}
\endm
\begin{mmraw}%
|- ( A o. B ) = \{ <. x , y >. | E. z ( <. x , z
>. e. B \TAND <. z , y >. e. A ) \} \$.
\end{mmraw}

\noindent Define a function\index{function}.  Definition 6.4(4) of Takeuti and
Zaring, p.~24.

\vskip 0.5ex
\setbox\startprefix=\hbox{\tt \ \ df-fun\ \$a\ }
\setbox\contprefix=\hbox{\tt \ \ \ \ \ \ \ \ \ \ \ \ }
\startm
\m{\vdash}\m{(}\m{\mbox{\rm Fun}}\m{\,A}\m{\leftrightarrow}\m{(}
\m{\mbox{\rm Rel}}\m{\,A}\m{\wedge}
\m{\forall}\m{x}\m{\exists}\m{z}\m{\forall}\m{y}\m{(}
\m{\langle}\m{x}\m{,}\m{y}\m{\rangle}\m{\in}\m{A}\m{\rightarrow}\m{y}\m{=}\m{z}
\m{)}\m{)}\m{)}
\endm
\begin{mmraw}%
|- ( Fun A <-> ( Rel A /\ A. x E. z A. y ( <. x
   , y >. e. A -> y = z ) ) ) \$.
\end{mmraw}

\noindent Define a function with domain.  Definition 6.15(1) of Takeuti and
Zaring, p.~27.

\vskip 0.5ex
\setbox\startprefix=\hbox{\tt \ \ df-fn\ \$a\ }
\setbox\contprefix=\hbox{\tt \ \ \ \ \ \ \ \ \ \ \ }
\startm
\m{\vdash}\m{(}\m{A}\m{\,\mbox{\rm Fn}}\m{\,B}\m{\leftrightarrow}\m{(}
\m{\mbox{\rm Fun}}\m{\,A}\m{\wedge}\m{\mbox{\rm dom}}\m{\,A}\m{=}\m{B}\m{)}
\m{)}
\endm
\begin{mmraw}%
|- ( A Fn B <-> ( Fun A \TAND dom A = B ) ) \$.
\end{mmraw}

\noindent Define a function with domain and co-domain.  Definition 6.15(3)
of Takeuti and Zaring, p.~27.

\vskip 0.5ex
\setbox\startprefix=\hbox{\tt \ \ df-f\ \$a\ }
\setbox\contprefix=\hbox{\tt \ \ \ \ \ \ \ \ \ \ }
\startm
\m{\vdash}\m{(}\m{F}\m{:}\m{A}\m{\longrightarrow}\m{B}\m{
\leftrightarrow}\m{(}\m{F}\m{\,\mbox{\rm Fn}}\m{\,A}\m{\wedge}\m{
\mbox{\rm ran}}\m{\,F}\m{\subseteq}\m{B}\m{)}\m{)}
\endm
\begin{mmraw}%
|- ( F : A --> B <-> ( F Fn A \TAND ran F C\_ B ) ) \$.
\end{mmraw}

\noindent Define a one-to-one function\index{one-to-one function}.  Compare
Definition 6.15(5) of Takeuti and Zaring, p.~27.

\vskip 0.5ex
\setbox\startprefix=\hbox{\tt \ \ df-f1\ \$a\ }
\setbox\contprefix=\hbox{\tt \ \ \ \ \ \ \ \ \ \ \ }
\startm
\m{\vdash}\m{(}\m{F}\m{:}\m{A}\m{
\raisebox{.5ex}{${\textstyle{\:}_{\mbox{\footnotesize\rm
1\tt -\rm 1}}}\atop{\textstyle{
\longrightarrow}\atop{\textstyle{}^{\mbox{\footnotesize\rm {\ }}}}}$}
}\m{B}
\m{\leftrightarrow}\m{(}\m{F}\m{:}\m{A}\m{\longrightarrow}\m{B}
\m{\wedge}\m{\forall}\m{y}\m{\exists}\m{z}\m{\forall}\m{x}\m{(}\m{\langle}\m{x}
\m{,}\m{y}\m{\rangle}\m{\in}\m{F}\m{\rightarrow}\m{x}\m{=}\m{z}\m{)}\m{)}\m{)}
\endm
\begin{mmraw}%
|- ( F : A -1-1-> B <-> ( F : A --> B \TAND
   A. y E. z A. x ( <. x , y >. e. F -> x = z ) ) ) \$.
\end{mmraw}

\noindent Define an onto function\index{onto function}.  Definition 6.15(4) of Takeuti and
Zaring, p.~27.

\vskip 0.5ex
\setbox\startprefix=\hbox{\tt \ \ df-fo\ \$a\ }
\setbox\contprefix=\hbox{\tt \ \ \ \ \ \ \ \ \ \ \ }
\startm
\m{\vdash}\m{(}\m{F}\m{:}\m{A}\m{
\raisebox{.5ex}{${\textstyle{\:}_{\mbox{\footnotesize\rm
{\ }}}}\atop{\textstyle{
\longrightarrow}\atop{\textstyle{}^{\mbox{\footnotesize\rm onto}}}}$}
}\m{B}
\m{\leftrightarrow}\m{(}\m{F}\m{\,\mbox{\rm Fn}}\m{\,A}\m{\wedge}
\m{\mbox{\rm ran}}\m{\,F}\m{=}\m{B}\m{)}\m{)}
\endm
\begin{mmraw}%
|- ( F : A -onto-> B <-> ( F Fn A /\ ran F = B ) ) \$.
\end{mmraw}

\noindent Define a one-to-one, onto function.  Compare Definition 6.15(6) of
Takeuti and Zaring, p.~27.

\vskip 0.5ex
\setbox\startprefix=\hbox{\tt \ \ df-f1o\ \$a\ }
\setbox\contprefix=\hbox{\tt \ \ \ \ \ \ \ \ \ \ \ \ }
\startm
\m{\vdash}\m{(}\m{F}\m{:}\m{A}
\m{
\raisebox{.5ex}{${\textstyle{\:}_{\mbox{\footnotesize\rm
1\tt -\rm 1}}}\atop{\textstyle{
\longrightarrow}\atop{\textstyle{}^{\mbox{\footnotesize\rm onto}}}}$}
}
\m{B}
\m{\leftrightarrow}\m{(}\m{F}\m{:}\m{A}
\m{
\raisebox{.5ex}{${\textstyle{\:}_{\mbox{\footnotesize\rm
1\tt -\rm 1}}}\atop{\textstyle{
\longrightarrow}\atop{\textstyle{}^{\mbox{\footnotesize\rm {\ }}}}}$}
}
\m{B}\m{\wedge}\m{F}\m{:}\m{A}
\m{
\raisebox{.5ex}{${\textstyle{\:}_{\mbox{\footnotesize\rm
{\ }}}}\atop{\textstyle{
\longrightarrow}\atop{\textstyle{}^{\mbox{\footnotesize\rm onto}}}}$}
}
\m{B}\m{)}\m{)}
\endm
\begin{mmraw}%
|- ( F : A -1-1-onto-> B <-> ( F : A -1-1-> B? \TAND F : A -onto-> B ) ) \$.?
\end{mmraw}

\noindent Define the value of a function\index{function value}.  This
definition applies to any class and evaluates to the empty set when it is not
meaningful. Note that $ F`A$ means the same thing as the more familiar $ F(A)$
notation for a function's value at $A$.  The $ F`A$ notation is common in
formal set theory.\label{df-fv}

\vskip 0.5ex
\setbox\startprefix=\hbox{\tt \ \ df-fv\ \$a\ }
\setbox\contprefix=\hbox{\tt \ \ \ \ \ \ \ \ \ \ \ }
\startm
\m{\vdash}\m{(}\m{F}\m{`}\m{A}\m{)}\m{=}\m{\bigcup}\m{\{}\m{x}\m{|}\m{(}\m{F}%
\m{``}\m{\{}\m{A}\m{\}}\m{)}\m{=}\m{\{}\m{x}\m{\}}\m{\}}
\endm
\begin{mmraw}%
|- ( F ` A ) = U. \{ x | ( F " \{ A \} ) = \{ x \} \} \$.
\end{mmraw}

\noindent Define the result of an operation.\index{operation}  Here, $F$ is
     an operation on two
     values (such as $+$ for real numbers).   This is defined for proper
     classes $A$ and $B$ even though not meaningful in that case.  However,
     the definition can be meaningful when $F$ is a proper class.\label{dfopr}

\vskip 0.5ex
\setbox\startprefix=\hbox{\tt \ \ df-opr\ \$a\ }
\setbox\contprefix=\hbox{\tt \ \ \ \ \ \ \ \ \ \ \ \ }
\startm
\m{\vdash}\m{(}\m{A}\m{\,F}\m{\,B}\m{)}\m{=}\m{(}\m{F}\m{`}\m{\langle}\m{A}%
\m{,}\m{B}\m{\rangle}\m{)}
\endm
\begin{mmraw}%
|- ( A F B ) = ( F ` <. A , B >. ) \$.
\end{mmraw}

\section{Tricks of the Trade}\label{tricks}

In the \texttt{set.mm}\index{set theory database (\texttt{set.mm})} database our goal
was usually to conform to modern notation.  However in some cases the
relationship to standard textbook language may be obscured by several
unconventional devices we used to simplify the development and to take
advantage of the Metamath language.  In this section we will describe some
common conventions used in \texttt{set.mm}.

\begin{itemize}
\item
The turnstile\index{turnstile ({$\,\vdash$})} symbol, $\vdash$, meaning ``it
is provable that,'' is the first token of all assertions and hypotheses that
aren't syntax constructions.  This is a standard convention in logic.  (We
mentioned this earlier, but this symbol is bothersome to some people without a
logic background.  It has no deeper meaning but just provides us with a way to
distinguish syntax constructions from ordinary mathematical statements.)

\item
A hypothesis of the form

\vskip 1ex
\setbox\startprefix=\hbox{\tt \ \ \ \ \ \ \ \ \ \$e\ }
\setbox\contprefix=\hbox{\tt \ \ \ \ \ \ \ \ \ \ }
\startm
\m{\vdash}\m{(}\m{\varphi}\m{\rightarrow}\m{\forall}\m{x}\m{\varphi}\m{)}
\endm
\vskip 1ex

should be read ``assume variable $x$ is (effectively) not free in wff
$\varphi$.''\index{effectively not free}
Literally, this says ``assume it is provable that $\varphi \rightarrow \forall
x\, \varphi$.''  This device lets us avoid the complexities associated with
the standard treatment of free and bound variables.
%% Uncomment this when uncommenting section {formalspec} below
The footnote on p.~\pageref{effectivelybound} discusses this further.

\item
A statement of one of the forms

\vskip 1ex
\setbox\startprefix=\hbox{\tt \ \ \ \ \ \ \ \ \ \$a\ }
\setbox\contprefix=\hbox{\tt \ \ \ \ \ \ \ \ \ \ }
\startm
\m{\vdash}\m{(}\m{\lnot}\m{\forall}\m{x}\m{\,x}\m{=}\m{y}\m{\rightarrow}
\m{\ldots}\m{)}
\endm
\setbox\startprefix=\hbox{\tt \ \ \ \ \ \ \ \ \ \$p\ }
\setbox\contprefix=\hbox{\tt \ \ \ \ \ \ \ \ \ \ }
\startm
\m{\vdash}\m{(}\m{\lnot}\m{\forall}\m{x}\m{\,x}\m{=}\m{y}\m{\rightarrow}
\m{\ldots}\m{)}
\endm
\vskip 1ex

should be read ``if $x$ and $y$ are distinct variables, then...''  This
antecedent provides us with a technical device to avoid the need for the
\texttt{\$d} statement early in our development of predicate calculus,
permitting symbol manipulations to be as conceptually simple as those in
propositional calculus.  However, the \texttt{\$d} statement eventually
becomes a requirement, and after that this device is rarely used.

\item
The statement

\vskip 1ex
\setbox\startprefix=\hbox{\tt \ \ \ \ \ \ \ \ \ \$d\ }
\setbox\contprefix=\hbox{\tt \ \ \ \ \ \ \ \ \ \ }
\startm
\m{x}\m{\,y}
\endm
\vskip 1ex

should be read ``assume $x$ and $y$ are distinct variables.''

\item
The statement

\vskip 1ex
\setbox\startprefix=\hbox{\tt \ \ \ \ \ \ \ \ \ \$d\ }
\setbox\contprefix=\hbox{\tt \ \ \ \ \ \ \ \ \ \ }
\startm
\m{x}\m{\,\varphi}
\endm
\vskip 1ex

should be read ``assume $x$ does not occur in $\varphi$.''

\item
The statement

\vskip 1ex
\setbox\startprefix=\hbox{\tt \ \ \ \ \ \ \ \ \ \$d\ }
\setbox\contprefix=\hbox{\tt \ \ \ \ \ \ \ \ \ \ }
\startm
\m{x}\m{\,A}
\endm
\vskip 1ex

should be read ``assume variable $x$ does not occur in class $A$.''

\item
The restriction and hypothesis group

\vskip 1ex
\setbox\startprefix=\hbox{\tt \ \ \ \ \ \ \ \ \ \$d\ }
\setbox\contprefix=\hbox{\tt \ \ \ \ \ \ \ \ \ \ }
\startm
\m{x}\m{\,A}
\endm
\setbox\startprefix=\hbox{\tt \ \ \ \ \ \ \ \ \ \$d\ }
\setbox\contprefix=\hbox{\tt \ \ \ \ \ \ \ \ \ \ }
\startm
\m{x}\m{\,\psi}
\endm
\setbox\startprefix=\hbox{\tt \ \ \ \ \ \ \ \ \ \$e\ }
\setbox\contprefix=\hbox{\tt \ \ \ \ \ \ \ \ \ \ }
\startm
\m{\vdash}\m{(}\m{x}\m{=}\m{A}\m{\rightarrow}\m{(}\m{\varphi}\m{\leftrightarrow}
\m{\psi}\m{)}\m{)}
\endm
\vskip 1ex

is frequently used in place of explicit substitution, meaning ``assume
$\psi$ results from the proper substitution of $A$ for $x$ in
$\varphi$.''  Sometimes ``\texttt{\$e} $\vdash ( \psi \rightarrow
\forall x \, \psi )$'' is used instead of ``\texttt{\$d} $x\, \psi $,''
which requires only that $x$ be effectively not free in $\varphi$ but
not necessarily absent from it.  The use of implicit
substitution\index{substitution!implicit} is partly a matter of personal
style, although it may make proofs somewhat shorter than would be the
case with explicit substitution.

\item
The hypothesis


\vskip 1ex
\setbox\startprefix=\hbox{\tt \ \ \ \ \ \ \ \ \ \$e\ }
\setbox\contprefix=\hbox{\tt \ \ \ \ \ \ \ \ \ \ }
\startm
\m{\vdash}\m{A}\m{\in}\m{{\rm V}}
\endm
\vskip 1ex

should be read ``assume class $A$ is a set (i.e.\ exists).''
This is a convenient convention used by Quine.

\item
The restriction and hypothesis

\vskip 1ex
\setbox\startprefix=\hbox{\tt \ \ \ \ \ \ \ \ \ \$d\ }
\setbox\contprefix=\hbox{\tt \ \ \ \ \ \ \ \ \ \ }
\startm
\m{x}\m{\,y}
\endm
\setbox\startprefix=\hbox{\tt \ \ \ \ \ \ \ \ \ \$e\ }
\setbox\contprefix=\hbox{\tt \ \ \ \ \ \ \ \ \ \ }
\startm
\m{\vdash}\m{(}\m{y}\m{\in}\m{A}\m{\rightarrow}\m{\forall}\m{x}\m{\,y}
\m{\in}\m{A}\m{)}
\endm
\vskip 1ex

should be read ``assume variable $x$ is
(effectively) not free in class $A$.''

\end{itemize}

\section{A Theorem Sampler}\label{sometheorems}

In this section we list some of the more important theorems that are proved in
the \texttt{set.mm} database, and they illustrate the kinds of things that can be
done with Metamath.  While all of these facts are well-known results,
Metamath offers the advantage of easily allowing you to trace their
derivation back to axioms.  Our intent here is not to try to explain the
details or motivation; for this we refer you to the textbooks that are
mentioned in the descriptions.  (The \texttt{set.mm} file has bibliographic
references for the text references.)  Their proofs often embody important
concepts you may wish to explore with the Metamath program (see
Section~\ref{exploring}).  All the symbols that are used here are defined in
Section~\ref{hierarchy}.  For brevity we haven't included the \texttt{\$d}
restrictions or \texttt{\$f} hypotheses for these theorems; when you are
uncertain consult the \texttt{set.mm} database.

We start with \texttt{syl} (principle of the syllogism).
In \textit{Principia Mathematica}
Whitehead and Russell call this ``the principle of the
syllogism... because... the syllogism in Barbara is derived from them''
\cite[quote after Theorem *2.06 p.~101]{PM}.
Some authors call this law a ``hypothetical syllogism.''
At the time of this writing it is the most commonly referenced assertion in the
\texttt{set.mm} database.\footnote{
The Metamath program command \texttt{show usage}
shows the number of references.
}

\vskip 2ex
\noindent Theorem syl (principle of the syllogism)\index{Syllogism}%
\index{\texttt{syl}}\label{syl}.

\vskip 0.5ex
\setbox\startprefix=\hbox{\tt \ \ syl.1\ \$e\ }
\setbox\contprefix=\hbox{\tt \ \ \ \ \ \ \ \ \ \ \ }
\startm
\m{\vdash}\m{(}\m{\varphi}\m{ \rightarrow }\m{\psi}\m{)}
\endm
\setbox\startprefix=\hbox{\tt \ \ syl.2\ \$e\ }
\setbox\contprefix=\hbox{\tt \ \ \ \ \ \ \ \ \ \ \ }
\startm
\m{\vdash}\m{(}\m{\psi}\m{ \rightarrow }\m{\chi}\m{)}
\endm
\setbox\startprefix=\hbox{\tt \ \ syl\ \$p\ }
\setbox\contprefix=\hbox{\tt \ \ \ \ \ \ \ \ \ }
\startm
\m{\vdash}\m{(}\m{\varphi}\m{ \rightarrow }\m{\chi}\m{)}
\endm
\vskip 2ex

The following theorem is not very deep but provides us with a notational device
that is frequently used.  It allows us to use the expression ``$A \in V$'' as
a compact way of saying that class $A$ exists, i.e.\ is a set.

\vskip 2ex
\noindent Two ways to say ``$A$ is a set'':  $A$ is a member of the universe
$V$ if and only if $A$ exists (i.e.\ there exists a set equal to $A$).
Theorem 6.9 of Quine, p. 43.

\vskip 0.5ex
\setbox\startprefix=\hbox{\tt \ \ isset\ \$p\ }
\setbox\contprefix=\hbox{\tt \ \ \ \ \ \ \ \ \ \ \ }
\startm
\m{\vdash}\m{(}\m{A}\m{\in}\m{{\rm V}}\m{\leftrightarrow}\m{\exists}\m{x}\m{\,x}\m{=}
\m{A}\m{)}
\endm
\vskip 1ex

Next we prove the axioms of standard ZF set theory that were missing from our
axiom system.  From our point of view they are theorems since they
can be derived from the other axioms.

\vskip 2ex
\noindent Axiom of Separation\index{Axiom of Separation}
(Aussonderung)\index{Aussonderung} proved from the other axioms of ZF set
theory.  Compare Exercise 4 of Takeuti and Zaring, p.~22.

\vskip 0.5ex
\setbox\startprefix=\hbox{\tt \ \ inex1.1\ \$e\ }
\setbox\contprefix=\hbox{\tt \ \ \ \ \ \ \ \ \ \ \ \ \ \ \ }
\startm
\m{\vdash}\m{A}\m{\in}\m{{\rm V}}
\endm
\setbox\startprefix=\hbox{\tt \ \ inex\ \$p\ }
\setbox\contprefix=\hbox{\tt \ \ \ \ \ \ \ \ \ \ \ \ \ }
\startm
\m{\vdash}\m{(}\m{A}\m{\cap}\m{B}\m{)}\m{\in}\m{{\rm V}}
\endm
\vskip 1ex

\noindent Axiom of the Null Set\index{Axiom of the Null Set} proved from the
other axioms of ZF set theory. Corollary 5.16 of Takeuti and Zaring, p.~20.

\vskip 0.5ex
\setbox\startprefix=\hbox{\tt \ \ 0ex\ \$p\ }
\setbox\contprefix=\hbox{\tt \ \ \ \ \ \ \ \ \ \ \ \ }
\startm
\m{\vdash}\m{\varnothing}\m{\in}\m{{\rm V}}
\endm
\vskip 1ex

\noindent The Axiom of Pairing\index{Axiom of Pairing} proved from the other
axioms of ZF set theory.  Theorem 7.13 of Quine, p.~51.
\vskip 0.5ex
\setbox\startprefix=\hbox{\tt \ \ prex\ \$p\ }
\setbox\contprefix=\hbox{\tt \ \ \ \ \ \ \ \ \ \ \ \ \ \ }
\startm
\m{\vdash}\m{\{}\m{A}\m{,}\m{B}\m{\}}\m{\in}\m{{\rm V}}
\endm
\vskip 2ex

Next we will list some famous or important theorems that are proved in
the \texttt{set.mm} database.  None of them except \texttt{omex}
require the Axiom of Infinity, as you can verify with the \texttt{show
trace{\char`\_}back} Metamath command.

\vskip 2ex
\noindent The resolution of Russell's paradox\index{Russell's paradox}.  There
exists no set containing the set of all sets which are not members of
themselves.  Proposition 4.14 of Takeuti and Zaring, p.~14.

\vskip 0.5ex
\setbox\startprefix=\hbox{\tt \ \ ru\ \$p\ }
\setbox\contprefix=\hbox{\tt \ \ \ \ \ \ \ \ }
\startm
\m{\vdash}\m{\lnot}\m{\exists}\m{x}\m{\,x}\m{=}\m{\{}\m{y}\m{|}\m{\lnot}\m{y}
\m{\in}\m{y}\m{\}}
\endm
\vskip 1ex

\noindent Cantor's theorem\index{Cantor's theorem}.  No set can be mapped onto
its power set.  Compare Theorem 6B(b) of Enderton, p.~132.

\vskip 0.5ex
\setbox\startprefix=\hbox{\tt \ \ canth.1\ \$e\ }
\setbox\contprefix=\hbox{\tt \ \ \ \ \ \ \ \ \ \ \ \ \ }
\startm
\m{\vdash}\m{A}\m{\in}\m{{\rm V}}
\endm
\setbox\startprefix=\hbox{\tt \ \ canth\ \$p\ }
\setbox\contprefix=\hbox{\tt \ \ \ \ \ \ \ \ \ \ \ }
\startm
\m{\vdash}\m{\lnot}\m{F}\m{:}\m{A}\m{\raisebox{.5ex}{${\textstyle{\:}_{
\mbox{\footnotesize\rm {\ }}}}\atop{\textstyle{\longrightarrow}\atop{
\textstyle{}^{\mbox{\footnotesize\rm onto}}}}$}}\m{{\cal P}}\m{A}
\endm
\vskip 1ex

\noindent The Burali-Forti paradox\index{Burali-Forti paradox}.  No set
contains all ordinal numbers. Enderton, p.~194.  (Burali-Forti was one person,
not two.)

\vskip 0.5ex
\setbox\startprefix=\hbox{\tt \ \ onprc\ \$p\ }
\setbox\contprefix=\hbox{\tt \ \ \ \ \ \ \ \ \ \ \ \ }
\startm
\m{\vdash}\m{\lnot}\m{\mbox{\rm On}}\m{\in}\m{{\rm V}}
\endm
\vskip 1ex

\noindent Peano's postulates\index{Peano's postulates} for arithmetic.
Proposition 7.30 of Takeuti and Zaring, pp.~42--43.  The objects being
described are the members of $\omega$ i.e.\ the natural numbers 0, 1,
2,\ldots.  The successor\index{successor} operation suc means ``plus
one.''  \texttt{peano1} says that 0 (which is defined as the empty set)
is a natural number.  \texttt{peano2} says that if $A$ is a natural
number, so is $A+1$.  \texttt{peano3} says that 0 is not the successor
of any natural number.  \texttt{peano4} says that two natural numbers
are equal if and only if their successors are equal.  \texttt{peano5} is
essentially the same as mathematical induction.

\vskip 1ex
\setbox\startprefix=\hbox{\tt \ \ peano1\ \$p\ }
\setbox\contprefix=\hbox{\tt \ \ \ \ \ \ \ \ \ \ \ \ }
\startm
\m{\vdash}\m{\varnothing}\m{\in}\m{\omega}
\endm
\vskip 1.5ex

\setbox\startprefix=\hbox{\tt \ \ peano2\ \$p\ }
\setbox\contprefix=\hbox{\tt \ \ \ \ \ \ \ \ \ \ \ \ }
\startm
\m{\vdash}\m{(}\m{A}\m{\in}\m{\omega}\m{\rightarrow}\m{{\rm suc}}\m{A}\m{\in}%
\m{\omega}\m{)}
\endm
\vskip 1.5ex

\setbox\startprefix=\hbox{\tt \ \ peano3\ \$p\ }
\setbox\contprefix=\hbox{\tt \ \ \ \ \ \ \ \ \ \ \ \ }
\startm
\m{\vdash}\m{(}\m{A}\m{\in}\m{\omega}\m{\rightarrow}\m{\lnot}\m{{\rm suc}}%
\m{A}\m{=}\m{\varnothing}\m{)}
\endm
\vskip 1.5ex

\setbox\startprefix=\hbox{\tt \ \ peano4\ \$p\ }
\setbox\contprefix=\hbox{\tt \ \ \ \ \ \ \ \ \ \ \ \ }
\startm
\m{\vdash}\m{(}\m{(}\m{A}\m{\in}\m{\omega}\m{\wedge}\m{B}\m{\in}\m{\omega}%
\m{)}\m{\rightarrow}\m{(}\m{{\rm suc}}\m{A}\m{=}\m{{\rm suc}}\m{B}\m{%
\leftrightarrow}\m{A}\m{=}\m{B}\m{)}\m{)}
\endm
\vskip 1.5ex

\setbox\startprefix=\hbox{\tt \ \ peano5\ \$p\ }
\setbox\contprefix=\hbox{\tt \ \ \ \ \ \ \ \ \ \ \ \ }
\startm
\m{\vdash}\m{(}\m{(}\m{\varnothing}\m{\in}\m{A}\m{\wedge}\m{\forall}\m{x}\m{%
\in}\m{\omega}\m{(}\m{x}\m{\in}\m{A}\m{\rightarrow}\m{{\rm suc}}\m{x}\m{\in}%
\m{A}\m{)}\m{)}\m{\rightarrow}\m{\omega}\m{\subseteq}\m{A}\m{)}
\endm
\vskip 1.5ex

\noindent Finite Induction (mathematical induction).\index{finite
induction}\index{mathematical induction} The first hypothesis is the
basis and the second is the induction hypothesis.  Theorem Schema 22 of
Suppes, p.~136.

\vskip 0.5ex
\setbox\startprefix=\hbox{\tt \ \ findes.1\ \$e\ }
\setbox\contprefix=\hbox{\tt \ \ \ \ \ \ \ \ \ \ \ \ \ \ }
\startm
\m{\vdash}\m{[}\m{\varnothing}\m{/}\m{x}\m{]}\m{\varphi}
\endm
\setbox\startprefix=\hbox{\tt \ \ findes.2\ \$e\ }
\setbox\contprefix=\hbox{\tt \ \ \ \ \ \ \ \ \ \ \ \ \ \ }
\startm
\m{\vdash}\m{(}\m{x}\m{\in}\m{\omega}\m{\rightarrow}\m{(}\m{\varphi}\m{%
\rightarrow}\m{[}\m{{\rm suc}}\m{x}\m{/}\m{x}\m{]}\m{\varphi}\m{)}\m{)}
\endm
\setbox\startprefix=\hbox{\tt \ \ findes\ \$p\ }
\setbox\contprefix=\hbox{\tt \ \ \ \ \ \ \ \ \ \ \ \ }
\startm
\m{\vdash}\m{(}\m{x}\m{\in}\m{\omega}\m{\rightarrow}\m{\varphi}\m{)}
\endm
\vskip 1ex

\noindent Transfinite Induction with explicit substitution.  The first
hypothesis is the basis, the second is the induction hypothesis for
successors, and the third is the induction hypothesis for limit
ordinals.  Theorem Schema 4 of Suppes, p. 197.

\vskip 0.5ex
\setbox\startprefix=\hbox{\tt \ \ tfindes.1\ \$e\ }
\setbox\contprefix=\hbox{\tt \ \ \ \ \ \ \ \ \ \ \ \ \ \ \ }
\startm
\m{\vdash}\m{[}\m{\varnothing}\m{/}\m{x}\m{]}\m{\varphi}
\endm
\setbox\startprefix=\hbox{\tt \ \ tfindes.2\ \$e\ }
\setbox\contprefix=\hbox{\tt \ \ \ \ \ \ \ \ \ \ \ \ \ \ \ }
\startm
\m{\vdash}\m{(}\m{x}\m{\in}\m{{\rm On}}\m{\rightarrow}\m{(}\m{\varphi}\m{%
\rightarrow}\m{[}\m{{\rm suc}}\m{x}\m{/}\m{x}\m{]}\m{\varphi}\m{)}\m{)}
\endm
\setbox\startprefix=\hbox{\tt \ \ tfindes.3\ \$e\ }
\setbox\contprefix=\hbox{\tt \ \ \ \ \ \ \ \ \ \ \ \ \ \ \ }
\startm
\m{\vdash}\m{(}\m{{\rm Lim}}\m{y}\m{\rightarrow}\m{(}\m{\forall}\m{x}\m{\in}%
\m{y}\m{\varphi}\m{\rightarrow}\m{[}\m{y}\m{/}\m{x}\m{]}\m{\varphi}\m{)}\m{)}
\endm
\setbox\startprefix=\hbox{\tt \ \ tfindes\ \$p\ }
\setbox\contprefix=\hbox{\tt \ \ \ \ \ \ \ \ \ \ \ \ \ }
\startm
\m{\vdash}\m{(}\m{x}\m{\in}\m{{\rm On}}\m{\rightarrow}\m{\varphi}\m{)}
\endm
\vskip 1ex

\noindent Principle of Transfinite Recursion.\index{transfinite
recursion} Theorem 7.41 of Takeuti and Zaring, p.~47.  Transfinite
recursion is the key theorem that allows arithmetic of ordinals to be
rigorously defined, and has many other important uses as well.
Hypotheses \texttt{tfr.1} and \texttt{tfr.2} specify a certain (proper)
class $ F$.  The complicated definition of $ F$ is not important in
itself; what is important is that there be such an $ F$ with the
required properties, and we show this by displaying $ F$ explicitly.
\texttt{tfr1} states that $ F$ is a function whose domain is the set of
ordinal numbers.  \texttt{tfr2} states that any value of $ F$ is
completely determined by its previous values and the values of an
auxiliary function, $G$.  \texttt{tfr3} states that $ F$ is unique,
i.e.\ it is the only function that satisfies \texttt{tfr1} and
\texttt{tfr2}.  Note that $ f$ is an individual variable like $x$ and
$y$; it is just a mnemonic to remind us that $A$ is a collection of
functions.

\vskip 0.5ex
\setbox\startprefix=\hbox{\tt \ \ tfr.1\ \$e\ }
\setbox\contprefix=\hbox{\tt \ \ \ \ \ \ \ \ \ \ \ }
\startm
\m{\vdash}\m{A}\m{=}\m{\{}\m{f}\m{|}\m{\exists}\m{x}\m{\in}\m{{\rm On}}\m{(}%
\m{f}\m{{\rm Fn}}\m{x}\m{\wedge}\m{\forall}\m{y}\m{\in}\m{x}\m{(}\m{f}\m{`}%
\m{y}\m{)}\m{=}\m{(}\m{G}\m{`}\m{(}\m{f}\m{\restriction}\m{y}\m{)}\m{)}\m{)}%
\m{\}}
\endm
\setbox\startprefix=\hbox{\tt \ \ tfr.2\ \$e\ }
\setbox\contprefix=\hbox{\tt \ \ \ \ \ \ \ \ \ \ \ }
\startm
\m{\vdash}\m{F}\m{=}\m{\bigcup}\m{A}
\endm
\setbox\startprefix=\hbox{\tt \ \ tfr1\ \$p\ }
\setbox\contprefix=\hbox{\tt \ \ \ \ \ \ \ \ \ \ }
\startm
\m{\vdash}\m{F}\m{{\rm Fn}}\m{{\rm On}}
\endm
\setbox\startprefix=\hbox{\tt \ \ tfr2\ \$p\ }
\setbox\contprefix=\hbox{\tt \ \ \ \ \ \ \ \ \ \ }
\startm
\m{\vdash}\m{(}\m{z}\m{\in}\m{{\rm On}}\m{\rightarrow}\m{(}\m{F}\m{`}\m{z}%
\m{)}\m{=}\m{(}\m{G}\m{`}\m{(}\m{F}\m{\restriction}\m{z}\m{)}\m{)}\m{)}
\endm
\setbox\startprefix=\hbox{\tt \ \ tfr3\ \$p\ }
\setbox\contprefix=\hbox{\tt \ \ \ \ \ \ \ \ \ \ }
\startm
\m{\vdash}\m{(}\m{(}\m{B}\m{{\rm Fn}}\m{{\rm On}}\m{\wedge}\m{\forall}\m{x}\m{%
\in}\m{{\rm On}}\m{(}\m{B}\m{`}\m{x}\m{)}\m{=}\m{(}\m{G}\m{`}\m{(}\m{B}\m{%
\restriction}\m{x}\m{)}\m{)}\m{)}\m{\rightarrow}\m{B}\m{=}\m{F}\m{)}
\endm
\vskip 1ex

\noindent The existence of omega (the class of natural numbers).\index{natural
number}\index{omega ($\omega$)}\index{Axiom of Infinity}  Axiom 7 of Takeuti
and Zaring, p.~43.  (This is the only theorem in this section requiring the
Axiom of Infinity.)

\vskip 0.5ex
\setbox\startprefix=\hbox{\tt \
\ omex\ \$p\ }
\setbox\contprefix=\hbox{\tt \ \ \ \ \ \ \ \ \ \ }
\startm
\m{\vdash}\m{\omega}\m{\in}\m{{\rm V}}
\endm
%\vskip 2ex


\section{Axioms for Real and Complex Numbers}\label{real}
\index{real and complex numbers!axioms for}

This section presents the axioms for real and complex numbers.  Analysis
textbooks implicitly or explicitly use these axioms or their equivalents
as their starting point.  In the database \texttt{set.mm}, we define real
and complex numbers as (rather complicated) specific sets and derive these
axioms as {\em theorems} from the axioms of ZF set theory, using a method
called Dedekind cuts.  We omit the details of this construction, which you can
follow if you wish using the \texttt{set.mm} database in conjunction with the
textbooks referenced therein.

Once we prove those theorems, we then restate these proven theorems as axioms.
This lets us easily identify which axioms are needed for a particular complex number proof, without the obfuscation of the set theory used to derive them.
As a result,
the construction is actually unimportant other
than to show that sets exist that satisfy the axioms, and thus that the axioms
are consistent if ZF set theory is consistent.  When working with real numbers
you can think of them as being the actual sets resulting
from the construction (for definiteness), or you can
think of them as otherwise unspecified sets that happen to satisfy the axioms.
The derivation is not easy, but the fact that it works is quite remarkable
and lends support to the idea that ZFC set theory is all we need to
provide a foundation for essentially all of mathematics.

For the axioms we are given (or postulate) 8 classes:  $\mathbb{C}$ (the
set of complex numbers), $\mathbb{R}$ (the set of real numbers, a subset
of $\mathbb{C}$), $0$ (zero), $1$ (one), $i$ (square root of
$-1$), $+$ (plus), $\cdot$ (times), and
$<_{\mathbb{R}}$ (less than for just the real numbers).
Subtraction and division are defined terms and are not part of the
axioms; for their definitions see \texttt{set.mm}.

Note that the notation $(A+B)$ (and similarly $(A\cdot B)$) specifies a class
called an {\em operation},\index{operation} and is the function value of the
class $+$ at ordered pair $\langle A,B \rangle$.  An operation is defined by
statement \texttt{df-opr} on p.~\pageref{dfopr}.
The notation $A <_{\mathbb{R}} B$ specifies a
wff called a {\em binary relation}\index{binary relation} and means $\langle A,B \rangle \in \,<_{\mathbb{R}}$, as defined by statement \texttt{df-br} on p.~\pageref{dfbr}.

Our set of 8 given classes is assumed to satisfy the following 22 axioms
(in the axioms listed below, $<$ really means $<_{\mathbb{R}}$).

\vskip 2ex

\noindent 1. The real numbers are a subset of the complex numbers.

%\vskip 0.5ex
\setbox\startprefix=\hbox{\tt \ \ ax-resscn\ \$p\ }
\setbox\contprefix=\hbox{\tt \ \ \ \ \ \ \ \ \ \ \ \ \ \ }
\startm
\m{\vdash}\m{\mathbb{R}}\m{\subseteq}\m{\mathbb{C}}
\endm
%\vskip 1ex

\noindent 2. One is a complex number.

%\vskip 0.5ex
\setbox\startprefix=\hbox{\tt \ \ ax-1cn\ \$p\ }
\setbox\contprefix=\hbox{\tt \ \ \ \ \ \ \ \ \ \ \ }
\startm
\m{\vdash}\m{1}\m{\in}\m{\mathbb{C}}
\endm
%\vskip 1ex

\noindent 3. The imaginary number $i$ is a complex number.

%\vskip 0.5ex
\setbox\startprefix=\hbox{\tt \ \ ax-icn\ \$p\ }
\setbox\contprefix=\hbox{\tt \ \ \ \ \ \ \ \ \ \ \ }
\startm
\m{\vdash}\m{i}\m{\in}\m{\mathbb{C}}
\endm
%\vskip 1ex

\noindent 4. Complex numbers are closed under addition.

%\vskip 0.5ex
\setbox\startprefix=\hbox{\tt \ \ ax-addcl\ \$p\ }
\setbox\contprefix=\hbox{\tt \ \ \ \ \ \ \ \ \ \ \ \ \ }
\startm
\m{\vdash}\m{(}\m{(}\m{A}\m{\in}\m{\mathbb{C}}\m{\wedge}\m{B}\m{\in}\m{\mathbb{C}}%
\m{)}\m{\rightarrow}\m{(}\m{A}\m{+}\m{B}\m{)}\m{\in}\m{\mathbb{C}}\m{)}
\endm
%\vskip 1ex

\noindent 5. Real numbers are closed under addition.

%\vskip 0.5ex
\setbox\startprefix=\hbox{\tt \ \ ax-addrcl\ \$p\ }
\setbox\contprefix=\hbox{\tt \ \ \ \ \ \ \ \ \ \ \ \ \ \ }
\startm
\m{\vdash}\m{(}\m{(}\m{A}\m{\in}\m{\mathbb{R}}\m{\wedge}\m{B}\m{\in}\m{\mathbb{R}}%
\m{)}\m{\rightarrow}\m{(}\m{A}\m{+}\m{B}\m{)}\m{\in}\m{\mathbb{R}}\m{)}
\endm
%\vskip 1ex

\noindent 6. Complex numbers are closed under multiplication.

%\vskip 0.5ex
\setbox\startprefix=\hbox{\tt \ \ ax-mulcl\ \$p\ }
\setbox\contprefix=\hbox{\tt \ \ \ \ \ \ \ \ \ \ \ \ \ }
\startm
\m{\vdash}\m{(}\m{(}\m{A}\m{\in}\m{\mathbb{C}}\m{\wedge}\m{B}\m{\in}\m{\mathbb{C}}%
\m{)}\m{\rightarrow}\m{(}\m{A}\m{\cdot}\m{B}\m{)}\m{\in}\m{\mathbb{C}}\m{)}
\endm
%\vskip 1ex

\noindent 7. Real numbers are closed under multiplication.

%\vskip 0.5ex
\setbox\startprefix=\hbox{\tt \ \ ax-mulrcl\ \$p\ }
\setbox\contprefix=\hbox{\tt \ \ \ \ \ \ \ \ \ \ \ \ \ \ }
\startm
\m{\vdash}\m{(}\m{(}\m{A}\m{\in}\m{\mathbb{R}}\m{\wedge}\m{B}\m{\in}\m{\mathbb{R}}%
\m{)}\m{\rightarrow}\m{(}\m{A}\m{\cdot}\m{B}\m{)}\m{\in}\m{\mathbb{R}}\m{)}
\endm
%\vskip 1ex

\noindent 8. Multiplication of complex numbers is commutative.

%\vskip 0.5ex
\setbox\startprefix=\hbox{\tt \ \ ax-mulcom\ \$p\ }
\setbox\contprefix=\hbox{\tt \ \ \ \ \ \ \ \ \ \ \ \ \ \ }
\startm
\m{\vdash}\m{(}\m{(}\m{A}\m{\in}\m{\mathbb{C}}\m{\wedge}\m{B}\m{\in}\m{\mathbb{C}}%
\m{)}\m{\rightarrow}\m{(}\m{A}\m{\cdot}\m{B}\m{)}\m{=}\m{(}\m{B}\m{\cdot}\m{A}%
\m{)}\m{)}
\endm
%\vskip 1ex

\noindent 9. Addition of complex numbers is associative.

%\vskip 0.5ex
\setbox\startprefix=\hbox{\tt \ \ ax-addass\ \$p\ }
\setbox\contprefix=\hbox{\tt \ \ \ \ \ \ \ \ \ \ \ \ \ \ }
\startm
\m{\vdash}\m{(}\m{(}\m{A}\m{\in}\m{\mathbb{C}}\m{\wedge}\m{B}\m{\in}\m{\mathbb{C}}%
\m{\wedge}\m{C}\m{\in}\m{\mathbb{C}}\m{)}\m{\rightarrow}\m{(}\m{(}\m{A}\m{+}%
\m{B}\m{)}\m{+}\m{C}\m{)}\m{=}\m{(}\m{A}\m{+}\m{(}\m{B}\m{+}\m{C}\m{)}\m{)}%
\m{)}
\endm
%\vskip 1ex

\noindent 10. Multiplication of complex numbers is associative.

%\vskip 0.5ex
\setbox\startprefix=\hbox{\tt \ \ ax-mulass\ \$p\ }
\setbox\contprefix=\hbox{\tt \ \ \ \ \ \ \ \ \ \ \ \ \ \ }
\startm
\m{\vdash}\m{(}\m{(}\m{A}\m{\in}\m{\mathbb{C}}\m{\wedge}\m{B}\m{\in}\m{\mathbb{C}}%
\m{\wedge}\m{C}\m{\in}\m{\mathbb{C}}\m{)}\m{\rightarrow}\m{(}\m{(}\m{A}\m{\cdot}%
\m{B}\m{)}\m{\cdot}\m{C}\m{)}\m{=}\m{(}\m{A}\m{\cdot}\m{(}\m{B}\m{\cdot}\m{C}%
\m{)}\m{)}\m{)}
\endm
%\vskip 1ex

\noindent 11. Multiplication distributes over addition for complex numbers.

%\vskip 0.5ex
\setbox\startprefix=\hbox{\tt \ \ ax-distr\ \$p\ }
\setbox\contprefix=\hbox{\tt \ \ \ \ \ \ \ \ \ \ \ \ \ }
\startm
\m{\vdash}\m{(}\m{(}\m{A}\m{\in}\m{\mathbb{C}}\m{\wedge}\m{B}\m{\in}\m{\mathbb{C}}%
\m{\wedge}\m{C}\m{\in}\m{\mathbb{C}}\m{)}\m{\rightarrow}\m{(}\m{A}\m{\cdot}\m{(}%
\m{B}\m{+}\m{C}\m{)}\m{)}\m{=}\m{(}\m{(}\m{A}\m{\cdot}\m{B}\m{)}\m{+}\m{(}%
\m{A}\m{\cdot}\m{C}\m{)}\m{)}\m{)}
\endm
%\vskip 1ex

\noindent 12. The square of $i$ equals $-1$ (expressed as $i$-squared plus 1 is
0).

%\vskip 0.5ex
\setbox\startprefix=\hbox{\tt \ \ ax-i2m1\ \$p\ }
\setbox\contprefix=\hbox{\tt \ \ \ \ \ \ \ \ \ \ \ \ }
\startm
\m{\vdash}\m{(}\m{(}\m{i}\m{\cdot}\m{i}\m{)}\m{+}\m{1}\m{)}\m{=}\m{0}
\endm
%\vskip 1ex

\noindent 13. One and zero are distinct.

%\vskip 0.5ex
\setbox\startprefix=\hbox{\tt \ \ ax-1ne0\ \$p\ }
\setbox\contprefix=\hbox{\tt \ \ \ \ \ \ \ \ \ \ \ \ }
\startm
\m{\vdash}\m{1}\m{\ne}\m{0}
\endm
%\vskip 1ex

\noindent 14. One is an identity element for real multiplication.

%\vskip 0.5ex
\setbox\startprefix=\hbox{\tt \ \ ax-1rid\ \$p\ }
\setbox\contprefix=\hbox{\tt \ \ \ \ \ \ \ \ \ \ \ }
\startm
\m{\vdash}\m{(}\m{A}\m{\in}\m{\mathbb{R}}\m{\rightarrow}\m{(}\m{A}\m{\cdot}\m{1}%
\m{)}\m{=}\m{A}\m{)}
\endm
%\vskip 1ex

\noindent 15. Every real number has a negative.

%\vskip 0.5ex
\setbox\startprefix=\hbox{\tt \ \ ax-rnegex\ \$p\ }
\setbox\contprefix=\hbox{\tt \ \ \ \ \ \ \ \ \ \ \ \ \ \ }
\startm
\m{\vdash}\m{(}\m{A}\m{\in}\m{\mathbb{R}}\m{\rightarrow}\m{\exists}\m{x}\m{\in}%
\m{\mathbb{R}}\m{(}\m{A}\m{+}\m{x}\m{)}\m{=}\m{0}\m{)}
\endm
%\vskip 1ex

\noindent 16. Every nonzero real number has a reciprocal.

%\vskip 0.5ex
\setbox\startprefix=\hbox{\tt \ \ ax-rrecex\ \$p\ }
\setbox\contprefix=\hbox{\tt \ \ \ \ \ \ \ \ \ \ \ \ \ \ }
\startm
\m{\vdash}\m{(}\m{A}\m{\in}\m{\mathbb{R}}\m{\rightarrow}\m{(}\m{A}\m{\ne}\m{0}%
\m{\rightarrow}\m{\exists}\m{x}\m{\in}\m{\mathbb{R}}\m{(}\m{A}\m{\cdot}%
\m{x}\m{)}\m{=}\m{1}\m{)}\m{)}
\endm
%\vskip 1ex

\noindent 17. A complex number can be expressed in terms of two reals.

%\vskip 0.5ex
\setbox\startprefix=\hbox{\tt \ \ ax-cnre\ \$p\ }
\setbox\contprefix=\hbox{\tt \ \ \ \ \ \ \ \ \ \ \ \ }
\startm
\m{\vdash}\m{(}\m{A}\m{\in}\m{\mathbb{C}}\m{\rightarrow}\m{\exists}\m{x}\m{\in}%
\m{\mathbb{R}}\m{\exists}\m{y}\m{\in}\m{\mathbb{R}}\m{A}\m{=}\m{(}\m{x}\m{+}\m{(}%
\m{y}\m{\cdot}\m{i}\m{)}\m{)}\m{)}
\endm
%\vskip 1ex

\noindent 18. Ordering on reals satisfies strict trichotomy.

%\vskip 0.5ex
\setbox\startprefix=\hbox{\tt \ \ ax-pre-lttri\ \$p\ }
\setbox\contprefix=\hbox{\tt \ \ \ \ \ \ \ \ \ \ \ \ \ }
\startm
\m{\vdash}\m{(}\m{(}\m{A}\m{\in}\m{\mathbb{R}}\m{\wedge}\m{B}\m{\in}\m{\mathbb{R}}%
\m{)}\m{\rightarrow}\m{(}\m{A}\m{<}\m{B}\m{\leftrightarrow}\m{\lnot}\m{(}\m{A}%
\m{=}\m{B}\m{\vee}\m{B}\m{<}\m{A}\m{)}\m{)}\m{)}
\endm
%\vskip 1ex

\noindent 19. Ordering on reals is transitive.

%\vskip 0.5ex
\setbox\startprefix=\hbox{\tt \ \ ax-pre-lttrn\ \$p\ }
\setbox\contprefix=\hbox{\tt \ \ \ \ \ \ \ \ \ \ \ \ \ }
\startm
\m{\vdash}\m{(}\m{(}\m{A}\m{\in}\m{\mathbb{R}}\m{\wedge}\m{B}\m{\in}\m{\mathbb{R}}%
\m{\wedge}\m{C}\m{\in}\m{\mathbb{R}}\m{)}\m{\rightarrow}\m{(}\m{(}\m{A}\m{<}%
\m{B}\m{\wedge}\m{B}\m{<}\m{C}\m{)}\m{\rightarrow}\m{A}\m{<}\m{C}\m{)}\m{)}
\endm
%\vskip 1ex

\noindent 20. Ordering on reals is preserved after addition to both sides.

%\vskip 0.5ex
\setbox\startprefix=\hbox{\tt \ \ ax-pre-ltadd\ \$p\ }
\setbox\contprefix=\hbox{\tt \ \ \ \ \ \ \ \ \ \ \ \ \ }
\startm
\m{\vdash}\m{(}\m{(}\m{A}\m{\in}\m{\mathbb{R}}\m{\wedge}\m{B}\m{\in}\m{\mathbb{R}}%
\m{\wedge}\m{C}\m{\in}\m{\mathbb{R}}\m{)}\m{\rightarrow}\m{(}\m{A}\m{<}\m{B}\m{%
\rightarrow}\m{(}\m{C}\m{+}\m{A}\m{)}\m{<}\m{(}\m{C}\m{+}\m{B}\m{)}\m{)}\m{)}
\endm
%\vskip 1ex

\noindent 21. The product of two positive reals is positive.

%\vskip 0.5ex
\setbox\startprefix=\hbox{\tt \ \ ax-pre-mulgt0\ \$p\ }
\setbox\contprefix=\hbox{\tt \ \ \ \ \ \ \ \ \ \ \ \ \ \ }
\startm
\m{\vdash}\m{(}\m{(}\m{A}\m{\in}\m{\mathbb{R}}\m{\wedge}\m{B}\m{\in}\m{\mathbb{R}}%
\m{)}\m{\rightarrow}\m{(}\m{(}\m{0}\m{<}\m{A}\m{\wedge}\m{0}%
\m{<}\m{B}\m{)}\m{\rightarrow}\m{0}\m{<}\m{(}\m{A}\m{\cdot}\m{B}\m{)}%
\m{)}\m{)}
\endm
%\vskip 1ex

\noindent 22. A non-empty, bounded-above set of reals has a supremum.

%\vskip 0.5ex
\setbox\startprefix=\hbox{\tt \ \ ax-pre-sup\ \$p\ }
\setbox\contprefix=\hbox{\tt \ \ \ \ \ \ \ \ \ \ \ }
\startm
\m{\vdash}\m{(}\m{(}\m{A}\m{\subseteq}\m{\mathbb{R}}\m{\wedge}\m{A}\m{\ne}\m{%
\varnothing}\m{\wedge}\m{\exists}\m{x}\m{\in}\m{\mathbb{R}}\m{\forall}\m{y}\m{%
\in}\m{A}\m{\,y}\m{<}\m{x}\m{)}\m{\rightarrow}\m{\exists}\m{x}\m{\in}\m{%
\mathbb{R}}\m{(}\m{\forall}\m{y}\m{\in}\m{A}\m{\lnot}\m{x}\m{<}\m{y}\m{\wedge}\m{%
\forall}\m{y}\m{\in}\m{\mathbb{R}}\m{(}\m{y}\m{<}\m{x}\m{\rightarrow}\m{\exists}%
\m{z}\m{\in}\m{A}\m{\,y}\m{<}\m{z}\m{)}\m{)}\m{)}
\endm

% NOTE: The \m{...} expressions above could be represented as
% $ \vdash ( ( A \subseteq \mathbb{R} \wedge A \ne \varnothing \wedge \exists x \in \mathbb{R} \forall y \in A \,y < x ) \rightarrow \exists x \in \mathbb{R} ( \forall y \in A \lnot x < y \wedge \forall y \in \mathbb{R} ( y < x \rightarrow \exists z \in A \,y < z ) ) ) $

\vskip 2ex

This completes the set of axioms for real and complex numbers.  You may
wish to look at how subtraction, division, and decimal numbers
are defined in \texttt{set.mm}, and for fun look at the proof of $2+
2 = 4$ (theorem \texttt{2p2e4} in \texttt{set.mm})
as discussed in section \ref{2p2e4}.

In \texttt{set.mm} we define the non-negative integers $\mathbb{N}$, the integers
$\mathbb{Z}$, and the rationals $\mathbb{Q}$ as subsets of $\mathbb{R}$.  This leads
to the nice inclusion $\mathbb{N} \subseteq \mathbb{Z} \subseteq \mathbb{Q} \subseteq
\mathbb{R} \subseteq \mathbb{C}$, giving us a uniform framework in which, for
example, a property such as commutativity of complex number addition
automatically applies to integers.  The natural numbers $\mathbb{N}$
are different from the set $\omega$ we defined earlier, but both satisfy
Peano's postulates.

\subsubsection{Complex Number Axioms in Analysis Texts}

Most analysis texts construct complex numbers as ordered pairs of reals,
leading to construction-dependent properties that satisfy these axioms
but are not stated in their pure form.  (This is also done in
\texttt{set.mm} but our axioms are extracted from that construction.)
Other texts will simply state that $\mathbb{R}$ is a ``complete ordered
subfield of $\mathbb{C}$,'' leading to redundant axioms when this phrase
is completely expanded out.  In fact I have not seen a text with the
axioms in the explicit form above.
None of these axioms is unique individually, but this carefully worked out
collection of axioms is the result of years of work
by the Metamath community.

\subsubsection{Eliminating Unnecessary Complex Number Axioms}

We once had more axioms for real and complex numbers, but over years of time
we (the Metamath community)
have found ways to eliminate them (by proving them from other axioms)
or weaken them (by making weaker claims without reducing what
can be proved).
In particular, here are statements that used to be complex number
axioms but have since been formally proven (with Metamath) to be redundant:

\begin{itemize}
\item
  $\mathbb{C} \in V$.
  At one time this was listed as a ``complex number axiom''
  However, this is not properly speaking a complex number axiom,
  and in any case its proof uses axioms of set theory.
  Proven redundant by Mario Carneiro\index{Carneiro, Mario} on
  17-Nov-2014 (see \texttt{axcnex}).
\item
  $((A \in \mathbb{C} \land B \in \mathbb{C}$) $\rightarrow$
  $(A + B) = (B + A))$.
  Proved redundant by Eric Schmidt\index{Schmidt, Eric} on 19-Jun-2012,
  and formalized by Scott Fenton\index{Fenton, Scott} on 3-Jan-2013
  (see \texttt{addcom}).
\item
  $(A \in \mathbb{C} \rightarrow (A + 0) = A)$.
  Proved redundant by Eric Schmidt on 19-Jun-2012,
  and formalized by Scott Fenton on 3-Jan-2013
  (see \texttt{addid1}).
\item
  $(A \in \mathbb{C} \rightarrow \exists x \in \mathbb{C} (A + x) = 0)$.
  Proved redundant by Eric Schmidt and formalized on 21-May-2007
  (see \texttt{cnegex}).
\item
  $((A \in \mathbb{C} \land A \ne 0) \rightarrow \exists x \in \mathbb{C} (A \cdot x) = 1)$.
  Proved redundant by Eric Schmidt and formalized on 22-May-2007
  (see \texttt{recex}).
\item
  $0 \in \mathbb{R}$.
  Proved redundant by Eric Schmidt on 19-Feb-2005 and formalized 21-May-2007
  (see \texttt{0re}).
\end{itemize}

We could eliminate 0 as an axiomatic object by defining it as
$( ( i \cdot i ) + 1 )$
and replacing it with this expression throughout the axioms. If this
is done, axiom ax-i2m1 becomes redundant. However, the remaining axioms
would become longer and less intuitive.

Eric Schmidt's paper analyzing this axiom system \cite{Schmidt}
presented a proof that these remaining axioms,
with the possible exception of ax-mulcom, are independent of the others.
It is currently an open question if ax-mulcom is independent of the others.

\section{Two Plus Two Equals Four}\label{2p2e4}

Here is a proof that $2 + 2 = 4$, as proven in the theorem \texttt{2p2e4}
in the database \texttt{set.mm}.
This is a useful demonstration of what a Metamath proof can look like.
This proof may have more steps than you're used to, but each step is rigorously
proven all the way back to the axioms of logic and set theory.
This display was originally generated by the Metamath program
as an {\sc HTML} file.

In the table showing the proof ``Step'' is the sequential step number,
while its associated ``Expression'' is an expression that we have proved.
``Ref'' is the name of a theorem or axiom that justifies that expression,
and ``Hyp'' refers to previous steps (if any) that the theorem or axiom
needs so that we can use it.  Expressions are indented further than
the expressions that depend on them to show their interdependencies.

\begin{table}[!htbp]
\caption{Two plus two equals four}
\begin{tabular}{lllll}
\textbf{Step} & \textbf{Hyp} & \textbf{Ref} & \textbf{Expression} & \\
1  &       & df-2    & $ \; \; \vdash 2 = 1 + 1$  & \\
2  & 1     & oveq2i  & $ \; \vdash (2 + 2) = (2 + (1 + 1))$ & \\
3  &       & df-4    & $ \; \; \vdash 4 = (3 + 1)$ & \\
4  &       & df-3    & $ \; \; \; \vdash 3 = (2 + 1)$ & \\
5  & 4     & oveq1i  & $ \; \; \vdash (3 + 1) = ((2 + 1) + 1)$ & \\
6  &       & 2cn     & $ \; \; \; \vdash 2 \in \mathbb{C}$ & \\
7  &       & ax-1cn  & $ \; \; \; \vdash 1 \in \mathbb{C}$ & \\
8  & 6,7,7 & addassi & $ \; \; \vdash ((2 + 1) + 1) = (2 + (1 + 1))$ & \\
9  & 3,5,8 & 3eqtri  & $ \; \vdash 4 = (2 + (1 + 1))$ & \\
10 & 2,9   & eqtr4i  & $ \vdash (2 + 2) = 4$ & \\
\end{tabular}
\end{table}

Step 1 says that we can assert that $2 = 1 + 1$ because it is
justified by \texttt{df-2}.
What is \texttt{df-2}?
It is simply the definition of $2$, which in our system is defined as being
equal to $1 + 1$.  This shows how we can use definitions in proofs.

Look at Step 2 of the proof. In the Ref column, we see that it references
a previously proved theorem, \texttt{oveq2i}.
It turns out that
theorem \texttt{oveq2i} requires a
hypothesis, and in the Hyp column of Step 2 we indicate that Step 1 will
satisfy (match) this hypothesis.
If we looked at \texttt{oveq2i}
we would find that it proves that given some hypothesis
$A = B$, we can prove that $( C F A ) = ( C F B )$.
If we use \texttt{oveq2i} and apply step 1's result as the hypothesis,
that will mean that $A = 2$ and $B = ( 1 + 1 )$ within this use of
\texttt{oveq2i}.
We are free to select any value of $C$ and $F$ (subject to syntax constraints),
so we are free to select $C = 2$ and $F = +$,
producing our desired result,
$ (2 + 2) = (2 + (1 + 1))$.

Step 2 is an example of substitution.
In the end, every step in every proof uses only this one substitution rule.
All the rules of logic, and all the axioms, are expressed so that
they can be used via this one substitution rule.
So once you master substitution, you can master every Metamath proof,
no exceptions.

Each step is clear and can be immediately checked.
In the {\sc HTML} display you can even click on each reference to see why it is
justified, making it easy to see why the proof works.

\section{Deduction}\label{deduction}

Strictly speaking,
a deduction (also called an inference) is a kind of statement that needs
some hypotheses to be true in order for its conclusion to be true.
A theorem, on the other hand, has no hypotheses.
Informally we often call both of them theorems, but in this section we
will stick to the strict definitions.

It sometimes happens that we have proved a deduction of the form
$\varphi \Rightarrow \psi$\index{$\Rightarrow$}
(given hypothesis $\varphi$ we can prove $\psi$)
and we want to then prove a theorem of the form
$\varphi \rightarrow \psi$.

Converting a deduction (which uses a hypothesis) into a theorem
(which does not) is not as simple as you might think.
The deduction says, ``if we can prove $\varphi$ then we can prove $\psi$,''
which is in some sense weaker than saying
``$\varphi$ implies $\psi$.''
There is no axiom of logic that permits us to directly obtain the theorem
given the deduction.\footnote{
The conversion of a deduction to a theorem does not even hold in general
for quantum propositional calculus,
which is a weak subset of classical propositional calculus.
It has been shown that adding the Standard Deduction Theorem (discussed below)
to quantum propositional calculus turns it into classical
propositional calculus!
}

This is in contrast to going the other way.
If we have the theorem ($\varphi \rightarrow \psi$),
it is easy to recover the deduction
($\varphi \Rightarrow \psi$)
using modus ponens\index{modus ponens}
(\texttt{ax-mp}; see section \ref{axmp}).

In the following subsections we first discuss the standard deduction theorem
(the traditional but awkward way to convert deductions into theorems) and
the weak deduction theorem (a limited version of the standard deduction
theorem that is easier to use and was once widely used in
\texttt{set.mm}\index{set theory database (\texttt{set.mm})}).
In section \ref{deductionstyle} we discuss
deduction style, the newer approach we now recommend in most cases.
Deduction style uses ``deduction form,'' a form that
prefixes each hypothesis (other than definitions) and the
conclusion with a universal antecedent (``$\varphi \rightarrow$'').
Deduction style is widely used in \texttt{set.mm},
so it is useful to understand it and \textit{why} it is widely used.
Section \ref{naturaldeduction}
briefly discusses our approach for using natural deduction
within \texttt{set.mm},
as that approach is deeply related to deduction style.
We conclude with a summary of the strengths of
our approach, which we believe are compelling.

\subsection{The Standard Deduction Theorem}\label{standarddeductiontheorem}

It is possible to make use of information
contained in the deduction or its proof to assist us with the proof of
the related theorem.
In traditional logic books, there is a metatheorem called the
Deduction Theorem\index{Deduction Theorem}\index{Standard Deduction Theorem},
discovered independently by Herbrand and Tarski around 1930.
The Deduction Theorem, which we often call the Standard Deduction Theorem,
provides an algorithm for constructing a proof of a theorem from the
proof of its corresponding deduction. See, for example,
\cite[p.~56]{Margaris}\index{Margaris, Angelo}.
To construct a proof for a theorem, the
algorithm looks at each step in the proof of the original deduction and
rewrites the step with several steps wherein the hypothesis is eliminated
and becomes an antecedent.

In ordinary mathematics, no one actually carries out the algorithm,
because (in its most basic form) it involves an exponential explosion of
the number of proof steps as more hypotheses are eliminated. Instead,
the Standard Deduction Theorem is invoked simply to claim that it can
be done in principle, without actually doing it.
What's more, the algorithm is not as simple as it might first appear
when applying it rigorously.
There is a subtle restriction on the Standard Deduction Theorem
that must be taken into account involving the axiom of generalization
when working with predicate calculus (see the literature for more detail).

One of the goals of Metamath is to let you plainly see, with as few
underlying concepts as possible, how mathematics can be derived directly
from the axioms, and not indirectly according to some hidden rules
buried inside a program or understood only by logicians. If we added
the Standard Deduction Theorem to the language and proof verifier,
that would greatly complicate both and largely defeat Metamath's goal
of simplicity. In principle, we could show direct proofs by expanding
out the proof steps generated by the algorithm of the Standard Deduction
Theorem, but that is not feasible in practice because the number of proof
steps quickly becomes huge, even astronomical.
Since the algorithm of the Standard Deduction Theorem is driven by the proof,
we would have to go through that proof
all over again---starting from axioms---in order to obtain the theorem form.
In terms of proof length, there would be no savings over just
proving the theorem directly instead of first proving the deduction form.

\subsection{Weak Deduction Theorem}\label{weakdeductiontheorem}

We have developed
a more efficient method for proving a theorem from a deduction
that can be used instead of the Standard Deduction Theorem
in many (but not all) cases.
We call this more efficient method the
Weak Deduction Theorem\index{Weak Deduction Theorem}.\footnote{
There is also an unrelated ``Weak Deduction Theorem''
in the field of relevance logic, so to avoid confusion we could call
ours the ``Weak Deduction Theorem for Classical Logic.''}
Unlike the Standard Deduction Theorem, the Weak Deduction Theorem produces the
theorem directly from a special substitution instance of the deduction,
using a small, fixed number of steps roughly proportional to the length
of the final theorem.

If you come to a proof referencing the Weak Deduction Theorem
\texttt{dedth} (or one of its variants \texttt{dedthxx}),
here is how to follow the proof without getting into the details:
just click on the theorem referenced in the step
just before the reference to \texttt{dedth} and ignore everything else.
Theorem \texttt{dedth} simply turns a hypothesis into an antecedent
(i.e. the hypothesis followed by $\rightarrow$
is placed in front of the assertion, and the hypothesis
itself is eliminated) given certain conditions.

The Weak Deduction Theorem
eliminates a hypothesis $\varphi$, making it become an antecedent.
It does this by proving an expression
$ \varphi \rightarrow \psi $ given two hypotheses:
(1)
$ ( A = {\rm if} ( \varphi , A , B ) \rightarrow ( \varphi \leftrightarrow \chi ) ) $
and
(2) $\chi$.
Note that it requires that a proof exists for $\varphi$ when the class variable
$A$ is replaced with a specific class $B$. The hypothesis $\chi$
should be assigned to the inference.
You can see the details of the proof of the Weak Deduction Theorem
in theorem \texttt{dedth}.

The Weak Deduction Theorem
is probably easier to understand by studying proofs that make use of it.
For example, let's look at the proof of \texttt{renegcl}, which proves that
$ \vdash ( A \in \mathbb{R} \rightarrow - A \in \mathbb{R} )$:

\needspace{4\baselineskip}
\begin{longtabu} {l l l X}
\textbf{Step} & \textbf{Hyp} & \textbf{Ref} & \textbf{Expression} \\
  1 &  & negeq &
  $\vdash$ $($ $A$ $=$ ${\rm if}$ $($ $A$ $\in$ $\mathbb{R}$ $,$ $A$ $,$ $1$ $)$ $\rightarrow$
  $\textrm{-}$ $A$ $=$ $\textrm{-}$ ${\rm if}$ $($ $A$ $\in$ $\mathbb{R}$
  $,$ $A$ $,$ $1$ $)$ $)$ \\
 2 & 1 & eleq1d &
    $\vdash$ $($ $A$ $=$ ${\rm if}$ $($ $A$ $\in$ $\mathbb{R}$ $,$ $A$ $,$ $1$ $)$ $\rightarrow$ $($
    $\textrm{-}$ $A$ $\in$ $\mathbb{R}$ $\leftrightarrow$
    $\textrm{-}$ ${\rm if}$ $($ $A$ $\in$ $\mathbb{R}$ $,$ $A$ $,$ $1$ $)$ $\in$
    $\mathbb{R}$ $)$ $)$ \\
 3 &  & 1re & $\vdash 1 \in \mathbb{R}$ \\
 4 & 3 & elimel &
   $\vdash {\rm if} ( A \in \mathbb{R} , A , 1 ) \in \mathbb{R}$ \\
 5 & 4 & renegcli &
   $\vdash \textrm{-} {\rm if} ( A \in \mathbb{R} , A , 1 ) \in \mathbb{R}$ \\
 6 & 2,5 & dedth &
   $\vdash ( A \in \mathbb{R} \rightarrow \textrm{-} A \in \mathbb{R}$ ) \\
\end{longtabu}

The somewhat strange-looking steps in \texttt{renegcl} before step 5 are
technical stuff that makes this magic work, and they can be ignored
for a quick overview of the proof. To continue following the ``important''
part of the proof of \texttt{renegcl},
you can look at the reference to \texttt{renegcli} at step 5.

That said, let's briefly look at how
\texttt{renegcl} uses the
Weak Deduction Theorem (\texttt{dedth}) to do its job,
in case you want to do something similar or want understand it more deeply.
Let's work backwards in the proof of \texttt{renegcl}.
Step 6 applies \texttt{dedth} to produce our goal result
$ \vdash ( A \in \mathbb{R} \rightarrow\, - A \in \mathbb{R} )$.
This requires on the one hand the (substituted) deduction
\texttt{renegcli} in step 5.
By itself \texttt{renegcli} proves the deduction
$ \vdash A \in \mathbb{R} \Rightarrow\, \vdash - A \in \mathbb{R}$;
this is the deduction form we are trying to turn into theorem form,
and thus
\texttt{renegcli} has a separate hypothesis that must be fulfilled.
To fulfill the hypothesis of the invocation of
\texttt{renegcli} in step 5, it is eventually
reduced to the already proven theorem $1 \in \mathbb{R}$ in step 3.
Step 4 connects steps 3 and 5; step 4 invokes
\texttt{elimel}, a special case of \texttt{elimhyp} that eliminates
a membership hypothesis for the weak deduction theorem.
On the other hand, the equivalence of the conclusion of
\texttt{renegcl}
$( - A \in \mathbb{R} )$ and the substituted conclusion of
\texttt{renegcli} must be proven, which is done in steps 2 and 1.

The weak deduction theorem has limitations.
In particular, we must be able to prove a special case of the deduction's
hypothesis as a stand-alone theorem.
For example, we used $1 \in \mathbb{R}$ in step 3 of \texttt{renegcl}.

We used to use the weak deduction theorem
extensively within \texttt{set.mm}.
However, we now recommend applying ``deduction style''
instead in most cases, as deduction style is
often an easier and clearer approach.
Therefore, we will now describe deduction style.

\subsection{Deduction Style}\label{deductionstyle}

We now prefer to write assertions in ``deduction form''
instead of writing a proof that would require use of the standard or
weak deduction theorem.
We call this appraoch
``deduction style.''\index{deduction style}

It will be easier to explain this by first defining some terms:

\begin{itemize}
\item \textbf{closed form}\index{closed form}\index{forms!closed}:
A kind of assertion (theorem) with no hypotheses.
Typically its label has no special suffix.
An example is \texttt{unss}, which states:
$\vdash ( ( A \subseteq C \wedge B \subseteq C ) \leftrightarrow ( A \cup B )
\subseteq C )\label{eq:unss}$
\item \textbf{deduction form}\index{deduction form}\index{forms!deduction}:
A kind of assertion with one or more hypotheses
where the conclusion is an implication with
a wff variable as the antecedent (usually $\varphi$), and every hypothesis
(\$e statement)
is either (1) an implication with the same antecedent as the conclusion or
(2) a definition.
A definition
can be for a class variable (this is a class variable followed by ``='')
or a wff variable (this is a wff variable followed by $\leftrightarrow$);
class variable definitions are more common.
In practice, a proof
in deduction form will also contain many steps that are implications
where the antecedent is either that wff variable (normally $\varphi$)
or is
a conjunction (...$\land$...) including that wff variable ($\varphi$).
If an assertion is in deduction form, and other forms are also available,
then we suffix its label with ``d.''
An example is \texttt{unssd}, which states\footnote{
For brevity we show here (and in other places)
a $\&$\index{$\&$} between hypotheses\index{hypotheses}
and a $\Rightarrow$\index{$\Rightarrow$}\index{conclusion}
between the hypotheses and the conclusion.
This notation is technically not part of the Metamath language, but is
instead a convenient abbreviation to show both the hypotheses and conclusion.}:
$\vdash ( \varphi \rightarrow A \subseteq C )\quad\&\quad \vdash ( \varphi
    \rightarrow B \subseteq C )\quad\Rightarrow\quad \vdash ( \varphi
    \rightarrow ( A \cup B ) \subseteq C )\label{eq:unssd}$
\item \textbf{inference form}\index{inference form}\index{forms!inference}:
A kind of assertion with one or more hypotheses that is not in deduction form
(e.g., there is no common antecedent).
If an assertion is in inference form, and other forms are also available,
then we suffix its label with ``i.''
An example is \texttt{unssi}, which states:
$\vdash A \subseteq C\quad\&\quad \vdash B \subseteq C\quad\Rightarrow\quad
    \vdash ( A \cup B ) \subseteq C\label{eq:unssi}$
\end{itemize}

When using deduction style we express an assertion in deduction form.
This form prefixes each hypothesis (other than definitions) and the
conclusion with a universal antecedent (``$\varphi \rightarrow$'').
The antecedent (e.g., $\varphi$)
mimics the context handled in the deduction theorem, eliminating
the need to directly use the deduction theorem.

Once you have an assertion in deduction form, you can easily convert it
to inference form or closed form:

\begin{itemize}
\item To
prove some assertion Ti in inference form, given assertion Td in deduction
form, there is a simple mechanical process you can use. First take each
Ti hypothesis and insert a \texttt{T.} $\rightarrow$ prefix (``true implies'')
using \texttt{a1i}. You
can then use the existing assertion Td to prove the resulting conclusion
with a \texttt{T.} $\rightarrow$ prefix.
Finally, you can remove that prefix using \texttt{trud},
resulting in the conclusion you wanted to prove.
\item To
prove some assertion T in closed form, given assertion Td in deduction
form, there is another simple mechanical process you can use. First,
select an expression that is the conjunction (...$\land$...) of all of the
consequents of every hypothesis of Td. Next, prove that this expression
implies each of the separate hypotheses of Td in turn by eliminating
conjuncts (there are a variety of proven assertions to do this, including
\texttt{simpl},
\texttt{simpr},
\texttt{3simpa},
\texttt{3simpb},
\texttt{3simpc},
\texttt{simp1},
\texttt{simp2},
and
\texttt{simp3}).
If the
expression has nested conjunctions, inner conjuncts can be broken out by
chaining the above theorems with \texttt{syl}
(see section \ref{syl}).\footnote{
There are actually many theorems
(labeled simp* such as \texttt{simp333}) that break out inner conjuncts in one
step, but rather than learning them you can just use the chaining we
just described to prove them, and then let the Metamath program command
\texttt{minimize{\char`\_}with}\index{\texttt{minimize{\char`\_}with} command}
figure out the right ones needed to collapse them.}
As your final step, you can then apply the already-proven assertion Td
(which is in deduction form), proving assertion T in closed form.
\end{itemize}

We can also easily convert any assertion T in closed form to its related
assertion Ti in inference form by applying
modus ponens\index{modus ponens} (see section \ref{axmp}).

The deduction form antecedent can also be used to represent the context
necessary to support natural deduction systems, so we will now
discuss natural deduction.

\subsection{Natural Deduction}\label{naturaldeduction}

Natural deduction\index{natural deduction}
(ND) systems, as such, were originally introduced in
1934 by two logicians working independently: Ja\'skowski and Gentzen. ND
systems are supposed to reconstruct, in a formally proper way, traditional
ways of mathematical reasoning (such as conditional proof, indirect proof,
and proof by cases). As reconstructions they were naturally influenced
by previous work, and many specific ND systems and notations have been
developed since their original work.

There are many ND variants, but
Indrzejczak \cite[p.~31-32]{Indrzejczak}\index{Indrzejczak, Andrzej}
suggests that any natural deductive system must satisfy at
least these three criteria:

\begin{itemize}
\item ``There are some means for entering assumptions into a proof and
also for eliminating them. Usually it requires some bookkeeping devices
for indicating the scope of an assumption, and showing that a part of
a proof depending on eliminated assumption is discharged.
\item There are no (or, at least, a very limited set of) axioms, because
their role is taken over by the set of primitive rules for introduction
and elimination of logical constants which means that elementary
inferences instead of formulae are taken as primitive.
\item (A genuine) ND system admits a lot of freedom in proof construction
and possibility of applying several proof search strategies, like
conditional proof, proof by cases, proof by reductio ad absurdum etc.''
\end{itemize}

The Metamath Proof Explorer (MPE) as defined in \texttt{set.mm}
is fundamentally a Hilbert-style system.
That is, MPE is based on a larger number of axioms (compared
to natural deduction systems), a very small set of rules of inference
(modus ponens), and the context is not changed by the rules of inference
in the middle of a proof. That said, MPE proofs can be developed using
the natural deduction (ND) approach as originally developed by Ja\'skowski
and Gentzen.

The most common and recommended approach for applying ND in MPE is to use
deduction form\index{deduction form}%
\index{forms!deduction}
and apply the MPE proven assertions that are equivalent to ND rules.
For example, MPE's \texttt{jca} is equivalent to ND rule $\land$-I
(and-insertion).
We maintain a list of equivalences that you may consult.
This approach for applying an ND approach within MPE relies on Metamath's
wff metavariables in an essential way, and is described in more detail
in the presentation ``Natural Deductions in the Metamath Proof Language''
by Mario Carneiro \cite{CarneiroND}\index{Carneiro, Mario}.

In this style many steps are an implication, whose antecedent mimics
the context ($\Gamma$) of most ND systems. To add an assumption, simply add
it to the implication antecedent (typically using
\texttt{simpr}),
and use that
new antecedent for all later claims in the same scope. If you wish to
use an assertion in an ND hypothesis scope that is outside the current
ND hypothesis scope, modify the assertion so that the ND hypothesis
assumption is added to its antecedent (typically using \texttt{adantr}). Most
proof steps will be proved using rules that have hypotheses and results
of the form $\varphi \rightarrow$ ...

An example may make this clearer.
Let's show theorem 5.5 of
\cite[p.~18]{Clemente}\index{Clemente Laboreo, Daniel}
along with a line by line translation using the usual
translation of natural deduction (ND) in the Metamath Proof Explorer
(MPE) notation (this is proof \texttt{ex-natded5.5}).
The proof's original goal was to prove
$\lnot \psi$ given two hypotheses,
$( \psi \rightarrow \chi )$ and $ \lnot \chi$.
We will translate these statements into MPE deduction form
by prefixing them all with $\varphi \rightarrow$.
As a result, in MPE the goal is stated as
$( \varphi \rightarrow \lnot \psi )$, and the two hypotheses are stated as
$( \varphi \rightarrow ( \psi \rightarrow \chi ) )$ and
$( \varphi \rightarrow \lnot \chi )$.

The following table shows the proof in Fitch natural deduction style
and its MPE equivalent.
The \textit{\#} column shows the original numbering,
\textit{MPE\#} shows the number in the equivalent MPE proof
(which we will show later),
\textit{ND Expression} shows the original proof claim in ND notation,
and \textit{MPE Translation} shows its translation into MPE
as discussed in this section.
The final columns show the rationale in ND and MPE respectively.

\needspace{4\baselineskip}
{\setlength{\extrarowsep}{4pt} % Keep rows from being too close together
\begin{longtabu}   { @{} c c X X X X }
\textbf{\#} & \textbf{MPE\#} & \textbf{ND Ex\-pres\-sion} &
\textbf{MPE Trans\-lation} & \textbf{ND Ration\-ale} &
\textbf{MPE Ra\-tio\-nale} \\
\endhead

1 & 2;3 &
$( \psi \rightarrow \chi )$ &
$( \varphi \rightarrow ( \psi \rightarrow \chi ) )$ &
Given &
\$e; \texttt{adantr} to put in ND hypothesis \\

2 & 5 &
$ \lnot \chi$ &
$( \varphi \rightarrow \lnot \chi )$ &
Given &
\$e; \texttt{adantr} to put in ND hypothesis \\

3 & 1 &
... $\vert$ $\psi$ &
$( \varphi \rightarrow \psi )$ &
ND hypothesis assumption &
\texttt{simpr} \\

4 & 4 &
... $\chi$ &
$( ( \varphi \land \psi ) \rightarrow \chi )$ &
$\rightarrow$\,E 1,3 &
\texttt{mpd} 1,3 \\

5 & 6 &
... $\lnot \chi$ &
$( ( \varphi \land \psi ) \rightarrow \lnot \chi )$ &
IT 2 &
\texttt{adantr} 5 \\

6 & 7 &
$\lnot \psi$ &
$( \varphi \rightarrow \lnot \psi )$ &
$\land$\,I 3,4,5 &
\texttt{pm2.65da} 4,6 \\

\end{longtabu}
}


The original used Latin letters; we have replaced them with Greek letters
to follow Metamath naming conventions and so that it is easier to follow
the Metamath translation. The Metamath line-for-line translation of
this natural deduction approach precedes every line with an antecedent
including $\varphi$ and uses the Metamath equivalents of the natural deduction
rules. To add an assumption, the antecedent is modified to include it
(typically by using \texttt{adantr};
\texttt{simpr} is useful when you want to
depend directly on the new assumption, as is shown here).

In Metamath we can represent the two given statements as these hypotheses:

\needspace{2\baselineskip}
\begin{itemize}
\item ex-natded5.5.1 $\vdash ( \varphi \rightarrow ( \psi \rightarrow \chi ) )$
\item ex-natded5.5.2 $\vdash ( \varphi \rightarrow \lnot \chi )$
\end{itemize}

\needspace{4\baselineskip}
Here is the proof in Metamath as a line-by-line translation:

\begin{longtabu}   { l l l X }
\textbf{Step} & \textbf{Hyp} & \textbf{Ref} & \textbf{Ex\-pres\-sion} \\
\endhead
1 & & simpr & $\vdash ( ( \varphi \land \psi ) \rightarrow \psi )$ \\
2 & & ex-natded5.5.1 &
  $\vdash ( \varphi \rightarrow ( \psi \rightarrow \chi ) )$ \\
3 & 2 & adantr &
 $\vdash ( ( \varphi \land \psi ) \rightarrow ( \psi \rightarrow \chi ) )$ \\
4 & 1, 3 & mpd &
 $\vdash ( ( \varphi \land \psi ) \rightarrow \chi ) $ \\
5 & & ex-natded5.5.2 &
 $\vdash ( \varphi \rightarrow \lnot \chi )$ \\
6 & 5 & adantr &
 $\vdash ( ( \varphi \land \psi ) \rightarrow \lnot \chi )$ \\
7 & 4, 6 & pm2.65da &
 $\vdash ( \varphi \rightarrow \lnot \psi )$ \\
\end{longtabu}

Only using specific natural deduction rules directly can lead to very
long proofs, for exactly the same reason that only using axioms directly
in Hilbert-style proofs can lead to very long proofs.
If the goal is short and clear proofs,
then it is better to reuse already-proven assertions
in deduction form than to start from scratch each time
and using only basic natural deduction rules.

\subsection{Strengths of Our Approach}

As far as we know there is nothing else in the literature like either the
weak deduction theorem or Mario Carneiro\index{Carneiro, Mario}'s
natural deduction method.
In order to
transform a hypothesis into an antecedent, the literature's standard
``Deduction Theorem''\index{Deduction Theorem}\index{Standard Deduction Theorem}
requires metalogic outside of the notions provided
by the axiom system. We instead generally prefer to use Mario Carneiro's
natural deduction method, then use the weak deduction theorem in cases
where that is difficult to apply, and only then use the full standard
deduction theorem as a last resort.

The weak deduction theorem\index{Weak Deduction Theorem}
does not require any additional metalogic
but converts an inference directly into a closed form theorem, with
a rigorous proof that uses only the axiom system. Unlike the standard
Deduction Theorem, there is no implicit external justification that we
have to trust in order to use it.

Mario Carneiro's natural deduction\index{natural deduction}
method also does not require any new metalogical
notions. It avoids the Deduction Theorem's metalogic by prefixing the
hypotheses and conclusion of every would-be inference with a universal
antecedent (``$\varphi \rightarrow$'') from the very start.

We think it is impressive and satisfying that we can do so much in a
practical sense without stepping outside of our Hilbert-style axiom system.
Of course our axiomatization, which is in the form of schemes,
contains a metalogic of its own that we exploit. But this metalogic
is relatively simple, and for our Deduction Theorem alternatives,
we primarily use just the direct substitution of expressions for
metavariables.

\begin{sloppy}
\section{Exploring the Set The\-o\-ry Data\-base}\label{exploring}
\end{sloppy}
% NOTE: All examples performed in this section are
% recorded wtih "set width 61" % on set.mm as of 2019-05-28
% commit c1e7849557661260f77cfdf0f97ac4354fbb4f4d.

At this point you may wish to study the \texttt{set.mm}\index{set theory
database (\texttt{set.mm})} file in more detail.  Pay particular
attention to the assumptions needed to define wffs\index{well-formed
formula (wff)} (which are not included above), the variable types
(\texttt{\$f}\index{\texttt{\$f} statement} statements), and the
definitions that are introduced.  Start with some simple theorems in
propositional calculus, making sure you understand in detail each step
of a proof.  Once you get past the first few proofs and become familiar
with the Metamath language, any part of the \texttt{set.mm} database
will be as easy to follow, step by step, as any other part---you won't
have to undergo a ``quantum leap'' in mathematical sophistication to be
able to follow a deep proof in set theory.

Next, you may want to explore how concepts such as natural numbers are
defined and described.  This is probably best done in conjunction with
standard set theory textbooks, which can help give you a higher-level
understanding.  The \texttt{set.mm} database provides references that will get
you started.  From there, you will be on your way towards a very deep,
rigorous understanding of abstract mathematics.

The Metamath\index{Metamath} program can help you peruse a Metamath data\-base,
wheth\-er you are trying to figure out how a certain step follows in a proof or
just have a general curiosity.  We will go through some examples of the
commands, using the \texttt{set.mm}\index{set theory database (\texttt{set.mm})}
database provided with the Metamath software.  These should help get you
started.  See Chapter~\ref{commands} for a more detailed description of
the commands.  Note that we have included the full spelling of all commands to
prevent ambiguity with future commands.  In practice you may type just the
characters needed to specify each command keyword\index{command keyword}
unambiguously, often just one or two characters per keyword, and you don't
need to type them in upper case.

First run the Metamath program as described earlier.  You should see the
\verb/MM>/ prompt.  Read in the \texttt{set.mm} file:\index{\texttt{read}
command}

\begin{verbatim}
MM> read set.mm
Reading source file "set.mm"... 34554442 bytes
34554442 bytes were read into the source buffer.
The source has 155711 statements; 2254 are $a and 32250 are $p.
No errors were found.  However, proofs were not checked.
Type VERIFY PROOF * if you want to check them.
\end{verbatim}

As with most examples in this book, what you will see
will be slightly different because we are continuously
improving our databases (including \texttt{set.mm}).

Let's check the database integrity.  This may take a minute or two to run if
your computer is slow.

\begin{verbatim}
MM> verify proof *
0 10%  20%  30%  40%  50%  60%  70%  80%  90% 100%
..................................................
All proofs in the database were verified in 2.84 s.
\end{verbatim}

No errors were reported, so every proof is correct.

You need to know the names (labels) of theorems before you can look at them.
Often just examining the database file(s) with a text editor is the best
approach.  In \texttt{set.mm} there are many detailed comments, especially near
the beginning, that can help guide you. The \texttt{search} command in the
Metamath program is also handy.  The \texttt{comments} qualifier will list the
statements whose associated comment (the one immediately before it) contain a
string you give it.  For example, if you are studying Enderton's {\em Elements
of Set Theory} \cite{Enderton}\index{Enderton, Herbert B.} you may want to see
the references to it in the database.  The search string \texttt{enderton} is not
case sensitive.  (This will not show you all the database theorems that are in
Enderton's book because there is usually only one citation for a given
theorem, which may appear in several textbooks.)\index{\texttt{search}
command}

\begin{verbatim}
MM> search * "enderton" / comments
12067 unineq $p "... Exercise 20 of [Enderton] p. 32 and ..."
12459 undif2 $p "...Corollary 6K of [Enderton] p. 144. (C..."
12953 df-tp $a "...s. Definition of [Enderton] p. 19. (Co..."
13689 unissb $p ".... Exercise 5 of [Enderton] p. 26 and ..."
\end{verbatim}
\begin{center}
(etc.)
\end{center}

Or you may want to see what theorems have something to do with
conjunction (logical {\sc and}).  The quotes around the search
string are optional when there's no ambiguity.\index{\texttt{search}
command}

\begin{verbatim}
MM> search * conjunction / comments
120 a1d $p "...be replaced with a conjunction ( ~ df-an )..."
662 df-bi $a "...viated form after conjunction is introdu..."
1319 wa $a "...ff definition to include conjunction ('and')."
1321 df-an $a "Define conjunction (logical 'and'). Defini..."
1420 imnan $p "...tion in terms of conjunction. (Contribu..."
\end{verbatim}
\begin{center}
(etc.)
\end{center}


Now we will start to look at some details.  Let's look at the first
axiom of propositional calculus
(we could use \texttt{sh st} to abbreviate
\texttt{show statement}).\index{\texttt{show statement} command}

\begin{verbatim}
MM> show statement ax-1/full
Statement 19 is located on line 881 of the file "set.mm".
"Axiom _Simp_.  Axiom A1 of [Margaris] p. 49.  One of the 3
axioms of propositional calculus.  The 3 axioms are also
        ...
19 ax-1 $a |- ( ph -> ( ps -> ph ) ) $.
Its mandatory hypotheses in RPN order are:
  wph $f wff ph $.
  wps $f wff ps $.
The statement and its hypotheses require the variables:  ph
      ps
The variables it contains are:  ph ps


Statement 49 is located on line 11182 of the file "set.mm".
Its statement number for HTML pages is 6.
"Axiom _Simp_.  Axiom A1 of [Margaris] p. 49.  One of the 3
axioms of propositional calculus.  The 3 axioms are also
given as Definition 2.1 of [Hamilton] p. 28.
...
49 ax-1 $a |- ( ph -> ( ps -> ph ) ) $.
Its mandatory hypotheses in RPN order are:
  wph $f wff ph $.
  wps $f wff ps $.
The statement and its hypotheses require the variables:
  ph ps
The variables it contains are:  ph ps
\end{verbatim}

Compare this to \texttt{ax-1} on p.~\pageref{ax1}.  You can see that the
symbol \texttt{ph} is the {\sc ascii} notation for $\varphi$, etc.  To
see the mathematical symbols for any expression you may typeset it in
\LaTeX\ (type \texttt{help tex} for instructions)\index{latex@{\LaTeX}}
or, easier, just use a text editor to look at the comments where symbols
are first introduced in \texttt{set.mm}.  The hypotheses \texttt{wph}
and \texttt{wps} required by \texttt{ax-1} mean that variables
\texttt{ph} and \texttt{ps} must be wffs.

Next we'll pick a simple theorem of propositional calculus, the Principle of
Identity, which is proved directly from the axioms.  We'll look at the
statement then its proof.\index{\texttt{show statement}
command}

\begin{verbatim}
MM> show statement id1/full
Statement 116 is located on line 11371 of the file "set.mm".
Its statement number for HTML pages is 22.
"Principle of identity.  Theorem *2.08 of [WhiteheadRussell]
p. 101.  This version is proved directly from the axioms for
demonstration purposes.
...
116 id1 $p |- ( ph -> ph ) $= ... $.
Its mandatory hypotheses in RPN order are:
  wph $f wff ph $.
Its optional hypotheses are:  wps wch wth wta wet
      wze wsi wrh wmu wla wka
The statement and its hypotheses require the variables:  ph
These additional variables are allowed in its proof:
      ps ch th ta et ze si rh mu la ka
The variables it contains are:  ph
\end{verbatim}

The optional variables\index{optional variable} \texttt{ps}, \texttt{ch}, etc.\ are
available for use in a proof of this statement if we wish, and were we to do
so we would make use of optional hypotheses \texttt{wps}, \texttt{wch}, etc.  (See
Section~\ref{dollaref} for the meaning of ``optional
hypothesis.''\index{optional hypothesis}) The reason these show up in the
statement display is that statement \texttt{id1} happens to be in their scope
(see Section~\ref{scoping} for the definition of ``scope''\index{scope}), but
in fact in propositional calculus we will never make use of optional
hypotheses or variables.  This becomes important after quantifiers are
introduced, where ``dummy'' variables\index{dummy variable} are often needed
in the middle of a proof.

Let's look at the proof of statement \texttt{id1}.  We'll use the
\texttt{show proof} command, which by default suppresses the
``non-essential'' steps that construct the wffs.\index{\texttt{show proof}
command}
We will display the proof in ``lemmon' format (a non-indented format
with explicit previous step number references) and renumber the
displayed steps:

\begin{verbatim}
MM> show proof id1 /lemmon/renumber
1 ax-1           $a |- ( ph -> ( ph -> ph ) )
2 ax-1           $a |- ( ph -> ( ( ph -> ph ) -> ph ) )
3 ax-2           $a |- ( ( ph -> ( ( ph -> ph ) -> ph ) ) ->
                     ( ( ph -> ( ph -> ph ) ) -> ( ph -> ph )
                                                          ) )
4 2,3 ax-mp      $a |- ( ( ph -> ( ph -> ph ) ) -> ( ph -> ph
                                                          ) )
5 1,4 ax-mp      $a |- ( ph -> ph )
\end{verbatim}

If you have read Section~\ref{trialrun}, you'll know how to interpret this
proof.  Step~2, for example, is an application of axiom \texttt{ax-1}.  This
proof is identical to the one in Hamilton's {\em Logic for Mathematicians}
\cite[p.~32]{Hamilton}\index{Hamilton, Alan G.}.

You may want to look at what
substitutions\index{substitution!variable}\index{variable substitution} are
made into \texttt{ax-1} to arrive at step~2.  The command to do this needs to
know the ``real'' step number, so we'll display the proof again without
the \texttt{renumber} qualifier.\index{\texttt{show proof}
command}

\begin{verbatim}
MM> show proof id1 /lemmon
 9 ax-1          $a |- ( ph -> ( ph -> ph ) )
20 ax-1          $a |- ( ph -> ( ( ph -> ph ) -> ph ) )
24 ax-2          $a |- ( ( ph -> ( ( ph -> ph ) -> ph ) ) ->
                     ( ( ph -> ( ph -> ph ) ) -> ( ph -> ph )
                                                          ) )
25 20,24 ax-mp   $a |- ( ( ph -> ( ph -> ph ) ) -> ( ph -> ph
                                                          ) )
26 9,25 ax-mp    $a |- ( ph -> ph )
\end{verbatim}

The ``real'' step number is 20.  Let's look at its details.

\begin{verbatim}
MM> show proof id1 /detailed_step 20
Proof step 20:  min=ax-1 $a |- ( ph -> ( ( ph -> ph ) -> ph )
  )
This step assigns source "ax-1" ($a) to target "min" ($e).
The source assertion requires the hypotheses "wph" ($f, step
18) and "wps" ($f, step 19).  The parent assertion of the
target hypothesis is "ax-mp" ($a, step 25).
The source assertion before substitution was:
    ax-1 $a |- ( ph -> ( ps -> ph ) )
The following substitutions were made to the source
assertion:
    Variable  Substituted with
     ph        ph
     ps        ( ph -> ph )
The target hypothesis before substitution was:
    min $e |- ph
The following substitution was made to the target hypothesis:
    Variable  Substituted with
     ph        ( ph -> ( ( ph -> ph ) -> ph ) )
\end{verbatim}

This shows the substitutions\index{substitution!variable}\index{variable
substitution} made to the variables in \texttt{ax-1}.  References are made to
steps 18 and 19 which are not shown in our proof display.  To see these steps,
you can display the proof with the \texttt{all} qualifier.

Let's look at a slightly more advanced proof of propositional calculus.  Note
that \verb+/\+ is the symbol for $\wedge$ (logical {\sc and}, also
called conjunction).\index{conjunction ($\wedge$)}
\index{logical {\sc and} ($\wedge$)}

\begin{verbatim}
MM> show statement prth/full
Statement 1791 is located on line 15503 of the file "set.mm".
Its statement number for HTML pages is 559.
"Conjoin antecedents and consequents of two premises.  This
is the closed theorem form of ~ anim12d .  Theorem *3.47 of
[WhiteheadRussell] p. 113.  It was proved by Leibniz,
and it evidently pleased him enough to call it
_praeclarum theorema_ (splendid theorem).
...
1791 prth $p |- ( ( ( ph -> ps ) /\ ( ch -> th ) ) -> ( ( ph
      /\ ch ) -> ( ps /\ th ) ) ) $= ... $.
Its mandatory hypotheses in RPN order are:
  wph $f wff ph $.
  wps $f wff ps $.
  wch $f wff ch $.
  wth $f wff th $.
Its optional hypotheses are:  wta wet wze wsi wrh wmu wla wka
The statement and its hypotheses require the variables:  ph
      ps ch th
These additional variables are allowed in its proof:  ta et
      ze si rh mu la ka
The variables it contains are:  ph ps ch th


MM> show proof prth /lemmon/renumber
1 simpl          $p |- ( ( ( ph -> ps ) /\ ( ch -> th ) ) ->
                                               ( ph -> ps ) )
2 simpr          $p |- ( ( ( ph -> ps ) /\ ( ch -> th ) ) ->
                                               ( ch -> th ) )
3 1,2 anim12d    $p |- ( ( ( ph -> ps ) /\ ( ch -> th ) ) ->
                           ( ( ph /\ ch ) -> ( ps /\ th ) ) )
\end{verbatim}

There are references to a lot of unfamiliar statements.  To see what they are,
you may type the following:

\begin{verbatim}
MM> show proof prth /statement_summary
Summary of statements used in the proof of "prth":

Statement simpl is located on line 14748 of the file
"set.mm".
"Elimination of a conjunct.  Theorem *3.26 (Simp) of
[WhiteheadRussell] p. 112. ..."
  simpl $p |- ( ( ph /\ ps ) -> ph ) $= ... $.

Statement simpr is located on line 14777 of the file
"set.mm".
"Elimination of a conjunct.  Theorem *3.27 (Simp) of
[WhiteheadRussell] ..."
  simpr $p |- ( ( ph /\ ps ) -> ps ) $= ... $.

Statement anim12d is located on line 15445 of the file
"set.mm".
"Conjoin antecedents and consequents in a deduction.
..."
  anim12d.1 $e |- ( ph -> ( ps -> ch ) ) $.
  anim12d.2 $e |- ( ph -> ( th -> ta ) ) $.
  anim12d $p |- ( ph -> ( ( ps /\ th ) -> ( ch /\ ta ) ) )
      $= ... $.
\end{verbatim}
\begin{center}
(etc.)
\end{center}

Of course you can look at each of these statements and their proofs, and
so on, back to the axioms of propositional calculus if you wish.

The \texttt{search} command is useful for finding statements when you
know all or part of their contents.  The following example finds all
statements containing \verb@ph -> ps@ followed by \verb@ch -> th@.  The
\verb@$*@ is a wildcard that matches anything; the \texttt{\$} before the
\verb$*$ prevents conflicts with math symbol token names.  The \verb@*@ after
\texttt{SEARCH} is also a wildcard that in this case means ``match any label.''
\index{\texttt{search} command}

% I'm omitting this one, since readers are unlikely to see it:
% 1096 bisymOLD $p |- ( ( ( ph -> ps ) -> ( ch -> th ) ) -> ( (
%   ( ps -> ph ) -> ( th -> ch ) ) -> ( ( ph <-> ps ) -> ( ch
%    <-> th ) ) ) )
\begin{verbatim}
MM> search * "ph -> ps $* ch -> th"
1791 prth $p |- ( ( ( ph -> ps ) /\ ( ch -> th ) ) -> ( ( ph
    /\ ch ) -> ( ps /\ th ) ) )
2455 pm3.48 $p |- ( ( ( ph -> ps ) /\ ( ch -> th ) ) -> ( (
    ph \/ ch ) -> ( ps \/ th ) ) )
117859 pm11.71 $p |- ( ( E. x ph /\ E. y ch ) -> ( ( A. x (
    ph -> ps ) /\ A. y ( ch -> th ) ) <-> A. x A. y ( ( ph /\
    ch ) -> ( ps /\ th ) ) ) )
\end{verbatim}

Three statements, \texttt{prth}, \texttt{pm3.48},
 and \texttt{pm11.71}, were found to match.

To see what axioms\index{axiom} and definitions\index{definition}
\texttt{prth} ultimately depends on for its proof, you can have the
program backtrack through the hierarchy\index{hierarchy} of theorems and
definitions.\index{\texttt{show trace{\char`\_}back} command}

\begin{verbatim}
MM> show trace_back prth /essential/axioms
Statement "prth" assumes the following axioms ($a
statements):
  ax-1 ax-2 ax-3 ax-mp df-bi df-an
\end{verbatim}

Note that the 3 axioms of propositional calculus and the modus ponens rule are
needed (as expected); in addition, there are a couple of definitions that are used
along the way.  Note that Metamath makes no distinction\index{axiom vs.\
definition} between axioms\index{axiom} and definitions\index{definition}.  In
\texttt{set.mm} they have been distinguished artificially by prefixing their
labels\index{labels in \texttt{set.mm}} with \texttt{ax-} and \texttt{df-}
respectively.  For example, \texttt{df-an} defines conjunction (logical {\sc
and}), which is represented by the symbol \verb+/\+.
Section~\ref{definitions} discusses the philosophy of definitions, and the
Metamath language takes a particularly simple, conservative approach by using
the \texttt{\$a}\index{\texttt{\$a} statement} statement for both axioms and
definitions.

You can also have the program compute how many steps a proof
has\index{proof length} if we were to follow it all the way back to
\texttt{\$a} statements.

\begin{verbatim}
MM> show trace_back prth /essential/count_steps
The statement's actual proof has 3 steps.  Backtracking, a
total of 79 different subtheorems are used.  The statement
and subtheorems have a total of 274 actual steps.  If
subtheorems used only once were eliminated, there would be a
total of 38 subtheorems, and the statement and subtheorems
would have a total of 185 steps.  The proof would have 28349
steps if fully expanded back to axiom references.  The
maximum path length is 38.  A longest path is:  prth <-
anim12d <- syl2and <- sylan2d <- ancomsd <- ancom <- pm3.22
<- pm3.21 <- pm3.2 <- ex <- sylbir <- biimpri <- bicomi <-
bicom1 <- bi2 <- dfbi1 <- impbii <- bi3 <- simprim <- impi <-
con1i <- nsyl2 <- mt3d <- con1d <- notnot1 <- con2i <- nsyl3
<- mt2d <- con2d <- notnot2 <- pm2.18d <- pm2.18 <- pm2.21 <-
pm2.21d <- a1d <- syl <- mpd <- a2i <- a2i.1 .
\end{verbatim}

This tells us that we would have to inspect 274 steps if we want to
verify the proof completely starting from the axioms.  A few more
statistics are also shown.  There are one or more paths back to axioms
that are the longest; this command ferrets out one of them and shows it
to you.  There may be a sense in which the longest path length is
related to how ``deep'' the theorem is.

We might also be curious about what proofs depend on the theorem
\texttt{prth}.  If it is never used later on, we could eliminate it as
redundant if it has no intrinsic interest by itself.\index{\texttt{show
usage} command}

% I decided to show the OLD values here.
\begin{verbatim}
MM> show usage prth
Statement "prth" is directly referenced in the proofs of 18
statements:
  mo3 moOLD 2mo 2moOLD euind reuind reuss2 reusv3i opelopabt
  wemaplem2 rexanre rlimcn2 o1of2 o1rlimmul 2sqlem6 spanuni
  heicant pm11.71
\end{verbatim}

Thus \texttt{prth} is directly used by 18 proofs.
We can use the \texttt{/recursive} qualifier to include indirect use:

\begin{verbatim}
MM> show usage prth /recursive
Statement "prth" directly or indirectly affects the proofs of
24214 statements:
  mo3 mo mo3OLD eu2 moOLD eu2OLD eu3OLD mo4f mo4 eu4 mopick
...
\end{verbatim}

\subsection{A Note on the ``Compact'' Proof Format}

The present version of Metamath\index{Metamath!limitations of
version 0.07.30} will display proofs in a ``compact''\index{compact proof}
format whenever the proof is stored in compressed format in the database.  It
may be be slightly confusing unless you know how to interpret it.  (A
future version may eliminate this format from displays.)  For example,
if you display the complete proof of theorem \texttt{id1} it will start
off as follows:

\begin{verbatim}
MM> show proof id1 /lemmon/all
 1 wph           $f wff ph
 2 wph           $f wff ph
 3 wph           $f wff ph
 4 2,3 wi    @4: $a wff ( ph -> ph )
 5 1,4 wi    @5: $a wff ( ph -> ( ph -> ph ) )
 6 @4            $a wff ( ph -> ph )
\end{verbatim}

\begin{center}
{etc.}
\end{center}

Step 4 has a ``local label,''\index{local label} \texttt{@4}, assigned to it.
Later on, at step 6, the label \texttt{@1} is referenced instead of
displaying the explicit proof for that step.  This technique takes advantage
of the fact that steps in a proof often repeat, especially during the
construction of wffs.  The compact format reduces the number of steps in the
proof display and may actually be preferred by some people, which is one
reason that it is still displayed by the program.

If you want to see the normal format with the ``true'' step numbers, you can
use the following workaround:\index{\texttt{save proof} command}

\begin{verbatim}
MM> save proof id1 /normal
The proof of "id1" has been reformatted and saved internally.
Remember to use WRITE SOURCE to save it permanently.
MM> show proof id1 /lemmon/all
 1 wph           $f wff ph
 2 wph           $f wff ph
 3 wph           $f wff ph
 4 2,3 wi        $a wff ( ph -> ph )
 5 1,4 wi        $a wff ( ph -> ( ph -> ph ) )
 6 wph           $f wff ph
 7 wph           $f wff ph
 8 6,7 wi        $a wff ( ph -> ph )
\end{verbatim}

\begin{center}
{etc.}
\end{center}

Note that the original 6 steps are now 8 steps.  However, the format is
now the same as that described in Chapter~\ref{using}.

\chapter{The Metamath Language}
\label{languagespec}

\begin{quote}
  {\em Thus mathematics may be defined as the subject in which we never know
what we are talking about, nor whether what we are saying is true.}
    \flushright\sc  Bertrand Russell\footnote{\cite[p.~84]{Russell2}.}\\
\end{quote}\index{Russell, Bertrand}

Probably the most striking feature of the Metamath language is its almost
complete absence of hard-wired syntax. Metamath\index{Metamath} does not
understand any mathematics or logic other than that needed to construct finite
sequences of symbols according to a small set of simple, built-in rules.  The
only rule it uses in a proof is the substitution of an expression (symbol
sequence) for a variable, subject to a simple constraint to prevent
bound-variable clashes.  The primitive notions built into Metamath involve the
simple manipulation of finite objects (symbols) that we as humans can easily
visualize and that computers can easily deal with.  They seem to be just
about the simplest notions possible that are required to do standard
mathematics.

This chapter serves as a reference manual for the Metamath\index{Metamath}
language. It covers the tedious technical details of the language, some of
which you may wish to skim in a first reading.  On the other hand, you should
pay close attention to the defined terms in {\bf boldface}; they have precise
meanings that are important to keep in mind for later understanding.  It may
be best to first become familiar with the examples in Chapter~\ref{using} to
gain some motivation for the language.

%% Uncomment this when uncommenting section {formalspec} below
If you have some knowledge of set theory, you may wish to study this
chapter in conjunction with the formal set-theoretical description of the
Metamath language in Appendix~\ref{formalspec}.

We will use the name ``Metamath''\index{Metamath} to mean either the Metamath
computer language or the Metamath software associated with the computer
language.  We will not distinguish these two when the context is clear.

The next section contains the complete specification of the Metamath
language.
It serves as an
authoritative reference and presents the syntax in enough detail to
write a parser\index{parsing Metamath} and proof verifier.  The
specification is terse and it is probably hard to learn the language
directly from it, but we include it here for those impatient people who
prefer to see everything up front before looking at verbose expository
material.  Later sections explain this material and provide examples.
We will repeat the definitions in those sections, and you may skip the
next section at first reading and proceed to Section~\ref{tut1}
(p.~\pageref{tut1}).

\section{Specification of the Metamath Language}\label{spec}
\index{Metamath!specification}

\begin{quote}
  {\em Sometimes one has to say difficult things, but one ought to say
them as simply as one knows how.}
    \flushright\sc  G. H. Hardy\footnote{As quoted in
    \cite{deMillo}, p.~273.}\\
\end{quote}\index{Hardy, G. H.}

\subsection{Preliminaries}\label{spec1}

% Space is technically a printable character, so we'll word things
% carefully so it's unambiguous.
A Metamath {\bf database}\index{database} is built up from a top-level source
file together with any source files that are brought in through file inclusion
commands (see below).  The only characters that are allowed to appear in a
Metamath source file are the 94 non-whitespace printable {\sc
ascii}\index{ascii@{\sc ascii}} characters, which are digits, upper and lower
case letters, and the following 32 special
characters\index{special characters}:\label{spec1chars}

\begin{verbatim}
! " # $ % & ' ( ) * + , - . / :
; < = > ? @ [ \ ] ^ _ ` { | } ~
\end{verbatim}
plus the following characters which are the ``white space'' characters:
space (a printable character),
tab, carriage return, line feed, and form feed.\label{whitespace}
We will use \texttt{typewriter}
font to display the printable characters.

A Metamath database consists of a sequence of three kinds of {\bf
tokens}\index{token} separated by {\bf white space}\index{white space}
(which is any sequence of one or more white space characters).  The set
of {\bf keyword}\index{keyword} tokens is \texttt{\$\char`\{},
\texttt{\$\char`\}}, \texttt{\$c}, \texttt{\$v}, \texttt{\$f},
\texttt{\$e}, \texttt{\$d}, \texttt{\$a}, \texttt{\$p}, \texttt{\$.},
\texttt{\$=}, \texttt{\$(}, \texttt{\$)}, \texttt{\$[}, and
\texttt{\$]}.  The last four are called {\bf auxiliary}\index{auxiliary
keyword} or preprocessing keywords.  A {\bf label}\index{label} token
consists of any combination of letters, digits, and the characters
hyphen, underscore, and period.  A {\bf math symbol}\index{math symbol}
token may consist of any combination of the 92 printable standard {\sc
ascii} characters other than space or \texttt{\$}~. All tokens are
case-sensitive.

\subsection{Preprocessing}

The token \texttt{\$(} begins a {\bf comment} and
\texttt{\$)} ends a comment.\index{\texttt{\$(}
and \texttt{\$)} auxiliary keywords}\index{comment}
Comments may contain any of
the 94 non-whitespace printable characters and white space,
except they may not contain the
2-character sequences \texttt{\$(} or \texttt{\$)} (comments do not nest).
Comments are ignored (treated
like white space) for the purpose of parsing, e.g.,
\texttt{\$( \$[ \$)} is a comment.
See p.~\pageref{mathcomments} for comment typesetting conventions; these
conventions may be ignored for the purpose of parsing.

A {\bf file inclusion command} consists of \texttt{\$[} followed by a file name
followed by \texttt{\$]}.\index{\texttt{\$[} and \texttt{\$]} auxiliary
keywords}\index{included file}\index{file inclusion}
It is only allowed in the outermost scope (i.e., not between
\texttt{\$\char`\{} and \texttt{\$\char`\}})
and must not be inside a statement (e.g., it may not occur
between the label of a \texttt{\$a} statement and its \texttt{\$.}).
The file name may not
contain a \texttt{\$} or white space.  The file must exist.
The case-sensitivity
of its name follows the conventions of the operating system.  The contents of
the file replace the inclusion command.
Included files may include other files.
Only the first reference to a given file is included; any later
references to the same file (whether in the top-level file or in included
files) cause the inclusion command to be ignored (treated like white space).
A verifier may assume that file names with different strings
refer to different files for the purpose of ignoring later references.
A file self-reference is ignored, as is any reference to the top-level file
(to avoid loops).
Included files may not include a \texttt{\$(} without a matching \texttt{\$)},
may not include a \texttt{\$[} without a matching \texttt{\$]}, and may
not include incomplete statements (e.g., a \texttt{\$a} without a matching
\texttt{\$.}).
It is currently unspecified if path references are relative to the process'
current directory or the file's containing directory, so databases should
avoid using pathname separators (e.g., ``/'') in file names.

Like all tokens, the \texttt{\$(}, \texttt{\$)}, \texttt{\$[}, and \texttt{\$]} keywords
must be surrounded by white space.

\subsection{Basic Syntax}

After preprocessing, a database will consist of a sequence of {\bf
statements}.
These are the scoping statements \texttt{\$\char`\{} and
\texttt{\$\char`\}}, along with the \texttt{\$c}, \texttt{\$v},
\texttt{\$f}, \texttt{\$e}, \texttt{\$d}, \texttt{\$a}, and \texttt{\$p}
statements.

A {\bf scoping statement}\index{scoping statement} consists only of its
keyword, \texttt{\$\char`\{} or \texttt{\$\char`\}}.
A \texttt{\$\char`\{} begins a {\bf
block}\index{block} and a matching \texttt{\$\char`\}} ends the block.
Every \texttt{\$\char`\{}
must have a matching \texttt{\$\char`\}}.
Defining it recursively, we say a block
contains a sequence of zero or more tokens other
than \texttt{\$\char`\{} and \texttt{\$\char`\}} and
possibly other blocks.  There is an {\bf outermost
block}\index{block!outermost} not bracketed by \texttt{\$\char`\{} \texttt{\$\char`\}}; the end
of the outermost block is the end of the database.

% LaTeX bug? can't do \bf\tt

A {\bf \$v} or {\bf \$c statement}\index{\texttt{\$v} statement}\index{\texttt{\$c}
statement} consists of the keyword token \texttt{\$v} or \texttt{\$c} respectively,
followed by one or more math symbols,
% The word "token" is used to distinguish "$." from the sentence-ending period.
followed by the \texttt{\$.}\ token.
These
statements {\bf declare}\index{declaration} the math symbols to be {\bf
variables}\index{variable!Metamath} or {\bf constants}\index{constant}
respectively. The same math symbol may not occur twice in a given \texttt{\$v} or
\texttt{\$c} statement.

%c%A math symbol becomes an {\bf active}\index{active math symbol}
%c%when declared and stays active until the end of the block in which it is
%c%declared.  A math symbol may not be declared a second time while it is active,
%c%but it may be declared again after it becomes inactive.

A math symbol becomes {\bf active}\index{active math symbol} when declared
and stays active until the end of the block in which it is declared.  A
variable may not be declared a second time while it is active, but it
may be declared again (as a variable, but not as a constant) after it
becomes inactive.  A constant must be declared in the outermost block and may
not be declared a second time.\footnote{The rules for redeclaration may
become more general in the future;
see p.~\pageref{redeclarationf}.}\index{redeclaration of symbols}

A {\bf \$f statement}\index{\texttt{\$f} statement} consists of a label,
followed by \texttt{\$f}, followed by its typecode (an active constant),
followed by an
active variable, followed by the \texttt{\$.}\ token.  A {\bf \$e
statement}\index{\texttt{\$e} statement} consists of a label, followed
by \texttt{\$e}, followed by its typecode (an active constant),
followed by zero or more
active math symbols, followed by the \texttt{\$.}\ token.  A {\bf
hypothesis}\index{hypothesis} is a \texttt{\$f} or \texttt{\$e}
statement.
The type declared by a \texttt{\$f} statement for a given label
is global even if the variable is not
(e.g., a database may not have \texttt{wff P} in one local scope
nd \texttt{class P} in another).

A {\bf simple \$d statement}\index{\texttt{\$d} statement!simple}
consists of \texttt{\$d}, followed by two different active variables,
followed by the \texttt{\$.}\ token.  A {\bf compound \$d
statement}\index{\texttt{\$d} statement!compound} consists of
\texttt{\$d}, followed by three or more variables (all different),
followed by the \texttt{\$.}\ token.  The order of the variables in a
\texttt{\$d} statement is unimportant.  A compound \texttt{\$d}
statement is equivalent to a set of simple \texttt{\$d} statements, one
for each possible pair of variables occurring in the compound
\texttt{\$d} statement.  Henceforth in this specification we shall
assume all \texttt{\$d} statements are simple.  A \texttt{\$d} statement
is also called a {\bf disjoint} (or {\bf distinct}) {\bf variable
restriction}.\index{disjoint-variable restriction}

A {\bf \$a statement}\index{\texttt{\$a} statement} consists of a label,
followed by \texttt{\$a}, followed by its typecode (an active constant),
followed by
zero or more active math symbols, followed by the \texttt{\$.}\ token.  A {\bf
\$p statement}\index{\texttt{\$p} statement} consists of a label,
followed by \texttt{\$p}, followed by its typecode (an active constant),
followed by
zero or more active math symbols, followed by \texttt{\$=}, followed by
a sequence of labels, followed by the \texttt{\$.}\ token.  An {\bf
assertion}\index{assertion} is a \texttt{\$a} or \texttt{\$p} statement.

A \texttt{\$f}, \texttt{\$e}, or \texttt{\$d} statement is {\bf active}\index{active
statement} from the place it occurs until the end of the block it occurs in.
A \texttt{\$a} or \texttt{\$p} statement is {\bf active} from the place it occurs
through the end of the database.
There may not be two active \texttt{\$f} statements containing the same
variable.  Each variable in a \texttt{\$e}, \texttt{\$a}, or
\texttt{\$p} statement must exist in an active \texttt{\$f}
statement.\footnote{This requirement can greatly simplify the
unification algorithm (substitution calculation) required by proof
verification.}

%The label that begins each \texttt{\$f}, \texttt{\$e}, \texttt{\$a}, and
%\texttt{\$p} statement must be unique.
Each label token must be unique, and
no label token may match any math symbol
token.\label{namespace}\footnote{This
restriction was added on June 24, 2006.
It is not theoretically necessary but is imposed to make it easier to
write certain parsers.}

The set of {\bf mandatory variables}\index{mandatory variable} associated with
an assertion is the set of (zero or more) variables in the assertion and in any
active \texttt{\$e} statements.  The (possibly empty) set of {\bf mandatory
hypotheses}\index{mandatory hypothesis} is the set of all active \texttt{\$f}
statements containing mandatory variables, together with all active \texttt{\$e}
statements.
The set of {\bf mandatory {\bf \$d} statements}\index{mandatory
disjoint-variable restriction} associated with an assertion are those active
\texttt{\$d} statements whose variables are both among the assertion's
mandatory variables.

\subsection{Proof Verification}\label{spec4}

The sequence of labels between the \texttt{\$=} and \texttt{\$.}\ tokens
in a \texttt{\$p} statement is a {\bf proof}.\index{proof!Metamath} Each
label in a proof must be the label of an active statement other than the
\texttt{\$p} statement itself; thus a label must refer either to an
active hypothesis of the \texttt{\$p} statement or to an earlier
assertion.

An {\bf expression}\index{expression} is a sequence of math symbols. A {\bf
substitution map}\index{substitution map} associates a set of variables with a
set of expressions.  It is acceptable for a variable to be mapped to an
expression containing it.  A {\bf
substitution}\index{substitution!variable}\index{variable substitution} is the
simultaneous replacement of all variables in one or more expressions with the
expressions that the variables map to.

A proof is scanned in order of its label sequence.  If the label refers to an
active hypothesis, the expression in the hypothesis is pushed onto a
stack.\index{stack}\index{RPN stack}  If the label refers to an assertion, a
(unique) substitution must exist that, when made to the mandatory hypotheses
of the referenced assertion, causes them to match the topmost (i.e.\ most
recent) entries of the stack, in order of occurrence of the mandatory
hypotheses, with the topmost stack entry matching the last mandatory
hypothesis of the referenced assertion.  As many stack entries as there are
mandatory hypotheses are then popped from the stack.  The same substitution is
made to the referenced assertion, and the result is pushed onto the stack.
After the last label in the proof is processed, the stack must have a single
entry that matches the expression in the \texttt{\$p} statement containing the
proof.

%c%{\footnotesize\begin{quotation}\index{redeclaration of symbols}
%c%{{\em Comment.}\label{spec4comment} Whenever a math symbol token occurs in a
%c%{\texttt{\$c} or \texttt{\$v} statement, it is considered to designate a distinct new
%c%{symbol, even if the same token was previously declared (and is now inactive).
%c%{Thus a math token declared as a constant in two different blocks is considered
%c%{to designate two distinct constants (even though they have the same name).
%c%{The two constants will not match in a proof that references both blocks.
%c%{However, a proof referencing both blocks is acceptable as long as it doesn't
%c%{require that the constants match.  Similarly, a token declared to be a
%c%{constant for a referenced assertion will not match the same token declared to
%c%{be a variable for the \texttt{\$p} statement containing the proof.  In the case
%c%{of a token declared to be a variable for a referenced assertion, this is not
%c%{an issue since the variable can be substituted with whatever expression is
%c%{needed to achieve the required match.
%c%{\end{quotation}}
%c2%A proof may reference an assertion that contains or whose hypotheses contain a
%c2%constant that is not active for the \texttt{\$p} statement containing the proof.
%c2%However, the final result of the proof may not contain that constant. A proof
%c2%may also reference an assertion that contains or whose hypotheses contain a
%c2%variable that is not active for the \texttt{\$p} statement containing the proof.
%c2%That variable, of course, will be substituted with whatever expression is
%c2%needed to achieve the required match.

A proof may contain a \texttt{?}\ in place of a label to indicate an unknown step
(Section~\ref{unknown}).  A proof verifier may ignore any proof containing
\texttt{?}\ but should warn the user that the proof is incomplete.

A {\bf compressed proof}\index{compressed proof}\index{proof!compressed} is an
alternate proof notation described in Appen\-dix~\ref{compressed}; also see
references to ``compressed proof'' in the Index.  Compressed proofs are a
Metamath language extension which a complete proof verifier should be able to
parse and verify.

\subsubsection{Verifying Disjoint Variable Restrictions}

Each substitution made in a proof must be checked to verify that any
disjoint variable restrictions are satisfied, as follows.

If two variables replaced by a substitution exist in a mandatory \texttt{\$d}
statement\index{\texttt{\$d} statement} of the assertion referenced, the two
expressions resulting from the substitution must satisfy the following
conditions.  First, the two expressions must have no variables in common.
Second, each possible pair of variables, one from each expression, must exist
in an active \texttt{\$d} statement of the \texttt{\$p} statement containing the
proof.

\vskip 1ex

This ends the specification of the Metamath language;
see Appendix \ref{BNF} for its syntax in
Extended Backus--Naur Form (EBNF)\index{Extended Backus--Naur Form}\index{EBNF}.

\section{The Basic Keywords}\label{tut1}

Our expository material begins here.

Like most computer languages, Metamath\index{Metamath} takes its input from
one or more {\bf source files}\index{source file} which contain characters
expressed in the standard {\sc ascii} (American Standard Code for Information
Interchange)\index{ascii@{\sc ascii}} code for computers.  A source file
consists of a series of {\bf tokens}\index{token}, which are strings of
non-whitespace
printable characters (from the set of 94 shown on p.~\pageref{spec1chars})
separated by {\bf white space}\index{white space} (spaces, tabs, carriage
returns, line feeds, and form feeds). Any string consisting only of these
characters is treated the same as a single space.  The non-whitespace printable
characters\index{printable character} that Metamath recognizes are the 94
characters on standard {\sc ascii} keyboards.

Metamath has the ability to join several files together to form its
input (Section~\ref{include}).  We call the aggregate contents of all
the files after they have been joined together a {\bf
database}\index{database} to distinguish it from an individual source
file.  The tokens in a database consist of {\bf
keywords}\index{keyword}, which are built into the language, together
with two kinds of user-defined tokens called {\bf labels}\index{label}
and {\bf math symbols}\index{math symbol}.  (Often we will simply say
{\bf symbol}\index{symbol} instead of math symbol for brevity).  The set
of {\bf basic keywords}\index{basic keyword} is
\texttt{\$c}\index{\texttt{\$c} statement},
\texttt{\$v}\index{\texttt{\$v} statement},
\texttt{\$e}\index{\texttt{\$e} statement},
\texttt{\$f}\index{\texttt{\$f} statement},
\texttt{\$d}\index{\texttt{\$d} statement},
\texttt{\$a}\index{\texttt{\$a} statement},
\texttt{\$p}\index{\texttt{\$p} statement},
\texttt{\$=}\index{\texttt{\$=} keyword},
\texttt{\$.}\index{\texttt{\$.}\ keyword},
\texttt{\$\char`\{}\index{\texttt{\$\char`\{} and \texttt{\$\char`\}}
keywords}, and \texttt{\$\char`\}}.  This is the complete set of
syntactical elements of what we call the {\bf basic
language}\index{basic language} of Metamath, and with them you can
express all of the mathematics that were intended by the design of
Metamath.  You should make it a point to become very familiar with them.
Table~\ref{basickeywords} lists the basic keywords along with a brief
description of their functions.  For now, this description will give you
only a vague notion of what the keywords are for; later we will describe
the keywords in detail.


\begin{table}[htp] \caption{Summary of the basic Metamath
keywords} \label{basickeywords}
\begin{center}
\begin{tabular}{|p{4pc}|l|}
\hline
\em \centering Keyword&\em Description\\
\hline
\hline
\centering
   \texttt{\$c}&Constant symbol declaration\\
\hline
\centering
   \texttt{\$v}&Variable symbol declaration\\
\hline
\centering
   \texttt{\$d}&Disjoint variable restriction\\
\hline
\centering
   \texttt{\$f}&Variable-type (``floating'') hypothesis\\
\hline
\centering
   \texttt{\$e}&Logical (``essential'') hypothesis\\
\hline
\centering
   \texttt{\$a}&Axiomatic assertion\\
\hline
\centering
   \texttt{\$p}&Provable assertion\\
\hline
\centering
   \texttt{\$=}&Start of proof in \texttt{\$p} statement\\
\hline
\centering
   \texttt{\$.}&End of the above statement types\\
\hline
\centering
   \texttt{\$\char`\{}&Start of block\\
\hline
\centering
   \texttt{\$\char`\}}&End of block\\
\hline
\end{tabular}
\end{center}
\end{table}

%For LaTeX bug(?) where it puts tables on blank page instead of btwn text
%May have to adjust if text changes
%\newpage

There are some additional keywords, called {\bf auxiliary
keywords}\index{auxiliary keyword} that help make Metamath\index{Metamath}
more practical. These are part of the {\bf extended language}\index{extended
language}. They provide you with a means to put comments into a Metamath
source file\index{source file} and reference other source files.  We will
introduce these in later sections. Table~\ref{otherkeywords} summarizes them
so that you can recognize them now if you want to peruse some source
files while learning the basic keywords.


\begin{table}[htp] \caption{Auxiliary Metamath
keywords} \label{otherkeywords}
\begin{center}
\begin{tabular}{|p{4pc}|l|}
\hline
\em \centering Keyword&\em Description\\
\hline
\hline
\centering
   \texttt{\$(}&Start of comment\\
\hline
\centering
   \texttt{\$)}&End of comment\\
\hline
\centering
   \texttt{\$[}&Start of included source file name\\
\hline
\centering
   \texttt{\$]}&End of included source file name\\
\hline
\end{tabular}
\end{center}
\end{table}
\index{\texttt{\$(} and \texttt{\$)} auxiliary keywords}
\index{\texttt{\$[} and \texttt{\$]} auxiliary keywords}


Unlike those in some computer languages, the keywords\index{keyword} are short
two-character sequences rather than English-like words.  While this may make
them slightly more difficult to remember at first, their brevity allows
them to blend in with the mathematics being described, not
distract from it, like punctuation marks.


\subsection{User-Defined Tokens}\label{dollardollar}\index{token}

As you may have noticed, all keywords\index{keyword} begin with the \texttt{\$}
character.  This mundane monetary symbol is not ordinarily used in higher
mathematics (outside of grant proposals), so we have appropriated it to
distinguish the Metamath\index{Metamath} keywords from ordinary mathematical
symbols. The \texttt{\$} character is thus considered special and may not be
used as a character in a user-defined token.  All tokens and keywords are
case-sensitive; for example, \texttt{n} is considered to be a different character
from \texttt{N}.  Case-sensitivity makes the available {\sc ascii} character set
as rich as possible.

\subsubsection{Math Symbol Tokens}\index{token}

Math symbols\index{math symbol} are tokens used to represent the symbols
that appear in ordinary mathematical formulas.  They may consist of any
combination of the 93 non-whitespace printable {\sc ascii} characters other than
\texttt{\$}~. Some examples are \texttt{x}, \texttt{+}, \texttt{(},
\texttt{|-}, \verb$!%@?&$, and \texttt{bounded}.  For readability, it is
best to try to make these look as similar to actual mathematical symbols
as possible, within the constraints of the {\sc ascii} character set, in
order to make the resulting mathematical expressions more readable.

In the Metamath\index{Metamath} language, you express ordinary
mathematical formulas and statements as sequences of math symbols such
as \texttt{2 + 2 = 4} (five symbols, all constants).\footnote{To
eliminate ambiguity with other expressions, this is expressed in the set
theory database \texttt{set.mm} as \texttt{|- ( 2 + 2
 ) = 4 }, whose \LaTeX\ equivalent is $\vdash
(2+2)=4$.  The \,$\vdash$ means ``is a theorem'' and the
parentheses allow explicit associative grouping.}\index{turnstile
({$\,\vdash$})} They may even be English
sentences, as in \texttt{E is closed and bounded} (five symbols)---here
\texttt{E} would be a variable and the other four symbols constants.  In
principle, a Metamath database could be constructed to work with almost
any unambiguous English-language mathematical statement, but as a
practical matter the definitions needed to provide for all possible
syntax variations would be cumbersome and distracting and possibly have
subtle pitfalls accidentally built in.  We generally recommend that you
express mathematical statements with compact standard mathematical
symbols whenever possible and put their English-language descriptions in
comments.  Axioms\index{axiom} and definitions\index{definition}
(\texttt{\$a}\index{\texttt{\$a} statement} statements) are the only
places where Metamath will not detect an error, and doing this will help
reduce the number of definitions needed.

You are free to use any tokens\index{token} you like for math
symbols\index{math symbol}.  Appendix~\ref{ASCII} recommends token names to
use for symbols in set theory, and we suggest you adopt these in order to be
able to include the \texttt{set.mm} set theory database in your database.  For
printouts, you can convert the tokens in a database
to standard mathematical symbols with the \LaTeX\ typesetting program.  The
Metamath command \texttt{open tex} {\em filename}\index{\texttt{open tex} command}
produces output that can be read by \LaTeX.\index{latex@{\LaTeX}}
The correspondence
between tokens and the actual symbols is made by \texttt{latexdef}
statements inside a special database comment tagged
with \texttt{\$t}.\index{\texttt{\$t} comment}\index{typesetting comment}
  You can edit
this comment to change the definitions or add new ones.
Appendix~\ref{ASCII} describes how to do this in more detail.

% White space\index{white space} is normally used to separate math
% symbol\index{math symbol} tokens, but they may be juxtaposed without white
% space in \texttt{\$d}\index{\texttt{\$d} statement}, \texttt{\$e}\index{\texttt{\$e}
% statement}, \texttt{\$f}\index{\texttt{\$f} statement}, \texttt{\$a}\index{\texttt{\$a}
% statement}, and \texttt{\$p}\index{\texttt{\$p} statement} statements when no
% ambiguity will result.  Specifically, Metamath parses the math symbol sequence
% in one of these statements in the following manner:  when the math symbol
% sequence has been broken up into tokens\index{token} up to a given character,
% the next token is the longest string of characters that could constitute a
% math symbol that is active\index{active
% math symbol} at that point.  (See Section~\ref{scoping} for the
% definition of an active math symbol.)  For example, if \texttt{-}, \texttt{>}, and
% \texttt{->} are the only active math symbols, the juxtaposition \texttt{>-} will be
% interpreted as the two symbols \texttt{>} and \texttt{-}, whereas \texttt{->} will
% always be interpreted as that single symbol.\footnote{For better readability we
% recommend a white space between each token.  This also makes searching for a
% symbol easier to do with an editor.  Omission of optional white space is useful
% for reducing typing when assigning an expression to a temporary
% variable\index{temporary variable} with the \texttt{let variable} Metamath
% program command.}\index{\texttt{let variable} command}
%
% Keywords\index{keyword} may be placed next to math symbols without white
% space\index{white space} between them.\footnote{Again, we do not recommend
% this for readability.}
%
% The math symbols\index{math symbol} in \texttt{\$c}\index{\texttt{\$c} statement}
% and \texttt{\$v}\index{\texttt{\$v} statement} statements must always be separated
% by white space\index{white
% space}, for the obvious reason that these statements define the names
% of the symbols.
%
% Math symbols referred to in comments (see Section~\ref{comments}) must also be
% separated by white space.  This allows you to make comments about symbols that
% are not yet active\index{active
% math symbol}.  (The ``math mode'' feature of comments is also a quick and
% easy way to obtain word processing text with embedded mathematical symbols,
% independently of the main purpose of Metamath; the way to do this is described
% in Section~\ref{comments})

\subsubsection{Label Tokens}\index{token}\index{label}

Label tokens are used to identify Metamath\index{Metamath} statements for
later reference. Label tokens may contain only letters, digits, and the three
characters period, hyphen, and underscore:
\begin{verbatim}
. - _
\end{verbatim}

A label is {\bf declared}\index{label declaration} by placing it immediately
before the keyword of the statement it identifies.  For example, the label
\texttt{axiom.1} might be declared as follows:
\begin{verbatim}
axiom.1 $a |- x = x $.
\end{verbatim}

Each \texttt{\$e}\index{\texttt{\$e} statement},
\texttt{\$f}\index{\texttt{\$f} statement},
\texttt{\$a}\index{\texttt{\$a} statement}, and
\texttt{\$p}\index{\texttt{\$p} statement} statement in a database must
have a label declared for it.  No other statement types may have label
declarations.  Every label must be unique.

A label (and the statement it identifies) is {\bf referenced}\index{label
reference} by including the label between the \texttt{\$=}\index{\texttt{\$=}
keyword} and \texttt{\$.}\index{\texttt{\$.}\ keyword}\ keywords in a \texttt{\$p}
statement.  The sequence of labels\index{label sequence} between these two
keywords is called a {\bf proof}\index{proof}.  An example of a statement with
a proof that we will encounter later (Section~\ref{proof}) is
\begin{verbatim}
wnew $p wff ( s -> ( r -> p ) )
     $= ws wr wp w2 w2 $.
\end{verbatim}

You don't have to know what this means just yet, but you should know that the
label \texttt{wnew} is declared by this \texttt{\$p} statement and that the labels
\texttt{ws}, \texttt{wr}, \texttt{wp}, and \texttt{w2} are assumed to have been declared
earlier in the database and are referenced here.

\subsection{Constants and Variables}
\index{constant}
\index{variable}

An {\bf expression}\index{expression} is any sequence of math
symbols, possibly empty.

The basic Metamath\index{Metamath} language\index{basic language} has two
kinds of math symbols\index{math symbol}:  {\bf constants}\index{constant} and
{\bf variables}\index{variable}.  In a Metamath proof, a constant may not be
substituted with any expression.  A variable can be
substituted\index{substitution!variable}\index{variable substitution} with any
expression.  This sequence may include other variables and may even include
the variable being substituted.  This substitution takes place when proofs are
verified, and it will be described in Section~\ref{proof}.  The \texttt{\$f}
statement (described later in Section~\ref{dollaref}) is used to specify the
{\bf type} of a variable (i.e.\ what kind of
variable it is)\index{variable type}\index{type} and
give it a meaning typically
associated with a ``metavariable''\index{metavariable}\footnote{A metavariable
is a variable that ranges over the syntactical elements of the object language
being discussed; for example, one metavariable might represent a variable of
the object language and another metavariable might represent a formula in the
object language.} in ordinary mathematics; for example, a variable may be
specified to be a wff or well-formed formula (in logic), a set (in set
theory), or a non-negative integer (in number theory).

%\subsection{The \texttt{\$c} and \texttt{\$v} Declaration Statements}
\subsection{The \texttt{\$c} and \texttt{\$v} Declaration Statements}
\index{\texttt{\$c} statement}
\index{constant declaration}
\index{\texttt{\$v} statement}
\index{variable declaration}

Constants are introduced or {\bf declared}\index{constant declaration}
with \texttt{\$c}\index{\texttt{\$c} statement} statements, and
variables are declared\index{variable declaration} with
\texttt{\$v}\index{\texttt{\$v} statement} statements.  A {\bf simple}
declaration\index{simple declaration} statement introduces a single
constant or variable.  Its syntax is one of the following:
\begin{center}
  \texttt{\$c} {\em math-symbol} \texttt{\$.}\\
  \texttt{\$v} {\em math-symbol} \texttt{\$.}
\end{center}
The notation {\em math-symbol} means any math symbol token\index{token}.

Some examples of simple declaration statements are:
\begin{center}
  \texttt{\$c + \$.}\\
  \texttt{\$c -> \$.}\\
  \texttt{\$c ( \$.}\\
  \texttt{\$v x \$.}\\
  \texttt{\$v y2 \$.}
\end{center}

The characters in a math symbol\index{math symbol} being declared are
irrelevant to Meta\-math; for example, we could declare a right parenthesis to
be a variable,
\begin{center}
  \texttt{\$v ) \$.}\\
\end{center}
although this would be unconventional.

A {\bf compound} declaration\index{compound declaration} statement is a
shorthand for declaring several symbols at once.  Its syntax is one of the
following:
\begin{center}
  \texttt{\$c} {\em math-symbol}\ \,$\cdots$\ {\em math-symbol} \texttt{\$.}\\
  \texttt{\$v} {\em math-symbol}\ \,$\cdots$\ {\em math-symbol} \texttt{\$.}
\end{center}\index{\texttt{\$c} statement}
Here, the ellipsis (\ldots) means any number of {\em math-symbol}\,s.

An example of a compound declaration statement is:
\begin{center}
  \texttt{\$v x y mu \$.}\\
\end{center}
This is equivalent to the three simple declaration statements
\begin{center}
  \texttt{\$v x \$.}\\
  \texttt{\$v y \$.}\\
  \texttt{\$v mu \$.}\\
\end{center}
\index{\texttt{\$v} statement}

There are certain rules on where in the database math symbols may be declared,
what sections of the database are aware of them (i.e.\ where they are
``active''), and when they may be declared more than once.  These will be
discussed in Section~\ref{scoping} and specifically on
p.~\pageref{redeclaration}.

\subsection{The \texttt{\$d} Statement}\label{dollard}
\index{\texttt{\$d} statement}

The \texttt{\$d} statement is called a {\bf disjoint-variable restriction}.  The
syntax of the {\bf simple} version of this statement is
\begin{center}
  \texttt{\$d} {\em variable variable} \texttt{\$.}
\end{center}
where each {\em variable} is a previously declared variable and the two {\em
variable}\,s are different.  (More specifically, each  {\em variable} must be
an {\bf active} variable\index{active math symbol}, which means there must be
a previous \texttt{\$v} statement whose {\bf scope}\index{scope} includes the
\texttt{\$d} statement.  These terms will be defined when we discuss scoping
statements in Section~\ref{scoping}.)

In ordinary mathematics, formulas may arise that are true if the variables in
them are distinct\index{distinct variables}, but become false when those
variables are made identical. For example, the formula in logic $\exists x\,x
\neq y$, which means ``for a given $y$, there exists an $x$ that is not equal
to $y$,'' is true in most mathematical theories (namely all non-trivial
theories\index{non-trivial theory}, i.e.\ those that describe more than one
individual, such as arithmetic).  However, if we substitute $y$ with $x$, we
obtain $\exists x\,x \neq x$, which is always false, as it means ``there
exists something that is not equal to itself.''\footnote{If you are a
logician, you will recognize this as the improper substitution\index{proper
substitution}\index{substitution!proper} of a free variable\index{free
variable} with a bound variable\index{bound variable}.  Metamath makes no
inherent distinction between free and bound variables; instead, you let
Metamath know what substitutions are permissible by using \texttt{\$d} statements
in the right way in your axiom system.}\index{free vs.\ bound variable}  The
\texttt{\$d} statement allows you to specify a restriction that forbids the
substitution of one variable with another.  In
this case, we would use the statement
\begin{center}
  \texttt{\$d x y \$.}
\end{center}\index{\texttt{\$d} statement}
to specify this restriction.

The order in which the variables appear in a \texttt{\$d} statement is not
important.  We could also use
\begin{center}
  \texttt{\$d y x \$.}
\end{center}

The \texttt{\$d} statement is actually more general than this, as the
``disjoint''\index{disjoint variables} in its name suggests.  The full meaning
is that if any substitution is made to its two variables (during the
course of a proof that references a \texttt{\$a} or \texttt{\$p} statement
associated with the \texttt{\$d}), the two expressions that result from the
substitution must have no variables in common.  In addition, each possible
pair of variables, one from each expression, must be in a \texttt{\$d} statement
associated with the statement being proved.  (This requirement forces the
statement being proved to ``inherit'' the original disjoint variable
restriction.)

For example, suppose \texttt{u} is a variable.  If the restriction
\begin{center}
  \texttt{\$d A B \$.}
\end{center}
has been specified for a theorem referenced in a
proof, we may not substitute \texttt{A} with \mbox{\tt a + u} and
\texttt{B} with \mbox{\tt b + u} because these two symbol sequences have the
variable \texttt{u} in common.  Furthermore, if \texttt{a} and \texttt{b} are
variables, we may not substitute \texttt{A} with \texttt{a} and \texttt{B} with \texttt{b}
unless we have also specified \texttt{\$d a b} for the theorem being proved; in
other words, the \texttt{\$d} property associated with a pair of variables must
be effectively preserved after substitution.

The \texttt{\$d}\index{\texttt{\$d} statement} statement does {\em not} mean ``the
two variables may not be substituted with the same thing,'' as you might think
at first.  For example, substituting each of \texttt{A} and \texttt{B} in the above
example with identical symbol sequences consisting only of constants does not
cause a disjoint variable conflict, because two symbol sequences have no
variables in common (since they have no variables, period).  Similarly, a
conflict will not occur by substituting the two variables in a \texttt{\$d}
statement with the empty symbol sequence\index{empty substitution}.

The \texttt{\$d} statement does not have a direct counterpart in
ordinary mathematics, partly because the variables\index{variable} of
Metamath are not really the same as the variables\index{variable!in
ordinary mathematics} of ordinary mathematics but rather are
metavariables\index{metavariable} ranging over them (as well as over
other kinds of symbols and groups of symbols).  Depending on the
situation, we may informally interpret the \texttt{\$d} statement in
different ways.  Suppose, for example, that \texttt{x} and \texttt{y}
are variables ranging over numbers (more precisely, that \texttt{x} and
\texttt{y} are metavariables ranging over variables that range over
numbers), and that \texttt{ph} ($\varphi$) and \texttt{ps} ($\psi$) are
variables (more precisely, metavariables) ranging over formulas.  We can
make the following interpretations that correspond to the informal
language of ordinary mathematics:
\begin{quote}
\begin{tabbing}
\texttt{\$d x y \$.} means ``assume $x$ and $y$ are
distinct variables.''\\
\texttt{\$d x ph \$.} means ``assume $x$ does not
occur in $\varphi$.''\\
\texttt{\$d ph ps \$.} \=means ``assume $\varphi$ and
$\psi$ have no variables\\ \>in common.''
\end{tabbing}
\end{quote}\index{\texttt{\$d} statement}

\subsubsection{Compound \texttt{\$d} Statements}

The {\bf compound} version of the \texttt{\$d} statement is a shorthand for
specifying several variables whose substitutions must be pairwise disjoint.
Its syntax is:
\begin{center}
  \texttt{\$d} {\em variable}\ \,$\cdots$\ {\em variable} \texttt{\$.}
\end{center}\index{\texttt{\$d} statement}
Here, {\em variable} represents the token of a previously declared
variable (specifically, an active variable) and all {\em variable}\,s are
different.  The compound \texttt{\$d}
statement is internally broken up by Metamath into one simple \texttt{\$d}
statement for each possible pair of variables in the original \texttt{\$d}
statement.  For example,
\begin{center}
  \texttt{\$d w x y z \$.}
\end{center}
is equivalent to
\begin{center}
  \texttt{\$d w x \$.}\\
  \texttt{\$d w y \$.}\\
  \texttt{\$d w z \$.}\\
  \texttt{\$d x y \$.}\\
  \texttt{\$d x z \$.}\\
  \texttt{\$d y z \$.}
\end{center}

Two or more simple \texttt{\$d} statements specifying the same variable pair are
internally combined into a single \texttt{\$d} statement.  Thus the set of three
statements
\begin{center}
  \texttt{\$d x y \$.}
  \texttt{\$d x y \$.}
  \texttt{\$d y x \$.}
\end{center}
is equivalent to
\begin{center}
  \texttt{\$d x y \$.}
\end{center}

Similarly, compound \texttt{\$d} statements, after being internally broken up,
internally have their common variable pairs combined.  For example the
set of statements
\begin{center}
  \texttt{\$d x y A \$.}
  \texttt{\$d x y B \$.}
\end{center}
is equivalent to
\begin{center}
  \texttt{\$d x y \$.}
  \texttt{\$d x A \$.}
  \texttt{\$d y A \$.}
  \texttt{\$d x y \$.}
  \texttt{\$d x B \$.}
  \texttt{\$d y B \$.}
\end{center}
which is equivalent to
\begin{center}
  \texttt{\$d x y \$.}
  \texttt{\$d x A \$.}
  \texttt{\$d y A \$.}
  \texttt{\$d x B \$.}
  \texttt{\$d y B \$.}
\end{center}

Metamath\index{Metamath} automatically verifies that all \texttt{\$d}
restrictions are met whenever it verifies proofs.  \texttt{\$d} statements are
never referenced directly in proofs (this is why they do not have
labels\index{label}), but Metamath is always aware of which ones must be
satisfied (i.e.\ are active) and will notify you with an error message if any
violation occurs.

To illustrate how Metamath detects a missing \texttt{\$d}
statement, we will look at the following example from the
\texttt{set.mm} database.

\begin{verbatim}
$d x z $.  $d y z $.
$( Theorem to add distinct quantifier to atomic formula. $)
ax17eq $p |- ( x = y -> A. z x = y ) $=...
\end{verbatim}

This statement has the obvious requirement that $z$ must be
distinct\index{distinct variables} from $x$ in theorem \texttt{ax17eq} that
states $x=y \rightarrow \forall z \, x=y$ (well, obvious if you're a logician,
for otherwise we could conclude  $x=y \rightarrow \forall x \, x=y$, which is
false when the free variables $x$ and $y$ are equal.).

Let's look at what happens if we edit the database to comment out this
requirement.

\begin{verbatim}
$( $d x z $. $) $d y z $.
$( Theorem to add distinct quantifier to atomic formula. $)
ax17eq $p |- ( x = y -> A. z x = y ) $=...
\end{verbatim}

When it tries to verify the proof, Metamath will tell you that \texttt{x} and
\texttt{z} must be disjoint, because one of its steps references an axiom or
theorem that has this requirement.

\begin{verbatim}
MM> verify proof ax17eq
ax17eq ?Error at statement 1918, label "ax17eq", type "$p":
      vz wal wi vx vy vz ax-13 vx vy weq vz vx ax-c16 vx vy
                                               ^^^^^
There is a disjoint variable ($d) violation at proof step 29.
Assertion "ax-c16" requires that variables "x" and "y" be
disjoint.  But "x" was substituted with "z" and "y" was
substituted with "x".  The assertion being proved, "ax17eq",
does not require that variables "z" and "x" be disjoint.
\end{verbatim}

We can see the substitutions into \texttt{ax-c16} with the following command.

\begin{verbatim}
MM> show proof ax17eq / detailed_step 29
Proof step 29:  pm2.61dd.2=ax-c16 $a |- ( A. z z = x -> ( x =
  y -> A. z x = y ) )
This step assigns source "ax-c16" ($a) to target "pm2.61dd.2"
($e).  The source assertion requires the hypotheses "wph"
($f, step 26), "vx" ($f, step 27), and "vy" ($f, step 28).
The parent assertion of the target hypothesis is "pm2.61dd"
($p, step 36).
The source assertion before substitution was:
    ax-c16 $a |- ( A. x x = y -> ( ph -> A. x ph ) )
The following substitutions were made to the source
assertion:
    Variable  Substituted with
     x         z
     y         x
     ph        x = y
The target hypothesis before substitution was:
    pm2.61dd.2 $e |- ( ph -> ch )
The following substitutions were made to the target
hypothesis:
    Variable  Substituted with
     ph        A. z z = x
     ch        ( x = y -> A. z x = y )
\end{verbatim}

The disjoint variable restrictions of \texttt{ax-c16} can be seen from the
\texttt{show state\-ment} command.  The line that begins ``\texttt{Its mandatory
dis\-joint var\-i\-able pairs are:}\ldots'' lists any \texttt{\$d} variable
pairs in brackets.

\begin{verbatim}
MM> show statement ax-c16/full
Statement 3033 is located on line 9338 of the file "set.mm".
"Axiom of Distinct Variables. ..."
  ax-c16 $a |- ( A. x x = y -> ( ph -> A. x ph ) ) $.
Its mandatory hypotheses in RPN order are:
  wph $f wff ph $.
  vx $f set x $.
  vy $f set y $.
Its mandatory disjoint variable pairs are:  <x,y>
The statement and its hypotheses require the variables:  x y
      ph
The variables it contains are:  x y ph
\end{verbatim}

Since Metamath will always detect when \texttt{\$d}\index{\texttt{\$d} statement}
statements are needed for a proof, you don't have to worry too much about
forgetting to put one in; it can always be added if you see the error message
above.  If you put in unnecessary \texttt{\$d} statements, the worst that could
happen is that your theorem might not be as general as it could be, and this
may limit its use later on.

On the other hand, when you introduce axioms (\texttt{\$a}\index{\texttt{\$a}
statement} statements), you must be very careful to properly specify the
necessary associated \texttt{\$d} statements since Metamath has no way of knowing
whether your axioms are correct.  For example, Metamath would have no idea
that \texttt{ax-c16}, which we are telling it is an axiom of logic, would lead to
contradictions if we omitted its associated \texttt{\$d} statement.

% This was previously a comment in footnote-sized type, but it can be
% hard to read this much text in a small size.
% As a result, it's been changed to normally-sized text.
\label{nodd}
You may wonder if it is possible to develop standard
mathematics in the Metamath language without the \texttt{\$d}\index{\texttt{\$d}
statement} statement, since it seems like a nuisance that complicates proof
verification. The \texttt{\$d} statement is not needed in certain subsets of
mathematics such as propositional calculus.  However, dummy
variables\index{dummy variable!eliminating} and their associated \texttt{\$d}
statements are impossible to avoid in proofs in standard first-order logic as
well as in the variant used in \texttt{set.mm}.  In fact, there is no upper bound to
the number of dummy variables that might be needed in a proof of a theorem of
first-order logic containing 3 or more variables, as shown by H.\
Andr\'{e}ka\index{Andr{\'{e}}ka, H.} \cite{Nemeti}.  A first-order system that
avoids them entirely is given in \cite{Megill}\index{Megill, Norman}; the
trick there is simply to embed harmlessly the necessary dummy variables into a
theorem being proved so that they aren't ``dummy'' anymore, then interpret the
resulting longer theorem so as to ignore the embedded dummy variables.  If
this interests you, the system in \texttt{set.mm} obtained from \texttt{ax-1}
through \texttt{ax-c14} in \texttt{set.mm}, and deleting \texttt{ax-c16} and \texttt{ax-5},
requires no \texttt{\$d} statements but is logically complete in the sense
described in \cite{Megill}.  This means it can prove any theorem of
first-order logic as long as we add to the theorem an antecedent that embeds
dummy and any other variables that must be distinct.  In a similar fashion,
axioms for set theory can be devised that
do not require distinct variable
provisos\index{Set theory without distinct variable provisos},
as explained at
\url{http://us.metamath.org/mpeuni/mmzfcnd.html}.
Together, these in principle allow all of
mathematics to be developed under Metamath without a \texttt{\$d} statement,
although the length of the resulting theorems will grow as more and
more dummy variables become required in their proofs.

\subsection{The \texttt{\$f}
and \texttt{\$e} Statements}\label{dollaref}
\index{\texttt{\$e} statement}
\index{\texttt{\$f} statement}
\index{floating hypothesis}
\index{essential hypothesis}
\index{variable-type hypothesis}
\index{logical hypothesis}
\index{hypothesis}

Metamath has two kinds of hypo\-theses, the \texttt{\$f}\index{\texttt{\$f}
statement} or {\bf variable-type} hypothesis and the \texttt{\$e} or {\bf logical}
hypo\-the\-sis.\index{\texttt{\$d} statement}\footnote{Strictly speaking, the
\texttt{\$d} statement is also a hypothesis, but it is never directly referenced
in a proof, so we call it a restriction rather than a hypothesis to lessen
confusion.  The checking for violations of \texttt{\$d} restrictions is automatic
and built into Metamath's proof-checking algorithm.} The letters \texttt{f} and
\texttt{e} stand for ``floating''\index{floating hypothesis} (roughly meaning
used only if relevant) and ``essential''\index{essential hypothesis} (meaning
always used) respectively, for reasons that will become apparent
when we discuss frames in
Section~\ref{frames} and scoping in Section~\ref{scoping}. The syntax of these
are as follows:
\begin{center}
  {\em label} \texttt{\$f} {\em typecode} {\em variable} \texttt{\$.}\\
  {\em label} \texttt{\$e} {\em typecode}
      {\em math-symbol}\ \,$\cdots$\ {\em math-symbol} \texttt{\$.}\\
\end{center}
\index{\texttt{\$e} statement}
\index{\texttt{\$f} statement}
A hypothesis must have a {\em label}\index{label}.  The expression in a
\texttt{\$e} hypothesis consists of a typecode (an active constant math symbol)
followed by a sequence
of zero or more math symbols. Each math symbol (including {\em constant}
and {\em variable}) must be a previously declared constant or variable.  (In
addition, each math symbol must be active, which will be covered when we
discuss scoping statements in Section~\ref{scoping}.)  You use a \texttt{\$f}
hypothesis to specify the
nature or {\bf type}\index{variable type}\index{type} of a variable (such as ``let $x$ be an
integer'') and use a \texttt{\$e} hypothesis to express a logical truth (such as
``assume $x$ is prime'') that must be established in order for an assertion
requiring it to also be true.

A variable must have its type specified in a \texttt{\$f} statement before
it may be used in a \texttt{\$e}, \texttt{\$a}, or \texttt{\$p}
statement.  There may be only one (active) \texttt{\$f} statement for a
given variable.  (``Active'' is defined in Section~\ref{scoping}.)

In ordinary mathematics, theorems\index{theorem} are often expressed in the
form ``Assume $P$; then $Q$,'' where $Q$ is a statement that you can derive
if you start with statement $P$.\index{free variable}\footnote{A stronger
version of a theorem like this would be the {\em single} formula $P\rightarrow
Q$ ($P$ implies $Q$) from which the weaker version above follows by the rule
of modus ponens in logic.  We are not discussing this stronger form here.  In
the weaker form, we are saying only that if we can {\em prove} $P$, then we can
{\em prove} $Q$.  In a logician's language, if $x$ is the only free variable
in $P$ and $Q$, the stronger form is equivalent to $\forall x ( P \rightarrow
Q)$ (for all $x$, $P$ implies $Q$), whereas the weaker form is equivalent to
$\forall x P \rightarrow \forall x Q$. The stronger form implies the weaker,
but not vice-versa.  To be precise, the weaker form of the theorem is more
properly called an ``inference'' rather than a theorem.}\index{inference}
In the
Metamath\index{Metamath} language, you would express mathematical statement
$P$ as a hypothesis (a \texttt{\$e} Metamath language statement in this case) and
statement $Q$ as a provable assertion (a \texttt{\$p}\index{\texttt{\$p} statement}
statement).

Some examples of hypotheses you might encounter in logic and set theory are
\begin{center}
  \texttt{stmt1 \$f wff P \$.}\\
  \texttt{stmt2 \$f set x \$.}\\
  \texttt{stmt3 \$e |- ( P -> Q ) \$.}
\end{center}
\index{\texttt{\$e} statement}
\index{\texttt{\$f} statement}
Informally, these would be read, ``Let $P$ be a well-formed-formula,'' ``Let
$x$ be an (individual) variable,'' and ``Assume we have proved $P \rightarrow
Q$.''  The turnstile symbol \,$\vdash$\index{turnstile ({$\,\vdash$})} is
commonly used in logic texts to mean ``a proof exists for.''

To summarize:
\begin{itemize}
\item A \texttt{\$f} hypothesis tells Metamath the type or kind of its variable.
It is analogous to a variable declaration in a computer language that
tells the compiler that a variable is an integer or a floating-point
number.
\item The \texttt{\$e} hypothesis corresponds to what you would usually call a
``hypothesis'' in ordinary mathematics.
\end{itemize}

Before an assertion\index{assertion} (\texttt{\$a} or \texttt{\$p} statement) can be
referenced in a proof, all of its associated \texttt{\$f} and \texttt{\$e} hypotheses
(i.e.\ those \texttt{\$e} hypotheses that are active) must be satisfied (i.e.
established by the proof).  The meaning of ``associated'' (which we will call
{\bf mandatory} in Section~\ref{frames}) will become clear when we discuss
scoping later.

Note that after any \texttt{\$f}, \texttt{\$e},
\texttt{\$a}, or \texttt{\$p} token there is a required
\textit{typecode}\index{typecode}.
The typecode is a constant used to enforce types of expressions.
This will become clearer once we learn more about
assertions (\texttt{\$a} and \texttt{\$p} statements).
An example may also clarify their purpose.
In the
\texttt{set.mm}\index{set theory database (\texttt{set.mm})}%
\index{Metamath Proof Explorer}
database,
the following typecodes are used:

\begin{itemize}
\item \texttt{wff} :
  Well-formed formula (wff) symbol
  (read: ``the following symbol sequence is a wff'').
% The *textual* typecode for turnstile is "|-", but when read it's a little
% confusing, so I intentionally display the mathematical symbol here instead
% (I think it's clearer in this context).
\item \texttt{$\vdash$} :
  Turnstile (read: ``the following symbol sequence is provable'' or
  ``a proof exists for'').
\item \texttt{setvar} :
  Individual set variable type (read: ``the following is an
  individual set variable'').
  Note that this is \textit{not} the type of an arbitrary set expression,
  instead, it is used to ensure that there is only a single symbol used
  after quantifiers like for-all ($\forall$) and there-exists ($\exists$).
\item \texttt{class} :
  An expression that is a syntactically valid class expression.
  All valid set expressions are also valid class expression, so expressions
  of sets normally have the \texttt{class} typecode.
  Use the \texttt{class} typecode,
  \textit{not} the \texttt{setvar} typecode,
  for the type of set expressions unless you are specifically identifying
  a single set variable.
\end{itemize}

\subsection{Assertions (\texttt{\$a} and \texttt{\$p} Statements)}
\index{\texttt{\$a} statement}
\index{\texttt{\$p} statement}\index{assertion}\index{axiomatic assertion}
\index{provable assertion}

There are two types of assertions, \texttt{\$a}\index{\texttt{\$a} statement}
statements ({\bf axiomatic assertions}) and \texttt{\$p} statements ({\bf
provable assertions}).  Their syntax is as follows:
\begin{center}
  {\em label} \texttt{\$a} {\em typecode} {\em math-symbol} \ldots
         {\em math-symbol} \texttt{\$.}\\
  {\em label} \texttt{\$p} {\em typecode} {\em math-symbol} \ldots
        {\em math-symbol} \texttt{\$=} {\em proof} \texttt{\$.}
\end{center}
\index{\texttt{\$a} statement}
\index{\texttt{\$p} statement}
\index{\texttt{\$=} keyword}
An assertion always requires a {\em label}\index{label}. The expression in an
assertion consists of a typecode (an active constant)
followed by a sequence of zero
or more math symbols.  Each math symbol, including any {\em constant}, must be a
previously declared constant or variable.  (In addition, each math symbol
must be active, which will be covered when we discuss scoping statements in
Section~\ref{scoping}.)

A \texttt{\$a} statement is usually a definition of syntax (for example, if $P$
and $Q$ are wffs then so is $(P\to Q)$), an axiom\index{axiom} of ordinary
mathematics (for example, $x=x$), or a definition\index{definition} of
ordinary mathematics (for example, $x\ne y$ means $\lnot x=y$). A \texttt{\$p}
statement is a claim that a certain combination of math symbols follows from
previous assertions and is accompanied by a proof that demonstrates it.

Assertions can also be referenced in (later) proofs in order to derive new
assertions from them. The label of an assertion is used to refer to it in a
proof. Section~\ref{proof} will describe the proof in detail.

Assertions also provide the primary means for communicating the mathematical
results in the database to people.  Proofs (when conveniently displayed)
communicate to people how the results were arrived at.

\subsubsection{The \texttt{\$a} Statement}
\index{\texttt{\$a} statement}

Axiomatic assertions (\texttt{\$a} statements) represent the starting points from
which other assertions (\texttt{\$p}\index{\texttt{\$p} statement} statements) are
derived.  Their most obvious use is for specifying ordinary mathematical
axioms\index{axiom}, but they are also used for two other purposes.

First, Metamath\index{Metamath} needs to know the syntax of symbol
sequences that constitute valid mathematical statements.  A Metamath
proof must be broken down into much more detail than ordinary
mathematical proofs that you may be used to thinking of (even the
``complete'' proofs of formal logic\index{formal logic}).  This is one
of the things that makes Metamath a general-purpose language,
independent of any system of logic or even syntax.  If you want to use a
substitution instance of an assertion as a step in a proof, you must
first prove that the substitution is syntactically correct (or if you
prefer, you must ``construct'' it), showing for example that the
expression you are substituting for a wff metavariable is a valid wff.
The \texttt{\$a}\index{\texttt{\$a} statement} statement is used to
specify those combinations of symbols that are considered syntactically
valid, such as the legal forms of wffs.

Second, \texttt{\$a} statements are used to specify what are ordinarily thought of
as definitions, i.e.\ new combinations of symbols that abbreviate other
combinations of symbols.  Metamath makes no distinction\index{axiom vs.\
definition} between axioms\index{axiom} and definitions\index{definition}.
Indeed, it has been argued that such distinction should not be made even in
ordinary mathematics; see Section~\ref{definitions}, which discusses the
philosophy of definitions.  Section~\ref{hierarchy} discusses some
technical requirements for definitions.  In \texttt{set.mm} we adopt the
convention of prefixing axiom labels with \texttt{ax-} and definition labels with
\texttt{df-}\index{label}.

The results that can be derived with the Metamath language are only as good as
the \texttt{\$a}\index{\texttt{\$a} statement} statements used as their starting
point.  We cannot stress this too strongly.  For example, Metamath will
not prevent you from specifying $x\neq x$ as an axiom of logic.  It is
essential that you scrutinize all \texttt{\$a} statements with great care.
Because they are a source of potential pitfalls, it is best not to add new
ones (usually new definitions) casually; rather you should carefully evaluate
each one's necessity and advantages.

Once you have in place all of the basic axioms\index{axiom} and
rules\index{rule} of a mathematical theory, the only \texttt{\$a} statements that
you will be adding will be what are ordinarily called definitions.  In
principle, definitions should be in some sense eliminable from the language of
a theory according to some convention (usually involving logical equivalence
or equality).  The most common convention is that any formula that was
syntactically valid but not provable before the definition was introduced will
not become provable after the definition is introduced.  In an ideal world,
definitions should not be present at all if one is to have absolute confidence
in a mathematical result.  However, they are necessary to make
mathematics practical, for otherwise the resulting formulas would be
extremely long and incomprehensible.  Since the nature of definitions (in the
most general sense) does not permit them to automatically be verified as
``proper,''\index{proper definition}\index{definition!proper} the judgment of
the mathematician is required to ensure it.  (In \texttt{set.mm} effort was made
to make almost all definitions directly eliminable and thus minimize the need
for such judgment.)

If you are not a mathematician, it may be best not to add or change any
\texttt{\$a}\index{\texttt{\$a} statement} statements but instead use
the mathematical language already provided in standard databases.  This
way Metamath will not allow you to make a mistake (i.e.\ prove a false
result).


\subsection{Frames}\label{frames}

We now introduce the concept of a collection of related Metamath statements
called a frame.  Every assertion (\texttt{\$a} or \texttt{\$p} statement) in the database has
an associated frame.

A {\bf frame}\index{frame} is a sequence of \texttt{\$d}, \texttt{\$f},
and \texttt{\$e} statements (zero or more of each) followed by one
\texttt{\$a} or \texttt{\$p} statement, subject to certain conditions we
will describe.  For simplicity we will assume that all math symbol
tokens used are declared at the beginning of the database with
\texttt{\$c} and \texttt{\$v} statements (which are not properly part of
a frame).  Also for simplicity we will assume there are only simple
\texttt{\$d} statements (those with only two variables) and imagine any
compound \texttt{\$d} statements (those with more than two variables) as
broken up into simple ones.

A frame groups together those hypotheses (and \texttt{\$d} statements) relevant
to an assertion (\texttt{\$a} or \texttt{\$p} statement).  The statements in a frame
may or may not be physically adjacent in a database; we will cover
this in our discussion of scoping statements
in Section~\ref{scoping}.

A frame has the following properties:
\begin{enumerate}
 \item The set of variables contained in its \texttt{\$f} statements must
be identical to the set of variables contained in its \texttt{\$e},
\texttt{\$a}, and/or \texttt{\$p} statements.  In other words, each
variable in a \texttt{\$e}, \texttt{\$a}, or \texttt{\$p} statement must
have an associated ``variable type'' defined for it in a \texttt{\$f}
statement.
  \item No two \texttt{\$f} statements may contain the same variable.
  \item Any \texttt{\$f} statement
must occur before a \texttt{\$e} statement in which its variable occurs.
\end{enumerate}

The first property determines the set of variables occurring in a frame.
These are the {\bf mandatory
variables}\index{mandatory variable} of the frame.  The second property
tells us there must be only one type specified for a variable.
The last property is not a theoretical requirement but it
makes parsing of the database easier.

For our examples, we assume our database has the following declarations:

\begin{verbatim}
$v P Q R $.
$c -> ( ) |- wff $.
\end{verbatim}

The following sequence of statements, describing the modus ponens inference
rule, is an example of a frame:

\begin{verbatim}
wp  $f wff P $.
wq  $f wff Q $.
maj $e |- ( P -> Q ) $.
min $e |- P $.
mp  $a |- Q $.
\end{verbatim}

The following sequence of statements is not a frame because \texttt{R} does not
occur in the \texttt{\$e}'s or the \texttt{\$a}:

\begin{verbatim}
wp  $f wff P $.
wq  $f wff Q $.
wr  $f wff R $.
maj $e |- ( P -> Q ) $.
min $e |- P $.
mp  $a |- Q $.
\end{verbatim}

The following sequence of statements is not a frame because \texttt{Q} does not
occur in a \texttt{\$f}:

\begin{verbatim}
wp  $f wff P $.
maj $e |- ( P -> Q ) $.
min $e |- P $.
mp  $a |- Q $.
\end{verbatim}

The following sequence of statements is not a frame because the \texttt{\$a} statement is
not the last one:

\begin{verbatim}
wp  $f wff P $.
wq  $f wff Q $.
maj $e |- ( P -> Q ) $.
mp  $a |- Q $.
min $e |- P $.
\end{verbatim}

Associated with a frame is a sequence of {\bf mandatory
hypotheses}\index{mandatory hypothesis}.  This is simply the set of all
\texttt{\$f} and \texttt{\$e} statements in the frame, in the order they
appear.  A frame can be referenced in a later proof using the label of
the \texttt{\$a} or \texttt{\$p} assertion statement, and the proof
makes an assignment to each mandatory hypothesis in the order in which
it appears.  This means the order of the hypotheses, once chosen, must
not be changed so as not to affect later proofs referencing the frame's
assertion statement.  (The Metamath proof verifier will, of course, flag
an error if a proof becomes incorrect by doing this.)  Since proofs make
use of ``Reverse Polish notation,'' described in Section~\ref{proof}, we
call this order the {\bf RPN order}\index{RPN order} of the hypotheses.

Note that \texttt{\$d} statements are not part of the set of mandatory
hypotheses, and their order doesn't matter (as long as they satisfy the
fourth property for a frame described above).  The \texttt{\$d}
statements specify restrictions on variables that must be satisfied (and
are checked by the proof verifier) when expressions are substituted for
them in a proof, and the \texttt{\$d} statements themselves are never
referenced directly in a proof.

A frame with a \texttt{\$p} (provable) statement requires a proof as part of the
\texttt{\$p} statement.  Sometimes in a proof we want to make use of temporary or
dummy variables\index{dummy variable} that do not occur in the \texttt{\$p}
statement or its mandatory hypotheses.  To accommodate this we define an {\bf
extended frame}\index{extended frame} as a frame together with zero or more
\texttt{\$d} and \texttt{\$f} statements that reference variables not among the
mandatory variables of the frame.  Any new variables referenced are called the
{\bf optional variables}\index{optional variable} of the extended frame. If a
\texttt{\$f} statement references an optional variable it is called an {\bf
optional hypothesis}\index{optional hypothesis}, and if one or both of the
variables in a \texttt{\$d} statement are optional variables it is called an {\bf
optional disjoint-variable restriction}\index{optional disjoint-variable
restriction}.  Properties 2, 3, and 4 for a frame also apply to an extended
frame.

The concept of optional variables is not meaningful for frames with \texttt{\$a}
statements, since those statements have no proofs that might make use of them.
There is no restriction on including optional hypotheses in the extended frame
for a \texttt{\$a} statement, but they serve no purpose.

The following set of statements is an example of an extended frame, which
contains an optional variable \texttt{R} and an optional hypothesis \texttt{wr}.  In
this example, we suppose the rule of modus ponens is not an axiom but is
derived as a theorem from earlier statements (we omit its presumed proof).
Variable \texttt{R} may be used in its proof if desired (although this would
probably have no advantage in propositional calculus).  Note that the sequence
of mandatory hypotheses in RPN order is still \texttt{wp}, \texttt{wq}, \texttt{maj},
\texttt{min} (i.e.\ \texttt{wr} is omitted), and this sequence is still assumed
whenever the assertion \texttt{mp} is referenced in a subsequent proof.

\begin{verbatim}
wp  $f wff P $.
wq  $f wff Q $.
wr  $f wff R $.
maj $e |- ( P -> Q ) $.
min $e |- P $.
mp  $p |- Q $= ... $.
\end{verbatim}

Every frame is an extended frame, but not every extended frame is a frame, as
this example shows.  The underlying frame for an extended frame is
obtained by simply removing all statements containing optional variables.
Any proof referencing an assertion will ignore any extensions to its
frame, which means we may add or delete optional hypotheses at will without
affecting subsequent proofs.

The conceptually simplest way of organizing a Metamath database is as a
sequence of extended frames.  The scoping statements
\texttt{\$\char`\{}\index{\texttt{\$\char`\{} and \texttt{\$\char`\}}
keywords} and \texttt{\$\char`\}} can be used to delimit the start and
end of an extended frame, leading to the following possible structure for a
database.  \label{framelist}

\vskip 2ex
\setbox\startprefix=\hbox{\tt \ \ \ \ \ \ \ \ }
\setbox\contprefix=\hbox{}
\startm
\m{\mbox{(\texttt{\$v} {\em and} \texttt{\$c}\,{\em statements})}}
\endm
\startm
\m{\mbox{\texttt{\$\char`\{}}}
\endm
\startm
\m{\mbox{\texttt{\ \ } {\em extended frame}}}
\endm
\startm
\m{\mbox{\texttt{\$\char`\}}}}
\endm
\startm
\m{\mbox{\texttt{\$\char`\{}}}
\endm
\startm
\m{\mbox{\texttt{\ \ } {\em extended frame}}}
\endm
\startm
\m{\mbox{\texttt{\$\char`\}}}}
\endm
\startm
\m{\mbox{\texttt{\ \ \ \ \ \ \ \ \ }}\vdots}
\endm
\vskip 2ex

In practice, this structure is inconvenient because we have to repeat
any \texttt{\$f}, \texttt{\$e}, and \texttt{\$d} statements over and
over again rather than stating them once for use by several assertions.
The scoping statements, which we will discuss next, allow this to be
done.  In principle, any Metamath database can be converted to the above
format, and the above format is the most convenient to use when studying
a Metamath database as a formal system%
%% Uncomment this when uncommenting section {formalspec} below
   (Appendix \ref{formalspec})%
.
In fact, Metamath internally converts the database to the above format.
The command \texttt{show statement} in the Metamath program will show
you the contents of the frame for any \texttt{\$a} or \texttt{\$p}
statement, as well as its extension in the case of a \texttt{\$p}
statement.

%c%(provided that all ``local'' variables and constants with limited scope have
%c%unique names),

During our discussion of scoping statements, it may be helpful to
think in terms of the equivalent sequence of frames that will result when
the database is parsed.  Scoping (other than the limited
use above to delimit frames) is not a theoretical requirement for
Metamath but makes it more convenient.


\subsection{Scoping Statements (\texttt{\$\{} and \texttt{\$\}})}\label{scoping}
\index{\texttt{\$\char`\{} and \texttt{\$\char`\}} keywords}\index{scoping statement}

%c%Some Metamath statements may be needed only temporarily to
%c%serve a specific purpose, and after we're done with them we would like to
%c%disregard or ignore them.  For example, when we're finished using a variable,
%c%we might want to
%c%we might want to free up the token\index{token} used to name it so that the
%c%token can be used for other purposes later on, such as a different kind of
%c%variable or even a constant.  In the terminology of computer programming, we
%c%might want to let some symbol declarations be ``local'' rather than ``global.''
%c%\index{local symbol}\index{global symbol}

The {\bf scoping} statements, \texttt{\$\char`\{} ({\bf start of block}) and \texttt{\$\char`\}}
({\bf end of block})\index{block}, provide a means for controlling the portion
of a database over which certain statement types are recognized.  The
syntax of a scoping statement is very simple; it just consists of the
statement's keyword:
\begin{center}
\texttt{\$\char`\{}\\
\texttt{\$\char`\}}
\end{center}
\index{\texttt{\$\char`\{} and \texttt{\$\char`\}} keywords}

For example, consider the following database where we have stripped out
all tokens except the scoping statement keywords.  For the purpose of the
discussion, we have added subscripts to the scoping statements; these subscripts
do not appear in the actual database.
\[
 \mbox{\tt \ \$\char`\{}_1
 \mbox{\tt \ \$\char`\{}_2
 \mbox{\tt \ \$\char`\}}_2
 \mbox{\tt \ \$\char`\{}_3
 \mbox{\tt \ \$\char`\{}_4
 \mbox{\tt \ \$\char`\}}_4
 \mbox{\tt \ \$\char`\}}_3
 \mbox{\tt \ \$\char`\}}_1
\]
Each \texttt{\$\char`\{} statement in this example is said to be {\bf
matched} with the \texttt{\$\char`\}} statement that has the same
subscript.  Each pair of matched scoping statements defines a region of
the database called a {\bf block}.\index{block} Blocks can be {\bf
nested}\index{nested block} inside other blocks; in the example, the
block defined by $\mbox{\tt \$\char`\{}_4$ and $\mbox{\tt \$\char`\}}_4$
is nested inside the block defined by $\mbox{\tt \$\char`\{}_3$ and
$\mbox{\tt \$\char`\}}_3$ as well as inside the block defined by
$\mbox{\tt \$\char`\{}_1$ and $\mbox{\tt \$\char`\}}_1$.  In general, a
block may be empty, it may contain only non-scoping
statements,\footnote{Those statements other than \texttt{\$\char`\{} and
\texttt{\$\char`\}}.}\index{non-scoping statement} or it may contain any
mixture of other blocks and non-scoping statements.  (This is called a
``recursive'' definition\index{recursive definition} of a block.)

Associated with each block is a number called its {\bf nesting
level}\index{nesting level} that indicates how deeply the block is nested.
The nesting levels of the blocks in our example are as follows:
\[
  \underbrace{
    \mbox{\tt \ }
    \underbrace{
     \mbox{\tt \$\char`\{\ }
     \underbrace{
       \mbox{\tt \$\char`\{\ }
       \mbox{\tt \$\char`\}}
     }_{2}
     \mbox{\tt \ }
     \underbrace{
       \mbox{\tt \$\char`\{\ }
       \underbrace{
         \mbox{\tt \$\char`\{\ }
         \mbox{\tt \$\char`\}}
       }_{3}
       \mbox{\tt \ \$\char`\}}
     }_{2}
     \mbox{\tt \ \$\char`\}}
   }_{1}
   \mbox{\tt \ }
 }_{0}
\]
\index{\texttt{\$\char`\{} and \texttt{\$\char`\}} keywords}
The entire database is considered to be one big block (the {\bf outermost}
block) with a nesting level of 0.  The outermost block is {\em not} bracketed
by scoping statements.\footnote{The language was designed this way so that
several source files can be joined together more easily.}\index{outermost
block}

All non-scoping Metamath statements become recognized or {\bf
active}\index{active statement} at the place where they appear.\footnote{To
keep things slightly simpler, we do not bother to define the concept of
``active'' for the scoping statements.}  Certain of these statement types
become inactive at the end of the block in which they appear; these statement
types are:
\begin{center}
  \texttt{\$c}, \texttt{\$v}, \texttt{\$d}, \texttt{\$e}, and \texttt{\$f}.
%  \texttt{\$v}, \texttt{\$f}, \texttt{\$e}, and \texttt{\$d}.
\end{center}
\index{\texttt{\$c} statement}
\index{\texttt{\$d} statement}
\index{\texttt{\$e} statement}
\index{\texttt{\$f} statement}
\index{\texttt{\$v} statement}
The other statement types remain active forever (i.e.\ through the end of the
database); they are:
\begin{center}
  \texttt{\$a} and \texttt{\$p}.
%  \texttt{\$c}, \texttt{\$a}, and \texttt{\$p}.
\end{center}
\index{\texttt{\$a} statement}
\index{\texttt{\$p} statement}
Any statement (of these 7 types) located in the outermost
block\index{outermost block} will remain active through the end of the
database and thus are effectively ``global'' statements.\index{global
statement}

All \texttt{\$c} statements must be placed in the outermost block.  Since they are
therefore always global, they could be considered as belonging to both of the
above categories.

The {\bf scope}\index{scope} of a statement is the set of statements that
recognize it as active.

%c%The concept of ``active'' is also defined for math symbols\index{math
%c%symbol}.  Math symbols (constants\index{constant} and
%c%variables\index{variable}) become {\bf active}\index{active
%c%math symbol} in the \texttt{\$c}\index{\texttt{\$c}
%c%statement} and \texttt{\$v}\index{\texttt{\$v} statement} statements that
%c%declare them.  They become inactive when their declaration statements become
%c%inactive.

The concept of ``active'' is also defined for math symbols\index{math
symbol}.  Math symbols (constants\index{constant} and
variables\index{variable}) become {\bf active}\index{active math symbol}
in the \texttt{\$c}\index{\texttt{\$c} statement} and
\texttt{\$v}\index{\texttt{\$v} statement} statements that declare them.
A variable becomes inactive when its declaration statement becomes
inactive.  Because all \texttt{\$c} statements must be in the outermost
block, a constant will never become inactive after it is declared.

\subsubsection{Redeclaration of Math Symbols}
\index{redeclaration of symbols}\label{redeclaration}

%c%A math symbol may not be declared a second time while it is active, but it may
%c%be declared again after it becomes inactive.

A variable may not be declared a second time while it is active, but it may be
declared again after it becomes inactive.  This provides a convenient way to
introduce ``local'' variables,\index{local variable} i.e.\ temporary variables
for use in the frame of an assertion or in a proof without keeping them around
forever.  A previously declared variable may not be redeclared as a constant.

A constant may not be redeclared.  And, as mentioned above, constants must be
declared in the outermost block.

The reason variables may have limited scope but not constants is that an
assertion (\texttt{\$a} or \texttt{\$p} statement) remains available for use in
proofs through the end of the database.  Variables in an assertion's frame may
be substituted with whatever is needed in a proof step that references the
assertion, whereas constants remain fixed and may not be substituted with
anything.  The particular token used for a variable in an assertion's frame is
irrelevant when the assertion is referenced in a proof, and it doesn't matter
if that token is not available outside of the referenced assertion's frame.
Constants, however, must be globally fixed.

In the present version of the Metamath language, there is no theoretical
benefit for the feature allowing variables to be active for limited scopes
rather than global. It is just a convenience that allows them, for example, to
be locally grouped together with their corresponding \texttt{\$f} variable-type
declarations.\footnote{A future version\label{redeclarationf} of the Metamath
language may extend the language to include true local variables as well as
local constants.  I have not decided upon the best way to do this or even if
it should be done at all; the main issue is whether the benefits offset
greater confusion in learning the language.  In any case, the rules for
the present language were chosen to let existing databases be compatible with
any future language extensions.}

%c%If you declare a math symbol more than once, internally Metamath considers it a
%c%new distinct symbol, even though it has the same name.  If you are unaware of
%c%this, you may find that what you think are correct proofs are incorrectly
%c%rejected as invalid, because Metamath may tell you that a constant you
%c%previously declared does not match a newly declared math symbol with the same
%c%name.  For details on this subtle point, see the Comment on
%c%p.~\pageref{spec4comment}.  This is done purposely to allow temporary
%c%constants to be introduced while developing a subtheory, then allow their math
%c%symbol tokens to be reused later on; in general they will not refer to the
%c%same thing.  In practice, you would not ordinarily reuse the names of
%c%constants because it would tend to be confusing to the reader.  The reuse of
%c%names of variables, on the other hand, is something that is often useful to do
%c%(for example it is done frequently in \texttt{set.mm}).  Since variables in an
%c%assertion referenced in a proof can be substituted as needed to achieve a
%c%symbol match, this is not an issue.

% (This section covers a somewhat advanced topic you may want to skip
% at first reading.)
%
% Under certain circumstances, math symbol\index{math symbol}
% tokens\index{token} may be redeclared (i.e.\ the token
% may appear in more than
% one \texttt{\$c}\index{\texttt{\$c} statement} or \texttt{\$v}\index{\texttt{\$v}
% statement} statement).  You might want to do this say, to make temporary use
% of a variable name without having to worry about its affect elsewhere,
% somewhat analogous to declaring a local variable in a standard computer
% language.  Understanding what goes on when math symbol tokens are redeclared
% is a little tricky to understand at first, since it requires that we
% distinguish the token itself from the math symbol that it names.  It will help
% if we first take a peek at the internal workings of the
% Metamath\index{Metamath} program.
%
% Metamath reserves a memory location for each occurrence of a
% token\index{token} in a declaration statement (\texttt{\$c}\index{\texttt{\$c}
% statement} or \texttt{\$v}\index{\texttt{\$v} statement}).  If a given token appears
% in more than one declaration statement, it will refer to more than one memory
% locations.  A math symbol\index{math symbol} may be thought of as being one of
% these memory locations rather than as the token itself.  Only one of the
% memory locations associated with a given token may be active at any one time.
% The math symbol (memory location) that gets looked up when the token appears
% in a non-declaration statement is the one that happens to be active at that
% time.
%
% We now look at the rules for the redeclaration\index{redeclaration of symbols}
% of math symbol tokens.
% \begin{itemize}
% \item A math symbol token may not be declared twice in the
% same block.\footnote{While there is no theoretical reason for disallowing
% this, it was decided in the design of Metamath that allowing it would offer no
% advantage and might cause confusion.}
% \item An inactive math symbol may always be
% redeclared.
% \item  An active math symbol may be redeclared in a different (i.e.\
% inner) block\index{block} from the one it became active in.
% \end{itemize}
%
% When a math symbol token is redeclared, it conceptually refers to a different
% math symbol, just as it would be if it were called a different name.  In
% addition, the original math symbol that it referred to, if it was active,
% temporarily becomes inactive.  At the end of the block in which the
% redeclaration occurred, the new math symbol\index{math symbol} becomes
% inactive and the original symbol becomes active again.  This concept is
% illustrated in the following example, where the symbol \texttt{e} is
% ordinarily a constant (say Euler's constant, 2.71828...) but
% temporarily we want to use it as a ``local'' variable, say as a coefficient
% in the equation $a x^4 + b x^3 + c x^2 + d x + e$:
% \[
%   \mbox{\tt \$\char`\{\ \$c e \$.}
%   \underbrace{
%     \ \ldots\ %
%     \mbox{\tt \$\char`\{}\ \ldots\ %
%   }_{\mbox{\rm region A}}
%   \mbox{\tt \$v e \$.}
%   \underbrace{
%     \mbox{\ \ \ \ldots\ \ \ }
%   }_{\mbox{\rm region B}}
%   \mbox{\tt \$\char`\}}
%   \underbrace{
%     \mbox{\ \ \ \ldots\ \ \ }
%   }_{\mbox{\rm region C}}
%   \mbox{\tt \$\char`\}}
% \]
% \index{\texttt{\$\char`\{} and \texttt{\$\char`\}} keywords}
% In region A, the token \texttt{e} refers to a constant.  It is redeclared as a
% variable in region B, and any reference to it in this region will refer to this
% variable.  In region C, the redeclaration becomes inactive, and the original
% declaration becomes active again.  In region C, the token \texttt{x} refers to the
% original constant.
%
% As a practical matter, overuse of math symbol\index{math symbol}
% redeclarations\index{redeclaration of symbols} can be confusing (even though
% it is well-defined) and is best avoided when possible.  Here are some good
% general guidelines you can follow.  Usually, you should declare all
% constants\index{constant} in the outermost block\index{outermost block},
% especially if they are general-purpose (such as the token \verb$A.$, meaning
% $\forall$ or ``for all'').  This will make them ``globally'' active (although
% as in the example above local redeclarations will temporarily make them
% inactive.)  Most or all variables\index{variable}, on the other hand, could be
% declared in inner blocks, so that the token for them can be used later for a
% different type of variable or a constant.  (The names of the variables you
% choose are not used when you refer to an assertion\index{assertion} in a
% proof, whereas constants must match exactly.  A locally declared constant will
% not match a globally declared constant in a proof, even if they use the same
% token, because Metamath internally considers them to be different math
% symbols.)  To avoid confusion, you should generally avoid redeclaring active
% variables.  If you must redeclare them, do so at the beginning of a block.
% The temporary declaration of constants in inner blocks might be occasionally
% appropriate when you make use of a temporary definition to prove lemmas
% leading to a main result that does not make direct use of the definition.
% This way, you will not clutter up your database with a large number of
% seldom-used global constant symbols.  You might want to note that while
% inactive constants may not appear directly in an assertion (a \texttt{\$a}\index{\texttt{\$a}
% statement} or \texttt{\$p}\index{\texttt{\$p} statement}
% statement), they may be indirectly used in the proof of a \texttt{\$p} statement
% so long as they do not appear in the final math symbol sequence constructed by
% the proof.  In the end, you will have to use your best judgment, taking into
% account standard mathematical usage of the symbols as well as consideration
% for the reader of your work.
%
% \subsubsection{Reuse of Labels}\index{reuse of labels}\index{label}
%
% The \texttt{\$e}\index{\texttt{\$e} statement}, \texttt{\$f}\index{\texttt{\$f}
% statement}, \texttt{\$a}\index{\texttt{\$a} statement}, and
% \texttt{\$p}\index{\texttt{\$p}
% statement} statement types require labels, which allow them to be
% referenced later inside proofs.  A label is considered {\bf
% active}\index{active label} when the statement it is associated with is
% active.  The token\index{token} for a label may be reused
% (redeclared)\index{redeclaration of labels} provided that it is not being used
% for a currently active label.  (Unlike the tokens for math symbols, active
% label tokens may not be redeclared in an inner scope.)  Note that the labels
% of \texttt{\$a} and \texttt{\$p} statements can never be reused after these
% statements appear, because these statements remain active through the end of
% the database.
%
% You might find the reuse of labels a convenient way to have standard names for
% temporary hypotheses, such as \texttt{h1}, \texttt{h2}, etc.  This way you don't have
% to invent unique names for each of them, and in some cases it may be less
% confusing to the reader (although in other cases it might be more confusing, if
% the hypothesis is located far away from the assertion that uses
% it).\footnote{The current implementation requires that all labels, even
% inactive ones, be unique.}

\subsubsection{Frames Revisited}\index{frames and scoping statements}

Now that we have covered scoping, we will look at how an arbitrary
Metamath database can be converted to the simple sequence of extended
frames described on p.~\pageref{framelist}.  This is also how Metamath
stores the database internally when it reads in the database
source.\label{frameconvert} The method is simple.  First, we collect all
constant and variable (\texttt{\$c} and \texttt{\$v}) declarations in
the database, ignoring duplicate declarations of the same variable in
different scopes.  We then put our collected \texttt{\$c} and
\texttt{\$v} declarations at the beginning of the database, so that
their scope is the entire database.  Next, for each assertion in the
database, we determine its frame and extended frame.  The extended frame
is simply the \texttt{\$f}, \texttt{\$e}, and \texttt{\$d} statements
that are active.  The frame is the extended frame with all optional
hypotheses removed.

An equivalent way of saying this is that the extended frame of an assertion
is the collection of all \texttt{\$f}, \texttt{\$e}, and \texttt{\$d} statements
whose scope includes the assertion.
The \texttt{\$f} and \texttt{\$e} statements
occur in the order they appear
(order is irrelevant for \texttt{\$d} statements).

%c%, renaming any
%c%redeclared variables as needed so that all of them have unique names.  (The
%c%exact renaming convention is unimportant.  You might imagine renaming
%c%different declarations of math symbol \texttt{a} as \texttt{a\$1}, \texttt{a\$2}, etc.\
%c%which would prevent any conflicts since \texttt{\$} is not a legal character in a
%c%math symbol token.)

\section{The Anatomy of a Proof} \label{proof}
\index{proof!Metamath, description of}

Each provable assertion (\texttt{\$p}\index{\texttt{\$p} statement} statement) in a
database must include a {\bf proof}\index{proof}.  The proof is located
between the \texttt{\$=}\index{\texttt{\$=} keyword} and \texttt{\$.}\ keywords in the
\texttt{\$p} statement.

In the basic Metamath language\index{basic language}, a proof is a
sequence of statement labels.  This label sequence\index{label sequence}
serves as a set of instructions that the Metamath program uses to
construct a series of math symbol sequences.  The construction must
ultimately result in the math symbol sequence contained between the
\texttt{\$p}\index{\texttt{\$p} statement} and
\texttt{\$=}\index{\texttt{\$=} keyword} keywords of the \texttt{\$p}
statement.  Otherwise, the Metamath program will consider the proof
incorrect, and it will notify you with an appropriate error message when
you ask it to verify the proof.\footnote{To make the loading faster, the
Metamath program does not automatically verify proofs when you
\texttt{read} in a database unless you use the \texttt{/verify}
qualifier.  After a database has been read in, you may use the
\texttt{verify proof *} command to verify proofs.}\index{\texttt{verify
proof} command} Each label in a proof is said to {\bf
reference}\index{label reference} its corresponding statement.

Associated with any assertion\index{assertion} (\texttt{\$p} or
\texttt{\$a}\index{\texttt{\$a} statement} statement) is a set of
hypotheses (\texttt{\$f}\index{\texttt{\$f} statement} or
\texttt{\$e}\index{\texttt{\$e} statement} statements) that are active
with respect to that assertion.  Some are mandatory and the others are
optional.  You should review these concepts if necessary.

Each label\index{label} in a proof must be either the label of a
previous assertion (\texttt{\$a}\index{\texttt{\$a} statement} or
\texttt{\$p}\index{\texttt{\$p} statement} statement) or the label of an
active hypothesis (\texttt{\$e} or \texttt{\$f}\index{\texttt{\$f}
statement} statement) of the \texttt{\$p} statement containing the
proof.  Hypothesis labels may reference both the
mandatory\index{mandatory hypothesis} and the optional hypotheses of the
\texttt{\$p} statement.

The label sequence in a proof specifies a construction in {\bf reverse Polish
notation}\index{reverse Polish notation (RPN)} (RPN).  You may be familiar
with RPN if you have used older
Hewlett--Packard or similar hand-held calculators.
In the calculator analogy, a hypothesis label\index{hypothesis label} is like
a number and an assertion label\index{assertion label} is like an operation
(more precisely, an $n$-ary operation when the
assertion has $n$ \texttt{\$e}-hypotheses).
On an RPN calculator, an operation takes one or more previous numbers in an
input sequence, performs a calculation on them, and replaces those numbers and
itself with the result of the calculation.  For example, the input sequence
$2,3,+$ on an RPN calculator results in $5$, and the input sequence
$2,3,5,{\times},+$ results in $2,15,+$ which results in $17$.

Understanding how RPN is processed involves the concept of a {\bf
stack}\index{stack}\index{RPN stack}, which can be thought of as a set of
temporary memory locations that hold intermediate results.  When Metamath
encounters a hypothesis label it places or {\bf pushes}\index{push} the math
symbol sequence of the hypothesis onto the stack.  When Metamath encounters an
assertion label, it associates the most recent stack entries with the {\em
mandatory} hypotheses\index{mandatory hypothesis} of the assertion, in the
order where the most recent stack entry is associated with the last mandatory
hypothesis of the assertion.  It then determines what
substitutions\index{substitution!variable}\index{variable substitution} have
to be made into the variables of the assertion's mandatory hypotheses to make
them identical to the associated stack entries.  It then makes those same
substitutions into the assertion itself.  Finally, Metamath removes or {\bf
pops}\index{pop} the matched hypotheses from the stack and pushes the
substituted assertion onto the stack.

For the purpose of matching the mandatory hypothesis to the most recent stack
entries, whether a hypothesis is a \texttt{\$e} or \texttt{\$f} statement is
irrelevant.  The only important thing is that a set of
substitutions\footnote{In the Metamath spec (Section~\ref{spec}), we use the
singular term ``substitution'' to refer to the set of substitutions we talk
about here.} exist that allow a match (and if they don't, the proof verifier
will let you know with an error message).  The Metamath language is specified
in such a way that if a set of substitutions exists, it will be unique.
Specifically, the requirement that each variable have a type specified for it
with a \texttt{\$f} statement ensures the uniqueness.

We will illustrate this with an example.
Consider the following Metamath source file:
\begin{verbatim}
$c ( ) -> wff $.
$v p q r s $.
wp $f wff p $.
wq $f wff q $.
wr $f wff r $.
ws $f wff s $.
w2 $a wff ( p -> q ) $.
wnew $p wff ( s -> ( r -> p ) ) $= ws wr wp w2 w2 $.
\end{verbatim}
This Metamath source example shows the definition and ``proof'' (i.e.,
construction) of a well-formed formula (wff)\index{well-formed formula (wff)}
in propositional calculus.  (You may wish to type this example into a file to
experiment with the Metamath program.)  The first two statements declare
(introduce the names of) four constants and four variables.  The next four
statements specify the variable types, namely that
each variable is assumed to be a wff.  Statement \texttt{w2} defines (postulates)
a way to produce a new wff, \texttt{( p -> q )}, from two given wffs \texttt{p} and
\texttt{q}. The mandatory hypotheses of \texttt{w2} are \texttt{wp} and \texttt{wq}.
Statement \texttt{wnew} claims that \texttt{( s -> ( r -> p ) )} is a wff given
three wffs \texttt{s}, \texttt{r}, and \texttt{p}.  More precisely, \texttt{wnew} claims
that the sequence of ten symbols \texttt{wff ( s -> ( r -> p ) )} is provable from
previous assertions and the hypotheses of \texttt{wnew}.  Metamath does not know
or care what a wff is, and as far as it is concerned
the typecode \texttt{wff} is just an
arbitrary constant symbol in a math symbol sequence.  The mandatory hypotheses
of \texttt{wnew} are \texttt{wp}, \texttt{wr}, and \texttt{ws}; \texttt{wq} is an optional
hypothesis.  In our particular proof, the optional hypothesis is not
referenced, but in general, any combination of active (i.e.\ optional and
mandatory) hypotheses could be referenced.  The proof of statement \texttt{wnew}
is the sequence of five labels starting with \texttt{ws} (step~1) and ending with
\texttt{w2} (step~5).

When Metamath verifies the proof, it scans the proof from left to right.  We
will examine what happens at each step of the proof.  The stack starts off
empty.  At step 1, Metamath looks up label \texttt{ws} and determines that it is a
hypothesis, so it pushes the symbol sequence of statement \texttt{ws} onto the
stack:

\begin{center}\begin{tabular}{|l|l|}\hline
{Stack location} & {Contents} \\ \hline \hline
1 & \texttt{wff s} \\ \hline
\end{tabular}\end{center}

Metamath sees that the labels \texttt{wr} and \texttt{wp} in steps~2 and 3 are also
hypotheses, so it pushes them onto the stack.  After step~3, the stack looks
like
this:

\begin{center}\begin{tabular}{|l|l|}\hline
{Stack location} & {Contents} \\ \hline \hline
3 & \texttt{wff p} \\ \hline
2 & \texttt{wff r} \\ \hline
1 & \texttt{wff s} \\ \hline
\end{tabular}\end{center}

At step 4, Metamath sees that label \texttt{w2} is an assertion, so it must do
some processing.  First, it associates the mandatory hypotheses of \texttt{w2},
which are \texttt{wp} and \texttt{wq}, with stack locations~2 and 3, {\em in that
order}. Metamath determines that the only possible way
to make hypothesis \texttt{wp} match (become identical to) stack location~2 and
\texttt{wq} match stack location 3 is to substitute variable \texttt{p} with \texttt{r}
and \texttt{q} with \texttt{p}.  Metamath makes these substitutions into \texttt{w2} and
obtains the symbol sequence \texttt{wff ( r -> p )}.  It removes the hypotheses
from stack locations~2 and 3, then places the result into stack location~2:

\begin{center}\begin{tabular}{|l|l|}\hline
{Stack location} & {Contents} \\ \hline \hline
2 & \texttt{wff ( r -> p )} \\ \hline
1 & \texttt{wff s} \\ \hline
\end{tabular}\end{center}

At step 5, Metamath sees that label \texttt{w2} is an assertion, so it must again
do some processing.  First, it matches the mandatory hypotheses of \texttt{w2},
which are \texttt{wp} and \texttt{wq}, to stack locations 1 and 2.
Metamath determines that the only possible way to make the
hypotheses match is to substitute variable \texttt{p} with \texttt{s} and \texttt{q} with
\texttt{( r -> p )}.  Metamath makes these substitutions into \texttt{w2} and obtains
the symbol
sequence \texttt{wff ( s -> ( r -> p ) )}.  It removes stack
locations 1 and 2, then places the result into stack location~1:

\begin{center}\begin{tabular}{|l|l|}\hline
{Stack location} & {Contents} \\ \hline \hline
1 & \texttt{wff ( s -> ( r -> p ) )} \\ \hline
\end{tabular}\end{center}

After Metamath finishes processing the proof, it checks to see that the
stack contains exactly one element and that this element is
the same as the math symbol sequence in the
\texttt{\$p}\index{\texttt{\$p} statement} statement.  This is the case for our
proof of \texttt{wnew},
so we have proved \texttt{wnew} successfully.  If the result
differs, Metamath will notify you with an error message.  An error message
will also result if the stack contains more than one entry at the end of the
proof, or if the stack did not contain enough entries at any point in the
proof to match all of the mandatory hypotheses\index{mandatory hypothesis} of
an assertion.  Finally, Metamath will notify you with an error message if no
substitution is possible that will make a referenced assertion's hypothesis
match the
stack entries.  You may want to experiment with the different kinds of errors
that Metamath will detect by making some small changes in the proof of our
example.

Metamath's proof notation was designed primarily to express proofs in a
relatively compact manner, not for readability by humans.  Metamath can display
proofs in a number of different ways with the \texttt{show proof}\index{\texttt{show
proof} command} command.  The
\texttt{/lemmon} qualifier displays it in a format that is easier to read when the
proofs are short, and you saw examples of its use in Chapter~\ref{using}.  For
longer proofs, it is useful to see the tree structure of the proof.  A tree
structure is displayed when the \texttt{/lemmon} qualifier is omitted.  You will
probably find this display more convenient as you get used to it. The tree
display of the proof in our example looks like
this:\label{treeproof}\index{tree-style proof}\index{proof!tree-style}
\begin{verbatim}
1     wp=ws    $f wff s
2        wp=wr    $f wff r
3        wq=wp    $f wff p
4     wq=w2    $a wff ( r -> p )
5  wnew=w2  $a wff ( s -> ( r -> p ) )
\end{verbatim}
The number to the left of each line is the step number.  Following it is a
{\bf hypothesis association}\index{hypothesis association}, consisting of two
labels\index{label} separated by \texttt{=}.  To the left of the \texttt{=} (except
in the last step) is the label of a hypothesis of an assertion referenced
later in the proof; here, steps 1 and 4 are the hypothesis associations for
the assertion \texttt{w2} that is referenced in step 5.  A hypothesis association
is indented one level more than the assertion that uses it, so it is easy to
find the corresponding assertion by moving directly down until the indentation
level decreases to one less than where you started from.  To the right of each
\texttt{=} is the proof step label for that proof step.  The statement keyword of
the proof step label is listed next, followed by the content of the top of the
stack (the most recent stack entry) as it exists after that proof step is
processed.  With a little practice, you should have no trouble reading proofs
displayed in this format.

Metamath proofs include the syntax construction of a formula.
In standard mathematics, this kind of
construction is not considered a proper part of the proof at all, and it
certainly becomes rather boring after a while.
Therefore,
by default the \texttt{show proof}\index{\texttt{show proof}
command} command does not show the syntax construction.
Historically \texttt{show proof} command
\textit{did} show the syntax construction, and you needed to add the
\texttt{/essential} option to hide, them, but today
\texttt{/essential} is the default and you need to use
\texttt{/all} to see the syntax constructions.

When verifying a proof, Metamath will check that no mandatory
\texttt{\$d}\index{\texttt{\$d} statement}\index{mandatory \texttt{\$d}
statement} statement of an assertion referenced in a proof is violated
when substitutions\index{substitution!variable}\index{variable
substitution} are made to the variables in the assertion.  For details
see Section~\ref{spec4} or \ref{dollard}.

\subsection{The Concept of Unification} \label{unify}

During the course of verifying a proof, when Metamath\index{Metamath}
encounters an assertion label\index{assertion label}, it associates the
mandatory hypotheses\index{mandatory hypothesis} of the assertion with the top
entries of the RPN stack\index{stack}\index{RPN stack}.  Metamath then
determines what substitutions\index{substitution!variable}\index{variable
substitution} it must make to the variables in the assertion's mandatory
hypotheses in order for these hypotheses to become identical to their
corresponding stack entries.  This process is called {\bf
unification}\index{unification}.  (We also informally use the term
``unification'' to refer to a set of substitutions that results from the
process, as in ``two unifications are possible.'')  After the substitutions
are made, the hypotheses are said to be {\bf unified}.

If no such substitutions are possible, Metamath will consider the proof
incorrect and notify you with an error message.
% (deleted 3/10/07, per suggestion of Mel O'Cat:)
% The syntax of the
% Metamath language ensures that if a set of substitutions exists, it
% will be unique.

The general algorithm for unification described in the literature is
somewhat complex.
However, in the case of Metamath it is intentionally trivial.
Mandatory hypotheses must be
pushed on the proof stack in the order in which they appear.
In addition, each variable must have its type specified
with a \texttt{\$f} hypothesis before it is used
and that each \texttt{\$f} hypothesis
have the restricted syntax of a typecode (a constant) followed by a variable.
The typecode in the \texttt{\$f} hypothesis must match the first symbol of
the corresponding RPN stack entry (which will also be a constant), so
the only possible match for the variable in the \texttt{\$f} hypothesis is
the sequence of symbols in the stack entry after the initial constant.

In the Proof Assistant\index{Proof Assistant}, a more general unification
algorithm is used.  While a proof is being developed, sometimes not enough
information is available to determine a unique unification.  In this case
Metamath will ask you to pick the correct one.\index{ambiguous
unification}\index{unification!ambiguous}

\section{Extensions to the Metamath Language}\index{extended
language}

\subsection{Comments in the Metamath Language}\label{comments}
\index{markup notation}
\index{comments!markup notation}

The commenting feature allows you to annotate the contents of
a database.  Just as with most
computer languages, comments are ignored for the purpose of interpreting the
contents of the database. Comments effectively act as
additional white space\index{white
space} between tokens
when a database is parsed.

A comment may be placed at the beginning, end, or
between any two tokens\index{token} in a source file.

Comments have the following syntax:
\begin{center}
 \texttt{\$(} {\em text} \texttt{\$)}
\end{center}
Here,\index{\texttt{\$(} and \texttt{\$)} auxiliary
keywords}\index{comment} {\em text} is a string, possibly empty, of any
characters in Metamath's character set (p.~\pageref{spec1chars}), except
that the character strings \texttt{\$(} and \texttt{\$)} may not appear
in {\em text}.  Thus nested comments are not
permitted:\footnote{Computer languages have differing standards for
nested comments, and rather than picking one it was felt simplest not to
allow them at all, at least in the current version of
Metamath\index{Metamath!limitations of version 0.07.30}.} Metamath will
complain if you give it
\begin{center}
 \texttt{\$( This is a \$( nested \$) comment.\ \$)}
\end{center}
To compensate for this non-nesting behavior, I often change all \texttt{\$}'s
to \texttt{@}'s in sections of Metamath code I wish to comment out.

The Metamath program supports a number of markup mechanisms and conventions
to generate good-looking results in \LaTeX\ and {\sc html},
as discussed below.
These markup features have to do only with how the comments are typeset,
and have no effect on how Metamath verifies the proofs in the database.
The improper
use of them may result in incorrectly typeset output, but no Metamath
error messages will result during the \texttt{read} and \texttt{verify
proof} commands.  (However, the \texttt{write
theorem\texttt{\char`\_}list} command
will check for markup errors as a side-effect of its
{\sc html} generation.)
Section~\ref{texout} has instructions for creating \LaTeX\ output, and
section~\ref{htmlout} has instructions for creating
{\sc html}\index{HTML} output.

\subsubsection{Headings}\label{commentheadings}

If the \texttt{\$(} is immediately followed by a new line
starting with a heading marker, it is a header.
This can start with:

\begin{itemize}
 \item[] \texttt{\#\#\#\#} - major part header
 \item[] \texttt{\#*\#*} - section header
 \item[] \texttt{=-=-} - subsection header
 \item[] \texttt{-.-.} - subsubsection header
\end{itemize}

The line following the marker line
will be used for the table of contents entry, after trimming spaces.
The next line should be another (closing) matching marker line.
Any text after that
but before the closing \texttt{\$}, such as an extended description of the
section, will be included on the \texttt{mmtheoremsNNN.html} page.

For more information, run
\texttt{help write theorem\char`\_list}.

\subsubsection{Math mode}
\label{mathcomments}
\index{\texttt{`} inside comments}
\index{\texttt{\char`\~} inside comments}
\index{math mode}

Inside of comments, a string of tokens\index{token} enclosed in
grave accents\index{grave accent (\texttt{`})} (\texttt{`}) will be converted
to standard mathematical symbols during
{\sc HTML}\index{HTML} or \LaTeX\ output
typesetting,\index{latex@{\LaTeX}} according to the information in the
special \texttt{\$t}\index{\texttt{\$t} comment}\index{typesetting
comment} comment in the database
(see section~\ref{tcomment} for information about the typesetting
comment, and Appendix~\ref{ASCII} to see examples of its results).

The first grave accent\index{grave accent (\texttt{`})} \texttt{`}
causes the output processor to enter {\bf math mode}\index{math mode}
and the second one exits it.
In this
mode, the characters following the \texttt{`} are interpreted as a
sequence of math symbol tokens separated by white space\index{white
space}.  The tokens are looked up in the \texttt{\$t}
comment\index{\texttt{\$t} comment}\index{typesetting comment} and if
found, they will be replaced by the standard mathematical symbols that
they correspond to before being placed in the typeset output file.  If
not found, the symbol will be output as is and a warning will be issued.
The tokens do not have to be active in the database, although a warning
will be issued if they are not declared with \texttt{\$c} or
\texttt{\$v} statements.

Two consecutive
grave accents \texttt{``} are treated as a single actual grave accent
(both inside and outside of math mode) and will not cause the output
processor to enter or exit math mode.

Here is an example of its use\index{Pierce's axiom}:
\begin{center}
\texttt{\$( Pierce's axiom, ` ( ( ph -> ps ) -> ph ) -> ph ` ,\\
         is not very intuitive. \$)}
\end{center}
becomes
\begin{center}
   \texttt{\$(} Pierce's axiom, $((\varphi \rightarrow \psi)\rightarrow
\varphi)\rightarrow \varphi$, is not very intuitive. \texttt{\$)}
\end{center}

Note that the math symbol tokens\index{token} must be surrounded by white
space\index{white space}.
%, since there is no context that allows ambiguity to be
%resolved, as is the case with math symbol sequences in some of the Metamath
%statements.
White space should also surround the \texttt{`}
delimiters.

The math mode feature also gives you a quick and easy way to generate
text containing mathematical symbols, independently of the intended
purpose of Metamath.\index{Metamath!using as a math editor} To do this,
simply create your text with grave accents surrounding your formulas,
after making sure that your math symbols are mapped to \LaTeX\ symbols
as described in Appendix~\ref{ASCII}.  It is easier if you start with a
database with predefined symbols such as \texttt{set.mm}.  Use your
grave-quoted math string to replace an existing comment, then typeset
the statement corresponding to that comment following the instructions
from the \texttt{help tex} command in the Metamath program.  You will
then probably want to edit the resulting file with a text editor to fine
tune it to your exact needs.

\subsubsection{Label Mode}\index{label mode}

Outside of math mode, a tilde\index{tilde (\texttt{\char`\~})} \verb/~/
indicates to Metamath's\index{Metamath} output processor that the
token\index{token} that follows (i.e.\ the characters up to the next
white space\index{white space}) represents a statement label or URL.
This formatting mode is called {\bf label mode}\index{label mode}.
If a literal tilde
is desired (outside of math mode) instead of label mode,
use two tildes in a row to represent it.

When generating a \LaTeX\ output file,
the following token will be formatted in \texttt{typewriter}
font, and the tilde removed, to make it stand out from the rest of the text.
This formatting will be applied to all characters after the
tilde up to the first white space\index{white space}.
Whether
or not the token is an actual statement label is not checked, and the
token does not have to have the correct syntax for a label; no error
messages will be produced.  The only effect of the label mode on the
output is that typewriter font will be used for the tokens that are
placed in the \LaTeX\ output file.

When generating {\sc html},
the tokens after the tilde {\em must} be a URL (either http: or https:)
or a valid label.
Error messages will be issued during that output if they aren't.
A hyperlink will be generated to that URL or label.

\subsubsection{Link to bibliographical reference}\index{citation}%
\index{link to bibliographical reference}

Bibliographical references are handled specially when generating
{\sc html} if formatted specially.
Text in the form \texttt{[}{\em author}\texttt{]}
is considered a link to a bibliographical reference.
See \texttt{help html} and \texttt{help write
bibliography} in the Metamath program for more
information.
% \index{\texttt{\char`\[}\ldots\texttt{]} inside comments}
See also Sections~\ref{tcomment} and \ref{wrbib}.

The \texttt{[}{\em author}\texttt{]} notation will also create an entry in
the bibliography cross-reference file generated by \texttt{write
bibliography} (Section~\ref{wrbib}) for {\sc HTML}.
For this to work properly, the
surrounding comment must be formatted as follows:
\begin{quote}
    {\em keyword} {\em label} {\em noise-word}
     \texttt{[}{\em author}\texttt{] p.} {\em number}
\end{quote}
for example
\begin{verbatim}
     Theorem 5.2 of [Monk] p. 223
\end{verbatim}
The {\em keyword} is not case sensitive and must be one of the following:
\begin{verbatim}
     theorem lemma definition compare proposition corollary
     axiom rule remark exercise problem notation example
     property figure postulate equation scheme chapter
\end{verbatim}
The optional {\em label} may consist of more than one
(non-{\em keyword} and non-{\em noise-word}) word.
The optional {\em noise-word} is one of:
\begin{verbatim}
     of in from on
\end{verbatim}
and is  ignored when the cross-reference file is created.  The
\texttt{write
biblio\-graphy} command will perform error checking to verify the
above format.\index{error checking}

\subsubsection{Parentheticals}\label{parentheticals}

The end of a comment may include one or more parenthicals, that is,
statements enclosed in parentheses.
The Metamath program looks for certain parentheticals and can issue
warnings based on them.
They are:

\begin{itemize}
 \item[] \texttt{(Contributed by }
   \textit{NAME}\texttt{,} \textit{DATE}\texttt{.)} -
   document the original contributor's name and the date it was created.
 \item[] \texttt{(Revised by }
   \textit{NAME}\texttt{,} \textit{DATE}\texttt{.)} -
   document the contributor's name and creation date
   that resulted in significant revision
   (not just an automated minimization or shortening).
 \item[] \texttt{(Proof shortened by }
   \textit{NAME}\texttt{,} \textit{DATE}\texttt{.)} -
   document the contributor's name and date that developed a significant
   shortening of the proof (not just an automated minimization).
 \item[] \texttt{(Proof modification is discouraged.)} -
   Note that this proof should normally not be modified.
 \item[] \texttt{(New usage is discouraged.)} -
   Note that this assertion should normally not be used.
\end{itemize}

The \textit{DATE} must be in form YYYY-MMM-DD, where MMM is the
English abbreviation of that month.

\subsubsection{Other markup}\label{othermarkup}
\index{markup notation}

There are other markup notations for generating good-looking results
beyond math mode and label mode:

\begin{itemize}
 \item[]
         \texttt{\char`\_} (underscore)\index{\texttt{\char`\_} inside comments} -
             Italicize text starting from
              {\em space}\texttt{\char`\_}{\em non-space} (i.e.\ \texttt{\char`\_}
              with a space before it and a non-space character after it) until
             the next
             {\em non-space}\texttt{\char`\_}{\em space}.  Normal
             punctuation (e.g.\ a trailing
             comma or period) is ignored when determining {\em space}.
 \item[]
         \texttt{\char`\_} (underscore) - {\em
         non-space}\texttt{\char`\_}{\em non-space-string}, where
          {\em non-space-string} is a string of non-space characters,
         will make {\em non-space-string} become a subscript.
 \item[]
         \texttt{<HTML>}...\texttt{</HTML>} - do not convert
         ``\texttt{<}'' and ``\texttt{>}''
         in the enclosed text when generating {\sc HTML},
         otherwise process markup normally. This allows direct insertion
         of {\sc html} commands.
 \item[]
       ``\texttt{\&}ref\texttt{;}'' - insert an {\sc HTML}
         character reference.
         This is how to insert arbitrary Unicode characters
         (such as accented characters).  Currently only directly supported
         when generating {\sc HTML}.
\end{itemize}

It is recommended that spaces surround any \texttt{\char`\~} and
\texttt{`} tokens in the comment and that a space follow the {\em label}
after a \texttt{\char`\~} token.  This will make global substitutions
to change labels and symbol names much easier and also eliminate any
future chance of ambiguity.  Spaces around these tokens are automatically
removed in the final output to conform with normal rules of punctuation;
for example, a space between a trailing \texttt{`} and a left parenthesis
will be removed.

A good way to become familiar with the markup notation is to look at
the extensive examples in the \texttt{set.mm} database.

\subsection{The Typesetting Comment (\texttt{\$t})}\label{tcomment}

The typesetting comment \texttt{\$t} in the input database file
provides the information necessary to produce good-looking results.
It provides \LaTeX\ and {\sc html}
definitions for math symbols,
as well supporting as some
customization of the generated web page.
If you add a new token to a database, you should also
update the \texttt{\$t} comment information if you want to eventually
create output in \LaTeX\ or {\sc HTML}.
See the
\texttt{set.mm}\index{set theory database (\texttt{set.mm})} database
file for an extensive example of a \texttt{\$t} comment illustrating
many of the features described below.

Programs that do not need to generate good-looking presentation results,
such as programs that only verify Metamath databases,
can completely ignore typesetting comments
and just treat them as normal comments.
Even the Metamath program only consults the
\texttt{\$t} comment information when it needs to generate typeset output
in \LaTeX\ or {\sc HTML}
(e.g., when you open a \LaTeX\ output file with the \texttt{open tex} command).

We will first discuss the syntax of typesetting comments, and then
briefly discuss how this can be used within the Metamath program.

\subsubsection{Typesetting Comment Syntax Overview}

The typesetting comment is identified by the token
\texttt{\$t}\index{\texttt{\$t} comment}\index{typesetting comment} in
the comment, and the typesetting comment ends at the matching
\texttt{\$)}:
\[
  \mbox{\tt \$(\ }
  \mbox{\tt \$t\ }
  \underbrace{
    \mbox{\tt \ \ \ \ \ \ \ \ \ \ \ }
    \cdots
    \mbox{\tt \ \ \ \ \ \ \ \ \ \ \ }
  }_{\mbox{Typesetting definitions go here}}
  \mbox{\tt \ \$)}
\]

There must be one or more white space characters, and only white space
characters, between the \texttt{\$(} that starts the comment
and the \texttt{\$t} symbol,
and the \texttt{\$t} must be followed by one
or more white space characters
(see section \ref{whitespace} for the definition of white space characters).
The typesetting comment continues until the comment end token \texttt{\$)}
(which must be preceded by one or more white space characters).

In version 0.07.30\index{Metamath!limitations of version 0.07.30} of the
Metamath program, there may be only one \texttt{\$t} comment in a
database.  This restriction may be lifted in the future to allow
many \texttt{\$t} comments in a database.

Between the \texttt{\$t} symbol (and its following white space) and the
comment end token \texttt{\$)} (and its preceding white space)
is a sequence of one or more typesetting definitions, where
each definition has the form
\textit{definition-type arg arg ... ;}.
Each of the zero or more \textit{arg} values
can be either a typesetting data or a keyword
(what keywords are allowed, and where, depends on the specific
\textit{definition-type}).
The \textit{definition-type}, and each argument \textit{arg},
are separated by one or more white space characters.
Every definition ends in an unquoted semicolon;
white space is not required before the terminating semicolon of a definition.
Each definition should start on a new line.\footnote{This
restriction of the current version of Metamath
(0.07.30)\index{Metamath!limitations of version 0.07.30} may be removed
in a future version, but you should do it anyway for readability.}

For example, this typesetting definition:
\begin{center}
 \verb$latexdef "C_" as "\subseteq";$
\end{center}
defines the token \verb$C_$ as the \LaTeX\ symbol $\subseteq$ (which means
``subset'').

Typesetting data is a sequence of one or more quoted strings
(if there is more than one, they are connected by \texttt{\char`\+}).
Often a single quoted string is used to provide data for a definition, using
either double (\texttt{\char`\"}) or single (\texttt{'}) quotation marks.
However,
{\em a quoted string (enclosed in quotation marks) may not include
line breaks.}
A quoted string
may include a quotation mark that matches the enclosing quotes by repeating
the quotation mark twice.  Here are some examples:

\begin{tabu}   { l l }
\textbf{Example} & \textbf{Meaning} \\
\texttt{\char`\"a\char`\"\char`\"b\char`\"} & \texttt{a\char`\"b} \\
\texttt{'c''d'} & \texttt{c'd} \\
\texttt{\char`\"e''f\char`\"} & \texttt{e''f} \\
\texttt{'g\char`\"\char`\"h'} & \texttt{g\char`\"\char`\"h} \\
\end{tabu}

Finally, a long quoted string
may be broken up into multiple quoted strings (considered, as a whole,
a single quoted string) and joined with \texttt{\char`\+}.
You can even use multiple lines as long as a
'+' is at the end of every line except the last one.
The \texttt{\char`\+} should be preceded and followed by at least one
white space character.
Thus, for example,
\begin{center}
 \texttt{\char`\"ab\char`\"\ \char`\+\ \char`\"cd\char`\"
    \ \char`\+\ \\ 'ef'}
\end{center}
is the same as
\begin{center}
 \texttt{\char`\"abcdef\char`\"}
\end{center}

{\sc c}-style comments \texttt{/*}\ldots\texttt{*/} are also supported.

In practice, whenever you add a new math token you will often want to add
typesetting definitions using
\texttt{latexdef}, \texttt{htmldef}, and
\texttt{althtmldef}, as described below.
That way, they will all be up to date.
Of course, whether or not you want to use all three definitions will
depend on how the database is intended to be used.

Below we discuss the different possible \textit{definition-kind} options.
We will show data surrounded by double quotes (in practice they can also use
single quotes and/or be a sequence joined by \texttt{+}s).
We will use specific names for the \textit{data} to make clear what
the data is used for, such as
{\em math-token} (for a Metamath math token,
{\em latex-string} (for string to be placed in a \LaTeX\ stream),
{\em {\sc html}-code} (for {\sc html} code),
and {\em filename} (for a filename).

\subsubsection{Typesetting Comment - \LaTeX}

The syntax for a \LaTeX\ definition is:
\begin{center}
 \texttt{latexdef "}{\em math-token}\texttt{" as "}{\em latex-string}\texttt{";}
\end{center}
\index{latex definitions@\LaTeX\ definitions}%
\index{\texttt{latexdef} statement}

The {\em token-string} and {\em latex-string} are the data
(character strings) for
the token and the \LaTeX\ definition of the token, respectively,

These \LaTeX\ definitions are used by the Metamath program
when it is asked to product \LaTeX output using
the \texttt{write tex} command.

\subsubsection{Typesetting Comment - {\sc html}}

The key kinds of {\sc HTML} definitions have the following syntax:

\vskip 1ex
    \texttt{htmldef "}{\em math-token}\texttt{" as "}{\em
    {\sc html}-code}\texttt{";}\index{\texttt{htmldef} statement}
                    \ \ \ \ \ \ldots

    \texttt{althtmldef "}{\em math-token}\texttt{" as "}{\em
{\sc html}-code}\texttt{";}\index{\texttt{althtmldef} statement}

                    \ \ \ \ \ \ldots

Note that in {\sc HTML} there are two possible definitions for math tokens.
This feature is useful when
an alternate representation of symbols is desired, for example one that
uses Unicode entities and another uses {\sc gif} images.

There are many other typesetting definitions that can control {\sc HTML}.
These include:

\vskip 1ex

    \texttt{htmldef "}{\em math-token}\texttt{" as "}{\em {\sc
    html}-code}\texttt{";}

    \texttt{htmltitle "}{\em {\sc html}-code}\texttt{";}%
\index{\texttt{htmltitle} statement}

    \texttt{htmlhome "}{\em {\sc html}-code}\texttt{";}%
\index{\texttt{htmlhome} statement}

    \texttt{htmlvarcolor "}{\em {\sc html}-code}\texttt{";}%
\index{\texttt{htmlvarcolor} statement}

    \texttt{htmlbibliography "}{\em filename}\texttt{";}%
\index{\texttt{htmlbibliography} statement}

\vskip 1ex

\noindent The \texttt{htmltitle} is the {\sc html} code for a common
title, such as ``Metamath Proof Explorer.''  The \texttt{htmlhome} is
code for a link back to the home page.  The \texttt{htmlvarcolor} is
code for a color key that appears at the bottom of each proof.  The file
specified by {\em filename} is an {\sc html} file that is assumed to
have a \texttt{<A NAME=}\ldots\texttt{>} tag for each bibiographic
reference in the database comments.  For example, if
\texttt{[Monk]}\index{\texttt{\char`\[}\ldots\texttt{]} inside comments}
occurs in the comment for a theorem, then \texttt{<A NAME='Monk'>} must
be present in the file; if not, a warning message is given.

Associated with
\texttt{althtmldef}
are the statements
\vskip 1ex

    \texttt{htmldir "}{\em
      directoryname}\texttt{";}\index{\texttt{htmldir} statement}

    \texttt{althtmldir "}{\em
     directoryname}\texttt{";}\index{\texttt{althtmldir} statement}

\vskip 1ex
\noindent giving the directories of the {\sc gif} and Unicode versions
respectively; their purpose is to provide cross-linking between the
two versions in the generated web pages.

When two different types of pages need to be produced from a single
database, such as the Hilbert Space Explorer that extends the Metamath
Proof Explorer, ``extended'' variables may be declared in the
\texttt{\$t} comment:
\vskip 1ex

    \texttt{exthtmltitle "}{\em {\sc html}-code}\texttt{";}%
\index{\texttt{exthtmltitle} statement}

    \texttt{exthtmlhome "}{\em {\sc html}-code}\texttt{";}%
\index{\texttt{exthtmlhome} statement}

    \texttt{exthtmlbibliography "}{\em filename}\texttt{";}%
\index{\texttt{exthtmlbibliography} statement}

\vskip 1ex
\noindent When these are declared, you also must declare
\vskip 1ex

    \texttt{exthtmllabel "}{\em label}\texttt{";}%
\index{\texttt{exthtmllabel} statement}

\vskip 1ex \noindent that identifies the database statement where the
``extended'' section of the database starts (in our example, where the
Hilbert Space Explorer starts).  During the generation of web pages for
that starting statement and the statements after it, the {\sc html} code
assigned to \texttt{exthtmltitle} and \texttt{exthtmlhome} is used
instead of that assigned to \texttt{htmltitle} and \texttt{htmlhome},
respectively.

\begin{sloppy}
\subsection{Additional Information Com\-ment (\texttt{\$j})} \label{jcomment}
\end{sloppy}

The additional information comment, aka the
\texttt{\$j}\index{\texttt{\$j} comment}\index{additional information comment}
comment,
provides a way to add additional structured information that can
be optionally parsed by systems.

The additional information comment is parsed the same way as the
typesetting comment (\texttt{\$t}) (see section \ref{tcomment}).
That is,
the additional information comment begins with the token
\texttt{\$j} within a comment,
and continues until the comment close \texttt{\$)}.
Within an additional information comment is a sequence of one or more
commands of the form \texttt{command arg arg ... ;}
where each of the zero or more \texttt{arg} values
can be either a quoted string or a keyword.
Note that every command ends in an unquoted semicolon.
If a verifier is parsing an additional information comment, but
doesn't recognize a particular command, it must skip the command
by finding the end of the command (an unquoted semicolon).

A database may have 0 or more additional information comments.
Note, however, that a verifier may ignore these comments entirely or only
process certain commands in an additional information comment.
The \texttt{mmj2} verifier supports many commands in additional information
comments.
We encourage systems that process additional information comments
to coordinate so that they will use the same command for the same effect.

Examples of additional information comments with various commands
(from the \texttt{set.mm} database) are:

\begin{itemize}
   \item Define the syntax and logical typecodes,
     and declare that our grammar is
     unambiguous (verifiable using the KLR parser, with compositing depth 5).
\begin{verbatim}
  $( $j
    syntax 'wff';
    syntax '|-' as 'wff';
    unambiguous 'klr 5';
  $)
\end{verbatim}

   \item Register $\lnot$ and $\rightarrow$ as primitive expressions
           (lacking definitions).
\begin{verbatim}
  $( $j primitive 'wn' 'wi'; $)
\end{verbatim}

   \item There is a special justification for \texttt{df-bi}.
\begin{verbatim}
  $( $j justification 'bijust' for 'df-bi'; $)
\end{verbatim}

   \item Register $\leftrightarrow$ as an equality for its type (wff).
\begin{verbatim}
  $( $j
    equality 'wb' from 'biid' 'bicomi' 'bitri';
    definition 'dfbi1' for 'wb';
  $)
\end{verbatim}

   \item Theorem \texttt{notbii} is the congruence law for negation.
\begin{verbatim}
  $( $j congruence 'notbii'; $)
\end{verbatim}

   \item Add \texttt{setvar} as a typecode.
\begin{verbatim}
  $( $j syntax 'setvar'; $)
\end{verbatim}

   \item Register $=$ as an equality for its type (\texttt{class}).
\begin{verbatim}
  $( $j equality 'wceq' from 'eqid' 'eqcomi' 'eqtri'; $)
\end{verbatim}

\end{itemize}


\subsection{Including Other Files in a Metamath Source File} \label{include}
\index{\texttt{\$[} and \texttt{\$]} auxiliary keywords}

The keywords \texttt{\$[} and \texttt{\$]} specify a file to be
included\index{included file}\index{file inclusion} at that point in a
Metamath\index{Metamath} source file\index{source file}.  The syntax for
including a file is as follows:
\begin{center}
\texttt{\$[} {\em file-name} \texttt{\$]}
\end{center}

The {\em file-name} should be a single token\index{token} with the same syntax
as a math symbol (i.e., all 93 non-whitespace
printable characters other than \texttt{\$} are
allowed, subject to the file-naming limitations of your operating system).
Comments may appear between the \texttt{\$[} and \texttt{\$]} keywords.  Included
files may include other files, which may in turn include other files, and so
on.

For example, suppose you want to use the set theory database as the starting
point for your own theory.  The first line in your file could be
\begin{center}
\texttt{\$[ set.mm \$]}
\end{center} All of the information (axioms, theorems,
etc.) in \texttt{set.mm} and any files that {\em it} includes will become
available for you to reference in your file. This can help make your work more
modular. A drawback to including files is that if you change the name of a
symbol or the label of a statement, you must also remember to update any
references in any file that includes it.


The naming conventions for included files are the same as those of your
operating system.\footnote{On the Macintosh, prior to Mac OS X,
 a colon is used to separate disk
and folder names from your file name.  For example, {\em volume}\texttt{:}{\em
file-name} refers to the root directory, {\em volume}\texttt{:}{\em
folder-name}\texttt{:}{\em file-name} refers to a folder in root, and {\em
volume}\texttt{:}{\em folder-name}\texttt{:}\ldots\texttt{:}{\em file-name} refers to a
deeper folder.  A simple {\em file-name} refers to a file in the folder from
which you launch the Metamath application.  Under Mac OS X and later,
the Metamath program is run under the Terminal application, which
conforms to Unix naming conventions.}\index{Macintosh file
names}\index{file names!Macintosh}\label{includef} For compatibility among
operating systems, you should keep the file names as simple as possible.  A
good convention to use is {\em file}\texttt{.mm} where {\em file} is eight
characters or less, in lower case.

There is no limit to the nesting depth of included files.  One thing that you
should be aware of is that if two included files themselves include a common
third file, only the {\em first} reference to this common file will be read
in.  This allows you to include two or more files that build on a common
starting file without having to worry about label and symbol conflicts that
would occur if the common file were read in more than once.  (In fact, if a
file includes itself, the self-reference will be ignored, although of course
it would not make any sense to do that.)  This feature also means, however,
that if you try to include a common file in several inner blocks, the result
might not be what you expect, since only the first reference will be replaced
with the included file (unlike the include statement in most other computer
languages).  Thus you would normally include common files only in the
outermost block\index{outermost block}.

\subsection{Compressed Proof Format}\label{compressed1}\index{compressed
proof}\index{proof!compressed}

The proof notation presented in Section~\ref{proof} is called a
{\bf normal proof}\index{normal proof}\index{proof!normal} and in principle is
sufficient to express any proof.  However, proofs often contain steps and
subproofs that are identical.  This is particularly true in typical
Metamath\index{Metamath} applications, because Metamath requires that the math
symbol sequence (usually containing a formula) at each step be separately
constructed, that is, built up piece by piece. As a result, a lot of
repetition often results.  The {\bf compressed proof} format allows Metamath
to take advantage of this redundancy to shorten proofs.

The specification for the compressed proof format is given in
Appen\-dix~\ref{compressed}.

Normally you need not concern yourself with the details of the compressed
proof format, since the Metamath program will allow you to convert from
the normal format to the compressed format with ease, and will also
automatically convert from the compressed format when proofs are displayed.
The overall structure of the compressed format is as follows:
\begin{center}
  \texttt{\$= ( } {\em label-list} \texttt{) } {\em compressed-proof\ }\ \texttt{\$.}
\end{center}
\index{\texttt{\$=} keyword}
The first \texttt{(} serves as a flag to Metamath that a compressed proof
follows.  The {\em label-list} includes all statements referred to by the
proof except the mandatory hypotheses\index{mandatory hypothesis}.  The {\em
compressed-proof} is a compact encoding of the proof, using upper-case
letters, and can be thought of as a large integer in base 26.  White
space\index{white space} inside a {\em compressed-proof} is
optional and is ignored.

It is important to note that the order of the mandatory hypotheses of
the statement being proved must not be changed if the compressed proof
format is used, otherwise the proof will become incorrect.  The reason
for this is that the mandatory hypotheses are not mentioned explicitly
in the compressed proof in order to make the compression more efficient.
If you wish to change the order of mandatory hypotheses, you must first
convert the proof back to normal format using the \texttt{save proof
{\em statement} /normal}\index{\texttt{save proof} command} command.
Later, you can go back to compressed format with \texttt{save proof {\em
statement} /compressed}.

During error checking with the \texttt{verify proof} command, an error
found in a compressed proof may point to a character in {\em
compressed-proof}, which may not be very meaningful to you.  In this
case, try to \texttt{save proof /normal} first, then do the
\texttt{verify proof} again.  In general, it is best to make sure a
proof is correct before saving it in compressed format, because severe
errors are less likely to be recoverable than in normal format.

\subsection{Specifying Unknown Proofs or Subproofs}\label{unknown}

In a proof under development, any step or subproof that is not yet known
may be represented with a single \texttt{?}.  For the purposes of
parsing the proof, the \texttt{?}\ \index{\texttt{]}@\texttt{?}\ inside
proofs} will push a single entry onto the RPN stack just as if it were a
hypothesis.  While developing a proof with the Proof
Assistant\index{Proof Assistant}, a partially developed proof may be
saved with the \texttt{save new{\char`\_}proof}\index{\texttt{save
new{\char`\_}proof} command} command, and \texttt{?}'s will be placed at
the appropriate places.

All \texttt{\$p}\index{\texttt{\$p} statement} statements must have
proofs, even if they are entirely unknown.  Before creating a proof with
the Proof Assistant, you should specify a completely unknown proof as
follows:
\begin{center}
  {\em label} \texttt{\$p} {\em statement} \texttt{\$= ?\ \$.}
\end{center}
\index{\texttt{\$=} keyword}
\index{\texttt{]}@\texttt{?}\ inside proofs}

The \texttt{verify proof}\index{\texttt{verify proof} command} command
will check the known portions of a partial proof for errors, but will
warn you that the statement has not been proved.

Note that partially developed proofs may be saved in compressed format
if desired.  In this case, you will see one or more \texttt{?}'s in the
{\em compressed-proof} part.\index{compressed
proof}\index{proof!compressed}

\section{Axioms vs.\ Definitions}\label{definitions}

The \textit{basic}
Metamath\index{Metamath} language and program
make no distinction\index{axiom vs.\
definition} between axioms\index{axiom} and
definitions.\index{definition} The \texttt{\$a}\index{\texttt{\$a}
statement} statement is used for both.  At first, this may seem
puzzling.  In the minds of many mathematicians, the distinction is
clear, even obvious, and hardly worth discussing.  A definition is
considered to be merely an abbreviation that can be replaced by the
expression for which it stands; although unless one actually does this,
to be precise then one should say that a theorem\index{theorem} is a
consequence of the axioms {\em and} the definitions that are used in the
formulation of the theorem \cite[p.~20]{Behnke}.\index{Behnke, H.}

\subsection{What is a Definition?}

What is a definition?  In its simplest form, a definition introduces a new
symbol and provides an unambiguous rule to transform an expression containing
the new symbol to one without it.  The concept of a ``proper
definition''\index{proper definition}\index{definition!proper} (as opposed to
a creative definition)\index{creative definition}\index{definition!creative}
that is usually agreed upon is (1) the definition should not strengthen the
language and (2) any symbols introduced by the definition should be eliminable
from the language \cite{Nemesszeghy}\index{Nemesszeghy, E. Z.}.  In other
words, they are mere typographical conveniences that do not belong to the
system and are theoretically superfluous.  This may seem obvious, but in fact
the nature of definitions can be subtle, sometimes requiring difficult
metatheorems to establish that they are not creative.

A more conservative stance was taken by logician S.
Le\'{s}niewski.\index{Le\'{s}niewski, S.}
\begin{quote}
Le\'{s}niewski
regards definitions as theses of the system.  In this respect they do
not differ either from the axioms or from theorems, i.e.\ from the
theses added to the system on the basis of the rule of substitution or
the rule of detachment [modus ponens].  Once definitions have been
accepted as theses of the system, it becomes necessary to consider them
as true propositions in the same sense in which axioms are true
\cite{Lejewski}.
\end{quote}\index{Lejewski, Czeslaw}

Let us look at some simple examples of definitions in propositional
calculus.  Consider the definition of logical {\sc or}
(disjunction):\index{disjunction ($\vee$)} ``$P\vee Q$ denotes $\neg P
\rightarrow Q$ (not $P$ implies $Q$).''  It is very easy to recognize a
statement making use of this definition, because it introduces the new
symbol $\vee$ that did not previously exist in the language.  It is easy
to see that no new theorems of the original language will result from
this definition.

Next, consider a definition that eliminates parentheses:  ``$P
\rightarrow Q\rightarrow R$ denotes $P\rightarrow (Q \rightarrow R)$.''
This is more subtle, because no new symbols are introduced.  The reason
this definition is considered proper is that no new symbol sequences
that are valid wffs (well-formed formulas)\index{well-formed formula
(wff)} in the original language will result from the definition, since
``$P \rightarrow Q\rightarrow R$'' is not a wff in the original
language.  Here, we implicitly make use of the fact that there is a
decision procedure that allows us to determine whether or not a symbol
sequence is a wff, and this fact allows us to use symbol sequences that
are not wffs to represent other things (such as wffs) by means of the
definition.  However, to justify the definition as not being creative we
need to prove that ``$P \rightarrow Q\rightarrow R$'' is in fact not a
wff in the original language, and this is more difficult than in the
case where we simply introduce a new symbol.

%Now let's take this reasoning to an extreme.  Propositional calculus is a
%decidable theory,\footnote{This means that a mechanical algorithm exists to
%determine whether or not a wff is a theorem.} so in principle we could make use
%of symbol sequences that are not theorems to represent other things (say, to
%encode actual theorems in a more compact way).  For example, let us extend the
%language by defining a wff ``$P$'' in the extended language as the theorem
%``$P\rightarrow P$''\footnote{This is one of the first theorems proved in the
%Metamath database \texttt{set.mm}.}\index{set
%theory database (\texttt{set.mm})} in the original language whenever ``$P$'' is
%not a theorem in the original language.  In the extended language, any wff
%``$Q$'' thus represents a theorem; to find out what theorem (in the original
%language) ``$Q$'' represents, we determine whether ``$Q$'' is a theorem in the
%original language (before the definition was introduced).  If so, we're done; if
%not, we replace ``$Q$'' by ``$Q\rightarrow Q$'' to eliminate the definition.
%This definition is therefore eliminable, and it does not ``strengthen'' the
%language because any wff that is not a theorem is not in the set of statements
%provable in the original language and thus is available for use by definitions.
%
%Of course, a definition such as this would render practically useless the
%communication of theorems of propositional calculus; but
%this is just a human shortcoming, since we can't always easily discern what is
%and is not a theorem by inspection.  In fact, the extended theory with this
%definition has no more and no less information than the original theory; it just
%expresses certain theorems of the form ``$P\rightarrow P$''
%in a more compact way.
%
%The point here is that what constitutes a proper definition is a matter of
%judgment about whether a symbol sequence can easily be recognized by a human
%as invalid in some sense (for example, not a wff); if so, the symbol sequence
%can be appropriated for use by a definition in order to make the extended
%language more compact.  Metamath\index{Metamath} lacks the ability to make this
%judgment, since as far as Metamath is concerned the definition of a wff, for
%example, is arbitrary.  You define for Metamath how wffs\index{well-formed
%formula (wff)} are constructed according to your own preferred style.  The
%concept of a wff may not even exist in a given formal system\index{formal
%system}.  Metamath treats all definitions as if they were new axioms, and it
%is up to the human mathematician to judge whether the definition is ``proper''
%'\index{proper definition}\index{definition!proper} in some agreed-upon way.

What constitutes a definition\index{definition} versus\index{axiom vs.\
definition} an axiom\index{axiom} is sometimes arbitrary in mathematical
literature.  For example, the connectives $\vee$ ({\sc or}), $\wedge$
({\sc and}), and $\leftrightarrow$ (equivalent to) in propositional
calculus are usually considered defined symbols that can be used as
abbreviations for expressions containing the ``primitive'' connectives
$\rightarrow$ and $\neg$.  This is the way we treat them in the standard
logic and set theory database \texttt{set.mm}\index{set theory database
(\texttt{set.mm})}.  However, the first three connectives can also be
considered ``primitive,'' and axiom systems have been devised that treat
all of them as such.  For example,
\cite[p.~35]{Goodstein}\index{Goodstein, R. L.} presents one with 15
axioms, some of which in fact coincide with what we have chosen to call
definitions in \texttt{set.mm}.  In certain subsets of classical
propositional calculus, such as the intuitionist
fragment\index{intuitionism}, it can be shown that one cannot make do
with just $\rightarrow$ and $\neg$ but must treat additional connectives
as primitive in order for the system to make sense.\footnote{Two nice
systems that make the transition from intuitionistic and other weak
fragments to classical logic just by adding axioms are given in
\cite{Robinsont}\index{Robinson, T. Thacher}.}

\subsection{The Approach to Definitions in \texttt{set.mm}}

In set theory, recursive definitions define a newly introduced symbol in
terms of itself.  The justification of recursive definitions, using
several ``recursion theorems,'' is usually one of the first
sophisticated proofs a student encounters when learning set theory, and
there is a significant amount of implicit metalogic behind a recursive
definition even though the definition itself is typically simple to
state.  It is, however, possible to substitute one kind of complexity
for another.  We can eliminate the need for metalogical justification by
defining the operation directly with an explicit (but complicated)
expression, then deriving the recursive definition directly as a
theorem, using a recursion theorem ``in reverse.''  We do this in
\texttt{set.mm}, as follows.

In \texttt{set.mm} our goal was to introduce almost all definitions in
the form of two expressions connected by either $\leftrightarrow$ or
$=$, where the thing being defined does not appear on the right hand
side.  Quine calls this form ``a genuine or direct definition'' \cite[p.
174]{Quine}\index{Quine, Willard Van Orman}, which makes the definitions
very easy to eliminate and the metalogic\index{metalogic} needed to
justify them as simple as possible.

We achieved this goal in almost all cases in \texttt{set.mm}.
Sometimes this makes the definitions more complex and less
intuitive.  For example, the traditional way to define addition of
natural numbers is to define an operation called {\em
successor}\index{successor} (which means ``plus one'' and is denoted by
``${\rm suc}$''), then define addition recursively\index{recursive
definition} with the two definitions $n + 0 = n$ and $m + {\rm suc}\,n =
{\rm suc} (m + n)$.  Although this definition seems simple and obvious,
the method to eliminate the definition is not obvious:  in the second
part of the definition, addition is defined in terms of itself.  By
eliminating the definition, we don't mean repeatedly applying it to
specific $m$ and $n$ but rather showing the explicit, closed-form
set-theoretical expression that $m + n$ represents, that will work for
any $m$ and $n$ and that does not have a $+$ sign on its right-hand
side.  For a recursive definition like this not to be circular
(creative), there are some hidden, underlying assumptions we must make,
for example that the natural numbers have a certain kind of order.  In
\texttt{set.mm} we chose to start with the direct (though complex and
nonintuitive) definition then derive from it the standard recursive
definition.\footnote{The closed-form definition used in \texttt{set.mm}
for the addition operation on ordinals\index{ordinal
addition}\index{addition!of ordinals} (of which natural numbers are a
subset) is

\setbox\startprefix=\hbox{\tt \ \ df-oadd\ \$a\ }
\setbox\contprefix=\hbox{\tt \ \ \ \ \ \ \ \ \ \ \ \ \ }
\startm
\m{\vdash}\m{+_o}\m{=}\m{\{}\m{\langle}\m{\langle}\m{x}\m{,}\m{y}\m{\rangle}%
\m{,}\m{z}\m{\rangle}\m{|}\m{(}\m{(}\m{x}\m{\in}\m{{\rm On}}\m{\wedge}\m{y}\m{%
\in}\m{{\rm On}}\m{)}\m{\wedge}\m{z}\m{=}\m{(}\m{{\rm rec}}\m{(}\m{\{}\m{%
\langle}\m{w}\m{,}\m{v}\m{\rangle}\m{|}\m{v}\m{=}\m{{\rm suc}}\m{w}\m{\}}\m{,}%
\m{x}\m{)}\m{`}\m{y}\m{)}\m{)}\m{\}}
\endm

\noindent Here, the abstraction class of nested ordered pairs\index{abstraction
class!of nested ordered pairs} is defined by \texttt{df-oprab} in \texttt{set.mm},
and ${\rm rec}$ is a ``recursion operator''\index{recursion operator} with
the definition

\setbox\startprefix=\hbox{\tt \ \ df-rfg\ \$a\ }
\setbox\contprefix=\hbox{\tt \ \ \ \ \ \ \ \ \ \ \ \ }
\startm
\m{\vdash}\m{{\rm rec}}\m{(}\m{F}\m{,}\m{A}\m{)}\m{=}\m{\bigcup}\m{\{}\m{f}%
\m{|}\m{\exists}\m{x}\m{(}\m{x}\m{\in}\m{{\rm On}}\m{\wedge}\m{(}\m{f}\m{{%
\rm Fn}}\m{x}\m{\wedge}\m{\forall}\m{y}\m{(}\m{y}\m{\in}\m{x}\m{\rightarrow}%
\m{(}\m{f}\m{`}\m{y}\m{)}\m{=}\m{(}\m{\{}\m{\langle}\m{g}\m{,}\m{z}\m{\rangle}%
\m{|}\m{(}\m{(}\m{g}\m{=}\m{\varnothing}\m{\wedge}\m{z}\m{=}\m{A}\m{)}\m{\vee}%
\m{(}\m{\lnot}\m{(}\m{g}\m{=}\m{\varnothing}\m{\vee}\m{{\rm Lim}}\m{{\rm dom}}%
\m{g}\m{)}\m{\wedge}\m{z}\m{=}\m{(}\m{F}\m{`}\m{(}\m{g}\m{`}\m{\bigcup}\m{{%
\rm dom}}\m{g}\m{)}\m{)}\m{)}\m{\vee}\m{(}\m{{\rm Lim}}\m{{\rm dom}}\m{g}\m{%
\wedge}\m{z}\m{=}\m{\bigcup}\m{{\rm ran}}\m{g}\m{)}\m{)}\m{\}}\m{`}\m{(}\m{f}%
\m{\restriction}\m{y}\m{)}\m{)}\m{)}\m{)}\m{)}\m{\}}
\endm

\noindent which can be further broken down with definitions shown in
Section~\ref{setdefinitions}.  You may be surprised at the complexity of what
seems like such a simple notion.  From these definitions the simpler, more
intuitive recursive definition is derived as a set of theorems.}\index{natural
number}\index{addition}\index{recursive definition}\index{ordinal addition}
The end result is the same, but we completely eliminate the rather complex
metalogic that justifies the recursive definition.  (For a mathematician,
recursive definitions are more efficient and intuitive than
direct ones once the metalogic has been learned or possibly just accepted as
correct.  However, it was felt that direct definition in \texttt{set.mm}
maximizes rigor by minimizing metalogic.  It can be eliminated effortlessly,
something difficult to do with a recursive definition.)

\subsection{Adding Constraints on Definitions}

The basic Metamath language and the Metamath program do
not have any built-in constraints on definitions, since they are just
\$a statements.

However, nothing prevents a verification system from verifying
additional rules to impose further limitations on definitions.
For example, the \texttt{mmj2}\index{mmj2} program
supports various kinds of
additional information comments (see section \ref{jcomment}).
One of their uses is to optionally verify additional constraints,
including constraints to verify that definitions meet certain
requirements.
These additional checks are required by the
continuous integration (CI)\index{continuous integration (CI)}
checks of the
\texttt{set.mm}\index{set theory database (\texttt{set.mm})}%
\index{Metamath Proof Explorer}
database.
This approach enables us to optionally impose additional requirements
on definitions if we wish, without necessarily imposing those rules on
all databases or requiring all verification systems to implement them.
In addition, this allows us to impose specialized constraints tailored
to one database while not requiring other systems to implement
those specialized constraints.

We impose two constraints on the
\texttt{set.mm}\index{set theory database (\texttt{set.mm})}%
\index{Metamath Proof Explorer} database
via the \texttt{mmj2}\index{mmj2} program that are worth discussing here:
a parse check and a definition soundness check.

% On February 11, 2019 8:32:32 PM EST, saueran@oregonstate.edu wrote:
% The following addition to the end of set.mm is accepted by the mmj2
% parser and definition checker and the metamath verifier(at least it was
% when I checked, you should check it too), and creates a contradiction by
% proving the theorem |- ph.
% ${
% wleftp $a wff ( ( ph ) $.
% wbothp $a wff ( ph ) $.
% df-leftp $a |- ( ( ( ph ) <-> -. ph ) $.
% df-bothp $a |- ( ( ph ) <-> ph ) $.
% anything $p |- ph $=
%   ( wbothp wn wi wleftp df-leftp biimpi df-bothp mpbir mpbi simplim ax-mp)
%   ABZAMACZDZCZMOEZOCQAEZNDZRNAFGSHIOFJMNKLAHJ $.
% $}
%
% This particular problem is countered by enabling, within mmj2,
% SetParser,mmj.verify.LRParser

First,
we enable a parse check in \texttt{mmj2} (through its
\texttt{SetParser} command) that requires that all new definitions
must \textit{not} create an ambiguous parse for a KLR(5) parser.
This prevents some errors such as definitions with imbalanced parentheses.

Second, we run a definition soundness check specific to
\texttt{set.mm} or databases similar to it.
(through the \texttt{definitionCheck} macro).
Some \texttt{\$a} statements (including all ax-* statemnets)
are excluded from these checks, as they will
always fail this simple check,
but they are appropriate for most definitions.
This check imposes a set of additional rules:

\begin{enumerate}

\item New definitions must be introduced using $=$ or $\leftrightarrow$.

\item No \texttt{\$a} statement introduced before this one is allowed to use the
symbol being defined in this definition, and the definition is not
allowed to use itself (except once, in the definiendum).

\item Every variable in the definiens should not be distinct

\item Every dummy variable in the definiendum
are required to be distinct from each other and from variables in
the definiendum.
To determine this, the system will look for a "justification" theorem
in the database, and if it is not there, attempt to briefly prove
$( \varphi \rightarrow \forall x \varphi )$  for each dummy variable x.

\item Every dummy variable should be a set variable,
unless there is a justification theorem available.

\item Every dummy variable must be bound
(if the system cannot determine this a justification theorem must be
provided).

\end{enumerate}

\subsection{Summary of Approach to Definitions}

In short, when being rigorous it turns out that
definitions can be subtle, sometimes requiring difficult
metatheorems to establish that they are not creative.

Instead of building such complications into the Metamath language itself,
the basic Metmath language and program simply treat traditional
axioms and definitions as the same kind of \texttt{\$a} statement.
We have then built various tools to enable people to
verify additional conditions as their creators believe is appropriate
for those specific databases, without complicating the Metamath foundations.

\chapter{The Metamath Program}\label{commands}

This chapter provides a reference manual for the
Metamath program.\index{Metamath!commands}

Current instructions for obtaining and installing the Metamath program
can be found at the \url{http://metamath.org} web site.
For Windows, there is a pre-compiled version called
\texttt{metamath.exe}.  For Unix, Linux, and Mac OS X
(which we will refer to collectively as ``Unix''), the Metamath program
can be compiled from its source code with the command
\begin{verbatim}
gcc *.c -o metamath
\end{verbatim}
using the \texttt{gcc} {\sc c} compiler available on those systems.

In the command syntax descriptions below, fields enclosed in square
brackets [\ ] are optional.  File names may be optionally enclosed in
single or double quotes.  This is useful if the file name contains
spaces or
slashes (\texttt{/}), such as in Unix path names, \index{Unix file
names}\index{file names!Unix} that might be confused with Metamath
command qualifiers.\index{Unix file names}\index{file names!Unix}

\section{Invoking Metamath}

Unix, Linux, and Mac OS X
have a command-line interface called the {\em
bash shell}.  (In Mac OS X, select the Terminal application from
Applications/Utilities.)  To invoke Metamath from the bash shell prompt,
assuming that the Metamath program is in the current directory, type
\begin{verbatim}
bash$ ./metamath
\end{verbatim}

To invoke Metamath from a Windows DOS or Command Prompt, assuming that
the Metamath program is in the current directory (or in a directory
included in the Path system environment variable), type
\begin{verbatim}
C:\metamath>metamath
\end{verbatim}

To use command-line arguments at invocation, the command-line arguments
should be a list of Metamath commands, surrounded by quotes if they
contain spaces.  In Windows, the surrounding quotes must be double (not
single) quotes.  For example, to read the database file \texttt{set.mm},
verify all proofs, and exit the program, type (under Unix)
\begin{verbatim}
bash$ ./metamath 'read set.mm' 'verify proof *' exit
\end{verbatim}
Note that in Unix, any directory path with \texttt{/}'s must be
surrounded by quotes so Metamath will not interpret the \texttt{/} as a
command qualifier.  So if \texttt{set.mm} is in the \texttt{/tmp}
directory, use for the above example
\begin{verbatim}
bash$ ./metamath 'read "/tmp/set.mm"' 'verify proof *' exit
\end{verbatim}

For convenience, if the command-line has one argument and no spaces in
the argument, the command is implicitly assumed to be \texttt{read}.  In
this one special case, \texttt{/}'s are not interpreted as command
qualifiers, so you don't need quotes around a Unix file name.  Thus
\begin{verbatim}
bash$ ./metamath /tmp/set.mm
\end{verbatim}
and
\begin{verbatim}
bash$ ./metamath "read '/tmp/set.mm'"
\end{verbatim}
are equivalent.


\section{Controlling Metamath}

The Metamath program was first developed on a {\sc vax/vms} system, and
some aspects of its command line behavior reflect this heritage.
Hopefully you will find it reasonably user-friendly once you get used to
it.

Each command line is a sequence of English-like words separated by
spaces, as in \texttt{show settings}.  Command words are not case
sensitive, and only as many letters are needed as are necessary to
eliminate ambiguity; for example, \texttt{sh se} would work for the
command \texttt{show settings}.  In some cases arguments such as file
names, statement labels, or symbol names are required; these are
case-sensitive (although file names may not be on some operating
systems).

A command line is entered by typing it in then pressing the {\em return}
({\em enter}) key.  To find out what commands are available, type
\texttt{?} at the \texttt{MM>} prompt.  To find out the choices at any
point in a command, press {\em return} and you will be prompted for
them.  The default choice (the one selected if you just press {\em
return}) is shown in brackets (\texttt{<>}).

You may also type \texttt{?} in place of a command word to force
Metamath to tell you what the choices are.  The \texttt{?} method won't
work, though, if a non-keyword argument such as a file name is expected
at that point, because the program will think that \texttt{?} is the
value of the argument.

Some commands have one or more optional qualifiers which modify the
behavior of the command.  Qualifiers are preceded by a slash
(\texttt{/}), such as in \texttt{read set.mm / verify}.  Spaces are
optional around the \texttt{/}.  If you need to use a space or
slash in a command
argument, as in a Unix file name, put single or double quotes around the
command argument.

The \texttt{open log} command will save everything you see on the
screen and is useful to help you recover should something go wrong in a
proof, or if you want to document a bug.

If a command responds with more than a screenful, you will be
prompted to \texttt{<return> to continue, Q to quit, or S to scroll to
end}.  \texttt{Q} or \texttt{q} (not case-sensitive) will complete the
command internally but will suppress further output until the next
\texttt{MM>} prompt.  \texttt{s} will suppress further pausing until the
next \texttt{MM>} prompt.  After the first screen, you are also
presented with the choice of \texttt{b} to go back a screenful.  Note
that \texttt{b} may also be entered at the \texttt{MM>} prompt
immediately after a command to scroll back through the output of that
command.

A command line enclosed in quotes is executed by your operating system.
See Section~\ref{oscmd}.

{\em Warning:} Pressing {\sc ctrl-c} will abort the Metamath program
unconditionally.  This means any unsaved work will be lost.


\subsection{\texttt{exit} Command}\index{\texttt{exit} command}

Syntax:  \texttt{exit} [\texttt{/force}]

This command exits from Metamath.  If there have been changes to the
source with the \texttt{save proof} or \texttt{save new{\char`\_}proof}
commands, you will be given an opportunity to \texttt{write source} to
permanently save the changes.

In Proof Assistant\index{Proof Assistant} mode, the \texttt{exit} command will
return to the \verb/MM>/ prompt. If there were changes to the proof, you will
be given an opportunity to \texttt{save new{\char`\_}proof}.

The \texttt{quit} command is a synonym for \texttt{exit}.

Optional qualifier:
    \texttt{/force} - Do not prompt if changes were not saved.  This qualifier is
        useful in \texttt{submit} command files (Section~\ref{sbmt})
        to ensure predictable behavior.





\subsection{\texttt{open log} Command}\index{\texttt{open log} command}
Syntax:  \texttt{open log} {\em file-name}

This command will open a log file that will store everything you see on
the screen.  It is useful to help recovery from a mistake in a long Proof
Assistant session, or to document bugs.\index{Metamath!bugs}

The log file can be closed with \texttt{close log}.  It will automatically be
closed upon exiting Metamath.



\subsection{\texttt{close log} Command}\index{\texttt{close log} command}
Syntax:  \texttt{close log}

The \texttt{close log} command closes a log file if one is open.  See
also \texttt{open log}.




\subsection{\texttt{submit} Command}\index{\texttt{submit} command}\label{sbmt}
Syntax:  \texttt{submit} {\em filename}

This command causes further command lines to be taken from the specified
file.  Note that any line beginning with an exclamation point (\texttt{!}) is
treated as a comment (i.e.\ ignored).  Also note that the scrolling
of the screen output is continuous, so you may want to open a log file
(see \texttt{open log}) to record the results that fly by on the screen.
After the lines in the file are exhausted, Metamath returns to its
normal user interface mode.

Currently, the \texttt{submit} command is not recursive.  In other words,
\texttt{submit} commands are not allowed inside a command file.


\subsection{\texttt{erase} Command}\index{\texttt{erase} command}
Syntax:  \texttt{erase}

This command will reset Metamath to its starting state, deleting any
data\-base that was \texttt{read} in.
 If there have been changes to the
source with the \texttt{save proof} or \texttt{save new{\char`\_}proof}
commands, you will be given an opportunity to \texttt{write source} to
permanently save the changes.



\subsection{\texttt{set echo} Command}\index{\texttt{set echo} command}
Syntax:  \texttt{set echo on} or \texttt{set echo off}

The \texttt{set echo on} command will cause command lines to be echoed with any
abbreviations expanded.  While learning the Metamath commands, this
feature will show you the exact command that your abbreviated input
corresponds to.



\subsection{\texttt{set scroll} Command}\index{\texttt{set scroll} command}
Syntax:  \texttt{set scroll prompted} or \texttt{set scroll continuous}

The Metamath command line interface starts off in the \texttt{prompted} mode,
which means that you will be prompted to continue or quit after each
full screen in a long listing.  In \texttt{continuous} mode, long listings will be
scrolled without pausing.

% LaTeX bug? (1) \texttt{\_} puts out different character than
% \texttt{{\char`\_}}
%  = \verb$_$  (2) \texttt{{\char`\_}} puts out garbage in \subsection
%  argument
\subsection{\texttt{set width} Command}\index{\texttt{set
width} command}
Syntax:  \texttt{set width} {\em number}

Metamath assumes the width of your screen is 79 characters (chosen
because the Command Prompt in Windows XP has a wrapping bug at column
80).  If your screen is wider or narrower, this command allows you to
change this default screen width.  A larger width is advantageous for
logging proofs to an output file to be printed on a wide printer.  A
smaller width may be necessary on some terminals; in this case, the
wrapping of the information messages may sometimes seem somewhat
unnatural, however.  In \LaTeX\index{latex@{\LaTeX}!characters per line},
there is normally a maximum of 61 characters per line with typewriter
font.  (The examples in this book were produced with 61 characters per
line.)

\subsection{\texttt{set height} Command}\index{\texttt{set
height} command}
Syntax:  \texttt{set height} {\em number}

Metamath assumes your screen height is 24 lines of characters.  If your
screen is taller or shorter, this command lets you to change the number
of lines at which the display pauses and prompts you to continue.

\subsection{\texttt{beep} Command}\index{\texttt{beep} command}

Syntax:  \texttt{beep}

This command will produce a beep.  By typing it ahead after a
long-running command has started, it will alert you that the command is
finished.  For convenience, \texttt{b} is an abbreviation for
\texttt{beep}.

Note:  If \texttt{b} is typed at the \texttt{MM>} prompt immediately
after the end of a multiple-page display paged with ``\texttt{Press
<return> for more}...'' prompts, then the \texttt{b} will back up to the
previous page rather than perform the \texttt{beep} command.
In that case you must type the unabbreviated \texttt{beep} form
of the command.

\subsection{\texttt{more} Command}\index{\texttt{more} command}

Syntax:  \texttt{more} {\em filename}

This command will display the contents of an {\sc ascii} file on your
screen.  (This command is provided for convenience but is not very
powerful.  See Section~\ref{oscmd} to invoke your operating system's
command to do this, such as the \texttt{more} command in Unix.)

\subsection{Operating System Commands}\index{operating system
command}\label{oscmd}

A line enclosed in single or double quotes will be executed by your
computer's operating system if it has a command line interface.  For
example, on a {\sc vax/vms} system,
\verb/MM> 'dir'/
will print disk directory contents.  Note that this feature will not
work on the Macintosh prior to Mac OS X, which does not have a command
line interface.

For your convenience, the trailing quote is optional.

\subsection{Size Limitations in Metamath}

In general, there are no fixed, predefined limits\index{Metamath!memory
limits} on how many labels, tokens\index{token}, statements, etc.\ that
you may have in a database file.  The Metamath program uses 32-bit
variables (64-bit on 64-bit CPUs) as indices for almost all internal
arrays, which are allocated dynamically as needed.



\section{Reading and Writing Files}

The following commands create new files:  the \texttt{open} commands;
the \texttt{write} commands; the \texttt{/html},
\texttt{/alt{\char`\_}html}, \texttt{/brief{\char`\_}html},
\texttt{/brief{\char`\_}alt{\char`\_}html} qualifiers of \texttt{show
statement}, and \texttt{midi}.  The following commands append to files
previously opened:  the \texttt{/tex} qualifier of \texttt{show proof}
and \texttt{show new{\char`\_}proof}; the \texttt{/tex} and
\texttt{/simple{\char`\_}tex} qualifiers of \texttt{show statement}; the
\texttt{close} commands; and all screen dialog between \texttt{open log}
and \texttt{close log}.

The commands that create new files will not overwrite an existing {\em
filename} but will rename the existing one to {\em
filename}\texttt{{\char`\~}1}.  An existing {\em
filename}\texttt{{\char`\~}1} is renamed {\em
filename}\texttt{{\char`\~}2}, etc.\ up to {\em
filename}\texttt{{\char`\~}9}.  An existing {\em
filename}\texttt{{\char`\~}9} is deleted.  This makes recovery from
mistakes easier but also will clutter up your directory, so occasionally
you may want to clean up (delete) these \texttt{{\char`\~}}$n$ files.


\subsection{\texttt{read} Command}\index{\texttt{read} command}
Syntax:  \texttt{read} {\em file-name} [\texttt{/verify}]

This command will read in a Metamath language source file and any included
files.  Normally it will be the first thing you do when entering Metamath.
Statement syntax is checked, but proof syntax is not checked.
Note that the file name may be enclosed in single or double quotes;
this is useful if the file name contains slashes, as might be the case
under Unix.

If you are getting an ``\texttt{?Expected VERIFY}'' error
when trying to read a Unix file name with slashes, you probably haven't
quoted it.\index{Unix file names}\index{file names!Unix}

If you are prompted for the file name (by pressing {\em return}
 after \texttt{read})
you should {\em not} put quotes around it, even if it is a Unix file name
with slashes.

Optional command qualifier:

    \texttt{/verify} - Verify all proofs as the database is read in.  This
         qualifier will slow down reading in the file.  See \texttt{verify
         proof} for more information on file error-checking.

See also \texttt{erase}.



\subsection{\texttt{write source} Command}\index{\texttt{write source} command}
Syntax:  \texttt{write source} {\em filename}
[\texttt{/rewrap}]
[\texttt{/split}]
[\texttt{/keep\_includes}]
[\texttt{/no\_versioning}]

This command will write the contents of a Metamath\index{database}
database into a file.\index{source file}

Optional command qualifiers:

\texttt{/rewrap} -
Reformats statements and comments according to the
convention used in the set.mm database.
It unwraps the
lines in the comment before each \texttt{\$a} and \texttt{\$p} statement,
then it
rewraps the line.  You should compare the output to the original
to make sure that the desired effect results; if not, go back to
the original source.  The wrapped line length honors the
\texttt{set width}
parameter currently in effect.  Note:  Text
enclosed in \texttt{<HTML>}...\texttt{</HTML>} tags is not modified by the
\texttt{/rewrap} qualifier.
Proofs themselves are not reformatted;
use \texttt{save proof * / compressed} to do that.
An isolated tilde (\~{}) is always kept on the same line as the following
symbol, so you can find all comment references to a symbol by
searching for \~{} followed by a space and that symbol
(this is useful for finding cross-references).
Incidentally, \texttt{save proof} also honors the \texttt{set width}
parameter currently in effect.

\texttt{/split} - Files included in the source using the expression
\$[ \textit{inclfile} \$] will be
written into separate files instead of being included in a single output
file.  The name of each separately written included file will be the
\textit{inclfile} argument of its inclusion command.

\texttt{/keep\_includes} - If a source file has includes but is written as a
single file by omitting \texttt{/split}, by default the included files will
be deleted (actually just renamed with a \char`\~1 suffix unless
\texttt{/no\_versioning} is specified) to prevent the possibly confusing
source duplication in both the output file and the included file.
The \texttt{/keep\_includes} qualifier will prevent this deletion.

\texttt{/no\_versioning} - Backup files suffixed with \char`\~1 are not created.


\section{Showing Status and Statements}



\subsection{\texttt{show settings} Command}\index{\texttt{show settings} command}
Syntax:  \texttt{show settings}

This command shows the state of various parameters.

\subsection{\texttt{show memory} Command}\index{\texttt{show memory} command}
Syntax:  \texttt{show memory}

This command shows the available memory left.  It is not meaningful
on most modern operating systems,
which have virtual memory.\index{Metamath!memory usage}


\subsection{\texttt{show labels} Command}\index{\texttt{show labels} command}
Syntax:  \texttt{show labels} {\em label-match} [\texttt{/all}]
   [\texttt{/linear}]

This command shows the labels of \texttt{\$a} and \texttt{\$p}
statements that match {\em label-match}.  A \verb$*$ in {label-match}
matches zero or more characters.  For example, \verb$*abc*def$ will match all
labels containing \verb$abc$ and ending with \verb$def$.

Optional command qualifiers:

   \texttt{/all} - Include matches for \texttt{\$e} and \texttt{\$f}
   statement labels.

   \texttt{/linear} - Display only one label per line.  This can be useful for
       building scripts in conjunction with the utilities under the
       \texttt{tools}\index{\texttt{tools} command} command.



\subsection{\texttt{show statement} Command}\index{\texttt{show statement} command}
Syntax:  \texttt{show statement} {\em label-match} [{\em qualifiers} (see below)]

This command provides information about a statement.  Only statements
that have labels (\texttt{\$f}\index{\texttt{\$f} statement},
\texttt{\$e}\index{\texttt{\$e} statement},
\texttt{\$a}\index{\texttt{\$a} statement}, and
\texttt{\$p}\index{\texttt{\$p} statement}) may be specified.
If {\em label-match}
contains wildcard (\verb$*$) characters, all matching statements will be
displayed in the order they occur in the database.

Optional qualifiers (only one qualifier at a time is allowed):

    \texttt{/comment} - This qualifier includes the comment that immediately
       precedes the statement.

    \texttt{/full} - Show complete information about each statement,
       and show all
       statements matching {\em label} (including \texttt{\$e}
       and \texttt{\$f} statements).

    \texttt{/tex} - This qualifier will write the statement information to the
       \LaTeX\ file previously opened with \texttt{open tex}.  See
       Section~\ref{texout}.

    \texttt{/simple{\char`\_}tex} - The same as \texttt{/tex}, except that
       \LaTeX\ macros are not used for formatting equations, allowing easier
       manual edits of the output for slide presentations, etc.

    \texttt{/html}\index{html generation@{\sc html} generation},
       \texttt{/alt{\char`\_}html}, \texttt{/brief{\char`\_}html},
       \texttt{/brief{\char`\_}alt{\char`\_}html} -
       These qualifiers invoke a special mode of
       \texttt{show statement} that
       creates a web page for the statement.  They may not be used with
       any other qualifier.  See Section~\ref{htmlout} or
       \texttt{help html} in the program.


\subsection{\texttt{search} Command}\index{\texttt{search} command}
Syntax:  search {\em label-match}
\texttt{"}{\em symbol-match}{\tt}" [\texttt{/all}] [\texttt{/comments}]
[\texttt{/join}]

This command searches all \texttt{\$a} and \texttt{\$p} statements
matching {\em label-match} for occurrences of {\em symbol-match}.  A
\verb@*@ in {\em label-match} matches any label character.  A \verb@$*@
in {\em symbol-match} matches any sequence of symbols.  The symbols in
{\em symbol-match} must be separated by white space.  The quotes
surrounding {\em symbol-match} may be single or double quotes.  For
example, \texttt{search b}\verb@* "-> $* ch"@ will list all statements
whose labels begin with \texttt{b} and contain the symbols \verb@->@ and
\texttt{ch} surrounding any symbol sequence (including no symbol
sequence).  The wildcards \texttt{?} and \texttt{\$?} are also available
to match individual characters in labels and symbols respectively; see
\texttt{help search} in the Metamath program for details on their usage.

Optional command qualifiers:

    \texttt{/all} - Also search \texttt{\$e} and \texttt{\$f} statements.

    \texttt{/comments} - Search the comment that immediately precedes each
        label-matched statement for {\em symbol-match}.  In this case
        {\em symbol-match} is an arbitrary, non-case-sensitive character
        string.  Quotes around {\em symbol-match} are optional if there
        is no ambiguity.

    \texttt{/join} - In the case of a \texttt{\$a} or \texttt{\$p} statement,
	prepend its \texttt{\$e}
	hypotheses for searching. The
	\texttt{/join} qualifier has no effect in \texttt{/comments} mode.

\section{Displaying and Verifying Proofs}


\subsection{\texttt{show proof} Command}\index{\texttt{show proof} command}
Syntax:  \texttt{show proof} {\em label-match} [{\em qualifiers} (see below)]

This command displays the proof of the specified
\texttt{\$p}\index{\texttt{\$p} statement} statement in various formats.
The {\em label-match} may contain wildcard (\verb@$*@) characters to match
multiple statements.  Without any qualifiers, only the logical steps
will be shown (i.e.\ syntax construction steps will be omitted), in an
indented format.

Most of the time, you will use
    \texttt{show proof} {\em label}
to see just the proof steps corresponding to logical inferences.

Optional command qualifiers:

    \texttt{/essential} - The proof tree is trimmed of all
        \texttt{\$f}\index{\texttt{\$f} statement} hypotheses before
        being displayed.  (This is the default, and it is redundant to
        specify it.)

    \texttt{/all} - the proof tree is not trimmed of all \texttt{\$f} hypotheses before
        being displayed.  \texttt{/essential} and \texttt{/all} are mutually exclusive.

    \texttt{/from{\char`\_}step} {\em step} - The display starts at the specified
        step.  If
        this qualifier is omitted, the display starts at the first step.

    \texttt{/to{\char`\_}step} {\em step} - The display ends at the specified
        step.  If this
        qualifier is omitted, the display ends at the last step.

    \texttt{/tree{\char`\_}depth} {\em number} - Only
         steps at less than the specified proof
        tree depth are displayed.  Sometimes useful for obtaining an overview of
        the proof.

    \texttt{/reverse} - The steps are displayed in reverse order.

    \texttt{/renumber} - When used with \texttt{/essential}, the steps are renumbered
        to correspond only to the essential steps.

    \texttt{/tex} - The proof is converted to \LaTeX\ and\index{latex@{\LaTeX}}
        stored in the file opened
        with \texttt{open tex}.  See Section~\ref{texout} or
        \texttt{help tex} in the program.

    \texttt{/lemmon} - The proof is displayed in a non-indented format known
        as Lemmon style, with explicit previous step number references.
        If this qualifier is omitted, steps are indented in a tree format.

    \texttt{/start{\char`\_}column} {\em number} - Overrides the default column
        (16)
        at which the formula display starts in a Lemmon-style display.  May be
        used only in conjunction with \texttt{/lemmon}.

    \texttt{/normal} - The proof is displayed in normal format suitable for
        inclusion in a Metamath source file.  May not be used with any other
        qualifier.

    \texttt{/compressed} - The proof is displayed in compressed format
        suitable for inclusion in a Metamath source file.  May not be used with
        any other qualifier.

    \texttt{/statement{\char`\_}summary} - Summarizes all statements (like a
        brief \texttt{show statement})
        used by the proof.  It may not be used with any other qualifier
        except \texttt{/essential}.

    \texttt{/detailed{\char`\_}step} {\em step} - Shows the details of what is
        happening at
        a specific proof step.  May not be used with any other qualifier.
        The {\em step} is the step number shown when displaying a
        proof without the \texttt{/renumber} qualifier.


\subsection{\texttt{show usage} Command}\index{\texttt{show usage} command}
Syntax:  \texttt{show usage} {\em label-match} [\texttt{/recursive}]

This command lists the statements whose proofs make direct reference to
the statement specified.

Optional command qualifier:

    \texttt{/recursive} - Also include statements whose proofs ultimately
        depend on the statement specified.



\subsection{\texttt{show trace\_back} Command}\index{\texttt{show
       trace{\char`\_}back} command}
Syntax:  \texttt{show trace{\char`\_}back} {\em label-match} [\texttt{/essential}] [\texttt{/axioms}]
    [\texttt{/tree}] [\texttt{/depth} {\em number}]

This command lists all statements that the proof of the \texttt{\$p}
statement(s) specified by {\em label-match} depends on.

Optional command qualifiers:

    \texttt{/essential} - Restrict the trace-back to \texttt{\$e}
        \index{\texttt{\$e} statement} hypotheses of proof trees.

    \texttt{/axioms} - List only the axioms that the proof ultimately depends on.

    \texttt{/tree} - Display the trace-back in an indented tree format.

    \texttt{/depth} {\em number} - Restrict the \texttt{/tree} trace-back to the
        specified indentation depth.

    \texttt{/count{\char`\_}steps} - Count the number of steps the proof has
       all the way back to axioms.  If \texttt{/essential} is specified,
       expansions of variable-type hypotheses (syntax constructions) are not counted.

\subsection{\texttt{verify proof} Command}\index{\texttt{verify proof} command}
Syntax:  \texttt{verify proof} {\em label-match} [\texttt{/syntax{\char`\_}only}]

This command verifies the proofs of the specified statements.  {\em
label-match} may contain wild card characters (\texttt{*}) to verify
more than one proof; for example \verb/*abc*def/ will match all labels
containing \texttt{abc} and ending with \texttt{def}.
The command \texttt{verify proof *} will verify all proofs in the database.

Optional command qualifier:

    \texttt{/syntax{\char`\_}only} - This qualifier will perform a check of syntax
        and RPN
        stack violations only.  It will not verify that the proof is
        correct.  This qualifier is useful for quickly determining which
        proofs are incomplete (i.e.\ are under development and have \texttt{?}'s
        in them).

{\em Note:} \texttt{read}, followed by \texttt{verify proof *}, will ensure
 the database is free
from errors in the Metamath language but will not check the markup notation
in comments.
You can also check the markup notation by running \texttt{verify markup *}
as discussed in Section~\ref{verifymarkup}; see also the discussion
on generating {\sc HTML} in Section~\ref{htmlout}.

\subsection{\texttt{verify markup} Command}\index{\texttt{verify markup} command}\label{verifymarkup}
Syntax:  \texttt{verify markup} {\em label-match}
[\texttt{/date{\char`\_}skip}]
[\texttt{/top{\char`\_}date{\char`\_}skip}]
[\texttt{/file{\char`\_}skip}]
[\texttt{/verbose}]

This command checks comment markup and other informal conventions we have
adopted.  It error-checks the latexdef, htmldef, and althtmldef statements
in the \texttt{\$t} statement of a Metamath source file.\index{error checking}
It error-checks any \texttt{`}...\texttt{`},
\texttt{\char`\~}~\textit{label},
and bibliographic markups in statement descriptions.
It checks that
\texttt{\$p} and \texttt{\$a} statements
have the same content when their labels start with
``ax'' and ``ax-'' respectively but are otherwise identical, for example
ax4 and ax-4.
It also verifies the date consistency of ``(Contributed by...),''
``(Revised by...),'' and ``(Proof shortened by...)'' tags in the comment
above each \texttt{\$a} and \texttt{\$p} statement.

Optional command qualifiers:

    \texttt{/date{\char`\_}skip} - This qualifier will
        skip date consistency checking,
        which is usually not required for databases other than
	\texttt{set.mm}.

    \texttt{/top{\char`\_}date{\char`\_}skip} - This qualifier will check date consistency except
        that the version date at the top of the database file will not
        be checked.  Only one of
        \texttt{/date{\char`\_}skip} and
        \texttt{/top{\char`\_}date{\char`\_}skip} may be
        specified.

    \texttt{/file{\char`\_}skip} - This qualifier will skip checks that require
        external files to be present, such as checking GIF existence and
        bibliographic links to mmset.html or equivalent.  It is useful
        for doing a quick check from a directory without these files.

    \texttt{/verbose} - Provides more information.  Currently it provides a list
        of axXXX vs. ax-XXX matches.

\subsection{\texttt{save proof} Command}\index{\texttt{save proof} command}
Syntax:  \texttt{save proof} {\em label-match} [\texttt{/normal}]
   [\texttt{/compressed}]

The \texttt{save proof} command will reformat a proof in one of two formats and
replace the existing proof in the source buffer\index{source
buffer}.  It is useful for
converting between proof formats.  Note that a proof will not be
permanently saved until a \texttt{write source} command is issued.

Optional command qualifiers:

    \texttt{/normal} - The proof is saved in the normal format (i.e., as a
        sequence
        of labels, which is the defined format of the basic Metamath
        language).\index{basic language}  This is the default format that
        is used if a qualifier
        is omitted.

    \texttt{/compressed} - The proof is saved in the compressed format which
        reduces storage requirements for a database.
        See Appendix~\ref{compressed}.




\section{Creating Proofs}\label{pfcommands}\index{Proof Assistant}

Before using the Proof Assistant, you must add a \texttt{\$p} to your
source file (using a text editor) containing the statement you want to
prove.  Its proof should consist of a single \texttt{?}, meaning
``unknown step.''  Example:
\begin{verbatim}
equid $p x = x $= ? $.
\end{verbatim}

To enter the Proof assistant, type \texttt{prove} {\em label}, e.g.
\texttt{prove equid}.  Metamath will respond with the \texttt{MM-PA>}
prompt.

Proofs are created working backwards from the statement being proved,
primarily using a series of \texttt{assign} commands.  A proof is
complete when all steps are assigned to statements and all steps are
unified and completely known.  During the creation of a proof, Metamath
will allow only operations that are legal based on what is known up to
that point.  For example, it will not allow an \texttt{assign} of a
statement that cannot be unified with the unknown proof step being
assigned.

{\em Important:}
The Proof Assistant is
{\em not} a tool to help you discover proofs.  It is just a tool to help
you add them to the database.  For a tutorial read
Section~\ref{frstprf}.
To practice using the Proof Assistant, you may
want to \texttt{prove} an existing theorem, then delete all steps with
\texttt{delete all}, then re-create it with the Proof Assistant while
looking at its proof display (before deletion).
You might want to figure out your first few proofs completely
and write them down by hand, before using the Proof Assistant, though
not everyone finds that effective.

{\em Important:}
The \texttt{undo} command if very helpful when entering a proof, because
it allows you to undo a previously-entered step.
In addition, we suggest that you
keep track of your work with a log file (\texttt{open
log}) and save it frequently (\texttt{save new{\char`\_}proof},
\texttt{write source}).
You can use \texttt{delete} to reverse an \texttt{assign}.
You can also do \texttt{delete floating{\char`\_}hypotheses}, then
\texttt{initialize all}, then \texttt{unify all /interactive} to
reinitialize bad unifications made accidentally or by bad
\texttt{assign}s.  You cannot reverse a \texttt{delete} except by
a relevant \texttt{undo} or using
\texttt{exit /force} then reentering the Proof Assistant to recover from
the last \texttt{save new{\char`\_}proof}.

The following commands are available in the Proof Assistant (at the
\texttt{MM-PA>} prompt) to help you create your proof.  See the
individual commands for more detail.

\begin{itemize}
\item[]
    \texttt{show new{\char`\_}proof} [\texttt{/all},...] - Displays the
        proof in progress.  You will use this command a lot; see \texttt{help
        show new{\char`\_}proof} to become familiar with its qualifiers.  The
        qualifiers \texttt{/unknown} and \texttt{/not{\char`\_}unified} are
        useful for seeing the work remaining to be done.  The combination
        \texttt{/all/unknown} is useful for identifying dummy variables that must be
        assigned, or attempts to use illegal syntax, when \texttt{improve all}
        is unable to complete the syntax constructions.  Unknown variables are
        shown as \texttt{\$1}, \texttt{\$2},...
\item[]
    \texttt{assign} {\em step} {\em label} - Assigns an unknown {\em step}
        number with the statement
        specified by {\em label}.
\item[]
    \texttt{let variable} {\em variable}
        \texttt{= "}{\em symbol sequence}\texttt{"}
          - Forces a symbol
        sequence to replace an unknown variable (such as \texttt{\$1}) in a proof.
        It is useful
        for helping difficult unifications, and it is necessary when you have
        dummy variables that eventually must be assigned a name.
\item[]
    \texttt{let step} {\em step} \texttt{= "}{\em symbol sequence}\texttt{"} -
          Forces a symbol sequence
        to replace the contents of a proof step, provided it can be
        unified with the existing step contents.  (I rarely use this.)
\item[]
    \texttt{unify step} {\em step} (or \texttt{unify all}) - Unifies
        the source and target of
        a step.  If you specify a specific step, you will be prompted
        to select among the unifications that are possible.  If you
        specify \texttt{all}, all steps with unique unifications, but only
        those steps, will be
        unified.  \texttt{unify all /interactive} goes through all non-unified
        steps.
\item[]
    \texttt{initialize} {\em step} (or \texttt{all}) - De-unifies the target and source of
        a step (or all steps), as well as the hypotheses of the source,
        and makes all variables in the source unknown.  Useful to recover from
        an \texttt{assign} or \texttt{let} mistake that
        resulted in incorrect unifications.
\item[]
    \texttt{delete} {\em step} (or \texttt{all} or \texttt{floating{\char`\_}hypotheses}) -
        Deletes the specified
        step(s).  \texttt{delete floating{\char`\_}hypotheses}, then \texttt{initialize all}, then
        \texttt{unify all /interactive} is useful for recovering from mistakes
        where incorrect unifications assigned wrong math symbol strings to
        variables.
\item[]
    \texttt{improve} {\em step} (or \texttt{all}) -
      Automatically creates a proof for steps (with no unknown variables)
      whose proof requires no statements with \texttt{\$e} hypotheses.  Useful
      for filling in proofs of \texttt{\$f} hypotheses.  The \texttt{/depth}
      qualifier will also try statements whose \texttt{\$e} hypotheses contain
      no new variables.  {\em Warning:} Save your work (with \texttt{save
      new{\char`\_}proof} then \texttt{write source}) before using
      \texttt{/depth = 2} or greater, since the search time grows
      exponentially and may never terminate in a reasonable time, and you
      cannot interrupt the search.  I have found that it is rare for
      \texttt{/depth = 3} or greater to be useful.
 \item[]
    \texttt{save new{\char`\_}proof} - Saves the proof in progress in the program's
        internal database buffer.  To save it permanently into the database file,
        use \texttt{write source} after
        \texttt{save new{\char`\_}proof}.  To revert to the last
        \texttt{save new{\char`\_}proof},
        \texttt{exit /force} from the Proof Assistant then re-enter the Proof
        Assistant.
 \item[]
    \texttt{match step} {\em step} (or \texttt{match all}) - Shows what
        statements are
        possibilities for the \texttt{assign} statement. (This command
        is not very
        useful in its present form and hopefully will be improved
        eventually.  In the meantime, use the \texttt{search} statement for
        candidates matching specific math token combinations.)
 \item[]
 \texttt{minimize{\char`\_}with}\index{\texttt{minimize{\char`\_}with} command}
% 3/10/07 Note: line-breaking the above results in duplicate index entries
     - After a proof is complete, this command will attempt
        to match other database theorems to the proof to see if the proof
        size can be reduced as a result.  See \texttt{help
        minimize{\char`\_}with} in the
        Metamath program for its usage.
 \item[]
 \texttt{undo}\index{\texttt{undo} command}
    - Undo the effect of a proof-changing command (all but the
      \texttt{show} and \texttt{save} commands above).
 \item[]
 \texttt{redo}\index{\texttt{redo} command}
    - Reverse the previous \texttt{undo}.
\end{itemize}

The following commands set parameters that may be relevant to your proof.
Consult the individual \texttt{help set}... commands.
\begin{itemize}
   \item[] \texttt{set unification{\char`\_}timeout}
 \item[]
    \texttt{set search{\char`\_}limit}
  \item[]
    \texttt{set empty{\char`\_}substitution} - note that default is \texttt{off}
\end{itemize}

Type \texttt{exit} to exit the \texttt{MM-PA>}
 prompt and get back to the \texttt{MM>} prompt.
Another \texttt{exit} will then get you out of Metamath.



\subsection{\texttt{prove} Command}\index{\texttt{prove} command}
Syntax:  \texttt{prove} {\em label}

This command will enter the Proof Assistant\index{Proof Assistant}, which will
allow you to create or edit the proof of the specified statement.
The command-line prompt will change from \texttt{MM>} to \texttt{MM-PA>}.

Note:  In the present version of
Metamath\index{Metamath!limitations of version 0.07.30}, the Proof
Assistant does not verify that \texttt{\$d}\index{\texttt{\$d}
statement} restrictions are met as a proof is being built.  After you
have completed a proof, you should type \texttt{save new{\char`\_}proof}
followed by \texttt{verify proof} {\em label} (where {\em label} is the
statement you are proving with the \texttt{prove} command) to verify the
\texttt{\$d} restrictions.

See also: \texttt{exit}

\subsection{\texttt{set unification\_timeout} Command}\index{\texttt{set
unification{\char`\_}timeout} command}
Syntax:  \texttt{set unification{\char`\_}timeout} {\em number}

(This command is available outside the Proof Assistant but affects the
Proof Assistant\index{Proof Assistant} only.)

Sometimes the Proof Assistant will inform you that a unification
time-out occurred.  This may happen when you try to \texttt{unify}
formulas with many temporary variables\index{temporary variable}
(\texttt{\$1}, \texttt{\$2}, etc.), since the time to compute all possible
unifications may grow exponentially with the number of variables.  If
you want Metamath to try harder (and you're willing to wait longer) you
may increase this parameter.  \texttt{show settings} will show you the
current value.

\subsection{\texttt{set empty\_substitution} Command}\index{\texttt{set
empty{\char`\_}substitution} command}
% These long names can't break well in narrow mode, and even "sloppy"
% is not enough. Work around this by not demanding justification.
\begin{flushleft}
Syntax:  \texttt{set empty{\char`\_}substitution on} or \texttt{set
empty{\char`\_}substitution off}
\end{flushleft}

(This command is available outside the Proof Assistant but affects the
Proof Assistant\index{Proof Assistant} only.)

The Metamath language allows variables to be
substituted\index{substitution!variable}\index{variable substitution}
with empty symbol sequences\index{empty substitution}.  However, in many
formal systems\index{formal system} this will never happen in a valid
proof.  Allowing for this possibility increases the likelihood of
ambiguous unifications\index{ambiguous
unification}\index{unification!ambiguous} during proof creation.
The default is that
empty substitutions are not allowed; for formal systems requiring them,
you must \texttt{set empty{\char`\_}substitution on}.
(An example where this must be \texttt{on}
would be a system that implements a Deduction Rule and in
which deductions from empty assumption lists would be permissible.  The
MIU-system\index{MIU-system} described in Appendix~\ref{MIU} is another
example.)
Note that empty substitutions are
always permissible in proof verification (VERIFY PROOF...) outside the
Proof Assistant.  (See the MIU system in the Metamath book for an example
of a system needing empty substitutions; another example would be a
system that implements a Deduction Rule and in which deductions from
empty assumption lists would be permissible.)

It is better to leave this \texttt{off} when working with \texttt{set.mm}.
Note that this command does not affect the way proofs are verified with
the \texttt{verify proof} command.  Outside of the Proof Assistant,
substitution of empty sequences for math symbols is always allowed.

\subsection{\texttt{set search\_limit} Command}\index{\texttt{set
search{\char`\_}limit} command} Syntax:  \texttt{set search{\char`\_}limit} {\em
number}

(This command is available outside the Proof Assistant but affects the
Proof Assistant\index{Proof Assistant} only.)

This command sets a parameter that determines when the \texttt{improve} command
in Proof Assistant mode gives up.  If you want \texttt{improve} to search harder,
you may increase it.  The \texttt{show settings} command tells you its current
value.


\subsection{\texttt{show new\_proof} Command}\index{\texttt{show
new{\char`\_}proof} command}
Syntax:  \texttt{show new{\char`\_}proof} [{\em
qualifiers} (see below)]

This command (available only in Proof Assistant mode) displays the proof
in progress.  It is identical to the \texttt{show proof} command, except that
there is no statement argument (since it is the statement being proved) and
the following qualifiers are not available:

    \texttt{/statement{\char`\_}summary}

    \texttt{/detailed{\char`\_}step}

Also, the following additional qualifiers are available:

    \texttt{/unknown} - Shows only steps that have no statement assigned.

    \texttt{/not{\char`\_}unified} - Shows only steps that have not been unified.

Note that \texttt{/essential}, \texttt{/depth}, \texttt{/unknown}, and
\texttt{/not{\char`\_}unified} may be
used in any combination; each of them effectively filters out additional
steps from the proof display.

See also:  \texttt{show proof}






\subsection{\texttt{assign} Command}\index{\texttt{assign} command}
Syntax:   \texttt{assign} {\em step} {\em label} [\texttt{/no{\char`\_}unify}]

   and:   \texttt{assign first} {\em label}

   and:   \texttt{assign last} {\em label}


This command, available in the Proof Assistant only, assigns an unknown
step (one with \texttt{?} in the \texttt{show new{\char`\_}proof}
listing) with the statement specified by {\em label}.  The assignment
will not be allowed if the statement cannot be unified with the step.

If \texttt{last} is specified instead of {\em step} number, the last
step that is shown by \texttt{show new{\char`\_}proof /unknown} will be
used.  This can be useful for building a proof with a command file (see
\texttt{help submit}).  It also makes building proofs faster when you know
the assignment for the last step.

If \texttt{first} is specified instead of {\em step} number, the first
step that is shown by \texttt{show new{\char`\_}proof /unknown} will be
used.

If {\em step} is zero or negative, the -{\em step}th from last unknown
step, as shown by \texttt{show new{\char`\_}proof /unknown}, will be
used.  \texttt{assign -1} {\em label} will assign the penultimate
unknown step, \texttt{assign -2} {\em label} the antepenultimate, and
\texttt{assign 0} {\em label} is the same as \texttt{assign last} {\em
label}.

Optional command qualifier:

    \texttt{/no{\char`\_}unify} - do not prompt user to select a unification if there is
        more than one possibility.  This is useful for noninteractive
        command files.  Later, the user can \texttt{unify all /interactive}.
        (The assignment will still be automatically unified if there is only
        one possibility and will be refused if unification is not possible.)



\subsection{\texttt{match} Command}\index{\texttt{match} command}
Syntax:  \texttt{match step} {\em step} [\texttt{/max{\char`\_}essential{\char`\_}hyp}
{\em number}]

    and:  \texttt{match all} [\texttt{/essential}]
          [\texttt{/max{\char`\_}essential{\char`\_}hyp} {\em number}]

This command, available in the Proof Assistant only, shows what
statements can be unified with the specified step(s).  {\em Note:} In
its current form, this command is not very useful because of the large
number of matches it reports.  I rarely use it.
It may be enhanced in the future.

Optional command qualifiers:

    \texttt{/max{\char`\_}essential{\char`\_}hyp} {\em number} - filters out
        of the list any statements
        with more than the specified number of
        \texttt{\$e}\index{\texttt{\$e} statement} hypotheses

    \texttt{/essential{\char`\_}only} - in the \texttt{match all} statement, only
        the steps that
        would be listed in the \texttt{show new{\char`\_}proof /essential} display are
        matched.



\subsection{\texttt{let} Command}\index{\texttt{let} command}
Syntax: \texttt{let variable} {\em variable} = \verb/"/{\em symbol-sequence}\verb/"/

  and: \texttt{let step} {\em step} = \verb/"/{\em symbol-sequence}\verb/"/

These commands, available in the Proof Assistant\index{Proof Assistant}
only, assign a temporary variable\index{temporary variable} or unknown
step with a specific symbol sequence.  They are useful in the middle of
creating a proof, when you know what should be in the proof step but the
unification algorithm doesn't yet have enough information to completely
specify the temporary variables.  A ``temporary variable'' is one that
has the form \texttt{\$}{\em nn} in the proof display, such as
\texttt{\$1}, \texttt{\$2}, etc.  The {\em symbol-sequence} may contain
other unknown variables if desired.  Examples:

    \verb/let variable $32 = "A = B"/

    \verb/let variable $32 = "A = $35"/

    \verb/let step 10 = '|- x = x'/

    \verb/let step -2 = "|- ( $7 = ph )"/

Any symbol sequence will be accepted for the \texttt{let variable}
command.  Only those symbol sequences that can be unified with the step
will be accepted for \texttt{let step}.

The \texttt{let} commands ``zap'' the proof with information that can
only be verified when the proof is built up further.  If you make an
error, the command sequence \texttt{delete
floating{\char`\_}hypotheses}, \texttt{initialize all}, and
\texttt{unify all /interactive} will undo a bad \texttt{let} assignment.

If {\em step} is zero or negative, the -{\em step}th from last unknown
step, as shown by \texttt{show new{\char`\_}proof /unknown}, will be
used.  The command \texttt{let step 0} = \verb/"/{\em
symbol-sequence}\verb/"/ will use the last unknown step, \texttt{let
step -1} = \verb/"/{\em symbol-sequence}\verb/"/ the penultimate, etc.
If {\em step} is positive, \texttt{let step} may be used to assign known
(in the sense of having previously been assigned a label with
\texttt{assign}) as well as unknown steps.

Either single or double quotes can surround the {\em symbol-sequence} as
long as they are different from any quotes inside a {\em
symbol-sequence}.  If {\em symbol-sequence} contains both kinds of
quotes, see the instructions at the end of \texttt{help let} in the
Metamath program.


\subsection{\texttt{unify} Command}\index{\texttt{unify} command}
Syntax:  \texttt{unify step} {\em step}

      and:   \texttt{unify all} [\texttt{/interactive}]

These commands, available in the Proof Assistant only, unify the source
and target of the specified step(s). If you specify a specific step, you
will be prompted to select among the unifications that are possible.  If
you specify \texttt{all}, only those steps with unique unifications will be
unified.

Optional command qualifier for \texttt{unify all}:

    \texttt{/interactive} - You will be prompted to select among the
        unifications
        that are possible for any steps that do not have unique
        unifications.  (Otherwise \texttt{unify all} will bypass these.)

See also \texttt{set unification{\char`\_}timeout}.  The default is
100000, but increasing it to 1000000 can help difficult cases.  Manually
assigning some or all of the unknown variables with the \texttt{let
variable} command also helps difficult cases.



\subsection{\texttt{initialize} Command}\index{\texttt{initialize} command}
Syntax:  \texttt{initialize step} {\em step}

    and: \texttt{initialize all}

These commands, available in the Proof Assistant\index{Proof Assistant}
only, ``de-unify'' the target and source of a step (or all steps), as
well as the hypotheses of the source, and makes all variables in the
source and the source's hypotheses unknown.  This command is useful to
help recover from incorrect unifications that resulted from an incorrect
\texttt{assign}, \texttt{let}, or unification choice.  Part or all of
the command sequence \texttt{delete floating{\char`\_}hypotheses},
\texttt{initialize all}, and \texttt{unify all /interactive} will recover
from incorrect unifications.

See also:  \texttt{unify} and \texttt{delete}



\subsection{\texttt{delete} Command}\index{\texttt{delete} command}
Syntax:  \texttt{delete step} {\em step}

   and:      \texttt{delete all} -- {\em Warning: dangerous!}

   and:      \texttt{delete floating{\char`\_}hypotheses}

These commands are available in the Proof Assistant only.  The
\texttt{delete step} command deletes the proof tree section that
branches off of the specified step and makes the step become unknown.
\texttt{delete all} is equivalent to \texttt{delete step} {\em step}
where {\em step} is the last step in the proof (i.e.\ the beginning of
the proof tree).

In most cases the \texttt{undo} command is the best way to undo
a previous step.
An alternative is to salvage your last \texttt{save
new{\char`\_}proof} by exiting and reentering the Proof Assistant.
For this to work, keep a log file open to record your work
and to do \texttt{save new{\char`\_}proof} frequently, especially before
\texttt{delete}.

\texttt{delete floating{\char`\_}hypotheses} will delete all sections of
the proof that branch off of \texttt{\$f}\index{\texttt{\$f} statement}
statements.  It is sometimes useful to do this before an
\texttt{initialize} command to recover from an error.  Note that once a
proof step with a \texttt{\$f} hypothesis as the target is completely
known, the \texttt{improve} command can usually fill in the proof for
that step.  Unlike the deletion of logical steps, \texttt{delete
floating{\char`\_}hypotheses} is a relatively safe command that is
usually easy to recover from.



\subsection{\texttt{improve} Command}\index{\texttt{improve} command}
\label{improve}
Syntax:  \texttt{improve} {\em step} [\texttt{/depth} {\em number}]
                                               [\texttt{/no{\char`\_}distinct}]

   and:   \texttt{improve first} [\texttt{/depth} {\em number}]
                                              [\texttt{/no{\char`\_}distinct}]

   and:   \texttt{improve last} [\texttt{/depth} {\em number}]
                                              [\texttt{/no{\char`\_}distinct}]

   and:   \texttt{improve all} [\texttt{/depth} {\em number}]
                                              [\texttt{/no{\char`\_}distinct}]

These commands, available in the Proof Assistant\index{Proof Assistant}
only, try to find proofs automatically for unknown steps whose symbol
sequences are completely known.  They are primarily useful for filling in
proofs of \texttt{\$f}\index{\texttt{\$f} statement} hypotheses.  The
search will be restricted to statements having no
\texttt{\$e}\index{\texttt{\$e} statement} hypotheses.

\begin{sloppypar} % narrow
Note:  If memory is limited, \texttt{improve all} on a large proof may
overflow memory.  If you use \texttt{set unification{\char`\_}timeout 1}
before \texttt{improve all}, there will usually be sufficient
improvement to easily recover and completely \texttt{improve} the proof
later on a larger computer.  Warning:  Once memory has overflowed, there
is no recovery.  If in doubt, save the intermediate proof (\texttt{save
new{\char`\_}proof} then \texttt{write source}) before \texttt{improve
all}.
\end{sloppypar}

If \texttt{last} is specified instead of {\em step} number, the last
step that is shown by \texttt{show new{\char`\_}proof /unknown} will be
used.

If \texttt{first} is specified instead of {\em step} number, the first
step that is shown by \texttt{show new{\char`\_}proof /unknown} will be
used.

If {\em step} is zero or negative, the -{\em step}th from last unknown
step, as shown by \texttt{show new{\char`\_}proof /unknown}, will be
used.  \texttt{improve -1} will use the penultimate
unknown step, \texttt{improve -2} {\em label} the antepenultimate, and
\texttt{improve 0} is the same as \texttt{improve last}.

Optional command qualifier:

    \texttt{/depth} {\em number} - This qualifier will cause the search
        to include
        statements with \texttt{\$e} hypotheses (but no new variables in
        the \texttt{\$e}
        hypotheses), provided that the backtracking has not exceeded the
        specified depth. {\em Warning:}  Try \texttt{/depth 1},
        then \texttt{2}, then \texttt{3}, etc.
        in sequence because of possible exponential blowups.  Save your
        work before trying \texttt{/depth} greater than \texttt{1}!

    \texttt{/no{\char`\_}distinct} - Skip trial statements that have
        \texttt{\$d}\index{\texttt{\$d} statement} requirements.
        This qualifier will prevent assignments that might violate \texttt{\$d}
        requirements but it also could miss possible legal assignments.

See also: \texttt{set search{\char`\_}limit}

\subsection{\texttt{save new\_proof} Command}\index{\texttt{save
new{\char`\_}proof} command}
Syntax:  \texttt{save new{\char`\_}proof} {\em label} [\texttt{/normal}]
   [\texttt{/compressed}]

The \texttt{save new{\char`\_}proof} command is available in the Proof
Assistant only.  It saves the proof in progress in the source
buffer\index{source buffer}.  \texttt{save new{\char`\_}proof} may be
used to save a completed proof, or it may be used to save a proof in
progress in order to work on it later.  If an incomplete proof is saved,
any user assignments with \texttt{let step} or \texttt{let variable}
will be lost, as will any ambiguous unifications\index{ambiguous
unification}\index{unification!ambiguous} that were resolved manually.
To help make recovery easier, it can be helpful to \texttt{improve all}
before \texttt{save new{\char`\_}proof} so that the incomplete proof
will have as much information as possible.

Note that the proof will not be permanently saved until a \texttt{write
source} command is issued.

Optional command qualifiers:

    \texttt{/normal} - The proof is saved in the normal format (i.e., as a
        sequence of labels, which is the defined format of the basic Metamath
        language).\index{basic language}  This is the default format that
        is used if a qualifier is omitted.

    \texttt{/compressed} - The proof is saved in the compressed format, which
        reduces storage requirements for a database.
        (See Appendix~\ref{compressed}.)


\section{Creating \LaTeX\ Output}\label{texout}\index{latex@{\LaTeX}}

You can generate \LaTeX\ output given the
information in a database.
The database must already include the necessary typesetting information
(see section \ref{tcomment} for how to provide this information).

The \texttt{show statement} and \texttt{show proof} commands each have a
special \texttt{/tex} command qualifier that produces \LaTeX\ output.
(The \texttt{show statement} command also has the
\texttt{/simple{\char`\_}tex} qualifier for output that is easier to
edit by hand.)  Before you can use them, you must open a \LaTeX\ file to
which to send their output.  A typical complete session will use this
sequence of Metamath commands:

\begin{verbatim}
read set.mm
open tex example.tex
show statement a1i /tex
show proof a1i /all/lemmon/renumber/tex
show statement uneq2 /tex
show proof uneq2 /all/lemmon/renumber/tex
close tex
\end{verbatim}

See Section~\ref{mathcomments} for information on comment markup and
Appendix~\ref{ASCII} for information on how math symbol translation is
specified.

To format and print the \LaTeX\ source, you will need the \LaTeX\
program, which is standard on most Linux installations and available for
Windows.  On Linux, in order to create a {\sc pdf} file, you will
typically type at the shell prompt
\begin{verbatim}
$ pdflatex example.tex
\end{verbatim}

\subsection{\texttt{open tex} Command}\index{\texttt{open tex} command}
Syntax:  \texttt{open tex} {\em file-name} [\texttt{/no{\char`\_}header}]

This command opens a file for writing \LaTeX\
source\index{latex@{\LaTeX}} and writes a \LaTeX\ header to the file.
\LaTeX\ source can be written with the \texttt{show proof}, \texttt{show
new{\char`\_}proof}, and \texttt{show statement} commands using the
\texttt{/tex} qualifier.

The mapping to \LaTeX\ symbols is defined in a special comment
containing a \texttt{\$t} token, described in Appendix~\ref{ASCII}.

There is an optional command qualifier:

    \texttt{/no{\char`\_}header} - This qualifier prevents a standard
        \LaTeX\ header and trailer
        from being included with the output \LaTeX\ code.


\subsection{\texttt{close tex} Command}\index{\texttt{close tex} command}
Syntax:  \texttt{close tex}

This command writes a trailer to any \LaTeX\ file\index{latex@{\LaTeX}}
that was opened with \texttt{open tex} (unless
\texttt{/no{\char`\_}header} was used with \texttt{open tex}) and closes
the \LaTeX\ file.


\section{Creating {\sc HTML} Output}\label{htmlout}

You can generate {\sc html} web pages given the
information in a database.
The database must already include the necessary typesetting information
(see section \ref{tcomment} for how to provide this information).
The ability to produce {\sc html} web pages was added in Metamath version
0.07.30.

To create an {\sc html} output file(s) for \texttt{\$a} or \texttt{\$p}
statement(s), use
\begin{quote}
    \texttt{show statement} {\em label-match} \texttt{/html}
\end{quote}
The output file will be named {\em label-match}\texttt{.html}
for each match.  When {\em
label-match} has wildcard (\texttt{*}) characters, all statements with
matching labels will have {\sc html} files produced for them.  Also,
when {\em label-match} has a wildcard (\texttt{*}) character, two additional
files, \texttt{mmdefinitions.html} and \texttt{mmascii.html} will be
produced.  To produce {\em only} these two additional files, you can use
\texttt{?*}, which will not match any statement label, in place of {\em
label-match}.

There are three other qualifiers for \texttt{show statement} that also
generate {\sc HTML} code.  These are \texttt{/alt{\char`\_}html},
\texttt{/brief{\char`\_}html}, and
\texttt{/brief{\char`\_}alt{\char`\_}html}, and are described in the
next section.

The command
\begin{quote}
    \texttt{show statement} {\em label-match} \texttt{/alt{\char`\_}html}
\end{quote}
does the same as \texttt{show statement} {\em label-match} \texttt{/html},
except that the {\sc html} code for the symbols is taken from
\texttt{althtmldef} statements instead of \texttt{htmldef} statements in
the \texttt{\$t} comment.

The command
\begin{verbatim}
show statement * /brief_html
\end{verbatim}
invokes a special mode that just produces definition and theorem lists
accompanied by their symbol strings, in a format suitable for copying and
pasting into another web page (such as the tutorial pages on the
Metamath web site).

Finally, the command
\begin{verbatim}
show statement * /brief_alt_html
\end{verbatim}
does the same as \texttt{show statement * / brief{\char`\_}html}
for the alternate {\sc html}
symbol representation.

A statement's comment can include a special notation that provides a
certain amount of control over the {\sc HTML} version of the comment.  See
Section~\ref{mathcomments} (p.~\pageref{mathcomments}) for the comment
markup features.

The \texttt{write theorem{\char`\_}list} and \texttt{write bibliography}
commands, which are described below, provide as a side effect complete
error checking for all of the features described in this
section.\index{error checking}

\subsection{\texttt{write theorem\_list}
Command}\index{\texttt{write theorem{\char`\_}list} command}
Syntax:  \texttt{write theorem{\char`\_}list}
[\texttt{/theorems{\char`\_}per{\char`\_}page} {\em number}]

This command writes a list of all of the \texttt{\$a} and \texttt{\$p}
statements in the database into a web page file
 called \texttt{mmtheorems.html}.
When additional files are needed, they are called
\texttt{mmtheorems2.html}, \texttt{mmtheorems3.html}, etc.

Optional command qualifier:

    \texttt{/theorems{\char`\_}per{\char`\_}page} {\em number} -
 This qualifier specifies the number of statements to
        write per web page.  The default is 100.

{\em Note:} In version 0.07.30\index{Metamath!limitations of version
0.07.30} of Metamath, the ``Nearby theorems'' links on the individual
web pages presuppose 100 theorems per page when linking to the theorem
list pages.  Therefore the \texttt{/theorems{\char`\_}per{\char`\_}page}
qualifier, if it specifies a number other than 100, will cause the
individual web pages to be out of sync and should not be used to
generate the main theorem list for the web site.  This is expected to be
fixed in a future version.


\subsection{\texttt{write bibliography}\label{wrbib}
Command}\index{\texttt{write bibliography} command}
Syntax:  \texttt{write bibliography} {\em filename}

This command reads an existing {\sc html} bibliographic cross-reference
file, normally called \texttt{mmbiblio.html}, and updates it per the
bibliographic links in the database comments.  The file is updated
between the {\sc html} comment lines \texttt{<!--
{\char`\#}START{\char`\#} -->} and \texttt{<!-- {\char`\#}END{\char`\#}
-->}.  The original input file is renamed to {\em
filename}\texttt{{\char`\~}1}.

A bibliographic reference is indicated with the reference name
in brackets, such as  \texttt{Theorem 3.1 of
[Monk] p.\ 22}.
See Section~\ref{htmlout} (p.~\pageref{htmlout}) for
syntax details.


\subsection{\texttt{write recent\_additions}
Command}\index{\texttt{write recent{\char`\_}additions} command}
Syntax:  \texttt{write recent{\char`\_}additions} {\em filename}
[\texttt{/limit} {\em number}]

This command reads an existing ``Recent Additions'' {\sc html} file,
normally called \texttt{mmrecent.html}, and updates it with the
descriptions of the most recently added theorems to the database.
 The file is updated between
the {\sc html} comment lines \texttt{<!-- {\char`\#}START{\char`\#} -->}
and \texttt{<!-- {\char`\#}END{\char`\#} -->}.  The original input file
is renamed to {\em filename}\texttt{{\char`\~}1}.

Optional command qualifier:

    \texttt{/limit} {\em number} -
 This qualifier specifies the number of most recent theorems to
   write to the output file.  The default is 100.


\section{Text File Utilities}

\subsection{\texttt{tools} Command}\index{\texttt{tools} command}
Syntax:  \texttt{tools}

This command invokes an easy-to-use, general purpose utility for
manipulating the contents of {\sc ascii} text files.  Upon typing
\texttt{tools}, the command-line prompt will change to \texttt{TOOLS>}
until you type \texttt{exit}.  The \texttt{tools} commands can be used
to perform simple, global edits on an input/output file,
such as making a character string substitution on each line, adding a
string to each line, and so on.  A typical use of this utility is
to build a \texttt{submit} input file to perform a common operation on a
list of statements obtained from \texttt{show label} or \texttt{show
usage}.

The actions of most of the \texttt{tools} commands can also be
performed with equivalent (and more powerful) Unix shell commands, and
some users may find those more efficient.  But for Windows users or
users not comfortable with Unix, \texttt{tools} provides an
easy-to-learn alternative that is adequate for most of the
script-building tasks needed to use the Metamath program effectively.

\subsection{\texttt{help} Command (in \texttt{tools})}
Syntax:  \texttt{help}

The \texttt{help} command lists the commands available in the
\texttt{tools} utility, along with a brief description.  Each command,
in turn, has its own help, such as \texttt{help add}.  As with
Metamath's \texttt{MM>} prompt, a complete command can be entered at
once, or just the command word can be typed, causing the program to
prompt for each argument.

\vskip 1ex
\noindent Line-by-line editing commands:

  \texttt{add} - Add a specified string to each line in a file.

  \texttt{clean} - Trim spaces and tabs on each line in a file; convert
         characters.

  \texttt{delete} - Delete a section of each line in a file.

  \texttt{insert} - Insert a string at a specified column in each line of
        a file.

  \texttt{substitute} - Make a simple substitution on each line of the file.

  \texttt{tag} - Like \texttt{add}, but restricted to a range of lines.

  \texttt{swap} - Swap the two halves of each line in a file.

\vskip 1ex
\noindent Other file-processing commands:

  \texttt{break} - Break up (tokenize) a file into a list of tokens (one per
        line).

  \texttt{build} - Build a file with multiple tokens per line from a list.

  \texttt{count} - Count the occurrences in a file of a specified string.

  \texttt{number} - Create a list of numbers.

  \texttt{parallel} - Put two files in parallel.

  \texttt{reverse} - Reverse the order of the lines in a file.

  \texttt{right} - Right-justify lines in a file (useful before sorting
         numbers).

%  \texttt{tag} - Tag edit updates in a program for revision control.

  \texttt{sort} - Sort the lines in a file with key starting at
         specified string.

  \texttt{match} - Extract lines containing (or not) a specified string.

  \texttt{unduplicate} - Eliminate duplicate occurrences of lines in a file.

  \texttt{duplicate} - Extract first occurrence of any line occurring
         more than

   \ \ \    once in a file, discarding lines occurring exactly once.

  \texttt{unique} - Extract lines occurring exactly once in a file.

  \texttt{type} (10 lines) - Display the first few lines in a file.
                  Similar to Unix \texttt{head}.

  \texttt{copy} - Similar to Unix \texttt{cat} but safe (same input
         and output file allowed).

  \texttt{submit} - Run a script containing \texttt{tools} commands.

\vskip 1ex

\noindent Note:
  \texttt{unduplicate}, \texttt{duplicate}, and \texttt{unique} also
 sort the lines as a side effect.


\subsection{Using \texttt{tools} to Build Metamath \texttt{submit}
Scripts}

The \texttt{break} command is typically used to break up a series of
statement labels, such as the output of Metamath's \texttt{show usage},
into one label per line.  The other \texttt{tools} commands can then be
used to add strings before and after each statement label to specify
commands to be performed on the statement.  The \texttt{parallel}
command is useful when a statement label must be mentioned more than
once on a line.

Very often a \texttt{submit} script for Metamath will require multiple
command lines for each statement being processed.  For example, you may
want to enter the Proof Assistant, \texttt{minimize{\char`\_}with} your
latest theorem, \texttt{save} the new proof, and \texttt{exit} the Proof
Assistant.  To accomplish this, you can build a file with these four
commands for each statement on a single line, separating each command
with a designated character such as \texttt{@}.  Then at the end you can
\texttt{substitute} each \texttt{@} with \texttt{{\char`\\}n} to break
up the lines into individual command lines (see \texttt{help
substitute}).


\subsection{Example of a \texttt{tools} Session}

To give you a quick feel for the \texttt{tools} utility, we show a
simple session where we create a file \texttt{n.txt} with 3 lines, add
strings before and after each line, and display the lines on the screen.
You can experiment with the various commands to gain experience with the
\texttt{tools} utility.

\begin{verbatim}
MM> tools
Entering the Text Tools utilities.
Type HELP for help, EXIT to exit.
TOOLS> number
Output file <n.tmp>? n.txt
First number <1>?
Last number <10>? 3
Increment <1>?
TOOLS> add
Input/output file? n.txt
String to add to beginning of each line <>? This is line
String to add to end of each line <>? .
The file n.txt has 3 lines; 3 were changed.
First change is on line 1:
This is line 1.
TOOLS> type n.txt
This is line 1.
This is line 2.
This is line 3.
TOOLS> exit
Exiting the Text Tools.
Type EXIT again to exit Metamath.
MM>
\end{verbatim}



\appendix
\chapter{Sample Representations}
\label{ASCII}

This Appendix provides a sample of {\sc ASCII} representations,
their corresponding traditional mathematical symbols,
and a discussion of their meanings
in the \texttt{set.mm} database.
These are provided in order of appearance.
This is only a partial list, and new definitions are routinely added.
A complete list is available at \url{http://metamath.org}.

These {\sc ASCII} representations, along
with information on how to display them,
are defined in the \texttt{set.mm} database file inside
a special comment called a \texttt{\$t} {\em
comment}\index{\texttt{\$t} comment} or {\em typesetting
comment.}\index{typesetting comment}
A typesetting comment
is indicated by the appearance of the
two-character string \texttt{\$t} at the beginning of the comment.
For more information,
see Section~\ref{tcomment}, p.~\pageref{tcomment}.

In the following table the ``{\sc ASCII}'' column shows the {\sc ASCII}
representation,
``Symbol'' shows the mathematical symbolic display
that corresponds to that {\sc ASCII} representation, ``Labels'' shows
the key label(s) that define the representation, and
``Description'' provides a description about the symbol.
As usual, ``iff'' is short for ``if and only if.''\index{iff}
In most cases the ``{\sc ASCII}'' column only shows
the key token, but it will sometimes show a sequence of tokens
if that is necessary for clarity.

{\setlength{\extrarowsep}{4pt} % Keep rows from being too close together
\begin{longtabu}   { @{} c c l X }
\textbf{ASCII} & \textbf{Symbol} & \textbf{Labels} & \textbf{Description} \\
\endhead
\texttt{|-} & $\vdash$ & &
  ``It is provable that...'' \\
\texttt{ph} & $\varphi$ & \texttt{wph} &
  The wff (boolean) variable phi,
  conventionally the first wff variable. \\
\texttt{ps} & $\psi$ & \texttt{wps} &
  The wff (boolean) variable psi,
  conventionally the second wff variable. \\
\texttt{ch} & $\chi$ & \texttt{wch} &
  The wff (boolean) variable chi,
  conventionally the third wff variable. \\
\texttt{-.} & $\lnot$ & \texttt{wn} &
  Logical not. E.g., if $\varphi$ is true, then $\lnot \varphi$ is false. \\
\texttt{->} & $\rightarrow$ & \texttt{wi} &
  Implies, also known as material implication.
  In classical logic the expression $\varphi \rightarrow \psi$ is true
  if either $\varphi$ is false or $\psi$ is true (or both), that is,
  $\varphi \rightarrow \psi$ has the same meaning as
  $\lnot \varphi \lor \psi$ (as proven in theorem \texttt{imor}). \\
\texttt{<->} & $\leftrightarrow$ &
  \hyperref[df-bi]{\texttt{df-bi}} &
  Biconditional (aka is-equals for boolean values).
  $\varphi \leftrightarrow \psi$ is true iff
  $\varphi$ and $\psi$ have the same value. \\
\texttt{\char`\\/} & $\lor$ &
  \makecell[tl]{{\hyperref[df-or]{\texttt{df-or}}}, \\
	         \hyperref[df-3or]{\texttt{df-3or}}} &
  Disjunction (logical ``or''). $\varphi \lor \psi$ is true iff
  $\varphi$, $\psi$, or both are true. \\
\texttt{/\char`\\} & $\land$ &
  \makecell[tl]{{\hyperref[df-an]{\texttt{df-an}}}, \\
                 \hyperref[df-3an]{\texttt{df-3an}}} &
  Conjunction (logical ``and''). $\varphi \land \psi$ is true iff
  both $\varphi$ and $\psi$ are true. \\
\texttt{A.} & $\forall$ &
  \texttt{wal} &
  For all; the wff $\forall x \varphi$ is true iff
  $\varphi$ is true for all values of $x$. \\
\texttt{E.} & $\exists$ &
  \hyperref[df-ex]{\texttt{df-ex}} &
  There exists; the wff
  $\exists x \varphi$ is true iff
  there is at least one $x$ where $\varphi$ is true. \\
\texttt{[ y / x ]} & $[ y / x ]$ &
  \hyperref[df-sb]{\texttt{df-sb}} &
  The wff $[ y / x ] \varphi$ produces
  the result when $y$ is properly substituted for $x$ in $\varphi$
  ($y$ replaces $x$).
  % This is elsb4
  % ( [ x / y ] z e. y <-> z e. x )
  For example,
  $[ x / y ] z \in y$ is the same as $z \in x$. \\
\texttt{E!} & $\exists !$ &
  \hyperref[df-eu]{\texttt{df-eu}} &
  There exists exactly one;
  $\exists ! x \varphi$ is true iff
  there is at least one $x$ where $\varphi$ is true. \\
\texttt{\{ y | phi \}}  & $ \{ y | \varphi \}$ &
  \hyperref[df-clab]{\texttt{df-clab}} &
  The class of all sets where $\varphi$ is true. \\
\texttt{=} & $ = $ &
  \hyperref[df-cleq]{\texttt{df-cleq}} &
  Class equality; $A = B$ iff $A$ equals $B$. \\
\texttt{e.} & $ \in $ &
  \hyperref[df-clel]{\texttt{df-clel}} &
  Class membership; $A \in B$ if $A$ is a member of $B$. \\
\texttt{{\char`\_}V} & {\rm V} &
  \hyperref[df-v]{\texttt{df-v}} &
  Class of all sets (not itself a set). \\
\texttt{C\_} & $ \subseteq $ &
  \hyperref[df-ss]{\texttt{df-ss}} &
  Subclass (subset); $A \subseteq B$ is true iff
  $A$ is a subclass of $B$. \\
\texttt{u.} & $ \cup $ &
  \hyperref[df-un]{\texttt{df-un}} &
  $A \cup B$ is the union of classes $A$ and $B$. \\
\texttt{i^i} & $ \cap $ &
  \hyperref[df-in]{\texttt{df-in}} &
  $A \cap B$ is the intersection of classes $A$ and $B$. \\
\texttt{\char`\\} & $ \setminus $ &
  \hyperref[df-dif]{\texttt{df-dif}} &
  $A \setminus B$ (set difference)
  is the class of all sets in $A$ except for those in $B$. \\
\texttt{(/)} & $ \varnothing $ &
  \hyperref[df-nul]{\texttt{df-nul}} &
  $ \varnothing $ is the empty set (aka null set). \\
\texttt{\char`\~P} & $ \cal P $ &
  \hyperref[df-pw]{\texttt{df-pw}} &
  Power class. \\
\texttt{<.\ A , B >.} & $\langle A , B \rangle$ &
  \hyperref[df-op]{\texttt{df-op}} &
  The ordered pair $\langle A , B \rangle$. \\
\texttt{( F ` A )} & $ ( F ` A ) $ &
  \hyperref[df-fv]{\texttt{df-fv}} &
  The value of function $F$ when applied to $A$. \\
\texttt{\_i} & $ i $ &
  \texttt{df-i} &
  The square root of negative one. \\
\texttt{x.} & $ \cdot $ &
  \texttt{df-mul} &
  Complex number multiplication; $2~\cdot~3~=~6$. \\
\texttt{CC} & $ \mathbb{C} $ &
  \texttt{df-c} &
  The set of complex numbers. \\
\texttt{RR} & $ \mathbb{R} $ &
  \texttt{df-r} &
  The set of real numbers. \\
\end{longtabu}
} % end of extrarowsep

\chapter{Compressed Proofs}
\label{compressed}\index{compressed proof}\index{proof!compressed}

The proofs in the \texttt{set.mm} set theory database are stored in compressed
format for efficiency.  Normally you needn't concern yourself with the
compressed format, since you can display it with the usual proof display tools
in the Metamath program (\texttt{show proof}\ldots) or convert it to the normal
RPN proof format described in Section~\ref{proof} (with \texttt{save proof}
{\em label} \texttt{/normal}).  However for sake of completeness we describe the
format here and show how it maps to the normal RPN proof format.

A compressed proof, located between \texttt{\$=} and \texttt{\$.}\ keywords, consists
of a left parenthesis, a sequence of statement labels, a right parenthesis,
and a sequence of upper-case letters \texttt{A} through \texttt{Z} (with optional
white space between them).  White space must surround the parentheses
and the labels.  The left parenthesis tells Metamath that a
compressed proof follows.  (A normal RPN proof consists of just a sequence of
labels, and a parenthesis is not a legal character in a label.)

The sequence of upper-case letters corresponds to a sequence of integers
with the following mapping.  Each integer corresponds to a proof step as
described later.
\begin{center}
  \texttt{A} = 1 \\
  \texttt{B} = 2 \\
   \ldots \\
  \texttt{T} = 20 \\
  \texttt{UA} = 21 \\
  \texttt{UB} = 22 \\
   \ldots \\
  \texttt{UT} = 40 \\
  \texttt{VA} = 41 \\
  \texttt{VB} = 42 \\
   \ldots \\
  \texttt{YT} = 120 \\
  \texttt{UUA} = 121 \\
   \ldots \\
  \texttt{YYT} = 620 \\
  \texttt{UUUA} = 621 \\
   etc.
\end{center}

In other words, \texttt{A} through \texttt{T} represent the
least-significant digit in base 20, and \texttt{U} through \texttt{Y}
represent zero or more most-significant digits in base 5, where the
digits start counting at 1 instead of the usual 0. With this scheme, we
don't need white space between these ``numbers.''

(In the design of the compressed proof format, only upper-case letters,
as opposed to say all non-whitespace printable {\sc ascii} characters other than
%\texttt{\$}, was chosen to make the compressed proof a little less
%displeasing to the eye, at the expense of a typical 20\% compression
\texttt{\$}, were chosen so as not to collide with most text editor
searches, at the expense of a typical 20\% compression
loss.  The base 5/base 20 grouping, as opposed to say base 6/base 19,
was chosen by experimentally determining the grouping that resulted in
best typical compression.)

The letter \texttt{Z} identifies (tags) a proof step that is identical to one
that occurs later on in the proof; it helps shorten the proof by not requiring
that identical proof steps be proved over and over again (which happens often
when building wff's).  The \texttt{Z} is placed immediately after the
least-significant digit (letters \texttt{A} through \texttt{T}) that ends the integer
corresponding to the step to later be referenced.

The integers that the upper-case letters correspond to are mapped to labels as
follows.  If the statement being proved has $m$ mandatory hypotheses, integers
1 through $m$ correspond to the labels of these hypotheses in the order shown
by the \texttt{show statement ... / full} command, i.e., the RPN order\index{RPN
order} of the mandatory
hypotheses.  Integers $m+1$ through $m+n$ correspond to the labels enclosed in
the parentheses of the compressed proof, in the order that they appear, where
$n$ is the number of those labels.  Integers $m+n+1$ on up don't directly
correspond to statement labels but point to proof steps identified with the
letter \texttt{Z}, so that these proof steps can be referenced later in the
proof.  Integer $m+n+1$ corresponds to the first step tagged with a \texttt{Z},
$m+n+2$ to the second step tagged with a \texttt{Z}, etc.  When the compressed
proof is converted to a normal proof, the entire subproof of a step tagged
with \texttt{Z} replaces the reference to that step.

For efficiency, Metamath works with compressed proofs directly, without
converting them internally to normal proofs.  In addition to the usual
error-checking, an error message is given if (1) a label in the label list in
parentheses does not refer to a previous \texttt{\$p} or \texttt{\$a} statement or a
non-mandatory hypothesis of the statement being proved and (2) a proof step
tagged with \texttt{Z} is referenced before the step tagged with the \texttt{Z}.

Just as in a normal proof under development (Section~\ref{unknown}), any step
or subproof that is not yet known may be represented with a single \texttt{?}.
White space does not have to appear between the \texttt{?}\ and the upper-case
letters (or other \texttt{?}'s) representing the remainder of the proof.

% April 1, 2004 Appendix C has been added back in with corrections.
%
% May 20, 2003 Appendix C was removed for now because there was a problem found
% by Bob Solovay
%
% Also, removed earlier \ref{formalspec} 's (3 cases above)
%
% Bob Solovay wrote on 30 Nov 2002:
%%%%%%%%%%%%% (start of email comment )
%      3. My next set of comments concern appendix C. I read this before I
% read Chapter 4. So I first noted that the system as presented in the
% Appendix lacked a certain formal property that I thought desirable. I
% then came up with a revised formal system that had this property. Upon
% reading Chapter 4, I noticed that the revised system was closer to the
% treatment in Chapter 4 than the system in Appendix C.
%
%         First a very minor correction:
%
%         On page 142 line 2: The condition that V(e) != V(f) should only be
% required of e, f in T such that e != f.
%
%         Here is a natural property [transitivity] that one would like
% the formal system to have:
%
%         Let Gamma be a set of statements. Suppose that the statement Phi
% is provable from Gamma and that the statement Psi is provable from Gamma
% \cup {Phi}. Then Psi is provable from Gamma.
%
%         I shall present an example to show that this property does not
% hold for the formal systems of Appendix C:
%
%         I write the example in metamath style:
%
% $c A B C D E $.
% $v x y
%
% ${
% tx $f A x $.
% ty $f B y $.
% ax1 $a C x y $.
% $}
%
% ${
% tx $f A x $.
% ty $f B y $.
% ax2-h1 $e C x y $.
% ax2 $a D y $.
% $}
%
% ${
% ty $f B y $.
% ax3-h1 $e D y $.
% ax3 $a E y $.
% $}
%
% $(These three axioms are Gamma $)
%
% ${
% tx $f A x $.
% ty $f B y $.
% Phi $p D y $=
% tx ty tx ty ax1 ax2 $.
% $}
%
% ${
% ty $f B y $.
% Psi $p E y $=
% ty ty Phi ax3 $.
% $}
%
%
% I omit the formal proofs of the following claims. [I will be glad to
% supply them upon request.]
%
% 1) Psi is not provable from Gamma;
%
% 2) Psi is provable from Gamma + Phi.
%
% Here "provable" refers to the formalism of Appendix C.
%
% The trouble of course is that Psi is lacking the variable declaration
%
% $f Ax $.
%
% In the Metamath system there is no trouble proving Psi. I attach a
% metamath file that shows this and which has been checked by the
% metamath program.
%
% I next want to indicate how I think the treatment in Appendix C should
% be revised so as to conform more closely to the metamath system of the
% main text. The revised system *does* have the transitivity property.
%
% We want to give revised definitions of "statement" and
% "provable". [cf. sections C.2.4. and C.2.5] Our new definitions will
% use the definitions given in Appendix C. So we take the following
% tack. We refer to the original notions as o-statement and o-provable. And
% we refer to the notions we are defining as n-statement and n-provable.
%
%         A n-statement is an o-statement in which the only variables
% that appear in the T component are mandatory.
%
%         To any o-statement we can associate its reduct which is a
% n-statement by dropping all the elements of T or D which contain
% non-mandatory variables.
%
%         An n-statement gamma is n-provable if there is an o-statement
% gamma' which has gamma as its reduct andf such that gamma' is
% o-provable.
%
%         It seems to me [though I am not completely sure on this point]
% that n-provability corresponds to metamath provability as discussed
% say in Chapter 4.
%
%         Attached to this letter is the metamath proof of Phi and Psi
% from Gamma discussed above.
%
%         I am still brooding over the question of whether metamath
% correctly formalizes set-theory. No doubt I will have some questions
% re this after my thoughts become clearer.
%%%%%%%%%%%%%%%% (end of email comment)

%%%%%%%%%%%%%%%% (start of 2nd email comment from Bob Solovay 1-Apr-04)
%
%         I hope that Appendix C is the one that gives a "formal" treatment
% of Metamath. At any rate, thats the appendix I want to comment on.
%
%         I'm going to suggest two changes in the definition.
%
%         First change (in the definition of statement): Require that the
% sets D, T, and E be finite.
%
%         Probably things are fine as you give them. But in the applications
% to the main metamath system they will always be finite, and its useful in
% thinking about things [at least for me] to stick to the finite case.
%
%         Second change:
%
%         First let me give an approximate description. Remove the dummy
% variables from the statement. Instead, include them in the proof.
%
%         More formally: Require that T consists of type declarations only
% for mandatory variables. Require that all the pairs in D consist of
% mandatory variables.
%
%         At the start of a proof we are allowed to declare a finite number
% of dummy variables [provided that none of them appear in any of the
% statements in E \cup {A}. We have to supply type declarations for all the
% dummy variables. We are allowed to add new $d statements referring to
% either the mandatory or dummy variables. But we require that no new $d
% statement references only mandatory variables.
%
%         I find this way of doing things more conceptual than the treatment
% in Appendix C. But the change [which I will use implicitly in later
% letters about doing Peano] is mainly aesthetic. I definitely claim that my
% results on doing Peano all apply to Metamath as it is presented in your
% book.
%
%         --Bob
%
%%%%%%%%%%%%%%%% (end of 2nd email comment)

%%
%% When uncommenting the below, also uncomment references above to {formalspec}
%%
\chapter{Metamath's Formal System}\label{formalspec}\index{Metamath!as a formal
system}

\section{Introduction}

\begin{quote}
  {\em Perfection is when there is no longer anything more to take away.}
    \flushright\sc Antoine de
     Saint-Exupery\footnote{\cite[p.~3-25]{Campbell}.}\\
\end{quote}\index{de Saint-Exupery, Antoine}

This appendix describes the theory behind the Metamath language in an abstract
way intended for mathematicians.  Specifically, we construct two
set-theo\-ret\-i\-cal objects:  a ``formal system'' (roughly, a set of syntax
rules, axioms, and logical rules) and its ``universe'' (roughly, the set of
theorems derivable in the formal system).  The Metamath computer language
provides us with a way to describe specific formal systems and, with the aid of
a proof provided by the user, to verify that given theorems
belong to their universes.

To understand this appendix, you need a basic knowledge of informal set theory.
It should be sufficient to understand, for example, Ch.\ 1 of Munkres' {\em
Topology} \cite{Munkres}\index{Munkres, James R.} or the
introductory set theory chapter
in many textbooks that introduce abstract mathematics. (Note that there are
minor notational differences among authors; e.g.\ Munkres uses $\subset$ instead
of our $\subseteq$ for ``subset.''  We use ``included in'' to mean ``a subset
of,'' and ``belongs to'' or ``is contained in'' to mean ``is an element of.'')
What we call a ``formal'' description here, unlike earlier, is actually an
informal description in the ordinary language of mathematicians.  However we
provide sufficient detail so that a mathematician could easily formalize it,
even in the language of Metamath itself if desired.  To understand the logic
examples at the end of this appendix, familiarity with an introductory book on
mathematical logic would be helpful.

\section{The Formal Description}

\subsection[Preliminaries]{Preliminaries\protect\footnotemark}%
\footnotetext{This section is taken mostly verbatim
from Tarski \cite[p.~63]{Tarski1965}\index{Tarski, Alfred}.}

By $\omega$ we denote the set of all natural numbers (non-negative integers).
Each natural number $n$ is identified with the set of all smaller numbers: $n =
\{ m | m < n \}$.  The formula $m < n$ is thus equivalent to the condition: $m
\in n$ and $m,n \in \omega$. In particular, 0 is the number zero and at the
same time the empty set $\varnothing$, $1=\{0\}$, $2=\{0,1\}$, etc. ${}^B A$
denotes the set of all functions on $B$ to $A$ (i.e.\ with domain $B$ and range
included in $A$).  The members of ${}^\omega A$ are what are called {\em simple
infinite sequences},\index{simple infinite sequence}
with all {\em terms}\index{term} in $A$.  In case $n \in \omega$, the
members of ${}^n A$ are referred to as {\em finite $n$-termed
sequences},\index{finite $n$-termed
sequence} again
with terms in $A$.  The consecutive terms (function values) of a finite or
infinite sequence $f$ are denoted by $f_0, f_1, \ldots ,f_n,\ldots$.  Every
finite sequence $f \in \bigcup _{n \in \omega} {}^n A$ uniquely determines the
number $n$ such that $f \in {}^n A$; $n$ is called the {\em
length}\index{length of a sequence ({$"|\ "|$})} of $f$ and
is denoted by $|f|$.  $\langle a \rangle$ is the sequence $f$ with $|f|=1$ and
$f_0=a$; $\langle a,b \rangle$ is the sequence $f$ with $|f|=2$, $f_0=a$,
$f_1=b$; etc.  Given two finite sequences $f$ and $g$, we denote by $f\frown g$
their {\em concatenation},\index{concatenation} i.e., the
finite sequence $h$ determined by the
conditions:
\begin{eqnarray*}
& |h| = |f|+|g|;&  \\
& h_n = f_n & \mbox{\ for\ } n < |f|;  \\
& h_{|f|+n} = g_n & \mbox{\ for\ } n < |g|.
\end{eqnarray*}

\subsection{Constants, Variables, and Expressions}

A formal system has a set of {\em symbols}\index{symbol!in
a formal system} denoted
by $\mbox{\em SM}$.  A
precise set-theo\-ret\-i\-cal definition of this set is unimportant; a symbol
could be considered a primitive or atomic element if we wish.  We assume this
set is divided into two disjoint subsets:  a set $\mbox{\em CN}$ of {\em
constants}\index{constant!in a formal system} and a set $\mbox{\em VR}$ of
{\em variables}.\index{variable!in a formal system}  $\mbox{\em CN}$ and
$\mbox{\em VR}$ are each assumed to consist of countably many symbols which
may be arranged in finite or simple infinite sequences $c_0, c_1, \ldots$ and
$v_0, v_1, \ldots$ respectively, without repeating terms.  We will represent
arbitrary symbols by metavariables $\alpha$, $\beta$, etc.

{\footnotesize\begin{quotation}
{\em Comment.} The variables $v_0, v_1, \ldots$ of our formal system
correspond to what are usually considered ``metavariables'' in
descriptions of specific formal systems in the literature.  Typically,
when describing a specific formal system a book will postulate a set of
primitive objects called variables, then proceed to describe their
properties using metavariables that range over them, never mentioning
again the actual variables themselves.  Our formal system does not
mention these primitive variable objects at all but deals directly with
metavariables, as its primitive objects, from the start.  This is a
subtle but key distinction you should keep in mind, and it makes our
definition of ``formal system'' somewhat different from that typically
found in the literature.  (So, the $\alpha$, $\beta$, etc.\ above are
actually ``metametavariables'' when used to represent $v_0, v_1,
\ldots$.)
\end{quotation}}

Finite sequences all terms of which are symbols are called {\em
expressions}.\index{expression!in a formal system}  $\mbox{\em EX}$ is
the set of all expressions; thus
\begin{displaymath}
\mbox{\em EX} = \bigcup _{n \in \omega} {}^n \mbox{\em SM}.
\end{displaymath}

A {\em constant-prefixed expression}\index{constant-prefixed expression}
is an expression of non-zero length
whose first term is a constant.  We denote the set of all constant-prefixed
expressions by $\mbox{\em EX}_C = \{ e \in \mbox{\em EX} | ( |e| > 0 \wedge
e_0 \in \mbox{\em CN} ) \}$.

A {\em constant-variable pair}\index{constant-variable pair}
is an expression of length 2 whose first term
is a constant and whose second term is a variable.  We denote the set of all
constant-variable pairs by $\mbox{\em EX}_2 = \{ e \in \mbox{\em EX}_C | ( |e|
= 2 \wedge e_1 \in \mbox{\em VR} ) \}$.


{\footnotesize\begin{quotation}
{\em Relationship to Metamath.} In general, the set $\mbox{\em SM}$
corresponds to the set of declared math symbols in a Metamath database, the
set $\mbox{\em CN}$ to those declared with \texttt{\$c} statements, and the set
$\mbox{\em VR}$ to those declared with \texttt{\$v} statements.  Of course a
Metamath database can only have a finite number of math symbols, whereas
formal systems in general can have an infinite number, although the number of
Metamath math symbols available is in principle unlimited.

The set $\mbox{\em EX}_C$ corresponds to the set of permissible expressions
for \texttt{\$e}, \texttt{\$a}, and \texttt{\$p} statements.  The set $\mbox{\em EX}_2$
corresponds to the set of permissible expressions for \texttt{\$f} statements.
\end{quotation}}

We denote by ${\cal V}(e)$ the set of all variables in an expression $e \in
\mbox{\em EX}$, i.e.\ the set of all $\alpha \in \mbox{\em VR}$ such that
$\alpha = e_n$ for some $n < |e|$.  We also denote (with abuse of notation) by
${\cal V}(E)$ the set of all variables in a collection of expressions $E
\subseteq \mbox{\em EX}$, i.e.\ $\bigcup _{e \in E} {\cal V}(e)$.


\subsection{Substitution}

Given a function $F$ from $\mbox{\em VR}$ to
$\mbox{\em EX}$, we
denote by $\sigma_{F}$ or just $\sigma$ the function from $\mbox{\em EX}$ to
$\mbox{\em EX}$ defined recursively for nonempty sequences by
\begin{eqnarray*}
& \sigma(<\alpha>) = F(\alpha) & \mbox{for\ } \alpha \in \mbox{\em VR}; \\
& \sigma(<\alpha>) = <\alpha> & \mbox{for\ } \alpha \not\in \mbox{\em VR}; \\
& \sigma(g \frown h) = \sigma(g) \frown
    \sigma(h) & \mbox{for\ } g,h \in \mbox{\em EX}.
\end{eqnarray*}
We also define $\sigma(\varnothing)=\varnothing$.  We call $\sigma$ a {\em
simultaneous substitution}\index{substitution!variable}\index{variable
substitution} (or just {\em substitution}) with {\em substitution
map}\index{substitution map} $F$.

We also denote (with abuse of notation) by $\sigma(E)$ a substitution on a
collection of expressions $E \subseteq \mbox{\em EX}$, i.e.\ the set $\{
\sigma(e) | e \in E \}$.  The collection $\sigma(E)$ may of course contain
fewer expressions than $E$ because duplicate expressions could result from the
substitution.

\subsection{Statements}

We denote by $\mbox{\em DV}$ the set of all
unordered pairs $\{\alpha, \beta \} \subseteq \mbox{\em VR}$ such that $\alpha
\neq \beta$.  $\mbox{\em DV}$ stands for ``distinct variables.''

A {\em pre-statement}\index{pre-statement!in a formal system} is a
quadruple $\langle D,T,H,A \rangle$ such that
$D\subseteq \mbox{\em DV}$, $T\subseteq \mbox{\em EX}_2$, $H\subseteq
\mbox{\em EX}_C$ and $H$ is finite,
$A\in \mbox{\em EX}_C$, ${\cal V}(H\cup\{A\}) \subseteq
{\cal V}(T)$, and $\forall e,f\in T {\ } {\cal V}(e) \neq {\cal V}(f)$ (or
equivalently, $e_1 \ne f_1$) whenever $e \neq f$. The terms of the quadruple are called {\em
distinct-variable restrictions},\index{disjoint-variable restriction!in a
formal system} {\em variable-type hypotheses},\index{variable-type
hypothesis!in a formal system} {\em logical hypotheses},\index{logical
hypothesis!in a formal system} and the {\em assertion}\index{assertion!in a
formal system} respectively.  We denote by $T_M$ ({\em mandatory variable-type
hypotheses}\index{mandatory variable-type hypothesis!in a formal system}) the
subset of $T$ such that ${\cal V}(T_M) ={\cal V}(H \cup \{A\})$.  We denote by
$D_M=\{\{\alpha,\beta\}\in D|\{\alpha,\beta\}\subseteq {\cal V}(T_M)\}$ the
{\em mandatory distinct-variable restrictions}\index{mandatory
disjoint-variable restriction!in a formal system} of the pre-statement.
The set
of {\em mandatory hypotheses}\index{mandatory hypothesis!in a formal system}
is $T_M\cup H$.  We call the quadruple $\langle D_M,T_M,H,A \rangle$
the {\em reduct}\index{reduct!in a formal system} of
the pre-statement $\langle D,T,H,A \rangle$.

A {\em statement} is the reduct of some pre-statement\index{statement!in a
formal system}.  A statement is therefore a special kind of pre-statement;
in particular, a statement is the reduct of itself.

{\footnotesize\begin{quotation}
{\em Comment.}  $T$ is a set of expressions, each of length 2, that associate
a set of constants (``variable types'') with a set of variables.  The
condition ${\cal V}(H\cup\{A\}) \subseteq {\cal V}(T) $
means that each variable occurring in a statement's logical
hypotheses or assertion must have an associated variable-type hypothesis or
``type declaration,'' in  analogy to a computer programming language, where a
variable must be declared to be say, a string or an integer.  The requirement
that $\forall e,f\in T \, e_1 \ne f_1$ for $e\neq f$
means that each variable must be
associated with a unique constant designating its variable type; e.g., a
variable might be a ``wff'' or a ``set'' but not both.

Distinct-variable restrictions are used to specify what variable substitutions
are permissible to make for the statement to remain valid.  For example, in
the theorem scheme of set theory $\lnot\forall x\,x=y$ we may not substitute
the same variable for both $x$ and $y$.  On the other hand, the theorem scheme
$x=y\to y=x$ does not require that $x$ and $y$ be distinct, so we do not
require a distinct-variable restriction, although having one
would cause no harm other than making the scheme less general.

A mandatory variable-type hypothesis is one whose variable exists in a logical
hypothesis or the assertion.  A provable pre-statement
(defined below) may require
non-mandatory variable-type hypotheses that effectively introduce ``dummy''
variables for use in its proof.  Any number of dummy variables might
be required by a specific proof; indeed, it has been shown by H.\
Andr\'{e}ka\index{Andr{\'{e}}ka, H.} \cite{Nemeti} that there is no finite
upper bound to the number of dummy variables needed to prove an arbitrary
theorem in first-order logic (with equality) having a fixed number $n>2$ of
individual variables.  (See also the Comment on p.~\pageref{nodd}.)
For this reason we do not set a finite size bound on the collections $D$ and
$T$, although in an actual application (Metamath database) these will of
course be finite, increased to whatever size is necessary as more
proofs are added.
\end{quotation}}

{\footnotesize\begin{quotation}
{\em Relationship to Metamath.} A pre-statement of a formal system
corresponds to an extended frame in a Metamath database
(Section~\ref{frames}).  The collections $D$, $T$, and $H$ correspond
respectively to the \texttt{\$d}, \texttt{\$f}, and \texttt{\$e}
statement collections in an extended frame.  The expression $A$
corresponds to the \texttt{\$a} (or \texttt{\$p}) statement in an
extended frame.

A statement of a formal system corresponds to a frame in a Metamath
database.
\end{quotation}}

\subsection{Formal Systems}

A {\em formal system}\index{formal system} is a
triple $\langle \mbox{\em CN},\mbox{\em
VR},\Gamma\rangle$ where $\Gamma$ is a set of statements.  The members of
$\Gamma$ are called {\em axiomatic statements}.\index{axiomatic
statement!in a formal system}  Sometimes we will refer to a
formal system by just $\Gamma$ when $\mbox{\em CN}$ and $\mbox{\em VR}$ are
understood.

Given a formal system $\Gamma$, the {\em closure}\index{closure}\footnote{This
definition of closure incorporates a simplification due to
Josh Purinton.\index{Purinton, Josh}.} of a
pre-statement
$\langle D,T,H,A \rangle$ is the smallest set $C$ of expressions
such that:
%\begin{enumerate}
%  \item $T\cup H\subseteq C$; and
%  \item If for some axiomatic statement
%    $\langle D_M',T_M',H',A' \rangle \in \Gamma_A$, for
%    some $E \subseteq C$, some $F \subseteq C-T$ (where ``-'' denotes
%    set difference), and some substitution
%    $\sigma$ we have
%    \begin{enumerate}
%       \item $\sigma(T_M') = E$ (where, as above, the $M$ denotes the
%           mandatory variable-type hypotheses of $T^A$);
%       \item $\sigma(H') = F$;
%       \item for all $\{\alpha,\beta\}\in D^A$ and $\subseteq
%         {\cal V}(T_M')$, for all $\gamma\in {\cal V}(\sigma(\langle \alpha
%         \rangle))$, and for all $\delta\in  {\cal V}(\sigma(\langle \beta
%         \rangle))$, we have $\{\gamma, \delta\} \in D$;
%   \end{enumerate}
%   then $\sigma(A') \in C$.
%\end{enumerate}
\begin{list}{}{\itemsep 0.0pt}
  \item[1.] $T\cup H\subseteq C$; and
  \item[2.] If for some axiomatic statement
    $\langle D_M',T_M',H',A' \rangle \in
       \Gamma$ and for some substitution
    $\sigma$ we have
    \begin{enumerate}
       \item[a.] $\sigma(T_M' \cup H') \subseteq C$; and
       \item[b.] for all $\{\alpha,\beta\}\in D_M'$, for all $\gamma\in
         {\cal V}(\sigma(\langle \alpha
         \rangle))$, and for all $\delta\in  {\cal V}(\sigma(\langle \beta
         \rangle))$, we have $\{\gamma, \delta\} \in D$;
   \end{enumerate}
   then $\sigma(A') \in C$.
\end{list}
A pre-statement $\langle D,T,H,A
\rangle$ is {\em provable}\index{provable statement!in a formal
system} if $A\in C$ i.e.\ if its assertion belongs to its
closure.  A statement is {\em provable} if it is
the reduct of a provable pre-statement.
The {\em universe}\index{universe of a formal system}
of a formal system is
the collection of all of its provable statements.  Note that the
set of axiomatic statements $\Gamma$ in a formal system is a subset of its
universe.

{\footnotesize\begin{quotation}
{\em Comment.} The first condition in the definition of closure simply says
that the hypotheses of the pre-statement are in its closure.

Condition 2(a) says that a substitution exists that makes the
mandatory hypotheses of an axiomatic statement exactly match some members of
the closure.  This is what we explicitly demonstrate in a Metamath language
proof.

%Conditions 2(a) and 2(b) say that a substitution exists that makes the
%(mandatory) hypotheses of an axiomatic statement exactly match some members of
%the closure.  This is what we explicitly demonstrate with a Metamath language
%proof.
%
%The set of expressions $F$ in condition 2(b) excludes the variable-type
%hypotheses; this is done because non-mandatory variable-type hypotheses are
%effectively ``dropped'' as irrelevant whereas logical hypotheses must be
%retained to achieve a consistent logical system.

Condition 2(b) describes how distinct-variable restrictions in the axiomatic
statement must be met.  It means that after a substitution for two variables
that must be distinct, the resulting two expressions must either contain no
variables, or if they do, they may not have variables in common, and each pair
of any variables they do have, with one variable from each expression, must be
specified as distinct in the original statement.
\end{quotation}}

{\footnotesize\begin{quotation}
{\em Relationship to Metamath.} Axiomatic statements
 and provable statements in a formal
system correspond to the frames for \texttt{\$a} and \texttt{\$p} statements
respectively in a Metamath database.  The set of axiomatic statements is a
subset of the set of provable statements in a formal system, although in a
Metamath database a \texttt{\$a} statement is distinguished by not having a
proof.  A Metamath language proof for a \texttt{\$p} statement tells the computer
how to explicitly construct a series of members of the closure ultimately
leading to a demonstration that the assertion
being proved is in the closure.  The actual closure typically contains
an infinite number of expressions.  A formal system itself does not have
an explicit object called a ``proof'' but rather the existence of a proof
is implied indirectly by membership of an assertion in a provable
statement's closure.  We do this to make the formal system easier
to describe in the language of set theory.

We also note that once established as provable, a statement may be considered
to acquire the same status as an axiomatic statement, because if the set of
axiomatic statements is extended with a provable statement, the universe of
the formal system remains unchanged (provided that $\mbox{\em VR}$ is
infinite).
In practice, this means we can build a hierarchy of provable statements to
more efficiently establish additional provable statements.  This is
what we do in Metamath when we allow proofs to reference previous
\texttt{\$p} statements as well as previous \texttt{\$a} statements.
\end{quotation}}

\section{Examples of Formal Systems}

{\footnotesize\begin{quotation}
{\em Relationship to Metamath.} The examples in this section, except Example~2,
are for the most part exact equivalents of the development in the set
theory database \texttt{set.mm}.  You may want to compare Examples~1, 3, and 5
to Section~\ref{metaaxioms}, Example 4 to Sections~\ref{metadefprop} and
\ref{metadefpred}, and Example 6 to
Section~\ref{setdefinitions}.\label{exampleref}
\end{quotation}}

\subsection{Example~1---Propositional Calculus}\index{propositional calculus}

Classical propositional calculus can be described by the following formal
system.  We assume the set of variables is infinite.  Rather than denoting the
constants and variables by $c_0, c_1, \ldots$ and $v_0, v_1, \ldots$, for
readability we will instead use more conventional symbols, with the
understanding of course that they denote distinct primitive objects.
Also for readability we may omit commas between successive terms of a
sequence; thus $\langle \mbox{wff\ } \varphi\rangle$ denotes
$\langle \mbox{wff}, \varphi\rangle$.

Let
\begin{itemize}
  \item[] $\mbox{\em CN}=\{\mbox{wff}, \vdash, \to, \lnot, (,)\}$
  \item[] $\mbox{\em VR}=\{\varphi,\psi,\chi,\ldots\}$
  \item[] $T = \{\langle \mbox{wff\ } \varphi\rangle,
             \langle \mbox{wff\ } \psi\rangle,
             \langle \mbox{wff\ } \chi\rangle,\ldots\}$, i.e.\ those
             expressions of length 2 whose first member is $\mbox{\rm wff}$
             and whose second member belongs to $\mbox{\em VR}$.\footnote{For
convenience we let $T$ be an infinite set; the definition of a statement
permits this in principle.  Since a Metamath source file has a finite size, in
practice we must of course use appropriate finite subsets of this $T$,
specifically ones containing at least the mandatory variable-type
hypotheses.  Similarly, in the source file we introduce new variables as
required, with the understanding that a potentially infinite number of
them are available.}
\noindent Then $\Gamma$ consists of the axiomatic statements that
are the reducts of the following pre-statements:
    \begin{itemize}
      \item[] $\langle\varnothing,T,\varnothing,
               \langle \mbox{wff\ }(\varphi\to\psi)\rangle\rangle$
      \item[] $\langle\varnothing,T,\varnothing,
               \langle \mbox{wff\ }\lnot\varphi\rangle\rangle$
      \item[] $\langle\varnothing,T,\varnothing,
               \langle \vdash(\varphi\to(\psi\to\varphi))
               \rangle\rangle$
      \item[] $\langle\varnothing,T,
               \varnothing,
               \langle \vdash((\varphi\to(\psi\to\chi))\to
               ((\varphi\to\psi)\to(\varphi\to\chi)))
               \rangle\rangle$
      \item[] $\langle\varnothing,T,
               \varnothing,
               \langle \vdash((\lnot\varphi\to\lnot\psi)\to
               (\psi\to\varphi))\rangle\rangle$
      \item[] $\langle\varnothing,T,
               \{\langle\vdash(\varphi\to\psi)\rangle,
                 \langle\vdash\varphi\rangle\},
               \langle\vdash\psi\rangle\rangle$
    \end{itemize}
\end{itemize}

(For example, the reduct of $\langle\varnothing,T,\varnothing,
               \langle \mbox{wff\ }(\varphi\to\psi)\rangle\rangle$
is
\begin{itemize}
\item[] $\langle\varnothing,
\{\langle \mbox{wff\ } \varphi\rangle,
             \langle \mbox{wff\ } \psi\rangle\},
             \varnothing,
               \langle \mbox{wff\ }(\varphi\to\psi)\rangle\rangle$,
\end{itemize}
which is the first axiomatic statement.)

We call the members of $\mbox{\em VR}$ {\em wff variables} or (in the context
of first-order logic which we will describe shortly) {\em wff metavariables}.
Note that the symbols $\phi$, $\psi$, etc.\ denote actual specific members of
$\mbox{\em VR}$; they are not metavariables of our expository language (which
we denote with $\alpha$, $\beta$, etc.) but are instead (meta)constant symbols
(members of $\mbox{\em SM}$) from the point of view of our expository
language.  The equivalent system of propositional calculus described in
\cite{Tarski1965} also uses the symbols $\phi$, $\psi$, etc.\ to denote wff
metavariables, but in \cite{Tarski1965} unlike here those are metavariables of
the expository language and not primitive symbols of the formal system.

The first two statements define wffs: if $\varphi$ and $\psi$ are wffs, so is
$(\varphi \to \psi)$; if $\varphi$ is a wff, so is $\lnot\varphi$. The next
three are the axioms of propositional calculus: if $\varphi$ and $\psi$ are
wffs, then $\vdash (\varphi \to (\psi \to \varphi))$ is an (axiomatic)
theorem; etc. The
last is the rule of modus ponens: if $\varphi$ and $\psi$ are wffs, and
$\vdash (\varphi\to\psi)$ and $\vdash \varphi$ are theorems, then $\vdash
\psi$ is a theorem.

The correspondence to ordinary propositional calculus is as follows.  We
consider only provable statements of the form $\langle\varnothing,
T,\varnothing,A\rangle$ with $T$ defined as above.  The first term of the
assertion $A$ of any such statement is either ``wff'' or ``$\vdash$''.  A
statement for which the first term is ``wff'' is a {\em wff} of propositional
calculus, and one where the first term is ``$\vdash$'' is a {\em
theorem (scheme)} of propositional calculus.

The universe of this formal system also contains many other provable
statements.  Those with distinct-variable restrictions are irrelevant because
propositional calculus has no constraints on substitutions.  Those that have
logical hypotheses we call {\em inferences}\index{inference} when
the logical hypotheses are of the form
$\langle\vdash\rangle\frown w$ where $w$ is a wff (with the leading constant
term ``wff'' removed).  Inferences (other than the modus ponens rule) are not a
proper part of propositional calculus but are convenient to use when building a
hierarchy of provable statements.  A provable statement with a nonsense
hypothesis such as $\langle \to,\vdash,\lnot\rangle$, and this same expression
as its assertion, we consider irrelevant; no use can be made of it in
proving theorems, since there is no way to eliminate the nonsense hypothesis.

{\footnotesize\begin{quotation}
{\em Comment.} Our use of parentheses in the definition of a wff illustrates
how axiomatic statements should be carefully stated in a way that
ties in unambiguously with the substitutions allowed by the formal system.
There are many ways we could have defined wffs---for example, Polish
prefix notation would have allowed us to omit parentheses entirely, at
the expense of readability---but we must define them in a way that is
unambiguous.  For example, if we had omitted parentheses from the
definition of $(\varphi\to \psi)$, the wff $\lnot\varphi\to \psi$ could
be interpreted as either $\lnot(\varphi\to\psi)$ or $(\lnot\varphi\to\psi)$
and would have allowed us to prove nonsense.  Note that there is no
concept of operator binding precedence built into our formal system.
\end{quotation}}

\begin{sloppy}
\subsection{Example~2---Predicate Calculus with Equality}\index{predicate
calculus}
\end{sloppy}

Here we extend Example~1 to include predicate calculus with equality,
illustrating the use of distinct-variable restrictions.  This system is the
same as Tarski's system $\mathfrak{S}_2$ in \cite{Tarski1965} (except that the
axioms of propositional calculus are different but equivalent, and a redundant
axiom is omitted).  We extend $\mbox{\em CN}$ with the constants
$\{\mbox{var},\forall,=\}$.  We extend $\mbox{\em VR}$ with an infinite set of
{\em individual metavariables}\index{individual
metavariable} $\{x,y,z,\ldots\}$ and denote this subset
$\mbox{\em Vr}$.

We also join to $\mbox{\em CN}$ a possibly infinite set $\mbox{\em Pr}$ of {\em
predicates} $\{R,S,\ldots\}$.  We associate with $\mbox{\em Pr}$ a function
$\mbox{rnk}$ from $\mbox{\em Pr}$ to $\omega$, and for $\alpha\in \mbox{\em
Pr}$ we call $\mbox{rnk}(\alpha)$ the {\em rank} of the predicate $\alpha$,
which is simply the number of ``arguments'' that the predicate has.  (Most
applications of predicate calculus will have a finite number of predicates;
for example, set theory has the single two-argument or binary predicate $\in$,
which is usually written with its arguments surrounding the predicate symbol
rather than with the prefix notation we will use for the general case.)  As a
device to facilitate our discussion, we will let $\mbox{\em Vs}$ be any fixed
one-to-one function from $\omega$ to $\mbox{\em Vr}$; thus $\mbox{\em Vs}$ is
any simple infinite sequence of individual metavariables with no repeating
terms.

In this example we will not include the function symbols that are often part of
formalizations of predicate calculus.  Using metalogical arguments that are
beyond the scope of our discussion, it can be shown that our formalization is
equivalent when functions are introduced via appropriate definitions.

We extend the set $T$ defined in Example~1 with the expressions
$\{\langle \mbox{var\ } x\rangle,$ $ \langle \mbox{var\ } y\rangle, \langle
\mbox{var\ } z\rangle,\ldots\}$.  We extend the $\Gamma$ above
with the axiomatic statements that are the reducts of the following
pre-statements:
\begin{list}{}{\itemsep 0.0pt}
      \item[] $\langle\varnothing,T,\varnothing,
               \langle \mbox{wff\ }\forall x\,\varphi\rangle\rangle$
      \item[] $\langle\varnothing,T,\varnothing,
               \langle \mbox{wff\ }x=y\rangle\rangle$
      \item[] $\langle\varnothing,T,
               \{\langle\vdash\varphi\rangle\},
               \langle\vdash\forall x\,\varphi\rangle\rangle$
      \item[] $\langle\varnothing,T,\varnothing,
               \langle \vdash((\forall x(\varphi\to\psi)
                  \to(\forall x\,\varphi\to\forall x\,\psi))
               \rangle\rangle$
      \item[] $\langle\{\{x,\varphi\}\},T,\varnothing,
               \langle \vdash(\varphi\to\forall x\,\varphi)
               \rangle\rangle$
      \item[] $\langle\{\{x,y\}\},T,\varnothing,
               \langle \vdash\lnot\forall x\lnot x=y
               \rangle\rangle$
      \item[] $\langle\varnothing,T,\varnothing,
               \langle \vdash(x=z
                  \to(x=y\to z=y))
               \rangle\rangle$
      \item[] $\langle\varnothing,T,\varnothing,
               \langle \vdash(y=z
                  \to(x=y\to x=z))
               \rangle\rangle$
\end{list}
These are the axioms not involving predicate symbols. The first two statements
extend the definition of a wff.  The third is the rule of generalization.  The
fifth states, in effect, ``For a wff $\varphi$ and variable $x$,
$\vdash(\varphi\to\forall x\,\varphi)$, provided that $x$ does not occur in
$\varphi$.''  The sixth states ``For variables $x$ and $y$,
$\vdash\lnot\forall x\lnot x = y$, provided that $x$ and $y$ are distinct.''
(This proviso is not necessary but was included by Tarski to
weaken the axiom and still show that the system is logically complete.)

Finally, for each predicate symbol $\alpha\in \mbox{\em Pr}$, we add to
$\Gamma$ an axiomatic statement, extending the definition of wff,
that is the reduct of the following pre-statement:
\begin{displaymath}
    \langle\varnothing,T,\varnothing,
            \langle \mbox{wff},\alpha\rangle\
            \frown \mbox{\em Vs}\restriction\mbox{rnk}(\alpha)\rangle
\end{displaymath}
and for each $\alpha\in \mbox{\em Pr}$ and each $n < \mbox{rnk}(\alpha)$
we add to $\Gamma$ an equality axiom that is the reduct of the
following pre-statement:
\begin{eqnarray*}
    \lefteqn{\langle\varnothing,T,\varnothing,
            \langle
      \vdash,(,\mbox{\em Vs}_n,=,\mbox{\em Vs}_{\mbox{rnk}(\alpha)},\to,
            (,\alpha\rangle\frown \mbox{\em Vs}\restriction\mbox{rnk}(\alpha)} \\
  & & \frown
            \langle\to,\alpha\rangle\frown \mbox{\em Vs}\restriction n\frown
            \langle \mbox{\em Vs}_{\mbox{rnk}(\alpha)}\rangle \\
 & & \frown
            \mbox{\em Vs}\restriction(\mbox{rnk}(\alpha)\setminus(n+1))\frown
            \langle),)\rangle\rangle
\end{eqnarray*}
where $\restriction$ denotes function domain restriction and $\setminus$
denotes set difference.  Recall that a subscript on $\mbox{\em Vs}$
denotes one of its terms.  (In the above two axiom sets commas are placed
between successive terms of sequences to prevent ambiguity, and if you examine
them with care you will be able to distinguish those parentheses that denote
constant symbols from those of our expository language that delimit function
arguments.  Although it might have been better to use boldface for our
primitive symbols, unfortunately boldface was not available for all characters
on the \LaTeX\ system used to typeset this text.)  These seemingly forbidding
axioms can be understood by analogy to concatenation of substrings in a
computer language.  They are actually relatively simple for each specific case
and will become clearer by looking at the special case of a binary predicate
$\alpha = R$ where $\mbox{rnk}(R)=2$.  Letting $\mbox{\em Vs}$ be the sequence
$\langle x,y,z,\ldots\rangle$, the axioms we would add to $\Gamma$ for this
case would be the wff extension and two equality axioms that are the
reducts of the pre-statements:
\begin{list}{}{\itemsep 0.0pt}
      \item[] $\langle\varnothing,T,\varnothing,
               \langle \mbox{wff\ }R x y\rangle\rangle$
      \item[] $\langle\varnothing,T,\varnothing,
               \langle \vdash(x=z
                  \to(R x y \to R z y))
               \rangle\rangle$
      \item[] $\langle\varnothing,T,\varnothing,
               \langle \vdash(y=z
                  \to(R x y \to R x z))
               \rangle\rangle$
\end{list}
Study these carefully to see how the general axioms above evaluate to
them.  In practice, typically only a few special cases such as this would be
needed, and in any case the Metamath language will only permit us to describe
a finite number of predicates, as opposed to the infinite number permitted by
the formal system.  (If an infinite number should be needed for some reason,
we could not define the formal system directly in the Metamath language but
could instead define it metalogically under set theory as we
do in this appendix, and only the underlying set theory, with its single
binary predicate, would be defined directly in the Metamath language.)


{\footnotesize\begin{quotation}
{\em Comment.}  As we noted earlier, the specific variables denoted by the
symbols $x,y,z,\ldots\in \mbox{\em Vr}\subseteq \mbox{\em VR}\subseteq
\mbox{\em SM}$ in Example~2 are not the actual variables of ordinary predicate
calculus but should be thought of as metavariables ranging over them.  For
example, a distinct-variable restriction would be meaningless for actual
variables of ordinary predicate calculus since two different actual variables
are by definition distinct.  And when we talk about an arbitrary
representative $\alpha\in \mbox{\em Vr}$, $\alpha$ is a metavariable (in our
expository language) that ranges over metavariables (which are primitives of
our formal system) each of which ranges over the actual individual variables
of predicate calculus (which are never mentioned in our formal system).

The constant called ``var'' above is called \texttt{setvar} in the
\texttt{set.mm} database file, but it means the same thing.  I felt
that ``var'' is a more meaningful name in the context of predicate
calculus, whose use is not limited to set theory.  For consistency we
stick with the name ``var'' throughout this Appendix, even after set
theory is introduced.
\end{quotation}}

\subsection{Free Variables and Proper Substitution}\index{free variable}
\index{proper substitution}\index{substitution!proper}

Typical representations of mathematical axioms use concepts such
as ``free variable,'' ``bound variable,'' and ``proper substitution''
as primitive notions.
A free variable is a variable that
is not a parameter of any container expression.
A bound variable is the opposite of a free variable; it is a
a variable that has been bound in a container expression.
For example, in the expression $\forall x \varphi$ (for all $x$, $\varphi$
is true), the variable $x$
is bound within the for-all ($\forall$) expression.
It is possible to change one variable to another, and that process is called
``proper substitution.''
In most books, proper substitution has a somewhat complicated recursive
definition with multiple cases based on the occurrences of free and
bound variables.
You may consult
\cite[ch.\ 3--4]{Hamilton}\index{Hamilton, Alan G.} (as well as
many other texts) for more formal details about these terms.

Using these concepts as \texttt{primitives} creates complications
for computer implementations.

In the system of Example~2, there are no primitive notions of free variable
and proper substitution.  Tarski \cite{Tarski1965} shows that this system is
logically equivalent to the more typical textbook systems that do have these
primitive notions, if we introduce these notions with appropriate definitions
and metalogic.  We could also define axioms for such systems directly,
although the recursive definitions of free variable and proper substitution
would be messy and awkward to work with.  Instead, we mention two devices that
can be used in practice to mimic these notions.  (1) Instead of introducing
special notation to express (as a logical hypothesis) ``where $x$ is not free
in $\varphi$'' we can use the logical hypothesis $\vdash(\varphi\to\forall
x\,\varphi)$.\label{effectivelybound}\index{effectively
not free}\footnote{This is a slightly weaker requirement than ``where $x$ is
not free in $\varphi$.''  If we let $\varphi$ be $x=x$, we have the theorem
$(x=x\to\forall x\,x=x)$ which satisfies the hypothesis, even though $x$ is
free in $x=x$ .  In a case like this we say that $x$ is {\em effectively not
free}\index{effectively not free} in $x=x$, since $x=x$ is logically
equivalent to $\forall x\,x=x$ in which $x$ is bound.} (2) It can be shown
that the wff $((x=y\to\varphi)\wedge\exists x(x=y\wedge\varphi))$ (with the
usual definitions of $\wedge$ and $\exists$; see Example~4 below) is logically
equivalent to ``the wff that results from proper substitution of $y$ for $x$
in $\varphi$.''  This works whether or not $x$ and $y$ are distinct.

\subsection{Metalogical Completeness}\index{metalogical completeness}

In the system of Example~2, the
following are provable pre-statements (and their reducts are
provable statements):
\begin{eqnarray*}
      & \langle\{\{x,y\}\},T,\varnothing,
               \langle \vdash\lnot\forall x\lnot x=y
               \rangle\rangle & \\
     &  \langle\varnothing,T,\varnothing,
               \langle \vdash\lnot\forall x\lnot x=x
               \rangle\rangle &
\end{eqnarray*}
whereas the following pre-statement is not to my knowledge provable (but
in any case we will pretend it's not for sake of illustration):
\begin{eqnarray*}
     &  \langle\varnothing,T,\varnothing,
               \langle \vdash\lnot\forall x\lnot x=y
               \rangle\rangle &
\end{eqnarray*}
In other words, we can prove ``$\lnot\forall x\lnot x=y$ where $x$ and $y$ are
distinct'' and separately prove ``$\lnot\forall x\lnot x=x$'', but we can't
prove the combined general case ``$\lnot\forall x\lnot x=y$'' that has no
proviso.  Now this does not compromise logical completeness, because the
variables are really metavariables and the two provable cases together cover
all possible cases.  The third case can be considered a metatheorem whose
direct proof, using the system of Example~2, lies outside the capability of the
formal system.

Also, in the system of Example~2 the following pre-statement is not to my
knowledge provable (again, a conjecture that we will pretend to be the case):
\begin{eqnarray*}
     & \langle\varnothing,T,\varnothing,
               \langle \vdash(\forall x\, \varphi\to\varphi)
               \rangle\rangle &
\end{eqnarray*}
Instead, we can only prove specific cases of $\varphi$ involving individual
metavariables, and by induction on formula length, prove as a metatheorem
outside of our formal system the general statement above.  The details of this
proof are found in \cite{Kalish}.

There does, however, exist a system of predicate calculus in which all such
``simple metatheorems'' as those above can be proved directly, and we present
it in Example~3. A {\em simple metatheorem}\index{simple metatheorem}
is any statement of the formal
system of Example~2 where all distinct variable restrictions consist of either
two individual metavariables or an individual metavariable and a wff
metavariable, and which is provable by combining cases outside the system as
above.  A system is {\em metalogically complete}\index{metalogical
completeness} if all of its simple
metatheorems are (directly) provable statements. The precise definition of
``simple metatheorem'' and the proof of the ``metalogical completeness'' of
Example~3 is found in Remark 9.6 and Theorem 9.7 of \cite{Megill}.\index{Megill,
Norman}

\begin{sloppy}
\subsection{Example~3---Metalogically Complete Predicate
Calculus with
Equality}
\end{sloppy}

For simplicity we will assume there is one binary predicate $R$;
this system suffices for set theory, where the $R$ is of course the $\in$
predicate.  We label the axioms as they appear in \cite{Megill}.  This
system is logically equivalent to that of Example~2 (when the latter is
restricted to this single binary predicate) but is also metalogically
complete.\index{metalogical completeness}

Let
\begin{itemize}
  \item[] $\mbox{\em CN}=\{\mbox{wff}, \mbox{var}, \vdash, \to, \lnot, (,),\forall,=,R\}$.
  \item[] $\mbox{\em VR}=\{\varphi,\psi,\chi,\ldots\}\cup\{x,y,z,\ldots\}$.
  \item[] $T = \{\langle \mbox{wff\ } \varphi\rangle,
             \langle \mbox{wff\ } \psi\rangle,
             \langle \mbox{wff\ } \chi\rangle,\ldots\}\cup
       \{\langle \mbox{var\ } x\rangle, \langle \mbox{var\ } y\rangle, \langle
       \mbox{var\ }z\rangle,\ldots\}$.

\noindent Then
  $\Gamma$ consists of the reducts of the following pre-statements:
    \begin{itemize}
      \item[] $\langle\varnothing,T,\varnothing,
               \langle \mbox{wff\ }(\varphi\to\psi)\rangle\rangle$
      \item[] $\langle\varnothing,T,\varnothing,
               \langle \mbox{wff\ }\lnot\varphi\rangle\rangle$
      \item[] $\langle\varnothing,T,\varnothing,
               \langle \mbox{wff\ }\forall x\,\varphi\rangle\rangle$
      \item[] $\langle\varnothing,T,\varnothing,
               \langle \mbox{wff\ }x=y\rangle\rangle$
      \item[] $\langle\varnothing,T,\varnothing,
               \langle \mbox{wff\ }Rxy\rangle\rangle$
      \item[(C1$'$)] $\langle\varnothing,T,\varnothing,
               \langle \vdash(\varphi\to(\psi\to\varphi))
               \rangle\rangle$
      \item[(C2$'$)] $\langle\varnothing,T,
               \varnothing,
               \langle \vdash((\varphi\to(\psi\to\chi))\to
               ((\varphi\to\psi)\to(\varphi\to\chi)))
               \rangle\rangle$
      \item[(C3$'$)] $\langle\varnothing,T,
               \varnothing,
               \langle \vdash((\lnot\varphi\to\lnot\psi)\to
               (\psi\to\varphi))\rangle\rangle$
      \item[(C4$'$)] $\langle\varnothing,T,
               \varnothing,
               \langle \vdash(\forall x(\forall x\,\varphi\to\psi)\to
                 (\forall x\,\varphi\to\forall x\,\psi))\rangle\rangle$
      \item[(C5$'$)] $\langle\varnothing,T,
               \varnothing,
               \langle \vdash(\forall x\,\varphi\to\varphi)\rangle\rangle$
      \item[(C6$'$)] $\langle\varnothing,T,
               \varnothing,
               \langle \vdash(\forall x\forall y\,\varphi\to
                 \forall y\forall x\,\varphi)\rangle\rangle$
      \item[(C7$'$)] $\langle\varnothing,T,
               \varnothing,
               \langle \vdash(\lnot\varphi\to\forall x\lnot\forall x\,\varphi
                 )\rangle\rangle$
      \item[(C8$'$)] $\langle\varnothing,T,
               \varnothing,
               \langle \vdash(x=y\to(x=z\to y=z))\rangle\rangle$
      \item[(C9$'$)] $\langle\varnothing,T,
               \varnothing,
               \langle \vdash(\lnot\forall x\, x=y\to(\lnot\forall x\, x=z\to
                 (y=z\to\forall x\, y=z)))\rangle\rangle$
      \item[(C10$'$)] $\langle\varnothing,T,
               \varnothing,
               \langle \vdash(\forall x(x=y\to\forall x\,\varphi)\to
                 \varphi))\rangle\rangle$
      \item[(C11$'$)] $\langle\varnothing,T,
               \varnothing,
               \langle \vdash(\forall x\, x=y\to(\forall x\,\varphi
               \to\forall y\,\varphi))\rangle\rangle$
      \item[(C12$'$)] $\langle\varnothing,T,
               \varnothing,
               \langle \vdash(x=y\to(Rxz\to Ryz))\rangle\rangle$
      \item[(C13$'$)] $\langle\varnothing,T,
               \varnothing,
               \langle \vdash(x=y\to(Rzx\to Rzy))\rangle\rangle$
      \item[(C15$'$)] $\langle\varnothing,T,
               \varnothing,
               \langle \vdash(\lnot\forall x\, x=y\to(x=y\to(\varphi
                 \to\forall x(x=y\to\varphi))))\rangle\rangle$
      \item[(C16$'$)] $\langle\{\{x,y\}\},T,
               \varnothing,
               \langle \vdash(\forall x\, x=y\to(\varphi\to\forall x\,\varphi)
                 )\rangle\rangle$
      \item[(C5)] $\langle\{\{x,\varphi\}\},T,\varnothing,
               \langle \vdash(\varphi\to\forall x\,\varphi)
               \rangle\rangle$
      \item[(MP)] $\langle\varnothing,T,
               \{\langle\vdash(\varphi\to\psi)\rangle,
                 \langle\vdash\varphi\rangle\},
               \langle\vdash\psi\rangle\rangle$
      \item[(Gen)] $\langle\varnothing,T,
               \{\langle\vdash\varphi\rangle\},
               \langle\vdash\forall x\,\varphi\rangle\rangle$
    \end{itemize}
\end{itemize}

While it is known that these axioms are ``metalogically complete,'' it is
not known whether they are independent (i.e.\ none is
redundant) in the metalogical sense; specifically, whether any axiom (possibly
with additional non-mandatory distinct-variable restrictions, for use with any
dummy variables in its proof) is provable from the others.  Note that
metalogical independence is a weaker requirement than independence in the
usual logical sense.  Not all of the above axioms are logically independent:
for example, C9$'$ can be proved as a metatheorem from the others, outside the
formal system, by combining the possible cases of distinct variables.

\subsection{Example~4---Adding Definitions}\index{definition}
There are several ways to add definitions to a formal system.  Probably the
most proper way is to consider definitions not as part of the formal system at
all but rather as abbreviations that are part of the expository metalogic
outside the formal system.  For convenience, though, we may use the formal
system itself to incorporate definitions, adding them as axiomatic extensions
to the system.  This could be done by adding a constant representing the
concept ``is defined as'' along with axioms for it. But there is a nicer way,
at least in this writer's opinion, that introduces definitions as direct
extensions to the language rather than as extralogical primitive notions.  We
introduce additional logical connectives and provide axioms for them.  For
systems of logic such as Examples 1 through 3, the additional axioms must be
conservative in the sense that no wff of the original system that was not a
theorem (when the initial term ``wff'' is replaced by ``$\vdash$'' of course)
becomes a theorem of the extended system.  In this example we extend Example~3
(or 2) with standard abbreviations of logic.

We extend $\mbox{\em CN}$ of Example~3 with new constants $\{\leftrightarrow,
\wedge,\vee,\exists\}$, corresponding to logical equivalence,\index{logical
equivalence ($\leftrightarrow$)}\index{biconditional ($\leftrightarrow$)}
conjunction,\index{conjunction ($\wedge$)} disjunction,\index{disjunction
($\vee$)} and the existential quantifier.\index{existential quantifier
($\exists$)}  We extend $\Gamma$ with the axiomatic statements that are
the reducts of the following pre-statements:
\begin{list}{}{\itemsep 0.0pt}
      \item[] $\langle\varnothing,T,\varnothing,
               \langle \mbox{wff\ }(\varphi\leftrightarrow\psi)\rangle\rangle$
      \item[] $\langle\varnothing,T,\varnothing,
               \langle \mbox{wff\ }(\varphi\vee\psi)\rangle\rangle$
      \item[] $\langle\varnothing,T,\varnothing,
               \langle \mbox{wff\ }(\varphi\wedge\psi)\rangle\rangle$
      \item[] $\langle\varnothing,T,\varnothing,
               \langle \mbox{wff\ }\exists x\, \varphi\rangle\rangle$
  \item[] $\langle\varnothing,T,\varnothing,
     \langle\vdash ( ( \varphi \leftrightarrow \psi ) \to
     ( \varphi \to \psi ) )\rangle\rangle$
  \item[] $\langle\varnothing,T,\varnothing,
     \langle\vdash ((\varphi\leftrightarrow\psi)\to
    (\psi\to\varphi))\rangle\rangle$
  \item[] $\langle\varnothing,T,\varnothing,
     \langle\vdash ((\varphi\to\psi)\to(
     (\psi\to\varphi)\to(\varphi
     \leftrightarrow\psi)))\rangle\rangle$
  \item[] $\langle\varnothing,T,\varnothing,
     \langle\vdash (( \varphi \wedge \psi ) \leftrightarrow\neg ( \varphi
     \to \neg \psi )) \rangle\rangle$
  \item[] $\langle\varnothing,T,\varnothing,
     \langle\vdash (( \varphi \vee \psi ) \leftrightarrow (\neg \varphi
     \to \psi )) \rangle\rangle$
  \item[] $\langle\varnothing,T,\varnothing,
     \langle\vdash (\exists x \,\varphi\leftrightarrow
     \lnot \forall x \lnot \varphi)\rangle\rangle$
\end{list}
The first three logical axioms (statements containing ``$\vdash$'') introduce
and effectively define logical equivalence, ``$\leftrightarrow$''.  The last
three use ``$\leftrightarrow$'' to effectively mean ``is defined as.''

\subsection{Example~5---ZFC Set Theory}\index{ZFC set theory}

Here we add to the system of Example~4 the axioms of Zermelo--Fraenkel set
theory with Choice.  For convenience we make use of the
definitions in Example~4.

In the $\mbox{\em CN}$ of Example~4 (which extends Example~3), we replace the symbol $R$
with the symbol $\in$.
More explicitly, we remove from $\Gamma$ of Example~4 the three
axiomatic statements containing $R$ and replace them with the
reducts of the following:
\begin{list}{}{\itemsep 0.0pt}
      \item[] $\langle\varnothing,T,\varnothing,
               \langle \mbox{wff\ }x\in y\rangle\rangle$
      \item[] $\langle\varnothing,T,
               \varnothing,
               \langle \vdash(x=y\to(x\in z\to y\in z))\rangle\rangle$
      \item[] $\langle\varnothing,T,
               \varnothing,
               \langle \vdash(x=y\to(z\in x\to z\in y))\rangle\rangle$
\end{list}
Letting $D=\{\{\alpha,\beta\}\in \mbox{\em DV}\,|\alpha,\beta\in \mbox{\em
Vr}\}$ (in other words all individual variables must be distinct), we extend
$\Gamma$ with the ZFC axioms, called
\index{Axiom of Extensionality}
\index{Axiom of Replacement}
\index{Axiom of Union}
\index{Axiom of Power Sets}
\index{Axiom of Regularity}
\index{Axiom of Infinity}
\index{Axiom of Choice}
Extensionality, Replacement, Union, Power
Set, Regularity, Infinity, and Choice, that are the reducts of:
\begin{list}{}{\itemsep 0.0pt}
      \item[Ext] $\langle D,T,
               \varnothing,
               \langle\vdash (\forall x(x\in y\leftrightarrow x \in z)\to y
               =z) \rangle\rangle$
      \item[Rep] $\langle D,T,
               \varnothing,
               \langle\vdash\exists x ( \exists y \forall z (\varphi \to z = y
                        ) \to
                        \forall z ( z \in x \leftrightarrow \exists x ( x \in
                        y \wedge \forall y\,\varphi ) ) )\rangle\rangle$
      \item[Un] $\langle D,T,
               \varnothing,
               \langle\vdash \exists x \forall y ( \exists x ( y \in x \wedge
               x \in z ) \to y \in x ) \rangle\rangle$
      \item[Pow] $\langle D,T,
               \varnothing,
               \langle\vdash \exists x \forall y ( \forall x ( x \in y \to x
               \in z ) \to y \in x ) \rangle\rangle$
      \item[Reg] $\langle D,T,
               \varnothing,
               \langle\vdash (  x \in y \to
                 \exists x ( x \in y \wedge \forall z ( z \in x \to \lnot z
                \in y ) ) ) \rangle\rangle$
      \item[Inf] $\langle D,T,
               \varnothing,
               \langle\vdash \exists x(y\in x\wedge\forall y(y\in
               x\to
               \exists z(y \in z\wedge z\in x))) \rangle\rangle$
      \item[AC] $\langle D,T,
               \varnothing,
               \langle\vdash \exists x \forall y \forall z ( ( y \in z
               \wedge z \in w ) \to \exists w \forall y ( \exists w
              ( ( y \in z \wedge z \in w ) \wedge ( y \in w \wedge w \in x
              ) ) \leftrightarrow y = w ) ) \rangle\rangle$
\end{list}

\subsection{Example~6---Class Notation in Set Theory}\label{class}

A powerful device that makes set theory easier (and that we have
been using all along in our informal expository language) is {\em class
abstraction notation}.\index{class abstraction}\index{abstraction class}  The
definitions we introduce are rigorously justified
as conservative by Takeuti and Zaring \cite{Takeuti}\index{Takeuti, Gaisi} or
Quine \cite{Quine}\index{Quine, Willard Van Orman}.  The key idea is to
introduce the notation $\{x|\mbox{---}\}$ which means ``the class of all $x$
such that ---'' for abstraction classes and introduce (meta)variables that
range over them.  An abstraction class may or may not be a set, depending on
whether it exists (as a set).  A class that does not exist is
called a {\em proper class}.\index{proper class}\index{class!proper}

To illustrate the use of abstraction classes we will provide some examples
of definitions that make use of them:  the empty set, class union, and
unordered pair.  Many other such definitions can be found in the
Metamath set theory database,
\texttt{set.mm}.\index{set theory database (\texttt{set.mm})}

% We intentionally break up the sequence of math symbols here
% because otherwise the overlong line goes beyond the page in narrow mode.
We extend $\mbox{\em CN}$ of Example~5 with new symbols $\{$
$\mbox{class},$ $\{,$ $|,$ $\},$ $\varnothing,$ $\cup,$ $,$ $\}$
where the inner braces and last comma are
constant symbols. (As before,
our dual use of some mathematical symbols for both our expository
language and as primitives of the formal system should be clear from context.)

We extend $\mbox{\em VR}$ of Example~5 with a set of {\em class
variables}\index{class variable}
$\{A,B,C,\ldots\}$. We extend the $T$ of Example~5 with $\{\langle
\mbox{class\ } A\rangle, \langle \mbox{class\ }B\rangle, \langle \mbox{class\ }
C\rangle,\ldots\}$.

To
introduce our definitions,
we add to $\Gamma$ of Example~5 the axiomatic statements
that are the reducts of the following pre-statements:
\begin{list}{}{\itemsep 0.0pt}
      \item[] $\langle\varnothing,T,\varnothing,
               \langle \mbox{class\ }x\rangle\rangle$
      \item[] $\langle\varnothing,T,\varnothing,
               \langle \mbox{class\ }\{x|\varphi\}\rangle\rangle$
      \item[] $\langle\varnothing,T,\varnothing,
               \langle \mbox{wff\ }A=B\rangle\rangle$
      \item[] $\langle\varnothing,T,\varnothing,
               \langle \mbox{wff\ }A\in B\rangle\rangle$
      \item[Ab] $\langle\varnothing,T,\varnothing,
               \langle \vdash ( y \in \{ x |\varphi\} \leftrightarrow
                  ( ( x = y \to\varphi) \wedge \exists x ( x = y
                  \wedge\varphi) ))
               \rangle\rangle$
      \item[Eq] $\langle\{\{x,A\},\{x,B\}\},T,\varnothing,
               \langle \vdash ( A = B \leftrightarrow
               \forall x ( x \in A \leftrightarrow x \in B ) )
               \rangle\rangle$
      \item[El] $\langle\{\{x,A\},\{x,B\}\},T,\varnothing,
               \langle \vdash ( A \in B \leftrightarrow \exists x
               ( x = A \wedge x \in B ) )
               \rangle\rangle$
\end{list}
Here we say that an individual variable is a class; $\{x|\varphi\}$ is a
class; and we extend the definition of a wff to include class equality and
membership.  Axiom Ab defines membership of a variable in a class abstraction;
the right-hand side can be read as ``the wff that results from proper
substitution of $y$ for $x$ in $\varphi$.''\footnote{Note that this definition
makes unnecessary the introduction of a separate notation similar to
$\varphi(x|y)$ for proper substitution, although we may choose to do so to be
conventional.  Incidentally, $\varphi(x|y)$ as it stands would be ambiguous in
the formal systems of our examples, since we wouldn't know whether
$\lnot\varphi(x|y)$ meant $\lnot(\varphi(x|y))$ or $(\lnot\varphi)(x|y)$.
Instead, we would have to use an unambiguous variant such as $(\varphi\,
x|y)$.}  Axioms Eq and El extend the meaning of the existing equality and
membership connectives.  This is potentially dangerous and requires careful
justification.  For example, from Eq we can derive the Axiom of Extensionality
with predicate logic alone; thus in principle we should include the Axiom of
Extensionality as a logical hypothesis.  However we do not bother to do this
since we have already presupposed that axiom earlier. The distinct variable
restrictions should be read ``where $x$ does not occur in $A$ or $B$.''  We
typically do this when the right-hand side of a definition involves an
individual variable not in the expression being defined; it is done so that
the right-hand side remains independent of the particular ``dummy'' variable
we use.

We continue to add to $\Gamma$ the following definitions
(i.e. the reducts of the following pre-statements) for empty
set,\index{empty set} class union,\index{union} and unordered
pair.\index{unordered pair}  They should be self-explanatory.  Analogous to our
use of ``$\leftrightarrow$'' to define new wffs in Example~4, we use ``$=$''
to define new abstraction terms, and both may be read informally as ``is
defined as'' in this context.
\begin{list}{}{\itemsep 0.0pt}
      \item[] $\langle\varnothing,T,\varnothing,
               \langle \mbox{class\ }\varnothing\rangle\rangle$
      \item[] $\langle\varnothing,T,\varnothing,
               \langle \vdash \varnothing = \{ x | \lnot x = x \}
               \rangle\rangle$
      \item[] $\langle\varnothing,T,\varnothing,
               \langle \mbox{class\ }(A\cup B)\rangle\rangle$
      \item[] $\langle\{\{x,A\},\{x,B\}\},T,\varnothing,
               \langle \vdash ( A \cup B ) = \{ x | ( x \in A \vee x \in B ) \}
               \rangle\rangle$
      \item[] $\langle\varnothing,T,\varnothing,
               \langle \mbox{class\ }\{A,B\}\rangle\rangle$
      \item[] $\langle\{\{x,A\},\{x,B\}\},T,\varnothing,
               \langle \vdash \{ A , B \} = \{ x | ( x = A \vee x = B ) \}
               \rangle\rangle$
\end{list}

\section{Metamath as a Formal System}\label{theorymm}

This section presupposes a familiarity with the Metamath computer language.

Our theory describes formal systems and their universes.  The Metamath
language provides a way of representing these set-theoretical objects to
a computer.  A Metamath database, being a finite set of {\sc ascii}
characters, can usually describe only a subset of a formal system and
its universe, which are typically infinite.  However the database can
contain as large a finite subset of the formal system and its universe
as we wish.  (Of course a Metamath set theory database can, in
principle, indirectly describe an entire infinite formal system by
formalizing the expository language in this Appendix.)

For purpose of our discussion, we assume the Metamath database
is in the simple form described on p.~\pageref{framelist},
consisting of all constant and variable declarations at the beginning,
followed by a sequence of extended frames each
delimited by \texttt{\$\char`\{} and \texttt{\$\char`\}}.  Any Metamath database can
be converted to this form, as described on p.~\pageref{frameconvert}.

The math symbol tokens of a Metamath source file, which are declared
with \texttt{\$c} and \texttt{\$v} statements, are names we assign to
representatives of $\mbox{\em CN}$ and $\mbox{\em VR}$.  For
definiteness we could assume that the first math symbol declared as a
variable corresponds to $v_0$, the second to $v_1$, etc., although the
exact correspondence we choose is not important.

In the Metamath language, each \texttt{\$d}, \texttt{\$f}, and
 \texttt{\$e} source
statement in an extended frame (Section~\ref{frames})
corresponds respectively to a member of the
collections $D$, $T$, and $H$ in a formal system statement $\langle
D_M,T_M,H,A\rangle$.  The math symbol strings following these Metamath keywords
correspond to a variable pair (in the case of \texttt{\$d}) or an expression (for
the other two keywords). The math symbol string following a \texttt{\$a} source
statement corresponds to expression $A$ in an axiomatic statement of the
formal system; the one following a \texttt{\$p} source statement corresponds to
$A$ in a provable statement that is not axiomatic.  In other words, each
extended frame in a Metamath database corresponds to
a pre-statement of the formal system, and a frame corresponds to
a statement of the formal system.  (Don't confuse the two meanings of
``statement'' here.  A statement of the formal system corresponds to the
several statements in a Metamath database that may constitute a
frame.)

In order for the computer to verify that a formal system statement is
provable, each \texttt{\$p} source statement is accompanied by a proof.
However, the proof does not correspond to anything in the formal system
but is simply a way of communicating to the computer the information
needed for its verification.  The proof tells the computer {\em how to
construct} specific members of closure of the formal system
pre-statement corresponding to the extended frame of the \texttt{\$p}
statement.  The final result of the construction is the member of the
closure that matches the \texttt{\$p} statement.  The abstract formal
system, on the other hand, is concerned only with the {\em existence} of
members of the closure.

As mentioned on p.~\pageref{exampleref}, Examples 1 and 3--6 in the
previous Section parallel the development of logic and set theory in the
Metamath database
\texttt{set.mm}.\index{set theory database (\texttt{set.mm})} You may
find it instructive to compare them.


\chapter{The MIU System}
\label{MIU}
\index{formal system}
\index{MIU-system}

The following is a listing of the file \texttt{miu.mm}.  It is self-explanatory.

%%%%%%%%%%%%%%%%%%%%%%%%%%%%%%%%%%%%%%%%%%%%%%%%%%%%%%%%%%%%

\begin{verbatim}
$( The MIU-system:  A simple formal system $)

$( Note:  This formal system is unusual in that it allows
empty wffs.  To work with a proof, you must type
SET EMPTY_SUBSTITUTION ON before using the PROVE command.
By default, this is OFF in order to reduce the number of
ambiguous unification possibilities that have to be selected
during the construction of a proof.  $)

$(
Hofstadter's MIU-system is a simple example of a formal
system that illustrates some concepts of Metamath.  See
Douglas R. Hofstadter, _Goedel, Escher, Bach:  An Eternal
Golden Braid_ (Vintage Books, New York, 1979), pp. 33ff. for
a description of the MIU-system.

The system has 3 constant symbols, M, I, and U.  The sole
axiom of the system is MI. There are 4 rules:
     Rule I:  If you possess a string whose last letter is I,
     you can add on a U at the end.
     Rule II:  Suppose you have Mx.  Then you may add Mxx to
     your collection.
     Rule III:  If III occurs in one of the strings in your
     collection, you may make a new string with U in place
     of III.
     Rule IV:  If UU occurs inside one of your strings, you
     can drop it.
Unfortunately, Rules III and IV do not have unique results:
strings could have more than one occurrence of III or UU.
This requires that we introduce the concept of an "MIU
well-formed formula" or wff, which allows us to construct
unique symbol sequences to which Rules III and IV can be
applied.
$)

$( First, we declare the constant symbols of the language.
Note that we need two symbols to distinguish the assertion
that a sequence is a wff from the assertion that it is a
theorem; we have arbitrarily chosen "wff" and "|-". $)
      $c M I U |- wff $. $( Declare constants $)

$( Next, we declare some variables. $)
     $v x y $.

$( Throughout our theory, we shall assume that these
variables represent wffs. $)
 wx   $f wff x $.
 wy   $f wff y $.

$( Define MIU-wffs.  We allow the empty sequence to be a
wff. $)

$( The empty sequence is a wff. $)
 we   $a wff $.
$( "M" after any wff is a wff. $)
 wM   $a wff x M $.
$( "I" after any wff is a wff. $)
 wI   $a wff x I $.
$( "U" after any wff is a wff. $)
 wU   $a wff x U $.

$( Assert the axiom. $)
 ax   $a |- M I $.

$( Assert the rules. $)
 ${
   Ia   $e |- x I $.
$( Given any theorem ending with "I", it remains a theorem
if "U" is added after it.  (We distinguish the label I_
from the math symbol I to conform to the 24-Jun-2006
Metamath spec.) $)
   I_    $a |- x I U $.
 $}
 ${
IIa  $e |- M x $.
$( Given any theorem starting with "M", it remains a theorem
if the part after the "M" is added again after it. $)
   II   $a |- M x x $.
 $}
 ${
   IIIa $e |- x I I I y $.
$( Given any theorem with "III" in the middle, it remains a
theorem if the "III" is replaced with "U". $)
   III  $a |- x U y $.
 $}
 ${
   IVa  $e |- x U U y $.
$( Given any theorem with "UU" in the middle, it remains a
theorem if the "UU" is deleted. $)
   IV   $a |- x y $.
  $}

$( Now we prove the theorem MUIIU.  You may be interested in
comparing this proof with that of Hofstadter (pp. 35 - 36).
$)
 theorem1  $p |- M U I I U $=
      we wM wU wI we wI wU we wU wI wU we wM we wI wU we wM
      wI wI wI we wI wI we wI ax II II I_ III II IV $.
\end{verbatim}\index{well-formed formula (wff)}

The \texttt{show proof /lemmon/renumber} command
yields the following display.  It is very similar
to the one in \cite[pp.~35--36]{Hofstadter}.\index{Hofstadter, Douglas R.}

\begin{verbatim}
1 ax             $a |- M I
2 1 II           $a |- M I I
3 2 II           $a |- M I I I I
4 3 I_           $a |- M I I I I U
5 4 III          $a |- M U I U
6 5 II           $a |- M U I U U I U
7 6 IV           $a |- M U I I U
\end{verbatim}

We note that Hofstadter's ``MU-puzzle,'' which asks whether
MU is a theorem of the MIU-system, cannot be answered using
the system above because the MU-puzzle is a question {\em
about} the system.  To prove the answer to the MU-puzzle,
a much more elaborate system is needed, namely one that
models the MIU-system within set theory.  (Incidentally, the
answer to the MU-puzzle is no.)

\chapter{Metamath Language EBNF}%
\label{BNF}%
\index{Metamath Language EBNF}

The following is a formal description of the basic Metamath language syntax
(with compressed proofs and support for unknown proof steps).
It is defined using the
Extended Backus--Naur Form (EBNF)\index{Extended Backus--Naur Form}\index{EBNF}
notation from W3C\index{W3C}
\textit{Extensible Markup Language (XML) 1.0 (Fifth Edition)}
(W3C Recommendation 26 November 2008) at
\url{https://www.w3.org/TR/xml/#sec-notation}.

The \texttt{database}
rule is processed until the end of the file (\texttt{EOF}).
The rules eventually require reading whitespace-separated tokens.
A token has an upper-case definition (see below)
or is a string constant in a non-token (such as \texttt{'\$a'}).
We intend for this to be correct, but if there is a conflict the
rules of section \ref{spec} govern. That section also discusses
non-syntax restrictions not shown here
(e.g., that each new label token
defined in a \texttt{hypothesis-stmt} or \texttt{assert-stmt}
must be unique).

\begin{verbatim}
database ::= outermost-scope-stmt*

outermost-scope-stmt ::=
  include-stmt | constant-stmt | stmt

/* File inclusion command; process file as a database.
   Databases should NOT have a comment in the filename. */
include-stmt ::= '$[' filename '$]'

/* Constant symbols declaration. */
constant-stmt ::= '$c' constant+ '$.'

/* A normal statement can occur in any scope. */
stmt ::= block | variable-stmt | disjoint-stmt |
  hypothesis-stmt | assert-stmt

/* A block. You can have 0 statements in a block. */
block ::= '${' stmt* '$}'

/* Variable symbols declaration. */
variable-stmt ::= '$v' variable+ '$.'

/* Disjoint variables. Simple disjoint statements have
   2 variables, i.e., "variable*" is empty for them. */
disjoint-stmt ::= '$d' variable variable variable* '$.'

hypothesis-stmt ::= floating-stmt | essential-stmt

/* Floating (variable-type) hypothesis. */
floating-stmt ::= LABEL '$f' typecode variable '$.'

/* Essential (logical) hypothesis. */
essential-stmt ::= LABEL '$e' typecode MATH-SYMBOL* '$.'

assert-stmt ::= axiom-stmt | provable-stmt

/* Axiomatic assertion. */
axiom-stmt ::= LABEL '$a' typecode MATH-SYMBOL* '$.'

/* Provable assertion. */
provable-stmt ::= LABEL '$p' typecode MATH-SYMBOL*
  '$=' proof '$.'

/* A proof. Proofs may be interspersed by comments.
   If '?' is in a proof it's an "incomplete" proof. */
proof ::= uncompressed-proof | compressed-proof
uncompressed-proof ::= (LABEL | '?')+
compressed-proof ::= '(' LABEL* ')' COMPRESSED-PROOF-BLOCK+

typecode ::= constant

filename ::= MATH-SYMBOL /* No whitespace or '$' */
constant ::= MATH-SYMBOL
variable ::= MATH-SYMBOL
\end{verbatim}

\needspace{2\baselineskip}
A \texttt{frame} is a sequence of 0 or more
\texttt{disjoint-{\allowbreak}stmt} and
\texttt{hypotheses-{\allowbreak}stmt} statements
(possibly interleaved with other non-\texttt{assert-stmt} statements)
followed by one \texttt{assert-stmt}.

\needspace{3\baselineskip}
Here are the rules for lexical processing (tokenization) beyond
the constant tokens shown above.
By convention these tokenization rules have upper-case names.
Every token is read for the longest possible length.
Whitespace-separated tokens are read sequentially;
note that the separating whitespace and \texttt{\$(} ... \texttt{\$)}
comments are skipped.

If a token definition uses another token definition, the whole thing
is considered a single token.
A pattern that is only part of a full token has a name beginning
with an underscore (``\_'').
An implementation could tokenize many tokens as a
\texttt{PRINTABLE-SEQUENCE}
and then check if it meets the more specific rule shown here.

Comments do not nest, and both \texttt{\$(} and \texttt{\$)}
have to be surrounded
by at least one whitespace character (\texttt{\_WHITECHAR}).
Technically comments end without consuming the trailing
\texttt{\_WHITECHAR}, but the trailing
\texttt{\_WHITECHAR} gets ignored anyway so we ignore that detail here.
Metamath language processors
are not required to support \texttt{\$)} followed
immediately by a bare end-of-file, because the closing
comment symbol is supposed to be followed by a
\texttt{\_WHITECHAR} such as a newline.

\begin{verbatim}
PRINTABLE-SEQUENCE ::= _PRINTABLE-CHARACTER+

MATH-SYMBOL ::= (_PRINTABLE-CHARACTER - '$')+

/* ASCII non-whitespace printable characters */
_PRINTABLE-CHARACTER ::= [#x21-#x7e]

LABEL ::= ( _LETTER-OR-DIGIT | '.' | '-' | '_' )+

_LETTER-OR-DIGIT ::= [A-Za-z0-9]

COMPRESSED-PROOF-BLOCK ::= ([A-Z] | '?')+

/* Define whitespace between tokens. The -> SKIP
   means that when whitespace is seen, it is
   skipped and we simply read again. */
WHITESPACE ::= (_WHITECHAR+ | _COMMENT) -> SKIP

/* Comments. $( ... $) and do not nest. */
_COMMENT ::= '$(' (_WHITECHAR+ (PRINTABLE-SEQUENCE - '$)')*
  _WHITECHAR+ '$)' _WHITECHAR

/* Whitespace: (' ' | '\t' | '\r' | '\n' | '\f') */
_WHITECHAR ::= [#x20#x09#x0d#x0a#x0c]
\end{verbatim}
% This EBNF was developed as a collaboration between
% David A. Wheeler\index{Wheeler, David A.},
% Mario Carneiro\index{Carneiro, Mario}, and
% Benoit Jubin\index{Jubin, Benoit}, inspired by a request
% (and a lot of initial work) by Benoit Jubin.
%
% \chapter{Disclaimer and Trademarks}
%
% Information in this document is subject to change without notice and does not
% represent a commitment on the part of Norman Megill.
% \vspace{2ex}
%
% \noindent Norman D. Megill makes no warranties, either express or implied,
% regarding the Metamath computer software package.
%
% \vspace{2ex}
%
% \noindent Any trademarks mentioned in this book are the property of
% their respective owners.  The name ``Metamath'' is a trademark of
% Norman Megill.
%
\cleardoublepage
\phantomsection  % fixes the link anchor
\addcontentsline{toc}{chapter}{\bibname}

\bibliography{metamath}
%% metamath.tex - Version of 24-May-2019
% If you change the date above, also change the "Printed date" below.
% SPDX-License-Identifier: CC0-1.0
%
%                              PUBLIC DOMAIN
%
% This file (specifically, the version of this file with the above date)
% has been released into the Public Domain per the
% Creative Commons CC0 1.0 Universal (CC0 1.0) Public Domain Dedication
% https://creativecommons.org/publicdomain/zero/1.0/
%
% The public domain release applies worldwide.  In case this is not
% legally possible, the right is granted to use the work for any purpose,
% without any conditions, unless such conditions are required by law.
%
% Several short, attributed quotations from copyrighted works
% appear in this file under the ``fair use'' provision of Section 107 of
% the United States Copyright Act (Title 17 of the {\em United States
% Code}).  The public-domain status of this file is not applicable to
% those quotations.
%
% Norman Megill - email: nm(at)alum(dot)mit(dot)edu
%
% David A. Wheeler also donates his improvements to this file to the
% public domain per the CC0.  He works at the Institute for Defense Analyses
% (IDA), but IDA has agreed that this Metamath work is outside its "lane"
% and is not a work by IDA.  This was specifically confirmed by
% Margaret E. Myers (Division Director of the Information Technology
% and Systems Division) on 2019-05-24 and by Ben Lindorf (General Counsel)
% on 2019-05-22.

% This file, 'metamath.tex', is self-contained with everything needed to
% generate the the PDF file 'metamath.pdf' (the _Metamath_ book) on
% standard LaTeX 2e installations.  The auxiliary files are embedded with
% "filecontents" commands.  To generate metamath.pdf file, run these
% commands under Linux or Cygwin in the directory that contains
% 'metamath.tex':
%
%   rm -f realref.sty metamath.bib
%   touch metamath.ind
%   pdflatex metamath
%   pdflatex metamath
%   bibtex metamath
%   makeindex metamath
%   pdflatex metamath
%   pdflatex metamath
%
% The warnings that occur in the initial runs of pdflatex can be ignored.
% For the final run,
%
%   egrep -i 'error|warn' metamath.log
%
% should show exactly these 5 warnings:
%
%   LaTeX Warning: File `realref.sty' already exists on the system.
%   LaTeX Warning: File `metamath.bib' already exists on the system.
%   LaTeX Font Warning: Font shape `OMS/cmtt/m/n' undefined
%   LaTeX Font Warning: Font shape `OMS/cmtt/bx/n' undefined
%   LaTeX Font Warning: Some font shapes were not available, defaults
%       substituted.
%
% Search for "Uncomment" below if you want to suppress hyperlink boxes
% in the PDF output file
%
% TYPOGRAPHICAL NOTES:
% * It is customary to use an en dash (--) to "connect" names of different
%   people (and to denote ranges), and use a hyphen (-) for a
%   single compound name. Examples of connected multiple people are
%   Zermelo--Fraenkel, Schr\"{o}der--Bernstein, Tarski--Grothendieck,
%   Hewlett--Packard, and Backus--Naur.  Examples of a single person with
%   a compound name include Levi-Civita, Mittag-Leffler, and Burali-Forti.
% * Use non-breaking spaces after page abbreviations, e.g.,
%   p.~\pageref{note2002}.
%
% --------------------------- Start of realref.sty -----------------------------
\begin{filecontents}{realref.sty}
% Save the following as realref.sty.
% You can then use it with \usepackage{realref}
%
% This has \pageref jumping to the page on which the ref appears,
% \ref jumping to the point of the anchor, and \sectionref
% jumping to the start of section.
%
% Author:  Anthony Williams
%          Software Engineer
%          Nortel Networks Optical Components Ltd
% Date:    9 Nov 2001 (posted to comp.text.tex)
%
% The following declaration was made by Anthony Williams on
% 24 Jul 2006 (private email to Norman Megill):
%
%   ``I hereby donate the code for realref.sty posted on the
%   comp.text.tex newsgroup on 9th November 2001, accessible from
%   http://groups.google.com/group/comp.text.tex/msg/5a0e1cc13ea7fbb2
%   to the public domain.''
%
\ProvidesPackage{realref}
\RequirePackage[plainpages=false,pdfpagelabels=true]{hyperref}
\def\realref@anchorname{}
\AtBeginDocument{%
% ensure every label is a possible hyperlink target
\let\realref@oldrefstepcounter\refstepcounter%
\DeclareRobustCommand{\refstepcounter}[1]{\realref@oldrefstepcounter{#1}
\edef\realref@anchorname{\string #1.\@currentlabel}%
}%
\let\realref@oldlabel\label%
\DeclareRobustCommand{\label}[1]{\realref@oldlabel{#1}\hypertarget{#1}{}%
\@bsphack\protected@write\@auxout{}{%
    \string\expandafter\gdef\protect\csname
    page@num.#1\string\endcsname{\thepage}%
    \string\expandafter\gdef\protect\csname
    ref@num.#1\string\endcsname{\@currentlabel}%
    \string\expandafter\gdef\protect\csname
    sectionref@name.#1\string\endcsname{\realref@anchorname}%
}\@esphack}%
\DeclareRobustCommand\pageref[1]{{\edef\a{\csname
            page@num.#1\endcsname}\expandafter\hyperlink{page.\a}{\a}}}%
\DeclareRobustCommand\ref[1]{{\edef\a{\csname
            ref@num.#1\endcsname}\hyperlink{#1}{\a}}}%
\DeclareRobustCommand\sectionref[1]{{\edef\a{\csname
            ref@num.#1\endcsname}\edef\b{\csname
            sectionref@name.#1\endcsname}\hyperlink{\b}{\a}}}%
}
\end{filecontents}
% ---------------------------- End of realref.sty ------------------------------

% --------------------------- Start of metamath.bib -----------------------------
\begin{filecontents}{metamath.bib}
@book{Albers, editor = "Donald J. Albers and G. L. Alexanderson",
  title = "Mathematical People",
  publisher = "Contemporary Books, Inc.",
  address = "Chicago",
  note = "[QA28.M37]",
  year = 1985 }
@book{Anderson, author = "Alan Ross Anderson and Nuel D. Belnap",
  title = "Entailment",
  publisher = "Princeton University Press",
  address = "Princeton",
  volume = 1,
  note = "[QA9.A634 1975 v.1]",
  year = 1975}
@book{Barrow, author = "John D. Barrow",
  title = "Theories of Everything:  The Quest for Ultimate Explanation",
  publisher = "Oxford University Press",
  address = "Oxford",
  note = "[Q175.B225]",
  year = 1991 }
@book{Behnke,
  editor = "H. Behnke and F. Backmann and K. Fladt and W. S{\"{u}}ss",
  title = "Fundamentals of Mathematics",
  volume = "I",
  publisher = "The MIT Press",
  address = "Cambridge, Massachusetts",
  note = "[QA37.2.B413]",
  year = 1974 }
@book{Bell, author = "J. L. Bell and M. Machover",
  title = "A Course in Mathematical Logic",
  publisher = "North-Holland",
  address = "Amsterdam",
  note = "[QA9.B3953]",
  year = 1977 }
@inproceedings{Blass, author = "Andrea Blass",
  title = "The Interaction Between Category Theory and Set Theory",
  pages = "5--29",
  booktitle = "Mathematical Applications of Category Theory (Proceedings
     of the Special Session on Mathematical Applications
     Category Theory, 89th Annual Meeting of the American Mathematical
     Society, held in Denver, Colorado January 5--9, 1983)",
  editor = "John Walter Gray",
  year = 1983,
  note = "[QA169.A47 1983]",
  publisher = "American Mathematical Society",
  address = "Providence, Rhode Island"}
@proceedings{Bledsoe, editor = "W. W. Bledsoe and D. W. Loveland",
  title = "Automated Theorem Proving:  After 25 Years (Proceedings
     of the Special Session on Automatic Theorem Proving,
     89th Annual Meeting of the American Mathematical
     Society, held in Denver, Colorado January 5--9, 1983)",
  year = 1983,
  note = "[QA76.9.A96.S64 1983]",
  publisher = "American Mathematical Society",
  address = "Providence, Rhode Island" }
@book{Boolos, author = "George S. Boolos and Richard C. Jeffrey",
  title = "Computability and Log\-ic",
  publisher = "Cambridge University Press",
  edition = "third",
  address = "Cambridge",
  note = "[QA9.59.B66 1989]",
  year = 1989 }
@book{Campbell, author = "John Campbell",
  title = "Programmer's Progress",
  publisher = "White Star Software",
  address = "Box 51623, Palo Alto, CA 94303",
  year = 1991 }
@article{DBLP:journals/corr/Carneiro14,
  author    = {Mario Carneiro},
  title     = {Conversion of {HOL} Light proofs into Metamath},
  journal   = {CoRR},
  volume    = {abs/1412.8091},
  year      = {2014},
  url       = {http://arxiv.org/abs/1412.8091},
  archivePrefix = {arXiv},
  eprint    = {1412.8091},
  timestamp = {Mon, 13 Aug 2018 16:47:05 +0200},
  biburl    = {https://dblp.org/rec/bib/journals/corr/Carneiro14},
  bibsource = {dblp computer science bibliography, https://dblp.org}
}
@article{CarneiroND,
  author    = {Mario Carneiro},
  title     = {Natural Deductions in the Metamath Proof Language},
  url       = {http://us.metamath.org/ocat/natded.pdf},
  year      = 2014
}
@inproceedings{Chou, author = "Shang-Ching Chou",
  title = "Proving Elementary Geometry Theorems Using {W}u's Algorithm",
  pages = "243--286",
  booktitle = "Automated Theorem Proving:  After 25 Years (Proceedings
     of the Special Session on Automatic Theorem Proving,
     89th Annual Meeting of the American Mathematical
     Society, held in Denver, Colorado January 5--9, 1983)",
  editor = "W. W. Bledsoe and D. W. Loveland",
  year = 1983,
  note = "[QA76.9.A96.S64 1983]",
  publisher = "American Mathematical Society",
  address = "Providence, Rhode Island" }
@book{Clemente, author = "Daniel Clemente Laboreo",
  title = "Introduction to natural deduction",
  year = 2014,
  url = "http://www.danielclemente.com/logica/dn.en.pdf" }
@incollection{Courant, author = "Richard Courant and Herbert Robbins",
  title = "Topology",
  pages = "573--590",
  booktitle = "The World of Mathematics, Volume One",
  editor = "James R. Newman",
  publisher = "Simon and Schuster",
  address = "New York",
  note = "[QA3.W67 1988]",
  year = 1956 }
@book{Curry, author = "Haskell B. Curry",
  title = "Foundations of Mathematical Logic",
  publisher = "Dover Publications, Inc.",
  address = "New York",
  note = "[QA9.C976 1977]",
  year = 1977 }
@book{Davis, author = "Philip J. Davis and Reuben Hersh",
  title = "The Mathematical Experience",
  publisher = "Birkh{\"{a}}user Boston",
  address = "Boston",
  note = "[QA8.4.D37 1982]",
  year = 1981 }
@incollection{deMillo,
  author = "Richard de Millo and Richard Lipton and Alan Perlis",
  title = "Social Processes and Proofs of Theorems and Programs",
  pages = "267--285",
  booktitle = "New Directions in the Philosophy of Mathematics",
  editor = "Thomas Tymoczko",
  publisher = "Birkh{\"{a}}user Boston, Inc.",
  address = "Boston",
  note = "[QA8.6.N48 1986]",
  year = 1986 }
@book{Edwards, author = "Robert E. Edwards",
  title = "A Formal Background to Mathematics",
  publisher = "Springer-Verlag",
  address = "New York",
  note = "[QA37.2.E38 v.1a]",
  year = 1979 }
@book{Enderton, author = "Herbert B. Enderton",
  title = "Elements of Set Theory",
  publisher = "Academic Press, Inc.",
  address = "San Diego",
  note = "[QA248.E5]",
  year = 1977 }
@book{Goodstein, author = "R. L. Goodstein",
  title = "Development of Mathematical Logic",
  publisher = "Springer-Verlag New York Inc.",
  address = "New York",
  note = "[QA9.G6554]",
  year = 1971 }
@book{Guillen, author = "Michael Guillen",
  title = "Bridges to Infinity",
  publisher = "Jeremy P. Tarcher, Inc.",
  address = "Los Angeles",
  note = "[QA93.G8]",
  year = 1983 }
@book{Hamilton, author = "Alan G. Hamilton",
  title = "Logic for Mathematicians",
  edition = "revised",
  publisher = "Cambridge University Press",
  address = "Cambridge",
  note = "[QA9.H298]",
  year = 1988 }
@unpublished{Harrison, author = "John Robert Harrison",
  title = "Metatheory and Reflection in Theorem Proving:
    A Survey and Critique",
  note = "Technical Report
    CRC-053.
    SRI Cambridge,
    Millers Yard, Cambridge, UK,
    1995.
    Available on the Web as
{\verb+http:+}\-{\verb+//www.cl.cam.ac.uk/users/jrh/papers/reflect.html+}"}
@TECHREPORT{Harrison-thesis,
        author          = "John Robert Harrison",
        title           = "Theorem Proving with the Real Numbers",
        institution   = "University of Cambridge Computer
                         Lab\-o\-ra\-to\-ry",
        address         = "New Museums Site, Pembroke Street, Cambridge,
                           CB2 3QG, UK",
        year            = 1996,
        number          = 408,
        type            = "Technical Report",
        note            = "Author's PhD thesis,
   available on the Web at
{\verb+http:+}\-{\verb+//www.cl.cam.ac.uk+}\-{\verb+/users+}\-{\verb+/jrh+}%
\-{\verb+/papers+}\-{\verb+/thesis.html+}"}
@book{Herrlich, author = "Horst Herrlich and George E. Strecker",
  title = "Category Theory:  An Introduction",
  publisher = "Allyn and Bacon Inc.",
  address = "Boston",
  note = "[QA169.H567]",
  year = 1973 }
@article{Hindley, author = "J. Roger Hindley and David Meredith",
  title = "Principal Type-Schemes and Condensed Detachment",
  journal = "The Journal of Symbolic Logic",
  volume = 55,
  year = 1990,
  note = "[QA.J87]",
  pages = "90--105" }
@book{Hofstadter, author = "Douglas R. Hofstadter",
  title = "G{\"{o}}del, Escher, Bach",
  publisher = "Basic Books, Inc.",
  address = "New York",
  note = "[QA9.H63 1980]",
  year = 1979 }
@article{Indrzejczak, author= "Andrzej Indrzejczak",
  title = "Natural Deduction, Hybrid Systems and Modal Logic",
  journal = "Trends in Logic",
  volume = 30,
  publisher = "Springer",
  year = 2010 }
@article{Kalish, author = "D. Kalish and R. Montague",
  title = "On {T}arski's Formalization of Predicate Logic with Identity",
  journal = "Archiv f{\"{u}}r Mathematische Logik und Grundlagenfor\-schung",
  volume = 7,
  year = 1965,
  note = "[QA.A673]",
  pages = "81--101" }
@article{Kalman, author = "J. A. Kalman",
  title = "Condensed Detachment as a Rule of Inference",
  journal = "Studia Logica",
  volume = 42,
  number = 4,
  year = 1983,
  note = "[B18.P6.S933]",
  pages = "443-451" }
@book{Kline, author = "Morris Kline",
  title = "Mathematical Thought from Ancient to Modern Times",
  publisher = "Oxford University Press",
  address = "New York",
  note = "[QA21.K516 1990 v.3]",
  year = 1972 }
@book{Klinel, author = "Morris Kline",
  title = "Mathematics, The Loss of Certainty",
  publisher = "Oxford University Press",
  address = "New York",
  note = "[QA21.K525]",
  year = 1980 }
@book{Kramer, author = "Edna E. Kramer",
  title = "The Nature and Growth of Modern Mathematics",
  publisher = "Princeton University Press",
  address = "Princeton, New Jersey",
  note = "[QA93.K89 1981]",
  year = 1981 }
@article{Knill, author = "Oliver Knill",
  title = "Some Fundamental Theorems in Mathematics",
  year = "2018",
  url = "https://arxiv.org/abs/1807.08416" }
@book{Landau, author = "Edmund Landau",
  title = "Foundations of Analysis",
  publisher = "Chelsea Publishing Company",
  address = "New York",
  edition = "second",
  note = "[QA241.L2541 1960]",
  year = 1960 }
@article{Leblanc, author = "Hugues Leblanc",
  title = "On {M}eyer and {L}ambert's Quantificational Calculus {FQ}",
  journal = "The Journal of Symbolic Logic",
  volume = 33,
  year = 1968,
  note = "[QA.J87]",
  pages = "275--280" }
@article{Lejewski, author = "Czeslaw Lejewski",
  title = "On Implicational Definitions",
  journal = "Studia Logica",
  volume = 8,
  year = 1958,
  note = "[B18.P6.S933]",
  pages = "189--208" }
@book{Levy, author = "Azriel Levy",
  title = "Basic Set Theory",
  publisher = "Dover Publications",
  address = "Mineola, NY",
  year = "2002"
}
@book{Margaris, author = "Angelo Margaris",
  title = "First Order Mathematical Logic",
  publisher = "Blaisdell Publishing Company",
  address = "Waltham, Massachusetts",
  note = "[QA9.M327]",
  year = 1967}
@book{Manin, author = "Yu I. Manin",
  title = "A Course in Mathematical Logic",
  publisher = "Springer-Verlag",
  address = "New York",
  note = "[QA9.M29613]",
  year = "1977" }
@article{Mathias, author = "Adrian R. D. Mathias",
  title = "A Term of Length 4,523,659,424,929",
  journal = "Synthese",
  volume = 133,
  year = 2002,
  note = "[Q.S993]",
  pages = "75--86" }
@article{Megill, author = "Norman D. Megill",
  title = "A Finitely Axiomatized Formalization of Predicate Calculus
     with Equality",
  journal = "Notre Dame Journal of Formal Logic",
  volume = 36,
  year = 1995,
  note = "[QA.N914]",
  pages = "435--453" }
@unpublished{Megillc, author = "Norman D. Megill",
  title = "A Shorter Equivalent of the Axiom of Choice",
  month = "June",
  note = "Unpublished",
  year = 1991 }
@article{MegillBunder, author = "Norman D. Megill and Martin W.
    Bunder",
  title = "Weaker {D}-Complete Logics",
  journal = "Journal of the IGPL",
  volume = 4,
  year = 1996,
  pages = "215--225",
  note = "Available on the Web at
{\verb+http:+}\-{\verb+//www.mpi-sb.mpg.de+}\-{\verb+/igpl+}%
\-{\verb+/Journal+}\-{\verb+/V4-2+}\-{\verb+/#Megill+}"}
}
@book{Mendelson, author = "Elliott Mendelson",
  title = "Introduction to Mathematical Logic",
  edition = "second",
  publisher = "D. Van Nostrand Company, Inc.",
  address = "New York",
  note = "[QA9.M537 1979]",
  year = 1979 }
@article{Meredith, author = "David Meredith",
  title = "In Memoriam {C}arew {A}rthur {M}eredith (1904-1976)",
  journal = "Notre Dame Journal of Formal Logic",
  volume = 18,
  year = 1977,
  note = "[QA.N914]",
  pages = "513--516" }
@article{CAMeredith, author = "C. A. Meredith",
  title = "Single Axioms for the Systems ({C},{N}), ({C},{O}) and ({A},{N})
      of the Two-Valued Propositional Calculus",
  journal = "The Journal of Computing Systems",
  volume = 3,
  year = 1953,
  pages = "155--164" }
@article{Monk, author = "J. Donald Monk",
  title = "Provability With Finitely Many Variables",
  journal = "The Journal of Symbolic Logic",
  volume = 27,
  year = 1971,
  note = "[QA.J87]",
  pages = "353--358" }
@article{Monks, author = "J. Donald Monk",
  title = "Substitutionless Predicate Logic With Identity",
  journal = "Archiv f{\"{u}}r Mathematische Logik und Grundlagenfor\-schung",
  volume = 7,
  year = 1965,
  pages = "103--121" }
  %% Took out this from above to prevent LaTeX underfull warning:
  % note = "[QA.A673]",
@book{Moore, author = "A. W. Moore",
  title = "The Infinite",
  publisher = "Routledge",
  address = "New York",
  note = "[BD411.M59]",
  year = 1989}
@book{Munkres, author = "James R. Munkres",
  title = "Topology: A First Course",
  publisher = "Prentice-Hall, Inc.",
  address = "Englewood Cliffs, New Jersey",
  note = "[QA611.M82]",
  year = 1975}
@article{Nemesszeghy, author = "E. Z. Nemesszeghy and E. A. Nemesszeghy",
  title = "On Strongly Creative Definitions:  A Reply to {V}. {F}. {R}ickey",
  journal = "Logique et Analyse (N.\ S.)",
  year = 1977,
  volume = 20,
  note = "[BC.L832]",
  pages = "111--115" }
@unpublished{Nemeti, author = "N{\'{e}}meti, I.",
  title = "Algebraizations of Quantifier Logics, an Overview",
  note = "Version 11.4, preprint, Mathematical Institute, Budapest,
    1994.  A shortened version without proofs appeared in
    ``Algebraizations of quantifier logics, an introductory overview,''
   {\em Studia Logica}, 50:485--569, 1991 [B18.P6.S933]"}
@article{Pavicic, author = "M. Pavi{\v{c}}i{\'{c}}",
  title = "A New Axiomatization of Unified Quantum Logic",
  journal = "International Journal of Theoretical Physics",
  year = 1992,
  volume = 31,
  note = "[QC.I626]",
  pages = "1753 --1766" }
@book{Penrose, author = "Roger Penrose",
  title = "The Emperor's New Mind",
  publisher = "Oxford University Press",
  address = "New York",
  note = "[Q335.P415]",
  year = 1989 }
@book{PetersonI, author = "Ivars Peterson",
  title = "The Mathematical Tourist",
  publisher = "W. H. Freeman and Company",
  address = "New York",
  note = "[QA93.P475]",
  year = 1988 }
@article{Peterson, author = "Jeremy George Peterson",
  title = "An automatic theorem prover for substitution and detachment systems",
  journal = "Notre Dame Journal of Formal Logic",
  volume = 19,
  year = 1978,
  note = "[QA.N914]",
  pages = "119--122" }
@book{Quine, author = "Willard Van Orman Quine",
  title = "Set Theory and Its Logic",
  edition = "revised",
  publisher = "The Belknap Press of Harvard University Press",
  address = "Cambridge, Massachusetts",
  note = "[QA248.Q7 1969]",
  year = 1969 }
@article{Robinson, author = "J. A. Robinson",
  title = "A Machine-Oriented Logic Based on the Resolution Principle",
  journal = "Journal of the Association for Computing Machinery",
  year = 1965,
  volume = 12,
  pages = "23--41" }
@article{RobinsonT, author = "T. Thacher Robinson",
  title = "Independence of Two Nice Sets of Axioms for the Propositional
    Calculus",
  journal = "The Journal of Symbolic Logic",
  volume = 33,
  year = 1968,
  note = "[QA.J87]",
  pages = "265--270" }
@book{Rucker, author = "Rudy Rucker",
  title = "Infinity and the Mind:  The Science and Philosophy of the
    Infinite",
  publisher = "Bantam Books, Inc.",
  address = "New York",
  note = "[QA9.R79 1982]",
  year = 1982 }
@book{Russell, author = "Bertrand Russell",
  title = "Mysticism and Logic, and Other Essays",
  publisher = "Barnes \& Noble Books",
  address = "Totowa, New Jersey",
  note = "[B1649.R963.M9 1981]",
  year = 1981 }
@article{Russell2, author = "Bertrand Russell",
  title = "Recent Work on the Principles of Mathematics",
  journal = "International Monthly",
  volume = 4,
  year = 1901,
  pages = "84"}
@article{Schmidt, author = "Eric Schmidt",
  title = "Reductions in Norman Megill's axiom system for complex numbers",
  url = "http://us.metamath.org/downloads/schmidt-cnaxioms.pdf",
  year = "2012" }
@book{Shoenfield, author = "Joseph R. Shoenfield",
  title = "Mathematical Logic",
  publisher = "Addison-Wesley Publishing Company, Inc.",
  address = "Reading, Massachusetts",
  year = 1967,
  note = "[QA9.S52]" }
@book{Smullyan, author = "Raymond M. Smullyan",
  title = "Theory of Formal Systems",
  publisher = "Princeton University Press",
  address = "Princeton, New Jersey",
  year = 1961,
  note = "[QA248.5.S55]" }
@book{Solow, author = "Daniel Solow",
  title = "How to Read and Do Proofs:  An Introduction to Mathematical
    Thought Process",
  publisher = "John Wiley \& Sons",
  address = "New York",
  year = 1982,
  note = "[QA9.S577]" }
@book{Stark, author = "Harold M. Stark",
  title = "An Introduction to Number Theory",
  publisher = "Markham Publishing Company",
  address = "Chicago",
  note = "[QA241.S72 1978]",
  year = 1970 }
@article{Swart, author = "E. R. Swart",
  title = "The Philosophical Implications of the Four-Color Problem",
  journal = "American Mathematical Monthly",
  year = 1980,
  volume = 87,
  month = "November",
  note = "[QA.A5125]",
  pages = "697--707" }
@book{Szpiro, author = "George G. Szpiro",
  title = "Poincar{\'{e}}'s Prize: The Hundred-Year Quest to Solve One
    of Math's Greatest Puzzles",
  publisher = "Penguin Books Ltd",
  address = "London",
  note = "[QA43.S985 2007]",
  year = 2007}
@book{Takeuti, author = "Gaisi Takeuti and Wilson M. Zaring",
  title = "Introduction to Axiomatic Set Theory",
  edition = "second",
  publisher = "Springer-Verlag New York Inc.",
  address = "New York",
  note = "[QA248.T136 1982]",
  year = 1982}
@inproceedings{Tarski, author = "Alfred Tarski",
  title = "What is Elementary Geometry",
  pages = "16--29",
  booktitle = "The Axiomatic Method, with Special Reference to Geometry and
     Physics (Proceedings of an International Symposium held at the University
     of California, Berkeley, December 26, 1957 --- January 4, 1958)",
  editor = "Leon Henkin and Patrick Suppes and Alfred Tarski",
  year = 1959,
  publisher = "North-Holland Publishing Company",
  address = "Amsterdam"}
@article{Tarski1965, author = "Alfred Tarski",
  title = "A Simplified Formalization of Predicate Logic with Identity",
  journal = "Archiv f{\"{u}}r Mathematische Logik und Grundlagenforschung",
  volume = 7,
  year = 1965,
  note = "[QA.A673]",
  pages = "61--79" }
@book{Tymoczko,
  title = "New Directions in the Philosophy of Mathematics",
  editor = "Thomas Tymoczko",
  publisher = "Birkh{\"{a}}user Boston, Inc.",
  address = "Boston",
  note = "[QA8.6.N48 1986]",
  year = 1986 }
@incollection{Wang,
  author = "Hao Wang",
  title = "Theory and Practice in Mathematics",
  pages = "129--152",
  booktitle = "New Directions in the Philosophy of Mathematics",
  editor = "Thomas Tymoczko",
  publisher = "Birkh{\"{a}}user Boston, Inc.",
  address = "Boston",
  note = "[QA8.6.N48 1986]",
  year = 1986 }
@manual{Webster,
  title = "Webster's New Collegiate Dictionary",
  organization = "G. \& C. Merriam Co.",
  address = "Springfield, Massachusetts",
  note = "[PE1628.W4M4 1977]",
  year = 1977 }
@manual{Whitehead, author = "Alfred North Whitehead",
  title = "An Introduction to Mathematics",
  year = 1911 }
@book{PM, author = "Alfred North Whitehead and Bertrand Russell",
  title = "Principia Mathematica",
  edition = "second",
  publisher = "Cambridge University Press",
  address = "Cambridge",
  year = "1927",
  note = "(3 vols.) [QA9.W592 1927]" }
@article{DBLP:journals/corr/Whalen16,
  author    = {Daniel Whalen},
  title     = {Holophrasm: a neural Automated Theorem Prover for higher-order logic},
  journal   = {CoRR},
  volume    = {abs/1608.02644},
  year      = {2016},
  url       = {http://arxiv.org/abs/1608.02644},
  archivePrefix = {arXiv},
  eprint    = {1608.02644},
  timestamp = {Mon, 13 Aug 2018 16:46:19 +0200},
  biburl    = {https://dblp.org/rec/bib/journals/corr/Whalen16},
  bibsource = {dblp computer science bibliography, https://dblp.org} }
@article{Wiedijk-revisited,
  author = {Freek Wiedijk},
  title = {The QED Manifesto Revisited},
  year = {2007},
  url = {http://mizar.org/trybulec65/8.pdf} }
@book{Wolfram,
  author = "Stephen Wolfram",
  title = "Mathematica:  A System for Doing Mathematics by Computer",
  edition = "second",
  publisher = "Addison-Wesley Publishing Co.",
  address = "Redwood City, California",
  note = "[QA76.95.W65 1991]",
  year = 1991 }
@book{Wos, author = "Larry Wos and Ross Overbeek and Ewing Lusk and Jim Boyle",
  title = "Automated Reasoning:  Introduction and Applications",
  edition = "second",
  publisher = "McGraw-Hill, Inc.",
  address = "New York",
  note = "[QA76.9.A96.A93 1992]",
  year = 1992 }

%
%
%[1] Church, Alonzo, Introduction to Mathematical Logic,
% Volume 1, Princeton University Press, Princeton, N. J., 1956.
%
%[2] Cohen, Paul J., Set Theory and the Continuum Hypothesis,
% W. A. Benjamin, Inc., Reading, Mass., 1966.
%
%[3] Hamilton, Alan G., Logic for Mathematicians, Cambridge
% University Press,
% Cambridge, 1988.

%[6] Kleene, Stephen Cole, Introduction to Metamathematics, D.  Van
% Nostrand Company, Inc., Princeton (1952).

%[13] Tarski, Alfred, "A simplified formalization of predicate
% logic with identity," Archiv fur Mathematische Logik und
% Grundlagenforschung, vol. 7 (1965), pp. 61-79.

%[14] Tarski, Alfred and Steven Givant, A Formalization of Set
% Theory Without Variables, American Mathematical Society Colloquium
% Publications, vol. 41, American Mathematical Society,
% Providence, R. I., 1987.

%[15] Zeman, J. J., Modal Logic, Oxford University Press, Oxford, 1973.
\end{filecontents}
% --------------------------- End of metamath.bib -----------------------------


%Book: Metamath
%Author:  Norman Megill Email:  nm at alum.mit.edu
%Author:  David A. Wheeler Email:  dwheeler at dwheeler.com

% A book template example
% http://www.stsci.edu/ftp/software/tex/bookstuff/book.template

\documentclass[leqno]{book} % LaTeX 2e. 10pt. Use [leqno,12pt] for 12pt
% hyperref 2002/05/27 v6.72r  (couldn't get pagebackref to work)
\usepackage[plainpages=false,pdfpagelabels=true]{hyperref}

\usepackage{needspace}     % Enable control over page breaks
\usepackage{breqn}         % automatic equation breaking
\usepackage{microtype}     % microtypography, reduces hyphenation

% Packages for flexible tables.  We need to be able to
% wrap text within a cell (with automatically-determined widths) AND
% split a table automatically across multiple pages.
% * "tabularx" wraps text in cells but only 1 page
% * "longtable" goes across pages but by itself is incompatible with tabularx
% * "ltxtable" combines longtable and tabularx, but table contents
%    must be in a separate file.
% * "ltablex" combines tabularx and longtable - must install specially
% * "booktabs" is recommended as a way to improve the look of tables,
%   but doesn't add these capabilities.
% * "tabu" much more capable and seems to be recommended. So use that.

\usepackage{makecell}      % Enable forced line splits within a table cell
\usepackage{longtable}     % Enable multi-page tables
\usepackage{tabu}          % Multi-page tables with wrapped text in a cell

% You can find more Tex packages using commands like:
% tlmgr search --file tabu.sty
% find /usr/share/texmf-dist/ -name '*tab*'
%
%%%%%%%%%%%%%%%%%%%%%%%%%%%%%%%%%%%%%%%%%%%%%%%%%%%%%%%%%%%%%%%%%%%%%%%%%%%%
% Uncomment the next 3 lines to suppress boxes and colors on the hyperlinks
%%%%%%%%%%%%%%%%%%%%%%%%%%%%%%%%%%%%%%%%%%%%%%%%%%%%%%%%%%%%%%%%%%%%%%%%%%%%
%\hypersetup{
%colorlinks,citecolor=black,filecolor=black,linkcolor=black,urlcolor=black
%}
%
\usepackage{realref}

% Restarting page numbers: try?
%   \printglossary
%   \cleardoublepage
%   \pagenumbering{arabic}
%   \setcounter{page}{1}    ???needed
%   \include{chap1}

% not used:
% \def\R2Lurl#1#2{\mbox{\href{#1}\texttt{#2}}}

\usepackage{amssymb}

% Version 1 of book: margins: t=.4, b=.2, ll=.4, rr=.55
% \usepackage{anysize}
% % \papersize{<height>}{<width>}
% % \marginsize{<left>}{<right>}{<top>}{<bottom>}
% \papersize{9in}{6in}
% % l/r 0.6124-0.6170 works t/b 0.2418-0.3411 = 192pp. 0.2926-03118=exact
% \marginsize{0.7147in}{0.5147in}{0.4012in}{0.2012in}

\usepackage{anysize}
% \papersize{<height>}{<width>}
% \marginsize{<left>}{<right>}{<top>}{<bottom>}
\papersize{9in}{6in}
% l/r 0.85in&0.6431-0.6539 works t/b ?-?
%\marginsize{0.85in}{0.6485in}{0.55in}{0.35in}
\marginsize{0.8in}{0.65in}{0.5in}{0.3in}

% \usepackage[papersize={3.6in,4.8in},hmargin=0.1in,vmargin={0.1in,0.1in}]{geometry}  % page geometry
\usepackage{special-settings}

\raggedbottom
\makeindex

\begin{document}
% Discourage page widows and orphans:
\clubpenalty=300
\widowpenalty=300

%%%%%%% load in AMS fonts %%%%%%% % LaTeX 2.09 - obsolete in LaTeX 2e
%\input{amssym.def}
%\input{amssym.tex}
%\input{c:/texmf/tex/plain/amsfonts/amssym.def}
%\input{c:/texmf/tex/plain/amsfonts/amssym.tex}

\bibliographystyle{plain}
\pagenumbering{roman}
\pagestyle{headings}

\thispagestyle{empty}

\hfill
\vfill

\begin{center}
{\LARGE\bf Metamath} \\
\vspace{1ex}
{\large A Computer Language for Mathematical Proofs} \\
\vspace{7ex}
{\large Norman Megill} \\
\vspace{7ex}
with extensive revisions by \\
\vspace{1ex}
{\large David A. Wheeler} \\
\vspace{7ex}
% Printed date. If changing the date below, also fix the date at the beginning.
2019-05-24
\end{center}

\vfill
\hfill

\newpage
\thispagestyle{empty}

\hfill
\vfill

\begin{center}
$\sim$\ {\sc Public Domain}\ $\sim$

\vspace{2ex}
This book (including its later revisions)
has been released into the Public Domain by Norman Megill per the
Creative Commons CC0 1.0 Universal (CC0 1.0) Public Domain Dedication
(\url{https://creativecommons.org/publicdomain/zero/1.0/}).
David A. Wheeler has done the same.
The public domain release applies worldwide.  In case this is not
legally possible, the right is granted to use the work for any purpose,
without any conditions, unless such conditions are required by law.

\vspace{3ex}
Several short, attributed quotations from copyrighted works
appear in this book under the ``fair use'' provision of Section 107 of
the United States Copyright Act (Title 17 of the {\em United States
Code}).  The public-domain status of this book is not applicable to
those quotations.

\vspace{3ex}
Any trademarks used in this book are the property of their owners.

% QA76.9.L63.M??

% \vspace{1ex}
%
% \vspace{1ex}
% {\small Permission is granted to make and distribute verbatim copies of this
% book
% provided the copyright notice and this
% permission notice are preserved on all copies.}
%
% \vspace{1ex}
% {\small Permission is granted to copy and distribute modified versions of this
% book under the conditions for verbatim copying, provided that the
% entire
% resulting derived work is distributed under the terms of a permission
% notice
% identical to this one.}
%
% \vspace{1ex}
% {\small Permission is granted to copy and distribute translations of this
% book into another language, under the above conditions for modified
% versions,
% except that this permission notice may be stated in a translation
% approved by the
% author.}
%
% \vspace{1ex}
% %{\small   For a copy of the \LaTeX\ source files for this book, contact
% %the author.} \\
% \ \\
% \ \\

\vspace{7ex}
% ISBN: 1-4116-3724-0 \\
ISBN: 978-1-4116-3724-5 \\
{\ } \\
Lulu Press \\
Morrisville, North Carolina\\
USA


\hfill
\vfill

Norman Megill\\ 93 Bridge St., Lexington, MA 02421 \\
E-mail address: \texttt{nm{\char`\@}alum.mit.edu} \\
\vspace{7ex}
David A. Wheeler \\
E-mail address: \texttt{dwheeler{\char`\@}dwheeler.com} \\
% See notes added at end of Preface for revision history. \\
% For current information on the Metamath software see \\
\vspace{7ex}
\url{http://metamath.org}
\end{center}

% \newpage
% \thispagestyle{empty}
%
% \hfill
% \vfill
%
% \begin{center}
% {\it To my son Robin Dwight Megill}
% \end{center}
%
% \vfill
% \hfill
%
% \newpage

\tableofcontents
%\listoftables

\chapter*{Preface}
\markboth{PREFACE}{PREFACE}
\addcontentsline{toc}{section}{Preface}


% (For current information, see the notes added at the
% end of this preface on p.~\pageref{note2002}.)

\subsubsection{Overview}

Metamath\index{Metamath} is a computer language and an associated computer
program for archiving, verifying, and studying mathematical proofs at a very
detailed level.  The Metamath language incorporates no mathematics per se but
treats all mathematical statements as mere sequences of symbols.  You provide
Metamath with certain special sequences (axioms) that tell it what rules
of inference are allowed.  Metamath is not limited to any specific field of
mathematics.  The Metamath language is simple and robust, with an
almost total absence of hard-wired syntax, and
I\footnote{Unless otherwise noted, the words
``I,'' ``me,'' and ``my'' refer to Norman Megill\index{Megill, Norman}, while
``we,'' ``us,'' and ``our'' refer to Norman Megill and
David A. Wheeler\index{Wheeler, David A.}.}
believe that it
provides about the simplest possible framework that allows essentially all of
mathematics to be expressed with absolute rigor.

% index test
%\newcommand{\nn}[1]{#1n}
%\index{aaa@bbb}
%\index{abc!def}
%\index{abd|see{qqq}}
%\index{abe|nn}
%\index{abf|emph}
%\index{abg|(}
%\index{abg|)}

Using the Metamath language, you can build formal or mathematical
systems\index{formal system}\footnote{A formal or mathematical system consists
of a collection of symbols (such as $2$, $4$, $+$ and $=$), syntax rules that
describe how symbols may be combined to form a legal expression (called a
well-formed formula or {\em wff}, pronounced ``whiff''), some starting wffs
called axioms, and inference rules that describe how theorems may be derived
(proved) from the axioms.  A theorem is a mathematical fact such as $2+2=4$.
Strictly speaking, even an obvious fact such as this must be proved from
axioms to be formally acceptable to a mathematician.}\index{theorem}
\index{axiom}\index{rule}\index{well-formed formula (wff)} that involve
inferences from axioms.  Although a database is provided
that includes a recommended set of axioms for standard mathematics, if you
wish you can supply your own symbols, syntax, axioms, rules, and definitions.

The name ``Metamath'' was chosen to suggest that the language provides a
means for {\em describing} mathematics rather than {\em being} the
mathematics itself.  Actually in some sense any mathematical language is
metamathematical.  Symbols written on paper, or stored in a computer,
are not mathematics itself but rather a way of expressing mathematics.
For example ``7'' and ``VII'' are symbols for denoting the number seven
in Arabic and Roman numerals; neither {\em is} the number seven.

If you are able to understand and write computer programs, you should be able
to follow abstract mathematics with the aid of Metamath.  Used in conjunction
with standard textbooks, Metamath can guide you step-by-step towards an
understanding of abstract mathematics from a very rigorous viewpoint, even if
you have no formal abstract mathematics background.  By using a single,
consistent notation to express proofs, once you grasp its basic concepts
Metamath provides you with the ability to immediately follow and dissect
proofs even in totally unfamiliar areas.

Of course, just being able follow a proof will not necessarily give you an
intuitive familiarity with mathematics.  Memorizing the rules of chess does not
give you the ability to appreciate the game of a master, and knowing how the
notes on a musical score map to piano keys does not give you the ability to
hear in your head how it would sound.  But each of these can be a first step.

Metamath allows you to explore proofs in the sense that you can see the
theorem referenced at any step expanded in as much detail as you want, right
down to the underlying axioms of logic and set theory (in the case of the set
theory database provided).  While Metamath will not replace the higher-level
understanding that can only be acquired through exercises and hard work, being
able to see how gaps in a proof are filled in can give you increased
confidence that can speed up the learning process and save you time when you
get stuck.

The Metamath language breaks down a mathematical proof into its tiniest
possible parts.  These can be pieced together, like interlocking
pieces in a puzzle, only in a way that produces correct and absolutely rigorous
mathematics.

The nature of Metamath\index{Metamath} enforces very precise mathematical
thinking, similar to that involved in writing a computer program.  A crucial
difference, though, is that once a proof is verified (by the Metamath program)
to be correct, it is definitely correct; it can never have a hidden
``bug.''\index{computer program bugs}  After getting used to the kind of rigor
and accuracy provided by Metamath, you might even be tempted to
adopt the attitude that a proof should never be considered correct until it
has been verified by a computer, just as you would not completely trust a
manual calculation until you have verified it on a
calculator.

My goal
for Metamath was a system for describing and verifying
mathematics that is completely universal yet conceptually as simple as
possible.  In approaching mathematics from an axiomatic, formal viewpoint, I
wanted Metamath to be able to handle almost any mathematical system, not
necessarily with ease, but at least in principle and hopefully in practice. I
wanted it to verify proofs with absolute rigor, and for this reason Metamath
is what might be thought of as a ``compile-only'' language rather than an
algorithmic or Turing-machine language (Pascal, C, Prolog, Mathematica,
etc.).  In other words, a database written in the Metamath
language doesn't ``do'' anything; it merely exhibits mathematical knowledge
and permits this knowledge to be verified as being correct.  A program in an
algorithmic language can potentially have hidden bugs\index{computer program
bugs} as well as possibly being hard to understand.  But each token in a
Metamath database must be consistent with the database's earlier
contents according to simple, fixed rules.
If a database is verified
to be correct,\footnote{This includes
verification that a sequential list of proof steps results in the specified
theorem.} then the mathematical content is correct if the
verifier is correct and the axioms are correct.
The verification program could be incorrect, but the verification algorithm
is relatively simple (making it unlikely to be implemented incorrectly
by the Metamath program),
and there are over a dozen Metamath database verifiers
written by different people in different programming languages
(so these different verifiers can act as multiple reviewers of a database).
The most-used Metamath database, the Metamath Proof Explorer
(aka \texttt{set.mm}\index{set theory database (\texttt{set.mm})}%
\index{Metamath Proof Explorer}),
is currently verified by four different Metamath verifiers written by
four different people in four different languages, including the
original Metamath program described in this book.
The only ``bugs'' that can exist are in the statement of the axioms,
for example if the axioms are inconsistent (a famous problem shown to be
unsolvable by G\"{o}del's incompleteness theorem\index{G\"{o}del's
incompleteness theorem}).
However, real mathematical systems have very few axioms, and these can
be carefully studied.
All of this provides extraordinarily high confidence that the verified database
is in fact correct.

The Metamath program
doesn't prove theorems automatically but is designed to verify proofs
that you supply to it.
The underlying Metamath language is completely general and has no built-in,
preconceived notions about your formal system\index{formal system}, its logic
or its syntax.
For constructing proofs, the Metamath program has a Proof Assistant\index{Proof
Assistant} which helps you fill in some of a proof step's details, shows you
what choices you have at any step, and verifies the proof as you build it; but
you are still expected to provide the proof.

There are many other programs that can process or generate information
in the Metamath language, and more continue to be written.
This is in part because the Metamath language itself is very simple
and intentionally easy to automatically process.
Some programs, such as \texttt{mmj2}\index{mmj2}, include a proof assistant
that can automate some steps beyond what the Metamath program can do.
Mario Carneiro has developed an algorithm for converting proofs from
the OpenTheory interchange format, which can be translated to and from
any of the HOL family of proof languages (HOL4, HOL Light, ProofPower,
and Isabelle), into the
Metamath language.\cite{DBLP:journals/corr/Carneiro14}\index{Carneiro, Mario}.
Daniel Whalen has developed Holophrasm, which can automatically
prove many Metamath proofs using
machine learning\index{machine learning}\index{artificial intelligence}
approaches
(including multiple neural networks\index{neural networks})\cite{DBLP:journals/corr/Whalen16}\index{Whalen, Daniel}.
However,
a discussion of these other programs is beyond the scope of this book.

Like most computer languages, the Metamath\index{Metamath} language uses the
standard ({\sc ascii}) characters on a computer keyboard, so it cannot
directly represent many of the special symbols that mathematicians use.  A
useful feature of the Metamath program is its ability to convert its notation
into the \LaTeX\ typesetting language.\index{latex@{\LaTeX}}  This feature
lets you convert the {\sc ascii} tokens you've defined into standard
mathematical symbols, so you end up with symbols and formulas you are familiar
with instead of somewhat cryptic {\sc ascii} representations of them.
The Metamath program can also generate HTML\index{HTML}, making it easy
to view results on the web and to see related information by using
hypertext links.

Metamath is probably conceptually different from anything you've seen
before and some aspects may take some getting used to.  This book will
help you decide whether Metamath suits your specific needs.

\subsubsection{Setting Your Expectations}
It is important for you to understand what Metamath\index{Metamath} is and is
not.  As mentioned, the Metamath program
is {\em not} an automated theorem prover but
rather a proof verifier.  Developing a database can be tedious, hard work,
especially if you want to make the proofs as short as possible, but it becomes
easier as you build up a collection of useful theorems.  The purpose of
Metamath is simply to document existing mathematics in an absolutely rigorous,
computer-verifiable way, not to aid directly in the creation of new
mathematics.  It also is not a magic solution for learning abstract
mathematics, although it may be helpful to be able to actually see the implied
rigor behind what you are learning from textbooks, as well as providing hints
to work out proofs that you are stumped on.

As of this writing, a sizable set theory database has been developed to
provide a foundation for many fields of mathematics, but much more work would
be required to develop useful databases for specific fields.

Metamath\index{Metamath} ``knows no math;'' it just provides a framework in
which to express mathematics.  Its language is very small.  You can define two
kinds of symbols, constants\index{constant} and variables\index{variable}.
The only thing Metamath knows how to do is to substitute strings of symbols
for the variables\index{substitution!variable}\index{variable substitution} in
an expression based on instructions you provide it in a proof, subject to
certain constraints you specify for the variables.  Even the decimal
representation of a number is merely a string of certain constants (digits)
which together, in a specific context, correspond to whatever mathematical
object you choose to define for it; unlike other computer languages, there is
no actual number stored inside the computer.  In a proof, you in effect
instruct Metamath what symbol substitutions to make in previous axioms or
theorems and join a sequence of them together to result in the desired
theorem.  This kind of symbol manipulation captures the essence of mathematics
at a preaxiomatic level.

\subsubsection{Metamath and Mathematical Literature}

In advanced mathematical literature, proofs are usually presented in the form
of short outlines that often only an expert can follow.  This is partly out of
a desire for brevity, but it would also be unwise (even if it were practical)
to present proofs in complete formal detail, since the overall picture would
be lost.\index{formal proof}

A solution I envision\label{envision} that would allow mathematics to remain
acceptable to the expert, yet increase its accessibility to non-specialists,
consists of a combination of the traditional short, informal proof in print
accompanied by a complete formal proof stored in a computer database.  In an
analogy with a computer program, the informal proof is like a set of comments
that describe the overall reasoning and content of the proof, whereas the
computer database is like the actual program and provides a means for anyone,
even a non-expert, to follow the proof in as much detail as desired, exploring
it back through layers of theorems (like subroutines that call other
subroutines) all the way back to the axioms of the theory.  In addition, the
computer database would have the advantage of providing absolute assurance
that the proof is correct, since each step can be verified automatically.

There are several other approaches besides Metamath to a project such
as this.  Section~\ref{proofverifiers} discusses some of these.

To us, a noble goal would be a database with hundreds of thousands of
theorems and their computer-verifiable proofs, encompassing a significant
fraction of known mathematics and available for instant access.
These would be fully verified by multiple independently-implemented verifiers,
to provide extremely high confidence that the proofs are completely correct.
The database would allow people to investigate whatever details they were
interested in, so that they could confirm whatever portions they wished.
Whether or not Metamath is an appropriate choice remains to be seen, but in
principle we believe it is sufficient.

\subsubsection{Formalism}

Over the past fifty years, a group of French mathematicians working
collectively under the pseudonym of Bourbaki\index{Bourbaki, Nicolas} have
co-authored a series of monographs that attempt to rigorously and
consistently formalize large bodies of mathematics from foundations.  On the
one hand, certainly such an effort has its merits; on the other hand, the
Bourbaki project has been criticized for its ``scholasticism'' and
``hyperaxiomatics'' that hide the intuitive steps that lead to the results
\cite[p.~191]{Barrow}\index{Barrow, John D.}.

Metamath unabashedly carries this philosophy to its extreme and no doubt is
subject to the same kind of criticism.  Nonetheless I think that in
conjunction with conventional approaches to mathematics Metamath can serve a
useful purpose.  The Bourbaki approach is essentially pedagogic, requiring the
reader to become intimately familiar with each detail in a very large
hierarchy before he or she can proceed to the next step.  The difference with
Metamath is that the ``reader'' (user) knows that all details are contained in
its computer database, available as needed; it does not demand that the user
know everything but conveniently makes available those portions that are of
interest.  As the body of all mathematical knowledge grows larger and larger,
no one individual can have a thorough grasp of its entirety.  Metamath
can finalize and put to rest any questions about the validity of any part of it
and can make any part of it accessible, in principle, to a non-specialist.

\subsubsection{A Personal Note}
Why did I develop Metamath\index{Metamath}?  I enjoy abstract mathematics, but
I sometimes get lost in a barrage of definitions and start to lose confidence
that my proofs are correct.  Or I reach a point where I lose sight of how
anything I'm doing relates to the axioms that a theory is based on and am
sometimes suspicious that there may be some overlooked implicit axiom
accidentally introduced along the way (as happened historically with Euclidean
geometry\index{Euclidean geometry}, whose omission of Pasch's
axiom\index{Pasch's axiom} went unnoticed for 2000 years
\cite[p.~160]{Davis}!). I'm also somewhat lazy and wish to avoid the effort
involved in re-verifying the gaps in informal proofs ``left to the reader;'' I
prefer to figure them out just once and not have to go through the same
frustration a year from now when I've forgotten what I did.  Metamath provides
better recovery of my efforts than scraps of paper that I can't
decipher anymore.  But mostly I find very appealing the idea of rigorously
archiving mathematical knowledge in a computer database, providing precision,
certainty, and elimination of human error.

\subsubsection{Note on Bibliography and Index}

The Bibliography usually includes the Library of Congress classification
for a work to make it easier for you to find it in on a university
library shelf.  The Index has author references to pages where their works
are cited, even though the authors' names may not appear on those pages.

\subsubsection{Acknowledgments}

Acknowledgments are first due to my wife, Deborah (who passed away on
September 4, 1998), for critiquing the manu\-script but most of all for
her patience and support.  I also wish to thank Joe Wright, Richard
Becker, Clarke Evans, Buddha Buck, and Jeremy Henty for helpful
comments.  Any errors, omissions, and other shortcomings are of course
my responsibility.

\subsubsection{Note Added June 22, 2005}\label{note2002}

The original, unpublished version of this book was written in 1997 and
distributed via the web.  The present edition has been updated to
reflect the current Metamath program and databases, as well as more
current {\sc url}s for Internet sites.  Thanks to Josh
Purinton\index{Purinton, Josh}, One Hand
Clapping, Mel L.\ O'Cat, and Roy F. Longton for pointing out
typographical and other errors.  I have also benefitted from numerous
discussions with Raph Levien\index{Levien, Raph}, who has extended
Metamath's philosophy of rigor to result in his {\em
Ghilbert}\index{Ghilbert} proof language (\url{http://ghilbert.org}).

Robert (Bob) Solovay\index{Solovay, Robert} communicated a new result of
A.~R.~D.~Mathias on the system of Bourbaki, and the text has been
updated accordingly (p.~\pageref{bourbaki}).

Bob also pointed out a clarification of the literature regarding
category theory and inaccessible cardinals\index{category
theory}\index{cardinal, inaccessible} (p.~\pageref{categoryth}),
and a misleading statement was removed from the text.  Specifically,
contrary to a statement in previous editions, it is possible to express
``There is a proper class of inaccessible cardinals'' in the language of
ZFC.  This can be done as follows:  ``For every set $x$ there is an
inaccessible cardinal $\kappa$ such that $\kappa$ is not in $x$.''
Bob writes:\footnote{Private communication, Nov.~30, 2002.}
\begin{quotation}
     This axiom is how Grothendieck presents category theory.  To each
inaccessible cardinal $\kappa$ one associates a Grothendieck universe
\index{Grothendieck, Alexander} $U(\kappa)$.  $U(\kappa)$ consists of
those sets which lie in a transitive set of cardinality less than
$\kappa$.  Instead of the ``category of all groups,'' one works relative
to a universe [considering the category of groups of cardinality less
than $\kappa$].  Now the category whose objects are all categories
``relative to the universe $U(\kappa)$'' will be a category not
relative to this universe but to the next universe.

     All of the things category theorists like to do can be done in this
framework.  The only controversial point is whether the Grothen\-dieck
axiom is too strong for the needs of category theorists.  Mac Lane
\index{Mac Lane, Saunders} argues that ``one universe is enough'' and
Feferman\index{Feferman, Solomon} has argued that one can get by with
ordinary ZFC.  I don't find Feferman's arguments persuasive.  Mac Lane
may be right, but when I think about category theory I do it \`{a} la
Grothendieck.

        By the way Mizar\index{Mizar} adds the axiom ``there is a proper
class of inaccessibles'' precisely so as to do category theory.
\end{quotation}

The most current information on the Metamath program and databases can
always be found at \url{http://metamath.org}.


\subsubsection{Note Added June 24, 2006}\label{note2006}

The Metamath spec was restricted slightly to make parsers easier to
write.  See the footnote on p.~\pageref{namespace}.

%\subsubsection{Note Added July 24, 2006}\label{note2006b}
\subsubsection{Note Added March 10, 2007}\label{note2006b}

I am grateful to Anthony Williams\index{Williams, Anthony} for writing
the \LaTeX\ package called {\tt realref.sty} and contributing it to the
public domain.  This package allows the internal hyperlinks in a {\sc
pdf} file to anchor to specific page numbers instead of just section
titles, making the navigation of the {\sc pdf} file for this book much
more pleasant and ``logical.''

A typographical error found by Martin Kiselkov was corrected.
A confusing remark about unification was deleted per suggestion of
Mel O'Cat.

\subsubsection{Note Added May 27, 2009}\label{note2009}

Several typos found by Kim Sparre were corrected.  A note was added that
the Poincar\'{e} conjecture has been proved (p.~\pageref{poincare}).

\subsubsection{Note Added Nov. 17, 2014}\label{note2014}

The statement of the Schr\"{o}der--Bernstein theorem was corrected in
Section~\ref{trust}.  Thanks to Bob Solovay for pointing out the error.

\subsubsection{Note Added May 25, 2016}\label{note2016}

Thanks to Jerry James for correcting 16 typos.

\subsubsection{Note Added February 25, 2019}\label{note201902}

David A. Wheeler\index{Wheeler, David A.}
made a large number of improvements and updates,
in coordination with Norman Megill.
The predicate calculus axioms were renumbered, and the text makes
it clear that they are based on Tarski's system S2;
the one slight deviation in axiom ax-6 is explained and justified.
The real and complex number axioms were modified to be consistent with
\texttt{set.mm}\index{set theory database (\texttt{set.mm})}%
\index{Metamath Proof Explorer}.
Long-awaited specification changes ``1--8'' were made,
which clarified previously ambiguous points.
Some errors in the text involving \texttt{\$f} and
\texttt{\$d} statements were corrected (the spec was correct, but
the in-book explanations unintentionally contradicted the spec).
We now have a system for automatically generating narrow PDFs,
so that those with smartphones can have easy access to the current
version of this document.
A new section on deduction was added;
it discusses the standard deduction theorem,
the weak deduction theorem,
deduction style, and natural deduction.
Many minor corrections (too numerous to list here) were also made.

\subsubsection{Note Added March 7, 2019}\label{note201903}

This added a description of the Matamath language syntax in
Extended Backus--Naur Form (EBNF)\index{Extended Backus--Naur Form}\index{EBNF}
in Appendix \ref{BNF}, added a brief explanation about typecodes,
inserted more examples in the deduction section,
and added a variety of smaller improvements.

\subsubsection{Note Added April 7, 2019}\label{note201904}

This version clarified the proper substitution notation, improved the
discussion on the weak deduction theorem and natural deduction,
documented the \texttt{undo} command, updated the information on
\texttt{write source}, changed the typecode
from \texttt{set} to \texttt{setvar} to be consistent with the current
version of \texttt{set.mm}, added more documentation about comment markup
(e.g., documented how to create headings), and clarified the
differences between various assertion forms (in particular deduction form).

\subsubsection{Note Added May 20, 2019}\label{note201905}

This version fixes various small issues and added specific examples
of forms (deduction form, inference form, and closed form).

\chapter{Introduction}
\pagenumbering{arabic}

\begin{quotation}
  {\em {\em I.M.:}  No, no.  There's nothing subjective about it!  Everybody
knows what a proof is.  Just read some books, take courses from a competent
mathematician, and you'll catch on.

{\em Student:}  Are you sure?

{\em I.M.:}  Well---it is possible that you won't, if you don't have any
aptitude for it.  That can happen, too.

{\em Student:}  Then {\em you} decide what a proof is, and if I don't learn
to decide in the same way, you decide I don't have any aptitude.

{\em I.M.:}  If not me, then who?}
    \flushright\sc  ``The Ideal Mathematician''
    \index{Davis, Phillip J.}
    \footnote{\cite{Davis}, p.~40.}\\
\end{quotation}

In the past century, brilliant mathematicians have discovered almost
unimaginably profound results that rank among the crowning intellectual
achievements of mankind.  However, there is a sense in which modern abstract
mathematics is behind the times, stuck in an era before computers existed.
While no one disputes the remarkable results that have been achieved,
communicating these results in a precise way to the uninitiated is virtually
impossible.  To describe these results, a terse informal language is used which
despite its elegance is very difficult to learn.  This informal language is not
imprecise, far from it, but rather it often has omitted detail
and symbols with hidden context that are
implicitly understood by an expert but few others.  Extremely complex technical
meanings are associated with innocent-sounding English words such as
``compact'' and ``measurable'' that barely hint at what is actually being
said.  Anyone who does not keep the precise technical meaning constantly in
mind is bound to fail, and acquiring the ability to do this can be achieved
only through much practice and hard work.  Only the few who complete the
painful learning experience can join the small in-group of pure
mathematicians.  The informal language effectively cuts off the true nature of
their knowledge from most everyone else.

Metamath\index{Metamath} makes abstract mathematics more concrete.  It allows
a computer to keep track of the complexity associated with each word or symbol
with absolute rigor.  You can explore this complexity at your leisure, to
whatever degree you desire.  Whether or not you believe that concepts such as
infinity actually ``exist'' outside of the mind, Metamath lets you get to the
foundation for what's really being said.

Metamath also enables completely rigorous and thorough proof verification.
Its language is simple enough so that you
don't have to rely on the authority of experts but can verify the results
yourself, step by step.  If you want to attempt to derive your own results,
Metamath will not let you make a mistake in reasoning.
Even professional mathematicians make mistakes; Metamath makes it possible
to thoroughly verify that proofs are correct.

Metamath\index{Metamath} is a computer language and an associated computer
program for archiving, verifying, and studying mathematical proofs at a very
detailed level.
The Metamath language
describes formal\index{formal system} mathematical
systems and expresses proofs of theorems in those systems.  Such a language
is called a metalanguage\index{metalanguage} by mathematicians.
The Metamath program is a computer program that verifies
proofs expressed in the Metamath language.
The Metamath program does not have the built-in
ability to make logical inferences; it just makes a series of symbol
substitutions according to instructions given to it in a proof
and verifies that the result matches the expected theorem.  It makes logical
inferences based only on rules of logic that are contained in a set of
axioms\index{axiom}, or first principles, that you provide to it as the
starting point for proofs.

The complete specification of the Metamath language is only four pages long
(Section~\ref{spec}, p.~\pageref{spec}).  Its simplicity may at first make you
wonder how it can do much of anything at all.  But in fact the kinds of
symbol manipulations it performs are the ones that are implicitly done in all
mathematical systems at the lowest level.  You can learn it relatively quickly
and have complete confidence in any mathematical proof that it verifies.  On
the other hand, it is powerful and general enough so that virtually any
mathematical theory, from the most basic to the deeply abstract, can be
described with it.

Although in principle Metamath can be used with any
kind of mathematics, it is best suited for abstract or ``pure'' mathematics
that is mostly concerned with theorems and their proofs, as opposed to the
kind of mathematics that deals with the practical manipulation of numbers.
Examples of branches of pure mathematics are logic\index{logic},\footnote{Logic
is the study of statements that are universally true regardless of the objects
being described by the statements.  An example is the statement, ``if $P$
implies $Q$, then either $P$ is false or $Q$ is true.''} set theory\index{set
theory},\footnote{Set theory is the study of general-purpose mathematical objects called
``sets,'' and from it essentially all of mathematics can be derived.  For
example, numbers can be defined as specific sets, and their properties
can be explored using the tools of set theory.} number theory\index{number
theory},\footnote{Number theory deals with the properties of positive and
negative integers (whole numbers).} group theory\index{group
theory},\footnote{Group theory studies the properties of mathematical objects
called groups that obey a simple set of axioms and have properties of symmetry
that make them useful in many other fields.} abstract algebra\index{abstract
algebra},\footnote{Abstract algebra includes group theory and also studies
groups with additional properties that qualify them as ``rings'' and
``fields.''  The set of real numbers is a familiar example of a field.},
analysis\index{analysis} \index{real and complex numbers}\footnote{Analysis is
the study of real and complex numbers.} and
topology\index{topology}.\footnote{One area studied by topology are properties
that remain unchanged when geometrical objects undergo stretching
deformations; for example a doughnut and a coffee cup each have one hole (the
cup's hole is in its handle) and are thus considered topologically
equivalent.  In general, though, topology is the study of abstract
mathematical objects that obey a certain (surprisingly simple) set of axioms.
See, for example, Munkres \cite{Munkres}\index{Munkres, James R.}.} Even in
physics, Metamath could be applied to certain branches that make use of
abstract mathematics, such as quantum logic\index{quantum logic} (used to study
aspects of quantum mechanics\index{quantum mechanics}).

On the other hand, Metamath\index{Metamath} is less suited to applications
that deal primarily with intensive numeric computations.  Metamath does not
have any built-in representation of numbers\index{Metamath!representation of
numbers}; instead, a specific string of symbols (digits) must be syntactically
constructed as part of any proof in which an ordinary number is used.  For
this reason, numbers in Metamath are best limited to specific constants that
arise during the course of a theorem or its proof.  Numbers are only a tiny
part of the world of abstract mathematics.  The exclusion of built-in numbers
was a conscious decision to help achieve Metamath's simplicity, and there are
other software tools if you have different mathematical needs.
If you wish to quickly solve algebraic problems, the computer algebra
programs\index{computer algebra system} {\sc
macsyma}\index{macsyma@{\sc macsyma}}, Mathematica\index{Mathematica}, and
Maple\index{Maple} are specifically suited to handling numbers and
algebra efficiently.
If you wish to simply calculate numeric or matrix expressions easily,
tools such as Octave\index{Octave} may be a better choice.

After learning Metamath's basic statement types, any
tech\-ni\-cal\-ly ori\-ent\-ed person, mathematician or not, can
immediately trace
any theorem proved in the language as far back as he or she wants, all the way
to the axioms on which the theorem is based.  This ability suggests a
non-traditional way of learning about pure mathematics.  Used in conjunction
with traditional methods, Metamath could make pure mathematics accessible to
people who are not sufficiently skilled to figure out the implicit detail in
ordinary textbook proofs.  Once you learn the axioms of a theory, you can have
complete confidence that everything you need to understand a proof you are
studying is all there, at your beck and call, allowing you to focus in on any
proof step you don't understand in as much depth as you need, without worrying
about getting stuck on a step you can't figure out.\footnote{On the other
hand, writing proofs in the Metamath language is challenging, requiring
a degree of rigor far in excess of that normally taught to students.  In a
classroom setting, I doubt that writing Metamath proofs would ever replace
traditional homework exercises involving informal proofs, because the time
needed to work out the details would not allow a course to
cover much material.  For students who have trouble grasping the implied rigor
in traditional material, writing a few simple proofs in the Metamath language
might help clarify fuzzy thought processes.  Although somewhat difficult at
first, it eventually becomes fun to do, like solving a puzzle, because of the
instant feedback provided by the computer.}

Metamath\index{Metamath} is probably unlike anything you have
encountered before.  In this first chapter we will look at the philosophy and
use of computers in mathematics in order to better understand the motivation
behind Metamath.  The material in this chapter is not required in order to use
Metamath.  You may skip it if you are impatient, but I hope you will find it
educational and enjoyable.  If you want to start experimenting with the
Metamath program right away, proceed directly to Chapter~\ref{using}
(p.~\pageref{using}).  To
learn the Metamath language, skim Chapter~\ref{using} then proceed to
Chapter~\ref{languagespec} (p.~\pageref{languagespec}).

\section{Mathematics as a Computer Language}

\begin{quote}
  {\em The study of mathematics is apt to commence in
dis\-ap\-point\-ment.\ldots \\
We are told that by its aid the stars are weighted
and the billions of molecules in a drop of water are counted.  Yet, like the
ghost of Hamlet's father, this great science eludes the efforts of our mental
weapons to grasp it.}
  \flushright\sc  Alfred North Whitehead\footnote{\cite{Whitehead}, ch.\ 1.}\\
\end{quote}\index{Whitehead, Alfred North}

\subsection{Is Mathematics ``User-Friendly''?}

Suppose you have no formal training in abstract mathematics.  But popular
books you've read offer tempting glimpses of this world filled with profound
ideas that have stirred the human spirit.  You are not satisfied with the
informal, watered-down descriptions you've read but feel it is important to
grasp the underlying mathematics itself to understand its true meaning. It's
not practical to go back to school to learn it, though; you don't want to
dedicate years of your life to it.  There are many important things in life,
and you have to set priorities for what's important to you.  What would happen
if you tried to pursue it on your own, in your spare time?

After all, you were able to learn a computer programming language such as
Pascal on your own without too much difficulty, even though you had no formal
training in computers.  You don't claim to be an expert in software design,
but you can write a passable program when necessary to suit your needs.  Even
more important, you know that you can look at anyone else's Pascal program, no
matter how complex, and with enough patience figure out exactly how it works,
even though you are not a specialist.  Pascal allows you do anything that a
computer can do, at least in principle.  Thus you know you have the ability,
in principle, to follow anything that a computer program can do:  you just
have to break it down into small enough pieces.

Here's an imaginary scenario of what might happen if you na\-ive\-ly a\-dopted
this same view of abstract mathematics and tried to pick it up on your own, in
a period of time comparable to, say, learning a computer programming
language.

\subsubsection{A Non-Mathematician's Quest for Truth}

\begin{quote}
  {\em \ldots my daughters have been studying (chemistry) for several
se\-mes\-ters, think they have learned differential and integral calculus in
school, and yet even today don't know why $x\cdot y=y\cdot x$ is true.}
  \flushright\sc  Edmund Landau\footnote{\cite{Landau}, p.~vi.}\\
\end{quote}\index{Landau, Edmund}

\begin{quote}
  {\em Minus times minus is plus,\\
The reason for this we need not discuss.}
  \flushright\sc W.\ H.\ Auden\footnote{As quoted in \cite{Guillen}, p.~64.}\\
\end{quote}\index{Auden, W.\ H.}\index{Guillen, Michael}

We'll suppose you are a technically oriented professional, perhaps an engineer, a
computer programmer, or a physicist, but probably not a mathematician.  You
consider yourself reasonably intelligent.  You did well in school, learning a
variety of methods and techniques in practical mathematics such as calculus and
differential equations.  But rarely did your courses get into anything
resembling modern abstract mathematics, and proofs were something that appeared
only occasionally in your textbooks, a kind of necessary evil that was
supposed to convince you of a certain key result.  Most of your
homework consisted of exercises that gave you practice in the techniques, and
you were hardly ever asked to come up with a proof of your own.

You find yourself curious about advanced, abstract mathematics.  You are
driven by an inner conviction that it is important to understand and
appreciate some of the most profound knowledge discovered by mankind.  But it
seems very hard to learn, something that only certain gifted longhairs can
access and understand.  You are frustrated that it seems forever cut off from
you.

Eventually your curiosity drives you to do something about it.
You set for yourself a goal of ``really'' understanding mathematics:  not just
how to manipulate equations in algebra or calculus according to cookbook
rules, but rather to gain a deep understanding of where those rules come from.
In fact, you're not thinking about this kind of ordinary mathematics at all,
but about a much more abstract, ethereal realm of pure mathematics, where
famous results such as G\"{o}del's incompleteness theorem\index{G\"{o}del's
incompleteness theorem} and Cantor's different kinds of infinities
reside.

You have probably read a number of popular books, with titles like {\em
Infinity and the Mind} \cite{Rucker}\index{Rucker, Rudy}, on topics such as
these.  You found them inspiring but at the same time somewhat
unsatisfactory.  They gave you a general idea of what these results are about,
but if someone asked you to prove them, you wouldn't have the faintest idea of
where to begin.   Sure, you could give the same overall outline that you
learned from the popular books; and in a general sort of way, you do have an
understanding.  But deep down inside, you know that there is a rigor that is
missing, that probably there are many subtle steps and pitfalls along the way,
and ultimately it seems you have to place your trust in the experts in the
field.  You don't like this; you want to be able to verify these results for
yourself.

So where do you go next?  As a first step, you decide to look up some of the
original papers on the theorems you are curious about, or better, obtain some
standard textbooks in the field.  You look up a theorem you want to
understand.  Sure enough, it's there, but it's expressed with strange
terms and odd symbols that mean absolutely nothing to you.  It might as well be written in
a foreign language you've never seen before, whose symbols are totally alien.
You look at the proof, and you haven't the foggiest notion what each step
means, much less how one step follows from another.  Well, obviously you have
a lot to learn if you want to understand this stuff.

You feel that you could probably understand it by
going back to college for another three to six years and getting a math
degree.  But that does not fit in with your career and the other things in
your life and would serve no practical purpose.  You decide to seek a quicker
path.  You figure you'll just trace your way back to the beginning, step by
step, as you would do with a computer program, until you understand it.  But
you quickly find that this is not possible, since you can't even understand
enough to know what you have to trace back to.

Maybe a different approach is in order---maybe you should start at the
beginning and work your way up.  First, you read the introduction to the book
to find out what the prerequisites are.  In a similar fashion, you trace your
way back through two or three more books, finally arriving at one that seems
to start at a beginning:  it lists the axioms of arithmetic.  ``Aha!'' you
naively think, ``This must be the starting point, the source of all mathematical
knowledge.'' Or at least the starting point for mathematics dealing with
numbers; you have to start somewhere and have no idea what the starting point
for other mathematics would be.  But the word ``axioms'' looks promising.  So
you eagerly read along and work through some elementary exercises at the
beginning of the book.  You feel vaguely bothered:  these
don't seem like axioms at all, at least not in the sense that you want to
think of axioms.  Axioms imply a starting point from which everything else can
be built up, according to precise rules specified in the axiom system.  Even
though you can understand the first few proofs in an informal way,
and are able to do some of the
exercises, it's hard to pin down precisely what the
rules are.   Sure, each step seems to follow logically from the others, but
exactly what does that mean?  Is the ``logic'' just a matter of common sense,
something vague that we all understand but can never quite state precisely?

You've spent a number of years, off and on, programming computers, and you
know that in the case of computer languages there is no question of what the
rules are---they are precise and crystal clear.  If you follow them, your
program will work, and if you don't, it won't.  No matter how complex a
program, it can always be broken down into simpler and simpler pieces, until
you can ultimately identify the bits that are moved around to perform a
specific function.  Some programs might require a lot of perseverance to
accomplish this, but if you focus on a specific portion of it, you don't even
necessarily have to know how the rest of it works. Shouldn't there be an
analogy in mathematics?

You decide to apply the ultimate test:  you ask yourself how a computer could
verify or ensure that the steps in these proofs follow from one another.
Certainly mathematics must be at least as precisely defined as a computer
language, if not more so; after all, computer science itself is based on it.
If you can get a computer to verify these proofs, then you should also be
able, in principle, to understand them yourself in a very crystal clear,
precise way.

You're in for a surprise:  you can conceive of no way to convert the
proofs, which are in English, to a form that the computer can understand.
The proofs are filled with phrases such as ``assume there exists a unique
$x$\ldots'' and ``given any $y$, let $z$ be the number such that\ldots''  This
isn't the kind of logic you are used to in computer programming, where
everything, even arithmetic, reduces to Boolean ones and zeroes if you care to
break it down sufficiently.  Even though you think you understand the proofs,
there seems to be some kind of higher reasoning involved rather than precise
rules that define how you manipulate the symbols in the axioms.  Whatever it
is, it just isn't obvious how you would express it to a computer, and the more
you think about it, the more puzzled and confused you get, to the point where
you even wonder whether {\em you} really understand it.  There's a lot more to
these axioms of arithmetic than meets the eye.

Nobody ever talked about this in school in your applied math and engineering
courses.  You just learned the rules they gave you, not quite understanding
how or why they worked, sometimes vaguely suspicious or uncertain of them, and
through homework problems and osmosis learned how to present solutions that
satisfied the instructor and earned you an ``A.''  Rarely did you actually
``prove'' anything in a rigorous way, and the math majors who did do stuff
like that seemed to be in a different world.

Of course, there are computer algebra programs that can do mathematics, and
rather impressively.  They can instantly solve the integrals that you
struggled with in freshman calculus, and do much, much more.  But when you
look at these programs, what you see is a big collection of algorithms and
techniques that evolved and were added to over time, along with some basic
software that manipulates symbols.  Each algorithm that is built in is the
result of someone's theorem whose proof is omitted; you just have to trust the
person who proved it and the person who programmed it in and hope there are no
bugs.\index{computer program bugs}  Somehow this doesn't seem to be the
essence of mathematics.  Although computer algebra systems can generate
theorems with amazing speed, they can't actually prove a single one of them.

After some puzzlement, you revisit some popular books on what mathematics is
all about.  Somewhere you read that all of mathematics is actually derived
from something called ``set theory.''  This is a little confusing, because
nowhere in the book that presented the axioms of arithmetic was there any
mention of set theory, or if there was, it seemed to be just a tool that helps
you describe things better---the set of even numbers, that sort of thing.  If
set theory is the basis for all mathematics, then why are additional axioms
needed for arithmetic?

Something is wrong but you're not sure what.  One of your friends is a pure
mathematician.  He knows he is unable to communicate to you what he does for a
living and seems to have little interest in trying.  You do know that for him,
proofs are what mathematics is all about. You ask him what a proof is, and he
essentially tells you that, while of course it's based on logic, really it's
something you learn by doing it over and over until you pick it up.  He refers
you to a book, {\em How to Read and Do Proofs} \cite{Solow}.\index{Solow,
Daniel}  Although this book helps you understand traditional informal proofs,
there is still something missing you can't seem to pin down yet.

You ask your friend how you would go about having a computer verify a proof.
At first he seems puzzled by the question; why would you want to do that?
Then he says it's not something that would make any sense to do, but he's
heard that you'd have to break the proof down into thousands or even millions
of individual steps to do such a thing, because the reasoning involved is at
such a high level of abstraction.  He says that maybe it's something you could
do up to a point, but the computer would be completely impractical once you
get into any meaningful mathematics.  There, the only way you can verify a
proof is by hand, and you can only acquire the ability to do this by
specializing in the field for a couple of years in grad school.  Anyway, he
thinks it all has to do with set theory, although he has never taken a formal
course in set theory but just learned what he needed as he went along.

You are intrigued and amazed.  Apparently a mathematician can grasp as a
single concept something that would take a computer a thousand or a million
steps to verify, and have complete confidence in it.  Each one of these
thousand or million steps must be absolutely correct, or else the whole proof
is meaningless.  If you added a million numbers by hand, would you trust the
result?  How do you really know that all these steps are correct, that there
isn't some subtle pitfall in one of these million steps, like a bug in a
computer program?\index{computer program bugs}  After all, you've read that
famous mathematicians have occasionally made mistakes, and you certainly know
you've made your share on your math homework problems in school.

You recall the analogy with a computer program.  Sure, you can understand what
a large computer program such as a word processor does, as a single high-level
concept or a small set of such concepts, but your ability to understand it in
no way ensures that the program is correct and doesn't have hidden bugs.  Even
if you wrote the program yourself you can't really know this; most large
programs that you've written have had bugs that crop up at some later date, no
matter how careful you tried to be while writing them.

OK, so now it seems the reason you can't figure out how to make a
computer verify proofs is because each step really corresponds to a
million small steps.  Well, you say, a computer can do a million
calculations in a second, so maybe it's still practical to do.  Now the
puzzle becomes how to figure out what the million steps are that each
English-language step corresponds to.  Your mathematician friend hasn't
a clue, but suggests that maybe you would find the answer by studying
set theory.  Actually, your friend thinks you're a little off the wall
for even wondering such a thing.  For him, this is not what mathematics
is all about.

The subject of set theory keeps popping up, so you decide it's
time to look it up.

You decide to start off on a careful footing, so you start reading a couple of
very elementary books on set theory.  A lot of it seems pretty obvious, like
intersections, subsets, and Venn diagrams.  You thumb through one of the
books; nowhere is anything about axioms mentioned. The other book relegates to
an appendix a brief discussion that mentions a set of axioms called
``Zermelo--Fraenkel set theory''\index{Zermelo--Fraenkel set theory} and states
them in English.  You look at them and have no idea what they really mean or
what you can do with them.  The comments in this appendix say that the purpose
of mentioning them is to expose you to the idea, but imply that they are not
necessary for basic understanding and that they are really the subject matter
of advanced treatments where fine points such as a certain paradox (Russell's
paradox\index{Russell's paradox}\footnote{Russell's paradox assumes that there
exists a set $S$ that is a collection of all sets that don't contain
themselves.  Now, either $S$ contains itself or it doesn't.  If it contains
itself, it contradicts its definition.  But if it doesn't contain itself, it
also contradicts its definition.  Russell's paradox is resolved in ZF set
theory by denying that such a set $S$ exists.}) are resolved.  Wait a
minute---shouldn't the axioms be a starting point, not an ending point?  If
there are paradoxes that arise without the axioms, how do you know you won't
stumble across one accidentally when using the informal approach?

And nowhere do these books describe how ``all of mathematics can be
derived from set theory'' which by now you've heard a few times.

You find a more advanced book on set theory.  This one actually lists the
axioms of ZF set theory in plain English on page one.  {\em Now} you think
your quest has ended and you've finally found the source of all mathematical
knowledge; you just have to understand what it means.  Here, in one place, is
the basis for all of mathematics!  You stare at the axioms in awe, puzzle over
them, memorize them, hoping that if you just meditate on them long enough they
will become clear.  Of course, you haven't the slightest idea how the rest of
mathematics is ``derived'' from them; in particular, if these are the axioms
of mathematics, then why do arithmetic, group theory, and so on need their own
axioms?

You start reading this advanced book carefully, pondering the meaning of every
word, because by now you're really determined to get to the bottom of this.
The first thing the book does is explain how the axioms came about, which was
to resolve Russell's paradox.\index{Russell's paradox}  In fact that seems to
be the main purpose of their existence; that they supposedly can be used to
derive all of mathematics seems irrelevant and is not even mentioned.  Well,
you go on.  You hope the book will explain to you clearly, step by step, how
to derive things from the axioms.  After all, this is the starting point of
mathematics, like a book that explains the basics of a computer programming
language.  But something is missing.  You find you can't even understand the
first proof or do the first exercise.  Symbols such as $\exists$ and $\forall$
permeate the page without any mention of where they came from or how to
manipulate them; the author assumes you are totally familiar with them and
doesn't even tell you what they mean.  By now you know that $\exists$ means
``there exists'' and $\forall$ means ``for all,'' but shouldn't the rules for
manipulating these symbols be part of the axioms?  You still have no idea
how you could even describe the axioms to a computer.

Certainly there is something much different here from the technical
literature you're used to reading.  A computer language manual almost
always explains very clearly what all the symbols mean, precisely what
they do, and the rules used for combining them, and you work your way up
from there.

After glancing at four or five other such books, you come to the realization
that there is another whole field of study that you need just to get to the
point at which you can understand the axioms of set theory.  The field is
called ``logic.''  In fact, some of the books did recommend it as a
prerequisite, but it just didn't sink in.  You assumed logic was, well, just
logic, something that a person with common sense intuitively understood.  Why
waste your time reading boring treatises on symbolic logic, the manipulation
of 1's and 0's that computers do, when you already know that?  But this is a
different kind of logic, quite alien to you.  The subject of {\sc nand} and
{\sc nor} gates is not even touched upon or in any case has to do with only a
very small part of this field.

So your quest continues.  Skimming through the first couple of introductory
books, you get a general idea of what logic is about and what quantifiers
(``for all,'' ``there exists'') mean, but you find their examples somewhat
trivial and mildly annoying (``all dogs are animals,'' ``some animals are
dogs,'' and such).  But all you want to know is what the rules are for
manipulating the symbols so you can apply them to set theory.  Some formulas
describing the relationships among quantifiers ($\exists$ and $\forall$) are
listed in tables, along with some verbal reasoning to justify them.
Presumably, if you want to find out if a formula is correct, you go through
this same kind of mental reasoning process, possibly using images of dogs and
animals. Intuitively, the formulas seem to make sense.  But when you ask
yourself, ``What are the rules I need to get a computer to figure out whether
this formula is correct?'', you still don't know.  Certainly you don't ask the
computer to imagine dogs and animals.

You look at some more advanced logic books.  Many of them have an introductory
chapter summarizing set theory, which turns out to be a prerequisite.  You
need logic to understand set theory, but it seems you also need set theory to
understand logic!  These books jump right into proving rather advanced
theorems about logic, without offering the faintest clue about where the logic
came from that allows them to prove these theorems.

Luckily, you come across an elementary book of logic that, halfway through,
after the usual truth tables and metaphors, presents in a clear, precise way
what you've been looking for all along: the axioms!  They're divided into
propositional calculus (also called sentential logic) and predicate calculus
(also called first-order logic),\index{first-order logic} with rules so simple
and crystal clear that now you can finally program a computer to understand
them.  Indeed, they're no harder than learning how to play a game of chess.
As far as what you seem to need is concerned, the whole book could have been
written in five pages!

{\em Now} you think you've found the ultimate source of mathematical
truth.  So---the axioms of mathematics consist of these axioms of logic,
together with the axioms of ZF set theory. (By now you've also been able to
figure out how to translate the ZF axioms from English into the
actual symbols of logic which you can now manipulate according to
precise, easy-to-understand rules.)

Of course, you still don't understand how ``all of mathematics can be
derived from set theory,'' but maybe this will reveal itself in due
course.

You eagerly set out to program the axioms and rules into a computer and start
to look at the theorems you will have to prove as the logic is developed.  All
sorts of important theorems start popping up:  the deduction
theorem,\index{deduction theorem} the substitution theorem,\index{substitution
theorem} the completeness theorem of propositional calculus,\index{first-order
logic!completeness} the completeness theorem of predicate calculus.  Uh-oh,
there seems to be trouble.  They all get harder and harder, and not one of
them can be derived with the axioms and rules of logic you've just been
handed.  Instead, they all require ``metalogic'' for their proofs, a kind of
mixture of logic and set theory that allows you to prove things {\em about}
the axioms and theorems of logic rather than {\em with} them.

You plow ahead anyway.  A month later, you've spent much of your
free time getting the computer to verify proofs in propositional calculus.
You've programmed in the axioms, but you've also had to program in the
deduction theorem, the substitution theorem, and the completeness theorem of
propositional calculus, which by now you've resigned yourself to treating as
rather complex additional axioms, since they can't be proved from the axioms
you were given.  You can now get the computer to verify and even generate
complete, rigorous, formal proofs\index{formal proof}.  Never mind that they
may have 100,000 steps---at least now you can have complete, absolute
confidence in them.  Unfortunately, the only theorems you have proved are
pretty trivial and you can easily verify them in a few minutes with truth
tables, if not by inspection.

It looks like your mathematician friend was right.  Getting the computer to do
serious mathematics with this kind of rigor seems almost hopeless.  Even
worse, it seems that the further along you get, the more ``axioms'' you have
to add, as each new theorem seems to involve additional ``metamathematical''
reasoning that hasn't been formalized, and none of it can be derived from the
axioms of logic.  Not only do the proofs keep growing exponentially as you get
further along, but the program to verify them keeps getting bigger and bigger
as you program in more ``metatheorems.''\index{metatheorem}\footnote{A
metatheorem is usually a statement that is too general to be directly provable
in a theory.  For example, ``if $n_1$, $n_2$, and $n_3$ are integers, then
$n_1+n_2+n_3$ is an integer'' is a theorem of number theory.  But ``for any
integer $k > 1$, if $n_1, \ldots, n_k$ are integers, then $n_1+\ldots +n_k$ is
an integer'' is a metatheorem, in other words a family of theorems, one for
every $k$.  The reason it is not a theorem is that the general sum $n_1+\ldots
+n_k$ (as a function of $k$) is not an operation that can be defined directly
in number theory.} The bugs\index{computer program bugs} that have cropped up
so far have already made you start to lose faith in the rigor you seem to have
achieved, and you know it's just going to get worse as your program gets larger.

\subsection{Mathematics and the Non-Specialist}

\begin{quote}
  {\em A real proof is not checkable by a machine, or even by any mathematician
not privy to the gestalt, the mode of thought of the particular field of
mathematics in which the proof is located.}
  \flushright\sc  Davis and Hersh\index{Davis, Phillip J.}
  \footnote{\cite{Davis}, p.~354.}\\
\end{quote}

The bulk of abstract or theoretical mathematics is ordinarily outside
the reach of anyone but a few specialists in each field who have completed
the necessary difficult internship in order to enter its coterie.  The
typical intelligent layperson has no reasonable hope of understanding much of
it, nor even the specialist mathematician of understanding other fields.  It
is like a foreign language that has no dictionary to look up the translation;
the only way you can learn it is by living in the country for a few years.  It
is argued that the effort involved in learning a specialty is a necessary
process for acquiring a deep understanding.  Of course, this is almost certainly
true if one is to make significant contributions to a field; in particular,
``doing'' proofs is probably the most important part of a mathematician's
training.  But is it also necessary to deny outsiders access to it?  Is it
necessary that abstract mathematics be so hard for a layperson to grasp?

A computer normally is of no help whatsoever.  Most published proofs are
actually just series of hints written in an informal style that requires
considerable knowledge of the field to understand.  These are the ``real
proofs'' referred to by Davis and Hersh.\index{informal proof}  There is an
implicit understanding that, in principle, such a proof could be converted to
a complete formal proof\index{formal proof}.  However, it is said that no one
would ever attempt such a conversion, even if they could, because that would
presumably require millions of steps (Section~\ref{dream}).  Unfortunately the
informal style automatically excludes the understanding of the proof
by anyone who hasn't gone through the necessary apprenticeship. The
best that the intelligent layperson can do is to read popular books about deep
and famous results; while this can be helpful, it can also be misleading, and
the lack of detail usually leaves the reader with no ability whatsoever to
explore any aspect of the field being described.

The statements of theorems often use sophisticated notation that makes them
inaccessible to the non-specialist.  For a non-specialist who wants to achieve
a deeper understanding of a proof, the process of tracing definitions and
lemmas back through their hierarchy\index{hierarchy} quickly becomes confusing
and discouraging.  Textbooks are usually written to train mathematicians or to
communicate to people who are already mathematicians, and large gaps in proofs
are often left as exercises to the reader who is left at an impasse if he or
she becomes stuck.

I believe that eventually computers will enable non-specialists and even
intelligent laypersons to follow almost any mathematical proof in any field.
Metamath is an attempt in that direction.  If all of mathematics were as
easily accessible as a computer programming language, I could envision
computer programmers and hobbyists who otherwise lack mathematical
sophistication exploring and being amazed by the world of theorems and proofs
in obscure specialties, perhaps even coming up with results of their own.  A
tremendous advantage would be that anyone could experiment with conjectures in
any field---the computer would offer instant feedback as to whether
an inference step was correct.

Mathematicians sometimes have to put up with the annoyance of
cranks\index{cranks} who lack a fundamental understanding of mathematics but
insist that their ``proofs'' of, say, Fermat's Last Theorem\index{Fermat's
Last Theorem} be taken seriously.  I think part of the problem is that these
people are misled by informal mathematical language, treating it as if they
were reading ordinary expository English and failing to appreciate the
implicit underlying rigor.  Such cranks are rare in the field of computers,
because computer languages are much more explicit, and ultimately the proof is
in whether a computer program works or not.  With easily accessible
computer-based abstract mathematics, a mathematician could say to a crank,
``don't bother me until you've demonstrated your claim on the computer!''

% 22-May-04 nm
% Attempt to move De Millo quote so it doesn't separate from attribution
% CHANGE THIS NUMBER (AND ELIMINATE IF POSSIBLE) WHEN ABOVE TEXT CHANGES
\vspace{-0.5em}

\subsection{An Impossible Dream?}\label{dream}

\begin{quote}
  {\em Even quite basic theorems would demand almost unbelievably vast
  books to display their proofs.}
    \flushright\sc  Robert E. Edwards\footnote{\cite{Edwards}, p.~68.}\\
\end{quote}\index{Edwards, Robert E.}

\begin{quote}
  {\em Oh, of course no one ever really {\em does} it.  It would take
  forever!  You just show that you could do it, that's sufficient.}
    \flushright\sc  ``The Ideal Mathematician''
    \index{Davis, Phillip J.}\footnote{\cite{Davis},
p.~40.}\\
\end{quote}

\begin{quote}
  {\em There is a theorem in the primitive notation of set theory that
  corresponds to the arithmetic theorem `$1000+2000=3000$'.  The formula
  would be forbiddingly long\ldots even if [one] knows the definitions
  and is asked to simplify the long formula according to them, chances are
  he will make errors and arrive at some incorrect result.}
    \flushright\sc  Hao Wang\footnote{\cite{Wang}, p.~140.}\\
\end{quote}\index{Wang, Hao}

% 22-May-04 nm
% Attempt to move De Millo quote so it doesn't separate from attribution
% CHANGE THIS NUMBER (AND ELIMINATE IF POSSIBLE) WHEN ABOVE TEXT CHANGES
\vspace{-0.5em}

\begin{quote}
  {\em The {\em Principia Mathematica} was the crowning achievement of the
  formalists.  It was also the deathblow of the formalist view.\ldots
  {[Rus\-sell]} failed, in three enormous volumes, to get beyond the elementary
  facts of arithmetic.  He showed what can be done in principle and what
  cannot be done in practice.  If the mathematical process were really
  one of strict, logical progression, we would still be counting our
  fingers.\ldots
  One theoretician estimates\ldots that a demonstration of one of
  Ramanujan's conjectures assuming set theory and elementary analysis would
  take about two thousand pages; the length of a deduction from first principles
  is nearly in\-con\-ceiv\-a\-ble\ldots The probabilists argue that\ldots any
  very long proof can at best be viewed as only probably correct\ldots}
  \flushright\sc Richard de Millo et. al.\footnote{\cite{deMillo}, pp.~269,
  271.}\\
\end{quote}\index{de Millo, Richard}

A number of writers have conveyed the impression that the kind of absolute
rigor provided by Metamath\index{Metamath} is an impossible dream, suggesting
that a complete, formal verification\index{formal proof} of a typical theorem
would take millions of steps in untold volumes of books.  Even if it could be
done, the thinking sometimes goes, all meaning would be lost in such a
monstrous, tedious verification.\index{informal proof}\index{proof length}

These writers assume, however, that in order to achieve the kind of complete
formal verification they desire one must break down a proof into individual
primitive steps that make direct reference to the axioms.  This is
not necessary.  There is no reason not to make use of previously proved
theorems rather than proving them over and over.

Just as important, definitions\index{definition} can be introduced along
the way, allowing very complex formulas to be represented with few
symbols.  Not doing this can lead to absurdly long formulas.  For
example, the mere statement of
G\"{o}del's incompleteness theorem\index{G\"{o}del's
incompleteness theorem}, which can be expressed with a small number of
defined symbols, would require about 20,000 primitive symbols to express
it.\index{Boolos, George S.}\footnote{George S.\ Boolos, lecture at
Massachusetts Institute of Technology, spring 1990.} An extreme example
is Bourbaki's\label{bourbaki} language for set theory, which requires
4,523,659,424,929 symbols plus 1,179,618,517,981 disambiguatory links
(lines connecting symbol pairs, usually drawn below or above the
formula) to express the number
``one'' \cite{Mathias}.\index{Mathias, Adrian R. D.}\index{Bourbaki,
Nicolas}
% http://www.dpmms.cam.ac.uk/~ardm/

A hierarchy\index{hierarchy} of theorems and definitions permits an
exponential growth in the formula sizes and primitive proof steps to be
described with only a linear growth in the number of symbols used.  Of course,
this is how ordinary informal mathematics is normally done anyway, but with
Metamath\index{Metamath} it can be done with absolute rigor and precision.

\subsection{Beauty}


\begin{quote}
  {\em No one shall be able to drive us from the paradise that Cantor has
created for us.}
   \flushright\sc  David Hilbert\footnote{As quoted in \cite{Moore}, p.~131.}\\
\end{quote}\index{Hilbert, David}

\needspace{3\baselineskip}
\begin{quote}
  {\em Mathematics possesses not only truth, but some supreme beauty ---a
  beauty cold and austere, like that of a sculpture.}
    \flushright\sc  Bertrand
    Russell\footnote{\cite{Russell}.}\\
\end{quote}\index{Russell, Bertrand}

\begin{quote}
  {\em Euclid alone has looked on Beauty bare.}
  \flushright\sc Edna Millay\footnote{As quoted in \cite{Davis}, p.~150.}\\
\end{quote}\index{Millay, Edna}

For most people, abstract mathematics is distant, strange, and
incomprehensible.  Many popular books have tried to convey some of the sense
of beauty in famous theorems.  But even an intelligent layperson is left with
only a general idea of what a theorem is about and is hardly given the tools
needed to make use of it.  Traditionally, it is only after years of arduous
study that one can grasp the concepts needed for deep understanding.
Metamath\index{Metamath} allows you to approach the proof of the theorem from
a quite different perspective, peeling apart the formulas and definitions
layer by layer until an entirely different kind of understanding is achieved.
Every step of the proof is there, pieced together with absolute precision and
instantly available for inspection through a microscope with a magnification
as powerful as you desire.

A proof in itself can be considered an object of beauty.  Constructing an
elegant proof is an art.  Once a famous theorem has been proved, often
considerable effort is made to find simpler and more easily understood
proofs.  Creating and communicating elegant proofs is a major concern of
mathematicians.  Metamath is one way of providing a common language for
archiving and preserving this information.

The length of a proof can, to a certain extent, be considered an
objective measure of its ``beauty,'' since shorter proofs are usually
considered more elegant.  In the set theory database
\texttt{set.mm}\index{set theory database (\texttt{set.mm})}%
\index{Metamath Proof Explorer}
provided with Metamath, one goal was to make all proofs as short as possible.

\needspace{4\baselineskip}
\subsection{Simplicity}

\begin{quote}
  {\em God made man simple; man's complex problems are of his own
  devising.}
    \flushright\sc Eccles. 7:29\footnote{Jerusalem Bible.}\\
\end{quote}\index{Bible}

\needspace{3\baselineskip}
\begin{quote}
  {\em God made integers, all else is the work of man.}
    \flushright\sc Leopold Kronecker\footnote{{\em Jahresbericht
	der Deutschen Mathematiker-Vereinigung }, vol. 2, p. 19.}\\
\end{quote}\index{Kronecker, Leopold}

\needspace{3\baselineskip}
\begin{quote}
  {\em For what is clear and easily comprehended attracts; the
  complicated repels.}
    \flushright\sc David Hilbert\footnote{As quoted in \cite{deMillo},
p.~273.}\\
\end{quote}\index{Hilbert, David}

The Metamath\index{Metamath} language is simple and Spartan.  Metamath treats
all mathematical expressions as simple sequences of symbols, devoid of meaning.
The higher-level or ``metamathematical'' notions underlying Metamath are about
as simple as they could possibly be.  Each individual step in a proof involves
a single basic concept, the substitution of an expression for a variable, so
that in principle almost anyone, whether mathematician or not, can
completely understand how it was arrived at.

In one of its most basic applications, Metamath\index{Metamath} can be used to
develop the foundations of mathematics\index{foundations of mathematics} from
the very beginning.  This is done in the set theory database that is provided
with the Metamath package and is the subject matter
of Chapter~\ref{fol}. Any language (a metalanguage\index{metalanguage})
used to describe mathematics (an object language\index{object language}) must
have a mathematical content of its own, but it is desirable to keep this
content down to a bare minimum, namely that needed to make use of the
inference rules specified by the axioms.  With any metalanguage there is a
``chicken and egg'' problem somewhat like circular reasoning:  you must assume
the validity of the mathematics of the metalanguage in order to prove the
validity of the mathematics of the object language.  The mathematical content
of Metamath itself is quite limited.  Like the rules of a game of chess, the
essential concepts are simple enough so that virtually anyone should be able to
understand them (although that in itself will not let you play like
a master).  The symbols that Metamath manipulates do not in themselves
have any intrinsic meaning.  Your interpretation of the axioms that you supply
to Metamath is what gives them meaning.  Metamath is an attempt to strip down
mathematical thought to its bare essence and show you exactly how the symbols
are manipulated.

Philosophers and logicians, with various motivations, have often thought it
important to study ``weak'' fragments of logic\index{weak logic}
\cite{Anderson}\index{Anderson, Alan Ross} \cite{MegillBunder}\index{Megill,
Norman}\index{Bunder, Martin}, other unconventional systems of logic (such as
``modal'' logic\index{modal logic} \cite[ch.\ 27]{Boolos}\index{Boolos, George
S.}), and quantum logic\index{quantum logic} in physics
\cite{Pavicic}\index{Pavi{\v{c}}i{\'{c}}, M.}.  Metamath\index{Metamath}
provides a framework in which such systems can be expressed, with an absolute
precision that makes all underlying metamathematical assumptions rigorous and
crystal clear.

Some schools of philosophical thought, for example
intuitionism\index{intuitionism} and constructivism\index{constructivism},
demand that the notions underlying any mathematical system be as simple and
concrete as possible.  Metamath should meet the requirements of these
philosophies.  Metamath must be taught the symbols, axioms\index{axiom}, and
rules\index{rule} for a specific theory, from the skeptical (such as
intuitionism\index{intuitionism}\footnote{Intuitionism does not accept the law
of excluded middle (``either something is true or it is not true'').  See
\cite[p.~xi]{Tymoczko}\index{Tymoczko, Thomas} for discussion and references
on this topic.  Consider the theorem, ``There exist irrational numbers $a$ and
$b$ such that $a^b$ is rational.''  An intuitionist would reject the following
proof:  If $\sqrt{2}^{\sqrt{2}}$ is rational, we are done.  Otherwise, let
$a=\sqrt{2}^{\sqrt{2}}$ and $b=\sqrt{2}$. Then $a^b=2$, which is rational.})
to the bold (such as the axiom of choice in set theory\footnote{The axiom of
choice\index{Axiom of Choice} asserts that given any collection of pairwise
disjoint nonempty sets, there exists a set that has exactly one element in
common with each set of the collection.  It is used to prove many important
theorems in standard mathematics.  Some philosophers object to it because it
asserts the existence of a set without specifying what the set contains
\cite[p.~154]{Enderton}\index{Enderton, Herbert B.}.  In one foundation for
mathematics due to Quine\index{Quine, Willard Van Orman}, that has not been
otherwise shown to be inconsistent, the axiom of choice turns out to be false
\cite[p.~23]{Curry}\index{Curry, Haskell B.}.  The \texttt{show
trace{\char`\_}back} command of the Metamath program allows you to find out
whether the axiom of choice, or any other axiom, was assumed by a
proof.}\index{\texttt{show trace{\char`\_}back} command}).

The simplicity of the Metamath language lets the algorithm (computer program)
that verifies the validity of a Metamath proof be straightforward and
robust.  You can have confidence that the theorems it verifies really can be
derived from your axioms.

\subsection{Rigor}

\begin{quote}
  {\em Rigor became a goal with the Greeks\ldots But the efforts to
  pursue rigor to the utmost have led to an impasse in which there is
  no longer any agreement on what it really means.  Mathematics remains
  alive and vital, but only on a pragmatic basis.}
    \flushright\sc  Morris Kline\footnote{\cite{Kline}, p.~1209.}\\
\end{quote}\index{Kline, Morris}

Kline refers to a much deeper kind of rigor than that which we will discuss in
this section.  G\"{o}del's incompleteness theorem\index{G\"{o}del's
incompleteness theorem} showed that it is impossible to achieve absolute rigor
in standard mathematics because we can never prove that mathematics is
consistent (free from contradictions).\index{consistent theory}  If
mathematics is consistent, we will never know it, but must rely on faith.  If
mathematics is inconsistent, the best we can hope for is that some clever
future mathematician will discover the inconsistency.  In this case, the
axioms would probably be revised slightly to eliminate the inconsistency, as
was done in the case of Russell's paradox,\index{Russell's paradox} but the
bulk of mathematics would probably not be affected by such a discovery.
Russell's paradox, for example, did not affect most of the remarkable results
achieved by 19th-century and earlier mathematicians.  It mainly invalidated
some of Gottlob Frege's\index{Frege, Gottlob} work on the foundations of
mathematics in the late 1800's; in fact Frege's work inspired Russell's
discovery.  Despite the paradox, Frege's work contains important concepts that
have significantly influenced modern logic.  Kline's {\em Mathematics, The
Loss of Certainty} \cite{Klinel}\index{Kline, Morris} has an interesting
discussion of this topic.

What {\em can} be achieved with absolute certainty\index{certainty} is the
knowledge that if we assume the axioms are consistent and true, then the
results derived from them are true.  Part of the beauty of mathematics is that
it is the one area of human endeavor where absolute certainty can be achieved
in this sense.  A mathematical truth will remain such for eternity.  However,
our actual knowledge of whether a particular statement is a mathematical truth
is only as certain as the correctness of the proof that establishes it.  If
the proof of a statement is questionable or vague, we can't have absolute
confidence in the truth that the statement claims.

Let us look at some traditional ways of expressing proofs.

Except in the field of formal logic\index{formal logic}, almost all
traditional proofs in mathematics are really not proofs at all, but rather
proof outlines or hints as to how to go about constructing the proof.  Many
gaps\index{gaps in proofs} are left for the reader to fill in. There are
several reasons for this.  First, it is usually assumed in mathematical
literature that the person reading the proof is a mathematician familiar with
the specialty being described, and that the missing steps are obvious to such
a reader or at least that the reader is capable of filling them in.  This
attitude is fine for professional mathematicians in the specialty, but
unfortunately it often has the drawback of cutting off the rest of the world,
including mathematicians in other specialties, from understanding the proof.
We discussed one possible resolution to this on p.~\pageref{envision}.
Second, it is often assumed that a complete formal proof\index{formal proof}
would require countless millions of symbols (Section~\ref{dream}). This might
be true if the proof were to be expressed directly in terms of the axioms of
logic and set theory,\index{set theory} but it is usually not true if we allow
ourselves a hierarchy\index{hierarchy} of definitions and theorems to build
upon, using a notation that allows us to introduce new symbols, definitions,
and theorems in a precisely specified way.

Even in formal logic,\index{formal logic} formal proofs\index{formal proof}
that are considered complete still contain hidden or implicit information.
For example, a ``proof'' is usually defined as a sequence of
wffs,\index{well-formed formula (wff)}\footnote{A {\em wff} or well-formed
formula is a mathematical expression (string of symbols) constructed according
to some precise rules.  A formal mathematical system\index{formal system}
contains (1) the rules for constructing syntactically correct
wffs,\index{syntax rules} (2) a list of starting wffs called
axioms,\index{axiom} and (3) one or more rules prescribing how to derive new
wffs, called theorems\index{theorem}, from the axioms or previously derived
theorems.  An example of such a system is contained in
Metamath's\index{Metamath} set theory database, which defines a formal
system\index{formal system} from which all of standard mathematics can be
derived.  Section~\ref{startf} steps you through a complete example of a formal
system, and you may want to skim it now if you are unfamiliar with the
concept.} each of which is an axiom or follows from a rule applied to previous
wffs in the sequence.  The implicit part of the proof is the algorithm by
which a sequence of symbols is verified to be a valid wff, given the
definition of a wff.  The algorithm in this case is rather simple, but for a
computer to verify the proof,\index{automated proof verification} it must have
the algorithm built into its verification program.\footnote{It is possible, of
course, to specify wff construction syntax outside of the program itself
with a suitable input language (the Metamath language being an example), but
some proof-verification or theorem-proving programs lack the ability to extend
wff syntax in such a fashion.} If one deals exclusively with axioms and
elementary wffs, it is straightforward to implement such an algorithm.  But as
more and more definitions are added to the theory in order to make the
expression of wffs more compact, the algorithm becomes more and more
complicated.  A computer program that implements the algorithm becomes larger
and harder to understand as each definition is introduced, and thus more prone
to bugs.\index{computer program bugs}  The larger the program, the
more suspicious the mathematician may be about
the validity of its algorithms.  This is especially true because
computer programs are inherently hard to follow to begin with, and few people
enjoy verifying them manually in detail.

Metamath\index{Metamath} takes a different approach.  Metamath's ``knowledge''
is limited to the ability to substitute variables for expressions, subject to
some simple constraints.  Once the basic algorithm of Metamath is assumed to
be debugged, and perhaps independently confirmed, it
can be trusted once and for all.  The information that Metamath needs to
``understand'' mathematics is contained entirely in the body of knowledge
presented to Metamath.  Any errors in reasoning can only be errors in the
axioms or definitions contained in this body of knowledge.  As a
``constructive'' language\index{constructive language} Metamath has no
conditional branches or loops like the ones that make computer programs hard
to decipher; instead, the language can only build new sequences of symbols
from earlier sequences  of symbols.

The simplicity of the rules that underlie Metamath not only makes Metamath
easy to learn but also gives Metamath a great deal of flexibility. For
example, Metamath is not limited to describing standard first-order
logic\index{first-order logic}; higher-order logics\index{higher-order logic}
and fragments of logic\index{weak logic} can be described just as easily.
Metamath gives you the freedom to define whatever wff notation you prefer; it
has no built-in conception of the syntax of a wff.\index{well-formed formula
(wff)}  With suitable axioms and definitions, Metamath can even describe and
prove things about itself.\index{Metamath!self-description}  (John
Harrison\index{Harrison, John} discusses the ``reflection''
principle\index{reflection principle} involved in self-descriptive systems in
\cite{Harrison}.)

The flexibility of Metamath requires that its proofs specify a lot of detail,
much more than in an ordinary ``formal'' proof.\index{formal proof}  For
example, in an ordinary formal proof, a single step consists of displaying the
wff that constitutes that step.  In order for a computer program to verify
that the step is acceptable, it first must verify that the symbol sequence
being displayed is an acceptable wff.\index{automated proof verification} Most
proof verifiers have at least basic wff syntax built into their programs.
Metamath has no hard-wired knowledge of what constitutes a wff built into it;
instead every wff must be explicitly constructed based on rules defining wffs
that are present in a database.  Thus a single step in an ordinary formal
proof may be correspond to many steps in a Metamath proof. Despite the larger
number of steps, though, this does not mean that a Metamath proof must be
significantly larger than an ordinary formal proof. The reason is that since
we have constructed the wff from scratch, we know what the wff is, so there is
no reason to display it.  We only need to refer to a sequence of statements
that construct it.  In a sense, the display of the wff in an ordinary formal
proof is an implicit proof of its own validity as a wff; Metamath just makes
the proof explicit. (Section~\ref{proof} describes Metamath's proof notation.)

\section{Computers and Mathematicians}

\begin{quote}
  {\em The computer is important, but not to mathematics.}
    \flushright\sc  Paul Halmos\footnote{As quoted in \cite{Albers}, p.~121.}\\
\end{quote}\index{Halmos, Paul}

Pure mathematicians have traditionally been indifferent to computers, even to
the point of disdain.\index{computers and pure mathematics}  Computer science
itself is sometimes considered to fall in the mundane realm of ``applied''
mathematics, perhaps essential for the real world but intellectually unexciting
to those who seek the deepest truths in mathematics.  Perhaps a reason for this
attitude towards computers is that there is little or no computer software that
meets their needs, and there may be a general feeling that such software could
not even exist.  On the one hand, there are the practical computer algebra
systems, which can perform amazing symbolic manipulations in algebra and
calculus,\index{computer algebra system} yet can't prove the simplest
existence theorem, if the idea of a proof is present at all.  On the other
hand, there are specialized automated theorem provers that technically speaking
may generate correct proofs.\index{automated theorem proving}  But sometimes
their specialized input notation may be cryptic and their output perceived to
be long, inelegant, incomprehensible proofs.    The output
may be viewed with suspicion, since the program that generates it tends to be
very large, and its size increases the potential for bugs\index{computer
program bugs}.  Such a proof may be considered trustworthy only if
independently verified and ``understood'' by a human, but no one wants to
waste their time on such a boring, unrewarding chore.



\needspace{4\baselineskip}
\subsection{Trusting the Computer}

\begin{quote}
  {\em \ldots I continue to find the quasi-empirical interpretation of
  computer proofs to be the more plausible.\ldots Since not
  everything that claims to be a computer proof can be
  accepted as valid, what are the mathematical criteria for acceptable
  computer proofs?}
    \flushright\sc  Thomas Tymoczko\footnote{\cite{Tymoczko}, p.~245.}\\
\end{quote}\index{Tymoczko, Thomas}

In some cases, computers have been essential tools for proving famous
theorems.  But if a proof is so long and obscure that it can be verified in a
practical way only with a computer, it is vaguely felt to be suspicious.  For
example, proving the famous four-color theorem\index{four-color
theorem}\index{proof length} (``a map needs no more than four colors to
prevent any two adjacent countries from having the same color'') can presently
only be done with the aid of a very complex computer program which originally
required 1200 hours of computer time. There has been considerable debate about
whether such a proof can be trusted and whether such a proof is ``real''
mathematics \cite{Swart}\index{Swart, E. R.}.\index{trusting computers}

However, under normal circumstances even a skeptical mathematician would have a
great deal of confidence in the result of multiplying two numbers on a pocket
calculator, even though the precise details of what goes on are hidden from its
user.  Even the verification on a supercomputer that a huge number is prime is
trusted, especially if there is independent verification; no one bothers to
debate the philosophical significance of its ``proof,'' even though the actual
proof would be so large that it would be completely impractical to ever write
it down on paper.  It seems that if the algorithm used by the computer is
simple enough to be readily understood, then the computer can be trusted.

Metamath\index{Metamath} adopts this philosophy.  The simplicity of its
language makes it easy to learn, and because of its simplicity one can have
essentially absolute confidence that a proof is correct. All axioms, rules, and
definitions are available for inspection at any time because they are defined
by the user; there are no hidden or built-in rules that may be prone to subtle
bugs\index{computer program bugs}.  The basic algorithm at the heart of
Metamath is simple and fixed, and it can be assumed to be bug-free and robust
with a degree of confidence approaching certainty.
Independently written implementations of the Metamath verifier
can reduce any residual doubt on the part of a skeptic even further;
there are now over a dozen such implementations, written by many people.

\subsection{Trusting the Mathematician}\label{trust}

\begin{quote}
  {\em There is no Algebraist nor Mathematician so expert in his science, as
  to place entire confidence in any truth immediately upon his discovery of it,
  or regard it as any thing, but a mere probability.  Every time he runs over
  his proofs, his confidence encreases; but still more by the approbation of
  his friends; and is rais'd to its utmost perfection by the universal assent
  and applauses of the learned world.}
  \flushright\sc David Hume\footnote{{\em A Treatise of Human Nature}, as
  quoted in \cite{deMillo}, p.~267.}\\
\end{quote}\index{Hume, David}

\begin{quote}
  {\em Stanislaw Ulam estimates that mathematicians publish 200,000 theorems
  every year.  A number of these are subsequently contradicted or otherwise
  disallowed, others are thrown into doubt, and most are ignored.}
  \flushright\sc Richard de Millo et. al.\footnote{\cite{deMillo}, p.~269.}\\
\end{quote}\index{Ulam, Stanislaw}

Whether or not the computer can be trusted, humans  of course will occasionally
err. Only the most memorable proofs get independently verified, and of these
only a handful of truly great ones achieve the status of being ``known''
mathematical truths that are used without giving a second thought to their
correctness.

There are many famous examples of incorrect theorems and proofs in
mathematical literature.\index{errors in proofs}

\begin{itemize}
\item There have been thousands of purported proofs of Fermat's Last
Theorem\index{Fermat's Last Theorem} (``no integer solutions exist to $x^n +
y^n = z^n$ for $n > 2$''), by amateurs, cranks, and well-regarded
mathematicians \cite[p.~5]{Stark}\index{Stark, Harold M}.  Fermat wrote a note
in his copy of Bachet's {\em Diophantus} that he found ``a truly marvelous
proof of this theorem but this margin is too narrow to contain it''
\cite[p.~507]{Kramer}.  A recent, much publicized proof by Yoichi
Miyaoka\index{Miyaoka, Yoichi} was shown to be incorrect ({\em Science News},
April 9, 1988, p.~230).  The theorem was finally proved by Andrew
Wiles\index{Wiles, Andrew} ({\em Science News}, July 3, 1993, p.~5), but it
initially had some gaps and took over a year after its announcement to be
checked thoroughly by experts.  On Oct. 25, 1994, Wiles announced that the last
gap found in his proof had been filled in.
  \item In 1882, M. Pasch discovered that an axiom was omitted from Euclid's
formulation of geometry\index{Euclidean geometry}; without it, the proofs of
important theorems of Euclid are not valid.  Pasch's axiom\index{Pasch's
axiom} states that a line that intersects one side of a triangle must also
intersect another side, provided that it does not touch any of the triangle's
vertices.  The omission of Pasch's axiom went unnoticed for 2000
years \cite[p.~160]{Davis}, in spite of (one presumes) the thousands of
students, instructors, and mathematicians who studied Euclid.
  \item The first published proof of the famous Schr\"{o}der--Bernstein
theorem\index{Schr\"{o}der--Bernstein theorem} in set theory was incorrect
\cite[p.~148]{Enderton}\index{Enderton, Herbert B.}.  This theorem states
that if there exists a 1-to-1 function\footnote{A {\em set}\index{set} is any
collection of objects. A {\em function}\index{function} or {\em
mapping}\index{mapping} is a rule that assigns to each element of one set
(called the function's {\em domain}\index{domain}) an element from another
set.} from set $A$ into set $B$ and vice-versa, then sets $A$ and $B$ have
a 1-to-1 correspondence.  Although it sounds simple and obvious,
the standard proof is quite long and complex.
  \item In the early 1900's, Hilbert\index{Hilbert, David} published a
purported proof of the continuum hypothesis\index{continuum hypothesis}, which
was eventually established as unprovable by Cohen\index{Cohen, Paul} in 1963
\cite[p.~166]{Enderton}.  The continuum hypothesis states that no
infinity\index{infinity} (``transfinite cardinal number'')\index{cardinal,
transfinite} exists whose size (or ``cardinality''\index{cardinality}) is
between the size of the set of integers and the size of the set of real
numbers.  This hypothesis originated with German mathematician Georg
Cantor\index{Cantor, Georg} in the late 1800's, and his inability to prove it
is said to have contributed to mental illness that afflicted him in his later
years.
  \item An incorrect proof of the four-color theorem\index{four-color theorem}
was published by Kempe\index{Kempe, A. B.} in 1879
\cite[p.~582]{Courant}\index{Courant, Richard}; it stood for 11 years before
its flaw was discovered.  This theorem states that any map can be colored
using only four colors, so that no two adjacent countries have the same
color.  In 1976 the theorem was finally proved by the famous computer-assisted
proof of Haken, Appel, and Koch \cite{Swart}\index{Appel, K.}\index{Haken,
W.}\index{Koch, K.}.  Or at least it seems that way.  Mathematician
H.~S.~M.~Coxeter\index{Coxeter, H. S. M.} has doubts \cite[p.~58]{Davis}:  ``I
have a feeling that that is an untidy kind of use of the computers, and the more
you correspond with Haken and Appel, the more shaky you seem to be.''
  \item Many false ``proofs'' of the Poincar\'{e}
conjecture\index{Poincar\'{e} conjecture} have been proposed over the years.
This conjecture states that any object that mathematically behaves like a
three-dimensional sphere is a three-dimensional sphere topologically,
regardless of how it is distorted.  In March 1986, mathematicians Colin
Rourke\index{Rourke, Colin} and Eduardo R\^{e}go\index{R\^{e}go, Eduardo}
caused  a stir in the mathematical community by announcing that they had found
a proof; in November of that year the proof was found to be false \cite[p.
218]{PetersonI}.  It was finally proved in 2003 by Grigory Perelman
\label{poincare}\index{Szpiro, George}\index{Perelman, Grigory}\cite{Szpiro}.
 \end{itemize}

Many counterexamples to ``theorems'' in recent mathematical
literature related to Clifford algebras\index{Clifford algebras}
 have been found by Pertti
Lounesto (who passed away in 2002).\index{Lounesto, Pertti}
See the web page \url{http://mathforum.org/library/view/4933.html}.
% http://users.tkk.fi/~ppuska/mirror/Lounesto/counterexamples.htm

One of the purposes of Metamath\index{Metamath} is to allow proofs to be
expressed with absolute precision.  Developing a proof in the Metamath
language can be challenging, because Metamath will not permit even the
tiniest mistake.\index{errors in proofs}  But once the proof is created, its
correctness can be trusted immediately, without having to depend on the
process of peer review for confirmation.

\section{The Use of Computers in Mathematics}

\subsection{Computer Algebra Systems}

For the most part, you will find that Metamath\index{Metamath} is not a
practical tool for manipulating numbers.  (Even proving that $2 + 2 = 4$, if
you start with set theory, can be quite complex!)  Several commercial
mathematics packages are quite good at arithmetic, algebra, and calculus, and
as practical tools they are invaluable.\index{computer algebra system} But
they have no notion of proof, and cannot understand statements starting with
``there exists such and such...''.

Software packages such as Mathematica \cite{Wolfram}\index{Mathematica} do not
concern themselves with proofs but instead work directly with known results.
These packages primarily emphasize heuristic rules such as the substitution of
equals for equals to achieve simpler expressions or expressions in a different
form.  Starting with a rich collection of built-in rules and algorithms, users
can add to the collection by means of a powerful programming language.
However, results such as, say, the existence of a certain abstract object
without displaying the actual object cannot be expressed (directly) in their
languages.  The idea of a proof from a small set of axioms is absent.  Instead
this software simply assumes that each fact or rule you add to the built-in
collection of algorithms is valid.  One way to view the software is as a large
collection of axioms from which the software, with certain goals, attempts to
derive new theorems, for example equating a complex expression with a simpler
equivalent. But the terms ``theorem''\index{theorem} and
``proof,''\index{proof} for example, are not even mentioned in the index of
the user's manual for Mathematica.\index{Mathematica and proofs}  What is also
unsatisfactory from a philosophical point of view is that there is no way to
ensure the validity of the results other than by trusting the writer of each
application module or tediously checking each module by hand, similar to
checking a computer program for bugs.\index{computer program
bugs}\footnote{Two examples illustrate why the knowledge database of computer
algebra systems should sometimes be regarded with a certain caution.  If you
ask Mathematica (version 3.0) to \texttt{Solve[x\^{ }n + y\^{ }n == z\^{ }n , n]}
it will respond with \texttt{\{\{n-\char`\>-2\}, \{n-\char`\>-1\},
\{n-\char`\>1\}, \{n-\char`\>2\}\}}. In other words, Mathematica seems to
``know'' that Fermat's Last Theorem\index{Fermat's Last Theorem} is true!  (At
the time this version of Mathematica was released this fact was unknown.)  If
you ask Maple\index{Maple} to \texttt{solve(x\^{ }2 = 2\^{ }x)} then
\texttt{simplify(\{"\})}, it returns the solution set \texttt{\{2, 4\}}, apparently
unaware that 0.7666647\ldots is also a solution.} While of course extremely
valuable in applied mathematics, computer algebra systems tend to be of little
interest to the theoretical mathematician except as aids for exploring certain
specific problems.

Because of possible bugs, trusting the output of a computer algebra system for
use as theorems in a proof-verifier would defeat the latter's goal of rigor.
On the other hand, a fact such that a certain relatively large number is
prime, while easy for a computer algebra system to derive, might have a long,
tedious proof that could overwhelm a proof-verifier. One approach for linking
computer algebra systems to a proof-verifier while retaining the advantages of
both is to add a hypothesis to each such theorem indicating its source.  For
example, a constant {\sc maple} could indicate the theorem came from the Maple
package, and would mean ``assuming Maple is consistent, then\ldots''  This and
many other topics concerning the formalization of mathematics are discussed in
John Harrison's\index{Harrison, John} very interesting
PhD thesis~\cite{Harrison-thesis}.

\subsection{Automated Theorem Provers}\label{theoremprovers}

A mathematical theory is ``decidable''\index{decidable theory} if a mechanical
method or algorithm exists that is guaranteed to determine whether or not a
particular formula is a theorem.  Among the few theories that are decidable is
elementary geometry,\index{Euclidean geometry} as was shown by a classic
result of logician Alfred Tarski\index{Tarski, Alfred} in 1948
\cite{Tarski}.\footnote{Tarski's result actually applies to a subset of the
geometry discussed in elementary textbooks.  This subset includes most of what
would be considered elementary geometry but it is not powerful enough to
express, among other things, the notions of the circumference and area of a
circle.  Extending the theory in a way that includes notions such as these
makes the theory undecidable, as was also shown by Tarski.  Tarski's algorithm
is far too inefficient to implement practically on a computer.  A practical
algorithm for proving a smaller subset of geometry theorems (those not
involving concepts of ``order'' or ``continuity'') was discovered by Chinese
mathematician Wu Wen-ts\"{u}n in 1977 \cite{Chou}\index{Chou,
Shang-Ching}.}\index{Wen-ts{\"{u}}n, Wu}  But most theories, including
elementary arithmetic, are undecidable.  This fact contributes to keeping
mathematics alive and well, since many mathematicians believe
that they will never be
replaced by computers (if they believe Roger Penrose's argument that a
computer can never replace the brain \cite{Penrose}\index{Penrose, Roger}).
In fact,  elementary geometry is often considered a ``dead'' field
for the simple reason that it is decidable.

On the other hand, the undecidability of a theory does not mean that one cannot
use a computer to search for proofs, providing one is willing to give up if a
proof is not found after a reasonable amount of time.  The field of automated
theorem proving\index{automated theorem proving} specializes in pursuing such
computer searches.  Among the more successful results to date are those based
on an algorithm known as Robinson's resolution principle
\cite{Robinson}\index{Robinson's resolution principle}.

Automated theorem provers can be excellent tools for those willing to learn
how to use them.  But they are not widely used in mainstream pure
mathematics, even though they could probably be useful in many areas.  There
are several reasons for this.  Probably most important, the main goal in pure
mathematics is to arrive at results that are considered to be deep or
important; proving them is essential but secondary.  Usually, an automated
theorem prover cannot assist in this main goal, and by the time the main goal
is achieved, the mathematician may have already figured out the proof as a
by-product.  There is also a notational problem.  Mathematicians are used to
using very compact syntax where one or two symbols (heavily dependent on
context) can represent very complex concepts; this is part of the
hierarchy\index{hierarchy} they have built up to tackle difficult problems.  A
theorem prover on the other hand might require that a theorem be expressed in
``first-order logic,''\index{first-order logic} which is the logic on which
most of mathematics is ultimately based but which is not ordinarily used
directly because expressions can become very long.  Some automated theorem
provers are experimental programs, limited in their use to very specialized
areas, and the goal of many is simply research into the nature of automated
theorem proving itself.  Finally, much research remains to be done to enable
them to prove very deep theorems.  One significant result was a
computer proof by Larry Wos\index{Wos, Larry} and colleagues that every Robbins
algebra\index{Robbins algebra} is a Boolean  algebra\index{Boolean algebra}
({\em New York Times}, Dec. 10, 1996).\footnote{In 1933, E.~V.\
Huntington\index{Huntington, E. V.}
presented the following axiom system for
Boolean algebra with a unary operation $n$ and a binary operation $+$:
\begin{center}
    $x + y = y + x$ \\
    $(x + y) + z = x + (y + z)$ \\
    $n(n(x) + y) + n(n(x) + n(y)) = x$
\end{center}
Herbert Robbins\index{Robbins, Herbert}, a student of Huntington, conjectured
that the last equation can be replaced with a simpler one:
\begin{center}
    $n(n(x + y) + n(x + n(y))) = x$
\end{center}
Robbins and Huntington could not find a proof.  The problem was
later studied unsuccessfully by Tarski\index{Tarski, Alfred} and his
students, and it remained an unsolved problem until a
computer found the proof in 1996.  For more information on
the Robbins algebra problem see \cite{Wos}.}

How does Metamath\index{Metamath} relate to automated theorem provers?  A
theorem prover is primarily concerned with one theorem at a time (perhaps
tapping into a small database of known theorems) whereas Metamath is more like
a theorem archiving system, storing both the theorem and its proof in a
database for access and verification.  Metamath is one answer to ``what do you
do with the output of a theorem prover?''  and could be viewed as the
next step in the process.  Automated theorem provers could be useful tools for
helping develop its database.
Note that very long, automatically
generated proofs can make your database fat and ugly and cause Metamath's proof
verification to take a long time to run.  Unless you have a particularly good
program that generates very concise proofs, it might be best to consider the
use of automatically generated proofs as a quick-and-dirty approach, to be
manually rewritten at some later date.

The program {\sc otter}\index{otter@{\sc otter}}\footnote{\url{http://www.cs.unm.edu/\~mccune/otter/}.}, later succeeded by
prover9\index{prover9}\footnote{\url{https://www.cs.unm.edu/~mccune/mace4/}.},
have been historically influential.
The E prover\index{E prover}\footnote{\url{https://github.com/eprover/eprover}.}
is a maintained automated theorem prover
for full first-order logic with equality.
There are many other automated theorem provers as well.

If you want to combine automated theorem provers with Metamath
consider investigating
the book {\em Automated Reasoning:  Introduction and Applications}
\cite{Wos}\index{Wos, Larry}.  This book discusses
how to use {\sc otter} in a way that can
not only able to generate
relatively efficient proofs, it can even be instructed to search for
shorter proofs.  The effective use of {\sc otter} (and similar tools)
does require a certain
amount of experience, skill, and patience.  The axiom system used in the
\texttt{set.mm}\index{set theory database (\texttt{set.mm})} set theory
database can be expressed to {\sc otter} using a method described in
\cite{Megill}.\index{Megill, Norman}\footnote{To use those axioms with
{\sc otter}, they must be restated in a way that eliminates the need for
``dummy variables.''\index{dummy variable!eliminating} See the Comment
on p.~\pageref{nodd}.} When successful, this method tends to generate
short and clever proofs, but my experiments with it indicate that the
method will find proofs within a reasonable time only for relatively
easy theorems.  It is still fun to experiment with.

Reference \cite{Bledsoe}\index{Bledsoe, W. W.} surveys a number of approaches
people have explored in the field of automated theorem proving\index{automated
theorem proving}.

\subsection{Interactive Theorem Provers}\label{interactivetheoremprovers}

Finding proofs completely automatically is difficult, so there
are some interactive theorem provers that allow a human to guide the
computer to find a proof.
Examples include
HOL Light\index{HOL light}%
\footnote{\url{https://www.cl.cam.ac.uk/~jrh13/hol-light/}.},
Isabelle\index{Isabelle}%
\footnote{\url{http://www.cl.cam.ac.uk/Research/HVG/Isabelle}.},
{\sc hol}\index{hol@{\sc hol}}%
\footnote{\url{https://hol-theorem-prover.org/}.},
and
Coq\index{Coq}\footnote{\url{https://coq.inria.fr/}.}.

A major difference between most of these tools and Metamath is that the
``proofs'' are actually programs that guide the program to find a proof,
and not the proof itself.
For example, an Isabelle/HOL proof might apply a step
\texttt{apply (blast dest: rearrange reduction)}. The \texttt{blast}
instruction applies
an automatic tableux prover and returns if it found a sequence of proof
steps that work... but the sequence is not considered part of the proof.

A good overview of
higher-level proof verification languages (such as {\sc lcf}\index{lcf@{\sc
lcf}} and {\sc hol}\index{hol@{\sc hol}})
is given in \cite{Harrison}.  All of these languages are fundamentally
different from Metamath in that much of the mathematical foundational
knowledge is embedded in the underlying proof-verification program, rather
than placed directly in the database that is being verified.
These can have a steep learning curve for those without a mathematical
background.  For example, one usually must have a fair understanding of
mathematical logic in order to follow their proofs.

\subsection{Proof Verifiers}\label{proofverifiers}

A proof verifier is a program that doesn't generate proofs but instead
verifies proofs that you give it.  Many proof verifiers have limited built-in
automated proof capabilities, such as figuring out simple logical inferences
(while still being guided by a person who provides the overall proof).  Metamath
has no built-in automated proof capability other than the limited
capability of its Proof Assistant.

Proof-verification languages are not used as frequently as they might be.
Pure mathematicians are more concerned with producing new results, and such
detail and rigor would interfere with that goal.  The use of computers in pure
mathematics is primarily focused on automated theorem provers (not verifiers),
again with the ultimate goal of aiding the creation of new mathematics.
Automated theorem provers are usually concerned with attacking one theorem at
time rather than making a large, organized database easily available to the
user.  Metamath is one way to help close this gap.

By itself Metamath is a mostly a proof verifier.
This does not mean that other approaches can't be used; the difference
is that in Metamath, the results of various provers must be recorded
step-by-step so that they can be verified.

Another proof-verification language is Mizar,\index{Mizar} which can display
its proofs in the informal language that mathematicians are accustomed to.
Information on the Mizar language is available at \url{http://mizar.org}.

For the working mathematician, Mizar is an excellent tool for rigorously
documenting proofs. Mizar typesets its proofs in the informal English used by
mathematicians (and, while fine for them, are just as inscrutable by
laypersons!). A price paid for Mizar is a relatively steep learning curve of a
couple of weeks.  Several mathematicians are actively formalizing different
areas of mathematics using Mizar and publishing the proofs in a dedicated
journal. Unfortunately the task of formalizing mathematics is still looked
down upon to a certain extent since it doesn't involve the creation of ``new''
mathematics.

The closest system to Metamath is
the {\em Ghilbert}\index{Ghilbert} proof language (\url{http://ghilbert.org})
system developed by
Raph Levien\index{Levien, Raph}.
Ghilbert is a formal proof checker heavily inspired by Metamath.
Ghilbert statements are s-expressions (a la Lisp), which is easy
for computers to parse but many people find them hard to read.
There are a number of differences in their specific constructs, but
there is at least one tool to translate some Metamath materials into Ghilbert.
As of 2019 the Ghilbert community is smaller and less active than the
Metamath community.
That said, the Metamath and Ghilbert communities overlap, and fruitful
conversations between them have occurred many times over the years.

\subsection{Creating a Database of Formalized Mathematics}\label{mathdatabase}

Besides Metamath, there are several other ongoing projects with the goal of
formalizing mathematics into computer-verifiable databases.
Understanding some history will help.

The {\sc qed}\index{qed project@{\sc qed} project}%
\footnote{\url{http://www-unix.mcs.anl.gov/qed}.}
project arose in 1993 and its goals were outlined in the
{\sc qed} manifesto.
The {\sc qed} manifesto was
a proposal for a computer-based database of all mathematical knowledge,
strictly formalized and with all proofs having been checked automatically.
The project had a conference in 1994 and another in 1995;
there was also a ``twenty years of the {\sc qed} manifesto'' workshop
in 2014.
Its ideals are regularly reraised.

In a 2007 paper, Freek Wiedijk identified two reasons
for the failure of the {\sc qed} project as originally envisioned:%
\cite{Wiedijk-revisited}\index{Wiedijk, Freek}

\begin{itemize}
\item Very few people are working on formalization of mathematics. There is no compelling application for fully mechanized mathematics.
\item Formalized mathematics does not yet resemble traditional mathematics. This is partly due to the complexity of mathematical notation, and partly to the limitations of existing theorem provers and proof assistants.
\end{itemize}

But this did not end the dream of
formalizing mathematics into computer-verifiable databases.
The problems that led to the {\sc qed} manifesto are still with us,
even though the challenges were harder than originally considered.
What has happened instead is that various independent projects have
worked towards formalizing mathematics into computer-verifiable databases,
each simultaneously competing and cooperating with each other.

A concrete way to see this is
Freek Wiedijk's ``Formalizing 100 Theorems'' list%
\footnote{\url{http://www.cs.ru.nl/\%7Efreek/100/}.}
which shows the progress different systems have made on a challenge list
of 100 mathematical theorems.%
\footnote{ This is not the only list of ``interesting'' theorems.
Another interesting list was posted by Oliver Knill's list
\cite{Knill}\index{Knill, Oliver}.}
The top systems as of February 2019
(in order of the number of challenges completed) are
HOL Light, Isabelle, Metamath, Coq, and Mizar.

The Metamath 100%
\footnote{\url{http://us.metamath.org/mm\_100.html}}
page (maintained by David A. Wheeler\index{Wheeler, David A.})
shows the progress of Metamath (specifically its \texttt{set.mm} database)
against this challenge list maintained by Freek Wiedijk.
The Metamath \texttt{set.mm} database
has made a lot of progress over the years,
in part because working to prove those challenge theorems required
defining various terms and proving their properties as a prerequisite.
Here are just a few of the many statements that have been
formally proven with Metamath:

% The entries of this cause the narrow display to break poorly,
% since the short amount of text means LaTeX doesn't get a lot to work with
% and the itemize format gives it even *less* margin than usual.
% No one will mind if we make just this list flushleft, since this list
% will be internally consistent.
\begin{flushleft}
\begin{itemize}
\item 1. The Irrationality of the Square Root of 2
  (\texttt{sqr2irr}, by Norman Megill, 2001-08-20)
\item 2. The Fundamental Theorem of Algebra
  (\texttt{fta}, by Mario Carneiro, 2014-09-15)
\item 22. The Non-Denumerability of the Continuum
  (\texttt{ruc}, by Norman Megill, 2004-08-13)
\item 54. The Konigsberg Bridge Problem
  (\texttt{konigsberg}, by Mario Carneiro, 2015-04-16)
\item 83. The Friendship Theorem
  (\texttt{friendship}, by Alexander W. van der Vekens, 2018-10-09)
\end{itemize}
\end{flushleft}

We thank all of those who have developed at least one of the Metamath 100
proofs, and we particularly thank
Mario Carneiro\index{Carneiro, Mario}
who has contributed the most Metamath 100 proofs as of 2019.
The Metamath 100 page shows the list of all people who have contributed a
proof, and links to graphs and charts showing progress over time.
We encourage others to work on proving theorems not yet proven in Metamath,
since doing so improves the work as a whole.

Each of the math formalization systems (including Metamath)
has different strengths and weaknesses, depending on what you value.
Key aspects that differentiate Metamath from the other top systems are:

\begin{itemize}
\item Metamath is not tied to any particular set of axioms.
\item Metamath can show every step of every proof, no exceptions.
  Most other provers only assert that a proof can be found, and do not
  show every step. This also makes verification fast, because
  the system does not need to rediscover proof details.
\item The Metamath verifier has been re-implemented in many different
  programming languages, so verification can be done by multiple
  implementations.  In particular, the
  \texttt{set.mm}\index{set theory database (\texttt{set.mm})}%
  \index{Metamath Proof Explorer} database is verified by
  four different verifiers
  written in four different languages by four different authors.
  This greatly reduces the risk of accepting an invalid
  proof due to an error in the verifier.
\item Proofs stay proven.  In some systems, changes to the system's
  syntax or how a tactic works causes proofs to fail in later versions,
  causing older work to become essentially lost.
  Metamath's language is
  extremely small and fixed, so once a proof is added to a database,
  the database can be rechecked with later versions of the Metamath program
  and with other verifiers of Metamath databases.
  If an axiom or key definition needs to be changed, it is easy to
  manipulate the database as a whole to handle the change
  without touching the underlying verifier.
  Since re-verification of an entire database takes seconds, there
  is never a reason to delay complete verification.
  This aspect is especially compelling if your
  goal is to have a long-term database of proofs.
\item Licensing is generous.  The main Metamath databases are released to
  the public domain, and the main Metamath program is open source software
  under a standard, widely-used license.
\item Substitutions are easy to understand, even by those who are not
  professional mathematicians.
\end{itemize}

Of course, other systems may have advantages over Metamath
that are more compelling, depending on what you value.
In any case, we hope this helps you understand Metamath
within a wider context.

\subsection{In Summary}\label{computers-summary}

To summarize our discussions of computers and mathematics, computer algebra
systems can be viewed as theorem generators focusing on a narrow realm of
mathematics (numbers and their properties), automated theorem provers as proof
generators for specific theorems in a much broader realm covered by a built-in
formal system such as first-order logic, interactive theorem
provers require human guidance, proof verifiers verify proofs but
historically they have been
restricted to first-order logic.
Metamath, in contrast,
is a proof verifier and documenter whose realm is essentially unlimited.

\section{Mathematics and Metamath}

\subsection{Standard Mathematics}

There are a number of ways that Metamath\index{Metamath} can be used with
standard mathematics.  The most satisfying way philosophically is to start at
the very beginning, and develop the desired mathematics from the axioms of
logic and set theory.\index{set theory}  This is the approach taken in the
\texttt{set.mm}\index{set theory database (\texttt{set.mm})}%
\index{Metamath Proof Explorer}
database (also known as the Metamath Proof Explorer).
Among other things, this database builds up to the
axioms of real and complex numbers\index{analysis}\index{real and complex
numbers} (see Section~\ref{real}), and a standard development of analysis, for
example, could start at that point, using it as a basis.   Besides this
philosophical advantage, there are practical advantages to having all of the
tools of set theory available in the supporting infrastructure.

On the other hand, you may wish to start with the standard axioms of a
mathematical theory without going through the set theoretical proofs of those
axioms.  You will need mathematical logic to make inferences, but if you wish
you can simply introduce theorems\index{theorem} of logic as
``axioms''\index{axiom} wherever you need them, with the implicit assumption
that in principle they can be proved, if they are obvious to you.  If you
choose this approach, you will probably want to review the notation used in
\texttt{set.mm}\index{set theory database (\texttt{set.mm})} so that your own
notation will be consistent with it.

\subsection{Other Formal Systems}
\index{formal system}

Unlike some programs, Metamath\index{Metamath} is not limited to any specific
area of mathematics, nor committed to any particular mathematical philosophy
such as classical logic versus intuitionism, nor limited, say, to expressions
in first-order logic.  Although the database \texttt{set.mm}
describes standard logic and set theory, Meta\-math
is actually a general-purpose language for describing a wide variety of formal
systems.\index{formal system}  Non-standard systems such as modal
logic,\index{modal logic} intuitionist logic\index{intuitionism}, higher-order
logic\index{higher-order logic}, quantum logic\index{quantum logic}, and
category theory\index{category theory} can all be described with the Metamath
language.  You define the symbols you prefer and tell Metamath the axioms and
rules you want to start from, and Metamath will verify any inferences you make
from those axioms and rules.  A simple example of a non-standard formal system
is Hofstadter's\index{Hofstadter, Douglas R.} MIU system,\index{MIU-system}
whose Metamath description is presented in Appendix~\ref{MIU}.

This is not hypothetical.
The largest Metamath database is
\texttt{set.mm}\index{set theory database (\texttt{set.mm}}%
\index{Metamath Proof Explorer}), aka the Metamath Proof Explorer,
which uses the most common axioms for mathematical foundations
(specifically classical logic combined with Zermelo--Fraenkel
set theory\index{Zermelo--Fraenkel set theory} with the Axiom of Choice).
But other Metamath databases are available:

\begin{itemize}
\item The database
  \texttt{iset.mm}\index{intuitionistic logic database (\texttt{iset.mm})},
  aka the
  Intuitionistic Logic Explorer\index{Intuitionistic Logic Explorer},
  uses intuitionistic logic (a constructivist point of view)
  instead of classical logic.
\item The database
  \texttt{nf.mm}\index{New Foundations database (\texttt{nf.mm})},
  aka the
  New Foundations Explorer\index{New Foundations Explorer},
  constructs mathematics from scratch,
  starting from Quine's New Foundations (NF) set theory axioms.
\item The database
  \texttt{hol.mm}\index{Higher-order Logic database (\texttt{hol.mm})},
  aka the
  Higher-Order Logic (HOL) Explorer\index{Higher-Order Logic (HOL) Explorer},
  starts with HOL (also called simple type theory) and derives
  equivalents to ZFC axioms, connecting the two approaches.
\end{itemize}

Since the days of David Hilbert,\index{Hilbert, David} mathematicians have
been concerned with the fact that the metalanguage\index{metalanguage} used to
describe mathematics may be stronger than the mathematics being described.
Metamath\index{Metamath}'s underlying finitary\index{finitary proof},
constructive nature provides a good philosophical basis for studying even the
weakest logics.\index{weak logic}

The usual treatment of many non-standard formal systems\index{formal
system} uses model theory\index{model theory} or proof theory\index{proof
theory} to describe these systems; these theories, in turn, are based on
standard set theory.  In other words, a non-standard formal system is defined
as a set with certain properties, and standard set theory is used to derive
additional properties of this set.  The standard set theory database provided
with Metamath can be used for this purpose, and when used this way
the development of a special
axiom system for the non-standard formal system becomes unnecessary.  The
model- or proof-theoretic approach often allows you to prove much deeper
results with less effort.

Metamath supports both approaches.  You can define the non-standard
formal system directly, or define the non-standard formal system as
a set with certain properties, whichever you find most helpful.

%\section{Additional Remarks}

\subsection{Metamath and Its Philosophy}

Closely related to Metamath\index{Metamath} is a philosophy or way of looking
at mathematics. This philosophy is related to the formalist
philosophy\index{formalism} of Hilbert\index{Hilbert, David} and his followers
\cite[pp.~1203--1208]{Kline}\index{Kline, Morris}
\cite[p.~6]{Behnke}\index{Behnke, H.}. In this philosophy, mathematics is
viewed as nothing more than a set of rules that manipulate symbols, together
with the consequences of those rules.  While the mathematics being described
may be complex, the rules used to describe it (the
``metamathematics''\index{metamathematics}) should be as simple as possible.
In particular, proofs should be restricted to dealing with concrete objects
(the symbols we write on paper rather than the abstract concepts they
represent) in a constructive manner; these are called ``finitary''
proofs\index{finitary proof} \cite[pp.~2--3]{Shoenfield}\index{Shoenfield,
Joseph R.}.

Whether or not you find Metamath interesting or useful will in part depend on
the appeal you find in its philosophy, and this appeal will probably depend on
your particular goals with respect to mathematics.  For example, if you are a
pure mathematician at the forefront of discovering new mathematical knowledge,
you will probably find that the rigid formality of Metamath stifles your
creativity.  On the other hand, we would argue that once this knowledge is
discovered, there are advantages to documenting it in a standard format that
will make it accessible to others.  Sixty years from now, your field may be
dormant, and as Davis and Hersh put it, your ``writings would become less
translatable than those of the Maya'' \cite[p.~37]{Davis}\index{Davis, Phillip
J.}.


\subsection{A History of the Approach Behind Metamath}

Probably the one work that has had the most motivating influence on
Metamath\index{Metamath} is Whitehead and Russell's monumental {\em Principia
Mathematica} \cite{PM}\index{Whitehead, Alfred North}\index{Russell,
Bertrand}\index{principia mathematica@{\em Principia Mathematica}}, whose aim
was to deduce all of mathematics from a small number of primitive ideas, in a
very explicit way that in principle anyone could understand and follow.  While
this work was tremendously influential in its time, from a modern perspective
it suffers from several drawbacks.  Both its notation and its underlying
axioms are now considered dated and are no longer used.  From our point of
view, its development is not really as accessible as we would like to see; for
practical reasons, proofs become more and more sketchy as its mathematics
progresses, and working them out in fine detail requires a degree of
mathematical skill and patience that many people don't have.  There are
numerous small errors, which is understandable given the tedious, technical
nature of the proofs and the lack of a computer to verify the details.
However, even today {\em Principia Mathematica} stands out as the work closest
in spirit to Metamath.  It remains a mind-boggling work, and one can't help
but be amazed at seeing ``$1+1=2$'' finally appear on page 83 of Volume II
(Theorem *110.643).

The origin of the proof notation used by Metamath dates back to the 1950's,
when the logician C.~A.~Meredith expressed his proofs in a compact notation
called ``condensed detachment''\index{condensed detachment}
\cite{Hindley}\index{Hindley, J. Roger} \cite{Kalman}\index{Kalman, J. A.}
\cite{Meredith}\index{Meredith, C. A.} \cite{Peterson}\index{Peterson, Jeremy
George}.  This notation allows proofs to be communicated unambiguously by
merely referencing the axiom\index{axiom}, rule\index{rule}, or
theorem\index{theorem} used at each step, without explicitly indicating the
substitutions\index{substitution!variable}\index{variable substitution} that
have to be made to the variables in that axiom, rule, or theorem.  Ordinarily,
condensed detachment is more or less limited to propositional
calculus\index{propositional calculus}.  The concept has been extended to
first-order logic\index{first-order logic} in \cite{Megill}\index{Megill,
Norman}, making it is easy to write a small computer program to verify proofs
of simple first-order logic theorems.\index{condensed detachment!and
first-order logic}

A key concept behind the notation of condensed detachment is called
``unification,''\index{unification} which is an algorithm for determining what
substitutions\index{substitution!variable}\index{variable substitution} to
variables have to be made to make two expressions match each other.
Unification was first precisely defined by the logician J.~A.~Robinson, who
used it in the development of a powerful
theorem-proving technique called the ``resolution principle''
\cite{Robinson}\index{Robinson's resolution principle}. Metamath does not make
use of the resolution principle, which is intended for systems of first-order
logic.\index{first-order logic}  Metamath's use is not restricted to
first-order logic, and as we have mentioned it does not automatically discover
proofs.  However, unification is a key idea behind Metamath's proof
notation, and Metamath makes use of a very simple version of it
(Section~\ref{unify}).

\subsection{Metamath and First-Order Logic}

First-order logic\index{first-order logic} is the supporting structure
for standard mathematics.  On top of it is set theory, which contains
the axioms from which virtually all of mathematics can be derived---a
remarkable fact.\index{category
theory}\index{cardinal, inaccessible}\label{categoryth}\footnote{An exception seems
to be category theory.  There are several schools of thought on whether
category theory is derivable from set theory.  At a minimum, it appears
that an additional axiom is needed that asserts the existence of an
``inaccessible cardinal'' (a type of infinity so large that standard set
theory can't prove or deny that it exists).
%
%%%% (I took this out that was in previous editions:)
% But it is also argued that not just one but a ``proper class'' of them
% is needed, and the existence of proper classes is impossible in standard
% set theory.  (A proper class is a collection of sets so huge that no set
% can contain it as an element.  Proper classes can lead to
% inconsistencies such as ``Russell's paradox.''  The axioms of standard
% set theory are devised so as to deny the existence of proper classes.)
%
For more information, see
\cite[pp.~328--331]{Herrlich}\index{Herrlich, Horst} and
\cite{Blass}\index{Blass, Andrea}.}

One of the things that makes Metamath\index{Metamath} more practical for
first-order theories is a set of axioms for first-order logic designed
specifically with Metamath's approach in mind.  These are included in
the database \texttt{set.mm}\index{set theory database (\texttt{set.mm})}.
See Chapter~\ref{fol} for a detailed
description; the axioms are shown in Section~\ref{metaaxioms}.  While
logically equivalent to standard axiom systems, our axiom system breaks
up the standard axioms into smaller pieces such that from them, you can
directly derive what in other systems can only be derived as higher-level
``metatheorems.''\index{metatheorem}  In other words, it is more powerful than
the standard axioms from a metalogical point of view.  A rigorous
justification for this system and its ``metalogical
completeness''\index{metalogical completeness} is found in
\cite{Megill}\index{Megill, Norman}.  The system is closely related to a
system developed by Monk\index{Monk, J. Donald} and Tarski\index{Tarski,
Alfred} in 1965 \cite{Monks}.

For example, the formula $\exists x \, x = y $ (given $y$, there exists some
$x$ equal to it) is a theorem of logic,\footnote{Specifically, it is a theorem
of those systems of logic that assume non-empty domains.  It is not a theorem
of more general systems that include the empty domain\index{empty domain}, in
which nothing exists, period!  Such systems are called ``free
logics.''\index{free logic} For a discussion of these systems, see
\cite{Leblanc}\index{Leblanc, Hugues}.  Since our use for logic is as a basis
for set theory, which has a non-empty domain, it is more convenient (and more
traditional) to use a less general system.  An interesting curiosity is that,
using a free logic as a basis for Zermelo--Fraenkel set
theory\index{Zermelo--Fraenkel set theory} (with the redundant Axiom of the
Null Set omitted),\index{Axiom of the Null Set} we cannot even prove the
existence of a single set without assuming the axiom of infinity!\index{Axiom
of Infinity}} whether or not $x$ and $y$ are distinct variables\index{distinct
variables}.  In many systems of logic, we would have to prove two theorems to
arrive at this result.  First we would prove ``$\exists x \, x = x $,'' then
we would separately prove ``$\exists x \, x = y $, where $x$ and $y$ are
distinct variables.''  We would then combine these two special cases ``outside
of the system'' (i.e.\ in our heads) to be able to claim, ``$\exists x \, x =
y $, regardless of whether $x$ and $y$ are distinct.''  In other words, the
combination of the two special cases is a
metatheorem.  In the system of logic
used in Metamath's set theory\index{set theory database (\texttt{set.mm})}
database, the axioms of logic are broken down into small pieces that allow
them to be reassembled in such a way that theorems such as these can be proved
directly.

Breaking down the axioms in this way makes them look peculiar and not very
intuitive at first, but rest assured that they are correct and complete.  Their
correctness is ensured because they are theorem schemes of standard first-order
logic (which you can easily verify if you are a logician).  Their completeness
follows from the fact that we explicitly derive the standard axioms of
first-order logic as theorems.  Deriving the standard axioms is somewhat
tricky, but once we're there, we have at our disposal a system that is less
awkward to work with in formal proofs\index{formal proof}.  In technical terms
that logicians understand, we eliminate the cumbersome concepts of ``free
variable,''\index{free variable} ``bound variable,''\index{bound variable} and
``proper substitution''\index{proper substitution}\index{substitution!proper}
as primitive notions.  These concepts are present in our system but are
defined in terms of concepts expressed by the axioms and can be eliminated in
principle.  In standard systems, these concepts are really like additional,
implicit axioms\index{implicit axiom} that are somewhat complex and cannot be
eliminated.

The traditional approach to logic, wherein free variables and proper
substitution is defined, is also possible to do directly in the Metamath
language.  However, the notation tends to become awkward, and there are
disadvantages:  for example, extending the definition of a wff with a
definition is awkward, because the free variable and proper substitution
concepts have to have their definitions also extended.  Our choice of
axioms for \texttt{set.mm} is to a certain extent a matter of style, in
an attempt to achieve overall simplicity, but you should also be aware
that the traditional approach is possible as well if you should choose
it.

\chapter{Using the Metamath Program}
\label{using}

\section{Installation}

The way that you install Metamath\index{Metamath!installation} on your
computer system will vary for different computers.  Current
instructions are provided with the Metamath program download at
\url{http://metamath.org}.  In general, the installation is simple.
There is one file containing the Metamath program itself.  This file is
usually called \texttt{metamath} or \texttt{metamath.}{\em xxx} where
{\em xxx} is the convention (such as \texttt{exe}) for an executable
program on your operating system.  There are several additional files
containing samples of the Metamath language, all ending with
\texttt{.mm}.  The file \texttt{set.mm}\index{set theory database
(\texttt{set.mm})} contains logic and set theory and can be used as a
starting point for other areas of mathematics.

You will also need a text editor\index{text editor} capable of editing plain
{\sc ascii}\footnote{American Standard Code for Information Interchange.} text
in order to prepare your input files.\index{ascii@{\sc ascii}}  Most computers
have this capability built in.  Note that plain text is not necessarily the
default for some word processing programs\index{word processor}, especially if
they can handle different fonts; for example, with Microsoft Word\index{Word
(Microsoft)}, you must save the file in the format ``Text Only With Line
Breaks'' to get a plain text\index{plain text} file.\footnote{It is
recommended that all lines in a Metamath source file be 79 characters or less
in length for compatibility among different computer terminals.  When creating
a source file on an editor such as Word, select a monospaced
font\index{monospaced font} such as Courier\index{Courier font} or
Monaco\index{Monaco font} to make this easier to achieve.  Better yet,
just use a plain text editor such as Notepad.}

On some computer systems, Metamath does not have the capability to print
its output directly; instead, you send its output to a file (using the
\texttt{open} commands described later).  The way you print this output
file depends on your computer.\index{printers} Some computers have a
print command, whereas with others, you may have to read the file into
an editor and print it from there.

If you want to print your Metamath source files with typeset formulas
containing standard mathematical symbols, you will need the \LaTeX\
typesetting program\index{latex@{\LaTeX}}, which is widely and freely
available for most operating systems.  It runs natively on Unix and
Linux, and can be installed on Windows as part of the free Cygwin
package (\url{http://cygwin.com}).

You can also produce {\sc html}\footnote{HyperText Markup Language.}
web pages.  The {\tt help html} command in the Metamath program will
assist you with this feature.

\section{Your First Formal System}\label{start}
\subsection{From Nothing to Zero}\label{startf}

To give you a feel for what the Metamath\index{Metamath} language looks like,
we will take a look at a very simple example from formal number
theory\index{number theory}.  This example is taken from
Mendelson\index{Mendelson, Elliot} \cite[p. 123]{Mendelson}.\footnote{To keep
the example simple, we have changed the formalism slightly, and what we call
axioms\index{axiom} are strictly speaking theorems\index{theorem} in
\cite{Mendelson}.}  We will look at a small subset of this theory, namely that
part needed for the first number theory theorem proved in \cite{Mendelson}.

First we will look at a standard formal proof\index{formal proof} for the
example we have picked, then we will look at the Metamath version.  If you
have never been exposed to formal proofs, the notation may seem to be such
overkill to express such simple notions that you may wonder if you are missing
something.  You aren't.  The concepts involved are in fact very simple, and a
detailed breakdown in this fashion is necessary to express the proof in a way
that can be verified mechanically.  And as you will see, Metamath breaks the
proof down into even finer pieces so that the mechanical verification process
can be about as simple as possible.

Before we can introduce the axioms\index{axiom} of the theory, we must define
the syntax rules for forming legal expressions\index{syntax rules}
(combinations of symbols) with which those axioms can be used. The number 0 is
a {\bf term}\index{term}; and if $ t$ and $r$ are terms, so is $(t+r)$. Here,
$ t$ and $r$ are ``metavariables''\index{metavariable} ranging over terms; they
themselves do not appear as symbols in an actual term.  Some examples of
actual terms are $(0 + 0)$ and $((0+0)+0)$.  (Note that our theory describes
only the number zero and sums of zeroes.  Of course, not much can be done with
such a trivial theory, but remember that we have picked a very small subset of
complete number theory for our example.  The important thing for you to focus
on is our definitions that describe how symbols are combined to form valid
expressions, and not on the content or meaning of those expressions.) If $ t$
and $r$ are terms, an expression of the form $ t=r$ is a {\bf wff}
(well-formed formula)\index{well-formed formula (wff)}; and if $P$ and $Q$ are
wffs, so is $(P\rightarrow Q)$ (which means ``$P$ implies
$Q$''\index{implication ($\rightarrow$)} or ``if $P$ then $Q$'').
Here $P$ and $Q$ are metavariables ranging over wffs.  Examples of actual
wffs are $0=0$, $(0+0)=0$, $(0=0 \rightarrow (0+0)=0)$, and $(0=0\rightarrow
(0=0\rightarrow 0=(0+0)))$.  (Our notation makes use of more parentheses than
are customary, but the elimination of ambiguity this way simplifies our
example by avoiding the need to define operator precedence\index{operator
precedence}.)

The {\bf axioms}\index{axiom} of our theory are all wffs of the following
form, where $ t$, $r$, and $s$ are any terms:

%Latex p. 92
\renewcommand{\theequation}{A\arabic{equation}}

\begin{equation}
(t=r\rightarrow (t=s\rightarrow r=s))
\end{equation}
\begin{equation}
(t+0)=t
\end{equation}

Note that there are an infinite number of axioms since there are an infinite
number of possible terms.  A1 and A2 are properly called ``axiom
schemes,''\index{axiom scheme} but we will refer to them as ``axioms'' for
brevity.

An axiom is a {\bf theorem}; and if $P$ and $(P\rightarrow Q)$ are theorems
(where $P$ and $Q$ are wffs), then $Q$ is also a theorem.\index{theorem}  The
second part of this definition is called the modus ponens (MP) rule of
inference\index{inference rule}\index{modus ponens}.  It allows us to obtain
new theorems from old ones.

The {\bf proof}\index{proof} of a theorem is a sequence of one or more
theorems, each of which is either an axiom or the result of modus ponens
applied to two previous theorems in the sequence, and the last of which is the
theorem being proved.

The theorem we will prove for our example is very simple:  $ t=t$.  The proof of
our theorem follows.  Study it carefully until you feel sure you
understand it.\label{zeroproof}

% Use tabu so that lines will wrap automatically as needed.
\begin{tabu} { l X X }
1. & $(t+0)=t$ & (by axiom A2) \\
2. & $(t+0)=t$ & (by axiom A2) \\
3. & $((t+0)=t \rightarrow ((t+0)=t\rightarrow t=t))$ & (by axiom A1) \\
4. & $((t+0)=t\rightarrow t=t)$ & (by MP applied to steps 2 and 3) \\
5. & $t=t$ & (by MP applied to steps 1 and 4) \\
\end{tabu}

(You may wonder why step 1 is repeated twice.  This is not necessary in the
formal language we have defined, but in Metamath's ``reverse Polish
notation''\index{reverse Polish notation (RPN)} for proofs, a previous step
can be referred to only once.  The repetition of step~1 here will enable you
to see more clearly the correspondence of this proof with the
Metamath\index{Metamath} version on p.~\pageref{demoproof}.)

Our theorem is more properly called a ``theorem scheme,''\index{theorem
scheme} for it represents an infinite number of theorems, one for each
possible term $ t$.  Two examples of actual theorems would be $0=0$ and
$(0+0)=(0+0)$.  Rarely do we prove actual theorems, since by proving schemes
we can prove an infinite number of theorems in one fell swoop.  Similarly, our
proof should really be called a ``proof scheme.''\index{proof scheme}  To
obtain an actual proof, pick an actual term to use in place of $ t$, and
substitute it for $ t$ throughout the proof.

Let's discuss what we have done here.  The axioms\index{axiom} of our theory,
A1 and A2, are trivial and obvious.  Everyone knows that adding zero to
something doesn't change it, and also that if two things are equal to a third,
then they are equal to each other. In fact, stating the trivial and obvious is
a goal to strive for in any axiomatic system.  From trivial and obvious truths
that everyone agrees upon, we can prove results that are not so obvious yet
have absolute faith in them.  If we trust the axioms and the rules, we must,
by definition, trust the consequences of those axioms and rules, if logic is
to mean anything at all.

Our rule of inference\index{rule}, modus ponens\index{modus ponens}, is also
pretty obvious once you understand what it means.  If we prove a fact $P$, and
we also prove that $P$ implies $Q$, then $Q$ necessarily follows as a new
fact.  The rule provides us with a means for obtaining new facts (i.e.\
theorems\index{theorem}) from old ones.

The theorem that we have proved, $ t=t$, is so fundamental that you may wonder
why it isn't one of the axioms\index{axiom}.  In some axiom systems of
arithmetic, it {\em is} an axiom.  The choice of axioms in a theory is to some
extent arbitrary and even an art form, constrained only by the requirement
that any two equivalent axiom systems be able to derive each other as
theorems.  We could imagine that the inventor of our axiom system originally
included $ t=t$ as an axiom, then discovered that it could be derived as a
theorem from the other axioms.  Because of this, it was not necessary to
keep it as an axiom.  By eliminating it, the final set of axioms became
that much simpler.

Unless you have worked with formal proofs\index{formal proof} before, it
probably wasn't apparent to you that $ t=t$ could be derived from our two
axioms until you saw the proof. While you certainly believe that $ t=t$ is
true, you might not be able to convince an imaginary skeptic who believes only
in our two axioms until you produce the proof.  Formal proofs such as this are
hard to come up with when you first start working with them, but after you get
used to them they can become interesting and fun.  Once you understand the
idea behind formal proofs you will have grasped the fundamental principle that
underlies all of mathematics.  As the mathematics becomes more sophisticated,
its proofs become more challenging, but ultimately they all can be broken down
into individual steps as simple as the ones in our proof above.

Mendelson's\index{Mendelson, Elliot} book, from which our example was taken,
contains a number of detailed formal proofs such as these, and you may be
interested in looking it up.  The book is intended for mathematicians,
however, and most of it is rather advanced.  Popular literature describing
formal proofs\index{formal proof} include \cite[p.~296]{Rucker}\index{Rucker,
Rudy} and \cite[pp.~204--230]{Hofstadter}\index{Hofstadter, Douglas R.}.

\subsection{Converting It to Metamath}\label{convert}

Formal proofs\index{formal proof} such as the one in our example break down
logical reasoning into small, precise steps that leave little doubt that the
results follow from the axioms\index{axiom}.  You might think that this would
be the finest breakdown we can achieve in mathematics.  However, there is more
to the proof than meets the eye. Although our axioms were rather simple, a lot
of verbiage was needed before we could even state them:  we needed to define
``term,'' ``wff,'' and so on.  In addition, there are a number of implied
rules that we haven't even mentioned. For example, how do we know that step 3
of our proof follows from axiom A1? There is some hidden reasoning involved in
determining this.  Axiom A1 has two occurrences of the letter $ t$.  One of
the implied rules states that whatever we substitute for $ t$ must be a legal
term\index{term}.\footnote{Some authors make this implied rule explicit by
stating, ``only expressions of the above form are terms,'' after defining
``term.''}  The expression $ t+0$ is pretty obviously a legal term whenever $
t$ is, but suppose we wanted to substitute a huge term with thousands of
symbols?  Certainly a lot of work would be involved in determining that it
really is a term, but in ordinary formal proofs all of this work would be
considered a single ``step.''

To express our axiom system in the Metamath\index{Metamath} language, we must
describe this auxiliary information in addition to the axioms themselves.
Metamath does not know what a ``term'' or a ``wff''\index{well-formed formula
(wff)} is.  In Metamath, the specification of the ways in which we can combine
symbols to obtain terms and wffs are like little axioms in themselves.  These
auxiliary axioms are expressed in the same notation as the ``real''
axioms\index{axiom}, and Metamath does not distinguish between the two.  The
distinction is made by you, i.e.\ by the way in which you interpret the
notation you have chosen to express these two kinds of axioms.

The Metamath language breaks down mathematical proofs into tiny pieces, much
more so than in ordinary formal proofs\index{formal proof}.  If a single
step\index{proof step} involves the
substitution\index{substitution!variable}\index{variable substitution} of a
complex term for one of its variables, Metamath must see this single step
broken down into many small steps.  This fine-grained breakdown is what gives
Metamath generality and flexibility as it lets it not be limited to any
particular mathematical notation.

Metamath's proof notation is not, in itself, intended to be read by humans but
rather is in a compact format intended for a machine.  The Metamath program
will convert this notation to a form you can understand, using the \texttt{show
proof}\index{\texttt{show proof} command} command.  You can tell the program what
level of detail of the proof you want to look at.  You may want to look at
just the logical inference steps that correspond
to ordinary formal proof steps,
or you may want to see the fine-grained steps that prove that an expression is
a term.

Here, without further ado, is our example converted to the
Metamath\index{Metamath} language:\index{metavariable}\label{demo0}

\begin{verbatim}
$( Declare the constant symbols we will use $)
    $c 0 + = -> ( ) term wff |- $.
$( Declare the metavariables we will use $)
    $v t r s P Q $.
$( Specify properties of the metavariables $)
    tt $f term t $.
    tr $f term r $.
    ts $f term s $.
    wp $f wff P $.
    wq $f wff Q $.
$( Define "term" and "wff" $)
    tze $a term 0 $.
    tpl $a term ( t + r ) $.
    weq $a wff t = r $.
    wim $a wff ( P -> Q ) $.
$( State the axioms $)
    a1 $a |- ( t = r -> ( t = s -> r = s ) ) $.
    a2 $a |- ( t + 0 ) = t $.
$( Define the modus ponens inference rule $)
    ${
       min $e |- P $.
       maj $e |- ( P -> Q ) $.
       mp  $a |- Q $.
    $}
$( Prove a theorem $)
    th1 $p |- t = t $=
  $( Here is its proof: $)
       tt tze tpl tt weq tt tt weq tt a2 tt tze tpl
       tt weq tt tze tpl tt weq tt tt weq wim tt a2
       tt tze tpl tt tt a1 mp mp
     $.
\end{verbatim}\index{metavariable}

A ``database''\index{database} is a set of one or more {\sc ascii} source
files.  Here's a brief description of this Metamath\index{Metamath} database
(which consists of this single source file), so that you can understand in
general terms what is going on.  To understand the source file in detail, you
should read Chapter~\ref{languagespec}.

The database is a sequence of ``tokens,''\index{token} which are normally
separated by spaces or line breaks.  The only tokens that are built into
the Metamath language are those beginning with \texttt{\$}.  These tokens
are called ``keywords.''\index{keyword}  All other tokens are
user-defined, and their names are arbitrary.

As you might have guessed, the Metamath token \texttt{\$(}\index{\texttt{\$(} and
\texttt{\$)} auxiliary keywords} starts a comment and \texttt{\$)} ends a comment.

The Metamath tokens \texttt{\$c}\index{\texttt{\$c} statement},
\texttt{\$v}\index{\texttt{\$v} statement},
\texttt{\$e}\index{\texttt{\$e} statement},
\texttt{\$f}\index{\texttt{\$f} statement},
\texttt{\$a}\index{\texttt{\$a} statement}, and
\texttt{\$p}\index{\texttt{\$p} statement} specify ``statements'' that
end with \texttt{\$.}\,.\index{\texttt{\$.}\ keyword}

The Metamath tokens \texttt{\$c} and \texttt{\$v} each declare\index{constant
declaration}\index{variable declaration} a list of user-defined tokens, called
``math symbols,''\index{math symbol} that the database will reference later
on.  All of the math symbols they define you have seen earlier except the
turnstile symbol \texttt{|-} ($\vdash$)\index{turnstile ({$\,\vdash$})}, which is
commonly used by logicians to mean ``a proof exists for.''  For us
the turnstile is just a
convenient symbol that distinguishes expressions that are axioms\index{axiom}
or theorems\index{theorem} from expressions that are terms or wffs.

The \texttt{\$c} statement declares ``constants''\index{constant} and
the \texttt{\$v} statement declares
``variables''\index{variable}\index{constant declaration}\index{variable
declaration} (or more precisely, metavariables\index{metavariable}).  A
variable may be substituted\index{substitution!variable}\index{variable
substitution} with sequences of math symbols whereas a constant may not
be substituted with anything.

It may seem redundant to require both \texttt{\$c}\index{\texttt{\$c} statement} and
\texttt{\$v}\index{\texttt{\$v} statement} statements (since any math
symbol\index{math symbol} not specified with a \texttt{\$c} statement could be
presumed to be a variable), but this provides for better error checking and
also allows math symbols to be redeclared\index{redeclaration of symbols}
(Section~\ref{scoping}).

The token \texttt{\$f}\index{\texttt{\$f} statement} specifies a
statement called a ``variable-type hypothesis'' (also called a
``floating hypothesis'') and \texttt{\$e}\index{\texttt{\$e} statement}
specifies a ``logical hypothesis'' (also called an ``essential
hypothesis'').\index{hypothesis}\index{variable-type
hypothesis}\index{logical hypothesis}\index{floating
hypothesis}\index{essential hypothesis} The token
\texttt{\$a}\index{\texttt{\$a} statement} specifies an ``axiomatic
assertion,''\index{axiomatic assertion} and
\texttt{\$p}\index{\texttt{\$p} statement} specifies a ``provable
assertion.''\index{provable assertion} To the left of each occurrence of
these four tokens is a ``label''\index{label} that identifies the
hypothesis or assertion for later reference.  For example, the label of
the first axiomatic assertion is \texttt{tze}.  A \texttt{\$f} statement
must contain exactly two math symbols, a constant followed by a
variable.  The \texttt{\$e}, \texttt{\$a}, and \texttt{\$p} statements
each start with a constant followed by, in general, an arbitrary
sequence of math symbols.

Associated with each assertion\index{assertion} is a set of hypotheses
that must be satisfied in order for the assertion to be used in a proof.
These are called the ``mandatory hypotheses''\index{mandatory
hypothesis} of the assertion.  Among those hypotheses whose ``scope''
(described below) includes the assertion, \texttt{\$e} hypotheses are
always mandatory and \texttt{\$f}\index{\texttt{\$f} statement}
hypotheses are mandatory when they share their variable with the
assertion or its \texttt{\$e} hypotheses.  The exact rules for
determining which hypotheses are mandatory are described in detail in
Sections~\ref{frames} and \ref{scoping}.  For example, the mandatory
hypotheses of assertion \texttt{tpl} are \texttt{tt} and \texttt{tr},
whereas assertion \texttt{tze} has no mandatory hypotheses because it
contains no variables and has no \texttt{\$e}\index{\texttt{\$e}
statement} hypothesis.  Metamath's \texttt{show statement}
command\index{\texttt{show statement} command}, described in the next
section, will show you a statement's mandatory hypotheses.

Sometimes we need to make a hypothesis relevant to only certain
assertions.  The set of statements to which a hypothesis is relevant is
called its ``scope.''  The Metamath brackets,
\texttt{\$\char`\{}\index{\texttt{\$\char`\{} and \texttt{\$\char`\}}
keywords} and \texttt{\$\char`\}}, define a ``block''\index{block} that
delimits the scope of any hypothesis contained between them.  The
assertion \texttt{mp} has mandatory hypotheses \texttt{wp}, \texttt{wq},
\texttt{min}, and \texttt{maj}.  The only mandatory hypothesis of
\texttt{th1}, on the other hand, is \texttt{tt}, since \texttt{th1}
occurs outside of the block containing \texttt{min} and \texttt{maj}.

Note that \texttt{\$\char`\{} and \texttt{\$\char`\}} do not affect the
scope of assertions (\texttt{\$a} and \texttt{\$p}).  Assertions are always
available to be referenced by any later proof in the source file.

Each provable assertion (\texttt{\$p}\index{\texttt{\$p} statement}
statement) has two parts.  The first part is the
assertion\index{assertion} itself, which is a sequence of math
symbol\index{math symbol} tokens placed between the \texttt{\$p} token
and a \texttt{\$=}\index{\texttt{\$=} keyword} token.  The second part
is a ``proof,'' which is a list of label tokens placed between the
\texttt{\$=} token and the \texttt{\$.}\index{\texttt{\$.}\ keyword}\
token that ends the statement.\footnote{If you've looked at the
\texttt{set.mm} database, you may have noticed another notation used for
proofs.  The other notation is called ``compressed.''\index{compressed
proof}\index{proof!compressed} It reduces the amount of space needed to
store a proof in the database and is described in
Appendix~\ref{compressed}.  In the example above, we use
``normal''\index{normal proof}\index{proof!normal} notation.} The proof
acts as a series of instructions to the Metamath program, telling it how
to build up the sequence of math symbols contained in the assertion part of
the \texttt{\$p} statement, making use of the hypotheses of the
\texttt{\$p} statement and previous assertions.  The construction takes
place according to precise rules.  If the list of labels in the proof
causes these rules to be violated, or if the final sequence that results
does not match the assertion, the Metamath program will notify you with
an error message.

If you are familiar with reverse Polish notation (RPN), which is sometimes used
on pocket calculators, here in a nutshell is how a proof works.  Each
hypothesis label\index{hypothesis label} in the proof is pushed\index{push}
onto the RPN stack\index{stack}\index{RPN stack} as it is encountered. Each
assertion label\index{assertion label} pops\index{pop} off the stack as many
entries as the referenced assertion has mandatory hypotheses.  Variable
substitutions\index{substitution!variable}\index{variable substitution} are
computed which, when made to the referenced assertion's mandatory hypotheses,
cause these hypotheses to match the stack entries. These same substitutions
are then made to the variables in the referenced assertion itself, which is
then pushed onto the stack.  At the end of the proof, there should be one
stack entry, namely the assertion being proved.  This process is explained in
detail in Section~\ref{proof}.

Metamath's proof notation is not very readable for humans, but it allows the
proof to be stored compactly in a file.  The Metamath\index{Metamath} program
has proof display features that let you see what's going on in a more
readable way, as you will see in the next section.

The rules used in verifying a proof are not based on any built-in syntax of the
symbol sequence in an assertion\index{assertion} nor on any built-in meanings
attached to specific symbol names.  They are based strictly on symbol
matching:  constants\index{constant} must match themselves, and
variables\index{variable} may be replaced with anything that allows a match to
occur.  For example, instead of \texttt{term}, \texttt{0}, and \verb$|-$ we could
have just as well used \texttt{yellow}, \texttt{zero}, and \texttt{provable}, as long
as we did so consistently throughout the database.  Also, we could have used
\texttt{is provable} (two tokens) instead of \verb$|-$ (one token) throughout the
database.  In each of these cases, the proof would be exactly the same.  The
independence of proofs and notation means that you have a lot of flexibility to
change the notation you use without having to change any proofs.

\section{A Trial Run}\label{trialrun}

Now you are ready to try out the Metamath\index{Metamath} program.

On all computer systems, Metamath has a standard ``command line
interface'' (CLI)\index{command line interface (CLI)} that allows you to
interact with it.  You supply commands to the CLI by typing them on the
keyboard and pressing your keyboard's {\em return} key after each line
you enter.  The CLI is designed to be easy to use and has built-in help
features.

The first thing you should do is to use a text editor to create a file
called \texttt{demo0.mm} and type into it the Metamath source shown on
p.~\pageref{demo0}.  Actually, this file is included with your Metamath
software package, so check that first.  If you type it in, make sure
that you save it in the form of ``plain {\sc ascii} text with line
breaks.''  Most word processors will have this feature.

Next you must run the Metamath program.  Depending on your computer
system and how Metamath is installed, this could range from clicking the
mouse on the Metamath icon to typing \texttt{run metamath} to typing
simply \texttt{metamath}.  (Metamath's {\tt help invoke} command describes
alternate ways of invoking the Metamath program.)

When you first enter Metamath\index{Metamath}, it will be at the CLI, waiting
for your input. You will see something like the following on your screen:
\begin{verbatim}
Metamath - Version 0.172 25-Jan-2019
Type HELP for help, EXIT to exit.
MM>
\end{verbatim}
The \texttt{MM>} prompt means that Metamath is waiting for a command.
Command keywords\index{command keyword} are not case sensitive;
we will use lower-case commands in our examples.
The version number and its release date will probably be different on your
system from the one we show above.

The first thing that you need to do is to read in your
database:\index{\texttt{read} command}\footnote{If a directory path is
needed on Unix,\index{Unix file names}\index{file names!Unix} you should
enclose the path/file name in quotes to prevent Metamath from thinking
that the \texttt{/} in the path name is a command qualifier, e.g.,
\texttt{read \char`\"db/set.mm\char`\"}.  Quotes are optional when there
is no ambiguity.}
\begin{verbatim}
MM> read demo0.mm
\end{verbatim}
Remember to press the {\em return} key after entering this command.  If
you omit the file name, Metamath will prompt you for one.   The syntax for
specifying a Macintosh file name path is given in a footnote on
p.~\pageref{includef}.\index{Macintosh file names}\index{file
names!Macintosh}

If there are any syntax errors in the database, Metamath will let you know
when it reads in the file.  The one thing that Metamath does not check when
reading in a database is that all proofs are correct, because this would
slow it down too much.  It is a good idea to periodically verify the proofs in
a database you are making changes to.  To do this, use the following command
(and do it for your \texttt{demo0.mm} file now).  Note that the \texttt{*} is a
``wild card'' meaning all proofs in the file.\index{\texttt{verify proof} command}
\begin{verbatim}
MM> verify proof *
\end{verbatim}
Metamath will report any proofs that are incorrect.

It is often useful to save the information that the Metamath program displays
on the screen. You can save everything that happens on the screen by opening a
log file. You may want to do this before you read in a database so that you
can examine any errors later on.  To open a log file, type
\begin{verbatim}
MM> open log abc.log
\end{verbatim}
This will open a file called \texttt{abc.log}, and everything that appears on the
screen from this point on will be stored in this file.  The name of the log file
is arbitrary. To close the log file, type
\begin{verbatim}
MM> close log
\end{verbatim}

Several commands let you examine what's inside your database.
Section~\ref{exploring} has an overview of some useful ones.  The
\texttt{show labels} command lets you see what statement
labels\index{label} exist.  A \texttt{*} matches any combination of
characters, and \texttt{t*} refers to all labels starting with the
letter \texttt{t}.\index{\texttt{show labels} command} The \texttt{/all}
is a ``command qualifier''\index{command qualifier} that tells Metamath
to include labels of hypotheses.  (To see the syntax explained, type
\texttt{help show labels}.)  Type
\begin{verbatim}
MM> show labels t* /all
\end{verbatim}
Metamath will respond with
\begin{verbatim}
The statement number, label, and type are shown.
3 tt $f       4 tr $f       5 ts $f       8 tze $a
9 tpl $a      19 th1 $p
\end{verbatim}

You can use the \texttt{show statement} command to get information about a
particular statement.\index{\texttt{show statement} command}
For example, you can get information about the statement with label \texttt{mp}
by typing
\begin{verbatim}
MM> show statement mp /full
\end{verbatim}
Metamath will respond with
\begin{verbatim}
Statement 17 is located on line 43 of the file
"demo0.mm".
"Define the modus ponens inference rule"
17 mp $a |- Q $.
Its mandatory hypotheses in RPN order are:
  wp $f wff P $.
  wq $f wff Q $.
  min $e |- P $.
  maj $e |- ( P -> Q ) $.
The statement and its hypotheses require the
      variables:  Q P
The variables it contains are:  Q P
\end{verbatim}
The mandatory hypotheses\index{mandatory hypothesis} and their
order\index{RPN order} are
useful to know when you are trying to understand or debug a proof.

Now you are ready to look at what's really inside our proof.  First, here is
how to look at every step in the proof---not just the ones corresponding to an
ordinary formal proof\index{formal proof}, but also the ones that build up the
formulas that appear in each ordinary formal proof step.\index{\texttt{show
proof} command}
\begin{verbatim}
MM> show proof th1 /lemmon /all
\end{verbatim}

This will display the proof on the screen in the following format:
\begin{verbatim}
 1 tt            $f term t
 2 tze           $a term 0
 3 1,2 tpl       $a term ( t + 0 )
 4 tt            $f term t
 5 3,4 weq       $a wff ( t + 0 ) = t
 6 tt            $f term t
 7 tt            $f term t
 8 6,7 weq       $a wff t = t
 9 tt            $f term t
10 9 a2          $a |- ( t + 0 ) = t
11 tt            $f term t
12 tze           $a term 0
13 11,12 tpl     $a term ( t + 0 )
14 tt            $f term t
15 13,14 weq     $a wff ( t + 0 ) = t
16 tt            $f term t
17 tze           $a term 0
18 16,17 tpl     $a term ( t + 0 )
19 tt            $f term t
20 18,19 weq     $a wff ( t + 0 ) = t
21 tt            $f term t
22 tt            $f term t
23 21,22 weq     $a wff t = t
24 20,23 wim     $a wff ( ( t + 0 ) = t -> t = t )
25 tt            $f term t
26 25 a2         $a |- ( t + 0 ) = t
27 tt            $f term t
28 tze           $a term 0
29 27,28 tpl     $a term ( t + 0 )
30 tt            $f term t
31 tt            $f term t
32 29,30,31 a1   $a |- ( ( t + 0 ) = t -> ( ( t + 0 )
                                     = t -> t = t ) )
33 15,24,26,32 mp  $a |- ( ( t + 0 ) = t -> t = t )
34 5,8,10,33 mp  $a |- t = t
\end{verbatim}

The \texttt{/lemmon} command qualifier specifies what is known as a Lemmon-style
display\index{Lemmon-style proof}\index{proof!Lemmon-style}.  Omitting the
\texttt{/lemmon} qualifier results in a tree-style proof (see
p.~\pageref{treeproof} for an example) that is somewhat less explicit but
easier to follow once you get used to it.\index{tree-style
proof}\index{proof!tree-style}

The first number on each line is the step
number of the proof.  Any numbers that follow are step numbers assigned to the
hypotheses of the statement referenced by that step.  Next is the label of
the statement referenced by the step.  The statement type of the statement
referenced comes next, followed by the math symbol\index{math symbol} string
constructed by the proof up to that step.

The last step, 34, contains the statement that is being proved.

Looking at a small piece of the proof, notice that steps 3 and 4 have
established that
\texttt{( t + 0 )} and \texttt{t} are \texttt{term}\,s, and step 5 makes use of steps 3 and
4 to establish that \texttt{( t + 0 ) = t} is a \texttt{wff}.  Let Metamath
itself tell us in detail what is happening in step 5.  Note that the
``target hypothesis'' refers to where step 5 is eventually used, i.e., in step
34.
\begin{verbatim}
MM> show proof th1 /detailed_step 5
Proof step 5:  wp=weq $a wff ( t + 0 ) = t
This step assigns source "weq" ($a) to target "wp"
($f).  The source assertion requires the hypotheses
"tt" ($f, step 3) and "tr" ($f, step 4).  The parent
assertion of the target hypothesis is "mp" ($a,
step 34).
The source assertion before substitution was:
    weq $a wff t = r
The following substitutions were made to the source
assertion:
    Variable  Substituted with
     t         ( t + 0 )
     r         t
The target hypothesis before substitution was:
    wp $f wff P
The following substitution was made to the target
hypothesis:
    Variable  Substituted with
     P         ( t + 0 ) = t
\end{verbatim}

The full proof just shown is useful to understand what is going on in detail.
However, most of the time you will just be interested in
the ``essential'' or logical steps of a proof, i.e.\ those steps
that correspond to an
ordinary formal proof\index{formal proof}.  If you type
\begin{verbatim}
MM> show proof th1 /lemmon /renumber
\end{verbatim}
you will see\label{demoproof}
\begin{verbatim}
1 a2             $a |- ( t + 0 ) = t
2 a2             $a |- ( t + 0 ) = t
3 a1             $a |- ( ( t + 0 ) = t -> ( ( t + 0 )
                                     = t -> t = t ) )
4 2,3 mp         $a |- ( ( t + 0 ) = t -> t = t )
5 1,4 mp         $a |- t = t
\end{verbatim}
Compare this to the formal proof on p.~\pageref{zeroproof} and
notice the resemblance.
By default Metamath
does not show \texttt{\$f}\index{\texttt{\$f}
statement} hypotheses and everything branching off of them in the proof tree
when the proof is displayed; this makes the proof look more like an ordinary
mathematical proof, which does not normally incorporate the explicit
construction of expressions.
This is called the ``essential'' view
(at one time you had to add the
\texttt{/essential} qualifier in the \texttt{show proof}
command to get this view, but this is now the default).
You can could use the \texttt{/all} qualifier in the \texttt{show
proof} command to also show the explicit construction of expressions.
The \texttt{/renumber} qualifier means to renumber
the steps to correspond only to what is displayed.\index{\texttt{show proof}
command}

To exit Metamath, type\index{\texttt{exit} command}
\begin{verbatim}
MM> exit
\end{verbatim}

\subsection{Some Hints for Using the Command Line Interface}

We will conclude this quick introduction to Metamath\index{Metamath} with some
helpful hints on how to navigate your way through the commands.
\index{command line interface (CLI)}

When you type commands into Metamath's CLI, you only have to type as many
characters of a command keyword\index{command keyword} as are needed to make
it unambiguous.  If you type too few characters, Metamath will tell you what
the choices are.  In the case of the \texttt{read} command, only the \texttt{r} is
needed to specify it unambiguously, so you could have typed\index{\texttt{read}
command}
\begin{verbatim}
MM> r demo0.mm
\end{verbatim}
instead of
\begin{verbatim}
MM> read demo0.mm
\end{verbatim}
In our description, we always show the full command words.  When using the
Metamath CLI commands in a command file (to be read with the \texttt{submit}
command)\index{\texttt{submit} command}, it is good practice to use
the unabbreviated command to ensure your instructions will not become ambiguous
if more commands are added to the Metamath program in the future.

The command keywords\index{command
keyword} are not case sensitive; you may type either \texttt{read} or
\texttt{ReAd}.  File names may or may not be case sensitive, depending on your
computer's operating system.  Metamath label\index{label} and math
symbol\index{math symbol} tokens\index{token} are case-sensitive.

The \texttt{help} command\index{\texttt{help} command} will provide you
with a list of topics you can get help on.  You can then type
\texttt{help} {\em topic} to get help on that topic.

If you are uncertain of a command's spelling, just type as many characters
as you remember of the command.  If you have not typed enough characters to
specify it unambiguously, Metamath will tell you what choices you have.

\begin{verbatim}
MM> show s
         ^
?Ambiguous keyword - please specify SETTINGS,
STATEMENT, or SOURCE.
\end{verbatim}

If you don't know what argument to use as part of a command, type a
\texttt{?}\index{\texttt{]}@\texttt{?}\ in command lines}\ at the
argument position.  Metamath will tell you what it expected there.

\begin{verbatim}
MM> show ?
         ^
?Expected SETTINGS, LABELS, STATEMENT, SOURCE, PROOF,
MEMORY, TRACE_BACK, or USAGE.
\end{verbatim}

Finally, you may type just the first word or words of a command followed
by {\em return}.  Metamath will prompt you for the remaining part of the
command, showing you the choices at each step.  For example, instead of
typing \texttt{show statement th1 /full} you could interact in the
following manner:
\begin{verbatim}
MM> show
SETTINGS, LABELS, STATEMENT, SOURCE, PROOF,
MEMORY, TRACE_BACK, or USAGE <SETTINGS>? st
What is the statement label <th1>?
/ or nothing <nothing>? /
TEX, COMMENT_ONLY, or FULL <TEX>? f
/ or nothing <nothing>?
19 th1 $p |- t = t $= ... $.
\end{verbatim}
After each \texttt{?}\ in this mode, you must give Metamath the
information it requests.  Sometimes Metamath gives you a list of choices
with the default choice indicated by brackets \texttt{< > }. Pressing
{\em return} after the \texttt{?}\ will select the default choice.
Answering anything else will override the default.  Note that the
\texttt{/} in command qualifiers is considered a separate
token\index{token} by the parser, and this is why it is asked for
separately.

\section{Your First Proof}\label{frstprf}

Proofs are developed with the aid of the Proof Assistant\index{Proof
Assistant}.  We will now show you how the proof of theorem \texttt{th1}
was built.  So that you can repeat these steps, we will first have the
Proof Assistant erase the proof in Metamath's source buffer\index{source
buffer}, then reconstruct it.  (The source buffer is the place in memory
where Metamath stores the information in the database when it is
\texttt{read}\index{\texttt{read} command} in.  New or modified proofs
are kept in the source buffer until a \texttt{write source}
command\index{\texttt{write source} command} is issued.)  In practice, you
would place a \texttt{?}\index{\texttt{]}@\texttt{?}\ inside proofs}\
between \texttt{\$=}\index{\texttt{\$=} keyword} and
\texttt{\$.}\index{\texttt{\$.}\ keyword}\ in the database to indicate
to Metamath\index{Metamath} that the proof is unknown, and that would be
your starting point.  Whenever the \texttt{verify proof} command encounters
a proof with a \texttt{?}\ in place of a proof step, the statement is
identified as not proved.

When I first started creating Metamath proofs, I would write down
on a piece of paper the complete
formal proof\index{formal proof} as it would appear
in a \texttt{show proof} command\index{\texttt{show proof} command}; see
the display of \texttt{show proof th1 /lemmon /re\-num\-ber} above as an
example.  After you get used to using the Proof Assistant\index{Proof
Assistant} you may get to a point where you can ``see'' the proof in your mind
and let the Proof Assistant guide you in filling in the details, at least for
simpler proofs, but until you gain that experience you may find it very useful
to write down all the details in advance.
Otherwise you may waste a lot of time as you let it take you down a wrong path.
However, others do not find this approach as helpful.
For example, Thomas Brendan Leahy\index{Leahy, Thomas Brendan}
finds that it is more helpful to him to interactively
work backward from a machine-readable statement.
David A. Wheeler\index{Wheeler, David A.}
writes down a general approach, but develops the proof
interactively by switching between
working forwards (from hypotheses and facts likely to be useful) and
backwards (from the goal) until the forwards and backwards approaches meet.
In the end, use whatever approach works for you.

A proof is developed with the Proof Assistant by working backwards, starting
with the theorem\index{theorem} to be proved, and assigning each unknown step
with a theorem or hypothesis until no more unknown steps remain.  The Proof
Assistant will not let you make an assignment unless it can be ``unified''
with the unknown step.  This means that a
substitution\index{substitution!variable}\index{variable substitution} of
variables exists that will make the assignment match the unknown step.  On the
other hand, in the middle of a proof, when working backwards, often more than
one unification\index{unification} (set of substitutions) is possible, since
there is not enough information available at that point to uniquely establish
it.  In this case you can tell Metamath which unification to choose, or you
can continue to assign unknown steps until enough information is available to
make the unification unique.

We will assume you have entered Metamath and read in the database as described
above.  The following dialog shows how the proof was developed.  For more
details on what some of the commands do, refer to Section~\ref{pfcommands}.
\index{\texttt{prove} command}

\begin{verbatim}
MM> prove th1
Entering the Proof Assistant.  Type HELP for help, EXIT
to exit.  You will be working on the proof of statement th1:
  $p |- t = t
Note:  The proof you are starting with is already complete.
MM-PA>
\end{verbatim}

The \verb/MM-PA>/ prompt means we are inside the Proof
Assistant.\index{Proof Assistant} Most of the regular Metamath commands
(\texttt{show statement}, etc.) are still available if you need them.

\begin{verbatim}
MM-PA> delete all
The entire proof was deleted.
\end{verbatim}

We have deleted the whole proof so we can start from scratch.

\begin{verbatim}
MM-PA> show new_proof/lemmon/all
1 ?              $? |- t = t
\end{verbatim}

The \texttt{show new{\char`\_}proof} command\index{\texttt{show
new{\char`\_}proof} command} is like \texttt{show proof} except that we
don't specify a statement; instead, the proof we're working on is
displayed.

\begin{verbatim}
MM-PA> assign 1 mp
To undo the assignment, DELETE STEP 5 and INITIALIZE, UNIFY
if needed.
3   min=?  $? |- $2
4   maj=?  $? |- ( $2 -> t = t )
\end{verbatim}

The \texttt{assign} command\index{\texttt{assign} command} above means
``assign step 1 with the statement whose label is \texttt{mp}.''  Note
that step renumbering will constantly occur as you assign steps in the
middle of a proof; in general all steps from the step you assign until
the end of the proof will get moved up.  In this case, what used to be
step 1 is now step 5, because the (partial) proof now has five steps:
the four hypotheses of the \texttt{mp} statement and the \texttt{mp}
statement itself.  Let's look at all the steps in our partial proof:

\begin{verbatim}
MM-PA> show new_proof/lemmon/all
1 ?              $? wff $2
2 ?              $? wff t = t
3 ?              $? |- $2
4 ?              $? |- ( $2 -> t = t )
5 1,2,3,4 mp     $a |- t = t
\end{verbatim}

The symbol \texttt{\$2} is a temporary variable\index{temporary
variable} that represents a symbol sequence not yet known.  In the final
proof, all temporary variables will be eliminated.  The general format
for a temporary variable is \texttt{\$} followed by an integer.  Note
that \texttt{\$} is not a legal character in a math symbol (see
Section~\ref{dollardollar}, p.~\pageref{dollardollar}), so there will
never be a naming conflict between real symbols and temporary variables.

Unknown steps 1 and 2 are constructions of the two wffs used by the
modus ponens rule.  As you will see at the end of this section, the
Proof Assistant\index{Proof Assistant} can usually figure these steps
out by itself, and we will not have to worry about them.  Therefore from
here on we will display only the ``essential'' hypotheses, i.e.\ those
steps that correspond to traditional formal proofs\index{formal proof}.

\begin{verbatim}
MM-PA> show new_proof/lemmon
3 ?              $? |- $2
4 ?              $? |- ( $2 -> t = t )
5 3,4 mp         $a |- t = t
\end{verbatim}

Unknown steps 3 and 4 are the ones we must focus on.  They correspond to the
minor and major premises of the modus ponens rule.  We will assign them as
follows.  Notice that because of the step renumbering that takes place
after an assignment, it is advantageous to assign unknown steps in reverse
order, because earlier steps will not get renumbered.

\begin{verbatim}
MM-PA> assign 4 mp
To undo the assignment, DELETE STEP 8 and INITIALIZE, UNIFY
if needed.
3   min=?  $? |- $2
6     min=?  $? |- $4
7     maj=?  $? |- ( $4 -> ( $2 -> t = t ) )
\end{verbatim}

We are now going to describe an obscure feature that you will probably
never use but should be aware of.  The Metamath language allows empty
symbol sequences to be substituted for variables, but in most formal
systems this feature is never used.  One of the few examples where is it
used is the MIU-system\index{MIU-system} described in
Appendix~\ref{MIU}.  But such systems are rare, and by default this
feature is turned off in the Proof Assistant.  (It is always allowed for
{\tt verify proof}.)  Let us turn it on and see what
happens.\index{\texttt{set empty{\char`\_}substitution} command}

\begin{verbatim}
MM-PA> set empty_substitution on
Substitutions with empty symbol sequences is now allowed.
\end{verbatim}

With this feature enabled, more unifications will be
ambiguous\index{ambiguous unification}\index{unification!ambiguous} in
the middle of a proof, because
substitution\index{substitution!variable}\index{variable substitution}
of variables with empty symbol sequences will become an additional
possibility.  Let's see what happens when we make our next assignment.

\begin{verbatim}
MM-PA> assign 3 a2
There are 2 possible unifications.  Please select the correct
    one or Q if you want to UNIFY later.
Unify:  |- $6
 with:  |- ( $9 + 0 ) = $9
Unification #1 of 2 (weight = 7):
  Replace "$6" with "( + 0 ) ="
  Replace "$9" with ""
  Accept (A), reject (R), or quit (Q) <A>? r
\end{verbatim}

The first choice presented is the wrong one.  If we had selected it,
temporary variable \texttt{\$6} would have been assigned a truncated
wff, and temporary variable \texttt{\$9} would have been assigned an
empty sequence (which is not allowed in our system).  With this choice,
eventually we would reach a point where we would get stuck because
we would end up with steps impossible to prove.  (You may want to
try it.)  We typed \texttt{r} to reject the choice.

\begin{verbatim}
Unification #2 of 2 (weight = 21):
  Replace "$6" with "( $9 + 0 ) = $9"
  Accept (A), reject (R), or quit (Q) <A>? q
To undo the assignment, DELETE STEP 4 and INITIALIZE, UNIFY
if needed.
 7     min=?  $? |- $8
 8     maj=?  $? |- ( $8 -> ( $6 -> t = t ) )
\end{verbatim}

The second choice is correct, and normally we would type \texttt{a}
to accept it.  But instead we typed \texttt{q} to show what will happen:
it will leave the step with an unknown unification, which can be
seen as follows:

\begin{verbatim}
MM-PA> show new_proof/not_unified
 4   min    $a |- $6
        =a2  = |- ( $9 + 0 ) = $9
\end{verbatim}

Later we can unify this with the \texttt{unify}
\texttt{all/interactive} command.

The important point to remember is that occasionally you will be
presented with several unification choices while entering a proof, when
the program determines that there is not enough information yet to make
an unambiguous choice automatically (and this can happen even with
\texttt{set empty{\char`\_}substitution} turned off).  Usually it is
obvious by inspection which choice is correct, since incorrect ones will
tend to be meaningless fragments of wffs.  In addition, the correct
choice will usually be the first one presented, unlike our example
above.

Enough of this digression.  Let us go back to the default setting.

\begin{verbatim}
MM-PA> set empty_substitution off
The ability to substitute empty expressions for variables
has been turned off.  Note that this may make the Proof
Assistant too restrictive in some cases.
\end{verbatim}

If we delete the proof, start over, and get to the point where
we digressed above, there will no longer be an ambiguous unification.

\begin{verbatim}
MM-PA> assign 3 a2
To undo the assignment, DELETE STEP 4 and INITIALIZE, UNIFY
if needed.
 7     min=?  $? |- $4
 8     maj=?  $? |- ( $4 -> ( ( $5 + 0 ) = $5 -> t = t ) )
\end{verbatim}

Let us look at our proof so far, and continue.

\begin{verbatim}
MM-PA> show new_proof/lemmon
 4 a2            $a |- ( $5 + 0 ) = $5
 7 ?             $? |- $4
 8 ?             $? |- ( $4 -> ( ( $5 + 0 ) = $5 -> t = t ) )
 9 7,8 mp        $a |- ( ( $5 + 0 ) = $5 -> t = t )
10 4,9 mp        $a |- t = t
MM-PA> assign 8 a1
To undo the assignment, DELETE STEP 11 and INITIALIZE, UNIFY
if needed.
 7     min=?  $? |- ( t + 0 ) = t
MM-PA> assign 7 a2
To undo the assignment, DELETE STEP 8 and INITIALIZE, UNIFY
if needed.
MM-PA> show new_proof/lemmon
 4 a2            $a |- ( t + 0 ) = t
 8 a2            $a |- ( t + 0 ) = t
12 a1            $a |- ( ( t + 0 ) = t -> ( ( t + 0 ) = t ->
                                                    t = t ) )
13 8,12 mp       $a |- ( ( t + 0 ) = t -> t = t )
14 4,13 mp       $a |- t = t
\end{verbatim}

Now all temporary variables and unknown steps have been eliminated from the
``essential'' part of the proof.  When this is achieved, the Proof
Assistant\index{Proof Assistant} can usually figure out the rest of the proof
automatically.  (Note that the \texttt{improve} command can occasionally be
useful for filling in essential steps as well, but it only tries to make use
of statements that introduce no new variables in their hypotheses, which is
not the case for \texttt{mp}. Also it will not try to improve steps containing
temporary variables.)  Let's look at the complete proof, then run
the \texttt{improve} command, then look at it again.

\begin{verbatim}
MM-PA> show new_proof/lemmon/all
 1 ?             $? wff ( t + 0 ) = t
 2 ?             $? wff t = t
 3 ?             $? term t
 4 3 a2          $a |- ( t + 0 ) = t
 5 ?             $? wff ( t + 0 ) = t
 6 ?             $? wff ( ( t + 0 ) = t -> t = t )
 7 ?             $? term t
 8 7 a2          $a |- ( t + 0 ) = t
 9 ?             $? term ( t + 0 )
10 ?             $? term t
11 ?             $? term t
12 9,10,11 a1    $a |- ( ( t + 0 ) = t -> ( ( t + 0 ) = t ->
                                                    t = t ) )
13 5,6,8,12 mp   $a |- ( ( t + 0 ) = t -> t = t )
14 1,2,4,13 mp   $a |- t = t
\end{verbatim}

\begin{verbatim}
MM-PA> improve all
A proof of length 1 was found for step 11.
A proof of length 1 was found for step 10.
A proof of length 3 was found for step 9.
A proof of length 1 was found for step 7.
A proof of length 9 was found for step 6.
A proof of length 5 was found for step 5.
A proof of length 1 was found for step 3.
A proof of length 3 was found for step 2.
A proof of length 5 was found for step 1.
Steps 1 and above have been renumbered.
CONGRATULATIONS!  The proof is complete.  Use SAVE
NEW_PROOF to save it.  Note:  The Proof Assistant does
not detect $d violations.  After saving the proof, you
should verify it with VERIFY PROOF.
\end{verbatim}

The \texttt{save new{\char`\_}proof} command\index{\texttt{save
new{\char`\_}proof} command} will save the proof in the database.  Here
we will just display it in a form that can be clipped out of a log file
and inserted manually into the database source file with a text
editor.\index{normal proof}\index{proof!normal}

\begin{verbatim}
MM-PA> show new_proof/normal
---------Clip out the proof below this line:
      tt tze tpl tt weq tt tt weq tt a2 tt tze tpl tt weq
      tt tze tpl tt weq tt tt weq wim tt a2 tt tze tpl tt
      tt a1 mp mp $.
---------The proof of 'th1' to clip out ends above this line.
\end{verbatim}

There is another proof format called ``compressed''\index{compressed
proof}\index{proof!compressed} that you will see in databases.  It is
not important to understand how it is encoded but only to recognize it
when you see it.  Its only purpose is to reduce storage requirements for
large proofs.  A compressed proof can always be converted to a normal
one and vice-versa, and the Metamath \texttt{show proof}
commands\index{\texttt{show proof} command} work equally well with
compressed proofs.  The compressed proof format is described in
Appendix~\ref{compressed}.

\begin{verbatim}
MM-PA> show new_proof/compressed
---------Clip out the proof below this line:
      ( tze tpl weq a2 wim a1 mp ) ABCZADZAADZAEZJJKFLIA
      AGHH $.
---------The proof of 'th1' to clip out ends above this line.
\end{verbatim}

Now we will exit the Proof Assistant.  Since we made changes to the proof,
it will warn us that we have not saved it.  In this case, we don't care.

\begin{verbatim}
MM-PA> exit
Warning:  You have not saved changes to the proof.
Do you want to EXIT anyway (Y, N) <N>? y
Exiting the Proof Assistant.
Type EXIT again to exit Metamath.
\end{verbatim}

The Proof Assistant\index{Proof Assistant} has several other commands
that can help you while creating proofs.  See Section~\ref{pfcommands}
for a list of them.

A command that is often useful is \texttt{minimize{\char`\_}with
*/brief}, which tries to shorten the proof.  It can make the process
more efficient by letting you write a somewhat ``sloppy'' proof then
clean up some of the fine details of optimization for you (although it
can't perform miracles such as restructuring the overall proof).

\section{A Note About Editing a Data\-base File}

Once your source file contains proofs, there are some restrictions on
how you can edit it so that the proofs remain valid.  Pay particular
attention to these rules, since otherwise you can lose a lot of work.
It is a good idea to periodically verify all proofs with \texttt{verify
proof *} to ensure their integrity.

If your file contains only normal (as opposed to compressed) proofs, the
main rule is that you may not change the order of the mandatory
hypotheses\index{mandatory hypothesis} of any statement referenced in a
later proof.  For example, if you swap the order of the major and minor
premise in the modus ponens rule, all proofs making use of that rule
will become incorrect.  The \texttt{show statement}
command\index{\texttt{show statement} command} will show you the
mandatory hypotheses of a statement and their order.

If a statement has a compressed proof, you also must not change the
order of {\em its} mandatory hypotheses.  The compressed proof format
makes use of this information as part of the compression technique.
Note that swapping the names of two variables in a theorem will change
the order of its mandatory hypotheses.

The safest way to edit a statement, say \texttt{mytheorem}, is to
duplicate it then rename the original to \texttt{mytheoremOLD}
throughout the database.  Once the edited version is re-proved, all
statements referencing \texttt{mytheoremOLD} can be updated in the Proof
Assistant using \texttt{minimize{\char`\_}with
mytheorem
/allow{\char`\_}growth}.\index{\texttt{minimize{\char`\_}with} command}
% 3/10/07 Note: line-breaking the above results in duplicate index entries

\chapter{Abstract Mathematics Revealed}\label{fol}

\section{Logic and Set Theory}\label{logicandsettheory}

\begin{quote}
  {\em Set theory can be viewed as a form of exact theology.}
  \flushright\sc  Rudy Rucker\footnote{\cite{Barrow}, p.~31.}\\
\end{quote}\index{Rucker, Rudy}

Despite its seeming complexity, all of standard mathematics, no matter how
deep or abstract, can amazingly enough be derived from a relatively small set
of axioms\index{axiom} or first principles. The development of these axioms is
among the most impressive and important accomplishments of mathematics in the
20th century. Ultimately, these axioms can be broken down into a set of rules
for manipulating symbols that any technically oriented person can follow.

We will not spend much time trying to convey a deep, higher-level
understanding of the meaning of the axioms. This kind of understanding
requires some mathematical sophistication as well as an understanding of the
philosophy underlying the foundations of mathematics and typically develops
over time as you work with mathematics.  Our goal, instead, is to give you the
immediate ability to follow how theorems\index{theorem} are derived from the
axioms and from other theorems.  This will be similar to learning the syntax
of a computer language, which lets you follow the details in a program but
does not necessarily give you the ability to write non-trivial programs on
your own, an ability that comes with practice. For now don't be alarmed by
abstract-sounding names of the axioms; just focus on the rules for
manipulating the symbols, which follow the simple conventions of the
Metamath\index{Metamath} language.

The axioms that underlie all of standard mathematics consist of axioms of logic
and axioms of set theory. The axioms of logic are divided into two
subcategories, propositional calculus\index{propositional calculus} (sometimes
called sentential logic\index{sentential logic}) and predicate calculus
(sometimes called first-order logic\index{first-order logic}\index{quantifier
theory}\index{predicate calculus} or quantifier theory).  Propositional
calculus is a prerequisite for predicate calculus, and predicate calculus is a
prerequisite for set theory.  The version of set theory most commonly used is
Zermelo--Fraenkel set theory\index{Zermelo--Fraenkel set theory}\index{set theory}
with the axiom of choice,
often abbreviated as ZFC\index{ZFC}.

Here in a nutshell is what the axioms are all about in an informal way. The
connection between this description and symbols we will show you won't be
immediately apparent and in principle needn't ever be.  Our description just
tries to summarize what mathematicians think about when they work with the
axioms.

Logic is a set of rules that allow us determine truths given other truths.
Put another way,
logic is more or less the translation of what we would consider common sense
into a rigorous set of axioms.\index{axioms of logic}  Suppose $\varphi$,
$\psi$, and $\chi$ (the Greek letters phi, psi, and chi) represent statements
that are either true or false, and $x$ is a variable\index{variable!in predicate
calculus} ranging over some group of mathematical objects (sets, integers,
real numbers, etc.). In mathematics, a ``statement'' really means a formula,
and $\psi$ could be for example ``$x = 2$.''
Propositional calculus\index{propositional calculus}
allows us to use variables that are either true or false
and make deductions such as
``if $\varphi$ implies $\psi$ and $\psi$ implies $\chi$, then $\varphi$
implies $\chi$.''
Predicate calculus\index{predicate calculus}
extends propositional calculus by also allowing us
to discuss statements about objects (not just true and false values), including
statements about ``all'' or ``at least one'' object.
For example, predicate calculus allows to say,
``if $\varphi$ is true for all $x$, then $\varphi$ is true for some $x$.''
The logic used in \texttt{set.mm} is standard classical logic
(as opposed to other logic systems like intuitionistic logic).

Set theory\index{set theory} has to do with the manipulation of objects and
collections of objects, specifically the abstract, imaginary objects that
mathematics deals with, such as numbers. Everything that is claimed to exist
in mathematics is considered to be a set.  A set called the empty
set\index{empty set} contains nothing.  We represent the empty set by
$\varnothing$.  Many sets can be built up from the empty set.  There is a set
represented by $\{\varnothing\}$ that contains the empty set, another set
represented by $\{\varnothing,\{\varnothing\}\}$ that contains this set as
well as the empty set, another set represented by $\{\{\varnothing\}\}$ that
contains just the set that contains the empty set, and so on ad infinitum. All
mathematical objects, no matter how complex, are defined as being identical to
certain sets: the integer\index{integer} 0 is defined as the empty set, the
integer 1 is defined as $\{\varnothing\}$, the integer 2 is defined as
$\{\varnothing,\{\varnothing\}\}$.  (How these definitions were chosen doesn't
matter now, but the idea behind it is that these sets have the properties we
expect of integers once suitable operations are defined.)  Mathematical
operations, such as addition, are defined in terms of operations on
sets---their union\index{set union}, intersection\index{set intersection}, and
so on---operations you may have used in elementary school when you worked
with groups of apples and oranges.

With a leap of faith, the axioms also postulate the existence of infinite
sets\index{infinite set}, such as the set of all non-negative integers ($0, 1,
2,\ldots$, also called ``natural numbers''\index{natural number}).  This set
can't be represented with the brace notation\index{brace notation} we just
showed you, but requires a more complicated notation called ``class
abstraction.''\index{class abstraction}\index{abstraction class}  For
example, the infinite set $\{ x |
\mbox{``$x$ is a natural number''} \} $ means the ``set of all objects $x$
such that $x$ is a natural number'' i.e.\ the set of natural numbers; here,
``$x$ is a natural number'' is a rather complicated formula when broken down
into the primitive symbols.\label{expandom}\footnote{The statement ``$x$ is a
natural number'' is formally expressed as ``$x \in \omega$,'' where $\in$
(stylized epsilon) means ``is in'' or ``is an element of'' and $\omega$
(omega) means ``the set of natural numbers.''  When ``$x\in\omega$'' is
completely expanded in terms of the primitive symbols of set theory, the
result is  $\lnot$ $($ $\lnot$ $($ $\forall$ $z$ $($ $\lnot$ $\forall$ $w$ $($
$z$ $\in$ $w$ $\rightarrow$ $\lnot$ $w$ $\in$ $x$ $)$ $\rightarrow$ $z$ $\in$
$x$ $)$ $\rightarrow$ $($ $\forall$ $z$ $($ $\lnot$ $($ $\forall$ $w$ $($ $w$
$\in$ $z$ $\rightarrow$ $w$ $\in$ $x$ $)$ $\rightarrow$ $\forall$ $w$ $\lnot$
$w$ $\in$ $z$ $)$ $\rightarrow$ $\lnot$ $\forall$ $w$ $($ $w$ $\in$ $z$
$\rightarrow$ $\lnot$ $\forall$ $v$ $($ $v$ $\in$ $z$ $\rightarrow$ $\lnot$
$v$ $\in$ $w$ $)$ $)$ $)$ $\rightarrow$ $\lnot$ $\forall$ $z$ $\forall$ $w$
$($ $\lnot$ $($ $z$ $\in$ $x$ $\rightarrow$ $\lnot$ $w$ $\in$ $x$ $)$
$\rightarrow$ $($ $\lnot$ $z$ $\in$ $w$ $\rightarrow$ $($ $\lnot$ $z$ $=$ $w$
$\rightarrow$ $w$ $\in$ $z$ $)$ $)$ $)$ $)$ $)$ $\rightarrow$ $\lnot$
$\forall$ $y$ $($ $\lnot$ $($ $\lnot$ $($ $\forall$ $z$ $($ $\lnot$ $\forall$
$w$ $($ $z$ $\in$ $w$ $\rightarrow$ $\lnot$ $w$ $\in$ $y$ $)$ $\rightarrow$
$z$ $\in$ $y$ $)$ $\rightarrow$ $($ $\forall$ $z$ $($ $\lnot$ $($ $\forall$
$w$ $($ $w$ $\in$ $z$ $\rightarrow$ $w$ $\in$ $y$ $)$ $\rightarrow$ $\forall$
$w$ $\lnot$ $w$ $\in$ $z$ $)$ $\rightarrow$ $\lnot$ $\forall$ $w$ $($ $w$
$\in$ $z$ $\rightarrow$ $\lnot$ $\forall$ $v$ $($ $v$ $\in$ $z$ $\rightarrow$
$\lnot$ $v$ $\in$ $w$ $)$ $)$ $)$ $\rightarrow$ $\lnot$ $\forall$ $z$
$\forall$ $w$ $($ $\lnot$ $($ $z$ $\in$ $y$ $\rightarrow$ $\lnot$ $w$ $\in$
$y$ $)$ $\rightarrow$ $($ $\lnot$ $z$ $\in$ $w$ $\rightarrow$ $($ $\lnot$ $z$
$=$ $w$ $\rightarrow$ $w$ $\in$ $z$ $)$ $)$ $)$ $)$ $\rightarrow$ $($
$\forall$ $z$ $\lnot$ $z$ $\in$ $y$ $\rightarrow$ $\lnot$ $\forall$ $w$ $($
$\lnot$ $($ $w$ $\in$ $y$ $\rightarrow$ $\lnot$ $\forall$ $z$ $($ $w$ $\in$
$z$ $\rightarrow$ $\lnot$ $z$ $\in$ $y$ $)$ $)$ $\rightarrow$ $\lnot$ $($
$\lnot$ $\forall$ $z$ $($ $w$ $\in$ $z$ $\rightarrow$ $\lnot$ $z$ $\in$ $y$
$)$ $\rightarrow$ $w$ $\in$ $y$ $)$ $)$ $)$ $)$ $\rightarrow$ $x$ $\in$ $y$
$)$ $)$ $)$. Section~\ref{hierarchy} shows the hierarchy of definitions that
leads up to this expression.}\index{stylized epsilon ($\in$)}\index{omega
($\omega$)}  Actually, the primitive symbols don't even include the brace
notation.  The brace notation is a high-level definition, which you can find in
Section~\ref{hierarchy}.

Interestingly, the arithmetic of integers\index{integer} and
rationals\index{rational number} can be developed without appealing to the
existence of an infinite set, whereas the arithmetic of real
numbers\index{real number} requires it.

Each variable\index{variable!in set theory} in the axioms of set theory
represents an arbitrary set, and the axioms specify the legal kinds of things
you can do with these variables at a very primitive level.

Now, you may think that numbers and arithmetic are a lot more intuitive and
fundamental than sets and therefore should be the foundation of mathematics.
What is really the case is that you've dealt with numbers all your life and
are comfortable with a few rules for manipulating them such as addition and
multiplication.  Those rules only cover a small portion of what can be done
with numbers and only a very tiny fraction of the rest of mathematics.  If you
look at any elementary book on number theory, you will quickly become lost if
these are the only rules that you know.  Even though such books may present a
list of ``axioms''\index{axiom} for arithmetic, the ability to use the axioms
and to understand proofs of theorems\index{theorem} (facts) about numbers
requires an implicit mathematical talent that frustrates many people
from studying abstract mathematics.  The kind of mathematics that most people
know limits them to the practical, everyday usage of blindly manipulating
numbers and formulas, without any understanding of why those rules are correct
nor any ability to go any further.  For example, do you know why multiplying
two negative numbers yields a positive number?  Starting with set theory, you
will also start off blindly manipulating symbols according to the rules we give
you, but with the advantage that these rules will allow you, in principle, to
access {\em all} of mathematics, not just a tiny part of it.

Of course, concrete examples are often helpful in the learning process. For
example, you can verify that $2\cdot 3=3 \cdot 2$ by actually grouping
objects and can easily ``see'' how it generalizes to $x\cdot y = y\cdot x$,
even though you might not be able to rigorously prove it.  Similarly, in set
theory it can be helpful to understand how the axioms of set theory apply to
(and are correct for) small finite collections of objects.  You should be aware
that in set theory intuition can be misleading for infinite collections, and
rigorous proofs become more important.  For example, while $x\cdot y = y\cdot
x$ is correct for finite ordinals (which are the natural numbers), it is not
usually true for infinite ordinals.

\section{The Axioms for All of Mathematics}

In this section\index{axioms for mathematics}, we will show you the axioms
for all of standard mathematics (i.e.\ logic and set theory) as they are
traditionally presented.  The traditional presentation is useful for someone
with the mathematical experience needed to correctly manipulate high-level
abstract concepts.  For someone without this talent, knowing how to actually
make use of these axioms can be difficult.  The purpose of this section is to
allow you to see how the version of the axioms used in the standard
Metamath\index{Metamath} database \texttt{set.mm}\index{set
theory database (\texttt{set.mm})} relates to  the typical version
in textbooks, and also to give you an informal feel for them.

\subsection{Propositional Calculus}

Propositional calculus\index{propositional calculus} concerns itself with
statements that can be interpreted as either true or false.  Some examples of
statements (outside of mathematics) that are either true or false are ``It is
raining today'' and ``The United States has a female president.'' In
mathematics, as we mentioned, statements are really formulas.

In propositional calculus, we don't care what the statements are.  We also
treat a logical combination of statements, such as ``It is raining today and
the United States has a female president,'' no differently from a single
statement.  Statements and their combinations are called well-formed formulas
(wffs)\index{well-formed formula (wff)}.  We define wffs only in terms of
other wffs and don't define what a ``starting'' wff is.  As is common practice
in the literature, we use Greek letters to represent wffs.

Specifically, suppose $\varphi$ and $\psi$ are wffs.  Then the combinations
$\varphi\rightarrow\psi$ (``$\varphi$ implies $\psi$,'' also read ``if
$\varphi$ then $\psi$'')\index{implication ($\rightarrow$)} and $\lnot\varphi$
(``not $\varphi$'')\index{negation ($\lnot$)} are also wffs.

The three axioms of propositional calculus\index{axioms of propositional
calculus} are all wffs of the following form:\footnote{A remarkable result of
C.~A.~Meredith\index{Meredith, C. A.} squeezes these three axioms into the
single axiom $((((\varphi\rightarrow \psi)\rightarrow(\neg \chi\rightarrow\neg
\theta))\rightarrow \chi )\rightarrow \tau)\rightarrow((\tau\rightarrow
\varphi)\rightarrow(\theta\rightarrow \varphi))$ \cite{CAMeredith},
which is believed to be the shortest possible.}
\begin{center}
     $\varphi\rightarrow(\psi\rightarrow \varphi)$\\

     $(\varphi\rightarrow (\psi\rightarrow \chi))\rightarrow
((\varphi\rightarrow  \psi)\rightarrow (\varphi\rightarrow \chi))$\\

     $(\neg \varphi\rightarrow \neg\psi)\rightarrow (\psi\rightarrow
\varphi)$
\end{center}

These three axioms are widely used.
They are attributed to Jan {\L}ukasiewicz
(pronounced woo-kah-SHAY-vitch) and was popularized by Alonzo Church,
who called it system P2. (Thanks to Ted Ulrich for this information.)

There are an infinite number of axioms, one for each possible
wff\index{well-formed formula (wff)} of the above form.  (For this reason,
axioms such as the above are often called ``axiom schemes.''\index{axiom
scheme})  Each Greek letter in the axioms may be substituted with a more
complex wff to result in another axiom.  For example, substituting
$\neg(\varphi\rightarrow\chi)$ for $\varphi$ in the first axiom yields
$\neg(\varphi\rightarrow\chi)\rightarrow(\psi\rightarrow
\neg(\varphi\rightarrow\chi))$, which is still an axiom.

To deduce new true statements (theorems\index{theorem}) from the axioms, a
rule\index{rule} called ``modus ponens''\index{modus ponens} is used.  This
rule states that if the wff $\varphi$ is an axiom or a theorem, and the wff
$\varphi\rightarrow\psi$ is an axiom or a theorem, then the wff $\psi$ is also
a theorem\index{theorem}.

As a non-mathematical example of modus ponens, suppose we have proved (or
taken as an axiom) ``Bob is a man'' and separately have proved (or taken as
an axiom) ``If Bob is a man, then Bob is a human.''  Using the rule of modus
ponens, we can logically deduce, ``Bob is a human.''

From Metamath's\index{Metamath} point of view, the axioms and the rule of
modus ponens just define a mechanical means for deducing new true statements
from existing true statements, and that is the complete content of
propositional calculus as far as Metamath is concerned.  You can read a logic
textbook to gain a better understanding of their meaning, or you can just let
their meaning slowly become apparent to you after you use them for a while.

It is actually rather easy to check to see if a formula is a theorem of
propositional calculus.  Theorems of propositional calculus are also called
``tautologies.''\index{tautology}  The technique to check whether a formula is
a tautology is called the ``truth table method,''\index{truth table} and it
works like this.  A wff $\varphi\rightarrow\psi$ is false whenever $\varphi$ is true
and $\psi$ is false.  Otherwise it is true.  A wff $\lnot\varphi$ is false
whenever $\varphi$ is true and false otherwise. To verify a tautology such as
$\varphi\rightarrow(\psi\rightarrow \varphi)$, you break it down into sub-wffs and
construct a truth table that accounts for all possible combinations of true
and false assigned to the wff metavariables:
\begin{center}\begin{tabular}{|c|c|c|c|}\hline
\mbox{$\varphi$} & \mbox{$\psi$} & \mbox{$\psi\rightarrow\varphi$}
    & \mbox{$\varphi\rightarrow(\psi\rightarrow \varphi)$} \\ \hline \hline
              T   &  T    &      T       &        T    \\ \hline
              T   &  F    &      T       &        T    \\ \hline
              F   &  T    &      F       &        T    \\ \hline
              F   &  F    &      T       &        T    \\ \hline
\end{tabular}\end{center}
If all entries in the last column are true, the formula is a tautology.

Now, the truth table method doesn't tell you how to prove the tautology from
the axioms, but only that a proof exists.  Finding an actual proof (especially
one that is short and elegant) can be challenging.  Methods do exist for
automatically generating proofs in propositional calculus, but the proofs that
result can sometimes be very long.  In the Metamath \texttt{set.mm}\index{set
theory database (\texttt{set.mm})} database, most
or all proofs were created manually.

Section \ref{metadefprop} discusses various definitions
that make propositional calculus easier to use.
For example, we define:

\begin{itemize}
\item $\varphi \vee \psi$
  is true if either $\varphi$ or $\psi$ (or both) are true
  (this is disjunction\index{disjunction ($\vee$)}
  aka logical {\sc or}\index{logical {\sc or} ($\vee$)}).

\item $\varphi \wedge \psi$
  is true if both $\varphi$ and $\psi$ are true
  (this is conjunction\index{conjunction ($\wedge$)}
  aka logical {\sc and}\index{logical {\sc and} ($\wedge$)}).

\item $\varphi \leftrightarrow \psi$
  is true if $\varphi$ and $\psi$ have the same value, that is,
  they are both true or both false
  (this is the biconditional\index{biconditional ($\leftrightarrow$)}).
\end{itemize}

\subsection{Predicate Calculus}

Predicate calculus\index{predicate calculus} introduces the concept of
``individual variables,''\index{variable!in predicate calculus}\index{individual
variable} which
we will usually just call ``variables.''
These variables can represent something other than true or false (wffs),
and will always represent sets when we get to set theory.  There are also
three new symbols $\forall$\index{universal quantifier ($\forall$)},
$=$\index{equality ($=$)}, and $\in$\index{stylized epsilon ($\in$)},
read ``for all,'' ``equals,'' and ``is an element of''
respectively.  We will represent variables with the letters $x$, $y$, $z$, and
$w$, as is common practice in the literature.
For example, $\forall x \varphi$ means ``for all possible values of
$x$, $\varphi$ is true.''

In predicate calculus, we extend the definition of a wff\index{well-formed
formula (wff)}.  If $\varphi$ is a wff and $x$ and $y$ are variables, then
$\forall x \, \varphi$, $x=y$, and $x\in y$ are wffs. Note that these three new
types of wffs can be considered ``starting'' wffs from which we can build
other wffs with $\rightarrow$ and $\neg$ .  The concept of a starting wff was
absent in propositional calculus.  But starting wff or not, all we are really
concerned with is whether our wffs are correctly constructed according to
these mechanical rules.

A quick aside:
To prevent confusion, it might be best at this point to think of the variables
of Metamath\index{Metamath} as ``metavariables,''\index{metavariable} because
they are not quite the same as the variables we are introducing here.  A
(meta)variable in Metamath can be a wff or an individual variable, as well
as many other things; in general, it represents a kind of place holder for an
unspecified sequence of math symbols\index{math symbol}.

Unlike propositional calculus, no decision procedure\index{decision procedure}
analogous to the truth table method exists (nor theoretically can exist) that
will definitely determine whether a formula is a theorem of predicate
calculus.  Much of the work in the field of automated theorem
proving\index{automated theorem proving} has been dedicated to coming up with
clever heuristics for proving theorems of predicate calculus, but they can
never be guaranteed to work always.

Section \ref{metadefpred} discusses various definitions
that make predicate calculus easier to use.
For example, we define
$\exists x \varphi$ to mean
``there exists at least one possible value of $x$ where $\varphi$ is true.''

We now turn to looking at how predicate calculus can be formally
represented.

\subsubsection{Common Axioms}

There is a new rule of inference in predicate calculus:  if $\varphi$ is
an axiom or a theorem, then $\forall x \,\varphi$ is also a
theorem\index{theorem}.  This is called the rule of
``generalization.''\index{rule of generalization}
This is easily represented in Metamath.

In standard texts of logic, there are often two axioms of predicate
calculus\index{axioms of predicate calculus}:
\begin{center}
  $\forall x \,\varphi ( x ) \rightarrow \varphi ( y )$,
      where ``$y$ is properly substituted for $x$.''\\
  $\forall x ( \varphi \rightarrow \psi )\rightarrow ( \varphi \rightarrow
    \forall x\, \psi )$,
    where ``$x$ is not free in $\varphi$.''
\end{center}

Now at first glance, this seems simple:  just two axioms.  However,
conditional clauses are attached to each axiom describing requirements that
may seem puzzling to you.  In addition, the first axiom puts a variable symbol
in parentheses after each wff, seemingly violating our definition of a
wff\index{well-formed formula (wff)}; this is just an informal way of
referring to some arbitrary variable that may occur in the wff.  The
conditional clauses do, of course, have a precise meaning, but as it turns out
the precise meaning is somewhat complicated and awkward to formalize in a
way that a computer can handle easily.  Unlike propositional calculus, a
certain amount of mathematical sophistication and practice is needed to be
able to easily grasp and manipulate these concepts correctly.

Predicate calculus may be presented with or without axioms for
equality\index{axioms of equality}\index{equality ($=$)}. We will require the
axioms of equality as a prerequisite for the version of set theory we will
use.  The axioms for equality, when included, are often represented using these
two axioms:
\begin{center}
$x=x$\\ \ \\
$x=y\rightarrow (\varphi(x,x)\rightarrow\varphi(x,y))$ where ``$\varphi(x,y)$
   arises from $\varphi(x,x)$ by replacing some, but not necessarily all,
   free\index{free variable}
   occurrences of $x$ by $y$,\\ provided that $y$ is free for $x$
   in $\varphi(x,x)$.'' \end{center}
% (Mendelson p. 95)
The first equality axiom is simple, but again,
the condition on the second one is
somewhat awkward to implement on a computer.

\subsubsection{Tarski System S2}

Of course, we are not the first to notice the complications of these
predicate calculus axioms when being rigorous.

Well-known logician Alfred Tarski published in 1965
a system he called system S2\cite[p.~77]{Tarski1965}.
Tarski's system is \textit{exactly equivalent} to the traditional textbook
formalization, but (by clever use of equality axioms) it eliminates the
latter's primitive notions of ``proper substitution'' and ``free variable,''
replacing them with direct substitution and the notion of a variable
not occurring in a formula (which we express with distinct variable
constraints).

In advocating his system, Tarski wrote, ``The relatively complicated
character of [free variables and proper substitution] is a source
of certain inconveniences of both practical and theoretical nature;
this is clearly experienced both in teaching an elementary course of
mathematical logic and in formalizing the syntax of predicate logic for
some theoretical purposes''\cite[p.~61]{Tarski1965}\index{Tarski, Alfred}.

\subsubsection{Developing a Metamath Representation}

The standard textbook axioms of predicate calculus are somewhat
cumbersome to implement on a computer because of the complex notions of
``free variable''\index{free variable} and ``proper
substitution.''\index{proper substitution}\index{substitution!proper}
While it is possible to use the Metamath\index{Metamath} language to
implement these concepts, we have chosen not to implement them
as primitive constructs in the
\texttt{set.mm} set theory database.  Instead, we have eliminated them
within the axioms
by carefully crafting the axioms so as to avoid them,
building on Tarski's system S2.  This makes it
easy for a beginner to follow the steps in a proof without knowing any
advanced concepts other than the simple concept of
replacing\index{substitution!variable}\index{variable substitution}
variables with expressions.

In order to develop the concepts of free variable and proper
substitution from the axioms, we use an additional
Metamath statement type called ``disjoint variable
restriction''\index{disjoint variables} that we have not encountered
before.  In the context of the axioms, the statement \texttt{\$d} $ x\,
y$\index{\texttt{\$d} statement} simply means that $x$ and $y$ must be
distinct\index{distinct variables}, i.e.\ they may not be simultaneously
substituted\index{substitution!variable}\index{variable substitution}
with the same variable.  The statement \texttt{\$d} $ x\, \varphi$ means
variable $x$ must not occur in wff $\varphi$.  For the precise
definition of \texttt{\$d}, see Section~\ref{dollard}.

\subsubsection{Metamath representation}

The Metamath axiom system for predicate calculus
defined in set.mm uses Tarski's system S2.
As noted above, this has a different representation
than the traditional textbook formalization,
but it is \textit{exactly equivalent} to the textbook formalization,
and it is \textit{much} easier to work with.
This is reproduced as system S3 in Section 6 of
Megill's formalization \cite{Megill}\index{Megill, Norman}.

There is one exception, Tarski's axiom of existence,
which we label as axiom ax-6.
In the case of ax-6, Tarski's version is weaker because it includes a
distinct variable proviso. If we wish, we can also weaken our version
in this way and still have a metalogically complete system. Theorem
ax6 shows this by deriving, in the presence of the other axioms, our
ax-6 from Tarski's weaker version ax6v. However, we chose the stronger
version for our system because it is simpler to state and easier to use.

Tarski's system was designed for proving specific theorems rather than
more general theorem schemes. However, theorem schemes are much more
efficient than specific theorems for building a body of mathematical
knowledge, since they can be reused with different instances as
needed. While Tarski does derive some theorem schemes from his axioms,
their proofs require concepts that are ``outside'' of the system, such as
induction on formula length. The verification of such proofs is difficult
to automate in a proof verifier. (Specifically, Tarski treats the formulas
of his system as set-theoretical objects. In order to verify the proofs
of his theorem schemes, a proof verifier would need a significant amount
of set theory built into it.)

The Metamath axiom system for predicate calculus extends
Tarski's system to eliminate this difficulty. The additional
``auxilliary'' axiom
schemes (as we will call them in this section; see below) endow Tarski's
system with a nice property we call
metalogical completeness \cite[Remark 9.6]{Megill}\index{Megill, Norman}.
As a result, we can prove any theorem scheme
expressable in the ``simple metalogic'' of Tarski's system by using
only Metamath's direct substitution rule applied to the axiom system
(and no other metalogical or set-theoretical notions ``outside'' of the
system). Simple metalogic consists of schemes containing wff metavariables
(with no arguments) and/or set (also called ``individual'') metavariables,
accompanied by optional provisos each stating that two specified set
metavariables must be distinct or that a specified set metavariable may
not occur in a specified wff metavariable. Metamath's logic and set theory
axiom and rule schemes are all examples of simple metalogic. The schemes
of traditional predicate calculus with equality are examples which are
not simple metalogic, because they use wff metavariables with arguments
and have ``free for'' and ``not free in'' side conditions.

A rigorous justification for this system, using an older but
exactly equivalent set of axioms, can be
found in \cite{Megill}\index{Megill, Norman}.

This allows us to
take a different approach in the Metamath\index{Metamath} database
\texttt{set.mm}\index{set theory database (\texttt{set.mm})}.  We do not
directly use the primitive notions of ``free variable''\index{free variable}
and ``proper substitution''\index{proper
substitution}\index{substitution!proper} at all as primitive constructs.
Instead, we use a set
of axioms that are almost as simple to manipulate as those of
propositional calculus.  Our axiom system avoids complex primitive
notions by effectively embedding the complexity into the axioms
themselves.  As a result, we will end up with a larger number of axioms,
but they are ideally suited for a computer language such as Metamath.
(Section~\ref{metaaxioms} shows these axioms.)

We will not elaborate further
on the ``free variable'' and ``proper substitution''
concepts here.  You may consult
\cite[ch.\ 3--4]{Hamilton}\index{Hamilton, Alan G.} (as well as
many other books) for a precise explanation
of these concepts.  If you intend to do serious mathematical work, it is wise
to become familiar with the traditional textbook approach; even though the
concepts embedded in their axioms require a higher level of sophistication,
they can be more practical to deal with on an everyday, informal basis.  Even
if you are just developing Metamath proofs, familiarity with the traditional
approach can help you arrive at a proof outline much faster, which you can
then convert to the detail required by Metamath.

We do develop proper substitution rules later on, but in set.mm
they are defined as derived constructs; they are not primitives.

You should also note that our system of predicate calculus is specifically
tailored for set theory; thus there are only two specific predicates $=$ and
$\in$ and no functions\index{function!in predicate calculus}
or constants\index{constant!in predicate calculus} unlike more general systems.
We later add these.

\subsection{Set Theory}

Traditional Zermelo--Fraenkel set theory\index{Zermelo--Fraenkel set
theory}\index{set theory} with the Axiom of Choice
has 10 axioms, which can be expressed in the
language of predicate calculus.  In this section, we will list only the
names and brief English descriptions of these axioms, since we will give
you the precise formulas used by the Metamath\index{Metamath} set theory
database \texttt{set.mm} later on.

In the descriptions of the axioms, we assume that $x$, $y$, $z$, $w$, and $v$
represent sets.  These are the same as the variables\index{variable!in set
theory} in our predicate calculus system above, except that now we informally
think of the variables as ranging over sets.  Note that the terms
``object,''\index{object} ``set,''\index{set} ``element,''\index{element}
``collection,''\index{collection} and ``family''\index{family} are synonymous,
as are ``is an element of,'' ``is a member of,''\index{member} ``is contained
in,'' and ``belongs to.''  The different terms are used for convenience; for
example, ``a collection of sets'' is less confusing than ``a set of sets.''
A set $x$ is said to be a ``subset''\index{subset} of $y$ if every element of
$x$ is also an element of $y$; we also say $x$ is ``included in''
$y$.

The axioms are very general and apply to almost any conceivable mathematical
object, and this level of abstraction can be overwhelming at first.  To gain an
intuitive feel, it can be helpful to draw a picture illustrating the concept;
for example, a circle containing dots could represent a collection of sets,
and a smaller circle drawn inside the circle could represent a subset.
Overlapping circles can illustrate intersection and union.  Circles that
illustrate the concepts of set theory are frequently used in elementary
textbooks and are called Venn diagrams\index{Venn diagram}.\index{axioms of
set theory}

1. Axiom of Extensionality:  Two sets are identical if they contain the same
   elements.\index{Axiom of Extensionality}

2. Axiom of Pairing:  The set $\{ x , y \}$ exists.\index{Axiom of Pairing}

3. Axiom of Power Sets:  The power set of a set (the collection of all of
   its subsets) exists.  For example, the power set of $\{x,y\}$ is
   $\{\varnothing,\{x\},\{y\},\{x,y\}\}$ and it exists.\index{Axiom
of Power Sets}

4. Axiom of the Null Set:  The empty set $\varnothing$ exists.\index{Axiom of
the Null Set}

5. Axiom of Union:  The union of a set (the set containing the elements of
   its members) exists.  For example, the union of $\{\{x,y\},\{z\}\}$ is
 $\{x,y,z\}$ and
   it exists.\index{Axiom of Union}

6. Axiom of Regularity:  Roughly, no set can contain itself, nor can there
   be membership ``loops,'' such as a set being an
   element of one of its members.\index{Axiom of Regularity}

7. Axiom of Infinity:  An infinite set exists.  An example of an infinite
   set is the set of all
   integers.\index{Axiom of Infinity}

8. Axiom of Separation:  The set exists that is obtained by restricting $x$
   with some property.  For example, if the set of all integers exists,
   then the set of all even integers exists.\index{Axiom of Separation}

9. Axiom of Replacement:  The range of a function whose domain is restricted
   to the elements of a set $x$, is also a set.  For example, there
   is a function
   from integers (the function's domain) to their squares (its
   range).  If we
   restrict the domain to even integers, its range will become the set of
   squares of even integers, so this axiom asserts that the set of
    squares of even numbers exists.  Technical note:  In general, the
   ``function'' need not be a set but can be a proper class.
   \index{Axiom of Replacement}

10. Axiom of Choice:  Let $x$ be a set whose members are pairwise
  disjoint\index{disjoint sets} (i.e,
  whose members contain no elements in common).  Then there exists another
  set containing one element from each member of $x$.  For
  example, if $x$ is
  $\{\{y,z\},\{w,v\}\}$, where $y$, $z$, $w$, and $v$ are
  different sets, then a set such as $\{z,w\}$
  exists (but the axiom doesn't tell
  us which one).  (Actually the Axiom
  of Choice is redundant if the set $x$, as in this example, has a finite
  number of elements.)\index{Axiom of Choice}

The Axiom of Choice is usually considered an extension of ZF set theory rather
than a proper part of it.  It is sometimes considered philosophically
controversial because it specifies the existence of a set without specifying
what the set is. Constructive logics, including intuitionistic logic,
do not accept the axiom of choice.
Since there is some lingering controversy, we often prefer proofs that do
not use the axiom of choice (where there is a known alternative), and
in some cases we will use weaker axioms than the full axiom of choice.
That said, the axiom of choice is a powerful and widely-accepted tool,
so we do use it when needed.
ZF set theory that includes the Axiom of Choice is
called Zermelo--Fraenkel set theory with choice (ZFC\index{ZFC set theory}).

When expressed symbolically, the Axiom of Separation and the Axiom of
Replacement contain wff symbols and therefore each represent infinitely many
axioms, one for each possible wff. For this reason, they are often called
axiom schemes\index{axiom scheme}\index{well-formed formula (wff)}.

It turns out that the Axiom of the Null Set, the Axiom of Pairing, and the
Axiom of Separation can be derived from the other axioms and are therefore
unnecessary, although they tend to be included in standard texts for various
reasons (historical, philosophical, and possibly because some authors may not
know this).  In the Metamath\index{Metamath} set theory database, these
redundant axioms are derived from the other ones instead of truly
being considered axioms.
This is in keeping with our general goal of minimizing the number of
axioms we must depend on.

\subsection{Other Axioms}

Above we qualified the phrase "all of mathematics" with "essentially."
The main important missing piece is the ability to do category theory,
which requires huge sets (inaccessible cardinals) larger than those
postulated by the ZFC axioms. The Tarski--Grothendieck Axiom postulates
the existence of such sets.
Note that this is the same axiom used by Mizar for supporting
category theory.
The Tarski--Grothendieck axiom
can be viewed as a very strong replacement of the Axiom of Infinity,
the Axiom of Choice, and the Axiom of Power Sets.
The \texttt{set.mm} database includes this axiom; see the database
for details about it.
Again, we only use this axiom when we need to.
You are only likely to encounter or use this axiom if you are doing
category theory, since its use is highly specialized,
so we will not list the Tarsky-Grothendieck axiom
in the short list of axioms below.

Can there be even more axioms?
Of course.
G\"{o}del showed that no finite set of axioms or axiom schemes can completely
describe any consistent theory strong enough to include arithmetic.
But practically speaking, the ones above are the accepted foundation that
almost all mathematicians explicitly or implicitly base their work on.

\section{The Axioms in the Metamath Language}\label{metaaxioms}

Here we list the axioms as they appear in
\texttt{set.mm}\index{set theory database (\texttt{set.mm})} so you can
look them up there easily.  Incidentally, the \texttt{show statement
/tex} command\index{\texttt{show statement} command} was used to
typeset them.

%macros from show statement /tex
\newbox\mlinebox
\newbox\mtrialbox
\newbox\startprefix  % Prefix for first line of a formula
\newbox\contprefix  % Prefix for continuation line of a formula
\def\startm{  % Initialize formula line
  \setbox\mlinebox=\hbox{\unhcopy\startprefix}
}
\def\m#1{  % Add a symbol to the formula
  \setbox\mtrialbox=\hbox{\unhcopy\mlinebox $\,#1$}
  \ifdim\wd\mtrialbox>\hsize
    \box\mlinebox
    \setbox\mlinebox=\hbox{\unhcopy\contprefix $\,#1$}
  \else
    \setbox\mlinebox=\hbox{\unhbox\mtrialbox}
  \fi
}
\def\endm{  % Output the last line of a formula
  \box\mlinebox
}

% \SLASH for \ , \TOR for \/ (text OR), \TAND for /\ (text and)
% This embeds a following forced space to force the space.
\newcommand\SLASH{\char`\\~}
\newcommand\TOR{\char`\\/~}
\newcommand\TAND{/\char`\\~}
%
% Macro to output metamath raw text.
% This assumes \startprefix and \contprefix are set.
% NOTE: "\" is tricky to escape, use \SLASH, \TOR, and \TAND inside.
% Any use of "$ { ~ ^" must be escaped; ~ and ^ must be escaped specially.
% We escape { and } for consistency.
% For more about how this macro written, see:
% https://stackoverflow.com/questions/4073674/
% how-to-disable-indentation-in-particular-section-in-latex/4075706
% Use frenchspacing, or "e." will get an extra space after it.
\newlength\mystoreparindent
\newlength\mystorehangindent
\newenvironment{mmraw}{%
\setlength{\mystoreparindent}{\the\parindent}
\setlength{\mystorehangindent}{\the\hangindent}
\setlength{\parindent}{0pt} % TODO - we'll put in the \startprefix instead
\setlength{\hangindent}{\wd\the\contprefix}
\begin{flushleft}
\begin{frenchspacing}
\begin{tt}
{\unhcopy\startprefix}%
}{%
\end{tt}
\end{frenchspacing}
\end{flushleft}
\setlength{\parindent}{\mystoreparindent}
\setlength{\hangindent}{\mystorehangindent}
\vskip 1ex
}

\needspace{5\baselineskip}
\subsection{Propositional Calculus}\label{propcalc}\index{axioms of
propositional calculus}

\needspace{2\baselineskip}
Axiom of Simplification.\label{ax1}

\setbox\startprefix=\hbox{\tt \ \ ax-1\ \$a\ }
\setbox\contprefix=\hbox{\tt \ \ \ \ \ \ \ \ \ \ }
\startm
\m{\vdash}\m{(}\m{\varphi}\m{\rightarrow}\m{(}\m{\psi}\m{\rightarrow}\m{\varphi}\m{)}
\m{)}
\endm

\needspace{3\baselineskip}
\noindent Axiom of Distribution.

\setbox\startprefix=\hbox{\tt \ \ ax-2\ \$a\ }
\setbox\contprefix=\hbox{\tt \ \ \ \ \ \ \ \ \ \ }
\startm
\m{\vdash}\m{(}\m{(}\m{\varphi}\m{\rightarrow}\m{(}\m{\psi}\m{\rightarrow}\m{\chi}
\m{)}\m{)}\m{\rightarrow}\m{(}\m{(}\m{\varphi}\m{\rightarrow}\m{\psi}\m{)}\m{
\rightarrow}\m{(}\m{\varphi}\m{\rightarrow}\m{\chi}\m{)}\m{)}\m{)}
\endm

\needspace{2\baselineskip}
\noindent Axiom of Contraposition.

\setbox\startprefix=\hbox{\tt \ \ ax-3\ \$a\ }
\setbox\contprefix=\hbox{\tt \ \ \ \ \ \ \ \ \ \ }
\startm
\m{\vdash}\m{(}\m{(}\m{\lnot}\m{\varphi}\m{\rightarrow}\m{\lnot}\m{\psi}\m{)}\m{
\rightarrow}\m{(}\m{\psi}\m{\rightarrow}\m{\varphi}\m{)}\m{)}
\endm


\needspace{4\baselineskip}
\noindent Rule of Modus Ponens.\label{axmp}\index{modus ponens}

\setbox\startprefix=\hbox{\tt \ \ maj\ \$e\ }
\setbox\contprefix=\hbox{\tt \ \ \ \ \ \ \ \ \ }
\startm
\m{\vdash}\m{(}\m{\varphi}\m{\rightarrow}\m{\psi}\m{)}
\endm

\setbox\startprefix=\hbox{\tt \ \ min\ \$e\ }
\setbox\contprefix=\hbox{\tt \ \ \ \ \ \ \ \ \ }
\startm
\m{\vdash}\m{\varphi}
\endm

\setbox\startprefix=\hbox{\tt \ \ ax-mp\ \$a\ }
\setbox\contprefix=\hbox{\tt \ \ \ \ \ \ \ \ \ \ \ }
\startm
\m{\vdash}\m{\psi}
\endm


\needspace{7\baselineskip}
\subsection{Axioms of Predicate Calculus with Equality---Tarski's S2}\index{axioms of predicate calculus}

\needspace{3\baselineskip}
\noindent Rule of Generalization.\index{rule of generalization}

\setbox\startprefix=\hbox{\tt \ \ ax-g.1\ \$e\ }
\setbox\contprefix=\hbox{\tt \ \ \ \ \ \ \ \ \ \ \ \ }
\startm
\m{\vdash}\m{\varphi}
\endm

\setbox\startprefix=\hbox{\tt \ \ ax-gen\ \$a\ }
\setbox\contprefix=\hbox{\tt \ \ \ \ \ \ \ \ \ \ \ \ }
\startm
\m{\vdash}\m{\forall}\m{x}\m{\varphi}
\endm

\needspace{2\baselineskip}
\noindent Axiom of Quantified Implication.

\setbox\startprefix=\hbox{\tt \ \ ax-4\ \$a\ }
\setbox\contprefix=\hbox{\tt \ \ \ \ \ \ \ \ \ \ }
\startm
\m{\vdash}\m{(}\m{\forall}\m{x}\m{(}\m{\forall}\m{x}\m{\varphi}\m{\rightarrow}\m{
\psi}\m{)}\m{\rightarrow}\m{(}\m{\forall}\m{x}\m{\varphi}\m{\rightarrow}\m{
\forall}\m{x}\m{\psi}\m{)}\m{)}
\endm

\needspace{3\baselineskip}
\noindent Axiom of Distinctness.

% Aka: Add $d x ph $.
\setbox\startprefix=\hbox{\tt \ \ ax-5\ \$a\ }
\setbox\contprefix=\hbox{\tt \ \ \ \ \ \ \ \ \ \ }
\startm
\m{\vdash}\m{(}\m{\varphi}\m{\rightarrow}\m{\forall}\m{x}\m{\varphi}\m{)}\m{where}\m{ }\m{\$d}\m{ }\m{x}\m{ }\m{\varphi}\m{ }\m{(}\m{x}\m{ }\m{does}\m{ }\m{not}\m{ }\m{occur}\m{ }\m{in}\m{ }\m{\varphi}\m{)}
\endm

\needspace{2\baselineskip}
\noindent Axiom of Existence.

\setbox\startprefix=\hbox{\tt \ \ ax-6\ \$a\ }
\setbox\contprefix=\hbox{\tt \ \ \ \ \ \ \ \ \ \ }
\startm
\m{\vdash}\m{(}\m{\forall}\m{x}\m{(}\m{x}\m{=}\m{y}\m{\rightarrow}\m{\forall}
\m{x}\m{\varphi}\m{)}\m{\rightarrow}\m{\varphi}\m{)}
\endm

\needspace{2\baselineskip}
\noindent Axiom of Equality.

\setbox\startprefix=\hbox{\tt \ \ ax-7\ \$a\ }
\setbox\contprefix=\hbox{\tt \ \ \ \ \ \ \ \ \ \ }
\startm
\m{\vdash}\m{(}\m{x}\m{=}\m{y}\m{\rightarrow}\m{(}\m{x}\m{=}\m{z}\m{
\rightarrow}\m{y}\m{=}\m{z}\m{)}\m{)}
\endm

\needspace{2\baselineskip}
\noindent Axiom of Left Equality for Binary Predicate.

\setbox\startprefix=\hbox{\tt \ \ ax-8\ \$a\ }
\setbox\contprefix=\hbox{\tt \ \ \ \ \ \ \ \ \ \ \ }
\startm
\m{\vdash}\m{(}\m{x}\m{=}\m{y}\m{\rightarrow}\m{(}\m{x}\m{\in}\m{z}\m{
\rightarrow}\m{y}\m{\in}\m{z}\m{)}\m{)}
\endm

\needspace{2\baselineskip}
\noindent Axiom of Right Equality for Binary Predicate.

\setbox\startprefix=\hbox{\tt \ \ ax-9\ \$a\ }
\setbox\contprefix=\hbox{\tt \ \ \ \ \ \ \ \ \ \ \ }
\startm
\m{\vdash}\m{(}\m{x}\m{=}\m{y}\m{\rightarrow}\m{(}\m{z}\m{\in}\m{x}\m{
\rightarrow}\m{z}\m{\in}\m{y}\m{)}\m{)}
\endm


\needspace{4\baselineskip}
\subsection{Axioms of Predicate Calculus with Equality---Auxiliary}\index{axioms of predicate calculus - auxiliary}

\needspace{2\baselineskip}
\noindent Axiom of Quantified Negation.

\setbox\startprefix=\hbox{\tt \ \ ax-10\ \$a\ }
\setbox\contprefix=\hbox{\tt \ \ \ \ \ \ \ \ \ \ }
\startm
\m{\vdash}\m{(}\m{\lnot}\m{\forall}\m{x}\m{\lnot}\m{\forall}\m{x}\m{\varphi}\m{
\rightarrow}\m{\varphi}\m{)}
\endm

\needspace{2\baselineskip}
\noindent Axiom of Quantifier Commutation.

\setbox\startprefix=\hbox{\tt \ \ ax-11\ \$a\ }
\setbox\contprefix=\hbox{\tt \ \ \ \ \ \ \ \ \ \ }
\startm
\m{\vdash}\m{(}\m{\forall}\m{x}\m{\forall}\m{y}\m{\varphi}\m{\rightarrow}\m{
\forall}\m{y}\m{\forall}\m{x}\m{\varphi}\m{)}
\endm

\needspace{3\baselineskip}
\noindent Axiom of Substitution.

\setbox\startprefix=\hbox{\tt \ \ ax-12\ \$a\ }
\setbox\contprefix=\hbox{\tt \ \ \ \ \ \ \ \ \ \ \ }
\startm
\m{\vdash}\m{(}\m{\lnot}\m{\forall}\m{x}\m{\,x}\m{=}\m{y}\m{\rightarrow}\m{(}
\m{x}\m{=}\m{y}\m{\rightarrow}\m{(}\m{\varphi}\m{\rightarrow}\m{\forall}\m{x}\m{(}
\m{x}\m{=}\m{y}\m{\rightarrow}\m{\varphi}\m{)}\m{)}\m{)}\m{)}
\endm

\needspace{3\baselineskip}
\noindent Axiom of Quantified Equality.

\setbox\startprefix=\hbox{\tt \ \ ax-13\ \$a\ }
\setbox\contprefix=\hbox{\tt \ \ \ \ \ \ \ \ \ \ \ }
\startm
\m{\vdash}\m{(}\m{\lnot}\m{\forall}\m{z}\m{\,z}\m{=}\m{x}\m{\rightarrow}\m{(}
\m{\lnot}\m{\forall}\m{z}\m{\,z}\m{=}\m{y}\m{\rightarrow}\m{(}\m{x}\m{=}\m{y}
\m{\rightarrow}\m{\forall}\m{z}\m{\,x}\m{=}\m{y}\m{)}\m{)}\m{)}
\endm

% \noindent Axiom of Quantifier Substitution
%
% \setbox\startprefix=\hbox{\tt \ \ ax-c11n\ \$a\ }
% \setbox\contprefix=\hbox{\tt \ \ \ \ \ \ \ \ \ \ \ }
% \startm
% \m{\vdash}\m{(}\m{\forall}\m{x}\m{\,x}\m{=}\m{y}\m{\rightarrow}\m{(}\m{\forall}
% \m{x}\m{\varphi}\m{\rightarrow}\m{\forall}\m{y}\m{\varphi}\m{)}\m{)}
% \endm
%
% \noindent Axiom of Distinct Variables. (This axiom requires
% that two individual variables
% be distinct\index{\texttt{\$d} statement}\index{distinct
% variables}.)
%
% \setbox\startprefix=\hbox{\tt \ \ \ \ \ \ \ \ \$d\ }
% \setbox\contprefix=\hbox{\tt \ \ \ \ \ \ \ \ \ \ \ }
% \startm
% \m{x}\m{\,}\m{y}
% \endm
%
% \setbox\startprefix=\hbox{\tt \ \ ax-c16\ \$a\ }
% \setbox\contprefix=\hbox{\tt \ \ \ \ \ \ \ \ \ \ \ }
% \startm
% \m{\vdash}\m{(}\m{\forall}\m{x}\m{\,x}\m{=}\m{y}\m{\rightarrow}\m{(}\m{\varphi}\m{
% \rightarrow}\m{\forall}\m{x}\m{\varphi}\m{)}\m{)}
% \endm

% \noindent Axiom of Quantifier Introduction (2).  (This axiom requires
% that the individual variable not occur in the
% wff\index{\texttt{\$d} statement}\index{distinct variables}.)
%
% \setbox\startprefix=\hbox{\tt \ \ \ \ \ \ \ \ \$d\ }
% \setbox\contprefix=\hbox{\tt \ \ \ \ \ \ \ \ \ \ \ }
% \startm
% \m{x}\m{\,}\m{\varphi}
% \endm
% \setbox\startprefix=\hbox{\tt \ \ ax-5\ \$a\ }
% \setbox\contprefix=\hbox{\tt \ \ \ \ \ \ \ \ \ \ \ }
% \startm
% \m{\vdash}\m{(}\m{\varphi}\m{\rightarrow}\m{\forall}\m{x}\m{\varphi}\m{)}
% \endm

\subsection{Set Theory}\label{mmsettheoryaxioms}

In order to make the axioms of set theory\index{axioms of set theory} a little
more compact, there are several definitions from logic that we make use of
implicitly, namely, ``logical {\sc and},''\index{conjunction ($\wedge$)}
\index{logical {\sc and} ($\wedge$)} ``logical equivalence,''\index{logical
equivalence ($\leftrightarrow$)}\index{biconditional ($\leftrightarrow$)} and
``there exists.''\index{existential quantifier ($\exists$)}

\begin{center}\begin{tabular}{rcl}
  $( \varphi \wedge \psi )$ &\mbox{stands for}& $\neg ( \varphi
     \rightarrow \neg \psi )$\\
  $( \varphi \leftrightarrow \psi )$& \mbox{stands
     for}& $( ( \varphi \rightarrow \psi ) \wedge
     ( \psi \rightarrow \varphi ) )$\\
  $\exists x \,\varphi$ &\mbox{stands for}& $\neg \forall x \neg \varphi$
\end{tabular}\end{center}

In addition, the axioms of set theory require that all variables be
dis\-tinct,\index{distinct variables}\footnote{Set theory axioms can be
devised so that {\em no} variables are required to be distinct,
provided we replace \texttt{ax-c16} with an axiom stating that ``at
least two things exist,'' thus
making \texttt{ax-5} the only other axiom requiring the
\texttt{\$d} statement.  These axioms are unconventional and are not
presented here, but they can be found on the \url{http://metamath.org}
web site.  See also the Comment on
p.~\pageref{nodd}.}\index{\texttt{\$d} statement} thus we also assume:
\begin{center}
  \texttt{\$d }$x\,y\,z\,w$
\end{center}

\needspace{2\baselineskip}
\noindent Axiom of Extensionality.\index{Axiom of Extensionality}

\setbox\startprefix=\hbox{\tt \ \ ax-ext\ \$a\ }
\setbox\contprefix=\hbox{\tt \ \ \ \ \ \ \ \ \ \ \ \ }
\startm
\m{\vdash}\m{(}\m{\forall}\m{x}\m{(}\m{x}\m{\in}\m{y}\m{\leftrightarrow}\m{x}
\m{\in}\m{z}\m{)}\m{\rightarrow}\m{y}\m{=}\m{z}\m{)}
\endm

\needspace{3\baselineskip}
\noindent Axiom of Replacement.\index{Axiom of Replacement}

\setbox\startprefix=\hbox{\tt \ \ ax-rep\ \$a\ }
\setbox\contprefix=\hbox{\tt \ \ \ \ \ \ \ \ \ \ \ \ }
\startm
\m{\vdash}\m{(}\m{\forall}\m{w}\m{\exists}\m{y}\m{\forall}\m{z}\m{(}\m{%
\forall}\m{y}\m{\varphi}\m{\rightarrow}\m{z}\m{=}\m{y}\m{)}\m{\rightarrow}\m{%
\exists}\m{y}\m{\forall}\m{z}\m{(}\m{z}\m{\in}\m{y}\m{\leftrightarrow}\m{%
\exists}\m{w}\m{(}\m{w}\m{\in}\m{x}\m{\wedge}\m{\forall}\m{y}\m{\varphi}\m{)}%
\m{)}\m{)}
\endm

\needspace{2\baselineskip}
\noindent Axiom of Union.\index{Axiom of Union}

\setbox\startprefix=\hbox{\tt \ \ ax-un\ \$a\ }
\setbox\contprefix=\hbox{\tt \ \ \ \ \ \ \ \ \ \ \ }
\startm
\m{\vdash}\m{\exists}\m{x}\m{\forall}\m{y}\m{(}\m{\exists}\m{x}\m{(}\m{y}\m{
\in}\m{x}\m{\wedge}\m{x}\m{\in}\m{z}\m{)}\m{\rightarrow}\m{y}\m{\in}\m{x}\m{)}
\endm

\needspace{2\baselineskip}
\noindent Axiom of Power Sets.\index{Axiom of Power Sets}

\setbox\startprefix=\hbox{\tt \ \ ax-pow\ \$a\ }
\setbox\contprefix=\hbox{\tt \ \ \ \ \ \ \ \ \ \ \ \ }
\startm
\m{\vdash}\m{\exists}\m{x}\m{\forall}\m{y}\m{(}\m{\forall}\m{x}\m{(}\m{x}\m{
\in}\m{y}\m{\rightarrow}\m{x}\m{\in}\m{z}\m{)}\m{\rightarrow}\m{y}\m{\in}\m{x}
\m{)}
\endm

\needspace{3\baselineskip}
\noindent Axiom of Regularity.\index{Axiom of Regularity}

\setbox\startprefix=\hbox{\tt \ \ ax-reg\ \$a\ }
\setbox\contprefix=\hbox{\tt \ \ \ \ \ \ \ \ \ \ \ \ }
\startm
\m{\vdash}\m{(}\m{\exists}\m{x}\m{\,x}\m{\in}\m{y}\m{\rightarrow}\m{\exists}
\m{x}\m{(}\m{x}\m{\in}\m{y}\m{\wedge}\m{\forall}\m{z}\m{(}\m{z}\m{\in}\m{x}\m{
\rightarrow}\m{\lnot}\m{z}\m{\in}\m{y}\m{)}\m{)}\m{)}
\endm

\needspace{3\baselineskip}
\noindent Axiom of Infinity.\index{Axiom of Infinity}

\setbox\startprefix=\hbox{\tt \ \ ax-inf\ \$a\ }
\setbox\contprefix=\hbox{\tt \ \ \ \ \ \ \ \ \ \ \ \ \ \ \ }
\startm
\m{\vdash}\m{\exists}\m{x}\m{(}\m{y}\m{\in}\m{x}\m{\wedge}\m{\forall}\m{y}%
\m{(}\m{y}\m{\in}\m{x}\m{\rightarrow}\m{\exists}\m{z}\m{(}\m{y}\m{\in}\m{z}\m{%
\wedge}\m{z}\m{\in}\m{x}\m{)}\m{)}\m{)}
\endm

\needspace{4\baselineskip}
\noindent Axiom of Choice.\index{Axiom of Choice}

\setbox\startprefix=\hbox{\tt \ \ ax-ac\ \$a\ }
\setbox\contprefix=\hbox{\tt \ \ \ \ \ \ \ \ \ \ \ \ \ \ }
\startm
\m{\vdash}\m{\exists}\m{x}\m{\forall}\m{y}\m{\forall}\m{z}\m{(}\m{(}\m{y}\m{%
\in}\m{z}\m{\wedge}\m{z}\m{\in}\m{w}\m{)}\m{\rightarrow}\m{\exists}\m{w}\m{%
\forall}\m{y}\m{(}\m{\exists}\m{w}\m{(}\m{(}\m{y}\m{\in}\m{z}\m{\wedge}\m{z}%
\m{\in}\m{w}\m{)}\m{\wedge}\m{(}\m{y}\m{\in}\m{w}\m{\wedge}\m{w}\m{\in}\m{x}%
\m{)}\m{)}\m{\leftrightarrow}\m{y}\m{=}\m{w}\m{)}\m{)}
\endm

\subsection{That's It}

There you have it, the axioms for (essentially) all of mathematics!
Wonder at them and stare at them in awe.  Put a copy in your wallet, and
you will carry in your pocket the encoding for all theorems ever proved
and that ever will be proved, from the most mundane to the most
profound.

\section{A Hierarchy of Definitions}\label{hierarchy}

The axioms in the previous section in principle embody everything that can be
done within standard mathematics.  However, it is impractical to accomplish
very much by using them directly, for even simple concepts (from a human
perspective) can involve extremely long, incomprehensible formulas.
Mathematics is made practical by introducing definitions\index{definition}.
Definitions usually introduce new symbols, or at least new relationships among
existing symbols, to abbreviate more complex formulas.  An important
requirement for a definition is that there exist a straightforward
(algorithmic) method for eliminating the abbreviation by expanding it into the
more primitive symbol string that it represents.  Some
important definitions included in
the file \texttt{set.mm} are listed in this section for reference, and also to
give you a feel for why something like $\omega$\index{omega ($\omega$)} (the
set of natural numbers\index{natural number} 0, 1, 2,\ldots) becomes very
complicated when completely expanded into primitive symbols.

What is the motivation for definitions, aside from allowing complicated
expressions to be expressed more simply?  In the case of  $\omega$, one goal is
to provide a basis for the theory of natural numbers.\index{natural number}
Before set theory was invented, a set of axioms for arithmetic, called Peano's
postulates\index{Peano's postulates}, was devised and shown to have the
properties one expects for natural numbers.  Now anyone can postulate a
set of axioms, but if the axioms are inconsistent contradictions can be derived
from them.  Once a contradiction is derived, anything can be trivially
proved, including
all the facts of arithmetic and their negations.  To ensure that an
axiom system is at least as reliable as the axioms for set theory, we can
define sets and operations on those sets that satisfy the new axioms. In the
\texttt{set.mm} Metamath database, we prove that the elements of $\omega$ satisfy
Peano's postulates, and it's a long and hard journey to get there directly
from the axioms of set theory.  But the result is confidence in the
foundations of arithmetic.  And there is another advantage:  we now have all
the tools of set theory at our disposal for manipulating objects that obey the
axioms for arithmetic.

What are the criteria we use for definitions?  First, and of utmost importance,
the definition should not be {\em creative}\index{creative
definition}\index{definition!creative}, that
is it should not allow an expression that previously qualified as a wff but
was not provable, to become provable.   Second, the definition should be {\em
eliminable}\index{definition!eliminability}, that is, there should exist an
algorithmic method for converting any expression using the definition into
a logically equivalent expression that previously qualified as a wff.

In almost all cases below, definitions connect two expressions with either
$\leftrightarrow$ or $=$.  Eliminating\footnote{Here we mean the
elimination that a human might do in his or her head.  To eliminate them as
part of a Metamath proof we would invoke one of a number of
theorems that deal with transitivity of equivalence or equality; there are
many such examples in the proofs in \texttt{set.mm}.} such a definition is a
simple matter of substituting the expression on the left-hand side ({\em
definiendum}\index{definiendum} or thing being defined) with the equivalent,
more primitive expression on the right-hand side ({\em
definiens}\index{definiens} or definition).

Often a definition has variables on the right-hand side which do not appear on
the left-hand side; these are called {\em dummy variables}.\index{dummy
variable!in definitions}  In this case, any
allowable substitution (such as a new, distinct
variable) can be used when the definition is eliminated.  Dummy variables may
be used only if they are {\em effectively bound}\index{effectively bound
variable}, meaning that the definition will remain logically equivalent upon
any substitution of a dummy variable with any other {\em qualifying
expression}\index{qualifying expression}, i.e.\ any symbol string (such as
another variable) that
meets the restrictions on the dummy variable imposed by \texttt{\$d} and
\texttt{\$f} statements.  For example, we could define a constant $\perp$
(inverted tee, meaning logical ``false'') as $( \varphi \wedge \lnot \varphi
)$, i.e.\ ``phi and not phi.''  Here $\varphi$ is effectively bound because the
definition remains logically equivalent when we replace $\varphi$ with any
other wff.  (It is actually \texttt{df-fal}
in \texttt{set.mm}, which defines $\perp$.)

There are two cases where eliminating definitions is a little more
complex.  These cases are the definitions \texttt{df-bi} and
\texttt{df-cleq}.  The first stretches the concept of a definition a
little, as in effect it ``defines a definition;'' however, it meets our
requirements for a definition in that it is eliminable and does not
strengthen the language.  Theorem \texttt{bii} shows the substitution
needed to eliminate the $\leftrightarrow$\index{logical equivalence
($\leftrightarrow$)}\index{biconditional ($\leftrightarrow$)} symbol.

Definition \texttt{df-cleq}\index{equality ($=$)} extends the usage of
the equality symbol to include ``classes''\index{class} in set theory.  The
reason it is potentially problematic is that it can lead to statements which
do not follow from logic alone but presuppose the Axiom of
Extensionality\index{Axiom of Extensionality}, so we include this axiom
as a hypothesis for the definition.  We could have made \texttt{df-cleq} directly
eliminable by introducing a new equality symbol, but have chosen not to do so
in keeping with standard textbook practice.  Definitions such as \texttt{df-cleq}
that extend the meaning of existing symbols must be introduced carefully so
that they do not lead to contradictions.  Definition \texttt{df-clel} also
extends the meaning of an existing symbol ($\in$); while it doesn't strengthen
the language like \texttt{df-cleq}, this is not obvious and it must also be
subject to the same scrutiny.

Exercise:  Study how the wff $x\in\omega$, meaning ``$x$ is a natural
number,'' could be expanded in terms of primitive symbols, starting with the
definitions \texttt{df-clel} on p.~\pageref{dfclel} and \texttt{df-om} on
p.~\pageref{dfom} and working your way back.  Don't bother to work out the
details; just make sure that you understand how you could do it in principle.
The answer is shown in the footnote on p.~\pageref{expandom}.  If you
actually do work it out, you won't get exactly the same answer because we used
a few simplifications such as discarding occurrences of $\lnot\lnot$ (double
negation).

In the definitions below, we have placed the {\sc ascii} Metamath source
below each of the formulas to help you become familiar with the
notation in the database.  For simplicity, the necessary \texttt{\$f}
and \texttt{\$d} statements are not shown.  If you are in doubt, use the
\texttt{show statement}\index{\texttt{show statement} command} command
in the Metamath program to see the full statement.
A selection of this notation is summarized in Appendix~\ref{ASCII}.

To understand the motivation for these definitions, you should consult the
references indicated:  Takeuti and Zaring \cite{Takeuti}\index{Takeuti, Gaisi},
Quine \cite{Quine}\index{Quine, Willard Van Orman}, Bell and Machover
\cite{Bell}\index{Bell, J. L.}, and Enderton \cite{Enderton}\index{Enderton,
Herbert B.}.  Our list of definitions is provided more for reference than as a
learning aid.  However, by looking at a few of them you can gain a feel for
how the hierarchy is built up.  The definitions are a representative sample of
the 130 or so in \texttt{set.mm}, but they are complete with respect to the
theorem examples we will present in Section~\ref{sometheorems}.  Also, some are
slightly different from, but logically equivalent to, the ones in \texttt{set.mm}
(some of which have been revised over time to shorten them, for example).

\subsection{Definitions for Propositional Calculus}\label{metadefprop}

The symbols $\varphi$, $\psi$, and $\chi$ represent wffs.

Our first definition introduces the biconditional
connective\footnote{The term ``connective'' is informally used to mean a
symbol that is placed between two variables or adjacent to a variable,
whereas a mathematical ``constant'' usually indicates a symbol such as
the number 0 that may replace a variable or metavariable.  From
Metamath's point of view, there is no distinction between a connective
and a constant; both are constants in the Metamath
language.}\index{connective}\index{constant} (also called logical
equivalence)\index{logical equivalence
($\leftrightarrow$)}\index{biconditional ($\leftrightarrow$)}.  Unlike
most traditional developments, we have chosen not to have a separate
symbol such as ``Df.'' to mean ``is defined as.''  Instead, we will use
the biconditional connective for this purpose, as it lets us use
logic to manipulate definitions directly.  Here we state the properties
of the biconditional connective with a carefully crafted \texttt{\$a}
statement, which effectively uses the biconditional connective to define
itself.  The $\leftrightarrow$ symbol can be eliminated from a formula
using theorem \texttt{bii}, which is derived later.

\vskip 2ex
\noindent Define the biconditional connective.\label{df-bi}

\vskip 0.5ex
\setbox\startprefix=\hbox{\tt \ \ df-bi\ \$a\ }
\setbox\contprefix=\hbox{\tt \ \ \ \ \ \ \ \ \ \ \ }
\startm
\m{\vdash}\m{\lnot}\m{(}\m{(}\m{(}\m{\varphi}\m{\leftrightarrow}\m{\psi}\m{)}%
\m{\rightarrow}\m{\lnot}\m{(}\m{(}\m{\varphi}\m{\rightarrow}\m{\psi}\m{)}\m{%
\rightarrow}\m{\lnot}\m{(}\m{\psi}\m{\rightarrow}\m{\varphi}\m{)}\m{)}\m{)}\m{%
\rightarrow}\m{\lnot}\m{(}\m{\lnot}\m{(}\m{(}\m{\varphi}\m{\rightarrow}\m{%
\psi}\m{)}\m{\rightarrow}\m{\lnot}\m{(}\m{\psi}\m{\rightarrow}\m{\varphi}\m{)}%
\m{)}\m{\rightarrow}\m{(}\m{\varphi}\m{\leftrightarrow}\m{\psi}\m{)}\m{)}\m{)}
\endm
\begin{mmraw}%
|- -. ( ( ( ph <-> ps ) -> -. ( ( ph -> ps ) ->
-. ( ps -> ph ) ) ) -> -. ( -. ( ( ph -> ps ) -> -. (
ps -> ph ) ) -> ( ph <-> ps ) ) ) \$.
\end{mmraw}

\noindent This theorem relates the biconditional connective to primitive
connectives and can be used to eliminate the $\leftrightarrow$ symbol from any
wff.

\vskip 0.5ex
\setbox\startprefix=\hbox{\tt \ \ bii\ \$p\ }
\setbox\contprefix=\hbox{\tt \ \ \ \ \ \ \ \ \ }
\startm
\m{\vdash}\m{(}\m{(}\m{\varphi}\m{\leftrightarrow}\m{\psi}\m{)}\m{\leftrightarrow}
\m{\lnot}\m{(}\m{(}\m{\varphi}\m{\rightarrow}\m{\psi}\m{)}\m{\rightarrow}\m{\lnot}
\m{(}\m{\psi}\m{\rightarrow}\m{\varphi}\m{)}\m{)}\m{)}
\endm
\begin{mmraw}%
|- ( ( ph <-> ps ) <-> -. ( ( ph -> ps ) -> -. ( ps -> ph ) ) ) \$= ... \$.
\end{mmraw}

\noindent Define disjunction ({\sc or}).\index{disjunction ($\vee$)}%
\index{logical or (vee)@logical {\sc or} ($\vee$)}%
\index{df-or@\texttt{df-or}}\label{df-or}

\vskip 0.5ex
\setbox\startprefix=\hbox{\tt \ \ df-or\ \$a\ }
\setbox\contprefix=\hbox{\tt \ \ \ \ \ \ \ \ \ \ \ }
\startm
\m{\vdash}\m{(}\m{(}\m{\varphi}\m{\vee}\m{\psi}\m{)}\m{\leftrightarrow}\m{(}\m{
\lnot}\m{\varphi}\m{\rightarrow}\m{\psi}\m{)}\m{)}
\endm
\begin{mmraw}%
|- ( ( ph \TOR ps ) <-> ( -. ph -> ps ) ) \$.
\end{mmraw}

\noindent Define conjunction ({\sc and}).\index{conjunction ($\wedge$)}%
\index{logical {\sc and} ($\wedge$)}%
\index{df-an@\texttt{df-an}}\label{df-an}

\vskip 0.5ex
\setbox\startprefix=\hbox{\tt \ \ df-an\ \$a\ }
\setbox\contprefix=\hbox{\tt \ \ \ \ \ \ \ \ \ \ \ }
\startm
\m{\vdash}\m{(}\m{(}\m{\varphi}\m{\wedge}\m{\psi}\m{)}\m{\leftrightarrow}\m{\lnot}
\m{(}\m{\varphi}\m{\rightarrow}\m{\lnot}\m{\psi}\m{)}\m{)}
\endm
\begin{mmraw}%
|- ( ( ph \TAND ps ) <-> -. ( ph -> -. ps ) ) \$.
\end{mmraw}

\noindent Define disjunction ({\sc or}) of 3 wffs.%
\index{df-3or@\texttt{df-3or}}\label{df-3or}

\vskip 0.5ex
\setbox\startprefix=\hbox{\tt \ \ df-3or\ \$a\ }
\setbox\contprefix=\hbox{\tt \ \ \ \ \ \ \ \ \ \ \ \ }
\startm
\m{\vdash}\m{(}\m{(}\m{\varphi}\m{\vee}\m{\psi}\m{\vee}\m{\chi}\m{)}\m{
\leftrightarrow}\m{(}\m{(}\m{\varphi}\m{\vee}\m{\psi}\m{)}\m{\vee}\m{\chi}\m{)}
\m{)}
\endm
\begin{mmraw}%
|- ( ( ph \TOR ps \TOR ch ) <-> ( ( ph \TOR ps ) \TOR ch ) ) \$.
\end{mmraw}

\noindent Define conjunction ({\sc and}) of 3 wffs.%
\index{df-3an}\label{df-3an}

\vskip 0.5ex
\setbox\startprefix=\hbox{\tt \ \ df-3an\ \$a\ }
\setbox\contprefix=\hbox{\tt \ \ \ \ \ \ \ \ \ \ \ \ }
\startm
\m{\vdash}\m{(}\m{(}\m{\varphi}\m{\wedge}\m{\psi}\m{\wedge}\m{\chi}\m{)}\m{
\leftrightarrow}\m{(}\m{(}\m{\varphi}\m{\wedge}\m{\psi}\m{)}\m{\wedge}\m{\chi}
\m{)}\m{)}
\endm

\begin{mmraw}%
|- ( ( ph \TAND ps \TAND ch ) <-> ( ( ph \TAND ps ) \TAND ch ) ) \$.
\end{mmraw}

\subsection{Definitions for Predicate Calculus}\label{metadefpred}

The symbols $x$, $y$, and $z$ represent individual variables of predicate
calculus.  In this section, they are not necessarily distinct unless it is
explicitly
mentioned.

\vskip 2ex
\noindent Define existential quantification.
The expression $\exists x \varphi$ means
``there exists an $x$ where $\varphi$ is true.''\index{existential quantifier
($\exists$)}\label{df-ex}

\vskip 0.5ex
\setbox\startprefix=\hbox{\tt \ \ df-ex\ \$a\ }
\setbox\contprefix=\hbox{\tt \ \ \ \ \ \ \ \ \ \ \ }
\startm
\m{\vdash}\m{(}\m{\exists}\m{x}\m{\varphi}\m{\leftrightarrow}\m{\lnot}\m{\forall}
\m{x}\m{\lnot}\m{\varphi}\m{)}
\endm
\begin{mmraw}%
|- ( E. x ph <-> -. A. x -. ph ) \$.
\end{mmraw}

\noindent Define proper substitution.\index{proper
substitution}\index{substitution!proper}\label{df-sb}
In our notation, we use $[ y / x ] \varphi$ to mean ``the wff that
results when $y$ is properly substituted for $x$ in the wff
$\varphi$.''\footnote{
This can also be described
as substituting $x$ with $y$, $y$ properly replaces $x$, or
$x$ is properly replaced by $y$.}
% This is elsb4, though it currently says: ( [ x / y ] z e. y <-> z e. x )
For example,
$[ y / x ] z \in x$ is the same as $z \in y$.
One way to remember this notation is to notice that it looks like division
and recall that $( y / x ) \cdot x $ is $y$ (when $x \neq 0$).
The notation is different from the notation $\varphi ( x | y )$
that is sometimes used, because the latter notation is ambiguous for us:
for example, we don't know whether $\lnot \varphi ( x | y )$ is to be
interpreted as $\lnot ( \varphi ( x | y ) )$ or
$( \lnot \varphi ) ( x | y )$.\footnote{Because of the way
we initially defined wffs, this is the case
with any postfix connective\index{postfix connective} (one occurring after the
symbols being connected) or infix connective\index{infix connective} (one
occurring between the symbols being connected).  Metamath does not have a
built-in notion of operator binding strength that could eliminate the
ambiguity.  The initial parenthesis effectively provides a prefix
connective\index{prefix connective} to eliminate ambiguity.  Some conventions,
such as Polish notation\index{Polish notation} used in the 1930's and 1940's
by Polish logicians, use only prefix connectives and thus allow the total
elimination of parentheses, at the expense of readability.  In Metamath we
could actually redefine all notation to be Polish if we wanted to without
having to change any proofs!}  Other texts often use $\varphi(y)$ to indicate
our $[ y / x ] \varphi$, but this notation is even more ambiguous since there is
no explicit indication of what is being substituted.
Note that this
definition is valid even when
$x$ and $y$ are the same variable.  The first conjunct is a ``trick'' used to
achieve this property, making the definition look somewhat peculiar at
first.

\vskip 0.5ex
\setbox\startprefix=\hbox{\tt \ \ df-sb\ \$a\ }
\setbox\contprefix=\hbox{\tt \ \ \ \ \ \ \ \ \ \ \ }
\startm
\m{\vdash}\m{(}\m{[}\m{y}\m{/}\m{x}\m{]}\m{\varphi}\m{\leftrightarrow}\m{(}%
\m{(}\m{x}\m{=}\m{y}\m{\rightarrow}\m{\varphi}\m{)}\m{\wedge}\m{\exists}\m{x}%
\m{(}\m{x}\m{=}\m{y}\m{\wedge}\m{\varphi}\m{)}\m{)}\m{)}
\endm
\begin{mmraw}%
|- ( [ y / x ] ph <-> ( ( x = y -> ph ) \TAND E. x ( x = y \TAND ph ) ) ) \$.
\end{mmraw}


\noindent Define existential uniqueness\index{existential uniqueness
quantifier ($\exists "!$)} (``there exists exactly one'').  Note that $y$ is a
variable distinct from $x$ and not occurring in $\varphi$.\label{df-eu}

\vskip 0.5ex
\setbox\startprefix=\hbox{\tt \ \ df-eu\ \$a\ }
\setbox\contprefix=\hbox{\tt \ \ \ \ \ \ \ \ \ \ \ }
\startm
\m{\vdash}\m{(}\m{\exists}\m{{!}}\m{x}\m{\varphi}\m{\leftrightarrow}\m{\exists}
\m{y}\m{\forall}\m{x}\m{(}\m{\varphi}\m{\leftrightarrow}\m{x}\m{=}\m{y}\m{)}\m{)}
\endm

\begin{mmraw}%
|- ( E! x ph <-> E. y A. x ( ph <-> x = y ) ) \$.
\end{mmraw}

\subsection{Definitions for Set Theory}\label{setdefinitions}

The symbols $x$, $y$, $z$, and $w$ represent individual variables of
predicate calculus, which in set theory are understood to be sets.
However, using only the constructs shown so far would be very inconvenient.

To make set theory more practical, we introduce the notion of a ``class.''
A class\index{class} is either a set variable (such as $x$) or an
expression of the form $\{ x | \varphi\}$ (called an ``abstraction
class''\index{abstraction class}\index{class abstraction}).  Note that
sets (i.e.\ individual variables) always exist (this is a theorem of
logic, namely $\exists y \, y = x$ for any set $x$), whereas classes may
or may not exist (i.e.\ $\exists y \, y = A$ may or may not be true).
If a class does not exist it is called a ``proper class.''\index{proper
class}\index{class!proper} Definitions \texttt{df-clab},
\texttt{df-cleq}, and \texttt{df-clel} can be used to convert an
expression containing classes into one containing only set variables and
wff metavariables.

The symbols $A$, $B$, $C$, $D$, $ F$, $G$, and $R$ are metavariables that range
over classes.  A class metavariable $A$ may be eliminated from a wff by
replacing it with $\{ x|\varphi\}$ where neither $x$ nor $\varphi$ occur in
the wff.

The theory of classes can be shown to be an eliminable and conservative
extension of set theory. The \textbf{eliminability}
property shows that for every
formula in the extended language we can build a logically equivalent
formula in the basic language; so that even if the extended language
provides more ease to convey and formulate mathematical ideas for set
theory, its expressive power does not in fact strengthen the basic
language's expressive power.
The \textbf{conservation} property shows that for
every proof of a formula of the basic language in the extended system
we can build another proof of the same formula in the basic system;
so that, concerning theorems on sets only, the deductive powers of
the extended system and of the basic system are identical. Together,
these properties mean that the extended language can be treated as a
definitional extension that is \textbf{sound}.

A rigorous justification, which we will not give here, can be found in
Levy \cite[pp.~357-366]{Levy} supplementing his informal introduction to class
theory on pp.~7-17. Two other good treatments of class theory are provided
by Quine \cite[pp.~15-21]{Quine}\index{Quine, Willard Van Orman}
and also \cite[pp.~10-14]{Takeuti}\index{Takeuti, Gaisi}.
Quine's exposition (he calls them virtual classes)
is nicely written and very readable.

In the rest of this
section, individual variables are always assumed to be distinct from
each other unless otherwise indicated.  In addition, dummy variables on the
right-hand side of a definition do not occur in the class and wff
metavariables in the definition.

The definitions we present here are a partial but self-contained
collection selected from several hundred that appear in the current
\texttt{set.mm} database.  They are adequate for a basic development of
elementary set theory.

\vskip 2ex
\noindent Define the abstraction class.\index{abstraction class}\index{class
abstraction}\label{df-clab}  $x$ and $y$
need not be distinct.  Definition 2.1 of Quine, p.~16.  This definition may
seem puzzling since it is shorter than the expression being defined and does not
buy us anything in terms of brevity.  The reason we introduce this definition
is because it fits in neatly with the extension of the $\in$ connective
provided by \texttt{df-clel}.

\vskip 0.5ex
\setbox\startprefix=\hbox{\tt \ \ df-clab\ \$a\ }
\setbox\contprefix=\hbox{\tt \ \ \ \ \ \ \ \ \ \ \ \ \ }
\startm
\m{\vdash}\m{(}\m{x}\m{\in}\m{\{}\m{y}\m{|}\m{\varphi}\m{\}}\m{%
\leftrightarrow}\m{[}\m{x}\m{/}\m{y}\m{]}\m{\varphi}\m{)}
\endm
\begin{mmraw}%
|- ( x e. \{ y | ph \} <-> [ x / y ] ph ) \$.
\end{mmraw}

\noindent Define the equality connective between classes\index{class
equality}\label{df-cleq}.  See Quine or Chapter 4 of Takeuti and Zaring for its
justification and methods for eliminating it.  This is an example of a
somewhat ``dangerous'' definition, because it extends the use of the
existing equality symbol rather than introducing a new symbol, allowing
us to make statements in the original language that may not be true.
For example, it permits us to deduce $y = z \leftrightarrow \forall x (
x \in y \leftrightarrow x \in z )$ which is not a theorem of logic but
rather presupposes the Axiom of Extensionality,\index{Axiom of
Extensionality} which we include as a hypothesis so that we can know
when this axiom is assumed in a proof (with the \texttt{show
trace{\char`\_}back} command).  We could avoid the danger by introducing
another symbol, say $\eqcirc$, in place of $=$; this
would also have the advantage of making elimination of the definition
straightforward and would eliminate the need for Extensionality as a
hypothesis.  We would then also have the advantage of being able to
identify exactly where Extensionality truly comes into play.  One of our
theorems would be $x \eqcirc y \leftrightarrow x = y$
by invoking Extensionality.  However in keeping with standard practice
we retain the ``dangerous'' definition.

\vskip 0.5ex
\setbox\startprefix=\hbox{\tt \ \ df-cleq.1\ \$e\ }
\setbox\contprefix=\hbox{\tt \ \ \ \ \ \ \ \ \ \ \ \ \ \ \ }
\startm
\m{\vdash}\m{(}\m{\forall}\m{x}\m{(}\m{x}\m{\in}\m{y}\m{\leftrightarrow}\m{x}
\m{\in}\m{z}\m{)}\m{\rightarrow}\m{y}\m{=}\m{z}\m{)}
\endm
\setbox\startprefix=\hbox{\tt \ \ df-cleq\ \$a\ }
\setbox\contprefix=\hbox{\tt \ \ \ \ \ \ \ \ \ \ \ \ \ }
\startm
\m{\vdash}\m{(}\m{A}\m{=}\m{B}\m{\leftrightarrow}\m{\forall}\m{x}\m{(}\m{x}\m{
\in}\m{A}\m{\leftrightarrow}\m{x}\m{\in}\m{B}\m{)}\m{)}
\endm
% We need to reset the startprefix and contprefix.
\setbox\startprefix=\hbox{\tt \ \ df-cleq.1\ \$e\ }
\setbox\contprefix=\hbox{\tt \ \ \ \ \ \ \ \ \ \ \ \ \ \ \ }
\begin{mmraw}%
|- ( A. x ( x e. y <-> x e. z ) -> y = z ) \$.
\end{mmraw}
\setbox\startprefix=\hbox{\tt \ \ df-cleq\ \$a\ }
\setbox\contprefix=\hbox{\tt \ \ \ \ \ \ \ \ \ \ \ \ \ }
\begin{mmraw}%
|- ( A = B <-> A. x ( x e. A <-> x e. B ) ) \$.
\end{mmraw}

\noindent Define the membership connective between classes\index{class
membership}.  Theorem 6.3 of Quine, p.~41, which we adopt as a definition.
Note that it extends the use of the existing membership symbol, but unlike
{\tt df-cleq} it does not extend the set of valid wffs of logic when the class
metavariables are replaced with set variables.\label{dfclel}\label{df-clel}

\vskip 0.5ex
\setbox\startprefix=\hbox{\tt \ \ df-clel\ \$a\ }
\setbox\contprefix=\hbox{\tt \ \ \ \ \ \ \ \ \ \ \ \ \ }
\startm
\m{\vdash}\m{(}\m{A}\m{\in}\m{B}\m{\leftrightarrow}\m{\exists}\m{x}\m{(}\m{x}
\m{=}\m{A}\m{\wedge}\m{x}\m{\in}\m{B}\m{)}\m{)}
\endm
\begin{mmraw}%
|- ( A e. B <-> E. x ( x = A \TAND x e. B ) ) \$.?
\end{mmraw}

\noindent Define inequality.

\vskip 0.5ex
\setbox\startprefix=\hbox{\tt \ \ df-ne\ \$a\ }
\setbox\contprefix=\hbox{\tt \ \ \ \ \ \ \ \ \ \ \ }
\startm
\m{\vdash}\m{(}\m{A}\m{\ne}\m{B}\m{\leftrightarrow}\m{\lnot}\m{A}\m{=}\m{B}%
\m{)}
\endm
\begin{mmraw}%
|- ( A =/= B <-> -. A = B ) \$.
\end{mmraw}

\noindent Define restricted universal quantification.\index{universal
quantifier ($\forall$)!restricted}  Enderton, p.~22.

\vskip 0.5ex
\setbox\startprefix=\hbox{\tt \ \ df-ral\ \$a\ }
\setbox\contprefix=\hbox{\tt \ \ \ \ \ \ \ \ \ \ \ \ }
\startm
\m{\vdash}\m{(}\m{\forall}\m{x}\m{\in}\m{A}\m{\varphi}\m{\leftrightarrow}\m{%
\forall}\m{x}\m{(}\m{x}\m{\in}\m{A}\m{\rightarrow}\m{\varphi}\m{)}\m{)}
\endm
\begin{mmraw}%
|- ( A. x e. A ph <-> A. x ( x e. A -> ph ) ) \$.
\end{mmraw}

\noindent Define restricted existential quantification.\index{existential
quantifier ($\exists$)!restricted}  Enderton, p.~22.

\vskip 0.5ex
\setbox\startprefix=\hbox{\tt \ \ df-rex\ \$a\ }
\setbox\contprefix=\hbox{\tt \ \ \ \ \ \ \ \ \ \ \ \ }
\startm
\m{\vdash}\m{(}\m{\exists}\m{x}\m{\in}\m{A}\m{\varphi}\m{\leftrightarrow}\m{%
\exists}\m{x}\m{(}\m{x}\m{\in}\m{A}\m{\wedge}\m{\varphi}\m{)}\m{)}
\endm
\begin{mmraw}%
|- ( E. x e. A ph <-> E. x ( x e. A \TAND ph ) ) \$.
\end{mmraw}

\noindent Define the universal class\index{universal class ($V$)}.  Definition
5.20, p.~21, of Takeuti and Zaring.\label{df-v}

\vskip 0.5ex
\setbox\startprefix=\hbox{\tt \ \ df-v\ \$a\ }
\setbox\contprefix=\hbox{\tt \ \ \ \ \ \ \ \ \ \ }
\startm
\m{\vdash}\m{{\rm V}}\m{=}\m{\{}\m{x}\m{|}\m{x}\m{=}\m{x}\m{\}}
\endm
\begin{mmraw}%
|- {\char`\_}V = \{ x | x = x \} \$.
\end{mmraw}

\noindent Define the subclass\index{subclass}\index{subset} relationship
between two classes (called the subset relation if the classes are sets i.e.\
are not proper).  Definition 5.9 of Takeuti and Zaring, p.~17.\label{df-ss}

\vskip 0.5ex
\setbox\startprefix=\hbox{\tt \ \ df-ss\ \$a\ }
\setbox\contprefix=\hbox{\tt \ \ \ \ \ \ \ \ \ \ \ }
\startm
\m{\vdash}\m{(}\m{A}\m{\subseteq}\m{B}\m{\leftrightarrow}\m{\forall}\m{x}\m{(}
\m{x}\m{\in}\m{A}\m{\rightarrow}\m{x}\m{\in}\m{B}\m{)}\m{)}
\endm
\begin{mmraw}%
|- ( A C\_ B <-> A. x ( x e. A -> x e. B ) ) \$.
\end{mmraw}

\noindent Define the union\index{union} of two classes.  Definition 5.6 of Takeuti and Zaring,
p.~16.\label{df-un}

\vskip 0.5ex
\setbox\startprefix=\hbox{\tt \ \ df-un\ \$a\ }
\setbox\contprefix=\hbox{\tt \ \ \ \ \ \ \ \ \ \ \ }
\startm
\m{\vdash}\m{(}\m{A}\m{\cup}\m{B}\m{)}\m{=}\m{\{}\m{x}\m{|}\m{(}\m{x}\m{\in}
\m{A}\m{\vee}\m{x}\m{\in}\m{B}\m{)}\m{\}}
\endm
\begin{mmraw}%
( A u. B ) = \{ x | ( x e. A \TOR x e. B ) \} \$.
\end{mmraw}

\noindent Define the intersection\index{intersection} of two classes.  Definition 5.6 of
Takeuti and Zaring, p.~16.\label{df-in}

\vskip 0.5ex
\setbox\startprefix=\hbox{\tt \ \ df-in\ \$a\ }
\setbox\contprefix=\hbox{\tt \ \ \ \ \ \ \ \ \ \ \ }
\startm
\m{\vdash}\m{(}\m{A}\m{\cap}\m{B}\m{)}\m{=}\m{\{}\m{x}\m{|}\m{(}\m{x}\m{\in}
\m{A}\m{\wedge}\m{x}\m{\in}\m{B}\m{)}\m{\}}
\endm
% Caret ^ requires special treatment
\begin{mmraw}%
|- ( A i\^{}i B ) = \{ x | ( x e. A \TAND x e. B ) \} \$.
\end{mmraw}

\noindent Define class difference\index{class difference}\index{set difference}.
Definition 5.12 of Takeuti and Zaring, p.~20.  Several notations are used in
the literature; we chose the $\setminus$ convention instead of a minus sign to
reserve the latter for later use in, e.g., arithmetic.\label{df-dif}

\vskip 0.5ex
\setbox\startprefix=\hbox{\tt \ \ df-dif\ \$a\ }
\setbox\contprefix=\hbox{\tt \ \ \ \ \ \ \ \ \ \ \ \ }
\startm
\m{\vdash}\m{(}\m{A}\m{\setminus}\m{B}\m{)}\m{=}\m{\{}\m{x}\m{|}\m{(}\m{x}\m{
\in}\m{A}\m{\wedge}\m{\lnot}\m{x}\m{\in}\m{B}\m{)}\m{\}}
\endm
\begin{mmraw}%
( A \SLASH B ) = \{ x | ( x e. A \TAND -. x e. B ) \} \$.
\end{mmraw}

\noindent Define the empty or null set\index{empty set}\index{null set}.
Compare  Definition 5.14 of Takeuti and Zaring, p.~20.\label{df-nul}

\vskip 0.5ex
\setbox\startprefix=\hbox{\tt \ \ df-nul\ \$a\ }
\setbox\contprefix=\hbox{\tt \ \ \ \ \ \ \ \ \ \ }
\startm
\m{\vdash}\m{\varnothing}\m{=}\m{(}\m{{\rm V}}\m{\setminus}\m{{\rm V}}\m{)}
\endm
\begin{mmraw}%
|- (/) = ( {\char`\_}V \SLASH {\char`\_}V ) \$.
\end{mmraw}

\noindent Define power class\index{power set}\index{power class}.  Definition 5.10 of
Takeuti and Zaring, p.~17, but we also let it apply to proper classes.  (Note
that \verb$~P$ is the symbol for calligraphic P, the tilde
suggesting ``curly;'' see Appendix~\ref{ASCII}.)\label{df-pw}

\vskip 0.5ex
\setbox\startprefix=\hbox{\tt \ \ df-pw\ \$a\ }
\setbox\contprefix=\hbox{\tt \ \ \ \ \ \ \ \ \ \ \ }
\startm
\m{\vdash}\m{{\cal P}}\m{A}\m{=}\m{\{}\m{x}\m{|}\m{x}\m{\subseteq}\m{A}\m{\}}
\endm
% Special incantation required to put ~ into the text
\begin{mmraw}%
|- \char`\~P~A = \{ x | x C\_ A \} \$.
\end{mmraw}

\noindent Define the singleton of a class\index{singleton}.  Definition 7.1 of
Quine, p.~48.  It is well-defined for proper classes, although
it is not very meaningful in this case, where it evaluates to the empty
set.

\vskip 0.5ex
\setbox\startprefix=\hbox{\tt \ \ df-sn\ \$a\ }
\setbox\contprefix=\hbox{\tt \ \ \ \ \ \ \ \ \ \ \ }
\startm
\m{\vdash}\m{\{}\m{A}\m{\}}\m{=}\m{\{}\m{x}\m{|}\m{x}\m{=}\m{A}\m{\}}
\endm
\begin{mmraw}%
|- \{ A \} = \{ x | x = A \} \$.
\end{mmraw}%

\noindent Define an unordered pair of classes\index{unordered pair}\index{pair}.  Definition
7.1 of Quine, p.~48.

\vskip 0.5ex
\setbox\startprefix=\hbox{\tt \ \ df-pr\ \$a\ }
\setbox\contprefix=\hbox{\tt \ \ \ \ \ \ \ \ \ \ \ }
\startm
\m{\vdash}\m{\{}\m{A}\m{,}\m{B}\m{\}}\m{=}\m{(}\m{\{}\m{A}\m{\}}\m{\cup}\m{\{}
\m{B}\m{\}}\m{)}
\endm
\begin{mmraw}%
|- \{ A , B \} = ( \{ A \} u. \{ B \} ) \$.
\end{mmraw}

\noindent Define an unordered triple of classes\index{unordered triple}.  Definition of
Enderton, p.~19.

\vskip 0.5ex
\setbox\startprefix=\hbox{\tt \ \ df-tp\ \$a\ }
\setbox\contprefix=\hbox{\tt \ \ \ \ \ \ \ \ \ \ \ }
\startm
\m{\vdash}\m{\{}\m{A}\m{,}\m{B}\m{,}\m{C}\m{\}}\m{=}\m{(}\m{\{}\m{A}\m{,}\m{B}
\m{\}}\m{\cup}\m{\{}\m{C}\m{\}}\m{)}
\endm
\begin{mmraw}%
|- \{ A , B , C \} = ( \{ A , B \} u. \{ C \} ) \$.
\end{mmraw}%

\noindent Kuratowski's\index{Kuratowski, Kazimierz} ordered pair\index{ordered
pair} definition.  Definition 9.1 of Quine, p.~58. For proper classes it is
not meaningful but is well-defined for convenience.  (Note that \verb$<.$
stands for $\langle$ whereas \verb$<$ stands for $<$, and similarly for
\verb$>.$\,.)\label{df-op}

\vskip 0.5ex
\setbox\startprefix=\hbox{\tt \ \ df-op\ \$a\ }
\setbox\contprefix=\hbox{\tt \ \ \ \ \ \ \ \ \ \ \ }
\startm
\m{\vdash}\m{\langle}\m{A}\m{,}\m{B}\m{\rangle}\m{=}\m{\{}\m{\{}\m{A}\m{\}}
\m{,}\m{\{}\m{A}\m{,}\m{B}\m{\}}\m{\}}
\endm
\begin{mmraw}%
|- <. A , B >. = \{ \{ A \} , \{ A , B \} \} \$.
\end{mmraw}

\noindent Define the union of a class\index{union}.  Definition 5.5, p.~16,
of Takeuti and Zaring.

\vskip 0.5ex
\setbox\startprefix=\hbox{\tt \ \ df-uni\ \$a\ }
\setbox\contprefix=\hbox{\tt \ \ \ \ \ \ \ \ \ \ \ \ }
\startm
\m{\vdash}\m{\bigcup}\m{A}\m{=}\m{\{}\m{x}\m{|}\m{\exists}\m{y}\m{(}\m{x}\m{
\in}\m{y}\m{\wedge}\m{y}\m{\in}\m{A}\m{)}\m{\}}
\endm
\begin{mmraw}%
|- U. A = \{ x | E. y ( x e. y \TAND y e. A ) \} \$.
\end{mmraw}

\noindent Define the intersection\index{intersection} of a class.  Definition 7.35,
p.~44, of Takeuti and Zaring.

\vskip 0.5ex
\setbox\startprefix=\hbox{\tt \ \ df-int\ \$a\ }
\setbox\contprefix=\hbox{\tt \ \ \ \ \ \ \ \ \ \ \ \ }
\startm
\m{\vdash}\m{\bigcap}\m{A}\m{=}\m{\{}\m{x}\m{|}\m{\forall}\m{y}\m{(}\m{y}\m{
\in}\m{A}\m{\rightarrow}\m{x}\m{\in}\m{y}\m{)}\m{\}}
\endm
\begin{mmraw}%
|- |\^{}| A = \{ x | A. y ( y e. A -> x e. y ) \} \$.
\end{mmraw}

\noindent Define a transitive class\index{transitive class}\index{transitive
set}.  This should not be confused with a transitive relation, which is a different
concept.  Definition from p.~71 of Enderton, extended to classes.

\vskip 0.5ex
\setbox\startprefix=\hbox{\tt \ \ df-tr\ \$a\ }
\setbox\contprefix=\hbox{\tt \ \ \ \ \ \ \ \ \ \ \ }
\startm
\m{\vdash}\m{(}\m{\mbox{\rm Tr}}\m{A}\m{\leftrightarrow}\m{\bigcup}\m{A}\m{
\subseteq}\m{A}\m{)}
\endm
\begin{mmraw}%
|- ( Tr A <-> U. A C\_ A ) \$.
\end{mmraw}

\noindent Define a notation for a general binary relation\index{binary
relation}.  Definition 6.18, p.~29, of Takeuti and Zaring, generalized to
arbitrary classes.  This definition is well-defined, although not very
meaningful, when classes $A$ and/or $B$ are proper.\label{dfbr}  The lack of
parentheses (or any other connective) creates no ambiguity since we are defining
an atomic wff.

\vskip 0.5ex
\setbox\startprefix=\hbox{\tt \ \ df-br\ \$a\ }
\setbox\contprefix=\hbox{\tt \ \ \ \ \ \ \ \ \ \ \ }
\startm
\m{\vdash}\m{(}\m{A}\m{\,R}\m{\,B}\m{\leftrightarrow}\m{\langle}\m{A}\m{,}\m{B}
\m{\rangle}\m{\in}\m{R}\m{)}
\endm
\begin{mmraw}%
|- ( A R B <-> <. A , B >. e. R ) \$.
\end{mmraw}

\noindent Define an abstraction class of ordered pairs\index{abstraction
class!of ordered
pairs}.  A special case of Definition 4.16, p.~14, of Takeuti and Zaring.
Note that $ z $ must be distinct from $ x $ and $ y $,
and $ z $ must not occur in $\varphi$, but $ x $ and $ y $ may be
identical and may appear in $\varphi$.

\vskip 0.5ex
\setbox\startprefix=\hbox{\tt \ \ df-opab\ \$a\ }
\setbox\contprefix=\hbox{\tt \ \ \ \ \ \ \ \ \ \ \ \ \ }
\startm
\m{\vdash}\m{\{}\m{\langle}\m{x}\m{,}\m{y}\m{\rangle}\m{|}\m{\varphi}\m{\}}\m{=}
\m{\{}\m{z}\m{|}\m{\exists}\m{x}\m{\exists}\m{y}\m{(}\m{z}\m{=}\m{\langle}\m{x}
\m{,}\m{y}\m{\rangle}\m{\wedge}\m{\varphi}\m{)}\m{\}}
\endm

\begin{mmraw}%
|- \{ <. x , y >. | ph \} = \{ z | E. x E. y ( z =
<. x , y >. /\ ph ) \} \$.
\end{mmraw}

\noindent Define the epsilon relation\index{epsilon relation}.  Similar to Definition
6.22, p.~30, of Takeuti and Zaring.

\vskip 0.5ex
\setbox\startprefix=\hbox{\tt \ \ df-eprel\ \$a\ }
\setbox\contprefix=\hbox{\tt \ \ \ \ \ \ \ \ \ \ \ \ \ \ }
\startm
\m{\vdash}\m{{\rm E}}\m{=}\m{\{}\m{\langle}\m{x}\m{,}\m{y}\m{\rangle}\m{|}\m{x}\m{
\in}\m{y}\m{\}}
\endm
\begin{mmraw}%
|- \_E = \{ <. x , y >. | x e. y \} \$.
\end{mmraw}

\noindent Define a founded relation\index{founded relation}.  $R$ is a founded
relation on $A$ iff\index{iff} (if and only if) each nonempty subset of $A$
has an ``$R$-minimal element.''  Similar to Definition 6.21, p.~30, of
Takeuti and Zaring.

\vskip 0.5ex
\setbox\startprefix=\hbox{\tt \ \ df-fr\ \$a\ }
\setbox\contprefix=\hbox{\tt \ \ \ \ \ \ \ \ \ \ \ }
\startm
\m{\vdash}\m{(}\m{R}\m{\,\mbox{\rm Fr}}\m{\,A}\m{\leftrightarrow}\m{\forall}\m{x}
\m{(}\m{(}\m{x}\m{\subseteq}\m{A}\m{\wedge}\m{\lnot}\m{x}\m{=}\m{\varnothing}
\m{)}\m{\rightarrow}\m{\exists}\m{y}\m{(}\m{y}\m{\in}\m{x}\m{\wedge}\m{(}\m{x}
\m{\cap}\m{\{}\m{z}\m{|}\m{z}\m{\,R}\m{\,y}\m{\}}\m{)}\m{=}\m{\varnothing}\m{)}
\m{)}\m{)}
\endm
\begin{mmraw}%
|- ( R Fr A <-> A. x ( ( x C\_ A \TAND -. x = (/) ) ->
E. y ( y e. x \TAND ( x i\^{}i \{ z | z R y \} ) = (/) ) ) ) \$.
\end{mmraw}

\noindent Define a well-ordering\index{well-ordering}.  $R$ is a well-ordering of $A$ iff
it is founded on $A$ and the elements of $A$ are pairwise $R$-comparable.
Similar to Definition 6.24(2), p.~30, of Takeuti and Zaring.

\vskip 0.5ex
\setbox\startprefix=\hbox{\tt \ \ df-we\ \$a\ }
\setbox\contprefix=\hbox{\tt \ \ \ \ \ \ \ \ \ \ \ }
\startm
\m{\vdash}\m{(}\m{R}\m{\,\mbox{\rm We}}\m{\,A}\m{\leftrightarrow}\m{(}\m{R}\m{\,
\mbox{\rm Fr}}\m{\,A}\m{\wedge}\m{\forall}\m{x}\m{\forall}\m{y}\m{(}\m{(}\m{x}\m{
\in}\m{A}\m{\wedge}\m{y}\m{\in}\m{A}\m{)}\m{\rightarrow}\m{(}\m{x}\m{\,R}\m{\,y}
\m{\vee}\m{x}\m{=}\m{y}\m{\vee}\m{y}\m{\,R}\m{\,x}\m{)}\m{)}\m{)}\m{)}
\endm
\begin{mmraw}%
( R We A <-> ( R Fr A \TAND A. x A. y ( ( x e.
A \TAND y e. A ) -> ( x R y \TOR x = y \TOR y R x ) ) ) ) \$.
\end{mmraw}

\noindent Define the ordinal predicate\index{ordinal predicate}, which is true for a class
that is transitive and is well-ordered by the epsilon relation.  Similar to
definition on p.~468, Bell and Machover.

\vskip 0.5ex
\setbox\startprefix=\hbox{\tt \ \ df-ord\ \$a\ }
\setbox\contprefix=\hbox{\tt \ \ \ \ \ \ \ \ \ \ \ \ }
\startm
\m{\vdash}\m{(}\m{\mbox{\rm Ord}}\m{\,A}\m{\leftrightarrow}\m{(}
\m{\mbox{\rm Tr}}\m{\,A}\m{\wedge}\m{E}\m{\,\mbox{\rm We}}\m{\,A}\m{)}\m{)}
\endm
\begin{mmraw}%
|- ( Ord A <-> ( Tr A \TAND E We A ) ) \$.
\end{mmraw}

\noindent Define the class of all ordinal numbers\index{ordinal number}.  An ordinal number is
a set that satisfies the ordinal predicate.  Definition 7.11 of Takeuti and
Zaring, p.~38.

\vskip 0.5ex
\setbox\startprefix=\hbox{\tt \ \ df-on\ \$a\ }
\setbox\contprefix=\hbox{\tt \ \ \ \ \ \ \ \ \ \ \ }
\startm
\m{\vdash}\m{\,\mbox{\rm On}}\m{=}\m{\{}\m{x}\m{|}\m{\mbox{\rm Ord}}\m{\,x}
\m{\}}
\endm
\begin{mmraw}%
|- On = \{ x | Ord x \} \$.
\end{mmraw}

\noindent Define the limit ordinal predicate\index{limit ordinal}, which is true for a
non-empty ordinal that is not a successor (i.e.\ that is the union of itself).
Compare Bell and Machover, p.~471 and Exercise (1), p.~42 of Takeuti and
Zaring.

\vskip 0.5ex
\setbox\startprefix=\hbox{\tt \ \ df-lim\ \$a\ }
\setbox\contprefix=\hbox{\tt \ \ \ \ \ \ \ \ \ \ \ \ }
\startm
\m{\vdash}\m{(}\m{\mbox{\rm Lim}}\m{\,A}\m{\leftrightarrow}\m{(}\m{\mbox{
\rm Ord}}\m{\,A}\m{\wedge}\m{\lnot}\m{A}\m{=}\m{\varnothing}\m{\wedge}\m{A}
\m{=}\m{\bigcup}\m{A}\m{)}\m{)}
\endm
\begin{mmraw}%
|- ( Lim A <-> ( Ord A \TAND -. A = (/) \TAND A = U. A ) ) \$.
\end{mmraw}

\noindent Define the successor\index{successor} of a class.  Definition 7.22 of Takeuti
and Zaring, p.~41.  Our definition is a generalization to classes, although it
is meaningless when classes are proper.

\vskip 0.5ex
\setbox\startprefix=\hbox{\tt \ \ df-suc\ \$a\ }
\setbox\contprefix=\hbox{\tt \ \ \ \ \ \ \ \ \ \ \ \ }
\startm
\m{\vdash}\m{\,\mbox{\rm suc}}\m{\,A}\m{=}\m{(}\m{A}\m{\cup}\m{\{}\m{A}\m{\}}
\m{)}
\endm
\begin{mmraw}%
|- suc A = ( A u. \{ A \} ) \$.
\end{mmraw}

\noindent Define the class of natural numbers\index{natural number}\index{omega
($\omega$)}.  Compare Bell and Machover, p.~471.\label{dfom}

\vskip 0.5ex
\setbox\startprefix=\hbox{\tt \ \ df-om\ \$a\ }
\setbox\contprefix=\hbox{\tt \ \ \ \ \ \ \ \ \ \ \ }
\startm
\m{\vdash}\m{\omega}\m{=}\m{\{}\m{x}\m{|}\m{(}\m{\mbox{\rm Ord}}\m{\,x}\m{
\wedge}\m{\forall}\m{y}\m{(}\m{\mbox{\rm Lim}}\m{\,y}\m{\rightarrow}\m{x}\m{
\in}\m{y}\m{)}\m{)}\m{\}}
\endm
\begin{mmraw}%
|- om = \{ x | ( Ord x \TAND A. y ( Lim y -> x e. y ) ) \} \$.
\end{mmraw}

\noindent Define the Cartesian product (also called the
cross product)\index{Cartesian product}\index{cross product}
of two classes.  Definition 9.11 of Quine, p.~64.

\vskip 0.5ex
\setbox\startprefix=\hbox{\tt \ \ df-xp\ \$a\ }
\setbox\contprefix=\hbox{\tt \ \ \ \ \ \ \ \ \ \ \ }
\startm
\m{\vdash}\m{(}\m{A}\m{\times}\m{B}\m{)}\m{=}\m{\{}\m{\langle}\m{x}\m{,}\m{y}
\m{\rangle}\m{|}\m{(}\m{x}\m{\in}\m{A}\m{\wedge}\m{y}\m{\in}\m{B}\m{)}\m{\}}
\endm
\begin{mmraw}%
|- ( A X. B ) = \{ <. x , y >. | ( x e. A \TAND y e. B) \} \$.
\end{mmraw}

\noindent Define a relation\index{relation}.  Definition 6.4(1) of Takeuti and
Zaring, p.~23.

\vskip 0.5ex
\setbox\startprefix=\hbox{\tt \ \ df-rel\ \$a\ }
\setbox\contprefix=\hbox{\tt \ \ \ \ \ \ \ \ \ \ \ \ }
\startm
\m{\vdash}\m{(}\m{\mbox{\rm Rel}}\m{\,A}\m{\leftrightarrow}\m{A}\m{\subseteq}
\m{(}\m{{\rm V}}\m{\times}\m{{\rm V}}\m{)}\m{)}
\endm
\begin{mmraw}%
|- ( Rel A <-> A C\_ ( {\char`\_}V X. {\char`\_}V ) ) \$.
\end{mmraw}

\noindent Define the domain\index{domain} of a class.  Definition 6.5(1) of
Takeuti and Zaring, p.~24.

\vskip 0.5ex
\setbox\startprefix=\hbox{\tt \ \ df-dm\ \$a\ }
\setbox\contprefix=\hbox{\tt \ \ \ \ \ \ \ \ \ \ \ }
\startm
\m{\vdash}\m{\,\mbox{\rm dom}}\m{A}\m{=}\m{\{}\m{x}\m{|}\m{\exists}\m{y}\m{
\langle}\m{x}\m{,}\m{y}\m{\rangle}\m{\in}\m{A}\m{\}}
\endm
\begin{mmraw}%
|- dom A = \{ x | E. y <. x , y >. e. A \} \$.
\end{mmraw}

\noindent Define the range\index{range} of a class.  Definition 6.5(2) of
Takeuti and Zaring, p.~24.

\vskip 0.5ex
\setbox\startprefix=\hbox{\tt \ \ df-rn\ \$a\ }
\setbox\contprefix=\hbox{\tt \ \ \ \ \ \ \ \ \ \ \ }
\startm
\m{\vdash}\m{\,\mbox{\rm ran}}\m{A}\m{=}\m{\{}\m{y}\m{|}\m{\exists}\m{x}\m{
\langle}\m{x}\m{,}\m{y}\m{\rangle}\m{\in}\m{A}\m{\}}
\endm
\begin{mmraw}%
|- ran A = \{ y | E. x <. x , y >. e. A \} \$.
\end{mmraw}

\noindent Define the restriction\index{restriction} of a class.  Definition
6.6(1) of Takeuti and Zaring, p.~24.

\vskip 0.5ex
\setbox\startprefix=\hbox{\tt \ \ df-res\ \$a\ }
\setbox\contprefix=\hbox{\tt \ \ \ \ \ \ \ \ \ \ \ \ }
\startm
\m{\vdash}\m{(}\m{A}\m{\restriction}\m{B}\m{)}\m{=}\m{(}\m{A}\m{\cap}\m{(}\m{B}
\m{\times}\m{{\rm V}}\m{)}\m{)}
\endm
\begin{mmraw}%
|- ( A |` B ) = ( A i\^{}i ( B X. {\char`\_}V ) ) \$.
\end{mmraw}

\noindent Define the image\index{image} of a class.  Definition 6.6(2) of
Takeuti and Zaring, p.~24.

\vskip 0.5ex
\setbox\startprefix=\hbox{\tt \ \ df-ima\ \$a\ }
\setbox\contprefix=\hbox{\tt \ \ \ \ \ \ \ \ \ \ \ \ }
\startm
\m{\vdash}\m{(}\m{A}\m{``}\m{B}\m{)}\m{=}\m{\,\mbox{\rm ran}}\m{\,(}\m{A}\m{
\restriction}\m{B}\m{)}
\endm
\begin{mmraw}%
|- ( A " B ) = ran ( A |` B ) \$.
\end{mmraw}

\noindent Define the composition\index{composition} of two classes.  Definition 6.6(3) of
Takeuti and Zaring, p.~24.

\vskip 0.5ex
\setbox\startprefix=\hbox{\tt \ \ df-co\ \$a\ }
\setbox\contprefix=\hbox{\tt \ \ \ \ \ \ \ \ \ \ \ \ }
\startm
\m{\vdash}\m{(}\m{A}\m{\circ}\m{B}\m{)}\m{=}\m{\{}\m{\langle}\m{x}\m{,}\m{y}\m{
\rangle}\m{|}\m{\exists}\m{z}\m{(}\m{\langle}\m{x}\m{,}\m{z}\m{\rangle}\m{\in}
\m{B}\m{\wedge}\m{\langle}\m{z}\m{,}\m{y}\m{\rangle}\m{\in}\m{A}\m{)}\m{\}}
\endm
\begin{mmraw}%
|- ( A o. B ) = \{ <. x , y >. | E. z ( <. x , z
>. e. B \TAND <. z , y >. e. A ) \} \$.
\end{mmraw}

\noindent Define a function\index{function}.  Definition 6.4(4) of Takeuti and
Zaring, p.~24.

\vskip 0.5ex
\setbox\startprefix=\hbox{\tt \ \ df-fun\ \$a\ }
\setbox\contprefix=\hbox{\tt \ \ \ \ \ \ \ \ \ \ \ \ }
\startm
\m{\vdash}\m{(}\m{\mbox{\rm Fun}}\m{\,A}\m{\leftrightarrow}\m{(}
\m{\mbox{\rm Rel}}\m{\,A}\m{\wedge}
\m{\forall}\m{x}\m{\exists}\m{z}\m{\forall}\m{y}\m{(}
\m{\langle}\m{x}\m{,}\m{y}\m{\rangle}\m{\in}\m{A}\m{\rightarrow}\m{y}\m{=}\m{z}
\m{)}\m{)}\m{)}
\endm
\begin{mmraw}%
|- ( Fun A <-> ( Rel A /\ A. x E. z A. y ( <. x
   , y >. e. A -> y = z ) ) ) \$.
\end{mmraw}

\noindent Define a function with domain.  Definition 6.15(1) of Takeuti and
Zaring, p.~27.

\vskip 0.5ex
\setbox\startprefix=\hbox{\tt \ \ df-fn\ \$a\ }
\setbox\contprefix=\hbox{\tt \ \ \ \ \ \ \ \ \ \ \ }
\startm
\m{\vdash}\m{(}\m{A}\m{\,\mbox{\rm Fn}}\m{\,B}\m{\leftrightarrow}\m{(}
\m{\mbox{\rm Fun}}\m{\,A}\m{\wedge}\m{\mbox{\rm dom}}\m{\,A}\m{=}\m{B}\m{)}
\m{)}
\endm
\begin{mmraw}%
|- ( A Fn B <-> ( Fun A \TAND dom A = B ) ) \$.
\end{mmraw}

\noindent Define a function with domain and co-domain.  Definition 6.15(3)
of Takeuti and Zaring, p.~27.

\vskip 0.5ex
\setbox\startprefix=\hbox{\tt \ \ df-f\ \$a\ }
\setbox\contprefix=\hbox{\tt \ \ \ \ \ \ \ \ \ \ }
\startm
\m{\vdash}\m{(}\m{F}\m{:}\m{A}\m{\longrightarrow}\m{B}\m{
\leftrightarrow}\m{(}\m{F}\m{\,\mbox{\rm Fn}}\m{\,A}\m{\wedge}\m{
\mbox{\rm ran}}\m{\,F}\m{\subseteq}\m{B}\m{)}\m{)}
\endm
\begin{mmraw}%
|- ( F : A --> B <-> ( F Fn A \TAND ran F C\_ B ) ) \$.
\end{mmraw}

\noindent Define a one-to-one function\index{one-to-one function}.  Compare
Definition 6.15(5) of Takeuti and Zaring, p.~27.

\vskip 0.5ex
\setbox\startprefix=\hbox{\tt \ \ df-f1\ \$a\ }
\setbox\contprefix=\hbox{\tt \ \ \ \ \ \ \ \ \ \ \ }
\startm
\m{\vdash}\m{(}\m{F}\m{:}\m{A}\m{
\raisebox{.5ex}{${\textstyle{\:}_{\mbox{\footnotesize\rm
1\tt -\rm 1}}}\atop{\textstyle{
\longrightarrow}\atop{\textstyle{}^{\mbox{\footnotesize\rm {\ }}}}}$}
}\m{B}
\m{\leftrightarrow}\m{(}\m{F}\m{:}\m{A}\m{\longrightarrow}\m{B}
\m{\wedge}\m{\forall}\m{y}\m{\exists}\m{z}\m{\forall}\m{x}\m{(}\m{\langle}\m{x}
\m{,}\m{y}\m{\rangle}\m{\in}\m{F}\m{\rightarrow}\m{x}\m{=}\m{z}\m{)}\m{)}\m{)}
\endm
\begin{mmraw}%
|- ( F : A -1-1-> B <-> ( F : A --> B \TAND
   A. y E. z A. x ( <. x , y >. e. F -> x = z ) ) ) \$.
\end{mmraw}

\noindent Define an onto function\index{onto function}.  Definition 6.15(4) of Takeuti and
Zaring, p.~27.

\vskip 0.5ex
\setbox\startprefix=\hbox{\tt \ \ df-fo\ \$a\ }
\setbox\contprefix=\hbox{\tt \ \ \ \ \ \ \ \ \ \ \ }
\startm
\m{\vdash}\m{(}\m{F}\m{:}\m{A}\m{
\raisebox{.5ex}{${\textstyle{\:}_{\mbox{\footnotesize\rm
{\ }}}}\atop{\textstyle{
\longrightarrow}\atop{\textstyle{}^{\mbox{\footnotesize\rm onto}}}}$}
}\m{B}
\m{\leftrightarrow}\m{(}\m{F}\m{\,\mbox{\rm Fn}}\m{\,A}\m{\wedge}
\m{\mbox{\rm ran}}\m{\,F}\m{=}\m{B}\m{)}\m{)}
\endm
\begin{mmraw}%
|- ( F : A -onto-> B <-> ( F Fn A /\ ran F = B ) ) \$.
\end{mmraw}

\noindent Define a one-to-one, onto function.  Compare Definition 6.15(6) of
Takeuti and Zaring, p.~27.

\vskip 0.5ex
\setbox\startprefix=\hbox{\tt \ \ df-f1o\ \$a\ }
\setbox\contprefix=\hbox{\tt \ \ \ \ \ \ \ \ \ \ \ \ }
\startm
\m{\vdash}\m{(}\m{F}\m{:}\m{A}
\m{
\raisebox{.5ex}{${\textstyle{\:}_{\mbox{\footnotesize\rm
1\tt -\rm 1}}}\atop{\textstyle{
\longrightarrow}\atop{\textstyle{}^{\mbox{\footnotesize\rm onto}}}}$}
}
\m{B}
\m{\leftrightarrow}\m{(}\m{F}\m{:}\m{A}
\m{
\raisebox{.5ex}{${\textstyle{\:}_{\mbox{\footnotesize\rm
1\tt -\rm 1}}}\atop{\textstyle{
\longrightarrow}\atop{\textstyle{}^{\mbox{\footnotesize\rm {\ }}}}}$}
}
\m{B}\m{\wedge}\m{F}\m{:}\m{A}
\m{
\raisebox{.5ex}{${\textstyle{\:}_{\mbox{\footnotesize\rm
{\ }}}}\atop{\textstyle{
\longrightarrow}\atop{\textstyle{}^{\mbox{\footnotesize\rm onto}}}}$}
}
\m{B}\m{)}\m{)}
\endm
\begin{mmraw}%
|- ( F : A -1-1-onto-> B <-> ( F : A -1-1-> B? \TAND F : A -onto-> B ) ) \$.?
\end{mmraw}

\noindent Define the value of a function\index{function value}.  This
definition applies to any class and evaluates to the empty set when it is not
meaningful. Note that $ F`A$ means the same thing as the more familiar $ F(A)$
notation for a function's value at $A$.  The $ F`A$ notation is common in
formal set theory.\label{df-fv}

\vskip 0.5ex
\setbox\startprefix=\hbox{\tt \ \ df-fv\ \$a\ }
\setbox\contprefix=\hbox{\tt \ \ \ \ \ \ \ \ \ \ \ }
\startm
\m{\vdash}\m{(}\m{F}\m{`}\m{A}\m{)}\m{=}\m{\bigcup}\m{\{}\m{x}\m{|}\m{(}\m{F}%
\m{``}\m{\{}\m{A}\m{\}}\m{)}\m{=}\m{\{}\m{x}\m{\}}\m{\}}
\endm
\begin{mmraw}%
|- ( F ` A ) = U. \{ x | ( F " \{ A \} ) = \{ x \} \} \$.
\end{mmraw}

\noindent Define the result of an operation.\index{operation}  Here, $F$ is
     an operation on two
     values (such as $+$ for real numbers).   This is defined for proper
     classes $A$ and $B$ even though not meaningful in that case.  However,
     the definition can be meaningful when $F$ is a proper class.\label{dfopr}

\vskip 0.5ex
\setbox\startprefix=\hbox{\tt \ \ df-opr\ \$a\ }
\setbox\contprefix=\hbox{\tt \ \ \ \ \ \ \ \ \ \ \ \ }
\startm
\m{\vdash}\m{(}\m{A}\m{\,F}\m{\,B}\m{)}\m{=}\m{(}\m{F}\m{`}\m{\langle}\m{A}%
\m{,}\m{B}\m{\rangle}\m{)}
\endm
\begin{mmraw}%
|- ( A F B ) = ( F ` <. A , B >. ) \$.
\end{mmraw}

\section{Tricks of the Trade}\label{tricks}

In the \texttt{set.mm}\index{set theory database (\texttt{set.mm})} database our goal
was usually to conform to modern notation.  However in some cases the
relationship to standard textbook language may be obscured by several
unconventional devices we used to simplify the development and to take
advantage of the Metamath language.  In this section we will describe some
common conventions used in \texttt{set.mm}.

\begin{itemize}
\item
The turnstile\index{turnstile ({$\,\vdash$})} symbol, $\vdash$, meaning ``it
is provable that,'' is the first token of all assertions and hypotheses that
aren't syntax constructions.  This is a standard convention in logic.  (We
mentioned this earlier, but this symbol is bothersome to some people without a
logic background.  It has no deeper meaning but just provides us with a way to
distinguish syntax constructions from ordinary mathematical statements.)

\item
A hypothesis of the form

\vskip 1ex
\setbox\startprefix=\hbox{\tt \ \ \ \ \ \ \ \ \ \$e\ }
\setbox\contprefix=\hbox{\tt \ \ \ \ \ \ \ \ \ \ }
\startm
\m{\vdash}\m{(}\m{\varphi}\m{\rightarrow}\m{\forall}\m{x}\m{\varphi}\m{)}
\endm
\vskip 1ex

should be read ``assume variable $x$ is (effectively) not free in wff
$\varphi$.''\index{effectively not free}
Literally, this says ``assume it is provable that $\varphi \rightarrow \forall
x\, \varphi$.''  This device lets us avoid the complexities associated with
the standard treatment of free and bound variables.
%% Uncomment this when uncommenting section {formalspec} below
The footnote on p.~\pageref{effectivelybound} discusses this further.

\item
A statement of one of the forms

\vskip 1ex
\setbox\startprefix=\hbox{\tt \ \ \ \ \ \ \ \ \ \$a\ }
\setbox\contprefix=\hbox{\tt \ \ \ \ \ \ \ \ \ \ }
\startm
\m{\vdash}\m{(}\m{\lnot}\m{\forall}\m{x}\m{\,x}\m{=}\m{y}\m{\rightarrow}
\m{\ldots}\m{)}
\endm
\setbox\startprefix=\hbox{\tt \ \ \ \ \ \ \ \ \ \$p\ }
\setbox\contprefix=\hbox{\tt \ \ \ \ \ \ \ \ \ \ }
\startm
\m{\vdash}\m{(}\m{\lnot}\m{\forall}\m{x}\m{\,x}\m{=}\m{y}\m{\rightarrow}
\m{\ldots}\m{)}
\endm
\vskip 1ex

should be read ``if $x$ and $y$ are distinct variables, then...''  This
antecedent provides us with a technical device to avoid the need for the
\texttt{\$d} statement early in our development of predicate calculus,
permitting symbol manipulations to be as conceptually simple as those in
propositional calculus.  However, the \texttt{\$d} statement eventually
becomes a requirement, and after that this device is rarely used.

\item
The statement

\vskip 1ex
\setbox\startprefix=\hbox{\tt \ \ \ \ \ \ \ \ \ \$d\ }
\setbox\contprefix=\hbox{\tt \ \ \ \ \ \ \ \ \ \ }
\startm
\m{x}\m{\,y}
\endm
\vskip 1ex

should be read ``assume $x$ and $y$ are distinct variables.''

\item
The statement

\vskip 1ex
\setbox\startprefix=\hbox{\tt \ \ \ \ \ \ \ \ \ \$d\ }
\setbox\contprefix=\hbox{\tt \ \ \ \ \ \ \ \ \ \ }
\startm
\m{x}\m{\,\varphi}
\endm
\vskip 1ex

should be read ``assume $x$ does not occur in $\varphi$.''

\item
The statement

\vskip 1ex
\setbox\startprefix=\hbox{\tt \ \ \ \ \ \ \ \ \ \$d\ }
\setbox\contprefix=\hbox{\tt \ \ \ \ \ \ \ \ \ \ }
\startm
\m{x}\m{\,A}
\endm
\vskip 1ex

should be read ``assume variable $x$ does not occur in class $A$.''

\item
The restriction and hypothesis group

\vskip 1ex
\setbox\startprefix=\hbox{\tt \ \ \ \ \ \ \ \ \ \$d\ }
\setbox\contprefix=\hbox{\tt \ \ \ \ \ \ \ \ \ \ }
\startm
\m{x}\m{\,A}
\endm
\setbox\startprefix=\hbox{\tt \ \ \ \ \ \ \ \ \ \$d\ }
\setbox\contprefix=\hbox{\tt \ \ \ \ \ \ \ \ \ \ }
\startm
\m{x}\m{\,\psi}
\endm
\setbox\startprefix=\hbox{\tt \ \ \ \ \ \ \ \ \ \$e\ }
\setbox\contprefix=\hbox{\tt \ \ \ \ \ \ \ \ \ \ }
\startm
\m{\vdash}\m{(}\m{x}\m{=}\m{A}\m{\rightarrow}\m{(}\m{\varphi}\m{\leftrightarrow}
\m{\psi}\m{)}\m{)}
\endm
\vskip 1ex

is frequently used in place of explicit substitution, meaning ``assume
$\psi$ results from the proper substitution of $A$ for $x$ in
$\varphi$.''  Sometimes ``\texttt{\$e} $\vdash ( \psi \rightarrow
\forall x \, \psi )$'' is used instead of ``\texttt{\$d} $x\, \psi $,''
which requires only that $x$ be effectively not free in $\varphi$ but
not necessarily absent from it.  The use of implicit
substitution\index{substitution!implicit} is partly a matter of personal
style, although it may make proofs somewhat shorter than would be the
case with explicit substitution.

\item
The hypothesis


\vskip 1ex
\setbox\startprefix=\hbox{\tt \ \ \ \ \ \ \ \ \ \$e\ }
\setbox\contprefix=\hbox{\tt \ \ \ \ \ \ \ \ \ \ }
\startm
\m{\vdash}\m{A}\m{\in}\m{{\rm V}}
\endm
\vskip 1ex

should be read ``assume class $A$ is a set (i.e.\ exists).''
This is a convenient convention used by Quine.

\item
The restriction and hypothesis

\vskip 1ex
\setbox\startprefix=\hbox{\tt \ \ \ \ \ \ \ \ \ \$d\ }
\setbox\contprefix=\hbox{\tt \ \ \ \ \ \ \ \ \ \ }
\startm
\m{x}\m{\,y}
\endm
\setbox\startprefix=\hbox{\tt \ \ \ \ \ \ \ \ \ \$e\ }
\setbox\contprefix=\hbox{\tt \ \ \ \ \ \ \ \ \ \ }
\startm
\m{\vdash}\m{(}\m{y}\m{\in}\m{A}\m{\rightarrow}\m{\forall}\m{x}\m{\,y}
\m{\in}\m{A}\m{)}
\endm
\vskip 1ex

should be read ``assume variable $x$ is
(effectively) not free in class $A$.''

\end{itemize}

\section{A Theorem Sampler}\label{sometheorems}

In this section we list some of the more important theorems that are proved in
the \texttt{set.mm} database, and they illustrate the kinds of things that can be
done with Metamath.  While all of these facts are well-known results,
Metamath offers the advantage of easily allowing you to trace their
derivation back to axioms.  Our intent here is not to try to explain the
details or motivation; for this we refer you to the textbooks that are
mentioned in the descriptions.  (The \texttt{set.mm} file has bibliographic
references for the text references.)  Their proofs often embody important
concepts you may wish to explore with the Metamath program (see
Section~\ref{exploring}).  All the symbols that are used here are defined in
Section~\ref{hierarchy}.  For brevity we haven't included the \texttt{\$d}
restrictions or \texttt{\$f} hypotheses for these theorems; when you are
uncertain consult the \texttt{set.mm} database.

We start with \texttt{syl} (principle of the syllogism).
In \textit{Principia Mathematica}
Whitehead and Russell call this ``the principle of the
syllogism... because... the syllogism in Barbara is derived from them''
\cite[quote after Theorem *2.06 p.~101]{PM}.
Some authors call this law a ``hypothetical syllogism.''
At the time of this writing it is the most commonly referenced assertion in the
\texttt{set.mm} database.\footnote{
The Metamath program command \texttt{show usage}
shows the number of references.
}

\vskip 2ex
\noindent Theorem syl (principle of the syllogism)\index{Syllogism}%
\index{\texttt{syl}}\label{syl}.

\vskip 0.5ex
\setbox\startprefix=\hbox{\tt \ \ syl.1\ \$e\ }
\setbox\contprefix=\hbox{\tt \ \ \ \ \ \ \ \ \ \ \ }
\startm
\m{\vdash}\m{(}\m{\varphi}\m{ \rightarrow }\m{\psi}\m{)}
\endm
\setbox\startprefix=\hbox{\tt \ \ syl.2\ \$e\ }
\setbox\contprefix=\hbox{\tt \ \ \ \ \ \ \ \ \ \ \ }
\startm
\m{\vdash}\m{(}\m{\psi}\m{ \rightarrow }\m{\chi}\m{)}
\endm
\setbox\startprefix=\hbox{\tt \ \ syl\ \$p\ }
\setbox\contprefix=\hbox{\tt \ \ \ \ \ \ \ \ \ }
\startm
\m{\vdash}\m{(}\m{\varphi}\m{ \rightarrow }\m{\chi}\m{)}
\endm
\vskip 2ex

The following theorem is not very deep but provides us with a notational device
that is frequently used.  It allows us to use the expression ``$A \in V$'' as
a compact way of saying that class $A$ exists, i.e.\ is a set.

\vskip 2ex
\noindent Two ways to say ``$A$ is a set'':  $A$ is a member of the universe
$V$ if and only if $A$ exists (i.e.\ there exists a set equal to $A$).
Theorem 6.9 of Quine, p. 43.

\vskip 0.5ex
\setbox\startprefix=\hbox{\tt \ \ isset\ \$p\ }
\setbox\contprefix=\hbox{\tt \ \ \ \ \ \ \ \ \ \ \ }
\startm
\m{\vdash}\m{(}\m{A}\m{\in}\m{{\rm V}}\m{\leftrightarrow}\m{\exists}\m{x}\m{\,x}\m{=}
\m{A}\m{)}
\endm
\vskip 1ex

Next we prove the axioms of standard ZF set theory that were missing from our
axiom system.  From our point of view they are theorems since they
can be derived from the other axioms.

\vskip 2ex
\noindent Axiom of Separation\index{Axiom of Separation}
(Aussonderung)\index{Aussonderung} proved from the other axioms of ZF set
theory.  Compare Exercise 4 of Takeuti and Zaring, p.~22.

\vskip 0.5ex
\setbox\startprefix=\hbox{\tt \ \ inex1.1\ \$e\ }
\setbox\contprefix=\hbox{\tt \ \ \ \ \ \ \ \ \ \ \ \ \ \ \ }
\startm
\m{\vdash}\m{A}\m{\in}\m{{\rm V}}
\endm
\setbox\startprefix=\hbox{\tt \ \ inex\ \$p\ }
\setbox\contprefix=\hbox{\tt \ \ \ \ \ \ \ \ \ \ \ \ \ }
\startm
\m{\vdash}\m{(}\m{A}\m{\cap}\m{B}\m{)}\m{\in}\m{{\rm V}}
\endm
\vskip 1ex

\noindent Axiom of the Null Set\index{Axiom of the Null Set} proved from the
other axioms of ZF set theory. Corollary 5.16 of Takeuti and Zaring, p.~20.

\vskip 0.5ex
\setbox\startprefix=\hbox{\tt \ \ 0ex\ \$p\ }
\setbox\contprefix=\hbox{\tt \ \ \ \ \ \ \ \ \ \ \ \ }
\startm
\m{\vdash}\m{\varnothing}\m{\in}\m{{\rm V}}
\endm
\vskip 1ex

\noindent The Axiom of Pairing\index{Axiom of Pairing} proved from the other
axioms of ZF set theory.  Theorem 7.13 of Quine, p.~51.
\vskip 0.5ex
\setbox\startprefix=\hbox{\tt \ \ prex\ \$p\ }
\setbox\contprefix=\hbox{\tt \ \ \ \ \ \ \ \ \ \ \ \ \ \ }
\startm
\m{\vdash}\m{\{}\m{A}\m{,}\m{B}\m{\}}\m{\in}\m{{\rm V}}
\endm
\vskip 2ex

Next we will list some famous or important theorems that are proved in
the \texttt{set.mm} database.  None of them except \texttt{omex}
require the Axiom of Infinity, as you can verify with the \texttt{show
trace{\char`\_}back} Metamath command.

\vskip 2ex
\noindent The resolution of Russell's paradox\index{Russell's paradox}.  There
exists no set containing the set of all sets which are not members of
themselves.  Proposition 4.14 of Takeuti and Zaring, p.~14.

\vskip 0.5ex
\setbox\startprefix=\hbox{\tt \ \ ru\ \$p\ }
\setbox\contprefix=\hbox{\tt \ \ \ \ \ \ \ \ }
\startm
\m{\vdash}\m{\lnot}\m{\exists}\m{x}\m{\,x}\m{=}\m{\{}\m{y}\m{|}\m{\lnot}\m{y}
\m{\in}\m{y}\m{\}}
\endm
\vskip 1ex

\noindent Cantor's theorem\index{Cantor's theorem}.  No set can be mapped onto
its power set.  Compare Theorem 6B(b) of Enderton, p.~132.

\vskip 0.5ex
\setbox\startprefix=\hbox{\tt \ \ canth.1\ \$e\ }
\setbox\contprefix=\hbox{\tt \ \ \ \ \ \ \ \ \ \ \ \ \ }
\startm
\m{\vdash}\m{A}\m{\in}\m{{\rm V}}
\endm
\setbox\startprefix=\hbox{\tt \ \ canth\ \$p\ }
\setbox\contprefix=\hbox{\tt \ \ \ \ \ \ \ \ \ \ \ }
\startm
\m{\vdash}\m{\lnot}\m{F}\m{:}\m{A}\m{\raisebox{.5ex}{${\textstyle{\:}_{
\mbox{\footnotesize\rm {\ }}}}\atop{\textstyle{\longrightarrow}\atop{
\textstyle{}^{\mbox{\footnotesize\rm onto}}}}$}}\m{{\cal P}}\m{A}
\endm
\vskip 1ex

\noindent The Burali-Forti paradox\index{Burali-Forti paradox}.  No set
contains all ordinal numbers. Enderton, p.~194.  (Burali-Forti was one person,
not two.)

\vskip 0.5ex
\setbox\startprefix=\hbox{\tt \ \ onprc\ \$p\ }
\setbox\contprefix=\hbox{\tt \ \ \ \ \ \ \ \ \ \ \ \ }
\startm
\m{\vdash}\m{\lnot}\m{\mbox{\rm On}}\m{\in}\m{{\rm V}}
\endm
\vskip 1ex

\noindent Peano's postulates\index{Peano's postulates} for arithmetic.
Proposition 7.30 of Takeuti and Zaring, pp.~42--43.  The objects being
described are the members of $\omega$ i.e.\ the natural numbers 0, 1,
2,\ldots.  The successor\index{successor} operation suc means ``plus
one.''  \texttt{peano1} says that 0 (which is defined as the empty set)
is a natural number.  \texttt{peano2} says that if $A$ is a natural
number, so is $A+1$.  \texttt{peano3} says that 0 is not the successor
of any natural number.  \texttt{peano4} says that two natural numbers
are equal if and only if their successors are equal.  \texttt{peano5} is
essentially the same as mathematical induction.

\vskip 1ex
\setbox\startprefix=\hbox{\tt \ \ peano1\ \$p\ }
\setbox\contprefix=\hbox{\tt \ \ \ \ \ \ \ \ \ \ \ \ }
\startm
\m{\vdash}\m{\varnothing}\m{\in}\m{\omega}
\endm
\vskip 1.5ex

\setbox\startprefix=\hbox{\tt \ \ peano2\ \$p\ }
\setbox\contprefix=\hbox{\tt \ \ \ \ \ \ \ \ \ \ \ \ }
\startm
\m{\vdash}\m{(}\m{A}\m{\in}\m{\omega}\m{\rightarrow}\m{{\rm suc}}\m{A}\m{\in}%
\m{\omega}\m{)}
\endm
\vskip 1.5ex

\setbox\startprefix=\hbox{\tt \ \ peano3\ \$p\ }
\setbox\contprefix=\hbox{\tt \ \ \ \ \ \ \ \ \ \ \ \ }
\startm
\m{\vdash}\m{(}\m{A}\m{\in}\m{\omega}\m{\rightarrow}\m{\lnot}\m{{\rm suc}}%
\m{A}\m{=}\m{\varnothing}\m{)}
\endm
\vskip 1.5ex

\setbox\startprefix=\hbox{\tt \ \ peano4\ \$p\ }
\setbox\contprefix=\hbox{\tt \ \ \ \ \ \ \ \ \ \ \ \ }
\startm
\m{\vdash}\m{(}\m{(}\m{A}\m{\in}\m{\omega}\m{\wedge}\m{B}\m{\in}\m{\omega}%
\m{)}\m{\rightarrow}\m{(}\m{{\rm suc}}\m{A}\m{=}\m{{\rm suc}}\m{B}\m{%
\leftrightarrow}\m{A}\m{=}\m{B}\m{)}\m{)}
\endm
\vskip 1.5ex

\setbox\startprefix=\hbox{\tt \ \ peano5\ \$p\ }
\setbox\contprefix=\hbox{\tt \ \ \ \ \ \ \ \ \ \ \ \ }
\startm
\m{\vdash}\m{(}\m{(}\m{\varnothing}\m{\in}\m{A}\m{\wedge}\m{\forall}\m{x}\m{%
\in}\m{\omega}\m{(}\m{x}\m{\in}\m{A}\m{\rightarrow}\m{{\rm suc}}\m{x}\m{\in}%
\m{A}\m{)}\m{)}\m{\rightarrow}\m{\omega}\m{\subseteq}\m{A}\m{)}
\endm
\vskip 1.5ex

\noindent Finite Induction (mathematical induction).\index{finite
induction}\index{mathematical induction} The first hypothesis is the
basis and the second is the induction hypothesis.  Theorem Schema 22 of
Suppes, p.~136.

\vskip 0.5ex
\setbox\startprefix=\hbox{\tt \ \ findes.1\ \$e\ }
\setbox\contprefix=\hbox{\tt \ \ \ \ \ \ \ \ \ \ \ \ \ \ }
\startm
\m{\vdash}\m{[}\m{\varnothing}\m{/}\m{x}\m{]}\m{\varphi}
\endm
\setbox\startprefix=\hbox{\tt \ \ findes.2\ \$e\ }
\setbox\contprefix=\hbox{\tt \ \ \ \ \ \ \ \ \ \ \ \ \ \ }
\startm
\m{\vdash}\m{(}\m{x}\m{\in}\m{\omega}\m{\rightarrow}\m{(}\m{\varphi}\m{%
\rightarrow}\m{[}\m{{\rm suc}}\m{x}\m{/}\m{x}\m{]}\m{\varphi}\m{)}\m{)}
\endm
\setbox\startprefix=\hbox{\tt \ \ findes\ \$p\ }
\setbox\contprefix=\hbox{\tt \ \ \ \ \ \ \ \ \ \ \ \ }
\startm
\m{\vdash}\m{(}\m{x}\m{\in}\m{\omega}\m{\rightarrow}\m{\varphi}\m{)}
\endm
\vskip 1ex

\noindent Transfinite Induction with explicit substitution.  The first
hypothesis is the basis, the second is the induction hypothesis for
successors, and the third is the induction hypothesis for limit
ordinals.  Theorem Schema 4 of Suppes, p. 197.

\vskip 0.5ex
\setbox\startprefix=\hbox{\tt \ \ tfindes.1\ \$e\ }
\setbox\contprefix=\hbox{\tt \ \ \ \ \ \ \ \ \ \ \ \ \ \ \ }
\startm
\m{\vdash}\m{[}\m{\varnothing}\m{/}\m{x}\m{]}\m{\varphi}
\endm
\setbox\startprefix=\hbox{\tt \ \ tfindes.2\ \$e\ }
\setbox\contprefix=\hbox{\tt \ \ \ \ \ \ \ \ \ \ \ \ \ \ \ }
\startm
\m{\vdash}\m{(}\m{x}\m{\in}\m{{\rm On}}\m{\rightarrow}\m{(}\m{\varphi}\m{%
\rightarrow}\m{[}\m{{\rm suc}}\m{x}\m{/}\m{x}\m{]}\m{\varphi}\m{)}\m{)}
\endm
\setbox\startprefix=\hbox{\tt \ \ tfindes.3\ \$e\ }
\setbox\contprefix=\hbox{\tt \ \ \ \ \ \ \ \ \ \ \ \ \ \ \ }
\startm
\m{\vdash}\m{(}\m{{\rm Lim}}\m{y}\m{\rightarrow}\m{(}\m{\forall}\m{x}\m{\in}%
\m{y}\m{\varphi}\m{\rightarrow}\m{[}\m{y}\m{/}\m{x}\m{]}\m{\varphi}\m{)}\m{)}
\endm
\setbox\startprefix=\hbox{\tt \ \ tfindes\ \$p\ }
\setbox\contprefix=\hbox{\tt \ \ \ \ \ \ \ \ \ \ \ \ \ }
\startm
\m{\vdash}\m{(}\m{x}\m{\in}\m{{\rm On}}\m{\rightarrow}\m{\varphi}\m{)}
\endm
\vskip 1ex

\noindent Principle of Transfinite Recursion.\index{transfinite
recursion} Theorem 7.41 of Takeuti and Zaring, p.~47.  Transfinite
recursion is the key theorem that allows arithmetic of ordinals to be
rigorously defined, and has many other important uses as well.
Hypotheses \texttt{tfr.1} and \texttt{tfr.2} specify a certain (proper)
class $ F$.  The complicated definition of $ F$ is not important in
itself; what is important is that there be such an $ F$ with the
required properties, and we show this by displaying $ F$ explicitly.
\texttt{tfr1} states that $ F$ is a function whose domain is the set of
ordinal numbers.  \texttt{tfr2} states that any value of $ F$ is
completely determined by its previous values and the values of an
auxiliary function, $G$.  \texttt{tfr3} states that $ F$ is unique,
i.e.\ it is the only function that satisfies \texttt{tfr1} and
\texttt{tfr2}.  Note that $ f$ is an individual variable like $x$ and
$y$; it is just a mnemonic to remind us that $A$ is a collection of
functions.

\vskip 0.5ex
\setbox\startprefix=\hbox{\tt \ \ tfr.1\ \$e\ }
\setbox\contprefix=\hbox{\tt \ \ \ \ \ \ \ \ \ \ \ }
\startm
\m{\vdash}\m{A}\m{=}\m{\{}\m{f}\m{|}\m{\exists}\m{x}\m{\in}\m{{\rm On}}\m{(}%
\m{f}\m{{\rm Fn}}\m{x}\m{\wedge}\m{\forall}\m{y}\m{\in}\m{x}\m{(}\m{f}\m{`}%
\m{y}\m{)}\m{=}\m{(}\m{G}\m{`}\m{(}\m{f}\m{\restriction}\m{y}\m{)}\m{)}\m{)}%
\m{\}}
\endm
\setbox\startprefix=\hbox{\tt \ \ tfr.2\ \$e\ }
\setbox\contprefix=\hbox{\tt \ \ \ \ \ \ \ \ \ \ \ }
\startm
\m{\vdash}\m{F}\m{=}\m{\bigcup}\m{A}
\endm
\setbox\startprefix=\hbox{\tt \ \ tfr1\ \$p\ }
\setbox\contprefix=\hbox{\tt \ \ \ \ \ \ \ \ \ \ }
\startm
\m{\vdash}\m{F}\m{{\rm Fn}}\m{{\rm On}}
\endm
\setbox\startprefix=\hbox{\tt \ \ tfr2\ \$p\ }
\setbox\contprefix=\hbox{\tt \ \ \ \ \ \ \ \ \ \ }
\startm
\m{\vdash}\m{(}\m{z}\m{\in}\m{{\rm On}}\m{\rightarrow}\m{(}\m{F}\m{`}\m{z}%
\m{)}\m{=}\m{(}\m{G}\m{`}\m{(}\m{F}\m{\restriction}\m{z}\m{)}\m{)}\m{)}
\endm
\setbox\startprefix=\hbox{\tt \ \ tfr3\ \$p\ }
\setbox\contprefix=\hbox{\tt \ \ \ \ \ \ \ \ \ \ }
\startm
\m{\vdash}\m{(}\m{(}\m{B}\m{{\rm Fn}}\m{{\rm On}}\m{\wedge}\m{\forall}\m{x}\m{%
\in}\m{{\rm On}}\m{(}\m{B}\m{`}\m{x}\m{)}\m{=}\m{(}\m{G}\m{`}\m{(}\m{B}\m{%
\restriction}\m{x}\m{)}\m{)}\m{)}\m{\rightarrow}\m{B}\m{=}\m{F}\m{)}
\endm
\vskip 1ex

\noindent The existence of omega (the class of natural numbers).\index{natural
number}\index{omega ($\omega$)}\index{Axiom of Infinity}  Axiom 7 of Takeuti
and Zaring, p.~43.  (This is the only theorem in this section requiring the
Axiom of Infinity.)

\vskip 0.5ex
\setbox\startprefix=\hbox{\tt \
\ omex\ \$p\ }
\setbox\contprefix=\hbox{\tt \ \ \ \ \ \ \ \ \ \ }
\startm
\m{\vdash}\m{\omega}\m{\in}\m{{\rm V}}
\endm
%\vskip 2ex


\section{Axioms for Real and Complex Numbers}\label{real}
\index{real and complex numbers!axioms for}

This section presents the axioms for real and complex numbers.  Analysis
textbooks implicitly or explicitly use these axioms or their equivalents
as their starting point.  In the database \texttt{set.mm}, we define real
and complex numbers as (rather complicated) specific sets and derive these
axioms as {\em theorems} from the axioms of ZF set theory, using a method
called Dedekind cuts.  We omit the details of this construction, which you can
follow if you wish using the \texttt{set.mm} database in conjunction with the
textbooks referenced therein.

Once we prove those theorems, we then restate these proven theorems as axioms.
This lets us easily identify which axioms are needed for a particular complex number proof, without the obfuscation of the set theory used to derive them.
As a result,
the construction is actually unimportant other
than to show that sets exist that satisfy the axioms, and thus that the axioms
are consistent if ZF set theory is consistent.  When working with real numbers
you can think of them as being the actual sets resulting
from the construction (for definiteness), or you can
think of them as otherwise unspecified sets that happen to satisfy the axioms.
The derivation is not easy, but the fact that it works is quite remarkable
and lends support to the idea that ZFC set theory is all we need to
provide a foundation for essentially all of mathematics.

For the axioms we are given (or postulate) 8 classes:  $\mathbb{C}$ (the
set of complex numbers), $\mathbb{R}$ (the set of real numbers, a subset
of $\mathbb{C}$), $0$ (zero), $1$ (one), $i$ (square root of
$-1$), $+$ (plus), $\cdot$ (times), and
$<_{\mathbb{R}}$ (less than for just the real numbers).
Subtraction and division are defined terms and are not part of the
axioms; for their definitions see \texttt{set.mm}.

Note that the notation $(A+B)$ (and similarly $(A\cdot B)$) specifies a class
called an {\em operation},\index{operation} and is the function value of the
class $+$ at ordered pair $\langle A,B \rangle$.  An operation is defined by
statement \texttt{df-opr} on p.~\pageref{dfopr}.
The notation $A <_{\mathbb{R}} B$ specifies a
wff called a {\em binary relation}\index{binary relation} and means $\langle A,B \rangle \in \,<_{\mathbb{R}}$, as defined by statement \texttt{df-br} on p.~\pageref{dfbr}.

Our set of 8 given classes is assumed to satisfy the following 22 axioms
(in the axioms listed below, $<$ really means $<_{\mathbb{R}}$).

\vskip 2ex

\noindent 1. The real numbers are a subset of the complex numbers.

%\vskip 0.5ex
\setbox\startprefix=\hbox{\tt \ \ ax-resscn\ \$p\ }
\setbox\contprefix=\hbox{\tt \ \ \ \ \ \ \ \ \ \ \ \ \ \ }
\startm
\m{\vdash}\m{\mathbb{R}}\m{\subseteq}\m{\mathbb{C}}
\endm
%\vskip 1ex

\noindent 2. One is a complex number.

%\vskip 0.5ex
\setbox\startprefix=\hbox{\tt \ \ ax-1cn\ \$p\ }
\setbox\contprefix=\hbox{\tt \ \ \ \ \ \ \ \ \ \ \ }
\startm
\m{\vdash}\m{1}\m{\in}\m{\mathbb{C}}
\endm
%\vskip 1ex

\noindent 3. The imaginary number $i$ is a complex number.

%\vskip 0.5ex
\setbox\startprefix=\hbox{\tt \ \ ax-icn\ \$p\ }
\setbox\contprefix=\hbox{\tt \ \ \ \ \ \ \ \ \ \ \ }
\startm
\m{\vdash}\m{i}\m{\in}\m{\mathbb{C}}
\endm
%\vskip 1ex

\noindent 4. Complex numbers are closed under addition.

%\vskip 0.5ex
\setbox\startprefix=\hbox{\tt \ \ ax-addcl\ \$p\ }
\setbox\contprefix=\hbox{\tt \ \ \ \ \ \ \ \ \ \ \ \ \ }
\startm
\m{\vdash}\m{(}\m{(}\m{A}\m{\in}\m{\mathbb{C}}\m{\wedge}\m{B}\m{\in}\m{\mathbb{C}}%
\m{)}\m{\rightarrow}\m{(}\m{A}\m{+}\m{B}\m{)}\m{\in}\m{\mathbb{C}}\m{)}
\endm
%\vskip 1ex

\noindent 5. Real numbers are closed under addition.

%\vskip 0.5ex
\setbox\startprefix=\hbox{\tt \ \ ax-addrcl\ \$p\ }
\setbox\contprefix=\hbox{\tt \ \ \ \ \ \ \ \ \ \ \ \ \ \ }
\startm
\m{\vdash}\m{(}\m{(}\m{A}\m{\in}\m{\mathbb{R}}\m{\wedge}\m{B}\m{\in}\m{\mathbb{R}}%
\m{)}\m{\rightarrow}\m{(}\m{A}\m{+}\m{B}\m{)}\m{\in}\m{\mathbb{R}}\m{)}
\endm
%\vskip 1ex

\noindent 6. Complex numbers are closed under multiplication.

%\vskip 0.5ex
\setbox\startprefix=\hbox{\tt \ \ ax-mulcl\ \$p\ }
\setbox\contprefix=\hbox{\tt \ \ \ \ \ \ \ \ \ \ \ \ \ }
\startm
\m{\vdash}\m{(}\m{(}\m{A}\m{\in}\m{\mathbb{C}}\m{\wedge}\m{B}\m{\in}\m{\mathbb{C}}%
\m{)}\m{\rightarrow}\m{(}\m{A}\m{\cdot}\m{B}\m{)}\m{\in}\m{\mathbb{C}}\m{)}
\endm
%\vskip 1ex

\noindent 7. Real numbers are closed under multiplication.

%\vskip 0.5ex
\setbox\startprefix=\hbox{\tt \ \ ax-mulrcl\ \$p\ }
\setbox\contprefix=\hbox{\tt \ \ \ \ \ \ \ \ \ \ \ \ \ \ }
\startm
\m{\vdash}\m{(}\m{(}\m{A}\m{\in}\m{\mathbb{R}}\m{\wedge}\m{B}\m{\in}\m{\mathbb{R}}%
\m{)}\m{\rightarrow}\m{(}\m{A}\m{\cdot}\m{B}\m{)}\m{\in}\m{\mathbb{R}}\m{)}
\endm
%\vskip 1ex

\noindent 8. Multiplication of complex numbers is commutative.

%\vskip 0.5ex
\setbox\startprefix=\hbox{\tt \ \ ax-mulcom\ \$p\ }
\setbox\contprefix=\hbox{\tt \ \ \ \ \ \ \ \ \ \ \ \ \ \ }
\startm
\m{\vdash}\m{(}\m{(}\m{A}\m{\in}\m{\mathbb{C}}\m{\wedge}\m{B}\m{\in}\m{\mathbb{C}}%
\m{)}\m{\rightarrow}\m{(}\m{A}\m{\cdot}\m{B}\m{)}\m{=}\m{(}\m{B}\m{\cdot}\m{A}%
\m{)}\m{)}
\endm
%\vskip 1ex

\noindent 9. Addition of complex numbers is associative.

%\vskip 0.5ex
\setbox\startprefix=\hbox{\tt \ \ ax-addass\ \$p\ }
\setbox\contprefix=\hbox{\tt \ \ \ \ \ \ \ \ \ \ \ \ \ \ }
\startm
\m{\vdash}\m{(}\m{(}\m{A}\m{\in}\m{\mathbb{C}}\m{\wedge}\m{B}\m{\in}\m{\mathbb{C}}%
\m{\wedge}\m{C}\m{\in}\m{\mathbb{C}}\m{)}\m{\rightarrow}\m{(}\m{(}\m{A}\m{+}%
\m{B}\m{)}\m{+}\m{C}\m{)}\m{=}\m{(}\m{A}\m{+}\m{(}\m{B}\m{+}\m{C}\m{)}\m{)}%
\m{)}
\endm
%\vskip 1ex

\noindent 10. Multiplication of complex numbers is associative.

%\vskip 0.5ex
\setbox\startprefix=\hbox{\tt \ \ ax-mulass\ \$p\ }
\setbox\contprefix=\hbox{\tt \ \ \ \ \ \ \ \ \ \ \ \ \ \ }
\startm
\m{\vdash}\m{(}\m{(}\m{A}\m{\in}\m{\mathbb{C}}\m{\wedge}\m{B}\m{\in}\m{\mathbb{C}}%
\m{\wedge}\m{C}\m{\in}\m{\mathbb{C}}\m{)}\m{\rightarrow}\m{(}\m{(}\m{A}\m{\cdot}%
\m{B}\m{)}\m{\cdot}\m{C}\m{)}\m{=}\m{(}\m{A}\m{\cdot}\m{(}\m{B}\m{\cdot}\m{C}%
\m{)}\m{)}\m{)}
\endm
%\vskip 1ex

\noindent 11. Multiplication distributes over addition for complex numbers.

%\vskip 0.5ex
\setbox\startprefix=\hbox{\tt \ \ ax-distr\ \$p\ }
\setbox\contprefix=\hbox{\tt \ \ \ \ \ \ \ \ \ \ \ \ \ }
\startm
\m{\vdash}\m{(}\m{(}\m{A}\m{\in}\m{\mathbb{C}}\m{\wedge}\m{B}\m{\in}\m{\mathbb{C}}%
\m{\wedge}\m{C}\m{\in}\m{\mathbb{C}}\m{)}\m{\rightarrow}\m{(}\m{A}\m{\cdot}\m{(}%
\m{B}\m{+}\m{C}\m{)}\m{)}\m{=}\m{(}\m{(}\m{A}\m{\cdot}\m{B}\m{)}\m{+}\m{(}%
\m{A}\m{\cdot}\m{C}\m{)}\m{)}\m{)}
\endm
%\vskip 1ex

\noindent 12. The square of $i$ equals $-1$ (expressed as $i$-squared plus 1 is
0).

%\vskip 0.5ex
\setbox\startprefix=\hbox{\tt \ \ ax-i2m1\ \$p\ }
\setbox\contprefix=\hbox{\tt \ \ \ \ \ \ \ \ \ \ \ \ }
\startm
\m{\vdash}\m{(}\m{(}\m{i}\m{\cdot}\m{i}\m{)}\m{+}\m{1}\m{)}\m{=}\m{0}
\endm
%\vskip 1ex

\noindent 13. One and zero are distinct.

%\vskip 0.5ex
\setbox\startprefix=\hbox{\tt \ \ ax-1ne0\ \$p\ }
\setbox\contprefix=\hbox{\tt \ \ \ \ \ \ \ \ \ \ \ \ }
\startm
\m{\vdash}\m{1}\m{\ne}\m{0}
\endm
%\vskip 1ex

\noindent 14. One is an identity element for real multiplication.

%\vskip 0.5ex
\setbox\startprefix=\hbox{\tt \ \ ax-1rid\ \$p\ }
\setbox\contprefix=\hbox{\tt \ \ \ \ \ \ \ \ \ \ \ }
\startm
\m{\vdash}\m{(}\m{A}\m{\in}\m{\mathbb{R}}\m{\rightarrow}\m{(}\m{A}\m{\cdot}\m{1}%
\m{)}\m{=}\m{A}\m{)}
\endm
%\vskip 1ex

\noindent 15. Every real number has a negative.

%\vskip 0.5ex
\setbox\startprefix=\hbox{\tt \ \ ax-rnegex\ \$p\ }
\setbox\contprefix=\hbox{\tt \ \ \ \ \ \ \ \ \ \ \ \ \ \ }
\startm
\m{\vdash}\m{(}\m{A}\m{\in}\m{\mathbb{R}}\m{\rightarrow}\m{\exists}\m{x}\m{\in}%
\m{\mathbb{R}}\m{(}\m{A}\m{+}\m{x}\m{)}\m{=}\m{0}\m{)}
\endm
%\vskip 1ex

\noindent 16. Every nonzero real number has a reciprocal.

%\vskip 0.5ex
\setbox\startprefix=\hbox{\tt \ \ ax-rrecex\ \$p\ }
\setbox\contprefix=\hbox{\tt \ \ \ \ \ \ \ \ \ \ \ \ \ \ }
\startm
\m{\vdash}\m{(}\m{A}\m{\in}\m{\mathbb{R}}\m{\rightarrow}\m{(}\m{A}\m{\ne}\m{0}%
\m{\rightarrow}\m{\exists}\m{x}\m{\in}\m{\mathbb{R}}\m{(}\m{A}\m{\cdot}%
\m{x}\m{)}\m{=}\m{1}\m{)}\m{)}
\endm
%\vskip 1ex

\noindent 17. A complex number can be expressed in terms of two reals.

%\vskip 0.5ex
\setbox\startprefix=\hbox{\tt \ \ ax-cnre\ \$p\ }
\setbox\contprefix=\hbox{\tt \ \ \ \ \ \ \ \ \ \ \ \ }
\startm
\m{\vdash}\m{(}\m{A}\m{\in}\m{\mathbb{C}}\m{\rightarrow}\m{\exists}\m{x}\m{\in}%
\m{\mathbb{R}}\m{\exists}\m{y}\m{\in}\m{\mathbb{R}}\m{A}\m{=}\m{(}\m{x}\m{+}\m{(}%
\m{y}\m{\cdot}\m{i}\m{)}\m{)}\m{)}
\endm
%\vskip 1ex

\noindent 18. Ordering on reals satisfies strict trichotomy.

%\vskip 0.5ex
\setbox\startprefix=\hbox{\tt \ \ ax-pre-lttri\ \$p\ }
\setbox\contprefix=\hbox{\tt \ \ \ \ \ \ \ \ \ \ \ \ \ }
\startm
\m{\vdash}\m{(}\m{(}\m{A}\m{\in}\m{\mathbb{R}}\m{\wedge}\m{B}\m{\in}\m{\mathbb{R}}%
\m{)}\m{\rightarrow}\m{(}\m{A}\m{<}\m{B}\m{\leftrightarrow}\m{\lnot}\m{(}\m{A}%
\m{=}\m{B}\m{\vee}\m{B}\m{<}\m{A}\m{)}\m{)}\m{)}
\endm
%\vskip 1ex

\noindent 19. Ordering on reals is transitive.

%\vskip 0.5ex
\setbox\startprefix=\hbox{\tt \ \ ax-pre-lttrn\ \$p\ }
\setbox\contprefix=\hbox{\tt \ \ \ \ \ \ \ \ \ \ \ \ \ }
\startm
\m{\vdash}\m{(}\m{(}\m{A}\m{\in}\m{\mathbb{R}}\m{\wedge}\m{B}\m{\in}\m{\mathbb{R}}%
\m{\wedge}\m{C}\m{\in}\m{\mathbb{R}}\m{)}\m{\rightarrow}\m{(}\m{(}\m{A}\m{<}%
\m{B}\m{\wedge}\m{B}\m{<}\m{C}\m{)}\m{\rightarrow}\m{A}\m{<}\m{C}\m{)}\m{)}
\endm
%\vskip 1ex

\noindent 20. Ordering on reals is preserved after addition to both sides.

%\vskip 0.5ex
\setbox\startprefix=\hbox{\tt \ \ ax-pre-ltadd\ \$p\ }
\setbox\contprefix=\hbox{\tt \ \ \ \ \ \ \ \ \ \ \ \ \ }
\startm
\m{\vdash}\m{(}\m{(}\m{A}\m{\in}\m{\mathbb{R}}\m{\wedge}\m{B}\m{\in}\m{\mathbb{R}}%
\m{\wedge}\m{C}\m{\in}\m{\mathbb{R}}\m{)}\m{\rightarrow}\m{(}\m{A}\m{<}\m{B}\m{%
\rightarrow}\m{(}\m{C}\m{+}\m{A}\m{)}\m{<}\m{(}\m{C}\m{+}\m{B}\m{)}\m{)}\m{)}
\endm
%\vskip 1ex

\noindent 21. The product of two positive reals is positive.

%\vskip 0.5ex
\setbox\startprefix=\hbox{\tt \ \ ax-pre-mulgt0\ \$p\ }
\setbox\contprefix=\hbox{\tt \ \ \ \ \ \ \ \ \ \ \ \ \ \ }
\startm
\m{\vdash}\m{(}\m{(}\m{A}\m{\in}\m{\mathbb{R}}\m{\wedge}\m{B}\m{\in}\m{\mathbb{R}}%
\m{)}\m{\rightarrow}\m{(}\m{(}\m{0}\m{<}\m{A}\m{\wedge}\m{0}%
\m{<}\m{B}\m{)}\m{\rightarrow}\m{0}\m{<}\m{(}\m{A}\m{\cdot}\m{B}\m{)}%
\m{)}\m{)}
\endm
%\vskip 1ex

\noindent 22. A non-empty, bounded-above set of reals has a supremum.

%\vskip 0.5ex
\setbox\startprefix=\hbox{\tt \ \ ax-pre-sup\ \$p\ }
\setbox\contprefix=\hbox{\tt \ \ \ \ \ \ \ \ \ \ \ }
\startm
\m{\vdash}\m{(}\m{(}\m{A}\m{\subseteq}\m{\mathbb{R}}\m{\wedge}\m{A}\m{\ne}\m{%
\varnothing}\m{\wedge}\m{\exists}\m{x}\m{\in}\m{\mathbb{R}}\m{\forall}\m{y}\m{%
\in}\m{A}\m{\,y}\m{<}\m{x}\m{)}\m{\rightarrow}\m{\exists}\m{x}\m{\in}\m{%
\mathbb{R}}\m{(}\m{\forall}\m{y}\m{\in}\m{A}\m{\lnot}\m{x}\m{<}\m{y}\m{\wedge}\m{%
\forall}\m{y}\m{\in}\m{\mathbb{R}}\m{(}\m{y}\m{<}\m{x}\m{\rightarrow}\m{\exists}%
\m{z}\m{\in}\m{A}\m{\,y}\m{<}\m{z}\m{)}\m{)}\m{)}
\endm

% NOTE: The \m{...} expressions above could be represented as
% $ \vdash ( ( A \subseteq \mathbb{R} \wedge A \ne \varnothing \wedge \exists x \in \mathbb{R} \forall y \in A \,y < x ) \rightarrow \exists x \in \mathbb{R} ( \forall y \in A \lnot x < y \wedge \forall y \in \mathbb{R} ( y < x \rightarrow \exists z \in A \,y < z ) ) ) $

\vskip 2ex

This completes the set of axioms for real and complex numbers.  You may
wish to look at how subtraction, division, and decimal numbers
are defined in \texttt{set.mm}, and for fun look at the proof of $2+
2 = 4$ (theorem \texttt{2p2e4} in \texttt{set.mm})
as discussed in section \ref{2p2e4}.

In \texttt{set.mm} we define the non-negative integers $\mathbb{N}$, the integers
$\mathbb{Z}$, and the rationals $\mathbb{Q}$ as subsets of $\mathbb{R}$.  This leads
to the nice inclusion $\mathbb{N} \subseteq \mathbb{Z} \subseteq \mathbb{Q} \subseteq
\mathbb{R} \subseteq \mathbb{C}$, giving us a uniform framework in which, for
example, a property such as commutativity of complex number addition
automatically applies to integers.  The natural numbers $\mathbb{N}$
are different from the set $\omega$ we defined earlier, but both satisfy
Peano's postulates.

\subsubsection{Complex Number Axioms in Analysis Texts}

Most analysis texts construct complex numbers as ordered pairs of reals,
leading to construction-dependent properties that satisfy these axioms
but are not stated in their pure form.  (This is also done in
\texttt{set.mm} but our axioms are extracted from that construction.)
Other texts will simply state that $\mathbb{R}$ is a ``complete ordered
subfield of $\mathbb{C}$,'' leading to redundant axioms when this phrase
is completely expanded out.  In fact I have not seen a text with the
axioms in the explicit form above.
None of these axioms is unique individually, but this carefully worked out
collection of axioms is the result of years of work
by the Metamath community.

\subsubsection{Eliminating Unnecessary Complex Number Axioms}

We once had more axioms for real and complex numbers, but over years of time
we (the Metamath community)
have found ways to eliminate them (by proving them from other axioms)
or weaken them (by making weaker claims without reducing what
can be proved).
In particular, here are statements that used to be complex number
axioms but have since been formally proven (with Metamath) to be redundant:

\begin{itemize}
\item
  $\mathbb{C} \in V$.
  At one time this was listed as a ``complex number axiom''
  However, this is not properly speaking a complex number axiom,
  and in any case its proof uses axioms of set theory.
  Proven redundant by Mario Carneiro\index{Carneiro, Mario} on
  17-Nov-2014 (see \texttt{axcnex}).
\item
  $((A \in \mathbb{C} \land B \in \mathbb{C}$) $\rightarrow$
  $(A + B) = (B + A))$.
  Proved redundant by Eric Schmidt\index{Schmidt, Eric} on 19-Jun-2012,
  and formalized by Scott Fenton\index{Fenton, Scott} on 3-Jan-2013
  (see \texttt{addcom}).
\item
  $(A \in \mathbb{C} \rightarrow (A + 0) = A)$.
  Proved redundant by Eric Schmidt on 19-Jun-2012,
  and formalized by Scott Fenton on 3-Jan-2013
  (see \texttt{addid1}).
\item
  $(A \in \mathbb{C} \rightarrow \exists x \in \mathbb{C} (A + x) = 0)$.
  Proved redundant by Eric Schmidt and formalized on 21-May-2007
  (see \texttt{cnegex}).
\item
  $((A \in \mathbb{C} \land A \ne 0) \rightarrow \exists x \in \mathbb{C} (A \cdot x) = 1)$.
  Proved redundant by Eric Schmidt and formalized on 22-May-2007
  (see \texttt{recex}).
\item
  $0 \in \mathbb{R}$.
  Proved redundant by Eric Schmidt on 19-Feb-2005 and formalized 21-May-2007
  (see \texttt{0re}).
\end{itemize}

We could eliminate 0 as an axiomatic object by defining it as
$( ( i \cdot i ) + 1 )$
and replacing it with this expression throughout the axioms. If this
is done, axiom ax-i2m1 becomes redundant. However, the remaining axioms
would become longer and less intuitive.

Eric Schmidt's paper analyzing this axiom system \cite{Schmidt}
presented a proof that these remaining axioms,
with the possible exception of ax-mulcom, are independent of the others.
It is currently an open question if ax-mulcom is independent of the others.

\section{Two Plus Two Equals Four}\label{2p2e4}

Here is a proof that $2 + 2 = 4$, as proven in the theorem \texttt{2p2e4}
in the database \texttt{set.mm}.
This is a useful demonstration of what a Metamath proof can look like.
This proof may have more steps than you're used to, but each step is rigorously
proven all the way back to the axioms of logic and set theory.
This display was originally generated by the Metamath program
as an {\sc HTML} file.

In the table showing the proof ``Step'' is the sequential step number,
while its associated ``Expression'' is an expression that we have proved.
``Ref'' is the name of a theorem or axiom that justifies that expression,
and ``Hyp'' refers to previous steps (if any) that the theorem or axiom
needs so that we can use it.  Expressions are indented further than
the expressions that depend on them to show their interdependencies.

\begin{table}[!htbp]
\caption{Two plus two equals four}
\begin{tabular}{lllll}
\textbf{Step} & \textbf{Hyp} & \textbf{Ref} & \textbf{Expression} & \\
1  &       & df-2    & $ \; \; \vdash 2 = 1 + 1$  & \\
2  & 1     & oveq2i  & $ \; \vdash (2 + 2) = (2 + (1 + 1))$ & \\
3  &       & df-4    & $ \; \; \vdash 4 = (3 + 1)$ & \\
4  &       & df-3    & $ \; \; \; \vdash 3 = (2 + 1)$ & \\
5  & 4     & oveq1i  & $ \; \; \vdash (3 + 1) = ((2 + 1) + 1)$ & \\
6  &       & 2cn     & $ \; \; \; \vdash 2 \in \mathbb{C}$ & \\
7  &       & ax-1cn  & $ \; \; \; \vdash 1 \in \mathbb{C}$ & \\
8  & 6,7,7 & addassi & $ \; \; \vdash ((2 + 1) + 1) = (2 + (1 + 1))$ & \\
9  & 3,5,8 & 3eqtri  & $ \; \vdash 4 = (2 + (1 + 1))$ & \\
10 & 2,9   & eqtr4i  & $ \vdash (2 + 2) = 4$ & \\
\end{tabular}
\end{table}

Step 1 says that we can assert that $2 = 1 + 1$ because it is
justified by \texttt{df-2}.
What is \texttt{df-2}?
It is simply the definition of $2$, which in our system is defined as being
equal to $1 + 1$.  This shows how we can use definitions in proofs.

Look at Step 2 of the proof. In the Ref column, we see that it references
a previously proved theorem, \texttt{oveq2i}.
It turns out that
theorem \texttt{oveq2i} requires a
hypothesis, and in the Hyp column of Step 2 we indicate that Step 1 will
satisfy (match) this hypothesis.
If we looked at \texttt{oveq2i}
we would find that it proves that given some hypothesis
$A = B$, we can prove that $( C F A ) = ( C F B )$.
If we use \texttt{oveq2i} and apply step 1's result as the hypothesis,
that will mean that $A = 2$ and $B = ( 1 + 1 )$ within this use of
\texttt{oveq2i}.
We are free to select any value of $C$ and $F$ (subject to syntax constraints),
so we are free to select $C = 2$ and $F = +$,
producing our desired result,
$ (2 + 2) = (2 + (1 + 1))$.

Step 2 is an example of substitution.
In the end, every step in every proof uses only this one substitution rule.
All the rules of logic, and all the axioms, are expressed so that
they can be used via this one substitution rule.
So once you master substitution, you can master every Metamath proof,
no exceptions.

Each step is clear and can be immediately checked.
In the {\sc HTML} display you can even click on each reference to see why it is
justified, making it easy to see why the proof works.

\section{Deduction}\label{deduction}

Strictly speaking,
a deduction (also called an inference) is a kind of statement that needs
some hypotheses to be true in order for its conclusion to be true.
A theorem, on the other hand, has no hypotheses.
Informally we often call both of them theorems, but in this section we
will stick to the strict definitions.

It sometimes happens that we have proved a deduction of the form
$\varphi \Rightarrow \psi$\index{$\Rightarrow$}
(given hypothesis $\varphi$ we can prove $\psi$)
and we want to then prove a theorem of the form
$\varphi \rightarrow \psi$.

Converting a deduction (which uses a hypothesis) into a theorem
(which does not) is not as simple as you might think.
The deduction says, ``if we can prove $\varphi$ then we can prove $\psi$,''
which is in some sense weaker than saying
``$\varphi$ implies $\psi$.''
There is no axiom of logic that permits us to directly obtain the theorem
given the deduction.\footnote{
The conversion of a deduction to a theorem does not even hold in general
for quantum propositional calculus,
which is a weak subset of classical propositional calculus.
It has been shown that adding the Standard Deduction Theorem (discussed below)
to quantum propositional calculus turns it into classical
propositional calculus!
}

This is in contrast to going the other way.
If we have the theorem ($\varphi \rightarrow \psi$),
it is easy to recover the deduction
($\varphi \Rightarrow \psi$)
using modus ponens\index{modus ponens}
(\texttt{ax-mp}; see section \ref{axmp}).

In the following subsections we first discuss the standard deduction theorem
(the traditional but awkward way to convert deductions into theorems) and
the weak deduction theorem (a limited version of the standard deduction
theorem that is easier to use and was once widely used in
\texttt{set.mm}\index{set theory database (\texttt{set.mm})}).
In section \ref{deductionstyle} we discuss
deduction style, the newer approach we now recommend in most cases.
Deduction style uses ``deduction form,'' a form that
prefixes each hypothesis (other than definitions) and the
conclusion with a universal antecedent (``$\varphi \rightarrow$'').
Deduction style is widely used in \texttt{set.mm},
so it is useful to understand it and \textit{why} it is widely used.
Section \ref{naturaldeduction}
briefly discusses our approach for using natural deduction
within \texttt{set.mm},
as that approach is deeply related to deduction style.
We conclude with a summary of the strengths of
our approach, which we believe are compelling.

\subsection{The Standard Deduction Theorem}\label{standarddeductiontheorem}

It is possible to make use of information
contained in the deduction or its proof to assist us with the proof of
the related theorem.
In traditional logic books, there is a metatheorem called the
Deduction Theorem\index{Deduction Theorem}\index{Standard Deduction Theorem},
discovered independently by Herbrand and Tarski around 1930.
The Deduction Theorem, which we often call the Standard Deduction Theorem,
provides an algorithm for constructing a proof of a theorem from the
proof of its corresponding deduction. See, for example,
\cite[p.~56]{Margaris}\index{Margaris, Angelo}.
To construct a proof for a theorem, the
algorithm looks at each step in the proof of the original deduction and
rewrites the step with several steps wherein the hypothesis is eliminated
and becomes an antecedent.

In ordinary mathematics, no one actually carries out the algorithm,
because (in its most basic form) it involves an exponential explosion of
the number of proof steps as more hypotheses are eliminated. Instead,
the Standard Deduction Theorem is invoked simply to claim that it can
be done in principle, without actually doing it.
What's more, the algorithm is not as simple as it might first appear
when applying it rigorously.
There is a subtle restriction on the Standard Deduction Theorem
that must be taken into account involving the axiom of generalization
when working with predicate calculus (see the literature for more detail).

One of the goals of Metamath is to let you plainly see, with as few
underlying concepts as possible, how mathematics can be derived directly
from the axioms, and not indirectly according to some hidden rules
buried inside a program or understood only by logicians. If we added
the Standard Deduction Theorem to the language and proof verifier,
that would greatly complicate both and largely defeat Metamath's goal
of simplicity. In principle, we could show direct proofs by expanding
out the proof steps generated by the algorithm of the Standard Deduction
Theorem, but that is not feasible in practice because the number of proof
steps quickly becomes huge, even astronomical.
Since the algorithm of the Standard Deduction Theorem is driven by the proof,
we would have to go through that proof
all over again---starting from axioms---in order to obtain the theorem form.
In terms of proof length, there would be no savings over just
proving the theorem directly instead of first proving the deduction form.

\subsection{Weak Deduction Theorem}\label{weakdeductiontheorem}

We have developed
a more efficient method for proving a theorem from a deduction
that can be used instead of the Standard Deduction Theorem
in many (but not all) cases.
We call this more efficient method the
Weak Deduction Theorem\index{Weak Deduction Theorem}.\footnote{
There is also an unrelated ``Weak Deduction Theorem''
in the field of relevance logic, so to avoid confusion we could call
ours the ``Weak Deduction Theorem for Classical Logic.''}
Unlike the Standard Deduction Theorem, the Weak Deduction Theorem produces the
theorem directly from a special substitution instance of the deduction,
using a small, fixed number of steps roughly proportional to the length
of the final theorem.

If you come to a proof referencing the Weak Deduction Theorem
\texttt{dedth} (or one of its variants \texttt{dedthxx}),
here is how to follow the proof without getting into the details:
just click on the theorem referenced in the step
just before the reference to \texttt{dedth} and ignore everything else.
Theorem \texttt{dedth} simply turns a hypothesis into an antecedent
(i.e. the hypothesis followed by $\rightarrow$
is placed in front of the assertion, and the hypothesis
itself is eliminated) given certain conditions.

The Weak Deduction Theorem
eliminates a hypothesis $\varphi$, making it become an antecedent.
It does this by proving an expression
$ \varphi \rightarrow \psi $ given two hypotheses:
(1)
$ ( A = {\rm if} ( \varphi , A , B ) \rightarrow ( \varphi \leftrightarrow \chi ) ) $
and
(2) $\chi$.
Note that it requires that a proof exists for $\varphi$ when the class variable
$A$ is replaced with a specific class $B$. The hypothesis $\chi$
should be assigned to the inference.
You can see the details of the proof of the Weak Deduction Theorem
in theorem \texttt{dedth}.

The Weak Deduction Theorem
is probably easier to understand by studying proofs that make use of it.
For example, let's look at the proof of \texttt{renegcl}, which proves that
$ \vdash ( A \in \mathbb{R} \rightarrow - A \in \mathbb{R} )$:

\needspace{4\baselineskip}
\begin{longtabu} {l l l X}
\textbf{Step} & \textbf{Hyp} & \textbf{Ref} & \textbf{Expression} \\
  1 &  & negeq &
  $\vdash$ $($ $A$ $=$ ${\rm if}$ $($ $A$ $\in$ $\mathbb{R}$ $,$ $A$ $,$ $1$ $)$ $\rightarrow$
  $\textrm{-}$ $A$ $=$ $\textrm{-}$ ${\rm if}$ $($ $A$ $\in$ $\mathbb{R}$
  $,$ $A$ $,$ $1$ $)$ $)$ \\
 2 & 1 & eleq1d &
    $\vdash$ $($ $A$ $=$ ${\rm if}$ $($ $A$ $\in$ $\mathbb{R}$ $,$ $A$ $,$ $1$ $)$ $\rightarrow$ $($
    $\textrm{-}$ $A$ $\in$ $\mathbb{R}$ $\leftrightarrow$
    $\textrm{-}$ ${\rm if}$ $($ $A$ $\in$ $\mathbb{R}$ $,$ $A$ $,$ $1$ $)$ $\in$
    $\mathbb{R}$ $)$ $)$ \\
 3 &  & 1re & $\vdash 1 \in \mathbb{R}$ \\
 4 & 3 & elimel &
   $\vdash {\rm if} ( A \in \mathbb{R} , A , 1 ) \in \mathbb{R}$ \\
 5 & 4 & renegcli &
   $\vdash \textrm{-} {\rm if} ( A \in \mathbb{R} , A , 1 ) \in \mathbb{R}$ \\
 6 & 2,5 & dedth &
   $\vdash ( A \in \mathbb{R} \rightarrow \textrm{-} A \in \mathbb{R}$ ) \\
\end{longtabu}

The somewhat strange-looking steps in \texttt{renegcl} before step 5 are
technical stuff that makes this magic work, and they can be ignored
for a quick overview of the proof. To continue following the ``important''
part of the proof of \texttt{renegcl},
you can look at the reference to \texttt{renegcli} at step 5.

That said, let's briefly look at how
\texttt{renegcl} uses the
Weak Deduction Theorem (\texttt{dedth}) to do its job,
in case you want to do something similar or want understand it more deeply.
Let's work backwards in the proof of \texttt{renegcl}.
Step 6 applies \texttt{dedth} to produce our goal result
$ \vdash ( A \in \mathbb{R} \rightarrow\, - A \in \mathbb{R} )$.
This requires on the one hand the (substituted) deduction
\texttt{renegcli} in step 5.
By itself \texttt{renegcli} proves the deduction
$ \vdash A \in \mathbb{R} \Rightarrow\, \vdash - A \in \mathbb{R}$;
this is the deduction form we are trying to turn into theorem form,
and thus
\texttt{renegcli} has a separate hypothesis that must be fulfilled.
To fulfill the hypothesis of the invocation of
\texttt{renegcli} in step 5, it is eventually
reduced to the already proven theorem $1 \in \mathbb{R}$ in step 3.
Step 4 connects steps 3 and 5; step 4 invokes
\texttt{elimel}, a special case of \texttt{elimhyp} that eliminates
a membership hypothesis for the weak deduction theorem.
On the other hand, the equivalence of the conclusion of
\texttt{renegcl}
$( - A \in \mathbb{R} )$ and the substituted conclusion of
\texttt{renegcli} must be proven, which is done in steps 2 and 1.

The weak deduction theorem has limitations.
In particular, we must be able to prove a special case of the deduction's
hypothesis as a stand-alone theorem.
For example, we used $1 \in \mathbb{R}$ in step 3 of \texttt{renegcl}.

We used to use the weak deduction theorem
extensively within \texttt{set.mm}.
However, we now recommend applying ``deduction style''
instead in most cases, as deduction style is
often an easier and clearer approach.
Therefore, we will now describe deduction style.

\subsection{Deduction Style}\label{deductionstyle}

We now prefer to write assertions in ``deduction form''
instead of writing a proof that would require use of the standard or
weak deduction theorem.
We call this appraoch
``deduction style.''\index{deduction style}

It will be easier to explain this by first defining some terms:

\begin{itemize}
\item \textbf{closed form}\index{closed form}\index{forms!closed}:
A kind of assertion (theorem) with no hypotheses.
Typically its label has no special suffix.
An example is \texttt{unss}, which states:
$\vdash ( ( A \subseteq C \wedge B \subseteq C ) \leftrightarrow ( A \cup B )
\subseteq C )\label{eq:unss}$
\item \textbf{deduction form}\index{deduction form}\index{forms!deduction}:
A kind of assertion with one or more hypotheses
where the conclusion is an implication with
a wff variable as the antecedent (usually $\varphi$), and every hypothesis
(\$e statement)
is either (1) an implication with the same antecedent as the conclusion or
(2) a definition.
A definition
can be for a class variable (this is a class variable followed by ``='')
or a wff variable (this is a wff variable followed by $\leftrightarrow$);
class variable definitions are more common.
In practice, a proof
in deduction form will also contain many steps that are implications
where the antecedent is either that wff variable (normally $\varphi$)
or is
a conjunction (...$\land$...) including that wff variable ($\varphi$).
If an assertion is in deduction form, and other forms are also available,
then we suffix its label with ``d.''
An example is \texttt{unssd}, which states\footnote{
For brevity we show here (and in other places)
a $\&$\index{$\&$} between hypotheses\index{hypotheses}
and a $\Rightarrow$\index{$\Rightarrow$}\index{conclusion}
between the hypotheses and the conclusion.
This notation is technically not part of the Metamath language, but is
instead a convenient abbreviation to show both the hypotheses and conclusion.}:
$\vdash ( \varphi \rightarrow A \subseteq C )\quad\&\quad \vdash ( \varphi
    \rightarrow B \subseteq C )\quad\Rightarrow\quad \vdash ( \varphi
    \rightarrow ( A \cup B ) \subseteq C )\label{eq:unssd}$
\item \textbf{inference form}\index{inference form}\index{forms!inference}:
A kind of assertion with one or more hypotheses that is not in deduction form
(e.g., there is no common antecedent).
If an assertion is in inference form, and other forms are also available,
then we suffix its label with ``i.''
An example is \texttt{unssi}, which states:
$\vdash A \subseteq C\quad\&\quad \vdash B \subseteq C\quad\Rightarrow\quad
    \vdash ( A \cup B ) \subseteq C\label{eq:unssi}$
\end{itemize}

When using deduction style we express an assertion in deduction form.
This form prefixes each hypothesis (other than definitions) and the
conclusion with a universal antecedent (``$\varphi \rightarrow$'').
The antecedent (e.g., $\varphi$)
mimics the context handled in the deduction theorem, eliminating
the need to directly use the deduction theorem.

Once you have an assertion in deduction form, you can easily convert it
to inference form or closed form:

\begin{itemize}
\item To
prove some assertion Ti in inference form, given assertion Td in deduction
form, there is a simple mechanical process you can use. First take each
Ti hypothesis and insert a \texttt{T.} $\rightarrow$ prefix (``true implies'')
using \texttt{a1i}. You
can then use the existing assertion Td to prove the resulting conclusion
with a \texttt{T.} $\rightarrow$ prefix.
Finally, you can remove that prefix using \texttt{trud},
resulting in the conclusion you wanted to prove.
\item To
prove some assertion T in closed form, given assertion Td in deduction
form, there is another simple mechanical process you can use. First,
select an expression that is the conjunction (...$\land$...) of all of the
consequents of every hypothesis of Td. Next, prove that this expression
implies each of the separate hypotheses of Td in turn by eliminating
conjuncts (there are a variety of proven assertions to do this, including
\texttt{simpl},
\texttt{simpr},
\texttt{3simpa},
\texttt{3simpb},
\texttt{3simpc},
\texttt{simp1},
\texttt{simp2},
and
\texttt{simp3}).
If the
expression has nested conjunctions, inner conjuncts can be broken out by
chaining the above theorems with \texttt{syl}
(see section \ref{syl}).\footnote{
There are actually many theorems
(labeled simp* such as \texttt{simp333}) that break out inner conjuncts in one
step, but rather than learning them you can just use the chaining we
just described to prove them, and then let the Metamath program command
\texttt{minimize{\char`\_}with}\index{\texttt{minimize{\char`\_}with} command}
figure out the right ones needed to collapse them.}
As your final step, you can then apply the already-proven assertion Td
(which is in deduction form), proving assertion T in closed form.
\end{itemize}

We can also easily convert any assertion T in closed form to its related
assertion Ti in inference form by applying
modus ponens\index{modus ponens} (see section \ref{axmp}).

The deduction form antecedent can also be used to represent the context
necessary to support natural deduction systems, so we will now
discuss natural deduction.

\subsection{Natural Deduction}\label{naturaldeduction}

Natural deduction\index{natural deduction}
(ND) systems, as such, were originally introduced in
1934 by two logicians working independently: Ja\'skowski and Gentzen. ND
systems are supposed to reconstruct, in a formally proper way, traditional
ways of mathematical reasoning (such as conditional proof, indirect proof,
and proof by cases). As reconstructions they were naturally influenced
by previous work, and many specific ND systems and notations have been
developed since their original work.

There are many ND variants, but
Indrzejczak \cite[p.~31-32]{Indrzejczak}\index{Indrzejczak, Andrzej}
suggests that any natural deductive system must satisfy at
least these three criteria:

\begin{itemize}
\item ``There are some means for entering assumptions into a proof and
also for eliminating them. Usually it requires some bookkeeping devices
for indicating the scope of an assumption, and showing that a part of
a proof depending on eliminated assumption is discharged.
\item There are no (or, at least, a very limited set of) axioms, because
their role is taken over by the set of primitive rules for introduction
and elimination of logical constants which means that elementary
inferences instead of formulae are taken as primitive.
\item (A genuine) ND system admits a lot of freedom in proof construction
and possibility of applying several proof search strategies, like
conditional proof, proof by cases, proof by reductio ad absurdum etc.''
\end{itemize}

The Metamath Proof Explorer (MPE) as defined in \texttt{set.mm}
is fundamentally a Hilbert-style system.
That is, MPE is based on a larger number of axioms (compared
to natural deduction systems), a very small set of rules of inference
(modus ponens), and the context is not changed by the rules of inference
in the middle of a proof. That said, MPE proofs can be developed using
the natural deduction (ND) approach as originally developed by Ja\'skowski
and Gentzen.

The most common and recommended approach for applying ND in MPE is to use
deduction form\index{deduction form}%
\index{forms!deduction}
and apply the MPE proven assertions that are equivalent to ND rules.
For example, MPE's \texttt{jca} is equivalent to ND rule $\land$-I
(and-insertion).
We maintain a list of equivalences that you may consult.
This approach for applying an ND approach within MPE relies on Metamath's
wff metavariables in an essential way, and is described in more detail
in the presentation ``Natural Deductions in the Metamath Proof Language''
by Mario Carneiro \cite{CarneiroND}\index{Carneiro, Mario}.

In this style many steps are an implication, whose antecedent mimics
the context ($\Gamma$) of most ND systems. To add an assumption, simply add
it to the implication antecedent (typically using
\texttt{simpr}),
and use that
new antecedent for all later claims in the same scope. If you wish to
use an assertion in an ND hypothesis scope that is outside the current
ND hypothesis scope, modify the assertion so that the ND hypothesis
assumption is added to its antecedent (typically using \texttt{adantr}). Most
proof steps will be proved using rules that have hypotheses and results
of the form $\varphi \rightarrow$ ...

An example may make this clearer.
Let's show theorem 5.5 of
\cite[p.~18]{Clemente}\index{Clemente Laboreo, Daniel}
along with a line by line translation using the usual
translation of natural deduction (ND) in the Metamath Proof Explorer
(MPE) notation (this is proof \texttt{ex-natded5.5}).
The proof's original goal was to prove
$\lnot \psi$ given two hypotheses,
$( \psi \rightarrow \chi )$ and $ \lnot \chi$.
We will translate these statements into MPE deduction form
by prefixing them all with $\varphi \rightarrow$.
As a result, in MPE the goal is stated as
$( \varphi \rightarrow \lnot \psi )$, and the two hypotheses are stated as
$( \varphi \rightarrow ( \psi \rightarrow \chi ) )$ and
$( \varphi \rightarrow \lnot \chi )$.

The following table shows the proof in Fitch natural deduction style
and its MPE equivalent.
The \textit{\#} column shows the original numbering,
\textit{MPE\#} shows the number in the equivalent MPE proof
(which we will show later),
\textit{ND Expression} shows the original proof claim in ND notation,
and \textit{MPE Translation} shows its translation into MPE
as discussed in this section.
The final columns show the rationale in ND and MPE respectively.

\needspace{4\baselineskip}
{\setlength{\extrarowsep}{4pt} % Keep rows from being too close together
\begin{longtabu}   { @{} c c X X X X }
\textbf{\#} & \textbf{MPE\#} & \textbf{ND Ex\-pres\-sion} &
\textbf{MPE Trans\-lation} & \textbf{ND Ration\-ale} &
\textbf{MPE Ra\-tio\-nale} \\
\endhead

1 & 2;3 &
$( \psi \rightarrow \chi )$ &
$( \varphi \rightarrow ( \psi \rightarrow \chi ) )$ &
Given &
\$e; \texttt{adantr} to put in ND hypothesis \\

2 & 5 &
$ \lnot \chi$ &
$( \varphi \rightarrow \lnot \chi )$ &
Given &
\$e; \texttt{adantr} to put in ND hypothesis \\

3 & 1 &
... $\vert$ $\psi$ &
$( \varphi \rightarrow \psi )$ &
ND hypothesis assumption &
\texttt{simpr} \\

4 & 4 &
... $\chi$ &
$( ( \varphi \land \psi ) \rightarrow \chi )$ &
$\rightarrow$\,E 1,3 &
\texttt{mpd} 1,3 \\

5 & 6 &
... $\lnot \chi$ &
$( ( \varphi \land \psi ) \rightarrow \lnot \chi )$ &
IT 2 &
\texttt{adantr} 5 \\

6 & 7 &
$\lnot \psi$ &
$( \varphi \rightarrow \lnot \psi )$ &
$\land$\,I 3,4,5 &
\texttt{pm2.65da} 4,6 \\

\end{longtabu}
}


The original used Latin letters; we have replaced them with Greek letters
to follow Metamath naming conventions and so that it is easier to follow
the Metamath translation. The Metamath line-for-line translation of
this natural deduction approach precedes every line with an antecedent
including $\varphi$ and uses the Metamath equivalents of the natural deduction
rules. To add an assumption, the antecedent is modified to include it
(typically by using \texttt{adantr};
\texttt{simpr} is useful when you want to
depend directly on the new assumption, as is shown here).

In Metamath we can represent the two given statements as these hypotheses:

\needspace{2\baselineskip}
\begin{itemize}
\item ex-natded5.5.1 $\vdash ( \varphi \rightarrow ( \psi \rightarrow \chi ) )$
\item ex-natded5.5.2 $\vdash ( \varphi \rightarrow \lnot \chi )$
\end{itemize}

\needspace{4\baselineskip}
Here is the proof in Metamath as a line-by-line translation:

\begin{longtabu}   { l l l X }
\textbf{Step} & \textbf{Hyp} & \textbf{Ref} & \textbf{Ex\-pres\-sion} \\
\endhead
1 & & simpr & $\vdash ( ( \varphi \land \psi ) \rightarrow \psi )$ \\
2 & & ex-natded5.5.1 &
  $\vdash ( \varphi \rightarrow ( \psi \rightarrow \chi ) )$ \\
3 & 2 & adantr &
 $\vdash ( ( \varphi \land \psi ) \rightarrow ( \psi \rightarrow \chi ) )$ \\
4 & 1, 3 & mpd &
 $\vdash ( ( \varphi \land \psi ) \rightarrow \chi ) $ \\
5 & & ex-natded5.5.2 &
 $\vdash ( \varphi \rightarrow \lnot \chi )$ \\
6 & 5 & adantr &
 $\vdash ( ( \varphi \land \psi ) \rightarrow \lnot \chi )$ \\
7 & 4, 6 & pm2.65da &
 $\vdash ( \varphi \rightarrow \lnot \psi )$ \\
\end{longtabu}

Only using specific natural deduction rules directly can lead to very
long proofs, for exactly the same reason that only using axioms directly
in Hilbert-style proofs can lead to very long proofs.
If the goal is short and clear proofs,
then it is better to reuse already-proven assertions
in deduction form than to start from scratch each time
and using only basic natural deduction rules.

\subsection{Strengths of Our Approach}

As far as we know there is nothing else in the literature like either the
weak deduction theorem or Mario Carneiro\index{Carneiro, Mario}'s
natural deduction method.
In order to
transform a hypothesis into an antecedent, the literature's standard
``Deduction Theorem''\index{Deduction Theorem}\index{Standard Deduction Theorem}
requires metalogic outside of the notions provided
by the axiom system. We instead generally prefer to use Mario Carneiro's
natural deduction method, then use the weak deduction theorem in cases
where that is difficult to apply, and only then use the full standard
deduction theorem as a last resort.

The weak deduction theorem\index{Weak Deduction Theorem}
does not require any additional metalogic
but converts an inference directly into a closed form theorem, with
a rigorous proof that uses only the axiom system. Unlike the standard
Deduction Theorem, there is no implicit external justification that we
have to trust in order to use it.

Mario Carneiro's natural deduction\index{natural deduction}
method also does not require any new metalogical
notions. It avoids the Deduction Theorem's metalogic by prefixing the
hypotheses and conclusion of every would-be inference with a universal
antecedent (``$\varphi \rightarrow$'') from the very start.

We think it is impressive and satisfying that we can do so much in a
practical sense without stepping outside of our Hilbert-style axiom system.
Of course our axiomatization, which is in the form of schemes,
contains a metalogic of its own that we exploit. But this metalogic
is relatively simple, and for our Deduction Theorem alternatives,
we primarily use just the direct substitution of expressions for
metavariables.

\begin{sloppy}
\section{Exploring the Set The\-o\-ry Data\-base}\label{exploring}
\end{sloppy}
% NOTE: All examples performed in this section are
% recorded wtih "set width 61" % on set.mm as of 2019-05-28
% commit c1e7849557661260f77cfdf0f97ac4354fbb4f4d.

At this point you may wish to study the \texttt{set.mm}\index{set theory
database (\texttt{set.mm})} file in more detail.  Pay particular
attention to the assumptions needed to define wffs\index{well-formed
formula (wff)} (which are not included above), the variable types
(\texttt{\$f}\index{\texttt{\$f} statement} statements), and the
definitions that are introduced.  Start with some simple theorems in
propositional calculus, making sure you understand in detail each step
of a proof.  Once you get past the first few proofs and become familiar
with the Metamath language, any part of the \texttt{set.mm} database
will be as easy to follow, step by step, as any other part---you won't
have to undergo a ``quantum leap'' in mathematical sophistication to be
able to follow a deep proof in set theory.

Next, you may want to explore how concepts such as natural numbers are
defined and described.  This is probably best done in conjunction with
standard set theory textbooks, which can help give you a higher-level
understanding.  The \texttt{set.mm} database provides references that will get
you started.  From there, you will be on your way towards a very deep,
rigorous understanding of abstract mathematics.

The Metamath\index{Metamath} program can help you peruse a Metamath data\-base,
wheth\-er you are trying to figure out how a certain step follows in a proof or
just have a general curiosity.  We will go through some examples of the
commands, using the \texttt{set.mm}\index{set theory database (\texttt{set.mm})}
database provided with the Metamath software.  These should help get you
started.  See Chapter~\ref{commands} for a more detailed description of
the commands.  Note that we have included the full spelling of all commands to
prevent ambiguity with future commands.  In practice you may type just the
characters needed to specify each command keyword\index{command keyword}
unambiguously, often just one or two characters per keyword, and you don't
need to type them in upper case.

First run the Metamath program as described earlier.  You should see the
\verb/MM>/ prompt.  Read in the \texttt{set.mm} file:\index{\texttt{read}
command}

\begin{verbatim}
MM> read set.mm
Reading source file "set.mm"... 34554442 bytes
34554442 bytes were read into the source buffer.
The source has 155711 statements; 2254 are $a and 32250 are $p.
No errors were found.  However, proofs were not checked.
Type VERIFY PROOF * if you want to check them.
\end{verbatim}

As with most examples in this book, what you will see
will be slightly different because we are continuously
improving our databases (including \texttt{set.mm}).

Let's check the database integrity.  This may take a minute or two to run if
your computer is slow.

\begin{verbatim}
MM> verify proof *
0 10%  20%  30%  40%  50%  60%  70%  80%  90% 100%
..................................................
All proofs in the database were verified in 2.84 s.
\end{verbatim}

No errors were reported, so every proof is correct.

You need to know the names (labels) of theorems before you can look at them.
Often just examining the database file(s) with a text editor is the best
approach.  In \texttt{set.mm} there are many detailed comments, especially near
the beginning, that can help guide you. The \texttt{search} command in the
Metamath program is also handy.  The \texttt{comments} qualifier will list the
statements whose associated comment (the one immediately before it) contain a
string you give it.  For example, if you are studying Enderton's {\em Elements
of Set Theory} \cite{Enderton}\index{Enderton, Herbert B.} you may want to see
the references to it in the database.  The search string \texttt{enderton} is not
case sensitive.  (This will not show you all the database theorems that are in
Enderton's book because there is usually only one citation for a given
theorem, which may appear in several textbooks.)\index{\texttt{search}
command}

\begin{verbatim}
MM> search * "enderton" / comments
12067 unineq $p "... Exercise 20 of [Enderton] p. 32 and ..."
12459 undif2 $p "...Corollary 6K of [Enderton] p. 144. (C..."
12953 df-tp $a "...s. Definition of [Enderton] p. 19. (Co..."
13689 unissb $p ".... Exercise 5 of [Enderton] p. 26 and ..."
\end{verbatim}
\begin{center}
(etc.)
\end{center}

Or you may want to see what theorems have something to do with
conjunction (logical {\sc and}).  The quotes around the search
string are optional when there's no ambiguity.\index{\texttt{search}
command}

\begin{verbatim}
MM> search * conjunction / comments
120 a1d $p "...be replaced with a conjunction ( ~ df-an )..."
662 df-bi $a "...viated form after conjunction is introdu..."
1319 wa $a "...ff definition to include conjunction ('and')."
1321 df-an $a "Define conjunction (logical 'and'). Defini..."
1420 imnan $p "...tion in terms of conjunction. (Contribu..."
\end{verbatim}
\begin{center}
(etc.)
\end{center}


Now we will start to look at some details.  Let's look at the first
axiom of propositional calculus
(we could use \texttt{sh st} to abbreviate
\texttt{show statement}).\index{\texttt{show statement} command}

\begin{verbatim}
MM> show statement ax-1/full
Statement 19 is located on line 881 of the file "set.mm".
"Axiom _Simp_.  Axiom A1 of [Margaris] p. 49.  One of the 3
axioms of propositional calculus.  The 3 axioms are also
        ...
19 ax-1 $a |- ( ph -> ( ps -> ph ) ) $.
Its mandatory hypotheses in RPN order are:
  wph $f wff ph $.
  wps $f wff ps $.
The statement and its hypotheses require the variables:  ph
      ps
The variables it contains are:  ph ps


Statement 49 is located on line 11182 of the file "set.mm".
Its statement number for HTML pages is 6.
"Axiom _Simp_.  Axiom A1 of [Margaris] p. 49.  One of the 3
axioms of propositional calculus.  The 3 axioms are also
given as Definition 2.1 of [Hamilton] p. 28.
...
49 ax-1 $a |- ( ph -> ( ps -> ph ) ) $.
Its mandatory hypotheses in RPN order are:
  wph $f wff ph $.
  wps $f wff ps $.
The statement and its hypotheses require the variables:
  ph ps
The variables it contains are:  ph ps
\end{verbatim}

Compare this to \texttt{ax-1} on p.~\pageref{ax1}.  You can see that the
symbol \texttt{ph} is the {\sc ascii} notation for $\varphi$, etc.  To
see the mathematical symbols for any expression you may typeset it in
\LaTeX\ (type \texttt{help tex} for instructions)\index{latex@{\LaTeX}}
or, easier, just use a text editor to look at the comments where symbols
are first introduced in \texttt{set.mm}.  The hypotheses \texttt{wph}
and \texttt{wps} required by \texttt{ax-1} mean that variables
\texttt{ph} and \texttt{ps} must be wffs.

Next we'll pick a simple theorem of propositional calculus, the Principle of
Identity, which is proved directly from the axioms.  We'll look at the
statement then its proof.\index{\texttt{show statement}
command}

\begin{verbatim}
MM> show statement id1/full
Statement 116 is located on line 11371 of the file "set.mm".
Its statement number for HTML pages is 22.
"Principle of identity.  Theorem *2.08 of [WhiteheadRussell]
p. 101.  This version is proved directly from the axioms for
demonstration purposes.
...
116 id1 $p |- ( ph -> ph ) $= ... $.
Its mandatory hypotheses in RPN order are:
  wph $f wff ph $.
Its optional hypotheses are:  wps wch wth wta wet
      wze wsi wrh wmu wla wka
The statement and its hypotheses require the variables:  ph
These additional variables are allowed in its proof:
      ps ch th ta et ze si rh mu la ka
The variables it contains are:  ph
\end{verbatim}

The optional variables\index{optional variable} \texttt{ps}, \texttt{ch}, etc.\ are
available for use in a proof of this statement if we wish, and were we to do
so we would make use of optional hypotheses \texttt{wps}, \texttt{wch}, etc.  (See
Section~\ref{dollaref} for the meaning of ``optional
hypothesis.''\index{optional hypothesis}) The reason these show up in the
statement display is that statement \texttt{id1} happens to be in their scope
(see Section~\ref{scoping} for the definition of ``scope''\index{scope}), but
in fact in propositional calculus we will never make use of optional
hypotheses or variables.  This becomes important after quantifiers are
introduced, where ``dummy'' variables\index{dummy variable} are often needed
in the middle of a proof.

Let's look at the proof of statement \texttt{id1}.  We'll use the
\texttt{show proof} command, which by default suppresses the
``non-essential'' steps that construct the wffs.\index{\texttt{show proof}
command}
We will display the proof in ``lemmon' format (a non-indented format
with explicit previous step number references) and renumber the
displayed steps:

\begin{verbatim}
MM> show proof id1 /lemmon/renumber
1 ax-1           $a |- ( ph -> ( ph -> ph ) )
2 ax-1           $a |- ( ph -> ( ( ph -> ph ) -> ph ) )
3 ax-2           $a |- ( ( ph -> ( ( ph -> ph ) -> ph ) ) ->
                     ( ( ph -> ( ph -> ph ) ) -> ( ph -> ph )
                                                          ) )
4 2,3 ax-mp      $a |- ( ( ph -> ( ph -> ph ) ) -> ( ph -> ph
                                                          ) )
5 1,4 ax-mp      $a |- ( ph -> ph )
\end{verbatim}

If you have read Section~\ref{trialrun}, you'll know how to interpret this
proof.  Step~2, for example, is an application of axiom \texttt{ax-1}.  This
proof is identical to the one in Hamilton's {\em Logic for Mathematicians}
\cite[p.~32]{Hamilton}\index{Hamilton, Alan G.}.

You may want to look at what
substitutions\index{substitution!variable}\index{variable substitution} are
made into \texttt{ax-1} to arrive at step~2.  The command to do this needs to
know the ``real'' step number, so we'll display the proof again without
the \texttt{renumber} qualifier.\index{\texttt{show proof}
command}

\begin{verbatim}
MM> show proof id1 /lemmon
 9 ax-1          $a |- ( ph -> ( ph -> ph ) )
20 ax-1          $a |- ( ph -> ( ( ph -> ph ) -> ph ) )
24 ax-2          $a |- ( ( ph -> ( ( ph -> ph ) -> ph ) ) ->
                     ( ( ph -> ( ph -> ph ) ) -> ( ph -> ph )
                                                          ) )
25 20,24 ax-mp   $a |- ( ( ph -> ( ph -> ph ) ) -> ( ph -> ph
                                                          ) )
26 9,25 ax-mp    $a |- ( ph -> ph )
\end{verbatim}

The ``real'' step number is 20.  Let's look at its details.

\begin{verbatim}
MM> show proof id1 /detailed_step 20
Proof step 20:  min=ax-1 $a |- ( ph -> ( ( ph -> ph ) -> ph )
  )
This step assigns source "ax-1" ($a) to target "min" ($e).
The source assertion requires the hypotheses "wph" ($f, step
18) and "wps" ($f, step 19).  The parent assertion of the
target hypothesis is "ax-mp" ($a, step 25).
The source assertion before substitution was:
    ax-1 $a |- ( ph -> ( ps -> ph ) )
The following substitutions were made to the source
assertion:
    Variable  Substituted with
     ph        ph
     ps        ( ph -> ph )
The target hypothesis before substitution was:
    min $e |- ph
The following substitution was made to the target hypothesis:
    Variable  Substituted with
     ph        ( ph -> ( ( ph -> ph ) -> ph ) )
\end{verbatim}

This shows the substitutions\index{substitution!variable}\index{variable
substitution} made to the variables in \texttt{ax-1}.  References are made to
steps 18 and 19 which are not shown in our proof display.  To see these steps,
you can display the proof with the \texttt{all} qualifier.

Let's look at a slightly more advanced proof of propositional calculus.  Note
that \verb+/\+ is the symbol for $\wedge$ (logical {\sc and}, also
called conjunction).\index{conjunction ($\wedge$)}
\index{logical {\sc and} ($\wedge$)}

\begin{verbatim}
MM> show statement prth/full
Statement 1791 is located on line 15503 of the file "set.mm".
Its statement number for HTML pages is 559.
"Conjoin antecedents and consequents of two premises.  This
is the closed theorem form of ~ anim12d .  Theorem *3.47 of
[WhiteheadRussell] p. 113.  It was proved by Leibniz,
and it evidently pleased him enough to call it
_praeclarum theorema_ (splendid theorem).
...
1791 prth $p |- ( ( ( ph -> ps ) /\ ( ch -> th ) ) -> ( ( ph
      /\ ch ) -> ( ps /\ th ) ) ) $= ... $.
Its mandatory hypotheses in RPN order are:
  wph $f wff ph $.
  wps $f wff ps $.
  wch $f wff ch $.
  wth $f wff th $.
Its optional hypotheses are:  wta wet wze wsi wrh wmu wla wka
The statement and its hypotheses require the variables:  ph
      ps ch th
These additional variables are allowed in its proof:  ta et
      ze si rh mu la ka
The variables it contains are:  ph ps ch th


MM> show proof prth /lemmon/renumber
1 simpl          $p |- ( ( ( ph -> ps ) /\ ( ch -> th ) ) ->
                                               ( ph -> ps ) )
2 simpr          $p |- ( ( ( ph -> ps ) /\ ( ch -> th ) ) ->
                                               ( ch -> th ) )
3 1,2 anim12d    $p |- ( ( ( ph -> ps ) /\ ( ch -> th ) ) ->
                           ( ( ph /\ ch ) -> ( ps /\ th ) ) )
\end{verbatim}

There are references to a lot of unfamiliar statements.  To see what they are,
you may type the following:

\begin{verbatim}
MM> show proof prth /statement_summary
Summary of statements used in the proof of "prth":

Statement simpl is located on line 14748 of the file
"set.mm".
"Elimination of a conjunct.  Theorem *3.26 (Simp) of
[WhiteheadRussell] p. 112. ..."
  simpl $p |- ( ( ph /\ ps ) -> ph ) $= ... $.

Statement simpr is located on line 14777 of the file
"set.mm".
"Elimination of a conjunct.  Theorem *3.27 (Simp) of
[WhiteheadRussell] ..."
  simpr $p |- ( ( ph /\ ps ) -> ps ) $= ... $.

Statement anim12d is located on line 15445 of the file
"set.mm".
"Conjoin antecedents and consequents in a deduction.
..."
  anim12d.1 $e |- ( ph -> ( ps -> ch ) ) $.
  anim12d.2 $e |- ( ph -> ( th -> ta ) ) $.
  anim12d $p |- ( ph -> ( ( ps /\ th ) -> ( ch /\ ta ) ) )
      $= ... $.
\end{verbatim}
\begin{center}
(etc.)
\end{center}

Of course you can look at each of these statements and their proofs, and
so on, back to the axioms of propositional calculus if you wish.

The \texttt{search} command is useful for finding statements when you
know all or part of their contents.  The following example finds all
statements containing \verb@ph -> ps@ followed by \verb@ch -> th@.  The
\verb@$*@ is a wildcard that matches anything; the \texttt{\$} before the
\verb$*$ prevents conflicts with math symbol token names.  The \verb@*@ after
\texttt{SEARCH} is also a wildcard that in this case means ``match any label.''
\index{\texttt{search} command}

% I'm omitting this one, since readers are unlikely to see it:
% 1096 bisymOLD $p |- ( ( ( ph -> ps ) -> ( ch -> th ) ) -> ( (
%   ( ps -> ph ) -> ( th -> ch ) ) -> ( ( ph <-> ps ) -> ( ch
%    <-> th ) ) ) )
\begin{verbatim}
MM> search * "ph -> ps $* ch -> th"
1791 prth $p |- ( ( ( ph -> ps ) /\ ( ch -> th ) ) -> ( ( ph
    /\ ch ) -> ( ps /\ th ) ) )
2455 pm3.48 $p |- ( ( ( ph -> ps ) /\ ( ch -> th ) ) -> ( (
    ph \/ ch ) -> ( ps \/ th ) ) )
117859 pm11.71 $p |- ( ( E. x ph /\ E. y ch ) -> ( ( A. x (
    ph -> ps ) /\ A. y ( ch -> th ) ) <-> A. x A. y ( ( ph /\
    ch ) -> ( ps /\ th ) ) ) )
\end{verbatim}

Three statements, \texttt{prth}, \texttt{pm3.48},
 and \texttt{pm11.71}, were found to match.

To see what axioms\index{axiom} and definitions\index{definition}
\texttt{prth} ultimately depends on for its proof, you can have the
program backtrack through the hierarchy\index{hierarchy} of theorems and
definitions.\index{\texttt{show trace{\char`\_}back} command}

\begin{verbatim}
MM> show trace_back prth /essential/axioms
Statement "prth" assumes the following axioms ($a
statements):
  ax-1 ax-2 ax-3 ax-mp df-bi df-an
\end{verbatim}

Note that the 3 axioms of propositional calculus and the modus ponens rule are
needed (as expected); in addition, there are a couple of definitions that are used
along the way.  Note that Metamath makes no distinction\index{axiom vs.\
definition} between axioms\index{axiom} and definitions\index{definition}.  In
\texttt{set.mm} they have been distinguished artificially by prefixing their
labels\index{labels in \texttt{set.mm}} with \texttt{ax-} and \texttt{df-}
respectively.  For example, \texttt{df-an} defines conjunction (logical {\sc
and}), which is represented by the symbol \verb+/\+.
Section~\ref{definitions} discusses the philosophy of definitions, and the
Metamath language takes a particularly simple, conservative approach by using
the \texttt{\$a}\index{\texttt{\$a} statement} statement for both axioms and
definitions.

You can also have the program compute how many steps a proof
has\index{proof length} if we were to follow it all the way back to
\texttt{\$a} statements.

\begin{verbatim}
MM> show trace_back prth /essential/count_steps
The statement's actual proof has 3 steps.  Backtracking, a
total of 79 different subtheorems are used.  The statement
and subtheorems have a total of 274 actual steps.  If
subtheorems used only once were eliminated, there would be a
total of 38 subtheorems, and the statement and subtheorems
would have a total of 185 steps.  The proof would have 28349
steps if fully expanded back to axiom references.  The
maximum path length is 38.  A longest path is:  prth <-
anim12d <- syl2and <- sylan2d <- ancomsd <- ancom <- pm3.22
<- pm3.21 <- pm3.2 <- ex <- sylbir <- biimpri <- bicomi <-
bicom1 <- bi2 <- dfbi1 <- impbii <- bi3 <- simprim <- impi <-
con1i <- nsyl2 <- mt3d <- con1d <- notnot1 <- con2i <- nsyl3
<- mt2d <- con2d <- notnot2 <- pm2.18d <- pm2.18 <- pm2.21 <-
pm2.21d <- a1d <- syl <- mpd <- a2i <- a2i.1 .
\end{verbatim}

This tells us that we would have to inspect 274 steps if we want to
verify the proof completely starting from the axioms.  A few more
statistics are also shown.  There are one or more paths back to axioms
that are the longest; this command ferrets out one of them and shows it
to you.  There may be a sense in which the longest path length is
related to how ``deep'' the theorem is.

We might also be curious about what proofs depend on the theorem
\texttt{prth}.  If it is never used later on, we could eliminate it as
redundant if it has no intrinsic interest by itself.\index{\texttt{show
usage} command}

% I decided to show the OLD values here.
\begin{verbatim}
MM> show usage prth
Statement "prth" is directly referenced in the proofs of 18
statements:
  mo3 moOLD 2mo 2moOLD euind reuind reuss2 reusv3i opelopabt
  wemaplem2 rexanre rlimcn2 o1of2 o1rlimmul 2sqlem6 spanuni
  heicant pm11.71
\end{verbatim}

Thus \texttt{prth} is directly used by 18 proofs.
We can use the \texttt{/recursive} qualifier to include indirect use:

\begin{verbatim}
MM> show usage prth /recursive
Statement "prth" directly or indirectly affects the proofs of
24214 statements:
  mo3 mo mo3OLD eu2 moOLD eu2OLD eu3OLD mo4f mo4 eu4 mopick
...
\end{verbatim}

\subsection{A Note on the ``Compact'' Proof Format}

The present version of Metamath\index{Metamath!limitations of
version 0.07.30} will display proofs in a ``compact''\index{compact proof}
format whenever the proof is stored in compressed format in the database.  It
may be be slightly confusing unless you know how to interpret it.  (A
future version may eliminate this format from displays.)  For example,
if you display the complete proof of theorem \texttt{id1} it will start
off as follows:

\begin{verbatim}
MM> show proof id1 /lemmon/all
 1 wph           $f wff ph
 2 wph           $f wff ph
 3 wph           $f wff ph
 4 2,3 wi    @4: $a wff ( ph -> ph )
 5 1,4 wi    @5: $a wff ( ph -> ( ph -> ph ) )
 6 @4            $a wff ( ph -> ph )
\end{verbatim}

\begin{center}
{etc.}
\end{center}

Step 4 has a ``local label,''\index{local label} \texttt{@4}, assigned to it.
Later on, at step 6, the label \texttt{@1} is referenced instead of
displaying the explicit proof for that step.  This technique takes advantage
of the fact that steps in a proof often repeat, especially during the
construction of wffs.  The compact format reduces the number of steps in the
proof display and may actually be preferred by some people, which is one
reason that it is still displayed by the program.

If you want to see the normal format with the ``true'' step numbers, you can
use the following workaround:\index{\texttt{save proof} command}

\begin{verbatim}
MM> save proof id1 /normal
The proof of "id1" has been reformatted and saved internally.
Remember to use WRITE SOURCE to save it permanently.
MM> show proof id1 /lemmon/all
 1 wph           $f wff ph
 2 wph           $f wff ph
 3 wph           $f wff ph
 4 2,3 wi        $a wff ( ph -> ph )
 5 1,4 wi        $a wff ( ph -> ( ph -> ph ) )
 6 wph           $f wff ph
 7 wph           $f wff ph
 8 6,7 wi        $a wff ( ph -> ph )
\end{verbatim}

\begin{center}
{etc.}
\end{center}

Note that the original 6 steps are now 8 steps.  However, the format is
now the same as that described in Chapter~\ref{using}.

\chapter{The Metamath Language}
\label{languagespec}

\begin{quote}
  {\em Thus mathematics may be defined as the subject in which we never know
what we are talking about, nor whether what we are saying is true.}
    \flushright\sc  Bertrand Russell\footnote{\cite[p.~84]{Russell2}.}\\
\end{quote}\index{Russell, Bertrand}

Probably the most striking feature of the Metamath language is its almost
complete absence of hard-wired syntax. Metamath\index{Metamath} does not
understand any mathematics or logic other than that needed to construct finite
sequences of symbols according to a small set of simple, built-in rules.  The
only rule it uses in a proof is the substitution of an expression (symbol
sequence) for a variable, subject to a simple constraint to prevent
bound-variable clashes.  The primitive notions built into Metamath involve the
simple manipulation of finite objects (symbols) that we as humans can easily
visualize and that computers can easily deal with.  They seem to be just
about the simplest notions possible that are required to do standard
mathematics.

This chapter serves as a reference manual for the Metamath\index{Metamath}
language. It covers the tedious technical details of the language, some of
which you may wish to skim in a first reading.  On the other hand, you should
pay close attention to the defined terms in {\bf boldface}; they have precise
meanings that are important to keep in mind for later understanding.  It may
be best to first become familiar with the examples in Chapter~\ref{using} to
gain some motivation for the language.

%% Uncomment this when uncommenting section {formalspec} below
If you have some knowledge of set theory, you may wish to study this
chapter in conjunction with the formal set-theoretical description of the
Metamath language in Appendix~\ref{formalspec}.

We will use the name ``Metamath''\index{Metamath} to mean either the Metamath
computer language or the Metamath software associated with the computer
language.  We will not distinguish these two when the context is clear.

The next section contains the complete specification of the Metamath
language.
It serves as an
authoritative reference and presents the syntax in enough detail to
write a parser\index{parsing Metamath} and proof verifier.  The
specification is terse and it is probably hard to learn the language
directly from it, but we include it here for those impatient people who
prefer to see everything up front before looking at verbose expository
material.  Later sections explain this material and provide examples.
We will repeat the definitions in those sections, and you may skip the
next section at first reading and proceed to Section~\ref{tut1}
(p.~\pageref{tut1}).

\section{Specification of the Metamath Language}\label{spec}
\index{Metamath!specification}

\begin{quote}
  {\em Sometimes one has to say difficult things, but one ought to say
them as simply as one knows how.}
    \flushright\sc  G. H. Hardy\footnote{As quoted in
    \cite{deMillo}, p.~273.}\\
\end{quote}\index{Hardy, G. H.}

\subsection{Preliminaries}\label{spec1}

% Space is technically a printable character, so we'll word things
% carefully so it's unambiguous.
A Metamath {\bf database}\index{database} is built up from a top-level source
file together with any source files that are brought in through file inclusion
commands (see below).  The only characters that are allowed to appear in a
Metamath source file are the 94 non-whitespace printable {\sc
ascii}\index{ascii@{\sc ascii}} characters, which are digits, upper and lower
case letters, and the following 32 special
characters\index{special characters}:\label{spec1chars}

\begin{verbatim}
! " # $ % & ' ( ) * + , - . / :
; < = > ? @ [ \ ] ^ _ ` { | } ~
\end{verbatim}
plus the following characters which are the ``white space'' characters:
space (a printable character),
tab, carriage return, line feed, and form feed.\label{whitespace}
We will use \texttt{typewriter}
font to display the printable characters.

A Metamath database consists of a sequence of three kinds of {\bf
tokens}\index{token} separated by {\bf white space}\index{white space}
(which is any sequence of one or more white space characters).  The set
of {\bf keyword}\index{keyword} tokens is \texttt{\$\char`\{},
\texttt{\$\char`\}}, \texttt{\$c}, \texttt{\$v}, \texttt{\$f},
\texttt{\$e}, \texttt{\$d}, \texttt{\$a}, \texttt{\$p}, \texttt{\$.},
\texttt{\$=}, \texttt{\$(}, \texttt{\$)}, \texttt{\$[}, and
\texttt{\$]}.  The last four are called {\bf auxiliary}\index{auxiliary
keyword} or preprocessing keywords.  A {\bf label}\index{label} token
consists of any combination of letters, digits, and the characters
hyphen, underscore, and period.  A {\bf math symbol}\index{math symbol}
token may consist of any combination of the 92 printable standard {\sc
ascii} characters other than space or \texttt{\$}~. All tokens are
case-sensitive.

\subsection{Preprocessing}

The token \texttt{\$(} begins a {\bf comment} and
\texttt{\$)} ends a comment.\index{\texttt{\$(}
and \texttt{\$)} auxiliary keywords}\index{comment}
Comments may contain any of
the 94 non-whitespace printable characters and white space,
except they may not contain the
2-character sequences \texttt{\$(} or \texttt{\$)} (comments do not nest).
Comments are ignored (treated
like white space) for the purpose of parsing, e.g.,
\texttt{\$( \$[ \$)} is a comment.
See p.~\pageref{mathcomments} for comment typesetting conventions; these
conventions may be ignored for the purpose of parsing.

A {\bf file inclusion command} consists of \texttt{\$[} followed by a file name
followed by \texttt{\$]}.\index{\texttt{\$[} and \texttt{\$]} auxiliary
keywords}\index{included file}\index{file inclusion}
It is only allowed in the outermost scope (i.e., not between
\texttt{\$\char`\{} and \texttt{\$\char`\}})
and must not be inside a statement (e.g., it may not occur
between the label of a \texttt{\$a} statement and its \texttt{\$.}).
The file name may not
contain a \texttt{\$} or white space.  The file must exist.
The case-sensitivity
of its name follows the conventions of the operating system.  The contents of
the file replace the inclusion command.
Included files may include other files.
Only the first reference to a given file is included; any later
references to the same file (whether in the top-level file or in included
files) cause the inclusion command to be ignored (treated like white space).
A verifier may assume that file names with different strings
refer to different files for the purpose of ignoring later references.
A file self-reference is ignored, as is any reference to the top-level file
(to avoid loops).
Included files may not include a \texttt{\$(} without a matching \texttt{\$)},
may not include a \texttt{\$[} without a matching \texttt{\$]}, and may
not include incomplete statements (e.g., a \texttt{\$a} without a matching
\texttt{\$.}).
It is currently unspecified if path references are relative to the process'
current directory or the file's containing directory, so databases should
avoid using pathname separators (e.g., ``/'') in file names.

Like all tokens, the \texttt{\$(}, \texttt{\$)}, \texttt{\$[}, and \texttt{\$]} keywords
must be surrounded by white space.

\subsection{Basic Syntax}

After preprocessing, a database will consist of a sequence of {\bf
statements}.
These are the scoping statements \texttt{\$\char`\{} and
\texttt{\$\char`\}}, along with the \texttt{\$c}, \texttt{\$v},
\texttt{\$f}, \texttt{\$e}, \texttt{\$d}, \texttt{\$a}, and \texttt{\$p}
statements.

A {\bf scoping statement}\index{scoping statement} consists only of its
keyword, \texttt{\$\char`\{} or \texttt{\$\char`\}}.
A \texttt{\$\char`\{} begins a {\bf
block}\index{block} and a matching \texttt{\$\char`\}} ends the block.
Every \texttt{\$\char`\{}
must have a matching \texttt{\$\char`\}}.
Defining it recursively, we say a block
contains a sequence of zero or more tokens other
than \texttt{\$\char`\{} and \texttt{\$\char`\}} and
possibly other blocks.  There is an {\bf outermost
block}\index{block!outermost} not bracketed by \texttt{\$\char`\{} \texttt{\$\char`\}}; the end
of the outermost block is the end of the database.

% LaTeX bug? can't do \bf\tt

A {\bf \$v} or {\bf \$c statement}\index{\texttt{\$v} statement}\index{\texttt{\$c}
statement} consists of the keyword token \texttt{\$v} or \texttt{\$c} respectively,
followed by one or more math symbols,
% The word "token" is used to distinguish "$." from the sentence-ending period.
followed by the \texttt{\$.}\ token.
These
statements {\bf declare}\index{declaration} the math symbols to be {\bf
variables}\index{variable!Metamath} or {\bf constants}\index{constant}
respectively. The same math symbol may not occur twice in a given \texttt{\$v} or
\texttt{\$c} statement.

%c%A math symbol becomes an {\bf active}\index{active math symbol}
%c%when declared and stays active until the end of the block in which it is
%c%declared.  A math symbol may not be declared a second time while it is active,
%c%but it may be declared again after it becomes inactive.

A math symbol becomes {\bf active}\index{active math symbol} when declared
and stays active until the end of the block in which it is declared.  A
variable may not be declared a second time while it is active, but it
may be declared again (as a variable, but not as a constant) after it
becomes inactive.  A constant must be declared in the outermost block and may
not be declared a second time.\footnote{The rules for redeclaration may
become more general in the future;
see p.~\pageref{redeclarationf}.}\index{redeclaration of symbols}

A {\bf \$f statement}\index{\texttt{\$f} statement} consists of a label,
followed by \texttt{\$f}, followed by its typecode (an active constant),
followed by an
active variable, followed by the \texttt{\$.}\ token.  A {\bf \$e
statement}\index{\texttt{\$e} statement} consists of a label, followed
by \texttt{\$e}, followed by its typecode (an active constant),
followed by zero or more
active math symbols, followed by the \texttt{\$.}\ token.  A {\bf
hypothesis}\index{hypothesis} is a \texttt{\$f} or \texttt{\$e}
statement.
The type declared by a \texttt{\$f} statement for a given label
is global even if the variable is not
(e.g., a database may not have \texttt{wff P} in one local scope
nd \texttt{class P} in another).

A {\bf simple \$d statement}\index{\texttt{\$d} statement!simple}
consists of \texttt{\$d}, followed by two different active variables,
followed by the \texttt{\$.}\ token.  A {\bf compound \$d
statement}\index{\texttt{\$d} statement!compound} consists of
\texttt{\$d}, followed by three or more variables (all different),
followed by the \texttt{\$.}\ token.  The order of the variables in a
\texttt{\$d} statement is unimportant.  A compound \texttt{\$d}
statement is equivalent to a set of simple \texttt{\$d} statements, one
for each possible pair of variables occurring in the compound
\texttt{\$d} statement.  Henceforth in this specification we shall
assume all \texttt{\$d} statements are simple.  A \texttt{\$d} statement
is also called a {\bf disjoint} (or {\bf distinct}) {\bf variable
restriction}.\index{disjoint-variable restriction}

A {\bf \$a statement}\index{\texttt{\$a} statement} consists of a label,
followed by \texttt{\$a}, followed by its typecode (an active constant),
followed by
zero or more active math symbols, followed by the \texttt{\$.}\ token.  A {\bf
\$p statement}\index{\texttt{\$p} statement} consists of a label,
followed by \texttt{\$p}, followed by its typecode (an active constant),
followed by
zero or more active math symbols, followed by \texttt{\$=}, followed by
a sequence of labels, followed by the \texttt{\$.}\ token.  An {\bf
assertion}\index{assertion} is a \texttt{\$a} or \texttt{\$p} statement.

A \texttt{\$f}, \texttt{\$e}, or \texttt{\$d} statement is {\bf active}\index{active
statement} from the place it occurs until the end of the block it occurs in.
A \texttt{\$a} or \texttt{\$p} statement is {\bf active} from the place it occurs
through the end of the database.
There may not be two active \texttt{\$f} statements containing the same
variable.  Each variable in a \texttt{\$e}, \texttt{\$a}, or
\texttt{\$p} statement must exist in an active \texttt{\$f}
statement.\footnote{This requirement can greatly simplify the
unification algorithm (substitution calculation) required by proof
verification.}

%The label that begins each \texttt{\$f}, \texttt{\$e}, \texttt{\$a}, and
%\texttt{\$p} statement must be unique.
Each label token must be unique, and
no label token may match any math symbol
token.\label{namespace}\footnote{This
restriction was added on June 24, 2006.
It is not theoretically necessary but is imposed to make it easier to
write certain parsers.}

The set of {\bf mandatory variables}\index{mandatory variable} associated with
an assertion is the set of (zero or more) variables in the assertion and in any
active \texttt{\$e} statements.  The (possibly empty) set of {\bf mandatory
hypotheses}\index{mandatory hypothesis} is the set of all active \texttt{\$f}
statements containing mandatory variables, together with all active \texttt{\$e}
statements.
The set of {\bf mandatory {\bf \$d} statements}\index{mandatory
disjoint-variable restriction} associated with an assertion are those active
\texttt{\$d} statements whose variables are both among the assertion's
mandatory variables.

\subsection{Proof Verification}\label{spec4}

The sequence of labels between the \texttt{\$=} and \texttt{\$.}\ tokens
in a \texttt{\$p} statement is a {\bf proof}.\index{proof!Metamath} Each
label in a proof must be the label of an active statement other than the
\texttt{\$p} statement itself; thus a label must refer either to an
active hypothesis of the \texttt{\$p} statement or to an earlier
assertion.

An {\bf expression}\index{expression} is a sequence of math symbols. A {\bf
substitution map}\index{substitution map} associates a set of variables with a
set of expressions.  It is acceptable for a variable to be mapped to an
expression containing it.  A {\bf
substitution}\index{substitution!variable}\index{variable substitution} is the
simultaneous replacement of all variables in one or more expressions with the
expressions that the variables map to.

A proof is scanned in order of its label sequence.  If the label refers to an
active hypothesis, the expression in the hypothesis is pushed onto a
stack.\index{stack}\index{RPN stack}  If the label refers to an assertion, a
(unique) substitution must exist that, when made to the mandatory hypotheses
of the referenced assertion, causes them to match the topmost (i.e.\ most
recent) entries of the stack, in order of occurrence of the mandatory
hypotheses, with the topmost stack entry matching the last mandatory
hypothesis of the referenced assertion.  As many stack entries as there are
mandatory hypotheses are then popped from the stack.  The same substitution is
made to the referenced assertion, and the result is pushed onto the stack.
After the last label in the proof is processed, the stack must have a single
entry that matches the expression in the \texttt{\$p} statement containing the
proof.

%c%{\footnotesize\begin{quotation}\index{redeclaration of symbols}
%c%{{\em Comment.}\label{spec4comment} Whenever a math symbol token occurs in a
%c%{\texttt{\$c} or \texttt{\$v} statement, it is considered to designate a distinct new
%c%{symbol, even if the same token was previously declared (and is now inactive).
%c%{Thus a math token declared as a constant in two different blocks is considered
%c%{to designate two distinct constants (even though they have the same name).
%c%{The two constants will not match in a proof that references both blocks.
%c%{However, a proof referencing both blocks is acceptable as long as it doesn't
%c%{require that the constants match.  Similarly, a token declared to be a
%c%{constant for a referenced assertion will not match the same token declared to
%c%{be a variable for the \texttt{\$p} statement containing the proof.  In the case
%c%{of a token declared to be a variable for a referenced assertion, this is not
%c%{an issue since the variable can be substituted with whatever expression is
%c%{needed to achieve the required match.
%c%{\end{quotation}}
%c2%A proof may reference an assertion that contains or whose hypotheses contain a
%c2%constant that is not active for the \texttt{\$p} statement containing the proof.
%c2%However, the final result of the proof may not contain that constant. A proof
%c2%may also reference an assertion that contains or whose hypotheses contain a
%c2%variable that is not active for the \texttt{\$p} statement containing the proof.
%c2%That variable, of course, will be substituted with whatever expression is
%c2%needed to achieve the required match.

A proof may contain a \texttt{?}\ in place of a label to indicate an unknown step
(Section~\ref{unknown}).  A proof verifier may ignore any proof containing
\texttt{?}\ but should warn the user that the proof is incomplete.

A {\bf compressed proof}\index{compressed proof}\index{proof!compressed} is an
alternate proof notation described in Appen\-dix~\ref{compressed}; also see
references to ``compressed proof'' in the Index.  Compressed proofs are a
Metamath language extension which a complete proof verifier should be able to
parse and verify.

\subsubsection{Verifying Disjoint Variable Restrictions}

Each substitution made in a proof must be checked to verify that any
disjoint variable restrictions are satisfied, as follows.

If two variables replaced by a substitution exist in a mandatory \texttt{\$d}
statement\index{\texttt{\$d} statement} of the assertion referenced, the two
expressions resulting from the substitution must satisfy the following
conditions.  First, the two expressions must have no variables in common.
Second, each possible pair of variables, one from each expression, must exist
in an active \texttt{\$d} statement of the \texttt{\$p} statement containing the
proof.

\vskip 1ex

This ends the specification of the Metamath language;
see Appendix \ref{BNF} for its syntax in
Extended Backus--Naur Form (EBNF)\index{Extended Backus--Naur Form}\index{EBNF}.

\section{The Basic Keywords}\label{tut1}

Our expository material begins here.

Like most computer languages, Metamath\index{Metamath} takes its input from
one or more {\bf source files}\index{source file} which contain characters
expressed in the standard {\sc ascii} (American Standard Code for Information
Interchange)\index{ascii@{\sc ascii}} code for computers.  A source file
consists of a series of {\bf tokens}\index{token}, which are strings of
non-whitespace
printable characters (from the set of 94 shown on p.~\pageref{spec1chars})
separated by {\bf white space}\index{white space} (spaces, tabs, carriage
returns, line feeds, and form feeds). Any string consisting only of these
characters is treated the same as a single space.  The non-whitespace printable
characters\index{printable character} that Metamath recognizes are the 94
characters on standard {\sc ascii} keyboards.

Metamath has the ability to join several files together to form its
input (Section~\ref{include}).  We call the aggregate contents of all
the files after they have been joined together a {\bf
database}\index{database} to distinguish it from an individual source
file.  The tokens in a database consist of {\bf
keywords}\index{keyword}, which are built into the language, together
with two kinds of user-defined tokens called {\bf labels}\index{label}
and {\bf math symbols}\index{math symbol}.  (Often we will simply say
{\bf symbol}\index{symbol} instead of math symbol for brevity).  The set
of {\bf basic keywords}\index{basic keyword} is
\texttt{\$c}\index{\texttt{\$c} statement},
\texttt{\$v}\index{\texttt{\$v} statement},
\texttt{\$e}\index{\texttt{\$e} statement},
\texttt{\$f}\index{\texttt{\$f} statement},
\texttt{\$d}\index{\texttt{\$d} statement},
\texttt{\$a}\index{\texttt{\$a} statement},
\texttt{\$p}\index{\texttt{\$p} statement},
\texttt{\$=}\index{\texttt{\$=} keyword},
\texttt{\$.}\index{\texttt{\$.}\ keyword},
\texttt{\$\char`\{}\index{\texttt{\$\char`\{} and \texttt{\$\char`\}}
keywords}, and \texttt{\$\char`\}}.  This is the complete set of
syntactical elements of what we call the {\bf basic
language}\index{basic language} of Metamath, and with them you can
express all of the mathematics that were intended by the design of
Metamath.  You should make it a point to become very familiar with them.
Table~\ref{basickeywords} lists the basic keywords along with a brief
description of their functions.  For now, this description will give you
only a vague notion of what the keywords are for; later we will describe
the keywords in detail.


\begin{table}[htp] \caption{Summary of the basic Metamath
keywords} \label{basickeywords}
\begin{center}
\begin{tabular}{|p{4pc}|l|}
\hline
\em \centering Keyword&\em Description\\
\hline
\hline
\centering
   \texttt{\$c}&Constant symbol declaration\\
\hline
\centering
   \texttt{\$v}&Variable symbol declaration\\
\hline
\centering
   \texttt{\$d}&Disjoint variable restriction\\
\hline
\centering
   \texttt{\$f}&Variable-type (``floating'') hypothesis\\
\hline
\centering
   \texttt{\$e}&Logical (``essential'') hypothesis\\
\hline
\centering
   \texttt{\$a}&Axiomatic assertion\\
\hline
\centering
   \texttt{\$p}&Provable assertion\\
\hline
\centering
   \texttt{\$=}&Start of proof in \texttt{\$p} statement\\
\hline
\centering
   \texttt{\$.}&End of the above statement types\\
\hline
\centering
   \texttt{\$\char`\{}&Start of block\\
\hline
\centering
   \texttt{\$\char`\}}&End of block\\
\hline
\end{tabular}
\end{center}
\end{table}

%For LaTeX bug(?) where it puts tables on blank page instead of btwn text
%May have to adjust if text changes
%\newpage

There are some additional keywords, called {\bf auxiliary
keywords}\index{auxiliary keyword} that help make Metamath\index{Metamath}
more practical. These are part of the {\bf extended language}\index{extended
language}. They provide you with a means to put comments into a Metamath
source file\index{source file} and reference other source files.  We will
introduce these in later sections. Table~\ref{otherkeywords} summarizes them
so that you can recognize them now if you want to peruse some source
files while learning the basic keywords.


\begin{table}[htp] \caption{Auxiliary Metamath
keywords} \label{otherkeywords}
\begin{center}
\begin{tabular}{|p{4pc}|l|}
\hline
\em \centering Keyword&\em Description\\
\hline
\hline
\centering
   \texttt{\$(}&Start of comment\\
\hline
\centering
   \texttt{\$)}&End of comment\\
\hline
\centering
   \texttt{\$[}&Start of included source file name\\
\hline
\centering
   \texttt{\$]}&End of included source file name\\
\hline
\end{tabular}
\end{center}
\end{table}
\index{\texttt{\$(} and \texttt{\$)} auxiliary keywords}
\index{\texttt{\$[} and \texttt{\$]} auxiliary keywords}


Unlike those in some computer languages, the keywords\index{keyword} are short
two-character sequences rather than English-like words.  While this may make
them slightly more difficult to remember at first, their brevity allows
them to blend in with the mathematics being described, not
distract from it, like punctuation marks.


\subsection{User-Defined Tokens}\label{dollardollar}\index{token}

As you may have noticed, all keywords\index{keyword} begin with the \texttt{\$}
character.  This mundane monetary symbol is not ordinarily used in higher
mathematics (outside of grant proposals), so we have appropriated it to
distinguish the Metamath\index{Metamath} keywords from ordinary mathematical
symbols. The \texttt{\$} character is thus considered special and may not be
used as a character in a user-defined token.  All tokens and keywords are
case-sensitive; for example, \texttt{n} is considered to be a different character
from \texttt{N}.  Case-sensitivity makes the available {\sc ascii} character set
as rich as possible.

\subsubsection{Math Symbol Tokens}\index{token}

Math symbols\index{math symbol} are tokens used to represent the symbols
that appear in ordinary mathematical formulas.  They may consist of any
combination of the 93 non-whitespace printable {\sc ascii} characters other than
\texttt{\$}~. Some examples are \texttt{x}, \texttt{+}, \texttt{(},
\texttt{|-}, \verb$!%@?&$, and \texttt{bounded}.  For readability, it is
best to try to make these look as similar to actual mathematical symbols
as possible, within the constraints of the {\sc ascii} character set, in
order to make the resulting mathematical expressions more readable.

In the Metamath\index{Metamath} language, you express ordinary
mathematical formulas and statements as sequences of math symbols such
as \texttt{2 + 2 = 4} (five symbols, all constants).\footnote{To
eliminate ambiguity with other expressions, this is expressed in the set
theory database \texttt{set.mm} as \texttt{|- ( 2 + 2
 ) = 4 }, whose \LaTeX\ equivalent is $\vdash
(2+2)=4$.  The \,$\vdash$ means ``is a theorem'' and the
parentheses allow explicit associative grouping.}\index{turnstile
({$\,\vdash$})} They may even be English
sentences, as in \texttt{E is closed and bounded} (five symbols)---here
\texttt{E} would be a variable and the other four symbols constants.  In
principle, a Metamath database could be constructed to work with almost
any unambiguous English-language mathematical statement, but as a
practical matter the definitions needed to provide for all possible
syntax variations would be cumbersome and distracting and possibly have
subtle pitfalls accidentally built in.  We generally recommend that you
express mathematical statements with compact standard mathematical
symbols whenever possible and put their English-language descriptions in
comments.  Axioms\index{axiom} and definitions\index{definition}
(\texttt{\$a}\index{\texttt{\$a} statement} statements) are the only
places where Metamath will not detect an error, and doing this will help
reduce the number of definitions needed.

You are free to use any tokens\index{token} you like for math
symbols\index{math symbol}.  Appendix~\ref{ASCII} recommends token names to
use for symbols in set theory, and we suggest you adopt these in order to be
able to include the \texttt{set.mm} set theory database in your database.  For
printouts, you can convert the tokens in a database
to standard mathematical symbols with the \LaTeX\ typesetting program.  The
Metamath command \texttt{open tex} {\em filename}\index{\texttt{open tex} command}
produces output that can be read by \LaTeX.\index{latex@{\LaTeX}}
The correspondence
between tokens and the actual symbols is made by \texttt{latexdef}
statements inside a special database comment tagged
with \texttt{\$t}.\index{\texttt{\$t} comment}\index{typesetting comment}
  You can edit
this comment to change the definitions or add new ones.
Appendix~\ref{ASCII} describes how to do this in more detail.

% White space\index{white space} is normally used to separate math
% symbol\index{math symbol} tokens, but they may be juxtaposed without white
% space in \texttt{\$d}\index{\texttt{\$d} statement}, \texttt{\$e}\index{\texttt{\$e}
% statement}, \texttt{\$f}\index{\texttt{\$f} statement}, \texttt{\$a}\index{\texttt{\$a}
% statement}, and \texttt{\$p}\index{\texttt{\$p} statement} statements when no
% ambiguity will result.  Specifically, Metamath parses the math symbol sequence
% in one of these statements in the following manner:  when the math symbol
% sequence has been broken up into tokens\index{token} up to a given character,
% the next token is the longest string of characters that could constitute a
% math symbol that is active\index{active
% math symbol} at that point.  (See Section~\ref{scoping} for the
% definition of an active math symbol.)  For example, if \texttt{-}, \texttt{>}, and
% \texttt{->} are the only active math symbols, the juxtaposition \texttt{>-} will be
% interpreted as the two symbols \texttt{>} and \texttt{-}, whereas \texttt{->} will
% always be interpreted as that single symbol.\footnote{For better readability we
% recommend a white space between each token.  This also makes searching for a
% symbol easier to do with an editor.  Omission of optional white space is useful
% for reducing typing when assigning an expression to a temporary
% variable\index{temporary variable} with the \texttt{let variable} Metamath
% program command.}\index{\texttt{let variable} command}
%
% Keywords\index{keyword} may be placed next to math symbols without white
% space\index{white space} between them.\footnote{Again, we do not recommend
% this for readability.}
%
% The math symbols\index{math symbol} in \texttt{\$c}\index{\texttt{\$c} statement}
% and \texttt{\$v}\index{\texttt{\$v} statement} statements must always be separated
% by white space\index{white
% space}, for the obvious reason that these statements define the names
% of the symbols.
%
% Math symbols referred to in comments (see Section~\ref{comments}) must also be
% separated by white space.  This allows you to make comments about symbols that
% are not yet active\index{active
% math symbol}.  (The ``math mode'' feature of comments is also a quick and
% easy way to obtain word processing text with embedded mathematical symbols,
% independently of the main purpose of Metamath; the way to do this is described
% in Section~\ref{comments})

\subsubsection{Label Tokens}\index{token}\index{label}

Label tokens are used to identify Metamath\index{Metamath} statements for
later reference. Label tokens may contain only letters, digits, and the three
characters period, hyphen, and underscore:
\begin{verbatim}
. - _
\end{verbatim}

A label is {\bf declared}\index{label declaration} by placing it immediately
before the keyword of the statement it identifies.  For example, the label
\texttt{axiom.1} might be declared as follows:
\begin{verbatim}
axiom.1 $a |- x = x $.
\end{verbatim}

Each \texttt{\$e}\index{\texttt{\$e} statement},
\texttt{\$f}\index{\texttt{\$f} statement},
\texttt{\$a}\index{\texttt{\$a} statement}, and
\texttt{\$p}\index{\texttt{\$p} statement} statement in a database must
have a label declared for it.  No other statement types may have label
declarations.  Every label must be unique.

A label (and the statement it identifies) is {\bf referenced}\index{label
reference} by including the label between the \texttt{\$=}\index{\texttt{\$=}
keyword} and \texttt{\$.}\index{\texttt{\$.}\ keyword}\ keywords in a \texttt{\$p}
statement.  The sequence of labels\index{label sequence} between these two
keywords is called a {\bf proof}\index{proof}.  An example of a statement with
a proof that we will encounter later (Section~\ref{proof}) is
\begin{verbatim}
wnew $p wff ( s -> ( r -> p ) )
     $= ws wr wp w2 w2 $.
\end{verbatim}

You don't have to know what this means just yet, but you should know that the
label \texttt{wnew} is declared by this \texttt{\$p} statement and that the labels
\texttt{ws}, \texttt{wr}, \texttt{wp}, and \texttt{w2} are assumed to have been declared
earlier in the database and are referenced here.

\subsection{Constants and Variables}
\index{constant}
\index{variable}

An {\bf expression}\index{expression} is any sequence of math
symbols, possibly empty.

The basic Metamath\index{Metamath} language\index{basic language} has two
kinds of math symbols\index{math symbol}:  {\bf constants}\index{constant} and
{\bf variables}\index{variable}.  In a Metamath proof, a constant may not be
substituted with any expression.  A variable can be
substituted\index{substitution!variable}\index{variable substitution} with any
expression.  This sequence may include other variables and may even include
the variable being substituted.  This substitution takes place when proofs are
verified, and it will be described in Section~\ref{proof}.  The \texttt{\$f}
statement (described later in Section~\ref{dollaref}) is used to specify the
{\bf type} of a variable (i.e.\ what kind of
variable it is)\index{variable type}\index{type} and
give it a meaning typically
associated with a ``metavariable''\index{metavariable}\footnote{A metavariable
is a variable that ranges over the syntactical elements of the object language
being discussed; for example, one metavariable might represent a variable of
the object language and another metavariable might represent a formula in the
object language.} in ordinary mathematics; for example, a variable may be
specified to be a wff or well-formed formula (in logic), a set (in set
theory), or a non-negative integer (in number theory).

%\subsection{The \texttt{\$c} and \texttt{\$v} Declaration Statements}
\subsection{The \texttt{\$c} and \texttt{\$v} Declaration Statements}
\index{\texttt{\$c} statement}
\index{constant declaration}
\index{\texttt{\$v} statement}
\index{variable declaration}

Constants are introduced or {\bf declared}\index{constant declaration}
with \texttt{\$c}\index{\texttt{\$c} statement} statements, and
variables are declared\index{variable declaration} with
\texttt{\$v}\index{\texttt{\$v} statement} statements.  A {\bf simple}
declaration\index{simple declaration} statement introduces a single
constant or variable.  Its syntax is one of the following:
\begin{center}
  \texttt{\$c} {\em math-symbol} \texttt{\$.}\\
  \texttt{\$v} {\em math-symbol} \texttt{\$.}
\end{center}
The notation {\em math-symbol} means any math symbol token\index{token}.

Some examples of simple declaration statements are:
\begin{center}
  \texttt{\$c + \$.}\\
  \texttt{\$c -> \$.}\\
  \texttt{\$c ( \$.}\\
  \texttt{\$v x \$.}\\
  \texttt{\$v y2 \$.}
\end{center}

The characters in a math symbol\index{math symbol} being declared are
irrelevant to Meta\-math; for example, we could declare a right parenthesis to
be a variable,
\begin{center}
  \texttt{\$v ) \$.}\\
\end{center}
although this would be unconventional.

A {\bf compound} declaration\index{compound declaration} statement is a
shorthand for declaring several symbols at once.  Its syntax is one of the
following:
\begin{center}
  \texttt{\$c} {\em math-symbol}\ \,$\cdots$\ {\em math-symbol} \texttt{\$.}\\
  \texttt{\$v} {\em math-symbol}\ \,$\cdots$\ {\em math-symbol} \texttt{\$.}
\end{center}\index{\texttt{\$c} statement}
Here, the ellipsis (\ldots) means any number of {\em math-symbol}\,s.

An example of a compound declaration statement is:
\begin{center}
  \texttt{\$v x y mu \$.}\\
\end{center}
This is equivalent to the three simple declaration statements
\begin{center}
  \texttt{\$v x \$.}\\
  \texttt{\$v y \$.}\\
  \texttt{\$v mu \$.}\\
\end{center}
\index{\texttt{\$v} statement}

There are certain rules on where in the database math symbols may be declared,
what sections of the database are aware of them (i.e.\ where they are
``active''), and when they may be declared more than once.  These will be
discussed in Section~\ref{scoping} and specifically on
p.~\pageref{redeclaration}.

\subsection{The \texttt{\$d} Statement}\label{dollard}
\index{\texttt{\$d} statement}

The \texttt{\$d} statement is called a {\bf disjoint-variable restriction}.  The
syntax of the {\bf simple} version of this statement is
\begin{center}
  \texttt{\$d} {\em variable variable} \texttt{\$.}
\end{center}
where each {\em variable} is a previously declared variable and the two {\em
variable}\,s are different.  (More specifically, each  {\em variable} must be
an {\bf active} variable\index{active math symbol}, which means there must be
a previous \texttt{\$v} statement whose {\bf scope}\index{scope} includes the
\texttt{\$d} statement.  These terms will be defined when we discuss scoping
statements in Section~\ref{scoping}.)

In ordinary mathematics, formulas may arise that are true if the variables in
them are distinct\index{distinct variables}, but become false when those
variables are made identical. For example, the formula in logic $\exists x\,x
\neq y$, which means ``for a given $y$, there exists an $x$ that is not equal
to $y$,'' is true in most mathematical theories (namely all non-trivial
theories\index{non-trivial theory}, i.e.\ those that describe more than one
individual, such as arithmetic).  However, if we substitute $y$ with $x$, we
obtain $\exists x\,x \neq x$, which is always false, as it means ``there
exists something that is not equal to itself.''\footnote{If you are a
logician, you will recognize this as the improper substitution\index{proper
substitution}\index{substitution!proper} of a free variable\index{free
variable} with a bound variable\index{bound variable}.  Metamath makes no
inherent distinction between free and bound variables; instead, you let
Metamath know what substitutions are permissible by using \texttt{\$d} statements
in the right way in your axiom system.}\index{free vs.\ bound variable}  The
\texttt{\$d} statement allows you to specify a restriction that forbids the
substitution of one variable with another.  In
this case, we would use the statement
\begin{center}
  \texttt{\$d x y \$.}
\end{center}\index{\texttt{\$d} statement}
to specify this restriction.

The order in which the variables appear in a \texttt{\$d} statement is not
important.  We could also use
\begin{center}
  \texttt{\$d y x \$.}
\end{center}

The \texttt{\$d} statement is actually more general than this, as the
``disjoint''\index{disjoint variables} in its name suggests.  The full meaning
is that if any substitution is made to its two variables (during the
course of a proof that references a \texttt{\$a} or \texttt{\$p} statement
associated with the \texttt{\$d}), the two expressions that result from the
substitution must have no variables in common.  In addition, each possible
pair of variables, one from each expression, must be in a \texttt{\$d} statement
associated with the statement being proved.  (This requirement forces the
statement being proved to ``inherit'' the original disjoint variable
restriction.)

For example, suppose \texttt{u} is a variable.  If the restriction
\begin{center}
  \texttt{\$d A B \$.}
\end{center}
has been specified for a theorem referenced in a
proof, we may not substitute \texttt{A} with \mbox{\tt a + u} and
\texttt{B} with \mbox{\tt b + u} because these two symbol sequences have the
variable \texttt{u} in common.  Furthermore, if \texttt{a} and \texttt{b} are
variables, we may not substitute \texttt{A} with \texttt{a} and \texttt{B} with \texttt{b}
unless we have also specified \texttt{\$d a b} for the theorem being proved; in
other words, the \texttt{\$d} property associated with a pair of variables must
be effectively preserved after substitution.

The \texttt{\$d}\index{\texttt{\$d} statement} statement does {\em not} mean ``the
two variables may not be substituted with the same thing,'' as you might think
at first.  For example, substituting each of \texttt{A} and \texttt{B} in the above
example with identical symbol sequences consisting only of constants does not
cause a disjoint variable conflict, because two symbol sequences have no
variables in common (since they have no variables, period).  Similarly, a
conflict will not occur by substituting the two variables in a \texttt{\$d}
statement with the empty symbol sequence\index{empty substitution}.

The \texttt{\$d} statement does not have a direct counterpart in
ordinary mathematics, partly because the variables\index{variable} of
Metamath are not really the same as the variables\index{variable!in
ordinary mathematics} of ordinary mathematics but rather are
metavariables\index{metavariable} ranging over them (as well as over
other kinds of symbols and groups of symbols).  Depending on the
situation, we may informally interpret the \texttt{\$d} statement in
different ways.  Suppose, for example, that \texttt{x} and \texttt{y}
are variables ranging over numbers (more precisely, that \texttt{x} and
\texttt{y} are metavariables ranging over variables that range over
numbers), and that \texttt{ph} ($\varphi$) and \texttt{ps} ($\psi$) are
variables (more precisely, metavariables) ranging over formulas.  We can
make the following interpretations that correspond to the informal
language of ordinary mathematics:
\begin{quote}
\begin{tabbing}
\texttt{\$d x y \$.} means ``assume $x$ and $y$ are
distinct variables.''\\
\texttt{\$d x ph \$.} means ``assume $x$ does not
occur in $\varphi$.''\\
\texttt{\$d ph ps \$.} \=means ``assume $\varphi$ and
$\psi$ have no variables\\ \>in common.''
\end{tabbing}
\end{quote}\index{\texttt{\$d} statement}

\subsubsection{Compound \texttt{\$d} Statements}

The {\bf compound} version of the \texttt{\$d} statement is a shorthand for
specifying several variables whose substitutions must be pairwise disjoint.
Its syntax is:
\begin{center}
  \texttt{\$d} {\em variable}\ \,$\cdots$\ {\em variable} \texttt{\$.}
\end{center}\index{\texttt{\$d} statement}
Here, {\em variable} represents the token of a previously declared
variable (specifically, an active variable) and all {\em variable}\,s are
different.  The compound \texttt{\$d}
statement is internally broken up by Metamath into one simple \texttt{\$d}
statement for each possible pair of variables in the original \texttt{\$d}
statement.  For example,
\begin{center}
  \texttt{\$d w x y z \$.}
\end{center}
is equivalent to
\begin{center}
  \texttt{\$d w x \$.}\\
  \texttt{\$d w y \$.}\\
  \texttt{\$d w z \$.}\\
  \texttt{\$d x y \$.}\\
  \texttt{\$d x z \$.}\\
  \texttt{\$d y z \$.}
\end{center}

Two or more simple \texttt{\$d} statements specifying the same variable pair are
internally combined into a single \texttt{\$d} statement.  Thus the set of three
statements
\begin{center}
  \texttt{\$d x y \$.}
  \texttt{\$d x y \$.}
  \texttt{\$d y x \$.}
\end{center}
is equivalent to
\begin{center}
  \texttt{\$d x y \$.}
\end{center}

Similarly, compound \texttt{\$d} statements, after being internally broken up,
internally have their common variable pairs combined.  For example the
set of statements
\begin{center}
  \texttt{\$d x y A \$.}
  \texttt{\$d x y B \$.}
\end{center}
is equivalent to
\begin{center}
  \texttt{\$d x y \$.}
  \texttt{\$d x A \$.}
  \texttt{\$d y A \$.}
  \texttt{\$d x y \$.}
  \texttt{\$d x B \$.}
  \texttt{\$d y B \$.}
\end{center}
which is equivalent to
\begin{center}
  \texttt{\$d x y \$.}
  \texttt{\$d x A \$.}
  \texttt{\$d y A \$.}
  \texttt{\$d x B \$.}
  \texttt{\$d y B \$.}
\end{center}

Metamath\index{Metamath} automatically verifies that all \texttt{\$d}
restrictions are met whenever it verifies proofs.  \texttt{\$d} statements are
never referenced directly in proofs (this is why they do not have
labels\index{label}), but Metamath is always aware of which ones must be
satisfied (i.e.\ are active) and will notify you with an error message if any
violation occurs.

To illustrate how Metamath detects a missing \texttt{\$d}
statement, we will look at the following example from the
\texttt{set.mm} database.

\begin{verbatim}
$d x z $.  $d y z $.
$( Theorem to add distinct quantifier to atomic formula. $)
ax17eq $p |- ( x = y -> A. z x = y ) $=...
\end{verbatim}

This statement has the obvious requirement that $z$ must be
distinct\index{distinct variables} from $x$ in theorem \texttt{ax17eq} that
states $x=y \rightarrow \forall z \, x=y$ (well, obvious if you're a logician,
for otherwise we could conclude  $x=y \rightarrow \forall x \, x=y$, which is
false when the free variables $x$ and $y$ are equal.).

Let's look at what happens if we edit the database to comment out this
requirement.

\begin{verbatim}
$( $d x z $. $) $d y z $.
$( Theorem to add distinct quantifier to atomic formula. $)
ax17eq $p |- ( x = y -> A. z x = y ) $=...
\end{verbatim}

When it tries to verify the proof, Metamath will tell you that \texttt{x} and
\texttt{z} must be disjoint, because one of its steps references an axiom or
theorem that has this requirement.

\begin{verbatim}
MM> verify proof ax17eq
ax17eq ?Error at statement 1918, label "ax17eq", type "$p":
      vz wal wi vx vy vz ax-13 vx vy weq vz vx ax-c16 vx vy
                                               ^^^^^
There is a disjoint variable ($d) violation at proof step 29.
Assertion "ax-c16" requires that variables "x" and "y" be
disjoint.  But "x" was substituted with "z" and "y" was
substituted with "x".  The assertion being proved, "ax17eq",
does not require that variables "z" and "x" be disjoint.
\end{verbatim}

We can see the substitutions into \texttt{ax-c16} with the following command.

\begin{verbatim}
MM> show proof ax17eq / detailed_step 29
Proof step 29:  pm2.61dd.2=ax-c16 $a |- ( A. z z = x -> ( x =
  y -> A. z x = y ) )
This step assigns source "ax-c16" ($a) to target "pm2.61dd.2"
($e).  The source assertion requires the hypotheses "wph"
($f, step 26), "vx" ($f, step 27), and "vy" ($f, step 28).
The parent assertion of the target hypothesis is "pm2.61dd"
($p, step 36).
The source assertion before substitution was:
    ax-c16 $a |- ( A. x x = y -> ( ph -> A. x ph ) )
The following substitutions were made to the source
assertion:
    Variable  Substituted with
     x         z
     y         x
     ph        x = y
The target hypothesis before substitution was:
    pm2.61dd.2 $e |- ( ph -> ch )
The following substitutions were made to the target
hypothesis:
    Variable  Substituted with
     ph        A. z z = x
     ch        ( x = y -> A. z x = y )
\end{verbatim}

The disjoint variable restrictions of \texttt{ax-c16} can be seen from the
\texttt{show state\-ment} command.  The line that begins ``\texttt{Its mandatory
dis\-joint var\-i\-able pairs are:}\ldots'' lists any \texttt{\$d} variable
pairs in brackets.

\begin{verbatim}
MM> show statement ax-c16/full
Statement 3033 is located on line 9338 of the file "set.mm".
"Axiom of Distinct Variables. ..."
  ax-c16 $a |- ( A. x x = y -> ( ph -> A. x ph ) ) $.
Its mandatory hypotheses in RPN order are:
  wph $f wff ph $.
  vx $f set x $.
  vy $f set y $.
Its mandatory disjoint variable pairs are:  <x,y>
The statement and its hypotheses require the variables:  x y
      ph
The variables it contains are:  x y ph
\end{verbatim}

Since Metamath will always detect when \texttt{\$d}\index{\texttt{\$d} statement}
statements are needed for a proof, you don't have to worry too much about
forgetting to put one in; it can always be added if you see the error message
above.  If you put in unnecessary \texttt{\$d} statements, the worst that could
happen is that your theorem might not be as general as it could be, and this
may limit its use later on.

On the other hand, when you introduce axioms (\texttt{\$a}\index{\texttt{\$a}
statement} statements), you must be very careful to properly specify the
necessary associated \texttt{\$d} statements since Metamath has no way of knowing
whether your axioms are correct.  For example, Metamath would have no idea
that \texttt{ax-c16}, which we are telling it is an axiom of logic, would lead to
contradictions if we omitted its associated \texttt{\$d} statement.

% This was previously a comment in footnote-sized type, but it can be
% hard to read this much text in a small size.
% As a result, it's been changed to normally-sized text.
\label{nodd}
You may wonder if it is possible to develop standard
mathematics in the Metamath language without the \texttt{\$d}\index{\texttt{\$d}
statement} statement, since it seems like a nuisance that complicates proof
verification. The \texttt{\$d} statement is not needed in certain subsets of
mathematics such as propositional calculus.  However, dummy
variables\index{dummy variable!eliminating} and their associated \texttt{\$d}
statements are impossible to avoid in proofs in standard first-order logic as
well as in the variant used in \texttt{set.mm}.  In fact, there is no upper bound to
the number of dummy variables that might be needed in a proof of a theorem of
first-order logic containing 3 or more variables, as shown by H.\
Andr\'{e}ka\index{Andr{\'{e}}ka, H.} \cite{Nemeti}.  A first-order system that
avoids them entirely is given in \cite{Megill}\index{Megill, Norman}; the
trick there is simply to embed harmlessly the necessary dummy variables into a
theorem being proved so that they aren't ``dummy'' anymore, then interpret the
resulting longer theorem so as to ignore the embedded dummy variables.  If
this interests you, the system in \texttt{set.mm} obtained from \texttt{ax-1}
through \texttt{ax-c14} in \texttt{set.mm}, and deleting \texttt{ax-c16} and \texttt{ax-5},
requires no \texttt{\$d} statements but is logically complete in the sense
described in \cite{Megill}.  This means it can prove any theorem of
first-order logic as long as we add to the theorem an antecedent that embeds
dummy and any other variables that must be distinct.  In a similar fashion,
axioms for set theory can be devised that
do not require distinct variable
provisos\index{Set theory without distinct variable provisos},
as explained at
\url{http://us.metamath.org/mpeuni/mmzfcnd.html}.
Together, these in principle allow all of
mathematics to be developed under Metamath without a \texttt{\$d} statement,
although the length of the resulting theorems will grow as more and
more dummy variables become required in their proofs.

\subsection{The \texttt{\$f}
and \texttt{\$e} Statements}\label{dollaref}
\index{\texttt{\$e} statement}
\index{\texttt{\$f} statement}
\index{floating hypothesis}
\index{essential hypothesis}
\index{variable-type hypothesis}
\index{logical hypothesis}
\index{hypothesis}

Metamath has two kinds of hypo\-theses, the \texttt{\$f}\index{\texttt{\$f}
statement} or {\bf variable-type} hypothesis and the \texttt{\$e} or {\bf logical}
hypo\-the\-sis.\index{\texttt{\$d} statement}\footnote{Strictly speaking, the
\texttt{\$d} statement is also a hypothesis, but it is never directly referenced
in a proof, so we call it a restriction rather than a hypothesis to lessen
confusion.  The checking for violations of \texttt{\$d} restrictions is automatic
and built into Metamath's proof-checking algorithm.} The letters \texttt{f} and
\texttt{e} stand for ``floating''\index{floating hypothesis} (roughly meaning
used only if relevant) and ``essential''\index{essential hypothesis} (meaning
always used) respectively, for reasons that will become apparent
when we discuss frames in
Section~\ref{frames} and scoping in Section~\ref{scoping}. The syntax of these
are as follows:
\begin{center}
  {\em label} \texttt{\$f} {\em typecode} {\em variable} \texttt{\$.}\\
  {\em label} \texttt{\$e} {\em typecode}
      {\em math-symbol}\ \,$\cdots$\ {\em math-symbol} \texttt{\$.}\\
\end{center}
\index{\texttt{\$e} statement}
\index{\texttt{\$f} statement}
A hypothesis must have a {\em label}\index{label}.  The expression in a
\texttt{\$e} hypothesis consists of a typecode (an active constant math symbol)
followed by a sequence
of zero or more math symbols. Each math symbol (including {\em constant}
and {\em variable}) must be a previously declared constant or variable.  (In
addition, each math symbol must be active, which will be covered when we
discuss scoping statements in Section~\ref{scoping}.)  You use a \texttt{\$f}
hypothesis to specify the
nature or {\bf type}\index{variable type}\index{type} of a variable (such as ``let $x$ be an
integer'') and use a \texttt{\$e} hypothesis to express a logical truth (such as
``assume $x$ is prime'') that must be established in order for an assertion
requiring it to also be true.

A variable must have its type specified in a \texttt{\$f} statement before
it may be used in a \texttt{\$e}, \texttt{\$a}, or \texttt{\$p}
statement.  There may be only one (active) \texttt{\$f} statement for a
given variable.  (``Active'' is defined in Section~\ref{scoping}.)

In ordinary mathematics, theorems\index{theorem} are often expressed in the
form ``Assume $P$; then $Q$,'' where $Q$ is a statement that you can derive
if you start with statement $P$.\index{free variable}\footnote{A stronger
version of a theorem like this would be the {\em single} formula $P\rightarrow
Q$ ($P$ implies $Q$) from which the weaker version above follows by the rule
of modus ponens in logic.  We are not discussing this stronger form here.  In
the weaker form, we are saying only that if we can {\em prove} $P$, then we can
{\em prove} $Q$.  In a logician's language, if $x$ is the only free variable
in $P$ and $Q$, the stronger form is equivalent to $\forall x ( P \rightarrow
Q)$ (for all $x$, $P$ implies $Q$), whereas the weaker form is equivalent to
$\forall x P \rightarrow \forall x Q$. The stronger form implies the weaker,
but not vice-versa.  To be precise, the weaker form of the theorem is more
properly called an ``inference'' rather than a theorem.}\index{inference}
In the
Metamath\index{Metamath} language, you would express mathematical statement
$P$ as a hypothesis (a \texttt{\$e} Metamath language statement in this case) and
statement $Q$ as a provable assertion (a \texttt{\$p}\index{\texttt{\$p} statement}
statement).

Some examples of hypotheses you might encounter in logic and set theory are
\begin{center}
  \texttt{stmt1 \$f wff P \$.}\\
  \texttt{stmt2 \$f set x \$.}\\
  \texttt{stmt3 \$e |- ( P -> Q ) \$.}
\end{center}
\index{\texttt{\$e} statement}
\index{\texttt{\$f} statement}
Informally, these would be read, ``Let $P$ be a well-formed-formula,'' ``Let
$x$ be an (individual) variable,'' and ``Assume we have proved $P \rightarrow
Q$.''  The turnstile symbol \,$\vdash$\index{turnstile ({$\,\vdash$})} is
commonly used in logic texts to mean ``a proof exists for.''

To summarize:
\begin{itemize}
\item A \texttt{\$f} hypothesis tells Metamath the type or kind of its variable.
It is analogous to a variable declaration in a computer language that
tells the compiler that a variable is an integer or a floating-point
number.
\item The \texttt{\$e} hypothesis corresponds to what you would usually call a
``hypothesis'' in ordinary mathematics.
\end{itemize}

Before an assertion\index{assertion} (\texttt{\$a} or \texttt{\$p} statement) can be
referenced in a proof, all of its associated \texttt{\$f} and \texttt{\$e} hypotheses
(i.e.\ those \texttt{\$e} hypotheses that are active) must be satisfied (i.e.
established by the proof).  The meaning of ``associated'' (which we will call
{\bf mandatory} in Section~\ref{frames}) will become clear when we discuss
scoping later.

Note that after any \texttt{\$f}, \texttt{\$e},
\texttt{\$a}, or \texttt{\$p} token there is a required
\textit{typecode}\index{typecode}.
The typecode is a constant used to enforce types of expressions.
This will become clearer once we learn more about
assertions (\texttt{\$a} and \texttt{\$p} statements).
An example may also clarify their purpose.
In the
\texttt{set.mm}\index{set theory database (\texttt{set.mm})}%
\index{Metamath Proof Explorer}
database,
the following typecodes are used:

\begin{itemize}
\item \texttt{wff} :
  Well-formed formula (wff) symbol
  (read: ``the following symbol sequence is a wff'').
% The *textual* typecode for turnstile is "|-", but when read it's a little
% confusing, so I intentionally display the mathematical symbol here instead
% (I think it's clearer in this context).
\item \texttt{$\vdash$} :
  Turnstile (read: ``the following symbol sequence is provable'' or
  ``a proof exists for'').
\item \texttt{setvar} :
  Individual set variable type (read: ``the following is an
  individual set variable'').
  Note that this is \textit{not} the type of an arbitrary set expression,
  instead, it is used to ensure that there is only a single symbol used
  after quantifiers like for-all ($\forall$) and there-exists ($\exists$).
\item \texttt{class} :
  An expression that is a syntactically valid class expression.
  All valid set expressions are also valid class expression, so expressions
  of sets normally have the \texttt{class} typecode.
  Use the \texttt{class} typecode,
  \textit{not} the \texttt{setvar} typecode,
  for the type of set expressions unless you are specifically identifying
  a single set variable.
\end{itemize}

\subsection{Assertions (\texttt{\$a} and \texttt{\$p} Statements)}
\index{\texttt{\$a} statement}
\index{\texttt{\$p} statement}\index{assertion}\index{axiomatic assertion}
\index{provable assertion}

There are two types of assertions, \texttt{\$a}\index{\texttt{\$a} statement}
statements ({\bf axiomatic assertions}) and \texttt{\$p} statements ({\bf
provable assertions}).  Their syntax is as follows:
\begin{center}
  {\em label} \texttt{\$a} {\em typecode} {\em math-symbol} \ldots
         {\em math-symbol} \texttt{\$.}\\
  {\em label} \texttt{\$p} {\em typecode} {\em math-symbol} \ldots
        {\em math-symbol} \texttt{\$=} {\em proof} \texttt{\$.}
\end{center}
\index{\texttt{\$a} statement}
\index{\texttt{\$p} statement}
\index{\texttt{\$=} keyword}
An assertion always requires a {\em label}\index{label}. The expression in an
assertion consists of a typecode (an active constant)
followed by a sequence of zero
or more math symbols.  Each math symbol, including any {\em constant}, must be a
previously declared constant or variable.  (In addition, each math symbol
must be active, which will be covered when we discuss scoping statements in
Section~\ref{scoping}.)

A \texttt{\$a} statement is usually a definition of syntax (for example, if $P$
and $Q$ are wffs then so is $(P\to Q)$), an axiom\index{axiom} of ordinary
mathematics (for example, $x=x$), or a definition\index{definition} of
ordinary mathematics (for example, $x\ne y$ means $\lnot x=y$). A \texttt{\$p}
statement is a claim that a certain combination of math symbols follows from
previous assertions and is accompanied by a proof that demonstrates it.

Assertions can also be referenced in (later) proofs in order to derive new
assertions from them. The label of an assertion is used to refer to it in a
proof. Section~\ref{proof} will describe the proof in detail.

Assertions also provide the primary means for communicating the mathematical
results in the database to people.  Proofs (when conveniently displayed)
communicate to people how the results were arrived at.

\subsubsection{The \texttt{\$a} Statement}
\index{\texttt{\$a} statement}

Axiomatic assertions (\texttt{\$a} statements) represent the starting points from
which other assertions (\texttt{\$p}\index{\texttt{\$p} statement} statements) are
derived.  Their most obvious use is for specifying ordinary mathematical
axioms\index{axiom}, but they are also used for two other purposes.

First, Metamath\index{Metamath} needs to know the syntax of symbol
sequences that constitute valid mathematical statements.  A Metamath
proof must be broken down into much more detail than ordinary
mathematical proofs that you may be used to thinking of (even the
``complete'' proofs of formal logic\index{formal logic}).  This is one
of the things that makes Metamath a general-purpose language,
independent of any system of logic or even syntax.  If you want to use a
substitution instance of an assertion as a step in a proof, you must
first prove that the substitution is syntactically correct (or if you
prefer, you must ``construct'' it), showing for example that the
expression you are substituting for a wff metavariable is a valid wff.
The \texttt{\$a}\index{\texttt{\$a} statement} statement is used to
specify those combinations of symbols that are considered syntactically
valid, such as the legal forms of wffs.

Second, \texttt{\$a} statements are used to specify what are ordinarily thought of
as definitions, i.e.\ new combinations of symbols that abbreviate other
combinations of symbols.  Metamath makes no distinction\index{axiom vs.\
definition} between axioms\index{axiom} and definitions\index{definition}.
Indeed, it has been argued that such distinction should not be made even in
ordinary mathematics; see Section~\ref{definitions}, which discusses the
philosophy of definitions.  Section~\ref{hierarchy} discusses some
technical requirements for definitions.  In \texttt{set.mm} we adopt the
convention of prefixing axiom labels with \texttt{ax-} and definition labels with
\texttt{df-}\index{label}.

The results that can be derived with the Metamath language are only as good as
the \texttt{\$a}\index{\texttt{\$a} statement} statements used as their starting
point.  We cannot stress this too strongly.  For example, Metamath will
not prevent you from specifying $x\neq x$ as an axiom of logic.  It is
essential that you scrutinize all \texttt{\$a} statements with great care.
Because they are a source of potential pitfalls, it is best not to add new
ones (usually new definitions) casually; rather you should carefully evaluate
each one's necessity and advantages.

Once you have in place all of the basic axioms\index{axiom} and
rules\index{rule} of a mathematical theory, the only \texttt{\$a} statements that
you will be adding will be what are ordinarily called definitions.  In
principle, definitions should be in some sense eliminable from the language of
a theory according to some convention (usually involving logical equivalence
or equality).  The most common convention is that any formula that was
syntactically valid but not provable before the definition was introduced will
not become provable after the definition is introduced.  In an ideal world,
definitions should not be present at all if one is to have absolute confidence
in a mathematical result.  However, they are necessary to make
mathematics practical, for otherwise the resulting formulas would be
extremely long and incomprehensible.  Since the nature of definitions (in the
most general sense) does not permit them to automatically be verified as
``proper,''\index{proper definition}\index{definition!proper} the judgment of
the mathematician is required to ensure it.  (In \texttt{set.mm} effort was made
to make almost all definitions directly eliminable and thus minimize the need
for such judgment.)

If you are not a mathematician, it may be best not to add or change any
\texttt{\$a}\index{\texttt{\$a} statement} statements but instead use
the mathematical language already provided in standard databases.  This
way Metamath will not allow you to make a mistake (i.e.\ prove a false
result).


\subsection{Frames}\label{frames}

We now introduce the concept of a collection of related Metamath statements
called a frame.  Every assertion (\texttt{\$a} or \texttt{\$p} statement) in the database has
an associated frame.

A {\bf frame}\index{frame} is a sequence of \texttt{\$d}, \texttt{\$f},
and \texttt{\$e} statements (zero or more of each) followed by one
\texttt{\$a} or \texttt{\$p} statement, subject to certain conditions we
will describe.  For simplicity we will assume that all math symbol
tokens used are declared at the beginning of the database with
\texttt{\$c} and \texttt{\$v} statements (which are not properly part of
a frame).  Also for simplicity we will assume there are only simple
\texttt{\$d} statements (those with only two variables) and imagine any
compound \texttt{\$d} statements (those with more than two variables) as
broken up into simple ones.

A frame groups together those hypotheses (and \texttt{\$d} statements) relevant
to an assertion (\texttt{\$a} or \texttt{\$p} statement).  The statements in a frame
may or may not be physically adjacent in a database; we will cover
this in our discussion of scoping statements
in Section~\ref{scoping}.

A frame has the following properties:
\begin{enumerate}
 \item The set of variables contained in its \texttt{\$f} statements must
be identical to the set of variables contained in its \texttt{\$e},
\texttt{\$a}, and/or \texttt{\$p} statements.  In other words, each
variable in a \texttt{\$e}, \texttt{\$a}, or \texttt{\$p} statement must
have an associated ``variable type'' defined for it in a \texttt{\$f}
statement.
  \item No two \texttt{\$f} statements may contain the same variable.
  \item Any \texttt{\$f} statement
must occur before a \texttt{\$e} statement in which its variable occurs.
\end{enumerate}

The first property determines the set of variables occurring in a frame.
These are the {\bf mandatory
variables}\index{mandatory variable} of the frame.  The second property
tells us there must be only one type specified for a variable.
The last property is not a theoretical requirement but it
makes parsing of the database easier.

For our examples, we assume our database has the following declarations:

\begin{verbatim}
$v P Q R $.
$c -> ( ) |- wff $.
\end{verbatim}

The following sequence of statements, describing the modus ponens inference
rule, is an example of a frame:

\begin{verbatim}
wp  $f wff P $.
wq  $f wff Q $.
maj $e |- ( P -> Q ) $.
min $e |- P $.
mp  $a |- Q $.
\end{verbatim}

The following sequence of statements is not a frame because \texttt{R} does not
occur in the \texttt{\$e}'s or the \texttt{\$a}:

\begin{verbatim}
wp  $f wff P $.
wq  $f wff Q $.
wr  $f wff R $.
maj $e |- ( P -> Q ) $.
min $e |- P $.
mp  $a |- Q $.
\end{verbatim}

The following sequence of statements is not a frame because \texttt{Q} does not
occur in a \texttt{\$f}:

\begin{verbatim}
wp  $f wff P $.
maj $e |- ( P -> Q ) $.
min $e |- P $.
mp  $a |- Q $.
\end{verbatim}

The following sequence of statements is not a frame because the \texttt{\$a} statement is
not the last one:

\begin{verbatim}
wp  $f wff P $.
wq  $f wff Q $.
maj $e |- ( P -> Q ) $.
mp  $a |- Q $.
min $e |- P $.
\end{verbatim}

Associated with a frame is a sequence of {\bf mandatory
hypotheses}\index{mandatory hypothesis}.  This is simply the set of all
\texttt{\$f} and \texttt{\$e} statements in the frame, in the order they
appear.  A frame can be referenced in a later proof using the label of
the \texttt{\$a} or \texttt{\$p} assertion statement, and the proof
makes an assignment to each mandatory hypothesis in the order in which
it appears.  This means the order of the hypotheses, once chosen, must
not be changed so as not to affect later proofs referencing the frame's
assertion statement.  (The Metamath proof verifier will, of course, flag
an error if a proof becomes incorrect by doing this.)  Since proofs make
use of ``Reverse Polish notation,'' described in Section~\ref{proof}, we
call this order the {\bf RPN order}\index{RPN order} of the hypotheses.

Note that \texttt{\$d} statements are not part of the set of mandatory
hypotheses, and their order doesn't matter (as long as they satisfy the
fourth property for a frame described above).  The \texttt{\$d}
statements specify restrictions on variables that must be satisfied (and
are checked by the proof verifier) when expressions are substituted for
them in a proof, and the \texttt{\$d} statements themselves are never
referenced directly in a proof.

A frame with a \texttt{\$p} (provable) statement requires a proof as part of the
\texttt{\$p} statement.  Sometimes in a proof we want to make use of temporary or
dummy variables\index{dummy variable} that do not occur in the \texttt{\$p}
statement or its mandatory hypotheses.  To accommodate this we define an {\bf
extended frame}\index{extended frame} as a frame together with zero or more
\texttt{\$d} and \texttt{\$f} statements that reference variables not among the
mandatory variables of the frame.  Any new variables referenced are called the
{\bf optional variables}\index{optional variable} of the extended frame. If a
\texttt{\$f} statement references an optional variable it is called an {\bf
optional hypothesis}\index{optional hypothesis}, and if one or both of the
variables in a \texttt{\$d} statement are optional variables it is called an {\bf
optional disjoint-variable restriction}\index{optional disjoint-variable
restriction}.  Properties 2, 3, and 4 for a frame also apply to an extended
frame.

The concept of optional variables is not meaningful for frames with \texttt{\$a}
statements, since those statements have no proofs that might make use of them.
There is no restriction on including optional hypotheses in the extended frame
for a \texttt{\$a} statement, but they serve no purpose.

The following set of statements is an example of an extended frame, which
contains an optional variable \texttt{R} and an optional hypothesis \texttt{wr}.  In
this example, we suppose the rule of modus ponens is not an axiom but is
derived as a theorem from earlier statements (we omit its presumed proof).
Variable \texttt{R} may be used in its proof if desired (although this would
probably have no advantage in propositional calculus).  Note that the sequence
of mandatory hypotheses in RPN order is still \texttt{wp}, \texttt{wq}, \texttt{maj},
\texttt{min} (i.e.\ \texttt{wr} is omitted), and this sequence is still assumed
whenever the assertion \texttt{mp} is referenced in a subsequent proof.

\begin{verbatim}
wp  $f wff P $.
wq  $f wff Q $.
wr  $f wff R $.
maj $e |- ( P -> Q ) $.
min $e |- P $.
mp  $p |- Q $= ... $.
\end{verbatim}

Every frame is an extended frame, but not every extended frame is a frame, as
this example shows.  The underlying frame for an extended frame is
obtained by simply removing all statements containing optional variables.
Any proof referencing an assertion will ignore any extensions to its
frame, which means we may add or delete optional hypotheses at will without
affecting subsequent proofs.

The conceptually simplest way of organizing a Metamath database is as a
sequence of extended frames.  The scoping statements
\texttt{\$\char`\{}\index{\texttt{\$\char`\{} and \texttt{\$\char`\}}
keywords} and \texttt{\$\char`\}} can be used to delimit the start and
end of an extended frame, leading to the following possible structure for a
database.  \label{framelist}

\vskip 2ex
\setbox\startprefix=\hbox{\tt \ \ \ \ \ \ \ \ }
\setbox\contprefix=\hbox{}
\startm
\m{\mbox{(\texttt{\$v} {\em and} \texttt{\$c}\,{\em statements})}}
\endm
\startm
\m{\mbox{\texttt{\$\char`\{}}}
\endm
\startm
\m{\mbox{\texttt{\ \ } {\em extended frame}}}
\endm
\startm
\m{\mbox{\texttt{\$\char`\}}}}
\endm
\startm
\m{\mbox{\texttt{\$\char`\{}}}
\endm
\startm
\m{\mbox{\texttt{\ \ } {\em extended frame}}}
\endm
\startm
\m{\mbox{\texttt{\$\char`\}}}}
\endm
\startm
\m{\mbox{\texttt{\ \ \ \ \ \ \ \ \ }}\vdots}
\endm
\vskip 2ex

In practice, this structure is inconvenient because we have to repeat
any \texttt{\$f}, \texttt{\$e}, and \texttt{\$d} statements over and
over again rather than stating them once for use by several assertions.
The scoping statements, which we will discuss next, allow this to be
done.  In principle, any Metamath database can be converted to the above
format, and the above format is the most convenient to use when studying
a Metamath database as a formal system%
%% Uncomment this when uncommenting section {formalspec} below
   (Appendix \ref{formalspec})%
.
In fact, Metamath internally converts the database to the above format.
The command \texttt{show statement} in the Metamath program will show
you the contents of the frame for any \texttt{\$a} or \texttt{\$p}
statement, as well as its extension in the case of a \texttt{\$p}
statement.

%c%(provided that all ``local'' variables and constants with limited scope have
%c%unique names),

During our discussion of scoping statements, it may be helpful to
think in terms of the equivalent sequence of frames that will result when
the database is parsed.  Scoping (other than the limited
use above to delimit frames) is not a theoretical requirement for
Metamath but makes it more convenient.


\subsection{Scoping Statements (\texttt{\$\{} and \texttt{\$\}})}\label{scoping}
\index{\texttt{\$\char`\{} and \texttt{\$\char`\}} keywords}\index{scoping statement}

%c%Some Metamath statements may be needed only temporarily to
%c%serve a specific purpose, and after we're done with them we would like to
%c%disregard or ignore them.  For example, when we're finished using a variable,
%c%we might want to
%c%we might want to free up the token\index{token} used to name it so that the
%c%token can be used for other purposes later on, such as a different kind of
%c%variable or even a constant.  In the terminology of computer programming, we
%c%might want to let some symbol declarations be ``local'' rather than ``global.''
%c%\index{local symbol}\index{global symbol}

The {\bf scoping} statements, \texttt{\$\char`\{} ({\bf start of block}) and \texttt{\$\char`\}}
({\bf end of block})\index{block}, provide a means for controlling the portion
of a database over which certain statement types are recognized.  The
syntax of a scoping statement is very simple; it just consists of the
statement's keyword:
\begin{center}
\texttt{\$\char`\{}\\
\texttt{\$\char`\}}
\end{center}
\index{\texttt{\$\char`\{} and \texttt{\$\char`\}} keywords}

For example, consider the following database where we have stripped out
all tokens except the scoping statement keywords.  For the purpose of the
discussion, we have added subscripts to the scoping statements; these subscripts
do not appear in the actual database.
\[
 \mbox{\tt \ \$\char`\{}_1
 \mbox{\tt \ \$\char`\{}_2
 \mbox{\tt \ \$\char`\}}_2
 \mbox{\tt \ \$\char`\{}_3
 \mbox{\tt \ \$\char`\{}_4
 \mbox{\tt \ \$\char`\}}_4
 \mbox{\tt \ \$\char`\}}_3
 \mbox{\tt \ \$\char`\}}_1
\]
Each \texttt{\$\char`\{} statement in this example is said to be {\bf
matched} with the \texttt{\$\char`\}} statement that has the same
subscript.  Each pair of matched scoping statements defines a region of
the database called a {\bf block}.\index{block} Blocks can be {\bf
nested}\index{nested block} inside other blocks; in the example, the
block defined by $\mbox{\tt \$\char`\{}_4$ and $\mbox{\tt \$\char`\}}_4$
is nested inside the block defined by $\mbox{\tt \$\char`\{}_3$ and
$\mbox{\tt \$\char`\}}_3$ as well as inside the block defined by
$\mbox{\tt \$\char`\{}_1$ and $\mbox{\tt \$\char`\}}_1$.  In general, a
block may be empty, it may contain only non-scoping
statements,\footnote{Those statements other than \texttt{\$\char`\{} and
\texttt{\$\char`\}}.}\index{non-scoping statement} or it may contain any
mixture of other blocks and non-scoping statements.  (This is called a
``recursive'' definition\index{recursive definition} of a block.)

Associated with each block is a number called its {\bf nesting
level}\index{nesting level} that indicates how deeply the block is nested.
The nesting levels of the blocks in our example are as follows:
\[
  \underbrace{
    \mbox{\tt \ }
    \underbrace{
     \mbox{\tt \$\char`\{\ }
     \underbrace{
       \mbox{\tt \$\char`\{\ }
       \mbox{\tt \$\char`\}}
     }_{2}
     \mbox{\tt \ }
     \underbrace{
       \mbox{\tt \$\char`\{\ }
       \underbrace{
         \mbox{\tt \$\char`\{\ }
         \mbox{\tt \$\char`\}}
       }_{3}
       \mbox{\tt \ \$\char`\}}
     }_{2}
     \mbox{\tt \ \$\char`\}}
   }_{1}
   \mbox{\tt \ }
 }_{0}
\]
\index{\texttt{\$\char`\{} and \texttt{\$\char`\}} keywords}
The entire database is considered to be one big block (the {\bf outermost}
block) with a nesting level of 0.  The outermost block is {\em not} bracketed
by scoping statements.\footnote{The language was designed this way so that
several source files can be joined together more easily.}\index{outermost
block}

All non-scoping Metamath statements become recognized or {\bf
active}\index{active statement} at the place where they appear.\footnote{To
keep things slightly simpler, we do not bother to define the concept of
``active'' for the scoping statements.}  Certain of these statement types
become inactive at the end of the block in which they appear; these statement
types are:
\begin{center}
  \texttt{\$c}, \texttt{\$v}, \texttt{\$d}, \texttt{\$e}, and \texttt{\$f}.
%  \texttt{\$v}, \texttt{\$f}, \texttt{\$e}, and \texttt{\$d}.
\end{center}
\index{\texttt{\$c} statement}
\index{\texttt{\$d} statement}
\index{\texttt{\$e} statement}
\index{\texttt{\$f} statement}
\index{\texttt{\$v} statement}
The other statement types remain active forever (i.e.\ through the end of the
database); they are:
\begin{center}
  \texttt{\$a} and \texttt{\$p}.
%  \texttt{\$c}, \texttt{\$a}, and \texttt{\$p}.
\end{center}
\index{\texttt{\$a} statement}
\index{\texttt{\$p} statement}
Any statement (of these 7 types) located in the outermost
block\index{outermost block} will remain active through the end of the
database and thus are effectively ``global'' statements.\index{global
statement}

All \texttt{\$c} statements must be placed in the outermost block.  Since they are
therefore always global, they could be considered as belonging to both of the
above categories.

The {\bf scope}\index{scope} of a statement is the set of statements that
recognize it as active.

%c%The concept of ``active'' is also defined for math symbols\index{math
%c%symbol}.  Math symbols (constants\index{constant} and
%c%variables\index{variable}) become {\bf active}\index{active
%c%math symbol} in the \texttt{\$c}\index{\texttt{\$c}
%c%statement} and \texttt{\$v}\index{\texttt{\$v} statement} statements that
%c%declare them.  They become inactive when their declaration statements become
%c%inactive.

The concept of ``active'' is also defined for math symbols\index{math
symbol}.  Math symbols (constants\index{constant} and
variables\index{variable}) become {\bf active}\index{active math symbol}
in the \texttt{\$c}\index{\texttt{\$c} statement} and
\texttt{\$v}\index{\texttt{\$v} statement} statements that declare them.
A variable becomes inactive when its declaration statement becomes
inactive.  Because all \texttt{\$c} statements must be in the outermost
block, a constant will never become inactive after it is declared.

\subsubsection{Redeclaration of Math Symbols}
\index{redeclaration of symbols}\label{redeclaration}

%c%A math symbol may not be declared a second time while it is active, but it may
%c%be declared again after it becomes inactive.

A variable may not be declared a second time while it is active, but it may be
declared again after it becomes inactive.  This provides a convenient way to
introduce ``local'' variables,\index{local variable} i.e.\ temporary variables
for use in the frame of an assertion or in a proof without keeping them around
forever.  A previously declared variable may not be redeclared as a constant.

A constant may not be redeclared.  And, as mentioned above, constants must be
declared in the outermost block.

The reason variables may have limited scope but not constants is that an
assertion (\texttt{\$a} or \texttt{\$p} statement) remains available for use in
proofs through the end of the database.  Variables in an assertion's frame may
be substituted with whatever is needed in a proof step that references the
assertion, whereas constants remain fixed and may not be substituted with
anything.  The particular token used for a variable in an assertion's frame is
irrelevant when the assertion is referenced in a proof, and it doesn't matter
if that token is not available outside of the referenced assertion's frame.
Constants, however, must be globally fixed.

In the present version of the Metamath language, there is no theoretical
benefit for the feature allowing variables to be active for limited scopes
rather than global. It is just a convenience that allows them, for example, to
be locally grouped together with their corresponding \texttt{\$f} variable-type
declarations.\footnote{A future version\label{redeclarationf} of the Metamath
language may extend the language to include true local variables as well as
local constants.  I have not decided upon the best way to do this or even if
it should be done at all; the main issue is whether the benefits offset
greater confusion in learning the language.  In any case, the rules for
the present language were chosen to let existing databases be compatible with
any future language extensions.}

%c%If you declare a math symbol more than once, internally Metamath considers it a
%c%new distinct symbol, even though it has the same name.  If you are unaware of
%c%this, you may find that what you think are correct proofs are incorrectly
%c%rejected as invalid, because Metamath may tell you that a constant you
%c%previously declared does not match a newly declared math symbol with the same
%c%name.  For details on this subtle point, see the Comment on
%c%p.~\pageref{spec4comment}.  This is done purposely to allow temporary
%c%constants to be introduced while developing a subtheory, then allow their math
%c%symbol tokens to be reused later on; in general they will not refer to the
%c%same thing.  In practice, you would not ordinarily reuse the names of
%c%constants because it would tend to be confusing to the reader.  The reuse of
%c%names of variables, on the other hand, is something that is often useful to do
%c%(for example it is done frequently in \texttt{set.mm}).  Since variables in an
%c%assertion referenced in a proof can be substituted as needed to achieve a
%c%symbol match, this is not an issue.

% (This section covers a somewhat advanced topic you may want to skip
% at first reading.)
%
% Under certain circumstances, math symbol\index{math symbol}
% tokens\index{token} may be redeclared (i.e.\ the token
% may appear in more than
% one \texttt{\$c}\index{\texttt{\$c} statement} or \texttt{\$v}\index{\texttt{\$v}
% statement} statement).  You might want to do this say, to make temporary use
% of a variable name without having to worry about its affect elsewhere,
% somewhat analogous to declaring a local variable in a standard computer
% language.  Understanding what goes on when math symbol tokens are redeclared
% is a little tricky to understand at first, since it requires that we
% distinguish the token itself from the math symbol that it names.  It will help
% if we first take a peek at the internal workings of the
% Metamath\index{Metamath} program.
%
% Metamath reserves a memory location for each occurrence of a
% token\index{token} in a declaration statement (\texttt{\$c}\index{\texttt{\$c}
% statement} or \texttt{\$v}\index{\texttt{\$v} statement}).  If a given token appears
% in more than one declaration statement, it will refer to more than one memory
% locations.  A math symbol\index{math symbol} may be thought of as being one of
% these memory locations rather than as the token itself.  Only one of the
% memory locations associated with a given token may be active at any one time.
% The math symbol (memory location) that gets looked up when the token appears
% in a non-declaration statement is the one that happens to be active at that
% time.
%
% We now look at the rules for the redeclaration\index{redeclaration of symbols}
% of math symbol tokens.
% \begin{itemize}
% \item A math symbol token may not be declared twice in the
% same block.\footnote{While there is no theoretical reason for disallowing
% this, it was decided in the design of Metamath that allowing it would offer no
% advantage and might cause confusion.}
% \item An inactive math symbol may always be
% redeclared.
% \item  An active math symbol may be redeclared in a different (i.e.\
% inner) block\index{block} from the one it became active in.
% \end{itemize}
%
% When a math symbol token is redeclared, it conceptually refers to a different
% math symbol, just as it would be if it were called a different name.  In
% addition, the original math symbol that it referred to, if it was active,
% temporarily becomes inactive.  At the end of the block in which the
% redeclaration occurred, the new math symbol\index{math symbol} becomes
% inactive and the original symbol becomes active again.  This concept is
% illustrated in the following example, where the symbol \texttt{e} is
% ordinarily a constant (say Euler's constant, 2.71828...) but
% temporarily we want to use it as a ``local'' variable, say as a coefficient
% in the equation $a x^4 + b x^3 + c x^2 + d x + e$:
% \[
%   \mbox{\tt \$\char`\{\ \$c e \$.}
%   \underbrace{
%     \ \ldots\ %
%     \mbox{\tt \$\char`\{}\ \ldots\ %
%   }_{\mbox{\rm region A}}
%   \mbox{\tt \$v e \$.}
%   \underbrace{
%     \mbox{\ \ \ \ldots\ \ \ }
%   }_{\mbox{\rm region B}}
%   \mbox{\tt \$\char`\}}
%   \underbrace{
%     \mbox{\ \ \ \ldots\ \ \ }
%   }_{\mbox{\rm region C}}
%   \mbox{\tt \$\char`\}}
% \]
% \index{\texttt{\$\char`\{} and \texttt{\$\char`\}} keywords}
% In region A, the token \texttt{e} refers to a constant.  It is redeclared as a
% variable in region B, and any reference to it in this region will refer to this
% variable.  In region C, the redeclaration becomes inactive, and the original
% declaration becomes active again.  In region C, the token \texttt{x} refers to the
% original constant.
%
% As a practical matter, overuse of math symbol\index{math symbol}
% redeclarations\index{redeclaration of symbols} can be confusing (even though
% it is well-defined) and is best avoided when possible.  Here are some good
% general guidelines you can follow.  Usually, you should declare all
% constants\index{constant} in the outermost block\index{outermost block},
% especially if they are general-purpose (such as the token \verb$A.$, meaning
% $\forall$ or ``for all'').  This will make them ``globally'' active (although
% as in the example above local redeclarations will temporarily make them
% inactive.)  Most or all variables\index{variable}, on the other hand, could be
% declared in inner blocks, so that the token for them can be used later for a
% different type of variable or a constant.  (The names of the variables you
% choose are not used when you refer to an assertion\index{assertion} in a
% proof, whereas constants must match exactly.  A locally declared constant will
% not match a globally declared constant in a proof, even if they use the same
% token, because Metamath internally considers them to be different math
% symbols.)  To avoid confusion, you should generally avoid redeclaring active
% variables.  If you must redeclare them, do so at the beginning of a block.
% The temporary declaration of constants in inner blocks might be occasionally
% appropriate when you make use of a temporary definition to prove lemmas
% leading to a main result that does not make direct use of the definition.
% This way, you will not clutter up your database with a large number of
% seldom-used global constant symbols.  You might want to note that while
% inactive constants may not appear directly in an assertion (a \texttt{\$a}\index{\texttt{\$a}
% statement} or \texttt{\$p}\index{\texttt{\$p} statement}
% statement), they may be indirectly used in the proof of a \texttt{\$p} statement
% so long as they do not appear in the final math symbol sequence constructed by
% the proof.  In the end, you will have to use your best judgment, taking into
% account standard mathematical usage of the symbols as well as consideration
% for the reader of your work.
%
% \subsubsection{Reuse of Labels}\index{reuse of labels}\index{label}
%
% The \texttt{\$e}\index{\texttt{\$e} statement}, \texttt{\$f}\index{\texttt{\$f}
% statement}, \texttt{\$a}\index{\texttt{\$a} statement}, and
% \texttt{\$p}\index{\texttt{\$p}
% statement} statement types require labels, which allow them to be
% referenced later inside proofs.  A label is considered {\bf
% active}\index{active label} when the statement it is associated with is
% active.  The token\index{token} for a label may be reused
% (redeclared)\index{redeclaration of labels} provided that it is not being used
% for a currently active label.  (Unlike the tokens for math symbols, active
% label tokens may not be redeclared in an inner scope.)  Note that the labels
% of \texttt{\$a} and \texttt{\$p} statements can never be reused after these
% statements appear, because these statements remain active through the end of
% the database.
%
% You might find the reuse of labels a convenient way to have standard names for
% temporary hypotheses, such as \texttt{h1}, \texttt{h2}, etc.  This way you don't have
% to invent unique names for each of them, and in some cases it may be less
% confusing to the reader (although in other cases it might be more confusing, if
% the hypothesis is located far away from the assertion that uses
% it).\footnote{The current implementation requires that all labels, even
% inactive ones, be unique.}

\subsubsection{Frames Revisited}\index{frames and scoping statements}

Now that we have covered scoping, we will look at how an arbitrary
Metamath database can be converted to the simple sequence of extended
frames described on p.~\pageref{framelist}.  This is also how Metamath
stores the database internally when it reads in the database
source.\label{frameconvert} The method is simple.  First, we collect all
constant and variable (\texttt{\$c} and \texttt{\$v}) declarations in
the database, ignoring duplicate declarations of the same variable in
different scopes.  We then put our collected \texttt{\$c} and
\texttt{\$v} declarations at the beginning of the database, so that
their scope is the entire database.  Next, for each assertion in the
database, we determine its frame and extended frame.  The extended frame
is simply the \texttt{\$f}, \texttt{\$e}, and \texttt{\$d} statements
that are active.  The frame is the extended frame with all optional
hypotheses removed.

An equivalent way of saying this is that the extended frame of an assertion
is the collection of all \texttt{\$f}, \texttt{\$e}, and \texttt{\$d} statements
whose scope includes the assertion.
The \texttt{\$f} and \texttt{\$e} statements
occur in the order they appear
(order is irrelevant for \texttt{\$d} statements).

%c%, renaming any
%c%redeclared variables as needed so that all of them have unique names.  (The
%c%exact renaming convention is unimportant.  You might imagine renaming
%c%different declarations of math symbol \texttt{a} as \texttt{a\$1}, \texttt{a\$2}, etc.\
%c%which would prevent any conflicts since \texttt{\$} is not a legal character in a
%c%math symbol token.)

\section{The Anatomy of a Proof} \label{proof}
\index{proof!Metamath, description of}

Each provable assertion (\texttt{\$p}\index{\texttt{\$p} statement} statement) in a
database must include a {\bf proof}\index{proof}.  The proof is located
between the \texttt{\$=}\index{\texttt{\$=} keyword} and \texttt{\$.}\ keywords in the
\texttt{\$p} statement.

In the basic Metamath language\index{basic language}, a proof is a
sequence of statement labels.  This label sequence\index{label sequence}
serves as a set of instructions that the Metamath program uses to
construct a series of math symbol sequences.  The construction must
ultimately result in the math symbol sequence contained between the
\texttt{\$p}\index{\texttt{\$p} statement} and
\texttt{\$=}\index{\texttt{\$=} keyword} keywords of the \texttt{\$p}
statement.  Otherwise, the Metamath program will consider the proof
incorrect, and it will notify you with an appropriate error message when
you ask it to verify the proof.\footnote{To make the loading faster, the
Metamath program does not automatically verify proofs when you
\texttt{read} in a database unless you use the \texttt{/verify}
qualifier.  After a database has been read in, you may use the
\texttt{verify proof *} command to verify proofs.}\index{\texttt{verify
proof} command} Each label in a proof is said to {\bf
reference}\index{label reference} its corresponding statement.

Associated with any assertion\index{assertion} (\texttt{\$p} or
\texttt{\$a}\index{\texttt{\$a} statement} statement) is a set of
hypotheses (\texttt{\$f}\index{\texttt{\$f} statement} or
\texttt{\$e}\index{\texttt{\$e} statement} statements) that are active
with respect to that assertion.  Some are mandatory and the others are
optional.  You should review these concepts if necessary.

Each label\index{label} in a proof must be either the label of a
previous assertion (\texttt{\$a}\index{\texttt{\$a} statement} or
\texttt{\$p}\index{\texttt{\$p} statement} statement) or the label of an
active hypothesis (\texttt{\$e} or \texttt{\$f}\index{\texttt{\$f}
statement} statement) of the \texttt{\$p} statement containing the
proof.  Hypothesis labels may reference both the
mandatory\index{mandatory hypothesis} and the optional hypotheses of the
\texttt{\$p} statement.

The label sequence in a proof specifies a construction in {\bf reverse Polish
notation}\index{reverse Polish notation (RPN)} (RPN).  You may be familiar
with RPN if you have used older
Hewlett--Packard or similar hand-held calculators.
In the calculator analogy, a hypothesis label\index{hypothesis label} is like
a number and an assertion label\index{assertion label} is like an operation
(more precisely, an $n$-ary operation when the
assertion has $n$ \texttt{\$e}-hypotheses).
On an RPN calculator, an operation takes one or more previous numbers in an
input sequence, performs a calculation on them, and replaces those numbers and
itself with the result of the calculation.  For example, the input sequence
$2,3,+$ on an RPN calculator results in $5$, and the input sequence
$2,3,5,{\times},+$ results in $2,15,+$ which results in $17$.

Understanding how RPN is processed involves the concept of a {\bf
stack}\index{stack}\index{RPN stack}, which can be thought of as a set of
temporary memory locations that hold intermediate results.  When Metamath
encounters a hypothesis label it places or {\bf pushes}\index{push} the math
symbol sequence of the hypothesis onto the stack.  When Metamath encounters an
assertion label, it associates the most recent stack entries with the {\em
mandatory} hypotheses\index{mandatory hypothesis} of the assertion, in the
order where the most recent stack entry is associated with the last mandatory
hypothesis of the assertion.  It then determines what
substitutions\index{substitution!variable}\index{variable substitution} have
to be made into the variables of the assertion's mandatory hypotheses to make
them identical to the associated stack entries.  It then makes those same
substitutions into the assertion itself.  Finally, Metamath removes or {\bf
pops}\index{pop} the matched hypotheses from the stack and pushes the
substituted assertion onto the stack.

For the purpose of matching the mandatory hypothesis to the most recent stack
entries, whether a hypothesis is a \texttt{\$e} or \texttt{\$f} statement is
irrelevant.  The only important thing is that a set of
substitutions\footnote{In the Metamath spec (Section~\ref{spec}), we use the
singular term ``substitution'' to refer to the set of substitutions we talk
about here.} exist that allow a match (and if they don't, the proof verifier
will let you know with an error message).  The Metamath language is specified
in such a way that if a set of substitutions exists, it will be unique.
Specifically, the requirement that each variable have a type specified for it
with a \texttt{\$f} statement ensures the uniqueness.

We will illustrate this with an example.
Consider the following Metamath source file:
\begin{verbatim}
$c ( ) -> wff $.
$v p q r s $.
wp $f wff p $.
wq $f wff q $.
wr $f wff r $.
ws $f wff s $.
w2 $a wff ( p -> q ) $.
wnew $p wff ( s -> ( r -> p ) ) $= ws wr wp w2 w2 $.
\end{verbatim}
This Metamath source example shows the definition and ``proof'' (i.e.,
construction) of a well-formed formula (wff)\index{well-formed formula (wff)}
in propositional calculus.  (You may wish to type this example into a file to
experiment with the Metamath program.)  The first two statements declare
(introduce the names of) four constants and four variables.  The next four
statements specify the variable types, namely that
each variable is assumed to be a wff.  Statement \texttt{w2} defines (postulates)
a way to produce a new wff, \texttt{( p -> q )}, from two given wffs \texttt{p} and
\texttt{q}. The mandatory hypotheses of \texttt{w2} are \texttt{wp} and \texttt{wq}.
Statement \texttt{wnew} claims that \texttt{( s -> ( r -> p ) )} is a wff given
three wffs \texttt{s}, \texttt{r}, and \texttt{p}.  More precisely, \texttt{wnew} claims
that the sequence of ten symbols \texttt{wff ( s -> ( r -> p ) )} is provable from
previous assertions and the hypotheses of \texttt{wnew}.  Metamath does not know
or care what a wff is, and as far as it is concerned
the typecode \texttt{wff} is just an
arbitrary constant symbol in a math symbol sequence.  The mandatory hypotheses
of \texttt{wnew} are \texttt{wp}, \texttt{wr}, and \texttt{ws}; \texttt{wq} is an optional
hypothesis.  In our particular proof, the optional hypothesis is not
referenced, but in general, any combination of active (i.e.\ optional and
mandatory) hypotheses could be referenced.  The proof of statement \texttt{wnew}
is the sequence of five labels starting with \texttt{ws} (step~1) and ending with
\texttt{w2} (step~5).

When Metamath verifies the proof, it scans the proof from left to right.  We
will examine what happens at each step of the proof.  The stack starts off
empty.  At step 1, Metamath looks up label \texttt{ws} and determines that it is a
hypothesis, so it pushes the symbol sequence of statement \texttt{ws} onto the
stack:

\begin{center}\begin{tabular}{|l|l|}\hline
{Stack location} & {Contents} \\ \hline \hline
1 & \texttt{wff s} \\ \hline
\end{tabular}\end{center}

Metamath sees that the labels \texttt{wr} and \texttt{wp} in steps~2 and 3 are also
hypotheses, so it pushes them onto the stack.  After step~3, the stack looks
like
this:

\begin{center}\begin{tabular}{|l|l|}\hline
{Stack location} & {Contents} \\ \hline \hline
3 & \texttt{wff p} \\ \hline
2 & \texttt{wff r} \\ \hline
1 & \texttt{wff s} \\ \hline
\end{tabular}\end{center}

At step 4, Metamath sees that label \texttt{w2} is an assertion, so it must do
some processing.  First, it associates the mandatory hypotheses of \texttt{w2},
which are \texttt{wp} and \texttt{wq}, with stack locations~2 and 3, {\em in that
order}. Metamath determines that the only possible way
to make hypothesis \texttt{wp} match (become identical to) stack location~2 and
\texttt{wq} match stack location 3 is to substitute variable \texttt{p} with \texttt{r}
and \texttt{q} with \texttt{p}.  Metamath makes these substitutions into \texttt{w2} and
obtains the symbol sequence \texttt{wff ( r -> p )}.  It removes the hypotheses
from stack locations~2 and 3, then places the result into stack location~2:

\begin{center}\begin{tabular}{|l|l|}\hline
{Stack location} & {Contents} \\ \hline \hline
2 & \texttt{wff ( r -> p )} \\ \hline
1 & \texttt{wff s} \\ \hline
\end{tabular}\end{center}

At step 5, Metamath sees that label \texttt{w2} is an assertion, so it must again
do some processing.  First, it matches the mandatory hypotheses of \texttt{w2},
which are \texttt{wp} and \texttt{wq}, to stack locations 1 and 2.
Metamath determines that the only possible way to make the
hypotheses match is to substitute variable \texttt{p} with \texttt{s} and \texttt{q} with
\texttt{( r -> p )}.  Metamath makes these substitutions into \texttt{w2} and obtains
the symbol
sequence \texttt{wff ( s -> ( r -> p ) )}.  It removes stack
locations 1 and 2, then places the result into stack location~1:

\begin{center}\begin{tabular}{|l|l|}\hline
{Stack location} & {Contents} \\ \hline \hline
1 & \texttt{wff ( s -> ( r -> p ) )} \\ \hline
\end{tabular}\end{center}

After Metamath finishes processing the proof, it checks to see that the
stack contains exactly one element and that this element is
the same as the math symbol sequence in the
\texttt{\$p}\index{\texttt{\$p} statement} statement.  This is the case for our
proof of \texttt{wnew},
so we have proved \texttt{wnew} successfully.  If the result
differs, Metamath will notify you with an error message.  An error message
will also result if the stack contains more than one entry at the end of the
proof, or if the stack did not contain enough entries at any point in the
proof to match all of the mandatory hypotheses\index{mandatory hypothesis} of
an assertion.  Finally, Metamath will notify you with an error message if no
substitution is possible that will make a referenced assertion's hypothesis
match the
stack entries.  You may want to experiment with the different kinds of errors
that Metamath will detect by making some small changes in the proof of our
example.

Metamath's proof notation was designed primarily to express proofs in a
relatively compact manner, not for readability by humans.  Metamath can display
proofs in a number of different ways with the \texttt{show proof}\index{\texttt{show
proof} command} command.  The
\texttt{/lemmon} qualifier displays it in a format that is easier to read when the
proofs are short, and you saw examples of its use in Chapter~\ref{using}.  For
longer proofs, it is useful to see the tree structure of the proof.  A tree
structure is displayed when the \texttt{/lemmon} qualifier is omitted.  You will
probably find this display more convenient as you get used to it. The tree
display of the proof in our example looks like
this:\label{treeproof}\index{tree-style proof}\index{proof!tree-style}
\begin{verbatim}
1     wp=ws    $f wff s
2        wp=wr    $f wff r
3        wq=wp    $f wff p
4     wq=w2    $a wff ( r -> p )
5  wnew=w2  $a wff ( s -> ( r -> p ) )
\end{verbatim}
The number to the left of each line is the step number.  Following it is a
{\bf hypothesis association}\index{hypothesis association}, consisting of two
labels\index{label} separated by \texttt{=}.  To the left of the \texttt{=} (except
in the last step) is the label of a hypothesis of an assertion referenced
later in the proof; here, steps 1 and 4 are the hypothesis associations for
the assertion \texttt{w2} that is referenced in step 5.  A hypothesis association
is indented one level more than the assertion that uses it, so it is easy to
find the corresponding assertion by moving directly down until the indentation
level decreases to one less than where you started from.  To the right of each
\texttt{=} is the proof step label for that proof step.  The statement keyword of
the proof step label is listed next, followed by the content of the top of the
stack (the most recent stack entry) as it exists after that proof step is
processed.  With a little practice, you should have no trouble reading proofs
displayed in this format.

Metamath proofs include the syntax construction of a formula.
In standard mathematics, this kind of
construction is not considered a proper part of the proof at all, and it
certainly becomes rather boring after a while.
Therefore,
by default the \texttt{show proof}\index{\texttt{show proof}
command} command does not show the syntax construction.
Historically \texttt{show proof} command
\textit{did} show the syntax construction, and you needed to add the
\texttt{/essential} option to hide, them, but today
\texttt{/essential} is the default and you need to use
\texttt{/all} to see the syntax constructions.

When verifying a proof, Metamath will check that no mandatory
\texttt{\$d}\index{\texttt{\$d} statement}\index{mandatory \texttt{\$d}
statement} statement of an assertion referenced in a proof is violated
when substitutions\index{substitution!variable}\index{variable
substitution} are made to the variables in the assertion.  For details
see Section~\ref{spec4} or \ref{dollard}.

\subsection{The Concept of Unification} \label{unify}

During the course of verifying a proof, when Metamath\index{Metamath}
encounters an assertion label\index{assertion label}, it associates the
mandatory hypotheses\index{mandatory hypothesis} of the assertion with the top
entries of the RPN stack\index{stack}\index{RPN stack}.  Metamath then
determines what substitutions\index{substitution!variable}\index{variable
substitution} it must make to the variables in the assertion's mandatory
hypotheses in order for these hypotheses to become identical to their
corresponding stack entries.  This process is called {\bf
unification}\index{unification}.  (We also informally use the term
``unification'' to refer to a set of substitutions that results from the
process, as in ``two unifications are possible.'')  After the substitutions
are made, the hypotheses are said to be {\bf unified}.

If no such substitutions are possible, Metamath will consider the proof
incorrect and notify you with an error message.
% (deleted 3/10/07, per suggestion of Mel O'Cat:)
% The syntax of the
% Metamath language ensures that if a set of substitutions exists, it
% will be unique.

The general algorithm for unification described in the literature is
somewhat complex.
However, in the case of Metamath it is intentionally trivial.
Mandatory hypotheses must be
pushed on the proof stack in the order in which they appear.
In addition, each variable must have its type specified
with a \texttt{\$f} hypothesis before it is used
and that each \texttt{\$f} hypothesis
have the restricted syntax of a typecode (a constant) followed by a variable.
The typecode in the \texttt{\$f} hypothesis must match the first symbol of
the corresponding RPN stack entry (which will also be a constant), so
the only possible match for the variable in the \texttt{\$f} hypothesis is
the sequence of symbols in the stack entry after the initial constant.

In the Proof Assistant\index{Proof Assistant}, a more general unification
algorithm is used.  While a proof is being developed, sometimes not enough
information is available to determine a unique unification.  In this case
Metamath will ask you to pick the correct one.\index{ambiguous
unification}\index{unification!ambiguous}

\section{Extensions to the Metamath Language}\index{extended
language}

\subsection{Comments in the Metamath Language}\label{comments}
\index{markup notation}
\index{comments!markup notation}

The commenting feature allows you to annotate the contents of
a database.  Just as with most
computer languages, comments are ignored for the purpose of interpreting the
contents of the database. Comments effectively act as
additional white space\index{white
space} between tokens
when a database is parsed.

A comment may be placed at the beginning, end, or
between any two tokens\index{token} in a source file.

Comments have the following syntax:
\begin{center}
 \texttt{\$(} {\em text} \texttt{\$)}
\end{center}
Here,\index{\texttt{\$(} and \texttt{\$)} auxiliary
keywords}\index{comment} {\em text} is a string, possibly empty, of any
characters in Metamath's character set (p.~\pageref{spec1chars}), except
that the character strings \texttt{\$(} and \texttt{\$)} may not appear
in {\em text}.  Thus nested comments are not
permitted:\footnote{Computer languages have differing standards for
nested comments, and rather than picking one it was felt simplest not to
allow them at all, at least in the current version of
Metamath\index{Metamath!limitations of version 0.07.30}.} Metamath will
complain if you give it
\begin{center}
 \texttt{\$( This is a \$( nested \$) comment.\ \$)}
\end{center}
To compensate for this non-nesting behavior, I often change all \texttt{\$}'s
to \texttt{@}'s in sections of Metamath code I wish to comment out.

The Metamath program supports a number of markup mechanisms and conventions
to generate good-looking results in \LaTeX\ and {\sc html},
as discussed below.
These markup features have to do only with how the comments are typeset,
and have no effect on how Metamath verifies the proofs in the database.
The improper
use of them may result in incorrectly typeset output, but no Metamath
error messages will result during the \texttt{read} and \texttt{verify
proof} commands.  (However, the \texttt{write
theorem\texttt{\char`\_}list} command
will check for markup errors as a side-effect of its
{\sc html} generation.)
Section~\ref{texout} has instructions for creating \LaTeX\ output, and
section~\ref{htmlout} has instructions for creating
{\sc html}\index{HTML} output.

\subsubsection{Headings}\label{commentheadings}

If the \texttt{\$(} is immediately followed by a new line
starting with a heading marker, it is a header.
This can start with:

\begin{itemize}
 \item[] \texttt{\#\#\#\#} - major part header
 \item[] \texttt{\#*\#*} - section header
 \item[] \texttt{=-=-} - subsection header
 \item[] \texttt{-.-.} - subsubsection header
\end{itemize}

The line following the marker line
will be used for the table of contents entry, after trimming spaces.
The next line should be another (closing) matching marker line.
Any text after that
but before the closing \texttt{\$}, such as an extended description of the
section, will be included on the \texttt{mmtheoremsNNN.html} page.

For more information, run
\texttt{help write theorem\char`\_list}.

\subsubsection{Math mode}
\label{mathcomments}
\index{\texttt{`} inside comments}
\index{\texttt{\char`\~} inside comments}
\index{math mode}

Inside of comments, a string of tokens\index{token} enclosed in
grave accents\index{grave accent (\texttt{`})} (\texttt{`}) will be converted
to standard mathematical symbols during
{\sc HTML}\index{HTML} or \LaTeX\ output
typesetting,\index{latex@{\LaTeX}} according to the information in the
special \texttt{\$t}\index{\texttt{\$t} comment}\index{typesetting
comment} comment in the database
(see section~\ref{tcomment} for information about the typesetting
comment, and Appendix~\ref{ASCII} to see examples of its results).

The first grave accent\index{grave accent (\texttt{`})} \texttt{`}
causes the output processor to enter {\bf math mode}\index{math mode}
and the second one exits it.
In this
mode, the characters following the \texttt{`} are interpreted as a
sequence of math symbol tokens separated by white space\index{white
space}.  The tokens are looked up in the \texttt{\$t}
comment\index{\texttt{\$t} comment}\index{typesetting comment} and if
found, they will be replaced by the standard mathematical symbols that
they correspond to before being placed in the typeset output file.  If
not found, the symbol will be output as is and a warning will be issued.
The tokens do not have to be active in the database, although a warning
will be issued if they are not declared with \texttt{\$c} or
\texttt{\$v} statements.

Two consecutive
grave accents \texttt{``} are treated as a single actual grave accent
(both inside and outside of math mode) and will not cause the output
processor to enter or exit math mode.

Here is an example of its use\index{Pierce's axiom}:
\begin{center}
\texttt{\$( Pierce's axiom, ` ( ( ph -> ps ) -> ph ) -> ph ` ,\\
         is not very intuitive. \$)}
\end{center}
becomes
\begin{center}
   \texttt{\$(} Pierce's axiom, $((\varphi \rightarrow \psi)\rightarrow
\varphi)\rightarrow \varphi$, is not very intuitive. \texttt{\$)}
\end{center}

Note that the math symbol tokens\index{token} must be surrounded by white
space\index{white space}.
%, since there is no context that allows ambiguity to be
%resolved, as is the case with math symbol sequences in some of the Metamath
%statements.
White space should also surround the \texttt{`}
delimiters.

The math mode feature also gives you a quick and easy way to generate
text containing mathematical symbols, independently of the intended
purpose of Metamath.\index{Metamath!using as a math editor} To do this,
simply create your text with grave accents surrounding your formulas,
after making sure that your math symbols are mapped to \LaTeX\ symbols
as described in Appendix~\ref{ASCII}.  It is easier if you start with a
database with predefined symbols such as \texttt{set.mm}.  Use your
grave-quoted math string to replace an existing comment, then typeset
the statement corresponding to that comment following the instructions
from the \texttt{help tex} command in the Metamath program.  You will
then probably want to edit the resulting file with a text editor to fine
tune it to your exact needs.

\subsubsection{Label Mode}\index{label mode}

Outside of math mode, a tilde\index{tilde (\texttt{\char`\~})} \verb/~/
indicates to Metamath's\index{Metamath} output processor that the
token\index{token} that follows (i.e.\ the characters up to the next
white space\index{white space}) represents a statement label or URL.
This formatting mode is called {\bf label mode}\index{label mode}.
If a literal tilde
is desired (outside of math mode) instead of label mode,
use two tildes in a row to represent it.

When generating a \LaTeX\ output file,
the following token will be formatted in \texttt{typewriter}
font, and the tilde removed, to make it stand out from the rest of the text.
This formatting will be applied to all characters after the
tilde up to the first white space\index{white space}.
Whether
or not the token is an actual statement label is not checked, and the
token does not have to have the correct syntax for a label; no error
messages will be produced.  The only effect of the label mode on the
output is that typewriter font will be used for the tokens that are
placed in the \LaTeX\ output file.

When generating {\sc html},
the tokens after the tilde {\em must} be a URL (either http: or https:)
or a valid label.
Error messages will be issued during that output if they aren't.
A hyperlink will be generated to that URL or label.

\subsubsection{Link to bibliographical reference}\index{citation}%
\index{link to bibliographical reference}

Bibliographical references are handled specially when generating
{\sc html} if formatted specially.
Text in the form \texttt{[}{\em author}\texttt{]}
is considered a link to a bibliographical reference.
See \texttt{help html} and \texttt{help write
bibliography} in the Metamath program for more
information.
% \index{\texttt{\char`\[}\ldots\texttt{]} inside comments}
See also Sections~\ref{tcomment} and \ref{wrbib}.

The \texttt{[}{\em author}\texttt{]} notation will also create an entry in
the bibliography cross-reference file generated by \texttt{write
bibliography} (Section~\ref{wrbib}) for {\sc HTML}.
For this to work properly, the
surrounding comment must be formatted as follows:
\begin{quote}
    {\em keyword} {\em label} {\em noise-word}
     \texttt{[}{\em author}\texttt{] p.} {\em number}
\end{quote}
for example
\begin{verbatim}
     Theorem 5.2 of [Monk] p. 223
\end{verbatim}
The {\em keyword} is not case sensitive and must be one of the following:
\begin{verbatim}
     theorem lemma definition compare proposition corollary
     axiom rule remark exercise problem notation example
     property figure postulate equation scheme chapter
\end{verbatim}
The optional {\em label} may consist of more than one
(non-{\em keyword} and non-{\em noise-word}) word.
The optional {\em noise-word} is one of:
\begin{verbatim}
     of in from on
\end{verbatim}
and is  ignored when the cross-reference file is created.  The
\texttt{write
biblio\-graphy} command will perform error checking to verify the
above format.\index{error checking}

\subsubsection{Parentheticals}\label{parentheticals}

The end of a comment may include one or more parenthicals, that is,
statements enclosed in parentheses.
The Metamath program looks for certain parentheticals and can issue
warnings based on them.
They are:

\begin{itemize}
 \item[] \texttt{(Contributed by }
   \textit{NAME}\texttt{,} \textit{DATE}\texttt{.)} -
   document the original contributor's name and the date it was created.
 \item[] \texttt{(Revised by }
   \textit{NAME}\texttt{,} \textit{DATE}\texttt{.)} -
   document the contributor's name and creation date
   that resulted in significant revision
   (not just an automated minimization or shortening).
 \item[] \texttt{(Proof shortened by }
   \textit{NAME}\texttt{,} \textit{DATE}\texttt{.)} -
   document the contributor's name and date that developed a significant
   shortening of the proof (not just an automated minimization).
 \item[] \texttt{(Proof modification is discouraged.)} -
   Note that this proof should normally not be modified.
 \item[] \texttt{(New usage is discouraged.)} -
   Note that this assertion should normally not be used.
\end{itemize}

The \textit{DATE} must be in form YYYY-MMM-DD, where MMM is the
English abbreviation of that month.

\subsubsection{Other markup}\label{othermarkup}
\index{markup notation}

There are other markup notations for generating good-looking results
beyond math mode and label mode:

\begin{itemize}
 \item[]
         \texttt{\char`\_} (underscore)\index{\texttt{\char`\_} inside comments} -
             Italicize text starting from
              {\em space}\texttt{\char`\_}{\em non-space} (i.e.\ \texttt{\char`\_}
              with a space before it and a non-space character after it) until
             the next
             {\em non-space}\texttt{\char`\_}{\em space}.  Normal
             punctuation (e.g.\ a trailing
             comma or period) is ignored when determining {\em space}.
 \item[]
         \texttt{\char`\_} (underscore) - {\em
         non-space}\texttt{\char`\_}{\em non-space-string}, where
          {\em non-space-string} is a string of non-space characters,
         will make {\em non-space-string} become a subscript.
 \item[]
         \texttt{<HTML>}...\texttt{</HTML>} - do not convert
         ``\texttt{<}'' and ``\texttt{>}''
         in the enclosed text when generating {\sc HTML},
         otherwise process markup normally. This allows direct insertion
         of {\sc html} commands.
 \item[]
       ``\texttt{\&}ref\texttt{;}'' - insert an {\sc HTML}
         character reference.
         This is how to insert arbitrary Unicode characters
         (such as accented characters).  Currently only directly supported
         when generating {\sc HTML}.
\end{itemize}

It is recommended that spaces surround any \texttt{\char`\~} and
\texttt{`} tokens in the comment and that a space follow the {\em label}
after a \texttt{\char`\~} token.  This will make global substitutions
to change labels and symbol names much easier and also eliminate any
future chance of ambiguity.  Spaces around these tokens are automatically
removed in the final output to conform with normal rules of punctuation;
for example, a space between a trailing \texttt{`} and a left parenthesis
will be removed.

A good way to become familiar with the markup notation is to look at
the extensive examples in the \texttt{set.mm} database.

\subsection{The Typesetting Comment (\texttt{\$t})}\label{tcomment}

The typesetting comment \texttt{\$t} in the input database file
provides the information necessary to produce good-looking results.
It provides \LaTeX\ and {\sc html}
definitions for math symbols,
as well supporting as some
customization of the generated web page.
If you add a new token to a database, you should also
update the \texttt{\$t} comment information if you want to eventually
create output in \LaTeX\ or {\sc HTML}.
See the
\texttt{set.mm}\index{set theory database (\texttt{set.mm})} database
file for an extensive example of a \texttt{\$t} comment illustrating
many of the features described below.

Programs that do not need to generate good-looking presentation results,
such as programs that only verify Metamath databases,
can completely ignore typesetting comments
and just treat them as normal comments.
Even the Metamath program only consults the
\texttt{\$t} comment information when it needs to generate typeset output
in \LaTeX\ or {\sc HTML}
(e.g., when you open a \LaTeX\ output file with the \texttt{open tex} command).

We will first discuss the syntax of typesetting comments, and then
briefly discuss how this can be used within the Metamath program.

\subsubsection{Typesetting Comment Syntax Overview}

The typesetting comment is identified by the token
\texttt{\$t}\index{\texttt{\$t} comment}\index{typesetting comment} in
the comment, and the typesetting comment ends at the matching
\texttt{\$)}:
\[
  \mbox{\tt \$(\ }
  \mbox{\tt \$t\ }
  \underbrace{
    \mbox{\tt \ \ \ \ \ \ \ \ \ \ \ }
    \cdots
    \mbox{\tt \ \ \ \ \ \ \ \ \ \ \ }
  }_{\mbox{Typesetting definitions go here}}
  \mbox{\tt \ \$)}
\]

There must be one or more white space characters, and only white space
characters, between the \texttt{\$(} that starts the comment
and the \texttt{\$t} symbol,
and the \texttt{\$t} must be followed by one
or more white space characters
(see section \ref{whitespace} for the definition of white space characters).
The typesetting comment continues until the comment end token \texttt{\$)}
(which must be preceded by one or more white space characters).

In version 0.07.30\index{Metamath!limitations of version 0.07.30} of the
Metamath program, there may be only one \texttt{\$t} comment in a
database.  This restriction may be lifted in the future to allow
many \texttt{\$t} comments in a database.

Between the \texttt{\$t} symbol (and its following white space) and the
comment end token \texttt{\$)} (and its preceding white space)
is a sequence of one or more typesetting definitions, where
each definition has the form
\textit{definition-type arg arg ... ;}.
Each of the zero or more \textit{arg} values
can be either a typesetting data or a keyword
(what keywords are allowed, and where, depends on the specific
\textit{definition-type}).
The \textit{definition-type}, and each argument \textit{arg},
are separated by one or more white space characters.
Every definition ends in an unquoted semicolon;
white space is not required before the terminating semicolon of a definition.
Each definition should start on a new line.\footnote{This
restriction of the current version of Metamath
(0.07.30)\index{Metamath!limitations of version 0.07.30} may be removed
in a future version, but you should do it anyway for readability.}

For example, this typesetting definition:
\begin{center}
 \verb$latexdef "C_" as "\subseteq";$
\end{center}
defines the token \verb$C_$ as the \LaTeX\ symbol $\subseteq$ (which means
``subset'').

Typesetting data is a sequence of one or more quoted strings
(if there is more than one, they are connected by \texttt{\char`\+}).
Often a single quoted string is used to provide data for a definition, using
either double (\texttt{\char`\"}) or single (\texttt{'}) quotation marks.
However,
{\em a quoted string (enclosed in quotation marks) may not include
line breaks.}
A quoted string
may include a quotation mark that matches the enclosing quotes by repeating
the quotation mark twice.  Here are some examples:

\begin{tabu}   { l l }
\textbf{Example} & \textbf{Meaning} \\
\texttt{\char`\"a\char`\"\char`\"b\char`\"} & \texttt{a\char`\"b} \\
\texttt{'c''d'} & \texttt{c'd} \\
\texttt{\char`\"e''f\char`\"} & \texttt{e''f} \\
\texttt{'g\char`\"\char`\"h'} & \texttt{g\char`\"\char`\"h} \\
\end{tabu}

Finally, a long quoted string
may be broken up into multiple quoted strings (considered, as a whole,
a single quoted string) and joined with \texttt{\char`\+}.
You can even use multiple lines as long as a
'+' is at the end of every line except the last one.
The \texttt{\char`\+} should be preceded and followed by at least one
white space character.
Thus, for example,
\begin{center}
 \texttt{\char`\"ab\char`\"\ \char`\+\ \char`\"cd\char`\"
    \ \char`\+\ \\ 'ef'}
\end{center}
is the same as
\begin{center}
 \texttt{\char`\"abcdef\char`\"}
\end{center}

{\sc c}-style comments \texttt{/*}\ldots\texttt{*/} are also supported.

In practice, whenever you add a new math token you will often want to add
typesetting definitions using
\texttt{latexdef}, \texttt{htmldef}, and
\texttt{althtmldef}, as described below.
That way, they will all be up to date.
Of course, whether or not you want to use all three definitions will
depend on how the database is intended to be used.

Below we discuss the different possible \textit{definition-kind} options.
We will show data surrounded by double quotes (in practice they can also use
single quotes and/or be a sequence joined by \texttt{+}s).
We will use specific names for the \textit{data} to make clear what
the data is used for, such as
{\em math-token} (for a Metamath math token,
{\em latex-string} (for string to be placed in a \LaTeX\ stream),
{\em {\sc html}-code} (for {\sc html} code),
and {\em filename} (for a filename).

\subsubsection{Typesetting Comment - \LaTeX}

The syntax for a \LaTeX\ definition is:
\begin{center}
 \texttt{latexdef "}{\em math-token}\texttt{" as "}{\em latex-string}\texttt{";}
\end{center}
\index{latex definitions@\LaTeX\ definitions}%
\index{\texttt{latexdef} statement}

The {\em token-string} and {\em latex-string} are the data
(character strings) for
the token and the \LaTeX\ definition of the token, respectively,

These \LaTeX\ definitions are used by the Metamath program
when it is asked to product \LaTeX output using
the \texttt{write tex} command.

\subsubsection{Typesetting Comment - {\sc html}}

The key kinds of {\sc HTML} definitions have the following syntax:

\vskip 1ex
    \texttt{htmldef "}{\em math-token}\texttt{" as "}{\em
    {\sc html}-code}\texttt{";}\index{\texttt{htmldef} statement}
                    \ \ \ \ \ \ldots

    \texttt{althtmldef "}{\em math-token}\texttt{" as "}{\em
{\sc html}-code}\texttt{";}\index{\texttt{althtmldef} statement}

                    \ \ \ \ \ \ldots

Note that in {\sc HTML} there are two possible definitions for math tokens.
This feature is useful when
an alternate representation of symbols is desired, for example one that
uses Unicode entities and another uses {\sc gif} images.

There are many other typesetting definitions that can control {\sc HTML}.
These include:

\vskip 1ex

    \texttt{htmldef "}{\em math-token}\texttt{" as "}{\em {\sc
    html}-code}\texttt{";}

    \texttt{htmltitle "}{\em {\sc html}-code}\texttt{";}%
\index{\texttt{htmltitle} statement}

    \texttt{htmlhome "}{\em {\sc html}-code}\texttt{";}%
\index{\texttt{htmlhome} statement}

    \texttt{htmlvarcolor "}{\em {\sc html}-code}\texttt{";}%
\index{\texttt{htmlvarcolor} statement}

    \texttt{htmlbibliography "}{\em filename}\texttt{";}%
\index{\texttt{htmlbibliography} statement}

\vskip 1ex

\noindent The \texttt{htmltitle} is the {\sc html} code for a common
title, such as ``Metamath Proof Explorer.''  The \texttt{htmlhome} is
code for a link back to the home page.  The \texttt{htmlvarcolor} is
code for a color key that appears at the bottom of each proof.  The file
specified by {\em filename} is an {\sc html} file that is assumed to
have a \texttt{<A NAME=}\ldots\texttt{>} tag for each bibiographic
reference in the database comments.  For example, if
\texttt{[Monk]}\index{\texttt{\char`\[}\ldots\texttt{]} inside comments}
occurs in the comment for a theorem, then \texttt{<A NAME='Monk'>} must
be present in the file; if not, a warning message is given.

Associated with
\texttt{althtmldef}
are the statements
\vskip 1ex

    \texttt{htmldir "}{\em
      directoryname}\texttt{";}\index{\texttt{htmldir} statement}

    \texttt{althtmldir "}{\em
     directoryname}\texttt{";}\index{\texttt{althtmldir} statement}

\vskip 1ex
\noindent giving the directories of the {\sc gif} and Unicode versions
respectively; their purpose is to provide cross-linking between the
two versions in the generated web pages.

When two different types of pages need to be produced from a single
database, such as the Hilbert Space Explorer that extends the Metamath
Proof Explorer, ``extended'' variables may be declared in the
\texttt{\$t} comment:
\vskip 1ex

    \texttt{exthtmltitle "}{\em {\sc html}-code}\texttt{";}%
\index{\texttt{exthtmltitle} statement}

    \texttt{exthtmlhome "}{\em {\sc html}-code}\texttt{";}%
\index{\texttt{exthtmlhome} statement}

    \texttt{exthtmlbibliography "}{\em filename}\texttt{";}%
\index{\texttt{exthtmlbibliography} statement}

\vskip 1ex
\noindent When these are declared, you also must declare
\vskip 1ex

    \texttt{exthtmllabel "}{\em label}\texttt{";}%
\index{\texttt{exthtmllabel} statement}

\vskip 1ex \noindent that identifies the database statement where the
``extended'' section of the database starts (in our example, where the
Hilbert Space Explorer starts).  During the generation of web pages for
that starting statement and the statements after it, the {\sc html} code
assigned to \texttt{exthtmltitle} and \texttt{exthtmlhome} is used
instead of that assigned to \texttt{htmltitle} and \texttt{htmlhome},
respectively.

\begin{sloppy}
\subsection{Additional Information Com\-ment (\texttt{\$j})} \label{jcomment}
\end{sloppy}

The additional information comment, aka the
\texttt{\$j}\index{\texttt{\$j} comment}\index{additional information comment}
comment,
provides a way to add additional structured information that can
be optionally parsed by systems.

The additional information comment is parsed the same way as the
typesetting comment (\texttt{\$t}) (see section \ref{tcomment}).
That is,
the additional information comment begins with the token
\texttt{\$j} within a comment,
and continues until the comment close \texttt{\$)}.
Within an additional information comment is a sequence of one or more
commands of the form \texttt{command arg arg ... ;}
where each of the zero or more \texttt{arg} values
can be either a quoted string or a keyword.
Note that every command ends in an unquoted semicolon.
If a verifier is parsing an additional information comment, but
doesn't recognize a particular command, it must skip the command
by finding the end of the command (an unquoted semicolon).

A database may have 0 or more additional information comments.
Note, however, that a verifier may ignore these comments entirely or only
process certain commands in an additional information comment.
The \texttt{mmj2} verifier supports many commands in additional information
comments.
We encourage systems that process additional information comments
to coordinate so that they will use the same command for the same effect.

Examples of additional information comments with various commands
(from the \texttt{set.mm} database) are:

\begin{itemize}
   \item Define the syntax and logical typecodes,
     and declare that our grammar is
     unambiguous (verifiable using the KLR parser, with compositing depth 5).
\begin{verbatim}
  $( $j
    syntax 'wff';
    syntax '|-' as 'wff';
    unambiguous 'klr 5';
  $)
\end{verbatim}

   \item Register $\lnot$ and $\rightarrow$ as primitive expressions
           (lacking definitions).
\begin{verbatim}
  $( $j primitive 'wn' 'wi'; $)
\end{verbatim}

   \item There is a special justification for \texttt{df-bi}.
\begin{verbatim}
  $( $j justification 'bijust' for 'df-bi'; $)
\end{verbatim}

   \item Register $\leftrightarrow$ as an equality for its type (wff).
\begin{verbatim}
  $( $j
    equality 'wb' from 'biid' 'bicomi' 'bitri';
    definition 'dfbi1' for 'wb';
  $)
\end{verbatim}

   \item Theorem \texttt{notbii} is the congruence law for negation.
\begin{verbatim}
  $( $j congruence 'notbii'; $)
\end{verbatim}

   \item Add \texttt{setvar} as a typecode.
\begin{verbatim}
  $( $j syntax 'setvar'; $)
\end{verbatim}

   \item Register $=$ as an equality for its type (\texttt{class}).
\begin{verbatim}
  $( $j equality 'wceq' from 'eqid' 'eqcomi' 'eqtri'; $)
\end{verbatim}

\end{itemize}


\subsection{Including Other Files in a Metamath Source File} \label{include}
\index{\texttt{\$[} and \texttt{\$]} auxiliary keywords}

The keywords \texttt{\$[} and \texttt{\$]} specify a file to be
included\index{included file}\index{file inclusion} at that point in a
Metamath\index{Metamath} source file\index{source file}.  The syntax for
including a file is as follows:
\begin{center}
\texttt{\$[} {\em file-name} \texttt{\$]}
\end{center}

The {\em file-name} should be a single token\index{token} with the same syntax
as a math symbol (i.e., all 93 non-whitespace
printable characters other than \texttt{\$} are
allowed, subject to the file-naming limitations of your operating system).
Comments may appear between the \texttt{\$[} and \texttt{\$]} keywords.  Included
files may include other files, which may in turn include other files, and so
on.

For example, suppose you want to use the set theory database as the starting
point for your own theory.  The first line in your file could be
\begin{center}
\texttt{\$[ set.mm \$]}
\end{center} All of the information (axioms, theorems,
etc.) in \texttt{set.mm} and any files that {\em it} includes will become
available for you to reference in your file. This can help make your work more
modular. A drawback to including files is that if you change the name of a
symbol or the label of a statement, you must also remember to update any
references in any file that includes it.


The naming conventions for included files are the same as those of your
operating system.\footnote{On the Macintosh, prior to Mac OS X,
 a colon is used to separate disk
and folder names from your file name.  For example, {\em volume}\texttt{:}{\em
file-name} refers to the root directory, {\em volume}\texttt{:}{\em
folder-name}\texttt{:}{\em file-name} refers to a folder in root, and {\em
volume}\texttt{:}{\em folder-name}\texttt{:}\ldots\texttt{:}{\em file-name} refers to a
deeper folder.  A simple {\em file-name} refers to a file in the folder from
which you launch the Metamath application.  Under Mac OS X and later,
the Metamath program is run under the Terminal application, which
conforms to Unix naming conventions.}\index{Macintosh file
names}\index{file names!Macintosh}\label{includef} For compatibility among
operating systems, you should keep the file names as simple as possible.  A
good convention to use is {\em file}\texttt{.mm} where {\em file} is eight
characters or less, in lower case.

There is no limit to the nesting depth of included files.  One thing that you
should be aware of is that if two included files themselves include a common
third file, only the {\em first} reference to this common file will be read
in.  This allows you to include two or more files that build on a common
starting file without having to worry about label and symbol conflicts that
would occur if the common file were read in more than once.  (In fact, if a
file includes itself, the self-reference will be ignored, although of course
it would not make any sense to do that.)  This feature also means, however,
that if you try to include a common file in several inner blocks, the result
might not be what you expect, since only the first reference will be replaced
with the included file (unlike the include statement in most other computer
languages).  Thus you would normally include common files only in the
outermost block\index{outermost block}.

\subsection{Compressed Proof Format}\label{compressed1}\index{compressed
proof}\index{proof!compressed}

The proof notation presented in Section~\ref{proof} is called a
{\bf normal proof}\index{normal proof}\index{proof!normal} and in principle is
sufficient to express any proof.  However, proofs often contain steps and
subproofs that are identical.  This is particularly true in typical
Metamath\index{Metamath} applications, because Metamath requires that the math
symbol sequence (usually containing a formula) at each step be separately
constructed, that is, built up piece by piece. As a result, a lot of
repetition often results.  The {\bf compressed proof} format allows Metamath
to take advantage of this redundancy to shorten proofs.

The specification for the compressed proof format is given in
Appen\-dix~\ref{compressed}.

Normally you need not concern yourself with the details of the compressed
proof format, since the Metamath program will allow you to convert from
the normal format to the compressed format with ease, and will also
automatically convert from the compressed format when proofs are displayed.
The overall structure of the compressed format is as follows:
\begin{center}
  \texttt{\$= ( } {\em label-list} \texttt{) } {\em compressed-proof\ }\ \texttt{\$.}
\end{center}
\index{\texttt{\$=} keyword}
The first \texttt{(} serves as a flag to Metamath that a compressed proof
follows.  The {\em label-list} includes all statements referred to by the
proof except the mandatory hypotheses\index{mandatory hypothesis}.  The {\em
compressed-proof} is a compact encoding of the proof, using upper-case
letters, and can be thought of as a large integer in base 26.  White
space\index{white space} inside a {\em compressed-proof} is
optional and is ignored.

It is important to note that the order of the mandatory hypotheses of
the statement being proved must not be changed if the compressed proof
format is used, otherwise the proof will become incorrect.  The reason
for this is that the mandatory hypotheses are not mentioned explicitly
in the compressed proof in order to make the compression more efficient.
If you wish to change the order of mandatory hypotheses, you must first
convert the proof back to normal format using the \texttt{save proof
{\em statement} /normal}\index{\texttt{save proof} command} command.
Later, you can go back to compressed format with \texttt{save proof {\em
statement} /compressed}.

During error checking with the \texttt{verify proof} command, an error
found in a compressed proof may point to a character in {\em
compressed-proof}, which may not be very meaningful to you.  In this
case, try to \texttt{save proof /normal} first, then do the
\texttt{verify proof} again.  In general, it is best to make sure a
proof is correct before saving it in compressed format, because severe
errors are less likely to be recoverable than in normal format.

\subsection{Specifying Unknown Proofs or Subproofs}\label{unknown}

In a proof under development, any step or subproof that is not yet known
may be represented with a single \texttt{?}.  For the purposes of
parsing the proof, the \texttt{?}\ \index{\texttt{]}@\texttt{?}\ inside
proofs} will push a single entry onto the RPN stack just as if it were a
hypothesis.  While developing a proof with the Proof
Assistant\index{Proof Assistant}, a partially developed proof may be
saved with the \texttt{save new{\char`\_}proof}\index{\texttt{save
new{\char`\_}proof} command} command, and \texttt{?}'s will be placed at
the appropriate places.

All \texttt{\$p}\index{\texttt{\$p} statement} statements must have
proofs, even if they are entirely unknown.  Before creating a proof with
the Proof Assistant, you should specify a completely unknown proof as
follows:
\begin{center}
  {\em label} \texttt{\$p} {\em statement} \texttt{\$= ?\ \$.}
\end{center}
\index{\texttt{\$=} keyword}
\index{\texttt{]}@\texttt{?}\ inside proofs}

The \texttt{verify proof}\index{\texttt{verify proof} command} command
will check the known portions of a partial proof for errors, but will
warn you that the statement has not been proved.

Note that partially developed proofs may be saved in compressed format
if desired.  In this case, you will see one or more \texttt{?}'s in the
{\em compressed-proof} part.\index{compressed
proof}\index{proof!compressed}

\section{Axioms vs.\ Definitions}\label{definitions}

The \textit{basic}
Metamath\index{Metamath} language and program
make no distinction\index{axiom vs.\
definition} between axioms\index{axiom} and
definitions.\index{definition} The \texttt{\$a}\index{\texttt{\$a}
statement} statement is used for both.  At first, this may seem
puzzling.  In the minds of many mathematicians, the distinction is
clear, even obvious, and hardly worth discussing.  A definition is
considered to be merely an abbreviation that can be replaced by the
expression for which it stands; although unless one actually does this,
to be precise then one should say that a theorem\index{theorem} is a
consequence of the axioms {\em and} the definitions that are used in the
formulation of the theorem \cite[p.~20]{Behnke}.\index{Behnke, H.}

\subsection{What is a Definition?}

What is a definition?  In its simplest form, a definition introduces a new
symbol and provides an unambiguous rule to transform an expression containing
the new symbol to one without it.  The concept of a ``proper
definition''\index{proper definition}\index{definition!proper} (as opposed to
a creative definition)\index{creative definition}\index{definition!creative}
that is usually agreed upon is (1) the definition should not strengthen the
language and (2) any symbols introduced by the definition should be eliminable
from the language \cite{Nemesszeghy}\index{Nemesszeghy, E. Z.}.  In other
words, they are mere typographical conveniences that do not belong to the
system and are theoretically superfluous.  This may seem obvious, but in fact
the nature of definitions can be subtle, sometimes requiring difficult
metatheorems to establish that they are not creative.

A more conservative stance was taken by logician S.
Le\'{s}niewski.\index{Le\'{s}niewski, S.}
\begin{quote}
Le\'{s}niewski
regards definitions as theses of the system.  In this respect they do
not differ either from the axioms or from theorems, i.e.\ from the
theses added to the system on the basis of the rule of substitution or
the rule of detachment [modus ponens].  Once definitions have been
accepted as theses of the system, it becomes necessary to consider them
as true propositions in the same sense in which axioms are true
\cite{Lejewski}.
\end{quote}\index{Lejewski, Czeslaw}

Let us look at some simple examples of definitions in propositional
calculus.  Consider the definition of logical {\sc or}
(disjunction):\index{disjunction ($\vee$)} ``$P\vee Q$ denotes $\neg P
\rightarrow Q$ (not $P$ implies $Q$).''  It is very easy to recognize a
statement making use of this definition, because it introduces the new
symbol $\vee$ that did not previously exist in the language.  It is easy
to see that no new theorems of the original language will result from
this definition.

Next, consider a definition that eliminates parentheses:  ``$P
\rightarrow Q\rightarrow R$ denotes $P\rightarrow (Q \rightarrow R)$.''
This is more subtle, because no new symbols are introduced.  The reason
this definition is considered proper is that no new symbol sequences
that are valid wffs (well-formed formulas)\index{well-formed formula
(wff)} in the original language will result from the definition, since
``$P \rightarrow Q\rightarrow R$'' is not a wff in the original
language.  Here, we implicitly make use of the fact that there is a
decision procedure that allows us to determine whether or not a symbol
sequence is a wff, and this fact allows us to use symbol sequences that
are not wffs to represent other things (such as wffs) by means of the
definition.  However, to justify the definition as not being creative we
need to prove that ``$P \rightarrow Q\rightarrow R$'' is in fact not a
wff in the original language, and this is more difficult than in the
case where we simply introduce a new symbol.

%Now let's take this reasoning to an extreme.  Propositional calculus is a
%decidable theory,\footnote{This means that a mechanical algorithm exists to
%determine whether or not a wff is a theorem.} so in principle we could make use
%of symbol sequences that are not theorems to represent other things (say, to
%encode actual theorems in a more compact way).  For example, let us extend the
%language by defining a wff ``$P$'' in the extended language as the theorem
%``$P\rightarrow P$''\footnote{This is one of the first theorems proved in the
%Metamath database \texttt{set.mm}.}\index{set
%theory database (\texttt{set.mm})} in the original language whenever ``$P$'' is
%not a theorem in the original language.  In the extended language, any wff
%``$Q$'' thus represents a theorem; to find out what theorem (in the original
%language) ``$Q$'' represents, we determine whether ``$Q$'' is a theorem in the
%original language (before the definition was introduced).  If so, we're done; if
%not, we replace ``$Q$'' by ``$Q\rightarrow Q$'' to eliminate the definition.
%This definition is therefore eliminable, and it does not ``strengthen'' the
%language because any wff that is not a theorem is not in the set of statements
%provable in the original language and thus is available for use by definitions.
%
%Of course, a definition such as this would render practically useless the
%communication of theorems of propositional calculus; but
%this is just a human shortcoming, since we can't always easily discern what is
%and is not a theorem by inspection.  In fact, the extended theory with this
%definition has no more and no less information than the original theory; it just
%expresses certain theorems of the form ``$P\rightarrow P$''
%in a more compact way.
%
%The point here is that what constitutes a proper definition is a matter of
%judgment about whether a symbol sequence can easily be recognized by a human
%as invalid in some sense (for example, not a wff); if so, the symbol sequence
%can be appropriated for use by a definition in order to make the extended
%language more compact.  Metamath\index{Metamath} lacks the ability to make this
%judgment, since as far as Metamath is concerned the definition of a wff, for
%example, is arbitrary.  You define for Metamath how wffs\index{well-formed
%formula (wff)} are constructed according to your own preferred style.  The
%concept of a wff may not even exist in a given formal system\index{formal
%system}.  Metamath treats all definitions as if they were new axioms, and it
%is up to the human mathematician to judge whether the definition is ``proper''
%'\index{proper definition}\index{definition!proper} in some agreed-upon way.

What constitutes a definition\index{definition} versus\index{axiom vs.\
definition} an axiom\index{axiom} is sometimes arbitrary in mathematical
literature.  For example, the connectives $\vee$ ({\sc or}), $\wedge$
({\sc and}), and $\leftrightarrow$ (equivalent to) in propositional
calculus are usually considered defined symbols that can be used as
abbreviations for expressions containing the ``primitive'' connectives
$\rightarrow$ and $\neg$.  This is the way we treat them in the standard
logic and set theory database \texttt{set.mm}\index{set theory database
(\texttt{set.mm})}.  However, the first three connectives can also be
considered ``primitive,'' and axiom systems have been devised that treat
all of them as such.  For example,
\cite[p.~35]{Goodstein}\index{Goodstein, R. L.} presents one with 15
axioms, some of which in fact coincide with what we have chosen to call
definitions in \texttt{set.mm}.  In certain subsets of classical
propositional calculus, such as the intuitionist
fragment\index{intuitionism}, it can be shown that one cannot make do
with just $\rightarrow$ and $\neg$ but must treat additional connectives
as primitive in order for the system to make sense.\footnote{Two nice
systems that make the transition from intuitionistic and other weak
fragments to classical logic just by adding axioms are given in
\cite{Robinsont}\index{Robinson, T. Thacher}.}

\subsection{The Approach to Definitions in \texttt{set.mm}}

In set theory, recursive definitions define a newly introduced symbol in
terms of itself.  The justification of recursive definitions, using
several ``recursion theorems,'' is usually one of the first
sophisticated proofs a student encounters when learning set theory, and
there is a significant amount of implicit metalogic behind a recursive
definition even though the definition itself is typically simple to
state.  It is, however, possible to substitute one kind of complexity
for another.  We can eliminate the need for metalogical justification by
defining the operation directly with an explicit (but complicated)
expression, then deriving the recursive definition directly as a
theorem, using a recursion theorem ``in reverse.''  We do this in
\texttt{set.mm}, as follows.

In \texttt{set.mm} our goal was to introduce almost all definitions in
the form of two expressions connected by either $\leftrightarrow$ or
$=$, where the thing being defined does not appear on the right hand
side.  Quine calls this form ``a genuine or direct definition'' \cite[p.
174]{Quine}\index{Quine, Willard Van Orman}, which makes the definitions
very easy to eliminate and the metalogic\index{metalogic} needed to
justify them as simple as possible.

We achieved this goal in almost all cases in \texttt{set.mm}.
Sometimes this makes the definitions more complex and less
intuitive.  For example, the traditional way to define addition of
natural numbers is to define an operation called {\em
successor}\index{successor} (which means ``plus one'' and is denoted by
``${\rm suc}$''), then define addition recursively\index{recursive
definition} with the two definitions $n + 0 = n$ and $m + {\rm suc}\,n =
{\rm suc} (m + n)$.  Although this definition seems simple and obvious,
the method to eliminate the definition is not obvious:  in the second
part of the definition, addition is defined in terms of itself.  By
eliminating the definition, we don't mean repeatedly applying it to
specific $m$ and $n$ but rather showing the explicit, closed-form
set-theoretical expression that $m + n$ represents, that will work for
any $m$ and $n$ and that does not have a $+$ sign on its right-hand
side.  For a recursive definition like this not to be circular
(creative), there are some hidden, underlying assumptions we must make,
for example that the natural numbers have a certain kind of order.  In
\texttt{set.mm} we chose to start with the direct (though complex and
nonintuitive) definition then derive from it the standard recursive
definition.\footnote{The closed-form definition used in \texttt{set.mm}
for the addition operation on ordinals\index{ordinal
addition}\index{addition!of ordinals} (of which natural numbers are a
subset) is

\setbox\startprefix=\hbox{\tt \ \ df-oadd\ \$a\ }
\setbox\contprefix=\hbox{\tt \ \ \ \ \ \ \ \ \ \ \ \ \ }
\startm
\m{\vdash}\m{+_o}\m{=}\m{\{}\m{\langle}\m{\langle}\m{x}\m{,}\m{y}\m{\rangle}%
\m{,}\m{z}\m{\rangle}\m{|}\m{(}\m{(}\m{x}\m{\in}\m{{\rm On}}\m{\wedge}\m{y}\m{%
\in}\m{{\rm On}}\m{)}\m{\wedge}\m{z}\m{=}\m{(}\m{{\rm rec}}\m{(}\m{\{}\m{%
\langle}\m{w}\m{,}\m{v}\m{\rangle}\m{|}\m{v}\m{=}\m{{\rm suc}}\m{w}\m{\}}\m{,}%
\m{x}\m{)}\m{`}\m{y}\m{)}\m{)}\m{\}}
\endm

\noindent Here, the abstraction class of nested ordered pairs\index{abstraction
class!of nested ordered pairs} is defined by \texttt{df-oprab} in \texttt{set.mm},
and ${\rm rec}$ is a ``recursion operator''\index{recursion operator} with
the definition

\setbox\startprefix=\hbox{\tt \ \ df-rfg\ \$a\ }
\setbox\contprefix=\hbox{\tt \ \ \ \ \ \ \ \ \ \ \ \ }
\startm
\m{\vdash}\m{{\rm rec}}\m{(}\m{F}\m{,}\m{A}\m{)}\m{=}\m{\bigcup}\m{\{}\m{f}%
\m{|}\m{\exists}\m{x}\m{(}\m{x}\m{\in}\m{{\rm On}}\m{\wedge}\m{(}\m{f}\m{{%
\rm Fn}}\m{x}\m{\wedge}\m{\forall}\m{y}\m{(}\m{y}\m{\in}\m{x}\m{\rightarrow}%
\m{(}\m{f}\m{`}\m{y}\m{)}\m{=}\m{(}\m{\{}\m{\langle}\m{g}\m{,}\m{z}\m{\rangle}%
\m{|}\m{(}\m{(}\m{g}\m{=}\m{\varnothing}\m{\wedge}\m{z}\m{=}\m{A}\m{)}\m{\vee}%
\m{(}\m{\lnot}\m{(}\m{g}\m{=}\m{\varnothing}\m{\vee}\m{{\rm Lim}}\m{{\rm dom}}%
\m{g}\m{)}\m{\wedge}\m{z}\m{=}\m{(}\m{F}\m{`}\m{(}\m{g}\m{`}\m{\bigcup}\m{{%
\rm dom}}\m{g}\m{)}\m{)}\m{)}\m{\vee}\m{(}\m{{\rm Lim}}\m{{\rm dom}}\m{g}\m{%
\wedge}\m{z}\m{=}\m{\bigcup}\m{{\rm ran}}\m{g}\m{)}\m{)}\m{\}}\m{`}\m{(}\m{f}%
\m{\restriction}\m{y}\m{)}\m{)}\m{)}\m{)}\m{)}\m{\}}
\endm

\noindent which can be further broken down with definitions shown in
Section~\ref{setdefinitions}.  You may be surprised at the complexity of what
seems like such a simple notion.  From these definitions the simpler, more
intuitive recursive definition is derived as a set of theorems.}\index{natural
number}\index{addition}\index{recursive definition}\index{ordinal addition}
The end result is the same, but we completely eliminate the rather complex
metalogic that justifies the recursive definition.  (For a mathematician,
recursive definitions are more efficient and intuitive than
direct ones once the metalogic has been learned or possibly just accepted as
correct.  However, it was felt that direct definition in \texttt{set.mm}
maximizes rigor by minimizing metalogic.  It can be eliminated effortlessly,
something difficult to do with a recursive definition.)

\subsection{Adding Constraints on Definitions}

The basic Metamath language and the Metamath program do
not have any built-in constraints on definitions, since they are just
\$a statements.

However, nothing prevents a verification system from verifying
additional rules to impose further limitations on definitions.
For example, the \texttt{mmj2}\index{mmj2} program
supports various kinds of
additional information comments (see section \ref{jcomment}).
One of their uses is to optionally verify additional constraints,
including constraints to verify that definitions meet certain
requirements.
These additional checks are required by the
continuous integration (CI)\index{continuous integration (CI)}
checks of the
\texttt{set.mm}\index{set theory database (\texttt{set.mm})}%
\index{Metamath Proof Explorer}
database.
This approach enables us to optionally impose additional requirements
on definitions if we wish, without necessarily imposing those rules on
all databases or requiring all verification systems to implement them.
In addition, this allows us to impose specialized constraints tailored
to one database while not requiring other systems to implement
those specialized constraints.

We impose two constraints on the
\texttt{set.mm}\index{set theory database (\texttt{set.mm})}%
\index{Metamath Proof Explorer} database
via the \texttt{mmj2}\index{mmj2} program that are worth discussing here:
a parse check and a definition soundness check.

% On February 11, 2019 8:32:32 PM EST, saueran@oregonstate.edu wrote:
% The following addition to the end of set.mm is accepted by the mmj2
% parser and definition checker and the metamath verifier(at least it was
% when I checked, you should check it too), and creates a contradiction by
% proving the theorem |- ph.
% ${
% wleftp $a wff ( ( ph ) $.
% wbothp $a wff ( ph ) $.
% df-leftp $a |- ( ( ( ph ) <-> -. ph ) $.
% df-bothp $a |- ( ( ph ) <-> ph ) $.
% anything $p |- ph $=
%   ( wbothp wn wi wleftp df-leftp biimpi df-bothp mpbir mpbi simplim ax-mp)
%   ABZAMACZDZCZMOEZOCQAEZNDZRNAFGSHIOFJMNKLAHJ $.
% $}
%
% This particular problem is countered by enabling, within mmj2,
% SetParser,mmj.verify.LRParser

First,
we enable a parse check in \texttt{mmj2} (through its
\texttt{SetParser} command) that requires that all new definitions
must \textit{not} create an ambiguous parse for a KLR(5) parser.
This prevents some errors such as definitions with imbalanced parentheses.

Second, we run a definition soundness check specific to
\texttt{set.mm} or databases similar to it.
(through the \texttt{definitionCheck} macro).
Some \texttt{\$a} statements (including all ax-* statemnets)
are excluded from these checks, as they will
always fail this simple check,
but they are appropriate for most definitions.
This check imposes a set of additional rules:

\begin{enumerate}

\item New definitions must be introduced using $=$ or $\leftrightarrow$.

\item No \texttt{\$a} statement introduced before this one is allowed to use the
symbol being defined in this definition, and the definition is not
allowed to use itself (except once, in the definiendum).

\item Every variable in the definiens should not be distinct

\item Every dummy variable in the definiendum
are required to be distinct from each other and from variables in
the definiendum.
To determine this, the system will look for a "justification" theorem
in the database, and if it is not there, attempt to briefly prove
$( \varphi \rightarrow \forall x \varphi )$  for each dummy variable x.

\item Every dummy variable should be a set variable,
unless there is a justification theorem available.

\item Every dummy variable must be bound
(if the system cannot determine this a justification theorem must be
provided).

\end{enumerate}

\subsection{Summary of Approach to Definitions}

In short, when being rigorous it turns out that
definitions can be subtle, sometimes requiring difficult
metatheorems to establish that they are not creative.

Instead of building such complications into the Metamath language itself,
the basic Metmath language and program simply treat traditional
axioms and definitions as the same kind of \texttt{\$a} statement.
We have then built various tools to enable people to
verify additional conditions as their creators believe is appropriate
for those specific databases, without complicating the Metamath foundations.

\chapter{The Metamath Program}\label{commands}

This chapter provides a reference manual for the
Metamath program.\index{Metamath!commands}

Current instructions for obtaining and installing the Metamath program
can be found at the \url{http://metamath.org} web site.
For Windows, there is a pre-compiled version called
\texttt{metamath.exe}.  For Unix, Linux, and Mac OS X
(which we will refer to collectively as ``Unix''), the Metamath program
can be compiled from its source code with the command
\begin{verbatim}
gcc *.c -o metamath
\end{verbatim}
using the \texttt{gcc} {\sc c} compiler available on those systems.

In the command syntax descriptions below, fields enclosed in square
brackets [\ ] are optional.  File names may be optionally enclosed in
single or double quotes.  This is useful if the file name contains
spaces or
slashes (\texttt{/}), such as in Unix path names, \index{Unix file
names}\index{file names!Unix} that might be confused with Metamath
command qualifiers.\index{Unix file names}\index{file names!Unix}

\section{Invoking Metamath}

Unix, Linux, and Mac OS X
have a command-line interface called the {\em
bash shell}.  (In Mac OS X, select the Terminal application from
Applications/Utilities.)  To invoke Metamath from the bash shell prompt,
assuming that the Metamath program is in the current directory, type
\begin{verbatim}
bash$ ./metamath
\end{verbatim}

To invoke Metamath from a Windows DOS or Command Prompt, assuming that
the Metamath program is in the current directory (or in a directory
included in the Path system environment variable), type
\begin{verbatim}
C:\metamath>metamath
\end{verbatim}

To use command-line arguments at invocation, the command-line arguments
should be a list of Metamath commands, surrounded by quotes if they
contain spaces.  In Windows, the surrounding quotes must be double (not
single) quotes.  For example, to read the database file \texttt{set.mm},
verify all proofs, and exit the program, type (under Unix)
\begin{verbatim}
bash$ ./metamath 'read set.mm' 'verify proof *' exit
\end{verbatim}
Note that in Unix, any directory path with \texttt{/}'s must be
surrounded by quotes so Metamath will not interpret the \texttt{/} as a
command qualifier.  So if \texttt{set.mm} is in the \texttt{/tmp}
directory, use for the above example
\begin{verbatim}
bash$ ./metamath 'read "/tmp/set.mm"' 'verify proof *' exit
\end{verbatim}

For convenience, if the command-line has one argument and no spaces in
the argument, the command is implicitly assumed to be \texttt{read}.  In
this one special case, \texttt{/}'s are not interpreted as command
qualifiers, so you don't need quotes around a Unix file name.  Thus
\begin{verbatim}
bash$ ./metamath /tmp/set.mm
\end{verbatim}
and
\begin{verbatim}
bash$ ./metamath "read '/tmp/set.mm'"
\end{verbatim}
are equivalent.


\section{Controlling Metamath}

The Metamath program was first developed on a {\sc vax/vms} system, and
some aspects of its command line behavior reflect this heritage.
Hopefully you will find it reasonably user-friendly once you get used to
it.

Each command line is a sequence of English-like words separated by
spaces, as in \texttt{show settings}.  Command words are not case
sensitive, and only as many letters are needed as are necessary to
eliminate ambiguity; for example, \texttt{sh se} would work for the
command \texttt{show settings}.  In some cases arguments such as file
names, statement labels, or symbol names are required; these are
case-sensitive (although file names may not be on some operating
systems).

A command line is entered by typing it in then pressing the {\em return}
({\em enter}) key.  To find out what commands are available, type
\texttt{?} at the \texttt{MM>} prompt.  To find out the choices at any
point in a command, press {\em return} and you will be prompted for
them.  The default choice (the one selected if you just press {\em
return}) is shown in brackets (\texttt{<>}).

You may also type \texttt{?} in place of a command word to force
Metamath to tell you what the choices are.  The \texttt{?} method won't
work, though, if a non-keyword argument such as a file name is expected
at that point, because the program will think that \texttt{?} is the
value of the argument.

Some commands have one or more optional qualifiers which modify the
behavior of the command.  Qualifiers are preceded by a slash
(\texttt{/}), such as in \texttt{read set.mm / verify}.  Spaces are
optional around the \texttt{/}.  If you need to use a space or
slash in a command
argument, as in a Unix file name, put single or double quotes around the
command argument.

The \texttt{open log} command will save everything you see on the
screen and is useful to help you recover should something go wrong in a
proof, or if you want to document a bug.

If a command responds with more than a screenful, you will be
prompted to \texttt{<return> to continue, Q to quit, or S to scroll to
end}.  \texttt{Q} or \texttt{q} (not case-sensitive) will complete the
command internally but will suppress further output until the next
\texttt{MM>} prompt.  \texttt{s} will suppress further pausing until the
next \texttt{MM>} prompt.  After the first screen, you are also
presented with the choice of \texttt{b} to go back a screenful.  Note
that \texttt{b} may also be entered at the \texttt{MM>} prompt
immediately after a command to scroll back through the output of that
command.

A command line enclosed in quotes is executed by your operating system.
See Section~\ref{oscmd}.

{\em Warning:} Pressing {\sc ctrl-c} will abort the Metamath program
unconditionally.  This means any unsaved work will be lost.


\subsection{\texttt{exit} Command}\index{\texttt{exit} command}

Syntax:  \texttt{exit} [\texttt{/force}]

This command exits from Metamath.  If there have been changes to the
source with the \texttt{save proof} or \texttt{save new{\char`\_}proof}
commands, you will be given an opportunity to \texttt{write source} to
permanently save the changes.

In Proof Assistant\index{Proof Assistant} mode, the \texttt{exit} command will
return to the \verb/MM>/ prompt. If there were changes to the proof, you will
be given an opportunity to \texttt{save new{\char`\_}proof}.

The \texttt{quit} command is a synonym for \texttt{exit}.

Optional qualifier:
    \texttt{/force} - Do not prompt if changes were not saved.  This qualifier is
        useful in \texttt{submit} command files (Section~\ref{sbmt})
        to ensure predictable behavior.





\subsection{\texttt{open log} Command}\index{\texttt{open log} command}
Syntax:  \texttt{open log} {\em file-name}

This command will open a log file that will store everything you see on
the screen.  It is useful to help recovery from a mistake in a long Proof
Assistant session, or to document bugs.\index{Metamath!bugs}

The log file can be closed with \texttt{close log}.  It will automatically be
closed upon exiting Metamath.



\subsection{\texttt{close log} Command}\index{\texttt{close log} command}
Syntax:  \texttt{close log}

The \texttt{close log} command closes a log file if one is open.  See
also \texttt{open log}.




\subsection{\texttt{submit} Command}\index{\texttt{submit} command}\label{sbmt}
Syntax:  \texttt{submit} {\em filename}

This command causes further command lines to be taken from the specified
file.  Note that any line beginning with an exclamation point (\texttt{!}) is
treated as a comment (i.e.\ ignored).  Also note that the scrolling
of the screen output is continuous, so you may want to open a log file
(see \texttt{open log}) to record the results that fly by on the screen.
After the lines in the file are exhausted, Metamath returns to its
normal user interface mode.

Currently, the \texttt{submit} command is not recursive.  In other words,
\texttt{submit} commands are not allowed inside a command file.


\subsection{\texttt{erase} Command}\index{\texttt{erase} command}
Syntax:  \texttt{erase}

This command will reset Metamath to its starting state, deleting any
data\-base that was \texttt{read} in.
 If there have been changes to the
source with the \texttt{save proof} or \texttt{save new{\char`\_}proof}
commands, you will be given an opportunity to \texttt{write source} to
permanently save the changes.



\subsection{\texttt{set echo} Command}\index{\texttt{set echo} command}
Syntax:  \texttt{set echo on} or \texttt{set echo off}

The \texttt{set echo on} command will cause command lines to be echoed with any
abbreviations expanded.  While learning the Metamath commands, this
feature will show you the exact command that your abbreviated input
corresponds to.



\subsection{\texttt{set scroll} Command}\index{\texttt{set scroll} command}
Syntax:  \texttt{set scroll prompted} or \texttt{set scroll continuous}

The Metamath command line interface starts off in the \texttt{prompted} mode,
which means that you will be prompted to continue or quit after each
full screen in a long listing.  In \texttt{continuous} mode, long listings will be
scrolled without pausing.

% LaTeX bug? (1) \texttt{\_} puts out different character than
% \texttt{{\char`\_}}
%  = \verb$_$  (2) \texttt{{\char`\_}} puts out garbage in \subsection
%  argument
\subsection{\texttt{set width} Command}\index{\texttt{set
width} command}
Syntax:  \texttt{set width} {\em number}

Metamath assumes the width of your screen is 79 characters (chosen
because the Command Prompt in Windows XP has a wrapping bug at column
80).  If your screen is wider or narrower, this command allows you to
change this default screen width.  A larger width is advantageous for
logging proofs to an output file to be printed on a wide printer.  A
smaller width may be necessary on some terminals; in this case, the
wrapping of the information messages may sometimes seem somewhat
unnatural, however.  In \LaTeX\index{latex@{\LaTeX}!characters per line},
there is normally a maximum of 61 characters per line with typewriter
font.  (The examples in this book were produced with 61 characters per
line.)

\subsection{\texttt{set height} Command}\index{\texttt{set
height} command}
Syntax:  \texttt{set height} {\em number}

Metamath assumes your screen height is 24 lines of characters.  If your
screen is taller or shorter, this command lets you to change the number
of lines at which the display pauses and prompts you to continue.

\subsection{\texttt{beep} Command}\index{\texttt{beep} command}

Syntax:  \texttt{beep}

This command will produce a beep.  By typing it ahead after a
long-running command has started, it will alert you that the command is
finished.  For convenience, \texttt{b} is an abbreviation for
\texttt{beep}.

Note:  If \texttt{b} is typed at the \texttt{MM>} prompt immediately
after the end of a multiple-page display paged with ``\texttt{Press
<return> for more}...'' prompts, then the \texttt{b} will back up to the
previous page rather than perform the \texttt{beep} command.
In that case you must type the unabbreviated \texttt{beep} form
of the command.

\subsection{\texttt{more} Command}\index{\texttt{more} command}

Syntax:  \texttt{more} {\em filename}

This command will display the contents of an {\sc ascii} file on your
screen.  (This command is provided for convenience but is not very
powerful.  See Section~\ref{oscmd} to invoke your operating system's
command to do this, such as the \texttt{more} command in Unix.)

\subsection{Operating System Commands}\index{operating system
command}\label{oscmd}

A line enclosed in single or double quotes will be executed by your
computer's operating system if it has a command line interface.  For
example, on a {\sc vax/vms} system,
\verb/MM> 'dir'/
will print disk directory contents.  Note that this feature will not
work on the Macintosh prior to Mac OS X, which does not have a command
line interface.

For your convenience, the trailing quote is optional.

\subsection{Size Limitations in Metamath}

In general, there are no fixed, predefined limits\index{Metamath!memory
limits} on how many labels, tokens\index{token}, statements, etc.\ that
you may have in a database file.  The Metamath program uses 32-bit
variables (64-bit on 64-bit CPUs) as indices for almost all internal
arrays, which are allocated dynamically as needed.



\section{Reading and Writing Files}

The following commands create new files:  the \texttt{open} commands;
the \texttt{write} commands; the \texttt{/html},
\texttt{/alt{\char`\_}html}, \texttt{/brief{\char`\_}html},
\texttt{/brief{\char`\_}alt{\char`\_}html} qualifiers of \texttt{show
statement}, and \texttt{midi}.  The following commands append to files
previously opened:  the \texttt{/tex} qualifier of \texttt{show proof}
and \texttt{show new{\char`\_}proof}; the \texttt{/tex} and
\texttt{/simple{\char`\_}tex} qualifiers of \texttt{show statement}; the
\texttt{close} commands; and all screen dialog between \texttt{open log}
and \texttt{close log}.

The commands that create new files will not overwrite an existing {\em
filename} but will rename the existing one to {\em
filename}\texttt{{\char`\~}1}.  An existing {\em
filename}\texttt{{\char`\~}1} is renamed {\em
filename}\texttt{{\char`\~}2}, etc.\ up to {\em
filename}\texttt{{\char`\~}9}.  An existing {\em
filename}\texttt{{\char`\~}9} is deleted.  This makes recovery from
mistakes easier but also will clutter up your directory, so occasionally
you may want to clean up (delete) these \texttt{{\char`\~}}$n$ files.


\subsection{\texttt{read} Command}\index{\texttt{read} command}
Syntax:  \texttt{read} {\em file-name} [\texttt{/verify}]

This command will read in a Metamath language source file and any included
files.  Normally it will be the first thing you do when entering Metamath.
Statement syntax is checked, but proof syntax is not checked.
Note that the file name may be enclosed in single or double quotes;
this is useful if the file name contains slashes, as might be the case
under Unix.

If you are getting an ``\texttt{?Expected VERIFY}'' error
when trying to read a Unix file name with slashes, you probably haven't
quoted it.\index{Unix file names}\index{file names!Unix}

If you are prompted for the file name (by pressing {\em return}
 after \texttt{read})
you should {\em not} put quotes around it, even if it is a Unix file name
with slashes.

Optional command qualifier:

    \texttt{/verify} - Verify all proofs as the database is read in.  This
         qualifier will slow down reading in the file.  See \texttt{verify
         proof} for more information on file error-checking.

See also \texttt{erase}.



\subsection{\texttt{write source} Command}\index{\texttt{write source} command}
Syntax:  \texttt{write source} {\em filename}
[\texttt{/rewrap}]
[\texttt{/split}]
[\texttt{/keep\_includes}]
[\texttt{/no\_versioning}]

This command will write the contents of a Metamath\index{database}
database into a file.\index{source file}

Optional command qualifiers:

\texttt{/rewrap} -
Reformats statements and comments according to the
convention used in the set.mm database.
It unwraps the
lines in the comment before each \texttt{\$a} and \texttt{\$p} statement,
then it
rewraps the line.  You should compare the output to the original
to make sure that the desired effect results; if not, go back to
the original source.  The wrapped line length honors the
\texttt{set width}
parameter currently in effect.  Note:  Text
enclosed in \texttt{<HTML>}...\texttt{</HTML>} tags is not modified by the
\texttt{/rewrap} qualifier.
Proofs themselves are not reformatted;
use \texttt{save proof * / compressed} to do that.
An isolated tilde (\~{}) is always kept on the same line as the following
symbol, so you can find all comment references to a symbol by
searching for \~{} followed by a space and that symbol
(this is useful for finding cross-references).
Incidentally, \texttt{save proof} also honors the \texttt{set width}
parameter currently in effect.

\texttt{/split} - Files included in the source using the expression
\$[ \textit{inclfile} \$] will be
written into separate files instead of being included in a single output
file.  The name of each separately written included file will be the
\textit{inclfile} argument of its inclusion command.

\texttt{/keep\_includes} - If a source file has includes but is written as a
single file by omitting \texttt{/split}, by default the included files will
be deleted (actually just renamed with a \char`\~1 suffix unless
\texttt{/no\_versioning} is specified) to prevent the possibly confusing
source duplication in both the output file and the included file.
The \texttt{/keep\_includes} qualifier will prevent this deletion.

\texttt{/no\_versioning} - Backup files suffixed with \char`\~1 are not created.


\section{Showing Status and Statements}



\subsection{\texttt{show settings} Command}\index{\texttt{show settings} command}
Syntax:  \texttt{show settings}

This command shows the state of various parameters.

\subsection{\texttt{show memory} Command}\index{\texttt{show memory} command}
Syntax:  \texttt{show memory}

This command shows the available memory left.  It is not meaningful
on most modern operating systems,
which have virtual memory.\index{Metamath!memory usage}


\subsection{\texttt{show labels} Command}\index{\texttt{show labels} command}
Syntax:  \texttt{show labels} {\em label-match} [\texttt{/all}]
   [\texttt{/linear}]

This command shows the labels of \texttt{\$a} and \texttt{\$p}
statements that match {\em label-match}.  A \verb$*$ in {label-match}
matches zero or more characters.  For example, \verb$*abc*def$ will match all
labels containing \verb$abc$ and ending with \verb$def$.

Optional command qualifiers:

   \texttt{/all} - Include matches for \texttt{\$e} and \texttt{\$f}
   statement labels.

   \texttt{/linear} - Display only one label per line.  This can be useful for
       building scripts in conjunction with the utilities under the
       \texttt{tools}\index{\texttt{tools} command} command.



\subsection{\texttt{show statement} Command}\index{\texttt{show statement} command}
Syntax:  \texttt{show statement} {\em label-match} [{\em qualifiers} (see below)]

This command provides information about a statement.  Only statements
that have labels (\texttt{\$f}\index{\texttt{\$f} statement},
\texttt{\$e}\index{\texttt{\$e} statement},
\texttt{\$a}\index{\texttt{\$a} statement}, and
\texttt{\$p}\index{\texttt{\$p} statement}) may be specified.
If {\em label-match}
contains wildcard (\verb$*$) characters, all matching statements will be
displayed in the order they occur in the database.

Optional qualifiers (only one qualifier at a time is allowed):

    \texttt{/comment} - This qualifier includes the comment that immediately
       precedes the statement.

    \texttt{/full} - Show complete information about each statement,
       and show all
       statements matching {\em label} (including \texttt{\$e}
       and \texttt{\$f} statements).

    \texttt{/tex} - This qualifier will write the statement information to the
       \LaTeX\ file previously opened with \texttt{open tex}.  See
       Section~\ref{texout}.

    \texttt{/simple{\char`\_}tex} - The same as \texttt{/tex}, except that
       \LaTeX\ macros are not used for formatting equations, allowing easier
       manual edits of the output for slide presentations, etc.

    \texttt{/html}\index{html generation@{\sc html} generation},
       \texttt{/alt{\char`\_}html}, \texttt{/brief{\char`\_}html},
       \texttt{/brief{\char`\_}alt{\char`\_}html} -
       These qualifiers invoke a special mode of
       \texttt{show statement} that
       creates a web page for the statement.  They may not be used with
       any other qualifier.  See Section~\ref{htmlout} or
       \texttt{help html} in the program.


\subsection{\texttt{search} Command}\index{\texttt{search} command}
Syntax:  search {\em label-match}
\texttt{"}{\em symbol-match}{\tt}" [\texttt{/all}] [\texttt{/comments}]
[\texttt{/join}]

This command searches all \texttt{\$a} and \texttt{\$p} statements
matching {\em label-match} for occurrences of {\em symbol-match}.  A
\verb@*@ in {\em label-match} matches any label character.  A \verb@$*@
in {\em symbol-match} matches any sequence of symbols.  The symbols in
{\em symbol-match} must be separated by white space.  The quotes
surrounding {\em symbol-match} may be single or double quotes.  For
example, \texttt{search b}\verb@* "-> $* ch"@ will list all statements
whose labels begin with \texttt{b} and contain the symbols \verb@->@ and
\texttt{ch} surrounding any symbol sequence (including no symbol
sequence).  The wildcards \texttt{?} and \texttt{\$?} are also available
to match individual characters in labels and symbols respectively; see
\texttt{help search} in the Metamath program for details on their usage.

Optional command qualifiers:

    \texttt{/all} - Also search \texttt{\$e} and \texttt{\$f} statements.

    \texttt{/comments} - Search the comment that immediately precedes each
        label-matched statement for {\em symbol-match}.  In this case
        {\em symbol-match} is an arbitrary, non-case-sensitive character
        string.  Quotes around {\em symbol-match} are optional if there
        is no ambiguity.

    \texttt{/join} - In the case of a \texttt{\$a} or \texttt{\$p} statement,
	prepend its \texttt{\$e}
	hypotheses for searching. The
	\texttt{/join} qualifier has no effect in \texttt{/comments} mode.

\section{Displaying and Verifying Proofs}


\subsection{\texttt{show proof} Command}\index{\texttt{show proof} command}
Syntax:  \texttt{show proof} {\em label-match} [{\em qualifiers} (see below)]

This command displays the proof of the specified
\texttt{\$p}\index{\texttt{\$p} statement} statement in various formats.
The {\em label-match} may contain wildcard (\verb@$*@) characters to match
multiple statements.  Without any qualifiers, only the logical steps
will be shown (i.e.\ syntax construction steps will be omitted), in an
indented format.

Most of the time, you will use
    \texttt{show proof} {\em label}
to see just the proof steps corresponding to logical inferences.

Optional command qualifiers:

    \texttt{/essential} - The proof tree is trimmed of all
        \texttt{\$f}\index{\texttt{\$f} statement} hypotheses before
        being displayed.  (This is the default, and it is redundant to
        specify it.)

    \texttt{/all} - the proof tree is not trimmed of all \texttt{\$f} hypotheses before
        being displayed.  \texttt{/essential} and \texttt{/all} are mutually exclusive.

    \texttt{/from{\char`\_}step} {\em step} - The display starts at the specified
        step.  If
        this qualifier is omitted, the display starts at the first step.

    \texttt{/to{\char`\_}step} {\em step} - The display ends at the specified
        step.  If this
        qualifier is omitted, the display ends at the last step.

    \texttt{/tree{\char`\_}depth} {\em number} - Only
         steps at less than the specified proof
        tree depth are displayed.  Sometimes useful for obtaining an overview of
        the proof.

    \texttt{/reverse} - The steps are displayed in reverse order.

    \texttt{/renumber} - When used with \texttt{/essential}, the steps are renumbered
        to correspond only to the essential steps.

    \texttt{/tex} - The proof is converted to \LaTeX\ and\index{latex@{\LaTeX}}
        stored in the file opened
        with \texttt{open tex}.  See Section~\ref{texout} or
        \texttt{help tex} in the program.

    \texttt{/lemmon} - The proof is displayed in a non-indented format known
        as Lemmon style, with explicit previous step number references.
        If this qualifier is omitted, steps are indented in a tree format.

    \texttt{/start{\char`\_}column} {\em number} - Overrides the default column
        (16)
        at which the formula display starts in a Lemmon-style display.  May be
        used only in conjunction with \texttt{/lemmon}.

    \texttt{/normal} - The proof is displayed in normal format suitable for
        inclusion in a Metamath source file.  May not be used with any other
        qualifier.

    \texttt{/compressed} - The proof is displayed in compressed format
        suitable for inclusion in a Metamath source file.  May not be used with
        any other qualifier.

    \texttt{/statement{\char`\_}summary} - Summarizes all statements (like a
        brief \texttt{show statement})
        used by the proof.  It may not be used with any other qualifier
        except \texttt{/essential}.

    \texttt{/detailed{\char`\_}step} {\em step} - Shows the details of what is
        happening at
        a specific proof step.  May not be used with any other qualifier.
        The {\em step} is the step number shown when displaying a
        proof without the \texttt{/renumber} qualifier.


\subsection{\texttt{show usage} Command}\index{\texttt{show usage} command}
Syntax:  \texttt{show usage} {\em label-match} [\texttt{/recursive}]

This command lists the statements whose proofs make direct reference to
the statement specified.

Optional command qualifier:

    \texttt{/recursive} - Also include statements whose proofs ultimately
        depend on the statement specified.



\subsection{\texttt{show trace\_back} Command}\index{\texttt{show
       trace{\char`\_}back} command}
Syntax:  \texttt{show trace{\char`\_}back} {\em label-match} [\texttt{/essential}] [\texttt{/axioms}]
    [\texttt{/tree}] [\texttt{/depth} {\em number}]

This command lists all statements that the proof of the \texttt{\$p}
statement(s) specified by {\em label-match} depends on.

Optional command qualifiers:

    \texttt{/essential} - Restrict the trace-back to \texttt{\$e}
        \index{\texttt{\$e} statement} hypotheses of proof trees.

    \texttt{/axioms} - List only the axioms that the proof ultimately depends on.

    \texttt{/tree} - Display the trace-back in an indented tree format.

    \texttt{/depth} {\em number} - Restrict the \texttt{/tree} trace-back to the
        specified indentation depth.

    \texttt{/count{\char`\_}steps} - Count the number of steps the proof has
       all the way back to axioms.  If \texttt{/essential} is specified,
       expansions of variable-type hypotheses (syntax constructions) are not counted.

\subsection{\texttt{verify proof} Command}\index{\texttt{verify proof} command}
Syntax:  \texttt{verify proof} {\em label-match} [\texttt{/syntax{\char`\_}only}]

This command verifies the proofs of the specified statements.  {\em
label-match} may contain wild card characters (\texttt{*}) to verify
more than one proof; for example \verb/*abc*def/ will match all labels
containing \texttt{abc} and ending with \texttt{def}.
The command \texttt{verify proof *} will verify all proofs in the database.

Optional command qualifier:

    \texttt{/syntax{\char`\_}only} - This qualifier will perform a check of syntax
        and RPN
        stack violations only.  It will not verify that the proof is
        correct.  This qualifier is useful for quickly determining which
        proofs are incomplete (i.e.\ are under development and have \texttt{?}'s
        in them).

{\em Note:} \texttt{read}, followed by \texttt{verify proof *}, will ensure
 the database is free
from errors in the Metamath language but will not check the markup notation
in comments.
You can also check the markup notation by running \texttt{verify markup *}
as discussed in Section~\ref{verifymarkup}; see also the discussion
on generating {\sc HTML} in Section~\ref{htmlout}.

\subsection{\texttt{verify markup} Command}\index{\texttt{verify markup} command}\label{verifymarkup}
Syntax:  \texttt{verify markup} {\em label-match}
[\texttt{/date{\char`\_}skip}]
[\texttt{/top{\char`\_}date{\char`\_}skip}]
[\texttt{/file{\char`\_}skip}]
[\texttt{/verbose}]

This command checks comment markup and other informal conventions we have
adopted.  It error-checks the latexdef, htmldef, and althtmldef statements
in the \texttt{\$t} statement of a Metamath source file.\index{error checking}
It error-checks any \texttt{`}...\texttt{`},
\texttt{\char`\~}~\textit{label},
and bibliographic markups in statement descriptions.
It checks that
\texttt{\$p} and \texttt{\$a} statements
have the same content when their labels start with
``ax'' and ``ax-'' respectively but are otherwise identical, for example
ax4 and ax-4.
It also verifies the date consistency of ``(Contributed by...),''
``(Revised by...),'' and ``(Proof shortened by...)'' tags in the comment
above each \texttt{\$a} and \texttt{\$p} statement.

Optional command qualifiers:

    \texttt{/date{\char`\_}skip} - This qualifier will
        skip date consistency checking,
        which is usually not required for databases other than
	\texttt{set.mm}.

    \texttt{/top{\char`\_}date{\char`\_}skip} - This qualifier will check date consistency except
        that the version date at the top of the database file will not
        be checked.  Only one of
        \texttt{/date{\char`\_}skip} and
        \texttt{/top{\char`\_}date{\char`\_}skip} may be
        specified.

    \texttt{/file{\char`\_}skip} - This qualifier will skip checks that require
        external files to be present, such as checking GIF existence and
        bibliographic links to mmset.html or equivalent.  It is useful
        for doing a quick check from a directory without these files.

    \texttt{/verbose} - Provides more information.  Currently it provides a list
        of axXXX vs. ax-XXX matches.

\subsection{\texttt{save proof} Command}\index{\texttt{save proof} command}
Syntax:  \texttt{save proof} {\em label-match} [\texttt{/normal}]
   [\texttt{/compressed}]

The \texttt{save proof} command will reformat a proof in one of two formats and
replace the existing proof in the source buffer\index{source
buffer}.  It is useful for
converting between proof formats.  Note that a proof will not be
permanently saved until a \texttt{write source} command is issued.

Optional command qualifiers:

    \texttt{/normal} - The proof is saved in the normal format (i.e., as a
        sequence
        of labels, which is the defined format of the basic Metamath
        language).\index{basic language}  This is the default format that
        is used if a qualifier
        is omitted.

    \texttt{/compressed} - The proof is saved in the compressed format which
        reduces storage requirements for a database.
        See Appendix~\ref{compressed}.




\section{Creating Proofs}\label{pfcommands}\index{Proof Assistant}

Before using the Proof Assistant, you must add a \texttt{\$p} to your
source file (using a text editor) containing the statement you want to
prove.  Its proof should consist of a single \texttt{?}, meaning
``unknown step.''  Example:
\begin{verbatim}
equid $p x = x $= ? $.
\end{verbatim}

To enter the Proof assistant, type \texttt{prove} {\em label}, e.g.
\texttt{prove equid}.  Metamath will respond with the \texttt{MM-PA>}
prompt.

Proofs are created working backwards from the statement being proved,
primarily using a series of \texttt{assign} commands.  A proof is
complete when all steps are assigned to statements and all steps are
unified and completely known.  During the creation of a proof, Metamath
will allow only operations that are legal based on what is known up to
that point.  For example, it will not allow an \texttt{assign} of a
statement that cannot be unified with the unknown proof step being
assigned.

{\em Important:}
The Proof Assistant is
{\em not} a tool to help you discover proofs.  It is just a tool to help
you add them to the database.  For a tutorial read
Section~\ref{frstprf}.
To practice using the Proof Assistant, you may
want to \texttt{prove} an existing theorem, then delete all steps with
\texttt{delete all}, then re-create it with the Proof Assistant while
looking at its proof display (before deletion).
You might want to figure out your first few proofs completely
and write them down by hand, before using the Proof Assistant, though
not everyone finds that effective.

{\em Important:}
The \texttt{undo} command if very helpful when entering a proof, because
it allows you to undo a previously-entered step.
In addition, we suggest that you
keep track of your work with a log file (\texttt{open
log}) and save it frequently (\texttt{save new{\char`\_}proof},
\texttt{write source}).
You can use \texttt{delete} to reverse an \texttt{assign}.
You can also do \texttt{delete floating{\char`\_}hypotheses}, then
\texttt{initialize all}, then \texttt{unify all /interactive} to
reinitialize bad unifications made accidentally or by bad
\texttt{assign}s.  You cannot reverse a \texttt{delete} except by
a relevant \texttt{undo} or using
\texttt{exit /force} then reentering the Proof Assistant to recover from
the last \texttt{save new{\char`\_}proof}.

The following commands are available in the Proof Assistant (at the
\texttt{MM-PA>} prompt) to help you create your proof.  See the
individual commands for more detail.

\begin{itemize}
\item[]
    \texttt{show new{\char`\_}proof} [\texttt{/all},...] - Displays the
        proof in progress.  You will use this command a lot; see \texttt{help
        show new{\char`\_}proof} to become familiar with its qualifiers.  The
        qualifiers \texttt{/unknown} and \texttt{/not{\char`\_}unified} are
        useful for seeing the work remaining to be done.  The combination
        \texttt{/all/unknown} is useful for identifying dummy variables that must be
        assigned, or attempts to use illegal syntax, when \texttt{improve all}
        is unable to complete the syntax constructions.  Unknown variables are
        shown as \texttt{\$1}, \texttt{\$2},...
\item[]
    \texttt{assign} {\em step} {\em label} - Assigns an unknown {\em step}
        number with the statement
        specified by {\em label}.
\item[]
    \texttt{let variable} {\em variable}
        \texttt{= "}{\em symbol sequence}\texttt{"}
          - Forces a symbol
        sequence to replace an unknown variable (such as \texttt{\$1}) in a proof.
        It is useful
        for helping difficult unifications, and it is necessary when you have
        dummy variables that eventually must be assigned a name.
\item[]
    \texttt{let step} {\em step} \texttt{= "}{\em symbol sequence}\texttt{"} -
          Forces a symbol sequence
        to replace the contents of a proof step, provided it can be
        unified with the existing step contents.  (I rarely use this.)
\item[]
    \texttt{unify step} {\em step} (or \texttt{unify all}) - Unifies
        the source and target of
        a step.  If you specify a specific step, you will be prompted
        to select among the unifications that are possible.  If you
        specify \texttt{all}, all steps with unique unifications, but only
        those steps, will be
        unified.  \texttt{unify all /interactive} goes through all non-unified
        steps.
\item[]
    \texttt{initialize} {\em step} (or \texttt{all}) - De-unifies the target and source of
        a step (or all steps), as well as the hypotheses of the source,
        and makes all variables in the source unknown.  Useful to recover from
        an \texttt{assign} or \texttt{let} mistake that
        resulted in incorrect unifications.
\item[]
    \texttt{delete} {\em step} (or \texttt{all} or \texttt{floating{\char`\_}hypotheses}) -
        Deletes the specified
        step(s).  \texttt{delete floating{\char`\_}hypotheses}, then \texttt{initialize all}, then
        \texttt{unify all /interactive} is useful for recovering from mistakes
        where incorrect unifications assigned wrong math symbol strings to
        variables.
\item[]
    \texttt{improve} {\em step} (or \texttt{all}) -
      Automatically creates a proof for steps (with no unknown variables)
      whose proof requires no statements with \texttt{\$e} hypotheses.  Useful
      for filling in proofs of \texttt{\$f} hypotheses.  The \texttt{/depth}
      qualifier will also try statements whose \texttt{\$e} hypotheses contain
      no new variables.  {\em Warning:} Save your work (with \texttt{save
      new{\char`\_}proof} then \texttt{write source}) before using
      \texttt{/depth = 2} or greater, since the search time grows
      exponentially and may never terminate in a reasonable time, and you
      cannot interrupt the search.  I have found that it is rare for
      \texttt{/depth = 3} or greater to be useful.
 \item[]
    \texttt{save new{\char`\_}proof} - Saves the proof in progress in the program's
        internal database buffer.  To save it permanently into the database file,
        use \texttt{write source} after
        \texttt{save new{\char`\_}proof}.  To revert to the last
        \texttt{save new{\char`\_}proof},
        \texttt{exit /force} from the Proof Assistant then re-enter the Proof
        Assistant.
 \item[]
    \texttt{match step} {\em step} (or \texttt{match all}) - Shows what
        statements are
        possibilities for the \texttt{assign} statement. (This command
        is not very
        useful in its present form and hopefully will be improved
        eventually.  In the meantime, use the \texttt{search} statement for
        candidates matching specific math token combinations.)
 \item[]
 \texttt{minimize{\char`\_}with}\index{\texttt{minimize{\char`\_}with} command}
% 3/10/07 Note: line-breaking the above results in duplicate index entries
     - After a proof is complete, this command will attempt
        to match other database theorems to the proof to see if the proof
        size can be reduced as a result.  See \texttt{help
        minimize{\char`\_}with} in the
        Metamath program for its usage.
 \item[]
 \texttt{undo}\index{\texttt{undo} command}
    - Undo the effect of a proof-changing command (all but the
      \texttt{show} and \texttt{save} commands above).
 \item[]
 \texttt{redo}\index{\texttt{redo} command}
    - Reverse the previous \texttt{undo}.
\end{itemize}

The following commands set parameters that may be relevant to your proof.
Consult the individual \texttt{help set}... commands.
\begin{itemize}
   \item[] \texttt{set unification{\char`\_}timeout}
 \item[]
    \texttt{set search{\char`\_}limit}
  \item[]
    \texttt{set empty{\char`\_}substitution} - note that default is \texttt{off}
\end{itemize}

Type \texttt{exit} to exit the \texttt{MM-PA>}
 prompt and get back to the \texttt{MM>} prompt.
Another \texttt{exit} will then get you out of Metamath.



\subsection{\texttt{prove} Command}\index{\texttt{prove} command}
Syntax:  \texttt{prove} {\em label}

This command will enter the Proof Assistant\index{Proof Assistant}, which will
allow you to create or edit the proof of the specified statement.
The command-line prompt will change from \texttt{MM>} to \texttt{MM-PA>}.

Note:  In the present version of
Metamath\index{Metamath!limitations of version 0.07.30}, the Proof
Assistant does not verify that \texttt{\$d}\index{\texttt{\$d}
statement} restrictions are met as a proof is being built.  After you
have completed a proof, you should type \texttt{save new{\char`\_}proof}
followed by \texttt{verify proof} {\em label} (where {\em label} is the
statement you are proving with the \texttt{prove} command) to verify the
\texttt{\$d} restrictions.

See also: \texttt{exit}

\subsection{\texttt{set unification\_timeout} Command}\index{\texttt{set
unification{\char`\_}timeout} command}
Syntax:  \texttt{set unification{\char`\_}timeout} {\em number}

(This command is available outside the Proof Assistant but affects the
Proof Assistant\index{Proof Assistant} only.)

Sometimes the Proof Assistant will inform you that a unification
time-out occurred.  This may happen when you try to \texttt{unify}
formulas with many temporary variables\index{temporary variable}
(\texttt{\$1}, \texttt{\$2}, etc.), since the time to compute all possible
unifications may grow exponentially with the number of variables.  If
you want Metamath to try harder (and you're willing to wait longer) you
may increase this parameter.  \texttt{show settings} will show you the
current value.

\subsection{\texttt{set empty\_substitution} Command}\index{\texttt{set
empty{\char`\_}substitution} command}
% These long names can't break well in narrow mode, and even "sloppy"
% is not enough. Work around this by not demanding justification.
\begin{flushleft}
Syntax:  \texttt{set empty{\char`\_}substitution on} or \texttt{set
empty{\char`\_}substitution off}
\end{flushleft}

(This command is available outside the Proof Assistant but affects the
Proof Assistant\index{Proof Assistant} only.)

The Metamath language allows variables to be
substituted\index{substitution!variable}\index{variable substitution}
with empty symbol sequences\index{empty substitution}.  However, in many
formal systems\index{formal system} this will never happen in a valid
proof.  Allowing for this possibility increases the likelihood of
ambiguous unifications\index{ambiguous
unification}\index{unification!ambiguous} during proof creation.
The default is that
empty substitutions are not allowed; for formal systems requiring them,
you must \texttt{set empty{\char`\_}substitution on}.
(An example where this must be \texttt{on}
would be a system that implements a Deduction Rule and in
which deductions from empty assumption lists would be permissible.  The
MIU-system\index{MIU-system} described in Appendix~\ref{MIU} is another
example.)
Note that empty substitutions are
always permissible in proof verification (VERIFY PROOF...) outside the
Proof Assistant.  (See the MIU system in the Metamath book for an example
of a system needing empty substitutions; another example would be a
system that implements a Deduction Rule and in which deductions from
empty assumption lists would be permissible.)

It is better to leave this \texttt{off} when working with \texttt{set.mm}.
Note that this command does not affect the way proofs are verified with
the \texttt{verify proof} command.  Outside of the Proof Assistant,
substitution of empty sequences for math symbols is always allowed.

\subsection{\texttt{set search\_limit} Command}\index{\texttt{set
search{\char`\_}limit} command} Syntax:  \texttt{set search{\char`\_}limit} {\em
number}

(This command is available outside the Proof Assistant but affects the
Proof Assistant\index{Proof Assistant} only.)

This command sets a parameter that determines when the \texttt{improve} command
in Proof Assistant mode gives up.  If you want \texttt{improve} to search harder,
you may increase it.  The \texttt{show settings} command tells you its current
value.


\subsection{\texttt{show new\_proof} Command}\index{\texttt{show
new{\char`\_}proof} command}
Syntax:  \texttt{show new{\char`\_}proof} [{\em
qualifiers} (see below)]

This command (available only in Proof Assistant mode) displays the proof
in progress.  It is identical to the \texttt{show proof} command, except that
there is no statement argument (since it is the statement being proved) and
the following qualifiers are not available:

    \texttt{/statement{\char`\_}summary}

    \texttt{/detailed{\char`\_}step}

Also, the following additional qualifiers are available:

    \texttt{/unknown} - Shows only steps that have no statement assigned.

    \texttt{/not{\char`\_}unified} - Shows only steps that have not been unified.

Note that \texttt{/essential}, \texttt{/depth}, \texttt{/unknown}, and
\texttt{/not{\char`\_}unified} may be
used in any combination; each of them effectively filters out additional
steps from the proof display.

See also:  \texttt{show proof}






\subsection{\texttt{assign} Command}\index{\texttt{assign} command}
Syntax:   \texttt{assign} {\em step} {\em label} [\texttt{/no{\char`\_}unify}]

   and:   \texttt{assign first} {\em label}

   and:   \texttt{assign last} {\em label}


This command, available in the Proof Assistant only, assigns an unknown
step (one with \texttt{?} in the \texttt{show new{\char`\_}proof}
listing) with the statement specified by {\em label}.  The assignment
will not be allowed if the statement cannot be unified with the step.

If \texttt{last} is specified instead of {\em step} number, the last
step that is shown by \texttt{show new{\char`\_}proof /unknown} will be
used.  This can be useful for building a proof with a command file (see
\texttt{help submit}).  It also makes building proofs faster when you know
the assignment for the last step.

If \texttt{first} is specified instead of {\em step} number, the first
step that is shown by \texttt{show new{\char`\_}proof /unknown} will be
used.

If {\em step} is zero or negative, the -{\em step}th from last unknown
step, as shown by \texttt{show new{\char`\_}proof /unknown}, will be
used.  \texttt{assign -1} {\em label} will assign the penultimate
unknown step, \texttt{assign -2} {\em label} the antepenultimate, and
\texttt{assign 0} {\em label} is the same as \texttt{assign last} {\em
label}.

Optional command qualifier:

    \texttt{/no{\char`\_}unify} - do not prompt user to select a unification if there is
        more than one possibility.  This is useful for noninteractive
        command files.  Later, the user can \texttt{unify all /interactive}.
        (The assignment will still be automatically unified if there is only
        one possibility and will be refused if unification is not possible.)



\subsection{\texttt{match} Command}\index{\texttt{match} command}
Syntax:  \texttt{match step} {\em step} [\texttt{/max{\char`\_}essential{\char`\_}hyp}
{\em number}]

    and:  \texttt{match all} [\texttt{/essential}]
          [\texttt{/max{\char`\_}essential{\char`\_}hyp} {\em number}]

This command, available in the Proof Assistant only, shows what
statements can be unified with the specified step(s).  {\em Note:} In
its current form, this command is not very useful because of the large
number of matches it reports.  I rarely use it.
It may be enhanced in the future.

Optional command qualifiers:

    \texttt{/max{\char`\_}essential{\char`\_}hyp} {\em number} - filters out
        of the list any statements
        with more than the specified number of
        \texttt{\$e}\index{\texttt{\$e} statement} hypotheses

    \texttt{/essential{\char`\_}only} - in the \texttt{match all} statement, only
        the steps that
        would be listed in the \texttt{show new{\char`\_}proof /essential} display are
        matched.



\subsection{\texttt{let} Command}\index{\texttt{let} command}
Syntax: \texttt{let variable} {\em variable} = \verb/"/{\em symbol-sequence}\verb/"/

  and: \texttt{let step} {\em step} = \verb/"/{\em symbol-sequence}\verb/"/

These commands, available in the Proof Assistant\index{Proof Assistant}
only, assign a temporary variable\index{temporary variable} or unknown
step with a specific symbol sequence.  They are useful in the middle of
creating a proof, when you know what should be in the proof step but the
unification algorithm doesn't yet have enough information to completely
specify the temporary variables.  A ``temporary variable'' is one that
has the form \texttt{\$}{\em nn} in the proof display, such as
\texttt{\$1}, \texttt{\$2}, etc.  The {\em symbol-sequence} may contain
other unknown variables if desired.  Examples:

    \verb/let variable $32 = "A = B"/

    \verb/let variable $32 = "A = $35"/

    \verb/let step 10 = '|- x = x'/

    \verb/let step -2 = "|- ( $7 = ph )"/

Any symbol sequence will be accepted for the \texttt{let variable}
command.  Only those symbol sequences that can be unified with the step
will be accepted for \texttt{let step}.

The \texttt{let} commands ``zap'' the proof with information that can
only be verified when the proof is built up further.  If you make an
error, the command sequence \texttt{delete
floating{\char`\_}hypotheses}, \texttt{initialize all}, and
\texttt{unify all /interactive} will undo a bad \texttt{let} assignment.

If {\em step} is zero or negative, the -{\em step}th from last unknown
step, as shown by \texttt{show new{\char`\_}proof /unknown}, will be
used.  The command \texttt{let step 0} = \verb/"/{\em
symbol-sequence}\verb/"/ will use the last unknown step, \texttt{let
step -1} = \verb/"/{\em symbol-sequence}\verb/"/ the penultimate, etc.
If {\em step} is positive, \texttt{let step} may be used to assign known
(in the sense of having previously been assigned a label with
\texttt{assign}) as well as unknown steps.

Either single or double quotes can surround the {\em symbol-sequence} as
long as they are different from any quotes inside a {\em
symbol-sequence}.  If {\em symbol-sequence} contains both kinds of
quotes, see the instructions at the end of \texttt{help let} in the
Metamath program.


\subsection{\texttt{unify} Command}\index{\texttt{unify} command}
Syntax:  \texttt{unify step} {\em step}

      and:   \texttt{unify all} [\texttt{/interactive}]

These commands, available in the Proof Assistant only, unify the source
and target of the specified step(s). If you specify a specific step, you
will be prompted to select among the unifications that are possible.  If
you specify \texttt{all}, only those steps with unique unifications will be
unified.

Optional command qualifier for \texttt{unify all}:

    \texttt{/interactive} - You will be prompted to select among the
        unifications
        that are possible for any steps that do not have unique
        unifications.  (Otherwise \texttt{unify all} will bypass these.)

See also \texttt{set unification{\char`\_}timeout}.  The default is
100000, but increasing it to 1000000 can help difficult cases.  Manually
assigning some or all of the unknown variables with the \texttt{let
variable} command also helps difficult cases.



\subsection{\texttt{initialize} Command}\index{\texttt{initialize} command}
Syntax:  \texttt{initialize step} {\em step}

    and: \texttt{initialize all}

These commands, available in the Proof Assistant\index{Proof Assistant}
only, ``de-unify'' the target and source of a step (or all steps), as
well as the hypotheses of the source, and makes all variables in the
source and the source's hypotheses unknown.  This command is useful to
help recover from incorrect unifications that resulted from an incorrect
\texttt{assign}, \texttt{let}, or unification choice.  Part or all of
the command sequence \texttt{delete floating{\char`\_}hypotheses},
\texttt{initialize all}, and \texttt{unify all /interactive} will recover
from incorrect unifications.

See also:  \texttt{unify} and \texttt{delete}



\subsection{\texttt{delete} Command}\index{\texttt{delete} command}
Syntax:  \texttt{delete step} {\em step}

   and:      \texttt{delete all} -- {\em Warning: dangerous!}

   and:      \texttt{delete floating{\char`\_}hypotheses}

These commands are available in the Proof Assistant only.  The
\texttt{delete step} command deletes the proof tree section that
branches off of the specified step and makes the step become unknown.
\texttt{delete all} is equivalent to \texttt{delete step} {\em step}
where {\em step} is the last step in the proof (i.e.\ the beginning of
the proof tree).

In most cases the \texttt{undo} command is the best way to undo
a previous step.
An alternative is to salvage your last \texttt{save
new{\char`\_}proof} by exiting and reentering the Proof Assistant.
For this to work, keep a log file open to record your work
and to do \texttt{save new{\char`\_}proof} frequently, especially before
\texttt{delete}.

\texttt{delete floating{\char`\_}hypotheses} will delete all sections of
the proof that branch off of \texttt{\$f}\index{\texttt{\$f} statement}
statements.  It is sometimes useful to do this before an
\texttt{initialize} command to recover from an error.  Note that once a
proof step with a \texttt{\$f} hypothesis as the target is completely
known, the \texttt{improve} command can usually fill in the proof for
that step.  Unlike the deletion of logical steps, \texttt{delete
floating{\char`\_}hypotheses} is a relatively safe command that is
usually easy to recover from.



\subsection{\texttt{improve} Command}\index{\texttt{improve} command}
\label{improve}
Syntax:  \texttt{improve} {\em step} [\texttt{/depth} {\em number}]
                                               [\texttt{/no{\char`\_}distinct}]

   and:   \texttt{improve first} [\texttt{/depth} {\em number}]
                                              [\texttt{/no{\char`\_}distinct}]

   and:   \texttt{improve last} [\texttt{/depth} {\em number}]
                                              [\texttt{/no{\char`\_}distinct}]

   and:   \texttt{improve all} [\texttt{/depth} {\em number}]
                                              [\texttt{/no{\char`\_}distinct}]

These commands, available in the Proof Assistant\index{Proof Assistant}
only, try to find proofs automatically for unknown steps whose symbol
sequences are completely known.  They are primarily useful for filling in
proofs of \texttt{\$f}\index{\texttt{\$f} statement} hypotheses.  The
search will be restricted to statements having no
\texttt{\$e}\index{\texttt{\$e} statement} hypotheses.

\begin{sloppypar} % narrow
Note:  If memory is limited, \texttt{improve all} on a large proof may
overflow memory.  If you use \texttt{set unification{\char`\_}timeout 1}
before \texttt{improve all}, there will usually be sufficient
improvement to easily recover and completely \texttt{improve} the proof
later on a larger computer.  Warning:  Once memory has overflowed, there
is no recovery.  If in doubt, save the intermediate proof (\texttt{save
new{\char`\_}proof} then \texttt{write source}) before \texttt{improve
all}.
\end{sloppypar}

If \texttt{last} is specified instead of {\em step} number, the last
step that is shown by \texttt{show new{\char`\_}proof /unknown} will be
used.

If \texttt{first} is specified instead of {\em step} number, the first
step that is shown by \texttt{show new{\char`\_}proof /unknown} will be
used.

If {\em step} is zero or negative, the -{\em step}th from last unknown
step, as shown by \texttt{show new{\char`\_}proof /unknown}, will be
used.  \texttt{improve -1} will use the penultimate
unknown step, \texttt{improve -2} {\em label} the antepenultimate, and
\texttt{improve 0} is the same as \texttt{improve last}.

Optional command qualifier:

    \texttt{/depth} {\em number} - This qualifier will cause the search
        to include
        statements with \texttt{\$e} hypotheses (but no new variables in
        the \texttt{\$e}
        hypotheses), provided that the backtracking has not exceeded the
        specified depth. {\em Warning:}  Try \texttt{/depth 1},
        then \texttt{2}, then \texttt{3}, etc.
        in sequence because of possible exponential blowups.  Save your
        work before trying \texttt{/depth} greater than \texttt{1}!

    \texttt{/no{\char`\_}distinct} - Skip trial statements that have
        \texttt{\$d}\index{\texttt{\$d} statement} requirements.
        This qualifier will prevent assignments that might violate \texttt{\$d}
        requirements but it also could miss possible legal assignments.

See also: \texttt{set search{\char`\_}limit}

\subsection{\texttt{save new\_proof} Command}\index{\texttt{save
new{\char`\_}proof} command}
Syntax:  \texttt{save new{\char`\_}proof} {\em label} [\texttt{/normal}]
   [\texttt{/compressed}]

The \texttt{save new{\char`\_}proof} command is available in the Proof
Assistant only.  It saves the proof in progress in the source
buffer\index{source buffer}.  \texttt{save new{\char`\_}proof} may be
used to save a completed proof, or it may be used to save a proof in
progress in order to work on it later.  If an incomplete proof is saved,
any user assignments with \texttt{let step} or \texttt{let variable}
will be lost, as will any ambiguous unifications\index{ambiguous
unification}\index{unification!ambiguous} that were resolved manually.
To help make recovery easier, it can be helpful to \texttt{improve all}
before \texttt{save new{\char`\_}proof} so that the incomplete proof
will have as much information as possible.

Note that the proof will not be permanently saved until a \texttt{write
source} command is issued.

Optional command qualifiers:

    \texttt{/normal} - The proof is saved in the normal format (i.e., as a
        sequence of labels, which is the defined format of the basic Metamath
        language).\index{basic language}  This is the default format that
        is used if a qualifier is omitted.

    \texttt{/compressed} - The proof is saved in the compressed format, which
        reduces storage requirements for a database.
        (See Appendix~\ref{compressed}.)


\section{Creating \LaTeX\ Output}\label{texout}\index{latex@{\LaTeX}}

You can generate \LaTeX\ output given the
information in a database.
The database must already include the necessary typesetting information
(see section \ref{tcomment} for how to provide this information).

The \texttt{show statement} and \texttt{show proof} commands each have a
special \texttt{/tex} command qualifier that produces \LaTeX\ output.
(The \texttt{show statement} command also has the
\texttt{/simple{\char`\_}tex} qualifier for output that is easier to
edit by hand.)  Before you can use them, you must open a \LaTeX\ file to
which to send their output.  A typical complete session will use this
sequence of Metamath commands:

\begin{verbatim}
read set.mm
open tex example.tex
show statement a1i /tex
show proof a1i /all/lemmon/renumber/tex
show statement uneq2 /tex
show proof uneq2 /all/lemmon/renumber/tex
close tex
\end{verbatim}

See Section~\ref{mathcomments} for information on comment markup and
Appendix~\ref{ASCII} for information on how math symbol translation is
specified.

To format and print the \LaTeX\ source, you will need the \LaTeX\
program, which is standard on most Linux installations and available for
Windows.  On Linux, in order to create a {\sc pdf} file, you will
typically type at the shell prompt
\begin{verbatim}
$ pdflatex example.tex
\end{verbatim}

\subsection{\texttt{open tex} Command}\index{\texttt{open tex} command}
Syntax:  \texttt{open tex} {\em file-name} [\texttt{/no{\char`\_}header}]

This command opens a file for writing \LaTeX\
source\index{latex@{\LaTeX}} and writes a \LaTeX\ header to the file.
\LaTeX\ source can be written with the \texttt{show proof}, \texttt{show
new{\char`\_}proof}, and \texttt{show statement} commands using the
\texttt{/tex} qualifier.

The mapping to \LaTeX\ symbols is defined in a special comment
containing a \texttt{\$t} token, described in Appendix~\ref{ASCII}.

There is an optional command qualifier:

    \texttt{/no{\char`\_}header} - This qualifier prevents a standard
        \LaTeX\ header and trailer
        from being included with the output \LaTeX\ code.


\subsection{\texttt{close tex} Command}\index{\texttt{close tex} command}
Syntax:  \texttt{close tex}

This command writes a trailer to any \LaTeX\ file\index{latex@{\LaTeX}}
that was opened with \texttt{open tex} (unless
\texttt{/no{\char`\_}header} was used with \texttt{open tex}) and closes
the \LaTeX\ file.


\section{Creating {\sc HTML} Output}\label{htmlout}

You can generate {\sc html} web pages given the
information in a database.
The database must already include the necessary typesetting information
(see section \ref{tcomment} for how to provide this information).
The ability to produce {\sc html} web pages was added in Metamath version
0.07.30.

To create an {\sc html} output file(s) for \texttt{\$a} or \texttt{\$p}
statement(s), use
\begin{quote}
    \texttt{show statement} {\em label-match} \texttt{/html}
\end{quote}
The output file will be named {\em label-match}\texttt{.html}
for each match.  When {\em
label-match} has wildcard (\texttt{*}) characters, all statements with
matching labels will have {\sc html} files produced for them.  Also,
when {\em label-match} has a wildcard (\texttt{*}) character, two additional
files, \texttt{mmdefinitions.html} and \texttt{mmascii.html} will be
produced.  To produce {\em only} these two additional files, you can use
\texttt{?*}, which will not match any statement label, in place of {\em
label-match}.

There are three other qualifiers for \texttt{show statement} that also
generate {\sc HTML} code.  These are \texttt{/alt{\char`\_}html},
\texttt{/brief{\char`\_}html}, and
\texttt{/brief{\char`\_}alt{\char`\_}html}, and are described in the
next section.

The command
\begin{quote}
    \texttt{show statement} {\em label-match} \texttt{/alt{\char`\_}html}
\end{quote}
does the same as \texttt{show statement} {\em label-match} \texttt{/html},
except that the {\sc html} code for the symbols is taken from
\texttt{althtmldef} statements instead of \texttt{htmldef} statements in
the \texttt{\$t} comment.

The command
\begin{verbatim}
show statement * /brief_html
\end{verbatim}
invokes a special mode that just produces definition and theorem lists
accompanied by their symbol strings, in a format suitable for copying and
pasting into another web page (such as the tutorial pages on the
Metamath web site).

Finally, the command
\begin{verbatim}
show statement * /brief_alt_html
\end{verbatim}
does the same as \texttt{show statement * / brief{\char`\_}html}
for the alternate {\sc html}
symbol representation.

A statement's comment can include a special notation that provides a
certain amount of control over the {\sc HTML} version of the comment.  See
Section~\ref{mathcomments} (p.~\pageref{mathcomments}) for the comment
markup features.

The \texttt{write theorem{\char`\_}list} and \texttt{write bibliography}
commands, which are described below, provide as a side effect complete
error checking for all of the features described in this
section.\index{error checking}

\subsection{\texttt{write theorem\_list}
Command}\index{\texttt{write theorem{\char`\_}list} command}
Syntax:  \texttt{write theorem{\char`\_}list}
[\texttt{/theorems{\char`\_}per{\char`\_}page} {\em number}]

This command writes a list of all of the \texttt{\$a} and \texttt{\$p}
statements in the database into a web page file
 called \texttt{mmtheorems.html}.
When additional files are needed, they are called
\texttt{mmtheorems2.html}, \texttt{mmtheorems3.html}, etc.

Optional command qualifier:

    \texttt{/theorems{\char`\_}per{\char`\_}page} {\em number} -
 This qualifier specifies the number of statements to
        write per web page.  The default is 100.

{\em Note:} In version 0.07.30\index{Metamath!limitations of version
0.07.30} of Metamath, the ``Nearby theorems'' links on the individual
web pages presuppose 100 theorems per page when linking to the theorem
list pages.  Therefore the \texttt{/theorems{\char`\_}per{\char`\_}page}
qualifier, if it specifies a number other than 100, will cause the
individual web pages to be out of sync and should not be used to
generate the main theorem list for the web site.  This is expected to be
fixed in a future version.


\subsection{\texttt{write bibliography}\label{wrbib}
Command}\index{\texttt{write bibliography} command}
Syntax:  \texttt{write bibliography} {\em filename}

This command reads an existing {\sc html} bibliographic cross-reference
file, normally called \texttt{mmbiblio.html}, and updates it per the
bibliographic links in the database comments.  The file is updated
between the {\sc html} comment lines \texttt{<!--
{\char`\#}START{\char`\#} -->} and \texttt{<!-- {\char`\#}END{\char`\#}
-->}.  The original input file is renamed to {\em
filename}\texttt{{\char`\~}1}.

A bibliographic reference is indicated with the reference name
in brackets, such as  \texttt{Theorem 3.1 of
[Monk] p.\ 22}.
See Section~\ref{htmlout} (p.~\pageref{htmlout}) for
syntax details.


\subsection{\texttt{write recent\_additions}
Command}\index{\texttt{write recent{\char`\_}additions} command}
Syntax:  \texttt{write recent{\char`\_}additions} {\em filename}
[\texttt{/limit} {\em number}]

This command reads an existing ``Recent Additions'' {\sc html} file,
normally called \texttt{mmrecent.html}, and updates it with the
descriptions of the most recently added theorems to the database.
 The file is updated between
the {\sc html} comment lines \texttt{<!-- {\char`\#}START{\char`\#} -->}
and \texttt{<!-- {\char`\#}END{\char`\#} -->}.  The original input file
is renamed to {\em filename}\texttt{{\char`\~}1}.

Optional command qualifier:

    \texttt{/limit} {\em number} -
 This qualifier specifies the number of most recent theorems to
   write to the output file.  The default is 100.


\section{Text File Utilities}

\subsection{\texttt{tools} Command}\index{\texttt{tools} command}
Syntax:  \texttt{tools}

This command invokes an easy-to-use, general purpose utility for
manipulating the contents of {\sc ascii} text files.  Upon typing
\texttt{tools}, the command-line prompt will change to \texttt{TOOLS>}
until you type \texttt{exit}.  The \texttt{tools} commands can be used
to perform simple, global edits on an input/output file,
such as making a character string substitution on each line, adding a
string to each line, and so on.  A typical use of this utility is
to build a \texttt{submit} input file to perform a common operation on a
list of statements obtained from \texttt{show label} or \texttt{show
usage}.

The actions of most of the \texttt{tools} commands can also be
performed with equivalent (and more powerful) Unix shell commands, and
some users may find those more efficient.  But for Windows users or
users not comfortable with Unix, \texttt{tools} provides an
easy-to-learn alternative that is adequate for most of the
script-building tasks needed to use the Metamath program effectively.

\subsection{\texttt{help} Command (in \texttt{tools})}
Syntax:  \texttt{help}

The \texttt{help} command lists the commands available in the
\texttt{tools} utility, along with a brief description.  Each command,
in turn, has its own help, such as \texttt{help add}.  As with
Metamath's \texttt{MM>} prompt, a complete command can be entered at
once, or just the command word can be typed, causing the program to
prompt for each argument.

\vskip 1ex
\noindent Line-by-line editing commands:

  \texttt{add} - Add a specified string to each line in a file.

  \texttt{clean} - Trim spaces and tabs on each line in a file; convert
         characters.

  \texttt{delete} - Delete a section of each line in a file.

  \texttt{insert} - Insert a string at a specified column in each line of
        a file.

  \texttt{substitute} - Make a simple substitution on each line of the file.

  \texttt{tag} - Like \texttt{add}, but restricted to a range of lines.

  \texttt{swap} - Swap the two halves of each line in a file.

\vskip 1ex
\noindent Other file-processing commands:

  \texttt{break} - Break up (tokenize) a file into a list of tokens (one per
        line).

  \texttt{build} - Build a file with multiple tokens per line from a list.

  \texttt{count} - Count the occurrences in a file of a specified string.

  \texttt{number} - Create a list of numbers.

  \texttt{parallel} - Put two files in parallel.

  \texttt{reverse} - Reverse the order of the lines in a file.

  \texttt{right} - Right-justify lines in a file (useful before sorting
         numbers).

%  \texttt{tag} - Tag edit updates in a program for revision control.

  \texttt{sort} - Sort the lines in a file with key starting at
         specified string.

  \texttt{match} - Extract lines containing (or not) a specified string.

  \texttt{unduplicate} - Eliminate duplicate occurrences of lines in a file.

  \texttt{duplicate} - Extract first occurrence of any line occurring
         more than

   \ \ \    once in a file, discarding lines occurring exactly once.

  \texttt{unique} - Extract lines occurring exactly once in a file.

  \texttt{type} (10 lines) - Display the first few lines in a file.
                  Similar to Unix \texttt{head}.

  \texttt{copy} - Similar to Unix \texttt{cat} but safe (same input
         and output file allowed).

  \texttt{submit} - Run a script containing \texttt{tools} commands.

\vskip 1ex

\noindent Note:
  \texttt{unduplicate}, \texttt{duplicate}, and \texttt{unique} also
 sort the lines as a side effect.


\subsection{Using \texttt{tools} to Build Metamath \texttt{submit}
Scripts}

The \texttt{break} command is typically used to break up a series of
statement labels, such as the output of Metamath's \texttt{show usage},
into one label per line.  The other \texttt{tools} commands can then be
used to add strings before and after each statement label to specify
commands to be performed on the statement.  The \texttt{parallel}
command is useful when a statement label must be mentioned more than
once on a line.

Very often a \texttt{submit} script for Metamath will require multiple
command lines for each statement being processed.  For example, you may
want to enter the Proof Assistant, \texttt{minimize{\char`\_}with} your
latest theorem, \texttt{save} the new proof, and \texttt{exit} the Proof
Assistant.  To accomplish this, you can build a file with these four
commands for each statement on a single line, separating each command
with a designated character such as \texttt{@}.  Then at the end you can
\texttt{substitute} each \texttt{@} with \texttt{{\char`\\}n} to break
up the lines into individual command lines (see \texttt{help
substitute}).


\subsection{Example of a \texttt{tools} Session}

To give you a quick feel for the \texttt{tools} utility, we show a
simple session where we create a file \texttt{n.txt} with 3 lines, add
strings before and after each line, and display the lines on the screen.
You can experiment with the various commands to gain experience with the
\texttt{tools} utility.

\begin{verbatim}
MM> tools
Entering the Text Tools utilities.
Type HELP for help, EXIT to exit.
TOOLS> number
Output file <n.tmp>? n.txt
First number <1>?
Last number <10>? 3
Increment <1>?
TOOLS> add
Input/output file? n.txt
String to add to beginning of each line <>? This is line
String to add to end of each line <>? .
The file n.txt has 3 lines; 3 were changed.
First change is on line 1:
This is line 1.
TOOLS> type n.txt
This is line 1.
This is line 2.
This is line 3.
TOOLS> exit
Exiting the Text Tools.
Type EXIT again to exit Metamath.
MM>
\end{verbatim}



\appendix
\chapter{Sample Representations}
\label{ASCII}

This Appendix provides a sample of {\sc ASCII} representations,
their corresponding traditional mathematical symbols,
and a discussion of their meanings
in the \texttt{set.mm} database.
These are provided in order of appearance.
This is only a partial list, and new definitions are routinely added.
A complete list is available at \url{http://metamath.org}.

These {\sc ASCII} representations, along
with information on how to display them,
are defined in the \texttt{set.mm} database file inside
a special comment called a \texttt{\$t} {\em
comment}\index{\texttt{\$t} comment} or {\em typesetting
comment.}\index{typesetting comment}
A typesetting comment
is indicated by the appearance of the
two-character string \texttt{\$t} at the beginning of the comment.
For more information,
see Section~\ref{tcomment}, p.~\pageref{tcomment}.

In the following table the ``{\sc ASCII}'' column shows the {\sc ASCII}
representation,
``Symbol'' shows the mathematical symbolic display
that corresponds to that {\sc ASCII} representation, ``Labels'' shows
the key label(s) that define the representation, and
``Description'' provides a description about the symbol.
As usual, ``iff'' is short for ``if and only if.''\index{iff}
In most cases the ``{\sc ASCII}'' column only shows
the key token, but it will sometimes show a sequence of tokens
if that is necessary for clarity.

{\setlength{\extrarowsep}{4pt} % Keep rows from being too close together
\begin{longtabu}   { @{} c c l X }
\textbf{ASCII} & \textbf{Symbol} & \textbf{Labels} & \textbf{Description} \\
\endhead
\texttt{|-} & $\vdash$ & &
  ``It is provable that...'' \\
\texttt{ph} & $\varphi$ & \texttt{wph} &
  The wff (boolean) variable phi,
  conventionally the first wff variable. \\
\texttt{ps} & $\psi$ & \texttt{wps} &
  The wff (boolean) variable psi,
  conventionally the second wff variable. \\
\texttt{ch} & $\chi$ & \texttt{wch} &
  The wff (boolean) variable chi,
  conventionally the third wff variable. \\
\texttt{-.} & $\lnot$ & \texttt{wn} &
  Logical not. E.g., if $\varphi$ is true, then $\lnot \varphi$ is false. \\
\texttt{->} & $\rightarrow$ & \texttt{wi} &
  Implies, also known as material implication.
  In classical logic the expression $\varphi \rightarrow \psi$ is true
  if either $\varphi$ is false or $\psi$ is true (or both), that is,
  $\varphi \rightarrow \psi$ has the same meaning as
  $\lnot \varphi \lor \psi$ (as proven in theorem \texttt{imor}). \\
\texttt{<->} & $\leftrightarrow$ &
  \hyperref[df-bi]{\texttt{df-bi}} &
  Biconditional (aka is-equals for boolean values).
  $\varphi \leftrightarrow \psi$ is true iff
  $\varphi$ and $\psi$ have the same value. \\
\texttt{\char`\\/} & $\lor$ &
  \makecell[tl]{{\hyperref[df-or]{\texttt{df-or}}}, \\
	         \hyperref[df-3or]{\texttt{df-3or}}} &
  Disjunction (logical ``or''). $\varphi \lor \psi$ is true iff
  $\varphi$, $\psi$, or both are true. \\
\texttt{/\char`\\} & $\land$ &
  \makecell[tl]{{\hyperref[df-an]{\texttt{df-an}}}, \\
                 \hyperref[df-3an]{\texttt{df-3an}}} &
  Conjunction (logical ``and''). $\varphi \land \psi$ is true iff
  both $\varphi$ and $\psi$ are true. \\
\texttt{A.} & $\forall$ &
  \texttt{wal} &
  For all; the wff $\forall x \varphi$ is true iff
  $\varphi$ is true for all values of $x$. \\
\texttt{E.} & $\exists$ &
  \hyperref[df-ex]{\texttt{df-ex}} &
  There exists; the wff
  $\exists x \varphi$ is true iff
  there is at least one $x$ where $\varphi$ is true. \\
\texttt{[ y / x ]} & $[ y / x ]$ &
  \hyperref[df-sb]{\texttt{df-sb}} &
  The wff $[ y / x ] \varphi$ produces
  the result when $y$ is properly substituted for $x$ in $\varphi$
  ($y$ replaces $x$).
  % This is elsb4
  % ( [ x / y ] z e. y <-> z e. x )
  For example,
  $[ x / y ] z \in y$ is the same as $z \in x$. \\
\texttt{E!} & $\exists !$ &
  \hyperref[df-eu]{\texttt{df-eu}} &
  There exists exactly one;
  $\exists ! x \varphi$ is true iff
  there is at least one $x$ where $\varphi$ is true. \\
\texttt{\{ y | phi \}}  & $ \{ y | \varphi \}$ &
  \hyperref[df-clab]{\texttt{df-clab}} &
  The class of all sets where $\varphi$ is true. \\
\texttt{=} & $ = $ &
  \hyperref[df-cleq]{\texttt{df-cleq}} &
  Class equality; $A = B$ iff $A$ equals $B$. \\
\texttt{e.} & $ \in $ &
  \hyperref[df-clel]{\texttt{df-clel}} &
  Class membership; $A \in B$ if $A$ is a member of $B$. \\
\texttt{{\char`\_}V} & {\rm V} &
  \hyperref[df-v]{\texttt{df-v}} &
  Class of all sets (not itself a set). \\
\texttt{C\_} & $ \subseteq $ &
  \hyperref[df-ss]{\texttt{df-ss}} &
  Subclass (subset); $A \subseteq B$ is true iff
  $A$ is a subclass of $B$. \\
\texttt{u.} & $ \cup $ &
  \hyperref[df-un]{\texttt{df-un}} &
  $A \cup B$ is the union of classes $A$ and $B$. \\
\texttt{i^i} & $ \cap $ &
  \hyperref[df-in]{\texttt{df-in}} &
  $A \cap B$ is the intersection of classes $A$ and $B$. \\
\texttt{\char`\\} & $ \setminus $ &
  \hyperref[df-dif]{\texttt{df-dif}} &
  $A \setminus B$ (set difference)
  is the class of all sets in $A$ except for those in $B$. \\
\texttt{(/)} & $ \varnothing $ &
  \hyperref[df-nul]{\texttt{df-nul}} &
  $ \varnothing $ is the empty set (aka null set). \\
\texttt{\char`\~P} & $ \cal P $ &
  \hyperref[df-pw]{\texttt{df-pw}} &
  Power class. \\
\texttt{<.\ A , B >.} & $\langle A , B \rangle$ &
  \hyperref[df-op]{\texttt{df-op}} &
  The ordered pair $\langle A , B \rangle$. \\
\texttt{( F ` A )} & $ ( F ` A ) $ &
  \hyperref[df-fv]{\texttt{df-fv}} &
  The value of function $F$ when applied to $A$. \\
\texttt{\_i} & $ i $ &
  \texttt{df-i} &
  The square root of negative one. \\
\texttt{x.} & $ \cdot $ &
  \texttt{df-mul} &
  Complex number multiplication; $2~\cdot~3~=~6$. \\
\texttt{CC} & $ \mathbb{C} $ &
  \texttt{df-c} &
  The set of complex numbers. \\
\texttt{RR} & $ \mathbb{R} $ &
  \texttt{df-r} &
  The set of real numbers. \\
\end{longtabu}
} % end of extrarowsep

\chapter{Compressed Proofs}
\label{compressed}\index{compressed proof}\index{proof!compressed}

The proofs in the \texttt{set.mm} set theory database are stored in compressed
format for efficiency.  Normally you needn't concern yourself with the
compressed format, since you can display it with the usual proof display tools
in the Metamath program (\texttt{show proof}\ldots) or convert it to the normal
RPN proof format described in Section~\ref{proof} (with \texttt{save proof}
{\em label} \texttt{/normal}).  However for sake of completeness we describe the
format here and show how it maps to the normal RPN proof format.

A compressed proof, located between \texttt{\$=} and \texttt{\$.}\ keywords, consists
of a left parenthesis, a sequence of statement labels, a right parenthesis,
and a sequence of upper-case letters \texttt{A} through \texttt{Z} (with optional
white space between them).  White space must surround the parentheses
and the labels.  The left parenthesis tells Metamath that a
compressed proof follows.  (A normal RPN proof consists of just a sequence of
labels, and a parenthesis is not a legal character in a label.)

The sequence of upper-case letters corresponds to a sequence of integers
with the following mapping.  Each integer corresponds to a proof step as
described later.
\begin{center}
  \texttt{A} = 1 \\
  \texttt{B} = 2 \\
   \ldots \\
  \texttt{T} = 20 \\
  \texttt{UA} = 21 \\
  \texttt{UB} = 22 \\
   \ldots \\
  \texttt{UT} = 40 \\
  \texttt{VA} = 41 \\
  \texttt{VB} = 42 \\
   \ldots \\
  \texttt{YT} = 120 \\
  \texttt{UUA} = 121 \\
   \ldots \\
  \texttt{YYT} = 620 \\
  \texttt{UUUA} = 621 \\
   etc.
\end{center}

In other words, \texttt{A} through \texttt{T} represent the
least-significant digit in base 20, and \texttt{U} through \texttt{Y}
represent zero or more most-significant digits in base 5, where the
digits start counting at 1 instead of the usual 0. With this scheme, we
don't need white space between these ``numbers.''

(In the design of the compressed proof format, only upper-case letters,
as opposed to say all non-whitespace printable {\sc ascii} characters other than
%\texttt{\$}, was chosen to make the compressed proof a little less
%displeasing to the eye, at the expense of a typical 20\% compression
\texttt{\$}, were chosen so as not to collide with most text editor
searches, at the expense of a typical 20\% compression
loss.  The base 5/base 20 grouping, as opposed to say base 6/base 19,
was chosen by experimentally determining the grouping that resulted in
best typical compression.)

The letter \texttt{Z} identifies (tags) a proof step that is identical to one
that occurs later on in the proof; it helps shorten the proof by not requiring
that identical proof steps be proved over and over again (which happens often
when building wff's).  The \texttt{Z} is placed immediately after the
least-significant digit (letters \texttt{A} through \texttt{T}) that ends the integer
corresponding to the step to later be referenced.

The integers that the upper-case letters correspond to are mapped to labels as
follows.  If the statement being proved has $m$ mandatory hypotheses, integers
1 through $m$ correspond to the labels of these hypotheses in the order shown
by the \texttt{show statement ... / full} command, i.e., the RPN order\index{RPN
order} of the mandatory
hypotheses.  Integers $m+1$ through $m+n$ correspond to the labels enclosed in
the parentheses of the compressed proof, in the order that they appear, where
$n$ is the number of those labels.  Integers $m+n+1$ on up don't directly
correspond to statement labels but point to proof steps identified with the
letter \texttt{Z}, so that these proof steps can be referenced later in the
proof.  Integer $m+n+1$ corresponds to the first step tagged with a \texttt{Z},
$m+n+2$ to the second step tagged with a \texttt{Z}, etc.  When the compressed
proof is converted to a normal proof, the entire subproof of a step tagged
with \texttt{Z} replaces the reference to that step.

For efficiency, Metamath works with compressed proofs directly, without
converting them internally to normal proofs.  In addition to the usual
error-checking, an error message is given if (1) a label in the label list in
parentheses does not refer to a previous \texttt{\$p} or \texttt{\$a} statement or a
non-mandatory hypothesis of the statement being proved and (2) a proof step
tagged with \texttt{Z} is referenced before the step tagged with the \texttt{Z}.

Just as in a normal proof under development (Section~\ref{unknown}), any step
or subproof that is not yet known may be represented with a single \texttt{?}.
White space does not have to appear between the \texttt{?}\ and the upper-case
letters (or other \texttt{?}'s) representing the remainder of the proof.

% April 1, 2004 Appendix C has been added back in with corrections.
%
% May 20, 2003 Appendix C was removed for now because there was a problem found
% by Bob Solovay
%
% Also, removed earlier \ref{formalspec} 's (3 cases above)
%
% Bob Solovay wrote on 30 Nov 2002:
%%%%%%%%%%%%% (start of email comment )
%      3. My next set of comments concern appendix C. I read this before I
% read Chapter 4. So I first noted that the system as presented in the
% Appendix lacked a certain formal property that I thought desirable. I
% then came up with a revised formal system that had this property. Upon
% reading Chapter 4, I noticed that the revised system was closer to the
% treatment in Chapter 4 than the system in Appendix C.
%
%         First a very minor correction:
%
%         On page 142 line 2: The condition that V(e) != V(f) should only be
% required of e, f in T such that e != f.
%
%         Here is a natural property [transitivity] that one would like
% the formal system to have:
%
%         Let Gamma be a set of statements. Suppose that the statement Phi
% is provable from Gamma and that the statement Psi is provable from Gamma
% \cup {Phi}. Then Psi is provable from Gamma.
%
%         I shall present an example to show that this property does not
% hold for the formal systems of Appendix C:
%
%         I write the example in metamath style:
%
% $c A B C D E $.
% $v x y
%
% ${
% tx $f A x $.
% ty $f B y $.
% ax1 $a C x y $.
% $}
%
% ${
% tx $f A x $.
% ty $f B y $.
% ax2-h1 $e C x y $.
% ax2 $a D y $.
% $}
%
% ${
% ty $f B y $.
% ax3-h1 $e D y $.
% ax3 $a E y $.
% $}
%
% $(These three axioms are Gamma $)
%
% ${
% tx $f A x $.
% ty $f B y $.
% Phi $p D y $=
% tx ty tx ty ax1 ax2 $.
% $}
%
% ${
% ty $f B y $.
% Psi $p E y $=
% ty ty Phi ax3 $.
% $}
%
%
% I omit the formal proofs of the following claims. [I will be glad to
% supply them upon request.]
%
% 1) Psi is not provable from Gamma;
%
% 2) Psi is provable from Gamma + Phi.
%
% Here "provable" refers to the formalism of Appendix C.
%
% The trouble of course is that Psi is lacking the variable declaration
%
% $f Ax $.
%
% In the Metamath system there is no trouble proving Psi. I attach a
% metamath file that shows this and which has been checked by the
% metamath program.
%
% I next want to indicate how I think the treatment in Appendix C should
% be revised so as to conform more closely to the metamath system of the
% main text. The revised system *does* have the transitivity property.
%
% We want to give revised definitions of "statement" and
% "provable". [cf. sections C.2.4. and C.2.5] Our new definitions will
% use the definitions given in Appendix C. So we take the following
% tack. We refer to the original notions as o-statement and o-provable. And
% we refer to the notions we are defining as n-statement and n-provable.
%
%         A n-statement is an o-statement in which the only variables
% that appear in the T component are mandatory.
%
%         To any o-statement we can associate its reduct which is a
% n-statement by dropping all the elements of T or D which contain
% non-mandatory variables.
%
%         An n-statement gamma is n-provable if there is an o-statement
% gamma' which has gamma as its reduct andf such that gamma' is
% o-provable.
%
%         It seems to me [though I am not completely sure on this point]
% that n-provability corresponds to metamath provability as discussed
% say in Chapter 4.
%
%         Attached to this letter is the metamath proof of Phi and Psi
% from Gamma discussed above.
%
%         I am still brooding over the question of whether metamath
% correctly formalizes set-theory. No doubt I will have some questions
% re this after my thoughts become clearer.
%%%%%%%%%%%%%%%% (end of email comment)

%%%%%%%%%%%%%%%% (start of 2nd email comment from Bob Solovay 1-Apr-04)
%
%         I hope that Appendix C is the one that gives a "formal" treatment
% of Metamath. At any rate, thats the appendix I want to comment on.
%
%         I'm going to suggest two changes in the definition.
%
%         First change (in the definition of statement): Require that the
% sets D, T, and E be finite.
%
%         Probably things are fine as you give them. But in the applications
% to the main metamath system they will always be finite, and its useful in
% thinking about things [at least for me] to stick to the finite case.
%
%         Second change:
%
%         First let me give an approximate description. Remove the dummy
% variables from the statement. Instead, include them in the proof.
%
%         More formally: Require that T consists of type declarations only
% for mandatory variables. Require that all the pairs in D consist of
% mandatory variables.
%
%         At the start of a proof we are allowed to declare a finite number
% of dummy variables [provided that none of them appear in any of the
% statements in E \cup {A}. We have to supply type declarations for all the
% dummy variables. We are allowed to add new $d statements referring to
% either the mandatory or dummy variables. But we require that no new $d
% statement references only mandatory variables.
%
%         I find this way of doing things more conceptual than the treatment
% in Appendix C. But the change [which I will use implicitly in later
% letters about doing Peano] is mainly aesthetic. I definitely claim that my
% results on doing Peano all apply to Metamath as it is presented in your
% book.
%
%         --Bob
%
%%%%%%%%%%%%%%%% (end of 2nd email comment)

%%
%% When uncommenting the below, also uncomment references above to {formalspec}
%%
\chapter{Metamath's Formal System}\label{formalspec}\index{Metamath!as a formal
system}

\section{Introduction}

\begin{quote}
  {\em Perfection is when there is no longer anything more to take away.}
    \flushright\sc Antoine de
     Saint-Exupery\footnote{\cite[p.~3-25]{Campbell}.}\\
\end{quote}\index{de Saint-Exupery, Antoine}

This appendix describes the theory behind the Metamath language in an abstract
way intended for mathematicians.  Specifically, we construct two
set-theo\-ret\-i\-cal objects:  a ``formal system'' (roughly, a set of syntax
rules, axioms, and logical rules) and its ``universe'' (roughly, the set of
theorems derivable in the formal system).  The Metamath computer language
provides us with a way to describe specific formal systems and, with the aid of
a proof provided by the user, to verify that given theorems
belong to their universes.

To understand this appendix, you need a basic knowledge of informal set theory.
It should be sufficient to understand, for example, Ch.\ 1 of Munkres' {\em
Topology} \cite{Munkres}\index{Munkres, James R.} or the
introductory set theory chapter
in many textbooks that introduce abstract mathematics. (Note that there are
minor notational differences among authors; e.g.\ Munkres uses $\subset$ instead
of our $\subseteq$ for ``subset.''  We use ``included in'' to mean ``a subset
of,'' and ``belongs to'' or ``is contained in'' to mean ``is an element of.'')
What we call a ``formal'' description here, unlike earlier, is actually an
informal description in the ordinary language of mathematicians.  However we
provide sufficient detail so that a mathematician could easily formalize it,
even in the language of Metamath itself if desired.  To understand the logic
examples at the end of this appendix, familiarity with an introductory book on
mathematical logic would be helpful.

\section{The Formal Description}

\subsection[Preliminaries]{Preliminaries\protect\footnotemark}%
\footnotetext{This section is taken mostly verbatim
from Tarski \cite[p.~63]{Tarski1965}\index{Tarski, Alfred}.}

By $\omega$ we denote the set of all natural numbers (non-negative integers).
Each natural number $n$ is identified with the set of all smaller numbers: $n =
\{ m | m < n \}$.  The formula $m < n$ is thus equivalent to the condition: $m
\in n$ and $m,n \in \omega$. In particular, 0 is the number zero and at the
same time the empty set $\varnothing$, $1=\{0\}$, $2=\{0,1\}$, etc. ${}^B A$
denotes the set of all functions on $B$ to $A$ (i.e.\ with domain $B$ and range
included in $A$).  The members of ${}^\omega A$ are what are called {\em simple
infinite sequences},\index{simple infinite sequence}
with all {\em terms}\index{term} in $A$.  In case $n \in \omega$, the
members of ${}^n A$ are referred to as {\em finite $n$-termed
sequences},\index{finite $n$-termed
sequence} again
with terms in $A$.  The consecutive terms (function values) of a finite or
infinite sequence $f$ are denoted by $f_0, f_1, \ldots ,f_n,\ldots$.  Every
finite sequence $f \in \bigcup _{n \in \omega} {}^n A$ uniquely determines the
number $n$ such that $f \in {}^n A$; $n$ is called the {\em
length}\index{length of a sequence ({$"|\ "|$})} of $f$ and
is denoted by $|f|$.  $\langle a \rangle$ is the sequence $f$ with $|f|=1$ and
$f_0=a$; $\langle a,b \rangle$ is the sequence $f$ with $|f|=2$, $f_0=a$,
$f_1=b$; etc.  Given two finite sequences $f$ and $g$, we denote by $f\frown g$
their {\em concatenation},\index{concatenation} i.e., the
finite sequence $h$ determined by the
conditions:
\begin{eqnarray*}
& |h| = |f|+|g|;&  \\
& h_n = f_n & \mbox{\ for\ } n < |f|;  \\
& h_{|f|+n} = g_n & \mbox{\ for\ } n < |g|.
\end{eqnarray*}

\subsection{Constants, Variables, and Expressions}

A formal system has a set of {\em symbols}\index{symbol!in
a formal system} denoted
by $\mbox{\em SM}$.  A
precise set-theo\-ret\-i\-cal definition of this set is unimportant; a symbol
could be considered a primitive or atomic element if we wish.  We assume this
set is divided into two disjoint subsets:  a set $\mbox{\em CN}$ of {\em
constants}\index{constant!in a formal system} and a set $\mbox{\em VR}$ of
{\em variables}.\index{variable!in a formal system}  $\mbox{\em CN}$ and
$\mbox{\em VR}$ are each assumed to consist of countably many symbols which
may be arranged in finite or simple infinite sequences $c_0, c_1, \ldots$ and
$v_0, v_1, \ldots$ respectively, without repeating terms.  We will represent
arbitrary symbols by metavariables $\alpha$, $\beta$, etc.

{\footnotesize\begin{quotation}
{\em Comment.} The variables $v_0, v_1, \ldots$ of our formal system
correspond to what are usually considered ``metavariables'' in
descriptions of specific formal systems in the literature.  Typically,
when describing a specific formal system a book will postulate a set of
primitive objects called variables, then proceed to describe their
properties using metavariables that range over them, never mentioning
again the actual variables themselves.  Our formal system does not
mention these primitive variable objects at all but deals directly with
metavariables, as its primitive objects, from the start.  This is a
subtle but key distinction you should keep in mind, and it makes our
definition of ``formal system'' somewhat different from that typically
found in the literature.  (So, the $\alpha$, $\beta$, etc.\ above are
actually ``metametavariables'' when used to represent $v_0, v_1,
\ldots$.)
\end{quotation}}

Finite sequences all terms of which are symbols are called {\em
expressions}.\index{expression!in a formal system}  $\mbox{\em EX}$ is
the set of all expressions; thus
\begin{displaymath}
\mbox{\em EX} = \bigcup _{n \in \omega} {}^n \mbox{\em SM}.
\end{displaymath}

A {\em constant-prefixed expression}\index{constant-prefixed expression}
is an expression of non-zero length
whose first term is a constant.  We denote the set of all constant-prefixed
expressions by $\mbox{\em EX}_C = \{ e \in \mbox{\em EX} | ( |e| > 0 \wedge
e_0 \in \mbox{\em CN} ) \}$.

A {\em constant-variable pair}\index{constant-variable pair}
is an expression of length 2 whose first term
is a constant and whose second term is a variable.  We denote the set of all
constant-variable pairs by $\mbox{\em EX}_2 = \{ e \in \mbox{\em EX}_C | ( |e|
= 2 \wedge e_1 \in \mbox{\em VR} ) \}$.


{\footnotesize\begin{quotation}
{\em Relationship to Metamath.} In general, the set $\mbox{\em SM}$
corresponds to the set of declared math symbols in a Metamath database, the
set $\mbox{\em CN}$ to those declared with \texttt{\$c} statements, and the set
$\mbox{\em VR}$ to those declared with \texttt{\$v} statements.  Of course a
Metamath database can only have a finite number of math symbols, whereas
formal systems in general can have an infinite number, although the number of
Metamath math symbols available is in principle unlimited.

The set $\mbox{\em EX}_C$ corresponds to the set of permissible expressions
for \texttt{\$e}, \texttt{\$a}, and \texttt{\$p} statements.  The set $\mbox{\em EX}_2$
corresponds to the set of permissible expressions for \texttt{\$f} statements.
\end{quotation}}

We denote by ${\cal V}(e)$ the set of all variables in an expression $e \in
\mbox{\em EX}$, i.e.\ the set of all $\alpha \in \mbox{\em VR}$ such that
$\alpha = e_n$ for some $n < |e|$.  We also denote (with abuse of notation) by
${\cal V}(E)$ the set of all variables in a collection of expressions $E
\subseteq \mbox{\em EX}$, i.e.\ $\bigcup _{e \in E} {\cal V}(e)$.


\subsection{Substitution}

Given a function $F$ from $\mbox{\em VR}$ to
$\mbox{\em EX}$, we
denote by $\sigma_{F}$ or just $\sigma$ the function from $\mbox{\em EX}$ to
$\mbox{\em EX}$ defined recursively for nonempty sequences by
\begin{eqnarray*}
& \sigma(<\alpha>) = F(\alpha) & \mbox{for\ } \alpha \in \mbox{\em VR}; \\
& \sigma(<\alpha>) = <\alpha> & \mbox{for\ } \alpha \not\in \mbox{\em VR}; \\
& \sigma(g \frown h) = \sigma(g) \frown
    \sigma(h) & \mbox{for\ } g,h \in \mbox{\em EX}.
\end{eqnarray*}
We also define $\sigma(\varnothing)=\varnothing$.  We call $\sigma$ a {\em
simultaneous substitution}\index{substitution!variable}\index{variable
substitution} (or just {\em substitution}) with {\em substitution
map}\index{substitution map} $F$.

We also denote (with abuse of notation) by $\sigma(E)$ a substitution on a
collection of expressions $E \subseteq \mbox{\em EX}$, i.e.\ the set $\{
\sigma(e) | e \in E \}$.  The collection $\sigma(E)$ may of course contain
fewer expressions than $E$ because duplicate expressions could result from the
substitution.

\subsection{Statements}

We denote by $\mbox{\em DV}$ the set of all
unordered pairs $\{\alpha, \beta \} \subseteq \mbox{\em VR}$ such that $\alpha
\neq \beta$.  $\mbox{\em DV}$ stands for ``distinct variables.''

A {\em pre-statement}\index{pre-statement!in a formal system} is a
quadruple $\langle D,T,H,A \rangle$ such that
$D\subseteq \mbox{\em DV}$, $T\subseteq \mbox{\em EX}_2$, $H\subseteq
\mbox{\em EX}_C$ and $H$ is finite,
$A\in \mbox{\em EX}_C$, ${\cal V}(H\cup\{A\}) \subseteq
{\cal V}(T)$, and $\forall e,f\in T {\ } {\cal V}(e) \neq {\cal V}(f)$ (or
equivalently, $e_1 \ne f_1$) whenever $e \neq f$. The terms of the quadruple are called {\em
distinct-variable restrictions},\index{disjoint-variable restriction!in a
formal system} {\em variable-type hypotheses},\index{variable-type
hypothesis!in a formal system} {\em logical hypotheses},\index{logical
hypothesis!in a formal system} and the {\em assertion}\index{assertion!in a
formal system} respectively.  We denote by $T_M$ ({\em mandatory variable-type
hypotheses}\index{mandatory variable-type hypothesis!in a formal system}) the
subset of $T$ such that ${\cal V}(T_M) ={\cal V}(H \cup \{A\})$.  We denote by
$D_M=\{\{\alpha,\beta\}\in D|\{\alpha,\beta\}\subseteq {\cal V}(T_M)\}$ the
{\em mandatory distinct-variable restrictions}\index{mandatory
disjoint-variable restriction!in a formal system} of the pre-statement.
The set
of {\em mandatory hypotheses}\index{mandatory hypothesis!in a formal system}
is $T_M\cup H$.  We call the quadruple $\langle D_M,T_M,H,A \rangle$
the {\em reduct}\index{reduct!in a formal system} of
the pre-statement $\langle D,T,H,A \rangle$.

A {\em statement} is the reduct of some pre-statement\index{statement!in a
formal system}.  A statement is therefore a special kind of pre-statement;
in particular, a statement is the reduct of itself.

{\footnotesize\begin{quotation}
{\em Comment.}  $T$ is a set of expressions, each of length 2, that associate
a set of constants (``variable types'') with a set of variables.  The
condition ${\cal V}(H\cup\{A\}) \subseteq {\cal V}(T) $
means that each variable occurring in a statement's logical
hypotheses or assertion must have an associated variable-type hypothesis or
``type declaration,'' in  analogy to a computer programming language, where a
variable must be declared to be say, a string or an integer.  The requirement
that $\forall e,f\in T \, e_1 \ne f_1$ for $e\neq f$
means that each variable must be
associated with a unique constant designating its variable type; e.g., a
variable might be a ``wff'' or a ``set'' but not both.

Distinct-variable restrictions are used to specify what variable substitutions
are permissible to make for the statement to remain valid.  For example, in
the theorem scheme of set theory $\lnot\forall x\,x=y$ we may not substitute
the same variable for both $x$ and $y$.  On the other hand, the theorem scheme
$x=y\to y=x$ does not require that $x$ and $y$ be distinct, so we do not
require a distinct-variable restriction, although having one
would cause no harm other than making the scheme less general.

A mandatory variable-type hypothesis is one whose variable exists in a logical
hypothesis or the assertion.  A provable pre-statement
(defined below) may require
non-mandatory variable-type hypotheses that effectively introduce ``dummy''
variables for use in its proof.  Any number of dummy variables might
be required by a specific proof; indeed, it has been shown by H.\
Andr\'{e}ka\index{Andr{\'{e}}ka, H.} \cite{Nemeti} that there is no finite
upper bound to the number of dummy variables needed to prove an arbitrary
theorem in first-order logic (with equality) having a fixed number $n>2$ of
individual variables.  (See also the Comment on p.~\pageref{nodd}.)
For this reason we do not set a finite size bound on the collections $D$ and
$T$, although in an actual application (Metamath database) these will of
course be finite, increased to whatever size is necessary as more
proofs are added.
\end{quotation}}

{\footnotesize\begin{quotation}
{\em Relationship to Metamath.} A pre-statement of a formal system
corresponds to an extended frame in a Metamath database
(Section~\ref{frames}).  The collections $D$, $T$, and $H$ correspond
respectively to the \texttt{\$d}, \texttt{\$f}, and \texttt{\$e}
statement collections in an extended frame.  The expression $A$
corresponds to the \texttt{\$a} (or \texttt{\$p}) statement in an
extended frame.

A statement of a formal system corresponds to a frame in a Metamath
database.
\end{quotation}}

\subsection{Formal Systems}

A {\em formal system}\index{formal system} is a
triple $\langle \mbox{\em CN},\mbox{\em
VR},\Gamma\rangle$ where $\Gamma$ is a set of statements.  The members of
$\Gamma$ are called {\em axiomatic statements}.\index{axiomatic
statement!in a formal system}  Sometimes we will refer to a
formal system by just $\Gamma$ when $\mbox{\em CN}$ and $\mbox{\em VR}$ are
understood.

Given a formal system $\Gamma$, the {\em closure}\index{closure}\footnote{This
definition of closure incorporates a simplification due to
Josh Purinton.\index{Purinton, Josh}.} of a
pre-statement
$\langle D,T,H,A \rangle$ is the smallest set $C$ of expressions
such that:
%\begin{enumerate}
%  \item $T\cup H\subseteq C$; and
%  \item If for some axiomatic statement
%    $\langle D_M',T_M',H',A' \rangle \in \Gamma_A$, for
%    some $E \subseteq C$, some $F \subseteq C-T$ (where ``-'' denotes
%    set difference), and some substitution
%    $\sigma$ we have
%    \begin{enumerate}
%       \item $\sigma(T_M') = E$ (where, as above, the $M$ denotes the
%           mandatory variable-type hypotheses of $T^A$);
%       \item $\sigma(H') = F$;
%       \item for all $\{\alpha,\beta\}\in D^A$ and $\subseteq
%         {\cal V}(T_M')$, for all $\gamma\in {\cal V}(\sigma(\langle \alpha
%         \rangle))$, and for all $\delta\in  {\cal V}(\sigma(\langle \beta
%         \rangle))$, we have $\{\gamma, \delta\} \in D$;
%   \end{enumerate}
%   then $\sigma(A') \in C$.
%\end{enumerate}
\begin{list}{}{\itemsep 0.0pt}
  \item[1.] $T\cup H\subseteq C$; and
  \item[2.] If for some axiomatic statement
    $\langle D_M',T_M',H',A' \rangle \in
       \Gamma$ and for some substitution
    $\sigma$ we have
    \begin{enumerate}
       \item[a.] $\sigma(T_M' \cup H') \subseteq C$; and
       \item[b.] for all $\{\alpha,\beta\}\in D_M'$, for all $\gamma\in
         {\cal V}(\sigma(\langle \alpha
         \rangle))$, and for all $\delta\in  {\cal V}(\sigma(\langle \beta
         \rangle))$, we have $\{\gamma, \delta\} \in D$;
   \end{enumerate}
   then $\sigma(A') \in C$.
\end{list}
A pre-statement $\langle D,T,H,A
\rangle$ is {\em provable}\index{provable statement!in a formal
system} if $A\in C$ i.e.\ if its assertion belongs to its
closure.  A statement is {\em provable} if it is
the reduct of a provable pre-statement.
The {\em universe}\index{universe of a formal system}
of a formal system is
the collection of all of its provable statements.  Note that the
set of axiomatic statements $\Gamma$ in a formal system is a subset of its
universe.

{\footnotesize\begin{quotation}
{\em Comment.} The first condition in the definition of closure simply says
that the hypotheses of the pre-statement are in its closure.

Condition 2(a) says that a substitution exists that makes the
mandatory hypotheses of an axiomatic statement exactly match some members of
the closure.  This is what we explicitly demonstrate in a Metamath language
proof.

%Conditions 2(a) and 2(b) say that a substitution exists that makes the
%(mandatory) hypotheses of an axiomatic statement exactly match some members of
%the closure.  This is what we explicitly demonstrate with a Metamath language
%proof.
%
%The set of expressions $F$ in condition 2(b) excludes the variable-type
%hypotheses; this is done because non-mandatory variable-type hypotheses are
%effectively ``dropped'' as irrelevant whereas logical hypotheses must be
%retained to achieve a consistent logical system.

Condition 2(b) describes how distinct-variable restrictions in the axiomatic
statement must be met.  It means that after a substitution for two variables
that must be distinct, the resulting two expressions must either contain no
variables, or if they do, they may not have variables in common, and each pair
of any variables they do have, with one variable from each expression, must be
specified as distinct in the original statement.
\end{quotation}}

{\footnotesize\begin{quotation}
{\em Relationship to Metamath.} Axiomatic statements
 and provable statements in a formal
system correspond to the frames for \texttt{\$a} and \texttt{\$p} statements
respectively in a Metamath database.  The set of axiomatic statements is a
subset of the set of provable statements in a formal system, although in a
Metamath database a \texttt{\$a} statement is distinguished by not having a
proof.  A Metamath language proof for a \texttt{\$p} statement tells the computer
how to explicitly construct a series of members of the closure ultimately
leading to a demonstration that the assertion
being proved is in the closure.  The actual closure typically contains
an infinite number of expressions.  A formal system itself does not have
an explicit object called a ``proof'' but rather the existence of a proof
is implied indirectly by membership of an assertion in a provable
statement's closure.  We do this to make the formal system easier
to describe in the language of set theory.

We also note that once established as provable, a statement may be considered
to acquire the same status as an axiomatic statement, because if the set of
axiomatic statements is extended with a provable statement, the universe of
the formal system remains unchanged (provided that $\mbox{\em VR}$ is
infinite).
In practice, this means we can build a hierarchy of provable statements to
more efficiently establish additional provable statements.  This is
what we do in Metamath when we allow proofs to reference previous
\texttt{\$p} statements as well as previous \texttt{\$a} statements.
\end{quotation}}

\section{Examples of Formal Systems}

{\footnotesize\begin{quotation}
{\em Relationship to Metamath.} The examples in this section, except Example~2,
are for the most part exact equivalents of the development in the set
theory database \texttt{set.mm}.  You may want to compare Examples~1, 3, and 5
to Section~\ref{metaaxioms}, Example 4 to Sections~\ref{metadefprop} and
\ref{metadefpred}, and Example 6 to
Section~\ref{setdefinitions}.\label{exampleref}
\end{quotation}}

\subsection{Example~1---Propositional Calculus}\index{propositional calculus}

Classical propositional calculus can be described by the following formal
system.  We assume the set of variables is infinite.  Rather than denoting the
constants and variables by $c_0, c_1, \ldots$ and $v_0, v_1, \ldots$, for
readability we will instead use more conventional symbols, with the
understanding of course that they denote distinct primitive objects.
Also for readability we may omit commas between successive terms of a
sequence; thus $\langle \mbox{wff\ } \varphi\rangle$ denotes
$\langle \mbox{wff}, \varphi\rangle$.

Let
\begin{itemize}
  \item[] $\mbox{\em CN}=\{\mbox{wff}, \vdash, \to, \lnot, (,)\}$
  \item[] $\mbox{\em VR}=\{\varphi,\psi,\chi,\ldots\}$
  \item[] $T = \{\langle \mbox{wff\ } \varphi\rangle,
             \langle \mbox{wff\ } \psi\rangle,
             \langle \mbox{wff\ } \chi\rangle,\ldots\}$, i.e.\ those
             expressions of length 2 whose first member is $\mbox{\rm wff}$
             and whose second member belongs to $\mbox{\em VR}$.\footnote{For
convenience we let $T$ be an infinite set; the definition of a statement
permits this in principle.  Since a Metamath source file has a finite size, in
practice we must of course use appropriate finite subsets of this $T$,
specifically ones containing at least the mandatory variable-type
hypotheses.  Similarly, in the source file we introduce new variables as
required, with the understanding that a potentially infinite number of
them are available.}
\noindent Then $\Gamma$ consists of the axiomatic statements that
are the reducts of the following pre-statements:
    \begin{itemize}
      \item[] $\langle\varnothing,T,\varnothing,
               \langle \mbox{wff\ }(\varphi\to\psi)\rangle\rangle$
      \item[] $\langle\varnothing,T,\varnothing,
               \langle \mbox{wff\ }\lnot\varphi\rangle\rangle$
      \item[] $\langle\varnothing,T,\varnothing,
               \langle \vdash(\varphi\to(\psi\to\varphi))
               \rangle\rangle$
      \item[] $\langle\varnothing,T,
               \varnothing,
               \langle \vdash((\varphi\to(\psi\to\chi))\to
               ((\varphi\to\psi)\to(\varphi\to\chi)))
               \rangle\rangle$
      \item[] $\langle\varnothing,T,
               \varnothing,
               \langle \vdash((\lnot\varphi\to\lnot\psi)\to
               (\psi\to\varphi))\rangle\rangle$
      \item[] $\langle\varnothing,T,
               \{\langle\vdash(\varphi\to\psi)\rangle,
                 \langle\vdash\varphi\rangle\},
               \langle\vdash\psi\rangle\rangle$
    \end{itemize}
\end{itemize}

(For example, the reduct of $\langle\varnothing,T,\varnothing,
               \langle \mbox{wff\ }(\varphi\to\psi)\rangle\rangle$
is
\begin{itemize}
\item[] $\langle\varnothing,
\{\langle \mbox{wff\ } \varphi\rangle,
             \langle \mbox{wff\ } \psi\rangle\},
             \varnothing,
               \langle \mbox{wff\ }(\varphi\to\psi)\rangle\rangle$,
\end{itemize}
which is the first axiomatic statement.)

We call the members of $\mbox{\em VR}$ {\em wff variables} or (in the context
of first-order logic which we will describe shortly) {\em wff metavariables}.
Note that the symbols $\phi$, $\psi$, etc.\ denote actual specific members of
$\mbox{\em VR}$; they are not metavariables of our expository language (which
we denote with $\alpha$, $\beta$, etc.) but are instead (meta)constant symbols
(members of $\mbox{\em SM}$) from the point of view of our expository
language.  The equivalent system of propositional calculus described in
\cite{Tarski1965} also uses the symbols $\phi$, $\psi$, etc.\ to denote wff
metavariables, but in \cite{Tarski1965} unlike here those are metavariables of
the expository language and not primitive symbols of the formal system.

The first two statements define wffs: if $\varphi$ and $\psi$ are wffs, so is
$(\varphi \to \psi)$; if $\varphi$ is a wff, so is $\lnot\varphi$. The next
three are the axioms of propositional calculus: if $\varphi$ and $\psi$ are
wffs, then $\vdash (\varphi \to (\psi \to \varphi))$ is an (axiomatic)
theorem; etc. The
last is the rule of modus ponens: if $\varphi$ and $\psi$ are wffs, and
$\vdash (\varphi\to\psi)$ and $\vdash \varphi$ are theorems, then $\vdash
\psi$ is a theorem.

The correspondence to ordinary propositional calculus is as follows.  We
consider only provable statements of the form $\langle\varnothing,
T,\varnothing,A\rangle$ with $T$ defined as above.  The first term of the
assertion $A$ of any such statement is either ``wff'' or ``$\vdash$''.  A
statement for which the first term is ``wff'' is a {\em wff} of propositional
calculus, and one where the first term is ``$\vdash$'' is a {\em
theorem (scheme)} of propositional calculus.

The universe of this formal system also contains many other provable
statements.  Those with distinct-variable restrictions are irrelevant because
propositional calculus has no constraints on substitutions.  Those that have
logical hypotheses we call {\em inferences}\index{inference} when
the logical hypotheses are of the form
$\langle\vdash\rangle\frown w$ where $w$ is a wff (with the leading constant
term ``wff'' removed).  Inferences (other than the modus ponens rule) are not a
proper part of propositional calculus but are convenient to use when building a
hierarchy of provable statements.  A provable statement with a nonsense
hypothesis such as $\langle \to,\vdash,\lnot\rangle$, and this same expression
as its assertion, we consider irrelevant; no use can be made of it in
proving theorems, since there is no way to eliminate the nonsense hypothesis.

{\footnotesize\begin{quotation}
{\em Comment.} Our use of parentheses in the definition of a wff illustrates
how axiomatic statements should be carefully stated in a way that
ties in unambiguously with the substitutions allowed by the formal system.
There are many ways we could have defined wffs---for example, Polish
prefix notation would have allowed us to omit parentheses entirely, at
the expense of readability---but we must define them in a way that is
unambiguous.  For example, if we had omitted parentheses from the
definition of $(\varphi\to \psi)$, the wff $\lnot\varphi\to \psi$ could
be interpreted as either $\lnot(\varphi\to\psi)$ or $(\lnot\varphi\to\psi)$
and would have allowed us to prove nonsense.  Note that there is no
concept of operator binding precedence built into our formal system.
\end{quotation}}

\begin{sloppy}
\subsection{Example~2---Predicate Calculus with Equality}\index{predicate
calculus}
\end{sloppy}

Here we extend Example~1 to include predicate calculus with equality,
illustrating the use of distinct-variable restrictions.  This system is the
same as Tarski's system $\mathfrak{S}_2$ in \cite{Tarski1965} (except that the
axioms of propositional calculus are different but equivalent, and a redundant
axiom is omitted).  We extend $\mbox{\em CN}$ with the constants
$\{\mbox{var},\forall,=\}$.  We extend $\mbox{\em VR}$ with an infinite set of
{\em individual metavariables}\index{individual
metavariable} $\{x,y,z,\ldots\}$ and denote this subset
$\mbox{\em Vr}$.

We also join to $\mbox{\em CN}$ a possibly infinite set $\mbox{\em Pr}$ of {\em
predicates} $\{R,S,\ldots\}$.  We associate with $\mbox{\em Pr}$ a function
$\mbox{rnk}$ from $\mbox{\em Pr}$ to $\omega$, and for $\alpha\in \mbox{\em
Pr}$ we call $\mbox{rnk}(\alpha)$ the {\em rank} of the predicate $\alpha$,
which is simply the number of ``arguments'' that the predicate has.  (Most
applications of predicate calculus will have a finite number of predicates;
for example, set theory has the single two-argument or binary predicate $\in$,
which is usually written with its arguments surrounding the predicate symbol
rather than with the prefix notation we will use for the general case.)  As a
device to facilitate our discussion, we will let $\mbox{\em Vs}$ be any fixed
one-to-one function from $\omega$ to $\mbox{\em Vr}$; thus $\mbox{\em Vs}$ is
any simple infinite sequence of individual metavariables with no repeating
terms.

In this example we will not include the function symbols that are often part of
formalizations of predicate calculus.  Using metalogical arguments that are
beyond the scope of our discussion, it can be shown that our formalization is
equivalent when functions are introduced via appropriate definitions.

We extend the set $T$ defined in Example~1 with the expressions
$\{\langle \mbox{var\ } x\rangle,$ $ \langle \mbox{var\ } y\rangle, \langle
\mbox{var\ } z\rangle,\ldots\}$.  We extend the $\Gamma$ above
with the axiomatic statements that are the reducts of the following
pre-statements:
\begin{list}{}{\itemsep 0.0pt}
      \item[] $\langle\varnothing,T,\varnothing,
               \langle \mbox{wff\ }\forall x\,\varphi\rangle\rangle$
      \item[] $\langle\varnothing,T,\varnothing,
               \langle \mbox{wff\ }x=y\rangle\rangle$
      \item[] $\langle\varnothing,T,
               \{\langle\vdash\varphi\rangle\},
               \langle\vdash\forall x\,\varphi\rangle\rangle$
      \item[] $\langle\varnothing,T,\varnothing,
               \langle \vdash((\forall x(\varphi\to\psi)
                  \to(\forall x\,\varphi\to\forall x\,\psi))
               \rangle\rangle$
      \item[] $\langle\{\{x,\varphi\}\},T,\varnothing,
               \langle \vdash(\varphi\to\forall x\,\varphi)
               \rangle\rangle$
      \item[] $\langle\{\{x,y\}\},T,\varnothing,
               \langle \vdash\lnot\forall x\lnot x=y
               \rangle\rangle$
      \item[] $\langle\varnothing,T,\varnothing,
               \langle \vdash(x=z
                  \to(x=y\to z=y))
               \rangle\rangle$
      \item[] $\langle\varnothing,T,\varnothing,
               \langle \vdash(y=z
                  \to(x=y\to x=z))
               \rangle\rangle$
\end{list}
These are the axioms not involving predicate symbols. The first two statements
extend the definition of a wff.  The third is the rule of generalization.  The
fifth states, in effect, ``For a wff $\varphi$ and variable $x$,
$\vdash(\varphi\to\forall x\,\varphi)$, provided that $x$ does not occur in
$\varphi$.''  The sixth states ``For variables $x$ and $y$,
$\vdash\lnot\forall x\lnot x = y$, provided that $x$ and $y$ are distinct.''
(This proviso is not necessary but was included by Tarski to
weaken the axiom and still show that the system is logically complete.)

Finally, for each predicate symbol $\alpha\in \mbox{\em Pr}$, we add to
$\Gamma$ an axiomatic statement, extending the definition of wff,
that is the reduct of the following pre-statement:
\begin{displaymath}
    \langle\varnothing,T,\varnothing,
            \langle \mbox{wff},\alpha\rangle\
            \frown \mbox{\em Vs}\restriction\mbox{rnk}(\alpha)\rangle
\end{displaymath}
and for each $\alpha\in \mbox{\em Pr}$ and each $n < \mbox{rnk}(\alpha)$
we add to $\Gamma$ an equality axiom that is the reduct of the
following pre-statement:
\begin{eqnarray*}
    \lefteqn{\langle\varnothing,T,\varnothing,
            \langle
      \vdash,(,\mbox{\em Vs}_n,=,\mbox{\em Vs}_{\mbox{rnk}(\alpha)},\to,
            (,\alpha\rangle\frown \mbox{\em Vs}\restriction\mbox{rnk}(\alpha)} \\
  & & \frown
            \langle\to,\alpha\rangle\frown \mbox{\em Vs}\restriction n\frown
            \langle \mbox{\em Vs}_{\mbox{rnk}(\alpha)}\rangle \\
 & & \frown
            \mbox{\em Vs}\restriction(\mbox{rnk}(\alpha)\setminus(n+1))\frown
            \langle),)\rangle\rangle
\end{eqnarray*}
where $\restriction$ denotes function domain restriction and $\setminus$
denotes set difference.  Recall that a subscript on $\mbox{\em Vs}$
denotes one of its terms.  (In the above two axiom sets commas are placed
between successive terms of sequences to prevent ambiguity, and if you examine
them with care you will be able to distinguish those parentheses that denote
constant symbols from those of our expository language that delimit function
arguments.  Although it might have been better to use boldface for our
primitive symbols, unfortunately boldface was not available for all characters
on the \LaTeX\ system used to typeset this text.)  These seemingly forbidding
axioms can be understood by analogy to concatenation of substrings in a
computer language.  They are actually relatively simple for each specific case
and will become clearer by looking at the special case of a binary predicate
$\alpha = R$ where $\mbox{rnk}(R)=2$.  Letting $\mbox{\em Vs}$ be the sequence
$\langle x,y,z,\ldots\rangle$, the axioms we would add to $\Gamma$ for this
case would be the wff extension and two equality axioms that are the
reducts of the pre-statements:
\begin{list}{}{\itemsep 0.0pt}
      \item[] $\langle\varnothing,T,\varnothing,
               \langle \mbox{wff\ }R x y\rangle\rangle$
      \item[] $\langle\varnothing,T,\varnothing,
               \langle \vdash(x=z
                  \to(R x y \to R z y))
               \rangle\rangle$
      \item[] $\langle\varnothing,T,\varnothing,
               \langle \vdash(y=z
                  \to(R x y \to R x z))
               \rangle\rangle$
\end{list}
Study these carefully to see how the general axioms above evaluate to
them.  In practice, typically only a few special cases such as this would be
needed, and in any case the Metamath language will only permit us to describe
a finite number of predicates, as opposed to the infinite number permitted by
the formal system.  (If an infinite number should be needed for some reason,
we could not define the formal system directly in the Metamath language but
could instead define it metalogically under set theory as we
do in this appendix, and only the underlying set theory, with its single
binary predicate, would be defined directly in the Metamath language.)


{\footnotesize\begin{quotation}
{\em Comment.}  As we noted earlier, the specific variables denoted by the
symbols $x,y,z,\ldots\in \mbox{\em Vr}\subseteq \mbox{\em VR}\subseteq
\mbox{\em SM}$ in Example~2 are not the actual variables of ordinary predicate
calculus but should be thought of as metavariables ranging over them.  For
example, a distinct-variable restriction would be meaningless for actual
variables of ordinary predicate calculus since two different actual variables
are by definition distinct.  And when we talk about an arbitrary
representative $\alpha\in \mbox{\em Vr}$, $\alpha$ is a metavariable (in our
expository language) that ranges over metavariables (which are primitives of
our formal system) each of which ranges over the actual individual variables
of predicate calculus (which are never mentioned in our formal system).

The constant called ``var'' above is called \texttt{setvar} in the
\texttt{set.mm} database file, but it means the same thing.  I felt
that ``var'' is a more meaningful name in the context of predicate
calculus, whose use is not limited to set theory.  For consistency we
stick with the name ``var'' throughout this Appendix, even after set
theory is introduced.
\end{quotation}}

\subsection{Free Variables and Proper Substitution}\index{free variable}
\index{proper substitution}\index{substitution!proper}

Typical representations of mathematical axioms use concepts such
as ``free variable,'' ``bound variable,'' and ``proper substitution''
as primitive notions.
A free variable is a variable that
is not a parameter of any container expression.
A bound variable is the opposite of a free variable; it is a
a variable that has been bound in a container expression.
For example, in the expression $\forall x \varphi$ (for all $x$, $\varphi$
is true), the variable $x$
is bound within the for-all ($\forall$) expression.
It is possible to change one variable to another, and that process is called
``proper substitution.''
In most books, proper substitution has a somewhat complicated recursive
definition with multiple cases based on the occurrences of free and
bound variables.
You may consult
\cite[ch.\ 3--4]{Hamilton}\index{Hamilton, Alan G.} (as well as
many other texts) for more formal details about these terms.

Using these concepts as \texttt{primitives} creates complications
for computer implementations.

In the system of Example~2, there are no primitive notions of free variable
and proper substitution.  Tarski \cite{Tarski1965} shows that this system is
logically equivalent to the more typical textbook systems that do have these
primitive notions, if we introduce these notions with appropriate definitions
and metalogic.  We could also define axioms for such systems directly,
although the recursive definitions of free variable and proper substitution
would be messy and awkward to work with.  Instead, we mention two devices that
can be used in practice to mimic these notions.  (1) Instead of introducing
special notation to express (as a logical hypothesis) ``where $x$ is not free
in $\varphi$'' we can use the logical hypothesis $\vdash(\varphi\to\forall
x\,\varphi)$.\label{effectivelybound}\index{effectively
not free}\footnote{This is a slightly weaker requirement than ``where $x$ is
not free in $\varphi$.''  If we let $\varphi$ be $x=x$, we have the theorem
$(x=x\to\forall x\,x=x)$ which satisfies the hypothesis, even though $x$ is
free in $x=x$ .  In a case like this we say that $x$ is {\em effectively not
free}\index{effectively not free} in $x=x$, since $x=x$ is logically
equivalent to $\forall x\,x=x$ in which $x$ is bound.} (2) It can be shown
that the wff $((x=y\to\varphi)\wedge\exists x(x=y\wedge\varphi))$ (with the
usual definitions of $\wedge$ and $\exists$; see Example~4 below) is logically
equivalent to ``the wff that results from proper substitution of $y$ for $x$
in $\varphi$.''  This works whether or not $x$ and $y$ are distinct.

\subsection{Metalogical Completeness}\index{metalogical completeness}

In the system of Example~2, the
following are provable pre-statements (and their reducts are
provable statements):
\begin{eqnarray*}
      & \langle\{\{x,y\}\},T,\varnothing,
               \langle \vdash\lnot\forall x\lnot x=y
               \rangle\rangle & \\
     &  \langle\varnothing,T,\varnothing,
               \langle \vdash\lnot\forall x\lnot x=x
               \rangle\rangle &
\end{eqnarray*}
whereas the following pre-statement is not to my knowledge provable (but
in any case we will pretend it's not for sake of illustration):
\begin{eqnarray*}
     &  \langle\varnothing,T,\varnothing,
               \langle \vdash\lnot\forall x\lnot x=y
               \rangle\rangle &
\end{eqnarray*}
In other words, we can prove ``$\lnot\forall x\lnot x=y$ where $x$ and $y$ are
distinct'' and separately prove ``$\lnot\forall x\lnot x=x$'', but we can't
prove the combined general case ``$\lnot\forall x\lnot x=y$'' that has no
proviso.  Now this does not compromise logical completeness, because the
variables are really metavariables and the two provable cases together cover
all possible cases.  The third case can be considered a metatheorem whose
direct proof, using the system of Example~2, lies outside the capability of the
formal system.

Also, in the system of Example~2 the following pre-statement is not to my
knowledge provable (again, a conjecture that we will pretend to be the case):
\begin{eqnarray*}
     & \langle\varnothing,T,\varnothing,
               \langle \vdash(\forall x\, \varphi\to\varphi)
               \rangle\rangle &
\end{eqnarray*}
Instead, we can only prove specific cases of $\varphi$ involving individual
metavariables, and by induction on formula length, prove as a metatheorem
outside of our formal system the general statement above.  The details of this
proof are found in \cite{Kalish}.

There does, however, exist a system of predicate calculus in which all such
``simple metatheorems'' as those above can be proved directly, and we present
it in Example~3. A {\em simple metatheorem}\index{simple metatheorem}
is any statement of the formal
system of Example~2 where all distinct variable restrictions consist of either
two individual metavariables or an individual metavariable and a wff
metavariable, and which is provable by combining cases outside the system as
above.  A system is {\em metalogically complete}\index{metalogical
completeness} if all of its simple
metatheorems are (directly) provable statements. The precise definition of
``simple metatheorem'' and the proof of the ``metalogical completeness'' of
Example~3 is found in Remark 9.6 and Theorem 9.7 of \cite{Megill}.\index{Megill,
Norman}

\begin{sloppy}
\subsection{Example~3---Metalogically Complete Predicate
Calculus with
Equality}
\end{sloppy}

For simplicity we will assume there is one binary predicate $R$;
this system suffices for set theory, where the $R$ is of course the $\in$
predicate.  We label the axioms as they appear in \cite{Megill}.  This
system is logically equivalent to that of Example~2 (when the latter is
restricted to this single binary predicate) but is also metalogically
complete.\index{metalogical completeness}

Let
\begin{itemize}
  \item[] $\mbox{\em CN}=\{\mbox{wff}, \mbox{var}, \vdash, \to, \lnot, (,),\forall,=,R\}$.
  \item[] $\mbox{\em VR}=\{\varphi,\psi,\chi,\ldots\}\cup\{x,y,z,\ldots\}$.
  \item[] $T = \{\langle \mbox{wff\ } \varphi\rangle,
             \langle \mbox{wff\ } \psi\rangle,
             \langle \mbox{wff\ } \chi\rangle,\ldots\}\cup
       \{\langle \mbox{var\ } x\rangle, \langle \mbox{var\ } y\rangle, \langle
       \mbox{var\ }z\rangle,\ldots\}$.

\noindent Then
  $\Gamma$ consists of the reducts of the following pre-statements:
    \begin{itemize}
      \item[] $\langle\varnothing,T,\varnothing,
               \langle \mbox{wff\ }(\varphi\to\psi)\rangle\rangle$
      \item[] $\langle\varnothing,T,\varnothing,
               \langle \mbox{wff\ }\lnot\varphi\rangle\rangle$
      \item[] $\langle\varnothing,T,\varnothing,
               \langle \mbox{wff\ }\forall x\,\varphi\rangle\rangle$
      \item[] $\langle\varnothing,T,\varnothing,
               \langle \mbox{wff\ }x=y\rangle\rangle$
      \item[] $\langle\varnothing,T,\varnothing,
               \langle \mbox{wff\ }Rxy\rangle\rangle$
      \item[(C1$'$)] $\langle\varnothing,T,\varnothing,
               \langle \vdash(\varphi\to(\psi\to\varphi))
               \rangle\rangle$
      \item[(C2$'$)] $\langle\varnothing,T,
               \varnothing,
               \langle \vdash((\varphi\to(\psi\to\chi))\to
               ((\varphi\to\psi)\to(\varphi\to\chi)))
               \rangle\rangle$
      \item[(C3$'$)] $\langle\varnothing,T,
               \varnothing,
               \langle \vdash((\lnot\varphi\to\lnot\psi)\to
               (\psi\to\varphi))\rangle\rangle$
      \item[(C4$'$)] $\langle\varnothing,T,
               \varnothing,
               \langle \vdash(\forall x(\forall x\,\varphi\to\psi)\to
                 (\forall x\,\varphi\to\forall x\,\psi))\rangle\rangle$
      \item[(C5$'$)] $\langle\varnothing,T,
               \varnothing,
               \langle \vdash(\forall x\,\varphi\to\varphi)\rangle\rangle$
      \item[(C6$'$)] $\langle\varnothing,T,
               \varnothing,
               \langle \vdash(\forall x\forall y\,\varphi\to
                 \forall y\forall x\,\varphi)\rangle\rangle$
      \item[(C7$'$)] $\langle\varnothing,T,
               \varnothing,
               \langle \vdash(\lnot\varphi\to\forall x\lnot\forall x\,\varphi
                 )\rangle\rangle$
      \item[(C8$'$)] $\langle\varnothing,T,
               \varnothing,
               \langle \vdash(x=y\to(x=z\to y=z))\rangle\rangle$
      \item[(C9$'$)] $\langle\varnothing,T,
               \varnothing,
               \langle \vdash(\lnot\forall x\, x=y\to(\lnot\forall x\, x=z\to
                 (y=z\to\forall x\, y=z)))\rangle\rangle$
      \item[(C10$'$)] $\langle\varnothing,T,
               \varnothing,
               \langle \vdash(\forall x(x=y\to\forall x\,\varphi)\to
                 \varphi))\rangle\rangle$
      \item[(C11$'$)] $\langle\varnothing,T,
               \varnothing,
               \langle \vdash(\forall x\, x=y\to(\forall x\,\varphi
               \to\forall y\,\varphi))\rangle\rangle$
      \item[(C12$'$)] $\langle\varnothing,T,
               \varnothing,
               \langle \vdash(x=y\to(Rxz\to Ryz))\rangle\rangle$
      \item[(C13$'$)] $\langle\varnothing,T,
               \varnothing,
               \langle \vdash(x=y\to(Rzx\to Rzy))\rangle\rangle$
      \item[(C15$'$)] $\langle\varnothing,T,
               \varnothing,
               \langle \vdash(\lnot\forall x\, x=y\to(x=y\to(\varphi
                 \to\forall x(x=y\to\varphi))))\rangle\rangle$
      \item[(C16$'$)] $\langle\{\{x,y\}\},T,
               \varnothing,
               \langle \vdash(\forall x\, x=y\to(\varphi\to\forall x\,\varphi)
                 )\rangle\rangle$
      \item[(C5)] $\langle\{\{x,\varphi\}\},T,\varnothing,
               \langle \vdash(\varphi\to\forall x\,\varphi)
               \rangle\rangle$
      \item[(MP)] $\langle\varnothing,T,
               \{\langle\vdash(\varphi\to\psi)\rangle,
                 \langle\vdash\varphi\rangle\},
               \langle\vdash\psi\rangle\rangle$
      \item[(Gen)] $\langle\varnothing,T,
               \{\langle\vdash\varphi\rangle\},
               \langle\vdash\forall x\,\varphi\rangle\rangle$
    \end{itemize}
\end{itemize}

While it is known that these axioms are ``metalogically complete,'' it is
not known whether they are independent (i.e.\ none is
redundant) in the metalogical sense; specifically, whether any axiom (possibly
with additional non-mandatory distinct-variable restrictions, for use with any
dummy variables in its proof) is provable from the others.  Note that
metalogical independence is a weaker requirement than independence in the
usual logical sense.  Not all of the above axioms are logically independent:
for example, C9$'$ can be proved as a metatheorem from the others, outside the
formal system, by combining the possible cases of distinct variables.

\subsection{Example~4---Adding Definitions}\index{definition}
There are several ways to add definitions to a formal system.  Probably the
most proper way is to consider definitions not as part of the formal system at
all but rather as abbreviations that are part of the expository metalogic
outside the formal system.  For convenience, though, we may use the formal
system itself to incorporate definitions, adding them as axiomatic extensions
to the system.  This could be done by adding a constant representing the
concept ``is defined as'' along with axioms for it. But there is a nicer way,
at least in this writer's opinion, that introduces definitions as direct
extensions to the language rather than as extralogical primitive notions.  We
introduce additional logical connectives and provide axioms for them.  For
systems of logic such as Examples 1 through 3, the additional axioms must be
conservative in the sense that no wff of the original system that was not a
theorem (when the initial term ``wff'' is replaced by ``$\vdash$'' of course)
becomes a theorem of the extended system.  In this example we extend Example~3
(or 2) with standard abbreviations of logic.

We extend $\mbox{\em CN}$ of Example~3 with new constants $\{\leftrightarrow,
\wedge,\vee,\exists\}$, corresponding to logical equivalence,\index{logical
equivalence ($\leftrightarrow$)}\index{biconditional ($\leftrightarrow$)}
conjunction,\index{conjunction ($\wedge$)} disjunction,\index{disjunction
($\vee$)} and the existential quantifier.\index{existential quantifier
($\exists$)}  We extend $\Gamma$ with the axiomatic statements that are
the reducts of the following pre-statements:
\begin{list}{}{\itemsep 0.0pt}
      \item[] $\langle\varnothing,T,\varnothing,
               \langle \mbox{wff\ }(\varphi\leftrightarrow\psi)\rangle\rangle$
      \item[] $\langle\varnothing,T,\varnothing,
               \langle \mbox{wff\ }(\varphi\vee\psi)\rangle\rangle$
      \item[] $\langle\varnothing,T,\varnothing,
               \langle \mbox{wff\ }(\varphi\wedge\psi)\rangle\rangle$
      \item[] $\langle\varnothing,T,\varnothing,
               \langle \mbox{wff\ }\exists x\, \varphi\rangle\rangle$
  \item[] $\langle\varnothing,T,\varnothing,
     \langle\vdash ( ( \varphi \leftrightarrow \psi ) \to
     ( \varphi \to \psi ) )\rangle\rangle$
  \item[] $\langle\varnothing,T,\varnothing,
     \langle\vdash ((\varphi\leftrightarrow\psi)\to
    (\psi\to\varphi))\rangle\rangle$
  \item[] $\langle\varnothing,T,\varnothing,
     \langle\vdash ((\varphi\to\psi)\to(
     (\psi\to\varphi)\to(\varphi
     \leftrightarrow\psi)))\rangle\rangle$
  \item[] $\langle\varnothing,T,\varnothing,
     \langle\vdash (( \varphi \wedge \psi ) \leftrightarrow\neg ( \varphi
     \to \neg \psi )) \rangle\rangle$
  \item[] $\langle\varnothing,T,\varnothing,
     \langle\vdash (( \varphi \vee \psi ) \leftrightarrow (\neg \varphi
     \to \psi )) \rangle\rangle$
  \item[] $\langle\varnothing,T,\varnothing,
     \langle\vdash (\exists x \,\varphi\leftrightarrow
     \lnot \forall x \lnot \varphi)\rangle\rangle$
\end{list}
The first three logical axioms (statements containing ``$\vdash$'') introduce
and effectively define logical equivalence, ``$\leftrightarrow$''.  The last
three use ``$\leftrightarrow$'' to effectively mean ``is defined as.''

\subsection{Example~5---ZFC Set Theory}\index{ZFC set theory}

Here we add to the system of Example~4 the axioms of Zermelo--Fraenkel set
theory with Choice.  For convenience we make use of the
definitions in Example~4.

In the $\mbox{\em CN}$ of Example~4 (which extends Example~3), we replace the symbol $R$
with the symbol $\in$.
More explicitly, we remove from $\Gamma$ of Example~4 the three
axiomatic statements containing $R$ and replace them with the
reducts of the following:
\begin{list}{}{\itemsep 0.0pt}
      \item[] $\langle\varnothing,T,\varnothing,
               \langle \mbox{wff\ }x\in y\rangle\rangle$
      \item[] $\langle\varnothing,T,
               \varnothing,
               \langle \vdash(x=y\to(x\in z\to y\in z))\rangle\rangle$
      \item[] $\langle\varnothing,T,
               \varnothing,
               \langle \vdash(x=y\to(z\in x\to z\in y))\rangle\rangle$
\end{list}
Letting $D=\{\{\alpha,\beta\}\in \mbox{\em DV}\,|\alpha,\beta\in \mbox{\em
Vr}\}$ (in other words all individual variables must be distinct), we extend
$\Gamma$ with the ZFC axioms, called
\index{Axiom of Extensionality}
\index{Axiom of Replacement}
\index{Axiom of Union}
\index{Axiom of Power Sets}
\index{Axiom of Regularity}
\index{Axiom of Infinity}
\index{Axiom of Choice}
Extensionality, Replacement, Union, Power
Set, Regularity, Infinity, and Choice, that are the reducts of:
\begin{list}{}{\itemsep 0.0pt}
      \item[Ext] $\langle D,T,
               \varnothing,
               \langle\vdash (\forall x(x\in y\leftrightarrow x \in z)\to y
               =z) \rangle\rangle$
      \item[Rep] $\langle D,T,
               \varnothing,
               \langle\vdash\exists x ( \exists y \forall z (\varphi \to z = y
                        ) \to
                        \forall z ( z \in x \leftrightarrow \exists x ( x \in
                        y \wedge \forall y\,\varphi ) ) )\rangle\rangle$
      \item[Un] $\langle D,T,
               \varnothing,
               \langle\vdash \exists x \forall y ( \exists x ( y \in x \wedge
               x \in z ) \to y \in x ) \rangle\rangle$
      \item[Pow] $\langle D,T,
               \varnothing,
               \langle\vdash \exists x \forall y ( \forall x ( x \in y \to x
               \in z ) \to y \in x ) \rangle\rangle$
      \item[Reg] $\langle D,T,
               \varnothing,
               \langle\vdash (  x \in y \to
                 \exists x ( x \in y \wedge \forall z ( z \in x \to \lnot z
                \in y ) ) ) \rangle\rangle$
      \item[Inf] $\langle D,T,
               \varnothing,
               \langle\vdash \exists x(y\in x\wedge\forall y(y\in
               x\to
               \exists z(y \in z\wedge z\in x))) \rangle\rangle$
      \item[AC] $\langle D,T,
               \varnothing,
               \langle\vdash \exists x \forall y \forall z ( ( y \in z
               \wedge z \in w ) \to \exists w \forall y ( \exists w
              ( ( y \in z \wedge z \in w ) \wedge ( y \in w \wedge w \in x
              ) ) \leftrightarrow y = w ) ) \rangle\rangle$
\end{list}

\subsection{Example~6---Class Notation in Set Theory}\label{class}

A powerful device that makes set theory easier (and that we have
been using all along in our informal expository language) is {\em class
abstraction notation}.\index{class abstraction}\index{abstraction class}  The
definitions we introduce are rigorously justified
as conservative by Takeuti and Zaring \cite{Takeuti}\index{Takeuti, Gaisi} or
Quine \cite{Quine}\index{Quine, Willard Van Orman}.  The key idea is to
introduce the notation $\{x|\mbox{---}\}$ which means ``the class of all $x$
such that ---'' for abstraction classes and introduce (meta)variables that
range over them.  An abstraction class may or may not be a set, depending on
whether it exists (as a set).  A class that does not exist is
called a {\em proper class}.\index{proper class}\index{class!proper}

To illustrate the use of abstraction classes we will provide some examples
of definitions that make use of them:  the empty set, class union, and
unordered pair.  Many other such definitions can be found in the
Metamath set theory database,
\texttt{set.mm}.\index{set theory database (\texttt{set.mm})}

% We intentionally break up the sequence of math symbols here
% because otherwise the overlong line goes beyond the page in narrow mode.
We extend $\mbox{\em CN}$ of Example~5 with new symbols $\{$
$\mbox{class},$ $\{,$ $|,$ $\},$ $\varnothing,$ $\cup,$ $,$ $\}$
where the inner braces and last comma are
constant symbols. (As before,
our dual use of some mathematical symbols for both our expository
language and as primitives of the formal system should be clear from context.)

We extend $\mbox{\em VR}$ of Example~5 with a set of {\em class
variables}\index{class variable}
$\{A,B,C,\ldots\}$. We extend the $T$ of Example~5 with $\{\langle
\mbox{class\ } A\rangle, \langle \mbox{class\ }B\rangle, \langle \mbox{class\ }
C\rangle,\ldots\}$.

To
introduce our definitions,
we add to $\Gamma$ of Example~5 the axiomatic statements
that are the reducts of the following pre-statements:
\begin{list}{}{\itemsep 0.0pt}
      \item[] $\langle\varnothing,T,\varnothing,
               \langle \mbox{class\ }x\rangle\rangle$
      \item[] $\langle\varnothing,T,\varnothing,
               \langle \mbox{class\ }\{x|\varphi\}\rangle\rangle$
      \item[] $\langle\varnothing,T,\varnothing,
               \langle \mbox{wff\ }A=B\rangle\rangle$
      \item[] $\langle\varnothing,T,\varnothing,
               \langle \mbox{wff\ }A\in B\rangle\rangle$
      \item[Ab] $\langle\varnothing,T,\varnothing,
               \langle \vdash ( y \in \{ x |\varphi\} \leftrightarrow
                  ( ( x = y \to\varphi) \wedge \exists x ( x = y
                  \wedge\varphi) ))
               \rangle\rangle$
      \item[Eq] $\langle\{\{x,A\},\{x,B\}\},T,\varnothing,
               \langle \vdash ( A = B \leftrightarrow
               \forall x ( x \in A \leftrightarrow x \in B ) )
               \rangle\rangle$
      \item[El] $\langle\{\{x,A\},\{x,B\}\},T,\varnothing,
               \langle \vdash ( A \in B \leftrightarrow \exists x
               ( x = A \wedge x \in B ) )
               \rangle\rangle$
\end{list}
Here we say that an individual variable is a class; $\{x|\varphi\}$ is a
class; and we extend the definition of a wff to include class equality and
membership.  Axiom Ab defines membership of a variable in a class abstraction;
the right-hand side can be read as ``the wff that results from proper
substitution of $y$ for $x$ in $\varphi$.''\footnote{Note that this definition
makes unnecessary the introduction of a separate notation similar to
$\varphi(x|y)$ for proper substitution, although we may choose to do so to be
conventional.  Incidentally, $\varphi(x|y)$ as it stands would be ambiguous in
the formal systems of our examples, since we wouldn't know whether
$\lnot\varphi(x|y)$ meant $\lnot(\varphi(x|y))$ or $(\lnot\varphi)(x|y)$.
Instead, we would have to use an unambiguous variant such as $(\varphi\,
x|y)$.}  Axioms Eq and El extend the meaning of the existing equality and
membership connectives.  This is potentially dangerous and requires careful
justification.  For example, from Eq we can derive the Axiom of Extensionality
with predicate logic alone; thus in principle we should include the Axiom of
Extensionality as a logical hypothesis.  However we do not bother to do this
since we have already presupposed that axiom earlier. The distinct variable
restrictions should be read ``where $x$ does not occur in $A$ or $B$.''  We
typically do this when the right-hand side of a definition involves an
individual variable not in the expression being defined; it is done so that
the right-hand side remains independent of the particular ``dummy'' variable
we use.

We continue to add to $\Gamma$ the following definitions
(i.e. the reducts of the following pre-statements) for empty
set,\index{empty set} class union,\index{union} and unordered
pair.\index{unordered pair}  They should be self-explanatory.  Analogous to our
use of ``$\leftrightarrow$'' to define new wffs in Example~4, we use ``$=$''
to define new abstraction terms, and both may be read informally as ``is
defined as'' in this context.
\begin{list}{}{\itemsep 0.0pt}
      \item[] $\langle\varnothing,T,\varnothing,
               \langle \mbox{class\ }\varnothing\rangle\rangle$
      \item[] $\langle\varnothing,T,\varnothing,
               \langle \vdash \varnothing = \{ x | \lnot x = x \}
               \rangle\rangle$
      \item[] $\langle\varnothing,T,\varnothing,
               \langle \mbox{class\ }(A\cup B)\rangle\rangle$
      \item[] $\langle\{\{x,A\},\{x,B\}\},T,\varnothing,
               \langle \vdash ( A \cup B ) = \{ x | ( x \in A \vee x \in B ) \}
               \rangle\rangle$
      \item[] $\langle\varnothing,T,\varnothing,
               \langle \mbox{class\ }\{A,B\}\rangle\rangle$
      \item[] $\langle\{\{x,A\},\{x,B\}\},T,\varnothing,
               \langle \vdash \{ A , B \} = \{ x | ( x = A \vee x = B ) \}
               \rangle\rangle$
\end{list}

\section{Metamath as a Formal System}\label{theorymm}

This section presupposes a familiarity with the Metamath computer language.

Our theory describes formal systems and their universes.  The Metamath
language provides a way of representing these set-theoretical objects to
a computer.  A Metamath database, being a finite set of {\sc ascii}
characters, can usually describe only a subset of a formal system and
its universe, which are typically infinite.  However the database can
contain as large a finite subset of the formal system and its universe
as we wish.  (Of course a Metamath set theory database can, in
principle, indirectly describe an entire infinite formal system by
formalizing the expository language in this Appendix.)

For purpose of our discussion, we assume the Metamath database
is in the simple form described on p.~\pageref{framelist},
consisting of all constant and variable declarations at the beginning,
followed by a sequence of extended frames each
delimited by \texttt{\$\char`\{} and \texttt{\$\char`\}}.  Any Metamath database can
be converted to this form, as described on p.~\pageref{frameconvert}.

The math symbol tokens of a Metamath source file, which are declared
with \texttt{\$c} and \texttt{\$v} statements, are names we assign to
representatives of $\mbox{\em CN}$ and $\mbox{\em VR}$.  For
definiteness we could assume that the first math symbol declared as a
variable corresponds to $v_0$, the second to $v_1$, etc., although the
exact correspondence we choose is not important.

In the Metamath language, each \texttt{\$d}, \texttt{\$f}, and
 \texttt{\$e} source
statement in an extended frame (Section~\ref{frames})
corresponds respectively to a member of the
collections $D$, $T$, and $H$ in a formal system statement $\langle
D_M,T_M,H,A\rangle$.  The math symbol strings following these Metamath keywords
correspond to a variable pair (in the case of \texttt{\$d}) or an expression (for
the other two keywords). The math symbol string following a \texttt{\$a} source
statement corresponds to expression $A$ in an axiomatic statement of the
formal system; the one following a \texttt{\$p} source statement corresponds to
$A$ in a provable statement that is not axiomatic.  In other words, each
extended frame in a Metamath database corresponds to
a pre-statement of the formal system, and a frame corresponds to
a statement of the formal system.  (Don't confuse the two meanings of
``statement'' here.  A statement of the formal system corresponds to the
several statements in a Metamath database that may constitute a
frame.)

In order for the computer to verify that a formal system statement is
provable, each \texttt{\$p} source statement is accompanied by a proof.
However, the proof does not correspond to anything in the formal system
but is simply a way of communicating to the computer the information
needed for its verification.  The proof tells the computer {\em how to
construct} specific members of closure of the formal system
pre-statement corresponding to the extended frame of the \texttt{\$p}
statement.  The final result of the construction is the member of the
closure that matches the \texttt{\$p} statement.  The abstract formal
system, on the other hand, is concerned only with the {\em existence} of
members of the closure.

As mentioned on p.~\pageref{exampleref}, Examples 1 and 3--6 in the
previous Section parallel the development of logic and set theory in the
Metamath database
\texttt{set.mm}.\index{set theory database (\texttt{set.mm})} You may
find it instructive to compare them.


\chapter{The MIU System}
\label{MIU}
\index{formal system}
\index{MIU-system}

The following is a listing of the file \texttt{miu.mm}.  It is self-explanatory.

%%%%%%%%%%%%%%%%%%%%%%%%%%%%%%%%%%%%%%%%%%%%%%%%%%%%%%%%%%%%

\begin{verbatim}
$( The MIU-system:  A simple formal system $)

$( Note:  This formal system is unusual in that it allows
empty wffs.  To work with a proof, you must type
SET EMPTY_SUBSTITUTION ON before using the PROVE command.
By default, this is OFF in order to reduce the number of
ambiguous unification possibilities that have to be selected
during the construction of a proof.  $)

$(
Hofstadter's MIU-system is a simple example of a formal
system that illustrates some concepts of Metamath.  See
Douglas R. Hofstadter, _Goedel, Escher, Bach:  An Eternal
Golden Braid_ (Vintage Books, New York, 1979), pp. 33ff. for
a description of the MIU-system.

The system has 3 constant symbols, M, I, and U.  The sole
axiom of the system is MI. There are 4 rules:
     Rule I:  If you possess a string whose last letter is I,
     you can add on a U at the end.
     Rule II:  Suppose you have Mx.  Then you may add Mxx to
     your collection.
     Rule III:  If III occurs in one of the strings in your
     collection, you may make a new string with U in place
     of III.
     Rule IV:  If UU occurs inside one of your strings, you
     can drop it.
Unfortunately, Rules III and IV do not have unique results:
strings could have more than one occurrence of III or UU.
This requires that we introduce the concept of an "MIU
well-formed formula" or wff, which allows us to construct
unique symbol sequences to which Rules III and IV can be
applied.
$)

$( First, we declare the constant symbols of the language.
Note that we need two symbols to distinguish the assertion
that a sequence is a wff from the assertion that it is a
theorem; we have arbitrarily chosen "wff" and "|-". $)
      $c M I U |- wff $. $( Declare constants $)

$( Next, we declare some variables. $)
     $v x y $.

$( Throughout our theory, we shall assume that these
variables represent wffs. $)
 wx   $f wff x $.
 wy   $f wff y $.

$( Define MIU-wffs.  We allow the empty sequence to be a
wff. $)

$( The empty sequence is a wff. $)
 we   $a wff $.
$( "M" after any wff is a wff. $)
 wM   $a wff x M $.
$( "I" after any wff is a wff. $)
 wI   $a wff x I $.
$( "U" after any wff is a wff. $)
 wU   $a wff x U $.

$( Assert the axiom. $)
 ax   $a |- M I $.

$( Assert the rules. $)
 ${
   Ia   $e |- x I $.
$( Given any theorem ending with "I", it remains a theorem
if "U" is added after it.  (We distinguish the label I_
from the math symbol I to conform to the 24-Jun-2006
Metamath spec.) $)
   I_    $a |- x I U $.
 $}
 ${
IIa  $e |- M x $.
$( Given any theorem starting with "M", it remains a theorem
if the part after the "M" is added again after it. $)
   II   $a |- M x x $.
 $}
 ${
   IIIa $e |- x I I I y $.
$( Given any theorem with "III" in the middle, it remains a
theorem if the "III" is replaced with "U". $)
   III  $a |- x U y $.
 $}
 ${
   IVa  $e |- x U U y $.
$( Given any theorem with "UU" in the middle, it remains a
theorem if the "UU" is deleted. $)
   IV   $a |- x y $.
  $}

$( Now we prove the theorem MUIIU.  You may be interested in
comparing this proof with that of Hofstadter (pp. 35 - 36).
$)
 theorem1  $p |- M U I I U $=
      we wM wU wI we wI wU we wU wI wU we wM we wI wU we wM
      wI wI wI we wI wI we wI ax II II I_ III II IV $.
\end{verbatim}\index{well-formed formula (wff)}

The \texttt{show proof /lemmon/renumber} command
yields the following display.  It is very similar
to the one in \cite[pp.~35--36]{Hofstadter}.\index{Hofstadter, Douglas R.}

\begin{verbatim}
1 ax             $a |- M I
2 1 II           $a |- M I I
3 2 II           $a |- M I I I I
4 3 I_           $a |- M I I I I U
5 4 III          $a |- M U I U
6 5 II           $a |- M U I U U I U
7 6 IV           $a |- M U I I U
\end{verbatim}

We note that Hofstadter's ``MU-puzzle,'' which asks whether
MU is a theorem of the MIU-system, cannot be answered using
the system above because the MU-puzzle is a question {\em
about} the system.  To prove the answer to the MU-puzzle,
a much more elaborate system is needed, namely one that
models the MIU-system within set theory.  (Incidentally, the
answer to the MU-puzzle is no.)

\chapter{Metamath Language EBNF}%
\label{BNF}%
\index{Metamath Language EBNF}

The following is a formal description of the basic Metamath language syntax
(with compressed proofs and support for unknown proof steps).
It is defined using the
Extended Backus--Naur Form (EBNF)\index{Extended Backus--Naur Form}\index{EBNF}
notation from W3C\index{W3C}
\textit{Extensible Markup Language (XML) 1.0 (Fifth Edition)}
(W3C Recommendation 26 November 2008) at
\url{https://www.w3.org/TR/xml/#sec-notation}.

The \texttt{database}
rule is processed until the end of the file (\texttt{EOF}).
The rules eventually require reading whitespace-separated tokens.
A token has an upper-case definition (see below)
or is a string constant in a non-token (such as \texttt{'\$a'}).
We intend for this to be correct, but if there is a conflict the
rules of section \ref{spec} govern. That section also discusses
non-syntax restrictions not shown here
(e.g., that each new label token
defined in a \texttt{hypothesis-stmt} or \texttt{assert-stmt}
must be unique).

\begin{verbatim}
database ::= outermost-scope-stmt*

outermost-scope-stmt ::=
  include-stmt | constant-stmt | stmt

/* File inclusion command; process file as a database.
   Databases should NOT have a comment in the filename. */
include-stmt ::= '$[' filename '$]'

/* Constant symbols declaration. */
constant-stmt ::= '$c' constant+ '$.'

/* A normal statement can occur in any scope. */
stmt ::= block | variable-stmt | disjoint-stmt |
  hypothesis-stmt | assert-stmt

/* A block. You can have 0 statements in a block. */
block ::= '${' stmt* '$}'

/* Variable symbols declaration. */
variable-stmt ::= '$v' variable+ '$.'

/* Disjoint variables. Simple disjoint statements have
   2 variables, i.e., "variable*" is empty for them. */
disjoint-stmt ::= '$d' variable variable variable* '$.'

hypothesis-stmt ::= floating-stmt | essential-stmt

/* Floating (variable-type) hypothesis. */
floating-stmt ::= LABEL '$f' typecode variable '$.'

/* Essential (logical) hypothesis. */
essential-stmt ::= LABEL '$e' typecode MATH-SYMBOL* '$.'

assert-stmt ::= axiom-stmt | provable-stmt

/* Axiomatic assertion. */
axiom-stmt ::= LABEL '$a' typecode MATH-SYMBOL* '$.'

/* Provable assertion. */
provable-stmt ::= LABEL '$p' typecode MATH-SYMBOL*
  '$=' proof '$.'

/* A proof. Proofs may be interspersed by comments.
   If '?' is in a proof it's an "incomplete" proof. */
proof ::= uncompressed-proof | compressed-proof
uncompressed-proof ::= (LABEL | '?')+
compressed-proof ::= '(' LABEL* ')' COMPRESSED-PROOF-BLOCK+

typecode ::= constant

filename ::= MATH-SYMBOL /* No whitespace or '$' */
constant ::= MATH-SYMBOL
variable ::= MATH-SYMBOL
\end{verbatim}

\needspace{2\baselineskip}
A \texttt{frame} is a sequence of 0 or more
\texttt{disjoint-{\allowbreak}stmt} and
\texttt{hypotheses-{\allowbreak}stmt} statements
(possibly interleaved with other non-\texttt{assert-stmt} statements)
followed by one \texttt{assert-stmt}.

\needspace{3\baselineskip}
Here are the rules for lexical processing (tokenization) beyond
the constant tokens shown above.
By convention these tokenization rules have upper-case names.
Every token is read for the longest possible length.
Whitespace-separated tokens are read sequentially;
note that the separating whitespace and \texttt{\$(} ... \texttt{\$)}
comments are skipped.

If a token definition uses another token definition, the whole thing
is considered a single token.
A pattern that is only part of a full token has a name beginning
with an underscore (``\_'').
An implementation could tokenize many tokens as a
\texttt{PRINTABLE-SEQUENCE}
and then check if it meets the more specific rule shown here.

Comments do not nest, and both \texttt{\$(} and \texttt{\$)}
have to be surrounded
by at least one whitespace character (\texttt{\_WHITECHAR}).
Technically comments end without consuming the trailing
\texttt{\_WHITECHAR}, but the trailing
\texttt{\_WHITECHAR} gets ignored anyway so we ignore that detail here.
Metamath language processors
are not required to support \texttt{\$)} followed
immediately by a bare end-of-file, because the closing
comment symbol is supposed to be followed by a
\texttt{\_WHITECHAR} such as a newline.

\begin{verbatim}
PRINTABLE-SEQUENCE ::= _PRINTABLE-CHARACTER+

MATH-SYMBOL ::= (_PRINTABLE-CHARACTER - '$')+

/* ASCII non-whitespace printable characters */
_PRINTABLE-CHARACTER ::= [#x21-#x7e]

LABEL ::= ( _LETTER-OR-DIGIT | '.' | '-' | '_' )+

_LETTER-OR-DIGIT ::= [A-Za-z0-9]

COMPRESSED-PROOF-BLOCK ::= ([A-Z] | '?')+

/* Define whitespace between tokens. The -> SKIP
   means that when whitespace is seen, it is
   skipped and we simply read again. */
WHITESPACE ::= (_WHITECHAR+ | _COMMENT) -> SKIP

/* Comments. $( ... $) and do not nest. */
_COMMENT ::= '$(' (_WHITECHAR+ (PRINTABLE-SEQUENCE - '$)')*
  _WHITECHAR+ '$)' _WHITECHAR

/* Whitespace: (' ' | '\t' | '\r' | '\n' | '\f') */
_WHITECHAR ::= [#x20#x09#x0d#x0a#x0c]
\end{verbatim}
% This EBNF was developed as a collaboration between
% David A. Wheeler\index{Wheeler, David A.},
% Mario Carneiro\index{Carneiro, Mario}, and
% Benoit Jubin\index{Jubin, Benoit}, inspired by a request
% (and a lot of initial work) by Benoit Jubin.
%
% \chapter{Disclaimer and Trademarks}
%
% Information in this document is subject to change without notice and does not
% represent a commitment on the part of Norman Megill.
% \vspace{2ex}
%
% \noindent Norman D. Megill makes no warranties, either express or implied,
% regarding the Metamath computer software package.
%
% \vspace{2ex}
%
% \noindent Any trademarks mentioned in this book are the property of
% their respective owners.  The name ``Metamath'' is a trademark of
% Norman Megill.
%
\cleardoublepage
\phantomsection  % fixes the link anchor
\addcontentsline{toc}{chapter}{\bibname}

\bibliography{metamath}
%% metamath.tex - Version of 24-May-2019
% If you change the date above, also change the "Printed date" below.
% SPDX-License-Identifier: CC0-1.0
%
%                              PUBLIC DOMAIN
%
% This file (specifically, the version of this file with the above date)
% has been released into the Public Domain per the
% Creative Commons CC0 1.0 Universal (CC0 1.0) Public Domain Dedication
% https://creativecommons.org/publicdomain/zero/1.0/
%
% The public domain release applies worldwide.  In case this is not
% legally possible, the right is granted to use the work for any purpose,
% without any conditions, unless such conditions are required by law.
%
% Several short, attributed quotations from copyrighted works
% appear in this file under the ``fair use'' provision of Section 107 of
% the United States Copyright Act (Title 17 of the {\em United States
% Code}).  The public-domain status of this file is not applicable to
% those quotations.
%
% Norman Megill - email: nm(at)alum(dot)mit(dot)edu
%
% David A. Wheeler also donates his improvements to this file to the
% public domain per the CC0.  He works at the Institute for Defense Analyses
% (IDA), but IDA has agreed that this Metamath work is outside its "lane"
% and is not a work by IDA.  This was specifically confirmed by
% Margaret E. Myers (Division Director of the Information Technology
% and Systems Division) on 2019-05-24 and by Ben Lindorf (General Counsel)
% on 2019-05-22.

% This file, 'metamath.tex', is self-contained with everything needed to
% generate the the PDF file 'metamath.pdf' (the _Metamath_ book) on
% standard LaTeX 2e installations.  The auxiliary files are embedded with
% "filecontents" commands.  To generate metamath.pdf file, run these
% commands under Linux or Cygwin in the directory that contains
% 'metamath.tex':
%
%   rm -f realref.sty metamath.bib
%   touch metamath.ind
%   pdflatex metamath
%   pdflatex metamath
%   bibtex metamath
%   makeindex metamath
%   pdflatex metamath
%   pdflatex metamath
%
% The warnings that occur in the initial runs of pdflatex can be ignored.
% For the final run,
%
%   egrep -i 'error|warn' metamath.log
%
% should show exactly these 5 warnings:
%
%   LaTeX Warning: File `realref.sty' already exists on the system.
%   LaTeX Warning: File `metamath.bib' already exists on the system.
%   LaTeX Font Warning: Font shape `OMS/cmtt/m/n' undefined
%   LaTeX Font Warning: Font shape `OMS/cmtt/bx/n' undefined
%   LaTeX Font Warning: Some font shapes were not available, defaults
%       substituted.
%
% Search for "Uncomment" below if you want to suppress hyperlink boxes
% in the PDF output file
%
% TYPOGRAPHICAL NOTES:
% * It is customary to use an en dash (--) to "connect" names of different
%   people (and to denote ranges), and use a hyphen (-) for a
%   single compound name. Examples of connected multiple people are
%   Zermelo--Fraenkel, Schr\"{o}der--Bernstein, Tarski--Grothendieck,
%   Hewlett--Packard, and Backus--Naur.  Examples of a single person with
%   a compound name include Levi-Civita, Mittag-Leffler, and Burali-Forti.
% * Use non-breaking spaces after page abbreviations, e.g.,
%   p.~\pageref{note2002}.
%
% --------------------------- Start of realref.sty -----------------------------
\begin{filecontents}{realref.sty}
% Save the following as realref.sty.
% You can then use it with \usepackage{realref}
%
% This has \pageref jumping to the page on which the ref appears,
% \ref jumping to the point of the anchor, and \sectionref
% jumping to the start of section.
%
% Author:  Anthony Williams
%          Software Engineer
%          Nortel Networks Optical Components Ltd
% Date:    9 Nov 2001 (posted to comp.text.tex)
%
% The following declaration was made by Anthony Williams on
% 24 Jul 2006 (private email to Norman Megill):
%
%   ``I hereby donate the code for realref.sty posted on the
%   comp.text.tex newsgroup on 9th November 2001, accessible from
%   http://groups.google.com/group/comp.text.tex/msg/5a0e1cc13ea7fbb2
%   to the public domain.''
%
\ProvidesPackage{realref}
\RequirePackage[plainpages=false,pdfpagelabels=true]{hyperref}
\def\realref@anchorname{}
\AtBeginDocument{%
% ensure every label is a possible hyperlink target
\let\realref@oldrefstepcounter\refstepcounter%
\DeclareRobustCommand{\refstepcounter}[1]{\realref@oldrefstepcounter{#1}
\edef\realref@anchorname{\string #1.\@currentlabel}%
}%
\let\realref@oldlabel\label%
\DeclareRobustCommand{\label}[1]{\realref@oldlabel{#1}\hypertarget{#1}{}%
\@bsphack\protected@write\@auxout{}{%
    \string\expandafter\gdef\protect\csname
    page@num.#1\string\endcsname{\thepage}%
    \string\expandafter\gdef\protect\csname
    ref@num.#1\string\endcsname{\@currentlabel}%
    \string\expandafter\gdef\protect\csname
    sectionref@name.#1\string\endcsname{\realref@anchorname}%
}\@esphack}%
\DeclareRobustCommand\pageref[1]{{\edef\a{\csname
            page@num.#1\endcsname}\expandafter\hyperlink{page.\a}{\a}}}%
\DeclareRobustCommand\ref[1]{{\edef\a{\csname
            ref@num.#1\endcsname}\hyperlink{#1}{\a}}}%
\DeclareRobustCommand\sectionref[1]{{\edef\a{\csname
            ref@num.#1\endcsname}\edef\b{\csname
            sectionref@name.#1\endcsname}\hyperlink{\b}{\a}}}%
}
\end{filecontents}
% ---------------------------- End of realref.sty ------------------------------

% --------------------------- Start of metamath.bib -----------------------------
\begin{filecontents}{metamath.bib}
@book{Albers, editor = "Donald J. Albers and G. L. Alexanderson",
  title = "Mathematical People",
  publisher = "Contemporary Books, Inc.",
  address = "Chicago",
  note = "[QA28.M37]",
  year = 1985 }
@book{Anderson, author = "Alan Ross Anderson and Nuel D. Belnap",
  title = "Entailment",
  publisher = "Princeton University Press",
  address = "Princeton",
  volume = 1,
  note = "[QA9.A634 1975 v.1]",
  year = 1975}
@book{Barrow, author = "John D. Barrow",
  title = "Theories of Everything:  The Quest for Ultimate Explanation",
  publisher = "Oxford University Press",
  address = "Oxford",
  note = "[Q175.B225]",
  year = 1991 }
@book{Behnke,
  editor = "H. Behnke and F. Backmann and K. Fladt and W. S{\"{u}}ss",
  title = "Fundamentals of Mathematics",
  volume = "I",
  publisher = "The MIT Press",
  address = "Cambridge, Massachusetts",
  note = "[QA37.2.B413]",
  year = 1974 }
@book{Bell, author = "J. L. Bell and M. Machover",
  title = "A Course in Mathematical Logic",
  publisher = "North-Holland",
  address = "Amsterdam",
  note = "[QA9.B3953]",
  year = 1977 }
@inproceedings{Blass, author = "Andrea Blass",
  title = "The Interaction Between Category Theory and Set Theory",
  pages = "5--29",
  booktitle = "Mathematical Applications of Category Theory (Proceedings
     of the Special Session on Mathematical Applications
     Category Theory, 89th Annual Meeting of the American Mathematical
     Society, held in Denver, Colorado January 5--9, 1983)",
  editor = "John Walter Gray",
  year = 1983,
  note = "[QA169.A47 1983]",
  publisher = "American Mathematical Society",
  address = "Providence, Rhode Island"}
@proceedings{Bledsoe, editor = "W. W. Bledsoe and D. W. Loveland",
  title = "Automated Theorem Proving:  After 25 Years (Proceedings
     of the Special Session on Automatic Theorem Proving,
     89th Annual Meeting of the American Mathematical
     Society, held in Denver, Colorado January 5--9, 1983)",
  year = 1983,
  note = "[QA76.9.A96.S64 1983]",
  publisher = "American Mathematical Society",
  address = "Providence, Rhode Island" }
@book{Boolos, author = "George S. Boolos and Richard C. Jeffrey",
  title = "Computability and Log\-ic",
  publisher = "Cambridge University Press",
  edition = "third",
  address = "Cambridge",
  note = "[QA9.59.B66 1989]",
  year = 1989 }
@book{Campbell, author = "John Campbell",
  title = "Programmer's Progress",
  publisher = "White Star Software",
  address = "Box 51623, Palo Alto, CA 94303",
  year = 1991 }
@article{DBLP:journals/corr/Carneiro14,
  author    = {Mario Carneiro},
  title     = {Conversion of {HOL} Light proofs into Metamath},
  journal   = {CoRR},
  volume    = {abs/1412.8091},
  year      = {2014},
  url       = {http://arxiv.org/abs/1412.8091},
  archivePrefix = {arXiv},
  eprint    = {1412.8091},
  timestamp = {Mon, 13 Aug 2018 16:47:05 +0200},
  biburl    = {https://dblp.org/rec/bib/journals/corr/Carneiro14},
  bibsource = {dblp computer science bibliography, https://dblp.org}
}
@article{CarneiroND,
  author    = {Mario Carneiro},
  title     = {Natural Deductions in the Metamath Proof Language},
  url       = {http://us.metamath.org/ocat/natded.pdf},
  year      = 2014
}
@inproceedings{Chou, author = "Shang-Ching Chou",
  title = "Proving Elementary Geometry Theorems Using {W}u's Algorithm",
  pages = "243--286",
  booktitle = "Automated Theorem Proving:  After 25 Years (Proceedings
     of the Special Session on Automatic Theorem Proving,
     89th Annual Meeting of the American Mathematical
     Society, held in Denver, Colorado January 5--9, 1983)",
  editor = "W. W. Bledsoe and D. W. Loveland",
  year = 1983,
  note = "[QA76.9.A96.S64 1983]",
  publisher = "American Mathematical Society",
  address = "Providence, Rhode Island" }
@book{Clemente, author = "Daniel Clemente Laboreo",
  title = "Introduction to natural deduction",
  year = 2014,
  url = "http://www.danielclemente.com/logica/dn.en.pdf" }
@incollection{Courant, author = "Richard Courant and Herbert Robbins",
  title = "Topology",
  pages = "573--590",
  booktitle = "The World of Mathematics, Volume One",
  editor = "James R. Newman",
  publisher = "Simon and Schuster",
  address = "New York",
  note = "[QA3.W67 1988]",
  year = 1956 }
@book{Curry, author = "Haskell B. Curry",
  title = "Foundations of Mathematical Logic",
  publisher = "Dover Publications, Inc.",
  address = "New York",
  note = "[QA9.C976 1977]",
  year = 1977 }
@book{Davis, author = "Philip J. Davis and Reuben Hersh",
  title = "The Mathematical Experience",
  publisher = "Birkh{\"{a}}user Boston",
  address = "Boston",
  note = "[QA8.4.D37 1982]",
  year = 1981 }
@incollection{deMillo,
  author = "Richard de Millo and Richard Lipton and Alan Perlis",
  title = "Social Processes and Proofs of Theorems and Programs",
  pages = "267--285",
  booktitle = "New Directions in the Philosophy of Mathematics",
  editor = "Thomas Tymoczko",
  publisher = "Birkh{\"{a}}user Boston, Inc.",
  address = "Boston",
  note = "[QA8.6.N48 1986]",
  year = 1986 }
@book{Edwards, author = "Robert E. Edwards",
  title = "A Formal Background to Mathematics",
  publisher = "Springer-Verlag",
  address = "New York",
  note = "[QA37.2.E38 v.1a]",
  year = 1979 }
@book{Enderton, author = "Herbert B. Enderton",
  title = "Elements of Set Theory",
  publisher = "Academic Press, Inc.",
  address = "San Diego",
  note = "[QA248.E5]",
  year = 1977 }
@book{Goodstein, author = "R. L. Goodstein",
  title = "Development of Mathematical Logic",
  publisher = "Springer-Verlag New York Inc.",
  address = "New York",
  note = "[QA9.G6554]",
  year = 1971 }
@book{Guillen, author = "Michael Guillen",
  title = "Bridges to Infinity",
  publisher = "Jeremy P. Tarcher, Inc.",
  address = "Los Angeles",
  note = "[QA93.G8]",
  year = 1983 }
@book{Hamilton, author = "Alan G. Hamilton",
  title = "Logic for Mathematicians",
  edition = "revised",
  publisher = "Cambridge University Press",
  address = "Cambridge",
  note = "[QA9.H298]",
  year = 1988 }
@unpublished{Harrison, author = "John Robert Harrison",
  title = "Metatheory and Reflection in Theorem Proving:
    A Survey and Critique",
  note = "Technical Report
    CRC-053.
    SRI Cambridge,
    Millers Yard, Cambridge, UK,
    1995.
    Available on the Web as
{\verb+http:+}\-{\verb+//www.cl.cam.ac.uk/users/jrh/papers/reflect.html+}"}
@TECHREPORT{Harrison-thesis,
        author          = "John Robert Harrison",
        title           = "Theorem Proving with the Real Numbers",
        institution   = "University of Cambridge Computer
                         Lab\-o\-ra\-to\-ry",
        address         = "New Museums Site, Pembroke Street, Cambridge,
                           CB2 3QG, UK",
        year            = 1996,
        number          = 408,
        type            = "Technical Report",
        note            = "Author's PhD thesis,
   available on the Web at
{\verb+http:+}\-{\verb+//www.cl.cam.ac.uk+}\-{\verb+/users+}\-{\verb+/jrh+}%
\-{\verb+/papers+}\-{\verb+/thesis.html+}"}
@book{Herrlich, author = "Horst Herrlich and George E. Strecker",
  title = "Category Theory:  An Introduction",
  publisher = "Allyn and Bacon Inc.",
  address = "Boston",
  note = "[QA169.H567]",
  year = 1973 }
@article{Hindley, author = "J. Roger Hindley and David Meredith",
  title = "Principal Type-Schemes and Condensed Detachment",
  journal = "The Journal of Symbolic Logic",
  volume = 55,
  year = 1990,
  note = "[QA.J87]",
  pages = "90--105" }
@book{Hofstadter, author = "Douglas R. Hofstadter",
  title = "G{\"{o}}del, Escher, Bach",
  publisher = "Basic Books, Inc.",
  address = "New York",
  note = "[QA9.H63 1980]",
  year = 1979 }
@article{Indrzejczak, author= "Andrzej Indrzejczak",
  title = "Natural Deduction, Hybrid Systems and Modal Logic",
  journal = "Trends in Logic",
  volume = 30,
  publisher = "Springer",
  year = 2010 }
@article{Kalish, author = "D. Kalish and R. Montague",
  title = "On {T}arski's Formalization of Predicate Logic with Identity",
  journal = "Archiv f{\"{u}}r Mathematische Logik und Grundlagenfor\-schung",
  volume = 7,
  year = 1965,
  note = "[QA.A673]",
  pages = "81--101" }
@article{Kalman, author = "J. A. Kalman",
  title = "Condensed Detachment as a Rule of Inference",
  journal = "Studia Logica",
  volume = 42,
  number = 4,
  year = 1983,
  note = "[B18.P6.S933]",
  pages = "443-451" }
@book{Kline, author = "Morris Kline",
  title = "Mathematical Thought from Ancient to Modern Times",
  publisher = "Oxford University Press",
  address = "New York",
  note = "[QA21.K516 1990 v.3]",
  year = 1972 }
@book{Klinel, author = "Morris Kline",
  title = "Mathematics, The Loss of Certainty",
  publisher = "Oxford University Press",
  address = "New York",
  note = "[QA21.K525]",
  year = 1980 }
@book{Kramer, author = "Edna E. Kramer",
  title = "The Nature and Growth of Modern Mathematics",
  publisher = "Princeton University Press",
  address = "Princeton, New Jersey",
  note = "[QA93.K89 1981]",
  year = 1981 }
@article{Knill, author = "Oliver Knill",
  title = "Some Fundamental Theorems in Mathematics",
  year = "2018",
  url = "https://arxiv.org/abs/1807.08416" }
@book{Landau, author = "Edmund Landau",
  title = "Foundations of Analysis",
  publisher = "Chelsea Publishing Company",
  address = "New York",
  edition = "second",
  note = "[QA241.L2541 1960]",
  year = 1960 }
@article{Leblanc, author = "Hugues Leblanc",
  title = "On {M}eyer and {L}ambert's Quantificational Calculus {FQ}",
  journal = "The Journal of Symbolic Logic",
  volume = 33,
  year = 1968,
  note = "[QA.J87]",
  pages = "275--280" }
@article{Lejewski, author = "Czeslaw Lejewski",
  title = "On Implicational Definitions",
  journal = "Studia Logica",
  volume = 8,
  year = 1958,
  note = "[B18.P6.S933]",
  pages = "189--208" }
@book{Levy, author = "Azriel Levy",
  title = "Basic Set Theory",
  publisher = "Dover Publications",
  address = "Mineola, NY",
  year = "2002"
}
@book{Margaris, author = "Angelo Margaris",
  title = "First Order Mathematical Logic",
  publisher = "Blaisdell Publishing Company",
  address = "Waltham, Massachusetts",
  note = "[QA9.M327]",
  year = 1967}
@book{Manin, author = "Yu I. Manin",
  title = "A Course in Mathematical Logic",
  publisher = "Springer-Verlag",
  address = "New York",
  note = "[QA9.M29613]",
  year = "1977" }
@article{Mathias, author = "Adrian R. D. Mathias",
  title = "A Term of Length 4,523,659,424,929",
  journal = "Synthese",
  volume = 133,
  year = 2002,
  note = "[Q.S993]",
  pages = "75--86" }
@article{Megill, author = "Norman D. Megill",
  title = "A Finitely Axiomatized Formalization of Predicate Calculus
     with Equality",
  journal = "Notre Dame Journal of Formal Logic",
  volume = 36,
  year = 1995,
  note = "[QA.N914]",
  pages = "435--453" }
@unpublished{Megillc, author = "Norman D. Megill",
  title = "A Shorter Equivalent of the Axiom of Choice",
  month = "June",
  note = "Unpublished",
  year = 1991 }
@article{MegillBunder, author = "Norman D. Megill and Martin W.
    Bunder",
  title = "Weaker {D}-Complete Logics",
  journal = "Journal of the IGPL",
  volume = 4,
  year = 1996,
  pages = "215--225",
  note = "Available on the Web at
{\verb+http:+}\-{\verb+//www.mpi-sb.mpg.de+}\-{\verb+/igpl+}%
\-{\verb+/Journal+}\-{\verb+/V4-2+}\-{\verb+/#Megill+}"}
}
@book{Mendelson, author = "Elliott Mendelson",
  title = "Introduction to Mathematical Logic",
  edition = "second",
  publisher = "D. Van Nostrand Company, Inc.",
  address = "New York",
  note = "[QA9.M537 1979]",
  year = 1979 }
@article{Meredith, author = "David Meredith",
  title = "In Memoriam {C}arew {A}rthur {M}eredith (1904-1976)",
  journal = "Notre Dame Journal of Formal Logic",
  volume = 18,
  year = 1977,
  note = "[QA.N914]",
  pages = "513--516" }
@article{CAMeredith, author = "C. A. Meredith",
  title = "Single Axioms for the Systems ({C},{N}), ({C},{O}) and ({A},{N})
      of the Two-Valued Propositional Calculus",
  journal = "The Journal of Computing Systems",
  volume = 3,
  year = 1953,
  pages = "155--164" }
@article{Monk, author = "J. Donald Monk",
  title = "Provability With Finitely Many Variables",
  journal = "The Journal of Symbolic Logic",
  volume = 27,
  year = 1971,
  note = "[QA.J87]",
  pages = "353--358" }
@article{Monks, author = "J. Donald Monk",
  title = "Substitutionless Predicate Logic With Identity",
  journal = "Archiv f{\"{u}}r Mathematische Logik und Grundlagenfor\-schung",
  volume = 7,
  year = 1965,
  pages = "103--121" }
  %% Took out this from above to prevent LaTeX underfull warning:
  % note = "[QA.A673]",
@book{Moore, author = "A. W. Moore",
  title = "The Infinite",
  publisher = "Routledge",
  address = "New York",
  note = "[BD411.M59]",
  year = 1989}
@book{Munkres, author = "James R. Munkres",
  title = "Topology: A First Course",
  publisher = "Prentice-Hall, Inc.",
  address = "Englewood Cliffs, New Jersey",
  note = "[QA611.M82]",
  year = 1975}
@article{Nemesszeghy, author = "E. Z. Nemesszeghy and E. A. Nemesszeghy",
  title = "On Strongly Creative Definitions:  A Reply to {V}. {F}. {R}ickey",
  journal = "Logique et Analyse (N.\ S.)",
  year = 1977,
  volume = 20,
  note = "[BC.L832]",
  pages = "111--115" }
@unpublished{Nemeti, author = "N{\'{e}}meti, I.",
  title = "Algebraizations of Quantifier Logics, an Overview",
  note = "Version 11.4, preprint, Mathematical Institute, Budapest,
    1994.  A shortened version without proofs appeared in
    ``Algebraizations of quantifier logics, an introductory overview,''
   {\em Studia Logica}, 50:485--569, 1991 [B18.P6.S933]"}
@article{Pavicic, author = "M. Pavi{\v{c}}i{\'{c}}",
  title = "A New Axiomatization of Unified Quantum Logic",
  journal = "International Journal of Theoretical Physics",
  year = 1992,
  volume = 31,
  note = "[QC.I626]",
  pages = "1753 --1766" }
@book{Penrose, author = "Roger Penrose",
  title = "The Emperor's New Mind",
  publisher = "Oxford University Press",
  address = "New York",
  note = "[Q335.P415]",
  year = 1989 }
@book{PetersonI, author = "Ivars Peterson",
  title = "The Mathematical Tourist",
  publisher = "W. H. Freeman and Company",
  address = "New York",
  note = "[QA93.P475]",
  year = 1988 }
@article{Peterson, author = "Jeremy George Peterson",
  title = "An automatic theorem prover for substitution and detachment systems",
  journal = "Notre Dame Journal of Formal Logic",
  volume = 19,
  year = 1978,
  note = "[QA.N914]",
  pages = "119--122" }
@book{Quine, author = "Willard Van Orman Quine",
  title = "Set Theory and Its Logic",
  edition = "revised",
  publisher = "The Belknap Press of Harvard University Press",
  address = "Cambridge, Massachusetts",
  note = "[QA248.Q7 1969]",
  year = 1969 }
@article{Robinson, author = "J. A. Robinson",
  title = "A Machine-Oriented Logic Based on the Resolution Principle",
  journal = "Journal of the Association for Computing Machinery",
  year = 1965,
  volume = 12,
  pages = "23--41" }
@article{RobinsonT, author = "T. Thacher Robinson",
  title = "Independence of Two Nice Sets of Axioms for the Propositional
    Calculus",
  journal = "The Journal of Symbolic Logic",
  volume = 33,
  year = 1968,
  note = "[QA.J87]",
  pages = "265--270" }
@book{Rucker, author = "Rudy Rucker",
  title = "Infinity and the Mind:  The Science and Philosophy of the
    Infinite",
  publisher = "Bantam Books, Inc.",
  address = "New York",
  note = "[QA9.R79 1982]",
  year = 1982 }
@book{Russell, author = "Bertrand Russell",
  title = "Mysticism and Logic, and Other Essays",
  publisher = "Barnes \& Noble Books",
  address = "Totowa, New Jersey",
  note = "[B1649.R963.M9 1981]",
  year = 1981 }
@article{Russell2, author = "Bertrand Russell",
  title = "Recent Work on the Principles of Mathematics",
  journal = "International Monthly",
  volume = 4,
  year = 1901,
  pages = "84"}
@article{Schmidt, author = "Eric Schmidt",
  title = "Reductions in Norman Megill's axiom system for complex numbers",
  url = "http://us.metamath.org/downloads/schmidt-cnaxioms.pdf",
  year = "2012" }
@book{Shoenfield, author = "Joseph R. Shoenfield",
  title = "Mathematical Logic",
  publisher = "Addison-Wesley Publishing Company, Inc.",
  address = "Reading, Massachusetts",
  year = 1967,
  note = "[QA9.S52]" }
@book{Smullyan, author = "Raymond M. Smullyan",
  title = "Theory of Formal Systems",
  publisher = "Princeton University Press",
  address = "Princeton, New Jersey",
  year = 1961,
  note = "[QA248.5.S55]" }
@book{Solow, author = "Daniel Solow",
  title = "How to Read and Do Proofs:  An Introduction to Mathematical
    Thought Process",
  publisher = "John Wiley \& Sons",
  address = "New York",
  year = 1982,
  note = "[QA9.S577]" }
@book{Stark, author = "Harold M. Stark",
  title = "An Introduction to Number Theory",
  publisher = "Markham Publishing Company",
  address = "Chicago",
  note = "[QA241.S72 1978]",
  year = 1970 }
@article{Swart, author = "E. R. Swart",
  title = "The Philosophical Implications of the Four-Color Problem",
  journal = "American Mathematical Monthly",
  year = 1980,
  volume = 87,
  month = "November",
  note = "[QA.A5125]",
  pages = "697--707" }
@book{Szpiro, author = "George G. Szpiro",
  title = "Poincar{\'{e}}'s Prize: The Hundred-Year Quest to Solve One
    of Math's Greatest Puzzles",
  publisher = "Penguin Books Ltd",
  address = "London",
  note = "[QA43.S985 2007]",
  year = 2007}
@book{Takeuti, author = "Gaisi Takeuti and Wilson M. Zaring",
  title = "Introduction to Axiomatic Set Theory",
  edition = "second",
  publisher = "Springer-Verlag New York Inc.",
  address = "New York",
  note = "[QA248.T136 1982]",
  year = 1982}
@inproceedings{Tarski, author = "Alfred Tarski",
  title = "What is Elementary Geometry",
  pages = "16--29",
  booktitle = "The Axiomatic Method, with Special Reference to Geometry and
     Physics (Proceedings of an International Symposium held at the University
     of California, Berkeley, December 26, 1957 --- January 4, 1958)",
  editor = "Leon Henkin and Patrick Suppes and Alfred Tarski",
  year = 1959,
  publisher = "North-Holland Publishing Company",
  address = "Amsterdam"}
@article{Tarski1965, author = "Alfred Tarski",
  title = "A Simplified Formalization of Predicate Logic with Identity",
  journal = "Archiv f{\"{u}}r Mathematische Logik und Grundlagenforschung",
  volume = 7,
  year = 1965,
  note = "[QA.A673]",
  pages = "61--79" }
@book{Tymoczko,
  title = "New Directions in the Philosophy of Mathematics",
  editor = "Thomas Tymoczko",
  publisher = "Birkh{\"{a}}user Boston, Inc.",
  address = "Boston",
  note = "[QA8.6.N48 1986]",
  year = 1986 }
@incollection{Wang,
  author = "Hao Wang",
  title = "Theory and Practice in Mathematics",
  pages = "129--152",
  booktitle = "New Directions in the Philosophy of Mathematics",
  editor = "Thomas Tymoczko",
  publisher = "Birkh{\"{a}}user Boston, Inc.",
  address = "Boston",
  note = "[QA8.6.N48 1986]",
  year = 1986 }
@manual{Webster,
  title = "Webster's New Collegiate Dictionary",
  organization = "G. \& C. Merriam Co.",
  address = "Springfield, Massachusetts",
  note = "[PE1628.W4M4 1977]",
  year = 1977 }
@manual{Whitehead, author = "Alfred North Whitehead",
  title = "An Introduction to Mathematics",
  year = 1911 }
@book{PM, author = "Alfred North Whitehead and Bertrand Russell",
  title = "Principia Mathematica",
  edition = "second",
  publisher = "Cambridge University Press",
  address = "Cambridge",
  year = "1927",
  note = "(3 vols.) [QA9.W592 1927]" }
@article{DBLP:journals/corr/Whalen16,
  author    = {Daniel Whalen},
  title     = {Holophrasm: a neural Automated Theorem Prover for higher-order logic},
  journal   = {CoRR},
  volume    = {abs/1608.02644},
  year      = {2016},
  url       = {http://arxiv.org/abs/1608.02644},
  archivePrefix = {arXiv},
  eprint    = {1608.02644},
  timestamp = {Mon, 13 Aug 2018 16:46:19 +0200},
  biburl    = {https://dblp.org/rec/bib/journals/corr/Whalen16},
  bibsource = {dblp computer science bibliography, https://dblp.org} }
@article{Wiedijk-revisited,
  author = {Freek Wiedijk},
  title = {The QED Manifesto Revisited},
  year = {2007},
  url = {http://mizar.org/trybulec65/8.pdf} }
@book{Wolfram,
  author = "Stephen Wolfram",
  title = "Mathematica:  A System for Doing Mathematics by Computer",
  edition = "second",
  publisher = "Addison-Wesley Publishing Co.",
  address = "Redwood City, California",
  note = "[QA76.95.W65 1991]",
  year = 1991 }
@book{Wos, author = "Larry Wos and Ross Overbeek and Ewing Lusk and Jim Boyle",
  title = "Automated Reasoning:  Introduction and Applications",
  edition = "second",
  publisher = "McGraw-Hill, Inc.",
  address = "New York",
  note = "[QA76.9.A96.A93 1992]",
  year = 1992 }

%
%
%[1] Church, Alonzo, Introduction to Mathematical Logic,
% Volume 1, Princeton University Press, Princeton, N. J., 1956.
%
%[2] Cohen, Paul J., Set Theory and the Continuum Hypothesis,
% W. A. Benjamin, Inc., Reading, Mass., 1966.
%
%[3] Hamilton, Alan G., Logic for Mathematicians, Cambridge
% University Press,
% Cambridge, 1988.

%[6] Kleene, Stephen Cole, Introduction to Metamathematics, D.  Van
% Nostrand Company, Inc., Princeton (1952).

%[13] Tarski, Alfred, "A simplified formalization of predicate
% logic with identity," Archiv fur Mathematische Logik und
% Grundlagenforschung, vol. 7 (1965), pp. 61-79.

%[14] Tarski, Alfred and Steven Givant, A Formalization of Set
% Theory Without Variables, American Mathematical Society Colloquium
% Publications, vol. 41, American Mathematical Society,
% Providence, R. I., 1987.

%[15] Zeman, J. J., Modal Logic, Oxford University Press, Oxford, 1973.
\end{filecontents}
% --------------------------- End of metamath.bib -----------------------------


%Book: Metamath
%Author:  Norman Megill Email:  nm at alum.mit.edu
%Author:  David A. Wheeler Email:  dwheeler at dwheeler.com

% A book template example
% http://www.stsci.edu/ftp/software/tex/bookstuff/book.template

\documentclass[leqno]{book} % LaTeX 2e. 10pt. Use [leqno,12pt] for 12pt
% hyperref 2002/05/27 v6.72r  (couldn't get pagebackref to work)
\usepackage[plainpages=false,pdfpagelabels=true]{hyperref}

\usepackage{needspace}     % Enable control over page breaks
\usepackage{breqn}         % automatic equation breaking
\usepackage{microtype}     % microtypography, reduces hyphenation

% Packages for flexible tables.  We need to be able to
% wrap text within a cell (with automatically-determined widths) AND
% split a table automatically across multiple pages.
% * "tabularx" wraps text in cells but only 1 page
% * "longtable" goes across pages but by itself is incompatible with tabularx
% * "ltxtable" combines longtable and tabularx, but table contents
%    must be in a separate file.
% * "ltablex" combines tabularx and longtable - must install specially
% * "booktabs" is recommended as a way to improve the look of tables,
%   but doesn't add these capabilities.
% * "tabu" much more capable and seems to be recommended. So use that.

\usepackage{makecell}      % Enable forced line splits within a table cell
\usepackage{longtable}     % Enable multi-page tables
\usepackage{tabu}          % Multi-page tables with wrapped text in a cell

% You can find more Tex packages using commands like:
% tlmgr search --file tabu.sty
% find /usr/share/texmf-dist/ -name '*tab*'
%
%%%%%%%%%%%%%%%%%%%%%%%%%%%%%%%%%%%%%%%%%%%%%%%%%%%%%%%%%%%%%%%%%%%%%%%%%%%%
% Uncomment the next 3 lines to suppress boxes and colors on the hyperlinks
%%%%%%%%%%%%%%%%%%%%%%%%%%%%%%%%%%%%%%%%%%%%%%%%%%%%%%%%%%%%%%%%%%%%%%%%%%%%
%\hypersetup{
%colorlinks,citecolor=black,filecolor=black,linkcolor=black,urlcolor=black
%}
%
\usepackage{realref}

% Restarting page numbers: try?
%   \printglossary
%   \cleardoublepage
%   \pagenumbering{arabic}
%   \setcounter{page}{1}    ???needed
%   \include{chap1}

% not used:
% \def\R2Lurl#1#2{\mbox{\href{#1}\texttt{#2}}}

\usepackage{amssymb}

% Version 1 of book: margins: t=.4, b=.2, ll=.4, rr=.55
% \usepackage{anysize}
% % \papersize{<height>}{<width>}
% % \marginsize{<left>}{<right>}{<top>}{<bottom>}
% \papersize{9in}{6in}
% % l/r 0.6124-0.6170 works t/b 0.2418-0.3411 = 192pp. 0.2926-03118=exact
% \marginsize{0.7147in}{0.5147in}{0.4012in}{0.2012in}

\usepackage{anysize}
% \papersize{<height>}{<width>}
% \marginsize{<left>}{<right>}{<top>}{<bottom>}
\papersize{9in}{6in}
% l/r 0.85in&0.6431-0.6539 works t/b ?-?
%\marginsize{0.85in}{0.6485in}{0.55in}{0.35in}
\marginsize{0.8in}{0.65in}{0.5in}{0.3in}

% \usepackage[papersize={3.6in,4.8in},hmargin=0.1in,vmargin={0.1in,0.1in}]{geometry}  % page geometry
\usepackage{special-settings}

\raggedbottom
\makeindex

\begin{document}
% Discourage page widows and orphans:
\clubpenalty=300
\widowpenalty=300

%%%%%%% load in AMS fonts %%%%%%% % LaTeX 2.09 - obsolete in LaTeX 2e
%\input{amssym.def}
%\input{amssym.tex}
%\input{c:/texmf/tex/plain/amsfonts/amssym.def}
%\input{c:/texmf/tex/plain/amsfonts/amssym.tex}

\bibliographystyle{plain}
\pagenumbering{roman}
\pagestyle{headings}

\thispagestyle{empty}

\hfill
\vfill

\begin{center}
{\LARGE\bf Metamath} \\
\vspace{1ex}
{\large A Computer Language for Mathematical Proofs} \\
\vspace{7ex}
{\large Norman Megill} \\
\vspace{7ex}
with extensive revisions by \\
\vspace{1ex}
{\large David A. Wheeler} \\
\vspace{7ex}
% Printed date. If changing the date below, also fix the date at the beginning.
2019-05-24
\end{center}

\vfill
\hfill

\newpage
\thispagestyle{empty}

\hfill
\vfill

\begin{center}
$\sim$\ {\sc Public Domain}\ $\sim$

\vspace{2ex}
This book (including its later revisions)
has been released into the Public Domain by Norman Megill per the
Creative Commons CC0 1.0 Universal (CC0 1.0) Public Domain Dedication
(\url{https://creativecommons.org/publicdomain/zero/1.0/}).
David A. Wheeler has done the same.
The public domain release applies worldwide.  In case this is not
legally possible, the right is granted to use the work for any purpose,
without any conditions, unless such conditions are required by law.

\vspace{3ex}
Several short, attributed quotations from copyrighted works
appear in this book under the ``fair use'' provision of Section 107 of
the United States Copyright Act (Title 17 of the {\em United States
Code}).  The public-domain status of this book is not applicable to
those quotations.

\vspace{3ex}
Any trademarks used in this book are the property of their owners.

% QA76.9.L63.M??

% \vspace{1ex}
%
% \vspace{1ex}
% {\small Permission is granted to make and distribute verbatim copies of this
% book
% provided the copyright notice and this
% permission notice are preserved on all copies.}
%
% \vspace{1ex}
% {\small Permission is granted to copy and distribute modified versions of this
% book under the conditions for verbatim copying, provided that the
% entire
% resulting derived work is distributed under the terms of a permission
% notice
% identical to this one.}
%
% \vspace{1ex}
% {\small Permission is granted to copy and distribute translations of this
% book into another language, under the above conditions for modified
% versions,
% except that this permission notice may be stated in a translation
% approved by the
% author.}
%
% \vspace{1ex}
% %{\small   For a copy of the \LaTeX\ source files for this book, contact
% %the author.} \\
% \ \\
% \ \\

\vspace{7ex}
% ISBN: 1-4116-3724-0 \\
ISBN: 978-1-4116-3724-5 \\
{\ } \\
Lulu Press \\
Morrisville, North Carolina\\
USA


\hfill
\vfill

Norman Megill\\ 93 Bridge St., Lexington, MA 02421 \\
E-mail address: \texttt{nm{\char`\@}alum.mit.edu} \\
\vspace{7ex}
David A. Wheeler \\
E-mail address: \texttt{dwheeler{\char`\@}dwheeler.com} \\
% See notes added at end of Preface for revision history. \\
% For current information on the Metamath software see \\
\vspace{7ex}
\url{http://metamath.org}
\end{center}

% \newpage
% \thispagestyle{empty}
%
% \hfill
% \vfill
%
% \begin{center}
% {\it To my son Robin Dwight Megill}
% \end{center}
%
% \vfill
% \hfill
%
% \newpage

\tableofcontents
%\listoftables

\chapter*{Preface}
\markboth{PREFACE}{PREFACE}
\addcontentsline{toc}{section}{Preface}


% (For current information, see the notes added at the
% end of this preface on p.~\pageref{note2002}.)

\subsubsection{Overview}

Metamath\index{Metamath} is a computer language and an associated computer
program for archiving, verifying, and studying mathematical proofs at a very
detailed level.  The Metamath language incorporates no mathematics per se but
treats all mathematical statements as mere sequences of symbols.  You provide
Metamath with certain special sequences (axioms) that tell it what rules
of inference are allowed.  Metamath is not limited to any specific field of
mathematics.  The Metamath language is simple and robust, with an
almost total absence of hard-wired syntax, and
I\footnote{Unless otherwise noted, the words
``I,'' ``me,'' and ``my'' refer to Norman Megill\index{Megill, Norman}, while
``we,'' ``us,'' and ``our'' refer to Norman Megill and
David A. Wheeler\index{Wheeler, David A.}.}
believe that it
provides about the simplest possible framework that allows essentially all of
mathematics to be expressed with absolute rigor.

% index test
%\newcommand{\nn}[1]{#1n}
%\index{aaa@bbb}
%\index{abc!def}
%\index{abd|see{qqq}}
%\index{abe|nn}
%\index{abf|emph}
%\index{abg|(}
%\index{abg|)}

Using the Metamath language, you can build formal or mathematical
systems\index{formal system}\footnote{A formal or mathematical system consists
of a collection of symbols (such as $2$, $4$, $+$ and $=$), syntax rules that
describe how symbols may be combined to form a legal expression (called a
well-formed formula or {\em wff}, pronounced ``whiff''), some starting wffs
called axioms, and inference rules that describe how theorems may be derived
(proved) from the axioms.  A theorem is a mathematical fact such as $2+2=4$.
Strictly speaking, even an obvious fact such as this must be proved from
axioms to be formally acceptable to a mathematician.}\index{theorem}
\index{axiom}\index{rule}\index{well-formed formula (wff)} that involve
inferences from axioms.  Although a database is provided
that includes a recommended set of axioms for standard mathematics, if you
wish you can supply your own symbols, syntax, axioms, rules, and definitions.

The name ``Metamath'' was chosen to suggest that the language provides a
means for {\em describing} mathematics rather than {\em being} the
mathematics itself.  Actually in some sense any mathematical language is
metamathematical.  Symbols written on paper, or stored in a computer,
are not mathematics itself but rather a way of expressing mathematics.
For example ``7'' and ``VII'' are symbols for denoting the number seven
in Arabic and Roman numerals; neither {\em is} the number seven.

If you are able to understand and write computer programs, you should be able
to follow abstract mathematics with the aid of Metamath.  Used in conjunction
with standard textbooks, Metamath can guide you step-by-step towards an
understanding of abstract mathematics from a very rigorous viewpoint, even if
you have no formal abstract mathematics background.  By using a single,
consistent notation to express proofs, once you grasp its basic concepts
Metamath provides you with the ability to immediately follow and dissect
proofs even in totally unfamiliar areas.

Of course, just being able follow a proof will not necessarily give you an
intuitive familiarity with mathematics.  Memorizing the rules of chess does not
give you the ability to appreciate the game of a master, and knowing how the
notes on a musical score map to piano keys does not give you the ability to
hear in your head how it would sound.  But each of these can be a first step.

Metamath allows you to explore proofs in the sense that you can see the
theorem referenced at any step expanded in as much detail as you want, right
down to the underlying axioms of logic and set theory (in the case of the set
theory database provided).  While Metamath will not replace the higher-level
understanding that can only be acquired through exercises and hard work, being
able to see how gaps in a proof are filled in can give you increased
confidence that can speed up the learning process and save you time when you
get stuck.

The Metamath language breaks down a mathematical proof into its tiniest
possible parts.  These can be pieced together, like interlocking
pieces in a puzzle, only in a way that produces correct and absolutely rigorous
mathematics.

The nature of Metamath\index{Metamath} enforces very precise mathematical
thinking, similar to that involved in writing a computer program.  A crucial
difference, though, is that once a proof is verified (by the Metamath program)
to be correct, it is definitely correct; it can never have a hidden
``bug.''\index{computer program bugs}  After getting used to the kind of rigor
and accuracy provided by Metamath, you might even be tempted to
adopt the attitude that a proof should never be considered correct until it
has been verified by a computer, just as you would not completely trust a
manual calculation until you have verified it on a
calculator.

My goal
for Metamath was a system for describing and verifying
mathematics that is completely universal yet conceptually as simple as
possible.  In approaching mathematics from an axiomatic, formal viewpoint, I
wanted Metamath to be able to handle almost any mathematical system, not
necessarily with ease, but at least in principle and hopefully in practice. I
wanted it to verify proofs with absolute rigor, and for this reason Metamath
is what might be thought of as a ``compile-only'' language rather than an
algorithmic or Turing-machine language (Pascal, C, Prolog, Mathematica,
etc.).  In other words, a database written in the Metamath
language doesn't ``do'' anything; it merely exhibits mathematical knowledge
and permits this knowledge to be verified as being correct.  A program in an
algorithmic language can potentially have hidden bugs\index{computer program
bugs} as well as possibly being hard to understand.  But each token in a
Metamath database must be consistent with the database's earlier
contents according to simple, fixed rules.
If a database is verified
to be correct,\footnote{This includes
verification that a sequential list of proof steps results in the specified
theorem.} then the mathematical content is correct if the
verifier is correct and the axioms are correct.
The verification program could be incorrect, but the verification algorithm
is relatively simple (making it unlikely to be implemented incorrectly
by the Metamath program),
and there are over a dozen Metamath database verifiers
written by different people in different programming languages
(so these different verifiers can act as multiple reviewers of a database).
The most-used Metamath database, the Metamath Proof Explorer
(aka \texttt{set.mm}\index{set theory database (\texttt{set.mm})}%
\index{Metamath Proof Explorer}),
is currently verified by four different Metamath verifiers written by
four different people in four different languages, including the
original Metamath program described in this book.
The only ``bugs'' that can exist are in the statement of the axioms,
for example if the axioms are inconsistent (a famous problem shown to be
unsolvable by G\"{o}del's incompleteness theorem\index{G\"{o}del's
incompleteness theorem}).
However, real mathematical systems have very few axioms, and these can
be carefully studied.
All of this provides extraordinarily high confidence that the verified database
is in fact correct.

The Metamath program
doesn't prove theorems automatically but is designed to verify proofs
that you supply to it.
The underlying Metamath language is completely general and has no built-in,
preconceived notions about your formal system\index{formal system}, its logic
or its syntax.
For constructing proofs, the Metamath program has a Proof Assistant\index{Proof
Assistant} which helps you fill in some of a proof step's details, shows you
what choices you have at any step, and verifies the proof as you build it; but
you are still expected to provide the proof.

There are many other programs that can process or generate information
in the Metamath language, and more continue to be written.
This is in part because the Metamath language itself is very simple
and intentionally easy to automatically process.
Some programs, such as \texttt{mmj2}\index{mmj2}, include a proof assistant
that can automate some steps beyond what the Metamath program can do.
Mario Carneiro has developed an algorithm for converting proofs from
the OpenTheory interchange format, which can be translated to and from
any of the HOL family of proof languages (HOL4, HOL Light, ProofPower,
and Isabelle), into the
Metamath language.\cite{DBLP:journals/corr/Carneiro14}\index{Carneiro, Mario}.
Daniel Whalen has developed Holophrasm, which can automatically
prove many Metamath proofs using
machine learning\index{machine learning}\index{artificial intelligence}
approaches
(including multiple neural networks\index{neural networks})\cite{DBLP:journals/corr/Whalen16}\index{Whalen, Daniel}.
However,
a discussion of these other programs is beyond the scope of this book.

Like most computer languages, the Metamath\index{Metamath} language uses the
standard ({\sc ascii}) characters on a computer keyboard, so it cannot
directly represent many of the special symbols that mathematicians use.  A
useful feature of the Metamath program is its ability to convert its notation
into the \LaTeX\ typesetting language.\index{latex@{\LaTeX}}  This feature
lets you convert the {\sc ascii} tokens you've defined into standard
mathematical symbols, so you end up with symbols and formulas you are familiar
with instead of somewhat cryptic {\sc ascii} representations of them.
The Metamath program can also generate HTML\index{HTML}, making it easy
to view results on the web and to see related information by using
hypertext links.

Metamath is probably conceptually different from anything you've seen
before and some aspects may take some getting used to.  This book will
help you decide whether Metamath suits your specific needs.

\subsubsection{Setting Your Expectations}
It is important for you to understand what Metamath\index{Metamath} is and is
not.  As mentioned, the Metamath program
is {\em not} an automated theorem prover but
rather a proof verifier.  Developing a database can be tedious, hard work,
especially if you want to make the proofs as short as possible, but it becomes
easier as you build up a collection of useful theorems.  The purpose of
Metamath is simply to document existing mathematics in an absolutely rigorous,
computer-verifiable way, not to aid directly in the creation of new
mathematics.  It also is not a magic solution for learning abstract
mathematics, although it may be helpful to be able to actually see the implied
rigor behind what you are learning from textbooks, as well as providing hints
to work out proofs that you are stumped on.

As of this writing, a sizable set theory database has been developed to
provide a foundation for many fields of mathematics, but much more work would
be required to develop useful databases for specific fields.

Metamath\index{Metamath} ``knows no math;'' it just provides a framework in
which to express mathematics.  Its language is very small.  You can define two
kinds of symbols, constants\index{constant} and variables\index{variable}.
The only thing Metamath knows how to do is to substitute strings of symbols
for the variables\index{substitution!variable}\index{variable substitution} in
an expression based on instructions you provide it in a proof, subject to
certain constraints you specify for the variables.  Even the decimal
representation of a number is merely a string of certain constants (digits)
which together, in a specific context, correspond to whatever mathematical
object you choose to define for it; unlike other computer languages, there is
no actual number stored inside the computer.  In a proof, you in effect
instruct Metamath what symbol substitutions to make in previous axioms or
theorems and join a sequence of them together to result in the desired
theorem.  This kind of symbol manipulation captures the essence of mathematics
at a preaxiomatic level.

\subsubsection{Metamath and Mathematical Literature}

In advanced mathematical literature, proofs are usually presented in the form
of short outlines that often only an expert can follow.  This is partly out of
a desire for brevity, but it would also be unwise (even if it were practical)
to present proofs in complete formal detail, since the overall picture would
be lost.\index{formal proof}

A solution I envision\label{envision} that would allow mathematics to remain
acceptable to the expert, yet increase its accessibility to non-specialists,
consists of a combination of the traditional short, informal proof in print
accompanied by a complete formal proof stored in a computer database.  In an
analogy with a computer program, the informal proof is like a set of comments
that describe the overall reasoning and content of the proof, whereas the
computer database is like the actual program and provides a means for anyone,
even a non-expert, to follow the proof in as much detail as desired, exploring
it back through layers of theorems (like subroutines that call other
subroutines) all the way back to the axioms of the theory.  In addition, the
computer database would have the advantage of providing absolute assurance
that the proof is correct, since each step can be verified automatically.

There are several other approaches besides Metamath to a project such
as this.  Section~\ref{proofverifiers} discusses some of these.

To us, a noble goal would be a database with hundreds of thousands of
theorems and their computer-verifiable proofs, encompassing a significant
fraction of known mathematics and available for instant access.
These would be fully verified by multiple independently-implemented verifiers,
to provide extremely high confidence that the proofs are completely correct.
The database would allow people to investigate whatever details they were
interested in, so that they could confirm whatever portions they wished.
Whether or not Metamath is an appropriate choice remains to be seen, but in
principle we believe it is sufficient.

\subsubsection{Formalism}

Over the past fifty years, a group of French mathematicians working
collectively under the pseudonym of Bourbaki\index{Bourbaki, Nicolas} have
co-authored a series of monographs that attempt to rigorously and
consistently formalize large bodies of mathematics from foundations.  On the
one hand, certainly such an effort has its merits; on the other hand, the
Bourbaki project has been criticized for its ``scholasticism'' and
``hyperaxiomatics'' that hide the intuitive steps that lead to the results
\cite[p.~191]{Barrow}\index{Barrow, John D.}.

Metamath unabashedly carries this philosophy to its extreme and no doubt is
subject to the same kind of criticism.  Nonetheless I think that in
conjunction with conventional approaches to mathematics Metamath can serve a
useful purpose.  The Bourbaki approach is essentially pedagogic, requiring the
reader to become intimately familiar with each detail in a very large
hierarchy before he or she can proceed to the next step.  The difference with
Metamath is that the ``reader'' (user) knows that all details are contained in
its computer database, available as needed; it does not demand that the user
know everything but conveniently makes available those portions that are of
interest.  As the body of all mathematical knowledge grows larger and larger,
no one individual can have a thorough grasp of its entirety.  Metamath
can finalize and put to rest any questions about the validity of any part of it
and can make any part of it accessible, in principle, to a non-specialist.

\subsubsection{A Personal Note}
Why did I develop Metamath\index{Metamath}?  I enjoy abstract mathematics, but
I sometimes get lost in a barrage of definitions and start to lose confidence
that my proofs are correct.  Or I reach a point where I lose sight of how
anything I'm doing relates to the axioms that a theory is based on and am
sometimes suspicious that there may be some overlooked implicit axiom
accidentally introduced along the way (as happened historically with Euclidean
geometry\index{Euclidean geometry}, whose omission of Pasch's
axiom\index{Pasch's axiom} went unnoticed for 2000 years
\cite[p.~160]{Davis}!). I'm also somewhat lazy and wish to avoid the effort
involved in re-verifying the gaps in informal proofs ``left to the reader;'' I
prefer to figure them out just once and not have to go through the same
frustration a year from now when I've forgotten what I did.  Metamath provides
better recovery of my efforts than scraps of paper that I can't
decipher anymore.  But mostly I find very appealing the idea of rigorously
archiving mathematical knowledge in a computer database, providing precision,
certainty, and elimination of human error.

\subsubsection{Note on Bibliography and Index}

The Bibliography usually includes the Library of Congress classification
for a work to make it easier for you to find it in on a university
library shelf.  The Index has author references to pages where their works
are cited, even though the authors' names may not appear on those pages.

\subsubsection{Acknowledgments}

Acknowledgments are first due to my wife, Deborah (who passed away on
September 4, 1998), for critiquing the manu\-script but most of all for
her patience and support.  I also wish to thank Joe Wright, Richard
Becker, Clarke Evans, Buddha Buck, and Jeremy Henty for helpful
comments.  Any errors, omissions, and other shortcomings are of course
my responsibility.

\subsubsection{Note Added June 22, 2005}\label{note2002}

The original, unpublished version of this book was written in 1997 and
distributed via the web.  The present edition has been updated to
reflect the current Metamath program and databases, as well as more
current {\sc url}s for Internet sites.  Thanks to Josh
Purinton\index{Purinton, Josh}, One Hand
Clapping, Mel L.\ O'Cat, and Roy F. Longton for pointing out
typographical and other errors.  I have also benefitted from numerous
discussions with Raph Levien\index{Levien, Raph}, who has extended
Metamath's philosophy of rigor to result in his {\em
Ghilbert}\index{Ghilbert} proof language (\url{http://ghilbert.org}).

Robert (Bob) Solovay\index{Solovay, Robert} communicated a new result of
A.~R.~D.~Mathias on the system of Bourbaki, and the text has been
updated accordingly (p.~\pageref{bourbaki}).

Bob also pointed out a clarification of the literature regarding
category theory and inaccessible cardinals\index{category
theory}\index{cardinal, inaccessible} (p.~\pageref{categoryth}),
and a misleading statement was removed from the text.  Specifically,
contrary to a statement in previous editions, it is possible to express
``There is a proper class of inaccessible cardinals'' in the language of
ZFC.  This can be done as follows:  ``For every set $x$ there is an
inaccessible cardinal $\kappa$ such that $\kappa$ is not in $x$.''
Bob writes:\footnote{Private communication, Nov.~30, 2002.}
\begin{quotation}
     This axiom is how Grothendieck presents category theory.  To each
inaccessible cardinal $\kappa$ one associates a Grothendieck universe
\index{Grothendieck, Alexander} $U(\kappa)$.  $U(\kappa)$ consists of
those sets which lie in a transitive set of cardinality less than
$\kappa$.  Instead of the ``category of all groups,'' one works relative
to a universe [considering the category of groups of cardinality less
than $\kappa$].  Now the category whose objects are all categories
``relative to the universe $U(\kappa)$'' will be a category not
relative to this universe but to the next universe.

     All of the things category theorists like to do can be done in this
framework.  The only controversial point is whether the Grothen\-dieck
axiom is too strong for the needs of category theorists.  Mac Lane
\index{Mac Lane, Saunders} argues that ``one universe is enough'' and
Feferman\index{Feferman, Solomon} has argued that one can get by with
ordinary ZFC.  I don't find Feferman's arguments persuasive.  Mac Lane
may be right, but when I think about category theory I do it \`{a} la
Grothendieck.

        By the way Mizar\index{Mizar} adds the axiom ``there is a proper
class of inaccessibles'' precisely so as to do category theory.
\end{quotation}

The most current information on the Metamath program and databases can
always be found at \url{http://metamath.org}.


\subsubsection{Note Added June 24, 2006}\label{note2006}

The Metamath spec was restricted slightly to make parsers easier to
write.  See the footnote on p.~\pageref{namespace}.

%\subsubsection{Note Added July 24, 2006}\label{note2006b}
\subsubsection{Note Added March 10, 2007}\label{note2006b}

I am grateful to Anthony Williams\index{Williams, Anthony} for writing
the \LaTeX\ package called {\tt realref.sty} and contributing it to the
public domain.  This package allows the internal hyperlinks in a {\sc
pdf} file to anchor to specific page numbers instead of just section
titles, making the navigation of the {\sc pdf} file for this book much
more pleasant and ``logical.''

A typographical error found by Martin Kiselkov was corrected.
A confusing remark about unification was deleted per suggestion of
Mel O'Cat.

\subsubsection{Note Added May 27, 2009}\label{note2009}

Several typos found by Kim Sparre were corrected.  A note was added that
the Poincar\'{e} conjecture has been proved (p.~\pageref{poincare}).

\subsubsection{Note Added Nov. 17, 2014}\label{note2014}

The statement of the Schr\"{o}der--Bernstein theorem was corrected in
Section~\ref{trust}.  Thanks to Bob Solovay for pointing out the error.

\subsubsection{Note Added May 25, 2016}\label{note2016}

Thanks to Jerry James for correcting 16 typos.

\subsubsection{Note Added February 25, 2019}\label{note201902}

David A. Wheeler\index{Wheeler, David A.}
made a large number of improvements and updates,
in coordination with Norman Megill.
The predicate calculus axioms were renumbered, and the text makes
it clear that they are based on Tarski's system S2;
the one slight deviation in axiom ax-6 is explained and justified.
The real and complex number axioms were modified to be consistent with
\texttt{set.mm}\index{set theory database (\texttt{set.mm})}%
\index{Metamath Proof Explorer}.
Long-awaited specification changes ``1--8'' were made,
which clarified previously ambiguous points.
Some errors in the text involving \texttt{\$f} and
\texttt{\$d} statements were corrected (the spec was correct, but
the in-book explanations unintentionally contradicted the spec).
We now have a system for automatically generating narrow PDFs,
so that those with smartphones can have easy access to the current
version of this document.
A new section on deduction was added;
it discusses the standard deduction theorem,
the weak deduction theorem,
deduction style, and natural deduction.
Many minor corrections (too numerous to list here) were also made.

\subsubsection{Note Added March 7, 2019}\label{note201903}

This added a description of the Matamath language syntax in
Extended Backus--Naur Form (EBNF)\index{Extended Backus--Naur Form}\index{EBNF}
in Appendix \ref{BNF}, added a brief explanation about typecodes,
inserted more examples in the deduction section,
and added a variety of smaller improvements.

\subsubsection{Note Added April 7, 2019}\label{note201904}

This version clarified the proper substitution notation, improved the
discussion on the weak deduction theorem and natural deduction,
documented the \texttt{undo} command, updated the information on
\texttt{write source}, changed the typecode
from \texttt{set} to \texttt{setvar} to be consistent with the current
version of \texttt{set.mm}, added more documentation about comment markup
(e.g., documented how to create headings), and clarified the
differences between various assertion forms (in particular deduction form).

\subsubsection{Note Added May 20, 2019}\label{note201905}

This version fixes various small issues and added specific examples
of forms (deduction form, inference form, and closed form).

\chapter{Introduction}
\pagenumbering{arabic}

\begin{quotation}
  {\em {\em I.M.:}  No, no.  There's nothing subjective about it!  Everybody
knows what a proof is.  Just read some books, take courses from a competent
mathematician, and you'll catch on.

{\em Student:}  Are you sure?

{\em I.M.:}  Well---it is possible that you won't, if you don't have any
aptitude for it.  That can happen, too.

{\em Student:}  Then {\em you} decide what a proof is, and if I don't learn
to decide in the same way, you decide I don't have any aptitude.

{\em I.M.:}  If not me, then who?}
    \flushright\sc  ``The Ideal Mathematician''
    \index{Davis, Phillip J.}
    \footnote{\cite{Davis}, p.~40.}\\
\end{quotation}

In the past century, brilliant mathematicians have discovered almost
unimaginably profound results that rank among the crowning intellectual
achievements of mankind.  However, there is a sense in which modern abstract
mathematics is behind the times, stuck in an era before computers existed.
While no one disputes the remarkable results that have been achieved,
communicating these results in a precise way to the uninitiated is virtually
impossible.  To describe these results, a terse informal language is used which
despite its elegance is very difficult to learn.  This informal language is not
imprecise, far from it, but rather it often has omitted detail
and symbols with hidden context that are
implicitly understood by an expert but few others.  Extremely complex technical
meanings are associated with innocent-sounding English words such as
``compact'' and ``measurable'' that barely hint at what is actually being
said.  Anyone who does not keep the precise technical meaning constantly in
mind is bound to fail, and acquiring the ability to do this can be achieved
only through much practice and hard work.  Only the few who complete the
painful learning experience can join the small in-group of pure
mathematicians.  The informal language effectively cuts off the true nature of
their knowledge from most everyone else.

Metamath\index{Metamath} makes abstract mathematics more concrete.  It allows
a computer to keep track of the complexity associated with each word or symbol
with absolute rigor.  You can explore this complexity at your leisure, to
whatever degree you desire.  Whether or not you believe that concepts such as
infinity actually ``exist'' outside of the mind, Metamath lets you get to the
foundation for what's really being said.

Metamath also enables completely rigorous and thorough proof verification.
Its language is simple enough so that you
don't have to rely on the authority of experts but can verify the results
yourself, step by step.  If you want to attempt to derive your own results,
Metamath will not let you make a mistake in reasoning.
Even professional mathematicians make mistakes; Metamath makes it possible
to thoroughly verify that proofs are correct.

Metamath\index{Metamath} is a computer language and an associated computer
program for archiving, verifying, and studying mathematical proofs at a very
detailed level.
The Metamath language
describes formal\index{formal system} mathematical
systems and expresses proofs of theorems in those systems.  Such a language
is called a metalanguage\index{metalanguage} by mathematicians.
The Metamath program is a computer program that verifies
proofs expressed in the Metamath language.
The Metamath program does not have the built-in
ability to make logical inferences; it just makes a series of symbol
substitutions according to instructions given to it in a proof
and verifies that the result matches the expected theorem.  It makes logical
inferences based only on rules of logic that are contained in a set of
axioms\index{axiom}, or first principles, that you provide to it as the
starting point for proofs.

The complete specification of the Metamath language is only four pages long
(Section~\ref{spec}, p.~\pageref{spec}).  Its simplicity may at first make you
wonder how it can do much of anything at all.  But in fact the kinds of
symbol manipulations it performs are the ones that are implicitly done in all
mathematical systems at the lowest level.  You can learn it relatively quickly
and have complete confidence in any mathematical proof that it verifies.  On
the other hand, it is powerful and general enough so that virtually any
mathematical theory, from the most basic to the deeply abstract, can be
described with it.

Although in principle Metamath can be used with any
kind of mathematics, it is best suited for abstract or ``pure'' mathematics
that is mostly concerned with theorems and their proofs, as opposed to the
kind of mathematics that deals with the practical manipulation of numbers.
Examples of branches of pure mathematics are logic\index{logic},\footnote{Logic
is the study of statements that are universally true regardless of the objects
being described by the statements.  An example is the statement, ``if $P$
implies $Q$, then either $P$ is false or $Q$ is true.''} set theory\index{set
theory},\footnote{Set theory is the study of general-purpose mathematical objects called
``sets,'' and from it essentially all of mathematics can be derived.  For
example, numbers can be defined as specific sets, and their properties
can be explored using the tools of set theory.} number theory\index{number
theory},\footnote{Number theory deals with the properties of positive and
negative integers (whole numbers).} group theory\index{group
theory},\footnote{Group theory studies the properties of mathematical objects
called groups that obey a simple set of axioms and have properties of symmetry
that make them useful in many other fields.} abstract algebra\index{abstract
algebra},\footnote{Abstract algebra includes group theory and also studies
groups with additional properties that qualify them as ``rings'' and
``fields.''  The set of real numbers is a familiar example of a field.},
analysis\index{analysis} \index{real and complex numbers}\footnote{Analysis is
the study of real and complex numbers.} and
topology\index{topology}.\footnote{One area studied by topology are properties
that remain unchanged when geometrical objects undergo stretching
deformations; for example a doughnut and a coffee cup each have one hole (the
cup's hole is in its handle) and are thus considered topologically
equivalent.  In general, though, topology is the study of abstract
mathematical objects that obey a certain (surprisingly simple) set of axioms.
See, for example, Munkres \cite{Munkres}\index{Munkres, James R.}.} Even in
physics, Metamath could be applied to certain branches that make use of
abstract mathematics, such as quantum logic\index{quantum logic} (used to study
aspects of quantum mechanics\index{quantum mechanics}).

On the other hand, Metamath\index{Metamath} is less suited to applications
that deal primarily with intensive numeric computations.  Metamath does not
have any built-in representation of numbers\index{Metamath!representation of
numbers}; instead, a specific string of symbols (digits) must be syntactically
constructed as part of any proof in which an ordinary number is used.  For
this reason, numbers in Metamath are best limited to specific constants that
arise during the course of a theorem or its proof.  Numbers are only a tiny
part of the world of abstract mathematics.  The exclusion of built-in numbers
was a conscious decision to help achieve Metamath's simplicity, and there are
other software tools if you have different mathematical needs.
If you wish to quickly solve algebraic problems, the computer algebra
programs\index{computer algebra system} {\sc
macsyma}\index{macsyma@{\sc macsyma}}, Mathematica\index{Mathematica}, and
Maple\index{Maple} are specifically suited to handling numbers and
algebra efficiently.
If you wish to simply calculate numeric or matrix expressions easily,
tools such as Octave\index{Octave} may be a better choice.

After learning Metamath's basic statement types, any
tech\-ni\-cal\-ly ori\-ent\-ed person, mathematician or not, can
immediately trace
any theorem proved in the language as far back as he or she wants, all the way
to the axioms on which the theorem is based.  This ability suggests a
non-traditional way of learning about pure mathematics.  Used in conjunction
with traditional methods, Metamath could make pure mathematics accessible to
people who are not sufficiently skilled to figure out the implicit detail in
ordinary textbook proofs.  Once you learn the axioms of a theory, you can have
complete confidence that everything you need to understand a proof you are
studying is all there, at your beck and call, allowing you to focus in on any
proof step you don't understand in as much depth as you need, without worrying
about getting stuck on a step you can't figure out.\footnote{On the other
hand, writing proofs in the Metamath language is challenging, requiring
a degree of rigor far in excess of that normally taught to students.  In a
classroom setting, I doubt that writing Metamath proofs would ever replace
traditional homework exercises involving informal proofs, because the time
needed to work out the details would not allow a course to
cover much material.  For students who have trouble grasping the implied rigor
in traditional material, writing a few simple proofs in the Metamath language
might help clarify fuzzy thought processes.  Although somewhat difficult at
first, it eventually becomes fun to do, like solving a puzzle, because of the
instant feedback provided by the computer.}

Metamath\index{Metamath} is probably unlike anything you have
encountered before.  In this first chapter we will look at the philosophy and
use of computers in mathematics in order to better understand the motivation
behind Metamath.  The material in this chapter is not required in order to use
Metamath.  You may skip it if you are impatient, but I hope you will find it
educational and enjoyable.  If you want to start experimenting with the
Metamath program right away, proceed directly to Chapter~\ref{using}
(p.~\pageref{using}).  To
learn the Metamath language, skim Chapter~\ref{using} then proceed to
Chapter~\ref{languagespec} (p.~\pageref{languagespec}).

\section{Mathematics as a Computer Language}

\begin{quote}
  {\em The study of mathematics is apt to commence in
dis\-ap\-point\-ment.\ldots \\
We are told that by its aid the stars are weighted
and the billions of molecules in a drop of water are counted.  Yet, like the
ghost of Hamlet's father, this great science eludes the efforts of our mental
weapons to grasp it.}
  \flushright\sc  Alfred North Whitehead\footnote{\cite{Whitehead}, ch.\ 1.}\\
\end{quote}\index{Whitehead, Alfred North}

\subsection{Is Mathematics ``User-Friendly''?}

Suppose you have no formal training in abstract mathematics.  But popular
books you've read offer tempting glimpses of this world filled with profound
ideas that have stirred the human spirit.  You are not satisfied with the
informal, watered-down descriptions you've read but feel it is important to
grasp the underlying mathematics itself to understand its true meaning. It's
not practical to go back to school to learn it, though; you don't want to
dedicate years of your life to it.  There are many important things in life,
and you have to set priorities for what's important to you.  What would happen
if you tried to pursue it on your own, in your spare time?

After all, you were able to learn a computer programming language such as
Pascal on your own without too much difficulty, even though you had no formal
training in computers.  You don't claim to be an expert in software design,
but you can write a passable program when necessary to suit your needs.  Even
more important, you know that you can look at anyone else's Pascal program, no
matter how complex, and with enough patience figure out exactly how it works,
even though you are not a specialist.  Pascal allows you do anything that a
computer can do, at least in principle.  Thus you know you have the ability,
in principle, to follow anything that a computer program can do:  you just
have to break it down into small enough pieces.

Here's an imaginary scenario of what might happen if you na\-ive\-ly a\-dopted
this same view of abstract mathematics and tried to pick it up on your own, in
a period of time comparable to, say, learning a computer programming
language.

\subsubsection{A Non-Mathematician's Quest for Truth}

\begin{quote}
  {\em \ldots my daughters have been studying (chemistry) for several
se\-mes\-ters, think they have learned differential and integral calculus in
school, and yet even today don't know why $x\cdot y=y\cdot x$ is true.}
  \flushright\sc  Edmund Landau\footnote{\cite{Landau}, p.~vi.}\\
\end{quote}\index{Landau, Edmund}

\begin{quote}
  {\em Minus times minus is plus,\\
The reason for this we need not discuss.}
  \flushright\sc W.\ H.\ Auden\footnote{As quoted in \cite{Guillen}, p.~64.}\\
\end{quote}\index{Auden, W.\ H.}\index{Guillen, Michael}

We'll suppose you are a technically oriented professional, perhaps an engineer, a
computer programmer, or a physicist, but probably not a mathematician.  You
consider yourself reasonably intelligent.  You did well in school, learning a
variety of methods and techniques in practical mathematics such as calculus and
differential equations.  But rarely did your courses get into anything
resembling modern abstract mathematics, and proofs were something that appeared
only occasionally in your textbooks, a kind of necessary evil that was
supposed to convince you of a certain key result.  Most of your
homework consisted of exercises that gave you practice in the techniques, and
you were hardly ever asked to come up with a proof of your own.

You find yourself curious about advanced, abstract mathematics.  You are
driven by an inner conviction that it is important to understand and
appreciate some of the most profound knowledge discovered by mankind.  But it
seems very hard to learn, something that only certain gifted longhairs can
access and understand.  You are frustrated that it seems forever cut off from
you.

Eventually your curiosity drives you to do something about it.
You set for yourself a goal of ``really'' understanding mathematics:  not just
how to manipulate equations in algebra or calculus according to cookbook
rules, but rather to gain a deep understanding of where those rules come from.
In fact, you're not thinking about this kind of ordinary mathematics at all,
but about a much more abstract, ethereal realm of pure mathematics, where
famous results such as G\"{o}del's incompleteness theorem\index{G\"{o}del's
incompleteness theorem} and Cantor's different kinds of infinities
reside.

You have probably read a number of popular books, with titles like {\em
Infinity and the Mind} \cite{Rucker}\index{Rucker, Rudy}, on topics such as
these.  You found them inspiring but at the same time somewhat
unsatisfactory.  They gave you a general idea of what these results are about,
but if someone asked you to prove them, you wouldn't have the faintest idea of
where to begin.   Sure, you could give the same overall outline that you
learned from the popular books; and in a general sort of way, you do have an
understanding.  But deep down inside, you know that there is a rigor that is
missing, that probably there are many subtle steps and pitfalls along the way,
and ultimately it seems you have to place your trust in the experts in the
field.  You don't like this; you want to be able to verify these results for
yourself.

So where do you go next?  As a first step, you decide to look up some of the
original papers on the theorems you are curious about, or better, obtain some
standard textbooks in the field.  You look up a theorem you want to
understand.  Sure enough, it's there, but it's expressed with strange
terms and odd symbols that mean absolutely nothing to you.  It might as well be written in
a foreign language you've never seen before, whose symbols are totally alien.
You look at the proof, and you haven't the foggiest notion what each step
means, much less how one step follows from another.  Well, obviously you have
a lot to learn if you want to understand this stuff.

You feel that you could probably understand it by
going back to college for another three to six years and getting a math
degree.  But that does not fit in with your career and the other things in
your life and would serve no practical purpose.  You decide to seek a quicker
path.  You figure you'll just trace your way back to the beginning, step by
step, as you would do with a computer program, until you understand it.  But
you quickly find that this is not possible, since you can't even understand
enough to know what you have to trace back to.

Maybe a different approach is in order---maybe you should start at the
beginning and work your way up.  First, you read the introduction to the book
to find out what the prerequisites are.  In a similar fashion, you trace your
way back through two or three more books, finally arriving at one that seems
to start at a beginning:  it lists the axioms of arithmetic.  ``Aha!'' you
naively think, ``This must be the starting point, the source of all mathematical
knowledge.'' Or at least the starting point for mathematics dealing with
numbers; you have to start somewhere and have no idea what the starting point
for other mathematics would be.  But the word ``axioms'' looks promising.  So
you eagerly read along and work through some elementary exercises at the
beginning of the book.  You feel vaguely bothered:  these
don't seem like axioms at all, at least not in the sense that you want to
think of axioms.  Axioms imply a starting point from which everything else can
be built up, according to precise rules specified in the axiom system.  Even
though you can understand the first few proofs in an informal way,
and are able to do some of the
exercises, it's hard to pin down precisely what the
rules are.   Sure, each step seems to follow logically from the others, but
exactly what does that mean?  Is the ``logic'' just a matter of common sense,
something vague that we all understand but can never quite state precisely?

You've spent a number of years, off and on, programming computers, and you
know that in the case of computer languages there is no question of what the
rules are---they are precise and crystal clear.  If you follow them, your
program will work, and if you don't, it won't.  No matter how complex a
program, it can always be broken down into simpler and simpler pieces, until
you can ultimately identify the bits that are moved around to perform a
specific function.  Some programs might require a lot of perseverance to
accomplish this, but if you focus on a specific portion of it, you don't even
necessarily have to know how the rest of it works. Shouldn't there be an
analogy in mathematics?

You decide to apply the ultimate test:  you ask yourself how a computer could
verify or ensure that the steps in these proofs follow from one another.
Certainly mathematics must be at least as precisely defined as a computer
language, if not more so; after all, computer science itself is based on it.
If you can get a computer to verify these proofs, then you should also be
able, in principle, to understand them yourself in a very crystal clear,
precise way.

You're in for a surprise:  you can conceive of no way to convert the
proofs, which are in English, to a form that the computer can understand.
The proofs are filled with phrases such as ``assume there exists a unique
$x$\ldots'' and ``given any $y$, let $z$ be the number such that\ldots''  This
isn't the kind of logic you are used to in computer programming, where
everything, even arithmetic, reduces to Boolean ones and zeroes if you care to
break it down sufficiently.  Even though you think you understand the proofs,
there seems to be some kind of higher reasoning involved rather than precise
rules that define how you manipulate the symbols in the axioms.  Whatever it
is, it just isn't obvious how you would express it to a computer, and the more
you think about it, the more puzzled and confused you get, to the point where
you even wonder whether {\em you} really understand it.  There's a lot more to
these axioms of arithmetic than meets the eye.

Nobody ever talked about this in school in your applied math and engineering
courses.  You just learned the rules they gave you, not quite understanding
how or why they worked, sometimes vaguely suspicious or uncertain of them, and
through homework problems and osmosis learned how to present solutions that
satisfied the instructor and earned you an ``A.''  Rarely did you actually
``prove'' anything in a rigorous way, and the math majors who did do stuff
like that seemed to be in a different world.

Of course, there are computer algebra programs that can do mathematics, and
rather impressively.  They can instantly solve the integrals that you
struggled with in freshman calculus, and do much, much more.  But when you
look at these programs, what you see is a big collection of algorithms and
techniques that evolved and were added to over time, along with some basic
software that manipulates symbols.  Each algorithm that is built in is the
result of someone's theorem whose proof is omitted; you just have to trust the
person who proved it and the person who programmed it in and hope there are no
bugs.\index{computer program bugs}  Somehow this doesn't seem to be the
essence of mathematics.  Although computer algebra systems can generate
theorems with amazing speed, they can't actually prove a single one of them.

After some puzzlement, you revisit some popular books on what mathematics is
all about.  Somewhere you read that all of mathematics is actually derived
from something called ``set theory.''  This is a little confusing, because
nowhere in the book that presented the axioms of arithmetic was there any
mention of set theory, or if there was, it seemed to be just a tool that helps
you describe things better---the set of even numbers, that sort of thing.  If
set theory is the basis for all mathematics, then why are additional axioms
needed for arithmetic?

Something is wrong but you're not sure what.  One of your friends is a pure
mathematician.  He knows he is unable to communicate to you what he does for a
living and seems to have little interest in trying.  You do know that for him,
proofs are what mathematics is all about. You ask him what a proof is, and he
essentially tells you that, while of course it's based on logic, really it's
something you learn by doing it over and over until you pick it up.  He refers
you to a book, {\em How to Read and Do Proofs} \cite{Solow}.\index{Solow,
Daniel}  Although this book helps you understand traditional informal proofs,
there is still something missing you can't seem to pin down yet.

You ask your friend how you would go about having a computer verify a proof.
At first he seems puzzled by the question; why would you want to do that?
Then he says it's not something that would make any sense to do, but he's
heard that you'd have to break the proof down into thousands or even millions
of individual steps to do such a thing, because the reasoning involved is at
such a high level of abstraction.  He says that maybe it's something you could
do up to a point, but the computer would be completely impractical once you
get into any meaningful mathematics.  There, the only way you can verify a
proof is by hand, and you can only acquire the ability to do this by
specializing in the field for a couple of years in grad school.  Anyway, he
thinks it all has to do with set theory, although he has never taken a formal
course in set theory but just learned what he needed as he went along.

You are intrigued and amazed.  Apparently a mathematician can grasp as a
single concept something that would take a computer a thousand or a million
steps to verify, and have complete confidence in it.  Each one of these
thousand or million steps must be absolutely correct, or else the whole proof
is meaningless.  If you added a million numbers by hand, would you trust the
result?  How do you really know that all these steps are correct, that there
isn't some subtle pitfall in one of these million steps, like a bug in a
computer program?\index{computer program bugs}  After all, you've read that
famous mathematicians have occasionally made mistakes, and you certainly know
you've made your share on your math homework problems in school.

You recall the analogy with a computer program.  Sure, you can understand what
a large computer program such as a word processor does, as a single high-level
concept or a small set of such concepts, but your ability to understand it in
no way ensures that the program is correct and doesn't have hidden bugs.  Even
if you wrote the program yourself you can't really know this; most large
programs that you've written have had bugs that crop up at some later date, no
matter how careful you tried to be while writing them.

OK, so now it seems the reason you can't figure out how to make a
computer verify proofs is because each step really corresponds to a
million small steps.  Well, you say, a computer can do a million
calculations in a second, so maybe it's still practical to do.  Now the
puzzle becomes how to figure out what the million steps are that each
English-language step corresponds to.  Your mathematician friend hasn't
a clue, but suggests that maybe you would find the answer by studying
set theory.  Actually, your friend thinks you're a little off the wall
for even wondering such a thing.  For him, this is not what mathematics
is all about.

The subject of set theory keeps popping up, so you decide it's
time to look it up.

You decide to start off on a careful footing, so you start reading a couple of
very elementary books on set theory.  A lot of it seems pretty obvious, like
intersections, subsets, and Venn diagrams.  You thumb through one of the
books; nowhere is anything about axioms mentioned. The other book relegates to
an appendix a brief discussion that mentions a set of axioms called
``Zermelo--Fraenkel set theory''\index{Zermelo--Fraenkel set theory} and states
them in English.  You look at them and have no idea what they really mean or
what you can do with them.  The comments in this appendix say that the purpose
of mentioning them is to expose you to the idea, but imply that they are not
necessary for basic understanding and that they are really the subject matter
of advanced treatments where fine points such as a certain paradox (Russell's
paradox\index{Russell's paradox}\footnote{Russell's paradox assumes that there
exists a set $S$ that is a collection of all sets that don't contain
themselves.  Now, either $S$ contains itself or it doesn't.  If it contains
itself, it contradicts its definition.  But if it doesn't contain itself, it
also contradicts its definition.  Russell's paradox is resolved in ZF set
theory by denying that such a set $S$ exists.}) are resolved.  Wait a
minute---shouldn't the axioms be a starting point, not an ending point?  If
there are paradoxes that arise without the axioms, how do you know you won't
stumble across one accidentally when using the informal approach?

And nowhere do these books describe how ``all of mathematics can be
derived from set theory'' which by now you've heard a few times.

You find a more advanced book on set theory.  This one actually lists the
axioms of ZF set theory in plain English on page one.  {\em Now} you think
your quest has ended and you've finally found the source of all mathematical
knowledge; you just have to understand what it means.  Here, in one place, is
the basis for all of mathematics!  You stare at the axioms in awe, puzzle over
them, memorize them, hoping that if you just meditate on them long enough they
will become clear.  Of course, you haven't the slightest idea how the rest of
mathematics is ``derived'' from them; in particular, if these are the axioms
of mathematics, then why do arithmetic, group theory, and so on need their own
axioms?

You start reading this advanced book carefully, pondering the meaning of every
word, because by now you're really determined to get to the bottom of this.
The first thing the book does is explain how the axioms came about, which was
to resolve Russell's paradox.\index{Russell's paradox}  In fact that seems to
be the main purpose of their existence; that they supposedly can be used to
derive all of mathematics seems irrelevant and is not even mentioned.  Well,
you go on.  You hope the book will explain to you clearly, step by step, how
to derive things from the axioms.  After all, this is the starting point of
mathematics, like a book that explains the basics of a computer programming
language.  But something is missing.  You find you can't even understand the
first proof or do the first exercise.  Symbols such as $\exists$ and $\forall$
permeate the page without any mention of where they came from or how to
manipulate them; the author assumes you are totally familiar with them and
doesn't even tell you what they mean.  By now you know that $\exists$ means
``there exists'' and $\forall$ means ``for all,'' but shouldn't the rules for
manipulating these symbols be part of the axioms?  You still have no idea
how you could even describe the axioms to a computer.

Certainly there is something much different here from the technical
literature you're used to reading.  A computer language manual almost
always explains very clearly what all the symbols mean, precisely what
they do, and the rules used for combining them, and you work your way up
from there.

After glancing at four or five other such books, you come to the realization
that there is another whole field of study that you need just to get to the
point at which you can understand the axioms of set theory.  The field is
called ``logic.''  In fact, some of the books did recommend it as a
prerequisite, but it just didn't sink in.  You assumed logic was, well, just
logic, something that a person with common sense intuitively understood.  Why
waste your time reading boring treatises on symbolic logic, the manipulation
of 1's and 0's that computers do, when you already know that?  But this is a
different kind of logic, quite alien to you.  The subject of {\sc nand} and
{\sc nor} gates is not even touched upon or in any case has to do with only a
very small part of this field.

So your quest continues.  Skimming through the first couple of introductory
books, you get a general idea of what logic is about and what quantifiers
(``for all,'' ``there exists'') mean, but you find their examples somewhat
trivial and mildly annoying (``all dogs are animals,'' ``some animals are
dogs,'' and such).  But all you want to know is what the rules are for
manipulating the symbols so you can apply them to set theory.  Some formulas
describing the relationships among quantifiers ($\exists$ and $\forall$) are
listed in tables, along with some verbal reasoning to justify them.
Presumably, if you want to find out if a formula is correct, you go through
this same kind of mental reasoning process, possibly using images of dogs and
animals. Intuitively, the formulas seem to make sense.  But when you ask
yourself, ``What are the rules I need to get a computer to figure out whether
this formula is correct?'', you still don't know.  Certainly you don't ask the
computer to imagine dogs and animals.

You look at some more advanced logic books.  Many of them have an introductory
chapter summarizing set theory, which turns out to be a prerequisite.  You
need logic to understand set theory, but it seems you also need set theory to
understand logic!  These books jump right into proving rather advanced
theorems about logic, without offering the faintest clue about where the logic
came from that allows them to prove these theorems.

Luckily, you come across an elementary book of logic that, halfway through,
after the usual truth tables and metaphors, presents in a clear, precise way
what you've been looking for all along: the axioms!  They're divided into
propositional calculus (also called sentential logic) and predicate calculus
(also called first-order logic),\index{first-order logic} with rules so simple
and crystal clear that now you can finally program a computer to understand
them.  Indeed, they're no harder than learning how to play a game of chess.
As far as what you seem to need is concerned, the whole book could have been
written in five pages!

{\em Now} you think you've found the ultimate source of mathematical
truth.  So---the axioms of mathematics consist of these axioms of logic,
together with the axioms of ZF set theory. (By now you've also been able to
figure out how to translate the ZF axioms from English into the
actual symbols of logic which you can now manipulate according to
precise, easy-to-understand rules.)

Of course, you still don't understand how ``all of mathematics can be
derived from set theory,'' but maybe this will reveal itself in due
course.

You eagerly set out to program the axioms and rules into a computer and start
to look at the theorems you will have to prove as the logic is developed.  All
sorts of important theorems start popping up:  the deduction
theorem,\index{deduction theorem} the substitution theorem,\index{substitution
theorem} the completeness theorem of propositional calculus,\index{first-order
logic!completeness} the completeness theorem of predicate calculus.  Uh-oh,
there seems to be trouble.  They all get harder and harder, and not one of
them can be derived with the axioms and rules of logic you've just been
handed.  Instead, they all require ``metalogic'' for their proofs, a kind of
mixture of logic and set theory that allows you to prove things {\em about}
the axioms and theorems of logic rather than {\em with} them.

You plow ahead anyway.  A month later, you've spent much of your
free time getting the computer to verify proofs in propositional calculus.
You've programmed in the axioms, but you've also had to program in the
deduction theorem, the substitution theorem, and the completeness theorem of
propositional calculus, which by now you've resigned yourself to treating as
rather complex additional axioms, since they can't be proved from the axioms
you were given.  You can now get the computer to verify and even generate
complete, rigorous, formal proofs\index{formal proof}.  Never mind that they
may have 100,000 steps---at least now you can have complete, absolute
confidence in them.  Unfortunately, the only theorems you have proved are
pretty trivial and you can easily verify them in a few minutes with truth
tables, if not by inspection.

It looks like your mathematician friend was right.  Getting the computer to do
serious mathematics with this kind of rigor seems almost hopeless.  Even
worse, it seems that the further along you get, the more ``axioms'' you have
to add, as each new theorem seems to involve additional ``metamathematical''
reasoning that hasn't been formalized, and none of it can be derived from the
axioms of logic.  Not only do the proofs keep growing exponentially as you get
further along, but the program to verify them keeps getting bigger and bigger
as you program in more ``metatheorems.''\index{metatheorem}\footnote{A
metatheorem is usually a statement that is too general to be directly provable
in a theory.  For example, ``if $n_1$, $n_2$, and $n_3$ are integers, then
$n_1+n_2+n_3$ is an integer'' is a theorem of number theory.  But ``for any
integer $k > 1$, if $n_1, \ldots, n_k$ are integers, then $n_1+\ldots +n_k$ is
an integer'' is a metatheorem, in other words a family of theorems, one for
every $k$.  The reason it is not a theorem is that the general sum $n_1+\ldots
+n_k$ (as a function of $k$) is not an operation that can be defined directly
in number theory.} The bugs\index{computer program bugs} that have cropped up
so far have already made you start to lose faith in the rigor you seem to have
achieved, and you know it's just going to get worse as your program gets larger.

\subsection{Mathematics and the Non-Specialist}

\begin{quote}
  {\em A real proof is not checkable by a machine, or even by any mathematician
not privy to the gestalt, the mode of thought of the particular field of
mathematics in which the proof is located.}
  \flushright\sc  Davis and Hersh\index{Davis, Phillip J.}
  \footnote{\cite{Davis}, p.~354.}\\
\end{quote}

The bulk of abstract or theoretical mathematics is ordinarily outside
the reach of anyone but a few specialists in each field who have completed
the necessary difficult internship in order to enter its coterie.  The
typical intelligent layperson has no reasonable hope of understanding much of
it, nor even the specialist mathematician of understanding other fields.  It
is like a foreign language that has no dictionary to look up the translation;
the only way you can learn it is by living in the country for a few years.  It
is argued that the effort involved in learning a specialty is a necessary
process for acquiring a deep understanding.  Of course, this is almost certainly
true if one is to make significant contributions to a field; in particular,
``doing'' proofs is probably the most important part of a mathematician's
training.  But is it also necessary to deny outsiders access to it?  Is it
necessary that abstract mathematics be so hard for a layperson to grasp?

A computer normally is of no help whatsoever.  Most published proofs are
actually just series of hints written in an informal style that requires
considerable knowledge of the field to understand.  These are the ``real
proofs'' referred to by Davis and Hersh.\index{informal proof}  There is an
implicit understanding that, in principle, such a proof could be converted to
a complete formal proof\index{formal proof}.  However, it is said that no one
would ever attempt such a conversion, even if they could, because that would
presumably require millions of steps (Section~\ref{dream}).  Unfortunately the
informal style automatically excludes the understanding of the proof
by anyone who hasn't gone through the necessary apprenticeship. The
best that the intelligent layperson can do is to read popular books about deep
and famous results; while this can be helpful, it can also be misleading, and
the lack of detail usually leaves the reader with no ability whatsoever to
explore any aspect of the field being described.

The statements of theorems often use sophisticated notation that makes them
inaccessible to the non-specialist.  For a non-specialist who wants to achieve
a deeper understanding of a proof, the process of tracing definitions and
lemmas back through their hierarchy\index{hierarchy} quickly becomes confusing
and discouraging.  Textbooks are usually written to train mathematicians or to
communicate to people who are already mathematicians, and large gaps in proofs
are often left as exercises to the reader who is left at an impasse if he or
she becomes stuck.

I believe that eventually computers will enable non-specialists and even
intelligent laypersons to follow almost any mathematical proof in any field.
Metamath is an attempt in that direction.  If all of mathematics were as
easily accessible as a computer programming language, I could envision
computer programmers and hobbyists who otherwise lack mathematical
sophistication exploring and being amazed by the world of theorems and proofs
in obscure specialties, perhaps even coming up with results of their own.  A
tremendous advantage would be that anyone could experiment with conjectures in
any field---the computer would offer instant feedback as to whether
an inference step was correct.

Mathematicians sometimes have to put up with the annoyance of
cranks\index{cranks} who lack a fundamental understanding of mathematics but
insist that their ``proofs'' of, say, Fermat's Last Theorem\index{Fermat's
Last Theorem} be taken seriously.  I think part of the problem is that these
people are misled by informal mathematical language, treating it as if they
were reading ordinary expository English and failing to appreciate the
implicit underlying rigor.  Such cranks are rare in the field of computers,
because computer languages are much more explicit, and ultimately the proof is
in whether a computer program works or not.  With easily accessible
computer-based abstract mathematics, a mathematician could say to a crank,
``don't bother me until you've demonstrated your claim on the computer!''

% 22-May-04 nm
% Attempt to move De Millo quote so it doesn't separate from attribution
% CHANGE THIS NUMBER (AND ELIMINATE IF POSSIBLE) WHEN ABOVE TEXT CHANGES
\vspace{-0.5em}

\subsection{An Impossible Dream?}\label{dream}

\begin{quote}
  {\em Even quite basic theorems would demand almost unbelievably vast
  books to display their proofs.}
    \flushright\sc  Robert E. Edwards\footnote{\cite{Edwards}, p.~68.}\\
\end{quote}\index{Edwards, Robert E.}

\begin{quote}
  {\em Oh, of course no one ever really {\em does} it.  It would take
  forever!  You just show that you could do it, that's sufficient.}
    \flushright\sc  ``The Ideal Mathematician''
    \index{Davis, Phillip J.}\footnote{\cite{Davis},
p.~40.}\\
\end{quote}

\begin{quote}
  {\em There is a theorem in the primitive notation of set theory that
  corresponds to the arithmetic theorem `$1000+2000=3000$'.  The formula
  would be forbiddingly long\ldots even if [one] knows the definitions
  and is asked to simplify the long formula according to them, chances are
  he will make errors and arrive at some incorrect result.}
    \flushright\sc  Hao Wang\footnote{\cite{Wang}, p.~140.}\\
\end{quote}\index{Wang, Hao}

% 22-May-04 nm
% Attempt to move De Millo quote so it doesn't separate from attribution
% CHANGE THIS NUMBER (AND ELIMINATE IF POSSIBLE) WHEN ABOVE TEXT CHANGES
\vspace{-0.5em}

\begin{quote}
  {\em The {\em Principia Mathematica} was the crowning achievement of the
  formalists.  It was also the deathblow of the formalist view.\ldots
  {[Rus\-sell]} failed, in three enormous volumes, to get beyond the elementary
  facts of arithmetic.  He showed what can be done in principle and what
  cannot be done in practice.  If the mathematical process were really
  one of strict, logical progression, we would still be counting our
  fingers.\ldots
  One theoretician estimates\ldots that a demonstration of one of
  Ramanujan's conjectures assuming set theory and elementary analysis would
  take about two thousand pages; the length of a deduction from first principles
  is nearly in\-con\-ceiv\-a\-ble\ldots The probabilists argue that\ldots any
  very long proof can at best be viewed as only probably correct\ldots}
  \flushright\sc Richard de Millo et. al.\footnote{\cite{deMillo}, pp.~269,
  271.}\\
\end{quote}\index{de Millo, Richard}

A number of writers have conveyed the impression that the kind of absolute
rigor provided by Metamath\index{Metamath} is an impossible dream, suggesting
that a complete, formal verification\index{formal proof} of a typical theorem
would take millions of steps in untold volumes of books.  Even if it could be
done, the thinking sometimes goes, all meaning would be lost in such a
monstrous, tedious verification.\index{informal proof}\index{proof length}

These writers assume, however, that in order to achieve the kind of complete
formal verification they desire one must break down a proof into individual
primitive steps that make direct reference to the axioms.  This is
not necessary.  There is no reason not to make use of previously proved
theorems rather than proving them over and over.

Just as important, definitions\index{definition} can be introduced along
the way, allowing very complex formulas to be represented with few
symbols.  Not doing this can lead to absurdly long formulas.  For
example, the mere statement of
G\"{o}del's incompleteness theorem\index{G\"{o}del's
incompleteness theorem}, which can be expressed with a small number of
defined symbols, would require about 20,000 primitive symbols to express
it.\index{Boolos, George S.}\footnote{George S.\ Boolos, lecture at
Massachusetts Institute of Technology, spring 1990.} An extreme example
is Bourbaki's\label{bourbaki} language for set theory, which requires
4,523,659,424,929 symbols plus 1,179,618,517,981 disambiguatory links
(lines connecting symbol pairs, usually drawn below or above the
formula) to express the number
``one'' \cite{Mathias}.\index{Mathias, Adrian R. D.}\index{Bourbaki,
Nicolas}
% http://www.dpmms.cam.ac.uk/~ardm/

A hierarchy\index{hierarchy} of theorems and definitions permits an
exponential growth in the formula sizes and primitive proof steps to be
described with only a linear growth in the number of symbols used.  Of course,
this is how ordinary informal mathematics is normally done anyway, but with
Metamath\index{Metamath} it can be done with absolute rigor and precision.

\subsection{Beauty}


\begin{quote}
  {\em No one shall be able to drive us from the paradise that Cantor has
created for us.}
   \flushright\sc  David Hilbert\footnote{As quoted in \cite{Moore}, p.~131.}\\
\end{quote}\index{Hilbert, David}

\needspace{3\baselineskip}
\begin{quote}
  {\em Mathematics possesses not only truth, but some supreme beauty ---a
  beauty cold and austere, like that of a sculpture.}
    \flushright\sc  Bertrand
    Russell\footnote{\cite{Russell}.}\\
\end{quote}\index{Russell, Bertrand}

\begin{quote}
  {\em Euclid alone has looked on Beauty bare.}
  \flushright\sc Edna Millay\footnote{As quoted in \cite{Davis}, p.~150.}\\
\end{quote}\index{Millay, Edna}

For most people, abstract mathematics is distant, strange, and
incomprehensible.  Many popular books have tried to convey some of the sense
of beauty in famous theorems.  But even an intelligent layperson is left with
only a general idea of what a theorem is about and is hardly given the tools
needed to make use of it.  Traditionally, it is only after years of arduous
study that one can grasp the concepts needed for deep understanding.
Metamath\index{Metamath} allows you to approach the proof of the theorem from
a quite different perspective, peeling apart the formulas and definitions
layer by layer until an entirely different kind of understanding is achieved.
Every step of the proof is there, pieced together with absolute precision and
instantly available for inspection through a microscope with a magnification
as powerful as you desire.

A proof in itself can be considered an object of beauty.  Constructing an
elegant proof is an art.  Once a famous theorem has been proved, often
considerable effort is made to find simpler and more easily understood
proofs.  Creating and communicating elegant proofs is a major concern of
mathematicians.  Metamath is one way of providing a common language for
archiving and preserving this information.

The length of a proof can, to a certain extent, be considered an
objective measure of its ``beauty,'' since shorter proofs are usually
considered more elegant.  In the set theory database
\texttt{set.mm}\index{set theory database (\texttt{set.mm})}%
\index{Metamath Proof Explorer}
provided with Metamath, one goal was to make all proofs as short as possible.

\needspace{4\baselineskip}
\subsection{Simplicity}

\begin{quote}
  {\em God made man simple; man's complex problems are of his own
  devising.}
    \flushright\sc Eccles. 7:29\footnote{Jerusalem Bible.}\\
\end{quote}\index{Bible}

\needspace{3\baselineskip}
\begin{quote}
  {\em God made integers, all else is the work of man.}
    \flushright\sc Leopold Kronecker\footnote{{\em Jahresbericht
	der Deutschen Mathematiker-Vereinigung }, vol. 2, p. 19.}\\
\end{quote}\index{Kronecker, Leopold}

\needspace{3\baselineskip}
\begin{quote}
  {\em For what is clear and easily comprehended attracts; the
  complicated repels.}
    \flushright\sc David Hilbert\footnote{As quoted in \cite{deMillo},
p.~273.}\\
\end{quote}\index{Hilbert, David}

The Metamath\index{Metamath} language is simple and Spartan.  Metamath treats
all mathematical expressions as simple sequences of symbols, devoid of meaning.
The higher-level or ``metamathematical'' notions underlying Metamath are about
as simple as they could possibly be.  Each individual step in a proof involves
a single basic concept, the substitution of an expression for a variable, so
that in principle almost anyone, whether mathematician or not, can
completely understand how it was arrived at.

In one of its most basic applications, Metamath\index{Metamath} can be used to
develop the foundations of mathematics\index{foundations of mathematics} from
the very beginning.  This is done in the set theory database that is provided
with the Metamath package and is the subject matter
of Chapter~\ref{fol}. Any language (a metalanguage\index{metalanguage})
used to describe mathematics (an object language\index{object language}) must
have a mathematical content of its own, but it is desirable to keep this
content down to a bare minimum, namely that needed to make use of the
inference rules specified by the axioms.  With any metalanguage there is a
``chicken and egg'' problem somewhat like circular reasoning:  you must assume
the validity of the mathematics of the metalanguage in order to prove the
validity of the mathematics of the object language.  The mathematical content
of Metamath itself is quite limited.  Like the rules of a game of chess, the
essential concepts are simple enough so that virtually anyone should be able to
understand them (although that in itself will not let you play like
a master).  The symbols that Metamath manipulates do not in themselves
have any intrinsic meaning.  Your interpretation of the axioms that you supply
to Metamath is what gives them meaning.  Metamath is an attempt to strip down
mathematical thought to its bare essence and show you exactly how the symbols
are manipulated.

Philosophers and logicians, with various motivations, have often thought it
important to study ``weak'' fragments of logic\index{weak logic}
\cite{Anderson}\index{Anderson, Alan Ross} \cite{MegillBunder}\index{Megill,
Norman}\index{Bunder, Martin}, other unconventional systems of logic (such as
``modal'' logic\index{modal logic} \cite[ch.\ 27]{Boolos}\index{Boolos, George
S.}), and quantum logic\index{quantum logic} in physics
\cite{Pavicic}\index{Pavi{\v{c}}i{\'{c}}, M.}.  Metamath\index{Metamath}
provides a framework in which such systems can be expressed, with an absolute
precision that makes all underlying metamathematical assumptions rigorous and
crystal clear.

Some schools of philosophical thought, for example
intuitionism\index{intuitionism} and constructivism\index{constructivism},
demand that the notions underlying any mathematical system be as simple and
concrete as possible.  Metamath should meet the requirements of these
philosophies.  Metamath must be taught the symbols, axioms\index{axiom}, and
rules\index{rule} for a specific theory, from the skeptical (such as
intuitionism\index{intuitionism}\footnote{Intuitionism does not accept the law
of excluded middle (``either something is true or it is not true'').  See
\cite[p.~xi]{Tymoczko}\index{Tymoczko, Thomas} for discussion and references
on this topic.  Consider the theorem, ``There exist irrational numbers $a$ and
$b$ such that $a^b$ is rational.''  An intuitionist would reject the following
proof:  If $\sqrt{2}^{\sqrt{2}}$ is rational, we are done.  Otherwise, let
$a=\sqrt{2}^{\sqrt{2}}$ and $b=\sqrt{2}$. Then $a^b=2$, which is rational.})
to the bold (such as the axiom of choice in set theory\footnote{The axiom of
choice\index{Axiom of Choice} asserts that given any collection of pairwise
disjoint nonempty sets, there exists a set that has exactly one element in
common with each set of the collection.  It is used to prove many important
theorems in standard mathematics.  Some philosophers object to it because it
asserts the existence of a set without specifying what the set contains
\cite[p.~154]{Enderton}\index{Enderton, Herbert B.}.  In one foundation for
mathematics due to Quine\index{Quine, Willard Van Orman}, that has not been
otherwise shown to be inconsistent, the axiom of choice turns out to be false
\cite[p.~23]{Curry}\index{Curry, Haskell B.}.  The \texttt{show
trace{\char`\_}back} command of the Metamath program allows you to find out
whether the axiom of choice, or any other axiom, was assumed by a
proof.}\index{\texttt{show trace{\char`\_}back} command}).

The simplicity of the Metamath language lets the algorithm (computer program)
that verifies the validity of a Metamath proof be straightforward and
robust.  You can have confidence that the theorems it verifies really can be
derived from your axioms.

\subsection{Rigor}

\begin{quote}
  {\em Rigor became a goal with the Greeks\ldots But the efforts to
  pursue rigor to the utmost have led to an impasse in which there is
  no longer any agreement on what it really means.  Mathematics remains
  alive and vital, but only on a pragmatic basis.}
    \flushright\sc  Morris Kline\footnote{\cite{Kline}, p.~1209.}\\
\end{quote}\index{Kline, Morris}

Kline refers to a much deeper kind of rigor than that which we will discuss in
this section.  G\"{o}del's incompleteness theorem\index{G\"{o}del's
incompleteness theorem} showed that it is impossible to achieve absolute rigor
in standard mathematics because we can never prove that mathematics is
consistent (free from contradictions).\index{consistent theory}  If
mathematics is consistent, we will never know it, but must rely on faith.  If
mathematics is inconsistent, the best we can hope for is that some clever
future mathematician will discover the inconsistency.  In this case, the
axioms would probably be revised slightly to eliminate the inconsistency, as
was done in the case of Russell's paradox,\index{Russell's paradox} but the
bulk of mathematics would probably not be affected by such a discovery.
Russell's paradox, for example, did not affect most of the remarkable results
achieved by 19th-century and earlier mathematicians.  It mainly invalidated
some of Gottlob Frege's\index{Frege, Gottlob} work on the foundations of
mathematics in the late 1800's; in fact Frege's work inspired Russell's
discovery.  Despite the paradox, Frege's work contains important concepts that
have significantly influenced modern logic.  Kline's {\em Mathematics, The
Loss of Certainty} \cite{Klinel}\index{Kline, Morris} has an interesting
discussion of this topic.

What {\em can} be achieved with absolute certainty\index{certainty} is the
knowledge that if we assume the axioms are consistent and true, then the
results derived from them are true.  Part of the beauty of mathematics is that
it is the one area of human endeavor where absolute certainty can be achieved
in this sense.  A mathematical truth will remain such for eternity.  However,
our actual knowledge of whether a particular statement is a mathematical truth
is only as certain as the correctness of the proof that establishes it.  If
the proof of a statement is questionable or vague, we can't have absolute
confidence in the truth that the statement claims.

Let us look at some traditional ways of expressing proofs.

Except in the field of formal logic\index{formal logic}, almost all
traditional proofs in mathematics are really not proofs at all, but rather
proof outlines or hints as to how to go about constructing the proof.  Many
gaps\index{gaps in proofs} are left for the reader to fill in. There are
several reasons for this.  First, it is usually assumed in mathematical
literature that the person reading the proof is a mathematician familiar with
the specialty being described, and that the missing steps are obvious to such
a reader or at least that the reader is capable of filling them in.  This
attitude is fine for professional mathematicians in the specialty, but
unfortunately it often has the drawback of cutting off the rest of the world,
including mathematicians in other specialties, from understanding the proof.
We discussed one possible resolution to this on p.~\pageref{envision}.
Second, it is often assumed that a complete formal proof\index{formal proof}
would require countless millions of symbols (Section~\ref{dream}). This might
be true if the proof were to be expressed directly in terms of the axioms of
logic and set theory,\index{set theory} but it is usually not true if we allow
ourselves a hierarchy\index{hierarchy} of definitions and theorems to build
upon, using a notation that allows us to introduce new symbols, definitions,
and theorems in a precisely specified way.

Even in formal logic,\index{formal logic} formal proofs\index{formal proof}
that are considered complete still contain hidden or implicit information.
For example, a ``proof'' is usually defined as a sequence of
wffs,\index{well-formed formula (wff)}\footnote{A {\em wff} or well-formed
formula is a mathematical expression (string of symbols) constructed according
to some precise rules.  A formal mathematical system\index{formal system}
contains (1) the rules for constructing syntactically correct
wffs,\index{syntax rules} (2) a list of starting wffs called
axioms,\index{axiom} and (3) one or more rules prescribing how to derive new
wffs, called theorems\index{theorem}, from the axioms or previously derived
theorems.  An example of such a system is contained in
Metamath's\index{Metamath} set theory database, which defines a formal
system\index{formal system} from which all of standard mathematics can be
derived.  Section~\ref{startf} steps you through a complete example of a formal
system, and you may want to skim it now if you are unfamiliar with the
concept.} each of which is an axiom or follows from a rule applied to previous
wffs in the sequence.  The implicit part of the proof is the algorithm by
which a sequence of symbols is verified to be a valid wff, given the
definition of a wff.  The algorithm in this case is rather simple, but for a
computer to verify the proof,\index{automated proof verification} it must have
the algorithm built into its verification program.\footnote{It is possible, of
course, to specify wff construction syntax outside of the program itself
with a suitable input language (the Metamath language being an example), but
some proof-verification or theorem-proving programs lack the ability to extend
wff syntax in such a fashion.} If one deals exclusively with axioms and
elementary wffs, it is straightforward to implement such an algorithm.  But as
more and more definitions are added to the theory in order to make the
expression of wffs more compact, the algorithm becomes more and more
complicated.  A computer program that implements the algorithm becomes larger
and harder to understand as each definition is introduced, and thus more prone
to bugs.\index{computer program bugs}  The larger the program, the
more suspicious the mathematician may be about
the validity of its algorithms.  This is especially true because
computer programs are inherently hard to follow to begin with, and few people
enjoy verifying them manually in detail.

Metamath\index{Metamath} takes a different approach.  Metamath's ``knowledge''
is limited to the ability to substitute variables for expressions, subject to
some simple constraints.  Once the basic algorithm of Metamath is assumed to
be debugged, and perhaps independently confirmed, it
can be trusted once and for all.  The information that Metamath needs to
``understand'' mathematics is contained entirely in the body of knowledge
presented to Metamath.  Any errors in reasoning can only be errors in the
axioms or definitions contained in this body of knowledge.  As a
``constructive'' language\index{constructive language} Metamath has no
conditional branches or loops like the ones that make computer programs hard
to decipher; instead, the language can only build new sequences of symbols
from earlier sequences  of symbols.

The simplicity of the rules that underlie Metamath not only makes Metamath
easy to learn but also gives Metamath a great deal of flexibility. For
example, Metamath is not limited to describing standard first-order
logic\index{first-order logic}; higher-order logics\index{higher-order logic}
and fragments of logic\index{weak logic} can be described just as easily.
Metamath gives you the freedom to define whatever wff notation you prefer; it
has no built-in conception of the syntax of a wff.\index{well-formed formula
(wff)}  With suitable axioms and definitions, Metamath can even describe and
prove things about itself.\index{Metamath!self-description}  (John
Harrison\index{Harrison, John} discusses the ``reflection''
principle\index{reflection principle} involved in self-descriptive systems in
\cite{Harrison}.)

The flexibility of Metamath requires that its proofs specify a lot of detail,
much more than in an ordinary ``formal'' proof.\index{formal proof}  For
example, in an ordinary formal proof, a single step consists of displaying the
wff that constitutes that step.  In order for a computer program to verify
that the step is acceptable, it first must verify that the symbol sequence
being displayed is an acceptable wff.\index{automated proof verification} Most
proof verifiers have at least basic wff syntax built into their programs.
Metamath has no hard-wired knowledge of what constitutes a wff built into it;
instead every wff must be explicitly constructed based on rules defining wffs
that are present in a database.  Thus a single step in an ordinary formal
proof may be correspond to many steps in a Metamath proof. Despite the larger
number of steps, though, this does not mean that a Metamath proof must be
significantly larger than an ordinary formal proof. The reason is that since
we have constructed the wff from scratch, we know what the wff is, so there is
no reason to display it.  We only need to refer to a sequence of statements
that construct it.  In a sense, the display of the wff in an ordinary formal
proof is an implicit proof of its own validity as a wff; Metamath just makes
the proof explicit. (Section~\ref{proof} describes Metamath's proof notation.)

\section{Computers and Mathematicians}

\begin{quote}
  {\em The computer is important, but not to mathematics.}
    \flushright\sc  Paul Halmos\footnote{As quoted in \cite{Albers}, p.~121.}\\
\end{quote}\index{Halmos, Paul}

Pure mathematicians have traditionally been indifferent to computers, even to
the point of disdain.\index{computers and pure mathematics}  Computer science
itself is sometimes considered to fall in the mundane realm of ``applied''
mathematics, perhaps essential for the real world but intellectually unexciting
to those who seek the deepest truths in mathematics.  Perhaps a reason for this
attitude towards computers is that there is little or no computer software that
meets their needs, and there may be a general feeling that such software could
not even exist.  On the one hand, there are the practical computer algebra
systems, which can perform amazing symbolic manipulations in algebra and
calculus,\index{computer algebra system} yet can't prove the simplest
existence theorem, if the idea of a proof is present at all.  On the other
hand, there are specialized automated theorem provers that technically speaking
may generate correct proofs.\index{automated theorem proving}  But sometimes
their specialized input notation may be cryptic and their output perceived to
be long, inelegant, incomprehensible proofs.    The output
may be viewed with suspicion, since the program that generates it tends to be
very large, and its size increases the potential for bugs\index{computer
program bugs}.  Such a proof may be considered trustworthy only if
independently verified and ``understood'' by a human, but no one wants to
waste their time on such a boring, unrewarding chore.



\needspace{4\baselineskip}
\subsection{Trusting the Computer}

\begin{quote}
  {\em \ldots I continue to find the quasi-empirical interpretation of
  computer proofs to be the more plausible.\ldots Since not
  everything that claims to be a computer proof can be
  accepted as valid, what are the mathematical criteria for acceptable
  computer proofs?}
    \flushright\sc  Thomas Tymoczko\footnote{\cite{Tymoczko}, p.~245.}\\
\end{quote}\index{Tymoczko, Thomas}

In some cases, computers have been essential tools for proving famous
theorems.  But if a proof is so long and obscure that it can be verified in a
practical way only with a computer, it is vaguely felt to be suspicious.  For
example, proving the famous four-color theorem\index{four-color
theorem}\index{proof length} (``a map needs no more than four colors to
prevent any two adjacent countries from having the same color'') can presently
only be done with the aid of a very complex computer program which originally
required 1200 hours of computer time. There has been considerable debate about
whether such a proof can be trusted and whether such a proof is ``real''
mathematics \cite{Swart}\index{Swart, E. R.}.\index{trusting computers}

However, under normal circumstances even a skeptical mathematician would have a
great deal of confidence in the result of multiplying two numbers on a pocket
calculator, even though the precise details of what goes on are hidden from its
user.  Even the verification on a supercomputer that a huge number is prime is
trusted, especially if there is independent verification; no one bothers to
debate the philosophical significance of its ``proof,'' even though the actual
proof would be so large that it would be completely impractical to ever write
it down on paper.  It seems that if the algorithm used by the computer is
simple enough to be readily understood, then the computer can be trusted.

Metamath\index{Metamath} adopts this philosophy.  The simplicity of its
language makes it easy to learn, and because of its simplicity one can have
essentially absolute confidence that a proof is correct. All axioms, rules, and
definitions are available for inspection at any time because they are defined
by the user; there are no hidden or built-in rules that may be prone to subtle
bugs\index{computer program bugs}.  The basic algorithm at the heart of
Metamath is simple and fixed, and it can be assumed to be bug-free and robust
with a degree of confidence approaching certainty.
Independently written implementations of the Metamath verifier
can reduce any residual doubt on the part of a skeptic even further;
there are now over a dozen such implementations, written by many people.

\subsection{Trusting the Mathematician}\label{trust}

\begin{quote}
  {\em There is no Algebraist nor Mathematician so expert in his science, as
  to place entire confidence in any truth immediately upon his discovery of it,
  or regard it as any thing, but a mere probability.  Every time he runs over
  his proofs, his confidence encreases; but still more by the approbation of
  his friends; and is rais'd to its utmost perfection by the universal assent
  and applauses of the learned world.}
  \flushright\sc David Hume\footnote{{\em A Treatise of Human Nature}, as
  quoted in \cite{deMillo}, p.~267.}\\
\end{quote}\index{Hume, David}

\begin{quote}
  {\em Stanislaw Ulam estimates that mathematicians publish 200,000 theorems
  every year.  A number of these are subsequently contradicted or otherwise
  disallowed, others are thrown into doubt, and most are ignored.}
  \flushright\sc Richard de Millo et. al.\footnote{\cite{deMillo}, p.~269.}\\
\end{quote}\index{Ulam, Stanislaw}

Whether or not the computer can be trusted, humans  of course will occasionally
err. Only the most memorable proofs get independently verified, and of these
only a handful of truly great ones achieve the status of being ``known''
mathematical truths that are used without giving a second thought to their
correctness.

There are many famous examples of incorrect theorems and proofs in
mathematical literature.\index{errors in proofs}

\begin{itemize}
\item There have been thousands of purported proofs of Fermat's Last
Theorem\index{Fermat's Last Theorem} (``no integer solutions exist to $x^n +
y^n = z^n$ for $n > 2$''), by amateurs, cranks, and well-regarded
mathematicians \cite[p.~5]{Stark}\index{Stark, Harold M}.  Fermat wrote a note
in his copy of Bachet's {\em Diophantus} that he found ``a truly marvelous
proof of this theorem but this margin is too narrow to contain it''
\cite[p.~507]{Kramer}.  A recent, much publicized proof by Yoichi
Miyaoka\index{Miyaoka, Yoichi} was shown to be incorrect ({\em Science News},
April 9, 1988, p.~230).  The theorem was finally proved by Andrew
Wiles\index{Wiles, Andrew} ({\em Science News}, July 3, 1993, p.~5), but it
initially had some gaps and took over a year after its announcement to be
checked thoroughly by experts.  On Oct. 25, 1994, Wiles announced that the last
gap found in his proof had been filled in.
  \item In 1882, M. Pasch discovered that an axiom was omitted from Euclid's
formulation of geometry\index{Euclidean geometry}; without it, the proofs of
important theorems of Euclid are not valid.  Pasch's axiom\index{Pasch's
axiom} states that a line that intersects one side of a triangle must also
intersect another side, provided that it does not touch any of the triangle's
vertices.  The omission of Pasch's axiom went unnoticed for 2000
years \cite[p.~160]{Davis}, in spite of (one presumes) the thousands of
students, instructors, and mathematicians who studied Euclid.
  \item The first published proof of the famous Schr\"{o}der--Bernstein
theorem\index{Schr\"{o}der--Bernstein theorem} in set theory was incorrect
\cite[p.~148]{Enderton}\index{Enderton, Herbert B.}.  This theorem states
that if there exists a 1-to-1 function\footnote{A {\em set}\index{set} is any
collection of objects. A {\em function}\index{function} or {\em
mapping}\index{mapping} is a rule that assigns to each element of one set
(called the function's {\em domain}\index{domain}) an element from another
set.} from set $A$ into set $B$ and vice-versa, then sets $A$ and $B$ have
a 1-to-1 correspondence.  Although it sounds simple and obvious,
the standard proof is quite long and complex.
  \item In the early 1900's, Hilbert\index{Hilbert, David} published a
purported proof of the continuum hypothesis\index{continuum hypothesis}, which
was eventually established as unprovable by Cohen\index{Cohen, Paul} in 1963
\cite[p.~166]{Enderton}.  The continuum hypothesis states that no
infinity\index{infinity} (``transfinite cardinal number'')\index{cardinal,
transfinite} exists whose size (or ``cardinality''\index{cardinality}) is
between the size of the set of integers and the size of the set of real
numbers.  This hypothesis originated with German mathematician Georg
Cantor\index{Cantor, Georg} in the late 1800's, and his inability to prove it
is said to have contributed to mental illness that afflicted him in his later
years.
  \item An incorrect proof of the four-color theorem\index{four-color theorem}
was published by Kempe\index{Kempe, A. B.} in 1879
\cite[p.~582]{Courant}\index{Courant, Richard}; it stood for 11 years before
its flaw was discovered.  This theorem states that any map can be colored
using only four colors, so that no two adjacent countries have the same
color.  In 1976 the theorem was finally proved by the famous computer-assisted
proof of Haken, Appel, and Koch \cite{Swart}\index{Appel, K.}\index{Haken,
W.}\index{Koch, K.}.  Or at least it seems that way.  Mathematician
H.~S.~M.~Coxeter\index{Coxeter, H. S. M.} has doubts \cite[p.~58]{Davis}:  ``I
have a feeling that that is an untidy kind of use of the computers, and the more
you correspond with Haken and Appel, the more shaky you seem to be.''
  \item Many false ``proofs'' of the Poincar\'{e}
conjecture\index{Poincar\'{e} conjecture} have been proposed over the years.
This conjecture states that any object that mathematically behaves like a
three-dimensional sphere is a three-dimensional sphere topologically,
regardless of how it is distorted.  In March 1986, mathematicians Colin
Rourke\index{Rourke, Colin} and Eduardo R\^{e}go\index{R\^{e}go, Eduardo}
caused  a stir in the mathematical community by announcing that they had found
a proof; in November of that year the proof was found to be false \cite[p.
218]{PetersonI}.  It was finally proved in 2003 by Grigory Perelman
\label{poincare}\index{Szpiro, George}\index{Perelman, Grigory}\cite{Szpiro}.
 \end{itemize}

Many counterexamples to ``theorems'' in recent mathematical
literature related to Clifford algebras\index{Clifford algebras}
 have been found by Pertti
Lounesto (who passed away in 2002).\index{Lounesto, Pertti}
See the web page \url{http://mathforum.org/library/view/4933.html}.
% http://users.tkk.fi/~ppuska/mirror/Lounesto/counterexamples.htm

One of the purposes of Metamath\index{Metamath} is to allow proofs to be
expressed with absolute precision.  Developing a proof in the Metamath
language can be challenging, because Metamath will not permit even the
tiniest mistake.\index{errors in proofs}  But once the proof is created, its
correctness can be trusted immediately, without having to depend on the
process of peer review for confirmation.

\section{The Use of Computers in Mathematics}

\subsection{Computer Algebra Systems}

For the most part, you will find that Metamath\index{Metamath} is not a
practical tool for manipulating numbers.  (Even proving that $2 + 2 = 4$, if
you start with set theory, can be quite complex!)  Several commercial
mathematics packages are quite good at arithmetic, algebra, and calculus, and
as practical tools they are invaluable.\index{computer algebra system} But
they have no notion of proof, and cannot understand statements starting with
``there exists such and such...''.

Software packages such as Mathematica \cite{Wolfram}\index{Mathematica} do not
concern themselves with proofs but instead work directly with known results.
These packages primarily emphasize heuristic rules such as the substitution of
equals for equals to achieve simpler expressions or expressions in a different
form.  Starting with a rich collection of built-in rules and algorithms, users
can add to the collection by means of a powerful programming language.
However, results such as, say, the existence of a certain abstract object
without displaying the actual object cannot be expressed (directly) in their
languages.  The idea of a proof from a small set of axioms is absent.  Instead
this software simply assumes that each fact or rule you add to the built-in
collection of algorithms is valid.  One way to view the software is as a large
collection of axioms from which the software, with certain goals, attempts to
derive new theorems, for example equating a complex expression with a simpler
equivalent. But the terms ``theorem''\index{theorem} and
``proof,''\index{proof} for example, are not even mentioned in the index of
the user's manual for Mathematica.\index{Mathematica and proofs}  What is also
unsatisfactory from a philosophical point of view is that there is no way to
ensure the validity of the results other than by trusting the writer of each
application module or tediously checking each module by hand, similar to
checking a computer program for bugs.\index{computer program
bugs}\footnote{Two examples illustrate why the knowledge database of computer
algebra systems should sometimes be regarded with a certain caution.  If you
ask Mathematica (version 3.0) to \texttt{Solve[x\^{ }n + y\^{ }n == z\^{ }n , n]}
it will respond with \texttt{\{\{n-\char`\>-2\}, \{n-\char`\>-1\},
\{n-\char`\>1\}, \{n-\char`\>2\}\}}. In other words, Mathematica seems to
``know'' that Fermat's Last Theorem\index{Fermat's Last Theorem} is true!  (At
the time this version of Mathematica was released this fact was unknown.)  If
you ask Maple\index{Maple} to \texttt{solve(x\^{ }2 = 2\^{ }x)} then
\texttt{simplify(\{"\})}, it returns the solution set \texttt{\{2, 4\}}, apparently
unaware that 0.7666647\ldots is also a solution.} While of course extremely
valuable in applied mathematics, computer algebra systems tend to be of little
interest to the theoretical mathematician except as aids for exploring certain
specific problems.

Because of possible bugs, trusting the output of a computer algebra system for
use as theorems in a proof-verifier would defeat the latter's goal of rigor.
On the other hand, a fact such that a certain relatively large number is
prime, while easy for a computer algebra system to derive, might have a long,
tedious proof that could overwhelm a proof-verifier. One approach for linking
computer algebra systems to a proof-verifier while retaining the advantages of
both is to add a hypothesis to each such theorem indicating its source.  For
example, a constant {\sc maple} could indicate the theorem came from the Maple
package, and would mean ``assuming Maple is consistent, then\ldots''  This and
many other topics concerning the formalization of mathematics are discussed in
John Harrison's\index{Harrison, John} very interesting
PhD thesis~\cite{Harrison-thesis}.

\subsection{Automated Theorem Provers}\label{theoremprovers}

A mathematical theory is ``decidable''\index{decidable theory} if a mechanical
method or algorithm exists that is guaranteed to determine whether or not a
particular formula is a theorem.  Among the few theories that are decidable is
elementary geometry,\index{Euclidean geometry} as was shown by a classic
result of logician Alfred Tarski\index{Tarski, Alfred} in 1948
\cite{Tarski}.\footnote{Tarski's result actually applies to a subset of the
geometry discussed in elementary textbooks.  This subset includes most of what
would be considered elementary geometry but it is not powerful enough to
express, among other things, the notions of the circumference and area of a
circle.  Extending the theory in a way that includes notions such as these
makes the theory undecidable, as was also shown by Tarski.  Tarski's algorithm
is far too inefficient to implement practically on a computer.  A practical
algorithm for proving a smaller subset of geometry theorems (those not
involving concepts of ``order'' or ``continuity'') was discovered by Chinese
mathematician Wu Wen-ts\"{u}n in 1977 \cite{Chou}\index{Chou,
Shang-Ching}.}\index{Wen-ts{\"{u}}n, Wu}  But most theories, including
elementary arithmetic, are undecidable.  This fact contributes to keeping
mathematics alive and well, since many mathematicians believe
that they will never be
replaced by computers (if they believe Roger Penrose's argument that a
computer can never replace the brain \cite{Penrose}\index{Penrose, Roger}).
In fact,  elementary geometry is often considered a ``dead'' field
for the simple reason that it is decidable.

On the other hand, the undecidability of a theory does not mean that one cannot
use a computer to search for proofs, providing one is willing to give up if a
proof is not found after a reasonable amount of time.  The field of automated
theorem proving\index{automated theorem proving} specializes in pursuing such
computer searches.  Among the more successful results to date are those based
on an algorithm known as Robinson's resolution principle
\cite{Robinson}\index{Robinson's resolution principle}.

Automated theorem provers can be excellent tools for those willing to learn
how to use them.  But they are not widely used in mainstream pure
mathematics, even though they could probably be useful in many areas.  There
are several reasons for this.  Probably most important, the main goal in pure
mathematics is to arrive at results that are considered to be deep or
important; proving them is essential but secondary.  Usually, an automated
theorem prover cannot assist in this main goal, and by the time the main goal
is achieved, the mathematician may have already figured out the proof as a
by-product.  There is also a notational problem.  Mathematicians are used to
using very compact syntax where one or two symbols (heavily dependent on
context) can represent very complex concepts; this is part of the
hierarchy\index{hierarchy} they have built up to tackle difficult problems.  A
theorem prover on the other hand might require that a theorem be expressed in
``first-order logic,''\index{first-order logic} which is the logic on which
most of mathematics is ultimately based but which is not ordinarily used
directly because expressions can become very long.  Some automated theorem
provers are experimental programs, limited in their use to very specialized
areas, and the goal of many is simply research into the nature of automated
theorem proving itself.  Finally, much research remains to be done to enable
them to prove very deep theorems.  One significant result was a
computer proof by Larry Wos\index{Wos, Larry} and colleagues that every Robbins
algebra\index{Robbins algebra} is a Boolean  algebra\index{Boolean algebra}
({\em New York Times}, Dec. 10, 1996).\footnote{In 1933, E.~V.\
Huntington\index{Huntington, E. V.}
presented the following axiom system for
Boolean algebra with a unary operation $n$ and a binary operation $+$:
\begin{center}
    $x + y = y + x$ \\
    $(x + y) + z = x + (y + z)$ \\
    $n(n(x) + y) + n(n(x) + n(y)) = x$
\end{center}
Herbert Robbins\index{Robbins, Herbert}, a student of Huntington, conjectured
that the last equation can be replaced with a simpler one:
\begin{center}
    $n(n(x + y) + n(x + n(y))) = x$
\end{center}
Robbins and Huntington could not find a proof.  The problem was
later studied unsuccessfully by Tarski\index{Tarski, Alfred} and his
students, and it remained an unsolved problem until a
computer found the proof in 1996.  For more information on
the Robbins algebra problem see \cite{Wos}.}

How does Metamath\index{Metamath} relate to automated theorem provers?  A
theorem prover is primarily concerned with one theorem at a time (perhaps
tapping into a small database of known theorems) whereas Metamath is more like
a theorem archiving system, storing both the theorem and its proof in a
database for access and verification.  Metamath is one answer to ``what do you
do with the output of a theorem prover?''  and could be viewed as the
next step in the process.  Automated theorem provers could be useful tools for
helping develop its database.
Note that very long, automatically
generated proofs can make your database fat and ugly and cause Metamath's proof
verification to take a long time to run.  Unless you have a particularly good
program that generates very concise proofs, it might be best to consider the
use of automatically generated proofs as a quick-and-dirty approach, to be
manually rewritten at some later date.

The program {\sc otter}\index{otter@{\sc otter}}\footnote{\url{http://www.cs.unm.edu/\~mccune/otter/}.}, later succeeded by
prover9\index{prover9}\footnote{\url{https://www.cs.unm.edu/~mccune/mace4/}.},
have been historically influential.
The E prover\index{E prover}\footnote{\url{https://github.com/eprover/eprover}.}
is a maintained automated theorem prover
for full first-order logic with equality.
There are many other automated theorem provers as well.

If you want to combine automated theorem provers with Metamath
consider investigating
the book {\em Automated Reasoning:  Introduction and Applications}
\cite{Wos}\index{Wos, Larry}.  This book discusses
how to use {\sc otter} in a way that can
not only able to generate
relatively efficient proofs, it can even be instructed to search for
shorter proofs.  The effective use of {\sc otter} (and similar tools)
does require a certain
amount of experience, skill, and patience.  The axiom system used in the
\texttt{set.mm}\index{set theory database (\texttt{set.mm})} set theory
database can be expressed to {\sc otter} using a method described in
\cite{Megill}.\index{Megill, Norman}\footnote{To use those axioms with
{\sc otter}, they must be restated in a way that eliminates the need for
``dummy variables.''\index{dummy variable!eliminating} See the Comment
on p.~\pageref{nodd}.} When successful, this method tends to generate
short and clever proofs, but my experiments with it indicate that the
method will find proofs within a reasonable time only for relatively
easy theorems.  It is still fun to experiment with.

Reference \cite{Bledsoe}\index{Bledsoe, W. W.} surveys a number of approaches
people have explored in the field of automated theorem proving\index{automated
theorem proving}.

\subsection{Interactive Theorem Provers}\label{interactivetheoremprovers}

Finding proofs completely automatically is difficult, so there
are some interactive theorem provers that allow a human to guide the
computer to find a proof.
Examples include
HOL Light\index{HOL light}%
\footnote{\url{https://www.cl.cam.ac.uk/~jrh13/hol-light/}.},
Isabelle\index{Isabelle}%
\footnote{\url{http://www.cl.cam.ac.uk/Research/HVG/Isabelle}.},
{\sc hol}\index{hol@{\sc hol}}%
\footnote{\url{https://hol-theorem-prover.org/}.},
and
Coq\index{Coq}\footnote{\url{https://coq.inria.fr/}.}.

A major difference between most of these tools and Metamath is that the
``proofs'' are actually programs that guide the program to find a proof,
and not the proof itself.
For example, an Isabelle/HOL proof might apply a step
\texttt{apply (blast dest: rearrange reduction)}. The \texttt{blast}
instruction applies
an automatic tableux prover and returns if it found a sequence of proof
steps that work... but the sequence is not considered part of the proof.

A good overview of
higher-level proof verification languages (such as {\sc lcf}\index{lcf@{\sc
lcf}} and {\sc hol}\index{hol@{\sc hol}})
is given in \cite{Harrison}.  All of these languages are fundamentally
different from Metamath in that much of the mathematical foundational
knowledge is embedded in the underlying proof-verification program, rather
than placed directly in the database that is being verified.
These can have a steep learning curve for those without a mathematical
background.  For example, one usually must have a fair understanding of
mathematical logic in order to follow their proofs.

\subsection{Proof Verifiers}\label{proofverifiers}

A proof verifier is a program that doesn't generate proofs but instead
verifies proofs that you give it.  Many proof verifiers have limited built-in
automated proof capabilities, such as figuring out simple logical inferences
(while still being guided by a person who provides the overall proof).  Metamath
has no built-in automated proof capability other than the limited
capability of its Proof Assistant.

Proof-verification languages are not used as frequently as they might be.
Pure mathematicians are more concerned with producing new results, and such
detail and rigor would interfere with that goal.  The use of computers in pure
mathematics is primarily focused on automated theorem provers (not verifiers),
again with the ultimate goal of aiding the creation of new mathematics.
Automated theorem provers are usually concerned with attacking one theorem at
time rather than making a large, organized database easily available to the
user.  Metamath is one way to help close this gap.

By itself Metamath is a mostly a proof verifier.
This does not mean that other approaches can't be used; the difference
is that in Metamath, the results of various provers must be recorded
step-by-step so that they can be verified.

Another proof-verification language is Mizar,\index{Mizar} which can display
its proofs in the informal language that mathematicians are accustomed to.
Information on the Mizar language is available at \url{http://mizar.org}.

For the working mathematician, Mizar is an excellent tool for rigorously
documenting proofs. Mizar typesets its proofs in the informal English used by
mathematicians (and, while fine for them, are just as inscrutable by
laypersons!). A price paid for Mizar is a relatively steep learning curve of a
couple of weeks.  Several mathematicians are actively formalizing different
areas of mathematics using Mizar and publishing the proofs in a dedicated
journal. Unfortunately the task of formalizing mathematics is still looked
down upon to a certain extent since it doesn't involve the creation of ``new''
mathematics.

The closest system to Metamath is
the {\em Ghilbert}\index{Ghilbert} proof language (\url{http://ghilbert.org})
system developed by
Raph Levien\index{Levien, Raph}.
Ghilbert is a formal proof checker heavily inspired by Metamath.
Ghilbert statements are s-expressions (a la Lisp), which is easy
for computers to parse but many people find them hard to read.
There are a number of differences in their specific constructs, but
there is at least one tool to translate some Metamath materials into Ghilbert.
As of 2019 the Ghilbert community is smaller and less active than the
Metamath community.
That said, the Metamath and Ghilbert communities overlap, and fruitful
conversations between them have occurred many times over the years.

\subsection{Creating a Database of Formalized Mathematics}\label{mathdatabase}

Besides Metamath, there are several other ongoing projects with the goal of
formalizing mathematics into computer-verifiable databases.
Understanding some history will help.

The {\sc qed}\index{qed project@{\sc qed} project}%
\footnote{\url{http://www-unix.mcs.anl.gov/qed}.}
project arose in 1993 and its goals were outlined in the
{\sc qed} manifesto.
The {\sc qed} manifesto was
a proposal for a computer-based database of all mathematical knowledge,
strictly formalized and with all proofs having been checked automatically.
The project had a conference in 1994 and another in 1995;
there was also a ``twenty years of the {\sc qed} manifesto'' workshop
in 2014.
Its ideals are regularly reraised.

In a 2007 paper, Freek Wiedijk identified two reasons
for the failure of the {\sc qed} project as originally envisioned:%
\cite{Wiedijk-revisited}\index{Wiedijk, Freek}

\begin{itemize}
\item Very few people are working on formalization of mathematics. There is no compelling application for fully mechanized mathematics.
\item Formalized mathematics does not yet resemble traditional mathematics. This is partly due to the complexity of mathematical notation, and partly to the limitations of existing theorem provers and proof assistants.
\end{itemize}

But this did not end the dream of
formalizing mathematics into computer-verifiable databases.
The problems that led to the {\sc qed} manifesto are still with us,
even though the challenges were harder than originally considered.
What has happened instead is that various independent projects have
worked towards formalizing mathematics into computer-verifiable databases,
each simultaneously competing and cooperating with each other.

A concrete way to see this is
Freek Wiedijk's ``Formalizing 100 Theorems'' list%
\footnote{\url{http://www.cs.ru.nl/\%7Efreek/100/}.}
which shows the progress different systems have made on a challenge list
of 100 mathematical theorems.%
\footnote{ This is not the only list of ``interesting'' theorems.
Another interesting list was posted by Oliver Knill's list
\cite{Knill}\index{Knill, Oliver}.}
The top systems as of February 2019
(in order of the number of challenges completed) are
HOL Light, Isabelle, Metamath, Coq, and Mizar.

The Metamath 100%
\footnote{\url{http://us.metamath.org/mm\_100.html}}
page (maintained by David A. Wheeler\index{Wheeler, David A.})
shows the progress of Metamath (specifically its \texttt{set.mm} database)
against this challenge list maintained by Freek Wiedijk.
The Metamath \texttt{set.mm} database
has made a lot of progress over the years,
in part because working to prove those challenge theorems required
defining various terms and proving their properties as a prerequisite.
Here are just a few of the many statements that have been
formally proven with Metamath:

% The entries of this cause the narrow display to break poorly,
% since the short amount of text means LaTeX doesn't get a lot to work with
% and the itemize format gives it even *less* margin than usual.
% No one will mind if we make just this list flushleft, since this list
% will be internally consistent.
\begin{flushleft}
\begin{itemize}
\item 1. The Irrationality of the Square Root of 2
  (\texttt{sqr2irr}, by Norman Megill, 2001-08-20)
\item 2. The Fundamental Theorem of Algebra
  (\texttt{fta}, by Mario Carneiro, 2014-09-15)
\item 22. The Non-Denumerability of the Continuum
  (\texttt{ruc}, by Norman Megill, 2004-08-13)
\item 54. The Konigsberg Bridge Problem
  (\texttt{konigsberg}, by Mario Carneiro, 2015-04-16)
\item 83. The Friendship Theorem
  (\texttt{friendship}, by Alexander W. van der Vekens, 2018-10-09)
\end{itemize}
\end{flushleft}

We thank all of those who have developed at least one of the Metamath 100
proofs, and we particularly thank
Mario Carneiro\index{Carneiro, Mario}
who has contributed the most Metamath 100 proofs as of 2019.
The Metamath 100 page shows the list of all people who have contributed a
proof, and links to graphs and charts showing progress over time.
We encourage others to work on proving theorems not yet proven in Metamath,
since doing so improves the work as a whole.

Each of the math formalization systems (including Metamath)
has different strengths and weaknesses, depending on what you value.
Key aspects that differentiate Metamath from the other top systems are:

\begin{itemize}
\item Metamath is not tied to any particular set of axioms.
\item Metamath can show every step of every proof, no exceptions.
  Most other provers only assert that a proof can be found, and do not
  show every step. This also makes verification fast, because
  the system does not need to rediscover proof details.
\item The Metamath verifier has been re-implemented in many different
  programming languages, so verification can be done by multiple
  implementations.  In particular, the
  \texttt{set.mm}\index{set theory database (\texttt{set.mm})}%
  \index{Metamath Proof Explorer} database is verified by
  four different verifiers
  written in four different languages by four different authors.
  This greatly reduces the risk of accepting an invalid
  proof due to an error in the verifier.
\item Proofs stay proven.  In some systems, changes to the system's
  syntax or how a tactic works causes proofs to fail in later versions,
  causing older work to become essentially lost.
  Metamath's language is
  extremely small and fixed, so once a proof is added to a database,
  the database can be rechecked with later versions of the Metamath program
  and with other verifiers of Metamath databases.
  If an axiom or key definition needs to be changed, it is easy to
  manipulate the database as a whole to handle the change
  without touching the underlying verifier.
  Since re-verification of an entire database takes seconds, there
  is never a reason to delay complete verification.
  This aspect is especially compelling if your
  goal is to have a long-term database of proofs.
\item Licensing is generous.  The main Metamath databases are released to
  the public domain, and the main Metamath program is open source software
  under a standard, widely-used license.
\item Substitutions are easy to understand, even by those who are not
  professional mathematicians.
\end{itemize}

Of course, other systems may have advantages over Metamath
that are more compelling, depending on what you value.
In any case, we hope this helps you understand Metamath
within a wider context.

\subsection{In Summary}\label{computers-summary}

To summarize our discussions of computers and mathematics, computer algebra
systems can be viewed as theorem generators focusing on a narrow realm of
mathematics (numbers and their properties), automated theorem provers as proof
generators for specific theorems in a much broader realm covered by a built-in
formal system such as first-order logic, interactive theorem
provers require human guidance, proof verifiers verify proofs but
historically they have been
restricted to first-order logic.
Metamath, in contrast,
is a proof verifier and documenter whose realm is essentially unlimited.

\section{Mathematics and Metamath}

\subsection{Standard Mathematics}

There are a number of ways that Metamath\index{Metamath} can be used with
standard mathematics.  The most satisfying way philosophically is to start at
the very beginning, and develop the desired mathematics from the axioms of
logic and set theory.\index{set theory}  This is the approach taken in the
\texttt{set.mm}\index{set theory database (\texttt{set.mm})}%
\index{Metamath Proof Explorer}
database (also known as the Metamath Proof Explorer).
Among other things, this database builds up to the
axioms of real and complex numbers\index{analysis}\index{real and complex
numbers} (see Section~\ref{real}), and a standard development of analysis, for
example, could start at that point, using it as a basis.   Besides this
philosophical advantage, there are practical advantages to having all of the
tools of set theory available in the supporting infrastructure.

On the other hand, you may wish to start with the standard axioms of a
mathematical theory without going through the set theoretical proofs of those
axioms.  You will need mathematical logic to make inferences, but if you wish
you can simply introduce theorems\index{theorem} of logic as
``axioms''\index{axiom} wherever you need them, with the implicit assumption
that in principle they can be proved, if they are obvious to you.  If you
choose this approach, you will probably want to review the notation used in
\texttt{set.mm}\index{set theory database (\texttt{set.mm})} so that your own
notation will be consistent with it.

\subsection{Other Formal Systems}
\index{formal system}

Unlike some programs, Metamath\index{Metamath} is not limited to any specific
area of mathematics, nor committed to any particular mathematical philosophy
such as classical logic versus intuitionism, nor limited, say, to expressions
in first-order logic.  Although the database \texttt{set.mm}
describes standard logic and set theory, Meta\-math
is actually a general-purpose language for describing a wide variety of formal
systems.\index{formal system}  Non-standard systems such as modal
logic,\index{modal logic} intuitionist logic\index{intuitionism}, higher-order
logic\index{higher-order logic}, quantum logic\index{quantum logic}, and
category theory\index{category theory} can all be described with the Metamath
language.  You define the symbols you prefer and tell Metamath the axioms and
rules you want to start from, and Metamath will verify any inferences you make
from those axioms and rules.  A simple example of a non-standard formal system
is Hofstadter's\index{Hofstadter, Douglas R.} MIU system,\index{MIU-system}
whose Metamath description is presented in Appendix~\ref{MIU}.

This is not hypothetical.
The largest Metamath database is
\texttt{set.mm}\index{set theory database (\texttt{set.mm}}%
\index{Metamath Proof Explorer}), aka the Metamath Proof Explorer,
which uses the most common axioms for mathematical foundations
(specifically classical logic combined with Zermelo--Fraenkel
set theory\index{Zermelo--Fraenkel set theory} with the Axiom of Choice).
But other Metamath databases are available:

\begin{itemize}
\item The database
  \texttt{iset.mm}\index{intuitionistic logic database (\texttt{iset.mm})},
  aka the
  Intuitionistic Logic Explorer\index{Intuitionistic Logic Explorer},
  uses intuitionistic logic (a constructivist point of view)
  instead of classical logic.
\item The database
  \texttt{nf.mm}\index{New Foundations database (\texttt{nf.mm})},
  aka the
  New Foundations Explorer\index{New Foundations Explorer},
  constructs mathematics from scratch,
  starting from Quine's New Foundations (NF) set theory axioms.
\item The database
  \texttt{hol.mm}\index{Higher-order Logic database (\texttt{hol.mm})},
  aka the
  Higher-Order Logic (HOL) Explorer\index{Higher-Order Logic (HOL) Explorer},
  starts with HOL (also called simple type theory) and derives
  equivalents to ZFC axioms, connecting the two approaches.
\end{itemize}

Since the days of David Hilbert,\index{Hilbert, David} mathematicians have
been concerned with the fact that the metalanguage\index{metalanguage} used to
describe mathematics may be stronger than the mathematics being described.
Metamath\index{Metamath}'s underlying finitary\index{finitary proof},
constructive nature provides a good philosophical basis for studying even the
weakest logics.\index{weak logic}

The usual treatment of many non-standard formal systems\index{formal
system} uses model theory\index{model theory} or proof theory\index{proof
theory} to describe these systems; these theories, in turn, are based on
standard set theory.  In other words, a non-standard formal system is defined
as a set with certain properties, and standard set theory is used to derive
additional properties of this set.  The standard set theory database provided
with Metamath can be used for this purpose, and when used this way
the development of a special
axiom system for the non-standard formal system becomes unnecessary.  The
model- or proof-theoretic approach often allows you to prove much deeper
results with less effort.

Metamath supports both approaches.  You can define the non-standard
formal system directly, or define the non-standard formal system as
a set with certain properties, whichever you find most helpful.

%\section{Additional Remarks}

\subsection{Metamath and Its Philosophy}

Closely related to Metamath\index{Metamath} is a philosophy or way of looking
at mathematics. This philosophy is related to the formalist
philosophy\index{formalism} of Hilbert\index{Hilbert, David} and his followers
\cite[pp.~1203--1208]{Kline}\index{Kline, Morris}
\cite[p.~6]{Behnke}\index{Behnke, H.}. In this philosophy, mathematics is
viewed as nothing more than a set of rules that manipulate symbols, together
with the consequences of those rules.  While the mathematics being described
may be complex, the rules used to describe it (the
``metamathematics''\index{metamathematics}) should be as simple as possible.
In particular, proofs should be restricted to dealing with concrete objects
(the symbols we write on paper rather than the abstract concepts they
represent) in a constructive manner; these are called ``finitary''
proofs\index{finitary proof} \cite[pp.~2--3]{Shoenfield}\index{Shoenfield,
Joseph R.}.

Whether or not you find Metamath interesting or useful will in part depend on
the appeal you find in its philosophy, and this appeal will probably depend on
your particular goals with respect to mathematics.  For example, if you are a
pure mathematician at the forefront of discovering new mathematical knowledge,
you will probably find that the rigid formality of Metamath stifles your
creativity.  On the other hand, we would argue that once this knowledge is
discovered, there are advantages to documenting it in a standard format that
will make it accessible to others.  Sixty years from now, your field may be
dormant, and as Davis and Hersh put it, your ``writings would become less
translatable than those of the Maya'' \cite[p.~37]{Davis}\index{Davis, Phillip
J.}.


\subsection{A History of the Approach Behind Metamath}

Probably the one work that has had the most motivating influence on
Metamath\index{Metamath} is Whitehead and Russell's monumental {\em Principia
Mathematica} \cite{PM}\index{Whitehead, Alfred North}\index{Russell,
Bertrand}\index{principia mathematica@{\em Principia Mathematica}}, whose aim
was to deduce all of mathematics from a small number of primitive ideas, in a
very explicit way that in principle anyone could understand and follow.  While
this work was tremendously influential in its time, from a modern perspective
it suffers from several drawbacks.  Both its notation and its underlying
axioms are now considered dated and are no longer used.  From our point of
view, its development is not really as accessible as we would like to see; for
practical reasons, proofs become more and more sketchy as its mathematics
progresses, and working them out in fine detail requires a degree of
mathematical skill and patience that many people don't have.  There are
numerous small errors, which is understandable given the tedious, technical
nature of the proofs and the lack of a computer to verify the details.
However, even today {\em Principia Mathematica} stands out as the work closest
in spirit to Metamath.  It remains a mind-boggling work, and one can't help
but be amazed at seeing ``$1+1=2$'' finally appear on page 83 of Volume II
(Theorem *110.643).

The origin of the proof notation used by Metamath dates back to the 1950's,
when the logician C.~A.~Meredith expressed his proofs in a compact notation
called ``condensed detachment''\index{condensed detachment}
\cite{Hindley}\index{Hindley, J. Roger} \cite{Kalman}\index{Kalman, J. A.}
\cite{Meredith}\index{Meredith, C. A.} \cite{Peterson}\index{Peterson, Jeremy
George}.  This notation allows proofs to be communicated unambiguously by
merely referencing the axiom\index{axiom}, rule\index{rule}, or
theorem\index{theorem} used at each step, without explicitly indicating the
substitutions\index{substitution!variable}\index{variable substitution} that
have to be made to the variables in that axiom, rule, or theorem.  Ordinarily,
condensed detachment is more or less limited to propositional
calculus\index{propositional calculus}.  The concept has been extended to
first-order logic\index{first-order logic} in \cite{Megill}\index{Megill,
Norman}, making it is easy to write a small computer program to verify proofs
of simple first-order logic theorems.\index{condensed detachment!and
first-order logic}

A key concept behind the notation of condensed detachment is called
``unification,''\index{unification} which is an algorithm for determining what
substitutions\index{substitution!variable}\index{variable substitution} to
variables have to be made to make two expressions match each other.
Unification was first precisely defined by the logician J.~A.~Robinson, who
used it in the development of a powerful
theorem-proving technique called the ``resolution principle''
\cite{Robinson}\index{Robinson's resolution principle}. Metamath does not make
use of the resolution principle, which is intended for systems of first-order
logic.\index{first-order logic}  Metamath's use is not restricted to
first-order logic, and as we have mentioned it does not automatically discover
proofs.  However, unification is a key idea behind Metamath's proof
notation, and Metamath makes use of a very simple version of it
(Section~\ref{unify}).

\subsection{Metamath and First-Order Logic}

First-order logic\index{first-order logic} is the supporting structure
for standard mathematics.  On top of it is set theory, which contains
the axioms from which virtually all of mathematics can be derived---a
remarkable fact.\index{category
theory}\index{cardinal, inaccessible}\label{categoryth}\footnote{An exception seems
to be category theory.  There are several schools of thought on whether
category theory is derivable from set theory.  At a minimum, it appears
that an additional axiom is needed that asserts the existence of an
``inaccessible cardinal'' (a type of infinity so large that standard set
theory can't prove or deny that it exists).
%
%%%% (I took this out that was in previous editions:)
% But it is also argued that not just one but a ``proper class'' of them
% is needed, and the existence of proper classes is impossible in standard
% set theory.  (A proper class is a collection of sets so huge that no set
% can contain it as an element.  Proper classes can lead to
% inconsistencies such as ``Russell's paradox.''  The axioms of standard
% set theory are devised so as to deny the existence of proper classes.)
%
For more information, see
\cite[pp.~328--331]{Herrlich}\index{Herrlich, Horst} and
\cite{Blass}\index{Blass, Andrea}.}

One of the things that makes Metamath\index{Metamath} more practical for
first-order theories is a set of axioms for first-order logic designed
specifically with Metamath's approach in mind.  These are included in
the database \texttt{set.mm}\index{set theory database (\texttt{set.mm})}.
See Chapter~\ref{fol} for a detailed
description; the axioms are shown in Section~\ref{metaaxioms}.  While
logically equivalent to standard axiom systems, our axiom system breaks
up the standard axioms into smaller pieces such that from them, you can
directly derive what in other systems can only be derived as higher-level
``metatheorems.''\index{metatheorem}  In other words, it is more powerful than
the standard axioms from a metalogical point of view.  A rigorous
justification for this system and its ``metalogical
completeness''\index{metalogical completeness} is found in
\cite{Megill}\index{Megill, Norman}.  The system is closely related to a
system developed by Monk\index{Monk, J. Donald} and Tarski\index{Tarski,
Alfred} in 1965 \cite{Monks}.

For example, the formula $\exists x \, x = y $ (given $y$, there exists some
$x$ equal to it) is a theorem of logic,\footnote{Specifically, it is a theorem
of those systems of logic that assume non-empty domains.  It is not a theorem
of more general systems that include the empty domain\index{empty domain}, in
which nothing exists, period!  Such systems are called ``free
logics.''\index{free logic} For a discussion of these systems, see
\cite{Leblanc}\index{Leblanc, Hugues}.  Since our use for logic is as a basis
for set theory, which has a non-empty domain, it is more convenient (and more
traditional) to use a less general system.  An interesting curiosity is that,
using a free logic as a basis for Zermelo--Fraenkel set
theory\index{Zermelo--Fraenkel set theory} (with the redundant Axiom of the
Null Set omitted),\index{Axiom of the Null Set} we cannot even prove the
existence of a single set without assuming the axiom of infinity!\index{Axiom
of Infinity}} whether or not $x$ and $y$ are distinct variables\index{distinct
variables}.  In many systems of logic, we would have to prove two theorems to
arrive at this result.  First we would prove ``$\exists x \, x = x $,'' then
we would separately prove ``$\exists x \, x = y $, where $x$ and $y$ are
distinct variables.''  We would then combine these two special cases ``outside
of the system'' (i.e.\ in our heads) to be able to claim, ``$\exists x \, x =
y $, regardless of whether $x$ and $y$ are distinct.''  In other words, the
combination of the two special cases is a
metatheorem.  In the system of logic
used in Metamath's set theory\index{set theory database (\texttt{set.mm})}
database, the axioms of logic are broken down into small pieces that allow
them to be reassembled in such a way that theorems such as these can be proved
directly.

Breaking down the axioms in this way makes them look peculiar and not very
intuitive at first, but rest assured that they are correct and complete.  Their
correctness is ensured because they are theorem schemes of standard first-order
logic (which you can easily verify if you are a logician).  Their completeness
follows from the fact that we explicitly derive the standard axioms of
first-order logic as theorems.  Deriving the standard axioms is somewhat
tricky, but once we're there, we have at our disposal a system that is less
awkward to work with in formal proofs\index{formal proof}.  In technical terms
that logicians understand, we eliminate the cumbersome concepts of ``free
variable,''\index{free variable} ``bound variable,''\index{bound variable} and
``proper substitution''\index{proper substitution}\index{substitution!proper}
as primitive notions.  These concepts are present in our system but are
defined in terms of concepts expressed by the axioms and can be eliminated in
principle.  In standard systems, these concepts are really like additional,
implicit axioms\index{implicit axiom} that are somewhat complex and cannot be
eliminated.

The traditional approach to logic, wherein free variables and proper
substitution is defined, is also possible to do directly in the Metamath
language.  However, the notation tends to become awkward, and there are
disadvantages:  for example, extending the definition of a wff with a
definition is awkward, because the free variable and proper substitution
concepts have to have their definitions also extended.  Our choice of
axioms for \texttt{set.mm} is to a certain extent a matter of style, in
an attempt to achieve overall simplicity, but you should also be aware
that the traditional approach is possible as well if you should choose
it.

\chapter{Using the Metamath Program}
\label{using}

\section{Installation}

The way that you install Metamath\index{Metamath!installation} on your
computer system will vary for different computers.  Current
instructions are provided with the Metamath program download at
\url{http://metamath.org}.  In general, the installation is simple.
There is one file containing the Metamath program itself.  This file is
usually called \texttt{metamath} or \texttt{metamath.}{\em xxx} where
{\em xxx} is the convention (such as \texttt{exe}) for an executable
program on your operating system.  There are several additional files
containing samples of the Metamath language, all ending with
\texttt{.mm}.  The file \texttt{set.mm}\index{set theory database
(\texttt{set.mm})} contains logic and set theory and can be used as a
starting point for other areas of mathematics.

You will also need a text editor\index{text editor} capable of editing plain
{\sc ascii}\footnote{American Standard Code for Information Interchange.} text
in order to prepare your input files.\index{ascii@{\sc ascii}}  Most computers
have this capability built in.  Note that plain text is not necessarily the
default for some word processing programs\index{word processor}, especially if
they can handle different fonts; for example, with Microsoft Word\index{Word
(Microsoft)}, you must save the file in the format ``Text Only With Line
Breaks'' to get a plain text\index{plain text} file.\footnote{It is
recommended that all lines in a Metamath source file be 79 characters or less
in length for compatibility among different computer terminals.  When creating
a source file on an editor such as Word, select a monospaced
font\index{monospaced font} such as Courier\index{Courier font} or
Monaco\index{Monaco font} to make this easier to achieve.  Better yet,
just use a plain text editor such as Notepad.}

On some computer systems, Metamath does not have the capability to print
its output directly; instead, you send its output to a file (using the
\texttt{open} commands described later).  The way you print this output
file depends on your computer.\index{printers} Some computers have a
print command, whereas with others, you may have to read the file into
an editor and print it from there.

If you want to print your Metamath source files with typeset formulas
containing standard mathematical symbols, you will need the \LaTeX\
typesetting program\index{latex@{\LaTeX}}, which is widely and freely
available for most operating systems.  It runs natively on Unix and
Linux, and can be installed on Windows as part of the free Cygwin
package (\url{http://cygwin.com}).

You can also produce {\sc html}\footnote{HyperText Markup Language.}
web pages.  The {\tt help html} command in the Metamath program will
assist you with this feature.

\section{Your First Formal System}\label{start}
\subsection{From Nothing to Zero}\label{startf}

To give you a feel for what the Metamath\index{Metamath} language looks like,
we will take a look at a very simple example from formal number
theory\index{number theory}.  This example is taken from
Mendelson\index{Mendelson, Elliot} \cite[p. 123]{Mendelson}.\footnote{To keep
the example simple, we have changed the formalism slightly, and what we call
axioms\index{axiom} are strictly speaking theorems\index{theorem} in
\cite{Mendelson}.}  We will look at a small subset of this theory, namely that
part needed for the first number theory theorem proved in \cite{Mendelson}.

First we will look at a standard formal proof\index{formal proof} for the
example we have picked, then we will look at the Metamath version.  If you
have never been exposed to formal proofs, the notation may seem to be such
overkill to express such simple notions that you may wonder if you are missing
something.  You aren't.  The concepts involved are in fact very simple, and a
detailed breakdown in this fashion is necessary to express the proof in a way
that can be verified mechanically.  And as you will see, Metamath breaks the
proof down into even finer pieces so that the mechanical verification process
can be about as simple as possible.

Before we can introduce the axioms\index{axiom} of the theory, we must define
the syntax rules for forming legal expressions\index{syntax rules}
(combinations of symbols) with which those axioms can be used. The number 0 is
a {\bf term}\index{term}; and if $ t$ and $r$ are terms, so is $(t+r)$. Here,
$ t$ and $r$ are ``metavariables''\index{metavariable} ranging over terms; they
themselves do not appear as symbols in an actual term.  Some examples of
actual terms are $(0 + 0)$ and $((0+0)+0)$.  (Note that our theory describes
only the number zero and sums of zeroes.  Of course, not much can be done with
such a trivial theory, but remember that we have picked a very small subset of
complete number theory for our example.  The important thing for you to focus
on is our definitions that describe how symbols are combined to form valid
expressions, and not on the content or meaning of those expressions.) If $ t$
and $r$ are terms, an expression of the form $ t=r$ is a {\bf wff}
(well-formed formula)\index{well-formed formula (wff)}; and if $P$ and $Q$ are
wffs, so is $(P\rightarrow Q)$ (which means ``$P$ implies
$Q$''\index{implication ($\rightarrow$)} or ``if $P$ then $Q$'').
Here $P$ and $Q$ are metavariables ranging over wffs.  Examples of actual
wffs are $0=0$, $(0+0)=0$, $(0=0 \rightarrow (0+0)=0)$, and $(0=0\rightarrow
(0=0\rightarrow 0=(0+0)))$.  (Our notation makes use of more parentheses than
are customary, but the elimination of ambiguity this way simplifies our
example by avoiding the need to define operator precedence\index{operator
precedence}.)

The {\bf axioms}\index{axiom} of our theory are all wffs of the following
form, where $ t$, $r$, and $s$ are any terms:

%Latex p. 92
\renewcommand{\theequation}{A\arabic{equation}}

\begin{equation}
(t=r\rightarrow (t=s\rightarrow r=s))
\end{equation}
\begin{equation}
(t+0)=t
\end{equation}

Note that there are an infinite number of axioms since there are an infinite
number of possible terms.  A1 and A2 are properly called ``axiom
schemes,''\index{axiom scheme} but we will refer to them as ``axioms'' for
brevity.

An axiom is a {\bf theorem}; and if $P$ and $(P\rightarrow Q)$ are theorems
(where $P$ and $Q$ are wffs), then $Q$ is also a theorem.\index{theorem}  The
second part of this definition is called the modus ponens (MP) rule of
inference\index{inference rule}\index{modus ponens}.  It allows us to obtain
new theorems from old ones.

The {\bf proof}\index{proof} of a theorem is a sequence of one or more
theorems, each of which is either an axiom or the result of modus ponens
applied to two previous theorems in the sequence, and the last of which is the
theorem being proved.

The theorem we will prove for our example is very simple:  $ t=t$.  The proof of
our theorem follows.  Study it carefully until you feel sure you
understand it.\label{zeroproof}

% Use tabu so that lines will wrap automatically as needed.
\begin{tabu} { l X X }
1. & $(t+0)=t$ & (by axiom A2) \\
2. & $(t+0)=t$ & (by axiom A2) \\
3. & $((t+0)=t \rightarrow ((t+0)=t\rightarrow t=t))$ & (by axiom A1) \\
4. & $((t+0)=t\rightarrow t=t)$ & (by MP applied to steps 2 and 3) \\
5. & $t=t$ & (by MP applied to steps 1 and 4) \\
\end{tabu}

(You may wonder why step 1 is repeated twice.  This is not necessary in the
formal language we have defined, but in Metamath's ``reverse Polish
notation''\index{reverse Polish notation (RPN)} for proofs, a previous step
can be referred to only once.  The repetition of step~1 here will enable you
to see more clearly the correspondence of this proof with the
Metamath\index{Metamath} version on p.~\pageref{demoproof}.)

Our theorem is more properly called a ``theorem scheme,''\index{theorem
scheme} for it represents an infinite number of theorems, one for each
possible term $ t$.  Two examples of actual theorems would be $0=0$ and
$(0+0)=(0+0)$.  Rarely do we prove actual theorems, since by proving schemes
we can prove an infinite number of theorems in one fell swoop.  Similarly, our
proof should really be called a ``proof scheme.''\index{proof scheme}  To
obtain an actual proof, pick an actual term to use in place of $ t$, and
substitute it for $ t$ throughout the proof.

Let's discuss what we have done here.  The axioms\index{axiom} of our theory,
A1 and A2, are trivial and obvious.  Everyone knows that adding zero to
something doesn't change it, and also that if two things are equal to a third,
then they are equal to each other. In fact, stating the trivial and obvious is
a goal to strive for in any axiomatic system.  From trivial and obvious truths
that everyone agrees upon, we can prove results that are not so obvious yet
have absolute faith in them.  If we trust the axioms and the rules, we must,
by definition, trust the consequences of those axioms and rules, if logic is
to mean anything at all.

Our rule of inference\index{rule}, modus ponens\index{modus ponens}, is also
pretty obvious once you understand what it means.  If we prove a fact $P$, and
we also prove that $P$ implies $Q$, then $Q$ necessarily follows as a new
fact.  The rule provides us with a means for obtaining new facts (i.e.\
theorems\index{theorem}) from old ones.

The theorem that we have proved, $ t=t$, is so fundamental that you may wonder
why it isn't one of the axioms\index{axiom}.  In some axiom systems of
arithmetic, it {\em is} an axiom.  The choice of axioms in a theory is to some
extent arbitrary and even an art form, constrained only by the requirement
that any two equivalent axiom systems be able to derive each other as
theorems.  We could imagine that the inventor of our axiom system originally
included $ t=t$ as an axiom, then discovered that it could be derived as a
theorem from the other axioms.  Because of this, it was not necessary to
keep it as an axiom.  By eliminating it, the final set of axioms became
that much simpler.

Unless you have worked with formal proofs\index{formal proof} before, it
probably wasn't apparent to you that $ t=t$ could be derived from our two
axioms until you saw the proof. While you certainly believe that $ t=t$ is
true, you might not be able to convince an imaginary skeptic who believes only
in our two axioms until you produce the proof.  Formal proofs such as this are
hard to come up with when you first start working with them, but after you get
used to them they can become interesting and fun.  Once you understand the
idea behind formal proofs you will have grasped the fundamental principle that
underlies all of mathematics.  As the mathematics becomes more sophisticated,
its proofs become more challenging, but ultimately they all can be broken down
into individual steps as simple as the ones in our proof above.

Mendelson's\index{Mendelson, Elliot} book, from which our example was taken,
contains a number of detailed formal proofs such as these, and you may be
interested in looking it up.  The book is intended for mathematicians,
however, and most of it is rather advanced.  Popular literature describing
formal proofs\index{formal proof} include \cite[p.~296]{Rucker}\index{Rucker,
Rudy} and \cite[pp.~204--230]{Hofstadter}\index{Hofstadter, Douglas R.}.

\subsection{Converting It to Metamath}\label{convert}

Formal proofs\index{formal proof} such as the one in our example break down
logical reasoning into small, precise steps that leave little doubt that the
results follow from the axioms\index{axiom}.  You might think that this would
be the finest breakdown we can achieve in mathematics.  However, there is more
to the proof than meets the eye. Although our axioms were rather simple, a lot
of verbiage was needed before we could even state them:  we needed to define
``term,'' ``wff,'' and so on.  In addition, there are a number of implied
rules that we haven't even mentioned. For example, how do we know that step 3
of our proof follows from axiom A1? There is some hidden reasoning involved in
determining this.  Axiom A1 has two occurrences of the letter $ t$.  One of
the implied rules states that whatever we substitute for $ t$ must be a legal
term\index{term}.\footnote{Some authors make this implied rule explicit by
stating, ``only expressions of the above form are terms,'' after defining
``term.''}  The expression $ t+0$ is pretty obviously a legal term whenever $
t$ is, but suppose we wanted to substitute a huge term with thousands of
symbols?  Certainly a lot of work would be involved in determining that it
really is a term, but in ordinary formal proofs all of this work would be
considered a single ``step.''

To express our axiom system in the Metamath\index{Metamath} language, we must
describe this auxiliary information in addition to the axioms themselves.
Metamath does not know what a ``term'' or a ``wff''\index{well-formed formula
(wff)} is.  In Metamath, the specification of the ways in which we can combine
symbols to obtain terms and wffs are like little axioms in themselves.  These
auxiliary axioms are expressed in the same notation as the ``real''
axioms\index{axiom}, and Metamath does not distinguish between the two.  The
distinction is made by you, i.e.\ by the way in which you interpret the
notation you have chosen to express these two kinds of axioms.

The Metamath language breaks down mathematical proofs into tiny pieces, much
more so than in ordinary formal proofs\index{formal proof}.  If a single
step\index{proof step} involves the
substitution\index{substitution!variable}\index{variable substitution} of a
complex term for one of its variables, Metamath must see this single step
broken down into many small steps.  This fine-grained breakdown is what gives
Metamath generality and flexibility as it lets it not be limited to any
particular mathematical notation.

Metamath's proof notation is not, in itself, intended to be read by humans but
rather is in a compact format intended for a machine.  The Metamath program
will convert this notation to a form you can understand, using the \texttt{show
proof}\index{\texttt{show proof} command} command.  You can tell the program what
level of detail of the proof you want to look at.  You may want to look at
just the logical inference steps that correspond
to ordinary formal proof steps,
or you may want to see the fine-grained steps that prove that an expression is
a term.

Here, without further ado, is our example converted to the
Metamath\index{Metamath} language:\index{metavariable}\label{demo0}

\begin{verbatim}
$( Declare the constant symbols we will use $)
    $c 0 + = -> ( ) term wff |- $.
$( Declare the metavariables we will use $)
    $v t r s P Q $.
$( Specify properties of the metavariables $)
    tt $f term t $.
    tr $f term r $.
    ts $f term s $.
    wp $f wff P $.
    wq $f wff Q $.
$( Define "term" and "wff" $)
    tze $a term 0 $.
    tpl $a term ( t + r ) $.
    weq $a wff t = r $.
    wim $a wff ( P -> Q ) $.
$( State the axioms $)
    a1 $a |- ( t = r -> ( t = s -> r = s ) ) $.
    a2 $a |- ( t + 0 ) = t $.
$( Define the modus ponens inference rule $)
    ${
       min $e |- P $.
       maj $e |- ( P -> Q ) $.
       mp  $a |- Q $.
    $}
$( Prove a theorem $)
    th1 $p |- t = t $=
  $( Here is its proof: $)
       tt tze tpl tt weq tt tt weq tt a2 tt tze tpl
       tt weq tt tze tpl tt weq tt tt weq wim tt a2
       tt tze tpl tt tt a1 mp mp
     $.
\end{verbatim}\index{metavariable}

A ``database''\index{database} is a set of one or more {\sc ascii} source
files.  Here's a brief description of this Metamath\index{Metamath} database
(which consists of this single source file), so that you can understand in
general terms what is going on.  To understand the source file in detail, you
should read Chapter~\ref{languagespec}.

The database is a sequence of ``tokens,''\index{token} which are normally
separated by spaces or line breaks.  The only tokens that are built into
the Metamath language are those beginning with \texttt{\$}.  These tokens
are called ``keywords.''\index{keyword}  All other tokens are
user-defined, and their names are arbitrary.

As you might have guessed, the Metamath token \texttt{\$(}\index{\texttt{\$(} and
\texttt{\$)} auxiliary keywords} starts a comment and \texttt{\$)} ends a comment.

The Metamath tokens \texttt{\$c}\index{\texttt{\$c} statement},
\texttt{\$v}\index{\texttt{\$v} statement},
\texttt{\$e}\index{\texttt{\$e} statement},
\texttt{\$f}\index{\texttt{\$f} statement},
\texttt{\$a}\index{\texttt{\$a} statement}, and
\texttt{\$p}\index{\texttt{\$p} statement} specify ``statements'' that
end with \texttt{\$.}\,.\index{\texttt{\$.}\ keyword}

The Metamath tokens \texttt{\$c} and \texttt{\$v} each declare\index{constant
declaration}\index{variable declaration} a list of user-defined tokens, called
``math symbols,''\index{math symbol} that the database will reference later
on.  All of the math symbols they define you have seen earlier except the
turnstile symbol \texttt{|-} ($\vdash$)\index{turnstile ({$\,\vdash$})}, which is
commonly used by logicians to mean ``a proof exists for.''  For us
the turnstile is just a
convenient symbol that distinguishes expressions that are axioms\index{axiom}
or theorems\index{theorem} from expressions that are terms or wffs.

The \texttt{\$c} statement declares ``constants''\index{constant} and
the \texttt{\$v} statement declares
``variables''\index{variable}\index{constant declaration}\index{variable
declaration} (or more precisely, metavariables\index{metavariable}).  A
variable may be substituted\index{substitution!variable}\index{variable
substitution} with sequences of math symbols whereas a constant may not
be substituted with anything.

It may seem redundant to require both \texttt{\$c}\index{\texttt{\$c} statement} and
\texttt{\$v}\index{\texttt{\$v} statement} statements (since any math
symbol\index{math symbol} not specified with a \texttt{\$c} statement could be
presumed to be a variable), but this provides for better error checking and
also allows math symbols to be redeclared\index{redeclaration of symbols}
(Section~\ref{scoping}).

The token \texttt{\$f}\index{\texttt{\$f} statement} specifies a
statement called a ``variable-type hypothesis'' (also called a
``floating hypothesis'') and \texttt{\$e}\index{\texttt{\$e} statement}
specifies a ``logical hypothesis'' (also called an ``essential
hypothesis'').\index{hypothesis}\index{variable-type
hypothesis}\index{logical hypothesis}\index{floating
hypothesis}\index{essential hypothesis} The token
\texttt{\$a}\index{\texttt{\$a} statement} specifies an ``axiomatic
assertion,''\index{axiomatic assertion} and
\texttt{\$p}\index{\texttt{\$p} statement} specifies a ``provable
assertion.''\index{provable assertion} To the left of each occurrence of
these four tokens is a ``label''\index{label} that identifies the
hypothesis or assertion for later reference.  For example, the label of
the first axiomatic assertion is \texttt{tze}.  A \texttt{\$f} statement
must contain exactly two math symbols, a constant followed by a
variable.  The \texttt{\$e}, \texttt{\$a}, and \texttt{\$p} statements
each start with a constant followed by, in general, an arbitrary
sequence of math symbols.

Associated with each assertion\index{assertion} is a set of hypotheses
that must be satisfied in order for the assertion to be used in a proof.
These are called the ``mandatory hypotheses''\index{mandatory
hypothesis} of the assertion.  Among those hypotheses whose ``scope''
(described below) includes the assertion, \texttt{\$e} hypotheses are
always mandatory and \texttt{\$f}\index{\texttt{\$f} statement}
hypotheses are mandatory when they share their variable with the
assertion or its \texttt{\$e} hypotheses.  The exact rules for
determining which hypotheses are mandatory are described in detail in
Sections~\ref{frames} and \ref{scoping}.  For example, the mandatory
hypotheses of assertion \texttt{tpl} are \texttt{tt} and \texttt{tr},
whereas assertion \texttt{tze} has no mandatory hypotheses because it
contains no variables and has no \texttt{\$e}\index{\texttt{\$e}
statement} hypothesis.  Metamath's \texttt{show statement}
command\index{\texttt{show statement} command}, described in the next
section, will show you a statement's mandatory hypotheses.

Sometimes we need to make a hypothesis relevant to only certain
assertions.  The set of statements to which a hypothesis is relevant is
called its ``scope.''  The Metamath brackets,
\texttt{\$\char`\{}\index{\texttt{\$\char`\{} and \texttt{\$\char`\}}
keywords} and \texttt{\$\char`\}}, define a ``block''\index{block} that
delimits the scope of any hypothesis contained between them.  The
assertion \texttt{mp} has mandatory hypotheses \texttt{wp}, \texttt{wq},
\texttt{min}, and \texttt{maj}.  The only mandatory hypothesis of
\texttt{th1}, on the other hand, is \texttt{tt}, since \texttt{th1}
occurs outside of the block containing \texttt{min} and \texttt{maj}.

Note that \texttt{\$\char`\{} and \texttt{\$\char`\}} do not affect the
scope of assertions (\texttt{\$a} and \texttt{\$p}).  Assertions are always
available to be referenced by any later proof in the source file.

Each provable assertion (\texttt{\$p}\index{\texttt{\$p} statement}
statement) has two parts.  The first part is the
assertion\index{assertion} itself, which is a sequence of math
symbol\index{math symbol} tokens placed between the \texttt{\$p} token
and a \texttt{\$=}\index{\texttt{\$=} keyword} token.  The second part
is a ``proof,'' which is a list of label tokens placed between the
\texttt{\$=} token and the \texttt{\$.}\index{\texttt{\$.}\ keyword}\
token that ends the statement.\footnote{If you've looked at the
\texttt{set.mm} database, you may have noticed another notation used for
proofs.  The other notation is called ``compressed.''\index{compressed
proof}\index{proof!compressed} It reduces the amount of space needed to
store a proof in the database and is described in
Appendix~\ref{compressed}.  In the example above, we use
``normal''\index{normal proof}\index{proof!normal} notation.} The proof
acts as a series of instructions to the Metamath program, telling it how
to build up the sequence of math symbols contained in the assertion part of
the \texttt{\$p} statement, making use of the hypotheses of the
\texttt{\$p} statement and previous assertions.  The construction takes
place according to precise rules.  If the list of labels in the proof
causes these rules to be violated, or if the final sequence that results
does not match the assertion, the Metamath program will notify you with
an error message.

If you are familiar with reverse Polish notation (RPN), which is sometimes used
on pocket calculators, here in a nutshell is how a proof works.  Each
hypothesis label\index{hypothesis label} in the proof is pushed\index{push}
onto the RPN stack\index{stack}\index{RPN stack} as it is encountered. Each
assertion label\index{assertion label} pops\index{pop} off the stack as many
entries as the referenced assertion has mandatory hypotheses.  Variable
substitutions\index{substitution!variable}\index{variable substitution} are
computed which, when made to the referenced assertion's mandatory hypotheses,
cause these hypotheses to match the stack entries. These same substitutions
are then made to the variables in the referenced assertion itself, which is
then pushed onto the stack.  At the end of the proof, there should be one
stack entry, namely the assertion being proved.  This process is explained in
detail in Section~\ref{proof}.

Metamath's proof notation is not very readable for humans, but it allows the
proof to be stored compactly in a file.  The Metamath\index{Metamath} program
has proof display features that let you see what's going on in a more
readable way, as you will see in the next section.

The rules used in verifying a proof are not based on any built-in syntax of the
symbol sequence in an assertion\index{assertion} nor on any built-in meanings
attached to specific symbol names.  They are based strictly on symbol
matching:  constants\index{constant} must match themselves, and
variables\index{variable} may be replaced with anything that allows a match to
occur.  For example, instead of \texttt{term}, \texttt{0}, and \verb$|-$ we could
have just as well used \texttt{yellow}, \texttt{zero}, and \texttt{provable}, as long
as we did so consistently throughout the database.  Also, we could have used
\texttt{is provable} (two tokens) instead of \verb$|-$ (one token) throughout the
database.  In each of these cases, the proof would be exactly the same.  The
independence of proofs and notation means that you have a lot of flexibility to
change the notation you use without having to change any proofs.

\section{A Trial Run}\label{trialrun}

Now you are ready to try out the Metamath\index{Metamath} program.

On all computer systems, Metamath has a standard ``command line
interface'' (CLI)\index{command line interface (CLI)} that allows you to
interact with it.  You supply commands to the CLI by typing them on the
keyboard and pressing your keyboard's {\em return} key after each line
you enter.  The CLI is designed to be easy to use and has built-in help
features.

The first thing you should do is to use a text editor to create a file
called \texttt{demo0.mm} and type into it the Metamath source shown on
p.~\pageref{demo0}.  Actually, this file is included with your Metamath
software package, so check that first.  If you type it in, make sure
that you save it in the form of ``plain {\sc ascii} text with line
breaks.''  Most word processors will have this feature.

Next you must run the Metamath program.  Depending on your computer
system and how Metamath is installed, this could range from clicking the
mouse on the Metamath icon to typing \texttt{run metamath} to typing
simply \texttt{metamath}.  (Metamath's {\tt help invoke} command describes
alternate ways of invoking the Metamath program.)

When you first enter Metamath\index{Metamath}, it will be at the CLI, waiting
for your input. You will see something like the following on your screen:
\begin{verbatim}
Metamath - Version 0.172 25-Jan-2019
Type HELP for help, EXIT to exit.
MM>
\end{verbatim}
The \texttt{MM>} prompt means that Metamath is waiting for a command.
Command keywords\index{command keyword} are not case sensitive;
we will use lower-case commands in our examples.
The version number and its release date will probably be different on your
system from the one we show above.

The first thing that you need to do is to read in your
database:\index{\texttt{read} command}\footnote{If a directory path is
needed on Unix,\index{Unix file names}\index{file names!Unix} you should
enclose the path/file name in quotes to prevent Metamath from thinking
that the \texttt{/} in the path name is a command qualifier, e.g.,
\texttt{read \char`\"db/set.mm\char`\"}.  Quotes are optional when there
is no ambiguity.}
\begin{verbatim}
MM> read demo0.mm
\end{verbatim}
Remember to press the {\em return} key after entering this command.  If
you omit the file name, Metamath will prompt you for one.   The syntax for
specifying a Macintosh file name path is given in a footnote on
p.~\pageref{includef}.\index{Macintosh file names}\index{file
names!Macintosh}

If there are any syntax errors in the database, Metamath will let you know
when it reads in the file.  The one thing that Metamath does not check when
reading in a database is that all proofs are correct, because this would
slow it down too much.  It is a good idea to periodically verify the proofs in
a database you are making changes to.  To do this, use the following command
(and do it for your \texttt{demo0.mm} file now).  Note that the \texttt{*} is a
``wild card'' meaning all proofs in the file.\index{\texttt{verify proof} command}
\begin{verbatim}
MM> verify proof *
\end{verbatim}
Metamath will report any proofs that are incorrect.

It is often useful to save the information that the Metamath program displays
on the screen. You can save everything that happens on the screen by opening a
log file. You may want to do this before you read in a database so that you
can examine any errors later on.  To open a log file, type
\begin{verbatim}
MM> open log abc.log
\end{verbatim}
This will open a file called \texttt{abc.log}, and everything that appears on the
screen from this point on will be stored in this file.  The name of the log file
is arbitrary. To close the log file, type
\begin{verbatim}
MM> close log
\end{verbatim}

Several commands let you examine what's inside your database.
Section~\ref{exploring} has an overview of some useful ones.  The
\texttt{show labels} command lets you see what statement
labels\index{label} exist.  A \texttt{*} matches any combination of
characters, and \texttt{t*} refers to all labels starting with the
letter \texttt{t}.\index{\texttt{show labels} command} The \texttt{/all}
is a ``command qualifier''\index{command qualifier} that tells Metamath
to include labels of hypotheses.  (To see the syntax explained, type
\texttt{help show labels}.)  Type
\begin{verbatim}
MM> show labels t* /all
\end{verbatim}
Metamath will respond with
\begin{verbatim}
The statement number, label, and type are shown.
3 tt $f       4 tr $f       5 ts $f       8 tze $a
9 tpl $a      19 th1 $p
\end{verbatim}

You can use the \texttt{show statement} command to get information about a
particular statement.\index{\texttt{show statement} command}
For example, you can get information about the statement with label \texttt{mp}
by typing
\begin{verbatim}
MM> show statement mp /full
\end{verbatim}
Metamath will respond with
\begin{verbatim}
Statement 17 is located on line 43 of the file
"demo0.mm".
"Define the modus ponens inference rule"
17 mp $a |- Q $.
Its mandatory hypotheses in RPN order are:
  wp $f wff P $.
  wq $f wff Q $.
  min $e |- P $.
  maj $e |- ( P -> Q ) $.
The statement and its hypotheses require the
      variables:  Q P
The variables it contains are:  Q P
\end{verbatim}
The mandatory hypotheses\index{mandatory hypothesis} and their
order\index{RPN order} are
useful to know when you are trying to understand or debug a proof.

Now you are ready to look at what's really inside our proof.  First, here is
how to look at every step in the proof---not just the ones corresponding to an
ordinary formal proof\index{formal proof}, but also the ones that build up the
formulas that appear in each ordinary formal proof step.\index{\texttt{show
proof} command}
\begin{verbatim}
MM> show proof th1 /lemmon /all
\end{verbatim}

This will display the proof on the screen in the following format:
\begin{verbatim}
 1 tt            $f term t
 2 tze           $a term 0
 3 1,2 tpl       $a term ( t + 0 )
 4 tt            $f term t
 5 3,4 weq       $a wff ( t + 0 ) = t
 6 tt            $f term t
 7 tt            $f term t
 8 6,7 weq       $a wff t = t
 9 tt            $f term t
10 9 a2          $a |- ( t + 0 ) = t
11 tt            $f term t
12 tze           $a term 0
13 11,12 tpl     $a term ( t + 0 )
14 tt            $f term t
15 13,14 weq     $a wff ( t + 0 ) = t
16 tt            $f term t
17 tze           $a term 0
18 16,17 tpl     $a term ( t + 0 )
19 tt            $f term t
20 18,19 weq     $a wff ( t + 0 ) = t
21 tt            $f term t
22 tt            $f term t
23 21,22 weq     $a wff t = t
24 20,23 wim     $a wff ( ( t + 0 ) = t -> t = t )
25 tt            $f term t
26 25 a2         $a |- ( t + 0 ) = t
27 tt            $f term t
28 tze           $a term 0
29 27,28 tpl     $a term ( t + 0 )
30 tt            $f term t
31 tt            $f term t
32 29,30,31 a1   $a |- ( ( t + 0 ) = t -> ( ( t + 0 )
                                     = t -> t = t ) )
33 15,24,26,32 mp  $a |- ( ( t + 0 ) = t -> t = t )
34 5,8,10,33 mp  $a |- t = t
\end{verbatim}

The \texttt{/lemmon} command qualifier specifies what is known as a Lemmon-style
display\index{Lemmon-style proof}\index{proof!Lemmon-style}.  Omitting the
\texttt{/lemmon} qualifier results in a tree-style proof (see
p.~\pageref{treeproof} for an example) that is somewhat less explicit but
easier to follow once you get used to it.\index{tree-style
proof}\index{proof!tree-style}

The first number on each line is the step
number of the proof.  Any numbers that follow are step numbers assigned to the
hypotheses of the statement referenced by that step.  Next is the label of
the statement referenced by the step.  The statement type of the statement
referenced comes next, followed by the math symbol\index{math symbol} string
constructed by the proof up to that step.

The last step, 34, contains the statement that is being proved.

Looking at a small piece of the proof, notice that steps 3 and 4 have
established that
\texttt{( t + 0 )} and \texttt{t} are \texttt{term}\,s, and step 5 makes use of steps 3 and
4 to establish that \texttt{( t + 0 ) = t} is a \texttt{wff}.  Let Metamath
itself tell us in detail what is happening in step 5.  Note that the
``target hypothesis'' refers to where step 5 is eventually used, i.e., in step
34.
\begin{verbatim}
MM> show proof th1 /detailed_step 5
Proof step 5:  wp=weq $a wff ( t + 0 ) = t
This step assigns source "weq" ($a) to target "wp"
($f).  The source assertion requires the hypotheses
"tt" ($f, step 3) and "tr" ($f, step 4).  The parent
assertion of the target hypothesis is "mp" ($a,
step 34).
The source assertion before substitution was:
    weq $a wff t = r
The following substitutions were made to the source
assertion:
    Variable  Substituted with
     t         ( t + 0 )
     r         t
The target hypothesis before substitution was:
    wp $f wff P
The following substitution was made to the target
hypothesis:
    Variable  Substituted with
     P         ( t + 0 ) = t
\end{verbatim}

The full proof just shown is useful to understand what is going on in detail.
However, most of the time you will just be interested in
the ``essential'' or logical steps of a proof, i.e.\ those steps
that correspond to an
ordinary formal proof\index{formal proof}.  If you type
\begin{verbatim}
MM> show proof th1 /lemmon /renumber
\end{verbatim}
you will see\label{demoproof}
\begin{verbatim}
1 a2             $a |- ( t + 0 ) = t
2 a2             $a |- ( t + 0 ) = t
3 a1             $a |- ( ( t + 0 ) = t -> ( ( t + 0 )
                                     = t -> t = t ) )
4 2,3 mp         $a |- ( ( t + 0 ) = t -> t = t )
5 1,4 mp         $a |- t = t
\end{verbatim}
Compare this to the formal proof on p.~\pageref{zeroproof} and
notice the resemblance.
By default Metamath
does not show \texttt{\$f}\index{\texttt{\$f}
statement} hypotheses and everything branching off of them in the proof tree
when the proof is displayed; this makes the proof look more like an ordinary
mathematical proof, which does not normally incorporate the explicit
construction of expressions.
This is called the ``essential'' view
(at one time you had to add the
\texttt{/essential} qualifier in the \texttt{show proof}
command to get this view, but this is now the default).
You can could use the \texttt{/all} qualifier in the \texttt{show
proof} command to also show the explicit construction of expressions.
The \texttt{/renumber} qualifier means to renumber
the steps to correspond only to what is displayed.\index{\texttt{show proof}
command}

To exit Metamath, type\index{\texttt{exit} command}
\begin{verbatim}
MM> exit
\end{verbatim}

\subsection{Some Hints for Using the Command Line Interface}

We will conclude this quick introduction to Metamath\index{Metamath} with some
helpful hints on how to navigate your way through the commands.
\index{command line interface (CLI)}

When you type commands into Metamath's CLI, you only have to type as many
characters of a command keyword\index{command keyword} as are needed to make
it unambiguous.  If you type too few characters, Metamath will tell you what
the choices are.  In the case of the \texttt{read} command, only the \texttt{r} is
needed to specify it unambiguously, so you could have typed\index{\texttt{read}
command}
\begin{verbatim}
MM> r demo0.mm
\end{verbatim}
instead of
\begin{verbatim}
MM> read demo0.mm
\end{verbatim}
In our description, we always show the full command words.  When using the
Metamath CLI commands in a command file (to be read with the \texttt{submit}
command)\index{\texttt{submit} command}, it is good practice to use
the unabbreviated command to ensure your instructions will not become ambiguous
if more commands are added to the Metamath program in the future.

The command keywords\index{command
keyword} are not case sensitive; you may type either \texttt{read} or
\texttt{ReAd}.  File names may or may not be case sensitive, depending on your
computer's operating system.  Metamath label\index{label} and math
symbol\index{math symbol} tokens\index{token} are case-sensitive.

The \texttt{help} command\index{\texttt{help} command} will provide you
with a list of topics you can get help on.  You can then type
\texttt{help} {\em topic} to get help on that topic.

If you are uncertain of a command's spelling, just type as many characters
as you remember of the command.  If you have not typed enough characters to
specify it unambiguously, Metamath will tell you what choices you have.

\begin{verbatim}
MM> show s
         ^
?Ambiguous keyword - please specify SETTINGS,
STATEMENT, or SOURCE.
\end{verbatim}

If you don't know what argument to use as part of a command, type a
\texttt{?}\index{\texttt{]}@\texttt{?}\ in command lines}\ at the
argument position.  Metamath will tell you what it expected there.

\begin{verbatim}
MM> show ?
         ^
?Expected SETTINGS, LABELS, STATEMENT, SOURCE, PROOF,
MEMORY, TRACE_BACK, or USAGE.
\end{verbatim}

Finally, you may type just the first word or words of a command followed
by {\em return}.  Metamath will prompt you for the remaining part of the
command, showing you the choices at each step.  For example, instead of
typing \texttt{show statement th1 /full} you could interact in the
following manner:
\begin{verbatim}
MM> show
SETTINGS, LABELS, STATEMENT, SOURCE, PROOF,
MEMORY, TRACE_BACK, or USAGE <SETTINGS>? st
What is the statement label <th1>?
/ or nothing <nothing>? /
TEX, COMMENT_ONLY, or FULL <TEX>? f
/ or nothing <nothing>?
19 th1 $p |- t = t $= ... $.
\end{verbatim}
After each \texttt{?}\ in this mode, you must give Metamath the
information it requests.  Sometimes Metamath gives you a list of choices
with the default choice indicated by brackets \texttt{< > }. Pressing
{\em return} after the \texttt{?}\ will select the default choice.
Answering anything else will override the default.  Note that the
\texttt{/} in command qualifiers is considered a separate
token\index{token} by the parser, and this is why it is asked for
separately.

\section{Your First Proof}\label{frstprf}

Proofs are developed with the aid of the Proof Assistant\index{Proof
Assistant}.  We will now show you how the proof of theorem \texttt{th1}
was built.  So that you can repeat these steps, we will first have the
Proof Assistant erase the proof in Metamath's source buffer\index{source
buffer}, then reconstruct it.  (The source buffer is the place in memory
where Metamath stores the information in the database when it is
\texttt{read}\index{\texttt{read} command} in.  New or modified proofs
are kept in the source buffer until a \texttt{write source}
command\index{\texttt{write source} command} is issued.)  In practice, you
would place a \texttt{?}\index{\texttt{]}@\texttt{?}\ inside proofs}\
between \texttt{\$=}\index{\texttt{\$=} keyword} and
\texttt{\$.}\index{\texttt{\$.}\ keyword}\ in the database to indicate
to Metamath\index{Metamath} that the proof is unknown, and that would be
your starting point.  Whenever the \texttt{verify proof} command encounters
a proof with a \texttt{?}\ in place of a proof step, the statement is
identified as not proved.

When I first started creating Metamath proofs, I would write down
on a piece of paper the complete
formal proof\index{formal proof} as it would appear
in a \texttt{show proof} command\index{\texttt{show proof} command}; see
the display of \texttt{show proof th1 /lemmon /re\-num\-ber} above as an
example.  After you get used to using the Proof Assistant\index{Proof
Assistant} you may get to a point where you can ``see'' the proof in your mind
and let the Proof Assistant guide you in filling in the details, at least for
simpler proofs, but until you gain that experience you may find it very useful
to write down all the details in advance.
Otherwise you may waste a lot of time as you let it take you down a wrong path.
However, others do not find this approach as helpful.
For example, Thomas Brendan Leahy\index{Leahy, Thomas Brendan}
finds that it is more helpful to him to interactively
work backward from a machine-readable statement.
David A. Wheeler\index{Wheeler, David A.}
writes down a general approach, but develops the proof
interactively by switching between
working forwards (from hypotheses and facts likely to be useful) and
backwards (from the goal) until the forwards and backwards approaches meet.
In the end, use whatever approach works for you.

A proof is developed with the Proof Assistant by working backwards, starting
with the theorem\index{theorem} to be proved, and assigning each unknown step
with a theorem or hypothesis until no more unknown steps remain.  The Proof
Assistant will not let you make an assignment unless it can be ``unified''
with the unknown step.  This means that a
substitution\index{substitution!variable}\index{variable substitution} of
variables exists that will make the assignment match the unknown step.  On the
other hand, in the middle of a proof, when working backwards, often more than
one unification\index{unification} (set of substitutions) is possible, since
there is not enough information available at that point to uniquely establish
it.  In this case you can tell Metamath which unification to choose, or you
can continue to assign unknown steps until enough information is available to
make the unification unique.

We will assume you have entered Metamath and read in the database as described
above.  The following dialog shows how the proof was developed.  For more
details on what some of the commands do, refer to Section~\ref{pfcommands}.
\index{\texttt{prove} command}

\begin{verbatim}
MM> prove th1
Entering the Proof Assistant.  Type HELP for help, EXIT
to exit.  You will be working on the proof of statement th1:
  $p |- t = t
Note:  The proof you are starting with is already complete.
MM-PA>
\end{verbatim}

The \verb/MM-PA>/ prompt means we are inside the Proof
Assistant.\index{Proof Assistant} Most of the regular Metamath commands
(\texttt{show statement}, etc.) are still available if you need them.

\begin{verbatim}
MM-PA> delete all
The entire proof was deleted.
\end{verbatim}

We have deleted the whole proof so we can start from scratch.

\begin{verbatim}
MM-PA> show new_proof/lemmon/all
1 ?              $? |- t = t
\end{verbatim}

The \texttt{show new{\char`\_}proof} command\index{\texttt{show
new{\char`\_}proof} command} is like \texttt{show proof} except that we
don't specify a statement; instead, the proof we're working on is
displayed.

\begin{verbatim}
MM-PA> assign 1 mp
To undo the assignment, DELETE STEP 5 and INITIALIZE, UNIFY
if needed.
3   min=?  $? |- $2
4   maj=?  $? |- ( $2 -> t = t )
\end{verbatim}

The \texttt{assign} command\index{\texttt{assign} command} above means
``assign step 1 with the statement whose label is \texttt{mp}.''  Note
that step renumbering will constantly occur as you assign steps in the
middle of a proof; in general all steps from the step you assign until
the end of the proof will get moved up.  In this case, what used to be
step 1 is now step 5, because the (partial) proof now has five steps:
the four hypotheses of the \texttt{mp} statement and the \texttt{mp}
statement itself.  Let's look at all the steps in our partial proof:

\begin{verbatim}
MM-PA> show new_proof/lemmon/all
1 ?              $? wff $2
2 ?              $? wff t = t
3 ?              $? |- $2
4 ?              $? |- ( $2 -> t = t )
5 1,2,3,4 mp     $a |- t = t
\end{verbatim}

The symbol \texttt{\$2} is a temporary variable\index{temporary
variable} that represents a symbol sequence not yet known.  In the final
proof, all temporary variables will be eliminated.  The general format
for a temporary variable is \texttt{\$} followed by an integer.  Note
that \texttt{\$} is not a legal character in a math symbol (see
Section~\ref{dollardollar}, p.~\pageref{dollardollar}), so there will
never be a naming conflict between real symbols and temporary variables.

Unknown steps 1 and 2 are constructions of the two wffs used by the
modus ponens rule.  As you will see at the end of this section, the
Proof Assistant\index{Proof Assistant} can usually figure these steps
out by itself, and we will not have to worry about them.  Therefore from
here on we will display only the ``essential'' hypotheses, i.e.\ those
steps that correspond to traditional formal proofs\index{formal proof}.

\begin{verbatim}
MM-PA> show new_proof/lemmon
3 ?              $? |- $2
4 ?              $? |- ( $2 -> t = t )
5 3,4 mp         $a |- t = t
\end{verbatim}

Unknown steps 3 and 4 are the ones we must focus on.  They correspond to the
minor and major premises of the modus ponens rule.  We will assign them as
follows.  Notice that because of the step renumbering that takes place
after an assignment, it is advantageous to assign unknown steps in reverse
order, because earlier steps will not get renumbered.

\begin{verbatim}
MM-PA> assign 4 mp
To undo the assignment, DELETE STEP 8 and INITIALIZE, UNIFY
if needed.
3   min=?  $? |- $2
6     min=?  $? |- $4
7     maj=?  $? |- ( $4 -> ( $2 -> t = t ) )
\end{verbatim}

We are now going to describe an obscure feature that you will probably
never use but should be aware of.  The Metamath language allows empty
symbol sequences to be substituted for variables, but in most formal
systems this feature is never used.  One of the few examples where is it
used is the MIU-system\index{MIU-system} described in
Appendix~\ref{MIU}.  But such systems are rare, and by default this
feature is turned off in the Proof Assistant.  (It is always allowed for
{\tt verify proof}.)  Let us turn it on and see what
happens.\index{\texttt{set empty{\char`\_}substitution} command}

\begin{verbatim}
MM-PA> set empty_substitution on
Substitutions with empty symbol sequences is now allowed.
\end{verbatim}

With this feature enabled, more unifications will be
ambiguous\index{ambiguous unification}\index{unification!ambiguous} in
the middle of a proof, because
substitution\index{substitution!variable}\index{variable substitution}
of variables with empty symbol sequences will become an additional
possibility.  Let's see what happens when we make our next assignment.

\begin{verbatim}
MM-PA> assign 3 a2
There are 2 possible unifications.  Please select the correct
    one or Q if you want to UNIFY later.
Unify:  |- $6
 with:  |- ( $9 + 0 ) = $9
Unification #1 of 2 (weight = 7):
  Replace "$6" with "( + 0 ) ="
  Replace "$9" with ""
  Accept (A), reject (R), or quit (Q) <A>? r
\end{verbatim}

The first choice presented is the wrong one.  If we had selected it,
temporary variable \texttt{\$6} would have been assigned a truncated
wff, and temporary variable \texttt{\$9} would have been assigned an
empty sequence (which is not allowed in our system).  With this choice,
eventually we would reach a point where we would get stuck because
we would end up with steps impossible to prove.  (You may want to
try it.)  We typed \texttt{r} to reject the choice.

\begin{verbatim}
Unification #2 of 2 (weight = 21):
  Replace "$6" with "( $9 + 0 ) = $9"
  Accept (A), reject (R), or quit (Q) <A>? q
To undo the assignment, DELETE STEP 4 and INITIALIZE, UNIFY
if needed.
 7     min=?  $? |- $8
 8     maj=?  $? |- ( $8 -> ( $6 -> t = t ) )
\end{verbatim}

The second choice is correct, and normally we would type \texttt{a}
to accept it.  But instead we typed \texttt{q} to show what will happen:
it will leave the step with an unknown unification, which can be
seen as follows:

\begin{verbatim}
MM-PA> show new_proof/not_unified
 4   min    $a |- $6
        =a2  = |- ( $9 + 0 ) = $9
\end{verbatim}

Later we can unify this with the \texttt{unify}
\texttt{all/interactive} command.

The important point to remember is that occasionally you will be
presented with several unification choices while entering a proof, when
the program determines that there is not enough information yet to make
an unambiguous choice automatically (and this can happen even with
\texttt{set empty{\char`\_}substitution} turned off).  Usually it is
obvious by inspection which choice is correct, since incorrect ones will
tend to be meaningless fragments of wffs.  In addition, the correct
choice will usually be the first one presented, unlike our example
above.

Enough of this digression.  Let us go back to the default setting.

\begin{verbatim}
MM-PA> set empty_substitution off
The ability to substitute empty expressions for variables
has been turned off.  Note that this may make the Proof
Assistant too restrictive in some cases.
\end{verbatim}

If we delete the proof, start over, and get to the point where
we digressed above, there will no longer be an ambiguous unification.

\begin{verbatim}
MM-PA> assign 3 a2
To undo the assignment, DELETE STEP 4 and INITIALIZE, UNIFY
if needed.
 7     min=?  $? |- $4
 8     maj=?  $? |- ( $4 -> ( ( $5 + 0 ) = $5 -> t = t ) )
\end{verbatim}

Let us look at our proof so far, and continue.

\begin{verbatim}
MM-PA> show new_proof/lemmon
 4 a2            $a |- ( $5 + 0 ) = $5
 7 ?             $? |- $4
 8 ?             $? |- ( $4 -> ( ( $5 + 0 ) = $5 -> t = t ) )
 9 7,8 mp        $a |- ( ( $5 + 0 ) = $5 -> t = t )
10 4,9 mp        $a |- t = t
MM-PA> assign 8 a1
To undo the assignment, DELETE STEP 11 and INITIALIZE, UNIFY
if needed.
 7     min=?  $? |- ( t + 0 ) = t
MM-PA> assign 7 a2
To undo the assignment, DELETE STEP 8 and INITIALIZE, UNIFY
if needed.
MM-PA> show new_proof/lemmon
 4 a2            $a |- ( t + 0 ) = t
 8 a2            $a |- ( t + 0 ) = t
12 a1            $a |- ( ( t + 0 ) = t -> ( ( t + 0 ) = t ->
                                                    t = t ) )
13 8,12 mp       $a |- ( ( t + 0 ) = t -> t = t )
14 4,13 mp       $a |- t = t
\end{verbatim}

Now all temporary variables and unknown steps have been eliminated from the
``essential'' part of the proof.  When this is achieved, the Proof
Assistant\index{Proof Assistant} can usually figure out the rest of the proof
automatically.  (Note that the \texttt{improve} command can occasionally be
useful for filling in essential steps as well, but it only tries to make use
of statements that introduce no new variables in their hypotheses, which is
not the case for \texttt{mp}. Also it will not try to improve steps containing
temporary variables.)  Let's look at the complete proof, then run
the \texttt{improve} command, then look at it again.

\begin{verbatim}
MM-PA> show new_proof/lemmon/all
 1 ?             $? wff ( t + 0 ) = t
 2 ?             $? wff t = t
 3 ?             $? term t
 4 3 a2          $a |- ( t + 0 ) = t
 5 ?             $? wff ( t + 0 ) = t
 6 ?             $? wff ( ( t + 0 ) = t -> t = t )
 7 ?             $? term t
 8 7 a2          $a |- ( t + 0 ) = t
 9 ?             $? term ( t + 0 )
10 ?             $? term t
11 ?             $? term t
12 9,10,11 a1    $a |- ( ( t + 0 ) = t -> ( ( t + 0 ) = t ->
                                                    t = t ) )
13 5,6,8,12 mp   $a |- ( ( t + 0 ) = t -> t = t )
14 1,2,4,13 mp   $a |- t = t
\end{verbatim}

\begin{verbatim}
MM-PA> improve all
A proof of length 1 was found for step 11.
A proof of length 1 was found for step 10.
A proof of length 3 was found for step 9.
A proof of length 1 was found for step 7.
A proof of length 9 was found for step 6.
A proof of length 5 was found for step 5.
A proof of length 1 was found for step 3.
A proof of length 3 was found for step 2.
A proof of length 5 was found for step 1.
Steps 1 and above have been renumbered.
CONGRATULATIONS!  The proof is complete.  Use SAVE
NEW_PROOF to save it.  Note:  The Proof Assistant does
not detect $d violations.  After saving the proof, you
should verify it with VERIFY PROOF.
\end{verbatim}

The \texttt{save new{\char`\_}proof} command\index{\texttt{save
new{\char`\_}proof} command} will save the proof in the database.  Here
we will just display it in a form that can be clipped out of a log file
and inserted manually into the database source file with a text
editor.\index{normal proof}\index{proof!normal}

\begin{verbatim}
MM-PA> show new_proof/normal
---------Clip out the proof below this line:
      tt tze tpl tt weq tt tt weq tt a2 tt tze tpl tt weq
      tt tze tpl tt weq tt tt weq wim tt a2 tt tze tpl tt
      tt a1 mp mp $.
---------The proof of 'th1' to clip out ends above this line.
\end{verbatim}

There is another proof format called ``compressed''\index{compressed
proof}\index{proof!compressed} that you will see in databases.  It is
not important to understand how it is encoded but only to recognize it
when you see it.  Its only purpose is to reduce storage requirements for
large proofs.  A compressed proof can always be converted to a normal
one and vice-versa, and the Metamath \texttt{show proof}
commands\index{\texttt{show proof} command} work equally well with
compressed proofs.  The compressed proof format is described in
Appendix~\ref{compressed}.

\begin{verbatim}
MM-PA> show new_proof/compressed
---------Clip out the proof below this line:
      ( tze tpl weq a2 wim a1 mp ) ABCZADZAADZAEZJJKFLIA
      AGHH $.
---------The proof of 'th1' to clip out ends above this line.
\end{verbatim}

Now we will exit the Proof Assistant.  Since we made changes to the proof,
it will warn us that we have not saved it.  In this case, we don't care.

\begin{verbatim}
MM-PA> exit
Warning:  You have not saved changes to the proof.
Do you want to EXIT anyway (Y, N) <N>? y
Exiting the Proof Assistant.
Type EXIT again to exit Metamath.
\end{verbatim}

The Proof Assistant\index{Proof Assistant} has several other commands
that can help you while creating proofs.  See Section~\ref{pfcommands}
for a list of them.

A command that is often useful is \texttt{minimize{\char`\_}with
*/brief}, which tries to shorten the proof.  It can make the process
more efficient by letting you write a somewhat ``sloppy'' proof then
clean up some of the fine details of optimization for you (although it
can't perform miracles such as restructuring the overall proof).

\section{A Note About Editing a Data\-base File}

Once your source file contains proofs, there are some restrictions on
how you can edit it so that the proofs remain valid.  Pay particular
attention to these rules, since otherwise you can lose a lot of work.
It is a good idea to periodically verify all proofs with \texttt{verify
proof *} to ensure their integrity.

If your file contains only normal (as opposed to compressed) proofs, the
main rule is that you may not change the order of the mandatory
hypotheses\index{mandatory hypothesis} of any statement referenced in a
later proof.  For example, if you swap the order of the major and minor
premise in the modus ponens rule, all proofs making use of that rule
will become incorrect.  The \texttt{show statement}
command\index{\texttt{show statement} command} will show you the
mandatory hypotheses of a statement and their order.

If a statement has a compressed proof, you also must not change the
order of {\em its} mandatory hypotheses.  The compressed proof format
makes use of this information as part of the compression technique.
Note that swapping the names of two variables in a theorem will change
the order of its mandatory hypotheses.

The safest way to edit a statement, say \texttt{mytheorem}, is to
duplicate it then rename the original to \texttt{mytheoremOLD}
throughout the database.  Once the edited version is re-proved, all
statements referencing \texttt{mytheoremOLD} can be updated in the Proof
Assistant using \texttt{minimize{\char`\_}with
mytheorem
/allow{\char`\_}growth}.\index{\texttt{minimize{\char`\_}with} command}
% 3/10/07 Note: line-breaking the above results in duplicate index entries

\chapter{Abstract Mathematics Revealed}\label{fol}

\section{Logic and Set Theory}\label{logicandsettheory}

\begin{quote}
  {\em Set theory can be viewed as a form of exact theology.}
  \flushright\sc  Rudy Rucker\footnote{\cite{Barrow}, p.~31.}\\
\end{quote}\index{Rucker, Rudy}

Despite its seeming complexity, all of standard mathematics, no matter how
deep or abstract, can amazingly enough be derived from a relatively small set
of axioms\index{axiom} or first principles. The development of these axioms is
among the most impressive and important accomplishments of mathematics in the
20th century. Ultimately, these axioms can be broken down into a set of rules
for manipulating symbols that any technically oriented person can follow.

We will not spend much time trying to convey a deep, higher-level
understanding of the meaning of the axioms. This kind of understanding
requires some mathematical sophistication as well as an understanding of the
philosophy underlying the foundations of mathematics and typically develops
over time as you work with mathematics.  Our goal, instead, is to give you the
immediate ability to follow how theorems\index{theorem} are derived from the
axioms and from other theorems.  This will be similar to learning the syntax
of a computer language, which lets you follow the details in a program but
does not necessarily give you the ability to write non-trivial programs on
your own, an ability that comes with practice. For now don't be alarmed by
abstract-sounding names of the axioms; just focus on the rules for
manipulating the symbols, which follow the simple conventions of the
Metamath\index{Metamath} language.

The axioms that underlie all of standard mathematics consist of axioms of logic
and axioms of set theory. The axioms of logic are divided into two
subcategories, propositional calculus\index{propositional calculus} (sometimes
called sentential logic\index{sentential logic}) and predicate calculus
(sometimes called first-order logic\index{first-order logic}\index{quantifier
theory}\index{predicate calculus} or quantifier theory).  Propositional
calculus is a prerequisite for predicate calculus, and predicate calculus is a
prerequisite for set theory.  The version of set theory most commonly used is
Zermelo--Fraenkel set theory\index{Zermelo--Fraenkel set theory}\index{set theory}
with the axiom of choice,
often abbreviated as ZFC\index{ZFC}.

Here in a nutshell is what the axioms are all about in an informal way. The
connection between this description and symbols we will show you won't be
immediately apparent and in principle needn't ever be.  Our description just
tries to summarize what mathematicians think about when they work with the
axioms.

Logic is a set of rules that allow us determine truths given other truths.
Put another way,
logic is more or less the translation of what we would consider common sense
into a rigorous set of axioms.\index{axioms of logic}  Suppose $\varphi$,
$\psi$, and $\chi$ (the Greek letters phi, psi, and chi) represent statements
that are either true or false, and $x$ is a variable\index{variable!in predicate
calculus} ranging over some group of mathematical objects (sets, integers,
real numbers, etc.). In mathematics, a ``statement'' really means a formula,
and $\psi$ could be for example ``$x = 2$.''
Propositional calculus\index{propositional calculus}
allows us to use variables that are either true or false
and make deductions such as
``if $\varphi$ implies $\psi$ and $\psi$ implies $\chi$, then $\varphi$
implies $\chi$.''
Predicate calculus\index{predicate calculus}
extends propositional calculus by also allowing us
to discuss statements about objects (not just true and false values), including
statements about ``all'' or ``at least one'' object.
For example, predicate calculus allows to say,
``if $\varphi$ is true for all $x$, then $\varphi$ is true for some $x$.''
The logic used in \texttt{set.mm} is standard classical logic
(as opposed to other logic systems like intuitionistic logic).

Set theory\index{set theory} has to do with the manipulation of objects and
collections of objects, specifically the abstract, imaginary objects that
mathematics deals with, such as numbers. Everything that is claimed to exist
in mathematics is considered to be a set.  A set called the empty
set\index{empty set} contains nothing.  We represent the empty set by
$\varnothing$.  Many sets can be built up from the empty set.  There is a set
represented by $\{\varnothing\}$ that contains the empty set, another set
represented by $\{\varnothing,\{\varnothing\}\}$ that contains this set as
well as the empty set, another set represented by $\{\{\varnothing\}\}$ that
contains just the set that contains the empty set, and so on ad infinitum. All
mathematical objects, no matter how complex, are defined as being identical to
certain sets: the integer\index{integer} 0 is defined as the empty set, the
integer 1 is defined as $\{\varnothing\}$, the integer 2 is defined as
$\{\varnothing,\{\varnothing\}\}$.  (How these definitions were chosen doesn't
matter now, but the idea behind it is that these sets have the properties we
expect of integers once suitable operations are defined.)  Mathematical
operations, such as addition, are defined in terms of operations on
sets---their union\index{set union}, intersection\index{set intersection}, and
so on---operations you may have used in elementary school when you worked
with groups of apples and oranges.

With a leap of faith, the axioms also postulate the existence of infinite
sets\index{infinite set}, such as the set of all non-negative integers ($0, 1,
2,\ldots$, also called ``natural numbers''\index{natural number}).  This set
can't be represented with the brace notation\index{brace notation} we just
showed you, but requires a more complicated notation called ``class
abstraction.''\index{class abstraction}\index{abstraction class}  For
example, the infinite set $\{ x |
\mbox{``$x$ is a natural number''} \} $ means the ``set of all objects $x$
such that $x$ is a natural number'' i.e.\ the set of natural numbers; here,
``$x$ is a natural number'' is a rather complicated formula when broken down
into the primitive symbols.\label{expandom}\footnote{The statement ``$x$ is a
natural number'' is formally expressed as ``$x \in \omega$,'' where $\in$
(stylized epsilon) means ``is in'' or ``is an element of'' and $\omega$
(omega) means ``the set of natural numbers.''  When ``$x\in\omega$'' is
completely expanded in terms of the primitive symbols of set theory, the
result is  $\lnot$ $($ $\lnot$ $($ $\forall$ $z$ $($ $\lnot$ $\forall$ $w$ $($
$z$ $\in$ $w$ $\rightarrow$ $\lnot$ $w$ $\in$ $x$ $)$ $\rightarrow$ $z$ $\in$
$x$ $)$ $\rightarrow$ $($ $\forall$ $z$ $($ $\lnot$ $($ $\forall$ $w$ $($ $w$
$\in$ $z$ $\rightarrow$ $w$ $\in$ $x$ $)$ $\rightarrow$ $\forall$ $w$ $\lnot$
$w$ $\in$ $z$ $)$ $\rightarrow$ $\lnot$ $\forall$ $w$ $($ $w$ $\in$ $z$
$\rightarrow$ $\lnot$ $\forall$ $v$ $($ $v$ $\in$ $z$ $\rightarrow$ $\lnot$
$v$ $\in$ $w$ $)$ $)$ $)$ $\rightarrow$ $\lnot$ $\forall$ $z$ $\forall$ $w$
$($ $\lnot$ $($ $z$ $\in$ $x$ $\rightarrow$ $\lnot$ $w$ $\in$ $x$ $)$
$\rightarrow$ $($ $\lnot$ $z$ $\in$ $w$ $\rightarrow$ $($ $\lnot$ $z$ $=$ $w$
$\rightarrow$ $w$ $\in$ $z$ $)$ $)$ $)$ $)$ $)$ $\rightarrow$ $\lnot$
$\forall$ $y$ $($ $\lnot$ $($ $\lnot$ $($ $\forall$ $z$ $($ $\lnot$ $\forall$
$w$ $($ $z$ $\in$ $w$ $\rightarrow$ $\lnot$ $w$ $\in$ $y$ $)$ $\rightarrow$
$z$ $\in$ $y$ $)$ $\rightarrow$ $($ $\forall$ $z$ $($ $\lnot$ $($ $\forall$
$w$ $($ $w$ $\in$ $z$ $\rightarrow$ $w$ $\in$ $y$ $)$ $\rightarrow$ $\forall$
$w$ $\lnot$ $w$ $\in$ $z$ $)$ $\rightarrow$ $\lnot$ $\forall$ $w$ $($ $w$
$\in$ $z$ $\rightarrow$ $\lnot$ $\forall$ $v$ $($ $v$ $\in$ $z$ $\rightarrow$
$\lnot$ $v$ $\in$ $w$ $)$ $)$ $)$ $\rightarrow$ $\lnot$ $\forall$ $z$
$\forall$ $w$ $($ $\lnot$ $($ $z$ $\in$ $y$ $\rightarrow$ $\lnot$ $w$ $\in$
$y$ $)$ $\rightarrow$ $($ $\lnot$ $z$ $\in$ $w$ $\rightarrow$ $($ $\lnot$ $z$
$=$ $w$ $\rightarrow$ $w$ $\in$ $z$ $)$ $)$ $)$ $)$ $\rightarrow$ $($
$\forall$ $z$ $\lnot$ $z$ $\in$ $y$ $\rightarrow$ $\lnot$ $\forall$ $w$ $($
$\lnot$ $($ $w$ $\in$ $y$ $\rightarrow$ $\lnot$ $\forall$ $z$ $($ $w$ $\in$
$z$ $\rightarrow$ $\lnot$ $z$ $\in$ $y$ $)$ $)$ $\rightarrow$ $\lnot$ $($
$\lnot$ $\forall$ $z$ $($ $w$ $\in$ $z$ $\rightarrow$ $\lnot$ $z$ $\in$ $y$
$)$ $\rightarrow$ $w$ $\in$ $y$ $)$ $)$ $)$ $)$ $\rightarrow$ $x$ $\in$ $y$
$)$ $)$ $)$. Section~\ref{hierarchy} shows the hierarchy of definitions that
leads up to this expression.}\index{stylized epsilon ($\in$)}\index{omega
($\omega$)}  Actually, the primitive symbols don't even include the brace
notation.  The brace notation is a high-level definition, which you can find in
Section~\ref{hierarchy}.

Interestingly, the arithmetic of integers\index{integer} and
rationals\index{rational number} can be developed without appealing to the
existence of an infinite set, whereas the arithmetic of real
numbers\index{real number} requires it.

Each variable\index{variable!in set theory} in the axioms of set theory
represents an arbitrary set, and the axioms specify the legal kinds of things
you can do with these variables at a very primitive level.

Now, you may think that numbers and arithmetic are a lot more intuitive and
fundamental than sets and therefore should be the foundation of mathematics.
What is really the case is that you've dealt with numbers all your life and
are comfortable with a few rules for manipulating them such as addition and
multiplication.  Those rules only cover a small portion of what can be done
with numbers and only a very tiny fraction of the rest of mathematics.  If you
look at any elementary book on number theory, you will quickly become lost if
these are the only rules that you know.  Even though such books may present a
list of ``axioms''\index{axiom} for arithmetic, the ability to use the axioms
and to understand proofs of theorems\index{theorem} (facts) about numbers
requires an implicit mathematical talent that frustrates many people
from studying abstract mathematics.  The kind of mathematics that most people
know limits them to the practical, everyday usage of blindly manipulating
numbers and formulas, without any understanding of why those rules are correct
nor any ability to go any further.  For example, do you know why multiplying
two negative numbers yields a positive number?  Starting with set theory, you
will also start off blindly manipulating symbols according to the rules we give
you, but with the advantage that these rules will allow you, in principle, to
access {\em all} of mathematics, not just a tiny part of it.

Of course, concrete examples are often helpful in the learning process. For
example, you can verify that $2\cdot 3=3 \cdot 2$ by actually grouping
objects and can easily ``see'' how it generalizes to $x\cdot y = y\cdot x$,
even though you might not be able to rigorously prove it.  Similarly, in set
theory it can be helpful to understand how the axioms of set theory apply to
(and are correct for) small finite collections of objects.  You should be aware
that in set theory intuition can be misleading for infinite collections, and
rigorous proofs become more important.  For example, while $x\cdot y = y\cdot
x$ is correct for finite ordinals (which are the natural numbers), it is not
usually true for infinite ordinals.

\section{The Axioms for All of Mathematics}

In this section\index{axioms for mathematics}, we will show you the axioms
for all of standard mathematics (i.e.\ logic and set theory) as they are
traditionally presented.  The traditional presentation is useful for someone
with the mathematical experience needed to correctly manipulate high-level
abstract concepts.  For someone without this talent, knowing how to actually
make use of these axioms can be difficult.  The purpose of this section is to
allow you to see how the version of the axioms used in the standard
Metamath\index{Metamath} database \texttt{set.mm}\index{set
theory database (\texttt{set.mm})} relates to  the typical version
in textbooks, and also to give you an informal feel for them.

\subsection{Propositional Calculus}

Propositional calculus\index{propositional calculus} concerns itself with
statements that can be interpreted as either true or false.  Some examples of
statements (outside of mathematics) that are either true or false are ``It is
raining today'' and ``The United States has a female president.'' In
mathematics, as we mentioned, statements are really formulas.

In propositional calculus, we don't care what the statements are.  We also
treat a logical combination of statements, such as ``It is raining today and
the United States has a female president,'' no differently from a single
statement.  Statements and their combinations are called well-formed formulas
(wffs)\index{well-formed formula (wff)}.  We define wffs only in terms of
other wffs and don't define what a ``starting'' wff is.  As is common practice
in the literature, we use Greek letters to represent wffs.

Specifically, suppose $\varphi$ and $\psi$ are wffs.  Then the combinations
$\varphi\rightarrow\psi$ (``$\varphi$ implies $\psi$,'' also read ``if
$\varphi$ then $\psi$'')\index{implication ($\rightarrow$)} and $\lnot\varphi$
(``not $\varphi$'')\index{negation ($\lnot$)} are also wffs.

The three axioms of propositional calculus\index{axioms of propositional
calculus} are all wffs of the following form:\footnote{A remarkable result of
C.~A.~Meredith\index{Meredith, C. A.} squeezes these three axioms into the
single axiom $((((\varphi\rightarrow \psi)\rightarrow(\neg \chi\rightarrow\neg
\theta))\rightarrow \chi )\rightarrow \tau)\rightarrow((\tau\rightarrow
\varphi)\rightarrow(\theta\rightarrow \varphi))$ \cite{CAMeredith},
which is believed to be the shortest possible.}
\begin{center}
     $\varphi\rightarrow(\psi\rightarrow \varphi)$\\

     $(\varphi\rightarrow (\psi\rightarrow \chi))\rightarrow
((\varphi\rightarrow  \psi)\rightarrow (\varphi\rightarrow \chi))$\\

     $(\neg \varphi\rightarrow \neg\psi)\rightarrow (\psi\rightarrow
\varphi)$
\end{center}

These three axioms are widely used.
They are attributed to Jan {\L}ukasiewicz
(pronounced woo-kah-SHAY-vitch) and was popularized by Alonzo Church,
who called it system P2. (Thanks to Ted Ulrich for this information.)

There are an infinite number of axioms, one for each possible
wff\index{well-formed formula (wff)} of the above form.  (For this reason,
axioms such as the above are often called ``axiom schemes.''\index{axiom
scheme})  Each Greek letter in the axioms may be substituted with a more
complex wff to result in another axiom.  For example, substituting
$\neg(\varphi\rightarrow\chi)$ for $\varphi$ in the first axiom yields
$\neg(\varphi\rightarrow\chi)\rightarrow(\psi\rightarrow
\neg(\varphi\rightarrow\chi))$, which is still an axiom.

To deduce new true statements (theorems\index{theorem}) from the axioms, a
rule\index{rule} called ``modus ponens''\index{modus ponens} is used.  This
rule states that if the wff $\varphi$ is an axiom or a theorem, and the wff
$\varphi\rightarrow\psi$ is an axiom or a theorem, then the wff $\psi$ is also
a theorem\index{theorem}.

As a non-mathematical example of modus ponens, suppose we have proved (or
taken as an axiom) ``Bob is a man'' and separately have proved (or taken as
an axiom) ``If Bob is a man, then Bob is a human.''  Using the rule of modus
ponens, we can logically deduce, ``Bob is a human.''

From Metamath's\index{Metamath} point of view, the axioms and the rule of
modus ponens just define a mechanical means for deducing new true statements
from existing true statements, and that is the complete content of
propositional calculus as far as Metamath is concerned.  You can read a logic
textbook to gain a better understanding of their meaning, or you can just let
their meaning slowly become apparent to you after you use them for a while.

It is actually rather easy to check to see if a formula is a theorem of
propositional calculus.  Theorems of propositional calculus are also called
``tautologies.''\index{tautology}  The technique to check whether a formula is
a tautology is called the ``truth table method,''\index{truth table} and it
works like this.  A wff $\varphi\rightarrow\psi$ is false whenever $\varphi$ is true
and $\psi$ is false.  Otherwise it is true.  A wff $\lnot\varphi$ is false
whenever $\varphi$ is true and false otherwise. To verify a tautology such as
$\varphi\rightarrow(\psi\rightarrow \varphi)$, you break it down into sub-wffs and
construct a truth table that accounts for all possible combinations of true
and false assigned to the wff metavariables:
\begin{center}\begin{tabular}{|c|c|c|c|}\hline
\mbox{$\varphi$} & \mbox{$\psi$} & \mbox{$\psi\rightarrow\varphi$}
    & \mbox{$\varphi\rightarrow(\psi\rightarrow \varphi)$} \\ \hline \hline
              T   &  T    &      T       &        T    \\ \hline
              T   &  F    &      T       &        T    \\ \hline
              F   &  T    &      F       &        T    \\ \hline
              F   &  F    &      T       &        T    \\ \hline
\end{tabular}\end{center}
If all entries in the last column are true, the formula is a tautology.

Now, the truth table method doesn't tell you how to prove the tautology from
the axioms, but only that a proof exists.  Finding an actual proof (especially
one that is short and elegant) can be challenging.  Methods do exist for
automatically generating proofs in propositional calculus, but the proofs that
result can sometimes be very long.  In the Metamath \texttt{set.mm}\index{set
theory database (\texttt{set.mm})} database, most
or all proofs were created manually.

Section \ref{metadefprop} discusses various definitions
that make propositional calculus easier to use.
For example, we define:

\begin{itemize}
\item $\varphi \vee \psi$
  is true if either $\varphi$ or $\psi$ (or both) are true
  (this is disjunction\index{disjunction ($\vee$)}
  aka logical {\sc or}\index{logical {\sc or} ($\vee$)}).

\item $\varphi \wedge \psi$
  is true if both $\varphi$ and $\psi$ are true
  (this is conjunction\index{conjunction ($\wedge$)}
  aka logical {\sc and}\index{logical {\sc and} ($\wedge$)}).

\item $\varphi \leftrightarrow \psi$
  is true if $\varphi$ and $\psi$ have the same value, that is,
  they are both true or both false
  (this is the biconditional\index{biconditional ($\leftrightarrow$)}).
\end{itemize}

\subsection{Predicate Calculus}

Predicate calculus\index{predicate calculus} introduces the concept of
``individual variables,''\index{variable!in predicate calculus}\index{individual
variable} which
we will usually just call ``variables.''
These variables can represent something other than true or false (wffs),
and will always represent sets when we get to set theory.  There are also
three new symbols $\forall$\index{universal quantifier ($\forall$)},
$=$\index{equality ($=$)}, and $\in$\index{stylized epsilon ($\in$)},
read ``for all,'' ``equals,'' and ``is an element of''
respectively.  We will represent variables with the letters $x$, $y$, $z$, and
$w$, as is common practice in the literature.
For example, $\forall x \varphi$ means ``for all possible values of
$x$, $\varphi$ is true.''

In predicate calculus, we extend the definition of a wff\index{well-formed
formula (wff)}.  If $\varphi$ is a wff and $x$ and $y$ are variables, then
$\forall x \, \varphi$, $x=y$, and $x\in y$ are wffs. Note that these three new
types of wffs can be considered ``starting'' wffs from which we can build
other wffs with $\rightarrow$ and $\neg$ .  The concept of a starting wff was
absent in propositional calculus.  But starting wff or not, all we are really
concerned with is whether our wffs are correctly constructed according to
these mechanical rules.

A quick aside:
To prevent confusion, it might be best at this point to think of the variables
of Metamath\index{Metamath} as ``metavariables,''\index{metavariable} because
they are not quite the same as the variables we are introducing here.  A
(meta)variable in Metamath can be a wff or an individual variable, as well
as many other things; in general, it represents a kind of place holder for an
unspecified sequence of math symbols\index{math symbol}.

Unlike propositional calculus, no decision procedure\index{decision procedure}
analogous to the truth table method exists (nor theoretically can exist) that
will definitely determine whether a formula is a theorem of predicate
calculus.  Much of the work in the field of automated theorem
proving\index{automated theorem proving} has been dedicated to coming up with
clever heuristics for proving theorems of predicate calculus, but they can
never be guaranteed to work always.

Section \ref{metadefpred} discusses various definitions
that make predicate calculus easier to use.
For example, we define
$\exists x \varphi$ to mean
``there exists at least one possible value of $x$ where $\varphi$ is true.''

We now turn to looking at how predicate calculus can be formally
represented.

\subsubsection{Common Axioms}

There is a new rule of inference in predicate calculus:  if $\varphi$ is
an axiom or a theorem, then $\forall x \,\varphi$ is also a
theorem\index{theorem}.  This is called the rule of
``generalization.''\index{rule of generalization}
This is easily represented in Metamath.

In standard texts of logic, there are often two axioms of predicate
calculus\index{axioms of predicate calculus}:
\begin{center}
  $\forall x \,\varphi ( x ) \rightarrow \varphi ( y )$,
      where ``$y$ is properly substituted for $x$.''\\
  $\forall x ( \varphi \rightarrow \psi )\rightarrow ( \varphi \rightarrow
    \forall x\, \psi )$,
    where ``$x$ is not free in $\varphi$.''
\end{center}

Now at first glance, this seems simple:  just two axioms.  However,
conditional clauses are attached to each axiom describing requirements that
may seem puzzling to you.  In addition, the first axiom puts a variable symbol
in parentheses after each wff, seemingly violating our definition of a
wff\index{well-formed formula (wff)}; this is just an informal way of
referring to some arbitrary variable that may occur in the wff.  The
conditional clauses do, of course, have a precise meaning, but as it turns out
the precise meaning is somewhat complicated and awkward to formalize in a
way that a computer can handle easily.  Unlike propositional calculus, a
certain amount of mathematical sophistication and practice is needed to be
able to easily grasp and manipulate these concepts correctly.

Predicate calculus may be presented with or without axioms for
equality\index{axioms of equality}\index{equality ($=$)}. We will require the
axioms of equality as a prerequisite for the version of set theory we will
use.  The axioms for equality, when included, are often represented using these
two axioms:
\begin{center}
$x=x$\\ \ \\
$x=y\rightarrow (\varphi(x,x)\rightarrow\varphi(x,y))$ where ``$\varphi(x,y)$
   arises from $\varphi(x,x)$ by replacing some, but not necessarily all,
   free\index{free variable}
   occurrences of $x$ by $y$,\\ provided that $y$ is free for $x$
   in $\varphi(x,x)$.'' \end{center}
% (Mendelson p. 95)
The first equality axiom is simple, but again,
the condition on the second one is
somewhat awkward to implement on a computer.

\subsubsection{Tarski System S2}

Of course, we are not the first to notice the complications of these
predicate calculus axioms when being rigorous.

Well-known logician Alfred Tarski published in 1965
a system he called system S2\cite[p.~77]{Tarski1965}.
Tarski's system is \textit{exactly equivalent} to the traditional textbook
formalization, but (by clever use of equality axioms) it eliminates the
latter's primitive notions of ``proper substitution'' and ``free variable,''
replacing them with direct substitution and the notion of a variable
not occurring in a formula (which we express with distinct variable
constraints).

In advocating his system, Tarski wrote, ``The relatively complicated
character of [free variables and proper substitution] is a source
of certain inconveniences of both practical and theoretical nature;
this is clearly experienced both in teaching an elementary course of
mathematical logic and in formalizing the syntax of predicate logic for
some theoretical purposes''\cite[p.~61]{Tarski1965}\index{Tarski, Alfred}.

\subsubsection{Developing a Metamath Representation}

The standard textbook axioms of predicate calculus are somewhat
cumbersome to implement on a computer because of the complex notions of
``free variable''\index{free variable} and ``proper
substitution.''\index{proper substitution}\index{substitution!proper}
While it is possible to use the Metamath\index{Metamath} language to
implement these concepts, we have chosen not to implement them
as primitive constructs in the
\texttt{set.mm} set theory database.  Instead, we have eliminated them
within the axioms
by carefully crafting the axioms so as to avoid them,
building on Tarski's system S2.  This makes it
easy for a beginner to follow the steps in a proof without knowing any
advanced concepts other than the simple concept of
replacing\index{substitution!variable}\index{variable substitution}
variables with expressions.

In order to develop the concepts of free variable and proper
substitution from the axioms, we use an additional
Metamath statement type called ``disjoint variable
restriction''\index{disjoint variables} that we have not encountered
before.  In the context of the axioms, the statement \texttt{\$d} $ x\,
y$\index{\texttt{\$d} statement} simply means that $x$ and $y$ must be
distinct\index{distinct variables}, i.e.\ they may not be simultaneously
substituted\index{substitution!variable}\index{variable substitution}
with the same variable.  The statement \texttt{\$d} $ x\, \varphi$ means
variable $x$ must not occur in wff $\varphi$.  For the precise
definition of \texttt{\$d}, see Section~\ref{dollard}.

\subsubsection{Metamath representation}

The Metamath axiom system for predicate calculus
defined in set.mm uses Tarski's system S2.
As noted above, this has a different representation
than the traditional textbook formalization,
but it is \textit{exactly equivalent} to the textbook formalization,
and it is \textit{much} easier to work with.
This is reproduced as system S3 in Section 6 of
Megill's formalization \cite{Megill}\index{Megill, Norman}.

There is one exception, Tarski's axiom of existence,
which we label as axiom ax-6.
In the case of ax-6, Tarski's version is weaker because it includes a
distinct variable proviso. If we wish, we can also weaken our version
in this way and still have a metalogically complete system. Theorem
ax6 shows this by deriving, in the presence of the other axioms, our
ax-6 from Tarski's weaker version ax6v. However, we chose the stronger
version for our system because it is simpler to state and easier to use.

Tarski's system was designed for proving specific theorems rather than
more general theorem schemes. However, theorem schemes are much more
efficient than specific theorems for building a body of mathematical
knowledge, since they can be reused with different instances as
needed. While Tarski does derive some theorem schemes from his axioms,
their proofs require concepts that are ``outside'' of the system, such as
induction on formula length. The verification of such proofs is difficult
to automate in a proof verifier. (Specifically, Tarski treats the formulas
of his system as set-theoretical objects. In order to verify the proofs
of his theorem schemes, a proof verifier would need a significant amount
of set theory built into it.)

The Metamath axiom system for predicate calculus extends
Tarski's system to eliminate this difficulty. The additional
``auxilliary'' axiom
schemes (as we will call them in this section; see below) endow Tarski's
system with a nice property we call
metalogical completeness \cite[Remark 9.6]{Megill}\index{Megill, Norman}.
As a result, we can prove any theorem scheme
expressable in the ``simple metalogic'' of Tarski's system by using
only Metamath's direct substitution rule applied to the axiom system
(and no other metalogical or set-theoretical notions ``outside'' of the
system). Simple metalogic consists of schemes containing wff metavariables
(with no arguments) and/or set (also called ``individual'') metavariables,
accompanied by optional provisos each stating that two specified set
metavariables must be distinct or that a specified set metavariable may
not occur in a specified wff metavariable. Metamath's logic and set theory
axiom and rule schemes are all examples of simple metalogic. The schemes
of traditional predicate calculus with equality are examples which are
not simple metalogic, because they use wff metavariables with arguments
and have ``free for'' and ``not free in'' side conditions.

A rigorous justification for this system, using an older but
exactly equivalent set of axioms, can be
found in \cite{Megill}\index{Megill, Norman}.

This allows us to
take a different approach in the Metamath\index{Metamath} database
\texttt{set.mm}\index{set theory database (\texttt{set.mm})}.  We do not
directly use the primitive notions of ``free variable''\index{free variable}
and ``proper substitution''\index{proper
substitution}\index{substitution!proper} at all as primitive constructs.
Instead, we use a set
of axioms that are almost as simple to manipulate as those of
propositional calculus.  Our axiom system avoids complex primitive
notions by effectively embedding the complexity into the axioms
themselves.  As a result, we will end up with a larger number of axioms,
but they are ideally suited for a computer language such as Metamath.
(Section~\ref{metaaxioms} shows these axioms.)

We will not elaborate further
on the ``free variable'' and ``proper substitution''
concepts here.  You may consult
\cite[ch.\ 3--4]{Hamilton}\index{Hamilton, Alan G.} (as well as
many other books) for a precise explanation
of these concepts.  If you intend to do serious mathematical work, it is wise
to become familiar with the traditional textbook approach; even though the
concepts embedded in their axioms require a higher level of sophistication,
they can be more practical to deal with on an everyday, informal basis.  Even
if you are just developing Metamath proofs, familiarity with the traditional
approach can help you arrive at a proof outline much faster, which you can
then convert to the detail required by Metamath.

We do develop proper substitution rules later on, but in set.mm
they are defined as derived constructs; they are not primitives.

You should also note that our system of predicate calculus is specifically
tailored for set theory; thus there are only two specific predicates $=$ and
$\in$ and no functions\index{function!in predicate calculus}
or constants\index{constant!in predicate calculus} unlike more general systems.
We later add these.

\subsection{Set Theory}

Traditional Zermelo--Fraenkel set theory\index{Zermelo--Fraenkel set
theory}\index{set theory} with the Axiom of Choice
has 10 axioms, which can be expressed in the
language of predicate calculus.  In this section, we will list only the
names and brief English descriptions of these axioms, since we will give
you the precise formulas used by the Metamath\index{Metamath} set theory
database \texttt{set.mm} later on.

In the descriptions of the axioms, we assume that $x$, $y$, $z$, $w$, and $v$
represent sets.  These are the same as the variables\index{variable!in set
theory} in our predicate calculus system above, except that now we informally
think of the variables as ranging over sets.  Note that the terms
``object,''\index{object} ``set,''\index{set} ``element,''\index{element}
``collection,''\index{collection} and ``family''\index{family} are synonymous,
as are ``is an element of,'' ``is a member of,''\index{member} ``is contained
in,'' and ``belongs to.''  The different terms are used for convenience; for
example, ``a collection of sets'' is less confusing than ``a set of sets.''
A set $x$ is said to be a ``subset''\index{subset} of $y$ if every element of
$x$ is also an element of $y$; we also say $x$ is ``included in''
$y$.

The axioms are very general and apply to almost any conceivable mathematical
object, and this level of abstraction can be overwhelming at first.  To gain an
intuitive feel, it can be helpful to draw a picture illustrating the concept;
for example, a circle containing dots could represent a collection of sets,
and a smaller circle drawn inside the circle could represent a subset.
Overlapping circles can illustrate intersection and union.  Circles that
illustrate the concepts of set theory are frequently used in elementary
textbooks and are called Venn diagrams\index{Venn diagram}.\index{axioms of
set theory}

1. Axiom of Extensionality:  Two sets are identical if they contain the same
   elements.\index{Axiom of Extensionality}

2. Axiom of Pairing:  The set $\{ x , y \}$ exists.\index{Axiom of Pairing}

3. Axiom of Power Sets:  The power set of a set (the collection of all of
   its subsets) exists.  For example, the power set of $\{x,y\}$ is
   $\{\varnothing,\{x\},\{y\},\{x,y\}\}$ and it exists.\index{Axiom
of Power Sets}

4. Axiom of the Null Set:  The empty set $\varnothing$ exists.\index{Axiom of
the Null Set}

5. Axiom of Union:  The union of a set (the set containing the elements of
   its members) exists.  For example, the union of $\{\{x,y\},\{z\}\}$ is
 $\{x,y,z\}$ and
   it exists.\index{Axiom of Union}

6. Axiom of Regularity:  Roughly, no set can contain itself, nor can there
   be membership ``loops,'' such as a set being an
   element of one of its members.\index{Axiom of Regularity}

7. Axiom of Infinity:  An infinite set exists.  An example of an infinite
   set is the set of all
   integers.\index{Axiom of Infinity}

8. Axiom of Separation:  The set exists that is obtained by restricting $x$
   with some property.  For example, if the set of all integers exists,
   then the set of all even integers exists.\index{Axiom of Separation}

9. Axiom of Replacement:  The range of a function whose domain is restricted
   to the elements of a set $x$, is also a set.  For example, there
   is a function
   from integers (the function's domain) to their squares (its
   range).  If we
   restrict the domain to even integers, its range will become the set of
   squares of even integers, so this axiom asserts that the set of
    squares of even numbers exists.  Technical note:  In general, the
   ``function'' need not be a set but can be a proper class.
   \index{Axiom of Replacement}

10. Axiom of Choice:  Let $x$ be a set whose members are pairwise
  disjoint\index{disjoint sets} (i.e,
  whose members contain no elements in common).  Then there exists another
  set containing one element from each member of $x$.  For
  example, if $x$ is
  $\{\{y,z\},\{w,v\}\}$, where $y$, $z$, $w$, and $v$ are
  different sets, then a set such as $\{z,w\}$
  exists (but the axiom doesn't tell
  us which one).  (Actually the Axiom
  of Choice is redundant if the set $x$, as in this example, has a finite
  number of elements.)\index{Axiom of Choice}

The Axiom of Choice is usually considered an extension of ZF set theory rather
than a proper part of it.  It is sometimes considered philosophically
controversial because it specifies the existence of a set without specifying
what the set is. Constructive logics, including intuitionistic logic,
do not accept the axiom of choice.
Since there is some lingering controversy, we often prefer proofs that do
not use the axiom of choice (where there is a known alternative), and
in some cases we will use weaker axioms than the full axiom of choice.
That said, the axiom of choice is a powerful and widely-accepted tool,
so we do use it when needed.
ZF set theory that includes the Axiom of Choice is
called Zermelo--Fraenkel set theory with choice (ZFC\index{ZFC set theory}).

When expressed symbolically, the Axiom of Separation and the Axiom of
Replacement contain wff symbols and therefore each represent infinitely many
axioms, one for each possible wff. For this reason, they are often called
axiom schemes\index{axiom scheme}\index{well-formed formula (wff)}.

It turns out that the Axiom of the Null Set, the Axiom of Pairing, and the
Axiom of Separation can be derived from the other axioms and are therefore
unnecessary, although they tend to be included in standard texts for various
reasons (historical, philosophical, and possibly because some authors may not
know this).  In the Metamath\index{Metamath} set theory database, these
redundant axioms are derived from the other ones instead of truly
being considered axioms.
This is in keeping with our general goal of minimizing the number of
axioms we must depend on.

\subsection{Other Axioms}

Above we qualified the phrase "all of mathematics" with "essentially."
The main important missing piece is the ability to do category theory,
which requires huge sets (inaccessible cardinals) larger than those
postulated by the ZFC axioms. The Tarski--Grothendieck Axiom postulates
the existence of such sets.
Note that this is the same axiom used by Mizar for supporting
category theory.
The Tarski--Grothendieck axiom
can be viewed as a very strong replacement of the Axiom of Infinity,
the Axiom of Choice, and the Axiom of Power Sets.
The \texttt{set.mm} database includes this axiom; see the database
for details about it.
Again, we only use this axiom when we need to.
You are only likely to encounter or use this axiom if you are doing
category theory, since its use is highly specialized,
so we will not list the Tarsky-Grothendieck axiom
in the short list of axioms below.

Can there be even more axioms?
Of course.
G\"{o}del showed that no finite set of axioms or axiom schemes can completely
describe any consistent theory strong enough to include arithmetic.
But practically speaking, the ones above are the accepted foundation that
almost all mathematicians explicitly or implicitly base their work on.

\section{The Axioms in the Metamath Language}\label{metaaxioms}

Here we list the axioms as they appear in
\texttt{set.mm}\index{set theory database (\texttt{set.mm})} so you can
look them up there easily.  Incidentally, the \texttt{show statement
/tex} command\index{\texttt{show statement} command} was used to
typeset them.

%macros from show statement /tex
\newbox\mlinebox
\newbox\mtrialbox
\newbox\startprefix  % Prefix for first line of a formula
\newbox\contprefix  % Prefix for continuation line of a formula
\def\startm{  % Initialize formula line
  \setbox\mlinebox=\hbox{\unhcopy\startprefix}
}
\def\m#1{  % Add a symbol to the formula
  \setbox\mtrialbox=\hbox{\unhcopy\mlinebox $\,#1$}
  \ifdim\wd\mtrialbox>\hsize
    \box\mlinebox
    \setbox\mlinebox=\hbox{\unhcopy\contprefix $\,#1$}
  \else
    \setbox\mlinebox=\hbox{\unhbox\mtrialbox}
  \fi
}
\def\endm{  % Output the last line of a formula
  \box\mlinebox
}

% \SLASH for \ , \TOR for \/ (text OR), \TAND for /\ (text and)
% This embeds a following forced space to force the space.
\newcommand\SLASH{\char`\\~}
\newcommand\TOR{\char`\\/~}
\newcommand\TAND{/\char`\\~}
%
% Macro to output metamath raw text.
% This assumes \startprefix and \contprefix are set.
% NOTE: "\" is tricky to escape, use \SLASH, \TOR, and \TAND inside.
% Any use of "$ { ~ ^" must be escaped; ~ and ^ must be escaped specially.
% We escape { and } for consistency.
% For more about how this macro written, see:
% https://stackoverflow.com/questions/4073674/
% how-to-disable-indentation-in-particular-section-in-latex/4075706
% Use frenchspacing, or "e." will get an extra space after it.
\newlength\mystoreparindent
\newlength\mystorehangindent
\newenvironment{mmraw}{%
\setlength{\mystoreparindent}{\the\parindent}
\setlength{\mystorehangindent}{\the\hangindent}
\setlength{\parindent}{0pt} % TODO - we'll put in the \startprefix instead
\setlength{\hangindent}{\wd\the\contprefix}
\begin{flushleft}
\begin{frenchspacing}
\begin{tt}
{\unhcopy\startprefix}%
}{%
\end{tt}
\end{frenchspacing}
\end{flushleft}
\setlength{\parindent}{\mystoreparindent}
\setlength{\hangindent}{\mystorehangindent}
\vskip 1ex
}

\needspace{5\baselineskip}
\subsection{Propositional Calculus}\label{propcalc}\index{axioms of
propositional calculus}

\needspace{2\baselineskip}
Axiom of Simplification.\label{ax1}

\setbox\startprefix=\hbox{\tt \ \ ax-1\ \$a\ }
\setbox\contprefix=\hbox{\tt \ \ \ \ \ \ \ \ \ \ }
\startm
\m{\vdash}\m{(}\m{\varphi}\m{\rightarrow}\m{(}\m{\psi}\m{\rightarrow}\m{\varphi}\m{)}
\m{)}
\endm

\needspace{3\baselineskip}
\noindent Axiom of Distribution.

\setbox\startprefix=\hbox{\tt \ \ ax-2\ \$a\ }
\setbox\contprefix=\hbox{\tt \ \ \ \ \ \ \ \ \ \ }
\startm
\m{\vdash}\m{(}\m{(}\m{\varphi}\m{\rightarrow}\m{(}\m{\psi}\m{\rightarrow}\m{\chi}
\m{)}\m{)}\m{\rightarrow}\m{(}\m{(}\m{\varphi}\m{\rightarrow}\m{\psi}\m{)}\m{
\rightarrow}\m{(}\m{\varphi}\m{\rightarrow}\m{\chi}\m{)}\m{)}\m{)}
\endm

\needspace{2\baselineskip}
\noindent Axiom of Contraposition.

\setbox\startprefix=\hbox{\tt \ \ ax-3\ \$a\ }
\setbox\contprefix=\hbox{\tt \ \ \ \ \ \ \ \ \ \ }
\startm
\m{\vdash}\m{(}\m{(}\m{\lnot}\m{\varphi}\m{\rightarrow}\m{\lnot}\m{\psi}\m{)}\m{
\rightarrow}\m{(}\m{\psi}\m{\rightarrow}\m{\varphi}\m{)}\m{)}
\endm


\needspace{4\baselineskip}
\noindent Rule of Modus Ponens.\label{axmp}\index{modus ponens}

\setbox\startprefix=\hbox{\tt \ \ maj\ \$e\ }
\setbox\contprefix=\hbox{\tt \ \ \ \ \ \ \ \ \ }
\startm
\m{\vdash}\m{(}\m{\varphi}\m{\rightarrow}\m{\psi}\m{)}
\endm

\setbox\startprefix=\hbox{\tt \ \ min\ \$e\ }
\setbox\contprefix=\hbox{\tt \ \ \ \ \ \ \ \ \ }
\startm
\m{\vdash}\m{\varphi}
\endm

\setbox\startprefix=\hbox{\tt \ \ ax-mp\ \$a\ }
\setbox\contprefix=\hbox{\tt \ \ \ \ \ \ \ \ \ \ \ }
\startm
\m{\vdash}\m{\psi}
\endm


\needspace{7\baselineskip}
\subsection{Axioms of Predicate Calculus with Equality---Tarski's S2}\index{axioms of predicate calculus}

\needspace{3\baselineskip}
\noindent Rule of Generalization.\index{rule of generalization}

\setbox\startprefix=\hbox{\tt \ \ ax-g.1\ \$e\ }
\setbox\contprefix=\hbox{\tt \ \ \ \ \ \ \ \ \ \ \ \ }
\startm
\m{\vdash}\m{\varphi}
\endm

\setbox\startprefix=\hbox{\tt \ \ ax-gen\ \$a\ }
\setbox\contprefix=\hbox{\tt \ \ \ \ \ \ \ \ \ \ \ \ }
\startm
\m{\vdash}\m{\forall}\m{x}\m{\varphi}
\endm

\needspace{2\baselineskip}
\noindent Axiom of Quantified Implication.

\setbox\startprefix=\hbox{\tt \ \ ax-4\ \$a\ }
\setbox\contprefix=\hbox{\tt \ \ \ \ \ \ \ \ \ \ }
\startm
\m{\vdash}\m{(}\m{\forall}\m{x}\m{(}\m{\forall}\m{x}\m{\varphi}\m{\rightarrow}\m{
\psi}\m{)}\m{\rightarrow}\m{(}\m{\forall}\m{x}\m{\varphi}\m{\rightarrow}\m{
\forall}\m{x}\m{\psi}\m{)}\m{)}
\endm

\needspace{3\baselineskip}
\noindent Axiom of Distinctness.

% Aka: Add $d x ph $.
\setbox\startprefix=\hbox{\tt \ \ ax-5\ \$a\ }
\setbox\contprefix=\hbox{\tt \ \ \ \ \ \ \ \ \ \ }
\startm
\m{\vdash}\m{(}\m{\varphi}\m{\rightarrow}\m{\forall}\m{x}\m{\varphi}\m{)}\m{where}\m{ }\m{\$d}\m{ }\m{x}\m{ }\m{\varphi}\m{ }\m{(}\m{x}\m{ }\m{does}\m{ }\m{not}\m{ }\m{occur}\m{ }\m{in}\m{ }\m{\varphi}\m{)}
\endm

\needspace{2\baselineskip}
\noindent Axiom of Existence.

\setbox\startprefix=\hbox{\tt \ \ ax-6\ \$a\ }
\setbox\contprefix=\hbox{\tt \ \ \ \ \ \ \ \ \ \ }
\startm
\m{\vdash}\m{(}\m{\forall}\m{x}\m{(}\m{x}\m{=}\m{y}\m{\rightarrow}\m{\forall}
\m{x}\m{\varphi}\m{)}\m{\rightarrow}\m{\varphi}\m{)}
\endm

\needspace{2\baselineskip}
\noindent Axiom of Equality.

\setbox\startprefix=\hbox{\tt \ \ ax-7\ \$a\ }
\setbox\contprefix=\hbox{\tt \ \ \ \ \ \ \ \ \ \ }
\startm
\m{\vdash}\m{(}\m{x}\m{=}\m{y}\m{\rightarrow}\m{(}\m{x}\m{=}\m{z}\m{
\rightarrow}\m{y}\m{=}\m{z}\m{)}\m{)}
\endm

\needspace{2\baselineskip}
\noindent Axiom of Left Equality for Binary Predicate.

\setbox\startprefix=\hbox{\tt \ \ ax-8\ \$a\ }
\setbox\contprefix=\hbox{\tt \ \ \ \ \ \ \ \ \ \ \ }
\startm
\m{\vdash}\m{(}\m{x}\m{=}\m{y}\m{\rightarrow}\m{(}\m{x}\m{\in}\m{z}\m{
\rightarrow}\m{y}\m{\in}\m{z}\m{)}\m{)}
\endm

\needspace{2\baselineskip}
\noindent Axiom of Right Equality for Binary Predicate.

\setbox\startprefix=\hbox{\tt \ \ ax-9\ \$a\ }
\setbox\contprefix=\hbox{\tt \ \ \ \ \ \ \ \ \ \ \ }
\startm
\m{\vdash}\m{(}\m{x}\m{=}\m{y}\m{\rightarrow}\m{(}\m{z}\m{\in}\m{x}\m{
\rightarrow}\m{z}\m{\in}\m{y}\m{)}\m{)}
\endm


\needspace{4\baselineskip}
\subsection{Axioms of Predicate Calculus with Equality---Auxiliary}\index{axioms of predicate calculus - auxiliary}

\needspace{2\baselineskip}
\noindent Axiom of Quantified Negation.

\setbox\startprefix=\hbox{\tt \ \ ax-10\ \$a\ }
\setbox\contprefix=\hbox{\tt \ \ \ \ \ \ \ \ \ \ }
\startm
\m{\vdash}\m{(}\m{\lnot}\m{\forall}\m{x}\m{\lnot}\m{\forall}\m{x}\m{\varphi}\m{
\rightarrow}\m{\varphi}\m{)}
\endm

\needspace{2\baselineskip}
\noindent Axiom of Quantifier Commutation.

\setbox\startprefix=\hbox{\tt \ \ ax-11\ \$a\ }
\setbox\contprefix=\hbox{\tt \ \ \ \ \ \ \ \ \ \ }
\startm
\m{\vdash}\m{(}\m{\forall}\m{x}\m{\forall}\m{y}\m{\varphi}\m{\rightarrow}\m{
\forall}\m{y}\m{\forall}\m{x}\m{\varphi}\m{)}
\endm

\needspace{3\baselineskip}
\noindent Axiom of Substitution.

\setbox\startprefix=\hbox{\tt \ \ ax-12\ \$a\ }
\setbox\contprefix=\hbox{\tt \ \ \ \ \ \ \ \ \ \ \ }
\startm
\m{\vdash}\m{(}\m{\lnot}\m{\forall}\m{x}\m{\,x}\m{=}\m{y}\m{\rightarrow}\m{(}
\m{x}\m{=}\m{y}\m{\rightarrow}\m{(}\m{\varphi}\m{\rightarrow}\m{\forall}\m{x}\m{(}
\m{x}\m{=}\m{y}\m{\rightarrow}\m{\varphi}\m{)}\m{)}\m{)}\m{)}
\endm

\needspace{3\baselineskip}
\noindent Axiom of Quantified Equality.

\setbox\startprefix=\hbox{\tt \ \ ax-13\ \$a\ }
\setbox\contprefix=\hbox{\tt \ \ \ \ \ \ \ \ \ \ \ }
\startm
\m{\vdash}\m{(}\m{\lnot}\m{\forall}\m{z}\m{\,z}\m{=}\m{x}\m{\rightarrow}\m{(}
\m{\lnot}\m{\forall}\m{z}\m{\,z}\m{=}\m{y}\m{\rightarrow}\m{(}\m{x}\m{=}\m{y}
\m{\rightarrow}\m{\forall}\m{z}\m{\,x}\m{=}\m{y}\m{)}\m{)}\m{)}
\endm

% \noindent Axiom of Quantifier Substitution
%
% \setbox\startprefix=\hbox{\tt \ \ ax-c11n\ \$a\ }
% \setbox\contprefix=\hbox{\tt \ \ \ \ \ \ \ \ \ \ \ }
% \startm
% \m{\vdash}\m{(}\m{\forall}\m{x}\m{\,x}\m{=}\m{y}\m{\rightarrow}\m{(}\m{\forall}
% \m{x}\m{\varphi}\m{\rightarrow}\m{\forall}\m{y}\m{\varphi}\m{)}\m{)}
% \endm
%
% \noindent Axiom of Distinct Variables. (This axiom requires
% that two individual variables
% be distinct\index{\texttt{\$d} statement}\index{distinct
% variables}.)
%
% \setbox\startprefix=\hbox{\tt \ \ \ \ \ \ \ \ \$d\ }
% \setbox\contprefix=\hbox{\tt \ \ \ \ \ \ \ \ \ \ \ }
% \startm
% \m{x}\m{\,}\m{y}
% \endm
%
% \setbox\startprefix=\hbox{\tt \ \ ax-c16\ \$a\ }
% \setbox\contprefix=\hbox{\tt \ \ \ \ \ \ \ \ \ \ \ }
% \startm
% \m{\vdash}\m{(}\m{\forall}\m{x}\m{\,x}\m{=}\m{y}\m{\rightarrow}\m{(}\m{\varphi}\m{
% \rightarrow}\m{\forall}\m{x}\m{\varphi}\m{)}\m{)}
% \endm

% \noindent Axiom of Quantifier Introduction (2).  (This axiom requires
% that the individual variable not occur in the
% wff\index{\texttt{\$d} statement}\index{distinct variables}.)
%
% \setbox\startprefix=\hbox{\tt \ \ \ \ \ \ \ \ \$d\ }
% \setbox\contprefix=\hbox{\tt \ \ \ \ \ \ \ \ \ \ \ }
% \startm
% \m{x}\m{\,}\m{\varphi}
% \endm
% \setbox\startprefix=\hbox{\tt \ \ ax-5\ \$a\ }
% \setbox\contprefix=\hbox{\tt \ \ \ \ \ \ \ \ \ \ \ }
% \startm
% \m{\vdash}\m{(}\m{\varphi}\m{\rightarrow}\m{\forall}\m{x}\m{\varphi}\m{)}
% \endm

\subsection{Set Theory}\label{mmsettheoryaxioms}

In order to make the axioms of set theory\index{axioms of set theory} a little
more compact, there are several definitions from logic that we make use of
implicitly, namely, ``logical {\sc and},''\index{conjunction ($\wedge$)}
\index{logical {\sc and} ($\wedge$)} ``logical equivalence,''\index{logical
equivalence ($\leftrightarrow$)}\index{biconditional ($\leftrightarrow$)} and
``there exists.''\index{existential quantifier ($\exists$)}

\begin{center}\begin{tabular}{rcl}
  $( \varphi \wedge \psi )$ &\mbox{stands for}& $\neg ( \varphi
     \rightarrow \neg \psi )$\\
  $( \varphi \leftrightarrow \psi )$& \mbox{stands
     for}& $( ( \varphi \rightarrow \psi ) \wedge
     ( \psi \rightarrow \varphi ) )$\\
  $\exists x \,\varphi$ &\mbox{stands for}& $\neg \forall x \neg \varphi$
\end{tabular}\end{center}

In addition, the axioms of set theory require that all variables be
dis\-tinct,\index{distinct variables}\footnote{Set theory axioms can be
devised so that {\em no} variables are required to be distinct,
provided we replace \texttt{ax-c16} with an axiom stating that ``at
least two things exist,'' thus
making \texttt{ax-5} the only other axiom requiring the
\texttt{\$d} statement.  These axioms are unconventional and are not
presented here, but they can be found on the \url{http://metamath.org}
web site.  See also the Comment on
p.~\pageref{nodd}.}\index{\texttt{\$d} statement} thus we also assume:
\begin{center}
  \texttt{\$d }$x\,y\,z\,w$
\end{center}

\needspace{2\baselineskip}
\noindent Axiom of Extensionality.\index{Axiom of Extensionality}

\setbox\startprefix=\hbox{\tt \ \ ax-ext\ \$a\ }
\setbox\contprefix=\hbox{\tt \ \ \ \ \ \ \ \ \ \ \ \ }
\startm
\m{\vdash}\m{(}\m{\forall}\m{x}\m{(}\m{x}\m{\in}\m{y}\m{\leftrightarrow}\m{x}
\m{\in}\m{z}\m{)}\m{\rightarrow}\m{y}\m{=}\m{z}\m{)}
\endm

\needspace{3\baselineskip}
\noindent Axiom of Replacement.\index{Axiom of Replacement}

\setbox\startprefix=\hbox{\tt \ \ ax-rep\ \$a\ }
\setbox\contprefix=\hbox{\tt \ \ \ \ \ \ \ \ \ \ \ \ }
\startm
\m{\vdash}\m{(}\m{\forall}\m{w}\m{\exists}\m{y}\m{\forall}\m{z}\m{(}\m{%
\forall}\m{y}\m{\varphi}\m{\rightarrow}\m{z}\m{=}\m{y}\m{)}\m{\rightarrow}\m{%
\exists}\m{y}\m{\forall}\m{z}\m{(}\m{z}\m{\in}\m{y}\m{\leftrightarrow}\m{%
\exists}\m{w}\m{(}\m{w}\m{\in}\m{x}\m{\wedge}\m{\forall}\m{y}\m{\varphi}\m{)}%
\m{)}\m{)}
\endm

\needspace{2\baselineskip}
\noindent Axiom of Union.\index{Axiom of Union}

\setbox\startprefix=\hbox{\tt \ \ ax-un\ \$a\ }
\setbox\contprefix=\hbox{\tt \ \ \ \ \ \ \ \ \ \ \ }
\startm
\m{\vdash}\m{\exists}\m{x}\m{\forall}\m{y}\m{(}\m{\exists}\m{x}\m{(}\m{y}\m{
\in}\m{x}\m{\wedge}\m{x}\m{\in}\m{z}\m{)}\m{\rightarrow}\m{y}\m{\in}\m{x}\m{)}
\endm

\needspace{2\baselineskip}
\noindent Axiom of Power Sets.\index{Axiom of Power Sets}

\setbox\startprefix=\hbox{\tt \ \ ax-pow\ \$a\ }
\setbox\contprefix=\hbox{\tt \ \ \ \ \ \ \ \ \ \ \ \ }
\startm
\m{\vdash}\m{\exists}\m{x}\m{\forall}\m{y}\m{(}\m{\forall}\m{x}\m{(}\m{x}\m{
\in}\m{y}\m{\rightarrow}\m{x}\m{\in}\m{z}\m{)}\m{\rightarrow}\m{y}\m{\in}\m{x}
\m{)}
\endm

\needspace{3\baselineskip}
\noindent Axiom of Regularity.\index{Axiom of Regularity}

\setbox\startprefix=\hbox{\tt \ \ ax-reg\ \$a\ }
\setbox\contprefix=\hbox{\tt \ \ \ \ \ \ \ \ \ \ \ \ }
\startm
\m{\vdash}\m{(}\m{\exists}\m{x}\m{\,x}\m{\in}\m{y}\m{\rightarrow}\m{\exists}
\m{x}\m{(}\m{x}\m{\in}\m{y}\m{\wedge}\m{\forall}\m{z}\m{(}\m{z}\m{\in}\m{x}\m{
\rightarrow}\m{\lnot}\m{z}\m{\in}\m{y}\m{)}\m{)}\m{)}
\endm

\needspace{3\baselineskip}
\noindent Axiom of Infinity.\index{Axiom of Infinity}

\setbox\startprefix=\hbox{\tt \ \ ax-inf\ \$a\ }
\setbox\contprefix=\hbox{\tt \ \ \ \ \ \ \ \ \ \ \ \ \ \ \ }
\startm
\m{\vdash}\m{\exists}\m{x}\m{(}\m{y}\m{\in}\m{x}\m{\wedge}\m{\forall}\m{y}%
\m{(}\m{y}\m{\in}\m{x}\m{\rightarrow}\m{\exists}\m{z}\m{(}\m{y}\m{\in}\m{z}\m{%
\wedge}\m{z}\m{\in}\m{x}\m{)}\m{)}\m{)}
\endm

\needspace{4\baselineskip}
\noindent Axiom of Choice.\index{Axiom of Choice}

\setbox\startprefix=\hbox{\tt \ \ ax-ac\ \$a\ }
\setbox\contprefix=\hbox{\tt \ \ \ \ \ \ \ \ \ \ \ \ \ \ }
\startm
\m{\vdash}\m{\exists}\m{x}\m{\forall}\m{y}\m{\forall}\m{z}\m{(}\m{(}\m{y}\m{%
\in}\m{z}\m{\wedge}\m{z}\m{\in}\m{w}\m{)}\m{\rightarrow}\m{\exists}\m{w}\m{%
\forall}\m{y}\m{(}\m{\exists}\m{w}\m{(}\m{(}\m{y}\m{\in}\m{z}\m{\wedge}\m{z}%
\m{\in}\m{w}\m{)}\m{\wedge}\m{(}\m{y}\m{\in}\m{w}\m{\wedge}\m{w}\m{\in}\m{x}%
\m{)}\m{)}\m{\leftrightarrow}\m{y}\m{=}\m{w}\m{)}\m{)}
\endm

\subsection{That's It}

There you have it, the axioms for (essentially) all of mathematics!
Wonder at them and stare at them in awe.  Put a copy in your wallet, and
you will carry in your pocket the encoding for all theorems ever proved
and that ever will be proved, from the most mundane to the most
profound.

\section{A Hierarchy of Definitions}\label{hierarchy}

The axioms in the previous section in principle embody everything that can be
done within standard mathematics.  However, it is impractical to accomplish
very much by using them directly, for even simple concepts (from a human
perspective) can involve extremely long, incomprehensible formulas.
Mathematics is made practical by introducing definitions\index{definition}.
Definitions usually introduce new symbols, or at least new relationships among
existing symbols, to abbreviate more complex formulas.  An important
requirement for a definition is that there exist a straightforward
(algorithmic) method for eliminating the abbreviation by expanding it into the
more primitive symbol string that it represents.  Some
important definitions included in
the file \texttt{set.mm} are listed in this section for reference, and also to
give you a feel for why something like $\omega$\index{omega ($\omega$)} (the
set of natural numbers\index{natural number} 0, 1, 2,\ldots) becomes very
complicated when completely expanded into primitive symbols.

What is the motivation for definitions, aside from allowing complicated
expressions to be expressed more simply?  In the case of  $\omega$, one goal is
to provide a basis for the theory of natural numbers.\index{natural number}
Before set theory was invented, a set of axioms for arithmetic, called Peano's
postulates\index{Peano's postulates}, was devised and shown to have the
properties one expects for natural numbers.  Now anyone can postulate a
set of axioms, but if the axioms are inconsistent contradictions can be derived
from them.  Once a contradiction is derived, anything can be trivially
proved, including
all the facts of arithmetic and their negations.  To ensure that an
axiom system is at least as reliable as the axioms for set theory, we can
define sets and operations on those sets that satisfy the new axioms. In the
\texttt{set.mm} Metamath database, we prove that the elements of $\omega$ satisfy
Peano's postulates, and it's a long and hard journey to get there directly
from the axioms of set theory.  But the result is confidence in the
foundations of arithmetic.  And there is another advantage:  we now have all
the tools of set theory at our disposal for manipulating objects that obey the
axioms for arithmetic.

What are the criteria we use for definitions?  First, and of utmost importance,
the definition should not be {\em creative}\index{creative
definition}\index{definition!creative}, that
is it should not allow an expression that previously qualified as a wff but
was not provable, to become provable.   Second, the definition should be {\em
eliminable}\index{definition!eliminability}, that is, there should exist an
algorithmic method for converting any expression using the definition into
a logically equivalent expression that previously qualified as a wff.

In almost all cases below, definitions connect two expressions with either
$\leftrightarrow$ or $=$.  Eliminating\footnote{Here we mean the
elimination that a human might do in his or her head.  To eliminate them as
part of a Metamath proof we would invoke one of a number of
theorems that deal with transitivity of equivalence or equality; there are
many such examples in the proofs in \texttt{set.mm}.} such a definition is a
simple matter of substituting the expression on the left-hand side ({\em
definiendum}\index{definiendum} or thing being defined) with the equivalent,
more primitive expression on the right-hand side ({\em
definiens}\index{definiens} or definition).

Often a definition has variables on the right-hand side which do not appear on
the left-hand side; these are called {\em dummy variables}.\index{dummy
variable!in definitions}  In this case, any
allowable substitution (such as a new, distinct
variable) can be used when the definition is eliminated.  Dummy variables may
be used only if they are {\em effectively bound}\index{effectively bound
variable}, meaning that the definition will remain logically equivalent upon
any substitution of a dummy variable with any other {\em qualifying
expression}\index{qualifying expression}, i.e.\ any symbol string (such as
another variable) that
meets the restrictions on the dummy variable imposed by \texttt{\$d} and
\texttt{\$f} statements.  For example, we could define a constant $\perp$
(inverted tee, meaning logical ``false'') as $( \varphi \wedge \lnot \varphi
)$, i.e.\ ``phi and not phi.''  Here $\varphi$ is effectively bound because the
definition remains logically equivalent when we replace $\varphi$ with any
other wff.  (It is actually \texttt{df-fal}
in \texttt{set.mm}, which defines $\perp$.)

There are two cases where eliminating definitions is a little more
complex.  These cases are the definitions \texttt{df-bi} and
\texttt{df-cleq}.  The first stretches the concept of a definition a
little, as in effect it ``defines a definition;'' however, it meets our
requirements for a definition in that it is eliminable and does not
strengthen the language.  Theorem \texttt{bii} shows the substitution
needed to eliminate the $\leftrightarrow$\index{logical equivalence
($\leftrightarrow$)}\index{biconditional ($\leftrightarrow$)} symbol.

Definition \texttt{df-cleq}\index{equality ($=$)} extends the usage of
the equality symbol to include ``classes''\index{class} in set theory.  The
reason it is potentially problematic is that it can lead to statements which
do not follow from logic alone but presuppose the Axiom of
Extensionality\index{Axiom of Extensionality}, so we include this axiom
as a hypothesis for the definition.  We could have made \texttt{df-cleq} directly
eliminable by introducing a new equality symbol, but have chosen not to do so
in keeping with standard textbook practice.  Definitions such as \texttt{df-cleq}
that extend the meaning of existing symbols must be introduced carefully so
that they do not lead to contradictions.  Definition \texttt{df-clel} also
extends the meaning of an existing symbol ($\in$); while it doesn't strengthen
the language like \texttt{df-cleq}, this is not obvious and it must also be
subject to the same scrutiny.

Exercise:  Study how the wff $x\in\omega$, meaning ``$x$ is a natural
number,'' could be expanded in terms of primitive symbols, starting with the
definitions \texttt{df-clel} on p.~\pageref{dfclel} and \texttt{df-om} on
p.~\pageref{dfom} and working your way back.  Don't bother to work out the
details; just make sure that you understand how you could do it in principle.
The answer is shown in the footnote on p.~\pageref{expandom}.  If you
actually do work it out, you won't get exactly the same answer because we used
a few simplifications such as discarding occurrences of $\lnot\lnot$ (double
negation).

In the definitions below, we have placed the {\sc ascii} Metamath source
below each of the formulas to help you become familiar with the
notation in the database.  For simplicity, the necessary \texttt{\$f}
and \texttt{\$d} statements are not shown.  If you are in doubt, use the
\texttt{show statement}\index{\texttt{show statement} command} command
in the Metamath program to see the full statement.
A selection of this notation is summarized in Appendix~\ref{ASCII}.

To understand the motivation for these definitions, you should consult the
references indicated:  Takeuti and Zaring \cite{Takeuti}\index{Takeuti, Gaisi},
Quine \cite{Quine}\index{Quine, Willard Van Orman}, Bell and Machover
\cite{Bell}\index{Bell, J. L.}, and Enderton \cite{Enderton}\index{Enderton,
Herbert B.}.  Our list of definitions is provided more for reference than as a
learning aid.  However, by looking at a few of them you can gain a feel for
how the hierarchy is built up.  The definitions are a representative sample of
the 130 or so in \texttt{set.mm}, but they are complete with respect to the
theorem examples we will present in Section~\ref{sometheorems}.  Also, some are
slightly different from, but logically equivalent to, the ones in \texttt{set.mm}
(some of which have been revised over time to shorten them, for example).

\subsection{Definitions for Propositional Calculus}\label{metadefprop}

The symbols $\varphi$, $\psi$, and $\chi$ represent wffs.

Our first definition introduces the biconditional
connective\footnote{The term ``connective'' is informally used to mean a
symbol that is placed between two variables or adjacent to a variable,
whereas a mathematical ``constant'' usually indicates a symbol such as
the number 0 that may replace a variable or metavariable.  From
Metamath's point of view, there is no distinction between a connective
and a constant; both are constants in the Metamath
language.}\index{connective}\index{constant} (also called logical
equivalence)\index{logical equivalence
($\leftrightarrow$)}\index{biconditional ($\leftrightarrow$)}.  Unlike
most traditional developments, we have chosen not to have a separate
symbol such as ``Df.'' to mean ``is defined as.''  Instead, we will use
the biconditional connective for this purpose, as it lets us use
logic to manipulate definitions directly.  Here we state the properties
of the biconditional connective with a carefully crafted \texttt{\$a}
statement, which effectively uses the biconditional connective to define
itself.  The $\leftrightarrow$ symbol can be eliminated from a formula
using theorem \texttt{bii}, which is derived later.

\vskip 2ex
\noindent Define the biconditional connective.\label{df-bi}

\vskip 0.5ex
\setbox\startprefix=\hbox{\tt \ \ df-bi\ \$a\ }
\setbox\contprefix=\hbox{\tt \ \ \ \ \ \ \ \ \ \ \ }
\startm
\m{\vdash}\m{\lnot}\m{(}\m{(}\m{(}\m{\varphi}\m{\leftrightarrow}\m{\psi}\m{)}%
\m{\rightarrow}\m{\lnot}\m{(}\m{(}\m{\varphi}\m{\rightarrow}\m{\psi}\m{)}\m{%
\rightarrow}\m{\lnot}\m{(}\m{\psi}\m{\rightarrow}\m{\varphi}\m{)}\m{)}\m{)}\m{%
\rightarrow}\m{\lnot}\m{(}\m{\lnot}\m{(}\m{(}\m{\varphi}\m{\rightarrow}\m{%
\psi}\m{)}\m{\rightarrow}\m{\lnot}\m{(}\m{\psi}\m{\rightarrow}\m{\varphi}\m{)}%
\m{)}\m{\rightarrow}\m{(}\m{\varphi}\m{\leftrightarrow}\m{\psi}\m{)}\m{)}\m{)}
\endm
\begin{mmraw}%
|- -. ( ( ( ph <-> ps ) -> -. ( ( ph -> ps ) ->
-. ( ps -> ph ) ) ) -> -. ( -. ( ( ph -> ps ) -> -. (
ps -> ph ) ) -> ( ph <-> ps ) ) ) \$.
\end{mmraw}

\noindent This theorem relates the biconditional connective to primitive
connectives and can be used to eliminate the $\leftrightarrow$ symbol from any
wff.

\vskip 0.5ex
\setbox\startprefix=\hbox{\tt \ \ bii\ \$p\ }
\setbox\contprefix=\hbox{\tt \ \ \ \ \ \ \ \ \ }
\startm
\m{\vdash}\m{(}\m{(}\m{\varphi}\m{\leftrightarrow}\m{\psi}\m{)}\m{\leftrightarrow}
\m{\lnot}\m{(}\m{(}\m{\varphi}\m{\rightarrow}\m{\psi}\m{)}\m{\rightarrow}\m{\lnot}
\m{(}\m{\psi}\m{\rightarrow}\m{\varphi}\m{)}\m{)}\m{)}
\endm
\begin{mmraw}%
|- ( ( ph <-> ps ) <-> -. ( ( ph -> ps ) -> -. ( ps -> ph ) ) ) \$= ... \$.
\end{mmraw}

\noindent Define disjunction ({\sc or}).\index{disjunction ($\vee$)}%
\index{logical or (vee)@logical {\sc or} ($\vee$)}%
\index{df-or@\texttt{df-or}}\label{df-or}

\vskip 0.5ex
\setbox\startprefix=\hbox{\tt \ \ df-or\ \$a\ }
\setbox\contprefix=\hbox{\tt \ \ \ \ \ \ \ \ \ \ \ }
\startm
\m{\vdash}\m{(}\m{(}\m{\varphi}\m{\vee}\m{\psi}\m{)}\m{\leftrightarrow}\m{(}\m{
\lnot}\m{\varphi}\m{\rightarrow}\m{\psi}\m{)}\m{)}
\endm
\begin{mmraw}%
|- ( ( ph \TOR ps ) <-> ( -. ph -> ps ) ) \$.
\end{mmraw}

\noindent Define conjunction ({\sc and}).\index{conjunction ($\wedge$)}%
\index{logical {\sc and} ($\wedge$)}%
\index{df-an@\texttt{df-an}}\label{df-an}

\vskip 0.5ex
\setbox\startprefix=\hbox{\tt \ \ df-an\ \$a\ }
\setbox\contprefix=\hbox{\tt \ \ \ \ \ \ \ \ \ \ \ }
\startm
\m{\vdash}\m{(}\m{(}\m{\varphi}\m{\wedge}\m{\psi}\m{)}\m{\leftrightarrow}\m{\lnot}
\m{(}\m{\varphi}\m{\rightarrow}\m{\lnot}\m{\psi}\m{)}\m{)}
\endm
\begin{mmraw}%
|- ( ( ph \TAND ps ) <-> -. ( ph -> -. ps ) ) \$.
\end{mmraw}

\noindent Define disjunction ({\sc or}) of 3 wffs.%
\index{df-3or@\texttt{df-3or}}\label{df-3or}

\vskip 0.5ex
\setbox\startprefix=\hbox{\tt \ \ df-3or\ \$a\ }
\setbox\contprefix=\hbox{\tt \ \ \ \ \ \ \ \ \ \ \ \ }
\startm
\m{\vdash}\m{(}\m{(}\m{\varphi}\m{\vee}\m{\psi}\m{\vee}\m{\chi}\m{)}\m{
\leftrightarrow}\m{(}\m{(}\m{\varphi}\m{\vee}\m{\psi}\m{)}\m{\vee}\m{\chi}\m{)}
\m{)}
\endm
\begin{mmraw}%
|- ( ( ph \TOR ps \TOR ch ) <-> ( ( ph \TOR ps ) \TOR ch ) ) \$.
\end{mmraw}

\noindent Define conjunction ({\sc and}) of 3 wffs.%
\index{df-3an}\label{df-3an}

\vskip 0.5ex
\setbox\startprefix=\hbox{\tt \ \ df-3an\ \$a\ }
\setbox\contprefix=\hbox{\tt \ \ \ \ \ \ \ \ \ \ \ \ }
\startm
\m{\vdash}\m{(}\m{(}\m{\varphi}\m{\wedge}\m{\psi}\m{\wedge}\m{\chi}\m{)}\m{
\leftrightarrow}\m{(}\m{(}\m{\varphi}\m{\wedge}\m{\psi}\m{)}\m{\wedge}\m{\chi}
\m{)}\m{)}
\endm

\begin{mmraw}%
|- ( ( ph \TAND ps \TAND ch ) <-> ( ( ph \TAND ps ) \TAND ch ) ) \$.
\end{mmraw}

\subsection{Definitions for Predicate Calculus}\label{metadefpred}

The symbols $x$, $y$, and $z$ represent individual variables of predicate
calculus.  In this section, they are not necessarily distinct unless it is
explicitly
mentioned.

\vskip 2ex
\noindent Define existential quantification.
The expression $\exists x \varphi$ means
``there exists an $x$ where $\varphi$ is true.''\index{existential quantifier
($\exists$)}\label{df-ex}

\vskip 0.5ex
\setbox\startprefix=\hbox{\tt \ \ df-ex\ \$a\ }
\setbox\contprefix=\hbox{\tt \ \ \ \ \ \ \ \ \ \ \ }
\startm
\m{\vdash}\m{(}\m{\exists}\m{x}\m{\varphi}\m{\leftrightarrow}\m{\lnot}\m{\forall}
\m{x}\m{\lnot}\m{\varphi}\m{)}
\endm
\begin{mmraw}%
|- ( E. x ph <-> -. A. x -. ph ) \$.
\end{mmraw}

\noindent Define proper substitution.\index{proper
substitution}\index{substitution!proper}\label{df-sb}
In our notation, we use $[ y / x ] \varphi$ to mean ``the wff that
results when $y$ is properly substituted for $x$ in the wff
$\varphi$.''\footnote{
This can also be described
as substituting $x$ with $y$, $y$ properly replaces $x$, or
$x$ is properly replaced by $y$.}
% This is elsb4, though it currently says: ( [ x / y ] z e. y <-> z e. x )
For example,
$[ y / x ] z \in x$ is the same as $z \in y$.
One way to remember this notation is to notice that it looks like division
and recall that $( y / x ) \cdot x $ is $y$ (when $x \neq 0$).
The notation is different from the notation $\varphi ( x | y )$
that is sometimes used, because the latter notation is ambiguous for us:
for example, we don't know whether $\lnot \varphi ( x | y )$ is to be
interpreted as $\lnot ( \varphi ( x | y ) )$ or
$( \lnot \varphi ) ( x | y )$.\footnote{Because of the way
we initially defined wffs, this is the case
with any postfix connective\index{postfix connective} (one occurring after the
symbols being connected) or infix connective\index{infix connective} (one
occurring between the symbols being connected).  Metamath does not have a
built-in notion of operator binding strength that could eliminate the
ambiguity.  The initial parenthesis effectively provides a prefix
connective\index{prefix connective} to eliminate ambiguity.  Some conventions,
such as Polish notation\index{Polish notation} used in the 1930's and 1940's
by Polish logicians, use only prefix connectives and thus allow the total
elimination of parentheses, at the expense of readability.  In Metamath we
could actually redefine all notation to be Polish if we wanted to without
having to change any proofs!}  Other texts often use $\varphi(y)$ to indicate
our $[ y / x ] \varphi$, but this notation is even more ambiguous since there is
no explicit indication of what is being substituted.
Note that this
definition is valid even when
$x$ and $y$ are the same variable.  The first conjunct is a ``trick'' used to
achieve this property, making the definition look somewhat peculiar at
first.

\vskip 0.5ex
\setbox\startprefix=\hbox{\tt \ \ df-sb\ \$a\ }
\setbox\contprefix=\hbox{\tt \ \ \ \ \ \ \ \ \ \ \ }
\startm
\m{\vdash}\m{(}\m{[}\m{y}\m{/}\m{x}\m{]}\m{\varphi}\m{\leftrightarrow}\m{(}%
\m{(}\m{x}\m{=}\m{y}\m{\rightarrow}\m{\varphi}\m{)}\m{\wedge}\m{\exists}\m{x}%
\m{(}\m{x}\m{=}\m{y}\m{\wedge}\m{\varphi}\m{)}\m{)}\m{)}
\endm
\begin{mmraw}%
|- ( [ y / x ] ph <-> ( ( x = y -> ph ) \TAND E. x ( x = y \TAND ph ) ) ) \$.
\end{mmraw}


\noindent Define existential uniqueness\index{existential uniqueness
quantifier ($\exists "!$)} (``there exists exactly one'').  Note that $y$ is a
variable distinct from $x$ and not occurring in $\varphi$.\label{df-eu}

\vskip 0.5ex
\setbox\startprefix=\hbox{\tt \ \ df-eu\ \$a\ }
\setbox\contprefix=\hbox{\tt \ \ \ \ \ \ \ \ \ \ \ }
\startm
\m{\vdash}\m{(}\m{\exists}\m{{!}}\m{x}\m{\varphi}\m{\leftrightarrow}\m{\exists}
\m{y}\m{\forall}\m{x}\m{(}\m{\varphi}\m{\leftrightarrow}\m{x}\m{=}\m{y}\m{)}\m{)}
\endm

\begin{mmraw}%
|- ( E! x ph <-> E. y A. x ( ph <-> x = y ) ) \$.
\end{mmraw}

\subsection{Definitions for Set Theory}\label{setdefinitions}

The symbols $x$, $y$, $z$, and $w$ represent individual variables of
predicate calculus, which in set theory are understood to be sets.
However, using only the constructs shown so far would be very inconvenient.

To make set theory more practical, we introduce the notion of a ``class.''
A class\index{class} is either a set variable (such as $x$) or an
expression of the form $\{ x | \varphi\}$ (called an ``abstraction
class''\index{abstraction class}\index{class abstraction}).  Note that
sets (i.e.\ individual variables) always exist (this is a theorem of
logic, namely $\exists y \, y = x$ for any set $x$), whereas classes may
or may not exist (i.e.\ $\exists y \, y = A$ may or may not be true).
If a class does not exist it is called a ``proper class.''\index{proper
class}\index{class!proper} Definitions \texttt{df-clab},
\texttt{df-cleq}, and \texttt{df-clel} can be used to convert an
expression containing classes into one containing only set variables and
wff metavariables.

The symbols $A$, $B$, $C$, $D$, $ F$, $G$, and $R$ are metavariables that range
over classes.  A class metavariable $A$ may be eliminated from a wff by
replacing it with $\{ x|\varphi\}$ where neither $x$ nor $\varphi$ occur in
the wff.

The theory of classes can be shown to be an eliminable and conservative
extension of set theory. The \textbf{eliminability}
property shows that for every
formula in the extended language we can build a logically equivalent
formula in the basic language; so that even if the extended language
provides more ease to convey and formulate mathematical ideas for set
theory, its expressive power does not in fact strengthen the basic
language's expressive power.
The \textbf{conservation} property shows that for
every proof of a formula of the basic language in the extended system
we can build another proof of the same formula in the basic system;
so that, concerning theorems on sets only, the deductive powers of
the extended system and of the basic system are identical. Together,
these properties mean that the extended language can be treated as a
definitional extension that is \textbf{sound}.

A rigorous justification, which we will not give here, can be found in
Levy \cite[pp.~357-366]{Levy} supplementing his informal introduction to class
theory on pp.~7-17. Two other good treatments of class theory are provided
by Quine \cite[pp.~15-21]{Quine}\index{Quine, Willard Van Orman}
and also \cite[pp.~10-14]{Takeuti}\index{Takeuti, Gaisi}.
Quine's exposition (he calls them virtual classes)
is nicely written and very readable.

In the rest of this
section, individual variables are always assumed to be distinct from
each other unless otherwise indicated.  In addition, dummy variables on the
right-hand side of a definition do not occur in the class and wff
metavariables in the definition.

The definitions we present here are a partial but self-contained
collection selected from several hundred that appear in the current
\texttt{set.mm} database.  They are adequate for a basic development of
elementary set theory.

\vskip 2ex
\noindent Define the abstraction class.\index{abstraction class}\index{class
abstraction}\label{df-clab}  $x$ and $y$
need not be distinct.  Definition 2.1 of Quine, p.~16.  This definition may
seem puzzling since it is shorter than the expression being defined and does not
buy us anything in terms of brevity.  The reason we introduce this definition
is because it fits in neatly with the extension of the $\in$ connective
provided by \texttt{df-clel}.

\vskip 0.5ex
\setbox\startprefix=\hbox{\tt \ \ df-clab\ \$a\ }
\setbox\contprefix=\hbox{\tt \ \ \ \ \ \ \ \ \ \ \ \ \ }
\startm
\m{\vdash}\m{(}\m{x}\m{\in}\m{\{}\m{y}\m{|}\m{\varphi}\m{\}}\m{%
\leftrightarrow}\m{[}\m{x}\m{/}\m{y}\m{]}\m{\varphi}\m{)}
\endm
\begin{mmraw}%
|- ( x e. \{ y | ph \} <-> [ x / y ] ph ) \$.
\end{mmraw}

\noindent Define the equality connective between classes\index{class
equality}\label{df-cleq}.  See Quine or Chapter 4 of Takeuti and Zaring for its
justification and methods for eliminating it.  This is an example of a
somewhat ``dangerous'' definition, because it extends the use of the
existing equality symbol rather than introducing a new symbol, allowing
us to make statements in the original language that may not be true.
For example, it permits us to deduce $y = z \leftrightarrow \forall x (
x \in y \leftrightarrow x \in z )$ which is not a theorem of logic but
rather presupposes the Axiom of Extensionality,\index{Axiom of
Extensionality} which we include as a hypothesis so that we can know
when this axiom is assumed in a proof (with the \texttt{show
trace{\char`\_}back} command).  We could avoid the danger by introducing
another symbol, say $\eqcirc$, in place of $=$; this
would also have the advantage of making elimination of the definition
straightforward and would eliminate the need for Extensionality as a
hypothesis.  We would then also have the advantage of being able to
identify exactly where Extensionality truly comes into play.  One of our
theorems would be $x \eqcirc y \leftrightarrow x = y$
by invoking Extensionality.  However in keeping with standard practice
we retain the ``dangerous'' definition.

\vskip 0.5ex
\setbox\startprefix=\hbox{\tt \ \ df-cleq.1\ \$e\ }
\setbox\contprefix=\hbox{\tt \ \ \ \ \ \ \ \ \ \ \ \ \ \ \ }
\startm
\m{\vdash}\m{(}\m{\forall}\m{x}\m{(}\m{x}\m{\in}\m{y}\m{\leftrightarrow}\m{x}
\m{\in}\m{z}\m{)}\m{\rightarrow}\m{y}\m{=}\m{z}\m{)}
\endm
\setbox\startprefix=\hbox{\tt \ \ df-cleq\ \$a\ }
\setbox\contprefix=\hbox{\tt \ \ \ \ \ \ \ \ \ \ \ \ \ }
\startm
\m{\vdash}\m{(}\m{A}\m{=}\m{B}\m{\leftrightarrow}\m{\forall}\m{x}\m{(}\m{x}\m{
\in}\m{A}\m{\leftrightarrow}\m{x}\m{\in}\m{B}\m{)}\m{)}
\endm
% We need to reset the startprefix and contprefix.
\setbox\startprefix=\hbox{\tt \ \ df-cleq.1\ \$e\ }
\setbox\contprefix=\hbox{\tt \ \ \ \ \ \ \ \ \ \ \ \ \ \ \ }
\begin{mmraw}%
|- ( A. x ( x e. y <-> x e. z ) -> y = z ) \$.
\end{mmraw}
\setbox\startprefix=\hbox{\tt \ \ df-cleq\ \$a\ }
\setbox\contprefix=\hbox{\tt \ \ \ \ \ \ \ \ \ \ \ \ \ }
\begin{mmraw}%
|- ( A = B <-> A. x ( x e. A <-> x e. B ) ) \$.
\end{mmraw}

\noindent Define the membership connective between classes\index{class
membership}.  Theorem 6.3 of Quine, p.~41, which we adopt as a definition.
Note that it extends the use of the existing membership symbol, but unlike
{\tt df-cleq} it does not extend the set of valid wffs of logic when the class
metavariables are replaced with set variables.\label{dfclel}\label{df-clel}

\vskip 0.5ex
\setbox\startprefix=\hbox{\tt \ \ df-clel\ \$a\ }
\setbox\contprefix=\hbox{\tt \ \ \ \ \ \ \ \ \ \ \ \ \ }
\startm
\m{\vdash}\m{(}\m{A}\m{\in}\m{B}\m{\leftrightarrow}\m{\exists}\m{x}\m{(}\m{x}
\m{=}\m{A}\m{\wedge}\m{x}\m{\in}\m{B}\m{)}\m{)}
\endm
\begin{mmraw}%
|- ( A e. B <-> E. x ( x = A \TAND x e. B ) ) \$.?
\end{mmraw}

\noindent Define inequality.

\vskip 0.5ex
\setbox\startprefix=\hbox{\tt \ \ df-ne\ \$a\ }
\setbox\contprefix=\hbox{\tt \ \ \ \ \ \ \ \ \ \ \ }
\startm
\m{\vdash}\m{(}\m{A}\m{\ne}\m{B}\m{\leftrightarrow}\m{\lnot}\m{A}\m{=}\m{B}%
\m{)}
\endm
\begin{mmraw}%
|- ( A =/= B <-> -. A = B ) \$.
\end{mmraw}

\noindent Define restricted universal quantification.\index{universal
quantifier ($\forall$)!restricted}  Enderton, p.~22.

\vskip 0.5ex
\setbox\startprefix=\hbox{\tt \ \ df-ral\ \$a\ }
\setbox\contprefix=\hbox{\tt \ \ \ \ \ \ \ \ \ \ \ \ }
\startm
\m{\vdash}\m{(}\m{\forall}\m{x}\m{\in}\m{A}\m{\varphi}\m{\leftrightarrow}\m{%
\forall}\m{x}\m{(}\m{x}\m{\in}\m{A}\m{\rightarrow}\m{\varphi}\m{)}\m{)}
\endm
\begin{mmraw}%
|- ( A. x e. A ph <-> A. x ( x e. A -> ph ) ) \$.
\end{mmraw}

\noindent Define restricted existential quantification.\index{existential
quantifier ($\exists$)!restricted}  Enderton, p.~22.

\vskip 0.5ex
\setbox\startprefix=\hbox{\tt \ \ df-rex\ \$a\ }
\setbox\contprefix=\hbox{\tt \ \ \ \ \ \ \ \ \ \ \ \ }
\startm
\m{\vdash}\m{(}\m{\exists}\m{x}\m{\in}\m{A}\m{\varphi}\m{\leftrightarrow}\m{%
\exists}\m{x}\m{(}\m{x}\m{\in}\m{A}\m{\wedge}\m{\varphi}\m{)}\m{)}
\endm
\begin{mmraw}%
|- ( E. x e. A ph <-> E. x ( x e. A \TAND ph ) ) \$.
\end{mmraw}

\noindent Define the universal class\index{universal class ($V$)}.  Definition
5.20, p.~21, of Takeuti and Zaring.\label{df-v}

\vskip 0.5ex
\setbox\startprefix=\hbox{\tt \ \ df-v\ \$a\ }
\setbox\contprefix=\hbox{\tt \ \ \ \ \ \ \ \ \ \ }
\startm
\m{\vdash}\m{{\rm V}}\m{=}\m{\{}\m{x}\m{|}\m{x}\m{=}\m{x}\m{\}}
\endm
\begin{mmraw}%
|- {\char`\_}V = \{ x | x = x \} \$.
\end{mmraw}

\noindent Define the subclass\index{subclass}\index{subset} relationship
between two classes (called the subset relation if the classes are sets i.e.\
are not proper).  Definition 5.9 of Takeuti and Zaring, p.~17.\label{df-ss}

\vskip 0.5ex
\setbox\startprefix=\hbox{\tt \ \ df-ss\ \$a\ }
\setbox\contprefix=\hbox{\tt \ \ \ \ \ \ \ \ \ \ \ }
\startm
\m{\vdash}\m{(}\m{A}\m{\subseteq}\m{B}\m{\leftrightarrow}\m{\forall}\m{x}\m{(}
\m{x}\m{\in}\m{A}\m{\rightarrow}\m{x}\m{\in}\m{B}\m{)}\m{)}
\endm
\begin{mmraw}%
|- ( A C\_ B <-> A. x ( x e. A -> x e. B ) ) \$.
\end{mmraw}

\noindent Define the union\index{union} of two classes.  Definition 5.6 of Takeuti and Zaring,
p.~16.\label{df-un}

\vskip 0.5ex
\setbox\startprefix=\hbox{\tt \ \ df-un\ \$a\ }
\setbox\contprefix=\hbox{\tt \ \ \ \ \ \ \ \ \ \ \ }
\startm
\m{\vdash}\m{(}\m{A}\m{\cup}\m{B}\m{)}\m{=}\m{\{}\m{x}\m{|}\m{(}\m{x}\m{\in}
\m{A}\m{\vee}\m{x}\m{\in}\m{B}\m{)}\m{\}}
\endm
\begin{mmraw}%
( A u. B ) = \{ x | ( x e. A \TOR x e. B ) \} \$.
\end{mmraw}

\noindent Define the intersection\index{intersection} of two classes.  Definition 5.6 of
Takeuti and Zaring, p.~16.\label{df-in}

\vskip 0.5ex
\setbox\startprefix=\hbox{\tt \ \ df-in\ \$a\ }
\setbox\contprefix=\hbox{\tt \ \ \ \ \ \ \ \ \ \ \ }
\startm
\m{\vdash}\m{(}\m{A}\m{\cap}\m{B}\m{)}\m{=}\m{\{}\m{x}\m{|}\m{(}\m{x}\m{\in}
\m{A}\m{\wedge}\m{x}\m{\in}\m{B}\m{)}\m{\}}
\endm
% Caret ^ requires special treatment
\begin{mmraw}%
|- ( A i\^{}i B ) = \{ x | ( x e. A \TAND x e. B ) \} \$.
\end{mmraw}

\noindent Define class difference\index{class difference}\index{set difference}.
Definition 5.12 of Takeuti and Zaring, p.~20.  Several notations are used in
the literature; we chose the $\setminus$ convention instead of a minus sign to
reserve the latter for later use in, e.g., arithmetic.\label{df-dif}

\vskip 0.5ex
\setbox\startprefix=\hbox{\tt \ \ df-dif\ \$a\ }
\setbox\contprefix=\hbox{\tt \ \ \ \ \ \ \ \ \ \ \ \ }
\startm
\m{\vdash}\m{(}\m{A}\m{\setminus}\m{B}\m{)}\m{=}\m{\{}\m{x}\m{|}\m{(}\m{x}\m{
\in}\m{A}\m{\wedge}\m{\lnot}\m{x}\m{\in}\m{B}\m{)}\m{\}}
\endm
\begin{mmraw}%
( A \SLASH B ) = \{ x | ( x e. A \TAND -. x e. B ) \} \$.
\end{mmraw}

\noindent Define the empty or null set\index{empty set}\index{null set}.
Compare  Definition 5.14 of Takeuti and Zaring, p.~20.\label{df-nul}

\vskip 0.5ex
\setbox\startprefix=\hbox{\tt \ \ df-nul\ \$a\ }
\setbox\contprefix=\hbox{\tt \ \ \ \ \ \ \ \ \ \ }
\startm
\m{\vdash}\m{\varnothing}\m{=}\m{(}\m{{\rm V}}\m{\setminus}\m{{\rm V}}\m{)}
\endm
\begin{mmraw}%
|- (/) = ( {\char`\_}V \SLASH {\char`\_}V ) \$.
\end{mmraw}

\noindent Define power class\index{power set}\index{power class}.  Definition 5.10 of
Takeuti and Zaring, p.~17, but we also let it apply to proper classes.  (Note
that \verb$~P$ is the symbol for calligraphic P, the tilde
suggesting ``curly;'' see Appendix~\ref{ASCII}.)\label{df-pw}

\vskip 0.5ex
\setbox\startprefix=\hbox{\tt \ \ df-pw\ \$a\ }
\setbox\contprefix=\hbox{\tt \ \ \ \ \ \ \ \ \ \ \ }
\startm
\m{\vdash}\m{{\cal P}}\m{A}\m{=}\m{\{}\m{x}\m{|}\m{x}\m{\subseteq}\m{A}\m{\}}
\endm
% Special incantation required to put ~ into the text
\begin{mmraw}%
|- \char`\~P~A = \{ x | x C\_ A \} \$.
\end{mmraw}

\noindent Define the singleton of a class\index{singleton}.  Definition 7.1 of
Quine, p.~48.  It is well-defined for proper classes, although
it is not very meaningful in this case, where it evaluates to the empty
set.

\vskip 0.5ex
\setbox\startprefix=\hbox{\tt \ \ df-sn\ \$a\ }
\setbox\contprefix=\hbox{\tt \ \ \ \ \ \ \ \ \ \ \ }
\startm
\m{\vdash}\m{\{}\m{A}\m{\}}\m{=}\m{\{}\m{x}\m{|}\m{x}\m{=}\m{A}\m{\}}
\endm
\begin{mmraw}%
|- \{ A \} = \{ x | x = A \} \$.
\end{mmraw}%

\noindent Define an unordered pair of classes\index{unordered pair}\index{pair}.  Definition
7.1 of Quine, p.~48.

\vskip 0.5ex
\setbox\startprefix=\hbox{\tt \ \ df-pr\ \$a\ }
\setbox\contprefix=\hbox{\tt \ \ \ \ \ \ \ \ \ \ \ }
\startm
\m{\vdash}\m{\{}\m{A}\m{,}\m{B}\m{\}}\m{=}\m{(}\m{\{}\m{A}\m{\}}\m{\cup}\m{\{}
\m{B}\m{\}}\m{)}
\endm
\begin{mmraw}%
|- \{ A , B \} = ( \{ A \} u. \{ B \} ) \$.
\end{mmraw}

\noindent Define an unordered triple of classes\index{unordered triple}.  Definition of
Enderton, p.~19.

\vskip 0.5ex
\setbox\startprefix=\hbox{\tt \ \ df-tp\ \$a\ }
\setbox\contprefix=\hbox{\tt \ \ \ \ \ \ \ \ \ \ \ }
\startm
\m{\vdash}\m{\{}\m{A}\m{,}\m{B}\m{,}\m{C}\m{\}}\m{=}\m{(}\m{\{}\m{A}\m{,}\m{B}
\m{\}}\m{\cup}\m{\{}\m{C}\m{\}}\m{)}
\endm
\begin{mmraw}%
|- \{ A , B , C \} = ( \{ A , B \} u. \{ C \} ) \$.
\end{mmraw}%

\noindent Kuratowski's\index{Kuratowski, Kazimierz} ordered pair\index{ordered
pair} definition.  Definition 9.1 of Quine, p.~58. For proper classes it is
not meaningful but is well-defined for convenience.  (Note that \verb$<.$
stands for $\langle$ whereas \verb$<$ stands for $<$, and similarly for
\verb$>.$\,.)\label{df-op}

\vskip 0.5ex
\setbox\startprefix=\hbox{\tt \ \ df-op\ \$a\ }
\setbox\contprefix=\hbox{\tt \ \ \ \ \ \ \ \ \ \ \ }
\startm
\m{\vdash}\m{\langle}\m{A}\m{,}\m{B}\m{\rangle}\m{=}\m{\{}\m{\{}\m{A}\m{\}}
\m{,}\m{\{}\m{A}\m{,}\m{B}\m{\}}\m{\}}
\endm
\begin{mmraw}%
|- <. A , B >. = \{ \{ A \} , \{ A , B \} \} \$.
\end{mmraw}

\noindent Define the union of a class\index{union}.  Definition 5.5, p.~16,
of Takeuti and Zaring.

\vskip 0.5ex
\setbox\startprefix=\hbox{\tt \ \ df-uni\ \$a\ }
\setbox\contprefix=\hbox{\tt \ \ \ \ \ \ \ \ \ \ \ \ }
\startm
\m{\vdash}\m{\bigcup}\m{A}\m{=}\m{\{}\m{x}\m{|}\m{\exists}\m{y}\m{(}\m{x}\m{
\in}\m{y}\m{\wedge}\m{y}\m{\in}\m{A}\m{)}\m{\}}
\endm
\begin{mmraw}%
|- U. A = \{ x | E. y ( x e. y \TAND y e. A ) \} \$.
\end{mmraw}

\noindent Define the intersection\index{intersection} of a class.  Definition 7.35,
p.~44, of Takeuti and Zaring.

\vskip 0.5ex
\setbox\startprefix=\hbox{\tt \ \ df-int\ \$a\ }
\setbox\contprefix=\hbox{\tt \ \ \ \ \ \ \ \ \ \ \ \ }
\startm
\m{\vdash}\m{\bigcap}\m{A}\m{=}\m{\{}\m{x}\m{|}\m{\forall}\m{y}\m{(}\m{y}\m{
\in}\m{A}\m{\rightarrow}\m{x}\m{\in}\m{y}\m{)}\m{\}}
\endm
\begin{mmraw}%
|- |\^{}| A = \{ x | A. y ( y e. A -> x e. y ) \} \$.
\end{mmraw}

\noindent Define a transitive class\index{transitive class}\index{transitive
set}.  This should not be confused with a transitive relation, which is a different
concept.  Definition from p.~71 of Enderton, extended to classes.

\vskip 0.5ex
\setbox\startprefix=\hbox{\tt \ \ df-tr\ \$a\ }
\setbox\contprefix=\hbox{\tt \ \ \ \ \ \ \ \ \ \ \ }
\startm
\m{\vdash}\m{(}\m{\mbox{\rm Tr}}\m{A}\m{\leftrightarrow}\m{\bigcup}\m{A}\m{
\subseteq}\m{A}\m{)}
\endm
\begin{mmraw}%
|- ( Tr A <-> U. A C\_ A ) \$.
\end{mmraw}

\noindent Define a notation for a general binary relation\index{binary
relation}.  Definition 6.18, p.~29, of Takeuti and Zaring, generalized to
arbitrary classes.  This definition is well-defined, although not very
meaningful, when classes $A$ and/or $B$ are proper.\label{dfbr}  The lack of
parentheses (or any other connective) creates no ambiguity since we are defining
an atomic wff.

\vskip 0.5ex
\setbox\startprefix=\hbox{\tt \ \ df-br\ \$a\ }
\setbox\contprefix=\hbox{\tt \ \ \ \ \ \ \ \ \ \ \ }
\startm
\m{\vdash}\m{(}\m{A}\m{\,R}\m{\,B}\m{\leftrightarrow}\m{\langle}\m{A}\m{,}\m{B}
\m{\rangle}\m{\in}\m{R}\m{)}
\endm
\begin{mmraw}%
|- ( A R B <-> <. A , B >. e. R ) \$.
\end{mmraw}

\noindent Define an abstraction class of ordered pairs\index{abstraction
class!of ordered
pairs}.  A special case of Definition 4.16, p.~14, of Takeuti and Zaring.
Note that $ z $ must be distinct from $ x $ and $ y $,
and $ z $ must not occur in $\varphi$, but $ x $ and $ y $ may be
identical and may appear in $\varphi$.

\vskip 0.5ex
\setbox\startprefix=\hbox{\tt \ \ df-opab\ \$a\ }
\setbox\contprefix=\hbox{\tt \ \ \ \ \ \ \ \ \ \ \ \ \ }
\startm
\m{\vdash}\m{\{}\m{\langle}\m{x}\m{,}\m{y}\m{\rangle}\m{|}\m{\varphi}\m{\}}\m{=}
\m{\{}\m{z}\m{|}\m{\exists}\m{x}\m{\exists}\m{y}\m{(}\m{z}\m{=}\m{\langle}\m{x}
\m{,}\m{y}\m{\rangle}\m{\wedge}\m{\varphi}\m{)}\m{\}}
\endm

\begin{mmraw}%
|- \{ <. x , y >. | ph \} = \{ z | E. x E. y ( z =
<. x , y >. /\ ph ) \} \$.
\end{mmraw}

\noindent Define the epsilon relation\index{epsilon relation}.  Similar to Definition
6.22, p.~30, of Takeuti and Zaring.

\vskip 0.5ex
\setbox\startprefix=\hbox{\tt \ \ df-eprel\ \$a\ }
\setbox\contprefix=\hbox{\tt \ \ \ \ \ \ \ \ \ \ \ \ \ \ }
\startm
\m{\vdash}\m{{\rm E}}\m{=}\m{\{}\m{\langle}\m{x}\m{,}\m{y}\m{\rangle}\m{|}\m{x}\m{
\in}\m{y}\m{\}}
\endm
\begin{mmraw}%
|- \_E = \{ <. x , y >. | x e. y \} \$.
\end{mmraw}

\noindent Define a founded relation\index{founded relation}.  $R$ is a founded
relation on $A$ iff\index{iff} (if and only if) each nonempty subset of $A$
has an ``$R$-minimal element.''  Similar to Definition 6.21, p.~30, of
Takeuti and Zaring.

\vskip 0.5ex
\setbox\startprefix=\hbox{\tt \ \ df-fr\ \$a\ }
\setbox\contprefix=\hbox{\tt \ \ \ \ \ \ \ \ \ \ \ }
\startm
\m{\vdash}\m{(}\m{R}\m{\,\mbox{\rm Fr}}\m{\,A}\m{\leftrightarrow}\m{\forall}\m{x}
\m{(}\m{(}\m{x}\m{\subseteq}\m{A}\m{\wedge}\m{\lnot}\m{x}\m{=}\m{\varnothing}
\m{)}\m{\rightarrow}\m{\exists}\m{y}\m{(}\m{y}\m{\in}\m{x}\m{\wedge}\m{(}\m{x}
\m{\cap}\m{\{}\m{z}\m{|}\m{z}\m{\,R}\m{\,y}\m{\}}\m{)}\m{=}\m{\varnothing}\m{)}
\m{)}\m{)}
\endm
\begin{mmraw}%
|- ( R Fr A <-> A. x ( ( x C\_ A \TAND -. x = (/) ) ->
E. y ( y e. x \TAND ( x i\^{}i \{ z | z R y \} ) = (/) ) ) ) \$.
\end{mmraw}

\noindent Define a well-ordering\index{well-ordering}.  $R$ is a well-ordering of $A$ iff
it is founded on $A$ and the elements of $A$ are pairwise $R$-comparable.
Similar to Definition 6.24(2), p.~30, of Takeuti and Zaring.

\vskip 0.5ex
\setbox\startprefix=\hbox{\tt \ \ df-we\ \$a\ }
\setbox\contprefix=\hbox{\tt \ \ \ \ \ \ \ \ \ \ \ }
\startm
\m{\vdash}\m{(}\m{R}\m{\,\mbox{\rm We}}\m{\,A}\m{\leftrightarrow}\m{(}\m{R}\m{\,
\mbox{\rm Fr}}\m{\,A}\m{\wedge}\m{\forall}\m{x}\m{\forall}\m{y}\m{(}\m{(}\m{x}\m{
\in}\m{A}\m{\wedge}\m{y}\m{\in}\m{A}\m{)}\m{\rightarrow}\m{(}\m{x}\m{\,R}\m{\,y}
\m{\vee}\m{x}\m{=}\m{y}\m{\vee}\m{y}\m{\,R}\m{\,x}\m{)}\m{)}\m{)}\m{)}
\endm
\begin{mmraw}%
( R We A <-> ( R Fr A \TAND A. x A. y ( ( x e.
A \TAND y e. A ) -> ( x R y \TOR x = y \TOR y R x ) ) ) ) \$.
\end{mmraw}

\noindent Define the ordinal predicate\index{ordinal predicate}, which is true for a class
that is transitive and is well-ordered by the epsilon relation.  Similar to
definition on p.~468, Bell and Machover.

\vskip 0.5ex
\setbox\startprefix=\hbox{\tt \ \ df-ord\ \$a\ }
\setbox\contprefix=\hbox{\tt \ \ \ \ \ \ \ \ \ \ \ \ }
\startm
\m{\vdash}\m{(}\m{\mbox{\rm Ord}}\m{\,A}\m{\leftrightarrow}\m{(}
\m{\mbox{\rm Tr}}\m{\,A}\m{\wedge}\m{E}\m{\,\mbox{\rm We}}\m{\,A}\m{)}\m{)}
\endm
\begin{mmraw}%
|- ( Ord A <-> ( Tr A \TAND E We A ) ) \$.
\end{mmraw}

\noindent Define the class of all ordinal numbers\index{ordinal number}.  An ordinal number is
a set that satisfies the ordinal predicate.  Definition 7.11 of Takeuti and
Zaring, p.~38.

\vskip 0.5ex
\setbox\startprefix=\hbox{\tt \ \ df-on\ \$a\ }
\setbox\contprefix=\hbox{\tt \ \ \ \ \ \ \ \ \ \ \ }
\startm
\m{\vdash}\m{\,\mbox{\rm On}}\m{=}\m{\{}\m{x}\m{|}\m{\mbox{\rm Ord}}\m{\,x}
\m{\}}
\endm
\begin{mmraw}%
|- On = \{ x | Ord x \} \$.
\end{mmraw}

\noindent Define the limit ordinal predicate\index{limit ordinal}, which is true for a
non-empty ordinal that is not a successor (i.e.\ that is the union of itself).
Compare Bell and Machover, p.~471 and Exercise (1), p.~42 of Takeuti and
Zaring.

\vskip 0.5ex
\setbox\startprefix=\hbox{\tt \ \ df-lim\ \$a\ }
\setbox\contprefix=\hbox{\tt \ \ \ \ \ \ \ \ \ \ \ \ }
\startm
\m{\vdash}\m{(}\m{\mbox{\rm Lim}}\m{\,A}\m{\leftrightarrow}\m{(}\m{\mbox{
\rm Ord}}\m{\,A}\m{\wedge}\m{\lnot}\m{A}\m{=}\m{\varnothing}\m{\wedge}\m{A}
\m{=}\m{\bigcup}\m{A}\m{)}\m{)}
\endm
\begin{mmraw}%
|- ( Lim A <-> ( Ord A \TAND -. A = (/) \TAND A = U. A ) ) \$.
\end{mmraw}

\noindent Define the successor\index{successor} of a class.  Definition 7.22 of Takeuti
and Zaring, p.~41.  Our definition is a generalization to classes, although it
is meaningless when classes are proper.

\vskip 0.5ex
\setbox\startprefix=\hbox{\tt \ \ df-suc\ \$a\ }
\setbox\contprefix=\hbox{\tt \ \ \ \ \ \ \ \ \ \ \ \ }
\startm
\m{\vdash}\m{\,\mbox{\rm suc}}\m{\,A}\m{=}\m{(}\m{A}\m{\cup}\m{\{}\m{A}\m{\}}
\m{)}
\endm
\begin{mmraw}%
|- suc A = ( A u. \{ A \} ) \$.
\end{mmraw}

\noindent Define the class of natural numbers\index{natural number}\index{omega
($\omega$)}.  Compare Bell and Machover, p.~471.\label{dfom}

\vskip 0.5ex
\setbox\startprefix=\hbox{\tt \ \ df-om\ \$a\ }
\setbox\contprefix=\hbox{\tt \ \ \ \ \ \ \ \ \ \ \ }
\startm
\m{\vdash}\m{\omega}\m{=}\m{\{}\m{x}\m{|}\m{(}\m{\mbox{\rm Ord}}\m{\,x}\m{
\wedge}\m{\forall}\m{y}\m{(}\m{\mbox{\rm Lim}}\m{\,y}\m{\rightarrow}\m{x}\m{
\in}\m{y}\m{)}\m{)}\m{\}}
\endm
\begin{mmraw}%
|- om = \{ x | ( Ord x \TAND A. y ( Lim y -> x e. y ) ) \} \$.
\end{mmraw}

\noindent Define the Cartesian product (also called the
cross product)\index{Cartesian product}\index{cross product}
of two classes.  Definition 9.11 of Quine, p.~64.

\vskip 0.5ex
\setbox\startprefix=\hbox{\tt \ \ df-xp\ \$a\ }
\setbox\contprefix=\hbox{\tt \ \ \ \ \ \ \ \ \ \ \ }
\startm
\m{\vdash}\m{(}\m{A}\m{\times}\m{B}\m{)}\m{=}\m{\{}\m{\langle}\m{x}\m{,}\m{y}
\m{\rangle}\m{|}\m{(}\m{x}\m{\in}\m{A}\m{\wedge}\m{y}\m{\in}\m{B}\m{)}\m{\}}
\endm
\begin{mmraw}%
|- ( A X. B ) = \{ <. x , y >. | ( x e. A \TAND y e. B) \} \$.
\end{mmraw}

\noindent Define a relation\index{relation}.  Definition 6.4(1) of Takeuti and
Zaring, p.~23.

\vskip 0.5ex
\setbox\startprefix=\hbox{\tt \ \ df-rel\ \$a\ }
\setbox\contprefix=\hbox{\tt \ \ \ \ \ \ \ \ \ \ \ \ }
\startm
\m{\vdash}\m{(}\m{\mbox{\rm Rel}}\m{\,A}\m{\leftrightarrow}\m{A}\m{\subseteq}
\m{(}\m{{\rm V}}\m{\times}\m{{\rm V}}\m{)}\m{)}
\endm
\begin{mmraw}%
|- ( Rel A <-> A C\_ ( {\char`\_}V X. {\char`\_}V ) ) \$.
\end{mmraw}

\noindent Define the domain\index{domain} of a class.  Definition 6.5(1) of
Takeuti and Zaring, p.~24.

\vskip 0.5ex
\setbox\startprefix=\hbox{\tt \ \ df-dm\ \$a\ }
\setbox\contprefix=\hbox{\tt \ \ \ \ \ \ \ \ \ \ \ }
\startm
\m{\vdash}\m{\,\mbox{\rm dom}}\m{A}\m{=}\m{\{}\m{x}\m{|}\m{\exists}\m{y}\m{
\langle}\m{x}\m{,}\m{y}\m{\rangle}\m{\in}\m{A}\m{\}}
\endm
\begin{mmraw}%
|- dom A = \{ x | E. y <. x , y >. e. A \} \$.
\end{mmraw}

\noindent Define the range\index{range} of a class.  Definition 6.5(2) of
Takeuti and Zaring, p.~24.

\vskip 0.5ex
\setbox\startprefix=\hbox{\tt \ \ df-rn\ \$a\ }
\setbox\contprefix=\hbox{\tt \ \ \ \ \ \ \ \ \ \ \ }
\startm
\m{\vdash}\m{\,\mbox{\rm ran}}\m{A}\m{=}\m{\{}\m{y}\m{|}\m{\exists}\m{x}\m{
\langle}\m{x}\m{,}\m{y}\m{\rangle}\m{\in}\m{A}\m{\}}
\endm
\begin{mmraw}%
|- ran A = \{ y | E. x <. x , y >. e. A \} \$.
\end{mmraw}

\noindent Define the restriction\index{restriction} of a class.  Definition
6.6(1) of Takeuti and Zaring, p.~24.

\vskip 0.5ex
\setbox\startprefix=\hbox{\tt \ \ df-res\ \$a\ }
\setbox\contprefix=\hbox{\tt \ \ \ \ \ \ \ \ \ \ \ \ }
\startm
\m{\vdash}\m{(}\m{A}\m{\restriction}\m{B}\m{)}\m{=}\m{(}\m{A}\m{\cap}\m{(}\m{B}
\m{\times}\m{{\rm V}}\m{)}\m{)}
\endm
\begin{mmraw}%
|- ( A |` B ) = ( A i\^{}i ( B X. {\char`\_}V ) ) \$.
\end{mmraw}

\noindent Define the image\index{image} of a class.  Definition 6.6(2) of
Takeuti and Zaring, p.~24.

\vskip 0.5ex
\setbox\startprefix=\hbox{\tt \ \ df-ima\ \$a\ }
\setbox\contprefix=\hbox{\tt \ \ \ \ \ \ \ \ \ \ \ \ }
\startm
\m{\vdash}\m{(}\m{A}\m{``}\m{B}\m{)}\m{=}\m{\,\mbox{\rm ran}}\m{\,(}\m{A}\m{
\restriction}\m{B}\m{)}
\endm
\begin{mmraw}%
|- ( A " B ) = ran ( A |` B ) \$.
\end{mmraw}

\noindent Define the composition\index{composition} of two classes.  Definition 6.6(3) of
Takeuti and Zaring, p.~24.

\vskip 0.5ex
\setbox\startprefix=\hbox{\tt \ \ df-co\ \$a\ }
\setbox\contprefix=\hbox{\tt \ \ \ \ \ \ \ \ \ \ \ \ }
\startm
\m{\vdash}\m{(}\m{A}\m{\circ}\m{B}\m{)}\m{=}\m{\{}\m{\langle}\m{x}\m{,}\m{y}\m{
\rangle}\m{|}\m{\exists}\m{z}\m{(}\m{\langle}\m{x}\m{,}\m{z}\m{\rangle}\m{\in}
\m{B}\m{\wedge}\m{\langle}\m{z}\m{,}\m{y}\m{\rangle}\m{\in}\m{A}\m{)}\m{\}}
\endm
\begin{mmraw}%
|- ( A o. B ) = \{ <. x , y >. | E. z ( <. x , z
>. e. B \TAND <. z , y >. e. A ) \} \$.
\end{mmraw}

\noindent Define a function\index{function}.  Definition 6.4(4) of Takeuti and
Zaring, p.~24.

\vskip 0.5ex
\setbox\startprefix=\hbox{\tt \ \ df-fun\ \$a\ }
\setbox\contprefix=\hbox{\tt \ \ \ \ \ \ \ \ \ \ \ \ }
\startm
\m{\vdash}\m{(}\m{\mbox{\rm Fun}}\m{\,A}\m{\leftrightarrow}\m{(}
\m{\mbox{\rm Rel}}\m{\,A}\m{\wedge}
\m{\forall}\m{x}\m{\exists}\m{z}\m{\forall}\m{y}\m{(}
\m{\langle}\m{x}\m{,}\m{y}\m{\rangle}\m{\in}\m{A}\m{\rightarrow}\m{y}\m{=}\m{z}
\m{)}\m{)}\m{)}
\endm
\begin{mmraw}%
|- ( Fun A <-> ( Rel A /\ A. x E. z A. y ( <. x
   , y >. e. A -> y = z ) ) ) \$.
\end{mmraw}

\noindent Define a function with domain.  Definition 6.15(1) of Takeuti and
Zaring, p.~27.

\vskip 0.5ex
\setbox\startprefix=\hbox{\tt \ \ df-fn\ \$a\ }
\setbox\contprefix=\hbox{\tt \ \ \ \ \ \ \ \ \ \ \ }
\startm
\m{\vdash}\m{(}\m{A}\m{\,\mbox{\rm Fn}}\m{\,B}\m{\leftrightarrow}\m{(}
\m{\mbox{\rm Fun}}\m{\,A}\m{\wedge}\m{\mbox{\rm dom}}\m{\,A}\m{=}\m{B}\m{)}
\m{)}
\endm
\begin{mmraw}%
|- ( A Fn B <-> ( Fun A \TAND dom A = B ) ) \$.
\end{mmraw}

\noindent Define a function with domain and co-domain.  Definition 6.15(3)
of Takeuti and Zaring, p.~27.

\vskip 0.5ex
\setbox\startprefix=\hbox{\tt \ \ df-f\ \$a\ }
\setbox\contprefix=\hbox{\tt \ \ \ \ \ \ \ \ \ \ }
\startm
\m{\vdash}\m{(}\m{F}\m{:}\m{A}\m{\longrightarrow}\m{B}\m{
\leftrightarrow}\m{(}\m{F}\m{\,\mbox{\rm Fn}}\m{\,A}\m{\wedge}\m{
\mbox{\rm ran}}\m{\,F}\m{\subseteq}\m{B}\m{)}\m{)}
\endm
\begin{mmraw}%
|- ( F : A --> B <-> ( F Fn A \TAND ran F C\_ B ) ) \$.
\end{mmraw}

\noindent Define a one-to-one function\index{one-to-one function}.  Compare
Definition 6.15(5) of Takeuti and Zaring, p.~27.

\vskip 0.5ex
\setbox\startprefix=\hbox{\tt \ \ df-f1\ \$a\ }
\setbox\contprefix=\hbox{\tt \ \ \ \ \ \ \ \ \ \ \ }
\startm
\m{\vdash}\m{(}\m{F}\m{:}\m{A}\m{
\raisebox{.5ex}{${\textstyle{\:}_{\mbox{\footnotesize\rm
1\tt -\rm 1}}}\atop{\textstyle{
\longrightarrow}\atop{\textstyle{}^{\mbox{\footnotesize\rm {\ }}}}}$}
}\m{B}
\m{\leftrightarrow}\m{(}\m{F}\m{:}\m{A}\m{\longrightarrow}\m{B}
\m{\wedge}\m{\forall}\m{y}\m{\exists}\m{z}\m{\forall}\m{x}\m{(}\m{\langle}\m{x}
\m{,}\m{y}\m{\rangle}\m{\in}\m{F}\m{\rightarrow}\m{x}\m{=}\m{z}\m{)}\m{)}\m{)}
\endm
\begin{mmraw}%
|- ( F : A -1-1-> B <-> ( F : A --> B \TAND
   A. y E. z A. x ( <. x , y >. e. F -> x = z ) ) ) \$.
\end{mmraw}

\noindent Define an onto function\index{onto function}.  Definition 6.15(4) of Takeuti and
Zaring, p.~27.

\vskip 0.5ex
\setbox\startprefix=\hbox{\tt \ \ df-fo\ \$a\ }
\setbox\contprefix=\hbox{\tt \ \ \ \ \ \ \ \ \ \ \ }
\startm
\m{\vdash}\m{(}\m{F}\m{:}\m{A}\m{
\raisebox{.5ex}{${\textstyle{\:}_{\mbox{\footnotesize\rm
{\ }}}}\atop{\textstyle{
\longrightarrow}\atop{\textstyle{}^{\mbox{\footnotesize\rm onto}}}}$}
}\m{B}
\m{\leftrightarrow}\m{(}\m{F}\m{\,\mbox{\rm Fn}}\m{\,A}\m{\wedge}
\m{\mbox{\rm ran}}\m{\,F}\m{=}\m{B}\m{)}\m{)}
\endm
\begin{mmraw}%
|- ( F : A -onto-> B <-> ( F Fn A /\ ran F = B ) ) \$.
\end{mmraw}

\noindent Define a one-to-one, onto function.  Compare Definition 6.15(6) of
Takeuti and Zaring, p.~27.

\vskip 0.5ex
\setbox\startprefix=\hbox{\tt \ \ df-f1o\ \$a\ }
\setbox\contprefix=\hbox{\tt \ \ \ \ \ \ \ \ \ \ \ \ }
\startm
\m{\vdash}\m{(}\m{F}\m{:}\m{A}
\m{
\raisebox{.5ex}{${\textstyle{\:}_{\mbox{\footnotesize\rm
1\tt -\rm 1}}}\atop{\textstyle{
\longrightarrow}\atop{\textstyle{}^{\mbox{\footnotesize\rm onto}}}}$}
}
\m{B}
\m{\leftrightarrow}\m{(}\m{F}\m{:}\m{A}
\m{
\raisebox{.5ex}{${\textstyle{\:}_{\mbox{\footnotesize\rm
1\tt -\rm 1}}}\atop{\textstyle{
\longrightarrow}\atop{\textstyle{}^{\mbox{\footnotesize\rm {\ }}}}}$}
}
\m{B}\m{\wedge}\m{F}\m{:}\m{A}
\m{
\raisebox{.5ex}{${\textstyle{\:}_{\mbox{\footnotesize\rm
{\ }}}}\atop{\textstyle{
\longrightarrow}\atop{\textstyle{}^{\mbox{\footnotesize\rm onto}}}}$}
}
\m{B}\m{)}\m{)}
\endm
\begin{mmraw}%
|- ( F : A -1-1-onto-> B <-> ( F : A -1-1-> B? \TAND F : A -onto-> B ) ) \$.?
\end{mmraw}

\noindent Define the value of a function\index{function value}.  This
definition applies to any class and evaluates to the empty set when it is not
meaningful. Note that $ F`A$ means the same thing as the more familiar $ F(A)$
notation for a function's value at $A$.  The $ F`A$ notation is common in
formal set theory.\label{df-fv}

\vskip 0.5ex
\setbox\startprefix=\hbox{\tt \ \ df-fv\ \$a\ }
\setbox\contprefix=\hbox{\tt \ \ \ \ \ \ \ \ \ \ \ }
\startm
\m{\vdash}\m{(}\m{F}\m{`}\m{A}\m{)}\m{=}\m{\bigcup}\m{\{}\m{x}\m{|}\m{(}\m{F}%
\m{``}\m{\{}\m{A}\m{\}}\m{)}\m{=}\m{\{}\m{x}\m{\}}\m{\}}
\endm
\begin{mmraw}%
|- ( F ` A ) = U. \{ x | ( F " \{ A \} ) = \{ x \} \} \$.
\end{mmraw}

\noindent Define the result of an operation.\index{operation}  Here, $F$ is
     an operation on two
     values (such as $+$ for real numbers).   This is defined for proper
     classes $A$ and $B$ even though not meaningful in that case.  However,
     the definition can be meaningful when $F$ is a proper class.\label{dfopr}

\vskip 0.5ex
\setbox\startprefix=\hbox{\tt \ \ df-opr\ \$a\ }
\setbox\contprefix=\hbox{\tt \ \ \ \ \ \ \ \ \ \ \ \ }
\startm
\m{\vdash}\m{(}\m{A}\m{\,F}\m{\,B}\m{)}\m{=}\m{(}\m{F}\m{`}\m{\langle}\m{A}%
\m{,}\m{B}\m{\rangle}\m{)}
\endm
\begin{mmraw}%
|- ( A F B ) = ( F ` <. A , B >. ) \$.
\end{mmraw}

\section{Tricks of the Trade}\label{tricks}

In the \texttt{set.mm}\index{set theory database (\texttt{set.mm})} database our goal
was usually to conform to modern notation.  However in some cases the
relationship to standard textbook language may be obscured by several
unconventional devices we used to simplify the development and to take
advantage of the Metamath language.  In this section we will describe some
common conventions used in \texttt{set.mm}.

\begin{itemize}
\item
The turnstile\index{turnstile ({$\,\vdash$})} symbol, $\vdash$, meaning ``it
is provable that,'' is the first token of all assertions and hypotheses that
aren't syntax constructions.  This is a standard convention in logic.  (We
mentioned this earlier, but this symbol is bothersome to some people without a
logic background.  It has no deeper meaning but just provides us with a way to
distinguish syntax constructions from ordinary mathematical statements.)

\item
A hypothesis of the form

\vskip 1ex
\setbox\startprefix=\hbox{\tt \ \ \ \ \ \ \ \ \ \$e\ }
\setbox\contprefix=\hbox{\tt \ \ \ \ \ \ \ \ \ \ }
\startm
\m{\vdash}\m{(}\m{\varphi}\m{\rightarrow}\m{\forall}\m{x}\m{\varphi}\m{)}
\endm
\vskip 1ex

should be read ``assume variable $x$ is (effectively) not free in wff
$\varphi$.''\index{effectively not free}
Literally, this says ``assume it is provable that $\varphi \rightarrow \forall
x\, \varphi$.''  This device lets us avoid the complexities associated with
the standard treatment of free and bound variables.
%% Uncomment this when uncommenting section {formalspec} below
The footnote on p.~\pageref{effectivelybound} discusses this further.

\item
A statement of one of the forms

\vskip 1ex
\setbox\startprefix=\hbox{\tt \ \ \ \ \ \ \ \ \ \$a\ }
\setbox\contprefix=\hbox{\tt \ \ \ \ \ \ \ \ \ \ }
\startm
\m{\vdash}\m{(}\m{\lnot}\m{\forall}\m{x}\m{\,x}\m{=}\m{y}\m{\rightarrow}
\m{\ldots}\m{)}
\endm
\setbox\startprefix=\hbox{\tt \ \ \ \ \ \ \ \ \ \$p\ }
\setbox\contprefix=\hbox{\tt \ \ \ \ \ \ \ \ \ \ }
\startm
\m{\vdash}\m{(}\m{\lnot}\m{\forall}\m{x}\m{\,x}\m{=}\m{y}\m{\rightarrow}
\m{\ldots}\m{)}
\endm
\vskip 1ex

should be read ``if $x$ and $y$ are distinct variables, then...''  This
antecedent provides us with a technical device to avoid the need for the
\texttt{\$d} statement early in our development of predicate calculus,
permitting symbol manipulations to be as conceptually simple as those in
propositional calculus.  However, the \texttt{\$d} statement eventually
becomes a requirement, and after that this device is rarely used.

\item
The statement

\vskip 1ex
\setbox\startprefix=\hbox{\tt \ \ \ \ \ \ \ \ \ \$d\ }
\setbox\contprefix=\hbox{\tt \ \ \ \ \ \ \ \ \ \ }
\startm
\m{x}\m{\,y}
\endm
\vskip 1ex

should be read ``assume $x$ and $y$ are distinct variables.''

\item
The statement

\vskip 1ex
\setbox\startprefix=\hbox{\tt \ \ \ \ \ \ \ \ \ \$d\ }
\setbox\contprefix=\hbox{\tt \ \ \ \ \ \ \ \ \ \ }
\startm
\m{x}\m{\,\varphi}
\endm
\vskip 1ex

should be read ``assume $x$ does not occur in $\varphi$.''

\item
The statement

\vskip 1ex
\setbox\startprefix=\hbox{\tt \ \ \ \ \ \ \ \ \ \$d\ }
\setbox\contprefix=\hbox{\tt \ \ \ \ \ \ \ \ \ \ }
\startm
\m{x}\m{\,A}
\endm
\vskip 1ex

should be read ``assume variable $x$ does not occur in class $A$.''

\item
The restriction and hypothesis group

\vskip 1ex
\setbox\startprefix=\hbox{\tt \ \ \ \ \ \ \ \ \ \$d\ }
\setbox\contprefix=\hbox{\tt \ \ \ \ \ \ \ \ \ \ }
\startm
\m{x}\m{\,A}
\endm
\setbox\startprefix=\hbox{\tt \ \ \ \ \ \ \ \ \ \$d\ }
\setbox\contprefix=\hbox{\tt \ \ \ \ \ \ \ \ \ \ }
\startm
\m{x}\m{\,\psi}
\endm
\setbox\startprefix=\hbox{\tt \ \ \ \ \ \ \ \ \ \$e\ }
\setbox\contprefix=\hbox{\tt \ \ \ \ \ \ \ \ \ \ }
\startm
\m{\vdash}\m{(}\m{x}\m{=}\m{A}\m{\rightarrow}\m{(}\m{\varphi}\m{\leftrightarrow}
\m{\psi}\m{)}\m{)}
\endm
\vskip 1ex

is frequently used in place of explicit substitution, meaning ``assume
$\psi$ results from the proper substitution of $A$ for $x$ in
$\varphi$.''  Sometimes ``\texttt{\$e} $\vdash ( \psi \rightarrow
\forall x \, \psi )$'' is used instead of ``\texttt{\$d} $x\, \psi $,''
which requires only that $x$ be effectively not free in $\varphi$ but
not necessarily absent from it.  The use of implicit
substitution\index{substitution!implicit} is partly a matter of personal
style, although it may make proofs somewhat shorter than would be the
case with explicit substitution.

\item
The hypothesis


\vskip 1ex
\setbox\startprefix=\hbox{\tt \ \ \ \ \ \ \ \ \ \$e\ }
\setbox\contprefix=\hbox{\tt \ \ \ \ \ \ \ \ \ \ }
\startm
\m{\vdash}\m{A}\m{\in}\m{{\rm V}}
\endm
\vskip 1ex

should be read ``assume class $A$ is a set (i.e.\ exists).''
This is a convenient convention used by Quine.

\item
The restriction and hypothesis

\vskip 1ex
\setbox\startprefix=\hbox{\tt \ \ \ \ \ \ \ \ \ \$d\ }
\setbox\contprefix=\hbox{\tt \ \ \ \ \ \ \ \ \ \ }
\startm
\m{x}\m{\,y}
\endm
\setbox\startprefix=\hbox{\tt \ \ \ \ \ \ \ \ \ \$e\ }
\setbox\contprefix=\hbox{\tt \ \ \ \ \ \ \ \ \ \ }
\startm
\m{\vdash}\m{(}\m{y}\m{\in}\m{A}\m{\rightarrow}\m{\forall}\m{x}\m{\,y}
\m{\in}\m{A}\m{)}
\endm
\vskip 1ex

should be read ``assume variable $x$ is
(effectively) not free in class $A$.''

\end{itemize}

\section{A Theorem Sampler}\label{sometheorems}

In this section we list some of the more important theorems that are proved in
the \texttt{set.mm} database, and they illustrate the kinds of things that can be
done with Metamath.  While all of these facts are well-known results,
Metamath offers the advantage of easily allowing you to trace their
derivation back to axioms.  Our intent here is not to try to explain the
details or motivation; for this we refer you to the textbooks that are
mentioned in the descriptions.  (The \texttt{set.mm} file has bibliographic
references for the text references.)  Their proofs often embody important
concepts you may wish to explore with the Metamath program (see
Section~\ref{exploring}).  All the symbols that are used here are defined in
Section~\ref{hierarchy}.  For brevity we haven't included the \texttt{\$d}
restrictions or \texttt{\$f} hypotheses for these theorems; when you are
uncertain consult the \texttt{set.mm} database.

We start with \texttt{syl} (principle of the syllogism).
In \textit{Principia Mathematica}
Whitehead and Russell call this ``the principle of the
syllogism... because... the syllogism in Barbara is derived from them''
\cite[quote after Theorem *2.06 p.~101]{PM}.
Some authors call this law a ``hypothetical syllogism.''
At the time of this writing it is the most commonly referenced assertion in the
\texttt{set.mm} database.\footnote{
The Metamath program command \texttt{show usage}
shows the number of references.
}

\vskip 2ex
\noindent Theorem syl (principle of the syllogism)\index{Syllogism}%
\index{\texttt{syl}}\label{syl}.

\vskip 0.5ex
\setbox\startprefix=\hbox{\tt \ \ syl.1\ \$e\ }
\setbox\contprefix=\hbox{\tt \ \ \ \ \ \ \ \ \ \ \ }
\startm
\m{\vdash}\m{(}\m{\varphi}\m{ \rightarrow }\m{\psi}\m{)}
\endm
\setbox\startprefix=\hbox{\tt \ \ syl.2\ \$e\ }
\setbox\contprefix=\hbox{\tt \ \ \ \ \ \ \ \ \ \ \ }
\startm
\m{\vdash}\m{(}\m{\psi}\m{ \rightarrow }\m{\chi}\m{)}
\endm
\setbox\startprefix=\hbox{\tt \ \ syl\ \$p\ }
\setbox\contprefix=\hbox{\tt \ \ \ \ \ \ \ \ \ }
\startm
\m{\vdash}\m{(}\m{\varphi}\m{ \rightarrow }\m{\chi}\m{)}
\endm
\vskip 2ex

The following theorem is not very deep but provides us with a notational device
that is frequently used.  It allows us to use the expression ``$A \in V$'' as
a compact way of saying that class $A$ exists, i.e.\ is a set.

\vskip 2ex
\noindent Two ways to say ``$A$ is a set'':  $A$ is a member of the universe
$V$ if and only if $A$ exists (i.e.\ there exists a set equal to $A$).
Theorem 6.9 of Quine, p. 43.

\vskip 0.5ex
\setbox\startprefix=\hbox{\tt \ \ isset\ \$p\ }
\setbox\contprefix=\hbox{\tt \ \ \ \ \ \ \ \ \ \ \ }
\startm
\m{\vdash}\m{(}\m{A}\m{\in}\m{{\rm V}}\m{\leftrightarrow}\m{\exists}\m{x}\m{\,x}\m{=}
\m{A}\m{)}
\endm
\vskip 1ex

Next we prove the axioms of standard ZF set theory that were missing from our
axiom system.  From our point of view they are theorems since they
can be derived from the other axioms.

\vskip 2ex
\noindent Axiom of Separation\index{Axiom of Separation}
(Aussonderung)\index{Aussonderung} proved from the other axioms of ZF set
theory.  Compare Exercise 4 of Takeuti and Zaring, p.~22.

\vskip 0.5ex
\setbox\startprefix=\hbox{\tt \ \ inex1.1\ \$e\ }
\setbox\contprefix=\hbox{\tt \ \ \ \ \ \ \ \ \ \ \ \ \ \ \ }
\startm
\m{\vdash}\m{A}\m{\in}\m{{\rm V}}
\endm
\setbox\startprefix=\hbox{\tt \ \ inex\ \$p\ }
\setbox\contprefix=\hbox{\tt \ \ \ \ \ \ \ \ \ \ \ \ \ }
\startm
\m{\vdash}\m{(}\m{A}\m{\cap}\m{B}\m{)}\m{\in}\m{{\rm V}}
\endm
\vskip 1ex

\noindent Axiom of the Null Set\index{Axiom of the Null Set} proved from the
other axioms of ZF set theory. Corollary 5.16 of Takeuti and Zaring, p.~20.

\vskip 0.5ex
\setbox\startprefix=\hbox{\tt \ \ 0ex\ \$p\ }
\setbox\contprefix=\hbox{\tt \ \ \ \ \ \ \ \ \ \ \ \ }
\startm
\m{\vdash}\m{\varnothing}\m{\in}\m{{\rm V}}
\endm
\vskip 1ex

\noindent The Axiom of Pairing\index{Axiom of Pairing} proved from the other
axioms of ZF set theory.  Theorem 7.13 of Quine, p.~51.
\vskip 0.5ex
\setbox\startprefix=\hbox{\tt \ \ prex\ \$p\ }
\setbox\contprefix=\hbox{\tt \ \ \ \ \ \ \ \ \ \ \ \ \ \ }
\startm
\m{\vdash}\m{\{}\m{A}\m{,}\m{B}\m{\}}\m{\in}\m{{\rm V}}
\endm
\vskip 2ex

Next we will list some famous or important theorems that are proved in
the \texttt{set.mm} database.  None of them except \texttt{omex}
require the Axiom of Infinity, as you can verify with the \texttt{show
trace{\char`\_}back} Metamath command.

\vskip 2ex
\noindent The resolution of Russell's paradox\index{Russell's paradox}.  There
exists no set containing the set of all sets which are not members of
themselves.  Proposition 4.14 of Takeuti and Zaring, p.~14.

\vskip 0.5ex
\setbox\startprefix=\hbox{\tt \ \ ru\ \$p\ }
\setbox\contprefix=\hbox{\tt \ \ \ \ \ \ \ \ }
\startm
\m{\vdash}\m{\lnot}\m{\exists}\m{x}\m{\,x}\m{=}\m{\{}\m{y}\m{|}\m{\lnot}\m{y}
\m{\in}\m{y}\m{\}}
\endm
\vskip 1ex

\noindent Cantor's theorem\index{Cantor's theorem}.  No set can be mapped onto
its power set.  Compare Theorem 6B(b) of Enderton, p.~132.

\vskip 0.5ex
\setbox\startprefix=\hbox{\tt \ \ canth.1\ \$e\ }
\setbox\contprefix=\hbox{\tt \ \ \ \ \ \ \ \ \ \ \ \ \ }
\startm
\m{\vdash}\m{A}\m{\in}\m{{\rm V}}
\endm
\setbox\startprefix=\hbox{\tt \ \ canth\ \$p\ }
\setbox\contprefix=\hbox{\tt \ \ \ \ \ \ \ \ \ \ \ }
\startm
\m{\vdash}\m{\lnot}\m{F}\m{:}\m{A}\m{\raisebox{.5ex}{${\textstyle{\:}_{
\mbox{\footnotesize\rm {\ }}}}\atop{\textstyle{\longrightarrow}\atop{
\textstyle{}^{\mbox{\footnotesize\rm onto}}}}$}}\m{{\cal P}}\m{A}
\endm
\vskip 1ex

\noindent The Burali-Forti paradox\index{Burali-Forti paradox}.  No set
contains all ordinal numbers. Enderton, p.~194.  (Burali-Forti was one person,
not two.)

\vskip 0.5ex
\setbox\startprefix=\hbox{\tt \ \ onprc\ \$p\ }
\setbox\contprefix=\hbox{\tt \ \ \ \ \ \ \ \ \ \ \ \ }
\startm
\m{\vdash}\m{\lnot}\m{\mbox{\rm On}}\m{\in}\m{{\rm V}}
\endm
\vskip 1ex

\noindent Peano's postulates\index{Peano's postulates} for arithmetic.
Proposition 7.30 of Takeuti and Zaring, pp.~42--43.  The objects being
described are the members of $\omega$ i.e.\ the natural numbers 0, 1,
2,\ldots.  The successor\index{successor} operation suc means ``plus
one.''  \texttt{peano1} says that 0 (which is defined as the empty set)
is a natural number.  \texttt{peano2} says that if $A$ is a natural
number, so is $A+1$.  \texttt{peano3} says that 0 is not the successor
of any natural number.  \texttt{peano4} says that two natural numbers
are equal if and only if their successors are equal.  \texttt{peano5} is
essentially the same as mathematical induction.

\vskip 1ex
\setbox\startprefix=\hbox{\tt \ \ peano1\ \$p\ }
\setbox\contprefix=\hbox{\tt \ \ \ \ \ \ \ \ \ \ \ \ }
\startm
\m{\vdash}\m{\varnothing}\m{\in}\m{\omega}
\endm
\vskip 1.5ex

\setbox\startprefix=\hbox{\tt \ \ peano2\ \$p\ }
\setbox\contprefix=\hbox{\tt \ \ \ \ \ \ \ \ \ \ \ \ }
\startm
\m{\vdash}\m{(}\m{A}\m{\in}\m{\omega}\m{\rightarrow}\m{{\rm suc}}\m{A}\m{\in}%
\m{\omega}\m{)}
\endm
\vskip 1.5ex

\setbox\startprefix=\hbox{\tt \ \ peano3\ \$p\ }
\setbox\contprefix=\hbox{\tt \ \ \ \ \ \ \ \ \ \ \ \ }
\startm
\m{\vdash}\m{(}\m{A}\m{\in}\m{\omega}\m{\rightarrow}\m{\lnot}\m{{\rm suc}}%
\m{A}\m{=}\m{\varnothing}\m{)}
\endm
\vskip 1.5ex

\setbox\startprefix=\hbox{\tt \ \ peano4\ \$p\ }
\setbox\contprefix=\hbox{\tt \ \ \ \ \ \ \ \ \ \ \ \ }
\startm
\m{\vdash}\m{(}\m{(}\m{A}\m{\in}\m{\omega}\m{\wedge}\m{B}\m{\in}\m{\omega}%
\m{)}\m{\rightarrow}\m{(}\m{{\rm suc}}\m{A}\m{=}\m{{\rm suc}}\m{B}\m{%
\leftrightarrow}\m{A}\m{=}\m{B}\m{)}\m{)}
\endm
\vskip 1.5ex

\setbox\startprefix=\hbox{\tt \ \ peano5\ \$p\ }
\setbox\contprefix=\hbox{\tt \ \ \ \ \ \ \ \ \ \ \ \ }
\startm
\m{\vdash}\m{(}\m{(}\m{\varnothing}\m{\in}\m{A}\m{\wedge}\m{\forall}\m{x}\m{%
\in}\m{\omega}\m{(}\m{x}\m{\in}\m{A}\m{\rightarrow}\m{{\rm suc}}\m{x}\m{\in}%
\m{A}\m{)}\m{)}\m{\rightarrow}\m{\omega}\m{\subseteq}\m{A}\m{)}
\endm
\vskip 1.5ex

\noindent Finite Induction (mathematical induction).\index{finite
induction}\index{mathematical induction} The first hypothesis is the
basis and the second is the induction hypothesis.  Theorem Schema 22 of
Suppes, p.~136.

\vskip 0.5ex
\setbox\startprefix=\hbox{\tt \ \ findes.1\ \$e\ }
\setbox\contprefix=\hbox{\tt \ \ \ \ \ \ \ \ \ \ \ \ \ \ }
\startm
\m{\vdash}\m{[}\m{\varnothing}\m{/}\m{x}\m{]}\m{\varphi}
\endm
\setbox\startprefix=\hbox{\tt \ \ findes.2\ \$e\ }
\setbox\contprefix=\hbox{\tt \ \ \ \ \ \ \ \ \ \ \ \ \ \ }
\startm
\m{\vdash}\m{(}\m{x}\m{\in}\m{\omega}\m{\rightarrow}\m{(}\m{\varphi}\m{%
\rightarrow}\m{[}\m{{\rm suc}}\m{x}\m{/}\m{x}\m{]}\m{\varphi}\m{)}\m{)}
\endm
\setbox\startprefix=\hbox{\tt \ \ findes\ \$p\ }
\setbox\contprefix=\hbox{\tt \ \ \ \ \ \ \ \ \ \ \ \ }
\startm
\m{\vdash}\m{(}\m{x}\m{\in}\m{\omega}\m{\rightarrow}\m{\varphi}\m{)}
\endm
\vskip 1ex

\noindent Transfinite Induction with explicit substitution.  The first
hypothesis is the basis, the second is the induction hypothesis for
successors, and the third is the induction hypothesis for limit
ordinals.  Theorem Schema 4 of Suppes, p. 197.

\vskip 0.5ex
\setbox\startprefix=\hbox{\tt \ \ tfindes.1\ \$e\ }
\setbox\contprefix=\hbox{\tt \ \ \ \ \ \ \ \ \ \ \ \ \ \ \ }
\startm
\m{\vdash}\m{[}\m{\varnothing}\m{/}\m{x}\m{]}\m{\varphi}
\endm
\setbox\startprefix=\hbox{\tt \ \ tfindes.2\ \$e\ }
\setbox\contprefix=\hbox{\tt \ \ \ \ \ \ \ \ \ \ \ \ \ \ \ }
\startm
\m{\vdash}\m{(}\m{x}\m{\in}\m{{\rm On}}\m{\rightarrow}\m{(}\m{\varphi}\m{%
\rightarrow}\m{[}\m{{\rm suc}}\m{x}\m{/}\m{x}\m{]}\m{\varphi}\m{)}\m{)}
\endm
\setbox\startprefix=\hbox{\tt \ \ tfindes.3\ \$e\ }
\setbox\contprefix=\hbox{\tt \ \ \ \ \ \ \ \ \ \ \ \ \ \ \ }
\startm
\m{\vdash}\m{(}\m{{\rm Lim}}\m{y}\m{\rightarrow}\m{(}\m{\forall}\m{x}\m{\in}%
\m{y}\m{\varphi}\m{\rightarrow}\m{[}\m{y}\m{/}\m{x}\m{]}\m{\varphi}\m{)}\m{)}
\endm
\setbox\startprefix=\hbox{\tt \ \ tfindes\ \$p\ }
\setbox\contprefix=\hbox{\tt \ \ \ \ \ \ \ \ \ \ \ \ \ }
\startm
\m{\vdash}\m{(}\m{x}\m{\in}\m{{\rm On}}\m{\rightarrow}\m{\varphi}\m{)}
\endm
\vskip 1ex

\noindent Principle of Transfinite Recursion.\index{transfinite
recursion} Theorem 7.41 of Takeuti and Zaring, p.~47.  Transfinite
recursion is the key theorem that allows arithmetic of ordinals to be
rigorously defined, and has many other important uses as well.
Hypotheses \texttt{tfr.1} and \texttt{tfr.2} specify a certain (proper)
class $ F$.  The complicated definition of $ F$ is not important in
itself; what is important is that there be such an $ F$ with the
required properties, and we show this by displaying $ F$ explicitly.
\texttt{tfr1} states that $ F$ is a function whose domain is the set of
ordinal numbers.  \texttt{tfr2} states that any value of $ F$ is
completely determined by its previous values and the values of an
auxiliary function, $G$.  \texttt{tfr3} states that $ F$ is unique,
i.e.\ it is the only function that satisfies \texttt{tfr1} and
\texttt{tfr2}.  Note that $ f$ is an individual variable like $x$ and
$y$; it is just a mnemonic to remind us that $A$ is a collection of
functions.

\vskip 0.5ex
\setbox\startprefix=\hbox{\tt \ \ tfr.1\ \$e\ }
\setbox\contprefix=\hbox{\tt \ \ \ \ \ \ \ \ \ \ \ }
\startm
\m{\vdash}\m{A}\m{=}\m{\{}\m{f}\m{|}\m{\exists}\m{x}\m{\in}\m{{\rm On}}\m{(}%
\m{f}\m{{\rm Fn}}\m{x}\m{\wedge}\m{\forall}\m{y}\m{\in}\m{x}\m{(}\m{f}\m{`}%
\m{y}\m{)}\m{=}\m{(}\m{G}\m{`}\m{(}\m{f}\m{\restriction}\m{y}\m{)}\m{)}\m{)}%
\m{\}}
\endm
\setbox\startprefix=\hbox{\tt \ \ tfr.2\ \$e\ }
\setbox\contprefix=\hbox{\tt \ \ \ \ \ \ \ \ \ \ \ }
\startm
\m{\vdash}\m{F}\m{=}\m{\bigcup}\m{A}
\endm
\setbox\startprefix=\hbox{\tt \ \ tfr1\ \$p\ }
\setbox\contprefix=\hbox{\tt \ \ \ \ \ \ \ \ \ \ }
\startm
\m{\vdash}\m{F}\m{{\rm Fn}}\m{{\rm On}}
\endm
\setbox\startprefix=\hbox{\tt \ \ tfr2\ \$p\ }
\setbox\contprefix=\hbox{\tt \ \ \ \ \ \ \ \ \ \ }
\startm
\m{\vdash}\m{(}\m{z}\m{\in}\m{{\rm On}}\m{\rightarrow}\m{(}\m{F}\m{`}\m{z}%
\m{)}\m{=}\m{(}\m{G}\m{`}\m{(}\m{F}\m{\restriction}\m{z}\m{)}\m{)}\m{)}
\endm
\setbox\startprefix=\hbox{\tt \ \ tfr3\ \$p\ }
\setbox\contprefix=\hbox{\tt \ \ \ \ \ \ \ \ \ \ }
\startm
\m{\vdash}\m{(}\m{(}\m{B}\m{{\rm Fn}}\m{{\rm On}}\m{\wedge}\m{\forall}\m{x}\m{%
\in}\m{{\rm On}}\m{(}\m{B}\m{`}\m{x}\m{)}\m{=}\m{(}\m{G}\m{`}\m{(}\m{B}\m{%
\restriction}\m{x}\m{)}\m{)}\m{)}\m{\rightarrow}\m{B}\m{=}\m{F}\m{)}
\endm
\vskip 1ex

\noindent The existence of omega (the class of natural numbers).\index{natural
number}\index{omega ($\omega$)}\index{Axiom of Infinity}  Axiom 7 of Takeuti
and Zaring, p.~43.  (This is the only theorem in this section requiring the
Axiom of Infinity.)

\vskip 0.5ex
\setbox\startprefix=\hbox{\tt \
\ omex\ \$p\ }
\setbox\contprefix=\hbox{\tt \ \ \ \ \ \ \ \ \ \ }
\startm
\m{\vdash}\m{\omega}\m{\in}\m{{\rm V}}
\endm
%\vskip 2ex


\section{Axioms for Real and Complex Numbers}\label{real}
\index{real and complex numbers!axioms for}

This section presents the axioms for real and complex numbers.  Analysis
textbooks implicitly or explicitly use these axioms or their equivalents
as their starting point.  In the database \texttt{set.mm}, we define real
and complex numbers as (rather complicated) specific sets and derive these
axioms as {\em theorems} from the axioms of ZF set theory, using a method
called Dedekind cuts.  We omit the details of this construction, which you can
follow if you wish using the \texttt{set.mm} database in conjunction with the
textbooks referenced therein.

Once we prove those theorems, we then restate these proven theorems as axioms.
This lets us easily identify which axioms are needed for a particular complex number proof, without the obfuscation of the set theory used to derive them.
As a result,
the construction is actually unimportant other
than to show that sets exist that satisfy the axioms, and thus that the axioms
are consistent if ZF set theory is consistent.  When working with real numbers
you can think of them as being the actual sets resulting
from the construction (for definiteness), or you can
think of them as otherwise unspecified sets that happen to satisfy the axioms.
The derivation is not easy, but the fact that it works is quite remarkable
and lends support to the idea that ZFC set theory is all we need to
provide a foundation for essentially all of mathematics.

For the axioms we are given (or postulate) 8 classes:  $\mathbb{C}$ (the
set of complex numbers), $\mathbb{R}$ (the set of real numbers, a subset
of $\mathbb{C}$), $0$ (zero), $1$ (one), $i$ (square root of
$-1$), $+$ (plus), $\cdot$ (times), and
$<_{\mathbb{R}}$ (less than for just the real numbers).
Subtraction and division are defined terms and are not part of the
axioms; for their definitions see \texttt{set.mm}.

Note that the notation $(A+B)$ (and similarly $(A\cdot B)$) specifies a class
called an {\em operation},\index{operation} and is the function value of the
class $+$ at ordered pair $\langle A,B \rangle$.  An operation is defined by
statement \texttt{df-opr} on p.~\pageref{dfopr}.
The notation $A <_{\mathbb{R}} B$ specifies a
wff called a {\em binary relation}\index{binary relation} and means $\langle A,B \rangle \in \,<_{\mathbb{R}}$, as defined by statement \texttt{df-br} on p.~\pageref{dfbr}.

Our set of 8 given classes is assumed to satisfy the following 22 axioms
(in the axioms listed below, $<$ really means $<_{\mathbb{R}}$).

\vskip 2ex

\noindent 1. The real numbers are a subset of the complex numbers.

%\vskip 0.5ex
\setbox\startprefix=\hbox{\tt \ \ ax-resscn\ \$p\ }
\setbox\contprefix=\hbox{\tt \ \ \ \ \ \ \ \ \ \ \ \ \ \ }
\startm
\m{\vdash}\m{\mathbb{R}}\m{\subseteq}\m{\mathbb{C}}
\endm
%\vskip 1ex

\noindent 2. One is a complex number.

%\vskip 0.5ex
\setbox\startprefix=\hbox{\tt \ \ ax-1cn\ \$p\ }
\setbox\contprefix=\hbox{\tt \ \ \ \ \ \ \ \ \ \ \ }
\startm
\m{\vdash}\m{1}\m{\in}\m{\mathbb{C}}
\endm
%\vskip 1ex

\noindent 3. The imaginary number $i$ is a complex number.

%\vskip 0.5ex
\setbox\startprefix=\hbox{\tt \ \ ax-icn\ \$p\ }
\setbox\contprefix=\hbox{\tt \ \ \ \ \ \ \ \ \ \ \ }
\startm
\m{\vdash}\m{i}\m{\in}\m{\mathbb{C}}
\endm
%\vskip 1ex

\noindent 4. Complex numbers are closed under addition.

%\vskip 0.5ex
\setbox\startprefix=\hbox{\tt \ \ ax-addcl\ \$p\ }
\setbox\contprefix=\hbox{\tt \ \ \ \ \ \ \ \ \ \ \ \ \ }
\startm
\m{\vdash}\m{(}\m{(}\m{A}\m{\in}\m{\mathbb{C}}\m{\wedge}\m{B}\m{\in}\m{\mathbb{C}}%
\m{)}\m{\rightarrow}\m{(}\m{A}\m{+}\m{B}\m{)}\m{\in}\m{\mathbb{C}}\m{)}
\endm
%\vskip 1ex

\noindent 5. Real numbers are closed under addition.

%\vskip 0.5ex
\setbox\startprefix=\hbox{\tt \ \ ax-addrcl\ \$p\ }
\setbox\contprefix=\hbox{\tt \ \ \ \ \ \ \ \ \ \ \ \ \ \ }
\startm
\m{\vdash}\m{(}\m{(}\m{A}\m{\in}\m{\mathbb{R}}\m{\wedge}\m{B}\m{\in}\m{\mathbb{R}}%
\m{)}\m{\rightarrow}\m{(}\m{A}\m{+}\m{B}\m{)}\m{\in}\m{\mathbb{R}}\m{)}
\endm
%\vskip 1ex

\noindent 6. Complex numbers are closed under multiplication.

%\vskip 0.5ex
\setbox\startprefix=\hbox{\tt \ \ ax-mulcl\ \$p\ }
\setbox\contprefix=\hbox{\tt \ \ \ \ \ \ \ \ \ \ \ \ \ }
\startm
\m{\vdash}\m{(}\m{(}\m{A}\m{\in}\m{\mathbb{C}}\m{\wedge}\m{B}\m{\in}\m{\mathbb{C}}%
\m{)}\m{\rightarrow}\m{(}\m{A}\m{\cdot}\m{B}\m{)}\m{\in}\m{\mathbb{C}}\m{)}
\endm
%\vskip 1ex

\noindent 7. Real numbers are closed under multiplication.

%\vskip 0.5ex
\setbox\startprefix=\hbox{\tt \ \ ax-mulrcl\ \$p\ }
\setbox\contprefix=\hbox{\tt \ \ \ \ \ \ \ \ \ \ \ \ \ \ }
\startm
\m{\vdash}\m{(}\m{(}\m{A}\m{\in}\m{\mathbb{R}}\m{\wedge}\m{B}\m{\in}\m{\mathbb{R}}%
\m{)}\m{\rightarrow}\m{(}\m{A}\m{\cdot}\m{B}\m{)}\m{\in}\m{\mathbb{R}}\m{)}
\endm
%\vskip 1ex

\noindent 8. Multiplication of complex numbers is commutative.

%\vskip 0.5ex
\setbox\startprefix=\hbox{\tt \ \ ax-mulcom\ \$p\ }
\setbox\contprefix=\hbox{\tt \ \ \ \ \ \ \ \ \ \ \ \ \ \ }
\startm
\m{\vdash}\m{(}\m{(}\m{A}\m{\in}\m{\mathbb{C}}\m{\wedge}\m{B}\m{\in}\m{\mathbb{C}}%
\m{)}\m{\rightarrow}\m{(}\m{A}\m{\cdot}\m{B}\m{)}\m{=}\m{(}\m{B}\m{\cdot}\m{A}%
\m{)}\m{)}
\endm
%\vskip 1ex

\noindent 9. Addition of complex numbers is associative.

%\vskip 0.5ex
\setbox\startprefix=\hbox{\tt \ \ ax-addass\ \$p\ }
\setbox\contprefix=\hbox{\tt \ \ \ \ \ \ \ \ \ \ \ \ \ \ }
\startm
\m{\vdash}\m{(}\m{(}\m{A}\m{\in}\m{\mathbb{C}}\m{\wedge}\m{B}\m{\in}\m{\mathbb{C}}%
\m{\wedge}\m{C}\m{\in}\m{\mathbb{C}}\m{)}\m{\rightarrow}\m{(}\m{(}\m{A}\m{+}%
\m{B}\m{)}\m{+}\m{C}\m{)}\m{=}\m{(}\m{A}\m{+}\m{(}\m{B}\m{+}\m{C}\m{)}\m{)}%
\m{)}
\endm
%\vskip 1ex

\noindent 10. Multiplication of complex numbers is associative.

%\vskip 0.5ex
\setbox\startprefix=\hbox{\tt \ \ ax-mulass\ \$p\ }
\setbox\contprefix=\hbox{\tt \ \ \ \ \ \ \ \ \ \ \ \ \ \ }
\startm
\m{\vdash}\m{(}\m{(}\m{A}\m{\in}\m{\mathbb{C}}\m{\wedge}\m{B}\m{\in}\m{\mathbb{C}}%
\m{\wedge}\m{C}\m{\in}\m{\mathbb{C}}\m{)}\m{\rightarrow}\m{(}\m{(}\m{A}\m{\cdot}%
\m{B}\m{)}\m{\cdot}\m{C}\m{)}\m{=}\m{(}\m{A}\m{\cdot}\m{(}\m{B}\m{\cdot}\m{C}%
\m{)}\m{)}\m{)}
\endm
%\vskip 1ex

\noindent 11. Multiplication distributes over addition for complex numbers.

%\vskip 0.5ex
\setbox\startprefix=\hbox{\tt \ \ ax-distr\ \$p\ }
\setbox\contprefix=\hbox{\tt \ \ \ \ \ \ \ \ \ \ \ \ \ }
\startm
\m{\vdash}\m{(}\m{(}\m{A}\m{\in}\m{\mathbb{C}}\m{\wedge}\m{B}\m{\in}\m{\mathbb{C}}%
\m{\wedge}\m{C}\m{\in}\m{\mathbb{C}}\m{)}\m{\rightarrow}\m{(}\m{A}\m{\cdot}\m{(}%
\m{B}\m{+}\m{C}\m{)}\m{)}\m{=}\m{(}\m{(}\m{A}\m{\cdot}\m{B}\m{)}\m{+}\m{(}%
\m{A}\m{\cdot}\m{C}\m{)}\m{)}\m{)}
\endm
%\vskip 1ex

\noindent 12. The square of $i$ equals $-1$ (expressed as $i$-squared plus 1 is
0).

%\vskip 0.5ex
\setbox\startprefix=\hbox{\tt \ \ ax-i2m1\ \$p\ }
\setbox\contprefix=\hbox{\tt \ \ \ \ \ \ \ \ \ \ \ \ }
\startm
\m{\vdash}\m{(}\m{(}\m{i}\m{\cdot}\m{i}\m{)}\m{+}\m{1}\m{)}\m{=}\m{0}
\endm
%\vskip 1ex

\noindent 13. One and zero are distinct.

%\vskip 0.5ex
\setbox\startprefix=\hbox{\tt \ \ ax-1ne0\ \$p\ }
\setbox\contprefix=\hbox{\tt \ \ \ \ \ \ \ \ \ \ \ \ }
\startm
\m{\vdash}\m{1}\m{\ne}\m{0}
\endm
%\vskip 1ex

\noindent 14. One is an identity element for real multiplication.

%\vskip 0.5ex
\setbox\startprefix=\hbox{\tt \ \ ax-1rid\ \$p\ }
\setbox\contprefix=\hbox{\tt \ \ \ \ \ \ \ \ \ \ \ }
\startm
\m{\vdash}\m{(}\m{A}\m{\in}\m{\mathbb{R}}\m{\rightarrow}\m{(}\m{A}\m{\cdot}\m{1}%
\m{)}\m{=}\m{A}\m{)}
\endm
%\vskip 1ex

\noindent 15. Every real number has a negative.

%\vskip 0.5ex
\setbox\startprefix=\hbox{\tt \ \ ax-rnegex\ \$p\ }
\setbox\contprefix=\hbox{\tt \ \ \ \ \ \ \ \ \ \ \ \ \ \ }
\startm
\m{\vdash}\m{(}\m{A}\m{\in}\m{\mathbb{R}}\m{\rightarrow}\m{\exists}\m{x}\m{\in}%
\m{\mathbb{R}}\m{(}\m{A}\m{+}\m{x}\m{)}\m{=}\m{0}\m{)}
\endm
%\vskip 1ex

\noindent 16. Every nonzero real number has a reciprocal.

%\vskip 0.5ex
\setbox\startprefix=\hbox{\tt \ \ ax-rrecex\ \$p\ }
\setbox\contprefix=\hbox{\tt \ \ \ \ \ \ \ \ \ \ \ \ \ \ }
\startm
\m{\vdash}\m{(}\m{A}\m{\in}\m{\mathbb{R}}\m{\rightarrow}\m{(}\m{A}\m{\ne}\m{0}%
\m{\rightarrow}\m{\exists}\m{x}\m{\in}\m{\mathbb{R}}\m{(}\m{A}\m{\cdot}%
\m{x}\m{)}\m{=}\m{1}\m{)}\m{)}
\endm
%\vskip 1ex

\noindent 17. A complex number can be expressed in terms of two reals.

%\vskip 0.5ex
\setbox\startprefix=\hbox{\tt \ \ ax-cnre\ \$p\ }
\setbox\contprefix=\hbox{\tt \ \ \ \ \ \ \ \ \ \ \ \ }
\startm
\m{\vdash}\m{(}\m{A}\m{\in}\m{\mathbb{C}}\m{\rightarrow}\m{\exists}\m{x}\m{\in}%
\m{\mathbb{R}}\m{\exists}\m{y}\m{\in}\m{\mathbb{R}}\m{A}\m{=}\m{(}\m{x}\m{+}\m{(}%
\m{y}\m{\cdot}\m{i}\m{)}\m{)}\m{)}
\endm
%\vskip 1ex

\noindent 18. Ordering on reals satisfies strict trichotomy.

%\vskip 0.5ex
\setbox\startprefix=\hbox{\tt \ \ ax-pre-lttri\ \$p\ }
\setbox\contprefix=\hbox{\tt \ \ \ \ \ \ \ \ \ \ \ \ \ }
\startm
\m{\vdash}\m{(}\m{(}\m{A}\m{\in}\m{\mathbb{R}}\m{\wedge}\m{B}\m{\in}\m{\mathbb{R}}%
\m{)}\m{\rightarrow}\m{(}\m{A}\m{<}\m{B}\m{\leftrightarrow}\m{\lnot}\m{(}\m{A}%
\m{=}\m{B}\m{\vee}\m{B}\m{<}\m{A}\m{)}\m{)}\m{)}
\endm
%\vskip 1ex

\noindent 19. Ordering on reals is transitive.

%\vskip 0.5ex
\setbox\startprefix=\hbox{\tt \ \ ax-pre-lttrn\ \$p\ }
\setbox\contprefix=\hbox{\tt \ \ \ \ \ \ \ \ \ \ \ \ \ }
\startm
\m{\vdash}\m{(}\m{(}\m{A}\m{\in}\m{\mathbb{R}}\m{\wedge}\m{B}\m{\in}\m{\mathbb{R}}%
\m{\wedge}\m{C}\m{\in}\m{\mathbb{R}}\m{)}\m{\rightarrow}\m{(}\m{(}\m{A}\m{<}%
\m{B}\m{\wedge}\m{B}\m{<}\m{C}\m{)}\m{\rightarrow}\m{A}\m{<}\m{C}\m{)}\m{)}
\endm
%\vskip 1ex

\noindent 20. Ordering on reals is preserved after addition to both sides.

%\vskip 0.5ex
\setbox\startprefix=\hbox{\tt \ \ ax-pre-ltadd\ \$p\ }
\setbox\contprefix=\hbox{\tt \ \ \ \ \ \ \ \ \ \ \ \ \ }
\startm
\m{\vdash}\m{(}\m{(}\m{A}\m{\in}\m{\mathbb{R}}\m{\wedge}\m{B}\m{\in}\m{\mathbb{R}}%
\m{\wedge}\m{C}\m{\in}\m{\mathbb{R}}\m{)}\m{\rightarrow}\m{(}\m{A}\m{<}\m{B}\m{%
\rightarrow}\m{(}\m{C}\m{+}\m{A}\m{)}\m{<}\m{(}\m{C}\m{+}\m{B}\m{)}\m{)}\m{)}
\endm
%\vskip 1ex

\noindent 21. The product of two positive reals is positive.

%\vskip 0.5ex
\setbox\startprefix=\hbox{\tt \ \ ax-pre-mulgt0\ \$p\ }
\setbox\contprefix=\hbox{\tt \ \ \ \ \ \ \ \ \ \ \ \ \ \ }
\startm
\m{\vdash}\m{(}\m{(}\m{A}\m{\in}\m{\mathbb{R}}\m{\wedge}\m{B}\m{\in}\m{\mathbb{R}}%
\m{)}\m{\rightarrow}\m{(}\m{(}\m{0}\m{<}\m{A}\m{\wedge}\m{0}%
\m{<}\m{B}\m{)}\m{\rightarrow}\m{0}\m{<}\m{(}\m{A}\m{\cdot}\m{B}\m{)}%
\m{)}\m{)}
\endm
%\vskip 1ex

\noindent 22. A non-empty, bounded-above set of reals has a supremum.

%\vskip 0.5ex
\setbox\startprefix=\hbox{\tt \ \ ax-pre-sup\ \$p\ }
\setbox\contprefix=\hbox{\tt \ \ \ \ \ \ \ \ \ \ \ }
\startm
\m{\vdash}\m{(}\m{(}\m{A}\m{\subseteq}\m{\mathbb{R}}\m{\wedge}\m{A}\m{\ne}\m{%
\varnothing}\m{\wedge}\m{\exists}\m{x}\m{\in}\m{\mathbb{R}}\m{\forall}\m{y}\m{%
\in}\m{A}\m{\,y}\m{<}\m{x}\m{)}\m{\rightarrow}\m{\exists}\m{x}\m{\in}\m{%
\mathbb{R}}\m{(}\m{\forall}\m{y}\m{\in}\m{A}\m{\lnot}\m{x}\m{<}\m{y}\m{\wedge}\m{%
\forall}\m{y}\m{\in}\m{\mathbb{R}}\m{(}\m{y}\m{<}\m{x}\m{\rightarrow}\m{\exists}%
\m{z}\m{\in}\m{A}\m{\,y}\m{<}\m{z}\m{)}\m{)}\m{)}
\endm

% NOTE: The \m{...} expressions above could be represented as
% $ \vdash ( ( A \subseteq \mathbb{R} \wedge A \ne \varnothing \wedge \exists x \in \mathbb{R} \forall y \in A \,y < x ) \rightarrow \exists x \in \mathbb{R} ( \forall y \in A \lnot x < y \wedge \forall y \in \mathbb{R} ( y < x \rightarrow \exists z \in A \,y < z ) ) ) $

\vskip 2ex

This completes the set of axioms for real and complex numbers.  You may
wish to look at how subtraction, division, and decimal numbers
are defined in \texttt{set.mm}, and for fun look at the proof of $2+
2 = 4$ (theorem \texttt{2p2e4} in \texttt{set.mm})
as discussed in section \ref{2p2e4}.

In \texttt{set.mm} we define the non-negative integers $\mathbb{N}$, the integers
$\mathbb{Z}$, and the rationals $\mathbb{Q}$ as subsets of $\mathbb{R}$.  This leads
to the nice inclusion $\mathbb{N} \subseteq \mathbb{Z} \subseteq \mathbb{Q} \subseteq
\mathbb{R} \subseteq \mathbb{C}$, giving us a uniform framework in which, for
example, a property such as commutativity of complex number addition
automatically applies to integers.  The natural numbers $\mathbb{N}$
are different from the set $\omega$ we defined earlier, but both satisfy
Peano's postulates.

\subsubsection{Complex Number Axioms in Analysis Texts}

Most analysis texts construct complex numbers as ordered pairs of reals,
leading to construction-dependent properties that satisfy these axioms
but are not stated in their pure form.  (This is also done in
\texttt{set.mm} but our axioms are extracted from that construction.)
Other texts will simply state that $\mathbb{R}$ is a ``complete ordered
subfield of $\mathbb{C}$,'' leading to redundant axioms when this phrase
is completely expanded out.  In fact I have not seen a text with the
axioms in the explicit form above.
None of these axioms is unique individually, but this carefully worked out
collection of axioms is the result of years of work
by the Metamath community.

\subsubsection{Eliminating Unnecessary Complex Number Axioms}

We once had more axioms for real and complex numbers, but over years of time
we (the Metamath community)
have found ways to eliminate them (by proving them from other axioms)
or weaken them (by making weaker claims without reducing what
can be proved).
In particular, here are statements that used to be complex number
axioms but have since been formally proven (with Metamath) to be redundant:

\begin{itemize}
\item
  $\mathbb{C} \in V$.
  At one time this was listed as a ``complex number axiom''
  However, this is not properly speaking a complex number axiom,
  and in any case its proof uses axioms of set theory.
  Proven redundant by Mario Carneiro\index{Carneiro, Mario} on
  17-Nov-2014 (see \texttt{axcnex}).
\item
  $((A \in \mathbb{C} \land B \in \mathbb{C}$) $\rightarrow$
  $(A + B) = (B + A))$.
  Proved redundant by Eric Schmidt\index{Schmidt, Eric} on 19-Jun-2012,
  and formalized by Scott Fenton\index{Fenton, Scott} on 3-Jan-2013
  (see \texttt{addcom}).
\item
  $(A \in \mathbb{C} \rightarrow (A + 0) = A)$.
  Proved redundant by Eric Schmidt on 19-Jun-2012,
  and formalized by Scott Fenton on 3-Jan-2013
  (see \texttt{addid1}).
\item
  $(A \in \mathbb{C} \rightarrow \exists x \in \mathbb{C} (A + x) = 0)$.
  Proved redundant by Eric Schmidt and formalized on 21-May-2007
  (see \texttt{cnegex}).
\item
  $((A \in \mathbb{C} \land A \ne 0) \rightarrow \exists x \in \mathbb{C} (A \cdot x) = 1)$.
  Proved redundant by Eric Schmidt and formalized on 22-May-2007
  (see \texttt{recex}).
\item
  $0 \in \mathbb{R}$.
  Proved redundant by Eric Schmidt on 19-Feb-2005 and formalized 21-May-2007
  (see \texttt{0re}).
\end{itemize}

We could eliminate 0 as an axiomatic object by defining it as
$( ( i \cdot i ) + 1 )$
and replacing it with this expression throughout the axioms. If this
is done, axiom ax-i2m1 becomes redundant. However, the remaining axioms
would become longer and less intuitive.

Eric Schmidt's paper analyzing this axiom system \cite{Schmidt}
presented a proof that these remaining axioms,
with the possible exception of ax-mulcom, are independent of the others.
It is currently an open question if ax-mulcom is independent of the others.

\section{Two Plus Two Equals Four}\label{2p2e4}

Here is a proof that $2 + 2 = 4$, as proven in the theorem \texttt{2p2e4}
in the database \texttt{set.mm}.
This is a useful demonstration of what a Metamath proof can look like.
This proof may have more steps than you're used to, but each step is rigorously
proven all the way back to the axioms of logic and set theory.
This display was originally generated by the Metamath program
as an {\sc HTML} file.

In the table showing the proof ``Step'' is the sequential step number,
while its associated ``Expression'' is an expression that we have proved.
``Ref'' is the name of a theorem or axiom that justifies that expression,
and ``Hyp'' refers to previous steps (if any) that the theorem or axiom
needs so that we can use it.  Expressions are indented further than
the expressions that depend on them to show their interdependencies.

\begin{table}[!htbp]
\caption{Two plus two equals four}
\begin{tabular}{lllll}
\textbf{Step} & \textbf{Hyp} & \textbf{Ref} & \textbf{Expression} & \\
1  &       & df-2    & $ \; \; \vdash 2 = 1 + 1$  & \\
2  & 1     & oveq2i  & $ \; \vdash (2 + 2) = (2 + (1 + 1))$ & \\
3  &       & df-4    & $ \; \; \vdash 4 = (3 + 1)$ & \\
4  &       & df-3    & $ \; \; \; \vdash 3 = (2 + 1)$ & \\
5  & 4     & oveq1i  & $ \; \; \vdash (3 + 1) = ((2 + 1) + 1)$ & \\
6  &       & 2cn     & $ \; \; \; \vdash 2 \in \mathbb{C}$ & \\
7  &       & ax-1cn  & $ \; \; \; \vdash 1 \in \mathbb{C}$ & \\
8  & 6,7,7 & addassi & $ \; \; \vdash ((2 + 1) + 1) = (2 + (1 + 1))$ & \\
9  & 3,5,8 & 3eqtri  & $ \; \vdash 4 = (2 + (1 + 1))$ & \\
10 & 2,9   & eqtr4i  & $ \vdash (2 + 2) = 4$ & \\
\end{tabular}
\end{table}

Step 1 says that we can assert that $2 = 1 + 1$ because it is
justified by \texttt{df-2}.
What is \texttt{df-2}?
It is simply the definition of $2$, which in our system is defined as being
equal to $1 + 1$.  This shows how we can use definitions in proofs.

Look at Step 2 of the proof. In the Ref column, we see that it references
a previously proved theorem, \texttt{oveq2i}.
It turns out that
theorem \texttt{oveq2i} requires a
hypothesis, and in the Hyp column of Step 2 we indicate that Step 1 will
satisfy (match) this hypothesis.
If we looked at \texttt{oveq2i}
we would find that it proves that given some hypothesis
$A = B$, we can prove that $( C F A ) = ( C F B )$.
If we use \texttt{oveq2i} and apply step 1's result as the hypothesis,
that will mean that $A = 2$ and $B = ( 1 + 1 )$ within this use of
\texttt{oveq2i}.
We are free to select any value of $C$ and $F$ (subject to syntax constraints),
so we are free to select $C = 2$ and $F = +$,
producing our desired result,
$ (2 + 2) = (2 + (1 + 1))$.

Step 2 is an example of substitution.
In the end, every step in every proof uses only this one substitution rule.
All the rules of logic, and all the axioms, are expressed so that
they can be used via this one substitution rule.
So once you master substitution, you can master every Metamath proof,
no exceptions.

Each step is clear and can be immediately checked.
In the {\sc HTML} display you can even click on each reference to see why it is
justified, making it easy to see why the proof works.

\section{Deduction}\label{deduction}

Strictly speaking,
a deduction (also called an inference) is a kind of statement that needs
some hypotheses to be true in order for its conclusion to be true.
A theorem, on the other hand, has no hypotheses.
Informally we often call both of them theorems, but in this section we
will stick to the strict definitions.

It sometimes happens that we have proved a deduction of the form
$\varphi \Rightarrow \psi$\index{$\Rightarrow$}
(given hypothesis $\varphi$ we can prove $\psi$)
and we want to then prove a theorem of the form
$\varphi \rightarrow \psi$.

Converting a deduction (which uses a hypothesis) into a theorem
(which does not) is not as simple as you might think.
The deduction says, ``if we can prove $\varphi$ then we can prove $\psi$,''
which is in some sense weaker than saying
``$\varphi$ implies $\psi$.''
There is no axiom of logic that permits us to directly obtain the theorem
given the deduction.\footnote{
The conversion of a deduction to a theorem does not even hold in general
for quantum propositional calculus,
which is a weak subset of classical propositional calculus.
It has been shown that adding the Standard Deduction Theorem (discussed below)
to quantum propositional calculus turns it into classical
propositional calculus!
}

This is in contrast to going the other way.
If we have the theorem ($\varphi \rightarrow \psi$),
it is easy to recover the deduction
($\varphi \Rightarrow \psi$)
using modus ponens\index{modus ponens}
(\texttt{ax-mp}; see section \ref{axmp}).

In the following subsections we first discuss the standard deduction theorem
(the traditional but awkward way to convert deductions into theorems) and
the weak deduction theorem (a limited version of the standard deduction
theorem that is easier to use and was once widely used in
\texttt{set.mm}\index{set theory database (\texttt{set.mm})}).
In section \ref{deductionstyle} we discuss
deduction style, the newer approach we now recommend in most cases.
Deduction style uses ``deduction form,'' a form that
prefixes each hypothesis (other than definitions) and the
conclusion with a universal antecedent (``$\varphi \rightarrow$'').
Deduction style is widely used in \texttt{set.mm},
so it is useful to understand it and \textit{why} it is widely used.
Section \ref{naturaldeduction}
briefly discusses our approach for using natural deduction
within \texttt{set.mm},
as that approach is deeply related to deduction style.
We conclude with a summary of the strengths of
our approach, which we believe are compelling.

\subsection{The Standard Deduction Theorem}\label{standarddeductiontheorem}

It is possible to make use of information
contained in the deduction or its proof to assist us with the proof of
the related theorem.
In traditional logic books, there is a metatheorem called the
Deduction Theorem\index{Deduction Theorem}\index{Standard Deduction Theorem},
discovered independently by Herbrand and Tarski around 1930.
The Deduction Theorem, which we often call the Standard Deduction Theorem,
provides an algorithm for constructing a proof of a theorem from the
proof of its corresponding deduction. See, for example,
\cite[p.~56]{Margaris}\index{Margaris, Angelo}.
To construct a proof for a theorem, the
algorithm looks at each step in the proof of the original deduction and
rewrites the step with several steps wherein the hypothesis is eliminated
and becomes an antecedent.

In ordinary mathematics, no one actually carries out the algorithm,
because (in its most basic form) it involves an exponential explosion of
the number of proof steps as more hypotheses are eliminated. Instead,
the Standard Deduction Theorem is invoked simply to claim that it can
be done in principle, without actually doing it.
What's more, the algorithm is not as simple as it might first appear
when applying it rigorously.
There is a subtle restriction on the Standard Deduction Theorem
that must be taken into account involving the axiom of generalization
when working with predicate calculus (see the literature for more detail).

One of the goals of Metamath is to let you plainly see, with as few
underlying concepts as possible, how mathematics can be derived directly
from the axioms, and not indirectly according to some hidden rules
buried inside a program or understood only by logicians. If we added
the Standard Deduction Theorem to the language and proof verifier,
that would greatly complicate both and largely defeat Metamath's goal
of simplicity. In principle, we could show direct proofs by expanding
out the proof steps generated by the algorithm of the Standard Deduction
Theorem, but that is not feasible in practice because the number of proof
steps quickly becomes huge, even astronomical.
Since the algorithm of the Standard Deduction Theorem is driven by the proof,
we would have to go through that proof
all over again---starting from axioms---in order to obtain the theorem form.
In terms of proof length, there would be no savings over just
proving the theorem directly instead of first proving the deduction form.

\subsection{Weak Deduction Theorem}\label{weakdeductiontheorem}

We have developed
a more efficient method for proving a theorem from a deduction
that can be used instead of the Standard Deduction Theorem
in many (but not all) cases.
We call this more efficient method the
Weak Deduction Theorem\index{Weak Deduction Theorem}.\footnote{
There is also an unrelated ``Weak Deduction Theorem''
in the field of relevance logic, so to avoid confusion we could call
ours the ``Weak Deduction Theorem for Classical Logic.''}
Unlike the Standard Deduction Theorem, the Weak Deduction Theorem produces the
theorem directly from a special substitution instance of the deduction,
using a small, fixed number of steps roughly proportional to the length
of the final theorem.

If you come to a proof referencing the Weak Deduction Theorem
\texttt{dedth} (or one of its variants \texttt{dedthxx}),
here is how to follow the proof without getting into the details:
just click on the theorem referenced in the step
just before the reference to \texttt{dedth} and ignore everything else.
Theorem \texttt{dedth} simply turns a hypothesis into an antecedent
(i.e. the hypothesis followed by $\rightarrow$
is placed in front of the assertion, and the hypothesis
itself is eliminated) given certain conditions.

The Weak Deduction Theorem
eliminates a hypothesis $\varphi$, making it become an antecedent.
It does this by proving an expression
$ \varphi \rightarrow \psi $ given two hypotheses:
(1)
$ ( A = {\rm if} ( \varphi , A , B ) \rightarrow ( \varphi \leftrightarrow \chi ) ) $
and
(2) $\chi$.
Note that it requires that a proof exists for $\varphi$ when the class variable
$A$ is replaced with a specific class $B$. The hypothesis $\chi$
should be assigned to the inference.
You can see the details of the proof of the Weak Deduction Theorem
in theorem \texttt{dedth}.

The Weak Deduction Theorem
is probably easier to understand by studying proofs that make use of it.
For example, let's look at the proof of \texttt{renegcl}, which proves that
$ \vdash ( A \in \mathbb{R} \rightarrow - A \in \mathbb{R} )$:

\needspace{4\baselineskip}
\begin{longtabu} {l l l X}
\textbf{Step} & \textbf{Hyp} & \textbf{Ref} & \textbf{Expression} \\
  1 &  & negeq &
  $\vdash$ $($ $A$ $=$ ${\rm if}$ $($ $A$ $\in$ $\mathbb{R}$ $,$ $A$ $,$ $1$ $)$ $\rightarrow$
  $\textrm{-}$ $A$ $=$ $\textrm{-}$ ${\rm if}$ $($ $A$ $\in$ $\mathbb{R}$
  $,$ $A$ $,$ $1$ $)$ $)$ \\
 2 & 1 & eleq1d &
    $\vdash$ $($ $A$ $=$ ${\rm if}$ $($ $A$ $\in$ $\mathbb{R}$ $,$ $A$ $,$ $1$ $)$ $\rightarrow$ $($
    $\textrm{-}$ $A$ $\in$ $\mathbb{R}$ $\leftrightarrow$
    $\textrm{-}$ ${\rm if}$ $($ $A$ $\in$ $\mathbb{R}$ $,$ $A$ $,$ $1$ $)$ $\in$
    $\mathbb{R}$ $)$ $)$ \\
 3 &  & 1re & $\vdash 1 \in \mathbb{R}$ \\
 4 & 3 & elimel &
   $\vdash {\rm if} ( A \in \mathbb{R} , A , 1 ) \in \mathbb{R}$ \\
 5 & 4 & renegcli &
   $\vdash \textrm{-} {\rm if} ( A \in \mathbb{R} , A , 1 ) \in \mathbb{R}$ \\
 6 & 2,5 & dedth &
   $\vdash ( A \in \mathbb{R} \rightarrow \textrm{-} A \in \mathbb{R}$ ) \\
\end{longtabu}

The somewhat strange-looking steps in \texttt{renegcl} before step 5 are
technical stuff that makes this magic work, and they can be ignored
for a quick overview of the proof. To continue following the ``important''
part of the proof of \texttt{renegcl},
you can look at the reference to \texttt{renegcli} at step 5.

That said, let's briefly look at how
\texttt{renegcl} uses the
Weak Deduction Theorem (\texttt{dedth}) to do its job,
in case you want to do something similar or want understand it more deeply.
Let's work backwards in the proof of \texttt{renegcl}.
Step 6 applies \texttt{dedth} to produce our goal result
$ \vdash ( A \in \mathbb{R} \rightarrow\, - A \in \mathbb{R} )$.
This requires on the one hand the (substituted) deduction
\texttt{renegcli} in step 5.
By itself \texttt{renegcli} proves the deduction
$ \vdash A \in \mathbb{R} \Rightarrow\, \vdash - A \in \mathbb{R}$;
this is the deduction form we are trying to turn into theorem form,
and thus
\texttt{renegcli} has a separate hypothesis that must be fulfilled.
To fulfill the hypothesis of the invocation of
\texttt{renegcli} in step 5, it is eventually
reduced to the already proven theorem $1 \in \mathbb{R}$ in step 3.
Step 4 connects steps 3 and 5; step 4 invokes
\texttt{elimel}, a special case of \texttt{elimhyp} that eliminates
a membership hypothesis for the weak deduction theorem.
On the other hand, the equivalence of the conclusion of
\texttt{renegcl}
$( - A \in \mathbb{R} )$ and the substituted conclusion of
\texttt{renegcli} must be proven, which is done in steps 2 and 1.

The weak deduction theorem has limitations.
In particular, we must be able to prove a special case of the deduction's
hypothesis as a stand-alone theorem.
For example, we used $1 \in \mathbb{R}$ in step 3 of \texttt{renegcl}.

We used to use the weak deduction theorem
extensively within \texttt{set.mm}.
However, we now recommend applying ``deduction style''
instead in most cases, as deduction style is
often an easier and clearer approach.
Therefore, we will now describe deduction style.

\subsection{Deduction Style}\label{deductionstyle}

We now prefer to write assertions in ``deduction form''
instead of writing a proof that would require use of the standard or
weak deduction theorem.
We call this appraoch
``deduction style.''\index{deduction style}

It will be easier to explain this by first defining some terms:

\begin{itemize}
\item \textbf{closed form}\index{closed form}\index{forms!closed}:
A kind of assertion (theorem) with no hypotheses.
Typically its label has no special suffix.
An example is \texttt{unss}, which states:
$\vdash ( ( A \subseteq C \wedge B \subseteq C ) \leftrightarrow ( A \cup B )
\subseteq C )\label{eq:unss}$
\item \textbf{deduction form}\index{deduction form}\index{forms!deduction}:
A kind of assertion with one or more hypotheses
where the conclusion is an implication with
a wff variable as the antecedent (usually $\varphi$), and every hypothesis
(\$e statement)
is either (1) an implication with the same antecedent as the conclusion or
(2) a definition.
A definition
can be for a class variable (this is a class variable followed by ``='')
or a wff variable (this is a wff variable followed by $\leftrightarrow$);
class variable definitions are more common.
In practice, a proof
in deduction form will also contain many steps that are implications
where the antecedent is either that wff variable (normally $\varphi$)
or is
a conjunction (...$\land$...) including that wff variable ($\varphi$).
If an assertion is in deduction form, and other forms are also available,
then we suffix its label with ``d.''
An example is \texttt{unssd}, which states\footnote{
For brevity we show here (and in other places)
a $\&$\index{$\&$} between hypotheses\index{hypotheses}
and a $\Rightarrow$\index{$\Rightarrow$}\index{conclusion}
between the hypotheses and the conclusion.
This notation is technically not part of the Metamath language, but is
instead a convenient abbreviation to show both the hypotheses and conclusion.}:
$\vdash ( \varphi \rightarrow A \subseteq C )\quad\&\quad \vdash ( \varphi
    \rightarrow B \subseteq C )\quad\Rightarrow\quad \vdash ( \varphi
    \rightarrow ( A \cup B ) \subseteq C )\label{eq:unssd}$
\item \textbf{inference form}\index{inference form}\index{forms!inference}:
A kind of assertion with one or more hypotheses that is not in deduction form
(e.g., there is no common antecedent).
If an assertion is in inference form, and other forms are also available,
then we suffix its label with ``i.''
An example is \texttt{unssi}, which states:
$\vdash A \subseteq C\quad\&\quad \vdash B \subseteq C\quad\Rightarrow\quad
    \vdash ( A \cup B ) \subseteq C\label{eq:unssi}$
\end{itemize}

When using deduction style we express an assertion in deduction form.
This form prefixes each hypothesis (other than definitions) and the
conclusion with a universal antecedent (``$\varphi \rightarrow$'').
The antecedent (e.g., $\varphi$)
mimics the context handled in the deduction theorem, eliminating
the need to directly use the deduction theorem.

Once you have an assertion in deduction form, you can easily convert it
to inference form or closed form:

\begin{itemize}
\item To
prove some assertion Ti in inference form, given assertion Td in deduction
form, there is a simple mechanical process you can use. First take each
Ti hypothesis and insert a \texttt{T.} $\rightarrow$ prefix (``true implies'')
using \texttt{a1i}. You
can then use the existing assertion Td to prove the resulting conclusion
with a \texttt{T.} $\rightarrow$ prefix.
Finally, you can remove that prefix using \texttt{trud},
resulting in the conclusion you wanted to prove.
\item To
prove some assertion T in closed form, given assertion Td in deduction
form, there is another simple mechanical process you can use. First,
select an expression that is the conjunction (...$\land$...) of all of the
consequents of every hypothesis of Td. Next, prove that this expression
implies each of the separate hypotheses of Td in turn by eliminating
conjuncts (there are a variety of proven assertions to do this, including
\texttt{simpl},
\texttt{simpr},
\texttt{3simpa},
\texttt{3simpb},
\texttt{3simpc},
\texttt{simp1},
\texttt{simp2},
and
\texttt{simp3}).
If the
expression has nested conjunctions, inner conjuncts can be broken out by
chaining the above theorems with \texttt{syl}
(see section \ref{syl}).\footnote{
There are actually many theorems
(labeled simp* such as \texttt{simp333}) that break out inner conjuncts in one
step, but rather than learning them you can just use the chaining we
just described to prove them, and then let the Metamath program command
\texttt{minimize{\char`\_}with}\index{\texttt{minimize{\char`\_}with} command}
figure out the right ones needed to collapse them.}
As your final step, you can then apply the already-proven assertion Td
(which is in deduction form), proving assertion T in closed form.
\end{itemize}

We can also easily convert any assertion T in closed form to its related
assertion Ti in inference form by applying
modus ponens\index{modus ponens} (see section \ref{axmp}).

The deduction form antecedent can also be used to represent the context
necessary to support natural deduction systems, so we will now
discuss natural deduction.

\subsection{Natural Deduction}\label{naturaldeduction}

Natural deduction\index{natural deduction}
(ND) systems, as such, were originally introduced in
1934 by two logicians working independently: Ja\'skowski and Gentzen. ND
systems are supposed to reconstruct, in a formally proper way, traditional
ways of mathematical reasoning (such as conditional proof, indirect proof,
and proof by cases). As reconstructions they were naturally influenced
by previous work, and many specific ND systems and notations have been
developed since their original work.

There are many ND variants, but
Indrzejczak \cite[p.~31-32]{Indrzejczak}\index{Indrzejczak, Andrzej}
suggests that any natural deductive system must satisfy at
least these three criteria:

\begin{itemize}
\item ``There are some means for entering assumptions into a proof and
also for eliminating them. Usually it requires some bookkeeping devices
for indicating the scope of an assumption, and showing that a part of
a proof depending on eliminated assumption is discharged.
\item There are no (or, at least, a very limited set of) axioms, because
their role is taken over by the set of primitive rules for introduction
and elimination of logical constants which means that elementary
inferences instead of formulae are taken as primitive.
\item (A genuine) ND system admits a lot of freedom in proof construction
and possibility of applying several proof search strategies, like
conditional proof, proof by cases, proof by reductio ad absurdum etc.''
\end{itemize}

The Metamath Proof Explorer (MPE) as defined in \texttt{set.mm}
is fundamentally a Hilbert-style system.
That is, MPE is based on a larger number of axioms (compared
to natural deduction systems), a very small set of rules of inference
(modus ponens), and the context is not changed by the rules of inference
in the middle of a proof. That said, MPE proofs can be developed using
the natural deduction (ND) approach as originally developed by Ja\'skowski
and Gentzen.

The most common and recommended approach for applying ND in MPE is to use
deduction form\index{deduction form}%
\index{forms!deduction}
and apply the MPE proven assertions that are equivalent to ND rules.
For example, MPE's \texttt{jca} is equivalent to ND rule $\land$-I
(and-insertion).
We maintain a list of equivalences that you may consult.
This approach for applying an ND approach within MPE relies on Metamath's
wff metavariables in an essential way, and is described in more detail
in the presentation ``Natural Deductions in the Metamath Proof Language''
by Mario Carneiro \cite{CarneiroND}\index{Carneiro, Mario}.

In this style many steps are an implication, whose antecedent mimics
the context ($\Gamma$) of most ND systems. To add an assumption, simply add
it to the implication antecedent (typically using
\texttt{simpr}),
and use that
new antecedent for all later claims in the same scope. If you wish to
use an assertion in an ND hypothesis scope that is outside the current
ND hypothesis scope, modify the assertion so that the ND hypothesis
assumption is added to its antecedent (typically using \texttt{adantr}). Most
proof steps will be proved using rules that have hypotheses and results
of the form $\varphi \rightarrow$ ...

An example may make this clearer.
Let's show theorem 5.5 of
\cite[p.~18]{Clemente}\index{Clemente Laboreo, Daniel}
along with a line by line translation using the usual
translation of natural deduction (ND) in the Metamath Proof Explorer
(MPE) notation (this is proof \texttt{ex-natded5.5}).
The proof's original goal was to prove
$\lnot \psi$ given two hypotheses,
$( \psi \rightarrow \chi )$ and $ \lnot \chi$.
We will translate these statements into MPE deduction form
by prefixing them all with $\varphi \rightarrow$.
As a result, in MPE the goal is stated as
$( \varphi \rightarrow \lnot \psi )$, and the two hypotheses are stated as
$( \varphi \rightarrow ( \psi \rightarrow \chi ) )$ and
$( \varphi \rightarrow \lnot \chi )$.

The following table shows the proof in Fitch natural deduction style
and its MPE equivalent.
The \textit{\#} column shows the original numbering,
\textit{MPE\#} shows the number in the equivalent MPE proof
(which we will show later),
\textit{ND Expression} shows the original proof claim in ND notation,
and \textit{MPE Translation} shows its translation into MPE
as discussed in this section.
The final columns show the rationale in ND and MPE respectively.

\needspace{4\baselineskip}
{\setlength{\extrarowsep}{4pt} % Keep rows from being too close together
\begin{longtabu}   { @{} c c X X X X }
\textbf{\#} & \textbf{MPE\#} & \textbf{ND Ex\-pres\-sion} &
\textbf{MPE Trans\-lation} & \textbf{ND Ration\-ale} &
\textbf{MPE Ra\-tio\-nale} \\
\endhead

1 & 2;3 &
$( \psi \rightarrow \chi )$ &
$( \varphi \rightarrow ( \psi \rightarrow \chi ) )$ &
Given &
\$e; \texttt{adantr} to put in ND hypothesis \\

2 & 5 &
$ \lnot \chi$ &
$( \varphi \rightarrow \lnot \chi )$ &
Given &
\$e; \texttt{adantr} to put in ND hypothesis \\

3 & 1 &
... $\vert$ $\psi$ &
$( \varphi \rightarrow \psi )$ &
ND hypothesis assumption &
\texttt{simpr} \\

4 & 4 &
... $\chi$ &
$( ( \varphi \land \psi ) \rightarrow \chi )$ &
$\rightarrow$\,E 1,3 &
\texttt{mpd} 1,3 \\

5 & 6 &
... $\lnot \chi$ &
$( ( \varphi \land \psi ) \rightarrow \lnot \chi )$ &
IT 2 &
\texttt{adantr} 5 \\

6 & 7 &
$\lnot \psi$ &
$( \varphi \rightarrow \lnot \psi )$ &
$\land$\,I 3,4,5 &
\texttt{pm2.65da} 4,6 \\

\end{longtabu}
}


The original used Latin letters; we have replaced them with Greek letters
to follow Metamath naming conventions and so that it is easier to follow
the Metamath translation. The Metamath line-for-line translation of
this natural deduction approach precedes every line with an antecedent
including $\varphi$ and uses the Metamath equivalents of the natural deduction
rules. To add an assumption, the antecedent is modified to include it
(typically by using \texttt{adantr};
\texttt{simpr} is useful when you want to
depend directly on the new assumption, as is shown here).

In Metamath we can represent the two given statements as these hypotheses:

\needspace{2\baselineskip}
\begin{itemize}
\item ex-natded5.5.1 $\vdash ( \varphi \rightarrow ( \psi \rightarrow \chi ) )$
\item ex-natded5.5.2 $\vdash ( \varphi \rightarrow \lnot \chi )$
\end{itemize}

\needspace{4\baselineskip}
Here is the proof in Metamath as a line-by-line translation:

\begin{longtabu}   { l l l X }
\textbf{Step} & \textbf{Hyp} & \textbf{Ref} & \textbf{Ex\-pres\-sion} \\
\endhead
1 & & simpr & $\vdash ( ( \varphi \land \psi ) \rightarrow \psi )$ \\
2 & & ex-natded5.5.1 &
  $\vdash ( \varphi \rightarrow ( \psi \rightarrow \chi ) )$ \\
3 & 2 & adantr &
 $\vdash ( ( \varphi \land \psi ) \rightarrow ( \psi \rightarrow \chi ) )$ \\
4 & 1, 3 & mpd &
 $\vdash ( ( \varphi \land \psi ) \rightarrow \chi ) $ \\
5 & & ex-natded5.5.2 &
 $\vdash ( \varphi \rightarrow \lnot \chi )$ \\
6 & 5 & adantr &
 $\vdash ( ( \varphi \land \psi ) \rightarrow \lnot \chi )$ \\
7 & 4, 6 & pm2.65da &
 $\vdash ( \varphi \rightarrow \lnot \psi )$ \\
\end{longtabu}

Only using specific natural deduction rules directly can lead to very
long proofs, for exactly the same reason that only using axioms directly
in Hilbert-style proofs can lead to very long proofs.
If the goal is short and clear proofs,
then it is better to reuse already-proven assertions
in deduction form than to start from scratch each time
and using only basic natural deduction rules.

\subsection{Strengths of Our Approach}

As far as we know there is nothing else in the literature like either the
weak deduction theorem or Mario Carneiro\index{Carneiro, Mario}'s
natural deduction method.
In order to
transform a hypothesis into an antecedent, the literature's standard
``Deduction Theorem''\index{Deduction Theorem}\index{Standard Deduction Theorem}
requires metalogic outside of the notions provided
by the axiom system. We instead generally prefer to use Mario Carneiro's
natural deduction method, then use the weak deduction theorem in cases
where that is difficult to apply, and only then use the full standard
deduction theorem as a last resort.

The weak deduction theorem\index{Weak Deduction Theorem}
does not require any additional metalogic
but converts an inference directly into a closed form theorem, with
a rigorous proof that uses only the axiom system. Unlike the standard
Deduction Theorem, there is no implicit external justification that we
have to trust in order to use it.

Mario Carneiro's natural deduction\index{natural deduction}
method also does not require any new metalogical
notions. It avoids the Deduction Theorem's metalogic by prefixing the
hypotheses and conclusion of every would-be inference with a universal
antecedent (``$\varphi \rightarrow$'') from the very start.

We think it is impressive and satisfying that we can do so much in a
practical sense without stepping outside of our Hilbert-style axiom system.
Of course our axiomatization, which is in the form of schemes,
contains a metalogic of its own that we exploit. But this metalogic
is relatively simple, and for our Deduction Theorem alternatives,
we primarily use just the direct substitution of expressions for
metavariables.

\begin{sloppy}
\section{Exploring the Set The\-o\-ry Data\-base}\label{exploring}
\end{sloppy}
% NOTE: All examples performed in this section are
% recorded wtih "set width 61" % on set.mm as of 2019-05-28
% commit c1e7849557661260f77cfdf0f97ac4354fbb4f4d.

At this point you may wish to study the \texttt{set.mm}\index{set theory
database (\texttt{set.mm})} file in more detail.  Pay particular
attention to the assumptions needed to define wffs\index{well-formed
formula (wff)} (which are not included above), the variable types
(\texttt{\$f}\index{\texttt{\$f} statement} statements), and the
definitions that are introduced.  Start with some simple theorems in
propositional calculus, making sure you understand in detail each step
of a proof.  Once you get past the first few proofs and become familiar
with the Metamath language, any part of the \texttt{set.mm} database
will be as easy to follow, step by step, as any other part---you won't
have to undergo a ``quantum leap'' in mathematical sophistication to be
able to follow a deep proof in set theory.

Next, you may want to explore how concepts such as natural numbers are
defined and described.  This is probably best done in conjunction with
standard set theory textbooks, which can help give you a higher-level
understanding.  The \texttt{set.mm} database provides references that will get
you started.  From there, you will be on your way towards a very deep,
rigorous understanding of abstract mathematics.

The Metamath\index{Metamath} program can help you peruse a Metamath data\-base,
wheth\-er you are trying to figure out how a certain step follows in a proof or
just have a general curiosity.  We will go through some examples of the
commands, using the \texttt{set.mm}\index{set theory database (\texttt{set.mm})}
database provided with the Metamath software.  These should help get you
started.  See Chapter~\ref{commands} for a more detailed description of
the commands.  Note that we have included the full spelling of all commands to
prevent ambiguity with future commands.  In practice you may type just the
characters needed to specify each command keyword\index{command keyword}
unambiguously, often just one or two characters per keyword, and you don't
need to type them in upper case.

First run the Metamath program as described earlier.  You should see the
\verb/MM>/ prompt.  Read in the \texttt{set.mm} file:\index{\texttt{read}
command}

\begin{verbatim}
MM> read set.mm
Reading source file "set.mm"... 34554442 bytes
34554442 bytes were read into the source buffer.
The source has 155711 statements; 2254 are $a and 32250 are $p.
No errors were found.  However, proofs were not checked.
Type VERIFY PROOF * if you want to check them.
\end{verbatim}

As with most examples in this book, what you will see
will be slightly different because we are continuously
improving our databases (including \texttt{set.mm}).

Let's check the database integrity.  This may take a minute or two to run if
your computer is slow.

\begin{verbatim}
MM> verify proof *
0 10%  20%  30%  40%  50%  60%  70%  80%  90% 100%
..................................................
All proofs in the database were verified in 2.84 s.
\end{verbatim}

No errors were reported, so every proof is correct.

You need to know the names (labels) of theorems before you can look at them.
Often just examining the database file(s) with a text editor is the best
approach.  In \texttt{set.mm} there are many detailed comments, especially near
the beginning, that can help guide you. The \texttt{search} command in the
Metamath program is also handy.  The \texttt{comments} qualifier will list the
statements whose associated comment (the one immediately before it) contain a
string you give it.  For example, if you are studying Enderton's {\em Elements
of Set Theory} \cite{Enderton}\index{Enderton, Herbert B.} you may want to see
the references to it in the database.  The search string \texttt{enderton} is not
case sensitive.  (This will not show you all the database theorems that are in
Enderton's book because there is usually only one citation for a given
theorem, which may appear in several textbooks.)\index{\texttt{search}
command}

\begin{verbatim}
MM> search * "enderton" / comments
12067 unineq $p "... Exercise 20 of [Enderton] p. 32 and ..."
12459 undif2 $p "...Corollary 6K of [Enderton] p. 144. (C..."
12953 df-tp $a "...s. Definition of [Enderton] p. 19. (Co..."
13689 unissb $p ".... Exercise 5 of [Enderton] p. 26 and ..."
\end{verbatim}
\begin{center}
(etc.)
\end{center}

Or you may want to see what theorems have something to do with
conjunction (logical {\sc and}).  The quotes around the search
string are optional when there's no ambiguity.\index{\texttt{search}
command}

\begin{verbatim}
MM> search * conjunction / comments
120 a1d $p "...be replaced with a conjunction ( ~ df-an )..."
662 df-bi $a "...viated form after conjunction is introdu..."
1319 wa $a "...ff definition to include conjunction ('and')."
1321 df-an $a "Define conjunction (logical 'and'). Defini..."
1420 imnan $p "...tion in terms of conjunction. (Contribu..."
\end{verbatim}
\begin{center}
(etc.)
\end{center}


Now we will start to look at some details.  Let's look at the first
axiom of propositional calculus
(we could use \texttt{sh st} to abbreviate
\texttt{show statement}).\index{\texttt{show statement} command}

\begin{verbatim}
MM> show statement ax-1/full
Statement 19 is located on line 881 of the file "set.mm".
"Axiom _Simp_.  Axiom A1 of [Margaris] p. 49.  One of the 3
axioms of propositional calculus.  The 3 axioms are also
        ...
19 ax-1 $a |- ( ph -> ( ps -> ph ) ) $.
Its mandatory hypotheses in RPN order are:
  wph $f wff ph $.
  wps $f wff ps $.
The statement and its hypotheses require the variables:  ph
      ps
The variables it contains are:  ph ps


Statement 49 is located on line 11182 of the file "set.mm".
Its statement number for HTML pages is 6.
"Axiom _Simp_.  Axiom A1 of [Margaris] p. 49.  One of the 3
axioms of propositional calculus.  The 3 axioms are also
given as Definition 2.1 of [Hamilton] p. 28.
...
49 ax-1 $a |- ( ph -> ( ps -> ph ) ) $.
Its mandatory hypotheses in RPN order are:
  wph $f wff ph $.
  wps $f wff ps $.
The statement and its hypotheses require the variables:
  ph ps
The variables it contains are:  ph ps
\end{verbatim}

Compare this to \texttt{ax-1} on p.~\pageref{ax1}.  You can see that the
symbol \texttt{ph} is the {\sc ascii} notation for $\varphi$, etc.  To
see the mathematical symbols for any expression you may typeset it in
\LaTeX\ (type \texttt{help tex} for instructions)\index{latex@{\LaTeX}}
or, easier, just use a text editor to look at the comments where symbols
are first introduced in \texttt{set.mm}.  The hypotheses \texttt{wph}
and \texttt{wps} required by \texttt{ax-1} mean that variables
\texttt{ph} and \texttt{ps} must be wffs.

Next we'll pick a simple theorem of propositional calculus, the Principle of
Identity, which is proved directly from the axioms.  We'll look at the
statement then its proof.\index{\texttt{show statement}
command}

\begin{verbatim}
MM> show statement id1/full
Statement 116 is located on line 11371 of the file "set.mm".
Its statement number for HTML pages is 22.
"Principle of identity.  Theorem *2.08 of [WhiteheadRussell]
p. 101.  This version is proved directly from the axioms for
demonstration purposes.
...
116 id1 $p |- ( ph -> ph ) $= ... $.
Its mandatory hypotheses in RPN order are:
  wph $f wff ph $.
Its optional hypotheses are:  wps wch wth wta wet
      wze wsi wrh wmu wla wka
The statement and its hypotheses require the variables:  ph
These additional variables are allowed in its proof:
      ps ch th ta et ze si rh mu la ka
The variables it contains are:  ph
\end{verbatim}

The optional variables\index{optional variable} \texttt{ps}, \texttt{ch}, etc.\ are
available for use in a proof of this statement if we wish, and were we to do
so we would make use of optional hypotheses \texttt{wps}, \texttt{wch}, etc.  (See
Section~\ref{dollaref} for the meaning of ``optional
hypothesis.''\index{optional hypothesis}) The reason these show up in the
statement display is that statement \texttt{id1} happens to be in their scope
(see Section~\ref{scoping} for the definition of ``scope''\index{scope}), but
in fact in propositional calculus we will never make use of optional
hypotheses or variables.  This becomes important after quantifiers are
introduced, where ``dummy'' variables\index{dummy variable} are often needed
in the middle of a proof.

Let's look at the proof of statement \texttt{id1}.  We'll use the
\texttt{show proof} command, which by default suppresses the
``non-essential'' steps that construct the wffs.\index{\texttt{show proof}
command}
We will display the proof in ``lemmon' format (a non-indented format
with explicit previous step number references) and renumber the
displayed steps:

\begin{verbatim}
MM> show proof id1 /lemmon/renumber
1 ax-1           $a |- ( ph -> ( ph -> ph ) )
2 ax-1           $a |- ( ph -> ( ( ph -> ph ) -> ph ) )
3 ax-2           $a |- ( ( ph -> ( ( ph -> ph ) -> ph ) ) ->
                     ( ( ph -> ( ph -> ph ) ) -> ( ph -> ph )
                                                          ) )
4 2,3 ax-mp      $a |- ( ( ph -> ( ph -> ph ) ) -> ( ph -> ph
                                                          ) )
5 1,4 ax-mp      $a |- ( ph -> ph )
\end{verbatim}

If you have read Section~\ref{trialrun}, you'll know how to interpret this
proof.  Step~2, for example, is an application of axiom \texttt{ax-1}.  This
proof is identical to the one in Hamilton's {\em Logic for Mathematicians}
\cite[p.~32]{Hamilton}\index{Hamilton, Alan G.}.

You may want to look at what
substitutions\index{substitution!variable}\index{variable substitution} are
made into \texttt{ax-1} to arrive at step~2.  The command to do this needs to
know the ``real'' step number, so we'll display the proof again without
the \texttt{renumber} qualifier.\index{\texttt{show proof}
command}

\begin{verbatim}
MM> show proof id1 /lemmon
 9 ax-1          $a |- ( ph -> ( ph -> ph ) )
20 ax-1          $a |- ( ph -> ( ( ph -> ph ) -> ph ) )
24 ax-2          $a |- ( ( ph -> ( ( ph -> ph ) -> ph ) ) ->
                     ( ( ph -> ( ph -> ph ) ) -> ( ph -> ph )
                                                          ) )
25 20,24 ax-mp   $a |- ( ( ph -> ( ph -> ph ) ) -> ( ph -> ph
                                                          ) )
26 9,25 ax-mp    $a |- ( ph -> ph )
\end{verbatim}

The ``real'' step number is 20.  Let's look at its details.

\begin{verbatim}
MM> show proof id1 /detailed_step 20
Proof step 20:  min=ax-1 $a |- ( ph -> ( ( ph -> ph ) -> ph )
  )
This step assigns source "ax-1" ($a) to target "min" ($e).
The source assertion requires the hypotheses "wph" ($f, step
18) and "wps" ($f, step 19).  The parent assertion of the
target hypothesis is "ax-mp" ($a, step 25).
The source assertion before substitution was:
    ax-1 $a |- ( ph -> ( ps -> ph ) )
The following substitutions were made to the source
assertion:
    Variable  Substituted with
     ph        ph
     ps        ( ph -> ph )
The target hypothesis before substitution was:
    min $e |- ph
The following substitution was made to the target hypothesis:
    Variable  Substituted with
     ph        ( ph -> ( ( ph -> ph ) -> ph ) )
\end{verbatim}

This shows the substitutions\index{substitution!variable}\index{variable
substitution} made to the variables in \texttt{ax-1}.  References are made to
steps 18 and 19 which are not shown in our proof display.  To see these steps,
you can display the proof with the \texttt{all} qualifier.

Let's look at a slightly more advanced proof of propositional calculus.  Note
that \verb+/\+ is the symbol for $\wedge$ (logical {\sc and}, also
called conjunction).\index{conjunction ($\wedge$)}
\index{logical {\sc and} ($\wedge$)}

\begin{verbatim}
MM> show statement prth/full
Statement 1791 is located on line 15503 of the file "set.mm".
Its statement number for HTML pages is 559.
"Conjoin antecedents and consequents of two premises.  This
is the closed theorem form of ~ anim12d .  Theorem *3.47 of
[WhiteheadRussell] p. 113.  It was proved by Leibniz,
and it evidently pleased him enough to call it
_praeclarum theorema_ (splendid theorem).
...
1791 prth $p |- ( ( ( ph -> ps ) /\ ( ch -> th ) ) -> ( ( ph
      /\ ch ) -> ( ps /\ th ) ) ) $= ... $.
Its mandatory hypotheses in RPN order are:
  wph $f wff ph $.
  wps $f wff ps $.
  wch $f wff ch $.
  wth $f wff th $.
Its optional hypotheses are:  wta wet wze wsi wrh wmu wla wka
The statement and its hypotheses require the variables:  ph
      ps ch th
These additional variables are allowed in its proof:  ta et
      ze si rh mu la ka
The variables it contains are:  ph ps ch th


MM> show proof prth /lemmon/renumber
1 simpl          $p |- ( ( ( ph -> ps ) /\ ( ch -> th ) ) ->
                                               ( ph -> ps ) )
2 simpr          $p |- ( ( ( ph -> ps ) /\ ( ch -> th ) ) ->
                                               ( ch -> th ) )
3 1,2 anim12d    $p |- ( ( ( ph -> ps ) /\ ( ch -> th ) ) ->
                           ( ( ph /\ ch ) -> ( ps /\ th ) ) )
\end{verbatim}

There are references to a lot of unfamiliar statements.  To see what they are,
you may type the following:

\begin{verbatim}
MM> show proof prth /statement_summary
Summary of statements used in the proof of "prth":

Statement simpl is located on line 14748 of the file
"set.mm".
"Elimination of a conjunct.  Theorem *3.26 (Simp) of
[WhiteheadRussell] p. 112. ..."
  simpl $p |- ( ( ph /\ ps ) -> ph ) $= ... $.

Statement simpr is located on line 14777 of the file
"set.mm".
"Elimination of a conjunct.  Theorem *3.27 (Simp) of
[WhiteheadRussell] ..."
  simpr $p |- ( ( ph /\ ps ) -> ps ) $= ... $.

Statement anim12d is located on line 15445 of the file
"set.mm".
"Conjoin antecedents and consequents in a deduction.
..."
  anim12d.1 $e |- ( ph -> ( ps -> ch ) ) $.
  anim12d.2 $e |- ( ph -> ( th -> ta ) ) $.
  anim12d $p |- ( ph -> ( ( ps /\ th ) -> ( ch /\ ta ) ) )
      $= ... $.
\end{verbatim}
\begin{center}
(etc.)
\end{center}

Of course you can look at each of these statements and their proofs, and
so on, back to the axioms of propositional calculus if you wish.

The \texttt{search} command is useful for finding statements when you
know all or part of their contents.  The following example finds all
statements containing \verb@ph -> ps@ followed by \verb@ch -> th@.  The
\verb@$*@ is a wildcard that matches anything; the \texttt{\$} before the
\verb$*$ prevents conflicts with math symbol token names.  The \verb@*@ after
\texttt{SEARCH} is also a wildcard that in this case means ``match any label.''
\index{\texttt{search} command}

% I'm omitting this one, since readers are unlikely to see it:
% 1096 bisymOLD $p |- ( ( ( ph -> ps ) -> ( ch -> th ) ) -> ( (
%   ( ps -> ph ) -> ( th -> ch ) ) -> ( ( ph <-> ps ) -> ( ch
%    <-> th ) ) ) )
\begin{verbatim}
MM> search * "ph -> ps $* ch -> th"
1791 prth $p |- ( ( ( ph -> ps ) /\ ( ch -> th ) ) -> ( ( ph
    /\ ch ) -> ( ps /\ th ) ) )
2455 pm3.48 $p |- ( ( ( ph -> ps ) /\ ( ch -> th ) ) -> ( (
    ph \/ ch ) -> ( ps \/ th ) ) )
117859 pm11.71 $p |- ( ( E. x ph /\ E. y ch ) -> ( ( A. x (
    ph -> ps ) /\ A. y ( ch -> th ) ) <-> A. x A. y ( ( ph /\
    ch ) -> ( ps /\ th ) ) ) )
\end{verbatim}

Three statements, \texttt{prth}, \texttt{pm3.48},
 and \texttt{pm11.71}, were found to match.

To see what axioms\index{axiom} and definitions\index{definition}
\texttt{prth} ultimately depends on for its proof, you can have the
program backtrack through the hierarchy\index{hierarchy} of theorems and
definitions.\index{\texttt{show trace{\char`\_}back} command}

\begin{verbatim}
MM> show trace_back prth /essential/axioms
Statement "prth" assumes the following axioms ($a
statements):
  ax-1 ax-2 ax-3 ax-mp df-bi df-an
\end{verbatim}

Note that the 3 axioms of propositional calculus and the modus ponens rule are
needed (as expected); in addition, there are a couple of definitions that are used
along the way.  Note that Metamath makes no distinction\index{axiom vs.\
definition} between axioms\index{axiom} and definitions\index{definition}.  In
\texttt{set.mm} they have been distinguished artificially by prefixing their
labels\index{labels in \texttt{set.mm}} with \texttt{ax-} and \texttt{df-}
respectively.  For example, \texttt{df-an} defines conjunction (logical {\sc
and}), which is represented by the symbol \verb+/\+.
Section~\ref{definitions} discusses the philosophy of definitions, and the
Metamath language takes a particularly simple, conservative approach by using
the \texttt{\$a}\index{\texttt{\$a} statement} statement for both axioms and
definitions.

You can also have the program compute how many steps a proof
has\index{proof length} if we were to follow it all the way back to
\texttt{\$a} statements.

\begin{verbatim}
MM> show trace_back prth /essential/count_steps
The statement's actual proof has 3 steps.  Backtracking, a
total of 79 different subtheorems are used.  The statement
and subtheorems have a total of 274 actual steps.  If
subtheorems used only once were eliminated, there would be a
total of 38 subtheorems, and the statement and subtheorems
would have a total of 185 steps.  The proof would have 28349
steps if fully expanded back to axiom references.  The
maximum path length is 38.  A longest path is:  prth <-
anim12d <- syl2and <- sylan2d <- ancomsd <- ancom <- pm3.22
<- pm3.21 <- pm3.2 <- ex <- sylbir <- biimpri <- bicomi <-
bicom1 <- bi2 <- dfbi1 <- impbii <- bi3 <- simprim <- impi <-
con1i <- nsyl2 <- mt3d <- con1d <- notnot1 <- con2i <- nsyl3
<- mt2d <- con2d <- notnot2 <- pm2.18d <- pm2.18 <- pm2.21 <-
pm2.21d <- a1d <- syl <- mpd <- a2i <- a2i.1 .
\end{verbatim}

This tells us that we would have to inspect 274 steps if we want to
verify the proof completely starting from the axioms.  A few more
statistics are also shown.  There are one or more paths back to axioms
that are the longest; this command ferrets out one of them and shows it
to you.  There may be a sense in which the longest path length is
related to how ``deep'' the theorem is.

We might also be curious about what proofs depend on the theorem
\texttt{prth}.  If it is never used later on, we could eliminate it as
redundant if it has no intrinsic interest by itself.\index{\texttt{show
usage} command}

% I decided to show the OLD values here.
\begin{verbatim}
MM> show usage prth
Statement "prth" is directly referenced in the proofs of 18
statements:
  mo3 moOLD 2mo 2moOLD euind reuind reuss2 reusv3i opelopabt
  wemaplem2 rexanre rlimcn2 o1of2 o1rlimmul 2sqlem6 spanuni
  heicant pm11.71
\end{verbatim}

Thus \texttt{prth} is directly used by 18 proofs.
We can use the \texttt{/recursive} qualifier to include indirect use:

\begin{verbatim}
MM> show usage prth /recursive
Statement "prth" directly or indirectly affects the proofs of
24214 statements:
  mo3 mo mo3OLD eu2 moOLD eu2OLD eu3OLD mo4f mo4 eu4 mopick
...
\end{verbatim}

\subsection{A Note on the ``Compact'' Proof Format}

The present version of Metamath\index{Metamath!limitations of
version 0.07.30} will display proofs in a ``compact''\index{compact proof}
format whenever the proof is stored in compressed format in the database.  It
may be be slightly confusing unless you know how to interpret it.  (A
future version may eliminate this format from displays.)  For example,
if you display the complete proof of theorem \texttt{id1} it will start
off as follows:

\begin{verbatim}
MM> show proof id1 /lemmon/all
 1 wph           $f wff ph
 2 wph           $f wff ph
 3 wph           $f wff ph
 4 2,3 wi    @4: $a wff ( ph -> ph )
 5 1,4 wi    @5: $a wff ( ph -> ( ph -> ph ) )
 6 @4            $a wff ( ph -> ph )
\end{verbatim}

\begin{center}
{etc.}
\end{center}

Step 4 has a ``local label,''\index{local label} \texttt{@4}, assigned to it.
Later on, at step 6, the label \texttt{@1} is referenced instead of
displaying the explicit proof for that step.  This technique takes advantage
of the fact that steps in a proof often repeat, especially during the
construction of wffs.  The compact format reduces the number of steps in the
proof display and may actually be preferred by some people, which is one
reason that it is still displayed by the program.

If you want to see the normal format with the ``true'' step numbers, you can
use the following workaround:\index{\texttt{save proof} command}

\begin{verbatim}
MM> save proof id1 /normal
The proof of "id1" has been reformatted and saved internally.
Remember to use WRITE SOURCE to save it permanently.
MM> show proof id1 /lemmon/all
 1 wph           $f wff ph
 2 wph           $f wff ph
 3 wph           $f wff ph
 4 2,3 wi        $a wff ( ph -> ph )
 5 1,4 wi        $a wff ( ph -> ( ph -> ph ) )
 6 wph           $f wff ph
 7 wph           $f wff ph
 8 6,7 wi        $a wff ( ph -> ph )
\end{verbatim}

\begin{center}
{etc.}
\end{center}

Note that the original 6 steps are now 8 steps.  However, the format is
now the same as that described in Chapter~\ref{using}.

\chapter{The Metamath Language}
\label{languagespec}

\begin{quote}
  {\em Thus mathematics may be defined as the subject in which we never know
what we are talking about, nor whether what we are saying is true.}
    \flushright\sc  Bertrand Russell\footnote{\cite[p.~84]{Russell2}.}\\
\end{quote}\index{Russell, Bertrand}

Probably the most striking feature of the Metamath language is its almost
complete absence of hard-wired syntax. Metamath\index{Metamath} does not
understand any mathematics or logic other than that needed to construct finite
sequences of symbols according to a small set of simple, built-in rules.  The
only rule it uses in a proof is the substitution of an expression (symbol
sequence) for a variable, subject to a simple constraint to prevent
bound-variable clashes.  The primitive notions built into Metamath involve the
simple manipulation of finite objects (symbols) that we as humans can easily
visualize and that computers can easily deal with.  They seem to be just
about the simplest notions possible that are required to do standard
mathematics.

This chapter serves as a reference manual for the Metamath\index{Metamath}
language. It covers the tedious technical details of the language, some of
which you may wish to skim in a first reading.  On the other hand, you should
pay close attention to the defined terms in {\bf boldface}; they have precise
meanings that are important to keep in mind for later understanding.  It may
be best to first become familiar with the examples in Chapter~\ref{using} to
gain some motivation for the language.

%% Uncomment this when uncommenting section {formalspec} below
If you have some knowledge of set theory, you may wish to study this
chapter in conjunction with the formal set-theoretical description of the
Metamath language in Appendix~\ref{formalspec}.

We will use the name ``Metamath''\index{Metamath} to mean either the Metamath
computer language or the Metamath software associated with the computer
language.  We will not distinguish these two when the context is clear.

The next section contains the complete specification of the Metamath
language.
It serves as an
authoritative reference and presents the syntax in enough detail to
write a parser\index{parsing Metamath} and proof verifier.  The
specification is terse and it is probably hard to learn the language
directly from it, but we include it here for those impatient people who
prefer to see everything up front before looking at verbose expository
material.  Later sections explain this material and provide examples.
We will repeat the definitions in those sections, and you may skip the
next section at first reading and proceed to Section~\ref{tut1}
(p.~\pageref{tut1}).

\section{Specification of the Metamath Language}\label{spec}
\index{Metamath!specification}

\begin{quote}
  {\em Sometimes one has to say difficult things, but one ought to say
them as simply as one knows how.}
    \flushright\sc  G. H. Hardy\footnote{As quoted in
    \cite{deMillo}, p.~273.}\\
\end{quote}\index{Hardy, G. H.}

\subsection{Preliminaries}\label{spec1}

% Space is technically a printable character, so we'll word things
% carefully so it's unambiguous.
A Metamath {\bf database}\index{database} is built up from a top-level source
file together with any source files that are brought in through file inclusion
commands (see below).  The only characters that are allowed to appear in a
Metamath source file are the 94 non-whitespace printable {\sc
ascii}\index{ascii@{\sc ascii}} characters, which are digits, upper and lower
case letters, and the following 32 special
characters\index{special characters}:\label{spec1chars}

\begin{verbatim}
! " # $ % & ' ( ) * + , - . / :
; < = > ? @ [ \ ] ^ _ ` { | } ~
\end{verbatim}
plus the following characters which are the ``white space'' characters:
space (a printable character),
tab, carriage return, line feed, and form feed.\label{whitespace}
We will use \texttt{typewriter}
font to display the printable characters.

A Metamath database consists of a sequence of three kinds of {\bf
tokens}\index{token} separated by {\bf white space}\index{white space}
(which is any sequence of one or more white space characters).  The set
of {\bf keyword}\index{keyword} tokens is \texttt{\$\char`\{},
\texttt{\$\char`\}}, \texttt{\$c}, \texttt{\$v}, \texttt{\$f},
\texttt{\$e}, \texttt{\$d}, \texttt{\$a}, \texttt{\$p}, \texttt{\$.},
\texttt{\$=}, \texttt{\$(}, \texttt{\$)}, \texttt{\$[}, and
\texttt{\$]}.  The last four are called {\bf auxiliary}\index{auxiliary
keyword} or preprocessing keywords.  A {\bf label}\index{label} token
consists of any combination of letters, digits, and the characters
hyphen, underscore, and period.  A {\bf math symbol}\index{math symbol}
token may consist of any combination of the 92 printable standard {\sc
ascii} characters other than space or \texttt{\$}~. All tokens are
case-sensitive.

\subsection{Preprocessing}

The token \texttt{\$(} begins a {\bf comment} and
\texttt{\$)} ends a comment.\index{\texttt{\$(}
and \texttt{\$)} auxiliary keywords}\index{comment}
Comments may contain any of
the 94 non-whitespace printable characters and white space,
except they may not contain the
2-character sequences \texttt{\$(} or \texttt{\$)} (comments do not nest).
Comments are ignored (treated
like white space) for the purpose of parsing, e.g.,
\texttt{\$( \$[ \$)} is a comment.
See p.~\pageref{mathcomments} for comment typesetting conventions; these
conventions may be ignored for the purpose of parsing.

A {\bf file inclusion command} consists of \texttt{\$[} followed by a file name
followed by \texttt{\$]}.\index{\texttt{\$[} and \texttt{\$]} auxiliary
keywords}\index{included file}\index{file inclusion}
It is only allowed in the outermost scope (i.e., not between
\texttt{\$\char`\{} and \texttt{\$\char`\}})
and must not be inside a statement (e.g., it may not occur
between the label of a \texttt{\$a} statement and its \texttt{\$.}).
The file name may not
contain a \texttt{\$} or white space.  The file must exist.
The case-sensitivity
of its name follows the conventions of the operating system.  The contents of
the file replace the inclusion command.
Included files may include other files.
Only the first reference to a given file is included; any later
references to the same file (whether in the top-level file or in included
files) cause the inclusion command to be ignored (treated like white space).
A verifier may assume that file names with different strings
refer to different files for the purpose of ignoring later references.
A file self-reference is ignored, as is any reference to the top-level file
(to avoid loops).
Included files may not include a \texttt{\$(} without a matching \texttt{\$)},
may not include a \texttt{\$[} without a matching \texttt{\$]}, and may
not include incomplete statements (e.g., a \texttt{\$a} without a matching
\texttt{\$.}).
It is currently unspecified if path references are relative to the process'
current directory or the file's containing directory, so databases should
avoid using pathname separators (e.g., ``/'') in file names.

Like all tokens, the \texttt{\$(}, \texttt{\$)}, \texttt{\$[}, and \texttt{\$]} keywords
must be surrounded by white space.

\subsection{Basic Syntax}

After preprocessing, a database will consist of a sequence of {\bf
statements}.
These are the scoping statements \texttt{\$\char`\{} and
\texttt{\$\char`\}}, along with the \texttt{\$c}, \texttt{\$v},
\texttt{\$f}, \texttt{\$e}, \texttt{\$d}, \texttt{\$a}, and \texttt{\$p}
statements.

A {\bf scoping statement}\index{scoping statement} consists only of its
keyword, \texttt{\$\char`\{} or \texttt{\$\char`\}}.
A \texttt{\$\char`\{} begins a {\bf
block}\index{block} and a matching \texttt{\$\char`\}} ends the block.
Every \texttt{\$\char`\{}
must have a matching \texttt{\$\char`\}}.
Defining it recursively, we say a block
contains a sequence of zero or more tokens other
than \texttt{\$\char`\{} and \texttt{\$\char`\}} and
possibly other blocks.  There is an {\bf outermost
block}\index{block!outermost} not bracketed by \texttt{\$\char`\{} \texttt{\$\char`\}}; the end
of the outermost block is the end of the database.

% LaTeX bug? can't do \bf\tt

A {\bf \$v} or {\bf \$c statement}\index{\texttt{\$v} statement}\index{\texttt{\$c}
statement} consists of the keyword token \texttt{\$v} or \texttt{\$c} respectively,
followed by one or more math symbols,
% The word "token" is used to distinguish "$." from the sentence-ending period.
followed by the \texttt{\$.}\ token.
These
statements {\bf declare}\index{declaration} the math symbols to be {\bf
variables}\index{variable!Metamath} or {\bf constants}\index{constant}
respectively. The same math symbol may not occur twice in a given \texttt{\$v} or
\texttt{\$c} statement.

%c%A math symbol becomes an {\bf active}\index{active math symbol}
%c%when declared and stays active until the end of the block in which it is
%c%declared.  A math symbol may not be declared a second time while it is active,
%c%but it may be declared again after it becomes inactive.

A math symbol becomes {\bf active}\index{active math symbol} when declared
and stays active until the end of the block in which it is declared.  A
variable may not be declared a second time while it is active, but it
may be declared again (as a variable, but not as a constant) after it
becomes inactive.  A constant must be declared in the outermost block and may
not be declared a second time.\footnote{The rules for redeclaration may
become more general in the future;
see p.~\pageref{redeclarationf}.}\index{redeclaration of symbols}

A {\bf \$f statement}\index{\texttt{\$f} statement} consists of a label,
followed by \texttt{\$f}, followed by its typecode (an active constant),
followed by an
active variable, followed by the \texttt{\$.}\ token.  A {\bf \$e
statement}\index{\texttt{\$e} statement} consists of a label, followed
by \texttt{\$e}, followed by its typecode (an active constant),
followed by zero or more
active math symbols, followed by the \texttt{\$.}\ token.  A {\bf
hypothesis}\index{hypothesis} is a \texttt{\$f} or \texttt{\$e}
statement.
The type declared by a \texttt{\$f} statement for a given label
is global even if the variable is not
(e.g., a database may not have \texttt{wff P} in one local scope
nd \texttt{class P} in another).

A {\bf simple \$d statement}\index{\texttt{\$d} statement!simple}
consists of \texttt{\$d}, followed by two different active variables,
followed by the \texttt{\$.}\ token.  A {\bf compound \$d
statement}\index{\texttt{\$d} statement!compound} consists of
\texttt{\$d}, followed by three or more variables (all different),
followed by the \texttt{\$.}\ token.  The order of the variables in a
\texttt{\$d} statement is unimportant.  A compound \texttt{\$d}
statement is equivalent to a set of simple \texttt{\$d} statements, one
for each possible pair of variables occurring in the compound
\texttt{\$d} statement.  Henceforth in this specification we shall
assume all \texttt{\$d} statements are simple.  A \texttt{\$d} statement
is also called a {\bf disjoint} (or {\bf distinct}) {\bf variable
restriction}.\index{disjoint-variable restriction}

A {\bf \$a statement}\index{\texttt{\$a} statement} consists of a label,
followed by \texttt{\$a}, followed by its typecode (an active constant),
followed by
zero or more active math symbols, followed by the \texttt{\$.}\ token.  A {\bf
\$p statement}\index{\texttt{\$p} statement} consists of a label,
followed by \texttt{\$p}, followed by its typecode (an active constant),
followed by
zero or more active math symbols, followed by \texttt{\$=}, followed by
a sequence of labels, followed by the \texttt{\$.}\ token.  An {\bf
assertion}\index{assertion} is a \texttt{\$a} or \texttt{\$p} statement.

A \texttt{\$f}, \texttt{\$e}, or \texttt{\$d} statement is {\bf active}\index{active
statement} from the place it occurs until the end of the block it occurs in.
A \texttt{\$a} or \texttt{\$p} statement is {\bf active} from the place it occurs
through the end of the database.
There may not be two active \texttt{\$f} statements containing the same
variable.  Each variable in a \texttt{\$e}, \texttt{\$a}, or
\texttt{\$p} statement must exist in an active \texttt{\$f}
statement.\footnote{This requirement can greatly simplify the
unification algorithm (substitution calculation) required by proof
verification.}

%The label that begins each \texttt{\$f}, \texttt{\$e}, \texttt{\$a}, and
%\texttt{\$p} statement must be unique.
Each label token must be unique, and
no label token may match any math symbol
token.\label{namespace}\footnote{This
restriction was added on June 24, 2006.
It is not theoretically necessary but is imposed to make it easier to
write certain parsers.}

The set of {\bf mandatory variables}\index{mandatory variable} associated with
an assertion is the set of (zero or more) variables in the assertion and in any
active \texttt{\$e} statements.  The (possibly empty) set of {\bf mandatory
hypotheses}\index{mandatory hypothesis} is the set of all active \texttt{\$f}
statements containing mandatory variables, together with all active \texttt{\$e}
statements.
The set of {\bf mandatory {\bf \$d} statements}\index{mandatory
disjoint-variable restriction} associated with an assertion are those active
\texttt{\$d} statements whose variables are both among the assertion's
mandatory variables.

\subsection{Proof Verification}\label{spec4}

The sequence of labels between the \texttt{\$=} and \texttt{\$.}\ tokens
in a \texttt{\$p} statement is a {\bf proof}.\index{proof!Metamath} Each
label in a proof must be the label of an active statement other than the
\texttt{\$p} statement itself; thus a label must refer either to an
active hypothesis of the \texttt{\$p} statement or to an earlier
assertion.

An {\bf expression}\index{expression} is a sequence of math symbols. A {\bf
substitution map}\index{substitution map} associates a set of variables with a
set of expressions.  It is acceptable for a variable to be mapped to an
expression containing it.  A {\bf
substitution}\index{substitution!variable}\index{variable substitution} is the
simultaneous replacement of all variables in one or more expressions with the
expressions that the variables map to.

A proof is scanned in order of its label sequence.  If the label refers to an
active hypothesis, the expression in the hypothesis is pushed onto a
stack.\index{stack}\index{RPN stack}  If the label refers to an assertion, a
(unique) substitution must exist that, when made to the mandatory hypotheses
of the referenced assertion, causes them to match the topmost (i.e.\ most
recent) entries of the stack, in order of occurrence of the mandatory
hypotheses, with the topmost stack entry matching the last mandatory
hypothesis of the referenced assertion.  As many stack entries as there are
mandatory hypotheses are then popped from the stack.  The same substitution is
made to the referenced assertion, and the result is pushed onto the stack.
After the last label in the proof is processed, the stack must have a single
entry that matches the expression in the \texttt{\$p} statement containing the
proof.

%c%{\footnotesize\begin{quotation}\index{redeclaration of symbols}
%c%{{\em Comment.}\label{spec4comment} Whenever a math symbol token occurs in a
%c%{\texttt{\$c} or \texttt{\$v} statement, it is considered to designate a distinct new
%c%{symbol, even if the same token was previously declared (and is now inactive).
%c%{Thus a math token declared as a constant in two different blocks is considered
%c%{to designate two distinct constants (even though they have the same name).
%c%{The two constants will not match in a proof that references both blocks.
%c%{However, a proof referencing both blocks is acceptable as long as it doesn't
%c%{require that the constants match.  Similarly, a token declared to be a
%c%{constant for a referenced assertion will not match the same token declared to
%c%{be a variable for the \texttt{\$p} statement containing the proof.  In the case
%c%{of a token declared to be a variable for a referenced assertion, this is not
%c%{an issue since the variable can be substituted with whatever expression is
%c%{needed to achieve the required match.
%c%{\end{quotation}}
%c2%A proof may reference an assertion that contains or whose hypotheses contain a
%c2%constant that is not active for the \texttt{\$p} statement containing the proof.
%c2%However, the final result of the proof may not contain that constant. A proof
%c2%may also reference an assertion that contains or whose hypotheses contain a
%c2%variable that is not active for the \texttt{\$p} statement containing the proof.
%c2%That variable, of course, will be substituted with whatever expression is
%c2%needed to achieve the required match.

A proof may contain a \texttt{?}\ in place of a label to indicate an unknown step
(Section~\ref{unknown}).  A proof verifier may ignore any proof containing
\texttt{?}\ but should warn the user that the proof is incomplete.

A {\bf compressed proof}\index{compressed proof}\index{proof!compressed} is an
alternate proof notation described in Appen\-dix~\ref{compressed}; also see
references to ``compressed proof'' in the Index.  Compressed proofs are a
Metamath language extension which a complete proof verifier should be able to
parse and verify.

\subsubsection{Verifying Disjoint Variable Restrictions}

Each substitution made in a proof must be checked to verify that any
disjoint variable restrictions are satisfied, as follows.

If two variables replaced by a substitution exist in a mandatory \texttt{\$d}
statement\index{\texttt{\$d} statement} of the assertion referenced, the two
expressions resulting from the substitution must satisfy the following
conditions.  First, the two expressions must have no variables in common.
Second, each possible pair of variables, one from each expression, must exist
in an active \texttt{\$d} statement of the \texttt{\$p} statement containing the
proof.

\vskip 1ex

This ends the specification of the Metamath language;
see Appendix \ref{BNF} for its syntax in
Extended Backus--Naur Form (EBNF)\index{Extended Backus--Naur Form}\index{EBNF}.

\section{The Basic Keywords}\label{tut1}

Our expository material begins here.

Like most computer languages, Metamath\index{Metamath} takes its input from
one or more {\bf source files}\index{source file} which contain characters
expressed in the standard {\sc ascii} (American Standard Code for Information
Interchange)\index{ascii@{\sc ascii}} code for computers.  A source file
consists of a series of {\bf tokens}\index{token}, which are strings of
non-whitespace
printable characters (from the set of 94 shown on p.~\pageref{spec1chars})
separated by {\bf white space}\index{white space} (spaces, tabs, carriage
returns, line feeds, and form feeds). Any string consisting only of these
characters is treated the same as a single space.  The non-whitespace printable
characters\index{printable character} that Metamath recognizes are the 94
characters on standard {\sc ascii} keyboards.

Metamath has the ability to join several files together to form its
input (Section~\ref{include}).  We call the aggregate contents of all
the files after they have been joined together a {\bf
database}\index{database} to distinguish it from an individual source
file.  The tokens in a database consist of {\bf
keywords}\index{keyword}, which are built into the language, together
with two kinds of user-defined tokens called {\bf labels}\index{label}
and {\bf math symbols}\index{math symbol}.  (Often we will simply say
{\bf symbol}\index{symbol} instead of math symbol for brevity).  The set
of {\bf basic keywords}\index{basic keyword} is
\texttt{\$c}\index{\texttt{\$c} statement},
\texttt{\$v}\index{\texttt{\$v} statement},
\texttt{\$e}\index{\texttt{\$e} statement},
\texttt{\$f}\index{\texttt{\$f} statement},
\texttt{\$d}\index{\texttt{\$d} statement},
\texttt{\$a}\index{\texttt{\$a} statement},
\texttt{\$p}\index{\texttt{\$p} statement},
\texttt{\$=}\index{\texttt{\$=} keyword},
\texttt{\$.}\index{\texttt{\$.}\ keyword},
\texttt{\$\char`\{}\index{\texttt{\$\char`\{} and \texttt{\$\char`\}}
keywords}, and \texttt{\$\char`\}}.  This is the complete set of
syntactical elements of what we call the {\bf basic
language}\index{basic language} of Metamath, and with them you can
express all of the mathematics that were intended by the design of
Metamath.  You should make it a point to become very familiar with them.
Table~\ref{basickeywords} lists the basic keywords along with a brief
description of their functions.  For now, this description will give you
only a vague notion of what the keywords are for; later we will describe
the keywords in detail.


\begin{table}[htp] \caption{Summary of the basic Metamath
keywords} \label{basickeywords}
\begin{center}
\begin{tabular}{|p{4pc}|l|}
\hline
\em \centering Keyword&\em Description\\
\hline
\hline
\centering
   \texttt{\$c}&Constant symbol declaration\\
\hline
\centering
   \texttt{\$v}&Variable symbol declaration\\
\hline
\centering
   \texttt{\$d}&Disjoint variable restriction\\
\hline
\centering
   \texttt{\$f}&Variable-type (``floating'') hypothesis\\
\hline
\centering
   \texttt{\$e}&Logical (``essential'') hypothesis\\
\hline
\centering
   \texttt{\$a}&Axiomatic assertion\\
\hline
\centering
   \texttt{\$p}&Provable assertion\\
\hline
\centering
   \texttt{\$=}&Start of proof in \texttt{\$p} statement\\
\hline
\centering
   \texttt{\$.}&End of the above statement types\\
\hline
\centering
   \texttt{\$\char`\{}&Start of block\\
\hline
\centering
   \texttt{\$\char`\}}&End of block\\
\hline
\end{tabular}
\end{center}
\end{table}

%For LaTeX bug(?) where it puts tables on blank page instead of btwn text
%May have to adjust if text changes
%\newpage

There are some additional keywords, called {\bf auxiliary
keywords}\index{auxiliary keyword} that help make Metamath\index{Metamath}
more practical. These are part of the {\bf extended language}\index{extended
language}. They provide you with a means to put comments into a Metamath
source file\index{source file} and reference other source files.  We will
introduce these in later sections. Table~\ref{otherkeywords} summarizes them
so that you can recognize them now if you want to peruse some source
files while learning the basic keywords.


\begin{table}[htp] \caption{Auxiliary Metamath
keywords} \label{otherkeywords}
\begin{center}
\begin{tabular}{|p{4pc}|l|}
\hline
\em \centering Keyword&\em Description\\
\hline
\hline
\centering
   \texttt{\$(}&Start of comment\\
\hline
\centering
   \texttt{\$)}&End of comment\\
\hline
\centering
   \texttt{\$[}&Start of included source file name\\
\hline
\centering
   \texttt{\$]}&End of included source file name\\
\hline
\end{tabular}
\end{center}
\end{table}
\index{\texttt{\$(} and \texttt{\$)} auxiliary keywords}
\index{\texttt{\$[} and \texttt{\$]} auxiliary keywords}


Unlike those in some computer languages, the keywords\index{keyword} are short
two-character sequences rather than English-like words.  While this may make
them slightly more difficult to remember at first, their brevity allows
them to blend in with the mathematics being described, not
distract from it, like punctuation marks.


\subsection{User-Defined Tokens}\label{dollardollar}\index{token}

As you may have noticed, all keywords\index{keyword} begin with the \texttt{\$}
character.  This mundane monetary symbol is not ordinarily used in higher
mathematics (outside of grant proposals), so we have appropriated it to
distinguish the Metamath\index{Metamath} keywords from ordinary mathematical
symbols. The \texttt{\$} character is thus considered special and may not be
used as a character in a user-defined token.  All tokens and keywords are
case-sensitive; for example, \texttt{n} is considered to be a different character
from \texttt{N}.  Case-sensitivity makes the available {\sc ascii} character set
as rich as possible.

\subsubsection{Math Symbol Tokens}\index{token}

Math symbols\index{math symbol} are tokens used to represent the symbols
that appear in ordinary mathematical formulas.  They may consist of any
combination of the 93 non-whitespace printable {\sc ascii} characters other than
\texttt{\$}~. Some examples are \texttt{x}, \texttt{+}, \texttt{(},
\texttt{|-}, \verb$!%@?&$, and \texttt{bounded}.  For readability, it is
best to try to make these look as similar to actual mathematical symbols
as possible, within the constraints of the {\sc ascii} character set, in
order to make the resulting mathematical expressions more readable.

In the Metamath\index{Metamath} language, you express ordinary
mathematical formulas and statements as sequences of math symbols such
as \texttt{2 + 2 = 4} (five symbols, all constants).\footnote{To
eliminate ambiguity with other expressions, this is expressed in the set
theory database \texttt{set.mm} as \texttt{|- ( 2 + 2
 ) = 4 }, whose \LaTeX\ equivalent is $\vdash
(2+2)=4$.  The \,$\vdash$ means ``is a theorem'' and the
parentheses allow explicit associative grouping.}\index{turnstile
({$\,\vdash$})} They may even be English
sentences, as in \texttt{E is closed and bounded} (five symbols)---here
\texttt{E} would be a variable and the other four symbols constants.  In
principle, a Metamath database could be constructed to work with almost
any unambiguous English-language mathematical statement, but as a
practical matter the definitions needed to provide for all possible
syntax variations would be cumbersome and distracting and possibly have
subtle pitfalls accidentally built in.  We generally recommend that you
express mathematical statements with compact standard mathematical
symbols whenever possible and put their English-language descriptions in
comments.  Axioms\index{axiom} and definitions\index{definition}
(\texttt{\$a}\index{\texttt{\$a} statement} statements) are the only
places where Metamath will not detect an error, and doing this will help
reduce the number of definitions needed.

You are free to use any tokens\index{token} you like for math
symbols\index{math symbol}.  Appendix~\ref{ASCII} recommends token names to
use for symbols in set theory, and we suggest you adopt these in order to be
able to include the \texttt{set.mm} set theory database in your database.  For
printouts, you can convert the tokens in a database
to standard mathematical symbols with the \LaTeX\ typesetting program.  The
Metamath command \texttt{open tex} {\em filename}\index{\texttt{open tex} command}
produces output that can be read by \LaTeX.\index{latex@{\LaTeX}}
The correspondence
between tokens and the actual symbols is made by \texttt{latexdef}
statements inside a special database comment tagged
with \texttt{\$t}.\index{\texttt{\$t} comment}\index{typesetting comment}
  You can edit
this comment to change the definitions or add new ones.
Appendix~\ref{ASCII} describes how to do this in more detail.

% White space\index{white space} is normally used to separate math
% symbol\index{math symbol} tokens, but they may be juxtaposed without white
% space in \texttt{\$d}\index{\texttt{\$d} statement}, \texttt{\$e}\index{\texttt{\$e}
% statement}, \texttt{\$f}\index{\texttt{\$f} statement}, \texttt{\$a}\index{\texttt{\$a}
% statement}, and \texttt{\$p}\index{\texttt{\$p} statement} statements when no
% ambiguity will result.  Specifically, Metamath parses the math symbol sequence
% in one of these statements in the following manner:  when the math symbol
% sequence has been broken up into tokens\index{token} up to a given character,
% the next token is the longest string of characters that could constitute a
% math symbol that is active\index{active
% math symbol} at that point.  (See Section~\ref{scoping} for the
% definition of an active math symbol.)  For example, if \texttt{-}, \texttt{>}, and
% \texttt{->} are the only active math symbols, the juxtaposition \texttt{>-} will be
% interpreted as the two symbols \texttt{>} and \texttt{-}, whereas \texttt{->} will
% always be interpreted as that single symbol.\footnote{For better readability we
% recommend a white space between each token.  This also makes searching for a
% symbol easier to do with an editor.  Omission of optional white space is useful
% for reducing typing when assigning an expression to a temporary
% variable\index{temporary variable} with the \texttt{let variable} Metamath
% program command.}\index{\texttt{let variable} command}
%
% Keywords\index{keyword} may be placed next to math symbols without white
% space\index{white space} between them.\footnote{Again, we do not recommend
% this for readability.}
%
% The math symbols\index{math symbol} in \texttt{\$c}\index{\texttt{\$c} statement}
% and \texttt{\$v}\index{\texttt{\$v} statement} statements must always be separated
% by white space\index{white
% space}, for the obvious reason that these statements define the names
% of the symbols.
%
% Math symbols referred to in comments (see Section~\ref{comments}) must also be
% separated by white space.  This allows you to make comments about symbols that
% are not yet active\index{active
% math symbol}.  (The ``math mode'' feature of comments is also a quick and
% easy way to obtain word processing text with embedded mathematical symbols,
% independently of the main purpose of Metamath; the way to do this is described
% in Section~\ref{comments})

\subsubsection{Label Tokens}\index{token}\index{label}

Label tokens are used to identify Metamath\index{Metamath} statements for
later reference. Label tokens may contain only letters, digits, and the three
characters period, hyphen, and underscore:
\begin{verbatim}
. - _
\end{verbatim}

A label is {\bf declared}\index{label declaration} by placing it immediately
before the keyword of the statement it identifies.  For example, the label
\texttt{axiom.1} might be declared as follows:
\begin{verbatim}
axiom.1 $a |- x = x $.
\end{verbatim}

Each \texttt{\$e}\index{\texttt{\$e} statement},
\texttt{\$f}\index{\texttt{\$f} statement},
\texttt{\$a}\index{\texttt{\$a} statement}, and
\texttt{\$p}\index{\texttt{\$p} statement} statement in a database must
have a label declared for it.  No other statement types may have label
declarations.  Every label must be unique.

A label (and the statement it identifies) is {\bf referenced}\index{label
reference} by including the label between the \texttt{\$=}\index{\texttt{\$=}
keyword} and \texttt{\$.}\index{\texttt{\$.}\ keyword}\ keywords in a \texttt{\$p}
statement.  The sequence of labels\index{label sequence} between these two
keywords is called a {\bf proof}\index{proof}.  An example of a statement with
a proof that we will encounter later (Section~\ref{proof}) is
\begin{verbatim}
wnew $p wff ( s -> ( r -> p ) )
     $= ws wr wp w2 w2 $.
\end{verbatim}

You don't have to know what this means just yet, but you should know that the
label \texttt{wnew} is declared by this \texttt{\$p} statement and that the labels
\texttt{ws}, \texttt{wr}, \texttt{wp}, and \texttt{w2} are assumed to have been declared
earlier in the database and are referenced here.

\subsection{Constants and Variables}
\index{constant}
\index{variable}

An {\bf expression}\index{expression} is any sequence of math
symbols, possibly empty.

The basic Metamath\index{Metamath} language\index{basic language} has two
kinds of math symbols\index{math symbol}:  {\bf constants}\index{constant} and
{\bf variables}\index{variable}.  In a Metamath proof, a constant may not be
substituted with any expression.  A variable can be
substituted\index{substitution!variable}\index{variable substitution} with any
expression.  This sequence may include other variables and may even include
the variable being substituted.  This substitution takes place when proofs are
verified, and it will be described in Section~\ref{proof}.  The \texttt{\$f}
statement (described later in Section~\ref{dollaref}) is used to specify the
{\bf type} of a variable (i.e.\ what kind of
variable it is)\index{variable type}\index{type} and
give it a meaning typically
associated with a ``metavariable''\index{metavariable}\footnote{A metavariable
is a variable that ranges over the syntactical elements of the object language
being discussed; for example, one metavariable might represent a variable of
the object language and another metavariable might represent a formula in the
object language.} in ordinary mathematics; for example, a variable may be
specified to be a wff or well-formed formula (in logic), a set (in set
theory), or a non-negative integer (in number theory).

%\subsection{The \texttt{\$c} and \texttt{\$v} Declaration Statements}
\subsection{The \texttt{\$c} and \texttt{\$v} Declaration Statements}
\index{\texttt{\$c} statement}
\index{constant declaration}
\index{\texttt{\$v} statement}
\index{variable declaration}

Constants are introduced or {\bf declared}\index{constant declaration}
with \texttt{\$c}\index{\texttt{\$c} statement} statements, and
variables are declared\index{variable declaration} with
\texttt{\$v}\index{\texttt{\$v} statement} statements.  A {\bf simple}
declaration\index{simple declaration} statement introduces a single
constant or variable.  Its syntax is one of the following:
\begin{center}
  \texttt{\$c} {\em math-symbol} \texttt{\$.}\\
  \texttt{\$v} {\em math-symbol} \texttt{\$.}
\end{center}
The notation {\em math-symbol} means any math symbol token\index{token}.

Some examples of simple declaration statements are:
\begin{center}
  \texttt{\$c + \$.}\\
  \texttt{\$c -> \$.}\\
  \texttt{\$c ( \$.}\\
  \texttt{\$v x \$.}\\
  \texttt{\$v y2 \$.}
\end{center}

The characters in a math symbol\index{math symbol} being declared are
irrelevant to Meta\-math; for example, we could declare a right parenthesis to
be a variable,
\begin{center}
  \texttt{\$v ) \$.}\\
\end{center}
although this would be unconventional.

A {\bf compound} declaration\index{compound declaration} statement is a
shorthand for declaring several symbols at once.  Its syntax is one of the
following:
\begin{center}
  \texttt{\$c} {\em math-symbol}\ \,$\cdots$\ {\em math-symbol} \texttt{\$.}\\
  \texttt{\$v} {\em math-symbol}\ \,$\cdots$\ {\em math-symbol} \texttt{\$.}
\end{center}\index{\texttt{\$c} statement}
Here, the ellipsis (\ldots) means any number of {\em math-symbol}\,s.

An example of a compound declaration statement is:
\begin{center}
  \texttt{\$v x y mu \$.}\\
\end{center}
This is equivalent to the three simple declaration statements
\begin{center}
  \texttt{\$v x \$.}\\
  \texttt{\$v y \$.}\\
  \texttt{\$v mu \$.}\\
\end{center}
\index{\texttt{\$v} statement}

There are certain rules on where in the database math symbols may be declared,
what sections of the database are aware of them (i.e.\ where they are
``active''), and when they may be declared more than once.  These will be
discussed in Section~\ref{scoping} and specifically on
p.~\pageref{redeclaration}.

\subsection{The \texttt{\$d} Statement}\label{dollard}
\index{\texttt{\$d} statement}

The \texttt{\$d} statement is called a {\bf disjoint-variable restriction}.  The
syntax of the {\bf simple} version of this statement is
\begin{center}
  \texttt{\$d} {\em variable variable} \texttt{\$.}
\end{center}
where each {\em variable} is a previously declared variable and the two {\em
variable}\,s are different.  (More specifically, each  {\em variable} must be
an {\bf active} variable\index{active math symbol}, which means there must be
a previous \texttt{\$v} statement whose {\bf scope}\index{scope} includes the
\texttt{\$d} statement.  These terms will be defined when we discuss scoping
statements in Section~\ref{scoping}.)

In ordinary mathematics, formulas may arise that are true if the variables in
them are distinct\index{distinct variables}, but become false when those
variables are made identical. For example, the formula in logic $\exists x\,x
\neq y$, which means ``for a given $y$, there exists an $x$ that is not equal
to $y$,'' is true in most mathematical theories (namely all non-trivial
theories\index{non-trivial theory}, i.e.\ those that describe more than one
individual, such as arithmetic).  However, if we substitute $y$ with $x$, we
obtain $\exists x\,x \neq x$, which is always false, as it means ``there
exists something that is not equal to itself.''\footnote{If you are a
logician, you will recognize this as the improper substitution\index{proper
substitution}\index{substitution!proper} of a free variable\index{free
variable} with a bound variable\index{bound variable}.  Metamath makes no
inherent distinction between free and bound variables; instead, you let
Metamath know what substitutions are permissible by using \texttt{\$d} statements
in the right way in your axiom system.}\index{free vs.\ bound variable}  The
\texttt{\$d} statement allows you to specify a restriction that forbids the
substitution of one variable with another.  In
this case, we would use the statement
\begin{center}
  \texttt{\$d x y \$.}
\end{center}\index{\texttt{\$d} statement}
to specify this restriction.

The order in which the variables appear in a \texttt{\$d} statement is not
important.  We could also use
\begin{center}
  \texttt{\$d y x \$.}
\end{center}

The \texttt{\$d} statement is actually more general than this, as the
``disjoint''\index{disjoint variables} in its name suggests.  The full meaning
is that if any substitution is made to its two variables (during the
course of a proof that references a \texttt{\$a} or \texttt{\$p} statement
associated with the \texttt{\$d}), the two expressions that result from the
substitution must have no variables in common.  In addition, each possible
pair of variables, one from each expression, must be in a \texttt{\$d} statement
associated with the statement being proved.  (This requirement forces the
statement being proved to ``inherit'' the original disjoint variable
restriction.)

For example, suppose \texttt{u} is a variable.  If the restriction
\begin{center}
  \texttt{\$d A B \$.}
\end{center}
has been specified for a theorem referenced in a
proof, we may not substitute \texttt{A} with \mbox{\tt a + u} and
\texttt{B} with \mbox{\tt b + u} because these two symbol sequences have the
variable \texttt{u} in common.  Furthermore, if \texttt{a} and \texttt{b} are
variables, we may not substitute \texttt{A} with \texttt{a} and \texttt{B} with \texttt{b}
unless we have also specified \texttt{\$d a b} for the theorem being proved; in
other words, the \texttt{\$d} property associated with a pair of variables must
be effectively preserved after substitution.

The \texttt{\$d}\index{\texttt{\$d} statement} statement does {\em not} mean ``the
two variables may not be substituted with the same thing,'' as you might think
at first.  For example, substituting each of \texttt{A} and \texttt{B} in the above
example with identical symbol sequences consisting only of constants does not
cause a disjoint variable conflict, because two symbol sequences have no
variables in common (since they have no variables, period).  Similarly, a
conflict will not occur by substituting the two variables in a \texttt{\$d}
statement with the empty symbol sequence\index{empty substitution}.

The \texttt{\$d} statement does not have a direct counterpart in
ordinary mathematics, partly because the variables\index{variable} of
Metamath are not really the same as the variables\index{variable!in
ordinary mathematics} of ordinary mathematics but rather are
metavariables\index{metavariable} ranging over them (as well as over
other kinds of symbols and groups of symbols).  Depending on the
situation, we may informally interpret the \texttt{\$d} statement in
different ways.  Suppose, for example, that \texttt{x} and \texttt{y}
are variables ranging over numbers (more precisely, that \texttt{x} and
\texttt{y} are metavariables ranging over variables that range over
numbers), and that \texttt{ph} ($\varphi$) and \texttt{ps} ($\psi$) are
variables (more precisely, metavariables) ranging over formulas.  We can
make the following interpretations that correspond to the informal
language of ordinary mathematics:
\begin{quote}
\begin{tabbing}
\texttt{\$d x y \$.} means ``assume $x$ and $y$ are
distinct variables.''\\
\texttt{\$d x ph \$.} means ``assume $x$ does not
occur in $\varphi$.''\\
\texttt{\$d ph ps \$.} \=means ``assume $\varphi$ and
$\psi$ have no variables\\ \>in common.''
\end{tabbing}
\end{quote}\index{\texttt{\$d} statement}

\subsubsection{Compound \texttt{\$d} Statements}

The {\bf compound} version of the \texttt{\$d} statement is a shorthand for
specifying several variables whose substitutions must be pairwise disjoint.
Its syntax is:
\begin{center}
  \texttt{\$d} {\em variable}\ \,$\cdots$\ {\em variable} \texttt{\$.}
\end{center}\index{\texttt{\$d} statement}
Here, {\em variable} represents the token of a previously declared
variable (specifically, an active variable) and all {\em variable}\,s are
different.  The compound \texttt{\$d}
statement is internally broken up by Metamath into one simple \texttt{\$d}
statement for each possible pair of variables in the original \texttt{\$d}
statement.  For example,
\begin{center}
  \texttt{\$d w x y z \$.}
\end{center}
is equivalent to
\begin{center}
  \texttt{\$d w x \$.}\\
  \texttt{\$d w y \$.}\\
  \texttt{\$d w z \$.}\\
  \texttt{\$d x y \$.}\\
  \texttt{\$d x z \$.}\\
  \texttt{\$d y z \$.}
\end{center}

Two or more simple \texttt{\$d} statements specifying the same variable pair are
internally combined into a single \texttt{\$d} statement.  Thus the set of three
statements
\begin{center}
  \texttt{\$d x y \$.}
  \texttt{\$d x y \$.}
  \texttt{\$d y x \$.}
\end{center}
is equivalent to
\begin{center}
  \texttt{\$d x y \$.}
\end{center}

Similarly, compound \texttt{\$d} statements, after being internally broken up,
internally have their common variable pairs combined.  For example the
set of statements
\begin{center}
  \texttt{\$d x y A \$.}
  \texttt{\$d x y B \$.}
\end{center}
is equivalent to
\begin{center}
  \texttt{\$d x y \$.}
  \texttt{\$d x A \$.}
  \texttt{\$d y A \$.}
  \texttt{\$d x y \$.}
  \texttt{\$d x B \$.}
  \texttt{\$d y B \$.}
\end{center}
which is equivalent to
\begin{center}
  \texttt{\$d x y \$.}
  \texttt{\$d x A \$.}
  \texttt{\$d y A \$.}
  \texttt{\$d x B \$.}
  \texttt{\$d y B \$.}
\end{center}

Metamath\index{Metamath} automatically verifies that all \texttt{\$d}
restrictions are met whenever it verifies proofs.  \texttt{\$d} statements are
never referenced directly in proofs (this is why they do not have
labels\index{label}), but Metamath is always aware of which ones must be
satisfied (i.e.\ are active) and will notify you with an error message if any
violation occurs.

To illustrate how Metamath detects a missing \texttt{\$d}
statement, we will look at the following example from the
\texttt{set.mm} database.

\begin{verbatim}
$d x z $.  $d y z $.
$( Theorem to add distinct quantifier to atomic formula. $)
ax17eq $p |- ( x = y -> A. z x = y ) $=...
\end{verbatim}

This statement has the obvious requirement that $z$ must be
distinct\index{distinct variables} from $x$ in theorem \texttt{ax17eq} that
states $x=y \rightarrow \forall z \, x=y$ (well, obvious if you're a logician,
for otherwise we could conclude  $x=y \rightarrow \forall x \, x=y$, which is
false when the free variables $x$ and $y$ are equal.).

Let's look at what happens if we edit the database to comment out this
requirement.

\begin{verbatim}
$( $d x z $. $) $d y z $.
$( Theorem to add distinct quantifier to atomic formula. $)
ax17eq $p |- ( x = y -> A. z x = y ) $=...
\end{verbatim}

When it tries to verify the proof, Metamath will tell you that \texttt{x} and
\texttt{z} must be disjoint, because one of its steps references an axiom or
theorem that has this requirement.

\begin{verbatim}
MM> verify proof ax17eq
ax17eq ?Error at statement 1918, label "ax17eq", type "$p":
      vz wal wi vx vy vz ax-13 vx vy weq vz vx ax-c16 vx vy
                                               ^^^^^
There is a disjoint variable ($d) violation at proof step 29.
Assertion "ax-c16" requires that variables "x" and "y" be
disjoint.  But "x" was substituted with "z" and "y" was
substituted with "x".  The assertion being proved, "ax17eq",
does not require that variables "z" and "x" be disjoint.
\end{verbatim}

We can see the substitutions into \texttt{ax-c16} with the following command.

\begin{verbatim}
MM> show proof ax17eq / detailed_step 29
Proof step 29:  pm2.61dd.2=ax-c16 $a |- ( A. z z = x -> ( x =
  y -> A. z x = y ) )
This step assigns source "ax-c16" ($a) to target "pm2.61dd.2"
($e).  The source assertion requires the hypotheses "wph"
($f, step 26), "vx" ($f, step 27), and "vy" ($f, step 28).
The parent assertion of the target hypothesis is "pm2.61dd"
($p, step 36).
The source assertion before substitution was:
    ax-c16 $a |- ( A. x x = y -> ( ph -> A. x ph ) )
The following substitutions were made to the source
assertion:
    Variable  Substituted with
     x         z
     y         x
     ph        x = y
The target hypothesis before substitution was:
    pm2.61dd.2 $e |- ( ph -> ch )
The following substitutions were made to the target
hypothesis:
    Variable  Substituted with
     ph        A. z z = x
     ch        ( x = y -> A. z x = y )
\end{verbatim}

The disjoint variable restrictions of \texttt{ax-c16} can be seen from the
\texttt{show state\-ment} command.  The line that begins ``\texttt{Its mandatory
dis\-joint var\-i\-able pairs are:}\ldots'' lists any \texttt{\$d} variable
pairs in brackets.

\begin{verbatim}
MM> show statement ax-c16/full
Statement 3033 is located on line 9338 of the file "set.mm".
"Axiom of Distinct Variables. ..."
  ax-c16 $a |- ( A. x x = y -> ( ph -> A. x ph ) ) $.
Its mandatory hypotheses in RPN order are:
  wph $f wff ph $.
  vx $f set x $.
  vy $f set y $.
Its mandatory disjoint variable pairs are:  <x,y>
The statement and its hypotheses require the variables:  x y
      ph
The variables it contains are:  x y ph
\end{verbatim}

Since Metamath will always detect when \texttt{\$d}\index{\texttt{\$d} statement}
statements are needed for a proof, you don't have to worry too much about
forgetting to put one in; it can always be added if you see the error message
above.  If you put in unnecessary \texttt{\$d} statements, the worst that could
happen is that your theorem might not be as general as it could be, and this
may limit its use later on.

On the other hand, when you introduce axioms (\texttt{\$a}\index{\texttt{\$a}
statement} statements), you must be very careful to properly specify the
necessary associated \texttt{\$d} statements since Metamath has no way of knowing
whether your axioms are correct.  For example, Metamath would have no idea
that \texttt{ax-c16}, which we are telling it is an axiom of logic, would lead to
contradictions if we omitted its associated \texttt{\$d} statement.

% This was previously a comment in footnote-sized type, but it can be
% hard to read this much text in a small size.
% As a result, it's been changed to normally-sized text.
\label{nodd}
You may wonder if it is possible to develop standard
mathematics in the Metamath language without the \texttt{\$d}\index{\texttt{\$d}
statement} statement, since it seems like a nuisance that complicates proof
verification. The \texttt{\$d} statement is not needed in certain subsets of
mathematics such as propositional calculus.  However, dummy
variables\index{dummy variable!eliminating} and their associated \texttt{\$d}
statements are impossible to avoid in proofs in standard first-order logic as
well as in the variant used in \texttt{set.mm}.  In fact, there is no upper bound to
the number of dummy variables that might be needed in a proof of a theorem of
first-order logic containing 3 or more variables, as shown by H.\
Andr\'{e}ka\index{Andr{\'{e}}ka, H.} \cite{Nemeti}.  A first-order system that
avoids them entirely is given in \cite{Megill}\index{Megill, Norman}; the
trick there is simply to embed harmlessly the necessary dummy variables into a
theorem being proved so that they aren't ``dummy'' anymore, then interpret the
resulting longer theorem so as to ignore the embedded dummy variables.  If
this interests you, the system in \texttt{set.mm} obtained from \texttt{ax-1}
through \texttt{ax-c14} in \texttt{set.mm}, and deleting \texttt{ax-c16} and \texttt{ax-5},
requires no \texttt{\$d} statements but is logically complete in the sense
described in \cite{Megill}.  This means it can prove any theorem of
first-order logic as long as we add to the theorem an antecedent that embeds
dummy and any other variables that must be distinct.  In a similar fashion,
axioms for set theory can be devised that
do not require distinct variable
provisos\index{Set theory without distinct variable provisos},
as explained at
\url{http://us.metamath.org/mpeuni/mmzfcnd.html}.
Together, these in principle allow all of
mathematics to be developed under Metamath without a \texttt{\$d} statement,
although the length of the resulting theorems will grow as more and
more dummy variables become required in their proofs.

\subsection{The \texttt{\$f}
and \texttt{\$e} Statements}\label{dollaref}
\index{\texttt{\$e} statement}
\index{\texttt{\$f} statement}
\index{floating hypothesis}
\index{essential hypothesis}
\index{variable-type hypothesis}
\index{logical hypothesis}
\index{hypothesis}

Metamath has two kinds of hypo\-theses, the \texttt{\$f}\index{\texttt{\$f}
statement} or {\bf variable-type} hypothesis and the \texttt{\$e} or {\bf logical}
hypo\-the\-sis.\index{\texttt{\$d} statement}\footnote{Strictly speaking, the
\texttt{\$d} statement is also a hypothesis, but it is never directly referenced
in a proof, so we call it a restriction rather than a hypothesis to lessen
confusion.  The checking for violations of \texttt{\$d} restrictions is automatic
and built into Metamath's proof-checking algorithm.} The letters \texttt{f} and
\texttt{e} stand for ``floating''\index{floating hypothesis} (roughly meaning
used only if relevant) and ``essential''\index{essential hypothesis} (meaning
always used) respectively, for reasons that will become apparent
when we discuss frames in
Section~\ref{frames} and scoping in Section~\ref{scoping}. The syntax of these
are as follows:
\begin{center}
  {\em label} \texttt{\$f} {\em typecode} {\em variable} \texttt{\$.}\\
  {\em label} \texttt{\$e} {\em typecode}
      {\em math-symbol}\ \,$\cdots$\ {\em math-symbol} \texttt{\$.}\\
\end{center}
\index{\texttt{\$e} statement}
\index{\texttt{\$f} statement}
A hypothesis must have a {\em label}\index{label}.  The expression in a
\texttt{\$e} hypothesis consists of a typecode (an active constant math symbol)
followed by a sequence
of zero or more math symbols. Each math symbol (including {\em constant}
and {\em variable}) must be a previously declared constant or variable.  (In
addition, each math symbol must be active, which will be covered when we
discuss scoping statements in Section~\ref{scoping}.)  You use a \texttt{\$f}
hypothesis to specify the
nature or {\bf type}\index{variable type}\index{type} of a variable (such as ``let $x$ be an
integer'') and use a \texttt{\$e} hypothesis to express a logical truth (such as
``assume $x$ is prime'') that must be established in order for an assertion
requiring it to also be true.

A variable must have its type specified in a \texttt{\$f} statement before
it may be used in a \texttt{\$e}, \texttt{\$a}, or \texttt{\$p}
statement.  There may be only one (active) \texttt{\$f} statement for a
given variable.  (``Active'' is defined in Section~\ref{scoping}.)

In ordinary mathematics, theorems\index{theorem} are often expressed in the
form ``Assume $P$; then $Q$,'' where $Q$ is a statement that you can derive
if you start with statement $P$.\index{free variable}\footnote{A stronger
version of a theorem like this would be the {\em single} formula $P\rightarrow
Q$ ($P$ implies $Q$) from which the weaker version above follows by the rule
of modus ponens in logic.  We are not discussing this stronger form here.  In
the weaker form, we are saying only that if we can {\em prove} $P$, then we can
{\em prove} $Q$.  In a logician's language, if $x$ is the only free variable
in $P$ and $Q$, the stronger form is equivalent to $\forall x ( P \rightarrow
Q)$ (for all $x$, $P$ implies $Q$), whereas the weaker form is equivalent to
$\forall x P \rightarrow \forall x Q$. The stronger form implies the weaker,
but not vice-versa.  To be precise, the weaker form of the theorem is more
properly called an ``inference'' rather than a theorem.}\index{inference}
In the
Metamath\index{Metamath} language, you would express mathematical statement
$P$ as a hypothesis (a \texttt{\$e} Metamath language statement in this case) and
statement $Q$ as a provable assertion (a \texttt{\$p}\index{\texttt{\$p} statement}
statement).

Some examples of hypotheses you might encounter in logic and set theory are
\begin{center}
  \texttt{stmt1 \$f wff P \$.}\\
  \texttt{stmt2 \$f set x \$.}\\
  \texttt{stmt3 \$e |- ( P -> Q ) \$.}
\end{center}
\index{\texttt{\$e} statement}
\index{\texttt{\$f} statement}
Informally, these would be read, ``Let $P$ be a well-formed-formula,'' ``Let
$x$ be an (individual) variable,'' and ``Assume we have proved $P \rightarrow
Q$.''  The turnstile symbol \,$\vdash$\index{turnstile ({$\,\vdash$})} is
commonly used in logic texts to mean ``a proof exists for.''

To summarize:
\begin{itemize}
\item A \texttt{\$f} hypothesis tells Metamath the type or kind of its variable.
It is analogous to a variable declaration in a computer language that
tells the compiler that a variable is an integer or a floating-point
number.
\item The \texttt{\$e} hypothesis corresponds to what you would usually call a
``hypothesis'' in ordinary mathematics.
\end{itemize}

Before an assertion\index{assertion} (\texttt{\$a} or \texttt{\$p} statement) can be
referenced in a proof, all of its associated \texttt{\$f} and \texttt{\$e} hypotheses
(i.e.\ those \texttt{\$e} hypotheses that are active) must be satisfied (i.e.
established by the proof).  The meaning of ``associated'' (which we will call
{\bf mandatory} in Section~\ref{frames}) will become clear when we discuss
scoping later.

Note that after any \texttt{\$f}, \texttt{\$e},
\texttt{\$a}, or \texttt{\$p} token there is a required
\textit{typecode}\index{typecode}.
The typecode is a constant used to enforce types of expressions.
This will become clearer once we learn more about
assertions (\texttt{\$a} and \texttt{\$p} statements).
An example may also clarify their purpose.
In the
\texttt{set.mm}\index{set theory database (\texttt{set.mm})}%
\index{Metamath Proof Explorer}
database,
the following typecodes are used:

\begin{itemize}
\item \texttt{wff} :
  Well-formed formula (wff) symbol
  (read: ``the following symbol sequence is a wff'').
% The *textual* typecode for turnstile is "|-", but when read it's a little
% confusing, so I intentionally display the mathematical symbol here instead
% (I think it's clearer in this context).
\item \texttt{$\vdash$} :
  Turnstile (read: ``the following symbol sequence is provable'' or
  ``a proof exists for'').
\item \texttt{setvar} :
  Individual set variable type (read: ``the following is an
  individual set variable'').
  Note that this is \textit{not} the type of an arbitrary set expression,
  instead, it is used to ensure that there is only a single symbol used
  after quantifiers like for-all ($\forall$) and there-exists ($\exists$).
\item \texttt{class} :
  An expression that is a syntactically valid class expression.
  All valid set expressions are also valid class expression, so expressions
  of sets normally have the \texttt{class} typecode.
  Use the \texttt{class} typecode,
  \textit{not} the \texttt{setvar} typecode,
  for the type of set expressions unless you are specifically identifying
  a single set variable.
\end{itemize}

\subsection{Assertions (\texttt{\$a} and \texttt{\$p} Statements)}
\index{\texttt{\$a} statement}
\index{\texttt{\$p} statement}\index{assertion}\index{axiomatic assertion}
\index{provable assertion}

There are two types of assertions, \texttt{\$a}\index{\texttt{\$a} statement}
statements ({\bf axiomatic assertions}) and \texttt{\$p} statements ({\bf
provable assertions}).  Their syntax is as follows:
\begin{center}
  {\em label} \texttt{\$a} {\em typecode} {\em math-symbol} \ldots
         {\em math-symbol} \texttt{\$.}\\
  {\em label} \texttt{\$p} {\em typecode} {\em math-symbol} \ldots
        {\em math-symbol} \texttt{\$=} {\em proof} \texttt{\$.}
\end{center}
\index{\texttt{\$a} statement}
\index{\texttt{\$p} statement}
\index{\texttt{\$=} keyword}
An assertion always requires a {\em label}\index{label}. The expression in an
assertion consists of a typecode (an active constant)
followed by a sequence of zero
or more math symbols.  Each math symbol, including any {\em constant}, must be a
previously declared constant or variable.  (In addition, each math symbol
must be active, which will be covered when we discuss scoping statements in
Section~\ref{scoping}.)

A \texttt{\$a} statement is usually a definition of syntax (for example, if $P$
and $Q$ are wffs then so is $(P\to Q)$), an axiom\index{axiom} of ordinary
mathematics (for example, $x=x$), or a definition\index{definition} of
ordinary mathematics (for example, $x\ne y$ means $\lnot x=y$). A \texttt{\$p}
statement is a claim that a certain combination of math symbols follows from
previous assertions and is accompanied by a proof that demonstrates it.

Assertions can also be referenced in (later) proofs in order to derive new
assertions from them. The label of an assertion is used to refer to it in a
proof. Section~\ref{proof} will describe the proof in detail.

Assertions also provide the primary means for communicating the mathematical
results in the database to people.  Proofs (when conveniently displayed)
communicate to people how the results were arrived at.

\subsubsection{The \texttt{\$a} Statement}
\index{\texttt{\$a} statement}

Axiomatic assertions (\texttt{\$a} statements) represent the starting points from
which other assertions (\texttt{\$p}\index{\texttt{\$p} statement} statements) are
derived.  Their most obvious use is for specifying ordinary mathematical
axioms\index{axiom}, but they are also used for two other purposes.

First, Metamath\index{Metamath} needs to know the syntax of symbol
sequences that constitute valid mathematical statements.  A Metamath
proof must be broken down into much more detail than ordinary
mathematical proofs that you may be used to thinking of (even the
``complete'' proofs of formal logic\index{formal logic}).  This is one
of the things that makes Metamath a general-purpose language,
independent of any system of logic or even syntax.  If you want to use a
substitution instance of an assertion as a step in a proof, you must
first prove that the substitution is syntactically correct (or if you
prefer, you must ``construct'' it), showing for example that the
expression you are substituting for a wff metavariable is a valid wff.
The \texttt{\$a}\index{\texttt{\$a} statement} statement is used to
specify those combinations of symbols that are considered syntactically
valid, such as the legal forms of wffs.

Second, \texttt{\$a} statements are used to specify what are ordinarily thought of
as definitions, i.e.\ new combinations of symbols that abbreviate other
combinations of symbols.  Metamath makes no distinction\index{axiom vs.\
definition} between axioms\index{axiom} and definitions\index{definition}.
Indeed, it has been argued that such distinction should not be made even in
ordinary mathematics; see Section~\ref{definitions}, which discusses the
philosophy of definitions.  Section~\ref{hierarchy} discusses some
technical requirements for definitions.  In \texttt{set.mm} we adopt the
convention of prefixing axiom labels with \texttt{ax-} and definition labels with
\texttt{df-}\index{label}.

The results that can be derived with the Metamath language are only as good as
the \texttt{\$a}\index{\texttt{\$a} statement} statements used as their starting
point.  We cannot stress this too strongly.  For example, Metamath will
not prevent you from specifying $x\neq x$ as an axiom of logic.  It is
essential that you scrutinize all \texttt{\$a} statements with great care.
Because they are a source of potential pitfalls, it is best not to add new
ones (usually new definitions) casually; rather you should carefully evaluate
each one's necessity and advantages.

Once you have in place all of the basic axioms\index{axiom} and
rules\index{rule} of a mathematical theory, the only \texttt{\$a} statements that
you will be adding will be what are ordinarily called definitions.  In
principle, definitions should be in some sense eliminable from the language of
a theory according to some convention (usually involving logical equivalence
or equality).  The most common convention is that any formula that was
syntactically valid but not provable before the definition was introduced will
not become provable after the definition is introduced.  In an ideal world,
definitions should not be present at all if one is to have absolute confidence
in a mathematical result.  However, they are necessary to make
mathematics practical, for otherwise the resulting formulas would be
extremely long and incomprehensible.  Since the nature of definitions (in the
most general sense) does not permit them to automatically be verified as
``proper,''\index{proper definition}\index{definition!proper} the judgment of
the mathematician is required to ensure it.  (In \texttt{set.mm} effort was made
to make almost all definitions directly eliminable and thus minimize the need
for such judgment.)

If you are not a mathematician, it may be best not to add or change any
\texttt{\$a}\index{\texttt{\$a} statement} statements but instead use
the mathematical language already provided in standard databases.  This
way Metamath will not allow you to make a mistake (i.e.\ prove a false
result).


\subsection{Frames}\label{frames}

We now introduce the concept of a collection of related Metamath statements
called a frame.  Every assertion (\texttt{\$a} or \texttt{\$p} statement) in the database has
an associated frame.

A {\bf frame}\index{frame} is a sequence of \texttt{\$d}, \texttt{\$f},
and \texttt{\$e} statements (zero or more of each) followed by one
\texttt{\$a} or \texttt{\$p} statement, subject to certain conditions we
will describe.  For simplicity we will assume that all math symbol
tokens used are declared at the beginning of the database with
\texttt{\$c} and \texttt{\$v} statements (which are not properly part of
a frame).  Also for simplicity we will assume there are only simple
\texttt{\$d} statements (those with only two variables) and imagine any
compound \texttt{\$d} statements (those with more than two variables) as
broken up into simple ones.

A frame groups together those hypotheses (and \texttt{\$d} statements) relevant
to an assertion (\texttt{\$a} or \texttt{\$p} statement).  The statements in a frame
may or may not be physically adjacent in a database; we will cover
this in our discussion of scoping statements
in Section~\ref{scoping}.

A frame has the following properties:
\begin{enumerate}
 \item The set of variables contained in its \texttt{\$f} statements must
be identical to the set of variables contained in its \texttt{\$e},
\texttt{\$a}, and/or \texttt{\$p} statements.  In other words, each
variable in a \texttt{\$e}, \texttt{\$a}, or \texttt{\$p} statement must
have an associated ``variable type'' defined for it in a \texttt{\$f}
statement.
  \item No two \texttt{\$f} statements may contain the same variable.
  \item Any \texttt{\$f} statement
must occur before a \texttt{\$e} statement in which its variable occurs.
\end{enumerate}

The first property determines the set of variables occurring in a frame.
These are the {\bf mandatory
variables}\index{mandatory variable} of the frame.  The second property
tells us there must be only one type specified for a variable.
The last property is not a theoretical requirement but it
makes parsing of the database easier.

For our examples, we assume our database has the following declarations:

\begin{verbatim}
$v P Q R $.
$c -> ( ) |- wff $.
\end{verbatim}

The following sequence of statements, describing the modus ponens inference
rule, is an example of a frame:

\begin{verbatim}
wp  $f wff P $.
wq  $f wff Q $.
maj $e |- ( P -> Q ) $.
min $e |- P $.
mp  $a |- Q $.
\end{verbatim}

The following sequence of statements is not a frame because \texttt{R} does not
occur in the \texttt{\$e}'s or the \texttt{\$a}:

\begin{verbatim}
wp  $f wff P $.
wq  $f wff Q $.
wr  $f wff R $.
maj $e |- ( P -> Q ) $.
min $e |- P $.
mp  $a |- Q $.
\end{verbatim}

The following sequence of statements is not a frame because \texttt{Q} does not
occur in a \texttt{\$f}:

\begin{verbatim}
wp  $f wff P $.
maj $e |- ( P -> Q ) $.
min $e |- P $.
mp  $a |- Q $.
\end{verbatim}

The following sequence of statements is not a frame because the \texttt{\$a} statement is
not the last one:

\begin{verbatim}
wp  $f wff P $.
wq  $f wff Q $.
maj $e |- ( P -> Q ) $.
mp  $a |- Q $.
min $e |- P $.
\end{verbatim}

Associated with a frame is a sequence of {\bf mandatory
hypotheses}\index{mandatory hypothesis}.  This is simply the set of all
\texttt{\$f} and \texttt{\$e} statements in the frame, in the order they
appear.  A frame can be referenced in a later proof using the label of
the \texttt{\$a} or \texttt{\$p} assertion statement, and the proof
makes an assignment to each mandatory hypothesis in the order in which
it appears.  This means the order of the hypotheses, once chosen, must
not be changed so as not to affect later proofs referencing the frame's
assertion statement.  (The Metamath proof verifier will, of course, flag
an error if a proof becomes incorrect by doing this.)  Since proofs make
use of ``Reverse Polish notation,'' described in Section~\ref{proof}, we
call this order the {\bf RPN order}\index{RPN order} of the hypotheses.

Note that \texttt{\$d} statements are not part of the set of mandatory
hypotheses, and their order doesn't matter (as long as they satisfy the
fourth property for a frame described above).  The \texttt{\$d}
statements specify restrictions on variables that must be satisfied (and
are checked by the proof verifier) when expressions are substituted for
them in a proof, and the \texttt{\$d} statements themselves are never
referenced directly in a proof.

A frame with a \texttt{\$p} (provable) statement requires a proof as part of the
\texttt{\$p} statement.  Sometimes in a proof we want to make use of temporary or
dummy variables\index{dummy variable} that do not occur in the \texttt{\$p}
statement or its mandatory hypotheses.  To accommodate this we define an {\bf
extended frame}\index{extended frame} as a frame together with zero or more
\texttt{\$d} and \texttt{\$f} statements that reference variables not among the
mandatory variables of the frame.  Any new variables referenced are called the
{\bf optional variables}\index{optional variable} of the extended frame. If a
\texttt{\$f} statement references an optional variable it is called an {\bf
optional hypothesis}\index{optional hypothesis}, and if one or both of the
variables in a \texttt{\$d} statement are optional variables it is called an {\bf
optional disjoint-variable restriction}\index{optional disjoint-variable
restriction}.  Properties 2, 3, and 4 for a frame also apply to an extended
frame.

The concept of optional variables is not meaningful for frames with \texttt{\$a}
statements, since those statements have no proofs that might make use of them.
There is no restriction on including optional hypotheses in the extended frame
for a \texttt{\$a} statement, but they serve no purpose.

The following set of statements is an example of an extended frame, which
contains an optional variable \texttt{R} and an optional hypothesis \texttt{wr}.  In
this example, we suppose the rule of modus ponens is not an axiom but is
derived as a theorem from earlier statements (we omit its presumed proof).
Variable \texttt{R} may be used in its proof if desired (although this would
probably have no advantage in propositional calculus).  Note that the sequence
of mandatory hypotheses in RPN order is still \texttt{wp}, \texttt{wq}, \texttt{maj},
\texttt{min} (i.e.\ \texttt{wr} is omitted), and this sequence is still assumed
whenever the assertion \texttt{mp} is referenced in a subsequent proof.

\begin{verbatim}
wp  $f wff P $.
wq  $f wff Q $.
wr  $f wff R $.
maj $e |- ( P -> Q ) $.
min $e |- P $.
mp  $p |- Q $= ... $.
\end{verbatim}

Every frame is an extended frame, but not every extended frame is a frame, as
this example shows.  The underlying frame for an extended frame is
obtained by simply removing all statements containing optional variables.
Any proof referencing an assertion will ignore any extensions to its
frame, which means we may add or delete optional hypotheses at will without
affecting subsequent proofs.

The conceptually simplest way of organizing a Metamath database is as a
sequence of extended frames.  The scoping statements
\texttt{\$\char`\{}\index{\texttt{\$\char`\{} and \texttt{\$\char`\}}
keywords} and \texttt{\$\char`\}} can be used to delimit the start and
end of an extended frame, leading to the following possible structure for a
database.  \label{framelist}

\vskip 2ex
\setbox\startprefix=\hbox{\tt \ \ \ \ \ \ \ \ }
\setbox\contprefix=\hbox{}
\startm
\m{\mbox{(\texttt{\$v} {\em and} \texttt{\$c}\,{\em statements})}}
\endm
\startm
\m{\mbox{\texttt{\$\char`\{}}}
\endm
\startm
\m{\mbox{\texttt{\ \ } {\em extended frame}}}
\endm
\startm
\m{\mbox{\texttt{\$\char`\}}}}
\endm
\startm
\m{\mbox{\texttt{\$\char`\{}}}
\endm
\startm
\m{\mbox{\texttt{\ \ } {\em extended frame}}}
\endm
\startm
\m{\mbox{\texttt{\$\char`\}}}}
\endm
\startm
\m{\mbox{\texttt{\ \ \ \ \ \ \ \ \ }}\vdots}
\endm
\vskip 2ex

In practice, this structure is inconvenient because we have to repeat
any \texttt{\$f}, \texttt{\$e}, and \texttt{\$d} statements over and
over again rather than stating them once for use by several assertions.
The scoping statements, which we will discuss next, allow this to be
done.  In principle, any Metamath database can be converted to the above
format, and the above format is the most convenient to use when studying
a Metamath database as a formal system%
%% Uncomment this when uncommenting section {formalspec} below
   (Appendix \ref{formalspec})%
.
In fact, Metamath internally converts the database to the above format.
The command \texttt{show statement} in the Metamath program will show
you the contents of the frame for any \texttt{\$a} or \texttt{\$p}
statement, as well as its extension in the case of a \texttt{\$p}
statement.

%c%(provided that all ``local'' variables and constants with limited scope have
%c%unique names),

During our discussion of scoping statements, it may be helpful to
think in terms of the equivalent sequence of frames that will result when
the database is parsed.  Scoping (other than the limited
use above to delimit frames) is not a theoretical requirement for
Metamath but makes it more convenient.


\subsection{Scoping Statements (\texttt{\$\{} and \texttt{\$\}})}\label{scoping}
\index{\texttt{\$\char`\{} and \texttt{\$\char`\}} keywords}\index{scoping statement}

%c%Some Metamath statements may be needed only temporarily to
%c%serve a specific purpose, and after we're done with them we would like to
%c%disregard or ignore them.  For example, when we're finished using a variable,
%c%we might want to
%c%we might want to free up the token\index{token} used to name it so that the
%c%token can be used for other purposes later on, such as a different kind of
%c%variable or even a constant.  In the terminology of computer programming, we
%c%might want to let some symbol declarations be ``local'' rather than ``global.''
%c%\index{local symbol}\index{global symbol}

The {\bf scoping} statements, \texttt{\$\char`\{} ({\bf start of block}) and \texttt{\$\char`\}}
({\bf end of block})\index{block}, provide a means for controlling the portion
of a database over which certain statement types are recognized.  The
syntax of a scoping statement is very simple; it just consists of the
statement's keyword:
\begin{center}
\texttt{\$\char`\{}\\
\texttt{\$\char`\}}
\end{center}
\index{\texttt{\$\char`\{} and \texttt{\$\char`\}} keywords}

For example, consider the following database where we have stripped out
all tokens except the scoping statement keywords.  For the purpose of the
discussion, we have added subscripts to the scoping statements; these subscripts
do not appear in the actual database.
\[
 \mbox{\tt \ \$\char`\{}_1
 \mbox{\tt \ \$\char`\{}_2
 \mbox{\tt \ \$\char`\}}_2
 \mbox{\tt \ \$\char`\{}_3
 \mbox{\tt \ \$\char`\{}_4
 \mbox{\tt \ \$\char`\}}_4
 \mbox{\tt \ \$\char`\}}_3
 \mbox{\tt \ \$\char`\}}_1
\]
Each \texttt{\$\char`\{} statement in this example is said to be {\bf
matched} with the \texttt{\$\char`\}} statement that has the same
subscript.  Each pair of matched scoping statements defines a region of
the database called a {\bf block}.\index{block} Blocks can be {\bf
nested}\index{nested block} inside other blocks; in the example, the
block defined by $\mbox{\tt \$\char`\{}_4$ and $\mbox{\tt \$\char`\}}_4$
is nested inside the block defined by $\mbox{\tt \$\char`\{}_3$ and
$\mbox{\tt \$\char`\}}_3$ as well as inside the block defined by
$\mbox{\tt \$\char`\{}_1$ and $\mbox{\tt \$\char`\}}_1$.  In general, a
block may be empty, it may contain only non-scoping
statements,\footnote{Those statements other than \texttt{\$\char`\{} and
\texttt{\$\char`\}}.}\index{non-scoping statement} or it may contain any
mixture of other blocks and non-scoping statements.  (This is called a
``recursive'' definition\index{recursive definition} of a block.)

Associated with each block is a number called its {\bf nesting
level}\index{nesting level} that indicates how deeply the block is nested.
The nesting levels of the blocks in our example are as follows:
\[
  \underbrace{
    \mbox{\tt \ }
    \underbrace{
     \mbox{\tt \$\char`\{\ }
     \underbrace{
       \mbox{\tt \$\char`\{\ }
       \mbox{\tt \$\char`\}}
     }_{2}
     \mbox{\tt \ }
     \underbrace{
       \mbox{\tt \$\char`\{\ }
       \underbrace{
         \mbox{\tt \$\char`\{\ }
         \mbox{\tt \$\char`\}}
       }_{3}
       \mbox{\tt \ \$\char`\}}
     }_{2}
     \mbox{\tt \ \$\char`\}}
   }_{1}
   \mbox{\tt \ }
 }_{0}
\]
\index{\texttt{\$\char`\{} and \texttt{\$\char`\}} keywords}
The entire database is considered to be one big block (the {\bf outermost}
block) with a nesting level of 0.  The outermost block is {\em not} bracketed
by scoping statements.\footnote{The language was designed this way so that
several source files can be joined together more easily.}\index{outermost
block}

All non-scoping Metamath statements become recognized or {\bf
active}\index{active statement} at the place where they appear.\footnote{To
keep things slightly simpler, we do not bother to define the concept of
``active'' for the scoping statements.}  Certain of these statement types
become inactive at the end of the block in which they appear; these statement
types are:
\begin{center}
  \texttt{\$c}, \texttt{\$v}, \texttt{\$d}, \texttt{\$e}, and \texttt{\$f}.
%  \texttt{\$v}, \texttt{\$f}, \texttt{\$e}, and \texttt{\$d}.
\end{center}
\index{\texttt{\$c} statement}
\index{\texttt{\$d} statement}
\index{\texttt{\$e} statement}
\index{\texttt{\$f} statement}
\index{\texttt{\$v} statement}
The other statement types remain active forever (i.e.\ through the end of the
database); they are:
\begin{center}
  \texttt{\$a} and \texttt{\$p}.
%  \texttt{\$c}, \texttt{\$a}, and \texttt{\$p}.
\end{center}
\index{\texttt{\$a} statement}
\index{\texttt{\$p} statement}
Any statement (of these 7 types) located in the outermost
block\index{outermost block} will remain active through the end of the
database and thus are effectively ``global'' statements.\index{global
statement}

All \texttt{\$c} statements must be placed in the outermost block.  Since they are
therefore always global, they could be considered as belonging to both of the
above categories.

The {\bf scope}\index{scope} of a statement is the set of statements that
recognize it as active.

%c%The concept of ``active'' is also defined for math symbols\index{math
%c%symbol}.  Math symbols (constants\index{constant} and
%c%variables\index{variable}) become {\bf active}\index{active
%c%math symbol} in the \texttt{\$c}\index{\texttt{\$c}
%c%statement} and \texttt{\$v}\index{\texttt{\$v} statement} statements that
%c%declare them.  They become inactive when their declaration statements become
%c%inactive.

The concept of ``active'' is also defined for math symbols\index{math
symbol}.  Math symbols (constants\index{constant} and
variables\index{variable}) become {\bf active}\index{active math symbol}
in the \texttt{\$c}\index{\texttt{\$c} statement} and
\texttt{\$v}\index{\texttt{\$v} statement} statements that declare them.
A variable becomes inactive when its declaration statement becomes
inactive.  Because all \texttt{\$c} statements must be in the outermost
block, a constant will never become inactive after it is declared.

\subsubsection{Redeclaration of Math Symbols}
\index{redeclaration of symbols}\label{redeclaration}

%c%A math symbol may not be declared a second time while it is active, but it may
%c%be declared again after it becomes inactive.

A variable may not be declared a second time while it is active, but it may be
declared again after it becomes inactive.  This provides a convenient way to
introduce ``local'' variables,\index{local variable} i.e.\ temporary variables
for use in the frame of an assertion or in a proof without keeping them around
forever.  A previously declared variable may not be redeclared as a constant.

A constant may not be redeclared.  And, as mentioned above, constants must be
declared in the outermost block.

The reason variables may have limited scope but not constants is that an
assertion (\texttt{\$a} or \texttt{\$p} statement) remains available for use in
proofs through the end of the database.  Variables in an assertion's frame may
be substituted with whatever is needed in a proof step that references the
assertion, whereas constants remain fixed and may not be substituted with
anything.  The particular token used for a variable in an assertion's frame is
irrelevant when the assertion is referenced in a proof, and it doesn't matter
if that token is not available outside of the referenced assertion's frame.
Constants, however, must be globally fixed.

In the present version of the Metamath language, there is no theoretical
benefit for the feature allowing variables to be active for limited scopes
rather than global. It is just a convenience that allows them, for example, to
be locally grouped together with their corresponding \texttt{\$f} variable-type
declarations.\footnote{A future version\label{redeclarationf} of the Metamath
language may extend the language to include true local variables as well as
local constants.  I have not decided upon the best way to do this or even if
it should be done at all; the main issue is whether the benefits offset
greater confusion in learning the language.  In any case, the rules for
the present language were chosen to let existing databases be compatible with
any future language extensions.}

%c%If you declare a math symbol more than once, internally Metamath considers it a
%c%new distinct symbol, even though it has the same name.  If you are unaware of
%c%this, you may find that what you think are correct proofs are incorrectly
%c%rejected as invalid, because Metamath may tell you that a constant you
%c%previously declared does not match a newly declared math symbol with the same
%c%name.  For details on this subtle point, see the Comment on
%c%p.~\pageref{spec4comment}.  This is done purposely to allow temporary
%c%constants to be introduced while developing a subtheory, then allow their math
%c%symbol tokens to be reused later on; in general they will not refer to the
%c%same thing.  In practice, you would not ordinarily reuse the names of
%c%constants because it would tend to be confusing to the reader.  The reuse of
%c%names of variables, on the other hand, is something that is often useful to do
%c%(for example it is done frequently in \texttt{set.mm}).  Since variables in an
%c%assertion referenced in a proof can be substituted as needed to achieve a
%c%symbol match, this is not an issue.

% (This section covers a somewhat advanced topic you may want to skip
% at first reading.)
%
% Under certain circumstances, math symbol\index{math symbol}
% tokens\index{token} may be redeclared (i.e.\ the token
% may appear in more than
% one \texttt{\$c}\index{\texttt{\$c} statement} or \texttt{\$v}\index{\texttt{\$v}
% statement} statement).  You might want to do this say, to make temporary use
% of a variable name without having to worry about its affect elsewhere,
% somewhat analogous to declaring a local variable in a standard computer
% language.  Understanding what goes on when math symbol tokens are redeclared
% is a little tricky to understand at first, since it requires that we
% distinguish the token itself from the math symbol that it names.  It will help
% if we first take a peek at the internal workings of the
% Metamath\index{Metamath} program.
%
% Metamath reserves a memory location for each occurrence of a
% token\index{token} in a declaration statement (\texttt{\$c}\index{\texttt{\$c}
% statement} or \texttt{\$v}\index{\texttt{\$v} statement}).  If a given token appears
% in more than one declaration statement, it will refer to more than one memory
% locations.  A math symbol\index{math symbol} may be thought of as being one of
% these memory locations rather than as the token itself.  Only one of the
% memory locations associated with a given token may be active at any one time.
% The math symbol (memory location) that gets looked up when the token appears
% in a non-declaration statement is the one that happens to be active at that
% time.
%
% We now look at the rules for the redeclaration\index{redeclaration of symbols}
% of math symbol tokens.
% \begin{itemize}
% \item A math symbol token may not be declared twice in the
% same block.\footnote{While there is no theoretical reason for disallowing
% this, it was decided in the design of Metamath that allowing it would offer no
% advantage and might cause confusion.}
% \item An inactive math symbol may always be
% redeclared.
% \item  An active math symbol may be redeclared in a different (i.e.\
% inner) block\index{block} from the one it became active in.
% \end{itemize}
%
% When a math symbol token is redeclared, it conceptually refers to a different
% math symbol, just as it would be if it were called a different name.  In
% addition, the original math symbol that it referred to, if it was active,
% temporarily becomes inactive.  At the end of the block in which the
% redeclaration occurred, the new math symbol\index{math symbol} becomes
% inactive and the original symbol becomes active again.  This concept is
% illustrated in the following example, where the symbol \texttt{e} is
% ordinarily a constant (say Euler's constant, 2.71828...) but
% temporarily we want to use it as a ``local'' variable, say as a coefficient
% in the equation $a x^4 + b x^3 + c x^2 + d x + e$:
% \[
%   \mbox{\tt \$\char`\{\ \$c e \$.}
%   \underbrace{
%     \ \ldots\ %
%     \mbox{\tt \$\char`\{}\ \ldots\ %
%   }_{\mbox{\rm region A}}
%   \mbox{\tt \$v e \$.}
%   \underbrace{
%     \mbox{\ \ \ \ldots\ \ \ }
%   }_{\mbox{\rm region B}}
%   \mbox{\tt \$\char`\}}
%   \underbrace{
%     \mbox{\ \ \ \ldots\ \ \ }
%   }_{\mbox{\rm region C}}
%   \mbox{\tt \$\char`\}}
% \]
% \index{\texttt{\$\char`\{} and \texttt{\$\char`\}} keywords}
% In region A, the token \texttt{e} refers to a constant.  It is redeclared as a
% variable in region B, and any reference to it in this region will refer to this
% variable.  In region C, the redeclaration becomes inactive, and the original
% declaration becomes active again.  In region C, the token \texttt{x} refers to the
% original constant.
%
% As a practical matter, overuse of math symbol\index{math symbol}
% redeclarations\index{redeclaration of symbols} can be confusing (even though
% it is well-defined) and is best avoided when possible.  Here are some good
% general guidelines you can follow.  Usually, you should declare all
% constants\index{constant} in the outermost block\index{outermost block},
% especially if they are general-purpose (such as the token \verb$A.$, meaning
% $\forall$ or ``for all'').  This will make them ``globally'' active (although
% as in the example above local redeclarations will temporarily make them
% inactive.)  Most or all variables\index{variable}, on the other hand, could be
% declared in inner blocks, so that the token for them can be used later for a
% different type of variable or a constant.  (The names of the variables you
% choose are not used when you refer to an assertion\index{assertion} in a
% proof, whereas constants must match exactly.  A locally declared constant will
% not match a globally declared constant in a proof, even if they use the same
% token, because Metamath internally considers them to be different math
% symbols.)  To avoid confusion, you should generally avoid redeclaring active
% variables.  If you must redeclare them, do so at the beginning of a block.
% The temporary declaration of constants in inner blocks might be occasionally
% appropriate when you make use of a temporary definition to prove lemmas
% leading to a main result that does not make direct use of the definition.
% This way, you will not clutter up your database with a large number of
% seldom-used global constant symbols.  You might want to note that while
% inactive constants may not appear directly in an assertion (a \texttt{\$a}\index{\texttt{\$a}
% statement} or \texttt{\$p}\index{\texttt{\$p} statement}
% statement), they may be indirectly used in the proof of a \texttt{\$p} statement
% so long as they do not appear in the final math symbol sequence constructed by
% the proof.  In the end, you will have to use your best judgment, taking into
% account standard mathematical usage of the symbols as well as consideration
% for the reader of your work.
%
% \subsubsection{Reuse of Labels}\index{reuse of labels}\index{label}
%
% The \texttt{\$e}\index{\texttt{\$e} statement}, \texttt{\$f}\index{\texttt{\$f}
% statement}, \texttt{\$a}\index{\texttt{\$a} statement}, and
% \texttt{\$p}\index{\texttt{\$p}
% statement} statement types require labels, which allow them to be
% referenced later inside proofs.  A label is considered {\bf
% active}\index{active label} when the statement it is associated with is
% active.  The token\index{token} for a label may be reused
% (redeclared)\index{redeclaration of labels} provided that it is not being used
% for a currently active label.  (Unlike the tokens for math symbols, active
% label tokens may not be redeclared in an inner scope.)  Note that the labels
% of \texttt{\$a} and \texttt{\$p} statements can never be reused after these
% statements appear, because these statements remain active through the end of
% the database.
%
% You might find the reuse of labels a convenient way to have standard names for
% temporary hypotheses, such as \texttt{h1}, \texttt{h2}, etc.  This way you don't have
% to invent unique names for each of them, and in some cases it may be less
% confusing to the reader (although in other cases it might be more confusing, if
% the hypothesis is located far away from the assertion that uses
% it).\footnote{The current implementation requires that all labels, even
% inactive ones, be unique.}

\subsubsection{Frames Revisited}\index{frames and scoping statements}

Now that we have covered scoping, we will look at how an arbitrary
Metamath database can be converted to the simple sequence of extended
frames described on p.~\pageref{framelist}.  This is also how Metamath
stores the database internally when it reads in the database
source.\label{frameconvert} The method is simple.  First, we collect all
constant and variable (\texttt{\$c} and \texttt{\$v}) declarations in
the database, ignoring duplicate declarations of the same variable in
different scopes.  We then put our collected \texttt{\$c} and
\texttt{\$v} declarations at the beginning of the database, so that
their scope is the entire database.  Next, for each assertion in the
database, we determine its frame and extended frame.  The extended frame
is simply the \texttt{\$f}, \texttt{\$e}, and \texttt{\$d} statements
that are active.  The frame is the extended frame with all optional
hypotheses removed.

An equivalent way of saying this is that the extended frame of an assertion
is the collection of all \texttt{\$f}, \texttt{\$e}, and \texttt{\$d} statements
whose scope includes the assertion.
The \texttt{\$f} and \texttt{\$e} statements
occur in the order they appear
(order is irrelevant for \texttt{\$d} statements).

%c%, renaming any
%c%redeclared variables as needed so that all of them have unique names.  (The
%c%exact renaming convention is unimportant.  You might imagine renaming
%c%different declarations of math symbol \texttt{a} as \texttt{a\$1}, \texttt{a\$2}, etc.\
%c%which would prevent any conflicts since \texttt{\$} is not a legal character in a
%c%math symbol token.)

\section{The Anatomy of a Proof} \label{proof}
\index{proof!Metamath, description of}

Each provable assertion (\texttt{\$p}\index{\texttt{\$p} statement} statement) in a
database must include a {\bf proof}\index{proof}.  The proof is located
between the \texttt{\$=}\index{\texttt{\$=} keyword} and \texttt{\$.}\ keywords in the
\texttt{\$p} statement.

In the basic Metamath language\index{basic language}, a proof is a
sequence of statement labels.  This label sequence\index{label sequence}
serves as a set of instructions that the Metamath program uses to
construct a series of math symbol sequences.  The construction must
ultimately result in the math symbol sequence contained between the
\texttt{\$p}\index{\texttt{\$p} statement} and
\texttt{\$=}\index{\texttt{\$=} keyword} keywords of the \texttt{\$p}
statement.  Otherwise, the Metamath program will consider the proof
incorrect, and it will notify you with an appropriate error message when
you ask it to verify the proof.\footnote{To make the loading faster, the
Metamath program does not automatically verify proofs when you
\texttt{read} in a database unless you use the \texttt{/verify}
qualifier.  After a database has been read in, you may use the
\texttt{verify proof *} command to verify proofs.}\index{\texttt{verify
proof} command} Each label in a proof is said to {\bf
reference}\index{label reference} its corresponding statement.

Associated with any assertion\index{assertion} (\texttt{\$p} or
\texttt{\$a}\index{\texttt{\$a} statement} statement) is a set of
hypotheses (\texttt{\$f}\index{\texttt{\$f} statement} or
\texttt{\$e}\index{\texttt{\$e} statement} statements) that are active
with respect to that assertion.  Some are mandatory and the others are
optional.  You should review these concepts if necessary.

Each label\index{label} in a proof must be either the label of a
previous assertion (\texttt{\$a}\index{\texttt{\$a} statement} or
\texttt{\$p}\index{\texttt{\$p} statement} statement) or the label of an
active hypothesis (\texttt{\$e} or \texttt{\$f}\index{\texttt{\$f}
statement} statement) of the \texttt{\$p} statement containing the
proof.  Hypothesis labels may reference both the
mandatory\index{mandatory hypothesis} and the optional hypotheses of the
\texttt{\$p} statement.

The label sequence in a proof specifies a construction in {\bf reverse Polish
notation}\index{reverse Polish notation (RPN)} (RPN).  You may be familiar
with RPN if you have used older
Hewlett--Packard or similar hand-held calculators.
In the calculator analogy, a hypothesis label\index{hypothesis label} is like
a number and an assertion label\index{assertion label} is like an operation
(more precisely, an $n$-ary operation when the
assertion has $n$ \texttt{\$e}-hypotheses).
On an RPN calculator, an operation takes one or more previous numbers in an
input sequence, performs a calculation on them, and replaces those numbers and
itself with the result of the calculation.  For example, the input sequence
$2,3,+$ on an RPN calculator results in $5$, and the input sequence
$2,3,5,{\times},+$ results in $2,15,+$ which results in $17$.

Understanding how RPN is processed involves the concept of a {\bf
stack}\index{stack}\index{RPN stack}, which can be thought of as a set of
temporary memory locations that hold intermediate results.  When Metamath
encounters a hypothesis label it places or {\bf pushes}\index{push} the math
symbol sequence of the hypothesis onto the stack.  When Metamath encounters an
assertion label, it associates the most recent stack entries with the {\em
mandatory} hypotheses\index{mandatory hypothesis} of the assertion, in the
order where the most recent stack entry is associated with the last mandatory
hypothesis of the assertion.  It then determines what
substitutions\index{substitution!variable}\index{variable substitution} have
to be made into the variables of the assertion's mandatory hypotheses to make
them identical to the associated stack entries.  It then makes those same
substitutions into the assertion itself.  Finally, Metamath removes or {\bf
pops}\index{pop} the matched hypotheses from the stack and pushes the
substituted assertion onto the stack.

For the purpose of matching the mandatory hypothesis to the most recent stack
entries, whether a hypothesis is a \texttt{\$e} or \texttt{\$f} statement is
irrelevant.  The only important thing is that a set of
substitutions\footnote{In the Metamath spec (Section~\ref{spec}), we use the
singular term ``substitution'' to refer to the set of substitutions we talk
about here.} exist that allow a match (and if they don't, the proof verifier
will let you know with an error message).  The Metamath language is specified
in such a way that if a set of substitutions exists, it will be unique.
Specifically, the requirement that each variable have a type specified for it
with a \texttt{\$f} statement ensures the uniqueness.

We will illustrate this with an example.
Consider the following Metamath source file:
\begin{verbatim}
$c ( ) -> wff $.
$v p q r s $.
wp $f wff p $.
wq $f wff q $.
wr $f wff r $.
ws $f wff s $.
w2 $a wff ( p -> q ) $.
wnew $p wff ( s -> ( r -> p ) ) $= ws wr wp w2 w2 $.
\end{verbatim}
This Metamath source example shows the definition and ``proof'' (i.e.,
construction) of a well-formed formula (wff)\index{well-formed formula (wff)}
in propositional calculus.  (You may wish to type this example into a file to
experiment with the Metamath program.)  The first two statements declare
(introduce the names of) four constants and four variables.  The next four
statements specify the variable types, namely that
each variable is assumed to be a wff.  Statement \texttt{w2} defines (postulates)
a way to produce a new wff, \texttt{( p -> q )}, from two given wffs \texttt{p} and
\texttt{q}. The mandatory hypotheses of \texttt{w2} are \texttt{wp} and \texttt{wq}.
Statement \texttt{wnew} claims that \texttt{( s -> ( r -> p ) )} is a wff given
three wffs \texttt{s}, \texttt{r}, and \texttt{p}.  More precisely, \texttt{wnew} claims
that the sequence of ten symbols \texttt{wff ( s -> ( r -> p ) )} is provable from
previous assertions and the hypotheses of \texttt{wnew}.  Metamath does not know
or care what a wff is, and as far as it is concerned
the typecode \texttt{wff} is just an
arbitrary constant symbol in a math symbol sequence.  The mandatory hypotheses
of \texttt{wnew} are \texttt{wp}, \texttt{wr}, and \texttt{ws}; \texttt{wq} is an optional
hypothesis.  In our particular proof, the optional hypothesis is not
referenced, but in general, any combination of active (i.e.\ optional and
mandatory) hypotheses could be referenced.  The proof of statement \texttt{wnew}
is the sequence of five labels starting with \texttt{ws} (step~1) and ending with
\texttt{w2} (step~5).

When Metamath verifies the proof, it scans the proof from left to right.  We
will examine what happens at each step of the proof.  The stack starts off
empty.  At step 1, Metamath looks up label \texttt{ws} and determines that it is a
hypothesis, so it pushes the symbol sequence of statement \texttt{ws} onto the
stack:

\begin{center}\begin{tabular}{|l|l|}\hline
{Stack location} & {Contents} \\ \hline \hline
1 & \texttt{wff s} \\ \hline
\end{tabular}\end{center}

Metamath sees that the labels \texttt{wr} and \texttt{wp} in steps~2 and 3 are also
hypotheses, so it pushes them onto the stack.  After step~3, the stack looks
like
this:

\begin{center}\begin{tabular}{|l|l|}\hline
{Stack location} & {Contents} \\ \hline \hline
3 & \texttt{wff p} \\ \hline
2 & \texttt{wff r} \\ \hline
1 & \texttt{wff s} \\ \hline
\end{tabular}\end{center}

At step 4, Metamath sees that label \texttt{w2} is an assertion, so it must do
some processing.  First, it associates the mandatory hypotheses of \texttt{w2},
which are \texttt{wp} and \texttt{wq}, with stack locations~2 and 3, {\em in that
order}. Metamath determines that the only possible way
to make hypothesis \texttt{wp} match (become identical to) stack location~2 and
\texttt{wq} match stack location 3 is to substitute variable \texttt{p} with \texttt{r}
and \texttt{q} with \texttt{p}.  Metamath makes these substitutions into \texttt{w2} and
obtains the symbol sequence \texttt{wff ( r -> p )}.  It removes the hypotheses
from stack locations~2 and 3, then places the result into stack location~2:

\begin{center}\begin{tabular}{|l|l|}\hline
{Stack location} & {Contents} \\ \hline \hline
2 & \texttt{wff ( r -> p )} \\ \hline
1 & \texttt{wff s} \\ \hline
\end{tabular}\end{center}

At step 5, Metamath sees that label \texttt{w2} is an assertion, so it must again
do some processing.  First, it matches the mandatory hypotheses of \texttt{w2},
which are \texttt{wp} and \texttt{wq}, to stack locations 1 and 2.
Metamath determines that the only possible way to make the
hypotheses match is to substitute variable \texttt{p} with \texttt{s} and \texttt{q} with
\texttt{( r -> p )}.  Metamath makes these substitutions into \texttt{w2} and obtains
the symbol
sequence \texttt{wff ( s -> ( r -> p ) )}.  It removes stack
locations 1 and 2, then places the result into stack location~1:

\begin{center}\begin{tabular}{|l|l|}\hline
{Stack location} & {Contents} \\ \hline \hline
1 & \texttt{wff ( s -> ( r -> p ) )} \\ \hline
\end{tabular}\end{center}

After Metamath finishes processing the proof, it checks to see that the
stack contains exactly one element and that this element is
the same as the math symbol sequence in the
\texttt{\$p}\index{\texttt{\$p} statement} statement.  This is the case for our
proof of \texttt{wnew},
so we have proved \texttt{wnew} successfully.  If the result
differs, Metamath will notify you with an error message.  An error message
will also result if the stack contains more than one entry at the end of the
proof, or if the stack did not contain enough entries at any point in the
proof to match all of the mandatory hypotheses\index{mandatory hypothesis} of
an assertion.  Finally, Metamath will notify you with an error message if no
substitution is possible that will make a referenced assertion's hypothesis
match the
stack entries.  You may want to experiment with the different kinds of errors
that Metamath will detect by making some small changes in the proof of our
example.

Metamath's proof notation was designed primarily to express proofs in a
relatively compact manner, not for readability by humans.  Metamath can display
proofs in a number of different ways with the \texttt{show proof}\index{\texttt{show
proof} command} command.  The
\texttt{/lemmon} qualifier displays it in a format that is easier to read when the
proofs are short, and you saw examples of its use in Chapter~\ref{using}.  For
longer proofs, it is useful to see the tree structure of the proof.  A tree
structure is displayed when the \texttt{/lemmon} qualifier is omitted.  You will
probably find this display more convenient as you get used to it. The tree
display of the proof in our example looks like
this:\label{treeproof}\index{tree-style proof}\index{proof!tree-style}
\begin{verbatim}
1     wp=ws    $f wff s
2        wp=wr    $f wff r
3        wq=wp    $f wff p
4     wq=w2    $a wff ( r -> p )
5  wnew=w2  $a wff ( s -> ( r -> p ) )
\end{verbatim}
The number to the left of each line is the step number.  Following it is a
{\bf hypothesis association}\index{hypothesis association}, consisting of two
labels\index{label} separated by \texttt{=}.  To the left of the \texttt{=} (except
in the last step) is the label of a hypothesis of an assertion referenced
later in the proof; here, steps 1 and 4 are the hypothesis associations for
the assertion \texttt{w2} that is referenced in step 5.  A hypothesis association
is indented one level more than the assertion that uses it, so it is easy to
find the corresponding assertion by moving directly down until the indentation
level decreases to one less than where you started from.  To the right of each
\texttt{=} is the proof step label for that proof step.  The statement keyword of
the proof step label is listed next, followed by the content of the top of the
stack (the most recent stack entry) as it exists after that proof step is
processed.  With a little practice, you should have no trouble reading proofs
displayed in this format.

Metamath proofs include the syntax construction of a formula.
In standard mathematics, this kind of
construction is not considered a proper part of the proof at all, and it
certainly becomes rather boring after a while.
Therefore,
by default the \texttt{show proof}\index{\texttt{show proof}
command} command does not show the syntax construction.
Historically \texttt{show proof} command
\textit{did} show the syntax construction, and you needed to add the
\texttt{/essential} option to hide, them, but today
\texttt{/essential} is the default and you need to use
\texttt{/all} to see the syntax constructions.

When verifying a proof, Metamath will check that no mandatory
\texttt{\$d}\index{\texttt{\$d} statement}\index{mandatory \texttt{\$d}
statement} statement of an assertion referenced in a proof is violated
when substitutions\index{substitution!variable}\index{variable
substitution} are made to the variables in the assertion.  For details
see Section~\ref{spec4} or \ref{dollard}.

\subsection{The Concept of Unification} \label{unify}

During the course of verifying a proof, when Metamath\index{Metamath}
encounters an assertion label\index{assertion label}, it associates the
mandatory hypotheses\index{mandatory hypothesis} of the assertion with the top
entries of the RPN stack\index{stack}\index{RPN stack}.  Metamath then
determines what substitutions\index{substitution!variable}\index{variable
substitution} it must make to the variables in the assertion's mandatory
hypotheses in order for these hypotheses to become identical to their
corresponding stack entries.  This process is called {\bf
unification}\index{unification}.  (We also informally use the term
``unification'' to refer to a set of substitutions that results from the
process, as in ``two unifications are possible.'')  After the substitutions
are made, the hypotheses are said to be {\bf unified}.

If no such substitutions are possible, Metamath will consider the proof
incorrect and notify you with an error message.
% (deleted 3/10/07, per suggestion of Mel O'Cat:)
% The syntax of the
% Metamath language ensures that if a set of substitutions exists, it
% will be unique.

The general algorithm for unification described in the literature is
somewhat complex.
However, in the case of Metamath it is intentionally trivial.
Mandatory hypotheses must be
pushed on the proof stack in the order in which they appear.
In addition, each variable must have its type specified
with a \texttt{\$f} hypothesis before it is used
and that each \texttt{\$f} hypothesis
have the restricted syntax of a typecode (a constant) followed by a variable.
The typecode in the \texttt{\$f} hypothesis must match the first symbol of
the corresponding RPN stack entry (which will also be a constant), so
the only possible match for the variable in the \texttt{\$f} hypothesis is
the sequence of symbols in the stack entry after the initial constant.

In the Proof Assistant\index{Proof Assistant}, a more general unification
algorithm is used.  While a proof is being developed, sometimes not enough
information is available to determine a unique unification.  In this case
Metamath will ask you to pick the correct one.\index{ambiguous
unification}\index{unification!ambiguous}

\section{Extensions to the Metamath Language}\index{extended
language}

\subsection{Comments in the Metamath Language}\label{comments}
\index{markup notation}
\index{comments!markup notation}

The commenting feature allows you to annotate the contents of
a database.  Just as with most
computer languages, comments are ignored for the purpose of interpreting the
contents of the database. Comments effectively act as
additional white space\index{white
space} between tokens
when a database is parsed.

A comment may be placed at the beginning, end, or
between any two tokens\index{token} in a source file.

Comments have the following syntax:
\begin{center}
 \texttt{\$(} {\em text} \texttt{\$)}
\end{center}
Here,\index{\texttt{\$(} and \texttt{\$)} auxiliary
keywords}\index{comment} {\em text} is a string, possibly empty, of any
characters in Metamath's character set (p.~\pageref{spec1chars}), except
that the character strings \texttt{\$(} and \texttt{\$)} may not appear
in {\em text}.  Thus nested comments are not
permitted:\footnote{Computer languages have differing standards for
nested comments, and rather than picking one it was felt simplest not to
allow them at all, at least in the current version of
Metamath\index{Metamath!limitations of version 0.07.30}.} Metamath will
complain if you give it
\begin{center}
 \texttt{\$( This is a \$( nested \$) comment.\ \$)}
\end{center}
To compensate for this non-nesting behavior, I often change all \texttt{\$}'s
to \texttt{@}'s in sections of Metamath code I wish to comment out.

The Metamath program supports a number of markup mechanisms and conventions
to generate good-looking results in \LaTeX\ and {\sc html},
as discussed below.
These markup features have to do only with how the comments are typeset,
and have no effect on how Metamath verifies the proofs in the database.
The improper
use of them may result in incorrectly typeset output, but no Metamath
error messages will result during the \texttt{read} and \texttt{verify
proof} commands.  (However, the \texttt{write
theorem\texttt{\char`\_}list} command
will check for markup errors as a side-effect of its
{\sc html} generation.)
Section~\ref{texout} has instructions for creating \LaTeX\ output, and
section~\ref{htmlout} has instructions for creating
{\sc html}\index{HTML} output.

\subsubsection{Headings}\label{commentheadings}

If the \texttt{\$(} is immediately followed by a new line
starting with a heading marker, it is a header.
This can start with:

\begin{itemize}
 \item[] \texttt{\#\#\#\#} - major part header
 \item[] \texttt{\#*\#*} - section header
 \item[] \texttt{=-=-} - subsection header
 \item[] \texttt{-.-.} - subsubsection header
\end{itemize}

The line following the marker line
will be used for the table of contents entry, after trimming spaces.
The next line should be another (closing) matching marker line.
Any text after that
but before the closing \texttt{\$}, such as an extended description of the
section, will be included on the \texttt{mmtheoremsNNN.html} page.

For more information, run
\texttt{help write theorem\char`\_list}.

\subsubsection{Math mode}
\label{mathcomments}
\index{\texttt{`} inside comments}
\index{\texttt{\char`\~} inside comments}
\index{math mode}

Inside of comments, a string of tokens\index{token} enclosed in
grave accents\index{grave accent (\texttt{`})} (\texttt{`}) will be converted
to standard mathematical symbols during
{\sc HTML}\index{HTML} or \LaTeX\ output
typesetting,\index{latex@{\LaTeX}} according to the information in the
special \texttt{\$t}\index{\texttt{\$t} comment}\index{typesetting
comment} comment in the database
(see section~\ref{tcomment} for information about the typesetting
comment, and Appendix~\ref{ASCII} to see examples of its results).

The first grave accent\index{grave accent (\texttt{`})} \texttt{`}
causes the output processor to enter {\bf math mode}\index{math mode}
and the second one exits it.
In this
mode, the characters following the \texttt{`} are interpreted as a
sequence of math symbol tokens separated by white space\index{white
space}.  The tokens are looked up in the \texttt{\$t}
comment\index{\texttt{\$t} comment}\index{typesetting comment} and if
found, they will be replaced by the standard mathematical symbols that
they correspond to before being placed in the typeset output file.  If
not found, the symbol will be output as is and a warning will be issued.
The tokens do not have to be active in the database, although a warning
will be issued if they are not declared with \texttt{\$c} or
\texttt{\$v} statements.

Two consecutive
grave accents \texttt{``} are treated as a single actual grave accent
(both inside and outside of math mode) and will not cause the output
processor to enter or exit math mode.

Here is an example of its use\index{Pierce's axiom}:
\begin{center}
\texttt{\$( Pierce's axiom, ` ( ( ph -> ps ) -> ph ) -> ph ` ,\\
         is not very intuitive. \$)}
\end{center}
becomes
\begin{center}
   \texttt{\$(} Pierce's axiom, $((\varphi \rightarrow \psi)\rightarrow
\varphi)\rightarrow \varphi$, is not very intuitive. \texttt{\$)}
\end{center}

Note that the math symbol tokens\index{token} must be surrounded by white
space\index{white space}.
%, since there is no context that allows ambiguity to be
%resolved, as is the case with math symbol sequences in some of the Metamath
%statements.
White space should also surround the \texttt{`}
delimiters.

The math mode feature also gives you a quick and easy way to generate
text containing mathematical symbols, independently of the intended
purpose of Metamath.\index{Metamath!using as a math editor} To do this,
simply create your text with grave accents surrounding your formulas,
after making sure that your math symbols are mapped to \LaTeX\ symbols
as described in Appendix~\ref{ASCII}.  It is easier if you start with a
database with predefined symbols such as \texttt{set.mm}.  Use your
grave-quoted math string to replace an existing comment, then typeset
the statement corresponding to that comment following the instructions
from the \texttt{help tex} command in the Metamath program.  You will
then probably want to edit the resulting file with a text editor to fine
tune it to your exact needs.

\subsubsection{Label Mode}\index{label mode}

Outside of math mode, a tilde\index{tilde (\texttt{\char`\~})} \verb/~/
indicates to Metamath's\index{Metamath} output processor that the
token\index{token} that follows (i.e.\ the characters up to the next
white space\index{white space}) represents a statement label or URL.
This formatting mode is called {\bf label mode}\index{label mode}.
If a literal tilde
is desired (outside of math mode) instead of label mode,
use two tildes in a row to represent it.

When generating a \LaTeX\ output file,
the following token will be formatted in \texttt{typewriter}
font, and the tilde removed, to make it stand out from the rest of the text.
This formatting will be applied to all characters after the
tilde up to the first white space\index{white space}.
Whether
or not the token is an actual statement label is not checked, and the
token does not have to have the correct syntax for a label; no error
messages will be produced.  The only effect of the label mode on the
output is that typewriter font will be used for the tokens that are
placed in the \LaTeX\ output file.

When generating {\sc html},
the tokens after the tilde {\em must} be a URL (either http: or https:)
or a valid label.
Error messages will be issued during that output if they aren't.
A hyperlink will be generated to that URL or label.

\subsubsection{Link to bibliographical reference}\index{citation}%
\index{link to bibliographical reference}

Bibliographical references are handled specially when generating
{\sc html} if formatted specially.
Text in the form \texttt{[}{\em author}\texttt{]}
is considered a link to a bibliographical reference.
See \texttt{help html} and \texttt{help write
bibliography} in the Metamath program for more
information.
% \index{\texttt{\char`\[}\ldots\texttt{]} inside comments}
See also Sections~\ref{tcomment} and \ref{wrbib}.

The \texttt{[}{\em author}\texttt{]} notation will also create an entry in
the bibliography cross-reference file generated by \texttt{write
bibliography} (Section~\ref{wrbib}) for {\sc HTML}.
For this to work properly, the
surrounding comment must be formatted as follows:
\begin{quote}
    {\em keyword} {\em label} {\em noise-word}
     \texttt{[}{\em author}\texttt{] p.} {\em number}
\end{quote}
for example
\begin{verbatim}
     Theorem 5.2 of [Monk] p. 223
\end{verbatim}
The {\em keyword} is not case sensitive and must be one of the following:
\begin{verbatim}
     theorem lemma definition compare proposition corollary
     axiom rule remark exercise problem notation example
     property figure postulate equation scheme chapter
\end{verbatim}
The optional {\em label} may consist of more than one
(non-{\em keyword} and non-{\em noise-word}) word.
The optional {\em noise-word} is one of:
\begin{verbatim}
     of in from on
\end{verbatim}
and is  ignored when the cross-reference file is created.  The
\texttt{write
biblio\-graphy} command will perform error checking to verify the
above format.\index{error checking}

\subsubsection{Parentheticals}\label{parentheticals}

The end of a comment may include one or more parenthicals, that is,
statements enclosed in parentheses.
The Metamath program looks for certain parentheticals and can issue
warnings based on them.
They are:

\begin{itemize}
 \item[] \texttt{(Contributed by }
   \textit{NAME}\texttt{,} \textit{DATE}\texttt{.)} -
   document the original contributor's name and the date it was created.
 \item[] \texttt{(Revised by }
   \textit{NAME}\texttt{,} \textit{DATE}\texttt{.)} -
   document the contributor's name and creation date
   that resulted in significant revision
   (not just an automated minimization or shortening).
 \item[] \texttt{(Proof shortened by }
   \textit{NAME}\texttt{,} \textit{DATE}\texttt{.)} -
   document the contributor's name and date that developed a significant
   shortening of the proof (not just an automated minimization).
 \item[] \texttt{(Proof modification is discouraged.)} -
   Note that this proof should normally not be modified.
 \item[] \texttt{(New usage is discouraged.)} -
   Note that this assertion should normally not be used.
\end{itemize}

The \textit{DATE} must be in form YYYY-MMM-DD, where MMM is the
English abbreviation of that month.

\subsubsection{Other markup}\label{othermarkup}
\index{markup notation}

There are other markup notations for generating good-looking results
beyond math mode and label mode:

\begin{itemize}
 \item[]
         \texttt{\char`\_} (underscore)\index{\texttt{\char`\_} inside comments} -
             Italicize text starting from
              {\em space}\texttt{\char`\_}{\em non-space} (i.e.\ \texttt{\char`\_}
              with a space before it and a non-space character after it) until
             the next
             {\em non-space}\texttt{\char`\_}{\em space}.  Normal
             punctuation (e.g.\ a trailing
             comma or period) is ignored when determining {\em space}.
 \item[]
         \texttt{\char`\_} (underscore) - {\em
         non-space}\texttt{\char`\_}{\em non-space-string}, where
          {\em non-space-string} is a string of non-space characters,
         will make {\em non-space-string} become a subscript.
 \item[]
         \texttt{<HTML>}...\texttt{</HTML>} - do not convert
         ``\texttt{<}'' and ``\texttt{>}''
         in the enclosed text when generating {\sc HTML},
         otherwise process markup normally. This allows direct insertion
         of {\sc html} commands.
 \item[]
       ``\texttt{\&}ref\texttt{;}'' - insert an {\sc HTML}
         character reference.
         This is how to insert arbitrary Unicode characters
         (such as accented characters).  Currently only directly supported
         when generating {\sc HTML}.
\end{itemize}

It is recommended that spaces surround any \texttt{\char`\~} and
\texttt{`} tokens in the comment and that a space follow the {\em label}
after a \texttt{\char`\~} token.  This will make global substitutions
to change labels and symbol names much easier and also eliminate any
future chance of ambiguity.  Spaces around these tokens are automatically
removed in the final output to conform with normal rules of punctuation;
for example, a space between a trailing \texttt{`} and a left parenthesis
will be removed.

A good way to become familiar with the markup notation is to look at
the extensive examples in the \texttt{set.mm} database.

\subsection{The Typesetting Comment (\texttt{\$t})}\label{tcomment}

The typesetting comment \texttt{\$t} in the input database file
provides the information necessary to produce good-looking results.
It provides \LaTeX\ and {\sc html}
definitions for math symbols,
as well supporting as some
customization of the generated web page.
If you add a new token to a database, you should also
update the \texttt{\$t} comment information if you want to eventually
create output in \LaTeX\ or {\sc HTML}.
See the
\texttt{set.mm}\index{set theory database (\texttt{set.mm})} database
file for an extensive example of a \texttt{\$t} comment illustrating
many of the features described below.

Programs that do not need to generate good-looking presentation results,
such as programs that only verify Metamath databases,
can completely ignore typesetting comments
and just treat them as normal comments.
Even the Metamath program only consults the
\texttt{\$t} comment information when it needs to generate typeset output
in \LaTeX\ or {\sc HTML}
(e.g., when you open a \LaTeX\ output file with the \texttt{open tex} command).

We will first discuss the syntax of typesetting comments, and then
briefly discuss how this can be used within the Metamath program.

\subsubsection{Typesetting Comment Syntax Overview}

The typesetting comment is identified by the token
\texttt{\$t}\index{\texttt{\$t} comment}\index{typesetting comment} in
the comment, and the typesetting comment ends at the matching
\texttt{\$)}:
\[
  \mbox{\tt \$(\ }
  \mbox{\tt \$t\ }
  \underbrace{
    \mbox{\tt \ \ \ \ \ \ \ \ \ \ \ }
    \cdots
    \mbox{\tt \ \ \ \ \ \ \ \ \ \ \ }
  }_{\mbox{Typesetting definitions go here}}
  \mbox{\tt \ \$)}
\]

There must be one or more white space characters, and only white space
characters, between the \texttt{\$(} that starts the comment
and the \texttt{\$t} symbol,
and the \texttt{\$t} must be followed by one
or more white space characters
(see section \ref{whitespace} for the definition of white space characters).
The typesetting comment continues until the comment end token \texttt{\$)}
(which must be preceded by one or more white space characters).

In version 0.07.30\index{Metamath!limitations of version 0.07.30} of the
Metamath program, there may be only one \texttt{\$t} comment in a
database.  This restriction may be lifted in the future to allow
many \texttt{\$t} comments in a database.

Between the \texttt{\$t} symbol (and its following white space) and the
comment end token \texttt{\$)} (and its preceding white space)
is a sequence of one or more typesetting definitions, where
each definition has the form
\textit{definition-type arg arg ... ;}.
Each of the zero or more \textit{arg} values
can be either a typesetting data or a keyword
(what keywords are allowed, and where, depends on the specific
\textit{definition-type}).
The \textit{definition-type}, and each argument \textit{arg},
are separated by one or more white space characters.
Every definition ends in an unquoted semicolon;
white space is not required before the terminating semicolon of a definition.
Each definition should start on a new line.\footnote{This
restriction of the current version of Metamath
(0.07.30)\index{Metamath!limitations of version 0.07.30} may be removed
in a future version, but you should do it anyway for readability.}

For example, this typesetting definition:
\begin{center}
 \verb$latexdef "C_" as "\subseteq";$
\end{center}
defines the token \verb$C_$ as the \LaTeX\ symbol $\subseteq$ (which means
``subset'').

Typesetting data is a sequence of one or more quoted strings
(if there is more than one, they are connected by \texttt{\char`\+}).
Often a single quoted string is used to provide data for a definition, using
either double (\texttt{\char`\"}) or single (\texttt{'}) quotation marks.
However,
{\em a quoted string (enclosed in quotation marks) may not include
line breaks.}
A quoted string
may include a quotation mark that matches the enclosing quotes by repeating
the quotation mark twice.  Here are some examples:

\begin{tabu}   { l l }
\textbf{Example} & \textbf{Meaning} \\
\texttt{\char`\"a\char`\"\char`\"b\char`\"} & \texttt{a\char`\"b} \\
\texttt{'c''d'} & \texttt{c'd} \\
\texttt{\char`\"e''f\char`\"} & \texttt{e''f} \\
\texttt{'g\char`\"\char`\"h'} & \texttt{g\char`\"\char`\"h} \\
\end{tabu}

Finally, a long quoted string
may be broken up into multiple quoted strings (considered, as a whole,
a single quoted string) and joined with \texttt{\char`\+}.
You can even use multiple lines as long as a
'+' is at the end of every line except the last one.
The \texttt{\char`\+} should be preceded and followed by at least one
white space character.
Thus, for example,
\begin{center}
 \texttt{\char`\"ab\char`\"\ \char`\+\ \char`\"cd\char`\"
    \ \char`\+\ \\ 'ef'}
\end{center}
is the same as
\begin{center}
 \texttt{\char`\"abcdef\char`\"}
\end{center}

{\sc c}-style comments \texttt{/*}\ldots\texttt{*/} are also supported.

In practice, whenever you add a new math token you will often want to add
typesetting definitions using
\texttt{latexdef}, \texttt{htmldef}, and
\texttt{althtmldef}, as described below.
That way, they will all be up to date.
Of course, whether or not you want to use all three definitions will
depend on how the database is intended to be used.

Below we discuss the different possible \textit{definition-kind} options.
We will show data surrounded by double quotes (in practice they can also use
single quotes and/or be a sequence joined by \texttt{+}s).
We will use specific names for the \textit{data} to make clear what
the data is used for, such as
{\em math-token} (for a Metamath math token,
{\em latex-string} (for string to be placed in a \LaTeX\ stream),
{\em {\sc html}-code} (for {\sc html} code),
and {\em filename} (for a filename).

\subsubsection{Typesetting Comment - \LaTeX}

The syntax for a \LaTeX\ definition is:
\begin{center}
 \texttt{latexdef "}{\em math-token}\texttt{" as "}{\em latex-string}\texttt{";}
\end{center}
\index{latex definitions@\LaTeX\ definitions}%
\index{\texttt{latexdef} statement}

The {\em token-string} and {\em latex-string} are the data
(character strings) for
the token and the \LaTeX\ definition of the token, respectively,

These \LaTeX\ definitions are used by the Metamath program
when it is asked to product \LaTeX output using
the \texttt{write tex} command.

\subsubsection{Typesetting Comment - {\sc html}}

The key kinds of {\sc HTML} definitions have the following syntax:

\vskip 1ex
    \texttt{htmldef "}{\em math-token}\texttt{" as "}{\em
    {\sc html}-code}\texttt{";}\index{\texttt{htmldef} statement}
                    \ \ \ \ \ \ldots

    \texttt{althtmldef "}{\em math-token}\texttt{" as "}{\em
{\sc html}-code}\texttt{";}\index{\texttt{althtmldef} statement}

                    \ \ \ \ \ \ldots

Note that in {\sc HTML} there are two possible definitions for math tokens.
This feature is useful when
an alternate representation of symbols is desired, for example one that
uses Unicode entities and another uses {\sc gif} images.

There are many other typesetting definitions that can control {\sc HTML}.
These include:

\vskip 1ex

    \texttt{htmldef "}{\em math-token}\texttt{" as "}{\em {\sc
    html}-code}\texttt{";}

    \texttt{htmltitle "}{\em {\sc html}-code}\texttt{";}%
\index{\texttt{htmltitle} statement}

    \texttt{htmlhome "}{\em {\sc html}-code}\texttt{";}%
\index{\texttt{htmlhome} statement}

    \texttt{htmlvarcolor "}{\em {\sc html}-code}\texttt{";}%
\index{\texttt{htmlvarcolor} statement}

    \texttt{htmlbibliography "}{\em filename}\texttt{";}%
\index{\texttt{htmlbibliography} statement}

\vskip 1ex

\noindent The \texttt{htmltitle} is the {\sc html} code for a common
title, such as ``Metamath Proof Explorer.''  The \texttt{htmlhome} is
code for a link back to the home page.  The \texttt{htmlvarcolor} is
code for a color key that appears at the bottom of each proof.  The file
specified by {\em filename} is an {\sc html} file that is assumed to
have a \texttt{<A NAME=}\ldots\texttt{>} tag for each bibiographic
reference in the database comments.  For example, if
\texttt{[Monk]}\index{\texttt{\char`\[}\ldots\texttt{]} inside comments}
occurs in the comment for a theorem, then \texttt{<A NAME='Monk'>} must
be present in the file; if not, a warning message is given.

Associated with
\texttt{althtmldef}
are the statements
\vskip 1ex

    \texttt{htmldir "}{\em
      directoryname}\texttt{";}\index{\texttt{htmldir} statement}

    \texttt{althtmldir "}{\em
     directoryname}\texttt{";}\index{\texttt{althtmldir} statement}

\vskip 1ex
\noindent giving the directories of the {\sc gif} and Unicode versions
respectively; their purpose is to provide cross-linking between the
two versions in the generated web pages.

When two different types of pages need to be produced from a single
database, such as the Hilbert Space Explorer that extends the Metamath
Proof Explorer, ``extended'' variables may be declared in the
\texttt{\$t} comment:
\vskip 1ex

    \texttt{exthtmltitle "}{\em {\sc html}-code}\texttt{";}%
\index{\texttt{exthtmltitle} statement}

    \texttt{exthtmlhome "}{\em {\sc html}-code}\texttt{";}%
\index{\texttt{exthtmlhome} statement}

    \texttt{exthtmlbibliography "}{\em filename}\texttt{";}%
\index{\texttt{exthtmlbibliography} statement}

\vskip 1ex
\noindent When these are declared, you also must declare
\vskip 1ex

    \texttt{exthtmllabel "}{\em label}\texttt{";}%
\index{\texttt{exthtmllabel} statement}

\vskip 1ex \noindent that identifies the database statement where the
``extended'' section of the database starts (in our example, where the
Hilbert Space Explorer starts).  During the generation of web pages for
that starting statement and the statements after it, the {\sc html} code
assigned to \texttt{exthtmltitle} and \texttt{exthtmlhome} is used
instead of that assigned to \texttt{htmltitle} and \texttt{htmlhome},
respectively.

\begin{sloppy}
\subsection{Additional Information Com\-ment (\texttt{\$j})} \label{jcomment}
\end{sloppy}

The additional information comment, aka the
\texttt{\$j}\index{\texttt{\$j} comment}\index{additional information comment}
comment,
provides a way to add additional structured information that can
be optionally parsed by systems.

The additional information comment is parsed the same way as the
typesetting comment (\texttt{\$t}) (see section \ref{tcomment}).
That is,
the additional information comment begins with the token
\texttt{\$j} within a comment,
and continues until the comment close \texttt{\$)}.
Within an additional information comment is a sequence of one or more
commands of the form \texttt{command arg arg ... ;}
where each of the zero or more \texttt{arg} values
can be either a quoted string or a keyword.
Note that every command ends in an unquoted semicolon.
If a verifier is parsing an additional information comment, but
doesn't recognize a particular command, it must skip the command
by finding the end of the command (an unquoted semicolon).

A database may have 0 or more additional information comments.
Note, however, that a verifier may ignore these comments entirely or only
process certain commands in an additional information comment.
The \texttt{mmj2} verifier supports many commands in additional information
comments.
We encourage systems that process additional information comments
to coordinate so that they will use the same command for the same effect.

Examples of additional information comments with various commands
(from the \texttt{set.mm} database) are:

\begin{itemize}
   \item Define the syntax and logical typecodes,
     and declare that our grammar is
     unambiguous (verifiable using the KLR parser, with compositing depth 5).
\begin{verbatim}
  $( $j
    syntax 'wff';
    syntax '|-' as 'wff';
    unambiguous 'klr 5';
  $)
\end{verbatim}

   \item Register $\lnot$ and $\rightarrow$ as primitive expressions
           (lacking definitions).
\begin{verbatim}
  $( $j primitive 'wn' 'wi'; $)
\end{verbatim}

   \item There is a special justification for \texttt{df-bi}.
\begin{verbatim}
  $( $j justification 'bijust' for 'df-bi'; $)
\end{verbatim}

   \item Register $\leftrightarrow$ as an equality for its type (wff).
\begin{verbatim}
  $( $j
    equality 'wb' from 'biid' 'bicomi' 'bitri';
    definition 'dfbi1' for 'wb';
  $)
\end{verbatim}

   \item Theorem \texttt{notbii} is the congruence law for negation.
\begin{verbatim}
  $( $j congruence 'notbii'; $)
\end{verbatim}

   \item Add \texttt{setvar} as a typecode.
\begin{verbatim}
  $( $j syntax 'setvar'; $)
\end{verbatim}

   \item Register $=$ as an equality for its type (\texttt{class}).
\begin{verbatim}
  $( $j equality 'wceq' from 'eqid' 'eqcomi' 'eqtri'; $)
\end{verbatim}

\end{itemize}


\subsection{Including Other Files in a Metamath Source File} \label{include}
\index{\texttt{\$[} and \texttt{\$]} auxiliary keywords}

The keywords \texttt{\$[} and \texttt{\$]} specify a file to be
included\index{included file}\index{file inclusion} at that point in a
Metamath\index{Metamath} source file\index{source file}.  The syntax for
including a file is as follows:
\begin{center}
\texttt{\$[} {\em file-name} \texttt{\$]}
\end{center}

The {\em file-name} should be a single token\index{token} with the same syntax
as a math symbol (i.e., all 93 non-whitespace
printable characters other than \texttt{\$} are
allowed, subject to the file-naming limitations of your operating system).
Comments may appear between the \texttt{\$[} and \texttt{\$]} keywords.  Included
files may include other files, which may in turn include other files, and so
on.

For example, suppose you want to use the set theory database as the starting
point for your own theory.  The first line in your file could be
\begin{center}
\texttt{\$[ set.mm \$]}
\end{center} All of the information (axioms, theorems,
etc.) in \texttt{set.mm} and any files that {\em it} includes will become
available for you to reference in your file. This can help make your work more
modular. A drawback to including files is that if you change the name of a
symbol or the label of a statement, you must also remember to update any
references in any file that includes it.


The naming conventions for included files are the same as those of your
operating system.\footnote{On the Macintosh, prior to Mac OS X,
 a colon is used to separate disk
and folder names from your file name.  For example, {\em volume}\texttt{:}{\em
file-name} refers to the root directory, {\em volume}\texttt{:}{\em
folder-name}\texttt{:}{\em file-name} refers to a folder in root, and {\em
volume}\texttt{:}{\em folder-name}\texttt{:}\ldots\texttt{:}{\em file-name} refers to a
deeper folder.  A simple {\em file-name} refers to a file in the folder from
which you launch the Metamath application.  Under Mac OS X and later,
the Metamath program is run under the Terminal application, which
conforms to Unix naming conventions.}\index{Macintosh file
names}\index{file names!Macintosh}\label{includef} For compatibility among
operating systems, you should keep the file names as simple as possible.  A
good convention to use is {\em file}\texttt{.mm} where {\em file} is eight
characters or less, in lower case.

There is no limit to the nesting depth of included files.  One thing that you
should be aware of is that if two included files themselves include a common
third file, only the {\em first} reference to this common file will be read
in.  This allows you to include two or more files that build on a common
starting file without having to worry about label and symbol conflicts that
would occur if the common file were read in more than once.  (In fact, if a
file includes itself, the self-reference will be ignored, although of course
it would not make any sense to do that.)  This feature also means, however,
that if you try to include a common file in several inner blocks, the result
might not be what you expect, since only the first reference will be replaced
with the included file (unlike the include statement in most other computer
languages).  Thus you would normally include common files only in the
outermost block\index{outermost block}.

\subsection{Compressed Proof Format}\label{compressed1}\index{compressed
proof}\index{proof!compressed}

The proof notation presented in Section~\ref{proof} is called a
{\bf normal proof}\index{normal proof}\index{proof!normal} and in principle is
sufficient to express any proof.  However, proofs often contain steps and
subproofs that are identical.  This is particularly true in typical
Metamath\index{Metamath} applications, because Metamath requires that the math
symbol sequence (usually containing a formula) at each step be separately
constructed, that is, built up piece by piece. As a result, a lot of
repetition often results.  The {\bf compressed proof} format allows Metamath
to take advantage of this redundancy to shorten proofs.

The specification for the compressed proof format is given in
Appen\-dix~\ref{compressed}.

Normally you need not concern yourself with the details of the compressed
proof format, since the Metamath program will allow you to convert from
the normal format to the compressed format with ease, and will also
automatically convert from the compressed format when proofs are displayed.
The overall structure of the compressed format is as follows:
\begin{center}
  \texttt{\$= ( } {\em label-list} \texttt{) } {\em compressed-proof\ }\ \texttt{\$.}
\end{center}
\index{\texttt{\$=} keyword}
The first \texttt{(} serves as a flag to Metamath that a compressed proof
follows.  The {\em label-list} includes all statements referred to by the
proof except the mandatory hypotheses\index{mandatory hypothesis}.  The {\em
compressed-proof} is a compact encoding of the proof, using upper-case
letters, and can be thought of as a large integer in base 26.  White
space\index{white space} inside a {\em compressed-proof} is
optional and is ignored.

It is important to note that the order of the mandatory hypotheses of
the statement being proved must not be changed if the compressed proof
format is used, otherwise the proof will become incorrect.  The reason
for this is that the mandatory hypotheses are not mentioned explicitly
in the compressed proof in order to make the compression more efficient.
If you wish to change the order of mandatory hypotheses, you must first
convert the proof back to normal format using the \texttt{save proof
{\em statement} /normal}\index{\texttt{save proof} command} command.
Later, you can go back to compressed format with \texttt{save proof {\em
statement} /compressed}.

During error checking with the \texttt{verify proof} command, an error
found in a compressed proof may point to a character in {\em
compressed-proof}, which may not be very meaningful to you.  In this
case, try to \texttt{save proof /normal} first, then do the
\texttt{verify proof} again.  In general, it is best to make sure a
proof is correct before saving it in compressed format, because severe
errors are less likely to be recoverable than in normal format.

\subsection{Specifying Unknown Proofs or Subproofs}\label{unknown}

In a proof under development, any step or subproof that is not yet known
may be represented with a single \texttt{?}.  For the purposes of
parsing the proof, the \texttt{?}\ \index{\texttt{]}@\texttt{?}\ inside
proofs} will push a single entry onto the RPN stack just as if it were a
hypothesis.  While developing a proof with the Proof
Assistant\index{Proof Assistant}, a partially developed proof may be
saved with the \texttt{save new{\char`\_}proof}\index{\texttt{save
new{\char`\_}proof} command} command, and \texttt{?}'s will be placed at
the appropriate places.

All \texttt{\$p}\index{\texttt{\$p} statement} statements must have
proofs, even if they are entirely unknown.  Before creating a proof with
the Proof Assistant, you should specify a completely unknown proof as
follows:
\begin{center}
  {\em label} \texttt{\$p} {\em statement} \texttt{\$= ?\ \$.}
\end{center}
\index{\texttt{\$=} keyword}
\index{\texttt{]}@\texttt{?}\ inside proofs}

The \texttt{verify proof}\index{\texttt{verify proof} command} command
will check the known portions of a partial proof for errors, but will
warn you that the statement has not been proved.

Note that partially developed proofs may be saved in compressed format
if desired.  In this case, you will see one or more \texttt{?}'s in the
{\em compressed-proof} part.\index{compressed
proof}\index{proof!compressed}

\section{Axioms vs.\ Definitions}\label{definitions}

The \textit{basic}
Metamath\index{Metamath} language and program
make no distinction\index{axiom vs.\
definition} between axioms\index{axiom} and
definitions.\index{definition} The \texttt{\$a}\index{\texttt{\$a}
statement} statement is used for both.  At first, this may seem
puzzling.  In the minds of many mathematicians, the distinction is
clear, even obvious, and hardly worth discussing.  A definition is
considered to be merely an abbreviation that can be replaced by the
expression for which it stands; although unless one actually does this,
to be precise then one should say that a theorem\index{theorem} is a
consequence of the axioms {\em and} the definitions that are used in the
formulation of the theorem \cite[p.~20]{Behnke}.\index{Behnke, H.}

\subsection{What is a Definition?}

What is a definition?  In its simplest form, a definition introduces a new
symbol and provides an unambiguous rule to transform an expression containing
the new symbol to one without it.  The concept of a ``proper
definition''\index{proper definition}\index{definition!proper} (as opposed to
a creative definition)\index{creative definition}\index{definition!creative}
that is usually agreed upon is (1) the definition should not strengthen the
language and (2) any symbols introduced by the definition should be eliminable
from the language \cite{Nemesszeghy}\index{Nemesszeghy, E. Z.}.  In other
words, they are mere typographical conveniences that do not belong to the
system and are theoretically superfluous.  This may seem obvious, but in fact
the nature of definitions can be subtle, sometimes requiring difficult
metatheorems to establish that they are not creative.

A more conservative stance was taken by logician S.
Le\'{s}niewski.\index{Le\'{s}niewski, S.}
\begin{quote}
Le\'{s}niewski
regards definitions as theses of the system.  In this respect they do
not differ either from the axioms or from theorems, i.e.\ from the
theses added to the system on the basis of the rule of substitution or
the rule of detachment [modus ponens].  Once definitions have been
accepted as theses of the system, it becomes necessary to consider them
as true propositions in the same sense in which axioms are true
\cite{Lejewski}.
\end{quote}\index{Lejewski, Czeslaw}

Let us look at some simple examples of definitions in propositional
calculus.  Consider the definition of logical {\sc or}
(disjunction):\index{disjunction ($\vee$)} ``$P\vee Q$ denotes $\neg P
\rightarrow Q$ (not $P$ implies $Q$).''  It is very easy to recognize a
statement making use of this definition, because it introduces the new
symbol $\vee$ that did not previously exist in the language.  It is easy
to see that no new theorems of the original language will result from
this definition.

Next, consider a definition that eliminates parentheses:  ``$P
\rightarrow Q\rightarrow R$ denotes $P\rightarrow (Q \rightarrow R)$.''
This is more subtle, because no new symbols are introduced.  The reason
this definition is considered proper is that no new symbol sequences
that are valid wffs (well-formed formulas)\index{well-formed formula
(wff)} in the original language will result from the definition, since
``$P \rightarrow Q\rightarrow R$'' is not a wff in the original
language.  Here, we implicitly make use of the fact that there is a
decision procedure that allows us to determine whether or not a symbol
sequence is a wff, and this fact allows us to use symbol sequences that
are not wffs to represent other things (such as wffs) by means of the
definition.  However, to justify the definition as not being creative we
need to prove that ``$P \rightarrow Q\rightarrow R$'' is in fact not a
wff in the original language, and this is more difficult than in the
case where we simply introduce a new symbol.

%Now let's take this reasoning to an extreme.  Propositional calculus is a
%decidable theory,\footnote{This means that a mechanical algorithm exists to
%determine whether or not a wff is a theorem.} so in principle we could make use
%of symbol sequences that are not theorems to represent other things (say, to
%encode actual theorems in a more compact way).  For example, let us extend the
%language by defining a wff ``$P$'' in the extended language as the theorem
%``$P\rightarrow P$''\footnote{This is one of the first theorems proved in the
%Metamath database \texttt{set.mm}.}\index{set
%theory database (\texttt{set.mm})} in the original language whenever ``$P$'' is
%not a theorem in the original language.  In the extended language, any wff
%``$Q$'' thus represents a theorem; to find out what theorem (in the original
%language) ``$Q$'' represents, we determine whether ``$Q$'' is a theorem in the
%original language (before the definition was introduced).  If so, we're done; if
%not, we replace ``$Q$'' by ``$Q\rightarrow Q$'' to eliminate the definition.
%This definition is therefore eliminable, and it does not ``strengthen'' the
%language because any wff that is not a theorem is not in the set of statements
%provable in the original language and thus is available for use by definitions.
%
%Of course, a definition such as this would render practically useless the
%communication of theorems of propositional calculus; but
%this is just a human shortcoming, since we can't always easily discern what is
%and is not a theorem by inspection.  In fact, the extended theory with this
%definition has no more and no less information than the original theory; it just
%expresses certain theorems of the form ``$P\rightarrow P$''
%in a more compact way.
%
%The point here is that what constitutes a proper definition is a matter of
%judgment about whether a symbol sequence can easily be recognized by a human
%as invalid in some sense (for example, not a wff); if so, the symbol sequence
%can be appropriated for use by a definition in order to make the extended
%language more compact.  Metamath\index{Metamath} lacks the ability to make this
%judgment, since as far as Metamath is concerned the definition of a wff, for
%example, is arbitrary.  You define for Metamath how wffs\index{well-formed
%formula (wff)} are constructed according to your own preferred style.  The
%concept of a wff may not even exist in a given formal system\index{formal
%system}.  Metamath treats all definitions as if they were new axioms, and it
%is up to the human mathematician to judge whether the definition is ``proper''
%'\index{proper definition}\index{definition!proper} in some agreed-upon way.

What constitutes a definition\index{definition} versus\index{axiom vs.\
definition} an axiom\index{axiom} is sometimes arbitrary in mathematical
literature.  For example, the connectives $\vee$ ({\sc or}), $\wedge$
({\sc and}), and $\leftrightarrow$ (equivalent to) in propositional
calculus are usually considered defined symbols that can be used as
abbreviations for expressions containing the ``primitive'' connectives
$\rightarrow$ and $\neg$.  This is the way we treat them in the standard
logic and set theory database \texttt{set.mm}\index{set theory database
(\texttt{set.mm})}.  However, the first three connectives can also be
considered ``primitive,'' and axiom systems have been devised that treat
all of them as such.  For example,
\cite[p.~35]{Goodstein}\index{Goodstein, R. L.} presents one with 15
axioms, some of which in fact coincide with what we have chosen to call
definitions in \texttt{set.mm}.  In certain subsets of classical
propositional calculus, such as the intuitionist
fragment\index{intuitionism}, it can be shown that one cannot make do
with just $\rightarrow$ and $\neg$ but must treat additional connectives
as primitive in order for the system to make sense.\footnote{Two nice
systems that make the transition from intuitionistic and other weak
fragments to classical logic just by adding axioms are given in
\cite{Robinsont}\index{Robinson, T. Thacher}.}

\subsection{The Approach to Definitions in \texttt{set.mm}}

In set theory, recursive definitions define a newly introduced symbol in
terms of itself.  The justification of recursive definitions, using
several ``recursion theorems,'' is usually one of the first
sophisticated proofs a student encounters when learning set theory, and
there is a significant amount of implicit metalogic behind a recursive
definition even though the definition itself is typically simple to
state.  It is, however, possible to substitute one kind of complexity
for another.  We can eliminate the need for metalogical justification by
defining the operation directly with an explicit (but complicated)
expression, then deriving the recursive definition directly as a
theorem, using a recursion theorem ``in reverse.''  We do this in
\texttt{set.mm}, as follows.

In \texttt{set.mm} our goal was to introduce almost all definitions in
the form of two expressions connected by either $\leftrightarrow$ or
$=$, where the thing being defined does not appear on the right hand
side.  Quine calls this form ``a genuine or direct definition'' \cite[p.
174]{Quine}\index{Quine, Willard Van Orman}, which makes the definitions
very easy to eliminate and the metalogic\index{metalogic} needed to
justify them as simple as possible.

We achieved this goal in almost all cases in \texttt{set.mm}.
Sometimes this makes the definitions more complex and less
intuitive.  For example, the traditional way to define addition of
natural numbers is to define an operation called {\em
successor}\index{successor} (which means ``plus one'' and is denoted by
``${\rm suc}$''), then define addition recursively\index{recursive
definition} with the two definitions $n + 0 = n$ and $m + {\rm suc}\,n =
{\rm suc} (m + n)$.  Although this definition seems simple and obvious,
the method to eliminate the definition is not obvious:  in the second
part of the definition, addition is defined in terms of itself.  By
eliminating the definition, we don't mean repeatedly applying it to
specific $m$ and $n$ but rather showing the explicit, closed-form
set-theoretical expression that $m + n$ represents, that will work for
any $m$ and $n$ and that does not have a $+$ sign on its right-hand
side.  For a recursive definition like this not to be circular
(creative), there are some hidden, underlying assumptions we must make,
for example that the natural numbers have a certain kind of order.  In
\texttt{set.mm} we chose to start with the direct (though complex and
nonintuitive) definition then derive from it the standard recursive
definition.\footnote{The closed-form definition used in \texttt{set.mm}
for the addition operation on ordinals\index{ordinal
addition}\index{addition!of ordinals} (of which natural numbers are a
subset) is

\setbox\startprefix=\hbox{\tt \ \ df-oadd\ \$a\ }
\setbox\contprefix=\hbox{\tt \ \ \ \ \ \ \ \ \ \ \ \ \ }
\startm
\m{\vdash}\m{+_o}\m{=}\m{\{}\m{\langle}\m{\langle}\m{x}\m{,}\m{y}\m{\rangle}%
\m{,}\m{z}\m{\rangle}\m{|}\m{(}\m{(}\m{x}\m{\in}\m{{\rm On}}\m{\wedge}\m{y}\m{%
\in}\m{{\rm On}}\m{)}\m{\wedge}\m{z}\m{=}\m{(}\m{{\rm rec}}\m{(}\m{\{}\m{%
\langle}\m{w}\m{,}\m{v}\m{\rangle}\m{|}\m{v}\m{=}\m{{\rm suc}}\m{w}\m{\}}\m{,}%
\m{x}\m{)}\m{`}\m{y}\m{)}\m{)}\m{\}}
\endm

\noindent Here, the abstraction class of nested ordered pairs\index{abstraction
class!of nested ordered pairs} is defined by \texttt{df-oprab} in \texttt{set.mm},
and ${\rm rec}$ is a ``recursion operator''\index{recursion operator} with
the definition

\setbox\startprefix=\hbox{\tt \ \ df-rfg\ \$a\ }
\setbox\contprefix=\hbox{\tt \ \ \ \ \ \ \ \ \ \ \ \ }
\startm
\m{\vdash}\m{{\rm rec}}\m{(}\m{F}\m{,}\m{A}\m{)}\m{=}\m{\bigcup}\m{\{}\m{f}%
\m{|}\m{\exists}\m{x}\m{(}\m{x}\m{\in}\m{{\rm On}}\m{\wedge}\m{(}\m{f}\m{{%
\rm Fn}}\m{x}\m{\wedge}\m{\forall}\m{y}\m{(}\m{y}\m{\in}\m{x}\m{\rightarrow}%
\m{(}\m{f}\m{`}\m{y}\m{)}\m{=}\m{(}\m{\{}\m{\langle}\m{g}\m{,}\m{z}\m{\rangle}%
\m{|}\m{(}\m{(}\m{g}\m{=}\m{\varnothing}\m{\wedge}\m{z}\m{=}\m{A}\m{)}\m{\vee}%
\m{(}\m{\lnot}\m{(}\m{g}\m{=}\m{\varnothing}\m{\vee}\m{{\rm Lim}}\m{{\rm dom}}%
\m{g}\m{)}\m{\wedge}\m{z}\m{=}\m{(}\m{F}\m{`}\m{(}\m{g}\m{`}\m{\bigcup}\m{{%
\rm dom}}\m{g}\m{)}\m{)}\m{)}\m{\vee}\m{(}\m{{\rm Lim}}\m{{\rm dom}}\m{g}\m{%
\wedge}\m{z}\m{=}\m{\bigcup}\m{{\rm ran}}\m{g}\m{)}\m{)}\m{\}}\m{`}\m{(}\m{f}%
\m{\restriction}\m{y}\m{)}\m{)}\m{)}\m{)}\m{)}\m{\}}
\endm

\noindent which can be further broken down with definitions shown in
Section~\ref{setdefinitions}.  You may be surprised at the complexity of what
seems like such a simple notion.  From these definitions the simpler, more
intuitive recursive definition is derived as a set of theorems.}\index{natural
number}\index{addition}\index{recursive definition}\index{ordinal addition}
The end result is the same, but we completely eliminate the rather complex
metalogic that justifies the recursive definition.  (For a mathematician,
recursive definitions are more efficient and intuitive than
direct ones once the metalogic has been learned or possibly just accepted as
correct.  However, it was felt that direct definition in \texttt{set.mm}
maximizes rigor by minimizing metalogic.  It can be eliminated effortlessly,
something difficult to do with a recursive definition.)

\subsection{Adding Constraints on Definitions}

The basic Metamath language and the Metamath program do
not have any built-in constraints on definitions, since they are just
\$a statements.

However, nothing prevents a verification system from verifying
additional rules to impose further limitations on definitions.
For example, the \texttt{mmj2}\index{mmj2} program
supports various kinds of
additional information comments (see section \ref{jcomment}).
One of their uses is to optionally verify additional constraints,
including constraints to verify that definitions meet certain
requirements.
These additional checks are required by the
continuous integration (CI)\index{continuous integration (CI)}
checks of the
\texttt{set.mm}\index{set theory database (\texttt{set.mm})}%
\index{Metamath Proof Explorer}
database.
This approach enables us to optionally impose additional requirements
on definitions if we wish, without necessarily imposing those rules on
all databases or requiring all verification systems to implement them.
In addition, this allows us to impose specialized constraints tailored
to one database while not requiring other systems to implement
those specialized constraints.

We impose two constraints on the
\texttt{set.mm}\index{set theory database (\texttt{set.mm})}%
\index{Metamath Proof Explorer} database
via the \texttt{mmj2}\index{mmj2} program that are worth discussing here:
a parse check and a definition soundness check.

% On February 11, 2019 8:32:32 PM EST, saueran@oregonstate.edu wrote:
% The following addition to the end of set.mm is accepted by the mmj2
% parser and definition checker and the metamath verifier(at least it was
% when I checked, you should check it too), and creates a contradiction by
% proving the theorem |- ph.
% ${
% wleftp $a wff ( ( ph ) $.
% wbothp $a wff ( ph ) $.
% df-leftp $a |- ( ( ( ph ) <-> -. ph ) $.
% df-bothp $a |- ( ( ph ) <-> ph ) $.
% anything $p |- ph $=
%   ( wbothp wn wi wleftp df-leftp biimpi df-bothp mpbir mpbi simplim ax-mp)
%   ABZAMACZDZCZMOEZOCQAEZNDZRNAFGSHIOFJMNKLAHJ $.
% $}
%
% This particular problem is countered by enabling, within mmj2,
% SetParser,mmj.verify.LRParser

First,
we enable a parse check in \texttt{mmj2} (through its
\texttt{SetParser} command) that requires that all new definitions
must \textit{not} create an ambiguous parse for a KLR(5) parser.
This prevents some errors such as definitions with imbalanced parentheses.

Second, we run a definition soundness check specific to
\texttt{set.mm} or databases similar to it.
(through the \texttt{definitionCheck} macro).
Some \texttt{\$a} statements (including all ax-* statemnets)
are excluded from these checks, as they will
always fail this simple check,
but they are appropriate for most definitions.
This check imposes a set of additional rules:

\begin{enumerate}

\item New definitions must be introduced using $=$ or $\leftrightarrow$.

\item No \texttt{\$a} statement introduced before this one is allowed to use the
symbol being defined in this definition, and the definition is not
allowed to use itself (except once, in the definiendum).

\item Every variable in the definiens should not be distinct

\item Every dummy variable in the definiendum
are required to be distinct from each other and from variables in
the definiendum.
To determine this, the system will look for a "justification" theorem
in the database, and if it is not there, attempt to briefly prove
$( \varphi \rightarrow \forall x \varphi )$  for each dummy variable x.

\item Every dummy variable should be a set variable,
unless there is a justification theorem available.

\item Every dummy variable must be bound
(if the system cannot determine this a justification theorem must be
provided).

\end{enumerate}

\subsection{Summary of Approach to Definitions}

In short, when being rigorous it turns out that
definitions can be subtle, sometimes requiring difficult
metatheorems to establish that they are not creative.

Instead of building such complications into the Metamath language itself,
the basic Metmath language and program simply treat traditional
axioms and definitions as the same kind of \texttt{\$a} statement.
We have then built various tools to enable people to
verify additional conditions as their creators believe is appropriate
for those specific databases, without complicating the Metamath foundations.

\chapter{The Metamath Program}\label{commands}

This chapter provides a reference manual for the
Metamath program.\index{Metamath!commands}

Current instructions for obtaining and installing the Metamath program
can be found at the \url{http://metamath.org} web site.
For Windows, there is a pre-compiled version called
\texttt{metamath.exe}.  For Unix, Linux, and Mac OS X
(which we will refer to collectively as ``Unix''), the Metamath program
can be compiled from its source code with the command
\begin{verbatim}
gcc *.c -o metamath
\end{verbatim}
using the \texttt{gcc} {\sc c} compiler available on those systems.

In the command syntax descriptions below, fields enclosed in square
brackets [\ ] are optional.  File names may be optionally enclosed in
single or double quotes.  This is useful if the file name contains
spaces or
slashes (\texttt{/}), such as in Unix path names, \index{Unix file
names}\index{file names!Unix} that might be confused with Metamath
command qualifiers.\index{Unix file names}\index{file names!Unix}

\section{Invoking Metamath}

Unix, Linux, and Mac OS X
have a command-line interface called the {\em
bash shell}.  (In Mac OS X, select the Terminal application from
Applications/Utilities.)  To invoke Metamath from the bash shell prompt,
assuming that the Metamath program is in the current directory, type
\begin{verbatim}
bash$ ./metamath
\end{verbatim}

To invoke Metamath from a Windows DOS or Command Prompt, assuming that
the Metamath program is in the current directory (or in a directory
included in the Path system environment variable), type
\begin{verbatim}
C:\metamath>metamath
\end{verbatim}

To use command-line arguments at invocation, the command-line arguments
should be a list of Metamath commands, surrounded by quotes if they
contain spaces.  In Windows, the surrounding quotes must be double (not
single) quotes.  For example, to read the database file \texttt{set.mm},
verify all proofs, and exit the program, type (under Unix)
\begin{verbatim}
bash$ ./metamath 'read set.mm' 'verify proof *' exit
\end{verbatim}
Note that in Unix, any directory path with \texttt{/}'s must be
surrounded by quotes so Metamath will not interpret the \texttt{/} as a
command qualifier.  So if \texttt{set.mm} is in the \texttt{/tmp}
directory, use for the above example
\begin{verbatim}
bash$ ./metamath 'read "/tmp/set.mm"' 'verify proof *' exit
\end{verbatim}

For convenience, if the command-line has one argument and no spaces in
the argument, the command is implicitly assumed to be \texttt{read}.  In
this one special case, \texttt{/}'s are not interpreted as command
qualifiers, so you don't need quotes around a Unix file name.  Thus
\begin{verbatim}
bash$ ./metamath /tmp/set.mm
\end{verbatim}
and
\begin{verbatim}
bash$ ./metamath "read '/tmp/set.mm'"
\end{verbatim}
are equivalent.


\section{Controlling Metamath}

The Metamath program was first developed on a {\sc vax/vms} system, and
some aspects of its command line behavior reflect this heritage.
Hopefully you will find it reasonably user-friendly once you get used to
it.

Each command line is a sequence of English-like words separated by
spaces, as in \texttt{show settings}.  Command words are not case
sensitive, and only as many letters are needed as are necessary to
eliminate ambiguity; for example, \texttt{sh se} would work for the
command \texttt{show settings}.  In some cases arguments such as file
names, statement labels, or symbol names are required; these are
case-sensitive (although file names may not be on some operating
systems).

A command line is entered by typing it in then pressing the {\em return}
({\em enter}) key.  To find out what commands are available, type
\texttt{?} at the \texttt{MM>} prompt.  To find out the choices at any
point in a command, press {\em return} and you will be prompted for
them.  The default choice (the one selected if you just press {\em
return}) is shown in brackets (\texttt{<>}).

You may also type \texttt{?} in place of a command word to force
Metamath to tell you what the choices are.  The \texttt{?} method won't
work, though, if a non-keyword argument such as a file name is expected
at that point, because the program will think that \texttt{?} is the
value of the argument.

Some commands have one or more optional qualifiers which modify the
behavior of the command.  Qualifiers are preceded by a slash
(\texttt{/}), such as in \texttt{read set.mm / verify}.  Spaces are
optional around the \texttt{/}.  If you need to use a space or
slash in a command
argument, as in a Unix file name, put single or double quotes around the
command argument.

The \texttt{open log} command will save everything you see on the
screen and is useful to help you recover should something go wrong in a
proof, or if you want to document a bug.

If a command responds with more than a screenful, you will be
prompted to \texttt{<return> to continue, Q to quit, or S to scroll to
end}.  \texttt{Q} or \texttt{q} (not case-sensitive) will complete the
command internally but will suppress further output until the next
\texttt{MM>} prompt.  \texttt{s} will suppress further pausing until the
next \texttt{MM>} prompt.  After the first screen, you are also
presented with the choice of \texttt{b} to go back a screenful.  Note
that \texttt{b} may also be entered at the \texttt{MM>} prompt
immediately after a command to scroll back through the output of that
command.

A command line enclosed in quotes is executed by your operating system.
See Section~\ref{oscmd}.

{\em Warning:} Pressing {\sc ctrl-c} will abort the Metamath program
unconditionally.  This means any unsaved work will be lost.


\subsection{\texttt{exit} Command}\index{\texttt{exit} command}

Syntax:  \texttt{exit} [\texttt{/force}]

This command exits from Metamath.  If there have been changes to the
source with the \texttt{save proof} or \texttt{save new{\char`\_}proof}
commands, you will be given an opportunity to \texttt{write source} to
permanently save the changes.

In Proof Assistant\index{Proof Assistant} mode, the \texttt{exit} command will
return to the \verb/MM>/ prompt. If there were changes to the proof, you will
be given an opportunity to \texttt{save new{\char`\_}proof}.

The \texttt{quit} command is a synonym for \texttt{exit}.

Optional qualifier:
    \texttt{/force} - Do not prompt if changes were not saved.  This qualifier is
        useful in \texttt{submit} command files (Section~\ref{sbmt})
        to ensure predictable behavior.





\subsection{\texttt{open log} Command}\index{\texttt{open log} command}
Syntax:  \texttt{open log} {\em file-name}

This command will open a log file that will store everything you see on
the screen.  It is useful to help recovery from a mistake in a long Proof
Assistant session, or to document bugs.\index{Metamath!bugs}

The log file can be closed with \texttt{close log}.  It will automatically be
closed upon exiting Metamath.



\subsection{\texttt{close log} Command}\index{\texttt{close log} command}
Syntax:  \texttt{close log}

The \texttt{close log} command closes a log file if one is open.  See
also \texttt{open log}.




\subsection{\texttt{submit} Command}\index{\texttt{submit} command}\label{sbmt}
Syntax:  \texttt{submit} {\em filename}

This command causes further command lines to be taken from the specified
file.  Note that any line beginning with an exclamation point (\texttt{!}) is
treated as a comment (i.e.\ ignored).  Also note that the scrolling
of the screen output is continuous, so you may want to open a log file
(see \texttt{open log}) to record the results that fly by on the screen.
After the lines in the file are exhausted, Metamath returns to its
normal user interface mode.

Currently, the \texttt{submit} command is not recursive.  In other words,
\texttt{submit} commands are not allowed inside a command file.


\subsection{\texttt{erase} Command}\index{\texttt{erase} command}
Syntax:  \texttt{erase}

This command will reset Metamath to its starting state, deleting any
data\-base that was \texttt{read} in.
 If there have been changes to the
source with the \texttt{save proof} or \texttt{save new{\char`\_}proof}
commands, you will be given an opportunity to \texttt{write source} to
permanently save the changes.



\subsection{\texttt{set echo} Command}\index{\texttt{set echo} command}
Syntax:  \texttt{set echo on} or \texttt{set echo off}

The \texttt{set echo on} command will cause command lines to be echoed with any
abbreviations expanded.  While learning the Metamath commands, this
feature will show you the exact command that your abbreviated input
corresponds to.



\subsection{\texttt{set scroll} Command}\index{\texttt{set scroll} command}
Syntax:  \texttt{set scroll prompted} or \texttt{set scroll continuous}

The Metamath command line interface starts off in the \texttt{prompted} mode,
which means that you will be prompted to continue or quit after each
full screen in a long listing.  In \texttt{continuous} mode, long listings will be
scrolled without pausing.

% LaTeX bug? (1) \texttt{\_} puts out different character than
% \texttt{{\char`\_}}
%  = \verb$_$  (2) \texttt{{\char`\_}} puts out garbage in \subsection
%  argument
\subsection{\texttt{set width} Command}\index{\texttt{set
width} command}
Syntax:  \texttt{set width} {\em number}

Metamath assumes the width of your screen is 79 characters (chosen
because the Command Prompt in Windows XP has a wrapping bug at column
80).  If your screen is wider or narrower, this command allows you to
change this default screen width.  A larger width is advantageous for
logging proofs to an output file to be printed on a wide printer.  A
smaller width may be necessary on some terminals; in this case, the
wrapping of the information messages may sometimes seem somewhat
unnatural, however.  In \LaTeX\index{latex@{\LaTeX}!characters per line},
there is normally a maximum of 61 characters per line with typewriter
font.  (The examples in this book were produced with 61 characters per
line.)

\subsection{\texttt{set height} Command}\index{\texttt{set
height} command}
Syntax:  \texttt{set height} {\em number}

Metamath assumes your screen height is 24 lines of characters.  If your
screen is taller or shorter, this command lets you to change the number
of lines at which the display pauses and prompts you to continue.

\subsection{\texttt{beep} Command}\index{\texttt{beep} command}

Syntax:  \texttt{beep}

This command will produce a beep.  By typing it ahead after a
long-running command has started, it will alert you that the command is
finished.  For convenience, \texttt{b} is an abbreviation for
\texttt{beep}.

Note:  If \texttt{b} is typed at the \texttt{MM>} prompt immediately
after the end of a multiple-page display paged with ``\texttt{Press
<return> for more}...'' prompts, then the \texttt{b} will back up to the
previous page rather than perform the \texttt{beep} command.
In that case you must type the unabbreviated \texttt{beep} form
of the command.

\subsection{\texttt{more} Command}\index{\texttt{more} command}

Syntax:  \texttt{more} {\em filename}

This command will display the contents of an {\sc ascii} file on your
screen.  (This command is provided for convenience but is not very
powerful.  See Section~\ref{oscmd} to invoke your operating system's
command to do this, such as the \texttt{more} command in Unix.)

\subsection{Operating System Commands}\index{operating system
command}\label{oscmd}

A line enclosed in single or double quotes will be executed by your
computer's operating system if it has a command line interface.  For
example, on a {\sc vax/vms} system,
\verb/MM> 'dir'/
will print disk directory contents.  Note that this feature will not
work on the Macintosh prior to Mac OS X, which does not have a command
line interface.

For your convenience, the trailing quote is optional.

\subsection{Size Limitations in Metamath}

In general, there are no fixed, predefined limits\index{Metamath!memory
limits} on how many labels, tokens\index{token}, statements, etc.\ that
you may have in a database file.  The Metamath program uses 32-bit
variables (64-bit on 64-bit CPUs) as indices for almost all internal
arrays, which are allocated dynamically as needed.



\section{Reading and Writing Files}

The following commands create new files:  the \texttt{open} commands;
the \texttt{write} commands; the \texttt{/html},
\texttt{/alt{\char`\_}html}, \texttt{/brief{\char`\_}html},
\texttt{/brief{\char`\_}alt{\char`\_}html} qualifiers of \texttt{show
statement}, and \texttt{midi}.  The following commands append to files
previously opened:  the \texttt{/tex} qualifier of \texttt{show proof}
and \texttt{show new{\char`\_}proof}; the \texttt{/tex} and
\texttt{/simple{\char`\_}tex} qualifiers of \texttt{show statement}; the
\texttt{close} commands; and all screen dialog between \texttt{open log}
and \texttt{close log}.

The commands that create new files will not overwrite an existing {\em
filename} but will rename the existing one to {\em
filename}\texttt{{\char`\~}1}.  An existing {\em
filename}\texttt{{\char`\~}1} is renamed {\em
filename}\texttt{{\char`\~}2}, etc.\ up to {\em
filename}\texttt{{\char`\~}9}.  An existing {\em
filename}\texttt{{\char`\~}9} is deleted.  This makes recovery from
mistakes easier but also will clutter up your directory, so occasionally
you may want to clean up (delete) these \texttt{{\char`\~}}$n$ files.


\subsection{\texttt{read} Command}\index{\texttt{read} command}
Syntax:  \texttt{read} {\em file-name} [\texttt{/verify}]

This command will read in a Metamath language source file and any included
files.  Normally it will be the first thing you do when entering Metamath.
Statement syntax is checked, but proof syntax is not checked.
Note that the file name may be enclosed in single or double quotes;
this is useful if the file name contains slashes, as might be the case
under Unix.

If you are getting an ``\texttt{?Expected VERIFY}'' error
when trying to read a Unix file name with slashes, you probably haven't
quoted it.\index{Unix file names}\index{file names!Unix}

If you are prompted for the file name (by pressing {\em return}
 after \texttt{read})
you should {\em not} put quotes around it, even if it is a Unix file name
with slashes.

Optional command qualifier:

    \texttt{/verify} - Verify all proofs as the database is read in.  This
         qualifier will slow down reading in the file.  See \texttt{verify
         proof} for more information on file error-checking.

See also \texttt{erase}.



\subsection{\texttt{write source} Command}\index{\texttt{write source} command}
Syntax:  \texttt{write source} {\em filename}
[\texttt{/rewrap}]
[\texttt{/split}]
[\texttt{/keep\_includes}]
[\texttt{/no\_versioning}]

This command will write the contents of a Metamath\index{database}
database into a file.\index{source file}

Optional command qualifiers:

\texttt{/rewrap} -
Reformats statements and comments according to the
convention used in the set.mm database.
It unwraps the
lines in the comment before each \texttt{\$a} and \texttt{\$p} statement,
then it
rewraps the line.  You should compare the output to the original
to make sure that the desired effect results; if not, go back to
the original source.  The wrapped line length honors the
\texttt{set width}
parameter currently in effect.  Note:  Text
enclosed in \texttt{<HTML>}...\texttt{</HTML>} tags is not modified by the
\texttt{/rewrap} qualifier.
Proofs themselves are not reformatted;
use \texttt{save proof * / compressed} to do that.
An isolated tilde (\~{}) is always kept on the same line as the following
symbol, so you can find all comment references to a symbol by
searching for \~{} followed by a space and that symbol
(this is useful for finding cross-references).
Incidentally, \texttt{save proof} also honors the \texttt{set width}
parameter currently in effect.

\texttt{/split} - Files included in the source using the expression
\$[ \textit{inclfile} \$] will be
written into separate files instead of being included in a single output
file.  The name of each separately written included file will be the
\textit{inclfile} argument of its inclusion command.

\texttt{/keep\_includes} - If a source file has includes but is written as a
single file by omitting \texttt{/split}, by default the included files will
be deleted (actually just renamed with a \char`\~1 suffix unless
\texttt{/no\_versioning} is specified) to prevent the possibly confusing
source duplication in both the output file and the included file.
The \texttt{/keep\_includes} qualifier will prevent this deletion.

\texttt{/no\_versioning} - Backup files suffixed with \char`\~1 are not created.


\section{Showing Status and Statements}



\subsection{\texttt{show settings} Command}\index{\texttt{show settings} command}
Syntax:  \texttt{show settings}

This command shows the state of various parameters.

\subsection{\texttt{show memory} Command}\index{\texttt{show memory} command}
Syntax:  \texttt{show memory}

This command shows the available memory left.  It is not meaningful
on most modern operating systems,
which have virtual memory.\index{Metamath!memory usage}


\subsection{\texttt{show labels} Command}\index{\texttt{show labels} command}
Syntax:  \texttt{show labels} {\em label-match} [\texttt{/all}]
   [\texttt{/linear}]

This command shows the labels of \texttt{\$a} and \texttt{\$p}
statements that match {\em label-match}.  A \verb$*$ in {label-match}
matches zero or more characters.  For example, \verb$*abc*def$ will match all
labels containing \verb$abc$ and ending with \verb$def$.

Optional command qualifiers:

   \texttt{/all} - Include matches for \texttt{\$e} and \texttt{\$f}
   statement labels.

   \texttt{/linear} - Display only one label per line.  This can be useful for
       building scripts in conjunction with the utilities under the
       \texttt{tools}\index{\texttt{tools} command} command.



\subsection{\texttt{show statement} Command}\index{\texttt{show statement} command}
Syntax:  \texttt{show statement} {\em label-match} [{\em qualifiers} (see below)]

This command provides information about a statement.  Only statements
that have labels (\texttt{\$f}\index{\texttt{\$f} statement},
\texttt{\$e}\index{\texttt{\$e} statement},
\texttt{\$a}\index{\texttt{\$a} statement}, and
\texttt{\$p}\index{\texttt{\$p} statement}) may be specified.
If {\em label-match}
contains wildcard (\verb$*$) characters, all matching statements will be
displayed in the order they occur in the database.

Optional qualifiers (only one qualifier at a time is allowed):

    \texttt{/comment} - This qualifier includes the comment that immediately
       precedes the statement.

    \texttt{/full} - Show complete information about each statement,
       and show all
       statements matching {\em label} (including \texttt{\$e}
       and \texttt{\$f} statements).

    \texttt{/tex} - This qualifier will write the statement information to the
       \LaTeX\ file previously opened with \texttt{open tex}.  See
       Section~\ref{texout}.

    \texttt{/simple{\char`\_}tex} - The same as \texttt{/tex}, except that
       \LaTeX\ macros are not used for formatting equations, allowing easier
       manual edits of the output for slide presentations, etc.

    \texttt{/html}\index{html generation@{\sc html} generation},
       \texttt{/alt{\char`\_}html}, \texttt{/brief{\char`\_}html},
       \texttt{/brief{\char`\_}alt{\char`\_}html} -
       These qualifiers invoke a special mode of
       \texttt{show statement} that
       creates a web page for the statement.  They may not be used with
       any other qualifier.  See Section~\ref{htmlout} or
       \texttt{help html} in the program.


\subsection{\texttt{search} Command}\index{\texttt{search} command}
Syntax:  search {\em label-match}
\texttt{"}{\em symbol-match}{\tt}" [\texttt{/all}] [\texttt{/comments}]
[\texttt{/join}]

This command searches all \texttt{\$a} and \texttt{\$p} statements
matching {\em label-match} for occurrences of {\em symbol-match}.  A
\verb@*@ in {\em label-match} matches any label character.  A \verb@$*@
in {\em symbol-match} matches any sequence of symbols.  The symbols in
{\em symbol-match} must be separated by white space.  The quotes
surrounding {\em symbol-match} may be single or double quotes.  For
example, \texttt{search b}\verb@* "-> $* ch"@ will list all statements
whose labels begin with \texttt{b} and contain the symbols \verb@->@ and
\texttt{ch} surrounding any symbol sequence (including no symbol
sequence).  The wildcards \texttt{?} and \texttt{\$?} are also available
to match individual characters in labels and symbols respectively; see
\texttt{help search} in the Metamath program for details on their usage.

Optional command qualifiers:

    \texttt{/all} - Also search \texttt{\$e} and \texttt{\$f} statements.

    \texttt{/comments} - Search the comment that immediately precedes each
        label-matched statement for {\em symbol-match}.  In this case
        {\em symbol-match} is an arbitrary, non-case-sensitive character
        string.  Quotes around {\em symbol-match} are optional if there
        is no ambiguity.

    \texttt{/join} - In the case of a \texttt{\$a} or \texttt{\$p} statement,
	prepend its \texttt{\$e}
	hypotheses for searching. The
	\texttt{/join} qualifier has no effect in \texttt{/comments} mode.

\section{Displaying and Verifying Proofs}


\subsection{\texttt{show proof} Command}\index{\texttt{show proof} command}
Syntax:  \texttt{show proof} {\em label-match} [{\em qualifiers} (see below)]

This command displays the proof of the specified
\texttt{\$p}\index{\texttt{\$p} statement} statement in various formats.
The {\em label-match} may contain wildcard (\verb@$*@) characters to match
multiple statements.  Without any qualifiers, only the logical steps
will be shown (i.e.\ syntax construction steps will be omitted), in an
indented format.

Most of the time, you will use
    \texttt{show proof} {\em label}
to see just the proof steps corresponding to logical inferences.

Optional command qualifiers:

    \texttt{/essential} - The proof tree is trimmed of all
        \texttt{\$f}\index{\texttt{\$f} statement} hypotheses before
        being displayed.  (This is the default, and it is redundant to
        specify it.)

    \texttt{/all} - the proof tree is not trimmed of all \texttt{\$f} hypotheses before
        being displayed.  \texttt{/essential} and \texttt{/all} are mutually exclusive.

    \texttt{/from{\char`\_}step} {\em step} - The display starts at the specified
        step.  If
        this qualifier is omitted, the display starts at the first step.

    \texttt{/to{\char`\_}step} {\em step} - The display ends at the specified
        step.  If this
        qualifier is omitted, the display ends at the last step.

    \texttt{/tree{\char`\_}depth} {\em number} - Only
         steps at less than the specified proof
        tree depth are displayed.  Sometimes useful for obtaining an overview of
        the proof.

    \texttt{/reverse} - The steps are displayed in reverse order.

    \texttt{/renumber} - When used with \texttt{/essential}, the steps are renumbered
        to correspond only to the essential steps.

    \texttt{/tex} - The proof is converted to \LaTeX\ and\index{latex@{\LaTeX}}
        stored in the file opened
        with \texttt{open tex}.  See Section~\ref{texout} or
        \texttt{help tex} in the program.

    \texttt{/lemmon} - The proof is displayed in a non-indented format known
        as Lemmon style, with explicit previous step number references.
        If this qualifier is omitted, steps are indented in a tree format.

    \texttt{/start{\char`\_}column} {\em number} - Overrides the default column
        (16)
        at which the formula display starts in a Lemmon-style display.  May be
        used only in conjunction with \texttt{/lemmon}.

    \texttt{/normal} - The proof is displayed in normal format suitable for
        inclusion in a Metamath source file.  May not be used with any other
        qualifier.

    \texttt{/compressed} - The proof is displayed in compressed format
        suitable for inclusion in a Metamath source file.  May not be used with
        any other qualifier.

    \texttt{/statement{\char`\_}summary} - Summarizes all statements (like a
        brief \texttt{show statement})
        used by the proof.  It may not be used with any other qualifier
        except \texttt{/essential}.

    \texttt{/detailed{\char`\_}step} {\em step} - Shows the details of what is
        happening at
        a specific proof step.  May not be used with any other qualifier.
        The {\em step} is the step number shown when displaying a
        proof without the \texttt{/renumber} qualifier.


\subsection{\texttt{show usage} Command}\index{\texttt{show usage} command}
Syntax:  \texttt{show usage} {\em label-match} [\texttt{/recursive}]

This command lists the statements whose proofs make direct reference to
the statement specified.

Optional command qualifier:

    \texttt{/recursive} - Also include statements whose proofs ultimately
        depend on the statement specified.



\subsection{\texttt{show trace\_back} Command}\index{\texttt{show
       trace{\char`\_}back} command}
Syntax:  \texttt{show trace{\char`\_}back} {\em label-match} [\texttt{/essential}] [\texttt{/axioms}]
    [\texttt{/tree}] [\texttt{/depth} {\em number}]

This command lists all statements that the proof of the \texttt{\$p}
statement(s) specified by {\em label-match} depends on.

Optional command qualifiers:

    \texttt{/essential} - Restrict the trace-back to \texttt{\$e}
        \index{\texttt{\$e} statement} hypotheses of proof trees.

    \texttt{/axioms} - List only the axioms that the proof ultimately depends on.

    \texttt{/tree} - Display the trace-back in an indented tree format.

    \texttt{/depth} {\em number} - Restrict the \texttt{/tree} trace-back to the
        specified indentation depth.

    \texttt{/count{\char`\_}steps} - Count the number of steps the proof has
       all the way back to axioms.  If \texttt{/essential} is specified,
       expansions of variable-type hypotheses (syntax constructions) are not counted.

\subsection{\texttt{verify proof} Command}\index{\texttt{verify proof} command}
Syntax:  \texttt{verify proof} {\em label-match} [\texttt{/syntax{\char`\_}only}]

This command verifies the proofs of the specified statements.  {\em
label-match} may contain wild card characters (\texttt{*}) to verify
more than one proof; for example \verb/*abc*def/ will match all labels
containing \texttt{abc} and ending with \texttt{def}.
The command \texttt{verify proof *} will verify all proofs in the database.

Optional command qualifier:

    \texttt{/syntax{\char`\_}only} - This qualifier will perform a check of syntax
        and RPN
        stack violations only.  It will not verify that the proof is
        correct.  This qualifier is useful for quickly determining which
        proofs are incomplete (i.e.\ are under development and have \texttt{?}'s
        in them).

{\em Note:} \texttt{read}, followed by \texttt{verify proof *}, will ensure
 the database is free
from errors in the Metamath language but will not check the markup notation
in comments.
You can also check the markup notation by running \texttt{verify markup *}
as discussed in Section~\ref{verifymarkup}; see also the discussion
on generating {\sc HTML} in Section~\ref{htmlout}.

\subsection{\texttt{verify markup} Command}\index{\texttt{verify markup} command}\label{verifymarkup}
Syntax:  \texttt{verify markup} {\em label-match}
[\texttt{/date{\char`\_}skip}]
[\texttt{/top{\char`\_}date{\char`\_}skip}]
[\texttt{/file{\char`\_}skip}]
[\texttt{/verbose}]

This command checks comment markup and other informal conventions we have
adopted.  It error-checks the latexdef, htmldef, and althtmldef statements
in the \texttt{\$t} statement of a Metamath source file.\index{error checking}
It error-checks any \texttt{`}...\texttt{`},
\texttt{\char`\~}~\textit{label},
and bibliographic markups in statement descriptions.
It checks that
\texttt{\$p} and \texttt{\$a} statements
have the same content when their labels start with
``ax'' and ``ax-'' respectively but are otherwise identical, for example
ax4 and ax-4.
It also verifies the date consistency of ``(Contributed by...),''
``(Revised by...),'' and ``(Proof shortened by...)'' tags in the comment
above each \texttt{\$a} and \texttt{\$p} statement.

Optional command qualifiers:

    \texttt{/date{\char`\_}skip} - This qualifier will
        skip date consistency checking,
        which is usually not required for databases other than
	\texttt{set.mm}.

    \texttt{/top{\char`\_}date{\char`\_}skip} - This qualifier will check date consistency except
        that the version date at the top of the database file will not
        be checked.  Only one of
        \texttt{/date{\char`\_}skip} and
        \texttt{/top{\char`\_}date{\char`\_}skip} may be
        specified.

    \texttt{/file{\char`\_}skip} - This qualifier will skip checks that require
        external files to be present, such as checking GIF existence and
        bibliographic links to mmset.html or equivalent.  It is useful
        for doing a quick check from a directory without these files.

    \texttt{/verbose} - Provides more information.  Currently it provides a list
        of axXXX vs. ax-XXX matches.

\subsection{\texttt{save proof} Command}\index{\texttt{save proof} command}
Syntax:  \texttt{save proof} {\em label-match} [\texttt{/normal}]
   [\texttt{/compressed}]

The \texttt{save proof} command will reformat a proof in one of two formats and
replace the existing proof in the source buffer\index{source
buffer}.  It is useful for
converting between proof formats.  Note that a proof will not be
permanently saved until a \texttt{write source} command is issued.

Optional command qualifiers:

    \texttt{/normal} - The proof is saved in the normal format (i.e., as a
        sequence
        of labels, which is the defined format of the basic Metamath
        language).\index{basic language}  This is the default format that
        is used if a qualifier
        is omitted.

    \texttt{/compressed} - The proof is saved in the compressed format which
        reduces storage requirements for a database.
        See Appendix~\ref{compressed}.




\section{Creating Proofs}\label{pfcommands}\index{Proof Assistant}

Before using the Proof Assistant, you must add a \texttt{\$p} to your
source file (using a text editor) containing the statement you want to
prove.  Its proof should consist of a single \texttt{?}, meaning
``unknown step.''  Example:
\begin{verbatim}
equid $p x = x $= ? $.
\end{verbatim}

To enter the Proof assistant, type \texttt{prove} {\em label}, e.g.
\texttt{prove equid}.  Metamath will respond with the \texttt{MM-PA>}
prompt.

Proofs are created working backwards from the statement being proved,
primarily using a series of \texttt{assign} commands.  A proof is
complete when all steps are assigned to statements and all steps are
unified and completely known.  During the creation of a proof, Metamath
will allow only operations that are legal based on what is known up to
that point.  For example, it will not allow an \texttt{assign} of a
statement that cannot be unified with the unknown proof step being
assigned.

{\em Important:}
The Proof Assistant is
{\em not} a tool to help you discover proofs.  It is just a tool to help
you add them to the database.  For a tutorial read
Section~\ref{frstprf}.
To practice using the Proof Assistant, you may
want to \texttt{prove} an existing theorem, then delete all steps with
\texttt{delete all}, then re-create it with the Proof Assistant while
looking at its proof display (before deletion).
You might want to figure out your first few proofs completely
and write them down by hand, before using the Proof Assistant, though
not everyone finds that effective.

{\em Important:}
The \texttt{undo} command if very helpful when entering a proof, because
it allows you to undo a previously-entered step.
In addition, we suggest that you
keep track of your work with a log file (\texttt{open
log}) and save it frequently (\texttt{save new{\char`\_}proof},
\texttt{write source}).
You can use \texttt{delete} to reverse an \texttt{assign}.
You can also do \texttt{delete floating{\char`\_}hypotheses}, then
\texttt{initialize all}, then \texttt{unify all /interactive} to
reinitialize bad unifications made accidentally or by bad
\texttt{assign}s.  You cannot reverse a \texttt{delete} except by
a relevant \texttt{undo} or using
\texttt{exit /force} then reentering the Proof Assistant to recover from
the last \texttt{save new{\char`\_}proof}.

The following commands are available in the Proof Assistant (at the
\texttt{MM-PA>} prompt) to help you create your proof.  See the
individual commands for more detail.

\begin{itemize}
\item[]
    \texttt{show new{\char`\_}proof} [\texttt{/all},...] - Displays the
        proof in progress.  You will use this command a lot; see \texttt{help
        show new{\char`\_}proof} to become familiar with its qualifiers.  The
        qualifiers \texttt{/unknown} and \texttt{/not{\char`\_}unified} are
        useful for seeing the work remaining to be done.  The combination
        \texttt{/all/unknown} is useful for identifying dummy variables that must be
        assigned, or attempts to use illegal syntax, when \texttt{improve all}
        is unable to complete the syntax constructions.  Unknown variables are
        shown as \texttt{\$1}, \texttt{\$2},...
\item[]
    \texttt{assign} {\em step} {\em label} - Assigns an unknown {\em step}
        number with the statement
        specified by {\em label}.
\item[]
    \texttt{let variable} {\em variable}
        \texttt{= "}{\em symbol sequence}\texttt{"}
          - Forces a symbol
        sequence to replace an unknown variable (such as \texttt{\$1}) in a proof.
        It is useful
        for helping difficult unifications, and it is necessary when you have
        dummy variables that eventually must be assigned a name.
\item[]
    \texttt{let step} {\em step} \texttt{= "}{\em symbol sequence}\texttt{"} -
          Forces a symbol sequence
        to replace the contents of a proof step, provided it can be
        unified with the existing step contents.  (I rarely use this.)
\item[]
    \texttt{unify step} {\em step} (or \texttt{unify all}) - Unifies
        the source and target of
        a step.  If you specify a specific step, you will be prompted
        to select among the unifications that are possible.  If you
        specify \texttt{all}, all steps with unique unifications, but only
        those steps, will be
        unified.  \texttt{unify all /interactive} goes through all non-unified
        steps.
\item[]
    \texttt{initialize} {\em step} (or \texttt{all}) - De-unifies the target and source of
        a step (or all steps), as well as the hypotheses of the source,
        and makes all variables in the source unknown.  Useful to recover from
        an \texttt{assign} or \texttt{let} mistake that
        resulted in incorrect unifications.
\item[]
    \texttt{delete} {\em step} (or \texttt{all} or \texttt{floating{\char`\_}hypotheses}) -
        Deletes the specified
        step(s).  \texttt{delete floating{\char`\_}hypotheses}, then \texttt{initialize all}, then
        \texttt{unify all /interactive} is useful for recovering from mistakes
        where incorrect unifications assigned wrong math symbol strings to
        variables.
\item[]
    \texttt{improve} {\em step} (or \texttt{all}) -
      Automatically creates a proof for steps (with no unknown variables)
      whose proof requires no statements with \texttt{\$e} hypotheses.  Useful
      for filling in proofs of \texttt{\$f} hypotheses.  The \texttt{/depth}
      qualifier will also try statements whose \texttt{\$e} hypotheses contain
      no new variables.  {\em Warning:} Save your work (with \texttt{save
      new{\char`\_}proof} then \texttt{write source}) before using
      \texttt{/depth = 2} or greater, since the search time grows
      exponentially and may never terminate in a reasonable time, and you
      cannot interrupt the search.  I have found that it is rare for
      \texttt{/depth = 3} or greater to be useful.
 \item[]
    \texttt{save new{\char`\_}proof} - Saves the proof in progress in the program's
        internal database buffer.  To save it permanently into the database file,
        use \texttt{write source} after
        \texttt{save new{\char`\_}proof}.  To revert to the last
        \texttt{save new{\char`\_}proof},
        \texttt{exit /force} from the Proof Assistant then re-enter the Proof
        Assistant.
 \item[]
    \texttt{match step} {\em step} (or \texttt{match all}) - Shows what
        statements are
        possibilities for the \texttt{assign} statement. (This command
        is not very
        useful in its present form and hopefully will be improved
        eventually.  In the meantime, use the \texttt{search} statement for
        candidates matching specific math token combinations.)
 \item[]
 \texttt{minimize{\char`\_}with}\index{\texttt{minimize{\char`\_}with} command}
% 3/10/07 Note: line-breaking the above results in duplicate index entries
     - After a proof is complete, this command will attempt
        to match other database theorems to the proof to see if the proof
        size can be reduced as a result.  See \texttt{help
        minimize{\char`\_}with} in the
        Metamath program for its usage.
 \item[]
 \texttt{undo}\index{\texttt{undo} command}
    - Undo the effect of a proof-changing command (all but the
      \texttt{show} and \texttt{save} commands above).
 \item[]
 \texttt{redo}\index{\texttt{redo} command}
    - Reverse the previous \texttt{undo}.
\end{itemize}

The following commands set parameters that may be relevant to your proof.
Consult the individual \texttt{help set}... commands.
\begin{itemize}
   \item[] \texttt{set unification{\char`\_}timeout}
 \item[]
    \texttt{set search{\char`\_}limit}
  \item[]
    \texttt{set empty{\char`\_}substitution} - note that default is \texttt{off}
\end{itemize}

Type \texttt{exit} to exit the \texttt{MM-PA>}
 prompt and get back to the \texttt{MM>} prompt.
Another \texttt{exit} will then get you out of Metamath.



\subsection{\texttt{prove} Command}\index{\texttt{prove} command}
Syntax:  \texttt{prove} {\em label}

This command will enter the Proof Assistant\index{Proof Assistant}, which will
allow you to create or edit the proof of the specified statement.
The command-line prompt will change from \texttt{MM>} to \texttt{MM-PA>}.

Note:  In the present version of
Metamath\index{Metamath!limitations of version 0.07.30}, the Proof
Assistant does not verify that \texttt{\$d}\index{\texttt{\$d}
statement} restrictions are met as a proof is being built.  After you
have completed a proof, you should type \texttt{save new{\char`\_}proof}
followed by \texttt{verify proof} {\em label} (where {\em label} is the
statement you are proving with the \texttt{prove} command) to verify the
\texttt{\$d} restrictions.

See also: \texttt{exit}

\subsection{\texttt{set unification\_timeout} Command}\index{\texttt{set
unification{\char`\_}timeout} command}
Syntax:  \texttt{set unification{\char`\_}timeout} {\em number}

(This command is available outside the Proof Assistant but affects the
Proof Assistant\index{Proof Assistant} only.)

Sometimes the Proof Assistant will inform you that a unification
time-out occurred.  This may happen when you try to \texttt{unify}
formulas with many temporary variables\index{temporary variable}
(\texttt{\$1}, \texttt{\$2}, etc.), since the time to compute all possible
unifications may grow exponentially with the number of variables.  If
you want Metamath to try harder (and you're willing to wait longer) you
may increase this parameter.  \texttt{show settings} will show you the
current value.

\subsection{\texttt{set empty\_substitution} Command}\index{\texttt{set
empty{\char`\_}substitution} command}
% These long names can't break well in narrow mode, and even "sloppy"
% is not enough. Work around this by not demanding justification.
\begin{flushleft}
Syntax:  \texttt{set empty{\char`\_}substitution on} or \texttt{set
empty{\char`\_}substitution off}
\end{flushleft}

(This command is available outside the Proof Assistant but affects the
Proof Assistant\index{Proof Assistant} only.)

The Metamath language allows variables to be
substituted\index{substitution!variable}\index{variable substitution}
with empty symbol sequences\index{empty substitution}.  However, in many
formal systems\index{formal system} this will never happen in a valid
proof.  Allowing for this possibility increases the likelihood of
ambiguous unifications\index{ambiguous
unification}\index{unification!ambiguous} during proof creation.
The default is that
empty substitutions are not allowed; for formal systems requiring them,
you must \texttt{set empty{\char`\_}substitution on}.
(An example where this must be \texttt{on}
would be a system that implements a Deduction Rule and in
which deductions from empty assumption lists would be permissible.  The
MIU-system\index{MIU-system} described in Appendix~\ref{MIU} is another
example.)
Note that empty substitutions are
always permissible in proof verification (VERIFY PROOF...) outside the
Proof Assistant.  (See the MIU system in the Metamath book for an example
of a system needing empty substitutions; another example would be a
system that implements a Deduction Rule and in which deductions from
empty assumption lists would be permissible.)

It is better to leave this \texttt{off} when working with \texttt{set.mm}.
Note that this command does not affect the way proofs are verified with
the \texttt{verify proof} command.  Outside of the Proof Assistant,
substitution of empty sequences for math symbols is always allowed.

\subsection{\texttt{set search\_limit} Command}\index{\texttt{set
search{\char`\_}limit} command} Syntax:  \texttt{set search{\char`\_}limit} {\em
number}

(This command is available outside the Proof Assistant but affects the
Proof Assistant\index{Proof Assistant} only.)

This command sets a parameter that determines when the \texttt{improve} command
in Proof Assistant mode gives up.  If you want \texttt{improve} to search harder,
you may increase it.  The \texttt{show settings} command tells you its current
value.


\subsection{\texttt{show new\_proof} Command}\index{\texttt{show
new{\char`\_}proof} command}
Syntax:  \texttt{show new{\char`\_}proof} [{\em
qualifiers} (see below)]

This command (available only in Proof Assistant mode) displays the proof
in progress.  It is identical to the \texttt{show proof} command, except that
there is no statement argument (since it is the statement being proved) and
the following qualifiers are not available:

    \texttt{/statement{\char`\_}summary}

    \texttt{/detailed{\char`\_}step}

Also, the following additional qualifiers are available:

    \texttt{/unknown} - Shows only steps that have no statement assigned.

    \texttt{/not{\char`\_}unified} - Shows only steps that have not been unified.

Note that \texttt{/essential}, \texttt{/depth}, \texttt{/unknown}, and
\texttt{/not{\char`\_}unified} may be
used in any combination; each of them effectively filters out additional
steps from the proof display.

See also:  \texttt{show proof}






\subsection{\texttt{assign} Command}\index{\texttt{assign} command}
Syntax:   \texttt{assign} {\em step} {\em label} [\texttt{/no{\char`\_}unify}]

   and:   \texttt{assign first} {\em label}

   and:   \texttt{assign last} {\em label}


This command, available in the Proof Assistant only, assigns an unknown
step (one with \texttt{?} in the \texttt{show new{\char`\_}proof}
listing) with the statement specified by {\em label}.  The assignment
will not be allowed if the statement cannot be unified with the step.

If \texttt{last} is specified instead of {\em step} number, the last
step that is shown by \texttt{show new{\char`\_}proof /unknown} will be
used.  This can be useful for building a proof with a command file (see
\texttt{help submit}).  It also makes building proofs faster when you know
the assignment for the last step.

If \texttt{first} is specified instead of {\em step} number, the first
step that is shown by \texttt{show new{\char`\_}proof /unknown} will be
used.

If {\em step} is zero or negative, the -{\em step}th from last unknown
step, as shown by \texttt{show new{\char`\_}proof /unknown}, will be
used.  \texttt{assign -1} {\em label} will assign the penultimate
unknown step, \texttt{assign -2} {\em label} the antepenultimate, and
\texttt{assign 0} {\em label} is the same as \texttt{assign last} {\em
label}.

Optional command qualifier:

    \texttt{/no{\char`\_}unify} - do not prompt user to select a unification if there is
        more than one possibility.  This is useful for noninteractive
        command files.  Later, the user can \texttt{unify all /interactive}.
        (The assignment will still be automatically unified if there is only
        one possibility and will be refused if unification is not possible.)



\subsection{\texttt{match} Command}\index{\texttt{match} command}
Syntax:  \texttt{match step} {\em step} [\texttt{/max{\char`\_}essential{\char`\_}hyp}
{\em number}]

    and:  \texttt{match all} [\texttt{/essential}]
          [\texttt{/max{\char`\_}essential{\char`\_}hyp} {\em number}]

This command, available in the Proof Assistant only, shows what
statements can be unified with the specified step(s).  {\em Note:} In
its current form, this command is not very useful because of the large
number of matches it reports.  I rarely use it.
It may be enhanced in the future.

Optional command qualifiers:

    \texttt{/max{\char`\_}essential{\char`\_}hyp} {\em number} - filters out
        of the list any statements
        with more than the specified number of
        \texttt{\$e}\index{\texttt{\$e} statement} hypotheses

    \texttt{/essential{\char`\_}only} - in the \texttt{match all} statement, only
        the steps that
        would be listed in the \texttt{show new{\char`\_}proof /essential} display are
        matched.



\subsection{\texttt{let} Command}\index{\texttt{let} command}
Syntax: \texttt{let variable} {\em variable} = \verb/"/{\em symbol-sequence}\verb/"/

  and: \texttt{let step} {\em step} = \verb/"/{\em symbol-sequence}\verb/"/

These commands, available in the Proof Assistant\index{Proof Assistant}
only, assign a temporary variable\index{temporary variable} or unknown
step with a specific symbol sequence.  They are useful in the middle of
creating a proof, when you know what should be in the proof step but the
unification algorithm doesn't yet have enough information to completely
specify the temporary variables.  A ``temporary variable'' is one that
has the form \texttt{\$}{\em nn} in the proof display, such as
\texttt{\$1}, \texttt{\$2}, etc.  The {\em symbol-sequence} may contain
other unknown variables if desired.  Examples:

    \verb/let variable $32 = "A = B"/

    \verb/let variable $32 = "A = $35"/

    \verb/let step 10 = '|- x = x'/

    \verb/let step -2 = "|- ( $7 = ph )"/

Any symbol sequence will be accepted for the \texttt{let variable}
command.  Only those symbol sequences that can be unified with the step
will be accepted for \texttt{let step}.

The \texttt{let} commands ``zap'' the proof with information that can
only be verified when the proof is built up further.  If you make an
error, the command sequence \texttt{delete
floating{\char`\_}hypotheses}, \texttt{initialize all}, and
\texttt{unify all /interactive} will undo a bad \texttt{let} assignment.

If {\em step} is zero or negative, the -{\em step}th from last unknown
step, as shown by \texttt{show new{\char`\_}proof /unknown}, will be
used.  The command \texttt{let step 0} = \verb/"/{\em
symbol-sequence}\verb/"/ will use the last unknown step, \texttt{let
step -1} = \verb/"/{\em symbol-sequence}\verb/"/ the penultimate, etc.
If {\em step} is positive, \texttt{let step} may be used to assign known
(in the sense of having previously been assigned a label with
\texttt{assign}) as well as unknown steps.

Either single or double quotes can surround the {\em symbol-sequence} as
long as they are different from any quotes inside a {\em
symbol-sequence}.  If {\em symbol-sequence} contains both kinds of
quotes, see the instructions at the end of \texttt{help let} in the
Metamath program.


\subsection{\texttt{unify} Command}\index{\texttt{unify} command}
Syntax:  \texttt{unify step} {\em step}

      and:   \texttt{unify all} [\texttt{/interactive}]

These commands, available in the Proof Assistant only, unify the source
and target of the specified step(s). If you specify a specific step, you
will be prompted to select among the unifications that are possible.  If
you specify \texttt{all}, only those steps with unique unifications will be
unified.

Optional command qualifier for \texttt{unify all}:

    \texttt{/interactive} - You will be prompted to select among the
        unifications
        that are possible for any steps that do not have unique
        unifications.  (Otherwise \texttt{unify all} will bypass these.)

See also \texttt{set unification{\char`\_}timeout}.  The default is
100000, but increasing it to 1000000 can help difficult cases.  Manually
assigning some or all of the unknown variables with the \texttt{let
variable} command also helps difficult cases.



\subsection{\texttt{initialize} Command}\index{\texttt{initialize} command}
Syntax:  \texttt{initialize step} {\em step}

    and: \texttt{initialize all}

These commands, available in the Proof Assistant\index{Proof Assistant}
only, ``de-unify'' the target and source of a step (or all steps), as
well as the hypotheses of the source, and makes all variables in the
source and the source's hypotheses unknown.  This command is useful to
help recover from incorrect unifications that resulted from an incorrect
\texttt{assign}, \texttt{let}, or unification choice.  Part or all of
the command sequence \texttt{delete floating{\char`\_}hypotheses},
\texttt{initialize all}, and \texttt{unify all /interactive} will recover
from incorrect unifications.

See also:  \texttt{unify} and \texttt{delete}



\subsection{\texttt{delete} Command}\index{\texttt{delete} command}
Syntax:  \texttt{delete step} {\em step}

   and:      \texttt{delete all} -- {\em Warning: dangerous!}

   and:      \texttt{delete floating{\char`\_}hypotheses}

These commands are available in the Proof Assistant only.  The
\texttt{delete step} command deletes the proof tree section that
branches off of the specified step and makes the step become unknown.
\texttt{delete all} is equivalent to \texttt{delete step} {\em step}
where {\em step} is the last step in the proof (i.e.\ the beginning of
the proof tree).

In most cases the \texttt{undo} command is the best way to undo
a previous step.
An alternative is to salvage your last \texttt{save
new{\char`\_}proof} by exiting and reentering the Proof Assistant.
For this to work, keep a log file open to record your work
and to do \texttt{save new{\char`\_}proof} frequently, especially before
\texttt{delete}.

\texttt{delete floating{\char`\_}hypotheses} will delete all sections of
the proof that branch off of \texttt{\$f}\index{\texttt{\$f} statement}
statements.  It is sometimes useful to do this before an
\texttt{initialize} command to recover from an error.  Note that once a
proof step with a \texttt{\$f} hypothesis as the target is completely
known, the \texttt{improve} command can usually fill in the proof for
that step.  Unlike the deletion of logical steps, \texttt{delete
floating{\char`\_}hypotheses} is a relatively safe command that is
usually easy to recover from.



\subsection{\texttt{improve} Command}\index{\texttt{improve} command}
\label{improve}
Syntax:  \texttt{improve} {\em step} [\texttt{/depth} {\em number}]
                                               [\texttt{/no{\char`\_}distinct}]

   and:   \texttt{improve first} [\texttt{/depth} {\em number}]
                                              [\texttt{/no{\char`\_}distinct}]

   and:   \texttt{improve last} [\texttt{/depth} {\em number}]
                                              [\texttt{/no{\char`\_}distinct}]

   and:   \texttt{improve all} [\texttt{/depth} {\em number}]
                                              [\texttt{/no{\char`\_}distinct}]

These commands, available in the Proof Assistant\index{Proof Assistant}
only, try to find proofs automatically for unknown steps whose symbol
sequences are completely known.  They are primarily useful for filling in
proofs of \texttt{\$f}\index{\texttt{\$f} statement} hypotheses.  The
search will be restricted to statements having no
\texttt{\$e}\index{\texttt{\$e} statement} hypotheses.

\begin{sloppypar} % narrow
Note:  If memory is limited, \texttt{improve all} on a large proof may
overflow memory.  If you use \texttt{set unification{\char`\_}timeout 1}
before \texttt{improve all}, there will usually be sufficient
improvement to easily recover and completely \texttt{improve} the proof
later on a larger computer.  Warning:  Once memory has overflowed, there
is no recovery.  If in doubt, save the intermediate proof (\texttt{save
new{\char`\_}proof} then \texttt{write source}) before \texttt{improve
all}.
\end{sloppypar}

If \texttt{last} is specified instead of {\em step} number, the last
step that is shown by \texttt{show new{\char`\_}proof /unknown} will be
used.

If \texttt{first} is specified instead of {\em step} number, the first
step that is shown by \texttt{show new{\char`\_}proof /unknown} will be
used.

If {\em step} is zero or negative, the -{\em step}th from last unknown
step, as shown by \texttt{show new{\char`\_}proof /unknown}, will be
used.  \texttt{improve -1} will use the penultimate
unknown step, \texttt{improve -2} {\em label} the antepenultimate, and
\texttt{improve 0} is the same as \texttt{improve last}.

Optional command qualifier:

    \texttt{/depth} {\em number} - This qualifier will cause the search
        to include
        statements with \texttt{\$e} hypotheses (but no new variables in
        the \texttt{\$e}
        hypotheses), provided that the backtracking has not exceeded the
        specified depth. {\em Warning:}  Try \texttt{/depth 1},
        then \texttt{2}, then \texttt{3}, etc.
        in sequence because of possible exponential blowups.  Save your
        work before trying \texttt{/depth} greater than \texttt{1}!

    \texttt{/no{\char`\_}distinct} - Skip trial statements that have
        \texttt{\$d}\index{\texttt{\$d} statement} requirements.
        This qualifier will prevent assignments that might violate \texttt{\$d}
        requirements but it also could miss possible legal assignments.

See also: \texttt{set search{\char`\_}limit}

\subsection{\texttt{save new\_proof} Command}\index{\texttt{save
new{\char`\_}proof} command}
Syntax:  \texttt{save new{\char`\_}proof} {\em label} [\texttt{/normal}]
   [\texttt{/compressed}]

The \texttt{save new{\char`\_}proof} command is available in the Proof
Assistant only.  It saves the proof in progress in the source
buffer\index{source buffer}.  \texttt{save new{\char`\_}proof} may be
used to save a completed proof, or it may be used to save a proof in
progress in order to work on it later.  If an incomplete proof is saved,
any user assignments with \texttt{let step} or \texttt{let variable}
will be lost, as will any ambiguous unifications\index{ambiguous
unification}\index{unification!ambiguous} that were resolved manually.
To help make recovery easier, it can be helpful to \texttt{improve all}
before \texttt{save new{\char`\_}proof} so that the incomplete proof
will have as much information as possible.

Note that the proof will not be permanently saved until a \texttt{write
source} command is issued.

Optional command qualifiers:

    \texttt{/normal} - The proof is saved in the normal format (i.e., as a
        sequence of labels, which is the defined format of the basic Metamath
        language).\index{basic language}  This is the default format that
        is used if a qualifier is omitted.

    \texttt{/compressed} - The proof is saved in the compressed format, which
        reduces storage requirements for a database.
        (See Appendix~\ref{compressed}.)


\section{Creating \LaTeX\ Output}\label{texout}\index{latex@{\LaTeX}}

You can generate \LaTeX\ output given the
information in a database.
The database must already include the necessary typesetting information
(see section \ref{tcomment} for how to provide this information).

The \texttt{show statement} and \texttt{show proof} commands each have a
special \texttt{/tex} command qualifier that produces \LaTeX\ output.
(The \texttt{show statement} command also has the
\texttt{/simple{\char`\_}tex} qualifier for output that is easier to
edit by hand.)  Before you can use them, you must open a \LaTeX\ file to
which to send their output.  A typical complete session will use this
sequence of Metamath commands:

\begin{verbatim}
read set.mm
open tex example.tex
show statement a1i /tex
show proof a1i /all/lemmon/renumber/tex
show statement uneq2 /tex
show proof uneq2 /all/lemmon/renumber/tex
close tex
\end{verbatim}

See Section~\ref{mathcomments} for information on comment markup and
Appendix~\ref{ASCII} for information on how math symbol translation is
specified.

To format and print the \LaTeX\ source, you will need the \LaTeX\
program, which is standard on most Linux installations and available for
Windows.  On Linux, in order to create a {\sc pdf} file, you will
typically type at the shell prompt
\begin{verbatim}
$ pdflatex example.tex
\end{verbatim}

\subsection{\texttt{open tex} Command}\index{\texttt{open tex} command}
Syntax:  \texttt{open tex} {\em file-name} [\texttt{/no{\char`\_}header}]

This command opens a file for writing \LaTeX\
source\index{latex@{\LaTeX}} and writes a \LaTeX\ header to the file.
\LaTeX\ source can be written with the \texttt{show proof}, \texttt{show
new{\char`\_}proof}, and \texttt{show statement} commands using the
\texttt{/tex} qualifier.

The mapping to \LaTeX\ symbols is defined in a special comment
containing a \texttt{\$t} token, described in Appendix~\ref{ASCII}.

There is an optional command qualifier:

    \texttt{/no{\char`\_}header} - This qualifier prevents a standard
        \LaTeX\ header and trailer
        from being included with the output \LaTeX\ code.


\subsection{\texttt{close tex} Command}\index{\texttt{close tex} command}
Syntax:  \texttt{close tex}

This command writes a trailer to any \LaTeX\ file\index{latex@{\LaTeX}}
that was opened with \texttt{open tex} (unless
\texttt{/no{\char`\_}header} was used with \texttt{open tex}) and closes
the \LaTeX\ file.


\section{Creating {\sc HTML} Output}\label{htmlout}

You can generate {\sc html} web pages given the
information in a database.
The database must already include the necessary typesetting information
(see section \ref{tcomment} for how to provide this information).
The ability to produce {\sc html} web pages was added in Metamath version
0.07.30.

To create an {\sc html} output file(s) for \texttt{\$a} or \texttt{\$p}
statement(s), use
\begin{quote}
    \texttt{show statement} {\em label-match} \texttt{/html}
\end{quote}
The output file will be named {\em label-match}\texttt{.html}
for each match.  When {\em
label-match} has wildcard (\texttt{*}) characters, all statements with
matching labels will have {\sc html} files produced for them.  Also,
when {\em label-match} has a wildcard (\texttt{*}) character, two additional
files, \texttt{mmdefinitions.html} and \texttt{mmascii.html} will be
produced.  To produce {\em only} these two additional files, you can use
\texttt{?*}, which will not match any statement label, in place of {\em
label-match}.

There are three other qualifiers for \texttt{show statement} that also
generate {\sc HTML} code.  These are \texttt{/alt{\char`\_}html},
\texttt{/brief{\char`\_}html}, and
\texttt{/brief{\char`\_}alt{\char`\_}html}, and are described in the
next section.

The command
\begin{quote}
    \texttt{show statement} {\em label-match} \texttt{/alt{\char`\_}html}
\end{quote}
does the same as \texttt{show statement} {\em label-match} \texttt{/html},
except that the {\sc html} code for the symbols is taken from
\texttt{althtmldef} statements instead of \texttt{htmldef} statements in
the \texttt{\$t} comment.

The command
\begin{verbatim}
show statement * /brief_html
\end{verbatim}
invokes a special mode that just produces definition and theorem lists
accompanied by their symbol strings, in a format suitable for copying and
pasting into another web page (such as the tutorial pages on the
Metamath web site).

Finally, the command
\begin{verbatim}
show statement * /brief_alt_html
\end{verbatim}
does the same as \texttt{show statement * / brief{\char`\_}html}
for the alternate {\sc html}
symbol representation.

A statement's comment can include a special notation that provides a
certain amount of control over the {\sc HTML} version of the comment.  See
Section~\ref{mathcomments} (p.~\pageref{mathcomments}) for the comment
markup features.

The \texttt{write theorem{\char`\_}list} and \texttt{write bibliography}
commands, which are described below, provide as a side effect complete
error checking for all of the features described in this
section.\index{error checking}

\subsection{\texttt{write theorem\_list}
Command}\index{\texttt{write theorem{\char`\_}list} command}
Syntax:  \texttt{write theorem{\char`\_}list}
[\texttt{/theorems{\char`\_}per{\char`\_}page} {\em number}]

This command writes a list of all of the \texttt{\$a} and \texttt{\$p}
statements in the database into a web page file
 called \texttt{mmtheorems.html}.
When additional files are needed, they are called
\texttt{mmtheorems2.html}, \texttt{mmtheorems3.html}, etc.

Optional command qualifier:

    \texttt{/theorems{\char`\_}per{\char`\_}page} {\em number} -
 This qualifier specifies the number of statements to
        write per web page.  The default is 100.

{\em Note:} In version 0.07.30\index{Metamath!limitations of version
0.07.30} of Metamath, the ``Nearby theorems'' links on the individual
web pages presuppose 100 theorems per page when linking to the theorem
list pages.  Therefore the \texttt{/theorems{\char`\_}per{\char`\_}page}
qualifier, if it specifies a number other than 100, will cause the
individual web pages to be out of sync and should not be used to
generate the main theorem list for the web site.  This is expected to be
fixed in a future version.


\subsection{\texttt{write bibliography}\label{wrbib}
Command}\index{\texttt{write bibliography} command}
Syntax:  \texttt{write bibliography} {\em filename}

This command reads an existing {\sc html} bibliographic cross-reference
file, normally called \texttt{mmbiblio.html}, and updates it per the
bibliographic links in the database comments.  The file is updated
between the {\sc html} comment lines \texttt{<!--
{\char`\#}START{\char`\#} -->} and \texttt{<!-- {\char`\#}END{\char`\#}
-->}.  The original input file is renamed to {\em
filename}\texttt{{\char`\~}1}.

A bibliographic reference is indicated with the reference name
in brackets, such as  \texttt{Theorem 3.1 of
[Monk] p.\ 22}.
See Section~\ref{htmlout} (p.~\pageref{htmlout}) for
syntax details.


\subsection{\texttt{write recent\_additions}
Command}\index{\texttt{write recent{\char`\_}additions} command}
Syntax:  \texttt{write recent{\char`\_}additions} {\em filename}
[\texttt{/limit} {\em number}]

This command reads an existing ``Recent Additions'' {\sc html} file,
normally called \texttt{mmrecent.html}, and updates it with the
descriptions of the most recently added theorems to the database.
 The file is updated between
the {\sc html} comment lines \texttt{<!-- {\char`\#}START{\char`\#} -->}
and \texttt{<!-- {\char`\#}END{\char`\#} -->}.  The original input file
is renamed to {\em filename}\texttt{{\char`\~}1}.

Optional command qualifier:

    \texttt{/limit} {\em number} -
 This qualifier specifies the number of most recent theorems to
   write to the output file.  The default is 100.


\section{Text File Utilities}

\subsection{\texttt{tools} Command}\index{\texttt{tools} command}
Syntax:  \texttt{tools}

This command invokes an easy-to-use, general purpose utility for
manipulating the contents of {\sc ascii} text files.  Upon typing
\texttt{tools}, the command-line prompt will change to \texttt{TOOLS>}
until you type \texttt{exit}.  The \texttt{tools} commands can be used
to perform simple, global edits on an input/output file,
such as making a character string substitution on each line, adding a
string to each line, and so on.  A typical use of this utility is
to build a \texttt{submit} input file to perform a common operation on a
list of statements obtained from \texttt{show label} or \texttt{show
usage}.

The actions of most of the \texttt{tools} commands can also be
performed with equivalent (and more powerful) Unix shell commands, and
some users may find those more efficient.  But for Windows users or
users not comfortable with Unix, \texttt{tools} provides an
easy-to-learn alternative that is adequate for most of the
script-building tasks needed to use the Metamath program effectively.

\subsection{\texttt{help} Command (in \texttt{tools})}
Syntax:  \texttt{help}

The \texttt{help} command lists the commands available in the
\texttt{tools} utility, along with a brief description.  Each command,
in turn, has its own help, such as \texttt{help add}.  As with
Metamath's \texttt{MM>} prompt, a complete command can be entered at
once, or just the command word can be typed, causing the program to
prompt for each argument.

\vskip 1ex
\noindent Line-by-line editing commands:

  \texttt{add} - Add a specified string to each line in a file.

  \texttt{clean} - Trim spaces and tabs on each line in a file; convert
         characters.

  \texttt{delete} - Delete a section of each line in a file.

  \texttt{insert} - Insert a string at a specified column in each line of
        a file.

  \texttt{substitute} - Make a simple substitution on each line of the file.

  \texttt{tag} - Like \texttt{add}, but restricted to a range of lines.

  \texttt{swap} - Swap the two halves of each line in a file.

\vskip 1ex
\noindent Other file-processing commands:

  \texttt{break} - Break up (tokenize) a file into a list of tokens (one per
        line).

  \texttt{build} - Build a file with multiple tokens per line from a list.

  \texttt{count} - Count the occurrences in a file of a specified string.

  \texttt{number} - Create a list of numbers.

  \texttt{parallel} - Put two files in parallel.

  \texttt{reverse} - Reverse the order of the lines in a file.

  \texttt{right} - Right-justify lines in a file (useful before sorting
         numbers).

%  \texttt{tag} - Tag edit updates in a program for revision control.

  \texttt{sort} - Sort the lines in a file with key starting at
         specified string.

  \texttt{match} - Extract lines containing (or not) a specified string.

  \texttt{unduplicate} - Eliminate duplicate occurrences of lines in a file.

  \texttt{duplicate} - Extract first occurrence of any line occurring
         more than

   \ \ \    once in a file, discarding lines occurring exactly once.

  \texttt{unique} - Extract lines occurring exactly once in a file.

  \texttt{type} (10 lines) - Display the first few lines in a file.
                  Similar to Unix \texttt{head}.

  \texttt{copy} - Similar to Unix \texttt{cat} but safe (same input
         and output file allowed).

  \texttt{submit} - Run a script containing \texttt{tools} commands.

\vskip 1ex

\noindent Note:
  \texttt{unduplicate}, \texttt{duplicate}, and \texttt{unique} also
 sort the lines as a side effect.


\subsection{Using \texttt{tools} to Build Metamath \texttt{submit}
Scripts}

The \texttt{break} command is typically used to break up a series of
statement labels, such as the output of Metamath's \texttt{show usage},
into one label per line.  The other \texttt{tools} commands can then be
used to add strings before and after each statement label to specify
commands to be performed on the statement.  The \texttt{parallel}
command is useful when a statement label must be mentioned more than
once on a line.

Very often a \texttt{submit} script for Metamath will require multiple
command lines for each statement being processed.  For example, you may
want to enter the Proof Assistant, \texttt{minimize{\char`\_}with} your
latest theorem, \texttt{save} the new proof, and \texttt{exit} the Proof
Assistant.  To accomplish this, you can build a file with these four
commands for each statement on a single line, separating each command
with a designated character such as \texttt{@}.  Then at the end you can
\texttt{substitute} each \texttt{@} with \texttt{{\char`\\}n} to break
up the lines into individual command lines (see \texttt{help
substitute}).


\subsection{Example of a \texttt{tools} Session}

To give you a quick feel for the \texttt{tools} utility, we show a
simple session where we create a file \texttt{n.txt} with 3 lines, add
strings before and after each line, and display the lines on the screen.
You can experiment with the various commands to gain experience with the
\texttt{tools} utility.

\begin{verbatim}
MM> tools
Entering the Text Tools utilities.
Type HELP for help, EXIT to exit.
TOOLS> number
Output file <n.tmp>? n.txt
First number <1>?
Last number <10>? 3
Increment <1>?
TOOLS> add
Input/output file? n.txt
String to add to beginning of each line <>? This is line
String to add to end of each line <>? .
The file n.txt has 3 lines; 3 were changed.
First change is on line 1:
This is line 1.
TOOLS> type n.txt
This is line 1.
This is line 2.
This is line 3.
TOOLS> exit
Exiting the Text Tools.
Type EXIT again to exit Metamath.
MM>
\end{verbatim}



\appendix
\chapter{Sample Representations}
\label{ASCII}

This Appendix provides a sample of {\sc ASCII} representations,
their corresponding traditional mathematical symbols,
and a discussion of their meanings
in the \texttt{set.mm} database.
These are provided in order of appearance.
This is only a partial list, and new definitions are routinely added.
A complete list is available at \url{http://metamath.org}.

These {\sc ASCII} representations, along
with information on how to display them,
are defined in the \texttt{set.mm} database file inside
a special comment called a \texttt{\$t} {\em
comment}\index{\texttt{\$t} comment} or {\em typesetting
comment.}\index{typesetting comment}
A typesetting comment
is indicated by the appearance of the
two-character string \texttt{\$t} at the beginning of the comment.
For more information,
see Section~\ref{tcomment}, p.~\pageref{tcomment}.

In the following table the ``{\sc ASCII}'' column shows the {\sc ASCII}
representation,
``Symbol'' shows the mathematical symbolic display
that corresponds to that {\sc ASCII} representation, ``Labels'' shows
the key label(s) that define the representation, and
``Description'' provides a description about the symbol.
As usual, ``iff'' is short for ``if and only if.''\index{iff}
In most cases the ``{\sc ASCII}'' column only shows
the key token, but it will sometimes show a sequence of tokens
if that is necessary for clarity.

{\setlength{\extrarowsep}{4pt} % Keep rows from being too close together
\begin{longtabu}   { @{} c c l X }
\textbf{ASCII} & \textbf{Symbol} & \textbf{Labels} & \textbf{Description} \\
\endhead
\texttt{|-} & $\vdash$ & &
  ``It is provable that...'' \\
\texttt{ph} & $\varphi$ & \texttt{wph} &
  The wff (boolean) variable phi,
  conventionally the first wff variable. \\
\texttt{ps} & $\psi$ & \texttt{wps} &
  The wff (boolean) variable psi,
  conventionally the second wff variable. \\
\texttt{ch} & $\chi$ & \texttt{wch} &
  The wff (boolean) variable chi,
  conventionally the third wff variable. \\
\texttt{-.} & $\lnot$ & \texttt{wn} &
  Logical not. E.g., if $\varphi$ is true, then $\lnot \varphi$ is false. \\
\texttt{->} & $\rightarrow$ & \texttt{wi} &
  Implies, also known as material implication.
  In classical logic the expression $\varphi \rightarrow \psi$ is true
  if either $\varphi$ is false or $\psi$ is true (or both), that is,
  $\varphi \rightarrow \psi$ has the same meaning as
  $\lnot \varphi \lor \psi$ (as proven in theorem \texttt{imor}). \\
\texttt{<->} & $\leftrightarrow$ &
  \hyperref[df-bi]{\texttt{df-bi}} &
  Biconditional (aka is-equals for boolean values).
  $\varphi \leftrightarrow \psi$ is true iff
  $\varphi$ and $\psi$ have the same value. \\
\texttt{\char`\\/} & $\lor$ &
  \makecell[tl]{{\hyperref[df-or]{\texttt{df-or}}}, \\
	         \hyperref[df-3or]{\texttt{df-3or}}} &
  Disjunction (logical ``or''). $\varphi \lor \psi$ is true iff
  $\varphi$, $\psi$, or both are true. \\
\texttt{/\char`\\} & $\land$ &
  \makecell[tl]{{\hyperref[df-an]{\texttt{df-an}}}, \\
                 \hyperref[df-3an]{\texttt{df-3an}}} &
  Conjunction (logical ``and''). $\varphi \land \psi$ is true iff
  both $\varphi$ and $\psi$ are true. \\
\texttt{A.} & $\forall$ &
  \texttt{wal} &
  For all; the wff $\forall x \varphi$ is true iff
  $\varphi$ is true for all values of $x$. \\
\texttt{E.} & $\exists$ &
  \hyperref[df-ex]{\texttt{df-ex}} &
  There exists; the wff
  $\exists x \varphi$ is true iff
  there is at least one $x$ where $\varphi$ is true. \\
\texttt{[ y / x ]} & $[ y / x ]$ &
  \hyperref[df-sb]{\texttt{df-sb}} &
  The wff $[ y / x ] \varphi$ produces
  the result when $y$ is properly substituted for $x$ in $\varphi$
  ($y$ replaces $x$).
  % This is elsb4
  % ( [ x / y ] z e. y <-> z e. x )
  For example,
  $[ x / y ] z \in y$ is the same as $z \in x$. \\
\texttt{E!} & $\exists !$ &
  \hyperref[df-eu]{\texttt{df-eu}} &
  There exists exactly one;
  $\exists ! x \varphi$ is true iff
  there is at least one $x$ where $\varphi$ is true. \\
\texttt{\{ y | phi \}}  & $ \{ y | \varphi \}$ &
  \hyperref[df-clab]{\texttt{df-clab}} &
  The class of all sets where $\varphi$ is true. \\
\texttt{=} & $ = $ &
  \hyperref[df-cleq]{\texttt{df-cleq}} &
  Class equality; $A = B$ iff $A$ equals $B$. \\
\texttt{e.} & $ \in $ &
  \hyperref[df-clel]{\texttt{df-clel}} &
  Class membership; $A \in B$ if $A$ is a member of $B$. \\
\texttt{{\char`\_}V} & {\rm V} &
  \hyperref[df-v]{\texttt{df-v}} &
  Class of all sets (not itself a set). \\
\texttt{C\_} & $ \subseteq $ &
  \hyperref[df-ss]{\texttt{df-ss}} &
  Subclass (subset); $A \subseteq B$ is true iff
  $A$ is a subclass of $B$. \\
\texttt{u.} & $ \cup $ &
  \hyperref[df-un]{\texttt{df-un}} &
  $A \cup B$ is the union of classes $A$ and $B$. \\
\texttt{i^i} & $ \cap $ &
  \hyperref[df-in]{\texttt{df-in}} &
  $A \cap B$ is the intersection of classes $A$ and $B$. \\
\texttt{\char`\\} & $ \setminus $ &
  \hyperref[df-dif]{\texttt{df-dif}} &
  $A \setminus B$ (set difference)
  is the class of all sets in $A$ except for those in $B$. \\
\texttt{(/)} & $ \varnothing $ &
  \hyperref[df-nul]{\texttt{df-nul}} &
  $ \varnothing $ is the empty set (aka null set). \\
\texttt{\char`\~P} & $ \cal P $ &
  \hyperref[df-pw]{\texttt{df-pw}} &
  Power class. \\
\texttt{<.\ A , B >.} & $\langle A , B \rangle$ &
  \hyperref[df-op]{\texttt{df-op}} &
  The ordered pair $\langle A , B \rangle$. \\
\texttt{( F ` A )} & $ ( F ` A ) $ &
  \hyperref[df-fv]{\texttt{df-fv}} &
  The value of function $F$ when applied to $A$. \\
\texttt{\_i} & $ i $ &
  \texttt{df-i} &
  The square root of negative one. \\
\texttt{x.} & $ \cdot $ &
  \texttt{df-mul} &
  Complex number multiplication; $2~\cdot~3~=~6$. \\
\texttt{CC} & $ \mathbb{C} $ &
  \texttt{df-c} &
  The set of complex numbers. \\
\texttt{RR} & $ \mathbb{R} $ &
  \texttt{df-r} &
  The set of real numbers. \\
\end{longtabu}
} % end of extrarowsep

\chapter{Compressed Proofs}
\label{compressed}\index{compressed proof}\index{proof!compressed}

The proofs in the \texttt{set.mm} set theory database are stored in compressed
format for efficiency.  Normally you needn't concern yourself with the
compressed format, since you can display it with the usual proof display tools
in the Metamath program (\texttt{show proof}\ldots) or convert it to the normal
RPN proof format described in Section~\ref{proof} (with \texttt{save proof}
{\em label} \texttt{/normal}).  However for sake of completeness we describe the
format here and show how it maps to the normal RPN proof format.

A compressed proof, located between \texttt{\$=} and \texttt{\$.}\ keywords, consists
of a left parenthesis, a sequence of statement labels, a right parenthesis,
and a sequence of upper-case letters \texttt{A} through \texttt{Z} (with optional
white space between them).  White space must surround the parentheses
and the labels.  The left parenthesis tells Metamath that a
compressed proof follows.  (A normal RPN proof consists of just a sequence of
labels, and a parenthesis is not a legal character in a label.)

The sequence of upper-case letters corresponds to a sequence of integers
with the following mapping.  Each integer corresponds to a proof step as
described later.
\begin{center}
  \texttt{A} = 1 \\
  \texttt{B} = 2 \\
   \ldots \\
  \texttt{T} = 20 \\
  \texttt{UA} = 21 \\
  \texttt{UB} = 22 \\
   \ldots \\
  \texttt{UT} = 40 \\
  \texttt{VA} = 41 \\
  \texttt{VB} = 42 \\
   \ldots \\
  \texttt{YT} = 120 \\
  \texttt{UUA} = 121 \\
   \ldots \\
  \texttt{YYT} = 620 \\
  \texttt{UUUA} = 621 \\
   etc.
\end{center}

In other words, \texttt{A} through \texttt{T} represent the
least-significant digit in base 20, and \texttt{U} through \texttt{Y}
represent zero or more most-significant digits in base 5, where the
digits start counting at 1 instead of the usual 0. With this scheme, we
don't need white space between these ``numbers.''

(In the design of the compressed proof format, only upper-case letters,
as opposed to say all non-whitespace printable {\sc ascii} characters other than
%\texttt{\$}, was chosen to make the compressed proof a little less
%displeasing to the eye, at the expense of a typical 20\% compression
\texttt{\$}, were chosen so as not to collide with most text editor
searches, at the expense of a typical 20\% compression
loss.  The base 5/base 20 grouping, as opposed to say base 6/base 19,
was chosen by experimentally determining the grouping that resulted in
best typical compression.)

The letter \texttt{Z} identifies (tags) a proof step that is identical to one
that occurs later on in the proof; it helps shorten the proof by not requiring
that identical proof steps be proved over and over again (which happens often
when building wff's).  The \texttt{Z} is placed immediately after the
least-significant digit (letters \texttt{A} through \texttt{T}) that ends the integer
corresponding to the step to later be referenced.

The integers that the upper-case letters correspond to are mapped to labels as
follows.  If the statement being proved has $m$ mandatory hypotheses, integers
1 through $m$ correspond to the labels of these hypotheses in the order shown
by the \texttt{show statement ... / full} command, i.e., the RPN order\index{RPN
order} of the mandatory
hypotheses.  Integers $m+1$ through $m+n$ correspond to the labels enclosed in
the parentheses of the compressed proof, in the order that they appear, where
$n$ is the number of those labels.  Integers $m+n+1$ on up don't directly
correspond to statement labels but point to proof steps identified with the
letter \texttt{Z}, so that these proof steps can be referenced later in the
proof.  Integer $m+n+1$ corresponds to the first step tagged with a \texttt{Z},
$m+n+2$ to the second step tagged with a \texttt{Z}, etc.  When the compressed
proof is converted to a normal proof, the entire subproof of a step tagged
with \texttt{Z} replaces the reference to that step.

For efficiency, Metamath works with compressed proofs directly, without
converting them internally to normal proofs.  In addition to the usual
error-checking, an error message is given if (1) a label in the label list in
parentheses does not refer to a previous \texttt{\$p} or \texttt{\$a} statement or a
non-mandatory hypothesis of the statement being proved and (2) a proof step
tagged with \texttt{Z} is referenced before the step tagged with the \texttt{Z}.

Just as in a normal proof under development (Section~\ref{unknown}), any step
or subproof that is not yet known may be represented with a single \texttt{?}.
White space does not have to appear between the \texttt{?}\ and the upper-case
letters (or other \texttt{?}'s) representing the remainder of the proof.

% April 1, 2004 Appendix C has been added back in with corrections.
%
% May 20, 2003 Appendix C was removed for now because there was a problem found
% by Bob Solovay
%
% Also, removed earlier \ref{formalspec} 's (3 cases above)
%
% Bob Solovay wrote on 30 Nov 2002:
%%%%%%%%%%%%% (start of email comment )
%      3. My next set of comments concern appendix C. I read this before I
% read Chapter 4. So I first noted that the system as presented in the
% Appendix lacked a certain formal property that I thought desirable. I
% then came up with a revised formal system that had this property. Upon
% reading Chapter 4, I noticed that the revised system was closer to the
% treatment in Chapter 4 than the system in Appendix C.
%
%         First a very minor correction:
%
%         On page 142 line 2: The condition that V(e) != V(f) should only be
% required of e, f in T such that e != f.
%
%         Here is a natural property [transitivity] that one would like
% the formal system to have:
%
%         Let Gamma be a set of statements. Suppose that the statement Phi
% is provable from Gamma and that the statement Psi is provable from Gamma
% \cup {Phi}. Then Psi is provable from Gamma.
%
%         I shall present an example to show that this property does not
% hold for the formal systems of Appendix C:
%
%         I write the example in metamath style:
%
% $c A B C D E $.
% $v x y
%
% ${
% tx $f A x $.
% ty $f B y $.
% ax1 $a C x y $.
% $}
%
% ${
% tx $f A x $.
% ty $f B y $.
% ax2-h1 $e C x y $.
% ax2 $a D y $.
% $}
%
% ${
% ty $f B y $.
% ax3-h1 $e D y $.
% ax3 $a E y $.
% $}
%
% $(These three axioms are Gamma $)
%
% ${
% tx $f A x $.
% ty $f B y $.
% Phi $p D y $=
% tx ty tx ty ax1 ax2 $.
% $}
%
% ${
% ty $f B y $.
% Psi $p E y $=
% ty ty Phi ax3 $.
% $}
%
%
% I omit the formal proofs of the following claims. [I will be glad to
% supply them upon request.]
%
% 1) Psi is not provable from Gamma;
%
% 2) Psi is provable from Gamma + Phi.
%
% Here "provable" refers to the formalism of Appendix C.
%
% The trouble of course is that Psi is lacking the variable declaration
%
% $f Ax $.
%
% In the Metamath system there is no trouble proving Psi. I attach a
% metamath file that shows this and which has been checked by the
% metamath program.
%
% I next want to indicate how I think the treatment in Appendix C should
% be revised so as to conform more closely to the metamath system of the
% main text. The revised system *does* have the transitivity property.
%
% We want to give revised definitions of "statement" and
% "provable". [cf. sections C.2.4. and C.2.5] Our new definitions will
% use the definitions given in Appendix C. So we take the following
% tack. We refer to the original notions as o-statement and o-provable. And
% we refer to the notions we are defining as n-statement and n-provable.
%
%         A n-statement is an o-statement in which the only variables
% that appear in the T component are mandatory.
%
%         To any o-statement we can associate its reduct which is a
% n-statement by dropping all the elements of T or D which contain
% non-mandatory variables.
%
%         An n-statement gamma is n-provable if there is an o-statement
% gamma' which has gamma as its reduct andf such that gamma' is
% o-provable.
%
%         It seems to me [though I am not completely sure on this point]
% that n-provability corresponds to metamath provability as discussed
% say in Chapter 4.
%
%         Attached to this letter is the metamath proof of Phi and Psi
% from Gamma discussed above.
%
%         I am still brooding over the question of whether metamath
% correctly formalizes set-theory. No doubt I will have some questions
% re this after my thoughts become clearer.
%%%%%%%%%%%%%%%% (end of email comment)

%%%%%%%%%%%%%%%% (start of 2nd email comment from Bob Solovay 1-Apr-04)
%
%         I hope that Appendix C is the one that gives a "formal" treatment
% of Metamath. At any rate, thats the appendix I want to comment on.
%
%         I'm going to suggest two changes in the definition.
%
%         First change (in the definition of statement): Require that the
% sets D, T, and E be finite.
%
%         Probably things are fine as you give them. But in the applications
% to the main metamath system they will always be finite, and its useful in
% thinking about things [at least for me] to stick to the finite case.
%
%         Second change:
%
%         First let me give an approximate description. Remove the dummy
% variables from the statement. Instead, include them in the proof.
%
%         More formally: Require that T consists of type declarations only
% for mandatory variables. Require that all the pairs in D consist of
% mandatory variables.
%
%         At the start of a proof we are allowed to declare a finite number
% of dummy variables [provided that none of them appear in any of the
% statements in E \cup {A}. We have to supply type declarations for all the
% dummy variables. We are allowed to add new $d statements referring to
% either the mandatory or dummy variables. But we require that no new $d
% statement references only mandatory variables.
%
%         I find this way of doing things more conceptual than the treatment
% in Appendix C. But the change [which I will use implicitly in later
% letters about doing Peano] is mainly aesthetic. I definitely claim that my
% results on doing Peano all apply to Metamath as it is presented in your
% book.
%
%         --Bob
%
%%%%%%%%%%%%%%%% (end of 2nd email comment)

%%
%% When uncommenting the below, also uncomment references above to {formalspec}
%%
\chapter{Metamath's Formal System}\label{formalspec}\index{Metamath!as a formal
system}

\section{Introduction}

\begin{quote}
  {\em Perfection is when there is no longer anything more to take away.}
    \flushright\sc Antoine de
     Saint-Exupery\footnote{\cite[p.~3-25]{Campbell}.}\\
\end{quote}\index{de Saint-Exupery, Antoine}

This appendix describes the theory behind the Metamath language in an abstract
way intended for mathematicians.  Specifically, we construct two
set-theo\-ret\-i\-cal objects:  a ``formal system'' (roughly, a set of syntax
rules, axioms, and logical rules) and its ``universe'' (roughly, the set of
theorems derivable in the formal system).  The Metamath computer language
provides us with a way to describe specific formal systems and, with the aid of
a proof provided by the user, to verify that given theorems
belong to their universes.

To understand this appendix, you need a basic knowledge of informal set theory.
It should be sufficient to understand, for example, Ch.\ 1 of Munkres' {\em
Topology} \cite{Munkres}\index{Munkres, James R.} or the
introductory set theory chapter
in many textbooks that introduce abstract mathematics. (Note that there are
minor notational differences among authors; e.g.\ Munkres uses $\subset$ instead
of our $\subseteq$ for ``subset.''  We use ``included in'' to mean ``a subset
of,'' and ``belongs to'' or ``is contained in'' to mean ``is an element of.'')
What we call a ``formal'' description here, unlike earlier, is actually an
informal description in the ordinary language of mathematicians.  However we
provide sufficient detail so that a mathematician could easily formalize it,
even in the language of Metamath itself if desired.  To understand the logic
examples at the end of this appendix, familiarity with an introductory book on
mathematical logic would be helpful.

\section{The Formal Description}

\subsection[Preliminaries]{Preliminaries\protect\footnotemark}%
\footnotetext{This section is taken mostly verbatim
from Tarski \cite[p.~63]{Tarski1965}\index{Tarski, Alfred}.}

By $\omega$ we denote the set of all natural numbers (non-negative integers).
Each natural number $n$ is identified with the set of all smaller numbers: $n =
\{ m | m < n \}$.  The formula $m < n$ is thus equivalent to the condition: $m
\in n$ and $m,n \in \omega$. In particular, 0 is the number zero and at the
same time the empty set $\varnothing$, $1=\{0\}$, $2=\{0,1\}$, etc. ${}^B A$
denotes the set of all functions on $B$ to $A$ (i.e.\ with domain $B$ and range
included in $A$).  The members of ${}^\omega A$ are what are called {\em simple
infinite sequences},\index{simple infinite sequence}
with all {\em terms}\index{term} in $A$.  In case $n \in \omega$, the
members of ${}^n A$ are referred to as {\em finite $n$-termed
sequences},\index{finite $n$-termed
sequence} again
with terms in $A$.  The consecutive terms (function values) of a finite or
infinite sequence $f$ are denoted by $f_0, f_1, \ldots ,f_n,\ldots$.  Every
finite sequence $f \in \bigcup _{n \in \omega} {}^n A$ uniquely determines the
number $n$ such that $f \in {}^n A$; $n$ is called the {\em
length}\index{length of a sequence ({$"|\ "|$})} of $f$ and
is denoted by $|f|$.  $\langle a \rangle$ is the sequence $f$ with $|f|=1$ and
$f_0=a$; $\langle a,b \rangle$ is the sequence $f$ with $|f|=2$, $f_0=a$,
$f_1=b$; etc.  Given two finite sequences $f$ and $g$, we denote by $f\frown g$
their {\em concatenation},\index{concatenation} i.e., the
finite sequence $h$ determined by the
conditions:
\begin{eqnarray*}
& |h| = |f|+|g|;&  \\
& h_n = f_n & \mbox{\ for\ } n < |f|;  \\
& h_{|f|+n} = g_n & \mbox{\ for\ } n < |g|.
\end{eqnarray*}

\subsection{Constants, Variables, and Expressions}

A formal system has a set of {\em symbols}\index{symbol!in
a formal system} denoted
by $\mbox{\em SM}$.  A
precise set-theo\-ret\-i\-cal definition of this set is unimportant; a symbol
could be considered a primitive or atomic element if we wish.  We assume this
set is divided into two disjoint subsets:  a set $\mbox{\em CN}$ of {\em
constants}\index{constant!in a formal system} and a set $\mbox{\em VR}$ of
{\em variables}.\index{variable!in a formal system}  $\mbox{\em CN}$ and
$\mbox{\em VR}$ are each assumed to consist of countably many symbols which
may be arranged in finite or simple infinite sequences $c_0, c_1, \ldots$ and
$v_0, v_1, \ldots$ respectively, without repeating terms.  We will represent
arbitrary symbols by metavariables $\alpha$, $\beta$, etc.

{\footnotesize\begin{quotation}
{\em Comment.} The variables $v_0, v_1, \ldots$ of our formal system
correspond to what are usually considered ``metavariables'' in
descriptions of specific formal systems in the literature.  Typically,
when describing a specific formal system a book will postulate a set of
primitive objects called variables, then proceed to describe their
properties using metavariables that range over them, never mentioning
again the actual variables themselves.  Our formal system does not
mention these primitive variable objects at all but deals directly with
metavariables, as its primitive objects, from the start.  This is a
subtle but key distinction you should keep in mind, and it makes our
definition of ``formal system'' somewhat different from that typically
found in the literature.  (So, the $\alpha$, $\beta$, etc.\ above are
actually ``metametavariables'' when used to represent $v_0, v_1,
\ldots$.)
\end{quotation}}

Finite sequences all terms of which are symbols are called {\em
expressions}.\index{expression!in a formal system}  $\mbox{\em EX}$ is
the set of all expressions; thus
\begin{displaymath}
\mbox{\em EX} = \bigcup _{n \in \omega} {}^n \mbox{\em SM}.
\end{displaymath}

A {\em constant-prefixed expression}\index{constant-prefixed expression}
is an expression of non-zero length
whose first term is a constant.  We denote the set of all constant-prefixed
expressions by $\mbox{\em EX}_C = \{ e \in \mbox{\em EX} | ( |e| > 0 \wedge
e_0 \in \mbox{\em CN} ) \}$.

A {\em constant-variable pair}\index{constant-variable pair}
is an expression of length 2 whose first term
is a constant and whose second term is a variable.  We denote the set of all
constant-variable pairs by $\mbox{\em EX}_2 = \{ e \in \mbox{\em EX}_C | ( |e|
= 2 \wedge e_1 \in \mbox{\em VR} ) \}$.


{\footnotesize\begin{quotation}
{\em Relationship to Metamath.} In general, the set $\mbox{\em SM}$
corresponds to the set of declared math symbols in a Metamath database, the
set $\mbox{\em CN}$ to those declared with \texttt{\$c} statements, and the set
$\mbox{\em VR}$ to those declared with \texttt{\$v} statements.  Of course a
Metamath database can only have a finite number of math symbols, whereas
formal systems in general can have an infinite number, although the number of
Metamath math symbols available is in principle unlimited.

The set $\mbox{\em EX}_C$ corresponds to the set of permissible expressions
for \texttt{\$e}, \texttt{\$a}, and \texttt{\$p} statements.  The set $\mbox{\em EX}_2$
corresponds to the set of permissible expressions for \texttt{\$f} statements.
\end{quotation}}

We denote by ${\cal V}(e)$ the set of all variables in an expression $e \in
\mbox{\em EX}$, i.e.\ the set of all $\alpha \in \mbox{\em VR}$ such that
$\alpha = e_n$ for some $n < |e|$.  We also denote (with abuse of notation) by
${\cal V}(E)$ the set of all variables in a collection of expressions $E
\subseteq \mbox{\em EX}$, i.e.\ $\bigcup _{e \in E} {\cal V}(e)$.


\subsection{Substitution}

Given a function $F$ from $\mbox{\em VR}$ to
$\mbox{\em EX}$, we
denote by $\sigma_{F}$ or just $\sigma$ the function from $\mbox{\em EX}$ to
$\mbox{\em EX}$ defined recursively for nonempty sequences by
\begin{eqnarray*}
& \sigma(<\alpha>) = F(\alpha) & \mbox{for\ } \alpha \in \mbox{\em VR}; \\
& \sigma(<\alpha>) = <\alpha> & \mbox{for\ } \alpha \not\in \mbox{\em VR}; \\
& \sigma(g \frown h) = \sigma(g) \frown
    \sigma(h) & \mbox{for\ } g,h \in \mbox{\em EX}.
\end{eqnarray*}
We also define $\sigma(\varnothing)=\varnothing$.  We call $\sigma$ a {\em
simultaneous substitution}\index{substitution!variable}\index{variable
substitution} (or just {\em substitution}) with {\em substitution
map}\index{substitution map} $F$.

We also denote (with abuse of notation) by $\sigma(E)$ a substitution on a
collection of expressions $E \subseteq \mbox{\em EX}$, i.e.\ the set $\{
\sigma(e) | e \in E \}$.  The collection $\sigma(E)$ may of course contain
fewer expressions than $E$ because duplicate expressions could result from the
substitution.

\subsection{Statements}

We denote by $\mbox{\em DV}$ the set of all
unordered pairs $\{\alpha, \beta \} \subseteq \mbox{\em VR}$ such that $\alpha
\neq \beta$.  $\mbox{\em DV}$ stands for ``distinct variables.''

A {\em pre-statement}\index{pre-statement!in a formal system} is a
quadruple $\langle D,T,H,A \rangle$ such that
$D\subseteq \mbox{\em DV}$, $T\subseteq \mbox{\em EX}_2$, $H\subseteq
\mbox{\em EX}_C$ and $H$ is finite,
$A\in \mbox{\em EX}_C$, ${\cal V}(H\cup\{A\}) \subseteq
{\cal V}(T)$, and $\forall e,f\in T {\ } {\cal V}(e) \neq {\cal V}(f)$ (or
equivalently, $e_1 \ne f_1$) whenever $e \neq f$. The terms of the quadruple are called {\em
distinct-variable restrictions},\index{disjoint-variable restriction!in a
formal system} {\em variable-type hypotheses},\index{variable-type
hypothesis!in a formal system} {\em logical hypotheses},\index{logical
hypothesis!in a formal system} and the {\em assertion}\index{assertion!in a
formal system} respectively.  We denote by $T_M$ ({\em mandatory variable-type
hypotheses}\index{mandatory variable-type hypothesis!in a formal system}) the
subset of $T$ such that ${\cal V}(T_M) ={\cal V}(H \cup \{A\})$.  We denote by
$D_M=\{\{\alpha,\beta\}\in D|\{\alpha,\beta\}\subseteq {\cal V}(T_M)\}$ the
{\em mandatory distinct-variable restrictions}\index{mandatory
disjoint-variable restriction!in a formal system} of the pre-statement.
The set
of {\em mandatory hypotheses}\index{mandatory hypothesis!in a formal system}
is $T_M\cup H$.  We call the quadruple $\langle D_M,T_M,H,A \rangle$
the {\em reduct}\index{reduct!in a formal system} of
the pre-statement $\langle D,T,H,A \rangle$.

A {\em statement} is the reduct of some pre-statement\index{statement!in a
formal system}.  A statement is therefore a special kind of pre-statement;
in particular, a statement is the reduct of itself.

{\footnotesize\begin{quotation}
{\em Comment.}  $T$ is a set of expressions, each of length 2, that associate
a set of constants (``variable types'') with a set of variables.  The
condition ${\cal V}(H\cup\{A\}) \subseteq {\cal V}(T) $
means that each variable occurring in a statement's logical
hypotheses or assertion must have an associated variable-type hypothesis or
``type declaration,'' in  analogy to a computer programming language, where a
variable must be declared to be say, a string or an integer.  The requirement
that $\forall e,f\in T \, e_1 \ne f_1$ for $e\neq f$
means that each variable must be
associated with a unique constant designating its variable type; e.g., a
variable might be a ``wff'' or a ``set'' but not both.

Distinct-variable restrictions are used to specify what variable substitutions
are permissible to make for the statement to remain valid.  For example, in
the theorem scheme of set theory $\lnot\forall x\,x=y$ we may not substitute
the same variable for both $x$ and $y$.  On the other hand, the theorem scheme
$x=y\to y=x$ does not require that $x$ and $y$ be distinct, so we do not
require a distinct-variable restriction, although having one
would cause no harm other than making the scheme less general.

A mandatory variable-type hypothesis is one whose variable exists in a logical
hypothesis or the assertion.  A provable pre-statement
(defined below) may require
non-mandatory variable-type hypotheses that effectively introduce ``dummy''
variables for use in its proof.  Any number of dummy variables might
be required by a specific proof; indeed, it has been shown by H.\
Andr\'{e}ka\index{Andr{\'{e}}ka, H.} \cite{Nemeti} that there is no finite
upper bound to the number of dummy variables needed to prove an arbitrary
theorem in first-order logic (with equality) having a fixed number $n>2$ of
individual variables.  (See also the Comment on p.~\pageref{nodd}.)
For this reason we do not set a finite size bound on the collections $D$ and
$T$, although in an actual application (Metamath database) these will of
course be finite, increased to whatever size is necessary as more
proofs are added.
\end{quotation}}

{\footnotesize\begin{quotation}
{\em Relationship to Metamath.} A pre-statement of a formal system
corresponds to an extended frame in a Metamath database
(Section~\ref{frames}).  The collections $D$, $T$, and $H$ correspond
respectively to the \texttt{\$d}, \texttt{\$f}, and \texttt{\$e}
statement collections in an extended frame.  The expression $A$
corresponds to the \texttt{\$a} (or \texttt{\$p}) statement in an
extended frame.

A statement of a formal system corresponds to a frame in a Metamath
database.
\end{quotation}}

\subsection{Formal Systems}

A {\em formal system}\index{formal system} is a
triple $\langle \mbox{\em CN},\mbox{\em
VR},\Gamma\rangle$ where $\Gamma$ is a set of statements.  The members of
$\Gamma$ are called {\em axiomatic statements}.\index{axiomatic
statement!in a formal system}  Sometimes we will refer to a
formal system by just $\Gamma$ when $\mbox{\em CN}$ and $\mbox{\em VR}$ are
understood.

Given a formal system $\Gamma$, the {\em closure}\index{closure}\footnote{This
definition of closure incorporates a simplification due to
Josh Purinton.\index{Purinton, Josh}.} of a
pre-statement
$\langle D,T,H,A \rangle$ is the smallest set $C$ of expressions
such that:
%\begin{enumerate}
%  \item $T\cup H\subseteq C$; and
%  \item If for some axiomatic statement
%    $\langle D_M',T_M',H',A' \rangle \in \Gamma_A$, for
%    some $E \subseteq C$, some $F \subseteq C-T$ (where ``-'' denotes
%    set difference), and some substitution
%    $\sigma$ we have
%    \begin{enumerate}
%       \item $\sigma(T_M') = E$ (where, as above, the $M$ denotes the
%           mandatory variable-type hypotheses of $T^A$);
%       \item $\sigma(H') = F$;
%       \item for all $\{\alpha,\beta\}\in D^A$ and $\subseteq
%         {\cal V}(T_M')$, for all $\gamma\in {\cal V}(\sigma(\langle \alpha
%         \rangle))$, and for all $\delta\in  {\cal V}(\sigma(\langle \beta
%         \rangle))$, we have $\{\gamma, \delta\} \in D$;
%   \end{enumerate}
%   then $\sigma(A') \in C$.
%\end{enumerate}
\begin{list}{}{\itemsep 0.0pt}
  \item[1.] $T\cup H\subseteq C$; and
  \item[2.] If for some axiomatic statement
    $\langle D_M',T_M',H',A' \rangle \in
       \Gamma$ and for some substitution
    $\sigma$ we have
    \begin{enumerate}
       \item[a.] $\sigma(T_M' \cup H') \subseteq C$; and
       \item[b.] for all $\{\alpha,\beta\}\in D_M'$, for all $\gamma\in
         {\cal V}(\sigma(\langle \alpha
         \rangle))$, and for all $\delta\in  {\cal V}(\sigma(\langle \beta
         \rangle))$, we have $\{\gamma, \delta\} \in D$;
   \end{enumerate}
   then $\sigma(A') \in C$.
\end{list}
A pre-statement $\langle D,T,H,A
\rangle$ is {\em provable}\index{provable statement!in a formal
system} if $A\in C$ i.e.\ if its assertion belongs to its
closure.  A statement is {\em provable} if it is
the reduct of a provable pre-statement.
The {\em universe}\index{universe of a formal system}
of a formal system is
the collection of all of its provable statements.  Note that the
set of axiomatic statements $\Gamma$ in a formal system is a subset of its
universe.

{\footnotesize\begin{quotation}
{\em Comment.} The first condition in the definition of closure simply says
that the hypotheses of the pre-statement are in its closure.

Condition 2(a) says that a substitution exists that makes the
mandatory hypotheses of an axiomatic statement exactly match some members of
the closure.  This is what we explicitly demonstrate in a Metamath language
proof.

%Conditions 2(a) and 2(b) say that a substitution exists that makes the
%(mandatory) hypotheses of an axiomatic statement exactly match some members of
%the closure.  This is what we explicitly demonstrate with a Metamath language
%proof.
%
%The set of expressions $F$ in condition 2(b) excludes the variable-type
%hypotheses; this is done because non-mandatory variable-type hypotheses are
%effectively ``dropped'' as irrelevant whereas logical hypotheses must be
%retained to achieve a consistent logical system.

Condition 2(b) describes how distinct-variable restrictions in the axiomatic
statement must be met.  It means that after a substitution for two variables
that must be distinct, the resulting two expressions must either contain no
variables, or if they do, they may not have variables in common, and each pair
of any variables they do have, with one variable from each expression, must be
specified as distinct in the original statement.
\end{quotation}}

{\footnotesize\begin{quotation}
{\em Relationship to Metamath.} Axiomatic statements
 and provable statements in a formal
system correspond to the frames for \texttt{\$a} and \texttt{\$p} statements
respectively in a Metamath database.  The set of axiomatic statements is a
subset of the set of provable statements in a formal system, although in a
Metamath database a \texttt{\$a} statement is distinguished by not having a
proof.  A Metamath language proof for a \texttt{\$p} statement tells the computer
how to explicitly construct a series of members of the closure ultimately
leading to a demonstration that the assertion
being proved is in the closure.  The actual closure typically contains
an infinite number of expressions.  A formal system itself does not have
an explicit object called a ``proof'' but rather the existence of a proof
is implied indirectly by membership of an assertion in a provable
statement's closure.  We do this to make the formal system easier
to describe in the language of set theory.

We also note that once established as provable, a statement may be considered
to acquire the same status as an axiomatic statement, because if the set of
axiomatic statements is extended with a provable statement, the universe of
the formal system remains unchanged (provided that $\mbox{\em VR}$ is
infinite).
In practice, this means we can build a hierarchy of provable statements to
more efficiently establish additional provable statements.  This is
what we do in Metamath when we allow proofs to reference previous
\texttt{\$p} statements as well as previous \texttt{\$a} statements.
\end{quotation}}

\section{Examples of Formal Systems}

{\footnotesize\begin{quotation}
{\em Relationship to Metamath.} The examples in this section, except Example~2,
are for the most part exact equivalents of the development in the set
theory database \texttt{set.mm}.  You may want to compare Examples~1, 3, and 5
to Section~\ref{metaaxioms}, Example 4 to Sections~\ref{metadefprop} and
\ref{metadefpred}, and Example 6 to
Section~\ref{setdefinitions}.\label{exampleref}
\end{quotation}}

\subsection{Example~1---Propositional Calculus}\index{propositional calculus}

Classical propositional calculus can be described by the following formal
system.  We assume the set of variables is infinite.  Rather than denoting the
constants and variables by $c_0, c_1, \ldots$ and $v_0, v_1, \ldots$, for
readability we will instead use more conventional symbols, with the
understanding of course that they denote distinct primitive objects.
Also for readability we may omit commas between successive terms of a
sequence; thus $\langle \mbox{wff\ } \varphi\rangle$ denotes
$\langle \mbox{wff}, \varphi\rangle$.

Let
\begin{itemize}
  \item[] $\mbox{\em CN}=\{\mbox{wff}, \vdash, \to, \lnot, (,)\}$
  \item[] $\mbox{\em VR}=\{\varphi,\psi,\chi,\ldots\}$
  \item[] $T = \{\langle \mbox{wff\ } \varphi\rangle,
             \langle \mbox{wff\ } \psi\rangle,
             \langle \mbox{wff\ } \chi\rangle,\ldots\}$, i.e.\ those
             expressions of length 2 whose first member is $\mbox{\rm wff}$
             and whose second member belongs to $\mbox{\em VR}$.\footnote{For
convenience we let $T$ be an infinite set; the definition of a statement
permits this in principle.  Since a Metamath source file has a finite size, in
practice we must of course use appropriate finite subsets of this $T$,
specifically ones containing at least the mandatory variable-type
hypotheses.  Similarly, in the source file we introduce new variables as
required, with the understanding that a potentially infinite number of
them are available.}
\noindent Then $\Gamma$ consists of the axiomatic statements that
are the reducts of the following pre-statements:
    \begin{itemize}
      \item[] $\langle\varnothing,T,\varnothing,
               \langle \mbox{wff\ }(\varphi\to\psi)\rangle\rangle$
      \item[] $\langle\varnothing,T,\varnothing,
               \langle \mbox{wff\ }\lnot\varphi\rangle\rangle$
      \item[] $\langle\varnothing,T,\varnothing,
               \langle \vdash(\varphi\to(\psi\to\varphi))
               \rangle\rangle$
      \item[] $\langle\varnothing,T,
               \varnothing,
               \langle \vdash((\varphi\to(\psi\to\chi))\to
               ((\varphi\to\psi)\to(\varphi\to\chi)))
               \rangle\rangle$
      \item[] $\langle\varnothing,T,
               \varnothing,
               \langle \vdash((\lnot\varphi\to\lnot\psi)\to
               (\psi\to\varphi))\rangle\rangle$
      \item[] $\langle\varnothing,T,
               \{\langle\vdash(\varphi\to\psi)\rangle,
                 \langle\vdash\varphi\rangle\},
               \langle\vdash\psi\rangle\rangle$
    \end{itemize}
\end{itemize}

(For example, the reduct of $\langle\varnothing,T,\varnothing,
               \langle \mbox{wff\ }(\varphi\to\psi)\rangle\rangle$
is
\begin{itemize}
\item[] $\langle\varnothing,
\{\langle \mbox{wff\ } \varphi\rangle,
             \langle \mbox{wff\ } \psi\rangle\},
             \varnothing,
               \langle \mbox{wff\ }(\varphi\to\psi)\rangle\rangle$,
\end{itemize}
which is the first axiomatic statement.)

We call the members of $\mbox{\em VR}$ {\em wff variables} or (in the context
of first-order logic which we will describe shortly) {\em wff metavariables}.
Note that the symbols $\phi$, $\psi$, etc.\ denote actual specific members of
$\mbox{\em VR}$; they are not metavariables of our expository language (which
we denote with $\alpha$, $\beta$, etc.) but are instead (meta)constant symbols
(members of $\mbox{\em SM}$) from the point of view of our expository
language.  The equivalent system of propositional calculus described in
\cite{Tarski1965} also uses the symbols $\phi$, $\psi$, etc.\ to denote wff
metavariables, but in \cite{Tarski1965} unlike here those are metavariables of
the expository language and not primitive symbols of the formal system.

The first two statements define wffs: if $\varphi$ and $\psi$ are wffs, so is
$(\varphi \to \psi)$; if $\varphi$ is a wff, so is $\lnot\varphi$. The next
three are the axioms of propositional calculus: if $\varphi$ and $\psi$ are
wffs, then $\vdash (\varphi \to (\psi \to \varphi))$ is an (axiomatic)
theorem; etc. The
last is the rule of modus ponens: if $\varphi$ and $\psi$ are wffs, and
$\vdash (\varphi\to\psi)$ and $\vdash \varphi$ are theorems, then $\vdash
\psi$ is a theorem.

The correspondence to ordinary propositional calculus is as follows.  We
consider only provable statements of the form $\langle\varnothing,
T,\varnothing,A\rangle$ with $T$ defined as above.  The first term of the
assertion $A$ of any such statement is either ``wff'' or ``$\vdash$''.  A
statement for which the first term is ``wff'' is a {\em wff} of propositional
calculus, and one where the first term is ``$\vdash$'' is a {\em
theorem (scheme)} of propositional calculus.

The universe of this formal system also contains many other provable
statements.  Those with distinct-variable restrictions are irrelevant because
propositional calculus has no constraints on substitutions.  Those that have
logical hypotheses we call {\em inferences}\index{inference} when
the logical hypotheses are of the form
$\langle\vdash\rangle\frown w$ where $w$ is a wff (with the leading constant
term ``wff'' removed).  Inferences (other than the modus ponens rule) are not a
proper part of propositional calculus but are convenient to use when building a
hierarchy of provable statements.  A provable statement with a nonsense
hypothesis such as $\langle \to,\vdash,\lnot\rangle$, and this same expression
as its assertion, we consider irrelevant; no use can be made of it in
proving theorems, since there is no way to eliminate the nonsense hypothesis.

{\footnotesize\begin{quotation}
{\em Comment.} Our use of parentheses in the definition of a wff illustrates
how axiomatic statements should be carefully stated in a way that
ties in unambiguously with the substitutions allowed by the formal system.
There are many ways we could have defined wffs---for example, Polish
prefix notation would have allowed us to omit parentheses entirely, at
the expense of readability---but we must define them in a way that is
unambiguous.  For example, if we had omitted parentheses from the
definition of $(\varphi\to \psi)$, the wff $\lnot\varphi\to \psi$ could
be interpreted as either $\lnot(\varphi\to\psi)$ or $(\lnot\varphi\to\psi)$
and would have allowed us to prove nonsense.  Note that there is no
concept of operator binding precedence built into our formal system.
\end{quotation}}

\begin{sloppy}
\subsection{Example~2---Predicate Calculus with Equality}\index{predicate
calculus}
\end{sloppy}

Here we extend Example~1 to include predicate calculus with equality,
illustrating the use of distinct-variable restrictions.  This system is the
same as Tarski's system $\mathfrak{S}_2$ in \cite{Tarski1965} (except that the
axioms of propositional calculus are different but equivalent, and a redundant
axiom is omitted).  We extend $\mbox{\em CN}$ with the constants
$\{\mbox{var},\forall,=\}$.  We extend $\mbox{\em VR}$ with an infinite set of
{\em individual metavariables}\index{individual
metavariable} $\{x,y,z,\ldots\}$ and denote this subset
$\mbox{\em Vr}$.

We also join to $\mbox{\em CN}$ a possibly infinite set $\mbox{\em Pr}$ of {\em
predicates} $\{R,S,\ldots\}$.  We associate with $\mbox{\em Pr}$ a function
$\mbox{rnk}$ from $\mbox{\em Pr}$ to $\omega$, and for $\alpha\in \mbox{\em
Pr}$ we call $\mbox{rnk}(\alpha)$ the {\em rank} of the predicate $\alpha$,
which is simply the number of ``arguments'' that the predicate has.  (Most
applications of predicate calculus will have a finite number of predicates;
for example, set theory has the single two-argument or binary predicate $\in$,
which is usually written with its arguments surrounding the predicate symbol
rather than with the prefix notation we will use for the general case.)  As a
device to facilitate our discussion, we will let $\mbox{\em Vs}$ be any fixed
one-to-one function from $\omega$ to $\mbox{\em Vr}$; thus $\mbox{\em Vs}$ is
any simple infinite sequence of individual metavariables with no repeating
terms.

In this example we will not include the function symbols that are often part of
formalizations of predicate calculus.  Using metalogical arguments that are
beyond the scope of our discussion, it can be shown that our formalization is
equivalent when functions are introduced via appropriate definitions.

We extend the set $T$ defined in Example~1 with the expressions
$\{\langle \mbox{var\ } x\rangle,$ $ \langle \mbox{var\ } y\rangle, \langle
\mbox{var\ } z\rangle,\ldots\}$.  We extend the $\Gamma$ above
with the axiomatic statements that are the reducts of the following
pre-statements:
\begin{list}{}{\itemsep 0.0pt}
      \item[] $\langle\varnothing,T,\varnothing,
               \langle \mbox{wff\ }\forall x\,\varphi\rangle\rangle$
      \item[] $\langle\varnothing,T,\varnothing,
               \langle \mbox{wff\ }x=y\rangle\rangle$
      \item[] $\langle\varnothing,T,
               \{\langle\vdash\varphi\rangle\},
               \langle\vdash\forall x\,\varphi\rangle\rangle$
      \item[] $\langle\varnothing,T,\varnothing,
               \langle \vdash((\forall x(\varphi\to\psi)
                  \to(\forall x\,\varphi\to\forall x\,\psi))
               \rangle\rangle$
      \item[] $\langle\{\{x,\varphi\}\},T,\varnothing,
               \langle \vdash(\varphi\to\forall x\,\varphi)
               \rangle\rangle$
      \item[] $\langle\{\{x,y\}\},T,\varnothing,
               \langle \vdash\lnot\forall x\lnot x=y
               \rangle\rangle$
      \item[] $\langle\varnothing,T,\varnothing,
               \langle \vdash(x=z
                  \to(x=y\to z=y))
               \rangle\rangle$
      \item[] $\langle\varnothing,T,\varnothing,
               \langle \vdash(y=z
                  \to(x=y\to x=z))
               \rangle\rangle$
\end{list}
These are the axioms not involving predicate symbols. The first two statements
extend the definition of a wff.  The third is the rule of generalization.  The
fifth states, in effect, ``For a wff $\varphi$ and variable $x$,
$\vdash(\varphi\to\forall x\,\varphi)$, provided that $x$ does not occur in
$\varphi$.''  The sixth states ``For variables $x$ and $y$,
$\vdash\lnot\forall x\lnot x = y$, provided that $x$ and $y$ are distinct.''
(This proviso is not necessary but was included by Tarski to
weaken the axiom and still show that the system is logically complete.)

Finally, for each predicate symbol $\alpha\in \mbox{\em Pr}$, we add to
$\Gamma$ an axiomatic statement, extending the definition of wff,
that is the reduct of the following pre-statement:
\begin{displaymath}
    \langle\varnothing,T,\varnothing,
            \langle \mbox{wff},\alpha\rangle\
            \frown \mbox{\em Vs}\restriction\mbox{rnk}(\alpha)\rangle
\end{displaymath}
and for each $\alpha\in \mbox{\em Pr}$ and each $n < \mbox{rnk}(\alpha)$
we add to $\Gamma$ an equality axiom that is the reduct of the
following pre-statement:
\begin{eqnarray*}
    \lefteqn{\langle\varnothing,T,\varnothing,
            \langle
      \vdash,(,\mbox{\em Vs}_n,=,\mbox{\em Vs}_{\mbox{rnk}(\alpha)},\to,
            (,\alpha\rangle\frown \mbox{\em Vs}\restriction\mbox{rnk}(\alpha)} \\
  & & \frown
            \langle\to,\alpha\rangle\frown \mbox{\em Vs}\restriction n\frown
            \langle \mbox{\em Vs}_{\mbox{rnk}(\alpha)}\rangle \\
 & & \frown
            \mbox{\em Vs}\restriction(\mbox{rnk}(\alpha)\setminus(n+1))\frown
            \langle),)\rangle\rangle
\end{eqnarray*}
where $\restriction$ denotes function domain restriction and $\setminus$
denotes set difference.  Recall that a subscript on $\mbox{\em Vs}$
denotes one of its terms.  (In the above two axiom sets commas are placed
between successive terms of sequences to prevent ambiguity, and if you examine
them with care you will be able to distinguish those parentheses that denote
constant symbols from those of our expository language that delimit function
arguments.  Although it might have been better to use boldface for our
primitive symbols, unfortunately boldface was not available for all characters
on the \LaTeX\ system used to typeset this text.)  These seemingly forbidding
axioms can be understood by analogy to concatenation of substrings in a
computer language.  They are actually relatively simple for each specific case
and will become clearer by looking at the special case of a binary predicate
$\alpha = R$ where $\mbox{rnk}(R)=2$.  Letting $\mbox{\em Vs}$ be the sequence
$\langle x,y,z,\ldots\rangle$, the axioms we would add to $\Gamma$ for this
case would be the wff extension and two equality axioms that are the
reducts of the pre-statements:
\begin{list}{}{\itemsep 0.0pt}
      \item[] $\langle\varnothing,T,\varnothing,
               \langle \mbox{wff\ }R x y\rangle\rangle$
      \item[] $\langle\varnothing,T,\varnothing,
               \langle \vdash(x=z
                  \to(R x y \to R z y))
               \rangle\rangle$
      \item[] $\langle\varnothing,T,\varnothing,
               \langle \vdash(y=z
                  \to(R x y \to R x z))
               \rangle\rangle$
\end{list}
Study these carefully to see how the general axioms above evaluate to
them.  In practice, typically only a few special cases such as this would be
needed, and in any case the Metamath language will only permit us to describe
a finite number of predicates, as opposed to the infinite number permitted by
the formal system.  (If an infinite number should be needed for some reason,
we could not define the formal system directly in the Metamath language but
could instead define it metalogically under set theory as we
do in this appendix, and only the underlying set theory, with its single
binary predicate, would be defined directly in the Metamath language.)


{\footnotesize\begin{quotation}
{\em Comment.}  As we noted earlier, the specific variables denoted by the
symbols $x,y,z,\ldots\in \mbox{\em Vr}\subseteq \mbox{\em VR}\subseteq
\mbox{\em SM}$ in Example~2 are not the actual variables of ordinary predicate
calculus but should be thought of as metavariables ranging over them.  For
example, a distinct-variable restriction would be meaningless for actual
variables of ordinary predicate calculus since two different actual variables
are by definition distinct.  And when we talk about an arbitrary
representative $\alpha\in \mbox{\em Vr}$, $\alpha$ is a metavariable (in our
expository language) that ranges over metavariables (which are primitives of
our formal system) each of which ranges over the actual individual variables
of predicate calculus (which are never mentioned in our formal system).

The constant called ``var'' above is called \texttt{setvar} in the
\texttt{set.mm} database file, but it means the same thing.  I felt
that ``var'' is a more meaningful name in the context of predicate
calculus, whose use is not limited to set theory.  For consistency we
stick with the name ``var'' throughout this Appendix, even after set
theory is introduced.
\end{quotation}}

\subsection{Free Variables and Proper Substitution}\index{free variable}
\index{proper substitution}\index{substitution!proper}

Typical representations of mathematical axioms use concepts such
as ``free variable,'' ``bound variable,'' and ``proper substitution''
as primitive notions.
A free variable is a variable that
is not a parameter of any container expression.
A bound variable is the opposite of a free variable; it is a
a variable that has been bound in a container expression.
For example, in the expression $\forall x \varphi$ (for all $x$, $\varphi$
is true), the variable $x$
is bound within the for-all ($\forall$) expression.
It is possible to change one variable to another, and that process is called
``proper substitution.''
In most books, proper substitution has a somewhat complicated recursive
definition with multiple cases based on the occurrences of free and
bound variables.
You may consult
\cite[ch.\ 3--4]{Hamilton}\index{Hamilton, Alan G.} (as well as
many other texts) for more formal details about these terms.

Using these concepts as \texttt{primitives} creates complications
for computer implementations.

In the system of Example~2, there are no primitive notions of free variable
and proper substitution.  Tarski \cite{Tarski1965} shows that this system is
logically equivalent to the more typical textbook systems that do have these
primitive notions, if we introduce these notions with appropriate definitions
and metalogic.  We could also define axioms for such systems directly,
although the recursive definitions of free variable and proper substitution
would be messy and awkward to work with.  Instead, we mention two devices that
can be used in practice to mimic these notions.  (1) Instead of introducing
special notation to express (as a logical hypothesis) ``where $x$ is not free
in $\varphi$'' we can use the logical hypothesis $\vdash(\varphi\to\forall
x\,\varphi)$.\label{effectivelybound}\index{effectively
not free}\footnote{This is a slightly weaker requirement than ``where $x$ is
not free in $\varphi$.''  If we let $\varphi$ be $x=x$, we have the theorem
$(x=x\to\forall x\,x=x)$ which satisfies the hypothesis, even though $x$ is
free in $x=x$ .  In a case like this we say that $x$ is {\em effectively not
free}\index{effectively not free} in $x=x$, since $x=x$ is logically
equivalent to $\forall x\,x=x$ in which $x$ is bound.} (2) It can be shown
that the wff $((x=y\to\varphi)\wedge\exists x(x=y\wedge\varphi))$ (with the
usual definitions of $\wedge$ and $\exists$; see Example~4 below) is logically
equivalent to ``the wff that results from proper substitution of $y$ for $x$
in $\varphi$.''  This works whether or not $x$ and $y$ are distinct.

\subsection{Metalogical Completeness}\index{metalogical completeness}

In the system of Example~2, the
following are provable pre-statements (and their reducts are
provable statements):
\begin{eqnarray*}
      & \langle\{\{x,y\}\},T,\varnothing,
               \langle \vdash\lnot\forall x\lnot x=y
               \rangle\rangle & \\
     &  \langle\varnothing,T,\varnothing,
               \langle \vdash\lnot\forall x\lnot x=x
               \rangle\rangle &
\end{eqnarray*}
whereas the following pre-statement is not to my knowledge provable (but
in any case we will pretend it's not for sake of illustration):
\begin{eqnarray*}
     &  \langle\varnothing,T,\varnothing,
               \langle \vdash\lnot\forall x\lnot x=y
               \rangle\rangle &
\end{eqnarray*}
In other words, we can prove ``$\lnot\forall x\lnot x=y$ where $x$ and $y$ are
distinct'' and separately prove ``$\lnot\forall x\lnot x=x$'', but we can't
prove the combined general case ``$\lnot\forall x\lnot x=y$'' that has no
proviso.  Now this does not compromise logical completeness, because the
variables are really metavariables and the two provable cases together cover
all possible cases.  The third case can be considered a metatheorem whose
direct proof, using the system of Example~2, lies outside the capability of the
formal system.

Also, in the system of Example~2 the following pre-statement is not to my
knowledge provable (again, a conjecture that we will pretend to be the case):
\begin{eqnarray*}
     & \langle\varnothing,T,\varnothing,
               \langle \vdash(\forall x\, \varphi\to\varphi)
               \rangle\rangle &
\end{eqnarray*}
Instead, we can only prove specific cases of $\varphi$ involving individual
metavariables, and by induction on formula length, prove as a metatheorem
outside of our formal system the general statement above.  The details of this
proof are found in \cite{Kalish}.

There does, however, exist a system of predicate calculus in which all such
``simple metatheorems'' as those above can be proved directly, and we present
it in Example~3. A {\em simple metatheorem}\index{simple metatheorem}
is any statement of the formal
system of Example~2 where all distinct variable restrictions consist of either
two individual metavariables or an individual metavariable and a wff
metavariable, and which is provable by combining cases outside the system as
above.  A system is {\em metalogically complete}\index{metalogical
completeness} if all of its simple
metatheorems are (directly) provable statements. The precise definition of
``simple metatheorem'' and the proof of the ``metalogical completeness'' of
Example~3 is found in Remark 9.6 and Theorem 9.7 of \cite{Megill}.\index{Megill,
Norman}

\begin{sloppy}
\subsection{Example~3---Metalogically Complete Predicate
Calculus with
Equality}
\end{sloppy}

For simplicity we will assume there is one binary predicate $R$;
this system suffices for set theory, where the $R$ is of course the $\in$
predicate.  We label the axioms as they appear in \cite{Megill}.  This
system is logically equivalent to that of Example~2 (when the latter is
restricted to this single binary predicate) but is also metalogically
complete.\index{metalogical completeness}

Let
\begin{itemize}
  \item[] $\mbox{\em CN}=\{\mbox{wff}, \mbox{var}, \vdash, \to, \lnot, (,),\forall,=,R\}$.
  \item[] $\mbox{\em VR}=\{\varphi,\psi,\chi,\ldots\}\cup\{x,y,z,\ldots\}$.
  \item[] $T = \{\langle \mbox{wff\ } \varphi\rangle,
             \langle \mbox{wff\ } \psi\rangle,
             \langle \mbox{wff\ } \chi\rangle,\ldots\}\cup
       \{\langle \mbox{var\ } x\rangle, \langle \mbox{var\ } y\rangle, \langle
       \mbox{var\ }z\rangle,\ldots\}$.

\noindent Then
  $\Gamma$ consists of the reducts of the following pre-statements:
    \begin{itemize}
      \item[] $\langle\varnothing,T,\varnothing,
               \langle \mbox{wff\ }(\varphi\to\psi)\rangle\rangle$
      \item[] $\langle\varnothing,T,\varnothing,
               \langle \mbox{wff\ }\lnot\varphi\rangle\rangle$
      \item[] $\langle\varnothing,T,\varnothing,
               \langle \mbox{wff\ }\forall x\,\varphi\rangle\rangle$
      \item[] $\langle\varnothing,T,\varnothing,
               \langle \mbox{wff\ }x=y\rangle\rangle$
      \item[] $\langle\varnothing,T,\varnothing,
               \langle \mbox{wff\ }Rxy\rangle\rangle$
      \item[(C1$'$)] $\langle\varnothing,T,\varnothing,
               \langle \vdash(\varphi\to(\psi\to\varphi))
               \rangle\rangle$
      \item[(C2$'$)] $\langle\varnothing,T,
               \varnothing,
               \langle \vdash((\varphi\to(\psi\to\chi))\to
               ((\varphi\to\psi)\to(\varphi\to\chi)))
               \rangle\rangle$
      \item[(C3$'$)] $\langle\varnothing,T,
               \varnothing,
               \langle \vdash((\lnot\varphi\to\lnot\psi)\to
               (\psi\to\varphi))\rangle\rangle$
      \item[(C4$'$)] $\langle\varnothing,T,
               \varnothing,
               \langle \vdash(\forall x(\forall x\,\varphi\to\psi)\to
                 (\forall x\,\varphi\to\forall x\,\psi))\rangle\rangle$
      \item[(C5$'$)] $\langle\varnothing,T,
               \varnothing,
               \langle \vdash(\forall x\,\varphi\to\varphi)\rangle\rangle$
      \item[(C6$'$)] $\langle\varnothing,T,
               \varnothing,
               \langle \vdash(\forall x\forall y\,\varphi\to
                 \forall y\forall x\,\varphi)\rangle\rangle$
      \item[(C7$'$)] $\langle\varnothing,T,
               \varnothing,
               \langle \vdash(\lnot\varphi\to\forall x\lnot\forall x\,\varphi
                 )\rangle\rangle$
      \item[(C8$'$)] $\langle\varnothing,T,
               \varnothing,
               \langle \vdash(x=y\to(x=z\to y=z))\rangle\rangle$
      \item[(C9$'$)] $\langle\varnothing,T,
               \varnothing,
               \langle \vdash(\lnot\forall x\, x=y\to(\lnot\forall x\, x=z\to
                 (y=z\to\forall x\, y=z)))\rangle\rangle$
      \item[(C10$'$)] $\langle\varnothing,T,
               \varnothing,
               \langle \vdash(\forall x(x=y\to\forall x\,\varphi)\to
                 \varphi))\rangle\rangle$
      \item[(C11$'$)] $\langle\varnothing,T,
               \varnothing,
               \langle \vdash(\forall x\, x=y\to(\forall x\,\varphi
               \to\forall y\,\varphi))\rangle\rangle$
      \item[(C12$'$)] $\langle\varnothing,T,
               \varnothing,
               \langle \vdash(x=y\to(Rxz\to Ryz))\rangle\rangle$
      \item[(C13$'$)] $\langle\varnothing,T,
               \varnothing,
               \langle \vdash(x=y\to(Rzx\to Rzy))\rangle\rangle$
      \item[(C15$'$)] $\langle\varnothing,T,
               \varnothing,
               \langle \vdash(\lnot\forall x\, x=y\to(x=y\to(\varphi
                 \to\forall x(x=y\to\varphi))))\rangle\rangle$
      \item[(C16$'$)] $\langle\{\{x,y\}\},T,
               \varnothing,
               \langle \vdash(\forall x\, x=y\to(\varphi\to\forall x\,\varphi)
                 )\rangle\rangle$
      \item[(C5)] $\langle\{\{x,\varphi\}\},T,\varnothing,
               \langle \vdash(\varphi\to\forall x\,\varphi)
               \rangle\rangle$
      \item[(MP)] $\langle\varnothing,T,
               \{\langle\vdash(\varphi\to\psi)\rangle,
                 \langle\vdash\varphi\rangle\},
               \langle\vdash\psi\rangle\rangle$
      \item[(Gen)] $\langle\varnothing,T,
               \{\langle\vdash\varphi\rangle\},
               \langle\vdash\forall x\,\varphi\rangle\rangle$
    \end{itemize}
\end{itemize}

While it is known that these axioms are ``metalogically complete,'' it is
not known whether they are independent (i.e.\ none is
redundant) in the metalogical sense; specifically, whether any axiom (possibly
with additional non-mandatory distinct-variable restrictions, for use with any
dummy variables in its proof) is provable from the others.  Note that
metalogical independence is a weaker requirement than independence in the
usual logical sense.  Not all of the above axioms are logically independent:
for example, C9$'$ can be proved as a metatheorem from the others, outside the
formal system, by combining the possible cases of distinct variables.

\subsection{Example~4---Adding Definitions}\index{definition}
There are several ways to add definitions to a formal system.  Probably the
most proper way is to consider definitions not as part of the formal system at
all but rather as abbreviations that are part of the expository metalogic
outside the formal system.  For convenience, though, we may use the formal
system itself to incorporate definitions, adding them as axiomatic extensions
to the system.  This could be done by adding a constant representing the
concept ``is defined as'' along with axioms for it. But there is a nicer way,
at least in this writer's opinion, that introduces definitions as direct
extensions to the language rather than as extralogical primitive notions.  We
introduce additional logical connectives and provide axioms for them.  For
systems of logic such as Examples 1 through 3, the additional axioms must be
conservative in the sense that no wff of the original system that was not a
theorem (when the initial term ``wff'' is replaced by ``$\vdash$'' of course)
becomes a theorem of the extended system.  In this example we extend Example~3
(or 2) with standard abbreviations of logic.

We extend $\mbox{\em CN}$ of Example~3 with new constants $\{\leftrightarrow,
\wedge,\vee,\exists\}$, corresponding to logical equivalence,\index{logical
equivalence ($\leftrightarrow$)}\index{biconditional ($\leftrightarrow$)}
conjunction,\index{conjunction ($\wedge$)} disjunction,\index{disjunction
($\vee$)} and the existential quantifier.\index{existential quantifier
($\exists$)}  We extend $\Gamma$ with the axiomatic statements that are
the reducts of the following pre-statements:
\begin{list}{}{\itemsep 0.0pt}
      \item[] $\langle\varnothing,T,\varnothing,
               \langle \mbox{wff\ }(\varphi\leftrightarrow\psi)\rangle\rangle$
      \item[] $\langle\varnothing,T,\varnothing,
               \langle \mbox{wff\ }(\varphi\vee\psi)\rangle\rangle$
      \item[] $\langle\varnothing,T,\varnothing,
               \langle \mbox{wff\ }(\varphi\wedge\psi)\rangle\rangle$
      \item[] $\langle\varnothing,T,\varnothing,
               \langle \mbox{wff\ }\exists x\, \varphi\rangle\rangle$
  \item[] $\langle\varnothing,T,\varnothing,
     \langle\vdash ( ( \varphi \leftrightarrow \psi ) \to
     ( \varphi \to \psi ) )\rangle\rangle$
  \item[] $\langle\varnothing,T,\varnothing,
     \langle\vdash ((\varphi\leftrightarrow\psi)\to
    (\psi\to\varphi))\rangle\rangle$
  \item[] $\langle\varnothing,T,\varnothing,
     \langle\vdash ((\varphi\to\psi)\to(
     (\psi\to\varphi)\to(\varphi
     \leftrightarrow\psi)))\rangle\rangle$
  \item[] $\langle\varnothing,T,\varnothing,
     \langle\vdash (( \varphi \wedge \psi ) \leftrightarrow\neg ( \varphi
     \to \neg \psi )) \rangle\rangle$
  \item[] $\langle\varnothing,T,\varnothing,
     \langle\vdash (( \varphi \vee \psi ) \leftrightarrow (\neg \varphi
     \to \psi )) \rangle\rangle$
  \item[] $\langle\varnothing,T,\varnothing,
     \langle\vdash (\exists x \,\varphi\leftrightarrow
     \lnot \forall x \lnot \varphi)\rangle\rangle$
\end{list}
The first three logical axioms (statements containing ``$\vdash$'') introduce
and effectively define logical equivalence, ``$\leftrightarrow$''.  The last
three use ``$\leftrightarrow$'' to effectively mean ``is defined as.''

\subsection{Example~5---ZFC Set Theory}\index{ZFC set theory}

Here we add to the system of Example~4 the axioms of Zermelo--Fraenkel set
theory with Choice.  For convenience we make use of the
definitions in Example~4.

In the $\mbox{\em CN}$ of Example~4 (which extends Example~3), we replace the symbol $R$
with the symbol $\in$.
More explicitly, we remove from $\Gamma$ of Example~4 the three
axiomatic statements containing $R$ and replace them with the
reducts of the following:
\begin{list}{}{\itemsep 0.0pt}
      \item[] $\langle\varnothing,T,\varnothing,
               \langle \mbox{wff\ }x\in y\rangle\rangle$
      \item[] $\langle\varnothing,T,
               \varnothing,
               \langle \vdash(x=y\to(x\in z\to y\in z))\rangle\rangle$
      \item[] $\langle\varnothing,T,
               \varnothing,
               \langle \vdash(x=y\to(z\in x\to z\in y))\rangle\rangle$
\end{list}
Letting $D=\{\{\alpha,\beta\}\in \mbox{\em DV}\,|\alpha,\beta\in \mbox{\em
Vr}\}$ (in other words all individual variables must be distinct), we extend
$\Gamma$ with the ZFC axioms, called
\index{Axiom of Extensionality}
\index{Axiom of Replacement}
\index{Axiom of Union}
\index{Axiom of Power Sets}
\index{Axiom of Regularity}
\index{Axiom of Infinity}
\index{Axiom of Choice}
Extensionality, Replacement, Union, Power
Set, Regularity, Infinity, and Choice, that are the reducts of:
\begin{list}{}{\itemsep 0.0pt}
      \item[Ext] $\langle D,T,
               \varnothing,
               \langle\vdash (\forall x(x\in y\leftrightarrow x \in z)\to y
               =z) \rangle\rangle$
      \item[Rep] $\langle D,T,
               \varnothing,
               \langle\vdash\exists x ( \exists y \forall z (\varphi \to z = y
                        ) \to
                        \forall z ( z \in x \leftrightarrow \exists x ( x \in
                        y \wedge \forall y\,\varphi ) ) )\rangle\rangle$
      \item[Un] $\langle D,T,
               \varnothing,
               \langle\vdash \exists x \forall y ( \exists x ( y \in x \wedge
               x \in z ) \to y \in x ) \rangle\rangle$
      \item[Pow] $\langle D,T,
               \varnothing,
               \langle\vdash \exists x \forall y ( \forall x ( x \in y \to x
               \in z ) \to y \in x ) \rangle\rangle$
      \item[Reg] $\langle D,T,
               \varnothing,
               \langle\vdash (  x \in y \to
                 \exists x ( x \in y \wedge \forall z ( z \in x \to \lnot z
                \in y ) ) ) \rangle\rangle$
      \item[Inf] $\langle D,T,
               \varnothing,
               \langle\vdash \exists x(y\in x\wedge\forall y(y\in
               x\to
               \exists z(y \in z\wedge z\in x))) \rangle\rangle$
      \item[AC] $\langle D,T,
               \varnothing,
               \langle\vdash \exists x \forall y \forall z ( ( y \in z
               \wedge z \in w ) \to \exists w \forall y ( \exists w
              ( ( y \in z \wedge z \in w ) \wedge ( y \in w \wedge w \in x
              ) ) \leftrightarrow y = w ) ) \rangle\rangle$
\end{list}

\subsection{Example~6---Class Notation in Set Theory}\label{class}

A powerful device that makes set theory easier (and that we have
been using all along in our informal expository language) is {\em class
abstraction notation}.\index{class abstraction}\index{abstraction class}  The
definitions we introduce are rigorously justified
as conservative by Takeuti and Zaring \cite{Takeuti}\index{Takeuti, Gaisi} or
Quine \cite{Quine}\index{Quine, Willard Van Orman}.  The key idea is to
introduce the notation $\{x|\mbox{---}\}$ which means ``the class of all $x$
such that ---'' for abstraction classes and introduce (meta)variables that
range over them.  An abstraction class may or may not be a set, depending on
whether it exists (as a set).  A class that does not exist is
called a {\em proper class}.\index{proper class}\index{class!proper}

To illustrate the use of abstraction classes we will provide some examples
of definitions that make use of them:  the empty set, class union, and
unordered pair.  Many other such definitions can be found in the
Metamath set theory database,
\texttt{set.mm}.\index{set theory database (\texttt{set.mm})}

% We intentionally break up the sequence of math symbols here
% because otherwise the overlong line goes beyond the page in narrow mode.
We extend $\mbox{\em CN}$ of Example~5 with new symbols $\{$
$\mbox{class},$ $\{,$ $|,$ $\},$ $\varnothing,$ $\cup,$ $,$ $\}$
where the inner braces and last comma are
constant symbols. (As before,
our dual use of some mathematical symbols for both our expository
language and as primitives of the formal system should be clear from context.)

We extend $\mbox{\em VR}$ of Example~5 with a set of {\em class
variables}\index{class variable}
$\{A,B,C,\ldots\}$. We extend the $T$ of Example~5 with $\{\langle
\mbox{class\ } A\rangle, \langle \mbox{class\ }B\rangle, \langle \mbox{class\ }
C\rangle,\ldots\}$.

To
introduce our definitions,
we add to $\Gamma$ of Example~5 the axiomatic statements
that are the reducts of the following pre-statements:
\begin{list}{}{\itemsep 0.0pt}
      \item[] $\langle\varnothing,T,\varnothing,
               \langle \mbox{class\ }x\rangle\rangle$
      \item[] $\langle\varnothing,T,\varnothing,
               \langle \mbox{class\ }\{x|\varphi\}\rangle\rangle$
      \item[] $\langle\varnothing,T,\varnothing,
               \langle \mbox{wff\ }A=B\rangle\rangle$
      \item[] $\langle\varnothing,T,\varnothing,
               \langle \mbox{wff\ }A\in B\rangle\rangle$
      \item[Ab] $\langle\varnothing,T,\varnothing,
               \langle \vdash ( y \in \{ x |\varphi\} \leftrightarrow
                  ( ( x = y \to\varphi) \wedge \exists x ( x = y
                  \wedge\varphi) ))
               \rangle\rangle$
      \item[Eq] $\langle\{\{x,A\},\{x,B\}\},T,\varnothing,
               \langle \vdash ( A = B \leftrightarrow
               \forall x ( x \in A \leftrightarrow x \in B ) )
               \rangle\rangle$
      \item[El] $\langle\{\{x,A\},\{x,B\}\},T,\varnothing,
               \langle \vdash ( A \in B \leftrightarrow \exists x
               ( x = A \wedge x \in B ) )
               \rangle\rangle$
\end{list}
Here we say that an individual variable is a class; $\{x|\varphi\}$ is a
class; and we extend the definition of a wff to include class equality and
membership.  Axiom Ab defines membership of a variable in a class abstraction;
the right-hand side can be read as ``the wff that results from proper
substitution of $y$ for $x$ in $\varphi$.''\footnote{Note that this definition
makes unnecessary the introduction of a separate notation similar to
$\varphi(x|y)$ for proper substitution, although we may choose to do so to be
conventional.  Incidentally, $\varphi(x|y)$ as it stands would be ambiguous in
the formal systems of our examples, since we wouldn't know whether
$\lnot\varphi(x|y)$ meant $\lnot(\varphi(x|y))$ or $(\lnot\varphi)(x|y)$.
Instead, we would have to use an unambiguous variant such as $(\varphi\,
x|y)$.}  Axioms Eq and El extend the meaning of the existing equality and
membership connectives.  This is potentially dangerous and requires careful
justification.  For example, from Eq we can derive the Axiom of Extensionality
with predicate logic alone; thus in principle we should include the Axiom of
Extensionality as a logical hypothesis.  However we do not bother to do this
since we have already presupposed that axiom earlier. The distinct variable
restrictions should be read ``where $x$ does not occur in $A$ or $B$.''  We
typically do this when the right-hand side of a definition involves an
individual variable not in the expression being defined; it is done so that
the right-hand side remains independent of the particular ``dummy'' variable
we use.

We continue to add to $\Gamma$ the following definitions
(i.e. the reducts of the following pre-statements) for empty
set,\index{empty set} class union,\index{union} and unordered
pair.\index{unordered pair}  They should be self-explanatory.  Analogous to our
use of ``$\leftrightarrow$'' to define new wffs in Example~4, we use ``$=$''
to define new abstraction terms, and both may be read informally as ``is
defined as'' in this context.
\begin{list}{}{\itemsep 0.0pt}
      \item[] $\langle\varnothing,T,\varnothing,
               \langle \mbox{class\ }\varnothing\rangle\rangle$
      \item[] $\langle\varnothing,T,\varnothing,
               \langle \vdash \varnothing = \{ x | \lnot x = x \}
               \rangle\rangle$
      \item[] $\langle\varnothing,T,\varnothing,
               \langle \mbox{class\ }(A\cup B)\rangle\rangle$
      \item[] $\langle\{\{x,A\},\{x,B\}\},T,\varnothing,
               \langle \vdash ( A \cup B ) = \{ x | ( x \in A \vee x \in B ) \}
               \rangle\rangle$
      \item[] $\langle\varnothing,T,\varnothing,
               \langle \mbox{class\ }\{A,B\}\rangle\rangle$
      \item[] $\langle\{\{x,A\},\{x,B\}\},T,\varnothing,
               \langle \vdash \{ A , B \} = \{ x | ( x = A \vee x = B ) \}
               \rangle\rangle$
\end{list}

\section{Metamath as a Formal System}\label{theorymm}

This section presupposes a familiarity with the Metamath computer language.

Our theory describes formal systems and their universes.  The Metamath
language provides a way of representing these set-theoretical objects to
a computer.  A Metamath database, being a finite set of {\sc ascii}
characters, can usually describe only a subset of a formal system and
its universe, which are typically infinite.  However the database can
contain as large a finite subset of the formal system and its universe
as we wish.  (Of course a Metamath set theory database can, in
principle, indirectly describe an entire infinite formal system by
formalizing the expository language in this Appendix.)

For purpose of our discussion, we assume the Metamath database
is in the simple form described on p.~\pageref{framelist},
consisting of all constant and variable declarations at the beginning,
followed by a sequence of extended frames each
delimited by \texttt{\$\char`\{} and \texttt{\$\char`\}}.  Any Metamath database can
be converted to this form, as described on p.~\pageref{frameconvert}.

The math symbol tokens of a Metamath source file, which are declared
with \texttt{\$c} and \texttt{\$v} statements, are names we assign to
representatives of $\mbox{\em CN}$ and $\mbox{\em VR}$.  For
definiteness we could assume that the first math symbol declared as a
variable corresponds to $v_0$, the second to $v_1$, etc., although the
exact correspondence we choose is not important.

In the Metamath language, each \texttt{\$d}, \texttt{\$f}, and
 \texttt{\$e} source
statement in an extended frame (Section~\ref{frames})
corresponds respectively to a member of the
collections $D$, $T$, and $H$ in a formal system statement $\langle
D_M,T_M,H,A\rangle$.  The math symbol strings following these Metamath keywords
correspond to a variable pair (in the case of \texttt{\$d}) or an expression (for
the other two keywords). The math symbol string following a \texttt{\$a} source
statement corresponds to expression $A$ in an axiomatic statement of the
formal system; the one following a \texttt{\$p} source statement corresponds to
$A$ in a provable statement that is not axiomatic.  In other words, each
extended frame in a Metamath database corresponds to
a pre-statement of the formal system, and a frame corresponds to
a statement of the formal system.  (Don't confuse the two meanings of
``statement'' here.  A statement of the formal system corresponds to the
several statements in a Metamath database that may constitute a
frame.)

In order for the computer to verify that a formal system statement is
provable, each \texttt{\$p} source statement is accompanied by a proof.
However, the proof does not correspond to anything in the formal system
but is simply a way of communicating to the computer the information
needed for its verification.  The proof tells the computer {\em how to
construct} specific members of closure of the formal system
pre-statement corresponding to the extended frame of the \texttt{\$p}
statement.  The final result of the construction is the member of the
closure that matches the \texttt{\$p} statement.  The abstract formal
system, on the other hand, is concerned only with the {\em existence} of
members of the closure.

As mentioned on p.~\pageref{exampleref}, Examples 1 and 3--6 in the
previous Section parallel the development of logic and set theory in the
Metamath database
\texttt{set.mm}.\index{set theory database (\texttt{set.mm})} You may
find it instructive to compare them.


\chapter{The MIU System}
\label{MIU}
\index{formal system}
\index{MIU-system}

The following is a listing of the file \texttt{miu.mm}.  It is self-explanatory.

%%%%%%%%%%%%%%%%%%%%%%%%%%%%%%%%%%%%%%%%%%%%%%%%%%%%%%%%%%%%

\begin{verbatim}
$( The MIU-system:  A simple formal system $)

$( Note:  This formal system is unusual in that it allows
empty wffs.  To work with a proof, you must type
SET EMPTY_SUBSTITUTION ON before using the PROVE command.
By default, this is OFF in order to reduce the number of
ambiguous unification possibilities that have to be selected
during the construction of a proof.  $)

$(
Hofstadter's MIU-system is a simple example of a formal
system that illustrates some concepts of Metamath.  See
Douglas R. Hofstadter, _Goedel, Escher, Bach:  An Eternal
Golden Braid_ (Vintage Books, New York, 1979), pp. 33ff. for
a description of the MIU-system.

The system has 3 constant symbols, M, I, and U.  The sole
axiom of the system is MI. There are 4 rules:
     Rule I:  If you possess a string whose last letter is I,
     you can add on a U at the end.
     Rule II:  Suppose you have Mx.  Then you may add Mxx to
     your collection.
     Rule III:  If III occurs in one of the strings in your
     collection, you may make a new string with U in place
     of III.
     Rule IV:  If UU occurs inside one of your strings, you
     can drop it.
Unfortunately, Rules III and IV do not have unique results:
strings could have more than one occurrence of III or UU.
This requires that we introduce the concept of an "MIU
well-formed formula" or wff, which allows us to construct
unique symbol sequences to which Rules III and IV can be
applied.
$)

$( First, we declare the constant symbols of the language.
Note that we need two symbols to distinguish the assertion
that a sequence is a wff from the assertion that it is a
theorem; we have arbitrarily chosen "wff" and "|-". $)
      $c M I U |- wff $. $( Declare constants $)

$( Next, we declare some variables. $)
     $v x y $.

$( Throughout our theory, we shall assume that these
variables represent wffs. $)
 wx   $f wff x $.
 wy   $f wff y $.

$( Define MIU-wffs.  We allow the empty sequence to be a
wff. $)

$( The empty sequence is a wff. $)
 we   $a wff $.
$( "M" after any wff is a wff. $)
 wM   $a wff x M $.
$( "I" after any wff is a wff. $)
 wI   $a wff x I $.
$( "U" after any wff is a wff. $)
 wU   $a wff x U $.

$( Assert the axiom. $)
 ax   $a |- M I $.

$( Assert the rules. $)
 ${
   Ia   $e |- x I $.
$( Given any theorem ending with "I", it remains a theorem
if "U" is added after it.  (We distinguish the label I_
from the math symbol I to conform to the 24-Jun-2006
Metamath spec.) $)
   I_    $a |- x I U $.
 $}
 ${
IIa  $e |- M x $.
$( Given any theorem starting with "M", it remains a theorem
if the part after the "M" is added again after it. $)
   II   $a |- M x x $.
 $}
 ${
   IIIa $e |- x I I I y $.
$( Given any theorem with "III" in the middle, it remains a
theorem if the "III" is replaced with "U". $)
   III  $a |- x U y $.
 $}
 ${
   IVa  $e |- x U U y $.
$( Given any theorem with "UU" in the middle, it remains a
theorem if the "UU" is deleted. $)
   IV   $a |- x y $.
  $}

$( Now we prove the theorem MUIIU.  You may be interested in
comparing this proof with that of Hofstadter (pp. 35 - 36).
$)
 theorem1  $p |- M U I I U $=
      we wM wU wI we wI wU we wU wI wU we wM we wI wU we wM
      wI wI wI we wI wI we wI ax II II I_ III II IV $.
\end{verbatim}\index{well-formed formula (wff)}

The \texttt{show proof /lemmon/renumber} command
yields the following display.  It is very similar
to the one in \cite[pp.~35--36]{Hofstadter}.\index{Hofstadter, Douglas R.}

\begin{verbatim}
1 ax             $a |- M I
2 1 II           $a |- M I I
3 2 II           $a |- M I I I I
4 3 I_           $a |- M I I I I U
5 4 III          $a |- M U I U
6 5 II           $a |- M U I U U I U
7 6 IV           $a |- M U I I U
\end{verbatim}

We note that Hofstadter's ``MU-puzzle,'' which asks whether
MU is a theorem of the MIU-system, cannot be answered using
the system above because the MU-puzzle is a question {\em
about} the system.  To prove the answer to the MU-puzzle,
a much more elaborate system is needed, namely one that
models the MIU-system within set theory.  (Incidentally, the
answer to the MU-puzzle is no.)

\chapter{Metamath Language EBNF}%
\label{BNF}%
\index{Metamath Language EBNF}

The following is a formal description of the basic Metamath language syntax
(with compressed proofs and support for unknown proof steps).
It is defined using the
Extended Backus--Naur Form (EBNF)\index{Extended Backus--Naur Form}\index{EBNF}
notation from W3C\index{W3C}
\textit{Extensible Markup Language (XML) 1.0 (Fifth Edition)}
(W3C Recommendation 26 November 2008) at
\url{https://www.w3.org/TR/xml/#sec-notation}.

The \texttt{database}
rule is processed until the end of the file (\texttt{EOF}).
The rules eventually require reading whitespace-separated tokens.
A token has an upper-case definition (see below)
or is a string constant in a non-token (such as \texttt{'\$a'}).
We intend for this to be correct, but if there is a conflict the
rules of section \ref{spec} govern. That section also discusses
non-syntax restrictions not shown here
(e.g., that each new label token
defined in a \texttt{hypothesis-stmt} or \texttt{assert-stmt}
must be unique).

\begin{verbatim}
database ::= outermost-scope-stmt*

outermost-scope-stmt ::=
  include-stmt | constant-stmt | stmt

/* File inclusion command; process file as a database.
   Databases should NOT have a comment in the filename. */
include-stmt ::= '$[' filename '$]'

/* Constant symbols declaration. */
constant-stmt ::= '$c' constant+ '$.'

/* A normal statement can occur in any scope. */
stmt ::= block | variable-stmt | disjoint-stmt |
  hypothesis-stmt | assert-stmt

/* A block. You can have 0 statements in a block. */
block ::= '${' stmt* '$}'

/* Variable symbols declaration. */
variable-stmt ::= '$v' variable+ '$.'

/* Disjoint variables. Simple disjoint statements have
   2 variables, i.e., "variable*" is empty for them. */
disjoint-stmt ::= '$d' variable variable variable* '$.'

hypothesis-stmt ::= floating-stmt | essential-stmt

/* Floating (variable-type) hypothesis. */
floating-stmt ::= LABEL '$f' typecode variable '$.'

/* Essential (logical) hypothesis. */
essential-stmt ::= LABEL '$e' typecode MATH-SYMBOL* '$.'

assert-stmt ::= axiom-stmt | provable-stmt

/* Axiomatic assertion. */
axiom-stmt ::= LABEL '$a' typecode MATH-SYMBOL* '$.'

/* Provable assertion. */
provable-stmt ::= LABEL '$p' typecode MATH-SYMBOL*
  '$=' proof '$.'

/* A proof. Proofs may be interspersed by comments.
   If '?' is in a proof it's an "incomplete" proof. */
proof ::= uncompressed-proof | compressed-proof
uncompressed-proof ::= (LABEL | '?')+
compressed-proof ::= '(' LABEL* ')' COMPRESSED-PROOF-BLOCK+

typecode ::= constant

filename ::= MATH-SYMBOL /* No whitespace or '$' */
constant ::= MATH-SYMBOL
variable ::= MATH-SYMBOL
\end{verbatim}

\needspace{2\baselineskip}
A \texttt{frame} is a sequence of 0 or more
\texttt{disjoint-{\allowbreak}stmt} and
\texttt{hypotheses-{\allowbreak}stmt} statements
(possibly interleaved with other non-\texttt{assert-stmt} statements)
followed by one \texttt{assert-stmt}.

\needspace{3\baselineskip}
Here are the rules for lexical processing (tokenization) beyond
the constant tokens shown above.
By convention these tokenization rules have upper-case names.
Every token is read for the longest possible length.
Whitespace-separated tokens are read sequentially;
note that the separating whitespace and \texttt{\$(} ... \texttt{\$)}
comments are skipped.

If a token definition uses another token definition, the whole thing
is considered a single token.
A pattern that is only part of a full token has a name beginning
with an underscore (``\_'').
An implementation could tokenize many tokens as a
\texttt{PRINTABLE-SEQUENCE}
and then check if it meets the more specific rule shown here.

Comments do not nest, and both \texttt{\$(} and \texttt{\$)}
have to be surrounded
by at least one whitespace character (\texttt{\_WHITECHAR}).
Technically comments end without consuming the trailing
\texttt{\_WHITECHAR}, but the trailing
\texttt{\_WHITECHAR} gets ignored anyway so we ignore that detail here.
Metamath language processors
are not required to support \texttt{\$)} followed
immediately by a bare end-of-file, because the closing
comment symbol is supposed to be followed by a
\texttt{\_WHITECHAR} such as a newline.

\begin{verbatim}
PRINTABLE-SEQUENCE ::= _PRINTABLE-CHARACTER+

MATH-SYMBOL ::= (_PRINTABLE-CHARACTER - '$')+

/* ASCII non-whitespace printable characters */
_PRINTABLE-CHARACTER ::= [#x21-#x7e]

LABEL ::= ( _LETTER-OR-DIGIT | '.' | '-' | '_' )+

_LETTER-OR-DIGIT ::= [A-Za-z0-9]

COMPRESSED-PROOF-BLOCK ::= ([A-Z] | '?')+

/* Define whitespace between tokens. The -> SKIP
   means that when whitespace is seen, it is
   skipped and we simply read again. */
WHITESPACE ::= (_WHITECHAR+ | _COMMENT) -> SKIP

/* Comments. $( ... $) and do not nest. */
_COMMENT ::= '$(' (_WHITECHAR+ (PRINTABLE-SEQUENCE - '$)')*
  _WHITECHAR+ '$)' _WHITECHAR

/* Whitespace: (' ' | '\t' | '\r' | '\n' | '\f') */
_WHITECHAR ::= [#x20#x09#x0d#x0a#x0c]
\end{verbatim}
% This EBNF was developed as a collaboration between
% David A. Wheeler\index{Wheeler, David A.},
% Mario Carneiro\index{Carneiro, Mario}, and
% Benoit Jubin\index{Jubin, Benoit}, inspired by a request
% (and a lot of initial work) by Benoit Jubin.
%
% \chapter{Disclaimer and Trademarks}
%
% Information in this document is subject to change without notice and does not
% represent a commitment on the part of Norman Megill.
% \vspace{2ex}
%
% \noindent Norman D. Megill makes no warranties, either express or implied,
% regarding the Metamath computer software package.
%
% \vspace{2ex}
%
% \noindent Any trademarks mentioned in this book are the property of
% their respective owners.  The name ``Metamath'' is a trademark of
% Norman Megill.
%
\cleardoublepage
\phantomsection  % fixes the link anchor
\addcontentsline{toc}{chapter}{\bibname}

\bibliography{metamath}
%\input{metamath.bbl}

\raggedright
\cleardoublepage
\phantomsection % fixes the link anchor
\addcontentsline{toc}{chapter}{\indexname}
%\printindex   ??
\input{metamath.ind}

\end{document}



\raggedright
\cleardoublepage
\phantomsection % fixes the link anchor
\addcontentsline{toc}{chapter}{\indexname}
%\printindex   ??
% metamath.tex - Version of 24-May-2019
% If you change the date above, also change the "Printed date" below.
% SPDX-License-Identifier: CC0-1.0
%
%                              PUBLIC DOMAIN
%
% This file (specifically, the version of this file with the above date)
% has been released into the Public Domain per the
% Creative Commons CC0 1.0 Universal (CC0 1.0) Public Domain Dedication
% https://creativecommons.org/publicdomain/zero/1.0/
%
% The public domain release applies worldwide.  In case this is not
% legally possible, the right is granted to use the work for any purpose,
% without any conditions, unless such conditions are required by law.
%
% Several short, attributed quotations from copyrighted works
% appear in this file under the ``fair use'' provision of Section 107 of
% the United States Copyright Act (Title 17 of the {\em United States
% Code}).  The public-domain status of this file is not applicable to
% those quotations.
%
% Norman Megill - email: nm(at)alum(dot)mit(dot)edu
%
% David A. Wheeler also donates his improvements to this file to the
% public domain per the CC0.  He works at the Institute for Defense Analyses
% (IDA), but IDA has agreed that this Metamath work is outside its "lane"
% and is not a work by IDA.  This was specifically confirmed by
% Margaret E. Myers (Division Director of the Information Technology
% and Systems Division) on 2019-05-24 and by Ben Lindorf (General Counsel)
% on 2019-05-22.

% This file, 'metamath.tex', is self-contained with everything needed to
% generate the the PDF file 'metamath.pdf' (the _Metamath_ book) on
% standard LaTeX 2e installations.  The auxiliary files are embedded with
% "filecontents" commands.  To generate metamath.pdf file, run these
% commands under Linux or Cygwin in the directory that contains
% 'metamath.tex':
%
%   rm -f realref.sty metamath.bib
%   touch metamath.ind
%   pdflatex metamath
%   pdflatex metamath
%   bibtex metamath
%   makeindex metamath
%   pdflatex metamath
%   pdflatex metamath
%
% The warnings that occur in the initial runs of pdflatex can be ignored.
% For the final run,
%
%   egrep -i 'error|warn' metamath.log
%
% should show exactly these 5 warnings:
%
%   LaTeX Warning: File `realref.sty' already exists on the system.
%   LaTeX Warning: File `metamath.bib' already exists on the system.
%   LaTeX Font Warning: Font shape `OMS/cmtt/m/n' undefined
%   LaTeX Font Warning: Font shape `OMS/cmtt/bx/n' undefined
%   LaTeX Font Warning: Some font shapes were not available, defaults
%       substituted.
%
% Search for "Uncomment" below if you want to suppress hyperlink boxes
% in the PDF output file
%
% TYPOGRAPHICAL NOTES:
% * It is customary to use an en dash (--) to "connect" names of different
%   people (and to denote ranges), and use a hyphen (-) for a
%   single compound name. Examples of connected multiple people are
%   Zermelo--Fraenkel, Schr\"{o}der--Bernstein, Tarski--Grothendieck,
%   Hewlett--Packard, and Backus--Naur.  Examples of a single person with
%   a compound name include Levi-Civita, Mittag-Leffler, and Burali-Forti.
% * Use non-breaking spaces after page abbreviations, e.g.,
%   p.~\pageref{note2002}.
%
% --------------------------- Start of realref.sty -----------------------------
\begin{filecontents}{realref.sty}
% Save the following as realref.sty.
% You can then use it with \usepackage{realref}
%
% This has \pageref jumping to the page on which the ref appears,
% \ref jumping to the point of the anchor, and \sectionref
% jumping to the start of section.
%
% Author:  Anthony Williams
%          Software Engineer
%          Nortel Networks Optical Components Ltd
% Date:    9 Nov 2001 (posted to comp.text.tex)
%
% The following declaration was made by Anthony Williams on
% 24 Jul 2006 (private email to Norman Megill):
%
%   ``I hereby donate the code for realref.sty posted on the
%   comp.text.tex newsgroup on 9th November 2001, accessible from
%   http://groups.google.com/group/comp.text.tex/msg/5a0e1cc13ea7fbb2
%   to the public domain.''
%
\ProvidesPackage{realref}
\RequirePackage[plainpages=false,pdfpagelabels=true]{hyperref}
\def\realref@anchorname{}
\AtBeginDocument{%
% ensure every label is a possible hyperlink target
\let\realref@oldrefstepcounter\refstepcounter%
\DeclareRobustCommand{\refstepcounter}[1]{\realref@oldrefstepcounter{#1}
\edef\realref@anchorname{\string #1.\@currentlabel}%
}%
\let\realref@oldlabel\label%
\DeclareRobustCommand{\label}[1]{\realref@oldlabel{#1}\hypertarget{#1}{}%
\@bsphack\protected@write\@auxout{}{%
    \string\expandafter\gdef\protect\csname
    page@num.#1\string\endcsname{\thepage}%
    \string\expandafter\gdef\protect\csname
    ref@num.#1\string\endcsname{\@currentlabel}%
    \string\expandafter\gdef\protect\csname
    sectionref@name.#1\string\endcsname{\realref@anchorname}%
}\@esphack}%
\DeclareRobustCommand\pageref[1]{{\edef\a{\csname
            page@num.#1\endcsname}\expandafter\hyperlink{page.\a}{\a}}}%
\DeclareRobustCommand\ref[1]{{\edef\a{\csname
            ref@num.#1\endcsname}\hyperlink{#1}{\a}}}%
\DeclareRobustCommand\sectionref[1]{{\edef\a{\csname
            ref@num.#1\endcsname}\edef\b{\csname
            sectionref@name.#1\endcsname}\hyperlink{\b}{\a}}}%
}
\end{filecontents}
% ---------------------------- End of realref.sty ------------------------------

% --------------------------- Start of metamath.bib -----------------------------
\begin{filecontents}{metamath.bib}
@book{Albers, editor = "Donald J. Albers and G. L. Alexanderson",
  title = "Mathematical People",
  publisher = "Contemporary Books, Inc.",
  address = "Chicago",
  note = "[QA28.M37]",
  year = 1985 }
@book{Anderson, author = "Alan Ross Anderson and Nuel D. Belnap",
  title = "Entailment",
  publisher = "Princeton University Press",
  address = "Princeton",
  volume = 1,
  note = "[QA9.A634 1975 v.1]",
  year = 1975}
@book{Barrow, author = "John D. Barrow",
  title = "Theories of Everything:  The Quest for Ultimate Explanation",
  publisher = "Oxford University Press",
  address = "Oxford",
  note = "[Q175.B225]",
  year = 1991 }
@book{Behnke,
  editor = "H. Behnke and F. Backmann and K. Fladt and W. S{\"{u}}ss",
  title = "Fundamentals of Mathematics",
  volume = "I",
  publisher = "The MIT Press",
  address = "Cambridge, Massachusetts",
  note = "[QA37.2.B413]",
  year = 1974 }
@book{Bell, author = "J. L. Bell and M. Machover",
  title = "A Course in Mathematical Logic",
  publisher = "North-Holland",
  address = "Amsterdam",
  note = "[QA9.B3953]",
  year = 1977 }
@inproceedings{Blass, author = "Andrea Blass",
  title = "The Interaction Between Category Theory and Set Theory",
  pages = "5--29",
  booktitle = "Mathematical Applications of Category Theory (Proceedings
     of the Special Session on Mathematical Applications
     Category Theory, 89th Annual Meeting of the American Mathematical
     Society, held in Denver, Colorado January 5--9, 1983)",
  editor = "John Walter Gray",
  year = 1983,
  note = "[QA169.A47 1983]",
  publisher = "American Mathematical Society",
  address = "Providence, Rhode Island"}
@proceedings{Bledsoe, editor = "W. W. Bledsoe and D. W. Loveland",
  title = "Automated Theorem Proving:  After 25 Years (Proceedings
     of the Special Session on Automatic Theorem Proving,
     89th Annual Meeting of the American Mathematical
     Society, held in Denver, Colorado January 5--9, 1983)",
  year = 1983,
  note = "[QA76.9.A96.S64 1983]",
  publisher = "American Mathematical Society",
  address = "Providence, Rhode Island" }
@book{Boolos, author = "George S. Boolos and Richard C. Jeffrey",
  title = "Computability and Log\-ic",
  publisher = "Cambridge University Press",
  edition = "third",
  address = "Cambridge",
  note = "[QA9.59.B66 1989]",
  year = 1989 }
@book{Campbell, author = "John Campbell",
  title = "Programmer's Progress",
  publisher = "White Star Software",
  address = "Box 51623, Palo Alto, CA 94303",
  year = 1991 }
@article{DBLP:journals/corr/Carneiro14,
  author    = {Mario Carneiro},
  title     = {Conversion of {HOL} Light proofs into Metamath},
  journal   = {CoRR},
  volume    = {abs/1412.8091},
  year      = {2014},
  url       = {http://arxiv.org/abs/1412.8091},
  archivePrefix = {arXiv},
  eprint    = {1412.8091},
  timestamp = {Mon, 13 Aug 2018 16:47:05 +0200},
  biburl    = {https://dblp.org/rec/bib/journals/corr/Carneiro14},
  bibsource = {dblp computer science bibliography, https://dblp.org}
}
@article{CarneiroND,
  author    = {Mario Carneiro},
  title     = {Natural Deductions in the Metamath Proof Language},
  url       = {http://us.metamath.org/ocat/natded.pdf},
  year      = 2014
}
@inproceedings{Chou, author = "Shang-Ching Chou",
  title = "Proving Elementary Geometry Theorems Using {W}u's Algorithm",
  pages = "243--286",
  booktitle = "Automated Theorem Proving:  After 25 Years (Proceedings
     of the Special Session on Automatic Theorem Proving,
     89th Annual Meeting of the American Mathematical
     Society, held in Denver, Colorado January 5--9, 1983)",
  editor = "W. W. Bledsoe and D. W. Loveland",
  year = 1983,
  note = "[QA76.9.A96.S64 1983]",
  publisher = "American Mathematical Society",
  address = "Providence, Rhode Island" }
@book{Clemente, author = "Daniel Clemente Laboreo",
  title = "Introduction to natural deduction",
  year = 2014,
  url = "http://www.danielclemente.com/logica/dn.en.pdf" }
@incollection{Courant, author = "Richard Courant and Herbert Robbins",
  title = "Topology",
  pages = "573--590",
  booktitle = "The World of Mathematics, Volume One",
  editor = "James R. Newman",
  publisher = "Simon and Schuster",
  address = "New York",
  note = "[QA3.W67 1988]",
  year = 1956 }
@book{Curry, author = "Haskell B. Curry",
  title = "Foundations of Mathematical Logic",
  publisher = "Dover Publications, Inc.",
  address = "New York",
  note = "[QA9.C976 1977]",
  year = 1977 }
@book{Davis, author = "Philip J. Davis and Reuben Hersh",
  title = "The Mathematical Experience",
  publisher = "Birkh{\"{a}}user Boston",
  address = "Boston",
  note = "[QA8.4.D37 1982]",
  year = 1981 }
@incollection{deMillo,
  author = "Richard de Millo and Richard Lipton and Alan Perlis",
  title = "Social Processes and Proofs of Theorems and Programs",
  pages = "267--285",
  booktitle = "New Directions in the Philosophy of Mathematics",
  editor = "Thomas Tymoczko",
  publisher = "Birkh{\"{a}}user Boston, Inc.",
  address = "Boston",
  note = "[QA8.6.N48 1986]",
  year = 1986 }
@book{Edwards, author = "Robert E. Edwards",
  title = "A Formal Background to Mathematics",
  publisher = "Springer-Verlag",
  address = "New York",
  note = "[QA37.2.E38 v.1a]",
  year = 1979 }
@book{Enderton, author = "Herbert B. Enderton",
  title = "Elements of Set Theory",
  publisher = "Academic Press, Inc.",
  address = "San Diego",
  note = "[QA248.E5]",
  year = 1977 }
@book{Goodstein, author = "R. L. Goodstein",
  title = "Development of Mathematical Logic",
  publisher = "Springer-Verlag New York Inc.",
  address = "New York",
  note = "[QA9.G6554]",
  year = 1971 }
@book{Guillen, author = "Michael Guillen",
  title = "Bridges to Infinity",
  publisher = "Jeremy P. Tarcher, Inc.",
  address = "Los Angeles",
  note = "[QA93.G8]",
  year = 1983 }
@book{Hamilton, author = "Alan G. Hamilton",
  title = "Logic for Mathematicians",
  edition = "revised",
  publisher = "Cambridge University Press",
  address = "Cambridge",
  note = "[QA9.H298]",
  year = 1988 }
@unpublished{Harrison, author = "John Robert Harrison",
  title = "Metatheory and Reflection in Theorem Proving:
    A Survey and Critique",
  note = "Technical Report
    CRC-053.
    SRI Cambridge,
    Millers Yard, Cambridge, UK,
    1995.
    Available on the Web as
{\verb+http:+}\-{\verb+//www.cl.cam.ac.uk/users/jrh/papers/reflect.html+}"}
@TECHREPORT{Harrison-thesis,
        author          = "John Robert Harrison",
        title           = "Theorem Proving with the Real Numbers",
        institution   = "University of Cambridge Computer
                         Lab\-o\-ra\-to\-ry",
        address         = "New Museums Site, Pembroke Street, Cambridge,
                           CB2 3QG, UK",
        year            = 1996,
        number          = 408,
        type            = "Technical Report",
        note            = "Author's PhD thesis,
   available on the Web at
{\verb+http:+}\-{\verb+//www.cl.cam.ac.uk+}\-{\verb+/users+}\-{\verb+/jrh+}%
\-{\verb+/papers+}\-{\verb+/thesis.html+}"}
@book{Herrlich, author = "Horst Herrlich and George E. Strecker",
  title = "Category Theory:  An Introduction",
  publisher = "Allyn and Bacon Inc.",
  address = "Boston",
  note = "[QA169.H567]",
  year = 1973 }
@article{Hindley, author = "J. Roger Hindley and David Meredith",
  title = "Principal Type-Schemes and Condensed Detachment",
  journal = "The Journal of Symbolic Logic",
  volume = 55,
  year = 1990,
  note = "[QA.J87]",
  pages = "90--105" }
@book{Hofstadter, author = "Douglas R. Hofstadter",
  title = "G{\"{o}}del, Escher, Bach",
  publisher = "Basic Books, Inc.",
  address = "New York",
  note = "[QA9.H63 1980]",
  year = 1979 }
@article{Indrzejczak, author= "Andrzej Indrzejczak",
  title = "Natural Deduction, Hybrid Systems and Modal Logic",
  journal = "Trends in Logic",
  volume = 30,
  publisher = "Springer",
  year = 2010 }
@article{Kalish, author = "D. Kalish and R. Montague",
  title = "On {T}arski's Formalization of Predicate Logic with Identity",
  journal = "Archiv f{\"{u}}r Mathematische Logik und Grundlagenfor\-schung",
  volume = 7,
  year = 1965,
  note = "[QA.A673]",
  pages = "81--101" }
@article{Kalman, author = "J. A. Kalman",
  title = "Condensed Detachment as a Rule of Inference",
  journal = "Studia Logica",
  volume = 42,
  number = 4,
  year = 1983,
  note = "[B18.P6.S933]",
  pages = "443-451" }
@book{Kline, author = "Morris Kline",
  title = "Mathematical Thought from Ancient to Modern Times",
  publisher = "Oxford University Press",
  address = "New York",
  note = "[QA21.K516 1990 v.3]",
  year = 1972 }
@book{Klinel, author = "Morris Kline",
  title = "Mathematics, The Loss of Certainty",
  publisher = "Oxford University Press",
  address = "New York",
  note = "[QA21.K525]",
  year = 1980 }
@book{Kramer, author = "Edna E. Kramer",
  title = "The Nature and Growth of Modern Mathematics",
  publisher = "Princeton University Press",
  address = "Princeton, New Jersey",
  note = "[QA93.K89 1981]",
  year = 1981 }
@article{Knill, author = "Oliver Knill",
  title = "Some Fundamental Theorems in Mathematics",
  year = "2018",
  url = "https://arxiv.org/abs/1807.08416" }
@book{Landau, author = "Edmund Landau",
  title = "Foundations of Analysis",
  publisher = "Chelsea Publishing Company",
  address = "New York",
  edition = "second",
  note = "[QA241.L2541 1960]",
  year = 1960 }
@article{Leblanc, author = "Hugues Leblanc",
  title = "On {M}eyer and {L}ambert's Quantificational Calculus {FQ}",
  journal = "The Journal of Symbolic Logic",
  volume = 33,
  year = 1968,
  note = "[QA.J87]",
  pages = "275--280" }
@article{Lejewski, author = "Czeslaw Lejewski",
  title = "On Implicational Definitions",
  journal = "Studia Logica",
  volume = 8,
  year = 1958,
  note = "[B18.P6.S933]",
  pages = "189--208" }
@book{Levy, author = "Azriel Levy",
  title = "Basic Set Theory",
  publisher = "Dover Publications",
  address = "Mineola, NY",
  year = "2002"
}
@book{Margaris, author = "Angelo Margaris",
  title = "First Order Mathematical Logic",
  publisher = "Blaisdell Publishing Company",
  address = "Waltham, Massachusetts",
  note = "[QA9.M327]",
  year = 1967}
@book{Manin, author = "Yu I. Manin",
  title = "A Course in Mathematical Logic",
  publisher = "Springer-Verlag",
  address = "New York",
  note = "[QA9.M29613]",
  year = "1977" }
@article{Mathias, author = "Adrian R. D. Mathias",
  title = "A Term of Length 4,523,659,424,929",
  journal = "Synthese",
  volume = 133,
  year = 2002,
  note = "[Q.S993]",
  pages = "75--86" }
@article{Megill, author = "Norman D. Megill",
  title = "A Finitely Axiomatized Formalization of Predicate Calculus
     with Equality",
  journal = "Notre Dame Journal of Formal Logic",
  volume = 36,
  year = 1995,
  note = "[QA.N914]",
  pages = "435--453" }
@unpublished{Megillc, author = "Norman D. Megill",
  title = "A Shorter Equivalent of the Axiom of Choice",
  month = "June",
  note = "Unpublished",
  year = 1991 }
@article{MegillBunder, author = "Norman D. Megill and Martin W.
    Bunder",
  title = "Weaker {D}-Complete Logics",
  journal = "Journal of the IGPL",
  volume = 4,
  year = 1996,
  pages = "215--225",
  note = "Available on the Web at
{\verb+http:+}\-{\verb+//www.mpi-sb.mpg.de+}\-{\verb+/igpl+}%
\-{\verb+/Journal+}\-{\verb+/V4-2+}\-{\verb+/#Megill+}"}
}
@book{Mendelson, author = "Elliott Mendelson",
  title = "Introduction to Mathematical Logic",
  edition = "second",
  publisher = "D. Van Nostrand Company, Inc.",
  address = "New York",
  note = "[QA9.M537 1979]",
  year = 1979 }
@article{Meredith, author = "David Meredith",
  title = "In Memoriam {C}arew {A}rthur {M}eredith (1904-1976)",
  journal = "Notre Dame Journal of Formal Logic",
  volume = 18,
  year = 1977,
  note = "[QA.N914]",
  pages = "513--516" }
@article{CAMeredith, author = "C. A. Meredith",
  title = "Single Axioms for the Systems ({C},{N}), ({C},{O}) and ({A},{N})
      of the Two-Valued Propositional Calculus",
  journal = "The Journal of Computing Systems",
  volume = 3,
  year = 1953,
  pages = "155--164" }
@article{Monk, author = "J. Donald Monk",
  title = "Provability With Finitely Many Variables",
  journal = "The Journal of Symbolic Logic",
  volume = 27,
  year = 1971,
  note = "[QA.J87]",
  pages = "353--358" }
@article{Monks, author = "J. Donald Monk",
  title = "Substitutionless Predicate Logic With Identity",
  journal = "Archiv f{\"{u}}r Mathematische Logik und Grundlagenfor\-schung",
  volume = 7,
  year = 1965,
  pages = "103--121" }
  %% Took out this from above to prevent LaTeX underfull warning:
  % note = "[QA.A673]",
@book{Moore, author = "A. W. Moore",
  title = "The Infinite",
  publisher = "Routledge",
  address = "New York",
  note = "[BD411.M59]",
  year = 1989}
@book{Munkres, author = "James R. Munkres",
  title = "Topology: A First Course",
  publisher = "Prentice-Hall, Inc.",
  address = "Englewood Cliffs, New Jersey",
  note = "[QA611.M82]",
  year = 1975}
@article{Nemesszeghy, author = "E. Z. Nemesszeghy and E. A. Nemesszeghy",
  title = "On Strongly Creative Definitions:  A Reply to {V}. {F}. {R}ickey",
  journal = "Logique et Analyse (N.\ S.)",
  year = 1977,
  volume = 20,
  note = "[BC.L832]",
  pages = "111--115" }
@unpublished{Nemeti, author = "N{\'{e}}meti, I.",
  title = "Algebraizations of Quantifier Logics, an Overview",
  note = "Version 11.4, preprint, Mathematical Institute, Budapest,
    1994.  A shortened version without proofs appeared in
    ``Algebraizations of quantifier logics, an introductory overview,''
   {\em Studia Logica}, 50:485--569, 1991 [B18.P6.S933]"}
@article{Pavicic, author = "M. Pavi{\v{c}}i{\'{c}}",
  title = "A New Axiomatization of Unified Quantum Logic",
  journal = "International Journal of Theoretical Physics",
  year = 1992,
  volume = 31,
  note = "[QC.I626]",
  pages = "1753 --1766" }
@book{Penrose, author = "Roger Penrose",
  title = "The Emperor's New Mind",
  publisher = "Oxford University Press",
  address = "New York",
  note = "[Q335.P415]",
  year = 1989 }
@book{PetersonI, author = "Ivars Peterson",
  title = "The Mathematical Tourist",
  publisher = "W. H. Freeman and Company",
  address = "New York",
  note = "[QA93.P475]",
  year = 1988 }
@article{Peterson, author = "Jeremy George Peterson",
  title = "An automatic theorem prover for substitution and detachment systems",
  journal = "Notre Dame Journal of Formal Logic",
  volume = 19,
  year = 1978,
  note = "[QA.N914]",
  pages = "119--122" }
@book{Quine, author = "Willard Van Orman Quine",
  title = "Set Theory and Its Logic",
  edition = "revised",
  publisher = "The Belknap Press of Harvard University Press",
  address = "Cambridge, Massachusetts",
  note = "[QA248.Q7 1969]",
  year = 1969 }
@article{Robinson, author = "J. A. Robinson",
  title = "A Machine-Oriented Logic Based on the Resolution Principle",
  journal = "Journal of the Association for Computing Machinery",
  year = 1965,
  volume = 12,
  pages = "23--41" }
@article{RobinsonT, author = "T. Thacher Robinson",
  title = "Independence of Two Nice Sets of Axioms for the Propositional
    Calculus",
  journal = "The Journal of Symbolic Logic",
  volume = 33,
  year = 1968,
  note = "[QA.J87]",
  pages = "265--270" }
@book{Rucker, author = "Rudy Rucker",
  title = "Infinity and the Mind:  The Science and Philosophy of the
    Infinite",
  publisher = "Bantam Books, Inc.",
  address = "New York",
  note = "[QA9.R79 1982]",
  year = 1982 }
@book{Russell, author = "Bertrand Russell",
  title = "Mysticism and Logic, and Other Essays",
  publisher = "Barnes \& Noble Books",
  address = "Totowa, New Jersey",
  note = "[B1649.R963.M9 1981]",
  year = 1981 }
@article{Russell2, author = "Bertrand Russell",
  title = "Recent Work on the Principles of Mathematics",
  journal = "International Monthly",
  volume = 4,
  year = 1901,
  pages = "84"}
@article{Schmidt, author = "Eric Schmidt",
  title = "Reductions in Norman Megill's axiom system for complex numbers",
  url = "http://us.metamath.org/downloads/schmidt-cnaxioms.pdf",
  year = "2012" }
@book{Shoenfield, author = "Joseph R. Shoenfield",
  title = "Mathematical Logic",
  publisher = "Addison-Wesley Publishing Company, Inc.",
  address = "Reading, Massachusetts",
  year = 1967,
  note = "[QA9.S52]" }
@book{Smullyan, author = "Raymond M. Smullyan",
  title = "Theory of Formal Systems",
  publisher = "Princeton University Press",
  address = "Princeton, New Jersey",
  year = 1961,
  note = "[QA248.5.S55]" }
@book{Solow, author = "Daniel Solow",
  title = "How to Read and Do Proofs:  An Introduction to Mathematical
    Thought Process",
  publisher = "John Wiley \& Sons",
  address = "New York",
  year = 1982,
  note = "[QA9.S577]" }
@book{Stark, author = "Harold M. Stark",
  title = "An Introduction to Number Theory",
  publisher = "Markham Publishing Company",
  address = "Chicago",
  note = "[QA241.S72 1978]",
  year = 1970 }
@article{Swart, author = "E. R. Swart",
  title = "The Philosophical Implications of the Four-Color Problem",
  journal = "American Mathematical Monthly",
  year = 1980,
  volume = 87,
  month = "November",
  note = "[QA.A5125]",
  pages = "697--707" }
@book{Szpiro, author = "George G. Szpiro",
  title = "Poincar{\'{e}}'s Prize: The Hundred-Year Quest to Solve One
    of Math's Greatest Puzzles",
  publisher = "Penguin Books Ltd",
  address = "London",
  note = "[QA43.S985 2007]",
  year = 2007}
@book{Takeuti, author = "Gaisi Takeuti and Wilson M. Zaring",
  title = "Introduction to Axiomatic Set Theory",
  edition = "second",
  publisher = "Springer-Verlag New York Inc.",
  address = "New York",
  note = "[QA248.T136 1982]",
  year = 1982}
@inproceedings{Tarski, author = "Alfred Tarski",
  title = "What is Elementary Geometry",
  pages = "16--29",
  booktitle = "The Axiomatic Method, with Special Reference to Geometry and
     Physics (Proceedings of an International Symposium held at the University
     of California, Berkeley, December 26, 1957 --- January 4, 1958)",
  editor = "Leon Henkin and Patrick Suppes and Alfred Tarski",
  year = 1959,
  publisher = "North-Holland Publishing Company",
  address = "Amsterdam"}
@article{Tarski1965, author = "Alfred Tarski",
  title = "A Simplified Formalization of Predicate Logic with Identity",
  journal = "Archiv f{\"{u}}r Mathematische Logik und Grundlagenforschung",
  volume = 7,
  year = 1965,
  note = "[QA.A673]",
  pages = "61--79" }
@book{Tymoczko,
  title = "New Directions in the Philosophy of Mathematics",
  editor = "Thomas Tymoczko",
  publisher = "Birkh{\"{a}}user Boston, Inc.",
  address = "Boston",
  note = "[QA8.6.N48 1986]",
  year = 1986 }
@incollection{Wang,
  author = "Hao Wang",
  title = "Theory and Practice in Mathematics",
  pages = "129--152",
  booktitle = "New Directions in the Philosophy of Mathematics",
  editor = "Thomas Tymoczko",
  publisher = "Birkh{\"{a}}user Boston, Inc.",
  address = "Boston",
  note = "[QA8.6.N48 1986]",
  year = 1986 }
@manual{Webster,
  title = "Webster's New Collegiate Dictionary",
  organization = "G. \& C. Merriam Co.",
  address = "Springfield, Massachusetts",
  note = "[PE1628.W4M4 1977]",
  year = 1977 }
@manual{Whitehead, author = "Alfred North Whitehead",
  title = "An Introduction to Mathematics",
  year = 1911 }
@book{PM, author = "Alfred North Whitehead and Bertrand Russell",
  title = "Principia Mathematica",
  edition = "second",
  publisher = "Cambridge University Press",
  address = "Cambridge",
  year = "1927",
  note = "(3 vols.) [QA9.W592 1927]" }
@article{DBLP:journals/corr/Whalen16,
  author    = {Daniel Whalen},
  title     = {Holophrasm: a neural Automated Theorem Prover for higher-order logic},
  journal   = {CoRR},
  volume    = {abs/1608.02644},
  year      = {2016},
  url       = {http://arxiv.org/abs/1608.02644},
  archivePrefix = {arXiv},
  eprint    = {1608.02644},
  timestamp = {Mon, 13 Aug 2018 16:46:19 +0200},
  biburl    = {https://dblp.org/rec/bib/journals/corr/Whalen16},
  bibsource = {dblp computer science bibliography, https://dblp.org} }
@article{Wiedijk-revisited,
  author = {Freek Wiedijk},
  title = {The QED Manifesto Revisited},
  year = {2007},
  url = {http://mizar.org/trybulec65/8.pdf} }
@book{Wolfram,
  author = "Stephen Wolfram",
  title = "Mathematica:  A System for Doing Mathematics by Computer",
  edition = "second",
  publisher = "Addison-Wesley Publishing Co.",
  address = "Redwood City, California",
  note = "[QA76.95.W65 1991]",
  year = 1991 }
@book{Wos, author = "Larry Wos and Ross Overbeek and Ewing Lusk and Jim Boyle",
  title = "Automated Reasoning:  Introduction and Applications",
  edition = "second",
  publisher = "McGraw-Hill, Inc.",
  address = "New York",
  note = "[QA76.9.A96.A93 1992]",
  year = 1992 }

%
%
%[1] Church, Alonzo, Introduction to Mathematical Logic,
% Volume 1, Princeton University Press, Princeton, N. J., 1956.
%
%[2] Cohen, Paul J., Set Theory and the Continuum Hypothesis,
% W. A. Benjamin, Inc., Reading, Mass., 1966.
%
%[3] Hamilton, Alan G., Logic for Mathematicians, Cambridge
% University Press,
% Cambridge, 1988.

%[6] Kleene, Stephen Cole, Introduction to Metamathematics, D.  Van
% Nostrand Company, Inc., Princeton (1952).

%[13] Tarski, Alfred, "A simplified formalization of predicate
% logic with identity," Archiv fur Mathematische Logik und
% Grundlagenforschung, vol. 7 (1965), pp. 61-79.

%[14] Tarski, Alfred and Steven Givant, A Formalization of Set
% Theory Without Variables, American Mathematical Society Colloquium
% Publications, vol. 41, American Mathematical Society,
% Providence, R. I., 1987.

%[15] Zeman, J. J., Modal Logic, Oxford University Press, Oxford, 1973.
\end{filecontents}
% --------------------------- End of metamath.bib -----------------------------


%Book: Metamath
%Author:  Norman Megill Email:  nm at alum.mit.edu
%Author:  David A. Wheeler Email:  dwheeler at dwheeler.com

% A book template example
% http://www.stsci.edu/ftp/software/tex/bookstuff/book.template

\documentclass[leqno]{book} % LaTeX 2e. 10pt. Use [leqno,12pt] for 12pt
% hyperref 2002/05/27 v6.72r  (couldn't get pagebackref to work)
\usepackage[plainpages=false,pdfpagelabels=true]{hyperref}

\usepackage{needspace}     % Enable control over page breaks
\usepackage{breqn}         % automatic equation breaking
\usepackage{microtype}     % microtypography, reduces hyphenation

% Packages for flexible tables.  We need to be able to
% wrap text within a cell (with automatically-determined widths) AND
% split a table automatically across multiple pages.
% * "tabularx" wraps text in cells but only 1 page
% * "longtable" goes across pages but by itself is incompatible with tabularx
% * "ltxtable" combines longtable and tabularx, but table contents
%    must be in a separate file.
% * "ltablex" combines tabularx and longtable - must install specially
% * "booktabs" is recommended as a way to improve the look of tables,
%   but doesn't add these capabilities.
% * "tabu" much more capable and seems to be recommended. So use that.

\usepackage{makecell}      % Enable forced line splits within a table cell
\usepackage{longtable}     % Enable multi-page tables
\usepackage{tabu}          % Multi-page tables with wrapped text in a cell

% You can find more Tex packages using commands like:
% tlmgr search --file tabu.sty
% find /usr/share/texmf-dist/ -name '*tab*'
%
%%%%%%%%%%%%%%%%%%%%%%%%%%%%%%%%%%%%%%%%%%%%%%%%%%%%%%%%%%%%%%%%%%%%%%%%%%%%
% Uncomment the next 3 lines to suppress boxes and colors on the hyperlinks
%%%%%%%%%%%%%%%%%%%%%%%%%%%%%%%%%%%%%%%%%%%%%%%%%%%%%%%%%%%%%%%%%%%%%%%%%%%%
%\hypersetup{
%colorlinks,citecolor=black,filecolor=black,linkcolor=black,urlcolor=black
%}
%
\usepackage{realref}

% Restarting page numbers: try?
%   \printglossary
%   \cleardoublepage
%   \pagenumbering{arabic}
%   \setcounter{page}{1}    ???needed
%   \include{chap1}

% not used:
% \def\R2Lurl#1#2{\mbox{\href{#1}\texttt{#2}}}

\usepackage{amssymb}

% Version 1 of book: margins: t=.4, b=.2, ll=.4, rr=.55
% \usepackage{anysize}
% % \papersize{<height>}{<width>}
% % \marginsize{<left>}{<right>}{<top>}{<bottom>}
% \papersize{9in}{6in}
% % l/r 0.6124-0.6170 works t/b 0.2418-0.3411 = 192pp. 0.2926-03118=exact
% \marginsize{0.7147in}{0.5147in}{0.4012in}{0.2012in}

\usepackage{anysize}
% \papersize{<height>}{<width>}
% \marginsize{<left>}{<right>}{<top>}{<bottom>}
\papersize{9in}{6in}
% l/r 0.85in&0.6431-0.6539 works t/b ?-?
%\marginsize{0.85in}{0.6485in}{0.55in}{0.35in}
\marginsize{0.8in}{0.65in}{0.5in}{0.3in}

% \usepackage[papersize={3.6in,4.8in},hmargin=0.1in,vmargin={0.1in,0.1in}]{geometry}  % page geometry
\usepackage{special-settings}

\raggedbottom
\makeindex

\begin{document}
% Discourage page widows and orphans:
\clubpenalty=300
\widowpenalty=300

%%%%%%% load in AMS fonts %%%%%%% % LaTeX 2.09 - obsolete in LaTeX 2e
%\input{amssym.def}
%\input{amssym.tex}
%\input{c:/texmf/tex/plain/amsfonts/amssym.def}
%\input{c:/texmf/tex/plain/amsfonts/amssym.tex}

\bibliographystyle{plain}
\pagenumbering{roman}
\pagestyle{headings}

\thispagestyle{empty}

\hfill
\vfill

\begin{center}
{\LARGE\bf Metamath} \\
\vspace{1ex}
{\large A Computer Language for Mathematical Proofs} \\
\vspace{7ex}
{\large Norman Megill} \\
\vspace{7ex}
with extensive revisions by \\
\vspace{1ex}
{\large David A. Wheeler} \\
\vspace{7ex}
% Printed date. If changing the date below, also fix the date at the beginning.
2019-05-24
\end{center}

\vfill
\hfill

\newpage
\thispagestyle{empty}

\hfill
\vfill

\begin{center}
$\sim$\ {\sc Public Domain}\ $\sim$

\vspace{2ex}
This book (including its later revisions)
has been released into the Public Domain by Norman Megill per the
Creative Commons CC0 1.0 Universal (CC0 1.0) Public Domain Dedication
(\url{https://creativecommons.org/publicdomain/zero/1.0/}).
David A. Wheeler has done the same.
The public domain release applies worldwide.  In case this is not
legally possible, the right is granted to use the work for any purpose,
without any conditions, unless such conditions are required by law.

\vspace{3ex}
Several short, attributed quotations from copyrighted works
appear in this book under the ``fair use'' provision of Section 107 of
the United States Copyright Act (Title 17 of the {\em United States
Code}).  The public-domain status of this book is not applicable to
those quotations.

\vspace{3ex}
Any trademarks used in this book are the property of their owners.

% QA76.9.L63.M??

% \vspace{1ex}
%
% \vspace{1ex}
% {\small Permission is granted to make and distribute verbatim copies of this
% book
% provided the copyright notice and this
% permission notice are preserved on all copies.}
%
% \vspace{1ex}
% {\small Permission is granted to copy and distribute modified versions of this
% book under the conditions for verbatim copying, provided that the
% entire
% resulting derived work is distributed under the terms of a permission
% notice
% identical to this one.}
%
% \vspace{1ex}
% {\small Permission is granted to copy and distribute translations of this
% book into another language, under the above conditions for modified
% versions,
% except that this permission notice may be stated in a translation
% approved by the
% author.}
%
% \vspace{1ex}
% %{\small   For a copy of the \LaTeX\ source files for this book, contact
% %the author.} \\
% \ \\
% \ \\

\vspace{7ex}
% ISBN: 1-4116-3724-0 \\
ISBN: 978-1-4116-3724-5 \\
{\ } \\
Lulu Press \\
Morrisville, North Carolina\\
USA


\hfill
\vfill

Norman Megill\\ 93 Bridge St., Lexington, MA 02421 \\
E-mail address: \texttt{nm{\char`\@}alum.mit.edu} \\
\vspace{7ex}
David A. Wheeler \\
E-mail address: \texttt{dwheeler{\char`\@}dwheeler.com} \\
% See notes added at end of Preface for revision history. \\
% For current information on the Metamath software see \\
\vspace{7ex}
\url{http://metamath.org}
\end{center}

% \newpage
% \thispagestyle{empty}
%
% \hfill
% \vfill
%
% \begin{center}
% {\it To my son Robin Dwight Megill}
% \end{center}
%
% \vfill
% \hfill
%
% \newpage

\tableofcontents
%\listoftables

\chapter*{Preface}
\markboth{PREFACE}{PREFACE}
\addcontentsline{toc}{section}{Preface}


% (For current information, see the notes added at the
% end of this preface on p.~\pageref{note2002}.)

\subsubsection{Overview}

Metamath\index{Metamath} is a computer language and an associated computer
program for archiving, verifying, and studying mathematical proofs at a very
detailed level.  The Metamath language incorporates no mathematics per se but
treats all mathematical statements as mere sequences of symbols.  You provide
Metamath with certain special sequences (axioms) that tell it what rules
of inference are allowed.  Metamath is not limited to any specific field of
mathematics.  The Metamath language is simple and robust, with an
almost total absence of hard-wired syntax, and
I\footnote{Unless otherwise noted, the words
``I,'' ``me,'' and ``my'' refer to Norman Megill\index{Megill, Norman}, while
``we,'' ``us,'' and ``our'' refer to Norman Megill and
David A. Wheeler\index{Wheeler, David A.}.}
believe that it
provides about the simplest possible framework that allows essentially all of
mathematics to be expressed with absolute rigor.

% index test
%\newcommand{\nn}[1]{#1n}
%\index{aaa@bbb}
%\index{abc!def}
%\index{abd|see{qqq}}
%\index{abe|nn}
%\index{abf|emph}
%\index{abg|(}
%\index{abg|)}

Using the Metamath language, you can build formal or mathematical
systems\index{formal system}\footnote{A formal or mathematical system consists
of a collection of symbols (such as $2$, $4$, $+$ and $=$), syntax rules that
describe how symbols may be combined to form a legal expression (called a
well-formed formula or {\em wff}, pronounced ``whiff''), some starting wffs
called axioms, and inference rules that describe how theorems may be derived
(proved) from the axioms.  A theorem is a mathematical fact such as $2+2=4$.
Strictly speaking, even an obvious fact such as this must be proved from
axioms to be formally acceptable to a mathematician.}\index{theorem}
\index{axiom}\index{rule}\index{well-formed formula (wff)} that involve
inferences from axioms.  Although a database is provided
that includes a recommended set of axioms for standard mathematics, if you
wish you can supply your own symbols, syntax, axioms, rules, and definitions.

The name ``Metamath'' was chosen to suggest that the language provides a
means for {\em describing} mathematics rather than {\em being} the
mathematics itself.  Actually in some sense any mathematical language is
metamathematical.  Symbols written on paper, or stored in a computer,
are not mathematics itself but rather a way of expressing mathematics.
For example ``7'' and ``VII'' are symbols for denoting the number seven
in Arabic and Roman numerals; neither {\em is} the number seven.

If you are able to understand and write computer programs, you should be able
to follow abstract mathematics with the aid of Metamath.  Used in conjunction
with standard textbooks, Metamath can guide you step-by-step towards an
understanding of abstract mathematics from a very rigorous viewpoint, even if
you have no formal abstract mathematics background.  By using a single,
consistent notation to express proofs, once you grasp its basic concepts
Metamath provides you with the ability to immediately follow and dissect
proofs even in totally unfamiliar areas.

Of course, just being able follow a proof will not necessarily give you an
intuitive familiarity with mathematics.  Memorizing the rules of chess does not
give you the ability to appreciate the game of a master, and knowing how the
notes on a musical score map to piano keys does not give you the ability to
hear in your head how it would sound.  But each of these can be a first step.

Metamath allows you to explore proofs in the sense that you can see the
theorem referenced at any step expanded in as much detail as you want, right
down to the underlying axioms of logic and set theory (in the case of the set
theory database provided).  While Metamath will not replace the higher-level
understanding that can only be acquired through exercises and hard work, being
able to see how gaps in a proof are filled in can give you increased
confidence that can speed up the learning process and save you time when you
get stuck.

The Metamath language breaks down a mathematical proof into its tiniest
possible parts.  These can be pieced together, like interlocking
pieces in a puzzle, only in a way that produces correct and absolutely rigorous
mathematics.

The nature of Metamath\index{Metamath} enforces very precise mathematical
thinking, similar to that involved in writing a computer program.  A crucial
difference, though, is that once a proof is verified (by the Metamath program)
to be correct, it is definitely correct; it can never have a hidden
``bug.''\index{computer program bugs}  After getting used to the kind of rigor
and accuracy provided by Metamath, you might even be tempted to
adopt the attitude that a proof should never be considered correct until it
has been verified by a computer, just as you would not completely trust a
manual calculation until you have verified it on a
calculator.

My goal
for Metamath was a system for describing and verifying
mathematics that is completely universal yet conceptually as simple as
possible.  In approaching mathematics from an axiomatic, formal viewpoint, I
wanted Metamath to be able to handle almost any mathematical system, not
necessarily with ease, but at least in principle and hopefully in practice. I
wanted it to verify proofs with absolute rigor, and for this reason Metamath
is what might be thought of as a ``compile-only'' language rather than an
algorithmic or Turing-machine language (Pascal, C, Prolog, Mathematica,
etc.).  In other words, a database written in the Metamath
language doesn't ``do'' anything; it merely exhibits mathematical knowledge
and permits this knowledge to be verified as being correct.  A program in an
algorithmic language can potentially have hidden bugs\index{computer program
bugs} as well as possibly being hard to understand.  But each token in a
Metamath database must be consistent with the database's earlier
contents according to simple, fixed rules.
If a database is verified
to be correct,\footnote{This includes
verification that a sequential list of proof steps results in the specified
theorem.} then the mathematical content is correct if the
verifier is correct and the axioms are correct.
The verification program could be incorrect, but the verification algorithm
is relatively simple (making it unlikely to be implemented incorrectly
by the Metamath program),
and there are over a dozen Metamath database verifiers
written by different people in different programming languages
(so these different verifiers can act as multiple reviewers of a database).
The most-used Metamath database, the Metamath Proof Explorer
(aka \texttt{set.mm}\index{set theory database (\texttt{set.mm})}%
\index{Metamath Proof Explorer}),
is currently verified by four different Metamath verifiers written by
four different people in four different languages, including the
original Metamath program described in this book.
The only ``bugs'' that can exist are in the statement of the axioms,
for example if the axioms are inconsistent (a famous problem shown to be
unsolvable by G\"{o}del's incompleteness theorem\index{G\"{o}del's
incompleteness theorem}).
However, real mathematical systems have very few axioms, and these can
be carefully studied.
All of this provides extraordinarily high confidence that the verified database
is in fact correct.

The Metamath program
doesn't prove theorems automatically but is designed to verify proofs
that you supply to it.
The underlying Metamath language is completely general and has no built-in,
preconceived notions about your formal system\index{formal system}, its logic
or its syntax.
For constructing proofs, the Metamath program has a Proof Assistant\index{Proof
Assistant} which helps you fill in some of a proof step's details, shows you
what choices you have at any step, and verifies the proof as you build it; but
you are still expected to provide the proof.

There are many other programs that can process or generate information
in the Metamath language, and more continue to be written.
This is in part because the Metamath language itself is very simple
and intentionally easy to automatically process.
Some programs, such as \texttt{mmj2}\index{mmj2}, include a proof assistant
that can automate some steps beyond what the Metamath program can do.
Mario Carneiro has developed an algorithm for converting proofs from
the OpenTheory interchange format, which can be translated to and from
any of the HOL family of proof languages (HOL4, HOL Light, ProofPower,
and Isabelle), into the
Metamath language.\cite{DBLP:journals/corr/Carneiro14}\index{Carneiro, Mario}.
Daniel Whalen has developed Holophrasm, which can automatically
prove many Metamath proofs using
machine learning\index{machine learning}\index{artificial intelligence}
approaches
(including multiple neural networks\index{neural networks})\cite{DBLP:journals/corr/Whalen16}\index{Whalen, Daniel}.
However,
a discussion of these other programs is beyond the scope of this book.

Like most computer languages, the Metamath\index{Metamath} language uses the
standard ({\sc ascii}) characters on a computer keyboard, so it cannot
directly represent many of the special symbols that mathematicians use.  A
useful feature of the Metamath program is its ability to convert its notation
into the \LaTeX\ typesetting language.\index{latex@{\LaTeX}}  This feature
lets you convert the {\sc ascii} tokens you've defined into standard
mathematical symbols, so you end up with symbols and formulas you are familiar
with instead of somewhat cryptic {\sc ascii} representations of them.
The Metamath program can also generate HTML\index{HTML}, making it easy
to view results on the web and to see related information by using
hypertext links.

Metamath is probably conceptually different from anything you've seen
before and some aspects may take some getting used to.  This book will
help you decide whether Metamath suits your specific needs.

\subsubsection{Setting Your Expectations}
It is important for you to understand what Metamath\index{Metamath} is and is
not.  As mentioned, the Metamath program
is {\em not} an automated theorem prover but
rather a proof verifier.  Developing a database can be tedious, hard work,
especially if you want to make the proofs as short as possible, but it becomes
easier as you build up a collection of useful theorems.  The purpose of
Metamath is simply to document existing mathematics in an absolutely rigorous,
computer-verifiable way, not to aid directly in the creation of new
mathematics.  It also is not a magic solution for learning abstract
mathematics, although it may be helpful to be able to actually see the implied
rigor behind what you are learning from textbooks, as well as providing hints
to work out proofs that you are stumped on.

As of this writing, a sizable set theory database has been developed to
provide a foundation for many fields of mathematics, but much more work would
be required to develop useful databases for specific fields.

Metamath\index{Metamath} ``knows no math;'' it just provides a framework in
which to express mathematics.  Its language is very small.  You can define two
kinds of symbols, constants\index{constant} and variables\index{variable}.
The only thing Metamath knows how to do is to substitute strings of symbols
for the variables\index{substitution!variable}\index{variable substitution} in
an expression based on instructions you provide it in a proof, subject to
certain constraints you specify for the variables.  Even the decimal
representation of a number is merely a string of certain constants (digits)
which together, in a specific context, correspond to whatever mathematical
object you choose to define for it; unlike other computer languages, there is
no actual number stored inside the computer.  In a proof, you in effect
instruct Metamath what symbol substitutions to make in previous axioms or
theorems and join a sequence of them together to result in the desired
theorem.  This kind of symbol manipulation captures the essence of mathematics
at a preaxiomatic level.

\subsubsection{Metamath and Mathematical Literature}

In advanced mathematical literature, proofs are usually presented in the form
of short outlines that often only an expert can follow.  This is partly out of
a desire for brevity, but it would also be unwise (even if it were practical)
to present proofs in complete formal detail, since the overall picture would
be lost.\index{formal proof}

A solution I envision\label{envision} that would allow mathematics to remain
acceptable to the expert, yet increase its accessibility to non-specialists,
consists of a combination of the traditional short, informal proof in print
accompanied by a complete formal proof stored in a computer database.  In an
analogy with a computer program, the informal proof is like a set of comments
that describe the overall reasoning and content of the proof, whereas the
computer database is like the actual program and provides a means for anyone,
even a non-expert, to follow the proof in as much detail as desired, exploring
it back through layers of theorems (like subroutines that call other
subroutines) all the way back to the axioms of the theory.  In addition, the
computer database would have the advantage of providing absolute assurance
that the proof is correct, since each step can be verified automatically.

There are several other approaches besides Metamath to a project such
as this.  Section~\ref{proofverifiers} discusses some of these.

To us, a noble goal would be a database with hundreds of thousands of
theorems and their computer-verifiable proofs, encompassing a significant
fraction of known mathematics and available for instant access.
These would be fully verified by multiple independently-implemented verifiers,
to provide extremely high confidence that the proofs are completely correct.
The database would allow people to investigate whatever details they were
interested in, so that they could confirm whatever portions they wished.
Whether or not Metamath is an appropriate choice remains to be seen, but in
principle we believe it is sufficient.

\subsubsection{Formalism}

Over the past fifty years, a group of French mathematicians working
collectively under the pseudonym of Bourbaki\index{Bourbaki, Nicolas} have
co-authored a series of monographs that attempt to rigorously and
consistently formalize large bodies of mathematics from foundations.  On the
one hand, certainly such an effort has its merits; on the other hand, the
Bourbaki project has been criticized for its ``scholasticism'' and
``hyperaxiomatics'' that hide the intuitive steps that lead to the results
\cite[p.~191]{Barrow}\index{Barrow, John D.}.

Metamath unabashedly carries this philosophy to its extreme and no doubt is
subject to the same kind of criticism.  Nonetheless I think that in
conjunction with conventional approaches to mathematics Metamath can serve a
useful purpose.  The Bourbaki approach is essentially pedagogic, requiring the
reader to become intimately familiar with each detail in a very large
hierarchy before he or she can proceed to the next step.  The difference with
Metamath is that the ``reader'' (user) knows that all details are contained in
its computer database, available as needed; it does not demand that the user
know everything but conveniently makes available those portions that are of
interest.  As the body of all mathematical knowledge grows larger and larger,
no one individual can have a thorough grasp of its entirety.  Metamath
can finalize and put to rest any questions about the validity of any part of it
and can make any part of it accessible, in principle, to a non-specialist.

\subsubsection{A Personal Note}
Why did I develop Metamath\index{Metamath}?  I enjoy abstract mathematics, but
I sometimes get lost in a barrage of definitions and start to lose confidence
that my proofs are correct.  Or I reach a point where I lose sight of how
anything I'm doing relates to the axioms that a theory is based on and am
sometimes suspicious that there may be some overlooked implicit axiom
accidentally introduced along the way (as happened historically with Euclidean
geometry\index{Euclidean geometry}, whose omission of Pasch's
axiom\index{Pasch's axiom} went unnoticed for 2000 years
\cite[p.~160]{Davis}!). I'm also somewhat lazy and wish to avoid the effort
involved in re-verifying the gaps in informal proofs ``left to the reader;'' I
prefer to figure them out just once and not have to go through the same
frustration a year from now when I've forgotten what I did.  Metamath provides
better recovery of my efforts than scraps of paper that I can't
decipher anymore.  But mostly I find very appealing the idea of rigorously
archiving mathematical knowledge in a computer database, providing precision,
certainty, and elimination of human error.

\subsubsection{Note on Bibliography and Index}

The Bibliography usually includes the Library of Congress classification
for a work to make it easier for you to find it in on a university
library shelf.  The Index has author references to pages where their works
are cited, even though the authors' names may not appear on those pages.

\subsubsection{Acknowledgments}

Acknowledgments are first due to my wife, Deborah (who passed away on
September 4, 1998), for critiquing the manu\-script but most of all for
her patience and support.  I also wish to thank Joe Wright, Richard
Becker, Clarke Evans, Buddha Buck, and Jeremy Henty for helpful
comments.  Any errors, omissions, and other shortcomings are of course
my responsibility.

\subsubsection{Note Added June 22, 2005}\label{note2002}

The original, unpublished version of this book was written in 1997 and
distributed via the web.  The present edition has been updated to
reflect the current Metamath program and databases, as well as more
current {\sc url}s for Internet sites.  Thanks to Josh
Purinton\index{Purinton, Josh}, One Hand
Clapping, Mel L.\ O'Cat, and Roy F. Longton for pointing out
typographical and other errors.  I have also benefitted from numerous
discussions with Raph Levien\index{Levien, Raph}, who has extended
Metamath's philosophy of rigor to result in his {\em
Ghilbert}\index{Ghilbert} proof language (\url{http://ghilbert.org}).

Robert (Bob) Solovay\index{Solovay, Robert} communicated a new result of
A.~R.~D.~Mathias on the system of Bourbaki, and the text has been
updated accordingly (p.~\pageref{bourbaki}).

Bob also pointed out a clarification of the literature regarding
category theory and inaccessible cardinals\index{category
theory}\index{cardinal, inaccessible} (p.~\pageref{categoryth}),
and a misleading statement was removed from the text.  Specifically,
contrary to a statement in previous editions, it is possible to express
``There is a proper class of inaccessible cardinals'' in the language of
ZFC.  This can be done as follows:  ``For every set $x$ there is an
inaccessible cardinal $\kappa$ such that $\kappa$ is not in $x$.''
Bob writes:\footnote{Private communication, Nov.~30, 2002.}
\begin{quotation}
     This axiom is how Grothendieck presents category theory.  To each
inaccessible cardinal $\kappa$ one associates a Grothendieck universe
\index{Grothendieck, Alexander} $U(\kappa)$.  $U(\kappa)$ consists of
those sets which lie in a transitive set of cardinality less than
$\kappa$.  Instead of the ``category of all groups,'' one works relative
to a universe [considering the category of groups of cardinality less
than $\kappa$].  Now the category whose objects are all categories
``relative to the universe $U(\kappa)$'' will be a category not
relative to this universe but to the next universe.

     All of the things category theorists like to do can be done in this
framework.  The only controversial point is whether the Grothen\-dieck
axiom is too strong for the needs of category theorists.  Mac Lane
\index{Mac Lane, Saunders} argues that ``one universe is enough'' and
Feferman\index{Feferman, Solomon} has argued that one can get by with
ordinary ZFC.  I don't find Feferman's arguments persuasive.  Mac Lane
may be right, but when I think about category theory I do it \`{a} la
Grothendieck.

        By the way Mizar\index{Mizar} adds the axiom ``there is a proper
class of inaccessibles'' precisely so as to do category theory.
\end{quotation}

The most current information on the Metamath program and databases can
always be found at \url{http://metamath.org}.


\subsubsection{Note Added June 24, 2006}\label{note2006}

The Metamath spec was restricted slightly to make parsers easier to
write.  See the footnote on p.~\pageref{namespace}.

%\subsubsection{Note Added July 24, 2006}\label{note2006b}
\subsubsection{Note Added March 10, 2007}\label{note2006b}

I am grateful to Anthony Williams\index{Williams, Anthony} for writing
the \LaTeX\ package called {\tt realref.sty} and contributing it to the
public domain.  This package allows the internal hyperlinks in a {\sc
pdf} file to anchor to specific page numbers instead of just section
titles, making the navigation of the {\sc pdf} file for this book much
more pleasant and ``logical.''

A typographical error found by Martin Kiselkov was corrected.
A confusing remark about unification was deleted per suggestion of
Mel O'Cat.

\subsubsection{Note Added May 27, 2009}\label{note2009}

Several typos found by Kim Sparre were corrected.  A note was added that
the Poincar\'{e} conjecture has been proved (p.~\pageref{poincare}).

\subsubsection{Note Added Nov. 17, 2014}\label{note2014}

The statement of the Schr\"{o}der--Bernstein theorem was corrected in
Section~\ref{trust}.  Thanks to Bob Solovay for pointing out the error.

\subsubsection{Note Added May 25, 2016}\label{note2016}

Thanks to Jerry James for correcting 16 typos.

\subsubsection{Note Added February 25, 2019}\label{note201902}

David A. Wheeler\index{Wheeler, David A.}
made a large number of improvements and updates,
in coordination with Norman Megill.
The predicate calculus axioms were renumbered, and the text makes
it clear that they are based on Tarski's system S2;
the one slight deviation in axiom ax-6 is explained and justified.
The real and complex number axioms were modified to be consistent with
\texttt{set.mm}\index{set theory database (\texttt{set.mm})}%
\index{Metamath Proof Explorer}.
Long-awaited specification changes ``1--8'' were made,
which clarified previously ambiguous points.
Some errors in the text involving \texttt{\$f} and
\texttt{\$d} statements were corrected (the spec was correct, but
the in-book explanations unintentionally contradicted the spec).
We now have a system for automatically generating narrow PDFs,
so that those with smartphones can have easy access to the current
version of this document.
A new section on deduction was added;
it discusses the standard deduction theorem,
the weak deduction theorem,
deduction style, and natural deduction.
Many minor corrections (too numerous to list here) were also made.

\subsubsection{Note Added March 7, 2019}\label{note201903}

This added a description of the Matamath language syntax in
Extended Backus--Naur Form (EBNF)\index{Extended Backus--Naur Form}\index{EBNF}
in Appendix \ref{BNF}, added a brief explanation about typecodes,
inserted more examples in the deduction section,
and added a variety of smaller improvements.

\subsubsection{Note Added April 7, 2019}\label{note201904}

This version clarified the proper substitution notation, improved the
discussion on the weak deduction theorem and natural deduction,
documented the \texttt{undo} command, updated the information on
\texttt{write source}, changed the typecode
from \texttt{set} to \texttt{setvar} to be consistent with the current
version of \texttt{set.mm}, added more documentation about comment markup
(e.g., documented how to create headings), and clarified the
differences between various assertion forms (in particular deduction form).

\subsubsection{Note Added May 20, 2019}\label{note201905}

This version fixes various small issues and added specific examples
of forms (deduction form, inference form, and closed form).

\chapter{Introduction}
\pagenumbering{arabic}

\begin{quotation}
  {\em {\em I.M.:}  No, no.  There's nothing subjective about it!  Everybody
knows what a proof is.  Just read some books, take courses from a competent
mathematician, and you'll catch on.

{\em Student:}  Are you sure?

{\em I.M.:}  Well---it is possible that you won't, if you don't have any
aptitude for it.  That can happen, too.

{\em Student:}  Then {\em you} decide what a proof is, and if I don't learn
to decide in the same way, you decide I don't have any aptitude.

{\em I.M.:}  If not me, then who?}
    \flushright\sc  ``The Ideal Mathematician''
    \index{Davis, Phillip J.}
    \footnote{\cite{Davis}, p.~40.}\\
\end{quotation}

In the past century, brilliant mathematicians have discovered almost
unimaginably profound results that rank among the crowning intellectual
achievements of mankind.  However, there is a sense in which modern abstract
mathematics is behind the times, stuck in an era before computers existed.
While no one disputes the remarkable results that have been achieved,
communicating these results in a precise way to the uninitiated is virtually
impossible.  To describe these results, a terse informal language is used which
despite its elegance is very difficult to learn.  This informal language is not
imprecise, far from it, but rather it often has omitted detail
and symbols with hidden context that are
implicitly understood by an expert but few others.  Extremely complex technical
meanings are associated with innocent-sounding English words such as
``compact'' and ``measurable'' that barely hint at what is actually being
said.  Anyone who does not keep the precise technical meaning constantly in
mind is bound to fail, and acquiring the ability to do this can be achieved
only through much practice and hard work.  Only the few who complete the
painful learning experience can join the small in-group of pure
mathematicians.  The informal language effectively cuts off the true nature of
their knowledge from most everyone else.

Metamath\index{Metamath} makes abstract mathematics more concrete.  It allows
a computer to keep track of the complexity associated with each word or symbol
with absolute rigor.  You can explore this complexity at your leisure, to
whatever degree you desire.  Whether or not you believe that concepts such as
infinity actually ``exist'' outside of the mind, Metamath lets you get to the
foundation for what's really being said.

Metamath also enables completely rigorous and thorough proof verification.
Its language is simple enough so that you
don't have to rely on the authority of experts but can verify the results
yourself, step by step.  If you want to attempt to derive your own results,
Metamath will not let you make a mistake in reasoning.
Even professional mathematicians make mistakes; Metamath makes it possible
to thoroughly verify that proofs are correct.

Metamath\index{Metamath} is a computer language and an associated computer
program for archiving, verifying, and studying mathematical proofs at a very
detailed level.
The Metamath language
describes formal\index{formal system} mathematical
systems and expresses proofs of theorems in those systems.  Such a language
is called a metalanguage\index{metalanguage} by mathematicians.
The Metamath program is a computer program that verifies
proofs expressed in the Metamath language.
The Metamath program does not have the built-in
ability to make logical inferences; it just makes a series of symbol
substitutions according to instructions given to it in a proof
and verifies that the result matches the expected theorem.  It makes logical
inferences based only on rules of logic that are contained in a set of
axioms\index{axiom}, or first principles, that you provide to it as the
starting point for proofs.

The complete specification of the Metamath language is only four pages long
(Section~\ref{spec}, p.~\pageref{spec}).  Its simplicity may at first make you
wonder how it can do much of anything at all.  But in fact the kinds of
symbol manipulations it performs are the ones that are implicitly done in all
mathematical systems at the lowest level.  You can learn it relatively quickly
and have complete confidence in any mathematical proof that it verifies.  On
the other hand, it is powerful and general enough so that virtually any
mathematical theory, from the most basic to the deeply abstract, can be
described with it.

Although in principle Metamath can be used with any
kind of mathematics, it is best suited for abstract or ``pure'' mathematics
that is mostly concerned with theorems and their proofs, as opposed to the
kind of mathematics that deals with the practical manipulation of numbers.
Examples of branches of pure mathematics are logic\index{logic},\footnote{Logic
is the study of statements that are universally true regardless of the objects
being described by the statements.  An example is the statement, ``if $P$
implies $Q$, then either $P$ is false or $Q$ is true.''} set theory\index{set
theory},\footnote{Set theory is the study of general-purpose mathematical objects called
``sets,'' and from it essentially all of mathematics can be derived.  For
example, numbers can be defined as specific sets, and their properties
can be explored using the tools of set theory.} number theory\index{number
theory},\footnote{Number theory deals with the properties of positive and
negative integers (whole numbers).} group theory\index{group
theory},\footnote{Group theory studies the properties of mathematical objects
called groups that obey a simple set of axioms and have properties of symmetry
that make them useful in many other fields.} abstract algebra\index{abstract
algebra},\footnote{Abstract algebra includes group theory and also studies
groups with additional properties that qualify them as ``rings'' and
``fields.''  The set of real numbers is a familiar example of a field.},
analysis\index{analysis} \index{real and complex numbers}\footnote{Analysis is
the study of real and complex numbers.} and
topology\index{topology}.\footnote{One area studied by topology are properties
that remain unchanged when geometrical objects undergo stretching
deformations; for example a doughnut and a coffee cup each have one hole (the
cup's hole is in its handle) and are thus considered topologically
equivalent.  In general, though, topology is the study of abstract
mathematical objects that obey a certain (surprisingly simple) set of axioms.
See, for example, Munkres \cite{Munkres}\index{Munkres, James R.}.} Even in
physics, Metamath could be applied to certain branches that make use of
abstract mathematics, such as quantum logic\index{quantum logic} (used to study
aspects of quantum mechanics\index{quantum mechanics}).

On the other hand, Metamath\index{Metamath} is less suited to applications
that deal primarily with intensive numeric computations.  Metamath does not
have any built-in representation of numbers\index{Metamath!representation of
numbers}; instead, a specific string of symbols (digits) must be syntactically
constructed as part of any proof in which an ordinary number is used.  For
this reason, numbers in Metamath are best limited to specific constants that
arise during the course of a theorem or its proof.  Numbers are only a tiny
part of the world of abstract mathematics.  The exclusion of built-in numbers
was a conscious decision to help achieve Metamath's simplicity, and there are
other software tools if you have different mathematical needs.
If you wish to quickly solve algebraic problems, the computer algebra
programs\index{computer algebra system} {\sc
macsyma}\index{macsyma@{\sc macsyma}}, Mathematica\index{Mathematica}, and
Maple\index{Maple} are specifically suited to handling numbers and
algebra efficiently.
If you wish to simply calculate numeric or matrix expressions easily,
tools such as Octave\index{Octave} may be a better choice.

After learning Metamath's basic statement types, any
tech\-ni\-cal\-ly ori\-ent\-ed person, mathematician or not, can
immediately trace
any theorem proved in the language as far back as he or she wants, all the way
to the axioms on which the theorem is based.  This ability suggests a
non-traditional way of learning about pure mathematics.  Used in conjunction
with traditional methods, Metamath could make pure mathematics accessible to
people who are not sufficiently skilled to figure out the implicit detail in
ordinary textbook proofs.  Once you learn the axioms of a theory, you can have
complete confidence that everything you need to understand a proof you are
studying is all there, at your beck and call, allowing you to focus in on any
proof step you don't understand in as much depth as you need, without worrying
about getting stuck on a step you can't figure out.\footnote{On the other
hand, writing proofs in the Metamath language is challenging, requiring
a degree of rigor far in excess of that normally taught to students.  In a
classroom setting, I doubt that writing Metamath proofs would ever replace
traditional homework exercises involving informal proofs, because the time
needed to work out the details would not allow a course to
cover much material.  For students who have trouble grasping the implied rigor
in traditional material, writing a few simple proofs in the Metamath language
might help clarify fuzzy thought processes.  Although somewhat difficult at
first, it eventually becomes fun to do, like solving a puzzle, because of the
instant feedback provided by the computer.}

Metamath\index{Metamath} is probably unlike anything you have
encountered before.  In this first chapter we will look at the philosophy and
use of computers in mathematics in order to better understand the motivation
behind Metamath.  The material in this chapter is not required in order to use
Metamath.  You may skip it if you are impatient, but I hope you will find it
educational and enjoyable.  If you want to start experimenting with the
Metamath program right away, proceed directly to Chapter~\ref{using}
(p.~\pageref{using}).  To
learn the Metamath language, skim Chapter~\ref{using} then proceed to
Chapter~\ref{languagespec} (p.~\pageref{languagespec}).

\section{Mathematics as a Computer Language}

\begin{quote}
  {\em The study of mathematics is apt to commence in
dis\-ap\-point\-ment.\ldots \\
We are told that by its aid the stars are weighted
and the billions of molecules in a drop of water are counted.  Yet, like the
ghost of Hamlet's father, this great science eludes the efforts of our mental
weapons to grasp it.}
  \flushright\sc  Alfred North Whitehead\footnote{\cite{Whitehead}, ch.\ 1.}\\
\end{quote}\index{Whitehead, Alfred North}

\subsection{Is Mathematics ``User-Friendly''?}

Suppose you have no formal training in abstract mathematics.  But popular
books you've read offer tempting glimpses of this world filled with profound
ideas that have stirred the human spirit.  You are not satisfied with the
informal, watered-down descriptions you've read but feel it is important to
grasp the underlying mathematics itself to understand its true meaning. It's
not practical to go back to school to learn it, though; you don't want to
dedicate years of your life to it.  There are many important things in life,
and you have to set priorities for what's important to you.  What would happen
if you tried to pursue it on your own, in your spare time?

After all, you were able to learn a computer programming language such as
Pascal on your own without too much difficulty, even though you had no formal
training in computers.  You don't claim to be an expert in software design,
but you can write a passable program when necessary to suit your needs.  Even
more important, you know that you can look at anyone else's Pascal program, no
matter how complex, and with enough patience figure out exactly how it works,
even though you are not a specialist.  Pascal allows you do anything that a
computer can do, at least in principle.  Thus you know you have the ability,
in principle, to follow anything that a computer program can do:  you just
have to break it down into small enough pieces.

Here's an imaginary scenario of what might happen if you na\-ive\-ly a\-dopted
this same view of abstract mathematics and tried to pick it up on your own, in
a period of time comparable to, say, learning a computer programming
language.

\subsubsection{A Non-Mathematician's Quest for Truth}

\begin{quote}
  {\em \ldots my daughters have been studying (chemistry) for several
se\-mes\-ters, think they have learned differential and integral calculus in
school, and yet even today don't know why $x\cdot y=y\cdot x$ is true.}
  \flushright\sc  Edmund Landau\footnote{\cite{Landau}, p.~vi.}\\
\end{quote}\index{Landau, Edmund}

\begin{quote}
  {\em Minus times minus is plus,\\
The reason for this we need not discuss.}
  \flushright\sc W.\ H.\ Auden\footnote{As quoted in \cite{Guillen}, p.~64.}\\
\end{quote}\index{Auden, W.\ H.}\index{Guillen, Michael}

We'll suppose you are a technically oriented professional, perhaps an engineer, a
computer programmer, or a physicist, but probably not a mathematician.  You
consider yourself reasonably intelligent.  You did well in school, learning a
variety of methods and techniques in practical mathematics such as calculus and
differential equations.  But rarely did your courses get into anything
resembling modern abstract mathematics, and proofs were something that appeared
only occasionally in your textbooks, a kind of necessary evil that was
supposed to convince you of a certain key result.  Most of your
homework consisted of exercises that gave you practice in the techniques, and
you were hardly ever asked to come up with a proof of your own.

You find yourself curious about advanced, abstract mathematics.  You are
driven by an inner conviction that it is important to understand and
appreciate some of the most profound knowledge discovered by mankind.  But it
seems very hard to learn, something that only certain gifted longhairs can
access and understand.  You are frustrated that it seems forever cut off from
you.

Eventually your curiosity drives you to do something about it.
You set for yourself a goal of ``really'' understanding mathematics:  not just
how to manipulate equations in algebra or calculus according to cookbook
rules, but rather to gain a deep understanding of where those rules come from.
In fact, you're not thinking about this kind of ordinary mathematics at all,
but about a much more abstract, ethereal realm of pure mathematics, where
famous results such as G\"{o}del's incompleteness theorem\index{G\"{o}del's
incompleteness theorem} and Cantor's different kinds of infinities
reside.

You have probably read a number of popular books, with titles like {\em
Infinity and the Mind} \cite{Rucker}\index{Rucker, Rudy}, on topics such as
these.  You found them inspiring but at the same time somewhat
unsatisfactory.  They gave you a general idea of what these results are about,
but if someone asked you to prove them, you wouldn't have the faintest idea of
where to begin.   Sure, you could give the same overall outline that you
learned from the popular books; and in a general sort of way, you do have an
understanding.  But deep down inside, you know that there is a rigor that is
missing, that probably there are many subtle steps and pitfalls along the way,
and ultimately it seems you have to place your trust in the experts in the
field.  You don't like this; you want to be able to verify these results for
yourself.

So where do you go next?  As a first step, you decide to look up some of the
original papers on the theorems you are curious about, or better, obtain some
standard textbooks in the field.  You look up a theorem you want to
understand.  Sure enough, it's there, but it's expressed with strange
terms and odd symbols that mean absolutely nothing to you.  It might as well be written in
a foreign language you've never seen before, whose symbols are totally alien.
You look at the proof, and you haven't the foggiest notion what each step
means, much less how one step follows from another.  Well, obviously you have
a lot to learn if you want to understand this stuff.

You feel that you could probably understand it by
going back to college for another three to six years and getting a math
degree.  But that does not fit in with your career and the other things in
your life and would serve no practical purpose.  You decide to seek a quicker
path.  You figure you'll just trace your way back to the beginning, step by
step, as you would do with a computer program, until you understand it.  But
you quickly find that this is not possible, since you can't even understand
enough to know what you have to trace back to.

Maybe a different approach is in order---maybe you should start at the
beginning and work your way up.  First, you read the introduction to the book
to find out what the prerequisites are.  In a similar fashion, you trace your
way back through two or three more books, finally arriving at one that seems
to start at a beginning:  it lists the axioms of arithmetic.  ``Aha!'' you
naively think, ``This must be the starting point, the source of all mathematical
knowledge.'' Or at least the starting point for mathematics dealing with
numbers; you have to start somewhere and have no idea what the starting point
for other mathematics would be.  But the word ``axioms'' looks promising.  So
you eagerly read along and work through some elementary exercises at the
beginning of the book.  You feel vaguely bothered:  these
don't seem like axioms at all, at least not in the sense that you want to
think of axioms.  Axioms imply a starting point from which everything else can
be built up, according to precise rules specified in the axiom system.  Even
though you can understand the first few proofs in an informal way,
and are able to do some of the
exercises, it's hard to pin down precisely what the
rules are.   Sure, each step seems to follow logically from the others, but
exactly what does that mean?  Is the ``logic'' just a matter of common sense,
something vague that we all understand but can never quite state precisely?

You've spent a number of years, off and on, programming computers, and you
know that in the case of computer languages there is no question of what the
rules are---they are precise and crystal clear.  If you follow them, your
program will work, and if you don't, it won't.  No matter how complex a
program, it can always be broken down into simpler and simpler pieces, until
you can ultimately identify the bits that are moved around to perform a
specific function.  Some programs might require a lot of perseverance to
accomplish this, but if you focus on a specific portion of it, you don't even
necessarily have to know how the rest of it works. Shouldn't there be an
analogy in mathematics?

You decide to apply the ultimate test:  you ask yourself how a computer could
verify or ensure that the steps in these proofs follow from one another.
Certainly mathematics must be at least as precisely defined as a computer
language, if not more so; after all, computer science itself is based on it.
If you can get a computer to verify these proofs, then you should also be
able, in principle, to understand them yourself in a very crystal clear,
precise way.

You're in for a surprise:  you can conceive of no way to convert the
proofs, which are in English, to a form that the computer can understand.
The proofs are filled with phrases such as ``assume there exists a unique
$x$\ldots'' and ``given any $y$, let $z$ be the number such that\ldots''  This
isn't the kind of logic you are used to in computer programming, where
everything, even arithmetic, reduces to Boolean ones and zeroes if you care to
break it down sufficiently.  Even though you think you understand the proofs,
there seems to be some kind of higher reasoning involved rather than precise
rules that define how you manipulate the symbols in the axioms.  Whatever it
is, it just isn't obvious how you would express it to a computer, and the more
you think about it, the more puzzled and confused you get, to the point where
you even wonder whether {\em you} really understand it.  There's a lot more to
these axioms of arithmetic than meets the eye.

Nobody ever talked about this in school in your applied math and engineering
courses.  You just learned the rules they gave you, not quite understanding
how or why they worked, sometimes vaguely suspicious or uncertain of them, and
through homework problems and osmosis learned how to present solutions that
satisfied the instructor and earned you an ``A.''  Rarely did you actually
``prove'' anything in a rigorous way, and the math majors who did do stuff
like that seemed to be in a different world.

Of course, there are computer algebra programs that can do mathematics, and
rather impressively.  They can instantly solve the integrals that you
struggled with in freshman calculus, and do much, much more.  But when you
look at these programs, what you see is a big collection of algorithms and
techniques that evolved and were added to over time, along with some basic
software that manipulates symbols.  Each algorithm that is built in is the
result of someone's theorem whose proof is omitted; you just have to trust the
person who proved it and the person who programmed it in and hope there are no
bugs.\index{computer program bugs}  Somehow this doesn't seem to be the
essence of mathematics.  Although computer algebra systems can generate
theorems with amazing speed, they can't actually prove a single one of them.

After some puzzlement, you revisit some popular books on what mathematics is
all about.  Somewhere you read that all of mathematics is actually derived
from something called ``set theory.''  This is a little confusing, because
nowhere in the book that presented the axioms of arithmetic was there any
mention of set theory, or if there was, it seemed to be just a tool that helps
you describe things better---the set of even numbers, that sort of thing.  If
set theory is the basis for all mathematics, then why are additional axioms
needed for arithmetic?

Something is wrong but you're not sure what.  One of your friends is a pure
mathematician.  He knows he is unable to communicate to you what he does for a
living and seems to have little interest in trying.  You do know that for him,
proofs are what mathematics is all about. You ask him what a proof is, and he
essentially tells you that, while of course it's based on logic, really it's
something you learn by doing it over and over until you pick it up.  He refers
you to a book, {\em How to Read and Do Proofs} \cite{Solow}.\index{Solow,
Daniel}  Although this book helps you understand traditional informal proofs,
there is still something missing you can't seem to pin down yet.

You ask your friend how you would go about having a computer verify a proof.
At first he seems puzzled by the question; why would you want to do that?
Then he says it's not something that would make any sense to do, but he's
heard that you'd have to break the proof down into thousands or even millions
of individual steps to do such a thing, because the reasoning involved is at
such a high level of abstraction.  He says that maybe it's something you could
do up to a point, but the computer would be completely impractical once you
get into any meaningful mathematics.  There, the only way you can verify a
proof is by hand, and you can only acquire the ability to do this by
specializing in the field for a couple of years in grad school.  Anyway, he
thinks it all has to do with set theory, although he has never taken a formal
course in set theory but just learned what he needed as he went along.

You are intrigued and amazed.  Apparently a mathematician can grasp as a
single concept something that would take a computer a thousand or a million
steps to verify, and have complete confidence in it.  Each one of these
thousand or million steps must be absolutely correct, or else the whole proof
is meaningless.  If you added a million numbers by hand, would you trust the
result?  How do you really know that all these steps are correct, that there
isn't some subtle pitfall in one of these million steps, like a bug in a
computer program?\index{computer program bugs}  After all, you've read that
famous mathematicians have occasionally made mistakes, and you certainly know
you've made your share on your math homework problems in school.

You recall the analogy with a computer program.  Sure, you can understand what
a large computer program such as a word processor does, as a single high-level
concept or a small set of such concepts, but your ability to understand it in
no way ensures that the program is correct and doesn't have hidden bugs.  Even
if you wrote the program yourself you can't really know this; most large
programs that you've written have had bugs that crop up at some later date, no
matter how careful you tried to be while writing them.

OK, so now it seems the reason you can't figure out how to make a
computer verify proofs is because each step really corresponds to a
million small steps.  Well, you say, a computer can do a million
calculations in a second, so maybe it's still practical to do.  Now the
puzzle becomes how to figure out what the million steps are that each
English-language step corresponds to.  Your mathematician friend hasn't
a clue, but suggests that maybe you would find the answer by studying
set theory.  Actually, your friend thinks you're a little off the wall
for even wondering such a thing.  For him, this is not what mathematics
is all about.

The subject of set theory keeps popping up, so you decide it's
time to look it up.

You decide to start off on a careful footing, so you start reading a couple of
very elementary books on set theory.  A lot of it seems pretty obvious, like
intersections, subsets, and Venn diagrams.  You thumb through one of the
books; nowhere is anything about axioms mentioned. The other book relegates to
an appendix a brief discussion that mentions a set of axioms called
``Zermelo--Fraenkel set theory''\index{Zermelo--Fraenkel set theory} and states
them in English.  You look at them and have no idea what they really mean or
what you can do with them.  The comments in this appendix say that the purpose
of mentioning them is to expose you to the idea, but imply that they are not
necessary for basic understanding and that they are really the subject matter
of advanced treatments where fine points such as a certain paradox (Russell's
paradox\index{Russell's paradox}\footnote{Russell's paradox assumes that there
exists a set $S$ that is a collection of all sets that don't contain
themselves.  Now, either $S$ contains itself or it doesn't.  If it contains
itself, it contradicts its definition.  But if it doesn't contain itself, it
also contradicts its definition.  Russell's paradox is resolved in ZF set
theory by denying that such a set $S$ exists.}) are resolved.  Wait a
minute---shouldn't the axioms be a starting point, not an ending point?  If
there are paradoxes that arise without the axioms, how do you know you won't
stumble across one accidentally when using the informal approach?

And nowhere do these books describe how ``all of mathematics can be
derived from set theory'' which by now you've heard a few times.

You find a more advanced book on set theory.  This one actually lists the
axioms of ZF set theory in plain English on page one.  {\em Now} you think
your quest has ended and you've finally found the source of all mathematical
knowledge; you just have to understand what it means.  Here, in one place, is
the basis for all of mathematics!  You stare at the axioms in awe, puzzle over
them, memorize them, hoping that if you just meditate on them long enough they
will become clear.  Of course, you haven't the slightest idea how the rest of
mathematics is ``derived'' from them; in particular, if these are the axioms
of mathematics, then why do arithmetic, group theory, and so on need their own
axioms?

You start reading this advanced book carefully, pondering the meaning of every
word, because by now you're really determined to get to the bottom of this.
The first thing the book does is explain how the axioms came about, which was
to resolve Russell's paradox.\index{Russell's paradox}  In fact that seems to
be the main purpose of their existence; that they supposedly can be used to
derive all of mathematics seems irrelevant and is not even mentioned.  Well,
you go on.  You hope the book will explain to you clearly, step by step, how
to derive things from the axioms.  After all, this is the starting point of
mathematics, like a book that explains the basics of a computer programming
language.  But something is missing.  You find you can't even understand the
first proof or do the first exercise.  Symbols such as $\exists$ and $\forall$
permeate the page without any mention of where they came from or how to
manipulate them; the author assumes you are totally familiar with them and
doesn't even tell you what they mean.  By now you know that $\exists$ means
``there exists'' and $\forall$ means ``for all,'' but shouldn't the rules for
manipulating these symbols be part of the axioms?  You still have no idea
how you could even describe the axioms to a computer.

Certainly there is something much different here from the technical
literature you're used to reading.  A computer language manual almost
always explains very clearly what all the symbols mean, precisely what
they do, and the rules used for combining them, and you work your way up
from there.

After glancing at four or five other such books, you come to the realization
that there is another whole field of study that you need just to get to the
point at which you can understand the axioms of set theory.  The field is
called ``logic.''  In fact, some of the books did recommend it as a
prerequisite, but it just didn't sink in.  You assumed logic was, well, just
logic, something that a person with common sense intuitively understood.  Why
waste your time reading boring treatises on symbolic logic, the manipulation
of 1's and 0's that computers do, when you already know that?  But this is a
different kind of logic, quite alien to you.  The subject of {\sc nand} and
{\sc nor} gates is not even touched upon or in any case has to do with only a
very small part of this field.

So your quest continues.  Skimming through the first couple of introductory
books, you get a general idea of what logic is about and what quantifiers
(``for all,'' ``there exists'') mean, but you find their examples somewhat
trivial and mildly annoying (``all dogs are animals,'' ``some animals are
dogs,'' and such).  But all you want to know is what the rules are for
manipulating the symbols so you can apply them to set theory.  Some formulas
describing the relationships among quantifiers ($\exists$ and $\forall$) are
listed in tables, along with some verbal reasoning to justify them.
Presumably, if you want to find out if a formula is correct, you go through
this same kind of mental reasoning process, possibly using images of dogs and
animals. Intuitively, the formulas seem to make sense.  But when you ask
yourself, ``What are the rules I need to get a computer to figure out whether
this formula is correct?'', you still don't know.  Certainly you don't ask the
computer to imagine dogs and animals.

You look at some more advanced logic books.  Many of them have an introductory
chapter summarizing set theory, which turns out to be a prerequisite.  You
need logic to understand set theory, but it seems you also need set theory to
understand logic!  These books jump right into proving rather advanced
theorems about logic, without offering the faintest clue about where the logic
came from that allows them to prove these theorems.

Luckily, you come across an elementary book of logic that, halfway through,
after the usual truth tables and metaphors, presents in a clear, precise way
what you've been looking for all along: the axioms!  They're divided into
propositional calculus (also called sentential logic) and predicate calculus
(also called first-order logic),\index{first-order logic} with rules so simple
and crystal clear that now you can finally program a computer to understand
them.  Indeed, they're no harder than learning how to play a game of chess.
As far as what you seem to need is concerned, the whole book could have been
written in five pages!

{\em Now} you think you've found the ultimate source of mathematical
truth.  So---the axioms of mathematics consist of these axioms of logic,
together with the axioms of ZF set theory. (By now you've also been able to
figure out how to translate the ZF axioms from English into the
actual symbols of logic which you can now manipulate according to
precise, easy-to-understand rules.)

Of course, you still don't understand how ``all of mathematics can be
derived from set theory,'' but maybe this will reveal itself in due
course.

You eagerly set out to program the axioms and rules into a computer and start
to look at the theorems you will have to prove as the logic is developed.  All
sorts of important theorems start popping up:  the deduction
theorem,\index{deduction theorem} the substitution theorem,\index{substitution
theorem} the completeness theorem of propositional calculus,\index{first-order
logic!completeness} the completeness theorem of predicate calculus.  Uh-oh,
there seems to be trouble.  They all get harder and harder, and not one of
them can be derived with the axioms and rules of logic you've just been
handed.  Instead, they all require ``metalogic'' for their proofs, a kind of
mixture of logic and set theory that allows you to prove things {\em about}
the axioms and theorems of logic rather than {\em with} them.

You plow ahead anyway.  A month later, you've spent much of your
free time getting the computer to verify proofs in propositional calculus.
You've programmed in the axioms, but you've also had to program in the
deduction theorem, the substitution theorem, and the completeness theorem of
propositional calculus, which by now you've resigned yourself to treating as
rather complex additional axioms, since they can't be proved from the axioms
you were given.  You can now get the computer to verify and even generate
complete, rigorous, formal proofs\index{formal proof}.  Never mind that they
may have 100,000 steps---at least now you can have complete, absolute
confidence in them.  Unfortunately, the only theorems you have proved are
pretty trivial and you can easily verify them in a few minutes with truth
tables, if not by inspection.

It looks like your mathematician friend was right.  Getting the computer to do
serious mathematics with this kind of rigor seems almost hopeless.  Even
worse, it seems that the further along you get, the more ``axioms'' you have
to add, as each new theorem seems to involve additional ``metamathematical''
reasoning that hasn't been formalized, and none of it can be derived from the
axioms of logic.  Not only do the proofs keep growing exponentially as you get
further along, but the program to verify them keeps getting bigger and bigger
as you program in more ``metatheorems.''\index{metatheorem}\footnote{A
metatheorem is usually a statement that is too general to be directly provable
in a theory.  For example, ``if $n_1$, $n_2$, and $n_3$ are integers, then
$n_1+n_2+n_3$ is an integer'' is a theorem of number theory.  But ``for any
integer $k > 1$, if $n_1, \ldots, n_k$ are integers, then $n_1+\ldots +n_k$ is
an integer'' is a metatheorem, in other words a family of theorems, one for
every $k$.  The reason it is not a theorem is that the general sum $n_1+\ldots
+n_k$ (as a function of $k$) is not an operation that can be defined directly
in number theory.} The bugs\index{computer program bugs} that have cropped up
so far have already made you start to lose faith in the rigor you seem to have
achieved, and you know it's just going to get worse as your program gets larger.

\subsection{Mathematics and the Non-Specialist}

\begin{quote}
  {\em A real proof is not checkable by a machine, or even by any mathematician
not privy to the gestalt, the mode of thought of the particular field of
mathematics in which the proof is located.}
  \flushright\sc  Davis and Hersh\index{Davis, Phillip J.}
  \footnote{\cite{Davis}, p.~354.}\\
\end{quote}

The bulk of abstract or theoretical mathematics is ordinarily outside
the reach of anyone but a few specialists in each field who have completed
the necessary difficult internship in order to enter its coterie.  The
typical intelligent layperson has no reasonable hope of understanding much of
it, nor even the specialist mathematician of understanding other fields.  It
is like a foreign language that has no dictionary to look up the translation;
the only way you can learn it is by living in the country for a few years.  It
is argued that the effort involved in learning a specialty is a necessary
process for acquiring a deep understanding.  Of course, this is almost certainly
true if one is to make significant contributions to a field; in particular,
``doing'' proofs is probably the most important part of a mathematician's
training.  But is it also necessary to deny outsiders access to it?  Is it
necessary that abstract mathematics be so hard for a layperson to grasp?

A computer normally is of no help whatsoever.  Most published proofs are
actually just series of hints written in an informal style that requires
considerable knowledge of the field to understand.  These are the ``real
proofs'' referred to by Davis and Hersh.\index{informal proof}  There is an
implicit understanding that, in principle, such a proof could be converted to
a complete formal proof\index{formal proof}.  However, it is said that no one
would ever attempt such a conversion, even if they could, because that would
presumably require millions of steps (Section~\ref{dream}).  Unfortunately the
informal style automatically excludes the understanding of the proof
by anyone who hasn't gone through the necessary apprenticeship. The
best that the intelligent layperson can do is to read popular books about deep
and famous results; while this can be helpful, it can also be misleading, and
the lack of detail usually leaves the reader with no ability whatsoever to
explore any aspect of the field being described.

The statements of theorems often use sophisticated notation that makes them
inaccessible to the non-specialist.  For a non-specialist who wants to achieve
a deeper understanding of a proof, the process of tracing definitions and
lemmas back through their hierarchy\index{hierarchy} quickly becomes confusing
and discouraging.  Textbooks are usually written to train mathematicians or to
communicate to people who are already mathematicians, and large gaps in proofs
are often left as exercises to the reader who is left at an impasse if he or
she becomes stuck.

I believe that eventually computers will enable non-specialists and even
intelligent laypersons to follow almost any mathematical proof in any field.
Metamath is an attempt in that direction.  If all of mathematics were as
easily accessible as a computer programming language, I could envision
computer programmers and hobbyists who otherwise lack mathematical
sophistication exploring and being amazed by the world of theorems and proofs
in obscure specialties, perhaps even coming up with results of their own.  A
tremendous advantage would be that anyone could experiment with conjectures in
any field---the computer would offer instant feedback as to whether
an inference step was correct.

Mathematicians sometimes have to put up with the annoyance of
cranks\index{cranks} who lack a fundamental understanding of mathematics but
insist that their ``proofs'' of, say, Fermat's Last Theorem\index{Fermat's
Last Theorem} be taken seriously.  I think part of the problem is that these
people are misled by informal mathematical language, treating it as if they
were reading ordinary expository English and failing to appreciate the
implicit underlying rigor.  Such cranks are rare in the field of computers,
because computer languages are much more explicit, and ultimately the proof is
in whether a computer program works or not.  With easily accessible
computer-based abstract mathematics, a mathematician could say to a crank,
``don't bother me until you've demonstrated your claim on the computer!''

% 22-May-04 nm
% Attempt to move De Millo quote so it doesn't separate from attribution
% CHANGE THIS NUMBER (AND ELIMINATE IF POSSIBLE) WHEN ABOVE TEXT CHANGES
\vspace{-0.5em}

\subsection{An Impossible Dream?}\label{dream}

\begin{quote}
  {\em Even quite basic theorems would demand almost unbelievably vast
  books to display their proofs.}
    \flushright\sc  Robert E. Edwards\footnote{\cite{Edwards}, p.~68.}\\
\end{quote}\index{Edwards, Robert E.}

\begin{quote}
  {\em Oh, of course no one ever really {\em does} it.  It would take
  forever!  You just show that you could do it, that's sufficient.}
    \flushright\sc  ``The Ideal Mathematician''
    \index{Davis, Phillip J.}\footnote{\cite{Davis},
p.~40.}\\
\end{quote}

\begin{quote}
  {\em There is a theorem in the primitive notation of set theory that
  corresponds to the arithmetic theorem `$1000+2000=3000$'.  The formula
  would be forbiddingly long\ldots even if [one] knows the definitions
  and is asked to simplify the long formula according to them, chances are
  he will make errors and arrive at some incorrect result.}
    \flushright\sc  Hao Wang\footnote{\cite{Wang}, p.~140.}\\
\end{quote}\index{Wang, Hao}

% 22-May-04 nm
% Attempt to move De Millo quote so it doesn't separate from attribution
% CHANGE THIS NUMBER (AND ELIMINATE IF POSSIBLE) WHEN ABOVE TEXT CHANGES
\vspace{-0.5em}

\begin{quote}
  {\em The {\em Principia Mathematica} was the crowning achievement of the
  formalists.  It was also the deathblow of the formalist view.\ldots
  {[Rus\-sell]} failed, in three enormous volumes, to get beyond the elementary
  facts of arithmetic.  He showed what can be done in principle and what
  cannot be done in practice.  If the mathematical process were really
  one of strict, logical progression, we would still be counting our
  fingers.\ldots
  One theoretician estimates\ldots that a demonstration of one of
  Ramanujan's conjectures assuming set theory and elementary analysis would
  take about two thousand pages; the length of a deduction from first principles
  is nearly in\-con\-ceiv\-a\-ble\ldots The probabilists argue that\ldots any
  very long proof can at best be viewed as only probably correct\ldots}
  \flushright\sc Richard de Millo et. al.\footnote{\cite{deMillo}, pp.~269,
  271.}\\
\end{quote}\index{de Millo, Richard}

A number of writers have conveyed the impression that the kind of absolute
rigor provided by Metamath\index{Metamath} is an impossible dream, suggesting
that a complete, formal verification\index{formal proof} of a typical theorem
would take millions of steps in untold volumes of books.  Even if it could be
done, the thinking sometimes goes, all meaning would be lost in such a
monstrous, tedious verification.\index{informal proof}\index{proof length}

These writers assume, however, that in order to achieve the kind of complete
formal verification they desire one must break down a proof into individual
primitive steps that make direct reference to the axioms.  This is
not necessary.  There is no reason not to make use of previously proved
theorems rather than proving them over and over.

Just as important, definitions\index{definition} can be introduced along
the way, allowing very complex formulas to be represented with few
symbols.  Not doing this can lead to absurdly long formulas.  For
example, the mere statement of
G\"{o}del's incompleteness theorem\index{G\"{o}del's
incompleteness theorem}, which can be expressed with a small number of
defined symbols, would require about 20,000 primitive symbols to express
it.\index{Boolos, George S.}\footnote{George S.\ Boolos, lecture at
Massachusetts Institute of Technology, spring 1990.} An extreme example
is Bourbaki's\label{bourbaki} language for set theory, which requires
4,523,659,424,929 symbols plus 1,179,618,517,981 disambiguatory links
(lines connecting symbol pairs, usually drawn below or above the
formula) to express the number
``one'' \cite{Mathias}.\index{Mathias, Adrian R. D.}\index{Bourbaki,
Nicolas}
% http://www.dpmms.cam.ac.uk/~ardm/

A hierarchy\index{hierarchy} of theorems and definitions permits an
exponential growth in the formula sizes and primitive proof steps to be
described with only a linear growth in the number of symbols used.  Of course,
this is how ordinary informal mathematics is normally done anyway, but with
Metamath\index{Metamath} it can be done with absolute rigor and precision.

\subsection{Beauty}


\begin{quote}
  {\em No one shall be able to drive us from the paradise that Cantor has
created for us.}
   \flushright\sc  David Hilbert\footnote{As quoted in \cite{Moore}, p.~131.}\\
\end{quote}\index{Hilbert, David}

\needspace{3\baselineskip}
\begin{quote}
  {\em Mathematics possesses not only truth, but some supreme beauty ---a
  beauty cold and austere, like that of a sculpture.}
    \flushright\sc  Bertrand
    Russell\footnote{\cite{Russell}.}\\
\end{quote}\index{Russell, Bertrand}

\begin{quote}
  {\em Euclid alone has looked on Beauty bare.}
  \flushright\sc Edna Millay\footnote{As quoted in \cite{Davis}, p.~150.}\\
\end{quote}\index{Millay, Edna}

For most people, abstract mathematics is distant, strange, and
incomprehensible.  Many popular books have tried to convey some of the sense
of beauty in famous theorems.  But even an intelligent layperson is left with
only a general idea of what a theorem is about and is hardly given the tools
needed to make use of it.  Traditionally, it is only after years of arduous
study that one can grasp the concepts needed for deep understanding.
Metamath\index{Metamath} allows you to approach the proof of the theorem from
a quite different perspective, peeling apart the formulas and definitions
layer by layer until an entirely different kind of understanding is achieved.
Every step of the proof is there, pieced together with absolute precision and
instantly available for inspection through a microscope with a magnification
as powerful as you desire.

A proof in itself can be considered an object of beauty.  Constructing an
elegant proof is an art.  Once a famous theorem has been proved, often
considerable effort is made to find simpler and more easily understood
proofs.  Creating and communicating elegant proofs is a major concern of
mathematicians.  Metamath is one way of providing a common language for
archiving and preserving this information.

The length of a proof can, to a certain extent, be considered an
objective measure of its ``beauty,'' since shorter proofs are usually
considered more elegant.  In the set theory database
\texttt{set.mm}\index{set theory database (\texttt{set.mm})}%
\index{Metamath Proof Explorer}
provided with Metamath, one goal was to make all proofs as short as possible.

\needspace{4\baselineskip}
\subsection{Simplicity}

\begin{quote}
  {\em God made man simple; man's complex problems are of his own
  devising.}
    \flushright\sc Eccles. 7:29\footnote{Jerusalem Bible.}\\
\end{quote}\index{Bible}

\needspace{3\baselineskip}
\begin{quote}
  {\em God made integers, all else is the work of man.}
    \flushright\sc Leopold Kronecker\footnote{{\em Jahresbericht
	der Deutschen Mathematiker-Vereinigung }, vol. 2, p. 19.}\\
\end{quote}\index{Kronecker, Leopold}

\needspace{3\baselineskip}
\begin{quote}
  {\em For what is clear and easily comprehended attracts; the
  complicated repels.}
    \flushright\sc David Hilbert\footnote{As quoted in \cite{deMillo},
p.~273.}\\
\end{quote}\index{Hilbert, David}

The Metamath\index{Metamath} language is simple and Spartan.  Metamath treats
all mathematical expressions as simple sequences of symbols, devoid of meaning.
The higher-level or ``metamathematical'' notions underlying Metamath are about
as simple as they could possibly be.  Each individual step in a proof involves
a single basic concept, the substitution of an expression for a variable, so
that in principle almost anyone, whether mathematician or not, can
completely understand how it was arrived at.

In one of its most basic applications, Metamath\index{Metamath} can be used to
develop the foundations of mathematics\index{foundations of mathematics} from
the very beginning.  This is done in the set theory database that is provided
with the Metamath package and is the subject matter
of Chapter~\ref{fol}. Any language (a metalanguage\index{metalanguage})
used to describe mathematics (an object language\index{object language}) must
have a mathematical content of its own, but it is desirable to keep this
content down to a bare minimum, namely that needed to make use of the
inference rules specified by the axioms.  With any metalanguage there is a
``chicken and egg'' problem somewhat like circular reasoning:  you must assume
the validity of the mathematics of the metalanguage in order to prove the
validity of the mathematics of the object language.  The mathematical content
of Metamath itself is quite limited.  Like the rules of a game of chess, the
essential concepts are simple enough so that virtually anyone should be able to
understand them (although that in itself will not let you play like
a master).  The symbols that Metamath manipulates do not in themselves
have any intrinsic meaning.  Your interpretation of the axioms that you supply
to Metamath is what gives them meaning.  Metamath is an attempt to strip down
mathematical thought to its bare essence and show you exactly how the symbols
are manipulated.

Philosophers and logicians, with various motivations, have often thought it
important to study ``weak'' fragments of logic\index{weak logic}
\cite{Anderson}\index{Anderson, Alan Ross} \cite{MegillBunder}\index{Megill,
Norman}\index{Bunder, Martin}, other unconventional systems of logic (such as
``modal'' logic\index{modal logic} \cite[ch.\ 27]{Boolos}\index{Boolos, George
S.}), and quantum logic\index{quantum logic} in physics
\cite{Pavicic}\index{Pavi{\v{c}}i{\'{c}}, M.}.  Metamath\index{Metamath}
provides a framework in which such systems can be expressed, with an absolute
precision that makes all underlying metamathematical assumptions rigorous and
crystal clear.

Some schools of philosophical thought, for example
intuitionism\index{intuitionism} and constructivism\index{constructivism},
demand that the notions underlying any mathematical system be as simple and
concrete as possible.  Metamath should meet the requirements of these
philosophies.  Metamath must be taught the symbols, axioms\index{axiom}, and
rules\index{rule} for a specific theory, from the skeptical (such as
intuitionism\index{intuitionism}\footnote{Intuitionism does not accept the law
of excluded middle (``either something is true or it is not true'').  See
\cite[p.~xi]{Tymoczko}\index{Tymoczko, Thomas} for discussion and references
on this topic.  Consider the theorem, ``There exist irrational numbers $a$ and
$b$ such that $a^b$ is rational.''  An intuitionist would reject the following
proof:  If $\sqrt{2}^{\sqrt{2}}$ is rational, we are done.  Otherwise, let
$a=\sqrt{2}^{\sqrt{2}}$ and $b=\sqrt{2}$. Then $a^b=2$, which is rational.})
to the bold (such as the axiom of choice in set theory\footnote{The axiom of
choice\index{Axiom of Choice} asserts that given any collection of pairwise
disjoint nonempty sets, there exists a set that has exactly one element in
common with each set of the collection.  It is used to prove many important
theorems in standard mathematics.  Some philosophers object to it because it
asserts the existence of a set without specifying what the set contains
\cite[p.~154]{Enderton}\index{Enderton, Herbert B.}.  In one foundation for
mathematics due to Quine\index{Quine, Willard Van Orman}, that has not been
otherwise shown to be inconsistent, the axiom of choice turns out to be false
\cite[p.~23]{Curry}\index{Curry, Haskell B.}.  The \texttt{show
trace{\char`\_}back} command of the Metamath program allows you to find out
whether the axiom of choice, or any other axiom, was assumed by a
proof.}\index{\texttt{show trace{\char`\_}back} command}).

The simplicity of the Metamath language lets the algorithm (computer program)
that verifies the validity of a Metamath proof be straightforward and
robust.  You can have confidence that the theorems it verifies really can be
derived from your axioms.

\subsection{Rigor}

\begin{quote}
  {\em Rigor became a goal with the Greeks\ldots But the efforts to
  pursue rigor to the utmost have led to an impasse in which there is
  no longer any agreement on what it really means.  Mathematics remains
  alive and vital, but only on a pragmatic basis.}
    \flushright\sc  Morris Kline\footnote{\cite{Kline}, p.~1209.}\\
\end{quote}\index{Kline, Morris}

Kline refers to a much deeper kind of rigor than that which we will discuss in
this section.  G\"{o}del's incompleteness theorem\index{G\"{o}del's
incompleteness theorem} showed that it is impossible to achieve absolute rigor
in standard mathematics because we can never prove that mathematics is
consistent (free from contradictions).\index{consistent theory}  If
mathematics is consistent, we will never know it, but must rely on faith.  If
mathematics is inconsistent, the best we can hope for is that some clever
future mathematician will discover the inconsistency.  In this case, the
axioms would probably be revised slightly to eliminate the inconsistency, as
was done in the case of Russell's paradox,\index{Russell's paradox} but the
bulk of mathematics would probably not be affected by such a discovery.
Russell's paradox, for example, did not affect most of the remarkable results
achieved by 19th-century and earlier mathematicians.  It mainly invalidated
some of Gottlob Frege's\index{Frege, Gottlob} work on the foundations of
mathematics in the late 1800's; in fact Frege's work inspired Russell's
discovery.  Despite the paradox, Frege's work contains important concepts that
have significantly influenced modern logic.  Kline's {\em Mathematics, The
Loss of Certainty} \cite{Klinel}\index{Kline, Morris} has an interesting
discussion of this topic.

What {\em can} be achieved with absolute certainty\index{certainty} is the
knowledge that if we assume the axioms are consistent and true, then the
results derived from them are true.  Part of the beauty of mathematics is that
it is the one area of human endeavor where absolute certainty can be achieved
in this sense.  A mathematical truth will remain such for eternity.  However,
our actual knowledge of whether a particular statement is a mathematical truth
is only as certain as the correctness of the proof that establishes it.  If
the proof of a statement is questionable or vague, we can't have absolute
confidence in the truth that the statement claims.

Let us look at some traditional ways of expressing proofs.

Except in the field of formal logic\index{formal logic}, almost all
traditional proofs in mathematics are really not proofs at all, but rather
proof outlines or hints as to how to go about constructing the proof.  Many
gaps\index{gaps in proofs} are left for the reader to fill in. There are
several reasons for this.  First, it is usually assumed in mathematical
literature that the person reading the proof is a mathematician familiar with
the specialty being described, and that the missing steps are obvious to such
a reader or at least that the reader is capable of filling them in.  This
attitude is fine for professional mathematicians in the specialty, but
unfortunately it often has the drawback of cutting off the rest of the world,
including mathematicians in other specialties, from understanding the proof.
We discussed one possible resolution to this on p.~\pageref{envision}.
Second, it is often assumed that a complete formal proof\index{formal proof}
would require countless millions of symbols (Section~\ref{dream}). This might
be true if the proof were to be expressed directly in terms of the axioms of
logic and set theory,\index{set theory} but it is usually not true if we allow
ourselves a hierarchy\index{hierarchy} of definitions and theorems to build
upon, using a notation that allows us to introduce new symbols, definitions,
and theorems in a precisely specified way.

Even in formal logic,\index{formal logic} formal proofs\index{formal proof}
that are considered complete still contain hidden or implicit information.
For example, a ``proof'' is usually defined as a sequence of
wffs,\index{well-formed formula (wff)}\footnote{A {\em wff} or well-formed
formula is a mathematical expression (string of symbols) constructed according
to some precise rules.  A formal mathematical system\index{formal system}
contains (1) the rules for constructing syntactically correct
wffs,\index{syntax rules} (2) a list of starting wffs called
axioms,\index{axiom} and (3) one or more rules prescribing how to derive new
wffs, called theorems\index{theorem}, from the axioms or previously derived
theorems.  An example of such a system is contained in
Metamath's\index{Metamath} set theory database, which defines a formal
system\index{formal system} from which all of standard mathematics can be
derived.  Section~\ref{startf} steps you through a complete example of a formal
system, and you may want to skim it now if you are unfamiliar with the
concept.} each of which is an axiom or follows from a rule applied to previous
wffs in the sequence.  The implicit part of the proof is the algorithm by
which a sequence of symbols is verified to be a valid wff, given the
definition of a wff.  The algorithm in this case is rather simple, but for a
computer to verify the proof,\index{automated proof verification} it must have
the algorithm built into its verification program.\footnote{It is possible, of
course, to specify wff construction syntax outside of the program itself
with a suitable input language (the Metamath language being an example), but
some proof-verification or theorem-proving programs lack the ability to extend
wff syntax in such a fashion.} If one deals exclusively with axioms and
elementary wffs, it is straightforward to implement such an algorithm.  But as
more and more definitions are added to the theory in order to make the
expression of wffs more compact, the algorithm becomes more and more
complicated.  A computer program that implements the algorithm becomes larger
and harder to understand as each definition is introduced, and thus more prone
to bugs.\index{computer program bugs}  The larger the program, the
more suspicious the mathematician may be about
the validity of its algorithms.  This is especially true because
computer programs are inherently hard to follow to begin with, and few people
enjoy verifying them manually in detail.

Metamath\index{Metamath} takes a different approach.  Metamath's ``knowledge''
is limited to the ability to substitute variables for expressions, subject to
some simple constraints.  Once the basic algorithm of Metamath is assumed to
be debugged, and perhaps independently confirmed, it
can be trusted once and for all.  The information that Metamath needs to
``understand'' mathematics is contained entirely in the body of knowledge
presented to Metamath.  Any errors in reasoning can only be errors in the
axioms or definitions contained in this body of knowledge.  As a
``constructive'' language\index{constructive language} Metamath has no
conditional branches or loops like the ones that make computer programs hard
to decipher; instead, the language can only build new sequences of symbols
from earlier sequences  of symbols.

The simplicity of the rules that underlie Metamath not only makes Metamath
easy to learn but also gives Metamath a great deal of flexibility. For
example, Metamath is not limited to describing standard first-order
logic\index{first-order logic}; higher-order logics\index{higher-order logic}
and fragments of logic\index{weak logic} can be described just as easily.
Metamath gives you the freedom to define whatever wff notation you prefer; it
has no built-in conception of the syntax of a wff.\index{well-formed formula
(wff)}  With suitable axioms and definitions, Metamath can even describe and
prove things about itself.\index{Metamath!self-description}  (John
Harrison\index{Harrison, John} discusses the ``reflection''
principle\index{reflection principle} involved in self-descriptive systems in
\cite{Harrison}.)

The flexibility of Metamath requires that its proofs specify a lot of detail,
much more than in an ordinary ``formal'' proof.\index{formal proof}  For
example, in an ordinary formal proof, a single step consists of displaying the
wff that constitutes that step.  In order for a computer program to verify
that the step is acceptable, it first must verify that the symbol sequence
being displayed is an acceptable wff.\index{automated proof verification} Most
proof verifiers have at least basic wff syntax built into their programs.
Metamath has no hard-wired knowledge of what constitutes a wff built into it;
instead every wff must be explicitly constructed based on rules defining wffs
that are present in a database.  Thus a single step in an ordinary formal
proof may be correspond to many steps in a Metamath proof. Despite the larger
number of steps, though, this does not mean that a Metamath proof must be
significantly larger than an ordinary formal proof. The reason is that since
we have constructed the wff from scratch, we know what the wff is, so there is
no reason to display it.  We only need to refer to a sequence of statements
that construct it.  In a sense, the display of the wff in an ordinary formal
proof is an implicit proof of its own validity as a wff; Metamath just makes
the proof explicit. (Section~\ref{proof} describes Metamath's proof notation.)

\section{Computers and Mathematicians}

\begin{quote}
  {\em The computer is important, but not to mathematics.}
    \flushright\sc  Paul Halmos\footnote{As quoted in \cite{Albers}, p.~121.}\\
\end{quote}\index{Halmos, Paul}

Pure mathematicians have traditionally been indifferent to computers, even to
the point of disdain.\index{computers and pure mathematics}  Computer science
itself is sometimes considered to fall in the mundane realm of ``applied''
mathematics, perhaps essential for the real world but intellectually unexciting
to those who seek the deepest truths in mathematics.  Perhaps a reason for this
attitude towards computers is that there is little or no computer software that
meets their needs, and there may be a general feeling that such software could
not even exist.  On the one hand, there are the practical computer algebra
systems, which can perform amazing symbolic manipulations in algebra and
calculus,\index{computer algebra system} yet can't prove the simplest
existence theorem, if the idea of a proof is present at all.  On the other
hand, there are specialized automated theorem provers that technically speaking
may generate correct proofs.\index{automated theorem proving}  But sometimes
their specialized input notation may be cryptic and their output perceived to
be long, inelegant, incomprehensible proofs.    The output
may be viewed with suspicion, since the program that generates it tends to be
very large, and its size increases the potential for bugs\index{computer
program bugs}.  Such a proof may be considered trustworthy only if
independently verified and ``understood'' by a human, but no one wants to
waste their time on such a boring, unrewarding chore.



\needspace{4\baselineskip}
\subsection{Trusting the Computer}

\begin{quote}
  {\em \ldots I continue to find the quasi-empirical interpretation of
  computer proofs to be the more plausible.\ldots Since not
  everything that claims to be a computer proof can be
  accepted as valid, what are the mathematical criteria for acceptable
  computer proofs?}
    \flushright\sc  Thomas Tymoczko\footnote{\cite{Tymoczko}, p.~245.}\\
\end{quote}\index{Tymoczko, Thomas}

In some cases, computers have been essential tools for proving famous
theorems.  But if a proof is so long and obscure that it can be verified in a
practical way only with a computer, it is vaguely felt to be suspicious.  For
example, proving the famous four-color theorem\index{four-color
theorem}\index{proof length} (``a map needs no more than four colors to
prevent any two adjacent countries from having the same color'') can presently
only be done with the aid of a very complex computer program which originally
required 1200 hours of computer time. There has been considerable debate about
whether such a proof can be trusted and whether such a proof is ``real''
mathematics \cite{Swart}\index{Swart, E. R.}.\index{trusting computers}

However, under normal circumstances even a skeptical mathematician would have a
great deal of confidence in the result of multiplying two numbers on a pocket
calculator, even though the precise details of what goes on are hidden from its
user.  Even the verification on a supercomputer that a huge number is prime is
trusted, especially if there is independent verification; no one bothers to
debate the philosophical significance of its ``proof,'' even though the actual
proof would be so large that it would be completely impractical to ever write
it down on paper.  It seems that if the algorithm used by the computer is
simple enough to be readily understood, then the computer can be trusted.

Metamath\index{Metamath} adopts this philosophy.  The simplicity of its
language makes it easy to learn, and because of its simplicity one can have
essentially absolute confidence that a proof is correct. All axioms, rules, and
definitions are available for inspection at any time because they are defined
by the user; there are no hidden or built-in rules that may be prone to subtle
bugs\index{computer program bugs}.  The basic algorithm at the heart of
Metamath is simple and fixed, and it can be assumed to be bug-free and robust
with a degree of confidence approaching certainty.
Independently written implementations of the Metamath verifier
can reduce any residual doubt on the part of a skeptic even further;
there are now over a dozen such implementations, written by many people.

\subsection{Trusting the Mathematician}\label{trust}

\begin{quote}
  {\em There is no Algebraist nor Mathematician so expert in his science, as
  to place entire confidence in any truth immediately upon his discovery of it,
  or regard it as any thing, but a mere probability.  Every time he runs over
  his proofs, his confidence encreases; but still more by the approbation of
  his friends; and is rais'd to its utmost perfection by the universal assent
  and applauses of the learned world.}
  \flushright\sc David Hume\footnote{{\em A Treatise of Human Nature}, as
  quoted in \cite{deMillo}, p.~267.}\\
\end{quote}\index{Hume, David}

\begin{quote}
  {\em Stanislaw Ulam estimates that mathematicians publish 200,000 theorems
  every year.  A number of these are subsequently contradicted or otherwise
  disallowed, others are thrown into doubt, and most are ignored.}
  \flushright\sc Richard de Millo et. al.\footnote{\cite{deMillo}, p.~269.}\\
\end{quote}\index{Ulam, Stanislaw}

Whether or not the computer can be trusted, humans  of course will occasionally
err. Only the most memorable proofs get independently verified, and of these
only a handful of truly great ones achieve the status of being ``known''
mathematical truths that are used without giving a second thought to their
correctness.

There are many famous examples of incorrect theorems and proofs in
mathematical literature.\index{errors in proofs}

\begin{itemize}
\item There have been thousands of purported proofs of Fermat's Last
Theorem\index{Fermat's Last Theorem} (``no integer solutions exist to $x^n +
y^n = z^n$ for $n > 2$''), by amateurs, cranks, and well-regarded
mathematicians \cite[p.~5]{Stark}\index{Stark, Harold M}.  Fermat wrote a note
in his copy of Bachet's {\em Diophantus} that he found ``a truly marvelous
proof of this theorem but this margin is too narrow to contain it''
\cite[p.~507]{Kramer}.  A recent, much publicized proof by Yoichi
Miyaoka\index{Miyaoka, Yoichi} was shown to be incorrect ({\em Science News},
April 9, 1988, p.~230).  The theorem was finally proved by Andrew
Wiles\index{Wiles, Andrew} ({\em Science News}, July 3, 1993, p.~5), but it
initially had some gaps and took over a year after its announcement to be
checked thoroughly by experts.  On Oct. 25, 1994, Wiles announced that the last
gap found in his proof had been filled in.
  \item In 1882, M. Pasch discovered that an axiom was omitted from Euclid's
formulation of geometry\index{Euclidean geometry}; without it, the proofs of
important theorems of Euclid are not valid.  Pasch's axiom\index{Pasch's
axiom} states that a line that intersects one side of a triangle must also
intersect another side, provided that it does not touch any of the triangle's
vertices.  The omission of Pasch's axiom went unnoticed for 2000
years \cite[p.~160]{Davis}, in spite of (one presumes) the thousands of
students, instructors, and mathematicians who studied Euclid.
  \item The first published proof of the famous Schr\"{o}der--Bernstein
theorem\index{Schr\"{o}der--Bernstein theorem} in set theory was incorrect
\cite[p.~148]{Enderton}\index{Enderton, Herbert B.}.  This theorem states
that if there exists a 1-to-1 function\footnote{A {\em set}\index{set} is any
collection of objects. A {\em function}\index{function} or {\em
mapping}\index{mapping} is a rule that assigns to each element of one set
(called the function's {\em domain}\index{domain}) an element from another
set.} from set $A$ into set $B$ and vice-versa, then sets $A$ and $B$ have
a 1-to-1 correspondence.  Although it sounds simple and obvious,
the standard proof is quite long and complex.
  \item In the early 1900's, Hilbert\index{Hilbert, David} published a
purported proof of the continuum hypothesis\index{continuum hypothesis}, which
was eventually established as unprovable by Cohen\index{Cohen, Paul} in 1963
\cite[p.~166]{Enderton}.  The continuum hypothesis states that no
infinity\index{infinity} (``transfinite cardinal number'')\index{cardinal,
transfinite} exists whose size (or ``cardinality''\index{cardinality}) is
between the size of the set of integers and the size of the set of real
numbers.  This hypothesis originated with German mathematician Georg
Cantor\index{Cantor, Georg} in the late 1800's, and his inability to prove it
is said to have contributed to mental illness that afflicted him in his later
years.
  \item An incorrect proof of the four-color theorem\index{four-color theorem}
was published by Kempe\index{Kempe, A. B.} in 1879
\cite[p.~582]{Courant}\index{Courant, Richard}; it stood for 11 years before
its flaw was discovered.  This theorem states that any map can be colored
using only four colors, so that no two adjacent countries have the same
color.  In 1976 the theorem was finally proved by the famous computer-assisted
proof of Haken, Appel, and Koch \cite{Swart}\index{Appel, K.}\index{Haken,
W.}\index{Koch, K.}.  Or at least it seems that way.  Mathematician
H.~S.~M.~Coxeter\index{Coxeter, H. S. M.} has doubts \cite[p.~58]{Davis}:  ``I
have a feeling that that is an untidy kind of use of the computers, and the more
you correspond with Haken and Appel, the more shaky you seem to be.''
  \item Many false ``proofs'' of the Poincar\'{e}
conjecture\index{Poincar\'{e} conjecture} have been proposed over the years.
This conjecture states that any object that mathematically behaves like a
three-dimensional sphere is a three-dimensional sphere topologically,
regardless of how it is distorted.  In March 1986, mathematicians Colin
Rourke\index{Rourke, Colin} and Eduardo R\^{e}go\index{R\^{e}go, Eduardo}
caused  a stir in the mathematical community by announcing that they had found
a proof; in November of that year the proof was found to be false \cite[p.
218]{PetersonI}.  It was finally proved in 2003 by Grigory Perelman
\label{poincare}\index{Szpiro, George}\index{Perelman, Grigory}\cite{Szpiro}.
 \end{itemize}

Many counterexamples to ``theorems'' in recent mathematical
literature related to Clifford algebras\index{Clifford algebras}
 have been found by Pertti
Lounesto (who passed away in 2002).\index{Lounesto, Pertti}
See the web page \url{http://mathforum.org/library/view/4933.html}.
% http://users.tkk.fi/~ppuska/mirror/Lounesto/counterexamples.htm

One of the purposes of Metamath\index{Metamath} is to allow proofs to be
expressed with absolute precision.  Developing a proof in the Metamath
language can be challenging, because Metamath will not permit even the
tiniest mistake.\index{errors in proofs}  But once the proof is created, its
correctness can be trusted immediately, without having to depend on the
process of peer review for confirmation.

\section{The Use of Computers in Mathematics}

\subsection{Computer Algebra Systems}

For the most part, you will find that Metamath\index{Metamath} is not a
practical tool for manipulating numbers.  (Even proving that $2 + 2 = 4$, if
you start with set theory, can be quite complex!)  Several commercial
mathematics packages are quite good at arithmetic, algebra, and calculus, and
as practical tools they are invaluable.\index{computer algebra system} But
they have no notion of proof, and cannot understand statements starting with
``there exists such and such...''.

Software packages such as Mathematica \cite{Wolfram}\index{Mathematica} do not
concern themselves with proofs but instead work directly with known results.
These packages primarily emphasize heuristic rules such as the substitution of
equals for equals to achieve simpler expressions or expressions in a different
form.  Starting with a rich collection of built-in rules and algorithms, users
can add to the collection by means of a powerful programming language.
However, results such as, say, the existence of a certain abstract object
without displaying the actual object cannot be expressed (directly) in their
languages.  The idea of a proof from a small set of axioms is absent.  Instead
this software simply assumes that each fact or rule you add to the built-in
collection of algorithms is valid.  One way to view the software is as a large
collection of axioms from which the software, with certain goals, attempts to
derive new theorems, for example equating a complex expression with a simpler
equivalent. But the terms ``theorem''\index{theorem} and
``proof,''\index{proof} for example, are not even mentioned in the index of
the user's manual for Mathematica.\index{Mathematica and proofs}  What is also
unsatisfactory from a philosophical point of view is that there is no way to
ensure the validity of the results other than by trusting the writer of each
application module or tediously checking each module by hand, similar to
checking a computer program for bugs.\index{computer program
bugs}\footnote{Two examples illustrate why the knowledge database of computer
algebra systems should sometimes be regarded with a certain caution.  If you
ask Mathematica (version 3.0) to \texttt{Solve[x\^{ }n + y\^{ }n == z\^{ }n , n]}
it will respond with \texttt{\{\{n-\char`\>-2\}, \{n-\char`\>-1\},
\{n-\char`\>1\}, \{n-\char`\>2\}\}}. In other words, Mathematica seems to
``know'' that Fermat's Last Theorem\index{Fermat's Last Theorem} is true!  (At
the time this version of Mathematica was released this fact was unknown.)  If
you ask Maple\index{Maple} to \texttt{solve(x\^{ }2 = 2\^{ }x)} then
\texttt{simplify(\{"\})}, it returns the solution set \texttt{\{2, 4\}}, apparently
unaware that 0.7666647\ldots is also a solution.} While of course extremely
valuable in applied mathematics, computer algebra systems tend to be of little
interest to the theoretical mathematician except as aids for exploring certain
specific problems.

Because of possible bugs, trusting the output of a computer algebra system for
use as theorems in a proof-verifier would defeat the latter's goal of rigor.
On the other hand, a fact such that a certain relatively large number is
prime, while easy for a computer algebra system to derive, might have a long,
tedious proof that could overwhelm a proof-verifier. One approach for linking
computer algebra systems to a proof-verifier while retaining the advantages of
both is to add a hypothesis to each such theorem indicating its source.  For
example, a constant {\sc maple} could indicate the theorem came from the Maple
package, and would mean ``assuming Maple is consistent, then\ldots''  This and
many other topics concerning the formalization of mathematics are discussed in
John Harrison's\index{Harrison, John} very interesting
PhD thesis~\cite{Harrison-thesis}.

\subsection{Automated Theorem Provers}\label{theoremprovers}

A mathematical theory is ``decidable''\index{decidable theory} if a mechanical
method or algorithm exists that is guaranteed to determine whether or not a
particular formula is a theorem.  Among the few theories that are decidable is
elementary geometry,\index{Euclidean geometry} as was shown by a classic
result of logician Alfred Tarski\index{Tarski, Alfred} in 1948
\cite{Tarski}.\footnote{Tarski's result actually applies to a subset of the
geometry discussed in elementary textbooks.  This subset includes most of what
would be considered elementary geometry but it is not powerful enough to
express, among other things, the notions of the circumference and area of a
circle.  Extending the theory in a way that includes notions such as these
makes the theory undecidable, as was also shown by Tarski.  Tarski's algorithm
is far too inefficient to implement practically on a computer.  A practical
algorithm for proving a smaller subset of geometry theorems (those not
involving concepts of ``order'' or ``continuity'') was discovered by Chinese
mathematician Wu Wen-ts\"{u}n in 1977 \cite{Chou}\index{Chou,
Shang-Ching}.}\index{Wen-ts{\"{u}}n, Wu}  But most theories, including
elementary arithmetic, are undecidable.  This fact contributes to keeping
mathematics alive and well, since many mathematicians believe
that they will never be
replaced by computers (if they believe Roger Penrose's argument that a
computer can never replace the brain \cite{Penrose}\index{Penrose, Roger}).
In fact,  elementary geometry is often considered a ``dead'' field
for the simple reason that it is decidable.

On the other hand, the undecidability of a theory does not mean that one cannot
use a computer to search for proofs, providing one is willing to give up if a
proof is not found after a reasonable amount of time.  The field of automated
theorem proving\index{automated theorem proving} specializes in pursuing such
computer searches.  Among the more successful results to date are those based
on an algorithm known as Robinson's resolution principle
\cite{Robinson}\index{Robinson's resolution principle}.

Automated theorem provers can be excellent tools for those willing to learn
how to use them.  But they are not widely used in mainstream pure
mathematics, even though they could probably be useful in many areas.  There
are several reasons for this.  Probably most important, the main goal in pure
mathematics is to arrive at results that are considered to be deep or
important; proving them is essential but secondary.  Usually, an automated
theorem prover cannot assist in this main goal, and by the time the main goal
is achieved, the mathematician may have already figured out the proof as a
by-product.  There is also a notational problem.  Mathematicians are used to
using very compact syntax where one or two symbols (heavily dependent on
context) can represent very complex concepts; this is part of the
hierarchy\index{hierarchy} they have built up to tackle difficult problems.  A
theorem prover on the other hand might require that a theorem be expressed in
``first-order logic,''\index{first-order logic} which is the logic on which
most of mathematics is ultimately based but which is not ordinarily used
directly because expressions can become very long.  Some automated theorem
provers are experimental programs, limited in their use to very specialized
areas, and the goal of many is simply research into the nature of automated
theorem proving itself.  Finally, much research remains to be done to enable
them to prove very deep theorems.  One significant result was a
computer proof by Larry Wos\index{Wos, Larry} and colleagues that every Robbins
algebra\index{Robbins algebra} is a Boolean  algebra\index{Boolean algebra}
({\em New York Times}, Dec. 10, 1996).\footnote{In 1933, E.~V.\
Huntington\index{Huntington, E. V.}
presented the following axiom system for
Boolean algebra with a unary operation $n$ and a binary operation $+$:
\begin{center}
    $x + y = y + x$ \\
    $(x + y) + z = x + (y + z)$ \\
    $n(n(x) + y) + n(n(x) + n(y)) = x$
\end{center}
Herbert Robbins\index{Robbins, Herbert}, a student of Huntington, conjectured
that the last equation can be replaced with a simpler one:
\begin{center}
    $n(n(x + y) + n(x + n(y))) = x$
\end{center}
Robbins and Huntington could not find a proof.  The problem was
later studied unsuccessfully by Tarski\index{Tarski, Alfred} and his
students, and it remained an unsolved problem until a
computer found the proof in 1996.  For more information on
the Robbins algebra problem see \cite{Wos}.}

How does Metamath\index{Metamath} relate to automated theorem provers?  A
theorem prover is primarily concerned with one theorem at a time (perhaps
tapping into a small database of known theorems) whereas Metamath is more like
a theorem archiving system, storing both the theorem and its proof in a
database for access and verification.  Metamath is one answer to ``what do you
do with the output of a theorem prover?''  and could be viewed as the
next step in the process.  Automated theorem provers could be useful tools for
helping develop its database.
Note that very long, automatically
generated proofs can make your database fat and ugly and cause Metamath's proof
verification to take a long time to run.  Unless you have a particularly good
program that generates very concise proofs, it might be best to consider the
use of automatically generated proofs as a quick-and-dirty approach, to be
manually rewritten at some later date.

The program {\sc otter}\index{otter@{\sc otter}}\footnote{\url{http://www.cs.unm.edu/\~mccune/otter/}.}, later succeeded by
prover9\index{prover9}\footnote{\url{https://www.cs.unm.edu/~mccune/mace4/}.},
have been historically influential.
The E prover\index{E prover}\footnote{\url{https://github.com/eprover/eprover}.}
is a maintained automated theorem prover
for full first-order logic with equality.
There are many other automated theorem provers as well.

If you want to combine automated theorem provers with Metamath
consider investigating
the book {\em Automated Reasoning:  Introduction and Applications}
\cite{Wos}\index{Wos, Larry}.  This book discusses
how to use {\sc otter} in a way that can
not only able to generate
relatively efficient proofs, it can even be instructed to search for
shorter proofs.  The effective use of {\sc otter} (and similar tools)
does require a certain
amount of experience, skill, and patience.  The axiom system used in the
\texttt{set.mm}\index{set theory database (\texttt{set.mm})} set theory
database can be expressed to {\sc otter} using a method described in
\cite{Megill}.\index{Megill, Norman}\footnote{To use those axioms with
{\sc otter}, they must be restated in a way that eliminates the need for
``dummy variables.''\index{dummy variable!eliminating} See the Comment
on p.~\pageref{nodd}.} When successful, this method tends to generate
short and clever proofs, but my experiments with it indicate that the
method will find proofs within a reasonable time only for relatively
easy theorems.  It is still fun to experiment with.

Reference \cite{Bledsoe}\index{Bledsoe, W. W.} surveys a number of approaches
people have explored in the field of automated theorem proving\index{automated
theorem proving}.

\subsection{Interactive Theorem Provers}\label{interactivetheoremprovers}

Finding proofs completely automatically is difficult, so there
are some interactive theorem provers that allow a human to guide the
computer to find a proof.
Examples include
HOL Light\index{HOL light}%
\footnote{\url{https://www.cl.cam.ac.uk/~jrh13/hol-light/}.},
Isabelle\index{Isabelle}%
\footnote{\url{http://www.cl.cam.ac.uk/Research/HVG/Isabelle}.},
{\sc hol}\index{hol@{\sc hol}}%
\footnote{\url{https://hol-theorem-prover.org/}.},
and
Coq\index{Coq}\footnote{\url{https://coq.inria.fr/}.}.

A major difference between most of these tools and Metamath is that the
``proofs'' are actually programs that guide the program to find a proof,
and not the proof itself.
For example, an Isabelle/HOL proof might apply a step
\texttt{apply (blast dest: rearrange reduction)}. The \texttt{blast}
instruction applies
an automatic tableux prover and returns if it found a sequence of proof
steps that work... but the sequence is not considered part of the proof.

A good overview of
higher-level proof verification languages (such as {\sc lcf}\index{lcf@{\sc
lcf}} and {\sc hol}\index{hol@{\sc hol}})
is given in \cite{Harrison}.  All of these languages are fundamentally
different from Metamath in that much of the mathematical foundational
knowledge is embedded in the underlying proof-verification program, rather
than placed directly in the database that is being verified.
These can have a steep learning curve for those without a mathematical
background.  For example, one usually must have a fair understanding of
mathematical logic in order to follow their proofs.

\subsection{Proof Verifiers}\label{proofverifiers}

A proof verifier is a program that doesn't generate proofs but instead
verifies proofs that you give it.  Many proof verifiers have limited built-in
automated proof capabilities, such as figuring out simple logical inferences
(while still being guided by a person who provides the overall proof).  Metamath
has no built-in automated proof capability other than the limited
capability of its Proof Assistant.

Proof-verification languages are not used as frequently as they might be.
Pure mathematicians are more concerned with producing new results, and such
detail and rigor would interfere with that goal.  The use of computers in pure
mathematics is primarily focused on automated theorem provers (not verifiers),
again with the ultimate goal of aiding the creation of new mathematics.
Automated theorem provers are usually concerned with attacking one theorem at
time rather than making a large, organized database easily available to the
user.  Metamath is one way to help close this gap.

By itself Metamath is a mostly a proof verifier.
This does not mean that other approaches can't be used; the difference
is that in Metamath, the results of various provers must be recorded
step-by-step so that they can be verified.

Another proof-verification language is Mizar,\index{Mizar} which can display
its proofs in the informal language that mathematicians are accustomed to.
Information on the Mizar language is available at \url{http://mizar.org}.

For the working mathematician, Mizar is an excellent tool for rigorously
documenting proofs. Mizar typesets its proofs in the informal English used by
mathematicians (and, while fine for them, are just as inscrutable by
laypersons!). A price paid for Mizar is a relatively steep learning curve of a
couple of weeks.  Several mathematicians are actively formalizing different
areas of mathematics using Mizar and publishing the proofs in a dedicated
journal. Unfortunately the task of formalizing mathematics is still looked
down upon to a certain extent since it doesn't involve the creation of ``new''
mathematics.

The closest system to Metamath is
the {\em Ghilbert}\index{Ghilbert} proof language (\url{http://ghilbert.org})
system developed by
Raph Levien\index{Levien, Raph}.
Ghilbert is a formal proof checker heavily inspired by Metamath.
Ghilbert statements are s-expressions (a la Lisp), which is easy
for computers to parse but many people find them hard to read.
There are a number of differences in their specific constructs, but
there is at least one tool to translate some Metamath materials into Ghilbert.
As of 2019 the Ghilbert community is smaller and less active than the
Metamath community.
That said, the Metamath and Ghilbert communities overlap, and fruitful
conversations between them have occurred many times over the years.

\subsection{Creating a Database of Formalized Mathematics}\label{mathdatabase}

Besides Metamath, there are several other ongoing projects with the goal of
formalizing mathematics into computer-verifiable databases.
Understanding some history will help.

The {\sc qed}\index{qed project@{\sc qed} project}%
\footnote{\url{http://www-unix.mcs.anl.gov/qed}.}
project arose in 1993 and its goals were outlined in the
{\sc qed} manifesto.
The {\sc qed} manifesto was
a proposal for a computer-based database of all mathematical knowledge,
strictly formalized and with all proofs having been checked automatically.
The project had a conference in 1994 and another in 1995;
there was also a ``twenty years of the {\sc qed} manifesto'' workshop
in 2014.
Its ideals are regularly reraised.

In a 2007 paper, Freek Wiedijk identified two reasons
for the failure of the {\sc qed} project as originally envisioned:%
\cite{Wiedijk-revisited}\index{Wiedijk, Freek}

\begin{itemize}
\item Very few people are working on formalization of mathematics. There is no compelling application for fully mechanized mathematics.
\item Formalized mathematics does not yet resemble traditional mathematics. This is partly due to the complexity of mathematical notation, and partly to the limitations of existing theorem provers and proof assistants.
\end{itemize}

But this did not end the dream of
formalizing mathematics into computer-verifiable databases.
The problems that led to the {\sc qed} manifesto are still with us,
even though the challenges were harder than originally considered.
What has happened instead is that various independent projects have
worked towards formalizing mathematics into computer-verifiable databases,
each simultaneously competing and cooperating with each other.

A concrete way to see this is
Freek Wiedijk's ``Formalizing 100 Theorems'' list%
\footnote{\url{http://www.cs.ru.nl/\%7Efreek/100/}.}
which shows the progress different systems have made on a challenge list
of 100 mathematical theorems.%
\footnote{ This is not the only list of ``interesting'' theorems.
Another interesting list was posted by Oliver Knill's list
\cite{Knill}\index{Knill, Oliver}.}
The top systems as of February 2019
(in order of the number of challenges completed) are
HOL Light, Isabelle, Metamath, Coq, and Mizar.

The Metamath 100%
\footnote{\url{http://us.metamath.org/mm\_100.html}}
page (maintained by David A. Wheeler\index{Wheeler, David A.})
shows the progress of Metamath (specifically its \texttt{set.mm} database)
against this challenge list maintained by Freek Wiedijk.
The Metamath \texttt{set.mm} database
has made a lot of progress over the years,
in part because working to prove those challenge theorems required
defining various terms and proving their properties as a prerequisite.
Here are just a few of the many statements that have been
formally proven with Metamath:

% The entries of this cause the narrow display to break poorly,
% since the short amount of text means LaTeX doesn't get a lot to work with
% and the itemize format gives it even *less* margin than usual.
% No one will mind if we make just this list flushleft, since this list
% will be internally consistent.
\begin{flushleft}
\begin{itemize}
\item 1. The Irrationality of the Square Root of 2
  (\texttt{sqr2irr}, by Norman Megill, 2001-08-20)
\item 2. The Fundamental Theorem of Algebra
  (\texttt{fta}, by Mario Carneiro, 2014-09-15)
\item 22. The Non-Denumerability of the Continuum
  (\texttt{ruc}, by Norman Megill, 2004-08-13)
\item 54. The Konigsberg Bridge Problem
  (\texttt{konigsberg}, by Mario Carneiro, 2015-04-16)
\item 83. The Friendship Theorem
  (\texttt{friendship}, by Alexander W. van der Vekens, 2018-10-09)
\end{itemize}
\end{flushleft}

We thank all of those who have developed at least one of the Metamath 100
proofs, and we particularly thank
Mario Carneiro\index{Carneiro, Mario}
who has contributed the most Metamath 100 proofs as of 2019.
The Metamath 100 page shows the list of all people who have contributed a
proof, and links to graphs and charts showing progress over time.
We encourage others to work on proving theorems not yet proven in Metamath,
since doing so improves the work as a whole.

Each of the math formalization systems (including Metamath)
has different strengths and weaknesses, depending on what you value.
Key aspects that differentiate Metamath from the other top systems are:

\begin{itemize}
\item Metamath is not tied to any particular set of axioms.
\item Metamath can show every step of every proof, no exceptions.
  Most other provers only assert that a proof can be found, and do not
  show every step. This also makes verification fast, because
  the system does not need to rediscover proof details.
\item The Metamath verifier has been re-implemented in many different
  programming languages, so verification can be done by multiple
  implementations.  In particular, the
  \texttt{set.mm}\index{set theory database (\texttt{set.mm})}%
  \index{Metamath Proof Explorer} database is verified by
  four different verifiers
  written in four different languages by four different authors.
  This greatly reduces the risk of accepting an invalid
  proof due to an error in the verifier.
\item Proofs stay proven.  In some systems, changes to the system's
  syntax or how a tactic works causes proofs to fail in later versions,
  causing older work to become essentially lost.
  Metamath's language is
  extremely small and fixed, so once a proof is added to a database,
  the database can be rechecked with later versions of the Metamath program
  and with other verifiers of Metamath databases.
  If an axiom or key definition needs to be changed, it is easy to
  manipulate the database as a whole to handle the change
  without touching the underlying verifier.
  Since re-verification of an entire database takes seconds, there
  is never a reason to delay complete verification.
  This aspect is especially compelling if your
  goal is to have a long-term database of proofs.
\item Licensing is generous.  The main Metamath databases are released to
  the public domain, and the main Metamath program is open source software
  under a standard, widely-used license.
\item Substitutions are easy to understand, even by those who are not
  professional mathematicians.
\end{itemize}

Of course, other systems may have advantages over Metamath
that are more compelling, depending on what you value.
In any case, we hope this helps you understand Metamath
within a wider context.

\subsection{In Summary}\label{computers-summary}

To summarize our discussions of computers and mathematics, computer algebra
systems can be viewed as theorem generators focusing on a narrow realm of
mathematics (numbers and their properties), automated theorem provers as proof
generators for specific theorems in a much broader realm covered by a built-in
formal system such as first-order logic, interactive theorem
provers require human guidance, proof verifiers verify proofs but
historically they have been
restricted to first-order logic.
Metamath, in contrast,
is a proof verifier and documenter whose realm is essentially unlimited.

\section{Mathematics and Metamath}

\subsection{Standard Mathematics}

There are a number of ways that Metamath\index{Metamath} can be used with
standard mathematics.  The most satisfying way philosophically is to start at
the very beginning, and develop the desired mathematics from the axioms of
logic and set theory.\index{set theory}  This is the approach taken in the
\texttt{set.mm}\index{set theory database (\texttt{set.mm})}%
\index{Metamath Proof Explorer}
database (also known as the Metamath Proof Explorer).
Among other things, this database builds up to the
axioms of real and complex numbers\index{analysis}\index{real and complex
numbers} (see Section~\ref{real}), and a standard development of analysis, for
example, could start at that point, using it as a basis.   Besides this
philosophical advantage, there are practical advantages to having all of the
tools of set theory available in the supporting infrastructure.

On the other hand, you may wish to start with the standard axioms of a
mathematical theory without going through the set theoretical proofs of those
axioms.  You will need mathematical logic to make inferences, but if you wish
you can simply introduce theorems\index{theorem} of logic as
``axioms''\index{axiom} wherever you need them, with the implicit assumption
that in principle they can be proved, if they are obvious to you.  If you
choose this approach, you will probably want to review the notation used in
\texttt{set.mm}\index{set theory database (\texttt{set.mm})} so that your own
notation will be consistent with it.

\subsection{Other Formal Systems}
\index{formal system}

Unlike some programs, Metamath\index{Metamath} is not limited to any specific
area of mathematics, nor committed to any particular mathematical philosophy
such as classical logic versus intuitionism, nor limited, say, to expressions
in first-order logic.  Although the database \texttt{set.mm}
describes standard logic and set theory, Meta\-math
is actually a general-purpose language for describing a wide variety of formal
systems.\index{formal system}  Non-standard systems such as modal
logic,\index{modal logic} intuitionist logic\index{intuitionism}, higher-order
logic\index{higher-order logic}, quantum logic\index{quantum logic}, and
category theory\index{category theory} can all be described with the Metamath
language.  You define the symbols you prefer and tell Metamath the axioms and
rules you want to start from, and Metamath will verify any inferences you make
from those axioms and rules.  A simple example of a non-standard formal system
is Hofstadter's\index{Hofstadter, Douglas R.} MIU system,\index{MIU-system}
whose Metamath description is presented in Appendix~\ref{MIU}.

This is not hypothetical.
The largest Metamath database is
\texttt{set.mm}\index{set theory database (\texttt{set.mm}}%
\index{Metamath Proof Explorer}), aka the Metamath Proof Explorer,
which uses the most common axioms for mathematical foundations
(specifically classical logic combined with Zermelo--Fraenkel
set theory\index{Zermelo--Fraenkel set theory} with the Axiom of Choice).
But other Metamath databases are available:

\begin{itemize}
\item The database
  \texttt{iset.mm}\index{intuitionistic logic database (\texttt{iset.mm})},
  aka the
  Intuitionistic Logic Explorer\index{Intuitionistic Logic Explorer},
  uses intuitionistic logic (a constructivist point of view)
  instead of classical logic.
\item The database
  \texttt{nf.mm}\index{New Foundations database (\texttt{nf.mm})},
  aka the
  New Foundations Explorer\index{New Foundations Explorer},
  constructs mathematics from scratch,
  starting from Quine's New Foundations (NF) set theory axioms.
\item The database
  \texttt{hol.mm}\index{Higher-order Logic database (\texttt{hol.mm})},
  aka the
  Higher-Order Logic (HOL) Explorer\index{Higher-Order Logic (HOL) Explorer},
  starts with HOL (also called simple type theory) and derives
  equivalents to ZFC axioms, connecting the two approaches.
\end{itemize}

Since the days of David Hilbert,\index{Hilbert, David} mathematicians have
been concerned with the fact that the metalanguage\index{metalanguage} used to
describe mathematics may be stronger than the mathematics being described.
Metamath\index{Metamath}'s underlying finitary\index{finitary proof},
constructive nature provides a good philosophical basis for studying even the
weakest logics.\index{weak logic}

The usual treatment of many non-standard formal systems\index{formal
system} uses model theory\index{model theory} or proof theory\index{proof
theory} to describe these systems; these theories, in turn, are based on
standard set theory.  In other words, a non-standard formal system is defined
as a set with certain properties, and standard set theory is used to derive
additional properties of this set.  The standard set theory database provided
with Metamath can be used for this purpose, and when used this way
the development of a special
axiom system for the non-standard formal system becomes unnecessary.  The
model- or proof-theoretic approach often allows you to prove much deeper
results with less effort.

Metamath supports both approaches.  You can define the non-standard
formal system directly, or define the non-standard formal system as
a set with certain properties, whichever you find most helpful.

%\section{Additional Remarks}

\subsection{Metamath and Its Philosophy}

Closely related to Metamath\index{Metamath} is a philosophy or way of looking
at mathematics. This philosophy is related to the formalist
philosophy\index{formalism} of Hilbert\index{Hilbert, David} and his followers
\cite[pp.~1203--1208]{Kline}\index{Kline, Morris}
\cite[p.~6]{Behnke}\index{Behnke, H.}. In this philosophy, mathematics is
viewed as nothing more than a set of rules that manipulate symbols, together
with the consequences of those rules.  While the mathematics being described
may be complex, the rules used to describe it (the
``metamathematics''\index{metamathematics}) should be as simple as possible.
In particular, proofs should be restricted to dealing with concrete objects
(the symbols we write on paper rather than the abstract concepts they
represent) in a constructive manner; these are called ``finitary''
proofs\index{finitary proof} \cite[pp.~2--3]{Shoenfield}\index{Shoenfield,
Joseph R.}.

Whether or not you find Metamath interesting or useful will in part depend on
the appeal you find in its philosophy, and this appeal will probably depend on
your particular goals with respect to mathematics.  For example, if you are a
pure mathematician at the forefront of discovering new mathematical knowledge,
you will probably find that the rigid formality of Metamath stifles your
creativity.  On the other hand, we would argue that once this knowledge is
discovered, there are advantages to documenting it in a standard format that
will make it accessible to others.  Sixty years from now, your field may be
dormant, and as Davis and Hersh put it, your ``writings would become less
translatable than those of the Maya'' \cite[p.~37]{Davis}\index{Davis, Phillip
J.}.


\subsection{A History of the Approach Behind Metamath}

Probably the one work that has had the most motivating influence on
Metamath\index{Metamath} is Whitehead and Russell's monumental {\em Principia
Mathematica} \cite{PM}\index{Whitehead, Alfred North}\index{Russell,
Bertrand}\index{principia mathematica@{\em Principia Mathematica}}, whose aim
was to deduce all of mathematics from a small number of primitive ideas, in a
very explicit way that in principle anyone could understand and follow.  While
this work was tremendously influential in its time, from a modern perspective
it suffers from several drawbacks.  Both its notation and its underlying
axioms are now considered dated and are no longer used.  From our point of
view, its development is not really as accessible as we would like to see; for
practical reasons, proofs become more and more sketchy as its mathematics
progresses, and working them out in fine detail requires a degree of
mathematical skill and patience that many people don't have.  There are
numerous small errors, which is understandable given the tedious, technical
nature of the proofs and the lack of a computer to verify the details.
However, even today {\em Principia Mathematica} stands out as the work closest
in spirit to Metamath.  It remains a mind-boggling work, and one can't help
but be amazed at seeing ``$1+1=2$'' finally appear on page 83 of Volume II
(Theorem *110.643).

The origin of the proof notation used by Metamath dates back to the 1950's,
when the logician C.~A.~Meredith expressed his proofs in a compact notation
called ``condensed detachment''\index{condensed detachment}
\cite{Hindley}\index{Hindley, J. Roger} \cite{Kalman}\index{Kalman, J. A.}
\cite{Meredith}\index{Meredith, C. A.} \cite{Peterson}\index{Peterson, Jeremy
George}.  This notation allows proofs to be communicated unambiguously by
merely referencing the axiom\index{axiom}, rule\index{rule}, or
theorem\index{theorem} used at each step, without explicitly indicating the
substitutions\index{substitution!variable}\index{variable substitution} that
have to be made to the variables in that axiom, rule, or theorem.  Ordinarily,
condensed detachment is more or less limited to propositional
calculus\index{propositional calculus}.  The concept has been extended to
first-order logic\index{first-order logic} in \cite{Megill}\index{Megill,
Norman}, making it is easy to write a small computer program to verify proofs
of simple first-order logic theorems.\index{condensed detachment!and
first-order logic}

A key concept behind the notation of condensed detachment is called
``unification,''\index{unification} which is an algorithm for determining what
substitutions\index{substitution!variable}\index{variable substitution} to
variables have to be made to make two expressions match each other.
Unification was first precisely defined by the logician J.~A.~Robinson, who
used it in the development of a powerful
theorem-proving technique called the ``resolution principle''
\cite{Robinson}\index{Robinson's resolution principle}. Metamath does not make
use of the resolution principle, which is intended for systems of first-order
logic.\index{first-order logic}  Metamath's use is not restricted to
first-order logic, and as we have mentioned it does not automatically discover
proofs.  However, unification is a key idea behind Metamath's proof
notation, and Metamath makes use of a very simple version of it
(Section~\ref{unify}).

\subsection{Metamath and First-Order Logic}

First-order logic\index{first-order logic} is the supporting structure
for standard mathematics.  On top of it is set theory, which contains
the axioms from which virtually all of mathematics can be derived---a
remarkable fact.\index{category
theory}\index{cardinal, inaccessible}\label{categoryth}\footnote{An exception seems
to be category theory.  There are several schools of thought on whether
category theory is derivable from set theory.  At a minimum, it appears
that an additional axiom is needed that asserts the existence of an
``inaccessible cardinal'' (a type of infinity so large that standard set
theory can't prove or deny that it exists).
%
%%%% (I took this out that was in previous editions:)
% But it is also argued that not just one but a ``proper class'' of them
% is needed, and the existence of proper classes is impossible in standard
% set theory.  (A proper class is a collection of sets so huge that no set
% can contain it as an element.  Proper classes can lead to
% inconsistencies such as ``Russell's paradox.''  The axioms of standard
% set theory are devised so as to deny the existence of proper classes.)
%
For more information, see
\cite[pp.~328--331]{Herrlich}\index{Herrlich, Horst} and
\cite{Blass}\index{Blass, Andrea}.}

One of the things that makes Metamath\index{Metamath} more practical for
first-order theories is a set of axioms for first-order logic designed
specifically with Metamath's approach in mind.  These are included in
the database \texttt{set.mm}\index{set theory database (\texttt{set.mm})}.
See Chapter~\ref{fol} for a detailed
description; the axioms are shown in Section~\ref{metaaxioms}.  While
logically equivalent to standard axiom systems, our axiom system breaks
up the standard axioms into smaller pieces such that from them, you can
directly derive what in other systems can only be derived as higher-level
``metatheorems.''\index{metatheorem}  In other words, it is more powerful than
the standard axioms from a metalogical point of view.  A rigorous
justification for this system and its ``metalogical
completeness''\index{metalogical completeness} is found in
\cite{Megill}\index{Megill, Norman}.  The system is closely related to a
system developed by Monk\index{Monk, J. Donald} and Tarski\index{Tarski,
Alfred} in 1965 \cite{Monks}.

For example, the formula $\exists x \, x = y $ (given $y$, there exists some
$x$ equal to it) is a theorem of logic,\footnote{Specifically, it is a theorem
of those systems of logic that assume non-empty domains.  It is not a theorem
of more general systems that include the empty domain\index{empty domain}, in
which nothing exists, period!  Such systems are called ``free
logics.''\index{free logic} For a discussion of these systems, see
\cite{Leblanc}\index{Leblanc, Hugues}.  Since our use for logic is as a basis
for set theory, which has a non-empty domain, it is more convenient (and more
traditional) to use a less general system.  An interesting curiosity is that,
using a free logic as a basis for Zermelo--Fraenkel set
theory\index{Zermelo--Fraenkel set theory} (with the redundant Axiom of the
Null Set omitted),\index{Axiom of the Null Set} we cannot even prove the
existence of a single set without assuming the axiom of infinity!\index{Axiom
of Infinity}} whether or not $x$ and $y$ are distinct variables\index{distinct
variables}.  In many systems of logic, we would have to prove two theorems to
arrive at this result.  First we would prove ``$\exists x \, x = x $,'' then
we would separately prove ``$\exists x \, x = y $, where $x$ and $y$ are
distinct variables.''  We would then combine these two special cases ``outside
of the system'' (i.e.\ in our heads) to be able to claim, ``$\exists x \, x =
y $, regardless of whether $x$ and $y$ are distinct.''  In other words, the
combination of the two special cases is a
metatheorem.  In the system of logic
used in Metamath's set theory\index{set theory database (\texttt{set.mm})}
database, the axioms of logic are broken down into small pieces that allow
them to be reassembled in such a way that theorems such as these can be proved
directly.

Breaking down the axioms in this way makes them look peculiar and not very
intuitive at first, but rest assured that they are correct and complete.  Their
correctness is ensured because they are theorem schemes of standard first-order
logic (which you can easily verify if you are a logician).  Their completeness
follows from the fact that we explicitly derive the standard axioms of
first-order logic as theorems.  Deriving the standard axioms is somewhat
tricky, but once we're there, we have at our disposal a system that is less
awkward to work with in formal proofs\index{formal proof}.  In technical terms
that logicians understand, we eliminate the cumbersome concepts of ``free
variable,''\index{free variable} ``bound variable,''\index{bound variable} and
``proper substitution''\index{proper substitution}\index{substitution!proper}
as primitive notions.  These concepts are present in our system but are
defined in terms of concepts expressed by the axioms and can be eliminated in
principle.  In standard systems, these concepts are really like additional,
implicit axioms\index{implicit axiom} that are somewhat complex and cannot be
eliminated.

The traditional approach to logic, wherein free variables and proper
substitution is defined, is also possible to do directly in the Metamath
language.  However, the notation tends to become awkward, and there are
disadvantages:  for example, extending the definition of a wff with a
definition is awkward, because the free variable and proper substitution
concepts have to have their definitions also extended.  Our choice of
axioms for \texttt{set.mm} is to a certain extent a matter of style, in
an attempt to achieve overall simplicity, but you should also be aware
that the traditional approach is possible as well if you should choose
it.

\chapter{Using the Metamath Program}
\label{using}

\section{Installation}

The way that you install Metamath\index{Metamath!installation} on your
computer system will vary for different computers.  Current
instructions are provided with the Metamath program download at
\url{http://metamath.org}.  In general, the installation is simple.
There is one file containing the Metamath program itself.  This file is
usually called \texttt{metamath} or \texttt{metamath.}{\em xxx} where
{\em xxx} is the convention (such as \texttt{exe}) for an executable
program on your operating system.  There are several additional files
containing samples of the Metamath language, all ending with
\texttt{.mm}.  The file \texttt{set.mm}\index{set theory database
(\texttt{set.mm})} contains logic and set theory and can be used as a
starting point for other areas of mathematics.

You will also need a text editor\index{text editor} capable of editing plain
{\sc ascii}\footnote{American Standard Code for Information Interchange.} text
in order to prepare your input files.\index{ascii@{\sc ascii}}  Most computers
have this capability built in.  Note that plain text is not necessarily the
default for some word processing programs\index{word processor}, especially if
they can handle different fonts; for example, with Microsoft Word\index{Word
(Microsoft)}, you must save the file in the format ``Text Only With Line
Breaks'' to get a plain text\index{plain text} file.\footnote{It is
recommended that all lines in a Metamath source file be 79 characters or less
in length for compatibility among different computer terminals.  When creating
a source file on an editor such as Word, select a monospaced
font\index{monospaced font} such as Courier\index{Courier font} or
Monaco\index{Monaco font} to make this easier to achieve.  Better yet,
just use a plain text editor such as Notepad.}

On some computer systems, Metamath does not have the capability to print
its output directly; instead, you send its output to a file (using the
\texttt{open} commands described later).  The way you print this output
file depends on your computer.\index{printers} Some computers have a
print command, whereas with others, you may have to read the file into
an editor and print it from there.

If you want to print your Metamath source files with typeset formulas
containing standard mathematical symbols, you will need the \LaTeX\
typesetting program\index{latex@{\LaTeX}}, which is widely and freely
available for most operating systems.  It runs natively on Unix and
Linux, and can be installed on Windows as part of the free Cygwin
package (\url{http://cygwin.com}).

You can also produce {\sc html}\footnote{HyperText Markup Language.}
web pages.  The {\tt help html} command in the Metamath program will
assist you with this feature.

\section{Your First Formal System}\label{start}
\subsection{From Nothing to Zero}\label{startf}

To give you a feel for what the Metamath\index{Metamath} language looks like,
we will take a look at a very simple example from formal number
theory\index{number theory}.  This example is taken from
Mendelson\index{Mendelson, Elliot} \cite[p. 123]{Mendelson}.\footnote{To keep
the example simple, we have changed the formalism slightly, and what we call
axioms\index{axiom} are strictly speaking theorems\index{theorem} in
\cite{Mendelson}.}  We will look at a small subset of this theory, namely that
part needed for the first number theory theorem proved in \cite{Mendelson}.

First we will look at a standard formal proof\index{formal proof} for the
example we have picked, then we will look at the Metamath version.  If you
have never been exposed to formal proofs, the notation may seem to be such
overkill to express such simple notions that you may wonder if you are missing
something.  You aren't.  The concepts involved are in fact very simple, and a
detailed breakdown in this fashion is necessary to express the proof in a way
that can be verified mechanically.  And as you will see, Metamath breaks the
proof down into even finer pieces so that the mechanical verification process
can be about as simple as possible.

Before we can introduce the axioms\index{axiom} of the theory, we must define
the syntax rules for forming legal expressions\index{syntax rules}
(combinations of symbols) with which those axioms can be used. The number 0 is
a {\bf term}\index{term}; and if $ t$ and $r$ are terms, so is $(t+r)$. Here,
$ t$ and $r$ are ``metavariables''\index{metavariable} ranging over terms; they
themselves do not appear as symbols in an actual term.  Some examples of
actual terms are $(0 + 0)$ and $((0+0)+0)$.  (Note that our theory describes
only the number zero and sums of zeroes.  Of course, not much can be done with
such a trivial theory, but remember that we have picked a very small subset of
complete number theory for our example.  The important thing for you to focus
on is our definitions that describe how symbols are combined to form valid
expressions, and not on the content or meaning of those expressions.) If $ t$
and $r$ are terms, an expression of the form $ t=r$ is a {\bf wff}
(well-formed formula)\index{well-formed formula (wff)}; and if $P$ and $Q$ are
wffs, so is $(P\rightarrow Q)$ (which means ``$P$ implies
$Q$''\index{implication ($\rightarrow$)} or ``if $P$ then $Q$'').
Here $P$ and $Q$ are metavariables ranging over wffs.  Examples of actual
wffs are $0=0$, $(0+0)=0$, $(0=0 \rightarrow (0+0)=0)$, and $(0=0\rightarrow
(0=0\rightarrow 0=(0+0)))$.  (Our notation makes use of more parentheses than
are customary, but the elimination of ambiguity this way simplifies our
example by avoiding the need to define operator precedence\index{operator
precedence}.)

The {\bf axioms}\index{axiom} of our theory are all wffs of the following
form, where $ t$, $r$, and $s$ are any terms:

%Latex p. 92
\renewcommand{\theequation}{A\arabic{equation}}

\begin{equation}
(t=r\rightarrow (t=s\rightarrow r=s))
\end{equation}
\begin{equation}
(t+0)=t
\end{equation}

Note that there are an infinite number of axioms since there are an infinite
number of possible terms.  A1 and A2 are properly called ``axiom
schemes,''\index{axiom scheme} but we will refer to them as ``axioms'' for
brevity.

An axiom is a {\bf theorem}; and if $P$ and $(P\rightarrow Q)$ are theorems
(where $P$ and $Q$ are wffs), then $Q$ is also a theorem.\index{theorem}  The
second part of this definition is called the modus ponens (MP) rule of
inference\index{inference rule}\index{modus ponens}.  It allows us to obtain
new theorems from old ones.

The {\bf proof}\index{proof} of a theorem is a sequence of one or more
theorems, each of which is either an axiom or the result of modus ponens
applied to two previous theorems in the sequence, and the last of which is the
theorem being proved.

The theorem we will prove for our example is very simple:  $ t=t$.  The proof of
our theorem follows.  Study it carefully until you feel sure you
understand it.\label{zeroproof}

% Use tabu so that lines will wrap automatically as needed.
\begin{tabu} { l X X }
1. & $(t+0)=t$ & (by axiom A2) \\
2. & $(t+0)=t$ & (by axiom A2) \\
3. & $((t+0)=t \rightarrow ((t+0)=t\rightarrow t=t))$ & (by axiom A1) \\
4. & $((t+0)=t\rightarrow t=t)$ & (by MP applied to steps 2 and 3) \\
5. & $t=t$ & (by MP applied to steps 1 and 4) \\
\end{tabu}

(You may wonder why step 1 is repeated twice.  This is not necessary in the
formal language we have defined, but in Metamath's ``reverse Polish
notation''\index{reverse Polish notation (RPN)} for proofs, a previous step
can be referred to only once.  The repetition of step~1 here will enable you
to see more clearly the correspondence of this proof with the
Metamath\index{Metamath} version on p.~\pageref{demoproof}.)

Our theorem is more properly called a ``theorem scheme,''\index{theorem
scheme} for it represents an infinite number of theorems, one for each
possible term $ t$.  Two examples of actual theorems would be $0=0$ and
$(0+0)=(0+0)$.  Rarely do we prove actual theorems, since by proving schemes
we can prove an infinite number of theorems in one fell swoop.  Similarly, our
proof should really be called a ``proof scheme.''\index{proof scheme}  To
obtain an actual proof, pick an actual term to use in place of $ t$, and
substitute it for $ t$ throughout the proof.

Let's discuss what we have done here.  The axioms\index{axiom} of our theory,
A1 and A2, are trivial and obvious.  Everyone knows that adding zero to
something doesn't change it, and also that if two things are equal to a third,
then they are equal to each other. In fact, stating the trivial and obvious is
a goal to strive for in any axiomatic system.  From trivial and obvious truths
that everyone agrees upon, we can prove results that are not so obvious yet
have absolute faith in them.  If we trust the axioms and the rules, we must,
by definition, trust the consequences of those axioms and rules, if logic is
to mean anything at all.

Our rule of inference\index{rule}, modus ponens\index{modus ponens}, is also
pretty obvious once you understand what it means.  If we prove a fact $P$, and
we also prove that $P$ implies $Q$, then $Q$ necessarily follows as a new
fact.  The rule provides us with a means for obtaining new facts (i.e.\
theorems\index{theorem}) from old ones.

The theorem that we have proved, $ t=t$, is so fundamental that you may wonder
why it isn't one of the axioms\index{axiom}.  In some axiom systems of
arithmetic, it {\em is} an axiom.  The choice of axioms in a theory is to some
extent arbitrary and even an art form, constrained only by the requirement
that any two equivalent axiom systems be able to derive each other as
theorems.  We could imagine that the inventor of our axiom system originally
included $ t=t$ as an axiom, then discovered that it could be derived as a
theorem from the other axioms.  Because of this, it was not necessary to
keep it as an axiom.  By eliminating it, the final set of axioms became
that much simpler.

Unless you have worked with formal proofs\index{formal proof} before, it
probably wasn't apparent to you that $ t=t$ could be derived from our two
axioms until you saw the proof. While you certainly believe that $ t=t$ is
true, you might not be able to convince an imaginary skeptic who believes only
in our two axioms until you produce the proof.  Formal proofs such as this are
hard to come up with when you first start working with them, but after you get
used to them they can become interesting and fun.  Once you understand the
idea behind formal proofs you will have grasped the fundamental principle that
underlies all of mathematics.  As the mathematics becomes more sophisticated,
its proofs become more challenging, but ultimately they all can be broken down
into individual steps as simple as the ones in our proof above.

Mendelson's\index{Mendelson, Elliot} book, from which our example was taken,
contains a number of detailed formal proofs such as these, and you may be
interested in looking it up.  The book is intended for mathematicians,
however, and most of it is rather advanced.  Popular literature describing
formal proofs\index{formal proof} include \cite[p.~296]{Rucker}\index{Rucker,
Rudy} and \cite[pp.~204--230]{Hofstadter}\index{Hofstadter, Douglas R.}.

\subsection{Converting It to Metamath}\label{convert}

Formal proofs\index{formal proof} such as the one in our example break down
logical reasoning into small, precise steps that leave little doubt that the
results follow from the axioms\index{axiom}.  You might think that this would
be the finest breakdown we can achieve in mathematics.  However, there is more
to the proof than meets the eye. Although our axioms were rather simple, a lot
of verbiage was needed before we could even state them:  we needed to define
``term,'' ``wff,'' and so on.  In addition, there are a number of implied
rules that we haven't even mentioned. For example, how do we know that step 3
of our proof follows from axiom A1? There is some hidden reasoning involved in
determining this.  Axiom A1 has two occurrences of the letter $ t$.  One of
the implied rules states that whatever we substitute for $ t$ must be a legal
term\index{term}.\footnote{Some authors make this implied rule explicit by
stating, ``only expressions of the above form are terms,'' after defining
``term.''}  The expression $ t+0$ is pretty obviously a legal term whenever $
t$ is, but suppose we wanted to substitute a huge term with thousands of
symbols?  Certainly a lot of work would be involved in determining that it
really is a term, but in ordinary formal proofs all of this work would be
considered a single ``step.''

To express our axiom system in the Metamath\index{Metamath} language, we must
describe this auxiliary information in addition to the axioms themselves.
Metamath does not know what a ``term'' or a ``wff''\index{well-formed formula
(wff)} is.  In Metamath, the specification of the ways in which we can combine
symbols to obtain terms and wffs are like little axioms in themselves.  These
auxiliary axioms are expressed in the same notation as the ``real''
axioms\index{axiom}, and Metamath does not distinguish between the two.  The
distinction is made by you, i.e.\ by the way in which you interpret the
notation you have chosen to express these two kinds of axioms.

The Metamath language breaks down mathematical proofs into tiny pieces, much
more so than in ordinary formal proofs\index{formal proof}.  If a single
step\index{proof step} involves the
substitution\index{substitution!variable}\index{variable substitution} of a
complex term for one of its variables, Metamath must see this single step
broken down into many small steps.  This fine-grained breakdown is what gives
Metamath generality and flexibility as it lets it not be limited to any
particular mathematical notation.

Metamath's proof notation is not, in itself, intended to be read by humans but
rather is in a compact format intended for a machine.  The Metamath program
will convert this notation to a form you can understand, using the \texttt{show
proof}\index{\texttt{show proof} command} command.  You can tell the program what
level of detail of the proof you want to look at.  You may want to look at
just the logical inference steps that correspond
to ordinary formal proof steps,
or you may want to see the fine-grained steps that prove that an expression is
a term.

Here, without further ado, is our example converted to the
Metamath\index{Metamath} language:\index{metavariable}\label{demo0}

\begin{verbatim}
$( Declare the constant symbols we will use $)
    $c 0 + = -> ( ) term wff |- $.
$( Declare the metavariables we will use $)
    $v t r s P Q $.
$( Specify properties of the metavariables $)
    tt $f term t $.
    tr $f term r $.
    ts $f term s $.
    wp $f wff P $.
    wq $f wff Q $.
$( Define "term" and "wff" $)
    tze $a term 0 $.
    tpl $a term ( t + r ) $.
    weq $a wff t = r $.
    wim $a wff ( P -> Q ) $.
$( State the axioms $)
    a1 $a |- ( t = r -> ( t = s -> r = s ) ) $.
    a2 $a |- ( t + 0 ) = t $.
$( Define the modus ponens inference rule $)
    ${
       min $e |- P $.
       maj $e |- ( P -> Q ) $.
       mp  $a |- Q $.
    $}
$( Prove a theorem $)
    th1 $p |- t = t $=
  $( Here is its proof: $)
       tt tze tpl tt weq tt tt weq tt a2 tt tze tpl
       tt weq tt tze tpl tt weq tt tt weq wim tt a2
       tt tze tpl tt tt a1 mp mp
     $.
\end{verbatim}\index{metavariable}

A ``database''\index{database} is a set of one or more {\sc ascii} source
files.  Here's a brief description of this Metamath\index{Metamath} database
(which consists of this single source file), so that you can understand in
general terms what is going on.  To understand the source file in detail, you
should read Chapter~\ref{languagespec}.

The database is a sequence of ``tokens,''\index{token} which are normally
separated by spaces or line breaks.  The only tokens that are built into
the Metamath language are those beginning with \texttt{\$}.  These tokens
are called ``keywords.''\index{keyword}  All other tokens are
user-defined, and their names are arbitrary.

As you might have guessed, the Metamath token \texttt{\$(}\index{\texttt{\$(} and
\texttt{\$)} auxiliary keywords} starts a comment and \texttt{\$)} ends a comment.

The Metamath tokens \texttt{\$c}\index{\texttt{\$c} statement},
\texttt{\$v}\index{\texttt{\$v} statement},
\texttt{\$e}\index{\texttt{\$e} statement},
\texttt{\$f}\index{\texttt{\$f} statement},
\texttt{\$a}\index{\texttt{\$a} statement}, and
\texttt{\$p}\index{\texttt{\$p} statement} specify ``statements'' that
end with \texttt{\$.}\,.\index{\texttt{\$.}\ keyword}

The Metamath tokens \texttt{\$c} and \texttt{\$v} each declare\index{constant
declaration}\index{variable declaration} a list of user-defined tokens, called
``math symbols,''\index{math symbol} that the database will reference later
on.  All of the math symbols they define you have seen earlier except the
turnstile symbol \texttt{|-} ($\vdash$)\index{turnstile ({$\,\vdash$})}, which is
commonly used by logicians to mean ``a proof exists for.''  For us
the turnstile is just a
convenient symbol that distinguishes expressions that are axioms\index{axiom}
or theorems\index{theorem} from expressions that are terms or wffs.

The \texttt{\$c} statement declares ``constants''\index{constant} and
the \texttt{\$v} statement declares
``variables''\index{variable}\index{constant declaration}\index{variable
declaration} (or more precisely, metavariables\index{metavariable}).  A
variable may be substituted\index{substitution!variable}\index{variable
substitution} with sequences of math symbols whereas a constant may not
be substituted with anything.

It may seem redundant to require both \texttt{\$c}\index{\texttt{\$c} statement} and
\texttt{\$v}\index{\texttt{\$v} statement} statements (since any math
symbol\index{math symbol} not specified with a \texttt{\$c} statement could be
presumed to be a variable), but this provides for better error checking and
also allows math symbols to be redeclared\index{redeclaration of symbols}
(Section~\ref{scoping}).

The token \texttt{\$f}\index{\texttt{\$f} statement} specifies a
statement called a ``variable-type hypothesis'' (also called a
``floating hypothesis'') and \texttt{\$e}\index{\texttt{\$e} statement}
specifies a ``logical hypothesis'' (also called an ``essential
hypothesis'').\index{hypothesis}\index{variable-type
hypothesis}\index{logical hypothesis}\index{floating
hypothesis}\index{essential hypothesis} The token
\texttt{\$a}\index{\texttt{\$a} statement} specifies an ``axiomatic
assertion,''\index{axiomatic assertion} and
\texttt{\$p}\index{\texttt{\$p} statement} specifies a ``provable
assertion.''\index{provable assertion} To the left of each occurrence of
these four tokens is a ``label''\index{label} that identifies the
hypothesis or assertion for later reference.  For example, the label of
the first axiomatic assertion is \texttt{tze}.  A \texttt{\$f} statement
must contain exactly two math symbols, a constant followed by a
variable.  The \texttt{\$e}, \texttt{\$a}, and \texttt{\$p} statements
each start with a constant followed by, in general, an arbitrary
sequence of math symbols.

Associated with each assertion\index{assertion} is a set of hypotheses
that must be satisfied in order for the assertion to be used in a proof.
These are called the ``mandatory hypotheses''\index{mandatory
hypothesis} of the assertion.  Among those hypotheses whose ``scope''
(described below) includes the assertion, \texttt{\$e} hypotheses are
always mandatory and \texttt{\$f}\index{\texttt{\$f} statement}
hypotheses are mandatory when they share their variable with the
assertion or its \texttt{\$e} hypotheses.  The exact rules for
determining which hypotheses are mandatory are described in detail in
Sections~\ref{frames} and \ref{scoping}.  For example, the mandatory
hypotheses of assertion \texttt{tpl} are \texttt{tt} and \texttt{tr},
whereas assertion \texttt{tze} has no mandatory hypotheses because it
contains no variables and has no \texttt{\$e}\index{\texttt{\$e}
statement} hypothesis.  Metamath's \texttt{show statement}
command\index{\texttt{show statement} command}, described in the next
section, will show you a statement's mandatory hypotheses.

Sometimes we need to make a hypothesis relevant to only certain
assertions.  The set of statements to which a hypothesis is relevant is
called its ``scope.''  The Metamath brackets,
\texttt{\$\char`\{}\index{\texttt{\$\char`\{} and \texttt{\$\char`\}}
keywords} and \texttt{\$\char`\}}, define a ``block''\index{block} that
delimits the scope of any hypothesis contained between them.  The
assertion \texttt{mp} has mandatory hypotheses \texttt{wp}, \texttt{wq},
\texttt{min}, and \texttt{maj}.  The only mandatory hypothesis of
\texttt{th1}, on the other hand, is \texttt{tt}, since \texttt{th1}
occurs outside of the block containing \texttt{min} and \texttt{maj}.

Note that \texttt{\$\char`\{} and \texttt{\$\char`\}} do not affect the
scope of assertions (\texttt{\$a} and \texttt{\$p}).  Assertions are always
available to be referenced by any later proof in the source file.

Each provable assertion (\texttt{\$p}\index{\texttt{\$p} statement}
statement) has two parts.  The first part is the
assertion\index{assertion} itself, which is a sequence of math
symbol\index{math symbol} tokens placed between the \texttt{\$p} token
and a \texttt{\$=}\index{\texttt{\$=} keyword} token.  The second part
is a ``proof,'' which is a list of label tokens placed between the
\texttt{\$=} token and the \texttt{\$.}\index{\texttt{\$.}\ keyword}\
token that ends the statement.\footnote{If you've looked at the
\texttt{set.mm} database, you may have noticed another notation used for
proofs.  The other notation is called ``compressed.''\index{compressed
proof}\index{proof!compressed} It reduces the amount of space needed to
store a proof in the database and is described in
Appendix~\ref{compressed}.  In the example above, we use
``normal''\index{normal proof}\index{proof!normal} notation.} The proof
acts as a series of instructions to the Metamath program, telling it how
to build up the sequence of math symbols contained in the assertion part of
the \texttt{\$p} statement, making use of the hypotheses of the
\texttt{\$p} statement and previous assertions.  The construction takes
place according to precise rules.  If the list of labels in the proof
causes these rules to be violated, or if the final sequence that results
does not match the assertion, the Metamath program will notify you with
an error message.

If you are familiar with reverse Polish notation (RPN), which is sometimes used
on pocket calculators, here in a nutshell is how a proof works.  Each
hypothesis label\index{hypothesis label} in the proof is pushed\index{push}
onto the RPN stack\index{stack}\index{RPN stack} as it is encountered. Each
assertion label\index{assertion label} pops\index{pop} off the stack as many
entries as the referenced assertion has mandatory hypotheses.  Variable
substitutions\index{substitution!variable}\index{variable substitution} are
computed which, when made to the referenced assertion's mandatory hypotheses,
cause these hypotheses to match the stack entries. These same substitutions
are then made to the variables in the referenced assertion itself, which is
then pushed onto the stack.  At the end of the proof, there should be one
stack entry, namely the assertion being proved.  This process is explained in
detail in Section~\ref{proof}.

Metamath's proof notation is not very readable for humans, but it allows the
proof to be stored compactly in a file.  The Metamath\index{Metamath} program
has proof display features that let you see what's going on in a more
readable way, as you will see in the next section.

The rules used in verifying a proof are not based on any built-in syntax of the
symbol sequence in an assertion\index{assertion} nor on any built-in meanings
attached to specific symbol names.  They are based strictly on symbol
matching:  constants\index{constant} must match themselves, and
variables\index{variable} may be replaced with anything that allows a match to
occur.  For example, instead of \texttt{term}, \texttt{0}, and \verb$|-$ we could
have just as well used \texttt{yellow}, \texttt{zero}, and \texttt{provable}, as long
as we did so consistently throughout the database.  Also, we could have used
\texttt{is provable} (two tokens) instead of \verb$|-$ (one token) throughout the
database.  In each of these cases, the proof would be exactly the same.  The
independence of proofs and notation means that you have a lot of flexibility to
change the notation you use without having to change any proofs.

\section{A Trial Run}\label{trialrun}

Now you are ready to try out the Metamath\index{Metamath} program.

On all computer systems, Metamath has a standard ``command line
interface'' (CLI)\index{command line interface (CLI)} that allows you to
interact with it.  You supply commands to the CLI by typing them on the
keyboard and pressing your keyboard's {\em return} key after each line
you enter.  The CLI is designed to be easy to use and has built-in help
features.

The first thing you should do is to use a text editor to create a file
called \texttt{demo0.mm} and type into it the Metamath source shown on
p.~\pageref{demo0}.  Actually, this file is included with your Metamath
software package, so check that first.  If you type it in, make sure
that you save it in the form of ``plain {\sc ascii} text with line
breaks.''  Most word processors will have this feature.

Next you must run the Metamath program.  Depending on your computer
system and how Metamath is installed, this could range from clicking the
mouse on the Metamath icon to typing \texttt{run metamath} to typing
simply \texttt{metamath}.  (Metamath's {\tt help invoke} command describes
alternate ways of invoking the Metamath program.)

When you first enter Metamath\index{Metamath}, it will be at the CLI, waiting
for your input. You will see something like the following on your screen:
\begin{verbatim}
Metamath - Version 0.172 25-Jan-2019
Type HELP for help, EXIT to exit.
MM>
\end{verbatim}
The \texttt{MM>} prompt means that Metamath is waiting for a command.
Command keywords\index{command keyword} are not case sensitive;
we will use lower-case commands in our examples.
The version number and its release date will probably be different on your
system from the one we show above.

The first thing that you need to do is to read in your
database:\index{\texttt{read} command}\footnote{If a directory path is
needed on Unix,\index{Unix file names}\index{file names!Unix} you should
enclose the path/file name in quotes to prevent Metamath from thinking
that the \texttt{/} in the path name is a command qualifier, e.g.,
\texttt{read \char`\"db/set.mm\char`\"}.  Quotes are optional when there
is no ambiguity.}
\begin{verbatim}
MM> read demo0.mm
\end{verbatim}
Remember to press the {\em return} key after entering this command.  If
you omit the file name, Metamath will prompt you for one.   The syntax for
specifying a Macintosh file name path is given in a footnote on
p.~\pageref{includef}.\index{Macintosh file names}\index{file
names!Macintosh}

If there are any syntax errors in the database, Metamath will let you know
when it reads in the file.  The one thing that Metamath does not check when
reading in a database is that all proofs are correct, because this would
slow it down too much.  It is a good idea to periodically verify the proofs in
a database you are making changes to.  To do this, use the following command
(and do it for your \texttt{demo0.mm} file now).  Note that the \texttt{*} is a
``wild card'' meaning all proofs in the file.\index{\texttt{verify proof} command}
\begin{verbatim}
MM> verify proof *
\end{verbatim}
Metamath will report any proofs that are incorrect.

It is often useful to save the information that the Metamath program displays
on the screen. You can save everything that happens on the screen by opening a
log file. You may want to do this before you read in a database so that you
can examine any errors later on.  To open a log file, type
\begin{verbatim}
MM> open log abc.log
\end{verbatim}
This will open a file called \texttt{abc.log}, and everything that appears on the
screen from this point on will be stored in this file.  The name of the log file
is arbitrary. To close the log file, type
\begin{verbatim}
MM> close log
\end{verbatim}

Several commands let you examine what's inside your database.
Section~\ref{exploring} has an overview of some useful ones.  The
\texttt{show labels} command lets you see what statement
labels\index{label} exist.  A \texttt{*} matches any combination of
characters, and \texttt{t*} refers to all labels starting with the
letter \texttt{t}.\index{\texttt{show labels} command} The \texttt{/all}
is a ``command qualifier''\index{command qualifier} that tells Metamath
to include labels of hypotheses.  (To see the syntax explained, type
\texttt{help show labels}.)  Type
\begin{verbatim}
MM> show labels t* /all
\end{verbatim}
Metamath will respond with
\begin{verbatim}
The statement number, label, and type are shown.
3 tt $f       4 tr $f       5 ts $f       8 tze $a
9 tpl $a      19 th1 $p
\end{verbatim}

You can use the \texttt{show statement} command to get information about a
particular statement.\index{\texttt{show statement} command}
For example, you can get information about the statement with label \texttt{mp}
by typing
\begin{verbatim}
MM> show statement mp /full
\end{verbatim}
Metamath will respond with
\begin{verbatim}
Statement 17 is located on line 43 of the file
"demo0.mm".
"Define the modus ponens inference rule"
17 mp $a |- Q $.
Its mandatory hypotheses in RPN order are:
  wp $f wff P $.
  wq $f wff Q $.
  min $e |- P $.
  maj $e |- ( P -> Q ) $.
The statement and its hypotheses require the
      variables:  Q P
The variables it contains are:  Q P
\end{verbatim}
The mandatory hypotheses\index{mandatory hypothesis} and their
order\index{RPN order} are
useful to know when you are trying to understand or debug a proof.

Now you are ready to look at what's really inside our proof.  First, here is
how to look at every step in the proof---not just the ones corresponding to an
ordinary formal proof\index{formal proof}, but also the ones that build up the
formulas that appear in each ordinary formal proof step.\index{\texttt{show
proof} command}
\begin{verbatim}
MM> show proof th1 /lemmon /all
\end{verbatim}

This will display the proof on the screen in the following format:
\begin{verbatim}
 1 tt            $f term t
 2 tze           $a term 0
 3 1,2 tpl       $a term ( t + 0 )
 4 tt            $f term t
 5 3,4 weq       $a wff ( t + 0 ) = t
 6 tt            $f term t
 7 tt            $f term t
 8 6,7 weq       $a wff t = t
 9 tt            $f term t
10 9 a2          $a |- ( t + 0 ) = t
11 tt            $f term t
12 tze           $a term 0
13 11,12 tpl     $a term ( t + 0 )
14 tt            $f term t
15 13,14 weq     $a wff ( t + 0 ) = t
16 tt            $f term t
17 tze           $a term 0
18 16,17 tpl     $a term ( t + 0 )
19 tt            $f term t
20 18,19 weq     $a wff ( t + 0 ) = t
21 tt            $f term t
22 tt            $f term t
23 21,22 weq     $a wff t = t
24 20,23 wim     $a wff ( ( t + 0 ) = t -> t = t )
25 tt            $f term t
26 25 a2         $a |- ( t + 0 ) = t
27 tt            $f term t
28 tze           $a term 0
29 27,28 tpl     $a term ( t + 0 )
30 tt            $f term t
31 tt            $f term t
32 29,30,31 a1   $a |- ( ( t + 0 ) = t -> ( ( t + 0 )
                                     = t -> t = t ) )
33 15,24,26,32 mp  $a |- ( ( t + 0 ) = t -> t = t )
34 5,8,10,33 mp  $a |- t = t
\end{verbatim}

The \texttt{/lemmon} command qualifier specifies what is known as a Lemmon-style
display\index{Lemmon-style proof}\index{proof!Lemmon-style}.  Omitting the
\texttt{/lemmon} qualifier results in a tree-style proof (see
p.~\pageref{treeproof} for an example) that is somewhat less explicit but
easier to follow once you get used to it.\index{tree-style
proof}\index{proof!tree-style}

The first number on each line is the step
number of the proof.  Any numbers that follow are step numbers assigned to the
hypotheses of the statement referenced by that step.  Next is the label of
the statement referenced by the step.  The statement type of the statement
referenced comes next, followed by the math symbol\index{math symbol} string
constructed by the proof up to that step.

The last step, 34, contains the statement that is being proved.

Looking at a small piece of the proof, notice that steps 3 and 4 have
established that
\texttt{( t + 0 )} and \texttt{t} are \texttt{term}\,s, and step 5 makes use of steps 3 and
4 to establish that \texttt{( t + 0 ) = t} is a \texttt{wff}.  Let Metamath
itself tell us in detail what is happening in step 5.  Note that the
``target hypothesis'' refers to where step 5 is eventually used, i.e., in step
34.
\begin{verbatim}
MM> show proof th1 /detailed_step 5
Proof step 5:  wp=weq $a wff ( t + 0 ) = t
This step assigns source "weq" ($a) to target "wp"
($f).  The source assertion requires the hypotheses
"tt" ($f, step 3) and "tr" ($f, step 4).  The parent
assertion of the target hypothesis is "mp" ($a,
step 34).
The source assertion before substitution was:
    weq $a wff t = r
The following substitutions were made to the source
assertion:
    Variable  Substituted with
     t         ( t + 0 )
     r         t
The target hypothesis before substitution was:
    wp $f wff P
The following substitution was made to the target
hypothesis:
    Variable  Substituted with
     P         ( t + 0 ) = t
\end{verbatim}

The full proof just shown is useful to understand what is going on in detail.
However, most of the time you will just be interested in
the ``essential'' or logical steps of a proof, i.e.\ those steps
that correspond to an
ordinary formal proof\index{formal proof}.  If you type
\begin{verbatim}
MM> show proof th1 /lemmon /renumber
\end{verbatim}
you will see\label{demoproof}
\begin{verbatim}
1 a2             $a |- ( t + 0 ) = t
2 a2             $a |- ( t + 0 ) = t
3 a1             $a |- ( ( t + 0 ) = t -> ( ( t + 0 )
                                     = t -> t = t ) )
4 2,3 mp         $a |- ( ( t + 0 ) = t -> t = t )
5 1,4 mp         $a |- t = t
\end{verbatim}
Compare this to the formal proof on p.~\pageref{zeroproof} and
notice the resemblance.
By default Metamath
does not show \texttt{\$f}\index{\texttt{\$f}
statement} hypotheses and everything branching off of them in the proof tree
when the proof is displayed; this makes the proof look more like an ordinary
mathematical proof, which does not normally incorporate the explicit
construction of expressions.
This is called the ``essential'' view
(at one time you had to add the
\texttt{/essential} qualifier in the \texttt{show proof}
command to get this view, but this is now the default).
You can could use the \texttt{/all} qualifier in the \texttt{show
proof} command to also show the explicit construction of expressions.
The \texttt{/renumber} qualifier means to renumber
the steps to correspond only to what is displayed.\index{\texttt{show proof}
command}

To exit Metamath, type\index{\texttt{exit} command}
\begin{verbatim}
MM> exit
\end{verbatim}

\subsection{Some Hints for Using the Command Line Interface}

We will conclude this quick introduction to Metamath\index{Metamath} with some
helpful hints on how to navigate your way through the commands.
\index{command line interface (CLI)}

When you type commands into Metamath's CLI, you only have to type as many
characters of a command keyword\index{command keyword} as are needed to make
it unambiguous.  If you type too few characters, Metamath will tell you what
the choices are.  In the case of the \texttt{read} command, only the \texttt{r} is
needed to specify it unambiguously, so you could have typed\index{\texttt{read}
command}
\begin{verbatim}
MM> r demo0.mm
\end{verbatim}
instead of
\begin{verbatim}
MM> read demo0.mm
\end{verbatim}
In our description, we always show the full command words.  When using the
Metamath CLI commands in a command file (to be read with the \texttt{submit}
command)\index{\texttt{submit} command}, it is good practice to use
the unabbreviated command to ensure your instructions will not become ambiguous
if more commands are added to the Metamath program in the future.

The command keywords\index{command
keyword} are not case sensitive; you may type either \texttt{read} or
\texttt{ReAd}.  File names may or may not be case sensitive, depending on your
computer's operating system.  Metamath label\index{label} and math
symbol\index{math symbol} tokens\index{token} are case-sensitive.

The \texttt{help} command\index{\texttt{help} command} will provide you
with a list of topics you can get help on.  You can then type
\texttt{help} {\em topic} to get help on that topic.

If you are uncertain of a command's spelling, just type as many characters
as you remember of the command.  If you have not typed enough characters to
specify it unambiguously, Metamath will tell you what choices you have.

\begin{verbatim}
MM> show s
         ^
?Ambiguous keyword - please specify SETTINGS,
STATEMENT, or SOURCE.
\end{verbatim}

If you don't know what argument to use as part of a command, type a
\texttt{?}\index{\texttt{]}@\texttt{?}\ in command lines}\ at the
argument position.  Metamath will tell you what it expected there.

\begin{verbatim}
MM> show ?
         ^
?Expected SETTINGS, LABELS, STATEMENT, SOURCE, PROOF,
MEMORY, TRACE_BACK, or USAGE.
\end{verbatim}

Finally, you may type just the first word or words of a command followed
by {\em return}.  Metamath will prompt you for the remaining part of the
command, showing you the choices at each step.  For example, instead of
typing \texttt{show statement th1 /full} you could interact in the
following manner:
\begin{verbatim}
MM> show
SETTINGS, LABELS, STATEMENT, SOURCE, PROOF,
MEMORY, TRACE_BACK, or USAGE <SETTINGS>? st
What is the statement label <th1>?
/ or nothing <nothing>? /
TEX, COMMENT_ONLY, or FULL <TEX>? f
/ or nothing <nothing>?
19 th1 $p |- t = t $= ... $.
\end{verbatim}
After each \texttt{?}\ in this mode, you must give Metamath the
information it requests.  Sometimes Metamath gives you a list of choices
with the default choice indicated by brackets \texttt{< > }. Pressing
{\em return} after the \texttt{?}\ will select the default choice.
Answering anything else will override the default.  Note that the
\texttt{/} in command qualifiers is considered a separate
token\index{token} by the parser, and this is why it is asked for
separately.

\section{Your First Proof}\label{frstprf}

Proofs are developed with the aid of the Proof Assistant\index{Proof
Assistant}.  We will now show you how the proof of theorem \texttt{th1}
was built.  So that you can repeat these steps, we will first have the
Proof Assistant erase the proof in Metamath's source buffer\index{source
buffer}, then reconstruct it.  (The source buffer is the place in memory
where Metamath stores the information in the database when it is
\texttt{read}\index{\texttt{read} command} in.  New or modified proofs
are kept in the source buffer until a \texttt{write source}
command\index{\texttt{write source} command} is issued.)  In practice, you
would place a \texttt{?}\index{\texttt{]}@\texttt{?}\ inside proofs}\
between \texttt{\$=}\index{\texttt{\$=} keyword} and
\texttt{\$.}\index{\texttt{\$.}\ keyword}\ in the database to indicate
to Metamath\index{Metamath} that the proof is unknown, and that would be
your starting point.  Whenever the \texttt{verify proof} command encounters
a proof with a \texttt{?}\ in place of a proof step, the statement is
identified as not proved.

When I first started creating Metamath proofs, I would write down
on a piece of paper the complete
formal proof\index{formal proof} as it would appear
in a \texttt{show proof} command\index{\texttt{show proof} command}; see
the display of \texttt{show proof th1 /lemmon /re\-num\-ber} above as an
example.  After you get used to using the Proof Assistant\index{Proof
Assistant} you may get to a point where you can ``see'' the proof in your mind
and let the Proof Assistant guide you in filling in the details, at least for
simpler proofs, but until you gain that experience you may find it very useful
to write down all the details in advance.
Otherwise you may waste a lot of time as you let it take you down a wrong path.
However, others do not find this approach as helpful.
For example, Thomas Brendan Leahy\index{Leahy, Thomas Brendan}
finds that it is more helpful to him to interactively
work backward from a machine-readable statement.
David A. Wheeler\index{Wheeler, David A.}
writes down a general approach, but develops the proof
interactively by switching between
working forwards (from hypotheses and facts likely to be useful) and
backwards (from the goal) until the forwards and backwards approaches meet.
In the end, use whatever approach works for you.

A proof is developed with the Proof Assistant by working backwards, starting
with the theorem\index{theorem} to be proved, and assigning each unknown step
with a theorem or hypothesis until no more unknown steps remain.  The Proof
Assistant will not let you make an assignment unless it can be ``unified''
with the unknown step.  This means that a
substitution\index{substitution!variable}\index{variable substitution} of
variables exists that will make the assignment match the unknown step.  On the
other hand, in the middle of a proof, when working backwards, often more than
one unification\index{unification} (set of substitutions) is possible, since
there is not enough information available at that point to uniquely establish
it.  In this case you can tell Metamath which unification to choose, or you
can continue to assign unknown steps until enough information is available to
make the unification unique.

We will assume you have entered Metamath and read in the database as described
above.  The following dialog shows how the proof was developed.  For more
details on what some of the commands do, refer to Section~\ref{pfcommands}.
\index{\texttt{prove} command}

\begin{verbatim}
MM> prove th1
Entering the Proof Assistant.  Type HELP for help, EXIT
to exit.  You will be working on the proof of statement th1:
  $p |- t = t
Note:  The proof you are starting with is already complete.
MM-PA>
\end{verbatim}

The \verb/MM-PA>/ prompt means we are inside the Proof
Assistant.\index{Proof Assistant} Most of the regular Metamath commands
(\texttt{show statement}, etc.) are still available if you need them.

\begin{verbatim}
MM-PA> delete all
The entire proof was deleted.
\end{verbatim}

We have deleted the whole proof so we can start from scratch.

\begin{verbatim}
MM-PA> show new_proof/lemmon/all
1 ?              $? |- t = t
\end{verbatim}

The \texttt{show new{\char`\_}proof} command\index{\texttt{show
new{\char`\_}proof} command} is like \texttt{show proof} except that we
don't specify a statement; instead, the proof we're working on is
displayed.

\begin{verbatim}
MM-PA> assign 1 mp
To undo the assignment, DELETE STEP 5 and INITIALIZE, UNIFY
if needed.
3   min=?  $? |- $2
4   maj=?  $? |- ( $2 -> t = t )
\end{verbatim}

The \texttt{assign} command\index{\texttt{assign} command} above means
``assign step 1 with the statement whose label is \texttt{mp}.''  Note
that step renumbering will constantly occur as you assign steps in the
middle of a proof; in general all steps from the step you assign until
the end of the proof will get moved up.  In this case, what used to be
step 1 is now step 5, because the (partial) proof now has five steps:
the four hypotheses of the \texttt{mp} statement and the \texttt{mp}
statement itself.  Let's look at all the steps in our partial proof:

\begin{verbatim}
MM-PA> show new_proof/lemmon/all
1 ?              $? wff $2
2 ?              $? wff t = t
3 ?              $? |- $2
4 ?              $? |- ( $2 -> t = t )
5 1,2,3,4 mp     $a |- t = t
\end{verbatim}

The symbol \texttt{\$2} is a temporary variable\index{temporary
variable} that represents a symbol sequence not yet known.  In the final
proof, all temporary variables will be eliminated.  The general format
for a temporary variable is \texttt{\$} followed by an integer.  Note
that \texttt{\$} is not a legal character in a math symbol (see
Section~\ref{dollardollar}, p.~\pageref{dollardollar}), so there will
never be a naming conflict between real symbols and temporary variables.

Unknown steps 1 and 2 are constructions of the two wffs used by the
modus ponens rule.  As you will see at the end of this section, the
Proof Assistant\index{Proof Assistant} can usually figure these steps
out by itself, and we will not have to worry about them.  Therefore from
here on we will display only the ``essential'' hypotheses, i.e.\ those
steps that correspond to traditional formal proofs\index{formal proof}.

\begin{verbatim}
MM-PA> show new_proof/lemmon
3 ?              $? |- $2
4 ?              $? |- ( $2 -> t = t )
5 3,4 mp         $a |- t = t
\end{verbatim}

Unknown steps 3 and 4 are the ones we must focus on.  They correspond to the
minor and major premises of the modus ponens rule.  We will assign them as
follows.  Notice that because of the step renumbering that takes place
after an assignment, it is advantageous to assign unknown steps in reverse
order, because earlier steps will not get renumbered.

\begin{verbatim}
MM-PA> assign 4 mp
To undo the assignment, DELETE STEP 8 and INITIALIZE, UNIFY
if needed.
3   min=?  $? |- $2
6     min=?  $? |- $4
7     maj=?  $? |- ( $4 -> ( $2 -> t = t ) )
\end{verbatim}

We are now going to describe an obscure feature that you will probably
never use but should be aware of.  The Metamath language allows empty
symbol sequences to be substituted for variables, but in most formal
systems this feature is never used.  One of the few examples where is it
used is the MIU-system\index{MIU-system} described in
Appendix~\ref{MIU}.  But such systems are rare, and by default this
feature is turned off in the Proof Assistant.  (It is always allowed for
{\tt verify proof}.)  Let us turn it on and see what
happens.\index{\texttt{set empty{\char`\_}substitution} command}

\begin{verbatim}
MM-PA> set empty_substitution on
Substitutions with empty symbol sequences is now allowed.
\end{verbatim}

With this feature enabled, more unifications will be
ambiguous\index{ambiguous unification}\index{unification!ambiguous} in
the middle of a proof, because
substitution\index{substitution!variable}\index{variable substitution}
of variables with empty symbol sequences will become an additional
possibility.  Let's see what happens when we make our next assignment.

\begin{verbatim}
MM-PA> assign 3 a2
There are 2 possible unifications.  Please select the correct
    one or Q if you want to UNIFY later.
Unify:  |- $6
 with:  |- ( $9 + 0 ) = $9
Unification #1 of 2 (weight = 7):
  Replace "$6" with "( + 0 ) ="
  Replace "$9" with ""
  Accept (A), reject (R), or quit (Q) <A>? r
\end{verbatim}

The first choice presented is the wrong one.  If we had selected it,
temporary variable \texttt{\$6} would have been assigned a truncated
wff, and temporary variable \texttt{\$9} would have been assigned an
empty sequence (which is not allowed in our system).  With this choice,
eventually we would reach a point where we would get stuck because
we would end up with steps impossible to prove.  (You may want to
try it.)  We typed \texttt{r} to reject the choice.

\begin{verbatim}
Unification #2 of 2 (weight = 21):
  Replace "$6" with "( $9 + 0 ) = $9"
  Accept (A), reject (R), or quit (Q) <A>? q
To undo the assignment, DELETE STEP 4 and INITIALIZE, UNIFY
if needed.
 7     min=?  $? |- $8
 8     maj=?  $? |- ( $8 -> ( $6 -> t = t ) )
\end{verbatim}

The second choice is correct, and normally we would type \texttt{a}
to accept it.  But instead we typed \texttt{q} to show what will happen:
it will leave the step with an unknown unification, which can be
seen as follows:

\begin{verbatim}
MM-PA> show new_proof/not_unified
 4   min    $a |- $6
        =a2  = |- ( $9 + 0 ) = $9
\end{verbatim}

Later we can unify this with the \texttt{unify}
\texttt{all/interactive} command.

The important point to remember is that occasionally you will be
presented with several unification choices while entering a proof, when
the program determines that there is not enough information yet to make
an unambiguous choice automatically (and this can happen even with
\texttt{set empty{\char`\_}substitution} turned off).  Usually it is
obvious by inspection which choice is correct, since incorrect ones will
tend to be meaningless fragments of wffs.  In addition, the correct
choice will usually be the first one presented, unlike our example
above.

Enough of this digression.  Let us go back to the default setting.

\begin{verbatim}
MM-PA> set empty_substitution off
The ability to substitute empty expressions for variables
has been turned off.  Note that this may make the Proof
Assistant too restrictive in some cases.
\end{verbatim}

If we delete the proof, start over, and get to the point where
we digressed above, there will no longer be an ambiguous unification.

\begin{verbatim}
MM-PA> assign 3 a2
To undo the assignment, DELETE STEP 4 and INITIALIZE, UNIFY
if needed.
 7     min=?  $? |- $4
 8     maj=?  $? |- ( $4 -> ( ( $5 + 0 ) = $5 -> t = t ) )
\end{verbatim}

Let us look at our proof so far, and continue.

\begin{verbatim}
MM-PA> show new_proof/lemmon
 4 a2            $a |- ( $5 + 0 ) = $5
 7 ?             $? |- $4
 8 ?             $? |- ( $4 -> ( ( $5 + 0 ) = $5 -> t = t ) )
 9 7,8 mp        $a |- ( ( $5 + 0 ) = $5 -> t = t )
10 4,9 mp        $a |- t = t
MM-PA> assign 8 a1
To undo the assignment, DELETE STEP 11 and INITIALIZE, UNIFY
if needed.
 7     min=?  $? |- ( t + 0 ) = t
MM-PA> assign 7 a2
To undo the assignment, DELETE STEP 8 and INITIALIZE, UNIFY
if needed.
MM-PA> show new_proof/lemmon
 4 a2            $a |- ( t + 0 ) = t
 8 a2            $a |- ( t + 0 ) = t
12 a1            $a |- ( ( t + 0 ) = t -> ( ( t + 0 ) = t ->
                                                    t = t ) )
13 8,12 mp       $a |- ( ( t + 0 ) = t -> t = t )
14 4,13 mp       $a |- t = t
\end{verbatim}

Now all temporary variables and unknown steps have been eliminated from the
``essential'' part of the proof.  When this is achieved, the Proof
Assistant\index{Proof Assistant} can usually figure out the rest of the proof
automatically.  (Note that the \texttt{improve} command can occasionally be
useful for filling in essential steps as well, but it only tries to make use
of statements that introduce no new variables in their hypotheses, which is
not the case for \texttt{mp}. Also it will not try to improve steps containing
temporary variables.)  Let's look at the complete proof, then run
the \texttt{improve} command, then look at it again.

\begin{verbatim}
MM-PA> show new_proof/lemmon/all
 1 ?             $? wff ( t + 0 ) = t
 2 ?             $? wff t = t
 3 ?             $? term t
 4 3 a2          $a |- ( t + 0 ) = t
 5 ?             $? wff ( t + 0 ) = t
 6 ?             $? wff ( ( t + 0 ) = t -> t = t )
 7 ?             $? term t
 8 7 a2          $a |- ( t + 0 ) = t
 9 ?             $? term ( t + 0 )
10 ?             $? term t
11 ?             $? term t
12 9,10,11 a1    $a |- ( ( t + 0 ) = t -> ( ( t + 0 ) = t ->
                                                    t = t ) )
13 5,6,8,12 mp   $a |- ( ( t + 0 ) = t -> t = t )
14 1,2,4,13 mp   $a |- t = t
\end{verbatim}

\begin{verbatim}
MM-PA> improve all
A proof of length 1 was found for step 11.
A proof of length 1 was found for step 10.
A proof of length 3 was found for step 9.
A proof of length 1 was found for step 7.
A proof of length 9 was found for step 6.
A proof of length 5 was found for step 5.
A proof of length 1 was found for step 3.
A proof of length 3 was found for step 2.
A proof of length 5 was found for step 1.
Steps 1 and above have been renumbered.
CONGRATULATIONS!  The proof is complete.  Use SAVE
NEW_PROOF to save it.  Note:  The Proof Assistant does
not detect $d violations.  After saving the proof, you
should verify it with VERIFY PROOF.
\end{verbatim}

The \texttt{save new{\char`\_}proof} command\index{\texttt{save
new{\char`\_}proof} command} will save the proof in the database.  Here
we will just display it in a form that can be clipped out of a log file
and inserted manually into the database source file with a text
editor.\index{normal proof}\index{proof!normal}

\begin{verbatim}
MM-PA> show new_proof/normal
---------Clip out the proof below this line:
      tt tze tpl tt weq tt tt weq tt a2 tt tze tpl tt weq
      tt tze tpl tt weq tt tt weq wim tt a2 tt tze tpl tt
      tt a1 mp mp $.
---------The proof of 'th1' to clip out ends above this line.
\end{verbatim}

There is another proof format called ``compressed''\index{compressed
proof}\index{proof!compressed} that you will see in databases.  It is
not important to understand how it is encoded but only to recognize it
when you see it.  Its only purpose is to reduce storage requirements for
large proofs.  A compressed proof can always be converted to a normal
one and vice-versa, and the Metamath \texttt{show proof}
commands\index{\texttt{show proof} command} work equally well with
compressed proofs.  The compressed proof format is described in
Appendix~\ref{compressed}.

\begin{verbatim}
MM-PA> show new_proof/compressed
---------Clip out the proof below this line:
      ( tze tpl weq a2 wim a1 mp ) ABCZADZAADZAEZJJKFLIA
      AGHH $.
---------The proof of 'th1' to clip out ends above this line.
\end{verbatim}

Now we will exit the Proof Assistant.  Since we made changes to the proof,
it will warn us that we have not saved it.  In this case, we don't care.

\begin{verbatim}
MM-PA> exit
Warning:  You have not saved changes to the proof.
Do you want to EXIT anyway (Y, N) <N>? y
Exiting the Proof Assistant.
Type EXIT again to exit Metamath.
\end{verbatim}

The Proof Assistant\index{Proof Assistant} has several other commands
that can help you while creating proofs.  See Section~\ref{pfcommands}
for a list of them.

A command that is often useful is \texttt{minimize{\char`\_}with
*/brief}, which tries to shorten the proof.  It can make the process
more efficient by letting you write a somewhat ``sloppy'' proof then
clean up some of the fine details of optimization for you (although it
can't perform miracles such as restructuring the overall proof).

\section{A Note About Editing a Data\-base File}

Once your source file contains proofs, there are some restrictions on
how you can edit it so that the proofs remain valid.  Pay particular
attention to these rules, since otherwise you can lose a lot of work.
It is a good idea to periodically verify all proofs with \texttt{verify
proof *} to ensure their integrity.

If your file contains only normal (as opposed to compressed) proofs, the
main rule is that you may not change the order of the mandatory
hypotheses\index{mandatory hypothesis} of any statement referenced in a
later proof.  For example, if you swap the order of the major and minor
premise in the modus ponens rule, all proofs making use of that rule
will become incorrect.  The \texttt{show statement}
command\index{\texttt{show statement} command} will show you the
mandatory hypotheses of a statement and their order.

If a statement has a compressed proof, you also must not change the
order of {\em its} mandatory hypotheses.  The compressed proof format
makes use of this information as part of the compression technique.
Note that swapping the names of two variables in a theorem will change
the order of its mandatory hypotheses.

The safest way to edit a statement, say \texttt{mytheorem}, is to
duplicate it then rename the original to \texttt{mytheoremOLD}
throughout the database.  Once the edited version is re-proved, all
statements referencing \texttt{mytheoremOLD} can be updated in the Proof
Assistant using \texttt{minimize{\char`\_}with
mytheorem
/allow{\char`\_}growth}.\index{\texttt{minimize{\char`\_}with} command}
% 3/10/07 Note: line-breaking the above results in duplicate index entries

\chapter{Abstract Mathematics Revealed}\label{fol}

\section{Logic and Set Theory}\label{logicandsettheory}

\begin{quote}
  {\em Set theory can be viewed as a form of exact theology.}
  \flushright\sc  Rudy Rucker\footnote{\cite{Barrow}, p.~31.}\\
\end{quote}\index{Rucker, Rudy}

Despite its seeming complexity, all of standard mathematics, no matter how
deep or abstract, can amazingly enough be derived from a relatively small set
of axioms\index{axiom} or first principles. The development of these axioms is
among the most impressive and important accomplishments of mathematics in the
20th century. Ultimately, these axioms can be broken down into a set of rules
for manipulating symbols that any technically oriented person can follow.

We will not spend much time trying to convey a deep, higher-level
understanding of the meaning of the axioms. This kind of understanding
requires some mathematical sophistication as well as an understanding of the
philosophy underlying the foundations of mathematics and typically develops
over time as you work with mathematics.  Our goal, instead, is to give you the
immediate ability to follow how theorems\index{theorem} are derived from the
axioms and from other theorems.  This will be similar to learning the syntax
of a computer language, which lets you follow the details in a program but
does not necessarily give you the ability to write non-trivial programs on
your own, an ability that comes with practice. For now don't be alarmed by
abstract-sounding names of the axioms; just focus on the rules for
manipulating the symbols, which follow the simple conventions of the
Metamath\index{Metamath} language.

The axioms that underlie all of standard mathematics consist of axioms of logic
and axioms of set theory. The axioms of logic are divided into two
subcategories, propositional calculus\index{propositional calculus} (sometimes
called sentential logic\index{sentential logic}) and predicate calculus
(sometimes called first-order logic\index{first-order logic}\index{quantifier
theory}\index{predicate calculus} or quantifier theory).  Propositional
calculus is a prerequisite for predicate calculus, and predicate calculus is a
prerequisite for set theory.  The version of set theory most commonly used is
Zermelo--Fraenkel set theory\index{Zermelo--Fraenkel set theory}\index{set theory}
with the axiom of choice,
often abbreviated as ZFC\index{ZFC}.

Here in a nutshell is what the axioms are all about in an informal way. The
connection between this description and symbols we will show you won't be
immediately apparent and in principle needn't ever be.  Our description just
tries to summarize what mathematicians think about when they work with the
axioms.

Logic is a set of rules that allow us determine truths given other truths.
Put another way,
logic is more or less the translation of what we would consider common sense
into a rigorous set of axioms.\index{axioms of logic}  Suppose $\varphi$,
$\psi$, and $\chi$ (the Greek letters phi, psi, and chi) represent statements
that are either true or false, and $x$ is a variable\index{variable!in predicate
calculus} ranging over some group of mathematical objects (sets, integers,
real numbers, etc.). In mathematics, a ``statement'' really means a formula,
and $\psi$ could be for example ``$x = 2$.''
Propositional calculus\index{propositional calculus}
allows us to use variables that are either true or false
and make deductions such as
``if $\varphi$ implies $\psi$ and $\psi$ implies $\chi$, then $\varphi$
implies $\chi$.''
Predicate calculus\index{predicate calculus}
extends propositional calculus by also allowing us
to discuss statements about objects (not just true and false values), including
statements about ``all'' or ``at least one'' object.
For example, predicate calculus allows to say,
``if $\varphi$ is true for all $x$, then $\varphi$ is true for some $x$.''
The logic used in \texttt{set.mm} is standard classical logic
(as opposed to other logic systems like intuitionistic logic).

Set theory\index{set theory} has to do with the manipulation of objects and
collections of objects, specifically the abstract, imaginary objects that
mathematics deals with, such as numbers. Everything that is claimed to exist
in mathematics is considered to be a set.  A set called the empty
set\index{empty set} contains nothing.  We represent the empty set by
$\varnothing$.  Many sets can be built up from the empty set.  There is a set
represented by $\{\varnothing\}$ that contains the empty set, another set
represented by $\{\varnothing,\{\varnothing\}\}$ that contains this set as
well as the empty set, another set represented by $\{\{\varnothing\}\}$ that
contains just the set that contains the empty set, and so on ad infinitum. All
mathematical objects, no matter how complex, are defined as being identical to
certain sets: the integer\index{integer} 0 is defined as the empty set, the
integer 1 is defined as $\{\varnothing\}$, the integer 2 is defined as
$\{\varnothing,\{\varnothing\}\}$.  (How these definitions were chosen doesn't
matter now, but the idea behind it is that these sets have the properties we
expect of integers once suitable operations are defined.)  Mathematical
operations, such as addition, are defined in terms of operations on
sets---their union\index{set union}, intersection\index{set intersection}, and
so on---operations you may have used in elementary school when you worked
with groups of apples and oranges.

With a leap of faith, the axioms also postulate the existence of infinite
sets\index{infinite set}, such as the set of all non-negative integers ($0, 1,
2,\ldots$, also called ``natural numbers''\index{natural number}).  This set
can't be represented with the brace notation\index{brace notation} we just
showed you, but requires a more complicated notation called ``class
abstraction.''\index{class abstraction}\index{abstraction class}  For
example, the infinite set $\{ x |
\mbox{``$x$ is a natural number''} \} $ means the ``set of all objects $x$
such that $x$ is a natural number'' i.e.\ the set of natural numbers; here,
``$x$ is a natural number'' is a rather complicated formula when broken down
into the primitive symbols.\label{expandom}\footnote{The statement ``$x$ is a
natural number'' is formally expressed as ``$x \in \omega$,'' where $\in$
(stylized epsilon) means ``is in'' or ``is an element of'' and $\omega$
(omega) means ``the set of natural numbers.''  When ``$x\in\omega$'' is
completely expanded in terms of the primitive symbols of set theory, the
result is  $\lnot$ $($ $\lnot$ $($ $\forall$ $z$ $($ $\lnot$ $\forall$ $w$ $($
$z$ $\in$ $w$ $\rightarrow$ $\lnot$ $w$ $\in$ $x$ $)$ $\rightarrow$ $z$ $\in$
$x$ $)$ $\rightarrow$ $($ $\forall$ $z$ $($ $\lnot$ $($ $\forall$ $w$ $($ $w$
$\in$ $z$ $\rightarrow$ $w$ $\in$ $x$ $)$ $\rightarrow$ $\forall$ $w$ $\lnot$
$w$ $\in$ $z$ $)$ $\rightarrow$ $\lnot$ $\forall$ $w$ $($ $w$ $\in$ $z$
$\rightarrow$ $\lnot$ $\forall$ $v$ $($ $v$ $\in$ $z$ $\rightarrow$ $\lnot$
$v$ $\in$ $w$ $)$ $)$ $)$ $\rightarrow$ $\lnot$ $\forall$ $z$ $\forall$ $w$
$($ $\lnot$ $($ $z$ $\in$ $x$ $\rightarrow$ $\lnot$ $w$ $\in$ $x$ $)$
$\rightarrow$ $($ $\lnot$ $z$ $\in$ $w$ $\rightarrow$ $($ $\lnot$ $z$ $=$ $w$
$\rightarrow$ $w$ $\in$ $z$ $)$ $)$ $)$ $)$ $)$ $\rightarrow$ $\lnot$
$\forall$ $y$ $($ $\lnot$ $($ $\lnot$ $($ $\forall$ $z$ $($ $\lnot$ $\forall$
$w$ $($ $z$ $\in$ $w$ $\rightarrow$ $\lnot$ $w$ $\in$ $y$ $)$ $\rightarrow$
$z$ $\in$ $y$ $)$ $\rightarrow$ $($ $\forall$ $z$ $($ $\lnot$ $($ $\forall$
$w$ $($ $w$ $\in$ $z$ $\rightarrow$ $w$ $\in$ $y$ $)$ $\rightarrow$ $\forall$
$w$ $\lnot$ $w$ $\in$ $z$ $)$ $\rightarrow$ $\lnot$ $\forall$ $w$ $($ $w$
$\in$ $z$ $\rightarrow$ $\lnot$ $\forall$ $v$ $($ $v$ $\in$ $z$ $\rightarrow$
$\lnot$ $v$ $\in$ $w$ $)$ $)$ $)$ $\rightarrow$ $\lnot$ $\forall$ $z$
$\forall$ $w$ $($ $\lnot$ $($ $z$ $\in$ $y$ $\rightarrow$ $\lnot$ $w$ $\in$
$y$ $)$ $\rightarrow$ $($ $\lnot$ $z$ $\in$ $w$ $\rightarrow$ $($ $\lnot$ $z$
$=$ $w$ $\rightarrow$ $w$ $\in$ $z$ $)$ $)$ $)$ $)$ $\rightarrow$ $($
$\forall$ $z$ $\lnot$ $z$ $\in$ $y$ $\rightarrow$ $\lnot$ $\forall$ $w$ $($
$\lnot$ $($ $w$ $\in$ $y$ $\rightarrow$ $\lnot$ $\forall$ $z$ $($ $w$ $\in$
$z$ $\rightarrow$ $\lnot$ $z$ $\in$ $y$ $)$ $)$ $\rightarrow$ $\lnot$ $($
$\lnot$ $\forall$ $z$ $($ $w$ $\in$ $z$ $\rightarrow$ $\lnot$ $z$ $\in$ $y$
$)$ $\rightarrow$ $w$ $\in$ $y$ $)$ $)$ $)$ $)$ $\rightarrow$ $x$ $\in$ $y$
$)$ $)$ $)$. Section~\ref{hierarchy} shows the hierarchy of definitions that
leads up to this expression.}\index{stylized epsilon ($\in$)}\index{omega
($\omega$)}  Actually, the primitive symbols don't even include the brace
notation.  The brace notation is a high-level definition, which you can find in
Section~\ref{hierarchy}.

Interestingly, the arithmetic of integers\index{integer} and
rationals\index{rational number} can be developed without appealing to the
existence of an infinite set, whereas the arithmetic of real
numbers\index{real number} requires it.

Each variable\index{variable!in set theory} in the axioms of set theory
represents an arbitrary set, and the axioms specify the legal kinds of things
you can do with these variables at a very primitive level.

Now, you may think that numbers and arithmetic are a lot more intuitive and
fundamental than sets and therefore should be the foundation of mathematics.
What is really the case is that you've dealt with numbers all your life and
are comfortable with a few rules for manipulating them such as addition and
multiplication.  Those rules only cover a small portion of what can be done
with numbers and only a very tiny fraction of the rest of mathematics.  If you
look at any elementary book on number theory, you will quickly become lost if
these are the only rules that you know.  Even though such books may present a
list of ``axioms''\index{axiom} for arithmetic, the ability to use the axioms
and to understand proofs of theorems\index{theorem} (facts) about numbers
requires an implicit mathematical talent that frustrates many people
from studying abstract mathematics.  The kind of mathematics that most people
know limits them to the practical, everyday usage of blindly manipulating
numbers and formulas, without any understanding of why those rules are correct
nor any ability to go any further.  For example, do you know why multiplying
two negative numbers yields a positive number?  Starting with set theory, you
will also start off blindly manipulating symbols according to the rules we give
you, but with the advantage that these rules will allow you, in principle, to
access {\em all} of mathematics, not just a tiny part of it.

Of course, concrete examples are often helpful in the learning process. For
example, you can verify that $2\cdot 3=3 \cdot 2$ by actually grouping
objects and can easily ``see'' how it generalizes to $x\cdot y = y\cdot x$,
even though you might not be able to rigorously prove it.  Similarly, in set
theory it can be helpful to understand how the axioms of set theory apply to
(and are correct for) small finite collections of objects.  You should be aware
that in set theory intuition can be misleading for infinite collections, and
rigorous proofs become more important.  For example, while $x\cdot y = y\cdot
x$ is correct for finite ordinals (which are the natural numbers), it is not
usually true for infinite ordinals.

\section{The Axioms for All of Mathematics}

In this section\index{axioms for mathematics}, we will show you the axioms
for all of standard mathematics (i.e.\ logic and set theory) as they are
traditionally presented.  The traditional presentation is useful for someone
with the mathematical experience needed to correctly manipulate high-level
abstract concepts.  For someone without this talent, knowing how to actually
make use of these axioms can be difficult.  The purpose of this section is to
allow you to see how the version of the axioms used in the standard
Metamath\index{Metamath} database \texttt{set.mm}\index{set
theory database (\texttt{set.mm})} relates to  the typical version
in textbooks, and also to give you an informal feel for them.

\subsection{Propositional Calculus}

Propositional calculus\index{propositional calculus} concerns itself with
statements that can be interpreted as either true or false.  Some examples of
statements (outside of mathematics) that are either true or false are ``It is
raining today'' and ``The United States has a female president.'' In
mathematics, as we mentioned, statements are really formulas.

In propositional calculus, we don't care what the statements are.  We also
treat a logical combination of statements, such as ``It is raining today and
the United States has a female president,'' no differently from a single
statement.  Statements and their combinations are called well-formed formulas
(wffs)\index{well-formed formula (wff)}.  We define wffs only in terms of
other wffs and don't define what a ``starting'' wff is.  As is common practice
in the literature, we use Greek letters to represent wffs.

Specifically, suppose $\varphi$ and $\psi$ are wffs.  Then the combinations
$\varphi\rightarrow\psi$ (``$\varphi$ implies $\psi$,'' also read ``if
$\varphi$ then $\psi$'')\index{implication ($\rightarrow$)} and $\lnot\varphi$
(``not $\varphi$'')\index{negation ($\lnot$)} are also wffs.

The three axioms of propositional calculus\index{axioms of propositional
calculus} are all wffs of the following form:\footnote{A remarkable result of
C.~A.~Meredith\index{Meredith, C. A.} squeezes these three axioms into the
single axiom $((((\varphi\rightarrow \psi)\rightarrow(\neg \chi\rightarrow\neg
\theta))\rightarrow \chi )\rightarrow \tau)\rightarrow((\tau\rightarrow
\varphi)\rightarrow(\theta\rightarrow \varphi))$ \cite{CAMeredith},
which is believed to be the shortest possible.}
\begin{center}
     $\varphi\rightarrow(\psi\rightarrow \varphi)$\\

     $(\varphi\rightarrow (\psi\rightarrow \chi))\rightarrow
((\varphi\rightarrow  \psi)\rightarrow (\varphi\rightarrow \chi))$\\

     $(\neg \varphi\rightarrow \neg\psi)\rightarrow (\psi\rightarrow
\varphi)$
\end{center}

These three axioms are widely used.
They are attributed to Jan {\L}ukasiewicz
(pronounced woo-kah-SHAY-vitch) and was popularized by Alonzo Church,
who called it system P2. (Thanks to Ted Ulrich for this information.)

There are an infinite number of axioms, one for each possible
wff\index{well-formed formula (wff)} of the above form.  (For this reason,
axioms such as the above are often called ``axiom schemes.''\index{axiom
scheme})  Each Greek letter in the axioms may be substituted with a more
complex wff to result in another axiom.  For example, substituting
$\neg(\varphi\rightarrow\chi)$ for $\varphi$ in the first axiom yields
$\neg(\varphi\rightarrow\chi)\rightarrow(\psi\rightarrow
\neg(\varphi\rightarrow\chi))$, which is still an axiom.

To deduce new true statements (theorems\index{theorem}) from the axioms, a
rule\index{rule} called ``modus ponens''\index{modus ponens} is used.  This
rule states that if the wff $\varphi$ is an axiom or a theorem, and the wff
$\varphi\rightarrow\psi$ is an axiom or a theorem, then the wff $\psi$ is also
a theorem\index{theorem}.

As a non-mathematical example of modus ponens, suppose we have proved (or
taken as an axiom) ``Bob is a man'' and separately have proved (or taken as
an axiom) ``If Bob is a man, then Bob is a human.''  Using the rule of modus
ponens, we can logically deduce, ``Bob is a human.''

From Metamath's\index{Metamath} point of view, the axioms and the rule of
modus ponens just define a mechanical means for deducing new true statements
from existing true statements, and that is the complete content of
propositional calculus as far as Metamath is concerned.  You can read a logic
textbook to gain a better understanding of their meaning, or you can just let
their meaning slowly become apparent to you after you use them for a while.

It is actually rather easy to check to see if a formula is a theorem of
propositional calculus.  Theorems of propositional calculus are also called
``tautologies.''\index{tautology}  The technique to check whether a formula is
a tautology is called the ``truth table method,''\index{truth table} and it
works like this.  A wff $\varphi\rightarrow\psi$ is false whenever $\varphi$ is true
and $\psi$ is false.  Otherwise it is true.  A wff $\lnot\varphi$ is false
whenever $\varphi$ is true and false otherwise. To verify a tautology such as
$\varphi\rightarrow(\psi\rightarrow \varphi)$, you break it down into sub-wffs and
construct a truth table that accounts for all possible combinations of true
and false assigned to the wff metavariables:
\begin{center}\begin{tabular}{|c|c|c|c|}\hline
\mbox{$\varphi$} & \mbox{$\psi$} & \mbox{$\psi\rightarrow\varphi$}
    & \mbox{$\varphi\rightarrow(\psi\rightarrow \varphi)$} \\ \hline \hline
              T   &  T    &      T       &        T    \\ \hline
              T   &  F    &      T       &        T    \\ \hline
              F   &  T    &      F       &        T    \\ \hline
              F   &  F    &      T       &        T    \\ \hline
\end{tabular}\end{center}
If all entries in the last column are true, the formula is a tautology.

Now, the truth table method doesn't tell you how to prove the tautology from
the axioms, but only that a proof exists.  Finding an actual proof (especially
one that is short and elegant) can be challenging.  Methods do exist for
automatically generating proofs in propositional calculus, but the proofs that
result can sometimes be very long.  In the Metamath \texttt{set.mm}\index{set
theory database (\texttt{set.mm})} database, most
or all proofs were created manually.

Section \ref{metadefprop} discusses various definitions
that make propositional calculus easier to use.
For example, we define:

\begin{itemize}
\item $\varphi \vee \psi$
  is true if either $\varphi$ or $\psi$ (or both) are true
  (this is disjunction\index{disjunction ($\vee$)}
  aka logical {\sc or}\index{logical {\sc or} ($\vee$)}).

\item $\varphi \wedge \psi$
  is true if both $\varphi$ and $\psi$ are true
  (this is conjunction\index{conjunction ($\wedge$)}
  aka logical {\sc and}\index{logical {\sc and} ($\wedge$)}).

\item $\varphi \leftrightarrow \psi$
  is true if $\varphi$ and $\psi$ have the same value, that is,
  they are both true or both false
  (this is the biconditional\index{biconditional ($\leftrightarrow$)}).
\end{itemize}

\subsection{Predicate Calculus}

Predicate calculus\index{predicate calculus} introduces the concept of
``individual variables,''\index{variable!in predicate calculus}\index{individual
variable} which
we will usually just call ``variables.''
These variables can represent something other than true or false (wffs),
and will always represent sets when we get to set theory.  There are also
three new symbols $\forall$\index{universal quantifier ($\forall$)},
$=$\index{equality ($=$)}, and $\in$\index{stylized epsilon ($\in$)},
read ``for all,'' ``equals,'' and ``is an element of''
respectively.  We will represent variables with the letters $x$, $y$, $z$, and
$w$, as is common practice in the literature.
For example, $\forall x \varphi$ means ``for all possible values of
$x$, $\varphi$ is true.''

In predicate calculus, we extend the definition of a wff\index{well-formed
formula (wff)}.  If $\varphi$ is a wff and $x$ and $y$ are variables, then
$\forall x \, \varphi$, $x=y$, and $x\in y$ are wffs. Note that these three new
types of wffs can be considered ``starting'' wffs from which we can build
other wffs with $\rightarrow$ and $\neg$ .  The concept of a starting wff was
absent in propositional calculus.  But starting wff or not, all we are really
concerned with is whether our wffs are correctly constructed according to
these mechanical rules.

A quick aside:
To prevent confusion, it might be best at this point to think of the variables
of Metamath\index{Metamath} as ``metavariables,''\index{metavariable} because
they are not quite the same as the variables we are introducing here.  A
(meta)variable in Metamath can be a wff or an individual variable, as well
as many other things; in general, it represents a kind of place holder for an
unspecified sequence of math symbols\index{math symbol}.

Unlike propositional calculus, no decision procedure\index{decision procedure}
analogous to the truth table method exists (nor theoretically can exist) that
will definitely determine whether a formula is a theorem of predicate
calculus.  Much of the work in the field of automated theorem
proving\index{automated theorem proving} has been dedicated to coming up with
clever heuristics for proving theorems of predicate calculus, but they can
never be guaranteed to work always.

Section \ref{metadefpred} discusses various definitions
that make predicate calculus easier to use.
For example, we define
$\exists x \varphi$ to mean
``there exists at least one possible value of $x$ where $\varphi$ is true.''

We now turn to looking at how predicate calculus can be formally
represented.

\subsubsection{Common Axioms}

There is a new rule of inference in predicate calculus:  if $\varphi$ is
an axiom or a theorem, then $\forall x \,\varphi$ is also a
theorem\index{theorem}.  This is called the rule of
``generalization.''\index{rule of generalization}
This is easily represented in Metamath.

In standard texts of logic, there are often two axioms of predicate
calculus\index{axioms of predicate calculus}:
\begin{center}
  $\forall x \,\varphi ( x ) \rightarrow \varphi ( y )$,
      where ``$y$ is properly substituted for $x$.''\\
  $\forall x ( \varphi \rightarrow \psi )\rightarrow ( \varphi \rightarrow
    \forall x\, \psi )$,
    where ``$x$ is not free in $\varphi$.''
\end{center}

Now at first glance, this seems simple:  just two axioms.  However,
conditional clauses are attached to each axiom describing requirements that
may seem puzzling to you.  In addition, the first axiom puts a variable symbol
in parentheses after each wff, seemingly violating our definition of a
wff\index{well-formed formula (wff)}; this is just an informal way of
referring to some arbitrary variable that may occur in the wff.  The
conditional clauses do, of course, have a precise meaning, but as it turns out
the precise meaning is somewhat complicated and awkward to formalize in a
way that a computer can handle easily.  Unlike propositional calculus, a
certain amount of mathematical sophistication and practice is needed to be
able to easily grasp and manipulate these concepts correctly.

Predicate calculus may be presented with or without axioms for
equality\index{axioms of equality}\index{equality ($=$)}. We will require the
axioms of equality as a prerequisite for the version of set theory we will
use.  The axioms for equality, when included, are often represented using these
two axioms:
\begin{center}
$x=x$\\ \ \\
$x=y\rightarrow (\varphi(x,x)\rightarrow\varphi(x,y))$ where ``$\varphi(x,y)$
   arises from $\varphi(x,x)$ by replacing some, but not necessarily all,
   free\index{free variable}
   occurrences of $x$ by $y$,\\ provided that $y$ is free for $x$
   in $\varphi(x,x)$.'' \end{center}
% (Mendelson p. 95)
The first equality axiom is simple, but again,
the condition on the second one is
somewhat awkward to implement on a computer.

\subsubsection{Tarski System S2}

Of course, we are not the first to notice the complications of these
predicate calculus axioms when being rigorous.

Well-known logician Alfred Tarski published in 1965
a system he called system S2\cite[p.~77]{Tarski1965}.
Tarski's system is \textit{exactly equivalent} to the traditional textbook
formalization, but (by clever use of equality axioms) it eliminates the
latter's primitive notions of ``proper substitution'' and ``free variable,''
replacing them with direct substitution and the notion of a variable
not occurring in a formula (which we express with distinct variable
constraints).

In advocating his system, Tarski wrote, ``The relatively complicated
character of [free variables and proper substitution] is a source
of certain inconveniences of both practical and theoretical nature;
this is clearly experienced both in teaching an elementary course of
mathematical logic and in formalizing the syntax of predicate logic for
some theoretical purposes''\cite[p.~61]{Tarski1965}\index{Tarski, Alfred}.

\subsubsection{Developing a Metamath Representation}

The standard textbook axioms of predicate calculus are somewhat
cumbersome to implement on a computer because of the complex notions of
``free variable''\index{free variable} and ``proper
substitution.''\index{proper substitution}\index{substitution!proper}
While it is possible to use the Metamath\index{Metamath} language to
implement these concepts, we have chosen not to implement them
as primitive constructs in the
\texttt{set.mm} set theory database.  Instead, we have eliminated them
within the axioms
by carefully crafting the axioms so as to avoid them,
building on Tarski's system S2.  This makes it
easy for a beginner to follow the steps in a proof without knowing any
advanced concepts other than the simple concept of
replacing\index{substitution!variable}\index{variable substitution}
variables with expressions.

In order to develop the concepts of free variable and proper
substitution from the axioms, we use an additional
Metamath statement type called ``disjoint variable
restriction''\index{disjoint variables} that we have not encountered
before.  In the context of the axioms, the statement \texttt{\$d} $ x\,
y$\index{\texttt{\$d} statement} simply means that $x$ and $y$ must be
distinct\index{distinct variables}, i.e.\ they may not be simultaneously
substituted\index{substitution!variable}\index{variable substitution}
with the same variable.  The statement \texttt{\$d} $ x\, \varphi$ means
variable $x$ must not occur in wff $\varphi$.  For the precise
definition of \texttt{\$d}, see Section~\ref{dollard}.

\subsubsection{Metamath representation}

The Metamath axiom system for predicate calculus
defined in set.mm uses Tarski's system S2.
As noted above, this has a different representation
than the traditional textbook formalization,
but it is \textit{exactly equivalent} to the textbook formalization,
and it is \textit{much} easier to work with.
This is reproduced as system S3 in Section 6 of
Megill's formalization \cite{Megill}\index{Megill, Norman}.

There is one exception, Tarski's axiom of existence,
which we label as axiom ax-6.
In the case of ax-6, Tarski's version is weaker because it includes a
distinct variable proviso. If we wish, we can also weaken our version
in this way and still have a metalogically complete system. Theorem
ax6 shows this by deriving, in the presence of the other axioms, our
ax-6 from Tarski's weaker version ax6v. However, we chose the stronger
version for our system because it is simpler to state and easier to use.

Tarski's system was designed for proving specific theorems rather than
more general theorem schemes. However, theorem schemes are much more
efficient than specific theorems for building a body of mathematical
knowledge, since they can be reused with different instances as
needed. While Tarski does derive some theorem schemes from his axioms,
their proofs require concepts that are ``outside'' of the system, such as
induction on formula length. The verification of such proofs is difficult
to automate in a proof verifier. (Specifically, Tarski treats the formulas
of his system as set-theoretical objects. In order to verify the proofs
of his theorem schemes, a proof verifier would need a significant amount
of set theory built into it.)

The Metamath axiom system for predicate calculus extends
Tarski's system to eliminate this difficulty. The additional
``auxilliary'' axiom
schemes (as we will call them in this section; see below) endow Tarski's
system with a nice property we call
metalogical completeness \cite[Remark 9.6]{Megill}\index{Megill, Norman}.
As a result, we can prove any theorem scheme
expressable in the ``simple metalogic'' of Tarski's system by using
only Metamath's direct substitution rule applied to the axiom system
(and no other metalogical or set-theoretical notions ``outside'' of the
system). Simple metalogic consists of schemes containing wff metavariables
(with no arguments) and/or set (also called ``individual'') metavariables,
accompanied by optional provisos each stating that two specified set
metavariables must be distinct or that a specified set metavariable may
not occur in a specified wff metavariable. Metamath's logic and set theory
axiom and rule schemes are all examples of simple metalogic. The schemes
of traditional predicate calculus with equality are examples which are
not simple metalogic, because they use wff metavariables with arguments
and have ``free for'' and ``not free in'' side conditions.

A rigorous justification for this system, using an older but
exactly equivalent set of axioms, can be
found in \cite{Megill}\index{Megill, Norman}.

This allows us to
take a different approach in the Metamath\index{Metamath} database
\texttt{set.mm}\index{set theory database (\texttt{set.mm})}.  We do not
directly use the primitive notions of ``free variable''\index{free variable}
and ``proper substitution''\index{proper
substitution}\index{substitution!proper} at all as primitive constructs.
Instead, we use a set
of axioms that are almost as simple to manipulate as those of
propositional calculus.  Our axiom system avoids complex primitive
notions by effectively embedding the complexity into the axioms
themselves.  As a result, we will end up with a larger number of axioms,
but they are ideally suited for a computer language such as Metamath.
(Section~\ref{metaaxioms} shows these axioms.)

We will not elaborate further
on the ``free variable'' and ``proper substitution''
concepts here.  You may consult
\cite[ch.\ 3--4]{Hamilton}\index{Hamilton, Alan G.} (as well as
many other books) for a precise explanation
of these concepts.  If you intend to do serious mathematical work, it is wise
to become familiar with the traditional textbook approach; even though the
concepts embedded in their axioms require a higher level of sophistication,
they can be more practical to deal with on an everyday, informal basis.  Even
if you are just developing Metamath proofs, familiarity with the traditional
approach can help you arrive at a proof outline much faster, which you can
then convert to the detail required by Metamath.

We do develop proper substitution rules later on, but in set.mm
they are defined as derived constructs; they are not primitives.

You should also note that our system of predicate calculus is specifically
tailored for set theory; thus there are only two specific predicates $=$ and
$\in$ and no functions\index{function!in predicate calculus}
or constants\index{constant!in predicate calculus} unlike more general systems.
We later add these.

\subsection{Set Theory}

Traditional Zermelo--Fraenkel set theory\index{Zermelo--Fraenkel set
theory}\index{set theory} with the Axiom of Choice
has 10 axioms, which can be expressed in the
language of predicate calculus.  In this section, we will list only the
names and brief English descriptions of these axioms, since we will give
you the precise formulas used by the Metamath\index{Metamath} set theory
database \texttt{set.mm} later on.

In the descriptions of the axioms, we assume that $x$, $y$, $z$, $w$, and $v$
represent sets.  These are the same as the variables\index{variable!in set
theory} in our predicate calculus system above, except that now we informally
think of the variables as ranging over sets.  Note that the terms
``object,''\index{object} ``set,''\index{set} ``element,''\index{element}
``collection,''\index{collection} and ``family''\index{family} are synonymous,
as are ``is an element of,'' ``is a member of,''\index{member} ``is contained
in,'' and ``belongs to.''  The different terms are used for convenience; for
example, ``a collection of sets'' is less confusing than ``a set of sets.''
A set $x$ is said to be a ``subset''\index{subset} of $y$ if every element of
$x$ is also an element of $y$; we also say $x$ is ``included in''
$y$.

The axioms are very general and apply to almost any conceivable mathematical
object, and this level of abstraction can be overwhelming at first.  To gain an
intuitive feel, it can be helpful to draw a picture illustrating the concept;
for example, a circle containing dots could represent a collection of sets,
and a smaller circle drawn inside the circle could represent a subset.
Overlapping circles can illustrate intersection and union.  Circles that
illustrate the concepts of set theory are frequently used in elementary
textbooks and are called Venn diagrams\index{Venn diagram}.\index{axioms of
set theory}

1. Axiom of Extensionality:  Two sets are identical if they contain the same
   elements.\index{Axiom of Extensionality}

2. Axiom of Pairing:  The set $\{ x , y \}$ exists.\index{Axiom of Pairing}

3. Axiom of Power Sets:  The power set of a set (the collection of all of
   its subsets) exists.  For example, the power set of $\{x,y\}$ is
   $\{\varnothing,\{x\},\{y\},\{x,y\}\}$ and it exists.\index{Axiom
of Power Sets}

4. Axiom of the Null Set:  The empty set $\varnothing$ exists.\index{Axiom of
the Null Set}

5. Axiom of Union:  The union of a set (the set containing the elements of
   its members) exists.  For example, the union of $\{\{x,y\},\{z\}\}$ is
 $\{x,y,z\}$ and
   it exists.\index{Axiom of Union}

6. Axiom of Regularity:  Roughly, no set can contain itself, nor can there
   be membership ``loops,'' such as a set being an
   element of one of its members.\index{Axiom of Regularity}

7. Axiom of Infinity:  An infinite set exists.  An example of an infinite
   set is the set of all
   integers.\index{Axiom of Infinity}

8. Axiom of Separation:  The set exists that is obtained by restricting $x$
   with some property.  For example, if the set of all integers exists,
   then the set of all even integers exists.\index{Axiom of Separation}

9. Axiom of Replacement:  The range of a function whose domain is restricted
   to the elements of a set $x$, is also a set.  For example, there
   is a function
   from integers (the function's domain) to their squares (its
   range).  If we
   restrict the domain to even integers, its range will become the set of
   squares of even integers, so this axiom asserts that the set of
    squares of even numbers exists.  Technical note:  In general, the
   ``function'' need not be a set but can be a proper class.
   \index{Axiom of Replacement}

10. Axiom of Choice:  Let $x$ be a set whose members are pairwise
  disjoint\index{disjoint sets} (i.e,
  whose members contain no elements in common).  Then there exists another
  set containing one element from each member of $x$.  For
  example, if $x$ is
  $\{\{y,z\},\{w,v\}\}$, where $y$, $z$, $w$, and $v$ are
  different sets, then a set such as $\{z,w\}$
  exists (but the axiom doesn't tell
  us which one).  (Actually the Axiom
  of Choice is redundant if the set $x$, as in this example, has a finite
  number of elements.)\index{Axiom of Choice}

The Axiom of Choice is usually considered an extension of ZF set theory rather
than a proper part of it.  It is sometimes considered philosophically
controversial because it specifies the existence of a set without specifying
what the set is. Constructive logics, including intuitionistic logic,
do not accept the axiom of choice.
Since there is some lingering controversy, we often prefer proofs that do
not use the axiom of choice (where there is a known alternative), and
in some cases we will use weaker axioms than the full axiom of choice.
That said, the axiom of choice is a powerful and widely-accepted tool,
so we do use it when needed.
ZF set theory that includes the Axiom of Choice is
called Zermelo--Fraenkel set theory with choice (ZFC\index{ZFC set theory}).

When expressed symbolically, the Axiom of Separation and the Axiom of
Replacement contain wff symbols and therefore each represent infinitely many
axioms, one for each possible wff. For this reason, they are often called
axiom schemes\index{axiom scheme}\index{well-formed formula (wff)}.

It turns out that the Axiom of the Null Set, the Axiom of Pairing, and the
Axiom of Separation can be derived from the other axioms and are therefore
unnecessary, although they tend to be included in standard texts for various
reasons (historical, philosophical, and possibly because some authors may not
know this).  In the Metamath\index{Metamath} set theory database, these
redundant axioms are derived from the other ones instead of truly
being considered axioms.
This is in keeping with our general goal of minimizing the number of
axioms we must depend on.

\subsection{Other Axioms}

Above we qualified the phrase "all of mathematics" with "essentially."
The main important missing piece is the ability to do category theory,
which requires huge sets (inaccessible cardinals) larger than those
postulated by the ZFC axioms. The Tarski--Grothendieck Axiom postulates
the existence of such sets.
Note that this is the same axiom used by Mizar for supporting
category theory.
The Tarski--Grothendieck axiom
can be viewed as a very strong replacement of the Axiom of Infinity,
the Axiom of Choice, and the Axiom of Power Sets.
The \texttt{set.mm} database includes this axiom; see the database
for details about it.
Again, we only use this axiom when we need to.
You are only likely to encounter or use this axiom if you are doing
category theory, since its use is highly specialized,
so we will not list the Tarsky-Grothendieck axiom
in the short list of axioms below.

Can there be even more axioms?
Of course.
G\"{o}del showed that no finite set of axioms or axiom schemes can completely
describe any consistent theory strong enough to include arithmetic.
But practically speaking, the ones above are the accepted foundation that
almost all mathematicians explicitly or implicitly base their work on.

\section{The Axioms in the Metamath Language}\label{metaaxioms}

Here we list the axioms as they appear in
\texttt{set.mm}\index{set theory database (\texttt{set.mm})} so you can
look them up there easily.  Incidentally, the \texttt{show statement
/tex} command\index{\texttt{show statement} command} was used to
typeset them.

%macros from show statement /tex
\newbox\mlinebox
\newbox\mtrialbox
\newbox\startprefix  % Prefix for first line of a formula
\newbox\contprefix  % Prefix for continuation line of a formula
\def\startm{  % Initialize formula line
  \setbox\mlinebox=\hbox{\unhcopy\startprefix}
}
\def\m#1{  % Add a symbol to the formula
  \setbox\mtrialbox=\hbox{\unhcopy\mlinebox $\,#1$}
  \ifdim\wd\mtrialbox>\hsize
    \box\mlinebox
    \setbox\mlinebox=\hbox{\unhcopy\contprefix $\,#1$}
  \else
    \setbox\mlinebox=\hbox{\unhbox\mtrialbox}
  \fi
}
\def\endm{  % Output the last line of a formula
  \box\mlinebox
}

% \SLASH for \ , \TOR for \/ (text OR), \TAND for /\ (text and)
% This embeds a following forced space to force the space.
\newcommand\SLASH{\char`\\~}
\newcommand\TOR{\char`\\/~}
\newcommand\TAND{/\char`\\~}
%
% Macro to output metamath raw text.
% This assumes \startprefix and \contprefix are set.
% NOTE: "\" is tricky to escape, use \SLASH, \TOR, and \TAND inside.
% Any use of "$ { ~ ^" must be escaped; ~ and ^ must be escaped specially.
% We escape { and } for consistency.
% For more about how this macro written, see:
% https://stackoverflow.com/questions/4073674/
% how-to-disable-indentation-in-particular-section-in-latex/4075706
% Use frenchspacing, or "e." will get an extra space after it.
\newlength\mystoreparindent
\newlength\mystorehangindent
\newenvironment{mmraw}{%
\setlength{\mystoreparindent}{\the\parindent}
\setlength{\mystorehangindent}{\the\hangindent}
\setlength{\parindent}{0pt} % TODO - we'll put in the \startprefix instead
\setlength{\hangindent}{\wd\the\contprefix}
\begin{flushleft}
\begin{frenchspacing}
\begin{tt}
{\unhcopy\startprefix}%
}{%
\end{tt}
\end{frenchspacing}
\end{flushleft}
\setlength{\parindent}{\mystoreparindent}
\setlength{\hangindent}{\mystorehangindent}
\vskip 1ex
}

\needspace{5\baselineskip}
\subsection{Propositional Calculus}\label{propcalc}\index{axioms of
propositional calculus}

\needspace{2\baselineskip}
Axiom of Simplification.\label{ax1}

\setbox\startprefix=\hbox{\tt \ \ ax-1\ \$a\ }
\setbox\contprefix=\hbox{\tt \ \ \ \ \ \ \ \ \ \ }
\startm
\m{\vdash}\m{(}\m{\varphi}\m{\rightarrow}\m{(}\m{\psi}\m{\rightarrow}\m{\varphi}\m{)}
\m{)}
\endm

\needspace{3\baselineskip}
\noindent Axiom of Distribution.

\setbox\startprefix=\hbox{\tt \ \ ax-2\ \$a\ }
\setbox\contprefix=\hbox{\tt \ \ \ \ \ \ \ \ \ \ }
\startm
\m{\vdash}\m{(}\m{(}\m{\varphi}\m{\rightarrow}\m{(}\m{\psi}\m{\rightarrow}\m{\chi}
\m{)}\m{)}\m{\rightarrow}\m{(}\m{(}\m{\varphi}\m{\rightarrow}\m{\psi}\m{)}\m{
\rightarrow}\m{(}\m{\varphi}\m{\rightarrow}\m{\chi}\m{)}\m{)}\m{)}
\endm

\needspace{2\baselineskip}
\noindent Axiom of Contraposition.

\setbox\startprefix=\hbox{\tt \ \ ax-3\ \$a\ }
\setbox\contprefix=\hbox{\tt \ \ \ \ \ \ \ \ \ \ }
\startm
\m{\vdash}\m{(}\m{(}\m{\lnot}\m{\varphi}\m{\rightarrow}\m{\lnot}\m{\psi}\m{)}\m{
\rightarrow}\m{(}\m{\psi}\m{\rightarrow}\m{\varphi}\m{)}\m{)}
\endm


\needspace{4\baselineskip}
\noindent Rule of Modus Ponens.\label{axmp}\index{modus ponens}

\setbox\startprefix=\hbox{\tt \ \ maj\ \$e\ }
\setbox\contprefix=\hbox{\tt \ \ \ \ \ \ \ \ \ }
\startm
\m{\vdash}\m{(}\m{\varphi}\m{\rightarrow}\m{\psi}\m{)}
\endm

\setbox\startprefix=\hbox{\tt \ \ min\ \$e\ }
\setbox\contprefix=\hbox{\tt \ \ \ \ \ \ \ \ \ }
\startm
\m{\vdash}\m{\varphi}
\endm

\setbox\startprefix=\hbox{\tt \ \ ax-mp\ \$a\ }
\setbox\contprefix=\hbox{\tt \ \ \ \ \ \ \ \ \ \ \ }
\startm
\m{\vdash}\m{\psi}
\endm


\needspace{7\baselineskip}
\subsection{Axioms of Predicate Calculus with Equality---Tarski's S2}\index{axioms of predicate calculus}

\needspace{3\baselineskip}
\noindent Rule of Generalization.\index{rule of generalization}

\setbox\startprefix=\hbox{\tt \ \ ax-g.1\ \$e\ }
\setbox\contprefix=\hbox{\tt \ \ \ \ \ \ \ \ \ \ \ \ }
\startm
\m{\vdash}\m{\varphi}
\endm

\setbox\startprefix=\hbox{\tt \ \ ax-gen\ \$a\ }
\setbox\contprefix=\hbox{\tt \ \ \ \ \ \ \ \ \ \ \ \ }
\startm
\m{\vdash}\m{\forall}\m{x}\m{\varphi}
\endm

\needspace{2\baselineskip}
\noindent Axiom of Quantified Implication.

\setbox\startprefix=\hbox{\tt \ \ ax-4\ \$a\ }
\setbox\contprefix=\hbox{\tt \ \ \ \ \ \ \ \ \ \ }
\startm
\m{\vdash}\m{(}\m{\forall}\m{x}\m{(}\m{\forall}\m{x}\m{\varphi}\m{\rightarrow}\m{
\psi}\m{)}\m{\rightarrow}\m{(}\m{\forall}\m{x}\m{\varphi}\m{\rightarrow}\m{
\forall}\m{x}\m{\psi}\m{)}\m{)}
\endm

\needspace{3\baselineskip}
\noindent Axiom of Distinctness.

% Aka: Add $d x ph $.
\setbox\startprefix=\hbox{\tt \ \ ax-5\ \$a\ }
\setbox\contprefix=\hbox{\tt \ \ \ \ \ \ \ \ \ \ }
\startm
\m{\vdash}\m{(}\m{\varphi}\m{\rightarrow}\m{\forall}\m{x}\m{\varphi}\m{)}\m{where}\m{ }\m{\$d}\m{ }\m{x}\m{ }\m{\varphi}\m{ }\m{(}\m{x}\m{ }\m{does}\m{ }\m{not}\m{ }\m{occur}\m{ }\m{in}\m{ }\m{\varphi}\m{)}
\endm

\needspace{2\baselineskip}
\noindent Axiom of Existence.

\setbox\startprefix=\hbox{\tt \ \ ax-6\ \$a\ }
\setbox\contprefix=\hbox{\tt \ \ \ \ \ \ \ \ \ \ }
\startm
\m{\vdash}\m{(}\m{\forall}\m{x}\m{(}\m{x}\m{=}\m{y}\m{\rightarrow}\m{\forall}
\m{x}\m{\varphi}\m{)}\m{\rightarrow}\m{\varphi}\m{)}
\endm

\needspace{2\baselineskip}
\noindent Axiom of Equality.

\setbox\startprefix=\hbox{\tt \ \ ax-7\ \$a\ }
\setbox\contprefix=\hbox{\tt \ \ \ \ \ \ \ \ \ \ }
\startm
\m{\vdash}\m{(}\m{x}\m{=}\m{y}\m{\rightarrow}\m{(}\m{x}\m{=}\m{z}\m{
\rightarrow}\m{y}\m{=}\m{z}\m{)}\m{)}
\endm

\needspace{2\baselineskip}
\noindent Axiom of Left Equality for Binary Predicate.

\setbox\startprefix=\hbox{\tt \ \ ax-8\ \$a\ }
\setbox\contprefix=\hbox{\tt \ \ \ \ \ \ \ \ \ \ \ }
\startm
\m{\vdash}\m{(}\m{x}\m{=}\m{y}\m{\rightarrow}\m{(}\m{x}\m{\in}\m{z}\m{
\rightarrow}\m{y}\m{\in}\m{z}\m{)}\m{)}
\endm

\needspace{2\baselineskip}
\noindent Axiom of Right Equality for Binary Predicate.

\setbox\startprefix=\hbox{\tt \ \ ax-9\ \$a\ }
\setbox\contprefix=\hbox{\tt \ \ \ \ \ \ \ \ \ \ \ }
\startm
\m{\vdash}\m{(}\m{x}\m{=}\m{y}\m{\rightarrow}\m{(}\m{z}\m{\in}\m{x}\m{
\rightarrow}\m{z}\m{\in}\m{y}\m{)}\m{)}
\endm


\needspace{4\baselineskip}
\subsection{Axioms of Predicate Calculus with Equality---Auxiliary}\index{axioms of predicate calculus - auxiliary}

\needspace{2\baselineskip}
\noindent Axiom of Quantified Negation.

\setbox\startprefix=\hbox{\tt \ \ ax-10\ \$a\ }
\setbox\contprefix=\hbox{\tt \ \ \ \ \ \ \ \ \ \ }
\startm
\m{\vdash}\m{(}\m{\lnot}\m{\forall}\m{x}\m{\lnot}\m{\forall}\m{x}\m{\varphi}\m{
\rightarrow}\m{\varphi}\m{)}
\endm

\needspace{2\baselineskip}
\noindent Axiom of Quantifier Commutation.

\setbox\startprefix=\hbox{\tt \ \ ax-11\ \$a\ }
\setbox\contprefix=\hbox{\tt \ \ \ \ \ \ \ \ \ \ }
\startm
\m{\vdash}\m{(}\m{\forall}\m{x}\m{\forall}\m{y}\m{\varphi}\m{\rightarrow}\m{
\forall}\m{y}\m{\forall}\m{x}\m{\varphi}\m{)}
\endm

\needspace{3\baselineskip}
\noindent Axiom of Substitution.

\setbox\startprefix=\hbox{\tt \ \ ax-12\ \$a\ }
\setbox\contprefix=\hbox{\tt \ \ \ \ \ \ \ \ \ \ \ }
\startm
\m{\vdash}\m{(}\m{\lnot}\m{\forall}\m{x}\m{\,x}\m{=}\m{y}\m{\rightarrow}\m{(}
\m{x}\m{=}\m{y}\m{\rightarrow}\m{(}\m{\varphi}\m{\rightarrow}\m{\forall}\m{x}\m{(}
\m{x}\m{=}\m{y}\m{\rightarrow}\m{\varphi}\m{)}\m{)}\m{)}\m{)}
\endm

\needspace{3\baselineskip}
\noindent Axiom of Quantified Equality.

\setbox\startprefix=\hbox{\tt \ \ ax-13\ \$a\ }
\setbox\contprefix=\hbox{\tt \ \ \ \ \ \ \ \ \ \ \ }
\startm
\m{\vdash}\m{(}\m{\lnot}\m{\forall}\m{z}\m{\,z}\m{=}\m{x}\m{\rightarrow}\m{(}
\m{\lnot}\m{\forall}\m{z}\m{\,z}\m{=}\m{y}\m{\rightarrow}\m{(}\m{x}\m{=}\m{y}
\m{\rightarrow}\m{\forall}\m{z}\m{\,x}\m{=}\m{y}\m{)}\m{)}\m{)}
\endm

% \noindent Axiom of Quantifier Substitution
%
% \setbox\startprefix=\hbox{\tt \ \ ax-c11n\ \$a\ }
% \setbox\contprefix=\hbox{\tt \ \ \ \ \ \ \ \ \ \ \ }
% \startm
% \m{\vdash}\m{(}\m{\forall}\m{x}\m{\,x}\m{=}\m{y}\m{\rightarrow}\m{(}\m{\forall}
% \m{x}\m{\varphi}\m{\rightarrow}\m{\forall}\m{y}\m{\varphi}\m{)}\m{)}
% \endm
%
% \noindent Axiom of Distinct Variables. (This axiom requires
% that two individual variables
% be distinct\index{\texttt{\$d} statement}\index{distinct
% variables}.)
%
% \setbox\startprefix=\hbox{\tt \ \ \ \ \ \ \ \ \$d\ }
% \setbox\contprefix=\hbox{\tt \ \ \ \ \ \ \ \ \ \ \ }
% \startm
% \m{x}\m{\,}\m{y}
% \endm
%
% \setbox\startprefix=\hbox{\tt \ \ ax-c16\ \$a\ }
% \setbox\contprefix=\hbox{\tt \ \ \ \ \ \ \ \ \ \ \ }
% \startm
% \m{\vdash}\m{(}\m{\forall}\m{x}\m{\,x}\m{=}\m{y}\m{\rightarrow}\m{(}\m{\varphi}\m{
% \rightarrow}\m{\forall}\m{x}\m{\varphi}\m{)}\m{)}
% \endm

% \noindent Axiom of Quantifier Introduction (2).  (This axiom requires
% that the individual variable not occur in the
% wff\index{\texttt{\$d} statement}\index{distinct variables}.)
%
% \setbox\startprefix=\hbox{\tt \ \ \ \ \ \ \ \ \$d\ }
% \setbox\contprefix=\hbox{\tt \ \ \ \ \ \ \ \ \ \ \ }
% \startm
% \m{x}\m{\,}\m{\varphi}
% \endm
% \setbox\startprefix=\hbox{\tt \ \ ax-5\ \$a\ }
% \setbox\contprefix=\hbox{\tt \ \ \ \ \ \ \ \ \ \ \ }
% \startm
% \m{\vdash}\m{(}\m{\varphi}\m{\rightarrow}\m{\forall}\m{x}\m{\varphi}\m{)}
% \endm

\subsection{Set Theory}\label{mmsettheoryaxioms}

In order to make the axioms of set theory\index{axioms of set theory} a little
more compact, there are several definitions from logic that we make use of
implicitly, namely, ``logical {\sc and},''\index{conjunction ($\wedge$)}
\index{logical {\sc and} ($\wedge$)} ``logical equivalence,''\index{logical
equivalence ($\leftrightarrow$)}\index{biconditional ($\leftrightarrow$)} and
``there exists.''\index{existential quantifier ($\exists$)}

\begin{center}\begin{tabular}{rcl}
  $( \varphi \wedge \psi )$ &\mbox{stands for}& $\neg ( \varphi
     \rightarrow \neg \psi )$\\
  $( \varphi \leftrightarrow \psi )$& \mbox{stands
     for}& $( ( \varphi \rightarrow \psi ) \wedge
     ( \psi \rightarrow \varphi ) )$\\
  $\exists x \,\varphi$ &\mbox{stands for}& $\neg \forall x \neg \varphi$
\end{tabular}\end{center}

In addition, the axioms of set theory require that all variables be
dis\-tinct,\index{distinct variables}\footnote{Set theory axioms can be
devised so that {\em no} variables are required to be distinct,
provided we replace \texttt{ax-c16} with an axiom stating that ``at
least two things exist,'' thus
making \texttt{ax-5} the only other axiom requiring the
\texttt{\$d} statement.  These axioms are unconventional and are not
presented here, but they can be found on the \url{http://metamath.org}
web site.  See also the Comment on
p.~\pageref{nodd}.}\index{\texttt{\$d} statement} thus we also assume:
\begin{center}
  \texttt{\$d }$x\,y\,z\,w$
\end{center}

\needspace{2\baselineskip}
\noindent Axiom of Extensionality.\index{Axiom of Extensionality}

\setbox\startprefix=\hbox{\tt \ \ ax-ext\ \$a\ }
\setbox\contprefix=\hbox{\tt \ \ \ \ \ \ \ \ \ \ \ \ }
\startm
\m{\vdash}\m{(}\m{\forall}\m{x}\m{(}\m{x}\m{\in}\m{y}\m{\leftrightarrow}\m{x}
\m{\in}\m{z}\m{)}\m{\rightarrow}\m{y}\m{=}\m{z}\m{)}
\endm

\needspace{3\baselineskip}
\noindent Axiom of Replacement.\index{Axiom of Replacement}

\setbox\startprefix=\hbox{\tt \ \ ax-rep\ \$a\ }
\setbox\contprefix=\hbox{\tt \ \ \ \ \ \ \ \ \ \ \ \ }
\startm
\m{\vdash}\m{(}\m{\forall}\m{w}\m{\exists}\m{y}\m{\forall}\m{z}\m{(}\m{%
\forall}\m{y}\m{\varphi}\m{\rightarrow}\m{z}\m{=}\m{y}\m{)}\m{\rightarrow}\m{%
\exists}\m{y}\m{\forall}\m{z}\m{(}\m{z}\m{\in}\m{y}\m{\leftrightarrow}\m{%
\exists}\m{w}\m{(}\m{w}\m{\in}\m{x}\m{\wedge}\m{\forall}\m{y}\m{\varphi}\m{)}%
\m{)}\m{)}
\endm

\needspace{2\baselineskip}
\noindent Axiom of Union.\index{Axiom of Union}

\setbox\startprefix=\hbox{\tt \ \ ax-un\ \$a\ }
\setbox\contprefix=\hbox{\tt \ \ \ \ \ \ \ \ \ \ \ }
\startm
\m{\vdash}\m{\exists}\m{x}\m{\forall}\m{y}\m{(}\m{\exists}\m{x}\m{(}\m{y}\m{
\in}\m{x}\m{\wedge}\m{x}\m{\in}\m{z}\m{)}\m{\rightarrow}\m{y}\m{\in}\m{x}\m{)}
\endm

\needspace{2\baselineskip}
\noindent Axiom of Power Sets.\index{Axiom of Power Sets}

\setbox\startprefix=\hbox{\tt \ \ ax-pow\ \$a\ }
\setbox\contprefix=\hbox{\tt \ \ \ \ \ \ \ \ \ \ \ \ }
\startm
\m{\vdash}\m{\exists}\m{x}\m{\forall}\m{y}\m{(}\m{\forall}\m{x}\m{(}\m{x}\m{
\in}\m{y}\m{\rightarrow}\m{x}\m{\in}\m{z}\m{)}\m{\rightarrow}\m{y}\m{\in}\m{x}
\m{)}
\endm

\needspace{3\baselineskip}
\noindent Axiom of Regularity.\index{Axiom of Regularity}

\setbox\startprefix=\hbox{\tt \ \ ax-reg\ \$a\ }
\setbox\contprefix=\hbox{\tt \ \ \ \ \ \ \ \ \ \ \ \ }
\startm
\m{\vdash}\m{(}\m{\exists}\m{x}\m{\,x}\m{\in}\m{y}\m{\rightarrow}\m{\exists}
\m{x}\m{(}\m{x}\m{\in}\m{y}\m{\wedge}\m{\forall}\m{z}\m{(}\m{z}\m{\in}\m{x}\m{
\rightarrow}\m{\lnot}\m{z}\m{\in}\m{y}\m{)}\m{)}\m{)}
\endm

\needspace{3\baselineskip}
\noindent Axiom of Infinity.\index{Axiom of Infinity}

\setbox\startprefix=\hbox{\tt \ \ ax-inf\ \$a\ }
\setbox\contprefix=\hbox{\tt \ \ \ \ \ \ \ \ \ \ \ \ \ \ \ }
\startm
\m{\vdash}\m{\exists}\m{x}\m{(}\m{y}\m{\in}\m{x}\m{\wedge}\m{\forall}\m{y}%
\m{(}\m{y}\m{\in}\m{x}\m{\rightarrow}\m{\exists}\m{z}\m{(}\m{y}\m{\in}\m{z}\m{%
\wedge}\m{z}\m{\in}\m{x}\m{)}\m{)}\m{)}
\endm

\needspace{4\baselineskip}
\noindent Axiom of Choice.\index{Axiom of Choice}

\setbox\startprefix=\hbox{\tt \ \ ax-ac\ \$a\ }
\setbox\contprefix=\hbox{\tt \ \ \ \ \ \ \ \ \ \ \ \ \ \ }
\startm
\m{\vdash}\m{\exists}\m{x}\m{\forall}\m{y}\m{\forall}\m{z}\m{(}\m{(}\m{y}\m{%
\in}\m{z}\m{\wedge}\m{z}\m{\in}\m{w}\m{)}\m{\rightarrow}\m{\exists}\m{w}\m{%
\forall}\m{y}\m{(}\m{\exists}\m{w}\m{(}\m{(}\m{y}\m{\in}\m{z}\m{\wedge}\m{z}%
\m{\in}\m{w}\m{)}\m{\wedge}\m{(}\m{y}\m{\in}\m{w}\m{\wedge}\m{w}\m{\in}\m{x}%
\m{)}\m{)}\m{\leftrightarrow}\m{y}\m{=}\m{w}\m{)}\m{)}
\endm

\subsection{That's It}

There you have it, the axioms for (essentially) all of mathematics!
Wonder at them and stare at them in awe.  Put a copy in your wallet, and
you will carry in your pocket the encoding for all theorems ever proved
and that ever will be proved, from the most mundane to the most
profound.

\section{A Hierarchy of Definitions}\label{hierarchy}

The axioms in the previous section in principle embody everything that can be
done within standard mathematics.  However, it is impractical to accomplish
very much by using them directly, for even simple concepts (from a human
perspective) can involve extremely long, incomprehensible formulas.
Mathematics is made practical by introducing definitions\index{definition}.
Definitions usually introduce new symbols, or at least new relationships among
existing symbols, to abbreviate more complex formulas.  An important
requirement for a definition is that there exist a straightforward
(algorithmic) method for eliminating the abbreviation by expanding it into the
more primitive symbol string that it represents.  Some
important definitions included in
the file \texttt{set.mm} are listed in this section for reference, and also to
give you a feel for why something like $\omega$\index{omega ($\omega$)} (the
set of natural numbers\index{natural number} 0, 1, 2,\ldots) becomes very
complicated when completely expanded into primitive symbols.

What is the motivation for definitions, aside from allowing complicated
expressions to be expressed more simply?  In the case of  $\omega$, one goal is
to provide a basis for the theory of natural numbers.\index{natural number}
Before set theory was invented, a set of axioms for arithmetic, called Peano's
postulates\index{Peano's postulates}, was devised and shown to have the
properties one expects for natural numbers.  Now anyone can postulate a
set of axioms, but if the axioms are inconsistent contradictions can be derived
from them.  Once a contradiction is derived, anything can be trivially
proved, including
all the facts of arithmetic and their negations.  To ensure that an
axiom system is at least as reliable as the axioms for set theory, we can
define sets and operations on those sets that satisfy the new axioms. In the
\texttt{set.mm} Metamath database, we prove that the elements of $\omega$ satisfy
Peano's postulates, and it's a long and hard journey to get there directly
from the axioms of set theory.  But the result is confidence in the
foundations of arithmetic.  And there is another advantage:  we now have all
the tools of set theory at our disposal for manipulating objects that obey the
axioms for arithmetic.

What are the criteria we use for definitions?  First, and of utmost importance,
the definition should not be {\em creative}\index{creative
definition}\index{definition!creative}, that
is it should not allow an expression that previously qualified as a wff but
was not provable, to become provable.   Second, the definition should be {\em
eliminable}\index{definition!eliminability}, that is, there should exist an
algorithmic method for converting any expression using the definition into
a logically equivalent expression that previously qualified as a wff.

In almost all cases below, definitions connect two expressions with either
$\leftrightarrow$ or $=$.  Eliminating\footnote{Here we mean the
elimination that a human might do in his or her head.  To eliminate them as
part of a Metamath proof we would invoke one of a number of
theorems that deal with transitivity of equivalence or equality; there are
many such examples in the proofs in \texttt{set.mm}.} such a definition is a
simple matter of substituting the expression on the left-hand side ({\em
definiendum}\index{definiendum} or thing being defined) with the equivalent,
more primitive expression on the right-hand side ({\em
definiens}\index{definiens} or definition).

Often a definition has variables on the right-hand side which do not appear on
the left-hand side; these are called {\em dummy variables}.\index{dummy
variable!in definitions}  In this case, any
allowable substitution (such as a new, distinct
variable) can be used when the definition is eliminated.  Dummy variables may
be used only if they are {\em effectively bound}\index{effectively bound
variable}, meaning that the definition will remain logically equivalent upon
any substitution of a dummy variable with any other {\em qualifying
expression}\index{qualifying expression}, i.e.\ any symbol string (such as
another variable) that
meets the restrictions on the dummy variable imposed by \texttt{\$d} and
\texttt{\$f} statements.  For example, we could define a constant $\perp$
(inverted tee, meaning logical ``false'') as $( \varphi \wedge \lnot \varphi
)$, i.e.\ ``phi and not phi.''  Here $\varphi$ is effectively bound because the
definition remains logically equivalent when we replace $\varphi$ with any
other wff.  (It is actually \texttt{df-fal}
in \texttt{set.mm}, which defines $\perp$.)

There are two cases where eliminating definitions is a little more
complex.  These cases are the definitions \texttt{df-bi} and
\texttt{df-cleq}.  The first stretches the concept of a definition a
little, as in effect it ``defines a definition;'' however, it meets our
requirements for a definition in that it is eliminable and does not
strengthen the language.  Theorem \texttt{bii} shows the substitution
needed to eliminate the $\leftrightarrow$\index{logical equivalence
($\leftrightarrow$)}\index{biconditional ($\leftrightarrow$)} symbol.

Definition \texttt{df-cleq}\index{equality ($=$)} extends the usage of
the equality symbol to include ``classes''\index{class} in set theory.  The
reason it is potentially problematic is that it can lead to statements which
do not follow from logic alone but presuppose the Axiom of
Extensionality\index{Axiom of Extensionality}, so we include this axiom
as a hypothesis for the definition.  We could have made \texttt{df-cleq} directly
eliminable by introducing a new equality symbol, but have chosen not to do so
in keeping with standard textbook practice.  Definitions such as \texttt{df-cleq}
that extend the meaning of existing symbols must be introduced carefully so
that they do not lead to contradictions.  Definition \texttt{df-clel} also
extends the meaning of an existing symbol ($\in$); while it doesn't strengthen
the language like \texttt{df-cleq}, this is not obvious and it must also be
subject to the same scrutiny.

Exercise:  Study how the wff $x\in\omega$, meaning ``$x$ is a natural
number,'' could be expanded in terms of primitive symbols, starting with the
definitions \texttt{df-clel} on p.~\pageref{dfclel} and \texttt{df-om} on
p.~\pageref{dfom} and working your way back.  Don't bother to work out the
details; just make sure that you understand how you could do it in principle.
The answer is shown in the footnote on p.~\pageref{expandom}.  If you
actually do work it out, you won't get exactly the same answer because we used
a few simplifications such as discarding occurrences of $\lnot\lnot$ (double
negation).

In the definitions below, we have placed the {\sc ascii} Metamath source
below each of the formulas to help you become familiar with the
notation in the database.  For simplicity, the necessary \texttt{\$f}
and \texttt{\$d} statements are not shown.  If you are in doubt, use the
\texttt{show statement}\index{\texttt{show statement} command} command
in the Metamath program to see the full statement.
A selection of this notation is summarized in Appendix~\ref{ASCII}.

To understand the motivation for these definitions, you should consult the
references indicated:  Takeuti and Zaring \cite{Takeuti}\index{Takeuti, Gaisi},
Quine \cite{Quine}\index{Quine, Willard Van Orman}, Bell and Machover
\cite{Bell}\index{Bell, J. L.}, and Enderton \cite{Enderton}\index{Enderton,
Herbert B.}.  Our list of definitions is provided more for reference than as a
learning aid.  However, by looking at a few of them you can gain a feel for
how the hierarchy is built up.  The definitions are a representative sample of
the 130 or so in \texttt{set.mm}, but they are complete with respect to the
theorem examples we will present in Section~\ref{sometheorems}.  Also, some are
slightly different from, but logically equivalent to, the ones in \texttt{set.mm}
(some of which have been revised over time to shorten them, for example).

\subsection{Definitions for Propositional Calculus}\label{metadefprop}

The symbols $\varphi$, $\psi$, and $\chi$ represent wffs.

Our first definition introduces the biconditional
connective\footnote{The term ``connective'' is informally used to mean a
symbol that is placed between two variables or adjacent to a variable,
whereas a mathematical ``constant'' usually indicates a symbol such as
the number 0 that may replace a variable or metavariable.  From
Metamath's point of view, there is no distinction between a connective
and a constant; both are constants in the Metamath
language.}\index{connective}\index{constant} (also called logical
equivalence)\index{logical equivalence
($\leftrightarrow$)}\index{biconditional ($\leftrightarrow$)}.  Unlike
most traditional developments, we have chosen not to have a separate
symbol such as ``Df.'' to mean ``is defined as.''  Instead, we will use
the biconditional connective for this purpose, as it lets us use
logic to manipulate definitions directly.  Here we state the properties
of the biconditional connective with a carefully crafted \texttt{\$a}
statement, which effectively uses the biconditional connective to define
itself.  The $\leftrightarrow$ symbol can be eliminated from a formula
using theorem \texttt{bii}, which is derived later.

\vskip 2ex
\noindent Define the biconditional connective.\label{df-bi}

\vskip 0.5ex
\setbox\startprefix=\hbox{\tt \ \ df-bi\ \$a\ }
\setbox\contprefix=\hbox{\tt \ \ \ \ \ \ \ \ \ \ \ }
\startm
\m{\vdash}\m{\lnot}\m{(}\m{(}\m{(}\m{\varphi}\m{\leftrightarrow}\m{\psi}\m{)}%
\m{\rightarrow}\m{\lnot}\m{(}\m{(}\m{\varphi}\m{\rightarrow}\m{\psi}\m{)}\m{%
\rightarrow}\m{\lnot}\m{(}\m{\psi}\m{\rightarrow}\m{\varphi}\m{)}\m{)}\m{)}\m{%
\rightarrow}\m{\lnot}\m{(}\m{\lnot}\m{(}\m{(}\m{\varphi}\m{\rightarrow}\m{%
\psi}\m{)}\m{\rightarrow}\m{\lnot}\m{(}\m{\psi}\m{\rightarrow}\m{\varphi}\m{)}%
\m{)}\m{\rightarrow}\m{(}\m{\varphi}\m{\leftrightarrow}\m{\psi}\m{)}\m{)}\m{)}
\endm
\begin{mmraw}%
|- -. ( ( ( ph <-> ps ) -> -. ( ( ph -> ps ) ->
-. ( ps -> ph ) ) ) -> -. ( -. ( ( ph -> ps ) -> -. (
ps -> ph ) ) -> ( ph <-> ps ) ) ) \$.
\end{mmraw}

\noindent This theorem relates the biconditional connective to primitive
connectives and can be used to eliminate the $\leftrightarrow$ symbol from any
wff.

\vskip 0.5ex
\setbox\startprefix=\hbox{\tt \ \ bii\ \$p\ }
\setbox\contprefix=\hbox{\tt \ \ \ \ \ \ \ \ \ }
\startm
\m{\vdash}\m{(}\m{(}\m{\varphi}\m{\leftrightarrow}\m{\psi}\m{)}\m{\leftrightarrow}
\m{\lnot}\m{(}\m{(}\m{\varphi}\m{\rightarrow}\m{\psi}\m{)}\m{\rightarrow}\m{\lnot}
\m{(}\m{\psi}\m{\rightarrow}\m{\varphi}\m{)}\m{)}\m{)}
\endm
\begin{mmraw}%
|- ( ( ph <-> ps ) <-> -. ( ( ph -> ps ) -> -. ( ps -> ph ) ) ) \$= ... \$.
\end{mmraw}

\noindent Define disjunction ({\sc or}).\index{disjunction ($\vee$)}%
\index{logical or (vee)@logical {\sc or} ($\vee$)}%
\index{df-or@\texttt{df-or}}\label{df-or}

\vskip 0.5ex
\setbox\startprefix=\hbox{\tt \ \ df-or\ \$a\ }
\setbox\contprefix=\hbox{\tt \ \ \ \ \ \ \ \ \ \ \ }
\startm
\m{\vdash}\m{(}\m{(}\m{\varphi}\m{\vee}\m{\psi}\m{)}\m{\leftrightarrow}\m{(}\m{
\lnot}\m{\varphi}\m{\rightarrow}\m{\psi}\m{)}\m{)}
\endm
\begin{mmraw}%
|- ( ( ph \TOR ps ) <-> ( -. ph -> ps ) ) \$.
\end{mmraw}

\noindent Define conjunction ({\sc and}).\index{conjunction ($\wedge$)}%
\index{logical {\sc and} ($\wedge$)}%
\index{df-an@\texttt{df-an}}\label{df-an}

\vskip 0.5ex
\setbox\startprefix=\hbox{\tt \ \ df-an\ \$a\ }
\setbox\contprefix=\hbox{\tt \ \ \ \ \ \ \ \ \ \ \ }
\startm
\m{\vdash}\m{(}\m{(}\m{\varphi}\m{\wedge}\m{\psi}\m{)}\m{\leftrightarrow}\m{\lnot}
\m{(}\m{\varphi}\m{\rightarrow}\m{\lnot}\m{\psi}\m{)}\m{)}
\endm
\begin{mmraw}%
|- ( ( ph \TAND ps ) <-> -. ( ph -> -. ps ) ) \$.
\end{mmraw}

\noindent Define disjunction ({\sc or}) of 3 wffs.%
\index{df-3or@\texttt{df-3or}}\label{df-3or}

\vskip 0.5ex
\setbox\startprefix=\hbox{\tt \ \ df-3or\ \$a\ }
\setbox\contprefix=\hbox{\tt \ \ \ \ \ \ \ \ \ \ \ \ }
\startm
\m{\vdash}\m{(}\m{(}\m{\varphi}\m{\vee}\m{\psi}\m{\vee}\m{\chi}\m{)}\m{
\leftrightarrow}\m{(}\m{(}\m{\varphi}\m{\vee}\m{\psi}\m{)}\m{\vee}\m{\chi}\m{)}
\m{)}
\endm
\begin{mmraw}%
|- ( ( ph \TOR ps \TOR ch ) <-> ( ( ph \TOR ps ) \TOR ch ) ) \$.
\end{mmraw}

\noindent Define conjunction ({\sc and}) of 3 wffs.%
\index{df-3an}\label{df-3an}

\vskip 0.5ex
\setbox\startprefix=\hbox{\tt \ \ df-3an\ \$a\ }
\setbox\contprefix=\hbox{\tt \ \ \ \ \ \ \ \ \ \ \ \ }
\startm
\m{\vdash}\m{(}\m{(}\m{\varphi}\m{\wedge}\m{\psi}\m{\wedge}\m{\chi}\m{)}\m{
\leftrightarrow}\m{(}\m{(}\m{\varphi}\m{\wedge}\m{\psi}\m{)}\m{\wedge}\m{\chi}
\m{)}\m{)}
\endm

\begin{mmraw}%
|- ( ( ph \TAND ps \TAND ch ) <-> ( ( ph \TAND ps ) \TAND ch ) ) \$.
\end{mmraw}

\subsection{Definitions for Predicate Calculus}\label{metadefpred}

The symbols $x$, $y$, and $z$ represent individual variables of predicate
calculus.  In this section, they are not necessarily distinct unless it is
explicitly
mentioned.

\vskip 2ex
\noindent Define existential quantification.
The expression $\exists x \varphi$ means
``there exists an $x$ where $\varphi$ is true.''\index{existential quantifier
($\exists$)}\label{df-ex}

\vskip 0.5ex
\setbox\startprefix=\hbox{\tt \ \ df-ex\ \$a\ }
\setbox\contprefix=\hbox{\tt \ \ \ \ \ \ \ \ \ \ \ }
\startm
\m{\vdash}\m{(}\m{\exists}\m{x}\m{\varphi}\m{\leftrightarrow}\m{\lnot}\m{\forall}
\m{x}\m{\lnot}\m{\varphi}\m{)}
\endm
\begin{mmraw}%
|- ( E. x ph <-> -. A. x -. ph ) \$.
\end{mmraw}

\noindent Define proper substitution.\index{proper
substitution}\index{substitution!proper}\label{df-sb}
In our notation, we use $[ y / x ] \varphi$ to mean ``the wff that
results when $y$ is properly substituted for $x$ in the wff
$\varphi$.''\footnote{
This can also be described
as substituting $x$ with $y$, $y$ properly replaces $x$, or
$x$ is properly replaced by $y$.}
% This is elsb4, though it currently says: ( [ x / y ] z e. y <-> z e. x )
For example,
$[ y / x ] z \in x$ is the same as $z \in y$.
One way to remember this notation is to notice that it looks like division
and recall that $( y / x ) \cdot x $ is $y$ (when $x \neq 0$).
The notation is different from the notation $\varphi ( x | y )$
that is sometimes used, because the latter notation is ambiguous for us:
for example, we don't know whether $\lnot \varphi ( x | y )$ is to be
interpreted as $\lnot ( \varphi ( x | y ) )$ or
$( \lnot \varphi ) ( x | y )$.\footnote{Because of the way
we initially defined wffs, this is the case
with any postfix connective\index{postfix connective} (one occurring after the
symbols being connected) or infix connective\index{infix connective} (one
occurring between the symbols being connected).  Metamath does not have a
built-in notion of operator binding strength that could eliminate the
ambiguity.  The initial parenthesis effectively provides a prefix
connective\index{prefix connective} to eliminate ambiguity.  Some conventions,
such as Polish notation\index{Polish notation} used in the 1930's and 1940's
by Polish logicians, use only prefix connectives and thus allow the total
elimination of parentheses, at the expense of readability.  In Metamath we
could actually redefine all notation to be Polish if we wanted to without
having to change any proofs!}  Other texts often use $\varphi(y)$ to indicate
our $[ y / x ] \varphi$, but this notation is even more ambiguous since there is
no explicit indication of what is being substituted.
Note that this
definition is valid even when
$x$ and $y$ are the same variable.  The first conjunct is a ``trick'' used to
achieve this property, making the definition look somewhat peculiar at
first.

\vskip 0.5ex
\setbox\startprefix=\hbox{\tt \ \ df-sb\ \$a\ }
\setbox\contprefix=\hbox{\tt \ \ \ \ \ \ \ \ \ \ \ }
\startm
\m{\vdash}\m{(}\m{[}\m{y}\m{/}\m{x}\m{]}\m{\varphi}\m{\leftrightarrow}\m{(}%
\m{(}\m{x}\m{=}\m{y}\m{\rightarrow}\m{\varphi}\m{)}\m{\wedge}\m{\exists}\m{x}%
\m{(}\m{x}\m{=}\m{y}\m{\wedge}\m{\varphi}\m{)}\m{)}\m{)}
\endm
\begin{mmraw}%
|- ( [ y / x ] ph <-> ( ( x = y -> ph ) \TAND E. x ( x = y \TAND ph ) ) ) \$.
\end{mmraw}


\noindent Define existential uniqueness\index{existential uniqueness
quantifier ($\exists "!$)} (``there exists exactly one'').  Note that $y$ is a
variable distinct from $x$ and not occurring in $\varphi$.\label{df-eu}

\vskip 0.5ex
\setbox\startprefix=\hbox{\tt \ \ df-eu\ \$a\ }
\setbox\contprefix=\hbox{\tt \ \ \ \ \ \ \ \ \ \ \ }
\startm
\m{\vdash}\m{(}\m{\exists}\m{{!}}\m{x}\m{\varphi}\m{\leftrightarrow}\m{\exists}
\m{y}\m{\forall}\m{x}\m{(}\m{\varphi}\m{\leftrightarrow}\m{x}\m{=}\m{y}\m{)}\m{)}
\endm

\begin{mmraw}%
|- ( E! x ph <-> E. y A. x ( ph <-> x = y ) ) \$.
\end{mmraw}

\subsection{Definitions for Set Theory}\label{setdefinitions}

The symbols $x$, $y$, $z$, and $w$ represent individual variables of
predicate calculus, which in set theory are understood to be sets.
However, using only the constructs shown so far would be very inconvenient.

To make set theory more practical, we introduce the notion of a ``class.''
A class\index{class} is either a set variable (such as $x$) or an
expression of the form $\{ x | \varphi\}$ (called an ``abstraction
class''\index{abstraction class}\index{class abstraction}).  Note that
sets (i.e.\ individual variables) always exist (this is a theorem of
logic, namely $\exists y \, y = x$ for any set $x$), whereas classes may
or may not exist (i.e.\ $\exists y \, y = A$ may or may not be true).
If a class does not exist it is called a ``proper class.''\index{proper
class}\index{class!proper} Definitions \texttt{df-clab},
\texttt{df-cleq}, and \texttt{df-clel} can be used to convert an
expression containing classes into one containing only set variables and
wff metavariables.

The symbols $A$, $B$, $C$, $D$, $ F$, $G$, and $R$ are metavariables that range
over classes.  A class metavariable $A$ may be eliminated from a wff by
replacing it with $\{ x|\varphi\}$ where neither $x$ nor $\varphi$ occur in
the wff.

The theory of classes can be shown to be an eliminable and conservative
extension of set theory. The \textbf{eliminability}
property shows that for every
formula in the extended language we can build a logically equivalent
formula in the basic language; so that even if the extended language
provides more ease to convey and formulate mathematical ideas for set
theory, its expressive power does not in fact strengthen the basic
language's expressive power.
The \textbf{conservation} property shows that for
every proof of a formula of the basic language in the extended system
we can build another proof of the same formula in the basic system;
so that, concerning theorems on sets only, the deductive powers of
the extended system and of the basic system are identical. Together,
these properties mean that the extended language can be treated as a
definitional extension that is \textbf{sound}.

A rigorous justification, which we will not give here, can be found in
Levy \cite[pp.~357-366]{Levy} supplementing his informal introduction to class
theory on pp.~7-17. Two other good treatments of class theory are provided
by Quine \cite[pp.~15-21]{Quine}\index{Quine, Willard Van Orman}
and also \cite[pp.~10-14]{Takeuti}\index{Takeuti, Gaisi}.
Quine's exposition (he calls them virtual classes)
is nicely written and very readable.

In the rest of this
section, individual variables are always assumed to be distinct from
each other unless otherwise indicated.  In addition, dummy variables on the
right-hand side of a definition do not occur in the class and wff
metavariables in the definition.

The definitions we present here are a partial but self-contained
collection selected from several hundred that appear in the current
\texttt{set.mm} database.  They are adequate for a basic development of
elementary set theory.

\vskip 2ex
\noindent Define the abstraction class.\index{abstraction class}\index{class
abstraction}\label{df-clab}  $x$ and $y$
need not be distinct.  Definition 2.1 of Quine, p.~16.  This definition may
seem puzzling since it is shorter than the expression being defined and does not
buy us anything in terms of brevity.  The reason we introduce this definition
is because it fits in neatly with the extension of the $\in$ connective
provided by \texttt{df-clel}.

\vskip 0.5ex
\setbox\startprefix=\hbox{\tt \ \ df-clab\ \$a\ }
\setbox\contprefix=\hbox{\tt \ \ \ \ \ \ \ \ \ \ \ \ \ }
\startm
\m{\vdash}\m{(}\m{x}\m{\in}\m{\{}\m{y}\m{|}\m{\varphi}\m{\}}\m{%
\leftrightarrow}\m{[}\m{x}\m{/}\m{y}\m{]}\m{\varphi}\m{)}
\endm
\begin{mmraw}%
|- ( x e. \{ y | ph \} <-> [ x / y ] ph ) \$.
\end{mmraw}

\noindent Define the equality connective between classes\index{class
equality}\label{df-cleq}.  See Quine or Chapter 4 of Takeuti and Zaring for its
justification and methods for eliminating it.  This is an example of a
somewhat ``dangerous'' definition, because it extends the use of the
existing equality symbol rather than introducing a new symbol, allowing
us to make statements in the original language that may not be true.
For example, it permits us to deduce $y = z \leftrightarrow \forall x (
x \in y \leftrightarrow x \in z )$ which is not a theorem of logic but
rather presupposes the Axiom of Extensionality,\index{Axiom of
Extensionality} which we include as a hypothesis so that we can know
when this axiom is assumed in a proof (with the \texttt{show
trace{\char`\_}back} command).  We could avoid the danger by introducing
another symbol, say $\eqcirc$, in place of $=$; this
would also have the advantage of making elimination of the definition
straightforward and would eliminate the need for Extensionality as a
hypothesis.  We would then also have the advantage of being able to
identify exactly where Extensionality truly comes into play.  One of our
theorems would be $x \eqcirc y \leftrightarrow x = y$
by invoking Extensionality.  However in keeping with standard practice
we retain the ``dangerous'' definition.

\vskip 0.5ex
\setbox\startprefix=\hbox{\tt \ \ df-cleq.1\ \$e\ }
\setbox\contprefix=\hbox{\tt \ \ \ \ \ \ \ \ \ \ \ \ \ \ \ }
\startm
\m{\vdash}\m{(}\m{\forall}\m{x}\m{(}\m{x}\m{\in}\m{y}\m{\leftrightarrow}\m{x}
\m{\in}\m{z}\m{)}\m{\rightarrow}\m{y}\m{=}\m{z}\m{)}
\endm
\setbox\startprefix=\hbox{\tt \ \ df-cleq\ \$a\ }
\setbox\contprefix=\hbox{\tt \ \ \ \ \ \ \ \ \ \ \ \ \ }
\startm
\m{\vdash}\m{(}\m{A}\m{=}\m{B}\m{\leftrightarrow}\m{\forall}\m{x}\m{(}\m{x}\m{
\in}\m{A}\m{\leftrightarrow}\m{x}\m{\in}\m{B}\m{)}\m{)}
\endm
% We need to reset the startprefix and contprefix.
\setbox\startprefix=\hbox{\tt \ \ df-cleq.1\ \$e\ }
\setbox\contprefix=\hbox{\tt \ \ \ \ \ \ \ \ \ \ \ \ \ \ \ }
\begin{mmraw}%
|- ( A. x ( x e. y <-> x e. z ) -> y = z ) \$.
\end{mmraw}
\setbox\startprefix=\hbox{\tt \ \ df-cleq\ \$a\ }
\setbox\contprefix=\hbox{\tt \ \ \ \ \ \ \ \ \ \ \ \ \ }
\begin{mmraw}%
|- ( A = B <-> A. x ( x e. A <-> x e. B ) ) \$.
\end{mmraw}

\noindent Define the membership connective between classes\index{class
membership}.  Theorem 6.3 of Quine, p.~41, which we adopt as a definition.
Note that it extends the use of the existing membership symbol, but unlike
{\tt df-cleq} it does not extend the set of valid wffs of logic when the class
metavariables are replaced with set variables.\label{dfclel}\label{df-clel}

\vskip 0.5ex
\setbox\startprefix=\hbox{\tt \ \ df-clel\ \$a\ }
\setbox\contprefix=\hbox{\tt \ \ \ \ \ \ \ \ \ \ \ \ \ }
\startm
\m{\vdash}\m{(}\m{A}\m{\in}\m{B}\m{\leftrightarrow}\m{\exists}\m{x}\m{(}\m{x}
\m{=}\m{A}\m{\wedge}\m{x}\m{\in}\m{B}\m{)}\m{)}
\endm
\begin{mmraw}%
|- ( A e. B <-> E. x ( x = A \TAND x e. B ) ) \$.?
\end{mmraw}

\noindent Define inequality.

\vskip 0.5ex
\setbox\startprefix=\hbox{\tt \ \ df-ne\ \$a\ }
\setbox\contprefix=\hbox{\tt \ \ \ \ \ \ \ \ \ \ \ }
\startm
\m{\vdash}\m{(}\m{A}\m{\ne}\m{B}\m{\leftrightarrow}\m{\lnot}\m{A}\m{=}\m{B}%
\m{)}
\endm
\begin{mmraw}%
|- ( A =/= B <-> -. A = B ) \$.
\end{mmraw}

\noindent Define restricted universal quantification.\index{universal
quantifier ($\forall$)!restricted}  Enderton, p.~22.

\vskip 0.5ex
\setbox\startprefix=\hbox{\tt \ \ df-ral\ \$a\ }
\setbox\contprefix=\hbox{\tt \ \ \ \ \ \ \ \ \ \ \ \ }
\startm
\m{\vdash}\m{(}\m{\forall}\m{x}\m{\in}\m{A}\m{\varphi}\m{\leftrightarrow}\m{%
\forall}\m{x}\m{(}\m{x}\m{\in}\m{A}\m{\rightarrow}\m{\varphi}\m{)}\m{)}
\endm
\begin{mmraw}%
|- ( A. x e. A ph <-> A. x ( x e. A -> ph ) ) \$.
\end{mmraw}

\noindent Define restricted existential quantification.\index{existential
quantifier ($\exists$)!restricted}  Enderton, p.~22.

\vskip 0.5ex
\setbox\startprefix=\hbox{\tt \ \ df-rex\ \$a\ }
\setbox\contprefix=\hbox{\tt \ \ \ \ \ \ \ \ \ \ \ \ }
\startm
\m{\vdash}\m{(}\m{\exists}\m{x}\m{\in}\m{A}\m{\varphi}\m{\leftrightarrow}\m{%
\exists}\m{x}\m{(}\m{x}\m{\in}\m{A}\m{\wedge}\m{\varphi}\m{)}\m{)}
\endm
\begin{mmraw}%
|- ( E. x e. A ph <-> E. x ( x e. A \TAND ph ) ) \$.
\end{mmraw}

\noindent Define the universal class\index{universal class ($V$)}.  Definition
5.20, p.~21, of Takeuti and Zaring.\label{df-v}

\vskip 0.5ex
\setbox\startprefix=\hbox{\tt \ \ df-v\ \$a\ }
\setbox\contprefix=\hbox{\tt \ \ \ \ \ \ \ \ \ \ }
\startm
\m{\vdash}\m{{\rm V}}\m{=}\m{\{}\m{x}\m{|}\m{x}\m{=}\m{x}\m{\}}
\endm
\begin{mmraw}%
|- {\char`\_}V = \{ x | x = x \} \$.
\end{mmraw}

\noindent Define the subclass\index{subclass}\index{subset} relationship
between two classes (called the subset relation if the classes are sets i.e.\
are not proper).  Definition 5.9 of Takeuti and Zaring, p.~17.\label{df-ss}

\vskip 0.5ex
\setbox\startprefix=\hbox{\tt \ \ df-ss\ \$a\ }
\setbox\contprefix=\hbox{\tt \ \ \ \ \ \ \ \ \ \ \ }
\startm
\m{\vdash}\m{(}\m{A}\m{\subseteq}\m{B}\m{\leftrightarrow}\m{\forall}\m{x}\m{(}
\m{x}\m{\in}\m{A}\m{\rightarrow}\m{x}\m{\in}\m{B}\m{)}\m{)}
\endm
\begin{mmraw}%
|- ( A C\_ B <-> A. x ( x e. A -> x e. B ) ) \$.
\end{mmraw}

\noindent Define the union\index{union} of two classes.  Definition 5.6 of Takeuti and Zaring,
p.~16.\label{df-un}

\vskip 0.5ex
\setbox\startprefix=\hbox{\tt \ \ df-un\ \$a\ }
\setbox\contprefix=\hbox{\tt \ \ \ \ \ \ \ \ \ \ \ }
\startm
\m{\vdash}\m{(}\m{A}\m{\cup}\m{B}\m{)}\m{=}\m{\{}\m{x}\m{|}\m{(}\m{x}\m{\in}
\m{A}\m{\vee}\m{x}\m{\in}\m{B}\m{)}\m{\}}
\endm
\begin{mmraw}%
( A u. B ) = \{ x | ( x e. A \TOR x e. B ) \} \$.
\end{mmraw}

\noindent Define the intersection\index{intersection} of two classes.  Definition 5.6 of
Takeuti and Zaring, p.~16.\label{df-in}

\vskip 0.5ex
\setbox\startprefix=\hbox{\tt \ \ df-in\ \$a\ }
\setbox\contprefix=\hbox{\tt \ \ \ \ \ \ \ \ \ \ \ }
\startm
\m{\vdash}\m{(}\m{A}\m{\cap}\m{B}\m{)}\m{=}\m{\{}\m{x}\m{|}\m{(}\m{x}\m{\in}
\m{A}\m{\wedge}\m{x}\m{\in}\m{B}\m{)}\m{\}}
\endm
% Caret ^ requires special treatment
\begin{mmraw}%
|- ( A i\^{}i B ) = \{ x | ( x e. A \TAND x e. B ) \} \$.
\end{mmraw}

\noindent Define class difference\index{class difference}\index{set difference}.
Definition 5.12 of Takeuti and Zaring, p.~20.  Several notations are used in
the literature; we chose the $\setminus$ convention instead of a minus sign to
reserve the latter for later use in, e.g., arithmetic.\label{df-dif}

\vskip 0.5ex
\setbox\startprefix=\hbox{\tt \ \ df-dif\ \$a\ }
\setbox\contprefix=\hbox{\tt \ \ \ \ \ \ \ \ \ \ \ \ }
\startm
\m{\vdash}\m{(}\m{A}\m{\setminus}\m{B}\m{)}\m{=}\m{\{}\m{x}\m{|}\m{(}\m{x}\m{
\in}\m{A}\m{\wedge}\m{\lnot}\m{x}\m{\in}\m{B}\m{)}\m{\}}
\endm
\begin{mmraw}%
( A \SLASH B ) = \{ x | ( x e. A \TAND -. x e. B ) \} \$.
\end{mmraw}

\noindent Define the empty or null set\index{empty set}\index{null set}.
Compare  Definition 5.14 of Takeuti and Zaring, p.~20.\label{df-nul}

\vskip 0.5ex
\setbox\startprefix=\hbox{\tt \ \ df-nul\ \$a\ }
\setbox\contprefix=\hbox{\tt \ \ \ \ \ \ \ \ \ \ }
\startm
\m{\vdash}\m{\varnothing}\m{=}\m{(}\m{{\rm V}}\m{\setminus}\m{{\rm V}}\m{)}
\endm
\begin{mmraw}%
|- (/) = ( {\char`\_}V \SLASH {\char`\_}V ) \$.
\end{mmraw}

\noindent Define power class\index{power set}\index{power class}.  Definition 5.10 of
Takeuti and Zaring, p.~17, but we also let it apply to proper classes.  (Note
that \verb$~P$ is the symbol for calligraphic P, the tilde
suggesting ``curly;'' see Appendix~\ref{ASCII}.)\label{df-pw}

\vskip 0.5ex
\setbox\startprefix=\hbox{\tt \ \ df-pw\ \$a\ }
\setbox\contprefix=\hbox{\tt \ \ \ \ \ \ \ \ \ \ \ }
\startm
\m{\vdash}\m{{\cal P}}\m{A}\m{=}\m{\{}\m{x}\m{|}\m{x}\m{\subseteq}\m{A}\m{\}}
\endm
% Special incantation required to put ~ into the text
\begin{mmraw}%
|- \char`\~P~A = \{ x | x C\_ A \} \$.
\end{mmraw}

\noindent Define the singleton of a class\index{singleton}.  Definition 7.1 of
Quine, p.~48.  It is well-defined for proper classes, although
it is not very meaningful in this case, where it evaluates to the empty
set.

\vskip 0.5ex
\setbox\startprefix=\hbox{\tt \ \ df-sn\ \$a\ }
\setbox\contprefix=\hbox{\tt \ \ \ \ \ \ \ \ \ \ \ }
\startm
\m{\vdash}\m{\{}\m{A}\m{\}}\m{=}\m{\{}\m{x}\m{|}\m{x}\m{=}\m{A}\m{\}}
\endm
\begin{mmraw}%
|- \{ A \} = \{ x | x = A \} \$.
\end{mmraw}%

\noindent Define an unordered pair of classes\index{unordered pair}\index{pair}.  Definition
7.1 of Quine, p.~48.

\vskip 0.5ex
\setbox\startprefix=\hbox{\tt \ \ df-pr\ \$a\ }
\setbox\contprefix=\hbox{\tt \ \ \ \ \ \ \ \ \ \ \ }
\startm
\m{\vdash}\m{\{}\m{A}\m{,}\m{B}\m{\}}\m{=}\m{(}\m{\{}\m{A}\m{\}}\m{\cup}\m{\{}
\m{B}\m{\}}\m{)}
\endm
\begin{mmraw}%
|- \{ A , B \} = ( \{ A \} u. \{ B \} ) \$.
\end{mmraw}

\noindent Define an unordered triple of classes\index{unordered triple}.  Definition of
Enderton, p.~19.

\vskip 0.5ex
\setbox\startprefix=\hbox{\tt \ \ df-tp\ \$a\ }
\setbox\contprefix=\hbox{\tt \ \ \ \ \ \ \ \ \ \ \ }
\startm
\m{\vdash}\m{\{}\m{A}\m{,}\m{B}\m{,}\m{C}\m{\}}\m{=}\m{(}\m{\{}\m{A}\m{,}\m{B}
\m{\}}\m{\cup}\m{\{}\m{C}\m{\}}\m{)}
\endm
\begin{mmraw}%
|- \{ A , B , C \} = ( \{ A , B \} u. \{ C \} ) \$.
\end{mmraw}%

\noindent Kuratowski's\index{Kuratowski, Kazimierz} ordered pair\index{ordered
pair} definition.  Definition 9.1 of Quine, p.~58. For proper classes it is
not meaningful but is well-defined for convenience.  (Note that \verb$<.$
stands for $\langle$ whereas \verb$<$ stands for $<$, and similarly for
\verb$>.$\,.)\label{df-op}

\vskip 0.5ex
\setbox\startprefix=\hbox{\tt \ \ df-op\ \$a\ }
\setbox\contprefix=\hbox{\tt \ \ \ \ \ \ \ \ \ \ \ }
\startm
\m{\vdash}\m{\langle}\m{A}\m{,}\m{B}\m{\rangle}\m{=}\m{\{}\m{\{}\m{A}\m{\}}
\m{,}\m{\{}\m{A}\m{,}\m{B}\m{\}}\m{\}}
\endm
\begin{mmraw}%
|- <. A , B >. = \{ \{ A \} , \{ A , B \} \} \$.
\end{mmraw}

\noindent Define the union of a class\index{union}.  Definition 5.5, p.~16,
of Takeuti and Zaring.

\vskip 0.5ex
\setbox\startprefix=\hbox{\tt \ \ df-uni\ \$a\ }
\setbox\contprefix=\hbox{\tt \ \ \ \ \ \ \ \ \ \ \ \ }
\startm
\m{\vdash}\m{\bigcup}\m{A}\m{=}\m{\{}\m{x}\m{|}\m{\exists}\m{y}\m{(}\m{x}\m{
\in}\m{y}\m{\wedge}\m{y}\m{\in}\m{A}\m{)}\m{\}}
\endm
\begin{mmraw}%
|- U. A = \{ x | E. y ( x e. y \TAND y e. A ) \} \$.
\end{mmraw}

\noindent Define the intersection\index{intersection} of a class.  Definition 7.35,
p.~44, of Takeuti and Zaring.

\vskip 0.5ex
\setbox\startprefix=\hbox{\tt \ \ df-int\ \$a\ }
\setbox\contprefix=\hbox{\tt \ \ \ \ \ \ \ \ \ \ \ \ }
\startm
\m{\vdash}\m{\bigcap}\m{A}\m{=}\m{\{}\m{x}\m{|}\m{\forall}\m{y}\m{(}\m{y}\m{
\in}\m{A}\m{\rightarrow}\m{x}\m{\in}\m{y}\m{)}\m{\}}
\endm
\begin{mmraw}%
|- |\^{}| A = \{ x | A. y ( y e. A -> x e. y ) \} \$.
\end{mmraw}

\noindent Define a transitive class\index{transitive class}\index{transitive
set}.  This should not be confused with a transitive relation, which is a different
concept.  Definition from p.~71 of Enderton, extended to classes.

\vskip 0.5ex
\setbox\startprefix=\hbox{\tt \ \ df-tr\ \$a\ }
\setbox\contprefix=\hbox{\tt \ \ \ \ \ \ \ \ \ \ \ }
\startm
\m{\vdash}\m{(}\m{\mbox{\rm Tr}}\m{A}\m{\leftrightarrow}\m{\bigcup}\m{A}\m{
\subseteq}\m{A}\m{)}
\endm
\begin{mmraw}%
|- ( Tr A <-> U. A C\_ A ) \$.
\end{mmraw}

\noindent Define a notation for a general binary relation\index{binary
relation}.  Definition 6.18, p.~29, of Takeuti and Zaring, generalized to
arbitrary classes.  This definition is well-defined, although not very
meaningful, when classes $A$ and/or $B$ are proper.\label{dfbr}  The lack of
parentheses (or any other connective) creates no ambiguity since we are defining
an atomic wff.

\vskip 0.5ex
\setbox\startprefix=\hbox{\tt \ \ df-br\ \$a\ }
\setbox\contprefix=\hbox{\tt \ \ \ \ \ \ \ \ \ \ \ }
\startm
\m{\vdash}\m{(}\m{A}\m{\,R}\m{\,B}\m{\leftrightarrow}\m{\langle}\m{A}\m{,}\m{B}
\m{\rangle}\m{\in}\m{R}\m{)}
\endm
\begin{mmraw}%
|- ( A R B <-> <. A , B >. e. R ) \$.
\end{mmraw}

\noindent Define an abstraction class of ordered pairs\index{abstraction
class!of ordered
pairs}.  A special case of Definition 4.16, p.~14, of Takeuti and Zaring.
Note that $ z $ must be distinct from $ x $ and $ y $,
and $ z $ must not occur in $\varphi$, but $ x $ and $ y $ may be
identical and may appear in $\varphi$.

\vskip 0.5ex
\setbox\startprefix=\hbox{\tt \ \ df-opab\ \$a\ }
\setbox\contprefix=\hbox{\tt \ \ \ \ \ \ \ \ \ \ \ \ \ }
\startm
\m{\vdash}\m{\{}\m{\langle}\m{x}\m{,}\m{y}\m{\rangle}\m{|}\m{\varphi}\m{\}}\m{=}
\m{\{}\m{z}\m{|}\m{\exists}\m{x}\m{\exists}\m{y}\m{(}\m{z}\m{=}\m{\langle}\m{x}
\m{,}\m{y}\m{\rangle}\m{\wedge}\m{\varphi}\m{)}\m{\}}
\endm

\begin{mmraw}%
|- \{ <. x , y >. | ph \} = \{ z | E. x E. y ( z =
<. x , y >. /\ ph ) \} \$.
\end{mmraw}

\noindent Define the epsilon relation\index{epsilon relation}.  Similar to Definition
6.22, p.~30, of Takeuti and Zaring.

\vskip 0.5ex
\setbox\startprefix=\hbox{\tt \ \ df-eprel\ \$a\ }
\setbox\contprefix=\hbox{\tt \ \ \ \ \ \ \ \ \ \ \ \ \ \ }
\startm
\m{\vdash}\m{{\rm E}}\m{=}\m{\{}\m{\langle}\m{x}\m{,}\m{y}\m{\rangle}\m{|}\m{x}\m{
\in}\m{y}\m{\}}
\endm
\begin{mmraw}%
|- \_E = \{ <. x , y >. | x e. y \} \$.
\end{mmraw}

\noindent Define a founded relation\index{founded relation}.  $R$ is a founded
relation on $A$ iff\index{iff} (if and only if) each nonempty subset of $A$
has an ``$R$-minimal element.''  Similar to Definition 6.21, p.~30, of
Takeuti and Zaring.

\vskip 0.5ex
\setbox\startprefix=\hbox{\tt \ \ df-fr\ \$a\ }
\setbox\contprefix=\hbox{\tt \ \ \ \ \ \ \ \ \ \ \ }
\startm
\m{\vdash}\m{(}\m{R}\m{\,\mbox{\rm Fr}}\m{\,A}\m{\leftrightarrow}\m{\forall}\m{x}
\m{(}\m{(}\m{x}\m{\subseteq}\m{A}\m{\wedge}\m{\lnot}\m{x}\m{=}\m{\varnothing}
\m{)}\m{\rightarrow}\m{\exists}\m{y}\m{(}\m{y}\m{\in}\m{x}\m{\wedge}\m{(}\m{x}
\m{\cap}\m{\{}\m{z}\m{|}\m{z}\m{\,R}\m{\,y}\m{\}}\m{)}\m{=}\m{\varnothing}\m{)}
\m{)}\m{)}
\endm
\begin{mmraw}%
|- ( R Fr A <-> A. x ( ( x C\_ A \TAND -. x = (/) ) ->
E. y ( y e. x \TAND ( x i\^{}i \{ z | z R y \} ) = (/) ) ) ) \$.
\end{mmraw}

\noindent Define a well-ordering\index{well-ordering}.  $R$ is a well-ordering of $A$ iff
it is founded on $A$ and the elements of $A$ are pairwise $R$-comparable.
Similar to Definition 6.24(2), p.~30, of Takeuti and Zaring.

\vskip 0.5ex
\setbox\startprefix=\hbox{\tt \ \ df-we\ \$a\ }
\setbox\contprefix=\hbox{\tt \ \ \ \ \ \ \ \ \ \ \ }
\startm
\m{\vdash}\m{(}\m{R}\m{\,\mbox{\rm We}}\m{\,A}\m{\leftrightarrow}\m{(}\m{R}\m{\,
\mbox{\rm Fr}}\m{\,A}\m{\wedge}\m{\forall}\m{x}\m{\forall}\m{y}\m{(}\m{(}\m{x}\m{
\in}\m{A}\m{\wedge}\m{y}\m{\in}\m{A}\m{)}\m{\rightarrow}\m{(}\m{x}\m{\,R}\m{\,y}
\m{\vee}\m{x}\m{=}\m{y}\m{\vee}\m{y}\m{\,R}\m{\,x}\m{)}\m{)}\m{)}\m{)}
\endm
\begin{mmraw}%
( R We A <-> ( R Fr A \TAND A. x A. y ( ( x e.
A \TAND y e. A ) -> ( x R y \TOR x = y \TOR y R x ) ) ) ) \$.
\end{mmraw}

\noindent Define the ordinal predicate\index{ordinal predicate}, which is true for a class
that is transitive and is well-ordered by the epsilon relation.  Similar to
definition on p.~468, Bell and Machover.

\vskip 0.5ex
\setbox\startprefix=\hbox{\tt \ \ df-ord\ \$a\ }
\setbox\contprefix=\hbox{\tt \ \ \ \ \ \ \ \ \ \ \ \ }
\startm
\m{\vdash}\m{(}\m{\mbox{\rm Ord}}\m{\,A}\m{\leftrightarrow}\m{(}
\m{\mbox{\rm Tr}}\m{\,A}\m{\wedge}\m{E}\m{\,\mbox{\rm We}}\m{\,A}\m{)}\m{)}
\endm
\begin{mmraw}%
|- ( Ord A <-> ( Tr A \TAND E We A ) ) \$.
\end{mmraw}

\noindent Define the class of all ordinal numbers\index{ordinal number}.  An ordinal number is
a set that satisfies the ordinal predicate.  Definition 7.11 of Takeuti and
Zaring, p.~38.

\vskip 0.5ex
\setbox\startprefix=\hbox{\tt \ \ df-on\ \$a\ }
\setbox\contprefix=\hbox{\tt \ \ \ \ \ \ \ \ \ \ \ }
\startm
\m{\vdash}\m{\,\mbox{\rm On}}\m{=}\m{\{}\m{x}\m{|}\m{\mbox{\rm Ord}}\m{\,x}
\m{\}}
\endm
\begin{mmraw}%
|- On = \{ x | Ord x \} \$.
\end{mmraw}

\noindent Define the limit ordinal predicate\index{limit ordinal}, which is true for a
non-empty ordinal that is not a successor (i.e.\ that is the union of itself).
Compare Bell and Machover, p.~471 and Exercise (1), p.~42 of Takeuti and
Zaring.

\vskip 0.5ex
\setbox\startprefix=\hbox{\tt \ \ df-lim\ \$a\ }
\setbox\contprefix=\hbox{\tt \ \ \ \ \ \ \ \ \ \ \ \ }
\startm
\m{\vdash}\m{(}\m{\mbox{\rm Lim}}\m{\,A}\m{\leftrightarrow}\m{(}\m{\mbox{
\rm Ord}}\m{\,A}\m{\wedge}\m{\lnot}\m{A}\m{=}\m{\varnothing}\m{\wedge}\m{A}
\m{=}\m{\bigcup}\m{A}\m{)}\m{)}
\endm
\begin{mmraw}%
|- ( Lim A <-> ( Ord A \TAND -. A = (/) \TAND A = U. A ) ) \$.
\end{mmraw}

\noindent Define the successor\index{successor} of a class.  Definition 7.22 of Takeuti
and Zaring, p.~41.  Our definition is a generalization to classes, although it
is meaningless when classes are proper.

\vskip 0.5ex
\setbox\startprefix=\hbox{\tt \ \ df-suc\ \$a\ }
\setbox\contprefix=\hbox{\tt \ \ \ \ \ \ \ \ \ \ \ \ }
\startm
\m{\vdash}\m{\,\mbox{\rm suc}}\m{\,A}\m{=}\m{(}\m{A}\m{\cup}\m{\{}\m{A}\m{\}}
\m{)}
\endm
\begin{mmraw}%
|- suc A = ( A u. \{ A \} ) \$.
\end{mmraw}

\noindent Define the class of natural numbers\index{natural number}\index{omega
($\omega$)}.  Compare Bell and Machover, p.~471.\label{dfom}

\vskip 0.5ex
\setbox\startprefix=\hbox{\tt \ \ df-om\ \$a\ }
\setbox\contprefix=\hbox{\tt \ \ \ \ \ \ \ \ \ \ \ }
\startm
\m{\vdash}\m{\omega}\m{=}\m{\{}\m{x}\m{|}\m{(}\m{\mbox{\rm Ord}}\m{\,x}\m{
\wedge}\m{\forall}\m{y}\m{(}\m{\mbox{\rm Lim}}\m{\,y}\m{\rightarrow}\m{x}\m{
\in}\m{y}\m{)}\m{)}\m{\}}
\endm
\begin{mmraw}%
|- om = \{ x | ( Ord x \TAND A. y ( Lim y -> x e. y ) ) \} \$.
\end{mmraw}

\noindent Define the Cartesian product (also called the
cross product)\index{Cartesian product}\index{cross product}
of two classes.  Definition 9.11 of Quine, p.~64.

\vskip 0.5ex
\setbox\startprefix=\hbox{\tt \ \ df-xp\ \$a\ }
\setbox\contprefix=\hbox{\tt \ \ \ \ \ \ \ \ \ \ \ }
\startm
\m{\vdash}\m{(}\m{A}\m{\times}\m{B}\m{)}\m{=}\m{\{}\m{\langle}\m{x}\m{,}\m{y}
\m{\rangle}\m{|}\m{(}\m{x}\m{\in}\m{A}\m{\wedge}\m{y}\m{\in}\m{B}\m{)}\m{\}}
\endm
\begin{mmraw}%
|- ( A X. B ) = \{ <. x , y >. | ( x e. A \TAND y e. B) \} \$.
\end{mmraw}

\noindent Define a relation\index{relation}.  Definition 6.4(1) of Takeuti and
Zaring, p.~23.

\vskip 0.5ex
\setbox\startprefix=\hbox{\tt \ \ df-rel\ \$a\ }
\setbox\contprefix=\hbox{\tt \ \ \ \ \ \ \ \ \ \ \ \ }
\startm
\m{\vdash}\m{(}\m{\mbox{\rm Rel}}\m{\,A}\m{\leftrightarrow}\m{A}\m{\subseteq}
\m{(}\m{{\rm V}}\m{\times}\m{{\rm V}}\m{)}\m{)}
\endm
\begin{mmraw}%
|- ( Rel A <-> A C\_ ( {\char`\_}V X. {\char`\_}V ) ) \$.
\end{mmraw}

\noindent Define the domain\index{domain} of a class.  Definition 6.5(1) of
Takeuti and Zaring, p.~24.

\vskip 0.5ex
\setbox\startprefix=\hbox{\tt \ \ df-dm\ \$a\ }
\setbox\contprefix=\hbox{\tt \ \ \ \ \ \ \ \ \ \ \ }
\startm
\m{\vdash}\m{\,\mbox{\rm dom}}\m{A}\m{=}\m{\{}\m{x}\m{|}\m{\exists}\m{y}\m{
\langle}\m{x}\m{,}\m{y}\m{\rangle}\m{\in}\m{A}\m{\}}
\endm
\begin{mmraw}%
|- dom A = \{ x | E. y <. x , y >. e. A \} \$.
\end{mmraw}

\noindent Define the range\index{range} of a class.  Definition 6.5(2) of
Takeuti and Zaring, p.~24.

\vskip 0.5ex
\setbox\startprefix=\hbox{\tt \ \ df-rn\ \$a\ }
\setbox\contprefix=\hbox{\tt \ \ \ \ \ \ \ \ \ \ \ }
\startm
\m{\vdash}\m{\,\mbox{\rm ran}}\m{A}\m{=}\m{\{}\m{y}\m{|}\m{\exists}\m{x}\m{
\langle}\m{x}\m{,}\m{y}\m{\rangle}\m{\in}\m{A}\m{\}}
\endm
\begin{mmraw}%
|- ran A = \{ y | E. x <. x , y >. e. A \} \$.
\end{mmraw}

\noindent Define the restriction\index{restriction} of a class.  Definition
6.6(1) of Takeuti and Zaring, p.~24.

\vskip 0.5ex
\setbox\startprefix=\hbox{\tt \ \ df-res\ \$a\ }
\setbox\contprefix=\hbox{\tt \ \ \ \ \ \ \ \ \ \ \ \ }
\startm
\m{\vdash}\m{(}\m{A}\m{\restriction}\m{B}\m{)}\m{=}\m{(}\m{A}\m{\cap}\m{(}\m{B}
\m{\times}\m{{\rm V}}\m{)}\m{)}
\endm
\begin{mmraw}%
|- ( A |` B ) = ( A i\^{}i ( B X. {\char`\_}V ) ) \$.
\end{mmraw}

\noindent Define the image\index{image} of a class.  Definition 6.6(2) of
Takeuti and Zaring, p.~24.

\vskip 0.5ex
\setbox\startprefix=\hbox{\tt \ \ df-ima\ \$a\ }
\setbox\contprefix=\hbox{\tt \ \ \ \ \ \ \ \ \ \ \ \ }
\startm
\m{\vdash}\m{(}\m{A}\m{``}\m{B}\m{)}\m{=}\m{\,\mbox{\rm ran}}\m{\,(}\m{A}\m{
\restriction}\m{B}\m{)}
\endm
\begin{mmraw}%
|- ( A " B ) = ran ( A |` B ) \$.
\end{mmraw}

\noindent Define the composition\index{composition} of two classes.  Definition 6.6(3) of
Takeuti and Zaring, p.~24.

\vskip 0.5ex
\setbox\startprefix=\hbox{\tt \ \ df-co\ \$a\ }
\setbox\contprefix=\hbox{\tt \ \ \ \ \ \ \ \ \ \ \ \ }
\startm
\m{\vdash}\m{(}\m{A}\m{\circ}\m{B}\m{)}\m{=}\m{\{}\m{\langle}\m{x}\m{,}\m{y}\m{
\rangle}\m{|}\m{\exists}\m{z}\m{(}\m{\langle}\m{x}\m{,}\m{z}\m{\rangle}\m{\in}
\m{B}\m{\wedge}\m{\langle}\m{z}\m{,}\m{y}\m{\rangle}\m{\in}\m{A}\m{)}\m{\}}
\endm
\begin{mmraw}%
|- ( A o. B ) = \{ <. x , y >. | E. z ( <. x , z
>. e. B \TAND <. z , y >. e. A ) \} \$.
\end{mmraw}

\noindent Define a function\index{function}.  Definition 6.4(4) of Takeuti and
Zaring, p.~24.

\vskip 0.5ex
\setbox\startprefix=\hbox{\tt \ \ df-fun\ \$a\ }
\setbox\contprefix=\hbox{\tt \ \ \ \ \ \ \ \ \ \ \ \ }
\startm
\m{\vdash}\m{(}\m{\mbox{\rm Fun}}\m{\,A}\m{\leftrightarrow}\m{(}
\m{\mbox{\rm Rel}}\m{\,A}\m{\wedge}
\m{\forall}\m{x}\m{\exists}\m{z}\m{\forall}\m{y}\m{(}
\m{\langle}\m{x}\m{,}\m{y}\m{\rangle}\m{\in}\m{A}\m{\rightarrow}\m{y}\m{=}\m{z}
\m{)}\m{)}\m{)}
\endm
\begin{mmraw}%
|- ( Fun A <-> ( Rel A /\ A. x E. z A. y ( <. x
   , y >. e. A -> y = z ) ) ) \$.
\end{mmraw}

\noindent Define a function with domain.  Definition 6.15(1) of Takeuti and
Zaring, p.~27.

\vskip 0.5ex
\setbox\startprefix=\hbox{\tt \ \ df-fn\ \$a\ }
\setbox\contprefix=\hbox{\tt \ \ \ \ \ \ \ \ \ \ \ }
\startm
\m{\vdash}\m{(}\m{A}\m{\,\mbox{\rm Fn}}\m{\,B}\m{\leftrightarrow}\m{(}
\m{\mbox{\rm Fun}}\m{\,A}\m{\wedge}\m{\mbox{\rm dom}}\m{\,A}\m{=}\m{B}\m{)}
\m{)}
\endm
\begin{mmraw}%
|- ( A Fn B <-> ( Fun A \TAND dom A = B ) ) \$.
\end{mmraw}

\noindent Define a function with domain and co-domain.  Definition 6.15(3)
of Takeuti and Zaring, p.~27.

\vskip 0.5ex
\setbox\startprefix=\hbox{\tt \ \ df-f\ \$a\ }
\setbox\contprefix=\hbox{\tt \ \ \ \ \ \ \ \ \ \ }
\startm
\m{\vdash}\m{(}\m{F}\m{:}\m{A}\m{\longrightarrow}\m{B}\m{
\leftrightarrow}\m{(}\m{F}\m{\,\mbox{\rm Fn}}\m{\,A}\m{\wedge}\m{
\mbox{\rm ran}}\m{\,F}\m{\subseteq}\m{B}\m{)}\m{)}
\endm
\begin{mmraw}%
|- ( F : A --> B <-> ( F Fn A \TAND ran F C\_ B ) ) \$.
\end{mmraw}

\noindent Define a one-to-one function\index{one-to-one function}.  Compare
Definition 6.15(5) of Takeuti and Zaring, p.~27.

\vskip 0.5ex
\setbox\startprefix=\hbox{\tt \ \ df-f1\ \$a\ }
\setbox\contprefix=\hbox{\tt \ \ \ \ \ \ \ \ \ \ \ }
\startm
\m{\vdash}\m{(}\m{F}\m{:}\m{A}\m{
\raisebox{.5ex}{${\textstyle{\:}_{\mbox{\footnotesize\rm
1\tt -\rm 1}}}\atop{\textstyle{
\longrightarrow}\atop{\textstyle{}^{\mbox{\footnotesize\rm {\ }}}}}$}
}\m{B}
\m{\leftrightarrow}\m{(}\m{F}\m{:}\m{A}\m{\longrightarrow}\m{B}
\m{\wedge}\m{\forall}\m{y}\m{\exists}\m{z}\m{\forall}\m{x}\m{(}\m{\langle}\m{x}
\m{,}\m{y}\m{\rangle}\m{\in}\m{F}\m{\rightarrow}\m{x}\m{=}\m{z}\m{)}\m{)}\m{)}
\endm
\begin{mmraw}%
|- ( F : A -1-1-> B <-> ( F : A --> B \TAND
   A. y E. z A. x ( <. x , y >. e. F -> x = z ) ) ) \$.
\end{mmraw}

\noindent Define an onto function\index{onto function}.  Definition 6.15(4) of Takeuti and
Zaring, p.~27.

\vskip 0.5ex
\setbox\startprefix=\hbox{\tt \ \ df-fo\ \$a\ }
\setbox\contprefix=\hbox{\tt \ \ \ \ \ \ \ \ \ \ \ }
\startm
\m{\vdash}\m{(}\m{F}\m{:}\m{A}\m{
\raisebox{.5ex}{${\textstyle{\:}_{\mbox{\footnotesize\rm
{\ }}}}\atop{\textstyle{
\longrightarrow}\atop{\textstyle{}^{\mbox{\footnotesize\rm onto}}}}$}
}\m{B}
\m{\leftrightarrow}\m{(}\m{F}\m{\,\mbox{\rm Fn}}\m{\,A}\m{\wedge}
\m{\mbox{\rm ran}}\m{\,F}\m{=}\m{B}\m{)}\m{)}
\endm
\begin{mmraw}%
|- ( F : A -onto-> B <-> ( F Fn A /\ ran F = B ) ) \$.
\end{mmraw}

\noindent Define a one-to-one, onto function.  Compare Definition 6.15(6) of
Takeuti and Zaring, p.~27.

\vskip 0.5ex
\setbox\startprefix=\hbox{\tt \ \ df-f1o\ \$a\ }
\setbox\contprefix=\hbox{\tt \ \ \ \ \ \ \ \ \ \ \ \ }
\startm
\m{\vdash}\m{(}\m{F}\m{:}\m{A}
\m{
\raisebox{.5ex}{${\textstyle{\:}_{\mbox{\footnotesize\rm
1\tt -\rm 1}}}\atop{\textstyle{
\longrightarrow}\atop{\textstyle{}^{\mbox{\footnotesize\rm onto}}}}$}
}
\m{B}
\m{\leftrightarrow}\m{(}\m{F}\m{:}\m{A}
\m{
\raisebox{.5ex}{${\textstyle{\:}_{\mbox{\footnotesize\rm
1\tt -\rm 1}}}\atop{\textstyle{
\longrightarrow}\atop{\textstyle{}^{\mbox{\footnotesize\rm {\ }}}}}$}
}
\m{B}\m{\wedge}\m{F}\m{:}\m{A}
\m{
\raisebox{.5ex}{${\textstyle{\:}_{\mbox{\footnotesize\rm
{\ }}}}\atop{\textstyle{
\longrightarrow}\atop{\textstyle{}^{\mbox{\footnotesize\rm onto}}}}$}
}
\m{B}\m{)}\m{)}
\endm
\begin{mmraw}%
|- ( F : A -1-1-onto-> B <-> ( F : A -1-1-> B? \TAND F : A -onto-> B ) ) \$.?
\end{mmraw}

\noindent Define the value of a function\index{function value}.  This
definition applies to any class and evaluates to the empty set when it is not
meaningful. Note that $ F`A$ means the same thing as the more familiar $ F(A)$
notation for a function's value at $A$.  The $ F`A$ notation is common in
formal set theory.\label{df-fv}

\vskip 0.5ex
\setbox\startprefix=\hbox{\tt \ \ df-fv\ \$a\ }
\setbox\contprefix=\hbox{\tt \ \ \ \ \ \ \ \ \ \ \ }
\startm
\m{\vdash}\m{(}\m{F}\m{`}\m{A}\m{)}\m{=}\m{\bigcup}\m{\{}\m{x}\m{|}\m{(}\m{F}%
\m{``}\m{\{}\m{A}\m{\}}\m{)}\m{=}\m{\{}\m{x}\m{\}}\m{\}}
\endm
\begin{mmraw}%
|- ( F ` A ) = U. \{ x | ( F " \{ A \} ) = \{ x \} \} \$.
\end{mmraw}

\noindent Define the result of an operation.\index{operation}  Here, $F$ is
     an operation on two
     values (such as $+$ for real numbers).   This is defined for proper
     classes $A$ and $B$ even though not meaningful in that case.  However,
     the definition can be meaningful when $F$ is a proper class.\label{dfopr}

\vskip 0.5ex
\setbox\startprefix=\hbox{\tt \ \ df-opr\ \$a\ }
\setbox\contprefix=\hbox{\tt \ \ \ \ \ \ \ \ \ \ \ \ }
\startm
\m{\vdash}\m{(}\m{A}\m{\,F}\m{\,B}\m{)}\m{=}\m{(}\m{F}\m{`}\m{\langle}\m{A}%
\m{,}\m{B}\m{\rangle}\m{)}
\endm
\begin{mmraw}%
|- ( A F B ) = ( F ` <. A , B >. ) \$.
\end{mmraw}

\section{Tricks of the Trade}\label{tricks}

In the \texttt{set.mm}\index{set theory database (\texttt{set.mm})} database our goal
was usually to conform to modern notation.  However in some cases the
relationship to standard textbook language may be obscured by several
unconventional devices we used to simplify the development and to take
advantage of the Metamath language.  In this section we will describe some
common conventions used in \texttt{set.mm}.

\begin{itemize}
\item
The turnstile\index{turnstile ({$\,\vdash$})} symbol, $\vdash$, meaning ``it
is provable that,'' is the first token of all assertions and hypotheses that
aren't syntax constructions.  This is a standard convention in logic.  (We
mentioned this earlier, but this symbol is bothersome to some people without a
logic background.  It has no deeper meaning but just provides us with a way to
distinguish syntax constructions from ordinary mathematical statements.)

\item
A hypothesis of the form

\vskip 1ex
\setbox\startprefix=\hbox{\tt \ \ \ \ \ \ \ \ \ \$e\ }
\setbox\contprefix=\hbox{\tt \ \ \ \ \ \ \ \ \ \ }
\startm
\m{\vdash}\m{(}\m{\varphi}\m{\rightarrow}\m{\forall}\m{x}\m{\varphi}\m{)}
\endm
\vskip 1ex

should be read ``assume variable $x$ is (effectively) not free in wff
$\varphi$.''\index{effectively not free}
Literally, this says ``assume it is provable that $\varphi \rightarrow \forall
x\, \varphi$.''  This device lets us avoid the complexities associated with
the standard treatment of free and bound variables.
%% Uncomment this when uncommenting section {formalspec} below
The footnote on p.~\pageref{effectivelybound} discusses this further.

\item
A statement of one of the forms

\vskip 1ex
\setbox\startprefix=\hbox{\tt \ \ \ \ \ \ \ \ \ \$a\ }
\setbox\contprefix=\hbox{\tt \ \ \ \ \ \ \ \ \ \ }
\startm
\m{\vdash}\m{(}\m{\lnot}\m{\forall}\m{x}\m{\,x}\m{=}\m{y}\m{\rightarrow}
\m{\ldots}\m{)}
\endm
\setbox\startprefix=\hbox{\tt \ \ \ \ \ \ \ \ \ \$p\ }
\setbox\contprefix=\hbox{\tt \ \ \ \ \ \ \ \ \ \ }
\startm
\m{\vdash}\m{(}\m{\lnot}\m{\forall}\m{x}\m{\,x}\m{=}\m{y}\m{\rightarrow}
\m{\ldots}\m{)}
\endm
\vskip 1ex

should be read ``if $x$ and $y$ are distinct variables, then...''  This
antecedent provides us with a technical device to avoid the need for the
\texttt{\$d} statement early in our development of predicate calculus,
permitting symbol manipulations to be as conceptually simple as those in
propositional calculus.  However, the \texttt{\$d} statement eventually
becomes a requirement, and after that this device is rarely used.

\item
The statement

\vskip 1ex
\setbox\startprefix=\hbox{\tt \ \ \ \ \ \ \ \ \ \$d\ }
\setbox\contprefix=\hbox{\tt \ \ \ \ \ \ \ \ \ \ }
\startm
\m{x}\m{\,y}
\endm
\vskip 1ex

should be read ``assume $x$ and $y$ are distinct variables.''

\item
The statement

\vskip 1ex
\setbox\startprefix=\hbox{\tt \ \ \ \ \ \ \ \ \ \$d\ }
\setbox\contprefix=\hbox{\tt \ \ \ \ \ \ \ \ \ \ }
\startm
\m{x}\m{\,\varphi}
\endm
\vskip 1ex

should be read ``assume $x$ does not occur in $\varphi$.''

\item
The statement

\vskip 1ex
\setbox\startprefix=\hbox{\tt \ \ \ \ \ \ \ \ \ \$d\ }
\setbox\contprefix=\hbox{\tt \ \ \ \ \ \ \ \ \ \ }
\startm
\m{x}\m{\,A}
\endm
\vskip 1ex

should be read ``assume variable $x$ does not occur in class $A$.''

\item
The restriction and hypothesis group

\vskip 1ex
\setbox\startprefix=\hbox{\tt \ \ \ \ \ \ \ \ \ \$d\ }
\setbox\contprefix=\hbox{\tt \ \ \ \ \ \ \ \ \ \ }
\startm
\m{x}\m{\,A}
\endm
\setbox\startprefix=\hbox{\tt \ \ \ \ \ \ \ \ \ \$d\ }
\setbox\contprefix=\hbox{\tt \ \ \ \ \ \ \ \ \ \ }
\startm
\m{x}\m{\,\psi}
\endm
\setbox\startprefix=\hbox{\tt \ \ \ \ \ \ \ \ \ \$e\ }
\setbox\contprefix=\hbox{\tt \ \ \ \ \ \ \ \ \ \ }
\startm
\m{\vdash}\m{(}\m{x}\m{=}\m{A}\m{\rightarrow}\m{(}\m{\varphi}\m{\leftrightarrow}
\m{\psi}\m{)}\m{)}
\endm
\vskip 1ex

is frequently used in place of explicit substitution, meaning ``assume
$\psi$ results from the proper substitution of $A$ for $x$ in
$\varphi$.''  Sometimes ``\texttt{\$e} $\vdash ( \psi \rightarrow
\forall x \, \psi )$'' is used instead of ``\texttt{\$d} $x\, \psi $,''
which requires only that $x$ be effectively not free in $\varphi$ but
not necessarily absent from it.  The use of implicit
substitution\index{substitution!implicit} is partly a matter of personal
style, although it may make proofs somewhat shorter than would be the
case with explicit substitution.

\item
The hypothesis


\vskip 1ex
\setbox\startprefix=\hbox{\tt \ \ \ \ \ \ \ \ \ \$e\ }
\setbox\contprefix=\hbox{\tt \ \ \ \ \ \ \ \ \ \ }
\startm
\m{\vdash}\m{A}\m{\in}\m{{\rm V}}
\endm
\vskip 1ex

should be read ``assume class $A$ is a set (i.e.\ exists).''
This is a convenient convention used by Quine.

\item
The restriction and hypothesis

\vskip 1ex
\setbox\startprefix=\hbox{\tt \ \ \ \ \ \ \ \ \ \$d\ }
\setbox\contprefix=\hbox{\tt \ \ \ \ \ \ \ \ \ \ }
\startm
\m{x}\m{\,y}
\endm
\setbox\startprefix=\hbox{\tt \ \ \ \ \ \ \ \ \ \$e\ }
\setbox\contprefix=\hbox{\tt \ \ \ \ \ \ \ \ \ \ }
\startm
\m{\vdash}\m{(}\m{y}\m{\in}\m{A}\m{\rightarrow}\m{\forall}\m{x}\m{\,y}
\m{\in}\m{A}\m{)}
\endm
\vskip 1ex

should be read ``assume variable $x$ is
(effectively) not free in class $A$.''

\end{itemize}

\section{A Theorem Sampler}\label{sometheorems}

In this section we list some of the more important theorems that are proved in
the \texttt{set.mm} database, and they illustrate the kinds of things that can be
done with Metamath.  While all of these facts are well-known results,
Metamath offers the advantage of easily allowing you to trace their
derivation back to axioms.  Our intent here is not to try to explain the
details or motivation; for this we refer you to the textbooks that are
mentioned in the descriptions.  (The \texttt{set.mm} file has bibliographic
references for the text references.)  Their proofs often embody important
concepts you may wish to explore with the Metamath program (see
Section~\ref{exploring}).  All the symbols that are used here are defined in
Section~\ref{hierarchy}.  For brevity we haven't included the \texttt{\$d}
restrictions or \texttt{\$f} hypotheses for these theorems; when you are
uncertain consult the \texttt{set.mm} database.

We start with \texttt{syl} (principle of the syllogism).
In \textit{Principia Mathematica}
Whitehead and Russell call this ``the principle of the
syllogism... because... the syllogism in Barbara is derived from them''
\cite[quote after Theorem *2.06 p.~101]{PM}.
Some authors call this law a ``hypothetical syllogism.''
At the time of this writing it is the most commonly referenced assertion in the
\texttt{set.mm} database.\footnote{
The Metamath program command \texttt{show usage}
shows the number of references.
}

\vskip 2ex
\noindent Theorem syl (principle of the syllogism)\index{Syllogism}%
\index{\texttt{syl}}\label{syl}.

\vskip 0.5ex
\setbox\startprefix=\hbox{\tt \ \ syl.1\ \$e\ }
\setbox\contprefix=\hbox{\tt \ \ \ \ \ \ \ \ \ \ \ }
\startm
\m{\vdash}\m{(}\m{\varphi}\m{ \rightarrow }\m{\psi}\m{)}
\endm
\setbox\startprefix=\hbox{\tt \ \ syl.2\ \$e\ }
\setbox\contprefix=\hbox{\tt \ \ \ \ \ \ \ \ \ \ \ }
\startm
\m{\vdash}\m{(}\m{\psi}\m{ \rightarrow }\m{\chi}\m{)}
\endm
\setbox\startprefix=\hbox{\tt \ \ syl\ \$p\ }
\setbox\contprefix=\hbox{\tt \ \ \ \ \ \ \ \ \ }
\startm
\m{\vdash}\m{(}\m{\varphi}\m{ \rightarrow }\m{\chi}\m{)}
\endm
\vskip 2ex

The following theorem is not very deep but provides us with a notational device
that is frequently used.  It allows us to use the expression ``$A \in V$'' as
a compact way of saying that class $A$ exists, i.e.\ is a set.

\vskip 2ex
\noindent Two ways to say ``$A$ is a set'':  $A$ is a member of the universe
$V$ if and only if $A$ exists (i.e.\ there exists a set equal to $A$).
Theorem 6.9 of Quine, p. 43.

\vskip 0.5ex
\setbox\startprefix=\hbox{\tt \ \ isset\ \$p\ }
\setbox\contprefix=\hbox{\tt \ \ \ \ \ \ \ \ \ \ \ }
\startm
\m{\vdash}\m{(}\m{A}\m{\in}\m{{\rm V}}\m{\leftrightarrow}\m{\exists}\m{x}\m{\,x}\m{=}
\m{A}\m{)}
\endm
\vskip 1ex

Next we prove the axioms of standard ZF set theory that were missing from our
axiom system.  From our point of view they are theorems since they
can be derived from the other axioms.

\vskip 2ex
\noindent Axiom of Separation\index{Axiom of Separation}
(Aussonderung)\index{Aussonderung} proved from the other axioms of ZF set
theory.  Compare Exercise 4 of Takeuti and Zaring, p.~22.

\vskip 0.5ex
\setbox\startprefix=\hbox{\tt \ \ inex1.1\ \$e\ }
\setbox\contprefix=\hbox{\tt \ \ \ \ \ \ \ \ \ \ \ \ \ \ \ }
\startm
\m{\vdash}\m{A}\m{\in}\m{{\rm V}}
\endm
\setbox\startprefix=\hbox{\tt \ \ inex\ \$p\ }
\setbox\contprefix=\hbox{\tt \ \ \ \ \ \ \ \ \ \ \ \ \ }
\startm
\m{\vdash}\m{(}\m{A}\m{\cap}\m{B}\m{)}\m{\in}\m{{\rm V}}
\endm
\vskip 1ex

\noindent Axiom of the Null Set\index{Axiom of the Null Set} proved from the
other axioms of ZF set theory. Corollary 5.16 of Takeuti and Zaring, p.~20.

\vskip 0.5ex
\setbox\startprefix=\hbox{\tt \ \ 0ex\ \$p\ }
\setbox\contprefix=\hbox{\tt \ \ \ \ \ \ \ \ \ \ \ \ }
\startm
\m{\vdash}\m{\varnothing}\m{\in}\m{{\rm V}}
\endm
\vskip 1ex

\noindent The Axiom of Pairing\index{Axiom of Pairing} proved from the other
axioms of ZF set theory.  Theorem 7.13 of Quine, p.~51.
\vskip 0.5ex
\setbox\startprefix=\hbox{\tt \ \ prex\ \$p\ }
\setbox\contprefix=\hbox{\tt \ \ \ \ \ \ \ \ \ \ \ \ \ \ }
\startm
\m{\vdash}\m{\{}\m{A}\m{,}\m{B}\m{\}}\m{\in}\m{{\rm V}}
\endm
\vskip 2ex

Next we will list some famous or important theorems that are proved in
the \texttt{set.mm} database.  None of them except \texttt{omex}
require the Axiom of Infinity, as you can verify with the \texttt{show
trace{\char`\_}back} Metamath command.

\vskip 2ex
\noindent The resolution of Russell's paradox\index{Russell's paradox}.  There
exists no set containing the set of all sets which are not members of
themselves.  Proposition 4.14 of Takeuti and Zaring, p.~14.

\vskip 0.5ex
\setbox\startprefix=\hbox{\tt \ \ ru\ \$p\ }
\setbox\contprefix=\hbox{\tt \ \ \ \ \ \ \ \ }
\startm
\m{\vdash}\m{\lnot}\m{\exists}\m{x}\m{\,x}\m{=}\m{\{}\m{y}\m{|}\m{\lnot}\m{y}
\m{\in}\m{y}\m{\}}
\endm
\vskip 1ex

\noindent Cantor's theorem\index{Cantor's theorem}.  No set can be mapped onto
its power set.  Compare Theorem 6B(b) of Enderton, p.~132.

\vskip 0.5ex
\setbox\startprefix=\hbox{\tt \ \ canth.1\ \$e\ }
\setbox\contprefix=\hbox{\tt \ \ \ \ \ \ \ \ \ \ \ \ \ }
\startm
\m{\vdash}\m{A}\m{\in}\m{{\rm V}}
\endm
\setbox\startprefix=\hbox{\tt \ \ canth\ \$p\ }
\setbox\contprefix=\hbox{\tt \ \ \ \ \ \ \ \ \ \ \ }
\startm
\m{\vdash}\m{\lnot}\m{F}\m{:}\m{A}\m{\raisebox{.5ex}{${\textstyle{\:}_{
\mbox{\footnotesize\rm {\ }}}}\atop{\textstyle{\longrightarrow}\atop{
\textstyle{}^{\mbox{\footnotesize\rm onto}}}}$}}\m{{\cal P}}\m{A}
\endm
\vskip 1ex

\noindent The Burali-Forti paradox\index{Burali-Forti paradox}.  No set
contains all ordinal numbers. Enderton, p.~194.  (Burali-Forti was one person,
not two.)

\vskip 0.5ex
\setbox\startprefix=\hbox{\tt \ \ onprc\ \$p\ }
\setbox\contprefix=\hbox{\tt \ \ \ \ \ \ \ \ \ \ \ \ }
\startm
\m{\vdash}\m{\lnot}\m{\mbox{\rm On}}\m{\in}\m{{\rm V}}
\endm
\vskip 1ex

\noindent Peano's postulates\index{Peano's postulates} for arithmetic.
Proposition 7.30 of Takeuti and Zaring, pp.~42--43.  The objects being
described are the members of $\omega$ i.e.\ the natural numbers 0, 1,
2,\ldots.  The successor\index{successor} operation suc means ``plus
one.''  \texttt{peano1} says that 0 (which is defined as the empty set)
is a natural number.  \texttt{peano2} says that if $A$ is a natural
number, so is $A+1$.  \texttt{peano3} says that 0 is not the successor
of any natural number.  \texttt{peano4} says that two natural numbers
are equal if and only if their successors are equal.  \texttt{peano5} is
essentially the same as mathematical induction.

\vskip 1ex
\setbox\startprefix=\hbox{\tt \ \ peano1\ \$p\ }
\setbox\contprefix=\hbox{\tt \ \ \ \ \ \ \ \ \ \ \ \ }
\startm
\m{\vdash}\m{\varnothing}\m{\in}\m{\omega}
\endm
\vskip 1.5ex

\setbox\startprefix=\hbox{\tt \ \ peano2\ \$p\ }
\setbox\contprefix=\hbox{\tt \ \ \ \ \ \ \ \ \ \ \ \ }
\startm
\m{\vdash}\m{(}\m{A}\m{\in}\m{\omega}\m{\rightarrow}\m{{\rm suc}}\m{A}\m{\in}%
\m{\omega}\m{)}
\endm
\vskip 1.5ex

\setbox\startprefix=\hbox{\tt \ \ peano3\ \$p\ }
\setbox\contprefix=\hbox{\tt \ \ \ \ \ \ \ \ \ \ \ \ }
\startm
\m{\vdash}\m{(}\m{A}\m{\in}\m{\omega}\m{\rightarrow}\m{\lnot}\m{{\rm suc}}%
\m{A}\m{=}\m{\varnothing}\m{)}
\endm
\vskip 1.5ex

\setbox\startprefix=\hbox{\tt \ \ peano4\ \$p\ }
\setbox\contprefix=\hbox{\tt \ \ \ \ \ \ \ \ \ \ \ \ }
\startm
\m{\vdash}\m{(}\m{(}\m{A}\m{\in}\m{\omega}\m{\wedge}\m{B}\m{\in}\m{\omega}%
\m{)}\m{\rightarrow}\m{(}\m{{\rm suc}}\m{A}\m{=}\m{{\rm suc}}\m{B}\m{%
\leftrightarrow}\m{A}\m{=}\m{B}\m{)}\m{)}
\endm
\vskip 1.5ex

\setbox\startprefix=\hbox{\tt \ \ peano5\ \$p\ }
\setbox\contprefix=\hbox{\tt \ \ \ \ \ \ \ \ \ \ \ \ }
\startm
\m{\vdash}\m{(}\m{(}\m{\varnothing}\m{\in}\m{A}\m{\wedge}\m{\forall}\m{x}\m{%
\in}\m{\omega}\m{(}\m{x}\m{\in}\m{A}\m{\rightarrow}\m{{\rm suc}}\m{x}\m{\in}%
\m{A}\m{)}\m{)}\m{\rightarrow}\m{\omega}\m{\subseteq}\m{A}\m{)}
\endm
\vskip 1.5ex

\noindent Finite Induction (mathematical induction).\index{finite
induction}\index{mathematical induction} The first hypothesis is the
basis and the second is the induction hypothesis.  Theorem Schema 22 of
Suppes, p.~136.

\vskip 0.5ex
\setbox\startprefix=\hbox{\tt \ \ findes.1\ \$e\ }
\setbox\contprefix=\hbox{\tt \ \ \ \ \ \ \ \ \ \ \ \ \ \ }
\startm
\m{\vdash}\m{[}\m{\varnothing}\m{/}\m{x}\m{]}\m{\varphi}
\endm
\setbox\startprefix=\hbox{\tt \ \ findes.2\ \$e\ }
\setbox\contprefix=\hbox{\tt \ \ \ \ \ \ \ \ \ \ \ \ \ \ }
\startm
\m{\vdash}\m{(}\m{x}\m{\in}\m{\omega}\m{\rightarrow}\m{(}\m{\varphi}\m{%
\rightarrow}\m{[}\m{{\rm suc}}\m{x}\m{/}\m{x}\m{]}\m{\varphi}\m{)}\m{)}
\endm
\setbox\startprefix=\hbox{\tt \ \ findes\ \$p\ }
\setbox\contprefix=\hbox{\tt \ \ \ \ \ \ \ \ \ \ \ \ }
\startm
\m{\vdash}\m{(}\m{x}\m{\in}\m{\omega}\m{\rightarrow}\m{\varphi}\m{)}
\endm
\vskip 1ex

\noindent Transfinite Induction with explicit substitution.  The first
hypothesis is the basis, the second is the induction hypothesis for
successors, and the third is the induction hypothesis for limit
ordinals.  Theorem Schema 4 of Suppes, p. 197.

\vskip 0.5ex
\setbox\startprefix=\hbox{\tt \ \ tfindes.1\ \$e\ }
\setbox\contprefix=\hbox{\tt \ \ \ \ \ \ \ \ \ \ \ \ \ \ \ }
\startm
\m{\vdash}\m{[}\m{\varnothing}\m{/}\m{x}\m{]}\m{\varphi}
\endm
\setbox\startprefix=\hbox{\tt \ \ tfindes.2\ \$e\ }
\setbox\contprefix=\hbox{\tt \ \ \ \ \ \ \ \ \ \ \ \ \ \ \ }
\startm
\m{\vdash}\m{(}\m{x}\m{\in}\m{{\rm On}}\m{\rightarrow}\m{(}\m{\varphi}\m{%
\rightarrow}\m{[}\m{{\rm suc}}\m{x}\m{/}\m{x}\m{]}\m{\varphi}\m{)}\m{)}
\endm
\setbox\startprefix=\hbox{\tt \ \ tfindes.3\ \$e\ }
\setbox\contprefix=\hbox{\tt \ \ \ \ \ \ \ \ \ \ \ \ \ \ \ }
\startm
\m{\vdash}\m{(}\m{{\rm Lim}}\m{y}\m{\rightarrow}\m{(}\m{\forall}\m{x}\m{\in}%
\m{y}\m{\varphi}\m{\rightarrow}\m{[}\m{y}\m{/}\m{x}\m{]}\m{\varphi}\m{)}\m{)}
\endm
\setbox\startprefix=\hbox{\tt \ \ tfindes\ \$p\ }
\setbox\contprefix=\hbox{\tt \ \ \ \ \ \ \ \ \ \ \ \ \ }
\startm
\m{\vdash}\m{(}\m{x}\m{\in}\m{{\rm On}}\m{\rightarrow}\m{\varphi}\m{)}
\endm
\vskip 1ex

\noindent Principle of Transfinite Recursion.\index{transfinite
recursion} Theorem 7.41 of Takeuti and Zaring, p.~47.  Transfinite
recursion is the key theorem that allows arithmetic of ordinals to be
rigorously defined, and has many other important uses as well.
Hypotheses \texttt{tfr.1} and \texttt{tfr.2} specify a certain (proper)
class $ F$.  The complicated definition of $ F$ is not important in
itself; what is important is that there be such an $ F$ with the
required properties, and we show this by displaying $ F$ explicitly.
\texttt{tfr1} states that $ F$ is a function whose domain is the set of
ordinal numbers.  \texttt{tfr2} states that any value of $ F$ is
completely determined by its previous values and the values of an
auxiliary function, $G$.  \texttt{tfr3} states that $ F$ is unique,
i.e.\ it is the only function that satisfies \texttt{tfr1} and
\texttt{tfr2}.  Note that $ f$ is an individual variable like $x$ and
$y$; it is just a mnemonic to remind us that $A$ is a collection of
functions.

\vskip 0.5ex
\setbox\startprefix=\hbox{\tt \ \ tfr.1\ \$e\ }
\setbox\contprefix=\hbox{\tt \ \ \ \ \ \ \ \ \ \ \ }
\startm
\m{\vdash}\m{A}\m{=}\m{\{}\m{f}\m{|}\m{\exists}\m{x}\m{\in}\m{{\rm On}}\m{(}%
\m{f}\m{{\rm Fn}}\m{x}\m{\wedge}\m{\forall}\m{y}\m{\in}\m{x}\m{(}\m{f}\m{`}%
\m{y}\m{)}\m{=}\m{(}\m{G}\m{`}\m{(}\m{f}\m{\restriction}\m{y}\m{)}\m{)}\m{)}%
\m{\}}
\endm
\setbox\startprefix=\hbox{\tt \ \ tfr.2\ \$e\ }
\setbox\contprefix=\hbox{\tt \ \ \ \ \ \ \ \ \ \ \ }
\startm
\m{\vdash}\m{F}\m{=}\m{\bigcup}\m{A}
\endm
\setbox\startprefix=\hbox{\tt \ \ tfr1\ \$p\ }
\setbox\contprefix=\hbox{\tt \ \ \ \ \ \ \ \ \ \ }
\startm
\m{\vdash}\m{F}\m{{\rm Fn}}\m{{\rm On}}
\endm
\setbox\startprefix=\hbox{\tt \ \ tfr2\ \$p\ }
\setbox\contprefix=\hbox{\tt \ \ \ \ \ \ \ \ \ \ }
\startm
\m{\vdash}\m{(}\m{z}\m{\in}\m{{\rm On}}\m{\rightarrow}\m{(}\m{F}\m{`}\m{z}%
\m{)}\m{=}\m{(}\m{G}\m{`}\m{(}\m{F}\m{\restriction}\m{z}\m{)}\m{)}\m{)}
\endm
\setbox\startprefix=\hbox{\tt \ \ tfr3\ \$p\ }
\setbox\contprefix=\hbox{\tt \ \ \ \ \ \ \ \ \ \ }
\startm
\m{\vdash}\m{(}\m{(}\m{B}\m{{\rm Fn}}\m{{\rm On}}\m{\wedge}\m{\forall}\m{x}\m{%
\in}\m{{\rm On}}\m{(}\m{B}\m{`}\m{x}\m{)}\m{=}\m{(}\m{G}\m{`}\m{(}\m{B}\m{%
\restriction}\m{x}\m{)}\m{)}\m{)}\m{\rightarrow}\m{B}\m{=}\m{F}\m{)}
\endm
\vskip 1ex

\noindent The existence of omega (the class of natural numbers).\index{natural
number}\index{omega ($\omega$)}\index{Axiom of Infinity}  Axiom 7 of Takeuti
and Zaring, p.~43.  (This is the only theorem in this section requiring the
Axiom of Infinity.)

\vskip 0.5ex
\setbox\startprefix=\hbox{\tt \
\ omex\ \$p\ }
\setbox\contprefix=\hbox{\tt \ \ \ \ \ \ \ \ \ \ }
\startm
\m{\vdash}\m{\omega}\m{\in}\m{{\rm V}}
\endm
%\vskip 2ex


\section{Axioms for Real and Complex Numbers}\label{real}
\index{real and complex numbers!axioms for}

This section presents the axioms for real and complex numbers.  Analysis
textbooks implicitly or explicitly use these axioms or their equivalents
as their starting point.  In the database \texttt{set.mm}, we define real
and complex numbers as (rather complicated) specific sets and derive these
axioms as {\em theorems} from the axioms of ZF set theory, using a method
called Dedekind cuts.  We omit the details of this construction, which you can
follow if you wish using the \texttt{set.mm} database in conjunction with the
textbooks referenced therein.

Once we prove those theorems, we then restate these proven theorems as axioms.
This lets us easily identify which axioms are needed for a particular complex number proof, without the obfuscation of the set theory used to derive them.
As a result,
the construction is actually unimportant other
than to show that sets exist that satisfy the axioms, and thus that the axioms
are consistent if ZF set theory is consistent.  When working with real numbers
you can think of them as being the actual sets resulting
from the construction (for definiteness), or you can
think of them as otherwise unspecified sets that happen to satisfy the axioms.
The derivation is not easy, but the fact that it works is quite remarkable
and lends support to the idea that ZFC set theory is all we need to
provide a foundation for essentially all of mathematics.

For the axioms we are given (or postulate) 8 classes:  $\mathbb{C}$ (the
set of complex numbers), $\mathbb{R}$ (the set of real numbers, a subset
of $\mathbb{C}$), $0$ (zero), $1$ (one), $i$ (square root of
$-1$), $+$ (plus), $\cdot$ (times), and
$<_{\mathbb{R}}$ (less than for just the real numbers).
Subtraction and division are defined terms and are not part of the
axioms; for their definitions see \texttt{set.mm}.

Note that the notation $(A+B)$ (and similarly $(A\cdot B)$) specifies a class
called an {\em operation},\index{operation} and is the function value of the
class $+$ at ordered pair $\langle A,B \rangle$.  An operation is defined by
statement \texttt{df-opr} on p.~\pageref{dfopr}.
The notation $A <_{\mathbb{R}} B$ specifies a
wff called a {\em binary relation}\index{binary relation} and means $\langle A,B \rangle \in \,<_{\mathbb{R}}$, as defined by statement \texttt{df-br} on p.~\pageref{dfbr}.

Our set of 8 given classes is assumed to satisfy the following 22 axioms
(in the axioms listed below, $<$ really means $<_{\mathbb{R}}$).

\vskip 2ex

\noindent 1. The real numbers are a subset of the complex numbers.

%\vskip 0.5ex
\setbox\startprefix=\hbox{\tt \ \ ax-resscn\ \$p\ }
\setbox\contprefix=\hbox{\tt \ \ \ \ \ \ \ \ \ \ \ \ \ \ }
\startm
\m{\vdash}\m{\mathbb{R}}\m{\subseteq}\m{\mathbb{C}}
\endm
%\vskip 1ex

\noindent 2. One is a complex number.

%\vskip 0.5ex
\setbox\startprefix=\hbox{\tt \ \ ax-1cn\ \$p\ }
\setbox\contprefix=\hbox{\tt \ \ \ \ \ \ \ \ \ \ \ }
\startm
\m{\vdash}\m{1}\m{\in}\m{\mathbb{C}}
\endm
%\vskip 1ex

\noindent 3. The imaginary number $i$ is a complex number.

%\vskip 0.5ex
\setbox\startprefix=\hbox{\tt \ \ ax-icn\ \$p\ }
\setbox\contprefix=\hbox{\tt \ \ \ \ \ \ \ \ \ \ \ }
\startm
\m{\vdash}\m{i}\m{\in}\m{\mathbb{C}}
\endm
%\vskip 1ex

\noindent 4. Complex numbers are closed under addition.

%\vskip 0.5ex
\setbox\startprefix=\hbox{\tt \ \ ax-addcl\ \$p\ }
\setbox\contprefix=\hbox{\tt \ \ \ \ \ \ \ \ \ \ \ \ \ }
\startm
\m{\vdash}\m{(}\m{(}\m{A}\m{\in}\m{\mathbb{C}}\m{\wedge}\m{B}\m{\in}\m{\mathbb{C}}%
\m{)}\m{\rightarrow}\m{(}\m{A}\m{+}\m{B}\m{)}\m{\in}\m{\mathbb{C}}\m{)}
\endm
%\vskip 1ex

\noindent 5. Real numbers are closed under addition.

%\vskip 0.5ex
\setbox\startprefix=\hbox{\tt \ \ ax-addrcl\ \$p\ }
\setbox\contprefix=\hbox{\tt \ \ \ \ \ \ \ \ \ \ \ \ \ \ }
\startm
\m{\vdash}\m{(}\m{(}\m{A}\m{\in}\m{\mathbb{R}}\m{\wedge}\m{B}\m{\in}\m{\mathbb{R}}%
\m{)}\m{\rightarrow}\m{(}\m{A}\m{+}\m{B}\m{)}\m{\in}\m{\mathbb{R}}\m{)}
\endm
%\vskip 1ex

\noindent 6. Complex numbers are closed under multiplication.

%\vskip 0.5ex
\setbox\startprefix=\hbox{\tt \ \ ax-mulcl\ \$p\ }
\setbox\contprefix=\hbox{\tt \ \ \ \ \ \ \ \ \ \ \ \ \ }
\startm
\m{\vdash}\m{(}\m{(}\m{A}\m{\in}\m{\mathbb{C}}\m{\wedge}\m{B}\m{\in}\m{\mathbb{C}}%
\m{)}\m{\rightarrow}\m{(}\m{A}\m{\cdot}\m{B}\m{)}\m{\in}\m{\mathbb{C}}\m{)}
\endm
%\vskip 1ex

\noindent 7. Real numbers are closed under multiplication.

%\vskip 0.5ex
\setbox\startprefix=\hbox{\tt \ \ ax-mulrcl\ \$p\ }
\setbox\contprefix=\hbox{\tt \ \ \ \ \ \ \ \ \ \ \ \ \ \ }
\startm
\m{\vdash}\m{(}\m{(}\m{A}\m{\in}\m{\mathbb{R}}\m{\wedge}\m{B}\m{\in}\m{\mathbb{R}}%
\m{)}\m{\rightarrow}\m{(}\m{A}\m{\cdot}\m{B}\m{)}\m{\in}\m{\mathbb{R}}\m{)}
\endm
%\vskip 1ex

\noindent 8. Multiplication of complex numbers is commutative.

%\vskip 0.5ex
\setbox\startprefix=\hbox{\tt \ \ ax-mulcom\ \$p\ }
\setbox\contprefix=\hbox{\tt \ \ \ \ \ \ \ \ \ \ \ \ \ \ }
\startm
\m{\vdash}\m{(}\m{(}\m{A}\m{\in}\m{\mathbb{C}}\m{\wedge}\m{B}\m{\in}\m{\mathbb{C}}%
\m{)}\m{\rightarrow}\m{(}\m{A}\m{\cdot}\m{B}\m{)}\m{=}\m{(}\m{B}\m{\cdot}\m{A}%
\m{)}\m{)}
\endm
%\vskip 1ex

\noindent 9. Addition of complex numbers is associative.

%\vskip 0.5ex
\setbox\startprefix=\hbox{\tt \ \ ax-addass\ \$p\ }
\setbox\contprefix=\hbox{\tt \ \ \ \ \ \ \ \ \ \ \ \ \ \ }
\startm
\m{\vdash}\m{(}\m{(}\m{A}\m{\in}\m{\mathbb{C}}\m{\wedge}\m{B}\m{\in}\m{\mathbb{C}}%
\m{\wedge}\m{C}\m{\in}\m{\mathbb{C}}\m{)}\m{\rightarrow}\m{(}\m{(}\m{A}\m{+}%
\m{B}\m{)}\m{+}\m{C}\m{)}\m{=}\m{(}\m{A}\m{+}\m{(}\m{B}\m{+}\m{C}\m{)}\m{)}%
\m{)}
\endm
%\vskip 1ex

\noindent 10. Multiplication of complex numbers is associative.

%\vskip 0.5ex
\setbox\startprefix=\hbox{\tt \ \ ax-mulass\ \$p\ }
\setbox\contprefix=\hbox{\tt \ \ \ \ \ \ \ \ \ \ \ \ \ \ }
\startm
\m{\vdash}\m{(}\m{(}\m{A}\m{\in}\m{\mathbb{C}}\m{\wedge}\m{B}\m{\in}\m{\mathbb{C}}%
\m{\wedge}\m{C}\m{\in}\m{\mathbb{C}}\m{)}\m{\rightarrow}\m{(}\m{(}\m{A}\m{\cdot}%
\m{B}\m{)}\m{\cdot}\m{C}\m{)}\m{=}\m{(}\m{A}\m{\cdot}\m{(}\m{B}\m{\cdot}\m{C}%
\m{)}\m{)}\m{)}
\endm
%\vskip 1ex

\noindent 11. Multiplication distributes over addition for complex numbers.

%\vskip 0.5ex
\setbox\startprefix=\hbox{\tt \ \ ax-distr\ \$p\ }
\setbox\contprefix=\hbox{\tt \ \ \ \ \ \ \ \ \ \ \ \ \ }
\startm
\m{\vdash}\m{(}\m{(}\m{A}\m{\in}\m{\mathbb{C}}\m{\wedge}\m{B}\m{\in}\m{\mathbb{C}}%
\m{\wedge}\m{C}\m{\in}\m{\mathbb{C}}\m{)}\m{\rightarrow}\m{(}\m{A}\m{\cdot}\m{(}%
\m{B}\m{+}\m{C}\m{)}\m{)}\m{=}\m{(}\m{(}\m{A}\m{\cdot}\m{B}\m{)}\m{+}\m{(}%
\m{A}\m{\cdot}\m{C}\m{)}\m{)}\m{)}
\endm
%\vskip 1ex

\noindent 12. The square of $i$ equals $-1$ (expressed as $i$-squared plus 1 is
0).

%\vskip 0.5ex
\setbox\startprefix=\hbox{\tt \ \ ax-i2m1\ \$p\ }
\setbox\contprefix=\hbox{\tt \ \ \ \ \ \ \ \ \ \ \ \ }
\startm
\m{\vdash}\m{(}\m{(}\m{i}\m{\cdot}\m{i}\m{)}\m{+}\m{1}\m{)}\m{=}\m{0}
\endm
%\vskip 1ex

\noindent 13. One and zero are distinct.

%\vskip 0.5ex
\setbox\startprefix=\hbox{\tt \ \ ax-1ne0\ \$p\ }
\setbox\contprefix=\hbox{\tt \ \ \ \ \ \ \ \ \ \ \ \ }
\startm
\m{\vdash}\m{1}\m{\ne}\m{0}
\endm
%\vskip 1ex

\noindent 14. One is an identity element for real multiplication.

%\vskip 0.5ex
\setbox\startprefix=\hbox{\tt \ \ ax-1rid\ \$p\ }
\setbox\contprefix=\hbox{\tt \ \ \ \ \ \ \ \ \ \ \ }
\startm
\m{\vdash}\m{(}\m{A}\m{\in}\m{\mathbb{R}}\m{\rightarrow}\m{(}\m{A}\m{\cdot}\m{1}%
\m{)}\m{=}\m{A}\m{)}
\endm
%\vskip 1ex

\noindent 15. Every real number has a negative.

%\vskip 0.5ex
\setbox\startprefix=\hbox{\tt \ \ ax-rnegex\ \$p\ }
\setbox\contprefix=\hbox{\tt \ \ \ \ \ \ \ \ \ \ \ \ \ \ }
\startm
\m{\vdash}\m{(}\m{A}\m{\in}\m{\mathbb{R}}\m{\rightarrow}\m{\exists}\m{x}\m{\in}%
\m{\mathbb{R}}\m{(}\m{A}\m{+}\m{x}\m{)}\m{=}\m{0}\m{)}
\endm
%\vskip 1ex

\noindent 16. Every nonzero real number has a reciprocal.

%\vskip 0.5ex
\setbox\startprefix=\hbox{\tt \ \ ax-rrecex\ \$p\ }
\setbox\contprefix=\hbox{\tt \ \ \ \ \ \ \ \ \ \ \ \ \ \ }
\startm
\m{\vdash}\m{(}\m{A}\m{\in}\m{\mathbb{R}}\m{\rightarrow}\m{(}\m{A}\m{\ne}\m{0}%
\m{\rightarrow}\m{\exists}\m{x}\m{\in}\m{\mathbb{R}}\m{(}\m{A}\m{\cdot}%
\m{x}\m{)}\m{=}\m{1}\m{)}\m{)}
\endm
%\vskip 1ex

\noindent 17. A complex number can be expressed in terms of two reals.

%\vskip 0.5ex
\setbox\startprefix=\hbox{\tt \ \ ax-cnre\ \$p\ }
\setbox\contprefix=\hbox{\tt \ \ \ \ \ \ \ \ \ \ \ \ }
\startm
\m{\vdash}\m{(}\m{A}\m{\in}\m{\mathbb{C}}\m{\rightarrow}\m{\exists}\m{x}\m{\in}%
\m{\mathbb{R}}\m{\exists}\m{y}\m{\in}\m{\mathbb{R}}\m{A}\m{=}\m{(}\m{x}\m{+}\m{(}%
\m{y}\m{\cdot}\m{i}\m{)}\m{)}\m{)}
\endm
%\vskip 1ex

\noindent 18. Ordering on reals satisfies strict trichotomy.

%\vskip 0.5ex
\setbox\startprefix=\hbox{\tt \ \ ax-pre-lttri\ \$p\ }
\setbox\contprefix=\hbox{\tt \ \ \ \ \ \ \ \ \ \ \ \ \ }
\startm
\m{\vdash}\m{(}\m{(}\m{A}\m{\in}\m{\mathbb{R}}\m{\wedge}\m{B}\m{\in}\m{\mathbb{R}}%
\m{)}\m{\rightarrow}\m{(}\m{A}\m{<}\m{B}\m{\leftrightarrow}\m{\lnot}\m{(}\m{A}%
\m{=}\m{B}\m{\vee}\m{B}\m{<}\m{A}\m{)}\m{)}\m{)}
\endm
%\vskip 1ex

\noindent 19. Ordering on reals is transitive.

%\vskip 0.5ex
\setbox\startprefix=\hbox{\tt \ \ ax-pre-lttrn\ \$p\ }
\setbox\contprefix=\hbox{\tt \ \ \ \ \ \ \ \ \ \ \ \ \ }
\startm
\m{\vdash}\m{(}\m{(}\m{A}\m{\in}\m{\mathbb{R}}\m{\wedge}\m{B}\m{\in}\m{\mathbb{R}}%
\m{\wedge}\m{C}\m{\in}\m{\mathbb{R}}\m{)}\m{\rightarrow}\m{(}\m{(}\m{A}\m{<}%
\m{B}\m{\wedge}\m{B}\m{<}\m{C}\m{)}\m{\rightarrow}\m{A}\m{<}\m{C}\m{)}\m{)}
\endm
%\vskip 1ex

\noindent 20. Ordering on reals is preserved after addition to both sides.

%\vskip 0.5ex
\setbox\startprefix=\hbox{\tt \ \ ax-pre-ltadd\ \$p\ }
\setbox\contprefix=\hbox{\tt \ \ \ \ \ \ \ \ \ \ \ \ \ }
\startm
\m{\vdash}\m{(}\m{(}\m{A}\m{\in}\m{\mathbb{R}}\m{\wedge}\m{B}\m{\in}\m{\mathbb{R}}%
\m{\wedge}\m{C}\m{\in}\m{\mathbb{R}}\m{)}\m{\rightarrow}\m{(}\m{A}\m{<}\m{B}\m{%
\rightarrow}\m{(}\m{C}\m{+}\m{A}\m{)}\m{<}\m{(}\m{C}\m{+}\m{B}\m{)}\m{)}\m{)}
\endm
%\vskip 1ex

\noindent 21. The product of two positive reals is positive.

%\vskip 0.5ex
\setbox\startprefix=\hbox{\tt \ \ ax-pre-mulgt0\ \$p\ }
\setbox\contprefix=\hbox{\tt \ \ \ \ \ \ \ \ \ \ \ \ \ \ }
\startm
\m{\vdash}\m{(}\m{(}\m{A}\m{\in}\m{\mathbb{R}}\m{\wedge}\m{B}\m{\in}\m{\mathbb{R}}%
\m{)}\m{\rightarrow}\m{(}\m{(}\m{0}\m{<}\m{A}\m{\wedge}\m{0}%
\m{<}\m{B}\m{)}\m{\rightarrow}\m{0}\m{<}\m{(}\m{A}\m{\cdot}\m{B}\m{)}%
\m{)}\m{)}
\endm
%\vskip 1ex

\noindent 22. A non-empty, bounded-above set of reals has a supremum.

%\vskip 0.5ex
\setbox\startprefix=\hbox{\tt \ \ ax-pre-sup\ \$p\ }
\setbox\contprefix=\hbox{\tt \ \ \ \ \ \ \ \ \ \ \ }
\startm
\m{\vdash}\m{(}\m{(}\m{A}\m{\subseteq}\m{\mathbb{R}}\m{\wedge}\m{A}\m{\ne}\m{%
\varnothing}\m{\wedge}\m{\exists}\m{x}\m{\in}\m{\mathbb{R}}\m{\forall}\m{y}\m{%
\in}\m{A}\m{\,y}\m{<}\m{x}\m{)}\m{\rightarrow}\m{\exists}\m{x}\m{\in}\m{%
\mathbb{R}}\m{(}\m{\forall}\m{y}\m{\in}\m{A}\m{\lnot}\m{x}\m{<}\m{y}\m{\wedge}\m{%
\forall}\m{y}\m{\in}\m{\mathbb{R}}\m{(}\m{y}\m{<}\m{x}\m{\rightarrow}\m{\exists}%
\m{z}\m{\in}\m{A}\m{\,y}\m{<}\m{z}\m{)}\m{)}\m{)}
\endm

% NOTE: The \m{...} expressions above could be represented as
% $ \vdash ( ( A \subseteq \mathbb{R} \wedge A \ne \varnothing \wedge \exists x \in \mathbb{R} \forall y \in A \,y < x ) \rightarrow \exists x \in \mathbb{R} ( \forall y \in A \lnot x < y \wedge \forall y \in \mathbb{R} ( y < x \rightarrow \exists z \in A \,y < z ) ) ) $

\vskip 2ex

This completes the set of axioms for real and complex numbers.  You may
wish to look at how subtraction, division, and decimal numbers
are defined in \texttt{set.mm}, and for fun look at the proof of $2+
2 = 4$ (theorem \texttt{2p2e4} in \texttt{set.mm})
as discussed in section \ref{2p2e4}.

In \texttt{set.mm} we define the non-negative integers $\mathbb{N}$, the integers
$\mathbb{Z}$, and the rationals $\mathbb{Q}$ as subsets of $\mathbb{R}$.  This leads
to the nice inclusion $\mathbb{N} \subseteq \mathbb{Z} \subseteq \mathbb{Q} \subseteq
\mathbb{R} \subseteq \mathbb{C}$, giving us a uniform framework in which, for
example, a property such as commutativity of complex number addition
automatically applies to integers.  The natural numbers $\mathbb{N}$
are different from the set $\omega$ we defined earlier, but both satisfy
Peano's postulates.

\subsubsection{Complex Number Axioms in Analysis Texts}

Most analysis texts construct complex numbers as ordered pairs of reals,
leading to construction-dependent properties that satisfy these axioms
but are not stated in their pure form.  (This is also done in
\texttt{set.mm} but our axioms are extracted from that construction.)
Other texts will simply state that $\mathbb{R}$ is a ``complete ordered
subfield of $\mathbb{C}$,'' leading to redundant axioms when this phrase
is completely expanded out.  In fact I have not seen a text with the
axioms in the explicit form above.
None of these axioms is unique individually, but this carefully worked out
collection of axioms is the result of years of work
by the Metamath community.

\subsubsection{Eliminating Unnecessary Complex Number Axioms}

We once had more axioms for real and complex numbers, but over years of time
we (the Metamath community)
have found ways to eliminate them (by proving them from other axioms)
or weaken them (by making weaker claims without reducing what
can be proved).
In particular, here are statements that used to be complex number
axioms but have since been formally proven (with Metamath) to be redundant:

\begin{itemize}
\item
  $\mathbb{C} \in V$.
  At one time this was listed as a ``complex number axiom''
  However, this is not properly speaking a complex number axiom,
  and in any case its proof uses axioms of set theory.
  Proven redundant by Mario Carneiro\index{Carneiro, Mario} on
  17-Nov-2014 (see \texttt{axcnex}).
\item
  $((A \in \mathbb{C} \land B \in \mathbb{C}$) $\rightarrow$
  $(A + B) = (B + A))$.
  Proved redundant by Eric Schmidt\index{Schmidt, Eric} on 19-Jun-2012,
  and formalized by Scott Fenton\index{Fenton, Scott} on 3-Jan-2013
  (see \texttt{addcom}).
\item
  $(A \in \mathbb{C} \rightarrow (A + 0) = A)$.
  Proved redundant by Eric Schmidt on 19-Jun-2012,
  and formalized by Scott Fenton on 3-Jan-2013
  (see \texttt{addid1}).
\item
  $(A \in \mathbb{C} \rightarrow \exists x \in \mathbb{C} (A + x) = 0)$.
  Proved redundant by Eric Schmidt and formalized on 21-May-2007
  (see \texttt{cnegex}).
\item
  $((A \in \mathbb{C} \land A \ne 0) \rightarrow \exists x \in \mathbb{C} (A \cdot x) = 1)$.
  Proved redundant by Eric Schmidt and formalized on 22-May-2007
  (see \texttt{recex}).
\item
  $0 \in \mathbb{R}$.
  Proved redundant by Eric Schmidt on 19-Feb-2005 and formalized 21-May-2007
  (see \texttt{0re}).
\end{itemize}

We could eliminate 0 as an axiomatic object by defining it as
$( ( i \cdot i ) + 1 )$
and replacing it with this expression throughout the axioms. If this
is done, axiom ax-i2m1 becomes redundant. However, the remaining axioms
would become longer and less intuitive.

Eric Schmidt's paper analyzing this axiom system \cite{Schmidt}
presented a proof that these remaining axioms,
with the possible exception of ax-mulcom, are independent of the others.
It is currently an open question if ax-mulcom is independent of the others.

\section{Two Plus Two Equals Four}\label{2p2e4}

Here is a proof that $2 + 2 = 4$, as proven in the theorem \texttt{2p2e4}
in the database \texttt{set.mm}.
This is a useful demonstration of what a Metamath proof can look like.
This proof may have more steps than you're used to, but each step is rigorously
proven all the way back to the axioms of logic and set theory.
This display was originally generated by the Metamath program
as an {\sc HTML} file.

In the table showing the proof ``Step'' is the sequential step number,
while its associated ``Expression'' is an expression that we have proved.
``Ref'' is the name of a theorem or axiom that justifies that expression,
and ``Hyp'' refers to previous steps (if any) that the theorem or axiom
needs so that we can use it.  Expressions are indented further than
the expressions that depend on them to show their interdependencies.

\begin{table}[!htbp]
\caption{Two plus two equals four}
\begin{tabular}{lllll}
\textbf{Step} & \textbf{Hyp} & \textbf{Ref} & \textbf{Expression} & \\
1  &       & df-2    & $ \; \; \vdash 2 = 1 + 1$  & \\
2  & 1     & oveq2i  & $ \; \vdash (2 + 2) = (2 + (1 + 1))$ & \\
3  &       & df-4    & $ \; \; \vdash 4 = (3 + 1)$ & \\
4  &       & df-3    & $ \; \; \; \vdash 3 = (2 + 1)$ & \\
5  & 4     & oveq1i  & $ \; \; \vdash (3 + 1) = ((2 + 1) + 1)$ & \\
6  &       & 2cn     & $ \; \; \; \vdash 2 \in \mathbb{C}$ & \\
7  &       & ax-1cn  & $ \; \; \; \vdash 1 \in \mathbb{C}$ & \\
8  & 6,7,7 & addassi & $ \; \; \vdash ((2 + 1) + 1) = (2 + (1 + 1))$ & \\
9  & 3,5,8 & 3eqtri  & $ \; \vdash 4 = (2 + (1 + 1))$ & \\
10 & 2,9   & eqtr4i  & $ \vdash (2 + 2) = 4$ & \\
\end{tabular}
\end{table}

Step 1 says that we can assert that $2 = 1 + 1$ because it is
justified by \texttt{df-2}.
What is \texttt{df-2}?
It is simply the definition of $2$, which in our system is defined as being
equal to $1 + 1$.  This shows how we can use definitions in proofs.

Look at Step 2 of the proof. In the Ref column, we see that it references
a previously proved theorem, \texttt{oveq2i}.
It turns out that
theorem \texttt{oveq2i} requires a
hypothesis, and in the Hyp column of Step 2 we indicate that Step 1 will
satisfy (match) this hypothesis.
If we looked at \texttt{oveq2i}
we would find that it proves that given some hypothesis
$A = B$, we can prove that $( C F A ) = ( C F B )$.
If we use \texttt{oveq2i} and apply step 1's result as the hypothesis,
that will mean that $A = 2$ and $B = ( 1 + 1 )$ within this use of
\texttt{oveq2i}.
We are free to select any value of $C$ and $F$ (subject to syntax constraints),
so we are free to select $C = 2$ and $F = +$,
producing our desired result,
$ (2 + 2) = (2 + (1 + 1))$.

Step 2 is an example of substitution.
In the end, every step in every proof uses only this one substitution rule.
All the rules of logic, and all the axioms, are expressed so that
they can be used via this one substitution rule.
So once you master substitution, you can master every Metamath proof,
no exceptions.

Each step is clear and can be immediately checked.
In the {\sc HTML} display you can even click on each reference to see why it is
justified, making it easy to see why the proof works.

\section{Deduction}\label{deduction}

Strictly speaking,
a deduction (also called an inference) is a kind of statement that needs
some hypotheses to be true in order for its conclusion to be true.
A theorem, on the other hand, has no hypotheses.
Informally we often call both of them theorems, but in this section we
will stick to the strict definitions.

It sometimes happens that we have proved a deduction of the form
$\varphi \Rightarrow \psi$\index{$\Rightarrow$}
(given hypothesis $\varphi$ we can prove $\psi$)
and we want to then prove a theorem of the form
$\varphi \rightarrow \psi$.

Converting a deduction (which uses a hypothesis) into a theorem
(which does not) is not as simple as you might think.
The deduction says, ``if we can prove $\varphi$ then we can prove $\psi$,''
which is in some sense weaker than saying
``$\varphi$ implies $\psi$.''
There is no axiom of logic that permits us to directly obtain the theorem
given the deduction.\footnote{
The conversion of a deduction to a theorem does not even hold in general
for quantum propositional calculus,
which is a weak subset of classical propositional calculus.
It has been shown that adding the Standard Deduction Theorem (discussed below)
to quantum propositional calculus turns it into classical
propositional calculus!
}

This is in contrast to going the other way.
If we have the theorem ($\varphi \rightarrow \psi$),
it is easy to recover the deduction
($\varphi \Rightarrow \psi$)
using modus ponens\index{modus ponens}
(\texttt{ax-mp}; see section \ref{axmp}).

In the following subsections we first discuss the standard deduction theorem
(the traditional but awkward way to convert deductions into theorems) and
the weak deduction theorem (a limited version of the standard deduction
theorem that is easier to use and was once widely used in
\texttt{set.mm}\index{set theory database (\texttt{set.mm})}).
In section \ref{deductionstyle} we discuss
deduction style, the newer approach we now recommend in most cases.
Deduction style uses ``deduction form,'' a form that
prefixes each hypothesis (other than definitions) and the
conclusion with a universal antecedent (``$\varphi \rightarrow$'').
Deduction style is widely used in \texttt{set.mm},
so it is useful to understand it and \textit{why} it is widely used.
Section \ref{naturaldeduction}
briefly discusses our approach for using natural deduction
within \texttt{set.mm},
as that approach is deeply related to deduction style.
We conclude with a summary of the strengths of
our approach, which we believe are compelling.

\subsection{The Standard Deduction Theorem}\label{standarddeductiontheorem}

It is possible to make use of information
contained in the deduction or its proof to assist us with the proof of
the related theorem.
In traditional logic books, there is a metatheorem called the
Deduction Theorem\index{Deduction Theorem}\index{Standard Deduction Theorem},
discovered independently by Herbrand and Tarski around 1930.
The Deduction Theorem, which we often call the Standard Deduction Theorem,
provides an algorithm for constructing a proof of a theorem from the
proof of its corresponding deduction. See, for example,
\cite[p.~56]{Margaris}\index{Margaris, Angelo}.
To construct a proof for a theorem, the
algorithm looks at each step in the proof of the original deduction and
rewrites the step with several steps wherein the hypothesis is eliminated
and becomes an antecedent.

In ordinary mathematics, no one actually carries out the algorithm,
because (in its most basic form) it involves an exponential explosion of
the number of proof steps as more hypotheses are eliminated. Instead,
the Standard Deduction Theorem is invoked simply to claim that it can
be done in principle, without actually doing it.
What's more, the algorithm is not as simple as it might first appear
when applying it rigorously.
There is a subtle restriction on the Standard Deduction Theorem
that must be taken into account involving the axiom of generalization
when working with predicate calculus (see the literature for more detail).

One of the goals of Metamath is to let you plainly see, with as few
underlying concepts as possible, how mathematics can be derived directly
from the axioms, and not indirectly according to some hidden rules
buried inside a program or understood only by logicians. If we added
the Standard Deduction Theorem to the language and proof verifier,
that would greatly complicate both and largely defeat Metamath's goal
of simplicity. In principle, we could show direct proofs by expanding
out the proof steps generated by the algorithm of the Standard Deduction
Theorem, but that is not feasible in practice because the number of proof
steps quickly becomes huge, even astronomical.
Since the algorithm of the Standard Deduction Theorem is driven by the proof,
we would have to go through that proof
all over again---starting from axioms---in order to obtain the theorem form.
In terms of proof length, there would be no savings over just
proving the theorem directly instead of first proving the deduction form.

\subsection{Weak Deduction Theorem}\label{weakdeductiontheorem}

We have developed
a more efficient method for proving a theorem from a deduction
that can be used instead of the Standard Deduction Theorem
in many (but not all) cases.
We call this more efficient method the
Weak Deduction Theorem\index{Weak Deduction Theorem}.\footnote{
There is also an unrelated ``Weak Deduction Theorem''
in the field of relevance logic, so to avoid confusion we could call
ours the ``Weak Deduction Theorem for Classical Logic.''}
Unlike the Standard Deduction Theorem, the Weak Deduction Theorem produces the
theorem directly from a special substitution instance of the deduction,
using a small, fixed number of steps roughly proportional to the length
of the final theorem.

If you come to a proof referencing the Weak Deduction Theorem
\texttt{dedth} (or one of its variants \texttt{dedthxx}),
here is how to follow the proof without getting into the details:
just click on the theorem referenced in the step
just before the reference to \texttt{dedth} and ignore everything else.
Theorem \texttt{dedth} simply turns a hypothesis into an antecedent
(i.e. the hypothesis followed by $\rightarrow$
is placed in front of the assertion, and the hypothesis
itself is eliminated) given certain conditions.

The Weak Deduction Theorem
eliminates a hypothesis $\varphi$, making it become an antecedent.
It does this by proving an expression
$ \varphi \rightarrow \psi $ given two hypotheses:
(1)
$ ( A = {\rm if} ( \varphi , A , B ) \rightarrow ( \varphi \leftrightarrow \chi ) ) $
and
(2) $\chi$.
Note that it requires that a proof exists for $\varphi$ when the class variable
$A$ is replaced with a specific class $B$. The hypothesis $\chi$
should be assigned to the inference.
You can see the details of the proof of the Weak Deduction Theorem
in theorem \texttt{dedth}.

The Weak Deduction Theorem
is probably easier to understand by studying proofs that make use of it.
For example, let's look at the proof of \texttt{renegcl}, which proves that
$ \vdash ( A \in \mathbb{R} \rightarrow - A \in \mathbb{R} )$:

\needspace{4\baselineskip}
\begin{longtabu} {l l l X}
\textbf{Step} & \textbf{Hyp} & \textbf{Ref} & \textbf{Expression} \\
  1 &  & negeq &
  $\vdash$ $($ $A$ $=$ ${\rm if}$ $($ $A$ $\in$ $\mathbb{R}$ $,$ $A$ $,$ $1$ $)$ $\rightarrow$
  $\textrm{-}$ $A$ $=$ $\textrm{-}$ ${\rm if}$ $($ $A$ $\in$ $\mathbb{R}$
  $,$ $A$ $,$ $1$ $)$ $)$ \\
 2 & 1 & eleq1d &
    $\vdash$ $($ $A$ $=$ ${\rm if}$ $($ $A$ $\in$ $\mathbb{R}$ $,$ $A$ $,$ $1$ $)$ $\rightarrow$ $($
    $\textrm{-}$ $A$ $\in$ $\mathbb{R}$ $\leftrightarrow$
    $\textrm{-}$ ${\rm if}$ $($ $A$ $\in$ $\mathbb{R}$ $,$ $A$ $,$ $1$ $)$ $\in$
    $\mathbb{R}$ $)$ $)$ \\
 3 &  & 1re & $\vdash 1 \in \mathbb{R}$ \\
 4 & 3 & elimel &
   $\vdash {\rm if} ( A \in \mathbb{R} , A , 1 ) \in \mathbb{R}$ \\
 5 & 4 & renegcli &
   $\vdash \textrm{-} {\rm if} ( A \in \mathbb{R} , A , 1 ) \in \mathbb{R}$ \\
 6 & 2,5 & dedth &
   $\vdash ( A \in \mathbb{R} \rightarrow \textrm{-} A \in \mathbb{R}$ ) \\
\end{longtabu}

The somewhat strange-looking steps in \texttt{renegcl} before step 5 are
technical stuff that makes this magic work, and they can be ignored
for a quick overview of the proof. To continue following the ``important''
part of the proof of \texttt{renegcl},
you can look at the reference to \texttt{renegcli} at step 5.

That said, let's briefly look at how
\texttt{renegcl} uses the
Weak Deduction Theorem (\texttt{dedth}) to do its job,
in case you want to do something similar or want understand it more deeply.
Let's work backwards in the proof of \texttt{renegcl}.
Step 6 applies \texttt{dedth} to produce our goal result
$ \vdash ( A \in \mathbb{R} \rightarrow\, - A \in \mathbb{R} )$.
This requires on the one hand the (substituted) deduction
\texttt{renegcli} in step 5.
By itself \texttt{renegcli} proves the deduction
$ \vdash A \in \mathbb{R} \Rightarrow\, \vdash - A \in \mathbb{R}$;
this is the deduction form we are trying to turn into theorem form,
and thus
\texttt{renegcli} has a separate hypothesis that must be fulfilled.
To fulfill the hypothesis of the invocation of
\texttt{renegcli} in step 5, it is eventually
reduced to the already proven theorem $1 \in \mathbb{R}$ in step 3.
Step 4 connects steps 3 and 5; step 4 invokes
\texttt{elimel}, a special case of \texttt{elimhyp} that eliminates
a membership hypothesis for the weak deduction theorem.
On the other hand, the equivalence of the conclusion of
\texttt{renegcl}
$( - A \in \mathbb{R} )$ and the substituted conclusion of
\texttt{renegcli} must be proven, which is done in steps 2 and 1.

The weak deduction theorem has limitations.
In particular, we must be able to prove a special case of the deduction's
hypothesis as a stand-alone theorem.
For example, we used $1 \in \mathbb{R}$ in step 3 of \texttt{renegcl}.

We used to use the weak deduction theorem
extensively within \texttt{set.mm}.
However, we now recommend applying ``deduction style''
instead in most cases, as deduction style is
often an easier and clearer approach.
Therefore, we will now describe deduction style.

\subsection{Deduction Style}\label{deductionstyle}

We now prefer to write assertions in ``deduction form''
instead of writing a proof that would require use of the standard or
weak deduction theorem.
We call this appraoch
``deduction style.''\index{deduction style}

It will be easier to explain this by first defining some terms:

\begin{itemize}
\item \textbf{closed form}\index{closed form}\index{forms!closed}:
A kind of assertion (theorem) with no hypotheses.
Typically its label has no special suffix.
An example is \texttt{unss}, which states:
$\vdash ( ( A \subseteq C \wedge B \subseteq C ) \leftrightarrow ( A \cup B )
\subseteq C )\label{eq:unss}$
\item \textbf{deduction form}\index{deduction form}\index{forms!deduction}:
A kind of assertion with one or more hypotheses
where the conclusion is an implication with
a wff variable as the antecedent (usually $\varphi$), and every hypothesis
(\$e statement)
is either (1) an implication with the same antecedent as the conclusion or
(2) a definition.
A definition
can be for a class variable (this is a class variable followed by ``='')
or a wff variable (this is a wff variable followed by $\leftrightarrow$);
class variable definitions are more common.
In practice, a proof
in deduction form will also contain many steps that are implications
where the antecedent is either that wff variable (normally $\varphi$)
or is
a conjunction (...$\land$...) including that wff variable ($\varphi$).
If an assertion is in deduction form, and other forms are also available,
then we suffix its label with ``d.''
An example is \texttt{unssd}, which states\footnote{
For brevity we show here (and in other places)
a $\&$\index{$\&$} between hypotheses\index{hypotheses}
and a $\Rightarrow$\index{$\Rightarrow$}\index{conclusion}
between the hypotheses and the conclusion.
This notation is technically not part of the Metamath language, but is
instead a convenient abbreviation to show both the hypotheses and conclusion.}:
$\vdash ( \varphi \rightarrow A \subseteq C )\quad\&\quad \vdash ( \varphi
    \rightarrow B \subseteq C )\quad\Rightarrow\quad \vdash ( \varphi
    \rightarrow ( A \cup B ) \subseteq C )\label{eq:unssd}$
\item \textbf{inference form}\index{inference form}\index{forms!inference}:
A kind of assertion with one or more hypotheses that is not in deduction form
(e.g., there is no common antecedent).
If an assertion is in inference form, and other forms are also available,
then we suffix its label with ``i.''
An example is \texttt{unssi}, which states:
$\vdash A \subseteq C\quad\&\quad \vdash B \subseteq C\quad\Rightarrow\quad
    \vdash ( A \cup B ) \subseteq C\label{eq:unssi}$
\end{itemize}

When using deduction style we express an assertion in deduction form.
This form prefixes each hypothesis (other than definitions) and the
conclusion with a universal antecedent (``$\varphi \rightarrow$'').
The antecedent (e.g., $\varphi$)
mimics the context handled in the deduction theorem, eliminating
the need to directly use the deduction theorem.

Once you have an assertion in deduction form, you can easily convert it
to inference form or closed form:

\begin{itemize}
\item To
prove some assertion Ti in inference form, given assertion Td in deduction
form, there is a simple mechanical process you can use. First take each
Ti hypothesis and insert a \texttt{T.} $\rightarrow$ prefix (``true implies'')
using \texttt{a1i}. You
can then use the existing assertion Td to prove the resulting conclusion
with a \texttt{T.} $\rightarrow$ prefix.
Finally, you can remove that prefix using \texttt{trud},
resulting in the conclusion you wanted to prove.
\item To
prove some assertion T in closed form, given assertion Td in deduction
form, there is another simple mechanical process you can use. First,
select an expression that is the conjunction (...$\land$...) of all of the
consequents of every hypothesis of Td. Next, prove that this expression
implies each of the separate hypotheses of Td in turn by eliminating
conjuncts (there are a variety of proven assertions to do this, including
\texttt{simpl},
\texttt{simpr},
\texttt{3simpa},
\texttt{3simpb},
\texttt{3simpc},
\texttt{simp1},
\texttt{simp2},
and
\texttt{simp3}).
If the
expression has nested conjunctions, inner conjuncts can be broken out by
chaining the above theorems with \texttt{syl}
(see section \ref{syl}).\footnote{
There are actually many theorems
(labeled simp* such as \texttt{simp333}) that break out inner conjuncts in one
step, but rather than learning them you can just use the chaining we
just described to prove them, and then let the Metamath program command
\texttt{minimize{\char`\_}with}\index{\texttt{minimize{\char`\_}with} command}
figure out the right ones needed to collapse them.}
As your final step, you can then apply the already-proven assertion Td
(which is in deduction form), proving assertion T in closed form.
\end{itemize}

We can also easily convert any assertion T in closed form to its related
assertion Ti in inference form by applying
modus ponens\index{modus ponens} (see section \ref{axmp}).

The deduction form antecedent can also be used to represent the context
necessary to support natural deduction systems, so we will now
discuss natural deduction.

\subsection{Natural Deduction}\label{naturaldeduction}

Natural deduction\index{natural deduction}
(ND) systems, as such, were originally introduced in
1934 by two logicians working independently: Ja\'skowski and Gentzen. ND
systems are supposed to reconstruct, in a formally proper way, traditional
ways of mathematical reasoning (such as conditional proof, indirect proof,
and proof by cases). As reconstructions they were naturally influenced
by previous work, and many specific ND systems and notations have been
developed since their original work.

There are many ND variants, but
Indrzejczak \cite[p.~31-32]{Indrzejczak}\index{Indrzejczak, Andrzej}
suggests that any natural deductive system must satisfy at
least these three criteria:

\begin{itemize}
\item ``There are some means for entering assumptions into a proof and
also for eliminating them. Usually it requires some bookkeeping devices
for indicating the scope of an assumption, and showing that a part of
a proof depending on eliminated assumption is discharged.
\item There are no (or, at least, a very limited set of) axioms, because
their role is taken over by the set of primitive rules for introduction
and elimination of logical constants which means that elementary
inferences instead of formulae are taken as primitive.
\item (A genuine) ND system admits a lot of freedom in proof construction
and possibility of applying several proof search strategies, like
conditional proof, proof by cases, proof by reductio ad absurdum etc.''
\end{itemize}

The Metamath Proof Explorer (MPE) as defined in \texttt{set.mm}
is fundamentally a Hilbert-style system.
That is, MPE is based on a larger number of axioms (compared
to natural deduction systems), a very small set of rules of inference
(modus ponens), and the context is not changed by the rules of inference
in the middle of a proof. That said, MPE proofs can be developed using
the natural deduction (ND) approach as originally developed by Ja\'skowski
and Gentzen.

The most common and recommended approach for applying ND in MPE is to use
deduction form\index{deduction form}%
\index{forms!deduction}
and apply the MPE proven assertions that are equivalent to ND rules.
For example, MPE's \texttt{jca} is equivalent to ND rule $\land$-I
(and-insertion).
We maintain a list of equivalences that you may consult.
This approach for applying an ND approach within MPE relies on Metamath's
wff metavariables in an essential way, and is described in more detail
in the presentation ``Natural Deductions in the Metamath Proof Language''
by Mario Carneiro \cite{CarneiroND}\index{Carneiro, Mario}.

In this style many steps are an implication, whose antecedent mimics
the context ($\Gamma$) of most ND systems. To add an assumption, simply add
it to the implication antecedent (typically using
\texttt{simpr}),
and use that
new antecedent for all later claims in the same scope. If you wish to
use an assertion in an ND hypothesis scope that is outside the current
ND hypothesis scope, modify the assertion so that the ND hypothesis
assumption is added to its antecedent (typically using \texttt{adantr}). Most
proof steps will be proved using rules that have hypotheses and results
of the form $\varphi \rightarrow$ ...

An example may make this clearer.
Let's show theorem 5.5 of
\cite[p.~18]{Clemente}\index{Clemente Laboreo, Daniel}
along with a line by line translation using the usual
translation of natural deduction (ND) in the Metamath Proof Explorer
(MPE) notation (this is proof \texttt{ex-natded5.5}).
The proof's original goal was to prove
$\lnot \psi$ given two hypotheses,
$( \psi \rightarrow \chi )$ and $ \lnot \chi$.
We will translate these statements into MPE deduction form
by prefixing them all with $\varphi \rightarrow$.
As a result, in MPE the goal is stated as
$( \varphi \rightarrow \lnot \psi )$, and the two hypotheses are stated as
$( \varphi \rightarrow ( \psi \rightarrow \chi ) )$ and
$( \varphi \rightarrow \lnot \chi )$.

The following table shows the proof in Fitch natural deduction style
and its MPE equivalent.
The \textit{\#} column shows the original numbering,
\textit{MPE\#} shows the number in the equivalent MPE proof
(which we will show later),
\textit{ND Expression} shows the original proof claim in ND notation,
and \textit{MPE Translation} shows its translation into MPE
as discussed in this section.
The final columns show the rationale in ND and MPE respectively.

\needspace{4\baselineskip}
{\setlength{\extrarowsep}{4pt} % Keep rows from being too close together
\begin{longtabu}   { @{} c c X X X X }
\textbf{\#} & \textbf{MPE\#} & \textbf{ND Ex\-pres\-sion} &
\textbf{MPE Trans\-lation} & \textbf{ND Ration\-ale} &
\textbf{MPE Ra\-tio\-nale} \\
\endhead

1 & 2;3 &
$( \psi \rightarrow \chi )$ &
$( \varphi \rightarrow ( \psi \rightarrow \chi ) )$ &
Given &
\$e; \texttt{adantr} to put in ND hypothesis \\

2 & 5 &
$ \lnot \chi$ &
$( \varphi \rightarrow \lnot \chi )$ &
Given &
\$e; \texttt{adantr} to put in ND hypothesis \\

3 & 1 &
... $\vert$ $\psi$ &
$( \varphi \rightarrow \psi )$ &
ND hypothesis assumption &
\texttt{simpr} \\

4 & 4 &
... $\chi$ &
$( ( \varphi \land \psi ) \rightarrow \chi )$ &
$\rightarrow$\,E 1,3 &
\texttt{mpd} 1,3 \\

5 & 6 &
... $\lnot \chi$ &
$( ( \varphi \land \psi ) \rightarrow \lnot \chi )$ &
IT 2 &
\texttt{adantr} 5 \\

6 & 7 &
$\lnot \psi$ &
$( \varphi \rightarrow \lnot \psi )$ &
$\land$\,I 3,4,5 &
\texttt{pm2.65da} 4,6 \\

\end{longtabu}
}


The original used Latin letters; we have replaced them with Greek letters
to follow Metamath naming conventions and so that it is easier to follow
the Metamath translation. The Metamath line-for-line translation of
this natural deduction approach precedes every line with an antecedent
including $\varphi$ and uses the Metamath equivalents of the natural deduction
rules. To add an assumption, the antecedent is modified to include it
(typically by using \texttt{adantr};
\texttt{simpr} is useful when you want to
depend directly on the new assumption, as is shown here).

In Metamath we can represent the two given statements as these hypotheses:

\needspace{2\baselineskip}
\begin{itemize}
\item ex-natded5.5.1 $\vdash ( \varphi \rightarrow ( \psi \rightarrow \chi ) )$
\item ex-natded5.5.2 $\vdash ( \varphi \rightarrow \lnot \chi )$
\end{itemize}

\needspace{4\baselineskip}
Here is the proof in Metamath as a line-by-line translation:

\begin{longtabu}   { l l l X }
\textbf{Step} & \textbf{Hyp} & \textbf{Ref} & \textbf{Ex\-pres\-sion} \\
\endhead
1 & & simpr & $\vdash ( ( \varphi \land \psi ) \rightarrow \psi )$ \\
2 & & ex-natded5.5.1 &
  $\vdash ( \varphi \rightarrow ( \psi \rightarrow \chi ) )$ \\
3 & 2 & adantr &
 $\vdash ( ( \varphi \land \psi ) \rightarrow ( \psi \rightarrow \chi ) )$ \\
4 & 1, 3 & mpd &
 $\vdash ( ( \varphi \land \psi ) \rightarrow \chi ) $ \\
5 & & ex-natded5.5.2 &
 $\vdash ( \varphi \rightarrow \lnot \chi )$ \\
6 & 5 & adantr &
 $\vdash ( ( \varphi \land \psi ) \rightarrow \lnot \chi )$ \\
7 & 4, 6 & pm2.65da &
 $\vdash ( \varphi \rightarrow \lnot \psi )$ \\
\end{longtabu}

Only using specific natural deduction rules directly can lead to very
long proofs, for exactly the same reason that only using axioms directly
in Hilbert-style proofs can lead to very long proofs.
If the goal is short and clear proofs,
then it is better to reuse already-proven assertions
in deduction form than to start from scratch each time
and using only basic natural deduction rules.

\subsection{Strengths of Our Approach}

As far as we know there is nothing else in the literature like either the
weak deduction theorem or Mario Carneiro\index{Carneiro, Mario}'s
natural deduction method.
In order to
transform a hypothesis into an antecedent, the literature's standard
``Deduction Theorem''\index{Deduction Theorem}\index{Standard Deduction Theorem}
requires metalogic outside of the notions provided
by the axiom system. We instead generally prefer to use Mario Carneiro's
natural deduction method, then use the weak deduction theorem in cases
where that is difficult to apply, and only then use the full standard
deduction theorem as a last resort.

The weak deduction theorem\index{Weak Deduction Theorem}
does not require any additional metalogic
but converts an inference directly into a closed form theorem, with
a rigorous proof that uses only the axiom system. Unlike the standard
Deduction Theorem, there is no implicit external justification that we
have to trust in order to use it.

Mario Carneiro's natural deduction\index{natural deduction}
method also does not require any new metalogical
notions. It avoids the Deduction Theorem's metalogic by prefixing the
hypotheses and conclusion of every would-be inference with a universal
antecedent (``$\varphi \rightarrow$'') from the very start.

We think it is impressive and satisfying that we can do so much in a
practical sense without stepping outside of our Hilbert-style axiom system.
Of course our axiomatization, which is in the form of schemes,
contains a metalogic of its own that we exploit. But this metalogic
is relatively simple, and for our Deduction Theorem alternatives,
we primarily use just the direct substitution of expressions for
metavariables.

\begin{sloppy}
\section{Exploring the Set The\-o\-ry Data\-base}\label{exploring}
\end{sloppy}
% NOTE: All examples performed in this section are
% recorded wtih "set width 61" % on set.mm as of 2019-05-28
% commit c1e7849557661260f77cfdf0f97ac4354fbb4f4d.

At this point you may wish to study the \texttt{set.mm}\index{set theory
database (\texttt{set.mm})} file in more detail.  Pay particular
attention to the assumptions needed to define wffs\index{well-formed
formula (wff)} (which are not included above), the variable types
(\texttt{\$f}\index{\texttt{\$f} statement} statements), and the
definitions that are introduced.  Start with some simple theorems in
propositional calculus, making sure you understand in detail each step
of a proof.  Once you get past the first few proofs and become familiar
with the Metamath language, any part of the \texttt{set.mm} database
will be as easy to follow, step by step, as any other part---you won't
have to undergo a ``quantum leap'' in mathematical sophistication to be
able to follow a deep proof in set theory.

Next, you may want to explore how concepts such as natural numbers are
defined and described.  This is probably best done in conjunction with
standard set theory textbooks, which can help give you a higher-level
understanding.  The \texttt{set.mm} database provides references that will get
you started.  From there, you will be on your way towards a very deep,
rigorous understanding of abstract mathematics.

The Metamath\index{Metamath} program can help you peruse a Metamath data\-base,
wheth\-er you are trying to figure out how a certain step follows in a proof or
just have a general curiosity.  We will go through some examples of the
commands, using the \texttt{set.mm}\index{set theory database (\texttt{set.mm})}
database provided with the Metamath software.  These should help get you
started.  See Chapter~\ref{commands} for a more detailed description of
the commands.  Note that we have included the full spelling of all commands to
prevent ambiguity with future commands.  In practice you may type just the
characters needed to specify each command keyword\index{command keyword}
unambiguously, often just one or two characters per keyword, and you don't
need to type them in upper case.

First run the Metamath program as described earlier.  You should see the
\verb/MM>/ prompt.  Read in the \texttt{set.mm} file:\index{\texttt{read}
command}

\begin{verbatim}
MM> read set.mm
Reading source file "set.mm"... 34554442 bytes
34554442 bytes were read into the source buffer.
The source has 155711 statements; 2254 are $a and 32250 are $p.
No errors were found.  However, proofs were not checked.
Type VERIFY PROOF * if you want to check them.
\end{verbatim}

As with most examples in this book, what you will see
will be slightly different because we are continuously
improving our databases (including \texttt{set.mm}).

Let's check the database integrity.  This may take a minute or two to run if
your computer is slow.

\begin{verbatim}
MM> verify proof *
0 10%  20%  30%  40%  50%  60%  70%  80%  90% 100%
..................................................
All proofs in the database were verified in 2.84 s.
\end{verbatim}

No errors were reported, so every proof is correct.

You need to know the names (labels) of theorems before you can look at them.
Often just examining the database file(s) with a text editor is the best
approach.  In \texttt{set.mm} there are many detailed comments, especially near
the beginning, that can help guide you. The \texttt{search} command in the
Metamath program is also handy.  The \texttt{comments} qualifier will list the
statements whose associated comment (the one immediately before it) contain a
string you give it.  For example, if you are studying Enderton's {\em Elements
of Set Theory} \cite{Enderton}\index{Enderton, Herbert B.} you may want to see
the references to it in the database.  The search string \texttt{enderton} is not
case sensitive.  (This will not show you all the database theorems that are in
Enderton's book because there is usually only one citation for a given
theorem, which may appear in several textbooks.)\index{\texttt{search}
command}

\begin{verbatim}
MM> search * "enderton" / comments
12067 unineq $p "... Exercise 20 of [Enderton] p. 32 and ..."
12459 undif2 $p "...Corollary 6K of [Enderton] p. 144. (C..."
12953 df-tp $a "...s. Definition of [Enderton] p. 19. (Co..."
13689 unissb $p ".... Exercise 5 of [Enderton] p. 26 and ..."
\end{verbatim}
\begin{center}
(etc.)
\end{center}

Or you may want to see what theorems have something to do with
conjunction (logical {\sc and}).  The quotes around the search
string are optional when there's no ambiguity.\index{\texttt{search}
command}

\begin{verbatim}
MM> search * conjunction / comments
120 a1d $p "...be replaced with a conjunction ( ~ df-an )..."
662 df-bi $a "...viated form after conjunction is introdu..."
1319 wa $a "...ff definition to include conjunction ('and')."
1321 df-an $a "Define conjunction (logical 'and'). Defini..."
1420 imnan $p "...tion in terms of conjunction. (Contribu..."
\end{verbatim}
\begin{center}
(etc.)
\end{center}


Now we will start to look at some details.  Let's look at the first
axiom of propositional calculus
(we could use \texttt{sh st} to abbreviate
\texttt{show statement}).\index{\texttt{show statement} command}

\begin{verbatim}
MM> show statement ax-1/full
Statement 19 is located on line 881 of the file "set.mm".
"Axiom _Simp_.  Axiom A1 of [Margaris] p. 49.  One of the 3
axioms of propositional calculus.  The 3 axioms are also
        ...
19 ax-1 $a |- ( ph -> ( ps -> ph ) ) $.
Its mandatory hypotheses in RPN order are:
  wph $f wff ph $.
  wps $f wff ps $.
The statement and its hypotheses require the variables:  ph
      ps
The variables it contains are:  ph ps


Statement 49 is located on line 11182 of the file "set.mm".
Its statement number for HTML pages is 6.
"Axiom _Simp_.  Axiom A1 of [Margaris] p. 49.  One of the 3
axioms of propositional calculus.  The 3 axioms are also
given as Definition 2.1 of [Hamilton] p. 28.
...
49 ax-1 $a |- ( ph -> ( ps -> ph ) ) $.
Its mandatory hypotheses in RPN order are:
  wph $f wff ph $.
  wps $f wff ps $.
The statement and its hypotheses require the variables:
  ph ps
The variables it contains are:  ph ps
\end{verbatim}

Compare this to \texttt{ax-1} on p.~\pageref{ax1}.  You can see that the
symbol \texttt{ph} is the {\sc ascii} notation for $\varphi$, etc.  To
see the mathematical symbols for any expression you may typeset it in
\LaTeX\ (type \texttt{help tex} for instructions)\index{latex@{\LaTeX}}
or, easier, just use a text editor to look at the comments where symbols
are first introduced in \texttt{set.mm}.  The hypotheses \texttt{wph}
and \texttt{wps} required by \texttt{ax-1} mean that variables
\texttt{ph} and \texttt{ps} must be wffs.

Next we'll pick a simple theorem of propositional calculus, the Principle of
Identity, which is proved directly from the axioms.  We'll look at the
statement then its proof.\index{\texttt{show statement}
command}

\begin{verbatim}
MM> show statement id1/full
Statement 116 is located on line 11371 of the file "set.mm".
Its statement number for HTML pages is 22.
"Principle of identity.  Theorem *2.08 of [WhiteheadRussell]
p. 101.  This version is proved directly from the axioms for
demonstration purposes.
...
116 id1 $p |- ( ph -> ph ) $= ... $.
Its mandatory hypotheses in RPN order are:
  wph $f wff ph $.
Its optional hypotheses are:  wps wch wth wta wet
      wze wsi wrh wmu wla wka
The statement and its hypotheses require the variables:  ph
These additional variables are allowed in its proof:
      ps ch th ta et ze si rh mu la ka
The variables it contains are:  ph
\end{verbatim}

The optional variables\index{optional variable} \texttt{ps}, \texttt{ch}, etc.\ are
available for use in a proof of this statement if we wish, and were we to do
so we would make use of optional hypotheses \texttt{wps}, \texttt{wch}, etc.  (See
Section~\ref{dollaref} for the meaning of ``optional
hypothesis.''\index{optional hypothesis}) The reason these show up in the
statement display is that statement \texttt{id1} happens to be in their scope
(see Section~\ref{scoping} for the definition of ``scope''\index{scope}), but
in fact in propositional calculus we will never make use of optional
hypotheses or variables.  This becomes important after quantifiers are
introduced, where ``dummy'' variables\index{dummy variable} are often needed
in the middle of a proof.

Let's look at the proof of statement \texttt{id1}.  We'll use the
\texttt{show proof} command, which by default suppresses the
``non-essential'' steps that construct the wffs.\index{\texttt{show proof}
command}
We will display the proof in ``lemmon' format (a non-indented format
with explicit previous step number references) and renumber the
displayed steps:

\begin{verbatim}
MM> show proof id1 /lemmon/renumber
1 ax-1           $a |- ( ph -> ( ph -> ph ) )
2 ax-1           $a |- ( ph -> ( ( ph -> ph ) -> ph ) )
3 ax-2           $a |- ( ( ph -> ( ( ph -> ph ) -> ph ) ) ->
                     ( ( ph -> ( ph -> ph ) ) -> ( ph -> ph )
                                                          ) )
4 2,3 ax-mp      $a |- ( ( ph -> ( ph -> ph ) ) -> ( ph -> ph
                                                          ) )
5 1,4 ax-mp      $a |- ( ph -> ph )
\end{verbatim}

If you have read Section~\ref{trialrun}, you'll know how to interpret this
proof.  Step~2, for example, is an application of axiom \texttt{ax-1}.  This
proof is identical to the one in Hamilton's {\em Logic for Mathematicians}
\cite[p.~32]{Hamilton}\index{Hamilton, Alan G.}.

You may want to look at what
substitutions\index{substitution!variable}\index{variable substitution} are
made into \texttt{ax-1} to arrive at step~2.  The command to do this needs to
know the ``real'' step number, so we'll display the proof again without
the \texttt{renumber} qualifier.\index{\texttt{show proof}
command}

\begin{verbatim}
MM> show proof id1 /lemmon
 9 ax-1          $a |- ( ph -> ( ph -> ph ) )
20 ax-1          $a |- ( ph -> ( ( ph -> ph ) -> ph ) )
24 ax-2          $a |- ( ( ph -> ( ( ph -> ph ) -> ph ) ) ->
                     ( ( ph -> ( ph -> ph ) ) -> ( ph -> ph )
                                                          ) )
25 20,24 ax-mp   $a |- ( ( ph -> ( ph -> ph ) ) -> ( ph -> ph
                                                          ) )
26 9,25 ax-mp    $a |- ( ph -> ph )
\end{verbatim}

The ``real'' step number is 20.  Let's look at its details.

\begin{verbatim}
MM> show proof id1 /detailed_step 20
Proof step 20:  min=ax-1 $a |- ( ph -> ( ( ph -> ph ) -> ph )
  )
This step assigns source "ax-1" ($a) to target "min" ($e).
The source assertion requires the hypotheses "wph" ($f, step
18) and "wps" ($f, step 19).  The parent assertion of the
target hypothesis is "ax-mp" ($a, step 25).
The source assertion before substitution was:
    ax-1 $a |- ( ph -> ( ps -> ph ) )
The following substitutions were made to the source
assertion:
    Variable  Substituted with
     ph        ph
     ps        ( ph -> ph )
The target hypothesis before substitution was:
    min $e |- ph
The following substitution was made to the target hypothesis:
    Variable  Substituted with
     ph        ( ph -> ( ( ph -> ph ) -> ph ) )
\end{verbatim}

This shows the substitutions\index{substitution!variable}\index{variable
substitution} made to the variables in \texttt{ax-1}.  References are made to
steps 18 and 19 which are not shown in our proof display.  To see these steps,
you can display the proof with the \texttt{all} qualifier.

Let's look at a slightly more advanced proof of propositional calculus.  Note
that \verb+/\+ is the symbol for $\wedge$ (logical {\sc and}, also
called conjunction).\index{conjunction ($\wedge$)}
\index{logical {\sc and} ($\wedge$)}

\begin{verbatim}
MM> show statement prth/full
Statement 1791 is located on line 15503 of the file "set.mm".
Its statement number for HTML pages is 559.
"Conjoin antecedents and consequents of two premises.  This
is the closed theorem form of ~ anim12d .  Theorem *3.47 of
[WhiteheadRussell] p. 113.  It was proved by Leibniz,
and it evidently pleased him enough to call it
_praeclarum theorema_ (splendid theorem).
...
1791 prth $p |- ( ( ( ph -> ps ) /\ ( ch -> th ) ) -> ( ( ph
      /\ ch ) -> ( ps /\ th ) ) ) $= ... $.
Its mandatory hypotheses in RPN order are:
  wph $f wff ph $.
  wps $f wff ps $.
  wch $f wff ch $.
  wth $f wff th $.
Its optional hypotheses are:  wta wet wze wsi wrh wmu wla wka
The statement and its hypotheses require the variables:  ph
      ps ch th
These additional variables are allowed in its proof:  ta et
      ze si rh mu la ka
The variables it contains are:  ph ps ch th


MM> show proof prth /lemmon/renumber
1 simpl          $p |- ( ( ( ph -> ps ) /\ ( ch -> th ) ) ->
                                               ( ph -> ps ) )
2 simpr          $p |- ( ( ( ph -> ps ) /\ ( ch -> th ) ) ->
                                               ( ch -> th ) )
3 1,2 anim12d    $p |- ( ( ( ph -> ps ) /\ ( ch -> th ) ) ->
                           ( ( ph /\ ch ) -> ( ps /\ th ) ) )
\end{verbatim}

There are references to a lot of unfamiliar statements.  To see what they are,
you may type the following:

\begin{verbatim}
MM> show proof prth /statement_summary
Summary of statements used in the proof of "prth":

Statement simpl is located on line 14748 of the file
"set.mm".
"Elimination of a conjunct.  Theorem *3.26 (Simp) of
[WhiteheadRussell] p. 112. ..."
  simpl $p |- ( ( ph /\ ps ) -> ph ) $= ... $.

Statement simpr is located on line 14777 of the file
"set.mm".
"Elimination of a conjunct.  Theorem *3.27 (Simp) of
[WhiteheadRussell] ..."
  simpr $p |- ( ( ph /\ ps ) -> ps ) $= ... $.

Statement anim12d is located on line 15445 of the file
"set.mm".
"Conjoin antecedents and consequents in a deduction.
..."
  anim12d.1 $e |- ( ph -> ( ps -> ch ) ) $.
  anim12d.2 $e |- ( ph -> ( th -> ta ) ) $.
  anim12d $p |- ( ph -> ( ( ps /\ th ) -> ( ch /\ ta ) ) )
      $= ... $.
\end{verbatim}
\begin{center}
(etc.)
\end{center}

Of course you can look at each of these statements and their proofs, and
so on, back to the axioms of propositional calculus if you wish.

The \texttt{search} command is useful for finding statements when you
know all or part of their contents.  The following example finds all
statements containing \verb@ph -> ps@ followed by \verb@ch -> th@.  The
\verb@$*@ is a wildcard that matches anything; the \texttt{\$} before the
\verb$*$ prevents conflicts with math symbol token names.  The \verb@*@ after
\texttt{SEARCH} is also a wildcard that in this case means ``match any label.''
\index{\texttt{search} command}

% I'm omitting this one, since readers are unlikely to see it:
% 1096 bisymOLD $p |- ( ( ( ph -> ps ) -> ( ch -> th ) ) -> ( (
%   ( ps -> ph ) -> ( th -> ch ) ) -> ( ( ph <-> ps ) -> ( ch
%    <-> th ) ) ) )
\begin{verbatim}
MM> search * "ph -> ps $* ch -> th"
1791 prth $p |- ( ( ( ph -> ps ) /\ ( ch -> th ) ) -> ( ( ph
    /\ ch ) -> ( ps /\ th ) ) )
2455 pm3.48 $p |- ( ( ( ph -> ps ) /\ ( ch -> th ) ) -> ( (
    ph \/ ch ) -> ( ps \/ th ) ) )
117859 pm11.71 $p |- ( ( E. x ph /\ E. y ch ) -> ( ( A. x (
    ph -> ps ) /\ A. y ( ch -> th ) ) <-> A. x A. y ( ( ph /\
    ch ) -> ( ps /\ th ) ) ) )
\end{verbatim}

Three statements, \texttt{prth}, \texttt{pm3.48},
 and \texttt{pm11.71}, were found to match.

To see what axioms\index{axiom} and definitions\index{definition}
\texttt{prth} ultimately depends on for its proof, you can have the
program backtrack through the hierarchy\index{hierarchy} of theorems and
definitions.\index{\texttt{show trace{\char`\_}back} command}

\begin{verbatim}
MM> show trace_back prth /essential/axioms
Statement "prth" assumes the following axioms ($a
statements):
  ax-1 ax-2 ax-3 ax-mp df-bi df-an
\end{verbatim}

Note that the 3 axioms of propositional calculus and the modus ponens rule are
needed (as expected); in addition, there are a couple of definitions that are used
along the way.  Note that Metamath makes no distinction\index{axiom vs.\
definition} between axioms\index{axiom} and definitions\index{definition}.  In
\texttt{set.mm} they have been distinguished artificially by prefixing their
labels\index{labels in \texttt{set.mm}} with \texttt{ax-} and \texttt{df-}
respectively.  For example, \texttt{df-an} defines conjunction (logical {\sc
and}), which is represented by the symbol \verb+/\+.
Section~\ref{definitions} discusses the philosophy of definitions, and the
Metamath language takes a particularly simple, conservative approach by using
the \texttt{\$a}\index{\texttt{\$a} statement} statement for both axioms and
definitions.

You can also have the program compute how many steps a proof
has\index{proof length} if we were to follow it all the way back to
\texttt{\$a} statements.

\begin{verbatim}
MM> show trace_back prth /essential/count_steps
The statement's actual proof has 3 steps.  Backtracking, a
total of 79 different subtheorems are used.  The statement
and subtheorems have a total of 274 actual steps.  If
subtheorems used only once were eliminated, there would be a
total of 38 subtheorems, and the statement and subtheorems
would have a total of 185 steps.  The proof would have 28349
steps if fully expanded back to axiom references.  The
maximum path length is 38.  A longest path is:  prth <-
anim12d <- syl2and <- sylan2d <- ancomsd <- ancom <- pm3.22
<- pm3.21 <- pm3.2 <- ex <- sylbir <- biimpri <- bicomi <-
bicom1 <- bi2 <- dfbi1 <- impbii <- bi3 <- simprim <- impi <-
con1i <- nsyl2 <- mt3d <- con1d <- notnot1 <- con2i <- nsyl3
<- mt2d <- con2d <- notnot2 <- pm2.18d <- pm2.18 <- pm2.21 <-
pm2.21d <- a1d <- syl <- mpd <- a2i <- a2i.1 .
\end{verbatim}

This tells us that we would have to inspect 274 steps if we want to
verify the proof completely starting from the axioms.  A few more
statistics are also shown.  There are one or more paths back to axioms
that are the longest; this command ferrets out one of them and shows it
to you.  There may be a sense in which the longest path length is
related to how ``deep'' the theorem is.

We might also be curious about what proofs depend on the theorem
\texttt{prth}.  If it is never used later on, we could eliminate it as
redundant if it has no intrinsic interest by itself.\index{\texttt{show
usage} command}

% I decided to show the OLD values here.
\begin{verbatim}
MM> show usage prth
Statement "prth" is directly referenced in the proofs of 18
statements:
  mo3 moOLD 2mo 2moOLD euind reuind reuss2 reusv3i opelopabt
  wemaplem2 rexanre rlimcn2 o1of2 o1rlimmul 2sqlem6 spanuni
  heicant pm11.71
\end{verbatim}

Thus \texttt{prth} is directly used by 18 proofs.
We can use the \texttt{/recursive} qualifier to include indirect use:

\begin{verbatim}
MM> show usage prth /recursive
Statement "prth" directly or indirectly affects the proofs of
24214 statements:
  mo3 mo mo3OLD eu2 moOLD eu2OLD eu3OLD mo4f mo4 eu4 mopick
...
\end{verbatim}

\subsection{A Note on the ``Compact'' Proof Format}

The present version of Metamath\index{Metamath!limitations of
version 0.07.30} will display proofs in a ``compact''\index{compact proof}
format whenever the proof is stored in compressed format in the database.  It
may be be slightly confusing unless you know how to interpret it.  (A
future version may eliminate this format from displays.)  For example,
if you display the complete proof of theorem \texttt{id1} it will start
off as follows:

\begin{verbatim}
MM> show proof id1 /lemmon/all
 1 wph           $f wff ph
 2 wph           $f wff ph
 3 wph           $f wff ph
 4 2,3 wi    @4: $a wff ( ph -> ph )
 5 1,4 wi    @5: $a wff ( ph -> ( ph -> ph ) )
 6 @4            $a wff ( ph -> ph )
\end{verbatim}

\begin{center}
{etc.}
\end{center}

Step 4 has a ``local label,''\index{local label} \texttt{@4}, assigned to it.
Later on, at step 6, the label \texttt{@1} is referenced instead of
displaying the explicit proof for that step.  This technique takes advantage
of the fact that steps in a proof often repeat, especially during the
construction of wffs.  The compact format reduces the number of steps in the
proof display and may actually be preferred by some people, which is one
reason that it is still displayed by the program.

If you want to see the normal format with the ``true'' step numbers, you can
use the following workaround:\index{\texttt{save proof} command}

\begin{verbatim}
MM> save proof id1 /normal
The proof of "id1" has been reformatted and saved internally.
Remember to use WRITE SOURCE to save it permanently.
MM> show proof id1 /lemmon/all
 1 wph           $f wff ph
 2 wph           $f wff ph
 3 wph           $f wff ph
 4 2,3 wi        $a wff ( ph -> ph )
 5 1,4 wi        $a wff ( ph -> ( ph -> ph ) )
 6 wph           $f wff ph
 7 wph           $f wff ph
 8 6,7 wi        $a wff ( ph -> ph )
\end{verbatim}

\begin{center}
{etc.}
\end{center}

Note that the original 6 steps are now 8 steps.  However, the format is
now the same as that described in Chapter~\ref{using}.

\chapter{The Metamath Language}
\label{languagespec}

\begin{quote}
  {\em Thus mathematics may be defined as the subject in which we never know
what we are talking about, nor whether what we are saying is true.}
    \flushright\sc  Bertrand Russell\footnote{\cite[p.~84]{Russell2}.}\\
\end{quote}\index{Russell, Bertrand}

Probably the most striking feature of the Metamath language is its almost
complete absence of hard-wired syntax. Metamath\index{Metamath} does not
understand any mathematics or logic other than that needed to construct finite
sequences of symbols according to a small set of simple, built-in rules.  The
only rule it uses in a proof is the substitution of an expression (symbol
sequence) for a variable, subject to a simple constraint to prevent
bound-variable clashes.  The primitive notions built into Metamath involve the
simple manipulation of finite objects (symbols) that we as humans can easily
visualize and that computers can easily deal with.  They seem to be just
about the simplest notions possible that are required to do standard
mathematics.

This chapter serves as a reference manual for the Metamath\index{Metamath}
language. It covers the tedious technical details of the language, some of
which you may wish to skim in a first reading.  On the other hand, you should
pay close attention to the defined terms in {\bf boldface}; they have precise
meanings that are important to keep in mind for later understanding.  It may
be best to first become familiar with the examples in Chapter~\ref{using} to
gain some motivation for the language.

%% Uncomment this when uncommenting section {formalspec} below
If you have some knowledge of set theory, you may wish to study this
chapter in conjunction with the formal set-theoretical description of the
Metamath language in Appendix~\ref{formalspec}.

We will use the name ``Metamath''\index{Metamath} to mean either the Metamath
computer language or the Metamath software associated with the computer
language.  We will not distinguish these two when the context is clear.

The next section contains the complete specification of the Metamath
language.
It serves as an
authoritative reference and presents the syntax in enough detail to
write a parser\index{parsing Metamath} and proof verifier.  The
specification is terse and it is probably hard to learn the language
directly from it, but we include it here for those impatient people who
prefer to see everything up front before looking at verbose expository
material.  Later sections explain this material and provide examples.
We will repeat the definitions in those sections, and you may skip the
next section at first reading and proceed to Section~\ref{tut1}
(p.~\pageref{tut1}).

\section{Specification of the Metamath Language}\label{spec}
\index{Metamath!specification}

\begin{quote}
  {\em Sometimes one has to say difficult things, but one ought to say
them as simply as one knows how.}
    \flushright\sc  G. H. Hardy\footnote{As quoted in
    \cite{deMillo}, p.~273.}\\
\end{quote}\index{Hardy, G. H.}

\subsection{Preliminaries}\label{spec1}

% Space is technically a printable character, so we'll word things
% carefully so it's unambiguous.
A Metamath {\bf database}\index{database} is built up from a top-level source
file together with any source files that are brought in through file inclusion
commands (see below).  The only characters that are allowed to appear in a
Metamath source file are the 94 non-whitespace printable {\sc
ascii}\index{ascii@{\sc ascii}} characters, which are digits, upper and lower
case letters, and the following 32 special
characters\index{special characters}:\label{spec1chars}

\begin{verbatim}
! " # $ % & ' ( ) * + , - . / :
; < = > ? @ [ \ ] ^ _ ` { | } ~
\end{verbatim}
plus the following characters which are the ``white space'' characters:
space (a printable character),
tab, carriage return, line feed, and form feed.\label{whitespace}
We will use \texttt{typewriter}
font to display the printable characters.

A Metamath database consists of a sequence of three kinds of {\bf
tokens}\index{token} separated by {\bf white space}\index{white space}
(which is any sequence of one or more white space characters).  The set
of {\bf keyword}\index{keyword} tokens is \texttt{\$\char`\{},
\texttt{\$\char`\}}, \texttt{\$c}, \texttt{\$v}, \texttt{\$f},
\texttt{\$e}, \texttt{\$d}, \texttt{\$a}, \texttt{\$p}, \texttt{\$.},
\texttt{\$=}, \texttt{\$(}, \texttt{\$)}, \texttt{\$[}, and
\texttt{\$]}.  The last four are called {\bf auxiliary}\index{auxiliary
keyword} or preprocessing keywords.  A {\bf label}\index{label} token
consists of any combination of letters, digits, and the characters
hyphen, underscore, and period.  A {\bf math symbol}\index{math symbol}
token may consist of any combination of the 92 printable standard {\sc
ascii} characters other than space or \texttt{\$}~. All tokens are
case-sensitive.

\subsection{Preprocessing}

The token \texttt{\$(} begins a {\bf comment} and
\texttt{\$)} ends a comment.\index{\texttt{\$(}
and \texttt{\$)} auxiliary keywords}\index{comment}
Comments may contain any of
the 94 non-whitespace printable characters and white space,
except they may not contain the
2-character sequences \texttt{\$(} or \texttt{\$)} (comments do not nest).
Comments are ignored (treated
like white space) for the purpose of parsing, e.g.,
\texttt{\$( \$[ \$)} is a comment.
See p.~\pageref{mathcomments} for comment typesetting conventions; these
conventions may be ignored for the purpose of parsing.

A {\bf file inclusion command} consists of \texttt{\$[} followed by a file name
followed by \texttt{\$]}.\index{\texttt{\$[} and \texttt{\$]} auxiliary
keywords}\index{included file}\index{file inclusion}
It is only allowed in the outermost scope (i.e., not between
\texttt{\$\char`\{} and \texttt{\$\char`\}})
and must not be inside a statement (e.g., it may not occur
between the label of a \texttt{\$a} statement and its \texttt{\$.}).
The file name may not
contain a \texttt{\$} or white space.  The file must exist.
The case-sensitivity
of its name follows the conventions of the operating system.  The contents of
the file replace the inclusion command.
Included files may include other files.
Only the first reference to a given file is included; any later
references to the same file (whether in the top-level file or in included
files) cause the inclusion command to be ignored (treated like white space).
A verifier may assume that file names with different strings
refer to different files for the purpose of ignoring later references.
A file self-reference is ignored, as is any reference to the top-level file
(to avoid loops).
Included files may not include a \texttt{\$(} without a matching \texttt{\$)},
may not include a \texttt{\$[} without a matching \texttt{\$]}, and may
not include incomplete statements (e.g., a \texttt{\$a} without a matching
\texttt{\$.}).
It is currently unspecified if path references are relative to the process'
current directory or the file's containing directory, so databases should
avoid using pathname separators (e.g., ``/'') in file names.

Like all tokens, the \texttt{\$(}, \texttt{\$)}, \texttt{\$[}, and \texttt{\$]} keywords
must be surrounded by white space.

\subsection{Basic Syntax}

After preprocessing, a database will consist of a sequence of {\bf
statements}.
These are the scoping statements \texttt{\$\char`\{} and
\texttt{\$\char`\}}, along with the \texttt{\$c}, \texttt{\$v},
\texttt{\$f}, \texttt{\$e}, \texttt{\$d}, \texttt{\$a}, and \texttt{\$p}
statements.

A {\bf scoping statement}\index{scoping statement} consists only of its
keyword, \texttt{\$\char`\{} or \texttt{\$\char`\}}.
A \texttt{\$\char`\{} begins a {\bf
block}\index{block} and a matching \texttt{\$\char`\}} ends the block.
Every \texttt{\$\char`\{}
must have a matching \texttt{\$\char`\}}.
Defining it recursively, we say a block
contains a sequence of zero or more tokens other
than \texttt{\$\char`\{} and \texttt{\$\char`\}} and
possibly other blocks.  There is an {\bf outermost
block}\index{block!outermost} not bracketed by \texttt{\$\char`\{} \texttt{\$\char`\}}; the end
of the outermost block is the end of the database.

% LaTeX bug? can't do \bf\tt

A {\bf \$v} or {\bf \$c statement}\index{\texttt{\$v} statement}\index{\texttt{\$c}
statement} consists of the keyword token \texttt{\$v} or \texttt{\$c} respectively,
followed by one or more math symbols,
% The word "token" is used to distinguish "$." from the sentence-ending period.
followed by the \texttt{\$.}\ token.
These
statements {\bf declare}\index{declaration} the math symbols to be {\bf
variables}\index{variable!Metamath} or {\bf constants}\index{constant}
respectively. The same math symbol may not occur twice in a given \texttt{\$v} or
\texttt{\$c} statement.

%c%A math symbol becomes an {\bf active}\index{active math symbol}
%c%when declared and stays active until the end of the block in which it is
%c%declared.  A math symbol may not be declared a second time while it is active,
%c%but it may be declared again after it becomes inactive.

A math symbol becomes {\bf active}\index{active math symbol} when declared
and stays active until the end of the block in which it is declared.  A
variable may not be declared a second time while it is active, but it
may be declared again (as a variable, but not as a constant) after it
becomes inactive.  A constant must be declared in the outermost block and may
not be declared a second time.\footnote{The rules for redeclaration may
become more general in the future;
see p.~\pageref{redeclarationf}.}\index{redeclaration of symbols}

A {\bf \$f statement}\index{\texttt{\$f} statement} consists of a label,
followed by \texttt{\$f}, followed by its typecode (an active constant),
followed by an
active variable, followed by the \texttt{\$.}\ token.  A {\bf \$e
statement}\index{\texttt{\$e} statement} consists of a label, followed
by \texttt{\$e}, followed by its typecode (an active constant),
followed by zero or more
active math symbols, followed by the \texttt{\$.}\ token.  A {\bf
hypothesis}\index{hypothesis} is a \texttt{\$f} or \texttt{\$e}
statement.
The type declared by a \texttt{\$f} statement for a given label
is global even if the variable is not
(e.g., a database may not have \texttt{wff P} in one local scope
nd \texttt{class P} in another).

A {\bf simple \$d statement}\index{\texttt{\$d} statement!simple}
consists of \texttt{\$d}, followed by two different active variables,
followed by the \texttt{\$.}\ token.  A {\bf compound \$d
statement}\index{\texttt{\$d} statement!compound} consists of
\texttt{\$d}, followed by three or more variables (all different),
followed by the \texttt{\$.}\ token.  The order of the variables in a
\texttt{\$d} statement is unimportant.  A compound \texttt{\$d}
statement is equivalent to a set of simple \texttt{\$d} statements, one
for each possible pair of variables occurring in the compound
\texttt{\$d} statement.  Henceforth in this specification we shall
assume all \texttt{\$d} statements are simple.  A \texttt{\$d} statement
is also called a {\bf disjoint} (or {\bf distinct}) {\bf variable
restriction}.\index{disjoint-variable restriction}

A {\bf \$a statement}\index{\texttt{\$a} statement} consists of a label,
followed by \texttt{\$a}, followed by its typecode (an active constant),
followed by
zero or more active math symbols, followed by the \texttt{\$.}\ token.  A {\bf
\$p statement}\index{\texttt{\$p} statement} consists of a label,
followed by \texttt{\$p}, followed by its typecode (an active constant),
followed by
zero or more active math symbols, followed by \texttt{\$=}, followed by
a sequence of labels, followed by the \texttt{\$.}\ token.  An {\bf
assertion}\index{assertion} is a \texttt{\$a} or \texttt{\$p} statement.

A \texttt{\$f}, \texttt{\$e}, or \texttt{\$d} statement is {\bf active}\index{active
statement} from the place it occurs until the end of the block it occurs in.
A \texttt{\$a} or \texttt{\$p} statement is {\bf active} from the place it occurs
through the end of the database.
There may not be two active \texttt{\$f} statements containing the same
variable.  Each variable in a \texttt{\$e}, \texttt{\$a}, or
\texttt{\$p} statement must exist in an active \texttt{\$f}
statement.\footnote{This requirement can greatly simplify the
unification algorithm (substitution calculation) required by proof
verification.}

%The label that begins each \texttt{\$f}, \texttt{\$e}, \texttt{\$a}, and
%\texttt{\$p} statement must be unique.
Each label token must be unique, and
no label token may match any math symbol
token.\label{namespace}\footnote{This
restriction was added on June 24, 2006.
It is not theoretically necessary but is imposed to make it easier to
write certain parsers.}

The set of {\bf mandatory variables}\index{mandatory variable} associated with
an assertion is the set of (zero or more) variables in the assertion and in any
active \texttt{\$e} statements.  The (possibly empty) set of {\bf mandatory
hypotheses}\index{mandatory hypothesis} is the set of all active \texttt{\$f}
statements containing mandatory variables, together with all active \texttt{\$e}
statements.
The set of {\bf mandatory {\bf \$d} statements}\index{mandatory
disjoint-variable restriction} associated with an assertion are those active
\texttt{\$d} statements whose variables are both among the assertion's
mandatory variables.

\subsection{Proof Verification}\label{spec4}

The sequence of labels between the \texttt{\$=} and \texttt{\$.}\ tokens
in a \texttt{\$p} statement is a {\bf proof}.\index{proof!Metamath} Each
label in a proof must be the label of an active statement other than the
\texttt{\$p} statement itself; thus a label must refer either to an
active hypothesis of the \texttt{\$p} statement or to an earlier
assertion.

An {\bf expression}\index{expression} is a sequence of math symbols. A {\bf
substitution map}\index{substitution map} associates a set of variables with a
set of expressions.  It is acceptable for a variable to be mapped to an
expression containing it.  A {\bf
substitution}\index{substitution!variable}\index{variable substitution} is the
simultaneous replacement of all variables in one or more expressions with the
expressions that the variables map to.

A proof is scanned in order of its label sequence.  If the label refers to an
active hypothesis, the expression in the hypothesis is pushed onto a
stack.\index{stack}\index{RPN stack}  If the label refers to an assertion, a
(unique) substitution must exist that, when made to the mandatory hypotheses
of the referenced assertion, causes them to match the topmost (i.e.\ most
recent) entries of the stack, in order of occurrence of the mandatory
hypotheses, with the topmost stack entry matching the last mandatory
hypothesis of the referenced assertion.  As many stack entries as there are
mandatory hypotheses are then popped from the stack.  The same substitution is
made to the referenced assertion, and the result is pushed onto the stack.
After the last label in the proof is processed, the stack must have a single
entry that matches the expression in the \texttt{\$p} statement containing the
proof.

%c%{\footnotesize\begin{quotation}\index{redeclaration of symbols}
%c%{{\em Comment.}\label{spec4comment} Whenever a math symbol token occurs in a
%c%{\texttt{\$c} or \texttt{\$v} statement, it is considered to designate a distinct new
%c%{symbol, even if the same token was previously declared (and is now inactive).
%c%{Thus a math token declared as a constant in two different blocks is considered
%c%{to designate two distinct constants (even though they have the same name).
%c%{The two constants will not match in a proof that references both blocks.
%c%{However, a proof referencing both blocks is acceptable as long as it doesn't
%c%{require that the constants match.  Similarly, a token declared to be a
%c%{constant for a referenced assertion will not match the same token declared to
%c%{be a variable for the \texttt{\$p} statement containing the proof.  In the case
%c%{of a token declared to be a variable for a referenced assertion, this is not
%c%{an issue since the variable can be substituted with whatever expression is
%c%{needed to achieve the required match.
%c%{\end{quotation}}
%c2%A proof may reference an assertion that contains or whose hypotheses contain a
%c2%constant that is not active for the \texttt{\$p} statement containing the proof.
%c2%However, the final result of the proof may not contain that constant. A proof
%c2%may also reference an assertion that contains or whose hypotheses contain a
%c2%variable that is not active for the \texttt{\$p} statement containing the proof.
%c2%That variable, of course, will be substituted with whatever expression is
%c2%needed to achieve the required match.

A proof may contain a \texttt{?}\ in place of a label to indicate an unknown step
(Section~\ref{unknown}).  A proof verifier may ignore any proof containing
\texttt{?}\ but should warn the user that the proof is incomplete.

A {\bf compressed proof}\index{compressed proof}\index{proof!compressed} is an
alternate proof notation described in Appen\-dix~\ref{compressed}; also see
references to ``compressed proof'' in the Index.  Compressed proofs are a
Metamath language extension which a complete proof verifier should be able to
parse and verify.

\subsubsection{Verifying Disjoint Variable Restrictions}

Each substitution made in a proof must be checked to verify that any
disjoint variable restrictions are satisfied, as follows.

If two variables replaced by a substitution exist in a mandatory \texttt{\$d}
statement\index{\texttt{\$d} statement} of the assertion referenced, the two
expressions resulting from the substitution must satisfy the following
conditions.  First, the two expressions must have no variables in common.
Second, each possible pair of variables, one from each expression, must exist
in an active \texttt{\$d} statement of the \texttt{\$p} statement containing the
proof.

\vskip 1ex

This ends the specification of the Metamath language;
see Appendix \ref{BNF} for its syntax in
Extended Backus--Naur Form (EBNF)\index{Extended Backus--Naur Form}\index{EBNF}.

\section{The Basic Keywords}\label{tut1}

Our expository material begins here.

Like most computer languages, Metamath\index{Metamath} takes its input from
one or more {\bf source files}\index{source file} which contain characters
expressed in the standard {\sc ascii} (American Standard Code for Information
Interchange)\index{ascii@{\sc ascii}} code for computers.  A source file
consists of a series of {\bf tokens}\index{token}, which are strings of
non-whitespace
printable characters (from the set of 94 shown on p.~\pageref{spec1chars})
separated by {\bf white space}\index{white space} (spaces, tabs, carriage
returns, line feeds, and form feeds). Any string consisting only of these
characters is treated the same as a single space.  The non-whitespace printable
characters\index{printable character} that Metamath recognizes are the 94
characters on standard {\sc ascii} keyboards.

Metamath has the ability to join several files together to form its
input (Section~\ref{include}).  We call the aggregate contents of all
the files after they have been joined together a {\bf
database}\index{database} to distinguish it from an individual source
file.  The tokens in a database consist of {\bf
keywords}\index{keyword}, which are built into the language, together
with two kinds of user-defined tokens called {\bf labels}\index{label}
and {\bf math symbols}\index{math symbol}.  (Often we will simply say
{\bf symbol}\index{symbol} instead of math symbol for brevity).  The set
of {\bf basic keywords}\index{basic keyword} is
\texttt{\$c}\index{\texttt{\$c} statement},
\texttt{\$v}\index{\texttt{\$v} statement},
\texttt{\$e}\index{\texttt{\$e} statement},
\texttt{\$f}\index{\texttt{\$f} statement},
\texttt{\$d}\index{\texttt{\$d} statement},
\texttt{\$a}\index{\texttt{\$a} statement},
\texttt{\$p}\index{\texttt{\$p} statement},
\texttt{\$=}\index{\texttt{\$=} keyword},
\texttt{\$.}\index{\texttt{\$.}\ keyword},
\texttt{\$\char`\{}\index{\texttt{\$\char`\{} and \texttt{\$\char`\}}
keywords}, and \texttt{\$\char`\}}.  This is the complete set of
syntactical elements of what we call the {\bf basic
language}\index{basic language} of Metamath, and with them you can
express all of the mathematics that were intended by the design of
Metamath.  You should make it a point to become very familiar with them.
Table~\ref{basickeywords} lists the basic keywords along with a brief
description of their functions.  For now, this description will give you
only a vague notion of what the keywords are for; later we will describe
the keywords in detail.


\begin{table}[htp] \caption{Summary of the basic Metamath
keywords} \label{basickeywords}
\begin{center}
\begin{tabular}{|p{4pc}|l|}
\hline
\em \centering Keyword&\em Description\\
\hline
\hline
\centering
   \texttt{\$c}&Constant symbol declaration\\
\hline
\centering
   \texttt{\$v}&Variable symbol declaration\\
\hline
\centering
   \texttt{\$d}&Disjoint variable restriction\\
\hline
\centering
   \texttt{\$f}&Variable-type (``floating'') hypothesis\\
\hline
\centering
   \texttt{\$e}&Logical (``essential'') hypothesis\\
\hline
\centering
   \texttt{\$a}&Axiomatic assertion\\
\hline
\centering
   \texttt{\$p}&Provable assertion\\
\hline
\centering
   \texttt{\$=}&Start of proof in \texttt{\$p} statement\\
\hline
\centering
   \texttt{\$.}&End of the above statement types\\
\hline
\centering
   \texttt{\$\char`\{}&Start of block\\
\hline
\centering
   \texttt{\$\char`\}}&End of block\\
\hline
\end{tabular}
\end{center}
\end{table}

%For LaTeX bug(?) where it puts tables on blank page instead of btwn text
%May have to adjust if text changes
%\newpage

There are some additional keywords, called {\bf auxiliary
keywords}\index{auxiliary keyword} that help make Metamath\index{Metamath}
more practical. These are part of the {\bf extended language}\index{extended
language}. They provide you with a means to put comments into a Metamath
source file\index{source file} and reference other source files.  We will
introduce these in later sections. Table~\ref{otherkeywords} summarizes them
so that you can recognize them now if you want to peruse some source
files while learning the basic keywords.


\begin{table}[htp] \caption{Auxiliary Metamath
keywords} \label{otherkeywords}
\begin{center}
\begin{tabular}{|p{4pc}|l|}
\hline
\em \centering Keyword&\em Description\\
\hline
\hline
\centering
   \texttt{\$(}&Start of comment\\
\hline
\centering
   \texttt{\$)}&End of comment\\
\hline
\centering
   \texttt{\$[}&Start of included source file name\\
\hline
\centering
   \texttt{\$]}&End of included source file name\\
\hline
\end{tabular}
\end{center}
\end{table}
\index{\texttt{\$(} and \texttt{\$)} auxiliary keywords}
\index{\texttt{\$[} and \texttt{\$]} auxiliary keywords}


Unlike those in some computer languages, the keywords\index{keyword} are short
two-character sequences rather than English-like words.  While this may make
them slightly more difficult to remember at first, their brevity allows
them to blend in with the mathematics being described, not
distract from it, like punctuation marks.


\subsection{User-Defined Tokens}\label{dollardollar}\index{token}

As you may have noticed, all keywords\index{keyword} begin with the \texttt{\$}
character.  This mundane monetary symbol is not ordinarily used in higher
mathematics (outside of grant proposals), so we have appropriated it to
distinguish the Metamath\index{Metamath} keywords from ordinary mathematical
symbols. The \texttt{\$} character is thus considered special and may not be
used as a character in a user-defined token.  All tokens and keywords are
case-sensitive; for example, \texttt{n} is considered to be a different character
from \texttt{N}.  Case-sensitivity makes the available {\sc ascii} character set
as rich as possible.

\subsubsection{Math Symbol Tokens}\index{token}

Math symbols\index{math symbol} are tokens used to represent the symbols
that appear in ordinary mathematical formulas.  They may consist of any
combination of the 93 non-whitespace printable {\sc ascii} characters other than
\texttt{\$}~. Some examples are \texttt{x}, \texttt{+}, \texttt{(},
\texttt{|-}, \verb$!%@?&$, and \texttt{bounded}.  For readability, it is
best to try to make these look as similar to actual mathematical symbols
as possible, within the constraints of the {\sc ascii} character set, in
order to make the resulting mathematical expressions more readable.

In the Metamath\index{Metamath} language, you express ordinary
mathematical formulas and statements as sequences of math symbols such
as \texttt{2 + 2 = 4} (five symbols, all constants).\footnote{To
eliminate ambiguity with other expressions, this is expressed in the set
theory database \texttt{set.mm} as \texttt{|- ( 2 + 2
 ) = 4 }, whose \LaTeX\ equivalent is $\vdash
(2+2)=4$.  The \,$\vdash$ means ``is a theorem'' and the
parentheses allow explicit associative grouping.}\index{turnstile
({$\,\vdash$})} They may even be English
sentences, as in \texttt{E is closed and bounded} (five symbols)---here
\texttt{E} would be a variable and the other four symbols constants.  In
principle, a Metamath database could be constructed to work with almost
any unambiguous English-language mathematical statement, but as a
practical matter the definitions needed to provide for all possible
syntax variations would be cumbersome and distracting and possibly have
subtle pitfalls accidentally built in.  We generally recommend that you
express mathematical statements with compact standard mathematical
symbols whenever possible and put their English-language descriptions in
comments.  Axioms\index{axiom} and definitions\index{definition}
(\texttt{\$a}\index{\texttt{\$a} statement} statements) are the only
places where Metamath will not detect an error, and doing this will help
reduce the number of definitions needed.

You are free to use any tokens\index{token} you like for math
symbols\index{math symbol}.  Appendix~\ref{ASCII} recommends token names to
use for symbols in set theory, and we suggest you adopt these in order to be
able to include the \texttt{set.mm} set theory database in your database.  For
printouts, you can convert the tokens in a database
to standard mathematical symbols with the \LaTeX\ typesetting program.  The
Metamath command \texttt{open tex} {\em filename}\index{\texttt{open tex} command}
produces output that can be read by \LaTeX.\index{latex@{\LaTeX}}
The correspondence
between tokens and the actual symbols is made by \texttt{latexdef}
statements inside a special database comment tagged
with \texttt{\$t}.\index{\texttt{\$t} comment}\index{typesetting comment}
  You can edit
this comment to change the definitions or add new ones.
Appendix~\ref{ASCII} describes how to do this in more detail.

% White space\index{white space} is normally used to separate math
% symbol\index{math symbol} tokens, but they may be juxtaposed without white
% space in \texttt{\$d}\index{\texttt{\$d} statement}, \texttt{\$e}\index{\texttt{\$e}
% statement}, \texttt{\$f}\index{\texttt{\$f} statement}, \texttt{\$a}\index{\texttt{\$a}
% statement}, and \texttt{\$p}\index{\texttt{\$p} statement} statements when no
% ambiguity will result.  Specifically, Metamath parses the math symbol sequence
% in one of these statements in the following manner:  when the math symbol
% sequence has been broken up into tokens\index{token} up to a given character,
% the next token is the longest string of characters that could constitute a
% math symbol that is active\index{active
% math symbol} at that point.  (See Section~\ref{scoping} for the
% definition of an active math symbol.)  For example, if \texttt{-}, \texttt{>}, and
% \texttt{->} are the only active math symbols, the juxtaposition \texttt{>-} will be
% interpreted as the two symbols \texttt{>} and \texttt{-}, whereas \texttt{->} will
% always be interpreted as that single symbol.\footnote{For better readability we
% recommend a white space between each token.  This also makes searching for a
% symbol easier to do with an editor.  Omission of optional white space is useful
% for reducing typing when assigning an expression to a temporary
% variable\index{temporary variable} with the \texttt{let variable} Metamath
% program command.}\index{\texttt{let variable} command}
%
% Keywords\index{keyword} may be placed next to math symbols without white
% space\index{white space} between them.\footnote{Again, we do not recommend
% this for readability.}
%
% The math symbols\index{math symbol} in \texttt{\$c}\index{\texttt{\$c} statement}
% and \texttt{\$v}\index{\texttt{\$v} statement} statements must always be separated
% by white space\index{white
% space}, for the obvious reason that these statements define the names
% of the symbols.
%
% Math symbols referred to in comments (see Section~\ref{comments}) must also be
% separated by white space.  This allows you to make comments about symbols that
% are not yet active\index{active
% math symbol}.  (The ``math mode'' feature of comments is also a quick and
% easy way to obtain word processing text with embedded mathematical symbols,
% independently of the main purpose of Metamath; the way to do this is described
% in Section~\ref{comments})

\subsubsection{Label Tokens}\index{token}\index{label}

Label tokens are used to identify Metamath\index{Metamath} statements for
later reference. Label tokens may contain only letters, digits, and the three
characters period, hyphen, and underscore:
\begin{verbatim}
. - _
\end{verbatim}

A label is {\bf declared}\index{label declaration} by placing it immediately
before the keyword of the statement it identifies.  For example, the label
\texttt{axiom.1} might be declared as follows:
\begin{verbatim}
axiom.1 $a |- x = x $.
\end{verbatim}

Each \texttt{\$e}\index{\texttt{\$e} statement},
\texttt{\$f}\index{\texttt{\$f} statement},
\texttt{\$a}\index{\texttt{\$a} statement}, and
\texttt{\$p}\index{\texttt{\$p} statement} statement in a database must
have a label declared for it.  No other statement types may have label
declarations.  Every label must be unique.

A label (and the statement it identifies) is {\bf referenced}\index{label
reference} by including the label between the \texttt{\$=}\index{\texttt{\$=}
keyword} and \texttt{\$.}\index{\texttt{\$.}\ keyword}\ keywords in a \texttt{\$p}
statement.  The sequence of labels\index{label sequence} between these two
keywords is called a {\bf proof}\index{proof}.  An example of a statement with
a proof that we will encounter later (Section~\ref{proof}) is
\begin{verbatim}
wnew $p wff ( s -> ( r -> p ) )
     $= ws wr wp w2 w2 $.
\end{verbatim}

You don't have to know what this means just yet, but you should know that the
label \texttt{wnew} is declared by this \texttt{\$p} statement and that the labels
\texttt{ws}, \texttt{wr}, \texttt{wp}, and \texttt{w2} are assumed to have been declared
earlier in the database and are referenced here.

\subsection{Constants and Variables}
\index{constant}
\index{variable}

An {\bf expression}\index{expression} is any sequence of math
symbols, possibly empty.

The basic Metamath\index{Metamath} language\index{basic language} has two
kinds of math symbols\index{math symbol}:  {\bf constants}\index{constant} and
{\bf variables}\index{variable}.  In a Metamath proof, a constant may not be
substituted with any expression.  A variable can be
substituted\index{substitution!variable}\index{variable substitution} with any
expression.  This sequence may include other variables and may even include
the variable being substituted.  This substitution takes place when proofs are
verified, and it will be described in Section~\ref{proof}.  The \texttt{\$f}
statement (described later in Section~\ref{dollaref}) is used to specify the
{\bf type} of a variable (i.e.\ what kind of
variable it is)\index{variable type}\index{type} and
give it a meaning typically
associated with a ``metavariable''\index{metavariable}\footnote{A metavariable
is a variable that ranges over the syntactical elements of the object language
being discussed; for example, one metavariable might represent a variable of
the object language and another metavariable might represent a formula in the
object language.} in ordinary mathematics; for example, a variable may be
specified to be a wff or well-formed formula (in logic), a set (in set
theory), or a non-negative integer (in number theory).

%\subsection{The \texttt{\$c} and \texttt{\$v} Declaration Statements}
\subsection{The \texttt{\$c} and \texttt{\$v} Declaration Statements}
\index{\texttt{\$c} statement}
\index{constant declaration}
\index{\texttt{\$v} statement}
\index{variable declaration}

Constants are introduced or {\bf declared}\index{constant declaration}
with \texttt{\$c}\index{\texttt{\$c} statement} statements, and
variables are declared\index{variable declaration} with
\texttt{\$v}\index{\texttt{\$v} statement} statements.  A {\bf simple}
declaration\index{simple declaration} statement introduces a single
constant or variable.  Its syntax is one of the following:
\begin{center}
  \texttt{\$c} {\em math-symbol} \texttt{\$.}\\
  \texttt{\$v} {\em math-symbol} \texttt{\$.}
\end{center}
The notation {\em math-symbol} means any math symbol token\index{token}.

Some examples of simple declaration statements are:
\begin{center}
  \texttt{\$c + \$.}\\
  \texttt{\$c -> \$.}\\
  \texttt{\$c ( \$.}\\
  \texttt{\$v x \$.}\\
  \texttt{\$v y2 \$.}
\end{center}

The characters in a math symbol\index{math symbol} being declared are
irrelevant to Meta\-math; for example, we could declare a right parenthesis to
be a variable,
\begin{center}
  \texttt{\$v ) \$.}\\
\end{center}
although this would be unconventional.

A {\bf compound} declaration\index{compound declaration} statement is a
shorthand for declaring several symbols at once.  Its syntax is one of the
following:
\begin{center}
  \texttt{\$c} {\em math-symbol}\ \,$\cdots$\ {\em math-symbol} \texttt{\$.}\\
  \texttt{\$v} {\em math-symbol}\ \,$\cdots$\ {\em math-symbol} \texttt{\$.}
\end{center}\index{\texttt{\$c} statement}
Here, the ellipsis (\ldots) means any number of {\em math-symbol}\,s.

An example of a compound declaration statement is:
\begin{center}
  \texttt{\$v x y mu \$.}\\
\end{center}
This is equivalent to the three simple declaration statements
\begin{center}
  \texttt{\$v x \$.}\\
  \texttt{\$v y \$.}\\
  \texttt{\$v mu \$.}\\
\end{center}
\index{\texttt{\$v} statement}

There are certain rules on where in the database math symbols may be declared,
what sections of the database are aware of them (i.e.\ where they are
``active''), and when they may be declared more than once.  These will be
discussed in Section~\ref{scoping} and specifically on
p.~\pageref{redeclaration}.

\subsection{The \texttt{\$d} Statement}\label{dollard}
\index{\texttt{\$d} statement}

The \texttt{\$d} statement is called a {\bf disjoint-variable restriction}.  The
syntax of the {\bf simple} version of this statement is
\begin{center}
  \texttt{\$d} {\em variable variable} \texttt{\$.}
\end{center}
where each {\em variable} is a previously declared variable and the two {\em
variable}\,s are different.  (More specifically, each  {\em variable} must be
an {\bf active} variable\index{active math symbol}, which means there must be
a previous \texttt{\$v} statement whose {\bf scope}\index{scope} includes the
\texttt{\$d} statement.  These terms will be defined when we discuss scoping
statements in Section~\ref{scoping}.)

In ordinary mathematics, formulas may arise that are true if the variables in
them are distinct\index{distinct variables}, but become false when those
variables are made identical. For example, the formula in logic $\exists x\,x
\neq y$, which means ``for a given $y$, there exists an $x$ that is not equal
to $y$,'' is true in most mathematical theories (namely all non-trivial
theories\index{non-trivial theory}, i.e.\ those that describe more than one
individual, such as arithmetic).  However, if we substitute $y$ with $x$, we
obtain $\exists x\,x \neq x$, which is always false, as it means ``there
exists something that is not equal to itself.''\footnote{If you are a
logician, you will recognize this as the improper substitution\index{proper
substitution}\index{substitution!proper} of a free variable\index{free
variable} with a bound variable\index{bound variable}.  Metamath makes no
inherent distinction between free and bound variables; instead, you let
Metamath know what substitutions are permissible by using \texttt{\$d} statements
in the right way in your axiom system.}\index{free vs.\ bound variable}  The
\texttt{\$d} statement allows you to specify a restriction that forbids the
substitution of one variable with another.  In
this case, we would use the statement
\begin{center}
  \texttt{\$d x y \$.}
\end{center}\index{\texttt{\$d} statement}
to specify this restriction.

The order in which the variables appear in a \texttt{\$d} statement is not
important.  We could also use
\begin{center}
  \texttt{\$d y x \$.}
\end{center}

The \texttt{\$d} statement is actually more general than this, as the
``disjoint''\index{disjoint variables} in its name suggests.  The full meaning
is that if any substitution is made to its two variables (during the
course of a proof that references a \texttt{\$a} or \texttt{\$p} statement
associated with the \texttt{\$d}), the two expressions that result from the
substitution must have no variables in common.  In addition, each possible
pair of variables, one from each expression, must be in a \texttt{\$d} statement
associated with the statement being proved.  (This requirement forces the
statement being proved to ``inherit'' the original disjoint variable
restriction.)

For example, suppose \texttt{u} is a variable.  If the restriction
\begin{center}
  \texttt{\$d A B \$.}
\end{center}
has been specified for a theorem referenced in a
proof, we may not substitute \texttt{A} with \mbox{\tt a + u} and
\texttt{B} with \mbox{\tt b + u} because these two symbol sequences have the
variable \texttt{u} in common.  Furthermore, if \texttt{a} and \texttt{b} are
variables, we may not substitute \texttt{A} with \texttt{a} and \texttt{B} with \texttt{b}
unless we have also specified \texttt{\$d a b} for the theorem being proved; in
other words, the \texttt{\$d} property associated with a pair of variables must
be effectively preserved after substitution.

The \texttt{\$d}\index{\texttt{\$d} statement} statement does {\em not} mean ``the
two variables may not be substituted with the same thing,'' as you might think
at first.  For example, substituting each of \texttt{A} and \texttt{B} in the above
example with identical symbol sequences consisting only of constants does not
cause a disjoint variable conflict, because two symbol sequences have no
variables in common (since they have no variables, period).  Similarly, a
conflict will not occur by substituting the two variables in a \texttt{\$d}
statement with the empty symbol sequence\index{empty substitution}.

The \texttt{\$d} statement does not have a direct counterpart in
ordinary mathematics, partly because the variables\index{variable} of
Metamath are not really the same as the variables\index{variable!in
ordinary mathematics} of ordinary mathematics but rather are
metavariables\index{metavariable} ranging over them (as well as over
other kinds of symbols and groups of symbols).  Depending on the
situation, we may informally interpret the \texttt{\$d} statement in
different ways.  Suppose, for example, that \texttt{x} and \texttt{y}
are variables ranging over numbers (more precisely, that \texttt{x} and
\texttt{y} are metavariables ranging over variables that range over
numbers), and that \texttt{ph} ($\varphi$) and \texttt{ps} ($\psi$) are
variables (more precisely, metavariables) ranging over formulas.  We can
make the following interpretations that correspond to the informal
language of ordinary mathematics:
\begin{quote}
\begin{tabbing}
\texttt{\$d x y \$.} means ``assume $x$ and $y$ are
distinct variables.''\\
\texttt{\$d x ph \$.} means ``assume $x$ does not
occur in $\varphi$.''\\
\texttt{\$d ph ps \$.} \=means ``assume $\varphi$ and
$\psi$ have no variables\\ \>in common.''
\end{tabbing}
\end{quote}\index{\texttt{\$d} statement}

\subsubsection{Compound \texttt{\$d} Statements}

The {\bf compound} version of the \texttt{\$d} statement is a shorthand for
specifying several variables whose substitutions must be pairwise disjoint.
Its syntax is:
\begin{center}
  \texttt{\$d} {\em variable}\ \,$\cdots$\ {\em variable} \texttt{\$.}
\end{center}\index{\texttt{\$d} statement}
Here, {\em variable} represents the token of a previously declared
variable (specifically, an active variable) and all {\em variable}\,s are
different.  The compound \texttt{\$d}
statement is internally broken up by Metamath into one simple \texttt{\$d}
statement for each possible pair of variables in the original \texttt{\$d}
statement.  For example,
\begin{center}
  \texttt{\$d w x y z \$.}
\end{center}
is equivalent to
\begin{center}
  \texttt{\$d w x \$.}\\
  \texttt{\$d w y \$.}\\
  \texttt{\$d w z \$.}\\
  \texttt{\$d x y \$.}\\
  \texttt{\$d x z \$.}\\
  \texttt{\$d y z \$.}
\end{center}

Two or more simple \texttt{\$d} statements specifying the same variable pair are
internally combined into a single \texttt{\$d} statement.  Thus the set of three
statements
\begin{center}
  \texttt{\$d x y \$.}
  \texttt{\$d x y \$.}
  \texttt{\$d y x \$.}
\end{center}
is equivalent to
\begin{center}
  \texttt{\$d x y \$.}
\end{center}

Similarly, compound \texttt{\$d} statements, after being internally broken up,
internally have their common variable pairs combined.  For example the
set of statements
\begin{center}
  \texttt{\$d x y A \$.}
  \texttt{\$d x y B \$.}
\end{center}
is equivalent to
\begin{center}
  \texttt{\$d x y \$.}
  \texttt{\$d x A \$.}
  \texttt{\$d y A \$.}
  \texttt{\$d x y \$.}
  \texttt{\$d x B \$.}
  \texttt{\$d y B \$.}
\end{center}
which is equivalent to
\begin{center}
  \texttt{\$d x y \$.}
  \texttt{\$d x A \$.}
  \texttt{\$d y A \$.}
  \texttt{\$d x B \$.}
  \texttt{\$d y B \$.}
\end{center}

Metamath\index{Metamath} automatically verifies that all \texttt{\$d}
restrictions are met whenever it verifies proofs.  \texttt{\$d} statements are
never referenced directly in proofs (this is why they do not have
labels\index{label}), but Metamath is always aware of which ones must be
satisfied (i.e.\ are active) and will notify you with an error message if any
violation occurs.

To illustrate how Metamath detects a missing \texttt{\$d}
statement, we will look at the following example from the
\texttt{set.mm} database.

\begin{verbatim}
$d x z $.  $d y z $.
$( Theorem to add distinct quantifier to atomic formula. $)
ax17eq $p |- ( x = y -> A. z x = y ) $=...
\end{verbatim}

This statement has the obvious requirement that $z$ must be
distinct\index{distinct variables} from $x$ in theorem \texttt{ax17eq} that
states $x=y \rightarrow \forall z \, x=y$ (well, obvious if you're a logician,
for otherwise we could conclude  $x=y \rightarrow \forall x \, x=y$, which is
false when the free variables $x$ and $y$ are equal.).

Let's look at what happens if we edit the database to comment out this
requirement.

\begin{verbatim}
$( $d x z $. $) $d y z $.
$( Theorem to add distinct quantifier to atomic formula. $)
ax17eq $p |- ( x = y -> A. z x = y ) $=...
\end{verbatim}

When it tries to verify the proof, Metamath will tell you that \texttt{x} and
\texttt{z} must be disjoint, because one of its steps references an axiom or
theorem that has this requirement.

\begin{verbatim}
MM> verify proof ax17eq
ax17eq ?Error at statement 1918, label "ax17eq", type "$p":
      vz wal wi vx vy vz ax-13 vx vy weq vz vx ax-c16 vx vy
                                               ^^^^^
There is a disjoint variable ($d) violation at proof step 29.
Assertion "ax-c16" requires that variables "x" and "y" be
disjoint.  But "x" was substituted with "z" and "y" was
substituted with "x".  The assertion being proved, "ax17eq",
does not require that variables "z" and "x" be disjoint.
\end{verbatim}

We can see the substitutions into \texttt{ax-c16} with the following command.

\begin{verbatim}
MM> show proof ax17eq / detailed_step 29
Proof step 29:  pm2.61dd.2=ax-c16 $a |- ( A. z z = x -> ( x =
  y -> A. z x = y ) )
This step assigns source "ax-c16" ($a) to target "pm2.61dd.2"
($e).  The source assertion requires the hypotheses "wph"
($f, step 26), "vx" ($f, step 27), and "vy" ($f, step 28).
The parent assertion of the target hypothesis is "pm2.61dd"
($p, step 36).
The source assertion before substitution was:
    ax-c16 $a |- ( A. x x = y -> ( ph -> A. x ph ) )
The following substitutions were made to the source
assertion:
    Variable  Substituted with
     x         z
     y         x
     ph        x = y
The target hypothesis before substitution was:
    pm2.61dd.2 $e |- ( ph -> ch )
The following substitutions were made to the target
hypothesis:
    Variable  Substituted with
     ph        A. z z = x
     ch        ( x = y -> A. z x = y )
\end{verbatim}

The disjoint variable restrictions of \texttt{ax-c16} can be seen from the
\texttt{show state\-ment} command.  The line that begins ``\texttt{Its mandatory
dis\-joint var\-i\-able pairs are:}\ldots'' lists any \texttt{\$d} variable
pairs in brackets.

\begin{verbatim}
MM> show statement ax-c16/full
Statement 3033 is located on line 9338 of the file "set.mm".
"Axiom of Distinct Variables. ..."
  ax-c16 $a |- ( A. x x = y -> ( ph -> A. x ph ) ) $.
Its mandatory hypotheses in RPN order are:
  wph $f wff ph $.
  vx $f set x $.
  vy $f set y $.
Its mandatory disjoint variable pairs are:  <x,y>
The statement and its hypotheses require the variables:  x y
      ph
The variables it contains are:  x y ph
\end{verbatim}

Since Metamath will always detect when \texttt{\$d}\index{\texttt{\$d} statement}
statements are needed for a proof, you don't have to worry too much about
forgetting to put one in; it can always be added if you see the error message
above.  If you put in unnecessary \texttt{\$d} statements, the worst that could
happen is that your theorem might not be as general as it could be, and this
may limit its use later on.

On the other hand, when you introduce axioms (\texttt{\$a}\index{\texttt{\$a}
statement} statements), you must be very careful to properly specify the
necessary associated \texttt{\$d} statements since Metamath has no way of knowing
whether your axioms are correct.  For example, Metamath would have no idea
that \texttt{ax-c16}, which we are telling it is an axiom of logic, would lead to
contradictions if we omitted its associated \texttt{\$d} statement.

% This was previously a comment in footnote-sized type, but it can be
% hard to read this much text in a small size.
% As a result, it's been changed to normally-sized text.
\label{nodd}
You may wonder if it is possible to develop standard
mathematics in the Metamath language without the \texttt{\$d}\index{\texttt{\$d}
statement} statement, since it seems like a nuisance that complicates proof
verification. The \texttt{\$d} statement is not needed in certain subsets of
mathematics such as propositional calculus.  However, dummy
variables\index{dummy variable!eliminating} and their associated \texttt{\$d}
statements are impossible to avoid in proofs in standard first-order logic as
well as in the variant used in \texttt{set.mm}.  In fact, there is no upper bound to
the number of dummy variables that might be needed in a proof of a theorem of
first-order logic containing 3 or more variables, as shown by H.\
Andr\'{e}ka\index{Andr{\'{e}}ka, H.} \cite{Nemeti}.  A first-order system that
avoids them entirely is given in \cite{Megill}\index{Megill, Norman}; the
trick there is simply to embed harmlessly the necessary dummy variables into a
theorem being proved so that they aren't ``dummy'' anymore, then interpret the
resulting longer theorem so as to ignore the embedded dummy variables.  If
this interests you, the system in \texttt{set.mm} obtained from \texttt{ax-1}
through \texttt{ax-c14} in \texttt{set.mm}, and deleting \texttt{ax-c16} and \texttt{ax-5},
requires no \texttt{\$d} statements but is logically complete in the sense
described in \cite{Megill}.  This means it can prove any theorem of
first-order logic as long as we add to the theorem an antecedent that embeds
dummy and any other variables that must be distinct.  In a similar fashion,
axioms for set theory can be devised that
do not require distinct variable
provisos\index{Set theory without distinct variable provisos},
as explained at
\url{http://us.metamath.org/mpeuni/mmzfcnd.html}.
Together, these in principle allow all of
mathematics to be developed under Metamath without a \texttt{\$d} statement,
although the length of the resulting theorems will grow as more and
more dummy variables become required in their proofs.

\subsection{The \texttt{\$f}
and \texttt{\$e} Statements}\label{dollaref}
\index{\texttt{\$e} statement}
\index{\texttt{\$f} statement}
\index{floating hypothesis}
\index{essential hypothesis}
\index{variable-type hypothesis}
\index{logical hypothesis}
\index{hypothesis}

Metamath has two kinds of hypo\-theses, the \texttt{\$f}\index{\texttt{\$f}
statement} or {\bf variable-type} hypothesis and the \texttt{\$e} or {\bf logical}
hypo\-the\-sis.\index{\texttt{\$d} statement}\footnote{Strictly speaking, the
\texttt{\$d} statement is also a hypothesis, but it is never directly referenced
in a proof, so we call it a restriction rather than a hypothesis to lessen
confusion.  The checking for violations of \texttt{\$d} restrictions is automatic
and built into Metamath's proof-checking algorithm.} The letters \texttt{f} and
\texttt{e} stand for ``floating''\index{floating hypothesis} (roughly meaning
used only if relevant) and ``essential''\index{essential hypothesis} (meaning
always used) respectively, for reasons that will become apparent
when we discuss frames in
Section~\ref{frames} and scoping in Section~\ref{scoping}. The syntax of these
are as follows:
\begin{center}
  {\em label} \texttt{\$f} {\em typecode} {\em variable} \texttt{\$.}\\
  {\em label} \texttt{\$e} {\em typecode}
      {\em math-symbol}\ \,$\cdots$\ {\em math-symbol} \texttt{\$.}\\
\end{center}
\index{\texttt{\$e} statement}
\index{\texttt{\$f} statement}
A hypothesis must have a {\em label}\index{label}.  The expression in a
\texttt{\$e} hypothesis consists of a typecode (an active constant math symbol)
followed by a sequence
of zero or more math symbols. Each math symbol (including {\em constant}
and {\em variable}) must be a previously declared constant or variable.  (In
addition, each math symbol must be active, which will be covered when we
discuss scoping statements in Section~\ref{scoping}.)  You use a \texttt{\$f}
hypothesis to specify the
nature or {\bf type}\index{variable type}\index{type} of a variable (such as ``let $x$ be an
integer'') and use a \texttt{\$e} hypothesis to express a logical truth (such as
``assume $x$ is prime'') that must be established in order for an assertion
requiring it to also be true.

A variable must have its type specified in a \texttt{\$f} statement before
it may be used in a \texttt{\$e}, \texttt{\$a}, or \texttt{\$p}
statement.  There may be only one (active) \texttt{\$f} statement for a
given variable.  (``Active'' is defined in Section~\ref{scoping}.)

In ordinary mathematics, theorems\index{theorem} are often expressed in the
form ``Assume $P$; then $Q$,'' where $Q$ is a statement that you can derive
if you start with statement $P$.\index{free variable}\footnote{A stronger
version of a theorem like this would be the {\em single} formula $P\rightarrow
Q$ ($P$ implies $Q$) from which the weaker version above follows by the rule
of modus ponens in logic.  We are not discussing this stronger form here.  In
the weaker form, we are saying only that if we can {\em prove} $P$, then we can
{\em prove} $Q$.  In a logician's language, if $x$ is the only free variable
in $P$ and $Q$, the stronger form is equivalent to $\forall x ( P \rightarrow
Q)$ (for all $x$, $P$ implies $Q$), whereas the weaker form is equivalent to
$\forall x P \rightarrow \forall x Q$. The stronger form implies the weaker,
but not vice-versa.  To be precise, the weaker form of the theorem is more
properly called an ``inference'' rather than a theorem.}\index{inference}
In the
Metamath\index{Metamath} language, you would express mathematical statement
$P$ as a hypothesis (a \texttt{\$e} Metamath language statement in this case) and
statement $Q$ as a provable assertion (a \texttt{\$p}\index{\texttt{\$p} statement}
statement).

Some examples of hypotheses you might encounter in logic and set theory are
\begin{center}
  \texttt{stmt1 \$f wff P \$.}\\
  \texttt{stmt2 \$f set x \$.}\\
  \texttt{stmt3 \$e |- ( P -> Q ) \$.}
\end{center}
\index{\texttt{\$e} statement}
\index{\texttt{\$f} statement}
Informally, these would be read, ``Let $P$ be a well-formed-formula,'' ``Let
$x$ be an (individual) variable,'' and ``Assume we have proved $P \rightarrow
Q$.''  The turnstile symbol \,$\vdash$\index{turnstile ({$\,\vdash$})} is
commonly used in logic texts to mean ``a proof exists for.''

To summarize:
\begin{itemize}
\item A \texttt{\$f} hypothesis tells Metamath the type or kind of its variable.
It is analogous to a variable declaration in a computer language that
tells the compiler that a variable is an integer or a floating-point
number.
\item The \texttt{\$e} hypothesis corresponds to what you would usually call a
``hypothesis'' in ordinary mathematics.
\end{itemize}

Before an assertion\index{assertion} (\texttt{\$a} or \texttt{\$p} statement) can be
referenced in a proof, all of its associated \texttt{\$f} and \texttt{\$e} hypotheses
(i.e.\ those \texttt{\$e} hypotheses that are active) must be satisfied (i.e.
established by the proof).  The meaning of ``associated'' (which we will call
{\bf mandatory} in Section~\ref{frames}) will become clear when we discuss
scoping later.

Note that after any \texttt{\$f}, \texttt{\$e},
\texttt{\$a}, or \texttt{\$p} token there is a required
\textit{typecode}\index{typecode}.
The typecode is a constant used to enforce types of expressions.
This will become clearer once we learn more about
assertions (\texttt{\$a} and \texttt{\$p} statements).
An example may also clarify their purpose.
In the
\texttt{set.mm}\index{set theory database (\texttt{set.mm})}%
\index{Metamath Proof Explorer}
database,
the following typecodes are used:

\begin{itemize}
\item \texttt{wff} :
  Well-formed formula (wff) symbol
  (read: ``the following symbol sequence is a wff'').
% The *textual* typecode for turnstile is "|-", but when read it's a little
% confusing, so I intentionally display the mathematical symbol here instead
% (I think it's clearer in this context).
\item \texttt{$\vdash$} :
  Turnstile (read: ``the following symbol sequence is provable'' or
  ``a proof exists for'').
\item \texttt{setvar} :
  Individual set variable type (read: ``the following is an
  individual set variable'').
  Note that this is \textit{not} the type of an arbitrary set expression,
  instead, it is used to ensure that there is only a single symbol used
  after quantifiers like for-all ($\forall$) and there-exists ($\exists$).
\item \texttt{class} :
  An expression that is a syntactically valid class expression.
  All valid set expressions are also valid class expression, so expressions
  of sets normally have the \texttt{class} typecode.
  Use the \texttt{class} typecode,
  \textit{not} the \texttt{setvar} typecode,
  for the type of set expressions unless you are specifically identifying
  a single set variable.
\end{itemize}

\subsection{Assertions (\texttt{\$a} and \texttt{\$p} Statements)}
\index{\texttt{\$a} statement}
\index{\texttt{\$p} statement}\index{assertion}\index{axiomatic assertion}
\index{provable assertion}

There are two types of assertions, \texttt{\$a}\index{\texttt{\$a} statement}
statements ({\bf axiomatic assertions}) and \texttt{\$p} statements ({\bf
provable assertions}).  Their syntax is as follows:
\begin{center}
  {\em label} \texttt{\$a} {\em typecode} {\em math-symbol} \ldots
         {\em math-symbol} \texttt{\$.}\\
  {\em label} \texttt{\$p} {\em typecode} {\em math-symbol} \ldots
        {\em math-symbol} \texttt{\$=} {\em proof} \texttt{\$.}
\end{center}
\index{\texttt{\$a} statement}
\index{\texttt{\$p} statement}
\index{\texttt{\$=} keyword}
An assertion always requires a {\em label}\index{label}. The expression in an
assertion consists of a typecode (an active constant)
followed by a sequence of zero
or more math symbols.  Each math symbol, including any {\em constant}, must be a
previously declared constant or variable.  (In addition, each math symbol
must be active, which will be covered when we discuss scoping statements in
Section~\ref{scoping}.)

A \texttt{\$a} statement is usually a definition of syntax (for example, if $P$
and $Q$ are wffs then so is $(P\to Q)$), an axiom\index{axiom} of ordinary
mathematics (for example, $x=x$), or a definition\index{definition} of
ordinary mathematics (for example, $x\ne y$ means $\lnot x=y$). A \texttt{\$p}
statement is a claim that a certain combination of math symbols follows from
previous assertions and is accompanied by a proof that demonstrates it.

Assertions can also be referenced in (later) proofs in order to derive new
assertions from them. The label of an assertion is used to refer to it in a
proof. Section~\ref{proof} will describe the proof in detail.

Assertions also provide the primary means for communicating the mathematical
results in the database to people.  Proofs (when conveniently displayed)
communicate to people how the results were arrived at.

\subsubsection{The \texttt{\$a} Statement}
\index{\texttt{\$a} statement}

Axiomatic assertions (\texttt{\$a} statements) represent the starting points from
which other assertions (\texttt{\$p}\index{\texttt{\$p} statement} statements) are
derived.  Their most obvious use is for specifying ordinary mathematical
axioms\index{axiom}, but they are also used for two other purposes.

First, Metamath\index{Metamath} needs to know the syntax of symbol
sequences that constitute valid mathematical statements.  A Metamath
proof must be broken down into much more detail than ordinary
mathematical proofs that you may be used to thinking of (even the
``complete'' proofs of formal logic\index{formal logic}).  This is one
of the things that makes Metamath a general-purpose language,
independent of any system of logic or even syntax.  If you want to use a
substitution instance of an assertion as a step in a proof, you must
first prove that the substitution is syntactically correct (or if you
prefer, you must ``construct'' it), showing for example that the
expression you are substituting for a wff metavariable is a valid wff.
The \texttt{\$a}\index{\texttt{\$a} statement} statement is used to
specify those combinations of symbols that are considered syntactically
valid, such as the legal forms of wffs.

Second, \texttt{\$a} statements are used to specify what are ordinarily thought of
as definitions, i.e.\ new combinations of symbols that abbreviate other
combinations of symbols.  Metamath makes no distinction\index{axiom vs.\
definition} between axioms\index{axiom} and definitions\index{definition}.
Indeed, it has been argued that such distinction should not be made even in
ordinary mathematics; see Section~\ref{definitions}, which discusses the
philosophy of definitions.  Section~\ref{hierarchy} discusses some
technical requirements for definitions.  In \texttt{set.mm} we adopt the
convention of prefixing axiom labels with \texttt{ax-} and definition labels with
\texttt{df-}\index{label}.

The results that can be derived with the Metamath language are only as good as
the \texttt{\$a}\index{\texttt{\$a} statement} statements used as their starting
point.  We cannot stress this too strongly.  For example, Metamath will
not prevent you from specifying $x\neq x$ as an axiom of logic.  It is
essential that you scrutinize all \texttt{\$a} statements with great care.
Because they are a source of potential pitfalls, it is best not to add new
ones (usually new definitions) casually; rather you should carefully evaluate
each one's necessity and advantages.

Once you have in place all of the basic axioms\index{axiom} and
rules\index{rule} of a mathematical theory, the only \texttt{\$a} statements that
you will be adding will be what are ordinarily called definitions.  In
principle, definitions should be in some sense eliminable from the language of
a theory according to some convention (usually involving logical equivalence
or equality).  The most common convention is that any formula that was
syntactically valid but not provable before the definition was introduced will
not become provable after the definition is introduced.  In an ideal world,
definitions should not be present at all if one is to have absolute confidence
in a mathematical result.  However, they are necessary to make
mathematics practical, for otherwise the resulting formulas would be
extremely long and incomprehensible.  Since the nature of definitions (in the
most general sense) does not permit them to automatically be verified as
``proper,''\index{proper definition}\index{definition!proper} the judgment of
the mathematician is required to ensure it.  (In \texttt{set.mm} effort was made
to make almost all definitions directly eliminable and thus minimize the need
for such judgment.)

If you are not a mathematician, it may be best not to add or change any
\texttt{\$a}\index{\texttt{\$a} statement} statements but instead use
the mathematical language already provided in standard databases.  This
way Metamath will not allow you to make a mistake (i.e.\ prove a false
result).


\subsection{Frames}\label{frames}

We now introduce the concept of a collection of related Metamath statements
called a frame.  Every assertion (\texttt{\$a} or \texttt{\$p} statement) in the database has
an associated frame.

A {\bf frame}\index{frame} is a sequence of \texttt{\$d}, \texttt{\$f},
and \texttt{\$e} statements (zero or more of each) followed by one
\texttt{\$a} or \texttt{\$p} statement, subject to certain conditions we
will describe.  For simplicity we will assume that all math symbol
tokens used are declared at the beginning of the database with
\texttt{\$c} and \texttt{\$v} statements (which are not properly part of
a frame).  Also for simplicity we will assume there are only simple
\texttt{\$d} statements (those with only two variables) and imagine any
compound \texttt{\$d} statements (those with more than two variables) as
broken up into simple ones.

A frame groups together those hypotheses (and \texttt{\$d} statements) relevant
to an assertion (\texttt{\$a} or \texttt{\$p} statement).  The statements in a frame
may or may not be physically adjacent in a database; we will cover
this in our discussion of scoping statements
in Section~\ref{scoping}.

A frame has the following properties:
\begin{enumerate}
 \item The set of variables contained in its \texttt{\$f} statements must
be identical to the set of variables contained in its \texttt{\$e},
\texttt{\$a}, and/or \texttt{\$p} statements.  In other words, each
variable in a \texttt{\$e}, \texttt{\$a}, or \texttt{\$p} statement must
have an associated ``variable type'' defined for it in a \texttt{\$f}
statement.
  \item No two \texttt{\$f} statements may contain the same variable.
  \item Any \texttt{\$f} statement
must occur before a \texttt{\$e} statement in which its variable occurs.
\end{enumerate}

The first property determines the set of variables occurring in a frame.
These are the {\bf mandatory
variables}\index{mandatory variable} of the frame.  The second property
tells us there must be only one type specified for a variable.
The last property is not a theoretical requirement but it
makes parsing of the database easier.

For our examples, we assume our database has the following declarations:

\begin{verbatim}
$v P Q R $.
$c -> ( ) |- wff $.
\end{verbatim}

The following sequence of statements, describing the modus ponens inference
rule, is an example of a frame:

\begin{verbatim}
wp  $f wff P $.
wq  $f wff Q $.
maj $e |- ( P -> Q ) $.
min $e |- P $.
mp  $a |- Q $.
\end{verbatim}

The following sequence of statements is not a frame because \texttt{R} does not
occur in the \texttt{\$e}'s or the \texttt{\$a}:

\begin{verbatim}
wp  $f wff P $.
wq  $f wff Q $.
wr  $f wff R $.
maj $e |- ( P -> Q ) $.
min $e |- P $.
mp  $a |- Q $.
\end{verbatim}

The following sequence of statements is not a frame because \texttt{Q} does not
occur in a \texttt{\$f}:

\begin{verbatim}
wp  $f wff P $.
maj $e |- ( P -> Q ) $.
min $e |- P $.
mp  $a |- Q $.
\end{verbatim}

The following sequence of statements is not a frame because the \texttt{\$a} statement is
not the last one:

\begin{verbatim}
wp  $f wff P $.
wq  $f wff Q $.
maj $e |- ( P -> Q ) $.
mp  $a |- Q $.
min $e |- P $.
\end{verbatim}

Associated with a frame is a sequence of {\bf mandatory
hypotheses}\index{mandatory hypothesis}.  This is simply the set of all
\texttt{\$f} and \texttt{\$e} statements in the frame, in the order they
appear.  A frame can be referenced in a later proof using the label of
the \texttt{\$a} or \texttt{\$p} assertion statement, and the proof
makes an assignment to each mandatory hypothesis in the order in which
it appears.  This means the order of the hypotheses, once chosen, must
not be changed so as not to affect later proofs referencing the frame's
assertion statement.  (The Metamath proof verifier will, of course, flag
an error if a proof becomes incorrect by doing this.)  Since proofs make
use of ``Reverse Polish notation,'' described in Section~\ref{proof}, we
call this order the {\bf RPN order}\index{RPN order} of the hypotheses.

Note that \texttt{\$d} statements are not part of the set of mandatory
hypotheses, and their order doesn't matter (as long as they satisfy the
fourth property for a frame described above).  The \texttt{\$d}
statements specify restrictions on variables that must be satisfied (and
are checked by the proof verifier) when expressions are substituted for
them in a proof, and the \texttt{\$d} statements themselves are never
referenced directly in a proof.

A frame with a \texttt{\$p} (provable) statement requires a proof as part of the
\texttt{\$p} statement.  Sometimes in a proof we want to make use of temporary or
dummy variables\index{dummy variable} that do not occur in the \texttt{\$p}
statement or its mandatory hypotheses.  To accommodate this we define an {\bf
extended frame}\index{extended frame} as a frame together with zero or more
\texttt{\$d} and \texttt{\$f} statements that reference variables not among the
mandatory variables of the frame.  Any new variables referenced are called the
{\bf optional variables}\index{optional variable} of the extended frame. If a
\texttt{\$f} statement references an optional variable it is called an {\bf
optional hypothesis}\index{optional hypothesis}, and if one or both of the
variables in a \texttt{\$d} statement are optional variables it is called an {\bf
optional disjoint-variable restriction}\index{optional disjoint-variable
restriction}.  Properties 2, 3, and 4 for a frame also apply to an extended
frame.

The concept of optional variables is not meaningful for frames with \texttt{\$a}
statements, since those statements have no proofs that might make use of them.
There is no restriction on including optional hypotheses in the extended frame
for a \texttt{\$a} statement, but they serve no purpose.

The following set of statements is an example of an extended frame, which
contains an optional variable \texttt{R} and an optional hypothesis \texttt{wr}.  In
this example, we suppose the rule of modus ponens is not an axiom but is
derived as a theorem from earlier statements (we omit its presumed proof).
Variable \texttt{R} may be used in its proof if desired (although this would
probably have no advantage in propositional calculus).  Note that the sequence
of mandatory hypotheses in RPN order is still \texttt{wp}, \texttt{wq}, \texttt{maj},
\texttt{min} (i.e.\ \texttt{wr} is omitted), and this sequence is still assumed
whenever the assertion \texttt{mp} is referenced in a subsequent proof.

\begin{verbatim}
wp  $f wff P $.
wq  $f wff Q $.
wr  $f wff R $.
maj $e |- ( P -> Q ) $.
min $e |- P $.
mp  $p |- Q $= ... $.
\end{verbatim}

Every frame is an extended frame, but not every extended frame is a frame, as
this example shows.  The underlying frame for an extended frame is
obtained by simply removing all statements containing optional variables.
Any proof referencing an assertion will ignore any extensions to its
frame, which means we may add or delete optional hypotheses at will without
affecting subsequent proofs.

The conceptually simplest way of organizing a Metamath database is as a
sequence of extended frames.  The scoping statements
\texttt{\$\char`\{}\index{\texttt{\$\char`\{} and \texttt{\$\char`\}}
keywords} and \texttt{\$\char`\}} can be used to delimit the start and
end of an extended frame, leading to the following possible structure for a
database.  \label{framelist}

\vskip 2ex
\setbox\startprefix=\hbox{\tt \ \ \ \ \ \ \ \ }
\setbox\contprefix=\hbox{}
\startm
\m{\mbox{(\texttt{\$v} {\em and} \texttt{\$c}\,{\em statements})}}
\endm
\startm
\m{\mbox{\texttt{\$\char`\{}}}
\endm
\startm
\m{\mbox{\texttt{\ \ } {\em extended frame}}}
\endm
\startm
\m{\mbox{\texttt{\$\char`\}}}}
\endm
\startm
\m{\mbox{\texttt{\$\char`\{}}}
\endm
\startm
\m{\mbox{\texttt{\ \ } {\em extended frame}}}
\endm
\startm
\m{\mbox{\texttt{\$\char`\}}}}
\endm
\startm
\m{\mbox{\texttt{\ \ \ \ \ \ \ \ \ }}\vdots}
\endm
\vskip 2ex

In practice, this structure is inconvenient because we have to repeat
any \texttt{\$f}, \texttt{\$e}, and \texttt{\$d} statements over and
over again rather than stating them once for use by several assertions.
The scoping statements, which we will discuss next, allow this to be
done.  In principle, any Metamath database can be converted to the above
format, and the above format is the most convenient to use when studying
a Metamath database as a formal system%
%% Uncomment this when uncommenting section {formalspec} below
   (Appendix \ref{formalspec})%
.
In fact, Metamath internally converts the database to the above format.
The command \texttt{show statement} in the Metamath program will show
you the contents of the frame for any \texttt{\$a} or \texttt{\$p}
statement, as well as its extension in the case of a \texttt{\$p}
statement.

%c%(provided that all ``local'' variables and constants with limited scope have
%c%unique names),

During our discussion of scoping statements, it may be helpful to
think in terms of the equivalent sequence of frames that will result when
the database is parsed.  Scoping (other than the limited
use above to delimit frames) is not a theoretical requirement for
Metamath but makes it more convenient.


\subsection{Scoping Statements (\texttt{\$\{} and \texttt{\$\}})}\label{scoping}
\index{\texttt{\$\char`\{} and \texttt{\$\char`\}} keywords}\index{scoping statement}

%c%Some Metamath statements may be needed only temporarily to
%c%serve a specific purpose, and after we're done with them we would like to
%c%disregard or ignore them.  For example, when we're finished using a variable,
%c%we might want to
%c%we might want to free up the token\index{token} used to name it so that the
%c%token can be used for other purposes later on, such as a different kind of
%c%variable or even a constant.  In the terminology of computer programming, we
%c%might want to let some symbol declarations be ``local'' rather than ``global.''
%c%\index{local symbol}\index{global symbol}

The {\bf scoping} statements, \texttt{\$\char`\{} ({\bf start of block}) and \texttt{\$\char`\}}
({\bf end of block})\index{block}, provide a means for controlling the portion
of a database over which certain statement types are recognized.  The
syntax of a scoping statement is very simple; it just consists of the
statement's keyword:
\begin{center}
\texttt{\$\char`\{}\\
\texttt{\$\char`\}}
\end{center}
\index{\texttt{\$\char`\{} and \texttt{\$\char`\}} keywords}

For example, consider the following database where we have stripped out
all tokens except the scoping statement keywords.  For the purpose of the
discussion, we have added subscripts to the scoping statements; these subscripts
do not appear in the actual database.
\[
 \mbox{\tt \ \$\char`\{}_1
 \mbox{\tt \ \$\char`\{}_2
 \mbox{\tt \ \$\char`\}}_2
 \mbox{\tt \ \$\char`\{}_3
 \mbox{\tt \ \$\char`\{}_4
 \mbox{\tt \ \$\char`\}}_4
 \mbox{\tt \ \$\char`\}}_3
 \mbox{\tt \ \$\char`\}}_1
\]
Each \texttt{\$\char`\{} statement in this example is said to be {\bf
matched} with the \texttt{\$\char`\}} statement that has the same
subscript.  Each pair of matched scoping statements defines a region of
the database called a {\bf block}.\index{block} Blocks can be {\bf
nested}\index{nested block} inside other blocks; in the example, the
block defined by $\mbox{\tt \$\char`\{}_4$ and $\mbox{\tt \$\char`\}}_4$
is nested inside the block defined by $\mbox{\tt \$\char`\{}_3$ and
$\mbox{\tt \$\char`\}}_3$ as well as inside the block defined by
$\mbox{\tt \$\char`\{}_1$ and $\mbox{\tt \$\char`\}}_1$.  In general, a
block may be empty, it may contain only non-scoping
statements,\footnote{Those statements other than \texttt{\$\char`\{} and
\texttt{\$\char`\}}.}\index{non-scoping statement} or it may contain any
mixture of other blocks and non-scoping statements.  (This is called a
``recursive'' definition\index{recursive definition} of a block.)

Associated with each block is a number called its {\bf nesting
level}\index{nesting level} that indicates how deeply the block is nested.
The nesting levels of the blocks in our example are as follows:
\[
  \underbrace{
    \mbox{\tt \ }
    \underbrace{
     \mbox{\tt \$\char`\{\ }
     \underbrace{
       \mbox{\tt \$\char`\{\ }
       \mbox{\tt \$\char`\}}
     }_{2}
     \mbox{\tt \ }
     \underbrace{
       \mbox{\tt \$\char`\{\ }
       \underbrace{
         \mbox{\tt \$\char`\{\ }
         \mbox{\tt \$\char`\}}
       }_{3}
       \mbox{\tt \ \$\char`\}}
     }_{2}
     \mbox{\tt \ \$\char`\}}
   }_{1}
   \mbox{\tt \ }
 }_{0}
\]
\index{\texttt{\$\char`\{} and \texttt{\$\char`\}} keywords}
The entire database is considered to be one big block (the {\bf outermost}
block) with a nesting level of 0.  The outermost block is {\em not} bracketed
by scoping statements.\footnote{The language was designed this way so that
several source files can be joined together more easily.}\index{outermost
block}

All non-scoping Metamath statements become recognized or {\bf
active}\index{active statement} at the place where they appear.\footnote{To
keep things slightly simpler, we do not bother to define the concept of
``active'' for the scoping statements.}  Certain of these statement types
become inactive at the end of the block in which they appear; these statement
types are:
\begin{center}
  \texttt{\$c}, \texttt{\$v}, \texttt{\$d}, \texttt{\$e}, and \texttt{\$f}.
%  \texttt{\$v}, \texttt{\$f}, \texttt{\$e}, and \texttt{\$d}.
\end{center}
\index{\texttt{\$c} statement}
\index{\texttt{\$d} statement}
\index{\texttt{\$e} statement}
\index{\texttt{\$f} statement}
\index{\texttt{\$v} statement}
The other statement types remain active forever (i.e.\ through the end of the
database); they are:
\begin{center}
  \texttt{\$a} and \texttt{\$p}.
%  \texttt{\$c}, \texttt{\$a}, and \texttt{\$p}.
\end{center}
\index{\texttt{\$a} statement}
\index{\texttt{\$p} statement}
Any statement (of these 7 types) located in the outermost
block\index{outermost block} will remain active through the end of the
database and thus are effectively ``global'' statements.\index{global
statement}

All \texttt{\$c} statements must be placed in the outermost block.  Since they are
therefore always global, they could be considered as belonging to both of the
above categories.

The {\bf scope}\index{scope} of a statement is the set of statements that
recognize it as active.

%c%The concept of ``active'' is also defined for math symbols\index{math
%c%symbol}.  Math symbols (constants\index{constant} and
%c%variables\index{variable}) become {\bf active}\index{active
%c%math symbol} in the \texttt{\$c}\index{\texttt{\$c}
%c%statement} and \texttt{\$v}\index{\texttt{\$v} statement} statements that
%c%declare them.  They become inactive when their declaration statements become
%c%inactive.

The concept of ``active'' is also defined for math symbols\index{math
symbol}.  Math symbols (constants\index{constant} and
variables\index{variable}) become {\bf active}\index{active math symbol}
in the \texttt{\$c}\index{\texttt{\$c} statement} and
\texttt{\$v}\index{\texttt{\$v} statement} statements that declare them.
A variable becomes inactive when its declaration statement becomes
inactive.  Because all \texttt{\$c} statements must be in the outermost
block, a constant will never become inactive after it is declared.

\subsubsection{Redeclaration of Math Symbols}
\index{redeclaration of symbols}\label{redeclaration}

%c%A math symbol may not be declared a second time while it is active, but it may
%c%be declared again after it becomes inactive.

A variable may not be declared a second time while it is active, but it may be
declared again after it becomes inactive.  This provides a convenient way to
introduce ``local'' variables,\index{local variable} i.e.\ temporary variables
for use in the frame of an assertion or in a proof without keeping them around
forever.  A previously declared variable may not be redeclared as a constant.

A constant may not be redeclared.  And, as mentioned above, constants must be
declared in the outermost block.

The reason variables may have limited scope but not constants is that an
assertion (\texttt{\$a} or \texttt{\$p} statement) remains available for use in
proofs through the end of the database.  Variables in an assertion's frame may
be substituted with whatever is needed in a proof step that references the
assertion, whereas constants remain fixed and may not be substituted with
anything.  The particular token used for a variable in an assertion's frame is
irrelevant when the assertion is referenced in a proof, and it doesn't matter
if that token is not available outside of the referenced assertion's frame.
Constants, however, must be globally fixed.

In the present version of the Metamath language, there is no theoretical
benefit for the feature allowing variables to be active for limited scopes
rather than global. It is just a convenience that allows them, for example, to
be locally grouped together with their corresponding \texttt{\$f} variable-type
declarations.\footnote{A future version\label{redeclarationf} of the Metamath
language may extend the language to include true local variables as well as
local constants.  I have not decided upon the best way to do this or even if
it should be done at all; the main issue is whether the benefits offset
greater confusion in learning the language.  In any case, the rules for
the present language were chosen to let existing databases be compatible with
any future language extensions.}

%c%If you declare a math symbol more than once, internally Metamath considers it a
%c%new distinct symbol, even though it has the same name.  If you are unaware of
%c%this, you may find that what you think are correct proofs are incorrectly
%c%rejected as invalid, because Metamath may tell you that a constant you
%c%previously declared does not match a newly declared math symbol with the same
%c%name.  For details on this subtle point, see the Comment on
%c%p.~\pageref{spec4comment}.  This is done purposely to allow temporary
%c%constants to be introduced while developing a subtheory, then allow their math
%c%symbol tokens to be reused later on; in general they will not refer to the
%c%same thing.  In practice, you would not ordinarily reuse the names of
%c%constants because it would tend to be confusing to the reader.  The reuse of
%c%names of variables, on the other hand, is something that is often useful to do
%c%(for example it is done frequently in \texttt{set.mm}).  Since variables in an
%c%assertion referenced in a proof can be substituted as needed to achieve a
%c%symbol match, this is not an issue.

% (This section covers a somewhat advanced topic you may want to skip
% at first reading.)
%
% Under certain circumstances, math symbol\index{math symbol}
% tokens\index{token} may be redeclared (i.e.\ the token
% may appear in more than
% one \texttt{\$c}\index{\texttt{\$c} statement} or \texttt{\$v}\index{\texttt{\$v}
% statement} statement).  You might want to do this say, to make temporary use
% of a variable name without having to worry about its affect elsewhere,
% somewhat analogous to declaring a local variable in a standard computer
% language.  Understanding what goes on when math symbol tokens are redeclared
% is a little tricky to understand at first, since it requires that we
% distinguish the token itself from the math symbol that it names.  It will help
% if we first take a peek at the internal workings of the
% Metamath\index{Metamath} program.
%
% Metamath reserves a memory location for each occurrence of a
% token\index{token} in a declaration statement (\texttt{\$c}\index{\texttt{\$c}
% statement} or \texttt{\$v}\index{\texttt{\$v} statement}).  If a given token appears
% in more than one declaration statement, it will refer to more than one memory
% locations.  A math symbol\index{math symbol} may be thought of as being one of
% these memory locations rather than as the token itself.  Only one of the
% memory locations associated with a given token may be active at any one time.
% The math symbol (memory location) that gets looked up when the token appears
% in a non-declaration statement is the one that happens to be active at that
% time.
%
% We now look at the rules for the redeclaration\index{redeclaration of symbols}
% of math symbol tokens.
% \begin{itemize}
% \item A math symbol token may not be declared twice in the
% same block.\footnote{While there is no theoretical reason for disallowing
% this, it was decided in the design of Metamath that allowing it would offer no
% advantage and might cause confusion.}
% \item An inactive math symbol may always be
% redeclared.
% \item  An active math symbol may be redeclared in a different (i.e.\
% inner) block\index{block} from the one it became active in.
% \end{itemize}
%
% When a math symbol token is redeclared, it conceptually refers to a different
% math symbol, just as it would be if it were called a different name.  In
% addition, the original math symbol that it referred to, if it was active,
% temporarily becomes inactive.  At the end of the block in which the
% redeclaration occurred, the new math symbol\index{math symbol} becomes
% inactive and the original symbol becomes active again.  This concept is
% illustrated in the following example, where the symbol \texttt{e} is
% ordinarily a constant (say Euler's constant, 2.71828...) but
% temporarily we want to use it as a ``local'' variable, say as a coefficient
% in the equation $a x^4 + b x^3 + c x^2 + d x + e$:
% \[
%   \mbox{\tt \$\char`\{\ \$c e \$.}
%   \underbrace{
%     \ \ldots\ %
%     \mbox{\tt \$\char`\{}\ \ldots\ %
%   }_{\mbox{\rm region A}}
%   \mbox{\tt \$v e \$.}
%   \underbrace{
%     \mbox{\ \ \ \ldots\ \ \ }
%   }_{\mbox{\rm region B}}
%   \mbox{\tt \$\char`\}}
%   \underbrace{
%     \mbox{\ \ \ \ldots\ \ \ }
%   }_{\mbox{\rm region C}}
%   \mbox{\tt \$\char`\}}
% \]
% \index{\texttt{\$\char`\{} and \texttt{\$\char`\}} keywords}
% In region A, the token \texttt{e} refers to a constant.  It is redeclared as a
% variable in region B, and any reference to it in this region will refer to this
% variable.  In region C, the redeclaration becomes inactive, and the original
% declaration becomes active again.  In region C, the token \texttt{x} refers to the
% original constant.
%
% As a practical matter, overuse of math symbol\index{math symbol}
% redeclarations\index{redeclaration of symbols} can be confusing (even though
% it is well-defined) and is best avoided when possible.  Here are some good
% general guidelines you can follow.  Usually, you should declare all
% constants\index{constant} in the outermost block\index{outermost block},
% especially if they are general-purpose (such as the token \verb$A.$, meaning
% $\forall$ or ``for all'').  This will make them ``globally'' active (although
% as in the example above local redeclarations will temporarily make them
% inactive.)  Most or all variables\index{variable}, on the other hand, could be
% declared in inner blocks, so that the token for them can be used later for a
% different type of variable or a constant.  (The names of the variables you
% choose are not used when you refer to an assertion\index{assertion} in a
% proof, whereas constants must match exactly.  A locally declared constant will
% not match a globally declared constant in a proof, even if they use the same
% token, because Metamath internally considers them to be different math
% symbols.)  To avoid confusion, you should generally avoid redeclaring active
% variables.  If you must redeclare them, do so at the beginning of a block.
% The temporary declaration of constants in inner blocks might be occasionally
% appropriate when you make use of a temporary definition to prove lemmas
% leading to a main result that does not make direct use of the definition.
% This way, you will not clutter up your database with a large number of
% seldom-used global constant symbols.  You might want to note that while
% inactive constants may not appear directly in an assertion (a \texttt{\$a}\index{\texttt{\$a}
% statement} or \texttt{\$p}\index{\texttt{\$p} statement}
% statement), they may be indirectly used in the proof of a \texttt{\$p} statement
% so long as they do not appear in the final math symbol sequence constructed by
% the proof.  In the end, you will have to use your best judgment, taking into
% account standard mathematical usage of the symbols as well as consideration
% for the reader of your work.
%
% \subsubsection{Reuse of Labels}\index{reuse of labels}\index{label}
%
% The \texttt{\$e}\index{\texttt{\$e} statement}, \texttt{\$f}\index{\texttt{\$f}
% statement}, \texttt{\$a}\index{\texttt{\$a} statement}, and
% \texttt{\$p}\index{\texttt{\$p}
% statement} statement types require labels, which allow them to be
% referenced later inside proofs.  A label is considered {\bf
% active}\index{active label} when the statement it is associated with is
% active.  The token\index{token} for a label may be reused
% (redeclared)\index{redeclaration of labels} provided that it is not being used
% for a currently active label.  (Unlike the tokens for math symbols, active
% label tokens may not be redeclared in an inner scope.)  Note that the labels
% of \texttt{\$a} and \texttt{\$p} statements can never be reused after these
% statements appear, because these statements remain active through the end of
% the database.
%
% You might find the reuse of labels a convenient way to have standard names for
% temporary hypotheses, such as \texttt{h1}, \texttt{h2}, etc.  This way you don't have
% to invent unique names for each of them, and in some cases it may be less
% confusing to the reader (although in other cases it might be more confusing, if
% the hypothesis is located far away from the assertion that uses
% it).\footnote{The current implementation requires that all labels, even
% inactive ones, be unique.}

\subsubsection{Frames Revisited}\index{frames and scoping statements}

Now that we have covered scoping, we will look at how an arbitrary
Metamath database can be converted to the simple sequence of extended
frames described on p.~\pageref{framelist}.  This is also how Metamath
stores the database internally when it reads in the database
source.\label{frameconvert} The method is simple.  First, we collect all
constant and variable (\texttt{\$c} and \texttt{\$v}) declarations in
the database, ignoring duplicate declarations of the same variable in
different scopes.  We then put our collected \texttt{\$c} and
\texttt{\$v} declarations at the beginning of the database, so that
their scope is the entire database.  Next, for each assertion in the
database, we determine its frame and extended frame.  The extended frame
is simply the \texttt{\$f}, \texttt{\$e}, and \texttt{\$d} statements
that are active.  The frame is the extended frame with all optional
hypotheses removed.

An equivalent way of saying this is that the extended frame of an assertion
is the collection of all \texttt{\$f}, \texttt{\$e}, and \texttt{\$d} statements
whose scope includes the assertion.
The \texttt{\$f} and \texttt{\$e} statements
occur in the order they appear
(order is irrelevant for \texttt{\$d} statements).

%c%, renaming any
%c%redeclared variables as needed so that all of them have unique names.  (The
%c%exact renaming convention is unimportant.  You might imagine renaming
%c%different declarations of math symbol \texttt{a} as \texttt{a\$1}, \texttt{a\$2}, etc.\
%c%which would prevent any conflicts since \texttt{\$} is not a legal character in a
%c%math symbol token.)

\section{The Anatomy of a Proof} \label{proof}
\index{proof!Metamath, description of}

Each provable assertion (\texttt{\$p}\index{\texttt{\$p} statement} statement) in a
database must include a {\bf proof}\index{proof}.  The proof is located
between the \texttt{\$=}\index{\texttt{\$=} keyword} and \texttt{\$.}\ keywords in the
\texttt{\$p} statement.

In the basic Metamath language\index{basic language}, a proof is a
sequence of statement labels.  This label sequence\index{label sequence}
serves as a set of instructions that the Metamath program uses to
construct a series of math symbol sequences.  The construction must
ultimately result in the math symbol sequence contained between the
\texttt{\$p}\index{\texttt{\$p} statement} and
\texttt{\$=}\index{\texttt{\$=} keyword} keywords of the \texttt{\$p}
statement.  Otherwise, the Metamath program will consider the proof
incorrect, and it will notify you with an appropriate error message when
you ask it to verify the proof.\footnote{To make the loading faster, the
Metamath program does not automatically verify proofs when you
\texttt{read} in a database unless you use the \texttt{/verify}
qualifier.  After a database has been read in, you may use the
\texttt{verify proof *} command to verify proofs.}\index{\texttt{verify
proof} command} Each label in a proof is said to {\bf
reference}\index{label reference} its corresponding statement.

Associated with any assertion\index{assertion} (\texttt{\$p} or
\texttt{\$a}\index{\texttt{\$a} statement} statement) is a set of
hypotheses (\texttt{\$f}\index{\texttt{\$f} statement} or
\texttt{\$e}\index{\texttt{\$e} statement} statements) that are active
with respect to that assertion.  Some are mandatory and the others are
optional.  You should review these concepts if necessary.

Each label\index{label} in a proof must be either the label of a
previous assertion (\texttt{\$a}\index{\texttt{\$a} statement} or
\texttt{\$p}\index{\texttt{\$p} statement} statement) or the label of an
active hypothesis (\texttt{\$e} or \texttt{\$f}\index{\texttt{\$f}
statement} statement) of the \texttt{\$p} statement containing the
proof.  Hypothesis labels may reference both the
mandatory\index{mandatory hypothesis} and the optional hypotheses of the
\texttt{\$p} statement.

The label sequence in a proof specifies a construction in {\bf reverse Polish
notation}\index{reverse Polish notation (RPN)} (RPN).  You may be familiar
with RPN if you have used older
Hewlett--Packard or similar hand-held calculators.
In the calculator analogy, a hypothesis label\index{hypothesis label} is like
a number and an assertion label\index{assertion label} is like an operation
(more precisely, an $n$-ary operation when the
assertion has $n$ \texttt{\$e}-hypotheses).
On an RPN calculator, an operation takes one or more previous numbers in an
input sequence, performs a calculation on them, and replaces those numbers and
itself with the result of the calculation.  For example, the input sequence
$2,3,+$ on an RPN calculator results in $5$, and the input sequence
$2,3,5,{\times},+$ results in $2,15,+$ which results in $17$.

Understanding how RPN is processed involves the concept of a {\bf
stack}\index{stack}\index{RPN stack}, which can be thought of as a set of
temporary memory locations that hold intermediate results.  When Metamath
encounters a hypothesis label it places or {\bf pushes}\index{push} the math
symbol sequence of the hypothesis onto the stack.  When Metamath encounters an
assertion label, it associates the most recent stack entries with the {\em
mandatory} hypotheses\index{mandatory hypothesis} of the assertion, in the
order where the most recent stack entry is associated with the last mandatory
hypothesis of the assertion.  It then determines what
substitutions\index{substitution!variable}\index{variable substitution} have
to be made into the variables of the assertion's mandatory hypotheses to make
them identical to the associated stack entries.  It then makes those same
substitutions into the assertion itself.  Finally, Metamath removes or {\bf
pops}\index{pop} the matched hypotheses from the stack and pushes the
substituted assertion onto the stack.

For the purpose of matching the mandatory hypothesis to the most recent stack
entries, whether a hypothesis is a \texttt{\$e} or \texttt{\$f} statement is
irrelevant.  The only important thing is that a set of
substitutions\footnote{In the Metamath spec (Section~\ref{spec}), we use the
singular term ``substitution'' to refer to the set of substitutions we talk
about here.} exist that allow a match (and if they don't, the proof verifier
will let you know with an error message).  The Metamath language is specified
in such a way that if a set of substitutions exists, it will be unique.
Specifically, the requirement that each variable have a type specified for it
with a \texttt{\$f} statement ensures the uniqueness.

We will illustrate this with an example.
Consider the following Metamath source file:
\begin{verbatim}
$c ( ) -> wff $.
$v p q r s $.
wp $f wff p $.
wq $f wff q $.
wr $f wff r $.
ws $f wff s $.
w2 $a wff ( p -> q ) $.
wnew $p wff ( s -> ( r -> p ) ) $= ws wr wp w2 w2 $.
\end{verbatim}
This Metamath source example shows the definition and ``proof'' (i.e.,
construction) of a well-formed formula (wff)\index{well-formed formula (wff)}
in propositional calculus.  (You may wish to type this example into a file to
experiment with the Metamath program.)  The first two statements declare
(introduce the names of) four constants and four variables.  The next four
statements specify the variable types, namely that
each variable is assumed to be a wff.  Statement \texttt{w2} defines (postulates)
a way to produce a new wff, \texttt{( p -> q )}, from two given wffs \texttt{p} and
\texttt{q}. The mandatory hypotheses of \texttt{w2} are \texttt{wp} and \texttt{wq}.
Statement \texttt{wnew} claims that \texttt{( s -> ( r -> p ) )} is a wff given
three wffs \texttt{s}, \texttt{r}, and \texttt{p}.  More precisely, \texttt{wnew} claims
that the sequence of ten symbols \texttt{wff ( s -> ( r -> p ) )} is provable from
previous assertions and the hypotheses of \texttt{wnew}.  Metamath does not know
or care what a wff is, and as far as it is concerned
the typecode \texttt{wff} is just an
arbitrary constant symbol in a math symbol sequence.  The mandatory hypotheses
of \texttt{wnew} are \texttt{wp}, \texttt{wr}, and \texttt{ws}; \texttt{wq} is an optional
hypothesis.  In our particular proof, the optional hypothesis is not
referenced, but in general, any combination of active (i.e.\ optional and
mandatory) hypotheses could be referenced.  The proof of statement \texttt{wnew}
is the sequence of five labels starting with \texttt{ws} (step~1) and ending with
\texttt{w2} (step~5).

When Metamath verifies the proof, it scans the proof from left to right.  We
will examine what happens at each step of the proof.  The stack starts off
empty.  At step 1, Metamath looks up label \texttt{ws} and determines that it is a
hypothesis, so it pushes the symbol sequence of statement \texttt{ws} onto the
stack:

\begin{center}\begin{tabular}{|l|l|}\hline
{Stack location} & {Contents} \\ \hline \hline
1 & \texttt{wff s} \\ \hline
\end{tabular}\end{center}

Metamath sees that the labels \texttt{wr} and \texttt{wp} in steps~2 and 3 are also
hypotheses, so it pushes them onto the stack.  After step~3, the stack looks
like
this:

\begin{center}\begin{tabular}{|l|l|}\hline
{Stack location} & {Contents} \\ \hline \hline
3 & \texttt{wff p} \\ \hline
2 & \texttt{wff r} \\ \hline
1 & \texttt{wff s} \\ \hline
\end{tabular}\end{center}

At step 4, Metamath sees that label \texttt{w2} is an assertion, so it must do
some processing.  First, it associates the mandatory hypotheses of \texttt{w2},
which are \texttt{wp} and \texttt{wq}, with stack locations~2 and 3, {\em in that
order}. Metamath determines that the only possible way
to make hypothesis \texttt{wp} match (become identical to) stack location~2 and
\texttt{wq} match stack location 3 is to substitute variable \texttt{p} with \texttt{r}
and \texttt{q} with \texttt{p}.  Metamath makes these substitutions into \texttt{w2} and
obtains the symbol sequence \texttt{wff ( r -> p )}.  It removes the hypotheses
from stack locations~2 and 3, then places the result into stack location~2:

\begin{center}\begin{tabular}{|l|l|}\hline
{Stack location} & {Contents} \\ \hline \hline
2 & \texttt{wff ( r -> p )} \\ \hline
1 & \texttt{wff s} \\ \hline
\end{tabular}\end{center}

At step 5, Metamath sees that label \texttt{w2} is an assertion, so it must again
do some processing.  First, it matches the mandatory hypotheses of \texttt{w2},
which are \texttt{wp} and \texttt{wq}, to stack locations 1 and 2.
Metamath determines that the only possible way to make the
hypotheses match is to substitute variable \texttt{p} with \texttt{s} and \texttt{q} with
\texttt{( r -> p )}.  Metamath makes these substitutions into \texttt{w2} and obtains
the symbol
sequence \texttt{wff ( s -> ( r -> p ) )}.  It removes stack
locations 1 and 2, then places the result into stack location~1:

\begin{center}\begin{tabular}{|l|l|}\hline
{Stack location} & {Contents} \\ \hline \hline
1 & \texttt{wff ( s -> ( r -> p ) )} \\ \hline
\end{tabular}\end{center}

After Metamath finishes processing the proof, it checks to see that the
stack contains exactly one element and that this element is
the same as the math symbol sequence in the
\texttt{\$p}\index{\texttt{\$p} statement} statement.  This is the case for our
proof of \texttt{wnew},
so we have proved \texttt{wnew} successfully.  If the result
differs, Metamath will notify you with an error message.  An error message
will also result if the stack contains more than one entry at the end of the
proof, or if the stack did not contain enough entries at any point in the
proof to match all of the mandatory hypotheses\index{mandatory hypothesis} of
an assertion.  Finally, Metamath will notify you with an error message if no
substitution is possible that will make a referenced assertion's hypothesis
match the
stack entries.  You may want to experiment with the different kinds of errors
that Metamath will detect by making some small changes in the proof of our
example.

Metamath's proof notation was designed primarily to express proofs in a
relatively compact manner, not for readability by humans.  Metamath can display
proofs in a number of different ways with the \texttt{show proof}\index{\texttt{show
proof} command} command.  The
\texttt{/lemmon} qualifier displays it in a format that is easier to read when the
proofs are short, and you saw examples of its use in Chapter~\ref{using}.  For
longer proofs, it is useful to see the tree structure of the proof.  A tree
structure is displayed when the \texttt{/lemmon} qualifier is omitted.  You will
probably find this display more convenient as you get used to it. The tree
display of the proof in our example looks like
this:\label{treeproof}\index{tree-style proof}\index{proof!tree-style}
\begin{verbatim}
1     wp=ws    $f wff s
2        wp=wr    $f wff r
3        wq=wp    $f wff p
4     wq=w2    $a wff ( r -> p )
5  wnew=w2  $a wff ( s -> ( r -> p ) )
\end{verbatim}
The number to the left of each line is the step number.  Following it is a
{\bf hypothesis association}\index{hypothesis association}, consisting of two
labels\index{label} separated by \texttt{=}.  To the left of the \texttt{=} (except
in the last step) is the label of a hypothesis of an assertion referenced
later in the proof; here, steps 1 and 4 are the hypothesis associations for
the assertion \texttt{w2} that is referenced in step 5.  A hypothesis association
is indented one level more than the assertion that uses it, so it is easy to
find the corresponding assertion by moving directly down until the indentation
level decreases to one less than where you started from.  To the right of each
\texttt{=} is the proof step label for that proof step.  The statement keyword of
the proof step label is listed next, followed by the content of the top of the
stack (the most recent stack entry) as it exists after that proof step is
processed.  With a little practice, you should have no trouble reading proofs
displayed in this format.

Metamath proofs include the syntax construction of a formula.
In standard mathematics, this kind of
construction is not considered a proper part of the proof at all, and it
certainly becomes rather boring after a while.
Therefore,
by default the \texttt{show proof}\index{\texttt{show proof}
command} command does not show the syntax construction.
Historically \texttt{show proof} command
\textit{did} show the syntax construction, and you needed to add the
\texttt{/essential} option to hide, them, but today
\texttt{/essential} is the default and you need to use
\texttt{/all} to see the syntax constructions.

When verifying a proof, Metamath will check that no mandatory
\texttt{\$d}\index{\texttt{\$d} statement}\index{mandatory \texttt{\$d}
statement} statement of an assertion referenced in a proof is violated
when substitutions\index{substitution!variable}\index{variable
substitution} are made to the variables in the assertion.  For details
see Section~\ref{spec4} or \ref{dollard}.

\subsection{The Concept of Unification} \label{unify}

During the course of verifying a proof, when Metamath\index{Metamath}
encounters an assertion label\index{assertion label}, it associates the
mandatory hypotheses\index{mandatory hypothesis} of the assertion with the top
entries of the RPN stack\index{stack}\index{RPN stack}.  Metamath then
determines what substitutions\index{substitution!variable}\index{variable
substitution} it must make to the variables in the assertion's mandatory
hypotheses in order for these hypotheses to become identical to their
corresponding stack entries.  This process is called {\bf
unification}\index{unification}.  (We also informally use the term
``unification'' to refer to a set of substitutions that results from the
process, as in ``two unifications are possible.'')  After the substitutions
are made, the hypotheses are said to be {\bf unified}.

If no such substitutions are possible, Metamath will consider the proof
incorrect and notify you with an error message.
% (deleted 3/10/07, per suggestion of Mel O'Cat:)
% The syntax of the
% Metamath language ensures that if a set of substitutions exists, it
% will be unique.

The general algorithm for unification described in the literature is
somewhat complex.
However, in the case of Metamath it is intentionally trivial.
Mandatory hypotheses must be
pushed on the proof stack in the order in which they appear.
In addition, each variable must have its type specified
with a \texttt{\$f} hypothesis before it is used
and that each \texttt{\$f} hypothesis
have the restricted syntax of a typecode (a constant) followed by a variable.
The typecode in the \texttt{\$f} hypothesis must match the first symbol of
the corresponding RPN stack entry (which will also be a constant), so
the only possible match for the variable in the \texttt{\$f} hypothesis is
the sequence of symbols in the stack entry after the initial constant.

In the Proof Assistant\index{Proof Assistant}, a more general unification
algorithm is used.  While a proof is being developed, sometimes not enough
information is available to determine a unique unification.  In this case
Metamath will ask you to pick the correct one.\index{ambiguous
unification}\index{unification!ambiguous}

\section{Extensions to the Metamath Language}\index{extended
language}

\subsection{Comments in the Metamath Language}\label{comments}
\index{markup notation}
\index{comments!markup notation}

The commenting feature allows you to annotate the contents of
a database.  Just as with most
computer languages, comments are ignored for the purpose of interpreting the
contents of the database. Comments effectively act as
additional white space\index{white
space} between tokens
when a database is parsed.

A comment may be placed at the beginning, end, or
between any two tokens\index{token} in a source file.

Comments have the following syntax:
\begin{center}
 \texttt{\$(} {\em text} \texttt{\$)}
\end{center}
Here,\index{\texttt{\$(} and \texttt{\$)} auxiliary
keywords}\index{comment} {\em text} is a string, possibly empty, of any
characters in Metamath's character set (p.~\pageref{spec1chars}), except
that the character strings \texttt{\$(} and \texttt{\$)} may not appear
in {\em text}.  Thus nested comments are not
permitted:\footnote{Computer languages have differing standards for
nested comments, and rather than picking one it was felt simplest not to
allow them at all, at least in the current version of
Metamath\index{Metamath!limitations of version 0.07.30}.} Metamath will
complain if you give it
\begin{center}
 \texttt{\$( This is a \$( nested \$) comment.\ \$)}
\end{center}
To compensate for this non-nesting behavior, I often change all \texttt{\$}'s
to \texttt{@}'s in sections of Metamath code I wish to comment out.

The Metamath program supports a number of markup mechanisms and conventions
to generate good-looking results in \LaTeX\ and {\sc html},
as discussed below.
These markup features have to do only with how the comments are typeset,
and have no effect on how Metamath verifies the proofs in the database.
The improper
use of them may result in incorrectly typeset output, but no Metamath
error messages will result during the \texttt{read} and \texttt{verify
proof} commands.  (However, the \texttt{write
theorem\texttt{\char`\_}list} command
will check for markup errors as a side-effect of its
{\sc html} generation.)
Section~\ref{texout} has instructions for creating \LaTeX\ output, and
section~\ref{htmlout} has instructions for creating
{\sc html}\index{HTML} output.

\subsubsection{Headings}\label{commentheadings}

If the \texttt{\$(} is immediately followed by a new line
starting with a heading marker, it is a header.
This can start with:

\begin{itemize}
 \item[] \texttt{\#\#\#\#} - major part header
 \item[] \texttt{\#*\#*} - section header
 \item[] \texttt{=-=-} - subsection header
 \item[] \texttt{-.-.} - subsubsection header
\end{itemize}

The line following the marker line
will be used for the table of contents entry, after trimming spaces.
The next line should be another (closing) matching marker line.
Any text after that
but before the closing \texttt{\$}, such as an extended description of the
section, will be included on the \texttt{mmtheoremsNNN.html} page.

For more information, run
\texttt{help write theorem\char`\_list}.

\subsubsection{Math mode}
\label{mathcomments}
\index{\texttt{`} inside comments}
\index{\texttt{\char`\~} inside comments}
\index{math mode}

Inside of comments, a string of tokens\index{token} enclosed in
grave accents\index{grave accent (\texttt{`})} (\texttt{`}) will be converted
to standard mathematical symbols during
{\sc HTML}\index{HTML} or \LaTeX\ output
typesetting,\index{latex@{\LaTeX}} according to the information in the
special \texttt{\$t}\index{\texttt{\$t} comment}\index{typesetting
comment} comment in the database
(see section~\ref{tcomment} for information about the typesetting
comment, and Appendix~\ref{ASCII} to see examples of its results).

The first grave accent\index{grave accent (\texttt{`})} \texttt{`}
causes the output processor to enter {\bf math mode}\index{math mode}
and the second one exits it.
In this
mode, the characters following the \texttt{`} are interpreted as a
sequence of math symbol tokens separated by white space\index{white
space}.  The tokens are looked up in the \texttt{\$t}
comment\index{\texttt{\$t} comment}\index{typesetting comment} and if
found, they will be replaced by the standard mathematical symbols that
they correspond to before being placed in the typeset output file.  If
not found, the symbol will be output as is and a warning will be issued.
The tokens do not have to be active in the database, although a warning
will be issued if they are not declared with \texttt{\$c} or
\texttt{\$v} statements.

Two consecutive
grave accents \texttt{``} are treated as a single actual grave accent
(both inside and outside of math mode) and will not cause the output
processor to enter or exit math mode.

Here is an example of its use\index{Pierce's axiom}:
\begin{center}
\texttt{\$( Pierce's axiom, ` ( ( ph -> ps ) -> ph ) -> ph ` ,\\
         is not very intuitive. \$)}
\end{center}
becomes
\begin{center}
   \texttt{\$(} Pierce's axiom, $((\varphi \rightarrow \psi)\rightarrow
\varphi)\rightarrow \varphi$, is not very intuitive. \texttt{\$)}
\end{center}

Note that the math symbol tokens\index{token} must be surrounded by white
space\index{white space}.
%, since there is no context that allows ambiguity to be
%resolved, as is the case with math symbol sequences in some of the Metamath
%statements.
White space should also surround the \texttt{`}
delimiters.

The math mode feature also gives you a quick and easy way to generate
text containing mathematical symbols, independently of the intended
purpose of Metamath.\index{Metamath!using as a math editor} To do this,
simply create your text with grave accents surrounding your formulas,
after making sure that your math symbols are mapped to \LaTeX\ symbols
as described in Appendix~\ref{ASCII}.  It is easier if you start with a
database with predefined symbols such as \texttt{set.mm}.  Use your
grave-quoted math string to replace an existing comment, then typeset
the statement corresponding to that comment following the instructions
from the \texttt{help tex} command in the Metamath program.  You will
then probably want to edit the resulting file with a text editor to fine
tune it to your exact needs.

\subsubsection{Label Mode}\index{label mode}

Outside of math mode, a tilde\index{tilde (\texttt{\char`\~})} \verb/~/
indicates to Metamath's\index{Metamath} output processor that the
token\index{token} that follows (i.e.\ the characters up to the next
white space\index{white space}) represents a statement label or URL.
This formatting mode is called {\bf label mode}\index{label mode}.
If a literal tilde
is desired (outside of math mode) instead of label mode,
use two tildes in a row to represent it.

When generating a \LaTeX\ output file,
the following token will be formatted in \texttt{typewriter}
font, and the tilde removed, to make it stand out from the rest of the text.
This formatting will be applied to all characters after the
tilde up to the first white space\index{white space}.
Whether
or not the token is an actual statement label is not checked, and the
token does not have to have the correct syntax for a label; no error
messages will be produced.  The only effect of the label mode on the
output is that typewriter font will be used for the tokens that are
placed in the \LaTeX\ output file.

When generating {\sc html},
the tokens after the tilde {\em must} be a URL (either http: or https:)
or a valid label.
Error messages will be issued during that output if they aren't.
A hyperlink will be generated to that URL or label.

\subsubsection{Link to bibliographical reference}\index{citation}%
\index{link to bibliographical reference}

Bibliographical references are handled specially when generating
{\sc html} if formatted specially.
Text in the form \texttt{[}{\em author}\texttt{]}
is considered a link to a bibliographical reference.
See \texttt{help html} and \texttt{help write
bibliography} in the Metamath program for more
information.
% \index{\texttt{\char`\[}\ldots\texttt{]} inside comments}
See also Sections~\ref{tcomment} and \ref{wrbib}.

The \texttt{[}{\em author}\texttt{]} notation will also create an entry in
the bibliography cross-reference file generated by \texttt{write
bibliography} (Section~\ref{wrbib}) for {\sc HTML}.
For this to work properly, the
surrounding comment must be formatted as follows:
\begin{quote}
    {\em keyword} {\em label} {\em noise-word}
     \texttt{[}{\em author}\texttt{] p.} {\em number}
\end{quote}
for example
\begin{verbatim}
     Theorem 5.2 of [Monk] p. 223
\end{verbatim}
The {\em keyword} is not case sensitive and must be one of the following:
\begin{verbatim}
     theorem lemma definition compare proposition corollary
     axiom rule remark exercise problem notation example
     property figure postulate equation scheme chapter
\end{verbatim}
The optional {\em label} may consist of more than one
(non-{\em keyword} and non-{\em noise-word}) word.
The optional {\em noise-word} is one of:
\begin{verbatim}
     of in from on
\end{verbatim}
and is  ignored when the cross-reference file is created.  The
\texttt{write
biblio\-graphy} command will perform error checking to verify the
above format.\index{error checking}

\subsubsection{Parentheticals}\label{parentheticals}

The end of a comment may include one or more parenthicals, that is,
statements enclosed in parentheses.
The Metamath program looks for certain parentheticals and can issue
warnings based on them.
They are:

\begin{itemize}
 \item[] \texttt{(Contributed by }
   \textit{NAME}\texttt{,} \textit{DATE}\texttt{.)} -
   document the original contributor's name and the date it was created.
 \item[] \texttt{(Revised by }
   \textit{NAME}\texttt{,} \textit{DATE}\texttt{.)} -
   document the contributor's name and creation date
   that resulted in significant revision
   (not just an automated minimization or shortening).
 \item[] \texttt{(Proof shortened by }
   \textit{NAME}\texttt{,} \textit{DATE}\texttt{.)} -
   document the contributor's name and date that developed a significant
   shortening of the proof (not just an automated minimization).
 \item[] \texttt{(Proof modification is discouraged.)} -
   Note that this proof should normally not be modified.
 \item[] \texttt{(New usage is discouraged.)} -
   Note that this assertion should normally not be used.
\end{itemize}

The \textit{DATE} must be in form YYYY-MMM-DD, where MMM is the
English abbreviation of that month.

\subsubsection{Other markup}\label{othermarkup}
\index{markup notation}

There are other markup notations for generating good-looking results
beyond math mode and label mode:

\begin{itemize}
 \item[]
         \texttt{\char`\_} (underscore)\index{\texttt{\char`\_} inside comments} -
             Italicize text starting from
              {\em space}\texttt{\char`\_}{\em non-space} (i.e.\ \texttt{\char`\_}
              with a space before it and a non-space character after it) until
             the next
             {\em non-space}\texttt{\char`\_}{\em space}.  Normal
             punctuation (e.g.\ a trailing
             comma or period) is ignored when determining {\em space}.
 \item[]
         \texttt{\char`\_} (underscore) - {\em
         non-space}\texttt{\char`\_}{\em non-space-string}, where
          {\em non-space-string} is a string of non-space characters,
         will make {\em non-space-string} become a subscript.
 \item[]
         \texttt{<HTML>}...\texttt{</HTML>} - do not convert
         ``\texttt{<}'' and ``\texttt{>}''
         in the enclosed text when generating {\sc HTML},
         otherwise process markup normally. This allows direct insertion
         of {\sc html} commands.
 \item[]
       ``\texttt{\&}ref\texttt{;}'' - insert an {\sc HTML}
         character reference.
         This is how to insert arbitrary Unicode characters
         (such as accented characters).  Currently only directly supported
         when generating {\sc HTML}.
\end{itemize}

It is recommended that spaces surround any \texttt{\char`\~} and
\texttt{`} tokens in the comment and that a space follow the {\em label}
after a \texttt{\char`\~} token.  This will make global substitutions
to change labels and symbol names much easier and also eliminate any
future chance of ambiguity.  Spaces around these tokens are automatically
removed in the final output to conform with normal rules of punctuation;
for example, a space between a trailing \texttt{`} and a left parenthesis
will be removed.

A good way to become familiar with the markup notation is to look at
the extensive examples in the \texttt{set.mm} database.

\subsection{The Typesetting Comment (\texttt{\$t})}\label{tcomment}

The typesetting comment \texttt{\$t} in the input database file
provides the information necessary to produce good-looking results.
It provides \LaTeX\ and {\sc html}
definitions for math symbols,
as well supporting as some
customization of the generated web page.
If you add a new token to a database, you should also
update the \texttt{\$t} comment information if you want to eventually
create output in \LaTeX\ or {\sc HTML}.
See the
\texttt{set.mm}\index{set theory database (\texttt{set.mm})} database
file for an extensive example of a \texttt{\$t} comment illustrating
many of the features described below.

Programs that do not need to generate good-looking presentation results,
such as programs that only verify Metamath databases,
can completely ignore typesetting comments
and just treat them as normal comments.
Even the Metamath program only consults the
\texttt{\$t} comment information when it needs to generate typeset output
in \LaTeX\ or {\sc HTML}
(e.g., when you open a \LaTeX\ output file with the \texttt{open tex} command).

We will first discuss the syntax of typesetting comments, and then
briefly discuss how this can be used within the Metamath program.

\subsubsection{Typesetting Comment Syntax Overview}

The typesetting comment is identified by the token
\texttt{\$t}\index{\texttt{\$t} comment}\index{typesetting comment} in
the comment, and the typesetting comment ends at the matching
\texttt{\$)}:
\[
  \mbox{\tt \$(\ }
  \mbox{\tt \$t\ }
  \underbrace{
    \mbox{\tt \ \ \ \ \ \ \ \ \ \ \ }
    \cdots
    \mbox{\tt \ \ \ \ \ \ \ \ \ \ \ }
  }_{\mbox{Typesetting definitions go here}}
  \mbox{\tt \ \$)}
\]

There must be one or more white space characters, and only white space
characters, between the \texttt{\$(} that starts the comment
and the \texttt{\$t} symbol,
and the \texttt{\$t} must be followed by one
or more white space characters
(see section \ref{whitespace} for the definition of white space characters).
The typesetting comment continues until the comment end token \texttt{\$)}
(which must be preceded by one or more white space characters).

In version 0.07.30\index{Metamath!limitations of version 0.07.30} of the
Metamath program, there may be only one \texttt{\$t} comment in a
database.  This restriction may be lifted in the future to allow
many \texttt{\$t} comments in a database.

Between the \texttt{\$t} symbol (and its following white space) and the
comment end token \texttt{\$)} (and its preceding white space)
is a sequence of one or more typesetting definitions, where
each definition has the form
\textit{definition-type arg arg ... ;}.
Each of the zero or more \textit{arg} values
can be either a typesetting data or a keyword
(what keywords are allowed, and where, depends on the specific
\textit{definition-type}).
The \textit{definition-type}, and each argument \textit{arg},
are separated by one or more white space characters.
Every definition ends in an unquoted semicolon;
white space is not required before the terminating semicolon of a definition.
Each definition should start on a new line.\footnote{This
restriction of the current version of Metamath
(0.07.30)\index{Metamath!limitations of version 0.07.30} may be removed
in a future version, but you should do it anyway for readability.}

For example, this typesetting definition:
\begin{center}
 \verb$latexdef "C_" as "\subseteq";$
\end{center}
defines the token \verb$C_$ as the \LaTeX\ symbol $\subseteq$ (which means
``subset'').

Typesetting data is a sequence of one or more quoted strings
(if there is more than one, they are connected by \texttt{\char`\+}).
Often a single quoted string is used to provide data for a definition, using
either double (\texttt{\char`\"}) or single (\texttt{'}) quotation marks.
However,
{\em a quoted string (enclosed in quotation marks) may not include
line breaks.}
A quoted string
may include a quotation mark that matches the enclosing quotes by repeating
the quotation mark twice.  Here are some examples:

\begin{tabu}   { l l }
\textbf{Example} & \textbf{Meaning} \\
\texttt{\char`\"a\char`\"\char`\"b\char`\"} & \texttt{a\char`\"b} \\
\texttt{'c''d'} & \texttt{c'd} \\
\texttt{\char`\"e''f\char`\"} & \texttt{e''f} \\
\texttt{'g\char`\"\char`\"h'} & \texttt{g\char`\"\char`\"h} \\
\end{tabu}

Finally, a long quoted string
may be broken up into multiple quoted strings (considered, as a whole,
a single quoted string) and joined with \texttt{\char`\+}.
You can even use multiple lines as long as a
'+' is at the end of every line except the last one.
The \texttt{\char`\+} should be preceded and followed by at least one
white space character.
Thus, for example,
\begin{center}
 \texttt{\char`\"ab\char`\"\ \char`\+\ \char`\"cd\char`\"
    \ \char`\+\ \\ 'ef'}
\end{center}
is the same as
\begin{center}
 \texttt{\char`\"abcdef\char`\"}
\end{center}

{\sc c}-style comments \texttt{/*}\ldots\texttt{*/} are also supported.

In practice, whenever you add a new math token you will often want to add
typesetting definitions using
\texttt{latexdef}, \texttt{htmldef}, and
\texttt{althtmldef}, as described below.
That way, they will all be up to date.
Of course, whether or not you want to use all three definitions will
depend on how the database is intended to be used.

Below we discuss the different possible \textit{definition-kind} options.
We will show data surrounded by double quotes (in practice they can also use
single quotes and/or be a sequence joined by \texttt{+}s).
We will use specific names for the \textit{data} to make clear what
the data is used for, such as
{\em math-token} (for a Metamath math token,
{\em latex-string} (for string to be placed in a \LaTeX\ stream),
{\em {\sc html}-code} (for {\sc html} code),
and {\em filename} (for a filename).

\subsubsection{Typesetting Comment - \LaTeX}

The syntax for a \LaTeX\ definition is:
\begin{center}
 \texttt{latexdef "}{\em math-token}\texttt{" as "}{\em latex-string}\texttt{";}
\end{center}
\index{latex definitions@\LaTeX\ definitions}%
\index{\texttt{latexdef} statement}

The {\em token-string} and {\em latex-string} are the data
(character strings) for
the token and the \LaTeX\ definition of the token, respectively,

These \LaTeX\ definitions are used by the Metamath program
when it is asked to product \LaTeX output using
the \texttt{write tex} command.

\subsubsection{Typesetting Comment - {\sc html}}

The key kinds of {\sc HTML} definitions have the following syntax:

\vskip 1ex
    \texttt{htmldef "}{\em math-token}\texttt{" as "}{\em
    {\sc html}-code}\texttt{";}\index{\texttt{htmldef} statement}
                    \ \ \ \ \ \ldots

    \texttt{althtmldef "}{\em math-token}\texttt{" as "}{\em
{\sc html}-code}\texttt{";}\index{\texttt{althtmldef} statement}

                    \ \ \ \ \ \ldots

Note that in {\sc HTML} there are two possible definitions for math tokens.
This feature is useful when
an alternate representation of symbols is desired, for example one that
uses Unicode entities and another uses {\sc gif} images.

There are many other typesetting definitions that can control {\sc HTML}.
These include:

\vskip 1ex

    \texttt{htmldef "}{\em math-token}\texttt{" as "}{\em {\sc
    html}-code}\texttt{";}

    \texttt{htmltitle "}{\em {\sc html}-code}\texttt{";}%
\index{\texttt{htmltitle} statement}

    \texttt{htmlhome "}{\em {\sc html}-code}\texttt{";}%
\index{\texttt{htmlhome} statement}

    \texttt{htmlvarcolor "}{\em {\sc html}-code}\texttt{";}%
\index{\texttt{htmlvarcolor} statement}

    \texttt{htmlbibliography "}{\em filename}\texttt{";}%
\index{\texttt{htmlbibliography} statement}

\vskip 1ex

\noindent The \texttt{htmltitle} is the {\sc html} code for a common
title, such as ``Metamath Proof Explorer.''  The \texttt{htmlhome} is
code for a link back to the home page.  The \texttt{htmlvarcolor} is
code for a color key that appears at the bottom of each proof.  The file
specified by {\em filename} is an {\sc html} file that is assumed to
have a \texttt{<A NAME=}\ldots\texttt{>} tag for each bibiographic
reference in the database comments.  For example, if
\texttt{[Monk]}\index{\texttt{\char`\[}\ldots\texttt{]} inside comments}
occurs in the comment for a theorem, then \texttt{<A NAME='Monk'>} must
be present in the file; if not, a warning message is given.

Associated with
\texttt{althtmldef}
are the statements
\vskip 1ex

    \texttt{htmldir "}{\em
      directoryname}\texttt{";}\index{\texttt{htmldir} statement}

    \texttt{althtmldir "}{\em
     directoryname}\texttt{";}\index{\texttt{althtmldir} statement}

\vskip 1ex
\noindent giving the directories of the {\sc gif} and Unicode versions
respectively; their purpose is to provide cross-linking between the
two versions in the generated web pages.

When two different types of pages need to be produced from a single
database, such as the Hilbert Space Explorer that extends the Metamath
Proof Explorer, ``extended'' variables may be declared in the
\texttt{\$t} comment:
\vskip 1ex

    \texttt{exthtmltitle "}{\em {\sc html}-code}\texttt{";}%
\index{\texttt{exthtmltitle} statement}

    \texttt{exthtmlhome "}{\em {\sc html}-code}\texttt{";}%
\index{\texttt{exthtmlhome} statement}

    \texttt{exthtmlbibliography "}{\em filename}\texttt{";}%
\index{\texttt{exthtmlbibliography} statement}

\vskip 1ex
\noindent When these are declared, you also must declare
\vskip 1ex

    \texttt{exthtmllabel "}{\em label}\texttt{";}%
\index{\texttt{exthtmllabel} statement}

\vskip 1ex \noindent that identifies the database statement where the
``extended'' section of the database starts (in our example, where the
Hilbert Space Explorer starts).  During the generation of web pages for
that starting statement and the statements after it, the {\sc html} code
assigned to \texttt{exthtmltitle} and \texttt{exthtmlhome} is used
instead of that assigned to \texttt{htmltitle} and \texttt{htmlhome},
respectively.

\begin{sloppy}
\subsection{Additional Information Com\-ment (\texttt{\$j})} \label{jcomment}
\end{sloppy}

The additional information comment, aka the
\texttt{\$j}\index{\texttt{\$j} comment}\index{additional information comment}
comment,
provides a way to add additional structured information that can
be optionally parsed by systems.

The additional information comment is parsed the same way as the
typesetting comment (\texttt{\$t}) (see section \ref{tcomment}).
That is,
the additional information comment begins with the token
\texttt{\$j} within a comment,
and continues until the comment close \texttt{\$)}.
Within an additional information comment is a sequence of one or more
commands of the form \texttt{command arg arg ... ;}
where each of the zero or more \texttt{arg} values
can be either a quoted string or a keyword.
Note that every command ends in an unquoted semicolon.
If a verifier is parsing an additional information comment, but
doesn't recognize a particular command, it must skip the command
by finding the end of the command (an unquoted semicolon).

A database may have 0 or more additional information comments.
Note, however, that a verifier may ignore these comments entirely or only
process certain commands in an additional information comment.
The \texttt{mmj2} verifier supports many commands in additional information
comments.
We encourage systems that process additional information comments
to coordinate so that they will use the same command for the same effect.

Examples of additional information comments with various commands
(from the \texttt{set.mm} database) are:

\begin{itemize}
   \item Define the syntax and logical typecodes,
     and declare that our grammar is
     unambiguous (verifiable using the KLR parser, with compositing depth 5).
\begin{verbatim}
  $( $j
    syntax 'wff';
    syntax '|-' as 'wff';
    unambiguous 'klr 5';
  $)
\end{verbatim}

   \item Register $\lnot$ and $\rightarrow$ as primitive expressions
           (lacking definitions).
\begin{verbatim}
  $( $j primitive 'wn' 'wi'; $)
\end{verbatim}

   \item There is a special justification for \texttt{df-bi}.
\begin{verbatim}
  $( $j justification 'bijust' for 'df-bi'; $)
\end{verbatim}

   \item Register $\leftrightarrow$ as an equality for its type (wff).
\begin{verbatim}
  $( $j
    equality 'wb' from 'biid' 'bicomi' 'bitri';
    definition 'dfbi1' for 'wb';
  $)
\end{verbatim}

   \item Theorem \texttt{notbii} is the congruence law for negation.
\begin{verbatim}
  $( $j congruence 'notbii'; $)
\end{verbatim}

   \item Add \texttt{setvar} as a typecode.
\begin{verbatim}
  $( $j syntax 'setvar'; $)
\end{verbatim}

   \item Register $=$ as an equality for its type (\texttt{class}).
\begin{verbatim}
  $( $j equality 'wceq' from 'eqid' 'eqcomi' 'eqtri'; $)
\end{verbatim}

\end{itemize}


\subsection{Including Other Files in a Metamath Source File} \label{include}
\index{\texttt{\$[} and \texttt{\$]} auxiliary keywords}

The keywords \texttt{\$[} and \texttt{\$]} specify a file to be
included\index{included file}\index{file inclusion} at that point in a
Metamath\index{Metamath} source file\index{source file}.  The syntax for
including a file is as follows:
\begin{center}
\texttt{\$[} {\em file-name} \texttt{\$]}
\end{center}

The {\em file-name} should be a single token\index{token} with the same syntax
as a math symbol (i.e., all 93 non-whitespace
printable characters other than \texttt{\$} are
allowed, subject to the file-naming limitations of your operating system).
Comments may appear between the \texttt{\$[} and \texttt{\$]} keywords.  Included
files may include other files, which may in turn include other files, and so
on.

For example, suppose you want to use the set theory database as the starting
point for your own theory.  The first line in your file could be
\begin{center}
\texttt{\$[ set.mm \$]}
\end{center} All of the information (axioms, theorems,
etc.) in \texttt{set.mm} and any files that {\em it} includes will become
available for you to reference in your file. This can help make your work more
modular. A drawback to including files is that if you change the name of a
symbol or the label of a statement, you must also remember to update any
references in any file that includes it.


The naming conventions for included files are the same as those of your
operating system.\footnote{On the Macintosh, prior to Mac OS X,
 a colon is used to separate disk
and folder names from your file name.  For example, {\em volume}\texttt{:}{\em
file-name} refers to the root directory, {\em volume}\texttt{:}{\em
folder-name}\texttt{:}{\em file-name} refers to a folder in root, and {\em
volume}\texttt{:}{\em folder-name}\texttt{:}\ldots\texttt{:}{\em file-name} refers to a
deeper folder.  A simple {\em file-name} refers to a file in the folder from
which you launch the Metamath application.  Under Mac OS X and later,
the Metamath program is run under the Terminal application, which
conforms to Unix naming conventions.}\index{Macintosh file
names}\index{file names!Macintosh}\label{includef} For compatibility among
operating systems, you should keep the file names as simple as possible.  A
good convention to use is {\em file}\texttt{.mm} where {\em file} is eight
characters or less, in lower case.

There is no limit to the nesting depth of included files.  One thing that you
should be aware of is that if two included files themselves include a common
third file, only the {\em first} reference to this common file will be read
in.  This allows you to include two or more files that build on a common
starting file without having to worry about label and symbol conflicts that
would occur if the common file were read in more than once.  (In fact, if a
file includes itself, the self-reference will be ignored, although of course
it would not make any sense to do that.)  This feature also means, however,
that if you try to include a common file in several inner blocks, the result
might not be what you expect, since only the first reference will be replaced
with the included file (unlike the include statement in most other computer
languages).  Thus you would normally include common files only in the
outermost block\index{outermost block}.

\subsection{Compressed Proof Format}\label{compressed1}\index{compressed
proof}\index{proof!compressed}

The proof notation presented in Section~\ref{proof} is called a
{\bf normal proof}\index{normal proof}\index{proof!normal} and in principle is
sufficient to express any proof.  However, proofs often contain steps and
subproofs that are identical.  This is particularly true in typical
Metamath\index{Metamath} applications, because Metamath requires that the math
symbol sequence (usually containing a formula) at each step be separately
constructed, that is, built up piece by piece. As a result, a lot of
repetition often results.  The {\bf compressed proof} format allows Metamath
to take advantage of this redundancy to shorten proofs.

The specification for the compressed proof format is given in
Appen\-dix~\ref{compressed}.

Normally you need not concern yourself with the details of the compressed
proof format, since the Metamath program will allow you to convert from
the normal format to the compressed format with ease, and will also
automatically convert from the compressed format when proofs are displayed.
The overall structure of the compressed format is as follows:
\begin{center}
  \texttt{\$= ( } {\em label-list} \texttt{) } {\em compressed-proof\ }\ \texttt{\$.}
\end{center}
\index{\texttt{\$=} keyword}
The first \texttt{(} serves as a flag to Metamath that a compressed proof
follows.  The {\em label-list} includes all statements referred to by the
proof except the mandatory hypotheses\index{mandatory hypothesis}.  The {\em
compressed-proof} is a compact encoding of the proof, using upper-case
letters, and can be thought of as a large integer in base 26.  White
space\index{white space} inside a {\em compressed-proof} is
optional and is ignored.

It is important to note that the order of the mandatory hypotheses of
the statement being proved must not be changed if the compressed proof
format is used, otherwise the proof will become incorrect.  The reason
for this is that the mandatory hypotheses are not mentioned explicitly
in the compressed proof in order to make the compression more efficient.
If you wish to change the order of mandatory hypotheses, you must first
convert the proof back to normal format using the \texttt{save proof
{\em statement} /normal}\index{\texttt{save proof} command} command.
Later, you can go back to compressed format with \texttt{save proof {\em
statement} /compressed}.

During error checking with the \texttt{verify proof} command, an error
found in a compressed proof may point to a character in {\em
compressed-proof}, which may not be very meaningful to you.  In this
case, try to \texttt{save proof /normal} first, then do the
\texttt{verify proof} again.  In general, it is best to make sure a
proof is correct before saving it in compressed format, because severe
errors are less likely to be recoverable than in normal format.

\subsection{Specifying Unknown Proofs or Subproofs}\label{unknown}

In a proof under development, any step or subproof that is not yet known
may be represented with a single \texttt{?}.  For the purposes of
parsing the proof, the \texttt{?}\ \index{\texttt{]}@\texttt{?}\ inside
proofs} will push a single entry onto the RPN stack just as if it were a
hypothesis.  While developing a proof with the Proof
Assistant\index{Proof Assistant}, a partially developed proof may be
saved with the \texttt{save new{\char`\_}proof}\index{\texttt{save
new{\char`\_}proof} command} command, and \texttt{?}'s will be placed at
the appropriate places.

All \texttt{\$p}\index{\texttt{\$p} statement} statements must have
proofs, even if they are entirely unknown.  Before creating a proof with
the Proof Assistant, you should specify a completely unknown proof as
follows:
\begin{center}
  {\em label} \texttt{\$p} {\em statement} \texttt{\$= ?\ \$.}
\end{center}
\index{\texttt{\$=} keyword}
\index{\texttt{]}@\texttt{?}\ inside proofs}

The \texttt{verify proof}\index{\texttt{verify proof} command} command
will check the known portions of a partial proof for errors, but will
warn you that the statement has not been proved.

Note that partially developed proofs may be saved in compressed format
if desired.  In this case, you will see one or more \texttt{?}'s in the
{\em compressed-proof} part.\index{compressed
proof}\index{proof!compressed}

\section{Axioms vs.\ Definitions}\label{definitions}

The \textit{basic}
Metamath\index{Metamath} language and program
make no distinction\index{axiom vs.\
definition} between axioms\index{axiom} and
definitions.\index{definition} The \texttt{\$a}\index{\texttt{\$a}
statement} statement is used for both.  At first, this may seem
puzzling.  In the minds of many mathematicians, the distinction is
clear, even obvious, and hardly worth discussing.  A definition is
considered to be merely an abbreviation that can be replaced by the
expression for which it stands; although unless one actually does this,
to be precise then one should say that a theorem\index{theorem} is a
consequence of the axioms {\em and} the definitions that are used in the
formulation of the theorem \cite[p.~20]{Behnke}.\index{Behnke, H.}

\subsection{What is a Definition?}

What is a definition?  In its simplest form, a definition introduces a new
symbol and provides an unambiguous rule to transform an expression containing
the new symbol to one without it.  The concept of a ``proper
definition''\index{proper definition}\index{definition!proper} (as opposed to
a creative definition)\index{creative definition}\index{definition!creative}
that is usually agreed upon is (1) the definition should not strengthen the
language and (2) any symbols introduced by the definition should be eliminable
from the language \cite{Nemesszeghy}\index{Nemesszeghy, E. Z.}.  In other
words, they are mere typographical conveniences that do not belong to the
system and are theoretically superfluous.  This may seem obvious, but in fact
the nature of definitions can be subtle, sometimes requiring difficult
metatheorems to establish that they are not creative.

A more conservative stance was taken by logician S.
Le\'{s}niewski.\index{Le\'{s}niewski, S.}
\begin{quote}
Le\'{s}niewski
regards definitions as theses of the system.  In this respect they do
not differ either from the axioms or from theorems, i.e.\ from the
theses added to the system on the basis of the rule of substitution or
the rule of detachment [modus ponens].  Once definitions have been
accepted as theses of the system, it becomes necessary to consider them
as true propositions in the same sense in which axioms are true
\cite{Lejewski}.
\end{quote}\index{Lejewski, Czeslaw}

Let us look at some simple examples of definitions in propositional
calculus.  Consider the definition of logical {\sc or}
(disjunction):\index{disjunction ($\vee$)} ``$P\vee Q$ denotes $\neg P
\rightarrow Q$ (not $P$ implies $Q$).''  It is very easy to recognize a
statement making use of this definition, because it introduces the new
symbol $\vee$ that did not previously exist in the language.  It is easy
to see that no new theorems of the original language will result from
this definition.

Next, consider a definition that eliminates parentheses:  ``$P
\rightarrow Q\rightarrow R$ denotes $P\rightarrow (Q \rightarrow R)$.''
This is more subtle, because no new symbols are introduced.  The reason
this definition is considered proper is that no new symbol sequences
that are valid wffs (well-formed formulas)\index{well-formed formula
(wff)} in the original language will result from the definition, since
``$P \rightarrow Q\rightarrow R$'' is not a wff in the original
language.  Here, we implicitly make use of the fact that there is a
decision procedure that allows us to determine whether or not a symbol
sequence is a wff, and this fact allows us to use symbol sequences that
are not wffs to represent other things (such as wffs) by means of the
definition.  However, to justify the definition as not being creative we
need to prove that ``$P \rightarrow Q\rightarrow R$'' is in fact not a
wff in the original language, and this is more difficult than in the
case where we simply introduce a new symbol.

%Now let's take this reasoning to an extreme.  Propositional calculus is a
%decidable theory,\footnote{This means that a mechanical algorithm exists to
%determine whether or not a wff is a theorem.} so in principle we could make use
%of symbol sequences that are not theorems to represent other things (say, to
%encode actual theorems in a more compact way).  For example, let us extend the
%language by defining a wff ``$P$'' in the extended language as the theorem
%``$P\rightarrow P$''\footnote{This is one of the first theorems proved in the
%Metamath database \texttt{set.mm}.}\index{set
%theory database (\texttt{set.mm})} in the original language whenever ``$P$'' is
%not a theorem in the original language.  In the extended language, any wff
%``$Q$'' thus represents a theorem; to find out what theorem (in the original
%language) ``$Q$'' represents, we determine whether ``$Q$'' is a theorem in the
%original language (before the definition was introduced).  If so, we're done; if
%not, we replace ``$Q$'' by ``$Q\rightarrow Q$'' to eliminate the definition.
%This definition is therefore eliminable, and it does not ``strengthen'' the
%language because any wff that is not a theorem is not in the set of statements
%provable in the original language and thus is available for use by definitions.
%
%Of course, a definition such as this would render practically useless the
%communication of theorems of propositional calculus; but
%this is just a human shortcoming, since we can't always easily discern what is
%and is not a theorem by inspection.  In fact, the extended theory with this
%definition has no more and no less information than the original theory; it just
%expresses certain theorems of the form ``$P\rightarrow P$''
%in a more compact way.
%
%The point here is that what constitutes a proper definition is a matter of
%judgment about whether a symbol sequence can easily be recognized by a human
%as invalid in some sense (for example, not a wff); if so, the symbol sequence
%can be appropriated for use by a definition in order to make the extended
%language more compact.  Metamath\index{Metamath} lacks the ability to make this
%judgment, since as far as Metamath is concerned the definition of a wff, for
%example, is arbitrary.  You define for Metamath how wffs\index{well-formed
%formula (wff)} are constructed according to your own preferred style.  The
%concept of a wff may not even exist in a given formal system\index{formal
%system}.  Metamath treats all definitions as if they were new axioms, and it
%is up to the human mathematician to judge whether the definition is ``proper''
%'\index{proper definition}\index{definition!proper} in some agreed-upon way.

What constitutes a definition\index{definition} versus\index{axiom vs.\
definition} an axiom\index{axiom} is sometimes arbitrary in mathematical
literature.  For example, the connectives $\vee$ ({\sc or}), $\wedge$
({\sc and}), and $\leftrightarrow$ (equivalent to) in propositional
calculus are usually considered defined symbols that can be used as
abbreviations for expressions containing the ``primitive'' connectives
$\rightarrow$ and $\neg$.  This is the way we treat them in the standard
logic and set theory database \texttt{set.mm}\index{set theory database
(\texttt{set.mm})}.  However, the first three connectives can also be
considered ``primitive,'' and axiom systems have been devised that treat
all of them as such.  For example,
\cite[p.~35]{Goodstein}\index{Goodstein, R. L.} presents one with 15
axioms, some of which in fact coincide with what we have chosen to call
definitions in \texttt{set.mm}.  In certain subsets of classical
propositional calculus, such as the intuitionist
fragment\index{intuitionism}, it can be shown that one cannot make do
with just $\rightarrow$ and $\neg$ but must treat additional connectives
as primitive in order for the system to make sense.\footnote{Two nice
systems that make the transition from intuitionistic and other weak
fragments to classical logic just by adding axioms are given in
\cite{Robinsont}\index{Robinson, T. Thacher}.}

\subsection{The Approach to Definitions in \texttt{set.mm}}

In set theory, recursive definitions define a newly introduced symbol in
terms of itself.  The justification of recursive definitions, using
several ``recursion theorems,'' is usually one of the first
sophisticated proofs a student encounters when learning set theory, and
there is a significant amount of implicit metalogic behind a recursive
definition even though the definition itself is typically simple to
state.  It is, however, possible to substitute one kind of complexity
for another.  We can eliminate the need for metalogical justification by
defining the operation directly with an explicit (but complicated)
expression, then deriving the recursive definition directly as a
theorem, using a recursion theorem ``in reverse.''  We do this in
\texttt{set.mm}, as follows.

In \texttt{set.mm} our goal was to introduce almost all definitions in
the form of two expressions connected by either $\leftrightarrow$ or
$=$, where the thing being defined does not appear on the right hand
side.  Quine calls this form ``a genuine or direct definition'' \cite[p.
174]{Quine}\index{Quine, Willard Van Orman}, which makes the definitions
very easy to eliminate and the metalogic\index{metalogic} needed to
justify them as simple as possible.

We achieved this goal in almost all cases in \texttt{set.mm}.
Sometimes this makes the definitions more complex and less
intuitive.  For example, the traditional way to define addition of
natural numbers is to define an operation called {\em
successor}\index{successor} (which means ``plus one'' and is denoted by
``${\rm suc}$''), then define addition recursively\index{recursive
definition} with the two definitions $n + 0 = n$ and $m + {\rm suc}\,n =
{\rm suc} (m + n)$.  Although this definition seems simple and obvious,
the method to eliminate the definition is not obvious:  in the second
part of the definition, addition is defined in terms of itself.  By
eliminating the definition, we don't mean repeatedly applying it to
specific $m$ and $n$ but rather showing the explicit, closed-form
set-theoretical expression that $m + n$ represents, that will work for
any $m$ and $n$ and that does not have a $+$ sign on its right-hand
side.  For a recursive definition like this not to be circular
(creative), there are some hidden, underlying assumptions we must make,
for example that the natural numbers have a certain kind of order.  In
\texttt{set.mm} we chose to start with the direct (though complex and
nonintuitive) definition then derive from it the standard recursive
definition.\footnote{The closed-form definition used in \texttt{set.mm}
for the addition operation on ordinals\index{ordinal
addition}\index{addition!of ordinals} (of which natural numbers are a
subset) is

\setbox\startprefix=\hbox{\tt \ \ df-oadd\ \$a\ }
\setbox\contprefix=\hbox{\tt \ \ \ \ \ \ \ \ \ \ \ \ \ }
\startm
\m{\vdash}\m{+_o}\m{=}\m{\{}\m{\langle}\m{\langle}\m{x}\m{,}\m{y}\m{\rangle}%
\m{,}\m{z}\m{\rangle}\m{|}\m{(}\m{(}\m{x}\m{\in}\m{{\rm On}}\m{\wedge}\m{y}\m{%
\in}\m{{\rm On}}\m{)}\m{\wedge}\m{z}\m{=}\m{(}\m{{\rm rec}}\m{(}\m{\{}\m{%
\langle}\m{w}\m{,}\m{v}\m{\rangle}\m{|}\m{v}\m{=}\m{{\rm suc}}\m{w}\m{\}}\m{,}%
\m{x}\m{)}\m{`}\m{y}\m{)}\m{)}\m{\}}
\endm

\noindent Here, the abstraction class of nested ordered pairs\index{abstraction
class!of nested ordered pairs} is defined by \texttt{df-oprab} in \texttt{set.mm},
and ${\rm rec}$ is a ``recursion operator''\index{recursion operator} with
the definition

\setbox\startprefix=\hbox{\tt \ \ df-rfg\ \$a\ }
\setbox\contprefix=\hbox{\tt \ \ \ \ \ \ \ \ \ \ \ \ }
\startm
\m{\vdash}\m{{\rm rec}}\m{(}\m{F}\m{,}\m{A}\m{)}\m{=}\m{\bigcup}\m{\{}\m{f}%
\m{|}\m{\exists}\m{x}\m{(}\m{x}\m{\in}\m{{\rm On}}\m{\wedge}\m{(}\m{f}\m{{%
\rm Fn}}\m{x}\m{\wedge}\m{\forall}\m{y}\m{(}\m{y}\m{\in}\m{x}\m{\rightarrow}%
\m{(}\m{f}\m{`}\m{y}\m{)}\m{=}\m{(}\m{\{}\m{\langle}\m{g}\m{,}\m{z}\m{\rangle}%
\m{|}\m{(}\m{(}\m{g}\m{=}\m{\varnothing}\m{\wedge}\m{z}\m{=}\m{A}\m{)}\m{\vee}%
\m{(}\m{\lnot}\m{(}\m{g}\m{=}\m{\varnothing}\m{\vee}\m{{\rm Lim}}\m{{\rm dom}}%
\m{g}\m{)}\m{\wedge}\m{z}\m{=}\m{(}\m{F}\m{`}\m{(}\m{g}\m{`}\m{\bigcup}\m{{%
\rm dom}}\m{g}\m{)}\m{)}\m{)}\m{\vee}\m{(}\m{{\rm Lim}}\m{{\rm dom}}\m{g}\m{%
\wedge}\m{z}\m{=}\m{\bigcup}\m{{\rm ran}}\m{g}\m{)}\m{)}\m{\}}\m{`}\m{(}\m{f}%
\m{\restriction}\m{y}\m{)}\m{)}\m{)}\m{)}\m{)}\m{\}}
\endm

\noindent which can be further broken down with definitions shown in
Section~\ref{setdefinitions}.  You may be surprised at the complexity of what
seems like such a simple notion.  From these definitions the simpler, more
intuitive recursive definition is derived as a set of theorems.}\index{natural
number}\index{addition}\index{recursive definition}\index{ordinal addition}
The end result is the same, but we completely eliminate the rather complex
metalogic that justifies the recursive definition.  (For a mathematician,
recursive definitions are more efficient and intuitive than
direct ones once the metalogic has been learned or possibly just accepted as
correct.  However, it was felt that direct definition in \texttt{set.mm}
maximizes rigor by minimizing metalogic.  It can be eliminated effortlessly,
something difficult to do with a recursive definition.)

\subsection{Adding Constraints on Definitions}

The basic Metamath language and the Metamath program do
not have any built-in constraints on definitions, since they are just
\$a statements.

However, nothing prevents a verification system from verifying
additional rules to impose further limitations on definitions.
For example, the \texttt{mmj2}\index{mmj2} program
supports various kinds of
additional information comments (see section \ref{jcomment}).
One of their uses is to optionally verify additional constraints,
including constraints to verify that definitions meet certain
requirements.
These additional checks are required by the
continuous integration (CI)\index{continuous integration (CI)}
checks of the
\texttt{set.mm}\index{set theory database (\texttt{set.mm})}%
\index{Metamath Proof Explorer}
database.
This approach enables us to optionally impose additional requirements
on definitions if we wish, without necessarily imposing those rules on
all databases or requiring all verification systems to implement them.
In addition, this allows us to impose specialized constraints tailored
to one database while not requiring other systems to implement
those specialized constraints.

We impose two constraints on the
\texttt{set.mm}\index{set theory database (\texttt{set.mm})}%
\index{Metamath Proof Explorer} database
via the \texttt{mmj2}\index{mmj2} program that are worth discussing here:
a parse check and a definition soundness check.

% On February 11, 2019 8:32:32 PM EST, saueran@oregonstate.edu wrote:
% The following addition to the end of set.mm is accepted by the mmj2
% parser and definition checker and the metamath verifier(at least it was
% when I checked, you should check it too), and creates a contradiction by
% proving the theorem |- ph.
% ${
% wleftp $a wff ( ( ph ) $.
% wbothp $a wff ( ph ) $.
% df-leftp $a |- ( ( ( ph ) <-> -. ph ) $.
% df-bothp $a |- ( ( ph ) <-> ph ) $.
% anything $p |- ph $=
%   ( wbothp wn wi wleftp df-leftp biimpi df-bothp mpbir mpbi simplim ax-mp)
%   ABZAMACZDZCZMOEZOCQAEZNDZRNAFGSHIOFJMNKLAHJ $.
% $}
%
% This particular problem is countered by enabling, within mmj2,
% SetParser,mmj.verify.LRParser

First,
we enable a parse check in \texttt{mmj2} (through its
\texttt{SetParser} command) that requires that all new definitions
must \textit{not} create an ambiguous parse for a KLR(5) parser.
This prevents some errors such as definitions with imbalanced parentheses.

Second, we run a definition soundness check specific to
\texttt{set.mm} or databases similar to it.
(through the \texttt{definitionCheck} macro).
Some \texttt{\$a} statements (including all ax-* statemnets)
are excluded from these checks, as they will
always fail this simple check,
but they are appropriate for most definitions.
This check imposes a set of additional rules:

\begin{enumerate}

\item New definitions must be introduced using $=$ or $\leftrightarrow$.

\item No \texttt{\$a} statement introduced before this one is allowed to use the
symbol being defined in this definition, and the definition is not
allowed to use itself (except once, in the definiendum).

\item Every variable in the definiens should not be distinct

\item Every dummy variable in the definiendum
are required to be distinct from each other and from variables in
the definiendum.
To determine this, the system will look for a "justification" theorem
in the database, and if it is not there, attempt to briefly prove
$( \varphi \rightarrow \forall x \varphi )$  for each dummy variable x.

\item Every dummy variable should be a set variable,
unless there is a justification theorem available.

\item Every dummy variable must be bound
(if the system cannot determine this a justification theorem must be
provided).

\end{enumerate}

\subsection{Summary of Approach to Definitions}

In short, when being rigorous it turns out that
definitions can be subtle, sometimes requiring difficult
metatheorems to establish that they are not creative.

Instead of building such complications into the Metamath language itself,
the basic Metmath language and program simply treat traditional
axioms and definitions as the same kind of \texttt{\$a} statement.
We have then built various tools to enable people to
verify additional conditions as their creators believe is appropriate
for those specific databases, without complicating the Metamath foundations.

\chapter{The Metamath Program}\label{commands}

This chapter provides a reference manual for the
Metamath program.\index{Metamath!commands}

Current instructions for obtaining and installing the Metamath program
can be found at the \url{http://metamath.org} web site.
For Windows, there is a pre-compiled version called
\texttt{metamath.exe}.  For Unix, Linux, and Mac OS X
(which we will refer to collectively as ``Unix''), the Metamath program
can be compiled from its source code with the command
\begin{verbatim}
gcc *.c -o metamath
\end{verbatim}
using the \texttt{gcc} {\sc c} compiler available on those systems.

In the command syntax descriptions below, fields enclosed in square
brackets [\ ] are optional.  File names may be optionally enclosed in
single or double quotes.  This is useful if the file name contains
spaces or
slashes (\texttt{/}), such as in Unix path names, \index{Unix file
names}\index{file names!Unix} that might be confused with Metamath
command qualifiers.\index{Unix file names}\index{file names!Unix}

\section{Invoking Metamath}

Unix, Linux, and Mac OS X
have a command-line interface called the {\em
bash shell}.  (In Mac OS X, select the Terminal application from
Applications/Utilities.)  To invoke Metamath from the bash shell prompt,
assuming that the Metamath program is in the current directory, type
\begin{verbatim}
bash$ ./metamath
\end{verbatim}

To invoke Metamath from a Windows DOS or Command Prompt, assuming that
the Metamath program is in the current directory (or in a directory
included in the Path system environment variable), type
\begin{verbatim}
C:\metamath>metamath
\end{verbatim}

To use command-line arguments at invocation, the command-line arguments
should be a list of Metamath commands, surrounded by quotes if they
contain spaces.  In Windows, the surrounding quotes must be double (not
single) quotes.  For example, to read the database file \texttt{set.mm},
verify all proofs, and exit the program, type (under Unix)
\begin{verbatim}
bash$ ./metamath 'read set.mm' 'verify proof *' exit
\end{verbatim}
Note that in Unix, any directory path with \texttt{/}'s must be
surrounded by quotes so Metamath will not interpret the \texttt{/} as a
command qualifier.  So if \texttt{set.mm} is in the \texttt{/tmp}
directory, use for the above example
\begin{verbatim}
bash$ ./metamath 'read "/tmp/set.mm"' 'verify proof *' exit
\end{verbatim}

For convenience, if the command-line has one argument and no spaces in
the argument, the command is implicitly assumed to be \texttt{read}.  In
this one special case, \texttt{/}'s are not interpreted as command
qualifiers, so you don't need quotes around a Unix file name.  Thus
\begin{verbatim}
bash$ ./metamath /tmp/set.mm
\end{verbatim}
and
\begin{verbatim}
bash$ ./metamath "read '/tmp/set.mm'"
\end{verbatim}
are equivalent.


\section{Controlling Metamath}

The Metamath program was first developed on a {\sc vax/vms} system, and
some aspects of its command line behavior reflect this heritage.
Hopefully you will find it reasonably user-friendly once you get used to
it.

Each command line is a sequence of English-like words separated by
spaces, as in \texttt{show settings}.  Command words are not case
sensitive, and only as many letters are needed as are necessary to
eliminate ambiguity; for example, \texttt{sh se} would work for the
command \texttt{show settings}.  In some cases arguments such as file
names, statement labels, or symbol names are required; these are
case-sensitive (although file names may not be on some operating
systems).

A command line is entered by typing it in then pressing the {\em return}
({\em enter}) key.  To find out what commands are available, type
\texttt{?} at the \texttt{MM>} prompt.  To find out the choices at any
point in a command, press {\em return} and you will be prompted for
them.  The default choice (the one selected if you just press {\em
return}) is shown in brackets (\texttt{<>}).

You may also type \texttt{?} in place of a command word to force
Metamath to tell you what the choices are.  The \texttt{?} method won't
work, though, if a non-keyword argument such as a file name is expected
at that point, because the program will think that \texttt{?} is the
value of the argument.

Some commands have one or more optional qualifiers which modify the
behavior of the command.  Qualifiers are preceded by a slash
(\texttt{/}), such as in \texttt{read set.mm / verify}.  Spaces are
optional around the \texttt{/}.  If you need to use a space or
slash in a command
argument, as in a Unix file name, put single or double quotes around the
command argument.

The \texttt{open log} command will save everything you see on the
screen and is useful to help you recover should something go wrong in a
proof, or if you want to document a bug.

If a command responds with more than a screenful, you will be
prompted to \texttt{<return> to continue, Q to quit, or S to scroll to
end}.  \texttt{Q} or \texttt{q} (not case-sensitive) will complete the
command internally but will suppress further output until the next
\texttt{MM>} prompt.  \texttt{s} will suppress further pausing until the
next \texttt{MM>} prompt.  After the first screen, you are also
presented with the choice of \texttt{b} to go back a screenful.  Note
that \texttt{b} may also be entered at the \texttt{MM>} prompt
immediately after a command to scroll back through the output of that
command.

A command line enclosed in quotes is executed by your operating system.
See Section~\ref{oscmd}.

{\em Warning:} Pressing {\sc ctrl-c} will abort the Metamath program
unconditionally.  This means any unsaved work will be lost.


\subsection{\texttt{exit} Command}\index{\texttt{exit} command}

Syntax:  \texttt{exit} [\texttt{/force}]

This command exits from Metamath.  If there have been changes to the
source with the \texttt{save proof} or \texttt{save new{\char`\_}proof}
commands, you will be given an opportunity to \texttt{write source} to
permanently save the changes.

In Proof Assistant\index{Proof Assistant} mode, the \texttt{exit} command will
return to the \verb/MM>/ prompt. If there were changes to the proof, you will
be given an opportunity to \texttt{save new{\char`\_}proof}.

The \texttt{quit} command is a synonym for \texttt{exit}.

Optional qualifier:
    \texttt{/force} - Do not prompt if changes were not saved.  This qualifier is
        useful in \texttt{submit} command files (Section~\ref{sbmt})
        to ensure predictable behavior.





\subsection{\texttt{open log} Command}\index{\texttt{open log} command}
Syntax:  \texttt{open log} {\em file-name}

This command will open a log file that will store everything you see on
the screen.  It is useful to help recovery from a mistake in a long Proof
Assistant session, or to document bugs.\index{Metamath!bugs}

The log file can be closed with \texttt{close log}.  It will automatically be
closed upon exiting Metamath.



\subsection{\texttt{close log} Command}\index{\texttt{close log} command}
Syntax:  \texttt{close log}

The \texttt{close log} command closes a log file if one is open.  See
also \texttt{open log}.




\subsection{\texttt{submit} Command}\index{\texttt{submit} command}\label{sbmt}
Syntax:  \texttt{submit} {\em filename}

This command causes further command lines to be taken from the specified
file.  Note that any line beginning with an exclamation point (\texttt{!}) is
treated as a comment (i.e.\ ignored).  Also note that the scrolling
of the screen output is continuous, so you may want to open a log file
(see \texttt{open log}) to record the results that fly by on the screen.
After the lines in the file are exhausted, Metamath returns to its
normal user interface mode.

Currently, the \texttt{submit} command is not recursive.  In other words,
\texttt{submit} commands are not allowed inside a command file.


\subsection{\texttt{erase} Command}\index{\texttt{erase} command}
Syntax:  \texttt{erase}

This command will reset Metamath to its starting state, deleting any
data\-base that was \texttt{read} in.
 If there have been changes to the
source with the \texttt{save proof} or \texttt{save new{\char`\_}proof}
commands, you will be given an opportunity to \texttt{write source} to
permanently save the changes.



\subsection{\texttt{set echo} Command}\index{\texttt{set echo} command}
Syntax:  \texttt{set echo on} or \texttt{set echo off}

The \texttt{set echo on} command will cause command lines to be echoed with any
abbreviations expanded.  While learning the Metamath commands, this
feature will show you the exact command that your abbreviated input
corresponds to.



\subsection{\texttt{set scroll} Command}\index{\texttt{set scroll} command}
Syntax:  \texttt{set scroll prompted} or \texttt{set scroll continuous}

The Metamath command line interface starts off in the \texttt{prompted} mode,
which means that you will be prompted to continue or quit after each
full screen in a long listing.  In \texttt{continuous} mode, long listings will be
scrolled without pausing.

% LaTeX bug? (1) \texttt{\_} puts out different character than
% \texttt{{\char`\_}}
%  = \verb$_$  (2) \texttt{{\char`\_}} puts out garbage in \subsection
%  argument
\subsection{\texttt{set width} Command}\index{\texttt{set
width} command}
Syntax:  \texttt{set width} {\em number}

Metamath assumes the width of your screen is 79 characters (chosen
because the Command Prompt in Windows XP has a wrapping bug at column
80).  If your screen is wider or narrower, this command allows you to
change this default screen width.  A larger width is advantageous for
logging proofs to an output file to be printed on a wide printer.  A
smaller width may be necessary on some terminals; in this case, the
wrapping of the information messages may sometimes seem somewhat
unnatural, however.  In \LaTeX\index{latex@{\LaTeX}!characters per line},
there is normally a maximum of 61 characters per line with typewriter
font.  (The examples in this book were produced with 61 characters per
line.)

\subsection{\texttt{set height} Command}\index{\texttt{set
height} command}
Syntax:  \texttt{set height} {\em number}

Metamath assumes your screen height is 24 lines of characters.  If your
screen is taller or shorter, this command lets you to change the number
of lines at which the display pauses and prompts you to continue.

\subsection{\texttt{beep} Command}\index{\texttt{beep} command}

Syntax:  \texttt{beep}

This command will produce a beep.  By typing it ahead after a
long-running command has started, it will alert you that the command is
finished.  For convenience, \texttt{b} is an abbreviation for
\texttt{beep}.

Note:  If \texttt{b} is typed at the \texttt{MM>} prompt immediately
after the end of a multiple-page display paged with ``\texttt{Press
<return> for more}...'' prompts, then the \texttt{b} will back up to the
previous page rather than perform the \texttt{beep} command.
In that case you must type the unabbreviated \texttt{beep} form
of the command.

\subsection{\texttt{more} Command}\index{\texttt{more} command}

Syntax:  \texttt{more} {\em filename}

This command will display the contents of an {\sc ascii} file on your
screen.  (This command is provided for convenience but is not very
powerful.  See Section~\ref{oscmd} to invoke your operating system's
command to do this, such as the \texttt{more} command in Unix.)

\subsection{Operating System Commands}\index{operating system
command}\label{oscmd}

A line enclosed in single or double quotes will be executed by your
computer's operating system if it has a command line interface.  For
example, on a {\sc vax/vms} system,
\verb/MM> 'dir'/
will print disk directory contents.  Note that this feature will not
work on the Macintosh prior to Mac OS X, which does not have a command
line interface.

For your convenience, the trailing quote is optional.

\subsection{Size Limitations in Metamath}

In general, there are no fixed, predefined limits\index{Metamath!memory
limits} on how many labels, tokens\index{token}, statements, etc.\ that
you may have in a database file.  The Metamath program uses 32-bit
variables (64-bit on 64-bit CPUs) as indices for almost all internal
arrays, which are allocated dynamically as needed.



\section{Reading and Writing Files}

The following commands create new files:  the \texttt{open} commands;
the \texttt{write} commands; the \texttt{/html},
\texttt{/alt{\char`\_}html}, \texttt{/brief{\char`\_}html},
\texttt{/brief{\char`\_}alt{\char`\_}html} qualifiers of \texttt{show
statement}, and \texttt{midi}.  The following commands append to files
previously opened:  the \texttt{/tex} qualifier of \texttt{show proof}
and \texttt{show new{\char`\_}proof}; the \texttt{/tex} and
\texttt{/simple{\char`\_}tex} qualifiers of \texttt{show statement}; the
\texttt{close} commands; and all screen dialog between \texttt{open log}
and \texttt{close log}.

The commands that create new files will not overwrite an existing {\em
filename} but will rename the existing one to {\em
filename}\texttt{{\char`\~}1}.  An existing {\em
filename}\texttt{{\char`\~}1} is renamed {\em
filename}\texttt{{\char`\~}2}, etc.\ up to {\em
filename}\texttt{{\char`\~}9}.  An existing {\em
filename}\texttt{{\char`\~}9} is deleted.  This makes recovery from
mistakes easier but also will clutter up your directory, so occasionally
you may want to clean up (delete) these \texttt{{\char`\~}}$n$ files.


\subsection{\texttt{read} Command}\index{\texttt{read} command}
Syntax:  \texttt{read} {\em file-name} [\texttt{/verify}]

This command will read in a Metamath language source file and any included
files.  Normally it will be the first thing you do when entering Metamath.
Statement syntax is checked, but proof syntax is not checked.
Note that the file name may be enclosed in single or double quotes;
this is useful if the file name contains slashes, as might be the case
under Unix.

If you are getting an ``\texttt{?Expected VERIFY}'' error
when trying to read a Unix file name with slashes, you probably haven't
quoted it.\index{Unix file names}\index{file names!Unix}

If you are prompted for the file name (by pressing {\em return}
 after \texttt{read})
you should {\em not} put quotes around it, even if it is a Unix file name
with slashes.

Optional command qualifier:

    \texttt{/verify} - Verify all proofs as the database is read in.  This
         qualifier will slow down reading in the file.  See \texttt{verify
         proof} for more information on file error-checking.

See also \texttt{erase}.



\subsection{\texttt{write source} Command}\index{\texttt{write source} command}
Syntax:  \texttt{write source} {\em filename}
[\texttt{/rewrap}]
[\texttt{/split}]
[\texttt{/keep\_includes}]
[\texttt{/no\_versioning}]

This command will write the contents of a Metamath\index{database}
database into a file.\index{source file}

Optional command qualifiers:

\texttt{/rewrap} -
Reformats statements and comments according to the
convention used in the set.mm database.
It unwraps the
lines in the comment before each \texttt{\$a} and \texttt{\$p} statement,
then it
rewraps the line.  You should compare the output to the original
to make sure that the desired effect results; if not, go back to
the original source.  The wrapped line length honors the
\texttt{set width}
parameter currently in effect.  Note:  Text
enclosed in \texttt{<HTML>}...\texttt{</HTML>} tags is not modified by the
\texttt{/rewrap} qualifier.
Proofs themselves are not reformatted;
use \texttt{save proof * / compressed} to do that.
An isolated tilde (\~{}) is always kept on the same line as the following
symbol, so you can find all comment references to a symbol by
searching for \~{} followed by a space and that symbol
(this is useful for finding cross-references).
Incidentally, \texttt{save proof} also honors the \texttt{set width}
parameter currently in effect.

\texttt{/split} - Files included in the source using the expression
\$[ \textit{inclfile} \$] will be
written into separate files instead of being included in a single output
file.  The name of each separately written included file will be the
\textit{inclfile} argument of its inclusion command.

\texttt{/keep\_includes} - If a source file has includes but is written as a
single file by omitting \texttt{/split}, by default the included files will
be deleted (actually just renamed with a \char`\~1 suffix unless
\texttt{/no\_versioning} is specified) to prevent the possibly confusing
source duplication in both the output file and the included file.
The \texttt{/keep\_includes} qualifier will prevent this deletion.

\texttt{/no\_versioning} - Backup files suffixed with \char`\~1 are not created.


\section{Showing Status and Statements}



\subsection{\texttt{show settings} Command}\index{\texttt{show settings} command}
Syntax:  \texttt{show settings}

This command shows the state of various parameters.

\subsection{\texttt{show memory} Command}\index{\texttt{show memory} command}
Syntax:  \texttt{show memory}

This command shows the available memory left.  It is not meaningful
on most modern operating systems,
which have virtual memory.\index{Metamath!memory usage}


\subsection{\texttt{show labels} Command}\index{\texttt{show labels} command}
Syntax:  \texttt{show labels} {\em label-match} [\texttt{/all}]
   [\texttt{/linear}]

This command shows the labels of \texttt{\$a} and \texttt{\$p}
statements that match {\em label-match}.  A \verb$*$ in {label-match}
matches zero or more characters.  For example, \verb$*abc*def$ will match all
labels containing \verb$abc$ and ending with \verb$def$.

Optional command qualifiers:

   \texttt{/all} - Include matches for \texttt{\$e} and \texttt{\$f}
   statement labels.

   \texttt{/linear} - Display only one label per line.  This can be useful for
       building scripts in conjunction with the utilities under the
       \texttt{tools}\index{\texttt{tools} command} command.



\subsection{\texttt{show statement} Command}\index{\texttt{show statement} command}
Syntax:  \texttt{show statement} {\em label-match} [{\em qualifiers} (see below)]

This command provides information about a statement.  Only statements
that have labels (\texttt{\$f}\index{\texttt{\$f} statement},
\texttt{\$e}\index{\texttt{\$e} statement},
\texttt{\$a}\index{\texttt{\$a} statement}, and
\texttt{\$p}\index{\texttt{\$p} statement}) may be specified.
If {\em label-match}
contains wildcard (\verb$*$) characters, all matching statements will be
displayed in the order they occur in the database.

Optional qualifiers (only one qualifier at a time is allowed):

    \texttt{/comment} - This qualifier includes the comment that immediately
       precedes the statement.

    \texttt{/full} - Show complete information about each statement,
       and show all
       statements matching {\em label} (including \texttt{\$e}
       and \texttt{\$f} statements).

    \texttt{/tex} - This qualifier will write the statement information to the
       \LaTeX\ file previously opened with \texttt{open tex}.  See
       Section~\ref{texout}.

    \texttt{/simple{\char`\_}tex} - The same as \texttt{/tex}, except that
       \LaTeX\ macros are not used for formatting equations, allowing easier
       manual edits of the output for slide presentations, etc.

    \texttt{/html}\index{html generation@{\sc html} generation},
       \texttt{/alt{\char`\_}html}, \texttt{/brief{\char`\_}html},
       \texttt{/brief{\char`\_}alt{\char`\_}html} -
       These qualifiers invoke a special mode of
       \texttt{show statement} that
       creates a web page for the statement.  They may not be used with
       any other qualifier.  See Section~\ref{htmlout} or
       \texttt{help html} in the program.


\subsection{\texttt{search} Command}\index{\texttt{search} command}
Syntax:  search {\em label-match}
\texttt{"}{\em symbol-match}{\tt}" [\texttt{/all}] [\texttt{/comments}]
[\texttt{/join}]

This command searches all \texttt{\$a} and \texttt{\$p} statements
matching {\em label-match} for occurrences of {\em symbol-match}.  A
\verb@*@ in {\em label-match} matches any label character.  A \verb@$*@
in {\em symbol-match} matches any sequence of symbols.  The symbols in
{\em symbol-match} must be separated by white space.  The quotes
surrounding {\em symbol-match} may be single or double quotes.  For
example, \texttt{search b}\verb@* "-> $* ch"@ will list all statements
whose labels begin with \texttt{b} and contain the symbols \verb@->@ and
\texttt{ch} surrounding any symbol sequence (including no symbol
sequence).  The wildcards \texttt{?} and \texttt{\$?} are also available
to match individual characters in labels and symbols respectively; see
\texttt{help search} in the Metamath program for details on their usage.

Optional command qualifiers:

    \texttt{/all} - Also search \texttt{\$e} and \texttt{\$f} statements.

    \texttt{/comments} - Search the comment that immediately precedes each
        label-matched statement for {\em symbol-match}.  In this case
        {\em symbol-match} is an arbitrary, non-case-sensitive character
        string.  Quotes around {\em symbol-match} are optional if there
        is no ambiguity.

    \texttt{/join} - In the case of a \texttt{\$a} or \texttt{\$p} statement,
	prepend its \texttt{\$e}
	hypotheses for searching. The
	\texttt{/join} qualifier has no effect in \texttt{/comments} mode.

\section{Displaying and Verifying Proofs}


\subsection{\texttt{show proof} Command}\index{\texttt{show proof} command}
Syntax:  \texttt{show proof} {\em label-match} [{\em qualifiers} (see below)]

This command displays the proof of the specified
\texttt{\$p}\index{\texttt{\$p} statement} statement in various formats.
The {\em label-match} may contain wildcard (\verb@$*@) characters to match
multiple statements.  Without any qualifiers, only the logical steps
will be shown (i.e.\ syntax construction steps will be omitted), in an
indented format.

Most of the time, you will use
    \texttt{show proof} {\em label}
to see just the proof steps corresponding to logical inferences.

Optional command qualifiers:

    \texttt{/essential} - The proof tree is trimmed of all
        \texttt{\$f}\index{\texttt{\$f} statement} hypotheses before
        being displayed.  (This is the default, and it is redundant to
        specify it.)

    \texttt{/all} - the proof tree is not trimmed of all \texttt{\$f} hypotheses before
        being displayed.  \texttt{/essential} and \texttt{/all} are mutually exclusive.

    \texttt{/from{\char`\_}step} {\em step} - The display starts at the specified
        step.  If
        this qualifier is omitted, the display starts at the first step.

    \texttt{/to{\char`\_}step} {\em step} - The display ends at the specified
        step.  If this
        qualifier is omitted, the display ends at the last step.

    \texttt{/tree{\char`\_}depth} {\em number} - Only
         steps at less than the specified proof
        tree depth are displayed.  Sometimes useful for obtaining an overview of
        the proof.

    \texttt{/reverse} - The steps are displayed in reverse order.

    \texttt{/renumber} - When used with \texttt{/essential}, the steps are renumbered
        to correspond only to the essential steps.

    \texttt{/tex} - The proof is converted to \LaTeX\ and\index{latex@{\LaTeX}}
        stored in the file opened
        with \texttt{open tex}.  See Section~\ref{texout} or
        \texttt{help tex} in the program.

    \texttt{/lemmon} - The proof is displayed in a non-indented format known
        as Lemmon style, with explicit previous step number references.
        If this qualifier is omitted, steps are indented in a tree format.

    \texttt{/start{\char`\_}column} {\em number} - Overrides the default column
        (16)
        at which the formula display starts in a Lemmon-style display.  May be
        used only in conjunction with \texttt{/lemmon}.

    \texttt{/normal} - The proof is displayed in normal format suitable for
        inclusion in a Metamath source file.  May not be used with any other
        qualifier.

    \texttt{/compressed} - The proof is displayed in compressed format
        suitable for inclusion in a Metamath source file.  May not be used with
        any other qualifier.

    \texttt{/statement{\char`\_}summary} - Summarizes all statements (like a
        brief \texttt{show statement})
        used by the proof.  It may not be used with any other qualifier
        except \texttt{/essential}.

    \texttt{/detailed{\char`\_}step} {\em step} - Shows the details of what is
        happening at
        a specific proof step.  May not be used with any other qualifier.
        The {\em step} is the step number shown when displaying a
        proof without the \texttt{/renumber} qualifier.


\subsection{\texttt{show usage} Command}\index{\texttt{show usage} command}
Syntax:  \texttt{show usage} {\em label-match} [\texttt{/recursive}]

This command lists the statements whose proofs make direct reference to
the statement specified.

Optional command qualifier:

    \texttt{/recursive} - Also include statements whose proofs ultimately
        depend on the statement specified.



\subsection{\texttt{show trace\_back} Command}\index{\texttt{show
       trace{\char`\_}back} command}
Syntax:  \texttt{show trace{\char`\_}back} {\em label-match} [\texttt{/essential}] [\texttt{/axioms}]
    [\texttt{/tree}] [\texttt{/depth} {\em number}]

This command lists all statements that the proof of the \texttt{\$p}
statement(s) specified by {\em label-match} depends on.

Optional command qualifiers:

    \texttt{/essential} - Restrict the trace-back to \texttt{\$e}
        \index{\texttt{\$e} statement} hypotheses of proof trees.

    \texttt{/axioms} - List only the axioms that the proof ultimately depends on.

    \texttt{/tree} - Display the trace-back in an indented tree format.

    \texttt{/depth} {\em number} - Restrict the \texttt{/tree} trace-back to the
        specified indentation depth.

    \texttt{/count{\char`\_}steps} - Count the number of steps the proof has
       all the way back to axioms.  If \texttt{/essential} is specified,
       expansions of variable-type hypotheses (syntax constructions) are not counted.

\subsection{\texttt{verify proof} Command}\index{\texttt{verify proof} command}
Syntax:  \texttt{verify proof} {\em label-match} [\texttt{/syntax{\char`\_}only}]

This command verifies the proofs of the specified statements.  {\em
label-match} may contain wild card characters (\texttt{*}) to verify
more than one proof; for example \verb/*abc*def/ will match all labels
containing \texttt{abc} and ending with \texttt{def}.
The command \texttt{verify proof *} will verify all proofs in the database.

Optional command qualifier:

    \texttt{/syntax{\char`\_}only} - This qualifier will perform a check of syntax
        and RPN
        stack violations only.  It will not verify that the proof is
        correct.  This qualifier is useful for quickly determining which
        proofs are incomplete (i.e.\ are under development and have \texttt{?}'s
        in them).

{\em Note:} \texttt{read}, followed by \texttt{verify proof *}, will ensure
 the database is free
from errors in the Metamath language but will not check the markup notation
in comments.
You can also check the markup notation by running \texttt{verify markup *}
as discussed in Section~\ref{verifymarkup}; see also the discussion
on generating {\sc HTML} in Section~\ref{htmlout}.

\subsection{\texttt{verify markup} Command}\index{\texttt{verify markup} command}\label{verifymarkup}
Syntax:  \texttt{verify markup} {\em label-match}
[\texttt{/date{\char`\_}skip}]
[\texttt{/top{\char`\_}date{\char`\_}skip}]
[\texttt{/file{\char`\_}skip}]
[\texttt{/verbose}]

This command checks comment markup and other informal conventions we have
adopted.  It error-checks the latexdef, htmldef, and althtmldef statements
in the \texttt{\$t} statement of a Metamath source file.\index{error checking}
It error-checks any \texttt{`}...\texttt{`},
\texttt{\char`\~}~\textit{label},
and bibliographic markups in statement descriptions.
It checks that
\texttt{\$p} and \texttt{\$a} statements
have the same content when their labels start with
``ax'' and ``ax-'' respectively but are otherwise identical, for example
ax4 and ax-4.
It also verifies the date consistency of ``(Contributed by...),''
``(Revised by...),'' and ``(Proof shortened by...)'' tags in the comment
above each \texttt{\$a} and \texttt{\$p} statement.

Optional command qualifiers:

    \texttt{/date{\char`\_}skip} - This qualifier will
        skip date consistency checking,
        which is usually not required for databases other than
	\texttt{set.mm}.

    \texttt{/top{\char`\_}date{\char`\_}skip} - This qualifier will check date consistency except
        that the version date at the top of the database file will not
        be checked.  Only one of
        \texttt{/date{\char`\_}skip} and
        \texttt{/top{\char`\_}date{\char`\_}skip} may be
        specified.

    \texttt{/file{\char`\_}skip} - This qualifier will skip checks that require
        external files to be present, such as checking GIF existence and
        bibliographic links to mmset.html or equivalent.  It is useful
        for doing a quick check from a directory without these files.

    \texttt{/verbose} - Provides more information.  Currently it provides a list
        of axXXX vs. ax-XXX matches.

\subsection{\texttt{save proof} Command}\index{\texttt{save proof} command}
Syntax:  \texttt{save proof} {\em label-match} [\texttt{/normal}]
   [\texttt{/compressed}]

The \texttt{save proof} command will reformat a proof in one of two formats and
replace the existing proof in the source buffer\index{source
buffer}.  It is useful for
converting between proof formats.  Note that a proof will not be
permanently saved until a \texttt{write source} command is issued.

Optional command qualifiers:

    \texttt{/normal} - The proof is saved in the normal format (i.e., as a
        sequence
        of labels, which is the defined format of the basic Metamath
        language).\index{basic language}  This is the default format that
        is used if a qualifier
        is omitted.

    \texttt{/compressed} - The proof is saved in the compressed format which
        reduces storage requirements for a database.
        See Appendix~\ref{compressed}.




\section{Creating Proofs}\label{pfcommands}\index{Proof Assistant}

Before using the Proof Assistant, you must add a \texttt{\$p} to your
source file (using a text editor) containing the statement you want to
prove.  Its proof should consist of a single \texttt{?}, meaning
``unknown step.''  Example:
\begin{verbatim}
equid $p x = x $= ? $.
\end{verbatim}

To enter the Proof assistant, type \texttt{prove} {\em label}, e.g.
\texttt{prove equid}.  Metamath will respond with the \texttt{MM-PA>}
prompt.

Proofs are created working backwards from the statement being proved,
primarily using a series of \texttt{assign} commands.  A proof is
complete when all steps are assigned to statements and all steps are
unified and completely known.  During the creation of a proof, Metamath
will allow only operations that are legal based on what is known up to
that point.  For example, it will not allow an \texttt{assign} of a
statement that cannot be unified with the unknown proof step being
assigned.

{\em Important:}
The Proof Assistant is
{\em not} a tool to help you discover proofs.  It is just a tool to help
you add them to the database.  For a tutorial read
Section~\ref{frstprf}.
To practice using the Proof Assistant, you may
want to \texttt{prove} an existing theorem, then delete all steps with
\texttt{delete all}, then re-create it with the Proof Assistant while
looking at its proof display (before deletion).
You might want to figure out your first few proofs completely
and write them down by hand, before using the Proof Assistant, though
not everyone finds that effective.

{\em Important:}
The \texttt{undo} command if very helpful when entering a proof, because
it allows you to undo a previously-entered step.
In addition, we suggest that you
keep track of your work with a log file (\texttt{open
log}) and save it frequently (\texttt{save new{\char`\_}proof},
\texttt{write source}).
You can use \texttt{delete} to reverse an \texttt{assign}.
You can also do \texttt{delete floating{\char`\_}hypotheses}, then
\texttt{initialize all}, then \texttt{unify all /interactive} to
reinitialize bad unifications made accidentally or by bad
\texttt{assign}s.  You cannot reverse a \texttt{delete} except by
a relevant \texttt{undo} or using
\texttt{exit /force} then reentering the Proof Assistant to recover from
the last \texttt{save new{\char`\_}proof}.

The following commands are available in the Proof Assistant (at the
\texttt{MM-PA>} prompt) to help you create your proof.  See the
individual commands for more detail.

\begin{itemize}
\item[]
    \texttt{show new{\char`\_}proof} [\texttt{/all},...] - Displays the
        proof in progress.  You will use this command a lot; see \texttt{help
        show new{\char`\_}proof} to become familiar with its qualifiers.  The
        qualifiers \texttt{/unknown} and \texttt{/not{\char`\_}unified} are
        useful for seeing the work remaining to be done.  The combination
        \texttt{/all/unknown} is useful for identifying dummy variables that must be
        assigned, or attempts to use illegal syntax, when \texttt{improve all}
        is unable to complete the syntax constructions.  Unknown variables are
        shown as \texttt{\$1}, \texttt{\$2},...
\item[]
    \texttt{assign} {\em step} {\em label} - Assigns an unknown {\em step}
        number with the statement
        specified by {\em label}.
\item[]
    \texttt{let variable} {\em variable}
        \texttt{= "}{\em symbol sequence}\texttt{"}
          - Forces a symbol
        sequence to replace an unknown variable (such as \texttt{\$1}) in a proof.
        It is useful
        for helping difficult unifications, and it is necessary when you have
        dummy variables that eventually must be assigned a name.
\item[]
    \texttt{let step} {\em step} \texttt{= "}{\em symbol sequence}\texttt{"} -
          Forces a symbol sequence
        to replace the contents of a proof step, provided it can be
        unified with the existing step contents.  (I rarely use this.)
\item[]
    \texttt{unify step} {\em step} (or \texttt{unify all}) - Unifies
        the source and target of
        a step.  If you specify a specific step, you will be prompted
        to select among the unifications that are possible.  If you
        specify \texttt{all}, all steps with unique unifications, but only
        those steps, will be
        unified.  \texttt{unify all /interactive} goes through all non-unified
        steps.
\item[]
    \texttt{initialize} {\em step} (or \texttt{all}) - De-unifies the target and source of
        a step (or all steps), as well as the hypotheses of the source,
        and makes all variables in the source unknown.  Useful to recover from
        an \texttt{assign} or \texttt{let} mistake that
        resulted in incorrect unifications.
\item[]
    \texttt{delete} {\em step} (or \texttt{all} or \texttt{floating{\char`\_}hypotheses}) -
        Deletes the specified
        step(s).  \texttt{delete floating{\char`\_}hypotheses}, then \texttt{initialize all}, then
        \texttt{unify all /interactive} is useful for recovering from mistakes
        where incorrect unifications assigned wrong math symbol strings to
        variables.
\item[]
    \texttt{improve} {\em step} (or \texttt{all}) -
      Automatically creates a proof for steps (with no unknown variables)
      whose proof requires no statements with \texttt{\$e} hypotheses.  Useful
      for filling in proofs of \texttt{\$f} hypotheses.  The \texttt{/depth}
      qualifier will also try statements whose \texttt{\$e} hypotheses contain
      no new variables.  {\em Warning:} Save your work (with \texttt{save
      new{\char`\_}proof} then \texttt{write source}) before using
      \texttt{/depth = 2} or greater, since the search time grows
      exponentially and may never terminate in a reasonable time, and you
      cannot interrupt the search.  I have found that it is rare for
      \texttt{/depth = 3} or greater to be useful.
 \item[]
    \texttt{save new{\char`\_}proof} - Saves the proof in progress in the program's
        internal database buffer.  To save it permanently into the database file,
        use \texttt{write source} after
        \texttt{save new{\char`\_}proof}.  To revert to the last
        \texttt{save new{\char`\_}proof},
        \texttt{exit /force} from the Proof Assistant then re-enter the Proof
        Assistant.
 \item[]
    \texttt{match step} {\em step} (or \texttt{match all}) - Shows what
        statements are
        possibilities for the \texttt{assign} statement. (This command
        is not very
        useful in its present form and hopefully will be improved
        eventually.  In the meantime, use the \texttt{search} statement for
        candidates matching specific math token combinations.)
 \item[]
 \texttt{minimize{\char`\_}with}\index{\texttt{minimize{\char`\_}with} command}
% 3/10/07 Note: line-breaking the above results in duplicate index entries
     - After a proof is complete, this command will attempt
        to match other database theorems to the proof to see if the proof
        size can be reduced as a result.  See \texttt{help
        minimize{\char`\_}with} in the
        Metamath program for its usage.
 \item[]
 \texttt{undo}\index{\texttt{undo} command}
    - Undo the effect of a proof-changing command (all but the
      \texttt{show} and \texttt{save} commands above).
 \item[]
 \texttt{redo}\index{\texttt{redo} command}
    - Reverse the previous \texttt{undo}.
\end{itemize}

The following commands set parameters that may be relevant to your proof.
Consult the individual \texttt{help set}... commands.
\begin{itemize}
   \item[] \texttt{set unification{\char`\_}timeout}
 \item[]
    \texttt{set search{\char`\_}limit}
  \item[]
    \texttt{set empty{\char`\_}substitution} - note that default is \texttt{off}
\end{itemize}

Type \texttt{exit} to exit the \texttt{MM-PA>}
 prompt and get back to the \texttt{MM>} prompt.
Another \texttt{exit} will then get you out of Metamath.



\subsection{\texttt{prove} Command}\index{\texttt{prove} command}
Syntax:  \texttt{prove} {\em label}

This command will enter the Proof Assistant\index{Proof Assistant}, which will
allow you to create or edit the proof of the specified statement.
The command-line prompt will change from \texttt{MM>} to \texttt{MM-PA>}.

Note:  In the present version of
Metamath\index{Metamath!limitations of version 0.07.30}, the Proof
Assistant does not verify that \texttt{\$d}\index{\texttt{\$d}
statement} restrictions are met as a proof is being built.  After you
have completed a proof, you should type \texttt{save new{\char`\_}proof}
followed by \texttt{verify proof} {\em label} (where {\em label} is the
statement you are proving with the \texttt{prove} command) to verify the
\texttt{\$d} restrictions.

See also: \texttt{exit}

\subsection{\texttt{set unification\_timeout} Command}\index{\texttt{set
unification{\char`\_}timeout} command}
Syntax:  \texttt{set unification{\char`\_}timeout} {\em number}

(This command is available outside the Proof Assistant but affects the
Proof Assistant\index{Proof Assistant} only.)

Sometimes the Proof Assistant will inform you that a unification
time-out occurred.  This may happen when you try to \texttt{unify}
formulas with many temporary variables\index{temporary variable}
(\texttt{\$1}, \texttt{\$2}, etc.), since the time to compute all possible
unifications may grow exponentially with the number of variables.  If
you want Metamath to try harder (and you're willing to wait longer) you
may increase this parameter.  \texttt{show settings} will show you the
current value.

\subsection{\texttt{set empty\_substitution} Command}\index{\texttt{set
empty{\char`\_}substitution} command}
% These long names can't break well in narrow mode, and even "sloppy"
% is not enough. Work around this by not demanding justification.
\begin{flushleft}
Syntax:  \texttt{set empty{\char`\_}substitution on} or \texttt{set
empty{\char`\_}substitution off}
\end{flushleft}

(This command is available outside the Proof Assistant but affects the
Proof Assistant\index{Proof Assistant} only.)

The Metamath language allows variables to be
substituted\index{substitution!variable}\index{variable substitution}
with empty symbol sequences\index{empty substitution}.  However, in many
formal systems\index{formal system} this will never happen in a valid
proof.  Allowing for this possibility increases the likelihood of
ambiguous unifications\index{ambiguous
unification}\index{unification!ambiguous} during proof creation.
The default is that
empty substitutions are not allowed; for formal systems requiring them,
you must \texttt{set empty{\char`\_}substitution on}.
(An example where this must be \texttt{on}
would be a system that implements a Deduction Rule and in
which deductions from empty assumption lists would be permissible.  The
MIU-system\index{MIU-system} described in Appendix~\ref{MIU} is another
example.)
Note that empty substitutions are
always permissible in proof verification (VERIFY PROOF...) outside the
Proof Assistant.  (See the MIU system in the Metamath book for an example
of a system needing empty substitutions; another example would be a
system that implements a Deduction Rule and in which deductions from
empty assumption lists would be permissible.)

It is better to leave this \texttt{off} when working with \texttt{set.mm}.
Note that this command does not affect the way proofs are verified with
the \texttt{verify proof} command.  Outside of the Proof Assistant,
substitution of empty sequences for math symbols is always allowed.

\subsection{\texttt{set search\_limit} Command}\index{\texttt{set
search{\char`\_}limit} command} Syntax:  \texttt{set search{\char`\_}limit} {\em
number}

(This command is available outside the Proof Assistant but affects the
Proof Assistant\index{Proof Assistant} only.)

This command sets a parameter that determines when the \texttt{improve} command
in Proof Assistant mode gives up.  If you want \texttt{improve} to search harder,
you may increase it.  The \texttt{show settings} command tells you its current
value.


\subsection{\texttt{show new\_proof} Command}\index{\texttt{show
new{\char`\_}proof} command}
Syntax:  \texttt{show new{\char`\_}proof} [{\em
qualifiers} (see below)]

This command (available only in Proof Assistant mode) displays the proof
in progress.  It is identical to the \texttt{show proof} command, except that
there is no statement argument (since it is the statement being proved) and
the following qualifiers are not available:

    \texttt{/statement{\char`\_}summary}

    \texttt{/detailed{\char`\_}step}

Also, the following additional qualifiers are available:

    \texttt{/unknown} - Shows only steps that have no statement assigned.

    \texttt{/not{\char`\_}unified} - Shows only steps that have not been unified.

Note that \texttt{/essential}, \texttt{/depth}, \texttt{/unknown}, and
\texttt{/not{\char`\_}unified} may be
used in any combination; each of them effectively filters out additional
steps from the proof display.

See also:  \texttt{show proof}






\subsection{\texttt{assign} Command}\index{\texttt{assign} command}
Syntax:   \texttt{assign} {\em step} {\em label} [\texttt{/no{\char`\_}unify}]

   and:   \texttt{assign first} {\em label}

   and:   \texttt{assign last} {\em label}


This command, available in the Proof Assistant only, assigns an unknown
step (one with \texttt{?} in the \texttt{show new{\char`\_}proof}
listing) with the statement specified by {\em label}.  The assignment
will not be allowed if the statement cannot be unified with the step.

If \texttt{last} is specified instead of {\em step} number, the last
step that is shown by \texttt{show new{\char`\_}proof /unknown} will be
used.  This can be useful for building a proof with a command file (see
\texttt{help submit}).  It also makes building proofs faster when you know
the assignment for the last step.

If \texttt{first} is specified instead of {\em step} number, the first
step that is shown by \texttt{show new{\char`\_}proof /unknown} will be
used.

If {\em step} is zero or negative, the -{\em step}th from last unknown
step, as shown by \texttt{show new{\char`\_}proof /unknown}, will be
used.  \texttt{assign -1} {\em label} will assign the penultimate
unknown step, \texttt{assign -2} {\em label} the antepenultimate, and
\texttt{assign 0} {\em label} is the same as \texttt{assign last} {\em
label}.

Optional command qualifier:

    \texttt{/no{\char`\_}unify} - do not prompt user to select a unification if there is
        more than one possibility.  This is useful for noninteractive
        command files.  Later, the user can \texttt{unify all /interactive}.
        (The assignment will still be automatically unified if there is only
        one possibility and will be refused if unification is not possible.)



\subsection{\texttt{match} Command}\index{\texttt{match} command}
Syntax:  \texttt{match step} {\em step} [\texttt{/max{\char`\_}essential{\char`\_}hyp}
{\em number}]

    and:  \texttt{match all} [\texttt{/essential}]
          [\texttt{/max{\char`\_}essential{\char`\_}hyp} {\em number}]

This command, available in the Proof Assistant only, shows what
statements can be unified with the specified step(s).  {\em Note:} In
its current form, this command is not very useful because of the large
number of matches it reports.  I rarely use it.
It may be enhanced in the future.

Optional command qualifiers:

    \texttt{/max{\char`\_}essential{\char`\_}hyp} {\em number} - filters out
        of the list any statements
        with more than the specified number of
        \texttt{\$e}\index{\texttt{\$e} statement} hypotheses

    \texttt{/essential{\char`\_}only} - in the \texttt{match all} statement, only
        the steps that
        would be listed in the \texttt{show new{\char`\_}proof /essential} display are
        matched.



\subsection{\texttt{let} Command}\index{\texttt{let} command}
Syntax: \texttt{let variable} {\em variable} = \verb/"/{\em symbol-sequence}\verb/"/

  and: \texttt{let step} {\em step} = \verb/"/{\em symbol-sequence}\verb/"/

These commands, available in the Proof Assistant\index{Proof Assistant}
only, assign a temporary variable\index{temporary variable} or unknown
step with a specific symbol sequence.  They are useful in the middle of
creating a proof, when you know what should be in the proof step but the
unification algorithm doesn't yet have enough information to completely
specify the temporary variables.  A ``temporary variable'' is one that
has the form \texttt{\$}{\em nn} in the proof display, such as
\texttt{\$1}, \texttt{\$2}, etc.  The {\em symbol-sequence} may contain
other unknown variables if desired.  Examples:

    \verb/let variable $32 = "A = B"/

    \verb/let variable $32 = "A = $35"/

    \verb/let step 10 = '|- x = x'/

    \verb/let step -2 = "|- ( $7 = ph )"/

Any symbol sequence will be accepted for the \texttt{let variable}
command.  Only those symbol sequences that can be unified with the step
will be accepted for \texttt{let step}.

The \texttt{let} commands ``zap'' the proof with information that can
only be verified when the proof is built up further.  If you make an
error, the command sequence \texttt{delete
floating{\char`\_}hypotheses}, \texttt{initialize all}, and
\texttt{unify all /interactive} will undo a bad \texttt{let} assignment.

If {\em step} is zero or negative, the -{\em step}th from last unknown
step, as shown by \texttt{show new{\char`\_}proof /unknown}, will be
used.  The command \texttt{let step 0} = \verb/"/{\em
symbol-sequence}\verb/"/ will use the last unknown step, \texttt{let
step -1} = \verb/"/{\em symbol-sequence}\verb/"/ the penultimate, etc.
If {\em step} is positive, \texttt{let step} may be used to assign known
(in the sense of having previously been assigned a label with
\texttt{assign}) as well as unknown steps.

Either single or double quotes can surround the {\em symbol-sequence} as
long as they are different from any quotes inside a {\em
symbol-sequence}.  If {\em symbol-sequence} contains both kinds of
quotes, see the instructions at the end of \texttt{help let} in the
Metamath program.


\subsection{\texttt{unify} Command}\index{\texttt{unify} command}
Syntax:  \texttt{unify step} {\em step}

      and:   \texttt{unify all} [\texttt{/interactive}]

These commands, available in the Proof Assistant only, unify the source
and target of the specified step(s). If you specify a specific step, you
will be prompted to select among the unifications that are possible.  If
you specify \texttt{all}, only those steps with unique unifications will be
unified.

Optional command qualifier for \texttt{unify all}:

    \texttt{/interactive} - You will be prompted to select among the
        unifications
        that are possible for any steps that do not have unique
        unifications.  (Otherwise \texttt{unify all} will bypass these.)

See also \texttt{set unification{\char`\_}timeout}.  The default is
100000, but increasing it to 1000000 can help difficult cases.  Manually
assigning some or all of the unknown variables with the \texttt{let
variable} command also helps difficult cases.



\subsection{\texttt{initialize} Command}\index{\texttt{initialize} command}
Syntax:  \texttt{initialize step} {\em step}

    and: \texttt{initialize all}

These commands, available in the Proof Assistant\index{Proof Assistant}
only, ``de-unify'' the target and source of a step (or all steps), as
well as the hypotheses of the source, and makes all variables in the
source and the source's hypotheses unknown.  This command is useful to
help recover from incorrect unifications that resulted from an incorrect
\texttt{assign}, \texttt{let}, or unification choice.  Part or all of
the command sequence \texttt{delete floating{\char`\_}hypotheses},
\texttt{initialize all}, and \texttt{unify all /interactive} will recover
from incorrect unifications.

See also:  \texttt{unify} and \texttt{delete}



\subsection{\texttt{delete} Command}\index{\texttt{delete} command}
Syntax:  \texttt{delete step} {\em step}

   and:      \texttt{delete all} -- {\em Warning: dangerous!}

   and:      \texttt{delete floating{\char`\_}hypotheses}

These commands are available in the Proof Assistant only.  The
\texttt{delete step} command deletes the proof tree section that
branches off of the specified step and makes the step become unknown.
\texttt{delete all} is equivalent to \texttt{delete step} {\em step}
where {\em step} is the last step in the proof (i.e.\ the beginning of
the proof tree).

In most cases the \texttt{undo} command is the best way to undo
a previous step.
An alternative is to salvage your last \texttt{save
new{\char`\_}proof} by exiting and reentering the Proof Assistant.
For this to work, keep a log file open to record your work
and to do \texttt{save new{\char`\_}proof} frequently, especially before
\texttt{delete}.

\texttt{delete floating{\char`\_}hypotheses} will delete all sections of
the proof that branch off of \texttt{\$f}\index{\texttt{\$f} statement}
statements.  It is sometimes useful to do this before an
\texttt{initialize} command to recover from an error.  Note that once a
proof step with a \texttt{\$f} hypothesis as the target is completely
known, the \texttt{improve} command can usually fill in the proof for
that step.  Unlike the deletion of logical steps, \texttt{delete
floating{\char`\_}hypotheses} is a relatively safe command that is
usually easy to recover from.



\subsection{\texttt{improve} Command}\index{\texttt{improve} command}
\label{improve}
Syntax:  \texttt{improve} {\em step} [\texttt{/depth} {\em number}]
                                               [\texttt{/no{\char`\_}distinct}]

   and:   \texttt{improve first} [\texttt{/depth} {\em number}]
                                              [\texttt{/no{\char`\_}distinct}]

   and:   \texttt{improve last} [\texttt{/depth} {\em number}]
                                              [\texttt{/no{\char`\_}distinct}]

   and:   \texttt{improve all} [\texttt{/depth} {\em number}]
                                              [\texttt{/no{\char`\_}distinct}]

These commands, available in the Proof Assistant\index{Proof Assistant}
only, try to find proofs automatically for unknown steps whose symbol
sequences are completely known.  They are primarily useful for filling in
proofs of \texttt{\$f}\index{\texttt{\$f} statement} hypotheses.  The
search will be restricted to statements having no
\texttt{\$e}\index{\texttt{\$e} statement} hypotheses.

\begin{sloppypar} % narrow
Note:  If memory is limited, \texttt{improve all} on a large proof may
overflow memory.  If you use \texttt{set unification{\char`\_}timeout 1}
before \texttt{improve all}, there will usually be sufficient
improvement to easily recover and completely \texttt{improve} the proof
later on a larger computer.  Warning:  Once memory has overflowed, there
is no recovery.  If in doubt, save the intermediate proof (\texttt{save
new{\char`\_}proof} then \texttt{write source}) before \texttt{improve
all}.
\end{sloppypar}

If \texttt{last} is specified instead of {\em step} number, the last
step that is shown by \texttt{show new{\char`\_}proof /unknown} will be
used.

If \texttt{first} is specified instead of {\em step} number, the first
step that is shown by \texttt{show new{\char`\_}proof /unknown} will be
used.

If {\em step} is zero or negative, the -{\em step}th from last unknown
step, as shown by \texttt{show new{\char`\_}proof /unknown}, will be
used.  \texttt{improve -1} will use the penultimate
unknown step, \texttt{improve -2} {\em label} the antepenultimate, and
\texttt{improve 0} is the same as \texttt{improve last}.

Optional command qualifier:

    \texttt{/depth} {\em number} - This qualifier will cause the search
        to include
        statements with \texttt{\$e} hypotheses (but no new variables in
        the \texttt{\$e}
        hypotheses), provided that the backtracking has not exceeded the
        specified depth. {\em Warning:}  Try \texttt{/depth 1},
        then \texttt{2}, then \texttt{3}, etc.
        in sequence because of possible exponential blowups.  Save your
        work before trying \texttt{/depth} greater than \texttt{1}!

    \texttt{/no{\char`\_}distinct} - Skip trial statements that have
        \texttt{\$d}\index{\texttt{\$d} statement} requirements.
        This qualifier will prevent assignments that might violate \texttt{\$d}
        requirements but it also could miss possible legal assignments.

See also: \texttt{set search{\char`\_}limit}

\subsection{\texttt{save new\_proof} Command}\index{\texttt{save
new{\char`\_}proof} command}
Syntax:  \texttt{save new{\char`\_}proof} {\em label} [\texttt{/normal}]
   [\texttt{/compressed}]

The \texttt{save new{\char`\_}proof} command is available in the Proof
Assistant only.  It saves the proof in progress in the source
buffer\index{source buffer}.  \texttt{save new{\char`\_}proof} may be
used to save a completed proof, or it may be used to save a proof in
progress in order to work on it later.  If an incomplete proof is saved,
any user assignments with \texttt{let step} or \texttt{let variable}
will be lost, as will any ambiguous unifications\index{ambiguous
unification}\index{unification!ambiguous} that were resolved manually.
To help make recovery easier, it can be helpful to \texttt{improve all}
before \texttt{save new{\char`\_}proof} so that the incomplete proof
will have as much information as possible.

Note that the proof will not be permanently saved until a \texttt{write
source} command is issued.

Optional command qualifiers:

    \texttt{/normal} - The proof is saved in the normal format (i.e., as a
        sequence of labels, which is the defined format of the basic Metamath
        language).\index{basic language}  This is the default format that
        is used if a qualifier is omitted.

    \texttt{/compressed} - The proof is saved in the compressed format, which
        reduces storage requirements for a database.
        (See Appendix~\ref{compressed}.)


\section{Creating \LaTeX\ Output}\label{texout}\index{latex@{\LaTeX}}

You can generate \LaTeX\ output given the
information in a database.
The database must already include the necessary typesetting information
(see section \ref{tcomment} for how to provide this information).

The \texttt{show statement} and \texttt{show proof} commands each have a
special \texttt{/tex} command qualifier that produces \LaTeX\ output.
(The \texttt{show statement} command also has the
\texttt{/simple{\char`\_}tex} qualifier for output that is easier to
edit by hand.)  Before you can use them, you must open a \LaTeX\ file to
which to send their output.  A typical complete session will use this
sequence of Metamath commands:

\begin{verbatim}
read set.mm
open tex example.tex
show statement a1i /tex
show proof a1i /all/lemmon/renumber/tex
show statement uneq2 /tex
show proof uneq2 /all/lemmon/renumber/tex
close tex
\end{verbatim}

See Section~\ref{mathcomments} for information on comment markup and
Appendix~\ref{ASCII} for information on how math symbol translation is
specified.

To format and print the \LaTeX\ source, you will need the \LaTeX\
program, which is standard on most Linux installations and available for
Windows.  On Linux, in order to create a {\sc pdf} file, you will
typically type at the shell prompt
\begin{verbatim}
$ pdflatex example.tex
\end{verbatim}

\subsection{\texttt{open tex} Command}\index{\texttt{open tex} command}
Syntax:  \texttt{open tex} {\em file-name} [\texttt{/no{\char`\_}header}]

This command opens a file for writing \LaTeX\
source\index{latex@{\LaTeX}} and writes a \LaTeX\ header to the file.
\LaTeX\ source can be written with the \texttt{show proof}, \texttt{show
new{\char`\_}proof}, and \texttt{show statement} commands using the
\texttt{/tex} qualifier.

The mapping to \LaTeX\ symbols is defined in a special comment
containing a \texttt{\$t} token, described in Appendix~\ref{ASCII}.

There is an optional command qualifier:

    \texttt{/no{\char`\_}header} - This qualifier prevents a standard
        \LaTeX\ header and trailer
        from being included with the output \LaTeX\ code.


\subsection{\texttt{close tex} Command}\index{\texttt{close tex} command}
Syntax:  \texttt{close tex}

This command writes a trailer to any \LaTeX\ file\index{latex@{\LaTeX}}
that was opened with \texttt{open tex} (unless
\texttt{/no{\char`\_}header} was used with \texttt{open tex}) and closes
the \LaTeX\ file.


\section{Creating {\sc HTML} Output}\label{htmlout}

You can generate {\sc html} web pages given the
information in a database.
The database must already include the necessary typesetting information
(see section \ref{tcomment} for how to provide this information).
The ability to produce {\sc html} web pages was added in Metamath version
0.07.30.

To create an {\sc html} output file(s) for \texttt{\$a} or \texttt{\$p}
statement(s), use
\begin{quote}
    \texttt{show statement} {\em label-match} \texttt{/html}
\end{quote}
The output file will be named {\em label-match}\texttt{.html}
for each match.  When {\em
label-match} has wildcard (\texttt{*}) characters, all statements with
matching labels will have {\sc html} files produced for them.  Also,
when {\em label-match} has a wildcard (\texttt{*}) character, two additional
files, \texttt{mmdefinitions.html} and \texttt{mmascii.html} will be
produced.  To produce {\em only} these two additional files, you can use
\texttt{?*}, which will not match any statement label, in place of {\em
label-match}.

There are three other qualifiers for \texttt{show statement} that also
generate {\sc HTML} code.  These are \texttt{/alt{\char`\_}html},
\texttt{/brief{\char`\_}html}, and
\texttt{/brief{\char`\_}alt{\char`\_}html}, and are described in the
next section.

The command
\begin{quote}
    \texttt{show statement} {\em label-match} \texttt{/alt{\char`\_}html}
\end{quote}
does the same as \texttt{show statement} {\em label-match} \texttt{/html},
except that the {\sc html} code for the symbols is taken from
\texttt{althtmldef} statements instead of \texttt{htmldef} statements in
the \texttt{\$t} comment.

The command
\begin{verbatim}
show statement * /brief_html
\end{verbatim}
invokes a special mode that just produces definition and theorem lists
accompanied by their symbol strings, in a format suitable for copying and
pasting into another web page (such as the tutorial pages on the
Metamath web site).

Finally, the command
\begin{verbatim}
show statement * /brief_alt_html
\end{verbatim}
does the same as \texttt{show statement * / brief{\char`\_}html}
for the alternate {\sc html}
symbol representation.

A statement's comment can include a special notation that provides a
certain amount of control over the {\sc HTML} version of the comment.  See
Section~\ref{mathcomments} (p.~\pageref{mathcomments}) for the comment
markup features.

The \texttt{write theorem{\char`\_}list} and \texttt{write bibliography}
commands, which are described below, provide as a side effect complete
error checking for all of the features described in this
section.\index{error checking}

\subsection{\texttt{write theorem\_list}
Command}\index{\texttt{write theorem{\char`\_}list} command}
Syntax:  \texttt{write theorem{\char`\_}list}
[\texttt{/theorems{\char`\_}per{\char`\_}page} {\em number}]

This command writes a list of all of the \texttt{\$a} and \texttt{\$p}
statements in the database into a web page file
 called \texttt{mmtheorems.html}.
When additional files are needed, they are called
\texttt{mmtheorems2.html}, \texttt{mmtheorems3.html}, etc.

Optional command qualifier:

    \texttt{/theorems{\char`\_}per{\char`\_}page} {\em number} -
 This qualifier specifies the number of statements to
        write per web page.  The default is 100.

{\em Note:} In version 0.07.30\index{Metamath!limitations of version
0.07.30} of Metamath, the ``Nearby theorems'' links on the individual
web pages presuppose 100 theorems per page when linking to the theorem
list pages.  Therefore the \texttt{/theorems{\char`\_}per{\char`\_}page}
qualifier, if it specifies a number other than 100, will cause the
individual web pages to be out of sync and should not be used to
generate the main theorem list for the web site.  This is expected to be
fixed in a future version.


\subsection{\texttt{write bibliography}\label{wrbib}
Command}\index{\texttt{write bibliography} command}
Syntax:  \texttt{write bibliography} {\em filename}

This command reads an existing {\sc html} bibliographic cross-reference
file, normally called \texttt{mmbiblio.html}, and updates it per the
bibliographic links in the database comments.  The file is updated
between the {\sc html} comment lines \texttt{<!--
{\char`\#}START{\char`\#} -->} and \texttt{<!-- {\char`\#}END{\char`\#}
-->}.  The original input file is renamed to {\em
filename}\texttt{{\char`\~}1}.

A bibliographic reference is indicated with the reference name
in brackets, such as  \texttt{Theorem 3.1 of
[Monk] p.\ 22}.
See Section~\ref{htmlout} (p.~\pageref{htmlout}) for
syntax details.


\subsection{\texttt{write recent\_additions}
Command}\index{\texttt{write recent{\char`\_}additions} command}
Syntax:  \texttt{write recent{\char`\_}additions} {\em filename}
[\texttt{/limit} {\em number}]

This command reads an existing ``Recent Additions'' {\sc html} file,
normally called \texttt{mmrecent.html}, and updates it with the
descriptions of the most recently added theorems to the database.
 The file is updated between
the {\sc html} comment lines \texttt{<!-- {\char`\#}START{\char`\#} -->}
and \texttt{<!-- {\char`\#}END{\char`\#} -->}.  The original input file
is renamed to {\em filename}\texttt{{\char`\~}1}.

Optional command qualifier:

    \texttt{/limit} {\em number} -
 This qualifier specifies the number of most recent theorems to
   write to the output file.  The default is 100.


\section{Text File Utilities}

\subsection{\texttt{tools} Command}\index{\texttt{tools} command}
Syntax:  \texttt{tools}

This command invokes an easy-to-use, general purpose utility for
manipulating the contents of {\sc ascii} text files.  Upon typing
\texttt{tools}, the command-line prompt will change to \texttt{TOOLS>}
until you type \texttt{exit}.  The \texttt{tools} commands can be used
to perform simple, global edits on an input/output file,
such as making a character string substitution on each line, adding a
string to each line, and so on.  A typical use of this utility is
to build a \texttt{submit} input file to perform a common operation on a
list of statements obtained from \texttt{show label} or \texttt{show
usage}.

The actions of most of the \texttt{tools} commands can also be
performed with equivalent (and more powerful) Unix shell commands, and
some users may find those more efficient.  But for Windows users or
users not comfortable with Unix, \texttt{tools} provides an
easy-to-learn alternative that is adequate for most of the
script-building tasks needed to use the Metamath program effectively.

\subsection{\texttt{help} Command (in \texttt{tools})}
Syntax:  \texttt{help}

The \texttt{help} command lists the commands available in the
\texttt{tools} utility, along with a brief description.  Each command,
in turn, has its own help, such as \texttt{help add}.  As with
Metamath's \texttt{MM>} prompt, a complete command can be entered at
once, or just the command word can be typed, causing the program to
prompt for each argument.

\vskip 1ex
\noindent Line-by-line editing commands:

  \texttt{add} - Add a specified string to each line in a file.

  \texttt{clean} - Trim spaces and tabs on each line in a file; convert
         characters.

  \texttt{delete} - Delete a section of each line in a file.

  \texttt{insert} - Insert a string at a specified column in each line of
        a file.

  \texttt{substitute} - Make a simple substitution on each line of the file.

  \texttt{tag} - Like \texttt{add}, but restricted to a range of lines.

  \texttt{swap} - Swap the two halves of each line in a file.

\vskip 1ex
\noindent Other file-processing commands:

  \texttt{break} - Break up (tokenize) a file into a list of tokens (one per
        line).

  \texttt{build} - Build a file with multiple tokens per line from a list.

  \texttt{count} - Count the occurrences in a file of a specified string.

  \texttt{number} - Create a list of numbers.

  \texttt{parallel} - Put two files in parallel.

  \texttt{reverse} - Reverse the order of the lines in a file.

  \texttt{right} - Right-justify lines in a file (useful before sorting
         numbers).

%  \texttt{tag} - Tag edit updates in a program for revision control.

  \texttt{sort} - Sort the lines in a file with key starting at
         specified string.

  \texttt{match} - Extract lines containing (or not) a specified string.

  \texttt{unduplicate} - Eliminate duplicate occurrences of lines in a file.

  \texttt{duplicate} - Extract first occurrence of any line occurring
         more than

   \ \ \    once in a file, discarding lines occurring exactly once.

  \texttt{unique} - Extract lines occurring exactly once in a file.

  \texttt{type} (10 lines) - Display the first few lines in a file.
                  Similar to Unix \texttt{head}.

  \texttt{copy} - Similar to Unix \texttt{cat} but safe (same input
         and output file allowed).

  \texttt{submit} - Run a script containing \texttt{tools} commands.

\vskip 1ex

\noindent Note:
  \texttt{unduplicate}, \texttt{duplicate}, and \texttt{unique} also
 sort the lines as a side effect.


\subsection{Using \texttt{tools} to Build Metamath \texttt{submit}
Scripts}

The \texttt{break} command is typically used to break up a series of
statement labels, such as the output of Metamath's \texttt{show usage},
into one label per line.  The other \texttt{tools} commands can then be
used to add strings before and after each statement label to specify
commands to be performed on the statement.  The \texttt{parallel}
command is useful when a statement label must be mentioned more than
once on a line.

Very often a \texttt{submit} script for Metamath will require multiple
command lines for each statement being processed.  For example, you may
want to enter the Proof Assistant, \texttt{minimize{\char`\_}with} your
latest theorem, \texttt{save} the new proof, and \texttt{exit} the Proof
Assistant.  To accomplish this, you can build a file with these four
commands for each statement on a single line, separating each command
with a designated character such as \texttt{@}.  Then at the end you can
\texttt{substitute} each \texttt{@} with \texttt{{\char`\\}n} to break
up the lines into individual command lines (see \texttt{help
substitute}).


\subsection{Example of a \texttt{tools} Session}

To give you a quick feel for the \texttt{tools} utility, we show a
simple session where we create a file \texttt{n.txt} with 3 lines, add
strings before and after each line, and display the lines on the screen.
You can experiment with the various commands to gain experience with the
\texttt{tools} utility.

\begin{verbatim}
MM> tools
Entering the Text Tools utilities.
Type HELP for help, EXIT to exit.
TOOLS> number
Output file <n.tmp>? n.txt
First number <1>?
Last number <10>? 3
Increment <1>?
TOOLS> add
Input/output file? n.txt
String to add to beginning of each line <>? This is line
String to add to end of each line <>? .
The file n.txt has 3 lines; 3 were changed.
First change is on line 1:
This is line 1.
TOOLS> type n.txt
This is line 1.
This is line 2.
This is line 3.
TOOLS> exit
Exiting the Text Tools.
Type EXIT again to exit Metamath.
MM>
\end{verbatim}



\appendix
\chapter{Sample Representations}
\label{ASCII}

This Appendix provides a sample of {\sc ASCII} representations,
their corresponding traditional mathematical symbols,
and a discussion of their meanings
in the \texttt{set.mm} database.
These are provided in order of appearance.
This is only a partial list, and new definitions are routinely added.
A complete list is available at \url{http://metamath.org}.

These {\sc ASCII} representations, along
with information on how to display them,
are defined in the \texttt{set.mm} database file inside
a special comment called a \texttt{\$t} {\em
comment}\index{\texttt{\$t} comment} or {\em typesetting
comment.}\index{typesetting comment}
A typesetting comment
is indicated by the appearance of the
two-character string \texttt{\$t} at the beginning of the comment.
For more information,
see Section~\ref{tcomment}, p.~\pageref{tcomment}.

In the following table the ``{\sc ASCII}'' column shows the {\sc ASCII}
representation,
``Symbol'' shows the mathematical symbolic display
that corresponds to that {\sc ASCII} representation, ``Labels'' shows
the key label(s) that define the representation, and
``Description'' provides a description about the symbol.
As usual, ``iff'' is short for ``if and only if.''\index{iff}
In most cases the ``{\sc ASCII}'' column only shows
the key token, but it will sometimes show a sequence of tokens
if that is necessary for clarity.

{\setlength{\extrarowsep}{4pt} % Keep rows from being too close together
\begin{longtabu}   { @{} c c l X }
\textbf{ASCII} & \textbf{Symbol} & \textbf{Labels} & \textbf{Description} \\
\endhead
\texttt{|-} & $\vdash$ & &
  ``It is provable that...'' \\
\texttt{ph} & $\varphi$ & \texttt{wph} &
  The wff (boolean) variable phi,
  conventionally the first wff variable. \\
\texttt{ps} & $\psi$ & \texttt{wps} &
  The wff (boolean) variable psi,
  conventionally the second wff variable. \\
\texttt{ch} & $\chi$ & \texttt{wch} &
  The wff (boolean) variable chi,
  conventionally the third wff variable. \\
\texttt{-.} & $\lnot$ & \texttt{wn} &
  Logical not. E.g., if $\varphi$ is true, then $\lnot \varphi$ is false. \\
\texttt{->} & $\rightarrow$ & \texttt{wi} &
  Implies, also known as material implication.
  In classical logic the expression $\varphi \rightarrow \psi$ is true
  if either $\varphi$ is false or $\psi$ is true (or both), that is,
  $\varphi \rightarrow \psi$ has the same meaning as
  $\lnot \varphi \lor \psi$ (as proven in theorem \texttt{imor}). \\
\texttt{<->} & $\leftrightarrow$ &
  \hyperref[df-bi]{\texttt{df-bi}} &
  Biconditional (aka is-equals for boolean values).
  $\varphi \leftrightarrow \psi$ is true iff
  $\varphi$ and $\psi$ have the same value. \\
\texttt{\char`\\/} & $\lor$ &
  \makecell[tl]{{\hyperref[df-or]{\texttt{df-or}}}, \\
	         \hyperref[df-3or]{\texttt{df-3or}}} &
  Disjunction (logical ``or''). $\varphi \lor \psi$ is true iff
  $\varphi$, $\psi$, or both are true. \\
\texttt{/\char`\\} & $\land$ &
  \makecell[tl]{{\hyperref[df-an]{\texttt{df-an}}}, \\
                 \hyperref[df-3an]{\texttt{df-3an}}} &
  Conjunction (logical ``and''). $\varphi \land \psi$ is true iff
  both $\varphi$ and $\psi$ are true. \\
\texttt{A.} & $\forall$ &
  \texttt{wal} &
  For all; the wff $\forall x \varphi$ is true iff
  $\varphi$ is true for all values of $x$. \\
\texttt{E.} & $\exists$ &
  \hyperref[df-ex]{\texttt{df-ex}} &
  There exists; the wff
  $\exists x \varphi$ is true iff
  there is at least one $x$ where $\varphi$ is true. \\
\texttt{[ y / x ]} & $[ y / x ]$ &
  \hyperref[df-sb]{\texttt{df-sb}} &
  The wff $[ y / x ] \varphi$ produces
  the result when $y$ is properly substituted for $x$ in $\varphi$
  ($y$ replaces $x$).
  % This is elsb4
  % ( [ x / y ] z e. y <-> z e. x )
  For example,
  $[ x / y ] z \in y$ is the same as $z \in x$. \\
\texttt{E!} & $\exists !$ &
  \hyperref[df-eu]{\texttt{df-eu}} &
  There exists exactly one;
  $\exists ! x \varphi$ is true iff
  there is at least one $x$ where $\varphi$ is true. \\
\texttt{\{ y | phi \}}  & $ \{ y | \varphi \}$ &
  \hyperref[df-clab]{\texttt{df-clab}} &
  The class of all sets where $\varphi$ is true. \\
\texttt{=} & $ = $ &
  \hyperref[df-cleq]{\texttt{df-cleq}} &
  Class equality; $A = B$ iff $A$ equals $B$. \\
\texttt{e.} & $ \in $ &
  \hyperref[df-clel]{\texttt{df-clel}} &
  Class membership; $A \in B$ if $A$ is a member of $B$. \\
\texttt{{\char`\_}V} & {\rm V} &
  \hyperref[df-v]{\texttt{df-v}} &
  Class of all sets (not itself a set). \\
\texttt{C\_} & $ \subseteq $ &
  \hyperref[df-ss]{\texttt{df-ss}} &
  Subclass (subset); $A \subseteq B$ is true iff
  $A$ is a subclass of $B$. \\
\texttt{u.} & $ \cup $ &
  \hyperref[df-un]{\texttt{df-un}} &
  $A \cup B$ is the union of classes $A$ and $B$. \\
\texttt{i^i} & $ \cap $ &
  \hyperref[df-in]{\texttt{df-in}} &
  $A \cap B$ is the intersection of classes $A$ and $B$. \\
\texttt{\char`\\} & $ \setminus $ &
  \hyperref[df-dif]{\texttt{df-dif}} &
  $A \setminus B$ (set difference)
  is the class of all sets in $A$ except for those in $B$. \\
\texttt{(/)} & $ \varnothing $ &
  \hyperref[df-nul]{\texttt{df-nul}} &
  $ \varnothing $ is the empty set (aka null set). \\
\texttt{\char`\~P} & $ \cal P $ &
  \hyperref[df-pw]{\texttt{df-pw}} &
  Power class. \\
\texttt{<.\ A , B >.} & $\langle A , B \rangle$ &
  \hyperref[df-op]{\texttt{df-op}} &
  The ordered pair $\langle A , B \rangle$. \\
\texttt{( F ` A )} & $ ( F ` A ) $ &
  \hyperref[df-fv]{\texttt{df-fv}} &
  The value of function $F$ when applied to $A$. \\
\texttt{\_i} & $ i $ &
  \texttt{df-i} &
  The square root of negative one. \\
\texttt{x.} & $ \cdot $ &
  \texttt{df-mul} &
  Complex number multiplication; $2~\cdot~3~=~6$. \\
\texttt{CC} & $ \mathbb{C} $ &
  \texttt{df-c} &
  The set of complex numbers. \\
\texttt{RR} & $ \mathbb{R} $ &
  \texttt{df-r} &
  The set of real numbers. \\
\end{longtabu}
} % end of extrarowsep

\chapter{Compressed Proofs}
\label{compressed}\index{compressed proof}\index{proof!compressed}

The proofs in the \texttt{set.mm} set theory database are stored in compressed
format for efficiency.  Normally you needn't concern yourself with the
compressed format, since you can display it with the usual proof display tools
in the Metamath program (\texttt{show proof}\ldots) or convert it to the normal
RPN proof format described in Section~\ref{proof} (with \texttt{save proof}
{\em label} \texttt{/normal}).  However for sake of completeness we describe the
format here and show how it maps to the normal RPN proof format.

A compressed proof, located between \texttt{\$=} and \texttt{\$.}\ keywords, consists
of a left parenthesis, a sequence of statement labels, a right parenthesis,
and a sequence of upper-case letters \texttt{A} through \texttt{Z} (with optional
white space between them).  White space must surround the parentheses
and the labels.  The left parenthesis tells Metamath that a
compressed proof follows.  (A normal RPN proof consists of just a sequence of
labels, and a parenthesis is not a legal character in a label.)

The sequence of upper-case letters corresponds to a sequence of integers
with the following mapping.  Each integer corresponds to a proof step as
described later.
\begin{center}
  \texttt{A} = 1 \\
  \texttt{B} = 2 \\
   \ldots \\
  \texttt{T} = 20 \\
  \texttt{UA} = 21 \\
  \texttt{UB} = 22 \\
   \ldots \\
  \texttt{UT} = 40 \\
  \texttt{VA} = 41 \\
  \texttt{VB} = 42 \\
   \ldots \\
  \texttt{YT} = 120 \\
  \texttt{UUA} = 121 \\
   \ldots \\
  \texttt{YYT} = 620 \\
  \texttt{UUUA} = 621 \\
   etc.
\end{center}

In other words, \texttt{A} through \texttt{T} represent the
least-significant digit in base 20, and \texttt{U} through \texttt{Y}
represent zero or more most-significant digits in base 5, where the
digits start counting at 1 instead of the usual 0. With this scheme, we
don't need white space between these ``numbers.''

(In the design of the compressed proof format, only upper-case letters,
as opposed to say all non-whitespace printable {\sc ascii} characters other than
%\texttt{\$}, was chosen to make the compressed proof a little less
%displeasing to the eye, at the expense of a typical 20\% compression
\texttt{\$}, were chosen so as not to collide with most text editor
searches, at the expense of a typical 20\% compression
loss.  The base 5/base 20 grouping, as opposed to say base 6/base 19,
was chosen by experimentally determining the grouping that resulted in
best typical compression.)

The letter \texttt{Z} identifies (tags) a proof step that is identical to one
that occurs later on in the proof; it helps shorten the proof by not requiring
that identical proof steps be proved over and over again (which happens often
when building wff's).  The \texttt{Z} is placed immediately after the
least-significant digit (letters \texttt{A} through \texttt{T}) that ends the integer
corresponding to the step to later be referenced.

The integers that the upper-case letters correspond to are mapped to labels as
follows.  If the statement being proved has $m$ mandatory hypotheses, integers
1 through $m$ correspond to the labels of these hypotheses in the order shown
by the \texttt{show statement ... / full} command, i.e., the RPN order\index{RPN
order} of the mandatory
hypotheses.  Integers $m+1$ through $m+n$ correspond to the labels enclosed in
the parentheses of the compressed proof, in the order that they appear, where
$n$ is the number of those labels.  Integers $m+n+1$ on up don't directly
correspond to statement labels but point to proof steps identified with the
letter \texttt{Z}, so that these proof steps can be referenced later in the
proof.  Integer $m+n+1$ corresponds to the first step tagged with a \texttt{Z},
$m+n+2$ to the second step tagged with a \texttt{Z}, etc.  When the compressed
proof is converted to a normal proof, the entire subproof of a step tagged
with \texttt{Z} replaces the reference to that step.

For efficiency, Metamath works with compressed proofs directly, without
converting them internally to normal proofs.  In addition to the usual
error-checking, an error message is given if (1) a label in the label list in
parentheses does not refer to a previous \texttt{\$p} or \texttt{\$a} statement or a
non-mandatory hypothesis of the statement being proved and (2) a proof step
tagged with \texttt{Z} is referenced before the step tagged with the \texttt{Z}.

Just as in a normal proof under development (Section~\ref{unknown}), any step
or subproof that is not yet known may be represented with a single \texttt{?}.
White space does not have to appear between the \texttt{?}\ and the upper-case
letters (or other \texttt{?}'s) representing the remainder of the proof.

% April 1, 2004 Appendix C has been added back in with corrections.
%
% May 20, 2003 Appendix C was removed for now because there was a problem found
% by Bob Solovay
%
% Also, removed earlier \ref{formalspec} 's (3 cases above)
%
% Bob Solovay wrote on 30 Nov 2002:
%%%%%%%%%%%%% (start of email comment )
%      3. My next set of comments concern appendix C. I read this before I
% read Chapter 4. So I first noted that the system as presented in the
% Appendix lacked a certain formal property that I thought desirable. I
% then came up with a revised formal system that had this property. Upon
% reading Chapter 4, I noticed that the revised system was closer to the
% treatment in Chapter 4 than the system in Appendix C.
%
%         First a very minor correction:
%
%         On page 142 line 2: The condition that V(e) != V(f) should only be
% required of e, f in T such that e != f.
%
%         Here is a natural property [transitivity] that one would like
% the formal system to have:
%
%         Let Gamma be a set of statements. Suppose that the statement Phi
% is provable from Gamma and that the statement Psi is provable from Gamma
% \cup {Phi}. Then Psi is provable from Gamma.
%
%         I shall present an example to show that this property does not
% hold for the formal systems of Appendix C:
%
%         I write the example in metamath style:
%
% $c A B C D E $.
% $v x y
%
% ${
% tx $f A x $.
% ty $f B y $.
% ax1 $a C x y $.
% $}
%
% ${
% tx $f A x $.
% ty $f B y $.
% ax2-h1 $e C x y $.
% ax2 $a D y $.
% $}
%
% ${
% ty $f B y $.
% ax3-h1 $e D y $.
% ax3 $a E y $.
% $}
%
% $(These three axioms are Gamma $)
%
% ${
% tx $f A x $.
% ty $f B y $.
% Phi $p D y $=
% tx ty tx ty ax1 ax2 $.
% $}
%
% ${
% ty $f B y $.
% Psi $p E y $=
% ty ty Phi ax3 $.
% $}
%
%
% I omit the formal proofs of the following claims. [I will be glad to
% supply them upon request.]
%
% 1) Psi is not provable from Gamma;
%
% 2) Psi is provable from Gamma + Phi.
%
% Here "provable" refers to the formalism of Appendix C.
%
% The trouble of course is that Psi is lacking the variable declaration
%
% $f Ax $.
%
% In the Metamath system there is no trouble proving Psi. I attach a
% metamath file that shows this and which has been checked by the
% metamath program.
%
% I next want to indicate how I think the treatment in Appendix C should
% be revised so as to conform more closely to the metamath system of the
% main text. The revised system *does* have the transitivity property.
%
% We want to give revised definitions of "statement" and
% "provable". [cf. sections C.2.4. and C.2.5] Our new definitions will
% use the definitions given in Appendix C. So we take the following
% tack. We refer to the original notions as o-statement and o-provable. And
% we refer to the notions we are defining as n-statement and n-provable.
%
%         A n-statement is an o-statement in which the only variables
% that appear in the T component are mandatory.
%
%         To any o-statement we can associate its reduct which is a
% n-statement by dropping all the elements of T or D which contain
% non-mandatory variables.
%
%         An n-statement gamma is n-provable if there is an o-statement
% gamma' which has gamma as its reduct andf such that gamma' is
% o-provable.
%
%         It seems to me [though I am not completely sure on this point]
% that n-provability corresponds to metamath provability as discussed
% say in Chapter 4.
%
%         Attached to this letter is the metamath proof of Phi and Psi
% from Gamma discussed above.
%
%         I am still brooding over the question of whether metamath
% correctly formalizes set-theory. No doubt I will have some questions
% re this after my thoughts become clearer.
%%%%%%%%%%%%%%%% (end of email comment)

%%%%%%%%%%%%%%%% (start of 2nd email comment from Bob Solovay 1-Apr-04)
%
%         I hope that Appendix C is the one that gives a "formal" treatment
% of Metamath. At any rate, thats the appendix I want to comment on.
%
%         I'm going to suggest two changes in the definition.
%
%         First change (in the definition of statement): Require that the
% sets D, T, and E be finite.
%
%         Probably things are fine as you give them. But in the applications
% to the main metamath system they will always be finite, and its useful in
% thinking about things [at least for me] to stick to the finite case.
%
%         Second change:
%
%         First let me give an approximate description. Remove the dummy
% variables from the statement. Instead, include them in the proof.
%
%         More formally: Require that T consists of type declarations only
% for mandatory variables. Require that all the pairs in D consist of
% mandatory variables.
%
%         At the start of a proof we are allowed to declare a finite number
% of dummy variables [provided that none of them appear in any of the
% statements in E \cup {A}. We have to supply type declarations for all the
% dummy variables. We are allowed to add new $d statements referring to
% either the mandatory or dummy variables. But we require that no new $d
% statement references only mandatory variables.
%
%         I find this way of doing things more conceptual than the treatment
% in Appendix C. But the change [which I will use implicitly in later
% letters about doing Peano] is mainly aesthetic. I definitely claim that my
% results on doing Peano all apply to Metamath as it is presented in your
% book.
%
%         --Bob
%
%%%%%%%%%%%%%%%% (end of 2nd email comment)

%%
%% When uncommenting the below, also uncomment references above to {formalspec}
%%
\chapter{Metamath's Formal System}\label{formalspec}\index{Metamath!as a formal
system}

\section{Introduction}

\begin{quote}
  {\em Perfection is when there is no longer anything more to take away.}
    \flushright\sc Antoine de
     Saint-Exupery\footnote{\cite[p.~3-25]{Campbell}.}\\
\end{quote}\index{de Saint-Exupery, Antoine}

This appendix describes the theory behind the Metamath language in an abstract
way intended for mathematicians.  Specifically, we construct two
set-theo\-ret\-i\-cal objects:  a ``formal system'' (roughly, a set of syntax
rules, axioms, and logical rules) and its ``universe'' (roughly, the set of
theorems derivable in the formal system).  The Metamath computer language
provides us with a way to describe specific formal systems and, with the aid of
a proof provided by the user, to verify that given theorems
belong to their universes.

To understand this appendix, you need a basic knowledge of informal set theory.
It should be sufficient to understand, for example, Ch.\ 1 of Munkres' {\em
Topology} \cite{Munkres}\index{Munkres, James R.} or the
introductory set theory chapter
in many textbooks that introduce abstract mathematics. (Note that there are
minor notational differences among authors; e.g.\ Munkres uses $\subset$ instead
of our $\subseteq$ for ``subset.''  We use ``included in'' to mean ``a subset
of,'' and ``belongs to'' or ``is contained in'' to mean ``is an element of.'')
What we call a ``formal'' description here, unlike earlier, is actually an
informal description in the ordinary language of mathematicians.  However we
provide sufficient detail so that a mathematician could easily formalize it,
even in the language of Metamath itself if desired.  To understand the logic
examples at the end of this appendix, familiarity with an introductory book on
mathematical logic would be helpful.

\section{The Formal Description}

\subsection[Preliminaries]{Preliminaries\protect\footnotemark}%
\footnotetext{This section is taken mostly verbatim
from Tarski \cite[p.~63]{Tarski1965}\index{Tarski, Alfred}.}

By $\omega$ we denote the set of all natural numbers (non-negative integers).
Each natural number $n$ is identified with the set of all smaller numbers: $n =
\{ m | m < n \}$.  The formula $m < n$ is thus equivalent to the condition: $m
\in n$ and $m,n \in \omega$. In particular, 0 is the number zero and at the
same time the empty set $\varnothing$, $1=\{0\}$, $2=\{0,1\}$, etc. ${}^B A$
denotes the set of all functions on $B$ to $A$ (i.e.\ with domain $B$ and range
included in $A$).  The members of ${}^\omega A$ are what are called {\em simple
infinite sequences},\index{simple infinite sequence}
with all {\em terms}\index{term} in $A$.  In case $n \in \omega$, the
members of ${}^n A$ are referred to as {\em finite $n$-termed
sequences},\index{finite $n$-termed
sequence} again
with terms in $A$.  The consecutive terms (function values) of a finite or
infinite sequence $f$ are denoted by $f_0, f_1, \ldots ,f_n,\ldots$.  Every
finite sequence $f \in \bigcup _{n \in \omega} {}^n A$ uniquely determines the
number $n$ such that $f \in {}^n A$; $n$ is called the {\em
length}\index{length of a sequence ({$"|\ "|$})} of $f$ and
is denoted by $|f|$.  $\langle a \rangle$ is the sequence $f$ with $|f|=1$ and
$f_0=a$; $\langle a,b \rangle$ is the sequence $f$ with $|f|=2$, $f_0=a$,
$f_1=b$; etc.  Given two finite sequences $f$ and $g$, we denote by $f\frown g$
their {\em concatenation},\index{concatenation} i.e., the
finite sequence $h$ determined by the
conditions:
\begin{eqnarray*}
& |h| = |f|+|g|;&  \\
& h_n = f_n & \mbox{\ for\ } n < |f|;  \\
& h_{|f|+n} = g_n & \mbox{\ for\ } n < |g|.
\end{eqnarray*}

\subsection{Constants, Variables, and Expressions}

A formal system has a set of {\em symbols}\index{symbol!in
a formal system} denoted
by $\mbox{\em SM}$.  A
precise set-theo\-ret\-i\-cal definition of this set is unimportant; a symbol
could be considered a primitive or atomic element if we wish.  We assume this
set is divided into two disjoint subsets:  a set $\mbox{\em CN}$ of {\em
constants}\index{constant!in a formal system} and a set $\mbox{\em VR}$ of
{\em variables}.\index{variable!in a formal system}  $\mbox{\em CN}$ and
$\mbox{\em VR}$ are each assumed to consist of countably many symbols which
may be arranged in finite or simple infinite sequences $c_0, c_1, \ldots$ and
$v_0, v_1, \ldots$ respectively, without repeating terms.  We will represent
arbitrary symbols by metavariables $\alpha$, $\beta$, etc.

{\footnotesize\begin{quotation}
{\em Comment.} The variables $v_0, v_1, \ldots$ of our formal system
correspond to what are usually considered ``metavariables'' in
descriptions of specific formal systems in the literature.  Typically,
when describing a specific formal system a book will postulate a set of
primitive objects called variables, then proceed to describe their
properties using metavariables that range over them, never mentioning
again the actual variables themselves.  Our formal system does not
mention these primitive variable objects at all but deals directly with
metavariables, as its primitive objects, from the start.  This is a
subtle but key distinction you should keep in mind, and it makes our
definition of ``formal system'' somewhat different from that typically
found in the literature.  (So, the $\alpha$, $\beta$, etc.\ above are
actually ``metametavariables'' when used to represent $v_0, v_1,
\ldots$.)
\end{quotation}}

Finite sequences all terms of which are symbols are called {\em
expressions}.\index{expression!in a formal system}  $\mbox{\em EX}$ is
the set of all expressions; thus
\begin{displaymath}
\mbox{\em EX} = \bigcup _{n \in \omega} {}^n \mbox{\em SM}.
\end{displaymath}

A {\em constant-prefixed expression}\index{constant-prefixed expression}
is an expression of non-zero length
whose first term is a constant.  We denote the set of all constant-prefixed
expressions by $\mbox{\em EX}_C = \{ e \in \mbox{\em EX} | ( |e| > 0 \wedge
e_0 \in \mbox{\em CN} ) \}$.

A {\em constant-variable pair}\index{constant-variable pair}
is an expression of length 2 whose first term
is a constant and whose second term is a variable.  We denote the set of all
constant-variable pairs by $\mbox{\em EX}_2 = \{ e \in \mbox{\em EX}_C | ( |e|
= 2 \wedge e_1 \in \mbox{\em VR} ) \}$.


{\footnotesize\begin{quotation}
{\em Relationship to Metamath.} In general, the set $\mbox{\em SM}$
corresponds to the set of declared math symbols in a Metamath database, the
set $\mbox{\em CN}$ to those declared with \texttt{\$c} statements, and the set
$\mbox{\em VR}$ to those declared with \texttt{\$v} statements.  Of course a
Metamath database can only have a finite number of math symbols, whereas
formal systems in general can have an infinite number, although the number of
Metamath math symbols available is in principle unlimited.

The set $\mbox{\em EX}_C$ corresponds to the set of permissible expressions
for \texttt{\$e}, \texttt{\$a}, and \texttt{\$p} statements.  The set $\mbox{\em EX}_2$
corresponds to the set of permissible expressions for \texttt{\$f} statements.
\end{quotation}}

We denote by ${\cal V}(e)$ the set of all variables in an expression $e \in
\mbox{\em EX}$, i.e.\ the set of all $\alpha \in \mbox{\em VR}$ such that
$\alpha = e_n$ for some $n < |e|$.  We also denote (with abuse of notation) by
${\cal V}(E)$ the set of all variables in a collection of expressions $E
\subseteq \mbox{\em EX}$, i.e.\ $\bigcup _{e \in E} {\cal V}(e)$.


\subsection{Substitution}

Given a function $F$ from $\mbox{\em VR}$ to
$\mbox{\em EX}$, we
denote by $\sigma_{F}$ or just $\sigma$ the function from $\mbox{\em EX}$ to
$\mbox{\em EX}$ defined recursively for nonempty sequences by
\begin{eqnarray*}
& \sigma(<\alpha>) = F(\alpha) & \mbox{for\ } \alpha \in \mbox{\em VR}; \\
& \sigma(<\alpha>) = <\alpha> & \mbox{for\ } \alpha \not\in \mbox{\em VR}; \\
& \sigma(g \frown h) = \sigma(g) \frown
    \sigma(h) & \mbox{for\ } g,h \in \mbox{\em EX}.
\end{eqnarray*}
We also define $\sigma(\varnothing)=\varnothing$.  We call $\sigma$ a {\em
simultaneous substitution}\index{substitution!variable}\index{variable
substitution} (or just {\em substitution}) with {\em substitution
map}\index{substitution map} $F$.

We also denote (with abuse of notation) by $\sigma(E)$ a substitution on a
collection of expressions $E \subseteq \mbox{\em EX}$, i.e.\ the set $\{
\sigma(e) | e \in E \}$.  The collection $\sigma(E)$ may of course contain
fewer expressions than $E$ because duplicate expressions could result from the
substitution.

\subsection{Statements}

We denote by $\mbox{\em DV}$ the set of all
unordered pairs $\{\alpha, \beta \} \subseteq \mbox{\em VR}$ such that $\alpha
\neq \beta$.  $\mbox{\em DV}$ stands for ``distinct variables.''

A {\em pre-statement}\index{pre-statement!in a formal system} is a
quadruple $\langle D,T,H,A \rangle$ such that
$D\subseteq \mbox{\em DV}$, $T\subseteq \mbox{\em EX}_2$, $H\subseteq
\mbox{\em EX}_C$ and $H$ is finite,
$A\in \mbox{\em EX}_C$, ${\cal V}(H\cup\{A\}) \subseteq
{\cal V}(T)$, and $\forall e,f\in T {\ } {\cal V}(e) \neq {\cal V}(f)$ (or
equivalently, $e_1 \ne f_1$) whenever $e \neq f$. The terms of the quadruple are called {\em
distinct-variable restrictions},\index{disjoint-variable restriction!in a
formal system} {\em variable-type hypotheses},\index{variable-type
hypothesis!in a formal system} {\em logical hypotheses},\index{logical
hypothesis!in a formal system} and the {\em assertion}\index{assertion!in a
formal system} respectively.  We denote by $T_M$ ({\em mandatory variable-type
hypotheses}\index{mandatory variable-type hypothesis!in a formal system}) the
subset of $T$ such that ${\cal V}(T_M) ={\cal V}(H \cup \{A\})$.  We denote by
$D_M=\{\{\alpha,\beta\}\in D|\{\alpha,\beta\}\subseteq {\cal V}(T_M)\}$ the
{\em mandatory distinct-variable restrictions}\index{mandatory
disjoint-variable restriction!in a formal system} of the pre-statement.
The set
of {\em mandatory hypotheses}\index{mandatory hypothesis!in a formal system}
is $T_M\cup H$.  We call the quadruple $\langle D_M,T_M,H,A \rangle$
the {\em reduct}\index{reduct!in a formal system} of
the pre-statement $\langle D,T,H,A \rangle$.

A {\em statement} is the reduct of some pre-statement\index{statement!in a
formal system}.  A statement is therefore a special kind of pre-statement;
in particular, a statement is the reduct of itself.

{\footnotesize\begin{quotation}
{\em Comment.}  $T$ is a set of expressions, each of length 2, that associate
a set of constants (``variable types'') with a set of variables.  The
condition ${\cal V}(H\cup\{A\}) \subseteq {\cal V}(T) $
means that each variable occurring in a statement's logical
hypotheses or assertion must have an associated variable-type hypothesis or
``type declaration,'' in  analogy to a computer programming language, where a
variable must be declared to be say, a string or an integer.  The requirement
that $\forall e,f\in T \, e_1 \ne f_1$ for $e\neq f$
means that each variable must be
associated with a unique constant designating its variable type; e.g., a
variable might be a ``wff'' or a ``set'' but not both.

Distinct-variable restrictions are used to specify what variable substitutions
are permissible to make for the statement to remain valid.  For example, in
the theorem scheme of set theory $\lnot\forall x\,x=y$ we may not substitute
the same variable for both $x$ and $y$.  On the other hand, the theorem scheme
$x=y\to y=x$ does not require that $x$ and $y$ be distinct, so we do not
require a distinct-variable restriction, although having one
would cause no harm other than making the scheme less general.

A mandatory variable-type hypothesis is one whose variable exists in a logical
hypothesis or the assertion.  A provable pre-statement
(defined below) may require
non-mandatory variable-type hypotheses that effectively introduce ``dummy''
variables for use in its proof.  Any number of dummy variables might
be required by a specific proof; indeed, it has been shown by H.\
Andr\'{e}ka\index{Andr{\'{e}}ka, H.} \cite{Nemeti} that there is no finite
upper bound to the number of dummy variables needed to prove an arbitrary
theorem in first-order logic (with equality) having a fixed number $n>2$ of
individual variables.  (See also the Comment on p.~\pageref{nodd}.)
For this reason we do not set a finite size bound on the collections $D$ and
$T$, although in an actual application (Metamath database) these will of
course be finite, increased to whatever size is necessary as more
proofs are added.
\end{quotation}}

{\footnotesize\begin{quotation}
{\em Relationship to Metamath.} A pre-statement of a formal system
corresponds to an extended frame in a Metamath database
(Section~\ref{frames}).  The collections $D$, $T$, and $H$ correspond
respectively to the \texttt{\$d}, \texttt{\$f}, and \texttt{\$e}
statement collections in an extended frame.  The expression $A$
corresponds to the \texttt{\$a} (or \texttt{\$p}) statement in an
extended frame.

A statement of a formal system corresponds to a frame in a Metamath
database.
\end{quotation}}

\subsection{Formal Systems}

A {\em formal system}\index{formal system} is a
triple $\langle \mbox{\em CN},\mbox{\em
VR},\Gamma\rangle$ where $\Gamma$ is a set of statements.  The members of
$\Gamma$ are called {\em axiomatic statements}.\index{axiomatic
statement!in a formal system}  Sometimes we will refer to a
formal system by just $\Gamma$ when $\mbox{\em CN}$ and $\mbox{\em VR}$ are
understood.

Given a formal system $\Gamma$, the {\em closure}\index{closure}\footnote{This
definition of closure incorporates a simplification due to
Josh Purinton.\index{Purinton, Josh}.} of a
pre-statement
$\langle D,T,H,A \rangle$ is the smallest set $C$ of expressions
such that:
%\begin{enumerate}
%  \item $T\cup H\subseteq C$; and
%  \item If for some axiomatic statement
%    $\langle D_M',T_M',H',A' \rangle \in \Gamma_A$, for
%    some $E \subseteq C$, some $F \subseteq C-T$ (where ``-'' denotes
%    set difference), and some substitution
%    $\sigma$ we have
%    \begin{enumerate}
%       \item $\sigma(T_M') = E$ (where, as above, the $M$ denotes the
%           mandatory variable-type hypotheses of $T^A$);
%       \item $\sigma(H') = F$;
%       \item for all $\{\alpha,\beta\}\in D^A$ and $\subseteq
%         {\cal V}(T_M')$, for all $\gamma\in {\cal V}(\sigma(\langle \alpha
%         \rangle))$, and for all $\delta\in  {\cal V}(\sigma(\langle \beta
%         \rangle))$, we have $\{\gamma, \delta\} \in D$;
%   \end{enumerate}
%   then $\sigma(A') \in C$.
%\end{enumerate}
\begin{list}{}{\itemsep 0.0pt}
  \item[1.] $T\cup H\subseteq C$; and
  \item[2.] If for some axiomatic statement
    $\langle D_M',T_M',H',A' \rangle \in
       \Gamma$ and for some substitution
    $\sigma$ we have
    \begin{enumerate}
       \item[a.] $\sigma(T_M' \cup H') \subseteq C$; and
       \item[b.] for all $\{\alpha,\beta\}\in D_M'$, for all $\gamma\in
         {\cal V}(\sigma(\langle \alpha
         \rangle))$, and for all $\delta\in  {\cal V}(\sigma(\langle \beta
         \rangle))$, we have $\{\gamma, \delta\} \in D$;
   \end{enumerate}
   then $\sigma(A') \in C$.
\end{list}
A pre-statement $\langle D,T,H,A
\rangle$ is {\em provable}\index{provable statement!in a formal
system} if $A\in C$ i.e.\ if its assertion belongs to its
closure.  A statement is {\em provable} if it is
the reduct of a provable pre-statement.
The {\em universe}\index{universe of a formal system}
of a formal system is
the collection of all of its provable statements.  Note that the
set of axiomatic statements $\Gamma$ in a formal system is a subset of its
universe.

{\footnotesize\begin{quotation}
{\em Comment.} The first condition in the definition of closure simply says
that the hypotheses of the pre-statement are in its closure.

Condition 2(a) says that a substitution exists that makes the
mandatory hypotheses of an axiomatic statement exactly match some members of
the closure.  This is what we explicitly demonstrate in a Metamath language
proof.

%Conditions 2(a) and 2(b) say that a substitution exists that makes the
%(mandatory) hypotheses of an axiomatic statement exactly match some members of
%the closure.  This is what we explicitly demonstrate with a Metamath language
%proof.
%
%The set of expressions $F$ in condition 2(b) excludes the variable-type
%hypotheses; this is done because non-mandatory variable-type hypotheses are
%effectively ``dropped'' as irrelevant whereas logical hypotheses must be
%retained to achieve a consistent logical system.

Condition 2(b) describes how distinct-variable restrictions in the axiomatic
statement must be met.  It means that after a substitution for two variables
that must be distinct, the resulting two expressions must either contain no
variables, or if they do, they may not have variables in common, and each pair
of any variables they do have, with one variable from each expression, must be
specified as distinct in the original statement.
\end{quotation}}

{\footnotesize\begin{quotation}
{\em Relationship to Metamath.} Axiomatic statements
 and provable statements in a formal
system correspond to the frames for \texttt{\$a} and \texttt{\$p} statements
respectively in a Metamath database.  The set of axiomatic statements is a
subset of the set of provable statements in a formal system, although in a
Metamath database a \texttt{\$a} statement is distinguished by not having a
proof.  A Metamath language proof for a \texttt{\$p} statement tells the computer
how to explicitly construct a series of members of the closure ultimately
leading to a demonstration that the assertion
being proved is in the closure.  The actual closure typically contains
an infinite number of expressions.  A formal system itself does not have
an explicit object called a ``proof'' but rather the existence of a proof
is implied indirectly by membership of an assertion in a provable
statement's closure.  We do this to make the formal system easier
to describe in the language of set theory.

We also note that once established as provable, a statement may be considered
to acquire the same status as an axiomatic statement, because if the set of
axiomatic statements is extended with a provable statement, the universe of
the formal system remains unchanged (provided that $\mbox{\em VR}$ is
infinite).
In practice, this means we can build a hierarchy of provable statements to
more efficiently establish additional provable statements.  This is
what we do in Metamath when we allow proofs to reference previous
\texttt{\$p} statements as well as previous \texttt{\$a} statements.
\end{quotation}}

\section{Examples of Formal Systems}

{\footnotesize\begin{quotation}
{\em Relationship to Metamath.} The examples in this section, except Example~2,
are for the most part exact equivalents of the development in the set
theory database \texttt{set.mm}.  You may want to compare Examples~1, 3, and 5
to Section~\ref{metaaxioms}, Example 4 to Sections~\ref{metadefprop} and
\ref{metadefpred}, and Example 6 to
Section~\ref{setdefinitions}.\label{exampleref}
\end{quotation}}

\subsection{Example~1---Propositional Calculus}\index{propositional calculus}

Classical propositional calculus can be described by the following formal
system.  We assume the set of variables is infinite.  Rather than denoting the
constants and variables by $c_0, c_1, \ldots$ and $v_0, v_1, \ldots$, for
readability we will instead use more conventional symbols, with the
understanding of course that they denote distinct primitive objects.
Also for readability we may omit commas between successive terms of a
sequence; thus $\langle \mbox{wff\ } \varphi\rangle$ denotes
$\langle \mbox{wff}, \varphi\rangle$.

Let
\begin{itemize}
  \item[] $\mbox{\em CN}=\{\mbox{wff}, \vdash, \to, \lnot, (,)\}$
  \item[] $\mbox{\em VR}=\{\varphi,\psi,\chi,\ldots\}$
  \item[] $T = \{\langle \mbox{wff\ } \varphi\rangle,
             \langle \mbox{wff\ } \psi\rangle,
             \langle \mbox{wff\ } \chi\rangle,\ldots\}$, i.e.\ those
             expressions of length 2 whose first member is $\mbox{\rm wff}$
             and whose second member belongs to $\mbox{\em VR}$.\footnote{For
convenience we let $T$ be an infinite set; the definition of a statement
permits this in principle.  Since a Metamath source file has a finite size, in
practice we must of course use appropriate finite subsets of this $T$,
specifically ones containing at least the mandatory variable-type
hypotheses.  Similarly, in the source file we introduce new variables as
required, with the understanding that a potentially infinite number of
them are available.}
\noindent Then $\Gamma$ consists of the axiomatic statements that
are the reducts of the following pre-statements:
    \begin{itemize}
      \item[] $\langle\varnothing,T,\varnothing,
               \langle \mbox{wff\ }(\varphi\to\psi)\rangle\rangle$
      \item[] $\langle\varnothing,T,\varnothing,
               \langle \mbox{wff\ }\lnot\varphi\rangle\rangle$
      \item[] $\langle\varnothing,T,\varnothing,
               \langle \vdash(\varphi\to(\psi\to\varphi))
               \rangle\rangle$
      \item[] $\langle\varnothing,T,
               \varnothing,
               \langle \vdash((\varphi\to(\psi\to\chi))\to
               ((\varphi\to\psi)\to(\varphi\to\chi)))
               \rangle\rangle$
      \item[] $\langle\varnothing,T,
               \varnothing,
               \langle \vdash((\lnot\varphi\to\lnot\psi)\to
               (\psi\to\varphi))\rangle\rangle$
      \item[] $\langle\varnothing,T,
               \{\langle\vdash(\varphi\to\psi)\rangle,
                 \langle\vdash\varphi\rangle\},
               \langle\vdash\psi\rangle\rangle$
    \end{itemize}
\end{itemize}

(For example, the reduct of $\langle\varnothing,T,\varnothing,
               \langle \mbox{wff\ }(\varphi\to\psi)\rangle\rangle$
is
\begin{itemize}
\item[] $\langle\varnothing,
\{\langle \mbox{wff\ } \varphi\rangle,
             \langle \mbox{wff\ } \psi\rangle\},
             \varnothing,
               \langle \mbox{wff\ }(\varphi\to\psi)\rangle\rangle$,
\end{itemize}
which is the first axiomatic statement.)

We call the members of $\mbox{\em VR}$ {\em wff variables} or (in the context
of first-order logic which we will describe shortly) {\em wff metavariables}.
Note that the symbols $\phi$, $\psi$, etc.\ denote actual specific members of
$\mbox{\em VR}$; they are not metavariables of our expository language (which
we denote with $\alpha$, $\beta$, etc.) but are instead (meta)constant symbols
(members of $\mbox{\em SM}$) from the point of view of our expository
language.  The equivalent system of propositional calculus described in
\cite{Tarski1965} also uses the symbols $\phi$, $\psi$, etc.\ to denote wff
metavariables, but in \cite{Tarski1965} unlike here those are metavariables of
the expository language and not primitive symbols of the formal system.

The first two statements define wffs: if $\varphi$ and $\psi$ are wffs, so is
$(\varphi \to \psi)$; if $\varphi$ is a wff, so is $\lnot\varphi$. The next
three are the axioms of propositional calculus: if $\varphi$ and $\psi$ are
wffs, then $\vdash (\varphi \to (\psi \to \varphi))$ is an (axiomatic)
theorem; etc. The
last is the rule of modus ponens: if $\varphi$ and $\psi$ are wffs, and
$\vdash (\varphi\to\psi)$ and $\vdash \varphi$ are theorems, then $\vdash
\psi$ is a theorem.

The correspondence to ordinary propositional calculus is as follows.  We
consider only provable statements of the form $\langle\varnothing,
T,\varnothing,A\rangle$ with $T$ defined as above.  The first term of the
assertion $A$ of any such statement is either ``wff'' or ``$\vdash$''.  A
statement for which the first term is ``wff'' is a {\em wff} of propositional
calculus, and one where the first term is ``$\vdash$'' is a {\em
theorem (scheme)} of propositional calculus.

The universe of this formal system also contains many other provable
statements.  Those with distinct-variable restrictions are irrelevant because
propositional calculus has no constraints on substitutions.  Those that have
logical hypotheses we call {\em inferences}\index{inference} when
the logical hypotheses are of the form
$\langle\vdash\rangle\frown w$ where $w$ is a wff (with the leading constant
term ``wff'' removed).  Inferences (other than the modus ponens rule) are not a
proper part of propositional calculus but are convenient to use when building a
hierarchy of provable statements.  A provable statement with a nonsense
hypothesis such as $\langle \to,\vdash,\lnot\rangle$, and this same expression
as its assertion, we consider irrelevant; no use can be made of it in
proving theorems, since there is no way to eliminate the nonsense hypothesis.

{\footnotesize\begin{quotation}
{\em Comment.} Our use of parentheses in the definition of a wff illustrates
how axiomatic statements should be carefully stated in a way that
ties in unambiguously with the substitutions allowed by the formal system.
There are many ways we could have defined wffs---for example, Polish
prefix notation would have allowed us to omit parentheses entirely, at
the expense of readability---but we must define them in a way that is
unambiguous.  For example, if we had omitted parentheses from the
definition of $(\varphi\to \psi)$, the wff $\lnot\varphi\to \psi$ could
be interpreted as either $\lnot(\varphi\to\psi)$ or $(\lnot\varphi\to\psi)$
and would have allowed us to prove nonsense.  Note that there is no
concept of operator binding precedence built into our formal system.
\end{quotation}}

\begin{sloppy}
\subsection{Example~2---Predicate Calculus with Equality}\index{predicate
calculus}
\end{sloppy}

Here we extend Example~1 to include predicate calculus with equality,
illustrating the use of distinct-variable restrictions.  This system is the
same as Tarski's system $\mathfrak{S}_2$ in \cite{Tarski1965} (except that the
axioms of propositional calculus are different but equivalent, and a redundant
axiom is omitted).  We extend $\mbox{\em CN}$ with the constants
$\{\mbox{var},\forall,=\}$.  We extend $\mbox{\em VR}$ with an infinite set of
{\em individual metavariables}\index{individual
metavariable} $\{x,y,z,\ldots\}$ and denote this subset
$\mbox{\em Vr}$.

We also join to $\mbox{\em CN}$ a possibly infinite set $\mbox{\em Pr}$ of {\em
predicates} $\{R,S,\ldots\}$.  We associate with $\mbox{\em Pr}$ a function
$\mbox{rnk}$ from $\mbox{\em Pr}$ to $\omega$, and for $\alpha\in \mbox{\em
Pr}$ we call $\mbox{rnk}(\alpha)$ the {\em rank} of the predicate $\alpha$,
which is simply the number of ``arguments'' that the predicate has.  (Most
applications of predicate calculus will have a finite number of predicates;
for example, set theory has the single two-argument or binary predicate $\in$,
which is usually written with its arguments surrounding the predicate symbol
rather than with the prefix notation we will use for the general case.)  As a
device to facilitate our discussion, we will let $\mbox{\em Vs}$ be any fixed
one-to-one function from $\omega$ to $\mbox{\em Vr}$; thus $\mbox{\em Vs}$ is
any simple infinite sequence of individual metavariables with no repeating
terms.

In this example we will not include the function symbols that are often part of
formalizations of predicate calculus.  Using metalogical arguments that are
beyond the scope of our discussion, it can be shown that our formalization is
equivalent when functions are introduced via appropriate definitions.

We extend the set $T$ defined in Example~1 with the expressions
$\{\langle \mbox{var\ } x\rangle,$ $ \langle \mbox{var\ } y\rangle, \langle
\mbox{var\ } z\rangle,\ldots\}$.  We extend the $\Gamma$ above
with the axiomatic statements that are the reducts of the following
pre-statements:
\begin{list}{}{\itemsep 0.0pt}
      \item[] $\langle\varnothing,T,\varnothing,
               \langle \mbox{wff\ }\forall x\,\varphi\rangle\rangle$
      \item[] $\langle\varnothing,T,\varnothing,
               \langle \mbox{wff\ }x=y\rangle\rangle$
      \item[] $\langle\varnothing,T,
               \{\langle\vdash\varphi\rangle\},
               \langle\vdash\forall x\,\varphi\rangle\rangle$
      \item[] $\langle\varnothing,T,\varnothing,
               \langle \vdash((\forall x(\varphi\to\psi)
                  \to(\forall x\,\varphi\to\forall x\,\psi))
               \rangle\rangle$
      \item[] $\langle\{\{x,\varphi\}\},T,\varnothing,
               \langle \vdash(\varphi\to\forall x\,\varphi)
               \rangle\rangle$
      \item[] $\langle\{\{x,y\}\},T,\varnothing,
               \langle \vdash\lnot\forall x\lnot x=y
               \rangle\rangle$
      \item[] $\langle\varnothing,T,\varnothing,
               \langle \vdash(x=z
                  \to(x=y\to z=y))
               \rangle\rangle$
      \item[] $\langle\varnothing,T,\varnothing,
               \langle \vdash(y=z
                  \to(x=y\to x=z))
               \rangle\rangle$
\end{list}
These are the axioms not involving predicate symbols. The first two statements
extend the definition of a wff.  The third is the rule of generalization.  The
fifth states, in effect, ``For a wff $\varphi$ and variable $x$,
$\vdash(\varphi\to\forall x\,\varphi)$, provided that $x$ does not occur in
$\varphi$.''  The sixth states ``For variables $x$ and $y$,
$\vdash\lnot\forall x\lnot x = y$, provided that $x$ and $y$ are distinct.''
(This proviso is not necessary but was included by Tarski to
weaken the axiom and still show that the system is logically complete.)

Finally, for each predicate symbol $\alpha\in \mbox{\em Pr}$, we add to
$\Gamma$ an axiomatic statement, extending the definition of wff,
that is the reduct of the following pre-statement:
\begin{displaymath}
    \langle\varnothing,T,\varnothing,
            \langle \mbox{wff},\alpha\rangle\
            \frown \mbox{\em Vs}\restriction\mbox{rnk}(\alpha)\rangle
\end{displaymath}
and for each $\alpha\in \mbox{\em Pr}$ and each $n < \mbox{rnk}(\alpha)$
we add to $\Gamma$ an equality axiom that is the reduct of the
following pre-statement:
\begin{eqnarray*}
    \lefteqn{\langle\varnothing,T,\varnothing,
            \langle
      \vdash,(,\mbox{\em Vs}_n,=,\mbox{\em Vs}_{\mbox{rnk}(\alpha)},\to,
            (,\alpha\rangle\frown \mbox{\em Vs}\restriction\mbox{rnk}(\alpha)} \\
  & & \frown
            \langle\to,\alpha\rangle\frown \mbox{\em Vs}\restriction n\frown
            \langle \mbox{\em Vs}_{\mbox{rnk}(\alpha)}\rangle \\
 & & \frown
            \mbox{\em Vs}\restriction(\mbox{rnk}(\alpha)\setminus(n+1))\frown
            \langle),)\rangle\rangle
\end{eqnarray*}
where $\restriction$ denotes function domain restriction and $\setminus$
denotes set difference.  Recall that a subscript on $\mbox{\em Vs}$
denotes one of its terms.  (In the above two axiom sets commas are placed
between successive terms of sequences to prevent ambiguity, and if you examine
them with care you will be able to distinguish those parentheses that denote
constant symbols from those of our expository language that delimit function
arguments.  Although it might have been better to use boldface for our
primitive symbols, unfortunately boldface was not available for all characters
on the \LaTeX\ system used to typeset this text.)  These seemingly forbidding
axioms can be understood by analogy to concatenation of substrings in a
computer language.  They are actually relatively simple for each specific case
and will become clearer by looking at the special case of a binary predicate
$\alpha = R$ where $\mbox{rnk}(R)=2$.  Letting $\mbox{\em Vs}$ be the sequence
$\langle x,y,z,\ldots\rangle$, the axioms we would add to $\Gamma$ for this
case would be the wff extension and two equality axioms that are the
reducts of the pre-statements:
\begin{list}{}{\itemsep 0.0pt}
      \item[] $\langle\varnothing,T,\varnothing,
               \langle \mbox{wff\ }R x y\rangle\rangle$
      \item[] $\langle\varnothing,T,\varnothing,
               \langle \vdash(x=z
                  \to(R x y \to R z y))
               \rangle\rangle$
      \item[] $\langle\varnothing,T,\varnothing,
               \langle \vdash(y=z
                  \to(R x y \to R x z))
               \rangle\rangle$
\end{list}
Study these carefully to see how the general axioms above evaluate to
them.  In practice, typically only a few special cases such as this would be
needed, and in any case the Metamath language will only permit us to describe
a finite number of predicates, as opposed to the infinite number permitted by
the formal system.  (If an infinite number should be needed for some reason,
we could not define the formal system directly in the Metamath language but
could instead define it metalogically under set theory as we
do in this appendix, and only the underlying set theory, with its single
binary predicate, would be defined directly in the Metamath language.)


{\footnotesize\begin{quotation}
{\em Comment.}  As we noted earlier, the specific variables denoted by the
symbols $x,y,z,\ldots\in \mbox{\em Vr}\subseteq \mbox{\em VR}\subseteq
\mbox{\em SM}$ in Example~2 are not the actual variables of ordinary predicate
calculus but should be thought of as metavariables ranging over them.  For
example, a distinct-variable restriction would be meaningless for actual
variables of ordinary predicate calculus since two different actual variables
are by definition distinct.  And when we talk about an arbitrary
representative $\alpha\in \mbox{\em Vr}$, $\alpha$ is a metavariable (in our
expository language) that ranges over metavariables (which are primitives of
our formal system) each of which ranges over the actual individual variables
of predicate calculus (which are never mentioned in our formal system).

The constant called ``var'' above is called \texttt{setvar} in the
\texttt{set.mm} database file, but it means the same thing.  I felt
that ``var'' is a more meaningful name in the context of predicate
calculus, whose use is not limited to set theory.  For consistency we
stick with the name ``var'' throughout this Appendix, even after set
theory is introduced.
\end{quotation}}

\subsection{Free Variables and Proper Substitution}\index{free variable}
\index{proper substitution}\index{substitution!proper}

Typical representations of mathematical axioms use concepts such
as ``free variable,'' ``bound variable,'' and ``proper substitution''
as primitive notions.
A free variable is a variable that
is not a parameter of any container expression.
A bound variable is the opposite of a free variable; it is a
a variable that has been bound in a container expression.
For example, in the expression $\forall x \varphi$ (for all $x$, $\varphi$
is true), the variable $x$
is bound within the for-all ($\forall$) expression.
It is possible to change one variable to another, and that process is called
``proper substitution.''
In most books, proper substitution has a somewhat complicated recursive
definition with multiple cases based on the occurrences of free and
bound variables.
You may consult
\cite[ch.\ 3--4]{Hamilton}\index{Hamilton, Alan G.} (as well as
many other texts) for more formal details about these terms.

Using these concepts as \texttt{primitives} creates complications
for computer implementations.

In the system of Example~2, there are no primitive notions of free variable
and proper substitution.  Tarski \cite{Tarski1965} shows that this system is
logically equivalent to the more typical textbook systems that do have these
primitive notions, if we introduce these notions with appropriate definitions
and metalogic.  We could also define axioms for such systems directly,
although the recursive definitions of free variable and proper substitution
would be messy and awkward to work with.  Instead, we mention two devices that
can be used in practice to mimic these notions.  (1) Instead of introducing
special notation to express (as a logical hypothesis) ``where $x$ is not free
in $\varphi$'' we can use the logical hypothesis $\vdash(\varphi\to\forall
x\,\varphi)$.\label{effectivelybound}\index{effectively
not free}\footnote{This is a slightly weaker requirement than ``where $x$ is
not free in $\varphi$.''  If we let $\varphi$ be $x=x$, we have the theorem
$(x=x\to\forall x\,x=x)$ which satisfies the hypothesis, even though $x$ is
free in $x=x$ .  In a case like this we say that $x$ is {\em effectively not
free}\index{effectively not free} in $x=x$, since $x=x$ is logically
equivalent to $\forall x\,x=x$ in which $x$ is bound.} (2) It can be shown
that the wff $((x=y\to\varphi)\wedge\exists x(x=y\wedge\varphi))$ (with the
usual definitions of $\wedge$ and $\exists$; see Example~4 below) is logically
equivalent to ``the wff that results from proper substitution of $y$ for $x$
in $\varphi$.''  This works whether or not $x$ and $y$ are distinct.

\subsection{Metalogical Completeness}\index{metalogical completeness}

In the system of Example~2, the
following are provable pre-statements (and their reducts are
provable statements):
\begin{eqnarray*}
      & \langle\{\{x,y\}\},T,\varnothing,
               \langle \vdash\lnot\forall x\lnot x=y
               \rangle\rangle & \\
     &  \langle\varnothing,T,\varnothing,
               \langle \vdash\lnot\forall x\lnot x=x
               \rangle\rangle &
\end{eqnarray*}
whereas the following pre-statement is not to my knowledge provable (but
in any case we will pretend it's not for sake of illustration):
\begin{eqnarray*}
     &  \langle\varnothing,T,\varnothing,
               \langle \vdash\lnot\forall x\lnot x=y
               \rangle\rangle &
\end{eqnarray*}
In other words, we can prove ``$\lnot\forall x\lnot x=y$ where $x$ and $y$ are
distinct'' and separately prove ``$\lnot\forall x\lnot x=x$'', but we can't
prove the combined general case ``$\lnot\forall x\lnot x=y$'' that has no
proviso.  Now this does not compromise logical completeness, because the
variables are really metavariables and the two provable cases together cover
all possible cases.  The third case can be considered a metatheorem whose
direct proof, using the system of Example~2, lies outside the capability of the
formal system.

Also, in the system of Example~2 the following pre-statement is not to my
knowledge provable (again, a conjecture that we will pretend to be the case):
\begin{eqnarray*}
     & \langle\varnothing,T,\varnothing,
               \langle \vdash(\forall x\, \varphi\to\varphi)
               \rangle\rangle &
\end{eqnarray*}
Instead, we can only prove specific cases of $\varphi$ involving individual
metavariables, and by induction on formula length, prove as a metatheorem
outside of our formal system the general statement above.  The details of this
proof are found in \cite{Kalish}.

There does, however, exist a system of predicate calculus in which all such
``simple metatheorems'' as those above can be proved directly, and we present
it in Example~3. A {\em simple metatheorem}\index{simple metatheorem}
is any statement of the formal
system of Example~2 where all distinct variable restrictions consist of either
two individual metavariables or an individual metavariable and a wff
metavariable, and which is provable by combining cases outside the system as
above.  A system is {\em metalogically complete}\index{metalogical
completeness} if all of its simple
metatheorems are (directly) provable statements. The precise definition of
``simple metatheorem'' and the proof of the ``metalogical completeness'' of
Example~3 is found in Remark 9.6 and Theorem 9.7 of \cite{Megill}.\index{Megill,
Norman}

\begin{sloppy}
\subsection{Example~3---Metalogically Complete Predicate
Calculus with
Equality}
\end{sloppy}

For simplicity we will assume there is one binary predicate $R$;
this system suffices for set theory, where the $R$ is of course the $\in$
predicate.  We label the axioms as they appear in \cite{Megill}.  This
system is logically equivalent to that of Example~2 (when the latter is
restricted to this single binary predicate) but is also metalogically
complete.\index{metalogical completeness}

Let
\begin{itemize}
  \item[] $\mbox{\em CN}=\{\mbox{wff}, \mbox{var}, \vdash, \to, \lnot, (,),\forall,=,R\}$.
  \item[] $\mbox{\em VR}=\{\varphi,\psi,\chi,\ldots\}\cup\{x,y,z,\ldots\}$.
  \item[] $T = \{\langle \mbox{wff\ } \varphi\rangle,
             \langle \mbox{wff\ } \psi\rangle,
             \langle \mbox{wff\ } \chi\rangle,\ldots\}\cup
       \{\langle \mbox{var\ } x\rangle, \langle \mbox{var\ } y\rangle, \langle
       \mbox{var\ }z\rangle,\ldots\}$.

\noindent Then
  $\Gamma$ consists of the reducts of the following pre-statements:
    \begin{itemize}
      \item[] $\langle\varnothing,T,\varnothing,
               \langle \mbox{wff\ }(\varphi\to\psi)\rangle\rangle$
      \item[] $\langle\varnothing,T,\varnothing,
               \langle \mbox{wff\ }\lnot\varphi\rangle\rangle$
      \item[] $\langle\varnothing,T,\varnothing,
               \langle \mbox{wff\ }\forall x\,\varphi\rangle\rangle$
      \item[] $\langle\varnothing,T,\varnothing,
               \langle \mbox{wff\ }x=y\rangle\rangle$
      \item[] $\langle\varnothing,T,\varnothing,
               \langle \mbox{wff\ }Rxy\rangle\rangle$
      \item[(C1$'$)] $\langle\varnothing,T,\varnothing,
               \langle \vdash(\varphi\to(\psi\to\varphi))
               \rangle\rangle$
      \item[(C2$'$)] $\langle\varnothing,T,
               \varnothing,
               \langle \vdash((\varphi\to(\psi\to\chi))\to
               ((\varphi\to\psi)\to(\varphi\to\chi)))
               \rangle\rangle$
      \item[(C3$'$)] $\langle\varnothing,T,
               \varnothing,
               \langle \vdash((\lnot\varphi\to\lnot\psi)\to
               (\psi\to\varphi))\rangle\rangle$
      \item[(C4$'$)] $\langle\varnothing,T,
               \varnothing,
               \langle \vdash(\forall x(\forall x\,\varphi\to\psi)\to
                 (\forall x\,\varphi\to\forall x\,\psi))\rangle\rangle$
      \item[(C5$'$)] $\langle\varnothing,T,
               \varnothing,
               \langle \vdash(\forall x\,\varphi\to\varphi)\rangle\rangle$
      \item[(C6$'$)] $\langle\varnothing,T,
               \varnothing,
               \langle \vdash(\forall x\forall y\,\varphi\to
                 \forall y\forall x\,\varphi)\rangle\rangle$
      \item[(C7$'$)] $\langle\varnothing,T,
               \varnothing,
               \langle \vdash(\lnot\varphi\to\forall x\lnot\forall x\,\varphi
                 )\rangle\rangle$
      \item[(C8$'$)] $\langle\varnothing,T,
               \varnothing,
               \langle \vdash(x=y\to(x=z\to y=z))\rangle\rangle$
      \item[(C9$'$)] $\langle\varnothing,T,
               \varnothing,
               \langle \vdash(\lnot\forall x\, x=y\to(\lnot\forall x\, x=z\to
                 (y=z\to\forall x\, y=z)))\rangle\rangle$
      \item[(C10$'$)] $\langle\varnothing,T,
               \varnothing,
               \langle \vdash(\forall x(x=y\to\forall x\,\varphi)\to
                 \varphi))\rangle\rangle$
      \item[(C11$'$)] $\langle\varnothing,T,
               \varnothing,
               \langle \vdash(\forall x\, x=y\to(\forall x\,\varphi
               \to\forall y\,\varphi))\rangle\rangle$
      \item[(C12$'$)] $\langle\varnothing,T,
               \varnothing,
               \langle \vdash(x=y\to(Rxz\to Ryz))\rangle\rangle$
      \item[(C13$'$)] $\langle\varnothing,T,
               \varnothing,
               \langle \vdash(x=y\to(Rzx\to Rzy))\rangle\rangle$
      \item[(C15$'$)] $\langle\varnothing,T,
               \varnothing,
               \langle \vdash(\lnot\forall x\, x=y\to(x=y\to(\varphi
                 \to\forall x(x=y\to\varphi))))\rangle\rangle$
      \item[(C16$'$)] $\langle\{\{x,y\}\},T,
               \varnothing,
               \langle \vdash(\forall x\, x=y\to(\varphi\to\forall x\,\varphi)
                 )\rangle\rangle$
      \item[(C5)] $\langle\{\{x,\varphi\}\},T,\varnothing,
               \langle \vdash(\varphi\to\forall x\,\varphi)
               \rangle\rangle$
      \item[(MP)] $\langle\varnothing,T,
               \{\langle\vdash(\varphi\to\psi)\rangle,
                 \langle\vdash\varphi\rangle\},
               \langle\vdash\psi\rangle\rangle$
      \item[(Gen)] $\langle\varnothing,T,
               \{\langle\vdash\varphi\rangle\},
               \langle\vdash\forall x\,\varphi\rangle\rangle$
    \end{itemize}
\end{itemize}

While it is known that these axioms are ``metalogically complete,'' it is
not known whether they are independent (i.e.\ none is
redundant) in the metalogical sense; specifically, whether any axiom (possibly
with additional non-mandatory distinct-variable restrictions, for use with any
dummy variables in its proof) is provable from the others.  Note that
metalogical independence is a weaker requirement than independence in the
usual logical sense.  Not all of the above axioms are logically independent:
for example, C9$'$ can be proved as a metatheorem from the others, outside the
formal system, by combining the possible cases of distinct variables.

\subsection{Example~4---Adding Definitions}\index{definition}
There are several ways to add definitions to a formal system.  Probably the
most proper way is to consider definitions not as part of the formal system at
all but rather as abbreviations that are part of the expository metalogic
outside the formal system.  For convenience, though, we may use the formal
system itself to incorporate definitions, adding them as axiomatic extensions
to the system.  This could be done by adding a constant representing the
concept ``is defined as'' along with axioms for it. But there is a nicer way,
at least in this writer's opinion, that introduces definitions as direct
extensions to the language rather than as extralogical primitive notions.  We
introduce additional logical connectives and provide axioms for them.  For
systems of logic such as Examples 1 through 3, the additional axioms must be
conservative in the sense that no wff of the original system that was not a
theorem (when the initial term ``wff'' is replaced by ``$\vdash$'' of course)
becomes a theorem of the extended system.  In this example we extend Example~3
(or 2) with standard abbreviations of logic.

We extend $\mbox{\em CN}$ of Example~3 with new constants $\{\leftrightarrow,
\wedge,\vee,\exists\}$, corresponding to logical equivalence,\index{logical
equivalence ($\leftrightarrow$)}\index{biconditional ($\leftrightarrow$)}
conjunction,\index{conjunction ($\wedge$)} disjunction,\index{disjunction
($\vee$)} and the existential quantifier.\index{existential quantifier
($\exists$)}  We extend $\Gamma$ with the axiomatic statements that are
the reducts of the following pre-statements:
\begin{list}{}{\itemsep 0.0pt}
      \item[] $\langle\varnothing,T,\varnothing,
               \langle \mbox{wff\ }(\varphi\leftrightarrow\psi)\rangle\rangle$
      \item[] $\langle\varnothing,T,\varnothing,
               \langle \mbox{wff\ }(\varphi\vee\psi)\rangle\rangle$
      \item[] $\langle\varnothing,T,\varnothing,
               \langle \mbox{wff\ }(\varphi\wedge\psi)\rangle\rangle$
      \item[] $\langle\varnothing,T,\varnothing,
               \langle \mbox{wff\ }\exists x\, \varphi\rangle\rangle$
  \item[] $\langle\varnothing,T,\varnothing,
     \langle\vdash ( ( \varphi \leftrightarrow \psi ) \to
     ( \varphi \to \psi ) )\rangle\rangle$
  \item[] $\langle\varnothing,T,\varnothing,
     \langle\vdash ((\varphi\leftrightarrow\psi)\to
    (\psi\to\varphi))\rangle\rangle$
  \item[] $\langle\varnothing,T,\varnothing,
     \langle\vdash ((\varphi\to\psi)\to(
     (\psi\to\varphi)\to(\varphi
     \leftrightarrow\psi)))\rangle\rangle$
  \item[] $\langle\varnothing,T,\varnothing,
     \langle\vdash (( \varphi \wedge \psi ) \leftrightarrow\neg ( \varphi
     \to \neg \psi )) \rangle\rangle$
  \item[] $\langle\varnothing,T,\varnothing,
     \langle\vdash (( \varphi \vee \psi ) \leftrightarrow (\neg \varphi
     \to \psi )) \rangle\rangle$
  \item[] $\langle\varnothing,T,\varnothing,
     \langle\vdash (\exists x \,\varphi\leftrightarrow
     \lnot \forall x \lnot \varphi)\rangle\rangle$
\end{list}
The first three logical axioms (statements containing ``$\vdash$'') introduce
and effectively define logical equivalence, ``$\leftrightarrow$''.  The last
three use ``$\leftrightarrow$'' to effectively mean ``is defined as.''

\subsection{Example~5---ZFC Set Theory}\index{ZFC set theory}

Here we add to the system of Example~4 the axioms of Zermelo--Fraenkel set
theory with Choice.  For convenience we make use of the
definitions in Example~4.

In the $\mbox{\em CN}$ of Example~4 (which extends Example~3), we replace the symbol $R$
with the symbol $\in$.
More explicitly, we remove from $\Gamma$ of Example~4 the three
axiomatic statements containing $R$ and replace them with the
reducts of the following:
\begin{list}{}{\itemsep 0.0pt}
      \item[] $\langle\varnothing,T,\varnothing,
               \langle \mbox{wff\ }x\in y\rangle\rangle$
      \item[] $\langle\varnothing,T,
               \varnothing,
               \langle \vdash(x=y\to(x\in z\to y\in z))\rangle\rangle$
      \item[] $\langle\varnothing,T,
               \varnothing,
               \langle \vdash(x=y\to(z\in x\to z\in y))\rangle\rangle$
\end{list}
Letting $D=\{\{\alpha,\beta\}\in \mbox{\em DV}\,|\alpha,\beta\in \mbox{\em
Vr}\}$ (in other words all individual variables must be distinct), we extend
$\Gamma$ with the ZFC axioms, called
\index{Axiom of Extensionality}
\index{Axiom of Replacement}
\index{Axiom of Union}
\index{Axiom of Power Sets}
\index{Axiom of Regularity}
\index{Axiom of Infinity}
\index{Axiom of Choice}
Extensionality, Replacement, Union, Power
Set, Regularity, Infinity, and Choice, that are the reducts of:
\begin{list}{}{\itemsep 0.0pt}
      \item[Ext] $\langle D,T,
               \varnothing,
               \langle\vdash (\forall x(x\in y\leftrightarrow x \in z)\to y
               =z) \rangle\rangle$
      \item[Rep] $\langle D,T,
               \varnothing,
               \langle\vdash\exists x ( \exists y \forall z (\varphi \to z = y
                        ) \to
                        \forall z ( z \in x \leftrightarrow \exists x ( x \in
                        y \wedge \forall y\,\varphi ) ) )\rangle\rangle$
      \item[Un] $\langle D,T,
               \varnothing,
               \langle\vdash \exists x \forall y ( \exists x ( y \in x \wedge
               x \in z ) \to y \in x ) \rangle\rangle$
      \item[Pow] $\langle D,T,
               \varnothing,
               \langle\vdash \exists x \forall y ( \forall x ( x \in y \to x
               \in z ) \to y \in x ) \rangle\rangle$
      \item[Reg] $\langle D,T,
               \varnothing,
               \langle\vdash (  x \in y \to
                 \exists x ( x \in y \wedge \forall z ( z \in x \to \lnot z
                \in y ) ) ) \rangle\rangle$
      \item[Inf] $\langle D,T,
               \varnothing,
               \langle\vdash \exists x(y\in x\wedge\forall y(y\in
               x\to
               \exists z(y \in z\wedge z\in x))) \rangle\rangle$
      \item[AC] $\langle D,T,
               \varnothing,
               \langle\vdash \exists x \forall y \forall z ( ( y \in z
               \wedge z \in w ) \to \exists w \forall y ( \exists w
              ( ( y \in z \wedge z \in w ) \wedge ( y \in w \wedge w \in x
              ) ) \leftrightarrow y = w ) ) \rangle\rangle$
\end{list}

\subsection{Example~6---Class Notation in Set Theory}\label{class}

A powerful device that makes set theory easier (and that we have
been using all along in our informal expository language) is {\em class
abstraction notation}.\index{class abstraction}\index{abstraction class}  The
definitions we introduce are rigorously justified
as conservative by Takeuti and Zaring \cite{Takeuti}\index{Takeuti, Gaisi} or
Quine \cite{Quine}\index{Quine, Willard Van Orman}.  The key idea is to
introduce the notation $\{x|\mbox{---}\}$ which means ``the class of all $x$
such that ---'' for abstraction classes and introduce (meta)variables that
range over them.  An abstraction class may or may not be a set, depending on
whether it exists (as a set).  A class that does not exist is
called a {\em proper class}.\index{proper class}\index{class!proper}

To illustrate the use of abstraction classes we will provide some examples
of definitions that make use of them:  the empty set, class union, and
unordered pair.  Many other such definitions can be found in the
Metamath set theory database,
\texttt{set.mm}.\index{set theory database (\texttt{set.mm})}

% We intentionally break up the sequence of math symbols here
% because otherwise the overlong line goes beyond the page in narrow mode.
We extend $\mbox{\em CN}$ of Example~5 with new symbols $\{$
$\mbox{class},$ $\{,$ $|,$ $\},$ $\varnothing,$ $\cup,$ $,$ $\}$
where the inner braces and last comma are
constant symbols. (As before,
our dual use of some mathematical symbols for both our expository
language and as primitives of the formal system should be clear from context.)

We extend $\mbox{\em VR}$ of Example~5 with a set of {\em class
variables}\index{class variable}
$\{A,B,C,\ldots\}$. We extend the $T$ of Example~5 with $\{\langle
\mbox{class\ } A\rangle, \langle \mbox{class\ }B\rangle, \langle \mbox{class\ }
C\rangle,\ldots\}$.

To
introduce our definitions,
we add to $\Gamma$ of Example~5 the axiomatic statements
that are the reducts of the following pre-statements:
\begin{list}{}{\itemsep 0.0pt}
      \item[] $\langle\varnothing,T,\varnothing,
               \langle \mbox{class\ }x\rangle\rangle$
      \item[] $\langle\varnothing,T,\varnothing,
               \langle \mbox{class\ }\{x|\varphi\}\rangle\rangle$
      \item[] $\langle\varnothing,T,\varnothing,
               \langle \mbox{wff\ }A=B\rangle\rangle$
      \item[] $\langle\varnothing,T,\varnothing,
               \langle \mbox{wff\ }A\in B\rangle\rangle$
      \item[Ab] $\langle\varnothing,T,\varnothing,
               \langle \vdash ( y \in \{ x |\varphi\} \leftrightarrow
                  ( ( x = y \to\varphi) \wedge \exists x ( x = y
                  \wedge\varphi) ))
               \rangle\rangle$
      \item[Eq] $\langle\{\{x,A\},\{x,B\}\},T,\varnothing,
               \langle \vdash ( A = B \leftrightarrow
               \forall x ( x \in A \leftrightarrow x \in B ) )
               \rangle\rangle$
      \item[El] $\langle\{\{x,A\},\{x,B\}\},T,\varnothing,
               \langle \vdash ( A \in B \leftrightarrow \exists x
               ( x = A \wedge x \in B ) )
               \rangle\rangle$
\end{list}
Here we say that an individual variable is a class; $\{x|\varphi\}$ is a
class; and we extend the definition of a wff to include class equality and
membership.  Axiom Ab defines membership of a variable in a class abstraction;
the right-hand side can be read as ``the wff that results from proper
substitution of $y$ for $x$ in $\varphi$.''\footnote{Note that this definition
makes unnecessary the introduction of a separate notation similar to
$\varphi(x|y)$ for proper substitution, although we may choose to do so to be
conventional.  Incidentally, $\varphi(x|y)$ as it stands would be ambiguous in
the formal systems of our examples, since we wouldn't know whether
$\lnot\varphi(x|y)$ meant $\lnot(\varphi(x|y))$ or $(\lnot\varphi)(x|y)$.
Instead, we would have to use an unambiguous variant such as $(\varphi\,
x|y)$.}  Axioms Eq and El extend the meaning of the existing equality and
membership connectives.  This is potentially dangerous and requires careful
justification.  For example, from Eq we can derive the Axiom of Extensionality
with predicate logic alone; thus in principle we should include the Axiom of
Extensionality as a logical hypothesis.  However we do not bother to do this
since we have already presupposed that axiom earlier. The distinct variable
restrictions should be read ``where $x$ does not occur in $A$ or $B$.''  We
typically do this when the right-hand side of a definition involves an
individual variable not in the expression being defined; it is done so that
the right-hand side remains independent of the particular ``dummy'' variable
we use.

We continue to add to $\Gamma$ the following definitions
(i.e. the reducts of the following pre-statements) for empty
set,\index{empty set} class union,\index{union} and unordered
pair.\index{unordered pair}  They should be self-explanatory.  Analogous to our
use of ``$\leftrightarrow$'' to define new wffs in Example~4, we use ``$=$''
to define new abstraction terms, and both may be read informally as ``is
defined as'' in this context.
\begin{list}{}{\itemsep 0.0pt}
      \item[] $\langle\varnothing,T,\varnothing,
               \langle \mbox{class\ }\varnothing\rangle\rangle$
      \item[] $\langle\varnothing,T,\varnothing,
               \langle \vdash \varnothing = \{ x | \lnot x = x \}
               \rangle\rangle$
      \item[] $\langle\varnothing,T,\varnothing,
               \langle \mbox{class\ }(A\cup B)\rangle\rangle$
      \item[] $\langle\{\{x,A\},\{x,B\}\},T,\varnothing,
               \langle \vdash ( A \cup B ) = \{ x | ( x \in A \vee x \in B ) \}
               \rangle\rangle$
      \item[] $\langle\varnothing,T,\varnothing,
               \langle \mbox{class\ }\{A,B\}\rangle\rangle$
      \item[] $\langle\{\{x,A\},\{x,B\}\},T,\varnothing,
               \langle \vdash \{ A , B \} = \{ x | ( x = A \vee x = B ) \}
               \rangle\rangle$
\end{list}

\section{Metamath as a Formal System}\label{theorymm}

This section presupposes a familiarity with the Metamath computer language.

Our theory describes formal systems and their universes.  The Metamath
language provides a way of representing these set-theoretical objects to
a computer.  A Metamath database, being a finite set of {\sc ascii}
characters, can usually describe only a subset of a formal system and
its universe, which are typically infinite.  However the database can
contain as large a finite subset of the formal system and its universe
as we wish.  (Of course a Metamath set theory database can, in
principle, indirectly describe an entire infinite formal system by
formalizing the expository language in this Appendix.)

For purpose of our discussion, we assume the Metamath database
is in the simple form described on p.~\pageref{framelist},
consisting of all constant and variable declarations at the beginning,
followed by a sequence of extended frames each
delimited by \texttt{\$\char`\{} and \texttt{\$\char`\}}.  Any Metamath database can
be converted to this form, as described on p.~\pageref{frameconvert}.

The math symbol tokens of a Metamath source file, which are declared
with \texttt{\$c} and \texttt{\$v} statements, are names we assign to
representatives of $\mbox{\em CN}$ and $\mbox{\em VR}$.  For
definiteness we could assume that the first math symbol declared as a
variable corresponds to $v_0$, the second to $v_1$, etc., although the
exact correspondence we choose is not important.

In the Metamath language, each \texttt{\$d}, \texttt{\$f}, and
 \texttt{\$e} source
statement in an extended frame (Section~\ref{frames})
corresponds respectively to a member of the
collections $D$, $T$, and $H$ in a formal system statement $\langle
D_M,T_M,H,A\rangle$.  The math symbol strings following these Metamath keywords
correspond to a variable pair (in the case of \texttt{\$d}) or an expression (for
the other two keywords). The math symbol string following a \texttt{\$a} source
statement corresponds to expression $A$ in an axiomatic statement of the
formal system; the one following a \texttt{\$p} source statement corresponds to
$A$ in a provable statement that is not axiomatic.  In other words, each
extended frame in a Metamath database corresponds to
a pre-statement of the formal system, and a frame corresponds to
a statement of the formal system.  (Don't confuse the two meanings of
``statement'' here.  A statement of the formal system corresponds to the
several statements in a Metamath database that may constitute a
frame.)

In order for the computer to verify that a formal system statement is
provable, each \texttt{\$p} source statement is accompanied by a proof.
However, the proof does not correspond to anything in the formal system
but is simply a way of communicating to the computer the information
needed for its verification.  The proof tells the computer {\em how to
construct} specific members of closure of the formal system
pre-statement corresponding to the extended frame of the \texttt{\$p}
statement.  The final result of the construction is the member of the
closure that matches the \texttt{\$p} statement.  The abstract formal
system, on the other hand, is concerned only with the {\em existence} of
members of the closure.

As mentioned on p.~\pageref{exampleref}, Examples 1 and 3--6 in the
previous Section parallel the development of logic and set theory in the
Metamath database
\texttt{set.mm}.\index{set theory database (\texttt{set.mm})} You may
find it instructive to compare them.


\chapter{The MIU System}
\label{MIU}
\index{formal system}
\index{MIU-system}

The following is a listing of the file \texttt{miu.mm}.  It is self-explanatory.

%%%%%%%%%%%%%%%%%%%%%%%%%%%%%%%%%%%%%%%%%%%%%%%%%%%%%%%%%%%%

\begin{verbatim}
$( The MIU-system:  A simple formal system $)

$( Note:  This formal system is unusual in that it allows
empty wffs.  To work with a proof, you must type
SET EMPTY_SUBSTITUTION ON before using the PROVE command.
By default, this is OFF in order to reduce the number of
ambiguous unification possibilities that have to be selected
during the construction of a proof.  $)

$(
Hofstadter's MIU-system is a simple example of a formal
system that illustrates some concepts of Metamath.  See
Douglas R. Hofstadter, _Goedel, Escher, Bach:  An Eternal
Golden Braid_ (Vintage Books, New York, 1979), pp. 33ff. for
a description of the MIU-system.

The system has 3 constant symbols, M, I, and U.  The sole
axiom of the system is MI. There are 4 rules:
     Rule I:  If you possess a string whose last letter is I,
     you can add on a U at the end.
     Rule II:  Suppose you have Mx.  Then you may add Mxx to
     your collection.
     Rule III:  If III occurs in one of the strings in your
     collection, you may make a new string with U in place
     of III.
     Rule IV:  If UU occurs inside one of your strings, you
     can drop it.
Unfortunately, Rules III and IV do not have unique results:
strings could have more than one occurrence of III or UU.
This requires that we introduce the concept of an "MIU
well-formed formula" or wff, which allows us to construct
unique symbol sequences to which Rules III and IV can be
applied.
$)

$( First, we declare the constant symbols of the language.
Note that we need two symbols to distinguish the assertion
that a sequence is a wff from the assertion that it is a
theorem; we have arbitrarily chosen "wff" and "|-". $)
      $c M I U |- wff $. $( Declare constants $)

$( Next, we declare some variables. $)
     $v x y $.

$( Throughout our theory, we shall assume that these
variables represent wffs. $)
 wx   $f wff x $.
 wy   $f wff y $.

$( Define MIU-wffs.  We allow the empty sequence to be a
wff. $)

$( The empty sequence is a wff. $)
 we   $a wff $.
$( "M" after any wff is a wff. $)
 wM   $a wff x M $.
$( "I" after any wff is a wff. $)
 wI   $a wff x I $.
$( "U" after any wff is a wff. $)
 wU   $a wff x U $.

$( Assert the axiom. $)
 ax   $a |- M I $.

$( Assert the rules. $)
 ${
   Ia   $e |- x I $.
$( Given any theorem ending with "I", it remains a theorem
if "U" is added after it.  (We distinguish the label I_
from the math symbol I to conform to the 24-Jun-2006
Metamath spec.) $)
   I_    $a |- x I U $.
 $}
 ${
IIa  $e |- M x $.
$( Given any theorem starting with "M", it remains a theorem
if the part after the "M" is added again after it. $)
   II   $a |- M x x $.
 $}
 ${
   IIIa $e |- x I I I y $.
$( Given any theorem with "III" in the middle, it remains a
theorem if the "III" is replaced with "U". $)
   III  $a |- x U y $.
 $}
 ${
   IVa  $e |- x U U y $.
$( Given any theorem with "UU" in the middle, it remains a
theorem if the "UU" is deleted. $)
   IV   $a |- x y $.
  $}

$( Now we prove the theorem MUIIU.  You may be interested in
comparing this proof with that of Hofstadter (pp. 35 - 36).
$)
 theorem1  $p |- M U I I U $=
      we wM wU wI we wI wU we wU wI wU we wM we wI wU we wM
      wI wI wI we wI wI we wI ax II II I_ III II IV $.
\end{verbatim}\index{well-formed formula (wff)}

The \texttt{show proof /lemmon/renumber} command
yields the following display.  It is very similar
to the one in \cite[pp.~35--36]{Hofstadter}.\index{Hofstadter, Douglas R.}

\begin{verbatim}
1 ax             $a |- M I
2 1 II           $a |- M I I
3 2 II           $a |- M I I I I
4 3 I_           $a |- M I I I I U
5 4 III          $a |- M U I U
6 5 II           $a |- M U I U U I U
7 6 IV           $a |- M U I I U
\end{verbatim}

We note that Hofstadter's ``MU-puzzle,'' which asks whether
MU is a theorem of the MIU-system, cannot be answered using
the system above because the MU-puzzle is a question {\em
about} the system.  To prove the answer to the MU-puzzle,
a much more elaborate system is needed, namely one that
models the MIU-system within set theory.  (Incidentally, the
answer to the MU-puzzle is no.)

\chapter{Metamath Language EBNF}%
\label{BNF}%
\index{Metamath Language EBNF}

The following is a formal description of the basic Metamath language syntax
(with compressed proofs and support for unknown proof steps).
It is defined using the
Extended Backus--Naur Form (EBNF)\index{Extended Backus--Naur Form}\index{EBNF}
notation from W3C\index{W3C}
\textit{Extensible Markup Language (XML) 1.0 (Fifth Edition)}
(W3C Recommendation 26 November 2008) at
\url{https://www.w3.org/TR/xml/#sec-notation}.

The \texttt{database}
rule is processed until the end of the file (\texttt{EOF}).
The rules eventually require reading whitespace-separated tokens.
A token has an upper-case definition (see below)
or is a string constant in a non-token (such as \texttt{'\$a'}).
We intend for this to be correct, but if there is a conflict the
rules of section \ref{spec} govern. That section also discusses
non-syntax restrictions not shown here
(e.g., that each new label token
defined in a \texttt{hypothesis-stmt} or \texttt{assert-stmt}
must be unique).

\begin{verbatim}
database ::= outermost-scope-stmt*

outermost-scope-stmt ::=
  include-stmt | constant-stmt | stmt

/* File inclusion command; process file as a database.
   Databases should NOT have a comment in the filename. */
include-stmt ::= '$[' filename '$]'

/* Constant symbols declaration. */
constant-stmt ::= '$c' constant+ '$.'

/* A normal statement can occur in any scope. */
stmt ::= block | variable-stmt | disjoint-stmt |
  hypothesis-stmt | assert-stmt

/* A block. You can have 0 statements in a block. */
block ::= '${' stmt* '$}'

/* Variable symbols declaration. */
variable-stmt ::= '$v' variable+ '$.'

/* Disjoint variables. Simple disjoint statements have
   2 variables, i.e., "variable*" is empty for them. */
disjoint-stmt ::= '$d' variable variable variable* '$.'

hypothesis-stmt ::= floating-stmt | essential-stmt

/* Floating (variable-type) hypothesis. */
floating-stmt ::= LABEL '$f' typecode variable '$.'

/* Essential (logical) hypothesis. */
essential-stmt ::= LABEL '$e' typecode MATH-SYMBOL* '$.'

assert-stmt ::= axiom-stmt | provable-stmt

/* Axiomatic assertion. */
axiom-stmt ::= LABEL '$a' typecode MATH-SYMBOL* '$.'

/* Provable assertion. */
provable-stmt ::= LABEL '$p' typecode MATH-SYMBOL*
  '$=' proof '$.'

/* A proof. Proofs may be interspersed by comments.
   If '?' is in a proof it's an "incomplete" proof. */
proof ::= uncompressed-proof | compressed-proof
uncompressed-proof ::= (LABEL | '?')+
compressed-proof ::= '(' LABEL* ')' COMPRESSED-PROOF-BLOCK+

typecode ::= constant

filename ::= MATH-SYMBOL /* No whitespace or '$' */
constant ::= MATH-SYMBOL
variable ::= MATH-SYMBOL
\end{verbatim}

\needspace{2\baselineskip}
A \texttt{frame} is a sequence of 0 or more
\texttt{disjoint-{\allowbreak}stmt} and
\texttt{hypotheses-{\allowbreak}stmt} statements
(possibly interleaved with other non-\texttt{assert-stmt} statements)
followed by one \texttt{assert-stmt}.

\needspace{3\baselineskip}
Here are the rules for lexical processing (tokenization) beyond
the constant tokens shown above.
By convention these tokenization rules have upper-case names.
Every token is read for the longest possible length.
Whitespace-separated tokens are read sequentially;
note that the separating whitespace and \texttt{\$(} ... \texttt{\$)}
comments are skipped.

If a token definition uses another token definition, the whole thing
is considered a single token.
A pattern that is only part of a full token has a name beginning
with an underscore (``\_'').
An implementation could tokenize many tokens as a
\texttt{PRINTABLE-SEQUENCE}
and then check if it meets the more specific rule shown here.

Comments do not nest, and both \texttt{\$(} and \texttt{\$)}
have to be surrounded
by at least one whitespace character (\texttt{\_WHITECHAR}).
Technically comments end without consuming the trailing
\texttt{\_WHITECHAR}, but the trailing
\texttt{\_WHITECHAR} gets ignored anyway so we ignore that detail here.
Metamath language processors
are not required to support \texttt{\$)} followed
immediately by a bare end-of-file, because the closing
comment symbol is supposed to be followed by a
\texttt{\_WHITECHAR} such as a newline.

\begin{verbatim}
PRINTABLE-SEQUENCE ::= _PRINTABLE-CHARACTER+

MATH-SYMBOL ::= (_PRINTABLE-CHARACTER - '$')+

/* ASCII non-whitespace printable characters */
_PRINTABLE-CHARACTER ::= [#x21-#x7e]

LABEL ::= ( _LETTER-OR-DIGIT | '.' | '-' | '_' )+

_LETTER-OR-DIGIT ::= [A-Za-z0-9]

COMPRESSED-PROOF-BLOCK ::= ([A-Z] | '?')+

/* Define whitespace between tokens. The -> SKIP
   means that when whitespace is seen, it is
   skipped and we simply read again. */
WHITESPACE ::= (_WHITECHAR+ | _COMMENT) -> SKIP

/* Comments. $( ... $) and do not nest. */
_COMMENT ::= '$(' (_WHITECHAR+ (PRINTABLE-SEQUENCE - '$)')*
  _WHITECHAR+ '$)' _WHITECHAR

/* Whitespace: (' ' | '\t' | '\r' | '\n' | '\f') */
_WHITECHAR ::= [#x20#x09#x0d#x0a#x0c]
\end{verbatim}
% This EBNF was developed as a collaboration between
% David A. Wheeler\index{Wheeler, David A.},
% Mario Carneiro\index{Carneiro, Mario}, and
% Benoit Jubin\index{Jubin, Benoit}, inspired by a request
% (and a lot of initial work) by Benoit Jubin.
%
% \chapter{Disclaimer and Trademarks}
%
% Information in this document is subject to change without notice and does not
% represent a commitment on the part of Norman Megill.
% \vspace{2ex}
%
% \noindent Norman D. Megill makes no warranties, either express or implied,
% regarding the Metamath computer software package.
%
% \vspace{2ex}
%
% \noindent Any trademarks mentioned in this book are the property of
% their respective owners.  The name ``Metamath'' is a trademark of
% Norman Megill.
%
\cleardoublepage
\phantomsection  % fixes the link anchor
\addcontentsline{toc}{chapter}{\bibname}

\bibliography{metamath}
%\input{metamath.bbl}

\raggedright
\cleardoublepage
\phantomsection % fixes the link anchor
\addcontentsline{toc}{chapter}{\indexname}
%\printindex   ??
\input{metamath.ind}

\end{document}



\end{document}



\raggedright
\cleardoublepage
\phantomsection % fixes the link anchor
\addcontentsline{toc}{chapter}{\indexname}
%\printindex   ??
% metamath.tex - Version of 24-May-2019
% If you change the date above, also change the "Printed date" below.
% SPDX-License-Identifier: CC0-1.0
%
%                              PUBLIC DOMAIN
%
% This file (specifically, the version of this file with the above date)
% has been released into the Public Domain per the
% Creative Commons CC0 1.0 Universal (CC0 1.0) Public Domain Dedication
% https://creativecommons.org/publicdomain/zero/1.0/
%
% The public domain release applies worldwide.  In case this is not
% legally possible, the right is granted to use the work for any purpose,
% without any conditions, unless such conditions are required by law.
%
% Several short, attributed quotations from copyrighted works
% appear in this file under the ``fair use'' provision of Section 107 of
% the United States Copyright Act (Title 17 of the {\em United States
% Code}).  The public-domain status of this file is not applicable to
% those quotations.
%
% Norman Megill - email: nm(at)alum(dot)mit(dot)edu
%
% David A. Wheeler also donates his improvements to this file to the
% public domain per the CC0.  He works at the Institute for Defense Analyses
% (IDA), but IDA has agreed that this Metamath work is outside its "lane"
% and is not a work by IDA.  This was specifically confirmed by
% Margaret E. Myers (Division Director of the Information Technology
% and Systems Division) on 2019-05-24 and by Ben Lindorf (General Counsel)
% on 2019-05-22.

% This file, 'metamath.tex', is self-contained with everything needed to
% generate the the PDF file 'metamath.pdf' (the _Metamath_ book) on
% standard LaTeX 2e installations.  The auxiliary files are embedded with
% "filecontents" commands.  To generate metamath.pdf file, run these
% commands under Linux or Cygwin in the directory that contains
% 'metamath.tex':
%
%   rm -f realref.sty metamath.bib
%   touch metamath.ind
%   pdflatex metamath
%   pdflatex metamath
%   bibtex metamath
%   makeindex metamath
%   pdflatex metamath
%   pdflatex metamath
%
% The warnings that occur in the initial runs of pdflatex can be ignored.
% For the final run,
%
%   egrep -i 'error|warn' metamath.log
%
% should show exactly these 5 warnings:
%
%   LaTeX Warning: File `realref.sty' already exists on the system.
%   LaTeX Warning: File `metamath.bib' already exists on the system.
%   LaTeX Font Warning: Font shape `OMS/cmtt/m/n' undefined
%   LaTeX Font Warning: Font shape `OMS/cmtt/bx/n' undefined
%   LaTeX Font Warning: Some font shapes were not available, defaults
%       substituted.
%
% Search for "Uncomment" below if you want to suppress hyperlink boxes
% in the PDF output file
%
% TYPOGRAPHICAL NOTES:
% * It is customary to use an en dash (--) to "connect" names of different
%   people (and to denote ranges), and use a hyphen (-) for a
%   single compound name. Examples of connected multiple people are
%   Zermelo--Fraenkel, Schr\"{o}der--Bernstein, Tarski--Grothendieck,
%   Hewlett--Packard, and Backus--Naur.  Examples of a single person with
%   a compound name include Levi-Civita, Mittag-Leffler, and Burali-Forti.
% * Use non-breaking spaces after page abbreviations, e.g.,
%   p.~\pageref{note2002}.
%
% --------------------------- Start of realref.sty -----------------------------
\begin{filecontents}{realref.sty}
% Save the following as realref.sty.
% You can then use it with \usepackage{realref}
%
% This has \pageref jumping to the page on which the ref appears,
% \ref jumping to the point of the anchor, and \sectionref
% jumping to the start of section.
%
% Author:  Anthony Williams
%          Software Engineer
%          Nortel Networks Optical Components Ltd
% Date:    9 Nov 2001 (posted to comp.text.tex)
%
% The following declaration was made by Anthony Williams on
% 24 Jul 2006 (private email to Norman Megill):
%
%   ``I hereby donate the code for realref.sty posted on the
%   comp.text.tex newsgroup on 9th November 2001, accessible from
%   http://groups.google.com/group/comp.text.tex/msg/5a0e1cc13ea7fbb2
%   to the public domain.''
%
\ProvidesPackage{realref}
\RequirePackage[plainpages=false,pdfpagelabels=true]{hyperref}
\def\realref@anchorname{}
\AtBeginDocument{%
% ensure every label is a possible hyperlink target
\let\realref@oldrefstepcounter\refstepcounter%
\DeclareRobustCommand{\refstepcounter}[1]{\realref@oldrefstepcounter{#1}
\edef\realref@anchorname{\string #1.\@currentlabel}%
}%
\let\realref@oldlabel\label%
\DeclareRobustCommand{\label}[1]{\realref@oldlabel{#1}\hypertarget{#1}{}%
\@bsphack\protected@write\@auxout{}{%
    \string\expandafter\gdef\protect\csname
    page@num.#1\string\endcsname{\thepage}%
    \string\expandafter\gdef\protect\csname
    ref@num.#1\string\endcsname{\@currentlabel}%
    \string\expandafter\gdef\protect\csname
    sectionref@name.#1\string\endcsname{\realref@anchorname}%
}\@esphack}%
\DeclareRobustCommand\pageref[1]{{\edef\a{\csname
            page@num.#1\endcsname}\expandafter\hyperlink{page.\a}{\a}}}%
\DeclareRobustCommand\ref[1]{{\edef\a{\csname
            ref@num.#1\endcsname}\hyperlink{#1}{\a}}}%
\DeclareRobustCommand\sectionref[1]{{\edef\a{\csname
            ref@num.#1\endcsname}\edef\b{\csname
            sectionref@name.#1\endcsname}\hyperlink{\b}{\a}}}%
}
\end{filecontents}
% ---------------------------- End of realref.sty ------------------------------

% --------------------------- Start of metamath.bib -----------------------------
\begin{filecontents}{metamath.bib}
@book{Albers, editor = "Donald J. Albers and G. L. Alexanderson",
  title = "Mathematical People",
  publisher = "Contemporary Books, Inc.",
  address = "Chicago",
  note = "[QA28.M37]",
  year = 1985 }
@book{Anderson, author = "Alan Ross Anderson and Nuel D. Belnap",
  title = "Entailment",
  publisher = "Princeton University Press",
  address = "Princeton",
  volume = 1,
  note = "[QA9.A634 1975 v.1]",
  year = 1975}
@book{Barrow, author = "John D. Barrow",
  title = "Theories of Everything:  The Quest for Ultimate Explanation",
  publisher = "Oxford University Press",
  address = "Oxford",
  note = "[Q175.B225]",
  year = 1991 }
@book{Behnke,
  editor = "H. Behnke and F. Backmann and K. Fladt and W. S{\"{u}}ss",
  title = "Fundamentals of Mathematics",
  volume = "I",
  publisher = "The MIT Press",
  address = "Cambridge, Massachusetts",
  note = "[QA37.2.B413]",
  year = 1974 }
@book{Bell, author = "J. L. Bell and M. Machover",
  title = "A Course in Mathematical Logic",
  publisher = "North-Holland",
  address = "Amsterdam",
  note = "[QA9.B3953]",
  year = 1977 }
@inproceedings{Blass, author = "Andrea Blass",
  title = "The Interaction Between Category Theory and Set Theory",
  pages = "5--29",
  booktitle = "Mathematical Applications of Category Theory (Proceedings
     of the Special Session on Mathematical Applications
     Category Theory, 89th Annual Meeting of the American Mathematical
     Society, held in Denver, Colorado January 5--9, 1983)",
  editor = "John Walter Gray",
  year = 1983,
  note = "[QA169.A47 1983]",
  publisher = "American Mathematical Society",
  address = "Providence, Rhode Island"}
@proceedings{Bledsoe, editor = "W. W. Bledsoe and D. W. Loveland",
  title = "Automated Theorem Proving:  After 25 Years (Proceedings
     of the Special Session on Automatic Theorem Proving,
     89th Annual Meeting of the American Mathematical
     Society, held in Denver, Colorado January 5--9, 1983)",
  year = 1983,
  note = "[QA76.9.A96.S64 1983]",
  publisher = "American Mathematical Society",
  address = "Providence, Rhode Island" }
@book{Boolos, author = "George S. Boolos and Richard C. Jeffrey",
  title = "Computability and Log\-ic",
  publisher = "Cambridge University Press",
  edition = "third",
  address = "Cambridge",
  note = "[QA9.59.B66 1989]",
  year = 1989 }
@book{Campbell, author = "John Campbell",
  title = "Programmer's Progress",
  publisher = "White Star Software",
  address = "Box 51623, Palo Alto, CA 94303",
  year = 1991 }
@article{DBLP:journals/corr/Carneiro14,
  author    = {Mario Carneiro},
  title     = {Conversion of {HOL} Light proofs into Metamath},
  journal   = {CoRR},
  volume    = {abs/1412.8091},
  year      = {2014},
  url       = {http://arxiv.org/abs/1412.8091},
  archivePrefix = {arXiv},
  eprint    = {1412.8091},
  timestamp = {Mon, 13 Aug 2018 16:47:05 +0200},
  biburl    = {https://dblp.org/rec/bib/journals/corr/Carneiro14},
  bibsource = {dblp computer science bibliography, https://dblp.org}
}
@article{CarneiroND,
  author    = {Mario Carneiro},
  title     = {Natural Deductions in the Metamath Proof Language},
  url       = {http://us.metamath.org/ocat/natded.pdf},
  year      = 2014
}
@inproceedings{Chou, author = "Shang-Ching Chou",
  title = "Proving Elementary Geometry Theorems Using {W}u's Algorithm",
  pages = "243--286",
  booktitle = "Automated Theorem Proving:  After 25 Years (Proceedings
     of the Special Session on Automatic Theorem Proving,
     89th Annual Meeting of the American Mathematical
     Society, held in Denver, Colorado January 5--9, 1983)",
  editor = "W. W. Bledsoe and D. W. Loveland",
  year = 1983,
  note = "[QA76.9.A96.S64 1983]",
  publisher = "American Mathematical Society",
  address = "Providence, Rhode Island" }
@book{Clemente, author = "Daniel Clemente Laboreo",
  title = "Introduction to natural deduction",
  year = 2014,
  url = "http://www.danielclemente.com/logica/dn.en.pdf" }
@incollection{Courant, author = "Richard Courant and Herbert Robbins",
  title = "Topology",
  pages = "573--590",
  booktitle = "The World of Mathematics, Volume One",
  editor = "James R. Newman",
  publisher = "Simon and Schuster",
  address = "New York",
  note = "[QA3.W67 1988]",
  year = 1956 }
@book{Curry, author = "Haskell B. Curry",
  title = "Foundations of Mathematical Logic",
  publisher = "Dover Publications, Inc.",
  address = "New York",
  note = "[QA9.C976 1977]",
  year = 1977 }
@book{Davis, author = "Philip J. Davis and Reuben Hersh",
  title = "The Mathematical Experience",
  publisher = "Birkh{\"{a}}user Boston",
  address = "Boston",
  note = "[QA8.4.D37 1982]",
  year = 1981 }
@incollection{deMillo,
  author = "Richard de Millo and Richard Lipton and Alan Perlis",
  title = "Social Processes and Proofs of Theorems and Programs",
  pages = "267--285",
  booktitle = "New Directions in the Philosophy of Mathematics",
  editor = "Thomas Tymoczko",
  publisher = "Birkh{\"{a}}user Boston, Inc.",
  address = "Boston",
  note = "[QA8.6.N48 1986]",
  year = 1986 }
@book{Edwards, author = "Robert E. Edwards",
  title = "A Formal Background to Mathematics",
  publisher = "Springer-Verlag",
  address = "New York",
  note = "[QA37.2.E38 v.1a]",
  year = 1979 }
@book{Enderton, author = "Herbert B. Enderton",
  title = "Elements of Set Theory",
  publisher = "Academic Press, Inc.",
  address = "San Diego",
  note = "[QA248.E5]",
  year = 1977 }
@book{Goodstein, author = "R. L. Goodstein",
  title = "Development of Mathematical Logic",
  publisher = "Springer-Verlag New York Inc.",
  address = "New York",
  note = "[QA9.G6554]",
  year = 1971 }
@book{Guillen, author = "Michael Guillen",
  title = "Bridges to Infinity",
  publisher = "Jeremy P. Tarcher, Inc.",
  address = "Los Angeles",
  note = "[QA93.G8]",
  year = 1983 }
@book{Hamilton, author = "Alan G. Hamilton",
  title = "Logic for Mathematicians",
  edition = "revised",
  publisher = "Cambridge University Press",
  address = "Cambridge",
  note = "[QA9.H298]",
  year = 1988 }
@unpublished{Harrison, author = "John Robert Harrison",
  title = "Metatheory and Reflection in Theorem Proving:
    A Survey and Critique",
  note = "Technical Report
    CRC-053.
    SRI Cambridge,
    Millers Yard, Cambridge, UK,
    1995.
    Available on the Web as
{\verb+http:+}\-{\verb+//www.cl.cam.ac.uk/users/jrh/papers/reflect.html+}"}
@TECHREPORT{Harrison-thesis,
        author          = "John Robert Harrison",
        title           = "Theorem Proving with the Real Numbers",
        institution   = "University of Cambridge Computer
                         Lab\-o\-ra\-to\-ry",
        address         = "New Museums Site, Pembroke Street, Cambridge,
                           CB2 3QG, UK",
        year            = 1996,
        number          = 408,
        type            = "Technical Report",
        note            = "Author's PhD thesis,
   available on the Web at
{\verb+http:+}\-{\verb+//www.cl.cam.ac.uk+}\-{\verb+/users+}\-{\verb+/jrh+}%
\-{\verb+/papers+}\-{\verb+/thesis.html+}"}
@book{Herrlich, author = "Horst Herrlich and George E. Strecker",
  title = "Category Theory:  An Introduction",
  publisher = "Allyn and Bacon Inc.",
  address = "Boston",
  note = "[QA169.H567]",
  year = 1973 }
@article{Hindley, author = "J. Roger Hindley and David Meredith",
  title = "Principal Type-Schemes and Condensed Detachment",
  journal = "The Journal of Symbolic Logic",
  volume = 55,
  year = 1990,
  note = "[QA.J87]",
  pages = "90--105" }
@book{Hofstadter, author = "Douglas R. Hofstadter",
  title = "G{\"{o}}del, Escher, Bach",
  publisher = "Basic Books, Inc.",
  address = "New York",
  note = "[QA9.H63 1980]",
  year = 1979 }
@article{Indrzejczak, author= "Andrzej Indrzejczak",
  title = "Natural Deduction, Hybrid Systems and Modal Logic",
  journal = "Trends in Logic",
  volume = 30,
  publisher = "Springer",
  year = 2010 }
@article{Kalish, author = "D. Kalish and R. Montague",
  title = "On {T}arski's Formalization of Predicate Logic with Identity",
  journal = "Archiv f{\"{u}}r Mathematische Logik und Grundlagenfor\-schung",
  volume = 7,
  year = 1965,
  note = "[QA.A673]",
  pages = "81--101" }
@article{Kalman, author = "J. A. Kalman",
  title = "Condensed Detachment as a Rule of Inference",
  journal = "Studia Logica",
  volume = 42,
  number = 4,
  year = 1983,
  note = "[B18.P6.S933]",
  pages = "443-451" }
@book{Kline, author = "Morris Kline",
  title = "Mathematical Thought from Ancient to Modern Times",
  publisher = "Oxford University Press",
  address = "New York",
  note = "[QA21.K516 1990 v.3]",
  year = 1972 }
@book{Klinel, author = "Morris Kline",
  title = "Mathematics, The Loss of Certainty",
  publisher = "Oxford University Press",
  address = "New York",
  note = "[QA21.K525]",
  year = 1980 }
@book{Kramer, author = "Edna E. Kramer",
  title = "The Nature and Growth of Modern Mathematics",
  publisher = "Princeton University Press",
  address = "Princeton, New Jersey",
  note = "[QA93.K89 1981]",
  year = 1981 }
@article{Knill, author = "Oliver Knill",
  title = "Some Fundamental Theorems in Mathematics",
  year = "2018",
  url = "https://arxiv.org/abs/1807.08416" }
@book{Landau, author = "Edmund Landau",
  title = "Foundations of Analysis",
  publisher = "Chelsea Publishing Company",
  address = "New York",
  edition = "second",
  note = "[QA241.L2541 1960]",
  year = 1960 }
@article{Leblanc, author = "Hugues Leblanc",
  title = "On {M}eyer and {L}ambert's Quantificational Calculus {FQ}",
  journal = "The Journal of Symbolic Logic",
  volume = 33,
  year = 1968,
  note = "[QA.J87]",
  pages = "275--280" }
@article{Lejewski, author = "Czeslaw Lejewski",
  title = "On Implicational Definitions",
  journal = "Studia Logica",
  volume = 8,
  year = 1958,
  note = "[B18.P6.S933]",
  pages = "189--208" }
@book{Levy, author = "Azriel Levy",
  title = "Basic Set Theory",
  publisher = "Dover Publications",
  address = "Mineola, NY",
  year = "2002"
}
@book{Margaris, author = "Angelo Margaris",
  title = "First Order Mathematical Logic",
  publisher = "Blaisdell Publishing Company",
  address = "Waltham, Massachusetts",
  note = "[QA9.M327]",
  year = 1967}
@book{Manin, author = "Yu I. Manin",
  title = "A Course in Mathematical Logic",
  publisher = "Springer-Verlag",
  address = "New York",
  note = "[QA9.M29613]",
  year = "1977" }
@article{Mathias, author = "Adrian R. D. Mathias",
  title = "A Term of Length 4,523,659,424,929",
  journal = "Synthese",
  volume = 133,
  year = 2002,
  note = "[Q.S993]",
  pages = "75--86" }
@article{Megill, author = "Norman D. Megill",
  title = "A Finitely Axiomatized Formalization of Predicate Calculus
     with Equality",
  journal = "Notre Dame Journal of Formal Logic",
  volume = 36,
  year = 1995,
  note = "[QA.N914]",
  pages = "435--453" }
@unpublished{Megillc, author = "Norman D. Megill",
  title = "A Shorter Equivalent of the Axiom of Choice",
  month = "June",
  note = "Unpublished",
  year = 1991 }
@article{MegillBunder, author = "Norman D. Megill and Martin W.
    Bunder",
  title = "Weaker {D}-Complete Logics",
  journal = "Journal of the IGPL",
  volume = 4,
  year = 1996,
  pages = "215--225",
  note = "Available on the Web at
{\verb+http:+}\-{\verb+//www.mpi-sb.mpg.de+}\-{\verb+/igpl+}%
\-{\verb+/Journal+}\-{\verb+/V4-2+}\-{\verb+/#Megill+}"}
}
@book{Mendelson, author = "Elliott Mendelson",
  title = "Introduction to Mathematical Logic",
  edition = "second",
  publisher = "D. Van Nostrand Company, Inc.",
  address = "New York",
  note = "[QA9.M537 1979]",
  year = 1979 }
@article{Meredith, author = "David Meredith",
  title = "In Memoriam {C}arew {A}rthur {M}eredith (1904-1976)",
  journal = "Notre Dame Journal of Formal Logic",
  volume = 18,
  year = 1977,
  note = "[QA.N914]",
  pages = "513--516" }
@article{CAMeredith, author = "C. A. Meredith",
  title = "Single Axioms for the Systems ({C},{N}), ({C},{O}) and ({A},{N})
      of the Two-Valued Propositional Calculus",
  journal = "The Journal of Computing Systems",
  volume = 3,
  year = 1953,
  pages = "155--164" }
@article{Monk, author = "J. Donald Monk",
  title = "Provability With Finitely Many Variables",
  journal = "The Journal of Symbolic Logic",
  volume = 27,
  year = 1971,
  note = "[QA.J87]",
  pages = "353--358" }
@article{Monks, author = "J. Donald Monk",
  title = "Substitutionless Predicate Logic With Identity",
  journal = "Archiv f{\"{u}}r Mathematische Logik und Grundlagenfor\-schung",
  volume = 7,
  year = 1965,
  pages = "103--121" }
  %% Took out this from above to prevent LaTeX underfull warning:
  % note = "[QA.A673]",
@book{Moore, author = "A. W. Moore",
  title = "The Infinite",
  publisher = "Routledge",
  address = "New York",
  note = "[BD411.M59]",
  year = 1989}
@book{Munkres, author = "James R. Munkres",
  title = "Topology: A First Course",
  publisher = "Prentice-Hall, Inc.",
  address = "Englewood Cliffs, New Jersey",
  note = "[QA611.M82]",
  year = 1975}
@article{Nemesszeghy, author = "E. Z. Nemesszeghy and E. A. Nemesszeghy",
  title = "On Strongly Creative Definitions:  A Reply to {V}. {F}. {R}ickey",
  journal = "Logique et Analyse (N.\ S.)",
  year = 1977,
  volume = 20,
  note = "[BC.L832]",
  pages = "111--115" }
@unpublished{Nemeti, author = "N{\'{e}}meti, I.",
  title = "Algebraizations of Quantifier Logics, an Overview",
  note = "Version 11.4, preprint, Mathematical Institute, Budapest,
    1994.  A shortened version without proofs appeared in
    ``Algebraizations of quantifier logics, an introductory overview,''
   {\em Studia Logica}, 50:485--569, 1991 [B18.P6.S933]"}
@article{Pavicic, author = "M. Pavi{\v{c}}i{\'{c}}",
  title = "A New Axiomatization of Unified Quantum Logic",
  journal = "International Journal of Theoretical Physics",
  year = 1992,
  volume = 31,
  note = "[QC.I626]",
  pages = "1753 --1766" }
@book{Penrose, author = "Roger Penrose",
  title = "The Emperor's New Mind",
  publisher = "Oxford University Press",
  address = "New York",
  note = "[Q335.P415]",
  year = 1989 }
@book{PetersonI, author = "Ivars Peterson",
  title = "The Mathematical Tourist",
  publisher = "W. H. Freeman and Company",
  address = "New York",
  note = "[QA93.P475]",
  year = 1988 }
@article{Peterson, author = "Jeremy George Peterson",
  title = "An automatic theorem prover for substitution and detachment systems",
  journal = "Notre Dame Journal of Formal Logic",
  volume = 19,
  year = 1978,
  note = "[QA.N914]",
  pages = "119--122" }
@book{Quine, author = "Willard Van Orman Quine",
  title = "Set Theory and Its Logic",
  edition = "revised",
  publisher = "The Belknap Press of Harvard University Press",
  address = "Cambridge, Massachusetts",
  note = "[QA248.Q7 1969]",
  year = 1969 }
@article{Robinson, author = "J. A. Robinson",
  title = "A Machine-Oriented Logic Based on the Resolution Principle",
  journal = "Journal of the Association for Computing Machinery",
  year = 1965,
  volume = 12,
  pages = "23--41" }
@article{RobinsonT, author = "T. Thacher Robinson",
  title = "Independence of Two Nice Sets of Axioms for the Propositional
    Calculus",
  journal = "The Journal of Symbolic Logic",
  volume = 33,
  year = 1968,
  note = "[QA.J87]",
  pages = "265--270" }
@book{Rucker, author = "Rudy Rucker",
  title = "Infinity and the Mind:  The Science and Philosophy of the
    Infinite",
  publisher = "Bantam Books, Inc.",
  address = "New York",
  note = "[QA9.R79 1982]",
  year = 1982 }
@book{Russell, author = "Bertrand Russell",
  title = "Mysticism and Logic, and Other Essays",
  publisher = "Barnes \& Noble Books",
  address = "Totowa, New Jersey",
  note = "[B1649.R963.M9 1981]",
  year = 1981 }
@article{Russell2, author = "Bertrand Russell",
  title = "Recent Work on the Principles of Mathematics",
  journal = "International Monthly",
  volume = 4,
  year = 1901,
  pages = "84"}
@article{Schmidt, author = "Eric Schmidt",
  title = "Reductions in Norman Megill's axiom system for complex numbers",
  url = "http://us.metamath.org/downloads/schmidt-cnaxioms.pdf",
  year = "2012" }
@book{Shoenfield, author = "Joseph R. Shoenfield",
  title = "Mathematical Logic",
  publisher = "Addison-Wesley Publishing Company, Inc.",
  address = "Reading, Massachusetts",
  year = 1967,
  note = "[QA9.S52]" }
@book{Smullyan, author = "Raymond M. Smullyan",
  title = "Theory of Formal Systems",
  publisher = "Princeton University Press",
  address = "Princeton, New Jersey",
  year = 1961,
  note = "[QA248.5.S55]" }
@book{Solow, author = "Daniel Solow",
  title = "How to Read and Do Proofs:  An Introduction to Mathematical
    Thought Process",
  publisher = "John Wiley \& Sons",
  address = "New York",
  year = 1982,
  note = "[QA9.S577]" }
@book{Stark, author = "Harold M. Stark",
  title = "An Introduction to Number Theory",
  publisher = "Markham Publishing Company",
  address = "Chicago",
  note = "[QA241.S72 1978]",
  year = 1970 }
@article{Swart, author = "E. R. Swart",
  title = "The Philosophical Implications of the Four-Color Problem",
  journal = "American Mathematical Monthly",
  year = 1980,
  volume = 87,
  month = "November",
  note = "[QA.A5125]",
  pages = "697--707" }
@book{Szpiro, author = "George G. Szpiro",
  title = "Poincar{\'{e}}'s Prize: The Hundred-Year Quest to Solve One
    of Math's Greatest Puzzles",
  publisher = "Penguin Books Ltd",
  address = "London",
  note = "[QA43.S985 2007]",
  year = 2007}
@book{Takeuti, author = "Gaisi Takeuti and Wilson M. Zaring",
  title = "Introduction to Axiomatic Set Theory",
  edition = "second",
  publisher = "Springer-Verlag New York Inc.",
  address = "New York",
  note = "[QA248.T136 1982]",
  year = 1982}
@inproceedings{Tarski, author = "Alfred Tarski",
  title = "What is Elementary Geometry",
  pages = "16--29",
  booktitle = "The Axiomatic Method, with Special Reference to Geometry and
     Physics (Proceedings of an International Symposium held at the University
     of California, Berkeley, December 26, 1957 --- January 4, 1958)",
  editor = "Leon Henkin and Patrick Suppes and Alfred Tarski",
  year = 1959,
  publisher = "North-Holland Publishing Company",
  address = "Amsterdam"}
@article{Tarski1965, author = "Alfred Tarski",
  title = "A Simplified Formalization of Predicate Logic with Identity",
  journal = "Archiv f{\"{u}}r Mathematische Logik und Grundlagenforschung",
  volume = 7,
  year = 1965,
  note = "[QA.A673]",
  pages = "61--79" }
@book{Tymoczko,
  title = "New Directions in the Philosophy of Mathematics",
  editor = "Thomas Tymoczko",
  publisher = "Birkh{\"{a}}user Boston, Inc.",
  address = "Boston",
  note = "[QA8.6.N48 1986]",
  year = 1986 }
@incollection{Wang,
  author = "Hao Wang",
  title = "Theory and Practice in Mathematics",
  pages = "129--152",
  booktitle = "New Directions in the Philosophy of Mathematics",
  editor = "Thomas Tymoczko",
  publisher = "Birkh{\"{a}}user Boston, Inc.",
  address = "Boston",
  note = "[QA8.6.N48 1986]",
  year = 1986 }
@manual{Webster,
  title = "Webster's New Collegiate Dictionary",
  organization = "G. \& C. Merriam Co.",
  address = "Springfield, Massachusetts",
  note = "[PE1628.W4M4 1977]",
  year = 1977 }
@manual{Whitehead, author = "Alfred North Whitehead",
  title = "An Introduction to Mathematics",
  year = 1911 }
@book{PM, author = "Alfred North Whitehead and Bertrand Russell",
  title = "Principia Mathematica",
  edition = "second",
  publisher = "Cambridge University Press",
  address = "Cambridge",
  year = "1927",
  note = "(3 vols.) [QA9.W592 1927]" }
@article{DBLP:journals/corr/Whalen16,
  author    = {Daniel Whalen},
  title     = {Holophrasm: a neural Automated Theorem Prover for higher-order logic},
  journal   = {CoRR},
  volume    = {abs/1608.02644},
  year      = {2016},
  url       = {http://arxiv.org/abs/1608.02644},
  archivePrefix = {arXiv},
  eprint    = {1608.02644},
  timestamp = {Mon, 13 Aug 2018 16:46:19 +0200},
  biburl    = {https://dblp.org/rec/bib/journals/corr/Whalen16},
  bibsource = {dblp computer science bibliography, https://dblp.org} }
@article{Wiedijk-revisited,
  author = {Freek Wiedijk},
  title = {The QED Manifesto Revisited},
  year = {2007},
  url = {http://mizar.org/trybulec65/8.pdf} }
@book{Wolfram,
  author = "Stephen Wolfram",
  title = "Mathematica:  A System for Doing Mathematics by Computer",
  edition = "second",
  publisher = "Addison-Wesley Publishing Co.",
  address = "Redwood City, California",
  note = "[QA76.95.W65 1991]",
  year = 1991 }
@book{Wos, author = "Larry Wos and Ross Overbeek and Ewing Lusk and Jim Boyle",
  title = "Automated Reasoning:  Introduction and Applications",
  edition = "second",
  publisher = "McGraw-Hill, Inc.",
  address = "New York",
  note = "[QA76.9.A96.A93 1992]",
  year = 1992 }

%
%
%[1] Church, Alonzo, Introduction to Mathematical Logic,
% Volume 1, Princeton University Press, Princeton, N. J., 1956.
%
%[2] Cohen, Paul J., Set Theory and the Continuum Hypothesis,
% W. A. Benjamin, Inc., Reading, Mass., 1966.
%
%[3] Hamilton, Alan G., Logic for Mathematicians, Cambridge
% University Press,
% Cambridge, 1988.

%[6] Kleene, Stephen Cole, Introduction to Metamathematics, D.  Van
% Nostrand Company, Inc., Princeton (1952).

%[13] Tarski, Alfred, "A simplified formalization of predicate
% logic with identity," Archiv fur Mathematische Logik und
% Grundlagenforschung, vol. 7 (1965), pp. 61-79.

%[14] Tarski, Alfred and Steven Givant, A Formalization of Set
% Theory Without Variables, American Mathematical Society Colloquium
% Publications, vol. 41, American Mathematical Society,
% Providence, R. I., 1987.

%[15] Zeman, J. J., Modal Logic, Oxford University Press, Oxford, 1973.
\end{filecontents}
% --------------------------- End of metamath.bib -----------------------------


%Book: Metamath
%Author:  Norman Megill Email:  nm at alum.mit.edu
%Author:  David A. Wheeler Email:  dwheeler at dwheeler.com

% A book template example
% http://www.stsci.edu/ftp/software/tex/bookstuff/book.template

\documentclass[leqno]{book} % LaTeX 2e. 10pt. Use [leqno,12pt] for 12pt
% hyperref 2002/05/27 v6.72r  (couldn't get pagebackref to work)
\usepackage[plainpages=false,pdfpagelabels=true]{hyperref}

\usepackage{needspace}     % Enable control over page breaks
\usepackage{breqn}         % automatic equation breaking
\usepackage{microtype}     % microtypography, reduces hyphenation

% Packages for flexible tables.  We need to be able to
% wrap text within a cell (with automatically-determined widths) AND
% split a table automatically across multiple pages.
% * "tabularx" wraps text in cells but only 1 page
% * "longtable" goes across pages but by itself is incompatible with tabularx
% * "ltxtable" combines longtable and tabularx, but table contents
%    must be in a separate file.
% * "ltablex" combines tabularx and longtable - must install specially
% * "booktabs" is recommended as a way to improve the look of tables,
%   but doesn't add these capabilities.
% * "tabu" much more capable and seems to be recommended. So use that.

\usepackage{makecell}      % Enable forced line splits within a table cell
\usepackage{longtable}     % Enable multi-page tables
\usepackage{tabu}          % Multi-page tables with wrapped text in a cell

% You can find more Tex packages using commands like:
% tlmgr search --file tabu.sty
% find /usr/share/texmf-dist/ -name '*tab*'
%
%%%%%%%%%%%%%%%%%%%%%%%%%%%%%%%%%%%%%%%%%%%%%%%%%%%%%%%%%%%%%%%%%%%%%%%%%%%%
% Uncomment the next 3 lines to suppress boxes and colors on the hyperlinks
%%%%%%%%%%%%%%%%%%%%%%%%%%%%%%%%%%%%%%%%%%%%%%%%%%%%%%%%%%%%%%%%%%%%%%%%%%%%
%\hypersetup{
%colorlinks,citecolor=black,filecolor=black,linkcolor=black,urlcolor=black
%}
%
\usepackage{realref}

% Restarting page numbers: try?
%   \printglossary
%   \cleardoublepage
%   \pagenumbering{arabic}
%   \setcounter{page}{1}    ???needed
%   \include{chap1}

% not used:
% \def\R2Lurl#1#2{\mbox{\href{#1}\texttt{#2}}}

\usepackage{amssymb}

% Version 1 of book: margins: t=.4, b=.2, ll=.4, rr=.55
% \usepackage{anysize}
% % \papersize{<height>}{<width>}
% % \marginsize{<left>}{<right>}{<top>}{<bottom>}
% \papersize{9in}{6in}
% % l/r 0.6124-0.6170 works t/b 0.2418-0.3411 = 192pp. 0.2926-03118=exact
% \marginsize{0.7147in}{0.5147in}{0.4012in}{0.2012in}

\usepackage{anysize}
% \papersize{<height>}{<width>}
% \marginsize{<left>}{<right>}{<top>}{<bottom>}
\papersize{9in}{6in}
% l/r 0.85in&0.6431-0.6539 works t/b ?-?
%\marginsize{0.85in}{0.6485in}{0.55in}{0.35in}
\marginsize{0.8in}{0.65in}{0.5in}{0.3in}

% \usepackage[papersize={3.6in,4.8in},hmargin=0.1in,vmargin={0.1in,0.1in}]{geometry}  % page geometry
\usepackage{special-settings}

\raggedbottom
\makeindex

\begin{document}
% Discourage page widows and orphans:
\clubpenalty=300
\widowpenalty=300

%%%%%%% load in AMS fonts %%%%%%% % LaTeX 2.09 - obsolete in LaTeX 2e
%\input{amssym.def}
%\input{amssym.tex}
%\input{c:/texmf/tex/plain/amsfonts/amssym.def}
%\input{c:/texmf/tex/plain/amsfonts/amssym.tex}

\bibliographystyle{plain}
\pagenumbering{roman}
\pagestyle{headings}

\thispagestyle{empty}

\hfill
\vfill

\begin{center}
{\LARGE\bf Metamath} \\
\vspace{1ex}
{\large A Computer Language for Mathematical Proofs} \\
\vspace{7ex}
{\large Norman Megill} \\
\vspace{7ex}
with extensive revisions by \\
\vspace{1ex}
{\large David A. Wheeler} \\
\vspace{7ex}
% Printed date. If changing the date below, also fix the date at the beginning.
2019-05-24
\end{center}

\vfill
\hfill

\newpage
\thispagestyle{empty}

\hfill
\vfill

\begin{center}
$\sim$\ {\sc Public Domain}\ $\sim$

\vspace{2ex}
This book (including its later revisions)
has been released into the Public Domain by Norman Megill per the
Creative Commons CC0 1.0 Universal (CC0 1.0) Public Domain Dedication
(\url{https://creativecommons.org/publicdomain/zero/1.0/}).
David A. Wheeler has done the same.
The public domain release applies worldwide.  In case this is not
legally possible, the right is granted to use the work for any purpose,
without any conditions, unless such conditions are required by law.

\vspace{3ex}
Several short, attributed quotations from copyrighted works
appear in this book under the ``fair use'' provision of Section 107 of
the United States Copyright Act (Title 17 of the {\em United States
Code}).  The public-domain status of this book is not applicable to
those quotations.

\vspace{3ex}
Any trademarks used in this book are the property of their owners.

% QA76.9.L63.M??

% \vspace{1ex}
%
% \vspace{1ex}
% {\small Permission is granted to make and distribute verbatim copies of this
% book
% provided the copyright notice and this
% permission notice are preserved on all copies.}
%
% \vspace{1ex}
% {\small Permission is granted to copy and distribute modified versions of this
% book under the conditions for verbatim copying, provided that the
% entire
% resulting derived work is distributed under the terms of a permission
% notice
% identical to this one.}
%
% \vspace{1ex}
% {\small Permission is granted to copy and distribute translations of this
% book into another language, under the above conditions for modified
% versions,
% except that this permission notice may be stated in a translation
% approved by the
% author.}
%
% \vspace{1ex}
% %{\small   For a copy of the \LaTeX\ source files for this book, contact
% %the author.} \\
% \ \\
% \ \\

\vspace{7ex}
% ISBN: 1-4116-3724-0 \\
ISBN: 978-1-4116-3724-5 \\
{\ } \\
Lulu Press \\
Morrisville, North Carolina\\
USA


\hfill
\vfill

Norman Megill\\ 93 Bridge St., Lexington, MA 02421 \\
E-mail address: \texttt{nm{\char`\@}alum.mit.edu} \\
\vspace{7ex}
David A. Wheeler \\
E-mail address: \texttt{dwheeler{\char`\@}dwheeler.com} \\
% See notes added at end of Preface for revision history. \\
% For current information on the Metamath software see \\
\vspace{7ex}
\url{http://metamath.org}
\end{center}

% \newpage
% \thispagestyle{empty}
%
% \hfill
% \vfill
%
% \begin{center}
% {\it To my son Robin Dwight Megill}
% \end{center}
%
% \vfill
% \hfill
%
% \newpage

\tableofcontents
%\listoftables

\chapter*{Preface}
\markboth{PREFACE}{PREFACE}
\addcontentsline{toc}{section}{Preface}


% (For current information, see the notes added at the
% end of this preface on p.~\pageref{note2002}.)

\subsubsection{Overview}

Metamath\index{Metamath} is a computer language and an associated computer
program for archiving, verifying, and studying mathematical proofs at a very
detailed level.  The Metamath language incorporates no mathematics per se but
treats all mathematical statements as mere sequences of symbols.  You provide
Metamath with certain special sequences (axioms) that tell it what rules
of inference are allowed.  Metamath is not limited to any specific field of
mathematics.  The Metamath language is simple and robust, with an
almost total absence of hard-wired syntax, and
I\footnote{Unless otherwise noted, the words
``I,'' ``me,'' and ``my'' refer to Norman Megill\index{Megill, Norman}, while
``we,'' ``us,'' and ``our'' refer to Norman Megill and
David A. Wheeler\index{Wheeler, David A.}.}
believe that it
provides about the simplest possible framework that allows essentially all of
mathematics to be expressed with absolute rigor.

% index test
%\newcommand{\nn}[1]{#1n}
%\index{aaa@bbb}
%\index{abc!def}
%\index{abd|see{qqq}}
%\index{abe|nn}
%\index{abf|emph}
%\index{abg|(}
%\index{abg|)}

Using the Metamath language, you can build formal or mathematical
systems\index{formal system}\footnote{A formal or mathematical system consists
of a collection of symbols (such as $2$, $4$, $+$ and $=$), syntax rules that
describe how symbols may be combined to form a legal expression (called a
well-formed formula or {\em wff}, pronounced ``whiff''), some starting wffs
called axioms, and inference rules that describe how theorems may be derived
(proved) from the axioms.  A theorem is a mathematical fact such as $2+2=4$.
Strictly speaking, even an obvious fact such as this must be proved from
axioms to be formally acceptable to a mathematician.}\index{theorem}
\index{axiom}\index{rule}\index{well-formed formula (wff)} that involve
inferences from axioms.  Although a database is provided
that includes a recommended set of axioms for standard mathematics, if you
wish you can supply your own symbols, syntax, axioms, rules, and definitions.

The name ``Metamath'' was chosen to suggest that the language provides a
means for {\em describing} mathematics rather than {\em being} the
mathematics itself.  Actually in some sense any mathematical language is
metamathematical.  Symbols written on paper, or stored in a computer,
are not mathematics itself but rather a way of expressing mathematics.
For example ``7'' and ``VII'' are symbols for denoting the number seven
in Arabic and Roman numerals; neither {\em is} the number seven.

If you are able to understand and write computer programs, you should be able
to follow abstract mathematics with the aid of Metamath.  Used in conjunction
with standard textbooks, Metamath can guide you step-by-step towards an
understanding of abstract mathematics from a very rigorous viewpoint, even if
you have no formal abstract mathematics background.  By using a single,
consistent notation to express proofs, once you grasp its basic concepts
Metamath provides you with the ability to immediately follow and dissect
proofs even in totally unfamiliar areas.

Of course, just being able follow a proof will not necessarily give you an
intuitive familiarity with mathematics.  Memorizing the rules of chess does not
give you the ability to appreciate the game of a master, and knowing how the
notes on a musical score map to piano keys does not give you the ability to
hear in your head how it would sound.  But each of these can be a first step.

Metamath allows you to explore proofs in the sense that you can see the
theorem referenced at any step expanded in as much detail as you want, right
down to the underlying axioms of logic and set theory (in the case of the set
theory database provided).  While Metamath will not replace the higher-level
understanding that can only be acquired through exercises and hard work, being
able to see how gaps in a proof are filled in can give you increased
confidence that can speed up the learning process and save you time when you
get stuck.

The Metamath language breaks down a mathematical proof into its tiniest
possible parts.  These can be pieced together, like interlocking
pieces in a puzzle, only in a way that produces correct and absolutely rigorous
mathematics.

The nature of Metamath\index{Metamath} enforces very precise mathematical
thinking, similar to that involved in writing a computer program.  A crucial
difference, though, is that once a proof is verified (by the Metamath program)
to be correct, it is definitely correct; it can never have a hidden
``bug.''\index{computer program bugs}  After getting used to the kind of rigor
and accuracy provided by Metamath, you might even be tempted to
adopt the attitude that a proof should never be considered correct until it
has been verified by a computer, just as you would not completely trust a
manual calculation until you have verified it on a
calculator.

My goal
for Metamath was a system for describing and verifying
mathematics that is completely universal yet conceptually as simple as
possible.  In approaching mathematics from an axiomatic, formal viewpoint, I
wanted Metamath to be able to handle almost any mathematical system, not
necessarily with ease, but at least in principle and hopefully in practice. I
wanted it to verify proofs with absolute rigor, and for this reason Metamath
is what might be thought of as a ``compile-only'' language rather than an
algorithmic or Turing-machine language (Pascal, C, Prolog, Mathematica,
etc.).  In other words, a database written in the Metamath
language doesn't ``do'' anything; it merely exhibits mathematical knowledge
and permits this knowledge to be verified as being correct.  A program in an
algorithmic language can potentially have hidden bugs\index{computer program
bugs} as well as possibly being hard to understand.  But each token in a
Metamath database must be consistent with the database's earlier
contents according to simple, fixed rules.
If a database is verified
to be correct,\footnote{This includes
verification that a sequential list of proof steps results in the specified
theorem.} then the mathematical content is correct if the
verifier is correct and the axioms are correct.
The verification program could be incorrect, but the verification algorithm
is relatively simple (making it unlikely to be implemented incorrectly
by the Metamath program),
and there are over a dozen Metamath database verifiers
written by different people in different programming languages
(so these different verifiers can act as multiple reviewers of a database).
The most-used Metamath database, the Metamath Proof Explorer
(aka \texttt{set.mm}\index{set theory database (\texttt{set.mm})}%
\index{Metamath Proof Explorer}),
is currently verified by four different Metamath verifiers written by
four different people in four different languages, including the
original Metamath program described in this book.
The only ``bugs'' that can exist are in the statement of the axioms,
for example if the axioms are inconsistent (a famous problem shown to be
unsolvable by G\"{o}del's incompleteness theorem\index{G\"{o}del's
incompleteness theorem}).
However, real mathematical systems have very few axioms, and these can
be carefully studied.
All of this provides extraordinarily high confidence that the verified database
is in fact correct.

The Metamath program
doesn't prove theorems automatically but is designed to verify proofs
that you supply to it.
The underlying Metamath language is completely general and has no built-in,
preconceived notions about your formal system\index{formal system}, its logic
or its syntax.
For constructing proofs, the Metamath program has a Proof Assistant\index{Proof
Assistant} which helps you fill in some of a proof step's details, shows you
what choices you have at any step, and verifies the proof as you build it; but
you are still expected to provide the proof.

There are many other programs that can process or generate information
in the Metamath language, and more continue to be written.
This is in part because the Metamath language itself is very simple
and intentionally easy to automatically process.
Some programs, such as \texttt{mmj2}\index{mmj2}, include a proof assistant
that can automate some steps beyond what the Metamath program can do.
Mario Carneiro has developed an algorithm for converting proofs from
the OpenTheory interchange format, which can be translated to and from
any of the HOL family of proof languages (HOL4, HOL Light, ProofPower,
and Isabelle), into the
Metamath language.\cite{DBLP:journals/corr/Carneiro14}\index{Carneiro, Mario}.
Daniel Whalen has developed Holophrasm, which can automatically
prove many Metamath proofs using
machine learning\index{machine learning}\index{artificial intelligence}
approaches
(including multiple neural networks\index{neural networks})\cite{DBLP:journals/corr/Whalen16}\index{Whalen, Daniel}.
However,
a discussion of these other programs is beyond the scope of this book.

Like most computer languages, the Metamath\index{Metamath} language uses the
standard ({\sc ascii}) characters on a computer keyboard, so it cannot
directly represent many of the special symbols that mathematicians use.  A
useful feature of the Metamath program is its ability to convert its notation
into the \LaTeX\ typesetting language.\index{latex@{\LaTeX}}  This feature
lets you convert the {\sc ascii} tokens you've defined into standard
mathematical symbols, so you end up with symbols and formulas you are familiar
with instead of somewhat cryptic {\sc ascii} representations of them.
The Metamath program can also generate HTML\index{HTML}, making it easy
to view results on the web and to see related information by using
hypertext links.

Metamath is probably conceptually different from anything you've seen
before and some aspects may take some getting used to.  This book will
help you decide whether Metamath suits your specific needs.

\subsubsection{Setting Your Expectations}
It is important for you to understand what Metamath\index{Metamath} is and is
not.  As mentioned, the Metamath program
is {\em not} an automated theorem prover but
rather a proof verifier.  Developing a database can be tedious, hard work,
especially if you want to make the proofs as short as possible, but it becomes
easier as you build up a collection of useful theorems.  The purpose of
Metamath is simply to document existing mathematics in an absolutely rigorous,
computer-verifiable way, not to aid directly in the creation of new
mathematics.  It also is not a magic solution for learning abstract
mathematics, although it may be helpful to be able to actually see the implied
rigor behind what you are learning from textbooks, as well as providing hints
to work out proofs that you are stumped on.

As of this writing, a sizable set theory database has been developed to
provide a foundation for many fields of mathematics, but much more work would
be required to develop useful databases for specific fields.

Metamath\index{Metamath} ``knows no math;'' it just provides a framework in
which to express mathematics.  Its language is very small.  You can define two
kinds of symbols, constants\index{constant} and variables\index{variable}.
The only thing Metamath knows how to do is to substitute strings of symbols
for the variables\index{substitution!variable}\index{variable substitution} in
an expression based on instructions you provide it in a proof, subject to
certain constraints you specify for the variables.  Even the decimal
representation of a number is merely a string of certain constants (digits)
which together, in a specific context, correspond to whatever mathematical
object you choose to define for it; unlike other computer languages, there is
no actual number stored inside the computer.  In a proof, you in effect
instruct Metamath what symbol substitutions to make in previous axioms or
theorems and join a sequence of them together to result in the desired
theorem.  This kind of symbol manipulation captures the essence of mathematics
at a preaxiomatic level.

\subsubsection{Metamath and Mathematical Literature}

In advanced mathematical literature, proofs are usually presented in the form
of short outlines that often only an expert can follow.  This is partly out of
a desire for brevity, but it would also be unwise (even if it were practical)
to present proofs in complete formal detail, since the overall picture would
be lost.\index{formal proof}

A solution I envision\label{envision} that would allow mathematics to remain
acceptable to the expert, yet increase its accessibility to non-specialists,
consists of a combination of the traditional short, informal proof in print
accompanied by a complete formal proof stored in a computer database.  In an
analogy with a computer program, the informal proof is like a set of comments
that describe the overall reasoning and content of the proof, whereas the
computer database is like the actual program and provides a means for anyone,
even a non-expert, to follow the proof in as much detail as desired, exploring
it back through layers of theorems (like subroutines that call other
subroutines) all the way back to the axioms of the theory.  In addition, the
computer database would have the advantage of providing absolute assurance
that the proof is correct, since each step can be verified automatically.

There are several other approaches besides Metamath to a project such
as this.  Section~\ref{proofverifiers} discusses some of these.

To us, a noble goal would be a database with hundreds of thousands of
theorems and their computer-verifiable proofs, encompassing a significant
fraction of known mathematics and available for instant access.
These would be fully verified by multiple independently-implemented verifiers,
to provide extremely high confidence that the proofs are completely correct.
The database would allow people to investigate whatever details they were
interested in, so that they could confirm whatever portions they wished.
Whether or not Metamath is an appropriate choice remains to be seen, but in
principle we believe it is sufficient.

\subsubsection{Formalism}

Over the past fifty years, a group of French mathematicians working
collectively under the pseudonym of Bourbaki\index{Bourbaki, Nicolas} have
co-authored a series of monographs that attempt to rigorously and
consistently formalize large bodies of mathematics from foundations.  On the
one hand, certainly such an effort has its merits; on the other hand, the
Bourbaki project has been criticized for its ``scholasticism'' and
``hyperaxiomatics'' that hide the intuitive steps that lead to the results
\cite[p.~191]{Barrow}\index{Barrow, John D.}.

Metamath unabashedly carries this philosophy to its extreme and no doubt is
subject to the same kind of criticism.  Nonetheless I think that in
conjunction with conventional approaches to mathematics Metamath can serve a
useful purpose.  The Bourbaki approach is essentially pedagogic, requiring the
reader to become intimately familiar with each detail in a very large
hierarchy before he or she can proceed to the next step.  The difference with
Metamath is that the ``reader'' (user) knows that all details are contained in
its computer database, available as needed; it does not demand that the user
know everything but conveniently makes available those portions that are of
interest.  As the body of all mathematical knowledge grows larger and larger,
no one individual can have a thorough grasp of its entirety.  Metamath
can finalize and put to rest any questions about the validity of any part of it
and can make any part of it accessible, in principle, to a non-specialist.

\subsubsection{A Personal Note}
Why did I develop Metamath\index{Metamath}?  I enjoy abstract mathematics, but
I sometimes get lost in a barrage of definitions and start to lose confidence
that my proofs are correct.  Or I reach a point where I lose sight of how
anything I'm doing relates to the axioms that a theory is based on and am
sometimes suspicious that there may be some overlooked implicit axiom
accidentally introduced along the way (as happened historically with Euclidean
geometry\index{Euclidean geometry}, whose omission of Pasch's
axiom\index{Pasch's axiom} went unnoticed for 2000 years
\cite[p.~160]{Davis}!). I'm also somewhat lazy and wish to avoid the effort
involved in re-verifying the gaps in informal proofs ``left to the reader;'' I
prefer to figure them out just once and not have to go through the same
frustration a year from now when I've forgotten what I did.  Metamath provides
better recovery of my efforts than scraps of paper that I can't
decipher anymore.  But mostly I find very appealing the idea of rigorously
archiving mathematical knowledge in a computer database, providing precision,
certainty, and elimination of human error.

\subsubsection{Note on Bibliography and Index}

The Bibliography usually includes the Library of Congress classification
for a work to make it easier for you to find it in on a university
library shelf.  The Index has author references to pages where their works
are cited, even though the authors' names may not appear on those pages.

\subsubsection{Acknowledgments}

Acknowledgments are first due to my wife, Deborah (who passed away on
September 4, 1998), for critiquing the manu\-script but most of all for
her patience and support.  I also wish to thank Joe Wright, Richard
Becker, Clarke Evans, Buddha Buck, and Jeremy Henty for helpful
comments.  Any errors, omissions, and other shortcomings are of course
my responsibility.

\subsubsection{Note Added June 22, 2005}\label{note2002}

The original, unpublished version of this book was written in 1997 and
distributed via the web.  The present edition has been updated to
reflect the current Metamath program and databases, as well as more
current {\sc url}s for Internet sites.  Thanks to Josh
Purinton\index{Purinton, Josh}, One Hand
Clapping, Mel L.\ O'Cat, and Roy F. Longton for pointing out
typographical and other errors.  I have also benefitted from numerous
discussions with Raph Levien\index{Levien, Raph}, who has extended
Metamath's philosophy of rigor to result in his {\em
Ghilbert}\index{Ghilbert} proof language (\url{http://ghilbert.org}).

Robert (Bob) Solovay\index{Solovay, Robert} communicated a new result of
A.~R.~D.~Mathias on the system of Bourbaki, and the text has been
updated accordingly (p.~\pageref{bourbaki}).

Bob also pointed out a clarification of the literature regarding
category theory and inaccessible cardinals\index{category
theory}\index{cardinal, inaccessible} (p.~\pageref{categoryth}),
and a misleading statement was removed from the text.  Specifically,
contrary to a statement in previous editions, it is possible to express
``There is a proper class of inaccessible cardinals'' in the language of
ZFC.  This can be done as follows:  ``For every set $x$ there is an
inaccessible cardinal $\kappa$ such that $\kappa$ is not in $x$.''
Bob writes:\footnote{Private communication, Nov.~30, 2002.}
\begin{quotation}
     This axiom is how Grothendieck presents category theory.  To each
inaccessible cardinal $\kappa$ one associates a Grothendieck universe
\index{Grothendieck, Alexander} $U(\kappa)$.  $U(\kappa)$ consists of
those sets which lie in a transitive set of cardinality less than
$\kappa$.  Instead of the ``category of all groups,'' one works relative
to a universe [considering the category of groups of cardinality less
than $\kappa$].  Now the category whose objects are all categories
``relative to the universe $U(\kappa)$'' will be a category not
relative to this universe but to the next universe.

     All of the things category theorists like to do can be done in this
framework.  The only controversial point is whether the Grothen\-dieck
axiom is too strong for the needs of category theorists.  Mac Lane
\index{Mac Lane, Saunders} argues that ``one universe is enough'' and
Feferman\index{Feferman, Solomon} has argued that one can get by with
ordinary ZFC.  I don't find Feferman's arguments persuasive.  Mac Lane
may be right, but when I think about category theory I do it \`{a} la
Grothendieck.

        By the way Mizar\index{Mizar} adds the axiom ``there is a proper
class of inaccessibles'' precisely so as to do category theory.
\end{quotation}

The most current information on the Metamath program and databases can
always be found at \url{http://metamath.org}.


\subsubsection{Note Added June 24, 2006}\label{note2006}

The Metamath spec was restricted slightly to make parsers easier to
write.  See the footnote on p.~\pageref{namespace}.

%\subsubsection{Note Added July 24, 2006}\label{note2006b}
\subsubsection{Note Added March 10, 2007}\label{note2006b}

I am grateful to Anthony Williams\index{Williams, Anthony} for writing
the \LaTeX\ package called {\tt realref.sty} and contributing it to the
public domain.  This package allows the internal hyperlinks in a {\sc
pdf} file to anchor to specific page numbers instead of just section
titles, making the navigation of the {\sc pdf} file for this book much
more pleasant and ``logical.''

A typographical error found by Martin Kiselkov was corrected.
A confusing remark about unification was deleted per suggestion of
Mel O'Cat.

\subsubsection{Note Added May 27, 2009}\label{note2009}

Several typos found by Kim Sparre were corrected.  A note was added that
the Poincar\'{e} conjecture has been proved (p.~\pageref{poincare}).

\subsubsection{Note Added Nov. 17, 2014}\label{note2014}

The statement of the Schr\"{o}der--Bernstein theorem was corrected in
Section~\ref{trust}.  Thanks to Bob Solovay for pointing out the error.

\subsubsection{Note Added May 25, 2016}\label{note2016}

Thanks to Jerry James for correcting 16 typos.

\subsubsection{Note Added February 25, 2019}\label{note201902}

David A. Wheeler\index{Wheeler, David A.}
made a large number of improvements and updates,
in coordination with Norman Megill.
The predicate calculus axioms were renumbered, and the text makes
it clear that they are based on Tarski's system S2;
the one slight deviation in axiom ax-6 is explained and justified.
The real and complex number axioms were modified to be consistent with
\texttt{set.mm}\index{set theory database (\texttt{set.mm})}%
\index{Metamath Proof Explorer}.
Long-awaited specification changes ``1--8'' were made,
which clarified previously ambiguous points.
Some errors in the text involving \texttt{\$f} and
\texttt{\$d} statements were corrected (the spec was correct, but
the in-book explanations unintentionally contradicted the spec).
We now have a system for automatically generating narrow PDFs,
so that those with smartphones can have easy access to the current
version of this document.
A new section on deduction was added;
it discusses the standard deduction theorem,
the weak deduction theorem,
deduction style, and natural deduction.
Many minor corrections (too numerous to list here) were also made.

\subsubsection{Note Added March 7, 2019}\label{note201903}

This added a description of the Matamath language syntax in
Extended Backus--Naur Form (EBNF)\index{Extended Backus--Naur Form}\index{EBNF}
in Appendix \ref{BNF}, added a brief explanation about typecodes,
inserted more examples in the deduction section,
and added a variety of smaller improvements.

\subsubsection{Note Added April 7, 2019}\label{note201904}

This version clarified the proper substitution notation, improved the
discussion on the weak deduction theorem and natural deduction,
documented the \texttt{undo} command, updated the information on
\texttt{write source}, changed the typecode
from \texttt{set} to \texttt{setvar} to be consistent with the current
version of \texttt{set.mm}, added more documentation about comment markup
(e.g., documented how to create headings), and clarified the
differences between various assertion forms (in particular deduction form).

\subsubsection{Note Added May 20, 2019}\label{note201905}

This version fixes various small issues and added specific examples
of forms (deduction form, inference form, and closed form).

\chapter{Introduction}
\pagenumbering{arabic}

\begin{quotation}
  {\em {\em I.M.:}  No, no.  There's nothing subjective about it!  Everybody
knows what a proof is.  Just read some books, take courses from a competent
mathematician, and you'll catch on.

{\em Student:}  Are you sure?

{\em I.M.:}  Well---it is possible that you won't, if you don't have any
aptitude for it.  That can happen, too.

{\em Student:}  Then {\em you} decide what a proof is, and if I don't learn
to decide in the same way, you decide I don't have any aptitude.

{\em I.M.:}  If not me, then who?}
    \flushright\sc  ``The Ideal Mathematician''
    \index{Davis, Phillip J.}
    \footnote{\cite{Davis}, p.~40.}\\
\end{quotation}

In the past century, brilliant mathematicians have discovered almost
unimaginably profound results that rank among the crowning intellectual
achievements of mankind.  However, there is a sense in which modern abstract
mathematics is behind the times, stuck in an era before computers existed.
While no one disputes the remarkable results that have been achieved,
communicating these results in a precise way to the uninitiated is virtually
impossible.  To describe these results, a terse informal language is used which
despite its elegance is very difficult to learn.  This informal language is not
imprecise, far from it, but rather it often has omitted detail
and symbols with hidden context that are
implicitly understood by an expert but few others.  Extremely complex technical
meanings are associated with innocent-sounding English words such as
``compact'' and ``measurable'' that barely hint at what is actually being
said.  Anyone who does not keep the precise technical meaning constantly in
mind is bound to fail, and acquiring the ability to do this can be achieved
only through much practice and hard work.  Only the few who complete the
painful learning experience can join the small in-group of pure
mathematicians.  The informal language effectively cuts off the true nature of
their knowledge from most everyone else.

Metamath\index{Metamath} makes abstract mathematics more concrete.  It allows
a computer to keep track of the complexity associated with each word or symbol
with absolute rigor.  You can explore this complexity at your leisure, to
whatever degree you desire.  Whether or not you believe that concepts such as
infinity actually ``exist'' outside of the mind, Metamath lets you get to the
foundation for what's really being said.

Metamath also enables completely rigorous and thorough proof verification.
Its language is simple enough so that you
don't have to rely on the authority of experts but can verify the results
yourself, step by step.  If you want to attempt to derive your own results,
Metamath will not let you make a mistake in reasoning.
Even professional mathematicians make mistakes; Metamath makes it possible
to thoroughly verify that proofs are correct.

Metamath\index{Metamath} is a computer language and an associated computer
program for archiving, verifying, and studying mathematical proofs at a very
detailed level.
The Metamath language
describes formal\index{formal system} mathematical
systems and expresses proofs of theorems in those systems.  Such a language
is called a metalanguage\index{metalanguage} by mathematicians.
The Metamath program is a computer program that verifies
proofs expressed in the Metamath language.
The Metamath program does not have the built-in
ability to make logical inferences; it just makes a series of symbol
substitutions according to instructions given to it in a proof
and verifies that the result matches the expected theorem.  It makes logical
inferences based only on rules of logic that are contained in a set of
axioms\index{axiom}, or first principles, that you provide to it as the
starting point for proofs.

The complete specification of the Metamath language is only four pages long
(Section~\ref{spec}, p.~\pageref{spec}).  Its simplicity may at first make you
wonder how it can do much of anything at all.  But in fact the kinds of
symbol manipulations it performs are the ones that are implicitly done in all
mathematical systems at the lowest level.  You can learn it relatively quickly
and have complete confidence in any mathematical proof that it verifies.  On
the other hand, it is powerful and general enough so that virtually any
mathematical theory, from the most basic to the deeply abstract, can be
described with it.

Although in principle Metamath can be used with any
kind of mathematics, it is best suited for abstract or ``pure'' mathematics
that is mostly concerned with theorems and their proofs, as opposed to the
kind of mathematics that deals with the practical manipulation of numbers.
Examples of branches of pure mathematics are logic\index{logic},\footnote{Logic
is the study of statements that are universally true regardless of the objects
being described by the statements.  An example is the statement, ``if $P$
implies $Q$, then either $P$ is false or $Q$ is true.''} set theory\index{set
theory},\footnote{Set theory is the study of general-purpose mathematical objects called
``sets,'' and from it essentially all of mathematics can be derived.  For
example, numbers can be defined as specific sets, and their properties
can be explored using the tools of set theory.} number theory\index{number
theory},\footnote{Number theory deals with the properties of positive and
negative integers (whole numbers).} group theory\index{group
theory},\footnote{Group theory studies the properties of mathematical objects
called groups that obey a simple set of axioms and have properties of symmetry
that make them useful in many other fields.} abstract algebra\index{abstract
algebra},\footnote{Abstract algebra includes group theory and also studies
groups with additional properties that qualify them as ``rings'' and
``fields.''  The set of real numbers is a familiar example of a field.},
analysis\index{analysis} \index{real and complex numbers}\footnote{Analysis is
the study of real and complex numbers.} and
topology\index{topology}.\footnote{One area studied by topology are properties
that remain unchanged when geometrical objects undergo stretching
deformations; for example a doughnut and a coffee cup each have one hole (the
cup's hole is in its handle) and are thus considered topologically
equivalent.  In general, though, topology is the study of abstract
mathematical objects that obey a certain (surprisingly simple) set of axioms.
See, for example, Munkres \cite{Munkres}\index{Munkres, James R.}.} Even in
physics, Metamath could be applied to certain branches that make use of
abstract mathematics, such as quantum logic\index{quantum logic} (used to study
aspects of quantum mechanics\index{quantum mechanics}).

On the other hand, Metamath\index{Metamath} is less suited to applications
that deal primarily with intensive numeric computations.  Metamath does not
have any built-in representation of numbers\index{Metamath!representation of
numbers}; instead, a specific string of symbols (digits) must be syntactically
constructed as part of any proof in which an ordinary number is used.  For
this reason, numbers in Metamath are best limited to specific constants that
arise during the course of a theorem or its proof.  Numbers are only a tiny
part of the world of abstract mathematics.  The exclusion of built-in numbers
was a conscious decision to help achieve Metamath's simplicity, and there are
other software tools if you have different mathematical needs.
If you wish to quickly solve algebraic problems, the computer algebra
programs\index{computer algebra system} {\sc
macsyma}\index{macsyma@{\sc macsyma}}, Mathematica\index{Mathematica}, and
Maple\index{Maple} are specifically suited to handling numbers and
algebra efficiently.
If you wish to simply calculate numeric or matrix expressions easily,
tools such as Octave\index{Octave} may be a better choice.

After learning Metamath's basic statement types, any
tech\-ni\-cal\-ly ori\-ent\-ed person, mathematician or not, can
immediately trace
any theorem proved in the language as far back as he or she wants, all the way
to the axioms on which the theorem is based.  This ability suggests a
non-traditional way of learning about pure mathematics.  Used in conjunction
with traditional methods, Metamath could make pure mathematics accessible to
people who are not sufficiently skilled to figure out the implicit detail in
ordinary textbook proofs.  Once you learn the axioms of a theory, you can have
complete confidence that everything you need to understand a proof you are
studying is all there, at your beck and call, allowing you to focus in on any
proof step you don't understand in as much depth as you need, without worrying
about getting stuck on a step you can't figure out.\footnote{On the other
hand, writing proofs in the Metamath language is challenging, requiring
a degree of rigor far in excess of that normally taught to students.  In a
classroom setting, I doubt that writing Metamath proofs would ever replace
traditional homework exercises involving informal proofs, because the time
needed to work out the details would not allow a course to
cover much material.  For students who have trouble grasping the implied rigor
in traditional material, writing a few simple proofs in the Metamath language
might help clarify fuzzy thought processes.  Although somewhat difficult at
first, it eventually becomes fun to do, like solving a puzzle, because of the
instant feedback provided by the computer.}

Metamath\index{Metamath} is probably unlike anything you have
encountered before.  In this first chapter we will look at the philosophy and
use of computers in mathematics in order to better understand the motivation
behind Metamath.  The material in this chapter is not required in order to use
Metamath.  You may skip it if you are impatient, but I hope you will find it
educational and enjoyable.  If you want to start experimenting with the
Metamath program right away, proceed directly to Chapter~\ref{using}
(p.~\pageref{using}).  To
learn the Metamath language, skim Chapter~\ref{using} then proceed to
Chapter~\ref{languagespec} (p.~\pageref{languagespec}).

\section{Mathematics as a Computer Language}

\begin{quote}
  {\em The study of mathematics is apt to commence in
dis\-ap\-point\-ment.\ldots \\
We are told that by its aid the stars are weighted
and the billions of molecules in a drop of water are counted.  Yet, like the
ghost of Hamlet's father, this great science eludes the efforts of our mental
weapons to grasp it.}
  \flushright\sc  Alfred North Whitehead\footnote{\cite{Whitehead}, ch.\ 1.}\\
\end{quote}\index{Whitehead, Alfred North}

\subsection{Is Mathematics ``User-Friendly''?}

Suppose you have no formal training in abstract mathematics.  But popular
books you've read offer tempting glimpses of this world filled with profound
ideas that have stirred the human spirit.  You are not satisfied with the
informal, watered-down descriptions you've read but feel it is important to
grasp the underlying mathematics itself to understand its true meaning. It's
not practical to go back to school to learn it, though; you don't want to
dedicate years of your life to it.  There are many important things in life,
and you have to set priorities for what's important to you.  What would happen
if you tried to pursue it on your own, in your spare time?

After all, you were able to learn a computer programming language such as
Pascal on your own without too much difficulty, even though you had no formal
training in computers.  You don't claim to be an expert in software design,
but you can write a passable program when necessary to suit your needs.  Even
more important, you know that you can look at anyone else's Pascal program, no
matter how complex, and with enough patience figure out exactly how it works,
even though you are not a specialist.  Pascal allows you do anything that a
computer can do, at least in principle.  Thus you know you have the ability,
in principle, to follow anything that a computer program can do:  you just
have to break it down into small enough pieces.

Here's an imaginary scenario of what might happen if you na\-ive\-ly a\-dopted
this same view of abstract mathematics and tried to pick it up on your own, in
a period of time comparable to, say, learning a computer programming
language.

\subsubsection{A Non-Mathematician's Quest for Truth}

\begin{quote}
  {\em \ldots my daughters have been studying (chemistry) for several
se\-mes\-ters, think they have learned differential and integral calculus in
school, and yet even today don't know why $x\cdot y=y\cdot x$ is true.}
  \flushright\sc  Edmund Landau\footnote{\cite{Landau}, p.~vi.}\\
\end{quote}\index{Landau, Edmund}

\begin{quote}
  {\em Minus times minus is plus,\\
The reason for this we need not discuss.}
  \flushright\sc W.\ H.\ Auden\footnote{As quoted in \cite{Guillen}, p.~64.}\\
\end{quote}\index{Auden, W.\ H.}\index{Guillen, Michael}

We'll suppose you are a technically oriented professional, perhaps an engineer, a
computer programmer, or a physicist, but probably not a mathematician.  You
consider yourself reasonably intelligent.  You did well in school, learning a
variety of methods and techniques in practical mathematics such as calculus and
differential equations.  But rarely did your courses get into anything
resembling modern abstract mathematics, and proofs were something that appeared
only occasionally in your textbooks, a kind of necessary evil that was
supposed to convince you of a certain key result.  Most of your
homework consisted of exercises that gave you practice in the techniques, and
you were hardly ever asked to come up with a proof of your own.

You find yourself curious about advanced, abstract mathematics.  You are
driven by an inner conviction that it is important to understand and
appreciate some of the most profound knowledge discovered by mankind.  But it
seems very hard to learn, something that only certain gifted longhairs can
access and understand.  You are frustrated that it seems forever cut off from
you.

Eventually your curiosity drives you to do something about it.
You set for yourself a goal of ``really'' understanding mathematics:  not just
how to manipulate equations in algebra or calculus according to cookbook
rules, but rather to gain a deep understanding of where those rules come from.
In fact, you're not thinking about this kind of ordinary mathematics at all,
but about a much more abstract, ethereal realm of pure mathematics, where
famous results such as G\"{o}del's incompleteness theorem\index{G\"{o}del's
incompleteness theorem} and Cantor's different kinds of infinities
reside.

You have probably read a number of popular books, with titles like {\em
Infinity and the Mind} \cite{Rucker}\index{Rucker, Rudy}, on topics such as
these.  You found them inspiring but at the same time somewhat
unsatisfactory.  They gave you a general idea of what these results are about,
but if someone asked you to prove them, you wouldn't have the faintest idea of
where to begin.   Sure, you could give the same overall outline that you
learned from the popular books; and in a general sort of way, you do have an
understanding.  But deep down inside, you know that there is a rigor that is
missing, that probably there are many subtle steps and pitfalls along the way,
and ultimately it seems you have to place your trust in the experts in the
field.  You don't like this; you want to be able to verify these results for
yourself.

So where do you go next?  As a first step, you decide to look up some of the
original papers on the theorems you are curious about, or better, obtain some
standard textbooks in the field.  You look up a theorem you want to
understand.  Sure enough, it's there, but it's expressed with strange
terms and odd symbols that mean absolutely nothing to you.  It might as well be written in
a foreign language you've never seen before, whose symbols are totally alien.
You look at the proof, and you haven't the foggiest notion what each step
means, much less how one step follows from another.  Well, obviously you have
a lot to learn if you want to understand this stuff.

You feel that you could probably understand it by
going back to college for another three to six years and getting a math
degree.  But that does not fit in with your career and the other things in
your life and would serve no practical purpose.  You decide to seek a quicker
path.  You figure you'll just trace your way back to the beginning, step by
step, as you would do with a computer program, until you understand it.  But
you quickly find that this is not possible, since you can't even understand
enough to know what you have to trace back to.

Maybe a different approach is in order---maybe you should start at the
beginning and work your way up.  First, you read the introduction to the book
to find out what the prerequisites are.  In a similar fashion, you trace your
way back through two or three more books, finally arriving at one that seems
to start at a beginning:  it lists the axioms of arithmetic.  ``Aha!'' you
naively think, ``This must be the starting point, the source of all mathematical
knowledge.'' Or at least the starting point for mathematics dealing with
numbers; you have to start somewhere and have no idea what the starting point
for other mathematics would be.  But the word ``axioms'' looks promising.  So
you eagerly read along and work through some elementary exercises at the
beginning of the book.  You feel vaguely bothered:  these
don't seem like axioms at all, at least not in the sense that you want to
think of axioms.  Axioms imply a starting point from which everything else can
be built up, according to precise rules specified in the axiom system.  Even
though you can understand the first few proofs in an informal way,
and are able to do some of the
exercises, it's hard to pin down precisely what the
rules are.   Sure, each step seems to follow logically from the others, but
exactly what does that mean?  Is the ``logic'' just a matter of common sense,
something vague that we all understand but can never quite state precisely?

You've spent a number of years, off and on, programming computers, and you
know that in the case of computer languages there is no question of what the
rules are---they are precise and crystal clear.  If you follow them, your
program will work, and if you don't, it won't.  No matter how complex a
program, it can always be broken down into simpler and simpler pieces, until
you can ultimately identify the bits that are moved around to perform a
specific function.  Some programs might require a lot of perseverance to
accomplish this, but if you focus on a specific portion of it, you don't even
necessarily have to know how the rest of it works. Shouldn't there be an
analogy in mathematics?

You decide to apply the ultimate test:  you ask yourself how a computer could
verify or ensure that the steps in these proofs follow from one another.
Certainly mathematics must be at least as precisely defined as a computer
language, if not more so; after all, computer science itself is based on it.
If you can get a computer to verify these proofs, then you should also be
able, in principle, to understand them yourself in a very crystal clear,
precise way.

You're in for a surprise:  you can conceive of no way to convert the
proofs, which are in English, to a form that the computer can understand.
The proofs are filled with phrases such as ``assume there exists a unique
$x$\ldots'' and ``given any $y$, let $z$ be the number such that\ldots''  This
isn't the kind of logic you are used to in computer programming, where
everything, even arithmetic, reduces to Boolean ones and zeroes if you care to
break it down sufficiently.  Even though you think you understand the proofs,
there seems to be some kind of higher reasoning involved rather than precise
rules that define how you manipulate the symbols in the axioms.  Whatever it
is, it just isn't obvious how you would express it to a computer, and the more
you think about it, the more puzzled and confused you get, to the point where
you even wonder whether {\em you} really understand it.  There's a lot more to
these axioms of arithmetic than meets the eye.

Nobody ever talked about this in school in your applied math and engineering
courses.  You just learned the rules they gave you, not quite understanding
how or why they worked, sometimes vaguely suspicious or uncertain of them, and
through homework problems and osmosis learned how to present solutions that
satisfied the instructor and earned you an ``A.''  Rarely did you actually
``prove'' anything in a rigorous way, and the math majors who did do stuff
like that seemed to be in a different world.

Of course, there are computer algebra programs that can do mathematics, and
rather impressively.  They can instantly solve the integrals that you
struggled with in freshman calculus, and do much, much more.  But when you
look at these programs, what you see is a big collection of algorithms and
techniques that evolved and were added to over time, along with some basic
software that manipulates symbols.  Each algorithm that is built in is the
result of someone's theorem whose proof is omitted; you just have to trust the
person who proved it and the person who programmed it in and hope there are no
bugs.\index{computer program bugs}  Somehow this doesn't seem to be the
essence of mathematics.  Although computer algebra systems can generate
theorems with amazing speed, they can't actually prove a single one of them.

After some puzzlement, you revisit some popular books on what mathematics is
all about.  Somewhere you read that all of mathematics is actually derived
from something called ``set theory.''  This is a little confusing, because
nowhere in the book that presented the axioms of arithmetic was there any
mention of set theory, or if there was, it seemed to be just a tool that helps
you describe things better---the set of even numbers, that sort of thing.  If
set theory is the basis for all mathematics, then why are additional axioms
needed for arithmetic?

Something is wrong but you're not sure what.  One of your friends is a pure
mathematician.  He knows he is unable to communicate to you what he does for a
living and seems to have little interest in trying.  You do know that for him,
proofs are what mathematics is all about. You ask him what a proof is, and he
essentially tells you that, while of course it's based on logic, really it's
something you learn by doing it over and over until you pick it up.  He refers
you to a book, {\em How to Read and Do Proofs} \cite{Solow}.\index{Solow,
Daniel}  Although this book helps you understand traditional informal proofs,
there is still something missing you can't seem to pin down yet.

You ask your friend how you would go about having a computer verify a proof.
At first he seems puzzled by the question; why would you want to do that?
Then he says it's not something that would make any sense to do, but he's
heard that you'd have to break the proof down into thousands or even millions
of individual steps to do such a thing, because the reasoning involved is at
such a high level of abstraction.  He says that maybe it's something you could
do up to a point, but the computer would be completely impractical once you
get into any meaningful mathematics.  There, the only way you can verify a
proof is by hand, and you can only acquire the ability to do this by
specializing in the field for a couple of years in grad school.  Anyway, he
thinks it all has to do with set theory, although he has never taken a formal
course in set theory but just learned what he needed as he went along.

You are intrigued and amazed.  Apparently a mathematician can grasp as a
single concept something that would take a computer a thousand or a million
steps to verify, and have complete confidence in it.  Each one of these
thousand or million steps must be absolutely correct, or else the whole proof
is meaningless.  If you added a million numbers by hand, would you trust the
result?  How do you really know that all these steps are correct, that there
isn't some subtle pitfall in one of these million steps, like a bug in a
computer program?\index{computer program bugs}  After all, you've read that
famous mathematicians have occasionally made mistakes, and you certainly know
you've made your share on your math homework problems in school.

You recall the analogy with a computer program.  Sure, you can understand what
a large computer program such as a word processor does, as a single high-level
concept or a small set of such concepts, but your ability to understand it in
no way ensures that the program is correct and doesn't have hidden bugs.  Even
if you wrote the program yourself you can't really know this; most large
programs that you've written have had bugs that crop up at some later date, no
matter how careful you tried to be while writing them.

OK, so now it seems the reason you can't figure out how to make a
computer verify proofs is because each step really corresponds to a
million small steps.  Well, you say, a computer can do a million
calculations in a second, so maybe it's still practical to do.  Now the
puzzle becomes how to figure out what the million steps are that each
English-language step corresponds to.  Your mathematician friend hasn't
a clue, but suggests that maybe you would find the answer by studying
set theory.  Actually, your friend thinks you're a little off the wall
for even wondering such a thing.  For him, this is not what mathematics
is all about.

The subject of set theory keeps popping up, so you decide it's
time to look it up.

You decide to start off on a careful footing, so you start reading a couple of
very elementary books on set theory.  A lot of it seems pretty obvious, like
intersections, subsets, and Venn diagrams.  You thumb through one of the
books; nowhere is anything about axioms mentioned. The other book relegates to
an appendix a brief discussion that mentions a set of axioms called
``Zermelo--Fraenkel set theory''\index{Zermelo--Fraenkel set theory} and states
them in English.  You look at them and have no idea what they really mean or
what you can do with them.  The comments in this appendix say that the purpose
of mentioning them is to expose you to the idea, but imply that they are not
necessary for basic understanding and that they are really the subject matter
of advanced treatments where fine points such as a certain paradox (Russell's
paradox\index{Russell's paradox}\footnote{Russell's paradox assumes that there
exists a set $S$ that is a collection of all sets that don't contain
themselves.  Now, either $S$ contains itself or it doesn't.  If it contains
itself, it contradicts its definition.  But if it doesn't contain itself, it
also contradicts its definition.  Russell's paradox is resolved in ZF set
theory by denying that such a set $S$ exists.}) are resolved.  Wait a
minute---shouldn't the axioms be a starting point, not an ending point?  If
there are paradoxes that arise without the axioms, how do you know you won't
stumble across one accidentally when using the informal approach?

And nowhere do these books describe how ``all of mathematics can be
derived from set theory'' which by now you've heard a few times.

You find a more advanced book on set theory.  This one actually lists the
axioms of ZF set theory in plain English on page one.  {\em Now} you think
your quest has ended and you've finally found the source of all mathematical
knowledge; you just have to understand what it means.  Here, in one place, is
the basis for all of mathematics!  You stare at the axioms in awe, puzzle over
them, memorize them, hoping that if you just meditate on them long enough they
will become clear.  Of course, you haven't the slightest idea how the rest of
mathematics is ``derived'' from them; in particular, if these are the axioms
of mathematics, then why do arithmetic, group theory, and so on need their own
axioms?

You start reading this advanced book carefully, pondering the meaning of every
word, because by now you're really determined to get to the bottom of this.
The first thing the book does is explain how the axioms came about, which was
to resolve Russell's paradox.\index{Russell's paradox}  In fact that seems to
be the main purpose of their existence; that they supposedly can be used to
derive all of mathematics seems irrelevant and is not even mentioned.  Well,
you go on.  You hope the book will explain to you clearly, step by step, how
to derive things from the axioms.  After all, this is the starting point of
mathematics, like a book that explains the basics of a computer programming
language.  But something is missing.  You find you can't even understand the
first proof or do the first exercise.  Symbols such as $\exists$ and $\forall$
permeate the page without any mention of where they came from or how to
manipulate them; the author assumes you are totally familiar with them and
doesn't even tell you what they mean.  By now you know that $\exists$ means
``there exists'' and $\forall$ means ``for all,'' but shouldn't the rules for
manipulating these symbols be part of the axioms?  You still have no idea
how you could even describe the axioms to a computer.

Certainly there is something much different here from the technical
literature you're used to reading.  A computer language manual almost
always explains very clearly what all the symbols mean, precisely what
they do, and the rules used for combining them, and you work your way up
from there.

After glancing at four or five other such books, you come to the realization
that there is another whole field of study that you need just to get to the
point at which you can understand the axioms of set theory.  The field is
called ``logic.''  In fact, some of the books did recommend it as a
prerequisite, but it just didn't sink in.  You assumed logic was, well, just
logic, something that a person with common sense intuitively understood.  Why
waste your time reading boring treatises on symbolic logic, the manipulation
of 1's and 0's that computers do, when you already know that?  But this is a
different kind of logic, quite alien to you.  The subject of {\sc nand} and
{\sc nor} gates is not even touched upon or in any case has to do with only a
very small part of this field.

So your quest continues.  Skimming through the first couple of introductory
books, you get a general idea of what logic is about and what quantifiers
(``for all,'' ``there exists'') mean, but you find their examples somewhat
trivial and mildly annoying (``all dogs are animals,'' ``some animals are
dogs,'' and such).  But all you want to know is what the rules are for
manipulating the symbols so you can apply them to set theory.  Some formulas
describing the relationships among quantifiers ($\exists$ and $\forall$) are
listed in tables, along with some verbal reasoning to justify them.
Presumably, if you want to find out if a formula is correct, you go through
this same kind of mental reasoning process, possibly using images of dogs and
animals. Intuitively, the formulas seem to make sense.  But when you ask
yourself, ``What are the rules I need to get a computer to figure out whether
this formula is correct?'', you still don't know.  Certainly you don't ask the
computer to imagine dogs and animals.

You look at some more advanced logic books.  Many of them have an introductory
chapter summarizing set theory, which turns out to be a prerequisite.  You
need logic to understand set theory, but it seems you also need set theory to
understand logic!  These books jump right into proving rather advanced
theorems about logic, without offering the faintest clue about where the logic
came from that allows them to prove these theorems.

Luckily, you come across an elementary book of logic that, halfway through,
after the usual truth tables and metaphors, presents in a clear, precise way
what you've been looking for all along: the axioms!  They're divided into
propositional calculus (also called sentential logic) and predicate calculus
(also called first-order logic),\index{first-order logic} with rules so simple
and crystal clear that now you can finally program a computer to understand
them.  Indeed, they're no harder than learning how to play a game of chess.
As far as what you seem to need is concerned, the whole book could have been
written in five pages!

{\em Now} you think you've found the ultimate source of mathematical
truth.  So---the axioms of mathematics consist of these axioms of logic,
together with the axioms of ZF set theory. (By now you've also been able to
figure out how to translate the ZF axioms from English into the
actual symbols of logic which you can now manipulate according to
precise, easy-to-understand rules.)

Of course, you still don't understand how ``all of mathematics can be
derived from set theory,'' but maybe this will reveal itself in due
course.

You eagerly set out to program the axioms and rules into a computer and start
to look at the theorems you will have to prove as the logic is developed.  All
sorts of important theorems start popping up:  the deduction
theorem,\index{deduction theorem} the substitution theorem,\index{substitution
theorem} the completeness theorem of propositional calculus,\index{first-order
logic!completeness} the completeness theorem of predicate calculus.  Uh-oh,
there seems to be trouble.  They all get harder and harder, and not one of
them can be derived with the axioms and rules of logic you've just been
handed.  Instead, they all require ``metalogic'' for their proofs, a kind of
mixture of logic and set theory that allows you to prove things {\em about}
the axioms and theorems of logic rather than {\em with} them.

You plow ahead anyway.  A month later, you've spent much of your
free time getting the computer to verify proofs in propositional calculus.
You've programmed in the axioms, but you've also had to program in the
deduction theorem, the substitution theorem, and the completeness theorem of
propositional calculus, which by now you've resigned yourself to treating as
rather complex additional axioms, since they can't be proved from the axioms
you were given.  You can now get the computer to verify and even generate
complete, rigorous, formal proofs\index{formal proof}.  Never mind that they
may have 100,000 steps---at least now you can have complete, absolute
confidence in them.  Unfortunately, the only theorems you have proved are
pretty trivial and you can easily verify them in a few minutes with truth
tables, if not by inspection.

It looks like your mathematician friend was right.  Getting the computer to do
serious mathematics with this kind of rigor seems almost hopeless.  Even
worse, it seems that the further along you get, the more ``axioms'' you have
to add, as each new theorem seems to involve additional ``metamathematical''
reasoning that hasn't been formalized, and none of it can be derived from the
axioms of logic.  Not only do the proofs keep growing exponentially as you get
further along, but the program to verify them keeps getting bigger and bigger
as you program in more ``metatheorems.''\index{metatheorem}\footnote{A
metatheorem is usually a statement that is too general to be directly provable
in a theory.  For example, ``if $n_1$, $n_2$, and $n_3$ are integers, then
$n_1+n_2+n_3$ is an integer'' is a theorem of number theory.  But ``for any
integer $k > 1$, if $n_1, \ldots, n_k$ are integers, then $n_1+\ldots +n_k$ is
an integer'' is a metatheorem, in other words a family of theorems, one for
every $k$.  The reason it is not a theorem is that the general sum $n_1+\ldots
+n_k$ (as a function of $k$) is not an operation that can be defined directly
in number theory.} The bugs\index{computer program bugs} that have cropped up
so far have already made you start to lose faith in the rigor you seem to have
achieved, and you know it's just going to get worse as your program gets larger.

\subsection{Mathematics and the Non-Specialist}

\begin{quote}
  {\em A real proof is not checkable by a machine, or even by any mathematician
not privy to the gestalt, the mode of thought of the particular field of
mathematics in which the proof is located.}
  \flushright\sc  Davis and Hersh\index{Davis, Phillip J.}
  \footnote{\cite{Davis}, p.~354.}\\
\end{quote}

The bulk of abstract or theoretical mathematics is ordinarily outside
the reach of anyone but a few specialists in each field who have completed
the necessary difficult internship in order to enter its coterie.  The
typical intelligent layperson has no reasonable hope of understanding much of
it, nor even the specialist mathematician of understanding other fields.  It
is like a foreign language that has no dictionary to look up the translation;
the only way you can learn it is by living in the country for a few years.  It
is argued that the effort involved in learning a specialty is a necessary
process for acquiring a deep understanding.  Of course, this is almost certainly
true if one is to make significant contributions to a field; in particular,
``doing'' proofs is probably the most important part of a mathematician's
training.  But is it also necessary to deny outsiders access to it?  Is it
necessary that abstract mathematics be so hard for a layperson to grasp?

A computer normally is of no help whatsoever.  Most published proofs are
actually just series of hints written in an informal style that requires
considerable knowledge of the field to understand.  These are the ``real
proofs'' referred to by Davis and Hersh.\index{informal proof}  There is an
implicit understanding that, in principle, such a proof could be converted to
a complete formal proof\index{formal proof}.  However, it is said that no one
would ever attempt such a conversion, even if they could, because that would
presumably require millions of steps (Section~\ref{dream}).  Unfortunately the
informal style automatically excludes the understanding of the proof
by anyone who hasn't gone through the necessary apprenticeship. The
best that the intelligent layperson can do is to read popular books about deep
and famous results; while this can be helpful, it can also be misleading, and
the lack of detail usually leaves the reader with no ability whatsoever to
explore any aspect of the field being described.

The statements of theorems often use sophisticated notation that makes them
inaccessible to the non-specialist.  For a non-specialist who wants to achieve
a deeper understanding of a proof, the process of tracing definitions and
lemmas back through their hierarchy\index{hierarchy} quickly becomes confusing
and discouraging.  Textbooks are usually written to train mathematicians or to
communicate to people who are already mathematicians, and large gaps in proofs
are often left as exercises to the reader who is left at an impasse if he or
she becomes stuck.

I believe that eventually computers will enable non-specialists and even
intelligent laypersons to follow almost any mathematical proof in any field.
Metamath is an attempt in that direction.  If all of mathematics were as
easily accessible as a computer programming language, I could envision
computer programmers and hobbyists who otherwise lack mathematical
sophistication exploring and being amazed by the world of theorems and proofs
in obscure specialties, perhaps even coming up with results of their own.  A
tremendous advantage would be that anyone could experiment with conjectures in
any field---the computer would offer instant feedback as to whether
an inference step was correct.

Mathematicians sometimes have to put up with the annoyance of
cranks\index{cranks} who lack a fundamental understanding of mathematics but
insist that their ``proofs'' of, say, Fermat's Last Theorem\index{Fermat's
Last Theorem} be taken seriously.  I think part of the problem is that these
people are misled by informal mathematical language, treating it as if they
were reading ordinary expository English and failing to appreciate the
implicit underlying rigor.  Such cranks are rare in the field of computers,
because computer languages are much more explicit, and ultimately the proof is
in whether a computer program works or not.  With easily accessible
computer-based abstract mathematics, a mathematician could say to a crank,
``don't bother me until you've demonstrated your claim on the computer!''

% 22-May-04 nm
% Attempt to move De Millo quote so it doesn't separate from attribution
% CHANGE THIS NUMBER (AND ELIMINATE IF POSSIBLE) WHEN ABOVE TEXT CHANGES
\vspace{-0.5em}

\subsection{An Impossible Dream?}\label{dream}

\begin{quote}
  {\em Even quite basic theorems would demand almost unbelievably vast
  books to display their proofs.}
    \flushright\sc  Robert E. Edwards\footnote{\cite{Edwards}, p.~68.}\\
\end{quote}\index{Edwards, Robert E.}

\begin{quote}
  {\em Oh, of course no one ever really {\em does} it.  It would take
  forever!  You just show that you could do it, that's sufficient.}
    \flushright\sc  ``The Ideal Mathematician''
    \index{Davis, Phillip J.}\footnote{\cite{Davis},
p.~40.}\\
\end{quote}

\begin{quote}
  {\em There is a theorem in the primitive notation of set theory that
  corresponds to the arithmetic theorem `$1000+2000=3000$'.  The formula
  would be forbiddingly long\ldots even if [one] knows the definitions
  and is asked to simplify the long formula according to them, chances are
  he will make errors and arrive at some incorrect result.}
    \flushright\sc  Hao Wang\footnote{\cite{Wang}, p.~140.}\\
\end{quote}\index{Wang, Hao}

% 22-May-04 nm
% Attempt to move De Millo quote so it doesn't separate from attribution
% CHANGE THIS NUMBER (AND ELIMINATE IF POSSIBLE) WHEN ABOVE TEXT CHANGES
\vspace{-0.5em}

\begin{quote}
  {\em The {\em Principia Mathematica} was the crowning achievement of the
  formalists.  It was also the deathblow of the formalist view.\ldots
  {[Rus\-sell]} failed, in three enormous volumes, to get beyond the elementary
  facts of arithmetic.  He showed what can be done in principle and what
  cannot be done in practice.  If the mathematical process were really
  one of strict, logical progression, we would still be counting our
  fingers.\ldots
  One theoretician estimates\ldots that a demonstration of one of
  Ramanujan's conjectures assuming set theory and elementary analysis would
  take about two thousand pages; the length of a deduction from first principles
  is nearly in\-con\-ceiv\-a\-ble\ldots The probabilists argue that\ldots any
  very long proof can at best be viewed as only probably correct\ldots}
  \flushright\sc Richard de Millo et. al.\footnote{\cite{deMillo}, pp.~269,
  271.}\\
\end{quote}\index{de Millo, Richard}

A number of writers have conveyed the impression that the kind of absolute
rigor provided by Metamath\index{Metamath} is an impossible dream, suggesting
that a complete, formal verification\index{formal proof} of a typical theorem
would take millions of steps in untold volumes of books.  Even if it could be
done, the thinking sometimes goes, all meaning would be lost in such a
monstrous, tedious verification.\index{informal proof}\index{proof length}

These writers assume, however, that in order to achieve the kind of complete
formal verification they desire one must break down a proof into individual
primitive steps that make direct reference to the axioms.  This is
not necessary.  There is no reason not to make use of previously proved
theorems rather than proving them over and over.

Just as important, definitions\index{definition} can be introduced along
the way, allowing very complex formulas to be represented with few
symbols.  Not doing this can lead to absurdly long formulas.  For
example, the mere statement of
G\"{o}del's incompleteness theorem\index{G\"{o}del's
incompleteness theorem}, which can be expressed with a small number of
defined symbols, would require about 20,000 primitive symbols to express
it.\index{Boolos, George S.}\footnote{George S.\ Boolos, lecture at
Massachusetts Institute of Technology, spring 1990.} An extreme example
is Bourbaki's\label{bourbaki} language for set theory, which requires
4,523,659,424,929 symbols plus 1,179,618,517,981 disambiguatory links
(lines connecting symbol pairs, usually drawn below or above the
formula) to express the number
``one'' \cite{Mathias}.\index{Mathias, Adrian R. D.}\index{Bourbaki,
Nicolas}
% http://www.dpmms.cam.ac.uk/~ardm/

A hierarchy\index{hierarchy} of theorems and definitions permits an
exponential growth in the formula sizes and primitive proof steps to be
described with only a linear growth in the number of symbols used.  Of course,
this is how ordinary informal mathematics is normally done anyway, but with
Metamath\index{Metamath} it can be done with absolute rigor and precision.

\subsection{Beauty}


\begin{quote}
  {\em No one shall be able to drive us from the paradise that Cantor has
created for us.}
   \flushright\sc  David Hilbert\footnote{As quoted in \cite{Moore}, p.~131.}\\
\end{quote}\index{Hilbert, David}

\needspace{3\baselineskip}
\begin{quote}
  {\em Mathematics possesses not only truth, but some supreme beauty ---a
  beauty cold and austere, like that of a sculpture.}
    \flushright\sc  Bertrand
    Russell\footnote{\cite{Russell}.}\\
\end{quote}\index{Russell, Bertrand}

\begin{quote}
  {\em Euclid alone has looked on Beauty bare.}
  \flushright\sc Edna Millay\footnote{As quoted in \cite{Davis}, p.~150.}\\
\end{quote}\index{Millay, Edna}

For most people, abstract mathematics is distant, strange, and
incomprehensible.  Many popular books have tried to convey some of the sense
of beauty in famous theorems.  But even an intelligent layperson is left with
only a general idea of what a theorem is about and is hardly given the tools
needed to make use of it.  Traditionally, it is only after years of arduous
study that one can grasp the concepts needed for deep understanding.
Metamath\index{Metamath} allows you to approach the proof of the theorem from
a quite different perspective, peeling apart the formulas and definitions
layer by layer until an entirely different kind of understanding is achieved.
Every step of the proof is there, pieced together with absolute precision and
instantly available for inspection through a microscope with a magnification
as powerful as you desire.

A proof in itself can be considered an object of beauty.  Constructing an
elegant proof is an art.  Once a famous theorem has been proved, often
considerable effort is made to find simpler and more easily understood
proofs.  Creating and communicating elegant proofs is a major concern of
mathematicians.  Metamath is one way of providing a common language for
archiving and preserving this information.

The length of a proof can, to a certain extent, be considered an
objective measure of its ``beauty,'' since shorter proofs are usually
considered more elegant.  In the set theory database
\texttt{set.mm}\index{set theory database (\texttt{set.mm})}%
\index{Metamath Proof Explorer}
provided with Metamath, one goal was to make all proofs as short as possible.

\needspace{4\baselineskip}
\subsection{Simplicity}

\begin{quote}
  {\em God made man simple; man's complex problems are of his own
  devising.}
    \flushright\sc Eccles. 7:29\footnote{Jerusalem Bible.}\\
\end{quote}\index{Bible}

\needspace{3\baselineskip}
\begin{quote}
  {\em God made integers, all else is the work of man.}
    \flushright\sc Leopold Kronecker\footnote{{\em Jahresbericht
	der Deutschen Mathematiker-Vereinigung }, vol. 2, p. 19.}\\
\end{quote}\index{Kronecker, Leopold}

\needspace{3\baselineskip}
\begin{quote}
  {\em For what is clear and easily comprehended attracts; the
  complicated repels.}
    \flushright\sc David Hilbert\footnote{As quoted in \cite{deMillo},
p.~273.}\\
\end{quote}\index{Hilbert, David}

The Metamath\index{Metamath} language is simple and Spartan.  Metamath treats
all mathematical expressions as simple sequences of symbols, devoid of meaning.
The higher-level or ``metamathematical'' notions underlying Metamath are about
as simple as they could possibly be.  Each individual step in a proof involves
a single basic concept, the substitution of an expression for a variable, so
that in principle almost anyone, whether mathematician or not, can
completely understand how it was arrived at.

In one of its most basic applications, Metamath\index{Metamath} can be used to
develop the foundations of mathematics\index{foundations of mathematics} from
the very beginning.  This is done in the set theory database that is provided
with the Metamath package and is the subject matter
of Chapter~\ref{fol}. Any language (a metalanguage\index{metalanguage})
used to describe mathematics (an object language\index{object language}) must
have a mathematical content of its own, but it is desirable to keep this
content down to a bare minimum, namely that needed to make use of the
inference rules specified by the axioms.  With any metalanguage there is a
``chicken and egg'' problem somewhat like circular reasoning:  you must assume
the validity of the mathematics of the metalanguage in order to prove the
validity of the mathematics of the object language.  The mathematical content
of Metamath itself is quite limited.  Like the rules of a game of chess, the
essential concepts are simple enough so that virtually anyone should be able to
understand them (although that in itself will not let you play like
a master).  The symbols that Metamath manipulates do not in themselves
have any intrinsic meaning.  Your interpretation of the axioms that you supply
to Metamath is what gives them meaning.  Metamath is an attempt to strip down
mathematical thought to its bare essence and show you exactly how the symbols
are manipulated.

Philosophers and logicians, with various motivations, have often thought it
important to study ``weak'' fragments of logic\index{weak logic}
\cite{Anderson}\index{Anderson, Alan Ross} \cite{MegillBunder}\index{Megill,
Norman}\index{Bunder, Martin}, other unconventional systems of logic (such as
``modal'' logic\index{modal logic} \cite[ch.\ 27]{Boolos}\index{Boolos, George
S.}), and quantum logic\index{quantum logic} in physics
\cite{Pavicic}\index{Pavi{\v{c}}i{\'{c}}, M.}.  Metamath\index{Metamath}
provides a framework in which such systems can be expressed, with an absolute
precision that makes all underlying metamathematical assumptions rigorous and
crystal clear.

Some schools of philosophical thought, for example
intuitionism\index{intuitionism} and constructivism\index{constructivism},
demand that the notions underlying any mathematical system be as simple and
concrete as possible.  Metamath should meet the requirements of these
philosophies.  Metamath must be taught the symbols, axioms\index{axiom}, and
rules\index{rule} for a specific theory, from the skeptical (such as
intuitionism\index{intuitionism}\footnote{Intuitionism does not accept the law
of excluded middle (``either something is true or it is not true'').  See
\cite[p.~xi]{Tymoczko}\index{Tymoczko, Thomas} for discussion and references
on this topic.  Consider the theorem, ``There exist irrational numbers $a$ and
$b$ such that $a^b$ is rational.''  An intuitionist would reject the following
proof:  If $\sqrt{2}^{\sqrt{2}}$ is rational, we are done.  Otherwise, let
$a=\sqrt{2}^{\sqrt{2}}$ and $b=\sqrt{2}$. Then $a^b=2$, which is rational.})
to the bold (such as the axiom of choice in set theory\footnote{The axiom of
choice\index{Axiom of Choice} asserts that given any collection of pairwise
disjoint nonempty sets, there exists a set that has exactly one element in
common with each set of the collection.  It is used to prove many important
theorems in standard mathematics.  Some philosophers object to it because it
asserts the existence of a set without specifying what the set contains
\cite[p.~154]{Enderton}\index{Enderton, Herbert B.}.  In one foundation for
mathematics due to Quine\index{Quine, Willard Van Orman}, that has not been
otherwise shown to be inconsistent, the axiom of choice turns out to be false
\cite[p.~23]{Curry}\index{Curry, Haskell B.}.  The \texttt{show
trace{\char`\_}back} command of the Metamath program allows you to find out
whether the axiom of choice, or any other axiom, was assumed by a
proof.}\index{\texttt{show trace{\char`\_}back} command}).

The simplicity of the Metamath language lets the algorithm (computer program)
that verifies the validity of a Metamath proof be straightforward and
robust.  You can have confidence that the theorems it verifies really can be
derived from your axioms.

\subsection{Rigor}

\begin{quote}
  {\em Rigor became a goal with the Greeks\ldots But the efforts to
  pursue rigor to the utmost have led to an impasse in which there is
  no longer any agreement on what it really means.  Mathematics remains
  alive and vital, but only on a pragmatic basis.}
    \flushright\sc  Morris Kline\footnote{\cite{Kline}, p.~1209.}\\
\end{quote}\index{Kline, Morris}

Kline refers to a much deeper kind of rigor than that which we will discuss in
this section.  G\"{o}del's incompleteness theorem\index{G\"{o}del's
incompleteness theorem} showed that it is impossible to achieve absolute rigor
in standard mathematics because we can never prove that mathematics is
consistent (free from contradictions).\index{consistent theory}  If
mathematics is consistent, we will never know it, but must rely on faith.  If
mathematics is inconsistent, the best we can hope for is that some clever
future mathematician will discover the inconsistency.  In this case, the
axioms would probably be revised slightly to eliminate the inconsistency, as
was done in the case of Russell's paradox,\index{Russell's paradox} but the
bulk of mathematics would probably not be affected by such a discovery.
Russell's paradox, for example, did not affect most of the remarkable results
achieved by 19th-century and earlier mathematicians.  It mainly invalidated
some of Gottlob Frege's\index{Frege, Gottlob} work on the foundations of
mathematics in the late 1800's; in fact Frege's work inspired Russell's
discovery.  Despite the paradox, Frege's work contains important concepts that
have significantly influenced modern logic.  Kline's {\em Mathematics, The
Loss of Certainty} \cite{Klinel}\index{Kline, Morris} has an interesting
discussion of this topic.

What {\em can} be achieved with absolute certainty\index{certainty} is the
knowledge that if we assume the axioms are consistent and true, then the
results derived from them are true.  Part of the beauty of mathematics is that
it is the one area of human endeavor where absolute certainty can be achieved
in this sense.  A mathematical truth will remain such for eternity.  However,
our actual knowledge of whether a particular statement is a mathematical truth
is only as certain as the correctness of the proof that establishes it.  If
the proof of a statement is questionable or vague, we can't have absolute
confidence in the truth that the statement claims.

Let us look at some traditional ways of expressing proofs.

Except in the field of formal logic\index{formal logic}, almost all
traditional proofs in mathematics are really not proofs at all, but rather
proof outlines or hints as to how to go about constructing the proof.  Many
gaps\index{gaps in proofs} are left for the reader to fill in. There are
several reasons for this.  First, it is usually assumed in mathematical
literature that the person reading the proof is a mathematician familiar with
the specialty being described, and that the missing steps are obvious to such
a reader or at least that the reader is capable of filling them in.  This
attitude is fine for professional mathematicians in the specialty, but
unfortunately it often has the drawback of cutting off the rest of the world,
including mathematicians in other specialties, from understanding the proof.
We discussed one possible resolution to this on p.~\pageref{envision}.
Second, it is often assumed that a complete formal proof\index{formal proof}
would require countless millions of symbols (Section~\ref{dream}). This might
be true if the proof were to be expressed directly in terms of the axioms of
logic and set theory,\index{set theory} but it is usually not true if we allow
ourselves a hierarchy\index{hierarchy} of definitions and theorems to build
upon, using a notation that allows us to introduce new symbols, definitions,
and theorems in a precisely specified way.

Even in formal logic,\index{formal logic} formal proofs\index{formal proof}
that are considered complete still contain hidden or implicit information.
For example, a ``proof'' is usually defined as a sequence of
wffs,\index{well-formed formula (wff)}\footnote{A {\em wff} or well-formed
formula is a mathematical expression (string of symbols) constructed according
to some precise rules.  A formal mathematical system\index{formal system}
contains (1) the rules for constructing syntactically correct
wffs,\index{syntax rules} (2) a list of starting wffs called
axioms,\index{axiom} and (3) one or more rules prescribing how to derive new
wffs, called theorems\index{theorem}, from the axioms or previously derived
theorems.  An example of such a system is contained in
Metamath's\index{Metamath} set theory database, which defines a formal
system\index{formal system} from which all of standard mathematics can be
derived.  Section~\ref{startf} steps you through a complete example of a formal
system, and you may want to skim it now if you are unfamiliar with the
concept.} each of which is an axiom or follows from a rule applied to previous
wffs in the sequence.  The implicit part of the proof is the algorithm by
which a sequence of symbols is verified to be a valid wff, given the
definition of a wff.  The algorithm in this case is rather simple, but for a
computer to verify the proof,\index{automated proof verification} it must have
the algorithm built into its verification program.\footnote{It is possible, of
course, to specify wff construction syntax outside of the program itself
with a suitable input language (the Metamath language being an example), but
some proof-verification or theorem-proving programs lack the ability to extend
wff syntax in such a fashion.} If one deals exclusively with axioms and
elementary wffs, it is straightforward to implement such an algorithm.  But as
more and more definitions are added to the theory in order to make the
expression of wffs more compact, the algorithm becomes more and more
complicated.  A computer program that implements the algorithm becomes larger
and harder to understand as each definition is introduced, and thus more prone
to bugs.\index{computer program bugs}  The larger the program, the
more suspicious the mathematician may be about
the validity of its algorithms.  This is especially true because
computer programs are inherently hard to follow to begin with, and few people
enjoy verifying them manually in detail.

Metamath\index{Metamath} takes a different approach.  Metamath's ``knowledge''
is limited to the ability to substitute variables for expressions, subject to
some simple constraints.  Once the basic algorithm of Metamath is assumed to
be debugged, and perhaps independently confirmed, it
can be trusted once and for all.  The information that Metamath needs to
``understand'' mathematics is contained entirely in the body of knowledge
presented to Metamath.  Any errors in reasoning can only be errors in the
axioms or definitions contained in this body of knowledge.  As a
``constructive'' language\index{constructive language} Metamath has no
conditional branches or loops like the ones that make computer programs hard
to decipher; instead, the language can only build new sequences of symbols
from earlier sequences  of symbols.

The simplicity of the rules that underlie Metamath not only makes Metamath
easy to learn but also gives Metamath a great deal of flexibility. For
example, Metamath is not limited to describing standard first-order
logic\index{first-order logic}; higher-order logics\index{higher-order logic}
and fragments of logic\index{weak logic} can be described just as easily.
Metamath gives you the freedom to define whatever wff notation you prefer; it
has no built-in conception of the syntax of a wff.\index{well-formed formula
(wff)}  With suitable axioms and definitions, Metamath can even describe and
prove things about itself.\index{Metamath!self-description}  (John
Harrison\index{Harrison, John} discusses the ``reflection''
principle\index{reflection principle} involved in self-descriptive systems in
\cite{Harrison}.)

The flexibility of Metamath requires that its proofs specify a lot of detail,
much more than in an ordinary ``formal'' proof.\index{formal proof}  For
example, in an ordinary formal proof, a single step consists of displaying the
wff that constitutes that step.  In order for a computer program to verify
that the step is acceptable, it first must verify that the symbol sequence
being displayed is an acceptable wff.\index{automated proof verification} Most
proof verifiers have at least basic wff syntax built into their programs.
Metamath has no hard-wired knowledge of what constitutes a wff built into it;
instead every wff must be explicitly constructed based on rules defining wffs
that are present in a database.  Thus a single step in an ordinary formal
proof may be correspond to many steps in a Metamath proof. Despite the larger
number of steps, though, this does not mean that a Metamath proof must be
significantly larger than an ordinary formal proof. The reason is that since
we have constructed the wff from scratch, we know what the wff is, so there is
no reason to display it.  We only need to refer to a sequence of statements
that construct it.  In a sense, the display of the wff in an ordinary formal
proof is an implicit proof of its own validity as a wff; Metamath just makes
the proof explicit. (Section~\ref{proof} describes Metamath's proof notation.)

\section{Computers and Mathematicians}

\begin{quote}
  {\em The computer is important, but not to mathematics.}
    \flushright\sc  Paul Halmos\footnote{As quoted in \cite{Albers}, p.~121.}\\
\end{quote}\index{Halmos, Paul}

Pure mathematicians have traditionally been indifferent to computers, even to
the point of disdain.\index{computers and pure mathematics}  Computer science
itself is sometimes considered to fall in the mundane realm of ``applied''
mathematics, perhaps essential for the real world but intellectually unexciting
to those who seek the deepest truths in mathematics.  Perhaps a reason for this
attitude towards computers is that there is little or no computer software that
meets their needs, and there may be a general feeling that such software could
not even exist.  On the one hand, there are the practical computer algebra
systems, which can perform amazing symbolic manipulations in algebra and
calculus,\index{computer algebra system} yet can't prove the simplest
existence theorem, if the idea of a proof is present at all.  On the other
hand, there are specialized automated theorem provers that technically speaking
may generate correct proofs.\index{automated theorem proving}  But sometimes
their specialized input notation may be cryptic and their output perceived to
be long, inelegant, incomprehensible proofs.    The output
may be viewed with suspicion, since the program that generates it tends to be
very large, and its size increases the potential for bugs\index{computer
program bugs}.  Such a proof may be considered trustworthy only if
independently verified and ``understood'' by a human, but no one wants to
waste their time on such a boring, unrewarding chore.



\needspace{4\baselineskip}
\subsection{Trusting the Computer}

\begin{quote}
  {\em \ldots I continue to find the quasi-empirical interpretation of
  computer proofs to be the more plausible.\ldots Since not
  everything that claims to be a computer proof can be
  accepted as valid, what are the mathematical criteria for acceptable
  computer proofs?}
    \flushright\sc  Thomas Tymoczko\footnote{\cite{Tymoczko}, p.~245.}\\
\end{quote}\index{Tymoczko, Thomas}

In some cases, computers have been essential tools for proving famous
theorems.  But if a proof is so long and obscure that it can be verified in a
practical way only with a computer, it is vaguely felt to be suspicious.  For
example, proving the famous four-color theorem\index{four-color
theorem}\index{proof length} (``a map needs no more than four colors to
prevent any two adjacent countries from having the same color'') can presently
only be done with the aid of a very complex computer program which originally
required 1200 hours of computer time. There has been considerable debate about
whether such a proof can be trusted and whether such a proof is ``real''
mathematics \cite{Swart}\index{Swart, E. R.}.\index{trusting computers}

However, under normal circumstances even a skeptical mathematician would have a
great deal of confidence in the result of multiplying two numbers on a pocket
calculator, even though the precise details of what goes on are hidden from its
user.  Even the verification on a supercomputer that a huge number is prime is
trusted, especially if there is independent verification; no one bothers to
debate the philosophical significance of its ``proof,'' even though the actual
proof would be so large that it would be completely impractical to ever write
it down on paper.  It seems that if the algorithm used by the computer is
simple enough to be readily understood, then the computer can be trusted.

Metamath\index{Metamath} adopts this philosophy.  The simplicity of its
language makes it easy to learn, and because of its simplicity one can have
essentially absolute confidence that a proof is correct. All axioms, rules, and
definitions are available for inspection at any time because they are defined
by the user; there are no hidden or built-in rules that may be prone to subtle
bugs\index{computer program bugs}.  The basic algorithm at the heart of
Metamath is simple and fixed, and it can be assumed to be bug-free and robust
with a degree of confidence approaching certainty.
Independently written implementations of the Metamath verifier
can reduce any residual doubt on the part of a skeptic even further;
there are now over a dozen such implementations, written by many people.

\subsection{Trusting the Mathematician}\label{trust}

\begin{quote}
  {\em There is no Algebraist nor Mathematician so expert in his science, as
  to place entire confidence in any truth immediately upon his discovery of it,
  or regard it as any thing, but a mere probability.  Every time he runs over
  his proofs, his confidence encreases; but still more by the approbation of
  his friends; and is rais'd to its utmost perfection by the universal assent
  and applauses of the learned world.}
  \flushright\sc David Hume\footnote{{\em A Treatise of Human Nature}, as
  quoted in \cite{deMillo}, p.~267.}\\
\end{quote}\index{Hume, David}

\begin{quote}
  {\em Stanislaw Ulam estimates that mathematicians publish 200,000 theorems
  every year.  A number of these are subsequently contradicted or otherwise
  disallowed, others are thrown into doubt, and most are ignored.}
  \flushright\sc Richard de Millo et. al.\footnote{\cite{deMillo}, p.~269.}\\
\end{quote}\index{Ulam, Stanislaw}

Whether or not the computer can be trusted, humans  of course will occasionally
err. Only the most memorable proofs get independently verified, and of these
only a handful of truly great ones achieve the status of being ``known''
mathematical truths that are used without giving a second thought to their
correctness.

There are many famous examples of incorrect theorems and proofs in
mathematical literature.\index{errors in proofs}

\begin{itemize}
\item There have been thousands of purported proofs of Fermat's Last
Theorem\index{Fermat's Last Theorem} (``no integer solutions exist to $x^n +
y^n = z^n$ for $n > 2$''), by amateurs, cranks, and well-regarded
mathematicians \cite[p.~5]{Stark}\index{Stark, Harold M}.  Fermat wrote a note
in his copy of Bachet's {\em Diophantus} that he found ``a truly marvelous
proof of this theorem but this margin is too narrow to contain it''
\cite[p.~507]{Kramer}.  A recent, much publicized proof by Yoichi
Miyaoka\index{Miyaoka, Yoichi} was shown to be incorrect ({\em Science News},
April 9, 1988, p.~230).  The theorem was finally proved by Andrew
Wiles\index{Wiles, Andrew} ({\em Science News}, July 3, 1993, p.~5), but it
initially had some gaps and took over a year after its announcement to be
checked thoroughly by experts.  On Oct. 25, 1994, Wiles announced that the last
gap found in his proof had been filled in.
  \item In 1882, M. Pasch discovered that an axiom was omitted from Euclid's
formulation of geometry\index{Euclidean geometry}; without it, the proofs of
important theorems of Euclid are not valid.  Pasch's axiom\index{Pasch's
axiom} states that a line that intersects one side of a triangle must also
intersect another side, provided that it does not touch any of the triangle's
vertices.  The omission of Pasch's axiom went unnoticed for 2000
years \cite[p.~160]{Davis}, in spite of (one presumes) the thousands of
students, instructors, and mathematicians who studied Euclid.
  \item The first published proof of the famous Schr\"{o}der--Bernstein
theorem\index{Schr\"{o}der--Bernstein theorem} in set theory was incorrect
\cite[p.~148]{Enderton}\index{Enderton, Herbert B.}.  This theorem states
that if there exists a 1-to-1 function\footnote{A {\em set}\index{set} is any
collection of objects. A {\em function}\index{function} or {\em
mapping}\index{mapping} is a rule that assigns to each element of one set
(called the function's {\em domain}\index{domain}) an element from another
set.} from set $A$ into set $B$ and vice-versa, then sets $A$ and $B$ have
a 1-to-1 correspondence.  Although it sounds simple and obvious,
the standard proof is quite long and complex.
  \item In the early 1900's, Hilbert\index{Hilbert, David} published a
purported proof of the continuum hypothesis\index{continuum hypothesis}, which
was eventually established as unprovable by Cohen\index{Cohen, Paul} in 1963
\cite[p.~166]{Enderton}.  The continuum hypothesis states that no
infinity\index{infinity} (``transfinite cardinal number'')\index{cardinal,
transfinite} exists whose size (or ``cardinality''\index{cardinality}) is
between the size of the set of integers and the size of the set of real
numbers.  This hypothesis originated with German mathematician Georg
Cantor\index{Cantor, Georg} in the late 1800's, and his inability to prove it
is said to have contributed to mental illness that afflicted him in his later
years.
  \item An incorrect proof of the four-color theorem\index{four-color theorem}
was published by Kempe\index{Kempe, A. B.} in 1879
\cite[p.~582]{Courant}\index{Courant, Richard}; it stood for 11 years before
its flaw was discovered.  This theorem states that any map can be colored
using only four colors, so that no two adjacent countries have the same
color.  In 1976 the theorem was finally proved by the famous computer-assisted
proof of Haken, Appel, and Koch \cite{Swart}\index{Appel, K.}\index{Haken,
W.}\index{Koch, K.}.  Or at least it seems that way.  Mathematician
H.~S.~M.~Coxeter\index{Coxeter, H. S. M.} has doubts \cite[p.~58]{Davis}:  ``I
have a feeling that that is an untidy kind of use of the computers, and the more
you correspond with Haken and Appel, the more shaky you seem to be.''
  \item Many false ``proofs'' of the Poincar\'{e}
conjecture\index{Poincar\'{e} conjecture} have been proposed over the years.
This conjecture states that any object that mathematically behaves like a
three-dimensional sphere is a three-dimensional sphere topologically,
regardless of how it is distorted.  In March 1986, mathematicians Colin
Rourke\index{Rourke, Colin} and Eduardo R\^{e}go\index{R\^{e}go, Eduardo}
caused  a stir in the mathematical community by announcing that they had found
a proof; in November of that year the proof was found to be false \cite[p.
218]{PetersonI}.  It was finally proved in 2003 by Grigory Perelman
\label{poincare}\index{Szpiro, George}\index{Perelman, Grigory}\cite{Szpiro}.
 \end{itemize}

Many counterexamples to ``theorems'' in recent mathematical
literature related to Clifford algebras\index{Clifford algebras}
 have been found by Pertti
Lounesto (who passed away in 2002).\index{Lounesto, Pertti}
See the web page \url{http://mathforum.org/library/view/4933.html}.
% http://users.tkk.fi/~ppuska/mirror/Lounesto/counterexamples.htm

One of the purposes of Metamath\index{Metamath} is to allow proofs to be
expressed with absolute precision.  Developing a proof in the Metamath
language can be challenging, because Metamath will not permit even the
tiniest mistake.\index{errors in proofs}  But once the proof is created, its
correctness can be trusted immediately, without having to depend on the
process of peer review for confirmation.

\section{The Use of Computers in Mathematics}

\subsection{Computer Algebra Systems}

For the most part, you will find that Metamath\index{Metamath} is not a
practical tool for manipulating numbers.  (Even proving that $2 + 2 = 4$, if
you start with set theory, can be quite complex!)  Several commercial
mathematics packages are quite good at arithmetic, algebra, and calculus, and
as practical tools they are invaluable.\index{computer algebra system} But
they have no notion of proof, and cannot understand statements starting with
``there exists such and such...''.

Software packages such as Mathematica \cite{Wolfram}\index{Mathematica} do not
concern themselves with proofs but instead work directly with known results.
These packages primarily emphasize heuristic rules such as the substitution of
equals for equals to achieve simpler expressions or expressions in a different
form.  Starting with a rich collection of built-in rules and algorithms, users
can add to the collection by means of a powerful programming language.
However, results such as, say, the existence of a certain abstract object
without displaying the actual object cannot be expressed (directly) in their
languages.  The idea of a proof from a small set of axioms is absent.  Instead
this software simply assumes that each fact or rule you add to the built-in
collection of algorithms is valid.  One way to view the software is as a large
collection of axioms from which the software, with certain goals, attempts to
derive new theorems, for example equating a complex expression with a simpler
equivalent. But the terms ``theorem''\index{theorem} and
``proof,''\index{proof} for example, are not even mentioned in the index of
the user's manual for Mathematica.\index{Mathematica and proofs}  What is also
unsatisfactory from a philosophical point of view is that there is no way to
ensure the validity of the results other than by trusting the writer of each
application module or tediously checking each module by hand, similar to
checking a computer program for bugs.\index{computer program
bugs}\footnote{Two examples illustrate why the knowledge database of computer
algebra systems should sometimes be regarded with a certain caution.  If you
ask Mathematica (version 3.0) to \texttt{Solve[x\^{ }n + y\^{ }n == z\^{ }n , n]}
it will respond with \texttt{\{\{n-\char`\>-2\}, \{n-\char`\>-1\},
\{n-\char`\>1\}, \{n-\char`\>2\}\}}. In other words, Mathematica seems to
``know'' that Fermat's Last Theorem\index{Fermat's Last Theorem} is true!  (At
the time this version of Mathematica was released this fact was unknown.)  If
you ask Maple\index{Maple} to \texttt{solve(x\^{ }2 = 2\^{ }x)} then
\texttt{simplify(\{"\})}, it returns the solution set \texttt{\{2, 4\}}, apparently
unaware that 0.7666647\ldots is also a solution.} While of course extremely
valuable in applied mathematics, computer algebra systems tend to be of little
interest to the theoretical mathematician except as aids for exploring certain
specific problems.

Because of possible bugs, trusting the output of a computer algebra system for
use as theorems in a proof-verifier would defeat the latter's goal of rigor.
On the other hand, a fact such that a certain relatively large number is
prime, while easy for a computer algebra system to derive, might have a long,
tedious proof that could overwhelm a proof-verifier. One approach for linking
computer algebra systems to a proof-verifier while retaining the advantages of
both is to add a hypothesis to each such theorem indicating its source.  For
example, a constant {\sc maple} could indicate the theorem came from the Maple
package, and would mean ``assuming Maple is consistent, then\ldots''  This and
many other topics concerning the formalization of mathematics are discussed in
John Harrison's\index{Harrison, John} very interesting
PhD thesis~\cite{Harrison-thesis}.

\subsection{Automated Theorem Provers}\label{theoremprovers}

A mathematical theory is ``decidable''\index{decidable theory} if a mechanical
method or algorithm exists that is guaranteed to determine whether or not a
particular formula is a theorem.  Among the few theories that are decidable is
elementary geometry,\index{Euclidean geometry} as was shown by a classic
result of logician Alfred Tarski\index{Tarski, Alfred} in 1948
\cite{Tarski}.\footnote{Tarski's result actually applies to a subset of the
geometry discussed in elementary textbooks.  This subset includes most of what
would be considered elementary geometry but it is not powerful enough to
express, among other things, the notions of the circumference and area of a
circle.  Extending the theory in a way that includes notions such as these
makes the theory undecidable, as was also shown by Tarski.  Tarski's algorithm
is far too inefficient to implement practically on a computer.  A practical
algorithm for proving a smaller subset of geometry theorems (those not
involving concepts of ``order'' or ``continuity'') was discovered by Chinese
mathematician Wu Wen-ts\"{u}n in 1977 \cite{Chou}\index{Chou,
Shang-Ching}.}\index{Wen-ts{\"{u}}n, Wu}  But most theories, including
elementary arithmetic, are undecidable.  This fact contributes to keeping
mathematics alive and well, since many mathematicians believe
that they will never be
replaced by computers (if they believe Roger Penrose's argument that a
computer can never replace the brain \cite{Penrose}\index{Penrose, Roger}).
In fact,  elementary geometry is often considered a ``dead'' field
for the simple reason that it is decidable.

On the other hand, the undecidability of a theory does not mean that one cannot
use a computer to search for proofs, providing one is willing to give up if a
proof is not found after a reasonable amount of time.  The field of automated
theorem proving\index{automated theorem proving} specializes in pursuing such
computer searches.  Among the more successful results to date are those based
on an algorithm known as Robinson's resolution principle
\cite{Robinson}\index{Robinson's resolution principle}.

Automated theorem provers can be excellent tools for those willing to learn
how to use them.  But they are not widely used in mainstream pure
mathematics, even though they could probably be useful in many areas.  There
are several reasons for this.  Probably most important, the main goal in pure
mathematics is to arrive at results that are considered to be deep or
important; proving them is essential but secondary.  Usually, an automated
theorem prover cannot assist in this main goal, and by the time the main goal
is achieved, the mathematician may have already figured out the proof as a
by-product.  There is also a notational problem.  Mathematicians are used to
using very compact syntax where one or two symbols (heavily dependent on
context) can represent very complex concepts; this is part of the
hierarchy\index{hierarchy} they have built up to tackle difficult problems.  A
theorem prover on the other hand might require that a theorem be expressed in
``first-order logic,''\index{first-order logic} which is the logic on which
most of mathematics is ultimately based but which is not ordinarily used
directly because expressions can become very long.  Some automated theorem
provers are experimental programs, limited in their use to very specialized
areas, and the goal of many is simply research into the nature of automated
theorem proving itself.  Finally, much research remains to be done to enable
them to prove very deep theorems.  One significant result was a
computer proof by Larry Wos\index{Wos, Larry} and colleagues that every Robbins
algebra\index{Robbins algebra} is a Boolean  algebra\index{Boolean algebra}
({\em New York Times}, Dec. 10, 1996).\footnote{In 1933, E.~V.\
Huntington\index{Huntington, E. V.}
presented the following axiom system for
Boolean algebra with a unary operation $n$ and a binary operation $+$:
\begin{center}
    $x + y = y + x$ \\
    $(x + y) + z = x + (y + z)$ \\
    $n(n(x) + y) + n(n(x) + n(y)) = x$
\end{center}
Herbert Robbins\index{Robbins, Herbert}, a student of Huntington, conjectured
that the last equation can be replaced with a simpler one:
\begin{center}
    $n(n(x + y) + n(x + n(y))) = x$
\end{center}
Robbins and Huntington could not find a proof.  The problem was
later studied unsuccessfully by Tarski\index{Tarski, Alfred} and his
students, and it remained an unsolved problem until a
computer found the proof in 1996.  For more information on
the Robbins algebra problem see \cite{Wos}.}

How does Metamath\index{Metamath} relate to automated theorem provers?  A
theorem prover is primarily concerned with one theorem at a time (perhaps
tapping into a small database of known theorems) whereas Metamath is more like
a theorem archiving system, storing both the theorem and its proof in a
database for access and verification.  Metamath is one answer to ``what do you
do with the output of a theorem prover?''  and could be viewed as the
next step in the process.  Automated theorem provers could be useful tools for
helping develop its database.
Note that very long, automatically
generated proofs can make your database fat and ugly and cause Metamath's proof
verification to take a long time to run.  Unless you have a particularly good
program that generates very concise proofs, it might be best to consider the
use of automatically generated proofs as a quick-and-dirty approach, to be
manually rewritten at some later date.

The program {\sc otter}\index{otter@{\sc otter}}\footnote{\url{http://www.cs.unm.edu/\~mccune/otter/}.}, later succeeded by
prover9\index{prover9}\footnote{\url{https://www.cs.unm.edu/~mccune/mace4/}.},
have been historically influential.
The E prover\index{E prover}\footnote{\url{https://github.com/eprover/eprover}.}
is a maintained automated theorem prover
for full first-order logic with equality.
There are many other automated theorem provers as well.

If you want to combine automated theorem provers with Metamath
consider investigating
the book {\em Automated Reasoning:  Introduction and Applications}
\cite{Wos}\index{Wos, Larry}.  This book discusses
how to use {\sc otter} in a way that can
not only able to generate
relatively efficient proofs, it can even be instructed to search for
shorter proofs.  The effective use of {\sc otter} (and similar tools)
does require a certain
amount of experience, skill, and patience.  The axiom system used in the
\texttt{set.mm}\index{set theory database (\texttt{set.mm})} set theory
database can be expressed to {\sc otter} using a method described in
\cite{Megill}.\index{Megill, Norman}\footnote{To use those axioms with
{\sc otter}, they must be restated in a way that eliminates the need for
``dummy variables.''\index{dummy variable!eliminating} See the Comment
on p.~\pageref{nodd}.} When successful, this method tends to generate
short and clever proofs, but my experiments with it indicate that the
method will find proofs within a reasonable time only for relatively
easy theorems.  It is still fun to experiment with.

Reference \cite{Bledsoe}\index{Bledsoe, W. W.} surveys a number of approaches
people have explored in the field of automated theorem proving\index{automated
theorem proving}.

\subsection{Interactive Theorem Provers}\label{interactivetheoremprovers}

Finding proofs completely automatically is difficult, so there
are some interactive theorem provers that allow a human to guide the
computer to find a proof.
Examples include
HOL Light\index{HOL light}%
\footnote{\url{https://www.cl.cam.ac.uk/~jrh13/hol-light/}.},
Isabelle\index{Isabelle}%
\footnote{\url{http://www.cl.cam.ac.uk/Research/HVG/Isabelle}.},
{\sc hol}\index{hol@{\sc hol}}%
\footnote{\url{https://hol-theorem-prover.org/}.},
and
Coq\index{Coq}\footnote{\url{https://coq.inria.fr/}.}.

A major difference between most of these tools and Metamath is that the
``proofs'' are actually programs that guide the program to find a proof,
and not the proof itself.
For example, an Isabelle/HOL proof might apply a step
\texttt{apply (blast dest: rearrange reduction)}. The \texttt{blast}
instruction applies
an automatic tableux prover and returns if it found a sequence of proof
steps that work... but the sequence is not considered part of the proof.

A good overview of
higher-level proof verification languages (such as {\sc lcf}\index{lcf@{\sc
lcf}} and {\sc hol}\index{hol@{\sc hol}})
is given in \cite{Harrison}.  All of these languages are fundamentally
different from Metamath in that much of the mathematical foundational
knowledge is embedded in the underlying proof-verification program, rather
than placed directly in the database that is being verified.
These can have a steep learning curve for those without a mathematical
background.  For example, one usually must have a fair understanding of
mathematical logic in order to follow their proofs.

\subsection{Proof Verifiers}\label{proofverifiers}

A proof verifier is a program that doesn't generate proofs but instead
verifies proofs that you give it.  Many proof verifiers have limited built-in
automated proof capabilities, such as figuring out simple logical inferences
(while still being guided by a person who provides the overall proof).  Metamath
has no built-in automated proof capability other than the limited
capability of its Proof Assistant.

Proof-verification languages are not used as frequently as they might be.
Pure mathematicians are more concerned with producing new results, and such
detail and rigor would interfere with that goal.  The use of computers in pure
mathematics is primarily focused on automated theorem provers (not verifiers),
again with the ultimate goal of aiding the creation of new mathematics.
Automated theorem provers are usually concerned with attacking one theorem at
time rather than making a large, organized database easily available to the
user.  Metamath is one way to help close this gap.

By itself Metamath is a mostly a proof verifier.
This does not mean that other approaches can't be used; the difference
is that in Metamath, the results of various provers must be recorded
step-by-step so that they can be verified.

Another proof-verification language is Mizar,\index{Mizar} which can display
its proofs in the informal language that mathematicians are accustomed to.
Information on the Mizar language is available at \url{http://mizar.org}.

For the working mathematician, Mizar is an excellent tool for rigorously
documenting proofs. Mizar typesets its proofs in the informal English used by
mathematicians (and, while fine for them, are just as inscrutable by
laypersons!). A price paid for Mizar is a relatively steep learning curve of a
couple of weeks.  Several mathematicians are actively formalizing different
areas of mathematics using Mizar and publishing the proofs in a dedicated
journal. Unfortunately the task of formalizing mathematics is still looked
down upon to a certain extent since it doesn't involve the creation of ``new''
mathematics.

The closest system to Metamath is
the {\em Ghilbert}\index{Ghilbert} proof language (\url{http://ghilbert.org})
system developed by
Raph Levien\index{Levien, Raph}.
Ghilbert is a formal proof checker heavily inspired by Metamath.
Ghilbert statements are s-expressions (a la Lisp), which is easy
for computers to parse but many people find them hard to read.
There are a number of differences in their specific constructs, but
there is at least one tool to translate some Metamath materials into Ghilbert.
As of 2019 the Ghilbert community is smaller and less active than the
Metamath community.
That said, the Metamath and Ghilbert communities overlap, and fruitful
conversations between them have occurred many times over the years.

\subsection{Creating a Database of Formalized Mathematics}\label{mathdatabase}

Besides Metamath, there are several other ongoing projects with the goal of
formalizing mathematics into computer-verifiable databases.
Understanding some history will help.

The {\sc qed}\index{qed project@{\sc qed} project}%
\footnote{\url{http://www-unix.mcs.anl.gov/qed}.}
project arose in 1993 and its goals were outlined in the
{\sc qed} manifesto.
The {\sc qed} manifesto was
a proposal for a computer-based database of all mathematical knowledge,
strictly formalized and with all proofs having been checked automatically.
The project had a conference in 1994 and another in 1995;
there was also a ``twenty years of the {\sc qed} manifesto'' workshop
in 2014.
Its ideals are regularly reraised.

In a 2007 paper, Freek Wiedijk identified two reasons
for the failure of the {\sc qed} project as originally envisioned:%
\cite{Wiedijk-revisited}\index{Wiedijk, Freek}

\begin{itemize}
\item Very few people are working on formalization of mathematics. There is no compelling application for fully mechanized mathematics.
\item Formalized mathematics does not yet resemble traditional mathematics. This is partly due to the complexity of mathematical notation, and partly to the limitations of existing theorem provers and proof assistants.
\end{itemize}

But this did not end the dream of
formalizing mathematics into computer-verifiable databases.
The problems that led to the {\sc qed} manifesto are still with us,
even though the challenges were harder than originally considered.
What has happened instead is that various independent projects have
worked towards formalizing mathematics into computer-verifiable databases,
each simultaneously competing and cooperating with each other.

A concrete way to see this is
Freek Wiedijk's ``Formalizing 100 Theorems'' list%
\footnote{\url{http://www.cs.ru.nl/\%7Efreek/100/}.}
which shows the progress different systems have made on a challenge list
of 100 mathematical theorems.%
\footnote{ This is not the only list of ``interesting'' theorems.
Another interesting list was posted by Oliver Knill's list
\cite{Knill}\index{Knill, Oliver}.}
The top systems as of February 2019
(in order of the number of challenges completed) are
HOL Light, Isabelle, Metamath, Coq, and Mizar.

The Metamath 100%
\footnote{\url{http://us.metamath.org/mm\_100.html}}
page (maintained by David A. Wheeler\index{Wheeler, David A.})
shows the progress of Metamath (specifically its \texttt{set.mm} database)
against this challenge list maintained by Freek Wiedijk.
The Metamath \texttt{set.mm} database
has made a lot of progress over the years,
in part because working to prove those challenge theorems required
defining various terms and proving their properties as a prerequisite.
Here are just a few of the many statements that have been
formally proven with Metamath:

% The entries of this cause the narrow display to break poorly,
% since the short amount of text means LaTeX doesn't get a lot to work with
% and the itemize format gives it even *less* margin than usual.
% No one will mind if we make just this list flushleft, since this list
% will be internally consistent.
\begin{flushleft}
\begin{itemize}
\item 1. The Irrationality of the Square Root of 2
  (\texttt{sqr2irr}, by Norman Megill, 2001-08-20)
\item 2. The Fundamental Theorem of Algebra
  (\texttt{fta}, by Mario Carneiro, 2014-09-15)
\item 22. The Non-Denumerability of the Continuum
  (\texttt{ruc}, by Norman Megill, 2004-08-13)
\item 54. The Konigsberg Bridge Problem
  (\texttt{konigsberg}, by Mario Carneiro, 2015-04-16)
\item 83. The Friendship Theorem
  (\texttt{friendship}, by Alexander W. van der Vekens, 2018-10-09)
\end{itemize}
\end{flushleft}

We thank all of those who have developed at least one of the Metamath 100
proofs, and we particularly thank
Mario Carneiro\index{Carneiro, Mario}
who has contributed the most Metamath 100 proofs as of 2019.
The Metamath 100 page shows the list of all people who have contributed a
proof, and links to graphs and charts showing progress over time.
We encourage others to work on proving theorems not yet proven in Metamath,
since doing so improves the work as a whole.

Each of the math formalization systems (including Metamath)
has different strengths and weaknesses, depending on what you value.
Key aspects that differentiate Metamath from the other top systems are:

\begin{itemize}
\item Metamath is not tied to any particular set of axioms.
\item Metamath can show every step of every proof, no exceptions.
  Most other provers only assert that a proof can be found, and do not
  show every step. This also makes verification fast, because
  the system does not need to rediscover proof details.
\item The Metamath verifier has been re-implemented in many different
  programming languages, so verification can be done by multiple
  implementations.  In particular, the
  \texttt{set.mm}\index{set theory database (\texttt{set.mm})}%
  \index{Metamath Proof Explorer} database is verified by
  four different verifiers
  written in four different languages by four different authors.
  This greatly reduces the risk of accepting an invalid
  proof due to an error in the verifier.
\item Proofs stay proven.  In some systems, changes to the system's
  syntax or how a tactic works causes proofs to fail in later versions,
  causing older work to become essentially lost.
  Metamath's language is
  extremely small and fixed, so once a proof is added to a database,
  the database can be rechecked with later versions of the Metamath program
  and with other verifiers of Metamath databases.
  If an axiom or key definition needs to be changed, it is easy to
  manipulate the database as a whole to handle the change
  without touching the underlying verifier.
  Since re-verification of an entire database takes seconds, there
  is never a reason to delay complete verification.
  This aspect is especially compelling if your
  goal is to have a long-term database of proofs.
\item Licensing is generous.  The main Metamath databases are released to
  the public domain, and the main Metamath program is open source software
  under a standard, widely-used license.
\item Substitutions are easy to understand, even by those who are not
  professional mathematicians.
\end{itemize}

Of course, other systems may have advantages over Metamath
that are more compelling, depending on what you value.
In any case, we hope this helps you understand Metamath
within a wider context.

\subsection{In Summary}\label{computers-summary}

To summarize our discussions of computers and mathematics, computer algebra
systems can be viewed as theorem generators focusing on a narrow realm of
mathematics (numbers and their properties), automated theorem provers as proof
generators for specific theorems in a much broader realm covered by a built-in
formal system such as first-order logic, interactive theorem
provers require human guidance, proof verifiers verify proofs but
historically they have been
restricted to first-order logic.
Metamath, in contrast,
is a proof verifier and documenter whose realm is essentially unlimited.

\section{Mathematics and Metamath}

\subsection{Standard Mathematics}

There are a number of ways that Metamath\index{Metamath} can be used with
standard mathematics.  The most satisfying way philosophically is to start at
the very beginning, and develop the desired mathematics from the axioms of
logic and set theory.\index{set theory}  This is the approach taken in the
\texttt{set.mm}\index{set theory database (\texttt{set.mm})}%
\index{Metamath Proof Explorer}
database (also known as the Metamath Proof Explorer).
Among other things, this database builds up to the
axioms of real and complex numbers\index{analysis}\index{real and complex
numbers} (see Section~\ref{real}), and a standard development of analysis, for
example, could start at that point, using it as a basis.   Besides this
philosophical advantage, there are practical advantages to having all of the
tools of set theory available in the supporting infrastructure.

On the other hand, you may wish to start with the standard axioms of a
mathematical theory without going through the set theoretical proofs of those
axioms.  You will need mathematical logic to make inferences, but if you wish
you can simply introduce theorems\index{theorem} of logic as
``axioms''\index{axiom} wherever you need them, with the implicit assumption
that in principle they can be proved, if they are obvious to you.  If you
choose this approach, you will probably want to review the notation used in
\texttt{set.mm}\index{set theory database (\texttt{set.mm})} so that your own
notation will be consistent with it.

\subsection{Other Formal Systems}
\index{formal system}

Unlike some programs, Metamath\index{Metamath} is not limited to any specific
area of mathematics, nor committed to any particular mathematical philosophy
such as classical logic versus intuitionism, nor limited, say, to expressions
in first-order logic.  Although the database \texttt{set.mm}
describes standard logic and set theory, Meta\-math
is actually a general-purpose language for describing a wide variety of formal
systems.\index{formal system}  Non-standard systems such as modal
logic,\index{modal logic} intuitionist logic\index{intuitionism}, higher-order
logic\index{higher-order logic}, quantum logic\index{quantum logic}, and
category theory\index{category theory} can all be described with the Metamath
language.  You define the symbols you prefer and tell Metamath the axioms and
rules you want to start from, and Metamath will verify any inferences you make
from those axioms and rules.  A simple example of a non-standard formal system
is Hofstadter's\index{Hofstadter, Douglas R.} MIU system,\index{MIU-system}
whose Metamath description is presented in Appendix~\ref{MIU}.

This is not hypothetical.
The largest Metamath database is
\texttt{set.mm}\index{set theory database (\texttt{set.mm}}%
\index{Metamath Proof Explorer}), aka the Metamath Proof Explorer,
which uses the most common axioms for mathematical foundations
(specifically classical logic combined with Zermelo--Fraenkel
set theory\index{Zermelo--Fraenkel set theory} with the Axiom of Choice).
But other Metamath databases are available:

\begin{itemize}
\item The database
  \texttt{iset.mm}\index{intuitionistic logic database (\texttt{iset.mm})},
  aka the
  Intuitionistic Logic Explorer\index{Intuitionistic Logic Explorer},
  uses intuitionistic logic (a constructivist point of view)
  instead of classical logic.
\item The database
  \texttt{nf.mm}\index{New Foundations database (\texttt{nf.mm})},
  aka the
  New Foundations Explorer\index{New Foundations Explorer},
  constructs mathematics from scratch,
  starting from Quine's New Foundations (NF) set theory axioms.
\item The database
  \texttt{hol.mm}\index{Higher-order Logic database (\texttt{hol.mm})},
  aka the
  Higher-Order Logic (HOL) Explorer\index{Higher-Order Logic (HOL) Explorer},
  starts with HOL (also called simple type theory) and derives
  equivalents to ZFC axioms, connecting the two approaches.
\end{itemize}

Since the days of David Hilbert,\index{Hilbert, David} mathematicians have
been concerned with the fact that the metalanguage\index{metalanguage} used to
describe mathematics may be stronger than the mathematics being described.
Metamath\index{Metamath}'s underlying finitary\index{finitary proof},
constructive nature provides a good philosophical basis for studying even the
weakest logics.\index{weak logic}

The usual treatment of many non-standard formal systems\index{formal
system} uses model theory\index{model theory} or proof theory\index{proof
theory} to describe these systems; these theories, in turn, are based on
standard set theory.  In other words, a non-standard formal system is defined
as a set with certain properties, and standard set theory is used to derive
additional properties of this set.  The standard set theory database provided
with Metamath can be used for this purpose, and when used this way
the development of a special
axiom system for the non-standard formal system becomes unnecessary.  The
model- or proof-theoretic approach often allows you to prove much deeper
results with less effort.

Metamath supports both approaches.  You can define the non-standard
formal system directly, or define the non-standard formal system as
a set with certain properties, whichever you find most helpful.

%\section{Additional Remarks}

\subsection{Metamath and Its Philosophy}

Closely related to Metamath\index{Metamath} is a philosophy or way of looking
at mathematics. This philosophy is related to the formalist
philosophy\index{formalism} of Hilbert\index{Hilbert, David} and his followers
\cite[pp.~1203--1208]{Kline}\index{Kline, Morris}
\cite[p.~6]{Behnke}\index{Behnke, H.}. In this philosophy, mathematics is
viewed as nothing more than a set of rules that manipulate symbols, together
with the consequences of those rules.  While the mathematics being described
may be complex, the rules used to describe it (the
``metamathematics''\index{metamathematics}) should be as simple as possible.
In particular, proofs should be restricted to dealing with concrete objects
(the symbols we write on paper rather than the abstract concepts they
represent) in a constructive manner; these are called ``finitary''
proofs\index{finitary proof} \cite[pp.~2--3]{Shoenfield}\index{Shoenfield,
Joseph R.}.

Whether or not you find Metamath interesting or useful will in part depend on
the appeal you find in its philosophy, and this appeal will probably depend on
your particular goals with respect to mathematics.  For example, if you are a
pure mathematician at the forefront of discovering new mathematical knowledge,
you will probably find that the rigid formality of Metamath stifles your
creativity.  On the other hand, we would argue that once this knowledge is
discovered, there are advantages to documenting it in a standard format that
will make it accessible to others.  Sixty years from now, your field may be
dormant, and as Davis and Hersh put it, your ``writings would become less
translatable than those of the Maya'' \cite[p.~37]{Davis}\index{Davis, Phillip
J.}.


\subsection{A History of the Approach Behind Metamath}

Probably the one work that has had the most motivating influence on
Metamath\index{Metamath} is Whitehead and Russell's monumental {\em Principia
Mathematica} \cite{PM}\index{Whitehead, Alfred North}\index{Russell,
Bertrand}\index{principia mathematica@{\em Principia Mathematica}}, whose aim
was to deduce all of mathematics from a small number of primitive ideas, in a
very explicit way that in principle anyone could understand and follow.  While
this work was tremendously influential in its time, from a modern perspective
it suffers from several drawbacks.  Both its notation and its underlying
axioms are now considered dated and are no longer used.  From our point of
view, its development is not really as accessible as we would like to see; for
practical reasons, proofs become more and more sketchy as its mathematics
progresses, and working them out in fine detail requires a degree of
mathematical skill and patience that many people don't have.  There are
numerous small errors, which is understandable given the tedious, technical
nature of the proofs and the lack of a computer to verify the details.
However, even today {\em Principia Mathematica} stands out as the work closest
in spirit to Metamath.  It remains a mind-boggling work, and one can't help
but be amazed at seeing ``$1+1=2$'' finally appear on page 83 of Volume II
(Theorem *110.643).

The origin of the proof notation used by Metamath dates back to the 1950's,
when the logician C.~A.~Meredith expressed his proofs in a compact notation
called ``condensed detachment''\index{condensed detachment}
\cite{Hindley}\index{Hindley, J. Roger} \cite{Kalman}\index{Kalman, J. A.}
\cite{Meredith}\index{Meredith, C. A.} \cite{Peterson}\index{Peterson, Jeremy
George}.  This notation allows proofs to be communicated unambiguously by
merely referencing the axiom\index{axiom}, rule\index{rule}, or
theorem\index{theorem} used at each step, without explicitly indicating the
substitutions\index{substitution!variable}\index{variable substitution} that
have to be made to the variables in that axiom, rule, or theorem.  Ordinarily,
condensed detachment is more or less limited to propositional
calculus\index{propositional calculus}.  The concept has been extended to
first-order logic\index{first-order logic} in \cite{Megill}\index{Megill,
Norman}, making it is easy to write a small computer program to verify proofs
of simple first-order logic theorems.\index{condensed detachment!and
first-order logic}

A key concept behind the notation of condensed detachment is called
``unification,''\index{unification} which is an algorithm for determining what
substitutions\index{substitution!variable}\index{variable substitution} to
variables have to be made to make two expressions match each other.
Unification was first precisely defined by the logician J.~A.~Robinson, who
used it in the development of a powerful
theorem-proving technique called the ``resolution principle''
\cite{Robinson}\index{Robinson's resolution principle}. Metamath does not make
use of the resolution principle, which is intended for systems of first-order
logic.\index{first-order logic}  Metamath's use is not restricted to
first-order logic, and as we have mentioned it does not automatically discover
proofs.  However, unification is a key idea behind Metamath's proof
notation, and Metamath makes use of a very simple version of it
(Section~\ref{unify}).

\subsection{Metamath and First-Order Logic}

First-order logic\index{first-order logic} is the supporting structure
for standard mathematics.  On top of it is set theory, which contains
the axioms from which virtually all of mathematics can be derived---a
remarkable fact.\index{category
theory}\index{cardinal, inaccessible}\label{categoryth}\footnote{An exception seems
to be category theory.  There are several schools of thought on whether
category theory is derivable from set theory.  At a minimum, it appears
that an additional axiom is needed that asserts the existence of an
``inaccessible cardinal'' (a type of infinity so large that standard set
theory can't prove or deny that it exists).
%
%%%% (I took this out that was in previous editions:)
% But it is also argued that not just one but a ``proper class'' of them
% is needed, and the existence of proper classes is impossible in standard
% set theory.  (A proper class is a collection of sets so huge that no set
% can contain it as an element.  Proper classes can lead to
% inconsistencies such as ``Russell's paradox.''  The axioms of standard
% set theory are devised so as to deny the existence of proper classes.)
%
For more information, see
\cite[pp.~328--331]{Herrlich}\index{Herrlich, Horst} and
\cite{Blass}\index{Blass, Andrea}.}

One of the things that makes Metamath\index{Metamath} more practical for
first-order theories is a set of axioms for first-order logic designed
specifically with Metamath's approach in mind.  These are included in
the database \texttt{set.mm}\index{set theory database (\texttt{set.mm})}.
See Chapter~\ref{fol} for a detailed
description; the axioms are shown in Section~\ref{metaaxioms}.  While
logically equivalent to standard axiom systems, our axiom system breaks
up the standard axioms into smaller pieces such that from them, you can
directly derive what in other systems can only be derived as higher-level
``metatheorems.''\index{metatheorem}  In other words, it is more powerful than
the standard axioms from a metalogical point of view.  A rigorous
justification for this system and its ``metalogical
completeness''\index{metalogical completeness} is found in
\cite{Megill}\index{Megill, Norman}.  The system is closely related to a
system developed by Monk\index{Monk, J. Donald} and Tarski\index{Tarski,
Alfred} in 1965 \cite{Monks}.

For example, the formula $\exists x \, x = y $ (given $y$, there exists some
$x$ equal to it) is a theorem of logic,\footnote{Specifically, it is a theorem
of those systems of logic that assume non-empty domains.  It is not a theorem
of more general systems that include the empty domain\index{empty domain}, in
which nothing exists, period!  Such systems are called ``free
logics.''\index{free logic} For a discussion of these systems, see
\cite{Leblanc}\index{Leblanc, Hugues}.  Since our use for logic is as a basis
for set theory, which has a non-empty domain, it is more convenient (and more
traditional) to use a less general system.  An interesting curiosity is that,
using a free logic as a basis for Zermelo--Fraenkel set
theory\index{Zermelo--Fraenkel set theory} (with the redundant Axiom of the
Null Set omitted),\index{Axiom of the Null Set} we cannot even prove the
existence of a single set without assuming the axiom of infinity!\index{Axiom
of Infinity}} whether or not $x$ and $y$ are distinct variables\index{distinct
variables}.  In many systems of logic, we would have to prove two theorems to
arrive at this result.  First we would prove ``$\exists x \, x = x $,'' then
we would separately prove ``$\exists x \, x = y $, where $x$ and $y$ are
distinct variables.''  We would then combine these two special cases ``outside
of the system'' (i.e.\ in our heads) to be able to claim, ``$\exists x \, x =
y $, regardless of whether $x$ and $y$ are distinct.''  In other words, the
combination of the two special cases is a
metatheorem.  In the system of logic
used in Metamath's set theory\index{set theory database (\texttt{set.mm})}
database, the axioms of logic are broken down into small pieces that allow
them to be reassembled in such a way that theorems such as these can be proved
directly.

Breaking down the axioms in this way makes them look peculiar and not very
intuitive at first, but rest assured that they are correct and complete.  Their
correctness is ensured because they are theorem schemes of standard first-order
logic (which you can easily verify if you are a logician).  Their completeness
follows from the fact that we explicitly derive the standard axioms of
first-order logic as theorems.  Deriving the standard axioms is somewhat
tricky, but once we're there, we have at our disposal a system that is less
awkward to work with in formal proofs\index{formal proof}.  In technical terms
that logicians understand, we eliminate the cumbersome concepts of ``free
variable,''\index{free variable} ``bound variable,''\index{bound variable} and
``proper substitution''\index{proper substitution}\index{substitution!proper}
as primitive notions.  These concepts are present in our system but are
defined in terms of concepts expressed by the axioms and can be eliminated in
principle.  In standard systems, these concepts are really like additional,
implicit axioms\index{implicit axiom} that are somewhat complex and cannot be
eliminated.

The traditional approach to logic, wherein free variables and proper
substitution is defined, is also possible to do directly in the Metamath
language.  However, the notation tends to become awkward, and there are
disadvantages:  for example, extending the definition of a wff with a
definition is awkward, because the free variable and proper substitution
concepts have to have their definitions also extended.  Our choice of
axioms for \texttt{set.mm} is to a certain extent a matter of style, in
an attempt to achieve overall simplicity, but you should also be aware
that the traditional approach is possible as well if you should choose
it.

\chapter{Using the Metamath Program}
\label{using}

\section{Installation}

The way that you install Metamath\index{Metamath!installation} on your
computer system will vary for different computers.  Current
instructions are provided with the Metamath program download at
\url{http://metamath.org}.  In general, the installation is simple.
There is one file containing the Metamath program itself.  This file is
usually called \texttt{metamath} or \texttt{metamath.}{\em xxx} where
{\em xxx} is the convention (such as \texttt{exe}) for an executable
program on your operating system.  There are several additional files
containing samples of the Metamath language, all ending with
\texttt{.mm}.  The file \texttt{set.mm}\index{set theory database
(\texttt{set.mm})} contains logic and set theory and can be used as a
starting point for other areas of mathematics.

You will also need a text editor\index{text editor} capable of editing plain
{\sc ascii}\footnote{American Standard Code for Information Interchange.} text
in order to prepare your input files.\index{ascii@{\sc ascii}}  Most computers
have this capability built in.  Note that plain text is not necessarily the
default for some word processing programs\index{word processor}, especially if
they can handle different fonts; for example, with Microsoft Word\index{Word
(Microsoft)}, you must save the file in the format ``Text Only With Line
Breaks'' to get a plain text\index{plain text} file.\footnote{It is
recommended that all lines in a Metamath source file be 79 characters or less
in length for compatibility among different computer terminals.  When creating
a source file on an editor such as Word, select a monospaced
font\index{monospaced font} such as Courier\index{Courier font} or
Monaco\index{Monaco font} to make this easier to achieve.  Better yet,
just use a plain text editor such as Notepad.}

On some computer systems, Metamath does not have the capability to print
its output directly; instead, you send its output to a file (using the
\texttt{open} commands described later).  The way you print this output
file depends on your computer.\index{printers} Some computers have a
print command, whereas with others, you may have to read the file into
an editor and print it from there.

If you want to print your Metamath source files with typeset formulas
containing standard mathematical symbols, you will need the \LaTeX\
typesetting program\index{latex@{\LaTeX}}, which is widely and freely
available for most operating systems.  It runs natively on Unix and
Linux, and can be installed on Windows as part of the free Cygwin
package (\url{http://cygwin.com}).

You can also produce {\sc html}\footnote{HyperText Markup Language.}
web pages.  The {\tt help html} command in the Metamath program will
assist you with this feature.

\section{Your First Formal System}\label{start}
\subsection{From Nothing to Zero}\label{startf}

To give you a feel for what the Metamath\index{Metamath} language looks like,
we will take a look at a very simple example from formal number
theory\index{number theory}.  This example is taken from
Mendelson\index{Mendelson, Elliot} \cite[p. 123]{Mendelson}.\footnote{To keep
the example simple, we have changed the formalism slightly, and what we call
axioms\index{axiom} are strictly speaking theorems\index{theorem} in
\cite{Mendelson}.}  We will look at a small subset of this theory, namely that
part needed for the first number theory theorem proved in \cite{Mendelson}.

First we will look at a standard formal proof\index{formal proof} for the
example we have picked, then we will look at the Metamath version.  If you
have never been exposed to formal proofs, the notation may seem to be such
overkill to express such simple notions that you may wonder if you are missing
something.  You aren't.  The concepts involved are in fact very simple, and a
detailed breakdown in this fashion is necessary to express the proof in a way
that can be verified mechanically.  And as you will see, Metamath breaks the
proof down into even finer pieces so that the mechanical verification process
can be about as simple as possible.

Before we can introduce the axioms\index{axiom} of the theory, we must define
the syntax rules for forming legal expressions\index{syntax rules}
(combinations of symbols) with which those axioms can be used. The number 0 is
a {\bf term}\index{term}; and if $ t$ and $r$ are terms, so is $(t+r)$. Here,
$ t$ and $r$ are ``metavariables''\index{metavariable} ranging over terms; they
themselves do not appear as symbols in an actual term.  Some examples of
actual terms are $(0 + 0)$ and $((0+0)+0)$.  (Note that our theory describes
only the number zero and sums of zeroes.  Of course, not much can be done with
such a trivial theory, but remember that we have picked a very small subset of
complete number theory for our example.  The important thing for you to focus
on is our definitions that describe how symbols are combined to form valid
expressions, and not on the content or meaning of those expressions.) If $ t$
and $r$ are terms, an expression of the form $ t=r$ is a {\bf wff}
(well-formed formula)\index{well-formed formula (wff)}; and if $P$ and $Q$ are
wffs, so is $(P\rightarrow Q)$ (which means ``$P$ implies
$Q$''\index{implication ($\rightarrow$)} or ``if $P$ then $Q$'').
Here $P$ and $Q$ are metavariables ranging over wffs.  Examples of actual
wffs are $0=0$, $(0+0)=0$, $(0=0 \rightarrow (0+0)=0)$, and $(0=0\rightarrow
(0=0\rightarrow 0=(0+0)))$.  (Our notation makes use of more parentheses than
are customary, but the elimination of ambiguity this way simplifies our
example by avoiding the need to define operator precedence\index{operator
precedence}.)

The {\bf axioms}\index{axiom} of our theory are all wffs of the following
form, where $ t$, $r$, and $s$ are any terms:

%Latex p. 92
\renewcommand{\theequation}{A\arabic{equation}}

\begin{equation}
(t=r\rightarrow (t=s\rightarrow r=s))
\end{equation}
\begin{equation}
(t+0)=t
\end{equation}

Note that there are an infinite number of axioms since there are an infinite
number of possible terms.  A1 and A2 are properly called ``axiom
schemes,''\index{axiom scheme} but we will refer to them as ``axioms'' for
brevity.

An axiom is a {\bf theorem}; and if $P$ and $(P\rightarrow Q)$ are theorems
(where $P$ and $Q$ are wffs), then $Q$ is also a theorem.\index{theorem}  The
second part of this definition is called the modus ponens (MP) rule of
inference\index{inference rule}\index{modus ponens}.  It allows us to obtain
new theorems from old ones.

The {\bf proof}\index{proof} of a theorem is a sequence of one or more
theorems, each of which is either an axiom or the result of modus ponens
applied to two previous theorems in the sequence, and the last of which is the
theorem being proved.

The theorem we will prove for our example is very simple:  $ t=t$.  The proof of
our theorem follows.  Study it carefully until you feel sure you
understand it.\label{zeroproof}

% Use tabu so that lines will wrap automatically as needed.
\begin{tabu} { l X X }
1. & $(t+0)=t$ & (by axiom A2) \\
2. & $(t+0)=t$ & (by axiom A2) \\
3. & $((t+0)=t \rightarrow ((t+0)=t\rightarrow t=t))$ & (by axiom A1) \\
4. & $((t+0)=t\rightarrow t=t)$ & (by MP applied to steps 2 and 3) \\
5. & $t=t$ & (by MP applied to steps 1 and 4) \\
\end{tabu}

(You may wonder why step 1 is repeated twice.  This is not necessary in the
formal language we have defined, but in Metamath's ``reverse Polish
notation''\index{reverse Polish notation (RPN)} for proofs, a previous step
can be referred to only once.  The repetition of step~1 here will enable you
to see more clearly the correspondence of this proof with the
Metamath\index{Metamath} version on p.~\pageref{demoproof}.)

Our theorem is more properly called a ``theorem scheme,''\index{theorem
scheme} for it represents an infinite number of theorems, one for each
possible term $ t$.  Two examples of actual theorems would be $0=0$ and
$(0+0)=(0+0)$.  Rarely do we prove actual theorems, since by proving schemes
we can prove an infinite number of theorems in one fell swoop.  Similarly, our
proof should really be called a ``proof scheme.''\index{proof scheme}  To
obtain an actual proof, pick an actual term to use in place of $ t$, and
substitute it for $ t$ throughout the proof.

Let's discuss what we have done here.  The axioms\index{axiom} of our theory,
A1 and A2, are trivial and obvious.  Everyone knows that adding zero to
something doesn't change it, and also that if two things are equal to a third,
then they are equal to each other. In fact, stating the trivial and obvious is
a goal to strive for in any axiomatic system.  From trivial and obvious truths
that everyone agrees upon, we can prove results that are not so obvious yet
have absolute faith in them.  If we trust the axioms and the rules, we must,
by definition, trust the consequences of those axioms and rules, if logic is
to mean anything at all.

Our rule of inference\index{rule}, modus ponens\index{modus ponens}, is also
pretty obvious once you understand what it means.  If we prove a fact $P$, and
we also prove that $P$ implies $Q$, then $Q$ necessarily follows as a new
fact.  The rule provides us with a means for obtaining new facts (i.e.\
theorems\index{theorem}) from old ones.

The theorem that we have proved, $ t=t$, is so fundamental that you may wonder
why it isn't one of the axioms\index{axiom}.  In some axiom systems of
arithmetic, it {\em is} an axiom.  The choice of axioms in a theory is to some
extent arbitrary and even an art form, constrained only by the requirement
that any two equivalent axiom systems be able to derive each other as
theorems.  We could imagine that the inventor of our axiom system originally
included $ t=t$ as an axiom, then discovered that it could be derived as a
theorem from the other axioms.  Because of this, it was not necessary to
keep it as an axiom.  By eliminating it, the final set of axioms became
that much simpler.

Unless you have worked with formal proofs\index{formal proof} before, it
probably wasn't apparent to you that $ t=t$ could be derived from our two
axioms until you saw the proof. While you certainly believe that $ t=t$ is
true, you might not be able to convince an imaginary skeptic who believes only
in our two axioms until you produce the proof.  Formal proofs such as this are
hard to come up with when you first start working with them, but after you get
used to them they can become interesting and fun.  Once you understand the
idea behind formal proofs you will have grasped the fundamental principle that
underlies all of mathematics.  As the mathematics becomes more sophisticated,
its proofs become more challenging, but ultimately they all can be broken down
into individual steps as simple as the ones in our proof above.

Mendelson's\index{Mendelson, Elliot} book, from which our example was taken,
contains a number of detailed formal proofs such as these, and you may be
interested in looking it up.  The book is intended for mathematicians,
however, and most of it is rather advanced.  Popular literature describing
formal proofs\index{formal proof} include \cite[p.~296]{Rucker}\index{Rucker,
Rudy} and \cite[pp.~204--230]{Hofstadter}\index{Hofstadter, Douglas R.}.

\subsection{Converting It to Metamath}\label{convert}

Formal proofs\index{formal proof} such as the one in our example break down
logical reasoning into small, precise steps that leave little doubt that the
results follow from the axioms\index{axiom}.  You might think that this would
be the finest breakdown we can achieve in mathematics.  However, there is more
to the proof than meets the eye. Although our axioms were rather simple, a lot
of verbiage was needed before we could even state them:  we needed to define
``term,'' ``wff,'' and so on.  In addition, there are a number of implied
rules that we haven't even mentioned. For example, how do we know that step 3
of our proof follows from axiom A1? There is some hidden reasoning involved in
determining this.  Axiom A1 has two occurrences of the letter $ t$.  One of
the implied rules states that whatever we substitute for $ t$ must be a legal
term\index{term}.\footnote{Some authors make this implied rule explicit by
stating, ``only expressions of the above form are terms,'' after defining
``term.''}  The expression $ t+0$ is pretty obviously a legal term whenever $
t$ is, but suppose we wanted to substitute a huge term with thousands of
symbols?  Certainly a lot of work would be involved in determining that it
really is a term, but in ordinary formal proofs all of this work would be
considered a single ``step.''

To express our axiom system in the Metamath\index{Metamath} language, we must
describe this auxiliary information in addition to the axioms themselves.
Metamath does not know what a ``term'' or a ``wff''\index{well-formed formula
(wff)} is.  In Metamath, the specification of the ways in which we can combine
symbols to obtain terms and wffs are like little axioms in themselves.  These
auxiliary axioms are expressed in the same notation as the ``real''
axioms\index{axiom}, and Metamath does not distinguish between the two.  The
distinction is made by you, i.e.\ by the way in which you interpret the
notation you have chosen to express these two kinds of axioms.

The Metamath language breaks down mathematical proofs into tiny pieces, much
more so than in ordinary formal proofs\index{formal proof}.  If a single
step\index{proof step} involves the
substitution\index{substitution!variable}\index{variable substitution} of a
complex term for one of its variables, Metamath must see this single step
broken down into many small steps.  This fine-grained breakdown is what gives
Metamath generality and flexibility as it lets it not be limited to any
particular mathematical notation.

Metamath's proof notation is not, in itself, intended to be read by humans but
rather is in a compact format intended for a machine.  The Metamath program
will convert this notation to a form you can understand, using the \texttt{show
proof}\index{\texttt{show proof} command} command.  You can tell the program what
level of detail of the proof you want to look at.  You may want to look at
just the logical inference steps that correspond
to ordinary formal proof steps,
or you may want to see the fine-grained steps that prove that an expression is
a term.

Here, without further ado, is our example converted to the
Metamath\index{Metamath} language:\index{metavariable}\label{demo0}

\begin{verbatim}
$( Declare the constant symbols we will use $)
    $c 0 + = -> ( ) term wff |- $.
$( Declare the metavariables we will use $)
    $v t r s P Q $.
$( Specify properties of the metavariables $)
    tt $f term t $.
    tr $f term r $.
    ts $f term s $.
    wp $f wff P $.
    wq $f wff Q $.
$( Define "term" and "wff" $)
    tze $a term 0 $.
    tpl $a term ( t + r ) $.
    weq $a wff t = r $.
    wim $a wff ( P -> Q ) $.
$( State the axioms $)
    a1 $a |- ( t = r -> ( t = s -> r = s ) ) $.
    a2 $a |- ( t + 0 ) = t $.
$( Define the modus ponens inference rule $)
    ${
       min $e |- P $.
       maj $e |- ( P -> Q ) $.
       mp  $a |- Q $.
    $}
$( Prove a theorem $)
    th1 $p |- t = t $=
  $( Here is its proof: $)
       tt tze tpl tt weq tt tt weq tt a2 tt tze tpl
       tt weq tt tze tpl tt weq tt tt weq wim tt a2
       tt tze tpl tt tt a1 mp mp
     $.
\end{verbatim}\index{metavariable}

A ``database''\index{database} is a set of one or more {\sc ascii} source
files.  Here's a brief description of this Metamath\index{Metamath} database
(which consists of this single source file), so that you can understand in
general terms what is going on.  To understand the source file in detail, you
should read Chapter~\ref{languagespec}.

The database is a sequence of ``tokens,''\index{token} which are normally
separated by spaces or line breaks.  The only tokens that are built into
the Metamath language are those beginning with \texttt{\$}.  These tokens
are called ``keywords.''\index{keyword}  All other tokens are
user-defined, and their names are arbitrary.

As you might have guessed, the Metamath token \texttt{\$(}\index{\texttt{\$(} and
\texttt{\$)} auxiliary keywords} starts a comment and \texttt{\$)} ends a comment.

The Metamath tokens \texttt{\$c}\index{\texttt{\$c} statement},
\texttt{\$v}\index{\texttt{\$v} statement},
\texttt{\$e}\index{\texttt{\$e} statement},
\texttt{\$f}\index{\texttt{\$f} statement},
\texttt{\$a}\index{\texttt{\$a} statement}, and
\texttt{\$p}\index{\texttt{\$p} statement} specify ``statements'' that
end with \texttt{\$.}\,.\index{\texttt{\$.}\ keyword}

The Metamath tokens \texttt{\$c} and \texttt{\$v} each declare\index{constant
declaration}\index{variable declaration} a list of user-defined tokens, called
``math symbols,''\index{math symbol} that the database will reference later
on.  All of the math symbols they define you have seen earlier except the
turnstile symbol \texttt{|-} ($\vdash$)\index{turnstile ({$\,\vdash$})}, which is
commonly used by logicians to mean ``a proof exists for.''  For us
the turnstile is just a
convenient symbol that distinguishes expressions that are axioms\index{axiom}
or theorems\index{theorem} from expressions that are terms or wffs.

The \texttt{\$c} statement declares ``constants''\index{constant} and
the \texttt{\$v} statement declares
``variables''\index{variable}\index{constant declaration}\index{variable
declaration} (or more precisely, metavariables\index{metavariable}).  A
variable may be substituted\index{substitution!variable}\index{variable
substitution} with sequences of math symbols whereas a constant may not
be substituted with anything.

It may seem redundant to require both \texttt{\$c}\index{\texttt{\$c} statement} and
\texttt{\$v}\index{\texttt{\$v} statement} statements (since any math
symbol\index{math symbol} not specified with a \texttt{\$c} statement could be
presumed to be a variable), but this provides for better error checking and
also allows math symbols to be redeclared\index{redeclaration of symbols}
(Section~\ref{scoping}).

The token \texttt{\$f}\index{\texttt{\$f} statement} specifies a
statement called a ``variable-type hypothesis'' (also called a
``floating hypothesis'') and \texttt{\$e}\index{\texttt{\$e} statement}
specifies a ``logical hypothesis'' (also called an ``essential
hypothesis'').\index{hypothesis}\index{variable-type
hypothesis}\index{logical hypothesis}\index{floating
hypothesis}\index{essential hypothesis} The token
\texttt{\$a}\index{\texttt{\$a} statement} specifies an ``axiomatic
assertion,''\index{axiomatic assertion} and
\texttt{\$p}\index{\texttt{\$p} statement} specifies a ``provable
assertion.''\index{provable assertion} To the left of each occurrence of
these four tokens is a ``label''\index{label} that identifies the
hypothesis or assertion for later reference.  For example, the label of
the first axiomatic assertion is \texttt{tze}.  A \texttt{\$f} statement
must contain exactly two math symbols, a constant followed by a
variable.  The \texttt{\$e}, \texttt{\$a}, and \texttt{\$p} statements
each start with a constant followed by, in general, an arbitrary
sequence of math symbols.

Associated with each assertion\index{assertion} is a set of hypotheses
that must be satisfied in order for the assertion to be used in a proof.
These are called the ``mandatory hypotheses''\index{mandatory
hypothesis} of the assertion.  Among those hypotheses whose ``scope''
(described below) includes the assertion, \texttt{\$e} hypotheses are
always mandatory and \texttt{\$f}\index{\texttt{\$f} statement}
hypotheses are mandatory when they share their variable with the
assertion or its \texttt{\$e} hypotheses.  The exact rules for
determining which hypotheses are mandatory are described in detail in
Sections~\ref{frames} and \ref{scoping}.  For example, the mandatory
hypotheses of assertion \texttt{tpl} are \texttt{tt} and \texttt{tr},
whereas assertion \texttt{tze} has no mandatory hypotheses because it
contains no variables and has no \texttt{\$e}\index{\texttt{\$e}
statement} hypothesis.  Metamath's \texttt{show statement}
command\index{\texttt{show statement} command}, described in the next
section, will show you a statement's mandatory hypotheses.

Sometimes we need to make a hypothesis relevant to only certain
assertions.  The set of statements to which a hypothesis is relevant is
called its ``scope.''  The Metamath brackets,
\texttt{\$\char`\{}\index{\texttt{\$\char`\{} and \texttt{\$\char`\}}
keywords} and \texttt{\$\char`\}}, define a ``block''\index{block} that
delimits the scope of any hypothesis contained between them.  The
assertion \texttt{mp} has mandatory hypotheses \texttt{wp}, \texttt{wq},
\texttt{min}, and \texttt{maj}.  The only mandatory hypothesis of
\texttt{th1}, on the other hand, is \texttt{tt}, since \texttt{th1}
occurs outside of the block containing \texttt{min} and \texttt{maj}.

Note that \texttt{\$\char`\{} and \texttt{\$\char`\}} do not affect the
scope of assertions (\texttt{\$a} and \texttt{\$p}).  Assertions are always
available to be referenced by any later proof in the source file.

Each provable assertion (\texttt{\$p}\index{\texttt{\$p} statement}
statement) has two parts.  The first part is the
assertion\index{assertion} itself, which is a sequence of math
symbol\index{math symbol} tokens placed between the \texttt{\$p} token
and a \texttt{\$=}\index{\texttt{\$=} keyword} token.  The second part
is a ``proof,'' which is a list of label tokens placed between the
\texttt{\$=} token and the \texttt{\$.}\index{\texttt{\$.}\ keyword}\
token that ends the statement.\footnote{If you've looked at the
\texttt{set.mm} database, you may have noticed another notation used for
proofs.  The other notation is called ``compressed.''\index{compressed
proof}\index{proof!compressed} It reduces the amount of space needed to
store a proof in the database and is described in
Appendix~\ref{compressed}.  In the example above, we use
``normal''\index{normal proof}\index{proof!normal} notation.} The proof
acts as a series of instructions to the Metamath program, telling it how
to build up the sequence of math symbols contained in the assertion part of
the \texttt{\$p} statement, making use of the hypotheses of the
\texttt{\$p} statement and previous assertions.  The construction takes
place according to precise rules.  If the list of labels in the proof
causes these rules to be violated, or if the final sequence that results
does not match the assertion, the Metamath program will notify you with
an error message.

If you are familiar with reverse Polish notation (RPN), which is sometimes used
on pocket calculators, here in a nutshell is how a proof works.  Each
hypothesis label\index{hypothesis label} in the proof is pushed\index{push}
onto the RPN stack\index{stack}\index{RPN stack} as it is encountered. Each
assertion label\index{assertion label} pops\index{pop} off the stack as many
entries as the referenced assertion has mandatory hypotheses.  Variable
substitutions\index{substitution!variable}\index{variable substitution} are
computed which, when made to the referenced assertion's mandatory hypotheses,
cause these hypotheses to match the stack entries. These same substitutions
are then made to the variables in the referenced assertion itself, which is
then pushed onto the stack.  At the end of the proof, there should be one
stack entry, namely the assertion being proved.  This process is explained in
detail in Section~\ref{proof}.

Metamath's proof notation is not very readable for humans, but it allows the
proof to be stored compactly in a file.  The Metamath\index{Metamath} program
has proof display features that let you see what's going on in a more
readable way, as you will see in the next section.

The rules used in verifying a proof are not based on any built-in syntax of the
symbol sequence in an assertion\index{assertion} nor on any built-in meanings
attached to specific symbol names.  They are based strictly on symbol
matching:  constants\index{constant} must match themselves, and
variables\index{variable} may be replaced with anything that allows a match to
occur.  For example, instead of \texttt{term}, \texttt{0}, and \verb$|-$ we could
have just as well used \texttt{yellow}, \texttt{zero}, and \texttt{provable}, as long
as we did so consistently throughout the database.  Also, we could have used
\texttt{is provable} (two tokens) instead of \verb$|-$ (one token) throughout the
database.  In each of these cases, the proof would be exactly the same.  The
independence of proofs and notation means that you have a lot of flexibility to
change the notation you use without having to change any proofs.

\section{A Trial Run}\label{trialrun}

Now you are ready to try out the Metamath\index{Metamath} program.

On all computer systems, Metamath has a standard ``command line
interface'' (CLI)\index{command line interface (CLI)} that allows you to
interact with it.  You supply commands to the CLI by typing them on the
keyboard and pressing your keyboard's {\em return} key after each line
you enter.  The CLI is designed to be easy to use and has built-in help
features.

The first thing you should do is to use a text editor to create a file
called \texttt{demo0.mm} and type into it the Metamath source shown on
p.~\pageref{demo0}.  Actually, this file is included with your Metamath
software package, so check that first.  If you type it in, make sure
that you save it in the form of ``plain {\sc ascii} text with line
breaks.''  Most word processors will have this feature.

Next you must run the Metamath program.  Depending on your computer
system and how Metamath is installed, this could range from clicking the
mouse on the Metamath icon to typing \texttt{run metamath} to typing
simply \texttt{metamath}.  (Metamath's {\tt help invoke} command describes
alternate ways of invoking the Metamath program.)

When you first enter Metamath\index{Metamath}, it will be at the CLI, waiting
for your input. You will see something like the following on your screen:
\begin{verbatim}
Metamath - Version 0.172 25-Jan-2019
Type HELP for help, EXIT to exit.
MM>
\end{verbatim}
The \texttt{MM>} prompt means that Metamath is waiting for a command.
Command keywords\index{command keyword} are not case sensitive;
we will use lower-case commands in our examples.
The version number and its release date will probably be different on your
system from the one we show above.

The first thing that you need to do is to read in your
database:\index{\texttt{read} command}\footnote{If a directory path is
needed on Unix,\index{Unix file names}\index{file names!Unix} you should
enclose the path/file name in quotes to prevent Metamath from thinking
that the \texttt{/} in the path name is a command qualifier, e.g.,
\texttt{read \char`\"db/set.mm\char`\"}.  Quotes are optional when there
is no ambiguity.}
\begin{verbatim}
MM> read demo0.mm
\end{verbatim}
Remember to press the {\em return} key after entering this command.  If
you omit the file name, Metamath will prompt you for one.   The syntax for
specifying a Macintosh file name path is given in a footnote on
p.~\pageref{includef}.\index{Macintosh file names}\index{file
names!Macintosh}

If there are any syntax errors in the database, Metamath will let you know
when it reads in the file.  The one thing that Metamath does not check when
reading in a database is that all proofs are correct, because this would
slow it down too much.  It is a good idea to periodically verify the proofs in
a database you are making changes to.  To do this, use the following command
(and do it for your \texttt{demo0.mm} file now).  Note that the \texttt{*} is a
``wild card'' meaning all proofs in the file.\index{\texttt{verify proof} command}
\begin{verbatim}
MM> verify proof *
\end{verbatim}
Metamath will report any proofs that are incorrect.

It is often useful to save the information that the Metamath program displays
on the screen. You can save everything that happens on the screen by opening a
log file. You may want to do this before you read in a database so that you
can examine any errors later on.  To open a log file, type
\begin{verbatim}
MM> open log abc.log
\end{verbatim}
This will open a file called \texttt{abc.log}, and everything that appears on the
screen from this point on will be stored in this file.  The name of the log file
is arbitrary. To close the log file, type
\begin{verbatim}
MM> close log
\end{verbatim}

Several commands let you examine what's inside your database.
Section~\ref{exploring} has an overview of some useful ones.  The
\texttt{show labels} command lets you see what statement
labels\index{label} exist.  A \texttt{*} matches any combination of
characters, and \texttt{t*} refers to all labels starting with the
letter \texttt{t}.\index{\texttt{show labels} command} The \texttt{/all}
is a ``command qualifier''\index{command qualifier} that tells Metamath
to include labels of hypotheses.  (To see the syntax explained, type
\texttt{help show labels}.)  Type
\begin{verbatim}
MM> show labels t* /all
\end{verbatim}
Metamath will respond with
\begin{verbatim}
The statement number, label, and type are shown.
3 tt $f       4 tr $f       5 ts $f       8 tze $a
9 tpl $a      19 th1 $p
\end{verbatim}

You can use the \texttt{show statement} command to get information about a
particular statement.\index{\texttt{show statement} command}
For example, you can get information about the statement with label \texttt{mp}
by typing
\begin{verbatim}
MM> show statement mp /full
\end{verbatim}
Metamath will respond with
\begin{verbatim}
Statement 17 is located on line 43 of the file
"demo0.mm".
"Define the modus ponens inference rule"
17 mp $a |- Q $.
Its mandatory hypotheses in RPN order are:
  wp $f wff P $.
  wq $f wff Q $.
  min $e |- P $.
  maj $e |- ( P -> Q ) $.
The statement and its hypotheses require the
      variables:  Q P
The variables it contains are:  Q P
\end{verbatim}
The mandatory hypotheses\index{mandatory hypothesis} and their
order\index{RPN order} are
useful to know when you are trying to understand or debug a proof.

Now you are ready to look at what's really inside our proof.  First, here is
how to look at every step in the proof---not just the ones corresponding to an
ordinary formal proof\index{formal proof}, but also the ones that build up the
formulas that appear in each ordinary formal proof step.\index{\texttt{show
proof} command}
\begin{verbatim}
MM> show proof th1 /lemmon /all
\end{verbatim}

This will display the proof on the screen in the following format:
\begin{verbatim}
 1 tt            $f term t
 2 tze           $a term 0
 3 1,2 tpl       $a term ( t + 0 )
 4 tt            $f term t
 5 3,4 weq       $a wff ( t + 0 ) = t
 6 tt            $f term t
 7 tt            $f term t
 8 6,7 weq       $a wff t = t
 9 tt            $f term t
10 9 a2          $a |- ( t + 0 ) = t
11 tt            $f term t
12 tze           $a term 0
13 11,12 tpl     $a term ( t + 0 )
14 tt            $f term t
15 13,14 weq     $a wff ( t + 0 ) = t
16 tt            $f term t
17 tze           $a term 0
18 16,17 tpl     $a term ( t + 0 )
19 tt            $f term t
20 18,19 weq     $a wff ( t + 0 ) = t
21 tt            $f term t
22 tt            $f term t
23 21,22 weq     $a wff t = t
24 20,23 wim     $a wff ( ( t + 0 ) = t -> t = t )
25 tt            $f term t
26 25 a2         $a |- ( t + 0 ) = t
27 tt            $f term t
28 tze           $a term 0
29 27,28 tpl     $a term ( t + 0 )
30 tt            $f term t
31 tt            $f term t
32 29,30,31 a1   $a |- ( ( t + 0 ) = t -> ( ( t + 0 )
                                     = t -> t = t ) )
33 15,24,26,32 mp  $a |- ( ( t + 0 ) = t -> t = t )
34 5,8,10,33 mp  $a |- t = t
\end{verbatim}

The \texttt{/lemmon} command qualifier specifies what is known as a Lemmon-style
display\index{Lemmon-style proof}\index{proof!Lemmon-style}.  Omitting the
\texttt{/lemmon} qualifier results in a tree-style proof (see
p.~\pageref{treeproof} for an example) that is somewhat less explicit but
easier to follow once you get used to it.\index{tree-style
proof}\index{proof!tree-style}

The first number on each line is the step
number of the proof.  Any numbers that follow are step numbers assigned to the
hypotheses of the statement referenced by that step.  Next is the label of
the statement referenced by the step.  The statement type of the statement
referenced comes next, followed by the math symbol\index{math symbol} string
constructed by the proof up to that step.

The last step, 34, contains the statement that is being proved.

Looking at a small piece of the proof, notice that steps 3 and 4 have
established that
\texttt{( t + 0 )} and \texttt{t} are \texttt{term}\,s, and step 5 makes use of steps 3 and
4 to establish that \texttt{( t + 0 ) = t} is a \texttt{wff}.  Let Metamath
itself tell us in detail what is happening in step 5.  Note that the
``target hypothesis'' refers to where step 5 is eventually used, i.e., in step
34.
\begin{verbatim}
MM> show proof th1 /detailed_step 5
Proof step 5:  wp=weq $a wff ( t + 0 ) = t
This step assigns source "weq" ($a) to target "wp"
($f).  The source assertion requires the hypotheses
"tt" ($f, step 3) and "tr" ($f, step 4).  The parent
assertion of the target hypothesis is "mp" ($a,
step 34).
The source assertion before substitution was:
    weq $a wff t = r
The following substitutions were made to the source
assertion:
    Variable  Substituted with
     t         ( t + 0 )
     r         t
The target hypothesis before substitution was:
    wp $f wff P
The following substitution was made to the target
hypothesis:
    Variable  Substituted with
     P         ( t + 0 ) = t
\end{verbatim}

The full proof just shown is useful to understand what is going on in detail.
However, most of the time you will just be interested in
the ``essential'' or logical steps of a proof, i.e.\ those steps
that correspond to an
ordinary formal proof\index{formal proof}.  If you type
\begin{verbatim}
MM> show proof th1 /lemmon /renumber
\end{verbatim}
you will see\label{demoproof}
\begin{verbatim}
1 a2             $a |- ( t + 0 ) = t
2 a2             $a |- ( t + 0 ) = t
3 a1             $a |- ( ( t + 0 ) = t -> ( ( t + 0 )
                                     = t -> t = t ) )
4 2,3 mp         $a |- ( ( t + 0 ) = t -> t = t )
5 1,4 mp         $a |- t = t
\end{verbatim}
Compare this to the formal proof on p.~\pageref{zeroproof} and
notice the resemblance.
By default Metamath
does not show \texttt{\$f}\index{\texttt{\$f}
statement} hypotheses and everything branching off of them in the proof tree
when the proof is displayed; this makes the proof look more like an ordinary
mathematical proof, which does not normally incorporate the explicit
construction of expressions.
This is called the ``essential'' view
(at one time you had to add the
\texttt{/essential} qualifier in the \texttt{show proof}
command to get this view, but this is now the default).
You can could use the \texttt{/all} qualifier in the \texttt{show
proof} command to also show the explicit construction of expressions.
The \texttt{/renumber} qualifier means to renumber
the steps to correspond only to what is displayed.\index{\texttt{show proof}
command}

To exit Metamath, type\index{\texttt{exit} command}
\begin{verbatim}
MM> exit
\end{verbatim}

\subsection{Some Hints for Using the Command Line Interface}

We will conclude this quick introduction to Metamath\index{Metamath} with some
helpful hints on how to navigate your way through the commands.
\index{command line interface (CLI)}

When you type commands into Metamath's CLI, you only have to type as many
characters of a command keyword\index{command keyword} as are needed to make
it unambiguous.  If you type too few characters, Metamath will tell you what
the choices are.  In the case of the \texttt{read} command, only the \texttt{r} is
needed to specify it unambiguously, so you could have typed\index{\texttt{read}
command}
\begin{verbatim}
MM> r demo0.mm
\end{verbatim}
instead of
\begin{verbatim}
MM> read demo0.mm
\end{verbatim}
In our description, we always show the full command words.  When using the
Metamath CLI commands in a command file (to be read with the \texttt{submit}
command)\index{\texttt{submit} command}, it is good practice to use
the unabbreviated command to ensure your instructions will not become ambiguous
if more commands are added to the Metamath program in the future.

The command keywords\index{command
keyword} are not case sensitive; you may type either \texttt{read} or
\texttt{ReAd}.  File names may or may not be case sensitive, depending on your
computer's operating system.  Metamath label\index{label} and math
symbol\index{math symbol} tokens\index{token} are case-sensitive.

The \texttt{help} command\index{\texttt{help} command} will provide you
with a list of topics you can get help on.  You can then type
\texttt{help} {\em topic} to get help on that topic.

If you are uncertain of a command's spelling, just type as many characters
as you remember of the command.  If you have not typed enough characters to
specify it unambiguously, Metamath will tell you what choices you have.

\begin{verbatim}
MM> show s
         ^
?Ambiguous keyword - please specify SETTINGS,
STATEMENT, or SOURCE.
\end{verbatim}

If you don't know what argument to use as part of a command, type a
\texttt{?}\index{\texttt{]}@\texttt{?}\ in command lines}\ at the
argument position.  Metamath will tell you what it expected there.

\begin{verbatim}
MM> show ?
         ^
?Expected SETTINGS, LABELS, STATEMENT, SOURCE, PROOF,
MEMORY, TRACE_BACK, or USAGE.
\end{verbatim}

Finally, you may type just the first word or words of a command followed
by {\em return}.  Metamath will prompt you for the remaining part of the
command, showing you the choices at each step.  For example, instead of
typing \texttt{show statement th1 /full} you could interact in the
following manner:
\begin{verbatim}
MM> show
SETTINGS, LABELS, STATEMENT, SOURCE, PROOF,
MEMORY, TRACE_BACK, or USAGE <SETTINGS>? st
What is the statement label <th1>?
/ or nothing <nothing>? /
TEX, COMMENT_ONLY, or FULL <TEX>? f
/ or nothing <nothing>?
19 th1 $p |- t = t $= ... $.
\end{verbatim}
After each \texttt{?}\ in this mode, you must give Metamath the
information it requests.  Sometimes Metamath gives you a list of choices
with the default choice indicated by brackets \texttt{< > }. Pressing
{\em return} after the \texttt{?}\ will select the default choice.
Answering anything else will override the default.  Note that the
\texttt{/} in command qualifiers is considered a separate
token\index{token} by the parser, and this is why it is asked for
separately.

\section{Your First Proof}\label{frstprf}

Proofs are developed with the aid of the Proof Assistant\index{Proof
Assistant}.  We will now show you how the proof of theorem \texttt{th1}
was built.  So that you can repeat these steps, we will first have the
Proof Assistant erase the proof in Metamath's source buffer\index{source
buffer}, then reconstruct it.  (The source buffer is the place in memory
where Metamath stores the information in the database when it is
\texttt{read}\index{\texttt{read} command} in.  New or modified proofs
are kept in the source buffer until a \texttt{write source}
command\index{\texttt{write source} command} is issued.)  In practice, you
would place a \texttt{?}\index{\texttt{]}@\texttt{?}\ inside proofs}\
between \texttt{\$=}\index{\texttt{\$=} keyword} and
\texttt{\$.}\index{\texttt{\$.}\ keyword}\ in the database to indicate
to Metamath\index{Metamath} that the proof is unknown, and that would be
your starting point.  Whenever the \texttt{verify proof} command encounters
a proof with a \texttt{?}\ in place of a proof step, the statement is
identified as not proved.

When I first started creating Metamath proofs, I would write down
on a piece of paper the complete
formal proof\index{formal proof} as it would appear
in a \texttt{show proof} command\index{\texttt{show proof} command}; see
the display of \texttt{show proof th1 /lemmon /re\-num\-ber} above as an
example.  After you get used to using the Proof Assistant\index{Proof
Assistant} you may get to a point where you can ``see'' the proof in your mind
and let the Proof Assistant guide you in filling in the details, at least for
simpler proofs, but until you gain that experience you may find it very useful
to write down all the details in advance.
Otherwise you may waste a lot of time as you let it take you down a wrong path.
However, others do not find this approach as helpful.
For example, Thomas Brendan Leahy\index{Leahy, Thomas Brendan}
finds that it is more helpful to him to interactively
work backward from a machine-readable statement.
David A. Wheeler\index{Wheeler, David A.}
writes down a general approach, but develops the proof
interactively by switching between
working forwards (from hypotheses and facts likely to be useful) and
backwards (from the goal) until the forwards and backwards approaches meet.
In the end, use whatever approach works for you.

A proof is developed with the Proof Assistant by working backwards, starting
with the theorem\index{theorem} to be proved, and assigning each unknown step
with a theorem or hypothesis until no more unknown steps remain.  The Proof
Assistant will not let you make an assignment unless it can be ``unified''
with the unknown step.  This means that a
substitution\index{substitution!variable}\index{variable substitution} of
variables exists that will make the assignment match the unknown step.  On the
other hand, in the middle of a proof, when working backwards, often more than
one unification\index{unification} (set of substitutions) is possible, since
there is not enough information available at that point to uniquely establish
it.  In this case you can tell Metamath which unification to choose, or you
can continue to assign unknown steps until enough information is available to
make the unification unique.

We will assume you have entered Metamath and read in the database as described
above.  The following dialog shows how the proof was developed.  For more
details on what some of the commands do, refer to Section~\ref{pfcommands}.
\index{\texttt{prove} command}

\begin{verbatim}
MM> prove th1
Entering the Proof Assistant.  Type HELP for help, EXIT
to exit.  You will be working on the proof of statement th1:
  $p |- t = t
Note:  The proof you are starting with is already complete.
MM-PA>
\end{verbatim}

The \verb/MM-PA>/ prompt means we are inside the Proof
Assistant.\index{Proof Assistant} Most of the regular Metamath commands
(\texttt{show statement}, etc.) are still available if you need them.

\begin{verbatim}
MM-PA> delete all
The entire proof was deleted.
\end{verbatim}

We have deleted the whole proof so we can start from scratch.

\begin{verbatim}
MM-PA> show new_proof/lemmon/all
1 ?              $? |- t = t
\end{verbatim}

The \texttt{show new{\char`\_}proof} command\index{\texttt{show
new{\char`\_}proof} command} is like \texttt{show proof} except that we
don't specify a statement; instead, the proof we're working on is
displayed.

\begin{verbatim}
MM-PA> assign 1 mp
To undo the assignment, DELETE STEP 5 and INITIALIZE, UNIFY
if needed.
3   min=?  $? |- $2
4   maj=?  $? |- ( $2 -> t = t )
\end{verbatim}

The \texttt{assign} command\index{\texttt{assign} command} above means
``assign step 1 with the statement whose label is \texttt{mp}.''  Note
that step renumbering will constantly occur as you assign steps in the
middle of a proof; in general all steps from the step you assign until
the end of the proof will get moved up.  In this case, what used to be
step 1 is now step 5, because the (partial) proof now has five steps:
the four hypotheses of the \texttt{mp} statement and the \texttt{mp}
statement itself.  Let's look at all the steps in our partial proof:

\begin{verbatim}
MM-PA> show new_proof/lemmon/all
1 ?              $? wff $2
2 ?              $? wff t = t
3 ?              $? |- $2
4 ?              $? |- ( $2 -> t = t )
5 1,2,3,4 mp     $a |- t = t
\end{verbatim}

The symbol \texttt{\$2} is a temporary variable\index{temporary
variable} that represents a symbol sequence not yet known.  In the final
proof, all temporary variables will be eliminated.  The general format
for a temporary variable is \texttt{\$} followed by an integer.  Note
that \texttt{\$} is not a legal character in a math symbol (see
Section~\ref{dollardollar}, p.~\pageref{dollardollar}), so there will
never be a naming conflict between real symbols and temporary variables.

Unknown steps 1 and 2 are constructions of the two wffs used by the
modus ponens rule.  As you will see at the end of this section, the
Proof Assistant\index{Proof Assistant} can usually figure these steps
out by itself, and we will not have to worry about them.  Therefore from
here on we will display only the ``essential'' hypotheses, i.e.\ those
steps that correspond to traditional formal proofs\index{formal proof}.

\begin{verbatim}
MM-PA> show new_proof/lemmon
3 ?              $? |- $2
4 ?              $? |- ( $2 -> t = t )
5 3,4 mp         $a |- t = t
\end{verbatim}

Unknown steps 3 and 4 are the ones we must focus on.  They correspond to the
minor and major premises of the modus ponens rule.  We will assign them as
follows.  Notice that because of the step renumbering that takes place
after an assignment, it is advantageous to assign unknown steps in reverse
order, because earlier steps will not get renumbered.

\begin{verbatim}
MM-PA> assign 4 mp
To undo the assignment, DELETE STEP 8 and INITIALIZE, UNIFY
if needed.
3   min=?  $? |- $2
6     min=?  $? |- $4
7     maj=?  $? |- ( $4 -> ( $2 -> t = t ) )
\end{verbatim}

We are now going to describe an obscure feature that you will probably
never use but should be aware of.  The Metamath language allows empty
symbol sequences to be substituted for variables, but in most formal
systems this feature is never used.  One of the few examples where is it
used is the MIU-system\index{MIU-system} described in
Appendix~\ref{MIU}.  But such systems are rare, and by default this
feature is turned off in the Proof Assistant.  (It is always allowed for
{\tt verify proof}.)  Let us turn it on and see what
happens.\index{\texttt{set empty{\char`\_}substitution} command}

\begin{verbatim}
MM-PA> set empty_substitution on
Substitutions with empty symbol sequences is now allowed.
\end{verbatim}

With this feature enabled, more unifications will be
ambiguous\index{ambiguous unification}\index{unification!ambiguous} in
the middle of a proof, because
substitution\index{substitution!variable}\index{variable substitution}
of variables with empty symbol sequences will become an additional
possibility.  Let's see what happens when we make our next assignment.

\begin{verbatim}
MM-PA> assign 3 a2
There are 2 possible unifications.  Please select the correct
    one or Q if you want to UNIFY later.
Unify:  |- $6
 with:  |- ( $9 + 0 ) = $9
Unification #1 of 2 (weight = 7):
  Replace "$6" with "( + 0 ) ="
  Replace "$9" with ""
  Accept (A), reject (R), or quit (Q) <A>? r
\end{verbatim}

The first choice presented is the wrong one.  If we had selected it,
temporary variable \texttt{\$6} would have been assigned a truncated
wff, and temporary variable \texttt{\$9} would have been assigned an
empty sequence (which is not allowed in our system).  With this choice,
eventually we would reach a point where we would get stuck because
we would end up with steps impossible to prove.  (You may want to
try it.)  We typed \texttt{r} to reject the choice.

\begin{verbatim}
Unification #2 of 2 (weight = 21):
  Replace "$6" with "( $9 + 0 ) = $9"
  Accept (A), reject (R), or quit (Q) <A>? q
To undo the assignment, DELETE STEP 4 and INITIALIZE, UNIFY
if needed.
 7     min=?  $? |- $8
 8     maj=?  $? |- ( $8 -> ( $6 -> t = t ) )
\end{verbatim}

The second choice is correct, and normally we would type \texttt{a}
to accept it.  But instead we typed \texttt{q} to show what will happen:
it will leave the step with an unknown unification, which can be
seen as follows:

\begin{verbatim}
MM-PA> show new_proof/not_unified
 4   min    $a |- $6
        =a2  = |- ( $9 + 0 ) = $9
\end{verbatim}

Later we can unify this with the \texttt{unify}
\texttt{all/interactive} command.

The important point to remember is that occasionally you will be
presented with several unification choices while entering a proof, when
the program determines that there is not enough information yet to make
an unambiguous choice automatically (and this can happen even with
\texttt{set empty{\char`\_}substitution} turned off).  Usually it is
obvious by inspection which choice is correct, since incorrect ones will
tend to be meaningless fragments of wffs.  In addition, the correct
choice will usually be the first one presented, unlike our example
above.

Enough of this digression.  Let us go back to the default setting.

\begin{verbatim}
MM-PA> set empty_substitution off
The ability to substitute empty expressions for variables
has been turned off.  Note that this may make the Proof
Assistant too restrictive in some cases.
\end{verbatim}

If we delete the proof, start over, and get to the point where
we digressed above, there will no longer be an ambiguous unification.

\begin{verbatim}
MM-PA> assign 3 a2
To undo the assignment, DELETE STEP 4 and INITIALIZE, UNIFY
if needed.
 7     min=?  $? |- $4
 8     maj=?  $? |- ( $4 -> ( ( $5 + 0 ) = $5 -> t = t ) )
\end{verbatim}

Let us look at our proof so far, and continue.

\begin{verbatim}
MM-PA> show new_proof/lemmon
 4 a2            $a |- ( $5 + 0 ) = $5
 7 ?             $? |- $4
 8 ?             $? |- ( $4 -> ( ( $5 + 0 ) = $5 -> t = t ) )
 9 7,8 mp        $a |- ( ( $5 + 0 ) = $5 -> t = t )
10 4,9 mp        $a |- t = t
MM-PA> assign 8 a1
To undo the assignment, DELETE STEP 11 and INITIALIZE, UNIFY
if needed.
 7     min=?  $? |- ( t + 0 ) = t
MM-PA> assign 7 a2
To undo the assignment, DELETE STEP 8 and INITIALIZE, UNIFY
if needed.
MM-PA> show new_proof/lemmon
 4 a2            $a |- ( t + 0 ) = t
 8 a2            $a |- ( t + 0 ) = t
12 a1            $a |- ( ( t + 0 ) = t -> ( ( t + 0 ) = t ->
                                                    t = t ) )
13 8,12 mp       $a |- ( ( t + 0 ) = t -> t = t )
14 4,13 mp       $a |- t = t
\end{verbatim}

Now all temporary variables and unknown steps have been eliminated from the
``essential'' part of the proof.  When this is achieved, the Proof
Assistant\index{Proof Assistant} can usually figure out the rest of the proof
automatically.  (Note that the \texttt{improve} command can occasionally be
useful for filling in essential steps as well, but it only tries to make use
of statements that introduce no new variables in their hypotheses, which is
not the case for \texttt{mp}. Also it will not try to improve steps containing
temporary variables.)  Let's look at the complete proof, then run
the \texttt{improve} command, then look at it again.

\begin{verbatim}
MM-PA> show new_proof/lemmon/all
 1 ?             $? wff ( t + 0 ) = t
 2 ?             $? wff t = t
 3 ?             $? term t
 4 3 a2          $a |- ( t + 0 ) = t
 5 ?             $? wff ( t + 0 ) = t
 6 ?             $? wff ( ( t + 0 ) = t -> t = t )
 7 ?             $? term t
 8 7 a2          $a |- ( t + 0 ) = t
 9 ?             $? term ( t + 0 )
10 ?             $? term t
11 ?             $? term t
12 9,10,11 a1    $a |- ( ( t + 0 ) = t -> ( ( t + 0 ) = t ->
                                                    t = t ) )
13 5,6,8,12 mp   $a |- ( ( t + 0 ) = t -> t = t )
14 1,2,4,13 mp   $a |- t = t
\end{verbatim}

\begin{verbatim}
MM-PA> improve all
A proof of length 1 was found for step 11.
A proof of length 1 was found for step 10.
A proof of length 3 was found for step 9.
A proof of length 1 was found for step 7.
A proof of length 9 was found for step 6.
A proof of length 5 was found for step 5.
A proof of length 1 was found for step 3.
A proof of length 3 was found for step 2.
A proof of length 5 was found for step 1.
Steps 1 and above have been renumbered.
CONGRATULATIONS!  The proof is complete.  Use SAVE
NEW_PROOF to save it.  Note:  The Proof Assistant does
not detect $d violations.  After saving the proof, you
should verify it with VERIFY PROOF.
\end{verbatim}

The \texttt{save new{\char`\_}proof} command\index{\texttt{save
new{\char`\_}proof} command} will save the proof in the database.  Here
we will just display it in a form that can be clipped out of a log file
and inserted manually into the database source file with a text
editor.\index{normal proof}\index{proof!normal}

\begin{verbatim}
MM-PA> show new_proof/normal
---------Clip out the proof below this line:
      tt tze tpl tt weq tt tt weq tt a2 tt tze tpl tt weq
      tt tze tpl tt weq tt tt weq wim tt a2 tt tze tpl tt
      tt a1 mp mp $.
---------The proof of 'th1' to clip out ends above this line.
\end{verbatim}

There is another proof format called ``compressed''\index{compressed
proof}\index{proof!compressed} that you will see in databases.  It is
not important to understand how it is encoded but only to recognize it
when you see it.  Its only purpose is to reduce storage requirements for
large proofs.  A compressed proof can always be converted to a normal
one and vice-versa, and the Metamath \texttt{show proof}
commands\index{\texttt{show proof} command} work equally well with
compressed proofs.  The compressed proof format is described in
Appendix~\ref{compressed}.

\begin{verbatim}
MM-PA> show new_proof/compressed
---------Clip out the proof below this line:
      ( tze tpl weq a2 wim a1 mp ) ABCZADZAADZAEZJJKFLIA
      AGHH $.
---------The proof of 'th1' to clip out ends above this line.
\end{verbatim}

Now we will exit the Proof Assistant.  Since we made changes to the proof,
it will warn us that we have not saved it.  In this case, we don't care.

\begin{verbatim}
MM-PA> exit
Warning:  You have not saved changes to the proof.
Do you want to EXIT anyway (Y, N) <N>? y
Exiting the Proof Assistant.
Type EXIT again to exit Metamath.
\end{verbatim}

The Proof Assistant\index{Proof Assistant} has several other commands
that can help you while creating proofs.  See Section~\ref{pfcommands}
for a list of them.

A command that is often useful is \texttt{minimize{\char`\_}with
*/brief}, which tries to shorten the proof.  It can make the process
more efficient by letting you write a somewhat ``sloppy'' proof then
clean up some of the fine details of optimization for you (although it
can't perform miracles such as restructuring the overall proof).

\section{A Note About Editing a Data\-base File}

Once your source file contains proofs, there are some restrictions on
how you can edit it so that the proofs remain valid.  Pay particular
attention to these rules, since otherwise you can lose a lot of work.
It is a good idea to periodically verify all proofs with \texttt{verify
proof *} to ensure their integrity.

If your file contains only normal (as opposed to compressed) proofs, the
main rule is that you may not change the order of the mandatory
hypotheses\index{mandatory hypothesis} of any statement referenced in a
later proof.  For example, if you swap the order of the major and minor
premise in the modus ponens rule, all proofs making use of that rule
will become incorrect.  The \texttt{show statement}
command\index{\texttt{show statement} command} will show you the
mandatory hypotheses of a statement and their order.

If a statement has a compressed proof, you also must not change the
order of {\em its} mandatory hypotheses.  The compressed proof format
makes use of this information as part of the compression technique.
Note that swapping the names of two variables in a theorem will change
the order of its mandatory hypotheses.

The safest way to edit a statement, say \texttt{mytheorem}, is to
duplicate it then rename the original to \texttt{mytheoremOLD}
throughout the database.  Once the edited version is re-proved, all
statements referencing \texttt{mytheoremOLD} can be updated in the Proof
Assistant using \texttt{minimize{\char`\_}with
mytheorem
/allow{\char`\_}growth}.\index{\texttt{minimize{\char`\_}with} command}
% 3/10/07 Note: line-breaking the above results in duplicate index entries

\chapter{Abstract Mathematics Revealed}\label{fol}

\section{Logic and Set Theory}\label{logicandsettheory}

\begin{quote}
  {\em Set theory can be viewed as a form of exact theology.}
  \flushright\sc  Rudy Rucker\footnote{\cite{Barrow}, p.~31.}\\
\end{quote}\index{Rucker, Rudy}

Despite its seeming complexity, all of standard mathematics, no matter how
deep or abstract, can amazingly enough be derived from a relatively small set
of axioms\index{axiom} or first principles. The development of these axioms is
among the most impressive and important accomplishments of mathematics in the
20th century. Ultimately, these axioms can be broken down into a set of rules
for manipulating symbols that any technically oriented person can follow.

We will not spend much time trying to convey a deep, higher-level
understanding of the meaning of the axioms. This kind of understanding
requires some mathematical sophistication as well as an understanding of the
philosophy underlying the foundations of mathematics and typically develops
over time as you work with mathematics.  Our goal, instead, is to give you the
immediate ability to follow how theorems\index{theorem} are derived from the
axioms and from other theorems.  This will be similar to learning the syntax
of a computer language, which lets you follow the details in a program but
does not necessarily give you the ability to write non-trivial programs on
your own, an ability that comes with practice. For now don't be alarmed by
abstract-sounding names of the axioms; just focus on the rules for
manipulating the symbols, which follow the simple conventions of the
Metamath\index{Metamath} language.

The axioms that underlie all of standard mathematics consist of axioms of logic
and axioms of set theory. The axioms of logic are divided into two
subcategories, propositional calculus\index{propositional calculus} (sometimes
called sentential logic\index{sentential logic}) and predicate calculus
(sometimes called first-order logic\index{first-order logic}\index{quantifier
theory}\index{predicate calculus} or quantifier theory).  Propositional
calculus is a prerequisite for predicate calculus, and predicate calculus is a
prerequisite for set theory.  The version of set theory most commonly used is
Zermelo--Fraenkel set theory\index{Zermelo--Fraenkel set theory}\index{set theory}
with the axiom of choice,
often abbreviated as ZFC\index{ZFC}.

Here in a nutshell is what the axioms are all about in an informal way. The
connection between this description and symbols we will show you won't be
immediately apparent and in principle needn't ever be.  Our description just
tries to summarize what mathematicians think about when they work with the
axioms.

Logic is a set of rules that allow us determine truths given other truths.
Put another way,
logic is more or less the translation of what we would consider common sense
into a rigorous set of axioms.\index{axioms of logic}  Suppose $\varphi$,
$\psi$, and $\chi$ (the Greek letters phi, psi, and chi) represent statements
that are either true or false, and $x$ is a variable\index{variable!in predicate
calculus} ranging over some group of mathematical objects (sets, integers,
real numbers, etc.). In mathematics, a ``statement'' really means a formula,
and $\psi$ could be for example ``$x = 2$.''
Propositional calculus\index{propositional calculus}
allows us to use variables that are either true or false
and make deductions such as
``if $\varphi$ implies $\psi$ and $\psi$ implies $\chi$, then $\varphi$
implies $\chi$.''
Predicate calculus\index{predicate calculus}
extends propositional calculus by also allowing us
to discuss statements about objects (not just true and false values), including
statements about ``all'' or ``at least one'' object.
For example, predicate calculus allows to say,
``if $\varphi$ is true for all $x$, then $\varphi$ is true for some $x$.''
The logic used in \texttt{set.mm} is standard classical logic
(as opposed to other logic systems like intuitionistic logic).

Set theory\index{set theory} has to do with the manipulation of objects and
collections of objects, specifically the abstract, imaginary objects that
mathematics deals with, such as numbers. Everything that is claimed to exist
in mathematics is considered to be a set.  A set called the empty
set\index{empty set} contains nothing.  We represent the empty set by
$\varnothing$.  Many sets can be built up from the empty set.  There is a set
represented by $\{\varnothing\}$ that contains the empty set, another set
represented by $\{\varnothing,\{\varnothing\}\}$ that contains this set as
well as the empty set, another set represented by $\{\{\varnothing\}\}$ that
contains just the set that contains the empty set, and so on ad infinitum. All
mathematical objects, no matter how complex, are defined as being identical to
certain sets: the integer\index{integer} 0 is defined as the empty set, the
integer 1 is defined as $\{\varnothing\}$, the integer 2 is defined as
$\{\varnothing,\{\varnothing\}\}$.  (How these definitions were chosen doesn't
matter now, but the idea behind it is that these sets have the properties we
expect of integers once suitable operations are defined.)  Mathematical
operations, such as addition, are defined in terms of operations on
sets---their union\index{set union}, intersection\index{set intersection}, and
so on---operations you may have used in elementary school when you worked
with groups of apples and oranges.

With a leap of faith, the axioms also postulate the existence of infinite
sets\index{infinite set}, such as the set of all non-negative integers ($0, 1,
2,\ldots$, also called ``natural numbers''\index{natural number}).  This set
can't be represented with the brace notation\index{brace notation} we just
showed you, but requires a more complicated notation called ``class
abstraction.''\index{class abstraction}\index{abstraction class}  For
example, the infinite set $\{ x |
\mbox{``$x$ is a natural number''} \} $ means the ``set of all objects $x$
such that $x$ is a natural number'' i.e.\ the set of natural numbers; here,
``$x$ is a natural number'' is a rather complicated formula when broken down
into the primitive symbols.\label{expandom}\footnote{The statement ``$x$ is a
natural number'' is formally expressed as ``$x \in \omega$,'' where $\in$
(stylized epsilon) means ``is in'' or ``is an element of'' and $\omega$
(omega) means ``the set of natural numbers.''  When ``$x\in\omega$'' is
completely expanded in terms of the primitive symbols of set theory, the
result is  $\lnot$ $($ $\lnot$ $($ $\forall$ $z$ $($ $\lnot$ $\forall$ $w$ $($
$z$ $\in$ $w$ $\rightarrow$ $\lnot$ $w$ $\in$ $x$ $)$ $\rightarrow$ $z$ $\in$
$x$ $)$ $\rightarrow$ $($ $\forall$ $z$ $($ $\lnot$ $($ $\forall$ $w$ $($ $w$
$\in$ $z$ $\rightarrow$ $w$ $\in$ $x$ $)$ $\rightarrow$ $\forall$ $w$ $\lnot$
$w$ $\in$ $z$ $)$ $\rightarrow$ $\lnot$ $\forall$ $w$ $($ $w$ $\in$ $z$
$\rightarrow$ $\lnot$ $\forall$ $v$ $($ $v$ $\in$ $z$ $\rightarrow$ $\lnot$
$v$ $\in$ $w$ $)$ $)$ $)$ $\rightarrow$ $\lnot$ $\forall$ $z$ $\forall$ $w$
$($ $\lnot$ $($ $z$ $\in$ $x$ $\rightarrow$ $\lnot$ $w$ $\in$ $x$ $)$
$\rightarrow$ $($ $\lnot$ $z$ $\in$ $w$ $\rightarrow$ $($ $\lnot$ $z$ $=$ $w$
$\rightarrow$ $w$ $\in$ $z$ $)$ $)$ $)$ $)$ $)$ $\rightarrow$ $\lnot$
$\forall$ $y$ $($ $\lnot$ $($ $\lnot$ $($ $\forall$ $z$ $($ $\lnot$ $\forall$
$w$ $($ $z$ $\in$ $w$ $\rightarrow$ $\lnot$ $w$ $\in$ $y$ $)$ $\rightarrow$
$z$ $\in$ $y$ $)$ $\rightarrow$ $($ $\forall$ $z$ $($ $\lnot$ $($ $\forall$
$w$ $($ $w$ $\in$ $z$ $\rightarrow$ $w$ $\in$ $y$ $)$ $\rightarrow$ $\forall$
$w$ $\lnot$ $w$ $\in$ $z$ $)$ $\rightarrow$ $\lnot$ $\forall$ $w$ $($ $w$
$\in$ $z$ $\rightarrow$ $\lnot$ $\forall$ $v$ $($ $v$ $\in$ $z$ $\rightarrow$
$\lnot$ $v$ $\in$ $w$ $)$ $)$ $)$ $\rightarrow$ $\lnot$ $\forall$ $z$
$\forall$ $w$ $($ $\lnot$ $($ $z$ $\in$ $y$ $\rightarrow$ $\lnot$ $w$ $\in$
$y$ $)$ $\rightarrow$ $($ $\lnot$ $z$ $\in$ $w$ $\rightarrow$ $($ $\lnot$ $z$
$=$ $w$ $\rightarrow$ $w$ $\in$ $z$ $)$ $)$ $)$ $)$ $\rightarrow$ $($
$\forall$ $z$ $\lnot$ $z$ $\in$ $y$ $\rightarrow$ $\lnot$ $\forall$ $w$ $($
$\lnot$ $($ $w$ $\in$ $y$ $\rightarrow$ $\lnot$ $\forall$ $z$ $($ $w$ $\in$
$z$ $\rightarrow$ $\lnot$ $z$ $\in$ $y$ $)$ $)$ $\rightarrow$ $\lnot$ $($
$\lnot$ $\forall$ $z$ $($ $w$ $\in$ $z$ $\rightarrow$ $\lnot$ $z$ $\in$ $y$
$)$ $\rightarrow$ $w$ $\in$ $y$ $)$ $)$ $)$ $)$ $\rightarrow$ $x$ $\in$ $y$
$)$ $)$ $)$. Section~\ref{hierarchy} shows the hierarchy of definitions that
leads up to this expression.}\index{stylized epsilon ($\in$)}\index{omega
($\omega$)}  Actually, the primitive symbols don't even include the brace
notation.  The brace notation is a high-level definition, which you can find in
Section~\ref{hierarchy}.

Interestingly, the arithmetic of integers\index{integer} and
rationals\index{rational number} can be developed without appealing to the
existence of an infinite set, whereas the arithmetic of real
numbers\index{real number} requires it.

Each variable\index{variable!in set theory} in the axioms of set theory
represents an arbitrary set, and the axioms specify the legal kinds of things
you can do with these variables at a very primitive level.

Now, you may think that numbers and arithmetic are a lot more intuitive and
fundamental than sets and therefore should be the foundation of mathematics.
What is really the case is that you've dealt with numbers all your life and
are comfortable with a few rules for manipulating them such as addition and
multiplication.  Those rules only cover a small portion of what can be done
with numbers and only a very tiny fraction of the rest of mathematics.  If you
look at any elementary book on number theory, you will quickly become lost if
these are the only rules that you know.  Even though such books may present a
list of ``axioms''\index{axiom} for arithmetic, the ability to use the axioms
and to understand proofs of theorems\index{theorem} (facts) about numbers
requires an implicit mathematical talent that frustrates many people
from studying abstract mathematics.  The kind of mathematics that most people
know limits them to the practical, everyday usage of blindly manipulating
numbers and formulas, without any understanding of why those rules are correct
nor any ability to go any further.  For example, do you know why multiplying
two negative numbers yields a positive number?  Starting with set theory, you
will also start off blindly manipulating symbols according to the rules we give
you, but with the advantage that these rules will allow you, in principle, to
access {\em all} of mathematics, not just a tiny part of it.

Of course, concrete examples are often helpful in the learning process. For
example, you can verify that $2\cdot 3=3 \cdot 2$ by actually grouping
objects and can easily ``see'' how it generalizes to $x\cdot y = y\cdot x$,
even though you might not be able to rigorously prove it.  Similarly, in set
theory it can be helpful to understand how the axioms of set theory apply to
(and are correct for) small finite collections of objects.  You should be aware
that in set theory intuition can be misleading for infinite collections, and
rigorous proofs become more important.  For example, while $x\cdot y = y\cdot
x$ is correct for finite ordinals (which are the natural numbers), it is not
usually true for infinite ordinals.

\section{The Axioms for All of Mathematics}

In this section\index{axioms for mathematics}, we will show you the axioms
for all of standard mathematics (i.e.\ logic and set theory) as they are
traditionally presented.  The traditional presentation is useful for someone
with the mathematical experience needed to correctly manipulate high-level
abstract concepts.  For someone without this talent, knowing how to actually
make use of these axioms can be difficult.  The purpose of this section is to
allow you to see how the version of the axioms used in the standard
Metamath\index{Metamath} database \texttt{set.mm}\index{set
theory database (\texttt{set.mm})} relates to  the typical version
in textbooks, and also to give you an informal feel for them.

\subsection{Propositional Calculus}

Propositional calculus\index{propositional calculus} concerns itself with
statements that can be interpreted as either true or false.  Some examples of
statements (outside of mathematics) that are either true or false are ``It is
raining today'' and ``The United States has a female president.'' In
mathematics, as we mentioned, statements are really formulas.

In propositional calculus, we don't care what the statements are.  We also
treat a logical combination of statements, such as ``It is raining today and
the United States has a female president,'' no differently from a single
statement.  Statements and their combinations are called well-formed formulas
(wffs)\index{well-formed formula (wff)}.  We define wffs only in terms of
other wffs and don't define what a ``starting'' wff is.  As is common practice
in the literature, we use Greek letters to represent wffs.

Specifically, suppose $\varphi$ and $\psi$ are wffs.  Then the combinations
$\varphi\rightarrow\psi$ (``$\varphi$ implies $\psi$,'' also read ``if
$\varphi$ then $\psi$'')\index{implication ($\rightarrow$)} and $\lnot\varphi$
(``not $\varphi$'')\index{negation ($\lnot$)} are also wffs.

The three axioms of propositional calculus\index{axioms of propositional
calculus} are all wffs of the following form:\footnote{A remarkable result of
C.~A.~Meredith\index{Meredith, C. A.} squeezes these three axioms into the
single axiom $((((\varphi\rightarrow \psi)\rightarrow(\neg \chi\rightarrow\neg
\theta))\rightarrow \chi )\rightarrow \tau)\rightarrow((\tau\rightarrow
\varphi)\rightarrow(\theta\rightarrow \varphi))$ \cite{CAMeredith},
which is believed to be the shortest possible.}
\begin{center}
     $\varphi\rightarrow(\psi\rightarrow \varphi)$\\

     $(\varphi\rightarrow (\psi\rightarrow \chi))\rightarrow
((\varphi\rightarrow  \psi)\rightarrow (\varphi\rightarrow \chi))$\\

     $(\neg \varphi\rightarrow \neg\psi)\rightarrow (\psi\rightarrow
\varphi)$
\end{center}

These three axioms are widely used.
They are attributed to Jan {\L}ukasiewicz
(pronounced woo-kah-SHAY-vitch) and was popularized by Alonzo Church,
who called it system P2. (Thanks to Ted Ulrich for this information.)

There are an infinite number of axioms, one for each possible
wff\index{well-formed formula (wff)} of the above form.  (For this reason,
axioms such as the above are often called ``axiom schemes.''\index{axiom
scheme})  Each Greek letter in the axioms may be substituted with a more
complex wff to result in another axiom.  For example, substituting
$\neg(\varphi\rightarrow\chi)$ for $\varphi$ in the first axiom yields
$\neg(\varphi\rightarrow\chi)\rightarrow(\psi\rightarrow
\neg(\varphi\rightarrow\chi))$, which is still an axiom.

To deduce new true statements (theorems\index{theorem}) from the axioms, a
rule\index{rule} called ``modus ponens''\index{modus ponens} is used.  This
rule states that if the wff $\varphi$ is an axiom or a theorem, and the wff
$\varphi\rightarrow\psi$ is an axiom or a theorem, then the wff $\psi$ is also
a theorem\index{theorem}.

As a non-mathematical example of modus ponens, suppose we have proved (or
taken as an axiom) ``Bob is a man'' and separately have proved (or taken as
an axiom) ``If Bob is a man, then Bob is a human.''  Using the rule of modus
ponens, we can logically deduce, ``Bob is a human.''

From Metamath's\index{Metamath} point of view, the axioms and the rule of
modus ponens just define a mechanical means for deducing new true statements
from existing true statements, and that is the complete content of
propositional calculus as far as Metamath is concerned.  You can read a logic
textbook to gain a better understanding of their meaning, or you can just let
their meaning slowly become apparent to you after you use them for a while.

It is actually rather easy to check to see if a formula is a theorem of
propositional calculus.  Theorems of propositional calculus are also called
``tautologies.''\index{tautology}  The technique to check whether a formula is
a tautology is called the ``truth table method,''\index{truth table} and it
works like this.  A wff $\varphi\rightarrow\psi$ is false whenever $\varphi$ is true
and $\psi$ is false.  Otherwise it is true.  A wff $\lnot\varphi$ is false
whenever $\varphi$ is true and false otherwise. To verify a tautology such as
$\varphi\rightarrow(\psi\rightarrow \varphi)$, you break it down into sub-wffs and
construct a truth table that accounts for all possible combinations of true
and false assigned to the wff metavariables:
\begin{center}\begin{tabular}{|c|c|c|c|}\hline
\mbox{$\varphi$} & \mbox{$\psi$} & \mbox{$\psi\rightarrow\varphi$}
    & \mbox{$\varphi\rightarrow(\psi\rightarrow \varphi)$} \\ \hline \hline
              T   &  T    &      T       &        T    \\ \hline
              T   &  F    &      T       &        T    \\ \hline
              F   &  T    &      F       &        T    \\ \hline
              F   &  F    &      T       &        T    \\ \hline
\end{tabular}\end{center}
If all entries in the last column are true, the formula is a tautology.

Now, the truth table method doesn't tell you how to prove the tautology from
the axioms, but only that a proof exists.  Finding an actual proof (especially
one that is short and elegant) can be challenging.  Methods do exist for
automatically generating proofs in propositional calculus, but the proofs that
result can sometimes be very long.  In the Metamath \texttt{set.mm}\index{set
theory database (\texttt{set.mm})} database, most
or all proofs were created manually.

Section \ref{metadefprop} discusses various definitions
that make propositional calculus easier to use.
For example, we define:

\begin{itemize}
\item $\varphi \vee \psi$
  is true if either $\varphi$ or $\psi$ (or both) are true
  (this is disjunction\index{disjunction ($\vee$)}
  aka logical {\sc or}\index{logical {\sc or} ($\vee$)}).

\item $\varphi \wedge \psi$
  is true if both $\varphi$ and $\psi$ are true
  (this is conjunction\index{conjunction ($\wedge$)}
  aka logical {\sc and}\index{logical {\sc and} ($\wedge$)}).

\item $\varphi \leftrightarrow \psi$
  is true if $\varphi$ and $\psi$ have the same value, that is,
  they are both true or both false
  (this is the biconditional\index{biconditional ($\leftrightarrow$)}).
\end{itemize}

\subsection{Predicate Calculus}

Predicate calculus\index{predicate calculus} introduces the concept of
``individual variables,''\index{variable!in predicate calculus}\index{individual
variable} which
we will usually just call ``variables.''
These variables can represent something other than true or false (wffs),
and will always represent sets when we get to set theory.  There are also
three new symbols $\forall$\index{universal quantifier ($\forall$)},
$=$\index{equality ($=$)}, and $\in$\index{stylized epsilon ($\in$)},
read ``for all,'' ``equals,'' and ``is an element of''
respectively.  We will represent variables with the letters $x$, $y$, $z$, and
$w$, as is common practice in the literature.
For example, $\forall x \varphi$ means ``for all possible values of
$x$, $\varphi$ is true.''

In predicate calculus, we extend the definition of a wff\index{well-formed
formula (wff)}.  If $\varphi$ is a wff and $x$ and $y$ are variables, then
$\forall x \, \varphi$, $x=y$, and $x\in y$ are wffs. Note that these three new
types of wffs can be considered ``starting'' wffs from which we can build
other wffs with $\rightarrow$ and $\neg$ .  The concept of a starting wff was
absent in propositional calculus.  But starting wff or not, all we are really
concerned with is whether our wffs are correctly constructed according to
these mechanical rules.

A quick aside:
To prevent confusion, it might be best at this point to think of the variables
of Metamath\index{Metamath} as ``metavariables,''\index{metavariable} because
they are not quite the same as the variables we are introducing here.  A
(meta)variable in Metamath can be a wff or an individual variable, as well
as many other things; in general, it represents a kind of place holder for an
unspecified sequence of math symbols\index{math symbol}.

Unlike propositional calculus, no decision procedure\index{decision procedure}
analogous to the truth table method exists (nor theoretically can exist) that
will definitely determine whether a formula is a theorem of predicate
calculus.  Much of the work in the field of automated theorem
proving\index{automated theorem proving} has been dedicated to coming up with
clever heuristics for proving theorems of predicate calculus, but they can
never be guaranteed to work always.

Section \ref{metadefpred} discusses various definitions
that make predicate calculus easier to use.
For example, we define
$\exists x \varphi$ to mean
``there exists at least one possible value of $x$ where $\varphi$ is true.''

We now turn to looking at how predicate calculus can be formally
represented.

\subsubsection{Common Axioms}

There is a new rule of inference in predicate calculus:  if $\varphi$ is
an axiom or a theorem, then $\forall x \,\varphi$ is also a
theorem\index{theorem}.  This is called the rule of
``generalization.''\index{rule of generalization}
This is easily represented in Metamath.

In standard texts of logic, there are often two axioms of predicate
calculus\index{axioms of predicate calculus}:
\begin{center}
  $\forall x \,\varphi ( x ) \rightarrow \varphi ( y )$,
      where ``$y$ is properly substituted for $x$.''\\
  $\forall x ( \varphi \rightarrow \psi )\rightarrow ( \varphi \rightarrow
    \forall x\, \psi )$,
    where ``$x$ is not free in $\varphi$.''
\end{center}

Now at first glance, this seems simple:  just two axioms.  However,
conditional clauses are attached to each axiom describing requirements that
may seem puzzling to you.  In addition, the first axiom puts a variable symbol
in parentheses after each wff, seemingly violating our definition of a
wff\index{well-formed formula (wff)}; this is just an informal way of
referring to some arbitrary variable that may occur in the wff.  The
conditional clauses do, of course, have a precise meaning, but as it turns out
the precise meaning is somewhat complicated and awkward to formalize in a
way that a computer can handle easily.  Unlike propositional calculus, a
certain amount of mathematical sophistication and practice is needed to be
able to easily grasp and manipulate these concepts correctly.

Predicate calculus may be presented with or without axioms for
equality\index{axioms of equality}\index{equality ($=$)}. We will require the
axioms of equality as a prerequisite for the version of set theory we will
use.  The axioms for equality, when included, are often represented using these
two axioms:
\begin{center}
$x=x$\\ \ \\
$x=y\rightarrow (\varphi(x,x)\rightarrow\varphi(x,y))$ where ``$\varphi(x,y)$
   arises from $\varphi(x,x)$ by replacing some, but not necessarily all,
   free\index{free variable}
   occurrences of $x$ by $y$,\\ provided that $y$ is free for $x$
   in $\varphi(x,x)$.'' \end{center}
% (Mendelson p. 95)
The first equality axiom is simple, but again,
the condition on the second one is
somewhat awkward to implement on a computer.

\subsubsection{Tarski System S2}

Of course, we are not the first to notice the complications of these
predicate calculus axioms when being rigorous.

Well-known logician Alfred Tarski published in 1965
a system he called system S2\cite[p.~77]{Tarski1965}.
Tarski's system is \textit{exactly equivalent} to the traditional textbook
formalization, but (by clever use of equality axioms) it eliminates the
latter's primitive notions of ``proper substitution'' and ``free variable,''
replacing them with direct substitution and the notion of a variable
not occurring in a formula (which we express with distinct variable
constraints).

In advocating his system, Tarski wrote, ``The relatively complicated
character of [free variables and proper substitution] is a source
of certain inconveniences of both practical and theoretical nature;
this is clearly experienced both in teaching an elementary course of
mathematical logic and in formalizing the syntax of predicate logic for
some theoretical purposes''\cite[p.~61]{Tarski1965}\index{Tarski, Alfred}.

\subsubsection{Developing a Metamath Representation}

The standard textbook axioms of predicate calculus are somewhat
cumbersome to implement on a computer because of the complex notions of
``free variable''\index{free variable} and ``proper
substitution.''\index{proper substitution}\index{substitution!proper}
While it is possible to use the Metamath\index{Metamath} language to
implement these concepts, we have chosen not to implement them
as primitive constructs in the
\texttt{set.mm} set theory database.  Instead, we have eliminated them
within the axioms
by carefully crafting the axioms so as to avoid them,
building on Tarski's system S2.  This makes it
easy for a beginner to follow the steps in a proof without knowing any
advanced concepts other than the simple concept of
replacing\index{substitution!variable}\index{variable substitution}
variables with expressions.

In order to develop the concepts of free variable and proper
substitution from the axioms, we use an additional
Metamath statement type called ``disjoint variable
restriction''\index{disjoint variables} that we have not encountered
before.  In the context of the axioms, the statement \texttt{\$d} $ x\,
y$\index{\texttt{\$d} statement} simply means that $x$ and $y$ must be
distinct\index{distinct variables}, i.e.\ they may not be simultaneously
substituted\index{substitution!variable}\index{variable substitution}
with the same variable.  The statement \texttt{\$d} $ x\, \varphi$ means
variable $x$ must not occur in wff $\varphi$.  For the precise
definition of \texttt{\$d}, see Section~\ref{dollard}.

\subsubsection{Metamath representation}

The Metamath axiom system for predicate calculus
defined in set.mm uses Tarski's system S2.
As noted above, this has a different representation
than the traditional textbook formalization,
but it is \textit{exactly equivalent} to the textbook formalization,
and it is \textit{much} easier to work with.
This is reproduced as system S3 in Section 6 of
Megill's formalization \cite{Megill}\index{Megill, Norman}.

There is one exception, Tarski's axiom of existence,
which we label as axiom ax-6.
In the case of ax-6, Tarski's version is weaker because it includes a
distinct variable proviso. If we wish, we can also weaken our version
in this way and still have a metalogically complete system. Theorem
ax6 shows this by deriving, in the presence of the other axioms, our
ax-6 from Tarski's weaker version ax6v. However, we chose the stronger
version for our system because it is simpler to state and easier to use.

Tarski's system was designed for proving specific theorems rather than
more general theorem schemes. However, theorem schemes are much more
efficient than specific theorems for building a body of mathematical
knowledge, since they can be reused with different instances as
needed. While Tarski does derive some theorem schemes from his axioms,
their proofs require concepts that are ``outside'' of the system, such as
induction on formula length. The verification of such proofs is difficult
to automate in a proof verifier. (Specifically, Tarski treats the formulas
of his system as set-theoretical objects. In order to verify the proofs
of his theorem schemes, a proof verifier would need a significant amount
of set theory built into it.)

The Metamath axiom system for predicate calculus extends
Tarski's system to eliminate this difficulty. The additional
``auxilliary'' axiom
schemes (as we will call them in this section; see below) endow Tarski's
system with a nice property we call
metalogical completeness \cite[Remark 9.6]{Megill}\index{Megill, Norman}.
As a result, we can prove any theorem scheme
expressable in the ``simple metalogic'' of Tarski's system by using
only Metamath's direct substitution rule applied to the axiom system
(and no other metalogical or set-theoretical notions ``outside'' of the
system). Simple metalogic consists of schemes containing wff metavariables
(with no arguments) and/or set (also called ``individual'') metavariables,
accompanied by optional provisos each stating that two specified set
metavariables must be distinct or that a specified set metavariable may
not occur in a specified wff metavariable. Metamath's logic and set theory
axiom and rule schemes are all examples of simple metalogic. The schemes
of traditional predicate calculus with equality are examples which are
not simple metalogic, because they use wff metavariables with arguments
and have ``free for'' and ``not free in'' side conditions.

A rigorous justification for this system, using an older but
exactly equivalent set of axioms, can be
found in \cite{Megill}\index{Megill, Norman}.

This allows us to
take a different approach in the Metamath\index{Metamath} database
\texttt{set.mm}\index{set theory database (\texttt{set.mm})}.  We do not
directly use the primitive notions of ``free variable''\index{free variable}
and ``proper substitution''\index{proper
substitution}\index{substitution!proper} at all as primitive constructs.
Instead, we use a set
of axioms that are almost as simple to manipulate as those of
propositional calculus.  Our axiom system avoids complex primitive
notions by effectively embedding the complexity into the axioms
themselves.  As a result, we will end up with a larger number of axioms,
but they are ideally suited for a computer language such as Metamath.
(Section~\ref{metaaxioms} shows these axioms.)

We will not elaborate further
on the ``free variable'' and ``proper substitution''
concepts here.  You may consult
\cite[ch.\ 3--4]{Hamilton}\index{Hamilton, Alan G.} (as well as
many other books) for a precise explanation
of these concepts.  If you intend to do serious mathematical work, it is wise
to become familiar with the traditional textbook approach; even though the
concepts embedded in their axioms require a higher level of sophistication,
they can be more practical to deal with on an everyday, informal basis.  Even
if you are just developing Metamath proofs, familiarity with the traditional
approach can help you arrive at a proof outline much faster, which you can
then convert to the detail required by Metamath.

We do develop proper substitution rules later on, but in set.mm
they are defined as derived constructs; they are not primitives.

You should also note that our system of predicate calculus is specifically
tailored for set theory; thus there are only two specific predicates $=$ and
$\in$ and no functions\index{function!in predicate calculus}
or constants\index{constant!in predicate calculus} unlike more general systems.
We later add these.

\subsection{Set Theory}

Traditional Zermelo--Fraenkel set theory\index{Zermelo--Fraenkel set
theory}\index{set theory} with the Axiom of Choice
has 10 axioms, which can be expressed in the
language of predicate calculus.  In this section, we will list only the
names and brief English descriptions of these axioms, since we will give
you the precise formulas used by the Metamath\index{Metamath} set theory
database \texttt{set.mm} later on.

In the descriptions of the axioms, we assume that $x$, $y$, $z$, $w$, and $v$
represent sets.  These are the same as the variables\index{variable!in set
theory} in our predicate calculus system above, except that now we informally
think of the variables as ranging over sets.  Note that the terms
``object,''\index{object} ``set,''\index{set} ``element,''\index{element}
``collection,''\index{collection} and ``family''\index{family} are synonymous,
as are ``is an element of,'' ``is a member of,''\index{member} ``is contained
in,'' and ``belongs to.''  The different terms are used for convenience; for
example, ``a collection of sets'' is less confusing than ``a set of sets.''
A set $x$ is said to be a ``subset''\index{subset} of $y$ if every element of
$x$ is also an element of $y$; we also say $x$ is ``included in''
$y$.

The axioms are very general and apply to almost any conceivable mathematical
object, and this level of abstraction can be overwhelming at first.  To gain an
intuitive feel, it can be helpful to draw a picture illustrating the concept;
for example, a circle containing dots could represent a collection of sets,
and a smaller circle drawn inside the circle could represent a subset.
Overlapping circles can illustrate intersection and union.  Circles that
illustrate the concepts of set theory are frequently used in elementary
textbooks and are called Venn diagrams\index{Venn diagram}.\index{axioms of
set theory}

1. Axiom of Extensionality:  Two sets are identical if they contain the same
   elements.\index{Axiom of Extensionality}

2. Axiom of Pairing:  The set $\{ x , y \}$ exists.\index{Axiom of Pairing}

3. Axiom of Power Sets:  The power set of a set (the collection of all of
   its subsets) exists.  For example, the power set of $\{x,y\}$ is
   $\{\varnothing,\{x\},\{y\},\{x,y\}\}$ and it exists.\index{Axiom
of Power Sets}

4. Axiom of the Null Set:  The empty set $\varnothing$ exists.\index{Axiom of
the Null Set}

5. Axiom of Union:  The union of a set (the set containing the elements of
   its members) exists.  For example, the union of $\{\{x,y\},\{z\}\}$ is
 $\{x,y,z\}$ and
   it exists.\index{Axiom of Union}

6. Axiom of Regularity:  Roughly, no set can contain itself, nor can there
   be membership ``loops,'' such as a set being an
   element of one of its members.\index{Axiom of Regularity}

7. Axiom of Infinity:  An infinite set exists.  An example of an infinite
   set is the set of all
   integers.\index{Axiom of Infinity}

8. Axiom of Separation:  The set exists that is obtained by restricting $x$
   with some property.  For example, if the set of all integers exists,
   then the set of all even integers exists.\index{Axiom of Separation}

9. Axiom of Replacement:  The range of a function whose domain is restricted
   to the elements of a set $x$, is also a set.  For example, there
   is a function
   from integers (the function's domain) to their squares (its
   range).  If we
   restrict the domain to even integers, its range will become the set of
   squares of even integers, so this axiom asserts that the set of
    squares of even numbers exists.  Technical note:  In general, the
   ``function'' need not be a set but can be a proper class.
   \index{Axiom of Replacement}

10. Axiom of Choice:  Let $x$ be a set whose members are pairwise
  disjoint\index{disjoint sets} (i.e,
  whose members contain no elements in common).  Then there exists another
  set containing one element from each member of $x$.  For
  example, if $x$ is
  $\{\{y,z\},\{w,v\}\}$, where $y$, $z$, $w$, and $v$ are
  different sets, then a set such as $\{z,w\}$
  exists (but the axiom doesn't tell
  us which one).  (Actually the Axiom
  of Choice is redundant if the set $x$, as in this example, has a finite
  number of elements.)\index{Axiom of Choice}

The Axiom of Choice is usually considered an extension of ZF set theory rather
than a proper part of it.  It is sometimes considered philosophically
controversial because it specifies the existence of a set without specifying
what the set is. Constructive logics, including intuitionistic logic,
do not accept the axiom of choice.
Since there is some lingering controversy, we often prefer proofs that do
not use the axiom of choice (where there is a known alternative), and
in some cases we will use weaker axioms than the full axiom of choice.
That said, the axiom of choice is a powerful and widely-accepted tool,
so we do use it when needed.
ZF set theory that includes the Axiom of Choice is
called Zermelo--Fraenkel set theory with choice (ZFC\index{ZFC set theory}).

When expressed symbolically, the Axiom of Separation and the Axiom of
Replacement contain wff symbols and therefore each represent infinitely many
axioms, one for each possible wff. For this reason, they are often called
axiom schemes\index{axiom scheme}\index{well-formed formula (wff)}.

It turns out that the Axiom of the Null Set, the Axiom of Pairing, and the
Axiom of Separation can be derived from the other axioms and are therefore
unnecessary, although they tend to be included in standard texts for various
reasons (historical, philosophical, and possibly because some authors may not
know this).  In the Metamath\index{Metamath} set theory database, these
redundant axioms are derived from the other ones instead of truly
being considered axioms.
This is in keeping with our general goal of minimizing the number of
axioms we must depend on.

\subsection{Other Axioms}

Above we qualified the phrase "all of mathematics" with "essentially."
The main important missing piece is the ability to do category theory,
which requires huge sets (inaccessible cardinals) larger than those
postulated by the ZFC axioms. The Tarski--Grothendieck Axiom postulates
the existence of such sets.
Note that this is the same axiom used by Mizar for supporting
category theory.
The Tarski--Grothendieck axiom
can be viewed as a very strong replacement of the Axiom of Infinity,
the Axiom of Choice, and the Axiom of Power Sets.
The \texttt{set.mm} database includes this axiom; see the database
for details about it.
Again, we only use this axiom when we need to.
You are only likely to encounter or use this axiom if you are doing
category theory, since its use is highly specialized,
so we will not list the Tarsky-Grothendieck axiom
in the short list of axioms below.

Can there be even more axioms?
Of course.
G\"{o}del showed that no finite set of axioms or axiom schemes can completely
describe any consistent theory strong enough to include arithmetic.
But practically speaking, the ones above are the accepted foundation that
almost all mathematicians explicitly or implicitly base their work on.

\section{The Axioms in the Metamath Language}\label{metaaxioms}

Here we list the axioms as they appear in
\texttt{set.mm}\index{set theory database (\texttt{set.mm})} so you can
look them up there easily.  Incidentally, the \texttt{show statement
/tex} command\index{\texttt{show statement} command} was used to
typeset them.

%macros from show statement /tex
\newbox\mlinebox
\newbox\mtrialbox
\newbox\startprefix  % Prefix for first line of a formula
\newbox\contprefix  % Prefix for continuation line of a formula
\def\startm{  % Initialize formula line
  \setbox\mlinebox=\hbox{\unhcopy\startprefix}
}
\def\m#1{  % Add a symbol to the formula
  \setbox\mtrialbox=\hbox{\unhcopy\mlinebox $\,#1$}
  \ifdim\wd\mtrialbox>\hsize
    \box\mlinebox
    \setbox\mlinebox=\hbox{\unhcopy\contprefix $\,#1$}
  \else
    \setbox\mlinebox=\hbox{\unhbox\mtrialbox}
  \fi
}
\def\endm{  % Output the last line of a formula
  \box\mlinebox
}

% \SLASH for \ , \TOR for \/ (text OR), \TAND for /\ (text and)
% This embeds a following forced space to force the space.
\newcommand\SLASH{\char`\\~}
\newcommand\TOR{\char`\\/~}
\newcommand\TAND{/\char`\\~}
%
% Macro to output metamath raw text.
% This assumes \startprefix and \contprefix are set.
% NOTE: "\" is tricky to escape, use \SLASH, \TOR, and \TAND inside.
% Any use of "$ { ~ ^" must be escaped; ~ and ^ must be escaped specially.
% We escape { and } for consistency.
% For more about how this macro written, see:
% https://stackoverflow.com/questions/4073674/
% how-to-disable-indentation-in-particular-section-in-latex/4075706
% Use frenchspacing, or "e." will get an extra space after it.
\newlength\mystoreparindent
\newlength\mystorehangindent
\newenvironment{mmraw}{%
\setlength{\mystoreparindent}{\the\parindent}
\setlength{\mystorehangindent}{\the\hangindent}
\setlength{\parindent}{0pt} % TODO - we'll put in the \startprefix instead
\setlength{\hangindent}{\wd\the\contprefix}
\begin{flushleft}
\begin{frenchspacing}
\begin{tt}
{\unhcopy\startprefix}%
}{%
\end{tt}
\end{frenchspacing}
\end{flushleft}
\setlength{\parindent}{\mystoreparindent}
\setlength{\hangindent}{\mystorehangindent}
\vskip 1ex
}

\needspace{5\baselineskip}
\subsection{Propositional Calculus}\label{propcalc}\index{axioms of
propositional calculus}

\needspace{2\baselineskip}
Axiom of Simplification.\label{ax1}

\setbox\startprefix=\hbox{\tt \ \ ax-1\ \$a\ }
\setbox\contprefix=\hbox{\tt \ \ \ \ \ \ \ \ \ \ }
\startm
\m{\vdash}\m{(}\m{\varphi}\m{\rightarrow}\m{(}\m{\psi}\m{\rightarrow}\m{\varphi}\m{)}
\m{)}
\endm

\needspace{3\baselineskip}
\noindent Axiom of Distribution.

\setbox\startprefix=\hbox{\tt \ \ ax-2\ \$a\ }
\setbox\contprefix=\hbox{\tt \ \ \ \ \ \ \ \ \ \ }
\startm
\m{\vdash}\m{(}\m{(}\m{\varphi}\m{\rightarrow}\m{(}\m{\psi}\m{\rightarrow}\m{\chi}
\m{)}\m{)}\m{\rightarrow}\m{(}\m{(}\m{\varphi}\m{\rightarrow}\m{\psi}\m{)}\m{
\rightarrow}\m{(}\m{\varphi}\m{\rightarrow}\m{\chi}\m{)}\m{)}\m{)}
\endm

\needspace{2\baselineskip}
\noindent Axiom of Contraposition.

\setbox\startprefix=\hbox{\tt \ \ ax-3\ \$a\ }
\setbox\contprefix=\hbox{\tt \ \ \ \ \ \ \ \ \ \ }
\startm
\m{\vdash}\m{(}\m{(}\m{\lnot}\m{\varphi}\m{\rightarrow}\m{\lnot}\m{\psi}\m{)}\m{
\rightarrow}\m{(}\m{\psi}\m{\rightarrow}\m{\varphi}\m{)}\m{)}
\endm


\needspace{4\baselineskip}
\noindent Rule of Modus Ponens.\label{axmp}\index{modus ponens}

\setbox\startprefix=\hbox{\tt \ \ maj\ \$e\ }
\setbox\contprefix=\hbox{\tt \ \ \ \ \ \ \ \ \ }
\startm
\m{\vdash}\m{(}\m{\varphi}\m{\rightarrow}\m{\psi}\m{)}
\endm

\setbox\startprefix=\hbox{\tt \ \ min\ \$e\ }
\setbox\contprefix=\hbox{\tt \ \ \ \ \ \ \ \ \ }
\startm
\m{\vdash}\m{\varphi}
\endm

\setbox\startprefix=\hbox{\tt \ \ ax-mp\ \$a\ }
\setbox\contprefix=\hbox{\tt \ \ \ \ \ \ \ \ \ \ \ }
\startm
\m{\vdash}\m{\psi}
\endm


\needspace{7\baselineskip}
\subsection{Axioms of Predicate Calculus with Equality---Tarski's S2}\index{axioms of predicate calculus}

\needspace{3\baselineskip}
\noindent Rule of Generalization.\index{rule of generalization}

\setbox\startprefix=\hbox{\tt \ \ ax-g.1\ \$e\ }
\setbox\contprefix=\hbox{\tt \ \ \ \ \ \ \ \ \ \ \ \ }
\startm
\m{\vdash}\m{\varphi}
\endm

\setbox\startprefix=\hbox{\tt \ \ ax-gen\ \$a\ }
\setbox\contprefix=\hbox{\tt \ \ \ \ \ \ \ \ \ \ \ \ }
\startm
\m{\vdash}\m{\forall}\m{x}\m{\varphi}
\endm

\needspace{2\baselineskip}
\noindent Axiom of Quantified Implication.

\setbox\startprefix=\hbox{\tt \ \ ax-4\ \$a\ }
\setbox\contprefix=\hbox{\tt \ \ \ \ \ \ \ \ \ \ }
\startm
\m{\vdash}\m{(}\m{\forall}\m{x}\m{(}\m{\forall}\m{x}\m{\varphi}\m{\rightarrow}\m{
\psi}\m{)}\m{\rightarrow}\m{(}\m{\forall}\m{x}\m{\varphi}\m{\rightarrow}\m{
\forall}\m{x}\m{\psi}\m{)}\m{)}
\endm

\needspace{3\baselineskip}
\noindent Axiom of Distinctness.

% Aka: Add $d x ph $.
\setbox\startprefix=\hbox{\tt \ \ ax-5\ \$a\ }
\setbox\contprefix=\hbox{\tt \ \ \ \ \ \ \ \ \ \ }
\startm
\m{\vdash}\m{(}\m{\varphi}\m{\rightarrow}\m{\forall}\m{x}\m{\varphi}\m{)}\m{where}\m{ }\m{\$d}\m{ }\m{x}\m{ }\m{\varphi}\m{ }\m{(}\m{x}\m{ }\m{does}\m{ }\m{not}\m{ }\m{occur}\m{ }\m{in}\m{ }\m{\varphi}\m{)}
\endm

\needspace{2\baselineskip}
\noindent Axiom of Existence.

\setbox\startprefix=\hbox{\tt \ \ ax-6\ \$a\ }
\setbox\contprefix=\hbox{\tt \ \ \ \ \ \ \ \ \ \ }
\startm
\m{\vdash}\m{(}\m{\forall}\m{x}\m{(}\m{x}\m{=}\m{y}\m{\rightarrow}\m{\forall}
\m{x}\m{\varphi}\m{)}\m{\rightarrow}\m{\varphi}\m{)}
\endm

\needspace{2\baselineskip}
\noindent Axiom of Equality.

\setbox\startprefix=\hbox{\tt \ \ ax-7\ \$a\ }
\setbox\contprefix=\hbox{\tt \ \ \ \ \ \ \ \ \ \ }
\startm
\m{\vdash}\m{(}\m{x}\m{=}\m{y}\m{\rightarrow}\m{(}\m{x}\m{=}\m{z}\m{
\rightarrow}\m{y}\m{=}\m{z}\m{)}\m{)}
\endm

\needspace{2\baselineskip}
\noindent Axiom of Left Equality for Binary Predicate.

\setbox\startprefix=\hbox{\tt \ \ ax-8\ \$a\ }
\setbox\contprefix=\hbox{\tt \ \ \ \ \ \ \ \ \ \ \ }
\startm
\m{\vdash}\m{(}\m{x}\m{=}\m{y}\m{\rightarrow}\m{(}\m{x}\m{\in}\m{z}\m{
\rightarrow}\m{y}\m{\in}\m{z}\m{)}\m{)}
\endm

\needspace{2\baselineskip}
\noindent Axiom of Right Equality for Binary Predicate.

\setbox\startprefix=\hbox{\tt \ \ ax-9\ \$a\ }
\setbox\contprefix=\hbox{\tt \ \ \ \ \ \ \ \ \ \ \ }
\startm
\m{\vdash}\m{(}\m{x}\m{=}\m{y}\m{\rightarrow}\m{(}\m{z}\m{\in}\m{x}\m{
\rightarrow}\m{z}\m{\in}\m{y}\m{)}\m{)}
\endm


\needspace{4\baselineskip}
\subsection{Axioms of Predicate Calculus with Equality---Auxiliary}\index{axioms of predicate calculus - auxiliary}

\needspace{2\baselineskip}
\noindent Axiom of Quantified Negation.

\setbox\startprefix=\hbox{\tt \ \ ax-10\ \$a\ }
\setbox\contprefix=\hbox{\tt \ \ \ \ \ \ \ \ \ \ }
\startm
\m{\vdash}\m{(}\m{\lnot}\m{\forall}\m{x}\m{\lnot}\m{\forall}\m{x}\m{\varphi}\m{
\rightarrow}\m{\varphi}\m{)}
\endm

\needspace{2\baselineskip}
\noindent Axiom of Quantifier Commutation.

\setbox\startprefix=\hbox{\tt \ \ ax-11\ \$a\ }
\setbox\contprefix=\hbox{\tt \ \ \ \ \ \ \ \ \ \ }
\startm
\m{\vdash}\m{(}\m{\forall}\m{x}\m{\forall}\m{y}\m{\varphi}\m{\rightarrow}\m{
\forall}\m{y}\m{\forall}\m{x}\m{\varphi}\m{)}
\endm

\needspace{3\baselineskip}
\noindent Axiom of Substitution.

\setbox\startprefix=\hbox{\tt \ \ ax-12\ \$a\ }
\setbox\contprefix=\hbox{\tt \ \ \ \ \ \ \ \ \ \ \ }
\startm
\m{\vdash}\m{(}\m{\lnot}\m{\forall}\m{x}\m{\,x}\m{=}\m{y}\m{\rightarrow}\m{(}
\m{x}\m{=}\m{y}\m{\rightarrow}\m{(}\m{\varphi}\m{\rightarrow}\m{\forall}\m{x}\m{(}
\m{x}\m{=}\m{y}\m{\rightarrow}\m{\varphi}\m{)}\m{)}\m{)}\m{)}
\endm

\needspace{3\baselineskip}
\noindent Axiom of Quantified Equality.

\setbox\startprefix=\hbox{\tt \ \ ax-13\ \$a\ }
\setbox\contprefix=\hbox{\tt \ \ \ \ \ \ \ \ \ \ \ }
\startm
\m{\vdash}\m{(}\m{\lnot}\m{\forall}\m{z}\m{\,z}\m{=}\m{x}\m{\rightarrow}\m{(}
\m{\lnot}\m{\forall}\m{z}\m{\,z}\m{=}\m{y}\m{\rightarrow}\m{(}\m{x}\m{=}\m{y}
\m{\rightarrow}\m{\forall}\m{z}\m{\,x}\m{=}\m{y}\m{)}\m{)}\m{)}
\endm

% \noindent Axiom of Quantifier Substitution
%
% \setbox\startprefix=\hbox{\tt \ \ ax-c11n\ \$a\ }
% \setbox\contprefix=\hbox{\tt \ \ \ \ \ \ \ \ \ \ \ }
% \startm
% \m{\vdash}\m{(}\m{\forall}\m{x}\m{\,x}\m{=}\m{y}\m{\rightarrow}\m{(}\m{\forall}
% \m{x}\m{\varphi}\m{\rightarrow}\m{\forall}\m{y}\m{\varphi}\m{)}\m{)}
% \endm
%
% \noindent Axiom of Distinct Variables. (This axiom requires
% that two individual variables
% be distinct\index{\texttt{\$d} statement}\index{distinct
% variables}.)
%
% \setbox\startprefix=\hbox{\tt \ \ \ \ \ \ \ \ \$d\ }
% \setbox\contprefix=\hbox{\tt \ \ \ \ \ \ \ \ \ \ \ }
% \startm
% \m{x}\m{\,}\m{y}
% \endm
%
% \setbox\startprefix=\hbox{\tt \ \ ax-c16\ \$a\ }
% \setbox\contprefix=\hbox{\tt \ \ \ \ \ \ \ \ \ \ \ }
% \startm
% \m{\vdash}\m{(}\m{\forall}\m{x}\m{\,x}\m{=}\m{y}\m{\rightarrow}\m{(}\m{\varphi}\m{
% \rightarrow}\m{\forall}\m{x}\m{\varphi}\m{)}\m{)}
% \endm

% \noindent Axiom of Quantifier Introduction (2).  (This axiom requires
% that the individual variable not occur in the
% wff\index{\texttt{\$d} statement}\index{distinct variables}.)
%
% \setbox\startprefix=\hbox{\tt \ \ \ \ \ \ \ \ \$d\ }
% \setbox\contprefix=\hbox{\tt \ \ \ \ \ \ \ \ \ \ \ }
% \startm
% \m{x}\m{\,}\m{\varphi}
% \endm
% \setbox\startprefix=\hbox{\tt \ \ ax-5\ \$a\ }
% \setbox\contprefix=\hbox{\tt \ \ \ \ \ \ \ \ \ \ \ }
% \startm
% \m{\vdash}\m{(}\m{\varphi}\m{\rightarrow}\m{\forall}\m{x}\m{\varphi}\m{)}
% \endm

\subsection{Set Theory}\label{mmsettheoryaxioms}

In order to make the axioms of set theory\index{axioms of set theory} a little
more compact, there are several definitions from logic that we make use of
implicitly, namely, ``logical {\sc and},''\index{conjunction ($\wedge$)}
\index{logical {\sc and} ($\wedge$)} ``logical equivalence,''\index{logical
equivalence ($\leftrightarrow$)}\index{biconditional ($\leftrightarrow$)} and
``there exists.''\index{existential quantifier ($\exists$)}

\begin{center}\begin{tabular}{rcl}
  $( \varphi \wedge \psi )$ &\mbox{stands for}& $\neg ( \varphi
     \rightarrow \neg \psi )$\\
  $( \varphi \leftrightarrow \psi )$& \mbox{stands
     for}& $( ( \varphi \rightarrow \psi ) \wedge
     ( \psi \rightarrow \varphi ) )$\\
  $\exists x \,\varphi$ &\mbox{stands for}& $\neg \forall x \neg \varphi$
\end{tabular}\end{center}

In addition, the axioms of set theory require that all variables be
dis\-tinct,\index{distinct variables}\footnote{Set theory axioms can be
devised so that {\em no} variables are required to be distinct,
provided we replace \texttt{ax-c16} with an axiom stating that ``at
least two things exist,'' thus
making \texttt{ax-5} the only other axiom requiring the
\texttt{\$d} statement.  These axioms are unconventional and are not
presented here, but they can be found on the \url{http://metamath.org}
web site.  See also the Comment on
p.~\pageref{nodd}.}\index{\texttt{\$d} statement} thus we also assume:
\begin{center}
  \texttt{\$d }$x\,y\,z\,w$
\end{center}

\needspace{2\baselineskip}
\noindent Axiom of Extensionality.\index{Axiom of Extensionality}

\setbox\startprefix=\hbox{\tt \ \ ax-ext\ \$a\ }
\setbox\contprefix=\hbox{\tt \ \ \ \ \ \ \ \ \ \ \ \ }
\startm
\m{\vdash}\m{(}\m{\forall}\m{x}\m{(}\m{x}\m{\in}\m{y}\m{\leftrightarrow}\m{x}
\m{\in}\m{z}\m{)}\m{\rightarrow}\m{y}\m{=}\m{z}\m{)}
\endm

\needspace{3\baselineskip}
\noindent Axiom of Replacement.\index{Axiom of Replacement}

\setbox\startprefix=\hbox{\tt \ \ ax-rep\ \$a\ }
\setbox\contprefix=\hbox{\tt \ \ \ \ \ \ \ \ \ \ \ \ }
\startm
\m{\vdash}\m{(}\m{\forall}\m{w}\m{\exists}\m{y}\m{\forall}\m{z}\m{(}\m{%
\forall}\m{y}\m{\varphi}\m{\rightarrow}\m{z}\m{=}\m{y}\m{)}\m{\rightarrow}\m{%
\exists}\m{y}\m{\forall}\m{z}\m{(}\m{z}\m{\in}\m{y}\m{\leftrightarrow}\m{%
\exists}\m{w}\m{(}\m{w}\m{\in}\m{x}\m{\wedge}\m{\forall}\m{y}\m{\varphi}\m{)}%
\m{)}\m{)}
\endm

\needspace{2\baselineskip}
\noindent Axiom of Union.\index{Axiom of Union}

\setbox\startprefix=\hbox{\tt \ \ ax-un\ \$a\ }
\setbox\contprefix=\hbox{\tt \ \ \ \ \ \ \ \ \ \ \ }
\startm
\m{\vdash}\m{\exists}\m{x}\m{\forall}\m{y}\m{(}\m{\exists}\m{x}\m{(}\m{y}\m{
\in}\m{x}\m{\wedge}\m{x}\m{\in}\m{z}\m{)}\m{\rightarrow}\m{y}\m{\in}\m{x}\m{)}
\endm

\needspace{2\baselineskip}
\noindent Axiom of Power Sets.\index{Axiom of Power Sets}

\setbox\startprefix=\hbox{\tt \ \ ax-pow\ \$a\ }
\setbox\contprefix=\hbox{\tt \ \ \ \ \ \ \ \ \ \ \ \ }
\startm
\m{\vdash}\m{\exists}\m{x}\m{\forall}\m{y}\m{(}\m{\forall}\m{x}\m{(}\m{x}\m{
\in}\m{y}\m{\rightarrow}\m{x}\m{\in}\m{z}\m{)}\m{\rightarrow}\m{y}\m{\in}\m{x}
\m{)}
\endm

\needspace{3\baselineskip}
\noindent Axiom of Regularity.\index{Axiom of Regularity}

\setbox\startprefix=\hbox{\tt \ \ ax-reg\ \$a\ }
\setbox\contprefix=\hbox{\tt \ \ \ \ \ \ \ \ \ \ \ \ }
\startm
\m{\vdash}\m{(}\m{\exists}\m{x}\m{\,x}\m{\in}\m{y}\m{\rightarrow}\m{\exists}
\m{x}\m{(}\m{x}\m{\in}\m{y}\m{\wedge}\m{\forall}\m{z}\m{(}\m{z}\m{\in}\m{x}\m{
\rightarrow}\m{\lnot}\m{z}\m{\in}\m{y}\m{)}\m{)}\m{)}
\endm

\needspace{3\baselineskip}
\noindent Axiom of Infinity.\index{Axiom of Infinity}

\setbox\startprefix=\hbox{\tt \ \ ax-inf\ \$a\ }
\setbox\contprefix=\hbox{\tt \ \ \ \ \ \ \ \ \ \ \ \ \ \ \ }
\startm
\m{\vdash}\m{\exists}\m{x}\m{(}\m{y}\m{\in}\m{x}\m{\wedge}\m{\forall}\m{y}%
\m{(}\m{y}\m{\in}\m{x}\m{\rightarrow}\m{\exists}\m{z}\m{(}\m{y}\m{\in}\m{z}\m{%
\wedge}\m{z}\m{\in}\m{x}\m{)}\m{)}\m{)}
\endm

\needspace{4\baselineskip}
\noindent Axiom of Choice.\index{Axiom of Choice}

\setbox\startprefix=\hbox{\tt \ \ ax-ac\ \$a\ }
\setbox\contprefix=\hbox{\tt \ \ \ \ \ \ \ \ \ \ \ \ \ \ }
\startm
\m{\vdash}\m{\exists}\m{x}\m{\forall}\m{y}\m{\forall}\m{z}\m{(}\m{(}\m{y}\m{%
\in}\m{z}\m{\wedge}\m{z}\m{\in}\m{w}\m{)}\m{\rightarrow}\m{\exists}\m{w}\m{%
\forall}\m{y}\m{(}\m{\exists}\m{w}\m{(}\m{(}\m{y}\m{\in}\m{z}\m{\wedge}\m{z}%
\m{\in}\m{w}\m{)}\m{\wedge}\m{(}\m{y}\m{\in}\m{w}\m{\wedge}\m{w}\m{\in}\m{x}%
\m{)}\m{)}\m{\leftrightarrow}\m{y}\m{=}\m{w}\m{)}\m{)}
\endm

\subsection{That's It}

There you have it, the axioms for (essentially) all of mathematics!
Wonder at them and stare at them in awe.  Put a copy in your wallet, and
you will carry in your pocket the encoding for all theorems ever proved
and that ever will be proved, from the most mundane to the most
profound.

\section{A Hierarchy of Definitions}\label{hierarchy}

The axioms in the previous section in principle embody everything that can be
done within standard mathematics.  However, it is impractical to accomplish
very much by using them directly, for even simple concepts (from a human
perspective) can involve extremely long, incomprehensible formulas.
Mathematics is made practical by introducing definitions\index{definition}.
Definitions usually introduce new symbols, or at least new relationships among
existing symbols, to abbreviate more complex formulas.  An important
requirement for a definition is that there exist a straightforward
(algorithmic) method for eliminating the abbreviation by expanding it into the
more primitive symbol string that it represents.  Some
important definitions included in
the file \texttt{set.mm} are listed in this section for reference, and also to
give you a feel for why something like $\omega$\index{omega ($\omega$)} (the
set of natural numbers\index{natural number} 0, 1, 2,\ldots) becomes very
complicated when completely expanded into primitive symbols.

What is the motivation for definitions, aside from allowing complicated
expressions to be expressed more simply?  In the case of  $\omega$, one goal is
to provide a basis for the theory of natural numbers.\index{natural number}
Before set theory was invented, a set of axioms for arithmetic, called Peano's
postulates\index{Peano's postulates}, was devised and shown to have the
properties one expects for natural numbers.  Now anyone can postulate a
set of axioms, but if the axioms are inconsistent contradictions can be derived
from them.  Once a contradiction is derived, anything can be trivially
proved, including
all the facts of arithmetic and their negations.  To ensure that an
axiom system is at least as reliable as the axioms for set theory, we can
define sets and operations on those sets that satisfy the new axioms. In the
\texttt{set.mm} Metamath database, we prove that the elements of $\omega$ satisfy
Peano's postulates, and it's a long and hard journey to get there directly
from the axioms of set theory.  But the result is confidence in the
foundations of arithmetic.  And there is another advantage:  we now have all
the tools of set theory at our disposal for manipulating objects that obey the
axioms for arithmetic.

What are the criteria we use for definitions?  First, and of utmost importance,
the definition should not be {\em creative}\index{creative
definition}\index{definition!creative}, that
is it should not allow an expression that previously qualified as a wff but
was not provable, to become provable.   Second, the definition should be {\em
eliminable}\index{definition!eliminability}, that is, there should exist an
algorithmic method for converting any expression using the definition into
a logically equivalent expression that previously qualified as a wff.

In almost all cases below, definitions connect two expressions with either
$\leftrightarrow$ or $=$.  Eliminating\footnote{Here we mean the
elimination that a human might do in his or her head.  To eliminate them as
part of a Metamath proof we would invoke one of a number of
theorems that deal with transitivity of equivalence or equality; there are
many such examples in the proofs in \texttt{set.mm}.} such a definition is a
simple matter of substituting the expression on the left-hand side ({\em
definiendum}\index{definiendum} or thing being defined) with the equivalent,
more primitive expression on the right-hand side ({\em
definiens}\index{definiens} or definition).

Often a definition has variables on the right-hand side which do not appear on
the left-hand side; these are called {\em dummy variables}.\index{dummy
variable!in definitions}  In this case, any
allowable substitution (such as a new, distinct
variable) can be used when the definition is eliminated.  Dummy variables may
be used only if they are {\em effectively bound}\index{effectively bound
variable}, meaning that the definition will remain logically equivalent upon
any substitution of a dummy variable with any other {\em qualifying
expression}\index{qualifying expression}, i.e.\ any symbol string (such as
another variable) that
meets the restrictions on the dummy variable imposed by \texttt{\$d} and
\texttt{\$f} statements.  For example, we could define a constant $\perp$
(inverted tee, meaning logical ``false'') as $( \varphi \wedge \lnot \varphi
)$, i.e.\ ``phi and not phi.''  Here $\varphi$ is effectively bound because the
definition remains logically equivalent when we replace $\varphi$ with any
other wff.  (It is actually \texttt{df-fal}
in \texttt{set.mm}, which defines $\perp$.)

There are two cases where eliminating definitions is a little more
complex.  These cases are the definitions \texttt{df-bi} and
\texttt{df-cleq}.  The first stretches the concept of a definition a
little, as in effect it ``defines a definition;'' however, it meets our
requirements for a definition in that it is eliminable and does not
strengthen the language.  Theorem \texttt{bii} shows the substitution
needed to eliminate the $\leftrightarrow$\index{logical equivalence
($\leftrightarrow$)}\index{biconditional ($\leftrightarrow$)} symbol.

Definition \texttt{df-cleq}\index{equality ($=$)} extends the usage of
the equality symbol to include ``classes''\index{class} in set theory.  The
reason it is potentially problematic is that it can lead to statements which
do not follow from logic alone but presuppose the Axiom of
Extensionality\index{Axiom of Extensionality}, so we include this axiom
as a hypothesis for the definition.  We could have made \texttt{df-cleq} directly
eliminable by introducing a new equality symbol, but have chosen not to do so
in keeping with standard textbook practice.  Definitions such as \texttt{df-cleq}
that extend the meaning of existing symbols must be introduced carefully so
that they do not lead to contradictions.  Definition \texttt{df-clel} also
extends the meaning of an existing symbol ($\in$); while it doesn't strengthen
the language like \texttt{df-cleq}, this is not obvious and it must also be
subject to the same scrutiny.

Exercise:  Study how the wff $x\in\omega$, meaning ``$x$ is a natural
number,'' could be expanded in terms of primitive symbols, starting with the
definitions \texttt{df-clel} on p.~\pageref{dfclel} and \texttt{df-om} on
p.~\pageref{dfom} and working your way back.  Don't bother to work out the
details; just make sure that you understand how you could do it in principle.
The answer is shown in the footnote on p.~\pageref{expandom}.  If you
actually do work it out, you won't get exactly the same answer because we used
a few simplifications such as discarding occurrences of $\lnot\lnot$ (double
negation).

In the definitions below, we have placed the {\sc ascii} Metamath source
below each of the formulas to help you become familiar with the
notation in the database.  For simplicity, the necessary \texttt{\$f}
and \texttt{\$d} statements are not shown.  If you are in doubt, use the
\texttt{show statement}\index{\texttt{show statement} command} command
in the Metamath program to see the full statement.
A selection of this notation is summarized in Appendix~\ref{ASCII}.

To understand the motivation for these definitions, you should consult the
references indicated:  Takeuti and Zaring \cite{Takeuti}\index{Takeuti, Gaisi},
Quine \cite{Quine}\index{Quine, Willard Van Orman}, Bell and Machover
\cite{Bell}\index{Bell, J. L.}, and Enderton \cite{Enderton}\index{Enderton,
Herbert B.}.  Our list of definitions is provided more for reference than as a
learning aid.  However, by looking at a few of them you can gain a feel for
how the hierarchy is built up.  The definitions are a representative sample of
the 130 or so in \texttt{set.mm}, but they are complete with respect to the
theorem examples we will present in Section~\ref{sometheorems}.  Also, some are
slightly different from, but logically equivalent to, the ones in \texttt{set.mm}
(some of which have been revised over time to shorten them, for example).

\subsection{Definitions for Propositional Calculus}\label{metadefprop}

The symbols $\varphi$, $\psi$, and $\chi$ represent wffs.

Our first definition introduces the biconditional
connective\footnote{The term ``connective'' is informally used to mean a
symbol that is placed between two variables or adjacent to a variable,
whereas a mathematical ``constant'' usually indicates a symbol such as
the number 0 that may replace a variable or metavariable.  From
Metamath's point of view, there is no distinction between a connective
and a constant; both are constants in the Metamath
language.}\index{connective}\index{constant} (also called logical
equivalence)\index{logical equivalence
($\leftrightarrow$)}\index{biconditional ($\leftrightarrow$)}.  Unlike
most traditional developments, we have chosen not to have a separate
symbol such as ``Df.'' to mean ``is defined as.''  Instead, we will use
the biconditional connective for this purpose, as it lets us use
logic to manipulate definitions directly.  Here we state the properties
of the biconditional connective with a carefully crafted \texttt{\$a}
statement, which effectively uses the biconditional connective to define
itself.  The $\leftrightarrow$ symbol can be eliminated from a formula
using theorem \texttt{bii}, which is derived later.

\vskip 2ex
\noindent Define the biconditional connective.\label{df-bi}

\vskip 0.5ex
\setbox\startprefix=\hbox{\tt \ \ df-bi\ \$a\ }
\setbox\contprefix=\hbox{\tt \ \ \ \ \ \ \ \ \ \ \ }
\startm
\m{\vdash}\m{\lnot}\m{(}\m{(}\m{(}\m{\varphi}\m{\leftrightarrow}\m{\psi}\m{)}%
\m{\rightarrow}\m{\lnot}\m{(}\m{(}\m{\varphi}\m{\rightarrow}\m{\psi}\m{)}\m{%
\rightarrow}\m{\lnot}\m{(}\m{\psi}\m{\rightarrow}\m{\varphi}\m{)}\m{)}\m{)}\m{%
\rightarrow}\m{\lnot}\m{(}\m{\lnot}\m{(}\m{(}\m{\varphi}\m{\rightarrow}\m{%
\psi}\m{)}\m{\rightarrow}\m{\lnot}\m{(}\m{\psi}\m{\rightarrow}\m{\varphi}\m{)}%
\m{)}\m{\rightarrow}\m{(}\m{\varphi}\m{\leftrightarrow}\m{\psi}\m{)}\m{)}\m{)}
\endm
\begin{mmraw}%
|- -. ( ( ( ph <-> ps ) -> -. ( ( ph -> ps ) ->
-. ( ps -> ph ) ) ) -> -. ( -. ( ( ph -> ps ) -> -. (
ps -> ph ) ) -> ( ph <-> ps ) ) ) \$.
\end{mmraw}

\noindent This theorem relates the biconditional connective to primitive
connectives and can be used to eliminate the $\leftrightarrow$ symbol from any
wff.

\vskip 0.5ex
\setbox\startprefix=\hbox{\tt \ \ bii\ \$p\ }
\setbox\contprefix=\hbox{\tt \ \ \ \ \ \ \ \ \ }
\startm
\m{\vdash}\m{(}\m{(}\m{\varphi}\m{\leftrightarrow}\m{\psi}\m{)}\m{\leftrightarrow}
\m{\lnot}\m{(}\m{(}\m{\varphi}\m{\rightarrow}\m{\psi}\m{)}\m{\rightarrow}\m{\lnot}
\m{(}\m{\psi}\m{\rightarrow}\m{\varphi}\m{)}\m{)}\m{)}
\endm
\begin{mmraw}%
|- ( ( ph <-> ps ) <-> -. ( ( ph -> ps ) -> -. ( ps -> ph ) ) ) \$= ... \$.
\end{mmraw}

\noindent Define disjunction ({\sc or}).\index{disjunction ($\vee$)}%
\index{logical or (vee)@logical {\sc or} ($\vee$)}%
\index{df-or@\texttt{df-or}}\label{df-or}

\vskip 0.5ex
\setbox\startprefix=\hbox{\tt \ \ df-or\ \$a\ }
\setbox\contprefix=\hbox{\tt \ \ \ \ \ \ \ \ \ \ \ }
\startm
\m{\vdash}\m{(}\m{(}\m{\varphi}\m{\vee}\m{\psi}\m{)}\m{\leftrightarrow}\m{(}\m{
\lnot}\m{\varphi}\m{\rightarrow}\m{\psi}\m{)}\m{)}
\endm
\begin{mmraw}%
|- ( ( ph \TOR ps ) <-> ( -. ph -> ps ) ) \$.
\end{mmraw}

\noindent Define conjunction ({\sc and}).\index{conjunction ($\wedge$)}%
\index{logical {\sc and} ($\wedge$)}%
\index{df-an@\texttt{df-an}}\label{df-an}

\vskip 0.5ex
\setbox\startprefix=\hbox{\tt \ \ df-an\ \$a\ }
\setbox\contprefix=\hbox{\tt \ \ \ \ \ \ \ \ \ \ \ }
\startm
\m{\vdash}\m{(}\m{(}\m{\varphi}\m{\wedge}\m{\psi}\m{)}\m{\leftrightarrow}\m{\lnot}
\m{(}\m{\varphi}\m{\rightarrow}\m{\lnot}\m{\psi}\m{)}\m{)}
\endm
\begin{mmraw}%
|- ( ( ph \TAND ps ) <-> -. ( ph -> -. ps ) ) \$.
\end{mmraw}

\noindent Define disjunction ({\sc or}) of 3 wffs.%
\index{df-3or@\texttt{df-3or}}\label{df-3or}

\vskip 0.5ex
\setbox\startprefix=\hbox{\tt \ \ df-3or\ \$a\ }
\setbox\contprefix=\hbox{\tt \ \ \ \ \ \ \ \ \ \ \ \ }
\startm
\m{\vdash}\m{(}\m{(}\m{\varphi}\m{\vee}\m{\psi}\m{\vee}\m{\chi}\m{)}\m{
\leftrightarrow}\m{(}\m{(}\m{\varphi}\m{\vee}\m{\psi}\m{)}\m{\vee}\m{\chi}\m{)}
\m{)}
\endm
\begin{mmraw}%
|- ( ( ph \TOR ps \TOR ch ) <-> ( ( ph \TOR ps ) \TOR ch ) ) \$.
\end{mmraw}

\noindent Define conjunction ({\sc and}) of 3 wffs.%
\index{df-3an}\label{df-3an}

\vskip 0.5ex
\setbox\startprefix=\hbox{\tt \ \ df-3an\ \$a\ }
\setbox\contprefix=\hbox{\tt \ \ \ \ \ \ \ \ \ \ \ \ }
\startm
\m{\vdash}\m{(}\m{(}\m{\varphi}\m{\wedge}\m{\psi}\m{\wedge}\m{\chi}\m{)}\m{
\leftrightarrow}\m{(}\m{(}\m{\varphi}\m{\wedge}\m{\psi}\m{)}\m{\wedge}\m{\chi}
\m{)}\m{)}
\endm

\begin{mmraw}%
|- ( ( ph \TAND ps \TAND ch ) <-> ( ( ph \TAND ps ) \TAND ch ) ) \$.
\end{mmraw}

\subsection{Definitions for Predicate Calculus}\label{metadefpred}

The symbols $x$, $y$, and $z$ represent individual variables of predicate
calculus.  In this section, they are not necessarily distinct unless it is
explicitly
mentioned.

\vskip 2ex
\noindent Define existential quantification.
The expression $\exists x \varphi$ means
``there exists an $x$ where $\varphi$ is true.''\index{existential quantifier
($\exists$)}\label{df-ex}

\vskip 0.5ex
\setbox\startprefix=\hbox{\tt \ \ df-ex\ \$a\ }
\setbox\contprefix=\hbox{\tt \ \ \ \ \ \ \ \ \ \ \ }
\startm
\m{\vdash}\m{(}\m{\exists}\m{x}\m{\varphi}\m{\leftrightarrow}\m{\lnot}\m{\forall}
\m{x}\m{\lnot}\m{\varphi}\m{)}
\endm
\begin{mmraw}%
|- ( E. x ph <-> -. A. x -. ph ) \$.
\end{mmraw}

\noindent Define proper substitution.\index{proper
substitution}\index{substitution!proper}\label{df-sb}
In our notation, we use $[ y / x ] \varphi$ to mean ``the wff that
results when $y$ is properly substituted for $x$ in the wff
$\varphi$.''\footnote{
This can also be described
as substituting $x$ with $y$, $y$ properly replaces $x$, or
$x$ is properly replaced by $y$.}
% This is elsb4, though it currently says: ( [ x / y ] z e. y <-> z e. x )
For example,
$[ y / x ] z \in x$ is the same as $z \in y$.
One way to remember this notation is to notice that it looks like division
and recall that $( y / x ) \cdot x $ is $y$ (when $x \neq 0$).
The notation is different from the notation $\varphi ( x | y )$
that is sometimes used, because the latter notation is ambiguous for us:
for example, we don't know whether $\lnot \varphi ( x | y )$ is to be
interpreted as $\lnot ( \varphi ( x | y ) )$ or
$( \lnot \varphi ) ( x | y )$.\footnote{Because of the way
we initially defined wffs, this is the case
with any postfix connective\index{postfix connective} (one occurring after the
symbols being connected) or infix connective\index{infix connective} (one
occurring between the symbols being connected).  Metamath does not have a
built-in notion of operator binding strength that could eliminate the
ambiguity.  The initial parenthesis effectively provides a prefix
connective\index{prefix connective} to eliminate ambiguity.  Some conventions,
such as Polish notation\index{Polish notation} used in the 1930's and 1940's
by Polish logicians, use only prefix connectives and thus allow the total
elimination of parentheses, at the expense of readability.  In Metamath we
could actually redefine all notation to be Polish if we wanted to without
having to change any proofs!}  Other texts often use $\varphi(y)$ to indicate
our $[ y / x ] \varphi$, but this notation is even more ambiguous since there is
no explicit indication of what is being substituted.
Note that this
definition is valid even when
$x$ and $y$ are the same variable.  The first conjunct is a ``trick'' used to
achieve this property, making the definition look somewhat peculiar at
first.

\vskip 0.5ex
\setbox\startprefix=\hbox{\tt \ \ df-sb\ \$a\ }
\setbox\contprefix=\hbox{\tt \ \ \ \ \ \ \ \ \ \ \ }
\startm
\m{\vdash}\m{(}\m{[}\m{y}\m{/}\m{x}\m{]}\m{\varphi}\m{\leftrightarrow}\m{(}%
\m{(}\m{x}\m{=}\m{y}\m{\rightarrow}\m{\varphi}\m{)}\m{\wedge}\m{\exists}\m{x}%
\m{(}\m{x}\m{=}\m{y}\m{\wedge}\m{\varphi}\m{)}\m{)}\m{)}
\endm
\begin{mmraw}%
|- ( [ y / x ] ph <-> ( ( x = y -> ph ) \TAND E. x ( x = y \TAND ph ) ) ) \$.
\end{mmraw}


\noindent Define existential uniqueness\index{existential uniqueness
quantifier ($\exists "!$)} (``there exists exactly one'').  Note that $y$ is a
variable distinct from $x$ and not occurring in $\varphi$.\label{df-eu}

\vskip 0.5ex
\setbox\startprefix=\hbox{\tt \ \ df-eu\ \$a\ }
\setbox\contprefix=\hbox{\tt \ \ \ \ \ \ \ \ \ \ \ }
\startm
\m{\vdash}\m{(}\m{\exists}\m{{!}}\m{x}\m{\varphi}\m{\leftrightarrow}\m{\exists}
\m{y}\m{\forall}\m{x}\m{(}\m{\varphi}\m{\leftrightarrow}\m{x}\m{=}\m{y}\m{)}\m{)}
\endm

\begin{mmraw}%
|- ( E! x ph <-> E. y A. x ( ph <-> x = y ) ) \$.
\end{mmraw}

\subsection{Definitions for Set Theory}\label{setdefinitions}

The symbols $x$, $y$, $z$, and $w$ represent individual variables of
predicate calculus, which in set theory are understood to be sets.
However, using only the constructs shown so far would be very inconvenient.

To make set theory more practical, we introduce the notion of a ``class.''
A class\index{class} is either a set variable (such as $x$) or an
expression of the form $\{ x | \varphi\}$ (called an ``abstraction
class''\index{abstraction class}\index{class abstraction}).  Note that
sets (i.e.\ individual variables) always exist (this is a theorem of
logic, namely $\exists y \, y = x$ for any set $x$), whereas classes may
or may not exist (i.e.\ $\exists y \, y = A$ may or may not be true).
If a class does not exist it is called a ``proper class.''\index{proper
class}\index{class!proper} Definitions \texttt{df-clab},
\texttt{df-cleq}, and \texttt{df-clel} can be used to convert an
expression containing classes into one containing only set variables and
wff metavariables.

The symbols $A$, $B$, $C$, $D$, $ F$, $G$, and $R$ are metavariables that range
over classes.  A class metavariable $A$ may be eliminated from a wff by
replacing it with $\{ x|\varphi\}$ where neither $x$ nor $\varphi$ occur in
the wff.

The theory of classes can be shown to be an eliminable and conservative
extension of set theory. The \textbf{eliminability}
property shows that for every
formula in the extended language we can build a logically equivalent
formula in the basic language; so that even if the extended language
provides more ease to convey and formulate mathematical ideas for set
theory, its expressive power does not in fact strengthen the basic
language's expressive power.
The \textbf{conservation} property shows that for
every proof of a formula of the basic language in the extended system
we can build another proof of the same formula in the basic system;
so that, concerning theorems on sets only, the deductive powers of
the extended system and of the basic system are identical. Together,
these properties mean that the extended language can be treated as a
definitional extension that is \textbf{sound}.

A rigorous justification, which we will not give here, can be found in
Levy \cite[pp.~357-366]{Levy} supplementing his informal introduction to class
theory on pp.~7-17. Two other good treatments of class theory are provided
by Quine \cite[pp.~15-21]{Quine}\index{Quine, Willard Van Orman}
and also \cite[pp.~10-14]{Takeuti}\index{Takeuti, Gaisi}.
Quine's exposition (he calls them virtual classes)
is nicely written and very readable.

In the rest of this
section, individual variables are always assumed to be distinct from
each other unless otherwise indicated.  In addition, dummy variables on the
right-hand side of a definition do not occur in the class and wff
metavariables in the definition.

The definitions we present here are a partial but self-contained
collection selected from several hundred that appear in the current
\texttt{set.mm} database.  They are adequate for a basic development of
elementary set theory.

\vskip 2ex
\noindent Define the abstraction class.\index{abstraction class}\index{class
abstraction}\label{df-clab}  $x$ and $y$
need not be distinct.  Definition 2.1 of Quine, p.~16.  This definition may
seem puzzling since it is shorter than the expression being defined and does not
buy us anything in terms of brevity.  The reason we introduce this definition
is because it fits in neatly with the extension of the $\in$ connective
provided by \texttt{df-clel}.

\vskip 0.5ex
\setbox\startprefix=\hbox{\tt \ \ df-clab\ \$a\ }
\setbox\contprefix=\hbox{\tt \ \ \ \ \ \ \ \ \ \ \ \ \ }
\startm
\m{\vdash}\m{(}\m{x}\m{\in}\m{\{}\m{y}\m{|}\m{\varphi}\m{\}}\m{%
\leftrightarrow}\m{[}\m{x}\m{/}\m{y}\m{]}\m{\varphi}\m{)}
\endm
\begin{mmraw}%
|- ( x e. \{ y | ph \} <-> [ x / y ] ph ) \$.
\end{mmraw}

\noindent Define the equality connective between classes\index{class
equality}\label{df-cleq}.  See Quine or Chapter 4 of Takeuti and Zaring for its
justification and methods for eliminating it.  This is an example of a
somewhat ``dangerous'' definition, because it extends the use of the
existing equality symbol rather than introducing a new symbol, allowing
us to make statements in the original language that may not be true.
For example, it permits us to deduce $y = z \leftrightarrow \forall x (
x \in y \leftrightarrow x \in z )$ which is not a theorem of logic but
rather presupposes the Axiom of Extensionality,\index{Axiom of
Extensionality} which we include as a hypothesis so that we can know
when this axiom is assumed in a proof (with the \texttt{show
trace{\char`\_}back} command).  We could avoid the danger by introducing
another symbol, say $\eqcirc$, in place of $=$; this
would also have the advantage of making elimination of the definition
straightforward and would eliminate the need for Extensionality as a
hypothesis.  We would then also have the advantage of being able to
identify exactly where Extensionality truly comes into play.  One of our
theorems would be $x \eqcirc y \leftrightarrow x = y$
by invoking Extensionality.  However in keeping with standard practice
we retain the ``dangerous'' definition.

\vskip 0.5ex
\setbox\startprefix=\hbox{\tt \ \ df-cleq.1\ \$e\ }
\setbox\contprefix=\hbox{\tt \ \ \ \ \ \ \ \ \ \ \ \ \ \ \ }
\startm
\m{\vdash}\m{(}\m{\forall}\m{x}\m{(}\m{x}\m{\in}\m{y}\m{\leftrightarrow}\m{x}
\m{\in}\m{z}\m{)}\m{\rightarrow}\m{y}\m{=}\m{z}\m{)}
\endm
\setbox\startprefix=\hbox{\tt \ \ df-cleq\ \$a\ }
\setbox\contprefix=\hbox{\tt \ \ \ \ \ \ \ \ \ \ \ \ \ }
\startm
\m{\vdash}\m{(}\m{A}\m{=}\m{B}\m{\leftrightarrow}\m{\forall}\m{x}\m{(}\m{x}\m{
\in}\m{A}\m{\leftrightarrow}\m{x}\m{\in}\m{B}\m{)}\m{)}
\endm
% We need to reset the startprefix and contprefix.
\setbox\startprefix=\hbox{\tt \ \ df-cleq.1\ \$e\ }
\setbox\contprefix=\hbox{\tt \ \ \ \ \ \ \ \ \ \ \ \ \ \ \ }
\begin{mmraw}%
|- ( A. x ( x e. y <-> x e. z ) -> y = z ) \$.
\end{mmraw}
\setbox\startprefix=\hbox{\tt \ \ df-cleq\ \$a\ }
\setbox\contprefix=\hbox{\tt \ \ \ \ \ \ \ \ \ \ \ \ \ }
\begin{mmraw}%
|- ( A = B <-> A. x ( x e. A <-> x e. B ) ) \$.
\end{mmraw}

\noindent Define the membership connective between classes\index{class
membership}.  Theorem 6.3 of Quine, p.~41, which we adopt as a definition.
Note that it extends the use of the existing membership symbol, but unlike
{\tt df-cleq} it does not extend the set of valid wffs of logic when the class
metavariables are replaced with set variables.\label{dfclel}\label{df-clel}

\vskip 0.5ex
\setbox\startprefix=\hbox{\tt \ \ df-clel\ \$a\ }
\setbox\contprefix=\hbox{\tt \ \ \ \ \ \ \ \ \ \ \ \ \ }
\startm
\m{\vdash}\m{(}\m{A}\m{\in}\m{B}\m{\leftrightarrow}\m{\exists}\m{x}\m{(}\m{x}
\m{=}\m{A}\m{\wedge}\m{x}\m{\in}\m{B}\m{)}\m{)}
\endm
\begin{mmraw}%
|- ( A e. B <-> E. x ( x = A \TAND x e. B ) ) \$.?
\end{mmraw}

\noindent Define inequality.

\vskip 0.5ex
\setbox\startprefix=\hbox{\tt \ \ df-ne\ \$a\ }
\setbox\contprefix=\hbox{\tt \ \ \ \ \ \ \ \ \ \ \ }
\startm
\m{\vdash}\m{(}\m{A}\m{\ne}\m{B}\m{\leftrightarrow}\m{\lnot}\m{A}\m{=}\m{B}%
\m{)}
\endm
\begin{mmraw}%
|- ( A =/= B <-> -. A = B ) \$.
\end{mmraw}

\noindent Define restricted universal quantification.\index{universal
quantifier ($\forall$)!restricted}  Enderton, p.~22.

\vskip 0.5ex
\setbox\startprefix=\hbox{\tt \ \ df-ral\ \$a\ }
\setbox\contprefix=\hbox{\tt \ \ \ \ \ \ \ \ \ \ \ \ }
\startm
\m{\vdash}\m{(}\m{\forall}\m{x}\m{\in}\m{A}\m{\varphi}\m{\leftrightarrow}\m{%
\forall}\m{x}\m{(}\m{x}\m{\in}\m{A}\m{\rightarrow}\m{\varphi}\m{)}\m{)}
\endm
\begin{mmraw}%
|- ( A. x e. A ph <-> A. x ( x e. A -> ph ) ) \$.
\end{mmraw}

\noindent Define restricted existential quantification.\index{existential
quantifier ($\exists$)!restricted}  Enderton, p.~22.

\vskip 0.5ex
\setbox\startprefix=\hbox{\tt \ \ df-rex\ \$a\ }
\setbox\contprefix=\hbox{\tt \ \ \ \ \ \ \ \ \ \ \ \ }
\startm
\m{\vdash}\m{(}\m{\exists}\m{x}\m{\in}\m{A}\m{\varphi}\m{\leftrightarrow}\m{%
\exists}\m{x}\m{(}\m{x}\m{\in}\m{A}\m{\wedge}\m{\varphi}\m{)}\m{)}
\endm
\begin{mmraw}%
|- ( E. x e. A ph <-> E. x ( x e. A \TAND ph ) ) \$.
\end{mmraw}

\noindent Define the universal class\index{universal class ($V$)}.  Definition
5.20, p.~21, of Takeuti and Zaring.\label{df-v}

\vskip 0.5ex
\setbox\startprefix=\hbox{\tt \ \ df-v\ \$a\ }
\setbox\contprefix=\hbox{\tt \ \ \ \ \ \ \ \ \ \ }
\startm
\m{\vdash}\m{{\rm V}}\m{=}\m{\{}\m{x}\m{|}\m{x}\m{=}\m{x}\m{\}}
\endm
\begin{mmraw}%
|- {\char`\_}V = \{ x | x = x \} \$.
\end{mmraw}

\noindent Define the subclass\index{subclass}\index{subset} relationship
between two classes (called the subset relation if the classes are sets i.e.\
are not proper).  Definition 5.9 of Takeuti and Zaring, p.~17.\label{df-ss}

\vskip 0.5ex
\setbox\startprefix=\hbox{\tt \ \ df-ss\ \$a\ }
\setbox\contprefix=\hbox{\tt \ \ \ \ \ \ \ \ \ \ \ }
\startm
\m{\vdash}\m{(}\m{A}\m{\subseteq}\m{B}\m{\leftrightarrow}\m{\forall}\m{x}\m{(}
\m{x}\m{\in}\m{A}\m{\rightarrow}\m{x}\m{\in}\m{B}\m{)}\m{)}
\endm
\begin{mmraw}%
|- ( A C\_ B <-> A. x ( x e. A -> x e. B ) ) \$.
\end{mmraw}

\noindent Define the union\index{union} of two classes.  Definition 5.6 of Takeuti and Zaring,
p.~16.\label{df-un}

\vskip 0.5ex
\setbox\startprefix=\hbox{\tt \ \ df-un\ \$a\ }
\setbox\contprefix=\hbox{\tt \ \ \ \ \ \ \ \ \ \ \ }
\startm
\m{\vdash}\m{(}\m{A}\m{\cup}\m{B}\m{)}\m{=}\m{\{}\m{x}\m{|}\m{(}\m{x}\m{\in}
\m{A}\m{\vee}\m{x}\m{\in}\m{B}\m{)}\m{\}}
\endm
\begin{mmraw}%
( A u. B ) = \{ x | ( x e. A \TOR x e. B ) \} \$.
\end{mmraw}

\noindent Define the intersection\index{intersection} of two classes.  Definition 5.6 of
Takeuti and Zaring, p.~16.\label{df-in}

\vskip 0.5ex
\setbox\startprefix=\hbox{\tt \ \ df-in\ \$a\ }
\setbox\contprefix=\hbox{\tt \ \ \ \ \ \ \ \ \ \ \ }
\startm
\m{\vdash}\m{(}\m{A}\m{\cap}\m{B}\m{)}\m{=}\m{\{}\m{x}\m{|}\m{(}\m{x}\m{\in}
\m{A}\m{\wedge}\m{x}\m{\in}\m{B}\m{)}\m{\}}
\endm
% Caret ^ requires special treatment
\begin{mmraw}%
|- ( A i\^{}i B ) = \{ x | ( x e. A \TAND x e. B ) \} \$.
\end{mmraw}

\noindent Define class difference\index{class difference}\index{set difference}.
Definition 5.12 of Takeuti and Zaring, p.~20.  Several notations are used in
the literature; we chose the $\setminus$ convention instead of a minus sign to
reserve the latter for later use in, e.g., arithmetic.\label{df-dif}

\vskip 0.5ex
\setbox\startprefix=\hbox{\tt \ \ df-dif\ \$a\ }
\setbox\contprefix=\hbox{\tt \ \ \ \ \ \ \ \ \ \ \ \ }
\startm
\m{\vdash}\m{(}\m{A}\m{\setminus}\m{B}\m{)}\m{=}\m{\{}\m{x}\m{|}\m{(}\m{x}\m{
\in}\m{A}\m{\wedge}\m{\lnot}\m{x}\m{\in}\m{B}\m{)}\m{\}}
\endm
\begin{mmraw}%
( A \SLASH B ) = \{ x | ( x e. A \TAND -. x e. B ) \} \$.
\end{mmraw}

\noindent Define the empty or null set\index{empty set}\index{null set}.
Compare  Definition 5.14 of Takeuti and Zaring, p.~20.\label{df-nul}

\vskip 0.5ex
\setbox\startprefix=\hbox{\tt \ \ df-nul\ \$a\ }
\setbox\contprefix=\hbox{\tt \ \ \ \ \ \ \ \ \ \ }
\startm
\m{\vdash}\m{\varnothing}\m{=}\m{(}\m{{\rm V}}\m{\setminus}\m{{\rm V}}\m{)}
\endm
\begin{mmraw}%
|- (/) = ( {\char`\_}V \SLASH {\char`\_}V ) \$.
\end{mmraw}

\noindent Define power class\index{power set}\index{power class}.  Definition 5.10 of
Takeuti and Zaring, p.~17, but we also let it apply to proper classes.  (Note
that \verb$~P$ is the symbol for calligraphic P, the tilde
suggesting ``curly;'' see Appendix~\ref{ASCII}.)\label{df-pw}

\vskip 0.5ex
\setbox\startprefix=\hbox{\tt \ \ df-pw\ \$a\ }
\setbox\contprefix=\hbox{\tt \ \ \ \ \ \ \ \ \ \ \ }
\startm
\m{\vdash}\m{{\cal P}}\m{A}\m{=}\m{\{}\m{x}\m{|}\m{x}\m{\subseteq}\m{A}\m{\}}
\endm
% Special incantation required to put ~ into the text
\begin{mmraw}%
|- \char`\~P~A = \{ x | x C\_ A \} \$.
\end{mmraw}

\noindent Define the singleton of a class\index{singleton}.  Definition 7.1 of
Quine, p.~48.  It is well-defined for proper classes, although
it is not very meaningful in this case, where it evaluates to the empty
set.

\vskip 0.5ex
\setbox\startprefix=\hbox{\tt \ \ df-sn\ \$a\ }
\setbox\contprefix=\hbox{\tt \ \ \ \ \ \ \ \ \ \ \ }
\startm
\m{\vdash}\m{\{}\m{A}\m{\}}\m{=}\m{\{}\m{x}\m{|}\m{x}\m{=}\m{A}\m{\}}
\endm
\begin{mmraw}%
|- \{ A \} = \{ x | x = A \} \$.
\end{mmraw}%

\noindent Define an unordered pair of classes\index{unordered pair}\index{pair}.  Definition
7.1 of Quine, p.~48.

\vskip 0.5ex
\setbox\startprefix=\hbox{\tt \ \ df-pr\ \$a\ }
\setbox\contprefix=\hbox{\tt \ \ \ \ \ \ \ \ \ \ \ }
\startm
\m{\vdash}\m{\{}\m{A}\m{,}\m{B}\m{\}}\m{=}\m{(}\m{\{}\m{A}\m{\}}\m{\cup}\m{\{}
\m{B}\m{\}}\m{)}
\endm
\begin{mmraw}%
|- \{ A , B \} = ( \{ A \} u. \{ B \} ) \$.
\end{mmraw}

\noindent Define an unordered triple of classes\index{unordered triple}.  Definition of
Enderton, p.~19.

\vskip 0.5ex
\setbox\startprefix=\hbox{\tt \ \ df-tp\ \$a\ }
\setbox\contprefix=\hbox{\tt \ \ \ \ \ \ \ \ \ \ \ }
\startm
\m{\vdash}\m{\{}\m{A}\m{,}\m{B}\m{,}\m{C}\m{\}}\m{=}\m{(}\m{\{}\m{A}\m{,}\m{B}
\m{\}}\m{\cup}\m{\{}\m{C}\m{\}}\m{)}
\endm
\begin{mmraw}%
|- \{ A , B , C \} = ( \{ A , B \} u. \{ C \} ) \$.
\end{mmraw}%

\noindent Kuratowski's\index{Kuratowski, Kazimierz} ordered pair\index{ordered
pair} definition.  Definition 9.1 of Quine, p.~58. For proper classes it is
not meaningful but is well-defined for convenience.  (Note that \verb$<.$
stands for $\langle$ whereas \verb$<$ stands for $<$, and similarly for
\verb$>.$\,.)\label{df-op}

\vskip 0.5ex
\setbox\startprefix=\hbox{\tt \ \ df-op\ \$a\ }
\setbox\contprefix=\hbox{\tt \ \ \ \ \ \ \ \ \ \ \ }
\startm
\m{\vdash}\m{\langle}\m{A}\m{,}\m{B}\m{\rangle}\m{=}\m{\{}\m{\{}\m{A}\m{\}}
\m{,}\m{\{}\m{A}\m{,}\m{B}\m{\}}\m{\}}
\endm
\begin{mmraw}%
|- <. A , B >. = \{ \{ A \} , \{ A , B \} \} \$.
\end{mmraw}

\noindent Define the union of a class\index{union}.  Definition 5.5, p.~16,
of Takeuti and Zaring.

\vskip 0.5ex
\setbox\startprefix=\hbox{\tt \ \ df-uni\ \$a\ }
\setbox\contprefix=\hbox{\tt \ \ \ \ \ \ \ \ \ \ \ \ }
\startm
\m{\vdash}\m{\bigcup}\m{A}\m{=}\m{\{}\m{x}\m{|}\m{\exists}\m{y}\m{(}\m{x}\m{
\in}\m{y}\m{\wedge}\m{y}\m{\in}\m{A}\m{)}\m{\}}
\endm
\begin{mmraw}%
|- U. A = \{ x | E. y ( x e. y \TAND y e. A ) \} \$.
\end{mmraw}

\noindent Define the intersection\index{intersection} of a class.  Definition 7.35,
p.~44, of Takeuti and Zaring.

\vskip 0.5ex
\setbox\startprefix=\hbox{\tt \ \ df-int\ \$a\ }
\setbox\contprefix=\hbox{\tt \ \ \ \ \ \ \ \ \ \ \ \ }
\startm
\m{\vdash}\m{\bigcap}\m{A}\m{=}\m{\{}\m{x}\m{|}\m{\forall}\m{y}\m{(}\m{y}\m{
\in}\m{A}\m{\rightarrow}\m{x}\m{\in}\m{y}\m{)}\m{\}}
\endm
\begin{mmraw}%
|- |\^{}| A = \{ x | A. y ( y e. A -> x e. y ) \} \$.
\end{mmraw}

\noindent Define a transitive class\index{transitive class}\index{transitive
set}.  This should not be confused with a transitive relation, which is a different
concept.  Definition from p.~71 of Enderton, extended to classes.

\vskip 0.5ex
\setbox\startprefix=\hbox{\tt \ \ df-tr\ \$a\ }
\setbox\contprefix=\hbox{\tt \ \ \ \ \ \ \ \ \ \ \ }
\startm
\m{\vdash}\m{(}\m{\mbox{\rm Tr}}\m{A}\m{\leftrightarrow}\m{\bigcup}\m{A}\m{
\subseteq}\m{A}\m{)}
\endm
\begin{mmraw}%
|- ( Tr A <-> U. A C\_ A ) \$.
\end{mmraw}

\noindent Define a notation for a general binary relation\index{binary
relation}.  Definition 6.18, p.~29, of Takeuti and Zaring, generalized to
arbitrary classes.  This definition is well-defined, although not very
meaningful, when classes $A$ and/or $B$ are proper.\label{dfbr}  The lack of
parentheses (or any other connective) creates no ambiguity since we are defining
an atomic wff.

\vskip 0.5ex
\setbox\startprefix=\hbox{\tt \ \ df-br\ \$a\ }
\setbox\contprefix=\hbox{\tt \ \ \ \ \ \ \ \ \ \ \ }
\startm
\m{\vdash}\m{(}\m{A}\m{\,R}\m{\,B}\m{\leftrightarrow}\m{\langle}\m{A}\m{,}\m{B}
\m{\rangle}\m{\in}\m{R}\m{)}
\endm
\begin{mmraw}%
|- ( A R B <-> <. A , B >. e. R ) \$.
\end{mmraw}

\noindent Define an abstraction class of ordered pairs\index{abstraction
class!of ordered
pairs}.  A special case of Definition 4.16, p.~14, of Takeuti and Zaring.
Note that $ z $ must be distinct from $ x $ and $ y $,
and $ z $ must not occur in $\varphi$, but $ x $ and $ y $ may be
identical and may appear in $\varphi$.

\vskip 0.5ex
\setbox\startprefix=\hbox{\tt \ \ df-opab\ \$a\ }
\setbox\contprefix=\hbox{\tt \ \ \ \ \ \ \ \ \ \ \ \ \ }
\startm
\m{\vdash}\m{\{}\m{\langle}\m{x}\m{,}\m{y}\m{\rangle}\m{|}\m{\varphi}\m{\}}\m{=}
\m{\{}\m{z}\m{|}\m{\exists}\m{x}\m{\exists}\m{y}\m{(}\m{z}\m{=}\m{\langle}\m{x}
\m{,}\m{y}\m{\rangle}\m{\wedge}\m{\varphi}\m{)}\m{\}}
\endm

\begin{mmraw}%
|- \{ <. x , y >. | ph \} = \{ z | E. x E. y ( z =
<. x , y >. /\ ph ) \} \$.
\end{mmraw}

\noindent Define the epsilon relation\index{epsilon relation}.  Similar to Definition
6.22, p.~30, of Takeuti and Zaring.

\vskip 0.5ex
\setbox\startprefix=\hbox{\tt \ \ df-eprel\ \$a\ }
\setbox\contprefix=\hbox{\tt \ \ \ \ \ \ \ \ \ \ \ \ \ \ }
\startm
\m{\vdash}\m{{\rm E}}\m{=}\m{\{}\m{\langle}\m{x}\m{,}\m{y}\m{\rangle}\m{|}\m{x}\m{
\in}\m{y}\m{\}}
\endm
\begin{mmraw}%
|- \_E = \{ <. x , y >. | x e. y \} \$.
\end{mmraw}

\noindent Define a founded relation\index{founded relation}.  $R$ is a founded
relation on $A$ iff\index{iff} (if and only if) each nonempty subset of $A$
has an ``$R$-minimal element.''  Similar to Definition 6.21, p.~30, of
Takeuti and Zaring.

\vskip 0.5ex
\setbox\startprefix=\hbox{\tt \ \ df-fr\ \$a\ }
\setbox\contprefix=\hbox{\tt \ \ \ \ \ \ \ \ \ \ \ }
\startm
\m{\vdash}\m{(}\m{R}\m{\,\mbox{\rm Fr}}\m{\,A}\m{\leftrightarrow}\m{\forall}\m{x}
\m{(}\m{(}\m{x}\m{\subseteq}\m{A}\m{\wedge}\m{\lnot}\m{x}\m{=}\m{\varnothing}
\m{)}\m{\rightarrow}\m{\exists}\m{y}\m{(}\m{y}\m{\in}\m{x}\m{\wedge}\m{(}\m{x}
\m{\cap}\m{\{}\m{z}\m{|}\m{z}\m{\,R}\m{\,y}\m{\}}\m{)}\m{=}\m{\varnothing}\m{)}
\m{)}\m{)}
\endm
\begin{mmraw}%
|- ( R Fr A <-> A. x ( ( x C\_ A \TAND -. x = (/) ) ->
E. y ( y e. x \TAND ( x i\^{}i \{ z | z R y \} ) = (/) ) ) ) \$.
\end{mmraw}

\noindent Define a well-ordering\index{well-ordering}.  $R$ is a well-ordering of $A$ iff
it is founded on $A$ and the elements of $A$ are pairwise $R$-comparable.
Similar to Definition 6.24(2), p.~30, of Takeuti and Zaring.

\vskip 0.5ex
\setbox\startprefix=\hbox{\tt \ \ df-we\ \$a\ }
\setbox\contprefix=\hbox{\tt \ \ \ \ \ \ \ \ \ \ \ }
\startm
\m{\vdash}\m{(}\m{R}\m{\,\mbox{\rm We}}\m{\,A}\m{\leftrightarrow}\m{(}\m{R}\m{\,
\mbox{\rm Fr}}\m{\,A}\m{\wedge}\m{\forall}\m{x}\m{\forall}\m{y}\m{(}\m{(}\m{x}\m{
\in}\m{A}\m{\wedge}\m{y}\m{\in}\m{A}\m{)}\m{\rightarrow}\m{(}\m{x}\m{\,R}\m{\,y}
\m{\vee}\m{x}\m{=}\m{y}\m{\vee}\m{y}\m{\,R}\m{\,x}\m{)}\m{)}\m{)}\m{)}
\endm
\begin{mmraw}%
( R We A <-> ( R Fr A \TAND A. x A. y ( ( x e.
A \TAND y e. A ) -> ( x R y \TOR x = y \TOR y R x ) ) ) ) \$.
\end{mmraw}

\noindent Define the ordinal predicate\index{ordinal predicate}, which is true for a class
that is transitive and is well-ordered by the epsilon relation.  Similar to
definition on p.~468, Bell and Machover.

\vskip 0.5ex
\setbox\startprefix=\hbox{\tt \ \ df-ord\ \$a\ }
\setbox\contprefix=\hbox{\tt \ \ \ \ \ \ \ \ \ \ \ \ }
\startm
\m{\vdash}\m{(}\m{\mbox{\rm Ord}}\m{\,A}\m{\leftrightarrow}\m{(}
\m{\mbox{\rm Tr}}\m{\,A}\m{\wedge}\m{E}\m{\,\mbox{\rm We}}\m{\,A}\m{)}\m{)}
\endm
\begin{mmraw}%
|- ( Ord A <-> ( Tr A \TAND E We A ) ) \$.
\end{mmraw}

\noindent Define the class of all ordinal numbers\index{ordinal number}.  An ordinal number is
a set that satisfies the ordinal predicate.  Definition 7.11 of Takeuti and
Zaring, p.~38.

\vskip 0.5ex
\setbox\startprefix=\hbox{\tt \ \ df-on\ \$a\ }
\setbox\contprefix=\hbox{\tt \ \ \ \ \ \ \ \ \ \ \ }
\startm
\m{\vdash}\m{\,\mbox{\rm On}}\m{=}\m{\{}\m{x}\m{|}\m{\mbox{\rm Ord}}\m{\,x}
\m{\}}
\endm
\begin{mmraw}%
|- On = \{ x | Ord x \} \$.
\end{mmraw}

\noindent Define the limit ordinal predicate\index{limit ordinal}, which is true for a
non-empty ordinal that is not a successor (i.e.\ that is the union of itself).
Compare Bell and Machover, p.~471 and Exercise (1), p.~42 of Takeuti and
Zaring.

\vskip 0.5ex
\setbox\startprefix=\hbox{\tt \ \ df-lim\ \$a\ }
\setbox\contprefix=\hbox{\tt \ \ \ \ \ \ \ \ \ \ \ \ }
\startm
\m{\vdash}\m{(}\m{\mbox{\rm Lim}}\m{\,A}\m{\leftrightarrow}\m{(}\m{\mbox{
\rm Ord}}\m{\,A}\m{\wedge}\m{\lnot}\m{A}\m{=}\m{\varnothing}\m{\wedge}\m{A}
\m{=}\m{\bigcup}\m{A}\m{)}\m{)}
\endm
\begin{mmraw}%
|- ( Lim A <-> ( Ord A \TAND -. A = (/) \TAND A = U. A ) ) \$.
\end{mmraw}

\noindent Define the successor\index{successor} of a class.  Definition 7.22 of Takeuti
and Zaring, p.~41.  Our definition is a generalization to classes, although it
is meaningless when classes are proper.

\vskip 0.5ex
\setbox\startprefix=\hbox{\tt \ \ df-suc\ \$a\ }
\setbox\contprefix=\hbox{\tt \ \ \ \ \ \ \ \ \ \ \ \ }
\startm
\m{\vdash}\m{\,\mbox{\rm suc}}\m{\,A}\m{=}\m{(}\m{A}\m{\cup}\m{\{}\m{A}\m{\}}
\m{)}
\endm
\begin{mmraw}%
|- suc A = ( A u. \{ A \} ) \$.
\end{mmraw}

\noindent Define the class of natural numbers\index{natural number}\index{omega
($\omega$)}.  Compare Bell and Machover, p.~471.\label{dfom}

\vskip 0.5ex
\setbox\startprefix=\hbox{\tt \ \ df-om\ \$a\ }
\setbox\contprefix=\hbox{\tt \ \ \ \ \ \ \ \ \ \ \ }
\startm
\m{\vdash}\m{\omega}\m{=}\m{\{}\m{x}\m{|}\m{(}\m{\mbox{\rm Ord}}\m{\,x}\m{
\wedge}\m{\forall}\m{y}\m{(}\m{\mbox{\rm Lim}}\m{\,y}\m{\rightarrow}\m{x}\m{
\in}\m{y}\m{)}\m{)}\m{\}}
\endm
\begin{mmraw}%
|- om = \{ x | ( Ord x \TAND A. y ( Lim y -> x e. y ) ) \} \$.
\end{mmraw}

\noindent Define the Cartesian product (also called the
cross product)\index{Cartesian product}\index{cross product}
of two classes.  Definition 9.11 of Quine, p.~64.

\vskip 0.5ex
\setbox\startprefix=\hbox{\tt \ \ df-xp\ \$a\ }
\setbox\contprefix=\hbox{\tt \ \ \ \ \ \ \ \ \ \ \ }
\startm
\m{\vdash}\m{(}\m{A}\m{\times}\m{B}\m{)}\m{=}\m{\{}\m{\langle}\m{x}\m{,}\m{y}
\m{\rangle}\m{|}\m{(}\m{x}\m{\in}\m{A}\m{\wedge}\m{y}\m{\in}\m{B}\m{)}\m{\}}
\endm
\begin{mmraw}%
|- ( A X. B ) = \{ <. x , y >. | ( x e. A \TAND y e. B) \} \$.
\end{mmraw}

\noindent Define a relation\index{relation}.  Definition 6.4(1) of Takeuti and
Zaring, p.~23.

\vskip 0.5ex
\setbox\startprefix=\hbox{\tt \ \ df-rel\ \$a\ }
\setbox\contprefix=\hbox{\tt \ \ \ \ \ \ \ \ \ \ \ \ }
\startm
\m{\vdash}\m{(}\m{\mbox{\rm Rel}}\m{\,A}\m{\leftrightarrow}\m{A}\m{\subseteq}
\m{(}\m{{\rm V}}\m{\times}\m{{\rm V}}\m{)}\m{)}
\endm
\begin{mmraw}%
|- ( Rel A <-> A C\_ ( {\char`\_}V X. {\char`\_}V ) ) \$.
\end{mmraw}

\noindent Define the domain\index{domain} of a class.  Definition 6.5(1) of
Takeuti and Zaring, p.~24.

\vskip 0.5ex
\setbox\startprefix=\hbox{\tt \ \ df-dm\ \$a\ }
\setbox\contprefix=\hbox{\tt \ \ \ \ \ \ \ \ \ \ \ }
\startm
\m{\vdash}\m{\,\mbox{\rm dom}}\m{A}\m{=}\m{\{}\m{x}\m{|}\m{\exists}\m{y}\m{
\langle}\m{x}\m{,}\m{y}\m{\rangle}\m{\in}\m{A}\m{\}}
\endm
\begin{mmraw}%
|- dom A = \{ x | E. y <. x , y >. e. A \} \$.
\end{mmraw}

\noindent Define the range\index{range} of a class.  Definition 6.5(2) of
Takeuti and Zaring, p.~24.

\vskip 0.5ex
\setbox\startprefix=\hbox{\tt \ \ df-rn\ \$a\ }
\setbox\contprefix=\hbox{\tt \ \ \ \ \ \ \ \ \ \ \ }
\startm
\m{\vdash}\m{\,\mbox{\rm ran}}\m{A}\m{=}\m{\{}\m{y}\m{|}\m{\exists}\m{x}\m{
\langle}\m{x}\m{,}\m{y}\m{\rangle}\m{\in}\m{A}\m{\}}
\endm
\begin{mmraw}%
|- ran A = \{ y | E. x <. x , y >. e. A \} \$.
\end{mmraw}

\noindent Define the restriction\index{restriction} of a class.  Definition
6.6(1) of Takeuti and Zaring, p.~24.

\vskip 0.5ex
\setbox\startprefix=\hbox{\tt \ \ df-res\ \$a\ }
\setbox\contprefix=\hbox{\tt \ \ \ \ \ \ \ \ \ \ \ \ }
\startm
\m{\vdash}\m{(}\m{A}\m{\restriction}\m{B}\m{)}\m{=}\m{(}\m{A}\m{\cap}\m{(}\m{B}
\m{\times}\m{{\rm V}}\m{)}\m{)}
\endm
\begin{mmraw}%
|- ( A |` B ) = ( A i\^{}i ( B X. {\char`\_}V ) ) \$.
\end{mmraw}

\noindent Define the image\index{image} of a class.  Definition 6.6(2) of
Takeuti and Zaring, p.~24.

\vskip 0.5ex
\setbox\startprefix=\hbox{\tt \ \ df-ima\ \$a\ }
\setbox\contprefix=\hbox{\tt \ \ \ \ \ \ \ \ \ \ \ \ }
\startm
\m{\vdash}\m{(}\m{A}\m{``}\m{B}\m{)}\m{=}\m{\,\mbox{\rm ran}}\m{\,(}\m{A}\m{
\restriction}\m{B}\m{)}
\endm
\begin{mmraw}%
|- ( A " B ) = ran ( A |` B ) \$.
\end{mmraw}

\noindent Define the composition\index{composition} of two classes.  Definition 6.6(3) of
Takeuti and Zaring, p.~24.

\vskip 0.5ex
\setbox\startprefix=\hbox{\tt \ \ df-co\ \$a\ }
\setbox\contprefix=\hbox{\tt \ \ \ \ \ \ \ \ \ \ \ \ }
\startm
\m{\vdash}\m{(}\m{A}\m{\circ}\m{B}\m{)}\m{=}\m{\{}\m{\langle}\m{x}\m{,}\m{y}\m{
\rangle}\m{|}\m{\exists}\m{z}\m{(}\m{\langle}\m{x}\m{,}\m{z}\m{\rangle}\m{\in}
\m{B}\m{\wedge}\m{\langle}\m{z}\m{,}\m{y}\m{\rangle}\m{\in}\m{A}\m{)}\m{\}}
\endm
\begin{mmraw}%
|- ( A o. B ) = \{ <. x , y >. | E. z ( <. x , z
>. e. B \TAND <. z , y >. e. A ) \} \$.
\end{mmraw}

\noindent Define a function\index{function}.  Definition 6.4(4) of Takeuti and
Zaring, p.~24.

\vskip 0.5ex
\setbox\startprefix=\hbox{\tt \ \ df-fun\ \$a\ }
\setbox\contprefix=\hbox{\tt \ \ \ \ \ \ \ \ \ \ \ \ }
\startm
\m{\vdash}\m{(}\m{\mbox{\rm Fun}}\m{\,A}\m{\leftrightarrow}\m{(}
\m{\mbox{\rm Rel}}\m{\,A}\m{\wedge}
\m{\forall}\m{x}\m{\exists}\m{z}\m{\forall}\m{y}\m{(}
\m{\langle}\m{x}\m{,}\m{y}\m{\rangle}\m{\in}\m{A}\m{\rightarrow}\m{y}\m{=}\m{z}
\m{)}\m{)}\m{)}
\endm
\begin{mmraw}%
|- ( Fun A <-> ( Rel A /\ A. x E. z A. y ( <. x
   , y >. e. A -> y = z ) ) ) \$.
\end{mmraw}

\noindent Define a function with domain.  Definition 6.15(1) of Takeuti and
Zaring, p.~27.

\vskip 0.5ex
\setbox\startprefix=\hbox{\tt \ \ df-fn\ \$a\ }
\setbox\contprefix=\hbox{\tt \ \ \ \ \ \ \ \ \ \ \ }
\startm
\m{\vdash}\m{(}\m{A}\m{\,\mbox{\rm Fn}}\m{\,B}\m{\leftrightarrow}\m{(}
\m{\mbox{\rm Fun}}\m{\,A}\m{\wedge}\m{\mbox{\rm dom}}\m{\,A}\m{=}\m{B}\m{)}
\m{)}
\endm
\begin{mmraw}%
|- ( A Fn B <-> ( Fun A \TAND dom A = B ) ) \$.
\end{mmraw}

\noindent Define a function with domain and co-domain.  Definition 6.15(3)
of Takeuti and Zaring, p.~27.

\vskip 0.5ex
\setbox\startprefix=\hbox{\tt \ \ df-f\ \$a\ }
\setbox\contprefix=\hbox{\tt \ \ \ \ \ \ \ \ \ \ }
\startm
\m{\vdash}\m{(}\m{F}\m{:}\m{A}\m{\longrightarrow}\m{B}\m{
\leftrightarrow}\m{(}\m{F}\m{\,\mbox{\rm Fn}}\m{\,A}\m{\wedge}\m{
\mbox{\rm ran}}\m{\,F}\m{\subseteq}\m{B}\m{)}\m{)}
\endm
\begin{mmraw}%
|- ( F : A --> B <-> ( F Fn A \TAND ran F C\_ B ) ) \$.
\end{mmraw}

\noindent Define a one-to-one function\index{one-to-one function}.  Compare
Definition 6.15(5) of Takeuti and Zaring, p.~27.

\vskip 0.5ex
\setbox\startprefix=\hbox{\tt \ \ df-f1\ \$a\ }
\setbox\contprefix=\hbox{\tt \ \ \ \ \ \ \ \ \ \ \ }
\startm
\m{\vdash}\m{(}\m{F}\m{:}\m{A}\m{
\raisebox{.5ex}{${\textstyle{\:}_{\mbox{\footnotesize\rm
1\tt -\rm 1}}}\atop{\textstyle{
\longrightarrow}\atop{\textstyle{}^{\mbox{\footnotesize\rm {\ }}}}}$}
}\m{B}
\m{\leftrightarrow}\m{(}\m{F}\m{:}\m{A}\m{\longrightarrow}\m{B}
\m{\wedge}\m{\forall}\m{y}\m{\exists}\m{z}\m{\forall}\m{x}\m{(}\m{\langle}\m{x}
\m{,}\m{y}\m{\rangle}\m{\in}\m{F}\m{\rightarrow}\m{x}\m{=}\m{z}\m{)}\m{)}\m{)}
\endm
\begin{mmraw}%
|- ( F : A -1-1-> B <-> ( F : A --> B \TAND
   A. y E. z A. x ( <. x , y >. e. F -> x = z ) ) ) \$.
\end{mmraw}

\noindent Define an onto function\index{onto function}.  Definition 6.15(4) of Takeuti and
Zaring, p.~27.

\vskip 0.5ex
\setbox\startprefix=\hbox{\tt \ \ df-fo\ \$a\ }
\setbox\contprefix=\hbox{\tt \ \ \ \ \ \ \ \ \ \ \ }
\startm
\m{\vdash}\m{(}\m{F}\m{:}\m{A}\m{
\raisebox{.5ex}{${\textstyle{\:}_{\mbox{\footnotesize\rm
{\ }}}}\atop{\textstyle{
\longrightarrow}\atop{\textstyle{}^{\mbox{\footnotesize\rm onto}}}}$}
}\m{B}
\m{\leftrightarrow}\m{(}\m{F}\m{\,\mbox{\rm Fn}}\m{\,A}\m{\wedge}
\m{\mbox{\rm ran}}\m{\,F}\m{=}\m{B}\m{)}\m{)}
\endm
\begin{mmraw}%
|- ( F : A -onto-> B <-> ( F Fn A /\ ran F = B ) ) \$.
\end{mmraw}

\noindent Define a one-to-one, onto function.  Compare Definition 6.15(6) of
Takeuti and Zaring, p.~27.

\vskip 0.5ex
\setbox\startprefix=\hbox{\tt \ \ df-f1o\ \$a\ }
\setbox\contprefix=\hbox{\tt \ \ \ \ \ \ \ \ \ \ \ \ }
\startm
\m{\vdash}\m{(}\m{F}\m{:}\m{A}
\m{
\raisebox{.5ex}{${\textstyle{\:}_{\mbox{\footnotesize\rm
1\tt -\rm 1}}}\atop{\textstyle{
\longrightarrow}\atop{\textstyle{}^{\mbox{\footnotesize\rm onto}}}}$}
}
\m{B}
\m{\leftrightarrow}\m{(}\m{F}\m{:}\m{A}
\m{
\raisebox{.5ex}{${\textstyle{\:}_{\mbox{\footnotesize\rm
1\tt -\rm 1}}}\atop{\textstyle{
\longrightarrow}\atop{\textstyle{}^{\mbox{\footnotesize\rm {\ }}}}}$}
}
\m{B}\m{\wedge}\m{F}\m{:}\m{A}
\m{
\raisebox{.5ex}{${\textstyle{\:}_{\mbox{\footnotesize\rm
{\ }}}}\atop{\textstyle{
\longrightarrow}\atop{\textstyle{}^{\mbox{\footnotesize\rm onto}}}}$}
}
\m{B}\m{)}\m{)}
\endm
\begin{mmraw}%
|- ( F : A -1-1-onto-> B <-> ( F : A -1-1-> B? \TAND F : A -onto-> B ) ) \$.?
\end{mmraw}

\noindent Define the value of a function\index{function value}.  This
definition applies to any class and evaluates to the empty set when it is not
meaningful. Note that $ F`A$ means the same thing as the more familiar $ F(A)$
notation for a function's value at $A$.  The $ F`A$ notation is common in
formal set theory.\label{df-fv}

\vskip 0.5ex
\setbox\startprefix=\hbox{\tt \ \ df-fv\ \$a\ }
\setbox\contprefix=\hbox{\tt \ \ \ \ \ \ \ \ \ \ \ }
\startm
\m{\vdash}\m{(}\m{F}\m{`}\m{A}\m{)}\m{=}\m{\bigcup}\m{\{}\m{x}\m{|}\m{(}\m{F}%
\m{``}\m{\{}\m{A}\m{\}}\m{)}\m{=}\m{\{}\m{x}\m{\}}\m{\}}
\endm
\begin{mmraw}%
|- ( F ` A ) = U. \{ x | ( F " \{ A \} ) = \{ x \} \} \$.
\end{mmraw}

\noindent Define the result of an operation.\index{operation}  Here, $F$ is
     an operation on two
     values (such as $+$ for real numbers).   This is defined for proper
     classes $A$ and $B$ even though not meaningful in that case.  However,
     the definition can be meaningful when $F$ is a proper class.\label{dfopr}

\vskip 0.5ex
\setbox\startprefix=\hbox{\tt \ \ df-opr\ \$a\ }
\setbox\contprefix=\hbox{\tt \ \ \ \ \ \ \ \ \ \ \ \ }
\startm
\m{\vdash}\m{(}\m{A}\m{\,F}\m{\,B}\m{)}\m{=}\m{(}\m{F}\m{`}\m{\langle}\m{A}%
\m{,}\m{B}\m{\rangle}\m{)}
\endm
\begin{mmraw}%
|- ( A F B ) = ( F ` <. A , B >. ) \$.
\end{mmraw}

\section{Tricks of the Trade}\label{tricks}

In the \texttt{set.mm}\index{set theory database (\texttt{set.mm})} database our goal
was usually to conform to modern notation.  However in some cases the
relationship to standard textbook language may be obscured by several
unconventional devices we used to simplify the development and to take
advantage of the Metamath language.  In this section we will describe some
common conventions used in \texttt{set.mm}.

\begin{itemize}
\item
The turnstile\index{turnstile ({$\,\vdash$})} symbol, $\vdash$, meaning ``it
is provable that,'' is the first token of all assertions and hypotheses that
aren't syntax constructions.  This is a standard convention in logic.  (We
mentioned this earlier, but this symbol is bothersome to some people without a
logic background.  It has no deeper meaning but just provides us with a way to
distinguish syntax constructions from ordinary mathematical statements.)

\item
A hypothesis of the form

\vskip 1ex
\setbox\startprefix=\hbox{\tt \ \ \ \ \ \ \ \ \ \$e\ }
\setbox\contprefix=\hbox{\tt \ \ \ \ \ \ \ \ \ \ }
\startm
\m{\vdash}\m{(}\m{\varphi}\m{\rightarrow}\m{\forall}\m{x}\m{\varphi}\m{)}
\endm
\vskip 1ex

should be read ``assume variable $x$ is (effectively) not free in wff
$\varphi$.''\index{effectively not free}
Literally, this says ``assume it is provable that $\varphi \rightarrow \forall
x\, \varphi$.''  This device lets us avoid the complexities associated with
the standard treatment of free and bound variables.
%% Uncomment this when uncommenting section {formalspec} below
The footnote on p.~\pageref{effectivelybound} discusses this further.

\item
A statement of one of the forms

\vskip 1ex
\setbox\startprefix=\hbox{\tt \ \ \ \ \ \ \ \ \ \$a\ }
\setbox\contprefix=\hbox{\tt \ \ \ \ \ \ \ \ \ \ }
\startm
\m{\vdash}\m{(}\m{\lnot}\m{\forall}\m{x}\m{\,x}\m{=}\m{y}\m{\rightarrow}
\m{\ldots}\m{)}
\endm
\setbox\startprefix=\hbox{\tt \ \ \ \ \ \ \ \ \ \$p\ }
\setbox\contprefix=\hbox{\tt \ \ \ \ \ \ \ \ \ \ }
\startm
\m{\vdash}\m{(}\m{\lnot}\m{\forall}\m{x}\m{\,x}\m{=}\m{y}\m{\rightarrow}
\m{\ldots}\m{)}
\endm
\vskip 1ex

should be read ``if $x$ and $y$ are distinct variables, then...''  This
antecedent provides us with a technical device to avoid the need for the
\texttt{\$d} statement early in our development of predicate calculus,
permitting symbol manipulations to be as conceptually simple as those in
propositional calculus.  However, the \texttt{\$d} statement eventually
becomes a requirement, and after that this device is rarely used.

\item
The statement

\vskip 1ex
\setbox\startprefix=\hbox{\tt \ \ \ \ \ \ \ \ \ \$d\ }
\setbox\contprefix=\hbox{\tt \ \ \ \ \ \ \ \ \ \ }
\startm
\m{x}\m{\,y}
\endm
\vskip 1ex

should be read ``assume $x$ and $y$ are distinct variables.''

\item
The statement

\vskip 1ex
\setbox\startprefix=\hbox{\tt \ \ \ \ \ \ \ \ \ \$d\ }
\setbox\contprefix=\hbox{\tt \ \ \ \ \ \ \ \ \ \ }
\startm
\m{x}\m{\,\varphi}
\endm
\vskip 1ex

should be read ``assume $x$ does not occur in $\varphi$.''

\item
The statement

\vskip 1ex
\setbox\startprefix=\hbox{\tt \ \ \ \ \ \ \ \ \ \$d\ }
\setbox\contprefix=\hbox{\tt \ \ \ \ \ \ \ \ \ \ }
\startm
\m{x}\m{\,A}
\endm
\vskip 1ex

should be read ``assume variable $x$ does not occur in class $A$.''

\item
The restriction and hypothesis group

\vskip 1ex
\setbox\startprefix=\hbox{\tt \ \ \ \ \ \ \ \ \ \$d\ }
\setbox\contprefix=\hbox{\tt \ \ \ \ \ \ \ \ \ \ }
\startm
\m{x}\m{\,A}
\endm
\setbox\startprefix=\hbox{\tt \ \ \ \ \ \ \ \ \ \$d\ }
\setbox\contprefix=\hbox{\tt \ \ \ \ \ \ \ \ \ \ }
\startm
\m{x}\m{\,\psi}
\endm
\setbox\startprefix=\hbox{\tt \ \ \ \ \ \ \ \ \ \$e\ }
\setbox\contprefix=\hbox{\tt \ \ \ \ \ \ \ \ \ \ }
\startm
\m{\vdash}\m{(}\m{x}\m{=}\m{A}\m{\rightarrow}\m{(}\m{\varphi}\m{\leftrightarrow}
\m{\psi}\m{)}\m{)}
\endm
\vskip 1ex

is frequently used in place of explicit substitution, meaning ``assume
$\psi$ results from the proper substitution of $A$ for $x$ in
$\varphi$.''  Sometimes ``\texttt{\$e} $\vdash ( \psi \rightarrow
\forall x \, \psi )$'' is used instead of ``\texttt{\$d} $x\, \psi $,''
which requires only that $x$ be effectively not free in $\varphi$ but
not necessarily absent from it.  The use of implicit
substitution\index{substitution!implicit} is partly a matter of personal
style, although it may make proofs somewhat shorter than would be the
case with explicit substitution.

\item
The hypothesis


\vskip 1ex
\setbox\startprefix=\hbox{\tt \ \ \ \ \ \ \ \ \ \$e\ }
\setbox\contprefix=\hbox{\tt \ \ \ \ \ \ \ \ \ \ }
\startm
\m{\vdash}\m{A}\m{\in}\m{{\rm V}}
\endm
\vskip 1ex

should be read ``assume class $A$ is a set (i.e.\ exists).''
This is a convenient convention used by Quine.

\item
The restriction and hypothesis

\vskip 1ex
\setbox\startprefix=\hbox{\tt \ \ \ \ \ \ \ \ \ \$d\ }
\setbox\contprefix=\hbox{\tt \ \ \ \ \ \ \ \ \ \ }
\startm
\m{x}\m{\,y}
\endm
\setbox\startprefix=\hbox{\tt \ \ \ \ \ \ \ \ \ \$e\ }
\setbox\contprefix=\hbox{\tt \ \ \ \ \ \ \ \ \ \ }
\startm
\m{\vdash}\m{(}\m{y}\m{\in}\m{A}\m{\rightarrow}\m{\forall}\m{x}\m{\,y}
\m{\in}\m{A}\m{)}
\endm
\vskip 1ex

should be read ``assume variable $x$ is
(effectively) not free in class $A$.''

\end{itemize}

\section{A Theorem Sampler}\label{sometheorems}

In this section we list some of the more important theorems that are proved in
the \texttt{set.mm} database, and they illustrate the kinds of things that can be
done with Metamath.  While all of these facts are well-known results,
Metamath offers the advantage of easily allowing you to trace their
derivation back to axioms.  Our intent here is not to try to explain the
details or motivation; for this we refer you to the textbooks that are
mentioned in the descriptions.  (The \texttt{set.mm} file has bibliographic
references for the text references.)  Their proofs often embody important
concepts you may wish to explore with the Metamath program (see
Section~\ref{exploring}).  All the symbols that are used here are defined in
Section~\ref{hierarchy}.  For brevity we haven't included the \texttt{\$d}
restrictions or \texttt{\$f} hypotheses for these theorems; when you are
uncertain consult the \texttt{set.mm} database.

We start with \texttt{syl} (principle of the syllogism).
In \textit{Principia Mathematica}
Whitehead and Russell call this ``the principle of the
syllogism... because... the syllogism in Barbara is derived from them''
\cite[quote after Theorem *2.06 p.~101]{PM}.
Some authors call this law a ``hypothetical syllogism.''
At the time of this writing it is the most commonly referenced assertion in the
\texttt{set.mm} database.\footnote{
The Metamath program command \texttt{show usage}
shows the number of references.
}

\vskip 2ex
\noindent Theorem syl (principle of the syllogism)\index{Syllogism}%
\index{\texttt{syl}}\label{syl}.

\vskip 0.5ex
\setbox\startprefix=\hbox{\tt \ \ syl.1\ \$e\ }
\setbox\contprefix=\hbox{\tt \ \ \ \ \ \ \ \ \ \ \ }
\startm
\m{\vdash}\m{(}\m{\varphi}\m{ \rightarrow }\m{\psi}\m{)}
\endm
\setbox\startprefix=\hbox{\tt \ \ syl.2\ \$e\ }
\setbox\contprefix=\hbox{\tt \ \ \ \ \ \ \ \ \ \ \ }
\startm
\m{\vdash}\m{(}\m{\psi}\m{ \rightarrow }\m{\chi}\m{)}
\endm
\setbox\startprefix=\hbox{\tt \ \ syl\ \$p\ }
\setbox\contprefix=\hbox{\tt \ \ \ \ \ \ \ \ \ }
\startm
\m{\vdash}\m{(}\m{\varphi}\m{ \rightarrow }\m{\chi}\m{)}
\endm
\vskip 2ex

The following theorem is not very deep but provides us with a notational device
that is frequently used.  It allows us to use the expression ``$A \in V$'' as
a compact way of saying that class $A$ exists, i.e.\ is a set.

\vskip 2ex
\noindent Two ways to say ``$A$ is a set'':  $A$ is a member of the universe
$V$ if and only if $A$ exists (i.e.\ there exists a set equal to $A$).
Theorem 6.9 of Quine, p. 43.

\vskip 0.5ex
\setbox\startprefix=\hbox{\tt \ \ isset\ \$p\ }
\setbox\contprefix=\hbox{\tt \ \ \ \ \ \ \ \ \ \ \ }
\startm
\m{\vdash}\m{(}\m{A}\m{\in}\m{{\rm V}}\m{\leftrightarrow}\m{\exists}\m{x}\m{\,x}\m{=}
\m{A}\m{)}
\endm
\vskip 1ex

Next we prove the axioms of standard ZF set theory that were missing from our
axiom system.  From our point of view they are theorems since they
can be derived from the other axioms.

\vskip 2ex
\noindent Axiom of Separation\index{Axiom of Separation}
(Aussonderung)\index{Aussonderung} proved from the other axioms of ZF set
theory.  Compare Exercise 4 of Takeuti and Zaring, p.~22.

\vskip 0.5ex
\setbox\startprefix=\hbox{\tt \ \ inex1.1\ \$e\ }
\setbox\contprefix=\hbox{\tt \ \ \ \ \ \ \ \ \ \ \ \ \ \ \ }
\startm
\m{\vdash}\m{A}\m{\in}\m{{\rm V}}
\endm
\setbox\startprefix=\hbox{\tt \ \ inex\ \$p\ }
\setbox\contprefix=\hbox{\tt \ \ \ \ \ \ \ \ \ \ \ \ \ }
\startm
\m{\vdash}\m{(}\m{A}\m{\cap}\m{B}\m{)}\m{\in}\m{{\rm V}}
\endm
\vskip 1ex

\noindent Axiom of the Null Set\index{Axiom of the Null Set} proved from the
other axioms of ZF set theory. Corollary 5.16 of Takeuti and Zaring, p.~20.

\vskip 0.5ex
\setbox\startprefix=\hbox{\tt \ \ 0ex\ \$p\ }
\setbox\contprefix=\hbox{\tt \ \ \ \ \ \ \ \ \ \ \ \ }
\startm
\m{\vdash}\m{\varnothing}\m{\in}\m{{\rm V}}
\endm
\vskip 1ex

\noindent The Axiom of Pairing\index{Axiom of Pairing} proved from the other
axioms of ZF set theory.  Theorem 7.13 of Quine, p.~51.
\vskip 0.5ex
\setbox\startprefix=\hbox{\tt \ \ prex\ \$p\ }
\setbox\contprefix=\hbox{\tt \ \ \ \ \ \ \ \ \ \ \ \ \ \ }
\startm
\m{\vdash}\m{\{}\m{A}\m{,}\m{B}\m{\}}\m{\in}\m{{\rm V}}
\endm
\vskip 2ex

Next we will list some famous or important theorems that are proved in
the \texttt{set.mm} database.  None of them except \texttt{omex}
require the Axiom of Infinity, as you can verify with the \texttt{show
trace{\char`\_}back} Metamath command.

\vskip 2ex
\noindent The resolution of Russell's paradox\index{Russell's paradox}.  There
exists no set containing the set of all sets which are not members of
themselves.  Proposition 4.14 of Takeuti and Zaring, p.~14.

\vskip 0.5ex
\setbox\startprefix=\hbox{\tt \ \ ru\ \$p\ }
\setbox\contprefix=\hbox{\tt \ \ \ \ \ \ \ \ }
\startm
\m{\vdash}\m{\lnot}\m{\exists}\m{x}\m{\,x}\m{=}\m{\{}\m{y}\m{|}\m{\lnot}\m{y}
\m{\in}\m{y}\m{\}}
\endm
\vskip 1ex

\noindent Cantor's theorem\index{Cantor's theorem}.  No set can be mapped onto
its power set.  Compare Theorem 6B(b) of Enderton, p.~132.

\vskip 0.5ex
\setbox\startprefix=\hbox{\tt \ \ canth.1\ \$e\ }
\setbox\contprefix=\hbox{\tt \ \ \ \ \ \ \ \ \ \ \ \ \ }
\startm
\m{\vdash}\m{A}\m{\in}\m{{\rm V}}
\endm
\setbox\startprefix=\hbox{\tt \ \ canth\ \$p\ }
\setbox\contprefix=\hbox{\tt \ \ \ \ \ \ \ \ \ \ \ }
\startm
\m{\vdash}\m{\lnot}\m{F}\m{:}\m{A}\m{\raisebox{.5ex}{${\textstyle{\:}_{
\mbox{\footnotesize\rm {\ }}}}\atop{\textstyle{\longrightarrow}\atop{
\textstyle{}^{\mbox{\footnotesize\rm onto}}}}$}}\m{{\cal P}}\m{A}
\endm
\vskip 1ex

\noindent The Burali-Forti paradox\index{Burali-Forti paradox}.  No set
contains all ordinal numbers. Enderton, p.~194.  (Burali-Forti was one person,
not two.)

\vskip 0.5ex
\setbox\startprefix=\hbox{\tt \ \ onprc\ \$p\ }
\setbox\contprefix=\hbox{\tt \ \ \ \ \ \ \ \ \ \ \ \ }
\startm
\m{\vdash}\m{\lnot}\m{\mbox{\rm On}}\m{\in}\m{{\rm V}}
\endm
\vskip 1ex

\noindent Peano's postulates\index{Peano's postulates} for arithmetic.
Proposition 7.30 of Takeuti and Zaring, pp.~42--43.  The objects being
described are the members of $\omega$ i.e.\ the natural numbers 0, 1,
2,\ldots.  The successor\index{successor} operation suc means ``plus
one.''  \texttt{peano1} says that 0 (which is defined as the empty set)
is a natural number.  \texttt{peano2} says that if $A$ is a natural
number, so is $A+1$.  \texttt{peano3} says that 0 is not the successor
of any natural number.  \texttt{peano4} says that two natural numbers
are equal if and only if their successors are equal.  \texttt{peano5} is
essentially the same as mathematical induction.

\vskip 1ex
\setbox\startprefix=\hbox{\tt \ \ peano1\ \$p\ }
\setbox\contprefix=\hbox{\tt \ \ \ \ \ \ \ \ \ \ \ \ }
\startm
\m{\vdash}\m{\varnothing}\m{\in}\m{\omega}
\endm
\vskip 1.5ex

\setbox\startprefix=\hbox{\tt \ \ peano2\ \$p\ }
\setbox\contprefix=\hbox{\tt \ \ \ \ \ \ \ \ \ \ \ \ }
\startm
\m{\vdash}\m{(}\m{A}\m{\in}\m{\omega}\m{\rightarrow}\m{{\rm suc}}\m{A}\m{\in}%
\m{\omega}\m{)}
\endm
\vskip 1.5ex

\setbox\startprefix=\hbox{\tt \ \ peano3\ \$p\ }
\setbox\contprefix=\hbox{\tt \ \ \ \ \ \ \ \ \ \ \ \ }
\startm
\m{\vdash}\m{(}\m{A}\m{\in}\m{\omega}\m{\rightarrow}\m{\lnot}\m{{\rm suc}}%
\m{A}\m{=}\m{\varnothing}\m{)}
\endm
\vskip 1.5ex

\setbox\startprefix=\hbox{\tt \ \ peano4\ \$p\ }
\setbox\contprefix=\hbox{\tt \ \ \ \ \ \ \ \ \ \ \ \ }
\startm
\m{\vdash}\m{(}\m{(}\m{A}\m{\in}\m{\omega}\m{\wedge}\m{B}\m{\in}\m{\omega}%
\m{)}\m{\rightarrow}\m{(}\m{{\rm suc}}\m{A}\m{=}\m{{\rm suc}}\m{B}\m{%
\leftrightarrow}\m{A}\m{=}\m{B}\m{)}\m{)}
\endm
\vskip 1.5ex

\setbox\startprefix=\hbox{\tt \ \ peano5\ \$p\ }
\setbox\contprefix=\hbox{\tt \ \ \ \ \ \ \ \ \ \ \ \ }
\startm
\m{\vdash}\m{(}\m{(}\m{\varnothing}\m{\in}\m{A}\m{\wedge}\m{\forall}\m{x}\m{%
\in}\m{\omega}\m{(}\m{x}\m{\in}\m{A}\m{\rightarrow}\m{{\rm suc}}\m{x}\m{\in}%
\m{A}\m{)}\m{)}\m{\rightarrow}\m{\omega}\m{\subseteq}\m{A}\m{)}
\endm
\vskip 1.5ex

\noindent Finite Induction (mathematical induction).\index{finite
induction}\index{mathematical induction} The first hypothesis is the
basis and the second is the induction hypothesis.  Theorem Schema 22 of
Suppes, p.~136.

\vskip 0.5ex
\setbox\startprefix=\hbox{\tt \ \ findes.1\ \$e\ }
\setbox\contprefix=\hbox{\tt \ \ \ \ \ \ \ \ \ \ \ \ \ \ }
\startm
\m{\vdash}\m{[}\m{\varnothing}\m{/}\m{x}\m{]}\m{\varphi}
\endm
\setbox\startprefix=\hbox{\tt \ \ findes.2\ \$e\ }
\setbox\contprefix=\hbox{\tt \ \ \ \ \ \ \ \ \ \ \ \ \ \ }
\startm
\m{\vdash}\m{(}\m{x}\m{\in}\m{\omega}\m{\rightarrow}\m{(}\m{\varphi}\m{%
\rightarrow}\m{[}\m{{\rm suc}}\m{x}\m{/}\m{x}\m{]}\m{\varphi}\m{)}\m{)}
\endm
\setbox\startprefix=\hbox{\tt \ \ findes\ \$p\ }
\setbox\contprefix=\hbox{\tt \ \ \ \ \ \ \ \ \ \ \ \ }
\startm
\m{\vdash}\m{(}\m{x}\m{\in}\m{\omega}\m{\rightarrow}\m{\varphi}\m{)}
\endm
\vskip 1ex

\noindent Transfinite Induction with explicit substitution.  The first
hypothesis is the basis, the second is the induction hypothesis for
successors, and the third is the induction hypothesis for limit
ordinals.  Theorem Schema 4 of Suppes, p. 197.

\vskip 0.5ex
\setbox\startprefix=\hbox{\tt \ \ tfindes.1\ \$e\ }
\setbox\contprefix=\hbox{\tt \ \ \ \ \ \ \ \ \ \ \ \ \ \ \ }
\startm
\m{\vdash}\m{[}\m{\varnothing}\m{/}\m{x}\m{]}\m{\varphi}
\endm
\setbox\startprefix=\hbox{\tt \ \ tfindes.2\ \$e\ }
\setbox\contprefix=\hbox{\tt \ \ \ \ \ \ \ \ \ \ \ \ \ \ \ }
\startm
\m{\vdash}\m{(}\m{x}\m{\in}\m{{\rm On}}\m{\rightarrow}\m{(}\m{\varphi}\m{%
\rightarrow}\m{[}\m{{\rm suc}}\m{x}\m{/}\m{x}\m{]}\m{\varphi}\m{)}\m{)}
\endm
\setbox\startprefix=\hbox{\tt \ \ tfindes.3\ \$e\ }
\setbox\contprefix=\hbox{\tt \ \ \ \ \ \ \ \ \ \ \ \ \ \ \ }
\startm
\m{\vdash}\m{(}\m{{\rm Lim}}\m{y}\m{\rightarrow}\m{(}\m{\forall}\m{x}\m{\in}%
\m{y}\m{\varphi}\m{\rightarrow}\m{[}\m{y}\m{/}\m{x}\m{]}\m{\varphi}\m{)}\m{)}
\endm
\setbox\startprefix=\hbox{\tt \ \ tfindes\ \$p\ }
\setbox\contprefix=\hbox{\tt \ \ \ \ \ \ \ \ \ \ \ \ \ }
\startm
\m{\vdash}\m{(}\m{x}\m{\in}\m{{\rm On}}\m{\rightarrow}\m{\varphi}\m{)}
\endm
\vskip 1ex

\noindent Principle of Transfinite Recursion.\index{transfinite
recursion} Theorem 7.41 of Takeuti and Zaring, p.~47.  Transfinite
recursion is the key theorem that allows arithmetic of ordinals to be
rigorously defined, and has many other important uses as well.
Hypotheses \texttt{tfr.1} and \texttt{tfr.2} specify a certain (proper)
class $ F$.  The complicated definition of $ F$ is not important in
itself; what is important is that there be such an $ F$ with the
required properties, and we show this by displaying $ F$ explicitly.
\texttt{tfr1} states that $ F$ is a function whose domain is the set of
ordinal numbers.  \texttt{tfr2} states that any value of $ F$ is
completely determined by its previous values and the values of an
auxiliary function, $G$.  \texttt{tfr3} states that $ F$ is unique,
i.e.\ it is the only function that satisfies \texttt{tfr1} and
\texttt{tfr2}.  Note that $ f$ is an individual variable like $x$ and
$y$; it is just a mnemonic to remind us that $A$ is a collection of
functions.

\vskip 0.5ex
\setbox\startprefix=\hbox{\tt \ \ tfr.1\ \$e\ }
\setbox\contprefix=\hbox{\tt \ \ \ \ \ \ \ \ \ \ \ }
\startm
\m{\vdash}\m{A}\m{=}\m{\{}\m{f}\m{|}\m{\exists}\m{x}\m{\in}\m{{\rm On}}\m{(}%
\m{f}\m{{\rm Fn}}\m{x}\m{\wedge}\m{\forall}\m{y}\m{\in}\m{x}\m{(}\m{f}\m{`}%
\m{y}\m{)}\m{=}\m{(}\m{G}\m{`}\m{(}\m{f}\m{\restriction}\m{y}\m{)}\m{)}\m{)}%
\m{\}}
\endm
\setbox\startprefix=\hbox{\tt \ \ tfr.2\ \$e\ }
\setbox\contprefix=\hbox{\tt \ \ \ \ \ \ \ \ \ \ \ }
\startm
\m{\vdash}\m{F}\m{=}\m{\bigcup}\m{A}
\endm
\setbox\startprefix=\hbox{\tt \ \ tfr1\ \$p\ }
\setbox\contprefix=\hbox{\tt \ \ \ \ \ \ \ \ \ \ }
\startm
\m{\vdash}\m{F}\m{{\rm Fn}}\m{{\rm On}}
\endm
\setbox\startprefix=\hbox{\tt \ \ tfr2\ \$p\ }
\setbox\contprefix=\hbox{\tt \ \ \ \ \ \ \ \ \ \ }
\startm
\m{\vdash}\m{(}\m{z}\m{\in}\m{{\rm On}}\m{\rightarrow}\m{(}\m{F}\m{`}\m{z}%
\m{)}\m{=}\m{(}\m{G}\m{`}\m{(}\m{F}\m{\restriction}\m{z}\m{)}\m{)}\m{)}
\endm
\setbox\startprefix=\hbox{\tt \ \ tfr3\ \$p\ }
\setbox\contprefix=\hbox{\tt \ \ \ \ \ \ \ \ \ \ }
\startm
\m{\vdash}\m{(}\m{(}\m{B}\m{{\rm Fn}}\m{{\rm On}}\m{\wedge}\m{\forall}\m{x}\m{%
\in}\m{{\rm On}}\m{(}\m{B}\m{`}\m{x}\m{)}\m{=}\m{(}\m{G}\m{`}\m{(}\m{B}\m{%
\restriction}\m{x}\m{)}\m{)}\m{)}\m{\rightarrow}\m{B}\m{=}\m{F}\m{)}
\endm
\vskip 1ex

\noindent The existence of omega (the class of natural numbers).\index{natural
number}\index{omega ($\omega$)}\index{Axiom of Infinity}  Axiom 7 of Takeuti
and Zaring, p.~43.  (This is the only theorem in this section requiring the
Axiom of Infinity.)

\vskip 0.5ex
\setbox\startprefix=\hbox{\tt \
\ omex\ \$p\ }
\setbox\contprefix=\hbox{\tt \ \ \ \ \ \ \ \ \ \ }
\startm
\m{\vdash}\m{\omega}\m{\in}\m{{\rm V}}
\endm
%\vskip 2ex


\section{Axioms for Real and Complex Numbers}\label{real}
\index{real and complex numbers!axioms for}

This section presents the axioms for real and complex numbers.  Analysis
textbooks implicitly or explicitly use these axioms or their equivalents
as their starting point.  In the database \texttt{set.mm}, we define real
and complex numbers as (rather complicated) specific sets and derive these
axioms as {\em theorems} from the axioms of ZF set theory, using a method
called Dedekind cuts.  We omit the details of this construction, which you can
follow if you wish using the \texttt{set.mm} database in conjunction with the
textbooks referenced therein.

Once we prove those theorems, we then restate these proven theorems as axioms.
This lets us easily identify which axioms are needed for a particular complex number proof, without the obfuscation of the set theory used to derive them.
As a result,
the construction is actually unimportant other
than to show that sets exist that satisfy the axioms, and thus that the axioms
are consistent if ZF set theory is consistent.  When working with real numbers
you can think of them as being the actual sets resulting
from the construction (for definiteness), or you can
think of them as otherwise unspecified sets that happen to satisfy the axioms.
The derivation is not easy, but the fact that it works is quite remarkable
and lends support to the idea that ZFC set theory is all we need to
provide a foundation for essentially all of mathematics.

For the axioms we are given (or postulate) 8 classes:  $\mathbb{C}$ (the
set of complex numbers), $\mathbb{R}$ (the set of real numbers, a subset
of $\mathbb{C}$), $0$ (zero), $1$ (one), $i$ (square root of
$-1$), $+$ (plus), $\cdot$ (times), and
$<_{\mathbb{R}}$ (less than for just the real numbers).
Subtraction and division are defined terms and are not part of the
axioms; for their definitions see \texttt{set.mm}.

Note that the notation $(A+B)$ (and similarly $(A\cdot B)$) specifies a class
called an {\em operation},\index{operation} and is the function value of the
class $+$ at ordered pair $\langle A,B \rangle$.  An operation is defined by
statement \texttt{df-opr} on p.~\pageref{dfopr}.
The notation $A <_{\mathbb{R}} B$ specifies a
wff called a {\em binary relation}\index{binary relation} and means $\langle A,B \rangle \in \,<_{\mathbb{R}}$, as defined by statement \texttt{df-br} on p.~\pageref{dfbr}.

Our set of 8 given classes is assumed to satisfy the following 22 axioms
(in the axioms listed below, $<$ really means $<_{\mathbb{R}}$).

\vskip 2ex

\noindent 1. The real numbers are a subset of the complex numbers.

%\vskip 0.5ex
\setbox\startprefix=\hbox{\tt \ \ ax-resscn\ \$p\ }
\setbox\contprefix=\hbox{\tt \ \ \ \ \ \ \ \ \ \ \ \ \ \ }
\startm
\m{\vdash}\m{\mathbb{R}}\m{\subseteq}\m{\mathbb{C}}
\endm
%\vskip 1ex

\noindent 2. One is a complex number.

%\vskip 0.5ex
\setbox\startprefix=\hbox{\tt \ \ ax-1cn\ \$p\ }
\setbox\contprefix=\hbox{\tt \ \ \ \ \ \ \ \ \ \ \ }
\startm
\m{\vdash}\m{1}\m{\in}\m{\mathbb{C}}
\endm
%\vskip 1ex

\noindent 3. The imaginary number $i$ is a complex number.

%\vskip 0.5ex
\setbox\startprefix=\hbox{\tt \ \ ax-icn\ \$p\ }
\setbox\contprefix=\hbox{\tt \ \ \ \ \ \ \ \ \ \ \ }
\startm
\m{\vdash}\m{i}\m{\in}\m{\mathbb{C}}
\endm
%\vskip 1ex

\noindent 4. Complex numbers are closed under addition.

%\vskip 0.5ex
\setbox\startprefix=\hbox{\tt \ \ ax-addcl\ \$p\ }
\setbox\contprefix=\hbox{\tt \ \ \ \ \ \ \ \ \ \ \ \ \ }
\startm
\m{\vdash}\m{(}\m{(}\m{A}\m{\in}\m{\mathbb{C}}\m{\wedge}\m{B}\m{\in}\m{\mathbb{C}}%
\m{)}\m{\rightarrow}\m{(}\m{A}\m{+}\m{B}\m{)}\m{\in}\m{\mathbb{C}}\m{)}
\endm
%\vskip 1ex

\noindent 5. Real numbers are closed under addition.

%\vskip 0.5ex
\setbox\startprefix=\hbox{\tt \ \ ax-addrcl\ \$p\ }
\setbox\contprefix=\hbox{\tt \ \ \ \ \ \ \ \ \ \ \ \ \ \ }
\startm
\m{\vdash}\m{(}\m{(}\m{A}\m{\in}\m{\mathbb{R}}\m{\wedge}\m{B}\m{\in}\m{\mathbb{R}}%
\m{)}\m{\rightarrow}\m{(}\m{A}\m{+}\m{B}\m{)}\m{\in}\m{\mathbb{R}}\m{)}
\endm
%\vskip 1ex

\noindent 6. Complex numbers are closed under multiplication.

%\vskip 0.5ex
\setbox\startprefix=\hbox{\tt \ \ ax-mulcl\ \$p\ }
\setbox\contprefix=\hbox{\tt \ \ \ \ \ \ \ \ \ \ \ \ \ }
\startm
\m{\vdash}\m{(}\m{(}\m{A}\m{\in}\m{\mathbb{C}}\m{\wedge}\m{B}\m{\in}\m{\mathbb{C}}%
\m{)}\m{\rightarrow}\m{(}\m{A}\m{\cdot}\m{B}\m{)}\m{\in}\m{\mathbb{C}}\m{)}
\endm
%\vskip 1ex

\noindent 7. Real numbers are closed under multiplication.

%\vskip 0.5ex
\setbox\startprefix=\hbox{\tt \ \ ax-mulrcl\ \$p\ }
\setbox\contprefix=\hbox{\tt \ \ \ \ \ \ \ \ \ \ \ \ \ \ }
\startm
\m{\vdash}\m{(}\m{(}\m{A}\m{\in}\m{\mathbb{R}}\m{\wedge}\m{B}\m{\in}\m{\mathbb{R}}%
\m{)}\m{\rightarrow}\m{(}\m{A}\m{\cdot}\m{B}\m{)}\m{\in}\m{\mathbb{R}}\m{)}
\endm
%\vskip 1ex

\noindent 8. Multiplication of complex numbers is commutative.

%\vskip 0.5ex
\setbox\startprefix=\hbox{\tt \ \ ax-mulcom\ \$p\ }
\setbox\contprefix=\hbox{\tt \ \ \ \ \ \ \ \ \ \ \ \ \ \ }
\startm
\m{\vdash}\m{(}\m{(}\m{A}\m{\in}\m{\mathbb{C}}\m{\wedge}\m{B}\m{\in}\m{\mathbb{C}}%
\m{)}\m{\rightarrow}\m{(}\m{A}\m{\cdot}\m{B}\m{)}\m{=}\m{(}\m{B}\m{\cdot}\m{A}%
\m{)}\m{)}
\endm
%\vskip 1ex

\noindent 9. Addition of complex numbers is associative.

%\vskip 0.5ex
\setbox\startprefix=\hbox{\tt \ \ ax-addass\ \$p\ }
\setbox\contprefix=\hbox{\tt \ \ \ \ \ \ \ \ \ \ \ \ \ \ }
\startm
\m{\vdash}\m{(}\m{(}\m{A}\m{\in}\m{\mathbb{C}}\m{\wedge}\m{B}\m{\in}\m{\mathbb{C}}%
\m{\wedge}\m{C}\m{\in}\m{\mathbb{C}}\m{)}\m{\rightarrow}\m{(}\m{(}\m{A}\m{+}%
\m{B}\m{)}\m{+}\m{C}\m{)}\m{=}\m{(}\m{A}\m{+}\m{(}\m{B}\m{+}\m{C}\m{)}\m{)}%
\m{)}
\endm
%\vskip 1ex

\noindent 10. Multiplication of complex numbers is associative.

%\vskip 0.5ex
\setbox\startprefix=\hbox{\tt \ \ ax-mulass\ \$p\ }
\setbox\contprefix=\hbox{\tt \ \ \ \ \ \ \ \ \ \ \ \ \ \ }
\startm
\m{\vdash}\m{(}\m{(}\m{A}\m{\in}\m{\mathbb{C}}\m{\wedge}\m{B}\m{\in}\m{\mathbb{C}}%
\m{\wedge}\m{C}\m{\in}\m{\mathbb{C}}\m{)}\m{\rightarrow}\m{(}\m{(}\m{A}\m{\cdot}%
\m{B}\m{)}\m{\cdot}\m{C}\m{)}\m{=}\m{(}\m{A}\m{\cdot}\m{(}\m{B}\m{\cdot}\m{C}%
\m{)}\m{)}\m{)}
\endm
%\vskip 1ex

\noindent 11. Multiplication distributes over addition for complex numbers.

%\vskip 0.5ex
\setbox\startprefix=\hbox{\tt \ \ ax-distr\ \$p\ }
\setbox\contprefix=\hbox{\tt \ \ \ \ \ \ \ \ \ \ \ \ \ }
\startm
\m{\vdash}\m{(}\m{(}\m{A}\m{\in}\m{\mathbb{C}}\m{\wedge}\m{B}\m{\in}\m{\mathbb{C}}%
\m{\wedge}\m{C}\m{\in}\m{\mathbb{C}}\m{)}\m{\rightarrow}\m{(}\m{A}\m{\cdot}\m{(}%
\m{B}\m{+}\m{C}\m{)}\m{)}\m{=}\m{(}\m{(}\m{A}\m{\cdot}\m{B}\m{)}\m{+}\m{(}%
\m{A}\m{\cdot}\m{C}\m{)}\m{)}\m{)}
\endm
%\vskip 1ex

\noindent 12. The square of $i$ equals $-1$ (expressed as $i$-squared plus 1 is
0).

%\vskip 0.5ex
\setbox\startprefix=\hbox{\tt \ \ ax-i2m1\ \$p\ }
\setbox\contprefix=\hbox{\tt \ \ \ \ \ \ \ \ \ \ \ \ }
\startm
\m{\vdash}\m{(}\m{(}\m{i}\m{\cdot}\m{i}\m{)}\m{+}\m{1}\m{)}\m{=}\m{0}
\endm
%\vskip 1ex

\noindent 13. One and zero are distinct.

%\vskip 0.5ex
\setbox\startprefix=\hbox{\tt \ \ ax-1ne0\ \$p\ }
\setbox\contprefix=\hbox{\tt \ \ \ \ \ \ \ \ \ \ \ \ }
\startm
\m{\vdash}\m{1}\m{\ne}\m{0}
\endm
%\vskip 1ex

\noindent 14. One is an identity element for real multiplication.

%\vskip 0.5ex
\setbox\startprefix=\hbox{\tt \ \ ax-1rid\ \$p\ }
\setbox\contprefix=\hbox{\tt \ \ \ \ \ \ \ \ \ \ \ }
\startm
\m{\vdash}\m{(}\m{A}\m{\in}\m{\mathbb{R}}\m{\rightarrow}\m{(}\m{A}\m{\cdot}\m{1}%
\m{)}\m{=}\m{A}\m{)}
\endm
%\vskip 1ex

\noindent 15. Every real number has a negative.

%\vskip 0.5ex
\setbox\startprefix=\hbox{\tt \ \ ax-rnegex\ \$p\ }
\setbox\contprefix=\hbox{\tt \ \ \ \ \ \ \ \ \ \ \ \ \ \ }
\startm
\m{\vdash}\m{(}\m{A}\m{\in}\m{\mathbb{R}}\m{\rightarrow}\m{\exists}\m{x}\m{\in}%
\m{\mathbb{R}}\m{(}\m{A}\m{+}\m{x}\m{)}\m{=}\m{0}\m{)}
\endm
%\vskip 1ex

\noindent 16. Every nonzero real number has a reciprocal.

%\vskip 0.5ex
\setbox\startprefix=\hbox{\tt \ \ ax-rrecex\ \$p\ }
\setbox\contprefix=\hbox{\tt \ \ \ \ \ \ \ \ \ \ \ \ \ \ }
\startm
\m{\vdash}\m{(}\m{A}\m{\in}\m{\mathbb{R}}\m{\rightarrow}\m{(}\m{A}\m{\ne}\m{0}%
\m{\rightarrow}\m{\exists}\m{x}\m{\in}\m{\mathbb{R}}\m{(}\m{A}\m{\cdot}%
\m{x}\m{)}\m{=}\m{1}\m{)}\m{)}
\endm
%\vskip 1ex

\noindent 17. A complex number can be expressed in terms of two reals.

%\vskip 0.5ex
\setbox\startprefix=\hbox{\tt \ \ ax-cnre\ \$p\ }
\setbox\contprefix=\hbox{\tt \ \ \ \ \ \ \ \ \ \ \ \ }
\startm
\m{\vdash}\m{(}\m{A}\m{\in}\m{\mathbb{C}}\m{\rightarrow}\m{\exists}\m{x}\m{\in}%
\m{\mathbb{R}}\m{\exists}\m{y}\m{\in}\m{\mathbb{R}}\m{A}\m{=}\m{(}\m{x}\m{+}\m{(}%
\m{y}\m{\cdot}\m{i}\m{)}\m{)}\m{)}
\endm
%\vskip 1ex

\noindent 18. Ordering on reals satisfies strict trichotomy.

%\vskip 0.5ex
\setbox\startprefix=\hbox{\tt \ \ ax-pre-lttri\ \$p\ }
\setbox\contprefix=\hbox{\tt \ \ \ \ \ \ \ \ \ \ \ \ \ }
\startm
\m{\vdash}\m{(}\m{(}\m{A}\m{\in}\m{\mathbb{R}}\m{\wedge}\m{B}\m{\in}\m{\mathbb{R}}%
\m{)}\m{\rightarrow}\m{(}\m{A}\m{<}\m{B}\m{\leftrightarrow}\m{\lnot}\m{(}\m{A}%
\m{=}\m{B}\m{\vee}\m{B}\m{<}\m{A}\m{)}\m{)}\m{)}
\endm
%\vskip 1ex

\noindent 19. Ordering on reals is transitive.

%\vskip 0.5ex
\setbox\startprefix=\hbox{\tt \ \ ax-pre-lttrn\ \$p\ }
\setbox\contprefix=\hbox{\tt \ \ \ \ \ \ \ \ \ \ \ \ \ }
\startm
\m{\vdash}\m{(}\m{(}\m{A}\m{\in}\m{\mathbb{R}}\m{\wedge}\m{B}\m{\in}\m{\mathbb{R}}%
\m{\wedge}\m{C}\m{\in}\m{\mathbb{R}}\m{)}\m{\rightarrow}\m{(}\m{(}\m{A}\m{<}%
\m{B}\m{\wedge}\m{B}\m{<}\m{C}\m{)}\m{\rightarrow}\m{A}\m{<}\m{C}\m{)}\m{)}
\endm
%\vskip 1ex

\noindent 20. Ordering on reals is preserved after addition to both sides.

%\vskip 0.5ex
\setbox\startprefix=\hbox{\tt \ \ ax-pre-ltadd\ \$p\ }
\setbox\contprefix=\hbox{\tt \ \ \ \ \ \ \ \ \ \ \ \ \ }
\startm
\m{\vdash}\m{(}\m{(}\m{A}\m{\in}\m{\mathbb{R}}\m{\wedge}\m{B}\m{\in}\m{\mathbb{R}}%
\m{\wedge}\m{C}\m{\in}\m{\mathbb{R}}\m{)}\m{\rightarrow}\m{(}\m{A}\m{<}\m{B}\m{%
\rightarrow}\m{(}\m{C}\m{+}\m{A}\m{)}\m{<}\m{(}\m{C}\m{+}\m{B}\m{)}\m{)}\m{)}
\endm
%\vskip 1ex

\noindent 21. The product of two positive reals is positive.

%\vskip 0.5ex
\setbox\startprefix=\hbox{\tt \ \ ax-pre-mulgt0\ \$p\ }
\setbox\contprefix=\hbox{\tt \ \ \ \ \ \ \ \ \ \ \ \ \ \ }
\startm
\m{\vdash}\m{(}\m{(}\m{A}\m{\in}\m{\mathbb{R}}\m{\wedge}\m{B}\m{\in}\m{\mathbb{R}}%
\m{)}\m{\rightarrow}\m{(}\m{(}\m{0}\m{<}\m{A}\m{\wedge}\m{0}%
\m{<}\m{B}\m{)}\m{\rightarrow}\m{0}\m{<}\m{(}\m{A}\m{\cdot}\m{B}\m{)}%
\m{)}\m{)}
\endm
%\vskip 1ex

\noindent 22. A non-empty, bounded-above set of reals has a supremum.

%\vskip 0.5ex
\setbox\startprefix=\hbox{\tt \ \ ax-pre-sup\ \$p\ }
\setbox\contprefix=\hbox{\tt \ \ \ \ \ \ \ \ \ \ \ }
\startm
\m{\vdash}\m{(}\m{(}\m{A}\m{\subseteq}\m{\mathbb{R}}\m{\wedge}\m{A}\m{\ne}\m{%
\varnothing}\m{\wedge}\m{\exists}\m{x}\m{\in}\m{\mathbb{R}}\m{\forall}\m{y}\m{%
\in}\m{A}\m{\,y}\m{<}\m{x}\m{)}\m{\rightarrow}\m{\exists}\m{x}\m{\in}\m{%
\mathbb{R}}\m{(}\m{\forall}\m{y}\m{\in}\m{A}\m{\lnot}\m{x}\m{<}\m{y}\m{\wedge}\m{%
\forall}\m{y}\m{\in}\m{\mathbb{R}}\m{(}\m{y}\m{<}\m{x}\m{\rightarrow}\m{\exists}%
\m{z}\m{\in}\m{A}\m{\,y}\m{<}\m{z}\m{)}\m{)}\m{)}
\endm

% NOTE: The \m{...} expressions above could be represented as
% $ \vdash ( ( A \subseteq \mathbb{R} \wedge A \ne \varnothing \wedge \exists x \in \mathbb{R} \forall y \in A \,y < x ) \rightarrow \exists x \in \mathbb{R} ( \forall y \in A \lnot x < y \wedge \forall y \in \mathbb{R} ( y < x \rightarrow \exists z \in A \,y < z ) ) ) $

\vskip 2ex

This completes the set of axioms for real and complex numbers.  You may
wish to look at how subtraction, division, and decimal numbers
are defined in \texttt{set.mm}, and for fun look at the proof of $2+
2 = 4$ (theorem \texttt{2p2e4} in \texttt{set.mm})
as discussed in section \ref{2p2e4}.

In \texttt{set.mm} we define the non-negative integers $\mathbb{N}$, the integers
$\mathbb{Z}$, and the rationals $\mathbb{Q}$ as subsets of $\mathbb{R}$.  This leads
to the nice inclusion $\mathbb{N} \subseteq \mathbb{Z} \subseteq \mathbb{Q} \subseteq
\mathbb{R} \subseteq \mathbb{C}$, giving us a uniform framework in which, for
example, a property such as commutativity of complex number addition
automatically applies to integers.  The natural numbers $\mathbb{N}$
are different from the set $\omega$ we defined earlier, but both satisfy
Peano's postulates.

\subsubsection{Complex Number Axioms in Analysis Texts}

Most analysis texts construct complex numbers as ordered pairs of reals,
leading to construction-dependent properties that satisfy these axioms
but are not stated in their pure form.  (This is also done in
\texttt{set.mm} but our axioms are extracted from that construction.)
Other texts will simply state that $\mathbb{R}$ is a ``complete ordered
subfield of $\mathbb{C}$,'' leading to redundant axioms when this phrase
is completely expanded out.  In fact I have not seen a text with the
axioms in the explicit form above.
None of these axioms is unique individually, but this carefully worked out
collection of axioms is the result of years of work
by the Metamath community.

\subsubsection{Eliminating Unnecessary Complex Number Axioms}

We once had more axioms for real and complex numbers, but over years of time
we (the Metamath community)
have found ways to eliminate them (by proving them from other axioms)
or weaken them (by making weaker claims without reducing what
can be proved).
In particular, here are statements that used to be complex number
axioms but have since been formally proven (with Metamath) to be redundant:

\begin{itemize}
\item
  $\mathbb{C} \in V$.
  At one time this was listed as a ``complex number axiom''
  However, this is not properly speaking a complex number axiom,
  and in any case its proof uses axioms of set theory.
  Proven redundant by Mario Carneiro\index{Carneiro, Mario} on
  17-Nov-2014 (see \texttt{axcnex}).
\item
  $((A \in \mathbb{C} \land B \in \mathbb{C}$) $\rightarrow$
  $(A + B) = (B + A))$.
  Proved redundant by Eric Schmidt\index{Schmidt, Eric} on 19-Jun-2012,
  and formalized by Scott Fenton\index{Fenton, Scott} on 3-Jan-2013
  (see \texttt{addcom}).
\item
  $(A \in \mathbb{C} \rightarrow (A + 0) = A)$.
  Proved redundant by Eric Schmidt on 19-Jun-2012,
  and formalized by Scott Fenton on 3-Jan-2013
  (see \texttt{addid1}).
\item
  $(A \in \mathbb{C} \rightarrow \exists x \in \mathbb{C} (A + x) = 0)$.
  Proved redundant by Eric Schmidt and formalized on 21-May-2007
  (see \texttt{cnegex}).
\item
  $((A \in \mathbb{C} \land A \ne 0) \rightarrow \exists x \in \mathbb{C} (A \cdot x) = 1)$.
  Proved redundant by Eric Schmidt and formalized on 22-May-2007
  (see \texttt{recex}).
\item
  $0 \in \mathbb{R}$.
  Proved redundant by Eric Schmidt on 19-Feb-2005 and formalized 21-May-2007
  (see \texttt{0re}).
\end{itemize}

We could eliminate 0 as an axiomatic object by defining it as
$( ( i \cdot i ) + 1 )$
and replacing it with this expression throughout the axioms. If this
is done, axiom ax-i2m1 becomes redundant. However, the remaining axioms
would become longer and less intuitive.

Eric Schmidt's paper analyzing this axiom system \cite{Schmidt}
presented a proof that these remaining axioms,
with the possible exception of ax-mulcom, are independent of the others.
It is currently an open question if ax-mulcom is independent of the others.

\section{Two Plus Two Equals Four}\label{2p2e4}

Here is a proof that $2 + 2 = 4$, as proven in the theorem \texttt{2p2e4}
in the database \texttt{set.mm}.
This is a useful demonstration of what a Metamath proof can look like.
This proof may have more steps than you're used to, but each step is rigorously
proven all the way back to the axioms of logic and set theory.
This display was originally generated by the Metamath program
as an {\sc HTML} file.

In the table showing the proof ``Step'' is the sequential step number,
while its associated ``Expression'' is an expression that we have proved.
``Ref'' is the name of a theorem or axiom that justifies that expression,
and ``Hyp'' refers to previous steps (if any) that the theorem or axiom
needs so that we can use it.  Expressions are indented further than
the expressions that depend on them to show their interdependencies.

\begin{table}[!htbp]
\caption{Two plus two equals four}
\begin{tabular}{lllll}
\textbf{Step} & \textbf{Hyp} & \textbf{Ref} & \textbf{Expression} & \\
1  &       & df-2    & $ \; \; \vdash 2 = 1 + 1$  & \\
2  & 1     & oveq2i  & $ \; \vdash (2 + 2) = (2 + (1 + 1))$ & \\
3  &       & df-4    & $ \; \; \vdash 4 = (3 + 1)$ & \\
4  &       & df-3    & $ \; \; \; \vdash 3 = (2 + 1)$ & \\
5  & 4     & oveq1i  & $ \; \; \vdash (3 + 1) = ((2 + 1) + 1)$ & \\
6  &       & 2cn     & $ \; \; \; \vdash 2 \in \mathbb{C}$ & \\
7  &       & ax-1cn  & $ \; \; \; \vdash 1 \in \mathbb{C}$ & \\
8  & 6,7,7 & addassi & $ \; \; \vdash ((2 + 1) + 1) = (2 + (1 + 1))$ & \\
9  & 3,5,8 & 3eqtri  & $ \; \vdash 4 = (2 + (1 + 1))$ & \\
10 & 2,9   & eqtr4i  & $ \vdash (2 + 2) = 4$ & \\
\end{tabular}
\end{table}

Step 1 says that we can assert that $2 = 1 + 1$ because it is
justified by \texttt{df-2}.
What is \texttt{df-2}?
It is simply the definition of $2$, which in our system is defined as being
equal to $1 + 1$.  This shows how we can use definitions in proofs.

Look at Step 2 of the proof. In the Ref column, we see that it references
a previously proved theorem, \texttt{oveq2i}.
It turns out that
theorem \texttt{oveq2i} requires a
hypothesis, and in the Hyp column of Step 2 we indicate that Step 1 will
satisfy (match) this hypothesis.
If we looked at \texttt{oveq2i}
we would find that it proves that given some hypothesis
$A = B$, we can prove that $( C F A ) = ( C F B )$.
If we use \texttt{oveq2i} and apply step 1's result as the hypothesis,
that will mean that $A = 2$ and $B = ( 1 + 1 )$ within this use of
\texttt{oveq2i}.
We are free to select any value of $C$ and $F$ (subject to syntax constraints),
so we are free to select $C = 2$ and $F = +$,
producing our desired result,
$ (2 + 2) = (2 + (1 + 1))$.

Step 2 is an example of substitution.
In the end, every step in every proof uses only this one substitution rule.
All the rules of logic, and all the axioms, are expressed so that
they can be used via this one substitution rule.
So once you master substitution, you can master every Metamath proof,
no exceptions.

Each step is clear and can be immediately checked.
In the {\sc HTML} display you can even click on each reference to see why it is
justified, making it easy to see why the proof works.

\section{Deduction}\label{deduction}

Strictly speaking,
a deduction (also called an inference) is a kind of statement that needs
some hypotheses to be true in order for its conclusion to be true.
A theorem, on the other hand, has no hypotheses.
Informally we often call both of them theorems, but in this section we
will stick to the strict definitions.

It sometimes happens that we have proved a deduction of the form
$\varphi \Rightarrow \psi$\index{$\Rightarrow$}
(given hypothesis $\varphi$ we can prove $\psi$)
and we want to then prove a theorem of the form
$\varphi \rightarrow \psi$.

Converting a deduction (which uses a hypothesis) into a theorem
(which does not) is not as simple as you might think.
The deduction says, ``if we can prove $\varphi$ then we can prove $\psi$,''
which is in some sense weaker than saying
``$\varphi$ implies $\psi$.''
There is no axiom of logic that permits us to directly obtain the theorem
given the deduction.\footnote{
The conversion of a deduction to a theorem does not even hold in general
for quantum propositional calculus,
which is a weak subset of classical propositional calculus.
It has been shown that adding the Standard Deduction Theorem (discussed below)
to quantum propositional calculus turns it into classical
propositional calculus!
}

This is in contrast to going the other way.
If we have the theorem ($\varphi \rightarrow \psi$),
it is easy to recover the deduction
($\varphi \Rightarrow \psi$)
using modus ponens\index{modus ponens}
(\texttt{ax-mp}; see section \ref{axmp}).

In the following subsections we first discuss the standard deduction theorem
(the traditional but awkward way to convert deductions into theorems) and
the weak deduction theorem (a limited version of the standard deduction
theorem that is easier to use and was once widely used in
\texttt{set.mm}\index{set theory database (\texttt{set.mm})}).
In section \ref{deductionstyle} we discuss
deduction style, the newer approach we now recommend in most cases.
Deduction style uses ``deduction form,'' a form that
prefixes each hypothesis (other than definitions) and the
conclusion with a universal antecedent (``$\varphi \rightarrow$'').
Deduction style is widely used in \texttt{set.mm},
so it is useful to understand it and \textit{why} it is widely used.
Section \ref{naturaldeduction}
briefly discusses our approach for using natural deduction
within \texttt{set.mm},
as that approach is deeply related to deduction style.
We conclude with a summary of the strengths of
our approach, which we believe are compelling.

\subsection{The Standard Deduction Theorem}\label{standarddeductiontheorem}

It is possible to make use of information
contained in the deduction or its proof to assist us with the proof of
the related theorem.
In traditional logic books, there is a metatheorem called the
Deduction Theorem\index{Deduction Theorem}\index{Standard Deduction Theorem},
discovered independently by Herbrand and Tarski around 1930.
The Deduction Theorem, which we often call the Standard Deduction Theorem,
provides an algorithm for constructing a proof of a theorem from the
proof of its corresponding deduction. See, for example,
\cite[p.~56]{Margaris}\index{Margaris, Angelo}.
To construct a proof for a theorem, the
algorithm looks at each step in the proof of the original deduction and
rewrites the step with several steps wherein the hypothesis is eliminated
and becomes an antecedent.

In ordinary mathematics, no one actually carries out the algorithm,
because (in its most basic form) it involves an exponential explosion of
the number of proof steps as more hypotheses are eliminated. Instead,
the Standard Deduction Theorem is invoked simply to claim that it can
be done in principle, without actually doing it.
What's more, the algorithm is not as simple as it might first appear
when applying it rigorously.
There is a subtle restriction on the Standard Deduction Theorem
that must be taken into account involving the axiom of generalization
when working with predicate calculus (see the literature for more detail).

One of the goals of Metamath is to let you plainly see, with as few
underlying concepts as possible, how mathematics can be derived directly
from the axioms, and not indirectly according to some hidden rules
buried inside a program or understood only by logicians. If we added
the Standard Deduction Theorem to the language and proof verifier,
that would greatly complicate both and largely defeat Metamath's goal
of simplicity. In principle, we could show direct proofs by expanding
out the proof steps generated by the algorithm of the Standard Deduction
Theorem, but that is not feasible in practice because the number of proof
steps quickly becomes huge, even astronomical.
Since the algorithm of the Standard Deduction Theorem is driven by the proof,
we would have to go through that proof
all over again---starting from axioms---in order to obtain the theorem form.
In terms of proof length, there would be no savings over just
proving the theorem directly instead of first proving the deduction form.

\subsection{Weak Deduction Theorem}\label{weakdeductiontheorem}

We have developed
a more efficient method for proving a theorem from a deduction
that can be used instead of the Standard Deduction Theorem
in many (but not all) cases.
We call this more efficient method the
Weak Deduction Theorem\index{Weak Deduction Theorem}.\footnote{
There is also an unrelated ``Weak Deduction Theorem''
in the field of relevance logic, so to avoid confusion we could call
ours the ``Weak Deduction Theorem for Classical Logic.''}
Unlike the Standard Deduction Theorem, the Weak Deduction Theorem produces the
theorem directly from a special substitution instance of the deduction,
using a small, fixed number of steps roughly proportional to the length
of the final theorem.

If you come to a proof referencing the Weak Deduction Theorem
\texttt{dedth} (or one of its variants \texttt{dedthxx}),
here is how to follow the proof without getting into the details:
just click on the theorem referenced in the step
just before the reference to \texttt{dedth} and ignore everything else.
Theorem \texttt{dedth} simply turns a hypothesis into an antecedent
(i.e. the hypothesis followed by $\rightarrow$
is placed in front of the assertion, and the hypothesis
itself is eliminated) given certain conditions.

The Weak Deduction Theorem
eliminates a hypothesis $\varphi$, making it become an antecedent.
It does this by proving an expression
$ \varphi \rightarrow \psi $ given two hypotheses:
(1)
$ ( A = {\rm if} ( \varphi , A , B ) \rightarrow ( \varphi \leftrightarrow \chi ) ) $
and
(2) $\chi$.
Note that it requires that a proof exists for $\varphi$ when the class variable
$A$ is replaced with a specific class $B$. The hypothesis $\chi$
should be assigned to the inference.
You can see the details of the proof of the Weak Deduction Theorem
in theorem \texttt{dedth}.

The Weak Deduction Theorem
is probably easier to understand by studying proofs that make use of it.
For example, let's look at the proof of \texttt{renegcl}, which proves that
$ \vdash ( A \in \mathbb{R} \rightarrow - A \in \mathbb{R} )$:

\needspace{4\baselineskip}
\begin{longtabu} {l l l X}
\textbf{Step} & \textbf{Hyp} & \textbf{Ref} & \textbf{Expression} \\
  1 &  & negeq &
  $\vdash$ $($ $A$ $=$ ${\rm if}$ $($ $A$ $\in$ $\mathbb{R}$ $,$ $A$ $,$ $1$ $)$ $\rightarrow$
  $\textrm{-}$ $A$ $=$ $\textrm{-}$ ${\rm if}$ $($ $A$ $\in$ $\mathbb{R}$
  $,$ $A$ $,$ $1$ $)$ $)$ \\
 2 & 1 & eleq1d &
    $\vdash$ $($ $A$ $=$ ${\rm if}$ $($ $A$ $\in$ $\mathbb{R}$ $,$ $A$ $,$ $1$ $)$ $\rightarrow$ $($
    $\textrm{-}$ $A$ $\in$ $\mathbb{R}$ $\leftrightarrow$
    $\textrm{-}$ ${\rm if}$ $($ $A$ $\in$ $\mathbb{R}$ $,$ $A$ $,$ $1$ $)$ $\in$
    $\mathbb{R}$ $)$ $)$ \\
 3 &  & 1re & $\vdash 1 \in \mathbb{R}$ \\
 4 & 3 & elimel &
   $\vdash {\rm if} ( A \in \mathbb{R} , A , 1 ) \in \mathbb{R}$ \\
 5 & 4 & renegcli &
   $\vdash \textrm{-} {\rm if} ( A \in \mathbb{R} , A , 1 ) \in \mathbb{R}$ \\
 6 & 2,5 & dedth &
   $\vdash ( A \in \mathbb{R} \rightarrow \textrm{-} A \in \mathbb{R}$ ) \\
\end{longtabu}

The somewhat strange-looking steps in \texttt{renegcl} before step 5 are
technical stuff that makes this magic work, and they can be ignored
for a quick overview of the proof. To continue following the ``important''
part of the proof of \texttt{renegcl},
you can look at the reference to \texttt{renegcli} at step 5.

That said, let's briefly look at how
\texttt{renegcl} uses the
Weak Deduction Theorem (\texttt{dedth}) to do its job,
in case you want to do something similar or want understand it more deeply.
Let's work backwards in the proof of \texttt{renegcl}.
Step 6 applies \texttt{dedth} to produce our goal result
$ \vdash ( A \in \mathbb{R} \rightarrow\, - A \in \mathbb{R} )$.
This requires on the one hand the (substituted) deduction
\texttt{renegcli} in step 5.
By itself \texttt{renegcli} proves the deduction
$ \vdash A \in \mathbb{R} \Rightarrow\, \vdash - A \in \mathbb{R}$;
this is the deduction form we are trying to turn into theorem form,
and thus
\texttt{renegcli} has a separate hypothesis that must be fulfilled.
To fulfill the hypothesis of the invocation of
\texttt{renegcli} in step 5, it is eventually
reduced to the already proven theorem $1 \in \mathbb{R}$ in step 3.
Step 4 connects steps 3 and 5; step 4 invokes
\texttt{elimel}, a special case of \texttt{elimhyp} that eliminates
a membership hypothesis for the weak deduction theorem.
On the other hand, the equivalence of the conclusion of
\texttt{renegcl}
$( - A \in \mathbb{R} )$ and the substituted conclusion of
\texttt{renegcli} must be proven, which is done in steps 2 and 1.

The weak deduction theorem has limitations.
In particular, we must be able to prove a special case of the deduction's
hypothesis as a stand-alone theorem.
For example, we used $1 \in \mathbb{R}$ in step 3 of \texttt{renegcl}.

We used to use the weak deduction theorem
extensively within \texttt{set.mm}.
However, we now recommend applying ``deduction style''
instead in most cases, as deduction style is
often an easier and clearer approach.
Therefore, we will now describe deduction style.

\subsection{Deduction Style}\label{deductionstyle}

We now prefer to write assertions in ``deduction form''
instead of writing a proof that would require use of the standard or
weak deduction theorem.
We call this appraoch
``deduction style.''\index{deduction style}

It will be easier to explain this by first defining some terms:

\begin{itemize}
\item \textbf{closed form}\index{closed form}\index{forms!closed}:
A kind of assertion (theorem) with no hypotheses.
Typically its label has no special suffix.
An example is \texttt{unss}, which states:
$\vdash ( ( A \subseteq C \wedge B \subseteq C ) \leftrightarrow ( A \cup B )
\subseteq C )\label{eq:unss}$
\item \textbf{deduction form}\index{deduction form}\index{forms!deduction}:
A kind of assertion with one or more hypotheses
where the conclusion is an implication with
a wff variable as the antecedent (usually $\varphi$), and every hypothesis
(\$e statement)
is either (1) an implication with the same antecedent as the conclusion or
(2) a definition.
A definition
can be for a class variable (this is a class variable followed by ``='')
or a wff variable (this is a wff variable followed by $\leftrightarrow$);
class variable definitions are more common.
In practice, a proof
in deduction form will also contain many steps that are implications
where the antecedent is either that wff variable (normally $\varphi$)
or is
a conjunction (...$\land$...) including that wff variable ($\varphi$).
If an assertion is in deduction form, and other forms are also available,
then we suffix its label with ``d.''
An example is \texttt{unssd}, which states\footnote{
For brevity we show here (and in other places)
a $\&$\index{$\&$} between hypotheses\index{hypotheses}
and a $\Rightarrow$\index{$\Rightarrow$}\index{conclusion}
between the hypotheses and the conclusion.
This notation is technically not part of the Metamath language, but is
instead a convenient abbreviation to show both the hypotheses and conclusion.}:
$\vdash ( \varphi \rightarrow A \subseteq C )\quad\&\quad \vdash ( \varphi
    \rightarrow B \subseteq C )\quad\Rightarrow\quad \vdash ( \varphi
    \rightarrow ( A \cup B ) \subseteq C )\label{eq:unssd}$
\item \textbf{inference form}\index{inference form}\index{forms!inference}:
A kind of assertion with one or more hypotheses that is not in deduction form
(e.g., there is no common antecedent).
If an assertion is in inference form, and other forms are also available,
then we suffix its label with ``i.''
An example is \texttt{unssi}, which states:
$\vdash A \subseteq C\quad\&\quad \vdash B \subseteq C\quad\Rightarrow\quad
    \vdash ( A \cup B ) \subseteq C\label{eq:unssi}$
\end{itemize}

When using deduction style we express an assertion in deduction form.
This form prefixes each hypothesis (other than definitions) and the
conclusion with a universal antecedent (``$\varphi \rightarrow$'').
The antecedent (e.g., $\varphi$)
mimics the context handled in the deduction theorem, eliminating
the need to directly use the deduction theorem.

Once you have an assertion in deduction form, you can easily convert it
to inference form or closed form:

\begin{itemize}
\item To
prove some assertion Ti in inference form, given assertion Td in deduction
form, there is a simple mechanical process you can use. First take each
Ti hypothesis and insert a \texttt{T.} $\rightarrow$ prefix (``true implies'')
using \texttt{a1i}. You
can then use the existing assertion Td to prove the resulting conclusion
with a \texttt{T.} $\rightarrow$ prefix.
Finally, you can remove that prefix using \texttt{trud},
resulting in the conclusion you wanted to prove.
\item To
prove some assertion T in closed form, given assertion Td in deduction
form, there is another simple mechanical process you can use. First,
select an expression that is the conjunction (...$\land$...) of all of the
consequents of every hypothesis of Td. Next, prove that this expression
implies each of the separate hypotheses of Td in turn by eliminating
conjuncts (there are a variety of proven assertions to do this, including
\texttt{simpl},
\texttt{simpr},
\texttt{3simpa},
\texttt{3simpb},
\texttt{3simpc},
\texttt{simp1},
\texttt{simp2},
and
\texttt{simp3}).
If the
expression has nested conjunctions, inner conjuncts can be broken out by
chaining the above theorems with \texttt{syl}
(see section \ref{syl}).\footnote{
There are actually many theorems
(labeled simp* such as \texttt{simp333}) that break out inner conjuncts in one
step, but rather than learning them you can just use the chaining we
just described to prove them, and then let the Metamath program command
\texttt{minimize{\char`\_}with}\index{\texttt{minimize{\char`\_}with} command}
figure out the right ones needed to collapse them.}
As your final step, you can then apply the already-proven assertion Td
(which is in deduction form), proving assertion T in closed form.
\end{itemize}

We can also easily convert any assertion T in closed form to its related
assertion Ti in inference form by applying
modus ponens\index{modus ponens} (see section \ref{axmp}).

The deduction form antecedent can also be used to represent the context
necessary to support natural deduction systems, so we will now
discuss natural deduction.

\subsection{Natural Deduction}\label{naturaldeduction}

Natural deduction\index{natural deduction}
(ND) systems, as such, were originally introduced in
1934 by two logicians working independently: Ja\'skowski and Gentzen. ND
systems are supposed to reconstruct, in a formally proper way, traditional
ways of mathematical reasoning (such as conditional proof, indirect proof,
and proof by cases). As reconstructions they were naturally influenced
by previous work, and many specific ND systems and notations have been
developed since their original work.

There are many ND variants, but
Indrzejczak \cite[p.~31-32]{Indrzejczak}\index{Indrzejczak, Andrzej}
suggests that any natural deductive system must satisfy at
least these three criteria:

\begin{itemize}
\item ``There are some means for entering assumptions into a proof and
also for eliminating them. Usually it requires some bookkeeping devices
for indicating the scope of an assumption, and showing that a part of
a proof depending on eliminated assumption is discharged.
\item There are no (or, at least, a very limited set of) axioms, because
their role is taken over by the set of primitive rules for introduction
and elimination of logical constants which means that elementary
inferences instead of formulae are taken as primitive.
\item (A genuine) ND system admits a lot of freedom in proof construction
and possibility of applying several proof search strategies, like
conditional proof, proof by cases, proof by reductio ad absurdum etc.''
\end{itemize}

The Metamath Proof Explorer (MPE) as defined in \texttt{set.mm}
is fundamentally a Hilbert-style system.
That is, MPE is based on a larger number of axioms (compared
to natural deduction systems), a very small set of rules of inference
(modus ponens), and the context is not changed by the rules of inference
in the middle of a proof. That said, MPE proofs can be developed using
the natural deduction (ND) approach as originally developed by Ja\'skowski
and Gentzen.

The most common and recommended approach for applying ND in MPE is to use
deduction form\index{deduction form}%
\index{forms!deduction}
and apply the MPE proven assertions that are equivalent to ND rules.
For example, MPE's \texttt{jca} is equivalent to ND rule $\land$-I
(and-insertion).
We maintain a list of equivalences that you may consult.
This approach for applying an ND approach within MPE relies on Metamath's
wff metavariables in an essential way, and is described in more detail
in the presentation ``Natural Deductions in the Metamath Proof Language''
by Mario Carneiro \cite{CarneiroND}\index{Carneiro, Mario}.

In this style many steps are an implication, whose antecedent mimics
the context ($\Gamma$) of most ND systems. To add an assumption, simply add
it to the implication antecedent (typically using
\texttt{simpr}),
and use that
new antecedent for all later claims in the same scope. If you wish to
use an assertion in an ND hypothesis scope that is outside the current
ND hypothesis scope, modify the assertion so that the ND hypothesis
assumption is added to its antecedent (typically using \texttt{adantr}). Most
proof steps will be proved using rules that have hypotheses and results
of the form $\varphi \rightarrow$ ...

An example may make this clearer.
Let's show theorem 5.5 of
\cite[p.~18]{Clemente}\index{Clemente Laboreo, Daniel}
along with a line by line translation using the usual
translation of natural deduction (ND) in the Metamath Proof Explorer
(MPE) notation (this is proof \texttt{ex-natded5.5}).
The proof's original goal was to prove
$\lnot \psi$ given two hypotheses,
$( \psi \rightarrow \chi )$ and $ \lnot \chi$.
We will translate these statements into MPE deduction form
by prefixing them all with $\varphi \rightarrow$.
As a result, in MPE the goal is stated as
$( \varphi \rightarrow \lnot \psi )$, and the two hypotheses are stated as
$( \varphi \rightarrow ( \psi \rightarrow \chi ) )$ and
$( \varphi \rightarrow \lnot \chi )$.

The following table shows the proof in Fitch natural deduction style
and its MPE equivalent.
The \textit{\#} column shows the original numbering,
\textit{MPE\#} shows the number in the equivalent MPE proof
(which we will show later),
\textit{ND Expression} shows the original proof claim in ND notation,
and \textit{MPE Translation} shows its translation into MPE
as discussed in this section.
The final columns show the rationale in ND and MPE respectively.

\needspace{4\baselineskip}
{\setlength{\extrarowsep}{4pt} % Keep rows from being too close together
\begin{longtabu}   { @{} c c X X X X }
\textbf{\#} & \textbf{MPE\#} & \textbf{ND Ex\-pres\-sion} &
\textbf{MPE Trans\-lation} & \textbf{ND Ration\-ale} &
\textbf{MPE Ra\-tio\-nale} \\
\endhead

1 & 2;3 &
$( \psi \rightarrow \chi )$ &
$( \varphi \rightarrow ( \psi \rightarrow \chi ) )$ &
Given &
\$e; \texttt{adantr} to put in ND hypothesis \\

2 & 5 &
$ \lnot \chi$ &
$( \varphi \rightarrow \lnot \chi )$ &
Given &
\$e; \texttt{adantr} to put in ND hypothesis \\

3 & 1 &
... $\vert$ $\psi$ &
$( \varphi \rightarrow \psi )$ &
ND hypothesis assumption &
\texttt{simpr} \\

4 & 4 &
... $\chi$ &
$( ( \varphi \land \psi ) \rightarrow \chi )$ &
$\rightarrow$\,E 1,3 &
\texttt{mpd} 1,3 \\

5 & 6 &
... $\lnot \chi$ &
$( ( \varphi \land \psi ) \rightarrow \lnot \chi )$ &
IT 2 &
\texttt{adantr} 5 \\

6 & 7 &
$\lnot \psi$ &
$( \varphi \rightarrow \lnot \psi )$ &
$\land$\,I 3,4,5 &
\texttt{pm2.65da} 4,6 \\

\end{longtabu}
}


The original used Latin letters; we have replaced them with Greek letters
to follow Metamath naming conventions and so that it is easier to follow
the Metamath translation. The Metamath line-for-line translation of
this natural deduction approach precedes every line with an antecedent
including $\varphi$ and uses the Metamath equivalents of the natural deduction
rules. To add an assumption, the antecedent is modified to include it
(typically by using \texttt{adantr};
\texttt{simpr} is useful when you want to
depend directly on the new assumption, as is shown here).

In Metamath we can represent the two given statements as these hypotheses:

\needspace{2\baselineskip}
\begin{itemize}
\item ex-natded5.5.1 $\vdash ( \varphi \rightarrow ( \psi \rightarrow \chi ) )$
\item ex-natded5.5.2 $\vdash ( \varphi \rightarrow \lnot \chi )$
\end{itemize}

\needspace{4\baselineskip}
Here is the proof in Metamath as a line-by-line translation:

\begin{longtabu}   { l l l X }
\textbf{Step} & \textbf{Hyp} & \textbf{Ref} & \textbf{Ex\-pres\-sion} \\
\endhead
1 & & simpr & $\vdash ( ( \varphi \land \psi ) \rightarrow \psi )$ \\
2 & & ex-natded5.5.1 &
  $\vdash ( \varphi \rightarrow ( \psi \rightarrow \chi ) )$ \\
3 & 2 & adantr &
 $\vdash ( ( \varphi \land \psi ) \rightarrow ( \psi \rightarrow \chi ) )$ \\
4 & 1, 3 & mpd &
 $\vdash ( ( \varphi \land \psi ) \rightarrow \chi ) $ \\
5 & & ex-natded5.5.2 &
 $\vdash ( \varphi \rightarrow \lnot \chi )$ \\
6 & 5 & adantr &
 $\vdash ( ( \varphi \land \psi ) \rightarrow \lnot \chi )$ \\
7 & 4, 6 & pm2.65da &
 $\vdash ( \varphi \rightarrow \lnot \psi )$ \\
\end{longtabu}

Only using specific natural deduction rules directly can lead to very
long proofs, for exactly the same reason that only using axioms directly
in Hilbert-style proofs can lead to very long proofs.
If the goal is short and clear proofs,
then it is better to reuse already-proven assertions
in deduction form than to start from scratch each time
and using only basic natural deduction rules.

\subsection{Strengths of Our Approach}

As far as we know there is nothing else in the literature like either the
weak deduction theorem or Mario Carneiro\index{Carneiro, Mario}'s
natural deduction method.
In order to
transform a hypothesis into an antecedent, the literature's standard
``Deduction Theorem''\index{Deduction Theorem}\index{Standard Deduction Theorem}
requires metalogic outside of the notions provided
by the axiom system. We instead generally prefer to use Mario Carneiro's
natural deduction method, then use the weak deduction theorem in cases
where that is difficult to apply, and only then use the full standard
deduction theorem as a last resort.

The weak deduction theorem\index{Weak Deduction Theorem}
does not require any additional metalogic
but converts an inference directly into a closed form theorem, with
a rigorous proof that uses only the axiom system. Unlike the standard
Deduction Theorem, there is no implicit external justification that we
have to trust in order to use it.

Mario Carneiro's natural deduction\index{natural deduction}
method also does not require any new metalogical
notions. It avoids the Deduction Theorem's metalogic by prefixing the
hypotheses and conclusion of every would-be inference with a universal
antecedent (``$\varphi \rightarrow$'') from the very start.

We think it is impressive and satisfying that we can do so much in a
practical sense without stepping outside of our Hilbert-style axiom system.
Of course our axiomatization, which is in the form of schemes,
contains a metalogic of its own that we exploit. But this metalogic
is relatively simple, and for our Deduction Theorem alternatives,
we primarily use just the direct substitution of expressions for
metavariables.

\begin{sloppy}
\section{Exploring the Set The\-o\-ry Data\-base}\label{exploring}
\end{sloppy}
% NOTE: All examples performed in this section are
% recorded wtih "set width 61" % on set.mm as of 2019-05-28
% commit c1e7849557661260f77cfdf0f97ac4354fbb4f4d.

At this point you may wish to study the \texttt{set.mm}\index{set theory
database (\texttt{set.mm})} file in more detail.  Pay particular
attention to the assumptions needed to define wffs\index{well-formed
formula (wff)} (which are not included above), the variable types
(\texttt{\$f}\index{\texttt{\$f} statement} statements), and the
definitions that are introduced.  Start with some simple theorems in
propositional calculus, making sure you understand in detail each step
of a proof.  Once you get past the first few proofs and become familiar
with the Metamath language, any part of the \texttt{set.mm} database
will be as easy to follow, step by step, as any other part---you won't
have to undergo a ``quantum leap'' in mathematical sophistication to be
able to follow a deep proof in set theory.

Next, you may want to explore how concepts such as natural numbers are
defined and described.  This is probably best done in conjunction with
standard set theory textbooks, which can help give you a higher-level
understanding.  The \texttt{set.mm} database provides references that will get
you started.  From there, you will be on your way towards a very deep,
rigorous understanding of abstract mathematics.

The Metamath\index{Metamath} program can help you peruse a Metamath data\-base,
wheth\-er you are trying to figure out how a certain step follows in a proof or
just have a general curiosity.  We will go through some examples of the
commands, using the \texttt{set.mm}\index{set theory database (\texttt{set.mm})}
database provided with the Metamath software.  These should help get you
started.  See Chapter~\ref{commands} for a more detailed description of
the commands.  Note that we have included the full spelling of all commands to
prevent ambiguity with future commands.  In practice you may type just the
characters needed to specify each command keyword\index{command keyword}
unambiguously, often just one or two characters per keyword, and you don't
need to type them in upper case.

First run the Metamath program as described earlier.  You should see the
\verb/MM>/ prompt.  Read in the \texttt{set.mm} file:\index{\texttt{read}
command}

\begin{verbatim}
MM> read set.mm
Reading source file "set.mm"... 34554442 bytes
34554442 bytes were read into the source buffer.
The source has 155711 statements; 2254 are $a and 32250 are $p.
No errors were found.  However, proofs were not checked.
Type VERIFY PROOF * if you want to check them.
\end{verbatim}

As with most examples in this book, what you will see
will be slightly different because we are continuously
improving our databases (including \texttt{set.mm}).

Let's check the database integrity.  This may take a minute or two to run if
your computer is slow.

\begin{verbatim}
MM> verify proof *
0 10%  20%  30%  40%  50%  60%  70%  80%  90% 100%
..................................................
All proofs in the database were verified in 2.84 s.
\end{verbatim}

No errors were reported, so every proof is correct.

You need to know the names (labels) of theorems before you can look at them.
Often just examining the database file(s) with a text editor is the best
approach.  In \texttt{set.mm} there are many detailed comments, especially near
the beginning, that can help guide you. The \texttt{search} command in the
Metamath program is also handy.  The \texttt{comments} qualifier will list the
statements whose associated comment (the one immediately before it) contain a
string you give it.  For example, if you are studying Enderton's {\em Elements
of Set Theory} \cite{Enderton}\index{Enderton, Herbert B.} you may want to see
the references to it in the database.  The search string \texttt{enderton} is not
case sensitive.  (This will not show you all the database theorems that are in
Enderton's book because there is usually only one citation for a given
theorem, which may appear in several textbooks.)\index{\texttt{search}
command}

\begin{verbatim}
MM> search * "enderton" / comments
12067 unineq $p "... Exercise 20 of [Enderton] p. 32 and ..."
12459 undif2 $p "...Corollary 6K of [Enderton] p. 144. (C..."
12953 df-tp $a "...s. Definition of [Enderton] p. 19. (Co..."
13689 unissb $p ".... Exercise 5 of [Enderton] p. 26 and ..."
\end{verbatim}
\begin{center}
(etc.)
\end{center}

Or you may want to see what theorems have something to do with
conjunction (logical {\sc and}).  The quotes around the search
string are optional when there's no ambiguity.\index{\texttt{search}
command}

\begin{verbatim}
MM> search * conjunction / comments
120 a1d $p "...be replaced with a conjunction ( ~ df-an )..."
662 df-bi $a "...viated form after conjunction is introdu..."
1319 wa $a "...ff definition to include conjunction ('and')."
1321 df-an $a "Define conjunction (logical 'and'). Defini..."
1420 imnan $p "...tion in terms of conjunction. (Contribu..."
\end{verbatim}
\begin{center}
(etc.)
\end{center}


Now we will start to look at some details.  Let's look at the first
axiom of propositional calculus
(we could use \texttt{sh st} to abbreviate
\texttt{show statement}).\index{\texttt{show statement} command}

\begin{verbatim}
MM> show statement ax-1/full
Statement 19 is located on line 881 of the file "set.mm".
"Axiom _Simp_.  Axiom A1 of [Margaris] p. 49.  One of the 3
axioms of propositional calculus.  The 3 axioms are also
        ...
19 ax-1 $a |- ( ph -> ( ps -> ph ) ) $.
Its mandatory hypotheses in RPN order are:
  wph $f wff ph $.
  wps $f wff ps $.
The statement and its hypotheses require the variables:  ph
      ps
The variables it contains are:  ph ps


Statement 49 is located on line 11182 of the file "set.mm".
Its statement number for HTML pages is 6.
"Axiom _Simp_.  Axiom A1 of [Margaris] p. 49.  One of the 3
axioms of propositional calculus.  The 3 axioms are also
given as Definition 2.1 of [Hamilton] p. 28.
...
49 ax-1 $a |- ( ph -> ( ps -> ph ) ) $.
Its mandatory hypotheses in RPN order are:
  wph $f wff ph $.
  wps $f wff ps $.
The statement and its hypotheses require the variables:
  ph ps
The variables it contains are:  ph ps
\end{verbatim}

Compare this to \texttt{ax-1} on p.~\pageref{ax1}.  You can see that the
symbol \texttt{ph} is the {\sc ascii} notation for $\varphi$, etc.  To
see the mathematical symbols for any expression you may typeset it in
\LaTeX\ (type \texttt{help tex} for instructions)\index{latex@{\LaTeX}}
or, easier, just use a text editor to look at the comments where symbols
are first introduced in \texttt{set.mm}.  The hypotheses \texttt{wph}
and \texttt{wps} required by \texttt{ax-1} mean that variables
\texttt{ph} and \texttt{ps} must be wffs.

Next we'll pick a simple theorem of propositional calculus, the Principle of
Identity, which is proved directly from the axioms.  We'll look at the
statement then its proof.\index{\texttt{show statement}
command}

\begin{verbatim}
MM> show statement id1/full
Statement 116 is located on line 11371 of the file "set.mm".
Its statement number for HTML pages is 22.
"Principle of identity.  Theorem *2.08 of [WhiteheadRussell]
p. 101.  This version is proved directly from the axioms for
demonstration purposes.
...
116 id1 $p |- ( ph -> ph ) $= ... $.
Its mandatory hypotheses in RPN order are:
  wph $f wff ph $.
Its optional hypotheses are:  wps wch wth wta wet
      wze wsi wrh wmu wla wka
The statement and its hypotheses require the variables:  ph
These additional variables are allowed in its proof:
      ps ch th ta et ze si rh mu la ka
The variables it contains are:  ph
\end{verbatim}

The optional variables\index{optional variable} \texttt{ps}, \texttt{ch}, etc.\ are
available for use in a proof of this statement if we wish, and were we to do
so we would make use of optional hypotheses \texttt{wps}, \texttt{wch}, etc.  (See
Section~\ref{dollaref} for the meaning of ``optional
hypothesis.''\index{optional hypothesis}) The reason these show up in the
statement display is that statement \texttt{id1} happens to be in their scope
(see Section~\ref{scoping} for the definition of ``scope''\index{scope}), but
in fact in propositional calculus we will never make use of optional
hypotheses or variables.  This becomes important after quantifiers are
introduced, where ``dummy'' variables\index{dummy variable} are often needed
in the middle of a proof.

Let's look at the proof of statement \texttt{id1}.  We'll use the
\texttt{show proof} command, which by default suppresses the
``non-essential'' steps that construct the wffs.\index{\texttt{show proof}
command}
We will display the proof in ``lemmon' format (a non-indented format
with explicit previous step number references) and renumber the
displayed steps:

\begin{verbatim}
MM> show proof id1 /lemmon/renumber
1 ax-1           $a |- ( ph -> ( ph -> ph ) )
2 ax-1           $a |- ( ph -> ( ( ph -> ph ) -> ph ) )
3 ax-2           $a |- ( ( ph -> ( ( ph -> ph ) -> ph ) ) ->
                     ( ( ph -> ( ph -> ph ) ) -> ( ph -> ph )
                                                          ) )
4 2,3 ax-mp      $a |- ( ( ph -> ( ph -> ph ) ) -> ( ph -> ph
                                                          ) )
5 1,4 ax-mp      $a |- ( ph -> ph )
\end{verbatim}

If you have read Section~\ref{trialrun}, you'll know how to interpret this
proof.  Step~2, for example, is an application of axiom \texttt{ax-1}.  This
proof is identical to the one in Hamilton's {\em Logic for Mathematicians}
\cite[p.~32]{Hamilton}\index{Hamilton, Alan G.}.

You may want to look at what
substitutions\index{substitution!variable}\index{variable substitution} are
made into \texttt{ax-1} to arrive at step~2.  The command to do this needs to
know the ``real'' step number, so we'll display the proof again without
the \texttt{renumber} qualifier.\index{\texttt{show proof}
command}

\begin{verbatim}
MM> show proof id1 /lemmon
 9 ax-1          $a |- ( ph -> ( ph -> ph ) )
20 ax-1          $a |- ( ph -> ( ( ph -> ph ) -> ph ) )
24 ax-2          $a |- ( ( ph -> ( ( ph -> ph ) -> ph ) ) ->
                     ( ( ph -> ( ph -> ph ) ) -> ( ph -> ph )
                                                          ) )
25 20,24 ax-mp   $a |- ( ( ph -> ( ph -> ph ) ) -> ( ph -> ph
                                                          ) )
26 9,25 ax-mp    $a |- ( ph -> ph )
\end{verbatim}

The ``real'' step number is 20.  Let's look at its details.

\begin{verbatim}
MM> show proof id1 /detailed_step 20
Proof step 20:  min=ax-1 $a |- ( ph -> ( ( ph -> ph ) -> ph )
  )
This step assigns source "ax-1" ($a) to target "min" ($e).
The source assertion requires the hypotheses "wph" ($f, step
18) and "wps" ($f, step 19).  The parent assertion of the
target hypothesis is "ax-mp" ($a, step 25).
The source assertion before substitution was:
    ax-1 $a |- ( ph -> ( ps -> ph ) )
The following substitutions were made to the source
assertion:
    Variable  Substituted with
     ph        ph
     ps        ( ph -> ph )
The target hypothesis before substitution was:
    min $e |- ph
The following substitution was made to the target hypothesis:
    Variable  Substituted with
     ph        ( ph -> ( ( ph -> ph ) -> ph ) )
\end{verbatim}

This shows the substitutions\index{substitution!variable}\index{variable
substitution} made to the variables in \texttt{ax-1}.  References are made to
steps 18 and 19 which are not shown in our proof display.  To see these steps,
you can display the proof with the \texttt{all} qualifier.

Let's look at a slightly more advanced proof of propositional calculus.  Note
that \verb+/\+ is the symbol for $\wedge$ (logical {\sc and}, also
called conjunction).\index{conjunction ($\wedge$)}
\index{logical {\sc and} ($\wedge$)}

\begin{verbatim}
MM> show statement prth/full
Statement 1791 is located on line 15503 of the file "set.mm".
Its statement number for HTML pages is 559.
"Conjoin antecedents and consequents of two premises.  This
is the closed theorem form of ~ anim12d .  Theorem *3.47 of
[WhiteheadRussell] p. 113.  It was proved by Leibniz,
and it evidently pleased him enough to call it
_praeclarum theorema_ (splendid theorem).
...
1791 prth $p |- ( ( ( ph -> ps ) /\ ( ch -> th ) ) -> ( ( ph
      /\ ch ) -> ( ps /\ th ) ) ) $= ... $.
Its mandatory hypotheses in RPN order are:
  wph $f wff ph $.
  wps $f wff ps $.
  wch $f wff ch $.
  wth $f wff th $.
Its optional hypotheses are:  wta wet wze wsi wrh wmu wla wka
The statement and its hypotheses require the variables:  ph
      ps ch th
These additional variables are allowed in its proof:  ta et
      ze si rh mu la ka
The variables it contains are:  ph ps ch th


MM> show proof prth /lemmon/renumber
1 simpl          $p |- ( ( ( ph -> ps ) /\ ( ch -> th ) ) ->
                                               ( ph -> ps ) )
2 simpr          $p |- ( ( ( ph -> ps ) /\ ( ch -> th ) ) ->
                                               ( ch -> th ) )
3 1,2 anim12d    $p |- ( ( ( ph -> ps ) /\ ( ch -> th ) ) ->
                           ( ( ph /\ ch ) -> ( ps /\ th ) ) )
\end{verbatim}

There are references to a lot of unfamiliar statements.  To see what they are,
you may type the following:

\begin{verbatim}
MM> show proof prth /statement_summary
Summary of statements used in the proof of "prth":

Statement simpl is located on line 14748 of the file
"set.mm".
"Elimination of a conjunct.  Theorem *3.26 (Simp) of
[WhiteheadRussell] p. 112. ..."
  simpl $p |- ( ( ph /\ ps ) -> ph ) $= ... $.

Statement simpr is located on line 14777 of the file
"set.mm".
"Elimination of a conjunct.  Theorem *3.27 (Simp) of
[WhiteheadRussell] ..."
  simpr $p |- ( ( ph /\ ps ) -> ps ) $= ... $.

Statement anim12d is located on line 15445 of the file
"set.mm".
"Conjoin antecedents and consequents in a deduction.
..."
  anim12d.1 $e |- ( ph -> ( ps -> ch ) ) $.
  anim12d.2 $e |- ( ph -> ( th -> ta ) ) $.
  anim12d $p |- ( ph -> ( ( ps /\ th ) -> ( ch /\ ta ) ) )
      $= ... $.
\end{verbatim}
\begin{center}
(etc.)
\end{center}

Of course you can look at each of these statements and their proofs, and
so on, back to the axioms of propositional calculus if you wish.

The \texttt{search} command is useful for finding statements when you
know all or part of their contents.  The following example finds all
statements containing \verb@ph -> ps@ followed by \verb@ch -> th@.  The
\verb@$*@ is a wildcard that matches anything; the \texttt{\$} before the
\verb$*$ prevents conflicts with math symbol token names.  The \verb@*@ after
\texttt{SEARCH} is also a wildcard that in this case means ``match any label.''
\index{\texttt{search} command}

% I'm omitting this one, since readers are unlikely to see it:
% 1096 bisymOLD $p |- ( ( ( ph -> ps ) -> ( ch -> th ) ) -> ( (
%   ( ps -> ph ) -> ( th -> ch ) ) -> ( ( ph <-> ps ) -> ( ch
%    <-> th ) ) ) )
\begin{verbatim}
MM> search * "ph -> ps $* ch -> th"
1791 prth $p |- ( ( ( ph -> ps ) /\ ( ch -> th ) ) -> ( ( ph
    /\ ch ) -> ( ps /\ th ) ) )
2455 pm3.48 $p |- ( ( ( ph -> ps ) /\ ( ch -> th ) ) -> ( (
    ph \/ ch ) -> ( ps \/ th ) ) )
117859 pm11.71 $p |- ( ( E. x ph /\ E. y ch ) -> ( ( A. x (
    ph -> ps ) /\ A. y ( ch -> th ) ) <-> A. x A. y ( ( ph /\
    ch ) -> ( ps /\ th ) ) ) )
\end{verbatim}

Three statements, \texttt{prth}, \texttt{pm3.48},
 and \texttt{pm11.71}, were found to match.

To see what axioms\index{axiom} and definitions\index{definition}
\texttt{prth} ultimately depends on for its proof, you can have the
program backtrack through the hierarchy\index{hierarchy} of theorems and
definitions.\index{\texttt{show trace{\char`\_}back} command}

\begin{verbatim}
MM> show trace_back prth /essential/axioms
Statement "prth" assumes the following axioms ($a
statements):
  ax-1 ax-2 ax-3 ax-mp df-bi df-an
\end{verbatim}

Note that the 3 axioms of propositional calculus and the modus ponens rule are
needed (as expected); in addition, there are a couple of definitions that are used
along the way.  Note that Metamath makes no distinction\index{axiom vs.\
definition} between axioms\index{axiom} and definitions\index{definition}.  In
\texttt{set.mm} they have been distinguished artificially by prefixing their
labels\index{labels in \texttt{set.mm}} with \texttt{ax-} and \texttt{df-}
respectively.  For example, \texttt{df-an} defines conjunction (logical {\sc
and}), which is represented by the symbol \verb+/\+.
Section~\ref{definitions} discusses the philosophy of definitions, and the
Metamath language takes a particularly simple, conservative approach by using
the \texttt{\$a}\index{\texttt{\$a} statement} statement for both axioms and
definitions.

You can also have the program compute how many steps a proof
has\index{proof length} if we were to follow it all the way back to
\texttt{\$a} statements.

\begin{verbatim}
MM> show trace_back prth /essential/count_steps
The statement's actual proof has 3 steps.  Backtracking, a
total of 79 different subtheorems are used.  The statement
and subtheorems have a total of 274 actual steps.  If
subtheorems used only once were eliminated, there would be a
total of 38 subtheorems, and the statement and subtheorems
would have a total of 185 steps.  The proof would have 28349
steps if fully expanded back to axiom references.  The
maximum path length is 38.  A longest path is:  prth <-
anim12d <- syl2and <- sylan2d <- ancomsd <- ancom <- pm3.22
<- pm3.21 <- pm3.2 <- ex <- sylbir <- biimpri <- bicomi <-
bicom1 <- bi2 <- dfbi1 <- impbii <- bi3 <- simprim <- impi <-
con1i <- nsyl2 <- mt3d <- con1d <- notnot1 <- con2i <- nsyl3
<- mt2d <- con2d <- notnot2 <- pm2.18d <- pm2.18 <- pm2.21 <-
pm2.21d <- a1d <- syl <- mpd <- a2i <- a2i.1 .
\end{verbatim}

This tells us that we would have to inspect 274 steps if we want to
verify the proof completely starting from the axioms.  A few more
statistics are also shown.  There are one or more paths back to axioms
that are the longest; this command ferrets out one of them and shows it
to you.  There may be a sense in which the longest path length is
related to how ``deep'' the theorem is.

We might also be curious about what proofs depend on the theorem
\texttt{prth}.  If it is never used later on, we could eliminate it as
redundant if it has no intrinsic interest by itself.\index{\texttt{show
usage} command}

% I decided to show the OLD values here.
\begin{verbatim}
MM> show usage prth
Statement "prth" is directly referenced in the proofs of 18
statements:
  mo3 moOLD 2mo 2moOLD euind reuind reuss2 reusv3i opelopabt
  wemaplem2 rexanre rlimcn2 o1of2 o1rlimmul 2sqlem6 spanuni
  heicant pm11.71
\end{verbatim}

Thus \texttt{prth} is directly used by 18 proofs.
We can use the \texttt{/recursive} qualifier to include indirect use:

\begin{verbatim}
MM> show usage prth /recursive
Statement "prth" directly or indirectly affects the proofs of
24214 statements:
  mo3 mo mo3OLD eu2 moOLD eu2OLD eu3OLD mo4f mo4 eu4 mopick
...
\end{verbatim}

\subsection{A Note on the ``Compact'' Proof Format}

The present version of Metamath\index{Metamath!limitations of
version 0.07.30} will display proofs in a ``compact''\index{compact proof}
format whenever the proof is stored in compressed format in the database.  It
may be be slightly confusing unless you know how to interpret it.  (A
future version may eliminate this format from displays.)  For example,
if you display the complete proof of theorem \texttt{id1} it will start
off as follows:

\begin{verbatim}
MM> show proof id1 /lemmon/all
 1 wph           $f wff ph
 2 wph           $f wff ph
 3 wph           $f wff ph
 4 2,3 wi    @4: $a wff ( ph -> ph )
 5 1,4 wi    @5: $a wff ( ph -> ( ph -> ph ) )
 6 @4            $a wff ( ph -> ph )
\end{verbatim}

\begin{center}
{etc.}
\end{center}

Step 4 has a ``local label,''\index{local label} \texttt{@4}, assigned to it.
Later on, at step 6, the label \texttt{@1} is referenced instead of
displaying the explicit proof for that step.  This technique takes advantage
of the fact that steps in a proof often repeat, especially during the
construction of wffs.  The compact format reduces the number of steps in the
proof display and may actually be preferred by some people, which is one
reason that it is still displayed by the program.

If you want to see the normal format with the ``true'' step numbers, you can
use the following workaround:\index{\texttt{save proof} command}

\begin{verbatim}
MM> save proof id1 /normal
The proof of "id1" has been reformatted and saved internally.
Remember to use WRITE SOURCE to save it permanently.
MM> show proof id1 /lemmon/all
 1 wph           $f wff ph
 2 wph           $f wff ph
 3 wph           $f wff ph
 4 2,3 wi        $a wff ( ph -> ph )
 5 1,4 wi        $a wff ( ph -> ( ph -> ph ) )
 6 wph           $f wff ph
 7 wph           $f wff ph
 8 6,7 wi        $a wff ( ph -> ph )
\end{verbatim}

\begin{center}
{etc.}
\end{center}

Note that the original 6 steps are now 8 steps.  However, the format is
now the same as that described in Chapter~\ref{using}.

\chapter{The Metamath Language}
\label{languagespec}

\begin{quote}
  {\em Thus mathematics may be defined as the subject in which we never know
what we are talking about, nor whether what we are saying is true.}
    \flushright\sc  Bertrand Russell\footnote{\cite[p.~84]{Russell2}.}\\
\end{quote}\index{Russell, Bertrand}

Probably the most striking feature of the Metamath language is its almost
complete absence of hard-wired syntax. Metamath\index{Metamath} does not
understand any mathematics or logic other than that needed to construct finite
sequences of symbols according to a small set of simple, built-in rules.  The
only rule it uses in a proof is the substitution of an expression (symbol
sequence) for a variable, subject to a simple constraint to prevent
bound-variable clashes.  The primitive notions built into Metamath involve the
simple manipulation of finite objects (symbols) that we as humans can easily
visualize and that computers can easily deal with.  They seem to be just
about the simplest notions possible that are required to do standard
mathematics.

This chapter serves as a reference manual for the Metamath\index{Metamath}
language. It covers the tedious technical details of the language, some of
which you may wish to skim in a first reading.  On the other hand, you should
pay close attention to the defined terms in {\bf boldface}; they have precise
meanings that are important to keep in mind for later understanding.  It may
be best to first become familiar with the examples in Chapter~\ref{using} to
gain some motivation for the language.

%% Uncomment this when uncommenting section {formalspec} below
If you have some knowledge of set theory, you may wish to study this
chapter in conjunction with the formal set-theoretical description of the
Metamath language in Appendix~\ref{formalspec}.

We will use the name ``Metamath''\index{Metamath} to mean either the Metamath
computer language or the Metamath software associated with the computer
language.  We will not distinguish these two when the context is clear.

The next section contains the complete specification of the Metamath
language.
It serves as an
authoritative reference and presents the syntax in enough detail to
write a parser\index{parsing Metamath} and proof verifier.  The
specification is terse and it is probably hard to learn the language
directly from it, but we include it here for those impatient people who
prefer to see everything up front before looking at verbose expository
material.  Later sections explain this material and provide examples.
We will repeat the definitions in those sections, and you may skip the
next section at first reading and proceed to Section~\ref{tut1}
(p.~\pageref{tut1}).

\section{Specification of the Metamath Language}\label{spec}
\index{Metamath!specification}

\begin{quote}
  {\em Sometimes one has to say difficult things, but one ought to say
them as simply as one knows how.}
    \flushright\sc  G. H. Hardy\footnote{As quoted in
    \cite{deMillo}, p.~273.}\\
\end{quote}\index{Hardy, G. H.}

\subsection{Preliminaries}\label{spec1}

% Space is technically a printable character, so we'll word things
% carefully so it's unambiguous.
A Metamath {\bf database}\index{database} is built up from a top-level source
file together with any source files that are brought in through file inclusion
commands (see below).  The only characters that are allowed to appear in a
Metamath source file are the 94 non-whitespace printable {\sc
ascii}\index{ascii@{\sc ascii}} characters, which are digits, upper and lower
case letters, and the following 32 special
characters\index{special characters}:\label{spec1chars}

\begin{verbatim}
! " # $ % & ' ( ) * + , - . / :
; < = > ? @ [ \ ] ^ _ ` { | } ~
\end{verbatim}
plus the following characters which are the ``white space'' characters:
space (a printable character),
tab, carriage return, line feed, and form feed.\label{whitespace}
We will use \texttt{typewriter}
font to display the printable characters.

A Metamath database consists of a sequence of three kinds of {\bf
tokens}\index{token} separated by {\bf white space}\index{white space}
(which is any sequence of one or more white space characters).  The set
of {\bf keyword}\index{keyword} tokens is \texttt{\$\char`\{},
\texttt{\$\char`\}}, \texttt{\$c}, \texttt{\$v}, \texttt{\$f},
\texttt{\$e}, \texttt{\$d}, \texttt{\$a}, \texttt{\$p}, \texttt{\$.},
\texttt{\$=}, \texttt{\$(}, \texttt{\$)}, \texttt{\$[}, and
\texttt{\$]}.  The last four are called {\bf auxiliary}\index{auxiliary
keyword} or preprocessing keywords.  A {\bf label}\index{label} token
consists of any combination of letters, digits, and the characters
hyphen, underscore, and period.  A {\bf math symbol}\index{math symbol}
token may consist of any combination of the 92 printable standard {\sc
ascii} characters other than space or \texttt{\$}~. All tokens are
case-sensitive.

\subsection{Preprocessing}

The token \texttt{\$(} begins a {\bf comment} and
\texttt{\$)} ends a comment.\index{\texttt{\$(}
and \texttt{\$)} auxiliary keywords}\index{comment}
Comments may contain any of
the 94 non-whitespace printable characters and white space,
except they may not contain the
2-character sequences \texttt{\$(} or \texttt{\$)} (comments do not nest).
Comments are ignored (treated
like white space) for the purpose of parsing, e.g.,
\texttt{\$( \$[ \$)} is a comment.
See p.~\pageref{mathcomments} for comment typesetting conventions; these
conventions may be ignored for the purpose of parsing.

A {\bf file inclusion command} consists of \texttt{\$[} followed by a file name
followed by \texttt{\$]}.\index{\texttt{\$[} and \texttt{\$]} auxiliary
keywords}\index{included file}\index{file inclusion}
It is only allowed in the outermost scope (i.e., not between
\texttt{\$\char`\{} and \texttt{\$\char`\}})
and must not be inside a statement (e.g., it may not occur
between the label of a \texttt{\$a} statement and its \texttt{\$.}).
The file name may not
contain a \texttt{\$} or white space.  The file must exist.
The case-sensitivity
of its name follows the conventions of the operating system.  The contents of
the file replace the inclusion command.
Included files may include other files.
Only the first reference to a given file is included; any later
references to the same file (whether in the top-level file or in included
files) cause the inclusion command to be ignored (treated like white space).
A verifier may assume that file names with different strings
refer to different files for the purpose of ignoring later references.
A file self-reference is ignored, as is any reference to the top-level file
(to avoid loops).
Included files may not include a \texttt{\$(} without a matching \texttt{\$)},
may not include a \texttt{\$[} without a matching \texttt{\$]}, and may
not include incomplete statements (e.g., a \texttt{\$a} without a matching
\texttt{\$.}).
It is currently unspecified if path references are relative to the process'
current directory or the file's containing directory, so databases should
avoid using pathname separators (e.g., ``/'') in file names.

Like all tokens, the \texttt{\$(}, \texttt{\$)}, \texttt{\$[}, and \texttt{\$]} keywords
must be surrounded by white space.

\subsection{Basic Syntax}

After preprocessing, a database will consist of a sequence of {\bf
statements}.
These are the scoping statements \texttt{\$\char`\{} and
\texttt{\$\char`\}}, along with the \texttt{\$c}, \texttt{\$v},
\texttt{\$f}, \texttt{\$e}, \texttt{\$d}, \texttt{\$a}, and \texttt{\$p}
statements.

A {\bf scoping statement}\index{scoping statement} consists only of its
keyword, \texttt{\$\char`\{} or \texttt{\$\char`\}}.
A \texttt{\$\char`\{} begins a {\bf
block}\index{block} and a matching \texttt{\$\char`\}} ends the block.
Every \texttt{\$\char`\{}
must have a matching \texttt{\$\char`\}}.
Defining it recursively, we say a block
contains a sequence of zero or more tokens other
than \texttt{\$\char`\{} and \texttt{\$\char`\}} and
possibly other blocks.  There is an {\bf outermost
block}\index{block!outermost} not bracketed by \texttt{\$\char`\{} \texttt{\$\char`\}}; the end
of the outermost block is the end of the database.

% LaTeX bug? can't do \bf\tt

A {\bf \$v} or {\bf \$c statement}\index{\texttt{\$v} statement}\index{\texttt{\$c}
statement} consists of the keyword token \texttt{\$v} or \texttt{\$c} respectively,
followed by one or more math symbols,
% The word "token" is used to distinguish "$." from the sentence-ending period.
followed by the \texttt{\$.}\ token.
These
statements {\bf declare}\index{declaration} the math symbols to be {\bf
variables}\index{variable!Metamath} or {\bf constants}\index{constant}
respectively. The same math symbol may not occur twice in a given \texttt{\$v} or
\texttt{\$c} statement.

%c%A math symbol becomes an {\bf active}\index{active math symbol}
%c%when declared and stays active until the end of the block in which it is
%c%declared.  A math symbol may not be declared a second time while it is active,
%c%but it may be declared again after it becomes inactive.

A math symbol becomes {\bf active}\index{active math symbol} when declared
and stays active until the end of the block in which it is declared.  A
variable may not be declared a second time while it is active, but it
may be declared again (as a variable, but not as a constant) after it
becomes inactive.  A constant must be declared in the outermost block and may
not be declared a second time.\footnote{The rules for redeclaration may
become more general in the future;
see p.~\pageref{redeclarationf}.}\index{redeclaration of symbols}

A {\bf \$f statement}\index{\texttt{\$f} statement} consists of a label,
followed by \texttt{\$f}, followed by its typecode (an active constant),
followed by an
active variable, followed by the \texttt{\$.}\ token.  A {\bf \$e
statement}\index{\texttt{\$e} statement} consists of a label, followed
by \texttt{\$e}, followed by its typecode (an active constant),
followed by zero or more
active math symbols, followed by the \texttt{\$.}\ token.  A {\bf
hypothesis}\index{hypothesis} is a \texttt{\$f} or \texttt{\$e}
statement.
The type declared by a \texttt{\$f} statement for a given label
is global even if the variable is not
(e.g., a database may not have \texttt{wff P} in one local scope
nd \texttt{class P} in another).

A {\bf simple \$d statement}\index{\texttt{\$d} statement!simple}
consists of \texttt{\$d}, followed by two different active variables,
followed by the \texttt{\$.}\ token.  A {\bf compound \$d
statement}\index{\texttt{\$d} statement!compound} consists of
\texttt{\$d}, followed by three or more variables (all different),
followed by the \texttt{\$.}\ token.  The order of the variables in a
\texttt{\$d} statement is unimportant.  A compound \texttt{\$d}
statement is equivalent to a set of simple \texttt{\$d} statements, one
for each possible pair of variables occurring in the compound
\texttt{\$d} statement.  Henceforth in this specification we shall
assume all \texttt{\$d} statements are simple.  A \texttt{\$d} statement
is also called a {\bf disjoint} (or {\bf distinct}) {\bf variable
restriction}.\index{disjoint-variable restriction}

A {\bf \$a statement}\index{\texttt{\$a} statement} consists of a label,
followed by \texttt{\$a}, followed by its typecode (an active constant),
followed by
zero or more active math symbols, followed by the \texttt{\$.}\ token.  A {\bf
\$p statement}\index{\texttt{\$p} statement} consists of a label,
followed by \texttt{\$p}, followed by its typecode (an active constant),
followed by
zero or more active math symbols, followed by \texttt{\$=}, followed by
a sequence of labels, followed by the \texttt{\$.}\ token.  An {\bf
assertion}\index{assertion} is a \texttt{\$a} or \texttt{\$p} statement.

A \texttt{\$f}, \texttt{\$e}, or \texttt{\$d} statement is {\bf active}\index{active
statement} from the place it occurs until the end of the block it occurs in.
A \texttt{\$a} or \texttt{\$p} statement is {\bf active} from the place it occurs
through the end of the database.
There may not be two active \texttt{\$f} statements containing the same
variable.  Each variable in a \texttt{\$e}, \texttt{\$a}, or
\texttt{\$p} statement must exist in an active \texttt{\$f}
statement.\footnote{This requirement can greatly simplify the
unification algorithm (substitution calculation) required by proof
verification.}

%The label that begins each \texttt{\$f}, \texttt{\$e}, \texttt{\$a}, and
%\texttt{\$p} statement must be unique.
Each label token must be unique, and
no label token may match any math symbol
token.\label{namespace}\footnote{This
restriction was added on June 24, 2006.
It is not theoretically necessary but is imposed to make it easier to
write certain parsers.}

The set of {\bf mandatory variables}\index{mandatory variable} associated with
an assertion is the set of (zero or more) variables in the assertion and in any
active \texttt{\$e} statements.  The (possibly empty) set of {\bf mandatory
hypotheses}\index{mandatory hypothesis} is the set of all active \texttt{\$f}
statements containing mandatory variables, together with all active \texttt{\$e}
statements.
The set of {\bf mandatory {\bf \$d} statements}\index{mandatory
disjoint-variable restriction} associated with an assertion are those active
\texttt{\$d} statements whose variables are both among the assertion's
mandatory variables.

\subsection{Proof Verification}\label{spec4}

The sequence of labels between the \texttt{\$=} and \texttt{\$.}\ tokens
in a \texttt{\$p} statement is a {\bf proof}.\index{proof!Metamath} Each
label in a proof must be the label of an active statement other than the
\texttt{\$p} statement itself; thus a label must refer either to an
active hypothesis of the \texttt{\$p} statement or to an earlier
assertion.

An {\bf expression}\index{expression} is a sequence of math symbols. A {\bf
substitution map}\index{substitution map} associates a set of variables with a
set of expressions.  It is acceptable for a variable to be mapped to an
expression containing it.  A {\bf
substitution}\index{substitution!variable}\index{variable substitution} is the
simultaneous replacement of all variables in one or more expressions with the
expressions that the variables map to.

A proof is scanned in order of its label sequence.  If the label refers to an
active hypothesis, the expression in the hypothesis is pushed onto a
stack.\index{stack}\index{RPN stack}  If the label refers to an assertion, a
(unique) substitution must exist that, when made to the mandatory hypotheses
of the referenced assertion, causes them to match the topmost (i.e.\ most
recent) entries of the stack, in order of occurrence of the mandatory
hypotheses, with the topmost stack entry matching the last mandatory
hypothesis of the referenced assertion.  As many stack entries as there are
mandatory hypotheses are then popped from the stack.  The same substitution is
made to the referenced assertion, and the result is pushed onto the stack.
After the last label in the proof is processed, the stack must have a single
entry that matches the expression in the \texttt{\$p} statement containing the
proof.

%c%{\footnotesize\begin{quotation}\index{redeclaration of symbols}
%c%{{\em Comment.}\label{spec4comment} Whenever a math symbol token occurs in a
%c%{\texttt{\$c} or \texttt{\$v} statement, it is considered to designate a distinct new
%c%{symbol, even if the same token was previously declared (and is now inactive).
%c%{Thus a math token declared as a constant in two different blocks is considered
%c%{to designate two distinct constants (even though they have the same name).
%c%{The two constants will not match in a proof that references both blocks.
%c%{However, a proof referencing both blocks is acceptable as long as it doesn't
%c%{require that the constants match.  Similarly, a token declared to be a
%c%{constant for a referenced assertion will not match the same token declared to
%c%{be a variable for the \texttt{\$p} statement containing the proof.  In the case
%c%{of a token declared to be a variable for a referenced assertion, this is not
%c%{an issue since the variable can be substituted with whatever expression is
%c%{needed to achieve the required match.
%c%{\end{quotation}}
%c2%A proof may reference an assertion that contains or whose hypotheses contain a
%c2%constant that is not active for the \texttt{\$p} statement containing the proof.
%c2%However, the final result of the proof may not contain that constant. A proof
%c2%may also reference an assertion that contains or whose hypotheses contain a
%c2%variable that is not active for the \texttt{\$p} statement containing the proof.
%c2%That variable, of course, will be substituted with whatever expression is
%c2%needed to achieve the required match.

A proof may contain a \texttt{?}\ in place of a label to indicate an unknown step
(Section~\ref{unknown}).  A proof verifier may ignore any proof containing
\texttt{?}\ but should warn the user that the proof is incomplete.

A {\bf compressed proof}\index{compressed proof}\index{proof!compressed} is an
alternate proof notation described in Appen\-dix~\ref{compressed}; also see
references to ``compressed proof'' in the Index.  Compressed proofs are a
Metamath language extension which a complete proof verifier should be able to
parse and verify.

\subsubsection{Verifying Disjoint Variable Restrictions}

Each substitution made in a proof must be checked to verify that any
disjoint variable restrictions are satisfied, as follows.

If two variables replaced by a substitution exist in a mandatory \texttt{\$d}
statement\index{\texttt{\$d} statement} of the assertion referenced, the two
expressions resulting from the substitution must satisfy the following
conditions.  First, the two expressions must have no variables in common.
Second, each possible pair of variables, one from each expression, must exist
in an active \texttt{\$d} statement of the \texttt{\$p} statement containing the
proof.

\vskip 1ex

This ends the specification of the Metamath language;
see Appendix \ref{BNF} for its syntax in
Extended Backus--Naur Form (EBNF)\index{Extended Backus--Naur Form}\index{EBNF}.

\section{The Basic Keywords}\label{tut1}

Our expository material begins here.

Like most computer languages, Metamath\index{Metamath} takes its input from
one or more {\bf source files}\index{source file} which contain characters
expressed in the standard {\sc ascii} (American Standard Code for Information
Interchange)\index{ascii@{\sc ascii}} code for computers.  A source file
consists of a series of {\bf tokens}\index{token}, which are strings of
non-whitespace
printable characters (from the set of 94 shown on p.~\pageref{spec1chars})
separated by {\bf white space}\index{white space} (spaces, tabs, carriage
returns, line feeds, and form feeds). Any string consisting only of these
characters is treated the same as a single space.  The non-whitespace printable
characters\index{printable character} that Metamath recognizes are the 94
characters on standard {\sc ascii} keyboards.

Metamath has the ability to join several files together to form its
input (Section~\ref{include}).  We call the aggregate contents of all
the files after they have been joined together a {\bf
database}\index{database} to distinguish it from an individual source
file.  The tokens in a database consist of {\bf
keywords}\index{keyword}, which are built into the language, together
with two kinds of user-defined tokens called {\bf labels}\index{label}
and {\bf math symbols}\index{math symbol}.  (Often we will simply say
{\bf symbol}\index{symbol} instead of math symbol for brevity).  The set
of {\bf basic keywords}\index{basic keyword} is
\texttt{\$c}\index{\texttt{\$c} statement},
\texttt{\$v}\index{\texttt{\$v} statement},
\texttt{\$e}\index{\texttt{\$e} statement},
\texttt{\$f}\index{\texttt{\$f} statement},
\texttt{\$d}\index{\texttt{\$d} statement},
\texttt{\$a}\index{\texttt{\$a} statement},
\texttt{\$p}\index{\texttt{\$p} statement},
\texttt{\$=}\index{\texttt{\$=} keyword},
\texttt{\$.}\index{\texttt{\$.}\ keyword},
\texttt{\$\char`\{}\index{\texttt{\$\char`\{} and \texttt{\$\char`\}}
keywords}, and \texttt{\$\char`\}}.  This is the complete set of
syntactical elements of what we call the {\bf basic
language}\index{basic language} of Metamath, and with them you can
express all of the mathematics that were intended by the design of
Metamath.  You should make it a point to become very familiar with them.
Table~\ref{basickeywords} lists the basic keywords along with a brief
description of their functions.  For now, this description will give you
only a vague notion of what the keywords are for; later we will describe
the keywords in detail.


\begin{table}[htp] \caption{Summary of the basic Metamath
keywords} \label{basickeywords}
\begin{center}
\begin{tabular}{|p{4pc}|l|}
\hline
\em \centering Keyword&\em Description\\
\hline
\hline
\centering
   \texttt{\$c}&Constant symbol declaration\\
\hline
\centering
   \texttt{\$v}&Variable symbol declaration\\
\hline
\centering
   \texttt{\$d}&Disjoint variable restriction\\
\hline
\centering
   \texttt{\$f}&Variable-type (``floating'') hypothesis\\
\hline
\centering
   \texttt{\$e}&Logical (``essential'') hypothesis\\
\hline
\centering
   \texttt{\$a}&Axiomatic assertion\\
\hline
\centering
   \texttt{\$p}&Provable assertion\\
\hline
\centering
   \texttt{\$=}&Start of proof in \texttt{\$p} statement\\
\hline
\centering
   \texttt{\$.}&End of the above statement types\\
\hline
\centering
   \texttt{\$\char`\{}&Start of block\\
\hline
\centering
   \texttt{\$\char`\}}&End of block\\
\hline
\end{tabular}
\end{center}
\end{table}

%For LaTeX bug(?) where it puts tables on blank page instead of btwn text
%May have to adjust if text changes
%\newpage

There are some additional keywords, called {\bf auxiliary
keywords}\index{auxiliary keyword} that help make Metamath\index{Metamath}
more practical. These are part of the {\bf extended language}\index{extended
language}. They provide you with a means to put comments into a Metamath
source file\index{source file} and reference other source files.  We will
introduce these in later sections. Table~\ref{otherkeywords} summarizes them
so that you can recognize them now if you want to peruse some source
files while learning the basic keywords.


\begin{table}[htp] \caption{Auxiliary Metamath
keywords} \label{otherkeywords}
\begin{center}
\begin{tabular}{|p{4pc}|l|}
\hline
\em \centering Keyword&\em Description\\
\hline
\hline
\centering
   \texttt{\$(}&Start of comment\\
\hline
\centering
   \texttt{\$)}&End of comment\\
\hline
\centering
   \texttt{\$[}&Start of included source file name\\
\hline
\centering
   \texttt{\$]}&End of included source file name\\
\hline
\end{tabular}
\end{center}
\end{table}
\index{\texttt{\$(} and \texttt{\$)} auxiliary keywords}
\index{\texttt{\$[} and \texttt{\$]} auxiliary keywords}


Unlike those in some computer languages, the keywords\index{keyword} are short
two-character sequences rather than English-like words.  While this may make
them slightly more difficult to remember at first, their brevity allows
them to blend in with the mathematics being described, not
distract from it, like punctuation marks.


\subsection{User-Defined Tokens}\label{dollardollar}\index{token}

As you may have noticed, all keywords\index{keyword} begin with the \texttt{\$}
character.  This mundane monetary symbol is not ordinarily used in higher
mathematics (outside of grant proposals), so we have appropriated it to
distinguish the Metamath\index{Metamath} keywords from ordinary mathematical
symbols. The \texttt{\$} character is thus considered special and may not be
used as a character in a user-defined token.  All tokens and keywords are
case-sensitive; for example, \texttt{n} is considered to be a different character
from \texttt{N}.  Case-sensitivity makes the available {\sc ascii} character set
as rich as possible.

\subsubsection{Math Symbol Tokens}\index{token}

Math symbols\index{math symbol} are tokens used to represent the symbols
that appear in ordinary mathematical formulas.  They may consist of any
combination of the 93 non-whitespace printable {\sc ascii} characters other than
\texttt{\$}~. Some examples are \texttt{x}, \texttt{+}, \texttt{(},
\texttt{|-}, \verb$!%@?&$, and \texttt{bounded}.  For readability, it is
best to try to make these look as similar to actual mathematical symbols
as possible, within the constraints of the {\sc ascii} character set, in
order to make the resulting mathematical expressions more readable.

In the Metamath\index{Metamath} language, you express ordinary
mathematical formulas and statements as sequences of math symbols such
as \texttt{2 + 2 = 4} (five symbols, all constants).\footnote{To
eliminate ambiguity with other expressions, this is expressed in the set
theory database \texttt{set.mm} as \texttt{|- ( 2 + 2
 ) = 4 }, whose \LaTeX\ equivalent is $\vdash
(2+2)=4$.  The \,$\vdash$ means ``is a theorem'' and the
parentheses allow explicit associative grouping.}\index{turnstile
({$\,\vdash$})} They may even be English
sentences, as in \texttt{E is closed and bounded} (five symbols)---here
\texttt{E} would be a variable and the other four symbols constants.  In
principle, a Metamath database could be constructed to work with almost
any unambiguous English-language mathematical statement, but as a
practical matter the definitions needed to provide for all possible
syntax variations would be cumbersome and distracting and possibly have
subtle pitfalls accidentally built in.  We generally recommend that you
express mathematical statements with compact standard mathematical
symbols whenever possible and put their English-language descriptions in
comments.  Axioms\index{axiom} and definitions\index{definition}
(\texttt{\$a}\index{\texttt{\$a} statement} statements) are the only
places where Metamath will not detect an error, and doing this will help
reduce the number of definitions needed.

You are free to use any tokens\index{token} you like for math
symbols\index{math symbol}.  Appendix~\ref{ASCII} recommends token names to
use for symbols in set theory, and we suggest you adopt these in order to be
able to include the \texttt{set.mm} set theory database in your database.  For
printouts, you can convert the tokens in a database
to standard mathematical symbols with the \LaTeX\ typesetting program.  The
Metamath command \texttt{open tex} {\em filename}\index{\texttt{open tex} command}
produces output that can be read by \LaTeX.\index{latex@{\LaTeX}}
The correspondence
between tokens and the actual symbols is made by \texttt{latexdef}
statements inside a special database comment tagged
with \texttt{\$t}.\index{\texttt{\$t} comment}\index{typesetting comment}
  You can edit
this comment to change the definitions or add new ones.
Appendix~\ref{ASCII} describes how to do this in more detail.

% White space\index{white space} is normally used to separate math
% symbol\index{math symbol} tokens, but they may be juxtaposed without white
% space in \texttt{\$d}\index{\texttt{\$d} statement}, \texttt{\$e}\index{\texttt{\$e}
% statement}, \texttt{\$f}\index{\texttt{\$f} statement}, \texttt{\$a}\index{\texttt{\$a}
% statement}, and \texttt{\$p}\index{\texttt{\$p} statement} statements when no
% ambiguity will result.  Specifically, Metamath parses the math symbol sequence
% in one of these statements in the following manner:  when the math symbol
% sequence has been broken up into tokens\index{token} up to a given character,
% the next token is the longest string of characters that could constitute a
% math symbol that is active\index{active
% math symbol} at that point.  (See Section~\ref{scoping} for the
% definition of an active math symbol.)  For example, if \texttt{-}, \texttt{>}, and
% \texttt{->} are the only active math symbols, the juxtaposition \texttt{>-} will be
% interpreted as the two symbols \texttt{>} and \texttt{-}, whereas \texttt{->} will
% always be interpreted as that single symbol.\footnote{For better readability we
% recommend a white space between each token.  This also makes searching for a
% symbol easier to do with an editor.  Omission of optional white space is useful
% for reducing typing when assigning an expression to a temporary
% variable\index{temporary variable} with the \texttt{let variable} Metamath
% program command.}\index{\texttt{let variable} command}
%
% Keywords\index{keyword} may be placed next to math symbols without white
% space\index{white space} between them.\footnote{Again, we do not recommend
% this for readability.}
%
% The math symbols\index{math symbol} in \texttt{\$c}\index{\texttt{\$c} statement}
% and \texttt{\$v}\index{\texttt{\$v} statement} statements must always be separated
% by white space\index{white
% space}, for the obvious reason that these statements define the names
% of the symbols.
%
% Math symbols referred to in comments (see Section~\ref{comments}) must also be
% separated by white space.  This allows you to make comments about symbols that
% are not yet active\index{active
% math symbol}.  (The ``math mode'' feature of comments is also a quick and
% easy way to obtain word processing text with embedded mathematical symbols,
% independently of the main purpose of Metamath; the way to do this is described
% in Section~\ref{comments})

\subsubsection{Label Tokens}\index{token}\index{label}

Label tokens are used to identify Metamath\index{Metamath} statements for
later reference. Label tokens may contain only letters, digits, and the three
characters period, hyphen, and underscore:
\begin{verbatim}
. - _
\end{verbatim}

A label is {\bf declared}\index{label declaration} by placing it immediately
before the keyword of the statement it identifies.  For example, the label
\texttt{axiom.1} might be declared as follows:
\begin{verbatim}
axiom.1 $a |- x = x $.
\end{verbatim}

Each \texttt{\$e}\index{\texttt{\$e} statement},
\texttt{\$f}\index{\texttt{\$f} statement},
\texttt{\$a}\index{\texttt{\$a} statement}, and
\texttt{\$p}\index{\texttt{\$p} statement} statement in a database must
have a label declared for it.  No other statement types may have label
declarations.  Every label must be unique.

A label (and the statement it identifies) is {\bf referenced}\index{label
reference} by including the label between the \texttt{\$=}\index{\texttt{\$=}
keyword} and \texttt{\$.}\index{\texttt{\$.}\ keyword}\ keywords in a \texttt{\$p}
statement.  The sequence of labels\index{label sequence} between these two
keywords is called a {\bf proof}\index{proof}.  An example of a statement with
a proof that we will encounter later (Section~\ref{proof}) is
\begin{verbatim}
wnew $p wff ( s -> ( r -> p ) )
     $= ws wr wp w2 w2 $.
\end{verbatim}

You don't have to know what this means just yet, but you should know that the
label \texttt{wnew} is declared by this \texttt{\$p} statement and that the labels
\texttt{ws}, \texttt{wr}, \texttt{wp}, and \texttt{w2} are assumed to have been declared
earlier in the database and are referenced here.

\subsection{Constants and Variables}
\index{constant}
\index{variable}

An {\bf expression}\index{expression} is any sequence of math
symbols, possibly empty.

The basic Metamath\index{Metamath} language\index{basic language} has two
kinds of math symbols\index{math symbol}:  {\bf constants}\index{constant} and
{\bf variables}\index{variable}.  In a Metamath proof, a constant may not be
substituted with any expression.  A variable can be
substituted\index{substitution!variable}\index{variable substitution} with any
expression.  This sequence may include other variables and may even include
the variable being substituted.  This substitution takes place when proofs are
verified, and it will be described in Section~\ref{proof}.  The \texttt{\$f}
statement (described later in Section~\ref{dollaref}) is used to specify the
{\bf type} of a variable (i.e.\ what kind of
variable it is)\index{variable type}\index{type} and
give it a meaning typically
associated with a ``metavariable''\index{metavariable}\footnote{A metavariable
is a variable that ranges over the syntactical elements of the object language
being discussed; for example, one metavariable might represent a variable of
the object language and another metavariable might represent a formula in the
object language.} in ordinary mathematics; for example, a variable may be
specified to be a wff or well-formed formula (in logic), a set (in set
theory), or a non-negative integer (in number theory).

%\subsection{The \texttt{\$c} and \texttt{\$v} Declaration Statements}
\subsection{The \texttt{\$c} and \texttt{\$v} Declaration Statements}
\index{\texttt{\$c} statement}
\index{constant declaration}
\index{\texttt{\$v} statement}
\index{variable declaration}

Constants are introduced or {\bf declared}\index{constant declaration}
with \texttt{\$c}\index{\texttt{\$c} statement} statements, and
variables are declared\index{variable declaration} with
\texttt{\$v}\index{\texttt{\$v} statement} statements.  A {\bf simple}
declaration\index{simple declaration} statement introduces a single
constant or variable.  Its syntax is one of the following:
\begin{center}
  \texttt{\$c} {\em math-symbol} \texttt{\$.}\\
  \texttt{\$v} {\em math-symbol} \texttt{\$.}
\end{center}
The notation {\em math-symbol} means any math symbol token\index{token}.

Some examples of simple declaration statements are:
\begin{center}
  \texttt{\$c + \$.}\\
  \texttt{\$c -> \$.}\\
  \texttt{\$c ( \$.}\\
  \texttt{\$v x \$.}\\
  \texttt{\$v y2 \$.}
\end{center}

The characters in a math symbol\index{math symbol} being declared are
irrelevant to Meta\-math; for example, we could declare a right parenthesis to
be a variable,
\begin{center}
  \texttt{\$v ) \$.}\\
\end{center}
although this would be unconventional.

A {\bf compound} declaration\index{compound declaration} statement is a
shorthand for declaring several symbols at once.  Its syntax is one of the
following:
\begin{center}
  \texttt{\$c} {\em math-symbol}\ \,$\cdots$\ {\em math-symbol} \texttt{\$.}\\
  \texttt{\$v} {\em math-symbol}\ \,$\cdots$\ {\em math-symbol} \texttt{\$.}
\end{center}\index{\texttt{\$c} statement}
Here, the ellipsis (\ldots) means any number of {\em math-symbol}\,s.

An example of a compound declaration statement is:
\begin{center}
  \texttt{\$v x y mu \$.}\\
\end{center}
This is equivalent to the three simple declaration statements
\begin{center}
  \texttt{\$v x \$.}\\
  \texttt{\$v y \$.}\\
  \texttt{\$v mu \$.}\\
\end{center}
\index{\texttt{\$v} statement}

There are certain rules on where in the database math symbols may be declared,
what sections of the database are aware of them (i.e.\ where they are
``active''), and when they may be declared more than once.  These will be
discussed in Section~\ref{scoping} and specifically on
p.~\pageref{redeclaration}.

\subsection{The \texttt{\$d} Statement}\label{dollard}
\index{\texttt{\$d} statement}

The \texttt{\$d} statement is called a {\bf disjoint-variable restriction}.  The
syntax of the {\bf simple} version of this statement is
\begin{center}
  \texttt{\$d} {\em variable variable} \texttt{\$.}
\end{center}
where each {\em variable} is a previously declared variable and the two {\em
variable}\,s are different.  (More specifically, each  {\em variable} must be
an {\bf active} variable\index{active math symbol}, which means there must be
a previous \texttt{\$v} statement whose {\bf scope}\index{scope} includes the
\texttt{\$d} statement.  These terms will be defined when we discuss scoping
statements in Section~\ref{scoping}.)

In ordinary mathematics, formulas may arise that are true if the variables in
them are distinct\index{distinct variables}, but become false when those
variables are made identical. For example, the formula in logic $\exists x\,x
\neq y$, which means ``for a given $y$, there exists an $x$ that is not equal
to $y$,'' is true in most mathematical theories (namely all non-trivial
theories\index{non-trivial theory}, i.e.\ those that describe more than one
individual, such as arithmetic).  However, if we substitute $y$ with $x$, we
obtain $\exists x\,x \neq x$, which is always false, as it means ``there
exists something that is not equal to itself.''\footnote{If you are a
logician, you will recognize this as the improper substitution\index{proper
substitution}\index{substitution!proper} of a free variable\index{free
variable} with a bound variable\index{bound variable}.  Metamath makes no
inherent distinction between free and bound variables; instead, you let
Metamath know what substitutions are permissible by using \texttt{\$d} statements
in the right way in your axiom system.}\index{free vs.\ bound variable}  The
\texttt{\$d} statement allows you to specify a restriction that forbids the
substitution of one variable with another.  In
this case, we would use the statement
\begin{center}
  \texttt{\$d x y \$.}
\end{center}\index{\texttt{\$d} statement}
to specify this restriction.

The order in which the variables appear in a \texttt{\$d} statement is not
important.  We could also use
\begin{center}
  \texttt{\$d y x \$.}
\end{center}

The \texttt{\$d} statement is actually more general than this, as the
``disjoint''\index{disjoint variables} in its name suggests.  The full meaning
is that if any substitution is made to its two variables (during the
course of a proof that references a \texttt{\$a} or \texttt{\$p} statement
associated with the \texttt{\$d}), the two expressions that result from the
substitution must have no variables in common.  In addition, each possible
pair of variables, one from each expression, must be in a \texttt{\$d} statement
associated with the statement being proved.  (This requirement forces the
statement being proved to ``inherit'' the original disjoint variable
restriction.)

For example, suppose \texttt{u} is a variable.  If the restriction
\begin{center}
  \texttt{\$d A B \$.}
\end{center}
has been specified for a theorem referenced in a
proof, we may not substitute \texttt{A} with \mbox{\tt a + u} and
\texttt{B} with \mbox{\tt b + u} because these two symbol sequences have the
variable \texttt{u} in common.  Furthermore, if \texttt{a} and \texttt{b} are
variables, we may not substitute \texttt{A} with \texttt{a} and \texttt{B} with \texttt{b}
unless we have also specified \texttt{\$d a b} for the theorem being proved; in
other words, the \texttt{\$d} property associated with a pair of variables must
be effectively preserved after substitution.

The \texttt{\$d}\index{\texttt{\$d} statement} statement does {\em not} mean ``the
two variables may not be substituted with the same thing,'' as you might think
at first.  For example, substituting each of \texttt{A} and \texttt{B} in the above
example with identical symbol sequences consisting only of constants does not
cause a disjoint variable conflict, because two symbol sequences have no
variables in common (since they have no variables, period).  Similarly, a
conflict will not occur by substituting the two variables in a \texttt{\$d}
statement with the empty symbol sequence\index{empty substitution}.

The \texttt{\$d} statement does not have a direct counterpart in
ordinary mathematics, partly because the variables\index{variable} of
Metamath are not really the same as the variables\index{variable!in
ordinary mathematics} of ordinary mathematics but rather are
metavariables\index{metavariable} ranging over them (as well as over
other kinds of symbols and groups of symbols).  Depending on the
situation, we may informally interpret the \texttt{\$d} statement in
different ways.  Suppose, for example, that \texttt{x} and \texttt{y}
are variables ranging over numbers (more precisely, that \texttt{x} and
\texttt{y} are metavariables ranging over variables that range over
numbers), and that \texttt{ph} ($\varphi$) and \texttt{ps} ($\psi$) are
variables (more precisely, metavariables) ranging over formulas.  We can
make the following interpretations that correspond to the informal
language of ordinary mathematics:
\begin{quote}
\begin{tabbing}
\texttt{\$d x y \$.} means ``assume $x$ and $y$ are
distinct variables.''\\
\texttt{\$d x ph \$.} means ``assume $x$ does not
occur in $\varphi$.''\\
\texttt{\$d ph ps \$.} \=means ``assume $\varphi$ and
$\psi$ have no variables\\ \>in common.''
\end{tabbing}
\end{quote}\index{\texttt{\$d} statement}

\subsubsection{Compound \texttt{\$d} Statements}

The {\bf compound} version of the \texttt{\$d} statement is a shorthand for
specifying several variables whose substitutions must be pairwise disjoint.
Its syntax is:
\begin{center}
  \texttt{\$d} {\em variable}\ \,$\cdots$\ {\em variable} \texttt{\$.}
\end{center}\index{\texttt{\$d} statement}
Here, {\em variable} represents the token of a previously declared
variable (specifically, an active variable) and all {\em variable}\,s are
different.  The compound \texttt{\$d}
statement is internally broken up by Metamath into one simple \texttt{\$d}
statement for each possible pair of variables in the original \texttt{\$d}
statement.  For example,
\begin{center}
  \texttt{\$d w x y z \$.}
\end{center}
is equivalent to
\begin{center}
  \texttt{\$d w x \$.}\\
  \texttt{\$d w y \$.}\\
  \texttt{\$d w z \$.}\\
  \texttt{\$d x y \$.}\\
  \texttt{\$d x z \$.}\\
  \texttt{\$d y z \$.}
\end{center}

Two or more simple \texttt{\$d} statements specifying the same variable pair are
internally combined into a single \texttt{\$d} statement.  Thus the set of three
statements
\begin{center}
  \texttt{\$d x y \$.}
  \texttt{\$d x y \$.}
  \texttt{\$d y x \$.}
\end{center}
is equivalent to
\begin{center}
  \texttt{\$d x y \$.}
\end{center}

Similarly, compound \texttt{\$d} statements, after being internally broken up,
internally have their common variable pairs combined.  For example the
set of statements
\begin{center}
  \texttt{\$d x y A \$.}
  \texttt{\$d x y B \$.}
\end{center}
is equivalent to
\begin{center}
  \texttt{\$d x y \$.}
  \texttt{\$d x A \$.}
  \texttt{\$d y A \$.}
  \texttt{\$d x y \$.}
  \texttt{\$d x B \$.}
  \texttt{\$d y B \$.}
\end{center}
which is equivalent to
\begin{center}
  \texttt{\$d x y \$.}
  \texttt{\$d x A \$.}
  \texttt{\$d y A \$.}
  \texttt{\$d x B \$.}
  \texttt{\$d y B \$.}
\end{center}

Metamath\index{Metamath} automatically verifies that all \texttt{\$d}
restrictions are met whenever it verifies proofs.  \texttt{\$d} statements are
never referenced directly in proofs (this is why they do not have
labels\index{label}), but Metamath is always aware of which ones must be
satisfied (i.e.\ are active) and will notify you with an error message if any
violation occurs.

To illustrate how Metamath detects a missing \texttt{\$d}
statement, we will look at the following example from the
\texttt{set.mm} database.

\begin{verbatim}
$d x z $.  $d y z $.
$( Theorem to add distinct quantifier to atomic formula. $)
ax17eq $p |- ( x = y -> A. z x = y ) $=...
\end{verbatim}

This statement has the obvious requirement that $z$ must be
distinct\index{distinct variables} from $x$ in theorem \texttt{ax17eq} that
states $x=y \rightarrow \forall z \, x=y$ (well, obvious if you're a logician,
for otherwise we could conclude  $x=y \rightarrow \forall x \, x=y$, which is
false when the free variables $x$ and $y$ are equal.).

Let's look at what happens if we edit the database to comment out this
requirement.

\begin{verbatim}
$( $d x z $. $) $d y z $.
$( Theorem to add distinct quantifier to atomic formula. $)
ax17eq $p |- ( x = y -> A. z x = y ) $=...
\end{verbatim}

When it tries to verify the proof, Metamath will tell you that \texttt{x} and
\texttt{z} must be disjoint, because one of its steps references an axiom or
theorem that has this requirement.

\begin{verbatim}
MM> verify proof ax17eq
ax17eq ?Error at statement 1918, label "ax17eq", type "$p":
      vz wal wi vx vy vz ax-13 vx vy weq vz vx ax-c16 vx vy
                                               ^^^^^
There is a disjoint variable ($d) violation at proof step 29.
Assertion "ax-c16" requires that variables "x" and "y" be
disjoint.  But "x" was substituted with "z" and "y" was
substituted with "x".  The assertion being proved, "ax17eq",
does not require that variables "z" and "x" be disjoint.
\end{verbatim}

We can see the substitutions into \texttt{ax-c16} with the following command.

\begin{verbatim}
MM> show proof ax17eq / detailed_step 29
Proof step 29:  pm2.61dd.2=ax-c16 $a |- ( A. z z = x -> ( x =
  y -> A. z x = y ) )
This step assigns source "ax-c16" ($a) to target "pm2.61dd.2"
($e).  The source assertion requires the hypotheses "wph"
($f, step 26), "vx" ($f, step 27), and "vy" ($f, step 28).
The parent assertion of the target hypothesis is "pm2.61dd"
($p, step 36).
The source assertion before substitution was:
    ax-c16 $a |- ( A. x x = y -> ( ph -> A. x ph ) )
The following substitutions were made to the source
assertion:
    Variable  Substituted with
     x         z
     y         x
     ph        x = y
The target hypothesis before substitution was:
    pm2.61dd.2 $e |- ( ph -> ch )
The following substitutions were made to the target
hypothesis:
    Variable  Substituted with
     ph        A. z z = x
     ch        ( x = y -> A. z x = y )
\end{verbatim}

The disjoint variable restrictions of \texttt{ax-c16} can be seen from the
\texttt{show state\-ment} command.  The line that begins ``\texttt{Its mandatory
dis\-joint var\-i\-able pairs are:}\ldots'' lists any \texttt{\$d} variable
pairs in brackets.

\begin{verbatim}
MM> show statement ax-c16/full
Statement 3033 is located on line 9338 of the file "set.mm".
"Axiom of Distinct Variables. ..."
  ax-c16 $a |- ( A. x x = y -> ( ph -> A. x ph ) ) $.
Its mandatory hypotheses in RPN order are:
  wph $f wff ph $.
  vx $f set x $.
  vy $f set y $.
Its mandatory disjoint variable pairs are:  <x,y>
The statement and its hypotheses require the variables:  x y
      ph
The variables it contains are:  x y ph
\end{verbatim}

Since Metamath will always detect when \texttt{\$d}\index{\texttt{\$d} statement}
statements are needed for a proof, you don't have to worry too much about
forgetting to put one in; it can always be added if you see the error message
above.  If you put in unnecessary \texttt{\$d} statements, the worst that could
happen is that your theorem might not be as general as it could be, and this
may limit its use later on.

On the other hand, when you introduce axioms (\texttt{\$a}\index{\texttt{\$a}
statement} statements), you must be very careful to properly specify the
necessary associated \texttt{\$d} statements since Metamath has no way of knowing
whether your axioms are correct.  For example, Metamath would have no idea
that \texttt{ax-c16}, which we are telling it is an axiom of logic, would lead to
contradictions if we omitted its associated \texttt{\$d} statement.

% This was previously a comment in footnote-sized type, but it can be
% hard to read this much text in a small size.
% As a result, it's been changed to normally-sized text.
\label{nodd}
You may wonder if it is possible to develop standard
mathematics in the Metamath language without the \texttt{\$d}\index{\texttt{\$d}
statement} statement, since it seems like a nuisance that complicates proof
verification. The \texttt{\$d} statement is not needed in certain subsets of
mathematics such as propositional calculus.  However, dummy
variables\index{dummy variable!eliminating} and their associated \texttt{\$d}
statements are impossible to avoid in proofs in standard first-order logic as
well as in the variant used in \texttt{set.mm}.  In fact, there is no upper bound to
the number of dummy variables that might be needed in a proof of a theorem of
first-order logic containing 3 or more variables, as shown by H.\
Andr\'{e}ka\index{Andr{\'{e}}ka, H.} \cite{Nemeti}.  A first-order system that
avoids them entirely is given in \cite{Megill}\index{Megill, Norman}; the
trick there is simply to embed harmlessly the necessary dummy variables into a
theorem being proved so that they aren't ``dummy'' anymore, then interpret the
resulting longer theorem so as to ignore the embedded dummy variables.  If
this interests you, the system in \texttt{set.mm} obtained from \texttt{ax-1}
through \texttt{ax-c14} in \texttt{set.mm}, and deleting \texttt{ax-c16} and \texttt{ax-5},
requires no \texttt{\$d} statements but is logically complete in the sense
described in \cite{Megill}.  This means it can prove any theorem of
first-order logic as long as we add to the theorem an antecedent that embeds
dummy and any other variables that must be distinct.  In a similar fashion,
axioms for set theory can be devised that
do not require distinct variable
provisos\index{Set theory without distinct variable provisos},
as explained at
\url{http://us.metamath.org/mpeuni/mmzfcnd.html}.
Together, these in principle allow all of
mathematics to be developed under Metamath without a \texttt{\$d} statement,
although the length of the resulting theorems will grow as more and
more dummy variables become required in their proofs.

\subsection{The \texttt{\$f}
and \texttt{\$e} Statements}\label{dollaref}
\index{\texttt{\$e} statement}
\index{\texttt{\$f} statement}
\index{floating hypothesis}
\index{essential hypothesis}
\index{variable-type hypothesis}
\index{logical hypothesis}
\index{hypothesis}

Metamath has two kinds of hypo\-theses, the \texttt{\$f}\index{\texttt{\$f}
statement} or {\bf variable-type} hypothesis and the \texttt{\$e} or {\bf logical}
hypo\-the\-sis.\index{\texttt{\$d} statement}\footnote{Strictly speaking, the
\texttt{\$d} statement is also a hypothesis, but it is never directly referenced
in a proof, so we call it a restriction rather than a hypothesis to lessen
confusion.  The checking for violations of \texttt{\$d} restrictions is automatic
and built into Metamath's proof-checking algorithm.} The letters \texttt{f} and
\texttt{e} stand for ``floating''\index{floating hypothesis} (roughly meaning
used only if relevant) and ``essential''\index{essential hypothesis} (meaning
always used) respectively, for reasons that will become apparent
when we discuss frames in
Section~\ref{frames} and scoping in Section~\ref{scoping}. The syntax of these
are as follows:
\begin{center}
  {\em label} \texttt{\$f} {\em typecode} {\em variable} \texttt{\$.}\\
  {\em label} \texttt{\$e} {\em typecode}
      {\em math-symbol}\ \,$\cdots$\ {\em math-symbol} \texttt{\$.}\\
\end{center}
\index{\texttt{\$e} statement}
\index{\texttt{\$f} statement}
A hypothesis must have a {\em label}\index{label}.  The expression in a
\texttt{\$e} hypothesis consists of a typecode (an active constant math symbol)
followed by a sequence
of zero or more math symbols. Each math symbol (including {\em constant}
and {\em variable}) must be a previously declared constant or variable.  (In
addition, each math symbol must be active, which will be covered when we
discuss scoping statements in Section~\ref{scoping}.)  You use a \texttt{\$f}
hypothesis to specify the
nature or {\bf type}\index{variable type}\index{type} of a variable (such as ``let $x$ be an
integer'') and use a \texttt{\$e} hypothesis to express a logical truth (such as
``assume $x$ is prime'') that must be established in order for an assertion
requiring it to also be true.

A variable must have its type specified in a \texttt{\$f} statement before
it may be used in a \texttt{\$e}, \texttt{\$a}, or \texttt{\$p}
statement.  There may be only one (active) \texttt{\$f} statement for a
given variable.  (``Active'' is defined in Section~\ref{scoping}.)

In ordinary mathematics, theorems\index{theorem} are often expressed in the
form ``Assume $P$; then $Q$,'' where $Q$ is a statement that you can derive
if you start with statement $P$.\index{free variable}\footnote{A stronger
version of a theorem like this would be the {\em single} formula $P\rightarrow
Q$ ($P$ implies $Q$) from which the weaker version above follows by the rule
of modus ponens in logic.  We are not discussing this stronger form here.  In
the weaker form, we are saying only that if we can {\em prove} $P$, then we can
{\em prove} $Q$.  In a logician's language, if $x$ is the only free variable
in $P$ and $Q$, the stronger form is equivalent to $\forall x ( P \rightarrow
Q)$ (for all $x$, $P$ implies $Q$), whereas the weaker form is equivalent to
$\forall x P \rightarrow \forall x Q$. The stronger form implies the weaker,
but not vice-versa.  To be precise, the weaker form of the theorem is more
properly called an ``inference'' rather than a theorem.}\index{inference}
In the
Metamath\index{Metamath} language, you would express mathematical statement
$P$ as a hypothesis (a \texttt{\$e} Metamath language statement in this case) and
statement $Q$ as a provable assertion (a \texttt{\$p}\index{\texttt{\$p} statement}
statement).

Some examples of hypotheses you might encounter in logic and set theory are
\begin{center}
  \texttt{stmt1 \$f wff P \$.}\\
  \texttt{stmt2 \$f set x \$.}\\
  \texttt{stmt3 \$e |- ( P -> Q ) \$.}
\end{center}
\index{\texttt{\$e} statement}
\index{\texttt{\$f} statement}
Informally, these would be read, ``Let $P$ be a well-formed-formula,'' ``Let
$x$ be an (individual) variable,'' and ``Assume we have proved $P \rightarrow
Q$.''  The turnstile symbol \,$\vdash$\index{turnstile ({$\,\vdash$})} is
commonly used in logic texts to mean ``a proof exists for.''

To summarize:
\begin{itemize}
\item A \texttt{\$f} hypothesis tells Metamath the type or kind of its variable.
It is analogous to a variable declaration in a computer language that
tells the compiler that a variable is an integer or a floating-point
number.
\item The \texttt{\$e} hypothesis corresponds to what you would usually call a
``hypothesis'' in ordinary mathematics.
\end{itemize}

Before an assertion\index{assertion} (\texttt{\$a} or \texttt{\$p} statement) can be
referenced in a proof, all of its associated \texttt{\$f} and \texttt{\$e} hypotheses
(i.e.\ those \texttt{\$e} hypotheses that are active) must be satisfied (i.e.
established by the proof).  The meaning of ``associated'' (which we will call
{\bf mandatory} in Section~\ref{frames}) will become clear when we discuss
scoping later.

Note that after any \texttt{\$f}, \texttt{\$e},
\texttt{\$a}, or \texttt{\$p} token there is a required
\textit{typecode}\index{typecode}.
The typecode is a constant used to enforce types of expressions.
This will become clearer once we learn more about
assertions (\texttt{\$a} and \texttt{\$p} statements).
An example may also clarify their purpose.
In the
\texttt{set.mm}\index{set theory database (\texttt{set.mm})}%
\index{Metamath Proof Explorer}
database,
the following typecodes are used:

\begin{itemize}
\item \texttt{wff} :
  Well-formed formula (wff) symbol
  (read: ``the following symbol sequence is a wff'').
% The *textual* typecode for turnstile is "|-", but when read it's a little
% confusing, so I intentionally display the mathematical symbol here instead
% (I think it's clearer in this context).
\item \texttt{$\vdash$} :
  Turnstile (read: ``the following symbol sequence is provable'' or
  ``a proof exists for'').
\item \texttt{setvar} :
  Individual set variable type (read: ``the following is an
  individual set variable'').
  Note that this is \textit{not} the type of an arbitrary set expression,
  instead, it is used to ensure that there is only a single symbol used
  after quantifiers like for-all ($\forall$) and there-exists ($\exists$).
\item \texttt{class} :
  An expression that is a syntactically valid class expression.
  All valid set expressions are also valid class expression, so expressions
  of sets normally have the \texttt{class} typecode.
  Use the \texttt{class} typecode,
  \textit{not} the \texttt{setvar} typecode,
  for the type of set expressions unless you are specifically identifying
  a single set variable.
\end{itemize}

\subsection{Assertions (\texttt{\$a} and \texttt{\$p} Statements)}
\index{\texttt{\$a} statement}
\index{\texttt{\$p} statement}\index{assertion}\index{axiomatic assertion}
\index{provable assertion}

There are two types of assertions, \texttt{\$a}\index{\texttt{\$a} statement}
statements ({\bf axiomatic assertions}) and \texttt{\$p} statements ({\bf
provable assertions}).  Their syntax is as follows:
\begin{center}
  {\em label} \texttt{\$a} {\em typecode} {\em math-symbol} \ldots
         {\em math-symbol} \texttt{\$.}\\
  {\em label} \texttt{\$p} {\em typecode} {\em math-symbol} \ldots
        {\em math-symbol} \texttt{\$=} {\em proof} \texttt{\$.}
\end{center}
\index{\texttt{\$a} statement}
\index{\texttt{\$p} statement}
\index{\texttt{\$=} keyword}
An assertion always requires a {\em label}\index{label}. The expression in an
assertion consists of a typecode (an active constant)
followed by a sequence of zero
or more math symbols.  Each math symbol, including any {\em constant}, must be a
previously declared constant or variable.  (In addition, each math symbol
must be active, which will be covered when we discuss scoping statements in
Section~\ref{scoping}.)

A \texttt{\$a} statement is usually a definition of syntax (for example, if $P$
and $Q$ are wffs then so is $(P\to Q)$), an axiom\index{axiom} of ordinary
mathematics (for example, $x=x$), or a definition\index{definition} of
ordinary mathematics (for example, $x\ne y$ means $\lnot x=y$). A \texttt{\$p}
statement is a claim that a certain combination of math symbols follows from
previous assertions and is accompanied by a proof that demonstrates it.

Assertions can also be referenced in (later) proofs in order to derive new
assertions from them. The label of an assertion is used to refer to it in a
proof. Section~\ref{proof} will describe the proof in detail.

Assertions also provide the primary means for communicating the mathematical
results in the database to people.  Proofs (when conveniently displayed)
communicate to people how the results were arrived at.

\subsubsection{The \texttt{\$a} Statement}
\index{\texttt{\$a} statement}

Axiomatic assertions (\texttt{\$a} statements) represent the starting points from
which other assertions (\texttt{\$p}\index{\texttt{\$p} statement} statements) are
derived.  Their most obvious use is for specifying ordinary mathematical
axioms\index{axiom}, but they are also used for two other purposes.

First, Metamath\index{Metamath} needs to know the syntax of symbol
sequences that constitute valid mathematical statements.  A Metamath
proof must be broken down into much more detail than ordinary
mathematical proofs that you may be used to thinking of (even the
``complete'' proofs of formal logic\index{formal logic}).  This is one
of the things that makes Metamath a general-purpose language,
independent of any system of logic or even syntax.  If you want to use a
substitution instance of an assertion as a step in a proof, you must
first prove that the substitution is syntactically correct (or if you
prefer, you must ``construct'' it), showing for example that the
expression you are substituting for a wff metavariable is a valid wff.
The \texttt{\$a}\index{\texttt{\$a} statement} statement is used to
specify those combinations of symbols that are considered syntactically
valid, such as the legal forms of wffs.

Second, \texttt{\$a} statements are used to specify what are ordinarily thought of
as definitions, i.e.\ new combinations of symbols that abbreviate other
combinations of symbols.  Metamath makes no distinction\index{axiom vs.\
definition} between axioms\index{axiom} and definitions\index{definition}.
Indeed, it has been argued that such distinction should not be made even in
ordinary mathematics; see Section~\ref{definitions}, which discusses the
philosophy of definitions.  Section~\ref{hierarchy} discusses some
technical requirements for definitions.  In \texttt{set.mm} we adopt the
convention of prefixing axiom labels with \texttt{ax-} and definition labels with
\texttt{df-}\index{label}.

The results that can be derived with the Metamath language are only as good as
the \texttt{\$a}\index{\texttt{\$a} statement} statements used as their starting
point.  We cannot stress this too strongly.  For example, Metamath will
not prevent you from specifying $x\neq x$ as an axiom of logic.  It is
essential that you scrutinize all \texttt{\$a} statements with great care.
Because they are a source of potential pitfalls, it is best not to add new
ones (usually new definitions) casually; rather you should carefully evaluate
each one's necessity and advantages.

Once you have in place all of the basic axioms\index{axiom} and
rules\index{rule} of a mathematical theory, the only \texttt{\$a} statements that
you will be adding will be what are ordinarily called definitions.  In
principle, definitions should be in some sense eliminable from the language of
a theory according to some convention (usually involving logical equivalence
or equality).  The most common convention is that any formula that was
syntactically valid but not provable before the definition was introduced will
not become provable after the definition is introduced.  In an ideal world,
definitions should not be present at all if one is to have absolute confidence
in a mathematical result.  However, they are necessary to make
mathematics practical, for otherwise the resulting formulas would be
extremely long and incomprehensible.  Since the nature of definitions (in the
most general sense) does not permit them to automatically be verified as
``proper,''\index{proper definition}\index{definition!proper} the judgment of
the mathematician is required to ensure it.  (In \texttt{set.mm} effort was made
to make almost all definitions directly eliminable and thus minimize the need
for such judgment.)

If you are not a mathematician, it may be best not to add or change any
\texttt{\$a}\index{\texttt{\$a} statement} statements but instead use
the mathematical language already provided in standard databases.  This
way Metamath will not allow you to make a mistake (i.e.\ prove a false
result).


\subsection{Frames}\label{frames}

We now introduce the concept of a collection of related Metamath statements
called a frame.  Every assertion (\texttt{\$a} or \texttt{\$p} statement) in the database has
an associated frame.

A {\bf frame}\index{frame} is a sequence of \texttt{\$d}, \texttt{\$f},
and \texttt{\$e} statements (zero or more of each) followed by one
\texttt{\$a} or \texttt{\$p} statement, subject to certain conditions we
will describe.  For simplicity we will assume that all math symbol
tokens used are declared at the beginning of the database with
\texttt{\$c} and \texttt{\$v} statements (which are not properly part of
a frame).  Also for simplicity we will assume there are only simple
\texttt{\$d} statements (those with only two variables) and imagine any
compound \texttt{\$d} statements (those with more than two variables) as
broken up into simple ones.

A frame groups together those hypotheses (and \texttt{\$d} statements) relevant
to an assertion (\texttt{\$a} or \texttt{\$p} statement).  The statements in a frame
may or may not be physically adjacent in a database; we will cover
this in our discussion of scoping statements
in Section~\ref{scoping}.

A frame has the following properties:
\begin{enumerate}
 \item The set of variables contained in its \texttt{\$f} statements must
be identical to the set of variables contained in its \texttt{\$e},
\texttt{\$a}, and/or \texttt{\$p} statements.  In other words, each
variable in a \texttt{\$e}, \texttt{\$a}, or \texttt{\$p} statement must
have an associated ``variable type'' defined for it in a \texttt{\$f}
statement.
  \item No two \texttt{\$f} statements may contain the same variable.
  \item Any \texttt{\$f} statement
must occur before a \texttt{\$e} statement in which its variable occurs.
\end{enumerate}

The first property determines the set of variables occurring in a frame.
These are the {\bf mandatory
variables}\index{mandatory variable} of the frame.  The second property
tells us there must be only one type specified for a variable.
The last property is not a theoretical requirement but it
makes parsing of the database easier.

For our examples, we assume our database has the following declarations:

\begin{verbatim}
$v P Q R $.
$c -> ( ) |- wff $.
\end{verbatim}

The following sequence of statements, describing the modus ponens inference
rule, is an example of a frame:

\begin{verbatim}
wp  $f wff P $.
wq  $f wff Q $.
maj $e |- ( P -> Q ) $.
min $e |- P $.
mp  $a |- Q $.
\end{verbatim}

The following sequence of statements is not a frame because \texttt{R} does not
occur in the \texttt{\$e}'s or the \texttt{\$a}:

\begin{verbatim}
wp  $f wff P $.
wq  $f wff Q $.
wr  $f wff R $.
maj $e |- ( P -> Q ) $.
min $e |- P $.
mp  $a |- Q $.
\end{verbatim}

The following sequence of statements is not a frame because \texttt{Q} does not
occur in a \texttt{\$f}:

\begin{verbatim}
wp  $f wff P $.
maj $e |- ( P -> Q ) $.
min $e |- P $.
mp  $a |- Q $.
\end{verbatim}

The following sequence of statements is not a frame because the \texttt{\$a} statement is
not the last one:

\begin{verbatim}
wp  $f wff P $.
wq  $f wff Q $.
maj $e |- ( P -> Q ) $.
mp  $a |- Q $.
min $e |- P $.
\end{verbatim}

Associated with a frame is a sequence of {\bf mandatory
hypotheses}\index{mandatory hypothesis}.  This is simply the set of all
\texttt{\$f} and \texttt{\$e} statements in the frame, in the order they
appear.  A frame can be referenced in a later proof using the label of
the \texttt{\$a} or \texttt{\$p} assertion statement, and the proof
makes an assignment to each mandatory hypothesis in the order in which
it appears.  This means the order of the hypotheses, once chosen, must
not be changed so as not to affect later proofs referencing the frame's
assertion statement.  (The Metamath proof verifier will, of course, flag
an error if a proof becomes incorrect by doing this.)  Since proofs make
use of ``Reverse Polish notation,'' described in Section~\ref{proof}, we
call this order the {\bf RPN order}\index{RPN order} of the hypotheses.

Note that \texttt{\$d} statements are not part of the set of mandatory
hypotheses, and their order doesn't matter (as long as they satisfy the
fourth property for a frame described above).  The \texttt{\$d}
statements specify restrictions on variables that must be satisfied (and
are checked by the proof verifier) when expressions are substituted for
them in a proof, and the \texttt{\$d} statements themselves are never
referenced directly in a proof.

A frame with a \texttt{\$p} (provable) statement requires a proof as part of the
\texttt{\$p} statement.  Sometimes in a proof we want to make use of temporary or
dummy variables\index{dummy variable} that do not occur in the \texttt{\$p}
statement or its mandatory hypotheses.  To accommodate this we define an {\bf
extended frame}\index{extended frame} as a frame together with zero or more
\texttt{\$d} and \texttt{\$f} statements that reference variables not among the
mandatory variables of the frame.  Any new variables referenced are called the
{\bf optional variables}\index{optional variable} of the extended frame. If a
\texttt{\$f} statement references an optional variable it is called an {\bf
optional hypothesis}\index{optional hypothesis}, and if one or both of the
variables in a \texttt{\$d} statement are optional variables it is called an {\bf
optional disjoint-variable restriction}\index{optional disjoint-variable
restriction}.  Properties 2, 3, and 4 for a frame also apply to an extended
frame.

The concept of optional variables is not meaningful for frames with \texttt{\$a}
statements, since those statements have no proofs that might make use of them.
There is no restriction on including optional hypotheses in the extended frame
for a \texttt{\$a} statement, but they serve no purpose.

The following set of statements is an example of an extended frame, which
contains an optional variable \texttt{R} and an optional hypothesis \texttt{wr}.  In
this example, we suppose the rule of modus ponens is not an axiom but is
derived as a theorem from earlier statements (we omit its presumed proof).
Variable \texttt{R} may be used in its proof if desired (although this would
probably have no advantage in propositional calculus).  Note that the sequence
of mandatory hypotheses in RPN order is still \texttt{wp}, \texttt{wq}, \texttt{maj},
\texttt{min} (i.e.\ \texttt{wr} is omitted), and this sequence is still assumed
whenever the assertion \texttt{mp} is referenced in a subsequent proof.

\begin{verbatim}
wp  $f wff P $.
wq  $f wff Q $.
wr  $f wff R $.
maj $e |- ( P -> Q ) $.
min $e |- P $.
mp  $p |- Q $= ... $.
\end{verbatim}

Every frame is an extended frame, but not every extended frame is a frame, as
this example shows.  The underlying frame for an extended frame is
obtained by simply removing all statements containing optional variables.
Any proof referencing an assertion will ignore any extensions to its
frame, which means we may add or delete optional hypotheses at will without
affecting subsequent proofs.

The conceptually simplest way of organizing a Metamath database is as a
sequence of extended frames.  The scoping statements
\texttt{\$\char`\{}\index{\texttt{\$\char`\{} and \texttt{\$\char`\}}
keywords} and \texttt{\$\char`\}} can be used to delimit the start and
end of an extended frame, leading to the following possible structure for a
database.  \label{framelist}

\vskip 2ex
\setbox\startprefix=\hbox{\tt \ \ \ \ \ \ \ \ }
\setbox\contprefix=\hbox{}
\startm
\m{\mbox{(\texttt{\$v} {\em and} \texttt{\$c}\,{\em statements})}}
\endm
\startm
\m{\mbox{\texttt{\$\char`\{}}}
\endm
\startm
\m{\mbox{\texttt{\ \ } {\em extended frame}}}
\endm
\startm
\m{\mbox{\texttt{\$\char`\}}}}
\endm
\startm
\m{\mbox{\texttt{\$\char`\{}}}
\endm
\startm
\m{\mbox{\texttt{\ \ } {\em extended frame}}}
\endm
\startm
\m{\mbox{\texttt{\$\char`\}}}}
\endm
\startm
\m{\mbox{\texttt{\ \ \ \ \ \ \ \ \ }}\vdots}
\endm
\vskip 2ex

In practice, this structure is inconvenient because we have to repeat
any \texttt{\$f}, \texttt{\$e}, and \texttt{\$d} statements over and
over again rather than stating them once for use by several assertions.
The scoping statements, which we will discuss next, allow this to be
done.  In principle, any Metamath database can be converted to the above
format, and the above format is the most convenient to use when studying
a Metamath database as a formal system%
%% Uncomment this when uncommenting section {formalspec} below
   (Appendix \ref{formalspec})%
.
In fact, Metamath internally converts the database to the above format.
The command \texttt{show statement} in the Metamath program will show
you the contents of the frame for any \texttt{\$a} or \texttt{\$p}
statement, as well as its extension in the case of a \texttt{\$p}
statement.

%c%(provided that all ``local'' variables and constants with limited scope have
%c%unique names),

During our discussion of scoping statements, it may be helpful to
think in terms of the equivalent sequence of frames that will result when
the database is parsed.  Scoping (other than the limited
use above to delimit frames) is not a theoretical requirement for
Metamath but makes it more convenient.


\subsection{Scoping Statements (\texttt{\$\{} and \texttt{\$\}})}\label{scoping}
\index{\texttt{\$\char`\{} and \texttt{\$\char`\}} keywords}\index{scoping statement}

%c%Some Metamath statements may be needed only temporarily to
%c%serve a specific purpose, and after we're done with them we would like to
%c%disregard or ignore them.  For example, when we're finished using a variable,
%c%we might want to
%c%we might want to free up the token\index{token} used to name it so that the
%c%token can be used for other purposes later on, such as a different kind of
%c%variable or even a constant.  In the terminology of computer programming, we
%c%might want to let some symbol declarations be ``local'' rather than ``global.''
%c%\index{local symbol}\index{global symbol}

The {\bf scoping} statements, \texttt{\$\char`\{} ({\bf start of block}) and \texttt{\$\char`\}}
({\bf end of block})\index{block}, provide a means for controlling the portion
of a database over which certain statement types are recognized.  The
syntax of a scoping statement is very simple; it just consists of the
statement's keyword:
\begin{center}
\texttt{\$\char`\{}\\
\texttt{\$\char`\}}
\end{center}
\index{\texttt{\$\char`\{} and \texttt{\$\char`\}} keywords}

For example, consider the following database where we have stripped out
all tokens except the scoping statement keywords.  For the purpose of the
discussion, we have added subscripts to the scoping statements; these subscripts
do not appear in the actual database.
\[
 \mbox{\tt \ \$\char`\{}_1
 \mbox{\tt \ \$\char`\{}_2
 \mbox{\tt \ \$\char`\}}_2
 \mbox{\tt \ \$\char`\{}_3
 \mbox{\tt \ \$\char`\{}_4
 \mbox{\tt \ \$\char`\}}_4
 \mbox{\tt \ \$\char`\}}_3
 \mbox{\tt \ \$\char`\}}_1
\]
Each \texttt{\$\char`\{} statement in this example is said to be {\bf
matched} with the \texttt{\$\char`\}} statement that has the same
subscript.  Each pair of matched scoping statements defines a region of
the database called a {\bf block}.\index{block} Blocks can be {\bf
nested}\index{nested block} inside other blocks; in the example, the
block defined by $\mbox{\tt \$\char`\{}_4$ and $\mbox{\tt \$\char`\}}_4$
is nested inside the block defined by $\mbox{\tt \$\char`\{}_3$ and
$\mbox{\tt \$\char`\}}_3$ as well as inside the block defined by
$\mbox{\tt \$\char`\{}_1$ and $\mbox{\tt \$\char`\}}_1$.  In general, a
block may be empty, it may contain only non-scoping
statements,\footnote{Those statements other than \texttt{\$\char`\{} and
\texttt{\$\char`\}}.}\index{non-scoping statement} or it may contain any
mixture of other blocks and non-scoping statements.  (This is called a
``recursive'' definition\index{recursive definition} of a block.)

Associated with each block is a number called its {\bf nesting
level}\index{nesting level} that indicates how deeply the block is nested.
The nesting levels of the blocks in our example are as follows:
\[
  \underbrace{
    \mbox{\tt \ }
    \underbrace{
     \mbox{\tt \$\char`\{\ }
     \underbrace{
       \mbox{\tt \$\char`\{\ }
       \mbox{\tt \$\char`\}}
     }_{2}
     \mbox{\tt \ }
     \underbrace{
       \mbox{\tt \$\char`\{\ }
       \underbrace{
         \mbox{\tt \$\char`\{\ }
         \mbox{\tt \$\char`\}}
       }_{3}
       \mbox{\tt \ \$\char`\}}
     }_{2}
     \mbox{\tt \ \$\char`\}}
   }_{1}
   \mbox{\tt \ }
 }_{0}
\]
\index{\texttt{\$\char`\{} and \texttt{\$\char`\}} keywords}
The entire database is considered to be one big block (the {\bf outermost}
block) with a nesting level of 0.  The outermost block is {\em not} bracketed
by scoping statements.\footnote{The language was designed this way so that
several source files can be joined together more easily.}\index{outermost
block}

All non-scoping Metamath statements become recognized or {\bf
active}\index{active statement} at the place where they appear.\footnote{To
keep things slightly simpler, we do not bother to define the concept of
``active'' for the scoping statements.}  Certain of these statement types
become inactive at the end of the block in which they appear; these statement
types are:
\begin{center}
  \texttt{\$c}, \texttt{\$v}, \texttt{\$d}, \texttt{\$e}, and \texttt{\$f}.
%  \texttt{\$v}, \texttt{\$f}, \texttt{\$e}, and \texttt{\$d}.
\end{center}
\index{\texttt{\$c} statement}
\index{\texttt{\$d} statement}
\index{\texttt{\$e} statement}
\index{\texttt{\$f} statement}
\index{\texttt{\$v} statement}
The other statement types remain active forever (i.e.\ through the end of the
database); they are:
\begin{center}
  \texttt{\$a} and \texttt{\$p}.
%  \texttt{\$c}, \texttt{\$a}, and \texttt{\$p}.
\end{center}
\index{\texttt{\$a} statement}
\index{\texttt{\$p} statement}
Any statement (of these 7 types) located in the outermost
block\index{outermost block} will remain active through the end of the
database and thus are effectively ``global'' statements.\index{global
statement}

All \texttt{\$c} statements must be placed in the outermost block.  Since they are
therefore always global, they could be considered as belonging to both of the
above categories.

The {\bf scope}\index{scope} of a statement is the set of statements that
recognize it as active.

%c%The concept of ``active'' is also defined for math symbols\index{math
%c%symbol}.  Math symbols (constants\index{constant} and
%c%variables\index{variable}) become {\bf active}\index{active
%c%math symbol} in the \texttt{\$c}\index{\texttt{\$c}
%c%statement} and \texttt{\$v}\index{\texttt{\$v} statement} statements that
%c%declare them.  They become inactive when their declaration statements become
%c%inactive.

The concept of ``active'' is also defined for math symbols\index{math
symbol}.  Math symbols (constants\index{constant} and
variables\index{variable}) become {\bf active}\index{active math symbol}
in the \texttt{\$c}\index{\texttt{\$c} statement} and
\texttt{\$v}\index{\texttt{\$v} statement} statements that declare them.
A variable becomes inactive when its declaration statement becomes
inactive.  Because all \texttt{\$c} statements must be in the outermost
block, a constant will never become inactive after it is declared.

\subsubsection{Redeclaration of Math Symbols}
\index{redeclaration of symbols}\label{redeclaration}

%c%A math symbol may not be declared a second time while it is active, but it may
%c%be declared again after it becomes inactive.

A variable may not be declared a second time while it is active, but it may be
declared again after it becomes inactive.  This provides a convenient way to
introduce ``local'' variables,\index{local variable} i.e.\ temporary variables
for use in the frame of an assertion or in a proof without keeping them around
forever.  A previously declared variable may not be redeclared as a constant.

A constant may not be redeclared.  And, as mentioned above, constants must be
declared in the outermost block.

The reason variables may have limited scope but not constants is that an
assertion (\texttt{\$a} or \texttt{\$p} statement) remains available for use in
proofs through the end of the database.  Variables in an assertion's frame may
be substituted with whatever is needed in a proof step that references the
assertion, whereas constants remain fixed and may not be substituted with
anything.  The particular token used for a variable in an assertion's frame is
irrelevant when the assertion is referenced in a proof, and it doesn't matter
if that token is not available outside of the referenced assertion's frame.
Constants, however, must be globally fixed.

In the present version of the Metamath language, there is no theoretical
benefit for the feature allowing variables to be active for limited scopes
rather than global. It is just a convenience that allows them, for example, to
be locally grouped together with their corresponding \texttt{\$f} variable-type
declarations.\footnote{A future version\label{redeclarationf} of the Metamath
language may extend the language to include true local variables as well as
local constants.  I have not decided upon the best way to do this or even if
it should be done at all; the main issue is whether the benefits offset
greater confusion in learning the language.  In any case, the rules for
the present language were chosen to let existing databases be compatible with
any future language extensions.}

%c%If you declare a math symbol more than once, internally Metamath considers it a
%c%new distinct symbol, even though it has the same name.  If you are unaware of
%c%this, you may find that what you think are correct proofs are incorrectly
%c%rejected as invalid, because Metamath may tell you that a constant you
%c%previously declared does not match a newly declared math symbol with the same
%c%name.  For details on this subtle point, see the Comment on
%c%p.~\pageref{spec4comment}.  This is done purposely to allow temporary
%c%constants to be introduced while developing a subtheory, then allow their math
%c%symbol tokens to be reused later on; in general they will not refer to the
%c%same thing.  In practice, you would not ordinarily reuse the names of
%c%constants because it would tend to be confusing to the reader.  The reuse of
%c%names of variables, on the other hand, is something that is often useful to do
%c%(for example it is done frequently in \texttt{set.mm}).  Since variables in an
%c%assertion referenced in a proof can be substituted as needed to achieve a
%c%symbol match, this is not an issue.

% (This section covers a somewhat advanced topic you may want to skip
% at first reading.)
%
% Under certain circumstances, math symbol\index{math symbol}
% tokens\index{token} may be redeclared (i.e.\ the token
% may appear in more than
% one \texttt{\$c}\index{\texttt{\$c} statement} or \texttt{\$v}\index{\texttt{\$v}
% statement} statement).  You might want to do this say, to make temporary use
% of a variable name without having to worry about its affect elsewhere,
% somewhat analogous to declaring a local variable in a standard computer
% language.  Understanding what goes on when math symbol tokens are redeclared
% is a little tricky to understand at first, since it requires that we
% distinguish the token itself from the math symbol that it names.  It will help
% if we first take a peek at the internal workings of the
% Metamath\index{Metamath} program.
%
% Metamath reserves a memory location for each occurrence of a
% token\index{token} in a declaration statement (\texttt{\$c}\index{\texttt{\$c}
% statement} or \texttt{\$v}\index{\texttt{\$v} statement}).  If a given token appears
% in more than one declaration statement, it will refer to more than one memory
% locations.  A math symbol\index{math symbol} may be thought of as being one of
% these memory locations rather than as the token itself.  Only one of the
% memory locations associated with a given token may be active at any one time.
% The math symbol (memory location) that gets looked up when the token appears
% in a non-declaration statement is the one that happens to be active at that
% time.
%
% We now look at the rules for the redeclaration\index{redeclaration of symbols}
% of math symbol tokens.
% \begin{itemize}
% \item A math symbol token may not be declared twice in the
% same block.\footnote{While there is no theoretical reason for disallowing
% this, it was decided in the design of Metamath that allowing it would offer no
% advantage and might cause confusion.}
% \item An inactive math symbol may always be
% redeclared.
% \item  An active math symbol may be redeclared in a different (i.e.\
% inner) block\index{block} from the one it became active in.
% \end{itemize}
%
% When a math symbol token is redeclared, it conceptually refers to a different
% math symbol, just as it would be if it were called a different name.  In
% addition, the original math symbol that it referred to, if it was active,
% temporarily becomes inactive.  At the end of the block in which the
% redeclaration occurred, the new math symbol\index{math symbol} becomes
% inactive and the original symbol becomes active again.  This concept is
% illustrated in the following example, where the symbol \texttt{e} is
% ordinarily a constant (say Euler's constant, 2.71828...) but
% temporarily we want to use it as a ``local'' variable, say as a coefficient
% in the equation $a x^4 + b x^3 + c x^2 + d x + e$:
% \[
%   \mbox{\tt \$\char`\{\ \$c e \$.}
%   \underbrace{
%     \ \ldots\ %
%     \mbox{\tt \$\char`\{}\ \ldots\ %
%   }_{\mbox{\rm region A}}
%   \mbox{\tt \$v e \$.}
%   \underbrace{
%     \mbox{\ \ \ \ldots\ \ \ }
%   }_{\mbox{\rm region B}}
%   \mbox{\tt \$\char`\}}
%   \underbrace{
%     \mbox{\ \ \ \ldots\ \ \ }
%   }_{\mbox{\rm region C}}
%   \mbox{\tt \$\char`\}}
% \]
% \index{\texttt{\$\char`\{} and \texttt{\$\char`\}} keywords}
% In region A, the token \texttt{e} refers to a constant.  It is redeclared as a
% variable in region B, and any reference to it in this region will refer to this
% variable.  In region C, the redeclaration becomes inactive, and the original
% declaration becomes active again.  In region C, the token \texttt{x} refers to the
% original constant.
%
% As a practical matter, overuse of math symbol\index{math symbol}
% redeclarations\index{redeclaration of symbols} can be confusing (even though
% it is well-defined) and is best avoided when possible.  Here are some good
% general guidelines you can follow.  Usually, you should declare all
% constants\index{constant} in the outermost block\index{outermost block},
% especially if they are general-purpose (such as the token \verb$A.$, meaning
% $\forall$ or ``for all'').  This will make them ``globally'' active (although
% as in the example above local redeclarations will temporarily make them
% inactive.)  Most or all variables\index{variable}, on the other hand, could be
% declared in inner blocks, so that the token for them can be used later for a
% different type of variable or a constant.  (The names of the variables you
% choose are not used when you refer to an assertion\index{assertion} in a
% proof, whereas constants must match exactly.  A locally declared constant will
% not match a globally declared constant in a proof, even if they use the same
% token, because Metamath internally considers them to be different math
% symbols.)  To avoid confusion, you should generally avoid redeclaring active
% variables.  If you must redeclare them, do so at the beginning of a block.
% The temporary declaration of constants in inner blocks might be occasionally
% appropriate when you make use of a temporary definition to prove lemmas
% leading to a main result that does not make direct use of the definition.
% This way, you will not clutter up your database with a large number of
% seldom-used global constant symbols.  You might want to note that while
% inactive constants may not appear directly in an assertion (a \texttt{\$a}\index{\texttt{\$a}
% statement} or \texttt{\$p}\index{\texttt{\$p} statement}
% statement), they may be indirectly used in the proof of a \texttt{\$p} statement
% so long as they do not appear in the final math symbol sequence constructed by
% the proof.  In the end, you will have to use your best judgment, taking into
% account standard mathematical usage of the symbols as well as consideration
% for the reader of your work.
%
% \subsubsection{Reuse of Labels}\index{reuse of labels}\index{label}
%
% The \texttt{\$e}\index{\texttt{\$e} statement}, \texttt{\$f}\index{\texttt{\$f}
% statement}, \texttt{\$a}\index{\texttt{\$a} statement}, and
% \texttt{\$p}\index{\texttt{\$p}
% statement} statement types require labels, which allow them to be
% referenced later inside proofs.  A label is considered {\bf
% active}\index{active label} when the statement it is associated with is
% active.  The token\index{token} for a label may be reused
% (redeclared)\index{redeclaration of labels} provided that it is not being used
% for a currently active label.  (Unlike the tokens for math symbols, active
% label tokens may not be redeclared in an inner scope.)  Note that the labels
% of \texttt{\$a} and \texttt{\$p} statements can never be reused after these
% statements appear, because these statements remain active through the end of
% the database.
%
% You might find the reuse of labels a convenient way to have standard names for
% temporary hypotheses, such as \texttt{h1}, \texttt{h2}, etc.  This way you don't have
% to invent unique names for each of them, and in some cases it may be less
% confusing to the reader (although in other cases it might be more confusing, if
% the hypothesis is located far away from the assertion that uses
% it).\footnote{The current implementation requires that all labels, even
% inactive ones, be unique.}

\subsubsection{Frames Revisited}\index{frames and scoping statements}

Now that we have covered scoping, we will look at how an arbitrary
Metamath database can be converted to the simple sequence of extended
frames described on p.~\pageref{framelist}.  This is also how Metamath
stores the database internally when it reads in the database
source.\label{frameconvert} The method is simple.  First, we collect all
constant and variable (\texttt{\$c} and \texttt{\$v}) declarations in
the database, ignoring duplicate declarations of the same variable in
different scopes.  We then put our collected \texttt{\$c} and
\texttt{\$v} declarations at the beginning of the database, so that
their scope is the entire database.  Next, for each assertion in the
database, we determine its frame and extended frame.  The extended frame
is simply the \texttt{\$f}, \texttt{\$e}, and \texttt{\$d} statements
that are active.  The frame is the extended frame with all optional
hypotheses removed.

An equivalent way of saying this is that the extended frame of an assertion
is the collection of all \texttt{\$f}, \texttt{\$e}, and \texttt{\$d} statements
whose scope includes the assertion.
The \texttt{\$f} and \texttt{\$e} statements
occur in the order they appear
(order is irrelevant for \texttt{\$d} statements).

%c%, renaming any
%c%redeclared variables as needed so that all of them have unique names.  (The
%c%exact renaming convention is unimportant.  You might imagine renaming
%c%different declarations of math symbol \texttt{a} as \texttt{a\$1}, \texttt{a\$2}, etc.\
%c%which would prevent any conflicts since \texttt{\$} is not a legal character in a
%c%math symbol token.)

\section{The Anatomy of a Proof} \label{proof}
\index{proof!Metamath, description of}

Each provable assertion (\texttt{\$p}\index{\texttt{\$p} statement} statement) in a
database must include a {\bf proof}\index{proof}.  The proof is located
between the \texttt{\$=}\index{\texttt{\$=} keyword} and \texttt{\$.}\ keywords in the
\texttt{\$p} statement.

In the basic Metamath language\index{basic language}, a proof is a
sequence of statement labels.  This label sequence\index{label sequence}
serves as a set of instructions that the Metamath program uses to
construct a series of math symbol sequences.  The construction must
ultimately result in the math symbol sequence contained between the
\texttt{\$p}\index{\texttt{\$p} statement} and
\texttt{\$=}\index{\texttt{\$=} keyword} keywords of the \texttt{\$p}
statement.  Otherwise, the Metamath program will consider the proof
incorrect, and it will notify you with an appropriate error message when
you ask it to verify the proof.\footnote{To make the loading faster, the
Metamath program does not automatically verify proofs when you
\texttt{read} in a database unless you use the \texttt{/verify}
qualifier.  After a database has been read in, you may use the
\texttt{verify proof *} command to verify proofs.}\index{\texttt{verify
proof} command} Each label in a proof is said to {\bf
reference}\index{label reference} its corresponding statement.

Associated with any assertion\index{assertion} (\texttt{\$p} or
\texttt{\$a}\index{\texttt{\$a} statement} statement) is a set of
hypotheses (\texttt{\$f}\index{\texttt{\$f} statement} or
\texttt{\$e}\index{\texttt{\$e} statement} statements) that are active
with respect to that assertion.  Some are mandatory and the others are
optional.  You should review these concepts if necessary.

Each label\index{label} in a proof must be either the label of a
previous assertion (\texttt{\$a}\index{\texttt{\$a} statement} or
\texttt{\$p}\index{\texttt{\$p} statement} statement) or the label of an
active hypothesis (\texttt{\$e} or \texttt{\$f}\index{\texttt{\$f}
statement} statement) of the \texttt{\$p} statement containing the
proof.  Hypothesis labels may reference both the
mandatory\index{mandatory hypothesis} and the optional hypotheses of the
\texttt{\$p} statement.

The label sequence in a proof specifies a construction in {\bf reverse Polish
notation}\index{reverse Polish notation (RPN)} (RPN).  You may be familiar
with RPN if you have used older
Hewlett--Packard or similar hand-held calculators.
In the calculator analogy, a hypothesis label\index{hypothesis label} is like
a number and an assertion label\index{assertion label} is like an operation
(more precisely, an $n$-ary operation when the
assertion has $n$ \texttt{\$e}-hypotheses).
On an RPN calculator, an operation takes one or more previous numbers in an
input sequence, performs a calculation on them, and replaces those numbers and
itself with the result of the calculation.  For example, the input sequence
$2,3,+$ on an RPN calculator results in $5$, and the input sequence
$2,3,5,{\times},+$ results in $2,15,+$ which results in $17$.

Understanding how RPN is processed involves the concept of a {\bf
stack}\index{stack}\index{RPN stack}, which can be thought of as a set of
temporary memory locations that hold intermediate results.  When Metamath
encounters a hypothesis label it places or {\bf pushes}\index{push} the math
symbol sequence of the hypothesis onto the stack.  When Metamath encounters an
assertion label, it associates the most recent stack entries with the {\em
mandatory} hypotheses\index{mandatory hypothesis} of the assertion, in the
order where the most recent stack entry is associated with the last mandatory
hypothesis of the assertion.  It then determines what
substitutions\index{substitution!variable}\index{variable substitution} have
to be made into the variables of the assertion's mandatory hypotheses to make
them identical to the associated stack entries.  It then makes those same
substitutions into the assertion itself.  Finally, Metamath removes or {\bf
pops}\index{pop} the matched hypotheses from the stack and pushes the
substituted assertion onto the stack.

For the purpose of matching the mandatory hypothesis to the most recent stack
entries, whether a hypothesis is a \texttt{\$e} or \texttt{\$f} statement is
irrelevant.  The only important thing is that a set of
substitutions\footnote{In the Metamath spec (Section~\ref{spec}), we use the
singular term ``substitution'' to refer to the set of substitutions we talk
about here.} exist that allow a match (and if they don't, the proof verifier
will let you know with an error message).  The Metamath language is specified
in such a way that if a set of substitutions exists, it will be unique.
Specifically, the requirement that each variable have a type specified for it
with a \texttt{\$f} statement ensures the uniqueness.

We will illustrate this with an example.
Consider the following Metamath source file:
\begin{verbatim}
$c ( ) -> wff $.
$v p q r s $.
wp $f wff p $.
wq $f wff q $.
wr $f wff r $.
ws $f wff s $.
w2 $a wff ( p -> q ) $.
wnew $p wff ( s -> ( r -> p ) ) $= ws wr wp w2 w2 $.
\end{verbatim}
This Metamath source example shows the definition and ``proof'' (i.e.,
construction) of a well-formed formula (wff)\index{well-formed formula (wff)}
in propositional calculus.  (You may wish to type this example into a file to
experiment with the Metamath program.)  The first two statements declare
(introduce the names of) four constants and four variables.  The next four
statements specify the variable types, namely that
each variable is assumed to be a wff.  Statement \texttt{w2} defines (postulates)
a way to produce a new wff, \texttt{( p -> q )}, from two given wffs \texttt{p} and
\texttt{q}. The mandatory hypotheses of \texttt{w2} are \texttt{wp} and \texttt{wq}.
Statement \texttt{wnew} claims that \texttt{( s -> ( r -> p ) )} is a wff given
three wffs \texttt{s}, \texttt{r}, and \texttt{p}.  More precisely, \texttt{wnew} claims
that the sequence of ten symbols \texttt{wff ( s -> ( r -> p ) )} is provable from
previous assertions and the hypotheses of \texttt{wnew}.  Metamath does not know
or care what a wff is, and as far as it is concerned
the typecode \texttt{wff} is just an
arbitrary constant symbol in a math symbol sequence.  The mandatory hypotheses
of \texttt{wnew} are \texttt{wp}, \texttt{wr}, and \texttt{ws}; \texttt{wq} is an optional
hypothesis.  In our particular proof, the optional hypothesis is not
referenced, but in general, any combination of active (i.e.\ optional and
mandatory) hypotheses could be referenced.  The proof of statement \texttt{wnew}
is the sequence of five labels starting with \texttt{ws} (step~1) and ending with
\texttt{w2} (step~5).

When Metamath verifies the proof, it scans the proof from left to right.  We
will examine what happens at each step of the proof.  The stack starts off
empty.  At step 1, Metamath looks up label \texttt{ws} and determines that it is a
hypothesis, so it pushes the symbol sequence of statement \texttt{ws} onto the
stack:

\begin{center}\begin{tabular}{|l|l|}\hline
{Stack location} & {Contents} \\ \hline \hline
1 & \texttt{wff s} \\ \hline
\end{tabular}\end{center}

Metamath sees that the labels \texttt{wr} and \texttt{wp} in steps~2 and 3 are also
hypotheses, so it pushes them onto the stack.  After step~3, the stack looks
like
this:

\begin{center}\begin{tabular}{|l|l|}\hline
{Stack location} & {Contents} \\ \hline \hline
3 & \texttt{wff p} \\ \hline
2 & \texttt{wff r} \\ \hline
1 & \texttt{wff s} \\ \hline
\end{tabular}\end{center}

At step 4, Metamath sees that label \texttt{w2} is an assertion, so it must do
some processing.  First, it associates the mandatory hypotheses of \texttt{w2},
which are \texttt{wp} and \texttt{wq}, with stack locations~2 and 3, {\em in that
order}. Metamath determines that the only possible way
to make hypothesis \texttt{wp} match (become identical to) stack location~2 and
\texttt{wq} match stack location 3 is to substitute variable \texttt{p} with \texttt{r}
and \texttt{q} with \texttt{p}.  Metamath makes these substitutions into \texttt{w2} and
obtains the symbol sequence \texttt{wff ( r -> p )}.  It removes the hypotheses
from stack locations~2 and 3, then places the result into stack location~2:

\begin{center}\begin{tabular}{|l|l|}\hline
{Stack location} & {Contents} \\ \hline \hline
2 & \texttt{wff ( r -> p )} \\ \hline
1 & \texttt{wff s} \\ \hline
\end{tabular}\end{center}

At step 5, Metamath sees that label \texttt{w2} is an assertion, so it must again
do some processing.  First, it matches the mandatory hypotheses of \texttt{w2},
which are \texttt{wp} and \texttt{wq}, to stack locations 1 and 2.
Metamath determines that the only possible way to make the
hypotheses match is to substitute variable \texttt{p} with \texttt{s} and \texttt{q} with
\texttt{( r -> p )}.  Metamath makes these substitutions into \texttt{w2} and obtains
the symbol
sequence \texttt{wff ( s -> ( r -> p ) )}.  It removes stack
locations 1 and 2, then places the result into stack location~1:

\begin{center}\begin{tabular}{|l|l|}\hline
{Stack location} & {Contents} \\ \hline \hline
1 & \texttt{wff ( s -> ( r -> p ) )} \\ \hline
\end{tabular}\end{center}

After Metamath finishes processing the proof, it checks to see that the
stack contains exactly one element and that this element is
the same as the math symbol sequence in the
\texttt{\$p}\index{\texttt{\$p} statement} statement.  This is the case for our
proof of \texttt{wnew},
so we have proved \texttt{wnew} successfully.  If the result
differs, Metamath will notify you with an error message.  An error message
will also result if the stack contains more than one entry at the end of the
proof, or if the stack did not contain enough entries at any point in the
proof to match all of the mandatory hypotheses\index{mandatory hypothesis} of
an assertion.  Finally, Metamath will notify you with an error message if no
substitution is possible that will make a referenced assertion's hypothesis
match the
stack entries.  You may want to experiment with the different kinds of errors
that Metamath will detect by making some small changes in the proof of our
example.

Metamath's proof notation was designed primarily to express proofs in a
relatively compact manner, not for readability by humans.  Metamath can display
proofs in a number of different ways with the \texttt{show proof}\index{\texttt{show
proof} command} command.  The
\texttt{/lemmon} qualifier displays it in a format that is easier to read when the
proofs are short, and you saw examples of its use in Chapter~\ref{using}.  For
longer proofs, it is useful to see the tree structure of the proof.  A tree
structure is displayed when the \texttt{/lemmon} qualifier is omitted.  You will
probably find this display more convenient as you get used to it. The tree
display of the proof in our example looks like
this:\label{treeproof}\index{tree-style proof}\index{proof!tree-style}
\begin{verbatim}
1     wp=ws    $f wff s
2        wp=wr    $f wff r
3        wq=wp    $f wff p
4     wq=w2    $a wff ( r -> p )
5  wnew=w2  $a wff ( s -> ( r -> p ) )
\end{verbatim}
The number to the left of each line is the step number.  Following it is a
{\bf hypothesis association}\index{hypothesis association}, consisting of two
labels\index{label} separated by \texttt{=}.  To the left of the \texttt{=} (except
in the last step) is the label of a hypothesis of an assertion referenced
later in the proof; here, steps 1 and 4 are the hypothesis associations for
the assertion \texttt{w2} that is referenced in step 5.  A hypothesis association
is indented one level more than the assertion that uses it, so it is easy to
find the corresponding assertion by moving directly down until the indentation
level decreases to one less than where you started from.  To the right of each
\texttt{=} is the proof step label for that proof step.  The statement keyword of
the proof step label is listed next, followed by the content of the top of the
stack (the most recent stack entry) as it exists after that proof step is
processed.  With a little practice, you should have no trouble reading proofs
displayed in this format.

Metamath proofs include the syntax construction of a formula.
In standard mathematics, this kind of
construction is not considered a proper part of the proof at all, and it
certainly becomes rather boring after a while.
Therefore,
by default the \texttt{show proof}\index{\texttt{show proof}
command} command does not show the syntax construction.
Historically \texttt{show proof} command
\textit{did} show the syntax construction, and you needed to add the
\texttt{/essential} option to hide, them, but today
\texttt{/essential} is the default and you need to use
\texttt{/all} to see the syntax constructions.

When verifying a proof, Metamath will check that no mandatory
\texttt{\$d}\index{\texttt{\$d} statement}\index{mandatory \texttt{\$d}
statement} statement of an assertion referenced in a proof is violated
when substitutions\index{substitution!variable}\index{variable
substitution} are made to the variables in the assertion.  For details
see Section~\ref{spec4} or \ref{dollard}.

\subsection{The Concept of Unification} \label{unify}

During the course of verifying a proof, when Metamath\index{Metamath}
encounters an assertion label\index{assertion label}, it associates the
mandatory hypotheses\index{mandatory hypothesis} of the assertion with the top
entries of the RPN stack\index{stack}\index{RPN stack}.  Metamath then
determines what substitutions\index{substitution!variable}\index{variable
substitution} it must make to the variables in the assertion's mandatory
hypotheses in order for these hypotheses to become identical to their
corresponding stack entries.  This process is called {\bf
unification}\index{unification}.  (We also informally use the term
``unification'' to refer to a set of substitutions that results from the
process, as in ``two unifications are possible.'')  After the substitutions
are made, the hypotheses are said to be {\bf unified}.

If no such substitutions are possible, Metamath will consider the proof
incorrect and notify you with an error message.
% (deleted 3/10/07, per suggestion of Mel O'Cat:)
% The syntax of the
% Metamath language ensures that if a set of substitutions exists, it
% will be unique.

The general algorithm for unification described in the literature is
somewhat complex.
However, in the case of Metamath it is intentionally trivial.
Mandatory hypotheses must be
pushed on the proof stack in the order in which they appear.
In addition, each variable must have its type specified
with a \texttt{\$f} hypothesis before it is used
and that each \texttt{\$f} hypothesis
have the restricted syntax of a typecode (a constant) followed by a variable.
The typecode in the \texttt{\$f} hypothesis must match the first symbol of
the corresponding RPN stack entry (which will also be a constant), so
the only possible match for the variable in the \texttt{\$f} hypothesis is
the sequence of symbols in the stack entry after the initial constant.

In the Proof Assistant\index{Proof Assistant}, a more general unification
algorithm is used.  While a proof is being developed, sometimes not enough
information is available to determine a unique unification.  In this case
Metamath will ask you to pick the correct one.\index{ambiguous
unification}\index{unification!ambiguous}

\section{Extensions to the Metamath Language}\index{extended
language}

\subsection{Comments in the Metamath Language}\label{comments}
\index{markup notation}
\index{comments!markup notation}

The commenting feature allows you to annotate the contents of
a database.  Just as with most
computer languages, comments are ignored for the purpose of interpreting the
contents of the database. Comments effectively act as
additional white space\index{white
space} between tokens
when a database is parsed.

A comment may be placed at the beginning, end, or
between any two tokens\index{token} in a source file.

Comments have the following syntax:
\begin{center}
 \texttt{\$(} {\em text} \texttt{\$)}
\end{center}
Here,\index{\texttt{\$(} and \texttt{\$)} auxiliary
keywords}\index{comment} {\em text} is a string, possibly empty, of any
characters in Metamath's character set (p.~\pageref{spec1chars}), except
that the character strings \texttt{\$(} and \texttt{\$)} may not appear
in {\em text}.  Thus nested comments are not
permitted:\footnote{Computer languages have differing standards for
nested comments, and rather than picking one it was felt simplest not to
allow them at all, at least in the current version of
Metamath\index{Metamath!limitations of version 0.07.30}.} Metamath will
complain if you give it
\begin{center}
 \texttt{\$( This is a \$( nested \$) comment.\ \$)}
\end{center}
To compensate for this non-nesting behavior, I often change all \texttt{\$}'s
to \texttt{@}'s in sections of Metamath code I wish to comment out.

The Metamath program supports a number of markup mechanisms and conventions
to generate good-looking results in \LaTeX\ and {\sc html},
as discussed below.
These markup features have to do only with how the comments are typeset,
and have no effect on how Metamath verifies the proofs in the database.
The improper
use of them may result in incorrectly typeset output, but no Metamath
error messages will result during the \texttt{read} and \texttt{verify
proof} commands.  (However, the \texttt{write
theorem\texttt{\char`\_}list} command
will check for markup errors as a side-effect of its
{\sc html} generation.)
Section~\ref{texout} has instructions for creating \LaTeX\ output, and
section~\ref{htmlout} has instructions for creating
{\sc html}\index{HTML} output.

\subsubsection{Headings}\label{commentheadings}

If the \texttt{\$(} is immediately followed by a new line
starting with a heading marker, it is a header.
This can start with:

\begin{itemize}
 \item[] \texttt{\#\#\#\#} - major part header
 \item[] \texttt{\#*\#*} - section header
 \item[] \texttt{=-=-} - subsection header
 \item[] \texttt{-.-.} - subsubsection header
\end{itemize}

The line following the marker line
will be used for the table of contents entry, after trimming spaces.
The next line should be another (closing) matching marker line.
Any text after that
but before the closing \texttt{\$}, such as an extended description of the
section, will be included on the \texttt{mmtheoremsNNN.html} page.

For more information, run
\texttt{help write theorem\char`\_list}.

\subsubsection{Math mode}
\label{mathcomments}
\index{\texttt{`} inside comments}
\index{\texttt{\char`\~} inside comments}
\index{math mode}

Inside of comments, a string of tokens\index{token} enclosed in
grave accents\index{grave accent (\texttt{`})} (\texttt{`}) will be converted
to standard mathematical symbols during
{\sc HTML}\index{HTML} or \LaTeX\ output
typesetting,\index{latex@{\LaTeX}} according to the information in the
special \texttt{\$t}\index{\texttt{\$t} comment}\index{typesetting
comment} comment in the database
(see section~\ref{tcomment} for information about the typesetting
comment, and Appendix~\ref{ASCII} to see examples of its results).

The first grave accent\index{grave accent (\texttt{`})} \texttt{`}
causes the output processor to enter {\bf math mode}\index{math mode}
and the second one exits it.
In this
mode, the characters following the \texttt{`} are interpreted as a
sequence of math symbol tokens separated by white space\index{white
space}.  The tokens are looked up in the \texttt{\$t}
comment\index{\texttt{\$t} comment}\index{typesetting comment} and if
found, they will be replaced by the standard mathematical symbols that
they correspond to before being placed in the typeset output file.  If
not found, the symbol will be output as is and a warning will be issued.
The tokens do not have to be active in the database, although a warning
will be issued if they are not declared with \texttt{\$c} or
\texttt{\$v} statements.

Two consecutive
grave accents \texttt{``} are treated as a single actual grave accent
(both inside and outside of math mode) and will not cause the output
processor to enter or exit math mode.

Here is an example of its use\index{Pierce's axiom}:
\begin{center}
\texttt{\$( Pierce's axiom, ` ( ( ph -> ps ) -> ph ) -> ph ` ,\\
         is not very intuitive. \$)}
\end{center}
becomes
\begin{center}
   \texttt{\$(} Pierce's axiom, $((\varphi \rightarrow \psi)\rightarrow
\varphi)\rightarrow \varphi$, is not very intuitive. \texttt{\$)}
\end{center}

Note that the math symbol tokens\index{token} must be surrounded by white
space\index{white space}.
%, since there is no context that allows ambiguity to be
%resolved, as is the case with math symbol sequences in some of the Metamath
%statements.
White space should also surround the \texttt{`}
delimiters.

The math mode feature also gives you a quick and easy way to generate
text containing mathematical symbols, independently of the intended
purpose of Metamath.\index{Metamath!using as a math editor} To do this,
simply create your text with grave accents surrounding your formulas,
after making sure that your math symbols are mapped to \LaTeX\ symbols
as described in Appendix~\ref{ASCII}.  It is easier if you start with a
database with predefined symbols such as \texttt{set.mm}.  Use your
grave-quoted math string to replace an existing comment, then typeset
the statement corresponding to that comment following the instructions
from the \texttt{help tex} command in the Metamath program.  You will
then probably want to edit the resulting file with a text editor to fine
tune it to your exact needs.

\subsubsection{Label Mode}\index{label mode}

Outside of math mode, a tilde\index{tilde (\texttt{\char`\~})} \verb/~/
indicates to Metamath's\index{Metamath} output processor that the
token\index{token} that follows (i.e.\ the characters up to the next
white space\index{white space}) represents a statement label or URL.
This formatting mode is called {\bf label mode}\index{label mode}.
If a literal tilde
is desired (outside of math mode) instead of label mode,
use two tildes in a row to represent it.

When generating a \LaTeX\ output file,
the following token will be formatted in \texttt{typewriter}
font, and the tilde removed, to make it stand out from the rest of the text.
This formatting will be applied to all characters after the
tilde up to the first white space\index{white space}.
Whether
or not the token is an actual statement label is not checked, and the
token does not have to have the correct syntax for a label; no error
messages will be produced.  The only effect of the label mode on the
output is that typewriter font will be used for the tokens that are
placed in the \LaTeX\ output file.

When generating {\sc html},
the tokens after the tilde {\em must} be a URL (either http: or https:)
or a valid label.
Error messages will be issued during that output if they aren't.
A hyperlink will be generated to that URL or label.

\subsubsection{Link to bibliographical reference}\index{citation}%
\index{link to bibliographical reference}

Bibliographical references are handled specially when generating
{\sc html} if formatted specially.
Text in the form \texttt{[}{\em author}\texttt{]}
is considered a link to a bibliographical reference.
See \texttt{help html} and \texttt{help write
bibliography} in the Metamath program for more
information.
% \index{\texttt{\char`\[}\ldots\texttt{]} inside comments}
See also Sections~\ref{tcomment} and \ref{wrbib}.

The \texttt{[}{\em author}\texttt{]} notation will also create an entry in
the bibliography cross-reference file generated by \texttt{write
bibliography} (Section~\ref{wrbib}) for {\sc HTML}.
For this to work properly, the
surrounding comment must be formatted as follows:
\begin{quote}
    {\em keyword} {\em label} {\em noise-word}
     \texttt{[}{\em author}\texttt{] p.} {\em number}
\end{quote}
for example
\begin{verbatim}
     Theorem 5.2 of [Monk] p. 223
\end{verbatim}
The {\em keyword} is not case sensitive and must be one of the following:
\begin{verbatim}
     theorem lemma definition compare proposition corollary
     axiom rule remark exercise problem notation example
     property figure postulate equation scheme chapter
\end{verbatim}
The optional {\em label} may consist of more than one
(non-{\em keyword} and non-{\em noise-word}) word.
The optional {\em noise-word} is one of:
\begin{verbatim}
     of in from on
\end{verbatim}
and is  ignored when the cross-reference file is created.  The
\texttt{write
biblio\-graphy} command will perform error checking to verify the
above format.\index{error checking}

\subsubsection{Parentheticals}\label{parentheticals}

The end of a comment may include one or more parenthicals, that is,
statements enclosed in parentheses.
The Metamath program looks for certain parentheticals and can issue
warnings based on them.
They are:

\begin{itemize}
 \item[] \texttt{(Contributed by }
   \textit{NAME}\texttt{,} \textit{DATE}\texttt{.)} -
   document the original contributor's name and the date it was created.
 \item[] \texttt{(Revised by }
   \textit{NAME}\texttt{,} \textit{DATE}\texttt{.)} -
   document the contributor's name and creation date
   that resulted in significant revision
   (not just an automated minimization or shortening).
 \item[] \texttt{(Proof shortened by }
   \textit{NAME}\texttt{,} \textit{DATE}\texttt{.)} -
   document the contributor's name and date that developed a significant
   shortening of the proof (not just an automated minimization).
 \item[] \texttt{(Proof modification is discouraged.)} -
   Note that this proof should normally not be modified.
 \item[] \texttt{(New usage is discouraged.)} -
   Note that this assertion should normally not be used.
\end{itemize}

The \textit{DATE} must be in form YYYY-MMM-DD, where MMM is the
English abbreviation of that month.

\subsubsection{Other markup}\label{othermarkup}
\index{markup notation}

There are other markup notations for generating good-looking results
beyond math mode and label mode:

\begin{itemize}
 \item[]
         \texttt{\char`\_} (underscore)\index{\texttt{\char`\_} inside comments} -
             Italicize text starting from
              {\em space}\texttt{\char`\_}{\em non-space} (i.e.\ \texttt{\char`\_}
              with a space before it and a non-space character after it) until
             the next
             {\em non-space}\texttt{\char`\_}{\em space}.  Normal
             punctuation (e.g.\ a trailing
             comma or period) is ignored when determining {\em space}.
 \item[]
         \texttt{\char`\_} (underscore) - {\em
         non-space}\texttt{\char`\_}{\em non-space-string}, where
          {\em non-space-string} is a string of non-space characters,
         will make {\em non-space-string} become a subscript.
 \item[]
         \texttt{<HTML>}...\texttt{</HTML>} - do not convert
         ``\texttt{<}'' and ``\texttt{>}''
         in the enclosed text when generating {\sc HTML},
         otherwise process markup normally. This allows direct insertion
         of {\sc html} commands.
 \item[]
       ``\texttt{\&}ref\texttt{;}'' - insert an {\sc HTML}
         character reference.
         This is how to insert arbitrary Unicode characters
         (such as accented characters).  Currently only directly supported
         when generating {\sc HTML}.
\end{itemize}

It is recommended that spaces surround any \texttt{\char`\~} and
\texttt{`} tokens in the comment and that a space follow the {\em label}
after a \texttt{\char`\~} token.  This will make global substitutions
to change labels and symbol names much easier and also eliminate any
future chance of ambiguity.  Spaces around these tokens are automatically
removed in the final output to conform with normal rules of punctuation;
for example, a space between a trailing \texttt{`} and a left parenthesis
will be removed.

A good way to become familiar with the markup notation is to look at
the extensive examples in the \texttt{set.mm} database.

\subsection{The Typesetting Comment (\texttt{\$t})}\label{tcomment}

The typesetting comment \texttt{\$t} in the input database file
provides the information necessary to produce good-looking results.
It provides \LaTeX\ and {\sc html}
definitions for math symbols,
as well supporting as some
customization of the generated web page.
If you add a new token to a database, you should also
update the \texttt{\$t} comment information if you want to eventually
create output in \LaTeX\ or {\sc HTML}.
See the
\texttt{set.mm}\index{set theory database (\texttt{set.mm})} database
file for an extensive example of a \texttt{\$t} comment illustrating
many of the features described below.

Programs that do not need to generate good-looking presentation results,
such as programs that only verify Metamath databases,
can completely ignore typesetting comments
and just treat them as normal comments.
Even the Metamath program only consults the
\texttt{\$t} comment information when it needs to generate typeset output
in \LaTeX\ or {\sc HTML}
(e.g., when you open a \LaTeX\ output file with the \texttt{open tex} command).

We will first discuss the syntax of typesetting comments, and then
briefly discuss how this can be used within the Metamath program.

\subsubsection{Typesetting Comment Syntax Overview}

The typesetting comment is identified by the token
\texttt{\$t}\index{\texttt{\$t} comment}\index{typesetting comment} in
the comment, and the typesetting comment ends at the matching
\texttt{\$)}:
\[
  \mbox{\tt \$(\ }
  \mbox{\tt \$t\ }
  \underbrace{
    \mbox{\tt \ \ \ \ \ \ \ \ \ \ \ }
    \cdots
    \mbox{\tt \ \ \ \ \ \ \ \ \ \ \ }
  }_{\mbox{Typesetting definitions go here}}
  \mbox{\tt \ \$)}
\]

There must be one or more white space characters, and only white space
characters, between the \texttt{\$(} that starts the comment
and the \texttt{\$t} symbol,
and the \texttt{\$t} must be followed by one
or more white space characters
(see section \ref{whitespace} for the definition of white space characters).
The typesetting comment continues until the comment end token \texttt{\$)}
(which must be preceded by one or more white space characters).

In version 0.07.30\index{Metamath!limitations of version 0.07.30} of the
Metamath program, there may be only one \texttt{\$t} comment in a
database.  This restriction may be lifted in the future to allow
many \texttt{\$t} comments in a database.

Between the \texttt{\$t} symbol (and its following white space) and the
comment end token \texttt{\$)} (and its preceding white space)
is a sequence of one or more typesetting definitions, where
each definition has the form
\textit{definition-type arg arg ... ;}.
Each of the zero or more \textit{arg} values
can be either a typesetting data or a keyword
(what keywords are allowed, and where, depends on the specific
\textit{definition-type}).
The \textit{definition-type}, and each argument \textit{arg},
are separated by one or more white space characters.
Every definition ends in an unquoted semicolon;
white space is not required before the terminating semicolon of a definition.
Each definition should start on a new line.\footnote{This
restriction of the current version of Metamath
(0.07.30)\index{Metamath!limitations of version 0.07.30} may be removed
in a future version, but you should do it anyway for readability.}

For example, this typesetting definition:
\begin{center}
 \verb$latexdef "C_" as "\subseteq";$
\end{center}
defines the token \verb$C_$ as the \LaTeX\ symbol $\subseteq$ (which means
``subset'').

Typesetting data is a sequence of one or more quoted strings
(if there is more than one, they are connected by \texttt{\char`\+}).
Often a single quoted string is used to provide data for a definition, using
either double (\texttt{\char`\"}) or single (\texttt{'}) quotation marks.
However,
{\em a quoted string (enclosed in quotation marks) may not include
line breaks.}
A quoted string
may include a quotation mark that matches the enclosing quotes by repeating
the quotation mark twice.  Here are some examples:

\begin{tabu}   { l l }
\textbf{Example} & \textbf{Meaning} \\
\texttt{\char`\"a\char`\"\char`\"b\char`\"} & \texttt{a\char`\"b} \\
\texttt{'c''d'} & \texttt{c'd} \\
\texttt{\char`\"e''f\char`\"} & \texttt{e''f} \\
\texttt{'g\char`\"\char`\"h'} & \texttt{g\char`\"\char`\"h} \\
\end{tabu}

Finally, a long quoted string
may be broken up into multiple quoted strings (considered, as a whole,
a single quoted string) and joined with \texttt{\char`\+}.
You can even use multiple lines as long as a
'+' is at the end of every line except the last one.
The \texttt{\char`\+} should be preceded and followed by at least one
white space character.
Thus, for example,
\begin{center}
 \texttt{\char`\"ab\char`\"\ \char`\+\ \char`\"cd\char`\"
    \ \char`\+\ \\ 'ef'}
\end{center}
is the same as
\begin{center}
 \texttt{\char`\"abcdef\char`\"}
\end{center}

{\sc c}-style comments \texttt{/*}\ldots\texttt{*/} are also supported.

In practice, whenever you add a new math token you will often want to add
typesetting definitions using
\texttt{latexdef}, \texttt{htmldef}, and
\texttt{althtmldef}, as described below.
That way, they will all be up to date.
Of course, whether or not you want to use all three definitions will
depend on how the database is intended to be used.

Below we discuss the different possible \textit{definition-kind} options.
We will show data surrounded by double quotes (in practice they can also use
single quotes and/or be a sequence joined by \texttt{+}s).
We will use specific names for the \textit{data} to make clear what
the data is used for, such as
{\em math-token} (for a Metamath math token,
{\em latex-string} (for string to be placed in a \LaTeX\ stream),
{\em {\sc html}-code} (for {\sc html} code),
and {\em filename} (for a filename).

\subsubsection{Typesetting Comment - \LaTeX}

The syntax for a \LaTeX\ definition is:
\begin{center}
 \texttt{latexdef "}{\em math-token}\texttt{" as "}{\em latex-string}\texttt{";}
\end{center}
\index{latex definitions@\LaTeX\ definitions}%
\index{\texttt{latexdef} statement}

The {\em token-string} and {\em latex-string} are the data
(character strings) for
the token and the \LaTeX\ definition of the token, respectively,

These \LaTeX\ definitions are used by the Metamath program
when it is asked to product \LaTeX output using
the \texttt{write tex} command.

\subsubsection{Typesetting Comment - {\sc html}}

The key kinds of {\sc HTML} definitions have the following syntax:

\vskip 1ex
    \texttt{htmldef "}{\em math-token}\texttt{" as "}{\em
    {\sc html}-code}\texttt{";}\index{\texttt{htmldef} statement}
                    \ \ \ \ \ \ldots

    \texttt{althtmldef "}{\em math-token}\texttt{" as "}{\em
{\sc html}-code}\texttt{";}\index{\texttt{althtmldef} statement}

                    \ \ \ \ \ \ldots

Note that in {\sc HTML} there are two possible definitions for math tokens.
This feature is useful when
an alternate representation of symbols is desired, for example one that
uses Unicode entities and another uses {\sc gif} images.

There are many other typesetting definitions that can control {\sc HTML}.
These include:

\vskip 1ex

    \texttt{htmldef "}{\em math-token}\texttt{" as "}{\em {\sc
    html}-code}\texttt{";}

    \texttt{htmltitle "}{\em {\sc html}-code}\texttt{";}%
\index{\texttt{htmltitle} statement}

    \texttt{htmlhome "}{\em {\sc html}-code}\texttt{";}%
\index{\texttt{htmlhome} statement}

    \texttt{htmlvarcolor "}{\em {\sc html}-code}\texttt{";}%
\index{\texttt{htmlvarcolor} statement}

    \texttt{htmlbibliography "}{\em filename}\texttt{";}%
\index{\texttt{htmlbibliography} statement}

\vskip 1ex

\noindent The \texttt{htmltitle} is the {\sc html} code for a common
title, such as ``Metamath Proof Explorer.''  The \texttt{htmlhome} is
code for a link back to the home page.  The \texttt{htmlvarcolor} is
code for a color key that appears at the bottom of each proof.  The file
specified by {\em filename} is an {\sc html} file that is assumed to
have a \texttt{<A NAME=}\ldots\texttt{>} tag for each bibiographic
reference in the database comments.  For example, if
\texttt{[Monk]}\index{\texttt{\char`\[}\ldots\texttt{]} inside comments}
occurs in the comment for a theorem, then \texttt{<A NAME='Monk'>} must
be present in the file; if not, a warning message is given.

Associated with
\texttt{althtmldef}
are the statements
\vskip 1ex

    \texttt{htmldir "}{\em
      directoryname}\texttt{";}\index{\texttt{htmldir} statement}

    \texttt{althtmldir "}{\em
     directoryname}\texttt{";}\index{\texttt{althtmldir} statement}

\vskip 1ex
\noindent giving the directories of the {\sc gif} and Unicode versions
respectively; their purpose is to provide cross-linking between the
two versions in the generated web pages.

When two different types of pages need to be produced from a single
database, such as the Hilbert Space Explorer that extends the Metamath
Proof Explorer, ``extended'' variables may be declared in the
\texttt{\$t} comment:
\vskip 1ex

    \texttt{exthtmltitle "}{\em {\sc html}-code}\texttt{";}%
\index{\texttt{exthtmltitle} statement}

    \texttt{exthtmlhome "}{\em {\sc html}-code}\texttt{";}%
\index{\texttt{exthtmlhome} statement}

    \texttt{exthtmlbibliography "}{\em filename}\texttt{";}%
\index{\texttt{exthtmlbibliography} statement}

\vskip 1ex
\noindent When these are declared, you also must declare
\vskip 1ex

    \texttt{exthtmllabel "}{\em label}\texttt{";}%
\index{\texttt{exthtmllabel} statement}

\vskip 1ex \noindent that identifies the database statement where the
``extended'' section of the database starts (in our example, where the
Hilbert Space Explorer starts).  During the generation of web pages for
that starting statement and the statements after it, the {\sc html} code
assigned to \texttt{exthtmltitle} and \texttt{exthtmlhome} is used
instead of that assigned to \texttt{htmltitle} and \texttt{htmlhome},
respectively.

\begin{sloppy}
\subsection{Additional Information Com\-ment (\texttt{\$j})} \label{jcomment}
\end{sloppy}

The additional information comment, aka the
\texttt{\$j}\index{\texttt{\$j} comment}\index{additional information comment}
comment,
provides a way to add additional structured information that can
be optionally parsed by systems.

The additional information comment is parsed the same way as the
typesetting comment (\texttt{\$t}) (see section \ref{tcomment}).
That is,
the additional information comment begins with the token
\texttt{\$j} within a comment,
and continues until the comment close \texttt{\$)}.
Within an additional information comment is a sequence of one or more
commands of the form \texttt{command arg arg ... ;}
where each of the zero or more \texttt{arg} values
can be either a quoted string or a keyword.
Note that every command ends in an unquoted semicolon.
If a verifier is parsing an additional information comment, but
doesn't recognize a particular command, it must skip the command
by finding the end of the command (an unquoted semicolon).

A database may have 0 or more additional information comments.
Note, however, that a verifier may ignore these comments entirely or only
process certain commands in an additional information comment.
The \texttt{mmj2} verifier supports many commands in additional information
comments.
We encourage systems that process additional information comments
to coordinate so that they will use the same command for the same effect.

Examples of additional information comments with various commands
(from the \texttt{set.mm} database) are:

\begin{itemize}
   \item Define the syntax and logical typecodes,
     and declare that our grammar is
     unambiguous (verifiable using the KLR parser, with compositing depth 5).
\begin{verbatim}
  $( $j
    syntax 'wff';
    syntax '|-' as 'wff';
    unambiguous 'klr 5';
  $)
\end{verbatim}

   \item Register $\lnot$ and $\rightarrow$ as primitive expressions
           (lacking definitions).
\begin{verbatim}
  $( $j primitive 'wn' 'wi'; $)
\end{verbatim}

   \item There is a special justification for \texttt{df-bi}.
\begin{verbatim}
  $( $j justification 'bijust' for 'df-bi'; $)
\end{verbatim}

   \item Register $\leftrightarrow$ as an equality for its type (wff).
\begin{verbatim}
  $( $j
    equality 'wb' from 'biid' 'bicomi' 'bitri';
    definition 'dfbi1' for 'wb';
  $)
\end{verbatim}

   \item Theorem \texttt{notbii} is the congruence law for negation.
\begin{verbatim}
  $( $j congruence 'notbii'; $)
\end{verbatim}

   \item Add \texttt{setvar} as a typecode.
\begin{verbatim}
  $( $j syntax 'setvar'; $)
\end{verbatim}

   \item Register $=$ as an equality for its type (\texttt{class}).
\begin{verbatim}
  $( $j equality 'wceq' from 'eqid' 'eqcomi' 'eqtri'; $)
\end{verbatim}

\end{itemize}


\subsection{Including Other Files in a Metamath Source File} \label{include}
\index{\texttt{\$[} and \texttt{\$]} auxiliary keywords}

The keywords \texttt{\$[} and \texttt{\$]} specify a file to be
included\index{included file}\index{file inclusion} at that point in a
Metamath\index{Metamath} source file\index{source file}.  The syntax for
including a file is as follows:
\begin{center}
\texttt{\$[} {\em file-name} \texttt{\$]}
\end{center}

The {\em file-name} should be a single token\index{token} with the same syntax
as a math symbol (i.e., all 93 non-whitespace
printable characters other than \texttt{\$} are
allowed, subject to the file-naming limitations of your operating system).
Comments may appear between the \texttt{\$[} and \texttt{\$]} keywords.  Included
files may include other files, which may in turn include other files, and so
on.

For example, suppose you want to use the set theory database as the starting
point for your own theory.  The first line in your file could be
\begin{center}
\texttt{\$[ set.mm \$]}
\end{center} All of the information (axioms, theorems,
etc.) in \texttt{set.mm} and any files that {\em it} includes will become
available for you to reference in your file. This can help make your work more
modular. A drawback to including files is that if you change the name of a
symbol or the label of a statement, you must also remember to update any
references in any file that includes it.


The naming conventions for included files are the same as those of your
operating system.\footnote{On the Macintosh, prior to Mac OS X,
 a colon is used to separate disk
and folder names from your file name.  For example, {\em volume}\texttt{:}{\em
file-name} refers to the root directory, {\em volume}\texttt{:}{\em
folder-name}\texttt{:}{\em file-name} refers to a folder in root, and {\em
volume}\texttt{:}{\em folder-name}\texttt{:}\ldots\texttt{:}{\em file-name} refers to a
deeper folder.  A simple {\em file-name} refers to a file in the folder from
which you launch the Metamath application.  Under Mac OS X and later,
the Metamath program is run under the Terminal application, which
conforms to Unix naming conventions.}\index{Macintosh file
names}\index{file names!Macintosh}\label{includef} For compatibility among
operating systems, you should keep the file names as simple as possible.  A
good convention to use is {\em file}\texttt{.mm} where {\em file} is eight
characters or less, in lower case.

There is no limit to the nesting depth of included files.  One thing that you
should be aware of is that if two included files themselves include a common
third file, only the {\em first} reference to this common file will be read
in.  This allows you to include two or more files that build on a common
starting file without having to worry about label and symbol conflicts that
would occur if the common file were read in more than once.  (In fact, if a
file includes itself, the self-reference will be ignored, although of course
it would not make any sense to do that.)  This feature also means, however,
that if you try to include a common file in several inner blocks, the result
might not be what you expect, since only the first reference will be replaced
with the included file (unlike the include statement in most other computer
languages).  Thus you would normally include common files only in the
outermost block\index{outermost block}.

\subsection{Compressed Proof Format}\label{compressed1}\index{compressed
proof}\index{proof!compressed}

The proof notation presented in Section~\ref{proof} is called a
{\bf normal proof}\index{normal proof}\index{proof!normal} and in principle is
sufficient to express any proof.  However, proofs often contain steps and
subproofs that are identical.  This is particularly true in typical
Metamath\index{Metamath} applications, because Metamath requires that the math
symbol sequence (usually containing a formula) at each step be separately
constructed, that is, built up piece by piece. As a result, a lot of
repetition often results.  The {\bf compressed proof} format allows Metamath
to take advantage of this redundancy to shorten proofs.

The specification for the compressed proof format is given in
Appen\-dix~\ref{compressed}.

Normally you need not concern yourself with the details of the compressed
proof format, since the Metamath program will allow you to convert from
the normal format to the compressed format with ease, and will also
automatically convert from the compressed format when proofs are displayed.
The overall structure of the compressed format is as follows:
\begin{center}
  \texttt{\$= ( } {\em label-list} \texttt{) } {\em compressed-proof\ }\ \texttt{\$.}
\end{center}
\index{\texttt{\$=} keyword}
The first \texttt{(} serves as a flag to Metamath that a compressed proof
follows.  The {\em label-list} includes all statements referred to by the
proof except the mandatory hypotheses\index{mandatory hypothesis}.  The {\em
compressed-proof} is a compact encoding of the proof, using upper-case
letters, and can be thought of as a large integer in base 26.  White
space\index{white space} inside a {\em compressed-proof} is
optional and is ignored.

It is important to note that the order of the mandatory hypotheses of
the statement being proved must not be changed if the compressed proof
format is used, otherwise the proof will become incorrect.  The reason
for this is that the mandatory hypotheses are not mentioned explicitly
in the compressed proof in order to make the compression more efficient.
If you wish to change the order of mandatory hypotheses, you must first
convert the proof back to normal format using the \texttt{save proof
{\em statement} /normal}\index{\texttt{save proof} command} command.
Later, you can go back to compressed format with \texttt{save proof {\em
statement} /compressed}.

During error checking with the \texttt{verify proof} command, an error
found in a compressed proof may point to a character in {\em
compressed-proof}, which may not be very meaningful to you.  In this
case, try to \texttt{save proof /normal} first, then do the
\texttt{verify proof} again.  In general, it is best to make sure a
proof is correct before saving it in compressed format, because severe
errors are less likely to be recoverable than in normal format.

\subsection{Specifying Unknown Proofs or Subproofs}\label{unknown}

In a proof under development, any step or subproof that is not yet known
may be represented with a single \texttt{?}.  For the purposes of
parsing the proof, the \texttt{?}\ \index{\texttt{]}@\texttt{?}\ inside
proofs} will push a single entry onto the RPN stack just as if it were a
hypothesis.  While developing a proof with the Proof
Assistant\index{Proof Assistant}, a partially developed proof may be
saved with the \texttt{save new{\char`\_}proof}\index{\texttt{save
new{\char`\_}proof} command} command, and \texttt{?}'s will be placed at
the appropriate places.

All \texttt{\$p}\index{\texttt{\$p} statement} statements must have
proofs, even if they are entirely unknown.  Before creating a proof with
the Proof Assistant, you should specify a completely unknown proof as
follows:
\begin{center}
  {\em label} \texttt{\$p} {\em statement} \texttt{\$= ?\ \$.}
\end{center}
\index{\texttt{\$=} keyword}
\index{\texttt{]}@\texttt{?}\ inside proofs}

The \texttt{verify proof}\index{\texttt{verify proof} command} command
will check the known portions of a partial proof for errors, but will
warn you that the statement has not been proved.

Note that partially developed proofs may be saved in compressed format
if desired.  In this case, you will see one or more \texttt{?}'s in the
{\em compressed-proof} part.\index{compressed
proof}\index{proof!compressed}

\section{Axioms vs.\ Definitions}\label{definitions}

The \textit{basic}
Metamath\index{Metamath} language and program
make no distinction\index{axiom vs.\
definition} between axioms\index{axiom} and
definitions.\index{definition} The \texttt{\$a}\index{\texttt{\$a}
statement} statement is used for both.  At first, this may seem
puzzling.  In the minds of many mathematicians, the distinction is
clear, even obvious, and hardly worth discussing.  A definition is
considered to be merely an abbreviation that can be replaced by the
expression for which it stands; although unless one actually does this,
to be precise then one should say that a theorem\index{theorem} is a
consequence of the axioms {\em and} the definitions that are used in the
formulation of the theorem \cite[p.~20]{Behnke}.\index{Behnke, H.}

\subsection{What is a Definition?}

What is a definition?  In its simplest form, a definition introduces a new
symbol and provides an unambiguous rule to transform an expression containing
the new symbol to one without it.  The concept of a ``proper
definition''\index{proper definition}\index{definition!proper} (as opposed to
a creative definition)\index{creative definition}\index{definition!creative}
that is usually agreed upon is (1) the definition should not strengthen the
language and (2) any symbols introduced by the definition should be eliminable
from the language \cite{Nemesszeghy}\index{Nemesszeghy, E. Z.}.  In other
words, they are mere typographical conveniences that do not belong to the
system and are theoretically superfluous.  This may seem obvious, but in fact
the nature of definitions can be subtle, sometimes requiring difficult
metatheorems to establish that they are not creative.

A more conservative stance was taken by logician S.
Le\'{s}niewski.\index{Le\'{s}niewski, S.}
\begin{quote}
Le\'{s}niewski
regards definitions as theses of the system.  In this respect they do
not differ either from the axioms or from theorems, i.e.\ from the
theses added to the system on the basis of the rule of substitution or
the rule of detachment [modus ponens].  Once definitions have been
accepted as theses of the system, it becomes necessary to consider them
as true propositions in the same sense in which axioms are true
\cite{Lejewski}.
\end{quote}\index{Lejewski, Czeslaw}

Let us look at some simple examples of definitions in propositional
calculus.  Consider the definition of logical {\sc or}
(disjunction):\index{disjunction ($\vee$)} ``$P\vee Q$ denotes $\neg P
\rightarrow Q$ (not $P$ implies $Q$).''  It is very easy to recognize a
statement making use of this definition, because it introduces the new
symbol $\vee$ that did not previously exist in the language.  It is easy
to see that no new theorems of the original language will result from
this definition.

Next, consider a definition that eliminates parentheses:  ``$P
\rightarrow Q\rightarrow R$ denotes $P\rightarrow (Q \rightarrow R)$.''
This is more subtle, because no new symbols are introduced.  The reason
this definition is considered proper is that no new symbol sequences
that are valid wffs (well-formed formulas)\index{well-formed formula
(wff)} in the original language will result from the definition, since
``$P \rightarrow Q\rightarrow R$'' is not a wff in the original
language.  Here, we implicitly make use of the fact that there is a
decision procedure that allows us to determine whether or not a symbol
sequence is a wff, and this fact allows us to use symbol sequences that
are not wffs to represent other things (such as wffs) by means of the
definition.  However, to justify the definition as not being creative we
need to prove that ``$P \rightarrow Q\rightarrow R$'' is in fact not a
wff in the original language, and this is more difficult than in the
case where we simply introduce a new symbol.

%Now let's take this reasoning to an extreme.  Propositional calculus is a
%decidable theory,\footnote{This means that a mechanical algorithm exists to
%determine whether or not a wff is a theorem.} so in principle we could make use
%of symbol sequences that are not theorems to represent other things (say, to
%encode actual theorems in a more compact way).  For example, let us extend the
%language by defining a wff ``$P$'' in the extended language as the theorem
%``$P\rightarrow P$''\footnote{This is one of the first theorems proved in the
%Metamath database \texttt{set.mm}.}\index{set
%theory database (\texttt{set.mm})} in the original language whenever ``$P$'' is
%not a theorem in the original language.  In the extended language, any wff
%``$Q$'' thus represents a theorem; to find out what theorem (in the original
%language) ``$Q$'' represents, we determine whether ``$Q$'' is a theorem in the
%original language (before the definition was introduced).  If so, we're done; if
%not, we replace ``$Q$'' by ``$Q\rightarrow Q$'' to eliminate the definition.
%This definition is therefore eliminable, and it does not ``strengthen'' the
%language because any wff that is not a theorem is not in the set of statements
%provable in the original language and thus is available for use by definitions.
%
%Of course, a definition such as this would render practically useless the
%communication of theorems of propositional calculus; but
%this is just a human shortcoming, since we can't always easily discern what is
%and is not a theorem by inspection.  In fact, the extended theory with this
%definition has no more and no less information than the original theory; it just
%expresses certain theorems of the form ``$P\rightarrow P$''
%in a more compact way.
%
%The point here is that what constitutes a proper definition is a matter of
%judgment about whether a symbol sequence can easily be recognized by a human
%as invalid in some sense (for example, not a wff); if so, the symbol sequence
%can be appropriated for use by a definition in order to make the extended
%language more compact.  Metamath\index{Metamath} lacks the ability to make this
%judgment, since as far as Metamath is concerned the definition of a wff, for
%example, is arbitrary.  You define for Metamath how wffs\index{well-formed
%formula (wff)} are constructed according to your own preferred style.  The
%concept of a wff may not even exist in a given formal system\index{formal
%system}.  Metamath treats all definitions as if they were new axioms, and it
%is up to the human mathematician to judge whether the definition is ``proper''
%'\index{proper definition}\index{definition!proper} in some agreed-upon way.

What constitutes a definition\index{definition} versus\index{axiom vs.\
definition} an axiom\index{axiom} is sometimes arbitrary in mathematical
literature.  For example, the connectives $\vee$ ({\sc or}), $\wedge$
({\sc and}), and $\leftrightarrow$ (equivalent to) in propositional
calculus are usually considered defined symbols that can be used as
abbreviations for expressions containing the ``primitive'' connectives
$\rightarrow$ and $\neg$.  This is the way we treat them in the standard
logic and set theory database \texttt{set.mm}\index{set theory database
(\texttt{set.mm})}.  However, the first three connectives can also be
considered ``primitive,'' and axiom systems have been devised that treat
all of them as such.  For example,
\cite[p.~35]{Goodstein}\index{Goodstein, R. L.} presents one with 15
axioms, some of which in fact coincide with what we have chosen to call
definitions in \texttt{set.mm}.  In certain subsets of classical
propositional calculus, such as the intuitionist
fragment\index{intuitionism}, it can be shown that one cannot make do
with just $\rightarrow$ and $\neg$ but must treat additional connectives
as primitive in order for the system to make sense.\footnote{Two nice
systems that make the transition from intuitionistic and other weak
fragments to classical logic just by adding axioms are given in
\cite{Robinsont}\index{Robinson, T. Thacher}.}

\subsection{The Approach to Definitions in \texttt{set.mm}}

In set theory, recursive definitions define a newly introduced symbol in
terms of itself.  The justification of recursive definitions, using
several ``recursion theorems,'' is usually one of the first
sophisticated proofs a student encounters when learning set theory, and
there is a significant amount of implicit metalogic behind a recursive
definition even though the definition itself is typically simple to
state.  It is, however, possible to substitute one kind of complexity
for another.  We can eliminate the need for metalogical justification by
defining the operation directly with an explicit (but complicated)
expression, then deriving the recursive definition directly as a
theorem, using a recursion theorem ``in reverse.''  We do this in
\texttt{set.mm}, as follows.

In \texttt{set.mm} our goal was to introduce almost all definitions in
the form of two expressions connected by either $\leftrightarrow$ or
$=$, where the thing being defined does not appear on the right hand
side.  Quine calls this form ``a genuine or direct definition'' \cite[p.
174]{Quine}\index{Quine, Willard Van Orman}, which makes the definitions
very easy to eliminate and the metalogic\index{metalogic} needed to
justify them as simple as possible.

We achieved this goal in almost all cases in \texttt{set.mm}.
Sometimes this makes the definitions more complex and less
intuitive.  For example, the traditional way to define addition of
natural numbers is to define an operation called {\em
successor}\index{successor} (which means ``plus one'' and is denoted by
``${\rm suc}$''), then define addition recursively\index{recursive
definition} with the two definitions $n + 0 = n$ and $m + {\rm suc}\,n =
{\rm suc} (m + n)$.  Although this definition seems simple and obvious,
the method to eliminate the definition is not obvious:  in the second
part of the definition, addition is defined in terms of itself.  By
eliminating the definition, we don't mean repeatedly applying it to
specific $m$ and $n$ but rather showing the explicit, closed-form
set-theoretical expression that $m + n$ represents, that will work for
any $m$ and $n$ and that does not have a $+$ sign on its right-hand
side.  For a recursive definition like this not to be circular
(creative), there are some hidden, underlying assumptions we must make,
for example that the natural numbers have a certain kind of order.  In
\texttt{set.mm} we chose to start with the direct (though complex and
nonintuitive) definition then derive from it the standard recursive
definition.\footnote{The closed-form definition used in \texttt{set.mm}
for the addition operation on ordinals\index{ordinal
addition}\index{addition!of ordinals} (of which natural numbers are a
subset) is

\setbox\startprefix=\hbox{\tt \ \ df-oadd\ \$a\ }
\setbox\contprefix=\hbox{\tt \ \ \ \ \ \ \ \ \ \ \ \ \ }
\startm
\m{\vdash}\m{+_o}\m{=}\m{\{}\m{\langle}\m{\langle}\m{x}\m{,}\m{y}\m{\rangle}%
\m{,}\m{z}\m{\rangle}\m{|}\m{(}\m{(}\m{x}\m{\in}\m{{\rm On}}\m{\wedge}\m{y}\m{%
\in}\m{{\rm On}}\m{)}\m{\wedge}\m{z}\m{=}\m{(}\m{{\rm rec}}\m{(}\m{\{}\m{%
\langle}\m{w}\m{,}\m{v}\m{\rangle}\m{|}\m{v}\m{=}\m{{\rm suc}}\m{w}\m{\}}\m{,}%
\m{x}\m{)}\m{`}\m{y}\m{)}\m{)}\m{\}}
\endm

\noindent Here, the abstraction class of nested ordered pairs\index{abstraction
class!of nested ordered pairs} is defined by \texttt{df-oprab} in \texttt{set.mm},
and ${\rm rec}$ is a ``recursion operator''\index{recursion operator} with
the definition

\setbox\startprefix=\hbox{\tt \ \ df-rfg\ \$a\ }
\setbox\contprefix=\hbox{\tt \ \ \ \ \ \ \ \ \ \ \ \ }
\startm
\m{\vdash}\m{{\rm rec}}\m{(}\m{F}\m{,}\m{A}\m{)}\m{=}\m{\bigcup}\m{\{}\m{f}%
\m{|}\m{\exists}\m{x}\m{(}\m{x}\m{\in}\m{{\rm On}}\m{\wedge}\m{(}\m{f}\m{{%
\rm Fn}}\m{x}\m{\wedge}\m{\forall}\m{y}\m{(}\m{y}\m{\in}\m{x}\m{\rightarrow}%
\m{(}\m{f}\m{`}\m{y}\m{)}\m{=}\m{(}\m{\{}\m{\langle}\m{g}\m{,}\m{z}\m{\rangle}%
\m{|}\m{(}\m{(}\m{g}\m{=}\m{\varnothing}\m{\wedge}\m{z}\m{=}\m{A}\m{)}\m{\vee}%
\m{(}\m{\lnot}\m{(}\m{g}\m{=}\m{\varnothing}\m{\vee}\m{{\rm Lim}}\m{{\rm dom}}%
\m{g}\m{)}\m{\wedge}\m{z}\m{=}\m{(}\m{F}\m{`}\m{(}\m{g}\m{`}\m{\bigcup}\m{{%
\rm dom}}\m{g}\m{)}\m{)}\m{)}\m{\vee}\m{(}\m{{\rm Lim}}\m{{\rm dom}}\m{g}\m{%
\wedge}\m{z}\m{=}\m{\bigcup}\m{{\rm ran}}\m{g}\m{)}\m{)}\m{\}}\m{`}\m{(}\m{f}%
\m{\restriction}\m{y}\m{)}\m{)}\m{)}\m{)}\m{)}\m{\}}
\endm

\noindent which can be further broken down with definitions shown in
Section~\ref{setdefinitions}.  You may be surprised at the complexity of what
seems like such a simple notion.  From these definitions the simpler, more
intuitive recursive definition is derived as a set of theorems.}\index{natural
number}\index{addition}\index{recursive definition}\index{ordinal addition}
The end result is the same, but we completely eliminate the rather complex
metalogic that justifies the recursive definition.  (For a mathematician,
recursive definitions are more efficient and intuitive than
direct ones once the metalogic has been learned or possibly just accepted as
correct.  However, it was felt that direct definition in \texttt{set.mm}
maximizes rigor by minimizing metalogic.  It can be eliminated effortlessly,
something difficult to do with a recursive definition.)

\subsection{Adding Constraints on Definitions}

The basic Metamath language and the Metamath program do
not have any built-in constraints on definitions, since they are just
\$a statements.

However, nothing prevents a verification system from verifying
additional rules to impose further limitations on definitions.
For example, the \texttt{mmj2}\index{mmj2} program
supports various kinds of
additional information comments (see section \ref{jcomment}).
One of their uses is to optionally verify additional constraints,
including constraints to verify that definitions meet certain
requirements.
These additional checks are required by the
continuous integration (CI)\index{continuous integration (CI)}
checks of the
\texttt{set.mm}\index{set theory database (\texttt{set.mm})}%
\index{Metamath Proof Explorer}
database.
This approach enables us to optionally impose additional requirements
on definitions if we wish, without necessarily imposing those rules on
all databases or requiring all verification systems to implement them.
In addition, this allows us to impose specialized constraints tailored
to one database while not requiring other systems to implement
those specialized constraints.

We impose two constraints on the
\texttt{set.mm}\index{set theory database (\texttt{set.mm})}%
\index{Metamath Proof Explorer} database
via the \texttt{mmj2}\index{mmj2} program that are worth discussing here:
a parse check and a definition soundness check.

% On February 11, 2019 8:32:32 PM EST, saueran@oregonstate.edu wrote:
% The following addition to the end of set.mm is accepted by the mmj2
% parser and definition checker and the metamath verifier(at least it was
% when I checked, you should check it too), and creates a contradiction by
% proving the theorem |- ph.
% ${
% wleftp $a wff ( ( ph ) $.
% wbothp $a wff ( ph ) $.
% df-leftp $a |- ( ( ( ph ) <-> -. ph ) $.
% df-bothp $a |- ( ( ph ) <-> ph ) $.
% anything $p |- ph $=
%   ( wbothp wn wi wleftp df-leftp biimpi df-bothp mpbir mpbi simplim ax-mp)
%   ABZAMACZDZCZMOEZOCQAEZNDZRNAFGSHIOFJMNKLAHJ $.
% $}
%
% This particular problem is countered by enabling, within mmj2,
% SetParser,mmj.verify.LRParser

First,
we enable a parse check in \texttt{mmj2} (through its
\texttt{SetParser} command) that requires that all new definitions
must \textit{not} create an ambiguous parse for a KLR(5) parser.
This prevents some errors such as definitions with imbalanced parentheses.

Second, we run a definition soundness check specific to
\texttt{set.mm} or databases similar to it.
(through the \texttt{definitionCheck} macro).
Some \texttt{\$a} statements (including all ax-* statemnets)
are excluded from these checks, as they will
always fail this simple check,
but they are appropriate for most definitions.
This check imposes a set of additional rules:

\begin{enumerate}

\item New definitions must be introduced using $=$ or $\leftrightarrow$.

\item No \texttt{\$a} statement introduced before this one is allowed to use the
symbol being defined in this definition, and the definition is not
allowed to use itself (except once, in the definiendum).

\item Every variable in the definiens should not be distinct

\item Every dummy variable in the definiendum
are required to be distinct from each other and from variables in
the definiendum.
To determine this, the system will look for a "justification" theorem
in the database, and if it is not there, attempt to briefly prove
$( \varphi \rightarrow \forall x \varphi )$  for each dummy variable x.

\item Every dummy variable should be a set variable,
unless there is a justification theorem available.

\item Every dummy variable must be bound
(if the system cannot determine this a justification theorem must be
provided).

\end{enumerate}

\subsection{Summary of Approach to Definitions}

In short, when being rigorous it turns out that
definitions can be subtle, sometimes requiring difficult
metatheorems to establish that they are not creative.

Instead of building such complications into the Metamath language itself,
the basic Metmath language and program simply treat traditional
axioms and definitions as the same kind of \texttt{\$a} statement.
We have then built various tools to enable people to
verify additional conditions as their creators believe is appropriate
for those specific databases, without complicating the Metamath foundations.

\chapter{The Metamath Program}\label{commands}

This chapter provides a reference manual for the
Metamath program.\index{Metamath!commands}

Current instructions for obtaining and installing the Metamath program
can be found at the \url{http://metamath.org} web site.
For Windows, there is a pre-compiled version called
\texttt{metamath.exe}.  For Unix, Linux, and Mac OS X
(which we will refer to collectively as ``Unix''), the Metamath program
can be compiled from its source code with the command
\begin{verbatim}
gcc *.c -o metamath
\end{verbatim}
using the \texttt{gcc} {\sc c} compiler available on those systems.

In the command syntax descriptions below, fields enclosed in square
brackets [\ ] are optional.  File names may be optionally enclosed in
single or double quotes.  This is useful if the file name contains
spaces or
slashes (\texttt{/}), such as in Unix path names, \index{Unix file
names}\index{file names!Unix} that might be confused with Metamath
command qualifiers.\index{Unix file names}\index{file names!Unix}

\section{Invoking Metamath}

Unix, Linux, and Mac OS X
have a command-line interface called the {\em
bash shell}.  (In Mac OS X, select the Terminal application from
Applications/Utilities.)  To invoke Metamath from the bash shell prompt,
assuming that the Metamath program is in the current directory, type
\begin{verbatim}
bash$ ./metamath
\end{verbatim}

To invoke Metamath from a Windows DOS or Command Prompt, assuming that
the Metamath program is in the current directory (or in a directory
included in the Path system environment variable), type
\begin{verbatim}
C:\metamath>metamath
\end{verbatim}

To use command-line arguments at invocation, the command-line arguments
should be a list of Metamath commands, surrounded by quotes if they
contain spaces.  In Windows, the surrounding quotes must be double (not
single) quotes.  For example, to read the database file \texttt{set.mm},
verify all proofs, and exit the program, type (under Unix)
\begin{verbatim}
bash$ ./metamath 'read set.mm' 'verify proof *' exit
\end{verbatim}
Note that in Unix, any directory path with \texttt{/}'s must be
surrounded by quotes so Metamath will not interpret the \texttt{/} as a
command qualifier.  So if \texttt{set.mm} is in the \texttt{/tmp}
directory, use for the above example
\begin{verbatim}
bash$ ./metamath 'read "/tmp/set.mm"' 'verify proof *' exit
\end{verbatim}

For convenience, if the command-line has one argument and no spaces in
the argument, the command is implicitly assumed to be \texttt{read}.  In
this one special case, \texttt{/}'s are not interpreted as command
qualifiers, so you don't need quotes around a Unix file name.  Thus
\begin{verbatim}
bash$ ./metamath /tmp/set.mm
\end{verbatim}
and
\begin{verbatim}
bash$ ./metamath "read '/tmp/set.mm'"
\end{verbatim}
are equivalent.


\section{Controlling Metamath}

The Metamath program was first developed on a {\sc vax/vms} system, and
some aspects of its command line behavior reflect this heritage.
Hopefully you will find it reasonably user-friendly once you get used to
it.

Each command line is a sequence of English-like words separated by
spaces, as in \texttt{show settings}.  Command words are not case
sensitive, and only as many letters are needed as are necessary to
eliminate ambiguity; for example, \texttt{sh se} would work for the
command \texttt{show settings}.  In some cases arguments such as file
names, statement labels, or symbol names are required; these are
case-sensitive (although file names may not be on some operating
systems).

A command line is entered by typing it in then pressing the {\em return}
({\em enter}) key.  To find out what commands are available, type
\texttt{?} at the \texttt{MM>} prompt.  To find out the choices at any
point in a command, press {\em return} and you will be prompted for
them.  The default choice (the one selected if you just press {\em
return}) is shown in brackets (\texttt{<>}).

You may also type \texttt{?} in place of a command word to force
Metamath to tell you what the choices are.  The \texttt{?} method won't
work, though, if a non-keyword argument such as a file name is expected
at that point, because the program will think that \texttt{?} is the
value of the argument.

Some commands have one or more optional qualifiers which modify the
behavior of the command.  Qualifiers are preceded by a slash
(\texttt{/}), such as in \texttt{read set.mm / verify}.  Spaces are
optional around the \texttt{/}.  If you need to use a space or
slash in a command
argument, as in a Unix file name, put single or double quotes around the
command argument.

The \texttt{open log} command will save everything you see on the
screen and is useful to help you recover should something go wrong in a
proof, or if you want to document a bug.

If a command responds with more than a screenful, you will be
prompted to \texttt{<return> to continue, Q to quit, or S to scroll to
end}.  \texttt{Q} or \texttt{q} (not case-sensitive) will complete the
command internally but will suppress further output until the next
\texttt{MM>} prompt.  \texttt{s} will suppress further pausing until the
next \texttt{MM>} prompt.  After the first screen, you are also
presented with the choice of \texttt{b} to go back a screenful.  Note
that \texttt{b} may also be entered at the \texttt{MM>} prompt
immediately after a command to scroll back through the output of that
command.

A command line enclosed in quotes is executed by your operating system.
See Section~\ref{oscmd}.

{\em Warning:} Pressing {\sc ctrl-c} will abort the Metamath program
unconditionally.  This means any unsaved work will be lost.


\subsection{\texttt{exit} Command}\index{\texttt{exit} command}

Syntax:  \texttt{exit} [\texttt{/force}]

This command exits from Metamath.  If there have been changes to the
source with the \texttt{save proof} or \texttt{save new{\char`\_}proof}
commands, you will be given an opportunity to \texttt{write source} to
permanently save the changes.

In Proof Assistant\index{Proof Assistant} mode, the \texttt{exit} command will
return to the \verb/MM>/ prompt. If there were changes to the proof, you will
be given an opportunity to \texttt{save new{\char`\_}proof}.

The \texttt{quit} command is a synonym for \texttt{exit}.

Optional qualifier:
    \texttt{/force} - Do not prompt if changes were not saved.  This qualifier is
        useful in \texttt{submit} command files (Section~\ref{sbmt})
        to ensure predictable behavior.





\subsection{\texttt{open log} Command}\index{\texttt{open log} command}
Syntax:  \texttt{open log} {\em file-name}

This command will open a log file that will store everything you see on
the screen.  It is useful to help recovery from a mistake in a long Proof
Assistant session, or to document bugs.\index{Metamath!bugs}

The log file can be closed with \texttt{close log}.  It will automatically be
closed upon exiting Metamath.



\subsection{\texttt{close log} Command}\index{\texttt{close log} command}
Syntax:  \texttt{close log}

The \texttt{close log} command closes a log file if one is open.  See
also \texttt{open log}.




\subsection{\texttt{submit} Command}\index{\texttt{submit} command}\label{sbmt}
Syntax:  \texttt{submit} {\em filename}

This command causes further command lines to be taken from the specified
file.  Note that any line beginning with an exclamation point (\texttt{!}) is
treated as a comment (i.e.\ ignored).  Also note that the scrolling
of the screen output is continuous, so you may want to open a log file
(see \texttt{open log}) to record the results that fly by on the screen.
After the lines in the file are exhausted, Metamath returns to its
normal user interface mode.

Currently, the \texttt{submit} command is not recursive.  In other words,
\texttt{submit} commands are not allowed inside a command file.


\subsection{\texttt{erase} Command}\index{\texttt{erase} command}
Syntax:  \texttt{erase}

This command will reset Metamath to its starting state, deleting any
data\-base that was \texttt{read} in.
 If there have been changes to the
source with the \texttt{save proof} or \texttt{save new{\char`\_}proof}
commands, you will be given an opportunity to \texttt{write source} to
permanently save the changes.



\subsection{\texttt{set echo} Command}\index{\texttt{set echo} command}
Syntax:  \texttt{set echo on} or \texttt{set echo off}

The \texttt{set echo on} command will cause command lines to be echoed with any
abbreviations expanded.  While learning the Metamath commands, this
feature will show you the exact command that your abbreviated input
corresponds to.



\subsection{\texttt{set scroll} Command}\index{\texttt{set scroll} command}
Syntax:  \texttt{set scroll prompted} or \texttt{set scroll continuous}

The Metamath command line interface starts off in the \texttt{prompted} mode,
which means that you will be prompted to continue or quit after each
full screen in a long listing.  In \texttt{continuous} mode, long listings will be
scrolled without pausing.

% LaTeX bug? (1) \texttt{\_} puts out different character than
% \texttt{{\char`\_}}
%  = \verb$_$  (2) \texttt{{\char`\_}} puts out garbage in \subsection
%  argument
\subsection{\texttt{set width} Command}\index{\texttt{set
width} command}
Syntax:  \texttt{set width} {\em number}

Metamath assumes the width of your screen is 79 characters (chosen
because the Command Prompt in Windows XP has a wrapping bug at column
80).  If your screen is wider or narrower, this command allows you to
change this default screen width.  A larger width is advantageous for
logging proofs to an output file to be printed on a wide printer.  A
smaller width may be necessary on some terminals; in this case, the
wrapping of the information messages may sometimes seem somewhat
unnatural, however.  In \LaTeX\index{latex@{\LaTeX}!characters per line},
there is normally a maximum of 61 characters per line with typewriter
font.  (The examples in this book were produced with 61 characters per
line.)

\subsection{\texttt{set height} Command}\index{\texttt{set
height} command}
Syntax:  \texttt{set height} {\em number}

Metamath assumes your screen height is 24 lines of characters.  If your
screen is taller or shorter, this command lets you to change the number
of lines at which the display pauses and prompts you to continue.

\subsection{\texttt{beep} Command}\index{\texttt{beep} command}

Syntax:  \texttt{beep}

This command will produce a beep.  By typing it ahead after a
long-running command has started, it will alert you that the command is
finished.  For convenience, \texttt{b} is an abbreviation for
\texttt{beep}.

Note:  If \texttt{b} is typed at the \texttt{MM>} prompt immediately
after the end of a multiple-page display paged with ``\texttt{Press
<return> for more}...'' prompts, then the \texttt{b} will back up to the
previous page rather than perform the \texttt{beep} command.
In that case you must type the unabbreviated \texttt{beep} form
of the command.

\subsection{\texttt{more} Command}\index{\texttt{more} command}

Syntax:  \texttt{more} {\em filename}

This command will display the contents of an {\sc ascii} file on your
screen.  (This command is provided for convenience but is not very
powerful.  See Section~\ref{oscmd} to invoke your operating system's
command to do this, such as the \texttt{more} command in Unix.)

\subsection{Operating System Commands}\index{operating system
command}\label{oscmd}

A line enclosed in single or double quotes will be executed by your
computer's operating system if it has a command line interface.  For
example, on a {\sc vax/vms} system,
\verb/MM> 'dir'/
will print disk directory contents.  Note that this feature will not
work on the Macintosh prior to Mac OS X, which does not have a command
line interface.

For your convenience, the trailing quote is optional.

\subsection{Size Limitations in Metamath}

In general, there are no fixed, predefined limits\index{Metamath!memory
limits} on how many labels, tokens\index{token}, statements, etc.\ that
you may have in a database file.  The Metamath program uses 32-bit
variables (64-bit on 64-bit CPUs) as indices for almost all internal
arrays, which are allocated dynamically as needed.



\section{Reading and Writing Files}

The following commands create new files:  the \texttt{open} commands;
the \texttt{write} commands; the \texttt{/html},
\texttt{/alt{\char`\_}html}, \texttt{/brief{\char`\_}html},
\texttt{/brief{\char`\_}alt{\char`\_}html} qualifiers of \texttt{show
statement}, and \texttt{midi}.  The following commands append to files
previously opened:  the \texttt{/tex} qualifier of \texttt{show proof}
and \texttt{show new{\char`\_}proof}; the \texttt{/tex} and
\texttt{/simple{\char`\_}tex} qualifiers of \texttt{show statement}; the
\texttt{close} commands; and all screen dialog between \texttt{open log}
and \texttt{close log}.

The commands that create new files will not overwrite an existing {\em
filename} but will rename the existing one to {\em
filename}\texttt{{\char`\~}1}.  An existing {\em
filename}\texttt{{\char`\~}1} is renamed {\em
filename}\texttt{{\char`\~}2}, etc.\ up to {\em
filename}\texttt{{\char`\~}9}.  An existing {\em
filename}\texttt{{\char`\~}9} is deleted.  This makes recovery from
mistakes easier but also will clutter up your directory, so occasionally
you may want to clean up (delete) these \texttt{{\char`\~}}$n$ files.


\subsection{\texttt{read} Command}\index{\texttt{read} command}
Syntax:  \texttt{read} {\em file-name} [\texttt{/verify}]

This command will read in a Metamath language source file and any included
files.  Normally it will be the first thing you do when entering Metamath.
Statement syntax is checked, but proof syntax is not checked.
Note that the file name may be enclosed in single or double quotes;
this is useful if the file name contains slashes, as might be the case
under Unix.

If you are getting an ``\texttt{?Expected VERIFY}'' error
when trying to read a Unix file name with slashes, you probably haven't
quoted it.\index{Unix file names}\index{file names!Unix}

If you are prompted for the file name (by pressing {\em return}
 after \texttt{read})
you should {\em not} put quotes around it, even if it is a Unix file name
with slashes.

Optional command qualifier:

    \texttt{/verify} - Verify all proofs as the database is read in.  This
         qualifier will slow down reading in the file.  See \texttt{verify
         proof} for more information on file error-checking.

See also \texttt{erase}.



\subsection{\texttt{write source} Command}\index{\texttt{write source} command}
Syntax:  \texttt{write source} {\em filename}
[\texttt{/rewrap}]
[\texttt{/split}]
[\texttt{/keep\_includes}]
[\texttt{/no\_versioning}]

This command will write the contents of a Metamath\index{database}
database into a file.\index{source file}

Optional command qualifiers:

\texttt{/rewrap} -
Reformats statements and comments according to the
convention used in the set.mm database.
It unwraps the
lines in the comment before each \texttt{\$a} and \texttt{\$p} statement,
then it
rewraps the line.  You should compare the output to the original
to make sure that the desired effect results; if not, go back to
the original source.  The wrapped line length honors the
\texttt{set width}
parameter currently in effect.  Note:  Text
enclosed in \texttt{<HTML>}...\texttt{</HTML>} tags is not modified by the
\texttt{/rewrap} qualifier.
Proofs themselves are not reformatted;
use \texttt{save proof * / compressed} to do that.
An isolated tilde (\~{}) is always kept on the same line as the following
symbol, so you can find all comment references to a symbol by
searching for \~{} followed by a space and that symbol
(this is useful for finding cross-references).
Incidentally, \texttt{save proof} also honors the \texttt{set width}
parameter currently in effect.

\texttt{/split} - Files included in the source using the expression
\$[ \textit{inclfile} \$] will be
written into separate files instead of being included in a single output
file.  The name of each separately written included file will be the
\textit{inclfile} argument of its inclusion command.

\texttt{/keep\_includes} - If a source file has includes but is written as a
single file by omitting \texttt{/split}, by default the included files will
be deleted (actually just renamed with a \char`\~1 suffix unless
\texttt{/no\_versioning} is specified) to prevent the possibly confusing
source duplication in both the output file and the included file.
The \texttt{/keep\_includes} qualifier will prevent this deletion.

\texttt{/no\_versioning} - Backup files suffixed with \char`\~1 are not created.


\section{Showing Status and Statements}



\subsection{\texttt{show settings} Command}\index{\texttt{show settings} command}
Syntax:  \texttt{show settings}

This command shows the state of various parameters.

\subsection{\texttt{show memory} Command}\index{\texttt{show memory} command}
Syntax:  \texttt{show memory}

This command shows the available memory left.  It is not meaningful
on most modern operating systems,
which have virtual memory.\index{Metamath!memory usage}


\subsection{\texttt{show labels} Command}\index{\texttt{show labels} command}
Syntax:  \texttt{show labels} {\em label-match} [\texttt{/all}]
   [\texttt{/linear}]

This command shows the labels of \texttt{\$a} and \texttt{\$p}
statements that match {\em label-match}.  A \verb$*$ in {label-match}
matches zero or more characters.  For example, \verb$*abc*def$ will match all
labels containing \verb$abc$ and ending with \verb$def$.

Optional command qualifiers:

   \texttt{/all} - Include matches for \texttt{\$e} and \texttt{\$f}
   statement labels.

   \texttt{/linear} - Display only one label per line.  This can be useful for
       building scripts in conjunction with the utilities under the
       \texttt{tools}\index{\texttt{tools} command} command.



\subsection{\texttt{show statement} Command}\index{\texttt{show statement} command}
Syntax:  \texttt{show statement} {\em label-match} [{\em qualifiers} (see below)]

This command provides information about a statement.  Only statements
that have labels (\texttt{\$f}\index{\texttt{\$f} statement},
\texttt{\$e}\index{\texttt{\$e} statement},
\texttt{\$a}\index{\texttt{\$a} statement}, and
\texttt{\$p}\index{\texttt{\$p} statement}) may be specified.
If {\em label-match}
contains wildcard (\verb$*$) characters, all matching statements will be
displayed in the order they occur in the database.

Optional qualifiers (only one qualifier at a time is allowed):

    \texttt{/comment} - This qualifier includes the comment that immediately
       precedes the statement.

    \texttt{/full} - Show complete information about each statement,
       and show all
       statements matching {\em label} (including \texttt{\$e}
       and \texttt{\$f} statements).

    \texttt{/tex} - This qualifier will write the statement information to the
       \LaTeX\ file previously opened with \texttt{open tex}.  See
       Section~\ref{texout}.

    \texttt{/simple{\char`\_}tex} - The same as \texttt{/tex}, except that
       \LaTeX\ macros are not used for formatting equations, allowing easier
       manual edits of the output for slide presentations, etc.

    \texttt{/html}\index{html generation@{\sc html} generation},
       \texttt{/alt{\char`\_}html}, \texttt{/brief{\char`\_}html},
       \texttt{/brief{\char`\_}alt{\char`\_}html} -
       These qualifiers invoke a special mode of
       \texttt{show statement} that
       creates a web page for the statement.  They may not be used with
       any other qualifier.  See Section~\ref{htmlout} or
       \texttt{help html} in the program.


\subsection{\texttt{search} Command}\index{\texttt{search} command}
Syntax:  search {\em label-match}
\texttt{"}{\em symbol-match}{\tt}" [\texttt{/all}] [\texttt{/comments}]
[\texttt{/join}]

This command searches all \texttt{\$a} and \texttt{\$p} statements
matching {\em label-match} for occurrences of {\em symbol-match}.  A
\verb@*@ in {\em label-match} matches any label character.  A \verb@$*@
in {\em symbol-match} matches any sequence of symbols.  The symbols in
{\em symbol-match} must be separated by white space.  The quotes
surrounding {\em symbol-match} may be single or double quotes.  For
example, \texttt{search b}\verb@* "-> $* ch"@ will list all statements
whose labels begin with \texttt{b} and contain the symbols \verb@->@ and
\texttt{ch} surrounding any symbol sequence (including no symbol
sequence).  The wildcards \texttt{?} and \texttt{\$?} are also available
to match individual characters in labels and symbols respectively; see
\texttt{help search} in the Metamath program for details on their usage.

Optional command qualifiers:

    \texttt{/all} - Also search \texttt{\$e} and \texttt{\$f} statements.

    \texttt{/comments} - Search the comment that immediately precedes each
        label-matched statement for {\em symbol-match}.  In this case
        {\em symbol-match} is an arbitrary, non-case-sensitive character
        string.  Quotes around {\em symbol-match} are optional if there
        is no ambiguity.

    \texttt{/join} - In the case of a \texttt{\$a} or \texttt{\$p} statement,
	prepend its \texttt{\$e}
	hypotheses for searching. The
	\texttt{/join} qualifier has no effect in \texttt{/comments} mode.

\section{Displaying and Verifying Proofs}


\subsection{\texttt{show proof} Command}\index{\texttt{show proof} command}
Syntax:  \texttt{show proof} {\em label-match} [{\em qualifiers} (see below)]

This command displays the proof of the specified
\texttt{\$p}\index{\texttt{\$p} statement} statement in various formats.
The {\em label-match} may contain wildcard (\verb@$*@) characters to match
multiple statements.  Without any qualifiers, only the logical steps
will be shown (i.e.\ syntax construction steps will be omitted), in an
indented format.

Most of the time, you will use
    \texttt{show proof} {\em label}
to see just the proof steps corresponding to logical inferences.

Optional command qualifiers:

    \texttt{/essential} - The proof tree is trimmed of all
        \texttt{\$f}\index{\texttt{\$f} statement} hypotheses before
        being displayed.  (This is the default, and it is redundant to
        specify it.)

    \texttt{/all} - the proof tree is not trimmed of all \texttt{\$f} hypotheses before
        being displayed.  \texttt{/essential} and \texttt{/all} are mutually exclusive.

    \texttt{/from{\char`\_}step} {\em step} - The display starts at the specified
        step.  If
        this qualifier is omitted, the display starts at the first step.

    \texttt{/to{\char`\_}step} {\em step} - The display ends at the specified
        step.  If this
        qualifier is omitted, the display ends at the last step.

    \texttt{/tree{\char`\_}depth} {\em number} - Only
         steps at less than the specified proof
        tree depth are displayed.  Sometimes useful for obtaining an overview of
        the proof.

    \texttt{/reverse} - The steps are displayed in reverse order.

    \texttt{/renumber} - When used with \texttt{/essential}, the steps are renumbered
        to correspond only to the essential steps.

    \texttt{/tex} - The proof is converted to \LaTeX\ and\index{latex@{\LaTeX}}
        stored in the file opened
        with \texttt{open tex}.  See Section~\ref{texout} or
        \texttt{help tex} in the program.

    \texttt{/lemmon} - The proof is displayed in a non-indented format known
        as Lemmon style, with explicit previous step number references.
        If this qualifier is omitted, steps are indented in a tree format.

    \texttt{/start{\char`\_}column} {\em number} - Overrides the default column
        (16)
        at which the formula display starts in a Lemmon-style display.  May be
        used only in conjunction with \texttt{/lemmon}.

    \texttt{/normal} - The proof is displayed in normal format suitable for
        inclusion in a Metamath source file.  May not be used with any other
        qualifier.

    \texttt{/compressed} - The proof is displayed in compressed format
        suitable for inclusion in a Metamath source file.  May not be used with
        any other qualifier.

    \texttt{/statement{\char`\_}summary} - Summarizes all statements (like a
        brief \texttt{show statement})
        used by the proof.  It may not be used with any other qualifier
        except \texttt{/essential}.

    \texttt{/detailed{\char`\_}step} {\em step} - Shows the details of what is
        happening at
        a specific proof step.  May not be used with any other qualifier.
        The {\em step} is the step number shown when displaying a
        proof without the \texttt{/renumber} qualifier.


\subsection{\texttt{show usage} Command}\index{\texttt{show usage} command}
Syntax:  \texttt{show usage} {\em label-match} [\texttt{/recursive}]

This command lists the statements whose proofs make direct reference to
the statement specified.

Optional command qualifier:

    \texttt{/recursive} - Also include statements whose proofs ultimately
        depend on the statement specified.



\subsection{\texttt{show trace\_back} Command}\index{\texttt{show
       trace{\char`\_}back} command}
Syntax:  \texttt{show trace{\char`\_}back} {\em label-match} [\texttt{/essential}] [\texttt{/axioms}]
    [\texttt{/tree}] [\texttt{/depth} {\em number}]

This command lists all statements that the proof of the \texttt{\$p}
statement(s) specified by {\em label-match} depends on.

Optional command qualifiers:

    \texttt{/essential} - Restrict the trace-back to \texttt{\$e}
        \index{\texttt{\$e} statement} hypotheses of proof trees.

    \texttt{/axioms} - List only the axioms that the proof ultimately depends on.

    \texttt{/tree} - Display the trace-back in an indented tree format.

    \texttt{/depth} {\em number} - Restrict the \texttt{/tree} trace-back to the
        specified indentation depth.

    \texttt{/count{\char`\_}steps} - Count the number of steps the proof has
       all the way back to axioms.  If \texttt{/essential} is specified,
       expansions of variable-type hypotheses (syntax constructions) are not counted.

\subsection{\texttt{verify proof} Command}\index{\texttt{verify proof} command}
Syntax:  \texttt{verify proof} {\em label-match} [\texttt{/syntax{\char`\_}only}]

This command verifies the proofs of the specified statements.  {\em
label-match} may contain wild card characters (\texttt{*}) to verify
more than one proof; for example \verb/*abc*def/ will match all labels
containing \texttt{abc} and ending with \texttt{def}.
The command \texttt{verify proof *} will verify all proofs in the database.

Optional command qualifier:

    \texttt{/syntax{\char`\_}only} - This qualifier will perform a check of syntax
        and RPN
        stack violations only.  It will not verify that the proof is
        correct.  This qualifier is useful for quickly determining which
        proofs are incomplete (i.e.\ are under development and have \texttt{?}'s
        in them).

{\em Note:} \texttt{read}, followed by \texttt{verify proof *}, will ensure
 the database is free
from errors in the Metamath language but will not check the markup notation
in comments.
You can also check the markup notation by running \texttt{verify markup *}
as discussed in Section~\ref{verifymarkup}; see also the discussion
on generating {\sc HTML} in Section~\ref{htmlout}.

\subsection{\texttt{verify markup} Command}\index{\texttt{verify markup} command}\label{verifymarkup}
Syntax:  \texttt{verify markup} {\em label-match}
[\texttt{/date{\char`\_}skip}]
[\texttt{/top{\char`\_}date{\char`\_}skip}]
[\texttt{/file{\char`\_}skip}]
[\texttt{/verbose}]

This command checks comment markup and other informal conventions we have
adopted.  It error-checks the latexdef, htmldef, and althtmldef statements
in the \texttt{\$t} statement of a Metamath source file.\index{error checking}
It error-checks any \texttt{`}...\texttt{`},
\texttt{\char`\~}~\textit{label},
and bibliographic markups in statement descriptions.
It checks that
\texttt{\$p} and \texttt{\$a} statements
have the same content when their labels start with
``ax'' and ``ax-'' respectively but are otherwise identical, for example
ax4 and ax-4.
It also verifies the date consistency of ``(Contributed by...),''
``(Revised by...),'' and ``(Proof shortened by...)'' tags in the comment
above each \texttt{\$a} and \texttt{\$p} statement.

Optional command qualifiers:

    \texttt{/date{\char`\_}skip} - This qualifier will
        skip date consistency checking,
        which is usually not required for databases other than
	\texttt{set.mm}.

    \texttt{/top{\char`\_}date{\char`\_}skip} - This qualifier will check date consistency except
        that the version date at the top of the database file will not
        be checked.  Only one of
        \texttt{/date{\char`\_}skip} and
        \texttt{/top{\char`\_}date{\char`\_}skip} may be
        specified.

    \texttt{/file{\char`\_}skip} - This qualifier will skip checks that require
        external files to be present, such as checking GIF existence and
        bibliographic links to mmset.html or equivalent.  It is useful
        for doing a quick check from a directory without these files.

    \texttt{/verbose} - Provides more information.  Currently it provides a list
        of axXXX vs. ax-XXX matches.

\subsection{\texttt{save proof} Command}\index{\texttt{save proof} command}
Syntax:  \texttt{save proof} {\em label-match} [\texttt{/normal}]
   [\texttt{/compressed}]

The \texttt{save proof} command will reformat a proof in one of two formats and
replace the existing proof in the source buffer\index{source
buffer}.  It is useful for
converting between proof formats.  Note that a proof will not be
permanently saved until a \texttt{write source} command is issued.

Optional command qualifiers:

    \texttt{/normal} - The proof is saved in the normal format (i.e., as a
        sequence
        of labels, which is the defined format of the basic Metamath
        language).\index{basic language}  This is the default format that
        is used if a qualifier
        is omitted.

    \texttt{/compressed} - The proof is saved in the compressed format which
        reduces storage requirements for a database.
        See Appendix~\ref{compressed}.




\section{Creating Proofs}\label{pfcommands}\index{Proof Assistant}

Before using the Proof Assistant, you must add a \texttt{\$p} to your
source file (using a text editor) containing the statement you want to
prove.  Its proof should consist of a single \texttt{?}, meaning
``unknown step.''  Example:
\begin{verbatim}
equid $p x = x $= ? $.
\end{verbatim}

To enter the Proof assistant, type \texttt{prove} {\em label}, e.g.
\texttt{prove equid}.  Metamath will respond with the \texttt{MM-PA>}
prompt.

Proofs are created working backwards from the statement being proved,
primarily using a series of \texttt{assign} commands.  A proof is
complete when all steps are assigned to statements and all steps are
unified and completely known.  During the creation of a proof, Metamath
will allow only operations that are legal based on what is known up to
that point.  For example, it will not allow an \texttt{assign} of a
statement that cannot be unified with the unknown proof step being
assigned.

{\em Important:}
The Proof Assistant is
{\em not} a tool to help you discover proofs.  It is just a tool to help
you add them to the database.  For a tutorial read
Section~\ref{frstprf}.
To practice using the Proof Assistant, you may
want to \texttt{prove} an existing theorem, then delete all steps with
\texttt{delete all}, then re-create it with the Proof Assistant while
looking at its proof display (before deletion).
You might want to figure out your first few proofs completely
and write them down by hand, before using the Proof Assistant, though
not everyone finds that effective.

{\em Important:}
The \texttt{undo} command if very helpful when entering a proof, because
it allows you to undo a previously-entered step.
In addition, we suggest that you
keep track of your work with a log file (\texttt{open
log}) and save it frequently (\texttt{save new{\char`\_}proof},
\texttt{write source}).
You can use \texttt{delete} to reverse an \texttt{assign}.
You can also do \texttt{delete floating{\char`\_}hypotheses}, then
\texttt{initialize all}, then \texttt{unify all /interactive} to
reinitialize bad unifications made accidentally or by bad
\texttt{assign}s.  You cannot reverse a \texttt{delete} except by
a relevant \texttt{undo} or using
\texttt{exit /force} then reentering the Proof Assistant to recover from
the last \texttt{save new{\char`\_}proof}.

The following commands are available in the Proof Assistant (at the
\texttt{MM-PA>} prompt) to help you create your proof.  See the
individual commands for more detail.

\begin{itemize}
\item[]
    \texttt{show new{\char`\_}proof} [\texttt{/all},...] - Displays the
        proof in progress.  You will use this command a lot; see \texttt{help
        show new{\char`\_}proof} to become familiar with its qualifiers.  The
        qualifiers \texttt{/unknown} and \texttt{/not{\char`\_}unified} are
        useful for seeing the work remaining to be done.  The combination
        \texttt{/all/unknown} is useful for identifying dummy variables that must be
        assigned, or attempts to use illegal syntax, when \texttt{improve all}
        is unable to complete the syntax constructions.  Unknown variables are
        shown as \texttt{\$1}, \texttt{\$2},...
\item[]
    \texttt{assign} {\em step} {\em label} - Assigns an unknown {\em step}
        number with the statement
        specified by {\em label}.
\item[]
    \texttt{let variable} {\em variable}
        \texttt{= "}{\em symbol sequence}\texttt{"}
          - Forces a symbol
        sequence to replace an unknown variable (such as \texttt{\$1}) in a proof.
        It is useful
        for helping difficult unifications, and it is necessary when you have
        dummy variables that eventually must be assigned a name.
\item[]
    \texttt{let step} {\em step} \texttt{= "}{\em symbol sequence}\texttt{"} -
          Forces a symbol sequence
        to replace the contents of a proof step, provided it can be
        unified with the existing step contents.  (I rarely use this.)
\item[]
    \texttt{unify step} {\em step} (or \texttt{unify all}) - Unifies
        the source and target of
        a step.  If you specify a specific step, you will be prompted
        to select among the unifications that are possible.  If you
        specify \texttt{all}, all steps with unique unifications, but only
        those steps, will be
        unified.  \texttt{unify all /interactive} goes through all non-unified
        steps.
\item[]
    \texttt{initialize} {\em step} (or \texttt{all}) - De-unifies the target and source of
        a step (or all steps), as well as the hypotheses of the source,
        and makes all variables in the source unknown.  Useful to recover from
        an \texttt{assign} or \texttt{let} mistake that
        resulted in incorrect unifications.
\item[]
    \texttt{delete} {\em step} (or \texttt{all} or \texttt{floating{\char`\_}hypotheses}) -
        Deletes the specified
        step(s).  \texttt{delete floating{\char`\_}hypotheses}, then \texttt{initialize all}, then
        \texttt{unify all /interactive} is useful for recovering from mistakes
        where incorrect unifications assigned wrong math symbol strings to
        variables.
\item[]
    \texttt{improve} {\em step} (or \texttt{all}) -
      Automatically creates a proof for steps (with no unknown variables)
      whose proof requires no statements with \texttt{\$e} hypotheses.  Useful
      for filling in proofs of \texttt{\$f} hypotheses.  The \texttt{/depth}
      qualifier will also try statements whose \texttt{\$e} hypotheses contain
      no new variables.  {\em Warning:} Save your work (with \texttt{save
      new{\char`\_}proof} then \texttt{write source}) before using
      \texttt{/depth = 2} or greater, since the search time grows
      exponentially and may never terminate in a reasonable time, and you
      cannot interrupt the search.  I have found that it is rare for
      \texttt{/depth = 3} or greater to be useful.
 \item[]
    \texttt{save new{\char`\_}proof} - Saves the proof in progress in the program's
        internal database buffer.  To save it permanently into the database file,
        use \texttt{write source} after
        \texttt{save new{\char`\_}proof}.  To revert to the last
        \texttt{save new{\char`\_}proof},
        \texttt{exit /force} from the Proof Assistant then re-enter the Proof
        Assistant.
 \item[]
    \texttt{match step} {\em step} (or \texttt{match all}) - Shows what
        statements are
        possibilities for the \texttt{assign} statement. (This command
        is not very
        useful in its present form and hopefully will be improved
        eventually.  In the meantime, use the \texttt{search} statement for
        candidates matching specific math token combinations.)
 \item[]
 \texttt{minimize{\char`\_}with}\index{\texttt{minimize{\char`\_}with} command}
% 3/10/07 Note: line-breaking the above results in duplicate index entries
     - After a proof is complete, this command will attempt
        to match other database theorems to the proof to see if the proof
        size can be reduced as a result.  See \texttt{help
        minimize{\char`\_}with} in the
        Metamath program for its usage.
 \item[]
 \texttt{undo}\index{\texttt{undo} command}
    - Undo the effect of a proof-changing command (all but the
      \texttt{show} and \texttt{save} commands above).
 \item[]
 \texttt{redo}\index{\texttt{redo} command}
    - Reverse the previous \texttt{undo}.
\end{itemize}

The following commands set parameters that may be relevant to your proof.
Consult the individual \texttt{help set}... commands.
\begin{itemize}
   \item[] \texttt{set unification{\char`\_}timeout}
 \item[]
    \texttt{set search{\char`\_}limit}
  \item[]
    \texttt{set empty{\char`\_}substitution} - note that default is \texttt{off}
\end{itemize}

Type \texttt{exit} to exit the \texttt{MM-PA>}
 prompt and get back to the \texttt{MM>} prompt.
Another \texttt{exit} will then get you out of Metamath.



\subsection{\texttt{prove} Command}\index{\texttt{prove} command}
Syntax:  \texttt{prove} {\em label}

This command will enter the Proof Assistant\index{Proof Assistant}, which will
allow you to create or edit the proof of the specified statement.
The command-line prompt will change from \texttt{MM>} to \texttt{MM-PA>}.

Note:  In the present version of
Metamath\index{Metamath!limitations of version 0.07.30}, the Proof
Assistant does not verify that \texttt{\$d}\index{\texttt{\$d}
statement} restrictions are met as a proof is being built.  After you
have completed a proof, you should type \texttt{save new{\char`\_}proof}
followed by \texttt{verify proof} {\em label} (where {\em label} is the
statement you are proving with the \texttt{prove} command) to verify the
\texttt{\$d} restrictions.

See also: \texttt{exit}

\subsection{\texttt{set unification\_timeout} Command}\index{\texttt{set
unification{\char`\_}timeout} command}
Syntax:  \texttt{set unification{\char`\_}timeout} {\em number}

(This command is available outside the Proof Assistant but affects the
Proof Assistant\index{Proof Assistant} only.)

Sometimes the Proof Assistant will inform you that a unification
time-out occurred.  This may happen when you try to \texttt{unify}
formulas with many temporary variables\index{temporary variable}
(\texttt{\$1}, \texttt{\$2}, etc.), since the time to compute all possible
unifications may grow exponentially with the number of variables.  If
you want Metamath to try harder (and you're willing to wait longer) you
may increase this parameter.  \texttt{show settings} will show you the
current value.

\subsection{\texttt{set empty\_substitution} Command}\index{\texttt{set
empty{\char`\_}substitution} command}
% These long names can't break well in narrow mode, and even "sloppy"
% is not enough. Work around this by not demanding justification.
\begin{flushleft}
Syntax:  \texttt{set empty{\char`\_}substitution on} or \texttt{set
empty{\char`\_}substitution off}
\end{flushleft}

(This command is available outside the Proof Assistant but affects the
Proof Assistant\index{Proof Assistant} only.)

The Metamath language allows variables to be
substituted\index{substitution!variable}\index{variable substitution}
with empty symbol sequences\index{empty substitution}.  However, in many
formal systems\index{formal system} this will never happen in a valid
proof.  Allowing for this possibility increases the likelihood of
ambiguous unifications\index{ambiguous
unification}\index{unification!ambiguous} during proof creation.
The default is that
empty substitutions are not allowed; for formal systems requiring them,
you must \texttt{set empty{\char`\_}substitution on}.
(An example where this must be \texttt{on}
would be a system that implements a Deduction Rule and in
which deductions from empty assumption lists would be permissible.  The
MIU-system\index{MIU-system} described in Appendix~\ref{MIU} is another
example.)
Note that empty substitutions are
always permissible in proof verification (VERIFY PROOF...) outside the
Proof Assistant.  (See the MIU system in the Metamath book for an example
of a system needing empty substitutions; another example would be a
system that implements a Deduction Rule and in which deductions from
empty assumption lists would be permissible.)

It is better to leave this \texttt{off} when working with \texttt{set.mm}.
Note that this command does not affect the way proofs are verified with
the \texttt{verify proof} command.  Outside of the Proof Assistant,
substitution of empty sequences for math symbols is always allowed.

\subsection{\texttt{set search\_limit} Command}\index{\texttt{set
search{\char`\_}limit} command} Syntax:  \texttt{set search{\char`\_}limit} {\em
number}

(This command is available outside the Proof Assistant but affects the
Proof Assistant\index{Proof Assistant} only.)

This command sets a parameter that determines when the \texttt{improve} command
in Proof Assistant mode gives up.  If you want \texttt{improve} to search harder,
you may increase it.  The \texttt{show settings} command tells you its current
value.


\subsection{\texttt{show new\_proof} Command}\index{\texttt{show
new{\char`\_}proof} command}
Syntax:  \texttt{show new{\char`\_}proof} [{\em
qualifiers} (see below)]

This command (available only in Proof Assistant mode) displays the proof
in progress.  It is identical to the \texttt{show proof} command, except that
there is no statement argument (since it is the statement being proved) and
the following qualifiers are not available:

    \texttt{/statement{\char`\_}summary}

    \texttt{/detailed{\char`\_}step}

Also, the following additional qualifiers are available:

    \texttt{/unknown} - Shows only steps that have no statement assigned.

    \texttt{/not{\char`\_}unified} - Shows only steps that have not been unified.

Note that \texttt{/essential}, \texttt{/depth}, \texttt{/unknown}, and
\texttt{/not{\char`\_}unified} may be
used in any combination; each of them effectively filters out additional
steps from the proof display.

See also:  \texttt{show proof}






\subsection{\texttt{assign} Command}\index{\texttt{assign} command}
Syntax:   \texttt{assign} {\em step} {\em label} [\texttt{/no{\char`\_}unify}]

   and:   \texttt{assign first} {\em label}

   and:   \texttt{assign last} {\em label}


This command, available in the Proof Assistant only, assigns an unknown
step (one with \texttt{?} in the \texttt{show new{\char`\_}proof}
listing) with the statement specified by {\em label}.  The assignment
will not be allowed if the statement cannot be unified with the step.

If \texttt{last} is specified instead of {\em step} number, the last
step that is shown by \texttt{show new{\char`\_}proof /unknown} will be
used.  This can be useful for building a proof with a command file (see
\texttt{help submit}).  It also makes building proofs faster when you know
the assignment for the last step.

If \texttt{first} is specified instead of {\em step} number, the first
step that is shown by \texttt{show new{\char`\_}proof /unknown} will be
used.

If {\em step} is zero or negative, the -{\em step}th from last unknown
step, as shown by \texttt{show new{\char`\_}proof /unknown}, will be
used.  \texttt{assign -1} {\em label} will assign the penultimate
unknown step, \texttt{assign -2} {\em label} the antepenultimate, and
\texttt{assign 0} {\em label} is the same as \texttt{assign last} {\em
label}.

Optional command qualifier:

    \texttt{/no{\char`\_}unify} - do not prompt user to select a unification if there is
        more than one possibility.  This is useful for noninteractive
        command files.  Later, the user can \texttt{unify all /interactive}.
        (The assignment will still be automatically unified if there is only
        one possibility and will be refused if unification is not possible.)



\subsection{\texttt{match} Command}\index{\texttt{match} command}
Syntax:  \texttt{match step} {\em step} [\texttt{/max{\char`\_}essential{\char`\_}hyp}
{\em number}]

    and:  \texttt{match all} [\texttt{/essential}]
          [\texttt{/max{\char`\_}essential{\char`\_}hyp} {\em number}]

This command, available in the Proof Assistant only, shows what
statements can be unified with the specified step(s).  {\em Note:} In
its current form, this command is not very useful because of the large
number of matches it reports.  I rarely use it.
It may be enhanced in the future.

Optional command qualifiers:

    \texttt{/max{\char`\_}essential{\char`\_}hyp} {\em number} - filters out
        of the list any statements
        with more than the specified number of
        \texttt{\$e}\index{\texttt{\$e} statement} hypotheses

    \texttt{/essential{\char`\_}only} - in the \texttt{match all} statement, only
        the steps that
        would be listed in the \texttt{show new{\char`\_}proof /essential} display are
        matched.



\subsection{\texttt{let} Command}\index{\texttt{let} command}
Syntax: \texttt{let variable} {\em variable} = \verb/"/{\em symbol-sequence}\verb/"/

  and: \texttt{let step} {\em step} = \verb/"/{\em symbol-sequence}\verb/"/

These commands, available in the Proof Assistant\index{Proof Assistant}
only, assign a temporary variable\index{temporary variable} or unknown
step with a specific symbol sequence.  They are useful in the middle of
creating a proof, when you know what should be in the proof step but the
unification algorithm doesn't yet have enough information to completely
specify the temporary variables.  A ``temporary variable'' is one that
has the form \texttt{\$}{\em nn} in the proof display, such as
\texttt{\$1}, \texttt{\$2}, etc.  The {\em symbol-sequence} may contain
other unknown variables if desired.  Examples:

    \verb/let variable $32 = "A = B"/

    \verb/let variable $32 = "A = $35"/

    \verb/let step 10 = '|- x = x'/

    \verb/let step -2 = "|- ( $7 = ph )"/

Any symbol sequence will be accepted for the \texttt{let variable}
command.  Only those symbol sequences that can be unified with the step
will be accepted for \texttt{let step}.

The \texttt{let} commands ``zap'' the proof with information that can
only be verified when the proof is built up further.  If you make an
error, the command sequence \texttt{delete
floating{\char`\_}hypotheses}, \texttt{initialize all}, and
\texttt{unify all /interactive} will undo a bad \texttt{let} assignment.

If {\em step} is zero or negative, the -{\em step}th from last unknown
step, as shown by \texttt{show new{\char`\_}proof /unknown}, will be
used.  The command \texttt{let step 0} = \verb/"/{\em
symbol-sequence}\verb/"/ will use the last unknown step, \texttt{let
step -1} = \verb/"/{\em symbol-sequence}\verb/"/ the penultimate, etc.
If {\em step} is positive, \texttt{let step} may be used to assign known
(in the sense of having previously been assigned a label with
\texttt{assign}) as well as unknown steps.

Either single or double quotes can surround the {\em symbol-sequence} as
long as they are different from any quotes inside a {\em
symbol-sequence}.  If {\em symbol-sequence} contains both kinds of
quotes, see the instructions at the end of \texttt{help let} in the
Metamath program.


\subsection{\texttt{unify} Command}\index{\texttt{unify} command}
Syntax:  \texttt{unify step} {\em step}

      and:   \texttt{unify all} [\texttt{/interactive}]

These commands, available in the Proof Assistant only, unify the source
and target of the specified step(s). If you specify a specific step, you
will be prompted to select among the unifications that are possible.  If
you specify \texttt{all}, only those steps with unique unifications will be
unified.

Optional command qualifier for \texttt{unify all}:

    \texttt{/interactive} - You will be prompted to select among the
        unifications
        that are possible for any steps that do not have unique
        unifications.  (Otherwise \texttt{unify all} will bypass these.)

See also \texttt{set unification{\char`\_}timeout}.  The default is
100000, but increasing it to 1000000 can help difficult cases.  Manually
assigning some or all of the unknown variables with the \texttt{let
variable} command also helps difficult cases.



\subsection{\texttt{initialize} Command}\index{\texttt{initialize} command}
Syntax:  \texttt{initialize step} {\em step}

    and: \texttt{initialize all}

These commands, available in the Proof Assistant\index{Proof Assistant}
only, ``de-unify'' the target and source of a step (or all steps), as
well as the hypotheses of the source, and makes all variables in the
source and the source's hypotheses unknown.  This command is useful to
help recover from incorrect unifications that resulted from an incorrect
\texttt{assign}, \texttt{let}, or unification choice.  Part or all of
the command sequence \texttt{delete floating{\char`\_}hypotheses},
\texttt{initialize all}, and \texttt{unify all /interactive} will recover
from incorrect unifications.

See also:  \texttt{unify} and \texttt{delete}



\subsection{\texttt{delete} Command}\index{\texttt{delete} command}
Syntax:  \texttt{delete step} {\em step}

   and:      \texttt{delete all} -- {\em Warning: dangerous!}

   and:      \texttt{delete floating{\char`\_}hypotheses}

These commands are available in the Proof Assistant only.  The
\texttt{delete step} command deletes the proof tree section that
branches off of the specified step and makes the step become unknown.
\texttt{delete all} is equivalent to \texttt{delete step} {\em step}
where {\em step} is the last step in the proof (i.e.\ the beginning of
the proof tree).

In most cases the \texttt{undo} command is the best way to undo
a previous step.
An alternative is to salvage your last \texttt{save
new{\char`\_}proof} by exiting and reentering the Proof Assistant.
For this to work, keep a log file open to record your work
and to do \texttt{save new{\char`\_}proof} frequently, especially before
\texttt{delete}.

\texttt{delete floating{\char`\_}hypotheses} will delete all sections of
the proof that branch off of \texttt{\$f}\index{\texttt{\$f} statement}
statements.  It is sometimes useful to do this before an
\texttt{initialize} command to recover from an error.  Note that once a
proof step with a \texttt{\$f} hypothesis as the target is completely
known, the \texttt{improve} command can usually fill in the proof for
that step.  Unlike the deletion of logical steps, \texttt{delete
floating{\char`\_}hypotheses} is a relatively safe command that is
usually easy to recover from.



\subsection{\texttt{improve} Command}\index{\texttt{improve} command}
\label{improve}
Syntax:  \texttt{improve} {\em step} [\texttt{/depth} {\em number}]
                                               [\texttt{/no{\char`\_}distinct}]

   and:   \texttt{improve first} [\texttt{/depth} {\em number}]
                                              [\texttt{/no{\char`\_}distinct}]

   and:   \texttt{improve last} [\texttt{/depth} {\em number}]
                                              [\texttt{/no{\char`\_}distinct}]

   and:   \texttt{improve all} [\texttt{/depth} {\em number}]
                                              [\texttt{/no{\char`\_}distinct}]

These commands, available in the Proof Assistant\index{Proof Assistant}
only, try to find proofs automatically for unknown steps whose symbol
sequences are completely known.  They are primarily useful for filling in
proofs of \texttt{\$f}\index{\texttt{\$f} statement} hypotheses.  The
search will be restricted to statements having no
\texttt{\$e}\index{\texttt{\$e} statement} hypotheses.

\begin{sloppypar} % narrow
Note:  If memory is limited, \texttt{improve all} on a large proof may
overflow memory.  If you use \texttt{set unification{\char`\_}timeout 1}
before \texttt{improve all}, there will usually be sufficient
improvement to easily recover and completely \texttt{improve} the proof
later on a larger computer.  Warning:  Once memory has overflowed, there
is no recovery.  If in doubt, save the intermediate proof (\texttt{save
new{\char`\_}proof} then \texttt{write source}) before \texttt{improve
all}.
\end{sloppypar}

If \texttt{last} is specified instead of {\em step} number, the last
step that is shown by \texttt{show new{\char`\_}proof /unknown} will be
used.

If \texttt{first} is specified instead of {\em step} number, the first
step that is shown by \texttt{show new{\char`\_}proof /unknown} will be
used.

If {\em step} is zero or negative, the -{\em step}th from last unknown
step, as shown by \texttt{show new{\char`\_}proof /unknown}, will be
used.  \texttt{improve -1} will use the penultimate
unknown step, \texttt{improve -2} {\em label} the antepenultimate, and
\texttt{improve 0} is the same as \texttt{improve last}.

Optional command qualifier:

    \texttt{/depth} {\em number} - This qualifier will cause the search
        to include
        statements with \texttt{\$e} hypotheses (but no new variables in
        the \texttt{\$e}
        hypotheses), provided that the backtracking has not exceeded the
        specified depth. {\em Warning:}  Try \texttt{/depth 1},
        then \texttt{2}, then \texttt{3}, etc.
        in sequence because of possible exponential blowups.  Save your
        work before trying \texttt{/depth} greater than \texttt{1}!

    \texttt{/no{\char`\_}distinct} - Skip trial statements that have
        \texttt{\$d}\index{\texttt{\$d} statement} requirements.
        This qualifier will prevent assignments that might violate \texttt{\$d}
        requirements but it also could miss possible legal assignments.

See also: \texttt{set search{\char`\_}limit}

\subsection{\texttt{save new\_proof} Command}\index{\texttt{save
new{\char`\_}proof} command}
Syntax:  \texttt{save new{\char`\_}proof} {\em label} [\texttt{/normal}]
   [\texttt{/compressed}]

The \texttt{save new{\char`\_}proof} command is available in the Proof
Assistant only.  It saves the proof in progress in the source
buffer\index{source buffer}.  \texttt{save new{\char`\_}proof} may be
used to save a completed proof, or it may be used to save a proof in
progress in order to work on it later.  If an incomplete proof is saved,
any user assignments with \texttt{let step} or \texttt{let variable}
will be lost, as will any ambiguous unifications\index{ambiguous
unification}\index{unification!ambiguous} that were resolved manually.
To help make recovery easier, it can be helpful to \texttt{improve all}
before \texttt{save new{\char`\_}proof} so that the incomplete proof
will have as much information as possible.

Note that the proof will not be permanently saved until a \texttt{write
source} command is issued.

Optional command qualifiers:

    \texttt{/normal} - The proof is saved in the normal format (i.e., as a
        sequence of labels, which is the defined format of the basic Metamath
        language).\index{basic language}  This is the default format that
        is used if a qualifier is omitted.

    \texttt{/compressed} - The proof is saved in the compressed format, which
        reduces storage requirements for a database.
        (See Appendix~\ref{compressed}.)


\section{Creating \LaTeX\ Output}\label{texout}\index{latex@{\LaTeX}}

You can generate \LaTeX\ output given the
information in a database.
The database must already include the necessary typesetting information
(see section \ref{tcomment} for how to provide this information).

The \texttt{show statement} and \texttt{show proof} commands each have a
special \texttt{/tex} command qualifier that produces \LaTeX\ output.
(The \texttt{show statement} command also has the
\texttt{/simple{\char`\_}tex} qualifier for output that is easier to
edit by hand.)  Before you can use them, you must open a \LaTeX\ file to
which to send their output.  A typical complete session will use this
sequence of Metamath commands:

\begin{verbatim}
read set.mm
open tex example.tex
show statement a1i /tex
show proof a1i /all/lemmon/renumber/tex
show statement uneq2 /tex
show proof uneq2 /all/lemmon/renumber/tex
close tex
\end{verbatim}

See Section~\ref{mathcomments} for information on comment markup and
Appendix~\ref{ASCII} for information on how math symbol translation is
specified.

To format and print the \LaTeX\ source, you will need the \LaTeX\
program, which is standard on most Linux installations and available for
Windows.  On Linux, in order to create a {\sc pdf} file, you will
typically type at the shell prompt
\begin{verbatim}
$ pdflatex example.tex
\end{verbatim}

\subsection{\texttt{open tex} Command}\index{\texttt{open tex} command}
Syntax:  \texttt{open tex} {\em file-name} [\texttt{/no{\char`\_}header}]

This command opens a file for writing \LaTeX\
source\index{latex@{\LaTeX}} and writes a \LaTeX\ header to the file.
\LaTeX\ source can be written with the \texttt{show proof}, \texttt{show
new{\char`\_}proof}, and \texttt{show statement} commands using the
\texttt{/tex} qualifier.

The mapping to \LaTeX\ symbols is defined in a special comment
containing a \texttt{\$t} token, described in Appendix~\ref{ASCII}.

There is an optional command qualifier:

    \texttt{/no{\char`\_}header} - This qualifier prevents a standard
        \LaTeX\ header and trailer
        from being included with the output \LaTeX\ code.


\subsection{\texttt{close tex} Command}\index{\texttt{close tex} command}
Syntax:  \texttt{close tex}

This command writes a trailer to any \LaTeX\ file\index{latex@{\LaTeX}}
that was opened with \texttt{open tex} (unless
\texttt{/no{\char`\_}header} was used with \texttt{open tex}) and closes
the \LaTeX\ file.


\section{Creating {\sc HTML} Output}\label{htmlout}

You can generate {\sc html} web pages given the
information in a database.
The database must already include the necessary typesetting information
(see section \ref{tcomment} for how to provide this information).
The ability to produce {\sc html} web pages was added in Metamath version
0.07.30.

To create an {\sc html} output file(s) for \texttt{\$a} or \texttt{\$p}
statement(s), use
\begin{quote}
    \texttt{show statement} {\em label-match} \texttt{/html}
\end{quote}
The output file will be named {\em label-match}\texttt{.html}
for each match.  When {\em
label-match} has wildcard (\texttt{*}) characters, all statements with
matching labels will have {\sc html} files produced for them.  Also,
when {\em label-match} has a wildcard (\texttt{*}) character, two additional
files, \texttt{mmdefinitions.html} and \texttt{mmascii.html} will be
produced.  To produce {\em only} these two additional files, you can use
\texttt{?*}, which will not match any statement label, in place of {\em
label-match}.

There are three other qualifiers for \texttt{show statement} that also
generate {\sc HTML} code.  These are \texttt{/alt{\char`\_}html},
\texttt{/brief{\char`\_}html}, and
\texttt{/brief{\char`\_}alt{\char`\_}html}, and are described in the
next section.

The command
\begin{quote}
    \texttt{show statement} {\em label-match} \texttt{/alt{\char`\_}html}
\end{quote}
does the same as \texttt{show statement} {\em label-match} \texttt{/html},
except that the {\sc html} code for the symbols is taken from
\texttt{althtmldef} statements instead of \texttt{htmldef} statements in
the \texttt{\$t} comment.

The command
\begin{verbatim}
show statement * /brief_html
\end{verbatim}
invokes a special mode that just produces definition and theorem lists
accompanied by their symbol strings, in a format suitable for copying and
pasting into another web page (such as the tutorial pages on the
Metamath web site).

Finally, the command
\begin{verbatim}
show statement * /brief_alt_html
\end{verbatim}
does the same as \texttt{show statement * / brief{\char`\_}html}
for the alternate {\sc html}
symbol representation.

A statement's comment can include a special notation that provides a
certain amount of control over the {\sc HTML} version of the comment.  See
Section~\ref{mathcomments} (p.~\pageref{mathcomments}) for the comment
markup features.

The \texttt{write theorem{\char`\_}list} and \texttt{write bibliography}
commands, which are described below, provide as a side effect complete
error checking for all of the features described in this
section.\index{error checking}

\subsection{\texttt{write theorem\_list}
Command}\index{\texttt{write theorem{\char`\_}list} command}
Syntax:  \texttt{write theorem{\char`\_}list}
[\texttt{/theorems{\char`\_}per{\char`\_}page} {\em number}]

This command writes a list of all of the \texttt{\$a} and \texttt{\$p}
statements in the database into a web page file
 called \texttt{mmtheorems.html}.
When additional files are needed, they are called
\texttt{mmtheorems2.html}, \texttt{mmtheorems3.html}, etc.

Optional command qualifier:

    \texttt{/theorems{\char`\_}per{\char`\_}page} {\em number} -
 This qualifier specifies the number of statements to
        write per web page.  The default is 100.

{\em Note:} In version 0.07.30\index{Metamath!limitations of version
0.07.30} of Metamath, the ``Nearby theorems'' links on the individual
web pages presuppose 100 theorems per page when linking to the theorem
list pages.  Therefore the \texttt{/theorems{\char`\_}per{\char`\_}page}
qualifier, if it specifies a number other than 100, will cause the
individual web pages to be out of sync and should not be used to
generate the main theorem list for the web site.  This is expected to be
fixed in a future version.


\subsection{\texttt{write bibliography}\label{wrbib}
Command}\index{\texttt{write bibliography} command}
Syntax:  \texttt{write bibliography} {\em filename}

This command reads an existing {\sc html} bibliographic cross-reference
file, normally called \texttt{mmbiblio.html}, and updates it per the
bibliographic links in the database comments.  The file is updated
between the {\sc html} comment lines \texttt{<!--
{\char`\#}START{\char`\#} -->} and \texttt{<!-- {\char`\#}END{\char`\#}
-->}.  The original input file is renamed to {\em
filename}\texttt{{\char`\~}1}.

A bibliographic reference is indicated with the reference name
in brackets, such as  \texttt{Theorem 3.1 of
[Monk] p.\ 22}.
See Section~\ref{htmlout} (p.~\pageref{htmlout}) for
syntax details.


\subsection{\texttt{write recent\_additions}
Command}\index{\texttt{write recent{\char`\_}additions} command}
Syntax:  \texttt{write recent{\char`\_}additions} {\em filename}
[\texttt{/limit} {\em number}]

This command reads an existing ``Recent Additions'' {\sc html} file,
normally called \texttt{mmrecent.html}, and updates it with the
descriptions of the most recently added theorems to the database.
 The file is updated between
the {\sc html} comment lines \texttt{<!-- {\char`\#}START{\char`\#} -->}
and \texttt{<!-- {\char`\#}END{\char`\#} -->}.  The original input file
is renamed to {\em filename}\texttt{{\char`\~}1}.

Optional command qualifier:

    \texttt{/limit} {\em number} -
 This qualifier specifies the number of most recent theorems to
   write to the output file.  The default is 100.


\section{Text File Utilities}

\subsection{\texttt{tools} Command}\index{\texttt{tools} command}
Syntax:  \texttt{tools}

This command invokes an easy-to-use, general purpose utility for
manipulating the contents of {\sc ascii} text files.  Upon typing
\texttt{tools}, the command-line prompt will change to \texttt{TOOLS>}
until you type \texttt{exit}.  The \texttt{tools} commands can be used
to perform simple, global edits on an input/output file,
such as making a character string substitution on each line, adding a
string to each line, and so on.  A typical use of this utility is
to build a \texttt{submit} input file to perform a common operation on a
list of statements obtained from \texttt{show label} or \texttt{show
usage}.

The actions of most of the \texttt{tools} commands can also be
performed with equivalent (and more powerful) Unix shell commands, and
some users may find those more efficient.  But for Windows users or
users not comfortable with Unix, \texttt{tools} provides an
easy-to-learn alternative that is adequate for most of the
script-building tasks needed to use the Metamath program effectively.

\subsection{\texttt{help} Command (in \texttt{tools})}
Syntax:  \texttt{help}

The \texttt{help} command lists the commands available in the
\texttt{tools} utility, along with a brief description.  Each command,
in turn, has its own help, such as \texttt{help add}.  As with
Metamath's \texttt{MM>} prompt, a complete command can be entered at
once, or just the command word can be typed, causing the program to
prompt for each argument.

\vskip 1ex
\noindent Line-by-line editing commands:

  \texttt{add} - Add a specified string to each line in a file.

  \texttt{clean} - Trim spaces and tabs on each line in a file; convert
         characters.

  \texttt{delete} - Delete a section of each line in a file.

  \texttt{insert} - Insert a string at a specified column in each line of
        a file.

  \texttt{substitute} - Make a simple substitution on each line of the file.

  \texttt{tag} - Like \texttt{add}, but restricted to a range of lines.

  \texttt{swap} - Swap the two halves of each line in a file.

\vskip 1ex
\noindent Other file-processing commands:

  \texttt{break} - Break up (tokenize) a file into a list of tokens (one per
        line).

  \texttt{build} - Build a file with multiple tokens per line from a list.

  \texttt{count} - Count the occurrences in a file of a specified string.

  \texttt{number} - Create a list of numbers.

  \texttt{parallel} - Put two files in parallel.

  \texttt{reverse} - Reverse the order of the lines in a file.

  \texttt{right} - Right-justify lines in a file (useful before sorting
         numbers).

%  \texttt{tag} - Tag edit updates in a program for revision control.

  \texttt{sort} - Sort the lines in a file with key starting at
         specified string.

  \texttt{match} - Extract lines containing (or not) a specified string.

  \texttt{unduplicate} - Eliminate duplicate occurrences of lines in a file.

  \texttt{duplicate} - Extract first occurrence of any line occurring
         more than

   \ \ \    once in a file, discarding lines occurring exactly once.

  \texttt{unique} - Extract lines occurring exactly once in a file.

  \texttt{type} (10 lines) - Display the first few lines in a file.
                  Similar to Unix \texttt{head}.

  \texttt{copy} - Similar to Unix \texttt{cat} but safe (same input
         and output file allowed).

  \texttt{submit} - Run a script containing \texttt{tools} commands.

\vskip 1ex

\noindent Note:
  \texttt{unduplicate}, \texttt{duplicate}, and \texttt{unique} also
 sort the lines as a side effect.


\subsection{Using \texttt{tools} to Build Metamath \texttt{submit}
Scripts}

The \texttt{break} command is typically used to break up a series of
statement labels, such as the output of Metamath's \texttt{show usage},
into one label per line.  The other \texttt{tools} commands can then be
used to add strings before and after each statement label to specify
commands to be performed on the statement.  The \texttt{parallel}
command is useful when a statement label must be mentioned more than
once on a line.

Very often a \texttt{submit} script for Metamath will require multiple
command lines for each statement being processed.  For example, you may
want to enter the Proof Assistant, \texttt{minimize{\char`\_}with} your
latest theorem, \texttt{save} the new proof, and \texttt{exit} the Proof
Assistant.  To accomplish this, you can build a file with these four
commands for each statement on a single line, separating each command
with a designated character such as \texttt{@}.  Then at the end you can
\texttt{substitute} each \texttt{@} with \texttt{{\char`\\}n} to break
up the lines into individual command lines (see \texttt{help
substitute}).


\subsection{Example of a \texttt{tools} Session}

To give you a quick feel for the \texttt{tools} utility, we show a
simple session where we create a file \texttt{n.txt} with 3 lines, add
strings before and after each line, and display the lines on the screen.
You can experiment with the various commands to gain experience with the
\texttt{tools} utility.

\begin{verbatim}
MM> tools
Entering the Text Tools utilities.
Type HELP for help, EXIT to exit.
TOOLS> number
Output file <n.tmp>? n.txt
First number <1>?
Last number <10>? 3
Increment <1>?
TOOLS> add
Input/output file? n.txt
String to add to beginning of each line <>? This is line
String to add to end of each line <>? .
The file n.txt has 3 lines; 3 were changed.
First change is on line 1:
This is line 1.
TOOLS> type n.txt
This is line 1.
This is line 2.
This is line 3.
TOOLS> exit
Exiting the Text Tools.
Type EXIT again to exit Metamath.
MM>
\end{verbatim}



\appendix
\chapter{Sample Representations}
\label{ASCII}

This Appendix provides a sample of {\sc ASCII} representations,
their corresponding traditional mathematical symbols,
and a discussion of their meanings
in the \texttt{set.mm} database.
These are provided in order of appearance.
This is only a partial list, and new definitions are routinely added.
A complete list is available at \url{http://metamath.org}.

These {\sc ASCII} representations, along
with information on how to display them,
are defined in the \texttt{set.mm} database file inside
a special comment called a \texttt{\$t} {\em
comment}\index{\texttt{\$t} comment} or {\em typesetting
comment.}\index{typesetting comment}
A typesetting comment
is indicated by the appearance of the
two-character string \texttt{\$t} at the beginning of the comment.
For more information,
see Section~\ref{tcomment}, p.~\pageref{tcomment}.

In the following table the ``{\sc ASCII}'' column shows the {\sc ASCII}
representation,
``Symbol'' shows the mathematical symbolic display
that corresponds to that {\sc ASCII} representation, ``Labels'' shows
the key label(s) that define the representation, and
``Description'' provides a description about the symbol.
As usual, ``iff'' is short for ``if and only if.''\index{iff}
In most cases the ``{\sc ASCII}'' column only shows
the key token, but it will sometimes show a sequence of tokens
if that is necessary for clarity.

{\setlength{\extrarowsep}{4pt} % Keep rows from being too close together
\begin{longtabu}   { @{} c c l X }
\textbf{ASCII} & \textbf{Symbol} & \textbf{Labels} & \textbf{Description} \\
\endhead
\texttt{|-} & $\vdash$ & &
  ``It is provable that...'' \\
\texttt{ph} & $\varphi$ & \texttt{wph} &
  The wff (boolean) variable phi,
  conventionally the first wff variable. \\
\texttt{ps} & $\psi$ & \texttt{wps} &
  The wff (boolean) variable psi,
  conventionally the second wff variable. \\
\texttt{ch} & $\chi$ & \texttt{wch} &
  The wff (boolean) variable chi,
  conventionally the third wff variable. \\
\texttt{-.} & $\lnot$ & \texttt{wn} &
  Logical not. E.g., if $\varphi$ is true, then $\lnot \varphi$ is false. \\
\texttt{->} & $\rightarrow$ & \texttt{wi} &
  Implies, also known as material implication.
  In classical logic the expression $\varphi \rightarrow \psi$ is true
  if either $\varphi$ is false or $\psi$ is true (or both), that is,
  $\varphi \rightarrow \psi$ has the same meaning as
  $\lnot \varphi \lor \psi$ (as proven in theorem \texttt{imor}). \\
\texttt{<->} & $\leftrightarrow$ &
  \hyperref[df-bi]{\texttt{df-bi}} &
  Biconditional (aka is-equals for boolean values).
  $\varphi \leftrightarrow \psi$ is true iff
  $\varphi$ and $\psi$ have the same value. \\
\texttt{\char`\\/} & $\lor$ &
  \makecell[tl]{{\hyperref[df-or]{\texttt{df-or}}}, \\
	         \hyperref[df-3or]{\texttt{df-3or}}} &
  Disjunction (logical ``or''). $\varphi \lor \psi$ is true iff
  $\varphi$, $\psi$, or both are true. \\
\texttt{/\char`\\} & $\land$ &
  \makecell[tl]{{\hyperref[df-an]{\texttt{df-an}}}, \\
                 \hyperref[df-3an]{\texttt{df-3an}}} &
  Conjunction (logical ``and''). $\varphi \land \psi$ is true iff
  both $\varphi$ and $\psi$ are true. \\
\texttt{A.} & $\forall$ &
  \texttt{wal} &
  For all; the wff $\forall x \varphi$ is true iff
  $\varphi$ is true for all values of $x$. \\
\texttt{E.} & $\exists$ &
  \hyperref[df-ex]{\texttt{df-ex}} &
  There exists; the wff
  $\exists x \varphi$ is true iff
  there is at least one $x$ where $\varphi$ is true. \\
\texttt{[ y / x ]} & $[ y / x ]$ &
  \hyperref[df-sb]{\texttt{df-sb}} &
  The wff $[ y / x ] \varphi$ produces
  the result when $y$ is properly substituted for $x$ in $\varphi$
  ($y$ replaces $x$).
  % This is elsb4
  % ( [ x / y ] z e. y <-> z e. x )
  For example,
  $[ x / y ] z \in y$ is the same as $z \in x$. \\
\texttt{E!} & $\exists !$ &
  \hyperref[df-eu]{\texttt{df-eu}} &
  There exists exactly one;
  $\exists ! x \varphi$ is true iff
  there is at least one $x$ where $\varphi$ is true. \\
\texttt{\{ y | phi \}}  & $ \{ y | \varphi \}$ &
  \hyperref[df-clab]{\texttt{df-clab}} &
  The class of all sets where $\varphi$ is true. \\
\texttt{=} & $ = $ &
  \hyperref[df-cleq]{\texttt{df-cleq}} &
  Class equality; $A = B$ iff $A$ equals $B$. \\
\texttt{e.} & $ \in $ &
  \hyperref[df-clel]{\texttt{df-clel}} &
  Class membership; $A \in B$ if $A$ is a member of $B$. \\
\texttt{{\char`\_}V} & {\rm V} &
  \hyperref[df-v]{\texttt{df-v}} &
  Class of all sets (not itself a set). \\
\texttt{C\_} & $ \subseteq $ &
  \hyperref[df-ss]{\texttt{df-ss}} &
  Subclass (subset); $A \subseteq B$ is true iff
  $A$ is a subclass of $B$. \\
\texttt{u.} & $ \cup $ &
  \hyperref[df-un]{\texttt{df-un}} &
  $A \cup B$ is the union of classes $A$ and $B$. \\
\texttt{i^i} & $ \cap $ &
  \hyperref[df-in]{\texttt{df-in}} &
  $A \cap B$ is the intersection of classes $A$ and $B$. \\
\texttt{\char`\\} & $ \setminus $ &
  \hyperref[df-dif]{\texttt{df-dif}} &
  $A \setminus B$ (set difference)
  is the class of all sets in $A$ except for those in $B$. \\
\texttt{(/)} & $ \varnothing $ &
  \hyperref[df-nul]{\texttt{df-nul}} &
  $ \varnothing $ is the empty set (aka null set). \\
\texttt{\char`\~P} & $ \cal P $ &
  \hyperref[df-pw]{\texttt{df-pw}} &
  Power class. \\
\texttt{<.\ A , B >.} & $\langle A , B \rangle$ &
  \hyperref[df-op]{\texttt{df-op}} &
  The ordered pair $\langle A , B \rangle$. \\
\texttt{( F ` A )} & $ ( F ` A ) $ &
  \hyperref[df-fv]{\texttt{df-fv}} &
  The value of function $F$ when applied to $A$. \\
\texttt{\_i} & $ i $ &
  \texttt{df-i} &
  The square root of negative one. \\
\texttt{x.} & $ \cdot $ &
  \texttt{df-mul} &
  Complex number multiplication; $2~\cdot~3~=~6$. \\
\texttt{CC} & $ \mathbb{C} $ &
  \texttt{df-c} &
  The set of complex numbers. \\
\texttt{RR} & $ \mathbb{R} $ &
  \texttt{df-r} &
  The set of real numbers. \\
\end{longtabu}
} % end of extrarowsep

\chapter{Compressed Proofs}
\label{compressed}\index{compressed proof}\index{proof!compressed}

The proofs in the \texttt{set.mm} set theory database are stored in compressed
format for efficiency.  Normally you needn't concern yourself with the
compressed format, since you can display it with the usual proof display tools
in the Metamath program (\texttt{show proof}\ldots) or convert it to the normal
RPN proof format described in Section~\ref{proof} (with \texttt{save proof}
{\em label} \texttt{/normal}).  However for sake of completeness we describe the
format here and show how it maps to the normal RPN proof format.

A compressed proof, located between \texttt{\$=} and \texttt{\$.}\ keywords, consists
of a left parenthesis, a sequence of statement labels, a right parenthesis,
and a sequence of upper-case letters \texttt{A} through \texttt{Z} (with optional
white space between them).  White space must surround the parentheses
and the labels.  The left parenthesis tells Metamath that a
compressed proof follows.  (A normal RPN proof consists of just a sequence of
labels, and a parenthesis is not a legal character in a label.)

The sequence of upper-case letters corresponds to a sequence of integers
with the following mapping.  Each integer corresponds to a proof step as
described later.
\begin{center}
  \texttt{A} = 1 \\
  \texttt{B} = 2 \\
   \ldots \\
  \texttt{T} = 20 \\
  \texttt{UA} = 21 \\
  \texttt{UB} = 22 \\
   \ldots \\
  \texttt{UT} = 40 \\
  \texttt{VA} = 41 \\
  \texttt{VB} = 42 \\
   \ldots \\
  \texttt{YT} = 120 \\
  \texttt{UUA} = 121 \\
   \ldots \\
  \texttt{YYT} = 620 \\
  \texttt{UUUA} = 621 \\
   etc.
\end{center}

In other words, \texttt{A} through \texttt{T} represent the
least-significant digit in base 20, and \texttt{U} through \texttt{Y}
represent zero or more most-significant digits in base 5, where the
digits start counting at 1 instead of the usual 0. With this scheme, we
don't need white space between these ``numbers.''

(In the design of the compressed proof format, only upper-case letters,
as opposed to say all non-whitespace printable {\sc ascii} characters other than
%\texttt{\$}, was chosen to make the compressed proof a little less
%displeasing to the eye, at the expense of a typical 20\% compression
\texttt{\$}, were chosen so as not to collide with most text editor
searches, at the expense of a typical 20\% compression
loss.  The base 5/base 20 grouping, as opposed to say base 6/base 19,
was chosen by experimentally determining the grouping that resulted in
best typical compression.)

The letter \texttt{Z} identifies (tags) a proof step that is identical to one
that occurs later on in the proof; it helps shorten the proof by not requiring
that identical proof steps be proved over and over again (which happens often
when building wff's).  The \texttt{Z} is placed immediately after the
least-significant digit (letters \texttt{A} through \texttt{T}) that ends the integer
corresponding to the step to later be referenced.

The integers that the upper-case letters correspond to are mapped to labels as
follows.  If the statement being proved has $m$ mandatory hypotheses, integers
1 through $m$ correspond to the labels of these hypotheses in the order shown
by the \texttt{show statement ... / full} command, i.e., the RPN order\index{RPN
order} of the mandatory
hypotheses.  Integers $m+1$ through $m+n$ correspond to the labels enclosed in
the parentheses of the compressed proof, in the order that they appear, where
$n$ is the number of those labels.  Integers $m+n+1$ on up don't directly
correspond to statement labels but point to proof steps identified with the
letter \texttt{Z}, so that these proof steps can be referenced later in the
proof.  Integer $m+n+1$ corresponds to the first step tagged with a \texttt{Z},
$m+n+2$ to the second step tagged with a \texttt{Z}, etc.  When the compressed
proof is converted to a normal proof, the entire subproof of a step tagged
with \texttt{Z} replaces the reference to that step.

For efficiency, Metamath works with compressed proofs directly, without
converting them internally to normal proofs.  In addition to the usual
error-checking, an error message is given if (1) a label in the label list in
parentheses does not refer to a previous \texttt{\$p} or \texttt{\$a} statement or a
non-mandatory hypothesis of the statement being proved and (2) a proof step
tagged with \texttt{Z} is referenced before the step tagged with the \texttt{Z}.

Just as in a normal proof under development (Section~\ref{unknown}), any step
or subproof that is not yet known may be represented with a single \texttt{?}.
White space does not have to appear between the \texttt{?}\ and the upper-case
letters (or other \texttt{?}'s) representing the remainder of the proof.

% April 1, 2004 Appendix C has been added back in with corrections.
%
% May 20, 2003 Appendix C was removed for now because there was a problem found
% by Bob Solovay
%
% Also, removed earlier \ref{formalspec} 's (3 cases above)
%
% Bob Solovay wrote on 30 Nov 2002:
%%%%%%%%%%%%% (start of email comment )
%      3. My next set of comments concern appendix C. I read this before I
% read Chapter 4. So I first noted that the system as presented in the
% Appendix lacked a certain formal property that I thought desirable. I
% then came up with a revised formal system that had this property. Upon
% reading Chapter 4, I noticed that the revised system was closer to the
% treatment in Chapter 4 than the system in Appendix C.
%
%         First a very minor correction:
%
%         On page 142 line 2: The condition that V(e) != V(f) should only be
% required of e, f in T such that e != f.
%
%         Here is a natural property [transitivity] that one would like
% the formal system to have:
%
%         Let Gamma be a set of statements. Suppose that the statement Phi
% is provable from Gamma and that the statement Psi is provable from Gamma
% \cup {Phi}. Then Psi is provable from Gamma.
%
%         I shall present an example to show that this property does not
% hold for the formal systems of Appendix C:
%
%         I write the example in metamath style:
%
% $c A B C D E $.
% $v x y
%
% ${
% tx $f A x $.
% ty $f B y $.
% ax1 $a C x y $.
% $}
%
% ${
% tx $f A x $.
% ty $f B y $.
% ax2-h1 $e C x y $.
% ax2 $a D y $.
% $}
%
% ${
% ty $f B y $.
% ax3-h1 $e D y $.
% ax3 $a E y $.
% $}
%
% $(These three axioms are Gamma $)
%
% ${
% tx $f A x $.
% ty $f B y $.
% Phi $p D y $=
% tx ty tx ty ax1 ax2 $.
% $}
%
% ${
% ty $f B y $.
% Psi $p E y $=
% ty ty Phi ax3 $.
% $}
%
%
% I omit the formal proofs of the following claims. [I will be glad to
% supply them upon request.]
%
% 1) Psi is not provable from Gamma;
%
% 2) Psi is provable from Gamma + Phi.
%
% Here "provable" refers to the formalism of Appendix C.
%
% The trouble of course is that Psi is lacking the variable declaration
%
% $f Ax $.
%
% In the Metamath system there is no trouble proving Psi. I attach a
% metamath file that shows this and which has been checked by the
% metamath program.
%
% I next want to indicate how I think the treatment in Appendix C should
% be revised so as to conform more closely to the metamath system of the
% main text. The revised system *does* have the transitivity property.
%
% We want to give revised definitions of "statement" and
% "provable". [cf. sections C.2.4. and C.2.5] Our new definitions will
% use the definitions given in Appendix C. So we take the following
% tack. We refer to the original notions as o-statement and o-provable. And
% we refer to the notions we are defining as n-statement and n-provable.
%
%         A n-statement is an o-statement in which the only variables
% that appear in the T component are mandatory.
%
%         To any o-statement we can associate its reduct which is a
% n-statement by dropping all the elements of T or D which contain
% non-mandatory variables.
%
%         An n-statement gamma is n-provable if there is an o-statement
% gamma' which has gamma as its reduct andf such that gamma' is
% o-provable.
%
%         It seems to me [though I am not completely sure on this point]
% that n-provability corresponds to metamath provability as discussed
% say in Chapter 4.
%
%         Attached to this letter is the metamath proof of Phi and Psi
% from Gamma discussed above.
%
%         I am still brooding over the question of whether metamath
% correctly formalizes set-theory. No doubt I will have some questions
% re this after my thoughts become clearer.
%%%%%%%%%%%%%%%% (end of email comment)

%%%%%%%%%%%%%%%% (start of 2nd email comment from Bob Solovay 1-Apr-04)
%
%         I hope that Appendix C is the one that gives a "formal" treatment
% of Metamath. At any rate, thats the appendix I want to comment on.
%
%         I'm going to suggest two changes in the definition.
%
%         First change (in the definition of statement): Require that the
% sets D, T, and E be finite.
%
%         Probably things are fine as you give them. But in the applications
% to the main metamath system they will always be finite, and its useful in
% thinking about things [at least for me] to stick to the finite case.
%
%         Second change:
%
%         First let me give an approximate description. Remove the dummy
% variables from the statement. Instead, include them in the proof.
%
%         More formally: Require that T consists of type declarations only
% for mandatory variables. Require that all the pairs in D consist of
% mandatory variables.
%
%         At the start of a proof we are allowed to declare a finite number
% of dummy variables [provided that none of them appear in any of the
% statements in E \cup {A}. We have to supply type declarations for all the
% dummy variables. We are allowed to add new $d statements referring to
% either the mandatory or dummy variables. But we require that no new $d
% statement references only mandatory variables.
%
%         I find this way of doing things more conceptual than the treatment
% in Appendix C. But the change [which I will use implicitly in later
% letters about doing Peano] is mainly aesthetic. I definitely claim that my
% results on doing Peano all apply to Metamath as it is presented in your
% book.
%
%         --Bob
%
%%%%%%%%%%%%%%%% (end of 2nd email comment)

%%
%% When uncommenting the below, also uncomment references above to {formalspec}
%%
\chapter{Metamath's Formal System}\label{formalspec}\index{Metamath!as a formal
system}

\section{Introduction}

\begin{quote}
  {\em Perfection is when there is no longer anything more to take away.}
    \flushright\sc Antoine de
     Saint-Exupery\footnote{\cite[p.~3-25]{Campbell}.}\\
\end{quote}\index{de Saint-Exupery, Antoine}

This appendix describes the theory behind the Metamath language in an abstract
way intended for mathematicians.  Specifically, we construct two
set-theo\-ret\-i\-cal objects:  a ``formal system'' (roughly, a set of syntax
rules, axioms, and logical rules) and its ``universe'' (roughly, the set of
theorems derivable in the formal system).  The Metamath computer language
provides us with a way to describe specific formal systems and, with the aid of
a proof provided by the user, to verify that given theorems
belong to their universes.

To understand this appendix, you need a basic knowledge of informal set theory.
It should be sufficient to understand, for example, Ch.\ 1 of Munkres' {\em
Topology} \cite{Munkres}\index{Munkres, James R.} or the
introductory set theory chapter
in many textbooks that introduce abstract mathematics. (Note that there are
minor notational differences among authors; e.g.\ Munkres uses $\subset$ instead
of our $\subseteq$ for ``subset.''  We use ``included in'' to mean ``a subset
of,'' and ``belongs to'' or ``is contained in'' to mean ``is an element of.'')
What we call a ``formal'' description here, unlike earlier, is actually an
informal description in the ordinary language of mathematicians.  However we
provide sufficient detail so that a mathematician could easily formalize it,
even in the language of Metamath itself if desired.  To understand the logic
examples at the end of this appendix, familiarity with an introductory book on
mathematical logic would be helpful.

\section{The Formal Description}

\subsection[Preliminaries]{Preliminaries\protect\footnotemark}%
\footnotetext{This section is taken mostly verbatim
from Tarski \cite[p.~63]{Tarski1965}\index{Tarski, Alfred}.}

By $\omega$ we denote the set of all natural numbers (non-negative integers).
Each natural number $n$ is identified with the set of all smaller numbers: $n =
\{ m | m < n \}$.  The formula $m < n$ is thus equivalent to the condition: $m
\in n$ and $m,n \in \omega$. In particular, 0 is the number zero and at the
same time the empty set $\varnothing$, $1=\{0\}$, $2=\{0,1\}$, etc. ${}^B A$
denotes the set of all functions on $B$ to $A$ (i.e.\ with domain $B$ and range
included in $A$).  The members of ${}^\omega A$ are what are called {\em simple
infinite sequences},\index{simple infinite sequence}
with all {\em terms}\index{term} in $A$.  In case $n \in \omega$, the
members of ${}^n A$ are referred to as {\em finite $n$-termed
sequences},\index{finite $n$-termed
sequence} again
with terms in $A$.  The consecutive terms (function values) of a finite or
infinite sequence $f$ are denoted by $f_0, f_1, \ldots ,f_n,\ldots$.  Every
finite sequence $f \in \bigcup _{n \in \omega} {}^n A$ uniquely determines the
number $n$ such that $f \in {}^n A$; $n$ is called the {\em
length}\index{length of a sequence ({$"|\ "|$})} of $f$ and
is denoted by $|f|$.  $\langle a \rangle$ is the sequence $f$ with $|f|=1$ and
$f_0=a$; $\langle a,b \rangle$ is the sequence $f$ with $|f|=2$, $f_0=a$,
$f_1=b$; etc.  Given two finite sequences $f$ and $g$, we denote by $f\frown g$
their {\em concatenation},\index{concatenation} i.e., the
finite sequence $h$ determined by the
conditions:
\begin{eqnarray*}
& |h| = |f|+|g|;&  \\
& h_n = f_n & \mbox{\ for\ } n < |f|;  \\
& h_{|f|+n} = g_n & \mbox{\ for\ } n < |g|.
\end{eqnarray*}

\subsection{Constants, Variables, and Expressions}

A formal system has a set of {\em symbols}\index{symbol!in
a formal system} denoted
by $\mbox{\em SM}$.  A
precise set-theo\-ret\-i\-cal definition of this set is unimportant; a symbol
could be considered a primitive or atomic element if we wish.  We assume this
set is divided into two disjoint subsets:  a set $\mbox{\em CN}$ of {\em
constants}\index{constant!in a formal system} and a set $\mbox{\em VR}$ of
{\em variables}.\index{variable!in a formal system}  $\mbox{\em CN}$ and
$\mbox{\em VR}$ are each assumed to consist of countably many symbols which
may be arranged in finite or simple infinite sequences $c_0, c_1, \ldots$ and
$v_0, v_1, \ldots$ respectively, without repeating terms.  We will represent
arbitrary symbols by metavariables $\alpha$, $\beta$, etc.

{\footnotesize\begin{quotation}
{\em Comment.} The variables $v_0, v_1, \ldots$ of our formal system
correspond to what are usually considered ``metavariables'' in
descriptions of specific formal systems in the literature.  Typically,
when describing a specific formal system a book will postulate a set of
primitive objects called variables, then proceed to describe their
properties using metavariables that range over them, never mentioning
again the actual variables themselves.  Our formal system does not
mention these primitive variable objects at all but deals directly with
metavariables, as its primitive objects, from the start.  This is a
subtle but key distinction you should keep in mind, and it makes our
definition of ``formal system'' somewhat different from that typically
found in the literature.  (So, the $\alpha$, $\beta$, etc.\ above are
actually ``metametavariables'' when used to represent $v_0, v_1,
\ldots$.)
\end{quotation}}

Finite sequences all terms of which are symbols are called {\em
expressions}.\index{expression!in a formal system}  $\mbox{\em EX}$ is
the set of all expressions; thus
\begin{displaymath}
\mbox{\em EX} = \bigcup _{n \in \omega} {}^n \mbox{\em SM}.
\end{displaymath}

A {\em constant-prefixed expression}\index{constant-prefixed expression}
is an expression of non-zero length
whose first term is a constant.  We denote the set of all constant-prefixed
expressions by $\mbox{\em EX}_C = \{ e \in \mbox{\em EX} | ( |e| > 0 \wedge
e_0 \in \mbox{\em CN} ) \}$.

A {\em constant-variable pair}\index{constant-variable pair}
is an expression of length 2 whose first term
is a constant and whose second term is a variable.  We denote the set of all
constant-variable pairs by $\mbox{\em EX}_2 = \{ e \in \mbox{\em EX}_C | ( |e|
= 2 \wedge e_1 \in \mbox{\em VR} ) \}$.


{\footnotesize\begin{quotation}
{\em Relationship to Metamath.} In general, the set $\mbox{\em SM}$
corresponds to the set of declared math symbols in a Metamath database, the
set $\mbox{\em CN}$ to those declared with \texttt{\$c} statements, and the set
$\mbox{\em VR}$ to those declared with \texttt{\$v} statements.  Of course a
Metamath database can only have a finite number of math symbols, whereas
formal systems in general can have an infinite number, although the number of
Metamath math symbols available is in principle unlimited.

The set $\mbox{\em EX}_C$ corresponds to the set of permissible expressions
for \texttt{\$e}, \texttt{\$a}, and \texttt{\$p} statements.  The set $\mbox{\em EX}_2$
corresponds to the set of permissible expressions for \texttt{\$f} statements.
\end{quotation}}

We denote by ${\cal V}(e)$ the set of all variables in an expression $e \in
\mbox{\em EX}$, i.e.\ the set of all $\alpha \in \mbox{\em VR}$ such that
$\alpha = e_n$ for some $n < |e|$.  We also denote (with abuse of notation) by
${\cal V}(E)$ the set of all variables in a collection of expressions $E
\subseteq \mbox{\em EX}$, i.e.\ $\bigcup _{e \in E} {\cal V}(e)$.


\subsection{Substitution}

Given a function $F$ from $\mbox{\em VR}$ to
$\mbox{\em EX}$, we
denote by $\sigma_{F}$ or just $\sigma$ the function from $\mbox{\em EX}$ to
$\mbox{\em EX}$ defined recursively for nonempty sequences by
\begin{eqnarray*}
& \sigma(<\alpha>) = F(\alpha) & \mbox{for\ } \alpha \in \mbox{\em VR}; \\
& \sigma(<\alpha>) = <\alpha> & \mbox{for\ } \alpha \not\in \mbox{\em VR}; \\
& \sigma(g \frown h) = \sigma(g) \frown
    \sigma(h) & \mbox{for\ } g,h \in \mbox{\em EX}.
\end{eqnarray*}
We also define $\sigma(\varnothing)=\varnothing$.  We call $\sigma$ a {\em
simultaneous substitution}\index{substitution!variable}\index{variable
substitution} (or just {\em substitution}) with {\em substitution
map}\index{substitution map} $F$.

We also denote (with abuse of notation) by $\sigma(E)$ a substitution on a
collection of expressions $E \subseteq \mbox{\em EX}$, i.e.\ the set $\{
\sigma(e) | e \in E \}$.  The collection $\sigma(E)$ may of course contain
fewer expressions than $E$ because duplicate expressions could result from the
substitution.

\subsection{Statements}

We denote by $\mbox{\em DV}$ the set of all
unordered pairs $\{\alpha, \beta \} \subseteq \mbox{\em VR}$ such that $\alpha
\neq \beta$.  $\mbox{\em DV}$ stands for ``distinct variables.''

A {\em pre-statement}\index{pre-statement!in a formal system} is a
quadruple $\langle D,T,H,A \rangle$ such that
$D\subseteq \mbox{\em DV}$, $T\subseteq \mbox{\em EX}_2$, $H\subseteq
\mbox{\em EX}_C$ and $H$ is finite,
$A\in \mbox{\em EX}_C$, ${\cal V}(H\cup\{A\}) \subseteq
{\cal V}(T)$, and $\forall e,f\in T {\ } {\cal V}(e) \neq {\cal V}(f)$ (or
equivalently, $e_1 \ne f_1$) whenever $e \neq f$. The terms of the quadruple are called {\em
distinct-variable restrictions},\index{disjoint-variable restriction!in a
formal system} {\em variable-type hypotheses},\index{variable-type
hypothesis!in a formal system} {\em logical hypotheses},\index{logical
hypothesis!in a formal system} and the {\em assertion}\index{assertion!in a
formal system} respectively.  We denote by $T_M$ ({\em mandatory variable-type
hypotheses}\index{mandatory variable-type hypothesis!in a formal system}) the
subset of $T$ such that ${\cal V}(T_M) ={\cal V}(H \cup \{A\})$.  We denote by
$D_M=\{\{\alpha,\beta\}\in D|\{\alpha,\beta\}\subseteq {\cal V}(T_M)\}$ the
{\em mandatory distinct-variable restrictions}\index{mandatory
disjoint-variable restriction!in a formal system} of the pre-statement.
The set
of {\em mandatory hypotheses}\index{mandatory hypothesis!in a formal system}
is $T_M\cup H$.  We call the quadruple $\langle D_M,T_M,H,A \rangle$
the {\em reduct}\index{reduct!in a formal system} of
the pre-statement $\langle D,T,H,A \rangle$.

A {\em statement} is the reduct of some pre-statement\index{statement!in a
formal system}.  A statement is therefore a special kind of pre-statement;
in particular, a statement is the reduct of itself.

{\footnotesize\begin{quotation}
{\em Comment.}  $T$ is a set of expressions, each of length 2, that associate
a set of constants (``variable types'') with a set of variables.  The
condition ${\cal V}(H\cup\{A\}) \subseteq {\cal V}(T) $
means that each variable occurring in a statement's logical
hypotheses or assertion must have an associated variable-type hypothesis or
``type declaration,'' in  analogy to a computer programming language, where a
variable must be declared to be say, a string or an integer.  The requirement
that $\forall e,f\in T \, e_1 \ne f_1$ for $e\neq f$
means that each variable must be
associated with a unique constant designating its variable type; e.g., a
variable might be a ``wff'' or a ``set'' but not both.

Distinct-variable restrictions are used to specify what variable substitutions
are permissible to make for the statement to remain valid.  For example, in
the theorem scheme of set theory $\lnot\forall x\,x=y$ we may not substitute
the same variable for both $x$ and $y$.  On the other hand, the theorem scheme
$x=y\to y=x$ does not require that $x$ and $y$ be distinct, so we do not
require a distinct-variable restriction, although having one
would cause no harm other than making the scheme less general.

A mandatory variable-type hypothesis is one whose variable exists in a logical
hypothesis or the assertion.  A provable pre-statement
(defined below) may require
non-mandatory variable-type hypotheses that effectively introduce ``dummy''
variables for use in its proof.  Any number of dummy variables might
be required by a specific proof; indeed, it has been shown by H.\
Andr\'{e}ka\index{Andr{\'{e}}ka, H.} \cite{Nemeti} that there is no finite
upper bound to the number of dummy variables needed to prove an arbitrary
theorem in first-order logic (with equality) having a fixed number $n>2$ of
individual variables.  (See also the Comment on p.~\pageref{nodd}.)
For this reason we do not set a finite size bound on the collections $D$ and
$T$, although in an actual application (Metamath database) these will of
course be finite, increased to whatever size is necessary as more
proofs are added.
\end{quotation}}

{\footnotesize\begin{quotation}
{\em Relationship to Metamath.} A pre-statement of a formal system
corresponds to an extended frame in a Metamath database
(Section~\ref{frames}).  The collections $D$, $T$, and $H$ correspond
respectively to the \texttt{\$d}, \texttt{\$f}, and \texttt{\$e}
statement collections in an extended frame.  The expression $A$
corresponds to the \texttt{\$a} (or \texttt{\$p}) statement in an
extended frame.

A statement of a formal system corresponds to a frame in a Metamath
database.
\end{quotation}}

\subsection{Formal Systems}

A {\em formal system}\index{formal system} is a
triple $\langle \mbox{\em CN},\mbox{\em
VR},\Gamma\rangle$ where $\Gamma$ is a set of statements.  The members of
$\Gamma$ are called {\em axiomatic statements}.\index{axiomatic
statement!in a formal system}  Sometimes we will refer to a
formal system by just $\Gamma$ when $\mbox{\em CN}$ and $\mbox{\em VR}$ are
understood.

Given a formal system $\Gamma$, the {\em closure}\index{closure}\footnote{This
definition of closure incorporates a simplification due to
Josh Purinton.\index{Purinton, Josh}.} of a
pre-statement
$\langle D,T,H,A \rangle$ is the smallest set $C$ of expressions
such that:
%\begin{enumerate}
%  \item $T\cup H\subseteq C$; and
%  \item If for some axiomatic statement
%    $\langle D_M',T_M',H',A' \rangle \in \Gamma_A$, for
%    some $E \subseteq C$, some $F \subseteq C-T$ (where ``-'' denotes
%    set difference), and some substitution
%    $\sigma$ we have
%    \begin{enumerate}
%       \item $\sigma(T_M') = E$ (where, as above, the $M$ denotes the
%           mandatory variable-type hypotheses of $T^A$);
%       \item $\sigma(H') = F$;
%       \item for all $\{\alpha,\beta\}\in D^A$ and $\subseteq
%         {\cal V}(T_M')$, for all $\gamma\in {\cal V}(\sigma(\langle \alpha
%         \rangle))$, and for all $\delta\in  {\cal V}(\sigma(\langle \beta
%         \rangle))$, we have $\{\gamma, \delta\} \in D$;
%   \end{enumerate}
%   then $\sigma(A') \in C$.
%\end{enumerate}
\begin{list}{}{\itemsep 0.0pt}
  \item[1.] $T\cup H\subseteq C$; and
  \item[2.] If for some axiomatic statement
    $\langle D_M',T_M',H',A' \rangle \in
       \Gamma$ and for some substitution
    $\sigma$ we have
    \begin{enumerate}
       \item[a.] $\sigma(T_M' \cup H') \subseteq C$; and
       \item[b.] for all $\{\alpha,\beta\}\in D_M'$, for all $\gamma\in
         {\cal V}(\sigma(\langle \alpha
         \rangle))$, and for all $\delta\in  {\cal V}(\sigma(\langle \beta
         \rangle))$, we have $\{\gamma, \delta\} \in D$;
   \end{enumerate}
   then $\sigma(A') \in C$.
\end{list}
A pre-statement $\langle D,T,H,A
\rangle$ is {\em provable}\index{provable statement!in a formal
system} if $A\in C$ i.e.\ if its assertion belongs to its
closure.  A statement is {\em provable} if it is
the reduct of a provable pre-statement.
The {\em universe}\index{universe of a formal system}
of a formal system is
the collection of all of its provable statements.  Note that the
set of axiomatic statements $\Gamma$ in a formal system is a subset of its
universe.

{\footnotesize\begin{quotation}
{\em Comment.} The first condition in the definition of closure simply says
that the hypotheses of the pre-statement are in its closure.

Condition 2(a) says that a substitution exists that makes the
mandatory hypotheses of an axiomatic statement exactly match some members of
the closure.  This is what we explicitly demonstrate in a Metamath language
proof.

%Conditions 2(a) and 2(b) say that a substitution exists that makes the
%(mandatory) hypotheses of an axiomatic statement exactly match some members of
%the closure.  This is what we explicitly demonstrate with a Metamath language
%proof.
%
%The set of expressions $F$ in condition 2(b) excludes the variable-type
%hypotheses; this is done because non-mandatory variable-type hypotheses are
%effectively ``dropped'' as irrelevant whereas logical hypotheses must be
%retained to achieve a consistent logical system.

Condition 2(b) describes how distinct-variable restrictions in the axiomatic
statement must be met.  It means that after a substitution for two variables
that must be distinct, the resulting two expressions must either contain no
variables, or if they do, they may not have variables in common, and each pair
of any variables they do have, with one variable from each expression, must be
specified as distinct in the original statement.
\end{quotation}}

{\footnotesize\begin{quotation}
{\em Relationship to Metamath.} Axiomatic statements
 and provable statements in a formal
system correspond to the frames for \texttt{\$a} and \texttt{\$p} statements
respectively in a Metamath database.  The set of axiomatic statements is a
subset of the set of provable statements in a formal system, although in a
Metamath database a \texttt{\$a} statement is distinguished by not having a
proof.  A Metamath language proof for a \texttt{\$p} statement tells the computer
how to explicitly construct a series of members of the closure ultimately
leading to a demonstration that the assertion
being proved is in the closure.  The actual closure typically contains
an infinite number of expressions.  A formal system itself does not have
an explicit object called a ``proof'' but rather the existence of a proof
is implied indirectly by membership of an assertion in a provable
statement's closure.  We do this to make the formal system easier
to describe in the language of set theory.

We also note that once established as provable, a statement may be considered
to acquire the same status as an axiomatic statement, because if the set of
axiomatic statements is extended with a provable statement, the universe of
the formal system remains unchanged (provided that $\mbox{\em VR}$ is
infinite).
In practice, this means we can build a hierarchy of provable statements to
more efficiently establish additional provable statements.  This is
what we do in Metamath when we allow proofs to reference previous
\texttt{\$p} statements as well as previous \texttt{\$a} statements.
\end{quotation}}

\section{Examples of Formal Systems}

{\footnotesize\begin{quotation}
{\em Relationship to Metamath.} The examples in this section, except Example~2,
are for the most part exact equivalents of the development in the set
theory database \texttt{set.mm}.  You may want to compare Examples~1, 3, and 5
to Section~\ref{metaaxioms}, Example 4 to Sections~\ref{metadefprop} and
\ref{metadefpred}, and Example 6 to
Section~\ref{setdefinitions}.\label{exampleref}
\end{quotation}}

\subsection{Example~1---Propositional Calculus}\index{propositional calculus}

Classical propositional calculus can be described by the following formal
system.  We assume the set of variables is infinite.  Rather than denoting the
constants and variables by $c_0, c_1, \ldots$ and $v_0, v_1, \ldots$, for
readability we will instead use more conventional symbols, with the
understanding of course that they denote distinct primitive objects.
Also for readability we may omit commas between successive terms of a
sequence; thus $\langle \mbox{wff\ } \varphi\rangle$ denotes
$\langle \mbox{wff}, \varphi\rangle$.

Let
\begin{itemize}
  \item[] $\mbox{\em CN}=\{\mbox{wff}, \vdash, \to, \lnot, (,)\}$
  \item[] $\mbox{\em VR}=\{\varphi,\psi,\chi,\ldots\}$
  \item[] $T = \{\langle \mbox{wff\ } \varphi\rangle,
             \langle \mbox{wff\ } \psi\rangle,
             \langle \mbox{wff\ } \chi\rangle,\ldots\}$, i.e.\ those
             expressions of length 2 whose first member is $\mbox{\rm wff}$
             and whose second member belongs to $\mbox{\em VR}$.\footnote{For
convenience we let $T$ be an infinite set; the definition of a statement
permits this in principle.  Since a Metamath source file has a finite size, in
practice we must of course use appropriate finite subsets of this $T$,
specifically ones containing at least the mandatory variable-type
hypotheses.  Similarly, in the source file we introduce new variables as
required, with the understanding that a potentially infinite number of
them are available.}
\noindent Then $\Gamma$ consists of the axiomatic statements that
are the reducts of the following pre-statements:
    \begin{itemize}
      \item[] $\langle\varnothing,T,\varnothing,
               \langle \mbox{wff\ }(\varphi\to\psi)\rangle\rangle$
      \item[] $\langle\varnothing,T,\varnothing,
               \langle \mbox{wff\ }\lnot\varphi\rangle\rangle$
      \item[] $\langle\varnothing,T,\varnothing,
               \langle \vdash(\varphi\to(\psi\to\varphi))
               \rangle\rangle$
      \item[] $\langle\varnothing,T,
               \varnothing,
               \langle \vdash((\varphi\to(\psi\to\chi))\to
               ((\varphi\to\psi)\to(\varphi\to\chi)))
               \rangle\rangle$
      \item[] $\langle\varnothing,T,
               \varnothing,
               \langle \vdash((\lnot\varphi\to\lnot\psi)\to
               (\psi\to\varphi))\rangle\rangle$
      \item[] $\langle\varnothing,T,
               \{\langle\vdash(\varphi\to\psi)\rangle,
                 \langle\vdash\varphi\rangle\},
               \langle\vdash\psi\rangle\rangle$
    \end{itemize}
\end{itemize}

(For example, the reduct of $\langle\varnothing,T,\varnothing,
               \langle \mbox{wff\ }(\varphi\to\psi)\rangle\rangle$
is
\begin{itemize}
\item[] $\langle\varnothing,
\{\langle \mbox{wff\ } \varphi\rangle,
             \langle \mbox{wff\ } \psi\rangle\},
             \varnothing,
               \langle \mbox{wff\ }(\varphi\to\psi)\rangle\rangle$,
\end{itemize}
which is the first axiomatic statement.)

We call the members of $\mbox{\em VR}$ {\em wff variables} or (in the context
of first-order logic which we will describe shortly) {\em wff metavariables}.
Note that the symbols $\phi$, $\psi$, etc.\ denote actual specific members of
$\mbox{\em VR}$; they are not metavariables of our expository language (which
we denote with $\alpha$, $\beta$, etc.) but are instead (meta)constant symbols
(members of $\mbox{\em SM}$) from the point of view of our expository
language.  The equivalent system of propositional calculus described in
\cite{Tarski1965} also uses the symbols $\phi$, $\psi$, etc.\ to denote wff
metavariables, but in \cite{Tarski1965} unlike here those are metavariables of
the expository language and not primitive symbols of the formal system.

The first two statements define wffs: if $\varphi$ and $\psi$ are wffs, so is
$(\varphi \to \psi)$; if $\varphi$ is a wff, so is $\lnot\varphi$. The next
three are the axioms of propositional calculus: if $\varphi$ and $\psi$ are
wffs, then $\vdash (\varphi \to (\psi \to \varphi))$ is an (axiomatic)
theorem; etc. The
last is the rule of modus ponens: if $\varphi$ and $\psi$ are wffs, and
$\vdash (\varphi\to\psi)$ and $\vdash \varphi$ are theorems, then $\vdash
\psi$ is a theorem.

The correspondence to ordinary propositional calculus is as follows.  We
consider only provable statements of the form $\langle\varnothing,
T,\varnothing,A\rangle$ with $T$ defined as above.  The first term of the
assertion $A$ of any such statement is either ``wff'' or ``$\vdash$''.  A
statement for which the first term is ``wff'' is a {\em wff} of propositional
calculus, and one where the first term is ``$\vdash$'' is a {\em
theorem (scheme)} of propositional calculus.

The universe of this formal system also contains many other provable
statements.  Those with distinct-variable restrictions are irrelevant because
propositional calculus has no constraints on substitutions.  Those that have
logical hypotheses we call {\em inferences}\index{inference} when
the logical hypotheses are of the form
$\langle\vdash\rangle\frown w$ where $w$ is a wff (with the leading constant
term ``wff'' removed).  Inferences (other than the modus ponens rule) are not a
proper part of propositional calculus but are convenient to use when building a
hierarchy of provable statements.  A provable statement with a nonsense
hypothesis such as $\langle \to,\vdash,\lnot\rangle$, and this same expression
as its assertion, we consider irrelevant; no use can be made of it in
proving theorems, since there is no way to eliminate the nonsense hypothesis.

{\footnotesize\begin{quotation}
{\em Comment.} Our use of parentheses in the definition of a wff illustrates
how axiomatic statements should be carefully stated in a way that
ties in unambiguously with the substitutions allowed by the formal system.
There are many ways we could have defined wffs---for example, Polish
prefix notation would have allowed us to omit parentheses entirely, at
the expense of readability---but we must define them in a way that is
unambiguous.  For example, if we had omitted parentheses from the
definition of $(\varphi\to \psi)$, the wff $\lnot\varphi\to \psi$ could
be interpreted as either $\lnot(\varphi\to\psi)$ or $(\lnot\varphi\to\psi)$
and would have allowed us to prove nonsense.  Note that there is no
concept of operator binding precedence built into our formal system.
\end{quotation}}

\begin{sloppy}
\subsection{Example~2---Predicate Calculus with Equality}\index{predicate
calculus}
\end{sloppy}

Here we extend Example~1 to include predicate calculus with equality,
illustrating the use of distinct-variable restrictions.  This system is the
same as Tarski's system $\mathfrak{S}_2$ in \cite{Tarski1965} (except that the
axioms of propositional calculus are different but equivalent, and a redundant
axiom is omitted).  We extend $\mbox{\em CN}$ with the constants
$\{\mbox{var},\forall,=\}$.  We extend $\mbox{\em VR}$ with an infinite set of
{\em individual metavariables}\index{individual
metavariable} $\{x,y,z,\ldots\}$ and denote this subset
$\mbox{\em Vr}$.

We also join to $\mbox{\em CN}$ a possibly infinite set $\mbox{\em Pr}$ of {\em
predicates} $\{R,S,\ldots\}$.  We associate with $\mbox{\em Pr}$ a function
$\mbox{rnk}$ from $\mbox{\em Pr}$ to $\omega$, and for $\alpha\in \mbox{\em
Pr}$ we call $\mbox{rnk}(\alpha)$ the {\em rank} of the predicate $\alpha$,
which is simply the number of ``arguments'' that the predicate has.  (Most
applications of predicate calculus will have a finite number of predicates;
for example, set theory has the single two-argument or binary predicate $\in$,
which is usually written with its arguments surrounding the predicate symbol
rather than with the prefix notation we will use for the general case.)  As a
device to facilitate our discussion, we will let $\mbox{\em Vs}$ be any fixed
one-to-one function from $\omega$ to $\mbox{\em Vr}$; thus $\mbox{\em Vs}$ is
any simple infinite sequence of individual metavariables with no repeating
terms.

In this example we will not include the function symbols that are often part of
formalizations of predicate calculus.  Using metalogical arguments that are
beyond the scope of our discussion, it can be shown that our formalization is
equivalent when functions are introduced via appropriate definitions.

We extend the set $T$ defined in Example~1 with the expressions
$\{\langle \mbox{var\ } x\rangle,$ $ \langle \mbox{var\ } y\rangle, \langle
\mbox{var\ } z\rangle,\ldots\}$.  We extend the $\Gamma$ above
with the axiomatic statements that are the reducts of the following
pre-statements:
\begin{list}{}{\itemsep 0.0pt}
      \item[] $\langle\varnothing,T,\varnothing,
               \langle \mbox{wff\ }\forall x\,\varphi\rangle\rangle$
      \item[] $\langle\varnothing,T,\varnothing,
               \langle \mbox{wff\ }x=y\rangle\rangle$
      \item[] $\langle\varnothing,T,
               \{\langle\vdash\varphi\rangle\},
               \langle\vdash\forall x\,\varphi\rangle\rangle$
      \item[] $\langle\varnothing,T,\varnothing,
               \langle \vdash((\forall x(\varphi\to\psi)
                  \to(\forall x\,\varphi\to\forall x\,\psi))
               \rangle\rangle$
      \item[] $\langle\{\{x,\varphi\}\},T,\varnothing,
               \langle \vdash(\varphi\to\forall x\,\varphi)
               \rangle\rangle$
      \item[] $\langle\{\{x,y\}\},T,\varnothing,
               \langle \vdash\lnot\forall x\lnot x=y
               \rangle\rangle$
      \item[] $\langle\varnothing,T,\varnothing,
               \langle \vdash(x=z
                  \to(x=y\to z=y))
               \rangle\rangle$
      \item[] $\langle\varnothing,T,\varnothing,
               \langle \vdash(y=z
                  \to(x=y\to x=z))
               \rangle\rangle$
\end{list}
These are the axioms not involving predicate symbols. The first two statements
extend the definition of a wff.  The third is the rule of generalization.  The
fifth states, in effect, ``For a wff $\varphi$ and variable $x$,
$\vdash(\varphi\to\forall x\,\varphi)$, provided that $x$ does not occur in
$\varphi$.''  The sixth states ``For variables $x$ and $y$,
$\vdash\lnot\forall x\lnot x = y$, provided that $x$ and $y$ are distinct.''
(This proviso is not necessary but was included by Tarski to
weaken the axiom and still show that the system is logically complete.)

Finally, for each predicate symbol $\alpha\in \mbox{\em Pr}$, we add to
$\Gamma$ an axiomatic statement, extending the definition of wff,
that is the reduct of the following pre-statement:
\begin{displaymath}
    \langle\varnothing,T,\varnothing,
            \langle \mbox{wff},\alpha\rangle\
            \frown \mbox{\em Vs}\restriction\mbox{rnk}(\alpha)\rangle
\end{displaymath}
and for each $\alpha\in \mbox{\em Pr}$ and each $n < \mbox{rnk}(\alpha)$
we add to $\Gamma$ an equality axiom that is the reduct of the
following pre-statement:
\begin{eqnarray*}
    \lefteqn{\langle\varnothing,T,\varnothing,
            \langle
      \vdash,(,\mbox{\em Vs}_n,=,\mbox{\em Vs}_{\mbox{rnk}(\alpha)},\to,
            (,\alpha\rangle\frown \mbox{\em Vs}\restriction\mbox{rnk}(\alpha)} \\
  & & \frown
            \langle\to,\alpha\rangle\frown \mbox{\em Vs}\restriction n\frown
            \langle \mbox{\em Vs}_{\mbox{rnk}(\alpha)}\rangle \\
 & & \frown
            \mbox{\em Vs}\restriction(\mbox{rnk}(\alpha)\setminus(n+1))\frown
            \langle),)\rangle\rangle
\end{eqnarray*}
where $\restriction$ denotes function domain restriction and $\setminus$
denotes set difference.  Recall that a subscript on $\mbox{\em Vs}$
denotes one of its terms.  (In the above two axiom sets commas are placed
between successive terms of sequences to prevent ambiguity, and if you examine
them with care you will be able to distinguish those parentheses that denote
constant symbols from those of our expository language that delimit function
arguments.  Although it might have been better to use boldface for our
primitive symbols, unfortunately boldface was not available for all characters
on the \LaTeX\ system used to typeset this text.)  These seemingly forbidding
axioms can be understood by analogy to concatenation of substrings in a
computer language.  They are actually relatively simple for each specific case
and will become clearer by looking at the special case of a binary predicate
$\alpha = R$ where $\mbox{rnk}(R)=2$.  Letting $\mbox{\em Vs}$ be the sequence
$\langle x,y,z,\ldots\rangle$, the axioms we would add to $\Gamma$ for this
case would be the wff extension and two equality axioms that are the
reducts of the pre-statements:
\begin{list}{}{\itemsep 0.0pt}
      \item[] $\langle\varnothing,T,\varnothing,
               \langle \mbox{wff\ }R x y\rangle\rangle$
      \item[] $\langle\varnothing,T,\varnothing,
               \langle \vdash(x=z
                  \to(R x y \to R z y))
               \rangle\rangle$
      \item[] $\langle\varnothing,T,\varnothing,
               \langle \vdash(y=z
                  \to(R x y \to R x z))
               \rangle\rangle$
\end{list}
Study these carefully to see how the general axioms above evaluate to
them.  In practice, typically only a few special cases such as this would be
needed, and in any case the Metamath language will only permit us to describe
a finite number of predicates, as opposed to the infinite number permitted by
the formal system.  (If an infinite number should be needed for some reason,
we could not define the formal system directly in the Metamath language but
could instead define it metalogically under set theory as we
do in this appendix, and only the underlying set theory, with its single
binary predicate, would be defined directly in the Metamath language.)


{\footnotesize\begin{quotation}
{\em Comment.}  As we noted earlier, the specific variables denoted by the
symbols $x,y,z,\ldots\in \mbox{\em Vr}\subseteq \mbox{\em VR}\subseteq
\mbox{\em SM}$ in Example~2 are not the actual variables of ordinary predicate
calculus but should be thought of as metavariables ranging over them.  For
example, a distinct-variable restriction would be meaningless for actual
variables of ordinary predicate calculus since two different actual variables
are by definition distinct.  And when we talk about an arbitrary
representative $\alpha\in \mbox{\em Vr}$, $\alpha$ is a metavariable (in our
expository language) that ranges over metavariables (which are primitives of
our formal system) each of which ranges over the actual individual variables
of predicate calculus (which are never mentioned in our formal system).

The constant called ``var'' above is called \texttt{setvar} in the
\texttt{set.mm} database file, but it means the same thing.  I felt
that ``var'' is a more meaningful name in the context of predicate
calculus, whose use is not limited to set theory.  For consistency we
stick with the name ``var'' throughout this Appendix, even after set
theory is introduced.
\end{quotation}}

\subsection{Free Variables and Proper Substitution}\index{free variable}
\index{proper substitution}\index{substitution!proper}

Typical representations of mathematical axioms use concepts such
as ``free variable,'' ``bound variable,'' and ``proper substitution''
as primitive notions.
A free variable is a variable that
is not a parameter of any container expression.
A bound variable is the opposite of a free variable; it is a
a variable that has been bound in a container expression.
For example, in the expression $\forall x \varphi$ (for all $x$, $\varphi$
is true), the variable $x$
is bound within the for-all ($\forall$) expression.
It is possible to change one variable to another, and that process is called
``proper substitution.''
In most books, proper substitution has a somewhat complicated recursive
definition with multiple cases based on the occurrences of free and
bound variables.
You may consult
\cite[ch.\ 3--4]{Hamilton}\index{Hamilton, Alan G.} (as well as
many other texts) for more formal details about these terms.

Using these concepts as \texttt{primitives} creates complications
for computer implementations.

In the system of Example~2, there are no primitive notions of free variable
and proper substitution.  Tarski \cite{Tarski1965} shows that this system is
logically equivalent to the more typical textbook systems that do have these
primitive notions, if we introduce these notions with appropriate definitions
and metalogic.  We could also define axioms for such systems directly,
although the recursive definitions of free variable and proper substitution
would be messy and awkward to work with.  Instead, we mention two devices that
can be used in practice to mimic these notions.  (1) Instead of introducing
special notation to express (as a logical hypothesis) ``where $x$ is not free
in $\varphi$'' we can use the logical hypothesis $\vdash(\varphi\to\forall
x\,\varphi)$.\label{effectivelybound}\index{effectively
not free}\footnote{This is a slightly weaker requirement than ``where $x$ is
not free in $\varphi$.''  If we let $\varphi$ be $x=x$, we have the theorem
$(x=x\to\forall x\,x=x)$ which satisfies the hypothesis, even though $x$ is
free in $x=x$ .  In a case like this we say that $x$ is {\em effectively not
free}\index{effectively not free} in $x=x$, since $x=x$ is logically
equivalent to $\forall x\,x=x$ in which $x$ is bound.} (2) It can be shown
that the wff $((x=y\to\varphi)\wedge\exists x(x=y\wedge\varphi))$ (with the
usual definitions of $\wedge$ and $\exists$; see Example~4 below) is logically
equivalent to ``the wff that results from proper substitution of $y$ for $x$
in $\varphi$.''  This works whether or not $x$ and $y$ are distinct.

\subsection{Metalogical Completeness}\index{metalogical completeness}

In the system of Example~2, the
following are provable pre-statements (and their reducts are
provable statements):
\begin{eqnarray*}
      & \langle\{\{x,y\}\},T,\varnothing,
               \langle \vdash\lnot\forall x\lnot x=y
               \rangle\rangle & \\
     &  \langle\varnothing,T,\varnothing,
               \langle \vdash\lnot\forall x\lnot x=x
               \rangle\rangle &
\end{eqnarray*}
whereas the following pre-statement is not to my knowledge provable (but
in any case we will pretend it's not for sake of illustration):
\begin{eqnarray*}
     &  \langle\varnothing,T,\varnothing,
               \langle \vdash\lnot\forall x\lnot x=y
               \rangle\rangle &
\end{eqnarray*}
In other words, we can prove ``$\lnot\forall x\lnot x=y$ where $x$ and $y$ are
distinct'' and separately prove ``$\lnot\forall x\lnot x=x$'', but we can't
prove the combined general case ``$\lnot\forall x\lnot x=y$'' that has no
proviso.  Now this does not compromise logical completeness, because the
variables are really metavariables and the two provable cases together cover
all possible cases.  The third case can be considered a metatheorem whose
direct proof, using the system of Example~2, lies outside the capability of the
formal system.

Also, in the system of Example~2 the following pre-statement is not to my
knowledge provable (again, a conjecture that we will pretend to be the case):
\begin{eqnarray*}
     & \langle\varnothing,T,\varnothing,
               \langle \vdash(\forall x\, \varphi\to\varphi)
               \rangle\rangle &
\end{eqnarray*}
Instead, we can only prove specific cases of $\varphi$ involving individual
metavariables, and by induction on formula length, prove as a metatheorem
outside of our formal system the general statement above.  The details of this
proof are found in \cite{Kalish}.

There does, however, exist a system of predicate calculus in which all such
``simple metatheorems'' as those above can be proved directly, and we present
it in Example~3. A {\em simple metatheorem}\index{simple metatheorem}
is any statement of the formal
system of Example~2 where all distinct variable restrictions consist of either
two individual metavariables or an individual metavariable and a wff
metavariable, and which is provable by combining cases outside the system as
above.  A system is {\em metalogically complete}\index{metalogical
completeness} if all of its simple
metatheorems are (directly) provable statements. The precise definition of
``simple metatheorem'' and the proof of the ``metalogical completeness'' of
Example~3 is found in Remark 9.6 and Theorem 9.7 of \cite{Megill}.\index{Megill,
Norman}

\begin{sloppy}
\subsection{Example~3---Metalogically Complete Predicate
Calculus with
Equality}
\end{sloppy}

For simplicity we will assume there is one binary predicate $R$;
this system suffices for set theory, where the $R$ is of course the $\in$
predicate.  We label the axioms as they appear in \cite{Megill}.  This
system is logically equivalent to that of Example~2 (when the latter is
restricted to this single binary predicate) but is also metalogically
complete.\index{metalogical completeness}

Let
\begin{itemize}
  \item[] $\mbox{\em CN}=\{\mbox{wff}, \mbox{var}, \vdash, \to, \lnot, (,),\forall,=,R\}$.
  \item[] $\mbox{\em VR}=\{\varphi,\psi,\chi,\ldots\}\cup\{x,y,z,\ldots\}$.
  \item[] $T = \{\langle \mbox{wff\ } \varphi\rangle,
             \langle \mbox{wff\ } \psi\rangle,
             \langle \mbox{wff\ } \chi\rangle,\ldots\}\cup
       \{\langle \mbox{var\ } x\rangle, \langle \mbox{var\ } y\rangle, \langle
       \mbox{var\ }z\rangle,\ldots\}$.

\noindent Then
  $\Gamma$ consists of the reducts of the following pre-statements:
    \begin{itemize}
      \item[] $\langle\varnothing,T,\varnothing,
               \langle \mbox{wff\ }(\varphi\to\psi)\rangle\rangle$
      \item[] $\langle\varnothing,T,\varnothing,
               \langle \mbox{wff\ }\lnot\varphi\rangle\rangle$
      \item[] $\langle\varnothing,T,\varnothing,
               \langle \mbox{wff\ }\forall x\,\varphi\rangle\rangle$
      \item[] $\langle\varnothing,T,\varnothing,
               \langle \mbox{wff\ }x=y\rangle\rangle$
      \item[] $\langle\varnothing,T,\varnothing,
               \langle \mbox{wff\ }Rxy\rangle\rangle$
      \item[(C1$'$)] $\langle\varnothing,T,\varnothing,
               \langle \vdash(\varphi\to(\psi\to\varphi))
               \rangle\rangle$
      \item[(C2$'$)] $\langle\varnothing,T,
               \varnothing,
               \langle \vdash((\varphi\to(\psi\to\chi))\to
               ((\varphi\to\psi)\to(\varphi\to\chi)))
               \rangle\rangle$
      \item[(C3$'$)] $\langle\varnothing,T,
               \varnothing,
               \langle \vdash((\lnot\varphi\to\lnot\psi)\to
               (\psi\to\varphi))\rangle\rangle$
      \item[(C4$'$)] $\langle\varnothing,T,
               \varnothing,
               \langle \vdash(\forall x(\forall x\,\varphi\to\psi)\to
                 (\forall x\,\varphi\to\forall x\,\psi))\rangle\rangle$
      \item[(C5$'$)] $\langle\varnothing,T,
               \varnothing,
               \langle \vdash(\forall x\,\varphi\to\varphi)\rangle\rangle$
      \item[(C6$'$)] $\langle\varnothing,T,
               \varnothing,
               \langle \vdash(\forall x\forall y\,\varphi\to
                 \forall y\forall x\,\varphi)\rangle\rangle$
      \item[(C7$'$)] $\langle\varnothing,T,
               \varnothing,
               \langle \vdash(\lnot\varphi\to\forall x\lnot\forall x\,\varphi
                 )\rangle\rangle$
      \item[(C8$'$)] $\langle\varnothing,T,
               \varnothing,
               \langle \vdash(x=y\to(x=z\to y=z))\rangle\rangle$
      \item[(C9$'$)] $\langle\varnothing,T,
               \varnothing,
               \langle \vdash(\lnot\forall x\, x=y\to(\lnot\forall x\, x=z\to
                 (y=z\to\forall x\, y=z)))\rangle\rangle$
      \item[(C10$'$)] $\langle\varnothing,T,
               \varnothing,
               \langle \vdash(\forall x(x=y\to\forall x\,\varphi)\to
                 \varphi))\rangle\rangle$
      \item[(C11$'$)] $\langle\varnothing,T,
               \varnothing,
               \langle \vdash(\forall x\, x=y\to(\forall x\,\varphi
               \to\forall y\,\varphi))\rangle\rangle$
      \item[(C12$'$)] $\langle\varnothing,T,
               \varnothing,
               \langle \vdash(x=y\to(Rxz\to Ryz))\rangle\rangle$
      \item[(C13$'$)] $\langle\varnothing,T,
               \varnothing,
               \langle \vdash(x=y\to(Rzx\to Rzy))\rangle\rangle$
      \item[(C15$'$)] $\langle\varnothing,T,
               \varnothing,
               \langle \vdash(\lnot\forall x\, x=y\to(x=y\to(\varphi
                 \to\forall x(x=y\to\varphi))))\rangle\rangle$
      \item[(C16$'$)] $\langle\{\{x,y\}\},T,
               \varnothing,
               \langle \vdash(\forall x\, x=y\to(\varphi\to\forall x\,\varphi)
                 )\rangle\rangle$
      \item[(C5)] $\langle\{\{x,\varphi\}\},T,\varnothing,
               \langle \vdash(\varphi\to\forall x\,\varphi)
               \rangle\rangle$
      \item[(MP)] $\langle\varnothing,T,
               \{\langle\vdash(\varphi\to\psi)\rangle,
                 \langle\vdash\varphi\rangle\},
               \langle\vdash\psi\rangle\rangle$
      \item[(Gen)] $\langle\varnothing,T,
               \{\langle\vdash\varphi\rangle\},
               \langle\vdash\forall x\,\varphi\rangle\rangle$
    \end{itemize}
\end{itemize}

While it is known that these axioms are ``metalogically complete,'' it is
not known whether they are independent (i.e.\ none is
redundant) in the metalogical sense; specifically, whether any axiom (possibly
with additional non-mandatory distinct-variable restrictions, for use with any
dummy variables in its proof) is provable from the others.  Note that
metalogical independence is a weaker requirement than independence in the
usual logical sense.  Not all of the above axioms are logically independent:
for example, C9$'$ can be proved as a metatheorem from the others, outside the
formal system, by combining the possible cases of distinct variables.

\subsection{Example~4---Adding Definitions}\index{definition}
There are several ways to add definitions to a formal system.  Probably the
most proper way is to consider definitions not as part of the formal system at
all but rather as abbreviations that are part of the expository metalogic
outside the formal system.  For convenience, though, we may use the formal
system itself to incorporate definitions, adding them as axiomatic extensions
to the system.  This could be done by adding a constant representing the
concept ``is defined as'' along with axioms for it. But there is a nicer way,
at least in this writer's opinion, that introduces definitions as direct
extensions to the language rather than as extralogical primitive notions.  We
introduce additional logical connectives and provide axioms for them.  For
systems of logic such as Examples 1 through 3, the additional axioms must be
conservative in the sense that no wff of the original system that was not a
theorem (when the initial term ``wff'' is replaced by ``$\vdash$'' of course)
becomes a theorem of the extended system.  In this example we extend Example~3
(or 2) with standard abbreviations of logic.

We extend $\mbox{\em CN}$ of Example~3 with new constants $\{\leftrightarrow,
\wedge,\vee,\exists\}$, corresponding to logical equivalence,\index{logical
equivalence ($\leftrightarrow$)}\index{biconditional ($\leftrightarrow$)}
conjunction,\index{conjunction ($\wedge$)} disjunction,\index{disjunction
($\vee$)} and the existential quantifier.\index{existential quantifier
($\exists$)}  We extend $\Gamma$ with the axiomatic statements that are
the reducts of the following pre-statements:
\begin{list}{}{\itemsep 0.0pt}
      \item[] $\langle\varnothing,T,\varnothing,
               \langle \mbox{wff\ }(\varphi\leftrightarrow\psi)\rangle\rangle$
      \item[] $\langle\varnothing,T,\varnothing,
               \langle \mbox{wff\ }(\varphi\vee\psi)\rangle\rangle$
      \item[] $\langle\varnothing,T,\varnothing,
               \langle \mbox{wff\ }(\varphi\wedge\psi)\rangle\rangle$
      \item[] $\langle\varnothing,T,\varnothing,
               \langle \mbox{wff\ }\exists x\, \varphi\rangle\rangle$
  \item[] $\langle\varnothing,T,\varnothing,
     \langle\vdash ( ( \varphi \leftrightarrow \psi ) \to
     ( \varphi \to \psi ) )\rangle\rangle$
  \item[] $\langle\varnothing,T,\varnothing,
     \langle\vdash ((\varphi\leftrightarrow\psi)\to
    (\psi\to\varphi))\rangle\rangle$
  \item[] $\langle\varnothing,T,\varnothing,
     \langle\vdash ((\varphi\to\psi)\to(
     (\psi\to\varphi)\to(\varphi
     \leftrightarrow\psi)))\rangle\rangle$
  \item[] $\langle\varnothing,T,\varnothing,
     \langle\vdash (( \varphi \wedge \psi ) \leftrightarrow\neg ( \varphi
     \to \neg \psi )) \rangle\rangle$
  \item[] $\langle\varnothing,T,\varnothing,
     \langle\vdash (( \varphi \vee \psi ) \leftrightarrow (\neg \varphi
     \to \psi )) \rangle\rangle$
  \item[] $\langle\varnothing,T,\varnothing,
     \langle\vdash (\exists x \,\varphi\leftrightarrow
     \lnot \forall x \lnot \varphi)\rangle\rangle$
\end{list}
The first three logical axioms (statements containing ``$\vdash$'') introduce
and effectively define logical equivalence, ``$\leftrightarrow$''.  The last
three use ``$\leftrightarrow$'' to effectively mean ``is defined as.''

\subsection{Example~5---ZFC Set Theory}\index{ZFC set theory}

Here we add to the system of Example~4 the axioms of Zermelo--Fraenkel set
theory with Choice.  For convenience we make use of the
definitions in Example~4.

In the $\mbox{\em CN}$ of Example~4 (which extends Example~3), we replace the symbol $R$
with the symbol $\in$.
More explicitly, we remove from $\Gamma$ of Example~4 the three
axiomatic statements containing $R$ and replace them with the
reducts of the following:
\begin{list}{}{\itemsep 0.0pt}
      \item[] $\langle\varnothing,T,\varnothing,
               \langle \mbox{wff\ }x\in y\rangle\rangle$
      \item[] $\langle\varnothing,T,
               \varnothing,
               \langle \vdash(x=y\to(x\in z\to y\in z))\rangle\rangle$
      \item[] $\langle\varnothing,T,
               \varnothing,
               \langle \vdash(x=y\to(z\in x\to z\in y))\rangle\rangle$
\end{list}
Letting $D=\{\{\alpha,\beta\}\in \mbox{\em DV}\,|\alpha,\beta\in \mbox{\em
Vr}\}$ (in other words all individual variables must be distinct), we extend
$\Gamma$ with the ZFC axioms, called
\index{Axiom of Extensionality}
\index{Axiom of Replacement}
\index{Axiom of Union}
\index{Axiom of Power Sets}
\index{Axiom of Regularity}
\index{Axiom of Infinity}
\index{Axiom of Choice}
Extensionality, Replacement, Union, Power
Set, Regularity, Infinity, and Choice, that are the reducts of:
\begin{list}{}{\itemsep 0.0pt}
      \item[Ext] $\langle D,T,
               \varnothing,
               \langle\vdash (\forall x(x\in y\leftrightarrow x \in z)\to y
               =z) \rangle\rangle$
      \item[Rep] $\langle D,T,
               \varnothing,
               \langle\vdash\exists x ( \exists y \forall z (\varphi \to z = y
                        ) \to
                        \forall z ( z \in x \leftrightarrow \exists x ( x \in
                        y \wedge \forall y\,\varphi ) ) )\rangle\rangle$
      \item[Un] $\langle D,T,
               \varnothing,
               \langle\vdash \exists x \forall y ( \exists x ( y \in x \wedge
               x \in z ) \to y \in x ) \rangle\rangle$
      \item[Pow] $\langle D,T,
               \varnothing,
               \langle\vdash \exists x \forall y ( \forall x ( x \in y \to x
               \in z ) \to y \in x ) \rangle\rangle$
      \item[Reg] $\langle D,T,
               \varnothing,
               \langle\vdash (  x \in y \to
                 \exists x ( x \in y \wedge \forall z ( z \in x \to \lnot z
                \in y ) ) ) \rangle\rangle$
      \item[Inf] $\langle D,T,
               \varnothing,
               \langle\vdash \exists x(y\in x\wedge\forall y(y\in
               x\to
               \exists z(y \in z\wedge z\in x))) \rangle\rangle$
      \item[AC] $\langle D,T,
               \varnothing,
               \langle\vdash \exists x \forall y \forall z ( ( y \in z
               \wedge z \in w ) \to \exists w \forall y ( \exists w
              ( ( y \in z \wedge z \in w ) \wedge ( y \in w \wedge w \in x
              ) ) \leftrightarrow y = w ) ) \rangle\rangle$
\end{list}

\subsection{Example~6---Class Notation in Set Theory}\label{class}

A powerful device that makes set theory easier (and that we have
been using all along in our informal expository language) is {\em class
abstraction notation}.\index{class abstraction}\index{abstraction class}  The
definitions we introduce are rigorously justified
as conservative by Takeuti and Zaring \cite{Takeuti}\index{Takeuti, Gaisi} or
Quine \cite{Quine}\index{Quine, Willard Van Orman}.  The key idea is to
introduce the notation $\{x|\mbox{---}\}$ which means ``the class of all $x$
such that ---'' for abstraction classes and introduce (meta)variables that
range over them.  An abstraction class may or may not be a set, depending on
whether it exists (as a set).  A class that does not exist is
called a {\em proper class}.\index{proper class}\index{class!proper}

To illustrate the use of abstraction classes we will provide some examples
of definitions that make use of them:  the empty set, class union, and
unordered pair.  Many other such definitions can be found in the
Metamath set theory database,
\texttt{set.mm}.\index{set theory database (\texttt{set.mm})}

% We intentionally break up the sequence of math symbols here
% because otherwise the overlong line goes beyond the page in narrow mode.
We extend $\mbox{\em CN}$ of Example~5 with new symbols $\{$
$\mbox{class},$ $\{,$ $|,$ $\},$ $\varnothing,$ $\cup,$ $,$ $\}$
where the inner braces and last comma are
constant symbols. (As before,
our dual use of some mathematical symbols for both our expository
language and as primitives of the formal system should be clear from context.)

We extend $\mbox{\em VR}$ of Example~5 with a set of {\em class
variables}\index{class variable}
$\{A,B,C,\ldots\}$. We extend the $T$ of Example~5 with $\{\langle
\mbox{class\ } A\rangle, \langle \mbox{class\ }B\rangle, \langle \mbox{class\ }
C\rangle,\ldots\}$.

To
introduce our definitions,
we add to $\Gamma$ of Example~5 the axiomatic statements
that are the reducts of the following pre-statements:
\begin{list}{}{\itemsep 0.0pt}
      \item[] $\langle\varnothing,T,\varnothing,
               \langle \mbox{class\ }x\rangle\rangle$
      \item[] $\langle\varnothing,T,\varnothing,
               \langle \mbox{class\ }\{x|\varphi\}\rangle\rangle$
      \item[] $\langle\varnothing,T,\varnothing,
               \langle \mbox{wff\ }A=B\rangle\rangle$
      \item[] $\langle\varnothing,T,\varnothing,
               \langle \mbox{wff\ }A\in B\rangle\rangle$
      \item[Ab] $\langle\varnothing,T,\varnothing,
               \langle \vdash ( y \in \{ x |\varphi\} \leftrightarrow
                  ( ( x = y \to\varphi) \wedge \exists x ( x = y
                  \wedge\varphi) ))
               \rangle\rangle$
      \item[Eq] $\langle\{\{x,A\},\{x,B\}\},T,\varnothing,
               \langle \vdash ( A = B \leftrightarrow
               \forall x ( x \in A \leftrightarrow x \in B ) )
               \rangle\rangle$
      \item[El] $\langle\{\{x,A\},\{x,B\}\},T,\varnothing,
               \langle \vdash ( A \in B \leftrightarrow \exists x
               ( x = A \wedge x \in B ) )
               \rangle\rangle$
\end{list}
Here we say that an individual variable is a class; $\{x|\varphi\}$ is a
class; and we extend the definition of a wff to include class equality and
membership.  Axiom Ab defines membership of a variable in a class abstraction;
the right-hand side can be read as ``the wff that results from proper
substitution of $y$ for $x$ in $\varphi$.''\footnote{Note that this definition
makes unnecessary the introduction of a separate notation similar to
$\varphi(x|y)$ for proper substitution, although we may choose to do so to be
conventional.  Incidentally, $\varphi(x|y)$ as it stands would be ambiguous in
the formal systems of our examples, since we wouldn't know whether
$\lnot\varphi(x|y)$ meant $\lnot(\varphi(x|y))$ or $(\lnot\varphi)(x|y)$.
Instead, we would have to use an unambiguous variant such as $(\varphi\,
x|y)$.}  Axioms Eq and El extend the meaning of the existing equality and
membership connectives.  This is potentially dangerous and requires careful
justification.  For example, from Eq we can derive the Axiom of Extensionality
with predicate logic alone; thus in principle we should include the Axiom of
Extensionality as a logical hypothesis.  However we do not bother to do this
since we have already presupposed that axiom earlier. The distinct variable
restrictions should be read ``where $x$ does not occur in $A$ or $B$.''  We
typically do this when the right-hand side of a definition involves an
individual variable not in the expression being defined; it is done so that
the right-hand side remains independent of the particular ``dummy'' variable
we use.

We continue to add to $\Gamma$ the following definitions
(i.e. the reducts of the following pre-statements) for empty
set,\index{empty set} class union,\index{union} and unordered
pair.\index{unordered pair}  They should be self-explanatory.  Analogous to our
use of ``$\leftrightarrow$'' to define new wffs in Example~4, we use ``$=$''
to define new abstraction terms, and both may be read informally as ``is
defined as'' in this context.
\begin{list}{}{\itemsep 0.0pt}
      \item[] $\langle\varnothing,T,\varnothing,
               \langle \mbox{class\ }\varnothing\rangle\rangle$
      \item[] $\langle\varnothing,T,\varnothing,
               \langle \vdash \varnothing = \{ x | \lnot x = x \}
               \rangle\rangle$
      \item[] $\langle\varnothing,T,\varnothing,
               \langle \mbox{class\ }(A\cup B)\rangle\rangle$
      \item[] $\langle\{\{x,A\},\{x,B\}\},T,\varnothing,
               \langle \vdash ( A \cup B ) = \{ x | ( x \in A \vee x \in B ) \}
               \rangle\rangle$
      \item[] $\langle\varnothing,T,\varnothing,
               \langle \mbox{class\ }\{A,B\}\rangle\rangle$
      \item[] $\langle\{\{x,A\},\{x,B\}\},T,\varnothing,
               \langle \vdash \{ A , B \} = \{ x | ( x = A \vee x = B ) \}
               \rangle\rangle$
\end{list}

\section{Metamath as a Formal System}\label{theorymm}

This section presupposes a familiarity with the Metamath computer language.

Our theory describes formal systems and their universes.  The Metamath
language provides a way of representing these set-theoretical objects to
a computer.  A Metamath database, being a finite set of {\sc ascii}
characters, can usually describe only a subset of a formal system and
its universe, which are typically infinite.  However the database can
contain as large a finite subset of the formal system and its universe
as we wish.  (Of course a Metamath set theory database can, in
principle, indirectly describe an entire infinite formal system by
formalizing the expository language in this Appendix.)

For purpose of our discussion, we assume the Metamath database
is in the simple form described on p.~\pageref{framelist},
consisting of all constant and variable declarations at the beginning,
followed by a sequence of extended frames each
delimited by \texttt{\$\char`\{} and \texttt{\$\char`\}}.  Any Metamath database can
be converted to this form, as described on p.~\pageref{frameconvert}.

The math symbol tokens of a Metamath source file, which are declared
with \texttt{\$c} and \texttt{\$v} statements, are names we assign to
representatives of $\mbox{\em CN}$ and $\mbox{\em VR}$.  For
definiteness we could assume that the first math symbol declared as a
variable corresponds to $v_0$, the second to $v_1$, etc., although the
exact correspondence we choose is not important.

In the Metamath language, each \texttt{\$d}, \texttt{\$f}, and
 \texttt{\$e} source
statement in an extended frame (Section~\ref{frames})
corresponds respectively to a member of the
collections $D$, $T$, and $H$ in a formal system statement $\langle
D_M,T_M,H,A\rangle$.  The math symbol strings following these Metamath keywords
correspond to a variable pair (in the case of \texttt{\$d}) or an expression (for
the other two keywords). The math symbol string following a \texttt{\$a} source
statement corresponds to expression $A$ in an axiomatic statement of the
formal system; the one following a \texttt{\$p} source statement corresponds to
$A$ in a provable statement that is not axiomatic.  In other words, each
extended frame in a Metamath database corresponds to
a pre-statement of the formal system, and a frame corresponds to
a statement of the formal system.  (Don't confuse the two meanings of
``statement'' here.  A statement of the formal system corresponds to the
several statements in a Metamath database that may constitute a
frame.)

In order for the computer to verify that a formal system statement is
provable, each \texttt{\$p} source statement is accompanied by a proof.
However, the proof does not correspond to anything in the formal system
but is simply a way of communicating to the computer the information
needed for its verification.  The proof tells the computer {\em how to
construct} specific members of closure of the formal system
pre-statement corresponding to the extended frame of the \texttt{\$p}
statement.  The final result of the construction is the member of the
closure that matches the \texttt{\$p} statement.  The abstract formal
system, on the other hand, is concerned only with the {\em existence} of
members of the closure.

As mentioned on p.~\pageref{exampleref}, Examples 1 and 3--6 in the
previous Section parallel the development of logic and set theory in the
Metamath database
\texttt{set.mm}.\index{set theory database (\texttt{set.mm})} You may
find it instructive to compare them.


\chapter{The MIU System}
\label{MIU}
\index{formal system}
\index{MIU-system}

The following is a listing of the file \texttt{miu.mm}.  It is self-explanatory.

%%%%%%%%%%%%%%%%%%%%%%%%%%%%%%%%%%%%%%%%%%%%%%%%%%%%%%%%%%%%

\begin{verbatim}
$( The MIU-system:  A simple formal system $)

$( Note:  This formal system is unusual in that it allows
empty wffs.  To work with a proof, you must type
SET EMPTY_SUBSTITUTION ON before using the PROVE command.
By default, this is OFF in order to reduce the number of
ambiguous unification possibilities that have to be selected
during the construction of a proof.  $)

$(
Hofstadter's MIU-system is a simple example of a formal
system that illustrates some concepts of Metamath.  See
Douglas R. Hofstadter, _Goedel, Escher, Bach:  An Eternal
Golden Braid_ (Vintage Books, New York, 1979), pp. 33ff. for
a description of the MIU-system.

The system has 3 constant symbols, M, I, and U.  The sole
axiom of the system is MI. There are 4 rules:
     Rule I:  If you possess a string whose last letter is I,
     you can add on a U at the end.
     Rule II:  Suppose you have Mx.  Then you may add Mxx to
     your collection.
     Rule III:  If III occurs in one of the strings in your
     collection, you may make a new string with U in place
     of III.
     Rule IV:  If UU occurs inside one of your strings, you
     can drop it.
Unfortunately, Rules III and IV do not have unique results:
strings could have more than one occurrence of III or UU.
This requires that we introduce the concept of an "MIU
well-formed formula" or wff, which allows us to construct
unique symbol sequences to which Rules III and IV can be
applied.
$)

$( First, we declare the constant symbols of the language.
Note that we need two symbols to distinguish the assertion
that a sequence is a wff from the assertion that it is a
theorem; we have arbitrarily chosen "wff" and "|-". $)
      $c M I U |- wff $. $( Declare constants $)

$( Next, we declare some variables. $)
     $v x y $.

$( Throughout our theory, we shall assume that these
variables represent wffs. $)
 wx   $f wff x $.
 wy   $f wff y $.

$( Define MIU-wffs.  We allow the empty sequence to be a
wff. $)

$( The empty sequence is a wff. $)
 we   $a wff $.
$( "M" after any wff is a wff. $)
 wM   $a wff x M $.
$( "I" after any wff is a wff. $)
 wI   $a wff x I $.
$( "U" after any wff is a wff. $)
 wU   $a wff x U $.

$( Assert the axiom. $)
 ax   $a |- M I $.

$( Assert the rules. $)
 ${
   Ia   $e |- x I $.
$( Given any theorem ending with "I", it remains a theorem
if "U" is added after it.  (We distinguish the label I_
from the math symbol I to conform to the 24-Jun-2006
Metamath spec.) $)
   I_    $a |- x I U $.
 $}
 ${
IIa  $e |- M x $.
$( Given any theorem starting with "M", it remains a theorem
if the part after the "M" is added again after it. $)
   II   $a |- M x x $.
 $}
 ${
   IIIa $e |- x I I I y $.
$( Given any theorem with "III" in the middle, it remains a
theorem if the "III" is replaced with "U". $)
   III  $a |- x U y $.
 $}
 ${
   IVa  $e |- x U U y $.
$( Given any theorem with "UU" in the middle, it remains a
theorem if the "UU" is deleted. $)
   IV   $a |- x y $.
  $}

$( Now we prove the theorem MUIIU.  You may be interested in
comparing this proof with that of Hofstadter (pp. 35 - 36).
$)
 theorem1  $p |- M U I I U $=
      we wM wU wI we wI wU we wU wI wU we wM we wI wU we wM
      wI wI wI we wI wI we wI ax II II I_ III II IV $.
\end{verbatim}\index{well-formed formula (wff)}

The \texttt{show proof /lemmon/renumber} command
yields the following display.  It is very similar
to the one in \cite[pp.~35--36]{Hofstadter}.\index{Hofstadter, Douglas R.}

\begin{verbatim}
1 ax             $a |- M I
2 1 II           $a |- M I I
3 2 II           $a |- M I I I I
4 3 I_           $a |- M I I I I U
5 4 III          $a |- M U I U
6 5 II           $a |- M U I U U I U
7 6 IV           $a |- M U I I U
\end{verbatim}

We note that Hofstadter's ``MU-puzzle,'' which asks whether
MU is a theorem of the MIU-system, cannot be answered using
the system above because the MU-puzzle is a question {\em
about} the system.  To prove the answer to the MU-puzzle,
a much more elaborate system is needed, namely one that
models the MIU-system within set theory.  (Incidentally, the
answer to the MU-puzzle is no.)

\chapter{Metamath Language EBNF}%
\label{BNF}%
\index{Metamath Language EBNF}

The following is a formal description of the basic Metamath language syntax
(with compressed proofs and support for unknown proof steps).
It is defined using the
Extended Backus--Naur Form (EBNF)\index{Extended Backus--Naur Form}\index{EBNF}
notation from W3C\index{W3C}
\textit{Extensible Markup Language (XML) 1.0 (Fifth Edition)}
(W3C Recommendation 26 November 2008) at
\url{https://www.w3.org/TR/xml/#sec-notation}.

The \texttt{database}
rule is processed until the end of the file (\texttt{EOF}).
The rules eventually require reading whitespace-separated tokens.
A token has an upper-case definition (see below)
or is a string constant in a non-token (such as \texttt{'\$a'}).
We intend for this to be correct, but if there is a conflict the
rules of section \ref{spec} govern. That section also discusses
non-syntax restrictions not shown here
(e.g., that each new label token
defined in a \texttt{hypothesis-stmt} or \texttt{assert-stmt}
must be unique).

\begin{verbatim}
database ::= outermost-scope-stmt*

outermost-scope-stmt ::=
  include-stmt | constant-stmt | stmt

/* File inclusion command; process file as a database.
   Databases should NOT have a comment in the filename. */
include-stmt ::= '$[' filename '$]'

/* Constant symbols declaration. */
constant-stmt ::= '$c' constant+ '$.'

/* A normal statement can occur in any scope. */
stmt ::= block | variable-stmt | disjoint-stmt |
  hypothesis-stmt | assert-stmt

/* A block. You can have 0 statements in a block. */
block ::= '${' stmt* '$}'

/* Variable symbols declaration. */
variable-stmt ::= '$v' variable+ '$.'

/* Disjoint variables. Simple disjoint statements have
   2 variables, i.e., "variable*" is empty for them. */
disjoint-stmt ::= '$d' variable variable variable* '$.'

hypothesis-stmt ::= floating-stmt | essential-stmt

/* Floating (variable-type) hypothesis. */
floating-stmt ::= LABEL '$f' typecode variable '$.'

/* Essential (logical) hypothesis. */
essential-stmt ::= LABEL '$e' typecode MATH-SYMBOL* '$.'

assert-stmt ::= axiom-stmt | provable-stmt

/* Axiomatic assertion. */
axiom-stmt ::= LABEL '$a' typecode MATH-SYMBOL* '$.'

/* Provable assertion. */
provable-stmt ::= LABEL '$p' typecode MATH-SYMBOL*
  '$=' proof '$.'

/* A proof. Proofs may be interspersed by comments.
   If '?' is in a proof it's an "incomplete" proof. */
proof ::= uncompressed-proof | compressed-proof
uncompressed-proof ::= (LABEL | '?')+
compressed-proof ::= '(' LABEL* ')' COMPRESSED-PROOF-BLOCK+

typecode ::= constant

filename ::= MATH-SYMBOL /* No whitespace or '$' */
constant ::= MATH-SYMBOL
variable ::= MATH-SYMBOL
\end{verbatim}

\needspace{2\baselineskip}
A \texttt{frame} is a sequence of 0 or more
\texttt{disjoint-{\allowbreak}stmt} and
\texttt{hypotheses-{\allowbreak}stmt} statements
(possibly interleaved with other non-\texttt{assert-stmt} statements)
followed by one \texttt{assert-stmt}.

\needspace{3\baselineskip}
Here are the rules for lexical processing (tokenization) beyond
the constant tokens shown above.
By convention these tokenization rules have upper-case names.
Every token is read for the longest possible length.
Whitespace-separated tokens are read sequentially;
note that the separating whitespace and \texttt{\$(} ... \texttt{\$)}
comments are skipped.

If a token definition uses another token definition, the whole thing
is considered a single token.
A pattern that is only part of a full token has a name beginning
with an underscore (``\_'').
An implementation could tokenize many tokens as a
\texttt{PRINTABLE-SEQUENCE}
and then check if it meets the more specific rule shown here.

Comments do not nest, and both \texttt{\$(} and \texttt{\$)}
have to be surrounded
by at least one whitespace character (\texttt{\_WHITECHAR}).
Technically comments end without consuming the trailing
\texttt{\_WHITECHAR}, but the trailing
\texttt{\_WHITECHAR} gets ignored anyway so we ignore that detail here.
Metamath language processors
are not required to support \texttt{\$)} followed
immediately by a bare end-of-file, because the closing
comment symbol is supposed to be followed by a
\texttt{\_WHITECHAR} such as a newline.

\begin{verbatim}
PRINTABLE-SEQUENCE ::= _PRINTABLE-CHARACTER+

MATH-SYMBOL ::= (_PRINTABLE-CHARACTER - '$')+

/* ASCII non-whitespace printable characters */
_PRINTABLE-CHARACTER ::= [#x21-#x7e]

LABEL ::= ( _LETTER-OR-DIGIT | '.' | '-' | '_' )+

_LETTER-OR-DIGIT ::= [A-Za-z0-9]

COMPRESSED-PROOF-BLOCK ::= ([A-Z] | '?')+

/* Define whitespace between tokens. The -> SKIP
   means that when whitespace is seen, it is
   skipped and we simply read again. */
WHITESPACE ::= (_WHITECHAR+ | _COMMENT) -> SKIP

/* Comments. $( ... $) and do not nest. */
_COMMENT ::= '$(' (_WHITECHAR+ (PRINTABLE-SEQUENCE - '$)')*
  _WHITECHAR+ '$)' _WHITECHAR

/* Whitespace: (' ' | '\t' | '\r' | '\n' | '\f') */
_WHITECHAR ::= [#x20#x09#x0d#x0a#x0c]
\end{verbatim}
% This EBNF was developed as a collaboration between
% David A. Wheeler\index{Wheeler, David A.},
% Mario Carneiro\index{Carneiro, Mario}, and
% Benoit Jubin\index{Jubin, Benoit}, inspired by a request
% (and a lot of initial work) by Benoit Jubin.
%
% \chapter{Disclaimer and Trademarks}
%
% Information in this document is subject to change without notice and does not
% represent a commitment on the part of Norman Megill.
% \vspace{2ex}
%
% \noindent Norman D. Megill makes no warranties, either express or implied,
% regarding the Metamath computer software package.
%
% \vspace{2ex}
%
% \noindent Any trademarks mentioned in this book are the property of
% their respective owners.  The name ``Metamath'' is a trademark of
% Norman Megill.
%
\cleardoublepage
\phantomsection  % fixes the link anchor
\addcontentsline{toc}{chapter}{\bibname}

\bibliography{metamath}
%% metamath.tex - Version of 24-May-2019
% If you change the date above, also change the "Printed date" below.
% SPDX-License-Identifier: CC0-1.0
%
%                              PUBLIC DOMAIN
%
% This file (specifically, the version of this file with the above date)
% has been released into the Public Domain per the
% Creative Commons CC0 1.0 Universal (CC0 1.0) Public Domain Dedication
% https://creativecommons.org/publicdomain/zero/1.0/
%
% The public domain release applies worldwide.  In case this is not
% legally possible, the right is granted to use the work for any purpose,
% without any conditions, unless such conditions are required by law.
%
% Several short, attributed quotations from copyrighted works
% appear in this file under the ``fair use'' provision of Section 107 of
% the United States Copyright Act (Title 17 of the {\em United States
% Code}).  The public-domain status of this file is not applicable to
% those quotations.
%
% Norman Megill - email: nm(at)alum(dot)mit(dot)edu
%
% David A. Wheeler also donates his improvements to this file to the
% public domain per the CC0.  He works at the Institute for Defense Analyses
% (IDA), but IDA has agreed that this Metamath work is outside its "lane"
% and is not a work by IDA.  This was specifically confirmed by
% Margaret E. Myers (Division Director of the Information Technology
% and Systems Division) on 2019-05-24 and by Ben Lindorf (General Counsel)
% on 2019-05-22.

% This file, 'metamath.tex', is self-contained with everything needed to
% generate the the PDF file 'metamath.pdf' (the _Metamath_ book) on
% standard LaTeX 2e installations.  The auxiliary files are embedded with
% "filecontents" commands.  To generate metamath.pdf file, run these
% commands under Linux or Cygwin in the directory that contains
% 'metamath.tex':
%
%   rm -f realref.sty metamath.bib
%   touch metamath.ind
%   pdflatex metamath
%   pdflatex metamath
%   bibtex metamath
%   makeindex metamath
%   pdflatex metamath
%   pdflatex metamath
%
% The warnings that occur in the initial runs of pdflatex can be ignored.
% For the final run,
%
%   egrep -i 'error|warn' metamath.log
%
% should show exactly these 5 warnings:
%
%   LaTeX Warning: File `realref.sty' already exists on the system.
%   LaTeX Warning: File `metamath.bib' already exists on the system.
%   LaTeX Font Warning: Font shape `OMS/cmtt/m/n' undefined
%   LaTeX Font Warning: Font shape `OMS/cmtt/bx/n' undefined
%   LaTeX Font Warning: Some font shapes were not available, defaults
%       substituted.
%
% Search for "Uncomment" below if you want to suppress hyperlink boxes
% in the PDF output file
%
% TYPOGRAPHICAL NOTES:
% * It is customary to use an en dash (--) to "connect" names of different
%   people (and to denote ranges), and use a hyphen (-) for a
%   single compound name. Examples of connected multiple people are
%   Zermelo--Fraenkel, Schr\"{o}der--Bernstein, Tarski--Grothendieck,
%   Hewlett--Packard, and Backus--Naur.  Examples of a single person with
%   a compound name include Levi-Civita, Mittag-Leffler, and Burali-Forti.
% * Use non-breaking spaces after page abbreviations, e.g.,
%   p.~\pageref{note2002}.
%
% --------------------------- Start of realref.sty -----------------------------
\begin{filecontents}{realref.sty}
% Save the following as realref.sty.
% You can then use it with \usepackage{realref}
%
% This has \pageref jumping to the page on which the ref appears,
% \ref jumping to the point of the anchor, and \sectionref
% jumping to the start of section.
%
% Author:  Anthony Williams
%          Software Engineer
%          Nortel Networks Optical Components Ltd
% Date:    9 Nov 2001 (posted to comp.text.tex)
%
% The following declaration was made by Anthony Williams on
% 24 Jul 2006 (private email to Norman Megill):
%
%   ``I hereby donate the code for realref.sty posted on the
%   comp.text.tex newsgroup on 9th November 2001, accessible from
%   http://groups.google.com/group/comp.text.tex/msg/5a0e1cc13ea7fbb2
%   to the public domain.''
%
\ProvidesPackage{realref}
\RequirePackage[plainpages=false,pdfpagelabels=true]{hyperref}
\def\realref@anchorname{}
\AtBeginDocument{%
% ensure every label is a possible hyperlink target
\let\realref@oldrefstepcounter\refstepcounter%
\DeclareRobustCommand{\refstepcounter}[1]{\realref@oldrefstepcounter{#1}
\edef\realref@anchorname{\string #1.\@currentlabel}%
}%
\let\realref@oldlabel\label%
\DeclareRobustCommand{\label}[1]{\realref@oldlabel{#1}\hypertarget{#1}{}%
\@bsphack\protected@write\@auxout{}{%
    \string\expandafter\gdef\protect\csname
    page@num.#1\string\endcsname{\thepage}%
    \string\expandafter\gdef\protect\csname
    ref@num.#1\string\endcsname{\@currentlabel}%
    \string\expandafter\gdef\protect\csname
    sectionref@name.#1\string\endcsname{\realref@anchorname}%
}\@esphack}%
\DeclareRobustCommand\pageref[1]{{\edef\a{\csname
            page@num.#1\endcsname}\expandafter\hyperlink{page.\a}{\a}}}%
\DeclareRobustCommand\ref[1]{{\edef\a{\csname
            ref@num.#1\endcsname}\hyperlink{#1}{\a}}}%
\DeclareRobustCommand\sectionref[1]{{\edef\a{\csname
            ref@num.#1\endcsname}\edef\b{\csname
            sectionref@name.#1\endcsname}\hyperlink{\b}{\a}}}%
}
\end{filecontents}
% ---------------------------- End of realref.sty ------------------------------

% --------------------------- Start of metamath.bib -----------------------------
\begin{filecontents}{metamath.bib}
@book{Albers, editor = "Donald J. Albers and G. L. Alexanderson",
  title = "Mathematical People",
  publisher = "Contemporary Books, Inc.",
  address = "Chicago",
  note = "[QA28.M37]",
  year = 1985 }
@book{Anderson, author = "Alan Ross Anderson and Nuel D. Belnap",
  title = "Entailment",
  publisher = "Princeton University Press",
  address = "Princeton",
  volume = 1,
  note = "[QA9.A634 1975 v.1]",
  year = 1975}
@book{Barrow, author = "John D. Barrow",
  title = "Theories of Everything:  The Quest for Ultimate Explanation",
  publisher = "Oxford University Press",
  address = "Oxford",
  note = "[Q175.B225]",
  year = 1991 }
@book{Behnke,
  editor = "H. Behnke and F. Backmann and K. Fladt and W. S{\"{u}}ss",
  title = "Fundamentals of Mathematics",
  volume = "I",
  publisher = "The MIT Press",
  address = "Cambridge, Massachusetts",
  note = "[QA37.2.B413]",
  year = 1974 }
@book{Bell, author = "J. L. Bell and M. Machover",
  title = "A Course in Mathematical Logic",
  publisher = "North-Holland",
  address = "Amsterdam",
  note = "[QA9.B3953]",
  year = 1977 }
@inproceedings{Blass, author = "Andrea Blass",
  title = "The Interaction Between Category Theory and Set Theory",
  pages = "5--29",
  booktitle = "Mathematical Applications of Category Theory (Proceedings
     of the Special Session on Mathematical Applications
     Category Theory, 89th Annual Meeting of the American Mathematical
     Society, held in Denver, Colorado January 5--9, 1983)",
  editor = "John Walter Gray",
  year = 1983,
  note = "[QA169.A47 1983]",
  publisher = "American Mathematical Society",
  address = "Providence, Rhode Island"}
@proceedings{Bledsoe, editor = "W. W. Bledsoe and D. W. Loveland",
  title = "Automated Theorem Proving:  After 25 Years (Proceedings
     of the Special Session on Automatic Theorem Proving,
     89th Annual Meeting of the American Mathematical
     Society, held in Denver, Colorado January 5--9, 1983)",
  year = 1983,
  note = "[QA76.9.A96.S64 1983]",
  publisher = "American Mathematical Society",
  address = "Providence, Rhode Island" }
@book{Boolos, author = "George S. Boolos and Richard C. Jeffrey",
  title = "Computability and Log\-ic",
  publisher = "Cambridge University Press",
  edition = "third",
  address = "Cambridge",
  note = "[QA9.59.B66 1989]",
  year = 1989 }
@book{Campbell, author = "John Campbell",
  title = "Programmer's Progress",
  publisher = "White Star Software",
  address = "Box 51623, Palo Alto, CA 94303",
  year = 1991 }
@article{DBLP:journals/corr/Carneiro14,
  author    = {Mario Carneiro},
  title     = {Conversion of {HOL} Light proofs into Metamath},
  journal   = {CoRR},
  volume    = {abs/1412.8091},
  year      = {2014},
  url       = {http://arxiv.org/abs/1412.8091},
  archivePrefix = {arXiv},
  eprint    = {1412.8091},
  timestamp = {Mon, 13 Aug 2018 16:47:05 +0200},
  biburl    = {https://dblp.org/rec/bib/journals/corr/Carneiro14},
  bibsource = {dblp computer science bibliography, https://dblp.org}
}
@article{CarneiroND,
  author    = {Mario Carneiro},
  title     = {Natural Deductions in the Metamath Proof Language},
  url       = {http://us.metamath.org/ocat/natded.pdf},
  year      = 2014
}
@inproceedings{Chou, author = "Shang-Ching Chou",
  title = "Proving Elementary Geometry Theorems Using {W}u's Algorithm",
  pages = "243--286",
  booktitle = "Automated Theorem Proving:  After 25 Years (Proceedings
     of the Special Session on Automatic Theorem Proving,
     89th Annual Meeting of the American Mathematical
     Society, held in Denver, Colorado January 5--9, 1983)",
  editor = "W. W. Bledsoe and D. W. Loveland",
  year = 1983,
  note = "[QA76.9.A96.S64 1983]",
  publisher = "American Mathematical Society",
  address = "Providence, Rhode Island" }
@book{Clemente, author = "Daniel Clemente Laboreo",
  title = "Introduction to natural deduction",
  year = 2014,
  url = "http://www.danielclemente.com/logica/dn.en.pdf" }
@incollection{Courant, author = "Richard Courant and Herbert Robbins",
  title = "Topology",
  pages = "573--590",
  booktitle = "The World of Mathematics, Volume One",
  editor = "James R. Newman",
  publisher = "Simon and Schuster",
  address = "New York",
  note = "[QA3.W67 1988]",
  year = 1956 }
@book{Curry, author = "Haskell B. Curry",
  title = "Foundations of Mathematical Logic",
  publisher = "Dover Publications, Inc.",
  address = "New York",
  note = "[QA9.C976 1977]",
  year = 1977 }
@book{Davis, author = "Philip J. Davis and Reuben Hersh",
  title = "The Mathematical Experience",
  publisher = "Birkh{\"{a}}user Boston",
  address = "Boston",
  note = "[QA8.4.D37 1982]",
  year = 1981 }
@incollection{deMillo,
  author = "Richard de Millo and Richard Lipton and Alan Perlis",
  title = "Social Processes and Proofs of Theorems and Programs",
  pages = "267--285",
  booktitle = "New Directions in the Philosophy of Mathematics",
  editor = "Thomas Tymoczko",
  publisher = "Birkh{\"{a}}user Boston, Inc.",
  address = "Boston",
  note = "[QA8.6.N48 1986]",
  year = 1986 }
@book{Edwards, author = "Robert E. Edwards",
  title = "A Formal Background to Mathematics",
  publisher = "Springer-Verlag",
  address = "New York",
  note = "[QA37.2.E38 v.1a]",
  year = 1979 }
@book{Enderton, author = "Herbert B. Enderton",
  title = "Elements of Set Theory",
  publisher = "Academic Press, Inc.",
  address = "San Diego",
  note = "[QA248.E5]",
  year = 1977 }
@book{Goodstein, author = "R. L. Goodstein",
  title = "Development of Mathematical Logic",
  publisher = "Springer-Verlag New York Inc.",
  address = "New York",
  note = "[QA9.G6554]",
  year = 1971 }
@book{Guillen, author = "Michael Guillen",
  title = "Bridges to Infinity",
  publisher = "Jeremy P. Tarcher, Inc.",
  address = "Los Angeles",
  note = "[QA93.G8]",
  year = 1983 }
@book{Hamilton, author = "Alan G. Hamilton",
  title = "Logic for Mathematicians",
  edition = "revised",
  publisher = "Cambridge University Press",
  address = "Cambridge",
  note = "[QA9.H298]",
  year = 1988 }
@unpublished{Harrison, author = "John Robert Harrison",
  title = "Metatheory and Reflection in Theorem Proving:
    A Survey and Critique",
  note = "Technical Report
    CRC-053.
    SRI Cambridge,
    Millers Yard, Cambridge, UK,
    1995.
    Available on the Web as
{\verb+http:+}\-{\verb+//www.cl.cam.ac.uk/users/jrh/papers/reflect.html+}"}
@TECHREPORT{Harrison-thesis,
        author          = "John Robert Harrison",
        title           = "Theorem Proving with the Real Numbers",
        institution   = "University of Cambridge Computer
                         Lab\-o\-ra\-to\-ry",
        address         = "New Museums Site, Pembroke Street, Cambridge,
                           CB2 3QG, UK",
        year            = 1996,
        number          = 408,
        type            = "Technical Report",
        note            = "Author's PhD thesis,
   available on the Web at
{\verb+http:+}\-{\verb+//www.cl.cam.ac.uk+}\-{\verb+/users+}\-{\verb+/jrh+}%
\-{\verb+/papers+}\-{\verb+/thesis.html+}"}
@book{Herrlich, author = "Horst Herrlich and George E. Strecker",
  title = "Category Theory:  An Introduction",
  publisher = "Allyn and Bacon Inc.",
  address = "Boston",
  note = "[QA169.H567]",
  year = 1973 }
@article{Hindley, author = "J. Roger Hindley and David Meredith",
  title = "Principal Type-Schemes and Condensed Detachment",
  journal = "The Journal of Symbolic Logic",
  volume = 55,
  year = 1990,
  note = "[QA.J87]",
  pages = "90--105" }
@book{Hofstadter, author = "Douglas R. Hofstadter",
  title = "G{\"{o}}del, Escher, Bach",
  publisher = "Basic Books, Inc.",
  address = "New York",
  note = "[QA9.H63 1980]",
  year = 1979 }
@article{Indrzejczak, author= "Andrzej Indrzejczak",
  title = "Natural Deduction, Hybrid Systems and Modal Logic",
  journal = "Trends in Logic",
  volume = 30,
  publisher = "Springer",
  year = 2010 }
@article{Kalish, author = "D. Kalish and R. Montague",
  title = "On {T}arski's Formalization of Predicate Logic with Identity",
  journal = "Archiv f{\"{u}}r Mathematische Logik und Grundlagenfor\-schung",
  volume = 7,
  year = 1965,
  note = "[QA.A673]",
  pages = "81--101" }
@article{Kalman, author = "J. A. Kalman",
  title = "Condensed Detachment as a Rule of Inference",
  journal = "Studia Logica",
  volume = 42,
  number = 4,
  year = 1983,
  note = "[B18.P6.S933]",
  pages = "443-451" }
@book{Kline, author = "Morris Kline",
  title = "Mathematical Thought from Ancient to Modern Times",
  publisher = "Oxford University Press",
  address = "New York",
  note = "[QA21.K516 1990 v.3]",
  year = 1972 }
@book{Klinel, author = "Morris Kline",
  title = "Mathematics, The Loss of Certainty",
  publisher = "Oxford University Press",
  address = "New York",
  note = "[QA21.K525]",
  year = 1980 }
@book{Kramer, author = "Edna E. Kramer",
  title = "The Nature and Growth of Modern Mathematics",
  publisher = "Princeton University Press",
  address = "Princeton, New Jersey",
  note = "[QA93.K89 1981]",
  year = 1981 }
@article{Knill, author = "Oliver Knill",
  title = "Some Fundamental Theorems in Mathematics",
  year = "2018",
  url = "https://arxiv.org/abs/1807.08416" }
@book{Landau, author = "Edmund Landau",
  title = "Foundations of Analysis",
  publisher = "Chelsea Publishing Company",
  address = "New York",
  edition = "second",
  note = "[QA241.L2541 1960]",
  year = 1960 }
@article{Leblanc, author = "Hugues Leblanc",
  title = "On {M}eyer and {L}ambert's Quantificational Calculus {FQ}",
  journal = "The Journal of Symbolic Logic",
  volume = 33,
  year = 1968,
  note = "[QA.J87]",
  pages = "275--280" }
@article{Lejewski, author = "Czeslaw Lejewski",
  title = "On Implicational Definitions",
  journal = "Studia Logica",
  volume = 8,
  year = 1958,
  note = "[B18.P6.S933]",
  pages = "189--208" }
@book{Levy, author = "Azriel Levy",
  title = "Basic Set Theory",
  publisher = "Dover Publications",
  address = "Mineola, NY",
  year = "2002"
}
@book{Margaris, author = "Angelo Margaris",
  title = "First Order Mathematical Logic",
  publisher = "Blaisdell Publishing Company",
  address = "Waltham, Massachusetts",
  note = "[QA9.M327]",
  year = 1967}
@book{Manin, author = "Yu I. Manin",
  title = "A Course in Mathematical Logic",
  publisher = "Springer-Verlag",
  address = "New York",
  note = "[QA9.M29613]",
  year = "1977" }
@article{Mathias, author = "Adrian R. D. Mathias",
  title = "A Term of Length 4,523,659,424,929",
  journal = "Synthese",
  volume = 133,
  year = 2002,
  note = "[Q.S993]",
  pages = "75--86" }
@article{Megill, author = "Norman D. Megill",
  title = "A Finitely Axiomatized Formalization of Predicate Calculus
     with Equality",
  journal = "Notre Dame Journal of Formal Logic",
  volume = 36,
  year = 1995,
  note = "[QA.N914]",
  pages = "435--453" }
@unpublished{Megillc, author = "Norman D. Megill",
  title = "A Shorter Equivalent of the Axiom of Choice",
  month = "June",
  note = "Unpublished",
  year = 1991 }
@article{MegillBunder, author = "Norman D. Megill and Martin W.
    Bunder",
  title = "Weaker {D}-Complete Logics",
  journal = "Journal of the IGPL",
  volume = 4,
  year = 1996,
  pages = "215--225",
  note = "Available on the Web at
{\verb+http:+}\-{\verb+//www.mpi-sb.mpg.de+}\-{\verb+/igpl+}%
\-{\verb+/Journal+}\-{\verb+/V4-2+}\-{\verb+/#Megill+}"}
}
@book{Mendelson, author = "Elliott Mendelson",
  title = "Introduction to Mathematical Logic",
  edition = "second",
  publisher = "D. Van Nostrand Company, Inc.",
  address = "New York",
  note = "[QA9.M537 1979]",
  year = 1979 }
@article{Meredith, author = "David Meredith",
  title = "In Memoriam {C}arew {A}rthur {M}eredith (1904-1976)",
  journal = "Notre Dame Journal of Formal Logic",
  volume = 18,
  year = 1977,
  note = "[QA.N914]",
  pages = "513--516" }
@article{CAMeredith, author = "C. A. Meredith",
  title = "Single Axioms for the Systems ({C},{N}), ({C},{O}) and ({A},{N})
      of the Two-Valued Propositional Calculus",
  journal = "The Journal of Computing Systems",
  volume = 3,
  year = 1953,
  pages = "155--164" }
@article{Monk, author = "J. Donald Monk",
  title = "Provability With Finitely Many Variables",
  journal = "The Journal of Symbolic Logic",
  volume = 27,
  year = 1971,
  note = "[QA.J87]",
  pages = "353--358" }
@article{Monks, author = "J. Donald Monk",
  title = "Substitutionless Predicate Logic With Identity",
  journal = "Archiv f{\"{u}}r Mathematische Logik und Grundlagenfor\-schung",
  volume = 7,
  year = 1965,
  pages = "103--121" }
  %% Took out this from above to prevent LaTeX underfull warning:
  % note = "[QA.A673]",
@book{Moore, author = "A. W. Moore",
  title = "The Infinite",
  publisher = "Routledge",
  address = "New York",
  note = "[BD411.M59]",
  year = 1989}
@book{Munkres, author = "James R. Munkres",
  title = "Topology: A First Course",
  publisher = "Prentice-Hall, Inc.",
  address = "Englewood Cliffs, New Jersey",
  note = "[QA611.M82]",
  year = 1975}
@article{Nemesszeghy, author = "E. Z. Nemesszeghy and E. A. Nemesszeghy",
  title = "On Strongly Creative Definitions:  A Reply to {V}. {F}. {R}ickey",
  journal = "Logique et Analyse (N.\ S.)",
  year = 1977,
  volume = 20,
  note = "[BC.L832]",
  pages = "111--115" }
@unpublished{Nemeti, author = "N{\'{e}}meti, I.",
  title = "Algebraizations of Quantifier Logics, an Overview",
  note = "Version 11.4, preprint, Mathematical Institute, Budapest,
    1994.  A shortened version without proofs appeared in
    ``Algebraizations of quantifier logics, an introductory overview,''
   {\em Studia Logica}, 50:485--569, 1991 [B18.P6.S933]"}
@article{Pavicic, author = "M. Pavi{\v{c}}i{\'{c}}",
  title = "A New Axiomatization of Unified Quantum Logic",
  journal = "International Journal of Theoretical Physics",
  year = 1992,
  volume = 31,
  note = "[QC.I626]",
  pages = "1753 --1766" }
@book{Penrose, author = "Roger Penrose",
  title = "The Emperor's New Mind",
  publisher = "Oxford University Press",
  address = "New York",
  note = "[Q335.P415]",
  year = 1989 }
@book{PetersonI, author = "Ivars Peterson",
  title = "The Mathematical Tourist",
  publisher = "W. H. Freeman and Company",
  address = "New York",
  note = "[QA93.P475]",
  year = 1988 }
@article{Peterson, author = "Jeremy George Peterson",
  title = "An automatic theorem prover for substitution and detachment systems",
  journal = "Notre Dame Journal of Formal Logic",
  volume = 19,
  year = 1978,
  note = "[QA.N914]",
  pages = "119--122" }
@book{Quine, author = "Willard Van Orman Quine",
  title = "Set Theory and Its Logic",
  edition = "revised",
  publisher = "The Belknap Press of Harvard University Press",
  address = "Cambridge, Massachusetts",
  note = "[QA248.Q7 1969]",
  year = 1969 }
@article{Robinson, author = "J. A. Robinson",
  title = "A Machine-Oriented Logic Based on the Resolution Principle",
  journal = "Journal of the Association for Computing Machinery",
  year = 1965,
  volume = 12,
  pages = "23--41" }
@article{RobinsonT, author = "T. Thacher Robinson",
  title = "Independence of Two Nice Sets of Axioms for the Propositional
    Calculus",
  journal = "The Journal of Symbolic Logic",
  volume = 33,
  year = 1968,
  note = "[QA.J87]",
  pages = "265--270" }
@book{Rucker, author = "Rudy Rucker",
  title = "Infinity and the Mind:  The Science and Philosophy of the
    Infinite",
  publisher = "Bantam Books, Inc.",
  address = "New York",
  note = "[QA9.R79 1982]",
  year = 1982 }
@book{Russell, author = "Bertrand Russell",
  title = "Mysticism and Logic, and Other Essays",
  publisher = "Barnes \& Noble Books",
  address = "Totowa, New Jersey",
  note = "[B1649.R963.M9 1981]",
  year = 1981 }
@article{Russell2, author = "Bertrand Russell",
  title = "Recent Work on the Principles of Mathematics",
  journal = "International Monthly",
  volume = 4,
  year = 1901,
  pages = "84"}
@article{Schmidt, author = "Eric Schmidt",
  title = "Reductions in Norman Megill's axiom system for complex numbers",
  url = "http://us.metamath.org/downloads/schmidt-cnaxioms.pdf",
  year = "2012" }
@book{Shoenfield, author = "Joseph R. Shoenfield",
  title = "Mathematical Logic",
  publisher = "Addison-Wesley Publishing Company, Inc.",
  address = "Reading, Massachusetts",
  year = 1967,
  note = "[QA9.S52]" }
@book{Smullyan, author = "Raymond M. Smullyan",
  title = "Theory of Formal Systems",
  publisher = "Princeton University Press",
  address = "Princeton, New Jersey",
  year = 1961,
  note = "[QA248.5.S55]" }
@book{Solow, author = "Daniel Solow",
  title = "How to Read and Do Proofs:  An Introduction to Mathematical
    Thought Process",
  publisher = "John Wiley \& Sons",
  address = "New York",
  year = 1982,
  note = "[QA9.S577]" }
@book{Stark, author = "Harold M. Stark",
  title = "An Introduction to Number Theory",
  publisher = "Markham Publishing Company",
  address = "Chicago",
  note = "[QA241.S72 1978]",
  year = 1970 }
@article{Swart, author = "E. R. Swart",
  title = "The Philosophical Implications of the Four-Color Problem",
  journal = "American Mathematical Monthly",
  year = 1980,
  volume = 87,
  month = "November",
  note = "[QA.A5125]",
  pages = "697--707" }
@book{Szpiro, author = "George G. Szpiro",
  title = "Poincar{\'{e}}'s Prize: The Hundred-Year Quest to Solve One
    of Math's Greatest Puzzles",
  publisher = "Penguin Books Ltd",
  address = "London",
  note = "[QA43.S985 2007]",
  year = 2007}
@book{Takeuti, author = "Gaisi Takeuti and Wilson M. Zaring",
  title = "Introduction to Axiomatic Set Theory",
  edition = "second",
  publisher = "Springer-Verlag New York Inc.",
  address = "New York",
  note = "[QA248.T136 1982]",
  year = 1982}
@inproceedings{Tarski, author = "Alfred Tarski",
  title = "What is Elementary Geometry",
  pages = "16--29",
  booktitle = "The Axiomatic Method, with Special Reference to Geometry and
     Physics (Proceedings of an International Symposium held at the University
     of California, Berkeley, December 26, 1957 --- January 4, 1958)",
  editor = "Leon Henkin and Patrick Suppes and Alfred Tarski",
  year = 1959,
  publisher = "North-Holland Publishing Company",
  address = "Amsterdam"}
@article{Tarski1965, author = "Alfred Tarski",
  title = "A Simplified Formalization of Predicate Logic with Identity",
  journal = "Archiv f{\"{u}}r Mathematische Logik und Grundlagenforschung",
  volume = 7,
  year = 1965,
  note = "[QA.A673]",
  pages = "61--79" }
@book{Tymoczko,
  title = "New Directions in the Philosophy of Mathematics",
  editor = "Thomas Tymoczko",
  publisher = "Birkh{\"{a}}user Boston, Inc.",
  address = "Boston",
  note = "[QA8.6.N48 1986]",
  year = 1986 }
@incollection{Wang,
  author = "Hao Wang",
  title = "Theory and Practice in Mathematics",
  pages = "129--152",
  booktitle = "New Directions in the Philosophy of Mathematics",
  editor = "Thomas Tymoczko",
  publisher = "Birkh{\"{a}}user Boston, Inc.",
  address = "Boston",
  note = "[QA8.6.N48 1986]",
  year = 1986 }
@manual{Webster,
  title = "Webster's New Collegiate Dictionary",
  organization = "G. \& C. Merriam Co.",
  address = "Springfield, Massachusetts",
  note = "[PE1628.W4M4 1977]",
  year = 1977 }
@manual{Whitehead, author = "Alfred North Whitehead",
  title = "An Introduction to Mathematics",
  year = 1911 }
@book{PM, author = "Alfred North Whitehead and Bertrand Russell",
  title = "Principia Mathematica",
  edition = "second",
  publisher = "Cambridge University Press",
  address = "Cambridge",
  year = "1927",
  note = "(3 vols.) [QA9.W592 1927]" }
@article{DBLP:journals/corr/Whalen16,
  author    = {Daniel Whalen},
  title     = {Holophrasm: a neural Automated Theorem Prover for higher-order logic},
  journal   = {CoRR},
  volume    = {abs/1608.02644},
  year      = {2016},
  url       = {http://arxiv.org/abs/1608.02644},
  archivePrefix = {arXiv},
  eprint    = {1608.02644},
  timestamp = {Mon, 13 Aug 2018 16:46:19 +0200},
  biburl    = {https://dblp.org/rec/bib/journals/corr/Whalen16},
  bibsource = {dblp computer science bibliography, https://dblp.org} }
@article{Wiedijk-revisited,
  author = {Freek Wiedijk},
  title = {The QED Manifesto Revisited},
  year = {2007},
  url = {http://mizar.org/trybulec65/8.pdf} }
@book{Wolfram,
  author = "Stephen Wolfram",
  title = "Mathematica:  A System for Doing Mathematics by Computer",
  edition = "second",
  publisher = "Addison-Wesley Publishing Co.",
  address = "Redwood City, California",
  note = "[QA76.95.W65 1991]",
  year = 1991 }
@book{Wos, author = "Larry Wos and Ross Overbeek and Ewing Lusk and Jim Boyle",
  title = "Automated Reasoning:  Introduction and Applications",
  edition = "second",
  publisher = "McGraw-Hill, Inc.",
  address = "New York",
  note = "[QA76.9.A96.A93 1992]",
  year = 1992 }

%
%
%[1] Church, Alonzo, Introduction to Mathematical Logic,
% Volume 1, Princeton University Press, Princeton, N. J., 1956.
%
%[2] Cohen, Paul J., Set Theory and the Continuum Hypothesis,
% W. A. Benjamin, Inc., Reading, Mass., 1966.
%
%[3] Hamilton, Alan G., Logic for Mathematicians, Cambridge
% University Press,
% Cambridge, 1988.

%[6] Kleene, Stephen Cole, Introduction to Metamathematics, D.  Van
% Nostrand Company, Inc., Princeton (1952).

%[13] Tarski, Alfred, "A simplified formalization of predicate
% logic with identity," Archiv fur Mathematische Logik und
% Grundlagenforschung, vol. 7 (1965), pp. 61-79.

%[14] Tarski, Alfred and Steven Givant, A Formalization of Set
% Theory Without Variables, American Mathematical Society Colloquium
% Publications, vol. 41, American Mathematical Society,
% Providence, R. I., 1987.

%[15] Zeman, J. J., Modal Logic, Oxford University Press, Oxford, 1973.
\end{filecontents}
% --------------------------- End of metamath.bib -----------------------------


%Book: Metamath
%Author:  Norman Megill Email:  nm at alum.mit.edu
%Author:  David A. Wheeler Email:  dwheeler at dwheeler.com

% A book template example
% http://www.stsci.edu/ftp/software/tex/bookstuff/book.template

\documentclass[leqno]{book} % LaTeX 2e. 10pt. Use [leqno,12pt] for 12pt
% hyperref 2002/05/27 v6.72r  (couldn't get pagebackref to work)
\usepackage[plainpages=false,pdfpagelabels=true]{hyperref}

\usepackage{needspace}     % Enable control over page breaks
\usepackage{breqn}         % automatic equation breaking
\usepackage{microtype}     % microtypography, reduces hyphenation

% Packages for flexible tables.  We need to be able to
% wrap text within a cell (with automatically-determined widths) AND
% split a table automatically across multiple pages.
% * "tabularx" wraps text in cells but only 1 page
% * "longtable" goes across pages but by itself is incompatible with tabularx
% * "ltxtable" combines longtable and tabularx, but table contents
%    must be in a separate file.
% * "ltablex" combines tabularx and longtable - must install specially
% * "booktabs" is recommended as a way to improve the look of tables,
%   but doesn't add these capabilities.
% * "tabu" much more capable and seems to be recommended. So use that.

\usepackage{makecell}      % Enable forced line splits within a table cell
\usepackage{longtable}     % Enable multi-page tables
\usepackage{tabu}          % Multi-page tables with wrapped text in a cell

% You can find more Tex packages using commands like:
% tlmgr search --file tabu.sty
% find /usr/share/texmf-dist/ -name '*tab*'
%
%%%%%%%%%%%%%%%%%%%%%%%%%%%%%%%%%%%%%%%%%%%%%%%%%%%%%%%%%%%%%%%%%%%%%%%%%%%%
% Uncomment the next 3 lines to suppress boxes and colors on the hyperlinks
%%%%%%%%%%%%%%%%%%%%%%%%%%%%%%%%%%%%%%%%%%%%%%%%%%%%%%%%%%%%%%%%%%%%%%%%%%%%
%\hypersetup{
%colorlinks,citecolor=black,filecolor=black,linkcolor=black,urlcolor=black
%}
%
\usepackage{realref}

% Restarting page numbers: try?
%   \printglossary
%   \cleardoublepage
%   \pagenumbering{arabic}
%   \setcounter{page}{1}    ???needed
%   \include{chap1}

% not used:
% \def\R2Lurl#1#2{\mbox{\href{#1}\texttt{#2}}}

\usepackage{amssymb}

% Version 1 of book: margins: t=.4, b=.2, ll=.4, rr=.55
% \usepackage{anysize}
% % \papersize{<height>}{<width>}
% % \marginsize{<left>}{<right>}{<top>}{<bottom>}
% \papersize{9in}{6in}
% % l/r 0.6124-0.6170 works t/b 0.2418-0.3411 = 192pp. 0.2926-03118=exact
% \marginsize{0.7147in}{0.5147in}{0.4012in}{0.2012in}

\usepackage{anysize}
% \papersize{<height>}{<width>}
% \marginsize{<left>}{<right>}{<top>}{<bottom>}
\papersize{9in}{6in}
% l/r 0.85in&0.6431-0.6539 works t/b ?-?
%\marginsize{0.85in}{0.6485in}{0.55in}{0.35in}
\marginsize{0.8in}{0.65in}{0.5in}{0.3in}

% \usepackage[papersize={3.6in,4.8in},hmargin=0.1in,vmargin={0.1in,0.1in}]{geometry}  % page geometry
\usepackage{special-settings}

\raggedbottom
\makeindex

\begin{document}
% Discourage page widows and orphans:
\clubpenalty=300
\widowpenalty=300

%%%%%%% load in AMS fonts %%%%%%% % LaTeX 2.09 - obsolete in LaTeX 2e
%\input{amssym.def}
%\input{amssym.tex}
%\input{c:/texmf/tex/plain/amsfonts/amssym.def}
%\input{c:/texmf/tex/plain/amsfonts/amssym.tex}

\bibliographystyle{plain}
\pagenumbering{roman}
\pagestyle{headings}

\thispagestyle{empty}

\hfill
\vfill

\begin{center}
{\LARGE\bf Metamath} \\
\vspace{1ex}
{\large A Computer Language for Mathematical Proofs} \\
\vspace{7ex}
{\large Norman Megill} \\
\vspace{7ex}
with extensive revisions by \\
\vspace{1ex}
{\large David A. Wheeler} \\
\vspace{7ex}
% Printed date. If changing the date below, also fix the date at the beginning.
2019-05-24
\end{center}

\vfill
\hfill

\newpage
\thispagestyle{empty}

\hfill
\vfill

\begin{center}
$\sim$\ {\sc Public Domain}\ $\sim$

\vspace{2ex}
This book (including its later revisions)
has been released into the Public Domain by Norman Megill per the
Creative Commons CC0 1.0 Universal (CC0 1.0) Public Domain Dedication
(\url{https://creativecommons.org/publicdomain/zero/1.0/}).
David A. Wheeler has done the same.
The public domain release applies worldwide.  In case this is not
legally possible, the right is granted to use the work for any purpose,
without any conditions, unless such conditions are required by law.

\vspace{3ex}
Several short, attributed quotations from copyrighted works
appear in this book under the ``fair use'' provision of Section 107 of
the United States Copyright Act (Title 17 of the {\em United States
Code}).  The public-domain status of this book is not applicable to
those quotations.

\vspace{3ex}
Any trademarks used in this book are the property of their owners.

% QA76.9.L63.M??

% \vspace{1ex}
%
% \vspace{1ex}
% {\small Permission is granted to make and distribute verbatim copies of this
% book
% provided the copyright notice and this
% permission notice are preserved on all copies.}
%
% \vspace{1ex}
% {\small Permission is granted to copy and distribute modified versions of this
% book under the conditions for verbatim copying, provided that the
% entire
% resulting derived work is distributed under the terms of a permission
% notice
% identical to this one.}
%
% \vspace{1ex}
% {\small Permission is granted to copy and distribute translations of this
% book into another language, under the above conditions for modified
% versions,
% except that this permission notice may be stated in a translation
% approved by the
% author.}
%
% \vspace{1ex}
% %{\small   For a copy of the \LaTeX\ source files for this book, contact
% %the author.} \\
% \ \\
% \ \\

\vspace{7ex}
% ISBN: 1-4116-3724-0 \\
ISBN: 978-1-4116-3724-5 \\
{\ } \\
Lulu Press \\
Morrisville, North Carolina\\
USA


\hfill
\vfill

Norman Megill\\ 93 Bridge St., Lexington, MA 02421 \\
E-mail address: \texttt{nm{\char`\@}alum.mit.edu} \\
\vspace{7ex}
David A. Wheeler \\
E-mail address: \texttt{dwheeler{\char`\@}dwheeler.com} \\
% See notes added at end of Preface for revision history. \\
% For current information on the Metamath software see \\
\vspace{7ex}
\url{http://metamath.org}
\end{center}

% \newpage
% \thispagestyle{empty}
%
% \hfill
% \vfill
%
% \begin{center}
% {\it To my son Robin Dwight Megill}
% \end{center}
%
% \vfill
% \hfill
%
% \newpage

\tableofcontents
%\listoftables

\chapter*{Preface}
\markboth{PREFACE}{PREFACE}
\addcontentsline{toc}{section}{Preface}


% (For current information, see the notes added at the
% end of this preface on p.~\pageref{note2002}.)

\subsubsection{Overview}

Metamath\index{Metamath} is a computer language and an associated computer
program for archiving, verifying, and studying mathematical proofs at a very
detailed level.  The Metamath language incorporates no mathematics per se but
treats all mathematical statements as mere sequences of symbols.  You provide
Metamath with certain special sequences (axioms) that tell it what rules
of inference are allowed.  Metamath is not limited to any specific field of
mathematics.  The Metamath language is simple and robust, with an
almost total absence of hard-wired syntax, and
I\footnote{Unless otherwise noted, the words
``I,'' ``me,'' and ``my'' refer to Norman Megill\index{Megill, Norman}, while
``we,'' ``us,'' and ``our'' refer to Norman Megill and
David A. Wheeler\index{Wheeler, David A.}.}
believe that it
provides about the simplest possible framework that allows essentially all of
mathematics to be expressed with absolute rigor.

% index test
%\newcommand{\nn}[1]{#1n}
%\index{aaa@bbb}
%\index{abc!def}
%\index{abd|see{qqq}}
%\index{abe|nn}
%\index{abf|emph}
%\index{abg|(}
%\index{abg|)}

Using the Metamath language, you can build formal or mathematical
systems\index{formal system}\footnote{A formal or mathematical system consists
of a collection of symbols (such as $2$, $4$, $+$ and $=$), syntax rules that
describe how symbols may be combined to form a legal expression (called a
well-formed formula or {\em wff}, pronounced ``whiff''), some starting wffs
called axioms, and inference rules that describe how theorems may be derived
(proved) from the axioms.  A theorem is a mathematical fact such as $2+2=4$.
Strictly speaking, even an obvious fact such as this must be proved from
axioms to be formally acceptable to a mathematician.}\index{theorem}
\index{axiom}\index{rule}\index{well-formed formula (wff)} that involve
inferences from axioms.  Although a database is provided
that includes a recommended set of axioms for standard mathematics, if you
wish you can supply your own symbols, syntax, axioms, rules, and definitions.

The name ``Metamath'' was chosen to suggest that the language provides a
means for {\em describing} mathematics rather than {\em being} the
mathematics itself.  Actually in some sense any mathematical language is
metamathematical.  Symbols written on paper, or stored in a computer,
are not mathematics itself but rather a way of expressing mathematics.
For example ``7'' and ``VII'' are symbols for denoting the number seven
in Arabic and Roman numerals; neither {\em is} the number seven.

If you are able to understand and write computer programs, you should be able
to follow abstract mathematics with the aid of Metamath.  Used in conjunction
with standard textbooks, Metamath can guide you step-by-step towards an
understanding of abstract mathematics from a very rigorous viewpoint, even if
you have no formal abstract mathematics background.  By using a single,
consistent notation to express proofs, once you grasp its basic concepts
Metamath provides you with the ability to immediately follow and dissect
proofs even in totally unfamiliar areas.

Of course, just being able follow a proof will not necessarily give you an
intuitive familiarity with mathematics.  Memorizing the rules of chess does not
give you the ability to appreciate the game of a master, and knowing how the
notes on a musical score map to piano keys does not give you the ability to
hear in your head how it would sound.  But each of these can be a first step.

Metamath allows you to explore proofs in the sense that you can see the
theorem referenced at any step expanded in as much detail as you want, right
down to the underlying axioms of logic and set theory (in the case of the set
theory database provided).  While Metamath will not replace the higher-level
understanding that can only be acquired through exercises and hard work, being
able to see how gaps in a proof are filled in can give you increased
confidence that can speed up the learning process and save you time when you
get stuck.

The Metamath language breaks down a mathematical proof into its tiniest
possible parts.  These can be pieced together, like interlocking
pieces in a puzzle, only in a way that produces correct and absolutely rigorous
mathematics.

The nature of Metamath\index{Metamath} enforces very precise mathematical
thinking, similar to that involved in writing a computer program.  A crucial
difference, though, is that once a proof is verified (by the Metamath program)
to be correct, it is definitely correct; it can never have a hidden
``bug.''\index{computer program bugs}  After getting used to the kind of rigor
and accuracy provided by Metamath, you might even be tempted to
adopt the attitude that a proof should never be considered correct until it
has been verified by a computer, just as you would not completely trust a
manual calculation until you have verified it on a
calculator.

My goal
for Metamath was a system for describing and verifying
mathematics that is completely universal yet conceptually as simple as
possible.  In approaching mathematics from an axiomatic, formal viewpoint, I
wanted Metamath to be able to handle almost any mathematical system, not
necessarily with ease, but at least in principle and hopefully in practice. I
wanted it to verify proofs with absolute rigor, and for this reason Metamath
is what might be thought of as a ``compile-only'' language rather than an
algorithmic or Turing-machine language (Pascal, C, Prolog, Mathematica,
etc.).  In other words, a database written in the Metamath
language doesn't ``do'' anything; it merely exhibits mathematical knowledge
and permits this knowledge to be verified as being correct.  A program in an
algorithmic language can potentially have hidden bugs\index{computer program
bugs} as well as possibly being hard to understand.  But each token in a
Metamath database must be consistent with the database's earlier
contents according to simple, fixed rules.
If a database is verified
to be correct,\footnote{This includes
verification that a sequential list of proof steps results in the specified
theorem.} then the mathematical content is correct if the
verifier is correct and the axioms are correct.
The verification program could be incorrect, but the verification algorithm
is relatively simple (making it unlikely to be implemented incorrectly
by the Metamath program),
and there are over a dozen Metamath database verifiers
written by different people in different programming languages
(so these different verifiers can act as multiple reviewers of a database).
The most-used Metamath database, the Metamath Proof Explorer
(aka \texttt{set.mm}\index{set theory database (\texttt{set.mm})}%
\index{Metamath Proof Explorer}),
is currently verified by four different Metamath verifiers written by
four different people in four different languages, including the
original Metamath program described in this book.
The only ``bugs'' that can exist are in the statement of the axioms,
for example if the axioms are inconsistent (a famous problem shown to be
unsolvable by G\"{o}del's incompleteness theorem\index{G\"{o}del's
incompleteness theorem}).
However, real mathematical systems have very few axioms, and these can
be carefully studied.
All of this provides extraordinarily high confidence that the verified database
is in fact correct.

The Metamath program
doesn't prove theorems automatically but is designed to verify proofs
that you supply to it.
The underlying Metamath language is completely general and has no built-in,
preconceived notions about your formal system\index{formal system}, its logic
or its syntax.
For constructing proofs, the Metamath program has a Proof Assistant\index{Proof
Assistant} which helps you fill in some of a proof step's details, shows you
what choices you have at any step, and verifies the proof as you build it; but
you are still expected to provide the proof.

There are many other programs that can process or generate information
in the Metamath language, and more continue to be written.
This is in part because the Metamath language itself is very simple
and intentionally easy to automatically process.
Some programs, such as \texttt{mmj2}\index{mmj2}, include a proof assistant
that can automate some steps beyond what the Metamath program can do.
Mario Carneiro has developed an algorithm for converting proofs from
the OpenTheory interchange format, which can be translated to and from
any of the HOL family of proof languages (HOL4, HOL Light, ProofPower,
and Isabelle), into the
Metamath language.\cite{DBLP:journals/corr/Carneiro14}\index{Carneiro, Mario}.
Daniel Whalen has developed Holophrasm, which can automatically
prove many Metamath proofs using
machine learning\index{machine learning}\index{artificial intelligence}
approaches
(including multiple neural networks\index{neural networks})\cite{DBLP:journals/corr/Whalen16}\index{Whalen, Daniel}.
However,
a discussion of these other programs is beyond the scope of this book.

Like most computer languages, the Metamath\index{Metamath} language uses the
standard ({\sc ascii}) characters on a computer keyboard, so it cannot
directly represent many of the special symbols that mathematicians use.  A
useful feature of the Metamath program is its ability to convert its notation
into the \LaTeX\ typesetting language.\index{latex@{\LaTeX}}  This feature
lets you convert the {\sc ascii} tokens you've defined into standard
mathematical symbols, so you end up with symbols and formulas you are familiar
with instead of somewhat cryptic {\sc ascii} representations of them.
The Metamath program can also generate HTML\index{HTML}, making it easy
to view results on the web and to see related information by using
hypertext links.

Metamath is probably conceptually different from anything you've seen
before and some aspects may take some getting used to.  This book will
help you decide whether Metamath suits your specific needs.

\subsubsection{Setting Your Expectations}
It is important for you to understand what Metamath\index{Metamath} is and is
not.  As mentioned, the Metamath program
is {\em not} an automated theorem prover but
rather a proof verifier.  Developing a database can be tedious, hard work,
especially if you want to make the proofs as short as possible, but it becomes
easier as you build up a collection of useful theorems.  The purpose of
Metamath is simply to document existing mathematics in an absolutely rigorous,
computer-verifiable way, not to aid directly in the creation of new
mathematics.  It also is not a magic solution for learning abstract
mathematics, although it may be helpful to be able to actually see the implied
rigor behind what you are learning from textbooks, as well as providing hints
to work out proofs that you are stumped on.

As of this writing, a sizable set theory database has been developed to
provide a foundation for many fields of mathematics, but much more work would
be required to develop useful databases for specific fields.

Metamath\index{Metamath} ``knows no math;'' it just provides a framework in
which to express mathematics.  Its language is very small.  You can define two
kinds of symbols, constants\index{constant} and variables\index{variable}.
The only thing Metamath knows how to do is to substitute strings of symbols
for the variables\index{substitution!variable}\index{variable substitution} in
an expression based on instructions you provide it in a proof, subject to
certain constraints you specify for the variables.  Even the decimal
representation of a number is merely a string of certain constants (digits)
which together, in a specific context, correspond to whatever mathematical
object you choose to define for it; unlike other computer languages, there is
no actual number stored inside the computer.  In a proof, you in effect
instruct Metamath what symbol substitutions to make in previous axioms or
theorems and join a sequence of them together to result in the desired
theorem.  This kind of symbol manipulation captures the essence of mathematics
at a preaxiomatic level.

\subsubsection{Metamath and Mathematical Literature}

In advanced mathematical literature, proofs are usually presented in the form
of short outlines that often only an expert can follow.  This is partly out of
a desire for brevity, but it would also be unwise (even if it were practical)
to present proofs in complete formal detail, since the overall picture would
be lost.\index{formal proof}

A solution I envision\label{envision} that would allow mathematics to remain
acceptable to the expert, yet increase its accessibility to non-specialists,
consists of a combination of the traditional short, informal proof in print
accompanied by a complete formal proof stored in a computer database.  In an
analogy with a computer program, the informal proof is like a set of comments
that describe the overall reasoning and content of the proof, whereas the
computer database is like the actual program and provides a means for anyone,
even a non-expert, to follow the proof in as much detail as desired, exploring
it back through layers of theorems (like subroutines that call other
subroutines) all the way back to the axioms of the theory.  In addition, the
computer database would have the advantage of providing absolute assurance
that the proof is correct, since each step can be verified automatically.

There are several other approaches besides Metamath to a project such
as this.  Section~\ref{proofverifiers} discusses some of these.

To us, a noble goal would be a database with hundreds of thousands of
theorems and their computer-verifiable proofs, encompassing a significant
fraction of known mathematics and available for instant access.
These would be fully verified by multiple independently-implemented verifiers,
to provide extremely high confidence that the proofs are completely correct.
The database would allow people to investigate whatever details they were
interested in, so that they could confirm whatever portions they wished.
Whether or not Metamath is an appropriate choice remains to be seen, but in
principle we believe it is sufficient.

\subsubsection{Formalism}

Over the past fifty years, a group of French mathematicians working
collectively under the pseudonym of Bourbaki\index{Bourbaki, Nicolas} have
co-authored a series of monographs that attempt to rigorously and
consistently formalize large bodies of mathematics from foundations.  On the
one hand, certainly such an effort has its merits; on the other hand, the
Bourbaki project has been criticized for its ``scholasticism'' and
``hyperaxiomatics'' that hide the intuitive steps that lead to the results
\cite[p.~191]{Barrow}\index{Barrow, John D.}.

Metamath unabashedly carries this philosophy to its extreme and no doubt is
subject to the same kind of criticism.  Nonetheless I think that in
conjunction with conventional approaches to mathematics Metamath can serve a
useful purpose.  The Bourbaki approach is essentially pedagogic, requiring the
reader to become intimately familiar with each detail in a very large
hierarchy before he or she can proceed to the next step.  The difference with
Metamath is that the ``reader'' (user) knows that all details are contained in
its computer database, available as needed; it does not demand that the user
know everything but conveniently makes available those portions that are of
interest.  As the body of all mathematical knowledge grows larger and larger,
no one individual can have a thorough grasp of its entirety.  Metamath
can finalize and put to rest any questions about the validity of any part of it
and can make any part of it accessible, in principle, to a non-specialist.

\subsubsection{A Personal Note}
Why did I develop Metamath\index{Metamath}?  I enjoy abstract mathematics, but
I sometimes get lost in a barrage of definitions and start to lose confidence
that my proofs are correct.  Or I reach a point where I lose sight of how
anything I'm doing relates to the axioms that a theory is based on and am
sometimes suspicious that there may be some overlooked implicit axiom
accidentally introduced along the way (as happened historically with Euclidean
geometry\index{Euclidean geometry}, whose omission of Pasch's
axiom\index{Pasch's axiom} went unnoticed for 2000 years
\cite[p.~160]{Davis}!). I'm also somewhat lazy and wish to avoid the effort
involved in re-verifying the gaps in informal proofs ``left to the reader;'' I
prefer to figure them out just once and not have to go through the same
frustration a year from now when I've forgotten what I did.  Metamath provides
better recovery of my efforts than scraps of paper that I can't
decipher anymore.  But mostly I find very appealing the idea of rigorously
archiving mathematical knowledge in a computer database, providing precision,
certainty, and elimination of human error.

\subsubsection{Note on Bibliography and Index}

The Bibliography usually includes the Library of Congress classification
for a work to make it easier for you to find it in on a university
library shelf.  The Index has author references to pages where their works
are cited, even though the authors' names may not appear on those pages.

\subsubsection{Acknowledgments}

Acknowledgments are first due to my wife, Deborah (who passed away on
September 4, 1998), for critiquing the manu\-script but most of all for
her patience and support.  I also wish to thank Joe Wright, Richard
Becker, Clarke Evans, Buddha Buck, and Jeremy Henty for helpful
comments.  Any errors, omissions, and other shortcomings are of course
my responsibility.

\subsubsection{Note Added June 22, 2005}\label{note2002}

The original, unpublished version of this book was written in 1997 and
distributed via the web.  The present edition has been updated to
reflect the current Metamath program and databases, as well as more
current {\sc url}s for Internet sites.  Thanks to Josh
Purinton\index{Purinton, Josh}, One Hand
Clapping, Mel L.\ O'Cat, and Roy F. Longton for pointing out
typographical and other errors.  I have also benefitted from numerous
discussions with Raph Levien\index{Levien, Raph}, who has extended
Metamath's philosophy of rigor to result in his {\em
Ghilbert}\index{Ghilbert} proof language (\url{http://ghilbert.org}).

Robert (Bob) Solovay\index{Solovay, Robert} communicated a new result of
A.~R.~D.~Mathias on the system of Bourbaki, and the text has been
updated accordingly (p.~\pageref{bourbaki}).

Bob also pointed out a clarification of the literature regarding
category theory and inaccessible cardinals\index{category
theory}\index{cardinal, inaccessible} (p.~\pageref{categoryth}),
and a misleading statement was removed from the text.  Specifically,
contrary to a statement in previous editions, it is possible to express
``There is a proper class of inaccessible cardinals'' in the language of
ZFC.  This can be done as follows:  ``For every set $x$ there is an
inaccessible cardinal $\kappa$ such that $\kappa$ is not in $x$.''
Bob writes:\footnote{Private communication, Nov.~30, 2002.}
\begin{quotation}
     This axiom is how Grothendieck presents category theory.  To each
inaccessible cardinal $\kappa$ one associates a Grothendieck universe
\index{Grothendieck, Alexander} $U(\kappa)$.  $U(\kappa)$ consists of
those sets which lie in a transitive set of cardinality less than
$\kappa$.  Instead of the ``category of all groups,'' one works relative
to a universe [considering the category of groups of cardinality less
than $\kappa$].  Now the category whose objects are all categories
``relative to the universe $U(\kappa)$'' will be a category not
relative to this universe but to the next universe.

     All of the things category theorists like to do can be done in this
framework.  The only controversial point is whether the Grothen\-dieck
axiom is too strong for the needs of category theorists.  Mac Lane
\index{Mac Lane, Saunders} argues that ``one universe is enough'' and
Feferman\index{Feferman, Solomon} has argued that one can get by with
ordinary ZFC.  I don't find Feferman's arguments persuasive.  Mac Lane
may be right, but when I think about category theory I do it \`{a} la
Grothendieck.

        By the way Mizar\index{Mizar} adds the axiom ``there is a proper
class of inaccessibles'' precisely so as to do category theory.
\end{quotation}

The most current information on the Metamath program and databases can
always be found at \url{http://metamath.org}.


\subsubsection{Note Added June 24, 2006}\label{note2006}

The Metamath spec was restricted slightly to make parsers easier to
write.  See the footnote on p.~\pageref{namespace}.

%\subsubsection{Note Added July 24, 2006}\label{note2006b}
\subsubsection{Note Added March 10, 2007}\label{note2006b}

I am grateful to Anthony Williams\index{Williams, Anthony} for writing
the \LaTeX\ package called {\tt realref.sty} and contributing it to the
public domain.  This package allows the internal hyperlinks in a {\sc
pdf} file to anchor to specific page numbers instead of just section
titles, making the navigation of the {\sc pdf} file for this book much
more pleasant and ``logical.''

A typographical error found by Martin Kiselkov was corrected.
A confusing remark about unification was deleted per suggestion of
Mel O'Cat.

\subsubsection{Note Added May 27, 2009}\label{note2009}

Several typos found by Kim Sparre were corrected.  A note was added that
the Poincar\'{e} conjecture has been proved (p.~\pageref{poincare}).

\subsubsection{Note Added Nov. 17, 2014}\label{note2014}

The statement of the Schr\"{o}der--Bernstein theorem was corrected in
Section~\ref{trust}.  Thanks to Bob Solovay for pointing out the error.

\subsubsection{Note Added May 25, 2016}\label{note2016}

Thanks to Jerry James for correcting 16 typos.

\subsubsection{Note Added February 25, 2019}\label{note201902}

David A. Wheeler\index{Wheeler, David A.}
made a large number of improvements and updates,
in coordination with Norman Megill.
The predicate calculus axioms were renumbered, and the text makes
it clear that they are based on Tarski's system S2;
the one slight deviation in axiom ax-6 is explained and justified.
The real and complex number axioms were modified to be consistent with
\texttt{set.mm}\index{set theory database (\texttt{set.mm})}%
\index{Metamath Proof Explorer}.
Long-awaited specification changes ``1--8'' were made,
which clarified previously ambiguous points.
Some errors in the text involving \texttt{\$f} and
\texttt{\$d} statements were corrected (the spec was correct, but
the in-book explanations unintentionally contradicted the spec).
We now have a system for automatically generating narrow PDFs,
so that those with smartphones can have easy access to the current
version of this document.
A new section on deduction was added;
it discusses the standard deduction theorem,
the weak deduction theorem,
deduction style, and natural deduction.
Many minor corrections (too numerous to list here) were also made.

\subsubsection{Note Added March 7, 2019}\label{note201903}

This added a description of the Matamath language syntax in
Extended Backus--Naur Form (EBNF)\index{Extended Backus--Naur Form}\index{EBNF}
in Appendix \ref{BNF}, added a brief explanation about typecodes,
inserted more examples in the deduction section,
and added a variety of smaller improvements.

\subsubsection{Note Added April 7, 2019}\label{note201904}

This version clarified the proper substitution notation, improved the
discussion on the weak deduction theorem and natural deduction,
documented the \texttt{undo} command, updated the information on
\texttt{write source}, changed the typecode
from \texttt{set} to \texttt{setvar} to be consistent with the current
version of \texttt{set.mm}, added more documentation about comment markup
(e.g., documented how to create headings), and clarified the
differences between various assertion forms (in particular deduction form).

\subsubsection{Note Added May 20, 2019}\label{note201905}

This version fixes various small issues and added specific examples
of forms (deduction form, inference form, and closed form).

\chapter{Introduction}
\pagenumbering{arabic}

\begin{quotation}
  {\em {\em I.M.:}  No, no.  There's nothing subjective about it!  Everybody
knows what a proof is.  Just read some books, take courses from a competent
mathematician, and you'll catch on.

{\em Student:}  Are you sure?

{\em I.M.:}  Well---it is possible that you won't, if you don't have any
aptitude for it.  That can happen, too.

{\em Student:}  Then {\em you} decide what a proof is, and if I don't learn
to decide in the same way, you decide I don't have any aptitude.

{\em I.M.:}  If not me, then who?}
    \flushright\sc  ``The Ideal Mathematician''
    \index{Davis, Phillip J.}
    \footnote{\cite{Davis}, p.~40.}\\
\end{quotation}

In the past century, brilliant mathematicians have discovered almost
unimaginably profound results that rank among the crowning intellectual
achievements of mankind.  However, there is a sense in which modern abstract
mathematics is behind the times, stuck in an era before computers existed.
While no one disputes the remarkable results that have been achieved,
communicating these results in a precise way to the uninitiated is virtually
impossible.  To describe these results, a terse informal language is used which
despite its elegance is very difficult to learn.  This informal language is not
imprecise, far from it, but rather it often has omitted detail
and symbols with hidden context that are
implicitly understood by an expert but few others.  Extremely complex technical
meanings are associated with innocent-sounding English words such as
``compact'' and ``measurable'' that barely hint at what is actually being
said.  Anyone who does not keep the precise technical meaning constantly in
mind is bound to fail, and acquiring the ability to do this can be achieved
only through much practice and hard work.  Only the few who complete the
painful learning experience can join the small in-group of pure
mathematicians.  The informal language effectively cuts off the true nature of
their knowledge from most everyone else.

Metamath\index{Metamath} makes abstract mathematics more concrete.  It allows
a computer to keep track of the complexity associated with each word or symbol
with absolute rigor.  You can explore this complexity at your leisure, to
whatever degree you desire.  Whether or not you believe that concepts such as
infinity actually ``exist'' outside of the mind, Metamath lets you get to the
foundation for what's really being said.

Metamath also enables completely rigorous and thorough proof verification.
Its language is simple enough so that you
don't have to rely on the authority of experts but can verify the results
yourself, step by step.  If you want to attempt to derive your own results,
Metamath will not let you make a mistake in reasoning.
Even professional mathematicians make mistakes; Metamath makes it possible
to thoroughly verify that proofs are correct.

Metamath\index{Metamath} is a computer language and an associated computer
program for archiving, verifying, and studying mathematical proofs at a very
detailed level.
The Metamath language
describes formal\index{formal system} mathematical
systems and expresses proofs of theorems in those systems.  Such a language
is called a metalanguage\index{metalanguage} by mathematicians.
The Metamath program is a computer program that verifies
proofs expressed in the Metamath language.
The Metamath program does not have the built-in
ability to make logical inferences; it just makes a series of symbol
substitutions according to instructions given to it in a proof
and verifies that the result matches the expected theorem.  It makes logical
inferences based only on rules of logic that are contained in a set of
axioms\index{axiom}, or first principles, that you provide to it as the
starting point for proofs.

The complete specification of the Metamath language is only four pages long
(Section~\ref{spec}, p.~\pageref{spec}).  Its simplicity may at first make you
wonder how it can do much of anything at all.  But in fact the kinds of
symbol manipulations it performs are the ones that are implicitly done in all
mathematical systems at the lowest level.  You can learn it relatively quickly
and have complete confidence in any mathematical proof that it verifies.  On
the other hand, it is powerful and general enough so that virtually any
mathematical theory, from the most basic to the deeply abstract, can be
described with it.

Although in principle Metamath can be used with any
kind of mathematics, it is best suited for abstract or ``pure'' mathematics
that is mostly concerned with theorems and their proofs, as opposed to the
kind of mathematics that deals with the practical manipulation of numbers.
Examples of branches of pure mathematics are logic\index{logic},\footnote{Logic
is the study of statements that are universally true regardless of the objects
being described by the statements.  An example is the statement, ``if $P$
implies $Q$, then either $P$ is false or $Q$ is true.''} set theory\index{set
theory},\footnote{Set theory is the study of general-purpose mathematical objects called
``sets,'' and from it essentially all of mathematics can be derived.  For
example, numbers can be defined as specific sets, and their properties
can be explored using the tools of set theory.} number theory\index{number
theory},\footnote{Number theory deals with the properties of positive and
negative integers (whole numbers).} group theory\index{group
theory},\footnote{Group theory studies the properties of mathematical objects
called groups that obey a simple set of axioms and have properties of symmetry
that make them useful in many other fields.} abstract algebra\index{abstract
algebra},\footnote{Abstract algebra includes group theory and also studies
groups with additional properties that qualify them as ``rings'' and
``fields.''  The set of real numbers is a familiar example of a field.},
analysis\index{analysis} \index{real and complex numbers}\footnote{Analysis is
the study of real and complex numbers.} and
topology\index{topology}.\footnote{One area studied by topology are properties
that remain unchanged when geometrical objects undergo stretching
deformations; for example a doughnut and a coffee cup each have one hole (the
cup's hole is in its handle) and are thus considered topologically
equivalent.  In general, though, topology is the study of abstract
mathematical objects that obey a certain (surprisingly simple) set of axioms.
See, for example, Munkres \cite{Munkres}\index{Munkres, James R.}.} Even in
physics, Metamath could be applied to certain branches that make use of
abstract mathematics, such as quantum logic\index{quantum logic} (used to study
aspects of quantum mechanics\index{quantum mechanics}).

On the other hand, Metamath\index{Metamath} is less suited to applications
that deal primarily with intensive numeric computations.  Metamath does not
have any built-in representation of numbers\index{Metamath!representation of
numbers}; instead, a specific string of symbols (digits) must be syntactically
constructed as part of any proof in which an ordinary number is used.  For
this reason, numbers in Metamath are best limited to specific constants that
arise during the course of a theorem or its proof.  Numbers are only a tiny
part of the world of abstract mathematics.  The exclusion of built-in numbers
was a conscious decision to help achieve Metamath's simplicity, and there are
other software tools if you have different mathematical needs.
If you wish to quickly solve algebraic problems, the computer algebra
programs\index{computer algebra system} {\sc
macsyma}\index{macsyma@{\sc macsyma}}, Mathematica\index{Mathematica}, and
Maple\index{Maple} are specifically suited to handling numbers and
algebra efficiently.
If you wish to simply calculate numeric or matrix expressions easily,
tools such as Octave\index{Octave} may be a better choice.

After learning Metamath's basic statement types, any
tech\-ni\-cal\-ly ori\-ent\-ed person, mathematician or not, can
immediately trace
any theorem proved in the language as far back as he or she wants, all the way
to the axioms on which the theorem is based.  This ability suggests a
non-traditional way of learning about pure mathematics.  Used in conjunction
with traditional methods, Metamath could make pure mathematics accessible to
people who are not sufficiently skilled to figure out the implicit detail in
ordinary textbook proofs.  Once you learn the axioms of a theory, you can have
complete confidence that everything you need to understand a proof you are
studying is all there, at your beck and call, allowing you to focus in on any
proof step you don't understand in as much depth as you need, without worrying
about getting stuck on a step you can't figure out.\footnote{On the other
hand, writing proofs in the Metamath language is challenging, requiring
a degree of rigor far in excess of that normally taught to students.  In a
classroom setting, I doubt that writing Metamath proofs would ever replace
traditional homework exercises involving informal proofs, because the time
needed to work out the details would not allow a course to
cover much material.  For students who have trouble grasping the implied rigor
in traditional material, writing a few simple proofs in the Metamath language
might help clarify fuzzy thought processes.  Although somewhat difficult at
first, it eventually becomes fun to do, like solving a puzzle, because of the
instant feedback provided by the computer.}

Metamath\index{Metamath} is probably unlike anything you have
encountered before.  In this first chapter we will look at the philosophy and
use of computers in mathematics in order to better understand the motivation
behind Metamath.  The material in this chapter is not required in order to use
Metamath.  You may skip it if you are impatient, but I hope you will find it
educational and enjoyable.  If you want to start experimenting with the
Metamath program right away, proceed directly to Chapter~\ref{using}
(p.~\pageref{using}).  To
learn the Metamath language, skim Chapter~\ref{using} then proceed to
Chapter~\ref{languagespec} (p.~\pageref{languagespec}).

\section{Mathematics as a Computer Language}

\begin{quote}
  {\em The study of mathematics is apt to commence in
dis\-ap\-point\-ment.\ldots \\
We are told that by its aid the stars are weighted
and the billions of molecules in a drop of water are counted.  Yet, like the
ghost of Hamlet's father, this great science eludes the efforts of our mental
weapons to grasp it.}
  \flushright\sc  Alfred North Whitehead\footnote{\cite{Whitehead}, ch.\ 1.}\\
\end{quote}\index{Whitehead, Alfred North}

\subsection{Is Mathematics ``User-Friendly''?}

Suppose you have no formal training in abstract mathematics.  But popular
books you've read offer tempting glimpses of this world filled with profound
ideas that have stirred the human spirit.  You are not satisfied with the
informal, watered-down descriptions you've read but feel it is important to
grasp the underlying mathematics itself to understand its true meaning. It's
not practical to go back to school to learn it, though; you don't want to
dedicate years of your life to it.  There are many important things in life,
and you have to set priorities for what's important to you.  What would happen
if you tried to pursue it on your own, in your spare time?

After all, you were able to learn a computer programming language such as
Pascal on your own without too much difficulty, even though you had no formal
training in computers.  You don't claim to be an expert in software design,
but you can write a passable program when necessary to suit your needs.  Even
more important, you know that you can look at anyone else's Pascal program, no
matter how complex, and with enough patience figure out exactly how it works,
even though you are not a specialist.  Pascal allows you do anything that a
computer can do, at least in principle.  Thus you know you have the ability,
in principle, to follow anything that a computer program can do:  you just
have to break it down into small enough pieces.

Here's an imaginary scenario of what might happen if you na\-ive\-ly a\-dopted
this same view of abstract mathematics and tried to pick it up on your own, in
a period of time comparable to, say, learning a computer programming
language.

\subsubsection{A Non-Mathematician's Quest for Truth}

\begin{quote}
  {\em \ldots my daughters have been studying (chemistry) for several
se\-mes\-ters, think they have learned differential and integral calculus in
school, and yet even today don't know why $x\cdot y=y\cdot x$ is true.}
  \flushright\sc  Edmund Landau\footnote{\cite{Landau}, p.~vi.}\\
\end{quote}\index{Landau, Edmund}

\begin{quote}
  {\em Minus times minus is plus,\\
The reason for this we need not discuss.}
  \flushright\sc W.\ H.\ Auden\footnote{As quoted in \cite{Guillen}, p.~64.}\\
\end{quote}\index{Auden, W.\ H.}\index{Guillen, Michael}

We'll suppose you are a technically oriented professional, perhaps an engineer, a
computer programmer, or a physicist, but probably not a mathematician.  You
consider yourself reasonably intelligent.  You did well in school, learning a
variety of methods and techniques in practical mathematics such as calculus and
differential equations.  But rarely did your courses get into anything
resembling modern abstract mathematics, and proofs were something that appeared
only occasionally in your textbooks, a kind of necessary evil that was
supposed to convince you of a certain key result.  Most of your
homework consisted of exercises that gave you practice in the techniques, and
you were hardly ever asked to come up with a proof of your own.

You find yourself curious about advanced, abstract mathematics.  You are
driven by an inner conviction that it is important to understand and
appreciate some of the most profound knowledge discovered by mankind.  But it
seems very hard to learn, something that only certain gifted longhairs can
access and understand.  You are frustrated that it seems forever cut off from
you.

Eventually your curiosity drives you to do something about it.
You set for yourself a goal of ``really'' understanding mathematics:  not just
how to manipulate equations in algebra or calculus according to cookbook
rules, but rather to gain a deep understanding of where those rules come from.
In fact, you're not thinking about this kind of ordinary mathematics at all,
but about a much more abstract, ethereal realm of pure mathematics, where
famous results such as G\"{o}del's incompleteness theorem\index{G\"{o}del's
incompleteness theorem} and Cantor's different kinds of infinities
reside.

You have probably read a number of popular books, with titles like {\em
Infinity and the Mind} \cite{Rucker}\index{Rucker, Rudy}, on topics such as
these.  You found them inspiring but at the same time somewhat
unsatisfactory.  They gave you a general idea of what these results are about,
but if someone asked you to prove them, you wouldn't have the faintest idea of
where to begin.   Sure, you could give the same overall outline that you
learned from the popular books; and in a general sort of way, you do have an
understanding.  But deep down inside, you know that there is a rigor that is
missing, that probably there are many subtle steps and pitfalls along the way,
and ultimately it seems you have to place your trust in the experts in the
field.  You don't like this; you want to be able to verify these results for
yourself.

So where do you go next?  As a first step, you decide to look up some of the
original papers on the theorems you are curious about, or better, obtain some
standard textbooks in the field.  You look up a theorem you want to
understand.  Sure enough, it's there, but it's expressed with strange
terms and odd symbols that mean absolutely nothing to you.  It might as well be written in
a foreign language you've never seen before, whose symbols are totally alien.
You look at the proof, and you haven't the foggiest notion what each step
means, much less how one step follows from another.  Well, obviously you have
a lot to learn if you want to understand this stuff.

You feel that you could probably understand it by
going back to college for another three to six years and getting a math
degree.  But that does not fit in with your career and the other things in
your life and would serve no practical purpose.  You decide to seek a quicker
path.  You figure you'll just trace your way back to the beginning, step by
step, as you would do with a computer program, until you understand it.  But
you quickly find that this is not possible, since you can't even understand
enough to know what you have to trace back to.

Maybe a different approach is in order---maybe you should start at the
beginning and work your way up.  First, you read the introduction to the book
to find out what the prerequisites are.  In a similar fashion, you trace your
way back through two or three more books, finally arriving at one that seems
to start at a beginning:  it lists the axioms of arithmetic.  ``Aha!'' you
naively think, ``This must be the starting point, the source of all mathematical
knowledge.'' Or at least the starting point for mathematics dealing with
numbers; you have to start somewhere and have no idea what the starting point
for other mathematics would be.  But the word ``axioms'' looks promising.  So
you eagerly read along and work through some elementary exercises at the
beginning of the book.  You feel vaguely bothered:  these
don't seem like axioms at all, at least not in the sense that you want to
think of axioms.  Axioms imply a starting point from which everything else can
be built up, according to precise rules specified in the axiom system.  Even
though you can understand the first few proofs in an informal way,
and are able to do some of the
exercises, it's hard to pin down precisely what the
rules are.   Sure, each step seems to follow logically from the others, but
exactly what does that mean?  Is the ``logic'' just a matter of common sense,
something vague that we all understand but can never quite state precisely?

You've spent a number of years, off and on, programming computers, and you
know that in the case of computer languages there is no question of what the
rules are---they are precise and crystal clear.  If you follow them, your
program will work, and if you don't, it won't.  No matter how complex a
program, it can always be broken down into simpler and simpler pieces, until
you can ultimately identify the bits that are moved around to perform a
specific function.  Some programs might require a lot of perseverance to
accomplish this, but if you focus on a specific portion of it, you don't even
necessarily have to know how the rest of it works. Shouldn't there be an
analogy in mathematics?

You decide to apply the ultimate test:  you ask yourself how a computer could
verify or ensure that the steps in these proofs follow from one another.
Certainly mathematics must be at least as precisely defined as a computer
language, if not more so; after all, computer science itself is based on it.
If you can get a computer to verify these proofs, then you should also be
able, in principle, to understand them yourself in a very crystal clear,
precise way.

You're in for a surprise:  you can conceive of no way to convert the
proofs, which are in English, to a form that the computer can understand.
The proofs are filled with phrases such as ``assume there exists a unique
$x$\ldots'' and ``given any $y$, let $z$ be the number such that\ldots''  This
isn't the kind of logic you are used to in computer programming, where
everything, even arithmetic, reduces to Boolean ones and zeroes if you care to
break it down sufficiently.  Even though you think you understand the proofs,
there seems to be some kind of higher reasoning involved rather than precise
rules that define how you manipulate the symbols in the axioms.  Whatever it
is, it just isn't obvious how you would express it to a computer, and the more
you think about it, the more puzzled and confused you get, to the point where
you even wonder whether {\em you} really understand it.  There's a lot more to
these axioms of arithmetic than meets the eye.

Nobody ever talked about this in school in your applied math and engineering
courses.  You just learned the rules they gave you, not quite understanding
how or why they worked, sometimes vaguely suspicious or uncertain of them, and
through homework problems and osmosis learned how to present solutions that
satisfied the instructor and earned you an ``A.''  Rarely did you actually
``prove'' anything in a rigorous way, and the math majors who did do stuff
like that seemed to be in a different world.

Of course, there are computer algebra programs that can do mathematics, and
rather impressively.  They can instantly solve the integrals that you
struggled with in freshman calculus, and do much, much more.  But when you
look at these programs, what you see is a big collection of algorithms and
techniques that evolved and were added to over time, along with some basic
software that manipulates symbols.  Each algorithm that is built in is the
result of someone's theorem whose proof is omitted; you just have to trust the
person who proved it and the person who programmed it in and hope there are no
bugs.\index{computer program bugs}  Somehow this doesn't seem to be the
essence of mathematics.  Although computer algebra systems can generate
theorems with amazing speed, they can't actually prove a single one of them.

After some puzzlement, you revisit some popular books on what mathematics is
all about.  Somewhere you read that all of mathematics is actually derived
from something called ``set theory.''  This is a little confusing, because
nowhere in the book that presented the axioms of arithmetic was there any
mention of set theory, or if there was, it seemed to be just a tool that helps
you describe things better---the set of even numbers, that sort of thing.  If
set theory is the basis for all mathematics, then why are additional axioms
needed for arithmetic?

Something is wrong but you're not sure what.  One of your friends is a pure
mathematician.  He knows he is unable to communicate to you what he does for a
living and seems to have little interest in trying.  You do know that for him,
proofs are what mathematics is all about. You ask him what a proof is, and he
essentially tells you that, while of course it's based on logic, really it's
something you learn by doing it over and over until you pick it up.  He refers
you to a book, {\em How to Read and Do Proofs} \cite{Solow}.\index{Solow,
Daniel}  Although this book helps you understand traditional informal proofs,
there is still something missing you can't seem to pin down yet.

You ask your friend how you would go about having a computer verify a proof.
At first he seems puzzled by the question; why would you want to do that?
Then he says it's not something that would make any sense to do, but he's
heard that you'd have to break the proof down into thousands or even millions
of individual steps to do such a thing, because the reasoning involved is at
such a high level of abstraction.  He says that maybe it's something you could
do up to a point, but the computer would be completely impractical once you
get into any meaningful mathematics.  There, the only way you can verify a
proof is by hand, and you can only acquire the ability to do this by
specializing in the field for a couple of years in grad school.  Anyway, he
thinks it all has to do with set theory, although he has never taken a formal
course in set theory but just learned what he needed as he went along.

You are intrigued and amazed.  Apparently a mathematician can grasp as a
single concept something that would take a computer a thousand or a million
steps to verify, and have complete confidence in it.  Each one of these
thousand or million steps must be absolutely correct, or else the whole proof
is meaningless.  If you added a million numbers by hand, would you trust the
result?  How do you really know that all these steps are correct, that there
isn't some subtle pitfall in one of these million steps, like a bug in a
computer program?\index{computer program bugs}  After all, you've read that
famous mathematicians have occasionally made mistakes, and you certainly know
you've made your share on your math homework problems in school.

You recall the analogy with a computer program.  Sure, you can understand what
a large computer program such as a word processor does, as a single high-level
concept or a small set of such concepts, but your ability to understand it in
no way ensures that the program is correct and doesn't have hidden bugs.  Even
if you wrote the program yourself you can't really know this; most large
programs that you've written have had bugs that crop up at some later date, no
matter how careful you tried to be while writing them.

OK, so now it seems the reason you can't figure out how to make a
computer verify proofs is because each step really corresponds to a
million small steps.  Well, you say, a computer can do a million
calculations in a second, so maybe it's still practical to do.  Now the
puzzle becomes how to figure out what the million steps are that each
English-language step corresponds to.  Your mathematician friend hasn't
a clue, but suggests that maybe you would find the answer by studying
set theory.  Actually, your friend thinks you're a little off the wall
for even wondering such a thing.  For him, this is not what mathematics
is all about.

The subject of set theory keeps popping up, so you decide it's
time to look it up.

You decide to start off on a careful footing, so you start reading a couple of
very elementary books on set theory.  A lot of it seems pretty obvious, like
intersections, subsets, and Venn diagrams.  You thumb through one of the
books; nowhere is anything about axioms mentioned. The other book relegates to
an appendix a brief discussion that mentions a set of axioms called
``Zermelo--Fraenkel set theory''\index{Zermelo--Fraenkel set theory} and states
them in English.  You look at them and have no idea what they really mean or
what you can do with them.  The comments in this appendix say that the purpose
of mentioning them is to expose you to the idea, but imply that they are not
necessary for basic understanding and that they are really the subject matter
of advanced treatments where fine points such as a certain paradox (Russell's
paradox\index{Russell's paradox}\footnote{Russell's paradox assumes that there
exists a set $S$ that is a collection of all sets that don't contain
themselves.  Now, either $S$ contains itself or it doesn't.  If it contains
itself, it contradicts its definition.  But if it doesn't contain itself, it
also contradicts its definition.  Russell's paradox is resolved in ZF set
theory by denying that such a set $S$ exists.}) are resolved.  Wait a
minute---shouldn't the axioms be a starting point, not an ending point?  If
there are paradoxes that arise without the axioms, how do you know you won't
stumble across one accidentally when using the informal approach?

And nowhere do these books describe how ``all of mathematics can be
derived from set theory'' which by now you've heard a few times.

You find a more advanced book on set theory.  This one actually lists the
axioms of ZF set theory in plain English on page one.  {\em Now} you think
your quest has ended and you've finally found the source of all mathematical
knowledge; you just have to understand what it means.  Here, in one place, is
the basis for all of mathematics!  You stare at the axioms in awe, puzzle over
them, memorize them, hoping that if you just meditate on them long enough they
will become clear.  Of course, you haven't the slightest idea how the rest of
mathematics is ``derived'' from them; in particular, if these are the axioms
of mathematics, then why do arithmetic, group theory, and so on need their own
axioms?

You start reading this advanced book carefully, pondering the meaning of every
word, because by now you're really determined to get to the bottom of this.
The first thing the book does is explain how the axioms came about, which was
to resolve Russell's paradox.\index{Russell's paradox}  In fact that seems to
be the main purpose of their existence; that they supposedly can be used to
derive all of mathematics seems irrelevant and is not even mentioned.  Well,
you go on.  You hope the book will explain to you clearly, step by step, how
to derive things from the axioms.  After all, this is the starting point of
mathematics, like a book that explains the basics of a computer programming
language.  But something is missing.  You find you can't even understand the
first proof or do the first exercise.  Symbols such as $\exists$ and $\forall$
permeate the page without any mention of where they came from or how to
manipulate them; the author assumes you are totally familiar with them and
doesn't even tell you what they mean.  By now you know that $\exists$ means
``there exists'' and $\forall$ means ``for all,'' but shouldn't the rules for
manipulating these symbols be part of the axioms?  You still have no idea
how you could even describe the axioms to a computer.

Certainly there is something much different here from the technical
literature you're used to reading.  A computer language manual almost
always explains very clearly what all the symbols mean, precisely what
they do, and the rules used for combining them, and you work your way up
from there.

After glancing at four or five other such books, you come to the realization
that there is another whole field of study that you need just to get to the
point at which you can understand the axioms of set theory.  The field is
called ``logic.''  In fact, some of the books did recommend it as a
prerequisite, but it just didn't sink in.  You assumed logic was, well, just
logic, something that a person with common sense intuitively understood.  Why
waste your time reading boring treatises on symbolic logic, the manipulation
of 1's and 0's that computers do, when you already know that?  But this is a
different kind of logic, quite alien to you.  The subject of {\sc nand} and
{\sc nor} gates is not even touched upon or in any case has to do with only a
very small part of this field.

So your quest continues.  Skimming through the first couple of introductory
books, you get a general idea of what logic is about and what quantifiers
(``for all,'' ``there exists'') mean, but you find their examples somewhat
trivial and mildly annoying (``all dogs are animals,'' ``some animals are
dogs,'' and such).  But all you want to know is what the rules are for
manipulating the symbols so you can apply them to set theory.  Some formulas
describing the relationships among quantifiers ($\exists$ and $\forall$) are
listed in tables, along with some verbal reasoning to justify them.
Presumably, if you want to find out if a formula is correct, you go through
this same kind of mental reasoning process, possibly using images of dogs and
animals. Intuitively, the formulas seem to make sense.  But when you ask
yourself, ``What are the rules I need to get a computer to figure out whether
this formula is correct?'', you still don't know.  Certainly you don't ask the
computer to imagine dogs and animals.

You look at some more advanced logic books.  Many of them have an introductory
chapter summarizing set theory, which turns out to be a prerequisite.  You
need logic to understand set theory, but it seems you also need set theory to
understand logic!  These books jump right into proving rather advanced
theorems about logic, without offering the faintest clue about where the logic
came from that allows them to prove these theorems.

Luckily, you come across an elementary book of logic that, halfway through,
after the usual truth tables and metaphors, presents in a clear, precise way
what you've been looking for all along: the axioms!  They're divided into
propositional calculus (also called sentential logic) and predicate calculus
(also called first-order logic),\index{first-order logic} with rules so simple
and crystal clear that now you can finally program a computer to understand
them.  Indeed, they're no harder than learning how to play a game of chess.
As far as what you seem to need is concerned, the whole book could have been
written in five pages!

{\em Now} you think you've found the ultimate source of mathematical
truth.  So---the axioms of mathematics consist of these axioms of logic,
together with the axioms of ZF set theory. (By now you've also been able to
figure out how to translate the ZF axioms from English into the
actual symbols of logic which you can now manipulate according to
precise, easy-to-understand rules.)

Of course, you still don't understand how ``all of mathematics can be
derived from set theory,'' but maybe this will reveal itself in due
course.

You eagerly set out to program the axioms and rules into a computer and start
to look at the theorems you will have to prove as the logic is developed.  All
sorts of important theorems start popping up:  the deduction
theorem,\index{deduction theorem} the substitution theorem,\index{substitution
theorem} the completeness theorem of propositional calculus,\index{first-order
logic!completeness} the completeness theorem of predicate calculus.  Uh-oh,
there seems to be trouble.  They all get harder and harder, and not one of
them can be derived with the axioms and rules of logic you've just been
handed.  Instead, they all require ``metalogic'' for their proofs, a kind of
mixture of logic and set theory that allows you to prove things {\em about}
the axioms and theorems of logic rather than {\em with} them.

You plow ahead anyway.  A month later, you've spent much of your
free time getting the computer to verify proofs in propositional calculus.
You've programmed in the axioms, but you've also had to program in the
deduction theorem, the substitution theorem, and the completeness theorem of
propositional calculus, which by now you've resigned yourself to treating as
rather complex additional axioms, since they can't be proved from the axioms
you were given.  You can now get the computer to verify and even generate
complete, rigorous, formal proofs\index{formal proof}.  Never mind that they
may have 100,000 steps---at least now you can have complete, absolute
confidence in them.  Unfortunately, the only theorems you have proved are
pretty trivial and you can easily verify them in a few minutes with truth
tables, if not by inspection.

It looks like your mathematician friend was right.  Getting the computer to do
serious mathematics with this kind of rigor seems almost hopeless.  Even
worse, it seems that the further along you get, the more ``axioms'' you have
to add, as each new theorem seems to involve additional ``metamathematical''
reasoning that hasn't been formalized, and none of it can be derived from the
axioms of logic.  Not only do the proofs keep growing exponentially as you get
further along, but the program to verify them keeps getting bigger and bigger
as you program in more ``metatheorems.''\index{metatheorem}\footnote{A
metatheorem is usually a statement that is too general to be directly provable
in a theory.  For example, ``if $n_1$, $n_2$, and $n_3$ are integers, then
$n_1+n_2+n_3$ is an integer'' is a theorem of number theory.  But ``for any
integer $k > 1$, if $n_1, \ldots, n_k$ are integers, then $n_1+\ldots +n_k$ is
an integer'' is a metatheorem, in other words a family of theorems, one for
every $k$.  The reason it is not a theorem is that the general sum $n_1+\ldots
+n_k$ (as a function of $k$) is not an operation that can be defined directly
in number theory.} The bugs\index{computer program bugs} that have cropped up
so far have already made you start to lose faith in the rigor you seem to have
achieved, and you know it's just going to get worse as your program gets larger.

\subsection{Mathematics and the Non-Specialist}

\begin{quote}
  {\em A real proof is not checkable by a machine, or even by any mathematician
not privy to the gestalt, the mode of thought of the particular field of
mathematics in which the proof is located.}
  \flushright\sc  Davis and Hersh\index{Davis, Phillip J.}
  \footnote{\cite{Davis}, p.~354.}\\
\end{quote}

The bulk of abstract or theoretical mathematics is ordinarily outside
the reach of anyone but a few specialists in each field who have completed
the necessary difficult internship in order to enter its coterie.  The
typical intelligent layperson has no reasonable hope of understanding much of
it, nor even the specialist mathematician of understanding other fields.  It
is like a foreign language that has no dictionary to look up the translation;
the only way you can learn it is by living in the country for a few years.  It
is argued that the effort involved in learning a specialty is a necessary
process for acquiring a deep understanding.  Of course, this is almost certainly
true if one is to make significant contributions to a field; in particular,
``doing'' proofs is probably the most important part of a mathematician's
training.  But is it also necessary to deny outsiders access to it?  Is it
necessary that abstract mathematics be so hard for a layperson to grasp?

A computer normally is of no help whatsoever.  Most published proofs are
actually just series of hints written in an informal style that requires
considerable knowledge of the field to understand.  These are the ``real
proofs'' referred to by Davis and Hersh.\index{informal proof}  There is an
implicit understanding that, in principle, such a proof could be converted to
a complete formal proof\index{formal proof}.  However, it is said that no one
would ever attempt such a conversion, even if they could, because that would
presumably require millions of steps (Section~\ref{dream}).  Unfortunately the
informal style automatically excludes the understanding of the proof
by anyone who hasn't gone through the necessary apprenticeship. The
best that the intelligent layperson can do is to read popular books about deep
and famous results; while this can be helpful, it can also be misleading, and
the lack of detail usually leaves the reader with no ability whatsoever to
explore any aspect of the field being described.

The statements of theorems often use sophisticated notation that makes them
inaccessible to the non-specialist.  For a non-specialist who wants to achieve
a deeper understanding of a proof, the process of tracing definitions and
lemmas back through their hierarchy\index{hierarchy} quickly becomes confusing
and discouraging.  Textbooks are usually written to train mathematicians or to
communicate to people who are already mathematicians, and large gaps in proofs
are often left as exercises to the reader who is left at an impasse if he or
she becomes stuck.

I believe that eventually computers will enable non-specialists and even
intelligent laypersons to follow almost any mathematical proof in any field.
Metamath is an attempt in that direction.  If all of mathematics were as
easily accessible as a computer programming language, I could envision
computer programmers and hobbyists who otherwise lack mathematical
sophistication exploring and being amazed by the world of theorems and proofs
in obscure specialties, perhaps even coming up with results of their own.  A
tremendous advantage would be that anyone could experiment with conjectures in
any field---the computer would offer instant feedback as to whether
an inference step was correct.

Mathematicians sometimes have to put up with the annoyance of
cranks\index{cranks} who lack a fundamental understanding of mathematics but
insist that their ``proofs'' of, say, Fermat's Last Theorem\index{Fermat's
Last Theorem} be taken seriously.  I think part of the problem is that these
people are misled by informal mathematical language, treating it as if they
were reading ordinary expository English and failing to appreciate the
implicit underlying rigor.  Such cranks are rare in the field of computers,
because computer languages are much more explicit, and ultimately the proof is
in whether a computer program works or not.  With easily accessible
computer-based abstract mathematics, a mathematician could say to a crank,
``don't bother me until you've demonstrated your claim on the computer!''

% 22-May-04 nm
% Attempt to move De Millo quote so it doesn't separate from attribution
% CHANGE THIS NUMBER (AND ELIMINATE IF POSSIBLE) WHEN ABOVE TEXT CHANGES
\vspace{-0.5em}

\subsection{An Impossible Dream?}\label{dream}

\begin{quote}
  {\em Even quite basic theorems would demand almost unbelievably vast
  books to display their proofs.}
    \flushright\sc  Robert E. Edwards\footnote{\cite{Edwards}, p.~68.}\\
\end{quote}\index{Edwards, Robert E.}

\begin{quote}
  {\em Oh, of course no one ever really {\em does} it.  It would take
  forever!  You just show that you could do it, that's sufficient.}
    \flushright\sc  ``The Ideal Mathematician''
    \index{Davis, Phillip J.}\footnote{\cite{Davis},
p.~40.}\\
\end{quote}

\begin{quote}
  {\em There is a theorem in the primitive notation of set theory that
  corresponds to the arithmetic theorem `$1000+2000=3000$'.  The formula
  would be forbiddingly long\ldots even if [one] knows the definitions
  and is asked to simplify the long formula according to them, chances are
  he will make errors and arrive at some incorrect result.}
    \flushright\sc  Hao Wang\footnote{\cite{Wang}, p.~140.}\\
\end{quote}\index{Wang, Hao}

% 22-May-04 nm
% Attempt to move De Millo quote so it doesn't separate from attribution
% CHANGE THIS NUMBER (AND ELIMINATE IF POSSIBLE) WHEN ABOVE TEXT CHANGES
\vspace{-0.5em}

\begin{quote}
  {\em The {\em Principia Mathematica} was the crowning achievement of the
  formalists.  It was also the deathblow of the formalist view.\ldots
  {[Rus\-sell]} failed, in three enormous volumes, to get beyond the elementary
  facts of arithmetic.  He showed what can be done in principle and what
  cannot be done in practice.  If the mathematical process were really
  one of strict, logical progression, we would still be counting our
  fingers.\ldots
  One theoretician estimates\ldots that a demonstration of one of
  Ramanujan's conjectures assuming set theory and elementary analysis would
  take about two thousand pages; the length of a deduction from first principles
  is nearly in\-con\-ceiv\-a\-ble\ldots The probabilists argue that\ldots any
  very long proof can at best be viewed as only probably correct\ldots}
  \flushright\sc Richard de Millo et. al.\footnote{\cite{deMillo}, pp.~269,
  271.}\\
\end{quote}\index{de Millo, Richard}

A number of writers have conveyed the impression that the kind of absolute
rigor provided by Metamath\index{Metamath} is an impossible dream, suggesting
that a complete, formal verification\index{formal proof} of a typical theorem
would take millions of steps in untold volumes of books.  Even if it could be
done, the thinking sometimes goes, all meaning would be lost in such a
monstrous, tedious verification.\index{informal proof}\index{proof length}

These writers assume, however, that in order to achieve the kind of complete
formal verification they desire one must break down a proof into individual
primitive steps that make direct reference to the axioms.  This is
not necessary.  There is no reason not to make use of previously proved
theorems rather than proving them over and over.

Just as important, definitions\index{definition} can be introduced along
the way, allowing very complex formulas to be represented with few
symbols.  Not doing this can lead to absurdly long formulas.  For
example, the mere statement of
G\"{o}del's incompleteness theorem\index{G\"{o}del's
incompleteness theorem}, which can be expressed with a small number of
defined symbols, would require about 20,000 primitive symbols to express
it.\index{Boolos, George S.}\footnote{George S.\ Boolos, lecture at
Massachusetts Institute of Technology, spring 1990.} An extreme example
is Bourbaki's\label{bourbaki} language for set theory, which requires
4,523,659,424,929 symbols plus 1,179,618,517,981 disambiguatory links
(lines connecting symbol pairs, usually drawn below or above the
formula) to express the number
``one'' \cite{Mathias}.\index{Mathias, Adrian R. D.}\index{Bourbaki,
Nicolas}
% http://www.dpmms.cam.ac.uk/~ardm/

A hierarchy\index{hierarchy} of theorems and definitions permits an
exponential growth in the formula sizes and primitive proof steps to be
described with only a linear growth in the number of symbols used.  Of course,
this is how ordinary informal mathematics is normally done anyway, but with
Metamath\index{Metamath} it can be done with absolute rigor and precision.

\subsection{Beauty}


\begin{quote}
  {\em No one shall be able to drive us from the paradise that Cantor has
created for us.}
   \flushright\sc  David Hilbert\footnote{As quoted in \cite{Moore}, p.~131.}\\
\end{quote}\index{Hilbert, David}

\needspace{3\baselineskip}
\begin{quote}
  {\em Mathematics possesses not only truth, but some supreme beauty ---a
  beauty cold and austere, like that of a sculpture.}
    \flushright\sc  Bertrand
    Russell\footnote{\cite{Russell}.}\\
\end{quote}\index{Russell, Bertrand}

\begin{quote}
  {\em Euclid alone has looked on Beauty bare.}
  \flushright\sc Edna Millay\footnote{As quoted in \cite{Davis}, p.~150.}\\
\end{quote}\index{Millay, Edna}

For most people, abstract mathematics is distant, strange, and
incomprehensible.  Many popular books have tried to convey some of the sense
of beauty in famous theorems.  But even an intelligent layperson is left with
only a general idea of what a theorem is about and is hardly given the tools
needed to make use of it.  Traditionally, it is only after years of arduous
study that one can grasp the concepts needed for deep understanding.
Metamath\index{Metamath} allows you to approach the proof of the theorem from
a quite different perspective, peeling apart the formulas and definitions
layer by layer until an entirely different kind of understanding is achieved.
Every step of the proof is there, pieced together with absolute precision and
instantly available for inspection through a microscope with a magnification
as powerful as you desire.

A proof in itself can be considered an object of beauty.  Constructing an
elegant proof is an art.  Once a famous theorem has been proved, often
considerable effort is made to find simpler and more easily understood
proofs.  Creating and communicating elegant proofs is a major concern of
mathematicians.  Metamath is one way of providing a common language for
archiving and preserving this information.

The length of a proof can, to a certain extent, be considered an
objective measure of its ``beauty,'' since shorter proofs are usually
considered more elegant.  In the set theory database
\texttt{set.mm}\index{set theory database (\texttt{set.mm})}%
\index{Metamath Proof Explorer}
provided with Metamath, one goal was to make all proofs as short as possible.

\needspace{4\baselineskip}
\subsection{Simplicity}

\begin{quote}
  {\em God made man simple; man's complex problems are of his own
  devising.}
    \flushright\sc Eccles. 7:29\footnote{Jerusalem Bible.}\\
\end{quote}\index{Bible}

\needspace{3\baselineskip}
\begin{quote}
  {\em God made integers, all else is the work of man.}
    \flushright\sc Leopold Kronecker\footnote{{\em Jahresbericht
	der Deutschen Mathematiker-Vereinigung }, vol. 2, p. 19.}\\
\end{quote}\index{Kronecker, Leopold}

\needspace{3\baselineskip}
\begin{quote}
  {\em For what is clear and easily comprehended attracts; the
  complicated repels.}
    \flushright\sc David Hilbert\footnote{As quoted in \cite{deMillo},
p.~273.}\\
\end{quote}\index{Hilbert, David}

The Metamath\index{Metamath} language is simple and Spartan.  Metamath treats
all mathematical expressions as simple sequences of symbols, devoid of meaning.
The higher-level or ``metamathematical'' notions underlying Metamath are about
as simple as they could possibly be.  Each individual step in a proof involves
a single basic concept, the substitution of an expression for a variable, so
that in principle almost anyone, whether mathematician or not, can
completely understand how it was arrived at.

In one of its most basic applications, Metamath\index{Metamath} can be used to
develop the foundations of mathematics\index{foundations of mathematics} from
the very beginning.  This is done in the set theory database that is provided
with the Metamath package and is the subject matter
of Chapter~\ref{fol}. Any language (a metalanguage\index{metalanguage})
used to describe mathematics (an object language\index{object language}) must
have a mathematical content of its own, but it is desirable to keep this
content down to a bare minimum, namely that needed to make use of the
inference rules specified by the axioms.  With any metalanguage there is a
``chicken and egg'' problem somewhat like circular reasoning:  you must assume
the validity of the mathematics of the metalanguage in order to prove the
validity of the mathematics of the object language.  The mathematical content
of Metamath itself is quite limited.  Like the rules of a game of chess, the
essential concepts are simple enough so that virtually anyone should be able to
understand them (although that in itself will not let you play like
a master).  The symbols that Metamath manipulates do not in themselves
have any intrinsic meaning.  Your interpretation of the axioms that you supply
to Metamath is what gives them meaning.  Metamath is an attempt to strip down
mathematical thought to its bare essence and show you exactly how the symbols
are manipulated.

Philosophers and logicians, with various motivations, have often thought it
important to study ``weak'' fragments of logic\index{weak logic}
\cite{Anderson}\index{Anderson, Alan Ross} \cite{MegillBunder}\index{Megill,
Norman}\index{Bunder, Martin}, other unconventional systems of logic (such as
``modal'' logic\index{modal logic} \cite[ch.\ 27]{Boolos}\index{Boolos, George
S.}), and quantum logic\index{quantum logic} in physics
\cite{Pavicic}\index{Pavi{\v{c}}i{\'{c}}, M.}.  Metamath\index{Metamath}
provides a framework in which such systems can be expressed, with an absolute
precision that makes all underlying metamathematical assumptions rigorous and
crystal clear.

Some schools of philosophical thought, for example
intuitionism\index{intuitionism} and constructivism\index{constructivism},
demand that the notions underlying any mathematical system be as simple and
concrete as possible.  Metamath should meet the requirements of these
philosophies.  Metamath must be taught the symbols, axioms\index{axiom}, and
rules\index{rule} for a specific theory, from the skeptical (such as
intuitionism\index{intuitionism}\footnote{Intuitionism does not accept the law
of excluded middle (``either something is true or it is not true'').  See
\cite[p.~xi]{Tymoczko}\index{Tymoczko, Thomas} for discussion and references
on this topic.  Consider the theorem, ``There exist irrational numbers $a$ and
$b$ such that $a^b$ is rational.''  An intuitionist would reject the following
proof:  If $\sqrt{2}^{\sqrt{2}}$ is rational, we are done.  Otherwise, let
$a=\sqrt{2}^{\sqrt{2}}$ and $b=\sqrt{2}$. Then $a^b=2$, which is rational.})
to the bold (such as the axiom of choice in set theory\footnote{The axiom of
choice\index{Axiom of Choice} asserts that given any collection of pairwise
disjoint nonempty sets, there exists a set that has exactly one element in
common with each set of the collection.  It is used to prove many important
theorems in standard mathematics.  Some philosophers object to it because it
asserts the existence of a set without specifying what the set contains
\cite[p.~154]{Enderton}\index{Enderton, Herbert B.}.  In one foundation for
mathematics due to Quine\index{Quine, Willard Van Orman}, that has not been
otherwise shown to be inconsistent, the axiom of choice turns out to be false
\cite[p.~23]{Curry}\index{Curry, Haskell B.}.  The \texttt{show
trace{\char`\_}back} command of the Metamath program allows you to find out
whether the axiom of choice, or any other axiom, was assumed by a
proof.}\index{\texttt{show trace{\char`\_}back} command}).

The simplicity of the Metamath language lets the algorithm (computer program)
that verifies the validity of a Metamath proof be straightforward and
robust.  You can have confidence that the theorems it verifies really can be
derived from your axioms.

\subsection{Rigor}

\begin{quote}
  {\em Rigor became a goal with the Greeks\ldots But the efforts to
  pursue rigor to the utmost have led to an impasse in which there is
  no longer any agreement on what it really means.  Mathematics remains
  alive and vital, but only on a pragmatic basis.}
    \flushright\sc  Morris Kline\footnote{\cite{Kline}, p.~1209.}\\
\end{quote}\index{Kline, Morris}

Kline refers to a much deeper kind of rigor than that which we will discuss in
this section.  G\"{o}del's incompleteness theorem\index{G\"{o}del's
incompleteness theorem} showed that it is impossible to achieve absolute rigor
in standard mathematics because we can never prove that mathematics is
consistent (free from contradictions).\index{consistent theory}  If
mathematics is consistent, we will never know it, but must rely on faith.  If
mathematics is inconsistent, the best we can hope for is that some clever
future mathematician will discover the inconsistency.  In this case, the
axioms would probably be revised slightly to eliminate the inconsistency, as
was done in the case of Russell's paradox,\index{Russell's paradox} but the
bulk of mathematics would probably not be affected by such a discovery.
Russell's paradox, for example, did not affect most of the remarkable results
achieved by 19th-century and earlier mathematicians.  It mainly invalidated
some of Gottlob Frege's\index{Frege, Gottlob} work on the foundations of
mathematics in the late 1800's; in fact Frege's work inspired Russell's
discovery.  Despite the paradox, Frege's work contains important concepts that
have significantly influenced modern logic.  Kline's {\em Mathematics, The
Loss of Certainty} \cite{Klinel}\index{Kline, Morris} has an interesting
discussion of this topic.

What {\em can} be achieved with absolute certainty\index{certainty} is the
knowledge that if we assume the axioms are consistent and true, then the
results derived from them are true.  Part of the beauty of mathematics is that
it is the one area of human endeavor where absolute certainty can be achieved
in this sense.  A mathematical truth will remain such for eternity.  However,
our actual knowledge of whether a particular statement is a mathematical truth
is only as certain as the correctness of the proof that establishes it.  If
the proof of a statement is questionable or vague, we can't have absolute
confidence in the truth that the statement claims.

Let us look at some traditional ways of expressing proofs.

Except in the field of formal logic\index{formal logic}, almost all
traditional proofs in mathematics are really not proofs at all, but rather
proof outlines or hints as to how to go about constructing the proof.  Many
gaps\index{gaps in proofs} are left for the reader to fill in. There are
several reasons for this.  First, it is usually assumed in mathematical
literature that the person reading the proof is a mathematician familiar with
the specialty being described, and that the missing steps are obvious to such
a reader or at least that the reader is capable of filling them in.  This
attitude is fine for professional mathematicians in the specialty, but
unfortunately it often has the drawback of cutting off the rest of the world,
including mathematicians in other specialties, from understanding the proof.
We discussed one possible resolution to this on p.~\pageref{envision}.
Second, it is often assumed that a complete formal proof\index{formal proof}
would require countless millions of symbols (Section~\ref{dream}). This might
be true if the proof were to be expressed directly in terms of the axioms of
logic and set theory,\index{set theory} but it is usually not true if we allow
ourselves a hierarchy\index{hierarchy} of definitions and theorems to build
upon, using a notation that allows us to introduce new symbols, definitions,
and theorems in a precisely specified way.

Even in formal logic,\index{formal logic} formal proofs\index{formal proof}
that are considered complete still contain hidden or implicit information.
For example, a ``proof'' is usually defined as a sequence of
wffs,\index{well-formed formula (wff)}\footnote{A {\em wff} or well-formed
formula is a mathematical expression (string of symbols) constructed according
to some precise rules.  A formal mathematical system\index{formal system}
contains (1) the rules for constructing syntactically correct
wffs,\index{syntax rules} (2) a list of starting wffs called
axioms,\index{axiom} and (3) one or more rules prescribing how to derive new
wffs, called theorems\index{theorem}, from the axioms or previously derived
theorems.  An example of such a system is contained in
Metamath's\index{Metamath} set theory database, which defines a formal
system\index{formal system} from which all of standard mathematics can be
derived.  Section~\ref{startf} steps you through a complete example of a formal
system, and you may want to skim it now if you are unfamiliar with the
concept.} each of which is an axiom or follows from a rule applied to previous
wffs in the sequence.  The implicit part of the proof is the algorithm by
which a sequence of symbols is verified to be a valid wff, given the
definition of a wff.  The algorithm in this case is rather simple, but for a
computer to verify the proof,\index{automated proof verification} it must have
the algorithm built into its verification program.\footnote{It is possible, of
course, to specify wff construction syntax outside of the program itself
with a suitable input language (the Metamath language being an example), but
some proof-verification or theorem-proving programs lack the ability to extend
wff syntax in such a fashion.} If one deals exclusively with axioms and
elementary wffs, it is straightforward to implement such an algorithm.  But as
more and more definitions are added to the theory in order to make the
expression of wffs more compact, the algorithm becomes more and more
complicated.  A computer program that implements the algorithm becomes larger
and harder to understand as each definition is introduced, and thus more prone
to bugs.\index{computer program bugs}  The larger the program, the
more suspicious the mathematician may be about
the validity of its algorithms.  This is especially true because
computer programs are inherently hard to follow to begin with, and few people
enjoy verifying them manually in detail.

Metamath\index{Metamath} takes a different approach.  Metamath's ``knowledge''
is limited to the ability to substitute variables for expressions, subject to
some simple constraints.  Once the basic algorithm of Metamath is assumed to
be debugged, and perhaps independently confirmed, it
can be trusted once and for all.  The information that Metamath needs to
``understand'' mathematics is contained entirely in the body of knowledge
presented to Metamath.  Any errors in reasoning can only be errors in the
axioms or definitions contained in this body of knowledge.  As a
``constructive'' language\index{constructive language} Metamath has no
conditional branches or loops like the ones that make computer programs hard
to decipher; instead, the language can only build new sequences of symbols
from earlier sequences  of symbols.

The simplicity of the rules that underlie Metamath not only makes Metamath
easy to learn but also gives Metamath a great deal of flexibility. For
example, Metamath is not limited to describing standard first-order
logic\index{first-order logic}; higher-order logics\index{higher-order logic}
and fragments of logic\index{weak logic} can be described just as easily.
Metamath gives you the freedom to define whatever wff notation you prefer; it
has no built-in conception of the syntax of a wff.\index{well-formed formula
(wff)}  With suitable axioms and definitions, Metamath can even describe and
prove things about itself.\index{Metamath!self-description}  (John
Harrison\index{Harrison, John} discusses the ``reflection''
principle\index{reflection principle} involved in self-descriptive systems in
\cite{Harrison}.)

The flexibility of Metamath requires that its proofs specify a lot of detail,
much more than in an ordinary ``formal'' proof.\index{formal proof}  For
example, in an ordinary formal proof, a single step consists of displaying the
wff that constitutes that step.  In order for a computer program to verify
that the step is acceptable, it first must verify that the symbol sequence
being displayed is an acceptable wff.\index{automated proof verification} Most
proof verifiers have at least basic wff syntax built into their programs.
Metamath has no hard-wired knowledge of what constitutes a wff built into it;
instead every wff must be explicitly constructed based on rules defining wffs
that are present in a database.  Thus a single step in an ordinary formal
proof may be correspond to many steps in a Metamath proof. Despite the larger
number of steps, though, this does not mean that a Metamath proof must be
significantly larger than an ordinary formal proof. The reason is that since
we have constructed the wff from scratch, we know what the wff is, so there is
no reason to display it.  We only need to refer to a sequence of statements
that construct it.  In a sense, the display of the wff in an ordinary formal
proof is an implicit proof of its own validity as a wff; Metamath just makes
the proof explicit. (Section~\ref{proof} describes Metamath's proof notation.)

\section{Computers and Mathematicians}

\begin{quote}
  {\em The computer is important, but not to mathematics.}
    \flushright\sc  Paul Halmos\footnote{As quoted in \cite{Albers}, p.~121.}\\
\end{quote}\index{Halmos, Paul}

Pure mathematicians have traditionally been indifferent to computers, even to
the point of disdain.\index{computers and pure mathematics}  Computer science
itself is sometimes considered to fall in the mundane realm of ``applied''
mathematics, perhaps essential for the real world but intellectually unexciting
to those who seek the deepest truths in mathematics.  Perhaps a reason for this
attitude towards computers is that there is little or no computer software that
meets their needs, and there may be a general feeling that such software could
not even exist.  On the one hand, there are the practical computer algebra
systems, which can perform amazing symbolic manipulations in algebra and
calculus,\index{computer algebra system} yet can't prove the simplest
existence theorem, if the idea of a proof is present at all.  On the other
hand, there are specialized automated theorem provers that technically speaking
may generate correct proofs.\index{automated theorem proving}  But sometimes
their specialized input notation may be cryptic and their output perceived to
be long, inelegant, incomprehensible proofs.    The output
may be viewed with suspicion, since the program that generates it tends to be
very large, and its size increases the potential for bugs\index{computer
program bugs}.  Such a proof may be considered trustworthy only if
independently verified and ``understood'' by a human, but no one wants to
waste their time on such a boring, unrewarding chore.



\needspace{4\baselineskip}
\subsection{Trusting the Computer}

\begin{quote}
  {\em \ldots I continue to find the quasi-empirical interpretation of
  computer proofs to be the more plausible.\ldots Since not
  everything that claims to be a computer proof can be
  accepted as valid, what are the mathematical criteria for acceptable
  computer proofs?}
    \flushright\sc  Thomas Tymoczko\footnote{\cite{Tymoczko}, p.~245.}\\
\end{quote}\index{Tymoczko, Thomas}

In some cases, computers have been essential tools for proving famous
theorems.  But if a proof is so long and obscure that it can be verified in a
practical way only with a computer, it is vaguely felt to be suspicious.  For
example, proving the famous four-color theorem\index{four-color
theorem}\index{proof length} (``a map needs no more than four colors to
prevent any two adjacent countries from having the same color'') can presently
only be done with the aid of a very complex computer program which originally
required 1200 hours of computer time. There has been considerable debate about
whether such a proof can be trusted and whether such a proof is ``real''
mathematics \cite{Swart}\index{Swart, E. R.}.\index{trusting computers}

However, under normal circumstances even a skeptical mathematician would have a
great deal of confidence in the result of multiplying two numbers on a pocket
calculator, even though the precise details of what goes on are hidden from its
user.  Even the verification on a supercomputer that a huge number is prime is
trusted, especially if there is independent verification; no one bothers to
debate the philosophical significance of its ``proof,'' even though the actual
proof would be so large that it would be completely impractical to ever write
it down on paper.  It seems that if the algorithm used by the computer is
simple enough to be readily understood, then the computer can be trusted.

Metamath\index{Metamath} adopts this philosophy.  The simplicity of its
language makes it easy to learn, and because of its simplicity one can have
essentially absolute confidence that a proof is correct. All axioms, rules, and
definitions are available for inspection at any time because they are defined
by the user; there are no hidden or built-in rules that may be prone to subtle
bugs\index{computer program bugs}.  The basic algorithm at the heart of
Metamath is simple and fixed, and it can be assumed to be bug-free and robust
with a degree of confidence approaching certainty.
Independently written implementations of the Metamath verifier
can reduce any residual doubt on the part of a skeptic even further;
there are now over a dozen such implementations, written by many people.

\subsection{Trusting the Mathematician}\label{trust}

\begin{quote}
  {\em There is no Algebraist nor Mathematician so expert in his science, as
  to place entire confidence in any truth immediately upon his discovery of it,
  or regard it as any thing, but a mere probability.  Every time he runs over
  his proofs, his confidence encreases; but still more by the approbation of
  his friends; and is rais'd to its utmost perfection by the universal assent
  and applauses of the learned world.}
  \flushright\sc David Hume\footnote{{\em A Treatise of Human Nature}, as
  quoted in \cite{deMillo}, p.~267.}\\
\end{quote}\index{Hume, David}

\begin{quote}
  {\em Stanislaw Ulam estimates that mathematicians publish 200,000 theorems
  every year.  A number of these are subsequently contradicted or otherwise
  disallowed, others are thrown into doubt, and most are ignored.}
  \flushright\sc Richard de Millo et. al.\footnote{\cite{deMillo}, p.~269.}\\
\end{quote}\index{Ulam, Stanislaw}

Whether or not the computer can be trusted, humans  of course will occasionally
err. Only the most memorable proofs get independently verified, and of these
only a handful of truly great ones achieve the status of being ``known''
mathematical truths that are used without giving a second thought to their
correctness.

There are many famous examples of incorrect theorems and proofs in
mathematical literature.\index{errors in proofs}

\begin{itemize}
\item There have been thousands of purported proofs of Fermat's Last
Theorem\index{Fermat's Last Theorem} (``no integer solutions exist to $x^n +
y^n = z^n$ for $n > 2$''), by amateurs, cranks, and well-regarded
mathematicians \cite[p.~5]{Stark}\index{Stark, Harold M}.  Fermat wrote a note
in his copy of Bachet's {\em Diophantus} that he found ``a truly marvelous
proof of this theorem but this margin is too narrow to contain it''
\cite[p.~507]{Kramer}.  A recent, much publicized proof by Yoichi
Miyaoka\index{Miyaoka, Yoichi} was shown to be incorrect ({\em Science News},
April 9, 1988, p.~230).  The theorem was finally proved by Andrew
Wiles\index{Wiles, Andrew} ({\em Science News}, July 3, 1993, p.~5), but it
initially had some gaps and took over a year after its announcement to be
checked thoroughly by experts.  On Oct. 25, 1994, Wiles announced that the last
gap found in his proof had been filled in.
  \item In 1882, M. Pasch discovered that an axiom was omitted from Euclid's
formulation of geometry\index{Euclidean geometry}; without it, the proofs of
important theorems of Euclid are not valid.  Pasch's axiom\index{Pasch's
axiom} states that a line that intersects one side of a triangle must also
intersect another side, provided that it does not touch any of the triangle's
vertices.  The omission of Pasch's axiom went unnoticed for 2000
years \cite[p.~160]{Davis}, in spite of (one presumes) the thousands of
students, instructors, and mathematicians who studied Euclid.
  \item The first published proof of the famous Schr\"{o}der--Bernstein
theorem\index{Schr\"{o}der--Bernstein theorem} in set theory was incorrect
\cite[p.~148]{Enderton}\index{Enderton, Herbert B.}.  This theorem states
that if there exists a 1-to-1 function\footnote{A {\em set}\index{set} is any
collection of objects. A {\em function}\index{function} or {\em
mapping}\index{mapping} is a rule that assigns to each element of one set
(called the function's {\em domain}\index{domain}) an element from another
set.} from set $A$ into set $B$ and vice-versa, then sets $A$ and $B$ have
a 1-to-1 correspondence.  Although it sounds simple and obvious,
the standard proof is quite long and complex.
  \item In the early 1900's, Hilbert\index{Hilbert, David} published a
purported proof of the continuum hypothesis\index{continuum hypothesis}, which
was eventually established as unprovable by Cohen\index{Cohen, Paul} in 1963
\cite[p.~166]{Enderton}.  The continuum hypothesis states that no
infinity\index{infinity} (``transfinite cardinal number'')\index{cardinal,
transfinite} exists whose size (or ``cardinality''\index{cardinality}) is
between the size of the set of integers and the size of the set of real
numbers.  This hypothesis originated with German mathematician Georg
Cantor\index{Cantor, Georg} in the late 1800's, and his inability to prove it
is said to have contributed to mental illness that afflicted him in his later
years.
  \item An incorrect proof of the four-color theorem\index{four-color theorem}
was published by Kempe\index{Kempe, A. B.} in 1879
\cite[p.~582]{Courant}\index{Courant, Richard}; it stood for 11 years before
its flaw was discovered.  This theorem states that any map can be colored
using only four colors, so that no two adjacent countries have the same
color.  In 1976 the theorem was finally proved by the famous computer-assisted
proof of Haken, Appel, and Koch \cite{Swart}\index{Appel, K.}\index{Haken,
W.}\index{Koch, K.}.  Or at least it seems that way.  Mathematician
H.~S.~M.~Coxeter\index{Coxeter, H. S. M.} has doubts \cite[p.~58]{Davis}:  ``I
have a feeling that that is an untidy kind of use of the computers, and the more
you correspond with Haken and Appel, the more shaky you seem to be.''
  \item Many false ``proofs'' of the Poincar\'{e}
conjecture\index{Poincar\'{e} conjecture} have been proposed over the years.
This conjecture states that any object that mathematically behaves like a
three-dimensional sphere is a three-dimensional sphere topologically,
regardless of how it is distorted.  In March 1986, mathematicians Colin
Rourke\index{Rourke, Colin} and Eduardo R\^{e}go\index{R\^{e}go, Eduardo}
caused  a stir in the mathematical community by announcing that they had found
a proof; in November of that year the proof was found to be false \cite[p.
218]{PetersonI}.  It was finally proved in 2003 by Grigory Perelman
\label{poincare}\index{Szpiro, George}\index{Perelman, Grigory}\cite{Szpiro}.
 \end{itemize}

Many counterexamples to ``theorems'' in recent mathematical
literature related to Clifford algebras\index{Clifford algebras}
 have been found by Pertti
Lounesto (who passed away in 2002).\index{Lounesto, Pertti}
See the web page \url{http://mathforum.org/library/view/4933.html}.
% http://users.tkk.fi/~ppuska/mirror/Lounesto/counterexamples.htm

One of the purposes of Metamath\index{Metamath} is to allow proofs to be
expressed with absolute precision.  Developing a proof in the Metamath
language can be challenging, because Metamath will not permit even the
tiniest mistake.\index{errors in proofs}  But once the proof is created, its
correctness can be trusted immediately, without having to depend on the
process of peer review for confirmation.

\section{The Use of Computers in Mathematics}

\subsection{Computer Algebra Systems}

For the most part, you will find that Metamath\index{Metamath} is not a
practical tool for manipulating numbers.  (Even proving that $2 + 2 = 4$, if
you start with set theory, can be quite complex!)  Several commercial
mathematics packages are quite good at arithmetic, algebra, and calculus, and
as practical tools they are invaluable.\index{computer algebra system} But
they have no notion of proof, and cannot understand statements starting with
``there exists such and such...''.

Software packages such as Mathematica \cite{Wolfram}\index{Mathematica} do not
concern themselves with proofs but instead work directly with known results.
These packages primarily emphasize heuristic rules such as the substitution of
equals for equals to achieve simpler expressions or expressions in a different
form.  Starting with a rich collection of built-in rules and algorithms, users
can add to the collection by means of a powerful programming language.
However, results such as, say, the existence of a certain abstract object
without displaying the actual object cannot be expressed (directly) in their
languages.  The idea of a proof from a small set of axioms is absent.  Instead
this software simply assumes that each fact or rule you add to the built-in
collection of algorithms is valid.  One way to view the software is as a large
collection of axioms from which the software, with certain goals, attempts to
derive new theorems, for example equating a complex expression with a simpler
equivalent. But the terms ``theorem''\index{theorem} and
``proof,''\index{proof} for example, are not even mentioned in the index of
the user's manual for Mathematica.\index{Mathematica and proofs}  What is also
unsatisfactory from a philosophical point of view is that there is no way to
ensure the validity of the results other than by trusting the writer of each
application module or tediously checking each module by hand, similar to
checking a computer program for bugs.\index{computer program
bugs}\footnote{Two examples illustrate why the knowledge database of computer
algebra systems should sometimes be regarded with a certain caution.  If you
ask Mathematica (version 3.0) to \texttt{Solve[x\^{ }n + y\^{ }n == z\^{ }n , n]}
it will respond with \texttt{\{\{n-\char`\>-2\}, \{n-\char`\>-1\},
\{n-\char`\>1\}, \{n-\char`\>2\}\}}. In other words, Mathematica seems to
``know'' that Fermat's Last Theorem\index{Fermat's Last Theorem} is true!  (At
the time this version of Mathematica was released this fact was unknown.)  If
you ask Maple\index{Maple} to \texttt{solve(x\^{ }2 = 2\^{ }x)} then
\texttt{simplify(\{"\})}, it returns the solution set \texttt{\{2, 4\}}, apparently
unaware that 0.7666647\ldots is also a solution.} While of course extremely
valuable in applied mathematics, computer algebra systems tend to be of little
interest to the theoretical mathematician except as aids for exploring certain
specific problems.

Because of possible bugs, trusting the output of a computer algebra system for
use as theorems in a proof-verifier would defeat the latter's goal of rigor.
On the other hand, a fact such that a certain relatively large number is
prime, while easy for a computer algebra system to derive, might have a long,
tedious proof that could overwhelm a proof-verifier. One approach for linking
computer algebra systems to a proof-verifier while retaining the advantages of
both is to add a hypothesis to each such theorem indicating its source.  For
example, a constant {\sc maple} could indicate the theorem came from the Maple
package, and would mean ``assuming Maple is consistent, then\ldots''  This and
many other topics concerning the formalization of mathematics are discussed in
John Harrison's\index{Harrison, John} very interesting
PhD thesis~\cite{Harrison-thesis}.

\subsection{Automated Theorem Provers}\label{theoremprovers}

A mathematical theory is ``decidable''\index{decidable theory} if a mechanical
method or algorithm exists that is guaranteed to determine whether or not a
particular formula is a theorem.  Among the few theories that are decidable is
elementary geometry,\index{Euclidean geometry} as was shown by a classic
result of logician Alfred Tarski\index{Tarski, Alfred} in 1948
\cite{Tarski}.\footnote{Tarski's result actually applies to a subset of the
geometry discussed in elementary textbooks.  This subset includes most of what
would be considered elementary geometry but it is not powerful enough to
express, among other things, the notions of the circumference and area of a
circle.  Extending the theory in a way that includes notions such as these
makes the theory undecidable, as was also shown by Tarski.  Tarski's algorithm
is far too inefficient to implement practically on a computer.  A practical
algorithm for proving a smaller subset of geometry theorems (those not
involving concepts of ``order'' or ``continuity'') was discovered by Chinese
mathematician Wu Wen-ts\"{u}n in 1977 \cite{Chou}\index{Chou,
Shang-Ching}.}\index{Wen-ts{\"{u}}n, Wu}  But most theories, including
elementary arithmetic, are undecidable.  This fact contributes to keeping
mathematics alive and well, since many mathematicians believe
that they will never be
replaced by computers (if they believe Roger Penrose's argument that a
computer can never replace the brain \cite{Penrose}\index{Penrose, Roger}).
In fact,  elementary geometry is often considered a ``dead'' field
for the simple reason that it is decidable.

On the other hand, the undecidability of a theory does not mean that one cannot
use a computer to search for proofs, providing one is willing to give up if a
proof is not found after a reasonable amount of time.  The field of automated
theorem proving\index{automated theorem proving} specializes in pursuing such
computer searches.  Among the more successful results to date are those based
on an algorithm known as Robinson's resolution principle
\cite{Robinson}\index{Robinson's resolution principle}.

Automated theorem provers can be excellent tools for those willing to learn
how to use them.  But they are not widely used in mainstream pure
mathematics, even though they could probably be useful in many areas.  There
are several reasons for this.  Probably most important, the main goal in pure
mathematics is to arrive at results that are considered to be deep or
important; proving them is essential but secondary.  Usually, an automated
theorem prover cannot assist in this main goal, and by the time the main goal
is achieved, the mathematician may have already figured out the proof as a
by-product.  There is also a notational problem.  Mathematicians are used to
using very compact syntax where one or two symbols (heavily dependent on
context) can represent very complex concepts; this is part of the
hierarchy\index{hierarchy} they have built up to tackle difficult problems.  A
theorem prover on the other hand might require that a theorem be expressed in
``first-order logic,''\index{first-order logic} which is the logic on which
most of mathematics is ultimately based but which is not ordinarily used
directly because expressions can become very long.  Some automated theorem
provers are experimental programs, limited in their use to very specialized
areas, and the goal of many is simply research into the nature of automated
theorem proving itself.  Finally, much research remains to be done to enable
them to prove very deep theorems.  One significant result was a
computer proof by Larry Wos\index{Wos, Larry} and colleagues that every Robbins
algebra\index{Robbins algebra} is a Boolean  algebra\index{Boolean algebra}
({\em New York Times}, Dec. 10, 1996).\footnote{In 1933, E.~V.\
Huntington\index{Huntington, E. V.}
presented the following axiom system for
Boolean algebra with a unary operation $n$ and a binary operation $+$:
\begin{center}
    $x + y = y + x$ \\
    $(x + y) + z = x + (y + z)$ \\
    $n(n(x) + y) + n(n(x) + n(y)) = x$
\end{center}
Herbert Robbins\index{Robbins, Herbert}, a student of Huntington, conjectured
that the last equation can be replaced with a simpler one:
\begin{center}
    $n(n(x + y) + n(x + n(y))) = x$
\end{center}
Robbins and Huntington could not find a proof.  The problem was
later studied unsuccessfully by Tarski\index{Tarski, Alfred} and his
students, and it remained an unsolved problem until a
computer found the proof in 1996.  For more information on
the Robbins algebra problem see \cite{Wos}.}

How does Metamath\index{Metamath} relate to automated theorem provers?  A
theorem prover is primarily concerned with one theorem at a time (perhaps
tapping into a small database of known theorems) whereas Metamath is more like
a theorem archiving system, storing both the theorem and its proof in a
database for access and verification.  Metamath is one answer to ``what do you
do with the output of a theorem prover?''  and could be viewed as the
next step in the process.  Automated theorem provers could be useful tools for
helping develop its database.
Note that very long, automatically
generated proofs can make your database fat and ugly and cause Metamath's proof
verification to take a long time to run.  Unless you have a particularly good
program that generates very concise proofs, it might be best to consider the
use of automatically generated proofs as a quick-and-dirty approach, to be
manually rewritten at some later date.

The program {\sc otter}\index{otter@{\sc otter}}\footnote{\url{http://www.cs.unm.edu/\~mccune/otter/}.}, later succeeded by
prover9\index{prover9}\footnote{\url{https://www.cs.unm.edu/~mccune/mace4/}.},
have been historically influential.
The E prover\index{E prover}\footnote{\url{https://github.com/eprover/eprover}.}
is a maintained automated theorem prover
for full first-order logic with equality.
There are many other automated theorem provers as well.

If you want to combine automated theorem provers with Metamath
consider investigating
the book {\em Automated Reasoning:  Introduction and Applications}
\cite{Wos}\index{Wos, Larry}.  This book discusses
how to use {\sc otter} in a way that can
not only able to generate
relatively efficient proofs, it can even be instructed to search for
shorter proofs.  The effective use of {\sc otter} (and similar tools)
does require a certain
amount of experience, skill, and patience.  The axiom system used in the
\texttt{set.mm}\index{set theory database (\texttt{set.mm})} set theory
database can be expressed to {\sc otter} using a method described in
\cite{Megill}.\index{Megill, Norman}\footnote{To use those axioms with
{\sc otter}, they must be restated in a way that eliminates the need for
``dummy variables.''\index{dummy variable!eliminating} See the Comment
on p.~\pageref{nodd}.} When successful, this method tends to generate
short and clever proofs, but my experiments with it indicate that the
method will find proofs within a reasonable time only for relatively
easy theorems.  It is still fun to experiment with.

Reference \cite{Bledsoe}\index{Bledsoe, W. W.} surveys a number of approaches
people have explored in the field of automated theorem proving\index{automated
theorem proving}.

\subsection{Interactive Theorem Provers}\label{interactivetheoremprovers}

Finding proofs completely automatically is difficult, so there
are some interactive theorem provers that allow a human to guide the
computer to find a proof.
Examples include
HOL Light\index{HOL light}%
\footnote{\url{https://www.cl.cam.ac.uk/~jrh13/hol-light/}.},
Isabelle\index{Isabelle}%
\footnote{\url{http://www.cl.cam.ac.uk/Research/HVG/Isabelle}.},
{\sc hol}\index{hol@{\sc hol}}%
\footnote{\url{https://hol-theorem-prover.org/}.},
and
Coq\index{Coq}\footnote{\url{https://coq.inria.fr/}.}.

A major difference between most of these tools and Metamath is that the
``proofs'' are actually programs that guide the program to find a proof,
and not the proof itself.
For example, an Isabelle/HOL proof might apply a step
\texttt{apply (blast dest: rearrange reduction)}. The \texttt{blast}
instruction applies
an automatic tableux prover and returns if it found a sequence of proof
steps that work... but the sequence is not considered part of the proof.

A good overview of
higher-level proof verification languages (such as {\sc lcf}\index{lcf@{\sc
lcf}} and {\sc hol}\index{hol@{\sc hol}})
is given in \cite{Harrison}.  All of these languages are fundamentally
different from Metamath in that much of the mathematical foundational
knowledge is embedded in the underlying proof-verification program, rather
than placed directly in the database that is being verified.
These can have a steep learning curve for those without a mathematical
background.  For example, one usually must have a fair understanding of
mathematical logic in order to follow their proofs.

\subsection{Proof Verifiers}\label{proofverifiers}

A proof verifier is a program that doesn't generate proofs but instead
verifies proofs that you give it.  Many proof verifiers have limited built-in
automated proof capabilities, such as figuring out simple logical inferences
(while still being guided by a person who provides the overall proof).  Metamath
has no built-in automated proof capability other than the limited
capability of its Proof Assistant.

Proof-verification languages are not used as frequently as they might be.
Pure mathematicians are more concerned with producing new results, and such
detail and rigor would interfere with that goal.  The use of computers in pure
mathematics is primarily focused on automated theorem provers (not verifiers),
again with the ultimate goal of aiding the creation of new mathematics.
Automated theorem provers are usually concerned with attacking one theorem at
time rather than making a large, organized database easily available to the
user.  Metamath is one way to help close this gap.

By itself Metamath is a mostly a proof verifier.
This does not mean that other approaches can't be used; the difference
is that in Metamath, the results of various provers must be recorded
step-by-step so that they can be verified.

Another proof-verification language is Mizar,\index{Mizar} which can display
its proofs in the informal language that mathematicians are accustomed to.
Information on the Mizar language is available at \url{http://mizar.org}.

For the working mathematician, Mizar is an excellent tool for rigorously
documenting proofs. Mizar typesets its proofs in the informal English used by
mathematicians (and, while fine for them, are just as inscrutable by
laypersons!). A price paid for Mizar is a relatively steep learning curve of a
couple of weeks.  Several mathematicians are actively formalizing different
areas of mathematics using Mizar and publishing the proofs in a dedicated
journal. Unfortunately the task of formalizing mathematics is still looked
down upon to a certain extent since it doesn't involve the creation of ``new''
mathematics.

The closest system to Metamath is
the {\em Ghilbert}\index{Ghilbert} proof language (\url{http://ghilbert.org})
system developed by
Raph Levien\index{Levien, Raph}.
Ghilbert is a formal proof checker heavily inspired by Metamath.
Ghilbert statements are s-expressions (a la Lisp), which is easy
for computers to parse but many people find them hard to read.
There are a number of differences in their specific constructs, but
there is at least one tool to translate some Metamath materials into Ghilbert.
As of 2019 the Ghilbert community is smaller and less active than the
Metamath community.
That said, the Metamath and Ghilbert communities overlap, and fruitful
conversations between them have occurred many times over the years.

\subsection{Creating a Database of Formalized Mathematics}\label{mathdatabase}

Besides Metamath, there are several other ongoing projects with the goal of
formalizing mathematics into computer-verifiable databases.
Understanding some history will help.

The {\sc qed}\index{qed project@{\sc qed} project}%
\footnote{\url{http://www-unix.mcs.anl.gov/qed}.}
project arose in 1993 and its goals were outlined in the
{\sc qed} manifesto.
The {\sc qed} manifesto was
a proposal for a computer-based database of all mathematical knowledge,
strictly formalized and with all proofs having been checked automatically.
The project had a conference in 1994 and another in 1995;
there was also a ``twenty years of the {\sc qed} manifesto'' workshop
in 2014.
Its ideals are regularly reraised.

In a 2007 paper, Freek Wiedijk identified two reasons
for the failure of the {\sc qed} project as originally envisioned:%
\cite{Wiedijk-revisited}\index{Wiedijk, Freek}

\begin{itemize}
\item Very few people are working on formalization of mathematics. There is no compelling application for fully mechanized mathematics.
\item Formalized mathematics does not yet resemble traditional mathematics. This is partly due to the complexity of mathematical notation, and partly to the limitations of existing theorem provers and proof assistants.
\end{itemize}

But this did not end the dream of
formalizing mathematics into computer-verifiable databases.
The problems that led to the {\sc qed} manifesto are still with us,
even though the challenges were harder than originally considered.
What has happened instead is that various independent projects have
worked towards formalizing mathematics into computer-verifiable databases,
each simultaneously competing and cooperating with each other.

A concrete way to see this is
Freek Wiedijk's ``Formalizing 100 Theorems'' list%
\footnote{\url{http://www.cs.ru.nl/\%7Efreek/100/}.}
which shows the progress different systems have made on a challenge list
of 100 mathematical theorems.%
\footnote{ This is not the only list of ``interesting'' theorems.
Another interesting list was posted by Oliver Knill's list
\cite{Knill}\index{Knill, Oliver}.}
The top systems as of February 2019
(in order of the number of challenges completed) are
HOL Light, Isabelle, Metamath, Coq, and Mizar.

The Metamath 100%
\footnote{\url{http://us.metamath.org/mm\_100.html}}
page (maintained by David A. Wheeler\index{Wheeler, David A.})
shows the progress of Metamath (specifically its \texttt{set.mm} database)
against this challenge list maintained by Freek Wiedijk.
The Metamath \texttt{set.mm} database
has made a lot of progress over the years,
in part because working to prove those challenge theorems required
defining various terms and proving their properties as a prerequisite.
Here are just a few of the many statements that have been
formally proven with Metamath:

% The entries of this cause the narrow display to break poorly,
% since the short amount of text means LaTeX doesn't get a lot to work with
% and the itemize format gives it even *less* margin than usual.
% No one will mind if we make just this list flushleft, since this list
% will be internally consistent.
\begin{flushleft}
\begin{itemize}
\item 1. The Irrationality of the Square Root of 2
  (\texttt{sqr2irr}, by Norman Megill, 2001-08-20)
\item 2. The Fundamental Theorem of Algebra
  (\texttt{fta}, by Mario Carneiro, 2014-09-15)
\item 22. The Non-Denumerability of the Continuum
  (\texttt{ruc}, by Norman Megill, 2004-08-13)
\item 54. The Konigsberg Bridge Problem
  (\texttt{konigsberg}, by Mario Carneiro, 2015-04-16)
\item 83. The Friendship Theorem
  (\texttt{friendship}, by Alexander W. van der Vekens, 2018-10-09)
\end{itemize}
\end{flushleft}

We thank all of those who have developed at least one of the Metamath 100
proofs, and we particularly thank
Mario Carneiro\index{Carneiro, Mario}
who has contributed the most Metamath 100 proofs as of 2019.
The Metamath 100 page shows the list of all people who have contributed a
proof, and links to graphs and charts showing progress over time.
We encourage others to work on proving theorems not yet proven in Metamath,
since doing so improves the work as a whole.

Each of the math formalization systems (including Metamath)
has different strengths and weaknesses, depending on what you value.
Key aspects that differentiate Metamath from the other top systems are:

\begin{itemize}
\item Metamath is not tied to any particular set of axioms.
\item Metamath can show every step of every proof, no exceptions.
  Most other provers only assert that a proof can be found, and do not
  show every step. This also makes verification fast, because
  the system does not need to rediscover proof details.
\item The Metamath verifier has been re-implemented in many different
  programming languages, so verification can be done by multiple
  implementations.  In particular, the
  \texttt{set.mm}\index{set theory database (\texttt{set.mm})}%
  \index{Metamath Proof Explorer} database is verified by
  four different verifiers
  written in four different languages by four different authors.
  This greatly reduces the risk of accepting an invalid
  proof due to an error in the verifier.
\item Proofs stay proven.  In some systems, changes to the system's
  syntax or how a tactic works causes proofs to fail in later versions,
  causing older work to become essentially lost.
  Metamath's language is
  extremely small and fixed, so once a proof is added to a database,
  the database can be rechecked with later versions of the Metamath program
  and with other verifiers of Metamath databases.
  If an axiom or key definition needs to be changed, it is easy to
  manipulate the database as a whole to handle the change
  without touching the underlying verifier.
  Since re-verification of an entire database takes seconds, there
  is never a reason to delay complete verification.
  This aspect is especially compelling if your
  goal is to have a long-term database of proofs.
\item Licensing is generous.  The main Metamath databases are released to
  the public domain, and the main Metamath program is open source software
  under a standard, widely-used license.
\item Substitutions are easy to understand, even by those who are not
  professional mathematicians.
\end{itemize}

Of course, other systems may have advantages over Metamath
that are more compelling, depending on what you value.
In any case, we hope this helps you understand Metamath
within a wider context.

\subsection{In Summary}\label{computers-summary}

To summarize our discussions of computers and mathematics, computer algebra
systems can be viewed as theorem generators focusing on a narrow realm of
mathematics (numbers and their properties), automated theorem provers as proof
generators for specific theorems in a much broader realm covered by a built-in
formal system such as first-order logic, interactive theorem
provers require human guidance, proof verifiers verify proofs but
historically they have been
restricted to first-order logic.
Metamath, in contrast,
is a proof verifier and documenter whose realm is essentially unlimited.

\section{Mathematics and Metamath}

\subsection{Standard Mathematics}

There are a number of ways that Metamath\index{Metamath} can be used with
standard mathematics.  The most satisfying way philosophically is to start at
the very beginning, and develop the desired mathematics from the axioms of
logic and set theory.\index{set theory}  This is the approach taken in the
\texttt{set.mm}\index{set theory database (\texttt{set.mm})}%
\index{Metamath Proof Explorer}
database (also known as the Metamath Proof Explorer).
Among other things, this database builds up to the
axioms of real and complex numbers\index{analysis}\index{real and complex
numbers} (see Section~\ref{real}), and a standard development of analysis, for
example, could start at that point, using it as a basis.   Besides this
philosophical advantage, there are practical advantages to having all of the
tools of set theory available in the supporting infrastructure.

On the other hand, you may wish to start with the standard axioms of a
mathematical theory without going through the set theoretical proofs of those
axioms.  You will need mathematical logic to make inferences, but if you wish
you can simply introduce theorems\index{theorem} of logic as
``axioms''\index{axiom} wherever you need them, with the implicit assumption
that in principle they can be proved, if they are obvious to you.  If you
choose this approach, you will probably want to review the notation used in
\texttt{set.mm}\index{set theory database (\texttt{set.mm})} so that your own
notation will be consistent with it.

\subsection{Other Formal Systems}
\index{formal system}

Unlike some programs, Metamath\index{Metamath} is not limited to any specific
area of mathematics, nor committed to any particular mathematical philosophy
such as classical logic versus intuitionism, nor limited, say, to expressions
in first-order logic.  Although the database \texttt{set.mm}
describes standard logic and set theory, Meta\-math
is actually a general-purpose language for describing a wide variety of formal
systems.\index{formal system}  Non-standard systems such as modal
logic,\index{modal logic} intuitionist logic\index{intuitionism}, higher-order
logic\index{higher-order logic}, quantum logic\index{quantum logic}, and
category theory\index{category theory} can all be described with the Metamath
language.  You define the symbols you prefer and tell Metamath the axioms and
rules you want to start from, and Metamath will verify any inferences you make
from those axioms and rules.  A simple example of a non-standard formal system
is Hofstadter's\index{Hofstadter, Douglas R.} MIU system,\index{MIU-system}
whose Metamath description is presented in Appendix~\ref{MIU}.

This is not hypothetical.
The largest Metamath database is
\texttt{set.mm}\index{set theory database (\texttt{set.mm}}%
\index{Metamath Proof Explorer}), aka the Metamath Proof Explorer,
which uses the most common axioms for mathematical foundations
(specifically classical logic combined with Zermelo--Fraenkel
set theory\index{Zermelo--Fraenkel set theory} with the Axiom of Choice).
But other Metamath databases are available:

\begin{itemize}
\item The database
  \texttt{iset.mm}\index{intuitionistic logic database (\texttt{iset.mm})},
  aka the
  Intuitionistic Logic Explorer\index{Intuitionistic Logic Explorer},
  uses intuitionistic logic (a constructivist point of view)
  instead of classical logic.
\item The database
  \texttt{nf.mm}\index{New Foundations database (\texttt{nf.mm})},
  aka the
  New Foundations Explorer\index{New Foundations Explorer},
  constructs mathematics from scratch,
  starting from Quine's New Foundations (NF) set theory axioms.
\item The database
  \texttt{hol.mm}\index{Higher-order Logic database (\texttt{hol.mm})},
  aka the
  Higher-Order Logic (HOL) Explorer\index{Higher-Order Logic (HOL) Explorer},
  starts with HOL (also called simple type theory) and derives
  equivalents to ZFC axioms, connecting the two approaches.
\end{itemize}

Since the days of David Hilbert,\index{Hilbert, David} mathematicians have
been concerned with the fact that the metalanguage\index{metalanguage} used to
describe mathematics may be stronger than the mathematics being described.
Metamath\index{Metamath}'s underlying finitary\index{finitary proof},
constructive nature provides a good philosophical basis for studying even the
weakest logics.\index{weak logic}

The usual treatment of many non-standard formal systems\index{formal
system} uses model theory\index{model theory} or proof theory\index{proof
theory} to describe these systems; these theories, in turn, are based on
standard set theory.  In other words, a non-standard formal system is defined
as a set with certain properties, and standard set theory is used to derive
additional properties of this set.  The standard set theory database provided
with Metamath can be used for this purpose, and when used this way
the development of a special
axiom system for the non-standard formal system becomes unnecessary.  The
model- or proof-theoretic approach often allows you to prove much deeper
results with less effort.

Metamath supports both approaches.  You can define the non-standard
formal system directly, or define the non-standard formal system as
a set with certain properties, whichever you find most helpful.

%\section{Additional Remarks}

\subsection{Metamath and Its Philosophy}

Closely related to Metamath\index{Metamath} is a philosophy or way of looking
at mathematics. This philosophy is related to the formalist
philosophy\index{formalism} of Hilbert\index{Hilbert, David} and his followers
\cite[pp.~1203--1208]{Kline}\index{Kline, Morris}
\cite[p.~6]{Behnke}\index{Behnke, H.}. In this philosophy, mathematics is
viewed as nothing more than a set of rules that manipulate symbols, together
with the consequences of those rules.  While the mathematics being described
may be complex, the rules used to describe it (the
``metamathematics''\index{metamathematics}) should be as simple as possible.
In particular, proofs should be restricted to dealing with concrete objects
(the symbols we write on paper rather than the abstract concepts they
represent) in a constructive manner; these are called ``finitary''
proofs\index{finitary proof} \cite[pp.~2--3]{Shoenfield}\index{Shoenfield,
Joseph R.}.

Whether or not you find Metamath interesting or useful will in part depend on
the appeal you find in its philosophy, and this appeal will probably depend on
your particular goals with respect to mathematics.  For example, if you are a
pure mathematician at the forefront of discovering new mathematical knowledge,
you will probably find that the rigid formality of Metamath stifles your
creativity.  On the other hand, we would argue that once this knowledge is
discovered, there are advantages to documenting it in a standard format that
will make it accessible to others.  Sixty years from now, your field may be
dormant, and as Davis and Hersh put it, your ``writings would become less
translatable than those of the Maya'' \cite[p.~37]{Davis}\index{Davis, Phillip
J.}.


\subsection{A History of the Approach Behind Metamath}

Probably the one work that has had the most motivating influence on
Metamath\index{Metamath} is Whitehead and Russell's monumental {\em Principia
Mathematica} \cite{PM}\index{Whitehead, Alfred North}\index{Russell,
Bertrand}\index{principia mathematica@{\em Principia Mathematica}}, whose aim
was to deduce all of mathematics from a small number of primitive ideas, in a
very explicit way that in principle anyone could understand and follow.  While
this work was tremendously influential in its time, from a modern perspective
it suffers from several drawbacks.  Both its notation and its underlying
axioms are now considered dated and are no longer used.  From our point of
view, its development is not really as accessible as we would like to see; for
practical reasons, proofs become more and more sketchy as its mathematics
progresses, and working them out in fine detail requires a degree of
mathematical skill and patience that many people don't have.  There are
numerous small errors, which is understandable given the tedious, technical
nature of the proofs and the lack of a computer to verify the details.
However, even today {\em Principia Mathematica} stands out as the work closest
in spirit to Metamath.  It remains a mind-boggling work, and one can't help
but be amazed at seeing ``$1+1=2$'' finally appear on page 83 of Volume II
(Theorem *110.643).

The origin of the proof notation used by Metamath dates back to the 1950's,
when the logician C.~A.~Meredith expressed his proofs in a compact notation
called ``condensed detachment''\index{condensed detachment}
\cite{Hindley}\index{Hindley, J. Roger} \cite{Kalman}\index{Kalman, J. A.}
\cite{Meredith}\index{Meredith, C. A.} \cite{Peterson}\index{Peterson, Jeremy
George}.  This notation allows proofs to be communicated unambiguously by
merely referencing the axiom\index{axiom}, rule\index{rule}, or
theorem\index{theorem} used at each step, without explicitly indicating the
substitutions\index{substitution!variable}\index{variable substitution} that
have to be made to the variables in that axiom, rule, or theorem.  Ordinarily,
condensed detachment is more or less limited to propositional
calculus\index{propositional calculus}.  The concept has been extended to
first-order logic\index{first-order logic} in \cite{Megill}\index{Megill,
Norman}, making it is easy to write a small computer program to verify proofs
of simple first-order logic theorems.\index{condensed detachment!and
first-order logic}

A key concept behind the notation of condensed detachment is called
``unification,''\index{unification} which is an algorithm for determining what
substitutions\index{substitution!variable}\index{variable substitution} to
variables have to be made to make two expressions match each other.
Unification was first precisely defined by the logician J.~A.~Robinson, who
used it in the development of a powerful
theorem-proving technique called the ``resolution principle''
\cite{Robinson}\index{Robinson's resolution principle}. Metamath does not make
use of the resolution principle, which is intended for systems of first-order
logic.\index{first-order logic}  Metamath's use is not restricted to
first-order logic, and as we have mentioned it does not automatically discover
proofs.  However, unification is a key idea behind Metamath's proof
notation, and Metamath makes use of a very simple version of it
(Section~\ref{unify}).

\subsection{Metamath and First-Order Logic}

First-order logic\index{first-order logic} is the supporting structure
for standard mathematics.  On top of it is set theory, which contains
the axioms from which virtually all of mathematics can be derived---a
remarkable fact.\index{category
theory}\index{cardinal, inaccessible}\label{categoryth}\footnote{An exception seems
to be category theory.  There are several schools of thought on whether
category theory is derivable from set theory.  At a minimum, it appears
that an additional axiom is needed that asserts the existence of an
``inaccessible cardinal'' (a type of infinity so large that standard set
theory can't prove or deny that it exists).
%
%%%% (I took this out that was in previous editions:)
% But it is also argued that not just one but a ``proper class'' of them
% is needed, and the existence of proper classes is impossible in standard
% set theory.  (A proper class is a collection of sets so huge that no set
% can contain it as an element.  Proper classes can lead to
% inconsistencies such as ``Russell's paradox.''  The axioms of standard
% set theory are devised so as to deny the existence of proper classes.)
%
For more information, see
\cite[pp.~328--331]{Herrlich}\index{Herrlich, Horst} and
\cite{Blass}\index{Blass, Andrea}.}

One of the things that makes Metamath\index{Metamath} more practical for
first-order theories is a set of axioms for first-order logic designed
specifically with Metamath's approach in mind.  These are included in
the database \texttt{set.mm}\index{set theory database (\texttt{set.mm})}.
See Chapter~\ref{fol} for a detailed
description; the axioms are shown in Section~\ref{metaaxioms}.  While
logically equivalent to standard axiom systems, our axiom system breaks
up the standard axioms into smaller pieces such that from them, you can
directly derive what in other systems can only be derived as higher-level
``metatheorems.''\index{metatheorem}  In other words, it is more powerful than
the standard axioms from a metalogical point of view.  A rigorous
justification for this system and its ``metalogical
completeness''\index{metalogical completeness} is found in
\cite{Megill}\index{Megill, Norman}.  The system is closely related to a
system developed by Monk\index{Monk, J. Donald} and Tarski\index{Tarski,
Alfred} in 1965 \cite{Monks}.

For example, the formula $\exists x \, x = y $ (given $y$, there exists some
$x$ equal to it) is a theorem of logic,\footnote{Specifically, it is a theorem
of those systems of logic that assume non-empty domains.  It is not a theorem
of more general systems that include the empty domain\index{empty domain}, in
which nothing exists, period!  Such systems are called ``free
logics.''\index{free logic} For a discussion of these systems, see
\cite{Leblanc}\index{Leblanc, Hugues}.  Since our use for logic is as a basis
for set theory, which has a non-empty domain, it is more convenient (and more
traditional) to use a less general system.  An interesting curiosity is that,
using a free logic as a basis for Zermelo--Fraenkel set
theory\index{Zermelo--Fraenkel set theory} (with the redundant Axiom of the
Null Set omitted),\index{Axiom of the Null Set} we cannot even prove the
existence of a single set without assuming the axiom of infinity!\index{Axiom
of Infinity}} whether or not $x$ and $y$ are distinct variables\index{distinct
variables}.  In many systems of logic, we would have to prove two theorems to
arrive at this result.  First we would prove ``$\exists x \, x = x $,'' then
we would separately prove ``$\exists x \, x = y $, where $x$ and $y$ are
distinct variables.''  We would then combine these two special cases ``outside
of the system'' (i.e.\ in our heads) to be able to claim, ``$\exists x \, x =
y $, regardless of whether $x$ and $y$ are distinct.''  In other words, the
combination of the two special cases is a
metatheorem.  In the system of logic
used in Metamath's set theory\index{set theory database (\texttt{set.mm})}
database, the axioms of logic are broken down into small pieces that allow
them to be reassembled in such a way that theorems such as these can be proved
directly.

Breaking down the axioms in this way makes them look peculiar and not very
intuitive at first, but rest assured that they are correct and complete.  Their
correctness is ensured because they are theorem schemes of standard first-order
logic (which you can easily verify if you are a logician).  Their completeness
follows from the fact that we explicitly derive the standard axioms of
first-order logic as theorems.  Deriving the standard axioms is somewhat
tricky, but once we're there, we have at our disposal a system that is less
awkward to work with in formal proofs\index{formal proof}.  In technical terms
that logicians understand, we eliminate the cumbersome concepts of ``free
variable,''\index{free variable} ``bound variable,''\index{bound variable} and
``proper substitution''\index{proper substitution}\index{substitution!proper}
as primitive notions.  These concepts are present in our system but are
defined in terms of concepts expressed by the axioms and can be eliminated in
principle.  In standard systems, these concepts are really like additional,
implicit axioms\index{implicit axiom} that are somewhat complex and cannot be
eliminated.

The traditional approach to logic, wherein free variables and proper
substitution is defined, is also possible to do directly in the Metamath
language.  However, the notation tends to become awkward, and there are
disadvantages:  for example, extending the definition of a wff with a
definition is awkward, because the free variable and proper substitution
concepts have to have their definitions also extended.  Our choice of
axioms for \texttt{set.mm} is to a certain extent a matter of style, in
an attempt to achieve overall simplicity, but you should also be aware
that the traditional approach is possible as well if you should choose
it.

\chapter{Using the Metamath Program}
\label{using}

\section{Installation}

The way that you install Metamath\index{Metamath!installation} on your
computer system will vary for different computers.  Current
instructions are provided with the Metamath program download at
\url{http://metamath.org}.  In general, the installation is simple.
There is one file containing the Metamath program itself.  This file is
usually called \texttt{metamath} or \texttt{metamath.}{\em xxx} where
{\em xxx} is the convention (such as \texttt{exe}) for an executable
program on your operating system.  There are several additional files
containing samples of the Metamath language, all ending with
\texttt{.mm}.  The file \texttt{set.mm}\index{set theory database
(\texttt{set.mm})} contains logic and set theory and can be used as a
starting point for other areas of mathematics.

You will also need a text editor\index{text editor} capable of editing plain
{\sc ascii}\footnote{American Standard Code for Information Interchange.} text
in order to prepare your input files.\index{ascii@{\sc ascii}}  Most computers
have this capability built in.  Note that plain text is not necessarily the
default for some word processing programs\index{word processor}, especially if
they can handle different fonts; for example, with Microsoft Word\index{Word
(Microsoft)}, you must save the file in the format ``Text Only With Line
Breaks'' to get a plain text\index{plain text} file.\footnote{It is
recommended that all lines in a Metamath source file be 79 characters or less
in length for compatibility among different computer terminals.  When creating
a source file on an editor such as Word, select a monospaced
font\index{monospaced font} such as Courier\index{Courier font} or
Monaco\index{Monaco font} to make this easier to achieve.  Better yet,
just use a plain text editor such as Notepad.}

On some computer systems, Metamath does not have the capability to print
its output directly; instead, you send its output to a file (using the
\texttt{open} commands described later).  The way you print this output
file depends on your computer.\index{printers} Some computers have a
print command, whereas with others, you may have to read the file into
an editor and print it from there.

If you want to print your Metamath source files with typeset formulas
containing standard mathematical symbols, you will need the \LaTeX\
typesetting program\index{latex@{\LaTeX}}, which is widely and freely
available for most operating systems.  It runs natively on Unix and
Linux, and can be installed on Windows as part of the free Cygwin
package (\url{http://cygwin.com}).

You can also produce {\sc html}\footnote{HyperText Markup Language.}
web pages.  The {\tt help html} command in the Metamath program will
assist you with this feature.

\section{Your First Formal System}\label{start}
\subsection{From Nothing to Zero}\label{startf}

To give you a feel for what the Metamath\index{Metamath} language looks like,
we will take a look at a very simple example from formal number
theory\index{number theory}.  This example is taken from
Mendelson\index{Mendelson, Elliot} \cite[p. 123]{Mendelson}.\footnote{To keep
the example simple, we have changed the formalism slightly, and what we call
axioms\index{axiom} are strictly speaking theorems\index{theorem} in
\cite{Mendelson}.}  We will look at a small subset of this theory, namely that
part needed for the first number theory theorem proved in \cite{Mendelson}.

First we will look at a standard formal proof\index{formal proof} for the
example we have picked, then we will look at the Metamath version.  If you
have never been exposed to formal proofs, the notation may seem to be such
overkill to express such simple notions that you may wonder if you are missing
something.  You aren't.  The concepts involved are in fact very simple, and a
detailed breakdown in this fashion is necessary to express the proof in a way
that can be verified mechanically.  And as you will see, Metamath breaks the
proof down into even finer pieces so that the mechanical verification process
can be about as simple as possible.

Before we can introduce the axioms\index{axiom} of the theory, we must define
the syntax rules for forming legal expressions\index{syntax rules}
(combinations of symbols) with which those axioms can be used. The number 0 is
a {\bf term}\index{term}; and if $ t$ and $r$ are terms, so is $(t+r)$. Here,
$ t$ and $r$ are ``metavariables''\index{metavariable} ranging over terms; they
themselves do not appear as symbols in an actual term.  Some examples of
actual terms are $(0 + 0)$ and $((0+0)+0)$.  (Note that our theory describes
only the number zero and sums of zeroes.  Of course, not much can be done with
such a trivial theory, but remember that we have picked a very small subset of
complete number theory for our example.  The important thing for you to focus
on is our definitions that describe how symbols are combined to form valid
expressions, and not on the content or meaning of those expressions.) If $ t$
and $r$ are terms, an expression of the form $ t=r$ is a {\bf wff}
(well-formed formula)\index{well-formed formula (wff)}; and if $P$ and $Q$ are
wffs, so is $(P\rightarrow Q)$ (which means ``$P$ implies
$Q$''\index{implication ($\rightarrow$)} or ``if $P$ then $Q$'').
Here $P$ and $Q$ are metavariables ranging over wffs.  Examples of actual
wffs are $0=0$, $(0+0)=0$, $(0=0 \rightarrow (0+0)=0)$, and $(0=0\rightarrow
(0=0\rightarrow 0=(0+0)))$.  (Our notation makes use of more parentheses than
are customary, but the elimination of ambiguity this way simplifies our
example by avoiding the need to define operator precedence\index{operator
precedence}.)

The {\bf axioms}\index{axiom} of our theory are all wffs of the following
form, where $ t$, $r$, and $s$ are any terms:

%Latex p. 92
\renewcommand{\theequation}{A\arabic{equation}}

\begin{equation}
(t=r\rightarrow (t=s\rightarrow r=s))
\end{equation}
\begin{equation}
(t+0)=t
\end{equation}

Note that there are an infinite number of axioms since there are an infinite
number of possible terms.  A1 and A2 are properly called ``axiom
schemes,''\index{axiom scheme} but we will refer to them as ``axioms'' for
brevity.

An axiom is a {\bf theorem}; and if $P$ and $(P\rightarrow Q)$ are theorems
(where $P$ and $Q$ are wffs), then $Q$ is also a theorem.\index{theorem}  The
second part of this definition is called the modus ponens (MP) rule of
inference\index{inference rule}\index{modus ponens}.  It allows us to obtain
new theorems from old ones.

The {\bf proof}\index{proof} of a theorem is a sequence of one or more
theorems, each of which is either an axiom or the result of modus ponens
applied to two previous theorems in the sequence, and the last of which is the
theorem being proved.

The theorem we will prove for our example is very simple:  $ t=t$.  The proof of
our theorem follows.  Study it carefully until you feel sure you
understand it.\label{zeroproof}

% Use tabu so that lines will wrap automatically as needed.
\begin{tabu} { l X X }
1. & $(t+0)=t$ & (by axiom A2) \\
2. & $(t+0)=t$ & (by axiom A2) \\
3. & $((t+0)=t \rightarrow ((t+0)=t\rightarrow t=t))$ & (by axiom A1) \\
4. & $((t+0)=t\rightarrow t=t)$ & (by MP applied to steps 2 and 3) \\
5. & $t=t$ & (by MP applied to steps 1 and 4) \\
\end{tabu}

(You may wonder why step 1 is repeated twice.  This is not necessary in the
formal language we have defined, but in Metamath's ``reverse Polish
notation''\index{reverse Polish notation (RPN)} for proofs, a previous step
can be referred to only once.  The repetition of step~1 here will enable you
to see more clearly the correspondence of this proof with the
Metamath\index{Metamath} version on p.~\pageref{demoproof}.)

Our theorem is more properly called a ``theorem scheme,''\index{theorem
scheme} for it represents an infinite number of theorems, one for each
possible term $ t$.  Two examples of actual theorems would be $0=0$ and
$(0+0)=(0+0)$.  Rarely do we prove actual theorems, since by proving schemes
we can prove an infinite number of theorems in one fell swoop.  Similarly, our
proof should really be called a ``proof scheme.''\index{proof scheme}  To
obtain an actual proof, pick an actual term to use in place of $ t$, and
substitute it for $ t$ throughout the proof.

Let's discuss what we have done here.  The axioms\index{axiom} of our theory,
A1 and A2, are trivial and obvious.  Everyone knows that adding zero to
something doesn't change it, and also that if two things are equal to a third,
then they are equal to each other. In fact, stating the trivial and obvious is
a goal to strive for in any axiomatic system.  From trivial and obvious truths
that everyone agrees upon, we can prove results that are not so obvious yet
have absolute faith in them.  If we trust the axioms and the rules, we must,
by definition, trust the consequences of those axioms and rules, if logic is
to mean anything at all.

Our rule of inference\index{rule}, modus ponens\index{modus ponens}, is also
pretty obvious once you understand what it means.  If we prove a fact $P$, and
we also prove that $P$ implies $Q$, then $Q$ necessarily follows as a new
fact.  The rule provides us with a means for obtaining new facts (i.e.\
theorems\index{theorem}) from old ones.

The theorem that we have proved, $ t=t$, is so fundamental that you may wonder
why it isn't one of the axioms\index{axiom}.  In some axiom systems of
arithmetic, it {\em is} an axiom.  The choice of axioms in a theory is to some
extent arbitrary and even an art form, constrained only by the requirement
that any two equivalent axiom systems be able to derive each other as
theorems.  We could imagine that the inventor of our axiom system originally
included $ t=t$ as an axiom, then discovered that it could be derived as a
theorem from the other axioms.  Because of this, it was not necessary to
keep it as an axiom.  By eliminating it, the final set of axioms became
that much simpler.

Unless you have worked with formal proofs\index{formal proof} before, it
probably wasn't apparent to you that $ t=t$ could be derived from our two
axioms until you saw the proof. While you certainly believe that $ t=t$ is
true, you might not be able to convince an imaginary skeptic who believes only
in our two axioms until you produce the proof.  Formal proofs such as this are
hard to come up with when you first start working with them, but after you get
used to them they can become interesting and fun.  Once you understand the
idea behind formal proofs you will have grasped the fundamental principle that
underlies all of mathematics.  As the mathematics becomes more sophisticated,
its proofs become more challenging, but ultimately they all can be broken down
into individual steps as simple as the ones in our proof above.

Mendelson's\index{Mendelson, Elliot} book, from which our example was taken,
contains a number of detailed formal proofs such as these, and you may be
interested in looking it up.  The book is intended for mathematicians,
however, and most of it is rather advanced.  Popular literature describing
formal proofs\index{formal proof} include \cite[p.~296]{Rucker}\index{Rucker,
Rudy} and \cite[pp.~204--230]{Hofstadter}\index{Hofstadter, Douglas R.}.

\subsection{Converting It to Metamath}\label{convert}

Formal proofs\index{formal proof} such as the one in our example break down
logical reasoning into small, precise steps that leave little doubt that the
results follow from the axioms\index{axiom}.  You might think that this would
be the finest breakdown we can achieve in mathematics.  However, there is more
to the proof than meets the eye. Although our axioms were rather simple, a lot
of verbiage was needed before we could even state them:  we needed to define
``term,'' ``wff,'' and so on.  In addition, there are a number of implied
rules that we haven't even mentioned. For example, how do we know that step 3
of our proof follows from axiom A1? There is some hidden reasoning involved in
determining this.  Axiom A1 has two occurrences of the letter $ t$.  One of
the implied rules states that whatever we substitute for $ t$ must be a legal
term\index{term}.\footnote{Some authors make this implied rule explicit by
stating, ``only expressions of the above form are terms,'' after defining
``term.''}  The expression $ t+0$ is pretty obviously a legal term whenever $
t$ is, but suppose we wanted to substitute a huge term with thousands of
symbols?  Certainly a lot of work would be involved in determining that it
really is a term, but in ordinary formal proofs all of this work would be
considered a single ``step.''

To express our axiom system in the Metamath\index{Metamath} language, we must
describe this auxiliary information in addition to the axioms themselves.
Metamath does not know what a ``term'' or a ``wff''\index{well-formed formula
(wff)} is.  In Metamath, the specification of the ways in which we can combine
symbols to obtain terms and wffs are like little axioms in themselves.  These
auxiliary axioms are expressed in the same notation as the ``real''
axioms\index{axiom}, and Metamath does not distinguish between the two.  The
distinction is made by you, i.e.\ by the way in which you interpret the
notation you have chosen to express these two kinds of axioms.

The Metamath language breaks down mathematical proofs into tiny pieces, much
more so than in ordinary formal proofs\index{formal proof}.  If a single
step\index{proof step} involves the
substitution\index{substitution!variable}\index{variable substitution} of a
complex term for one of its variables, Metamath must see this single step
broken down into many small steps.  This fine-grained breakdown is what gives
Metamath generality and flexibility as it lets it not be limited to any
particular mathematical notation.

Metamath's proof notation is not, in itself, intended to be read by humans but
rather is in a compact format intended for a machine.  The Metamath program
will convert this notation to a form you can understand, using the \texttt{show
proof}\index{\texttt{show proof} command} command.  You can tell the program what
level of detail of the proof you want to look at.  You may want to look at
just the logical inference steps that correspond
to ordinary formal proof steps,
or you may want to see the fine-grained steps that prove that an expression is
a term.

Here, without further ado, is our example converted to the
Metamath\index{Metamath} language:\index{metavariable}\label{demo0}

\begin{verbatim}
$( Declare the constant symbols we will use $)
    $c 0 + = -> ( ) term wff |- $.
$( Declare the metavariables we will use $)
    $v t r s P Q $.
$( Specify properties of the metavariables $)
    tt $f term t $.
    tr $f term r $.
    ts $f term s $.
    wp $f wff P $.
    wq $f wff Q $.
$( Define "term" and "wff" $)
    tze $a term 0 $.
    tpl $a term ( t + r ) $.
    weq $a wff t = r $.
    wim $a wff ( P -> Q ) $.
$( State the axioms $)
    a1 $a |- ( t = r -> ( t = s -> r = s ) ) $.
    a2 $a |- ( t + 0 ) = t $.
$( Define the modus ponens inference rule $)
    ${
       min $e |- P $.
       maj $e |- ( P -> Q ) $.
       mp  $a |- Q $.
    $}
$( Prove a theorem $)
    th1 $p |- t = t $=
  $( Here is its proof: $)
       tt tze tpl tt weq tt tt weq tt a2 tt tze tpl
       tt weq tt tze tpl tt weq tt tt weq wim tt a2
       tt tze tpl tt tt a1 mp mp
     $.
\end{verbatim}\index{metavariable}

A ``database''\index{database} is a set of one or more {\sc ascii} source
files.  Here's a brief description of this Metamath\index{Metamath} database
(which consists of this single source file), so that you can understand in
general terms what is going on.  To understand the source file in detail, you
should read Chapter~\ref{languagespec}.

The database is a sequence of ``tokens,''\index{token} which are normally
separated by spaces or line breaks.  The only tokens that are built into
the Metamath language are those beginning with \texttt{\$}.  These tokens
are called ``keywords.''\index{keyword}  All other tokens are
user-defined, and their names are arbitrary.

As you might have guessed, the Metamath token \texttt{\$(}\index{\texttt{\$(} and
\texttt{\$)} auxiliary keywords} starts a comment and \texttt{\$)} ends a comment.

The Metamath tokens \texttt{\$c}\index{\texttt{\$c} statement},
\texttt{\$v}\index{\texttt{\$v} statement},
\texttt{\$e}\index{\texttt{\$e} statement},
\texttt{\$f}\index{\texttt{\$f} statement},
\texttt{\$a}\index{\texttt{\$a} statement}, and
\texttt{\$p}\index{\texttt{\$p} statement} specify ``statements'' that
end with \texttt{\$.}\,.\index{\texttt{\$.}\ keyword}

The Metamath tokens \texttt{\$c} and \texttt{\$v} each declare\index{constant
declaration}\index{variable declaration} a list of user-defined tokens, called
``math symbols,''\index{math symbol} that the database will reference later
on.  All of the math symbols they define you have seen earlier except the
turnstile symbol \texttt{|-} ($\vdash$)\index{turnstile ({$\,\vdash$})}, which is
commonly used by logicians to mean ``a proof exists for.''  For us
the turnstile is just a
convenient symbol that distinguishes expressions that are axioms\index{axiom}
or theorems\index{theorem} from expressions that are terms or wffs.

The \texttt{\$c} statement declares ``constants''\index{constant} and
the \texttt{\$v} statement declares
``variables''\index{variable}\index{constant declaration}\index{variable
declaration} (or more precisely, metavariables\index{metavariable}).  A
variable may be substituted\index{substitution!variable}\index{variable
substitution} with sequences of math symbols whereas a constant may not
be substituted with anything.

It may seem redundant to require both \texttt{\$c}\index{\texttt{\$c} statement} and
\texttt{\$v}\index{\texttt{\$v} statement} statements (since any math
symbol\index{math symbol} not specified with a \texttt{\$c} statement could be
presumed to be a variable), but this provides for better error checking and
also allows math symbols to be redeclared\index{redeclaration of symbols}
(Section~\ref{scoping}).

The token \texttt{\$f}\index{\texttt{\$f} statement} specifies a
statement called a ``variable-type hypothesis'' (also called a
``floating hypothesis'') and \texttt{\$e}\index{\texttt{\$e} statement}
specifies a ``logical hypothesis'' (also called an ``essential
hypothesis'').\index{hypothesis}\index{variable-type
hypothesis}\index{logical hypothesis}\index{floating
hypothesis}\index{essential hypothesis} The token
\texttt{\$a}\index{\texttt{\$a} statement} specifies an ``axiomatic
assertion,''\index{axiomatic assertion} and
\texttt{\$p}\index{\texttt{\$p} statement} specifies a ``provable
assertion.''\index{provable assertion} To the left of each occurrence of
these four tokens is a ``label''\index{label} that identifies the
hypothesis or assertion for later reference.  For example, the label of
the first axiomatic assertion is \texttt{tze}.  A \texttt{\$f} statement
must contain exactly two math symbols, a constant followed by a
variable.  The \texttt{\$e}, \texttt{\$a}, and \texttt{\$p} statements
each start with a constant followed by, in general, an arbitrary
sequence of math symbols.

Associated with each assertion\index{assertion} is a set of hypotheses
that must be satisfied in order for the assertion to be used in a proof.
These are called the ``mandatory hypotheses''\index{mandatory
hypothesis} of the assertion.  Among those hypotheses whose ``scope''
(described below) includes the assertion, \texttt{\$e} hypotheses are
always mandatory and \texttt{\$f}\index{\texttt{\$f} statement}
hypotheses are mandatory when they share their variable with the
assertion or its \texttt{\$e} hypotheses.  The exact rules for
determining which hypotheses are mandatory are described in detail in
Sections~\ref{frames} and \ref{scoping}.  For example, the mandatory
hypotheses of assertion \texttt{tpl} are \texttt{tt} and \texttt{tr},
whereas assertion \texttt{tze} has no mandatory hypotheses because it
contains no variables and has no \texttt{\$e}\index{\texttt{\$e}
statement} hypothesis.  Metamath's \texttt{show statement}
command\index{\texttt{show statement} command}, described in the next
section, will show you a statement's mandatory hypotheses.

Sometimes we need to make a hypothesis relevant to only certain
assertions.  The set of statements to which a hypothesis is relevant is
called its ``scope.''  The Metamath brackets,
\texttt{\$\char`\{}\index{\texttt{\$\char`\{} and \texttt{\$\char`\}}
keywords} and \texttt{\$\char`\}}, define a ``block''\index{block} that
delimits the scope of any hypothesis contained between them.  The
assertion \texttt{mp} has mandatory hypotheses \texttt{wp}, \texttt{wq},
\texttt{min}, and \texttt{maj}.  The only mandatory hypothesis of
\texttt{th1}, on the other hand, is \texttt{tt}, since \texttt{th1}
occurs outside of the block containing \texttt{min} and \texttt{maj}.

Note that \texttt{\$\char`\{} and \texttt{\$\char`\}} do not affect the
scope of assertions (\texttt{\$a} and \texttt{\$p}).  Assertions are always
available to be referenced by any later proof in the source file.

Each provable assertion (\texttt{\$p}\index{\texttt{\$p} statement}
statement) has two parts.  The first part is the
assertion\index{assertion} itself, which is a sequence of math
symbol\index{math symbol} tokens placed between the \texttt{\$p} token
and a \texttt{\$=}\index{\texttt{\$=} keyword} token.  The second part
is a ``proof,'' which is a list of label tokens placed between the
\texttt{\$=} token and the \texttt{\$.}\index{\texttt{\$.}\ keyword}\
token that ends the statement.\footnote{If you've looked at the
\texttt{set.mm} database, you may have noticed another notation used for
proofs.  The other notation is called ``compressed.''\index{compressed
proof}\index{proof!compressed} It reduces the amount of space needed to
store a proof in the database and is described in
Appendix~\ref{compressed}.  In the example above, we use
``normal''\index{normal proof}\index{proof!normal} notation.} The proof
acts as a series of instructions to the Metamath program, telling it how
to build up the sequence of math symbols contained in the assertion part of
the \texttt{\$p} statement, making use of the hypotheses of the
\texttt{\$p} statement and previous assertions.  The construction takes
place according to precise rules.  If the list of labels in the proof
causes these rules to be violated, or if the final sequence that results
does not match the assertion, the Metamath program will notify you with
an error message.

If you are familiar with reverse Polish notation (RPN), which is sometimes used
on pocket calculators, here in a nutshell is how a proof works.  Each
hypothesis label\index{hypothesis label} in the proof is pushed\index{push}
onto the RPN stack\index{stack}\index{RPN stack} as it is encountered. Each
assertion label\index{assertion label} pops\index{pop} off the stack as many
entries as the referenced assertion has mandatory hypotheses.  Variable
substitutions\index{substitution!variable}\index{variable substitution} are
computed which, when made to the referenced assertion's mandatory hypotheses,
cause these hypotheses to match the stack entries. These same substitutions
are then made to the variables in the referenced assertion itself, which is
then pushed onto the stack.  At the end of the proof, there should be one
stack entry, namely the assertion being proved.  This process is explained in
detail in Section~\ref{proof}.

Metamath's proof notation is not very readable for humans, but it allows the
proof to be stored compactly in a file.  The Metamath\index{Metamath} program
has proof display features that let you see what's going on in a more
readable way, as you will see in the next section.

The rules used in verifying a proof are not based on any built-in syntax of the
symbol sequence in an assertion\index{assertion} nor on any built-in meanings
attached to specific symbol names.  They are based strictly on symbol
matching:  constants\index{constant} must match themselves, and
variables\index{variable} may be replaced with anything that allows a match to
occur.  For example, instead of \texttt{term}, \texttt{0}, and \verb$|-$ we could
have just as well used \texttt{yellow}, \texttt{zero}, and \texttt{provable}, as long
as we did so consistently throughout the database.  Also, we could have used
\texttt{is provable} (two tokens) instead of \verb$|-$ (one token) throughout the
database.  In each of these cases, the proof would be exactly the same.  The
independence of proofs and notation means that you have a lot of flexibility to
change the notation you use without having to change any proofs.

\section{A Trial Run}\label{trialrun}

Now you are ready to try out the Metamath\index{Metamath} program.

On all computer systems, Metamath has a standard ``command line
interface'' (CLI)\index{command line interface (CLI)} that allows you to
interact with it.  You supply commands to the CLI by typing them on the
keyboard and pressing your keyboard's {\em return} key after each line
you enter.  The CLI is designed to be easy to use and has built-in help
features.

The first thing you should do is to use a text editor to create a file
called \texttt{demo0.mm} and type into it the Metamath source shown on
p.~\pageref{demo0}.  Actually, this file is included with your Metamath
software package, so check that first.  If you type it in, make sure
that you save it in the form of ``plain {\sc ascii} text with line
breaks.''  Most word processors will have this feature.

Next you must run the Metamath program.  Depending on your computer
system and how Metamath is installed, this could range from clicking the
mouse on the Metamath icon to typing \texttt{run metamath} to typing
simply \texttt{metamath}.  (Metamath's {\tt help invoke} command describes
alternate ways of invoking the Metamath program.)

When you first enter Metamath\index{Metamath}, it will be at the CLI, waiting
for your input. You will see something like the following on your screen:
\begin{verbatim}
Metamath - Version 0.172 25-Jan-2019
Type HELP for help, EXIT to exit.
MM>
\end{verbatim}
The \texttt{MM>} prompt means that Metamath is waiting for a command.
Command keywords\index{command keyword} are not case sensitive;
we will use lower-case commands in our examples.
The version number and its release date will probably be different on your
system from the one we show above.

The first thing that you need to do is to read in your
database:\index{\texttt{read} command}\footnote{If a directory path is
needed on Unix,\index{Unix file names}\index{file names!Unix} you should
enclose the path/file name in quotes to prevent Metamath from thinking
that the \texttt{/} in the path name is a command qualifier, e.g.,
\texttt{read \char`\"db/set.mm\char`\"}.  Quotes are optional when there
is no ambiguity.}
\begin{verbatim}
MM> read demo0.mm
\end{verbatim}
Remember to press the {\em return} key after entering this command.  If
you omit the file name, Metamath will prompt you for one.   The syntax for
specifying a Macintosh file name path is given in a footnote on
p.~\pageref{includef}.\index{Macintosh file names}\index{file
names!Macintosh}

If there are any syntax errors in the database, Metamath will let you know
when it reads in the file.  The one thing that Metamath does not check when
reading in a database is that all proofs are correct, because this would
slow it down too much.  It is a good idea to periodically verify the proofs in
a database you are making changes to.  To do this, use the following command
(and do it for your \texttt{demo0.mm} file now).  Note that the \texttt{*} is a
``wild card'' meaning all proofs in the file.\index{\texttt{verify proof} command}
\begin{verbatim}
MM> verify proof *
\end{verbatim}
Metamath will report any proofs that are incorrect.

It is often useful to save the information that the Metamath program displays
on the screen. You can save everything that happens on the screen by opening a
log file. You may want to do this before you read in a database so that you
can examine any errors later on.  To open a log file, type
\begin{verbatim}
MM> open log abc.log
\end{verbatim}
This will open a file called \texttt{abc.log}, and everything that appears on the
screen from this point on will be stored in this file.  The name of the log file
is arbitrary. To close the log file, type
\begin{verbatim}
MM> close log
\end{verbatim}

Several commands let you examine what's inside your database.
Section~\ref{exploring} has an overview of some useful ones.  The
\texttt{show labels} command lets you see what statement
labels\index{label} exist.  A \texttt{*} matches any combination of
characters, and \texttt{t*} refers to all labels starting with the
letter \texttt{t}.\index{\texttt{show labels} command} The \texttt{/all}
is a ``command qualifier''\index{command qualifier} that tells Metamath
to include labels of hypotheses.  (To see the syntax explained, type
\texttt{help show labels}.)  Type
\begin{verbatim}
MM> show labels t* /all
\end{verbatim}
Metamath will respond with
\begin{verbatim}
The statement number, label, and type are shown.
3 tt $f       4 tr $f       5 ts $f       8 tze $a
9 tpl $a      19 th1 $p
\end{verbatim}

You can use the \texttt{show statement} command to get information about a
particular statement.\index{\texttt{show statement} command}
For example, you can get information about the statement with label \texttt{mp}
by typing
\begin{verbatim}
MM> show statement mp /full
\end{verbatim}
Metamath will respond with
\begin{verbatim}
Statement 17 is located on line 43 of the file
"demo0.mm".
"Define the modus ponens inference rule"
17 mp $a |- Q $.
Its mandatory hypotheses in RPN order are:
  wp $f wff P $.
  wq $f wff Q $.
  min $e |- P $.
  maj $e |- ( P -> Q ) $.
The statement and its hypotheses require the
      variables:  Q P
The variables it contains are:  Q P
\end{verbatim}
The mandatory hypotheses\index{mandatory hypothesis} and their
order\index{RPN order} are
useful to know when you are trying to understand or debug a proof.

Now you are ready to look at what's really inside our proof.  First, here is
how to look at every step in the proof---not just the ones corresponding to an
ordinary formal proof\index{formal proof}, but also the ones that build up the
formulas that appear in each ordinary formal proof step.\index{\texttt{show
proof} command}
\begin{verbatim}
MM> show proof th1 /lemmon /all
\end{verbatim}

This will display the proof on the screen in the following format:
\begin{verbatim}
 1 tt            $f term t
 2 tze           $a term 0
 3 1,2 tpl       $a term ( t + 0 )
 4 tt            $f term t
 5 3,4 weq       $a wff ( t + 0 ) = t
 6 tt            $f term t
 7 tt            $f term t
 8 6,7 weq       $a wff t = t
 9 tt            $f term t
10 9 a2          $a |- ( t + 0 ) = t
11 tt            $f term t
12 tze           $a term 0
13 11,12 tpl     $a term ( t + 0 )
14 tt            $f term t
15 13,14 weq     $a wff ( t + 0 ) = t
16 tt            $f term t
17 tze           $a term 0
18 16,17 tpl     $a term ( t + 0 )
19 tt            $f term t
20 18,19 weq     $a wff ( t + 0 ) = t
21 tt            $f term t
22 tt            $f term t
23 21,22 weq     $a wff t = t
24 20,23 wim     $a wff ( ( t + 0 ) = t -> t = t )
25 tt            $f term t
26 25 a2         $a |- ( t + 0 ) = t
27 tt            $f term t
28 tze           $a term 0
29 27,28 tpl     $a term ( t + 0 )
30 tt            $f term t
31 tt            $f term t
32 29,30,31 a1   $a |- ( ( t + 0 ) = t -> ( ( t + 0 )
                                     = t -> t = t ) )
33 15,24,26,32 mp  $a |- ( ( t + 0 ) = t -> t = t )
34 5,8,10,33 mp  $a |- t = t
\end{verbatim}

The \texttt{/lemmon} command qualifier specifies what is known as a Lemmon-style
display\index{Lemmon-style proof}\index{proof!Lemmon-style}.  Omitting the
\texttt{/lemmon} qualifier results in a tree-style proof (see
p.~\pageref{treeproof} for an example) that is somewhat less explicit but
easier to follow once you get used to it.\index{tree-style
proof}\index{proof!tree-style}

The first number on each line is the step
number of the proof.  Any numbers that follow are step numbers assigned to the
hypotheses of the statement referenced by that step.  Next is the label of
the statement referenced by the step.  The statement type of the statement
referenced comes next, followed by the math symbol\index{math symbol} string
constructed by the proof up to that step.

The last step, 34, contains the statement that is being proved.

Looking at a small piece of the proof, notice that steps 3 and 4 have
established that
\texttt{( t + 0 )} and \texttt{t} are \texttt{term}\,s, and step 5 makes use of steps 3 and
4 to establish that \texttt{( t + 0 ) = t} is a \texttt{wff}.  Let Metamath
itself tell us in detail what is happening in step 5.  Note that the
``target hypothesis'' refers to where step 5 is eventually used, i.e., in step
34.
\begin{verbatim}
MM> show proof th1 /detailed_step 5
Proof step 5:  wp=weq $a wff ( t + 0 ) = t
This step assigns source "weq" ($a) to target "wp"
($f).  The source assertion requires the hypotheses
"tt" ($f, step 3) and "tr" ($f, step 4).  The parent
assertion of the target hypothesis is "mp" ($a,
step 34).
The source assertion before substitution was:
    weq $a wff t = r
The following substitutions were made to the source
assertion:
    Variable  Substituted with
     t         ( t + 0 )
     r         t
The target hypothesis before substitution was:
    wp $f wff P
The following substitution was made to the target
hypothesis:
    Variable  Substituted with
     P         ( t + 0 ) = t
\end{verbatim}

The full proof just shown is useful to understand what is going on in detail.
However, most of the time you will just be interested in
the ``essential'' or logical steps of a proof, i.e.\ those steps
that correspond to an
ordinary formal proof\index{formal proof}.  If you type
\begin{verbatim}
MM> show proof th1 /lemmon /renumber
\end{verbatim}
you will see\label{demoproof}
\begin{verbatim}
1 a2             $a |- ( t + 0 ) = t
2 a2             $a |- ( t + 0 ) = t
3 a1             $a |- ( ( t + 0 ) = t -> ( ( t + 0 )
                                     = t -> t = t ) )
4 2,3 mp         $a |- ( ( t + 0 ) = t -> t = t )
5 1,4 mp         $a |- t = t
\end{verbatim}
Compare this to the formal proof on p.~\pageref{zeroproof} and
notice the resemblance.
By default Metamath
does not show \texttt{\$f}\index{\texttt{\$f}
statement} hypotheses and everything branching off of them in the proof tree
when the proof is displayed; this makes the proof look more like an ordinary
mathematical proof, which does not normally incorporate the explicit
construction of expressions.
This is called the ``essential'' view
(at one time you had to add the
\texttt{/essential} qualifier in the \texttt{show proof}
command to get this view, but this is now the default).
You can could use the \texttt{/all} qualifier in the \texttt{show
proof} command to also show the explicit construction of expressions.
The \texttt{/renumber} qualifier means to renumber
the steps to correspond only to what is displayed.\index{\texttt{show proof}
command}

To exit Metamath, type\index{\texttt{exit} command}
\begin{verbatim}
MM> exit
\end{verbatim}

\subsection{Some Hints for Using the Command Line Interface}

We will conclude this quick introduction to Metamath\index{Metamath} with some
helpful hints on how to navigate your way through the commands.
\index{command line interface (CLI)}

When you type commands into Metamath's CLI, you only have to type as many
characters of a command keyword\index{command keyword} as are needed to make
it unambiguous.  If you type too few characters, Metamath will tell you what
the choices are.  In the case of the \texttt{read} command, only the \texttt{r} is
needed to specify it unambiguously, so you could have typed\index{\texttt{read}
command}
\begin{verbatim}
MM> r demo0.mm
\end{verbatim}
instead of
\begin{verbatim}
MM> read demo0.mm
\end{verbatim}
In our description, we always show the full command words.  When using the
Metamath CLI commands in a command file (to be read with the \texttt{submit}
command)\index{\texttt{submit} command}, it is good practice to use
the unabbreviated command to ensure your instructions will not become ambiguous
if more commands are added to the Metamath program in the future.

The command keywords\index{command
keyword} are not case sensitive; you may type either \texttt{read} or
\texttt{ReAd}.  File names may or may not be case sensitive, depending on your
computer's operating system.  Metamath label\index{label} and math
symbol\index{math symbol} tokens\index{token} are case-sensitive.

The \texttt{help} command\index{\texttt{help} command} will provide you
with a list of topics you can get help on.  You can then type
\texttt{help} {\em topic} to get help on that topic.

If you are uncertain of a command's spelling, just type as many characters
as you remember of the command.  If you have not typed enough characters to
specify it unambiguously, Metamath will tell you what choices you have.

\begin{verbatim}
MM> show s
         ^
?Ambiguous keyword - please specify SETTINGS,
STATEMENT, or SOURCE.
\end{verbatim}

If you don't know what argument to use as part of a command, type a
\texttt{?}\index{\texttt{]}@\texttt{?}\ in command lines}\ at the
argument position.  Metamath will tell you what it expected there.

\begin{verbatim}
MM> show ?
         ^
?Expected SETTINGS, LABELS, STATEMENT, SOURCE, PROOF,
MEMORY, TRACE_BACK, or USAGE.
\end{verbatim}

Finally, you may type just the first word or words of a command followed
by {\em return}.  Metamath will prompt you for the remaining part of the
command, showing you the choices at each step.  For example, instead of
typing \texttt{show statement th1 /full} you could interact in the
following manner:
\begin{verbatim}
MM> show
SETTINGS, LABELS, STATEMENT, SOURCE, PROOF,
MEMORY, TRACE_BACK, or USAGE <SETTINGS>? st
What is the statement label <th1>?
/ or nothing <nothing>? /
TEX, COMMENT_ONLY, or FULL <TEX>? f
/ or nothing <nothing>?
19 th1 $p |- t = t $= ... $.
\end{verbatim}
After each \texttt{?}\ in this mode, you must give Metamath the
information it requests.  Sometimes Metamath gives you a list of choices
with the default choice indicated by brackets \texttt{< > }. Pressing
{\em return} after the \texttt{?}\ will select the default choice.
Answering anything else will override the default.  Note that the
\texttt{/} in command qualifiers is considered a separate
token\index{token} by the parser, and this is why it is asked for
separately.

\section{Your First Proof}\label{frstprf}

Proofs are developed with the aid of the Proof Assistant\index{Proof
Assistant}.  We will now show you how the proof of theorem \texttt{th1}
was built.  So that you can repeat these steps, we will first have the
Proof Assistant erase the proof in Metamath's source buffer\index{source
buffer}, then reconstruct it.  (The source buffer is the place in memory
where Metamath stores the information in the database when it is
\texttt{read}\index{\texttt{read} command} in.  New or modified proofs
are kept in the source buffer until a \texttt{write source}
command\index{\texttt{write source} command} is issued.)  In practice, you
would place a \texttt{?}\index{\texttt{]}@\texttt{?}\ inside proofs}\
between \texttt{\$=}\index{\texttt{\$=} keyword} and
\texttt{\$.}\index{\texttt{\$.}\ keyword}\ in the database to indicate
to Metamath\index{Metamath} that the proof is unknown, and that would be
your starting point.  Whenever the \texttt{verify proof} command encounters
a proof with a \texttt{?}\ in place of a proof step, the statement is
identified as not proved.

When I first started creating Metamath proofs, I would write down
on a piece of paper the complete
formal proof\index{formal proof} as it would appear
in a \texttt{show proof} command\index{\texttt{show proof} command}; see
the display of \texttt{show proof th1 /lemmon /re\-num\-ber} above as an
example.  After you get used to using the Proof Assistant\index{Proof
Assistant} you may get to a point where you can ``see'' the proof in your mind
and let the Proof Assistant guide you in filling in the details, at least for
simpler proofs, but until you gain that experience you may find it very useful
to write down all the details in advance.
Otherwise you may waste a lot of time as you let it take you down a wrong path.
However, others do not find this approach as helpful.
For example, Thomas Brendan Leahy\index{Leahy, Thomas Brendan}
finds that it is more helpful to him to interactively
work backward from a machine-readable statement.
David A. Wheeler\index{Wheeler, David A.}
writes down a general approach, but develops the proof
interactively by switching between
working forwards (from hypotheses and facts likely to be useful) and
backwards (from the goal) until the forwards and backwards approaches meet.
In the end, use whatever approach works for you.

A proof is developed with the Proof Assistant by working backwards, starting
with the theorem\index{theorem} to be proved, and assigning each unknown step
with a theorem or hypothesis until no more unknown steps remain.  The Proof
Assistant will not let you make an assignment unless it can be ``unified''
with the unknown step.  This means that a
substitution\index{substitution!variable}\index{variable substitution} of
variables exists that will make the assignment match the unknown step.  On the
other hand, in the middle of a proof, when working backwards, often more than
one unification\index{unification} (set of substitutions) is possible, since
there is not enough information available at that point to uniquely establish
it.  In this case you can tell Metamath which unification to choose, or you
can continue to assign unknown steps until enough information is available to
make the unification unique.

We will assume you have entered Metamath and read in the database as described
above.  The following dialog shows how the proof was developed.  For more
details on what some of the commands do, refer to Section~\ref{pfcommands}.
\index{\texttt{prove} command}

\begin{verbatim}
MM> prove th1
Entering the Proof Assistant.  Type HELP for help, EXIT
to exit.  You will be working on the proof of statement th1:
  $p |- t = t
Note:  The proof you are starting with is already complete.
MM-PA>
\end{verbatim}

The \verb/MM-PA>/ prompt means we are inside the Proof
Assistant.\index{Proof Assistant} Most of the regular Metamath commands
(\texttt{show statement}, etc.) are still available if you need them.

\begin{verbatim}
MM-PA> delete all
The entire proof was deleted.
\end{verbatim}

We have deleted the whole proof so we can start from scratch.

\begin{verbatim}
MM-PA> show new_proof/lemmon/all
1 ?              $? |- t = t
\end{verbatim}

The \texttt{show new{\char`\_}proof} command\index{\texttt{show
new{\char`\_}proof} command} is like \texttt{show proof} except that we
don't specify a statement; instead, the proof we're working on is
displayed.

\begin{verbatim}
MM-PA> assign 1 mp
To undo the assignment, DELETE STEP 5 and INITIALIZE, UNIFY
if needed.
3   min=?  $? |- $2
4   maj=?  $? |- ( $2 -> t = t )
\end{verbatim}

The \texttt{assign} command\index{\texttt{assign} command} above means
``assign step 1 with the statement whose label is \texttt{mp}.''  Note
that step renumbering will constantly occur as you assign steps in the
middle of a proof; in general all steps from the step you assign until
the end of the proof will get moved up.  In this case, what used to be
step 1 is now step 5, because the (partial) proof now has five steps:
the four hypotheses of the \texttt{mp} statement and the \texttt{mp}
statement itself.  Let's look at all the steps in our partial proof:

\begin{verbatim}
MM-PA> show new_proof/lemmon/all
1 ?              $? wff $2
2 ?              $? wff t = t
3 ?              $? |- $2
4 ?              $? |- ( $2 -> t = t )
5 1,2,3,4 mp     $a |- t = t
\end{verbatim}

The symbol \texttt{\$2} is a temporary variable\index{temporary
variable} that represents a symbol sequence not yet known.  In the final
proof, all temporary variables will be eliminated.  The general format
for a temporary variable is \texttt{\$} followed by an integer.  Note
that \texttt{\$} is not a legal character in a math symbol (see
Section~\ref{dollardollar}, p.~\pageref{dollardollar}), so there will
never be a naming conflict between real symbols and temporary variables.

Unknown steps 1 and 2 are constructions of the two wffs used by the
modus ponens rule.  As you will see at the end of this section, the
Proof Assistant\index{Proof Assistant} can usually figure these steps
out by itself, and we will not have to worry about them.  Therefore from
here on we will display only the ``essential'' hypotheses, i.e.\ those
steps that correspond to traditional formal proofs\index{formal proof}.

\begin{verbatim}
MM-PA> show new_proof/lemmon
3 ?              $? |- $2
4 ?              $? |- ( $2 -> t = t )
5 3,4 mp         $a |- t = t
\end{verbatim}

Unknown steps 3 and 4 are the ones we must focus on.  They correspond to the
minor and major premises of the modus ponens rule.  We will assign them as
follows.  Notice that because of the step renumbering that takes place
after an assignment, it is advantageous to assign unknown steps in reverse
order, because earlier steps will not get renumbered.

\begin{verbatim}
MM-PA> assign 4 mp
To undo the assignment, DELETE STEP 8 and INITIALIZE, UNIFY
if needed.
3   min=?  $? |- $2
6     min=?  $? |- $4
7     maj=?  $? |- ( $4 -> ( $2 -> t = t ) )
\end{verbatim}

We are now going to describe an obscure feature that you will probably
never use but should be aware of.  The Metamath language allows empty
symbol sequences to be substituted for variables, but in most formal
systems this feature is never used.  One of the few examples where is it
used is the MIU-system\index{MIU-system} described in
Appendix~\ref{MIU}.  But such systems are rare, and by default this
feature is turned off in the Proof Assistant.  (It is always allowed for
{\tt verify proof}.)  Let us turn it on and see what
happens.\index{\texttt{set empty{\char`\_}substitution} command}

\begin{verbatim}
MM-PA> set empty_substitution on
Substitutions with empty symbol sequences is now allowed.
\end{verbatim}

With this feature enabled, more unifications will be
ambiguous\index{ambiguous unification}\index{unification!ambiguous} in
the middle of a proof, because
substitution\index{substitution!variable}\index{variable substitution}
of variables with empty symbol sequences will become an additional
possibility.  Let's see what happens when we make our next assignment.

\begin{verbatim}
MM-PA> assign 3 a2
There are 2 possible unifications.  Please select the correct
    one or Q if you want to UNIFY later.
Unify:  |- $6
 with:  |- ( $9 + 0 ) = $9
Unification #1 of 2 (weight = 7):
  Replace "$6" with "( + 0 ) ="
  Replace "$9" with ""
  Accept (A), reject (R), or quit (Q) <A>? r
\end{verbatim}

The first choice presented is the wrong one.  If we had selected it,
temporary variable \texttt{\$6} would have been assigned a truncated
wff, and temporary variable \texttt{\$9} would have been assigned an
empty sequence (which is not allowed in our system).  With this choice,
eventually we would reach a point where we would get stuck because
we would end up with steps impossible to prove.  (You may want to
try it.)  We typed \texttt{r} to reject the choice.

\begin{verbatim}
Unification #2 of 2 (weight = 21):
  Replace "$6" with "( $9 + 0 ) = $9"
  Accept (A), reject (R), or quit (Q) <A>? q
To undo the assignment, DELETE STEP 4 and INITIALIZE, UNIFY
if needed.
 7     min=?  $? |- $8
 8     maj=?  $? |- ( $8 -> ( $6 -> t = t ) )
\end{verbatim}

The second choice is correct, and normally we would type \texttt{a}
to accept it.  But instead we typed \texttt{q} to show what will happen:
it will leave the step with an unknown unification, which can be
seen as follows:

\begin{verbatim}
MM-PA> show new_proof/not_unified
 4   min    $a |- $6
        =a2  = |- ( $9 + 0 ) = $9
\end{verbatim}

Later we can unify this with the \texttt{unify}
\texttt{all/interactive} command.

The important point to remember is that occasionally you will be
presented with several unification choices while entering a proof, when
the program determines that there is not enough information yet to make
an unambiguous choice automatically (and this can happen even with
\texttt{set empty{\char`\_}substitution} turned off).  Usually it is
obvious by inspection which choice is correct, since incorrect ones will
tend to be meaningless fragments of wffs.  In addition, the correct
choice will usually be the first one presented, unlike our example
above.

Enough of this digression.  Let us go back to the default setting.

\begin{verbatim}
MM-PA> set empty_substitution off
The ability to substitute empty expressions for variables
has been turned off.  Note that this may make the Proof
Assistant too restrictive in some cases.
\end{verbatim}

If we delete the proof, start over, and get to the point where
we digressed above, there will no longer be an ambiguous unification.

\begin{verbatim}
MM-PA> assign 3 a2
To undo the assignment, DELETE STEP 4 and INITIALIZE, UNIFY
if needed.
 7     min=?  $? |- $4
 8     maj=?  $? |- ( $4 -> ( ( $5 + 0 ) = $5 -> t = t ) )
\end{verbatim}

Let us look at our proof so far, and continue.

\begin{verbatim}
MM-PA> show new_proof/lemmon
 4 a2            $a |- ( $5 + 0 ) = $5
 7 ?             $? |- $4
 8 ?             $? |- ( $4 -> ( ( $5 + 0 ) = $5 -> t = t ) )
 9 7,8 mp        $a |- ( ( $5 + 0 ) = $5 -> t = t )
10 4,9 mp        $a |- t = t
MM-PA> assign 8 a1
To undo the assignment, DELETE STEP 11 and INITIALIZE, UNIFY
if needed.
 7     min=?  $? |- ( t + 0 ) = t
MM-PA> assign 7 a2
To undo the assignment, DELETE STEP 8 and INITIALIZE, UNIFY
if needed.
MM-PA> show new_proof/lemmon
 4 a2            $a |- ( t + 0 ) = t
 8 a2            $a |- ( t + 0 ) = t
12 a1            $a |- ( ( t + 0 ) = t -> ( ( t + 0 ) = t ->
                                                    t = t ) )
13 8,12 mp       $a |- ( ( t + 0 ) = t -> t = t )
14 4,13 mp       $a |- t = t
\end{verbatim}

Now all temporary variables and unknown steps have been eliminated from the
``essential'' part of the proof.  When this is achieved, the Proof
Assistant\index{Proof Assistant} can usually figure out the rest of the proof
automatically.  (Note that the \texttt{improve} command can occasionally be
useful for filling in essential steps as well, but it only tries to make use
of statements that introduce no new variables in their hypotheses, which is
not the case for \texttt{mp}. Also it will not try to improve steps containing
temporary variables.)  Let's look at the complete proof, then run
the \texttt{improve} command, then look at it again.

\begin{verbatim}
MM-PA> show new_proof/lemmon/all
 1 ?             $? wff ( t + 0 ) = t
 2 ?             $? wff t = t
 3 ?             $? term t
 4 3 a2          $a |- ( t + 0 ) = t
 5 ?             $? wff ( t + 0 ) = t
 6 ?             $? wff ( ( t + 0 ) = t -> t = t )
 7 ?             $? term t
 8 7 a2          $a |- ( t + 0 ) = t
 9 ?             $? term ( t + 0 )
10 ?             $? term t
11 ?             $? term t
12 9,10,11 a1    $a |- ( ( t + 0 ) = t -> ( ( t + 0 ) = t ->
                                                    t = t ) )
13 5,6,8,12 mp   $a |- ( ( t + 0 ) = t -> t = t )
14 1,2,4,13 mp   $a |- t = t
\end{verbatim}

\begin{verbatim}
MM-PA> improve all
A proof of length 1 was found for step 11.
A proof of length 1 was found for step 10.
A proof of length 3 was found for step 9.
A proof of length 1 was found for step 7.
A proof of length 9 was found for step 6.
A proof of length 5 was found for step 5.
A proof of length 1 was found for step 3.
A proof of length 3 was found for step 2.
A proof of length 5 was found for step 1.
Steps 1 and above have been renumbered.
CONGRATULATIONS!  The proof is complete.  Use SAVE
NEW_PROOF to save it.  Note:  The Proof Assistant does
not detect $d violations.  After saving the proof, you
should verify it with VERIFY PROOF.
\end{verbatim}

The \texttt{save new{\char`\_}proof} command\index{\texttt{save
new{\char`\_}proof} command} will save the proof in the database.  Here
we will just display it in a form that can be clipped out of a log file
and inserted manually into the database source file with a text
editor.\index{normal proof}\index{proof!normal}

\begin{verbatim}
MM-PA> show new_proof/normal
---------Clip out the proof below this line:
      tt tze tpl tt weq tt tt weq tt a2 tt tze tpl tt weq
      tt tze tpl tt weq tt tt weq wim tt a2 tt tze tpl tt
      tt a1 mp mp $.
---------The proof of 'th1' to clip out ends above this line.
\end{verbatim}

There is another proof format called ``compressed''\index{compressed
proof}\index{proof!compressed} that you will see in databases.  It is
not important to understand how it is encoded but only to recognize it
when you see it.  Its only purpose is to reduce storage requirements for
large proofs.  A compressed proof can always be converted to a normal
one and vice-versa, and the Metamath \texttt{show proof}
commands\index{\texttt{show proof} command} work equally well with
compressed proofs.  The compressed proof format is described in
Appendix~\ref{compressed}.

\begin{verbatim}
MM-PA> show new_proof/compressed
---------Clip out the proof below this line:
      ( tze tpl weq a2 wim a1 mp ) ABCZADZAADZAEZJJKFLIA
      AGHH $.
---------The proof of 'th1' to clip out ends above this line.
\end{verbatim}

Now we will exit the Proof Assistant.  Since we made changes to the proof,
it will warn us that we have not saved it.  In this case, we don't care.

\begin{verbatim}
MM-PA> exit
Warning:  You have not saved changes to the proof.
Do you want to EXIT anyway (Y, N) <N>? y
Exiting the Proof Assistant.
Type EXIT again to exit Metamath.
\end{verbatim}

The Proof Assistant\index{Proof Assistant} has several other commands
that can help you while creating proofs.  See Section~\ref{pfcommands}
for a list of them.

A command that is often useful is \texttt{minimize{\char`\_}with
*/brief}, which tries to shorten the proof.  It can make the process
more efficient by letting you write a somewhat ``sloppy'' proof then
clean up some of the fine details of optimization for you (although it
can't perform miracles such as restructuring the overall proof).

\section{A Note About Editing a Data\-base File}

Once your source file contains proofs, there are some restrictions on
how you can edit it so that the proofs remain valid.  Pay particular
attention to these rules, since otherwise you can lose a lot of work.
It is a good idea to periodically verify all proofs with \texttt{verify
proof *} to ensure their integrity.

If your file contains only normal (as opposed to compressed) proofs, the
main rule is that you may not change the order of the mandatory
hypotheses\index{mandatory hypothesis} of any statement referenced in a
later proof.  For example, if you swap the order of the major and minor
premise in the modus ponens rule, all proofs making use of that rule
will become incorrect.  The \texttt{show statement}
command\index{\texttt{show statement} command} will show you the
mandatory hypotheses of a statement and their order.

If a statement has a compressed proof, you also must not change the
order of {\em its} mandatory hypotheses.  The compressed proof format
makes use of this information as part of the compression technique.
Note that swapping the names of two variables in a theorem will change
the order of its mandatory hypotheses.

The safest way to edit a statement, say \texttt{mytheorem}, is to
duplicate it then rename the original to \texttt{mytheoremOLD}
throughout the database.  Once the edited version is re-proved, all
statements referencing \texttt{mytheoremOLD} can be updated in the Proof
Assistant using \texttt{minimize{\char`\_}with
mytheorem
/allow{\char`\_}growth}.\index{\texttt{minimize{\char`\_}with} command}
% 3/10/07 Note: line-breaking the above results in duplicate index entries

\chapter{Abstract Mathematics Revealed}\label{fol}

\section{Logic and Set Theory}\label{logicandsettheory}

\begin{quote}
  {\em Set theory can be viewed as a form of exact theology.}
  \flushright\sc  Rudy Rucker\footnote{\cite{Barrow}, p.~31.}\\
\end{quote}\index{Rucker, Rudy}

Despite its seeming complexity, all of standard mathematics, no matter how
deep or abstract, can amazingly enough be derived from a relatively small set
of axioms\index{axiom} or first principles. The development of these axioms is
among the most impressive and important accomplishments of mathematics in the
20th century. Ultimately, these axioms can be broken down into a set of rules
for manipulating symbols that any technically oriented person can follow.

We will not spend much time trying to convey a deep, higher-level
understanding of the meaning of the axioms. This kind of understanding
requires some mathematical sophistication as well as an understanding of the
philosophy underlying the foundations of mathematics and typically develops
over time as you work with mathematics.  Our goal, instead, is to give you the
immediate ability to follow how theorems\index{theorem} are derived from the
axioms and from other theorems.  This will be similar to learning the syntax
of a computer language, which lets you follow the details in a program but
does not necessarily give you the ability to write non-trivial programs on
your own, an ability that comes with practice. For now don't be alarmed by
abstract-sounding names of the axioms; just focus on the rules for
manipulating the symbols, which follow the simple conventions of the
Metamath\index{Metamath} language.

The axioms that underlie all of standard mathematics consist of axioms of logic
and axioms of set theory. The axioms of logic are divided into two
subcategories, propositional calculus\index{propositional calculus} (sometimes
called sentential logic\index{sentential logic}) and predicate calculus
(sometimes called first-order logic\index{first-order logic}\index{quantifier
theory}\index{predicate calculus} or quantifier theory).  Propositional
calculus is a prerequisite for predicate calculus, and predicate calculus is a
prerequisite for set theory.  The version of set theory most commonly used is
Zermelo--Fraenkel set theory\index{Zermelo--Fraenkel set theory}\index{set theory}
with the axiom of choice,
often abbreviated as ZFC\index{ZFC}.

Here in a nutshell is what the axioms are all about in an informal way. The
connection between this description and symbols we will show you won't be
immediately apparent and in principle needn't ever be.  Our description just
tries to summarize what mathematicians think about when they work with the
axioms.

Logic is a set of rules that allow us determine truths given other truths.
Put another way,
logic is more or less the translation of what we would consider common sense
into a rigorous set of axioms.\index{axioms of logic}  Suppose $\varphi$,
$\psi$, and $\chi$ (the Greek letters phi, psi, and chi) represent statements
that are either true or false, and $x$ is a variable\index{variable!in predicate
calculus} ranging over some group of mathematical objects (sets, integers,
real numbers, etc.). In mathematics, a ``statement'' really means a formula,
and $\psi$ could be for example ``$x = 2$.''
Propositional calculus\index{propositional calculus}
allows us to use variables that are either true or false
and make deductions such as
``if $\varphi$ implies $\psi$ and $\psi$ implies $\chi$, then $\varphi$
implies $\chi$.''
Predicate calculus\index{predicate calculus}
extends propositional calculus by also allowing us
to discuss statements about objects (not just true and false values), including
statements about ``all'' or ``at least one'' object.
For example, predicate calculus allows to say,
``if $\varphi$ is true for all $x$, then $\varphi$ is true for some $x$.''
The logic used in \texttt{set.mm} is standard classical logic
(as opposed to other logic systems like intuitionistic logic).

Set theory\index{set theory} has to do with the manipulation of objects and
collections of objects, specifically the abstract, imaginary objects that
mathematics deals with, such as numbers. Everything that is claimed to exist
in mathematics is considered to be a set.  A set called the empty
set\index{empty set} contains nothing.  We represent the empty set by
$\varnothing$.  Many sets can be built up from the empty set.  There is a set
represented by $\{\varnothing\}$ that contains the empty set, another set
represented by $\{\varnothing,\{\varnothing\}\}$ that contains this set as
well as the empty set, another set represented by $\{\{\varnothing\}\}$ that
contains just the set that contains the empty set, and so on ad infinitum. All
mathematical objects, no matter how complex, are defined as being identical to
certain sets: the integer\index{integer} 0 is defined as the empty set, the
integer 1 is defined as $\{\varnothing\}$, the integer 2 is defined as
$\{\varnothing,\{\varnothing\}\}$.  (How these definitions were chosen doesn't
matter now, but the idea behind it is that these sets have the properties we
expect of integers once suitable operations are defined.)  Mathematical
operations, such as addition, are defined in terms of operations on
sets---their union\index{set union}, intersection\index{set intersection}, and
so on---operations you may have used in elementary school when you worked
with groups of apples and oranges.

With a leap of faith, the axioms also postulate the existence of infinite
sets\index{infinite set}, such as the set of all non-negative integers ($0, 1,
2,\ldots$, also called ``natural numbers''\index{natural number}).  This set
can't be represented with the brace notation\index{brace notation} we just
showed you, but requires a more complicated notation called ``class
abstraction.''\index{class abstraction}\index{abstraction class}  For
example, the infinite set $\{ x |
\mbox{``$x$ is a natural number''} \} $ means the ``set of all objects $x$
such that $x$ is a natural number'' i.e.\ the set of natural numbers; here,
``$x$ is a natural number'' is a rather complicated formula when broken down
into the primitive symbols.\label{expandom}\footnote{The statement ``$x$ is a
natural number'' is formally expressed as ``$x \in \omega$,'' where $\in$
(stylized epsilon) means ``is in'' or ``is an element of'' and $\omega$
(omega) means ``the set of natural numbers.''  When ``$x\in\omega$'' is
completely expanded in terms of the primitive symbols of set theory, the
result is  $\lnot$ $($ $\lnot$ $($ $\forall$ $z$ $($ $\lnot$ $\forall$ $w$ $($
$z$ $\in$ $w$ $\rightarrow$ $\lnot$ $w$ $\in$ $x$ $)$ $\rightarrow$ $z$ $\in$
$x$ $)$ $\rightarrow$ $($ $\forall$ $z$ $($ $\lnot$ $($ $\forall$ $w$ $($ $w$
$\in$ $z$ $\rightarrow$ $w$ $\in$ $x$ $)$ $\rightarrow$ $\forall$ $w$ $\lnot$
$w$ $\in$ $z$ $)$ $\rightarrow$ $\lnot$ $\forall$ $w$ $($ $w$ $\in$ $z$
$\rightarrow$ $\lnot$ $\forall$ $v$ $($ $v$ $\in$ $z$ $\rightarrow$ $\lnot$
$v$ $\in$ $w$ $)$ $)$ $)$ $\rightarrow$ $\lnot$ $\forall$ $z$ $\forall$ $w$
$($ $\lnot$ $($ $z$ $\in$ $x$ $\rightarrow$ $\lnot$ $w$ $\in$ $x$ $)$
$\rightarrow$ $($ $\lnot$ $z$ $\in$ $w$ $\rightarrow$ $($ $\lnot$ $z$ $=$ $w$
$\rightarrow$ $w$ $\in$ $z$ $)$ $)$ $)$ $)$ $)$ $\rightarrow$ $\lnot$
$\forall$ $y$ $($ $\lnot$ $($ $\lnot$ $($ $\forall$ $z$ $($ $\lnot$ $\forall$
$w$ $($ $z$ $\in$ $w$ $\rightarrow$ $\lnot$ $w$ $\in$ $y$ $)$ $\rightarrow$
$z$ $\in$ $y$ $)$ $\rightarrow$ $($ $\forall$ $z$ $($ $\lnot$ $($ $\forall$
$w$ $($ $w$ $\in$ $z$ $\rightarrow$ $w$ $\in$ $y$ $)$ $\rightarrow$ $\forall$
$w$ $\lnot$ $w$ $\in$ $z$ $)$ $\rightarrow$ $\lnot$ $\forall$ $w$ $($ $w$
$\in$ $z$ $\rightarrow$ $\lnot$ $\forall$ $v$ $($ $v$ $\in$ $z$ $\rightarrow$
$\lnot$ $v$ $\in$ $w$ $)$ $)$ $)$ $\rightarrow$ $\lnot$ $\forall$ $z$
$\forall$ $w$ $($ $\lnot$ $($ $z$ $\in$ $y$ $\rightarrow$ $\lnot$ $w$ $\in$
$y$ $)$ $\rightarrow$ $($ $\lnot$ $z$ $\in$ $w$ $\rightarrow$ $($ $\lnot$ $z$
$=$ $w$ $\rightarrow$ $w$ $\in$ $z$ $)$ $)$ $)$ $)$ $\rightarrow$ $($
$\forall$ $z$ $\lnot$ $z$ $\in$ $y$ $\rightarrow$ $\lnot$ $\forall$ $w$ $($
$\lnot$ $($ $w$ $\in$ $y$ $\rightarrow$ $\lnot$ $\forall$ $z$ $($ $w$ $\in$
$z$ $\rightarrow$ $\lnot$ $z$ $\in$ $y$ $)$ $)$ $\rightarrow$ $\lnot$ $($
$\lnot$ $\forall$ $z$ $($ $w$ $\in$ $z$ $\rightarrow$ $\lnot$ $z$ $\in$ $y$
$)$ $\rightarrow$ $w$ $\in$ $y$ $)$ $)$ $)$ $)$ $\rightarrow$ $x$ $\in$ $y$
$)$ $)$ $)$. Section~\ref{hierarchy} shows the hierarchy of definitions that
leads up to this expression.}\index{stylized epsilon ($\in$)}\index{omega
($\omega$)}  Actually, the primitive symbols don't even include the brace
notation.  The brace notation is a high-level definition, which you can find in
Section~\ref{hierarchy}.

Interestingly, the arithmetic of integers\index{integer} and
rationals\index{rational number} can be developed without appealing to the
existence of an infinite set, whereas the arithmetic of real
numbers\index{real number} requires it.

Each variable\index{variable!in set theory} in the axioms of set theory
represents an arbitrary set, and the axioms specify the legal kinds of things
you can do with these variables at a very primitive level.

Now, you may think that numbers and arithmetic are a lot more intuitive and
fundamental than sets and therefore should be the foundation of mathematics.
What is really the case is that you've dealt with numbers all your life and
are comfortable with a few rules for manipulating them such as addition and
multiplication.  Those rules only cover a small portion of what can be done
with numbers and only a very tiny fraction of the rest of mathematics.  If you
look at any elementary book on number theory, you will quickly become lost if
these are the only rules that you know.  Even though such books may present a
list of ``axioms''\index{axiom} for arithmetic, the ability to use the axioms
and to understand proofs of theorems\index{theorem} (facts) about numbers
requires an implicit mathematical talent that frustrates many people
from studying abstract mathematics.  The kind of mathematics that most people
know limits them to the practical, everyday usage of blindly manipulating
numbers and formulas, without any understanding of why those rules are correct
nor any ability to go any further.  For example, do you know why multiplying
two negative numbers yields a positive number?  Starting with set theory, you
will also start off blindly manipulating symbols according to the rules we give
you, but with the advantage that these rules will allow you, in principle, to
access {\em all} of mathematics, not just a tiny part of it.

Of course, concrete examples are often helpful in the learning process. For
example, you can verify that $2\cdot 3=3 \cdot 2$ by actually grouping
objects and can easily ``see'' how it generalizes to $x\cdot y = y\cdot x$,
even though you might not be able to rigorously prove it.  Similarly, in set
theory it can be helpful to understand how the axioms of set theory apply to
(and are correct for) small finite collections of objects.  You should be aware
that in set theory intuition can be misleading for infinite collections, and
rigorous proofs become more important.  For example, while $x\cdot y = y\cdot
x$ is correct for finite ordinals (which are the natural numbers), it is not
usually true for infinite ordinals.

\section{The Axioms for All of Mathematics}

In this section\index{axioms for mathematics}, we will show you the axioms
for all of standard mathematics (i.e.\ logic and set theory) as they are
traditionally presented.  The traditional presentation is useful for someone
with the mathematical experience needed to correctly manipulate high-level
abstract concepts.  For someone without this talent, knowing how to actually
make use of these axioms can be difficult.  The purpose of this section is to
allow you to see how the version of the axioms used in the standard
Metamath\index{Metamath} database \texttt{set.mm}\index{set
theory database (\texttt{set.mm})} relates to  the typical version
in textbooks, and also to give you an informal feel for them.

\subsection{Propositional Calculus}

Propositional calculus\index{propositional calculus} concerns itself with
statements that can be interpreted as either true or false.  Some examples of
statements (outside of mathematics) that are either true or false are ``It is
raining today'' and ``The United States has a female president.'' In
mathematics, as we mentioned, statements are really formulas.

In propositional calculus, we don't care what the statements are.  We also
treat a logical combination of statements, such as ``It is raining today and
the United States has a female president,'' no differently from a single
statement.  Statements and their combinations are called well-formed formulas
(wffs)\index{well-formed formula (wff)}.  We define wffs only in terms of
other wffs and don't define what a ``starting'' wff is.  As is common practice
in the literature, we use Greek letters to represent wffs.

Specifically, suppose $\varphi$ and $\psi$ are wffs.  Then the combinations
$\varphi\rightarrow\psi$ (``$\varphi$ implies $\psi$,'' also read ``if
$\varphi$ then $\psi$'')\index{implication ($\rightarrow$)} and $\lnot\varphi$
(``not $\varphi$'')\index{negation ($\lnot$)} are also wffs.

The three axioms of propositional calculus\index{axioms of propositional
calculus} are all wffs of the following form:\footnote{A remarkable result of
C.~A.~Meredith\index{Meredith, C. A.} squeezes these three axioms into the
single axiom $((((\varphi\rightarrow \psi)\rightarrow(\neg \chi\rightarrow\neg
\theta))\rightarrow \chi )\rightarrow \tau)\rightarrow((\tau\rightarrow
\varphi)\rightarrow(\theta\rightarrow \varphi))$ \cite{CAMeredith},
which is believed to be the shortest possible.}
\begin{center}
     $\varphi\rightarrow(\psi\rightarrow \varphi)$\\

     $(\varphi\rightarrow (\psi\rightarrow \chi))\rightarrow
((\varphi\rightarrow  \psi)\rightarrow (\varphi\rightarrow \chi))$\\

     $(\neg \varphi\rightarrow \neg\psi)\rightarrow (\psi\rightarrow
\varphi)$
\end{center}

These three axioms are widely used.
They are attributed to Jan {\L}ukasiewicz
(pronounced woo-kah-SHAY-vitch) and was popularized by Alonzo Church,
who called it system P2. (Thanks to Ted Ulrich for this information.)

There are an infinite number of axioms, one for each possible
wff\index{well-formed formula (wff)} of the above form.  (For this reason,
axioms such as the above are often called ``axiom schemes.''\index{axiom
scheme})  Each Greek letter in the axioms may be substituted with a more
complex wff to result in another axiom.  For example, substituting
$\neg(\varphi\rightarrow\chi)$ for $\varphi$ in the first axiom yields
$\neg(\varphi\rightarrow\chi)\rightarrow(\psi\rightarrow
\neg(\varphi\rightarrow\chi))$, which is still an axiom.

To deduce new true statements (theorems\index{theorem}) from the axioms, a
rule\index{rule} called ``modus ponens''\index{modus ponens} is used.  This
rule states that if the wff $\varphi$ is an axiom or a theorem, and the wff
$\varphi\rightarrow\psi$ is an axiom or a theorem, then the wff $\psi$ is also
a theorem\index{theorem}.

As a non-mathematical example of modus ponens, suppose we have proved (or
taken as an axiom) ``Bob is a man'' and separately have proved (or taken as
an axiom) ``If Bob is a man, then Bob is a human.''  Using the rule of modus
ponens, we can logically deduce, ``Bob is a human.''

From Metamath's\index{Metamath} point of view, the axioms and the rule of
modus ponens just define a mechanical means for deducing new true statements
from existing true statements, and that is the complete content of
propositional calculus as far as Metamath is concerned.  You can read a logic
textbook to gain a better understanding of their meaning, or you can just let
their meaning slowly become apparent to you after you use them for a while.

It is actually rather easy to check to see if a formula is a theorem of
propositional calculus.  Theorems of propositional calculus are also called
``tautologies.''\index{tautology}  The technique to check whether a formula is
a tautology is called the ``truth table method,''\index{truth table} and it
works like this.  A wff $\varphi\rightarrow\psi$ is false whenever $\varphi$ is true
and $\psi$ is false.  Otherwise it is true.  A wff $\lnot\varphi$ is false
whenever $\varphi$ is true and false otherwise. To verify a tautology such as
$\varphi\rightarrow(\psi\rightarrow \varphi)$, you break it down into sub-wffs and
construct a truth table that accounts for all possible combinations of true
and false assigned to the wff metavariables:
\begin{center}\begin{tabular}{|c|c|c|c|}\hline
\mbox{$\varphi$} & \mbox{$\psi$} & \mbox{$\psi\rightarrow\varphi$}
    & \mbox{$\varphi\rightarrow(\psi\rightarrow \varphi)$} \\ \hline \hline
              T   &  T    &      T       &        T    \\ \hline
              T   &  F    &      T       &        T    \\ \hline
              F   &  T    &      F       &        T    \\ \hline
              F   &  F    &      T       &        T    \\ \hline
\end{tabular}\end{center}
If all entries in the last column are true, the formula is a tautology.

Now, the truth table method doesn't tell you how to prove the tautology from
the axioms, but only that a proof exists.  Finding an actual proof (especially
one that is short and elegant) can be challenging.  Methods do exist for
automatically generating proofs in propositional calculus, but the proofs that
result can sometimes be very long.  In the Metamath \texttt{set.mm}\index{set
theory database (\texttt{set.mm})} database, most
or all proofs were created manually.

Section \ref{metadefprop} discusses various definitions
that make propositional calculus easier to use.
For example, we define:

\begin{itemize}
\item $\varphi \vee \psi$
  is true if either $\varphi$ or $\psi$ (or both) are true
  (this is disjunction\index{disjunction ($\vee$)}
  aka logical {\sc or}\index{logical {\sc or} ($\vee$)}).

\item $\varphi \wedge \psi$
  is true if both $\varphi$ and $\psi$ are true
  (this is conjunction\index{conjunction ($\wedge$)}
  aka logical {\sc and}\index{logical {\sc and} ($\wedge$)}).

\item $\varphi \leftrightarrow \psi$
  is true if $\varphi$ and $\psi$ have the same value, that is,
  they are both true or both false
  (this is the biconditional\index{biconditional ($\leftrightarrow$)}).
\end{itemize}

\subsection{Predicate Calculus}

Predicate calculus\index{predicate calculus} introduces the concept of
``individual variables,''\index{variable!in predicate calculus}\index{individual
variable} which
we will usually just call ``variables.''
These variables can represent something other than true or false (wffs),
and will always represent sets when we get to set theory.  There are also
three new symbols $\forall$\index{universal quantifier ($\forall$)},
$=$\index{equality ($=$)}, and $\in$\index{stylized epsilon ($\in$)},
read ``for all,'' ``equals,'' and ``is an element of''
respectively.  We will represent variables with the letters $x$, $y$, $z$, and
$w$, as is common practice in the literature.
For example, $\forall x \varphi$ means ``for all possible values of
$x$, $\varphi$ is true.''

In predicate calculus, we extend the definition of a wff\index{well-formed
formula (wff)}.  If $\varphi$ is a wff and $x$ and $y$ are variables, then
$\forall x \, \varphi$, $x=y$, and $x\in y$ are wffs. Note that these three new
types of wffs can be considered ``starting'' wffs from which we can build
other wffs with $\rightarrow$ and $\neg$ .  The concept of a starting wff was
absent in propositional calculus.  But starting wff or not, all we are really
concerned with is whether our wffs are correctly constructed according to
these mechanical rules.

A quick aside:
To prevent confusion, it might be best at this point to think of the variables
of Metamath\index{Metamath} as ``metavariables,''\index{metavariable} because
they are not quite the same as the variables we are introducing here.  A
(meta)variable in Metamath can be a wff or an individual variable, as well
as many other things; in general, it represents a kind of place holder for an
unspecified sequence of math symbols\index{math symbol}.

Unlike propositional calculus, no decision procedure\index{decision procedure}
analogous to the truth table method exists (nor theoretically can exist) that
will definitely determine whether a formula is a theorem of predicate
calculus.  Much of the work in the field of automated theorem
proving\index{automated theorem proving} has been dedicated to coming up with
clever heuristics for proving theorems of predicate calculus, but they can
never be guaranteed to work always.

Section \ref{metadefpred} discusses various definitions
that make predicate calculus easier to use.
For example, we define
$\exists x \varphi$ to mean
``there exists at least one possible value of $x$ where $\varphi$ is true.''

We now turn to looking at how predicate calculus can be formally
represented.

\subsubsection{Common Axioms}

There is a new rule of inference in predicate calculus:  if $\varphi$ is
an axiom or a theorem, then $\forall x \,\varphi$ is also a
theorem\index{theorem}.  This is called the rule of
``generalization.''\index{rule of generalization}
This is easily represented in Metamath.

In standard texts of logic, there are often two axioms of predicate
calculus\index{axioms of predicate calculus}:
\begin{center}
  $\forall x \,\varphi ( x ) \rightarrow \varphi ( y )$,
      where ``$y$ is properly substituted for $x$.''\\
  $\forall x ( \varphi \rightarrow \psi )\rightarrow ( \varphi \rightarrow
    \forall x\, \psi )$,
    where ``$x$ is not free in $\varphi$.''
\end{center}

Now at first glance, this seems simple:  just two axioms.  However,
conditional clauses are attached to each axiom describing requirements that
may seem puzzling to you.  In addition, the first axiom puts a variable symbol
in parentheses after each wff, seemingly violating our definition of a
wff\index{well-formed formula (wff)}; this is just an informal way of
referring to some arbitrary variable that may occur in the wff.  The
conditional clauses do, of course, have a precise meaning, but as it turns out
the precise meaning is somewhat complicated and awkward to formalize in a
way that a computer can handle easily.  Unlike propositional calculus, a
certain amount of mathematical sophistication and practice is needed to be
able to easily grasp and manipulate these concepts correctly.

Predicate calculus may be presented with or without axioms for
equality\index{axioms of equality}\index{equality ($=$)}. We will require the
axioms of equality as a prerequisite for the version of set theory we will
use.  The axioms for equality, when included, are often represented using these
two axioms:
\begin{center}
$x=x$\\ \ \\
$x=y\rightarrow (\varphi(x,x)\rightarrow\varphi(x,y))$ where ``$\varphi(x,y)$
   arises from $\varphi(x,x)$ by replacing some, but not necessarily all,
   free\index{free variable}
   occurrences of $x$ by $y$,\\ provided that $y$ is free for $x$
   in $\varphi(x,x)$.'' \end{center}
% (Mendelson p. 95)
The first equality axiom is simple, but again,
the condition on the second one is
somewhat awkward to implement on a computer.

\subsubsection{Tarski System S2}

Of course, we are not the first to notice the complications of these
predicate calculus axioms when being rigorous.

Well-known logician Alfred Tarski published in 1965
a system he called system S2\cite[p.~77]{Tarski1965}.
Tarski's system is \textit{exactly equivalent} to the traditional textbook
formalization, but (by clever use of equality axioms) it eliminates the
latter's primitive notions of ``proper substitution'' and ``free variable,''
replacing them with direct substitution and the notion of a variable
not occurring in a formula (which we express with distinct variable
constraints).

In advocating his system, Tarski wrote, ``The relatively complicated
character of [free variables and proper substitution] is a source
of certain inconveniences of both practical and theoretical nature;
this is clearly experienced both in teaching an elementary course of
mathematical logic and in formalizing the syntax of predicate logic for
some theoretical purposes''\cite[p.~61]{Tarski1965}\index{Tarski, Alfred}.

\subsubsection{Developing a Metamath Representation}

The standard textbook axioms of predicate calculus are somewhat
cumbersome to implement on a computer because of the complex notions of
``free variable''\index{free variable} and ``proper
substitution.''\index{proper substitution}\index{substitution!proper}
While it is possible to use the Metamath\index{Metamath} language to
implement these concepts, we have chosen not to implement them
as primitive constructs in the
\texttt{set.mm} set theory database.  Instead, we have eliminated them
within the axioms
by carefully crafting the axioms so as to avoid them,
building on Tarski's system S2.  This makes it
easy for a beginner to follow the steps in a proof without knowing any
advanced concepts other than the simple concept of
replacing\index{substitution!variable}\index{variable substitution}
variables with expressions.

In order to develop the concepts of free variable and proper
substitution from the axioms, we use an additional
Metamath statement type called ``disjoint variable
restriction''\index{disjoint variables} that we have not encountered
before.  In the context of the axioms, the statement \texttt{\$d} $ x\,
y$\index{\texttt{\$d} statement} simply means that $x$ and $y$ must be
distinct\index{distinct variables}, i.e.\ they may not be simultaneously
substituted\index{substitution!variable}\index{variable substitution}
with the same variable.  The statement \texttt{\$d} $ x\, \varphi$ means
variable $x$ must not occur in wff $\varphi$.  For the precise
definition of \texttt{\$d}, see Section~\ref{dollard}.

\subsubsection{Metamath representation}

The Metamath axiom system for predicate calculus
defined in set.mm uses Tarski's system S2.
As noted above, this has a different representation
than the traditional textbook formalization,
but it is \textit{exactly equivalent} to the textbook formalization,
and it is \textit{much} easier to work with.
This is reproduced as system S3 in Section 6 of
Megill's formalization \cite{Megill}\index{Megill, Norman}.

There is one exception, Tarski's axiom of existence,
which we label as axiom ax-6.
In the case of ax-6, Tarski's version is weaker because it includes a
distinct variable proviso. If we wish, we can also weaken our version
in this way and still have a metalogically complete system. Theorem
ax6 shows this by deriving, in the presence of the other axioms, our
ax-6 from Tarski's weaker version ax6v. However, we chose the stronger
version for our system because it is simpler to state and easier to use.

Tarski's system was designed for proving specific theorems rather than
more general theorem schemes. However, theorem schemes are much more
efficient than specific theorems for building a body of mathematical
knowledge, since they can be reused with different instances as
needed. While Tarski does derive some theorem schemes from his axioms,
their proofs require concepts that are ``outside'' of the system, such as
induction on formula length. The verification of such proofs is difficult
to automate in a proof verifier. (Specifically, Tarski treats the formulas
of his system as set-theoretical objects. In order to verify the proofs
of his theorem schemes, a proof verifier would need a significant amount
of set theory built into it.)

The Metamath axiom system for predicate calculus extends
Tarski's system to eliminate this difficulty. The additional
``auxilliary'' axiom
schemes (as we will call them in this section; see below) endow Tarski's
system with a nice property we call
metalogical completeness \cite[Remark 9.6]{Megill}\index{Megill, Norman}.
As a result, we can prove any theorem scheme
expressable in the ``simple metalogic'' of Tarski's system by using
only Metamath's direct substitution rule applied to the axiom system
(and no other metalogical or set-theoretical notions ``outside'' of the
system). Simple metalogic consists of schemes containing wff metavariables
(with no arguments) and/or set (also called ``individual'') metavariables,
accompanied by optional provisos each stating that two specified set
metavariables must be distinct or that a specified set metavariable may
not occur in a specified wff metavariable. Metamath's logic and set theory
axiom and rule schemes are all examples of simple metalogic. The schemes
of traditional predicate calculus with equality are examples which are
not simple metalogic, because they use wff metavariables with arguments
and have ``free for'' and ``not free in'' side conditions.

A rigorous justification for this system, using an older but
exactly equivalent set of axioms, can be
found in \cite{Megill}\index{Megill, Norman}.

This allows us to
take a different approach in the Metamath\index{Metamath} database
\texttt{set.mm}\index{set theory database (\texttt{set.mm})}.  We do not
directly use the primitive notions of ``free variable''\index{free variable}
and ``proper substitution''\index{proper
substitution}\index{substitution!proper} at all as primitive constructs.
Instead, we use a set
of axioms that are almost as simple to manipulate as those of
propositional calculus.  Our axiom system avoids complex primitive
notions by effectively embedding the complexity into the axioms
themselves.  As a result, we will end up with a larger number of axioms,
but they are ideally suited for a computer language such as Metamath.
(Section~\ref{metaaxioms} shows these axioms.)

We will not elaborate further
on the ``free variable'' and ``proper substitution''
concepts here.  You may consult
\cite[ch.\ 3--4]{Hamilton}\index{Hamilton, Alan G.} (as well as
many other books) for a precise explanation
of these concepts.  If you intend to do serious mathematical work, it is wise
to become familiar with the traditional textbook approach; even though the
concepts embedded in their axioms require a higher level of sophistication,
they can be more practical to deal with on an everyday, informal basis.  Even
if you are just developing Metamath proofs, familiarity with the traditional
approach can help you arrive at a proof outline much faster, which you can
then convert to the detail required by Metamath.

We do develop proper substitution rules later on, but in set.mm
they are defined as derived constructs; they are not primitives.

You should also note that our system of predicate calculus is specifically
tailored for set theory; thus there are only two specific predicates $=$ and
$\in$ and no functions\index{function!in predicate calculus}
or constants\index{constant!in predicate calculus} unlike more general systems.
We later add these.

\subsection{Set Theory}

Traditional Zermelo--Fraenkel set theory\index{Zermelo--Fraenkel set
theory}\index{set theory} with the Axiom of Choice
has 10 axioms, which can be expressed in the
language of predicate calculus.  In this section, we will list only the
names and brief English descriptions of these axioms, since we will give
you the precise formulas used by the Metamath\index{Metamath} set theory
database \texttt{set.mm} later on.

In the descriptions of the axioms, we assume that $x$, $y$, $z$, $w$, and $v$
represent sets.  These are the same as the variables\index{variable!in set
theory} in our predicate calculus system above, except that now we informally
think of the variables as ranging over sets.  Note that the terms
``object,''\index{object} ``set,''\index{set} ``element,''\index{element}
``collection,''\index{collection} and ``family''\index{family} are synonymous,
as are ``is an element of,'' ``is a member of,''\index{member} ``is contained
in,'' and ``belongs to.''  The different terms are used for convenience; for
example, ``a collection of sets'' is less confusing than ``a set of sets.''
A set $x$ is said to be a ``subset''\index{subset} of $y$ if every element of
$x$ is also an element of $y$; we also say $x$ is ``included in''
$y$.

The axioms are very general and apply to almost any conceivable mathematical
object, and this level of abstraction can be overwhelming at first.  To gain an
intuitive feel, it can be helpful to draw a picture illustrating the concept;
for example, a circle containing dots could represent a collection of sets,
and a smaller circle drawn inside the circle could represent a subset.
Overlapping circles can illustrate intersection and union.  Circles that
illustrate the concepts of set theory are frequently used in elementary
textbooks and are called Venn diagrams\index{Venn diagram}.\index{axioms of
set theory}

1. Axiom of Extensionality:  Two sets are identical if they contain the same
   elements.\index{Axiom of Extensionality}

2. Axiom of Pairing:  The set $\{ x , y \}$ exists.\index{Axiom of Pairing}

3. Axiom of Power Sets:  The power set of a set (the collection of all of
   its subsets) exists.  For example, the power set of $\{x,y\}$ is
   $\{\varnothing,\{x\},\{y\},\{x,y\}\}$ and it exists.\index{Axiom
of Power Sets}

4. Axiom of the Null Set:  The empty set $\varnothing$ exists.\index{Axiom of
the Null Set}

5. Axiom of Union:  The union of a set (the set containing the elements of
   its members) exists.  For example, the union of $\{\{x,y\},\{z\}\}$ is
 $\{x,y,z\}$ and
   it exists.\index{Axiom of Union}

6. Axiom of Regularity:  Roughly, no set can contain itself, nor can there
   be membership ``loops,'' such as a set being an
   element of one of its members.\index{Axiom of Regularity}

7. Axiom of Infinity:  An infinite set exists.  An example of an infinite
   set is the set of all
   integers.\index{Axiom of Infinity}

8. Axiom of Separation:  The set exists that is obtained by restricting $x$
   with some property.  For example, if the set of all integers exists,
   then the set of all even integers exists.\index{Axiom of Separation}

9. Axiom of Replacement:  The range of a function whose domain is restricted
   to the elements of a set $x$, is also a set.  For example, there
   is a function
   from integers (the function's domain) to their squares (its
   range).  If we
   restrict the domain to even integers, its range will become the set of
   squares of even integers, so this axiom asserts that the set of
    squares of even numbers exists.  Technical note:  In general, the
   ``function'' need not be a set but can be a proper class.
   \index{Axiom of Replacement}

10. Axiom of Choice:  Let $x$ be a set whose members are pairwise
  disjoint\index{disjoint sets} (i.e,
  whose members contain no elements in common).  Then there exists another
  set containing one element from each member of $x$.  For
  example, if $x$ is
  $\{\{y,z\},\{w,v\}\}$, where $y$, $z$, $w$, and $v$ are
  different sets, then a set such as $\{z,w\}$
  exists (but the axiom doesn't tell
  us which one).  (Actually the Axiom
  of Choice is redundant if the set $x$, as in this example, has a finite
  number of elements.)\index{Axiom of Choice}

The Axiom of Choice is usually considered an extension of ZF set theory rather
than a proper part of it.  It is sometimes considered philosophically
controversial because it specifies the existence of a set without specifying
what the set is. Constructive logics, including intuitionistic logic,
do not accept the axiom of choice.
Since there is some lingering controversy, we often prefer proofs that do
not use the axiom of choice (where there is a known alternative), and
in some cases we will use weaker axioms than the full axiom of choice.
That said, the axiom of choice is a powerful and widely-accepted tool,
so we do use it when needed.
ZF set theory that includes the Axiom of Choice is
called Zermelo--Fraenkel set theory with choice (ZFC\index{ZFC set theory}).

When expressed symbolically, the Axiom of Separation and the Axiom of
Replacement contain wff symbols and therefore each represent infinitely many
axioms, one for each possible wff. For this reason, they are often called
axiom schemes\index{axiom scheme}\index{well-formed formula (wff)}.

It turns out that the Axiom of the Null Set, the Axiom of Pairing, and the
Axiom of Separation can be derived from the other axioms and are therefore
unnecessary, although they tend to be included in standard texts for various
reasons (historical, philosophical, and possibly because some authors may not
know this).  In the Metamath\index{Metamath} set theory database, these
redundant axioms are derived from the other ones instead of truly
being considered axioms.
This is in keeping with our general goal of minimizing the number of
axioms we must depend on.

\subsection{Other Axioms}

Above we qualified the phrase "all of mathematics" with "essentially."
The main important missing piece is the ability to do category theory,
which requires huge sets (inaccessible cardinals) larger than those
postulated by the ZFC axioms. The Tarski--Grothendieck Axiom postulates
the existence of such sets.
Note that this is the same axiom used by Mizar for supporting
category theory.
The Tarski--Grothendieck axiom
can be viewed as a very strong replacement of the Axiom of Infinity,
the Axiom of Choice, and the Axiom of Power Sets.
The \texttt{set.mm} database includes this axiom; see the database
for details about it.
Again, we only use this axiom when we need to.
You are only likely to encounter or use this axiom if you are doing
category theory, since its use is highly specialized,
so we will not list the Tarsky-Grothendieck axiom
in the short list of axioms below.

Can there be even more axioms?
Of course.
G\"{o}del showed that no finite set of axioms or axiom schemes can completely
describe any consistent theory strong enough to include arithmetic.
But practically speaking, the ones above are the accepted foundation that
almost all mathematicians explicitly or implicitly base their work on.

\section{The Axioms in the Metamath Language}\label{metaaxioms}

Here we list the axioms as they appear in
\texttt{set.mm}\index{set theory database (\texttt{set.mm})} so you can
look them up there easily.  Incidentally, the \texttt{show statement
/tex} command\index{\texttt{show statement} command} was used to
typeset them.

%macros from show statement /tex
\newbox\mlinebox
\newbox\mtrialbox
\newbox\startprefix  % Prefix for first line of a formula
\newbox\contprefix  % Prefix for continuation line of a formula
\def\startm{  % Initialize formula line
  \setbox\mlinebox=\hbox{\unhcopy\startprefix}
}
\def\m#1{  % Add a symbol to the formula
  \setbox\mtrialbox=\hbox{\unhcopy\mlinebox $\,#1$}
  \ifdim\wd\mtrialbox>\hsize
    \box\mlinebox
    \setbox\mlinebox=\hbox{\unhcopy\contprefix $\,#1$}
  \else
    \setbox\mlinebox=\hbox{\unhbox\mtrialbox}
  \fi
}
\def\endm{  % Output the last line of a formula
  \box\mlinebox
}

% \SLASH for \ , \TOR for \/ (text OR), \TAND for /\ (text and)
% This embeds a following forced space to force the space.
\newcommand\SLASH{\char`\\~}
\newcommand\TOR{\char`\\/~}
\newcommand\TAND{/\char`\\~}
%
% Macro to output metamath raw text.
% This assumes \startprefix and \contprefix are set.
% NOTE: "\" is tricky to escape, use \SLASH, \TOR, and \TAND inside.
% Any use of "$ { ~ ^" must be escaped; ~ and ^ must be escaped specially.
% We escape { and } for consistency.
% For more about how this macro written, see:
% https://stackoverflow.com/questions/4073674/
% how-to-disable-indentation-in-particular-section-in-latex/4075706
% Use frenchspacing, or "e." will get an extra space after it.
\newlength\mystoreparindent
\newlength\mystorehangindent
\newenvironment{mmraw}{%
\setlength{\mystoreparindent}{\the\parindent}
\setlength{\mystorehangindent}{\the\hangindent}
\setlength{\parindent}{0pt} % TODO - we'll put in the \startprefix instead
\setlength{\hangindent}{\wd\the\contprefix}
\begin{flushleft}
\begin{frenchspacing}
\begin{tt}
{\unhcopy\startprefix}%
}{%
\end{tt}
\end{frenchspacing}
\end{flushleft}
\setlength{\parindent}{\mystoreparindent}
\setlength{\hangindent}{\mystorehangindent}
\vskip 1ex
}

\needspace{5\baselineskip}
\subsection{Propositional Calculus}\label{propcalc}\index{axioms of
propositional calculus}

\needspace{2\baselineskip}
Axiom of Simplification.\label{ax1}

\setbox\startprefix=\hbox{\tt \ \ ax-1\ \$a\ }
\setbox\contprefix=\hbox{\tt \ \ \ \ \ \ \ \ \ \ }
\startm
\m{\vdash}\m{(}\m{\varphi}\m{\rightarrow}\m{(}\m{\psi}\m{\rightarrow}\m{\varphi}\m{)}
\m{)}
\endm

\needspace{3\baselineskip}
\noindent Axiom of Distribution.

\setbox\startprefix=\hbox{\tt \ \ ax-2\ \$a\ }
\setbox\contprefix=\hbox{\tt \ \ \ \ \ \ \ \ \ \ }
\startm
\m{\vdash}\m{(}\m{(}\m{\varphi}\m{\rightarrow}\m{(}\m{\psi}\m{\rightarrow}\m{\chi}
\m{)}\m{)}\m{\rightarrow}\m{(}\m{(}\m{\varphi}\m{\rightarrow}\m{\psi}\m{)}\m{
\rightarrow}\m{(}\m{\varphi}\m{\rightarrow}\m{\chi}\m{)}\m{)}\m{)}
\endm

\needspace{2\baselineskip}
\noindent Axiom of Contraposition.

\setbox\startprefix=\hbox{\tt \ \ ax-3\ \$a\ }
\setbox\contprefix=\hbox{\tt \ \ \ \ \ \ \ \ \ \ }
\startm
\m{\vdash}\m{(}\m{(}\m{\lnot}\m{\varphi}\m{\rightarrow}\m{\lnot}\m{\psi}\m{)}\m{
\rightarrow}\m{(}\m{\psi}\m{\rightarrow}\m{\varphi}\m{)}\m{)}
\endm


\needspace{4\baselineskip}
\noindent Rule of Modus Ponens.\label{axmp}\index{modus ponens}

\setbox\startprefix=\hbox{\tt \ \ maj\ \$e\ }
\setbox\contprefix=\hbox{\tt \ \ \ \ \ \ \ \ \ }
\startm
\m{\vdash}\m{(}\m{\varphi}\m{\rightarrow}\m{\psi}\m{)}
\endm

\setbox\startprefix=\hbox{\tt \ \ min\ \$e\ }
\setbox\contprefix=\hbox{\tt \ \ \ \ \ \ \ \ \ }
\startm
\m{\vdash}\m{\varphi}
\endm

\setbox\startprefix=\hbox{\tt \ \ ax-mp\ \$a\ }
\setbox\contprefix=\hbox{\tt \ \ \ \ \ \ \ \ \ \ \ }
\startm
\m{\vdash}\m{\psi}
\endm


\needspace{7\baselineskip}
\subsection{Axioms of Predicate Calculus with Equality---Tarski's S2}\index{axioms of predicate calculus}

\needspace{3\baselineskip}
\noindent Rule of Generalization.\index{rule of generalization}

\setbox\startprefix=\hbox{\tt \ \ ax-g.1\ \$e\ }
\setbox\contprefix=\hbox{\tt \ \ \ \ \ \ \ \ \ \ \ \ }
\startm
\m{\vdash}\m{\varphi}
\endm

\setbox\startprefix=\hbox{\tt \ \ ax-gen\ \$a\ }
\setbox\contprefix=\hbox{\tt \ \ \ \ \ \ \ \ \ \ \ \ }
\startm
\m{\vdash}\m{\forall}\m{x}\m{\varphi}
\endm

\needspace{2\baselineskip}
\noindent Axiom of Quantified Implication.

\setbox\startprefix=\hbox{\tt \ \ ax-4\ \$a\ }
\setbox\contprefix=\hbox{\tt \ \ \ \ \ \ \ \ \ \ }
\startm
\m{\vdash}\m{(}\m{\forall}\m{x}\m{(}\m{\forall}\m{x}\m{\varphi}\m{\rightarrow}\m{
\psi}\m{)}\m{\rightarrow}\m{(}\m{\forall}\m{x}\m{\varphi}\m{\rightarrow}\m{
\forall}\m{x}\m{\psi}\m{)}\m{)}
\endm

\needspace{3\baselineskip}
\noindent Axiom of Distinctness.

% Aka: Add $d x ph $.
\setbox\startprefix=\hbox{\tt \ \ ax-5\ \$a\ }
\setbox\contprefix=\hbox{\tt \ \ \ \ \ \ \ \ \ \ }
\startm
\m{\vdash}\m{(}\m{\varphi}\m{\rightarrow}\m{\forall}\m{x}\m{\varphi}\m{)}\m{where}\m{ }\m{\$d}\m{ }\m{x}\m{ }\m{\varphi}\m{ }\m{(}\m{x}\m{ }\m{does}\m{ }\m{not}\m{ }\m{occur}\m{ }\m{in}\m{ }\m{\varphi}\m{)}
\endm

\needspace{2\baselineskip}
\noindent Axiom of Existence.

\setbox\startprefix=\hbox{\tt \ \ ax-6\ \$a\ }
\setbox\contprefix=\hbox{\tt \ \ \ \ \ \ \ \ \ \ }
\startm
\m{\vdash}\m{(}\m{\forall}\m{x}\m{(}\m{x}\m{=}\m{y}\m{\rightarrow}\m{\forall}
\m{x}\m{\varphi}\m{)}\m{\rightarrow}\m{\varphi}\m{)}
\endm

\needspace{2\baselineskip}
\noindent Axiom of Equality.

\setbox\startprefix=\hbox{\tt \ \ ax-7\ \$a\ }
\setbox\contprefix=\hbox{\tt \ \ \ \ \ \ \ \ \ \ }
\startm
\m{\vdash}\m{(}\m{x}\m{=}\m{y}\m{\rightarrow}\m{(}\m{x}\m{=}\m{z}\m{
\rightarrow}\m{y}\m{=}\m{z}\m{)}\m{)}
\endm

\needspace{2\baselineskip}
\noindent Axiom of Left Equality for Binary Predicate.

\setbox\startprefix=\hbox{\tt \ \ ax-8\ \$a\ }
\setbox\contprefix=\hbox{\tt \ \ \ \ \ \ \ \ \ \ \ }
\startm
\m{\vdash}\m{(}\m{x}\m{=}\m{y}\m{\rightarrow}\m{(}\m{x}\m{\in}\m{z}\m{
\rightarrow}\m{y}\m{\in}\m{z}\m{)}\m{)}
\endm

\needspace{2\baselineskip}
\noindent Axiom of Right Equality for Binary Predicate.

\setbox\startprefix=\hbox{\tt \ \ ax-9\ \$a\ }
\setbox\contprefix=\hbox{\tt \ \ \ \ \ \ \ \ \ \ \ }
\startm
\m{\vdash}\m{(}\m{x}\m{=}\m{y}\m{\rightarrow}\m{(}\m{z}\m{\in}\m{x}\m{
\rightarrow}\m{z}\m{\in}\m{y}\m{)}\m{)}
\endm


\needspace{4\baselineskip}
\subsection{Axioms of Predicate Calculus with Equality---Auxiliary}\index{axioms of predicate calculus - auxiliary}

\needspace{2\baselineskip}
\noindent Axiom of Quantified Negation.

\setbox\startprefix=\hbox{\tt \ \ ax-10\ \$a\ }
\setbox\contprefix=\hbox{\tt \ \ \ \ \ \ \ \ \ \ }
\startm
\m{\vdash}\m{(}\m{\lnot}\m{\forall}\m{x}\m{\lnot}\m{\forall}\m{x}\m{\varphi}\m{
\rightarrow}\m{\varphi}\m{)}
\endm

\needspace{2\baselineskip}
\noindent Axiom of Quantifier Commutation.

\setbox\startprefix=\hbox{\tt \ \ ax-11\ \$a\ }
\setbox\contprefix=\hbox{\tt \ \ \ \ \ \ \ \ \ \ }
\startm
\m{\vdash}\m{(}\m{\forall}\m{x}\m{\forall}\m{y}\m{\varphi}\m{\rightarrow}\m{
\forall}\m{y}\m{\forall}\m{x}\m{\varphi}\m{)}
\endm

\needspace{3\baselineskip}
\noindent Axiom of Substitution.

\setbox\startprefix=\hbox{\tt \ \ ax-12\ \$a\ }
\setbox\contprefix=\hbox{\tt \ \ \ \ \ \ \ \ \ \ \ }
\startm
\m{\vdash}\m{(}\m{\lnot}\m{\forall}\m{x}\m{\,x}\m{=}\m{y}\m{\rightarrow}\m{(}
\m{x}\m{=}\m{y}\m{\rightarrow}\m{(}\m{\varphi}\m{\rightarrow}\m{\forall}\m{x}\m{(}
\m{x}\m{=}\m{y}\m{\rightarrow}\m{\varphi}\m{)}\m{)}\m{)}\m{)}
\endm

\needspace{3\baselineskip}
\noindent Axiom of Quantified Equality.

\setbox\startprefix=\hbox{\tt \ \ ax-13\ \$a\ }
\setbox\contprefix=\hbox{\tt \ \ \ \ \ \ \ \ \ \ \ }
\startm
\m{\vdash}\m{(}\m{\lnot}\m{\forall}\m{z}\m{\,z}\m{=}\m{x}\m{\rightarrow}\m{(}
\m{\lnot}\m{\forall}\m{z}\m{\,z}\m{=}\m{y}\m{\rightarrow}\m{(}\m{x}\m{=}\m{y}
\m{\rightarrow}\m{\forall}\m{z}\m{\,x}\m{=}\m{y}\m{)}\m{)}\m{)}
\endm

% \noindent Axiom of Quantifier Substitution
%
% \setbox\startprefix=\hbox{\tt \ \ ax-c11n\ \$a\ }
% \setbox\contprefix=\hbox{\tt \ \ \ \ \ \ \ \ \ \ \ }
% \startm
% \m{\vdash}\m{(}\m{\forall}\m{x}\m{\,x}\m{=}\m{y}\m{\rightarrow}\m{(}\m{\forall}
% \m{x}\m{\varphi}\m{\rightarrow}\m{\forall}\m{y}\m{\varphi}\m{)}\m{)}
% \endm
%
% \noindent Axiom of Distinct Variables. (This axiom requires
% that two individual variables
% be distinct\index{\texttt{\$d} statement}\index{distinct
% variables}.)
%
% \setbox\startprefix=\hbox{\tt \ \ \ \ \ \ \ \ \$d\ }
% \setbox\contprefix=\hbox{\tt \ \ \ \ \ \ \ \ \ \ \ }
% \startm
% \m{x}\m{\,}\m{y}
% \endm
%
% \setbox\startprefix=\hbox{\tt \ \ ax-c16\ \$a\ }
% \setbox\contprefix=\hbox{\tt \ \ \ \ \ \ \ \ \ \ \ }
% \startm
% \m{\vdash}\m{(}\m{\forall}\m{x}\m{\,x}\m{=}\m{y}\m{\rightarrow}\m{(}\m{\varphi}\m{
% \rightarrow}\m{\forall}\m{x}\m{\varphi}\m{)}\m{)}
% \endm

% \noindent Axiom of Quantifier Introduction (2).  (This axiom requires
% that the individual variable not occur in the
% wff\index{\texttt{\$d} statement}\index{distinct variables}.)
%
% \setbox\startprefix=\hbox{\tt \ \ \ \ \ \ \ \ \$d\ }
% \setbox\contprefix=\hbox{\tt \ \ \ \ \ \ \ \ \ \ \ }
% \startm
% \m{x}\m{\,}\m{\varphi}
% \endm
% \setbox\startprefix=\hbox{\tt \ \ ax-5\ \$a\ }
% \setbox\contprefix=\hbox{\tt \ \ \ \ \ \ \ \ \ \ \ }
% \startm
% \m{\vdash}\m{(}\m{\varphi}\m{\rightarrow}\m{\forall}\m{x}\m{\varphi}\m{)}
% \endm

\subsection{Set Theory}\label{mmsettheoryaxioms}

In order to make the axioms of set theory\index{axioms of set theory} a little
more compact, there are several definitions from logic that we make use of
implicitly, namely, ``logical {\sc and},''\index{conjunction ($\wedge$)}
\index{logical {\sc and} ($\wedge$)} ``logical equivalence,''\index{logical
equivalence ($\leftrightarrow$)}\index{biconditional ($\leftrightarrow$)} and
``there exists.''\index{existential quantifier ($\exists$)}

\begin{center}\begin{tabular}{rcl}
  $( \varphi \wedge \psi )$ &\mbox{stands for}& $\neg ( \varphi
     \rightarrow \neg \psi )$\\
  $( \varphi \leftrightarrow \psi )$& \mbox{stands
     for}& $( ( \varphi \rightarrow \psi ) \wedge
     ( \psi \rightarrow \varphi ) )$\\
  $\exists x \,\varphi$ &\mbox{stands for}& $\neg \forall x \neg \varphi$
\end{tabular}\end{center}

In addition, the axioms of set theory require that all variables be
dis\-tinct,\index{distinct variables}\footnote{Set theory axioms can be
devised so that {\em no} variables are required to be distinct,
provided we replace \texttt{ax-c16} with an axiom stating that ``at
least two things exist,'' thus
making \texttt{ax-5} the only other axiom requiring the
\texttt{\$d} statement.  These axioms are unconventional and are not
presented here, but they can be found on the \url{http://metamath.org}
web site.  See also the Comment on
p.~\pageref{nodd}.}\index{\texttt{\$d} statement} thus we also assume:
\begin{center}
  \texttt{\$d }$x\,y\,z\,w$
\end{center}

\needspace{2\baselineskip}
\noindent Axiom of Extensionality.\index{Axiom of Extensionality}

\setbox\startprefix=\hbox{\tt \ \ ax-ext\ \$a\ }
\setbox\contprefix=\hbox{\tt \ \ \ \ \ \ \ \ \ \ \ \ }
\startm
\m{\vdash}\m{(}\m{\forall}\m{x}\m{(}\m{x}\m{\in}\m{y}\m{\leftrightarrow}\m{x}
\m{\in}\m{z}\m{)}\m{\rightarrow}\m{y}\m{=}\m{z}\m{)}
\endm

\needspace{3\baselineskip}
\noindent Axiom of Replacement.\index{Axiom of Replacement}

\setbox\startprefix=\hbox{\tt \ \ ax-rep\ \$a\ }
\setbox\contprefix=\hbox{\tt \ \ \ \ \ \ \ \ \ \ \ \ }
\startm
\m{\vdash}\m{(}\m{\forall}\m{w}\m{\exists}\m{y}\m{\forall}\m{z}\m{(}\m{%
\forall}\m{y}\m{\varphi}\m{\rightarrow}\m{z}\m{=}\m{y}\m{)}\m{\rightarrow}\m{%
\exists}\m{y}\m{\forall}\m{z}\m{(}\m{z}\m{\in}\m{y}\m{\leftrightarrow}\m{%
\exists}\m{w}\m{(}\m{w}\m{\in}\m{x}\m{\wedge}\m{\forall}\m{y}\m{\varphi}\m{)}%
\m{)}\m{)}
\endm

\needspace{2\baselineskip}
\noindent Axiom of Union.\index{Axiom of Union}

\setbox\startprefix=\hbox{\tt \ \ ax-un\ \$a\ }
\setbox\contprefix=\hbox{\tt \ \ \ \ \ \ \ \ \ \ \ }
\startm
\m{\vdash}\m{\exists}\m{x}\m{\forall}\m{y}\m{(}\m{\exists}\m{x}\m{(}\m{y}\m{
\in}\m{x}\m{\wedge}\m{x}\m{\in}\m{z}\m{)}\m{\rightarrow}\m{y}\m{\in}\m{x}\m{)}
\endm

\needspace{2\baselineskip}
\noindent Axiom of Power Sets.\index{Axiom of Power Sets}

\setbox\startprefix=\hbox{\tt \ \ ax-pow\ \$a\ }
\setbox\contprefix=\hbox{\tt \ \ \ \ \ \ \ \ \ \ \ \ }
\startm
\m{\vdash}\m{\exists}\m{x}\m{\forall}\m{y}\m{(}\m{\forall}\m{x}\m{(}\m{x}\m{
\in}\m{y}\m{\rightarrow}\m{x}\m{\in}\m{z}\m{)}\m{\rightarrow}\m{y}\m{\in}\m{x}
\m{)}
\endm

\needspace{3\baselineskip}
\noindent Axiom of Regularity.\index{Axiom of Regularity}

\setbox\startprefix=\hbox{\tt \ \ ax-reg\ \$a\ }
\setbox\contprefix=\hbox{\tt \ \ \ \ \ \ \ \ \ \ \ \ }
\startm
\m{\vdash}\m{(}\m{\exists}\m{x}\m{\,x}\m{\in}\m{y}\m{\rightarrow}\m{\exists}
\m{x}\m{(}\m{x}\m{\in}\m{y}\m{\wedge}\m{\forall}\m{z}\m{(}\m{z}\m{\in}\m{x}\m{
\rightarrow}\m{\lnot}\m{z}\m{\in}\m{y}\m{)}\m{)}\m{)}
\endm

\needspace{3\baselineskip}
\noindent Axiom of Infinity.\index{Axiom of Infinity}

\setbox\startprefix=\hbox{\tt \ \ ax-inf\ \$a\ }
\setbox\contprefix=\hbox{\tt \ \ \ \ \ \ \ \ \ \ \ \ \ \ \ }
\startm
\m{\vdash}\m{\exists}\m{x}\m{(}\m{y}\m{\in}\m{x}\m{\wedge}\m{\forall}\m{y}%
\m{(}\m{y}\m{\in}\m{x}\m{\rightarrow}\m{\exists}\m{z}\m{(}\m{y}\m{\in}\m{z}\m{%
\wedge}\m{z}\m{\in}\m{x}\m{)}\m{)}\m{)}
\endm

\needspace{4\baselineskip}
\noindent Axiom of Choice.\index{Axiom of Choice}

\setbox\startprefix=\hbox{\tt \ \ ax-ac\ \$a\ }
\setbox\contprefix=\hbox{\tt \ \ \ \ \ \ \ \ \ \ \ \ \ \ }
\startm
\m{\vdash}\m{\exists}\m{x}\m{\forall}\m{y}\m{\forall}\m{z}\m{(}\m{(}\m{y}\m{%
\in}\m{z}\m{\wedge}\m{z}\m{\in}\m{w}\m{)}\m{\rightarrow}\m{\exists}\m{w}\m{%
\forall}\m{y}\m{(}\m{\exists}\m{w}\m{(}\m{(}\m{y}\m{\in}\m{z}\m{\wedge}\m{z}%
\m{\in}\m{w}\m{)}\m{\wedge}\m{(}\m{y}\m{\in}\m{w}\m{\wedge}\m{w}\m{\in}\m{x}%
\m{)}\m{)}\m{\leftrightarrow}\m{y}\m{=}\m{w}\m{)}\m{)}
\endm

\subsection{That's It}

There you have it, the axioms for (essentially) all of mathematics!
Wonder at them and stare at them in awe.  Put a copy in your wallet, and
you will carry in your pocket the encoding for all theorems ever proved
and that ever will be proved, from the most mundane to the most
profound.

\section{A Hierarchy of Definitions}\label{hierarchy}

The axioms in the previous section in principle embody everything that can be
done within standard mathematics.  However, it is impractical to accomplish
very much by using them directly, for even simple concepts (from a human
perspective) can involve extremely long, incomprehensible formulas.
Mathematics is made practical by introducing definitions\index{definition}.
Definitions usually introduce new symbols, or at least new relationships among
existing symbols, to abbreviate more complex formulas.  An important
requirement for a definition is that there exist a straightforward
(algorithmic) method for eliminating the abbreviation by expanding it into the
more primitive symbol string that it represents.  Some
important definitions included in
the file \texttt{set.mm} are listed in this section for reference, and also to
give you a feel for why something like $\omega$\index{omega ($\omega$)} (the
set of natural numbers\index{natural number} 0, 1, 2,\ldots) becomes very
complicated when completely expanded into primitive symbols.

What is the motivation for definitions, aside from allowing complicated
expressions to be expressed more simply?  In the case of  $\omega$, one goal is
to provide a basis for the theory of natural numbers.\index{natural number}
Before set theory was invented, a set of axioms for arithmetic, called Peano's
postulates\index{Peano's postulates}, was devised and shown to have the
properties one expects for natural numbers.  Now anyone can postulate a
set of axioms, but if the axioms are inconsistent contradictions can be derived
from them.  Once a contradiction is derived, anything can be trivially
proved, including
all the facts of arithmetic and their negations.  To ensure that an
axiom system is at least as reliable as the axioms for set theory, we can
define sets and operations on those sets that satisfy the new axioms. In the
\texttt{set.mm} Metamath database, we prove that the elements of $\omega$ satisfy
Peano's postulates, and it's a long and hard journey to get there directly
from the axioms of set theory.  But the result is confidence in the
foundations of arithmetic.  And there is another advantage:  we now have all
the tools of set theory at our disposal for manipulating objects that obey the
axioms for arithmetic.

What are the criteria we use for definitions?  First, and of utmost importance,
the definition should not be {\em creative}\index{creative
definition}\index{definition!creative}, that
is it should not allow an expression that previously qualified as a wff but
was not provable, to become provable.   Second, the definition should be {\em
eliminable}\index{definition!eliminability}, that is, there should exist an
algorithmic method for converting any expression using the definition into
a logically equivalent expression that previously qualified as a wff.

In almost all cases below, definitions connect two expressions with either
$\leftrightarrow$ or $=$.  Eliminating\footnote{Here we mean the
elimination that a human might do in his or her head.  To eliminate them as
part of a Metamath proof we would invoke one of a number of
theorems that deal with transitivity of equivalence or equality; there are
many such examples in the proofs in \texttt{set.mm}.} such a definition is a
simple matter of substituting the expression on the left-hand side ({\em
definiendum}\index{definiendum} or thing being defined) with the equivalent,
more primitive expression on the right-hand side ({\em
definiens}\index{definiens} or definition).

Often a definition has variables on the right-hand side which do not appear on
the left-hand side; these are called {\em dummy variables}.\index{dummy
variable!in definitions}  In this case, any
allowable substitution (such as a new, distinct
variable) can be used when the definition is eliminated.  Dummy variables may
be used only if they are {\em effectively bound}\index{effectively bound
variable}, meaning that the definition will remain logically equivalent upon
any substitution of a dummy variable with any other {\em qualifying
expression}\index{qualifying expression}, i.e.\ any symbol string (such as
another variable) that
meets the restrictions on the dummy variable imposed by \texttt{\$d} and
\texttt{\$f} statements.  For example, we could define a constant $\perp$
(inverted tee, meaning logical ``false'') as $( \varphi \wedge \lnot \varphi
)$, i.e.\ ``phi and not phi.''  Here $\varphi$ is effectively bound because the
definition remains logically equivalent when we replace $\varphi$ with any
other wff.  (It is actually \texttt{df-fal}
in \texttt{set.mm}, which defines $\perp$.)

There are two cases where eliminating definitions is a little more
complex.  These cases are the definitions \texttt{df-bi} and
\texttt{df-cleq}.  The first stretches the concept of a definition a
little, as in effect it ``defines a definition;'' however, it meets our
requirements for a definition in that it is eliminable and does not
strengthen the language.  Theorem \texttt{bii} shows the substitution
needed to eliminate the $\leftrightarrow$\index{logical equivalence
($\leftrightarrow$)}\index{biconditional ($\leftrightarrow$)} symbol.

Definition \texttt{df-cleq}\index{equality ($=$)} extends the usage of
the equality symbol to include ``classes''\index{class} in set theory.  The
reason it is potentially problematic is that it can lead to statements which
do not follow from logic alone but presuppose the Axiom of
Extensionality\index{Axiom of Extensionality}, so we include this axiom
as a hypothesis for the definition.  We could have made \texttt{df-cleq} directly
eliminable by introducing a new equality symbol, but have chosen not to do so
in keeping with standard textbook practice.  Definitions such as \texttt{df-cleq}
that extend the meaning of existing symbols must be introduced carefully so
that they do not lead to contradictions.  Definition \texttt{df-clel} also
extends the meaning of an existing symbol ($\in$); while it doesn't strengthen
the language like \texttt{df-cleq}, this is not obvious and it must also be
subject to the same scrutiny.

Exercise:  Study how the wff $x\in\omega$, meaning ``$x$ is a natural
number,'' could be expanded in terms of primitive symbols, starting with the
definitions \texttt{df-clel} on p.~\pageref{dfclel} and \texttt{df-om} on
p.~\pageref{dfom} and working your way back.  Don't bother to work out the
details; just make sure that you understand how you could do it in principle.
The answer is shown in the footnote on p.~\pageref{expandom}.  If you
actually do work it out, you won't get exactly the same answer because we used
a few simplifications such as discarding occurrences of $\lnot\lnot$ (double
negation).

In the definitions below, we have placed the {\sc ascii} Metamath source
below each of the formulas to help you become familiar with the
notation in the database.  For simplicity, the necessary \texttt{\$f}
and \texttt{\$d} statements are not shown.  If you are in doubt, use the
\texttt{show statement}\index{\texttt{show statement} command} command
in the Metamath program to see the full statement.
A selection of this notation is summarized in Appendix~\ref{ASCII}.

To understand the motivation for these definitions, you should consult the
references indicated:  Takeuti and Zaring \cite{Takeuti}\index{Takeuti, Gaisi},
Quine \cite{Quine}\index{Quine, Willard Van Orman}, Bell and Machover
\cite{Bell}\index{Bell, J. L.}, and Enderton \cite{Enderton}\index{Enderton,
Herbert B.}.  Our list of definitions is provided more for reference than as a
learning aid.  However, by looking at a few of them you can gain a feel for
how the hierarchy is built up.  The definitions are a representative sample of
the 130 or so in \texttt{set.mm}, but they are complete with respect to the
theorem examples we will present in Section~\ref{sometheorems}.  Also, some are
slightly different from, but logically equivalent to, the ones in \texttt{set.mm}
(some of which have been revised over time to shorten them, for example).

\subsection{Definitions for Propositional Calculus}\label{metadefprop}

The symbols $\varphi$, $\psi$, and $\chi$ represent wffs.

Our first definition introduces the biconditional
connective\footnote{The term ``connective'' is informally used to mean a
symbol that is placed between two variables or adjacent to a variable,
whereas a mathematical ``constant'' usually indicates a symbol such as
the number 0 that may replace a variable or metavariable.  From
Metamath's point of view, there is no distinction between a connective
and a constant; both are constants in the Metamath
language.}\index{connective}\index{constant} (also called logical
equivalence)\index{logical equivalence
($\leftrightarrow$)}\index{biconditional ($\leftrightarrow$)}.  Unlike
most traditional developments, we have chosen not to have a separate
symbol such as ``Df.'' to mean ``is defined as.''  Instead, we will use
the biconditional connective for this purpose, as it lets us use
logic to manipulate definitions directly.  Here we state the properties
of the biconditional connective with a carefully crafted \texttt{\$a}
statement, which effectively uses the biconditional connective to define
itself.  The $\leftrightarrow$ symbol can be eliminated from a formula
using theorem \texttt{bii}, which is derived later.

\vskip 2ex
\noindent Define the biconditional connective.\label{df-bi}

\vskip 0.5ex
\setbox\startprefix=\hbox{\tt \ \ df-bi\ \$a\ }
\setbox\contprefix=\hbox{\tt \ \ \ \ \ \ \ \ \ \ \ }
\startm
\m{\vdash}\m{\lnot}\m{(}\m{(}\m{(}\m{\varphi}\m{\leftrightarrow}\m{\psi}\m{)}%
\m{\rightarrow}\m{\lnot}\m{(}\m{(}\m{\varphi}\m{\rightarrow}\m{\psi}\m{)}\m{%
\rightarrow}\m{\lnot}\m{(}\m{\psi}\m{\rightarrow}\m{\varphi}\m{)}\m{)}\m{)}\m{%
\rightarrow}\m{\lnot}\m{(}\m{\lnot}\m{(}\m{(}\m{\varphi}\m{\rightarrow}\m{%
\psi}\m{)}\m{\rightarrow}\m{\lnot}\m{(}\m{\psi}\m{\rightarrow}\m{\varphi}\m{)}%
\m{)}\m{\rightarrow}\m{(}\m{\varphi}\m{\leftrightarrow}\m{\psi}\m{)}\m{)}\m{)}
\endm
\begin{mmraw}%
|- -. ( ( ( ph <-> ps ) -> -. ( ( ph -> ps ) ->
-. ( ps -> ph ) ) ) -> -. ( -. ( ( ph -> ps ) -> -. (
ps -> ph ) ) -> ( ph <-> ps ) ) ) \$.
\end{mmraw}

\noindent This theorem relates the biconditional connective to primitive
connectives and can be used to eliminate the $\leftrightarrow$ symbol from any
wff.

\vskip 0.5ex
\setbox\startprefix=\hbox{\tt \ \ bii\ \$p\ }
\setbox\contprefix=\hbox{\tt \ \ \ \ \ \ \ \ \ }
\startm
\m{\vdash}\m{(}\m{(}\m{\varphi}\m{\leftrightarrow}\m{\psi}\m{)}\m{\leftrightarrow}
\m{\lnot}\m{(}\m{(}\m{\varphi}\m{\rightarrow}\m{\psi}\m{)}\m{\rightarrow}\m{\lnot}
\m{(}\m{\psi}\m{\rightarrow}\m{\varphi}\m{)}\m{)}\m{)}
\endm
\begin{mmraw}%
|- ( ( ph <-> ps ) <-> -. ( ( ph -> ps ) -> -. ( ps -> ph ) ) ) \$= ... \$.
\end{mmraw}

\noindent Define disjunction ({\sc or}).\index{disjunction ($\vee$)}%
\index{logical or (vee)@logical {\sc or} ($\vee$)}%
\index{df-or@\texttt{df-or}}\label{df-or}

\vskip 0.5ex
\setbox\startprefix=\hbox{\tt \ \ df-or\ \$a\ }
\setbox\contprefix=\hbox{\tt \ \ \ \ \ \ \ \ \ \ \ }
\startm
\m{\vdash}\m{(}\m{(}\m{\varphi}\m{\vee}\m{\psi}\m{)}\m{\leftrightarrow}\m{(}\m{
\lnot}\m{\varphi}\m{\rightarrow}\m{\psi}\m{)}\m{)}
\endm
\begin{mmraw}%
|- ( ( ph \TOR ps ) <-> ( -. ph -> ps ) ) \$.
\end{mmraw}

\noindent Define conjunction ({\sc and}).\index{conjunction ($\wedge$)}%
\index{logical {\sc and} ($\wedge$)}%
\index{df-an@\texttt{df-an}}\label{df-an}

\vskip 0.5ex
\setbox\startprefix=\hbox{\tt \ \ df-an\ \$a\ }
\setbox\contprefix=\hbox{\tt \ \ \ \ \ \ \ \ \ \ \ }
\startm
\m{\vdash}\m{(}\m{(}\m{\varphi}\m{\wedge}\m{\psi}\m{)}\m{\leftrightarrow}\m{\lnot}
\m{(}\m{\varphi}\m{\rightarrow}\m{\lnot}\m{\psi}\m{)}\m{)}
\endm
\begin{mmraw}%
|- ( ( ph \TAND ps ) <-> -. ( ph -> -. ps ) ) \$.
\end{mmraw}

\noindent Define disjunction ({\sc or}) of 3 wffs.%
\index{df-3or@\texttt{df-3or}}\label{df-3or}

\vskip 0.5ex
\setbox\startprefix=\hbox{\tt \ \ df-3or\ \$a\ }
\setbox\contprefix=\hbox{\tt \ \ \ \ \ \ \ \ \ \ \ \ }
\startm
\m{\vdash}\m{(}\m{(}\m{\varphi}\m{\vee}\m{\psi}\m{\vee}\m{\chi}\m{)}\m{
\leftrightarrow}\m{(}\m{(}\m{\varphi}\m{\vee}\m{\psi}\m{)}\m{\vee}\m{\chi}\m{)}
\m{)}
\endm
\begin{mmraw}%
|- ( ( ph \TOR ps \TOR ch ) <-> ( ( ph \TOR ps ) \TOR ch ) ) \$.
\end{mmraw}

\noindent Define conjunction ({\sc and}) of 3 wffs.%
\index{df-3an}\label{df-3an}

\vskip 0.5ex
\setbox\startprefix=\hbox{\tt \ \ df-3an\ \$a\ }
\setbox\contprefix=\hbox{\tt \ \ \ \ \ \ \ \ \ \ \ \ }
\startm
\m{\vdash}\m{(}\m{(}\m{\varphi}\m{\wedge}\m{\psi}\m{\wedge}\m{\chi}\m{)}\m{
\leftrightarrow}\m{(}\m{(}\m{\varphi}\m{\wedge}\m{\psi}\m{)}\m{\wedge}\m{\chi}
\m{)}\m{)}
\endm

\begin{mmraw}%
|- ( ( ph \TAND ps \TAND ch ) <-> ( ( ph \TAND ps ) \TAND ch ) ) \$.
\end{mmraw}

\subsection{Definitions for Predicate Calculus}\label{metadefpred}

The symbols $x$, $y$, and $z$ represent individual variables of predicate
calculus.  In this section, they are not necessarily distinct unless it is
explicitly
mentioned.

\vskip 2ex
\noindent Define existential quantification.
The expression $\exists x \varphi$ means
``there exists an $x$ where $\varphi$ is true.''\index{existential quantifier
($\exists$)}\label{df-ex}

\vskip 0.5ex
\setbox\startprefix=\hbox{\tt \ \ df-ex\ \$a\ }
\setbox\contprefix=\hbox{\tt \ \ \ \ \ \ \ \ \ \ \ }
\startm
\m{\vdash}\m{(}\m{\exists}\m{x}\m{\varphi}\m{\leftrightarrow}\m{\lnot}\m{\forall}
\m{x}\m{\lnot}\m{\varphi}\m{)}
\endm
\begin{mmraw}%
|- ( E. x ph <-> -. A. x -. ph ) \$.
\end{mmraw}

\noindent Define proper substitution.\index{proper
substitution}\index{substitution!proper}\label{df-sb}
In our notation, we use $[ y / x ] \varphi$ to mean ``the wff that
results when $y$ is properly substituted for $x$ in the wff
$\varphi$.''\footnote{
This can also be described
as substituting $x$ with $y$, $y$ properly replaces $x$, or
$x$ is properly replaced by $y$.}
% This is elsb4, though it currently says: ( [ x / y ] z e. y <-> z e. x )
For example,
$[ y / x ] z \in x$ is the same as $z \in y$.
One way to remember this notation is to notice that it looks like division
and recall that $( y / x ) \cdot x $ is $y$ (when $x \neq 0$).
The notation is different from the notation $\varphi ( x | y )$
that is sometimes used, because the latter notation is ambiguous for us:
for example, we don't know whether $\lnot \varphi ( x | y )$ is to be
interpreted as $\lnot ( \varphi ( x | y ) )$ or
$( \lnot \varphi ) ( x | y )$.\footnote{Because of the way
we initially defined wffs, this is the case
with any postfix connective\index{postfix connective} (one occurring after the
symbols being connected) or infix connective\index{infix connective} (one
occurring between the symbols being connected).  Metamath does not have a
built-in notion of operator binding strength that could eliminate the
ambiguity.  The initial parenthesis effectively provides a prefix
connective\index{prefix connective} to eliminate ambiguity.  Some conventions,
such as Polish notation\index{Polish notation} used in the 1930's and 1940's
by Polish logicians, use only prefix connectives and thus allow the total
elimination of parentheses, at the expense of readability.  In Metamath we
could actually redefine all notation to be Polish if we wanted to without
having to change any proofs!}  Other texts often use $\varphi(y)$ to indicate
our $[ y / x ] \varphi$, but this notation is even more ambiguous since there is
no explicit indication of what is being substituted.
Note that this
definition is valid even when
$x$ and $y$ are the same variable.  The first conjunct is a ``trick'' used to
achieve this property, making the definition look somewhat peculiar at
first.

\vskip 0.5ex
\setbox\startprefix=\hbox{\tt \ \ df-sb\ \$a\ }
\setbox\contprefix=\hbox{\tt \ \ \ \ \ \ \ \ \ \ \ }
\startm
\m{\vdash}\m{(}\m{[}\m{y}\m{/}\m{x}\m{]}\m{\varphi}\m{\leftrightarrow}\m{(}%
\m{(}\m{x}\m{=}\m{y}\m{\rightarrow}\m{\varphi}\m{)}\m{\wedge}\m{\exists}\m{x}%
\m{(}\m{x}\m{=}\m{y}\m{\wedge}\m{\varphi}\m{)}\m{)}\m{)}
\endm
\begin{mmraw}%
|- ( [ y / x ] ph <-> ( ( x = y -> ph ) \TAND E. x ( x = y \TAND ph ) ) ) \$.
\end{mmraw}


\noindent Define existential uniqueness\index{existential uniqueness
quantifier ($\exists "!$)} (``there exists exactly one'').  Note that $y$ is a
variable distinct from $x$ and not occurring in $\varphi$.\label{df-eu}

\vskip 0.5ex
\setbox\startprefix=\hbox{\tt \ \ df-eu\ \$a\ }
\setbox\contprefix=\hbox{\tt \ \ \ \ \ \ \ \ \ \ \ }
\startm
\m{\vdash}\m{(}\m{\exists}\m{{!}}\m{x}\m{\varphi}\m{\leftrightarrow}\m{\exists}
\m{y}\m{\forall}\m{x}\m{(}\m{\varphi}\m{\leftrightarrow}\m{x}\m{=}\m{y}\m{)}\m{)}
\endm

\begin{mmraw}%
|- ( E! x ph <-> E. y A. x ( ph <-> x = y ) ) \$.
\end{mmraw}

\subsection{Definitions for Set Theory}\label{setdefinitions}

The symbols $x$, $y$, $z$, and $w$ represent individual variables of
predicate calculus, which in set theory are understood to be sets.
However, using only the constructs shown so far would be very inconvenient.

To make set theory more practical, we introduce the notion of a ``class.''
A class\index{class} is either a set variable (such as $x$) or an
expression of the form $\{ x | \varphi\}$ (called an ``abstraction
class''\index{abstraction class}\index{class abstraction}).  Note that
sets (i.e.\ individual variables) always exist (this is a theorem of
logic, namely $\exists y \, y = x$ for any set $x$), whereas classes may
or may not exist (i.e.\ $\exists y \, y = A$ may or may not be true).
If a class does not exist it is called a ``proper class.''\index{proper
class}\index{class!proper} Definitions \texttt{df-clab},
\texttt{df-cleq}, and \texttt{df-clel} can be used to convert an
expression containing classes into one containing only set variables and
wff metavariables.

The symbols $A$, $B$, $C$, $D$, $ F$, $G$, and $R$ are metavariables that range
over classes.  A class metavariable $A$ may be eliminated from a wff by
replacing it with $\{ x|\varphi\}$ where neither $x$ nor $\varphi$ occur in
the wff.

The theory of classes can be shown to be an eliminable and conservative
extension of set theory. The \textbf{eliminability}
property shows that for every
formula in the extended language we can build a logically equivalent
formula in the basic language; so that even if the extended language
provides more ease to convey and formulate mathematical ideas for set
theory, its expressive power does not in fact strengthen the basic
language's expressive power.
The \textbf{conservation} property shows that for
every proof of a formula of the basic language in the extended system
we can build another proof of the same formula in the basic system;
so that, concerning theorems on sets only, the deductive powers of
the extended system and of the basic system are identical. Together,
these properties mean that the extended language can be treated as a
definitional extension that is \textbf{sound}.

A rigorous justification, which we will not give here, can be found in
Levy \cite[pp.~357-366]{Levy} supplementing his informal introduction to class
theory on pp.~7-17. Two other good treatments of class theory are provided
by Quine \cite[pp.~15-21]{Quine}\index{Quine, Willard Van Orman}
and also \cite[pp.~10-14]{Takeuti}\index{Takeuti, Gaisi}.
Quine's exposition (he calls them virtual classes)
is nicely written and very readable.

In the rest of this
section, individual variables are always assumed to be distinct from
each other unless otherwise indicated.  In addition, dummy variables on the
right-hand side of a definition do not occur in the class and wff
metavariables in the definition.

The definitions we present here are a partial but self-contained
collection selected from several hundred that appear in the current
\texttt{set.mm} database.  They are adequate for a basic development of
elementary set theory.

\vskip 2ex
\noindent Define the abstraction class.\index{abstraction class}\index{class
abstraction}\label{df-clab}  $x$ and $y$
need not be distinct.  Definition 2.1 of Quine, p.~16.  This definition may
seem puzzling since it is shorter than the expression being defined and does not
buy us anything in terms of brevity.  The reason we introduce this definition
is because it fits in neatly with the extension of the $\in$ connective
provided by \texttt{df-clel}.

\vskip 0.5ex
\setbox\startprefix=\hbox{\tt \ \ df-clab\ \$a\ }
\setbox\contprefix=\hbox{\tt \ \ \ \ \ \ \ \ \ \ \ \ \ }
\startm
\m{\vdash}\m{(}\m{x}\m{\in}\m{\{}\m{y}\m{|}\m{\varphi}\m{\}}\m{%
\leftrightarrow}\m{[}\m{x}\m{/}\m{y}\m{]}\m{\varphi}\m{)}
\endm
\begin{mmraw}%
|- ( x e. \{ y | ph \} <-> [ x / y ] ph ) \$.
\end{mmraw}

\noindent Define the equality connective between classes\index{class
equality}\label{df-cleq}.  See Quine or Chapter 4 of Takeuti and Zaring for its
justification and methods for eliminating it.  This is an example of a
somewhat ``dangerous'' definition, because it extends the use of the
existing equality symbol rather than introducing a new symbol, allowing
us to make statements in the original language that may not be true.
For example, it permits us to deduce $y = z \leftrightarrow \forall x (
x \in y \leftrightarrow x \in z )$ which is not a theorem of logic but
rather presupposes the Axiom of Extensionality,\index{Axiom of
Extensionality} which we include as a hypothesis so that we can know
when this axiom is assumed in a proof (with the \texttt{show
trace{\char`\_}back} command).  We could avoid the danger by introducing
another symbol, say $\eqcirc$, in place of $=$; this
would also have the advantage of making elimination of the definition
straightforward and would eliminate the need for Extensionality as a
hypothesis.  We would then also have the advantage of being able to
identify exactly where Extensionality truly comes into play.  One of our
theorems would be $x \eqcirc y \leftrightarrow x = y$
by invoking Extensionality.  However in keeping with standard practice
we retain the ``dangerous'' definition.

\vskip 0.5ex
\setbox\startprefix=\hbox{\tt \ \ df-cleq.1\ \$e\ }
\setbox\contprefix=\hbox{\tt \ \ \ \ \ \ \ \ \ \ \ \ \ \ \ }
\startm
\m{\vdash}\m{(}\m{\forall}\m{x}\m{(}\m{x}\m{\in}\m{y}\m{\leftrightarrow}\m{x}
\m{\in}\m{z}\m{)}\m{\rightarrow}\m{y}\m{=}\m{z}\m{)}
\endm
\setbox\startprefix=\hbox{\tt \ \ df-cleq\ \$a\ }
\setbox\contprefix=\hbox{\tt \ \ \ \ \ \ \ \ \ \ \ \ \ }
\startm
\m{\vdash}\m{(}\m{A}\m{=}\m{B}\m{\leftrightarrow}\m{\forall}\m{x}\m{(}\m{x}\m{
\in}\m{A}\m{\leftrightarrow}\m{x}\m{\in}\m{B}\m{)}\m{)}
\endm
% We need to reset the startprefix and contprefix.
\setbox\startprefix=\hbox{\tt \ \ df-cleq.1\ \$e\ }
\setbox\contprefix=\hbox{\tt \ \ \ \ \ \ \ \ \ \ \ \ \ \ \ }
\begin{mmraw}%
|- ( A. x ( x e. y <-> x e. z ) -> y = z ) \$.
\end{mmraw}
\setbox\startprefix=\hbox{\tt \ \ df-cleq\ \$a\ }
\setbox\contprefix=\hbox{\tt \ \ \ \ \ \ \ \ \ \ \ \ \ }
\begin{mmraw}%
|- ( A = B <-> A. x ( x e. A <-> x e. B ) ) \$.
\end{mmraw}

\noindent Define the membership connective between classes\index{class
membership}.  Theorem 6.3 of Quine, p.~41, which we adopt as a definition.
Note that it extends the use of the existing membership symbol, but unlike
{\tt df-cleq} it does not extend the set of valid wffs of logic when the class
metavariables are replaced with set variables.\label{dfclel}\label{df-clel}

\vskip 0.5ex
\setbox\startprefix=\hbox{\tt \ \ df-clel\ \$a\ }
\setbox\contprefix=\hbox{\tt \ \ \ \ \ \ \ \ \ \ \ \ \ }
\startm
\m{\vdash}\m{(}\m{A}\m{\in}\m{B}\m{\leftrightarrow}\m{\exists}\m{x}\m{(}\m{x}
\m{=}\m{A}\m{\wedge}\m{x}\m{\in}\m{B}\m{)}\m{)}
\endm
\begin{mmraw}%
|- ( A e. B <-> E. x ( x = A \TAND x e. B ) ) \$.?
\end{mmraw}

\noindent Define inequality.

\vskip 0.5ex
\setbox\startprefix=\hbox{\tt \ \ df-ne\ \$a\ }
\setbox\contprefix=\hbox{\tt \ \ \ \ \ \ \ \ \ \ \ }
\startm
\m{\vdash}\m{(}\m{A}\m{\ne}\m{B}\m{\leftrightarrow}\m{\lnot}\m{A}\m{=}\m{B}%
\m{)}
\endm
\begin{mmraw}%
|- ( A =/= B <-> -. A = B ) \$.
\end{mmraw}

\noindent Define restricted universal quantification.\index{universal
quantifier ($\forall$)!restricted}  Enderton, p.~22.

\vskip 0.5ex
\setbox\startprefix=\hbox{\tt \ \ df-ral\ \$a\ }
\setbox\contprefix=\hbox{\tt \ \ \ \ \ \ \ \ \ \ \ \ }
\startm
\m{\vdash}\m{(}\m{\forall}\m{x}\m{\in}\m{A}\m{\varphi}\m{\leftrightarrow}\m{%
\forall}\m{x}\m{(}\m{x}\m{\in}\m{A}\m{\rightarrow}\m{\varphi}\m{)}\m{)}
\endm
\begin{mmraw}%
|- ( A. x e. A ph <-> A. x ( x e. A -> ph ) ) \$.
\end{mmraw}

\noindent Define restricted existential quantification.\index{existential
quantifier ($\exists$)!restricted}  Enderton, p.~22.

\vskip 0.5ex
\setbox\startprefix=\hbox{\tt \ \ df-rex\ \$a\ }
\setbox\contprefix=\hbox{\tt \ \ \ \ \ \ \ \ \ \ \ \ }
\startm
\m{\vdash}\m{(}\m{\exists}\m{x}\m{\in}\m{A}\m{\varphi}\m{\leftrightarrow}\m{%
\exists}\m{x}\m{(}\m{x}\m{\in}\m{A}\m{\wedge}\m{\varphi}\m{)}\m{)}
\endm
\begin{mmraw}%
|- ( E. x e. A ph <-> E. x ( x e. A \TAND ph ) ) \$.
\end{mmraw}

\noindent Define the universal class\index{universal class ($V$)}.  Definition
5.20, p.~21, of Takeuti and Zaring.\label{df-v}

\vskip 0.5ex
\setbox\startprefix=\hbox{\tt \ \ df-v\ \$a\ }
\setbox\contprefix=\hbox{\tt \ \ \ \ \ \ \ \ \ \ }
\startm
\m{\vdash}\m{{\rm V}}\m{=}\m{\{}\m{x}\m{|}\m{x}\m{=}\m{x}\m{\}}
\endm
\begin{mmraw}%
|- {\char`\_}V = \{ x | x = x \} \$.
\end{mmraw}

\noindent Define the subclass\index{subclass}\index{subset} relationship
between two classes (called the subset relation if the classes are sets i.e.\
are not proper).  Definition 5.9 of Takeuti and Zaring, p.~17.\label{df-ss}

\vskip 0.5ex
\setbox\startprefix=\hbox{\tt \ \ df-ss\ \$a\ }
\setbox\contprefix=\hbox{\tt \ \ \ \ \ \ \ \ \ \ \ }
\startm
\m{\vdash}\m{(}\m{A}\m{\subseteq}\m{B}\m{\leftrightarrow}\m{\forall}\m{x}\m{(}
\m{x}\m{\in}\m{A}\m{\rightarrow}\m{x}\m{\in}\m{B}\m{)}\m{)}
\endm
\begin{mmraw}%
|- ( A C\_ B <-> A. x ( x e. A -> x e. B ) ) \$.
\end{mmraw}

\noindent Define the union\index{union} of two classes.  Definition 5.6 of Takeuti and Zaring,
p.~16.\label{df-un}

\vskip 0.5ex
\setbox\startprefix=\hbox{\tt \ \ df-un\ \$a\ }
\setbox\contprefix=\hbox{\tt \ \ \ \ \ \ \ \ \ \ \ }
\startm
\m{\vdash}\m{(}\m{A}\m{\cup}\m{B}\m{)}\m{=}\m{\{}\m{x}\m{|}\m{(}\m{x}\m{\in}
\m{A}\m{\vee}\m{x}\m{\in}\m{B}\m{)}\m{\}}
\endm
\begin{mmraw}%
( A u. B ) = \{ x | ( x e. A \TOR x e. B ) \} \$.
\end{mmraw}

\noindent Define the intersection\index{intersection} of two classes.  Definition 5.6 of
Takeuti and Zaring, p.~16.\label{df-in}

\vskip 0.5ex
\setbox\startprefix=\hbox{\tt \ \ df-in\ \$a\ }
\setbox\contprefix=\hbox{\tt \ \ \ \ \ \ \ \ \ \ \ }
\startm
\m{\vdash}\m{(}\m{A}\m{\cap}\m{B}\m{)}\m{=}\m{\{}\m{x}\m{|}\m{(}\m{x}\m{\in}
\m{A}\m{\wedge}\m{x}\m{\in}\m{B}\m{)}\m{\}}
\endm
% Caret ^ requires special treatment
\begin{mmraw}%
|- ( A i\^{}i B ) = \{ x | ( x e. A \TAND x e. B ) \} \$.
\end{mmraw}

\noindent Define class difference\index{class difference}\index{set difference}.
Definition 5.12 of Takeuti and Zaring, p.~20.  Several notations are used in
the literature; we chose the $\setminus$ convention instead of a minus sign to
reserve the latter for later use in, e.g., arithmetic.\label{df-dif}

\vskip 0.5ex
\setbox\startprefix=\hbox{\tt \ \ df-dif\ \$a\ }
\setbox\contprefix=\hbox{\tt \ \ \ \ \ \ \ \ \ \ \ \ }
\startm
\m{\vdash}\m{(}\m{A}\m{\setminus}\m{B}\m{)}\m{=}\m{\{}\m{x}\m{|}\m{(}\m{x}\m{
\in}\m{A}\m{\wedge}\m{\lnot}\m{x}\m{\in}\m{B}\m{)}\m{\}}
\endm
\begin{mmraw}%
( A \SLASH B ) = \{ x | ( x e. A \TAND -. x e. B ) \} \$.
\end{mmraw}

\noindent Define the empty or null set\index{empty set}\index{null set}.
Compare  Definition 5.14 of Takeuti and Zaring, p.~20.\label{df-nul}

\vskip 0.5ex
\setbox\startprefix=\hbox{\tt \ \ df-nul\ \$a\ }
\setbox\contprefix=\hbox{\tt \ \ \ \ \ \ \ \ \ \ }
\startm
\m{\vdash}\m{\varnothing}\m{=}\m{(}\m{{\rm V}}\m{\setminus}\m{{\rm V}}\m{)}
\endm
\begin{mmraw}%
|- (/) = ( {\char`\_}V \SLASH {\char`\_}V ) \$.
\end{mmraw}

\noindent Define power class\index{power set}\index{power class}.  Definition 5.10 of
Takeuti and Zaring, p.~17, but we also let it apply to proper classes.  (Note
that \verb$~P$ is the symbol for calligraphic P, the tilde
suggesting ``curly;'' see Appendix~\ref{ASCII}.)\label{df-pw}

\vskip 0.5ex
\setbox\startprefix=\hbox{\tt \ \ df-pw\ \$a\ }
\setbox\contprefix=\hbox{\tt \ \ \ \ \ \ \ \ \ \ \ }
\startm
\m{\vdash}\m{{\cal P}}\m{A}\m{=}\m{\{}\m{x}\m{|}\m{x}\m{\subseteq}\m{A}\m{\}}
\endm
% Special incantation required to put ~ into the text
\begin{mmraw}%
|- \char`\~P~A = \{ x | x C\_ A \} \$.
\end{mmraw}

\noindent Define the singleton of a class\index{singleton}.  Definition 7.1 of
Quine, p.~48.  It is well-defined for proper classes, although
it is not very meaningful in this case, where it evaluates to the empty
set.

\vskip 0.5ex
\setbox\startprefix=\hbox{\tt \ \ df-sn\ \$a\ }
\setbox\contprefix=\hbox{\tt \ \ \ \ \ \ \ \ \ \ \ }
\startm
\m{\vdash}\m{\{}\m{A}\m{\}}\m{=}\m{\{}\m{x}\m{|}\m{x}\m{=}\m{A}\m{\}}
\endm
\begin{mmraw}%
|- \{ A \} = \{ x | x = A \} \$.
\end{mmraw}%

\noindent Define an unordered pair of classes\index{unordered pair}\index{pair}.  Definition
7.1 of Quine, p.~48.

\vskip 0.5ex
\setbox\startprefix=\hbox{\tt \ \ df-pr\ \$a\ }
\setbox\contprefix=\hbox{\tt \ \ \ \ \ \ \ \ \ \ \ }
\startm
\m{\vdash}\m{\{}\m{A}\m{,}\m{B}\m{\}}\m{=}\m{(}\m{\{}\m{A}\m{\}}\m{\cup}\m{\{}
\m{B}\m{\}}\m{)}
\endm
\begin{mmraw}%
|- \{ A , B \} = ( \{ A \} u. \{ B \} ) \$.
\end{mmraw}

\noindent Define an unordered triple of classes\index{unordered triple}.  Definition of
Enderton, p.~19.

\vskip 0.5ex
\setbox\startprefix=\hbox{\tt \ \ df-tp\ \$a\ }
\setbox\contprefix=\hbox{\tt \ \ \ \ \ \ \ \ \ \ \ }
\startm
\m{\vdash}\m{\{}\m{A}\m{,}\m{B}\m{,}\m{C}\m{\}}\m{=}\m{(}\m{\{}\m{A}\m{,}\m{B}
\m{\}}\m{\cup}\m{\{}\m{C}\m{\}}\m{)}
\endm
\begin{mmraw}%
|- \{ A , B , C \} = ( \{ A , B \} u. \{ C \} ) \$.
\end{mmraw}%

\noindent Kuratowski's\index{Kuratowski, Kazimierz} ordered pair\index{ordered
pair} definition.  Definition 9.1 of Quine, p.~58. For proper classes it is
not meaningful but is well-defined for convenience.  (Note that \verb$<.$
stands for $\langle$ whereas \verb$<$ stands for $<$, and similarly for
\verb$>.$\,.)\label{df-op}

\vskip 0.5ex
\setbox\startprefix=\hbox{\tt \ \ df-op\ \$a\ }
\setbox\contprefix=\hbox{\tt \ \ \ \ \ \ \ \ \ \ \ }
\startm
\m{\vdash}\m{\langle}\m{A}\m{,}\m{B}\m{\rangle}\m{=}\m{\{}\m{\{}\m{A}\m{\}}
\m{,}\m{\{}\m{A}\m{,}\m{B}\m{\}}\m{\}}
\endm
\begin{mmraw}%
|- <. A , B >. = \{ \{ A \} , \{ A , B \} \} \$.
\end{mmraw}

\noindent Define the union of a class\index{union}.  Definition 5.5, p.~16,
of Takeuti and Zaring.

\vskip 0.5ex
\setbox\startprefix=\hbox{\tt \ \ df-uni\ \$a\ }
\setbox\contprefix=\hbox{\tt \ \ \ \ \ \ \ \ \ \ \ \ }
\startm
\m{\vdash}\m{\bigcup}\m{A}\m{=}\m{\{}\m{x}\m{|}\m{\exists}\m{y}\m{(}\m{x}\m{
\in}\m{y}\m{\wedge}\m{y}\m{\in}\m{A}\m{)}\m{\}}
\endm
\begin{mmraw}%
|- U. A = \{ x | E. y ( x e. y \TAND y e. A ) \} \$.
\end{mmraw}

\noindent Define the intersection\index{intersection} of a class.  Definition 7.35,
p.~44, of Takeuti and Zaring.

\vskip 0.5ex
\setbox\startprefix=\hbox{\tt \ \ df-int\ \$a\ }
\setbox\contprefix=\hbox{\tt \ \ \ \ \ \ \ \ \ \ \ \ }
\startm
\m{\vdash}\m{\bigcap}\m{A}\m{=}\m{\{}\m{x}\m{|}\m{\forall}\m{y}\m{(}\m{y}\m{
\in}\m{A}\m{\rightarrow}\m{x}\m{\in}\m{y}\m{)}\m{\}}
\endm
\begin{mmraw}%
|- |\^{}| A = \{ x | A. y ( y e. A -> x e. y ) \} \$.
\end{mmraw}

\noindent Define a transitive class\index{transitive class}\index{transitive
set}.  This should not be confused with a transitive relation, which is a different
concept.  Definition from p.~71 of Enderton, extended to classes.

\vskip 0.5ex
\setbox\startprefix=\hbox{\tt \ \ df-tr\ \$a\ }
\setbox\contprefix=\hbox{\tt \ \ \ \ \ \ \ \ \ \ \ }
\startm
\m{\vdash}\m{(}\m{\mbox{\rm Tr}}\m{A}\m{\leftrightarrow}\m{\bigcup}\m{A}\m{
\subseteq}\m{A}\m{)}
\endm
\begin{mmraw}%
|- ( Tr A <-> U. A C\_ A ) \$.
\end{mmraw}

\noindent Define a notation for a general binary relation\index{binary
relation}.  Definition 6.18, p.~29, of Takeuti and Zaring, generalized to
arbitrary classes.  This definition is well-defined, although not very
meaningful, when classes $A$ and/or $B$ are proper.\label{dfbr}  The lack of
parentheses (or any other connective) creates no ambiguity since we are defining
an atomic wff.

\vskip 0.5ex
\setbox\startprefix=\hbox{\tt \ \ df-br\ \$a\ }
\setbox\contprefix=\hbox{\tt \ \ \ \ \ \ \ \ \ \ \ }
\startm
\m{\vdash}\m{(}\m{A}\m{\,R}\m{\,B}\m{\leftrightarrow}\m{\langle}\m{A}\m{,}\m{B}
\m{\rangle}\m{\in}\m{R}\m{)}
\endm
\begin{mmraw}%
|- ( A R B <-> <. A , B >. e. R ) \$.
\end{mmraw}

\noindent Define an abstraction class of ordered pairs\index{abstraction
class!of ordered
pairs}.  A special case of Definition 4.16, p.~14, of Takeuti and Zaring.
Note that $ z $ must be distinct from $ x $ and $ y $,
and $ z $ must not occur in $\varphi$, but $ x $ and $ y $ may be
identical and may appear in $\varphi$.

\vskip 0.5ex
\setbox\startprefix=\hbox{\tt \ \ df-opab\ \$a\ }
\setbox\contprefix=\hbox{\tt \ \ \ \ \ \ \ \ \ \ \ \ \ }
\startm
\m{\vdash}\m{\{}\m{\langle}\m{x}\m{,}\m{y}\m{\rangle}\m{|}\m{\varphi}\m{\}}\m{=}
\m{\{}\m{z}\m{|}\m{\exists}\m{x}\m{\exists}\m{y}\m{(}\m{z}\m{=}\m{\langle}\m{x}
\m{,}\m{y}\m{\rangle}\m{\wedge}\m{\varphi}\m{)}\m{\}}
\endm

\begin{mmraw}%
|- \{ <. x , y >. | ph \} = \{ z | E. x E. y ( z =
<. x , y >. /\ ph ) \} \$.
\end{mmraw}

\noindent Define the epsilon relation\index{epsilon relation}.  Similar to Definition
6.22, p.~30, of Takeuti and Zaring.

\vskip 0.5ex
\setbox\startprefix=\hbox{\tt \ \ df-eprel\ \$a\ }
\setbox\contprefix=\hbox{\tt \ \ \ \ \ \ \ \ \ \ \ \ \ \ }
\startm
\m{\vdash}\m{{\rm E}}\m{=}\m{\{}\m{\langle}\m{x}\m{,}\m{y}\m{\rangle}\m{|}\m{x}\m{
\in}\m{y}\m{\}}
\endm
\begin{mmraw}%
|- \_E = \{ <. x , y >. | x e. y \} \$.
\end{mmraw}

\noindent Define a founded relation\index{founded relation}.  $R$ is a founded
relation on $A$ iff\index{iff} (if and only if) each nonempty subset of $A$
has an ``$R$-minimal element.''  Similar to Definition 6.21, p.~30, of
Takeuti and Zaring.

\vskip 0.5ex
\setbox\startprefix=\hbox{\tt \ \ df-fr\ \$a\ }
\setbox\contprefix=\hbox{\tt \ \ \ \ \ \ \ \ \ \ \ }
\startm
\m{\vdash}\m{(}\m{R}\m{\,\mbox{\rm Fr}}\m{\,A}\m{\leftrightarrow}\m{\forall}\m{x}
\m{(}\m{(}\m{x}\m{\subseteq}\m{A}\m{\wedge}\m{\lnot}\m{x}\m{=}\m{\varnothing}
\m{)}\m{\rightarrow}\m{\exists}\m{y}\m{(}\m{y}\m{\in}\m{x}\m{\wedge}\m{(}\m{x}
\m{\cap}\m{\{}\m{z}\m{|}\m{z}\m{\,R}\m{\,y}\m{\}}\m{)}\m{=}\m{\varnothing}\m{)}
\m{)}\m{)}
\endm
\begin{mmraw}%
|- ( R Fr A <-> A. x ( ( x C\_ A \TAND -. x = (/) ) ->
E. y ( y e. x \TAND ( x i\^{}i \{ z | z R y \} ) = (/) ) ) ) \$.
\end{mmraw}

\noindent Define a well-ordering\index{well-ordering}.  $R$ is a well-ordering of $A$ iff
it is founded on $A$ and the elements of $A$ are pairwise $R$-comparable.
Similar to Definition 6.24(2), p.~30, of Takeuti and Zaring.

\vskip 0.5ex
\setbox\startprefix=\hbox{\tt \ \ df-we\ \$a\ }
\setbox\contprefix=\hbox{\tt \ \ \ \ \ \ \ \ \ \ \ }
\startm
\m{\vdash}\m{(}\m{R}\m{\,\mbox{\rm We}}\m{\,A}\m{\leftrightarrow}\m{(}\m{R}\m{\,
\mbox{\rm Fr}}\m{\,A}\m{\wedge}\m{\forall}\m{x}\m{\forall}\m{y}\m{(}\m{(}\m{x}\m{
\in}\m{A}\m{\wedge}\m{y}\m{\in}\m{A}\m{)}\m{\rightarrow}\m{(}\m{x}\m{\,R}\m{\,y}
\m{\vee}\m{x}\m{=}\m{y}\m{\vee}\m{y}\m{\,R}\m{\,x}\m{)}\m{)}\m{)}\m{)}
\endm
\begin{mmraw}%
( R We A <-> ( R Fr A \TAND A. x A. y ( ( x e.
A \TAND y e. A ) -> ( x R y \TOR x = y \TOR y R x ) ) ) ) \$.
\end{mmraw}

\noindent Define the ordinal predicate\index{ordinal predicate}, which is true for a class
that is transitive and is well-ordered by the epsilon relation.  Similar to
definition on p.~468, Bell and Machover.

\vskip 0.5ex
\setbox\startprefix=\hbox{\tt \ \ df-ord\ \$a\ }
\setbox\contprefix=\hbox{\tt \ \ \ \ \ \ \ \ \ \ \ \ }
\startm
\m{\vdash}\m{(}\m{\mbox{\rm Ord}}\m{\,A}\m{\leftrightarrow}\m{(}
\m{\mbox{\rm Tr}}\m{\,A}\m{\wedge}\m{E}\m{\,\mbox{\rm We}}\m{\,A}\m{)}\m{)}
\endm
\begin{mmraw}%
|- ( Ord A <-> ( Tr A \TAND E We A ) ) \$.
\end{mmraw}

\noindent Define the class of all ordinal numbers\index{ordinal number}.  An ordinal number is
a set that satisfies the ordinal predicate.  Definition 7.11 of Takeuti and
Zaring, p.~38.

\vskip 0.5ex
\setbox\startprefix=\hbox{\tt \ \ df-on\ \$a\ }
\setbox\contprefix=\hbox{\tt \ \ \ \ \ \ \ \ \ \ \ }
\startm
\m{\vdash}\m{\,\mbox{\rm On}}\m{=}\m{\{}\m{x}\m{|}\m{\mbox{\rm Ord}}\m{\,x}
\m{\}}
\endm
\begin{mmraw}%
|- On = \{ x | Ord x \} \$.
\end{mmraw}

\noindent Define the limit ordinal predicate\index{limit ordinal}, which is true for a
non-empty ordinal that is not a successor (i.e.\ that is the union of itself).
Compare Bell and Machover, p.~471 and Exercise (1), p.~42 of Takeuti and
Zaring.

\vskip 0.5ex
\setbox\startprefix=\hbox{\tt \ \ df-lim\ \$a\ }
\setbox\contprefix=\hbox{\tt \ \ \ \ \ \ \ \ \ \ \ \ }
\startm
\m{\vdash}\m{(}\m{\mbox{\rm Lim}}\m{\,A}\m{\leftrightarrow}\m{(}\m{\mbox{
\rm Ord}}\m{\,A}\m{\wedge}\m{\lnot}\m{A}\m{=}\m{\varnothing}\m{\wedge}\m{A}
\m{=}\m{\bigcup}\m{A}\m{)}\m{)}
\endm
\begin{mmraw}%
|- ( Lim A <-> ( Ord A \TAND -. A = (/) \TAND A = U. A ) ) \$.
\end{mmraw}

\noindent Define the successor\index{successor} of a class.  Definition 7.22 of Takeuti
and Zaring, p.~41.  Our definition is a generalization to classes, although it
is meaningless when classes are proper.

\vskip 0.5ex
\setbox\startprefix=\hbox{\tt \ \ df-suc\ \$a\ }
\setbox\contprefix=\hbox{\tt \ \ \ \ \ \ \ \ \ \ \ \ }
\startm
\m{\vdash}\m{\,\mbox{\rm suc}}\m{\,A}\m{=}\m{(}\m{A}\m{\cup}\m{\{}\m{A}\m{\}}
\m{)}
\endm
\begin{mmraw}%
|- suc A = ( A u. \{ A \} ) \$.
\end{mmraw}

\noindent Define the class of natural numbers\index{natural number}\index{omega
($\omega$)}.  Compare Bell and Machover, p.~471.\label{dfom}

\vskip 0.5ex
\setbox\startprefix=\hbox{\tt \ \ df-om\ \$a\ }
\setbox\contprefix=\hbox{\tt \ \ \ \ \ \ \ \ \ \ \ }
\startm
\m{\vdash}\m{\omega}\m{=}\m{\{}\m{x}\m{|}\m{(}\m{\mbox{\rm Ord}}\m{\,x}\m{
\wedge}\m{\forall}\m{y}\m{(}\m{\mbox{\rm Lim}}\m{\,y}\m{\rightarrow}\m{x}\m{
\in}\m{y}\m{)}\m{)}\m{\}}
\endm
\begin{mmraw}%
|- om = \{ x | ( Ord x \TAND A. y ( Lim y -> x e. y ) ) \} \$.
\end{mmraw}

\noindent Define the Cartesian product (also called the
cross product)\index{Cartesian product}\index{cross product}
of two classes.  Definition 9.11 of Quine, p.~64.

\vskip 0.5ex
\setbox\startprefix=\hbox{\tt \ \ df-xp\ \$a\ }
\setbox\contprefix=\hbox{\tt \ \ \ \ \ \ \ \ \ \ \ }
\startm
\m{\vdash}\m{(}\m{A}\m{\times}\m{B}\m{)}\m{=}\m{\{}\m{\langle}\m{x}\m{,}\m{y}
\m{\rangle}\m{|}\m{(}\m{x}\m{\in}\m{A}\m{\wedge}\m{y}\m{\in}\m{B}\m{)}\m{\}}
\endm
\begin{mmraw}%
|- ( A X. B ) = \{ <. x , y >. | ( x e. A \TAND y e. B) \} \$.
\end{mmraw}

\noindent Define a relation\index{relation}.  Definition 6.4(1) of Takeuti and
Zaring, p.~23.

\vskip 0.5ex
\setbox\startprefix=\hbox{\tt \ \ df-rel\ \$a\ }
\setbox\contprefix=\hbox{\tt \ \ \ \ \ \ \ \ \ \ \ \ }
\startm
\m{\vdash}\m{(}\m{\mbox{\rm Rel}}\m{\,A}\m{\leftrightarrow}\m{A}\m{\subseteq}
\m{(}\m{{\rm V}}\m{\times}\m{{\rm V}}\m{)}\m{)}
\endm
\begin{mmraw}%
|- ( Rel A <-> A C\_ ( {\char`\_}V X. {\char`\_}V ) ) \$.
\end{mmraw}

\noindent Define the domain\index{domain} of a class.  Definition 6.5(1) of
Takeuti and Zaring, p.~24.

\vskip 0.5ex
\setbox\startprefix=\hbox{\tt \ \ df-dm\ \$a\ }
\setbox\contprefix=\hbox{\tt \ \ \ \ \ \ \ \ \ \ \ }
\startm
\m{\vdash}\m{\,\mbox{\rm dom}}\m{A}\m{=}\m{\{}\m{x}\m{|}\m{\exists}\m{y}\m{
\langle}\m{x}\m{,}\m{y}\m{\rangle}\m{\in}\m{A}\m{\}}
\endm
\begin{mmraw}%
|- dom A = \{ x | E. y <. x , y >. e. A \} \$.
\end{mmraw}

\noindent Define the range\index{range} of a class.  Definition 6.5(2) of
Takeuti and Zaring, p.~24.

\vskip 0.5ex
\setbox\startprefix=\hbox{\tt \ \ df-rn\ \$a\ }
\setbox\contprefix=\hbox{\tt \ \ \ \ \ \ \ \ \ \ \ }
\startm
\m{\vdash}\m{\,\mbox{\rm ran}}\m{A}\m{=}\m{\{}\m{y}\m{|}\m{\exists}\m{x}\m{
\langle}\m{x}\m{,}\m{y}\m{\rangle}\m{\in}\m{A}\m{\}}
\endm
\begin{mmraw}%
|- ran A = \{ y | E. x <. x , y >. e. A \} \$.
\end{mmraw}

\noindent Define the restriction\index{restriction} of a class.  Definition
6.6(1) of Takeuti and Zaring, p.~24.

\vskip 0.5ex
\setbox\startprefix=\hbox{\tt \ \ df-res\ \$a\ }
\setbox\contprefix=\hbox{\tt \ \ \ \ \ \ \ \ \ \ \ \ }
\startm
\m{\vdash}\m{(}\m{A}\m{\restriction}\m{B}\m{)}\m{=}\m{(}\m{A}\m{\cap}\m{(}\m{B}
\m{\times}\m{{\rm V}}\m{)}\m{)}
\endm
\begin{mmraw}%
|- ( A |` B ) = ( A i\^{}i ( B X. {\char`\_}V ) ) \$.
\end{mmraw}

\noindent Define the image\index{image} of a class.  Definition 6.6(2) of
Takeuti and Zaring, p.~24.

\vskip 0.5ex
\setbox\startprefix=\hbox{\tt \ \ df-ima\ \$a\ }
\setbox\contprefix=\hbox{\tt \ \ \ \ \ \ \ \ \ \ \ \ }
\startm
\m{\vdash}\m{(}\m{A}\m{``}\m{B}\m{)}\m{=}\m{\,\mbox{\rm ran}}\m{\,(}\m{A}\m{
\restriction}\m{B}\m{)}
\endm
\begin{mmraw}%
|- ( A " B ) = ran ( A |` B ) \$.
\end{mmraw}

\noindent Define the composition\index{composition} of two classes.  Definition 6.6(3) of
Takeuti and Zaring, p.~24.

\vskip 0.5ex
\setbox\startprefix=\hbox{\tt \ \ df-co\ \$a\ }
\setbox\contprefix=\hbox{\tt \ \ \ \ \ \ \ \ \ \ \ \ }
\startm
\m{\vdash}\m{(}\m{A}\m{\circ}\m{B}\m{)}\m{=}\m{\{}\m{\langle}\m{x}\m{,}\m{y}\m{
\rangle}\m{|}\m{\exists}\m{z}\m{(}\m{\langle}\m{x}\m{,}\m{z}\m{\rangle}\m{\in}
\m{B}\m{\wedge}\m{\langle}\m{z}\m{,}\m{y}\m{\rangle}\m{\in}\m{A}\m{)}\m{\}}
\endm
\begin{mmraw}%
|- ( A o. B ) = \{ <. x , y >. | E. z ( <. x , z
>. e. B \TAND <. z , y >. e. A ) \} \$.
\end{mmraw}

\noindent Define a function\index{function}.  Definition 6.4(4) of Takeuti and
Zaring, p.~24.

\vskip 0.5ex
\setbox\startprefix=\hbox{\tt \ \ df-fun\ \$a\ }
\setbox\contprefix=\hbox{\tt \ \ \ \ \ \ \ \ \ \ \ \ }
\startm
\m{\vdash}\m{(}\m{\mbox{\rm Fun}}\m{\,A}\m{\leftrightarrow}\m{(}
\m{\mbox{\rm Rel}}\m{\,A}\m{\wedge}
\m{\forall}\m{x}\m{\exists}\m{z}\m{\forall}\m{y}\m{(}
\m{\langle}\m{x}\m{,}\m{y}\m{\rangle}\m{\in}\m{A}\m{\rightarrow}\m{y}\m{=}\m{z}
\m{)}\m{)}\m{)}
\endm
\begin{mmraw}%
|- ( Fun A <-> ( Rel A /\ A. x E. z A. y ( <. x
   , y >. e. A -> y = z ) ) ) \$.
\end{mmraw}

\noindent Define a function with domain.  Definition 6.15(1) of Takeuti and
Zaring, p.~27.

\vskip 0.5ex
\setbox\startprefix=\hbox{\tt \ \ df-fn\ \$a\ }
\setbox\contprefix=\hbox{\tt \ \ \ \ \ \ \ \ \ \ \ }
\startm
\m{\vdash}\m{(}\m{A}\m{\,\mbox{\rm Fn}}\m{\,B}\m{\leftrightarrow}\m{(}
\m{\mbox{\rm Fun}}\m{\,A}\m{\wedge}\m{\mbox{\rm dom}}\m{\,A}\m{=}\m{B}\m{)}
\m{)}
\endm
\begin{mmraw}%
|- ( A Fn B <-> ( Fun A \TAND dom A = B ) ) \$.
\end{mmraw}

\noindent Define a function with domain and co-domain.  Definition 6.15(3)
of Takeuti and Zaring, p.~27.

\vskip 0.5ex
\setbox\startprefix=\hbox{\tt \ \ df-f\ \$a\ }
\setbox\contprefix=\hbox{\tt \ \ \ \ \ \ \ \ \ \ }
\startm
\m{\vdash}\m{(}\m{F}\m{:}\m{A}\m{\longrightarrow}\m{B}\m{
\leftrightarrow}\m{(}\m{F}\m{\,\mbox{\rm Fn}}\m{\,A}\m{\wedge}\m{
\mbox{\rm ran}}\m{\,F}\m{\subseteq}\m{B}\m{)}\m{)}
\endm
\begin{mmraw}%
|- ( F : A --> B <-> ( F Fn A \TAND ran F C\_ B ) ) \$.
\end{mmraw}

\noindent Define a one-to-one function\index{one-to-one function}.  Compare
Definition 6.15(5) of Takeuti and Zaring, p.~27.

\vskip 0.5ex
\setbox\startprefix=\hbox{\tt \ \ df-f1\ \$a\ }
\setbox\contprefix=\hbox{\tt \ \ \ \ \ \ \ \ \ \ \ }
\startm
\m{\vdash}\m{(}\m{F}\m{:}\m{A}\m{
\raisebox{.5ex}{${\textstyle{\:}_{\mbox{\footnotesize\rm
1\tt -\rm 1}}}\atop{\textstyle{
\longrightarrow}\atop{\textstyle{}^{\mbox{\footnotesize\rm {\ }}}}}$}
}\m{B}
\m{\leftrightarrow}\m{(}\m{F}\m{:}\m{A}\m{\longrightarrow}\m{B}
\m{\wedge}\m{\forall}\m{y}\m{\exists}\m{z}\m{\forall}\m{x}\m{(}\m{\langle}\m{x}
\m{,}\m{y}\m{\rangle}\m{\in}\m{F}\m{\rightarrow}\m{x}\m{=}\m{z}\m{)}\m{)}\m{)}
\endm
\begin{mmraw}%
|- ( F : A -1-1-> B <-> ( F : A --> B \TAND
   A. y E. z A. x ( <. x , y >. e. F -> x = z ) ) ) \$.
\end{mmraw}

\noindent Define an onto function\index{onto function}.  Definition 6.15(4) of Takeuti and
Zaring, p.~27.

\vskip 0.5ex
\setbox\startprefix=\hbox{\tt \ \ df-fo\ \$a\ }
\setbox\contprefix=\hbox{\tt \ \ \ \ \ \ \ \ \ \ \ }
\startm
\m{\vdash}\m{(}\m{F}\m{:}\m{A}\m{
\raisebox{.5ex}{${\textstyle{\:}_{\mbox{\footnotesize\rm
{\ }}}}\atop{\textstyle{
\longrightarrow}\atop{\textstyle{}^{\mbox{\footnotesize\rm onto}}}}$}
}\m{B}
\m{\leftrightarrow}\m{(}\m{F}\m{\,\mbox{\rm Fn}}\m{\,A}\m{\wedge}
\m{\mbox{\rm ran}}\m{\,F}\m{=}\m{B}\m{)}\m{)}
\endm
\begin{mmraw}%
|- ( F : A -onto-> B <-> ( F Fn A /\ ran F = B ) ) \$.
\end{mmraw}

\noindent Define a one-to-one, onto function.  Compare Definition 6.15(6) of
Takeuti and Zaring, p.~27.

\vskip 0.5ex
\setbox\startprefix=\hbox{\tt \ \ df-f1o\ \$a\ }
\setbox\contprefix=\hbox{\tt \ \ \ \ \ \ \ \ \ \ \ \ }
\startm
\m{\vdash}\m{(}\m{F}\m{:}\m{A}
\m{
\raisebox{.5ex}{${\textstyle{\:}_{\mbox{\footnotesize\rm
1\tt -\rm 1}}}\atop{\textstyle{
\longrightarrow}\atop{\textstyle{}^{\mbox{\footnotesize\rm onto}}}}$}
}
\m{B}
\m{\leftrightarrow}\m{(}\m{F}\m{:}\m{A}
\m{
\raisebox{.5ex}{${\textstyle{\:}_{\mbox{\footnotesize\rm
1\tt -\rm 1}}}\atop{\textstyle{
\longrightarrow}\atop{\textstyle{}^{\mbox{\footnotesize\rm {\ }}}}}$}
}
\m{B}\m{\wedge}\m{F}\m{:}\m{A}
\m{
\raisebox{.5ex}{${\textstyle{\:}_{\mbox{\footnotesize\rm
{\ }}}}\atop{\textstyle{
\longrightarrow}\atop{\textstyle{}^{\mbox{\footnotesize\rm onto}}}}$}
}
\m{B}\m{)}\m{)}
\endm
\begin{mmraw}%
|- ( F : A -1-1-onto-> B <-> ( F : A -1-1-> B? \TAND F : A -onto-> B ) ) \$.?
\end{mmraw}

\noindent Define the value of a function\index{function value}.  This
definition applies to any class and evaluates to the empty set when it is not
meaningful. Note that $ F`A$ means the same thing as the more familiar $ F(A)$
notation for a function's value at $A$.  The $ F`A$ notation is common in
formal set theory.\label{df-fv}

\vskip 0.5ex
\setbox\startprefix=\hbox{\tt \ \ df-fv\ \$a\ }
\setbox\contprefix=\hbox{\tt \ \ \ \ \ \ \ \ \ \ \ }
\startm
\m{\vdash}\m{(}\m{F}\m{`}\m{A}\m{)}\m{=}\m{\bigcup}\m{\{}\m{x}\m{|}\m{(}\m{F}%
\m{``}\m{\{}\m{A}\m{\}}\m{)}\m{=}\m{\{}\m{x}\m{\}}\m{\}}
\endm
\begin{mmraw}%
|- ( F ` A ) = U. \{ x | ( F " \{ A \} ) = \{ x \} \} \$.
\end{mmraw}

\noindent Define the result of an operation.\index{operation}  Here, $F$ is
     an operation on two
     values (such as $+$ for real numbers).   This is defined for proper
     classes $A$ and $B$ even though not meaningful in that case.  However,
     the definition can be meaningful when $F$ is a proper class.\label{dfopr}

\vskip 0.5ex
\setbox\startprefix=\hbox{\tt \ \ df-opr\ \$a\ }
\setbox\contprefix=\hbox{\tt \ \ \ \ \ \ \ \ \ \ \ \ }
\startm
\m{\vdash}\m{(}\m{A}\m{\,F}\m{\,B}\m{)}\m{=}\m{(}\m{F}\m{`}\m{\langle}\m{A}%
\m{,}\m{B}\m{\rangle}\m{)}
\endm
\begin{mmraw}%
|- ( A F B ) = ( F ` <. A , B >. ) \$.
\end{mmraw}

\section{Tricks of the Trade}\label{tricks}

In the \texttt{set.mm}\index{set theory database (\texttt{set.mm})} database our goal
was usually to conform to modern notation.  However in some cases the
relationship to standard textbook language may be obscured by several
unconventional devices we used to simplify the development and to take
advantage of the Metamath language.  In this section we will describe some
common conventions used in \texttt{set.mm}.

\begin{itemize}
\item
The turnstile\index{turnstile ({$\,\vdash$})} symbol, $\vdash$, meaning ``it
is provable that,'' is the first token of all assertions and hypotheses that
aren't syntax constructions.  This is a standard convention in logic.  (We
mentioned this earlier, but this symbol is bothersome to some people without a
logic background.  It has no deeper meaning but just provides us with a way to
distinguish syntax constructions from ordinary mathematical statements.)

\item
A hypothesis of the form

\vskip 1ex
\setbox\startprefix=\hbox{\tt \ \ \ \ \ \ \ \ \ \$e\ }
\setbox\contprefix=\hbox{\tt \ \ \ \ \ \ \ \ \ \ }
\startm
\m{\vdash}\m{(}\m{\varphi}\m{\rightarrow}\m{\forall}\m{x}\m{\varphi}\m{)}
\endm
\vskip 1ex

should be read ``assume variable $x$ is (effectively) not free in wff
$\varphi$.''\index{effectively not free}
Literally, this says ``assume it is provable that $\varphi \rightarrow \forall
x\, \varphi$.''  This device lets us avoid the complexities associated with
the standard treatment of free and bound variables.
%% Uncomment this when uncommenting section {formalspec} below
The footnote on p.~\pageref{effectivelybound} discusses this further.

\item
A statement of one of the forms

\vskip 1ex
\setbox\startprefix=\hbox{\tt \ \ \ \ \ \ \ \ \ \$a\ }
\setbox\contprefix=\hbox{\tt \ \ \ \ \ \ \ \ \ \ }
\startm
\m{\vdash}\m{(}\m{\lnot}\m{\forall}\m{x}\m{\,x}\m{=}\m{y}\m{\rightarrow}
\m{\ldots}\m{)}
\endm
\setbox\startprefix=\hbox{\tt \ \ \ \ \ \ \ \ \ \$p\ }
\setbox\contprefix=\hbox{\tt \ \ \ \ \ \ \ \ \ \ }
\startm
\m{\vdash}\m{(}\m{\lnot}\m{\forall}\m{x}\m{\,x}\m{=}\m{y}\m{\rightarrow}
\m{\ldots}\m{)}
\endm
\vskip 1ex

should be read ``if $x$ and $y$ are distinct variables, then...''  This
antecedent provides us with a technical device to avoid the need for the
\texttt{\$d} statement early in our development of predicate calculus,
permitting symbol manipulations to be as conceptually simple as those in
propositional calculus.  However, the \texttt{\$d} statement eventually
becomes a requirement, and after that this device is rarely used.

\item
The statement

\vskip 1ex
\setbox\startprefix=\hbox{\tt \ \ \ \ \ \ \ \ \ \$d\ }
\setbox\contprefix=\hbox{\tt \ \ \ \ \ \ \ \ \ \ }
\startm
\m{x}\m{\,y}
\endm
\vskip 1ex

should be read ``assume $x$ and $y$ are distinct variables.''

\item
The statement

\vskip 1ex
\setbox\startprefix=\hbox{\tt \ \ \ \ \ \ \ \ \ \$d\ }
\setbox\contprefix=\hbox{\tt \ \ \ \ \ \ \ \ \ \ }
\startm
\m{x}\m{\,\varphi}
\endm
\vskip 1ex

should be read ``assume $x$ does not occur in $\varphi$.''

\item
The statement

\vskip 1ex
\setbox\startprefix=\hbox{\tt \ \ \ \ \ \ \ \ \ \$d\ }
\setbox\contprefix=\hbox{\tt \ \ \ \ \ \ \ \ \ \ }
\startm
\m{x}\m{\,A}
\endm
\vskip 1ex

should be read ``assume variable $x$ does not occur in class $A$.''

\item
The restriction and hypothesis group

\vskip 1ex
\setbox\startprefix=\hbox{\tt \ \ \ \ \ \ \ \ \ \$d\ }
\setbox\contprefix=\hbox{\tt \ \ \ \ \ \ \ \ \ \ }
\startm
\m{x}\m{\,A}
\endm
\setbox\startprefix=\hbox{\tt \ \ \ \ \ \ \ \ \ \$d\ }
\setbox\contprefix=\hbox{\tt \ \ \ \ \ \ \ \ \ \ }
\startm
\m{x}\m{\,\psi}
\endm
\setbox\startprefix=\hbox{\tt \ \ \ \ \ \ \ \ \ \$e\ }
\setbox\contprefix=\hbox{\tt \ \ \ \ \ \ \ \ \ \ }
\startm
\m{\vdash}\m{(}\m{x}\m{=}\m{A}\m{\rightarrow}\m{(}\m{\varphi}\m{\leftrightarrow}
\m{\psi}\m{)}\m{)}
\endm
\vskip 1ex

is frequently used in place of explicit substitution, meaning ``assume
$\psi$ results from the proper substitution of $A$ for $x$ in
$\varphi$.''  Sometimes ``\texttt{\$e} $\vdash ( \psi \rightarrow
\forall x \, \psi )$'' is used instead of ``\texttt{\$d} $x\, \psi $,''
which requires only that $x$ be effectively not free in $\varphi$ but
not necessarily absent from it.  The use of implicit
substitution\index{substitution!implicit} is partly a matter of personal
style, although it may make proofs somewhat shorter than would be the
case with explicit substitution.

\item
The hypothesis


\vskip 1ex
\setbox\startprefix=\hbox{\tt \ \ \ \ \ \ \ \ \ \$e\ }
\setbox\contprefix=\hbox{\tt \ \ \ \ \ \ \ \ \ \ }
\startm
\m{\vdash}\m{A}\m{\in}\m{{\rm V}}
\endm
\vskip 1ex

should be read ``assume class $A$ is a set (i.e.\ exists).''
This is a convenient convention used by Quine.

\item
The restriction and hypothesis

\vskip 1ex
\setbox\startprefix=\hbox{\tt \ \ \ \ \ \ \ \ \ \$d\ }
\setbox\contprefix=\hbox{\tt \ \ \ \ \ \ \ \ \ \ }
\startm
\m{x}\m{\,y}
\endm
\setbox\startprefix=\hbox{\tt \ \ \ \ \ \ \ \ \ \$e\ }
\setbox\contprefix=\hbox{\tt \ \ \ \ \ \ \ \ \ \ }
\startm
\m{\vdash}\m{(}\m{y}\m{\in}\m{A}\m{\rightarrow}\m{\forall}\m{x}\m{\,y}
\m{\in}\m{A}\m{)}
\endm
\vskip 1ex

should be read ``assume variable $x$ is
(effectively) not free in class $A$.''

\end{itemize}

\section{A Theorem Sampler}\label{sometheorems}

In this section we list some of the more important theorems that are proved in
the \texttt{set.mm} database, and they illustrate the kinds of things that can be
done with Metamath.  While all of these facts are well-known results,
Metamath offers the advantage of easily allowing you to trace their
derivation back to axioms.  Our intent here is not to try to explain the
details or motivation; for this we refer you to the textbooks that are
mentioned in the descriptions.  (The \texttt{set.mm} file has bibliographic
references for the text references.)  Their proofs often embody important
concepts you may wish to explore with the Metamath program (see
Section~\ref{exploring}).  All the symbols that are used here are defined in
Section~\ref{hierarchy}.  For brevity we haven't included the \texttt{\$d}
restrictions or \texttt{\$f} hypotheses for these theorems; when you are
uncertain consult the \texttt{set.mm} database.

We start with \texttt{syl} (principle of the syllogism).
In \textit{Principia Mathematica}
Whitehead and Russell call this ``the principle of the
syllogism... because... the syllogism in Barbara is derived from them''
\cite[quote after Theorem *2.06 p.~101]{PM}.
Some authors call this law a ``hypothetical syllogism.''
At the time of this writing it is the most commonly referenced assertion in the
\texttt{set.mm} database.\footnote{
The Metamath program command \texttt{show usage}
shows the number of references.
}

\vskip 2ex
\noindent Theorem syl (principle of the syllogism)\index{Syllogism}%
\index{\texttt{syl}}\label{syl}.

\vskip 0.5ex
\setbox\startprefix=\hbox{\tt \ \ syl.1\ \$e\ }
\setbox\contprefix=\hbox{\tt \ \ \ \ \ \ \ \ \ \ \ }
\startm
\m{\vdash}\m{(}\m{\varphi}\m{ \rightarrow }\m{\psi}\m{)}
\endm
\setbox\startprefix=\hbox{\tt \ \ syl.2\ \$e\ }
\setbox\contprefix=\hbox{\tt \ \ \ \ \ \ \ \ \ \ \ }
\startm
\m{\vdash}\m{(}\m{\psi}\m{ \rightarrow }\m{\chi}\m{)}
\endm
\setbox\startprefix=\hbox{\tt \ \ syl\ \$p\ }
\setbox\contprefix=\hbox{\tt \ \ \ \ \ \ \ \ \ }
\startm
\m{\vdash}\m{(}\m{\varphi}\m{ \rightarrow }\m{\chi}\m{)}
\endm
\vskip 2ex

The following theorem is not very deep but provides us with a notational device
that is frequently used.  It allows us to use the expression ``$A \in V$'' as
a compact way of saying that class $A$ exists, i.e.\ is a set.

\vskip 2ex
\noindent Two ways to say ``$A$ is a set'':  $A$ is a member of the universe
$V$ if and only if $A$ exists (i.e.\ there exists a set equal to $A$).
Theorem 6.9 of Quine, p. 43.

\vskip 0.5ex
\setbox\startprefix=\hbox{\tt \ \ isset\ \$p\ }
\setbox\contprefix=\hbox{\tt \ \ \ \ \ \ \ \ \ \ \ }
\startm
\m{\vdash}\m{(}\m{A}\m{\in}\m{{\rm V}}\m{\leftrightarrow}\m{\exists}\m{x}\m{\,x}\m{=}
\m{A}\m{)}
\endm
\vskip 1ex

Next we prove the axioms of standard ZF set theory that were missing from our
axiom system.  From our point of view they are theorems since they
can be derived from the other axioms.

\vskip 2ex
\noindent Axiom of Separation\index{Axiom of Separation}
(Aussonderung)\index{Aussonderung} proved from the other axioms of ZF set
theory.  Compare Exercise 4 of Takeuti and Zaring, p.~22.

\vskip 0.5ex
\setbox\startprefix=\hbox{\tt \ \ inex1.1\ \$e\ }
\setbox\contprefix=\hbox{\tt \ \ \ \ \ \ \ \ \ \ \ \ \ \ \ }
\startm
\m{\vdash}\m{A}\m{\in}\m{{\rm V}}
\endm
\setbox\startprefix=\hbox{\tt \ \ inex\ \$p\ }
\setbox\contprefix=\hbox{\tt \ \ \ \ \ \ \ \ \ \ \ \ \ }
\startm
\m{\vdash}\m{(}\m{A}\m{\cap}\m{B}\m{)}\m{\in}\m{{\rm V}}
\endm
\vskip 1ex

\noindent Axiom of the Null Set\index{Axiom of the Null Set} proved from the
other axioms of ZF set theory. Corollary 5.16 of Takeuti and Zaring, p.~20.

\vskip 0.5ex
\setbox\startprefix=\hbox{\tt \ \ 0ex\ \$p\ }
\setbox\contprefix=\hbox{\tt \ \ \ \ \ \ \ \ \ \ \ \ }
\startm
\m{\vdash}\m{\varnothing}\m{\in}\m{{\rm V}}
\endm
\vskip 1ex

\noindent The Axiom of Pairing\index{Axiom of Pairing} proved from the other
axioms of ZF set theory.  Theorem 7.13 of Quine, p.~51.
\vskip 0.5ex
\setbox\startprefix=\hbox{\tt \ \ prex\ \$p\ }
\setbox\contprefix=\hbox{\tt \ \ \ \ \ \ \ \ \ \ \ \ \ \ }
\startm
\m{\vdash}\m{\{}\m{A}\m{,}\m{B}\m{\}}\m{\in}\m{{\rm V}}
\endm
\vskip 2ex

Next we will list some famous or important theorems that are proved in
the \texttt{set.mm} database.  None of them except \texttt{omex}
require the Axiom of Infinity, as you can verify with the \texttt{show
trace{\char`\_}back} Metamath command.

\vskip 2ex
\noindent The resolution of Russell's paradox\index{Russell's paradox}.  There
exists no set containing the set of all sets which are not members of
themselves.  Proposition 4.14 of Takeuti and Zaring, p.~14.

\vskip 0.5ex
\setbox\startprefix=\hbox{\tt \ \ ru\ \$p\ }
\setbox\contprefix=\hbox{\tt \ \ \ \ \ \ \ \ }
\startm
\m{\vdash}\m{\lnot}\m{\exists}\m{x}\m{\,x}\m{=}\m{\{}\m{y}\m{|}\m{\lnot}\m{y}
\m{\in}\m{y}\m{\}}
\endm
\vskip 1ex

\noindent Cantor's theorem\index{Cantor's theorem}.  No set can be mapped onto
its power set.  Compare Theorem 6B(b) of Enderton, p.~132.

\vskip 0.5ex
\setbox\startprefix=\hbox{\tt \ \ canth.1\ \$e\ }
\setbox\contprefix=\hbox{\tt \ \ \ \ \ \ \ \ \ \ \ \ \ }
\startm
\m{\vdash}\m{A}\m{\in}\m{{\rm V}}
\endm
\setbox\startprefix=\hbox{\tt \ \ canth\ \$p\ }
\setbox\contprefix=\hbox{\tt \ \ \ \ \ \ \ \ \ \ \ }
\startm
\m{\vdash}\m{\lnot}\m{F}\m{:}\m{A}\m{\raisebox{.5ex}{${\textstyle{\:}_{
\mbox{\footnotesize\rm {\ }}}}\atop{\textstyle{\longrightarrow}\atop{
\textstyle{}^{\mbox{\footnotesize\rm onto}}}}$}}\m{{\cal P}}\m{A}
\endm
\vskip 1ex

\noindent The Burali-Forti paradox\index{Burali-Forti paradox}.  No set
contains all ordinal numbers. Enderton, p.~194.  (Burali-Forti was one person,
not two.)

\vskip 0.5ex
\setbox\startprefix=\hbox{\tt \ \ onprc\ \$p\ }
\setbox\contprefix=\hbox{\tt \ \ \ \ \ \ \ \ \ \ \ \ }
\startm
\m{\vdash}\m{\lnot}\m{\mbox{\rm On}}\m{\in}\m{{\rm V}}
\endm
\vskip 1ex

\noindent Peano's postulates\index{Peano's postulates} for arithmetic.
Proposition 7.30 of Takeuti and Zaring, pp.~42--43.  The objects being
described are the members of $\omega$ i.e.\ the natural numbers 0, 1,
2,\ldots.  The successor\index{successor} operation suc means ``plus
one.''  \texttt{peano1} says that 0 (which is defined as the empty set)
is a natural number.  \texttt{peano2} says that if $A$ is a natural
number, so is $A+1$.  \texttt{peano3} says that 0 is not the successor
of any natural number.  \texttt{peano4} says that two natural numbers
are equal if and only if their successors are equal.  \texttt{peano5} is
essentially the same as mathematical induction.

\vskip 1ex
\setbox\startprefix=\hbox{\tt \ \ peano1\ \$p\ }
\setbox\contprefix=\hbox{\tt \ \ \ \ \ \ \ \ \ \ \ \ }
\startm
\m{\vdash}\m{\varnothing}\m{\in}\m{\omega}
\endm
\vskip 1.5ex

\setbox\startprefix=\hbox{\tt \ \ peano2\ \$p\ }
\setbox\contprefix=\hbox{\tt \ \ \ \ \ \ \ \ \ \ \ \ }
\startm
\m{\vdash}\m{(}\m{A}\m{\in}\m{\omega}\m{\rightarrow}\m{{\rm suc}}\m{A}\m{\in}%
\m{\omega}\m{)}
\endm
\vskip 1.5ex

\setbox\startprefix=\hbox{\tt \ \ peano3\ \$p\ }
\setbox\contprefix=\hbox{\tt \ \ \ \ \ \ \ \ \ \ \ \ }
\startm
\m{\vdash}\m{(}\m{A}\m{\in}\m{\omega}\m{\rightarrow}\m{\lnot}\m{{\rm suc}}%
\m{A}\m{=}\m{\varnothing}\m{)}
\endm
\vskip 1.5ex

\setbox\startprefix=\hbox{\tt \ \ peano4\ \$p\ }
\setbox\contprefix=\hbox{\tt \ \ \ \ \ \ \ \ \ \ \ \ }
\startm
\m{\vdash}\m{(}\m{(}\m{A}\m{\in}\m{\omega}\m{\wedge}\m{B}\m{\in}\m{\omega}%
\m{)}\m{\rightarrow}\m{(}\m{{\rm suc}}\m{A}\m{=}\m{{\rm suc}}\m{B}\m{%
\leftrightarrow}\m{A}\m{=}\m{B}\m{)}\m{)}
\endm
\vskip 1.5ex

\setbox\startprefix=\hbox{\tt \ \ peano5\ \$p\ }
\setbox\contprefix=\hbox{\tt \ \ \ \ \ \ \ \ \ \ \ \ }
\startm
\m{\vdash}\m{(}\m{(}\m{\varnothing}\m{\in}\m{A}\m{\wedge}\m{\forall}\m{x}\m{%
\in}\m{\omega}\m{(}\m{x}\m{\in}\m{A}\m{\rightarrow}\m{{\rm suc}}\m{x}\m{\in}%
\m{A}\m{)}\m{)}\m{\rightarrow}\m{\omega}\m{\subseteq}\m{A}\m{)}
\endm
\vskip 1.5ex

\noindent Finite Induction (mathematical induction).\index{finite
induction}\index{mathematical induction} The first hypothesis is the
basis and the second is the induction hypothesis.  Theorem Schema 22 of
Suppes, p.~136.

\vskip 0.5ex
\setbox\startprefix=\hbox{\tt \ \ findes.1\ \$e\ }
\setbox\contprefix=\hbox{\tt \ \ \ \ \ \ \ \ \ \ \ \ \ \ }
\startm
\m{\vdash}\m{[}\m{\varnothing}\m{/}\m{x}\m{]}\m{\varphi}
\endm
\setbox\startprefix=\hbox{\tt \ \ findes.2\ \$e\ }
\setbox\contprefix=\hbox{\tt \ \ \ \ \ \ \ \ \ \ \ \ \ \ }
\startm
\m{\vdash}\m{(}\m{x}\m{\in}\m{\omega}\m{\rightarrow}\m{(}\m{\varphi}\m{%
\rightarrow}\m{[}\m{{\rm suc}}\m{x}\m{/}\m{x}\m{]}\m{\varphi}\m{)}\m{)}
\endm
\setbox\startprefix=\hbox{\tt \ \ findes\ \$p\ }
\setbox\contprefix=\hbox{\tt \ \ \ \ \ \ \ \ \ \ \ \ }
\startm
\m{\vdash}\m{(}\m{x}\m{\in}\m{\omega}\m{\rightarrow}\m{\varphi}\m{)}
\endm
\vskip 1ex

\noindent Transfinite Induction with explicit substitution.  The first
hypothesis is the basis, the second is the induction hypothesis for
successors, and the third is the induction hypothesis for limit
ordinals.  Theorem Schema 4 of Suppes, p. 197.

\vskip 0.5ex
\setbox\startprefix=\hbox{\tt \ \ tfindes.1\ \$e\ }
\setbox\contprefix=\hbox{\tt \ \ \ \ \ \ \ \ \ \ \ \ \ \ \ }
\startm
\m{\vdash}\m{[}\m{\varnothing}\m{/}\m{x}\m{]}\m{\varphi}
\endm
\setbox\startprefix=\hbox{\tt \ \ tfindes.2\ \$e\ }
\setbox\contprefix=\hbox{\tt \ \ \ \ \ \ \ \ \ \ \ \ \ \ \ }
\startm
\m{\vdash}\m{(}\m{x}\m{\in}\m{{\rm On}}\m{\rightarrow}\m{(}\m{\varphi}\m{%
\rightarrow}\m{[}\m{{\rm suc}}\m{x}\m{/}\m{x}\m{]}\m{\varphi}\m{)}\m{)}
\endm
\setbox\startprefix=\hbox{\tt \ \ tfindes.3\ \$e\ }
\setbox\contprefix=\hbox{\tt \ \ \ \ \ \ \ \ \ \ \ \ \ \ \ }
\startm
\m{\vdash}\m{(}\m{{\rm Lim}}\m{y}\m{\rightarrow}\m{(}\m{\forall}\m{x}\m{\in}%
\m{y}\m{\varphi}\m{\rightarrow}\m{[}\m{y}\m{/}\m{x}\m{]}\m{\varphi}\m{)}\m{)}
\endm
\setbox\startprefix=\hbox{\tt \ \ tfindes\ \$p\ }
\setbox\contprefix=\hbox{\tt \ \ \ \ \ \ \ \ \ \ \ \ \ }
\startm
\m{\vdash}\m{(}\m{x}\m{\in}\m{{\rm On}}\m{\rightarrow}\m{\varphi}\m{)}
\endm
\vskip 1ex

\noindent Principle of Transfinite Recursion.\index{transfinite
recursion} Theorem 7.41 of Takeuti and Zaring, p.~47.  Transfinite
recursion is the key theorem that allows arithmetic of ordinals to be
rigorously defined, and has many other important uses as well.
Hypotheses \texttt{tfr.1} and \texttt{tfr.2} specify a certain (proper)
class $ F$.  The complicated definition of $ F$ is not important in
itself; what is important is that there be such an $ F$ with the
required properties, and we show this by displaying $ F$ explicitly.
\texttt{tfr1} states that $ F$ is a function whose domain is the set of
ordinal numbers.  \texttt{tfr2} states that any value of $ F$ is
completely determined by its previous values and the values of an
auxiliary function, $G$.  \texttt{tfr3} states that $ F$ is unique,
i.e.\ it is the only function that satisfies \texttt{tfr1} and
\texttt{tfr2}.  Note that $ f$ is an individual variable like $x$ and
$y$; it is just a mnemonic to remind us that $A$ is a collection of
functions.

\vskip 0.5ex
\setbox\startprefix=\hbox{\tt \ \ tfr.1\ \$e\ }
\setbox\contprefix=\hbox{\tt \ \ \ \ \ \ \ \ \ \ \ }
\startm
\m{\vdash}\m{A}\m{=}\m{\{}\m{f}\m{|}\m{\exists}\m{x}\m{\in}\m{{\rm On}}\m{(}%
\m{f}\m{{\rm Fn}}\m{x}\m{\wedge}\m{\forall}\m{y}\m{\in}\m{x}\m{(}\m{f}\m{`}%
\m{y}\m{)}\m{=}\m{(}\m{G}\m{`}\m{(}\m{f}\m{\restriction}\m{y}\m{)}\m{)}\m{)}%
\m{\}}
\endm
\setbox\startprefix=\hbox{\tt \ \ tfr.2\ \$e\ }
\setbox\contprefix=\hbox{\tt \ \ \ \ \ \ \ \ \ \ \ }
\startm
\m{\vdash}\m{F}\m{=}\m{\bigcup}\m{A}
\endm
\setbox\startprefix=\hbox{\tt \ \ tfr1\ \$p\ }
\setbox\contprefix=\hbox{\tt \ \ \ \ \ \ \ \ \ \ }
\startm
\m{\vdash}\m{F}\m{{\rm Fn}}\m{{\rm On}}
\endm
\setbox\startprefix=\hbox{\tt \ \ tfr2\ \$p\ }
\setbox\contprefix=\hbox{\tt \ \ \ \ \ \ \ \ \ \ }
\startm
\m{\vdash}\m{(}\m{z}\m{\in}\m{{\rm On}}\m{\rightarrow}\m{(}\m{F}\m{`}\m{z}%
\m{)}\m{=}\m{(}\m{G}\m{`}\m{(}\m{F}\m{\restriction}\m{z}\m{)}\m{)}\m{)}
\endm
\setbox\startprefix=\hbox{\tt \ \ tfr3\ \$p\ }
\setbox\contprefix=\hbox{\tt \ \ \ \ \ \ \ \ \ \ }
\startm
\m{\vdash}\m{(}\m{(}\m{B}\m{{\rm Fn}}\m{{\rm On}}\m{\wedge}\m{\forall}\m{x}\m{%
\in}\m{{\rm On}}\m{(}\m{B}\m{`}\m{x}\m{)}\m{=}\m{(}\m{G}\m{`}\m{(}\m{B}\m{%
\restriction}\m{x}\m{)}\m{)}\m{)}\m{\rightarrow}\m{B}\m{=}\m{F}\m{)}
\endm
\vskip 1ex

\noindent The existence of omega (the class of natural numbers).\index{natural
number}\index{omega ($\omega$)}\index{Axiom of Infinity}  Axiom 7 of Takeuti
and Zaring, p.~43.  (This is the only theorem in this section requiring the
Axiom of Infinity.)

\vskip 0.5ex
\setbox\startprefix=\hbox{\tt \
\ omex\ \$p\ }
\setbox\contprefix=\hbox{\tt \ \ \ \ \ \ \ \ \ \ }
\startm
\m{\vdash}\m{\omega}\m{\in}\m{{\rm V}}
\endm
%\vskip 2ex


\section{Axioms for Real and Complex Numbers}\label{real}
\index{real and complex numbers!axioms for}

This section presents the axioms for real and complex numbers.  Analysis
textbooks implicitly or explicitly use these axioms or their equivalents
as their starting point.  In the database \texttt{set.mm}, we define real
and complex numbers as (rather complicated) specific sets and derive these
axioms as {\em theorems} from the axioms of ZF set theory, using a method
called Dedekind cuts.  We omit the details of this construction, which you can
follow if you wish using the \texttt{set.mm} database in conjunction with the
textbooks referenced therein.

Once we prove those theorems, we then restate these proven theorems as axioms.
This lets us easily identify which axioms are needed for a particular complex number proof, without the obfuscation of the set theory used to derive them.
As a result,
the construction is actually unimportant other
than to show that sets exist that satisfy the axioms, and thus that the axioms
are consistent if ZF set theory is consistent.  When working with real numbers
you can think of them as being the actual sets resulting
from the construction (for definiteness), or you can
think of them as otherwise unspecified sets that happen to satisfy the axioms.
The derivation is not easy, but the fact that it works is quite remarkable
and lends support to the idea that ZFC set theory is all we need to
provide a foundation for essentially all of mathematics.

For the axioms we are given (or postulate) 8 classes:  $\mathbb{C}$ (the
set of complex numbers), $\mathbb{R}$ (the set of real numbers, a subset
of $\mathbb{C}$), $0$ (zero), $1$ (one), $i$ (square root of
$-1$), $+$ (plus), $\cdot$ (times), and
$<_{\mathbb{R}}$ (less than for just the real numbers).
Subtraction and division are defined terms and are not part of the
axioms; for their definitions see \texttt{set.mm}.

Note that the notation $(A+B)$ (and similarly $(A\cdot B)$) specifies a class
called an {\em operation},\index{operation} and is the function value of the
class $+$ at ordered pair $\langle A,B \rangle$.  An operation is defined by
statement \texttt{df-opr} on p.~\pageref{dfopr}.
The notation $A <_{\mathbb{R}} B$ specifies a
wff called a {\em binary relation}\index{binary relation} and means $\langle A,B \rangle \in \,<_{\mathbb{R}}$, as defined by statement \texttt{df-br} on p.~\pageref{dfbr}.

Our set of 8 given classes is assumed to satisfy the following 22 axioms
(in the axioms listed below, $<$ really means $<_{\mathbb{R}}$).

\vskip 2ex

\noindent 1. The real numbers are a subset of the complex numbers.

%\vskip 0.5ex
\setbox\startprefix=\hbox{\tt \ \ ax-resscn\ \$p\ }
\setbox\contprefix=\hbox{\tt \ \ \ \ \ \ \ \ \ \ \ \ \ \ }
\startm
\m{\vdash}\m{\mathbb{R}}\m{\subseteq}\m{\mathbb{C}}
\endm
%\vskip 1ex

\noindent 2. One is a complex number.

%\vskip 0.5ex
\setbox\startprefix=\hbox{\tt \ \ ax-1cn\ \$p\ }
\setbox\contprefix=\hbox{\tt \ \ \ \ \ \ \ \ \ \ \ }
\startm
\m{\vdash}\m{1}\m{\in}\m{\mathbb{C}}
\endm
%\vskip 1ex

\noindent 3. The imaginary number $i$ is a complex number.

%\vskip 0.5ex
\setbox\startprefix=\hbox{\tt \ \ ax-icn\ \$p\ }
\setbox\contprefix=\hbox{\tt \ \ \ \ \ \ \ \ \ \ \ }
\startm
\m{\vdash}\m{i}\m{\in}\m{\mathbb{C}}
\endm
%\vskip 1ex

\noindent 4. Complex numbers are closed under addition.

%\vskip 0.5ex
\setbox\startprefix=\hbox{\tt \ \ ax-addcl\ \$p\ }
\setbox\contprefix=\hbox{\tt \ \ \ \ \ \ \ \ \ \ \ \ \ }
\startm
\m{\vdash}\m{(}\m{(}\m{A}\m{\in}\m{\mathbb{C}}\m{\wedge}\m{B}\m{\in}\m{\mathbb{C}}%
\m{)}\m{\rightarrow}\m{(}\m{A}\m{+}\m{B}\m{)}\m{\in}\m{\mathbb{C}}\m{)}
\endm
%\vskip 1ex

\noindent 5. Real numbers are closed under addition.

%\vskip 0.5ex
\setbox\startprefix=\hbox{\tt \ \ ax-addrcl\ \$p\ }
\setbox\contprefix=\hbox{\tt \ \ \ \ \ \ \ \ \ \ \ \ \ \ }
\startm
\m{\vdash}\m{(}\m{(}\m{A}\m{\in}\m{\mathbb{R}}\m{\wedge}\m{B}\m{\in}\m{\mathbb{R}}%
\m{)}\m{\rightarrow}\m{(}\m{A}\m{+}\m{B}\m{)}\m{\in}\m{\mathbb{R}}\m{)}
\endm
%\vskip 1ex

\noindent 6. Complex numbers are closed under multiplication.

%\vskip 0.5ex
\setbox\startprefix=\hbox{\tt \ \ ax-mulcl\ \$p\ }
\setbox\contprefix=\hbox{\tt \ \ \ \ \ \ \ \ \ \ \ \ \ }
\startm
\m{\vdash}\m{(}\m{(}\m{A}\m{\in}\m{\mathbb{C}}\m{\wedge}\m{B}\m{\in}\m{\mathbb{C}}%
\m{)}\m{\rightarrow}\m{(}\m{A}\m{\cdot}\m{B}\m{)}\m{\in}\m{\mathbb{C}}\m{)}
\endm
%\vskip 1ex

\noindent 7. Real numbers are closed under multiplication.

%\vskip 0.5ex
\setbox\startprefix=\hbox{\tt \ \ ax-mulrcl\ \$p\ }
\setbox\contprefix=\hbox{\tt \ \ \ \ \ \ \ \ \ \ \ \ \ \ }
\startm
\m{\vdash}\m{(}\m{(}\m{A}\m{\in}\m{\mathbb{R}}\m{\wedge}\m{B}\m{\in}\m{\mathbb{R}}%
\m{)}\m{\rightarrow}\m{(}\m{A}\m{\cdot}\m{B}\m{)}\m{\in}\m{\mathbb{R}}\m{)}
\endm
%\vskip 1ex

\noindent 8. Multiplication of complex numbers is commutative.

%\vskip 0.5ex
\setbox\startprefix=\hbox{\tt \ \ ax-mulcom\ \$p\ }
\setbox\contprefix=\hbox{\tt \ \ \ \ \ \ \ \ \ \ \ \ \ \ }
\startm
\m{\vdash}\m{(}\m{(}\m{A}\m{\in}\m{\mathbb{C}}\m{\wedge}\m{B}\m{\in}\m{\mathbb{C}}%
\m{)}\m{\rightarrow}\m{(}\m{A}\m{\cdot}\m{B}\m{)}\m{=}\m{(}\m{B}\m{\cdot}\m{A}%
\m{)}\m{)}
\endm
%\vskip 1ex

\noindent 9. Addition of complex numbers is associative.

%\vskip 0.5ex
\setbox\startprefix=\hbox{\tt \ \ ax-addass\ \$p\ }
\setbox\contprefix=\hbox{\tt \ \ \ \ \ \ \ \ \ \ \ \ \ \ }
\startm
\m{\vdash}\m{(}\m{(}\m{A}\m{\in}\m{\mathbb{C}}\m{\wedge}\m{B}\m{\in}\m{\mathbb{C}}%
\m{\wedge}\m{C}\m{\in}\m{\mathbb{C}}\m{)}\m{\rightarrow}\m{(}\m{(}\m{A}\m{+}%
\m{B}\m{)}\m{+}\m{C}\m{)}\m{=}\m{(}\m{A}\m{+}\m{(}\m{B}\m{+}\m{C}\m{)}\m{)}%
\m{)}
\endm
%\vskip 1ex

\noindent 10. Multiplication of complex numbers is associative.

%\vskip 0.5ex
\setbox\startprefix=\hbox{\tt \ \ ax-mulass\ \$p\ }
\setbox\contprefix=\hbox{\tt \ \ \ \ \ \ \ \ \ \ \ \ \ \ }
\startm
\m{\vdash}\m{(}\m{(}\m{A}\m{\in}\m{\mathbb{C}}\m{\wedge}\m{B}\m{\in}\m{\mathbb{C}}%
\m{\wedge}\m{C}\m{\in}\m{\mathbb{C}}\m{)}\m{\rightarrow}\m{(}\m{(}\m{A}\m{\cdot}%
\m{B}\m{)}\m{\cdot}\m{C}\m{)}\m{=}\m{(}\m{A}\m{\cdot}\m{(}\m{B}\m{\cdot}\m{C}%
\m{)}\m{)}\m{)}
\endm
%\vskip 1ex

\noindent 11. Multiplication distributes over addition for complex numbers.

%\vskip 0.5ex
\setbox\startprefix=\hbox{\tt \ \ ax-distr\ \$p\ }
\setbox\contprefix=\hbox{\tt \ \ \ \ \ \ \ \ \ \ \ \ \ }
\startm
\m{\vdash}\m{(}\m{(}\m{A}\m{\in}\m{\mathbb{C}}\m{\wedge}\m{B}\m{\in}\m{\mathbb{C}}%
\m{\wedge}\m{C}\m{\in}\m{\mathbb{C}}\m{)}\m{\rightarrow}\m{(}\m{A}\m{\cdot}\m{(}%
\m{B}\m{+}\m{C}\m{)}\m{)}\m{=}\m{(}\m{(}\m{A}\m{\cdot}\m{B}\m{)}\m{+}\m{(}%
\m{A}\m{\cdot}\m{C}\m{)}\m{)}\m{)}
\endm
%\vskip 1ex

\noindent 12. The square of $i$ equals $-1$ (expressed as $i$-squared plus 1 is
0).

%\vskip 0.5ex
\setbox\startprefix=\hbox{\tt \ \ ax-i2m1\ \$p\ }
\setbox\contprefix=\hbox{\tt \ \ \ \ \ \ \ \ \ \ \ \ }
\startm
\m{\vdash}\m{(}\m{(}\m{i}\m{\cdot}\m{i}\m{)}\m{+}\m{1}\m{)}\m{=}\m{0}
\endm
%\vskip 1ex

\noindent 13. One and zero are distinct.

%\vskip 0.5ex
\setbox\startprefix=\hbox{\tt \ \ ax-1ne0\ \$p\ }
\setbox\contprefix=\hbox{\tt \ \ \ \ \ \ \ \ \ \ \ \ }
\startm
\m{\vdash}\m{1}\m{\ne}\m{0}
\endm
%\vskip 1ex

\noindent 14. One is an identity element for real multiplication.

%\vskip 0.5ex
\setbox\startprefix=\hbox{\tt \ \ ax-1rid\ \$p\ }
\setbox\contprefix=\hbox{\tt \ \ \ \ \ \ \ \ \ \ \ }
\startm
\m{\vdash}\m{(}\m{A}\m{\in}\m{\mathbb{R}}\m{\rightarrow}\m{(}\m{A}\m{\cdot}\m{1}%
\m{)}\m{=}\m{A}\m{)}
\endm
%\vskip 1ex

\noindent 15. Every real number has a negative.

%\vskip 0.5ex
\setbox\startprefix=\hbox{\tt \ \ ax-rnegex\ \$p\ }
\setbox\contprefix=\hbox{\tt \ \ \ \ \ \ \ \ \ \ \ \ \ \ }
\startm
\m{\vdash}\m{(}\m{A}\m{\in}\m{\mathbb{R}}\m{\rightarrow}\m{\exists}\m{x}\m{\in}%
\m{\mathbb{R}}\m{(}\m{A}\m{+}\m{x}\m{)}\m{=}\m{0}\m{)}
\endm
%\vskip 1ex

\noindent 16. Every nonzero real number has a reciprocal.

%\vskip 0.5ex
\setbox\startprefix=\hbox{\tt \ \ ax-rrecex\ \$p\ }
\setbox\contprefix=\hbox{\tt \ \ \ \ \ \ \ \ \ \ \ \ \ \ }
\startm
\m{\vdash}\m{(}\m{A}\m{\in}\m{\mathbb{R}}\m{\rightarrow}\m{(}\m{A}\m{\ne}\m{0}%
\m{\rightarrow}\m{\exists}\m{x}\m{\in}\m{\mathbb{R}}\m{(}\m{A}\m{\cdot}%
\m{x}\m{)}\m{=}\m{1}\m{)}\m{)}
\endm
%\vskip 1ex

\noindent 17. A complex number can be expressed in terms of two reals.

%\vskip 0.5ex
\setbox\startprefix=\hbox{\tt \ \ ax-cnre\ \$p\ }
\setbox\contprefix=\hbox{\tt \ \ \ \ \ \ \ \ \ \ \ \ }
\startm
\m{\vdash}\m{(}\m{A}\m{\in}\m{\mathbb{C}}\m{\rightarrow}\m{\exists}\m{x}\m{\in}%
\m{\mathbb{R}}\m{\exists}\m{y}\m{\in}\m{\mathbb{R}}\m{A}\m{=}\m{(}\m{x}\m{+}\m{(}%
\m{y}\m{\cdot}\m{i}\m{)}\m{)}\m{)}
\endm
%\vskip 1ex

\noindent 18. Ordering on reals satisfies strict trichotomy.

%\vskip 0.5ex
\setbox\startprefix=\hbox{\tt \ \ ax-pre-lttri\ \$p\ }
\setbox\contprefix=\hbox{\tt \ \ \ \ \ \ \ \ \ \ \ \ \ }
\startm
\m{\vdash}\m{(}\m{(}\m{A}\m{\in}\m{\mathbb{R}}\m{\wedge}\m{B}\m{\in}\m{\mathbb{R}}%
\m{)}\m{\rightarrow}\m{(}\m{A}\m{<}\m{B}\m{\leftrightarrow}\m{\lnot}\m{(}\m{A}%
\m{=}\m{B}\m{\vee}\m{B}\m{<}\m{A}\m{)}\m{)}\m{)}
\endm
%\vskip 1ex

\noindent 19. Ordering on reals is transitive.

%\vskip 0.5ex
\setbox\startprefix=\hbox{\tt \ \ ax-pre-lttrn\ \$p\ }
\setbox\contprefix=\hbox{\tt \ \ \ \ \ \ \ \ \ \ \ \ \ }
\startm
\m{\vdash}\m{(}\m{(}\m{A}\m{\in}\m{\mathbb{R}}\m{\wedge}\m{B}\m{\in}\m{\mathbb{R}}%
\m{\wedge}\m{C}\m{\in}\m{\mathbb{R}}\m{)}\m{\rightarrow}\m{(}\m{(}\m{A}\m{<}%
\m{B}\m{\wedge}\m{B}\m{<}\m{C}\m{)}\m{\rightarrow}\m{A}\m{<}\m{C}\m{)}\m{)}
\endm
%\vskip 1ex

\noindent 20. Ordering on reals is preserved after addition to both sides.

%\vskip 0.5ex
\setbox\startprefix=\hbox{\tt \ \ ax-pre-ltadd\ \$p\ }
\setbox\contprefix=\hbox{\tt \ \ \ \ \ \ \ \ \ \ \ \ \ }
\startm
\m{\vdash}\m{(}\m{(}\m{A}\m{\in}\m{\mathbb{R}}\m{\wedge}\m{B}\m{\in}\m{\mathbb{R}}%
\m{\wedge}\m{C}\m{\in}\m{\mathbb{R}}\m{)}\m{\rightarrow}\m{(}\m{A}\m{<}\m{B}\m{%
\rightarrow}\m{(}\m{C}\m{+}\m{A}\m{)}\m{<}\m{(}\m{C}\m{+}\m{B}\m{)}\m{)}\m{)}
\endm
%\vskip 1ex

\noindent 21. The product of two positive reals is positive.

%\vskip 0.5ex
\setbox\startprefix=\hbox{\tt \ \ ax-pre-mulgt0\ \$p\ }
\setbox\contprefix=\hbox{\tt \ \ \ \ \ \ \ \ \ \ \ \ \ \ }
\startm
\m{\vdash}\m{(}\m{(}\m{A}\m{\in}\m{\mathbb{R}}\m{\wedge}\m{B}\m{\in}\m{\mathbb{R}}%
\m{)}\m{\rightarrow}\m{(}\m{(}\m{0}\m{<}\m{A}\m{\wedge}\m{0}%
\m{<}\m{B}\m{)}\m{\rightarrow}\m{0}\m{<}\m{(}\m{A}\m{\cdot}\m{B}\m{)}%
\m{)}\m{)}
\endm
%\vskip 1ex

\noindent 22. A non-empty, bounded-above set of reals has a supremum.

%\vskip 0.5ex
\setbox\startprefix=\hbox{\tt \ \ ax-pre-sup\ \$p\ }
\setbox\contprefix=\hbox{\tt \ \ \ \ \ \ \ \ \ \ \ }
\startm
\m{\vdash}\m{(}\m{(}\m{A}\m{\subseteq}\m{\mathbb{R}}\m{\wedge}\m{A}\m{\ne}\m{%
\varnothing}\m{\wedge}\m{\exists}\m{x}\m{\in}\m{\mathbb{R}}\m{\forall}\m{y}\m{%
\in}\m{A}\m{\,y}\m{<}\m{x}\m{)}\m{\rightarrow}\m{\exists}\m{x}\m{\in}\m{%
\mathbb{R}}\m{(}\m{\forall}\m{y}\m{\in}\m{A}\m{\lnot}\m{x}\m{<}\m{y}\m{\wedge}\m{%
\forall}\m{y}\m{\in}\m{\mathbb{R}}\m{(}\m{y}\m{<}\m{x}\m{\rightarrow}\m{\exists}%
\m{z}\m{\in}\m{A}\m{\,y}\m{<}\m{z}\m{)}\m{)}\m{)}
\endm

% NOTE: The \m{...} expressions above could be represented as
% $ \vdash ( ( A \subseteq \mathbb{R} \wedge A \ne \varnothing \wedge \exists x \in \mathbb{R} \forall y \in A \,y < x ) \rightarrow \exists x \in \mathbb{R} ( \forall y \in A \lnot x < y \wedge \forall y \in \mathbb{R} ( y < x \rightarrow \exists z \in A \,y < z ) ) ) $

\vskip 2ex

This completes the set of axioms for real and complex numbers.  You may
wish to look at how subtraction, division, and decimal numbers
are defined in \texttt{set.mm}, and for fun look at the proof of $2+
2 = 4$ (theorem \texttt{2p2e4} in \texttt{set.mm})
as discussed in section \ref{2p2e4}.

In \texttt{set.mm} we define the non-negative integers $\mathbb{N}$, the integers
$\mathbb{Z}$, and the rationals $\mathbb{Q}$ as subsets of $\mathbb{R}$.  This leads
to the nice inclusion $\mathbb{N} \subseteq \mathbb{Z} \subseteq \mathbb{Q} \subseteq
\mathbb{R} \subseteq \mathbb{C}$, giving us a uniform framework in which, for
example, a property such as commutativity of complex number addition
automatically applies to integers.  The natural numbers $\mathbb{N}$
are different from the set $\omega$ we defined earlier, but both satisfy
Peano's postulates.

\subsubsection{Complex Number Axioms in Analysis Texts}

Most analysis texts construct complex numbers as ordered pairs of reals,
leading to construction-dependent properties that satisfy these axioms
but are not stated in their pure form.  (This is also done in
\texttt{set.mm} but our axioms are extracted from that construction.)
Other texts will simply state that $\mathbb{R}$ is a ``complete ordered
subfield of $\mathbb{C}$,'' leading to redundant axioms when this phrase
is completely expanded out.  In fact I have not seen a text with the
axioms in the explicit form above.
None of these axioms is unique individually, but this carefully worked out
collection of axioms is the result of years of work
by the Metamath community.

\subsubsection{Eliminating Unnecessary Complex Number Axioms}

We once had more axioms for real and complex numbers, but over years of time
we (the Metamath community)
have found ways to eliminate them (by proving them from other axioms)
or weaken them (by making weaker claims without reducing what
can be proved).
In particular, here are statements that used to be complex number
axioms but have since been formally proven (with Metamath) to be redundant:

\begin{itemize}
\item
  $\mathbb{C} \in V$.
  At one time this was listed as a ``complex number axiom''
  However, this is not properly speaking a complex number axiom,
  and in any case its proof uses axioms of set theory.
  Proven redundant by Mario Carneiro\index{Carneiro, Mario} on
  17-Nov-2014 (see \texttt{axcnex}).
\item
  $((A \in \mathbb{C} \land B \in \mathbb{C}$) $\rightarrow$
  $(A + B) = (B + A))$.
  Proved redundant by Eric Schmidt\index{Schmidt, Eric} on 19-Jun-2012,
  and formalized by Scott Fenton\index{Fenton, Scott} on 3-Jan-2013
  (see \texttt{addcom}).
\item
  $(A \in \mathbb{C} \rightarrow (A + 0) = A)$.
  Proved redundant by Eric Schmidt on 19-Jun-2012,
  and formalized by Scott Fenton on 3-Jan-2013
  (see \texttt{addid1}).
\item
  $(A \in \mathbb{C} \rightarrow \exists x \in \mathbb{C} (A + x) = 0)$.
  Proved redundant by Eric Schmidt and formalized on 21-May-2007
  (see \texttt{cnegex}).
\item
  $((A \in \mathbb{C} \land A \ne 0) \rightarrow \exists x \in \mathbb{C} (A \cdot x) = 1)$.
  Proved redundant by Eric Schmidt and formalized on 22-May-2007
  (see \texttt{recex}).
\item
  $0 \in \mathbb{R}$.
  Proved redundant by Eric Schmidt on 19-Feb-2005 and formalized 21-May-2007
  (see \texttt{0re}).
\end{itemize}

We could eliminate 0 as an axiomatic object by defining it as
$( ( i \cdot i ) + 1 )$
and replacing it with this expression throughout the axioms. If this
is done, axiom ax-i2m1 becomes redundant. However, the remaining axioms
would become longer and less intuitive.

Eric Schmidt's paper analyzing this axiom system \cite{Schmidt}
presented a proof that these remaining axioms,
with the possible exception of ax-mulcom, are independent of the others.
It is currently an open question if ax-mulcom is independent of the others.

\section{Two Plus Two Equals Four}\label{2p2e4}

Here is a proof that $2 + 2 = 4$, as proven in the theorem \texttt{2p2e4}
in the database \texttt{set.mm}.
This is a useful demonstration of what a Metamath proof can look like.
This proof may have more steps than you're used to, but each step is rigorously
proven all the way back to the axioms of logic and set theory.
This display was originally generated by the Metamath program
as an {\sc HTML} file.

In the table showing the proof ``Step'' is the sequential step number,
while its associated ``Expression'' is an expression that we have proved.
``Ref'' is the name of a theorem or axiom that justifies that expression,
and ``Hyp'' refers to previous steps (if any) that the theorem or axiom
needs so that we can use it.  Expressions are indented further than
the expressions that depend on them to show their interdependencies.

\begin{table}[!htbp]
\caption{Two plus two equals four}
\begin{tabular}{lllll}
\textbf{Step} & \textbf{Hyp} & \textbf{Ref} & \textbf{Expression} & \\
1  &       & df-2    & $ \; \; \vdash 2 = 1 + 1$  & \\
2  & 1     & oveq2i  & $ \; \vdash (2 + 2) = (2 + (1 + 1))$ & \\
3  &       & df-4    & $ \; \; \vdash 4 = (3 + 1)$ & \\
4  &       & df-3    & $ \; \; \; \vdash 3 = (2 + 1)$ & \\
5  & 4     & oveq1i  & $ \; \; \vdash (3 + 1) = ((2 + 1) + 1)$ & \\
6  &       & 2cn     & $ \; \; \; \vdash 2 \in \mathbb{C}$ & \\
7  &       & ax-1cn  & $ \; \; \; \vdash 1 \in \mathbb{C}$ & \\
8  & 6,7,7 & addassi & $ \; \; \vdash ((2 + 1) + 1) = (2 + (1 + 1))$ & \\
9  & 3,5,8 & 3eqtri  & $ \; \vdash 4 = (2 + (1 + 1))$ & \\
10 & 2,9   & eqtr4i  & $ \vdash (2 + 2) = 4$ & \\
\end{tabular}
\end{table}

Step 1 says that we can assert that $2 = 1 + 1$ because it is
justified by \texttt{df-2}.
What is \texttt{df-2}?
It is simply the definition of $2$, which in our system is defined as being
equal to $1 + 1$.  This shows how we can use definitions in proofs.

Look at Step 2 of the proof. In the Ref column, we see that it references
a previously proved theorem, \texttt{oveq2i}.
It turns out that
theorem \texttt{oveq2i} requires a
hypothesis, and in the Hyp column of Step 2 we indicate that Step 1 will
satisfy (match) this hypothesis.
If we looked at \texttt{oveq2i}
we would find that it proves that given some hypothesis
$A = B$, we can prove that $( C F A ) = ( C F B )$.
If we use \texttt{oveq2i} and apply step 1's result as the hypothesis,
that will mean that $A = 2$ and $B = ( 1 + 1 )$ within this use of
\texttt{oveq2i}.
We are free to select any value of $C$ and $F$ (subject to syntax constraints),
so we are free to select $C = 2$ and $F = +$,
producing our desired result,
$ (2 + 2) = (2 + (1 + 1))$.

Step 2 is an example of substitution.
In the end, every step in every proof uses only this one substitution rule.
All the rules of logic, and all the axioms, are expressed so that
they can be used via this one substitution rule.
So once you master substitution, you can master every Metamath proof,
no exceptions.

Each step is clear and can be immediately checked.
In the {\sc HTML} display you can even click on each reference to see why it is
justified, making it easy to see why the proof works.

\section{Deduction}\label{deduction}

Strictly speaking,
a deduction (also called an inference) is a kind of statement that needs
some hypotheses to be true in order for its conclusion to be true.
A theorem, on the other hand, has no hypotheses.
Informally we often call both of them theorems, but in this section we
will stick to the strict definitions.

It sometimes happens that we have proved a deduction of the form
$\varphi \Rightarrow \psi$\index{$\Rightarrow$}
(given hypothesis $\varphi$ we can prove $\psi$)
and we want to then prove a theorem of the form
$\varphi \rightarrow \psi$.

Converting a deduction (which uses a hypothesis) into a theorem
(which does not) is not as simple as you might think.
The deduction says, ``if we can prove $\varphi$ then we can prove $\psi$,''
which is in some sense weaker than saying
``$\varphi$ implies $\psi$.''
There is no axiom of logic that permits us to directly obtain the theorem
given the deduction.\footnote{
The conversion of a deduction to a theorem does not even hold in general
for quantum propositional calculus,
which is a weak subset of classical propositional calculus.
It has been shown that adding the Standard Deduction Theorem (discussed below)
to quantum propositional calculus turns it into classical
propositional calculus!
}

This is in contrast to going the other way.
If we have the theorem ($\varphi \rightarrow \psi$),
it is easy to recover the deduction
($\varphi \Rightarrow \psi$)
using modus ponens\index{modus ponens}
(\texttt{ax-mp}; see section \ref{axmp}).

In the following subsections we first discuss the standard deduction theorem
(the traditional but awkward way to convert deductions into theorems) and
the weak deduction theorem (a limited version of the standard deduction
theorem that is easier to use and was once widely used in
\texttt{set.mm}\index{set theory database (\texttt{set.mm})}).
In section \ref{deductionstyle} we discuss
deduction style, the newer approach we now recommend in most cases.
Deduction style uses ``deduction form,'' a form that
prefixes each hypothesis (other than definitions) and the
conclusion with a universal antecedent (``$\varphi \rightarrow$'').
Deduction style is widely used in \texttt{set.mm},
so it is useful to understand it and \textit{why} it is widely used.
Section \ref{naturaldeduction}
briefly discusses our approach for using natural deduction
within \texttt{set.mm},
as that approach is deeply related to deduction style.
We conclude with a summary of the strengths of
our approach, which we believe are compelling.

\subsection{The Standard Deduction Theorem}\label{standarddeductiontheorem}

It is possible to make use of information
contained in the deduction or its proof to assist us with the proof of
the related theorem.
In traditional logic books, there is a metatheorem called the
Deduction Theorem\index{Deduction Theorem}\index{Standard Deduction Theorem},
discovered independently by Herbrand and Tarski around 1930.
The Deduction Theorem, which we often call the Standard Deduction Theorem,
provides an algorithm for constructing a proof of a theorem from the
proof of its corresponding deduction. See, for example,
\cite[p.~56]{Margaris}\index{Margaris, Angelo}.
To construct a proof for a theorem, the
algorithm looks at each step in the proof of the original deduction and
rewrites the step with several steps wherein the hypothesis is eliminated
and becomes an antecedent.

In ordinary mathematics, no one actually carries out the algorithm,
because (in its most basic form) it involves an exponential explosion of
the number of proof steps as more hypotheses are eliminated. Instead,
the Standard Deduction Theorem is invoked simply to claim that it can
be done in principle, without actually doing it.
What's more, the algorithm is not as simple as it might first appear
when applying it rigorously.
There is a subtle restriction on the Standard Deduction Theorem
that must be taken into account involving the axiom of generalization
when working with predicate calculus (see the literature for more detail).

One of the goals of Metamath is to let you plainly see, with as few
underlying concepts as possible, how mathematics can be derived directly
from the axioms, and not indirectly according to some hidden rules
buried inside a program or understood only by logicians. If we added
the Standard Deduction Theorem to the language and proof verifier,
that would greatly complicate both and largely defeat Metamath's goal
of simplicity. In principle, we could show direct proofs by expanding
out the proof steps generated by the algorithm of the Standard Deduction
Theorem, but that is not feasible in practice because the number of proof
steps quickly becomes huge, even astronomical.
Since the algorithm of the Standard Deduction Theorem is driven by the proof,
we would have to go through that proof
all over again---starting from axioms---in order to obtain the theorem form.
In terms of proof length, there would be no savings over just
proving the theorem directly instead of first proving the deduction form.

\subsection{Weak Deduction Theorem}\label{weakdeductiontheorem}

We have developed
a more efficient method for proving a theorem from a deduction
that can be used instead of the Standard Deduction Theorem
in many (but not all) cases.
We call this more efficient method the
Weak Deduction Theorem\index{Weak Deduction Theorem}.\footnote{
There is also an unrelated ``Weak Deduction Theorem''
in the field of relevance logic, so to avoid confusion we could call
ours the ``Weak Deduction Theorem for Classical Logic.''}
Unlike the Standard Deduction Theorem, the Weak Deduction Theorem produces the
theorem directly from a special substitution instance of the deduction,
using a small, fixed number of steps roughly proportional to the length
of the final theorem.

If you come to a proof referencing the Weak Deduction Theorem
\texttt{dedth} (or one of its variants \texttt{dedthxx}),
here is how to follow the proof without getting into the details:
just click on the theorem referenced in the step
just before the reference to \texttt{dedth} and ignore everything else.
Theorem \texttt{dedth} simply turns a hypothesis into an antecedent
(i.e. the hypothesis followed by $\rightarrow$
is placed in front of the assertion, and the hypothesis
itself is eliminated) given certain conditions.

The Weak Deduction Theorem
eliminates a hypothesis $\varphi$, making it become an antecedent.
It does this by proving an expression
$ \varphi \rightarrow \psi $ given two hypotheses:
(1)
$ ( A = {\rm if} ( \varphi , A , B ) \rightarrow ( \varphi \leftrightarrow \chi ) ) $
and
(2) $\chi$.
Note that it requires that a proof exists for $\varphi$ when the class variable
$A$ is replaced with a specific class $B$. The hypothesis $\chi$
should be assigned to the inference.
You can see the details of the proof of the Weak Deduction Theorem
in theorem \texttt{dedth}.

The Weak Deduction Theorem
is probably easier to understand by studying proofs that make use of it.
For example, let's look at the proof of \texttt{renegcl}, which proves that
$ \vdash ( A \in \mathbb{R} \rightarrow - A \in \mathbb{R} )$:

\needspace{4\baselineskip}
\begin{longtabu} {l l l X}
\textbf{Step} & \textbf{Hyp} & \textbf{Ref} & \textbf{Expression} \\
  1 &  & negeq &
  $\vdash$ $($ $A$ $=$ ${\rm if}$ $($ $A$ $\in$ $\mathbb{R}$ $,$ $A$ $,$ $1$ $)$ $\rightarrow$
  $\textrm{-}$ $A$ $=$ $\textrm{-}$ ${\rm if}$ $($ $A$ $\in$ $\mathbb{R}$
  $,$ $A$ $,$ $1$ $)$ $)$ \\
 2 & 1 & eleq1d &
    $\vdash$ $($ $A$ $=$ ${\rm if}$ $($ $A$ $\in$ $\mathbb{R}$ $,$ $A$ $,$ $1$ $)$ $\rightarrow$ $($
    $\textrm{-}$ $A$ $\in$ $\mathbb{R}$ $\leftrightarrow$
    $\textrm{-}$ ${\rm if}$ $($ $A$ $\in$ $\mathbb{R}$ $,$ $A$ $,$ $1$ $)$ $\in$
    $\mathbb{R}$ $)$ $)$ \\
 3 &  & 1re & $\vdash 1 \in \mathbb{R}$ \\
 4 & 3 & elimel &
   $\vdash {\rm if} ( A \in \mathbb{R} , A , 1 ) \in \mathbb{R}$ \\
 5 & 4 & renegcli &
   $\vdash \textrm{-} {\rm if} ( A \in \mathbb{R} , A , 1 ) \in \mathbb{R}$ \\
 6 & 2,5 & dedth &
   $\vdash ( A \in \mathbb{R} \rightarrow \textrm{-} A \in \mathbb{R}$ ) \\
\end{longtabu}

The somewhat strange-looking steps in \texttt{renegcl} before step 5 are
technical stuff that makes this magic work, and they can be ignored
for a quick overview of the proof. To continue following the ``important''
part of the proof of \texttt{renegcl},
you can look at the reference to \texttt{renegcli} at step 5.

That said, let's briefly look at how
\texttt{renegcl} uses the
Weak Deduction Theorem (\texttt{dedth}) to do its job,
in case you want to do something similar or want understand it more deeply.
Let's work backwards in the proof of \texttt{renegcl}.
Step 6 applies \texttt{dedth} to produce our goal result
$ \vdash ( A \in \mathbb{R} \rightarrow\, - A \in \mathbb{R} )$.
This requires on the one hand the (substituted) deduction
\texttt{renegcli} in step 5.
By itself \texttt{renegcli} proves the deduction
$ \vdash A \in \mathbb{R} \Rightarrow\, \vdash - A \in \mathbb{R}$;
this is the deduction form we are trying to turn into theorem form,
and thus
\texttt{renegcli} has a separate hypothesis that must be fulfilled.
To fulfill the hypothesis of the invocation of
\texttt{renegcli} in step 5, it is eventually
reduced to the already proven theorem $1 \in \mathbb{R}$ in step 3.
Step 4 connects steps 3 and 5; step 4 invokes
\texttt{elimel}, a special case of \texttt{elimhyp} that eliminates
a membership hypothesis for the weak deduction theorem.
On the other hand, the equivalence of the conclusion of
\texttt{renegcl}
$( - A \in \mathbb{R} )$ and the substituted conclusion of
\texttt{renegcli} must be proven, which is done in steps 2 and 1.

The weak deduction theorem has limitations.
In particular, we must be able to prove a special case of the deduction's
hypothesis as a stand-alone theorem.
For example, we used $1 \in \mathbb{R}$ in step 3 of \texttt{renegcl}.

We used to use the weak deduction theorem
extensively within \texttt{set.mm}.
However, we now recommend applying ``deduction style''
instead in most cases, as deduction style is
often an easier and clearer approach.
Therefore, we will now describe deduction style.

\subsection{Deduction Style}\label{deductionstyle}

We now prefer to write assertions in ``deduction form''
instead of writing a proof that would require use of the standard or
weak deduction theorem.
We call this appraoch
``deduction style.''\index{deduction style}

It will be easier to explain this by first defining some terms:

\begin{itemize}
\item \textbf{closed form}\index{closed form}\index{forms!closed}:
A kind of assertion (theorem) with no hypotheses.
Typically its label has no special suffix.
An example is \texttt{unss}, which states:
$\vdash ( ( A \subseteq C \wedge B \subseteq C ) \leftrightarrow ( A \cup B )
\subseteq C )\label{eq:unss}$
\item \textbf{deduction form}\index{deduction form}\index{forms!deduction}:
A kind of assertion with one or more hypotheses
where the conclusion is an implication with
a wff variable as the antecedent (usually $\varphi$), and every hypothesis
(\$e statement)
is either (1) an implication with the same antecedent as the conclusion or
(2) a definition.
A definition
can be for a class variable (this is a class variable followed by ``='')
or a wff variable (this is a wff variable followed by $\leftrightarrow$);
class variable definitions are more common.
In practice, a proof
in deduction form will also contain many steps that are implications
where the antecedent is either that wff variable (normally $\varphi$)
or is
a conjunction (...$\land$...) including that wff variable ($\varphi$).
If an assertion is in deduction form, and other forms are also available,
then we suffix its label with ``d.''
An example is \texttt{unssd}, which states\footnote{
For brevity we show here (and in other places)
a $\&$\index{$\&$} between hypotheses\index{hypotheses}
and a $\Rightarrow$\index{$\Rightarrow$}\index{conclusion}
between the hypotheses and the conclusion.
This notation is technically not part of the Metamath language, but is
instead a convenient abbreviation to show both the hypotheses and conclusion.}:
$\vdash ( \varphi \rightarrow A \subseteq C )\quad\&\quad \vdash ( \varphi
    \rightarrow B \subseteq C )\quad\Rightarrow\quad \vdash ( \varphi
    \rightarrow ( A \cup B ) \subseteq C )\label{eq:unssd}$
\item \textbf{inference form}\index{inference form}\index{forms!inference}:
A kind of assertion with one or more hypotheses that is not in deduction form
(e.g., there is no common antecedent).
If an assertion is in inference form, and other forms are also available,
then we suffix its label with ``i.''
An example is \texttt{unssi}, which states:
$\vdash A \subseteq C\quad\&\quad \vdash B \subseteq C\quad\Rightarrow\quad
    \vdash ( A \cup B ) \subseteq C\label{eq:unssi}$
\end{itemize}

When using deduction style we express an assertion in deduction form.
This form prefixes each hypothesis (other than definitions) and the
conclusion with a universal antecedent (``$\varphi \rightarrow$'').
The antecedent (e.g., $\varphi$)
mimics the context handled in the deduction theorem, eliminating
the need to directly use the deduction theorem.

Once you have an assertion in deduction form, you can easily convert it
to inference form or closed form:

\begin{itemize}
\item To
prove some assertion Ti in inference form, given assertion Td in deduction
form, there is a simple mechanical process you can use. First take each
Ti hypothesis and insert a \texttt{T.} $\rightarrow$ prefix (``true implies'')
using \texttt{a1i}. You
can then use the existing assertion Td to prove the resulting conclusion
with a \texttt{T.} $\rightarrow$ prefix.
Finally, you can remove that prefix using \texttt{trud},
resulting in the conclusion you wanted to prove.
\item To
prove some assertion T in closed form, given assertion Td in deduction
form, there is another simple mechanical process you can use. First,
select an expression that is the conjunction (...$\land$...) of all of the
consequents of every hypothesis of Td. Next, prove that this expression
implies each of the separate hypotheses of Td in turn by eliminating
conjuncts (there are a variety of proven assertions to do this, including
\texttt{simpl},
\texttt{simpr},
\texttt{3simpa},
\texttt{3simpb},
\texttt{3simpc},
\texttt{simp1},
\texttt{simp2},
and
\texttt{simp3}).
If the
expression has nested conjunctions, inner conjuncts can be broken out by
chaining the above theorems with \texttt{syl}
(see section \ref{syl}).\footnote{
There are actually many theorems
(labeled simp* such as \texttt{simp333}) that break out inner conjuncts in one
step, but rather than learning them you can just use the chaining we
just described to prove them, and then let the Metamath program command
\texttt{minimize{\char`\_}with}\index{\texttt{minimize{\char`\_}with} command}
figure out the right ones needed to collapse them.}
As your final step, you can then apply the already-proven assertion Td
(which is in deduction form), proving assertion T in closed form.
\end{itemize}

We can also easily convert any assertion T in closed form to its related
assertion Ti in inference form by applying
modus ponens\index{modus ponens} (see section \ref{axmp}).

The deduction form antecedent can also be used to represent the context
necessary to support natural deduction systems, so we will now
discuss natural deduction.

\subsection{Natural Deduction}\label{naturaldeduction}

Natural deduction\index{natural deduction}
(ND) systems, as such, were originally introduced in
1934 by two logicians working independently: Ja\'skowski and Gentzen. ND
systems are supposed to reconstruct, in a formally proper way, traditional
ways of mathematical reasoning (such as conditional proof, indirect proof,
and proof by cases). As reconstructions they were naturally influenced
by previous work, and many specific ND systems and notations have been
developed since their original work.

There are many ND variants, but
Indrzejczak \cite[p.~31-32]{Indrzejczak}\index{Indrzejczak, Andrzej}
suggests that any natural deductive system must satisfy at
least these three criteria:

\begin{itemize}
\item ``There are some means for entering assumptions into a proof and
also for eliminating them. Usually it requires some bookkeeping devices
for indicating the scope of an assumption, and showing that a part of
a proof depending on eliminated assumption is discharged.
\item There are no (or, at least, a very limited set of) axioms, because
their role is taken over by the set of primitive rules for introduction
and elimination of logical constants which means that elementary
inferences instead of formulae are taken as primitive.
\item (A genuine) ND system admits a lot of freedom in proof construction
and possibility of applying several proof search strategies, like
conditional proof, proof by cases, proof by reductio ad absurdum etc.''
\end{itemize}

The Metamath Proof Explorer (MPE) as defined in \texttt{set.mm}
is fundamentally a Hilbert-style system.
That is, MPE is based on a larger number of axioms (compared
to natural deduction systems), a very small set of rules of inference
(modus ponens), and the context is not changed by the rules of inference
in the middle of a proof. That said, MPE proofs can be developed using
the natural deduction (ND) approach as originally developed by Ja\'skowski
and Gentzen.

The most common and recommended approach for applying ND in MPE is to use
deduction form\index{deduction form}%
\index{forms!deduction}
and apply the MPE proven assertions that are equivalent to ND rules.
For example, MPE's \texttt{jca} is equivalent to ND rule $\land$-I
(and-insertion).
We maintain a list of equivalences that you may consult.
This approach for applying an ND approach within MPE relies on Metamath's
wff metavariables in an essential way, and is described in more detail
in the presentation ``Natural Deductions in the Metamath Proof Language''
by Mario Carneiro \cite{CarneiroND}\index{Carneiro, Mario}.

In this style many steps are an implication, whose antecedent mimics
the context ($\Gamma$) of most ND systems. To add an assumption, simply add
it to the implication antecedent (typically using
\texttt{simpr}),
and use that
new antecedent for all later claims in the same scope. If you wish to
use an assertion in an ND hypothesis scope that is outside the current
ND hypothesis scope, modify the assertion so that the ND hypothesis
assumption is added to its antecedent (typically using \texttt{adantr}). Most
proof steps will be proved using rules that have hypotheses and results
of the form $\varphi \rightarrow$ ...

An example may make this clearer.
Let's show theorem 5.5 of
\cite[p.~18]{Clemente}\index{Clemente Laboreo, Daniel}
along with a line by line translation using the usual
translation of natural deduction (ND) in the Metamath Proof Explorer
(MPE) notation (this is proof \texttt{ex-natded5.5}).
The proof's original goal was to prove
$\lnot \psi$ given two hypotheses,
$( \psi \rightarrow \chi )$ and $ \lnot \chi$.
We will translate these statements into MPE deduction form
by prefixing them all with $\varphi \rightarrow$.
As a result, in MPE the goal is stated as
$( \varphi \rightarrow \lnot \psi )$, and the two hypotheses are stated as
$( \varphi \rightarrow ( \psi \rightarrow \chi ) )$ and
$( \varphi \rightarrow \lnot \chi )$.

The following table shows the proof in Fitch natural deduction style
and its MPE equivalent.
The \textit{\#} column shows the original numbering,
\textit{MPE\#} shows the number in the equivalent MPE proof
(which we will show later),
\textit{ND Expression} shows the original proof claim in ND notation,
and \textit{MPE Translation} shows its translation into MPE
as discussed in this section.
The final columns show the rationale in ND and MPE respectively.

\needspace{4\baselineskip}
{\setlength{\extrarowsep}{4pt} % Keep rows from being too close together
\begin{longtabu}   { @{} c c X X X X }
\textbf{\#} & \textbf{MPE\#} & \textbf{ND Ex\-pres\-sion} &
\textbf{MPE Trans\-lation} & \textbf{ND Ration\-ale} &
\textbf{MPE Ra\-tio\-nale} \\
\endhead

1 & 2;3 &
$( \psi \rightarrow \chi )$ &
$( \varphi \rightarrow ( \psi \rightarrow \chi ) )$ &
Given &
\$e; \texttt{adantr} to put in ND hypothesis \\

2 & 5 &
$ \lnot \chi$ &
$( \varphi \rightarrow \lnot \chi )$ &
Given &
\$e; \texttt{adantr} to put in ND hypothesis \\

3 & 1 &
... $\vert$ $\psi$ &
$( \varphi \rightarrow \psi )$ &
ND hypothesis assumption &
\texttt{simpr} \\

4 & 4 &
... $\chi$ &
$( ( \varphi \land \psi ) \rightarrow \chi )$ &
$\rightarrow$\,E 1,3 &
\texttt{mpd} 1,3 \\

5 & 6 &
... $\lnot \chi$ &
$( ( \varphi \land \psi ) \rightarrow \lnot \chi )$ &
IT 2 &
\texttt{adantr} 5 \\

6 & 7 &
$\lnot \psi$ &
$( \varphi \rightarrow \lnot \psi )$ &
$\land$\,I 3,4,5 &
\texttt{pm2.65da} 4,6 \\

\end{longtabu}
}


The original used Latin letters; we have replaced them with Greek letters
to follow Metamath naming conventions and so that it is easier to follow
the Metamath translation. The Metamath line-for-line translation of
this natural deduction approach precedes every line with an antecedent
including $\varphi$ and uses the Metamath equivalents of the natural deduction
rules. To add an assumption, the antecedent is modified to include it
(typically by using \texttt{adantr};
\texttt{simpr} is useful when you want to
depend directly on the new assumption, as is shown here).

In Metamath we can represent the two given statements as these hypotheses:

\needspace{2\baselineskip}
\begin{itemize}
\item ex-natded5.5.1 $\vdash ( \varphi \rightarrow ( \psi \rightarrow \chi ) )$
\item ex-natded5.5.2 $\vdash ( \varphi \rightarrow \lnot \chi )$
\end{itemize}

\needspace{4\baselineskip}
Here is the proof in Metamath as a line-by-line translation:

\begin{longtabu}   { l l l X }
\textbf{Step} & \textbf{Hyp} & \textbf{Ref} & \textbf{Ex\-pres\-sion} \\
\endhead
1 & & simpr & $\vdash ( ( \varphi \land \psi ) \rightarrow \psi )$ \\
2 & & ex-natded5.5.1 &
  $\vdash ( \varphi \rightarrow ( \psi \rightarrow \chi ) )$ \\
3 & 2 & adantr &
 $\vdash ( ( \varphi \land \psi ) \rightarrow ( \psi \rightarrow \chi ) )$ \\
4 & 1, 3 & mpd &
 $\vdash ( ( \varphi \land \psi ) \rightarrow \chi ) $ \\
5 & & ex-natded5.5.2 &
 $\vdash ( \varphi \rightarrow \lnot \chi )$ \\
6 & 5 & adantr &
 $\vdash ( ( \varphi \land \psi ) \rightarrow \lnot \chi )$ \\
7 & 4, 6 & pm2.65da &
 $\vdash ( \varphi \rightarrow \lnot \psi )$ \\
\end{longtabu}

Only using specific natural deduction rules directly can lead to very
long proofs, for exactly the same reason that only using axioms directly
in Hilbert-style proofs can lead to very long proofs.
If the goal is short and clear proofs,
then it is better to reuse already-proven assertions
in deduction form than to start from scratch each time
and using only basic natural deduction rules.

\subsection{Strengths of Our Approach}

As far as we know there is nothing else in the literature like either the
weak deduction theorem or Mario Carneiro\index{Carneiro, Mario}'s
natural deduction method.
In order to
transform a hypothesis into an antecedent, the literature's standard
``Deduction Theorem''\index{Deduction Theorem}\index{Standard Deduction Theorem}
requires metalogic outside of the notions provided
by the axiom system. We instead generally prefer to use Mario Carneiro's
natural deduction method, then use the weak deduction theorem in cases
where that is difficult to apply, and only then use the full standard
deduction theorem as a last resort.

The weak deduction theorem\index{Weak Deduction Theorem}
does not require any additional metalogic
but converts an inference directly into a closed form theorem, with
a rigorous proof that uses only the axiom system. Unlike the standard
Deduction Theorem, there is no implicit external justification that we
have to trust in order to use it.

Mario Carneiro's natural deduction\index{natural deduction}
method also does not require any new metalogical
notions. It avoids the Deduction Theorem's metalogic by prefixing the
hypotheses and conclusion of every would-be inference with a universal
antecedent (``$\varphi \rightarrow$'') from the very start.

We think it is impressive and satisfying that we can do so much in a
practical sense without stepping outside of our Hilbert-style axiom system.
Of course our axiomatization, which is in the form of schemes,
contains a metalogic of its own that we exploit. But this metalogic
is relatively simple, and for our Deduction Theorem alternatives,
we primarily use just the direct substitution of expressions for
metavariables.

\begin{sloppy}
\section{Exploring the Set The\-o\-ry Data\-base}\label{exploring}
\end{sloppy}
% NOTE: All examples performed in this section are
% recorded wtih "set width 61" % on set.mm as of 2019-05-28
% commit c1e7849557661260f77cfdf0f97ac4354fbb4f4d.

At this point you may wish to study the \texttt{set.mm}\index{set theory
database (\texttt{set.mm})} file in more detail.  Pay particular
attention to the assumptions needed to define wffs\index{well-formed
formula (wff)} (which are not included above), the variable types
(\texttt{\$f}\index{\texttt{\$f} statement} statements), and the
definitions that are introduced.  Start with some simple theorems in
propositional calculus, making sure you understand in detail each step
of a proof.  Once you get past the first few proofs and become familiar
with the Metamath language, any part of the \texttt{set.mm} database
will be as easy to follow, step by step, as any other part---you won't
have to undergo a ``quantum leap'' in mathematical sophistication to be
able to follow a deep proof in set theory.

Next, you may want to explore how concepts such as natural numbers are
defined and described.  This is probably best done in conjunction with
standard set theory textbooks, which can help give you a higher-level
understanding.  The \texttt{set.mm} database provides references that will get
you started.  From there, you will be on your way towards a very deep,
rigorous understanding of abstract mathematics.

The Metamath\index{Metamath} program can help you peruse a Metamath data\-base,
wheth\-er you are trying to figure out how a certain step follows in a proof or
just have a general curiosity.  We will go through some examples of the
commands, using the \texttt{set.mm}\index{set theory database (\texttt{set.mm})}
database provided with the Metamath software.  These should help get you
started.  See Chapter~\ref{commands} for a more detailed description of
the commands.  Note that we have included the full spelling of all commands to
prevent ambiguity with future commands.  In practice you may type just the
characters needed to specify each command keyword\index{command keyword}
unambiguously, often just one or two characters per keyword, and you don't
need to type them in upper case.

First run the Metamath program as described earlier.  You should see the
\verb/MM>/ prompt.  Read in the \texttt{set.mm} file:\index{\texttt{read}
command}

\begin{verbatim}
MM> read set.mm
Reading source file "set.mm"... 34554442 bytes
34554442 bytes were read into the source buffer.
The source has 155711 statements; 2254 are $a and 32250 are $p.
No errors were found.  However, proofs were not checked.
Type VERIFY PROOF * if you want to check them.
\end{verbatim}

As with most examples in this book, what you will see
will be slightly different because we are continuously
improving our databases (including \texttt{set.mm}).

Let's check the database integrity.  This may take a minute or two to run if
your computer is slow.

\begin{verbatim}
MM> verify proof *
0 10%  20%  30%  40%  50%  60%  70%  80%  90% 100%
..................................................
All proofs in the database were verified in 2.84 s.
\end{verbatim}

No errors were reported, so every proof is correct.

You need to know the names (labels) of theorems before you can look at them.
Often just examining the database file(s) with a text editor is the best
approach.  In \texttt{set.mm} there are many detailed comments, especially near
the beginning, that can help guide you. The \texttt{search} command in the
Metamath program is also handy.  The \texttt{comments} qualifier will list the
statements whose associated comment (the one immediately before it) contain a
string you give it.  For example, if you are studying Enderton's {\em Elements
of Set Theory} \cite{Enderton}\index{Enderton, Herbert B.} you may want to see
the references to it in the database.  The search string \texttt{enderton} is not
case sensitive.  (This will not show you all the database theorems that are in
Enderton's book because there is usually only one citation for a given
theorem, which may appear in several textbooks.)\index{\texttt{search}
command}

\begin{verbatim}
MM> search * "enderton" / comments
12067 unineq $p "... Exercise 20 of [Enderton] p. 32 and ..."
12459 undif2 $p "...Corollary 6K of [Enderton] p. 144. (C..."
12953 df-tp $a "...s. Definition of [Enderton] p. 19. (Co..."
13689 unissb $p ".... Exercise 5 of [Enderton] p. 26 and ..."
\end{verbatim}
\begin{center}
(etc.)
\end{center}

Or you may want to see what theorems have something to do with
conjunction (logical {\sc and}).  The quotes around the search
string are optional when there's no ambiguity.\index{\texttt{search}
command}

\begin{verbatim}
MM> search * conjunction / comments
120 a1d $p "...be replaced with a conjunction ( ~ df-an )..."
662 df-bi $a "...viated form after conjunction is introdu..."
1319 wa $a "...ff definition to include conjunction ('and')."
1321 df-an $a "Define conjunction (logical 'and'). Defini..."
1420 imnan $p "...tion in terms of conjunction. (Contribu..."
\end{verbatim}
\begin{center}
(etc.)
\end{center}


Now we will start to look at some details.  Let's look at the first
axiom of propositional calculus
(we could use \texttt{sh st} to abbreviate
\texttt{show statement}).\index{\texttt{show statement} command}

\begin{verbatim}
MM> show statement ax-1/full
Statement 19 is located on line 881 of the file "set.mm".
"Axiom _Simp_.  Axiom A1 of [Margaris] p. 49.  One of the 3
axioms of propositional calculus.  The 3 axioms are also
        ...
19 ax-1 $a |- ( ph -> ( ps -> ph ) ) $.
Its mandatory hypotheses in RPN order are:
  wph $f wff ph $.
  wps $f wff ps $.
The statement and its hypotheses require the variables:  ph
      ps
The variables it contains are:  ph ps


Statement 49 is located on line 11182 of the file "set.mm".
Its statement number for HTML pages is 6.
"Axiom _Simp_.  Axiom A1 of [Margaris] p. 49.  One of the 3
axioms of propositional calculus.  The 3 axioms are also
given as Definition 2.1 of [Hamilton] p. 28.
...
49 ax-1 $a |- ( ph -> ( ps -> ph ) ) $.
Its mandatory hypotheses in RPN order are:
  wph $f wff ph $.
  wps $f wff ps $.
The statement and its hypotheses require the variables:
  ph ps
The variables it contains are:  ph ps
\end{verbatim}

Compare this to \texttt{ax-1} on p.~\pageref{ax1}.  You can see that the
symbol \texttt{ph} is the {\sc ascii} notation for $\varphi$, etc.  To
see the mathematical symbols for any expression you may typeset it in
\LaTeX\ (type \texttt{help tex} for instructions)\index{latex@{\LaTeX}}
or, easier, just use a text editor to look at the comments where symbols
are first introduced in \texttt{set.mm}.  The hypotheses \texttt{wph}
and \texttt{wps} required by \texttt{ax-1} mean that variables
\texttt{ph} and \texttt{ps} must be wffs.

Next we'll pick a simple theorem of propositional calculus, the Principle of
Identity, which is proved directly from the axioms.  We'll look at the
statement then its proof.\index{\texttt{show statement}
command}

\begin{verbatim}
MM> show statement id1/full
Statement 116 is located on line 11371 of the file "set.mm".
Its statement number for HTML pages is 22.
"Principle of identity.  Theorem *2.08 of [WhiteheadRussell]
p. 101.  This version is proved directly from the axioms for
demonstration purposes.
...
116 id1 $p |- ( ph -> ph ) $= ... $.
Its mandatory hypotheses in RPN order are:
  wph $f wff ph $.
Its optional hypotheses are:  wps wch wth wta wet
      wze wsi wrh wmu wla wka
The statement and its hypotheses require the variables:  ph
These additional variables are allowed in its proof:
      ps ch th ta et ze si rh mu la ka
The variables it contains are:  ph
\end{verbatim}

The optional variables\index{optional variable} \texttt{ps}, \texttt{ch}, etc.\ are
available for use in a proof of this statement if we wish, and were we to do
so we would make use of optional hypotheses \texttt{wps}, \texttt{wch}, etc.  (See
Section~\ref{dollaref} for the meaning of ``optional
hypothesis.''\index{optional hypothesis}) The reason these show up in the
statement display is that statement \texttt{id1} happens to be in their scope
(see Section~\ref{scoping} for the definition of ``scope''\index{scope}), but
in fact in propositional calculus we will never make use of optional
hypotheses or variables.  This becomes important after quantifiers are
introduced, where ``dummy'' variables\index{dummy variable} are often needed
in the middle of a proof.

Let's look at the proof of statement \texttt{id1}.  We'll use the
\texttt{show proof} command, which by default suppresses the
``non-essential'' steps that construct the wffs.\index{\texttt{show proof}
command}
We will display the proof in ``lemmon' format (a non-indented format
with explicit previous step number references) and renumber the
displayed steps:

\begin{verbatim}
MM> show proof id1 /lemmon/renumber
1 ax-1           $a |- ( ph -> ( ph -> ph ) )
2 ax-1           $a |- ( ph -> ( ( ph -> ph ) -> ph ) )
3 ax-2           $a |- ( ( ph -> ( ( ph -> ph ) -> ph ) ) ->
                     ( ( ph -> ( ph -> ph ) ) -> ( ph -> ph )
                                                          ) )
4 2,3 ax-mp      $a |- ( ( ph -> ( ph -> ph ) ) -> ( ph -> ph
                                                          ) )
5 1,4 ax-mp      $a |- ( ph -> ph )
\end{verbatim}

If you have read Section~\ref{trialrun}, you'll know how to interpret this
proof.  Step~2, for example, is an application of axiom \texttt{ax-1}.  This
proof is identical to the one in Hamilton's {\em Logic for Mathematicians}
\cite[p.~32]{Hamilton}\index{Hamilton, Alan G.}.

You may want to look at what
substitutions\index{substitution!variable}\index{variable substitution} are
made into \texttt{ax-1} to arrive at step~2.  The command to do this needs to
know the ``real'' step number, so we'll display the proof again without
the \texttt{renumber} qualifier.\index{\texttt{show proof}
command}

\begin{verbatim}
MM> show proof id1 /lemmon
 9 ax-1          $a |- ( ph -> ( ph -> ph ) )
20 ax-1          $a |- ( ph -> ( ( ph -> ph ) -> ph ) )
24 ax-2          $a |- ( ( ph -> ( ( ph -> ph ) -> ph ) ) ->
                     ( ( ph -> ( ph -> ph ) ) -> ( ph -> ph )
                                                          ) )
25 20,24 ax-mp   $a |- ( ( ph -> ( ph -> ph ) ) -> ( ph -> ph
                                                          ) )
26 9,25 ax-mp    $a |- ( ph -> ph )
\end{verbatim}

The ``real'' step number is 20.  Let's look at its details.

\begin{verbatim}
MM> show proof id1 /detailed_step 20
Proof step 20:  min=ax-1 $a |- ( ph -> ( ( ph -> ph ) -> ph )
  )
This step assigns source "ax-1" ($a) to target "min" ($e).
The source assertion requires the hypotheses "wph" ($f, step
18) and "wps" ($f, step 19).  The parent assertion of the
target hypothesis is "ax-mp" ($a, step 25).
The source assertion before substitution was:
    ax-1 $a |- ( ph -> ( ps -> ph ) )
The following substitutions were made to the source
assertion:
    Variable  Substituted with
     ph        ph
     ps        ( ph -> ph )
The target hypothesis before substitution was:
    min $e |- ph
The following substitution was made to the target hypothesis:
    Variable  Substituted with
     ph        ( ph -> ( ( ph -> ph ) -> ph ) )
\end{verbatim}

This shows the substitutions\index{substitution!variable}\index{variable
substitution} made to the variables in \texttt{ax-1}.  References are made to
steps 18 and 19 which are not shown in our proof display.  To see these steps,
you can display the proof with the \texttt{all} qualifier.

Let's look at a slightly more advanced proof of propositional calculus.  Note
that \verb+/\+ is the symbol for $\wedge$ (logical {\sc and}, also
called conjunction).\index{conjunction ($\wedge$)}
\index{logical {\sc and} ($\wedge$)}

\begin{verbatim}
MM> show statement prth/full
Statement 1791 is located on line 15503 of the file "set.mm".
Its statement number for HTML pages is 559.
"Conjoin antecedents and consequents of two premises.  This
is the closed theorem form of ~ anim12d .  Theorem *3.47 of
[WhiteheadRussell] p. 113.  It was proved by Leibniz,
and it evidently pleased him enough to call it
_praeclarum theorema_ (splendid theorem).
...
1791 prth $p |- ( ( ( ph -> ps ) /\ ( ch -> th ) ) -> ( ( ph
      /\ ch ) -> ( ps /\ th ) ) ) $= ... $.
Its mandatory hypotheses in RPN order are:
  wph $f wff ph $.
  wps $f wff ps $.
  wch $f wff ch $.
  wth $f wff th $.
Its optional hypotheses are:  wta wet wze wsi wrh wmu wla wka
The statement and its hypotheses require the variables:  ph
      ps ch th
These additional variables are allowed in its proof:  ta et
      ze si rh mu la ka
The variables it contains are:  ph ps ch th


MM> show proof prth /lemmon/renumber
1 simpl          $p |- ( ( ( ph -> ps ) /\ ( ch -> th ) ) ->
                                               ( ph -> ps ) )
2 simpr          $p |- ( ( ( ph -> ps ) /\ ( ch -> th ) ) ->
                                               ( ch -> th ) )
3 1,2 anim12d    $p |- ( ( ( ph -> ps ) /\ ( ch -> th ) ) ->
                           ( ( ph /\ ch ) -> ( ps /\ th ) ) )
\end{verbatim}

There are references to a lot of unfamiliar statements.  To see what they are,
you may type the following:

\begin{verbatim}
MM> show proof prth /statement_summary
Summary of statements used in the proof of "prth":

Statement simpl is located on line 14748 of the file
"set.mm".
"Elimination of a conjunct.  Theorem *3.26 (Simp) of
[WhiteheadRussell] p. 112. ..."
  simpl $p |- ( ( ph /\ ps ) -> ph ) $= ... $.

Statement simpr is located on line 14777 of the file
"set.mm".
"Elimination of a conjunct.  Theorem *3.27 (Simp) of
[WhiteheadRussell] ..."
  simpr $p |- ( ( ph /\ ps ) -> ps ) $= ... $.

Statement anim12d is located on line 15445 of the file
"set.mm".
"Conjoin antecedents and consequents in a deduction.
..."
  anim12d.1 $e |- ( ph -> ( ps -> ch ) ) $.
  anim12d.2 $e |- ( ph -> ( th -> ta ) ) $.
  anim12d $p |- ( ph -> ( ( ps /\ th ) -> ( ch /\ ta ) ) )
      $= ... $.
\end{verbatim}
\begin{center}
(etc.)
\end{center}

Of course you can look at each of these statements and their proofs, and
so on, back to the axioms of propositional calculus if you wish.

The \texttt{search} command is useful for finding statements when you
know all or part of their contents.  The following example finds all
statements containing \verb@ph -> ps@ followed by \verb@ch -> th@.  The
\verb@$*@ is a wildcard that matches anything; the \texttt{\$} before the
\verb$*$ prevents conflicts with math symbol token names.  The \verb@*@ after
\texttt{SEARCH} is also a wildcard that in this case means ``match any label.''
\index{\texttt{search} command}

% I'm omitting this one, since readers are unlikely to see it:
% 1096 bisymOLD $p |- ( ( ( ph -> ps ) -> ( ch -> th ) ) -> ( (
%   ( ps -> ph ) -> ( th -> ch ) ) -> ( ( ph <-> ps ) -> ( ch
%    <-> th ) ) ) )
\begin{verbatim}
MM> search * "ph -> ps $* ch -> th"
1791 prth $p |- ( ( ( ph -> ps ) /\ ( ch -> th ) ) -> ( ( ph
    /\ ch ) -> ( ps /\ th ) ) )
2455 pm3.48 $p |- ( ( ( ph -> ps ) /\ ( ch -> th ) ) -> ( (
    ph \/ ch ) -> ( ps \/ th ) ) )
117859 pm11.71 $p |- ( ( E. x ph /\ E. y ch ) -> ( ( A. x (
    ph -> ps ) /\ A. y ( ch -> th ) ) <-> A. x A. y ( ( ph /\
    ch ) -> ( ps /\ th ) ) ) )
\end{verbatim}

Three statements, \texttt{prth}, \texttt{pm3.48},
 and \texttt{pm11.71}, were found to match.

To see what axioms\index{axiom} and definitions\index{definition}
\texttt{prth} ultimately depends on for its proof, you can have the
program backtrack through the hierarchy\index{hierarchy} of theorems and
definitions.\index{\texttt{show trace{\char`\_}back} command}

\begin{verbatim}
MM> show trace_back prth /essential/axioms
Statement "prth" assumes the following axioms ($a
statements):
  ax-1 ax-2 ax-3 ax-mp df-bi df-an
\end{verbatim}

Note that the 3 axioms of propositional calculus and the modus ponens rule are
needed (as expected); in addition, there are a couple of definitions that are used
along the way.  Note that Metamath makes no distinction\index{axiom vs.\
definition} between axioms\index{axiom} and definitions\index{definition}.  In
\texttt{set.mm} they have been distinguished artificially by prefixing their
labels\index{labels in \texttt{set.mm}} with \texttt{ax-} and \texttt{df-}
respectively.  For example, \texttt{df-an} defines conjunction (logical {\sc
and}), which is represented by the symbol \verb+/\+.
Section~\ref{definitions} discusses the philosophy of definitions, and the
Metamath language takes a particularly simple, conservative approach by using
the \texttt{\$a}\index{\texttt{\$a} statement} statement for both axioms and
definitions.

You can also have the program compute how many steps a proof
has\index{proof length} if we were to follow it all the way back to
\texttt{\$a} statements.

\begin{verbatim}
MM> show trace_back prth /essential/count_steps
The statement's actual proof has 3 steps.  Backtracking, a
total of 79 different subtheorems are used.  The statement
and subtheorems have a total of 274 actual steps.  If
subtheorems used only once were eliminated, there would be a
total of 38 subtheorems, and the statement and subtheorems
would have a total of 185 steps.  The proof would have 28349
steps if fully expanded back to axiom references.  The
maximum path length is 38.  A longest path is:  prth <-
anim12d <- syl2and <- sylan2d <- ancomsd <- ancom <- pm3.22
<- pm3.21 <- pm3.2 <- ex <- sylbir <- biimpri <- bicomi <-
bicom1 <- bi2 <- dfbi1 <- impbii <- bi3 <- simprim <- impi <-
con1i <- nsyl2 <- mt3d <- con1d <- notnot1 <- con2i <- nsyl3
<- mt2d <- con2d <- notnot2 <- pm2.18d <- pm2.18 <- pm2.21 <-
pm2.21d <- a1d <- syl <- mpd <- a2i <- a2i.1 .
\end{verbatim}

This tells us that we would have to inspect 274 steps if we want to
verify the proof completely starting from the axioms.  A few more
statistics are also shown.  There are one or more paths back to axioms
that are the longest; this command ferrets out one of them and shows it
to you.  There may be a sense in which the longest path length is
related to how ``deep'' the theorem is.

We might also be curious about what proofs depend on the theorem
\texttt{prth}.  If it is never used later on, we could eliminate it as
redundant if it has no intrinsic interest by itself.\index{\texttt{show
usage} command}

% I decided to show the OLD values here.
\begin{verbatim}
MM> show usage prth
Statement "prth" is directly referenced in the proofs of 18
statements:
  mo3 moOLD 2mo 2moOLD euind reuind reuss2 reusv3i opelopabt
  wemaplem2 rexanre rlimcn2 o1of2 o1rlimmul 2sqlem6 spanuni
  heicant pm11.71
\end{verbatim}

Thus \texttt{prth} is directly used by 18 proofs.
We can use the \texttt{/recursive} qualifier to include indirect use:

\begin{verbatim}
MM> show usage prth /recursive
Statement "prth" directly or indirectly affects the proofs of
24214 statements:
  mo3 mo mo3OLD eu2 moOLD eu2OLD eu3OLD mo4f mo4 eu4 mopick
...
\end{verbatim}

\subsection{A Note on the ``Compact'' Proof Format}

The present version of Metamath\index{Metamath!limitations of
version 0.07.30} will display proofs in a ``compact''\index{compact proof}
format whenever the proof is stored in compressed format in the database.  It
may be be slightly confusing unless you know how to interpret it.  (A
future version may eliminate this format from displays.)  For example,
if you display the complete proof of theorem \texttt{id1} it will start
off as follows:

\begin{verbatim}
MM> show proof id1 /lemmon/all
 1 wph           $f wff ph
 2 wph           $f wff ph
 3 wph           $f wff ph
 4 2,3 wi    @4: $a wff ( ph -> ph )
 5 1,4 wi    @5: $a wff ( ph -> ( ph -> ph ) )
 6 @4            $a wff ( ph -> ph )
\end{verbatim}

\begin{center}
{etc.}
\end{center}

Step 4 has a ``local label,''\index{local label} \texttt{@4}, assigned to it.
Later on, at step 6, the label \texttt{@1} is referenced instead of
displaying the explicit proof for that step.  This technique takes advantage
of the fact that steps in a proof often repeat, especially during the
construction of wffs.  The compact format reduces the number of steps in the
proof display and may actually be preferred by some people, which is one
reason that it is still displayed by the program.

If you want to see the normal format with the ``true'' step numbers, you can
use the following workaround:\index{\texttt{save proof} command}

\begin{verbatim}
MM> save proof id1 /normal
The proof of "id1" has been reformatted and saved internally.
Remember to use WRITE SOURCE to save it permanently.
MM> show proof id1 /lemmon/all
 1 wph           $f wff ph
 2 wph           $f wff ph
 3 wph           $f wff ph
 4 2,3 wi        $a wff ( ph -> ph )
 5 1,4 wi        $a wff ( ph -> ( ph -> ph ) )
 6 wph           $f wff ph
 7 wph           $f wff ph
 8 6,7 wi        $a wff ( ph -> ph )
\end{verbatim}

\begin{center}
{etc.}
\end{center}

Note that the original 6 steps are now 8 steps.  However, the format is
now the same as that described in Chapter~\ref{using}.

\chapter{The Metamath Language}
\label{languagespec}

\begin{quote}
  {\em Thus mathematics may be defined as the subject in which we never know
what we are talking about, nor whether what we are saying is true.}
    \flushright\sc  Bertrand Russell\footnote{\cite[p.~84]{Russell2}.}\\
\end{quote}\index{Russell, Bertrand}

Probably the most striking feature of the Metamath language is its almost
complete absence of hard-wired syntax. Metamath\index{Metamath} does not
understand any mathematics or logic other than that needed to construct finite
sequences of symbols according to a small set of simple, built-in rules.  The
only rule it uses in a proof is the substitution of an expression (symbol
sequence) for a variable, subject to a simple constraint to prevent
bound-variable clashes.  The primitive notions built into Metamath involve the
simple manipulation of finite objects (symbols) that we as humans can easily
visualize and that computers can easily deal with.  They seem to be just
about the simplest notions possible that are required to do standard
mathematics.

This chapter serves as a reference manual for the Metamath\index{Metamath}
language. It covers the tedious technical details of the language, some of
which you may wish to skim in a first reading.  On the other hand, you should
pay close attention to the defined terms in {\bf boldface}; they have precise
meanings that are important to keep in mind for later understanding.  It may
be best to first become familiar with the examples in Chapter~\ref{using} to
gain some motivation for the language.

%% Uncomment this when uncommenting section {formalspec} below
If you have some knowledge of set theory, you may wish to study this
chapter in conjunction with the formal set-theoretical description of the
Metamath language in Appendix~\ref{formalspec}.

We will use the name ``Metamath''\index{Metamath} to mean either the Metamath
computer language or the Metamath software associated with the computer
language.  We will not distinguish these two when the context is clear.

The next section contains the complete specification of the Metamath
language.
It serves as an
authoritative reference and presents the syntax in enough detail to
write a parser\index{parsing Metamath} and proof verifier.  The
specification is terse and it is probably hard to learn the language
directly from it, but we include it here for those impatient people who
prefer to see everything up front before looking at verbose expository
material.  Later sections explain this material and provide examples.
We will repeat the definitions in those sections, and you may skip the
next section at first reading and proceed to Section~\ref{tut1}
(p.~\pageref{tut1}).

\section{Specification of the Metamath Language}\label{spec}
\index{Metamath!specification}

\begin{quote}
  {\em Sometimes one has to say difficult things, but one ought to say
them as simply as one knows how.}
    \flushright\sc  G. H. Hardy\footnote{As quoted in
    \cite{deMillo}, p.~273.}\\
\end{quote}\index{Hardy, G. H.}

\subsection{Preliminaries}\label{spec1}

% Space is technically a printable character, so we'll word things
% carefully so it's unambiguous.
A Metamath {\bf database}\index{database} is built up from a top-level source
file together with any source files that are brought in through file inclusion
commands (see below).  The only characters that are allowed to appear in a
Metamath source file are the 94 non-whitespace printable {\sc
ascii}\index{ascii@{\sc ascii}} characters, which are digits, upper and lower
case letters, and the following 32 special
characters\index{special characters}:\label{spec1chars}

\begin{verbatim}
! " # $ % & ' ( ) * + , - . / :
; < = > ? @ [ \ ] ^ _ ` { | } ~
\end{verbatim}
plus the following characters which are the ``white space'' characters:
space (a printable character),
tab, carriage return, line feed, and form feed.\label{whitespace}
We will use \texttt{typewriter}
font to display the printable characters.

A Metamath database consists of a sequence of three kinds of {\bf
tokens}\index{token} separated by {\bf white space}\index{white space}
(which is any sequence of one or more white space characters).  The set
of {\bf keyword}\index{keyword} tokens is \texttt{\$\char`\{},
\texttt{\$\char`\}}, \texttt{\$c}, \texttt{\$v}, \texttt{\$f},
\texttt{\$e}, \texttt{\$d}, \texttt{\$a}, \texttt{\$p}, \texttt{\$.},
\texttt{\$=}, \texttt{\$(}, \texttt{\$)}, \texttt{\$[}, and
\texttt{\$]}.  The last four are called {\bf auxiliary}\index{auxiliary
keyword} or preprocessing keywords.  A {\bf label}\index{label} token
consists of any combination of letters, digits, and the characters
hyphen, underscore, and period.  A {\bf math symbol}\index{math symbol}
token may consist of any combination of the 92 printable standard {\sc
ascii} characters other than space or \texttt{\$}~. All tokens are
case-sensitive.

\subsection{Preprocessing}

The token \texttt{\$(} begins a {\bf comment} and
\texttt{\$)} ends a comment.\index{\texttt{\$(}
and \texttt{\$)} auxiliary keywords}\index{comment}
Comments may contain any of
the 94 non-whitespace printable characters and white space,
except they may not contain the
2-character sequences \texttt{\$(} or \texttt{\$)} (comments do not nest).
Comments are ignored (treated
like white space) for the purpose of parsing, e.g.,
\texttt{\$( \$[ \$)} is a comment.
See p.~\pageref{mathcomments} for comment typesetting conventions; these
conventions may be ignored for the purpose of parsing.

A {\bf file inclusion command} consists of \texttt{\$[} followed by a file name
followed by \texttt{\$]}.\index{\texttt{\$[} and \texttt{\$]} auxiliary
keywords}\index{included file}\index{file inclusion}
It is only allowed in the outermost scope (i.e., not between
\texttt{\$\char`\{} and \texttt{\$\char`\}})
and must not be inside a statement (e.g., it may not occur
between the label of a \texttt{\$a} statement and its \texttt{\$.}).
The file name may not
contain a \texttt{\$} or white space.  The file must exist.
The case-sensitivity
of its name follows the conventions of the operating system.  The contents of
the file replace the inclusion command.
Included files may include other files.
Only the first reference to a given file is included; any later
references to the same file (whether in the top-level file or in included
files) cause the inclusion command to be ignored (treated like white space).
A verifier may assume that file names with different strings
refer to different files for the purpose of ignoring later references.
A file self-reference is ignored, as is any reference to the top-level file
(to avoid loops).
Included files may not include a \texttt{\$(} without a matching \texttt{\$)},
may not include a \texttt{\$[} without a matching \texttt{\$]}, and may
not include incomplete statements (e.g., a \texttt{\$a} without a matching
\texttt{\$.}).
It is currently unspecified if path references are relative to the process'
current directory or the file's containing directory, so databases should
avoid using pathname separators (e.g., ``/'') in file names.

Like all tokens, the \texttt{\$(}, \texttt{\$)}, \texttt{\$[}, and \texttt{\$]} keywords
must be surrounded by white space.

\subsection{Basic Syntax}

After preprocessing, a database will consist of a sequence of {\bf
statements}.
These are the scoping statements \texttt{\$\char`\{} and
\texttt{\$\char`\}}, along with the \texttt{\$c}, \texttt{\$v},
\texttt{\$f}, \texttt{\$e}, \texttt{\$d}, \texttt{\$a}, and \texttt{\$p}
statements.

A {\bf scoping statement}\index{scoping statement} consists only of its
keyword, \texttt{\$\char`\{} or \texttt{\$\char`\}}.
A \texttt{\$\char`\{} begins a {\bf
block}\index{block} and a matching \texttt{\$\char`\}} ends the block.
Every \texttt{\$\char`\{}
must have a matching \texttt{\$\char`\}}.
Defining it recursively, we say a block
contains a sequence of zero or more tokens other
than \texttt{\$\char`\{} and \texttt{\$\char`\}} and
possibly other blocks.  There is an {\bf outermost
block}\index{block!outermost} not bracketed by \texttt{\$\char`\{} \texttt{\$\char`\}}; the end
of the outermost block is the end of the database.

% LaTeX bug? can't do \bf\tt

A {\bf \$v} or {\bf \$c statement}\index{\texttt{\$v} statement}\index{\texttt{\$c}
statement} consists of the keyword token \texttt{\$v} or \texttt{\$c} respectively,
followed by one or more math symbols,
% The word "token" is used to distinguish "$." from the sentence-ending period.
followed by the \texttt{\$.}\ token.
These
statements {\bf declare}\index{declaration} the math symbols to be {\bf
variables}\index{variable!Metamath} or {\bf constants}\index{constant}
respectively. The same math symbol may not occur twice in a given \texttt{\$v} or
\texttt{\$c} statement.

%c%A math symbol becomes an {\bf active}\index{active math symbol}
%c%when declared and stays active until the end of the block in which it is
%c%declared.  A math symbol may not be declared a second time while it is active,
%c%but it may be declared again after it becomes inactive.

A math symbol becomes {\bf active}\index{active math symbol} when declared
and stays active until the end of the block in which it is declared.  A
variable may not be declared a second time while it is active, but it
may be declared again (as a variable, but not as a constant) after it
becomes inactive.  A constant must be declared in the outermost block and may
not be declared a second time.\footnote{The rules for redeclaration may
become more general in the future;
see p.~\pageref{redeclarationf}.}\index{redeclaration of symbols}

A {\bf \$f statement}\index{\texttt{\$f} statement} consists of a label,
followed by \texttt{\$f}, followed by its typecode (an active constant),
followed by an
active variable, followed by the \texttt{\$.}\ token.  A {\bf \$e
statement}\index{\texttt{\$e} statement} consists of a label, followed
by \texttt{\$e}, followed by its typecode (an active constant),
followed by zero or more
active math symbols, followed by the \texttt{\$.}\ token.  A {\bf
hypothesis}\index{hypothesis} is a \texttt{\$f} or \texttt{\$e}
statement.
The type declared by a \texttt{\$f} statement for a given label
is global even if the variable is not
(e.g., a database may not have \texttt{wff P} in one local scope
nd \texttt{class P} in another).

A {\bf simple \$d statement}\index{\texttt{\$d} statement!simple}
consists of \texttt{\$d}, followed by two different active variables,
followed by the \texttt{\$.}\ token.  A {\bf compound \$d
statement}\index{\texttt{\$d} statement!compound} consists of
\texttt{\$d}, followed by three or more variables (all different),
followed by the \texttt{\$.}\ token.  The order of the variables in a
\texttt{\$d} statement is unimportant.  A compound \texttt{\$d}
statement is equivalent to a set of simple \texttt{\$d} statements, one
for each possible pair of variables occurring in the compound
\texttt{\$d} statement.  Henceforth in this specification we shall
assume all \texttt{\$d} statements are simple.  A \texttt{\$d} statement
is also called a {\bf disjoint} (or {\bf distinct}) {\bf variable
restriction}.\index{disjoint-variable restriction}

A {\bf \$a statement}\index{\texttt{\$a} statement} consists of a label,
followed by \texttt{\$a}, followed by its typecode (an active constant),
followed by
zero or more active math symbols, followed by the \texttt{\$.}\ token.  A {\bf
\$p statement}\index{\texttt{\$p} statement} consists of a label,
followed by \texttt{\$p}, followed by its typecode (an active constant),
followed by
zero or more active math symbols, followed by \texttt{\$=}, followed by
a sequence of labels, followed by the \texttt{\$.}\ token.  An {\bf
assertion}\index{assertion} is a \texttt{\$a} or \texttt{\$p} statement.

A \texttt{\$f}, \texttt{\$e}, or \texttt{\$d} statement is {\bf active}\index{active
statement} from the place it occurs until the end of the block it occurs in.
A \texttt{\$a} or \texttt{\$p} statement is {\bf active} from the place it occurs
through the end of the database.
There may not be two active \texttt{\$f} statements containing the same
variable.  Each variable in a \texttt{\$e}, \texttt{\$a}, or
\texttt{\$p} statement must exist in an active \texttt{\$f}
statement.\footnote{This requirement can greatly simplify the
unification algorithm (substitution calculation) required by proof
verification.}

%The label that begins each \texttt{\$f}, \texttt{\$e}, \texttt{\$a}, and
%\texttt{\$p} statement must be unique.
Each label token must be unique, and
no label token may match any math symbol
token.\label{namespace}\footnote{This
restriction was added on June 24, 2006.
It is not theoretically necessary but is imposed to make it easier to
write certain parsers.}

The set of {\bf mandatory variables}\index{mandatory variable} associated with
an assertion is the set of (zero or more) variables in the assertion and in any
active \texttt{\$e} statements.  The (possibly empty) set of {\bf mandatory
hypotheses}\index{mandatory hypothesis} is the set of all active \texttt{\$f}
statements containing mandatory variables, together with all active \texttt{\$e}
statements.
The set of {\bf mandatory {\bf \$d} statements}\index{mandatory
disjoint-variable restriction} associated with an assertion are those active
\texttt{\$d} statements whose variables are both among the assertion's
mandatory variables.

\subsection{Proof Verification}\label{spec4}

The sequence of labels between the \texttt{\$=} and \texttt{\$.}\ tokens
in a \texttt{\$p} statement is a {\bf proof}.\index{proof!Metamath} Each
label in a proof must be the label of an active statement other than the
\texttt{\$p} statement itself; thus a label must refer either to an
active hypothesis of the \texttt{\$p} statement or to an earlier
assertion.

An {\bf expression}\index{expression} is a sequence of math symbols. A {\bf
substitution map}\index{substitution map} associates a set of variables with a
set of expressions.  It is acceptable for a variable to be mapped to an
expression containing it.  A {\bf
substitution}\index{substitution!variable}\index{variable substitution} is the
simultaneous replacement of all variables in one or more expressions with the
expressions that the variables map to.

A proof is scanned in order of its label sequence.  If the label refers to an
active hypothesis, the expression in the hypothesis is pushed onto a
stack.\index{stack}\index{RPN stack}  If the label refers to an assertion, a
(unique) substitution must exist that, when made to the mandatory hypotheses
of the referenced assertion, causes them to match the topmost (i.e.\ most
recent) entries of the stack, in order of occurrence of the mandatory
hypotheses, with the topmost stack entry matching the last mandatory
hypothesis of the referenced assertion.  As many stack entries as there are
mandatory hypotheses are then popped from the stack.  The same substitution is
made to the referenced assertion, and the result is pushed onto the stack.
After the last label in the proof is processed, the stack must have a single
entry that matches the expression in the \texttt{\$p} statement containing the
proof.

%c%{\footnotesize\begin{quotation}\index{redeclaration of symbols}
%c%{{\em Comment.}\label{spec4comment} Whenever a math symbol token occurs in a
%c%{\texttt{\$c} or \texttt{\$v} statement, it is considered to designate a distinct new
%c%{symbol, even if the same token was previously declared (and is now inactive).
%c%{Thus a math token declared as a constant in two different blocks is considered
%c%{to designate two distinct constants (even though they have the same name).
%c%{The two constants will not match in a proof that references both blocks.
%c%{However, a proof referencing both blocks is acceptable as long as it doesn't
%c%{require that the constants match.  Similarly, a token declared to be a
%c%{constant for a referenced assertion will not match the same token declared to
%c%{be a variable for the \texttt{\$p} statement containing the proof.  In the case
%c%{of a token declared to be a variable for a referenced assertion, this is not
%c%{an issue since the variable can be substituted with whatever expression is
%c%{needed to achieve the required match.
%c%{\end{quotation}}
%c2%A proof may reference an assertion that contains or whose hypotheses contain a
%c2%constant that is not active for the \texttt{\$p} statement containing the proof.
%c2%However, the final result of the proof may not contain that constant. A proof
%c2%may also reference an assertion that contains or whose hypotheses contain a
%c2%variable that is not active for the \texttt{\$p} statement containing the proof.
%c2%That variable, of course, will be substituted with whatever expression is
%c2%needed to achieve the required match.

A proof may contain a \texttt{?}\ in place of a label to indicate an unknown step
(Section~\ref{unknown}).  A proof verifier may ignore any proof containing
\texttt{?}\ but should warn the user that the proof is incomplete.

A {\bf compressed proof}\index{compressed proof}\index{proof!compressed} is an
alternate proof notation described in Appen\-dix~\ref{compressed}; also see
references to ``compressed proof'' in the Index.  Compressed proofs are a
Metamath language extension which a complete proof verifier should be able to
parse and verify.

\subsubsection{Verifying Disjoint Variable Restrictions}

Each substitution made in a proof must be checked to verify that any
disjoint variable restrictions are satisfied, as follows.

If two variables replaced by a substitution exist in a mandatory \texttt{\$d}
statement\index{\texttt{\$d} statement} of the assertion referenced, the two
expressions resulting from the substitution must satisfy the following
conditions.  First, the two expressions must have no variables in common.
Second, each possible pair of variables, one from each expression, must exist
in an active \texttt{\$d} statement of the \texttt{\$p} statement containing the
proof.

\vskip 1ex

This ends the specification of the Metamath language;
see Appendix \ref{BNF} for its syntax in
Extended Backus--Naur Form (EBNF)\index{Extended Backus--Naur Form}\index{EBNF}.

\section{The Basic Keywords}\label{tut1}

Our expository material begins here.

Like most computer languages, Metamath\index{Metamath} takes its input from
one or more {\bf source files}\index{source file} which contain characters
expressed in the standard {\sc ascii} (American Standard Code for Information
Interchange)\index{ascii@{\sc ascii}} code for computers.  A source file
consists of a series of {\bf tokens}\index{token}, which are strings of
non-whitespace
printable characters (from the set of 94 shown on p.~\pageref{spec1chars})
separated by {\bf white space}\index{white space} (spaces, tabs, carriage
returns, line feeds, and form feeds). Any string consisting only of these
characters is treated the same as a single space.  The non-whitespace printable
characters\index{printable character} that Metamath recognizes are the 94
characters on standard {\sc ascii} keyboards.

Metamath has the ability to join several files together to form its
input (Section~\ref{include}).  We call the aggregate contents of all
the files after they have been joined together a {\bf
database}\index{database} to distinguish it from an individual source
file.  The tokens in a database consist of {\bf
keywords}\index{keyword}, which are built into the language, together
with two kinds of user-defined tokens called {\bf labels}\index{label}
and {\bf math symbols}\index{math symbol}.  (Often we will simply say
{\bf symbol}\index{symbol} instead of math symbol for brevity).  The set
of {\bf basic keywords}\index{basic keyword} is
\texttt{\$c}\index{\texttt{\$c} statement},
\texttt{\$v}\index{\texttt{\$v} statement},
\texttt{\$e}\index{\texttt{\$e} statement},
\texttt{\$f}\index{\texttt{\$f} statement},
\texttt{\$d}\index{\texttt{\$d} statement},
\texttt{\$a}\index{\texttt{\$a} statement},
\texttt{\$p}\index{\texttt{\$p} statement},
\texttt{\$=}\index{\texttt{\$=} keyword},
\texttt{\$.}\index{\texttt{\$.}\ keyword},
\texttt{\$\char`\{}\index{\texttt{\$\char`\{} and \texttt{\$\char`\}}
keywords}, and \texttt{\$\char`\}}.  This is the complete set of
syntactical elements of what we call the {\bf basic
language}\index{basic language} of Metamath, and with them you can
express all of the mathematics that were intended by the design of
Metamath.  You should make it a point to become very familiar with them.
Table~\ref{basickeywords} lists the basic keywords along with a brief
description of their functions.  For now, this description will give you
only a vague notion of what the keywords are for; later we will describe
the keywords in detail.


\begin{table}[htp] \caption{Summary of the basic Metamath
keywords} \label{basickeywords}
\begin{center}
\begin{tabular}{|p{4pc}|l|}
\hline
\em \centering Keyword&\em Description\\
\hline
\hline
\centering
   \texttt{\$c}&Constant symbol declaration\\
\hline
\centering
   \texttt{\$v}&Variable symbol declaration\\
\hline
\centering
   \texttt{\$d}&Disjoint variable restriction\\
\hline
\centering
   \texttt{\$f}&Variable-type (``floating'') hypothesis\\
\hline
\centering
   \texttt{\$e}&Logical (``essential'') hypothesis\\
\hline
\centering
   \texttt{\$a}&Axiomatic assertion\\
\hline
\centering
   \texttt{\$p}&Provable assertion\\
\hline
\centering
   \texttt{\$=}&Start of proof in \texttt{\$p} statement\\
\hline
\centering
   \texttt{\$.}&End of the above statement types\\
\hline
\centering
   \texttt{\$\char`\{}&Start of block\\
\hline
\centering
   \texttt{\$\char`\}}&End of block\\
\hline
\end{tabular}
\end{center}
\end{table}

%For LaTeX bug(?) where it puts tables on blank page instead of btwn text
%May have to adjust if text changes
%\newpage

There are some additional keywords, called {\bf auxiliary
keywords}\index{auxiliary keyword} that help make Metamath\index{Metamath}
more practical. These are part of the {\bf extended language}\index{extended
language}. They provide you with a means to put comments into a Metamath
source file\index{source file} and reference other source files.  We will
introduce these in later sections. Table~\ref{otherkeywords} summarizes them
so that you can recognize them now if you want to peruse some source
files while learning the basic keywords.


\begin{table}[htp] \caption{Auxiliary Metamath
keywords} \label{otherkeywords}
\begin{center}
\begin{tabular}{|p{4pc}|l|}
\hline
\em \centering Keyword&\em Description\\
\hline
\hline
\centering
   \texttt{\$(}&Start of comment\\
\hline
\centering
   \texttt{\$)}&End of comment\\
\hline
\centering
   \texttt{\$[}&Start of included source file name\\
\hline
\centering
   \texttt{\$]}&End of included source file name\\
\hline
\end{tabular}
\end{center}
\end{table}
\index{\texttt{\$(} and \texttt{\$)} auxiliary keywords}
\index{\texttt{\$[} and \texttt{\$]} auxiliary keywords}


Unlike those in some computer languages, the keywords\index{keyword} are short
two-character sequences rather than English-like words.  While this may make
them slightly more difficult to remember at first, their brevity allows
them to blend in with the mathematics being described, not
distract from it, like punctuation marks.


\subsection{User-Defined Tokens}\label{dollardollar}\index{token}

As you may have noticed, all keywords\index{keyword} begin with the \texttt{\$}
character.  This mundane monetary symbol is not ordinarily used in higher
mathematics (outside of grant proposals), so we have appropriated it to
distinguish the Metamath\index{Metamath} keywords from ordinary mathematical
symbols. The \texttt{\$} character is thus considered special and may not be
used as a character in a user-defined token.  All tokens and keywords are
case-sensitive; for example, \texttt{n} is considered to be a different character
from \texttt{N}.  Case-sensitivity makes the available {\sc ascii} character set
as rich as possible.

\subsubsection{Math Symbol Tokens}\index{token}

Math symbols\index{math symbol} are tokens used to represent the symbols
that appear in ordinary mathematical formulas.  They may consist of any
combination of the 93 non-whitespace printable {\sc ascii} characters other than
\texttt{\$}~. Some examples are \texttt{x}, \texttt{+}, \texttt{(},
\texttt{|-}, \verb$!%@?&$, and \texttt{bounded}.  For readability, it is
best to try to make these look as similar to actual mathematical symbols
as possible, within the constraints of the {\sc ascii} character set, in
order to make the resulting mathematical expressions more readable.

In the Metamath\index{Metamath} language, you express ordinary
mathematical formulas and statements as sequences of math symbols such
as \texttt{2 + 2 = 4} (five symbols, all constants).\footnote{To
eliminate ambiguity with other expressions, this is expressed in the set
theory database \texttt{set.mm} as \texttt{|- ( 2 + 2
 ) = 4 }, whose \LaTeX\ equivalent is $\vdash
(2+2)=4$.  The \,$\vdash$ means ``is a theorem'' and the
parentheses allow explicit associative grouping.}\index{turnstile
({$\,\vdash$})} They may even be English
sentences, as in \texttt{E is closed and bounded} (five symbols)---here
\texttt{E} would be a variable and the other four symbols constants.  In
principle, a Metamath database could be constructed to work with almost
any unambiguous English-language mathematical statement, but as a
practical matter the definitions needed to provide for all possible
syntax variations would be cumbersome and distracting and possibly have
subtle pitfalls accidentally built in.  We generally recommend that you
express mathematical statements with compact standard mathematical
symbols whenever possible and put their English-language descriptions in
comments.  Axioms\index{axiom} and definitions\index{definition}
(\texttt{\$a}\index{\texttt{\$a} statement} statements) are the only
places where Metamath will not detect an error, and doing this will help
reduce the number of definitions needed.

You are free to use any tokens\index{token} you like for math
symbols\index{math symbol}.  Appendix~\ref{ASCII} recommends token names to
use for symbols in set theory, and we suggest you adopt these in order to be
able to include the \texttt{set.mm} set theory database in your database.  For
printouts, you can convert the tokens in a database
to standard mathematical symbols with the \LaTeX\ typesetting program.  The
Metamath command \texttt{open tex} {\em filename}\index{\texttt{open tex} command}
produces output that can be read by \LaTeX.\index{latex@{\LaTeX}}
The correspondence
between tokens and the actual symbols is made by \texttt{latexdef}
statements inside a special database comment tagged
with \texttt{\$t}.\index{\texttt{\$t} comment}\index{typesetting comment}
  You can edit
this comment to change the definitions or add new ones.
Appendix~\ref{ASCII} describes how to do this in more detail.

% White space\index{white space} is normally used to separate math
% symbol\index{math symbol} tokens, but they may be juxtaposed without white
% space in \texttt{\$d}\index{\texttt{\$d} statement}, \texttt{\$e}\index{\texttt{\$e}
% statement}, \texttt{\$f}\index{\texttt{\$f} statement}, \texttt{\$a}\index{\texttt{\$a}
% statement}, and \texttt{\$p}\index{\texttt{\$p} statement} statements when no
% ambiguity will result.  Specifically, Metamath parses the math symbol sequence
% in one of these statements in the following manner:  when the math symbol
% sequence has been broken up into tokens\index{token} up to a given character,
% the next token is the longest string of characters that could constitute a
% math symbol that is active\index{active
% math symbol} at that point.  (See Section~\ref{scoping} for the
% definition of an active math symbol.)  For example, if \texttt{-}, \texttt{>}, and
% \texttt{->} are the only active math symbols, the juxtaposition \texttt{>-} will be
% interpreted as the two symbols \texttt{>} and \texttt{-}, whereas \texttt{->} will
% always be interpreted as that single symbol.\footnote{For better readability we
% recommend a white space between each token.  This also makes searching for a
% symbol easier to do with an editor.  Omission of optional white space is useful
% for reducing typing when assigning an expression to a temporary
% variable\index{temporary variable} with the \texttt{let variable} Metamath
% program command.}\index{\texttt{let variable} command}
%
% Keywords\index{keyword} may be placed next to math symbols without white
% space\index{white space} between them.\footnote{Again, we do not recommend
% this for readability.}
%
% The math symbols\index{math symbol} in \texttt{\$c}\index{\texttt{\$c} statement}
% and \texttt{\$v}\index{\texttt{\$v} statement} statements must always be separated
% by white space\index{white
% space}, for the obvious reason that these statements define the names
% of the symbols.
%
% Math symbols referred to in comments (see Section~\ref{comments}) must also be
% separated by white space.  This allows you to make comments about symbols that
% are not yet active\index{active
% math symbol}.  (The ``math mode'' feature of comments is also a quick and
% easy way to obtain word processing text with embedded mathematical symbols,
% independently of the main purpose of Metamath; the way to do this is described
% in Section~\ref{comments})

\subsubsection{Label Tokens}\index{token}\index{label}

Label tokens are used to identify Metamath\index{Metamath} statements for
later reference. Label tokens may contain only letters, digits, and the three
characters period, hyphen, and underscore:
\begin{verbatim}
. - _
\end{verbatim}

A label is {\bf declared}\index{label declaration} by placing it immediately
before the keyword of the statement it identifies.  For example, the label
\texttt{axiom.1} might be declared as follows:
\begin{verbatim}
axiom.1 $a |- x = x $.
\end{verbatim}

Each \texttt{\$e}\index{\texttt{\$e} statement},
\texttt{\$f}\index{\texttt{\$f} statement},
\texttt{\$a}\index{\texttt{\$a} statement}, and
\texttt{\$p}\index{\texttt{\$p} statement} statement in a database must
have a label declared for it.  No other statement types may have label
declarations.  Every label must be unique.

A label (and the statement it identifies) is {\bf referenced}\index{label
reference} by including the label between the \texttt{\$=}\index{\texttt{\$=}
keyword} and \texttt{\$.}\index{\texttt{\$.}\ keyword}\ keywords in a \texttt{\$p}
statement.  The sequence of labels\index{label sequence} between these two
keywords is called a {\bf proof}\index{proof}.  An example of a statement with
a proof that we will encounter later (Section~\ref{proof}) is
\begin{verbatim}
wnew $p wff ( s -> ( r -> p ) )
     $= ws wr wp w2 w2 $.
\end{verbatim}

You don't have to know what this means just yet, but you should know that the
label \texttt{wnew} is declared by this \texttt{\$p} statement and that the labels
\texttt{ws}, \texttt{wr}, \texttt{wp}, and \texttt{w2} are assumed to have been declared
earlier in the database and are referenced here.

\subsection{Constants and Variables}
\index{constant}
\index{variable}

An {\bf expression}\index{expression} is any sequence of math
symbols, possibly empty.

The basic Metamath\index{Metamath} language\index{basic language} has two
kinds of math symbols\index{math symbol}:  {\bf constants}\index{constant} and
{\bf variables}\index{variable}.  In a Metamath proof, a constant may not be
substituted with any expression.  A variable can be
substituted\index{substitution!variable}\index{variable substitution} with any
expression.  This sequence may include other variables and may even include
the variable being substituted.  This substitution takes place when proofs are
verified, and it will be described in Section~\ref{proof}.  The \texttt{\$f}
statement (described later in Section~\ref{dollaref}) is used to specify the
{\bf type} of a variable (i.e.\ what kind of
variable it is)\index{variable type}\index{type} and
give it a meaning typically
associated with a ``metavariable''\index{metavariable}\footnote{A metavariable
is a variable that ranges over the syntactical elements of the object language
being discussed; for example, one metavariable might represent a variable of
the object language and another metavariable might represent a formula in the
object language.} in ordinary mathematics; for example, a variable may be
specified to be a wff or well-formed formula (in logic), a set (in set
theory), or a non-negative integer (in number theory).

%\subsection{The \texttt{\$c} and \texttt{\$v} Declaration Statements}
\subsection{The \texttt{\$c} and \texttt{\$v} Declaration Statements}
\index{\texttt{\$c} statement}
\index{constant declaration}
\index{\texttt{\$v} statement}
\index{variable declaration}

Constants are introduced or {\bf declared}\index{constant declaration}
with \texttt{\$c}\index{\texttt{\$c} statement} statements, and
variables are declared\index{variable declaration} with
\texttt{\$v}\index{\texttt{\$v} statement} statements.  A {\bf simple}
declaration\index{simple declaration} statement introduces a single
constant or variable.  Its syntax is one of the following:
\begin{center}
  \texttt{\$c} {\em math-symbol} \texttt{\$.}\\
  \texttt{\$v} {\em math-symbol} \texttt{\$.}
\end{center}
The notation {\em math-symbol} means any math symbol token\index{token}.

Some examples of simple declaration statements are:
\begin{center}
  \texttt{\$c + \$.}\\
  \texttt{\$c -> \$.}\\
  \texttt{\$c ( \$.}\\
  \texttt{\$v x \$.}\\
  \texttt{\$v y2 \$.}
\end{center}

The characters in a math symbol\index{math symbol} being declared are
irrelevant to Meta\-math; for example, we could declare a right parenthesis to
be a variable,
\begin{center}
  \texttt{\$v ) \$.}\\
\end{center}
although this would be unconventional.

A {\bf compound} declaration\index{compound declaration} statement is a
shorthand for declaring several symbols at once.  Its syntax is one of the
following:
\begin{center}
  \texttt{\$c} {\em math-symbol}\ \,$\cdots$\ {\em math-symbol} \texttt{\$.}\\
  \texttt{\$v} {\em math-symbol}\ \,$\cdots$\ {\em math-symbol} \texttt{\$.}
\end{center}\index{\texttt{\$c} statement}
Here, the ellipsis (\ldots) means any number of {\em math-symbol}\,s.

An example of a compound declaration statement is:
\begin{center}
  \texttt{\$v x y mu \$.}\\
\end{center}
This is equivalent to the three simple declaration statements
\begin{center}
  \texttt{\$v x \$.}\\
  \texttt{\$v y \$.}\\
  \texttt{\$v mu \$.}\\
\end{center}
\index{\texttt{\$v} statement}

There are certain rules on where in the database math symbols may be declared,
what sections of the database are aware of them (i.e.\ where they are
``active''), and when they may be declared more than once.  These will be
discussed in Section~\ref{scoping} and specifically on
p.~\pageref{redeclaration}.

\subsection{The \texttt{\$d} Statement}\label{dollard}
\index{\texttt{\$d} statement}

The \texttt{\$d} statement is called a {\bf disjoint-variable restriction}.  The
syntax of the {\bf simple} version of this statement is
\begin{center}
  \texttt{\$d} {\em variable variable} \texttt{\$.}
\end{center}
where each {\em variable} is a previously declared variable and the two {\em
variable}\,s are different.  (More specifically, each  {\em variable} must be
an {\bf active} variable\index{active math symbol}, which means there must be
a previous \texttt{\$v} statement whose {\bf scope}\index{scope} includes the
\texttt{\$d} statement.  These terms will be defined when we discuss scoping
statements in Section~\ref{scoping}.)

In ordinary mathematics, formulas may arise that are true if the variables in
them are distinct\index{distinct variables}, but become false when those
variables are made identical. For example, the formula in logic $\exists x\,x
\neq y$, which means ``for a given $y$, there exists an $x$ that is not equal
to $y$,'' is true in most mathematical theories (namely all non-trivial
theories\index{non-trivial theory}, i.e.\ those that describe more than one
individual, such as arithmetic).  However, if we substitute $y$ with $x$, we
obtain $\exists x\,x \neq x$, which is always false, as it means ``there
exists something that is not equal to itself.''\footnote{If you are a
logician, you will recognize this as the improper substitution\index{proper
substitution}\index{substitution!proper} of a free variable\index{free
variable} with a bound variable\index{bound variable}.  Metamath makes no
inherent distinction between free and bound variables; instead, you let
Metamath know what substitutions are permissible by using \texttt{\$d} statements
in the right way in your axiom system.}\index{free vs.\ bound variable}  The
\texttt{\$d} statement allows you to specify a restriction that forbids the
substitution of one variable with another.  In
this case, we would use the statement
\begin{center}
  \texttt{\$d x y \$.}
\end{center}\index{\texttt{\$d} statement}
to specify this restriction.

The order in which the variables appear in a \texttt{\$d} statement is not
important.  We could also use
\begin{center}
  \texttt{\$d y x \$.}
\end{center}

The \texttt{\$d} statement is actually more general than this, as the
``disjoint''\index{disjoint variables} in its name suggests.  The full meaning
is that if any substitution is made to its two variables (during the
course of a proof that references a \texttt{\$a} or \texttt{\$p} statement
associated with the \texttt{\$d}), the two expressions that result from the
substitution must have no variables in common.  In addition, each possible
pair of variables, one from each expression, must be in a \texttt{\$d} statement
associated with the statement being proved.  (This requirement forces the
statement being proved to ``inherit'' the original disjoint variable
restriction.)

For example, suppose \texttt{u} is a variable.  If the restriction
\begin{center}
  \texttt{\$d A B \$.}
\end{center}
has been specified for a theorem referenced in a
proof, we may not substitute \texttt{A} with \mbox{\tt a + u} and
\texttt{B} with \mbox{\tt b + u} because these two symbol sequences have the
variable \texttt{u} in common.  Furthermore, if \texttt{a} and \texttt{b} are
variables, we may not substitute \texttt{A} with \texttt{a} and \texttt{B} with \texttt{b}
unless we have also specified \texttt{\$d a b} for the theorem being proved; in
other words, the \texttt{\$d} property associated with a pair of variables must
be effectively preserved after substitution.

The \texttt{\$d}\index{\texttt{\$d} statement} statement does {\em not} mean ``the
two variables may not be substituted with the same thing,'' as you might think
at first.  For example, substituting each of \texttt{A} and \texttt{B} in the above
example with identical symbol sequences consisting only of constants does not
cause a disjoint variable conflict, because two symbol sequences have no
variables in common (since they have no variables, period).  Similarly, a
conflict will not occur by substituting the two variables in a \texttt{\$d}
statement with the empty symbol sequence\index{empty substitution}.

The \texttt{\$d} statement does not have a direct counterpart in
ordinary mathematics, partly because the variables\index{variable} of
Metamath are not really the same as the variables\index{variable!in
ordinary mathematics} of ordinary mathematics but rather are
metavariables\index{metavariable} ranging over them (as well as over
other kinds of symbols and groups of symbols).  Depending on the
situation, we may informally interpret the \texttt{\$d} statement in
different ways.  Suppose, for example, that \texttt{x} and \texttt{y}
are variables ranging over numbers (more precisely, that \texttt{x} and
\texttt{y} are metavariables ranging over variables that range over
numbers), and that \texttt{ph} ($\varphi$) and \texttt{ps} ($\psi$) are
variables (more precisely, metavariables) ranging over formulas.  We can
make the following interpretations that correspond to the informal
language of ordinary mathematics:
\begin{quote}
\begin{tabbing}
\texttt{\$d x y \$.} means ``assume $x$ and $y$ are
distinct variables.''\\
\texttt{\$d x ph \$.} means ``assume $x$ does not
occur in $\varphi$.''\\
\texttt{\$d ph ps \$.} \=means ``assume $\varphi$ and
$\psi$ have no variables\\ \>in common.''
\end{tabbing}
\end{quote}\index{\texttt{\$d} statement}

\subsubsection{Compound \texttt{\$d} Statements}

The {\bf compound} version of the \texttt{\$d} statement is a shorthand for
specifying several variables whose substitutions must be pairwise disjoint.
Its syntax is:
\begin{center}
  \texttt{\$d} {\em variable}\ \,$\cdots$\ {\em variable} \texttt{\$.}
\end{center}\index{\texttt{\$d} statement}
Here, {\em variable} represents the token of a previously declared
variable (specifically, an active variable) and all {\em variable}\,s are
different.  The compound \texttt{\$d}
statement is internally broken up by Metamath into one simple \texttt{\$d}
statement for each possible pair of variables in the original \texttt{\$d}
statement.  For example,
\begin{center}
  \texttt{\$d w x y z \$.}
\end{center}
is equivalent to
\begin{center}
  \texttt{\$d w x \$.}\\
  \texttt{\$d w y \$.}\\
  \texttt{\$d w z \$.}\\
  \texttt{\$d x y \$.}\\
  \texttt{\$d x z \$.}\\
  \texttt{\$d y z \$.}
\end{center}

Two or more simple \texttt{\$d} statements specifying the same variable pair are
internally combined into a single \texttt{\$d} statement.  Thus the set of three
statements
\begin{center}
  \texttt{\$d x y \$.}
  \texttt{\$d x y \$.}
  \texttt{\$d y x \$.}
\end{center}
is equivalent to
\begin{center}
  \texttt{\$d x y \$.}
\end{center}

Similarly, compound \texttt{\$d} statements, after being internally broken up,
internally have their common variable pairs combined.  For example the
set of statements
\begin{center}
  \texttt{\$d x y A \$.}
  \texttt{\$d x y B \$.}
\end{center}
is equivalent to
\begin{center}
  \texttt{\$d x y \$.}
  \texttt{\$d x A \$.}
  \texttt{\$d y A \$.}
  \texttt{\$d x y \$.}
  \texttt{\$d x B \$.}
  \texttt{\$d y B \$.}
\end{center}
which is equivalent to
\begin{center}
  \texttt{\$d x y \$.}
  \texttt{\$d x A \$.}
  \texttt{\$d y A \$.}
  \texttt{\$d x B \$.}
  \texttt{\$d y B \$.}
\end{center}

Metamath\index{Metamath} automatically verifies that all \texttt{\$d}
restrictions are met whenever it verifies proofs.  \texttt{\$d} statements are
never referenced directly in proofs (this is why they do not have
labels\index{label}), but Metamath is always aware of which ones must be
satisfied (i.e.\ are active) and will notify you with an error message if any
violation occurs.

To illustrate how Metamath detects a missing \texttt{\$d}
statement, we will look at the following example from the
\texttt{set.mm} database.

\begin{verbatim}
$d x z $.  $d y z $.
$( Theorem to add distinct quantifier to atomic formula. $)
ax17eq $p |- ( x = y -> A. z x = y ) $=...
\end{verbatim}

This statement has the obvious requirement that $z$ must be
distinct\index{distinct variables} from $x$ in theorem \texttt{ax17eq} that
states $x=y \rightarrow \forall z \, x=y$ (well, obvious if you're a logician,
for otherwise we could conclude  $x=y \rightarrow \forall x \, x=y$, which is
false when the free variables $x$ and $y$ are equal.).

Let's look at what happens if we edit the database to comment out this
requirement.

\begin{verbatim}
$( $d x z $. $) $d y z $.
$( Theorem to add distinct quantifier to atomic formula. $)
ax17eq $p |- ( x = y -> A. z x = y ) $=...
\end{verbatim}

When it tries to verify the proof, Metamath will tell you that \texttt{x} and
\texttt{z} must be disjoint, because one of its steps references an axiom or
theorem that has this requirement.

\begin{verbatim}
MM> verify proof ax17eq
ax17eq ?Error at statement 1918, label "ax17eq", type "$p":
      vz wal wi vx vy vz ax-13 vx vy weq vz vx ax-c16 vx vy
                                               ^^^^^
There is a disjoint variable ($d) violation at proof step 29.
Assertion "ax-c16" requires that variables "x" and "y" be
disjoint.  But "x" was substituted with "z" and "y" was
substituted with "x".  The assertion being proved, "ax17eq",
does not require that variables "z" and "x" be disjoint.
\end{verbatim}

We can see the substitutions into \texttt{ax-c16} with the following command.

\begin{verbatim}
MM> show proof ax17eq / detailed_step 29
Proof step 29:  pm2.61dd.2=ax-c16 $a |- ( A. z z = x -> ( x =
  y -> A. z x = y ) )
This step assigns source "ax-c16" ($a) to target "pm2.61dd.2"
($e).  The source assertion requires the hypotheses "wph"
($f, step 26), "vx" ($f, step 27), and "vy" ($f, step 28).
The parent assertion of the target hypothesis is "pm2.61dd"
($p, step 36).
The source assertion before substitution was:
    ax-c16 $a |- ( A. x x = y -> ( ph -> A. x ph ) )
The following substitutions were made to the source
assertion:
    Variable  Substituted with
     x         z
     y         x
     ph        x = y
The target hypothesis before substitution was:
    pm2.61dd.2 $e |- ( ph -> ch )
The following substitutions were made to the target
hypothesis:
    Variable  Substituted with
     ph        A. z z = x
     ch        ( x = y -> A. z x = y )
\end{verbatim}

The disjoint variable restrictions of \texttt{ax-c16} can be seen from the
\texttt{show state\-ment} command.  The line that begins ``\texttt{Its mandatory
dis\-joint var\-i\-able pairs are:}\ldots'' lists any \texttt{\$d} variable
pairs in brackets.

\begin{verbatim}
MM> show statement ax-c16/full
Statement 3033 is located on line 9338 of the file "set.mm".
"Axiom of Distinct Variables. ..."
  ax-c16 $a |- ( A. x x = y -> ( ph -> A. x ph ) ) $.
Its mandatory hypotheses in RPN order are:
  wph $f wff ph $.
  vx $f set x $.
  vy $f set y $.
Its mandatory disjoint variable pairs are:  <x,y>
The statement and its hypotheses require the variables:  x y
      ph
The variables it contains are:  x y ph
\end{verbatim}

Since Metamath will always detect when \texttt{\$d}\index{\texttt{\$d} statement}
statements are needed for a proof, you don't have to worry too much about
forgetting to put one in; it can always be added if you see the error message
above.  If you put in unnecessary \texttt{\$d} statements, the worst that could
happen is that your theorem might not be as general as it could be, and this
may limit its use later on.

On the other hand, when you introduce axioms (\texttt{\$a}\index{\texttt{\$a}
statement} statements), you must be very careful to properly specify the
necessary associated \texttt{\$d} statements since Metamath has no way of knowing
whether your axioms are correct.  For example, Metamath would have no idea
that \texttt{ax-c16}, which we are telling it is an axiom of logic, would lead to
contradictions if we omitted its associated \texttt{\$d} statement.

% This was previously a comment in footnote-sized type, but it can be
% hard to read this much text in a small size.
% As a result, it's been changed to normally-sized text.
\label{nodd}
You may wonder if it is possible to develop standard
mathematics in the Metamath language without the \texttt{\$d}\index{\texttt{\$d}
statement} statement, since it seems like a nuisance that complicates proof
verification. The \texttt{\$d} statement is not needed in certain subsets of
mathematics such as propositional calculus.  However, dummy
variables\index{dummy variable!eliminating} and their associated \texttt{\$d}
statements are impossible to avoid in proofs in standard first-order logic as
well as in the variant used in \texttt{set.mm}.  In fact, there is no upper bound to
the number of dummy variables that might be needed in a proof of a theorem of
first-order logic containing 3 or more variables, as shown by H.\
Andr\'{e}ka\index{Andr{\'{e}}ka, H.} \cite{Nemeti}.  A first-order system that
avoids them entirely is given in \cite{Megill}\index{Megill, Norman}; the
trick there is simply to embed harmlessly the necessary dummy variables into a
theorem being proved so that they aren't ``dummy'' anymore, then interpret the
resulting longer theorem so as to ignore the embedded dummy variables.  If
this interests you, the system in \texttt{set.mm} obtained from \texttt{ax-1}
through \texttt{ax-c14} in \texttt{set.mm}, and deleting \texttt{ax-c16} and \texttt{ax-5},
requires no \texttt{\$d} statements but is logically complete in the sense
described in \cite{Megill}.  This means it can prove any theorem of
first-order logic as long as we add to the theorem an antecedent that embeds
dummy and any other variables that must be distinct.  In a similar fashion,
axioms for set theory can be devised that
do not require distinct variable
provisos\index{Set theory without distinct variable provisos},
as explained at
\url{http://us.metamath.org/mpeuni/mmzfcnd.html}.
Together, these in principle allow all of
mathematics to be developed under Metamath without a \texttt{\$d} statement,
although the length of the resulting theorems will grow as more and
more dummy variables become required in their proofs.

\subsection{The \texttt{\$f}
and \texttt{\$e} Statements}\label{dollaref}
\index{\texttt{\$e} statement}
\index{\texttt{\$f} statement}
\index{floating hypothesis}
\index{essential hypothesis}
\index{variable-type hypothesis}
\index{logical hypothesis}
\index{hypothesis}

Metamath has two kinds of hypo\-theses, the \texttt{\$f}\index{\texttt{\$f}
statement} or {\bf variable-type} hypothesis and the \texttt{\$e} or {\bf logical}
hypo\-the\-sis.\index{\texttt{\$d} statement}\footnote{Strictly speaking, the
\texttt{\$d} statement is also a hypothesis, but it is never directly referenced
in a proof, so we call it a restriction rather than a hypothesis to lessen
confusion.  The checking for violations of \texttt{\$d} restrictions is automatic
and built into Metamath's proof-checking algorithm.} The letters \texttt{f} and
\texttt{e} stand for ``floating''\index{floating hypothesis} (roughly meaning
used only if relevant) and ``essential''\index{essential hypothesis} (meaning
always used) respectively, for reasons that will become apparent
when we discuss frames in
Section~\ref{frames} and scoping in Section~\ref{scoping}. The syntax of these
are as follows:
\begin{center}
  {\em label} \texttt{\$f} {\em typecode} {\em variable} \texttt{\$.}\\
  {\em label} \texttt{\$e} {\em typecode}
      {\em math-symbol}\ \,$\cdots$\ {\em math-symbol} \texttt{\$.}\\
\end{center}
\index{\texttt{\$e} statement}
\index{\texttt{\$f} statement}
A hypothesis must have a {\em label}\index{label}.  The expression in a
\texttt{\$e} hypothesis consists of a typecode (an active constant math symbol)
followed by a sequence
of zero or more math symbols. Each math symbol (including {\em constant}
and {\em variable}) must be a previously declared constant or variable.  (In
addition, each math symbol must be active, which will be covered when we
discuss scoping statements in Section~\ref{scoping}.)  You use a \texttt{\$f}
hypothesis to specify the
nature or {\bf type}\index{variable type}\index{type} of a variable (such as ``let $x$ be an
integer'') and use a \texttt{\$e} hypothesis to express a logical truth (such as
``assume $x$ is prime'') that must be established in order for an assertion
requiring it to also be true.

A variable must have its type specified in a \texttt{\$f} statement before
it may be used in a \texttt{\$e}, \texttt{\$a}, or \texttt{\$p}
statement.  There may be only one (active) \texttt{\$f} statement for a
given variable.  (``Active'' is defined in Section~\ref{scoping}.)

In ordinary mathematics, theorems\index{theorem} are often expressed in the
form ``Assume $P$; then $Q$,'' where $Q$ is a statement that you can derive
if you start with statement $P$.\index{free variable}\footnote{A stronger
version of a theorem like this would be the {\em single} formula $P\rightarrow
Q$ ($P$ implies $Q$) from which the weaker version above follows by the rule
of modus ponens in logic.  We are not discussing this stronger form here.  In
the weaker form, we are saying only that if we can {\em prove} $P$, then we can
{\em prove} $Q$.  In a logician's language, if $x$ is the only free variable
in $P$ and $Q$, the stronger form is equivalent to $\forall x ( P \rightarrow
Q)$ (for all $x$, $P$ implies $Q$), whereas the weaker form is equivalent to
$\forall x P \rightarrow \forall x Q$. The stronger form implies the weaker,
but not vice-versa.  To be precise, the weaker form of the theorem is more
properly called an ``inference'' rather than a theorem.}\index{inference}
In the
Metamath\index{Metamath} language, you would express mathematical statement
$P$ as a hypothesis (a \texttt{\$e} Metamath language statement in this case) and
statement $Q$ as a provable assertion (a \texttt{\$p}\index{\texttt{\$p} statement}
statement).

Some examples of hypotheses you might encounter in logic and set theory are
\begin{center}
  \texttt{stmt1 \$f wff P \$.}\\
  \texttt{stmt2 \$f set x \$.}\\
  \texttt{stmt3 \$e |- ( P -> Q ) \$.}
\end{center}
\index{\texttt{\$e} statement}
\index{\texttt{\$f} statement}
Informally, these would be read, ``Let $P$ be a well-formed-formula,'' ``Let
$x$ be an (individual) variable,'' and ``Assume we have proved $P \rightarrow
Q$.''  The turnstile symbol \,$\vdash$\index{turnstile ({$\,\vdash$})} is
commonly used in logic texts to mean ``a proof exists for.''

To summarize:
\begin{itemize}
\item A \texttt{\$f} hypothesis tells Metamath the type or kind of its variable.
It is analogous to a variable declaration in a computer language that
tells the compiler that a variable is an integer or a floating-point
number.
\item The \texttt{\$e} hypothesis corresponds to what you would usually call a
``hypothesis'' in ordinary mathematics.
\end{itemize}

Before an assertion\index{assertion} (\texttt{\$a} or \texttt{\$p} statement) can be
referenced in a proof, all of its associated \texttt{\$f} and \texttt{\$e} hypotheses
(i.e.\ those \texttt{\$e} hypotheses that are active) must be satisfied (i.e.
established by the proof).  The meaning of ``associated'' (which we will call
{\bf mandatory} in Section~\ref{frames}) will become clear when we discuss
scoping later.

Note that after any \texttt{\$f}, \texttt{\$e},
\texttt{\$a}, or \texttt{\$p} token there is a required
\textit{typecode}\index{typecode}.
The typecode is a constant used to enforce types of expressions.
This will become clearer once we learn more about
assertions (\texttt{\$a} and \texttt{\$p} statements).
An example may also clarify their purpose.
In the
\texttt{set.mm}\index{set theory database (\texttt{set.mm})}%
\index{Metamath Proof Explorer}
database,
the following typecodes are used:

\begin{itemize}
\item \texttt{wff} :
  Well-formed formula (wff) symbol
  (read: ``the following symbol sequence is a wff'').
% The *textual* typecode for turnstile is "|-", but when read it's a little
% confusing, so I intentionally display the mathematical symbol here instead
% (I think it's clearer in this context).
\item \texttt{$\vdash$} :
  Turnstile (read: ``the following symbol sequence is provable'' or
  ``a proof exists for'').
\item \texttt{setvar} :
  Individual set variable type (read: ``the following is an
  individual set variable'').
  Note that this is \textit{not} the type of an arbitrary set expression,
  instead, it is used to ensure that there is only a single symbol used
  after quantifiers like for-all ($\forall$) and there-exists ($\exists$).
\item \texttt{class} :
  An expression that is a syntactically valid class expression.
  All valid set expressions are also valid class expression, so expressions
  of sets normally have the \texttt{class} typecode.
  Use the \texttt{class} typecode,
  \textit{not} the \texttt{setvar} typecode,
  for the type of set expressions unless you are specifically identifying
  a single set variable.
\end{itemize}

\subsection{Assertions (\texttt{\$a} and \texttt{\$p} Statements)}
\index{\texttt{\$a} statement}
\index{\texttt{\$p} statement}\index{assertion}\index{axiomatic assertion}
\index{provable assertion}

There are two types of assertions, \texttt{\$a}\index{\texttt{\$a} statement}
statements ({\bf axiomatic assertions}) and \texttt{\$p} statements ({\bf
provable assertions}).  Their syntax is as follows:
\begin{center}
  {\em label} \texttt{\$a} {\em typecode} {\em math-symbol} \ldots
         {\em math-symbol} \texttt{\$.}\\
  {\em label} \texttt{\$p} {\em typecode} {\em math-symbol} \ldots
        {\em math-symbol} \texttt{\$=} {\em proof} \texttt{\$.}
\end{center}
\index{\texttt{\$a} statement}
\index{\texttt{\$p} statement}
\index{\texttt{\$=} keyword}
An assertion always requires a {\em label}\index{label}. The expression in an
assertion consists of a typecode (an active constant)
followed by a sequence of zero
or more math symbols.  Each math symbol, including any {\em constant}, must be a
previously declared constant or variable.  (In addition, each math symbol
must be active, which will be covered when we discuss scoping statements in
Section~\ref{scoping}.)

A \texttt{\$a} statement is usually a definition of syntax (for example, if $P$
and $Q$ are wffs then so is $(P\to Q)$), an axiom\index{axiom} of ordinary
mathematics (for example, $x=x$), or a definition\index{definition} of
ordinary mathematics (for example, $x\ne y$ means $\lnot x=y$). A \texttt{\$p}
statement is a claim that a certain combination of math symbols follows from
previous assertions and is accompanied by a proof that demonstrates it.

Assertions can also be referenced in (later) proofs in order to derive new
assertions from them. The label of an assertion is used to refer to it in a
proof. Section~\ref{proof} will describe the proof in detail.

Assertions also provide the primary means for communicating the mathematical
results in the database to people.  Proofs (when conveniently displayed)
communicate to people how the results were arrived at.

\subsubsection{The \texttt{\$a} Statement}
\index{\texttt{\$a} statement}

Axiomatic assertions (\texttt{\$a} statements) represent the starting points from
which other assertions (\texttt{\$p}\index{\texttt{\$p} statement} statements) are
derived.  Their most obvious use is for specifying ordinary mathematical
axioms\index{axiom}, but they are also used for two other purposes.

First, Metamath\index{Metamath} needs to know the syntax of symbol
sequences that constitute valid mathematical statements.  A Metamath
proof must be broken down into much more detail than ordinary
mathematical proofs that you may be used to thinking of (even the
``complete'' proofs of formal logic\index{formal logic}).  This is one
of the things that makes Metamath a general-purpose language,
independent of any system of logic or even syntax.  If you want to use a
substitution instance of an assertion as a step in a proof, you must
first prove that the substitution is syntactically correct (or if you
prefer, you must ``construct'' it), showing for example that the
expression you are substituting for a wff metavariable is a valid wff.
The \texttt{\$a}\index{\texttt{\$a} statement} statement is used to
specify those combinations of symbols that are considered syntactically
valid, such as the legal forms of wffs.

Second, \texttt{\$a} statements are used to specify what are ordinarily thought of
as definitions, i.e.\ new combinations of symbols that abbreviate other
combinations of symbols.  Metamath makes no distinction\index{axiom vs.\
definition} between axioms\index{axiom} and definitions\index{definition}.
Indeed, it has been argued that such distinction should not be made even in
ordinary mathematics; see Section~\ref{definitions}, which discusses the
philosophy of definitions.  Section~\ref{hierarchy} discusses some
technical requirements for definitions.  In \texttt{set.mm} we adopt the
convention of prefixing axiom labels with \texttt{ax-} and definition labels with
\texttt{df-}\index{label}.

The results that can be derived with the Metamath language are only as good as
the \texttt{\$a}\index{\texttt{\$a} statement} statements used as their starting
point.  We cannot stress this too strongly.  For example, Metamath will
not prevent you from specifying $x\neq x$ as an axiom of logic.  It is
essential that you scrutinize all \texttt{\$a} statements with great care.
Because they are a source of potential pitfalls, it is best not to add new
ones (usually new definitions) casually; rather you should carefully evaluate
each one's necessity and advantages.

Once you have in place all of the basic axioms\index{axiom} and
rules\index{rule} of a mathematical theory, the only \texttt{\$a} statements that
you will be adding will be what are ordinarily called definitions.  In
principle, definitions should be in some sense eliminable from the language of
a theory according to some convention (usually involving logical equivalence
or equality).  The most common convention is that any formula that was
syntactically valid but not provable before the definition was introduced will
not become provable after the definition is introduced.  In an ideal world,
definitions should not be present at all if one is to have absolute confidence
in a mathematical result.  However, they are necessary to make
mathematics practical, for otherwise the resulting formulas would be
extremely long and incomprehensible.  Since the nature of definitions (in the
most general sense) does not permit them to automatically be verified as
``proper,''\index{proper definition}\index{definition!proper} the judgment of
the mathematician is required to ensure it.  (In \texttt{set.mm} effort was made
to make almost all definitions directly eliminable and thus minimize the need
for such judgment.)

If you are not a mathematician, it may be best not to add or change any
\texttt{\$a}\index{\texttt{\$a} statement} statements but instead use
the mathematical language already provided in standard databases.  This
way Metamath will not allow you to make a mistake (i.e.\ prove a false
result).


\subsection{Frames}\label{frames}

We now introduce the concept of a collection of related Metamath statements
called a frame.  Every assertion (\texttt{\$a} or \texttt{\$p} statement) in the database has
an associated frame.

A {\bf frame}\index{frame} is a sequence of \texttt{\$d}, \texttt{\$f},
and \texttt{\$e} statements (zero or more of each) followed by one
\texttt{\$a} or \texttt{\$p} statement, subject to certain conditions we
will describe.  For simplicity we will assume that all math symbol
tokens used are declared at the beginning of the database with
\texttt{\$c} and \texttt{\$v} statements (which are not properly part of
a frame).  Also for simplicity we will assume there are only simple
\texttt{\$d} statements (those with only two variables) and imagine any
compound \texttt{\$d} statements (those with more than two variables) as
broken up into simple ones.

A frame groups together those hypotheses (and \texttt{\$d} statements) relevant
to an assertion (\texttt{\$a} or \texttt{\$p} statement).  The statements in a frame
may or may not be physically adjacent in a database; we will cover
this in our discussion of scoping statements
in Section~\ref{scoping}.

A frame has the following properties:
\begin{enumerate}
 \item The set of variables contained in its \texttt{\$f} statements must
be identical to the set of variables contained in its \texttt{\$e},
\texttt{\$a}, and/or \texttt{\$p} statements.  In other words, each
variable in a \texttt{\$e}, \texttt{\$a}, or \texttt{\$p} statement must
have an associated ``variable type'' defined for it in a \texttt{\$f}
statement.
  \item No two \texttt{\$f} statements may contain the same variable.
  \item Any \texttt{\$f} statement
must occur before a \texttt{\$e} statement in which its variable occurs.
\end{enumerate}

The first property determines the set of variables occurring in a frame.
These are the {\bf mandatory
variables}\index{mandatory variable} of the frame.  The second property
tells us there must be only one type specified for a variable.
The last property is not a theoretical requirement but it
makes parsing of the database easier.

For our examples, we assume our database has the following declarations:

\begin{verbatim}
$v P Q R $.
$c -> ( ) |- wff $.
\end{verbatim}

The following sequence of statements, describing the modus ponens inference
rule, is an example of a frame:

\begin{verbatim}
wp  $f wff P $.
wq  $f wff Q $.
maj $e |- ( P -> Q ) $.
min $e |- P $.
mp  $a |- Q $.
\end{verbatim}

The following sequence of statements is not a frame because \texttt{R} does not
occur in the \texttt{\$e}'s or the \texttt{\$a}:

\begin{verbatim}
wp  $f wff P $.
wq  $f wff Q $.
wr  $f wff R $.
maj $e |- ( P -> Q ) $.
min $e |- P $.
mp  $a |- Q $.
\end{verbatim}

The following sequence of statements is not a frame because \texttt{Q} does not
occur in a \texttt{\$f}:

\begin{verbatim}
wp  $f wff P $.
maj $e |- ( P -> Q ) $.
min $e |- P $.
mp  $a |- Q $.
\end{verbatim}

The following sequence of statements is not a frame because the \texttt{\$a} statement is
not the last one:

\begin{verbatim}
wp  $f wff P $.
wq  $f wff Q $.
maj $e |- ( P -> Q ) $.
mp  $a |- Q $.
min $e |- P $.
\end{verbatim}

Associated with a frame is a sequence of {\bf mandatory
hypotheses}\index{mandatory hypothesis}.  This is simply the set of all
\texttt{\$f} and \texttt{\$e} statements in the frame, in the order they
appear.  A frame can be referenced in a later proof using the label of
the \texttt{\$a} or \texttt{\$p} assertion statement, and the proof
makes an assignment to each mandatory hypothesis in the order in which
it appears.  This means the order of the hypotheses, once chosen, must
not be changed so as not to affect later proofs referencing the frame's
assertion statement.  (The Metamath proof verifier will, of course, flag
an error if a proof becomes incorrect by doing this.)  Since proofs make
use of ``Reverse Polish notation,'' described in Section~\ref{proof}, we
call this order the {\bf RPN order}\index{RPN order} of the hypotheses.

Note that \texttt{\$d} statements are not part of the set of mandatory
hypotheses, and their order doesn't matter (as long as they satisfy the
fourth property for a frame described above).  The \texttt{\$d}
statements specify restrictions on variables that must be satisfied (and
are checked by the proof verifier) when expressions are substituted for
them in a proof, and the \texttt{\$d} statements themselves are never
referenced directly in a proof.

A frame with a \texttt{\$p} (provable) statement requires a proof as part of the
\texttt{\$p} statement.  Sometimes in a proof we want to make use of temporary or
dummy variables\index{dummy variable} that do not occur in the \texttt{\$p}
statement or its mandatory hypotheses.  To accommodate this we define an {\bf
extended frame}\index{extended frame} as a frame together with zero or more
\texttt{\$d} and \texttt{\$f} statements that reference variables not among the
mandatory variables of the frame.  Any new variables referenced are called the
{\bf optional variables}\index{optional variable} of the extended frame. If a
\texttt{\$f} statement references an optional variable it is called an {\bf
optional hypothesis}\index{optional hypothesis}, and if one or both of the
variables in a \texttt{\$d} statement are optional variables it is called an {\bf
optional disjoint-variable restriction}\index{optional disjoint-variable
restriction}.  Properties 2, 3, and 4 for a frame also apply to an extended
frame.

The concept of optional variables is not meaningful for frames with \texttt{\$a}
statements, since those statements have no proofs that might make use of them.
There is no restriction on including optional hypotheses in the extended frame
for a \texttt{\$a} statement, but they serve no purpose.

The following set of statements is an example of an extended frame, which
contains an optional variable \texttt{R} and an optional hypothesis \texttt{wr}.  In
this example, we suppose the rule of modus ponens is not an axiom but is
derived as a theorem from earlier statements (we omit its presumed proof).
Variable \texttt{R} may be used in its proof if desired (although this would
probably have no advantage in propositional calculus).  Note that the sequence
of mandatory hypotheses in RPN order is still \texttt{wp}, \texttt{wq}, \texttt{maj},
\texttt{min} (i.e.\ \texttt{wr} is omitted), and this sequence is still assumed
whenever the assertion \texttt{mp} is referenced in a subsequent proof.

\begin{verbatim}
wp  $f wff P $.
wq  $f wff Q $.
wr  $f wff R $.
maj $e |- ( P -> Q ) $.
min $e |- P $.
mp  $p |- Q $= ... $.
\end{verbatim}

Every frame is an extended frame, but not every extended frame is a frame, as
this example shows.  The underlying frame for an extended frame is
obtained by simply removing all statements containing optional variables.
Any proof referencing an assertion will ignore any extensions to its
frame, which means we may add or delete optional hypotheses at will without
affecting subsequent proofs.

The conceptually simplest way of organizing a Metamath database is as a
sequence of extended frames.  The scoping statements
\texttt{\$\char`\{}\index{\texttt{\$\char`\{} and \texttt{\$\char`\}}
keywords} and \texttt{\$\char`\}} can be used to delimit the start and
end of an extended frame, leading to the following possible structure for a
database.  \label{framelist}

\vskip 2ex
\setbox\startprefix=\hbox{\tt \ \ \ \ \ \ \ \ }
\setbox\contprefix=\hbox{}
\startm
\m{\mbox{(\texttt{\$v} {\em and} \texttt{\$c}\,{\em statements})}}
\endm
\startm
\m{\mbox{\texttt{\$\char`\{}}}
\endm
\startm
\m{\mbox{\texttt{\ \ } {\em extended frame}}}
\endm
\startm
\m{\mbox{\texttt{\$\char`\}}}}
\endm
\startm
\m{\mbox{\texttt{\$\char`\{}}}
\endm
\startm
\m{\mbox{\texttt{\ \ } {\em extended frame}}}
\endm
\startm
\m{\mbox{\texttt{\$\char`\}}}}
\endm
\startm
\m{\mbox{\texttt{\ \ \ \ \ \ \ \ \ }}\vdots}
\endm
\vskip 2ex

In practice, this structure is inconvenient because we have to repeat
any \texttt{\$f}, \texttt{\$e}, and \texttt{\$d} statements over and
over again rather than stating them once for use by several assertions.
The scoping statements, which we will discuss next, allow this to be
done.  In principle, any Metamath database can be converted to the above
format, and the above format is the most convenient to use when studying
a Metamath database as a formal system%
%% Uncomment this when uncommenting section {formalspec} below
   (Appendix \ref{formalspec})%
.
In fact, Metamath internally converts the database to the above format.
The command \texttt{show statement} in the Metamath program will show
you the contents of the frame for any \texttt{\$a} or \texttt{\$p}
statement, as well as its extension in the case of a \texttt{\$p}
statement.

%c%(provided that all ``local'' variables and constants with limited scope have
%c%unique names),

During our discussion of scoping statements, it may be helpful to
think in terms of the equivalent sequence of frames that will result when
the database is parsed.  Scoping (other than the limited
use above to delimit frames) is not a theoretical requirement for
Metamath but makes it more convenient.


\subsection{Scoping Statements (\texttt{\$\{} and \texttt{\$\}})}\label{scoping}
\index{\texttt{\$\char`\{} and \texttt{\$\char`\}} keywords}\index{scoping statement}

%c%Some Metamath statements may be needed only temporarily to
%c%serve a specific purpose, and after we're done with them we would like to
%c%disregard or ignore them.  For example, when we're finished using a variable,
%c%we might want to
%c%we might want to free up the token\index{token} used to name it so that the
%c%token can be used for other purposes later on, such as a different kind of
%c%variable or even a constant.  In the terminology of computer programming, we
%c%might want to let some symbol declarations be ``local'' rather than ``global.''
%c%\index{local symbol}\index{global symbol}

The {\bf scoping} statements, \texttt{\$\char`\{} ({\bf start of block}) and \texttt{\$\char`\}}
({\bf end of block})\index{block}, provide a means for controlling the portion
of a database over which certain statement types are recognized.  The
syntax of a scoping statement is very simple; it just consists of the
statement's keyword:
\begin{center}
\texttt{\$\char`\{}\\
\texttt{\$\char`\}}
\end{center}
\index{\texttt{\$\char`\{} and \texttt{\$\char`\}} keywords}

For example, consider the following database where we have stripped out
all tokens except the scoping statement keywords.  For the purpose of the
discussion, we have added subscripts to the scoping statements; these subscripts
do not appear in the actual database.
\[
 \mbox{\tt \ \$\char`\{}_1
 \mbox{\tt \ \$\char`\{}_2
 \mbox{\tt \ \$\char`\}}_2
 \mbox{\tt \ \$\char`\{}_3
 \mbox{\tt \ \$\char`\{}_4
 \mbox{\tt \ \$\char`\}}_4
 \mbox{\tt \ \$\char`\}}_3
 \mbox{\tt \ \$\char`\}}_1
\]
Each \texttt{\$\char`\{} statement in this example is said to be {\bf
matched} with the \texttt{\$\char`\}} statement that has the same
subscript.  Each pair of matched scoping statements defines a region of
the database called a {\bf block}.\index{block} Blocks can be {\bf
nested}\index{nested block} inside other blocks; in the example, the
block defined by $\mbox{\tt \$\char`\{}_4$ and $\mbox{\tt \$\char`\}}_4$
is nested inside the block defined by $\mbox{\tt \$\char`\{}_3$ and
$\mbox{\tt \$\char`\}}_3$ as well as inside the block defined by
$\mbox{\tt \$\char`\{}_1$ and $\mbox{\tt \$\char`\}}_1$.  In general, a
block may be empty, it may contain only non-scoping
statements,\footnote{Those statements other than \texttt{\$\char`\{} and
\texttt{\$\char`\}}.}\index{non-scoping statement} or it may contain any
mixture of other blocks and non-scoping statements.  (This is called a
``recursive'' definition\index{recursive definition} of a block.)

Associated with each block is a number called its {\bf nesting
level}\index{nesting level} that indicates how deeply the block is nested.
The nesting levels of the blocks in our example are as follows:
\[
  \underbrace{
    \mbox{\tt \ }
    \underbrace{
     \mbox{\tt \$\char`\{\ }
     \underbrace{
       \mbox{\tt \$\char`\{\ }
       \mbox{\tt \$\char`\}}
     }_{2}
     \mbox{\tt \ }
     \underbrace{
       \mbox{\tt \$\char`\{\ }
       \underbrace{
         \mbox{\tt \$\char`\{\ }
         \mbox{\tt \$\char`\}}
       }_{3}
       \mbox{\tt \ \$\char`\}}
     }_{2}
     \mbox{\tt \ \$\char`\}}
   }_{1}
   \mbox{\tt \ }
 }_{0}
\]
\index{\texttt{\$\char`\{} and \texttt{\$\char`\}} keywords}
The entire database is considered to be one big block (the {\bf outermost}
block) with a nesting level of 0.  The outermost block is {\em not} bracketed
by scoping statements.\footnote{The language was designed this way so that
several source files can be joined together more easily.}\index{outermost
block}

All non-scoping Metamath statements become recognized or {\bf
active}\index{active statement} at the place where they appear.\footnote{To
keep things slightly simpler, we do not bother to define the concept of
``active'' for the scoping statements.}  Certain of these statement types
become inactive at the end of the block in which they appear; these statement
types are:
\begin{center}
  \texttt{\$c}, \texttt{\$v}, \texttt{\$d}, \texttt{\$e}, and \texttt{\$f}.
%  \texttt{\$v}, \texttt{\$f}, \texttt{\$e}, and \texttt{\$d}.
\end{center}
\index{\texttt{\$c} statement}
\index{\texttt{\$d} statement}
\index{\texttt{\$e} statement}
\index{\texttt{\$f} statement}
\index{\texttt{\$v} statement}
The other statement types remain active forever (i.e.\ through the end of the
database); they are:
\begin{center}
  \texttt{\$a} and \texttt{\$p}.
%  \texttt{\$c}, \texttt{\$a}, and \texttt{\$p}.
\end{center}
\index{\texttt{\$a} statement}
\index{\texttt{\$p} statement}
Any statement (of these 7 types) located in the outermost
block\index{outermost block} will remain active through the end of the
database and thus are effectively ``global'' statements.\index{global
statement}

All \texttt{\$c} statements must be placed in the outermost block.  Since they are
therefore always global, they could be considered as belonging to both of the
above categories.

The {\bf scope}\index{scope} of a statement is the set of statements that
recognize it as active.

%c%The concept of ``active'' is also defined for math symbols\index{math
%c%symbol}.  Math symbols (constants\index{constant} and
%c%variables\index{variable}) become {\bf active}\index{active
%c%math symbol} in the \texttt{\$c}\index{\texttt{\$c}
%c%statement} and \texttt{\$v}\index{\texttt{\$v} statement} statements that
%c%declare them.  They become inactive when their declaration statements become
%c%inactive.

The concept of ``active'' is also defined for math symbols\index{math
symbol}.  Math symbols (constants\index{constant} and
variables\index{variable}) become {\bf active}\index{active math symbol}
in the \texttt{\$c}\index{\texttt{\$c} statement} and
\texttt{\$v}\index{\texttt{\$v} statement} statements that declare them.
A variable becomes inactive when its declaration statement becomes
inactive.  Because all \texttt{\$c} statements must be in the outermost
block, a constant will never become inactive after it is declared.

\subsubsection{Redeclaration of Math Symbols}
\index{redeclaration of symbols}\label{redeclaration}

%c%A math symbol may not be declared a second time while it is active, but it may
%c%be declared again after it becomes inactive.

A variable may not be declared a second time while it is active, but it may be
declared again after it becomes inactive.  This provides a convenient way to
introduce ``local'' variables,\index{local variable} i.e.\ temporary variables
for use in the frame of an assertion or in a proof without keeping them around
forever.  A previously declared variable may not be redeclared as a constant.

A constant may not be redeclared.  And, as mentioned above, constants must be
declared in the outermost block.

The reason variables may have limited scope but not constants is that an
assertion (\texttt{\$a} or \texttt{\$p} statement) remains available for use in
proofs through the end of the database.  Variables in an assertion's frame may
be substituted with whatever is needed in a proof step that references the
assertion, whereas constants remain fixed and may not be substituted with
anything.  The particular token used for a variable in an assertion's frame is
irrelevant when the assertion is referenced in a proof, and it doesn't matter
if that token is not available outside of the referenced assertion's frame.
Constants, however, must be globally fixed.

In the present version of the Metamath language, there is no theoretical
benefit for the feature allowing variables to be active for limited scopes
rather than global. It is just a convenience that allows them, for example, to
be locally grouped together with their corresponding \texttt{\$f} variable-type
declarations.\footnote{A future version\label{redeclarationf} of the Metamath
language may extend the language to include true local variables as well as
local constants.  I have not decided upon the best way to do this or even if
it should be done at all; the main issue is whether the benefits offset
greater confusion in learning the language.  In any case, the rules for
the present language were chosen to let existing databases be compatible with
any future language extensions.}

%c%If you declare a math symbol more than once, internally Metamath considers it a
%c%new distinct symbol, even though it has the same name.  If you are unaware of
%c%this, you may find that what you think are correct proofs are incorrectly
%c%rejected as invalid, because Metamath may tell you that a constant you
%c%previously declared does not match a newly declared math symbol with the same
%c%name.  For details on this subtle point, see the Comment on
%c%p.~\pageref{spec4comment}.  This is done purposely to allow temporary
%c%constants to be introduced while developing a subtheory, then allow their math
%c%symbol tokens to be reused later on; in general they will not refer to the
%c%same thing.  In practice, you would not ordinarily reuse the names of
%c%constants because it would tend to be confusing to the reader.  The reuse of
%c%names of variables, on the other hand, is something that is often useful to do
%c%(for example it is done frequently in \texttt{set.mm}).  Since variables in an
%c%assertion referenced in a proof can be substituted as needed to achieve a
%c%symbol match, this is not an issue.

% (This section covers a somewhat advanced topic you may want to skip
% at first reading.)
%
% Under certain circumstances, math symbol\index{math symbol}
% tokens\index{token} may be redeclared (i.e.\ the token
% may appear in more than
% one \texttt{\$c}\index{\texttt{\$c} statement} or \texttt{\$v}\index{\texttt{\$v}
% statement} statement).  You might want to do this say, to make temporary use
% of a variable name without having to worry about its affect elsewhere,
% somewhat analogous to declaring a local variable in a standard computer
% language.  Understanding what goes on when math symbol tokens are redeclared
% is a little tricky to understand at first, since it requires that we
% distinguish the token itself from the math symbol that it names.  It will help
% if we first take a peek at the internal workings of the
% Metamath\index{Metamath} program.
%
% Metamath reserves a memory location for each occurrence of a
% token\index{token} in a declaration statement (\texttt{\$c}\index{\texttt{\$c}
% statement} or \texttt{\$v}\index{\texttt{\$v} statement}).  If a given token appears
% in more than one declaration statement, it will refer to more than one memory
% locations.  A math symbol\index{math symbol} may be thought of as being one of
% these memory locations rather than as the token itself.  Only one of the
% memory locations associated with a given token may be active at any one time.
% The math symbol (memory location) that gets looked up when the token appears
% in a non-declaration statement is the one that happens to be active at that
% time.
%
% We now look at the rules for the redeclaration\index{redeclaration of symbols}
% of math symbol tokens.
% \begin{itemize}
% \item A math symbol token may not be declared twice in the
% same block.\footnote{While there is no theoretical reason for disallowing
% this, it was decided in the design of Metamath that allowing it would offer no
% advantage and might cause confusion.}
% \item An inactive math symbol may always be
% redeclared.
% \item  An active math symbol may be redeclared in a different (i.e.\
% inner) block\index{block} from the one it became active in.
% \end{itemize}
%
% When a math symbol token is redeclared, it conceptually refers to a different
% math symbol, just as it would be if it were called a different name.  In
% addition, the original math symbol that it referred to, if it was active,
% temporarily becomes inactive.  At the end of the block in which the
% redeclaration occurred, the new math symbol\index{math symbol} becomes
% inactive and the original symbol becomes active again.  This concept is
% illustrated in the following example, where the symbol \texttt{e} is
% ordinarily a constant (say Euler's constant, 2.71828...) but
% temporarily we want to use it as a ``local'' variable, say as a coefficient
% in the equation $a x^4 + b x^3 + c x^2 + d x + e$:
% \[
%   \mbox{\tt \$\char`\{\ \$c e \$.}
%   \underbrace{
%     \ \ldots\ %
%     \mbox{\tt \$\char`\{}\ \ldots\ %
%   }_{\mbox{\rm region A}}
%   \mbox{\tt \$v e \$.}
%   \underbrace{
%     \mbox{\ \ \ \ldots\ \ \ }
%   }_{\mbox{\rm region B}}
%   \mbox{\tt \$\char`\}}
%   \underbrace{
%     \mbox{\ \ \ \ldots\ \ \ }
%   }_{\mbox{\rm region C}}
%   \mbox{\tt \$\char`\}}
% \]
% \index{\texttt{\$\char`\{} and \texttt{\$\char`\}} keywords}
% In region A, the token \texttt{e} refers to a constant.  It is redeclared as a
% variable in region B, and any reference to it in this region will refer to this
% variable.  In region C, the redeclaration becomes inactive, and the original
% declaration becomes active again.  In region C, the token \texttt{x} refers to the
% original constant.
%
% As a practical matter, overuse of math symbol\index{math symbol}
% redeclarations\index{redeclaration of symbols} can be confusing (even though
% it is well-defined) and is best avoided when possible.  Here are some good
% general guidelines you can follow.  Usually, you should declare all
% constants\index{constant} in the outermost block\index{outermost block},
% especially if they are general-purpose (such as the token \verb$A.$, meaning
% $\forall$ or ``for all'').  This will make them ``globally'' active (although
% as in the example above local redeclarations will temporarily make them
% inactive.)  Most or all variables\index{variable}, on the other hand, could be
% declared in inner blocks, so that the token for them can be used later for a
% different type of variable or a constant.  (The names of the variables you
% choose are not used when you refer to an assertion\index{assertion} in a
% proof, whereas constants must match exactly.  A locally declared constant will
% not match a globally declared constant in a proof, even if they use the same
% token, because Metamath internally considers them to be different math
% symbols.)  To avoid confusion, you should generally avoid redeclaring active
% variables.  If you must redeclare them, do so at the beginning of a block.
% The temporary declaration of constants in inner blocks might be occasionally
% appropriate when you make use of a temporary definition to prove lemmas
% leading to a main result that does not make direct use of the definition.
% This way, you will not clutter up your database with a large number of
% seldom-used global constant symbols.  You might want to note that while
% inactive constants may not appear directly in an assertion (a \texttt{\$a}\index{\texttt{\$a}
% statement} or \texttt{\$p}\index{\texttt{\$p} statement}
% statement), they may be indirectly used in the proof of a \texttt{\$p} statement
% so long as they do not appear in the final math symbol sequence constructed by
% the proof.  In the end, you will have to use your best judgment, taking into
% account standard mathematical usage of the symbols as well as consideration
% for the reader of your work.
%
% \subsubsection{Reuse of Labels}\index{reuse of labels}\index{label}
%
% The \texttt{\$e}\index{\texttt{\$e} statement}, \texttt{\$f}\index{\texttt{\$f}
% statement}, \texttt{\$a}\index{\texttt{\$a} statement}, and
% \texttt{\$p}\index{\texttt{\$p}
% statement} statement types require labels, which allow them to be
% referenced later inside proofs.  A label is considered {\bf
% active}\index{active label} when the statement it is associated with is
% active.  The token\index{token} for a label may be reused
% (redeclared)\index{redeclaration of labels} provided that it is not being used
% for a currently active label.  (Unlike the tokens for math symbols, active
% label tokens may not be redeclared in an inner scope.)  Note that the labels
% of \texttt{\$a} and \texttt{\$p} statements can never be reused after these
% statements appear, because these statements remain active through the end of
% the database.
%
% You might find the reuse of labels a convenient way to have standard names for
% temporary hypotheses, such as \texttt{h1}, \texttt{h2}, etc.  This way you don't have
% to invent unique names for each of them, and in some cases it may be less
% confusing to the reader (although in other cases it might be more confusing, if
% the hypothesis is located far away from the assertion that uses
% it).\footnote{The current implementation requires that all labels, even
% inactive ones, be unique.}

\subsubsection{Frames Revisited}\index{frames and scoping statements}

Now that we have covered scoping, we will look at how an arbitrary
Metamath database can be converted to the simple sequence of extended
frames described on p.~\pageref{framelist}.  This is also how Metamath
stores the database internally when it reads in the database
source.\label{frameconvert} The method is simple.  First, we collect all
constant and variable (\texttt{\$c} and \texttt{\$v}) declarations in
the database, ignoring duplicate declarations of the same variable in
different scopes.  We then put our collected \texttt{\$c} and
\texttt{\$v} declarations at the beginning of the database, so that
their scope is the entire database.  Next, for each assertion in the
database, we determine its frame and extended frame.  The extended frame
is simply the \texttt{\$f}, \texttt{\$e}, and \texttt{\$d} statements
that are active.  The frame is the extended frame with all optional
hypotheses removed.

An equivalent way of saying this is that the extended frame of an assertion
is the collection of all \texttt{\$f}, \texttt{\$e}, and \texttt{\$d} statements
whose scope includes the assertion.
The \texttt{\$f} and \texttt{\$e} statements
occur in the order they appear
(order is irrelevant for \texttt{\$d} statements).

%c%, renaming any
%c%redeclared variables as needed so that all of them have unique names.  (The
%c%exact renaming convention is unimportant.  You might imagine renaming
%c%different declarations of math symbol \texttt{a} as \texttt{a\$1}, \texttt{a\$2}, etc.\
%c%which would prevent any conflicts since \texttt{\$} is not a legal character in a
%c%math symbol token.)

\section{The Anatomy of a Proof} \label{proof}
\index{proof!Metamath, description of}

Each provable assertion (\texttt{\$p}\index{\texttt{\$p} statement} statement) in a
database must include a {\bf proof}\index{proof}.  The proof is located
between the \texttt{\$=}\index{\texttt{\$=} keyword} and \texttt{\$.}\ keywords in the
\texttt{\$p} statement.

In the basic Metamath language\index{basic language}, a proof is a
sequence of statement labels.  This label sequence\index{label sequence}
serves as a set of instructions that the Metamath program uses to
construct a series of math symbol sequences.  The construction must
ultimately result in the math symbol sequence contained between the
\texttt{\$p}\index{\texttt{\$p} statement} and
\texttt{\$=}\index{\texttt{\$=} keyword} keywords of the \texttt{\$p}
statement.  Otherwise, the Metamath program will consider the proof
incorrect, and it will notify you with an appropriate error message when
you ask it to verify the proof.\footnote{To make the loading faster, the
Metamath program does not automatically verify proofs when you
\texttt{read} in a database unless you use the \texttt{/verify}
qualifier.  After a database has been read in, you may use the
\texttt{verify proof *} command to verify proofs.}\index{\texttt{verify
proof} command} Each label in a proof is said to {\bf
reference}\index{label reference} its corresponding statement.

Associated with any assertion\index{assertion} (\texttt{\$p} or
\texttt{\$a}\index{\texttt{\$a} statement} statement) is a set of
hypotheses (\texttt{\$f}\index{\texttt{\$f} statement} or
\texttt{\$e}\index{\texttt{\$e} statement} statements) that are active
with respect to that assertion.  Some are mandatory and the others are
optional.  You should review these concepts if necessary.

Each label\index{label} in a proof must be either the label of a
previous assertion (\texttt{\$a}\index{\texttt{\$a} statement} or
\texttt{\$p}\index{\texttt{\$p} statement} statement) or the label of an
active hypothesis (\texttt{\$e} or \texttt{\$f}\index{\texttt{\$f}
statement} statement) of the \texttt{\$p} statement containing the
proof.  Hypothesis labels may reference both the
mandatory\index{mandatory hypothesis} and the optional hypotheses of the
\texttt{\$p} statement.

The label sequence in a proof specifies a construction in {\bf reverse Polish
notation}\index{reverse Polish notation (RPN)} (RPN).  You may be familiar
with RPN if you have used older
Hewlett--Packard or similar hand-held calculators.
In the calculator analogy, a hypothesis label\index{hypothesis label} is like
a number and an assertion label\index{assertion label} is like an operation
(more precisely, an $n$-ary operation when the
assertion has $n$ \texttt{\$e}-hypotheses).
On an RPN calculator, an operation takes one or more previous numbers in an
input sequence, performs a calculation on them, and replaces those numbers and
itself with the result of the calculation.  For example, the input sequence
$2,3,+$ on an RPN calculator results in $5$, and the input sequence
$2,3,5,{\times},+$ results in $2,15,+$ which results in $17$.

Understanding how RPN is processed involves the concept of a {\bf
stack}\index{stack}\index{RPN stack}, which can be thought of as a set of
temporary memory locations that hold intermediate results.  When Metamath
encounters a hypothesis label it places or {\bf pushes}\index{push} the math
symbol sequence of the hypothesis onto the stack.  When Metamath encounters an
assertion label, it associates the most recent stack entries with the {\em
mandatory} hypotheses\index{mandatory hypothesis} of the assertion, in the
order where the most recent stack entry is associated with the last mandatory
hypothesis of the assertion.  It then determines what
substitutions\index{substitution!variable}\index{variable substitution} have
to be made into the variables of the assertion's mandatory hypotheses to make
them identical to the associated stack entries.  It then makes those same
substitutions into the assertion itself.  Finally, Metamath removes or {\bf
pops}\index{pop} the matched hypotheses from the stack and pushes the
substituted assertion onto the stack.

For the purpose of matching the mandatory hypothesis to the most recent stack
entries, whether a hypothesis is a \texttt{\$e} or \texttt{\$f} statement is
irrelevant.  The only important thing is that a set of
substitutions\footnote{In the Metamath spec (Section~\ref{spec}), we use the
singular term ``substitution'' to refer to the set of substitutions we talk
about here.} exist that allow a match (and if they don't, the proof verifier
will let you know with an error message).  The Metamath language is specified
in such a way that if a set of substitutions exists, it will be unique.
Specifically, the requirement that each variable have a type specified for it
with a \texttt{\$f} statement ensures the uniqueness.

We will illustrate this with an example.
Consider the following Metamath source file:
\begin{verbatim}
$c ( ) -> wff $.
$v p q r s $.
wp $f wff p $.
wq $f wff q $.
wr $f wff r $.
ws $f wff s $.
w2 $a wff ( p -> q ) $.
wnew $p wff ( s -> ( r -> p ) ) $= ws wr wp w2 w2 $.
\end{verbatim}
This Metamath source example shows the definition and ``proof'' (i.e.,
construction) of a well-formed formula (wff)\index{well-formed formula (wff)}
in propositional calculus.  (You may wish to type this example into a file to
experiment with the Metamath program.)  The first two statements declare
(introduce the names of) four constants and four variables.  The next four
statements specify the variable types, namely that
each variable is assumed to be a wff.  Statement \texttt{w2} defines (postulates)
a way to produce a new wff, \texttt{( p -> q )}, from two given wffs \texttt{p} and
\texttt{q}. The mandatory hypotheses of \texttt{w2} are \texttt{wp} and \texttt{wq}.
Statement \texttt{wnew} claims that \texttt{( s -> ( r -> p ) )} is a wff given
three wffs \texttt{s}, \texttt{r}, and \texttt{p}.  More precisely, \texttt{wnew} claims
that the sequence of ten symbols \texttt{wff ( s -> ( r -> p ) )} is provable from
previous assertions and the hypotheses of \texttt{wnew}.  Metamath does not know
or care what a wff is, and as far as it is concerned
the typecode \texttt{wff} is just an
arbitrary constant symbol in a math symbol sequence.  The mandatory hypotheses
of \texttt{wnew} are \texttt{wp}, \texttt{wr}, and \texttt{ws}; \texttt{wq} is an optional
hypothesis.  In our particular proof, the optional hypothesis is not
referenced, but in general, any combination of active (i.e.\ optional and
mandatory) hypotheses could be referenced.  The proof of statement \texttt{wnew}
is the sequence of five labels starting with \texttt{ws} (step~1) and ending with
\texttt{w2} (step~5).

When Metamath verifies the proof, it scans the proof from left to right.  We
will examine what happens at each step of the proof.  The stack starts off
empty.  At step 1, Metamath looks up label \texttt{ws} and determines that it is a
hypothesis, so it pushes the symbol sequence of statement \texttt{ws} onto the
stack:

\begin{center}\begin{tabular}{|l|l|}\hline
{Stack location} & {Contents} \\ \hline \hline
1 & \texttt{wff s} \\ \hline
\end{tabular}\end{center}

Metamath sees that the labels \texttt{wr} and \texttt{wp} in steps~2 and 3 are also
hypotheses, so it pushes them onto the stack.  After step~3, the stack looks
like
this:

\begin{center}\begin{tabular}{|l|l|}\hline
{Stack location} & {Contents} \\ \hline \hline
3 & \texttt{wff p} \\ \hline
2 & \texttt{wff r} \\ \hline
1 & \texttt{wff s} \\ \hline
\end{tabular}\end{center}

At step 4, Metamath sees that label \texttt{w2} is an assertion, so it must do
some processing.  First, it associates the mandatory hypotheses of \texttt{w2},
which are \texttt{wp} and \texttt{wq}, with stack locations~2 and 3, {\em in that
order}. Metamath determines that the only possible way
to make hypothesis \texttt{wp} match (become identical to) stack location~2 and
\texttt{wq} match stack location 3 is to substitute variable \texttt{p} with \texttt{r}
and \texttt{q} with \texttt{p}.  Metamath makes these substitutions into \texttt{w2} and
obtains the symbol sequence \texttt{wff ( r -> p )}.  It removes the hypotheses
from stack locations~2 and 3, then places the result into stack location~2:

\begin{center}\begin{tabular}{|l|l|}\hline
{Stack location} & {Contents} \\ \hline \hline
2 & \texttt{wff ( r -> p )} \\ \hline
1 & \texttt{wff s} \\ \hline
\end{tabular}\end{center}

At step 5, Metamath sees that label \texttt{w2} is an assertion, so it must again
do some processing.  First, it matches the mandatory hypotheses of \texttt{w2},
which are \texttt{wp} and \texttt{wq}, to stack locations 1 and 2.
Metamath determines that the only possible way to make the
hypotheses match is to substitute variable \texttt{p} with \texttt{s} and \texttt{q} with
\texttt{( r -> p )}.  Metamath makes these substitutions into \texttt{w2} and obtains
the symbol
sequence \texttt{wff ( s -> ( r -> p ) )}.  It removes stack
locations 1 and 2, then places the result into stack location~1:

\begin{center}\begin{tabular}{|l|l|}\hline
{Stack location} & {Contents} \\ \hline \hline
1 & \texttt{wff ( s -> ( r -> p ) )} \\ \hline
\end{tabular}\end{center}

After Metamath finishes processing the proof, it checks to see that the
stack contains exactly one element and that this element is
the same as the math symbol sequence in the
\texttt{\$p}\index{\texttt{\$p} statement} statement.  This is the case for our
proof of \texttt{wnew},
so we have proved \texttt{wnew} successfully.  If the result
differs, Metamath will notify you with an error message.  An error message
will also result if the stack contains more than one entry at the end of the
proof, or if the stack did not contain enough entries at any point in the
proof to match all of the mandatory hypotheses\index{mandatory hypothesis} of
an assertion.  Finally, Metamath will notify you with an error message if no
substitution is possible that will make a referenced assertion's hypothesis
match the
stack entries.  You may want to experiment with the different kinds of errors
that Metamath will detect by making some small changes in the proof of our
example.

Metamath's proof notation was designed primarily to express proofs in a
relatively compact manner, not for readability by humans.  Metamath can display
proofs in a number of different ways with the \texttt{show proof}\index{\texttt{show
proof} command} command.  The
\texttt{/lemmon} qualifier displays it in a format that is easier to read when the
proofs are short, and you saw examples of its use in Chapter~\ref{using}.  For
longer proofs, it is useful to see the tree structure of the proof.  A tree
structure is displayed when the \texttt{/lemmon} qualifier is omitted.  You will
probably find this display more convenient as you get used to it. The tree
display of the proof in our example looks like
this:\label{treeproof}\index{tree-style proof}\index{proof!tree-style}
\begin{verbatim}
1     wp=ws    $f wff s
2        wp=wr    $f wff r
3        wq=wp    $f wff p
4     wq=w2    $a wff ( r -> p )
5  wnew=w2  $a wff ( s -> ( r -> p ) )
\end{verbatim}
The number to the left of each line is the step number.  Following it is a
{\bf hypothesis association}\index{hypothesis association}, consisting of two
labels\index{label} separated by \texttt{=}.  To the left of the \texttt{=} (except
in the last step) is the label of a hypothesis of an assertion referenced
later in the proof; here, steps 1 and 4 are the hypothesis associations for
the assertion \texttt{w2} that is referenced in step 5.  A hypothesis association
is indented one level more than the assertion that uses it, so it is easy to
find the corresponding assertion by moving directly down until the indentation
level decreases to one less than where you started from.  To the right of each
\texttt{=} is the proof step label for that proof step.  The statement keyword of
the proof step label is listed next, followed by the content of the top of the
stack (the most recent stack entry) as it exists after that proof step is
processed.  With a little practice, you should have no trouble reading proofs
displayed in this format.

Metamath proofs include the syntax construction of a formula.
In standard mathematics, this kind of
construction is not considered a proper part of the proof at all, and it
certainly becomes rather boring after a while.
Therefore,
by default the \texttt{show proof}\index{\texttt{show proof}
command} command does not show the syntax construction.
Historically \texttt{show proof} command
\textit{did} show the syntax construction, and you needed to add the
\texttt{/essential} option to hide, them, but today
\texttt{/essential} is the default and you need to use
\texttt{/all} to see the syntax constructions.

When verifying a proof, Metamath will check that no mandatory
\texttt{\$d}\index{\texttt{\$d} statement}\index{mandatory \texttt{\$d}
statement} statement of an assertion referenced in a proof is violated
when substitutions\index{substitution!variable}\index{variable
substitution} are made to the variables in the assertion.  For details
see Section~\ref{spec4} or \ref{dollard}.

\subsection{The Concept of Unification} \label{unify}

During the course of verifying a proof, when Metamath\index{Metamath}
encounters an assertion label\index{assertion label}, it associates the
mandatory hypotheses\index{mandatory hypothesis} of the assertion with the top
entries of the RPN stack\index{stack}\index{RPN stack}.  Metamath then
determines what substitutions\index{substitution!variable}\index{variable
substitution} it must make to the variables in the assertion's mandatory
hypotheses in order for these hypotheses to become identical to their
corresponding stack entries.  This process is called {\bf
unification}\index{unification}.  (We also informally use the term
``unification'' to refer to a set of substitutions that results from the
process, as in ``two unifications are possible.'')  After the substitutions
are made, the hypotheses are said to be {\bf unified}.

If no such substitutions are possible, Metamath will consider the proof
incorrect and notify you with an error message.
% (deleted 3/10/07, per suggestion of Mel O'Cat:)
% The syntax of the
% Metamath language ensures that if a set of substitutions exists, it
% will be unique.

The general algorithm for unification described in the literature is
somewhat complex.
However, in the case of Metamath it is intentionally trivial.
Mandatory hypotheses must be
pushed on the proof stack in the order in which they appear.
In addition, each variable must have its type specified
with a \texttt{\$f} hypothesis before it is used
and that each \texttt{\$f} hypothesis
have the restricted syntax of a typecode (a constant) followed by a variable.
The typecode in the \texttt{\$f} hypothesis must match the first symbol of
the corresponding RPN stack entry (which will also be a constant), so
the only possible match for the variable in the \texttt{\$f} hypothesis is
the sequence of symbols in the stack entry after the initial constant.

In the Proof Assistant\index{Proof Assistant}, a more general unification
algorithm is used.  While a proof is being developed, sometimes not enough
information is available to determine a unique unification.  In this case
Metamath will ask you to pick the correct one.\index{ambiguous
unification}\index{unification!ambiguous}

\section{Extensions to the Metamath Language}\index{extended
language}

\subsection{Comments in the Metamath Language}\label{comments}
\index{markup notation}
\index{comments!markup notation}

The commenting feature allows you to annotate the contents of
a database.  Just as with most
computer languages, comments are ignored for the purpose of interpreting the
contents of the database. Comments effectively act as
additional white space\index{white
space} between tokens
when a database is parsed.

A comment may be placed at the beginning, end, or
between any two tokens\index{token} in a source file.

Comments have the following syntax:
\begin{center}
 \texttt{\$(} {\em text} \texttt{\$)}
\end{center}
Here,\index{\texttt{\$(} and \texttt{\$)} auxiliary
keywords}\index{comment} {\em text} is a string, possibly empty, of any
characters in Metamath's character set (p.~\pageref{spec1chars}), except
that the character strings \texttt{\$(} and \texttt{\$)} may not appear
in {\em text}.  Thus nested comments are not
permitted:\footnote{Computer languages have differing standards for
nested comments, and rather than picking one it was felt simplest not to
allow them at all, at least in the current version of
Metamath\index{Metamath!limitations of version 0.07.30}.} Metamath will
complain if you give it
\begin{center}
 \texttt{\$( This is a \$( nested \$) comment.\ \$)}
\end{center}
To compensate for this non-nesting behavior, I often change all \texttt{\$}'s
to \texttt{@}'s in sections of Metamath code I wish to comment out.

The Metamath program supports a number of markup mechanisms and conventions
to generate good-looking results in \LaTeX\ and {\sc html},
as discussed below.
These markup features have to do only with how the comments are typeset,
and have no effect on how Metamath verifies the proofs in the database.
The improper
use of them may result in incorrectly typeset output, but no Metamath
error messages will result during the \texttt{read} and \texttt{verify
proof} commands.  (However, the \texttt{write
theorem\texttt{\char`\_}list} command
will check for markup errors as a side-effect of its
{\sc html} generation.)
Section~\ref{texout} has instructions for creating \LaTeX\ output, and
section~\ref{htmlout} has instructions for creating
{\sc html}\index{HTML} output.

\subsubsection{Headings}\label{commentheadings}

If the \texttt{\$(} is immediately followed by a new line
starting with a heading marker, it is a header.
This can start with:

\begin{itemize}
 \item[] \texttt{\#\#\#\#} - major part header
 \item[] \texttt{\#*\#*} - section header
 \item[] \texttt{=-=-} - subsection header
 \item[] \texttt{-.-.} - subsubsection header
\end{itemize}

The line following the marker line
will be used for the table of contents entry, after trimming spaces.
The next line should be another (closing) matching marker line.
Any text after that
but before the closing \texttt{\$}, such as an extended description of the
section, will be included on the \texttt{mmtheoremsNNN.html} page.

For more information, run
\texttt{help write theorem\char`\_list}.

\subsubsection{Math mode}
\label{mathcomments}
\index{\texttt{`} inside comments}
\index{\texttt{\char`\~} inside comments}
\index{math mode}

Inside of comments, a string of tokens\index{token} enclosed in
grave accents\index{grave accent (\texttt{`})} (\texttt{`}) will be converted
to standard mathematical symbols during
{\sc HTML}\index{HTML} or \LaTeX\ output
typesetting,\index{latex@{\LaTeX}} according to the information in the
special \texttt{\$t}\index{\texttt{\$t} comment}\index{typesetting
comment} comment in the database
(see section~\ref{tcomment} for information about the typesetting
comment, and Appendix~\ref{ASCII} to see examples of its results).

The first grave accent\index{grave accent (\texttt{`})} \texttt{`}
causes the output processor to enter {\bf math mode}\index{math mode}
and the second one exits it.
In this
mode, the characters following the \texttt{`} are interpreted as a
sequence of math symbol tokens separated by white space\index{white
space}.  The tokens are looked up in the \texttt{\$t}
comment\index{\texttt{\$t} comment}\index{typesetting comment} and if
found, they will be replaced by the standard mathematical symbols that
they correspond to before being placed in the typeset output file.  If
not found, the symbol will be output as is and a warning will be issued.
The tokens do not have to be active in the database, although a warning
will be issued if they are not declared with \texttt{\$c} or
\texttt{\$v} statements.

Two consecutive
grave accents \texttt{``} are treated as a single actual grave accent
(both inside and outside of math mode) and will not cause the output
processor to enter or exit math mode.

Here is an example of its use\index{Pierce's axiom}:
\begin{center}
\texttt{\$( Pierce's axiom, ` ( ( ph -> ps ) -> ph ) -> ph ` ,\\
         is not very intuitive. \$)}
\end{center}
becomes
\begin{center}
   \texttt{\$(} Pierce's axiom, $((\varphi \rightarrow \psi)\rightarrow
\varphi)\rightarrow \varphi$, is not very intuitive. \texttt{\$)}
\end{center}

Note that the math symbol tokens\index{token} must be surrounded by white
space\index{white space}.
%, since there is no context that allows ambiguity to be
%resolved, as is the case with math symbol sequences in some of the Metamath
%statements.
White space should also surround the \texttt{`}
delimiters.

The math mode feature also gives you a quick and easy way to generate
text containing mathematical symbols, independently of the intended
purpose of Metamath.\index{Metamath!using as a math editor} To do this,
simply create your text with grave accents surrounding your formulas,
after making sure that your math symbols are mapped to \LaTeX\ symbols
as described in Appendix~\ref{ASCII}.  It is easier if you start with a
database with predefined symbols such as \texttt{set.mm}.  Use your
grave-quoted math string to replace an existing comment, then typeset
the statement corresponding to that comment following the instructions
from the \texttt{help tex} command in the Metamath program.  You will
then probably want to edit the resulting file with a text editor to fine
tune it to your exact needs.

\subsubsection{Label Mode}\index{label mode}

Outside of math mode, a tilde\index{tilde (\texttt{\char`\~})} \verb/~/
indicates to Metamath's\index{Metamath} output processor that the
token\index{token} that follows (i.e.\ the characters up to the next
white space\index{white space}) represents a statement label or URL.
This formatting mode is called {\bf label mode}\index{label mode}.
If a literal tilde
is desired (outside of math mode) instead of label mode,
use two tildes in a row to represent it.

When generating a \LaTeX\ output file,
the following token will be formatted in \texttt{typewriter}
font, and the tilde removed, to make it stand out from the rest of the text.
This formatting will be applied to all characters after the
tilde up to the first white space\index{white space}.
Whether
or not the token is an actual statement label is not checked, and the
token does not have to have the correct syntax for a label; no error
messages will be produced.  The only effect of the label mode on the
output is that typewriter font will be used for the tokens that are
placed in the \LaTeX\ output file.

When generating {\sc html},
the tokens after the tilde {\em must} be a URL (either http: or https:)
or a valid label.
Error messages will be issued during that output if they aren't.
A hyperlink will be generated to that URL or label.

\subsubsection{Link to bibliographical reference}\index{citation}%
\index{link to bibliographical reference}

Bibliographical references are handled specially when generating
{\sc html} if formatted specially.
Text in the form \texttt{[}{\em author}\texttt{]}
is considered a link to a bibliographical reference.
See \texttt{help html} and \texttt{help write
bibliography} in the Metamath program for more
information.
% \index{\texttt{\char`\[}\ldots\texttt{]} inside comments}
See also Sections~\ref{tcomment} and \ref{wrbib}.

The \texttt{[}{\em author}\texttt{]} notation will also create an entry in
the bibliography cross-reference file generated by \texttt{write
bibliography} (Section~\ref{wrbib}) for {\sc HTML}.
For this to work properly, the
surrounding comment must be formatted as follows:
\begin{quote}
    {\em keyword} {\em label} {\em noise-word}
     \texttt{[}{\em author}\texttt{] p.} {\em number}
\end{quote}
for example
\begin{verbatim}
     Theorem 5.2 of [Monk] p. 223
\end{verbatim}
The {\em keyword} is not case sensitive and must be one of the following:
\begin{verbatim}
     theorem lemma definition compare proposition corollary
     axiom rule remark exercise problem notation example
     property figure postulate equation scheme chapter
\end{verbatim}
The optional {\em label} may consist of more than one
(non-{\em keyword} and non-{\em noise-word}) word.
The optional {\em noise-word} is one of:
\begin{verbatim}
     of in from on
\end{verbatim}
and is  ignored when the cross-reference file is created.  The
\texttt{write
biblio\-graphy} command will perform error checking to verify the
above format.\index{error checking}

\subsubsection{Parentheticals}\label{parentheticals}

The end of a comment may include one or more parenthicals, that is,
statements enclosed in parentheses.
The Metamath program looks for certain parentheticals and can issue
warnings based on them.
They are:

\begin{itemize}
 \item[] \texttt{(Contributed by }
   \textit{NAME}\texttt{,} \textit{DATE}\texttt{.)} -
   document the original contributor's name and the date it was created.
 \item[] \texttt{(Revised by }
   \textit{NAME}\texttt{,} \textit{DATE}\texttt{.)} -
   document the contributor's name and creation date
   that resulted in significant revision
   (not just an automated minimization or shortening).
 \item[] \texttt{(Proof shortened by }
   \textit{NAME}\texttt{,} \textit{DATE}\texttt{.)} -
   document the contributor's name and date that developed a significant
   shortening of the proof (not just an automated minimization).
 \item[] \texttt{(Proof modification is discouraged.)} -
   Note that this proof should normally not be modified.
 \item[] \texttt{(New usage is discouraged.)} -
   Note that this assertion should normally not be used.
\end{itemize}

The \textit{DATE} must be in form YYYY-MMM-DD, where MMM is the
English abbreviation of that month.

\subsubsection{Other markup}\label{othermarkup}
\index{markup notation}

There are other markup notations for generating good-looking results
beyond math mode and label mode:

\begin{itemize}
 \item[]
         \texttt{\char`\_} (underscore)\index{\texttt{\char`\_} inside comments} -
             Italicize text starting from
              {\em space}\texttt{\char`\_}{\em non-space} (i.e.\ \texttt{\char`\_}
              with a space before it and a non-space character after it) until
             the next
             {\em non-space}\texttt{\char`\_}{\em space}.  Normal
             punctuation (e.g.\ a trailing
             comma or period) is ignored when determining {\em space}.
 \item[]
         \texttt{\char`\_} (underscore) - {\em
         non-space}\texttt{\char`\_}{\em non-space-string}, where
          {\em non-space-string} is a string of non-space characters,
         will make {\em non-space-string} become a subscript.
 \item[]
         \texttt{<HTML>}...\texttt{</HTML>} - do not convert
         ``\texttt{<}'' and ``\texttt{>}''
         in the enclosed text when generating {\sc HTML},
         otherwise process markup normally. This allows direct insertion
         of {\sc html} commands.
 \item[]
       ``\texttt{\&}ref\texttt{;}'' - insert an {\sc HTML}
         character reference.
         This is how to insert arbitrary Unicode characters
         (such as accented characters).  Currently only directly supported
         when generating {\sc HTML}.
\end{itemize}

It is recommended that spaces surround any \texttt{\char`\~} and
\texttt{`} tokens in the comment and that a space follow the {\em label}
after a \texttt{\char`\~} token.  This will make global substitutions
to change labels and symbol names much easier and also eliminate any
future chance of ambiguity.  Spaces around these tokens are automatically
removed in the final output to conform with normal rules of punctuation;
for example, a space between a trailing \texttt{`} and a left parenthesis
will be removed.

A good way to become familiar with the markup notation is to look at
the extensive examples in the \texttt{set.mm} database.

\subsection{The Typesetting Comment (\texttt{\$t})}\label{tcomment}

The typesetting comment \texttt{\$t} in the input database file
provides the information necessary to produce good-looking results.
It provides \LaTeX\ and {\sc html}
definitions for math symbols,
as well supporting as some
customization of the generated web page.
If you add a new token to a database, you should also
update the \texttt{\$t} comment information if you want to eventually
create output in \LaTeX\ or {\sc HTML}.
See the
\texttt{set.mm}\index{set theory database (\texttt{set.mm})} database
file for an extensive example of a \texttt{\$t} comment illustrating
many of the features described below.

Programs that do not need to generate good-looking presentation results,
such as programs that only verify Metamath databases,
can completely ignore typesetting comments
and just treat them as normal comments.
Even the Metamath program only consults the
\texttt{\$t} comment information when it needs to generate typeset output
in \LaTeX\ or {\sc HTML}
(e.g., when you open a \LaTeX\ output file with the \texttt{open tex} command).

We will first discuss the syntax of typesetting comments, and then
briefly discuss how this can be used within the Metamath program.

\subsubsection{Typesetting Comment Syntax Overview}

The typesetting comment is identified by the token
\texttt{\$t}\index{\texttt{\$t} comment}\index{typesetting comment} in
the comment, and the typesetting comment ends at the matching
\texttt{\$)}:
\[
  \mbox{\tt \$(\ }
  \mbox{\tt \$t\ }
  \underbrace{
    \mbox{\tt \ \ \ \ \ \ \ \ \ \ \ }
    \cdots
    \mbox{\tt \ \ \ \ \ \ \ \ \ \ \ }
  }_{\mbox{Typesetting definitions go here}}
  \mbox{\tt \ \$)}
\]

There must be one or more white space characters, and only white space
characters, between the \texttt{\$(} that starts the comment
and the \texttt{\$t} symbol,
and the \texttt{\$t} must be followed by one
or more white space characters
(see section \ref{whitespace} for the definition of white space characters).
The typesetting comment continues until the comment end token \texttt{\$)}
(which must be preceded by one or more white space characters).

In version 0.07.30\index{Metamath!limitations of version 0.07.30} of the
Metamath program, there may be only one \texttt{\$t} comment in a
database.  This restriction may be lifted in the future to allow
many \texttt{\$t} comments in a database.

Between the \texttt{\$t} symbol (and its following white space) and the
comment end token \texttt{\$)} (and its preceding white space)
is a sequence of one or more typesetting definitions, where
each definition has the form
\textit{definition-type arg arg ... ;}.
Each of the zero or more \textit{arg} values
can be either a typesetting data or a keyword
(what keywords are allowed, and where, depends on the specific
\textit{definition-type}).
The \textit{definition-type}, and each argument \textit{arg},
are separated by one or more white space characters.
Every definition ends in an unquoted semicolon;
white space is not required before the terminating semicolon of a definition.
Each definition should start on a new line.\footnote{This
restriction of the current version of Metamath
(0.07.30)\index{Metamath!limitations of version 0.07.30} may be removed
in a future version, but you should do it anyway for readability.}

For example, this typesetting definition:
\begin{center}
 \verb$latexdef "C_" as "\subseteq";$
\end{center}
defines the token \verb$C_$ as the \LaTeX\ symbol $\subseteq$ (which means
``subset'').

Typesetting data is a sequence of one or more quoted strings
(if there is more than one, they are connected by \texttt{\char`\+}).
Often a single quoted string is used to provide data for a definition, using
either double (\texttt{\char`\"}) or single (\texttt{'}) quotation marks.
However,
{\em a quoted string (enclosed in quotation marks) may not include
line breaks.}
A quoted string
may include a quotation mark that matches the enclosing quotes by repeating
the quotation mark twice.  Here are some examples:

\begin{tabu}   { l l }
\textbf{Example} & \textbf{Meaning} \\
\texttt{\char`\"a\char`\"\char`\"b\char`\"} & \texttt{a\char`\"b} \\
\texttt{'c''d'} & \texttt{c'd} \\
\texttt{\char`\"e''f\char`\"} & \texttt{e''f} \\
\texttt{'g\char`\"\char`\"h'} & \texttt{g\char`\"\char`\"h} \\
\end{tabu}

Finally, a long quoted string
may be broken up into multiple quoted strings (considered, as a whole,
a single quoted string) and joined with \texttt{\char`\+}.
You can even use multiple lines as long as a
'+' is at the end of every line except the last one.
The \texttt{\char`\+} should be preceded and followed by at least one
white space character.
Thus, for example,
\begin{center}
 \texttt{\char`\"ab\char`\"\ \char`\+\ \char`\"cd\char`\"
    \ \char`\+\ \\ 'ef'}
\end{center}
is the same as
\begin{center}
 \texttt{\char`\"abcdef\char`\"}
\end{center}

{\sc c}-style comments \texttt{/*}\ldots\texttt{*/} are also supported.

In practice, whenever you add a new math token you will often want to add
typesetting definitions using
\texttt{latexdef}, \texttt{htmldef}, and
\texttt{althtmldef}, as described below.
That way, they will all be up to date.
Of course, whether or not you want to use all three definitions will
depend on how the database is intended to be used.

Below we discuss the different possible \textit{definition-kind} options.
We will show data surrounded by double quotes (in practice they can also use
single quotes and/or be a sequence joined by \texttt{+}s).
We will use specific names for the \textit{data} to make clear what
the data is used for, such as
{\em math-token} (for a Metamath math token,
{\em latex-string} (for string to be placed in a \LaTeX\ stream),
{\em {\sc html}-code} (for {\sc html} code),
and {\em filename} (for a filename).

\subsubsection{Typesetting Comment - \LaTeX}

The syntax for a \LaTeX\ definition is:
\begin{center}
 \texttt{latexdef "}{\em math-token}\texttt{" as "}{\em latex-string}\texttt{";}
\end{center}
\index{latex definitions@\LaTeX\ definitions}%
\index{\texttt{latexdef} statement}

The {\em token-string} and {\em latex-string} are the data
(character strings) for
the token and the \LaTeX\ definition of the token, respectively,

These \LaTeX\ definitions are used by the Metamath program
when it is asked to product \LaTeX output using
the \texttt{write tex} command.

\subsubsection{Typesetting Comment - {\sc html}}

The key kinds of {\sc HTML} definitions have the following syntax:

\vskip 1ex
    \texttt{htmldef "}{\em math-token}\texttt{" as "}{\em
    {\sc html}-code}\texttt{";}\index{\texttt{htmldef} statement}
                    \ \ \ \ \ \ldots

    \texttt{althtmldef "}{\em math-token}\texttt{" as "}{\em
{\sc html}-code}\texttt{";}\index{\texttt{althtmldef} statement}

                    \ \ \ \ \ \ldots

Note that in {\sc HTML} there are two possible definitions for math tokens.
This feature is useful when
an alternate representation of symbols is desired, for example one that
uses Unicode entities and another uses {\sc gif} images.

There are many other typesetting definitions that can control {\sc HTML}.
These include:

\vskip 1ex

    \texttt{htmldef "}{\em math-token}\texttt{" as "}{\em {\sc
    html}-code}\texttt{";}

    \texttt{htmltitle "}{\em {\sc html}-code}\texttt{";}%
\index{\texttt{htmltitle} statement}

    \texttt{htmlhome "}{\em {\sc html}-code}\texttt{";}%
\index{\texttt{htmlhome} statement}

    \texttt{htmlvarcolor "}{\em {\sc html}-code}\texttt{";}%
\index{\texttt{htmlvarcolor} statement}

    \texttt{htmlbibliography "}{\em filename}\texttt{";}%
\index{\texttt{htmlbibliography} statement}

\vskip 1ex

\noindent The \texttt{htmltitle} is the {\sc html} code for a common
title, such as ``Metamath Proof Explorer.''  The \texttt{htmlhome} is
code for a link back to the home page.  The \texttt{htmlvarcolor} is
code for a color key that appears at the bottom of each proof.  The file
specified by {\em filename} is an {\sc html} file that is assumed to
have a \texttt{<A NAME=}\ldots\texttt{>} tag for each bibiographic
reference in the database comments.  For example, if
\texttt{[Monk]}\index{\texttt{\char`\[}\ldots\texttt{]} inside comments}
occurs in the comment for a theorem, then \texttt{<A NAME='Monk'>} must
be present in the file; if not, a warning message is given.

Associated with
\texttt{althtmldef}
are the statements
\vskip 1ex

    \texttt{htmldir "}{\em
      directoryname}\texttt{";}\index{\texttt{htmldir} statement}

    \texttt{althtmldir "}{\em
     directoryname}\texttt{";}\index{\texttt{althtmldir} statement}

\vskip 1ex
\noindent giving the directories of the {\sc gif} and Unicode versions
respectively; their purpose is to provide cross-linking between the
two versions in the generated web pages.

When two different types of pages need to be produced from a single
database, such as the Hilbert Space Explorer that extends the Metamath
Proof Explorer, ``extended'' variables may be declared in the
\texttt{\$t} comment:
\vskip 1ex

    \texttt{exthtmltitle "}{\em {\sc html}-code}\texttt{";}%
\index{\texttt{exthtmltitle} statement}

    \texttt{exthtmlhome "}{\em {\sc html}-code}\texttt{";}%
\index{\texttt{exthtmlhome} statement}

    \texttt{exthtmlbibliography "}{\em filename}\texttt{";}%
\index{\texttt{exthtmlbibliography} statement}

\vskip 1ex
\noindent When these are declared, you also must declare
\vskip 1ex

    \texttt{exthtmllabel "}{\em label}\texttt{";}%
\index{\texttt{exthtmllabel} statement}

\vskip 1ex \noindent that identifies the database statement where the
``extended'' section of the database starts (in our example, where the
Hilbert Space Explorer starts).  During the generation of web pages for
that starting statement and the statements after it, the {\sc html} code
assigned to \texttt{exthtmltitle} and \texttt{exthtmlhome} is used
instead of that assigned to \texttt{htmltitle} and \texttt{htmlhome},
respectively.

\begin{sloppy}
\subsection{Additional Information Com\-ment (\texttt{\$j})} \label{jcomment}
\end{sloppy}

The additional information comment, aka the
\texttt{\$j}\index{\texttt{\$j} comment}\index{additional information comment}
comment,
provides a way to add additional structured information that can
be optionally parsed by systems.

The additional information comment is parsed the same way as the
typesetting comment (\texttt{\$t}) (see section \ref{tcomment}).
That is,
the additional information comment begins with the token
\texttt{\$j} within a comment,
and continues until the comment close \texttt{\$)}.
Within an additional information comment is a sequence of one or more
commands of the form \texttt{command arg arg ... ;}
where each of the zero or more \texttt{arg} values
can be either a quoted string or a keyword.
Note that every command ends in an unquoted semicolon.
If a verifier is parsing an additional information comment, but
doesn't recognize a particular command, it must skip the command
by finding the end of the command (an unquoted semicolon).

A database may have 0 or more additional information comments.
Note, however, that a verifier may ignore these comments entirely or only
process certain commands in an additional information comment.
The \texttt{mmj2} verifier supports many commands in additional information
comments.
We encourage systems that process additional information comments
to coordinate so that they will use the same command for the same effect.

Examples of additional information comments with various commands
(from the \texttt{set.mm} database) are:

\begin{itemize}
   \item Define the syntax and logical typecodes,
     and declare that our grammar is
     unambiguous (verifiable using the KLR parser, with compositing depth 5).
\begin{verbatim}
  $( $j
    syntax 'wff';
    syntax '|-' as 'wff';
    unambiguous 'klr 5';
  $)
\end{verbatim}

   \item Register $\lnot$ and $\rightarrow$ as primitive expressions
           (lacking definitions).
\begin{verbatim}
  $( $j primitive 'wn' 'wi'; $)
\end{verbatim}

   \item There is a special justification for \texttt{df-bi}.
\begin{verbatim}
  $( $j justification 'bijust' for 'df-bi'; $)
\end{verbatim}

   \item Register $\leftrightarrow$ as an equality for its type (wff).
\begin{verbatim}
  $( $j
    equality 'wb' from 'biid' 'bicomi' 'bitri';
    definition 'dfbi1' for 'wb';
  $)
\end{verbatim}

   \item Theorem \texttt{notbii} is the congruence law for negation.
\begin{verbatim}
  $( $j congruence 'notbii'; $)
\end{verbatim}

   \item Add \texttt{setvar} as a typecode.
\begin{verbatim}
  $( $j syntax 'setvar'; $)
\end{verbatim}

   \item Register $=$ as an equality for its type (\texttt{class}).
\begin{verbatim}
  $( $j equality 'wceq' from 'eqid' 'eqcomi' 'eqtri'; $)
\end{verbatim}

\end{itemize}


\subsection{Including Other Files in a Metamath Source File} \label{include}
\index{\texttt{\$[} and \texttt{\$]} auxiliary keywords}

The keywords \texttt{\$[} and \texttt{\$]} specify a file to be
included\index{included file}\index{file inclusion} at that point in a
Metamath\index{Metamath} source file\index{source file}.  The syntax for
including a file is as follows:
\begin{center}
\texttt{\$[} {\em file-name} \texttt{\$]}
\end{center}

The {\em file-name} should be a single token\index{token} with the same syntax
as a math symbol (i.e., all 93 non-whitespace
printable characters other than \texttt{\$} are
allowed, subject to the file-naming limitations of your operating system).
Comments may appear between the \texttt{\$[} and \texttt{\$]} keywords.  Included
files may include other files, which may in turn include other files, and so
on.

For example, suppose you want to use the set theory database as the starting
point for your own theory.  The first line in your file could be
\begin{center}
\texttt{\$[ set.mm \$]}
\end{center} All of the information (axioms, theorems,
etc.) in \texttt{set.mm} and any files that {\em it} includes will become
available for you to reference in your file. This can help make your work more
modular. A drawback to including files is that if you change the name of a
symbol or the label of a statement, you must also remember to update any
references in any file that includes it.


The naming conventions for included files are the same as those of your
operating system.\footnote{On the Macintosh, prior to Mac OS X,
 a colon is used to separate disk
and folder names from your file name.  For example, {\em volume}\texttt{:}{\em
file-name} refers to the root directory, {\em volume}\texttt{:}{\em
folder-name}\texttt{:}{\em file-name} refers to a folder in root, and {\em
volume}\texttt{:}{\em folder-name}\texttt{:}\ldots\texttt{:}{\em file-name} refers to a
deeper folder.  A simple {\em file-name} refers to a file in the folder from
which you launch the Metamath application.  Under Mac OS X and later,
the Metamath program is run under the Terminal application, which
conforms to Unix naming conventions.}\index{Macintosh file
names}\index{file names!Macintosh}\label{includef} For compatibility among
operating systems, you should keep the file names as simple as possible.  A
good convention to use is {\em file}\texttt{.mm} where {\em file} is eight
characters or less, in lower case.

There is no limit to the nesting depth of included files.  One thing that you
should be aware of is that if two included files themselves include a common
third file, only the {\em first} reference to this common file will be read
in.  This allows you to include two or more files that build on a common
starting file without having to worry about label and symbol conflicts that
would occur if the common file were read in more than once.  (In fact, if a
file includes itself, the self-reference will be ignored, although of course
it would not make any sense to do that.)  This feature also means, however,
that if you try to include a common file in several inner blocks, the result
might not be what you expect, since only the first reference will be replaced
with the included file (unlike the include statement in most other computer
languages).  Thus you would normally include common files only in the
outermost block\index{outermost block}.

\subsection{Compressed Proof Format}\label{compressed1}\index{compressed
proof}\index{proof!compressed}

The proof notation presented in Section~\ref{proof} is called a
{\bf normal proof}\index{normal proof}\index{proof!normal} and in principle is
sufficient to express any proof.  However, proofs often contain steps and
subproofs that are identical.  This is particularly true in typical
Metamath\index{Metamath} applications, because Metamath requires that the math
symbol sequence (usually containing a formula) at each step be separately
constructed, that is, built up piece by piece. As a result, a lot of
repetition often results.  The {\bf compressed proof} format allows Metamath
to take advantage of this redundancy to shorten proofs.

The specification for the compressed proof format is given in
Appen\-dix~\ref{compressed}.

Normally you need not concern yourself with the details of the compressed
proof format, since the Metamath program will allow you to convert from
the normal format to the compressed format with ease, and will also
automatically convert from the compressed format when proofs are displayed.
The overall structure of the compressed format is as follows:
\begin{center}
  \texttt{\$= ( } {\em label-list} \texttt{) } {\em compressed-proof\ }\ \texttt{\$.}
\end{center}
\index{\texttt{\$=} keyword}
The first \texttt{(} serves as a flag to Metamath that a compressed proof
follows.  The {\em label-list} includes all statements referred to by the
proof except the mandatory hypotheses\index{mandatory hypothesis}.  The {\em
compressed-proof} is a compact encoding of the proof, using upper-case
letters, and can be thought of as a large integer in base 26.  White
space\index{white space} inside a {\em compressed-proof} is
optional and is ignored.

It is important to note that the order of the mandatory hypotheses of
the statement being proved must not be changed if the compressed proof
format is used, otherwise the proof will become incorrect.  The reason
for this is that the mandatory hypotheses are not mentioned explicitly
in the compressed proof in order to make the compression more efficient.
If you wish to change the order of mandatory hypotheses, you must first
convert the proof back to normal format using the \texttt{save proof
{\em statement} /normal}\index{\texttt{save proof} command} command.
Later, you can go back to compressed format with \texttt{save proof {\em
statement} /compressed}.

During error checking with the \texttt{verify proof} command, an error
found in a compressed proof may point to a character in {\em
compressed-proof}, which may not be very meaningful to you.  In this
case, try to \texttt{save proof /normal} first, then do the
\texttt{verify proof} again.  In general, it is best to make sure a
proof is correct before saving it in compressed format, because severe
errors are less likely to be recoverable than in normal format.

\subsection{Specifying Unknown Proofs or Subproofs}\label{unknown}

In a proof under development, any step or subproof that is not yet known
may be represented with a single \texttt{?}.  For the purposes of
parsing the proof, the \texttt{?}\ \index{\texttt{]}@\texttt{?}\ inside
proofs} will push a single entry onto the RPN stack just as if it were a
hypothesis.  While developing a proof with the Proof
Assistant\index{Proof Assistant}, a partially developed proof may be
saved with the \texttt{save new{\char`\_}proof}\index{\texttt{save
new{\char`\_}proof} command} command, and \texttt{?}'s will be placed at
the appropriate places.

All \texttt{\$p}\index{\texttt{\$p} statement} statements must have
proofs, even if they are entirely unknown.  Before creating a proof with
the Proof Assistant, you should specify a completely unknown proof as
follows:
\begin{center}
  {\em label} \texttt{\$p} {\em statement} \texttt{\$= ?\ \$.}
\end{center}
\index{\texttt{\$=} keyword}
\index{\texttt{]}@\texttt{?}\ inside proofs}

The \texttt{verify proof}\index{\texttt{verify proof} command} command
will check the known portions of a partial proof for errors, but will
warn you that the statement has not been proved.

Note that partially developed proofs may be saved in compressed format
if desired.  In this case, you will see one or more \texttt{?}'s in the
{\em compressed-proof} part.\index{compressed
proof}\index{proof!compressed}

\section{Axioms vs.\ Definitions}\label{definitions}

The \textit{basic}
Metamath\index{Metamath} language and program
make no distinction\index{axiom vs.\
definition} between axioms\index{axiom} and
definitions.\index{definition} The \texttt{\$a}\index{\texttt{\$a}
statement} statement is used for both.  At first, this may seem
puzzling.  In the minds of many mathematicians, the distinction is
clear, even obvious, and hardly worth discussing.  A definition is
considered to be merely an abbreviation that can be replaced by the
expression for which it stands; although unless one actually does this,
to be precise then one should say that a theorem\index{theorem} is a
consequence of the axioms {\em and} the definitions that are used in the
formulation of the theorem \cite[p.~20]{Behnke}.\index{Behnke, H.}

\subsection{What is a Definition?}

What is a definition?  In its simplest form, a definition introduces a new
symbol and provides an unambiguous rule to transform an expression containing
the new symbol to one without it.  The concept of a ``proper
definition''\index{proper definition}\index{definition!proper} (as opposed to
a creative definition)\index{creative definition}\index{definition!creative}
that is usually agreed upon is (1) the definition should not strengthen the
language and (2) any symbols introduced by the definition should be eliminable
from the language \cite{Nemesszeghy}\index{Nemesszeghy, E. Z.}.  In other
words, they are mere typographical conveniences that do not belong to the
system and are theoretically superfluous.  This may seem obvious, but in fact
the nature of definitions can be subtle, sometimes requiring difficult
metatheorems to establish that they are not creative.

A more conservative stance was taken by logician S.
Le\'{s}niewski.\index{Le\'{s}niewski, S.}
\begin{quote}
Le\'{s}niewski
regards definitions as theses of the system.  In this respect they do
not differ either from the axioms or from theorems, i.e.\ from the
theses added to the system on the basis of the rule of substitution or
the rule of detachment [modus ponens].  Once definitions have been
accepted as theses of the system, it becomes necessary to consider them
as true propositions in the same sense in which axioms are true
\cite{Lejewski}.
\end{quote}\index{Lejewski, Czeslaw}

Let us look at some simple examples of definitions in propositional
calculus.  Consider the definition of logical {\sc or}
(disjunction):\index{disjunction ($\vee$)} ``$P\vee Q$ denotes $\neg P
\rightarrow Q$ (not $P$ implies $Q$).''  It is very easy to recognize a
statement making use of this definition, because it introduces the new
symbol $\vee$ that did not previously exist in the language.  It is easy
to see that no new theorems of the original language will result from
this definition.

Next, consider a definition that eliminates parentheses:  ``$P
\rightarrow Q\rightarrow R$ denotes $P\rightarrow (Q \rightarrow R)$.''
This is more subtle, because no new symbols are introduced.  The reason
this definition is considered proper is that no new symbol sequences
that are valid wffs (well-formed formulas)\index{well-formed formula
(wff)} in the original language will result from the definition, since
``$P \rightarrow Q\rightarrow R$'' is not a wff in the original
language.  Here, we implicitly make use of the fact that there is a
decision procedure that allows us to determine whether or not a symbol
sequence is a wff, and this fact allows us to use symbol sequences that
are not wffs to represent other things (such as wffs) by means of the
definition.  However, to justify the definition as not being creative we
need to prove that ``$P \rightarrow Q\rightarrow R$'' is in fact not a
wff in the original language, and this is more difficult than in the
case where we simply introduce a new symbol.

%Now let's take this reasoning to an extreme.  Propositional calculus is a
%decidable theory,\footnote{This means that a mechanical algorithm exists to
%determine whether or not a wff is a theorem.} so in principle we could make use
%of symbol sequences that are not theorems to represent other things (say, to
%encode actual theorems in a more compact way).  For example, let us extend the
%language by defining a wff ``$P$'' in the extended language as the theorem
%``$P\rightarrow P$''\footnote{This is one of the first theorems proved in the
%Metamath database \texttt{set.mm}.}\index{set
%theory database (\texttt{set.mm})} in the original language whenever ``$P$'' is
%not a theorem in the original language.  In the extended language, any wff
%``$Q$'' thus represents a theorem; to find out what theorem (in the original
%language) ``$Q$'' represents, we determine whether ``$Q$'' is a theorem in the
%original language (before the definition was introduced).  If so, we're done; if
%not, we replace ``$Q$'' by ``$Q\rightarrow Q$'' to eliminate the definition.
%This definition is therefore eliminable, and it does not ``strengthen'' the
%language because any wff that is not a theorem is not in the set of statements
%provable in the original language and thus is available for use by definitions.
%
%Of course, a definition such as this would render practically useless the
%communication of theorems of propositional calculus; but
%this is just a human shortcoming, since we can't always easily discern what is
%and is not a theorem by inspection.  In fact, the extended theory with this
%definition has no more and no less information than the original theory; it just
%expresses certain theorems of the form ``$P\rightarrow P$''
%in a more compact way.
%
%The point here is that what constitutes a proper definition is a matter of
%judgment about whether a symbol sequence can easily be recognized by a human
%as invalid in some sense (for example, not a wff); if so, the symbol sequence
%can be appropriated for use by a definition in order to make the extended
%language more compact.  Metamath\index{Metamath} lacks the ability to make this
%judgment, since as far as Metamath is concerned the definition of a wff, for
%example, is arbitrary.  You define for Metamath how wffs\index{well-formed
%formula (wff)} are constructed according to your own preferred style.  The
%concept of a wff may not even exist in a given formal system\index{formal
%system}.  Metamath treats all definitions as if they were new axioms, and it
%is up to the human mathematician to judge whether the definition is ``proper''
%'\index{proper definition}\index{definition!proper} in some agreed-upon way.

What constitutes a definition\index{definition} versus\index{axiom vs.\
definition} an axiom\index{axiom} is sometimes arbitrary in mathematical
literature.  For example, the connectives $\vee$ ({\sc or}), $\wedge$
({\sc and}), and $\leftrightarrow$ (equivalent to) in propositional
calculus are usually considered defined symbols that can be used as
abbreviations for expressions containing the ``primitive'' connectives
$\rightarrow$ and $\neg$.  This is the way we treat them in the standard
logic and set theory database \texttt{set.mm}\index{set theory database
(\texttt{set.mm})}.  However, the first three connectives can also be
considered ``primitive,'' and axiom systems have been devised that treat
all of them as such.  For example,
\cite[p.~35]{Goodstein}\index{Goodstein, R. L.} presents one with 15
axioms, some of which in fact coincide with what we have chosen to call
definitions in \texttt{set.mm}.  In certain subsets of classical
propositional calculus, such as the intuitionist
fragment\index{intuitionism}, it can be shown that one cannot make do
with just $\rightarrow$ and $\neg$ but must treat additional connectives
as primitive in order for the system to make sense.\footnote{Two nice
systems that make the transition from intuitionistic and other weak
fragments to classical logic just by adding axioms are given in
\cite{Robinsont}\index{Robinson, T. Thacher}.}

\subsection{The Approach to Definitions in \texttt{set.mm}}

In set theory, recursive definitions define a newly introduced symbol in
terms of itself.  The justification of recursive definitions, using
several ``recursion theorems,'' is usually one of the first
sophisticated proofs a student encounters when learning set theory, and
there is a significant amount of implicit metalogic behind a recursive
definition even though the definition itself is typically simple to
state.  It is, however, possible to substitute one kind of complexity
for another.  We can eliminate the need for metalogical justification by
defining the operation directly with an explicit (but complicated)
expression, then deriving the recursive definition directly as a
theorem, using a recursion theorem ``in reverse.''  We do this in
\texttt{set.mm}, as follows.

In \texttt{set.mm} our goal was to introduce almost all definitions in
the form of two expressions connected by either $\leftrightarrow$ or
$=$, where the thing being defined does not appear on the right hand
side.  Quine calls this form ``a genuine or direct definition'' \cite[p.
174]{Quine}\index{Quine, Willard Van Orman}, which makes the definitions
very easy to eliminate and the metalogic\index{metalogic} needed to
justify them as simple as possible.

We achieved this goal in almost all cases in \texttt{set.mm}.
Sometimes this makes the definitions more complex and less
intuitive.  For example, the traditional way to define addition of
natural numbers is to define an operation called {\em
successor}\index{successor} (which means ``plus one'' and is denoted by
``${\rm suc}$''), then define addition recursively\index{recursive
definition} with the two definitions $n + 0 = n$ and $m + {\rm suc}\,n =
{\rm suc} (m + n)$.  Although this definition seems simple and obvious,
the method to eliminate the definition is not obvious:  in the second
part of the definition, addition is defined in terms of itself.  By
eliminating the definition, we don't mean repeatedly applying it to
specific $m$ and $n$ but rather showing the explicit, closed-form
set-theoretical expression that $m + n$ represents, that will work for
any $m$ and $n$ and that does not have a $+$ sign on its right-hand
side.  For a recursive definition like this not to be circular
(creative), there are some hidden, underlying assumptions we must make,
for example that the natural numbers have a certain kind of order.  In
\texttt{set.mm} we chose to start with the direct (though complex and
nonintuitive) definition then derive from it the standard recursive
definition.\footnote{The closed-form definition used in \texttt{set.mm}
for the addition operation on ordinals\index{ordinal
addition}\index{addition!of ordinals} (of which natural numbers are a
subset) is

\setbox\startprefix=\hbox{\tt \ \ df-oadd\ \$a\ }
\setbox\contprefix=\hbox{\tt \ \ \ \ \ \ \ \ \ \ \ \ \ }
\startm
\m{\vdash}\m{+_o}\m{=}\m{\{}\m{\langle}\m{\langle}\m{x}\m{,}\m{y}\m{\rangle}%
\m{,}\m{z}\m{\rangle}\m{|}\m{(}\m{(}\m{x}\m{\in}\m{{\rm On}}\m{\wedge}\m{y}\m{%
\in}\m{{\rm On}}\m{)}\m{\wedge}\m{z}\m{=}\m{(}\m{{\rm rec}}\m{(}\m{\{}\m{%
\langle}\m{w}\m{,}\m{v}\m{\rangle}\m{|}\m{v}\m{=}\m{{\rm suc}}\m{w}\m{\}}\m{,}%
\m{x}\m{)}\m{`}\m{y}\m{)}\m{)}\m{\}}
\endm

\noindent Here, the abstraction class of nested ordered pairs\index{abstraction
class!of nested ordered pairs} is defined by \texttt{df-oprab} in \texttt{set.mm},
and ${\rm rec}$ is a ``recursion operator''\index{recursion operator} with
the definition

\setbox\startprefix=\hbox{\tt \ \ df-rfg\ \$a\ }
\setbox\contprefix=\hbox{\tt \ \ \ \ \ \ \ \ \ \ \ \ }
\startm
\m{\vdash}\m{{\rm rec}}\m{(}\m{F}\m{,}\m{A}\m{)}\m{=}\m{\bigcup}\m{\{}\m{f}%
\m{|}\m{\exists}\m{x}\m{(}\m{x}\m{\in}\m{{\rm On}}\m{\wedge}\m{(}\m{f}\m{{%
\rm Fn}}\m{x}\m{\wedge}\m{\forall}\m{y}\m{(}\m{y}\m{\in}\m{x}\m{\rightarrow}%
\m{(}\m{f}\m{`}\m{y}\m{)}\m{=}\m{(}\m{\{}\m{\langle}\m{g}\m{,}\m{z}\m{\rangle}%
\m{|}\m{(}\m{(}\m{g}\m{=}\m{\varnothing}\m{\wedge}\m{z}\m{=}\m{A}\m{)}\m{\vee}%
\m{(}\m{\lnot}\m{(}\m{g}\m{=}\m{\varnothing}\m{\vee}\m{{\rm Lim}}\m{{\rm dom}}%
\m{g}\m{)}\m{\wedge}\m{z}\m{=}\m{(}\m{F}\m{`}\m{(}\m{g}\m{`}\m{\bigcup}\m{{%
\rm dom}}\m{g}\m{)}\m{)}\m{)}\m{\vee}\m{(}\m{{\rm Lim}}\m{{\rm dom}}\m{g}\m{%
\wedge}\m{z}\m{=}\m{\bigcup}\m{{\rm ran}}\m{g}\m{)}\m{)}\m{\}}\m{`}\m{(}\m{f}%
\m{\restriction}\m{y}\m{)}\m{)}\m{)}\m{)}\m{)}\m{\}}
\endm

\noindent which can be further broken down with definitions shown in
Section~\ref{setdefinitions}.  You may be surprised at the complexity of what
seems like such a simple notion.  From these definitions the simpler, more
intuitive recursive definition is derived as a set of theorems.}\index{natural
number}\index{addition}\index{recursive definition}\index{ordinal addition}
The end result is the same, but we completely eliminate the rather complex
metalogic that justifies the recursive definition.  (For a mathematician,
recursive definitions are more efficient and intuitive than
direct ones once the metalogic has been learned or possibly just accepted as
correct.  However, it was felt that direct definition in \texttt{set.mm}
maximizes rigor by minimizing metalogic.  It can be eliminated effortlessly,
something difficult to do with a recursive definition.)

\subsection{Adding Constraints on Definitions}

The basic Metamath language and the Metamath program do
not have any built-in constraints on definitions, since they are just
\$a statements.

However, nothing prevents a verification system from verifying
additional rules to impose further limitations on definitions.
For example, the \texttt{mmj2}\index{mmj2} program
supports various kinds of
additional information comments (see section \ref{jcomment}).
One of their uses is to optionally verify additional constraints,
including constraints to verify that definitions meet certain
requirements.
These additional checks are required by the
continuous integration (CI)\index{continuous integration (CI)}
checks of the
\texttt{set.mm}\index{set theory database (\texttt{set.mm})}%
\index{Metamath Proof Explorer}
database.
This approach enables us to optionally impose additional requirements
on definitions if we wish, without necessarily imposing those rules on
all databases or requiring all verification systems to implement them.
In addition, this allows us to impose specialized constraints tailored
to one database while not requiring other systems to implement
those specialized constraints.

We impose two constraints on the
\texttt{set.mm}\index{set theory database (\texttt{set.mm})}%
\index{Metamath Proof Explorer} database
via the \texttt{mmj2}\index{mmj2} program that are worth discussing here:
a parse check and a definition soundness check.

% On February 11, 2019 8:32:32 PM EST, saueran@oregonstate.edu wrote:
% The following addition to the end of set.mm is accepted by the mmj2
% parser and definition checker and the metamath verifier(at least it was
% when I checked, you should check it too), and creates a contradiction by
% proving the theorem |- ph.
% ${
% wleftp $a wff ( ( ph ) $.
% wbothp $a wff ( ph ) $.
% df-leftp $a |- ( ( ( ph ) <-> -. ph ) $.
% df-bothp $a |- ( ( ph ) <-> ph ) $.
% anything $p |- ph $=
%   ( wbothp wn wi wleftp df-leftp biimpi df-bothp mpbir mpbi simplim ax-mp)
%   ABZAMACZDZCZMOEZOCQAEZNDZRNAFGSHIOFJMNKLAHJ $.
% $}
%
% This particular problem is countered by enabling, within mmj2,
% SetParser,mmj.verify.LRParser

First,
we enable a parse check in \texttt{mmj2} (through its
\texttt{SetParser} command) that requires that all new definitions
must \textit{not} create an ambiguous parse for a KLR(5) parser.
This prevents some errors such as definitions with imbalanced parentheses.

Second, we run a definition soundness check specific to
\texttt{set.mm} or databases similar to it.
(through the \texttt{definitionCheck} macro).
Some \texttt{\$a} statements (including all ax-* statemnets)
are excluded from these checks, as they will
always fail this simple check,
but they are appropriate for most definitions.
This check imposes a set of additional rules:

\begin{enumerate}

\item New definitions must be introduced using $=$ or $\leftrightarrow$.

\item No \texttt{\$a} statement introduced before this one is allowed to use the
symbol being defined in this definition, and the definition is not
allowed to use itself (except once, in the definiendum).

\item Every variable in the definiens should not be distinct

\item Every dummy variable in the definiendum
are required to be distinct from each other and from variables in
the definiendum.
To determine this, the system will look for a "justification" theorem
in the database, and if it is not there, attempt to briefly prove
$( \varphi \rightarrow \forall x \varphi )$  for each dummy variable x.

\item Every dummy variable should be a set variable,
unless there is a justification theorem available.

\item Every dummy variable must be bound
(if the system cannot determine this a justification theorem must be
provided).

\end{enumerate}

\subsection{Summary of Approach to Definitions}

In short, when being rigorous it turns out that
definitions can be subtle, sometimes requiring difficult
metatheorems to establish that they are not creative.

Instead of building such complications into the Metamath language itself,
the basic Metmath language and program simply treat traditional
axioms and definitions as the same kind of \texttt{\$a} statement.
We have then built various tools to enable people to
verify additional conditions as their creators believe is appropriate
for those specific databases, without complicating the Metamath foundations.

\chapter{The Metamath Program}\label{commands}

This chapter provides a reference manual for the
Metamath program.\index{Metamath!commands}

Current instructions for obtaining and installing the Metamath program
can be found at the \url{http://metamath.org} web site.
For Windows, there is a pre-compiled version called
\texttt{metamath.exe}.  For Unix, Linux, and Mac OS X
(which we will refer to collectively as ``Unix''), the Metamath program
can be compiled from its source code with the command
\begin{verbatim}
gcc *.c -o metamath
\end{verbatim}
using the \texttt{gcc} {\sc c} compiler available on those systems.

In the command syntax descriptions below, fields enclosed in square
brackets [\ ] are optional.  File names may be optionally enclosed in
single or double quotes.  This is useful if the file name contains
spaces or
slashes (\texttt{/}), such as in Unix path names, \index{Unix file
names}\index{file names!Unix} that might be confused with Metamath
command qualifiers.\index{Unix file names}\index{file names!Unix}

\section{Invoking Metamath}

Unix, Linux, and Mac OS X
have a command-line interface called the {\em
bash shell}.  (In Mac OS X, select the Terminal application from
Applications/Utilities.)  To invoke Metamath from the bash shell prompt,
assuming that the Metamath program is in the current directory, type
\begin{verbatim}
bash$ ./metamath
\end{verbatim}

To invoke Metamath from a Windows DOS or Command Prompt, assuming that
the Metamath program is in the current directory (or in a directory
included in the Path system environment variable), type
\begin{verbatim}
C:\metamath>metamath
\end{verbatim}

To use command-line arguments at invocation, the command-line arguments
should be a list of Metamath commands, surrounded by quotes if they
contain spaces.  In Windows, the surrounding quotes must be double (not
single) quotes.  For example, to read the database file \texttt{set.mm},
verify all proofs, and exit the program, type (under Unix)
\begin{verbatim}
bash$ ./metamath 'read set.mm' 'verify proof *' exit
\end{verbatim}
Note that in Unix, any directory path with \texttt{/}'s must be
surrounded by quotes so Metamath will not interpret the \texttt{/} as a
command qualifier.  So if \texttt{set.mm} is in the \texttt{/tmp}
directory, use for the above example
\begin{verbatim}
bash$ ./metamath 'read "/tmp/set.mm"' 'verify proof *' exit
\end{verbatim}

For convenience, if the command-line has one argument and no spaces in
the argument, the command is implicitly assumed to be \texttt{read}.  In
this one special case, \texttt{/}'s are not interpreted as command
qualifiers, so you don't need quotes around a Unix file name.  Thus
\begin{verbatim}
bash$ ./metamath /tmp/set.mm
\end{verbatim}
and
\begin{verbatim}
bash$ ./metamath "read '/tmp/set.mm'"
\end{verbatim}
are equivalent.


\section{Controlling Metamath}

The Metamath program was first developed on a {\sc vax/vms} system, and
some aspects of its command line behavior reflect this heritage.
Hopefully you will find it reasonably user-friendly once you get used to
it.

Each command line is a sequence of English-like words separated by
spaces, as in \texttt{show settings}.  Command words are not case
sensitive, and only as many letters are needed as are necessary to
eliminate ambiguity; for example, \texttt{sh se} would work for the
command \texttt{show settings}.  In some cases arguments such as file
names, statement labels, or symbol names are required; these are
case-sensitive (although file names may not be on some operating
systems).

A command line is entered by typing it in then pressing the {\em return}
({\em enter}) key.  To find out what commands are available, type
\texttt{?} at the \texttt{MM>} prompt.  To find out the choices at any
point in a command, press {\em return} and you will be prompted for
them.  The default choice (the one selected if you just press {\em
return}) is shown in brackets (\texttt{<>}).

You may also type \texttt{?} in place of a command word to force
Metamath to tell you what the choices are.  The \texttt{?} method won't
work, though, if a non-keyword argument such as a file name is expected
at that point, because the program will think that \texttt{?} is the
value of the argument.

Some commands have one or more optional qualifiers which modify the
behavior of the command.  Qualifiers are preceded by a slash
(\texttt{/}), such as in \texttt{read set.mm / verify}.  Spaces are
optional around the \texttt{/}.  If you need to use a space or
slash in a command
argument, as in a Unix file name, put single or double quotes around the
command argument.

The \texttt{open log} command will save everything you see on the
screen and is useful to help you recover should something go wrong in a
proof, or if you want to document a bug.

If a command responds with more than a screenful, you will be
prompted to \texttt{<return> to continue, Q to quit, or S to scroll to
end}.  \texttt{Q} or \texttt{q} (not case-sensitive) will complete the
command internally but will suppress further output until the next
\texttt{MM>} prompt.  \texttt{s} will suppress further pausing until the
next \texttt{MM>} prompt.  After the first screen, you are also
presented with the choice of \texttt{b} to go back a screenful.  Note
that \texttt{b} may also be entered at the \texttt{MM>} prompt
immediately after a command to scroll back through the output of that
command.

A command line enclosed in quotes is executed by your operating system.
See Section~\ref{oscmd}.

{\em Warning:} Pressing {\sc ctrl-c} will abort the Metamath program
unconditionally.  This means any unsaved work will be lost.


\subsection{\texttt{exit} Command}\index{\texttt{exit} command}

Syntax:  \texttt{exit} [\texttt{/force}]

This command exits from Metamath.  If there have been changes to the
source with the \texttt{save proof} or \texttt{save new{\char`\_}proof}
commands, you will be given an opportunity to \texttt{write source} to
permanently save the changes.

In Proof Assistant\index{Proof Assistant} mode, the \texttt{exit} command will
return to the \verb/MM>/ prompt. If there were changes to the proof, you will
be given an opportunity to \texttt{save new{\char`\_}proof}.

The \texttt{quit} command is a synonym for \texttt{exit}.

Optional qualifier:
    \texttt{/force} - Do not prompt if changes were not saved.  This qualifier is
        useful in \texttt{submit} command files (Section~\ref{sbmt})
        to ensure predictable behavior.





\subsection{\texttt{open log} Command}\index{\texttt{open log} command}
Syntax:  \texttt{open log} {\em file-name}

This command will open a log file that will store everything you see on
the screen.  It is useful to help recovery from a mistake in a long Proof
Assistant session, or to document bugs.\index{Metamath!bugs}

The log file can be closed with \texttt{close log}.  It will automatically be
closed upon exiting Metamath.



\subsection{\texttt{close log} Command}\index{\texttt{close log} command}
Syntax:  \texttt{close log}

The \texttt{close log} command closes a log file if one is open.  See
also \texttt{open log}.




\subsection{\texttt{submit} Command}\index{\texttt{submit} command}\label{sbmt}
Syntax:  \texttt{submit} {\em filename}

This command causes further command lines to be taken from the specified
file.  Note that any line beginning with an exclamation point (\texttt{!}) is
treated as a comment (i.e.\ ignored).  Also note that the scrolling
of the screen output is continuous, so you may want to open a log file
(see \texttt{open log}) to record the results that fly by on the screen.
After the lines in the file are exhausted, Metamath returns to its
normal user interface mode.

Currently, the \texttt{submit} command is not recursive.  In other words,
\texttt{submit} commands are not allowed inside a command file.


\subsection{\texttt{erase} Command}\index{\texttt{erase} command}
Syntax:  \texttt{erase}

This command will reset Metamath to its starting state, deleting any
data\-base that was \texttt{read} in.
 If there have been changes to the
source with the \texttt{save proof} or \texttt{save new{\char`\_}proof}
commands, you will be given an opportunity to \texttt{write source} to
permanently save the changes.



\subsection{\texttt{set echo} Command}\index{\texttt{set echo} command}
Syntax:  \texttt{set echo on} or \texttt{set echo off}

The \texttt{set echo on} command will cause command lines to be echoed with any
abbreviations expanded.  While learning the Metamath commands, this
feature will show you the exact command that your abbreviated input
corresponds to.



\subsection{\texttt{set scroll} Command}\index{\texttt{set scroll} command}
Syntax:  \texttt{set scroll prompted} or \texttt{set scroll continuous}

The Metamath command line interface starts off in the \texttt{prompted} mode,
which means that you will be prompted to continue or quit after each
full screen in a long listing.  In \texttt{continuous} mode, long listings will be
scrolled without pausing.

% LaTeX bug? (1) \texttt{\_} puts out different character than
% \texttt{{\char`\_}}
%  = \verb$_$  (2) \texttt{{\char`\_}} puts out garbage in \subsection
%  argument
\subsection{\texttt{set width} Command}\index{\texttt{set
width} command}
Syntax:  \texttt{set width} {\em number}

Metamath assumes the width of your screen is 79 characters (chosen
because the Command Prompt in Windows XP has a wrapping bug at column
80).  If your screen is wider or narrower, this command allows you to
change this default screen width.  A larger width is advantageous for
logging proofs to an output file to be printed on a wide printer.  A
smaller width may be necessary on some terminals; in this case, the
wrapping of the information messages may sometimes seem somewhat
unnatural, however.  In \LaTeX\index{latex@{\LaTeX}!characters per line},
there is normally a maximum of 61 characters per line with typewriter
font.  (The examples in this book were produced with 61 characters per
line.)

\subsection{\texttt{set height} Command}\index{\texttt{set
height} command}
Syntax:  \texttt{set height} {\em number}

Metamath assumes your screen height is 24 lines of characters.  If your
screen is taller or shorter, this command lets you to change the number
of lines at which the display pauses and prompts you to continue.

\subsection{\texttt{beep} Command}\index{\texttt{beep} command}

Syntax:  \texttt{beep}

This command will produce a beep.  By typing it ahead after a
long-running command has started, it will alert you that the command is
finished.  For convenience, \texttt{b} is an abbreviation for
\texttt{beep}.

Note:  If \texttt{b} is typed at the \texttt{MM>} prompt immediately
after the end of a multiple-page display paged with ``\texttt{Press
<return> for more}...'' prompts, then the \texttt{b} will back up to the
previous page rather than perform the \texttt{beep} command.
In that case you must type the unabbreviated \texttt{beep} form
of the command.

\subsection{\texttt{more} Command}\index{\texttt{more} command}

Syntax:  \texttt{more} {\em filename}

This command will display the contents of an {\sc ascii} file on your
screen.  (This command is provided for convenience but is not very
powerful.  See Section~\ref{oscmd} to invoke your operating system's
command to do this, such as the \texttt{more} command in Unix.)

\subsection{Operating System Commands}\index{operating system
command}\label{oscmd}

A line enclosed in single or double quotes will be executed by your
computer's operating system if it has a command line interface.  For
example, on a {\sc vax/vms} system,
\verb/MM> 'dir'/
will print disk directory contents.  Note that this feature will not
work on the Macintosh prior to Mac OS X, which does not have a command
line interface.

For your convenience, the trailing quote is optional.

\subsection{Size Limitations in Metamath}

In general, there are no fixed, predefined limits\index{Metamath!memory
limits} on how many labels, tokens\index{token}, statements, etc.\ that
you may have in a database file.  The Metamath program uses 32-bit
variables (64-bit on 64-bit CPUs) as indices for almost all internal
arrays, which are allocated dynamically as needed.



\section{Reading and Writing Files}

The following commands create new files:  the \texttt{open} commands;
the \texttt{write} commands; the \texttt{/html},
\texttt{/alt{\char`\_}html}, \texttt{/brief{\char`\_}html},
\texttt{/brief{\char`\_}alt{\char`\_}html} qualifiers of \texttt{show
statement}, and \texttt{midi}.  The following commands append to files
previously opened:  the \texttt{/tex} qualifier of \texttt{show proof}
and \texttt{show new{\char`\_}proof}; the \texttt{/tex} and
\texttt{/simple{\char`\_}tex} qualifiers of \texttt{show statement}; the
\texttt{close} commands; and all screen dialog between \texttt{open log}
and \texttt{close log}.

The commands that create new files will not overwrite an existing {\em
filename} but will rename the existing one to {\em
filename}\texttt{{\char`\~}1}.  An existing {\em
filename}\texttt{{\char`\~}1} is renamed {\em
filename}\texttt{{\char`\~}2}, etc.\ up to {\em
filename}\texttt{{\char`\~}9}.  An existing {\em
filename}\texttt{{\char`\~}9} is deleted.  This makes recovery from
mistakes easier but also will clutter up your directory, so occasionally
you may want to clean up (delete) these \texttt{{\char`\~}}$n$ files.


\subsection{\texttt{read} Command}\index{\texttt{read} command}
Syntax:  \texttt{read} {\em file-name} [\texttt{/verify}]

This command will read in a Metamath language source file and any included
files.  Normally it will be the first thing you do when entering Metamath.
Statement syntax is checked, but proof syntax is not checked.
Note that the file name may be enclosed in single or double quotes;
this is useful if the file name contains slashes, as might be the case
under Unix.

If you are getting an ``\texttt{?Expected VERIFY}'' error
when trying to read a Unix file name with slashes, you probably haven't
quoted it.\index{Unix file names}\index{file names!Unix}

If you are prompted for the file name (by pressing {\em return}
 after \texttt{read})
you should {\em not} put quotes around it, even if it is a Unix file name
with slashes.

Optional command qualifier:

    \texttt{/verify} - Verify all proofs as the database is read in.  This
         qualifier will slow down reading in the file.  See \texttt{verify
         proof} for more information on file error-checking.

See also \texttt{erase}.



\subsection{\texttt{write source} Command}\index{\texttt{write source} command}
Syntax:  \texttt{write source} {\em filename}
[\texttt{/rewrap}]
[\texttt{/split}]
[\texttt{/keep\_includes}]
[\texttt{/no\_versioning}]

This command will write the contents of a Metamath\index{database}
database into a file.\index{source file}

Optional command qualifiers:

\texttt{/rewrap} -
Reformats statements and comments according to the
convention used in the set.mm database.
It unwraps the
lines in the comment before each \texttt{\$a} and \texttt{\$p} statement,
then it
rewraps the line.  You should compare the output to the original
to make sure that the desired effect results; if not, go back to
the original source.  The wrapped line length honors the
\texttt{set width}
parameter currently in effect.  Note:  Text
enclosed in \texttt{<HTML>}...\texttt{</HTML>} tags is not modified by the
\texttt{/rewrap} qualifier.
Proofs themselves are not reformatted;
use \texttt{save proof * / compressed} to do that.
An isolated tilde (\~{}) is always kept on the same line as the following
symbol, so you can find all comment references to a symbol by
searching for \~{} followed by a space and that symbol
(this is useful for finding cross-references).
Incidentally, \texttt{save proof} also honors the \texttt{set width}
parameter currently in effect.

\texttt{/split} - Files included in the source using the expression
\$[ \textit{inclfile} \$] will be
written into separate files instead of being included in a single output
file.  The name of each separately written included file will be the
\textit{inclfile} argument of its inclusion command.

\texttt{/keep\_includes} - If a source file has includes but is written as a
single file by omitting \texttt{/split}, by default the included files will
be deleted (actually just renamed with a \char`\~1 suffix unless
\texttt{/no\_versioning} is specified) to prevent the possibly confusing
source duplication in both the output file and the included file.
The \texttt{/keep\_includes} qualifier will prevent this deletion.

\texttt{/no\_versioning} - Backup files suffixed with \char`\~1 are not created.


\section{Showing Status and Statements}



\subsection{\texttt{show settings} Command}\index{\texttt{show settings} command}
Syntax:  \texttt{show settings}

This command shows the state of various parameters.

\subsection{\texttt{show memory} Command}\index{\texttt{show memory} command}
Syntax:  \texttt{show memory}

This command shows the available memory left.  It is not meaningful
on most modern operating systems,
which have virtual memory.\index{Metamath!memory usage}


\subsection{\texttt{show labels} Command}\index{\texttt{show labels} command}
Syntax:  \texttt{show labels} {\em label-match} [\texttt{/all}]
   [\texttt{/linear}]

This command shows the labels of \texttt{\$a} and \texttt{\$p}
statements that match {\em label-match}.  A \verb$*$ in {label-match}
matches zero or more characters.  For example, \verb$*abc*def$ will match all
labels containing \verb$abc$ and ending with \verb$def$.

Optional command qualifiers:

   \texttt{/all} - Include matches for \texttt{\$e} and \texttt{\$f}
   statement labels.

   \texttt{/linear} - Display only one label per line.  This can be useful for
       building scripts in conjunction with the utilities under the
       \texttt{tools}\index{\texttt{tools} command} command.



\subsection{\texttt{show statement} Command}\index{\texttt{show statement} command}
Syntax:  \texttt{show statement} {\em label-match} [{\em qualifiers} (see below)]

This command provides information about a statement.  Only statements
that have labels (\texttt{\$f}\index{\texttt{\$f} statement},
\texttt{\$e}\index{\texttt{\$e} statement},
\texttt{\$a}\index{\texttt{\$a} statement}, and
\texttt{\$p}\index{\texttt{\$p} statement}) may be specified.
If {\em label-match}
contains wildcard (\verb$*$) characters, all matching statements will be
displayed in the order they occur in the database.

Optional qualifiers (only one qualifier at a time is allowed):

    \texttt{/comment} - This qualifier includes the comment that immediately
       precedes the statement.

    \texttt{/full} - Show complete information about each statement,
       and show all
       statements matching {\em label} (including \texttt{\$e}
       and \texttt{\$f} statements).

    \texttt{/tex} - This qualifier will write the statement information to the
       \LaTeX\ file previously opened with \texttt{open tex}.  See
       Section~\ref{texout}.

    \texttt{/simple{\char`\_}tex} - The same as \texttt{/tex}, except that
       \LaTeX\ macros are not used for formatting equations, allowing easier
       manual edits of the output for slide presentations, etc.

    \texttt{/html}\index{html generation@{\sc html} generation},
       \texttt{/alt{\char`\_}html}, \texttt{/brief{\char`\_}html},
       \texttt{/brief{\char`\_}alt{\char`\_}html} -
       These qualifiers invoke a special mode of
       \texttt{show statement} that
       creates a web page for the statement.  They may not be used with
       any other qualifier.  See Section~\ref{htmlout} or
       \texttt{help html} in the program.


\subsection{\texttt{search} Command}\index{\texttt{search} command}
Syntax:  search {\em label-match}
\texttt{"}{\em symbol-match}{\tt}" [\texttt{/all}] [\texttt{/comments}]
[\texttt{/join}]

This command searches all \texttt{\$a} and \texttt{\$p} statements
matching {\em label-match} for occurrences of {\em symbol-match}.  A
\verb@*@ in {\em label-match} matches any label character.  A \verb@$*@
in {\em symbol-match} matches any sequence of symbols.  The symbols in
{\em symbol-match} must be separated by white space.  The quotes
surrounding {\em symbol-match} may be single or double quotes.  For
example, \texttt{search b}\verb@* "-> $* ch"@ will list all statements
whose labels begin with \texttt{b} and contain the symbols \verb@->@ and
\texttt{ch} surrounding any symbol sequence (including no symbol
sequence).  The wildcards \texttt{?} and \texttt{\$?} are also available
to match individual characters in labels and symbols respectively; see
\texttt{help search} in the Metamath program for details on their usage.

Optional command qualifiers:

    \texttt{/all} - Also search \texttt{\$e} and \texttt{\$f} statements.

    \texttt{/comments} - Search the comment that immediately precedes each
        label-matched statement for {\em symbol-match}.  In this case
        {\em symbol-match} is an arbitrary, non-case-sensitive character
        string.  Quotes around {\em symbol-match} are optional if there
        is no ambiguity.

    \texttt{/join} - In the case of a \texttt{\$a} or \texttt{\$p} statement,
	prepend its \texttt{\$e}
	hypotheses for searching. The
	\texttt{/join} qualifier has no effect in \texttt{/comments} mode.

\section{Displaying and Verifying Proofs}


\subsection{\texttt{show proof} Command}\index{\texttt{show proof} command}
Syntax:  \texttt{show proof} {\em label-match} [{\em qualifiers} (see below)]

This command displays the proof of the specified
\texttt{\$p}\index{\texttt{\$p} statement} statement in various formats.
The {\em label-match} may contain wildcard (\verb@$*@) characters to match
multiple statements.  Without any qualifiers, only the logical steps
will be shown (i.e.\ syntax construction steps will be omitted), in an
indented format.

Most of the time, you will use
    \texttt{show proof} {\em label}
to see just the proof steps corresponding to logical inferences.

Optional command qualifiers:

    \texttt{/essential} - The proof tree is trimmed of all
        \texttt{\$f}\index{\texttt{\$f} statement} hypotheses before
        being displayed.  (This is the default, and it is redundant to
        specify it.)

    \texttt{/all} - the proof tree is not trimmed of all \texttt{\$f} hypotheses before
        being displayed.  \texttt{/essential} and \texttt{/all} are mutually exclusive.

    \texttt{/from{\char`\_}step} {\em step} - The display starts at the specified
        step.  If
        this qualifier is omitted, the display starts at the first step.

    \texttt{/to{\char`\_}step} {\em step} - The display ends at the specified
        step.  If this
        qualifier is omitted, the display ends at the last step.

    \texttt{/tree{\char`\_}depth} {\em number} - Only
         steps at less than the specified proof
        tree depth are displayed.  Sometimes useful for obtaining an overview of
        the proof.

    \texttt{/reverse} - The steps are displayed in reverse order.

    \texttt{/renumber} - When used with \texttt{/essential}, the steps are renumbered
        to correspond only to the essential steps.

    \texttt{/tex} - The proof is converted to \LaTeX\ and\index{latex@{\LaTeX}}
        stored in the file opened
        with \texttt{open tex}.  See Section~\ref{texout} or
        \texttt{help tex} in the program.

    \texttt{/lemmon} - The proof is displayed in a non-indented format known
        as Lemmon style, with explicit previous step number references.
        If this qualifier is omitted, steps are indented in a tree format.

    \texttt{/start{\char`\_}column} {\em number} - Overrides the default column
        (16)
        at which the formula display starts in a Lemmon-style display.  May be
        used only in conjunction with \texttt{/lemmon}.

    \texttt{/normal} - The proof is displayed in normal format suitable for
        inclusion in a Metamath source file.  May not be used with any other
        qualifier.

    \texttt{/compressed} - The proof is displayed in compressed format
        suitable for inclusion in a Metamath source file.  May not be used with
        any other qualifier.

    \texttt{/statement{\char`\_}summary} - Summarizes all statements (like a
        brief \texttt{show statement})
        used by the proof.  It may not be used with any other qualifier
        except \texttt{/essential}.

    \texttt{/detailed{\char`\_}step} {\em step} - Shows the details of what is
        happening at
        a specific proof step.  May not be used with any other qualifier.
        The {\em step} is the step number shown when displaying a
        proof without the \texttt{/renumber} qualifier.


\subsection{\texttt{show usage} Command}\index{\texttt{show usage} command}
Syntax:  \texttt{show usage} {\em label-match} [\texttt{/recursive}]

This command lists the statements whose proofs make direct reference to
the statement specified.

Optional command qualifier:

    \texttt{/recursive} - Also include statements whose proofs ultimately
        depend on the statement specified.



\subsection{\texttt{show trace\_back} Command}\index{\texttt{show
       trace{\char`\_}back} command}
Syntax:  \texttt{show trace{\char`\_}back} {\em label-match} [\texttt{/essential}] [\texttt{/axioms}]
    [\texttt{/tree}] [\texttt{/depth} {\em number}]

This command lists all statements that the proof of the \texttt{\$p}
statement(s) specified by {\em label-match} depends on.

Optional command qualifiers:

    \texttt{/essential} - Restrict the trace-back to \texttt{\$e}
        \index{\texttt{\$e} statement} hypotheses of proof trees.

    \texttt{/axioms} - List only the axioms that the proof ultimately depends on.

    \texttt{/tree} - Display the trace-back in an indented tree format.

    \texttt{/depth} {\em number} - Restrict the \texttt{/tree} trace-back to the
        specified indentation depth.

    \texttt{/count{\char`\_}steps} - Count the number of steps the proof has
       all the way back to axioms.  If \texttt{/essential} is specified,
       expansions of variable-type hypotheses (syntax constructions) are not counted.

\subsection{\texttt{verify proof} Command}\index{\texttt{verify proof} command}
Syntax:  \texttt{verify proof} {\em label-match} [\texttt{/syntax{\char`\_}only}]

This command verifies the proofs of the specified statements.  {\em
label-match} may contain wild card characters (\texttt{*}) to verify
more than one proof; for example \verb/*abc*def/ will match all labels
containing \texttt{abc} and ending with \texttt{def}.
The command \texttt{verify proof *} will verify all proofs in the database.

Optional command qualifier:

    \texttt{/syntax{\char`\_}only} - This qualifier will perform a check of syntax
        and RPN
        stack violations only.  It will not verify that the proof is
        correct.  This qualifier is useful for quickly determining which
        proofs are incomplete (i.e.\ are under development and have \texttt{?}'s
        in them).

{\em Note:} \texttt{read}, followed by \texttt{verify proof *}, will ensure
 the database is free
from errors in the Metamath language but will not check the markup notation
in comments.
You can also check the markup notation by running \texttt{verify markup *}
as discussed in Section~\ref{verifymarkup}; see also the discussion
on generating {\sc HTML} in Section~\ref{htmlout}.

\subsection{\texttt{verify markup} Command}\index{\texttt{verify markup} command}\label{verifymarkup}
Syntax:  \texttt{verify markup} {\em label-match}
[\texttt{/date{\char`\_}skip}]
[\texttt{/top{\char`\_}date{\char`\_}skip}]
[\texttt{/file{\char`\_}skip}]
[\texttt{/verbose}]

This command checks comment markup and other informal conventions we have
adopted.  It error-checks the latexdef, htmldef, and althtmldef statements
in the \texttt{\$t} statement of a Metamath source file.\index{error checking}
It error-checks any \texttt{`}...\texttt{`},
\texttt{\char`\~}~\textit{label},
and bibliographic markups in statement descriptions.
It checks that
\texttt{\$p} and \texttt{\$a} statements
have the same content when their labels start with
``ax'' and ``ax-'' respectively but are otherwise identical, for example
ax4 and ax-4.
It also verifies the date consistency of ``(Contributed by...),''
``(Revised by...),'' and ``(Proof shortened by...)'' tags in the comment
above each \texttt{\$a} and \texttt{\$p} statement.

Optional command qualifiers:

    \texttt{/date{\char`\_}skip} - This qualifier will
        skip date consistency checking,
        which is usually not required for databases other than
	\texttt{set.mm}.

    \texttt{/top{\char`\_}date{\char`\_}skip} - This qualifier will check date consistency except
        that the version date at the top of the database file will not
        be checked.  Only one of
        \texttt{/date{\char`\_}skip} and
        \texttt{/top{\char`\_}date{\char`\_}skip} may be
        specified.

    \texttt{/file{\char`\_}skip} - This qualifier will skip checks that require
        external files to be present, such as checking GIF existence and
        bibliographic links to mmset.html or equivalent.  It is useful
        for doing a quick check from a directory without these files.

    \texttt{/verbose} - Provides more information.  Currently it provides a list
        of axXXX vs. ax-XXX matches.

\subsection{\texttt{save proof} Command}\index{\texttt{save proof} command}
Syntax:  \texttt{save proof} {\em label-match} [\texttt{/normal}]
   [\texttt{/compressed}]

The \texttt{save proof} command will reformat a proof in one of two formats and
replace the existing proof in the source buffer\index{source
buffer}.  It is useful for
converting between proof formats.  Note that a proof will not be
permanently saved until a \texttt{write source} command is issued.

Optional command qualifiers:

    \texttt{/normal} - The proof is saved in the normal format (i.e., as a
        sequence
        of labels, which is the defined format of the basic Metamath
        language).\index{basic language}  This is the default format that
        is used if a qualifier
        is omitted.

    \texttt{/compressed} - The proof is saved in the compressed format which
        reduces storage requirements for a database.
        See Appendix~\ref{compressed}.




\section{Creating Proofs}\label{pfcommands}\index{Proof Assistant}

Before using the Proof Assistant, you must add a \texttt{\$p} to your
source file (using a text editor) containing the statement you want to
prove.  Its proof should consist of a single \texttt{?}, meaning
``unknown step.''  Example:
\begin{verbatim}
equid $p x = x $= ? $.
\end{verbatim}

To enter the Proof assistant, type \texttt{prove} {\em label}, e.g.
\texttt{prove equid}.  Metamath will respond with the \texttt{MM-PA>}
prompt.

Proofs are created working backwards from the statement being proved,
primarily using a series of \texttt{assign} commands.  A proof is
complete when all steps are assigned to statements and all steps are
unified and completely known.  During the creation of a proof, Metamath
will allow only operations that are legal based on what is known up to
that point.  For example, it will not allow an \texttt{assign} of a
statement that cannot be unified with the unknown proof step being
assigned.

{\em Important:}
The Proof Assistant is
{\em not} a tool to help you discover proofs.  It is just a tool to help
you add them to the database.  For a tutorial read
Section~\ref{frstprf}.
To practice using the Proof Assistant, you may
want to \texttt{prove} an existing theorem, then delete all steps with
\texttt{delete all}, then re-create it with the Proof Assistant while
looking at its proof display (before deletion).
You might want to figure out your first few proofs completely
and write them down by hand, before using the Proof Assistant, though
not everyone finds that effective.

{\em Important:}
The \texttt{undo} command if very helpful when entering a proof, because
it allows you to undo a previously-entered step.
In addition, we suggest that you
keep track of your work with a log file (\texttt{open
log}) and save it frequently (\texttt{save new{\char`\_}proof},
\texttt{write source}).
You can use \texttt{delete} to reverse an \texttt{assign}.
You can also do \texttt{delete floating{\char`\_}hypotheses}, then
\texttt{initialize all}, then \texttt{unify all /interactive} to
reinitialize bad unifications made accidentally or by bad
\texttt{assign}s.  You cannot reverse a \texttt{delete} except by
a relevant \texttt{undo} or using
\texttt{exit /force} then reentering the Proof Assistant to recover from
the last \texttt{save new{\char`\_}proof}.

The following commands are available in the Proof Assistant (at the
\texttt{MM-PA>} prompt) to help you create your proof.  See the
individual commands for more detail.

\begin{itemize}
\item[]
    \texttt{show new{\char`\_}proof} [\texttt{/all},...] - Displays the
        proof in progress.  You will use this command a lot; see \texttt{help
        show new{\char`\_}proof} to become familiar with its qualifiers.  The
        qualifiers \texttt{/unknown} and \texttt{/not{\char`\_}unified} are
        useful for seeing the work remaining to be done.  The combination
        \texttt{/all/unknown} is useful for identifying dummy variables that must be
        assigned, or attempts to use illegal syntax, when \texttt{improve all}
        is unable to complete the syntax constructions.  Unknown variables are
        shown as \texttt{\$1}, \texttt{\$2},...
\item[]
    \texttt{assign} {\em step} {\em label} - Assigns an unknown {\em step}
        number with the statement
        specified by {\em label}.
\item[]
    \texttt{let variable} {\em variable}
        \texttt{= "}{\em symbol sequence}\texttt{"}
          - Forces a symbol
        sequence to replace an unknown variable (such as \texttt{\$1}) in a proof.
        It is useful
        for helping difficult unifications, and it is necessary when you have
        dummy variables that eventually must be assigned a name.
\item[]
    \texttt{let step} {\em step} \texttt{= "}{\em symbol sequence}\texttt{"} -
          Forces a symbol sequence
        to replace the contents of a proof step, provided it can be
        unified with the existing step contents.  (I rarely use this.)
\item[]
    \texttt{unify step} {\em step} (or \texttt{unify all}) - Unifies
        the source and target of
        a step.  If you specify a specific step, you will be prompted
        to select among the unifications that are possible.  If you
        specify \texttt{all}, all steps with unique unifications, but only
        those steps, will be
        unified.  \texttt{unify all /interactive} goes through all non-unified
        steps.
\item[]
    \texttt{initialize} {\em step} (or \texttt{all}) - De-unifies the target and source of
        a step (or all steps), as well as the hypotheses of the source,
        and makes all variables in the source unknown.  Useful to recover from
        an \texttt{assign} or \texttt{let} mistake that
        resulted in incorrect unifications.
\item[]
    \texttt{delete} {\em step} (or \texttt{all} or \texttt{floating{\char`\_}hypotheses}) -
        Deletes the specified
        step(s).  \texttt{delete floating{\char`\_}hypotheses}, then \texttt{initialize all}, then
        \texttt{unify all /interactive} is useful for recovering from mistakes
        where incorrect unifications assigned wrong math symbol strings to
        variables.
\item[]
    \texttt{improve} {\em step} (or \texttt{all}) -
      Automatically creates a proof for steps (with no unknown variables)
      whose proof requires no statements with \texttt{\$e} hypotheses.  Useful
      for filling in proofs of \texttt{\$f} hypotheses.  The \texttt{/depth}
      qualifier will also try statements whose \texttt{\$e} hypotheses contain
      no new variables.  {\em Warning:} Save your work (with \texttt{save
      new{\char`\_}proof} then \texttt{write source}) before using
      \texttt{/depth = 2} or greater, since the search time grows
      exponentially and may never terminate in a reasonable time, and you
      cannot interrupt the search.  I have found that it is rare for
      \texttt{/depth = 3} or greater to be useful.
 \item[]
    \texttt{save new{\char`\_}proof} - Saves the proof in progress in the program's
        internal database buffer.  To save it permanently into the database file,
        use \texttt{write source} after
        \texttt{save new{\char`\_}proof}.  To revert to the last
        \texttt{save new{\char`\_}proof},
        \texttt{exit /force} from the Proof Assistant then re-enter the Proof
        Assistant.
 \item[]
    \texttt{match step} {\em step} (or \texttt{match all}) - Shows what
        statements are
        possibilities for the \texttt{assign} statement. (This command
        is not very
        useful in its present form and hopefully will be improved
        eventually.  In the meantime, use the \texttt{search} statement for
        candidates matching specific math token combinations.)
 \item[]
 \texttt{minimize{\char`\_}with}\index{\texttt{minimize{\char`\_}with} command}
% 3/10/07 Note: line-breaking the above results in duplicate index entries
     - After a proof is complete, this command will attempt
        to match other database theorems to the proof to see if the proof
        size can be reduced as a result.  See \texttt{help
        minimize{\char`\_}with} in the
        Metamath program for its usage.
 \item[]
 \texttt{undo}\index{\texttt{undo} command}
    - Undo the effect of a proof-changing command (all but the
      \texttt{show} and \texttt{save} commands above).
 \item[]
 \texttt{redo}\index{\texttt{redo} command}
    - Reverse the previous \texttt{undo}.
\end{itemize}

The following commands set parameters that may be relevant to your proof.
Consult the individual \texttt{help set}... commands.
\begin{itemize}
   \item[] \texttt{set unification{\char`\_}timeout}
 \item[]
    \texttt{set search{\char`\_}limit}
  \item[]
    \texttt{set empty{\char`\_}substitution} - note that default is \texttt{off}
\end{itemize}

Type \texttt{exit} to exit the \texttt{MM-PA>}
 prompt and get back to the \texttt{MM>} prompt.
Another \texttt{exit} will then get you out of Metamath.



\subsection{\texttt{prove} Command}\index{\texttt{prove} command}
Syntax:  \texttt{prove} {\em label}

This command will enter the Proof Assistant\index{Proof Assistant}, which will
allow you to create or edit the proof of the specified statement.
The command-line prompt will change from \texttt{MM>} to \texttt{MM-PA>}.

Note:  In the present version of
Metamath\index{Metamath!limitations of version 0.07.30}, the Proof
Assistant does not verify that \texttt{\$d}\index{\texttt{\$d}
statement} restrictions are met as a proof is being built.  After you
have completed a proof, you should type \texttt{save new{\char`\_}proof}
followed by \texttt{verify proof} {\em label} (where {\em label} is the
statement you are proving with the \texttt{prove} command) to verify the
\texttt{\$d} restrictions.

See also: \texttt{exit}

\subsection{\texttt{set unification\_timeout} Command}\index{\texttt{set
unification{\char`\_}timeout} command}
Syntax:  \texttt{set unification{\char`\_}timeout} {\em number}

(This command is available outside the Proof Assistant but affects the
Proof Assistant\index{Proof Assistant} only.)

Sometimes the Proof Assistant will inform you that a unification
time-out occurred.  This may happen when you try to \texttt{unify}
formulas with many temporary variables\index{temporary variable}
(\texttt{\$1}, \texttt{\$2}, etc.), since the time to compute all possible
unifications may grow exponentially with the number of variables.  If
you want Metamath to try harder (and you're willing to wait longer) you
may increase this parameter.  \texttt{show settings} will show you the
current value.

\subsection{\texttt{set empty\_substitution} Command}\index{\texttt{set
empty{\char`\_}substitution} command}
% These long names can't break well in narrow mode, and even "sloppy"
% is not enough. Work around this by not demanding justification.
\begin{flushleft}
Syntax:  \texttt{set empty{\char`\_}substitution on} or \texttt{set
empty{\char`\_}substitution off}
\end{flushleft}

(This command is available outside the Proof Assistant but affects the
Proof Assistant\index{Proof Assistant} only.)

The Metamath language allows variables to be
substituted\index{substitution!variable}\index{variable substitution}
with empty symbol sequences\index{empty substitution}.  However, in many
formal systems\index{formal system} this will never happen in a valid
proof.  Allowing for this possibility increases the likelihood of
ambiguous unifications\index{ambiguous
unification}\index{unification!ambiguous} during proof creation.
The default is that
empty substitutions are not allowed; for formal systems requiring them,
you must \texttt{set empty{\char`\_}substitution on}.
(An example where this must be \texttt{on}
would be a system that implements a Deduction Rule and in
which deductions from empty assumption lists would be permissible.  The
MIU-system\index{MIU-system} described in Appendix~\ref{MIU} is another
example.)
Note that empty substitutions are
always permissible in proof verification (VERIFY PROOF...) outside the
Proof Assistant.  (See the MIU system in the Metamath book for an example
of a system needing empty substitutions; another example would be a
system that implements a Deduction Rule and in which deductions from
empty assumption lists would be permissible.)

It is better to leave this \texttt{off} when working with \texttt{set.mm}.
Note that this command does not affect the way proofs are verified with
the \texttt{verify proof} command.  Outside of the Proof Assistant,
substitution of empty sequences for math symbols is always allowed.

\subsection{\texttt{set search\_limit} Command}\index{\texttt{set
search{\char`\_}limit} command} Syntax:  \texttt{set search{\char`\_}limit} {\em
number}

(This command is available outside the Proof Assistant but affects the
Proof Assistant\index{Proof Assistant} only.)

This command sets a parameter that determines when the \texttt{improve} command
in Proof Assistant mode gives up.  If you want \texttt{improve} to search harder,
you may increase it.  The \texttt{show settings} command tells you its current
value.


\subsection{\texttt{show new\_proof} Command}\index{\texttt{show
new{\char`\_}proof} command}
Syntax:  \texttt{show new{\char`\_}proof} [{\em
qualifiers} (see below)]

This command (available only in Proof Assistant mode) displays the proof
in progress.  It is identical to the \texttt{show proof} command, except that
there is no statement argument (since it is the statement being proved) and
the following qualifiers are not available:

    \texttt{/statement{\char`\_}summary}

    \texttt{/detailed{\char`\_}step}

Also, the following additional qualifiers are available:

    \texttt{/unknown} - Shows only steps that have no statement assigned.

    \texttt{/not{\char`\_}unified} - Shows only steps that have not been unified.

Note that \texttt{/essential}, \texttt{/depth}, \texttt{/unknown}, and
\texttt{/not{\char`\_}unified} may be
used in any combination; each of them effectively filters out additional
steps from the proof display.

See also:  \texttt{show proof}






\subsection{\texttt{assign} Command}\index{\texttt{assign} command}
Syntax:   \texttt{assign} {\em step} {\em label} [\texttt{/no{\char`\_}unify}]

   and:   \texttt{assign first} {\em label}

   and:   \texttt{assign last} {\em label}


This command, available in the Proof Assistant only, assigns an unknown
step (one with \texttt{?} in the \texttt{show new{\char`\_}proof}
listing) with the statement specified by {\em label}.  The assignment
will not be allowed if the statement cannot be unified with the step.

If \texttt{last} is specified instead of {\em step} number, the last
step that is shown by \texttt{show new{\char`\_}proof /unknown} will be
used.  This can be useful for building a proof with a command file (see
\texttt{help submit}).  It also makes building proofs faster when you know
the assignment for the last step.

If \texttt{first} is specified instead of {\em step} number, the first
step that is shown by \texttt{show new{\char`\_}proof /unknown} will be
used.

If {\em step} is zero or negative, the -{\em step}th from last unknown
step, as shown by \texttt{show new{\char`\_}proof /unknown}, will be
used.  \texttt{assign -1} {\em label} will assign the penultimate
unknown step, \texttt{assign -2} {\em label} the antepenultimate, and
\texttt{assign 0} {\em label} is the same as \texttt{assign last} {\em
label}.

Optional command qualifier:

    \texttt{/no{\char`\_}unify} - do not prompt user to select a unification if there is
        more than one possibility.  This is useful for noninteractive
        command files.  Later, the user can \texttt{unify all /interactive}.
        (The assignment will still be automatically unified if there is only
        one possibility and will be refused if unification is not possible.)



\subsection{\texttt{match} Command}\index{\texttt{match} command}
Syntax:  \texttt{match step} {\em step} [\texttt{/max{\char`\_}essential{\char`\_}hyp}
{\em number}]

    and:  \texttt{match all} [\texttt{/essential}]
          [\texttt{/max{\char`\_}essential{\char`\_}hyp} {\em number}]

This command, available in the Proof Assistant only, shows what
statements can be unified with the specified step(s).  {\em Note:} In
its current form, this command is not very useful because of the large
number of matches it reports.  I rarely use it.
It may be enhanced in the future.

Optional command qualifiers:

    \texttt{/max{\char`\_}essential{\char`\_}hyp} {\em number} - filters out
        of the list any statements
        with more than the specified number of
        \texttt{\$e}\index{\texttt{\$e} statement} hypotheses

    \texttt{/essential{\char`\_}only} - in the \texttt{match all} statement, only
        the steps that
        would be listed in the \texttt{show new{\char`\_}proof /essential} display are
        matched.



\subsection{\texttt{let} Command}\index{\texttt{let} command}
Syntax: \texttt{let variable} {\em variable} = \verb/"/{\em symbol-sequence}\verb/"/

  and: \texttt{let step} {\em step} = \verb/"/{\em symbol-sequence}\verb/"/

These commands, available in the Proof Assistant\index{Proof Assistant}
only, assign a temporary variable\index{temporary variable} or unknown
step with a specific symbol sequence.  They are useful in the middle of
creating a proof, when you know what should be in the proof step but the
unification algorithm doesn't yet have enough information to completely
specify the temporary variables.  A ``temporary variable'' is one that
has the form \texttt{\$}{\em nn} in the proof display, such as
\texttt{\$1}, \texttt{\$2}, etc.  The {\em symbol-sequence} may contain
other unknown variables if desired.  Examples:

    \verb/let variable $32 = "A = B"/

    \verb/let variable $32 = "A = $35"/

    \verb/let step 10 = '|- x = x'/

    \verb/let step -2 = "|- ( $7 = ph )"/

Any symbol sequence will be accepted for the \texttt{let variable}
command.  Only those symbol sequences that can be unified with the step
will be accepted for \texttt{let step}.

The \texttt{let} commands ``zap'' the proof with information that can
only be verified when the proof is built up further.  If you make an
error, the command sequence \texttt{delete
floating{\char`\_}hypotheses}, \texttt{initialize all}, and
\texttt{unify all /interactive} will undo a bad \texttt{let} assignment.

If {\em step} is zero or negative, the -{\em step}th from last unknown
step, as shown by \texttt{show new{\char`\_}proof /unknown}, will be
used.  The command \texttt{let step 0} = \verb/"/{\em
symbol-sequence}\verb/"/ will use the last unknown step, \texttt{let
step -1} = \verb/"/{\em symbol-sequence}\verb/"/ the penultimate, etc.
If {\em step} is positive, \texttt{let step} may be used to assign known
(in the sense of having previously been assigned a label with
\texttt{assign}) as well as unknown steps.

Either single or double quotes can surround the {\em symbol-sequence} as
long as they are different from any quotes inside a {\em
symbol-sequence}.  If {\em symbol-sequence} contains both kinds of
quotes, see the instructions at the end of \texttt{help let} in the
Metamath program.


\subsection{\texttt{unify} Command}\index{\texttt{unify} command}
Syntax:  \texttt{unify step} {\em step}

      and:   \texttt{unify all} [\texttt{/interactive}]

These commands, available in the Proof Assistant only, unify the source
and target of the specified step(s). If you specify a specific step, you
will be prompted to select among the unifications that are possible.  If
you specify \texttt{all}, only those steps with unique unifications will be
unified.

Optional command qualifier for \texttt{unify all}:

    \texttt{/interactive} - You will be prompted to select among the
        unifications
        that are possible for any steps that do not have unique
        unifications.  (Otherwise \texttt{unify all} will bypass these.)

See also \texttt{set unification{\char`\_}timeout}.  The default is
100000, but increasing it to 1000000 can help difficult cases.  Manually
assigning some or all of the unknown variables with the \texttt{let
variable} command also helps difficult cases.



\subsection{\texttt{initialize} Command}\index{\texttt{initialize} command}
Syntax:  \texttt{initialize step} {\em step}

    and: \texttt{initialize all}

These commands, available in the Proof Assistant\index{Proof Assistant}
only, ``de-unify'' the target and source of a step (or all steps), as
well as the hypotheses of the source, and makes all variables in the
source and the source's hypotheses unknown.  This command is useful to
help recover from incorrect unifications that resulted from an incorrect
\texttt{assign}, \texttt{let}, or unification choice.  Part or all of
the command sequence \texttt{delete floating{\char`\_}hypotheses},
\texttt{initialize all}, and \texttt{unify all /interactive} will recover
from incorrect unifications.

See also:  \texttt{unify} and \texttt{delete}



\subsection{\texttt{delete} Command}\index{\texttt{delete} command}
Syntax:  \texttt{delete step} {\em step}

   and:      \texttt{delete all} -- {\em Warning: dangerous!}

   and:      \texttt{delete floating{\char`\_}hypotheses}

These commands are available in the Proof Assistant only.  The
\texttt{delete step} command deletes the proof tree section that
branches off of the specified step and makes the step become unknown.
\texttt{delete all} is equivalent to \texttt{delete step} {\em step}
where {\em step} is the last step in the proof (i.e.\ the beginning of
the proof tree).

In most cases the \texttt{undo} command is the best way to undo
a previous step.
An alternative is to salvage your last \texttt{save
new{\char`\_}proof} by exiting and reentering the Proof Assistant.
For this to work, keep a log file open to record your work
and to do \texttt{save new{\char`\_}proof} frequently, especially before
\texttt{delete}.

\texttt{delete floating{\char`\_}hypotheses} will delete all sections of
the proof that branch off of \texttt{\$f}\index{\texttt{\$f} statement}
statements.  It is sometimes useful to do this before an
\texttt{initialize} command to recover from an error.  Note that once a
proof step with a \texttt{\$f} hypothesis as the target is completely
known, the \texttt{improve} command can usually fill in the proof for
that step.  Unlike the deletion of logical steps, \texttt{delete
floating{\char`\_}hypotheses} is a relatively safe command that is
usually easy to recover from.



\subsection{\texttt{improve} Command}\index{\texttt{improve} command}
\label{improve}
Syntax:  \texttt{improve} {\em step} [\texttt{/depth} {\em number}]
                                               [\texttt{/no{\char`\_}distinct}]

   and:   \texttt{improve first} [\texttt{/depth} {\em number}]
                                              [\texttt{/no{\char`\_}distinct}]

   and:   \texttt{improve last} [\texttt{/depth} {\em number}]
                                              [\texttt{/no{\char`\_}distinct}]

   and:   \texttt{improve all} [\texttt{/depth} {\em number}]
                                              [\texttt{/no{\char`\_}distinct}]

These commands, available in the Proof Assistant\index{Proof Assistant}
only, try to find proofs automatically for unknown steps whose symbol
sequences are completely known.  They are primarily useful for filling in
proofs of \texttt{\$f}\index{\texttt{\$f} statement} hypotheses.  The
search will be restricted to statements having no
\texttt{\$e}\index{\texttt{\$e} statement} hypotheses.

\begin{sloppypar} % narrow
Note:  If memory is limited, \texttt{improve all} on a large proof may
overflow memory.  If you use \texttt{set unification{\char`\_}timeout 1}
before \texttt{improve all}, there will usually be sufficient
improvement to easily recover and completely \texttt{improve} the proof
later on a larger computer.  Warning:  Once memory has overflowed, there
is no recovery.  If in doubt, save the intermediate proof (\texttt{save
new{\char`\_}proof} then \texttt{write source}) before \texttt{improve
all}.
\end{sloppypar}

If \texttt{last} is specified instead of {\em step} number, the last
step that is shown by \texttt{show new{\char`\_}proof /unknown} will be
used.

If \texttt{first} is specified instead of {\em step} number, the first
step that is shown by \texttt{show new{\char`\_}proof /unknown} will be
used.

If {\em step} is zero or negative, the -{\em step}th from last unknown
step, as shown by \texttt{show new{\char`\_}proof /unknown}, will be
used.  \texttt{improve -1} will use the penultimate
unknown step, \texttt{improve -2} {\em label} the antepenultimate, and
\texttt{improve 0} is the same as \texttt{improve last}.

Optional command qualifier:

    \texttt{/depth} {\em number} - This qualifier will cause the search
        to include
        statements with \texttt{\$e} hypotheses (but no new variables in
        the \texttt{\$e}
        hypotheses), provided that the backtracking has not exceeded the
        specified depth. {\em Warning:}  Try \texttt{/depth 1},
        then \texttt{2}, then \texttt{3}, etc.
        in sequence because of possible exponential blowups.  Save your
        work before trying \texttt{/depth} greater than \texttt{1}!

    \texttt{/no{\char`\_}distinct} - Skip trial statements that have
        \texttt{\$d}\index{\texttt{\$d} statement} requirements.
        This qualifier will prevent assignments that might violate \texttt{\$d}
        requirements but it also could miss possible legal assignments.

See also: \texttt{set search{\char`\_}limit}

\subsection{\texttt{save new\_proof} Command}\index{\texttt{save
new{\char`\_}proof} command}
Syntax:  \texttt{save new{\char`\_}proof} {\em label} [\texttt{/normal}]
   [\texttt{/compressed}]

The \texttt{save new{\char`\_}proof} command is available in the Proof
Assistant only.  It saves the proof in progress in the source
buffer\index{source buffer}.  \texttt{save new{\char`\_}proof} may be
used to save a completed proof, or it may be used to save a proof in
progress in order to work on it later.  If an incomplete proof is saved,
any user assignments with \texttt{let step} or \texttt{let variable}
will be lost, as will any ambiguous unifications\index{ambiguous
unification}\index{unification!ambiguous} that were resolved manually.
To help make recovery easier, it can be helpful to \texttt{improve all}
before \texttt{save new{\char`\_}proof} so that the incomplete proof
will have as much information as possible.

Note that the proof will not be permanently saved until a \texttt{write
source} command is issued.

Optional command qualifiers:

    \texttt{/normal} - The proof is saved in the normal format (i.e., as a
        sequence of labels, which is the defined format of the basic Metamath
        language).\index{basic language}  This is the default format that
        is used if a qualifier is omitted.

    \texttt{/compressed} - The proof is saved in the compressed format, which
        reduces storage requirements for a database.
        (See Appendix~\ref{compressed}.)


\section{Creating \LaTeX\ Output}\label{texout}\index{latex@{\LaTeX}}

You can generate \LaTeX\ output given the
information in a database.
The database must already include the necessary typesetting information
(see section \ref{tcomment} for how to provide this information).

The \texttt{show statement} and \texttt{show proof} commands each have a
special \texttt{/tex} command qualifier that produces \LaTeX\ output.
(The \texttt{show statement} command also has the
\texttt{/simple{\char`\_}tex} qualifier for output that is easier to
edit by hand.)  Before you can use them, you must open a \LaTeX\ file to
which to send their output.  A typical complete session will use this
sequence of Metamath commands:

\begin{verbatim}
read set.mm
open tex example.tex
show statement a1i /tex
show proof a1i /all/lemmon/renumber/tex
show statement uneq2 /tex
show proof uneq2 /all/lemmon/renumber/tex
close tex
\end{verbatim}

See Section~\ref{mathcomments} for information on comment markup and
Appendix~\ref{ASCII} for information on how math symbol translation is
specified.

To format and print the \LaTeX\ source, you will need the \LaTeX\
program, which is standard on most Linux installations and available for
Windows.  On Linux, in order to create a {\sc pdf} file, you will
typically type at the shell prompt
\begin{verbatim}
$ pdflatex example.tex
\end{verbatim}

\subsection{\texttt{open tex} Command}\index{\texttt{open tex} command}
Syntax:  \texttt{open tex} {\em file-name} [\texttt{/no{\char`\_}header}]

This command opens a file for writing \LaTeX\
source\index{latex@{\LaTeX}} and writes a \LaTeX\ header to the file.
\LaTeX\ source can be written with the \texttt{show proof}, \texttt{show
new{\char`\_}proof}, and \texttt{show statement} commands using the
\texttt{/tex} qualifier.

The mapping to \LaTeX\ symbols is defined in a special comment
containing a \texttt{\$t} token, described in Appendix~\ref{ASCII}.

There is an optional command qualifier:

    \texttt{/no{\char`\_}header} - This qualifier prevents a standard
        \LaTeX\ header and trailer
        from being included with the output \LaTeX\ code.


\subsection{\texttt{close tex} Command}\index{\texttt{close tex} command}
Syntax:  \texttt{close tex}

This command writes a trailer to any \LaTeX\ file\index{latex@{\LaTeX}}
that was opened with \texttt{open tex} (unless
\texttt{/no{\char`\_}header} was used with \texttt{open tex}) and closes
the \LaTeX\ file.


\section{Creating {\sc HTML} Output}\label{htmlout}

You can generate {\sc html} web pages given the
information in a database.
The database must already include the necessary typesetting information
(see section \ref{tcomment} for how to provide this information).
The ability to produce {\sc html} web pages was added in Metamath version
0.07.30.

To create an {\sc html} output file(s) for \texttt{\$a} or \texttt{\$p}
statement(s), use
\begin{quote}
    \texttt{show statement} {\em label-match} \texttt{/html}
\end{quote}
The output file will be named {\em label-match}\texttt{.html}
for each match.  When {\em
label-match} has wildcard (\texttt{*}) characters, all statements with
matching labels will have {\sc html} files produced for them.  Also,
when {\em label-match} has a wildcard (\texttt{*}) character, two additional
files, \texttt{mmdefinitions.html} and \texttt{mmascii.html} will be
produced.  To produce {\em only} these two additional files, you can use
\texttt{?*}, which will not match any statement label, in place of {\em
label-match}.

There are three other qualifiers for \texttt{show statement} that also
generate {\sc HTML} code.  These are \texttt{/alt{\char`\_}html},
\texttt{/brief{\char`\_}html}, and
\texttt{/brief{\char`\_}alt{\char`\_}html}, and are described in the
next section.

The command
\begin{quote}
    \texttt{show statement} {\em label-match} \texttt{/alt{\char`\_}html}
\end{quote}
does the same as \texttt{show statement} {\em label-match} \texttt{/html},
except that the {\sc html} code for the symbols is taken from
\texttt{althtmldef} statements instead of \texttt{htmldef} statements in
the \texttt{\$t} comment.

The command
\begin{verbatim}
show statement * /brief_html
\end{verbatim}
invokes a special mode that just produces definition and theorem lists
accompanied by their symbol strings, in a format suitable for copying and
pasting into another web page (such as the tutorial pages on the
Metamath web site).

Finally, the command
\begin{verbatim}
show statement * /brief_alt_html
\end{verbatim}
does the same as \texttt{show statement * / brief{\char`\_}html}
for the alternate {\sc html}
symbol representation.

A statement's comment can include a special notation that provides a
certain amount of control over the {\sc HTML} version of the comment.  See
Section~\ref{mathcomments} (p.~\pageref{mathcomments}) for the comment
markup features.

The \texttt{write theorem{\char`\_}list} and \texttt{write bibliography}
commands, which are described below, provide as a side effect complete
error checking for all of the features described in this
section.\index{error checking}

\subsection{\texttt{write theorem\_list}
Command}\index{\texttt{write theorem{\char`\_}list} command}
Syntax:  \texttt{write theorem{\char`\_}list}
[\texttt{/theorems{\char`\_}per{\char`\_}page} {\em number}]

This command writes a list of all of the \texttt{\$a} and \texttt{\$p}
statements in the database into a web page file
 called \texttt{mmtheorems.html}.
When additional files are needed, they are called
\texttt{mmtheorems2.html}, \texttt{mmtheorems3.html}, etc.

Optional command qualifier:

    \texttt{/theorems{\char`\_}per{\char`\_}page} {\em number} -
 This qualifier specifies the number of statements to
        write per web page.  The default is 100.

{\em Note:} In version 0.07.30\index{Metamath!limitations of version
0.07.30} of Metamath, the ``Nearby theorems'' links on the individual
web pages presuppose 100 theorems per page when linking to the theorem
list pages.  Therefore the \texttt{/theorems{\char`\_}per{\char`\_}page}
qualifier, if it specifies a number other than 100, will cause the
individual web pages to be out of sync and should not be used to
generate the main theorem list for the web site.  This is expected to be
fixed in a future version.


\subsection{\texttt{write bibliography}\label{wrbib}
Command}\index{\texttt{write bibliography} command}
Syntax:  \texttt{write bibliography} {\em filename}

This command reads an existing {\sc html} bibliographic cross-reference
file, normally called \texttt{mmbiblio.html}, and updates it per the
bibliographic links in the database comments.  The file is updated
between the {\sc html} comment lines \texttt{<!--
{\char`\#}START{\char`\#} -->} and \texttt{<!-- {\char`\#}END{\char`\#}
-->}.  The original input file is renamed to {\em
filename}\texttt{{\char`\~}1}.

A bibliographic reference is indicated with the reference name
in brackets, such as  \texttt{Theorem 3.1 of
[Monk] p.\ 22}.
See Section~\ref{htmlout} (p.~\pageref{htmlout}) for
syntax details.


\subsection{\texttt{write recent\_additions}
Command}\index{\texttt{write recent{\char`\_}additions} command}
Syntax:  \texttt{write recent{\char`\_}additions} {\em filename}
[\texttt{/limit} {\em number}]

This command reads an existing ``Recent Additions'' {\sc html} file,
normally called \texttt{mmrecent.html}, and updates it with the
descriptions of the most recently added theorems to the database.
 The file is updated between
the {\sc html} comment lines \texttt{<!-- {\char`\#}START{\char`\#} -->}
and \texttt{<!-- {\char`\#}END{\char`\#} -->}.  The original input file
is renamed to {\em filename}\texttt{{\char`\~}1}.

Optional command qualifier:

    \texttt{/limit} {\em number} -
 This qualifier specifies the number of most recent theorems to
   write to the output file.  The default is 100.


\section{Text File Utilities}

\subsection{\texttt{tools} Command}\index{\texttt{tools} command}
Syntax:  \texttt{tools}

This command invokes an easy-to-use, general purpose utility for
manipulating the contents of {\sc ascii} text files.  Upon typing
\texttt{tools}, the command-line prompt will change to \texttt{TOOLS>}
until you type \texttt{exit}.  The \texttt{tools} commands can be used
to perform simple, global edits on an input/output file,
such as making a character string substitution on each line, adding a
string to each line, and so on.  A typical use of this utility is
to build a \texttt{submit} input file to perform a common operation on a
list of statements obtained from \texttt{show label} or \texttt{show
usage}.

The actions of most of the \texttt{tools} commands can also be
performed with equivalent (and more powerful) Unix shell commands, and
some users may find those more efficient.  But for Windows users or
users not comfortable with Unix, \texttt{tools} provides an
easy-to-learn alternative that is adequate for most of the
script-building tasks needed to use the Metamath program effectively.

\subsection{\texttt{help} Command (in \texttt{tools})}
Syntax:  \texttt{help}

The \texttt{help} command lists the commands available in the
\texttt{tools} utility, along with a brief description.  Each command,
in turn, has its own help, such as \texttt{help add}.  As with
Metamath's \texttt{MM>} prompt, a complete command can be entered at
once, or just the command word can be typed, causing the program to
prompt for each argument.

\vskip 1ex
\noindent Line-by-line editing commands:

  \texttt{add} - Add a specified string to each line in a file.

  \texttt{clean} - Trim spaces and tabs on each line in a file; convert
         characters.

  \texttt{delete} - Delete a section of each line in a file.

  \texttt{insert} - Insert a string at a specified column in each line of
        a file.

  \texttt{substitute} - Make a simple substitution on each line of the file.

  \texttt{tag} - Like \texttt{add}, but restricted to a range of lines.

  \texttt{swap} - Swap the two halves of each line in a file.

\vskip 1ex
\noindent Other file-processing commands:

  \texttt{break} - Break up (tokenize) a file into a list of tokens (one per
        line).

  \texttt{build} - Build a file with multiple tokens per line from a list.

  \texttt{count} - Count the occurrences in a file of a specified string.

  \texttt{number} - Create a list of numbers.

  \texttt{parallel} - Put two files in parallel.

  \texttt{reverse} - Reverse the order of the lines in a file.

  \texttt{right} - Right-justify lines in a file (useful before sorting
         numbers).

%  \texttt{tag} - Tag edit updates in a program for revision control.

  \texttt{sort} - Sort the lines in a file with key starting at
         specified string.

  \texttt{match} - Extract lines containing (or not) a specified string.

  \texttt{unduplicate} - Eliminate duplicate occurrences of lines in a file.

  \texttt{duplicate} - Extract first occurrence of any line occurring
         more than

   \ \ \    once in a file, discarding lines occurring exactly once.

  \texttt{unique} - Extract lines occurring exactly once in a file.

  \texttt{type} (10 lines) - Display the first few lines in a file.
                  Similar to Unix \texttt{head}.

  \texttt{copy} - Similar to Unix \texttt{cat} but safe (same input
         and output file allowed).

  \texttt{submit} - Run a script containing \texttt{tools} commands.

\vskip 1ex

\noindent Note:
  \texttt{unduplicate}, \texttt{duplicate}, and \texttt{unique} also
 sort the lines as a side effect.


\subsection{Using \texttt{tools} to Build Metamath \texttt{submit}
Scripts}

The \texttt{break} command is typically used to break up a series of
statement labels, such as the output of Metamath's \texttt{show usage},
into one label per line.  The other \texttt{tools} commands can then be
used to add strings before and after each statement label to specify
commands to be performed on the statement.  The \texttt{parallel}
command is useful when a statement label must be mentioned more than
once on a line.

Very often a \texttt{submit} script for Metamath will require multiple
command lines for each statement being processed.  For example, you may
want to enter the Proof Assistant, \texttt{minimize{\char`\_}with} your
latest theorem, \texttt{save} the new proof, and \texttt{exit} the Proof
Assistant.  To accomplish this, you can build a file with these four
commands for each statement on a single line, separating each command
with a designated character such as \texttt{@}.  Then at the end you can
\texttt{substitute} each \texttt{@} with \texttt{{\char`\\}n} to break
up the lines into individual command lines (see \texttt{help
substitute}).


\subsection{Example of a \texttt{tools} Session}

To give you a quick feel for the \texttt{tools} utility, we show a
simple session where we create a file \texttt{n.txt} with 3 lines, add
strings before and after each line, and display the lines on the screen.
You can experiment with the various commands to gain experience with the
\texttt{tools} utility.

\begin{verbatim}
MM> tools
Entering the Text Tools utilities.
Type HELP for help, EXIT to exit.
TOOLS> number
Output file <n.tmp>? n.txt
First number <1>?
Last number <10>? 3
Increment <1>?
TOOLS> add
Input/output file? n.txt
String to add to beginning of each line <>? This is line
String to add to end of each line <>? .
The file n.txt has 3 lines; 3 were changed.
First change is on line 1:
This is line 1.
TOOLS> type n.txt
This is line 1.
This is line 2.
This is line 3.
TOOLS> exit
Exiting the Text Tools.
Type EXIT again to exit Metamath.
MM>
\end{verbatim}



\appendix
\chapter{Sample Representations}
\label{ASCII}

This Appendix provides a sample of {\sc ASCII} representations,
their corresponding traditional mathematical symbols,
and a discussion of their meanings
in the \texttt{set.mm} database.
These are provided in order of appearance.
This is only a partial list, and new definitions are routinely added.
A complete list is available at \url{http://metamath.org}.

These {\sc ASCII} representations, along
with information on how to display them,
are defined in the \texttt{set.mm} database file inside
a special comment called a \texttt{\$t} {\em
comment}\index{\texttt{\$t} comment} or {\em typesetting
comment.}\index{typesetting comment}
A typesetting comment
is indicated by the appearance of the
two-character string \texttt{\$t} at the beginning of the comment.
For more information,
see Section~\ref{tcomment}, p.~\pageref{tcomment}.

In the following table the ``{\sc ASCII}'' column shows the {\sc ASCII}
representation,
``Symbol'' shows the mathematical symbolic display
that corresponds to that {\sc ASCII} representation, ``Labels'' shows
the key label(s) that define the representation, and
``Description'' provides a description about the symbol.
As usual, ``iff'' is short for ``if and only if.''\index{iff}
In most cases the ``{\sc ASCII}'' column only shows
the key token, but it will sometimes show a sequence of tokens
if that is necessary for clarity.

{\setlength{\extrarowsep}{4pt} % Keep rows from being too close together
\begin{longtabu}   { @{} c c l X }
\textbf{ASCII} & \textbf{Symbol} & \textbf{Labels} & \textbf{Description} \\
\endhead
\texttt{|-} & $\vdash$ & &
  ``It is provable that...'' \\
\texttt{ph} & $\varphi$ & \texttt{wph} &
  The wff (boolean) variable phi,
  conventionally the first wff variable. \\
\texttt{ps} & $\psi$ & \texttt{wps} &
  The wff (boolean) variable psi,
  conventionally the second wff variable. \\
\texttt{ch} & $\chi$ & \texttt{wch} &
  The wff (boolean) variable chi,
  conventionally the third wff variable. \\
\texttt{-.} & $\lnot$ & \texttt{wn} &
  Logical not. E.g., if $\varphi$ is true, then $\lnot \varphi$ is false. \\
\texttt{->} & $\rightarrow$ & \texttt{wi} &
  Implies, also known as material implication.
  In classical logic the expression $\varphi \rightarrow \psi$ is true
  if either $\varphi$ is false or $\psi$ is true (or both), that is,
  $\varphi \rightarrow \psi$ has the same meaning as
  $\lnot \varphi \lor \psi$ (as proven in theorem \texttt{imor}). \\
\texttt{<->} & $\leftrightarrow$ &
  \hyperref[df-bi]{\texttt{df-bi}} &
  Biconditional (aka is-equals for boolean values).
  $\varphi \leftrightarrow \psi$ is true iff
  $\varphi$ and $\psi$ have the same value. \\
\texttt{\char`\\/} & $\lor$ &
  \makecell[tl]{{\hyperref[df-or]{\texttt{df-or}}}, \\
	         \hyperref[df-3or]{\texttt{df-3or}}} &
  Disjunction (logical ``or''). $\varphi \lor \psi$ is true iff
  $\varphi$, $\psi$, or both are true. \\
\texttt{/\char`\\} & $\land$ &
  \makecell[tl]{{\hyperref[df-an]{\texttt{df-an}}}, \\
                 \hyperref[df-3an]{\texttt{df-3an}}} &
  Conjunction (logical ``and''). $\varphi \land \psi$ is true iff
  both $\varphi$ and $\psi$ are true. \\
\texttt{A.} & $\forall$ &
  \texttt{wal} &
  For all; the wff $\forall x \varphi$ is true iff
  $\varphi$ is true for all values of $x$. \\
\texttt{E.} & $\exists$ &
  \hyperref[df-ex]{\texttt{df-ex}} &
  There exists; the wff
  $\exists x \varphi$ is true iff
  there is at least one $x$ where $\varphi$ is true. \\
\texttt{[ y / x ]} & $[ y / x ]$ &
  \hyperref[df-sb]{\texttt{df-sb}} &
  The wff $[ y / x ] \varphi$ produces
  the result when $y$ is properly substituted for $x$ in $\varphi$
  ($y$ replaces $x$).
  % This is elsb4
  % ( [ x / y ] z e. y <-> z e. x )
  For example,
  $[ x / y ] z \in y$ is the same as $z \in x$. \\
\texttt{E!} & $\exists !$ &
  \hyperref[df-eu]{\texttt{df-eu}} &
  There exists exactly one;
  $\exists ! x \varphi$ is true iff
  there is at least one $x$ where $\varphi$ is true. \\
\texttt{\{ y | phi \}}  & $ \{ y | \varphi \}$ &
  \hyperref[df-clab]{\texttt{df-clab}} &
  The class of all sets where $\varphi$ is true. \\
\texttt{=} & $ = $ &
  \hyperref[df-cleq]{\texttt{df-cleq}} &
  Class equality; $A = B$ iff $A$ equals $B$. \\
\texttt{e.} & $ \in $ &
  \hyperref[df-clel]{\texttt{df-clel}} &
  Class membership; $A \in B$ if $A$ is a member of $B$. \\
\texttt{{\char`\_}V} & {\rm V} &
  \hyperref[df-v]{\texttt{df-v}} &
  Class of all sets (not itself a set). \\
\texttt{C\_} & $ \subseteq $ &
  \hyperref[df-ss]{\texttt{df-ss}} &
  Subclass (subset); $A \subseteq B$ is true iff
  $A$ is a subclass of $B$. \\
\texttt{u.} & $ \cup $ &
  \hyperref[df-un]{\texttt{df-un}} &
  $A \cup B$ is the union of classes $A$ and $B$. \\
\texttt{i^i} & $ \cap $ &
  \hyperref[df-in]{\texttt{df-in}} &
  $A \cap B$ is the intersection of classes $A$ and $B$. \\
\texttt{\char`\\} & $ \setminus $ &
  \hyperref[df-dif]{\texttt{df-dif}} &
  $A \setminus B$ (set difference)
  is the class of all sets in $A$ except for those in $B$. \\
\texttt{(/)} & $ \varnothing $ &
  \hyperref[df-nul]{\texttt{df-nul}} &
  $ \varnothing $ is the empty set (aka null set). \\
\texttt{\char`\~P} & $ \cal P $ &
  \hyperref[df-pw]{\texttt{df-pw}} &
  Power class. \\
\texttt{<.\ A , B >.} & $\langle A , B \rangle$ &
  \hyperref[df-op]{\texttt{df-op}} &
  The ordered pair $\langle A , B \rangle$. \\
\texttt{( F ` A )} & $ ( F ` A ) $ &
  \hyperref[df-fv]{\texttt{df-fv}} &
  The value of function $F$ when applied to $A$. \\
\texttt{\_i} & $ i $ &
  \texttt{df-i} &
  The square root of negative one. \\
\texttt{x.} & $ \cdot $ &
  \texttt{df-mul} &
  Complex number multiplication; $2~\cdot~3~=~6$. \\
\texttt{CC} & $ \mathbb{C} $ &
  \texttt{df-c} &
  The set of complex numbers. \\
\texttt{RR} & $ \mathbb{R} $ &
  \texttt{df-r} &
  The set of real numbers. \\
\end{longtabu}
} % end of extrarowsep

\chapter{Compressed Proofs}
\label{compressed}\index{compressed proof}\index{proof!compressed}

The proofs in the \texttt{set.mm} set theory database are stored in compressed
format for efficiency.  Normally you needn't concern yourself with the
compressed format, since you can display it with the usual proof display tools
in the Metamath program (\texttt{show proof}\ldots) or convert it to the normal
RPN proof format described in Section~\ref{proof} (with \texttt{save proof}
{\em label} \texttt{/normal}).  However for sake of completeness we describe the
format here and show how it maps to the normal RPN proof format.

A compressed proof, located between \texttt{\$=} and \texttt{\$.}\ keywords, consists
of a left parenthesis, a sequence of statement labels, a right parenthesis,
and a sequence of upper-case letters \texttt{A} through \texttt{Z} (with optional
white space between them).  White space must surround the parentheses
and the labels.  The left parenthesis tells Metamath that a
compressed proof follows.  (A normal RPN proof consists of just a sequence of
labels, and a parenthesis is not a legal character in a label.)

The sequence of upper-case letters corresponds to a sequence of integers
with the following mapping.  Each integer corresponds to a proof step as
described later.
\begin{center}
  \texttt{A} = 1 \\
  \texttt{B} = 2 \\
   \ldots \\
  \texttt{T} = 20 \\
  \texttt{UA} = 21 \\
  \texttt{UB} = 22 \\
   \ldots \\
  \texttt{UT} = 40 \\
  \texttt{VA} = 41 \\
  \texttt{VB} = 42 \\
   \ldots \\
  \texttt{YT} = 120 \\
  \texttt{UUA} = 121 \\
   \ldots \\
  \texttt{YYT} = 620 \\
  \texttt{UUUA} = 621 \\
   etc.
\end{center}

In other words, \texttt{A} through \texttt{T} represent the
least-significant digit in base 20, and \texttt{U} through \texttt{Y}
represent zero or more most-significant digits in base 5, where the
digits start counting at 1 instead of the usual 0. With this scheme, we
don't need white space between these ``numbers.''

(In the design of the compressed proof format, only upper-case letters,
as opposed to say all non-whitespace printable {\sc ascii} characters other than
%\texttt{\$}, was chosen to make the compressed proof a little less
%displeasing to the eye, at the expense of a typical 20\% compression
\texttt{\$}, were chosen so as not to collide with most text editor
searches, at the expense of a typical 20\% compression
loss.  The base 5/base 20 grouping, as opposed to say base 6/base 19,
was chosen by experimentally determining the grouping that resulted in
best typical compression.)

The letter \texttt{Z} identifies (tags) a proof step that is identical to one
that occurs later on in the proof; it helps shorten the proof by not requiring
that identical proof steps be proved over and over again (which happens often
when building wff's).  The \texttt{Z} is placed immediately after the
least-significant digit (letters \texttt{A} through \texttt{T}) that ends the integer
corresponding to the step to later be referenced.

The integers that the upper-case letters correspond to are mapped to labels as
follows.  If the statement being proved has $m$ mandatory hypotheses, integers
1 through $m$ correspond to the labels of these hypotheses in the order shown
by the \texttt{show statement ... / full} command, i.e., the RPN order\index{RPN
order} of the mandatory
hypotheses.  Integers $m+1$ through $m+n$ correspond to the labels enclosed in
the parentheses of the compressed proof, in the order that they appear, where
$n$ is the number of those labels.  Integers $m+n+1$ on up don't directly
correspond to statement labels but point to proof steps identified with the
letter \texttt{Z}, so that these proof steps can be referenced later in the
proof.  Integer $m+n+1$ corresponds to the first step tagged with a \texttt{Z},
$m+n+2$ to the second step tagged with a \texttt{Z}, etc.  When the compressed
proof is converted to a normal proof, the entire subproof of a step tagged
with \texttt{Z} replaces the reference to that step.

For efficiency, Metamath works with compressed proofs directly, without
converting them internally to normal proofs.  In addition to the usual
error-checking, an error message is given if (1) a label in the label list in
parentheses does not refer to a previous \texttt{\$p} or \texttt{\$a} statement or a
non-mandatory hypothesis of the statement being proved and (2) a proof step
tagged with \texttt{Z} is referenced before the step tagged with the \texttt{Z}.

Just as in a normal proof under development (Section~\ref{unknown}), any step
or subproof that is not yet known may be represented with a single \texttt{?}.
White space does not have to appear between the \texttt{?}\ and the upper-case
letters (or other \texttt{?}'s) representing the remainder of the proof.

% April 1, 2004 Appendix C has been added back in with corrections.
%
% May 20, 2003 Appendix C was removed for now because there was a problem found
% by Bob Solovay
%
% Also, removed earlier \ref{formalspec} 's (3 cases above)
%
% Bob Solovay wrote on 30 Nov 2002:
%%%%%%%%%%%%% (start of email comment )
%      3. My next set of comments concern appendix C. I read this before I
% read Chapter 4. So I first noted that the system as presented in the
% Appendix lacked a certain formal property that I thought desirable. I
% then came up with a revised formal system that had this property. Upon
% reading Chapter 4, I noticed that the revised system was closer to the
% treatment in Chapter 4 than the system in Appendix C.
%
%         First a very minor correction:
%
%         On page 142 line 2: The condition that V(e) != V(f) should only be
% required of e, f in T such that e != f.
%
%         Here is a natural property [transitivity] that one would like
% the formal system to have:
%
%         Let Gamma be a set of statements. Suppose that the statement Phi
% is provable from Gamma and that the statement Psi is provable from Gamma
% \cup {Phi}. Then Psi is provable from Gamma.
%
%         I shall present an example to show that this property does not
% hold for the formal systems of Appendix C:
%
%         I write the example in metamath style:
%
% $c A B C D E $.
% $v x y
%
% ${
% tx $f A x $.
% ty $f B y $.
% ax1 $a C x y $.
% $}
%
% ${
% tx $f A x $.
% ty $f B y $.
% ax2-h1 $e C x y $.
% ax2 $a D y $.
% $}
%
% ${
% ty $f B y $.
% ax3-h1 $e D y $.
% ax3 $a E y $.
% $}
%
% $(These three axioms are Gamma $)
%
% ${
% tx $f A x $.
% ty $f B y $.
% Phi $p D y $=
% tx ty tx ty ax1 ax2 $.
% $}
%
% ${
% ty $f B y $.
% Psi $p E y $=
% ty ty Phi ax3 $.
% $}
%
%
% I omit the formal proofs of the following claims. [I will be glad to
% supply them upon request.]
%
% 1) Psi is not provable from Gamma;
%
% 2) Psi is provable from Gamma + Phi.
%
% Here "provable" refers to the formalism of Appendix C.
%
% The trouble of course is that Psi is lacking the variable declaration
%
% $f Ax $.
%
% In the Metamath system there is no trouble proving Psi. I attach a
% metamath file that shows this and which has been checked by the
% metamath program.
%
% I next want to indicate how I think the treatment in Appendix C should
% be revised so as to conform more closely to the metamath system of the
% main text. The revised system *does* have the transitivity property.
%
% We want to give revised definitions of "statement" and
% "provable". [cf. sections C.2.4. and C.2.5] Our new definitions will
% use the definitions given in Appendix C. So we take the following
% tack. We refer to the original notions as o-statement and o-provable. And
% we refer to the notions we are defining as n-statement and n-provable.
%
%         A n-statement is an o-statement in which the only variables
% that appear in the T component are mandatory.
%
%         To any o-statement we can associate its reduct which is a
% n-statement by dropping all the elements of T or D which contain
% non-mandatory variables.
%
%         An n-statement gamma is n-provable if there is an o-statement
% gamma' which has gamma as its reduct andf such that gamma' is
% o-provable.
%
%         It seems to me [though I am not completely sure on this point]
% that n-provability corresponds to metamath provability as discussed
% say in Chapter 4.
%
%         Attached to this letter is the metamath proof of Phi and Psi
% from Gamma discussed above.
%
%         I am still brooding over the question of whether metamath
% correctly formalizes set-theory. No doubt I will have some questions
% re this after my thoughts become clearer.
%%%%%%%%%%%%%%%% (end of email comment)

%%%%%%%%%%%%%%%% (start of 2nd email comment from Bob Solovay 1-Apr-04)
%
%         I hope that Appendix C is the one that gives a "formal" treatment
% of Metamath. At any rate, thats the appendix I want to comment on.
%
%         I'm going to suggest two changes in the definition.
%
%         First change (in the definition of statement): Require that the
% sets D, T, and E be finite.
%
%         Probably things are fine as you give them. But in the applications
% to the main metamath system they will always be finite, and its useful in
% thinking about things [at least for me] to stick to the finite case.
%
%         Second change:
%
%         First let me give an approximate description. Remove the dummy
% variables from the statement. Instead, include them in the proof.
%
%         More formally: Require that T consists of type declarations only
% for mandatory variables. Require that all the pairs in D consist of
% mandatory variables.
%
%         At the start of a proof we are allowed to declare a finite number
% of dummy variables [provided that none of them appear in any of the
% statements in E \cup {A}. We have to supply type declarations for all the
% dummy variables. We are allowed to add new $d statements referring to
% either the mandatory or dummy variables. But we require that no new $d
% statement references only mandatory variables.
%
%         I find this way of doing things more conceptual than the treatment
% in Appendix C. But the change [which I will use implicitly in later
% letters about doing Peano] is mainly aesthetic. I definitely claim that my
% results on doing Peano all apply to Metamath as it is presented in your
% book.
%
%         --Bob
%
%%%%%%%%%%%%%%%% (end of 2nd email comment)

%%
%% When uncommenting the below, also uncomment references above to {formalspec}
%%
\chapter{Metamath's Formal System}\label{formalspec}\index{Metamath!as a formal
system}

\section{Introduction}

\begin{quote}
  {\em Perfection is when there is no longer anything more to take away.}
    \flushright\sc Antoine de
     Saint-Exupery\footnote{\cite[p.~3-25]{Campbell}.}\\
\end{quote}\index{de Saint-Exupery, Antoine}

This appendix describes the theory behind the Metamath language in an abstract
way intended for mathematicians.  Specifically, we construct two
set-theo\-ret\-i\-cal objects:  a ``formal system'' (roughly, a set of syntax
rules, axioms, and logical rules) and its ``universe'' (roughly, the set of
theorems derivable in the formal system).  The Metamath computer language
provides us with a way to describe specific formal systems and, with the aid of
a proof provided by the user, to verify that given theorems
belong to their universes.

To understand this appendix, you need a basic knowledge of informal set theory.
It should be sufficient to understand, for example, Ch.\ 1 of Munkres' {\em
Topology} \cite{Munkres}\index{Munkres, James R.} or the
introductory set theory chapter
in many textbooks that introduce abstract mathematics. (Note that there are
minor notational differences among authors; e.g.\ Munkres uses $\subset$ instead
of our $\subseteq$ for ``subset.''  We use ``included in'' to mean ``a subset
of,'' and ``belongs to'' or ``is contained in'' to mean ``is an element of.'')
What we call a ``formal'' description here, unlike earlier, is actually an
informal description in the ordinary language of mathematicians.  However we
provide sufficient detail so that a mathematician could easily formalize it,
even in the language of Metamath itself if desired.  To understand the logic
examples at the end of this appendix, familiarity with an introductory book on
mathematical logic would be helpful.

\section{The Formal Description}

\subsection[Preliminaries]{Preliminaries\protect\footnotemark}%
\footnotetext{This section is taken mostly verbatim
from Tarski \cite[p.~63]{Tarski1965}\index{Tarski, Alfred}.}

By $\omega$ we denote the set of all natural numbers (non-negative integers).
Each natural number $n$ is identified with the set of all smaller numbers: $n =
\{ m | m < n \}$.  The formula $m < n$ is thus equivalent to the condition: $m
\in n$ and $m,n \in \omega$. In particular, 0 is the number zero and at the
same time the empty set $\varnothing$, $1=\{0\}$, $2=\{0,1\}$, etc. ${}^B A$
denotes the set of all functions on $B$ to $A$ (i.e.\ with domain $B$ and range
included in $A$).  The members of ${}^\omega A$ are what are called {\em simple
infinite sequences},\index{simple infinite sequence}
with all {\em terms}\index{term} in $A$.  In case $n \in \omega$, the
members of ${}^n A$ are referred to as {\em finite $n$-termed
sequences},\index{finite $n$-termed
sequence} again
with terms in $A$.  The consecutive terms (function values) of a finite or
infinite sequence $f$ are denoted by $f_0, f_1, \ldots ,f_n,\ldots$.  Every
finite sequence $f \in \bigcup _{n \in \omega} {}^n A$ uniquely determines the
number $n$ such that $f \in {}^n A$; $n$ is called the {\em
length}\index{length of a sequence ({$"|\ "|$})} of $f$ and
is denoted by $|f|$.  $\langle a \rangle$ is the sequence $f$ with $|f|=1$ and
$f_0=a$; $\langle a,b \rangle$ is the sequence $f$ with $|f|=2$, $f_0=a$,
$f_1=b$; etc.  Given two finite sequences $f$ and $g$, we denote by $f\frown g$
their {\em concatenation},\index{concatenation} i.e., the
finite sequence $h$ determined by the
conditions:
\begin{eqnarray*}
& |h| = |f|+|g|;&  \\
& h_n = f_n & \mbox{\ for\ } n < |f|;  \\
& h_{|f|+n} = g_n & \mbox{\ for\ } n < |g|.
\end{eqnarray*}

\subsection{Constants, Variables, and Expressions}

A formal system has a set of {\em symbols}\index{symbol!in
a formal system} denoted
by $\mbox{\em SM}$.  A
precise set-theo\-ret\-i\-cal definition of this set is unimportant; a symbol
could be considered a primitive or atomic element if we wish.  We assume this
set is divided into two disjoint subsets:  a set $\mbox{\em CN}$ of {\em
constants}\index{constant!in a formal system} and a set $\mbox{\em VR}$ of
{\em variables}.\index{variable!in a formal system}  $\mbox{\em CN}$ and
$\mbox{\em VR}$ are each assumed to consist of countably many symbols which
may be arranged in finite or simple infinite sequences $c_0, c_1, \ldots$ and
$v_0, v_1, \ldots$ respectively, without repeating terms.  We will represent
arbitrary symbols by metavariables $\alpha$, $\beta$, etc.

{\footnotesize\begin{quotation}
{\em Comment.} The variables $v_0, v_1, \ldots$ of our formal system
correspond to what are usually considered ``metavariables'' in
descriptions of specific formal systems in the literature.  Typically,
when describing a specific formal system a book will postulate a set of
primitive objects called variables, then proceed to describe their
properties using metavariables that range over them, never mentioning
again the actual variables themselves.  Our formal system does not
mention these primitive variable objects at all but deals directly with
metavariables, as its primitive objects, from the start.  This is a
subtle but key distinction you should keep in mind, and it makes our
definition of ``formal system'' somewhat different from that typically
found in the literature.  (So, the $\alpha$, $\beta$, etc.\ above are
actually ``metametavariables'' when used to represent $v_0, v_1,
\ldots$.)
\end{quotation}}

Finite sequences all terms of which are symbols are called {\em
expressions}.\index{expression!in a formal system}  $\mbox{\em EX}$ is
the set of all expressions; thus
\begin{displaymath}
\mbox{\em EX} = \bigcup _{n \in \omega} {}^n \mbox{\em SM}.
\end{displaymath}

A {\em constant-prefixed expression}\index{constant-prefixed expression}
is an expression of non-zero length
whose first term is a constant.  We denote the set of all constant-prefixed
expressions by $\mbox{\em EX}_C = \{ e \in \mbox{\em EX} | ( |e| > 0 \wedge
e_0 \in \mbox{\em CN} ) \}$.

A {\em constant-variable pair}\index{constant-variable pair}
is an expression of length 2 whose first term
is a constant and whose second term is a variable.  We denote the set of all
constant-variable pairs by $\mbox{\em EX}_2 = \{ e \in \mbox{\em EX}_C | ( |e|
= 2 \wedge e_1 \in \mbox{\em VR} ) \}$.


{\footnotesize\begin{quotation}
{\em Relationship to Metamath.} In general, the set $\mbox{\em SM}$
corresponds to the set of declared math symbols in a Metamath database, the
set $\mbox{\em CN}$ to those declared with \texttt{\$c} statements, and the set
$\mbox{\em VR}$ to those declared with \texttt{\$v} statements.  Of course a
Metamath database can only have a finite number of math symbols, whereas
formal systems in general can have an infinite number, although the number of
Metamath math symbols available is in principle unlimited.

The set $\mbox{\em EX}_C$ corresponds to the set of permissible expressions
for \texttt{\$e}, \texttt{\$a}, and \texttt{\$p} statements.  The set $\mbox{\em EX}_2$
corresponds to the set of permissible expressions for \texttt{\$f} statements.
\end{quotation}}

We denote by ${\cal V}(e)$ the set of all variables in an expression $e \in
\mbox{\em EX}$, i.e.\ the set of all $\alpha \in \mbox{\em VR}$ such that
$\alpha = e_n$ for some $n < |e|$.  We also denote (with abuse of notation) by
${\cal V}(E)$ the set of all variables in a collection of expressions $E
\subseteq \mbox{\em EX}$, i.e.\ $\bigcup _{e \in E} {\cal V}(e)$.


\subsection{Substitution}

Given a function $F$ from $\mbox{\em VR}$ to
$\mbox{\em EX}$, we
denote by $\sigma_{F}$ or just $\sigma$ the function from $\mbox{\em EX}$ to
$\mbox{\em EX}$ defined recursively for nonempty sequences by
\begin{eqnarray*}
& \sigma(<\alpha>) = F(\alpha) & \mbox{for\ } \alpha \in \mbox{\em VR}; \\
& \sigma(<\alpha>) = <\alpha> & \mbox{for\ } \alpha \not\in \mbox{\em VR}; \\
& \sigma(g \frown h) = \sigma(g) \frown
    \sigma(h) & \mbox{for\ } g,h \in \mbox{\em EX}.
\end{eqnarray*}
We also define $\sigma(\varnothing)=\varnothing$.  We call $\sigma$ a {\em
simultaneous substitution}\index{substitution!variable}\index{variable
substitution} (or just {\em substitution}) with {\em substitution
map}\index{substitution map} $F$.

We also denote (with abuse of notation) by $\sigma(E)$ a substitution on a
collection of expressions $E \subseteq \mbox{\em EX}$, i.e.\ the set $\{
\sigma(e) | e \in E \}$.  The collection $\sigma(E)$ may of course contain
fewer expressions than $E$ because duplicate expressions could result from the
substitution.

\subsection{Statements}

We denote by $\mbox{\em DV}$ the set of all
unordered pairs $\{\alpha, \beta \} \subseteq \mbox{\em VR}$ such that $\alpha
\neq \beta$.  $\mbox{\em DV}$ stands for ``distinct variables.''

A {\em pre-statement}\index{pre-statement!in a formal system} is a
quadruple $\langle D,T,H,A \rangle$ such that
$D\subseteq \mbox{\em DV}$, $T\subseteq \mbox{\em EX}_2$, $H\subseteq
\mbox{\em EX}_C$ and $H$ is finite,
$A\in \mbox{\em EX}_C$, ${\cal V}(H\cup\{A\}) \subseteq
{\cal V}(T)$, and $\forall e,f\in T {\ } {\cal V}(e) \neq {\cal V}(f)$ (or
equivalently, $e_1 \ne f_1$) whenever $e \neq f$. The terms of the quadruple are called {\em
distinct-variable restrictions},\index{disjoint-variable restriction!in a
formal system} {\em variable-type hypotheses},\index{variable-type
hypothesis!in a formal system} {\em logical hypotheses},\index{logical
hypothesis!in a formal system} and the {\em assertion}\index{assertion!in a
formal system} respectively.  We denote by $T_M$ ({\em mandatory variable-type
hypotheses}\index{mandatory variable-type hypothesis!in a formal system}) the
subset of $T$ such that ${\cal V}(T_M) ={\cal V}(H \cup \{A\})$.  We denote by
$D_M=\{\{\alpha,\beta\}\in D|\{\alpha,\beta\}\subseteq {\cal V}(T_M)\}$ the
{\em mandatory distinct-variable restrictions}\index{mandatory
disjoint-variable restriction!in a formal system} of the pre-statement.
The set
of {\em mandatory hypotheses}\index{mandatory hypothesis!in a formal system}
is $T_M\cup H$.  We call the quadruple $\langle D_M,T_M,H,A \rangle$
the {\em reduct}\index{reduct!in a formal system} of
the pre-statement $\langle D,T,H,A \rangle$.

A {\em statement} is the reduct of some pre-statement\index{statement!in a
formal system}.  A statement is therefore a special kind of pre-statement;
in particular, a statement is the reduct of itself.

{\footnotesize\begin{quotation}
{\em Comment.}  $T$ is a set of expressions, each of length 2, that associate
a set of constants (``variable types'') with a set of variables.  The
condition ${\cal V}(H\cup\{A\}) \subseteq {\cal V}(T) $
means that each variable occurring in a statement's logical
hypotheses or assertion must have an associated variable-type hypothesis or
``type declaration,'' in  analogy to a computer programming language, where a
variable must be declared to be say, a string or an integer.  The requirement
that $\forall e,f\in T \, e_1 \ne f_1$ for $e\neq f$
means that each variable must be
associated with a unique constant designating its variable type; e.g., a
variable might be a ``wff'' or a ``set'' but not both.

Distinct-variable restrictions are used to specify what variable substitutions
are permissible to make for the statement to remain valid.  For example, in
the theorem scheme of set theory $\lnot\forall x\,x=y$ we may not substitute
the same variable for both $x$ and $y$.  On the other hand, the theorem scheme
$x=y\to y=x$ does not require that $x$ and $y$ be distinct, so we do not
require a distinct-variable restriction, although having one
would cause no harm other than making the scheme less general.

A mandatory variable-type hypothesis is one whose variable exists in a logical
hypothesis or the assertion.  A provable pre-statement
(defined below) may require
non-mandatory variable-type hypotheses that effectively introduce ``dummy''
variables for use in its proof.  Any number of dummy variables might
be required by a specific proof; indeed, it has been shown by H.\
Andr\'{e}ka\index{Andr{\'{e}}ka, H.} \cite{Nemeti} that there is no finite
upper bound to the number of dummy variables needed to prove an arbitrary
theorem in first-order logic (with equality) having a fixed number $n>2$ of
individual variables.  (See also the Comment on p.~\pageref{nodd}.)
For this reason we do not set a finite size bound on the collections $D$ and
$T$, although in an actual application (Metamath database) these will of
course be finite, increased to whatever size is necessary as more
proofs are added.
\end{quotation}}

{\footnotesize\begin{quotation}
{\em Relationship to Metamath.} A pre-statement of a formal system
corresponds to an extended frame in a Metamath database
(Section~\ref{frames}).  The collections $D$, $T$, and $H$ correspond
respectively to the \texttt{\$d}, \texttt{\$f}, and \texttt{\$e}
statement collections in an extended frame.  The expression $A$
corresponds to the \texttt{\$a} (or \texttt{\$p}) statement in an
extended frame.

A statement of a formal system corresponds to a frame in a Metamath
database.
\end{quotation}}

\subsection{Formal Systems}

A {\em formal system}\index{formal system} is a
triple $\langle \mbox{\em CN},\mbox{\em
VR},\Gamma\rangle$ where $\Gamma$ is a set of statements.  The members of
$\Gamma$ are called {\em axiomatic statements}.\index{axiomatic
statement!in a formal system}  Sometimes we will refer to a
formal system by just $\Gamma$ when $\mbox{\em CN}$ and $\mbox{\em VR}$ are
understood.

Given a formal system $\Gamma$, the {\em closure}\index{closure}\footnote{This
definition of closure incorporates a simplification due to
Josh Purinton.\index{Purinton, Josh}.} of a
pre-statement
$\langle D,T,H,A \rangle$ is the smallest set $C$ of expressions
such that:
%\begin{enumerate}
%  \item $T\cup H\subseteq C$; and
%  \item If for some axiomatic statement
%    $\langle D_M',T_M',H',A' \rangle \in \Gamma_A$, for
%    some $E \subseteq C$, some $F \subseteq C-T$ (where ``-'' denotes
%    set difference), and some substitution
%    $\sigma$ we have
%    \begin{enumerate}
%       \item $\sigma(T_M') = E$ (where, as above, the $M$ denotes the
%           mandatory variable-type hypotheses of $T^A$);
%       \item $\sigma(H') = F$;
%       \item for all $\{\alpha,\beta\}\in D^A$ and $\subseteq
%         {\cal V}(T_M')$, for all $\gamma\in {\cal V}(\sigma(\langle \alpha
%         \rangle))$, and for all $\delta\in  {\cal V}(\sigma(\langle \beta
%         \rangle))$, we have $\{\gamma, \delta\} \in D$;
%   \end{enumerate}
%   then $\sigma(A') \in C$.
%\end{enumerate}
\begin{list}{}{\itemsep 0.0pt}
  \item[1.] $T\cup H\subseteq C$; and
  \item[2.] If for some axiomatic statement
    $\langle D_M',T_M',H',A' \rangle \in
       \Gamma$ and for some substitution
    $\sigma$ we have
    \begin{enumerate}
       \item[a.] $\sigma(T_M' \cup H') \subseteq C$; and
       \item[b.] for all $\{\alpha,\beta\}\in D_M'$, for all $\gamma\in
         {\cal V}(\sigma(\langle \alpha
         \rangle))$, and for all $\delta\in  {\cal V}(\sigma(\langle \beta
         \rangle))$, we have $\{\gamma, \delta\} \in D$;
   \end{enumerate}
   then $\sigma(A') \in C$.
\end{list}
A pre-statement $\langle D,T,H,A
\rangle$ is {\em provable}\index{provable statement!in a formal
system} if $A\in C$ i.e.\ if its assertion belongs to its
closure.  A statement is {\em provable} if it is
the reduct of a provable pre-statement.
The {\em universe}\index{universe of a formal system}
of a formal system is
the collection of all of its provable statements.  Note that the
set of axiomatic statements $\Gamma$ in a formal system is a subset of its
universe.

{\footnotesize\begin{quotation}
{\em Comment.} The first condition in the definition of closure simply says
that the hypotheses of the pre-statement are in its closure.

Condition 2(a) says that a substitution exists that makes the
mandatory hypotheses of an axiomatic statement exactly match some members of
the closure.  This is what we explicitly demonstrate in a Metamath language
proof.

%Conditions 2(a) and 2(b) say that a substitution exists that makes the
%(mandatory) hypotheses of an axiomatic statement exactly match some members of
%the closure.  This is what we explicitly demonstrate with a Metamath language
%proof.
%
%The set of expressions $F$ in condition 2(b) excludes the variable-type
%hypotheses; this is done because non-mandatory variable-type hypotheses are
%effectively ``dropped'' as irrelevant whereas logical hypotheses must be
%retained to achieve a consistent logical system.

Condition 2(b) describes how distinct-variable restrictions in the axiomatic
statement must be met.  It means that after a substitution for two variables
that must be distinct, the resulting two expressions must either contain no
variables, or if they do, they may not have variables in common, and each pair
of any variables they do have, with one variable from each expression, must be
specified as distinct in the original statement.
\end{quotation}}

{\footnotesize\begin{quotation}
{\em Relationship to Metamath.} Axiomatic statements
 and provable statements in a formal
system correspond to the frames for \texttt{\$a} and \texttt{\$p} statements
respectively in a Metamath database.  The set of axiomatic statements is a
subset of the set of provable statements in a formal system, although in a
Metamath database a \texttt{\$a} statement is distinguished by not having a
proof.  A Metamath language proof for a \texttt{\$p} statement tells the computer
how to explicitly construct a series of members of the closure ultimately
leading to a demonstration that the assertion
being proved is in the closure.  The actual closure typically contains
an infinite number of expressions.  A formal system itself does not have
an explicit object called a ``proof'' but rather the existence of a proof
is implied indirectly by membership of an assertion in a provable
statement's closure.  We do this to make the formal system easier
to describe in the language of set theory.

We also note that once established as provable, a statement may be considered
to acquire the same status as an axiomatic statement, because if the set of
axiomatic statements is extended with a provable statement, the universe of
the formal system remains unchanged (provided that $\mbox{\em VR}$ is
infinite).
In practice, this means we can build a hierarchy of provable statements to
more efficiently establish additional provable statements.  This is
what we do in Metamath when we allow proofs to reference previous
\texttt{\$p} statements as well as previous \texttt{\$a} statements.
\end{quotation}}

\section{Examples of Formal Systems}

{\footnotesize\begin{quotation}
{\em Relationship to Metamath.} The examples in this section, except Example~2,
are for the most part exact equivalents of the development in the set
theory database \texttt{set.mm}.  You may want to compare Examples~1, 3, and 5
to Section~\ref{metaaxioms}, Example 4 to Sections~\ref{metadefprop} and
\ref{metadefpred}, and Example 6 to
Section~\ref{setdefinitions}.\label{exampleref}
\end{quotation}}

\subsection{Example~1---Propositional Calculus}\index{propositional calculus}

Classical propositional calculus can be described by the following formal
system.  We assume the set of variables is infinite.  Rather than denoting the
constants and variables by $c_0, c_1, \ldots$ and $v_0, v_1, \ldots$, for
readability we will instead use more conventional symbols, with the
understanding of course that they denote distinct primitive objects.
Also for readability we may omit commas between successive terms of a
sequence; thus $\langle \mbox{wff\ } \varphi\rangle$ denotes
$\langle \mbox{wff}, \varphi\rangle$.

Let
\begin{itemize}
  \item[] $\mbox{\em CN}=\{\mbox{wff}, \vdash, \to, \lnot, (,)\}$
  \item[] $\mbox{\em VR}=\{\varphi,\psi,\chi,\ldots\}$
  \item[] $T = \{\langle \mbox{wff\ } \varphi\rangle,
             \langle \mbox{wff\ } \psi\rangle,
             \langle \mbox{wff\ } \chi\rangle,\ldots\}$, i.e.\ those
             expressions of length 2 whose first member is $\mbox{\rm wff}$
             and whose second member belongs to $\mbox{\em VR}$.\footnote{For
convenience we let $T$ be an infinite set; the definition of a statement
permits this in principle.  Since a Metamath source file has a finite size, in
practice we must of course use appropriate finite subsets of this $T$,
specifically ones containing at least the mandatory variable-type
hypotheses.  Similarly, in the source file we introduce new variables as
required, with the understanding that a potentially infinite number of
them are available.}
\noindent Then $\Gamma$ consists of the axiomatic statements that
are the reducts of the following pre-statements:
    \begin{itemize}
      \item[] $\langle\varnothing,T,\varnothing,
               \langle \mbox{wff\ }(\varphi\to\psi)\rangle\rangle$
      \item[] $\langle\varnothing,T,\varnothing,
               \langle \mbox{wff\ }\lnot\varphi\rangle\rangle$
      \item[] $\langle\varnothing,T,\varnothing,
               \langle \vdash(\varphi\to(\psi\to\varphi))
               \rangle\rangle$
      \item[] $\langle\varnothing,T,
               \varnothing,
               \langle \vdash((\varphi\to(\psi\to\chi))\to
               ((\varphi\to\psi)\to(\varphi\to\chi)))
               \rangle\rangle$
      \item[] $\langle\varnothing,T,
               \varnothing,
               \langle \vdash((\lnot\varphi\to\lnot\psi)\to
               (\psi\to\varphi))\rangle\rangle$
      \item[] $\langle\varnothing,T,
               \{\langle\vdash(\varphi\to\psi)\rangle,
                 \langle\vdash\varphi\rangle\},
               \langle\vdash\psi\rangle\rangle$
    \end{itemize}
\end{itemize}

(For example, the reduct of $\langle\varnothing,T,\varnothing,
               \langle \mbox{wff\ }(\varphi\to\psi)\rangle\rangle$
is
\begin{itemize}
\item[] $\langle\varnothing,
\{\langle \mbox{wff\ } \varphi\rangle,
             \langle \mbox{wff\ } \psi\rangle\},
             \varnothing,
               \langle \mbox{wff\ }(\varphi\to\psi)\rangle\rangle$,
\end{itemize}
which is the first axiomatic statement.)

We call the members of $\mbox{\em VR}$ {\em wff variables} or (in the context
of first-order logic which we will describe shortly) {\em wff metavariables}.
Note that the symbols $\phi$, $\psi$, etc.\ denote actual specific members of
$\mbox{\em VR}$; they are not metavariables of our expository language (which
we denote with $\alpha$, $\beta$, etc.) but are instead (meta)constant symbols
(members of $\mbox{\em SM}$) from the point of view of our expository
language.  The equivalent system of propositional calculus described in
\cite{Tarski1965} also uses the symbols $\phi$, $\psi$, etc.\ to denote wff
metavariables, but in \cite{Tarski1965} unlike here those are metavariables of
the expository language and not primitive symbols of the formal system.

The first two statements define wffs: if $\varphi$ and $\psi$ are wffs, so is
$(\varphi \to \psi)$; if $\varphi$ is a wff, so is $\lnot\varphi$. The next
three are the axioms of propositional calculus: if $\varphi$ and $\psi$ are
wffs, then $\vdash (\varphi \to (\psi \to \varphi))$ is an (axiomatic)
theorem; etc. The
last is the rule of modus ponens: if $\varphi$ and $\psi$ are wffs, and
$\vdash (\varphi\to\psi)$ and $\vdash \varphi$ are theorems, then $\vdash
\psi$ is a theorem.

The correspondence to ordinary propositional calculus is as follows.  We
consider only provable statements of the form $\langle\varnothing,
T,\varnothing,A\rangle$ with $T$ defined as above.  The first term of the
assertion $A$ of any such statement is either ``wff'' or ``$\vdash$''.  A
statement for which the first term is ``wff'' is a {\em wff} of propositional
calculus, and one where the first term is ``$\vdash$'' is a {\em
theorem (scheme)} of propositional calculus.

The universe of this formal system also contains many other provable
statements.  Those with distinct-variable restrictions are irrelevant because
propositional calculus has no constraints on substitutions.  Those that have
logical hypotheses we call {\em inferences}\index{inference} when
the logical hypotheses are of the form
$\langle\vdash\rangle\frown w$ where $w$ is a wff (with the leading constant
term ``wff'' removed).  Inferences (other than the modus ponens rule) are not a
proper part of propositional calculus but are convenient to use when building a
hierarchy of provable statements.  A provable statement with a nonsense
hypothesis such as $\langle \to,\vdash,\lnot\rangle$, and this same expression
as its assertion, we consider irrelevant; no use can be made of it in
proving theorems, since there is no way to eliminate the nonsense hypothesis.

{\footnotesize\begin{quotation}
{\em Comment.} Our use of parentheses in the definition of a wff illustrates
how axiomatic statements should be carefully stated in a way that
ties in unambiguously with the substitutions allowed by the formal system.
There are many ways we could have defined wffs---for example, Polish
prefix notation would have allowed us to omit parentheses entirely, at
the expense of readability---but we must define them in a way that is
unambiguous.  For example, if we had omitted parentheses from the
definition of $(\varphi\to \psi)$, the wff $\lnot\varphi\to \psi$ could
be interpreted as either $\lnot(\varphi\to\psi)$ or $(\lnot\varphi\to\psi)$
and would have allowed us to prove nonsense.  Note that there is no
concept of operator binding precedence built into our formal system.
\end{quotation}}

\begin{sloppy}
\subsection{Example~2---Predicate Calculus with Equality}\index{predicate
calculus}
\end{sloppy}

Here we extend Example~1 to include predicate calculus with equality,
illustrating the use of distinct-variable restrictions.  This system is the
same as Tarski's system $\mathfrak{S}_2$ in \cite{Tarski1965} (except that the
axioms of propositional calculus are different but equivalent, and a redundant
axiom is omitted).  We extend $\mbox{\em CN}$ with the constants
$\{\mbox{var},\forall,=\}$.  We extend $\mbox{\em VR}$ with an infinite set of
{\em individual metavariables}\index{individual
metavariable} $\{x,y,z,\ldots\}$ and denote this subset
$\mbox{\em Vr}$.

We also join to $\mbox{\em CN}$ a possibly infinite set $\mbox{\em Pr}$ of {\em
predicates} $\{R,S,\ldots\}$.  We associate with $\mbox{\em Pr}$ a function
$\mbox{rnk}$ from $\mbox{\em Pr}$ to $\omega$, and for $\alpha\in \mbox{\em
Pr}$ we call $\mbox{rnk}(\alpha)$ the {\em rank} of the predicate $\alpha$,
which is simply the number of ``arguments'' that the predicate has.  (Most
applications of predicate calculus will have a finite number of predicates;
for example, set theory has the single two-argument or binary predicate $\in$,
which is usually written with its arguments surrounding the predicate symbol
rather than with the prefix notation we will use for the general case.)  As a
device to facilitate our discussion, we will let $\mbox{\em Vs}$ be any fixed
one-to-one function from $\omega$ to $\mbox{\em Vr}$; thus $\mbox{\em Vs}$ is
any simple infinite sequence of individual metavariables with no repeating
terms.

In this example we will not include the function symbols that are often part of
formalizations of predicate calculus.  Using metalogical arguments that are
beyond the scope of our discussion, it can be shown that our formalization is
equivalent when functions are introduced via appropriate definitions.

We extend the set $T$ defined in Example~1 with the expressions
$\{\langle \mbox{var\ } x\rangle,$ $ \langle \mbox{var\ } y\rangle, \langle
\mbox{var\ } z\rangle,\ldots\}$.  We extend the $\Gamma$ above
with the axiomatic statements that are the reducts of the following
pre-statements:
\begin{list}{}{\itemsep 0.0pt}
      \item[] $\langle\varnothing,T,\varnothing,
               \langle \mbox{wff\ }\forall x\,\varphi\rangle\rangle$
      \item[] $\langle\varnothing,T,\varnothing,
               \langle \mbox{wff\ }x=y\rangle\rangle$
      \item[] $\langle\varnothing,T,
               \{\langle\vdash\varphi\rangle\},
               \langle\vdash\forall x\,\varphi\rangle\rangle$
      \item[] $\langle\varnothing,T,\varnothing,
               \langle \vdash((\forall x(\varphi\to\psi)
                  \to(\forall x\,\varphi\to\forall x\,\psi))
               \rangle\rangle$
      \item[] $\langle\{\{x,\varphi\}\},T,\varnothing,
               \langle \vdash(\varphi\to\forall x\,\varphi)
               \rangle\rangle$
      \item[] $\langle\{\{x,y\}\},T,\varnothing,
               \langle \vdash\lnot\forall x\lnot x=y
               \rangle\rangle$
      \item[] $\langle\varnothing,T,\varnothing,
               \langle \vdash(x=z
                  \to(x=y\to z=y))
               \rangle\rangle$
      \item[] $\langle\varnothing,T,\varnothing,
               \langle \vdash(y=z
                  \to(x=y\to x=z))
               \rangle\rangle$
\end{list}
These are the axioms not involving predicate symbols. The first two statements
extend the definition of a wff.  The third is the rule of generalization.  The
fifth states, in effect, ``For a wff $\varphi$ and variable $x$,
$\vdash(\varphi\to\forall x\,\varphi)$, provided that $x$ does not occur in
$\varphi$.''  The sixth states ``For variables $x$ and $y$,
$\vdash\lnot\forall x\lnot x = y$, provided that $x$ and $y$ are distinct.''
(This proviso is not necessary but was included by Tarski to
weaken the axiom and still show that the system is logically complete.)

Finally, for each predicate symbol $\alpha\in \mbox{\em Pr}$, we add to
$\Gamma$ an axiomatic statement, extending the definition of wff,
that is the reduct of the following pre-statement:
\begin{displaymath}
    \langle\varnothing,T,\varnothing,
            \langle \mbox{wff},\alpha\rangle\
            \frown \mbox{\em Vs}\restriction\mbox{rnk}(\alpha)\rangle
\end{displaymath}
and for each $\alpha\in \mbox{\em Pr}$ and each $n < \mbox{rnk}(\alpha)$
we add to $\Gamma$ an equality axiom that is the reduct of the
following pre-statement:
\begin{eqnarray*}
    \lefteqn{\langle\varnothing,T,\varnothing,
            \langle
      \vdash,(,\mbox{\em Vs}_n,=,\mbox{\em Vs}_{\mbox{rnk}(\alpha)},\to,
            (,\alpha\rangle\frown \mbox{\em Vs}\restriction\mbox{rnk}(\alpha)} \\
  & & \frown
            \langle\to,\alpha\rangle\frown \mbox{\em Vs}\restriction n\frown
            \langle \mbox{\em Vs}_{\mbox{rnk}(\alpha)}\rangle \\
 & & \frown
            \mbox{\em Vs}\restriction(\mbox{rnk}(\alpha)\setminus(n+1))\frown
            \langle),)\rangle\rangle
\end{eqnarray*}
where $\restriction$ denotes function domain restriction and $\setminus$
denotes set difference.  Recall that a subscript on $\mbox{\em Vs}$
denotes one of its terms.  (In the above two axiom sets commas are placed
between successive terms of sequences to prevent ambiguity, and if you examine
them with care you will be able to distinguish those parentheses that denote
constant symbols from those of our expository language that delimit function
arguments.  Although it might have been better to use boldface for our
primitive symbols, unfortunately boldface was not available for all characters
on the \LaTeX\ system used to typeset this text.)  These seemingly forbidding
axioms can be understood by analogy to concatenation of substrings in a
computer language.  They are actually relatively simple for each specific case
and will become clearer by looking at the special case of a binary predicate
$\alpha = R$ where $\mbox{rnk}(R)=2$.  Letting $\mbox{\em Vs}$ be the sequence
$\langle x,y,z,\ldots\rangle$, the axioms we would add to $\Gamma$ for this
case would be the wff extension and two equality axioms that are the
reducts of the pre-statements:
\begin{list}{}{\itemsep 0.0pt}
      \item[] $\langle\varnothing,T,\varnothing,
               \langle \mbox{wff\ }R x y\rangle\rangle$
      \item[] $\langle\varnothing,T,\varnothing,
               \langle \vdash(x=z
                  \to(R x y \to R z y))
               \rangle\rangle$
      \item[] $\langle\varnothing,T,\varnothing,
               \langle \vdash(y=z
                  \to(R x y \to R x z))
               \rangle\rangle$
\end{list}
Study these carefully to see how the general axioms above evaluate to
them.  In practice, typically only a few special cases such as this would be
needed, and in any case the Metamath language will only permit us to describe
a finite number of predicates, as opposed to the infinite number permitted by
the formal system.  (If an infinite number should be needed for some reason,
we could not define the formal system directly in the Metamath language but
could instead define it metalogically under set theory as we
do in this appendix, and only the underlying set theory, with its single
binary predicate, would be defined directly in the Metamath language.)


{\footnotesize\begin{quotation}
{\em Comment.}  As we noted earlier, the specific variables denoted by the
symbols $x,y,z,\ldots\in \mbox{\em Vr}\subseteq \mbox{\em VR}\subseteq
\mbox{\em SM}$ in Example~2 are not the actual variables of ordinary predicate
calculus but should be thought of as metavariables ranging over them.  For
example, a distinct-variable restriction would be meaningless for actual
variables of ordinary predicate calculus since two different actual variables
are by definition distinct.  And when we talk about an arbitrary
representative $\alpha\in \mbox{\em Vr}$, $\alpha$ is a metavariable (in our
expository language) that ranges over metavariables (which are primitives of
our formal system) each of which ranges over the actual individual variables
of predicate calculus (which are never mentioned in our formal system).

The constant called ``var'' above is called \texttt{setvar} in the
\texttt{set.mm} database file, but it means the same thing.  I felt
that ``var'' is a more meaningful name in the context of predicate
calculus, whose use is not limited to set theory.  For consistency we
stick with the name ``var'' throughout this Appendix, even after set
theory is introduced.
\end{quotation}}

\subsection{Free Variables and Proper Substitution}\index{free variable}
\index{proper substitution}\index{substitution!proper}

Typical representations of mathematical axioms use concepts such
as ``free variable,'' ``bound variable,'' and ``proper substitution''
as primitive notions.
A free variable is a variable that
is not a parameter of any container expression.
A bound variable is the opposite of a free variable; it is a
a variable that has been bound in a container expression.
For example, in the expression $\forall x \varphi$ (for all $x$, $\varphi$
is true), the variable $x$
is bound within the for-all ($\forall$) expression.
It is possible to change one variable to another, and that process is called
``proper substitution.''
In most books, proper substitution has a somewhat complicated recursive
definition with multiple cases based on the occurrences of free and
bound variables.
You may consult
\cite[ch.\ 3--4]{Hamilton}\index{Hamilton, Alan G.} (as well as
many other texts) for more formal details about these terms.

Using these concepts as \texttt{primitives} creates complications
for computer implementations.

In the system of Example~2, there are no primitive notions of free variable
and proper substitution.  Tarski \cite{Tarski1965} shows that this system is
logically equivalent to the more typical textbook systems that do have these
primitive notions, if we introduce these notions with appropriate definitions
and metalogic.  We could also define axioms for such systems directly,
although the recursive definitions of free variable and proper substitution
would be messy and awkward to work with.  Instead, we mention two devices that
can be used in practice to mimic these notions.  (1) Instead of introducing
special notation to express (as a logical hypothesis) ``where $x$ is not free
in $\varphi$'' we can use the logical hypothesis $\vdash(\varphi\to\forall
x\,\varphi)$.\label{effectivelybound}\index{effectively
not free}\footnote{This is a slightly weaker requirement than ``where $x$ is
not free in $\varphi$.''  If we let $\varphi$ be $x=x$, we have the theorem
$(x=x\to\forall x\,x=x)$ which satisfies the hypothesis, even though $x$ is
free in $x=x$ .  In a case like this we say that $x$ is {\em effectively not
free}\index{effectively not free} in $x=x$, since $x=x$ is logically
equivalent to $\forall x\,x=x$ in which $x$ is bound.} (2) It can be shown
that the wff $((x=y\to\varphi)\wedge\exists x(x=y\wedge\varphi))$ (with the
usual definitions of $\wedge$ and $\exists$; see Example~4 below) is logically
equivalent to ``the wff that results from proper substitution of $y$ for $x$
in $\varphi$.''  This works whether or not $x$ and $y$ are distinct.

\subsection{Metalogical Completeness}\index{metalogical completeness}

In the system of Example~2, the
following are provable pre-statements (and their reducts are
provable statements):
\begin{eqnarray*}
      & \langle\{\{x,y\}\},T,\varnothing,
               \langle \vdash\lnot\forall x\lnot x=y
               \rangle\rangle & \\
     &  \langle\varnothing,T,\varnothing,
               \langle \vdash\lnot\forall x\lnot x=x
               \rangle\rangle &
\end{eqnarray*}
whereas the following pre-statement is not to my knowledge provable (but
in any case we will pretend it's not for sake of illustration):
\begin{eqnarray*}
     &  \langle\varnothing,T,\varnothing,
               \langle \vdash\lnot\forall x\lnot x=y
               \rangle\rangle &
\end{eqnarray*}
In other words, we can prove ``$\lnot\forall x\lnot x=y$ where $x$ and $y$ are
distinct'' and separately prove ``$\lnot\forall x\lnot x=x$'', but we can't
prove the combined general case ``$\lnot\forall x\lnot x=y$'' that has no
proviso.  Now this does not compromise logical completeness, because the
variables are really metavariables and the two provable cases together cover
all possible cases.  The third case can be considered a metatheorem whose
direct proof, using the system of Example~2, lies outside the capability of the
formal system.

Also, in the system of Example~2 the following pre-statement is not to my
knowledge provable (again, a conjecture that we will pretend to be the case):
\begin{eqnarray*}
     & \langle\varnothing,T,\varnothing,
               \langle \vdash(\forall x\, \varphi\to\varphi)
               \rangle\rangle &
\end{eqnarray*}
Instead, we can only prove specific cases of $\varphi$ involving individual
metavariables, and by induction on formula length, prove as a metatheorem
outside of our formal system the general statement above.  The details of this
proof are found in \cite{Kalish}.

There does, however, exist a system of predicate calculus in which all such
``simple metatheorems'' as those above can be proved directly, and we present
it in Example~3. A {\em simple metatheorem}\index{simple metatheorem}
is any statement of the formal
system of Example~2 where all distinct variable restrictions consist of either
two individual metavariables or an individual metavariable and a wff
metavariable, and which is provable by combining cases outside the system as
above.  A system is {\em metalogically complete}\index{metalogical
completeness} if all of its simple
metatheorems are (directly) provable statements. The precise definition of
``simple metatheorem'' and the proof of the ``metalogical completeness'' of
Example~3 is found in Remark 9.6 and Theorem 9.7 of \cite{Megill}.\index{Megill,
Norman}

\begin{sloppy}
\subsection{Example~3---Metalogically Complete Predicate
Calculus with
Equality}
\end{sloppy}

For simplicity we will assume there is one binary predicate $R$;
this system suffices for set theory, where the $R$ is of course the $\in$
predicate.  We label the axioms as they appear in \cite{Megill}.  This
system is logically equivalent to that of Example~2 (when the latter is
restricted to this single binary predicate) but is also metalogically
complete.\index{metalogical completeness}

Let
\begin{itemize}
  \item[] $\mbox{\em CN}=\{\mbox{wff}, \mbox{var}, \vdash, \to, \lnot, (,),\forall,=,R\}$.
  \item[] $\mbox{\em VR}=\{\varphi,\psi,\chi,\ldots\}\cup\{x,y,z,\ldots\}$.
  \item[] $T = \{\langle \mbox{wff\ } \varphi\rangle,
             \langle \mbox{wff\ } \psi\rangle,
             \langle \mbox{wff\ } \chi\rangle,\ldots\}\cup
       \{\langle \mbox{var\ } x\rangle, \langle \mbox{var\ } y\rangle, \langle
       \mbox{var\ }z\rangle,\ldots\}$.

\noindent Then
  $\Gamma$ consists of the reducts of the following pre-statements:
    \begin{itemize}
      \item[] $\langle\varnothing,T,\varnothing,
               \langle \mbox{wff\ }(\varphi\to\psi)\rangle\rangle$
      \item[] $\langle\varnothing,T,\varnothing,
               \langle \mbox{wff\ }\lnot\varphi\rangle\rangle$
      \item[] $\langle\varnothing,T,\varnothing,
               \langle \mbox{wff\ }\forall x\,\varphi\rangle\rangle$
      \item[] $\langle\varnothing,T,\varnothing,
               \langle \mbox{wff\ }x=y\rangle\rangle$
      \item[] $\langle\varnothing,T,\varnothing,
               \langle \mbox{wff\ }Rxy\rangle\rangle$
      \item[(C1$'$)] $\langle\varnothing,T,\varnothing,
               \langle \vdash(\varphi\to(\psi\to\varphi))
               \rangle\rangle$
      \item[(C2$'$)] $\langle\varnothing,T,
               \varnothing,
               \langle \vdash((\varphi\to(\psi\to\chi))\to
               ((\varphi\to\psi)\to(\varphi\to\chi)))
               \rangle\rangle$
      \item[(C3$'$)] $\langle\varnothing,T,
               \varnothing,
               \langle \vdash((\lnot\varphi\to\lnot\psi)\to
               (\psi\to\varphi))\rangle\rangle$
      \item[(C4$'$)] $\langle\varnothing,T,
               \varnothing,
               \langle \vdash(\forall x(\forall x\,\varphi\to\psi)\to
                 (\forall x\,\varphi\to\forall x\,\psi))\rangle\rangle$
      \item[(C5$'$)] $\langle\varnothing,T,
               \varnothing,
               \langle \vdash(\forall x\,\varphi\to\varphi)\rangle\rangle$
      \item[(C6$'$)] $\langle\varnothing,T,
               \varnothing,
               \langle \vdash(\forall x\forall y\,\varphi\to
                 \forall y\forall x\,\varphi)\rangle\rangle$
      \item[(C7$'$)] $\langle\varnothing,T,
               \varnothing,
               \langle \vdash(\lnot\varphi\to\forall x\lnot\forall x\,\varphi
                 )\rangle\rangle$
      \item[(C8$'$)] $\langle\varnothing,T,
               \varnothing,
               \langle \vdash(x=y\to(x=z\to y=z))\rangle\rangle$
      \item[(C9$'$)] $\langle\varnothing,T,
               \varnothing,
               \langle \vdash(\lnot\forall x\, x=y\to(\lnot\forall x\, x=z\to
                 (y=z\to\forall x\, y=z)))\rangle\rangle$
      \item[(C10$'$)] $\langle\varnothing,T,
               \varnothing,
               \langle \vdash(\forall x(x=y\to\forall x\,\varphi)\to
                 \varphi))\rangle\rangle$
      \item[(C11$'$)] $\langle\varnothing,T,
               \varnothing,
               \langle \vdash(\forall x\, x=y\to(\forall x\,\varphi
               \to\forall y\,\varphi))\rangle\rangle$
      \item[(C12$'$)] $\langle\varnothing,T,
               \varnothing,
               \langle \vdash(x=y\to(Rxz\to Ryz))\rangle\rangle$
      \item[(C13$'$)] $\langle\varnothing,T,
               \varnothing,
               \langle \vdash(x=y\to(Rzx\to Rzy))\rangle\rangle$
      \item[(C15$'$)] $\langle\varnothing,T,
               \varnothing,
               \langle \vdash(\lnot\forall x\, x=y\to(x=y\to(\varphi
                 \to\forall x(x=y\to\varphi))))\rangle\rangle$
      \item[(C16$'$)] $\langle\{\{x,y\}\},T,
               \varnothing,
               \langle \vdash(\forall x\, x=y\to(\varphi\to\forall x\,\varphi)
                 )\rangle\rangle$
      \item[(C5)] $\langle\{\{x,\varphi\}\},T,\varnothing,
               \langle \vdash(\varphi\to\forall x\,\varphi)
               \rangle\rangle$
      \item[(MP)] $\langle\varnothing,T,
               \{\langle\vdash(\varphi\to\psi)\rangle,
                 \langle\vdash\varphi\rangle\},
               \langle\vdash\psi\rangle\rangle$
      \item[(Gen)] $\langle\varnothing,T,
               \{\langle\vdash\varphi\rangle\},
               \langle\vdash\forall x\,\varphi\rangle\rangle$
    \end{itemize}
\end{itemize}

While it is known that these axioms are ``metalogically complete,'' it is
not known whether they are independent (i.e.\ none is
redundant) in the metalogical sense; specifically, whether any axiom (possibly
with additional non-mandatory distinct-variable restrictions, for use with any
dummy variables in its proof) is provable from the others.  Note that
metalogical independence is a weaker requirement than independence in the
usual logical sense.  Not all of the above axioms are logically independent:
for example, C9$'$ can be proved as a metatheorem from the others, outside the
formal system, by combining the possible cases of distinct variables.

\subsection{Example~4---Adding Definitions}\index{definition}
There are several ways to add definitions to a formal system.  Probably the
most proper way is to consider definitions not as part of the formal system at
all but rather as abbreviations that are part of the expository metalogic
outside the formal system.  For convenience, though, we may use the formal
system itself to incorporate definitions, adding them as axiomatic extensions
to the system.  This could be done by adding a constant representing the
concept ``is defined as'' along with axioms for it. But there is a nicer way,
at least in this writer's opinion, that introduces definitions as direct
extensions to the language rather than as extralogical primitive notions.  We
introduce additional logical connectives and provide axioms for them.  For
systems of logic such as Examples 1 through 3, the additional axioms must be
conservative in the sense that no wff of the original system that was not a
theorem (when the initial term ``wff'' is replaced by ``$\vdash$'' of course)
becomes a theorem of the extended system.  In this example we extend Example~3
(or 2) with standard abbreviations of logic.

We extend $\mbox{\em CN}$ of Example~3 with new constants $\{\leftrightarrow,
\wedge,\vee,\exists\}$, corresponding to logical equivalence,\index{logical
equivalence ($\leftrightarrow$)}\index{biconditional ($\leftrightarrow$)}
conjunction,\index{conjunction ($\wedge$)} disjunction,\index{disjunction
($\vee$)} and the existential quantifier.\index{existential quantifier
($\exists$)}  We extend $\Gamma$ with the axiomatic statements that are
the reducts of the following pre-statements:
\begin{list}{}{\itemsep 0.0pt}
      \item[] $\langle\varnothing,T,\varnothing,
               \langle \mbox{wff\ }(\varphi\leftrightarrow\psi)\rangle\rangle$
      \item[] $\langle\varnothing,T,\varnothing,
               \langle \mbox{wff\ }(\varphi\vee\psi)\rangle\rangle$
      \item[] $\langle\varnothing,T,\varnothing,
               \langle \mbox{wff\ }(\varphi\wedge\psi)\rangle\rangle$
      \item[] $\langle\varnothing,T,\varnothing,
               \langle \mbox{wff\ }\exists x\, \varphi\rangle\rangle$
  \item[] $\langle\varnothing,T,\varnothing,
     \langle\vdash ( ( \varphi \leftrightarrow \psi ) \to
     ( \varphi \to \psi ) )\rangle\rangle$
  \item[] $\langle\varnothing,T,\varnothing,
     \langle\vdash ((\varphi\leftrightarrow\psi)\to
    (\psi\to\varphi))\rangle\rangle$
  \item[] $\langle\varnothing,T,\varnothing,
     \langle\vdash ((\varphi\to\psi)\to(
     (\psi\to\varphi)\to(\varphi
     \leftrightarrow\psi)))\rangle\rangle$
  \item[] $\langle\varnothing,T,\varnothing,
     \langle\vdash (( \varphi \wedge \psi ) \leftrightarrow\neg ( \varphi
     \to \neg \psi )) \rangle\rangle$
  \item[] $\langle\varnothing,T,\varnothing,
     \langle\vdash (( \varphi \vee \psi ) \leftrightarrow (\neg \varphi
     \to \psi )) \rangle\rangle$
  \item[] $\langle\varnothing,T,\varnothing,
     \langle\vdash (\exists x \,\varphi\leftrightarrow
     \lnot \forall x \lnot \varphi)\rangle\rangle$
\end{list}
The first three logical axioms (statements containing ``$\vdash$'') introduce
and effectively define logical equivalence, ``$\leftrightarrow$''.  The last
three use ``$\leftrightarrow$'' to effectively mean ``is defined as.''

\subsection{Example~5---ZFC Set Theory}\index{ZFC set theory}

Here we add to the system of Example~4 the axioms of Zermelo--Fraenkel set
theory with Choice.  For convenience we make use of the
definitions in Example~4.

In the $\mbox{\em CN}$ of Example~4 (which extends Example~3), we replace the symbol $R$
with the symbol $\in$.
More explicitly, we remove from $\Gamma$ of Example~4 the three
axiomatic statements containing $R$ and replace them with the
reducts of the following:
\begin{list}{}{\itemsep 0.0pt}
      \item[] $\langle\varnothing,T,\varnothing,
               \langle \mbox{wff\ }x\in y\rangle\rangle$
      \item[] $\langle\varnothing,T,
               \varnothing,
               \langle \vdash(x=y\to(x\in z\to y\in z))\rangle\rangle$
      \item[] $\langle\varnothing,T,
               \varnothing,
               \langle \vdash(x=y\to(z\in x\to z\in y))\rangle\rangle$
\end{list}
Letting $D=\{\{\alpha,\beta\}\in \mbox{\em DV}\,|\alpha,\beta\in \mbox{\em
Vr}\}$ (in other words all individual variables must be distinct), we extend
$\Gamma$ with the ZFC axioms, called
\index{Axiom of Extensionality}
\index{Axiom of Replacement}
\index{Axiom of Union}
\index{Axiom of Power Sets}
\index{Axiom of Regularity}
\index{Axiom of Infinity}
\index{Axiom of Choice}
Extensionality, Replacement, Union, Power
Set, Regularity, Infinity, and Choice, that are the reducts of:
\begin{list}{}{\itemsep 0.0pt}
      \item[Ext] $\langle D,T,
               \varnothing,
               \langle\vdash (\forall x(x\in y\leftrightarrow x \in z)\to y
               =z) \rangle\rangle$
      \item[Rep] $\langle D,T,
               \varnothing,
               \langle\vdash\exists x ( \exists y \forall z (\varphi \to z = y
                        ) \to
                        \forall z ( z \in x \leftrightarrow \exists x ( x \in
                        y \wedge \forall y\,\varphi ) ) )\rangle\rangle$
      \item[Un] $\langle D,T,
               \varnothing,
               \langle\vdash \exists x \forall y ( \exists x ( y \in x \wedge
               x \in z ) \to y \in x ) \rangle\rangle$
      \item[Pow] $\langle D,T,
               \varnothing,
               \langle\vdash \exists x \forall y ( \forall x ( x \in y \to x
               \in z ) \to y \in x ) \rangle\rangle$
      \item[Reg] $\langle D,T,
               \varnothing,
               \langle\vdash (  x \in y \to
                 \exists x ( x \in y \wedge \forall z ( z \in x \to \lnot z
                \in y ) ) ) \rangle\rangle$
      \item[Inf] $\langle D,T,
               \varnothing,
               \langle\vdash \exists x(y\in x\wedge\forall y(y\in
               x\to
               \exists z(y \in z\wedge z\in x))) \rangle\rangle$
      \item[AC] $\langle D,T,
               \varnothing,
               \langle\vdash \exists x \forall y \forall z ( ( y \in z
               \wedge z \in w ) \to \exists w \forall y ( \exists w
              ( ( y \in z \wedge z \in w ) \wedge ( y \in w \wedge w \in x
              ) ) \leftrightarrow y = w ) ) \rangle\rangle$
\end{list}

\subsection{Example~6---Class Notation in Set Theory}\label{class}

A powerful device that makes set theory easier (and that we have
been using all along in our informal expository language) is {\em class
abstraction notation}.\index{class abstraction}\index{abstraction class}  The
definitions we introduce are rigorously justified
as conservative by Takeuti and Zaring \cite{Takeuti}\index{Takeuti, Gaisi} or
Quine \cite{Quine}\index{Quine, Willard Van Orman}.  The key idea is to
introduce the notation $\{x|\mbox{---}\}$ which means ``the class of all $x$
such that ---'' for abstraction classes and introduce (meta)variables that
range over them.  An abstraction class may or may not be a set, depending on
whether it exists (as a set).  A class that does not exist is
called a {\em proper class}.\index{proper class}\index{class!proper}

To illustrate the use of abstraction classes we will provide some examples
of definitions that make use of them:  the empty set, class union, and
unordered pair.  Many other such definitions can be found in the
Metamath set theory database,
\texttt{set.mm}.\index{set theory database (\texttt{set.mm})}

% We intentionally break up the sequence of math symbols here
% because otherwise the overlong line goes beyond the page in narrow mode.
We extend $\mbox{\em CN}$ of Example~5 with new symbols $\{$
$\mbox{class},$ $\{,$ $|,$ $\},$ $\varnothing,$ $\cup,$ $,$ $\}$
where the inner braces and last comma are
constant symbols. (As before,
our dual use of some mathematical symbols for both our expository
language and as primitives of the formal system should be clear from context.)

We extend $\mbox{\em VR}$ of Example~5 with a set of {\em class
variables}\index{class variable}
$\{A,B,C,\ldots\}$. We extend the $T$ of Example~5 with $\{\langle
\mbox{class\ } A\rangle, \langle \mbox{class\ }B\rangle, \langle \mbox{class\ }
C\rangle,\ldots\}$.

To
introduce our definitions,
we add to $\Gamma$ of Example~5 the axiomatic statements
that are the reducts of the following pre-statements:
\begin{list}{}{\itemsep 0.0pt}
      \item[] $\langle\varnothing,T,\varnothing,
               \langle \mbox{class\ }x\rangle\rangle$
      \item[] $\langle\varnothing,T,\varnothing,
               \langle \mbox{class\ }\{x|\varphi\}\rangle\rangle$
      \item[] $\langle\varnothing,T,\varnothing,
               \langle \mbox{wff\ }A=B\rangle\rangle$
      \item[] $\langle\varnothing,T,\varnothing,
               \langle \mbox{wff\ }A\in B\rangle\rangle$
      \item[Ab] $\langle\varnothing,T,\varnothing,
               \langle \vdash ( y \in \{ x |\varphi\} \leftrightarrow
                  ( ( x = y \to\varphi) \wedge \exists x ( x = y
                  \wedge\varphi) ))
               \rangle\rangle$
      \item[Eq] $\langle\{\{x,A\},\{x,B\}\},T,\varnothing,
               \langle \vdash ( A = B \leftrightarrow
               \forall x ( x \in A \leftrightarrow x \in B ) )
               \rangle\rangle$
      \item[El] $\langle\{\{x,A\},\{x,B\}\},T,\varnothing,
               \langle \vdash ( A \in B \leftrightarrow \exists x
               ( x = A \wedge x \in B ) )
               \rangle\rangle$
\end{list}
Here we say that an individual variable is a class; $\{x|\varphi\}$ is a
class; and we extend the definition of a wff to include class equality and
membership.  Axiom Ab defines membership of a variable in a class abstraction;
the right-hand side can be read as ``the wff that results from proper
substitution of $y$ for $x$ in $\varphi$.''\footnote{Note that this definition
makes unnecessary the introduction of a separate notation similar to
$\varphi(x|y)$ for proper substitution, although we may choose to do so to be
conventional.  Incidentally, $\varphi(x|y)$ as it stands would be ambiguous in
the formal systems of our examples, since we wouldn't know whether
$\lnot\varphi(x|y)$ meant $\lnot(\varphi(x|y))$ or $(\lnot\varphi)(x|y)$.
Instead, we would have to use an unambiguous variant such as $(\varphi\,
x|y)$.}  Axioms Eq and El extend the meaning of the existing equality and
membership connectives.  This is potentially dangerous and requires careful
justification.  For example, from Eq we can derive the Axiom of Extensionality
with predicate logic alone; thus in principle we should include the Axiom of
Extensionality as a logical hypothesis.  However we do not bother to do this
since we have already presupposed that axiom earlier. The distinct variable
restrictions should be read ``where $x$ does not occur in $A$ or $B$.''  We
typically do this when the right-hand side of a definition involves an
individual variable not in the expression being defined; it is done so that
the right-hand side remains independent of the particular ``dummy'' variable
we use.

We continue to add to $\Gamma$ the following definitions
(i.e. the reducts of the following pre-statements) for empty
set,\index{empty set} class union,\index{union} and unordered
pair.\index{unordered pair}  They should be self-explanatory.  Analogous to our
use of ``$\leftrightarrow$'' to define new wffs in Example~4, we use ``$=$''
to define new abstraction terms, and both may be read informally as ``is
defined as'' in this context.
\begin{list}{}{\itemsep 0.0pt}
      \item[] $\langle\varnothing,T,\varnothing,
               \langle \mbox{class\ }\varnothing\rangle\rangle$
      \item[] $\langle\varnothing,T,\varnothing,
               \langle \vdash \varnothing = \{ x | \lnot x = x \}
               \rangle\rangle$
      \item[] $\langle\varnothing,T,\varnothing,
               \langle \mbox{class\ }(A\cup B)\rangle\rangle$
      \item[] $\langle\{\{x,A\},\{x,B\}\},T,\varnothing,
               \langle \vdash ( A \cup B ) = \{ x | ( x \in A \vee x \in B ) \}
               \rangle\rangle$
      \item[] $\langle\varnothing,T,\varnothing,
               \langle \mbox{class\ }\{A,B\}\rangle\rangle$
      \item[] $\langle\{\{x,A\},\{x,B\}\},T,\varnothing,
               \langle \vdash \{ A , B \} = \{ x | ( x = A \vee x = B ) \}
               \rangle\rangle$
\end{list}

\section{Metamath as a Formal System}\label{theorymm}

This section presupposes a familiarity with the Metamath computer language.

Our theory describes formal systems and their universes.  The Metamath
language provides a way of representing these set-theoretical objects to
a computer.  A Metamath database, being a finite set of {\sc ascii}
characters, can usually describe only a subset of a formal system and
its universe, which are typically infinite.  However the database can
contain as large a finite subset of the formal system and its universe
as we wish.  (Of course a Metamath set theory database can, in
principle, indirectly describe an entire infinite formal system by
formalizing the expository language in this Appendix.)

For purpose of our discussion, we assume the Metamath database
is in the simple form described on p.~\pageref{framelist},
consisting of all constant and variable declarations at the beginning,
followed by a sequence of extended frames each
delimited by \texttt{\$\char`\{} and \texttt{\$\char`\}}.  Any Metamath database can
be converted to this form, as described on p.~\pageref{frameconvert}.

The math symbol tokens of a Metamath source file, which are declared
with \texttt{\$c} and \texttt{\$v} statements, are names we assign to
representatives of $\mbox{\em CN}$ and $\mbox{\em VR}$.  For
definiteness we could assume that the first math symbol declared as a
variable corresponds to $v_0$, the second to $v_1$, etc., although the
exact correspondence we choose is not important.

In the Metamath language, each \texttt{\$d}, \texttt{\$f}, and
 \texttt{\$e} source
statement in an extended frame (Section~\ref{frames})
corresponds respectively to a member of the
collections $D$, $T$, and $H$ in a formal system statement $\langle
D_M,T_M,H,A\rangle$.  The math symbol strings following these Metamath keywords
correspond to a variable pair (in the case of \texttt{\$d}) or an expression (for
the other two keywords). The math symbol string following a \texttt{\$a} source
statement corresponds to expression $A$ in an axiomatic statement of the
formal system; the one following a \texttt{\$p} source statement corresponds to
$A$ in a provable statement that is not axiomatic.  In other words, each
extended frame in a Metamath database corresponds to
a pre-statement of the formal system, and a frame corresponds to
a statement of the formal system.  (Don't confuse the two meanings of
``statement'' here.  A statement of the formal system corresponds to the
several statements in a Metamath database that may constitute a
frame.)

In order for the computer to verify that a formal system statement is
provable, each \texttt{\$p} source statement is accompanied by a proof.
However, the proof does not correspond to anything in the formal system
but is simply a way of communicating to the computer the information
needed for its verification.  The proof tells the computer {\em how to
construct} specific members of closure of the formal system
pre-statement corresponding to the extended frame of the \texttt{\$p}
statement.  The final result of the construction is the member of the
closure that matches the \texttt{\$p} statement.  The abstract formal
system, on the other hand, is concerned only with the {\em existence} of
members of the closure.

As mentioned on p.~\pageref{exampleref}, Examples 1 and 3--6 in the
previous Section parallel the development of logic and set theory in the
Metamath database
\texttt{set.mm}.\index{set theory database (\texttt{set.mm})} You may
find it instructive to compare them.


\chapter{The MIU System}
\label{MIU}
\index{formal system}
\index{MIU-system}

The following is a listing of the file \texttt{miu.mm}.  It is self-explanatory.

%%%%%%%%%%%%%%%%%%%%%%%%%%%%%%%%%%%%%%%%%%%%%%%%%%%%%%%%%%%%

\begin{verbatim}
$( The MIU-system:  A simple formal system $)

$( Note:  This formal system is unusual in that it allows
empty wffs.  To work with a proof, you must type
SET EMPTY_SUBSTITUTION ON before using the PROVE command.
By default, this is OFF in order to reduce the number of
ambiguous unification possibilities that have to be selected
during the construction of a proof.  $)

$(
Hofstadter's MIU-system is a simple example of a formal
system that illustrates some concepts of Metamath.  See
Douglas R. Hofstadter, _Goedel, Escher, Bach:  An Eternal
Golden Braid_ (Vintage Books, New York, 1979), pp. 33ff. for
a description of the MIU-system.

The system has 3 constant symbols, M, I, and U.  The sole
axiom of the system is MI. There are 4 rules:
     Rule I:  If you possess a string whose last letter is I,
     you can add on a U at the end.
     Rule II:  Suppose you have Mx.  Then you may add Mxx to
     your collection.
     Rule III:  If III occurs in one of the strings in your
     collection, you may make a new string with U in place
     of III.
     Rule IV:  If UU occurs inside one of your strings, you
     can drop it.
Unfortunately, Rules III and IV do not have unique results:
strings could have more than one occurrence of III or UU.
This requires that we introduce the concept of an "MIU
well-formed formula" or wff, which allows us to construct
unique symbol sequences to which Rules III and IV can be
applied.
$)

$( First, we declare the constant symbols of the language.
Note that we need two symbols to distinguish the assertion
that a sequence is a wff from the assertion that it is a
theorem; we have arbitrarily chosen "wff" and "|-". $)
      $c M I U |- wff $. $( Declare constants $)

$( Next, we declare some variables. $)
     $v x y $.

$( Throughout our theory, we shall assume that these
variables represent wffs. $)
 wx   $f wff x $.
 wy   $f wff y $.

$( Define MIU-wffs.  We allow the empty sequence to be a
wff. $)

$( The empty sequence is a wff. $)
 we   $a wff $.
$( "M" after any wff is a wff. $)
 wM   $a wff x M $.
$( "I" after any wff is a wff. $)
 wI   $a wff x I $.
$( "U" after any wff is a wff. $)
 wU   $a wff x U $.

$( Assert the axiom. $)
 ax   $a |- M I $.

$( Assert the rules. $)
 ${
   Ia   $e |- x I $.
$( Given any theorem ending with "I", it remains a theorem
if "U" is added after it.  (We distinguish the label I_
from the math symbol I to conform to the 24-Jun-2006
Metamath spec.) $)
   I_    $a |- x I U $.
 $}
 ${
IIa  $e |- M x $.
$( Given any theorem starting with "M", it remains a theorem
if the part after the "M" is added again after it. $)
   II   $a |- M x x $.
 $}
 ${
   IIIa $e |- x I I I y $.
$( Given any theorem with "III" in the middle, it remains a
theorem if the "III" is replaced with "U". $)
   III  $a |- x U y $.
 $}
 ${
   IVa  $e |- x U U y $.
$( Given any theorem with "UU" in the middle, it remains a
theorem if the "UU" is deleted. $)
   IV   $a |- x y $.
  $}

$( Now we prove the theorem MUIIU.  You may be interested in
comparing this proof with that of Hofstadter (pp. 35 - 36).
$)
 theorem1  $p |- M U I I U $=
      we wM wU wI we wI wU we wU wI wU we wM we wI wU we wM
      wI wI wI we wI wI we wI ax II II I_ III II IV $.
\end{verbatim}\index{well-formed formula (wff)}

The \texttt{show proof /lemmon/renumber} command
yields the following display.  It is very similar
to the one in \cite[pp.~35--36]{Hofstadter}.\index{Hofstadter, Douglas R.}

\begin{verbatim}
1 ax             $a |- M I
2 1 II           $a |- M I I
3 2 II           $a |- M I I I I
4 3 I_           $a |- M I I I I U
5 4 III          $a |- M U I U
6 5 II           $a |- M U I U U I U
7 6 IV           $a |- M U I I U
\end{verbatim}

We note that Hofstadter's ``MU-puzzle,'' which asks whether
MU is a theorem of the MIU-system, cannot be answered using
the system above because the MU-puzzle is a question {\em
about} the system.  To prove the answer to the MU-puzzle,
a much more elaborate system is needed, namely one that
models the MIU-system within set theory.  (Incidentally, the
answer to the MU-puzzle is no.)

\chapter{Metamath Language EBNF}%
\label{BNF}%
\index{Metamath Language EBNF}

The following is a formal description of the basic Metamath language syntax
(with compressed proofs and support for unknown proof steps).
It is defined using the
Extended Backus--Naur Form (EBNF)\index{Extended Backus--Naur Form}\index{EBNF}
notation from W3C\index{W3C}
\textit{Extensible Markup Language (XML) 1.0 (Fifth Edition)}
(W3C Recommendation 26 November 2008) at
\url{https://www.w3.org/TR/xml/#sec-notation}.

The \texttt{database}
rule is processed until the end of the file (\texttt{EOF}).
The rules eventually require reading whitespace-separated tokens.
A token has an upper-case definition (see below)
or is a string constant in a non-token (such as \texttt{'\$a'}).
We intend for this to be correct, but if there is a conflict the
rules of section \ref{spec} govern. That section also discusses
non-syntax restrictions not shown here
(e.g., that each new label token
defined in a \texttt{hypothesis-stmt} or \texttt{assert-stmt}
must be unique).

\begin{verbatim}
database ::= outermost-scope-stmt*

outermost-scope-stmt ::=
  include-stmt | constant-stmt | stmt

/* File inclusion command; process file as a database.
   Databases should NOT have a comment in the filename. */
include-stmt ::= '$[' filename '$]'

/* Constant symbols declaration. */
constant-stmt ::= '$c' constant+ '$.'

/* A normal statement can occur in any scope. */
stmt ::= block | variable-stmt | disjoint-stmt |
  hypothesis-stmt | assert-stmt

/* A block. You can have 0 statements in a block. */
block ::= '${' stmt* '$}'

/* Variable symbols declaration. */
variable-stmt ::= '$v' variable+ '$.'

/* Disjoint variables. Simple disjoint statements have
   2 variables, i.e., "variable*" is empty for them. */
disjoint-stmt ::= '$d' variable variable variable* '$.'

hypothesis-stmt ::= floating-stmt | essential-stmt

/* Floating (variable-type) hypothesis. */
floating-stmt ::= LABEL '$f' typecode variable '$.'

/* Essential (logical) hypothesis. */
essential-stmt ::= LABEL '$e' typecode MATH-SYMBOL* '$.'

assert-stmt ::= axiom-stmt | provable-stmt

/* Axiomatic assertion. */
axiom-stmt ::= LABEL '$a' typecode MATH-SYMBOL* '$.'

/* Provable assertion. */
provable-stmt ::= LABEL '$p' typecode MATH-SYMBOL*
  '$=' proof '$.'

/* A proof. Proofs may be interspersed by comments.
   If '?' is in a proof it's an "incomplete" proof. */
proof ::= uncompressed-proof | compressed-proof
uncompressed-proof ::= (LABEL | '?')+
compressed-proof ::= '(' LABEL* ')' COMPRESSED-PROOF-BLOCK+

typecode ::= constant

filename ::= MATH-SYMBOL /* No whitespace or '$' */
constant ::= MATH-SYMBOL
variable ::= MATH-SYMBOL
\end{verbatim}

\needspace{2\baselineskip}
A \texttt{frame} is a sequence of 0 or more
\texttt{disjoint-{\allowbreak}stmt} and
\texttt{hypotheses-{\allowbreak}stmt} statements
(possibly interleaved with other non-\texttt{assert-stmt} statements)
followed by one \texttt{assert-stmt}.

\needspace{3\baselineskip}
Here are the rules for lexical processing (tokenization) beyond
the constant tokens shown above.
By convention these tokenization rules have upper-case names.
Every token is read for the longest possible length.
Whitespace-separated tokens are read sequentially;
note that the separating whitespace and \texttt{\$(} ... \texttt{\$)}
comments are skipped.

If a token definition uses another token definition, the whole thing
is considered a single token.
A pattern that is only part of a full token has a name beginning
with an underscore (``\_'').
An implementation could tokenize many tokens as a
\texttt{PRINTABLE-SEQUENCE}
and then check if it meets the more specific rule shown here.

Comments do not nest, and both \texttt{\$(} and \texttt{\$)}
have to be surrounded
by at least one whitespace character (\texttt{\_WHITECHAR}).
Technically comments end without consuming the trailing
\texttt{\_WHITECHAR}, but the trailing
\texttt{\_WHITECHAR} gets ignored anyway so we ignore that detail here.
Metamath language processors
are not required to support \texttt{\$)} followed
immediately by a bare end-of-file, because the closing
comment symbol is supposed to be followed by a
\texttt{\_WHITECHAR} such as a newline.

\begin{verbatim}
PRINTABLE-SEQUENCE ::= _PRINTABLE-CHARACTER+

MATH-SYMBOL ::= (_PRINTABLE-CHARACTER - '$')+

/* ASCII non-whitespace printable characters */
_PRINTABLE-CHARACTER ::= [#x21-#x7e]

LABEL ::= ( _LETTER-OR-DIGIT | '.' | '-' | '_' )+

_LETTER-OR-DIGIT ::= [A-Za-z0-9]

COMPRESSED-PROOF-BLOCK ::= ([A-Z] | '?')+

/* Define whitespace between tokens. The -> SKIP
   means that when whitespace is seen, it is
   skipped and we simply read again. */
WHITESPACE ::= (_WHITECHAR+ | _COMMENT) -> SKIP

/* Comments. $( ... $) and do not nest. */
_COMMENT ::= '$(' (_WHITECHAR+ (PRINTABLE-SEQUENCE - '$)')*
  _WHITECHAR+ '$)' _WHITECHAR

/* Whitespace: (' ' | '\t' | '\r' | '\n' | '\f') */
_WHITECHAR ::= [#x20#x09#x0d#x0a#x0c]
\end{verbatim}
% This EBNF was developed as a collaboration between
% David A. Wheeler\index{Wheeler, David A.},
% Mario Carneiro\index{Carneiro, Mario}, and
% Benoit Jubin\index{Jubin, Benoit}, inspired by a request
% (and a lot of initial work) by Benoit Jubin.
%
% \chapter{Disclaimer and Trademarks}
%
% Information in this document is subject to change without notice and does not
% represent a commitment on the part of Norman Megill.
% \vspace{2ex}
%
% \noindent Norman D. Megill makes no warranties, either express or implied,
% regarding the Metamath computer software package.
%
% \vspace{2ex}
%
% \noindent Any trademarks mentioned in this book are the property of
% their respective owners.  The name ``Metamath'' is a trademark of
% Norman Megill.
%
\cleardoublepage
\phantomsection  % fixes the link anchor
\addcontentsline{toc}{chapter}{\bibname}

\bibliography{metamath}
%\input{metamath.bbl}

\raggedright
\cleardoublepage
\phantomsection % fixes the link anchor
\addcontentsline{toc}{chapter}{\indexname}
%\printindex   ??
\input{metamath.ind}

\end{document}



\raggedright
\cleardoublepage
\phantomsection % fixes the link anchor
\addcontentsline{toc}{chapter}{\indexname}
%\printindex   ??
% metamath.tex - Version of 24-May-2019
% If you change the date above, also change the "Printed date" below.
% SPDX-License-Identifier: CC0-1.0
%
%                              PUBLIC DOMAIN
%
% This file (specifically, the version of this file with the above date)
% has been released into the Public Domain per the
% Creative Commons CC0 1.0 Universal (CC0 1.0) Public Domain Dedication
% https://creativecommons.org/publicdomain/zero/1.0/
%
% The public domain release applies worldwide.  In case this is not
% legally possible, the right is granted to use the work for any purpose,
% without any conditions, unless such conditions are required by law.
%
% Several short, attributed quotations from copyrighted works
% appear in this file under the ``fair use'' provision of Section 107 of
% the United States Copyright Act (Title 17 of the {\em United States
% Code}).  The public-domain status of this file is not applicable to
% those quotations.
%
% Norman Megill - email: nm(at)alum(dot)mit(dot)edu
%
% David A. Wheeler also donates his improvements to this file to the
% public domain per the CC0.  He works at the Institute for Defense Analyses
% (IDA), but IDA has agreed that this Metamath work is outside its "lane"
% and is not a work by IDA.  This was specifically confirmed by
% Margaret E. Myers (Division Director of the Information Technology
% and Systems Division) on 2019-05-24 and by Ben Lindorf (General Counsel)
% on 2019-05-22.

% This file, 'metamath.tex', is self-contained with everything needed to
% generate the the PDF file 'metamath.pdf' (the _Metamath_ book) on
% standard LaTeX 2e installations.  The auxiliary files are embedded with
% "filecontents" commands.  To generate metamath.pdf file, run these
% commands under Linux or Cygwin in the directory that contains
% 'metamath.tex':
%
%   rm -f realref.sty metamath.bib
%   touch metamath.ind
%   pdflatex metamath
%   pdflatex metamath
%   bibtex metamath
%   makeindex metamath
%   pdflatex metamath
%   pdflatex metamath
%
% The warnings that occur in the initial runs of pdflatex can be ignored.
% For the final run,
%
%   egrep -i 'error|warn' metamath.log
%
% should show exactly these 5 warnings:
%
%   LaTeX Warning: File `realref.sty' already exists on the system.
%   LaTeX Warning: File `metamath.bib' already exists on the system.
%   LaTeX Font Warning: Font shape `OMS/cmtt/m/n' undefined
%   LaTeX Font Warning: Font shape `OMS/cmtt/bx/n' undefined
%   LaTeX Font Warning: Some font shapes were not available, defaults
%       substituted.
%
% Search for "Uncomment" below if you want to suppress hyperlink boxes
% in the PDF output file
%
% TYPOGRAPHICAL NOTES:
% * It is customary to use an en dash (--) to "connect" names of different
%   people (and to denote ranges), and use a hyphen (-) for a
%   single compound name. Examples of connected multiple people are
%   Zermelo--Fraenkel, Schr\"{o}der--Bernstein, Tarski--Grothendieck,
%   Hewlett--Packard, and Backus--Naur.  Examples of a single person with
%   a compound name include Levi-Civita, Mittag-Leffler, and Burali-Forti.
% * Use non-breaking spaces after page abbreviations, e.g.,
%   p.~\pageref{note2002}.
%
% --------------------------- Start of realref.sty -----------------------------
\begin{filecontents}{realref.sty}
% Save the following as realref.sty.
% You can then use it with \usepackage{realref}
%
% This has \pageref jumping to the page on which the ref appears,
% \ref jumping to the point of the anchor, and \sectionref
% jumping to the start of section.
%
% Author:  Anthony Williams
%          Software Engineer
%          Nortel Networks Optical Components Ltd
% Date:    9 Nov 2001 (posted to comp.text.tex)
%
% The following declaration was made by Anthony Williams on
% 24 Jul 2006 (private email to Norman Megill):
%
%   ``I hereby donate the code for realref.sty posted on the
%   comp.text.tex newsgroup on 9th November 2001, accessible from
%   http://groups.google.com/group/comp.text.tex/msg/5a0e1cc13ea7fbb2
%   to the public domain.''
%
\ProvidesPackage{realref}
\RequirePackage[plainpages=false,pdfpagelabels=true]{hyperref}
\def\realref@anchorname{}
\AtBeginDocument{%
% ensure every label is a possible hyperlink target
\let\realref@oldrefstepcounter\refstepcounter%
\DeclareRobustCommand{\refstepcounter}[1]{\realref@oldrefstepcounter{#1}
\edef\realref@anchorname{\string #1.\@currentlabel}%
}%
\let\realref@oldlabel\label%
\DeclareRobustCommand{\label}[1]{\realref@oldlabel{#1}\hypertarget{#1}{}%
\@bsphack\protected@write\@auxout{}{%
    \string\expandafter\gdef\protect\csname
    page@num.#1\string\endcsname{\thepage}%
    \string\expandafter\gdef\protect\csname
    ref@num.#1\string\endcsname{\@currentlabel}%
    \string\expandafter\gdef\protect\csname
    sectionref@name.#1\string\endcsname{\realref@anchorname}%
}\@esphack}%
\DeclareRobustCommand\pageref[1]{{\edef\a{\csname
            page@num.#1\endcsname}\expandafter\hyperlink{page.\a}{\a}}}%
\DeclareRobustCommand\ref[1]{{\edef\a{\csname
            ref@num.#1\endcsname}\hyperlink{#1}{\a}}}%
\DeclareRobustCommand\sectionref[1]{{\edef\a{\csname
            ref@num.#1\endcsname}\edef\b{\csname
            sectionref@name.#1\endcsname}\hyperlink{\b}{\a}}}%
}
\end{filecontents}
% ---------------------------- End of realref.sty ------------------------------

% --------------------------- Start of metamath.bib -----------------------------
\begin{filecontents}{metamath.bib}
@book{Albers, editor = "Donald J. Albers and G. L. Alexanderson",
  title = "Mathematical People",
  publisher = "Contemporary Books, Inc.",
  address = "Chicago",
  note = "[QA28.M37]",
  year = 1985 }
@book{Anderson, author = "Alan Ross Anderson and Nuel D. Belnap",
  title = "Entailment",
  publisher = "Princeton University Press",
  address = "Princeton",
  volume = 1,
  note = "[QA9.A634 1975 v.1]",
  year = 1975}
@book{Barrow, author = "John D. Barrow",
  title = "Theories of Everything:  The Quest for Ultimate Explanation",
  publisher = "Oxford University Press",
  address = "Oxford",
  note = "[Q175.B225]",
  year = 1991 }
@book{Behnke,
  editor = "H. Behnke and F. Backmann and K. Fladt and W. S{\"{u}}ss",
  title = "Fundamentals of Mathematics",
  volume = "I",
  publisher = "The MIT Press",
  address = "Cambridge, Massachusetts",
  note = "[QA37.2.B413]",
  year = 1974 }
@book{Bell, author = "J. L. Bell and M. Machover",
  title = "A Course in Mathematical Logic",
  publisher = "North-Holland",
  address = "Amsterdam",
  note = "[QA9.B3953]",
  year = 1977 }
@inproceedings{Blass, author = "Andrea Blass",
  title = "The Interaction Between Category Theory and Set Theory",
  pages = "5--29",
  booktitle = "Mathematical Applications of Category Theory (Proceedings
     of the Special Session on Mathematical Applications
     Category Theory, 89th Annual Meeting of the American Mathematical
     Society, held in Denver, Colorado January 5--9, 1983)",
  editor = "John Walter Gray",
  year = 1983,
  note = "[QA169.A47 1983]",
  publisher = "American Mathematical Society",
  address = "Providence, Rhode Island"}
@proceedings{Bledsoe, editor = "W. W. Bledsoe and D. W. Loveland",
  title = "Automated Theorem Proving:  After 25 Years (Proceedings
     of the Special Session on Automatic Theorem Proving,
     89th Annual Meeting of the American Mathematical
     Society, held in Denver, Colorado January 5--9, 1983)",
  year = 1983,
  note = "[QA76.9.A96.S64 1983]",
  publisher = "American Mathematical Society",
  address = "Providence, Rhode Island" }
@book{Boolos, author = "George S. Boolos and Richard C. Jeffrey",
  title = "Computability and Log\-ic",
  publisher = "Cambridge University Press",
  edition = "third",
  address = "Cambridge",
  note = "[QA9.59.B66 1989]",
  year = 1989 }
@book{Campbell, author = "John Campbell",
  title = "Programmer's Progress",
  publisher = "White Star Software",
  address = "Box 51623, Palo Alto, CA 94303",
  year = 1991 }
@article{DBLP:journals/corr/Carneiro14,
  author    = {Mario Carneiro},
  title     = {Conversion of {HOL} Light proofs into Metamath},
  journal   = {CoRR},
  volume    = {abs/1412.8091},
  year      = {2014},
  url       = {http://arxiv.org/abs/1412.8091},
  archivePrefix = {arXiv},
  eprint    = {1412.8091},
  timestamp = {Mon, 13 Aug 2018 16:47:05 +0200},
  biburl    = {https://dblp.org/rec/bib/journals/corr/Carneiro14},
  bibsource = {dblp computer science bibliography, https://dblp.org}
}
@article{CarneiroND,
  author    = {Mario Carneiro},
  title     = {Natural Deductions in the Metamath Proof Language},
  url       = {http://us.metamath.org/ocat/natded.pdf},
  year      = 2014
}
@inproceedings{Chou, author = "Shang-Ching Chou",
  title = "Proving Elementary Geometry Theorems Using {W}u's Algorithm",
  pages = "243--286",
  booktitle = "Automated Theorem Proving:  After 25 Years (Proceedings
     of the Special Session on Automatic Theorem Proving,
     89th Annual Meeting of the American Mathematical
     Society, held in Denver, Colorado January 5--9, 1983)",
  editor = "W. W. Bledsoe and D. W. Loveland",
  year = 1983,
  note = "[QA76.9.A96.S64 1983]",
  publisher = "American Mathematical Society",
  address = "Providence, Rhode Island" }
@book{Clemente, author = "Daniel Clemente Laboreo",
  title = "Introduction to natural deduction",
  year = 2014,
  url = "http://www.danielclemente.com/logica/dn.en.pdf" }
@incollection{Courant, author = "Richard Courant and Herbert Robbins",
  title = "Topology",
  pages = "573--590",
  booktitle = "The World of Mathematics, Volume One",
  editor = "James R. Newman",
  publisher = "Simon and Schuster",
  address = "New York",
  note = "[QA3.W67 1988]",
  year = 1956 }
@book{Curry, author = "Haskell B. Curry",
  title = "Foundations of Mathematical Logic",
  publisher = "Dover Publications, Inc.",
  address = "New York",
  note = "[QA9.C976 1977]",
  year = 1977 }
@book{Davis, author = "Philip J. Davis and Reuben Hersh",
  title = "The Mathematical Experience",
  publisher = "Birkh{\"{a}}user Boston",
  address = "Boston",
  note = "[QA8.4.D37 1982]",
  year = 1981 }
@incollection{deMillo,
  author = "Richard de Millo and Richard Lipton and Alan Perlis",
  title = "Social Processes and Proofs of Theorems and Programs",
  pages = "267--285",
  booktitle = "New Directions in the Philosophy of Mathematics",
  editor = "Thomas Tymoczko",
  publisher = "Birkh{\"{a}}user Boston, Inc.",
  address = "Boston",
  note = "[QA8.6.N48 1986]",
  year = 1986 }
@book{Edwards, author = "Robert E. Edwards",
  title = "A Formal Background to Mathematics",
  publisher = "Springer-Verlag",
  address = "New York",
  note = "[QA37.2.E38 v.1a]",
  year = 1979 }
@book{Enderton, author = "Herbert B. Enderton",
  title = "Elements of Set Theory",
  publisher = "Academic Press, Inc.",
  address = "San Diego",
  note = "[QA248.E5]",
  year = 1977 }
@book{Goodstein, author = "R. L. Goodstein",
  title = "Development of Mathematical Logic",
  publisher = "Springer-Verlag New York Inc.",
  address = "New York",
  note = "[QA9.G6554]",
  year = 1971 }
@book{Guillen, author = "Michael Guillen",
  title = "Bridges to Infinity",
  publisher = "Jeremy P. Tarcher, Inc.",
  address = "Los Angeles",
  note = "[QA93.G8]",
  year = 1983 }
@book{Hamilton, author = "Alan G. Hamilton",
  title = "Logic for Mathematicians",
  edition = "revised",
  publisher = "Cambridge University Press",
  address = "Cambridge",
  note = "[QA9.H298]",
  year = 1988 }
@unpublished{Harrison, author = "John Robert Harrison",
  title = "Metatheory and Reflection in Theorem Proving:
    A Survey and Critique",
  note = "Technical Report
    CRC-053.
    SRI Cambridge,
    Millers Yard, Cambridge, UK,
    1995.
    Available on the Web as
{\verb+http:+}\-{\verb+//www.cl.cam.ac.uk/users/jrh/papers/reflect.html+}"}
@TECHREPORT{Harrison-thesis,
        author          = "John Robert Harrison",
        title           = "Theorem Proving with the Real Numbers",
        institution   = "University of Cambridge Computer
                         Lab\-o\-ra\-to\-ry",
        address         = "New Museums Site, Pembroke Street, Cambridge,
                           CB2 3QG, UK",
        year            = 1996,
        number          = 408,
        type            = "Technical Report",
        note            = "Author's PhD thesis,
   available on the Web at
{\verb+http:+}\-{\verb+//www.cl.cam.ac.uk+}\-{\verb+/users+}\-{\verb+/jrh+}%
\-{\verb+/papers+}\-{\verb+/thesis.html+}"}
@book{Herrlich, author = "Horst Herrlich and George E. Strecker",
  title = "Category Theory:  An Introduction",
  publisher = "Allyn and Bacon Inc.",
  address = "Boston",
  note = "[QA169.H567]",
  year = 1973 }
@article{Hindley, author = "J. Roger Hindley and David Meredith",
  title = "Principal Type-Schemes and Condensed Detachment",
  journal = "The Journal of Symbolic Logic",
  volume = 55,
  year = 1990,
  note = "[QA.J87]",
  pages = "90--105" }
@book{Hofstadter, author = "Douglas R. Hofstadter",
  title = "G{\"{o}}del, Escher, Bach",
  publisher = "Basic Books, Inc.",
  address = "New York",
  note = "[QA9.H63 1980]",
  year = 1979 }
@article{Indrzejczak, author= "Andrzej Indrzejczak",
  title = "Natural Deduction, Hybrid Systems and Modal Logic",
  journal = "Trends in Logic",
  volume = 30,
  publisher = "Springer",
  year = 2010 }
@article{Kalish, author = "D. Kalish and R. Montague",
  title = "On {T}arski's Formalization of Predicate Logic with Identity",
  journal = "Archiv f{\"{u}}r Mathematische Logik und Grundlagenfor\-schung",
  volume = 7,
  year = 1965,
  note = "[QA.A673]",
  pages = "81--101" }
@article{Kalman, author = "J. A. Kalman",
  title = "Condensed Detachment as a Rule of Inference",
  journal = "Studia Logica",
  volume = 42,
  number = 4,
  year = 1983,
  note = "[B18.P6.S933]",
  pages = "443-451" }
@book{Kline, author = "Morris Kline",
  title = "Mathematical Thought from Ancient to Modern Times",
  publisher = "Oxford University Press",
  address = "New York",
  note = "[QA21.K516 1990 v.3]",
  year = 1972 }
@book{Klinel, author = "Morris Kline",
  title = "Mathematics, The Loss of Certainty",
  publisher = "Oxford University Press",
  address = "New York",
  note = "[QA21.K525]",
  year = 1980 }
@book{Kramer, author = "Edna E. Kramer",
  title = "The Nature and Growth of Modern Mathematics",
  publisher = "Princeton University Press",
  address = "Princeton, New Jersey",
  note = "[QA93.K89 1981]",
  year = 1981 }
@article{Knill, author = "Oliver Knill",
  title = "Some Fundamental Theorems in Mathematics",
  year = "2018",
  url = "https://arxiv.org/abs/1807.08416" }
@book{Landau, author = "Edmund Landau",
  title = "Foundations of Analysis",
  publisher = "Chelsea Publishing Company",
  address = "New York",
  edition = "second",
  note = "[QA241.L2541 1960]",
  year = 1960 }
@article{Leblanc, author = "Hugues Leblanc",
  title = "On {M}eyer and {L}ambert's Quantificational Calculus {FQ}",
  journal = "The Journal of Symbolic Logic",
  volume = 33,
  year = 1968,
  note = "[QA.J87]",
  pages = "275--280" }
@article{Lejewski, author = "Czeslaw Lejewski",
  title = "On Implicational Definitions",
  journal = "Studia Logica",
  volume = 8,
  year = 1958,
  note = "[B18.P6.S933]",
  pages = "189--208" }
@book{Levy, author = "Azriel Levy",
  title = "Basic Set Theory",
  publisher = "Dover Publications",
  address = "Mineola, NY",
  year = "2002"
}
@book{Margaris, author = "Angelo Margaris",
  title = "First Order Mathematical Logic",
  publisher = "Blaisdell Publishing Company",
  address = "Waltham, Massachusetts",
  note = "[QA9.M327]",
  year = 1967}
@book{Manin, author = "Yu I. Manin",
  title = "A Course in Mathematical Logic",
  publisher = "Springer-Verlag",
  address = "New York",
  note = "[QA9.M29613]",
  year = "1977" }
@article{Mathias, author = "Adrian R. D. Mathias",
  title = "A Term of Length 4,523,659,424,929",
  journal = "Synthese",
  volume = 133,
  year = 2002,
  note = "[Q.S993]",
  pages = "75--86" }
@article{Megill, author = "Norman D. Megill",
  title = "A Finitely Axiomatized Formalization of Predicate Calculus
     with Equality",
  journal = "Notre Dame Journal of Formal Logic",
  volume = 36,
  year = 1995,
  note = "[QA.N914]",
  pages = "435--453" }
@unpublished{Megillc, author = "Norman D. Megill",
  title = "A Shorter Equivalent of the Axiom of Choice",
  month = "June",
  note = "Unpublished",
  year = 1991 }
@article{MegillBunder, author = "Norman D. Megill and Martin W.
    Bunder",
  title = "Weaker {D}-Complete Logics",
  journal = "Journal of the IGPL",
  volume = 4,
  year = 1996,
  pages = "215--225",
  note = "Available on the Web at
{\verb+http:+}\-{\verb+//www.mpi-sb.mpg.de+}\-{\verb+/igpl+}%
\-{\verb+/Journal+}\-{\verb+/V4-2+}\-{\verb+/#Megill+}"}
}
@book{Mendelson, author = "Elliott Mendelson",
  title = "Introduction to Mathematical Logic",
  edition = "second",
  publisher = "D. Van Nostrand Company, Inc.",
  address = "New York",
  note = "[QA9.M537 1979]",
  year = 1979 }
@article{Meredith, author = "David Meredith",
  title = "In Memoriam {C}arew {A}rthur {M}eredith (1904-1976)",
  journal = "Notre Dame Journal of Formal Logic",
  volume = 18,
  year = 1977,
  note = "[QA.N914]",
  pages = "513--516" }
@article{CAMeredith, author = "C. A. Meredith",
  title = "Single Axioms for the Systems ({C},{N}), ({C},{O}) and ({A},{N})
      of the Two-Valued Propositional Calculus",
  journal = "The Journal of Computing Systems",
  volume = 3,
  year = 1953,
  pages = "155--164" }
@article{Monk, author = "J. Donald Monk",
  title = "Provability With Finitely Many Variables",
  journal = "The Journal of Symbolic Logic",
  volume = 27,
  year = 1971,
  note = "[QA.J87]",
  pages = "353--358" }
@article{Monks, author = "J. Donald Monk",
  title = "Substitutionless Predicate Logic With Identity",
  journal = "Archiv f{\"{u}}r Mathematische Logik und Grundlagenfor\-schung",
  volume = 7,
  year = 1965,
  pages = "103--121" }
  %% Took out this from above to prevent LaTeX underfull warning:
  % note = "[QA.A673]",
@book{Moore, author = "A. W. Moore",
  title = "The Infinite",
  publisher = "Routledge",
  address = "New York",
  note = "[BD411.M59]",
  year = 1989}
@book{Munkres, author = "James R. Munkres",
  title = "Topology: A First Course",
  publisher = "Prentice-Hall, Inc.",
  address = "Englewood Cliffs, New Jersey",
  note = "[QA611.M82]",
  year = 1975}
@article{Nemesszeghy, author = "E. Z. Nemesszeghy and E. A. Nemesszeghy",
  title = "On Strongly Creative Definitions:  A Reply to {V}. {F}. {R}ickey",
  journal = "Logique et Analyse (N.\ S.)",
  year = 1977,
  volume = 20,
  note = "[BC.L832]",
  pages = "111--115" }
@unpublished{Nemeti, author = "N{\'{e}}meti, I.",
  title = "Algebraizations of Quantifier Logics, an Overview",
  note = "Version 11.4, preprint, Mathematical Institute, Budapest,
    1994.  A shortened version without proofs appeared in
    ``Algebraizations of quantifier logics, an introductory overview,''
   {\em Studia Logica}, 50:485--569, 1991 [B18.P6.S933]"}
@article{Pavicic, author = "M. Pavi{\v{c}}i{\'{c}}",
  title = "A New Axiomatization of Unified Quantum Logic",
  journal = "International Journal of Theoretical Physics",
  year = 1992,
  volume = 31,
  note = "[QC.I626]",
  pages = "1753 --1766" }
@book{Penrose, author = "Roger Penrose",
  title = "The Emperor's New Mind",
  publisher = "Oxford University Press",
  address = "New York",
  note = "[Q335.P415]",
  year = 1989 }
@book{PetersonI, author = "Ivars Peterson",
  title = "The Mathematical Tourist",
  publisher = "W. H. Freeman and Company",
  address = "New York",
  note = "[QA93.P475]",
  year = 1988 }
@article{Peterson, author = "Jeremy George Peterson",
  title = "An automatic theorem prover for substitution and detachment systems",
  journal = "Notre Dame Journal of Formal Logic",
  volume = 19,
  year = 1978,
  note = "[QA.N914]",
  pages = "119--122" }
@book{Quine, author = "Willard Van Orman Quine",
  title = "Set Theory and Its Logic",
  edition = "revised",
  publisher = "The Belknap Press of Harvard University Press",
  address = "Cambridge, Massachusetts",
  note = "[QA248.Q7 1969]",
  year = 1969 }
@article{Robinson, author = "J. A. Robinson",
  title = "A Machine-Oriented Logic Based on the Resolution Principle",
  journal = "Journal of the Association for Computing Machinery",
  year = 1965,
  volume = 12,
  pages = "23--41" }
@article{RobinsonT, author = "T. Thacher Robinson",
  title = "Independence of Two Nice Sets of Axioms for the Propositional
    Calculus",
  journal = "The Journal of Symbolic Logic",
  volume = 33,
  year = 1968,
  note = "[QA.J87]",
  pages = "265--270" }
@book{Rucker, author = "Rudy Rucker",
  title = "Infinity and the Mind:  The Science and Philosophy of the
    Infinite",
  publisher = "Bantam Books, Inc.",
  address = "New York",
  note = "[QA9.R79 1982]",
  year = 1982 }
@book{Russell, author = "Bertrand Russell",
  title = "Mysticism and Logic, and Other Essays",
  publisher = "Barnes \& Noble Books",
  address = "Totowa, New Jersey",
  note = "[B1649.R963.M9 1981]",
  year = 1981 }
@article{Russell2, author = "Bertrand Russell",
  title = "Recent Work on the Principles of Mathematics",
  journal = "International Monthly",
  volume = 4,
  year = 1901,
  pages = "84"}
@article{Schmidt, author = "Eric Schmidt",
  title = "Reductions in Norman Megill's axiom system for complex numbers",
  url = "http://us.metamath.org/downloads/schmidt-cnaxioms.pdf",
  year = "2012" }
@book{Shoenfield, author = "Joseph R. Shoenfield",
  title = "Mathematical Logic",
  publisher = "Addison-Wesley Publishing Company, Inc.",
  address = "Reading, Massachusetts",
  year = 1967,
  note = "[QA9.S52]" }
@book{Smullyan, author = "Raymond M. Smullyan",
  title = "Theory of Formal Systems",
  publisher = "Princeton University Press",
  address = "Princeton, New Jersey",
  year = 1961,
  note = "[QA248.5.S55]" }
@book{Solow, author = "Daniel Solow",
  title = "How to Read and Do Proofs:  An Introduction to Mathematical
    Thought Process",
  publisher = "John Wiley \& Sons",
  address = "New York",
  year = 1982,
  note = "[QA9.S577]" }
@book{Stark, author = "Harold M. Stark",
  title = "An Introduction to Number Theory",
  publisher = "Markham Publishing Company",
  address = "Chicago",
  note = "[QA241.S72 1978]",
  year = 1970 }
@article{Swart, author = "E. R. Swart",
  title = "The Philosophical Implications of the Four-Color Problem",
  journal = "American Mathematical Monthly",
  year = 1980,
  volume = 87,
  month = "November",
  note = "[QA.A5125]",
  pages = "697--707" }
@book{Szpiro, author = "George G. Szpiro",
  title = "Poincar{\'{e}}'s Prize: The Hundred-Year Quest to Solve One
    of Math's Greatest Puzzles",
  publisher = "Penguin Books Ltd",
  address = "London",
  note = "[QA43.S985 2007]",
  year = 2007}
@book{Takeuti, author = "Gaisi Takeuti and Wilson M. Zaring",
  title = "Introduction to Axiomatic Set Theory",
  edition = "second",
  publisher = "Springer-Verlag New York Inc.",
  address = "New York",
  note = "[QA248.T136 1982]",
  year = 1982}
@inproceedings{Tarski, author = "Alfred Tarski",
  title = "What is Elementary Geometry",
  pages = "16--29",
  booktitle = "The Axiomatic Method, with Special Reference to Geometry and
     Physics (Proceedings of an International Symposium held at the University
     of California, Berkeley, December 26, 1957 --- January 4, 1958)",
  editor = "Leon Henkin and Patrick Suppes and Alfred Tarski",
  year = 1959,
  publisher = "North-Holland Publishing Company",
  address = "Amsterdam"}
@article{Tarski1965, author = "Alfred Tarski",
  title = "A Simplified Formalization of Predicate Logic with Identity",
  journal = "Archiv f{\"{u}}r Mathematische Logik und Grundlagenforschung",
  volume = 7,
  year = 1965,
  note = "[QA.A673]",
  pages = "61--79" }
@book{Tymoczko,
  title = "New Directions in the Philosophy of Mathematics",
  editor = "Thomas Tymoczko",
  publisher = "Birkh{\"{a}}user Boston, Inc.",
  address = "Boston",
  note = "[QA8.6.N48 1986]",
  year = 1986 }
@incollection{Wang,
  author = "Hao Wang",
  title = "Theory and Practice in Mathematics",
  pages = "129--152",
  booktitle = "New Directions in the Philosophy of Mathematics",
  editor = "Thomas Tymoczko",
  publisher = "Birkh{\"{a}}user Boston, Inc.",
  address = "Boston",
  note = "[QA8.6.N48 1986]",
  year = 1986 }
@manual{Webster,
  title = "Webster's New Collegiate Dictionary",
  organization = "G. \& C. Merriam Co.",
  address = "Springfield, Massachusetts",
  note = "[PE1628.W4M4 1977]",
  year = 1977 }
@manual{Whitehead, author = "Alfred North Whitehead",
  title = "An Introduction to Mathematics",
  year = 1911 }
@book{PM, author = "Alfred North Whitehead and Bertrand Russell",
  title = "Principia Mathematica",
  edition = "second",
  publisher = "Cambridge University Press",
  address = "Cambridge",
  year = "1927",
  note = "(3 vols.) [QA9.W592 1927]" }
@article{DBLP:journals/corr/Whalen16,
  author    = {Daniel Whalen},
  title     = {Holophrasm: a neural Automated Theorem Prover for higher-order logic},
  journal   = {CoRR},
  volume    = {abs/1608.02644},
  year      = {2016},
  url       = {http://arxiv.org/abs/1608.02644},
  archivePrefix = {arXiv},
  eprint    = {1608.02644},
  timestamp = {Mon, 13 Aug 2018 16:46:19 +0200},
  biburl    = {https://dblp.org/rec/bib/journals/corr/Whalen16},
  bibsource = {dblp computer science bibliography, https://dblp.org} }
@article{Wiedijk-revisited,
  author = {Freek Wiedijk},
  title = {The QED Manifesto Revisited},
  year = {2007},
  url = {http://mizar.org/trybulec65/8.pdf} }
@book{Wolfram,
  author = "Stephen Wolfram",
  title = "Mathematica:  A System for Doing Mathematics by Computer",
  edition = "second",
  publisher = "Addison-Wesley Publishing Co.",
  address = "Redwood City, California",
  note = "[QA76.95.W65 1991]",
  year = 1991 }
@book{Wos, author = "Larry Wos and Ross Overbeek and Ewing Lusk and Jim Boyle",
  title = "Automated Reasoning:  Introduction and Applications",
  edition = "second",
  publisher = "McGraw-Hill, Inc.",
  address = "New York",
  note = "[QA76.9.A96.A93 1992]",
  year = 1992 }

%
%
%[1] Church, Alonzo, Introduction to Mathematical Logic,
% Volume 1, Princeton University Press, Princeton, N. J., 1956.
%
%[2] Cohen, Paul J., Set Theory and the Continuum Hypothesis,
% W. A. Benjamin, Inc., Reading, Mass., 1966.
%
%[3] Hamilton, Alan G., Logic for Mathematicians, Cambridge
% University Press,
% Cambridge, 1988.

%[6] Kleene, Stephen Cole, Introduction to Metamathematics, D.  Van
% Nostrand Company, Inc., Princeton (1952).

%[13] Tarski, Alfred, "A simplified formalization of predicate
% logic with identity," Archiv fur Mathematische Logik und
% Grundlagenforschung, vol. 7 (1965), pp. 61-79.

%[14] Tarski, Alfred and Steven Givant, A Formalization of Set
% Theory Without Variables, American Mathematical Society Colloquium
% Publications, vol. 41, American Mathematical Society,
% Providence, R. I., 1987.

%[15] Zeman, J. J., Modal Logic, Oxford University Press, Oxford, 1973.
\end{filecontents}
% --------------------------- End of metamath.bib -----------------------------


%Book: Metamath
%Author:  Norman Megill Email:  nm at alum.mit.edu
%Author:  David A. Wheeler Email:  dwheeler at dwheeler.com

% A book template example
% http://www.stsci.edu/ftp/software/tex/bookstuff/book.template

\documentclass[leqno]{book} % LaTeX 2e. 10pt. Use [leqno,12pt] for 12pt
% hyperref 2002/05/27 v6.72r  (couldn't get pagebackref to work)
\usepackage[plainpages=false,pdfpagelabels=true]{hyperref}

\usepackage{needspace}     % Enable control over page breaks
\usepackage{breqn}         % automatic equation breaking
\usepackage{microtype}     % microtypography, reduces hyphenation

% Packages for flexible tables.  We need to be able to
% wrap text within a cell (with automatically-determined widths) AND
% split a table automatically across multiple pages.
% * "tabularx" wraps text in cells but only 1 page
% * "longtable" goes across pages but by itself is incompatible with tabularx
% * "ltxtable" combines longtable and tabularx, but table contents
%    must be in a separate file.
% * "ltablex" combines tabularx and longtable - must install specially
% * "booktabs" is recommended as a way to improve the look of tables,
%   but doesn't add these capabilities.
% * "tabu" much more capable and seems to be recommended. So use that.

\usepackage{makecell}      % Enable forced line splits within a table cell
\usepackage{longtable}     % Enable multi-page tables
\usepackage{tabu}          % Multi-page tables with wrapped text in a cell

% You can find more Tex packages using commands like:
% tlmgr search --file tabu.sty
% find /usr/share/texmf-dist/ -name '*tab*'
%
%%%%%%%%%%%%%%%%%%%%%%%%%%%%%%%%%%%%%%%%%%%%%%%%%%%%%%%%%%%%%%%%%%%%%%%%%%%%
% Uncomment the next 3 lines to suppress boxes and colors on the hyperlinks
%%%%%%%%%%%%%%%%%%%%%%%%%%%%%%%%%%%%%%%%%%%%%%%%%%%%%%%%%%%%%%%%%%%%%%%%%%%%
%\hypersetup{
%colorlinks,citecolor=black,filecolor=black,linkcolor=black,urlcolor=black
%}
%
\usepackage{realref}

% Restarting page numbers: try?
%   \printglossary
%   \cleardoublepage
%   \pagenumbering{arabic}
%   \setcounter{page}{1}    ???needed
%   \include{chap1}

% not used:
% \def\R2Lurl#1#2{\mbox{\href{#1}\texttt{#2}}}

\usepackage{amssymb}

% Version 1 of book: margins: t=.4, b=.2, ll=.4, rr=.55
% \usepackage{anysize}
% % \papersize{<height>}{<width>}
% % \marginsize{<left>}{<right>}{<top>}{<bottom>}
% \papersize{9in}{6in}
% % l/r 0.6124-0.6170 works t/b 0.2418-0.3411 = 192pp. 0.2926-03118=exact
% \marginsize{0.7147in}{0.5147in}{0.4012in}{0.2012in}

\usepackage{anysize}
% \papersize{<height>}{<width>}
% \marginsize{<left>}{<right>}{<top>}{<bottom>}
\papersize{9in}{6in}
% l/r 0.85in&0.6431-0.6539 works t/b ?-?
%\marginsize{0.85in}{0.6485in}{0.55in}{0.35in}
\marginsize{0.8in}{0.65in}{0.5in}{0.3in}

% \usepackage[papersize={3.6in,4.8in},hmargin=0.1in,vmargin={0.1in,0.1in}]{geometry}  % page geometry
\usepackage{special-settings}

\raggedbottom
\makeindex

\begin{document}
% Discourage page widows and orphans:
\clubpenalty=300
\widowpenalty=300

%%%%%%% load in AMS fonts %%%%%%% % LaTeX 2.09 - obsolete in LaTeX 2e
%\input{amssym.def}
%\input{amssym.tex}
%\input{c:/texmf/tex/plain/amsfonts/amssym.def}
%\input{c:/texmf/tex/plain/amsfonts/amssym.tex}

\bibliographystyle{plain}
\pagenumbering{roman}
\pagestyle{headings}

\thispagestyle{empty}

\hfill
\vfill

\begin{center}
{\LARGE\bf Metamath} \\
\vspace{1ex}
{\large A Computer Language for Mathematical Proofs} \\
\vspace{7ex}
{\large Norman Megill} \\
\vspace{7ex}
with extensive revisions by \\
\vspace{1ex}
{\large David A. Wheeler} \\
\vspace{7ex}
% Printed date. If changing the date below, also fix the date at the beginning.
2019-05-24
\end{center}

\vfill
\hfill

\newpage
\thispagestyle{empty}

\hfill
\vfill

\begin{center}
$\sim$\ {\sc Public Domain}\ $\sim$

\vspace{2ex}
This book (including its later revisions)
has been released into the Public Domain by Norman Megill per the
Creative Commons CC0 1.0 Universal (CC0 1.0) Public Domain Dedication
(\url{https://creativecommons.org/publicdomain/zero/1.0/}).
David A. Wheeler has done the same.
The public domain release applies worldwide.  In case this is not
legally possible, the right is granted to use the work for any purpose,
without any conditions, unless such conditions are required by law.

\vspace{3ex}
Several short, attributed quotations from copyrighted works
appear in this book under the ``fair use'' provision of Section 107 of
the United States Copyright Act (Title 17 of the {\em United States
Code}).  The public-domain status of this book is not applicable to
those quotations.

\vspace{3ex}
Any trademarks used in this book are the property of their owners.

% QA76.9.L63.M??

% \vspace{1ex}
%
% \vspace{1ex}
% {\small Permission is granted to make and distribute verbatim copies of this
% book
% provided the copyright notice and this
% permission notice are preserved on all copies.}
%
% \vspace{1ex}
% {\small Permission is granted to copy and distribute modified versions of this
% book under the conditions for verbatim copying, provided that the
% entire
% resulting derived work is distributed under the terms of a permission
% notice
% identical to this one.}
%
% \vspace{1ex}
% {\small Permission is granted to copy and distribute translations of this
% book into another language, under the above conditions for modified
% versions,
% except that this permission notice may be stated in a translation
% approved by the
% author.}
%
% \vspace{1ex}
% %{\small   For a copy of the \LaTeX\ source files for this book, contact
% %the author.} \\
% \ \\
% \ \\

\vspace{7ex}
% ISBN: 1-4116-3724-0 \\
ISBN: 978-1-4116-3724-5 \\
{\ } \\
Lulu Press \\
Morrisville, North Carolina\\
USA


\hfill
\vfill

Norman Megill\\ 93 Bridge St., Lexington, MA 02421 \\
E-mail address: \texttt{nm{\char`\@}alum.mit.edu} \\
\vspace{7ex}
David A. Wheeler \\
E-mail address: \texttt{dwheeler{\char`\@}dwheeler.com} \\
% See notes added at end of Preface for revision history. \\
% For current information on the Metamath software see \\
\vspace{7ex}
\url{http://metamath.org}
\end{center}

% \newpage
% \thispagestyle{empty}
%
% \hfill
% \vfill
%
% \begin{center}
% {\it To my son Robin Dwight Megill}
% \end{center}
%
% \vfill
% \hfill
%
% \newpage

\tableofcontents
%\listoftables

\chapter*{Preface}
\markboth{PREFACE}{PREFACE}
\addcontentsline{toc}{section}{Preface}


% (For current information, see the notes added at the
% end of this preface on p.~\pageref{note2002}.)

\subsubsection{Overview}

Metamath\index{Metamath} is a computer language and an associated computer
program for archiving, verifying, and studying mathematical proofs at a very
detailed level.  The Metamath language incorporates no mathematics per se but
treats all mathematical statements as mere sequences of symbols.  You provide
Metamath with certain special sequences (axioms) that tell it what rules
of inference are allowed.  Metamath is not limited to any specific field of
mathematics.  The Metamath language is simple and robust, with an
almost total absence of hard-wired syntax, and
I\footnote{Unless otherwise noted, the words
``I,'' ``me,'' and ``my'' refer to Norman Megill\index{Megill, Norman}, while
``we,'' ``us,'' and ``our'' refer to Norman Megill and
David A. Wheeler\index{Wheeler, David A.}.}
believe that it
provides about the simplest possible framework that allows essentially all of
mathematics to be expressed with absolute rigor.

% index test
%\newcommand{\nn}[1]{#1n}
%\index{aaa@bbb}
%\index{abc!def}
%\index{abd|see{qqq}}
%\index{abe|nn}
%\index{abf|emph}
%\index{abg|(}
%\index{abg|)}

Using the Metamath language, you can build formal or mathematical
systems\index{formal system}\footnote{A formal or mathematical system consists
of a collection of symbols (such as $2$, $4$, $+$ and $=$), syntax rules that
describe how symbols may be combined to form a legal expression (called a
well-formed formula or {\em wff}, pronounced ``whiff''), some starting wffs
called axioms, and inference rules that describe how theorems may be derived
(proved) from the axioms.  A theorem is a mathematical fact such as $2+2=4$.
Strictly speaking, even an obvious fact such as this must be proved from
axioms to be formally acceptable to a mathematician.}\index{theorem}
\index{axiom}\index{rule}\index{well-formed formula (wff)} that involve
inferences from axioms.  Although a database is provided
that includes a recommended set of axioms for standard mathematics, if you
wish you can supply your own symbols, syntax, axioms, rules, and definitions.

The name ``Metamath'' was chosen to suggest that the language provides a
means for {\em describing} mathematics rather than {\em being} the
mathematics itself.  Actually in some sense any mathematical language is
metamathematical.  Symbols written on paper, or stored in a computer,
are not mathematics itself but rather a way of expressing mathematics.
For example ``7'' and ``VII'' are symbols for denoting the number seven
in Arabic and Roman numerals; neither {\em is} the number seven.

If you are able to understand and write computer programs, you should be able
to follow abstract mathematics with the aid of Metamath.  Used in conjunction
with standard textbooks, Metamath can guide you step-by-step towards an
understanding of abstract mathematics from a very rigorous viewpoint, even if
you have no formal abstract mathematics background.  By using a single,
consistent notation to express proofs, once you grasp its basic concepts
Metamath provides you with the ability to immediately follow and dissect
proofs even in totally unfamiliar areas.

Of course, just being able follow a proof will not necessarily give you an
intuitive familiarity with mathematics.  Memorizing the rules of chess does not
give you the ability to appreciate the game of a master, and knowing how the
notes on a musical score map to piano keys does not give you the ability to
hear in your head how it would sound.  But each of these can be a first step.

Metamath allows you to explore proofs in the sense that you can see the
theorem referenced at any step expanded in as much detail as you want, right
down to the underlying axioms of logic and set theory (in the case of the set
theory database provided).  While Metamath will not replace the higher-level
understanding that can only be acquired through exercises and hard work, being
able to see how gaps in a proof are filled in can give you increased
confidence that can speed up the learning process and save you time when you
get stuck.

The Metamath language breaks down a mathematical proof into its tiniest
possible parts.  These can be pieced together, like interlocking
pieces in a puzzle, only in a way that produces correct and absolutely rigorous
mathematics.

The nature of Metamath\index{Metamath} enforces very precise mathematical
thinking, similar to that involved in writing a computer program.  A crucial
difference, though, is that once a proof is verified (by the Metamath program)
to be correct, it is definitely correct; it can never have a hidden
``bug.''\index{computer program bugs}  After getting used to the kind of rigor
and accuracy provided by Metamath, you might even be tempted to
adopt the attitude that a proof should never be considered correct until it
has been verified by a computer, just as you would not completely trust a
manual calculation until you have verified it on a
calculator.

My goal
for Metamath was a system for describing and verifying
mathematics that is completely universal yet conceptually as simple as
possible.  In approaching mathematics from an axiomatic, formal viewpoint, I
wanted Metamath to be able to handle almost any mathematical system, not
necessarily with ease, but at least in principle and hopefully in practice. I
wanted it to verify proofs with absolute rigor, and for this reason Metamath
is what might be thought of as a ``compile-only'' language rather than an
algorithmic or Turing-machine language (Pascal, C, Prolog, Mathematica,
etc.).  In other words, a database written in the Metamath
language doesn't ``do'' anything; it merely exhibits mathematical knowledge
and permits this knowledge to be verified as being correct.  A program in an
algorithmic language can potentially have hidden bugs\index{computer program
bugs} as well as possibly being hard to understand.  But each token in a
Metamath database must be consistent with the database's earlier
contents according to simple, fixed rules.
If a database is verified
to be correct,\footnote{This includes
verification that a sequential list of proof steps results in the specified
theorem.} then the mathematical content is correct if the
verifier is correct and the axioms are correct.
The verification program could be incorrect, but the verification algorithm
is relatively simple (making it unlikely to be implemented incorrectly
by the Metamath program),
and there are over a dozen Metamath database verifiers
written by different people in different programming languages
(so these different verifiers can act as multiple reviewers of a database).
The most-used Metamath database, the Metamath Proof Explorer
(aka \texttt{set.mm}\index{set theory database (\texttt{set.mm})}%
\index{Metamath Proof Explorer}),
is currently verified by four different Metamath verifiers written by
four different people in four different languages, including the
original Metamath program described in this book.
The only ``bugs'' that can exist are in the statement of the axioms,
for example if the axioms are inconsistent (a famous problem shown to be
unsolvable by G\"{o}del's incompleteness theorem\index{G\"{o}del's
incompleteness theorem}).
However, real mathematical systems have very few axioms, and these can
be carefully studied.
All of this provides extraordinarily high confidence that the verified database
is in fact correct.

The Metamath program
doesn't prove theorems automatically but is designed to verify proofs
that you supply to it.
The underlying Metamath language is completely general and has no built-in,
preconceived notions about your formal system\index{formal system}, its logic
or its syntax.
For constructing proofs, the Metamath program has a Proof Assistant\index{Proof
Assistant} which helps you fill in some of a proof step's details, shows you
what choices you have at any step, and verifies the proof as you build it; but
you are still expected to provide the proof.

There are many other programs that can process or generate information
in the Metamath language, and more continue to be written.
This is in part because the Metamath language itself is very simple
and intentionally easy to automatically process.
Some programs, such as \texttt{mmj2}\index{mmj2}, include a proof assistant
that can automate some steps beyond what the Metamath program can do.
Mario Carneiro has developed an algorithm for converting proofs from
the OpenTheory interchange format, which can be translated to and from
any of the HOL family of proof languages (HOL4, HOL Light, ProofPower,
and Isabelle), into the
Metamath language.\cite{DBLP:journals/corr/Carneiro14}\index{Carneiro, Mario}.
Daniel Whalen has developed Holophrasm, which can automatically
prove many Metamath proofs using
machine learning\index{machine learning}\index{artificial intelligence}
approaches
(including multiple neural networks\index{neural networks})\cite{DBLP:journals/corr/Whalen16}\index{Whalen, Daniel}.
However,
a discussion of these other programs is beyond the scope of this book.

Like most computer languages, the Metamath\index{Metamath} language uses the
standard ({\sc ascii}) characters on a computer keyboard, so it cannot
directly represent many of the special symbols that mathematicians use.  A
useful feature of the Metamath program is its ability to convert its notation
into the \LaTeX\ typesetting language.\index{latex@{\LaTeX}}  This feature
lets you convert the {\sc ascii} tokens you've defined into standard
mathematical symbols, so you end up with symbols and formulas you are familiar
with instead of somewhat cryptic {\sc ascii} representations of them.
The Metamath program can also generate HTML\index{HTML}, making it easy
to view results on the web and to see related information by using
hypertext links.

Metamath is probably conceptually different from anything you've seen
before and some aspects may take some getting used to.  This book will
help you decide whether Metamath suits your specific needs.

\subsubsection{Setting Your Expectations}
It is important for you to understand what Metamath\index{Metamath} is and is
not.  As mentioned, the Metamath program
is {\em not} an automated theorem prover but
rather a proof verifier.  Developing a database can be tedious, hard work,
especially if you want to make the proofs as short as possible, but it becomes
easier as you build up a collection of useful theorems.  The purpose of
Metamath is simply to document existing mathematics in an absolutely rigorous,
computer-verifiable way, not to aid directly in the creation of new
mathematics.  It also is not a magic solution for learning abstract
mathematics, although it may be helpful to be able to actually see the implied
rigor behind what you are learning from textbooks, as well as providing hints
to work out proofs that you are stumped on.

As of this writing, a sizable set theory database has been developed to
provide a foundation for many fields of mathematics, but much more work would
be required to develop useful databases for specific fields.

Metamath\index{Metamath} ``knows no math;'' it just provides a framework in
which to express mathematics.  Its language is very small.  You can define two
kinds of symbols, constants\index{constant} and variables\index{variable}.
The only thing Metamath knows how to do is to substitute strings of symbols
for the variables\index{substitution!variable}\index{variable substitution} in
an expression based on instructions you provide it in a proof, subject to
certain constraints you specify for the variables.  Even the decimal
representation of a number is merely a string of certain constants (digits)
which together, in a specific context, correspond to whatever mathematical
object you choose to define for it; unlike other computer languages, there is
no actual number stored inside the computer.  In a proof, you in effect
instruct Metamath what symbol substitutions to make in previous axioms or
theorems and join a sequence of them together to result in the desired
theorem.  This kind of symbol manipulation captures the essence of mathematics
at a preaxiomatic level.

\subsubsection{Metamath and Mathematical Literature}

In advanced mathematical literature, proofs are usually presented in the form
of short outlines that often only an expert can follow.  This is partly out of
a desire for brevity, but it would also be unwise (even if it were practical)
to present proofs in complete formal detail, since the overall picture would
be lost.\index{formal proof}

A solution I envision\label{envision} that would allow mathematics to remain
acceptable to the expert, yet increase its accessibility to non-specialists,
consists of a combination of the traditional short, informal proof in print
accompanied by a complete formal proof stored in a computer database.  In an
analogy with a computer program, the informal proof is like a set of comments
that describe the overall reasoning and content of the proof, whereas the
computer database is like the actual program and provides a means for anyone,
even a non-expert, to follow the proof in as much detail as desired, exploring
it back through layers of theorems (like subroutines that call other
subroutines) all the way back to the axioms of the theory.  In addition, the
computer database would have the advantage of providing absolute assurance
that the proof is correct, since each step can be verified automatically.

There are several other approaches besides Metamath to a project such
as this.  Section~\ref{proofverifiers} discusses some of these.

To us, a noble goal would be a database with hundreds of thousands of
theorems and their computer-verifiable proofs, encompassing a significant
fraction of known mathematics and available for instant access.
These would be fully verified by multiple independently-implemented verifiers,
to provide extremely high confidence that the proofs are completely correct.
The database would allow people to investigate whatever details they were
interested in, so that they could confirm whatever portions they wished.
Whether or not Metamath is an appropriate choice remains to be seen, but in
principle we believe it is sufficient.

\subsubsection{Formalism}

Over the past fifty years, a group of French mathematicians working
collectively under the pseudonym of Bourbaki\index{Bourbaki, Nicolas} have
co-authored a series of monographs that attempt to rigorously and
consistently formalize large bodies of mathematics from foundations.  On the
one hand, certainly such an effort has its merits; on the other hand, the
Bourbaki project has been criticized for its ``scholasticism'' and
``hyperaxiomatics'' that hide the intuitive steps that lead to the results
\cite[p.~191]{Barrow}\index{Barrow, John D.}.

Metamath unabashedly carries this philosophy to its extreme and no doubt is
subject to the same kind of criticism.  Nonetheless I think that in
conjunction with conventional approaches to mathematics Metamath can serve a
useful purpose.  The Bourbaki approach is essentially pedagogic, requiring the
reader to become intimately familiar with each detail in a very large
hierarchy before he or she can proceed to the next step.  The difference with
Metamath is that the ``reader'' (user) knows that all details are contained in
its computer database, available as needed; it does not demand that the user
know everything but conveniently makes available those portions that are of
interest.  As the body of all mathematical knowledge grows larger and larger,
no one individual can have a thorough grasp of its entirety.  Metamath
can finalize and put to rest any questions about the validity of any part of it
and can make any part of it accessible, in principle, to a non-specialist.

\subsubsection{A Personal Note}
Why did I develop Metamath\index{Metamath}?  I enjoy abstract mathematics, but
I sometimes get lost in a barrage of definitions and start to lose confidence
that my proofs are correct.  Or I reach a point where I lose sight of how
anything I'm doing relates to the axioms that a theory is based on and am
sometimes suspicious that there may be some overlooked implicit axiom
accidentally introduced along the way (as happened historically with Euclidean
geometry\index{Euclidean geometry}, whose omission of Pasch's
axiom\index{Pasch's axiom} went unnoticed for 2000 years
\cite[p.~160]{Davis}!). I'm also somewhat lazy and wish to avoid the effort
involved in re-verifying the gaps in informal proofs ``left to the reader;'' I
prefer to figure them out just once and not have to go through the same
frustration a year from now when I've forgotten what I did.  Metamath provides
better recovery of my efforts than scraps of paper that I can't
decipher anymore.  But mostly I find very appealing the idea of rigorously
archiving mathematical knowledge in a computer database, providing precision,
certainty, and elimination of human error.

\subsubsection{Note on Bibliography and Index}

The Bibliography usually includes the Library of Congress classification
for a work to make it easier for you to find it in on a university
library shelf.  The Index has author references to pages where their works
are cited, even though the authors' names may not appear on those pages.

\subsubsection{Acknowledgments}

Acknowledgments are first due to my wife, Deborah (who passed away on
September 4, 1998), for critiquing the manu\-script but most of all for
her patience and support.  I also wish to thank Joe Wright, Richard
Becker, Clarke Evans, Buddha Buck, and Jeremy Henty for helpful
comments.  Any errors, omissions, and other shortcomings are of course
my responsibility.

\subsubsection{Note Added June 22, 2005}\label{note2002}

The original, unpublished version of this book was written in 1997 and
distributed via the web.  The present edition has been updated to
reflect the current Metamath program and databases, as well as more
current {\sc url}s for Internet sites.  Thanks to Josh
Purinton\index{Purinton, Josh}, One Hand
Clapping, Mel L.\ O'Cat, and Roy F. Longton for pointing out
typographical and other errors.  I have also benefitted from numerous
discussions with Raph Levien\index{Levien, Raph}, who has extended
Metamath's philosophy of rigor to result in his {\em
Ghilbert}\index{Ghilbert} proof language (\url{http://ghilbert.org}).

Robert (Bob) Solovay\index{Solovay, Robert} communicated a new result of
A.~R.~D.~Mathias on the system of Bourbaki, and the text has been
updated accordingly (p.~\pageref{bourbaki}).

Bob also pointed out a clarification of the literature regarding
category theory and inaccessible cardinals\index{category
theory}\index{cardinal, inaccessible} (p.~\pageref{categoryth}),
and a misleading statement was removed from the text.  Specifically,
contrary to a statement in previous editions, it is possible to express
``There is a proper class of inaccessible cardinals'' in the language of
ZFC.  This can be done as follows:  ``For every set $x$ there is an
inaccessible cardinal $\kappa$ such that $\kappa$ is not in $x$.''
Bob writes:\footnote{Private communication, Nov.~30, 2002.}
\begin{quotation}
     This axiom is how Grothendieck presents category theory.  To each
inaccessible cardinal $\kappa$ one associates a Grothendieck universe
\index{Grothendieck, Alexander} $U(\kappa)$.  $U(\kappa)$ consists of
those sets which lie in a transitive set of cardinality less than
$\kappa$.  Instead of the ``category of all groups,'' one works relative
to a universe [considering the category of groups of cardinality less
than $\kappa$].  Now the category whose objects are all categories
``relative to the universe $U(\kappa)$'' will be a category not
relative to this universe but to the next universe.

     All of the things category theorists like to do can be done in this
framework.  The only controversial point is whether the Grothen\-dieck
axiom is too strong for the needs of category theorists.  Mac Lane
\index{Mac Lane, Saunders} argues that ``one universe is enough'' and
Feferman\index{Feferman, Solomon} has argued that one can get by with
ordinary ZFC.  I don't find Feferman's arguments persuasive.  Mac Lane
may be right, but when I think about category theory I do it \`{a} la
Grothendieck.

        By the way Mizar\index{Mizar} adds the axiom ``there is a proper
class of inaccessibles'' precisely so as to do category theory.
\end{quotation}

The most current information on the Metamath program and databases can
always be found at \url{http://metamath.org}.


\subsubsection{Note Added June 24, 2006}\label{note2006}

The Metamath spec was restricted slightly to make parsers easier to
write.  See the footnote on p.~\pageref{namespace}.

%\subsubsection{Note Added July 24, 2006}\label{note2006b}
\subsubsection{Note Added March 10, 2007}\label{note2006b}

I am grateful to Anthony Williams\index{Williams, Anthony} for writing
the \LaTeX\ package called {\tt realref.sty} and contributing it to the
public domain.  This package allows the internal hyperlinks in a {\sc
pdf} file to anchor to specific page numbers instead of just section
titles, making the navigation of the {\sc pdf} file for this book much
more pleasant and ``logical.''

A typographical error found by Martin Kiselkov was corrected.
A confusing remark about unification was deleted per suggestion of
Mel O'Cat.

\subsubsection{Note Added May 27, 2009}\label{note2009}

Several typos found by Kim Sparre were corrected.  A note was added that
the Poincar\'{e} conjecture has been proved (p.~\pageref{poincare}).

\subsubsection{Note Added Nov. 17, 2014}\label{note2014}

The statement of the Schr\"{o}der--Bernstein theorem was corrected in
Section~\ref{trust}.  Thanks to Bob Solovay for pointing out the error.

\subsubsection{Note Added May 25, 2016}\label{note2016}

Thanks to Jerry James for correcting 16 typos.

\subsubsection{Note Added February 25, 2019}\label{note201902}

David A. Wheeler\index{Wheeler, David A.}
made a large number of improvements and updates,
in coordination with Norman Megill.
The predicate calculus axioms were renumbered, and the text makes
it clear that they are based on Tarski's system S2;
the one slight deviation in axiom ax-6 is explained and justified.
The real and complex number axioms were modified to be consistent with
\texttt{set.mm}\index{set theory database (\texttt{set.mm})}%
\index{Metamath Proof Explorer}.
Long-awaited specification changes ``1--8'' were made,
which clarified previously ambiguous points.
Some errors in the text involving \texttt{\$f} and
\texttt{\$d} statements were corrected (the spec was correct, but
the in-book explanations unintentionally contradicted the spec).
We now have a system for automatically generating narrow PDFs,
so that those with smartphones can have easy access to the current
version of this document.
A new section on deduction was added;
it discusses the standard deduction theorem,
the weak deduction theorem,
deduction style, and natural deduction.
Many minor corrections (too numerous to list here) were also made.

\subsubsection{Note Added March 7, 2019}\label{note201903}

This added a description of the Matamath language syntax in
Extended Backus--Naur Form (EBNF)\index{Extended Backus--Naur Form}\index{EBNF}
in Appendix \ref{BNF}, added a brief explanation about typecodes,
inserted more examples in the deduction section,
and added a variety of smaller improvements.

\subsubsection{Note Added April 7, 2019}\label{note201904}

This version clarified the proper substitution notation, improved the
discussion on the weak deduction theorem and natural deduction,
documented the \texttt{undo} command, updated the information on
\texttt{write source}, changed the typecode
from \texttt{set} to \texttt{setvar} to be consistent with the current
version of \texttt{set.mm}, added more documentation about comment markup
(e.g., documented how to create headings), and clarified the
differences between various assertion forms (in particular deduction form).

\subsubsection{Note Added May 20, 2019}\label{note201905}

This version fixes various small issues and added specific examples
of forms (deduction form, inference form, and closed form).

\chapter{Introduction}
\pagenumbering{arabic}

\begin{quotation}
  {\em {\em I.M.:}  No, no.  There's nothing subjective about it!  Everybody
knows what a proof is.  Just read some books, take courses from a competent
mathematician, and you'll catch on.

{\em Student:}  Are you sure?

{\em I.M.:}  Well---it is possible that you won't, if you don't have any
aptitude for it.  That can happen, too.

{\em Student:}  Then {\em you} decide what a proof is, and if I don't learn
to decide in the same way, you decide I don't have any aptitude.

{\em I.M.:}  If not me, then who?}
    \flushright\sc  ``The Ideal Mathematician''
    \index{Davis, Phillip J.}
    \footnote{\cite{Davis}, p.~40.}\\
\end{quotation}

In the past century, brilliant mathematicians have discovered almost
unimaginably profound results that rank among the crowning intellectual
achievements of mankind.  However, there is a sense in which modern abstract
mathematics is behind the times, stuck in an era before computers existed.
While no one disputes the remarkable results that have been achieved,
communicating these results in a precise way to the uninitiated is virtually
impossible.  To describe these results, a terse informal language is used which
despite its elegance is very difficult to learn.  This informal language is not
imprecise, far from it, but rather it often has omitted detail
and symbols with hidden context that are
implicitly understood by an expert but few others.  Extremely complex technical
meanings are associated with innocent-sounding English words such as
``compact'' and ``measurable'' that barely hint at what is actually being
said.  Anyone who does not keep the precise technical meaning constantly in
mind is bound to fail, and acquiring the ability to do this can be achieved
only through much practice and hard work.  Only the few who complete the
painful learning experience can join the small in-group of pure
mathematicians.  The informal language effectively cuts off the true nature of
their knowledge from most everyone else.

Metamath\index{Metamath} makes abstract mathematics more concrete.  It allows
a computer to keep track of the complexity associated with each word or symbol
with absolute rigor.  You can explore this complexity at your leisure, to
whatever degree you desire.  Whether or not you believe that concepts such as
infinity actually ``exist'' outside of the mind, Metamath lets you get to the
foundation for what's really being said.

Metamath also enables completely rigorous and thorough proof verification.
Its language is simple enough so that you
don't have to rely on the authority of experts but can verify the results
yourself, step by step.  If you want to attempt to derive your own results,
Metamath will not let you make a mistake in reasoning.
Even professional mathematicians make mistakes; Metamath makes it possible
to thoroughly verify that proofs are correct.

Metamath\index{Metamath} is a computer language and an associated computer
program for archiving, verifying, and studying mathematical proofs at a very
detailed level.
The Metamath language
describes formal\index{formal system} mathematical
systems and expresses proofs of theorems in those systems.  Such a language
is called a metalanguage\index{metalanguage} by mathematicians.
The Metamath program is a computer program that verifies
proofs expressed in the Metamath language.
The Metamath program does not have the built-in
ability to make logical inferences; it just makes a series of symbol
substitutions according to instructions given to it in a proof
and verifies that the result matches the expected theorem.  It makes logical
inferences based only on rules of logic that are contained in a set of
axioms\index{axiom}, or first principles, that you provide to it as the
starting point for proofs.

The complete specification of the Metamath language is only four pages long
(Section~\ref{spec}, p.~\pageref{spec}).  Its simplicity may at first make you
wonder how it can do much of anything at all.  But in fact the kinds of
symbol manipulations it performs are the ones that are implicitly done in all
mathematical systems at the lowest level.  You can learn it relatively quickly
and have complete confidence in any mathematical proof that it verifies.  On
the other hand, it is powerful and general enough so that virtually any
mathematical theory, from the most basic to the deeply abstract, can be
described with it.

Although in principle Metamath can be used with any
kind of mathematics, it is best suited for abstract or ``pure'' mathematics
that is mostly concerned with theorems and their proofs, as opposed to the
kind of mathematics that deals with the practical manipulation of numbers.
Examples of branches of pure mathematics are logic\index{logic},\footnote{Logic
is the study of statements that are universally true regardless of the objects
being described by the statements.  An example is the statement, ``if $P$
implies $Q$, then either $P$ is false or $Q$ is true.''} set theory\index{set
theory},\footnote{Set theory is the study of general-purpose mathematical objects called
``sets,'' and from it essentially all of mathematics can be derived.  For
example, numbers can be defined as specific sets, and their properties
can be explored using the tools of set theory.} number theory\index{number
theory},\footnote{Number theory deals with the properties of positive and
negative integers (whole numbers).} group theory\index{group
theory},\footnote{Group theory studies the properties of mathematical objects
called groups that obey a simple set of axioms and have properties of symmetry
that make them useful in many other fields.} abstract algebra\index{abstract
algebra},\footnote{Abstract algebra includes group theory and also studies
groups with additional properties that qualify them as ``rings'' and
``fields.''  The set of real numbers is a familiar example of a field.},
analysis\index{analysis} \index{real and complex numbers}\footnote{Analysis is
the study of real and complex numbers.} and
topology\index{topology}.\footnote{One area studied by topology are properties
that remain unchanged when geometrical objects undergo stretching
deformations; for example a doughnut and a coffee cup each have one hole (the
cup's hole is in its handle) and are thus considered topologically
equivalent.  In general, though, topology is the study of abstract
mathematical objects that obey a certain (surprisingly simple) set of axioms.
See, for example, Munkres \cite{Munkres}\index{Munkres, James R.}.} Even in
physics, Metamath could be applied to certain branches that make use of
abstract mathematics, such as quantum logic\index{quantum logic} (used to study
aspects of quantum mechanics\index{quantum mechanics}).

On the other hand, Metamath\index{Metamath} is less suited to applications
that deal primarily with intensive numeric computations.  Metamath does not
have any built-in representation of numbers\index{Metamath!representation of
numbers}; instead, a specific string of symbols (digits) must be syntactically
constructed as part of any proof in which an ordinary number is used.  For
this reason, numbers in Metamath are best limited to specific constants that
arise during the course of a theorem or its proof.  Numbers are only a tiny
part of the world of abstract mathematics.  The exclusion of built-in numbers
was a conscious decision to help achieve Metamath's simplicity, and there are
other software tools if you have different mathematical needs.
If you wish to quickly solve algebraic problems, the computer algebra
programs\index{computer algebra system} {\sc
macsyma}\index{macsyma@{\sc macsyma}}, Mathematica\index{Mathematica}, and
Maple\index{Maple} are specifically suited to handling numbers and
algebra efficiently.
If you wish to simply calculate numeric or matrix expressions easily,
tools such as Octave\index{Octave} may be a better choice.

After learning Metamath's basic statement types, any
tech\-ni\-cal\-ly ori\-ent\-ed person, mathematician or not, can
immediately trace
any theorem proved in the language as far back as he or she wants, all the way
to the axioms on which the theorem is based.  This ability suggests a
non-traditional way of learning about pure mathematics.  Used in conjunction
with traditional methods, Metamath could make pure mathematics accessible to
people who are not sufficiently skilled to figure out the implicit detail in
ordinary textbook proofs.  Once you learn the axioms of a theory, you can have
complete confidence that everything you need to understand a proof you are
studying is all there, at your beck and call, allowing you to focus in on any
proof step you don't understand in as much depth as you need, without worrying
about getting stuck on a step you can't figure out.\footnote{On the other
hand, writing proofs in the Metamath language is challenging, requiring
a degree of rigor far in excess of that normally taught to students.  In a
classroom setting, I doubt that writing Metamath proofs would ever replace
traditional homework exercises involving informal proofs, because the time
needed to work out the details would not allow a course to
cover much material.  For students who have trouble grasping the implied rigor
in traditional material, writing a few simple proofs in the Metamath language
might help clarify fuzzy thought processes.  Although somewhat difficult at
first, it eventually becomes fun to do, like solving a puzzle, because of the
instant feedback provided by the computer.}

Metamath\index{Metamath} is probably unlike anything you have
encountered before.  In this first chapter we will look at the philosophy and
use of computers in mathematics in order to better understand the motivation
behind Metamath.  The material in this chapter is not required in order to use
Metamath.  You may skip it if you are impatient, but I hope you will find it
educational and enjoyable.  If you want to start experimenting with the
Metamath program right away, proceed directly to Chapter~\ref{using}
(p.~\pageref{using}).  To
learn the Metamath language, skim Chapter~\ref{using} then proceed to
Chapter~\ref{languagespec} (p.~\pageref{languagespec}).

\section{Mathematics as a Computer Language}

\begin{quote}
  {\em The study of mathematics is apt to commence in
dis\-ap\-point\-ment.\ldots \\
We are told that by its aid the stars are weighted
and the billions of molecules in a drop of water are counted.  Yet, like the
ghost of Hamlet's father, this great science eludes the efforts of our mental
weapons to grasp it.}
  \flushright\sc  Alfred North Whitehead\footnote{\cite{Whitehead}, ch.\ 1.}\\
\end{quote}\index{Whitehead, Alfred North}

\subsection{Is Mathematics ``User-Friendly''?}

Suppose you have no formal training in abstract mathematics.  But popular
books you've read offer tempting glimpses of this world filled with profound
ideas that have stirred the human spirit.  You are not satisfied with the
informal, watered-down descriptions you've read but feel it is important to
grasp the underlying mathematics itself to understand its true meaning. It's
not practical to go back to school to learn it, though; you don't want to
dedicate years of your life to it.  There are many important things in life,
and you have to set priorities for what's important to you.  What would happen
if you tried to pursue it on your own, in your spare time?

After all, you were able to learn a computer programming language such as
Pascal on your own without too much difficulty, even though you had no formal
training in computers.  You don't claim to be an expert in software design,
but you can write a passable program when necessary to suit your needs.  Even
more important, you know that you can look at anyone else's Pascal program, no
matter how complex, and with enough patience figure out exactly how it works,
even though you are not a specialist.  Pascal allows you do anything that a
computer can do, at least in principle.  Thus you know you have the ability,
in principle, to follow anything that a computer program can do:  you just
have to break it down into small enough pieces.

Here's an imaginary scenario of what might happen if you na\-ive\-ly a\-dopted
this same view of abstract mathematics and tried to pick it up on your own, in
a period of time comparable to, say, learning a computer programming
language.

\subsubsection{A Non-Mathematician's Quest for Truth}

\begin{quote}
  {\em \ldots my daughters have been studying (chemistry) for several
se\-mes\-ters, think they have learned differential and integral calculus in
school, and yet even today don't know why $x\cdot y=y\cdot x$ is true.}
  \flushright\sc  Edmund Landau\footnote{\cite{Landau}, p.~vi.}\\
\end{quote}\index{Landau, Edmund}

\begin{quote}
  {\em Minus times minus is plus,\\
The reason for this we need not discuss.}
  \flushright\sc W.\ H.\ Auden\footnote{As quoted in \cite{Guillen}, p.~64.}\\
\end{quote}\index{Auden, W.\ H.}\index{Guillen, Michael}

We'll suppose you are a technically oriented professional, perhaps an engineer, a
computer programmer, or a physicist, but probably not a mathematician.  You
consider yourself reasonably intelligent.  You did well in school, learning a
variety of methods and techniques in practical mathematics such as calculus and
differential equations.  But rarely did your courses get into anything
resembling modern abstract mathematics, and proofs were something that appeared
only occasionally in your textbooks, a kind of necessary evil that was
supposed to convince you of a certain key result.  Most of your
homework consisted of exercises that gave you practice in the techniques, and
you were hardly ever asked to come up with a proof of your own.

You find yourself curious about advanced, abstract mathematics.  You are
driven by an inner conviction that it is important to understand and
appreciate some of the most profound knowledge discovered by mankind.  But it
seems very hard to learn, something that only certain gifted longhairs can
access and understand.  You are frustrated that it seems forever cut off from
you.

Eventually your curiosity drives you to do something about it.
You set for yourself a goal of ``really'' understanding mathematics:  not just
how to manipulate equations in algebra or calculus according to cookbook
rules, but rather to gain a deep understanding of where those rules come from.
In fact, you're not thinking about this kind of ordinary mathematics at all,
but about a much more abstract, ethereal realm of pure mathematics, where
famous results such as G\"{o}del's incompleteness theorem\index{G\"{o}del's
incompleteness theorem} and Cantor's different kinds of infinities
reside.

You have probably read a number of popular books, with titles like {\em
Infinity and the Mind} \cite{Rucker}\index{Rucker, Rudy}, on topics such as
these.  You found them inspiring but at the same time somewhat
unsatisfactory.  They gave you a general idea of what these results are about,
but if someone asked you to prove them, you wouldn't have the faintest idea of
where to begin.   Sure, you could give the same overall outline that you
learned from the popular books; and in a general sort of way, you do have an
understanding.  But deep down inside, you know that there is a rigor that is
missing, that probably there are many subtle steps and pitfalls along the way,
and ultimately it seems you have to place your trust in the experts in the
field.  You don't like this; you want to be able to verify these results for
yourself.

So where do you go next?  As a first step, you decide to look up some of the
original papers on the theorems you are curious about, or better, obtain some
standard textbooks in the field.  You look up a theorem you want to
understand.  Sure enough, it's there, but it's expressed with strange
terms and odd symbols that mean absolutely nothing to you.  It might as well be written in
a foreign language you've never seen before, whose symbols are totally alien.
You look at the proof, and you haven't the foggiest notion what each step
means, much less how one step follows from another.  Well, obviously you have
a lot to learn if you want to understand this stuff.

You feel that you could probably understand it by
going back to college for another three to six years and getting a math
degree.  But that does not fit in with your career and the other things in
your life and would serve no practical purpose.  You decide to seek a quicker
path.  You figure you'll just trace your way back to the beginning, step by
step, as you would do with a computer program, until you understand it.  But
you quickly find that this is not possible, since you can't even understand
enough to know what you have to trace back to.

Maybe a different approach is in order---maybe you should start at the
beginning and work your way up.  First, you read the introduction to the book
to find out what the prerequisites are.  In a similar fashion, you trace your
way back through two or three more books, finally arriving at one that seems
to start at a beginning:  it lists the axioms of arithmetic.  ``Aha!'' you
naively think, ``This must be the starting point, the source of all mathematical
knowledge.'' Or at least the starting point for mathematics dealing with
numbers; you have to start somewhere and have no idea what the starting point
for other mathematics would be.  But the word ``axioms'' looks promising.  So
you eagerly read along and work through some elementary exercises at the
beginning of the book.  You feel vaguely bothered:  these
don't seem like axioms at all, at least not in the sense that you want to
think of axioms.  Axioms imply a starting point from which everything else can
be built up, according to precise rules specified in the axiom system.  Even
though you can understand the first few proofs in an informal way,
and are able to do some of the
exercises, it's hard to pin down precisely what the
rules are.   Sure, each step seems to follow logically from the others, but
exactly what does that mean?  Is the ``logic'' just a matter of common sense,
something vague that we all understand but can never quite state precisely?

You've spent a number of years, off and on, programming computers, and you
know that in the case of computer languages there is no question of what the
rules are---they are precise and crystal clear.  If you follow them, your
program will work, and if you don't, it won't.  No matter how complex a
program, it can always be broken down into simpler and simpler pieces, until
you can ultimately identify the bits that are moved around to perform a
specific function.  Some programs might require a lot of perseverance to
accomplish this, but if you focus on a specific portion of it, you don't even
necessarily have to know how the rest of it works. Shouldn't there be an
analogy in mathematics?

You decide to apply the ultimate test:  you ask yourself how a computer could
verify or ensure that the steps in these proofs follow from one another.
Certainly mathematics must be at least as precisely defined as a computer
language, if not more so; after all, computer science itself is based on it.
If you can get a computer to verify these proofs, then you should also be
able, in principle, to understand them yourself in a very crystal clear,
precise way.

You're in for a surprise:  you can conceive of no way to convert the
proofs, which are in English, to a form that the computer can understand.
The proofs are filled with phrases such as ``assume there exists a unique
$x$\ldots'' and ``given any $y$, let $z$ be the number such that\ldots''  This
isn't the kind of logic you are used to in computer programming, where
everything, even arithmetic, reduces to Boolean ones and zeroes if you care to
break it down sufficiently.  Even though you think you understand the proofs,
there seems to be some kind of higher reasoning involved rather than precise
rules that define how you manipulate the symbols in the axioms.  Whatever it
is, it just isn't obvious how you would express it to a computer, and the more
you think about it, the more puzzled and confused you get, to the point where
you even wonder whether {\em you} really understand it.  There's a lot more to
these axioms of arithmetic than meets the eye.

Nobody ever talked about this in school in your applied math and engineering
courses.  You just learned the rules they gave you, not quite understanding
how or why they worked, sometimes vaguely suspicious or uncertain of them, and
through homework problems and osmosis learned how to present solutions that
satisfied the instructor and earned you an ``A.''  Rarely did you actually
``prove'' anything in a rigorous way, and the math majors who did do stuff
like that seemed to be in a different world.

Of course, there are computer algebra programs that can do mathematics, and
rather impressively.  They can instantly solve the integrals that you
struggled with in freshman calculus, and do much, much more.  But when you
look at these programs, what you see is a big collection of algorithms and
techniques that evolved and were added to over time, along with some basic
software that manipulates symbols.  Each algorithm that is built in is the
result of someone's theorem whose proof is omitted; you just have to trust the
person who proved it and the person who programmed it in and hope there are no
bugs.\index{computer program bugs}  Somehow this doesn't seem to be the
essence of mathematics.  Although computer algebra systems can generate
theorems with amazing speed, they can't actually prove a single one of them.

After some puzzlement, you revisit some popular books on what mathematics is
all about.  Somewhere you read that all of mathematics is actually derived
from something called ``set theory.''  This is a little confusing, because
nowhere in the book that presented the axioms of arithmetic was there any
mention of set theory, or if there was, it seemed to be just a tool that helps
you describe things better---the set of even numbers, that sort of thing.  If
set theory is the basis for all mathematics, then why are additional axioms
needed for arithmetic?

Something is wrong but you're not sure what.  One of your friends is a pure
mathematician.  He knows he is unable to communicate to you what he does for a
living and seems to have little interest in trying.  You do know that for him,
proofs are what mathematics is all about. You ask him what a proof is, and he
essentially tells you that, while of course it's based on logic, really it's
something you learn by doing it over and over until you pick it up.  He refers
you to a book, {\em How to Read and Do Proofs} \cite{Solow}.\index{Solow,
Daniel}  Although this book helps you understand traditional informal proofs,
there is still something missing you can't seem to pin down yet.

You ask your friend how you would go about having a computer verify a proof.
At first he seems puzzled by the question; why would you want to do that?
Then he says it's not something that would make any sense to do, but he's
heard that you'd have to break the proof down into thousands or even millions
of individual steps to do such a thing, because the reasoning involved is at
such a high level of abstraction.  He says that maybe it's something you could
do up to a point, but the computer would be completely impractical once you
get into any meaningful mathematics.  There, the only way you can verify a
proof is by hand, and you can only acquire the ability to do this by
specializing in the field for a couple of years in grad school.  Anyway, he
thinks it all has to do with set theory, although he has never taken a formal
course in set theory but just learned what he needed as he went along.

You are intrigued and amazed.  Apparently a mathematician can grasp as a
single concept something that would take a computer a thousand or a million
steps to verify, and have complete confidence in it.  Each one of these
thousand or million steps must be absolutely correct, or else the whole proof
is meaningless.  If you added a million numbers by hand, would you trust the
result?  How do you really know that all these steps are correct, that there
isn't some subtle pitfall in one of these million steps, like a bug in a
computer program?\index{computer program bugs}  After all, you've read that
famous mathematicians have occasionally made mistakes, and you certainly know
you've made your share on your math homework problems in school.

You recall the analogy with a computer program.  Sure, you can understand what
a large computer program such as a word processor does, as a single high-level
concept or a small set of such concepts, but your ability to understand it in
no way ensures that the program is correct and doesn't have hidden bugs.  Even
if you wrote the program yourself you can't really know this; most large
programs that you've written have had bugs that crop up at some later date, no
matter how careful you tried to be while writing them.

OK, so now it seems the reason you can't figure out how to make a
computer verify proofs is because each step really corresponds to a
million small steps.  Well, you say, a computer can do a million
calculations in a second, so maybe it's still practical to do.  Now the
puzzle becomes how to figure out what the million steps are that each
English-language step corresponds to.  Your mathematician friend hasn't
a clue, but suggests that maybe you would find the answer by studying
set theory.  Actually, your friend thinks you're a little off the wall
for even wondering such a thing.  For him, this is not what mathematics
is all about.

The subject of set theory keeps popping up, so you decide it's
time to look it up.

You decide to start off on a careful footing, so you start reading a couple of
very elementary books on set theory.  A lot of it seems pretty obvious, like
intersections, subsets, and Venn diagrams.  You thumb through one of the
books; nowhere is anything about axioms mentioned. The other book relegates to
an appendix a brief discussion that mentions a set of axioms called
``Zermelo--Fraenkel set theory''\index{Zermelo--Fraenkel set theory} and states
them in English.  You look at them and have no idea what they really mean or
what you can do with them.  The comments in this appendix say that the purpose
of mentioning them is to expose you to the idea, but imply that they are not
necessary for basic understanding and that they are really the subject matter
of advanced treatments where fine points such as a certain paradox (Russell's
paradox\index{Russell's paradox}\footnote{Russell's paradox assumes that there
exists a set $S$ that is a collection of all sets that don't contain
themselves.  Now, either $S$ contains itself or it doesn't.  If it contains
itself, it contradicts its definition.  But if it doesn't contain itself, it
also contradicts its definition.  Russell's paradox is resolved in ZF set
theory by denying that such a set $S$ exists.}) are resolved.  Wait a
minute---shouldn't the axioms be a starting point, not an ending point?  If
there are paradoxes that arise without the axioms, how do you know you won't
stumble across one accidentally when using the informal approach?

And nowhere do these books describe how ``all of mathematics can be
derived from set theory'' which by now you've heard a few times.

You find a more advanced book on set theory.  This one actually lists the
axioms of ZF set theory in plain English on page one.  {\em Now} you think
your quest has ended and you've finally found the source of all mathematical
knowledge; you just have to understand what it means.  Here, in one place, is
the basis for all of mathematics!  You stare at the axioms in awe, puzzle over
them, memorize them, hoping that if you just meditate on them long enough they
will become clear.  Of course, you haven't the slightest idea how the rest of
mathematics is ``derived'' from them; in particular, if these are the axioms
of mathematics, then why do arithmetic, group theory, and so on need their own
axioms?

You start reading this advanced book carefully, pondering the meaning of every
word, because by now you're really determined to get to the bottom of this.
The first thing the book does is explain how the axioms came about, which was
to resolve Russell's paradox.\index{Russell's paradox}  In fact that seems to
be the main purpose of their existence; that they supposedly can be used to
derive all of mathematics seems irrelevant and is not even mentioned.  Well,
you go on.  You hope the book will explain to you clearly, step by step, how
to derive things from the axioms.  After all, this is the starting point of
mathematics, like a book that explains the basics of a computer programming
language.  But something is missing.  You find you can't even understand the
first proof or do the first exercise.  Symbols such as $\exists$ and $\forall$
permeate the page without any mention of where they came from or how to
manipulate them; the author assumes you are totally familiar with them and
doesn't even tell you what they mean.  By now you know that $\exists$ means
``there exists'' and $\forall$ means ``for all,'' but shouldn't the rules for
manipulating these symbols be part of the axioms?  You still have no idea
how you could even describe the axioms to a computer.

Certainly there is something much different here from the technical
literature you're used to reading.  A computer language manual almost
always explains very clearly what all the symbols mean, precisely what
they do, and the rules used for combining them, and you work your way up
from there.

After glancing at four or five other such books, you come to the realization
that there is another whole field of study that you need just to get to the
point at which you can understand the axioms of set theory.  The field is
called ``logic.''  In fact, some of the books did recommend it as a
prerequisite, but it just didn't sink in.  You assumed logic was, well, just
logic, something that a person with common sense intuitively understood.  Why
waste your time reading boring treatises on symbolic logic, the manipulation
of 1's and 0's that computers do, when you already know that?  But this is a
different kind of logic, quite alien to you.  The subject of {\sc nand} and
{\sc nor} gates is not even touched upon or in any case has to do with only a
very small part of this field.

So your quest continues.  Skimming through the first couple of introductory
books, you get a general idea of what logic is about and what quantifiers
(``for all,'' ``there exists'') mean, but you find their examples somewhat
trivial and mildly annoying (``all dogs are animals,'' ``some animals are
dogs,'' and such).  But all you want to know is what the rules are for
manipulating the symbols so you can apply them to set theory.  Some formulas
describing the relationships among quantifiers ($\exists$ and $\forall$) are
listed in tables, along with some verbal reasoning to justify them.
Presumably, if you want to find out if a formula is correct, you go through
this same kind of mental reasoning process, possibly using images of dogs and
animals. Intuitively, the formulas seem to make sense.  But when you ask
yourself, ``What are the rules I need to get a computer to figure out whether
this formula is correct?'', you still don't know.  Certainly you don't ask the
computer to imagine dogs and animals.

You look at some more advanced logic books.  Many of them have an introductory
chapter summarizing set theory, which turns out to be a prerequisite.  You
need logic to understand set theory, but it seems you also need set theory to
understand logic!  These books jump right into proving rather advanced
theorems about logic, without offering the faintest clue about where the logic
came from that allows them to prove these theorems.

Luckily, you come across an elementary book of logic that, halfway through,
after the usual truth tables and metaphors, presents in a clear, precise way
what you've been looking for all along: the axioms!  They're divided into
propositional calculus (also called sentential logic) and predicate calculus
(also called first-order logic),\index{first-order logic} with rules so simple
and crystal clear that now you can finally program a computer to understand
them.  Indeed, they're no harder than learning how to play a game of chess.
As far as what you seem to need is concerned, the whole book could have been
written in five pages!

{\em Now} you think you've found the ultimate source of mathematical
truth.  So---the axioms of mathematics consist of these axioms of logic,
together with the axioms of ZF set theory. (By now you've also been able to
figure out how to translate the ZF axioms from English into the
actual symbols of logic which you can now manipulate according to
precise, easy-to-understand rules.)

Of course, you still don't understand how ``all of mathematics can be
derived from set theory,'' but maybe this will reveal itself in due
course.

You eagerly set out to program the axioms and rules into a computer and start
to look at the theorems you will have to prove as the logic is developed.  All
sorts of important theorems start popping up:  the deduction
theorem,\index{deduction theorem} the substitution theorem,\index{substitution
theorem} the completeness theorem of propositional calculus,\index{first-order
logic!completeness} the completeness theorem of predicate calculus.  Uh-oh,
there seems to be trouble.  They all get harder and harder, and not one of
them can be derived with the axioms and rules of logic you've just been
handed.  Instead, they all require ``metalogic'' for their proofs, a kind of
mixture of logic and set theory that allows you to prove things {\em about}
the axioms and theorems of logic rather than {\em with} them.

You plow ahead anyway.  A month later, you've spent much of your
free time getting the computer to verify proofs in propositional calculus.
You've programmed in the axioms, but you've also had to program in the
deduction theorem, the substitution theorem, and the completeness theorem of
propositional calculus, which by now you've resigned yourself to treating as
rather complex additional axioms, since they can't be proved from the axioms
you were given.  You can now get the computer to verify and even generate
complete, rigorous, formal proofs\index{formal proof}.  Never mind that they
may have 100,000 steps---at least now you can have complete, absolute
confidence in them.  Unfortunately, the only theorems you have proved are
pretty trivial and you can easily verify them in a few minutes with truth
tables, if not by inspection.

It looks like your mathematician friend was right.  Getting the computer to do
serious mathematics with this kind of rigor seems almost hopeless.  Even
worse, it seems that the further along you get, the more ``axioms'' you have
to add, as each new theorem seems to involve additional ``metamathematical''
reasoning that hasn't been formalized, and none of it can be derived from the
axioms of logic.  Not only do the proofs keep growing exponentially as you get
further along, but the program to verify them keeps getting bigger and bigger
as you program in more ``metatheorems.''\index{metatheorem}\footnote{A
metatheorem is usually a statement that is too general to be directly provable
in a theory.  For example, ``if $n_1$, $n_2$, and $n_3$ are integers, then
$n_1+n_2+n_3$ is an integer'' is a theorem of number theory.  But ``for any
integer $k > 1$, if $n_1, \ldots, n_k$ are integers, then $n_1+\ldots +n_k$ is
an integer'' is a metatheorem, in other words a family of theorems, one for
every $k$.  The reason it is not a theorem is that the general sum $n_1+\ldots
+n_k$ (as a function of $k$) is not an operation that can be defined directly
in number theory.} The bugs\index{computer program bugs} that have cropped up
so far have already made you start to lose faith in the rigor you seem to have
achieved, and you know it's just going to get worse as your program gets larger.

\subsection{Mathematics and the Non-Specialist}

\begin{quote}
  {\em A real proof is not checkable by a machine, or even by any mathematician
not privy to the gestalt, the mode of thought of the particular field of
mathematics in which the proof is located.}
  \flushright\sc  Davis and Hersh\index{Davis, Phillip J.}
  \footnote{\cite{Davis}, p.~354.}\\
\end{quote}

The bulk of abstract or theoretical mathematics is ordinarily outside
the reach of anyone but a few specialists in each field who have completed
the necessary difficult internship in order to enter its coterie.  The
typical intelligent layperson has no reasonable hope of understanding much of
it, nor even the specialist mathematician of understanding other fields.  It
is like a foreign language that has no dictionary to look up the translation;
the only way you can learn it is by living in the country for a few years.  It
is argued that the effort involved in learning a specialty is a necessary
process for acquiring a deep understanding.  Of course, this is almost certainly
true if one is to make significant contributions to a field; in particular,
``doing'' proofs is probably the most important part of a mathematician's
training.  But is it also necessary to deny outsiders access to it?  Is it
necessary that abstract mathematics be so hard for a layperson to grasp?

A computer normally is of no help whatsoever.  Most published proofs are
actually just series of hints written in an informal style that requires
considerable knowledge of the field to understand.  These are the ``real
proofs'' referred to by Davis and Hersh.\index{informal proof}  There is an
implicit understanding that, in principle, such a proof could be converted to
a complete formal proof\index{formal proof}.  However, it is said that no one
would ever attempt such a conversion, even if they could, because that would
presumably require millions of steps (Section~\ref{dream}).  Unfortunately the
informal style automatically excludes the understanding of the proof
by anyone who hasn't gone through the necessary apprenticeship. The
best that the intelligent layperson can do is to read popular books about deep
and famous results; while this can be helpful, it can also be misleading, and
the lack of detail usually leaves the reader with no ability whatsoever to
explore any aspect of the field being described.

The statements of theorems often use sophisticated notation that makes them
inaccessible to the non-specialist.  For a non-specialist who wants to achieve
a deeper understanding of a proof, the process of tracing definitions and
lemmas back through their hierarchy\index{hierarchy} quickly becomes confusing
and discouraging.  Textbooks are usually written to train mathematicians or to
communicate to people who are already mathematicians, and large gaps in proofs
are often left as exercises to the reader who is left at an impasse if he or
she becomes stuck.

I believe that eventually computers will enable non-specialists and even
intelligent laypersons to follow almost any mathematical proof in any field.
Metamath is an attempt in that direction.  If all of mathematics were as
easily accessible as a computer programming language, I could envision
computer programmers and hobbyists who otherwise lack mathematical
sophistication exploring and being amazed by the world of theorems and proofs
in obscure specialties, perhaps even coming up with results of their own.  A
tremendous advantage would be that anyone could experiment with conjectures in
any field---the computer would offer instant feedback as to whether
an inference step was correct.

Mathematicians sometimes have to put up with the annoyance of
cranks\index{cranks} who lack a fundamental understanding of mathematics but
insist that their ``proofs'' of, say, Fermat's Last Theorem\index{Fermat's
Last Theorem} be taken seriously.  I think part of the problem is that these
people are misled by informal mathematical language, treating it as if they
were reading ordinary expository English and failing to appreciate the
implicit underlying rigor.  Such cranks are rare in the field of computers,
because computer languages are much more explicit, and ultimately the proof is
in whether a computer program works or not.  With easily accessible
computer-based abstract mathematics, a mathematician could say to a crank,
``don't bother me until you've demonstrated your claim on the computer!''

% 22-May-04 nm
% Attempt to move De Millo quote so it doesn't separate from attribution
% CHANGE THIS NUMBER (AND ELIMINATE IF POSSIBLE) WHEN ABOVE TEXT CHANGES
\vspace{-0.5em}

\subsection{An Impossible Dream?}\label{dream}

\begin{quote}
  {\em Even quite basic theorems would demand almost unbelievably vast
  books to display their proofs.}
    \flushright\sc  Robert E. Edwards\footnote{\cite{Edwards}, p.~68.}\\
\end{quote}\index{Edwards, Robert E.}

\begin{quote}
  {\em Oh, of course no one ever really {\em does} it.  It would take
  forever!  You just show that you could do it, that's sufficient.}
    \flushright\sc  ``The Ideal Mathematician''
    \index{Davis, Phillip J.}\footnote{\cite{Davis},
p.~40.}\\
\end{quote}

\begin{quote}
  {\em There is a theorem in the primitive notation of set theory that
  corresponds to the arithmetic theorem `$1000+2000=3000$'.  The formula
  would be forbiddingly long\ldots even if [one] knows the definitions
  and is asked to simplify the long formula according to them, chances are
  he will make errors and arrive at some incorrect result.}
    \flushright\sc  Hao Wang\footnote{\cite{Wang}, p.~140.}\\
\end{quote}\index{Wang, Hao}

% 22-May-04 nm
% Attempt to move De Millo quote so it doesn't separate from attribution
% CHANGE THIS NUMBER (AND ELIMINATE IF POSSIBLE) WHEN ABOVE TEXT CHANGES
\vspace{-0.5em}

\begin{quote}
  {\em The {\em Principia Mathematica} was the crowning achievement of the
  formalists.  It was also the deathblow of the formalist view.\ldots
  {[Rus\-sell]} failed, in three enormous volumes, to get beyond the elementary
  facts of arithmetic.  He showed what can be done in principle and what
  cannot be done in practice.  If the mathematical process were really
  one of strict, logical progression, we would still be counting our
  fingers.\ldots
  One theoretician estimates\ldots that a demonstration of one of
  Ramanujan's conjectures assuming set theory and elementary analysis would
  take about two thousand pages; the length of a deduction from first principles
  is nearly in\-con\-ceiv\-a\-ble\ldots The probabilists argue that\ldots any
  very long proof can at best be viewed as only probably correct\ldots}
  \flushright\sc Richard de Millo et. al.\footnote{\cite{deMillo}, pp.~269,
  271.}\\
\end{quote}\index{de Millo, Richard}

A number of writers have conveyed the impression that the kind of absolute
rigor provided by Metamath\index{Metamath} is an impossible dream, suggesting
that a complete, formal verification\index{formal proof} of a typical theorem
would take millions of steps in untold volumes of books.  Even if it could be
done, the thinking sometimes goes, all meaning would be lost in such a
monstrous, tedious verification.\index{informal proof}\index{proof length}

These writers assume, however, that in order to achieve the kind of complete
formal verification they desire one must break down a proof into individual
primitive steps that make direct reference to the axioms.  This is
not necessary.  There is no reason not to make use of previously proved
theorems rather than proving them over and over.

Just as important, definitions\index{definition} can be introduced along
the way, allowing very complex formulas to be represented with few
symbols.  Not doing this can lead to absurdly long formulas.  For
example, the mere statement of
G\"{o}del's incompleteness theorem\index{G\"{o}del's
incompleteness theorem}, which can be expressed with a small number of
defined symbols, would require about 20,000 primitive symbols to express
it.\index{Boolos, George S.}\footnote{George S.\ Boolos, lecture at
Massachusetts Institute of Technology, spring 1990.} An extreme example
is Bourbaki's\label{bourbaki} language for set theory, which requires
4,523,659,424,929 symbols plus 1,179,618,517,981 disambiguatory links
(lines connecting symbol pairs, usually drawn below or above the
formula) to express the number
``one'' \cite{Mathias}.\index{Mathias, Adrian R. D.}\index{Bourbaki,
Nicolas}
% http://www.dpmms.cam.ac.uk/~ardm/

A hierarchy\index{hierarchy} of theorems and definitions permits an
exponential growth in the formula sizes and primitive proof steps to be
described with only a linear growth in the number of symbols used.  Of course,
this is how ordinary informal mathematics is normally done anyway, but with
Metamath\index{Metamath} it can be done with absolute rigor and precision.

\subsection{Beauty}


\begin{quote}
  {\em No one shall be able to drive us from the paradise that Cantor has
created for us.}
   \flushright\sc  David Hilbert\footnote{As quoted in \cite{Moore}, p.~131.}\\
\end{quote}\index{Hilbert, David}

\needspace{3\baselineskip}
\begin{quote}
  {\em Mathematics possesses not only truth, but some supreme beauty ---a
  beauty cold and austere, like that of a sculpture.}
    \flushright\sc  Bertrand
    Russell\footnote{\cite{Russell}.}\\
\end{quote}\index{Russell, Bertrand}

\begin{quote}
  {\em Euclid alone has looked on Beauty bare.}
  \flushright\sc Edna Millay\footnote{As quoted in \cite{Davis}, p.~150.}\\
\end{quote}\index{Millay, Edna}

For most people, abstract mathematics is distant, strange, and
incomprehensible.  Many popular books have tried to convey some of the sense
of beauty in famous theorems.  But even an intelligent layperson is left with
only a general idea of what a theorem is about and is hardly given the tools
needed to make use of it.  Traditionally, it is only after years of arduous
study that one can grasp the concepts needed for deep understanding.
Metamath\index{Metamath} allows you to approach the proof of the theorem from
a quite different perspective, peeling apart the formulas and definitions
layer by layer until an entirely different kind of understanding is achieved.
Every step of the proof is there, pieced together with absolute precision and
instantly available for inspection through a microscope with a magnification
as powerful as you desire.

A proof in itself can be considered an object of beauty.  Constructing an
elegant proof is an art.  Once a famous theorem has been proved, often
considerable effort is made to find simpler and more easily understood
proofs.  Creating and communicating elegant proofs is a major concern of
mathematicians.  Metamath is one way of providing a common language for
archiving and preserving this information.

The length of a proof can, to a certain extent, be considered an
objective measure of its ``beauty,'' since shorter proofs are usually
considered more elegant.  In the set theory database
\texttt{set.mm}\index{set theory database (\texttt{set.mm})}%
\index{Metamath Proof Explorer}
provided with Metamath, one goal was to make all proofs as short as possible.

\needspace{4\baselineskip}
\subsection{Simplicity}

\begin{quote}
  {\em God made man simple; man's complex problems are of his own
  devising.}
    \flushright\sc Eccles. 7:29\footnote{Jerusalem Bible.}\\
\end{quote}\index{Bible}

\needspace{3\baselineskip}
\begin{quote}
  {\em God made integers, all else is the work of man.}
    \flushright\sc Leopold Kronecker\footnote{{\em Jahresbericht
	der Deutschen Mathematiker-Vereinigung }, vol. 2, p. 19.}\\
\end{quote}\index{Kronecker, Leopold}

\needspace{3\baselineskip}
\begin{quote}
  {\em For what is clear and easily comprehended attracts; the
  complicated repels.}
    \flushright\sc David Hilbert\footnote{As quoted in \cite{deMillo},
p.~273.}\\
\end{quote}\index{Hilbert, David}

The Metamath\index{Metamath} language is simple and Spartan.  Metamath treats
all mathematical expressions as simple sequences of symbols, devoid of meaning.
The higher-level or ``metamathematical'' notions underlying Metamath are about
as simple as they could possibly be.  Each individual step in a proof involves
a single basic concept, the substitution of an expression for a variable, so
that in principle almost anyone, whether mathematician or not, can
completely understand how it was arrived at.

In one of its most basic applications, Metamath\index{Metamath} can be used to
develop the foundations of mathematics\index{foundations of mathematics} from
the very beginning.  This is done in the set theory database that is provided
with the Metamath package and is the subject matter
of Chapter~\ref{fol}. Any language (a metalanguage\index{metalanguage})
used to describe mathematics (an object language\index{object language}) must
have a mathematical content of its own, but it is desirable to keep this
content down to a bare minimum, namely that needed to make use of the
inference rules specified by the axioms.  With any metalanguage there is a
``chicken and egg'' problem somewhat like circular reasoning:  you must assume
the validity of the mathematics of the metalanguage in order to prove the
validity of the mathematics of the object language.  The mathematical content
of Metamath itself is quite limited.  Like the rules of a game of chess, the
essential concepts are simple enough so that virtually anyone should be able to
understand them (although that in itself will not let you play like
a master).  The symbols that Metamath manipulates do not in themselves
have any intrinsic meaning.  Your interpretation of the axioms that you supply
to Metamath is what gives them meaning.  Metamath is an attempt to strip down
mathematical thought to its bare essence and show you exactly how the symbols
are manipulated.

Philosophers and logicians, with various motivations, have often thought it
important to study ``weak'' fragments of logic\index{weak logic}
\cite{Anderson}\index{Anderson, Alan Ross} \cite{MegillBunder}\index{Megill,
Norman}\index{Bunder, Martin}, other unconventional systems of logic (such as
``modal'' logic\index{modal logic} \cite[ch.\ 27]{Boolos}\index{Boolos, George
S.}), and quantum logic\index{quantum logic} in physics
\cite{Pavicic}\index{Pavi{\v{c}}i{\'{c}}, M.}.  Metamath\index{Metamath}
provides a framework in which such systems can be expressed, with an absolute
precision that makes all underlying metamathematical assumptions rigorous and
crystal clear.

Some schools of philosophical thought, for example
intuitionism\index{intuitionism} and constructivism\index{constructivism},
demand that the notions underlying any mathematical system be as simple and
concrete as possible.  Metamath should meet the requirements of these
philosophies.  Metamath must be taught the symbols, axioms\index{axiom}, and
rules\index{rule} for a specific theory, from the skeptical (such as
intuitionism\index{intuitionism}\footnote{Intuitionism does not accept the law
of excluded middle (``either something is true or it is not true'').  See
\cite[p.~xi]{Tymoczko}\index{Tymoczko, Thomas} for discussion and references
on this topic.  Consider the theorem, ``There exist irrational numbers $a$ and
$b$ such that $a^b$ is rational.''  An intuitionist would reject the following
proof:  If $\sqrt{2}^{\sqrt{2}}$ is rational, we are done.  Otherwise, let
$a=\sqrt{2}^{\sqrt{2}}$ and $b=\sqrt{2}$. Then $a^b=2$, which is rational.})
to the bold (such as the axiom of choice in set theory\footnote{The axiom of
choice\index{Axiom of Choice} asserts that given any collection of pairwise
disjoint nonempty sets, there exists a set that has exactly one element in
common with each set of the collection.  It is used to prove many important
theorems in standard mathematics.  Some philosophers object to it because it
asserts the existence of a set without specifying what the set contains
\cite[p.~154]{Enderton}\index{Enderton, Herbert B.}.  In one foundation for
mathematics due to Quine\index{Quine, Willard Van Orman}, that has not been
otherwise shown to be inconsistent, the axiom of choice turns out to be false
\cite[p.~23]{Curry}\index{Curry, Haskell B.}.  The \texttt{show
trace{\char`\_}back} command of the Metamath program allows you to find out
whether the axiom of choice, or any other axiom, was assumed by a
proof.}\index{\texttt{show trace{\char`\_}back} command}).

The simplicity of the Metamath language lets the algorithm (computer program)
that verifies the validity of a Metamath proof be straightforward and
robust.  You can have confidence that the theorems it verifies really can be
derived from your axioms.

\subsection{Rigor}

\begin{quote}
  {\em Rigor became a goal with the Greeks\ldots But the efforts to
  pursue rigor to the utmost have led to an impasse in which there is
  no longer any agreement on what it really means.  Mathematics remains
  alive and vital, but only on a pragmatic basis.}
    \flushright\sc  Morris Kline\footnote{\cite{Kline}, p.~1209.}\\
\end{quote}\index{Kline, Morris}

Kline refers to a much deeper kind of rigor than that which we will discuss in
this section.  G\"{o}del's incompleteness theorem\index{G\"{o}del's
incompleteness theorem} showed that it is impossible to achieve absolute rigor
in standard mathematics because we can never prove that mathematics is
consistent (free from contradictions).\index{consistent theory}  If
mathematics is consistent, we will never know it, but must rely on faith.  If
mathematics is inconsistent, the best we can hope for is that some clever
future mathematician will discover the inconsistency.  In this case, the
axioms would probably be revised slightly to eliminate the inconsistency, as
was done in the case of Russell's paradox,\index{Russell's paradox} but the
bulk of mathematics would probably not be affected by such a discovery.
Russell's paradox, for example, did not affect most of the remarkable results
achieved by 19th-century and earlier mathematicians.  It mainly invalidated
some of Gottlob Frege's\index{Frege, Gottlob} work on the foundations of
mathematics in the late 1800's; in fact Frege's work inspired Russell's
discovery.  Despite the paradox, Frege's work contains important concepts that
have significantly influenced modern logic.  Kline's {\em Mathematics, The
Loss of Certainty} \cite{Klinel}\index{Kline, Morris} has an interesting
discussion of this topic.

What {\em can} be achieved with absolute certainty\index{certainty} is the
knowledge that if we assume the axioms are consistent and true, then the
results derived from them are true.  Part of the beauty of mathematics is that
it is the one area of human endeavor where absolute certainty can be achieved
in this sense.  A mathematical truth will remain such for eternity.  However,
our actual knowledge of whether a particular statement is a mathematical truth
is only as certain as the correctness of the proof that establishes it.  If
the proof of a statement is questionable or vague, we can't have absolute
confidence in the truth that the statement claims.

Let us look at some traditional ways of expressing proofs.

Except in the field of formal logic\index{formal logic}, almost all
traditional proofs in mathematics are really not proofs at all, but rather
proof outlines or hints as to how to go about constructing the proof.  Many
gaps\index{gaps in proofs} are left for the reader to fill in. There are
several reasons for this.  First, it is usually assumed in mathematical
literature that the person reading the proof is a mathematician familiar with
the specialty being described, and that the missing steps are obvious to such
a reader or at least that the reader is capable of filling them in.  This
attitude is fine for professional mathematicians in the specialty, but
unfortunately it often has the drawback of cutting off the rest of the world,
including mathematicians in other specialties, from understanding the proof.
We discussed one possible resolution to this on p.~\pageref{envision}.
Second, it is often assumed that a complete formal proof\index{formal proof}
would require countless millions of symbols (Section~\ref{dream}). This might
be true if the proof were to be expressed directly in terms of the axioms of
logic and set theory,\index{set theory} but it is usually not true if we allow
ourselves a hierarchy\index{hierarchy} of definitions and theorems to build
upon, using a notation that allows us to introduce new symbols, definitions,
and theorems in a precisely specified way.

Even in formal logic,\index{formal logic} formal proofs\index{formal proof}
that are considered complete still contain hidden or implicit information.
For example, a ``proof'' is usually defined as a sequence of
wffs,\index{well-formed formula (wff)}\footnote{A {\em wff} or well-formed
formula is a mathematical expression (string of symbols) constructed according
to some precise rules.  A formal mathematical system\index{formal system}
contains (1) the rules for constructing syntactically correct
wffs,\index{syntax rules} (2) a list of starting wffs called
axioms,\index{axiom} and (3) one or more rules prescribing how to derive new
wffs, called theorems\index{theorem}, from the axioms or previously derived
theorems.  An example of such a system is contained in
Metamath's\index{Metamath} set theory database, which defines a formal
system\index{formal system} from which all of standard mathematics can be
derived.  Section~\ref{startf} steps you through a complete example of a formal
system, and you may want to skim it now if you are unfamiliar with the
concept.} each of which is an axiom or follows from a rule applied to previous
wffs in the sequence.  The implicit part of the proof is the algorithm by
which a sequence of symbols is verified to be a valid wff, given the
definition of a wff.  The algorithm in this case is rather simple, but for a
computer to verify the proof,\index{automated proof verification} it must have
the algorithm built into its verification program.\footnote{It is possible, of
course, to specify wff construction syntax outside of the program itself
with a suitable input language (the Metamath language being an example), but
some proof-verification or theorem-proving programs lack the ability to extend
wff syntax in such a fashion.} If one deals exclusively with axioms and
elementary wffs, it is straightforward to implement such an algorithm.  But as
more and more definitions are added to the theory in order to make the
expression of wffs more compact, the algorithm becomes more and more
complicated.  A computer program that implements the algorithm becomes larger
and harder to understand as each definition is introduced, and thus more prone
to bugs.\index{computer program bugs}  The larger the program, the
more suspicious the mathematician may be about
the validity of its algorithms.  This is especially true because
computer programs are inherently hard to follow to begin with, and few people
enjoy verifying them manually in detail.

Metamath\index{Metamath} takes a different approach.  Metamath's ``knowledge''
is limited to the ability to substitute variables for expressions, subject to
some simple constraints.  Once the basic algorithm of Metamath is assumed to
be debugged, and perhaps independently confirmed, it
can be trusted once and for all.  The information that Metamath needs to
``understand'' mathematics is contained entirely in the body of knowledge
presented to Metamath.  Any errors in reasoning can only be errors in the
axioms or definitions contained in this body of knowledge.  As a
``constructive'' language\index{constructive language} Metamath has no
conditional branches or loops like the ones that make computer programs hard
to decipher; instead, the language can only build new sequences of symbols
from earlier sequences  of symbols.

The simplicity of the rules that underlie Metamath not only makes Metamath
easy to learn but also gives Metamath a great deal of flexibility. For
example, Metamath is not limited to describing standard first-order
logic\index{first-order logic}; higher-order logics\index{higher-order logic}
and fragments of logic\index{weak logic} can be described just as easily.
Metamath gives you the freedom to define whatever wff notation you prefer; it
has no built-in conception of the syntax of a wff.\index{well-formed formula
(wff)}  With suitable axioms and definitions, Metamath can even describe and
prove things about itself.\index{Metamath!self-description}  (John
Harrison\index{Harrison, John} discusses the ``reflection''
principle\index{reflection principle} involved in self-descriptive systems in
\cite{Harrison}.)

The flexibility of Metamath requires that its proofs specify a lot of detail,
much more than in an ordinary ``formal'' proof.\index{formal proof}  For
example, in an ordinary formal proof, a single step consists of displaying the
wff that constitutes that step.  In order for a computer program to verify
that the step is acceptable, it first must verify that the symbol sequence
being displayed is an acceptable wff.\index{automated proof verification} Most
proof verifiers have at least basic wff syntax built into their programs.
Metamath has no hard-wired knowledge of what constitutes a wff built into it;
instead every wff must be explicitly constructed based on rules defining wffs
that are present in a database.  Thus a single step in an ordinary formal
proof may be correspond to many steps in a Metamath proof. Despite the larger
number of steps, though, this does not mean that a Metamath proof must be
significantly larger than an ordinary formal proof. The reason is that since
we have constructed the wff from scratch, we know what the wff is, so there is
no reason to display it.  We only need to refer to a sequence of statements
that construct it.  In a sense, the display of the wff in an ordinary formal
proof is an implicit proof of its own validity as a wff; Metamath just makes
the proof explicit. (Section~\ref{proof} describes Metamath's proof notation.)

\section{Computers and Mathematicians}

\begin{quote}
  {\em The computer is important, but not to mathematics.}
    \flushright\sc  Paul Halmos\footnote{As quoted in \cite{Albers}, p.~121.}\\
\end{quote}\index{Halmos, Paul}

Pure mathematicians have traditionally been indifferent to computers, even to
the point of disdain.\index{computers and pure mathematics}  Computer science
itself is sometimes considered to fall in the mundane realm of ``applied''
mathematics, perhaps essential for the real world but intellectually unexciting
to those who seek the deepest truths in mathematics.  Perhaps a reason for this
attitude towards computers is that there is little or no computer software that
meets their needs, and there may be a general feeling that such software could
not even exist.  On the one hand, there are the practical computer algebra
systems, which can perform amazing symbolic manipulations in algebra and
calculus,\index{computer algebra system} yet can't prove the simplest
existence theorem, if the idea of a proof is present at all.  On the other
hand, there are specialized automated theorem provers that technically speaking
may generate correct proofs.\index{automated theorem proving}  But sometimes
their specialized input notation may be cryptic and their output perceived to
be long, inelegant, incomprehensible proofs.    The output
may be viewed with suspicion, since the program that generates it tends to be
very large, and its size increases the potential for bugs\index{computer
program bugs}.  Such a proof may be considered trustworthy only if
independently verified and ``understood'' by a human, but no one wants to
waste their time on such a boring, unrewarding chore.



\needspace{4\baselineskip}
\subsection{Trusting the Computer}

\begin{quote}
  {\em \ldots I continue to find the quasi-empirical interpretation of
  computer proofs to be the more plausible.\ldots Since not
  everything that claims to be a computer proof can be
  accepted as valid, what are the mathematical criteria for acceptable
  computer proofs?}
    \flushright\sc  Thomas Tymoczko\footnote{\cite{Tymoczko}, p.~245.}\\
\end{quote}\index{Tymoczko, Thomas}

In some cases, computers have been essential tools for proving famous
theorems.  But if a proof is so long and obscure that it can be verified in a
practical way only with a computer, it is vaguely felt to be suspicious.  For
example, proving the famous four-color theorem\index{four-color
theorem}\index{proof length} (``a map needs no more than four colors to
prevent any two adjacent countries from having the same color'') can presently
only be done with the aid of a very complex computer program which originally
required 1200 hours of computer time. There has been considerable debate about
whether such a proof can be trusted and whether such a proof is ``real''
mathematics \cite{Swart}\index{Swart, E. R.}.\index{trusting computers}

However, under normal circumstances even a skeptical mathematician would have a
great deal of confidence in the result of multiplying two numbers on a pocket
calculator, even though the precise details of what goes on are hidden from its
user.  Even the verification on a supercomputer that a huge number is prime is
trusted, especially if there is independent verification; no one bothers to
debate the philosophical significance of its ``proof,'' even though the actual
proof would be so large that it would be completely impractical to ever write
it down on paper.  It seems that if the algorithm used by the computer is
simple enough to be readily understood, then the computer can be trusted.

Metamath\index{Metamath} adopts this philosophy.  The simplicity of its
language makes it easy to learn, and because of its simplicity one can have
essentially absolute confidence that a proof is correct. All axioms, rules, and
definitions are available for inspection at any time because they are defined
by the user; there are no hidden or built-in rules that may be prone to subtle
bugs\index{computer program bugs}.  The basic algorithm at the heart of
Metamath is simple and fixed, and it can be assumed to be bug-free and robust
with a degree of confidence approaching certainty.
Independently written implementations of the Metamath verifier
can reduce any residual doubt on the part of a skeptic even further;
there are now over a dozen such implementations, written by many people.

\subsection{Trusting the Mathematician}\label{trust}

\begin{quote}
  {\em There is no Algebraist nor Mathematician so expert in his science, as
  to place entire confidence in any truth immediately upon his discovery of it,
  or regard it as any thing, but a mere probability.  Every time he runs over
  his proofs, his confidence encreases; but still more by the approbation of
  his friends; and is rais'd to its utmost perfection by the universal assent
  and applauses of the learned world.}
  \flushright\sc David Hume\footnote{{\em A Treatise of Human Nature}, as
  quoted in \cite{deMillo}, p.~267.}\\
\end{quote}\index{Hume, David}

\begin{quote}
  {\em Stanislaw Ulam estimates that mathematicians publish 200,000 theorems
  every year.  A number of these are subsequently contradicted or otherwise
  disallowed, others are thrown into doubt, and most are ignored.}
  \flushright\sc Richard de Millo et. al.\footnote{\cite{deMillo}, p.~269.}\\
\end{quote}\index{Ulam, Stanislaw}

Whether or not the computer can be trusted, humans  of course will occasionally
err. Only the most memorable proofs get independently verified, and of these
only a handful of truly great ones achieve the status of being ``known''
mathematical truths that are used without giving a second thought to their
correctness.

There are many famous examples of incorrect theorems and proofs in
mathematical literature.\index{errors in proofs}

\begin{itemize}
\item There have been thousands of purported proofs of Fermat's Last
Theorem\index{Fermat's Last Theorem} (``no integer solutions exist to $x^n +
y^n = z^n$ for $n > 2$''), by amateurs, cranks, and well-regarded
mathematicians \cite[p.~5]{Stark}\index{Stark, Harold M}.  Fermat wrote a note
in his copy of Bachet's {\em Diophantus} that he found ``a truly marvelous
proof of this theorem but this margin is too narrow to contain it''
\cite[p.~507]{Kramer}.  A recent, much publicized proof by Yoichi
Miyaoka\index{Miyaoka, Yoichi} was shown to be incorrect ({\em Science News},
April 9, 1988, p.~230).  The theorem was finally proved by Andrew
Wiles\index{Wiles, Andrew} ({\em Science News}, July 3, 1993, p.~5), but it
initially had some gaps and took over a year after its announcement to be
checked thoroughly by experts.  On Oct. 25, 1994, Wiles announced that the last
gap found in his proof had been filled in.
  \item In 1882, M. Pasch discovered that an axiom was omitted from Euclid's
formulation of geometry\index{Euclidean geometry}; without it, the proofs of
important theorems of Euclid are not valid.  Pasch's axiom\index{Pasch's
axiom} states that a line that intersects one side of a triangle must also
intersect another side, provided that it does not touch any of the triangle's
vertices.  The omission of Pasch's axiom went unnoticed for 2000
years \cite[p.~160]{Davis}, in spite of (one presumes) the thousands of
students, instructors, and mathematicians who studied Euclid.
  \item The first published proof of the famous Schr\"{o}der--Bernstein
theorem\index{Schr\"{o}der--Bernstein theorem} in set theory was incorrect
\cite[p.~148]{Enderton}\index{Enderton, Herbert B.}.  This theorem states
that if there exists a 1-to-1 function\footnote{A {\em set}\index{set} is any
collection of objects. A {\em function}\index{function} or {\em
mapping}\index{mapping} is a rule that assigns to each element of one set
(called the function's {\em domain}\index{domain}) an element from another
set.} from set $A$ into set $B$ and vice-versa, then sets $A$ and $B$ have
a 1-to-1 correspondence.  Although it sounds simple and obvious,
the standard proof is quite long and complex.
  \item In the early 1900's, Hilbert\index{Hilbert, David} published a
purported proof of the continuum hypothesis\index{continuum hypothesis}, which
was eventually established as unprovable by Cohen\index{Cohen, Paul} in 1963
\cite[p.~166]{Enderton}.  The continuum hypothesis states that no
infinity\index{infinity} (``transfinite cardinal number'')\index{cardinal,
transfinite} exists whose size (or ``cardinality''\index{cardinality}) is
between the size of the set of integers and the size of the set of real
numbers.  This hypothesis originated with German mathematician Georg
Cantor\index{Cantor, Georg} in the late 1800's, and his inability to prove it
is said to have contributed to mental illness that afflicted him in his later
years.
  \item An incorrect proof of the four-color theorem\index{four-color theorem}
was published by Kempe\index{Kempe, A. B.} in 1879
\cite[p.~582]{Courant}\index{Courant, Richard}; it stood for 11 years before
its flaw was discovered.  This theorem states that any map can be colored
using only four colors, so that no two adjacent countries have the same
color.  In 1976 the theorem was finally proved by the famous computer-assisted
proof of Haken, Appel, and Koch \cite{Swart}\index{Appel, K.}\index{Haken,
W.}\index{Koch, K.}.  Or at least it seems that way.  Mathematician
H.~S.~M.~Coxeter\index{Coxeter, H. S. M.} has doubts \cite[p.~58]{Davis}:  ``I
have a feeling that that is an untidy kind of use of the computers, and the more
you correspond with Haken and Appel, the more shaky you seem to be.''
  \item Many false ``proofs'' of the Poincar\'{e}
conjecture\index{Poincar\'{e} conjecture} have been proposed over the years.
This conjecture states that any object that mathematically behaves like a
three-dimensional sphere is a three-dimensional sphere topologically,
regardless of how it is distorted.  In March 1986, mathematicians Colin
Rourke\index{Rourke, Colin} and Eduardo R\^{e}go\index{R\^{e}go, Eduardo}
caused  a stir in the mathematical community by announcing that they had found
a proof; in November of that year the proof was found to be false \cite[p.
218]{PetersonI}.  It was finally proved in 2003 by Grigory Perelman
\label{poincare}\index{Szpiro, George}\index{Perelman, Grigory}\cite{Szpiro}.
 \end{itemize}

Many counterexamples to ``theorems'' in recent mathematical
literature related to Clifford algebras\index{Clifford algebras}
 have been found by Pertti
Lounesto (who passed away in 2002).\index{Lounesto, Pertti}
See the web page \url{http://mathforum.org/library/view/4933.html}.
% http://users.tkk.fi/~ppuska/mirror/Lounesto/counterexamples.htm

One of the purposes of Metamath\index{Metamath} is to allow proofs to be
expressed with absolute precision.  Developing a proof in the Metamath
language can be challenging, because Metamath will not permit even the
tiniest mistake.\index{errors in proofs}  But once the proof is created, its
correctness can be trusted immediately, without having to depend on the
process of peer review for confirmation.

\section{The Use of Computers in Mathematics}

\subsection{Computer Algebra Systems}

For the most part, you will find that Metamath\index{Metamath} is not a
practical tool for manipulating numbers.  (Even proving that $2 + 2 = 4$, if
you start with set theory, can be quite complex!)  Several commercial
mathematics packages are quite good at arithmetic, algebra, and calculus, and
as practical tools they are invaluable.\index{computer algebra system} But
they have no notion of proof, and cannot understand statements starting with
``there exists such and such...''.

Software packages such as Mathematica \cite{Wolfram}\index{Mathematica} do not
concern themselves with proofs but instead work directly with known results.
These packages primarily emphasize heuristic rules such as the substitution of
equals for equals to achieve simpler expressions or expressions in a different
form.  Starting with a rich collection of built-in rules and algorithms, users
can add to the collection by means of a powerful programming language.
However, results such as, say, the existence of a certain abstract object
without displaying the actual object cannot be expressed (directly) in their
languages.  The idea of a proof from a small set of axioms is absent.  Instead
this software simply assumes that each fact or rule you add to the built-in
collection of algorithms is valid.  One way to view the software is as a large
collection of axioms from which the software, with certain goals, attempts to
derive new theorems, for example equating a complex expression with a simpler
equivalent. But the terms ``theorem''\index{theorem} and
``proof,''\index{proof} for example, are not even mentioned in the index of
the user's manual for Mathematica.\index{Mathematica and proofs}  What is also
unsatisfactory from a philosophical point of view is that there is no way to
ensure the validity of the results other than by trusting the writer of each
application module or tediously checking each module by hand, similar to
checking a computer program for bugs.\index{computer program
bugs}\footnote{Two examples illustrate why the knowledge database of computer
algebra systems should sometimes be regarded with a certain caution.  If you
ask Mathematica (version 3.0) to \texttt{Solve[x\^{ }n + y\^{ }n == z\^{ }n , n]}
it will respond with \texttt{\{\{n-\char`\>-2\}, \{n-\char`\>-1\},
\{n-\char`\>1\}, \{n-\char`\>2\}\}}. In other words, Mathematica seems to
``know'' that Fermat's Last Theorem\index{Fermat's Last Theorem} is true!  (At
the time this version of Mathematica was released this fact was unknown.)  If
you ask Maple\index{Maple} to \texttt{solve(x\^{ }2 = 2\^{ }x)} then
\texttt{simplify(\{"\})}, it returns the solution set \texttt{\{2, 4\}}, apparently
unaware that 0.7666647\ldots is also a solution.} While of course extremely
valuable in applied mathematics, computer algebra systems tend to be of little
interest to the theoretical mathematician except as aids for exploring certain
specific problems.

Because of possible bugs, trusting the output of a computer algebra system for
use as theorems in a proof-verifier would defeat the latter's goal of rigor.
On the other hand, a fact such that a certain relatively large number is
prime, while easy for a computer algebra system to derive, might have a long,
tedious proof that could overwhelm a proof-verifier. One approach for linking
computer algebra systems to a proof-verifier while retaining the advantages of
both is to add a hypothesis to each such theorem indicating its source.  For
example, a constant {\sc maple} could indicate the theorem came from the Maple
package, and would mean ``assuming Maple is consistent, then\ldots''  This and
many other topics concerning the formalization of mathematics are discussed in
John Harrison's\index{Harrison, John} very interesting
PhD thesis~\cite{Harrison-thesis}.

\subsection{Automated Theorem Provers}\label{theoremprovers}

A mathematical theory is ``decidable''\index{decidable theory} if a mechanical
method or algorithm exists that is guaranteed to determine whether or not a
particular formula is a theorem.  Among the few theories that are decidable is
elementary geometry,\index{Euclidean geometry} as was shown by a classic
result of logician Alfred Tarski\index{Tarski, Alfred} in 1948
\cite{Tarski}.\footnote{Tarski's result actually applies to a subset of the
geometry discussed in elementary textbooks.  This subset includes most of what
would be considered elementary geometry but it is not powerful enough to
express, among other things, the notions of the circumference and area of a
circle.  Extending the theory in a way that includes notions such as these
makes the theory undecidable, as was also shown by Tarski.  Tarski's algorithm
is far too inefficient to implement practically on a computer.  A practical
algorithm for proving a smaller subset of geometry theorems (those not
involving concepts of ``order'' or ``continuity'') was discovered by Chinese
mathematician Wu Wen-ts\"{u}n in 1977 \cite{Chou}\index{Chou,
Shang-Ching}.}\index{Wen-ts{\"{u}}n, Wu}  But most theories, including
elementary arithmetic, are undecidable.  This fact contributes to keeping
mathematics alive and well, since many mathematicians believe
that they will never be
replaced by computers (if they believe Roger Penrose's argument that a
computer can never replace the brain \cite{Penrose}\index{Penrose, Roger}).
In fact,  elementary geometry is often considered a ``dead'' field
for the simple reason that it is decidable.

On the other hand, the undecidability of a theory does not mean that one cannot
use a computer to search for proofs, providing one is willing to give up if a
proof is not found after a reasonable amount of time.  The field of automated
theorem proving\index{automated theorem proving} specializes in pursuing such
computer searches.  Among the more successful results to date are those based
on an algorithm known as Robinson's resolution principle
\cite{Robinson}\index{Robinson's resolution principle}.

Automated theorem provers can be excellent tools for those willing to learn
how to use them.  But they are not widely used in mainstream pure
mathematics, even though they could probably be useful in many areas.  There
are several reasons for this.  Probably most important, the main goal in pure
mathematics is to arrive at results that are considered to be deep or
important; proving them is essential but secondary.  Usually, an automated
theorem prover cannot assist in this main goal, and by the time the main goal
is achieved, the mathematician may have already figured out the proof as a
by-product.  There is also a notational problem.  Mathematicians are used to
using very compact syntax where one or two symbols (heavily dependent on
context) can represent very complex concepts; this is part of the
hierarchy\index{hierarchy} they have built up to tackle difficult problems.  A
theorem prover on the other hand might require that a theorem be expressed in
``first-order logic,''\index{first-order logic} which is the logic on which
most of mathematics is ultimately based but which is not ordinarily used
directly because expressions can become very long.  Some automated theorem
provers are experimental programs, limited in their use to very specialized
areas, and the goal of many is simply research into the nature of automated
theorem proving itself.  Finally, much research remains to be done to enable
them to prove very deep theorems.  One significant result was a
computer proof by Larry Wos\index{Wos, Larry} and colleagues that every Robbins
algebra\index{Robbins algebra} is a Boolean  algebra\index{Boolean algebra}
({\em New York Times}, Dec. 10, 1996).\footnote{In 1933, E.~V.\
Huntington\index{Huntington, E. V.}
presented the following axiom system for
Boolean algebra with a unary operation $n$ and a binary operation $+$:
\begin{center}
    $x + y = y + x$ \\
    $(x + y) + z = x + (y + z)$ \\
    $n(n(x) + y) + n(n(x) + n(y)) = x$
\end{center}
Herbert Robbins\index{Robbins, Herbert}, a student of Huntington, conjectured
that the last equation can be replaced with a simpler one:
\begin{center}
    $n(n(x + y) + n(x + n(y))) = x$
\end{center}
Robbins and Huntington could not find a proof.  The problem was
later studied unsuccessfully by Tarski\index{Tarski, Alfred} and his
students, and it remained an unsolved problem until a
computer found the proof in 1996.  For more information on
the Robbins algebra problem see \cite{Wos}.}

How does Metamath\index{Metamath} relate to automated theorem provers?  A
theorem prover is primarily concerned with one theorem at a time (perhaps
tapping into a small database of known theorems) whereas Metamath is more like
a theorem archiving system, storing both the theorem and its proof in a
database for access and verification.  Metamath is one answer to ``what do you
do with the output of a theorem prover?''  and could be viewed as the
next step in the process.  Automated theorem provers could be useful tools for
helping develop its database.
Note that very long, automatically
generated proofs can make your database fat and ugly and cause Metamath's proof
verification to take a long time to run.  Unless you have a particularly good
program that generates very concise proofs, it might be best to consider the
use of automatically generated proofs as a quick-and-dirty approach, to be
manually rewritten at some later date.

The program {\sc otter}\index{otter@{\sc otter}}\footnote{\url{http://www.cs.unm.edu/\~mccune/otter/}.}, later succeeded by
prover9\index{prover9}\footnote{\url{https://www.cs.unm.edu/~mccune/mace4/}.},
have been historically influential.
The E prover\index{E prover}\footnote{\url{https://github.com/eprover/eprover}.}
is a maintained automated theorem prover
for full first-order logic with equality.
There are many other automated theorem provers as well.

If you want to combine automated theorem provers with Metamath
consider investigating
the book {\em Automated Reasoning:  Introduction and Applications}
\cite{Wos}\index{Wos, Larry}.  This book discusses
how to use {\sc otter} in a way that can
not only able to generate
relatively efficient proofs, it can even be instructed to search for
shorter proofs.  The effective use of {\sc otter} (and similar tools)
does require a certain
amount of experience, skill, and patience.  The axiom system used in the
\texttt{set.mm}\index{set theory database (\texttt{set.mm})} set theory
database can be expressed to {\sc otter} using a method described in
\cite{Megill}.\index{Megill, Norman}\footnote{To use those axioms with
{\sc otter}, they must be restated in a way that eliminates the need for
``dummy variables.''\index{dummy variable!eliminating} See the Comment
on p.~\pageref{nodd}.} When successful, this method tends to generate
short and clever proofs, but my experiments with it indicate that the
method will find proofs within a reasonable time only for relatively
easy theorems.  It is still fun to experiment with.

Reference \cite{Bledsoe}\index{Bledsoe, W. W.} surveys a number of approaches
people have explored in the field of automated theorem proving\index{automated
theorem proving}.

\subsection{Interactive Theorem Provers}\label{interactivetheoremprovers}

Finding proofs completely automatically is difficult, so there
are some interactive theorem provers that allow a human to guide the
computer to find a proof.
Examples include
HOL Light\index{HOL light}%
\footnote{\url{https://www.cl.cam.ac.uk/~jrh13/hol-light/}.},
Isabelle\index{Isabelle}%
\footnote{\url{http://www.cl.cam.ac.uk/Research/HVG/Isabelle}.},
{\sc hol}\index{hol@{\sc hol}}%
\footnote{\url{https://hol-theorem-prover.org/}.},
and
Coq\index{Coq}\footnote{\url{https://coq.inria.fr/}.}.

A major difference between most of these tools and Metamath is that the
``proofs'' are actually programs that guide the program to find a proof,
and not the proof itself.
For example, an Isabelle/HOL proof might apply a step
\texttt{apply (blast dest: rearrange reduction)}. The \texttt{blast}
instruction applies
an automatic tableux prover and returns if it found a sequence of proof
steps that work... but the sequence is not considered part of the proof.

A good overview of
higher-level proof verification languages (such as {\sc lcf}\index{lcf@{\sc
lcf}} and {\sc hol}\index{hol@{\sc hol}})
is given in \cite{Harrison}.  All of these languages are fundamentally
different from Metamath in that much of the mathematical foundational
knowledge is embedded in the underlying proof-verification program, rather
than placed directly in the database that is being verified.
These can have a steep learning curve for those without a mathematical
background.  For example, one usually must have a fair understanding of
mathematical logic in order to follow their proofs.

\subsection{Proof Verifiers}\label{proofverifiers}

A proof verifier is a program that doesn't generate proofs but instead
verifies proofs that you give it.  Many proof verifiers have limited built-in
automated proof capabilities, such as figuring out simple logical inferences
(while still being guided by a person who provides the overall proof).  Metamath
has no built-in automated proof capability other than the limited
capability of its Proof Assistant.

Proof-verification languages are not used as frequently as they might be.
Pure mathematicians are more concerned with producing new results, and such
detail and rigor would interfere with that goal.  The use of computers in pure
mathematics is primarily focused on automated theorem provers (not verifiers),
again with the ultimate goal of aiding the creation of new mathematics.
Automated theorem provers are usually concerned with attacking one theorem at
time rather than making a large, organized database easily available to the
user.  Metamath is one way to help close this gap.

By itself Metamath is a mostly a proof verifier.
This does not mean that other approaches can't be used; the difference
is that in Metamath, the results of various provers must be recorded
step-by-step so that they can be verified.

Another proof-verification language is Mizar,\index{Mizar} which can display
its proofs in the informal language that mathematicians are accustomed to.
Information on the Mizar language is available at \url{http://mizar.org}.

For the working mathematician, Mizar is an excellent tool for rigorously
documenting proofs. Mizar typesets its proofs in the informal English used by
mathematicians (and, while fine for them, are just as inscrutable by
laypersons!). A price paid for Mizar is a relatively steep learning curve of a
couple of weeks.  Several mathematicians are actively formalizing different
areas of mathematics using Mizar and publishing the proofs in a dedicated
journal. Unfortunately the task of formalizing mathematics is still looked
down upon to a certain extent since it doesn't involve the creation of ``new''
mathematics.

The closest system to Metamath is
the {\em Ghilbert}\index{Ghilbert} proof language (\url{http://ghilbert.org})
system developed by
Raph Levien\index{Levien, Raph}.
Ghilbert is a formal proof checker heavily inspired by Metamath.
Ghilbert statements are s-expressions (a la Lisp), which is easy
for computers to parse but many people find them hard to read.
There are a number of differences in their specific constructs, but
there is at least one tool to translate some Metamath materials into Ghilbert.
As of 2019 the Ghilbert community is smaller and less active than the
Metamath community.
That said, the Metamath and Ghilbert communities overlap, and fruitful
conversations between them have occurred many times over the years.

\subsection{Creating a Database of Formalized Mathematics}\label{mathdatabase}

Besides Metamath, there are several other ongoing projects with the goal of
formalizing mathematics into computer-verifiable databases.
Understanding some history will help.

The {\sc qed}\index{qed project@{\sc qed} project}%
\footnote{\url{http://www-unix.mcs.anl.gov/qed}.}
project arose in 1993 and its goals were outlined in the
{\sc qed} manifesto.
The {\sc qed} manifesto was
a proposal for a computer-based database of all mathematical knowledge,
strictly formalized and with all proofs having been checked automatically.
The project had a conference in 1994 and another in 1995;
there was also a ``twenty years of the {\sc qed} manifesto'' workshop
in 2014.
Its ideals are regularly reraised.

In a 2007 paper, Freek Wiedijk identified two reasons
for the failure of the {\sc qed} project as originally envisioned:%
\cite{Wiedijk-revisited}\index{Wiedijk, Freek}

\begin{itemize}
\item Very few people are working on formalization of mathematics. There is no compelling application for fully mechanized mathematics.
\item Formalized mathematics does not yet resemble traditional mathematics. This is partly due to the complexity of mathematical notation, and partly to the limitations of existing theorem provers and proof assistants.
\end{itemize}

But this did not end the dream of
formalizing mathematics into computer-verifiable databases.
The problems that led to the {\sc qed} manifesto are still with us,
even though the challenges were harder than originally considered.
What has happened instead is that various independent projects have
worked towards formalizing mathematics into computer-verifiable databases,
each simultaneously competing and cooperating with each other.

A concrete way to see this is
Freek Wiedijk's ``Formalizing 100 Theorems'' list%
\footnote{\url{http://www.cs.ru.nl/\%7Efreek/100/}.}
which shows the progress different systems have made on a challenge list
of 100 mathematical theorems.%
\footnote{ This is not the only list of ``interesting'' theorems.
Another interesting list was posted by Oliver Knill's list
\cite{Knill}\index{Knill, Oliver}.}
The top systems as of February 2019
(in order of the number of challenges completed) are
HOL Light, Isabelle, Metamath, Coq, and Mizar.

The Metamath 100%
\footnote{\url{http://us.metamath.org/mm\_100.html}}
page (maintained by David A. Wheeler\index{Wheeler, David A.})
shows the progress of Metamath (specifically its \texttt{set.mm} database)
against this challenge list maintained by Freek Wiedijk.
The Metamath \texttt{set.mm} database
has made a lot of progress over the years,
in part because working to prove those challenge theorems required
defining various terms and proving their properties as a prerequisite.
Here are just a few of the many statements that have been
formally proven with Metamath:

% The entries of this cause the narrow display to break poorly,
% since the short amount of text means LaTeX doesn't get a lot to work with
% and the itemize format gives it even *less* margin than usual.
% No one will mind if we make just this list flushleft, since this list
% will be internally consistent.
\begin{flushleft}
\begin{itemize}
\item 1. The Irrationality of the Square Root of 2
  (\texttt{sqr2irr}, by Norman Megill, 2001-08-20)
\item 2. The Fundamental Theorem of Algebra
  (\texttt{fta}, by Mario Carneiro, 2014-09-15)
\item 22. The Non-Denumerability of the Continuum
  (\texttt{ruc}, by Norman Megill, 2004-08-13)
\item 54. The Konigsberg Bridge Problem
  (\texttt{konigsberg}, by Mario Carneiro, 2015-04-16)
\item 83. The Friendship Theorem
  (\texttt{friendship}, by Alexander W. van der Vekens, 2018-10-09)
\end{itemize}
\end{flushleft}

We thank all of those who have developed at least one of the Metamath 100
proofs, and we particularly thank
Mario Carneiro\index{Carneiro, Mario}
who has contributed the most Metamath 100 proofs as of 2019.
The Metamath 100 page shows the list of all people who have contributed a
proof, and links to graphs and charts showing progress over time.
We encourage others to work on proving theorems not yet proven in Metamath,
since doing so improves the work as a whole.

Each of the math formalization systems (including Metamath)
has different strengths and weaknesses, depending on what you value.
Key aspects that differentiate Metamath from the other top systems are:

\begin{itemize}
\item Metamath is not tied to any particular set of axioms.
\item Metamath can show every step of every proof, no exceptions.
  Most other provers only assert that a proof can be found, and do not
  show every step. This also makes verification fast, because
  the system does not need to rediscover proof details.
\item The Metamath verifier has been re-implemented in many different
  programming languages, so verification can be done by multiple
  implementations.  In particular, the
  \texttt{set.mm}\index{set theory database (\texttt{set.mm})}%
  \index{Metamath Proof Explorer} database is verified by
  four different verifiers
  written in four different languages by four different authors.
  This greatly reduces the risk of accepting an invalid
  proof due to an error in the verifier.
\item Proofs stay proven.  In some systems, changes to the system's
  syntax or how a tactic works causes proofs to fail in later versions,
  causing older work to become essentially lost.
  Metamath's language is
  extremely small and fixed, so once a proof is added to a database,
  the database can be rechecked with later versions of the Metamath program
  and with other verifiers of Metamath databases.
  If an axiom or key definition needs to be changed, it is easy to
  manipulate the database as a whole to handle the change
  without touching the underlying verifier.
  Since re-verification of an entire database takes seconds, there
  is never a reason to delay complete verification.
  This aspect is especially compelling if your
  goal is to have a long-term database of proofs.
\item Licensing is generous.  The main Metamath databases are released to
  the public domain, and the main Metamath program is open source software
  under a standard, widely-used license.
\item Substitutions are easy to understand, even by those who are not
  professional mathematicians.
\end{itemize}

Of course, other systems may have advantages over Metamath
that are more compelling, depending on what you value.
In any case, we hope this helps you understand Metamath
within a wider context.

\subsection{In Summary}\label{computers-summary}

To summarize our discussions of computers and mathematics, computer algebra
systems can be viewed as theorem generators focusing on a narrow realm of
mathematics (numbers and their properties), automated theorem provers as proof
generators for specific theorems in a much broader realm covered by a built-in
formal system such as first-order logic, interactive theorem
provers require human guidance, proof verifiers verify proofs but
historically they have been
restricted to first-order logic.
Metamath, in contrast,
is a proof verifier and documenter whose realm is essentially unlimited.

\section{Mathematics and Metamath}

\subsection{Standard Mathematics}

There are a number of ways that Metamath\index{Metamath} can be used with
standard mathematics.  The most satisfying way philosophically is to start at
the very beginning, and develop the desired mathematics from the axioms of
logic and set theory.\index{set theory}  This is the approach taken in the
\texttt{set.mm}\index{set theory database (\texttt{set.mm})}%
\index{Metamath Proof Explorer}
database (also known as the Metamath Proof Explorer).
Among other things, this database builds up to the
axioms of real and complex numbers\index{analysis}\index{real and complex
numbers} (see Section~\ref{real}), and a standard development of analysis, for
example, could start at that point, using it as a basis.   Besides this
philosophical advantage, there are practical advantages to having all of the
tools of set theory available in the supporting infrastructure.

On the other hand, you may wish to start with the standard axioms of a
mathematical theory without going through the set theoretical proofs of those
axioms.  You will need mathematical logic to make inferences, but if you wish
you can simply introduce theorems\index{theorem} of logic as
``axioms''\index{axiom} wherever you need them, with the implicit assumption
that in principle they can be proved, if they are obvious to you.  If you
choose this approach, you will probably want to review the notation used in
\texttt{set.mm}\index{set theory database (\texttt{set.mm})} so that your own
notation will be consistent with it.

\subsection{Other Formal Systems}
\index{formal system}

Unlike some programs, Metamath\index{Metamath} is not limited to any specific
area of mathematics, nor committed to any particular mathematical philosophy
such as classical logic versus intuitionism, nor limited, say, to expressions
in first-order logic.  Although the database \texttt{set.mm}
describes standard logic and set theory, Meta\-math
is actually a general-purpose language for describing a wide variety of formal
systems.\index{formal system}  Non-standard systems such as modal
logic,\index{modal logic} intuitionist logic\index{intuitionism}, higher-order
logic\index{higher-order logic}, quantum logic\index{quantum logic}, and
category theory\index{category theory} can all be described with the Metamath
language.  You define the symbols you prefer and tell Metamath the axioms and
rules you want to start from, and Metamath will verify any inferences you make
from those axioms and rules.  A simple example of a non-standard formal system
is Hofstadter's\index{Hofstadter, Douglas R.} MIU system,\index{MIU-system}
whose Metamath description is presented in Appendix~\ref{MIU}.

This is not hypothetical.
The largest Metamath database is
\texttt{set.mm}\index{set theory database (\texttt{set.mm}}%
\index{Metamath Proof Explorer}), aka the Metamath Proof Explorer,
which uses the most common axioms for mathematical foundations
(specifically classical logic combined with Zermelo--Fraenkel
set theory\index{Zermelo--Fraenkel set theory} with the Axiom of Choice).
But other Metamath databases are available:

\begin{itemize}
\item The database
  \texttt{iset.mm}\index{intuitionistic logic database (\texttt{iset.mm})},
  aka the
  Intuitionistic Logic Explorer\index{Intuitionistic Logic Explorer},
  uses intuitionistic logic (a constructivist point of view)
  instead of classical logic.
\item The database
  \texttt{nf.mm}\index{New Foundations database (\texttt{nf.mm})},
  aka the
  New Foundations Explorer\index{New Foundations Explorer},
  constructs mathematics from scratch,
  starting from Quine's New Foundations (NF) set theory axioms.
\item The database
  \texttt{hol.mm}\index{Higher-order Logic database (\texttt{hol.mm})},
  aka the
  Higher-Order Logic (HOL) Explorer\index{Higher-Order Logic (HOL) Explorer},
  starts with HOL (also called simple type theory) and derives
  equivalents to ZFC axioms, connecting the two approaches.
\end{itemize}

Since the days of David Hilbert,\index{Hilbert, David} mathematicians have
been concerned with the fact that the metalanguage\index{metalanguage} used to
describe mathematics may be stronger than the mathematics being described.
Metamath\index{Metamath}'s underlying finitary\index{finitary proof},
constructive nature provides a good philosophical basis for studying even the
weakest logics.\index{weak logic}

The usual treatment of many non-standard formal systems\index{formal
system} uses model theory\index{model theory} or proof theory\index{proof
theory} to describe these systems; these theories, in turn, are based on
standard set theory.  In other words, a non-standard formal system is defined
as a set with certain properties, and standard set theory is used to derive
additional properties of this set.  The standard set theory database provided
with Metamath can be used for this purpose, and when used this way
the development of a special
axiom system for the non-standard formal system becomes unnecessary.  The
model- or proof-theoretic approach often allows you to prove much deeper
results with less effort.

Metamath supports both approaches.  You can define the non-standard
formal system directly, or define the non-standard formal system as
a set with certain properties, whichever you find most helpful.

%\section{Additional Remarks}

\subsection{Metamath and Its Philosophy}

Closely related to Metamath\index{Metamath} is a philosophy or way of looking
at mathematics. This philosophy is related to the formalist
philosophy\index{formalism} of Hilbert\index{Hilbert, David} and his followers
\cite[pp.~1203--1208]{Kline}\index{Kline, Morris}
\cite[p.~6]{Behnke}\index{Behnke, H.}. In this philosophy, mathematics is
viewed as nothing more than a set of rules that manipulate symbols, together
with the consequences of those rules.  While the mathematics being described
may be complex, the rules used to describe it (the
``metamathematics''\index{metamathematics}) should be as simple as possible.
In particular, proofs should be restricted to dealing with concrete objects
(the symbols we write on paper rather than the abstract concepts they
represent) in a constructive manner; these are called ``finitary''
proofs\index{finitary proof} \cite[pp.~2--3]{Shoenfield}\index{Shoenfield,
Joseph R.}.

Whether or not you find Metamath interesting or useful will in part depend on
the appeal you find in its philosophy, and this appeal will probably depend on
your particular goals with respect to mathematics.  For example, if you are a
pure mathematician at the forefront of discovering new mathematical knowledge,
you will probably find that the rigid formality of Metamath stifles your
creativity.  On the other hand, we would argue that once this knowledge is
discovered, there are advantages to documenting it in a standard format that
will make it accessible to others.  Sixty years from now, your field may be
dormant, and as Davis and Hersh put it, your ``writings would become less
translatable than those of the Maya'' \cite[p.~37]{Davis}\index{Davis, Phillip
J.}.


\subsection{A History of the Approach Behind Metamath}

Probably the one work that has had the most motivating influence on
Metamath\index{Metamath} is Whitehead and Russell's monumental {\em Principia
Mathematica} \cite{PM}\index{Whitehead, Alfred North}\index{Russell,
Bertrand}\index{principia mathematica@{\em Principia Mathematica}}, whose aim
was to deduce all of mathematics from a small number of primitive ideas, in a
very explicit way that in principle anyone could understand and follow.  While
this work was tremendously influential in its time, from a modern perspective
it suffers from several drawbacks.  Both its notation and its underlying
axioms are now considered dated and are no longer used.  From our point of
view, its development is not really as accessible as we would like to see; for
practical reasons, proofs become more and more sketchy as its mathematics
progresses, and working them out in fine detail requires a degree of
mathematical skill and patience that many people don't have.  There are
numerous small errors, which is understandable given the tedious, technical
nature of the proofs and the lack of a computer to verify the details.
However, even today {\em Principia Mathematica} stands out as the work closest
in spirit to Metamath.  It remains a mind-boggling work, and one can't help
but be amazed at seeing ``$1+1=2$'' finally appear on page 83 of Volume II
(Theorem *110.643).

The origin of the proof notation used by Metamath dates back to the 1950's,
when the logician C.~A.~Meredith expressed his proofs in a compact notation
called ``condensed detachment''\index{condensed detachment}
\cite{Hindley}\index{Hindley, J. Roger} \cite{Kalman}\index{Kalman, J. A.}
\cite{Meredith}\index{Meredith, C. A.} \cite{Peterson}\index{Peterson, Jeremy
George}.  This notation allows proofs to be communicated unambiguously by
merely referencing the axiom\index{axiom}, rule\index{rule}, or
theorem\index{theorem} used at each step, without explicitly indicating the
substitutions\index{substitution!variable}\index{variable substitution} that
have to be made to the variables in that axiom, rule, or theorem.  Ordinarily,
condensed detachment is more or less limited to propositional
calculus\index{propositional calculus}.  The concept has been extended to
first-order logic\index{first-order logic} in \cite{Megill}\index{Megill,
Norman}, making it is easy to write a small computer program to verify proofs
of simple first-order logic theorems.\index{condensed detachment!and
first-order logic}

A key concept behind the notation of condensed detachment is called
``unification,''\index{unification} which is an algorithm for determining what
substitutions\index{substitution!variable}\index{variable substitution} to
variables have to be made to make two expressions match each other.
Unification was first precisely defined by the logician J.~A.~Robinson, who
used it in the development of a powerful
theorem-proving technique called the ``resolution principle''
\cite{Robinson}\index{Robinson's resolution principle}. Metamath does not make
use of the resolution principle, which is intended for systems of first-order
logic.\index{first-order logic}  Metamath's use is not restricted to
first-order logic, and as we have mentioned it does not automatically discover
proofs.  However, unification is a key idea behind Metamath's proof
notation, and Metamath makes use of a very simple version of it
(Section~\ref{unify}).

\subsection{Metamath and First-Order Logic}

First-order logic\index{first-order logic} is the supporting structure
for standard mathematics.  On top of it is set theory, which contains
the axioms from which virtually all of mathematics can be derived---a
remarkable fact.\index{category
theory}\index{cardinal, inaccessible}\label{categoryth}\footnote{An exception seems
to be category theory.  There are several schools of thought on whether
category theory is derivable from set theory.  At a minimum, it appears
that an additional axiom is needed that asserts the existence of an
``inaccessible cardinal'' (a type of infinity so large that standard set
theory can't prove or deny that it exists).
%
%%%% (I took this out that was in previous editions:)
% But it is also argued that not just one but a ``proper class'' of them
% is needed, and the existence of proper classes is impossible in standard
% set theory.  (A proper class is a collection of sets so huge that no set
% can contain it as an element.  Proper classes can lead to
% inconsistencies such as ``Russell's paradox.''  The axioms of standard
% set theory are devised so as to deny the existence of proper classes.)
%
For more information, see
\cite[pp.~328--331]{Herrlich}\index{Herrlich, Horst} and
\cite{Blass}\index{Blass, Andrea}.}

One of the things that makes Metamath\index{Metamath} more practical for
first-order theories is a set of axioms for first-order logic designed
specifically with Metamath's approach in mind.  These are included in
the database \texttt{set.mm}\index{set theory database (\texttt{set.mm})}.
See Chapter~\ref{fol} for a detailed
description; the axioms are shown in Section~\ref{metaaxioms}.  While
logically equivalent to standard axiom systems, our axiom system breaks
up the standard axioms into smaller pieces such that from them, you can
directly derive what in other systems can only be derived as higher-level
``metatheorems.''\index{metatheorem}  In other words, it is more powerful than
the standard axioms from a metalogical point of view.  A rigorous
justification for this system and its ``metalogical
completeness''\index{metalogical completeness} is found in
\cite{Megill}\index{Megill, Norman}.  The system is closely related to a
system developed by Monk\index{Monk, J. Donald} and Tarski\index{Tarski,
Alfred} in 1965 \cite{Monks}.

For example, the formula $\exists x \, x = y $ (given $y$, there exists some
$x$ equal to it) is a theorem of logic,\footnote{Specifically, it is a theorem
of those systems of logic that assume non-empty domains.  It is not a theorem
of more general systems that include the empty domain\index{empty domain}, in
which nothing exists, period!  Such systems are called ``free
logics.''\index{free logic} For a discussion of these systems, see
\cite{Leblanc}\index{Leblanc, Hugues}.  Since our use for logic is as a basis
for set theory, which has a non-empty domain, it is more convenient (and more
traditional) to use a less general system.  An interesting curiosity is that,
using a free logic as a basis for Zermelo--Fraenkel set
theory\index{Zermelo--Fraenkel set theory} (with the redundant Axiom of the
Null Set omitted),\index{Axiom of the Null Set} we cannot even prove the
existence of a single set without assuming the axiom of infinity!\index{Axiom
of Infinity}} whether or not $x$ and $y$ are distinct variables\index{distinct
variables}.  In many systems of logic, we would have to prove two theorems to
arrive at this result.  First we would prove ``$\exists x \, x = x $,'' then
we would separately prove ``$\exists x \, x = y $, where $x$ and $y$ are
distinct variables.''  We would then combine these two special cases ``outside
of the system'' (i.e.\ in our heads) to be able to claim, ``$\exists x \, x =
y $, regardless of whether $x$ and $y$ are distinct.''  In other words, the
combination of the two special cases is a
metatheorem.  In the system of logic
used in Metamath's set theory\index{set theory database (\texttt{set.mm})}
database, the axioms of logic are broken down into small pieces that allow
them to be reassembled in such a way that theorems such as these can be proved
directly.

Breaking down the axioms in this way makes them look peculiar and not very
intuitive at first, but rest assured that they are correct and complete.  Their
correctness is ensured because they are theorem schemes of standard first-order
logic (which you can easily verify if you are a logician).  Their completeness
follows from the fact that we explicitly derive the standard axioms of
first-order logic as theorems.  Deriving the standard axioms is somewhat
tricky, but once we're there, we have at our disposal a system that is less
awkward to work with in formal proofs\index{formal proof}.  In technical terms
that logicians understand, we eliminate the cumbersome concepts of ``free
variable,''\index{free variable} ``bound variable,''\index{bound variable} and
``proper substitution''\index{proper substitution}\index{substitution!proper}
as primitive notions.  These concepts are present in our system but are
defined in terms of concepts expressed by the axioms and can be eliminated in
principle.  In standard systems, these concepts are really like additional,
implicit axioms\index{implicit axiom} that are somewhat complex and cannot be
eliminated.

The traditional approach to logic, wherein free variables and proper
substitution is defined, is also possible to do directly in the Metamath
language.  However, the notation tends to become awkward, and there are
disadvantages:  for example, extending the definition of a wff with a
definition is awkward, because the free variable and proper substitution
concepts have to have their definitions also extended.  Our choice of
axioms for \texttt{set.mm} is to a certain extent a matter of style, in
an attempt to achieve overall simplicity, but you should also be aware
that the traditional approach is possible as well if you should choose
it.

\chapter{Using the Metamath Program}
\label{using}

\section{Installation}

The way that you install Metamath\index{Metamath!installation} on your
computer system will vary for different computers.  Current
instructions are provided with the Metamath program download at
\url{http://metamath.org}.  In general, the installation is simple.
There is one file containing the Metamath program itself.  This file is
usually called \texttt{metamath} or \texttt{metamath.}{\em xxx} where
{\em xxx} is the convention (such as \texttt{exe}) for an executable
program on your operating system.  There are several additional files
containing samples of the Metamath language, all ending with
\texttt{.mm}.  The file \texttt{set.mm}\index{set theory database
(\texttt{set.mm})} contains logic and set theory and can be used as a
starting point for other areas of mathematics.

You will also need a text editor\index{text editor} capable of editing plain
{\sc ascii}\footnote{American Standard Code for Information Interchange.} text
in order to prepare your input files.\index{ascii@{\sc ascii}}  Most computers
have this capability built in.  Note that plain text is not necessarily the
default for some word processing programs\index{word processor}, especially if
they can handle different fonts; for example, with Microsoft Word\index{Word
(Microsoft)}, you must save the file in the format ``Text Only With Line
Breaks'' to get a plain text\index{plain text} file.\footnote{It is
recommended that all lines in a Metamath source file be 79 characters or less
in length for compatibility among different computer terminals.  When creating
a source file on an editor such as Word, select a monospaced
font\index{monospaced font} such as Courier\index{Courier font} or
Monaco\index{Monaco font} to make this easier to achieve.  Better yet,
just use a plain text editor such as Notepad.}

On some computer systems, Metamath does not have the capability to print
its output directly; instead, you send its output to a file (using the
\texttt{open} commands described later).  The way you print this output
file depends on your computer.\index{printers} Some computers have a
print command, whereas with others, you may have to read the file into
an editor and print it from there.

If you want to print your Metamath source files with typeset formulas
containing standard mathematical symbols, you will need the \LaTeX\
typesetting program\index{latex@{\LaTeX}}, which is widely and freely
available for most operating systems.  It runs natively on Unix and
Linux, and can be installed on Windows as part of the free Cygwin
package (\url{http://cygwin.com}).

You can also produce {\sc html}\footnote{HyperText Markup Language.}
web pages.  The {\tt help html} command in the Metamath program will
assist you with this feature.

\section{Your First Formal System}\label{start}
\subsection{From Nothing to Zero}\label{startf}

To give you a feel for what the Metamath\index{Metamath} language looks like,
we will take a look at a very simple example from formal number
theory\index{number theory}.  This example is taken from
Mendelson\index{Mendelson, Elliot} \cite[p. 123]{Mendelson}.\footnote{To keep
the example simple, we have changed the formalism slightly, and what we call
axioms\index{axiom} are strictly speaking theorems\index{theorem} in
\cite{Mendelson}.}  We will look at a small subset of this theory, namely that
part needed for the first number theory theorem proved in \cite{Mendelson}.

First we will look at a standard formal proof\index{formal proof} for the
example we have picked, then we will look at the Metamath version.  If you
have never been exposed to formal proofs, the notation may seem to be such
overkill to express such simple notions that you may wonder if you are missing
something.  You aren't.  The concepts involved are in fact very simple, and a
detailed breakdown in this fashion is necessary to express the proof in a way
that can be verified mechanically.  And as you will see, Metamath breaks the
proof down into even finer pieces so that the mechanical verification process
can be about as simple as possible.

Before we can introduce the axioms\index{axiom} of the theory, we must define
the syntax rules for forming legal expressions\index{syntax rules}
(combinations of symbols) with which those axioms can be used. The number 0 is
a {\bf term}\index{term}; and if $ t$ and $r$ are terms, so is $(t+r)$. Here,
$ t$ and $r$ are ``metavariables''\index{metavariable} ranging over terms; they
themselves do not appear as symbols in an actual term.  Some examples of
actual terms are $(0 + 0)$ and $((0+0)+0)$.  (Note that our theory describes
only the number zero and sums of zeroes.  Of course, not much can be done with
such a trivial theory, but remember that we have picked a very small subset of
complete number theory for our example.  The important thing for you to focus
on is our definitions that describe how symbols are combined to form valid
expressions, and not on the content or meaning of those expressions.) If $ t$
and $r$ are terms, an expression of the form $ t=r$ is a {\bf wff}
(well-formed formula)\index{well-formed formula (wff)}; and if $P$ and $Q$ are
wffs, so is $(P\rightarrow Q)$ (which means ``$P$ implies
$Q$''\index{implication ($\rightarrow$)} or ``if $P$ then $Q$'').
Here $P$ and $Q$ are metavariables ranging over wffs.  Examples of actual
wffs are $0=0$, $(0+0)=0$, $(0=0 \rightarrow (0+0)=0)$, and $(0=0\rightarrow
(0=0\rightarrow 0=(0+0)))$.  (Our notation makes use of more parentheses than
are customary, but the elimination of ambiguity this way simplifies our
example by avoiding the need to define operator precedence\index{operator
precedence}.)

The {\bf axioms}\index{axiom} of our theory are all wffs of the following
form, where $ t$, $r$, and $s$ are any terms:

%Latex p. 92
\renewcommand{\theequation}{A\arabic{equation}}

\begin{equation}
(t=r\rightarrow (t=s\rightarrow r=s))
\end{equation}
\begin{equation}
(t+0)=t
\end{equation}

Note that there are an infinite number of axioms since there are an infinite
number of possible terms.  A1 and A2 are properly called ``axiom
schemes,''\index{axiom scheme} but we will refer to them as ``axioms'' for
brevity.

An axiom is a {\bf theorem}; and if $P$ and $(P\rightarrow Q)$ are theorems
(where $P$ and $Q$ are wffs), then $Q$ is also a theorem.\index{theorem}  The
second part of this definition is called the modus ponens (MP) rule of
inference\index{inference rule}\index{modus ponens}.  It allows us to obtain
new theorems from old ones.

The {\bf proof}\index{proof} of a theorem is a sequence of one or more
theorems, each of which is either an axiom or the result of modus ponens
applied to two previous theorems in the sequence, and the last of which is the
theorem being proved.

The theorem we will prove for our example is very simple:  $ t=t$.  The proof of
our theorem follows.  Study it carefully until you feel sure you
understand it.\label{zeroproof}

% Use tabu so that lines will wrap automatically as needed.
\begin{tabu} { l X X }
1. & $(t+0)=t$ & (by axiom A2) \\
2. & $(t+0)=t$ & (by axiom A2) \\
3. & $((t+0)=t \rightarrow ((t+0)=t\rightarrow t=t))$ & (by axiom A1) \\
4. & $((t+0)=t\rightarrow t=t)$ & (by MP applied to steps 2 and 3) \\
5. & $t=t$ & (by MP applied to steps 1 and 4) \\
\end{tabu}

(You may wonder why step 1 is repeated twice.  This is not necessary in the
formal language we have defined, but in Metamath's ``reverse Polish
notation''\index{reverse Polish notation (RPN)} for proofs, a previous step
can be referred to only once.  The repetition of step~1 here will enable you
to see more clearly the correspondence of this proof with the
Metamath\index{Metamath} version on p.~\pageref{demoproof}.)

Our theorem is more properly called a ``theorem scheme,''\index{theorem
scheme} for it represents an infinite number of theorems, one for each
possible term $ t$.  Two examples of actual theorems would be $0=0$ and
$(0+0)=(0+0)$.  Rarely do we prove actual theorems, since by proving schemes
we can prove an infinite number of theorems in one fell swoop.  Similarly, our
proof should really be called a ``proof scheme.''\index{proof scheme}  To
obtain an actual proof, pick an actual term to use in place of $ t$, and
substitute it for $ t$ throughout the proof.

Let's discuss what we have done here.  The axioms\index{axiom} of our theory,
A1 and A2, are trivial and obvious.  Everyone knows that adding zero to
something doesn't change it, and also that if two things are equal to a third,
then they are equal to each other. In fact, stating the trivial and obvious is
a goal to strive for in any axiomatic system.  From trivial and obvious truths
that everyone agrees upon, we can prove results that are not so obvious yet
have absolute faith in them.  If we trust the axioms and the rules, we must,
by definition, trust the consequences of those axioms and rules, if logic is
to mean anything at all.

Our rule of inference\index{rule}, modus ponens\index{modus ponens}, is also
pretty obvious once you understand what it means.  If we prove a fact $P$, and
we also prove that $P$ implies $Q$, then $Q$ necessarily follows as a new
fact.  The rule provides us with a means for obtaining new facts (i.e.\
theorems\index{theorem}) from old ones.

The theorem that we have proved, $ t=t$, is so fundamental that you may wonder
why it isn't one of the axioms\index{axiom}.  In some axiom systems of
arithmetic, it {\em is} an axiom.  The choice of axioms in a theory is to some
extent arbitrary and even an art form, constrained only by the requirement
that any two equivalent axiom systems be able to derive each other as
theorems.  We could imagine that the inventor of our axiom system originally
included $ t=t$ as an axiom, then discovered that it could be derived as a
theorem from the other axioms.  Because of this, it was not necessary to
keep it as an axiom.  By eliminating it, the final set of axioms became
that much simpler.

Unless you have worked with formal proofs\index{formal proof} before, it
probably wasn't apparent to you that $ t=t$ could be derived from our two
axioms until you saw the proof. While you certainly believe that $ t=t$ is
true, you might not be able to convince an imaginary skeptic who believes only
in our two axioms until you produce the proof.  Formal proofs such as this are
hard to come up with when you first start working with them, but after you get
used to them they can become interesting and fun.  Once you understand the
idea behind formal proofs you will have grasped the fundamental principle that
underlies all of mathematics.  As the mathematics becomes more sophisticated,
its proofs become more challenging, but ultimately they all can be broken down
into individual steps as simple as the ones in our proof above.

Mendelson's\index{Mendelson, Elliot} book, from which our example was taken,
contains a number of detailed formal proofs such as these, and you may be
interested in looking it up.  The book is intended for mathematicians,
however, and most of it is rather advanced.  Popular literature describing
formal proofs\index{formal proof} include \cite[p.~296]{Rucker}\index{Rucker,
Rudy} and \cite[pp.~204--230]{Hofstadter}\index{Hofstadter, Douglas R.}.

\subsection{Converting It to Metamath}\label{convert}

Formal proofs\index{formal proof} such as the one in our example break down
logical reasoning into small, precise steps that leave little doubt that the
results follow from the axioms\index{axiom}.  You might think that this would
be the finest breakdown we can achieve in mathematics.  However, there is more
to the proof than meets the eye. Although our axioms were rather simple, a lot
of verbiage was needed before we could even state them:  we needed to define
``term,'' ``wff,'' and so on.  In addition, there are a number of implied
rules that we haven't even mentioned. For example, how do we know that step 3
of our proof follows from axiom A1? There is some hidden reasoning involved in
determining this.  Axiom A1 has two occurrences of the letter $ t$.  One of
the implied rules states that whatever we substitute for $ t$ must be a legal
term\index{term}.\footnote{Some authors make this implied rule explicit by
stating, ``only expressions of the above form are terms,'' after defining
``term.''}  The expression $ t+0$ is pretty obviously a legal term whenever $
t$ is, but suppose we wanted to substitute a huge term with thousands of
symbols?  Certainly a lot of work would be involved in determining that it
really is a term, but in ordinary formal proofs all of this work would be
considered a single ``step.''

To express our axiom system in the Metamath\index{Metamath} language, we must
describe this auxiliary information in addition to the axioms themselves.
Metamath does not know what a ``term'' or a ``wff''\index{well-formed formula
(wff)} is.  In Metamath, the specification of the ways in which we can combine
symbols to obtain terms and wffs are like little axioms in themselves.  These
auxiliary axioms are expressed in the same notation as the ``real''
axioms\index{axiom}, and Metamath does not distinguish between the two.  The
distinction is made by you, i.e.\ by the way in which you interpret the
notation you have chosen to express these two kinds of axioms.

The Metamath language breaks down mathematical proofs into tiny pieces, much
more so than in ordinary formal proofs\index{formal proof}.  If a single
step\index{proof step} involves the
substitution\index{substitution!variable}\index{variable substitution} of a
complex term for one of its variables, Metamath must see this single step
broken down into many small steps.  This fine-grained breakdown is what gives
Metamath generality and flexibility as it lets it not be limited to any
particular mathematical notation.

Metamath's proof notation is not, in itself, intended to be read by humans but
rather is in a compact format intended for a machine.  The Metamath program
will convert this notation to a form you can understand, using the \texttt{show
proof}\index{\texttt{show proof} command} command.  You can tell the program what
level of detail of the proof you want to look at.  You may want to look at
just the logical inference steps that correspond
to ordinary formal proof steps,
or you may want to see the fine-grained steps that prove that an expression is
a term.

Here, without further ado, is our example converted to the
Metamath\index{Metamath} language:\index{metavariable}\label{demo0}

\begin{verbatim}
$( Declare the constant symbols we will use $)
    $c 0 + = -> ( ) term wff |- $.
$( Declare the metavariables we will use $)
    $v t r s P Q $.
$( Specify properties of the metavariables $)
    tt $f term t $.
    tr $f term r $.
    ts $f term s $.
    wp $f wff P $.
    wq $f wff Q $.
$( Define "term" and "wff" $)
    tze $a term 0 $.
    tpl $a term ( t + r ) $.
    weq $a wff t = r $.
    wim $a wff ( P -> Q ) $.
$( State the axioms $)
    a1 $a |- ( t = r -> ( t = s -> r = s ) ) $.
    a2 $a |- ( t + 0 ) = t $.
$( Define the modus ponens inference rule $)
    ${
       min $e |- P $.
       maj $e |- ( P -> Q ) $.
       mp  $a |- Q $.
    $}
$( Prove a theorem $)
    th1 $p |- t = t $=
  $( Here is its proof: $)
       tt tze tpl tt weq tt tt weq tt a2 tt tze tpl
       tt weq tt tze tpl tt weq tt tt weq wim tt a2
       tt tze tpl tt tt a1 mp mp
     $.
\end{verbatim}\index{metavariable}

A ``database''\index{database} is a set of one or more {\sc ascii} source
files.  Here's a brief description of this Metamath\index{Metamath} database
(which consists of this single source file), so that you can understand in
general terms what is going on.  To understand the source file in detail, you
should read Chapter~\ref{languagespec}.

The database is a sequence of ``tokens,''\index{token} which are normally
separated by spaces or line breaks.  The only tokens that are built into
the Metamath language are those beginning with \texttt{\$}.  These tokens
are called ``keywords.''\index{keyword}  All other tokens are
user-defined, and their names are arbitrary.

As you might have guessed, the Metamath token \texttt{\$(}\index{\texttt{\$(} and
\texttt{\$)} auxiliary keywords} starts a comment and \texttt{\$)} ends a comment.

The Metamath tokens \texttt{\$c}\index{\texttt{\$c} statement},
\texttt{\$v}\index{\texttt{\$v} statement},
\texttt{\$e}\index{\texttt{\$e} statement},
\texttt{\$f}\index{\texttt{\$f} statement},
\texttt{\$a}\index{\texttt{\$a} statement}, and
\texttt{\$p}\index{\texttt{\$p} statement} specify ``statements'' that
end with \texttt{\$.}\,.\index{\texttt{\$.}\ keyword}

The Metamath tokens \texttt{\$c} and \texttt{\$v} each declare\index{constant
declaration}\index{variable declaration} a list of user-defined tokens, called
``math symbols,''\index{math symbol} that the database will reference later
on.  All of the math symbols they define you have seen earlier except the
turnstile symbol \texttt{|-} ($\vdash$)\index{turnstile ({$\,\vdash$})}, which is
commonly used by logicians to mean ``a proof exists for.''  For us
the turnstile is just a
convenient symbol that distinguishes expressions that are axioms\index{axiom}
or theorems\index{theorem} from expressions that are terms or wffs.

The \texttt{\$c} statement declares ``constants''\index{constant} and
the \texttt{\$v} statement declares
``variables''\index{variable}\index{constant declaration}\index{variable
declaration} (or more precisely, metavariables\index{metavariable}).  A
variable may be substituted\index{substitution!variable}\index{variable
substitution} with sequences of math symbols whereas a constant may not
be substituted with anything.

It may seem redundant to require both \texttt{\$c}\index{\texttt{\$c} statement} and
\texttt{\$v}\index{\texttt{\$v} statement} statements (since any math
symbol\index{math symbol} not specified with a \texttt{\$c} statement could be
presumed to be a variable), but this provides for better error checking and
also allows math symbols to be redeclared\index{redeclaration of symbols}
(Section~\ref{scoping}).

The token \texttt{\$f}\index{\texttt{\$f} statement} specifies a
statement called a ``variable-type hypothesis'' (also called a
``floating hypothesis'') and \texttt{\$e}\index{\texttt{\$e} statement}
specifies a ``logical hypothesis'' (also called an ``essential
hypothesis'').\index{hypothesis}\index{variable-type
hypothesis}\index{logical hypothesis}\index{floating
hypothesis}\index{essential hypothesis} The token
\texttt{\$a}\index{\texttt{\$a} statement} specifies an ``axiomatic
assertion,''\index{axiomatic assertion} and
\texttt{\$p}\index{\texttt{\$p} statement} specifies a ``provable
assertion.''\index{provable assertion} To the left of each occurrence of
these four tokens is a ``label''\index{label} that identifies the
hypothesis or assertion for later reference.  For example, the label of
the first axiomatic assertion is \texttt{tze}.  A \texttt{\$f} statement
must contain exactly two math symbols, a constant followed by a
variable.  The \texttt{\$e}, \texttt{\$a}, and \texttt{\$p} statements
each start with a constant followed by, in general, an arbitrary
sequence of math symbols.

Associated with each assertion\index{assertion} is a set of hypotheses
that must be satisfied in order for the assertion to be used in a proof.
These are called the ``mandatory hypotheses''\index{mandatory
hypothesis} of the assertion.  Among those hypotheses whose ``scope''
(described below) includes the assertion, \texttt{\$e} hypotheses are
always mandatory and \texttt{\$f}\index{\texttt{\$f} statement}
hypotheses are mandatory when they share their variable with the
assertion or its \texttt{\$e} hypotheses.  The exact rules for
determining which hypotheses are mandatory are described in detail in
Sections~\ref{frames} and \ref{scoping}.  For example, the mandatory
hypotheses of assertion \texttt{tpl} are \texttt{tt} and \texttt{tr},
whereas assertion \texttt{tze} has no mandatory hypotheses because it
contains no variables and has no \texttt{\$e}\index{\texttt{\$e}
statement} hypothesis.  Metamath's \texttt{show statement}
command\index{\texttt{show statement} command}, described in the next
section, will show you a statement's mandatory hypotheses.

Sometimes we need to make a hypothesis relevant to only certain
assertions.  The set of statements to which a hypothesis is relevant is
called its ``scope.''  The Metamath brackets,
\texttt{\$\char`\{}\index{\texttt{\$\char`\{} and \texttt{\$\char`\}}
keywords} and \texttt{\$\char`\}}, define a ``block''\index{block} that
delimits the scope of any hypothesis contained between them.  The
assertion \texttt{mp} has mandatory hypotheses \texttt{wp}, \texttt{wq},
\texttt{min}, and \texttt{maj}.  The only mandatory hypothesis of
\texttt{th1}, on the other hand, is \texttt{tt}, since \texttt{th1}
occurs outside of the block containing \texttt{min} and \texttt{maj}.

Note that \texttt{\$\char`\{} and \texttt{\$\char`\}} do not affect the
scope of assertions (\texttt{\$a} and \texttt{\$p}).  Assertions are always
available to be referenced by any later proof in the source file.

Each provable assertion (\texttt{\$p}\index{\texttt{\$p} statement}
statement) has two parts.  The first part is the
assertion\index{assertion} itself, which is a sequence of math
symbol\index{math symbol} tokens placed between the \texttt{\$p} token
and a \texttt{\$=}\index{\texttt{\$=} keyword} token.  The second part
is a ``proof,'' which is a list of label tokens placed between the
\texttt{\$=} token and the \texttt{\$.}\index{\texttt{\$.}\ keyword}\
token that ends the statement.\footnote{If you've looked at the
\texttt{set.mm} database, you may have noticed another notation used for
proofs.  The other notation is called ``compressed.''\index{compressed
proof}\index{proof!compressed} It reduces the amount of space needed to
store a proof in the database and is described in
Appendix~\ref{compressed}.  In the example above, we use
``normal''\index{normal proof}\index{proof!normal} notation.} The proof
acts as a series of instructions to the Metamath program, telling it how
to build up the sequence of math symbols contained in the assertion part of
the \texttt{\$p} statement, making use of the hypotheses of the
\texttt{\$p} statement and previous assertions.  The construction takes
place according to precise rules.  If the list of labels in the proof
causes these rules to be violated, or if the final sequence that results
does not match the assertion, the Metamath program will notify you with
an error message.

If you are familiar with reverse Polish notation (RPN), which is sometimes used
on pocket calculators, here in a nutshell is how a proof works.  Each
hypothesis label\index{hypothesis label} in the proof is pushed\index{push}
onto the RPN stack\index{stack}\index{RPN stack} as it is encountered. Each
assertion label\index{assertion label} pops\index{pop} off the stack as many
entries as the referenced assertion has mandatory hypotheses.  Variable
substitutions\index{substitution!variable}\index{variable substitution} are
computed which, when made to the referenced assertion's mandatory hypotheses,
cause these hypotheses to match the stack entries. These same substitutions
are then made to the variables in the referenced assertion itself, which is
then pushed onto the stack.  At the end of the proof, there should be one
stack entry, namely the assertion being proved.  This process is explained in
detail in Section~\ref{proof}.

Metamath's proof notation is not very readable for humans, but it allows the
proof to be stored compactly in a file.  The Metamath\index{Metamath} program
has proof display features that let you see what's going on in a more
readable way, as you will see in the next section.

The rules used in verifying a proof are not based on any built-in syntax of the
symbol sequence in an assertion\index{assertion} nor on any built-in meanings
attached to specific symbol names.  They are based strictly on symbol
matching:  constants\index{constant} must match themselves, and
variables\index{variable} may be replaced with anything that allows a match to
occur.  For example, instead of \texttt{term}, \texttt{0}, and \verb$|-$ we could
have just as well used \texttt{yellow}, \texttt{zero}, and \texttt{provable}, as long
as we did so consistently throughout the database.  Also, we could have used
\texttt{is provable} (two tokens) instead of \verb$|-$ (one token) throughout the
database.  In each of these cases, the proof would be exactly the same.  The
independence of proofs and notation means that you have a lot of flexibility to
change the notation you use without having to change any proofs.

\section{A Trial Run}\label{trialrun}

Now you are ready to try out the Metamath\index{Metamath} program.

On all computer systems, Metamath has a standard ``command line
interface'' (CLI)\index{command line interface (CLI)} that allows you to
interact with it.  You supply commands to the CLI by typing them on the
keyboard and pressing your keyboard's {\em return} key after each line
you enter.  The CLI is designed to be easy to use and has built-in help
features.

The first thing you should do is to use a text editor to create a file
called \texttt{demo0.mm} and type into it the Metamath source shown on
p.~\pageref{demo0}.  Actually, this file is included with your Metamath
software package, so check that first.  If you type it in, make sure
that you save it in the form of ``plain {\sc ascii} text with line
breaks.''  Most word processors will have this feature.

Next you must run the Metamath program.  Depending on your computer
system and how Metamath is installed, this could range from clicking the
mouse on the Metamath icon to typing \texttt{run metamath} to typing
simply \texttt{metamath}.  (Metamath's {\tt help invoke} command describes
alternate ways of invoking the Metamath program.)

When you first enter Metamath\index{Metamath}, it will be at the CLI, waiting
for your input. You will see something like the following on your screen:
\begin{verbatim}
Metamath - Version 0.172 25-Jan-2019
Type HELP for help, EXIT to exit.
MM>
\end{verbatim}
The \texttt{MM>} prompt means that Metamath is waiting for a command.
Command keywords\index{command keyword} are not case sensitive;
we will use lower-case commands in our examples.
The version number and its release date will probably be different on your
system from the one we show above.

The first thing that you need to do is to read in your
database:\index{\texttt{read} command}\footnote{If a directory path is
needed on Unix,\index{Unix file names}\index{file names!Unix} you should
enclose the path/file name in quotes to prevent Metamath from thinking
that the \texttt{/} in the path name is a command qualifier, e.g.,
\texttt{read \char`\"db/set.mm\char`\"}.  Quotes are optional when there
is no ambiguity.}
\begin{verbatim}
MM> read demo0.mm
\end{verbatim}
Remember to press the {\em return} key after entering this command.  If
you omit the file name, Metamath will prompt you for one.   The syntax for
specifying a Macintosh file name path is given in a footnote on
p.~\pageref{includef}.\index{Macintosh file names}\index{file
names!Macintosh}

If there are any syntax errors in the database, Metamath will let you know
when it reads in the file.  The one thing that Metamath does not check when
reading in a database is that all proofs are correct, because this would
slow it down too much.  It is a good idea to periodically verify the proofs in
a database you are making changes to.  To do this, use the following command
(and do it for your \texttt{demo0.mm} file now).  Note that the \texttt{*} is a
``wild card'' meaning all proofs in the file.\index{\texttt{verify proof} command}
\begin{verbatim}
MM> verify proof *
\end{verbatim}
Metamath will report any proofs that are incorrect.

It is often useful to save the information that the Metamath program displays
on the screen. You can save everything that happens on the screen by opening a
log file. You may want to do this before you read in a database so that you
can examine any errors later on.  To open a log file, type
\begin{verbatim}
MM> open log abc.log
\end{verbatim}
This will open a file called \texttt{abc.log}, and everything that appears on the
screen from this point on will be stored in this file.  The name of the log file
is arbitrary. To close the log file, type
\begin{verbatim}
MM> close log
\end{verbatim}

Several commands let you examine what's inside your database.
Section~\ref{exploring} has an overview of some useful ones.  The
\texttt{show labels} command lets you see what statement
labels\index{label} exist.  A \texttt{*} matches any combination of
characters, and \texttt{t*} refers to all labels starting with the
letter \texttt{t}.\index{\texttt{show labels} command} The \texttt{/all}
is a ``command qualifier''\index{command qualifier} that tells Metamath
to include labels of hypotheses.  (To see the syntax explained, type
\texttt{help show labels}.)  Type
\begin{verbatim}
MM> show labels t* /all
\end{verbatim}
Metamath will respond with
\begin{verbatim}
The statement number, label, and type are shown.
3 tt $f       4 tr $f       5 ts $f       8 tze $a
9 tpl $a      19 th1 $p
\end{verbatim}

You can use the \texttt{show statement} command to get information about a
particular statement.\index{\texttt{show statement} command}
For example, you can get information about the statement with label \texttt{mp}
by typing
\begin{verbatim}
MM> show statement mp /full
\end{verbatim}
Metamath will respond with
\begin{verbatim}
Statement 17 is located on line 43 of the file
"demo0.mm".
"Define the modus ponens inference rule"
17 mp $a |- Q $.
Its mandatory hypotheses in RPN order are:
  wp $f wff P $.
  wq $f wff Q $.
  min $e |- P $.
  maj $e |- ( P -> Q ) $.
The statement and its hypotheses require the
      variables:  Q P
The variables it contains are:  Q P
\end{verbatim}
The mandatory hypotheses\index{mandatory hypothesis} and their
order\index{RPN order} are
useful to know when you are trying to understand or debug a proof.

Now you are ready to look at what's really inside our proof.  First, here is
how to look at every step in the proof---not just the ones corresponding to an
ordinary formal proof\index{formal proof}, but also the ones that build up the
formulas that appear in each ordinary formal proof step.\index{\texttt{show
proof} command}
\begin{verbatim}
MM> show proof th1 /lemmon /all
\end{verbatim}

This will display the proof on the screen in the following format:
\begin{verbatim}
 1 tt            $f term t
 2 tze           $a term 0
 3 1,2 tpl       $a term ( t + 0 )
 4 tt            $f term t
 5 3,4 weq       $a wff ( t + 0 ) = t
 6 tt            $f term t
 7 tt            $f term t
 8 6,7 weq       $a wff t = t
 9 tt            $f term t
10 9 a2          $a |- ( t + 0 ) = t
11 tt            $f term t
12 tze           $a term 0
13 11,12 tpl     $a term ( t + 0 )
14 tt            $f term t
15 13,14 weq     $a wff ( t + 0 ) = t
16 tt            $f term t
17 tze           $a term 0
18 16,17 tpl     $a term ( t + 0 )
19 tt            $f term t
20 18,19 weq     $a wff ( t + 0 ) = t
21 tt            $f term t
22 tt            $f term t
23 21,22 weq     $a wff t = t
24 20,23 wim     $a wff ( ( t + 0 ) = t -> t = t )
25 tt            $f term t
26 25 a2         $a |- ( t + 0 ) = t
27 tt            $f term t
28 tze           $a term 0
29 27,28 tpl     $a term ( t + 0 )
30 tt            $f term t
31 tt            $f term t
32 29,30,31 a1   $a |- ( ( t + 0 ) = t -> ( ( t + 0 )
                                     = t -> t = t ) )
33 15,24,26,32 mp  $a |- ( ( t + 0 ) = t -> t = t )
34 5,8,10,33 mp  $a |- t = t
\end{verbatim}

The \texttt{/lemmon} command qualifier specifies what is known as a Lemmon-style
display\index{Lemmon-style proof}\index{proof!Lemmon-style}.  Omitting the
\texttt{/lemmon} qualifier results in a tree-style proof (see
p.~\pageref{treeproof} for an example) that is somewhat less explicit but
easier to follow once you get used to it.\index{tree-style
proof}\index{proof!tree-style}

The first number on each line is the step
number of the proof.  Any numbers that follow are step numbers assigned to the
hypotheses of the statement referenced by that step.  Next is the label of
the statement referenced by the step.  The statement type of the statement
referenced comes next, followed by the math symbol\index{math symbol} string
constructed by the proof up to that step.

The last step, 34, contains the statement that is being proved.

Looking at a small piece of the proof, notice that steps 3 and 4 have
established that
\texttt{( t + 0 )} and \texttt{t} are \texttt{term}\,s, and step 5 makes use of steps 3 and
4 to establish that \texttt{( t + 0 ) = t} is a \texttt{wff}.  Let Metamath
itself tell us in detail what is happening in step 5.  Note that the
``target hypothesis'' refers to where step 5 is eventually used, i.e., in step
34.
\begin{verbatim}
MM> show proof th1 /detailed_step 5
Proof step 5:  wp=weq $a wff ( t + 0 ) = t
This step assigns source "weq" ($a) to target "wp"
($f).  The source assertion requires the hypotheses
"tt" ($f, step 3) and "tr" ($f, step 4).  The parent
assertion of the target hypothesis is "mp" ($a,
step 34).
The source assertion before substitution was:
    weq $a wff t = r
The following substitutions were made to the source
assertion:
    Variable  Substituted with
     t         ( t + 0 )
     r         t
The target hypothesis before substitution was:
    wp $f wff P
The following substitution was made to the target
hypothesis:
    Variable  Substituted with
     P         ( t + 0 ) = t
\end{verbatim}

The full proof just shown is useful to understand what is going on in detail.
However, most of the time you will just be interested in
the ``essential'' or logical steps of a proof, i.e.\ those steps
that correspond to an
ordinary formal proof\index{formal proof}.  If you type
\begin{verbatim}
MM> show proof th1 /lemmon /renumber
\end{verbatim}
you will see\label{demoproof}
\begin{verbatim}
1 a2             $a |- ( t + 0 ) = t
2 a2             $a |- ( t + 0 ) = t
3 a1             $a |- ( ( t + 0 ) = t -> ( ( t + 0 )
                                     = t -> t = t ) )
4 2,3 mp         $a |- ( ( t + 0 ) = t -> t = t )
5 1,4 mp         $a |- t = t
\end{verbatim}
Compare this to the formal proof on p.~\pageref{zeroproof} and
notice the resemblance.
By default Metamath
does not show \texttt{\$f}\index{\texttt{\$f}
statement} hypotheses and everything branching off of them in the proof tree
when the proof is displayed; this makes the proof look more like an ordinary
mathematical proof, which does not normally incorporate the explicit
construction of expressions.
This is called the ``essential'' view
(at one time you had to add the
\texttt{/essential} qualifier in the \texttt{show proof}
command to get this view, but this is now the default).
You can could use the \texttt{/all} qualifier in the \texttt{show
proof} command to also show the explicit construction of expressions.
The \texttt{/renumber} qualifier means to renumber
the steps to correspond only to what is displayed.\index{\texttt{show proof}
command}

To exit Metamath, type\index{\texttt{exit} command}
\begin{verbatim}
MM> exit
\end{verbatim}

\subsection{Some Hints for Using the Command Line Interface}

We will conclude this quick introduction to Metamath\index{Metamath} with some
helpful hints on how to navigate your way through the commands.
\index{command line interface (CLI)}

When you type commands into Metamath's CLI, you only have to type as many
characters of a command keyword\index{command keyword} as are needed to make
it unambiguous.  If you type too few characters, Metamath will tell you what
the choices are.  In the case of the \texttt{read} command, only the \texttt{r} is
needed to specify it unambiguously, so you could have typed\index{\texttt{read}
command}
\begin{verbatim}
MM> r demo0.mm
\end{verbatim}
instead of
\begin{verbatim}
MM> read demo0.mm
\end{verbatim}
In our description, we always show the full command words.  When using the
Metamath CLI commands in a command file (to be read with the \texttt{submit}
command)\index{\texttt{submit} command}, it is good practice to use
the unabbreviated command to ensure your instructions will not become ambiguous
if more commands are added to the Metamath program in the future.

The command keywords\index{command
keyword} are not case sensitive; you may type either \texttt{read} or
\texttt{ReAd}.  File names may or may not be case sensitive, depending on your
computer's operating system.  Metamath label\index{label} and math
symbol\index{math symbol} tokens\index{token} are case-sensitive.

The \texttt{help} command\index{\texttt{help} command} will provide you
with a list of topics you can get help on.  You can then type
\texttt{help} {\em topic} to get help on that topic.

If you are uncertain of a command's spelling, just type as many characters
as you remember of the command.  If you have not typed enough characters to
specify it unambiguously, Metamath will tell you what choices you have.

\begin{verbatim}
MM> show s
         ^
?Ambiguous keyword - please specify SETTINGS,
STATEMENT, or SOURCE.
\end{verbatim}

If you don't know what argument to use as part of a command, type a
\texttt{?}\index{\texttt{]}@\texttt{?}\ in command lines}\ at the
argument position.  Metamath will tell you what it expected there.

\begin{verbatim}
MM> show ?
         ^
?Expected SETTINGS, LABELS, STATEMENT, SOURCE, PROOF,
MEMORY, TRACE_BACK, or USAGE.
\end{verbatim}

Finally, you may type just the first word or words of a command followed
by {\em return}.  Metamath will prompt you for the remaining part of the
command, showing you the choices at each step.  For example, instead of
typing \texttt{show statement th1 /full} you could interact in the
following manner:
\begin{verbatim}
MM> show
SETTINGS, LABELS, STATEMENT, SOURCE, PROOF,
MEMORY, TRACE_BACK, or USAGE <SETTINGS>? st
What is the statement label <th1>?
/ or nothing <nothing>? /
TEX, COMMENT_ONLY, or FULL <TEX>? f
/ or nothing <nothing>?
19 th1 $p |- t = t $= ... $.
\end{verbatim}
After each \texttt{?}\ in this mode, you must give Metamath the
information it requests.  Sometimes Metamath gives you a list of choices
with the default choice indicated by brackets \texttt{< > }. Pressing
{\em return} after the \texttt{?}\ will select the default choice.
Answering anything else will override the default.  Note that the
\texttt{/} in command qualifiers is considered a separate
token\index{token} by the parser, and this is why it is asked for
separately.

\section{Your First Proof}\label{frstprf}

Proofs are developed with the aid of the Proof Assistant\index{Proof
Assistant}.  We will now show you how the proof of theorem \texttt{th1}
was built.  So that you can repeat these steps, we will first have the
Proof Assistant erase the proof in Metamath's source buffer\index{source
buffer}, then reconstruct it.  (The source buffer is the place in memory
where Metamath stores the information in the database when it is
\texttt{read}\index{\texttt{read} command} in.  New or modified proofs
are kept in the source buffer until a \texttt{write source}
command\index{\texttt{write source} command} is issued.)  In practice, you
would place a \texttt{?}\index{\texttt{]}@\texttt{?}\ inside proofs}\
between \texttt{\$=}\index{\texttt{\$=} keyword} and
\texttt{\$.}\index{\texttt{\$.}\ keyword}\ in the database to indicate
to Metamath\index{Metamath} that the proof is unknown, and that would be
your starting point.  Whenever the \texttt{verify proof} command encounters
a proof with a \texttt{?}\ in place of a proof step, the statement is
identified as not proved.

When I first started creating Metamath proofs, I would write down
on a piece of paper the complete
formal proof\index{formal proof} as it would appear
in a \texttt{show proof} command\index{\texttt{show proof} command}; see
the display of \texttt{show proof th1 /lemmon /re\-num\-ber} above as an
example.  After you get used to using the Proof Assistant\index{Proof
Assistant} you may get to a point where you can ``see'' the proof in your mind
and let the Proof Assistant guide you in filling in the details, at least for
simpler proofs, but until you gain that experience you may find it very useful
to write down all the details in advance.
Otherwise you may waste a lot of time as you let it take you down a wrong path.
However, others do not find this approach as helpful.
For example, Thomas Brendan Leahy\index{Leahy, Thomas Brendan}
finds that it is more helpful to him to interactively
work backward from a machine-readable statement.
David A. Wheeler\index{Wheeler, David A.}
writes down a general approach, but develops the proof
interactively by switching between
working forwards (from hypotheses and facts likely to be useful) and
backwards (from the goal) until the forwards and backwards approaches meet.
In the end, use whatever approach works for you.

A proof is developed with the Proof Assistant by working backwards, starting
with the theorem\index{theorem} to be proved, and assigning each unknown step
with a theorem or hypothesis until no more unknown steps remain.  The Proof
Assistant will not let you make an assignment unless it can be ``unified''
with the unknown step.  This means that a
substitution\index{substitution!variable}\index{variable substitution} of
variables exists that will make the assignment match the unknown step.  On the
other hand, in the middle of a proof, when working backwards, often more than
one unification\index{unification} (set of substitutions) is possible, since
there is not enough information available at that point to uniquely establish
it.  In this case you can tell Metamath which unification to choose, or you
can continue to assign unknown steps until enough information is available to
make the unification unique.

We will assume you have entered Metamath and read in the database as described
above.  The following dialog shows how the proof was developed.  For more
details on what some of the commands do, refer to Section~\ref{pfcommands}.
\index{\texttt{prove} command}

\begin{verbatim}
MM> prove th1
Entering the Proof Assistant.  Type HELP for help, EXIT
to exit.  You will be working on the proof of statement th1:
  $p |- t = t
Note:  The proof you are starting with is already complete.
MM-PA>
\end{verbatim}

The \verb/MM-PA>/ prompt means we are inside the Proof
Assistant.\index{Proof Assistant} Most of the regular Metamath commands
(\texttt{show statement}, etc.) are still available if you need them.

\begin{verbatim}
MM-PA> delete all
The entire proof was deleted.
\end{verbatim}

We have deleted the whole proof so we can start from scratch.

\begin{verbatim}
MM-PA> show new_proof/lemmon/all
1 ?              $? |- t = t
\end{verbatim}

The \texttt{show new{\char`\_}proof} command\index{\texttt{show
new{\char`\_}proof} command} is like \texttt{show proof} except that we
don't specify a statement; instead, the proof we're working on is
displayed.

\begin{verbatim}
MM-PA> assign 1 mp
To undo the assignment, DELETE STEP 5 and INITIALIZE, UNIFY
if needed.
3   min=?  $? |- $2
4   maj=?  $? |- ( $2 -> t = t )
\end{verbatim}

The \texttt{assign} command\index{\texttt{assign} command} above means
``assign step 1 with the statement whose label is \texttt{mp}.''  Note
that step renumbering will constantly occur as you assign steps in the
middle of a proof; in general all steps from the step you assign until
the end of the proof will get moved up.  In this case, what used to be
step 1 is now step 5, because the (partial) proof now has five steps:
the four hypotheses of the \texttt{mp} statement and the \texttt{mp}
statement itself.  Let's look at all the steps in our partial proof:

\begin{verbatim}
MM-PA> show new_proof/lemmon/all
1 ?              $? wff $2
2 ?              $? wff t = t
3 ?              $? |- $2
4 ?              $? |- ( $2 -> t = t )
5 1,2,3,4 mp     $a |- t = t
\end{verbatim}

The symbol \texttt{\$2} is a temporary variable\index{temporary
variable} that represents a symbol sequence not yet known.  In the final
proof, all temporary variables will be eliminated.  The general format
for a temporary variable is \texttt{\$} followed by an integer.  Note
that \texttt{\$} is not a legal character in a math symbol (see
Section~\ref{dollardollar}, p.~\pageref{dollardollar}), so there will
never be a naming conflict between real symbols and temporary variables.

Unknown steps 1 and 2 are constructions of the two wffs used by the
modus ponens rule.  As you will see at the end of this section, the
Proof Assistant\index{Proof Assistant} can usually figure these steps
out by itself, and we will not have to worry about them.  Therefore from
here on we will display only the ``essential'' hypotheses, i.e.\ those
steps that correspond to traditional formal proofs\index{formal proof}.

\begin{verbatim}
MM-PA> show new_proof/lemmon
3 ?              $? |- $2
4 ?              $? |- ( $2 -> t = t )
5 3,4 mp         $a |- t = t
\end{verbatim}

Unknown steps 3 and 4 are the ones we must focus on.  They correspond to the
minor and major premises of the modus ponens rule.  We will assign them as
follows.  Notice that because of the step renumbering that takes place
after an assignment, it is advantageous to assign unknown steps in reverse
order, because earlier steps will not get renumbered.

\begin{verbatim}
MM-PA> assign 4 mp
To undo the assignment, DELETE STEP 8 and INITIALIZE, UNIFY
if needed.
3   min=?  $? |- $2
6     min=?  $? |- $4
7     maj=?  $? |- ( $4 -> ( $2 -> t = t ) )
\end{verbatim}

We are now going to describe an obscure feature that you will probably
never use but should be aware of.  The Metamath language allows empty
symbol sequences to be substituted for variables, but in most formal
systems this feature is never used.  One of the few examples where is it
used is the MIU-system\index{MIU-system} described in
Appendix~\ref{MIU}.  But such systems are rare, and by default this
feature is turned off in the Proof Assistant.  (It is always allowed for
{\tt verify proof}.)  Let us turn it on and see what
happens.\index{\texttt{set empty{\char`\_}substitution} command}

\begin{verbatim}
MM-PA> set empty_substitution on
Substitutions with empty symbol sequences is now allowed.
\end{verbatim}

With this feature enabled, more unifications will be
ambiguous\index{ambiguous unification}\index{unification!ambiguous} in
the middle of a proof, because
substitution\index{substitution!variable}\index{variable substitution}
of variables with empty symbol sequences will become an additional
possibility.  Let's see what happens when we make our next assignment.

\begin{verbatim}
MM-PA> assign 3 a2
There are 2 possible unifications.  Please select the correct
    one or Q if you want to UNIFY later.
Unify:  |- $6
 with:  |- ( $9 + 0 ) = $9
Unification #1 of 2 (weight = 7):
  Replace "$6" with "( + 0 ) ="
  Replace "$9" with ""
  Accept (A), reject (R), or quit (Q) <A>? r
\end{verbatim}

The first choice presented is the wrong one.  If we had selected it,
temporary variable \texttt{\$6} would have been assigned a truncated
wff, and temporary variable \texttt{\$9} would have been assigned an
empty sequence (which is not allowed in our system).  With this choice,
eventually we would reach a point where we would get stuck because
we would end up with steps impossible to prove.  (You may want to
try it.)  We typed \texttt{r} to reject the choice.

\begin{verbatim}
Unification #2 of 2 (weight = 21):
  Replace "$6" with "( $9 + 0 ) = $9"
  Accept (A), reject (R), or quit (Q) <A>? q
To undo the assignment, DELETE STEP 4 and INITIALIZE, UNIFY
if needed.
 7     min=?  $? |- $8
 8     maj=?  $? |- ( $8 -> ( $6 -> t = t ) )
\end{verbatim}

The second choice is correct, and normally we would type \texttt{a}
to accept it.  But instead we typed \texttt{q} to show what will happen:
it will leave the step with an unknown unification, which can be
seen as follows:

\begin{verbatim}
MM-PA> show new_proof/not_unified
 4   min    $a |- $6
        =a2  = |- ( $9 + 0 ) = $9
\end{verbatim}

Later we can unify this with the \texttt{unify}
\texttt{all/interactive} command.

The important point to remember is that occasionally you will be
presented with several unification choices while entering a proof, when
the program determines that there is not enough information yet to make
an unambiguous choice automatically (and this can happen even with
\texttt{set empty{\char`\_}substitution} turned off).  Usually it is
obvious by inspection which choice is correct, since incorrect ones will
tend to be meaningless fragments of wffs.  In addition, the correct
choice will usually be the first one presented, unlike our example
above.

Enough of this digression.  Let us go back to the default setting.

\begin{verbatim}
MM-PA> set empty_substitution off
The ability to substitute empty expressions for variables
has been turned off.  Note that this may make the Proof
Assistant too restrictive in some cases.
\end{verbatim}

If we delete the proof, start over, and get to the point where
we digressed above, there will no longer be an ambiguous unification.

\begin{verbatim}
MM-PA> assign 3 a2
To undo the assignment, DELETE STEP 4 and INITIALIZE, UNIFY
if needed.
 7     min=?  $? |- $4
 8     maj=?  $? |- ( $4 -> ( ( $5 + 0 ) = $5 -> t = t ) )
\end{verbatim}

Let us look at our proof so far, and continue.

\begin{verbatim}
MM-PA> show new_proof/lemmon
 4 a2            $a |- ( $5 + 0 ) = $5
 7 ?             $? |- $4
 8 ?             $? |- ( $4 -> ( ( $5 + 0 ) = $5 -> t = t ) )
 9 7,8 mp        $a |- ( ( $5 + 0 ) = $5 -> t = t )
10 4,9 mp        $a |- t = t
MM-PA> assign 8 a1
To undo the assignment, DELETE STEP 11 and INITIALIZE, UNIFY
if needed.
 7     min=?  $? |- ( t + 0 ) = t
MM-PA> assign 7 a2
To undo the assignment, DELETE STEP 8 and INITIALIZE, UNIFY
if needed.
MM-PA> show new_proof/lemmon
 4 a2            $a |- ( t + 0 ) = t
 8 a2            $a |- ( t + 0 ) = t
12 a1            $a |- ( ( t + 0 ) = t -> ( ( t + 0 ) = t ->
                                                    t = t ) )
13 8,12 mp       $a |- ( ( t + 0 ) = t -> t = t )
14 4,13 mp       $a |- t = t
\end{verbatim}

Now all temporary variables and unknown steps have been eliminated from the
``essential'' part of the proof.  When this is achieved, the Proof
Assistant\index{Proof Assistant} can usually figure out the rest of the proof
automatically.  (Note that the \texttt{improve} command can occasionally be
useful for filling in essential steps as well, but it only tries to make use
of statements that introduce no new variables in their hypotheses, which is
not the case for \texttt{mp}. Also it will not try to improve steps containing
temporary variables.)  Let's look at the complete proof, then run
the \texttt{improve} command, then look at it again.

\begin{verbatim}
MM-PA> show new_proof/lemmon/all
 1 ?             $? wff ( t + 0 ) = t
 2 ?             $? wff t = t
 3 ?             $? term t
 4 3 a2          $a |- ( t + 0 ) = t
 5 ?             $? wff ( t + 0 ) = t
 6 ?             $? wff ( ( t + 0 ) = t -> t = t )
 7 ?             $? term t
 8 7 a2          $a |- ( t + 0 ) = t
 9 ?             $? term ( t + 0 )
10 ?             $? term t
11 ?             $? term t
12 9,10,11 a1    $a |- ( ( t + 0 ) = t -> ( ( t + 0 ) = t ->
                                                    t = t ) )
13 5,6,8,12 mp   $a |- ( ( t + 0 ) = t -> t = t )
14 1,2,4,13 mp   $a |- t = t
\end{verbatim}

\begin{verbatim}
MM-PA> improve all
A proof of length 1 was found for step 11.
A proof of length 1 was found for step 10.
A proof of length 3 was found for step 9.
A proof of length 1 was found for step 7.
A proof of length 9 was found for step 6.
A proof of length 5 was found for step 5.
A proof of length 1 was found for step 3.
A proof of length 3 was found for step 2.
A proof of length 5 was found for step 1.
Steps 1 and above have been renumbered.
CONGRATULATIONS!  The proof is complete.  Use SAVE
NEW_PROOF to save it.  Note:  The Proof Assistant does
not detect $d violations.  After saving the proof, you
should verify it with VERIFY PROOF.
\end{verbatim}

The \texttt{save new{\char`\_}proof} command\index{\texttt{save
new{\char`\_}proof} command} will save the proof in the database.  Here
we will just display it in a form that can be clipped out of a log file
and inserted manually into the database source file with a text
editor.\index{normal proof}\index{proof!normal}

\begin{verbatim}
MM-PA> show new_proof/normal
---------Clip out the proof below this line:
      tt tze tpl tt weq tt tt weq tt a2 tt tze tpl tt weq
      tt tze tpl tt weq tt tt weq wim tt a2 tt tze tpl tt
      tt a1 mp mp $.
---------The proof of 'th1' to clip out ends above this line.
\end{verbatim}

There is another proof format called ``compressed''\index{compressed
proof}\index{proof!compressed} that you will see in databases.  It is
not important to understand how it is encoded but only to recognize it
when you see it.  Its only purpose is to reduce storage requirements for
large proofs.  A compressed proof can always be converted to a normal
one and vice-versa, and the Metamath \texttt{show proof}
commands\index{\texttt{show proof} command} work equally well with
compressed proofs.  The compressed proof format is described in
Appendix~\ref{compressed}.

\begin{verbatim}
MM-PA> show new_proof/compressed
---------Clip out the proof below this line:
      ( tze tpl weq a2 wim a1 mp ) ABCZADZAADZAEZJJKFLIA
      AGHH $.
---------The proof of 'th1' to clip out ends above this line.
\end{verbatim}

Now we will exit the Proof Assistant.  Since we made changes to the proof,
it will warn us that we have not saved it.  In this case, we don't care.

\begin{verbatim}
MM-PA> exit
Warning:  You have not saved changes to the proof.
Do you want to EXIT anyway (Y, N) <N>? y
Exiting the Proof Assistant.
Type EXIT again to exit Metamath.
\end{verbatim}

The Proof Assistant\index{Proof Assistant} has several other commands
that can help you while creating proofs.  See Section~\ref{pfcommands}
for a list of them.

A command that is often useful is \texttt{minimize{\char`\_}with
*/brief}, which tries to shorten the proof.  It can make the process
more efficient by letting you write a somewhat ``sloppy'' proof then
clean up some of the fine details of optimization for you (although it
can't perform miracles such as restructuring the overall proof).

\section{A Note About Editing a Data\-base File}

Once your source file contains proofs, there are some restrictions on
how you can edit it so that the proofs remain valid.  Pay particular
attention to these rules, since otherwise you can lose a lot of work.
It is a good idea to periodically verify all proofs with \texttt{verify
proof *} to ensure their integrity.

If your file contains only normal (as opposed to compressed) proofs, the
main rule is that you may not change the order of the mandatory
hypotheses\index{mandatory hypothesis} of any statement referenced in a
later proof.  For example, if you swap the order of the major and minor
premise in the modus ponens rule, all proofs making use of that rule
will become incorrect.  The \texttt{show statement}
command\index{\texttt{show statement} command} will show you the
mandatory hypotheses of a statement and their order.

If a statement has a compressed proof, you also must not change the
order of {\em its} mandatory hypotheses.  The compressed proof format
makes use of this information as part of the compression technique.
Note that swapping the names of two variables in a theorem will change
the order of its mandatory hypotheses.

The safest way to edit a statement, say \texttt{mytheorem}, is to
duplicate it then rename the original to \texttt{mytheoremOLD}
throughout the database.  Once the edited version is re-proved, all
statements referencing \texttt{mytheoremOLD} can be updated in the Proof
Assistant using \texttt{minimize{\char`\_}with
mytheorem
/allow{\char`\_}growth}.\index{\texttt{minimize{\char`\_}with} command}
% 3/10/07 Note: line-breaking the above results in duplicate index entries

\chapter{Abstract Mathematics Revealed}\label{fol}

\section{Logic and Set Theory}\label{logicandsettheory}

\begin{quote}
  {\em Set theory can be viewed as a form of exact theology.}
  \flushright\sc  Rudy Rucker\footnote{\cite{Barrow}, p.~31.}\\
\end{quote}\index{Rucker, Rudy}

Despite its seeming complexity, all of standard mathematics, no matter how
deep or abstract, can amazingly enough be derived from a relatively small set
of axioms\index{axiom} or first principles. The development of these axioms is
among the most impressive and important accomplishments of mathematics in the
20th century. Ultimately, these axioms can be broken down into a set of rules
for manipulating symbols that any technically oriented person can follow.

We will not spend much time trying to convey a deep, higher-level
understanding of the meaning of the axioms. This kind of understanding
requires some mathematical sophistication as well as an understanding of the
philosophy underlying the foundations of mathematics and typically develops
over time as you work with mathematics.  Our goal, instead, is to give you the
immediate ability to follow how theorems\index{theorem} are derived from the
axioms and from other theorems.  This will be similar to learning the syntax
of a computer language, which lets you follow the details in a program but
does not necessarily give you the ability to write non-trivial programs on
your own, an ability that comes with practice. For now don't be alarmed by
abstract-sounding names of the axioms; just focus on the rules for
manipulating the symbols, which follow the simple conventions of the
Metamath\index{Metamath} language.

The axioms that underlie all of standard mathematics consist of axioms of logic
and axioms of set theory. The axioms of logic are divided into two
subcategories, propositional calculus\index{propositional calculus} (sometimes
called sentential logic\index{sentential logic}) and predicate calculus
(sometimes called first-order logic\index{first-order logic}\index{quantifier
theory}\index{predicate calculus} or quantifier theory).  Propositional
calculus is a prerequisite for predicate calculus, and predicate calculus is a
prerequisite for set theory.  The version of set theory most commonly used is
Zermelo--Fraenkel set theory\index{Zermelo--Fraenkel set theory}\index{set theory}
with the axiom of choice,
often abbreviated as ZFC\index{ZFC}.

Here in a nutshell is what the axioms are all about in an informal way. The
connection between this description and symbols we will show you won't be
immediately apparent and in principle needn't ever be.  Our description just
tries to summarize what mathematicians think about when they work with the
axioms.

Logic is a set of rules that allow us determine truths given other truths.
Put another way,
logic is more or less the translation of what we would consider common sense
into a rigorous set of axioms.\index{axioms of logic}  Suppose $\varphi$,
$\psi$, and $\chi$ (the Greek letters phi, psi, and chi) represent statements
that are either true or false, and $x$ is a variable\index{variable!in predicate
calculus} ranging over some group of mathematical objects (sets, integers,
real numbers, etc.). In mathematics, a ``statement'' really means a formula,
and $\psi$ could be for example ``$x = 2$.''
Propositional calculus\index{propositional calculus}
allows us to use variables that are either true or false
and make deductions such as
``if $\varphi$ implies $\psi$ and $\psi$ implies $\chi$, then $\varphi$
implies $\chi$.''
Predicate calculus\index{predicate calculus}
extends propositional calculus by also allowing us
to discuss statements about objects (not just true and false values), including
statements about ``all'' or ``at least one'' object.
For example, predicate calculus allows to say,
``if $\varphi$ is true for all $x$, then $\varphi$ is true for some $x$.''
The logic used in \texttt{set.mm} is standard classical logic
(as opposed to other logic systems like intuitionistic logic).

Set theory\index{set theory} has to do with the manipulation of objects and
collections of objects, specifically the abstract, imaginary objects that
mathematics deals with, such as numbers. Everything that is claimed to exist
in mathematics is considered to be a set.  A set called the empty
set\index{empty set} contains nothing.  We represent the empty set by
$\varnothing$.  Many sets can be built up from the empty set.  There is a set
represented by $\{\varnothing\}$ that contains the empty set, another set
represented by $\{\varnothing,\{\varnothing\}\}$ that contains this set as
well as the empty set, another set represented by $\{\{\varnothing\}\}$ that
contains just the set that contains the empty set, and so on ad infinitum. All
mathematical objects, no matter how complex, are defined as being identical to
certain sets: the integer\index{integer} 0 is defined as the empty set, the
integer 1 is defined as $\{\varnothing\}$, the integer 2 is defined as
$\{\varnothing,\{\varnothing\}\}$.  (How these definitions were chosen doesn't
matter now, but the idea behind it is that these sets have the properties we
expect of integers once suitable operations are defined.)  Mathematical
operations, such as addition, are defined in terms of operations on
sets---their union\index{set union}, intersection\index{set intersection}, and
so on---operations you may have used in elementary school when you worked
with groups of apples and oranges.

With a leap of faith, the axioms also postulate the existence of infinite
sets\index{infinite set}, such as the set of all non-negative integers ($0, 1,
2,\ldots$, also called ``natural numbers''\index{natural number}).  This set
can't be represented with the brace notation\index{brace notation} we just
showed you, but requires a more complicated notation called ``class
abstraction.''\index{class abstraction}\index{abstraction class}  For
example, the infinite set $\{ x |
\mbox{``$x$ is a natural number''} \} $ means the ``set of all objects $x$
such that $x$ is a natural number'' i.e.\ the set of natural numbers; here,
``$x$ is a natural number'' is a rather complicated formula when broken down
into the primitive symbols.\label{expandom}\footnote{The statement ``$x$ is a
natural number'' is formally expressed as ``$x \in \omega$,'' where $\in$
(stylized epsilon) means ``is in'' or ``is an element of'' and $\omega$
(omega) means ``the set of natural numbers.''  When ``$x\in\omega$'' is
completely expanded in terms of the primitive symbols of set theory, the
result is  $\lnot$ $($ $\lnot$ $($ $\forall$ $z$ $($ $\lnot$ $\forall$ $w$ $($
$z$ $\in$ $w$ $\rightarrow$ $\lnot$ $w$ $\in$ $x$ $)$ $\rightarrow$ $z$ $\in$
$x$ $)$ $\rightarrow$ $($ $\forall$ $z$ $($ $\lnot$ $($ $\forall$ $w$ $($ $w$
$\in$ $z$ $\rightarrow$ $w$ $\in$ $x$ $)$ $\rightarrow$ $\forall$ $w$ $\lnot$
$w$ $\in$ $z$ $)$ $\rightarrow$ $\lnot$ $\forall$ $w$ $($ $w$ $\in$ $z$
$\rightarrow$ $\lnot$ $\forall$ $v$ $($ $v$ $\in$ $z$ $\rightarrow$ $\lnot$
$v$ $\in$ $w$ $)$ $)$ $)$ $\rightarrow$ $\lnot$ $\forall$ $z$ $\forall$ $w$
$($ $\lnot$ $($ $z$ $\in$ $x$ $\rightarrow$ $\lnot$ $w$ $\in$ $x$ $)$
$\rightarrow$ $($ $\lnot$ $z$ $\in$ $w$ $\rightarrow$ $($ $\lnot$ $z$ $=$ $w$
$\rightarrow$ $w$ $\in$ $z$ $)$ $)$ $)$ $)$ $)$ $\rightarrow$ $\lnot$
$\forall$ $y$ $($ $\lnot$ $($ $\lnot$ $($ $\forall$ $z$ $($ $\lnot$ $\forall$
$w$ $($ $z$ $\in$ $w$ $\rightarrow$ $\lnot$ $w$ $\in$ $y$ $)$ $\rightarrow$
$z$ $\in$ $y$ $)$ $\rightarrow$ $($ $\forall$ $z$ $($ $\lnot$ $($ $\forall$
$w$ $($ $w$ $\in$ $z$ $\rightarrow$ $w$ $\in$ $y$ $)$ $\rightarrow$ $\forall$
$w$ $\lnot$ $w$ $\in$ $z$ $)$ $\rightarrow$ $\lnot$ $\forall$ $w$ $($ $w$
$\in$ $z$ $\rightarrow$ $\lnot$ $\forall$ $v$ $($ $v$ $\in$ $z$ $\rightarrow$
$\lnot$ $v$ $\in$ $w$ $)$ $)$ $)$ $\rightarrow$ $\lnot$ $\forall$ $z$
$\forall$ $w$ $($ $\lnot$ $($ $z$ $\in$ $y$ $\rightarrow$ $\lnot$ $w$ $\in$
$y$ $)$ $\rightarrow$ $($ $\lnot$ $z$ $\in$ $w$ $\rightarrow$ $($ $\lnot$ $z$
$=$ $w$ $\rightarrow$ $w$ $\in$ $z$ $)$ $)$ $)$ $)$ $\rightarrow$ $($
$\forall$ $z$ $\lnot$ $z$ $\in$ $y$ $\rightarrow$ $\lnot$ $\forall$ $w$ $($
$\lnot$ $($ $w$ $\in$ $y$ $\rightarrow$ $\lnot$ $\forall$ $z$ $($ $w$ $\in$
$z$ $\rightarrow$ $\lnot$ $z$ $\in$ $y$ $)$ $)$ $\rightarrow$ $\lnot$ $($
$\lnot$ $\forall$ $z$ $($ $w$ $\in$ $z$ $\rightarrow$ $\lnot$ $z$ $\in$ $y$
$)$ $\rightarrow$ $w$ $\in$ $y$ $)$ $)$ $)$ $)$ $\rightarrow$ $x$ $\in$ $y$
$)$ $)$ $)$. Section~\ref{hierarchy} shows the hierarchy of definitions that
leads up to this expression.}\index{stylized epsilon ($\in$)}\index{omega
($\omega$)}  Actually, the primitive symbols don't even include the brace
notation.  The brace notation is a high-level definition, which you can find in
Section~\ref{hierarchy}.

Interestingly, the arithmetic of integers\index{integer} and
rationals\index{rational number} can be developed without appealing to the
existence of an infinite set, whereas the arithmetic of real
numbers\index{real number} requires it.

Each variable\index{variable!in set theory} in the axioms of set theory
represents an arbitrary set, and the axioms specify the legal kinds of things
you can do with these variables at a very primitive level.

Now, you may think that numbers and arithmetic are a lot more intuitive and
fundamental than sets and therefore should be the foundation of mathematics.
What is really the case is that you've dealt with numbers all your life and
are comfortable with a few rules for manipulating them such as addition and
multiplication.  Those rules only cover a small portion of what can be done
with numbers and only a very tiny fraction of the rest of mathematics.  If you
look at any elementary book on number theory, you will quickly become lost if
these are the only rules that you know.  Even though such books may present a
list of ``axioms''\index{axiom} for arithmetic, the ability to use the axioms
and to understand proofs of theorems\index{theorem} (facts) about numbers
requires an implicit mathematical talent that frustrates many people
from studying abstract mathematics.  The kind of mathematics that most people
know limits them to the practical, everyday usage of blindly manipulating
numbers and formulas, without any understanding of why those rules are correct
nor any ability to go any further.  For example, do you know why multiplying
two negative numbers yields a positive number?  Starting with set theory, you
will also start off blindly manipulating symbols according to the rules we give
you, but with the advantage that these rules will allow you, in principle, to
access {\em all} of mathematics, not just a tiny part of it.

Of course, concrete examples are often helpful in the learning process. For
example, you can verify that $2\cdot 3=3 \cdot 2$ by actually grouping
objects and can easily ``see'' how it generalizes to $x\cdot y = y\cdot x$,
even though you might not be able to rigorously prove it.  Similarly, in set
theory it can be helpful to understand how the axioms of set theory apply to
(and are correct for) small finite collections of objects.  You should be aware
that in set theory intuition can be misleading for infinite collections, and
rigorous proofs become more important.  For example, while $x\cdot y = y\cdot
x$ is correct for finite ordinals (which are the natural numbers), it is not
usually true for infinite ordinals.

\section{The Axioms for All of Mathematics}

In this section\index{axioms for mathematics}, we will show you the axioms
for all of standard mathematics (i.e.\ logic and set theory) as they are
traditionally presented.  The traditional presentation is useful for someone
with the mathematical experience needed to correctly manipulate high-level
abstract concepts.  For someone without this talent, knowing how to actually
make use of these axioms can be difficult.  The purpose of this section is to
allow you to see how the version of the axioms used in the standard
Metamath\index{Metamath} database \texttt{set.mm}\index{set
theory database (\texttt{set.mm})} relates to  the typical version
in textbooks, and also to give you an informal feel for them.

\subsection{Propositional Calculus}

Propositional calculus\index{propositional calculus} concerns itself with
statements that can be interpreted as either true or false.  Some examples of
statements (outside of mathematics) that are either true or false are ``It is
raining today'' and ``The United States has a female president.'' In
mathematics, as we mentioned, statements are really formulas.

In propositional calculus, we don't care what the statements are.  We also
treat a logical combination of statements, such as ``It is raining today and
the United States has a female president,'' no differently from a single
statement.  Statements and their combinations are called well-formed formulas
(wffs)\index{well-formed formula (wff)}.  We define wffs only in terms of
other wffs and don't define what a ``starting'' wff is.  As is common practice
in the literature, we use Greek letters to represent wffs.

Specifically, suppose $\varphi$ and $\psi$ are wffs.  Then the combinations
$\varphi\rightarrow\psi$ (``$\varphi$ implies $\psi$,'' also read ``if
$\varphi$ then $\psi$'')\index{implication ($\rightarrow$)} and $\lnot\varphi$
(``not $\varphi$'')\index{negation ($\lnot$)} are also wffs.

The three axioms of propositional calculus\index{axioms of propositional
calculus} are all wffs of the following form:\footnote{A remarkable result of
C.~A.~Meredith\index{Meredith, C. A.} squeezes these three axioms into the
single axiom $((((\varphi\rightarrow \psi)\rightarrow(\neg \chi\rightarrow\neg
\theta))\rightarrow \chi )\rightarrow \tau)\rightarrow((\tau\rightarrow
\varphi)\rightarrow(\theta\rightarrow \varphi))$ \cite{CAMeredith},
which is believed to be the shortest possible.}
\begin{center}
     $\varphi\rightarrow(\psi\rightarrow \varphi)$\\

     $(\varphi\rightarrow (\psi\rightarrow \chi))\rightarrow
((\varphi\rightarrow  \psi)\rightarrow (\varphi\rightarrow \chi))$\\

     $(\neg \varphi\rightarrow \neg\psi)\rightarrow (\psi\rightarrow
\varphi)$
\end{center}

These three axioms are widely used.
They are attributed to Jan {\L}ukasiewicz
(pronounced woo-kah-SHAY-vitch) and was popularized by Alonzo Church,
who called it system P2. (Thanks to Ted Ulrich for this information.)

There are an infinite number of axioms, one for each possible
wff\index{well-formed formula (wff)} of the above form.  (For this reason,
axioms such as the above are often called ``axiom schemes.''\index{axiom
scheme})  Each Greek letter in the axioms may be substituted with a more
complex wff to result in another axiom.  For example, substituting
$\neg(\varphi\rightarrow\chi)$ for $\varphi$ in the first axiom yields
$\neg(\varphi\rightarrow\chi)\rightarrow(\psi\rightarrow
\neg(\varphi\rightarrow\chi))$, which is still an axiom.

To deduce new true statements (theorems\index{theorem}) from the axioms, a
rule\index{rule} called ``modus ponens''\index{modus ponens} is used.  This
rule states that if the wff $\varphi$ is an axiom or a theorem, and the wff
$\varphi\rightarrow\psi$ is an axiom or a theorem, then the wff $\psi$ is also
a theorem\index{theorem}.

As a non-mathematical example of modus ponens, suppose we have proved (or
taken as an axiom) ``Bob is a man'' and separately have proved (or taken as
an axiom) ``If Bob is a man, then Bob is a human.''  Using the rule of modus
ponens, we can logically deduce, ``Bob is a human.''

From Metamath's\index{Metamath} point of view, the axioms and the rule of
modus ponens just define a mechanical means for deducing new true statements
from existing true statements, and that is the complete content of
propositional calculus as far as Metamath is concerned.  You can read a logic
textbook to gain a better understanding of their meaning, or you can just let
their meaning slowly become apparent to you after you use them for a while.

It is actually rather easy to check to see if a formula is a theorem of
propositional calculus.  Theorems of propositional calculus are also called
``tautologies.''\index{tautology}  The technique to check whether a formula is
a tautology is called the ``truth table method,''\index{truth table} and it
works like this.  A wff $\varphi\rightarrow\psi$ is false whenever $\varphi$ is true
and $\psi$ is false.  Otherwise it is true.  A wff $\lnot\varphi$ is false
whenever $\varphi$ is true and false otherwise. To verify a tautology such as
$\varphi\rightarrow(\psi\rightarrow \varphi)$, you break it down into sub-wffs and
construct a truth table that accounts for all possible combinations of true
and false assigned to the wff metavariables:
\begin{center}\begin{tabular}{|c|c|c|c|}\hline
\mbox{$\varphi$} & \mbox{$\psi$} & \mbox{$\psi\rightarrow\varphi$}
    & \mbox{$\varphi\rightarrow(\psi\rightarrow \varphi)$} \\ \hline \hline
              T   &  T    &      T       &        T    \\ \hline
              T   &  F    &      T       &        T    \\ \hline
              F   &  T    &      F       &        T    \\ \hline
              F   &  F    &      T       &        T    \\ \hline
\end{tabular}\end{center}
If all entries in the last column are true, the formula is a tautology.

Now, the truth table method doesn't tell you how to prove the tautology from
the axioms, but only that a proof exists.  Finding an actual proof (especially
one that is short and elegant) can be challenging.  Methods do exist for
automatically generating proofs in propositional calculus, but the proofs that
result can sometimes be very long.  In the Metamath \texttt{set.mm}\index{set
theory database (\texttt{set.mm})} database, most
or all proofs were created manually.

Section \ref{metadefprop} discusses various definitions
that make propositional calculus easier to use.
For example, we define:

\begin{itemize}
\item $\varphi \vee \psi$
  is true if either $\varphi$ or $\psi$ (or both) are true
  (this is disjunction\index{disjunction ($\vee$)}
  aka logical {\sc or}\index{logical {\sc or} ($\vee$)}).

\item $\varphi \wedge \psi$
  is true if both $\varphi$ and $\psi$ are true
  (this is conjunction\index{conjunction ($\wedge$)}
  aka logical {\sc and}\index{logical {\sc and} ($\wedge$)}).

\item $\varphi \leftrightarrow \psi$
  is true if $\varphi$ and $\psi$ have the same value, that is,
  they are both true or both false
  (this is the biconditional\index{biconditional ($\leftrightarrow$)}).
\end{itemize}

\subsection{Predicate Calculus}

Predicate calculus\index{predicate calculus} introduces the concept of
``individual variables,''\index{variable!in predicate calculus}\index{individual
variable} which
we will usually just call ``variables.''
These variables can represent something other than true or false (wffs),
and will always represent sets when we get to set theory.  There are also
three new symbols $\forall$\index{universal quantifier ($\forall$)},
$=$\index{equality ($=$)}, and $\in$\index{stylized epsilon ($\in$)},
read ``for all,'' ``equals,'' and ``is an element of''
respectively.  We will represent variables with the letters $x$, $y$, $z$, and
$w$, as is common practice in the literature.
For example, $\forall x \varphi$ means ``for all possible values of
$x$, $\varphi$ is true.''

In predicate calculus, we extend the definition of a wff\index{well-formed
formula (wff)}.  If $\varphi$ is a wff and $x$ and $y$ are variables, then
$\forall x \, \varphi$, $x=y$, and $x\in y$ are wffs. Note that these three new
types of wffs can be considered ``starting'' wffs from which we can build
other wffs with $\rightarrow$ and $\neg$ .  The concept of a starting wff was
absent in propositional calculus.  But starting wff or not, all we are really
concerned with is whether our wffs are correctly constructed according to
these mechanical rules.

A quick aside:
To prevent confusion, it might be best at this point to think of the variables
of Metamath\index{Metamath} as ``metavariables,''\index{metavariable} because
they are not quite the same as the variables we are introducing here.  A
(meta)variable in Metamath can be a wff or an individual variable, as well
as many other things; in general, it represents a kind of place holder for an
unspecified sequence of math symbols\index{math symbol}.

Unlike propositional calculus, no decision procedure\index{decision procedure}
analogous to the truth table method exists (nor theoretically can exist) that
will definitely determine whether a formula is a theorem of predicate
calculus.  Much of the work in the field of automated theorem
proving\index{automated theorem proving} has been dedicated to coming up with
clever heuristics for proving theorems of predicate calculus, but they can
never be guaranteed to work always.

Section \ref{metadefpred} discusses various definitions
that make predicate calculus easier to use.
For example, we define
$\exists x \varphi$ to mean
``there exists at least one possible value of $x$ where $\varphi$ is true.''

We now turn to looking at how predicate calculus can be formally
represented.

\subsubsection{Common Axioms}

There is a new rule of inference in predicate calculus:  if $\varphi$ is
an axiom or a theorem, then $\forall x \,\varphi$ is also a
theorem\index{theorem}.  This is called the rule of
``generalization.''\index{rule of generalization}
This is easily represented in Metamath.

In standard texts of logic, there are often two axioms of predicate
calculus\index{axioms of predicate calculus}:
\begin{center}
  $\forall x \,\varphi ( x ) \rightarrow \varphi ( y )$,
      where ``$y$ is properly substituted for $x$.''\\
  $\forall x ( \varphi \rightarrow \psi )\rightarrow ( \varphi \rightarrow
    \forall x\, \psi )$,
    where ``$x$ is not free in $\varphi$.''
\end{center}

Now at first glance, this seems simple:  just two axioms.  However,
conditional clauses are attached to each axiom describing requirements that
may seem puzzling to you.  In addition, the first axiom puts a variable symbol
in parentheses after each wff, seemingly violating our definition of a
wff\index{well-formed formula (wff)}; this is just an informal way of
referring to some arbitrary variable that may occur in the wff.  The
conditional clauses do, of course, have a precise meaning, but as it turns out
the precise meaning is somewhat complicated and awkward to formalize in a
way that a computer can handle easily.  Unlike propositional calculus, a
certain amount of mathematical sophistication and practice is needed to be
able to easily grasp and manipulate these concepts correctly.

Predicate calculus may be presented with or without axioms for
equality\index{axioms of equality}\index{equality ($=$)}. We will require the
axioms of equality as a prerequisite for the version of set theory we will
use.  The axioms for equality, when included, are often represented using these
two axioms:
\begin{center}
$x=x$\\ \ \\
$x=y\rightarrow (\varphi(x,x)\rightarrow\varphi(x,y))$ where ``$\varphi(x,y)$
   arises from $\varphi(x,x)$ by replacing some, but not necessarily all,
   free\index{free variable}
   occurrences of $x$ by $y$,\\ provided that $y$ is free for $x$
   in $\varphi(x,x)$.'' \end{center}
% (Mendelson p. 95)
The first equality axiom is simple, but again,
the condition on the second one is
somewhat awkward to implement on a computer.

\subsubsection{Tarski System S2}

Of course, we are not the first to notice the complications of these
predicate calculus axioms when being rigorous.

Well-known logician Alfred Tarski published in 1965
a system he called system S2\cite[p.~77]{Tarski1965}.
Tarski's system is \textit{exactly equivalent} to the traditional textbook
formalization, but (by clever use of equality axioms) it eliminates the
latter's primitive notions of ``proper substitution'' and ``free variable,''
replacing them with direct substitution and the notion of a variable
not occurring in a formula (which we express with distinct variable
constraints).

In advocating his system, Tarski wrote, ``The relatively complicated
character of [free variables and proper substitution] is a source
of certain inconveniences of both practical and theoretical nature;
this is clearly experienced both in teaching an elementary course of
mathematical logic and in formalizing the syntax of predicate logic for
some theoretical purposes''\cite[p.~61]{Tarski1965}\index{Tarski, Alfred}.

\subsubsection{Developing a Metamath Representation}

The standard textbook axioms of predicate calculus are somewhat
cumbersome to implement on a computer because of the complex notions of
``free variable''\index{free variable} and ``proper
substitution.''\index{proper substitution}\index{substitution!proper}
While it is possible to use the Metamath\index{Metamath} language to
implement these concepts, we have chosen not to implement them
as primitive constructs in the
\texttt{set.mm} set theory database.  Instead, we have eliminated them
within the axioms
by carefully crafting the axioms so as to avoid them,
building on Tarski's system S2.  This makes it
easy for a beginner to follow the steps in a proof without knowing any
advanced concepts other than the simple concept of
replacing\index{substitution!variable}\index{variable substitution}
variables with expressions.

In order to develop the concepts of free variable and proper
substitution from the axioms, we use an additional
Metamath statement type called ``disjoint variable
restriction''\index{disjoint variables} that we have not encountered
before.  In the context of the axioms, the statement \texttt{\$d} $ x\,
y$\index{\texttt{\$d} statement} simply means that $x$ and $y$ must be
distinct\index{distinct variables}, i.e.\ they may not be simultaneously
substituted\index{substitution!variable}\index{variable substitution}
with the same variable.  The statement \texttt{\$d} $ x\, \varphi$ means
variable $x$ must not occur in wff $\varphi$.  For the precise
definition of \texttt{\$d}, see Section~\ref{dollard}.

\subsubsection{Metamath representation}

The Metamath axiom system for predicate calculus
defined in set.mm uses Tarski's system S2.
As noted above, this has a different representation
than the traditional textbook formalization,
but it is \textit{exactly equivalent} to the textbook formalization,
and it is \textit{much} easier to work with.
This is reproduced as system S3 in Section 6 of
Megill's formalization \cite{Megill}\index{Megill, Norman}.

There is one exception, Tarski's axiom of existence,
which we label as axiom ax-6.
In the case of ax-6, Tarski's version is weaker because it includes a
distinct variable proviso. If we wish, we can also weaken our version
in this way and still have a metalogically complete system. Theorem
ax6 shows this by deriving, in the presence of the other axioms, our
ax-6 from Tarski's weaker version ax6v. However, we chose the stronger
version for our system because it is simpler to state and easier to use.

Tarski's system was designed for proving specific theorems rather than
more general theorem schemes. However, theorem schemes are much more
efficient than specific theorems for building a body of mathematical
knowledge, since they can be reused with different instances as
needed. While Tarski does derive some theorem schemes from his axioms,
their proofs require concepts that are ``outside'' of the system, such as
induction on formula length. The verification of such proofs is difficult
to automate in a proof verifier. (Specifically, Tarski treats the formulas
of his system as set-theoretical objects. In order to verify the proofs
of his theorem schemes, a proof verifier would need a significant amount
of set theory built into it.)

The Metamath axiom system for predicate calculus extends
Tarski's system to eliminate this difficulty. The additional
``auxilliary'' axiom
schemes (as we will call them in this section; see below) endow Tarski's
system with a nice property we call
metalogical completeness \cite[Remark 9.6]{Megill}\index{Megill, Norman}.
As a result, we can prove any theorem scheme
expressable in the ``simple metalogic'' of Tarski's system by using
only Metamath's direct substitution rule applied to the axiom system
(and no other metalogical or set-theoretical notions ``outside'' of the
system). Simple metalogic consists of schemes containing wff metavariables
(with no arguments) and/or set (also called ``individual'') metavariables,
accompanied by optional provisos each stating that two specified set
metavariables must be distinct or that a specified set metavariable may
not occur in a specified wff metavariable. Metamath's logic and set theory
axiom and rule schemes are all examples of simple metalogic. The schemes
of traditional predicate calculus with equality are examples which are
not simple metalogic, because they use wff metavariables with arguments
and have ``free for'' and ``not free in'' side conditions.

A rigorous justification for this system, using an older but
exactly equivalent set of axioms, can be
found in \cite{Megill}\index{Megill, Norman}.

This allows us to
take a different approach in the Metamath\index{Metamath} database
\texttt{set.mm}\index{set theory database (\texttt{set.mm})}.  We do not
directly use the primitive notions of ``free variable''\index{free variable}
and ``proper substitution''\index{proper
substitution}\index{substitution!proper} at all as primitive constructs.
Instead, we use a set
of axioms that are almost as simple to manipulate as those of
propositional calculus.  Our axiom system avoids complex primitive
notions by effectively embedding the complexity into the axioms
themselves.  As a result, we will end up with a larger number of axioms,
but they are ideally suited for a computer language such as Metamath.
(Section~\ref{metaaxioms} shows these axioms.)

We will not elaborate further
on the ``free variable'' and ``proper substitution''
concepts here.  You may consult
\cite[ch.\ 3--4]{Hamilton}\index{Hamilton, Alan G.} (as well as
many other books) for a precise explanation
of these concepts.  If you intend to do serious mathematical work, it is wise
to become familiar with the traditional textbook approach; even though the
concepts embedded in their axioms require a higher level of sophistication,
they can be more practical to deal with on an everyday, informal basis.  Even
if you are just developing Metamath proofs, familiarity with the traditional
approach can help you arrive at a proof outline much faster, which you can
then convert to the detail required by Metamath.

We do develop proper substitution rules later on, but in set.mm
they are defined as derived constructs; they are not primitives.

You should also note that our system of predicate calculus is specifically
tailored for set theory; thus there are only two specific predicates $=$ and
$\in$ and no functions\index{function!in predicate calculus}
or constants\index{constant!in predicate calculus} unlike more general systems.
We later add these.

\subsection{Set Theory}

Traditional Zermelo--Fraenkel set theory\index{Zermelo--Fraenkel set
theory}\index{set theory} with the Axiom of Choice
has 10 axioms, which can be expressed in the
language of predicate calculus.  In this section, we will list only the
names and brief English descriptions of these axioms, since we will give
you the precise formulas used by the Metamath\index{Metamath} set theory
database \texttt{set.mm} later on.

In the descriptions of the axioms, we assume that $x$, $y$, $z$, $w$, and $v$
represent sets.  These are the same as the variables\index{variable!in set
theory} in our predicate calculus system above, except that now we informally
think of the variables as ranging over sets.  Note that the terms
``object,''\index{object} ``set,''\index{set} ``element,''\index{element}
``collection,''\index{collection} and ``family''\index{family} are synonymous,
as are ``is an element of,'' ``is a member of,''\index{member} ``is contained
in,'' and ``belongs to.''  The different terms are used for convenience; for
example, ``a collection of sets'' is less confusing than ``a set of sets.''
A set $x$ is said to be a ``subset''\index{subset} of $y$ if every element of
$x$ is also an element of $y$; we also say $x$ is ``included in''
$y$.

The axioms are very general and apply to almost any conceivable mathematical
object, and this level of abstraction can be overwhelming at first.  To gain an
intuitive feel, it can be helpful to draw a picture illustrating the concept;
for example, a circle containing dots could represent a collection of sets,
and a smaller circle drawn inside the circle could represent a subset.
Overlapping circles can illustrate intersection and union.  Circles that
illustrate the concepts of set theory are frequently used in elementary
textbooks and are called Venn diagrams\index{Venn diagram}.\index{axioms of
set theory}

1. Axiom of Extensionality:  Two sets are identical if they contain the same
   elements.\index{Axiom of Extensionality}

2. Axiom of Pairing:  The set $\{ x , y \}$ exists.\index{Axiom of Pairing}

3. Axiom of Power Sets:  The power set of a set (the collection of all of
   its subsets) exists.  For example, the power set of $\{x,y\}$ is
   $\{\varnothing,\{x\},\{y\},\{x,y\}\}$ and it exists.\index{Axiom
of Power Sets}

4. Axiom of the Null Set:  The empty set $\varnothing$ exists.\index{Axiom of
the Null Set}

5. Axiom of Union:  The union of a set (the set containing the elements of
   its members) exists.  For example, the union of $\{\{x,y\},\{z\}\}$ is
 $\{x,y,z\}$ and
   it exists.\index{Axiom of Union}

6. Axiom of Regularity:  Roughly, no set can contain itself, nor can there
   be membership ``loops,'' such as a set being an
   element of one of its members.\index{Axiom of Regularity}

7. Axiom of Infinity:  An infinite set exists.  An example of an infinite
   set is the set of all
   integers.\index{Axiom of Infinity}

8. Axiom of Separation:  The set exists that is obtained by restricting $x$
   with some property.  For example, if the set of all integers exists,
   then the set of all even integers exists.\index{Axiom of Separation}

9. Axiom of Replacement:  The range of a function whose domain is restricted
   to the elements of a set $x$, is also a set.  For example, there
   is a function
   from integers (the function's domain) to their squares (its
   range).  If we
   restrict the domain to even integers, its range will become the set of
   squares of even integers, so this axiom asserts that the set of
    squares of even numbers exists.  Technical note:  In general, the
   ``function'' need not be a set but can be a proper class.
   \index{Axiom of Replacement}

10. Axiom of Choice:  Let $x$ be a set whose members are pairwise
  disjoint\index{disjoint sets} (i.e,
  whose members contain no elements in common).  Then there exists another
  set containing one element from each member of $x$.  For
  example, if $x$ is
  $\{\{y,z\},\{w,v\}\}$, where $y$, $z$, $w$, and $v$ are
  different sets, then a set such as $\{z,w\}$
  exists (but the axiom doesn't tell
  us which one).  (Actually the Axiom
  of Choice is redundant if the set $x$, as in this example, has a finite
  number of elements.)\index{Axiom of Choice}

The Axiom of Choice is usually considered an extension of ZF set theory rather
than a proper part of it.  It is sometimes considered philosophically
controversial because it specifies the existence of a set without specifying
what the set is. Constructive logics, including intuitionistic logic,
do not accept the axiom of choice.
Since there is some lingering controversy, we often prefer proofs that do
not use the axiom of choice (where there is a known alternative), and
in some cases we will use weaker axioms than the full axiom of choice.
That said, the axiom of choice is a powerful and widely-accepted tool,
so we do use it when needed.
ZF set theory that includes the Axiom of Choice is
called Zermelo--Fraenkel set theory with choice (ZFC\index{ZFC set theory}).

When expressed symbolically, the Axiom of Separation and the Axiom of
Replacement contain wff symbols and therefore each represent infinitely many
axioms, one for each possible wff. For this reason, they are often called
axiom schemes\index{axiom scheme}\index{well-formed formula (wff)}.

It turns out that the Axiom of the Null Set, the Axiom of Pairing, and the
Axiom of Separation can be derived from the other axioms and are therefore
unnecessary, although they tend to be included in standard texts for various
reasons (historical, philosophical, and possibly because some authors may not
know this).  In the Metamath\index{Metamath} set theory database, these
redundant axioms are derived from the other ones instead of truly
being considered axioms.
This is in keeping with our general goal of minimizing the number of
axioms we must depend on.

\subsection{Other Axioms}

Above we qualified the phrase "all of mathematics" with "essentially."
The main important missing piece is the ability to do category theory,
which requires huge sets (inaccessible cardinals) larger than those
postulated by the ZFC axioms. The Tarski--Grothendieck Axiom postulates
the existence of such sets.
Note that this is the same axiom used by Mizar for supporting
category theory.
The Tarski--Grothendieck axiom
can be viewed as a very strong replacement of the Axiom of Infinity,
the Axiom of Choice, and the Axiom of Power Sets.
The \texttt{set.mm} database includes this axiom; see the database
for details about it.
Again, we only use this axiom when we need to.
You are only likely to encounter or use this axiom if you are doing
category theory, since its use is highly specialized,
so we will not list the Tarsky-Grothendieck axiom
in the short list of axioms below.

Can there be even more axioms?
Of course.
G\"{o}del showed that no finite set of axioms or axiom schemes can completely
describe any consistent theory strong enough to include arithmetic.
But practically speaking, the ones above are the accepted foundation that
almost all mathematicians explicitly or implicitly base their work on.

\section{The Axioms in the Metamath Language}\label{metaaxioms}

Here we list the axioms as they appear in
\texttt{set.mm}\index{set theory database (\texttt{set.mm})} so you can
look them up there easily.  Incidentally, the \texttt{show statement
/tex} command\index{\texttt{show statement} command} was used to
typeset them.

%macros from show statement /tex
\newbox\mlinebox
\newbox\mtrialbox
\newbox\startprefix  % Prefix for first line of a formula
\newbox\contprefix  % Prefix for continuation line of a formula
\def\startm{  % Initialize formula line
  \setbox\mlinebox=\hbox{\unhcopy\startprefix}
}
\def\m#1{  % Add a symbol to the formula
  \setbox\mtrialbox=\hbox{\unhcopy\mlinebox $\,#1$}
  \ifdim\wd\mtrialbox>\hsize
    \box\mlinebox
    \setbox\mlinebox=\hbox{\unhcopy\contprefix $\,#1$}
  \else
    \setbox\mlinebox=\hbox{\unhbox\mtrialbox}
  \fi
}
\def\endm{  % Output the last line of a formula
  \box\mlinebox
}

% \SLASH for \ , \TOR for \/ (text OR), \TAND for /\ (text and)
% This embeds a following forced space to force the space.
\newcommand\SLASH{\char`\\~}
\newcommand\TOR{\char`\\/~}
\newcommand\TAND{/\char`\\~}
%
% Macro to output metamath raw text.
% This assumes \startprefix and \contprefix are set.
% NOTE: "\" is tricky to escape, use \SLASH, \TOR, and \TAND inside.
% Any use of "$ { ~ ^" must be escaped; ~ and ^ must be escaped specially.
% We escape { and } for consistency.
% For more about how this macro written, see:
% https://stackoverflow.com/questions/4073674/
% how-to-disable-indentation-in-particular-section-in-latex/4075706
% Use frenchspacing, or "e." will get an extra space after it.
\newlength\mystoreparindent
\newlength\mystorehangindent
\newenvironment{mmraw}{%
\setlength{\mystoreparindent}{\the\parindent}
\setlength{\mystorehangindent}{\the\hangindent}
\setlength{\parindent}{0pt} % TODO - we'll put in the \startprefix instead
\setlength{\hangindent}{\wd\the\contprefix}
\begin{flushleft}
\begin{frenchspacing}
\begin{tt}
{\unhcopy\startprefix}%
}{%
\end{tt}
\end{frenchspacing}
\end{flushleft}
\setlength{\parindent}{\mystoreparindent}
\setlength{\hangindent}{\mystorehangindent}
\vskip 1ex
}

\needspace{5\baselineskip}
\subsection{Propositional Calculus}\label{propcalc}\index{axioms of
propositional calculus}

\needspace{2\baselineskip}
Axiom of Simplification.\label{ax1}

\setbox\startprefix=\hbox{\tt \ \ ax-1\ \$a\ }
\setbox\contprefix=\hbox{\tt \ \ \ \ \ \ \ \ \ \ }
\startm
\m{\vdash}\m{(}\m{\varphi}\m{\rightarrow}\m{(}\m{\psi}\m{\rightarrow}\m{\varphi}\m{)}
\m{)}
\endm

\needspace{3\baselineskip}
\noindent Axiom of Distribution.

\setbox\startprefix=\hbox{\tt \ \ ax-2\ \$a\ }
\setbox\contprefix=\hbox{\tt \ \ \ \ \ \ \ \ \ \ }
\startm
\m{\vdash}\m{(}\m{(}\m{\varphi}\m{\rightarrow}\m{(}\m{\psi}\m{\rightarrow}\m{\chi}
\m{)}\m{)}\m{\rightarrow}\m{(}\m{(}\m{\varphi}\m{\rightarrow}\m{\psi}\m{)}\m{
\rightarrow}\m{(}\m{\varphi}\m{\rightarrow}\m{\chi}\m{)}\m{)}\m{)}
\endm

\needspace{2\baselineskip}
\noindent Axiom of Contraposition.

\setbox\startprefix=\hbox{\tt \ \ ax-3\ \$a\ }
\setbox\contprefix=\hbox{\tt \ \ \ \ \ \ \ \ \ \ }
\startm
\m{\vdash}\m{(}\m{(}\m{\lnot}\m{\varphi}\m{\rightarrow}\m{\lnot}\m{\psi}\m{)}\m{
\rightarrow}\m{(}\m{\psi}\m{\rightarrow}\m{\varphi}\m{)}\m{)}
\endm


\needspace{4\baselineskip}
\noindent Rule of Modus Ponens.\label{axmp}\index{modus ponens}

\setbox\startprefix=\hbox{\tt \ \ maj\ \$e\ }
\setbox\contprefix=\hbox{\tt \ \ \ \ \ \ \ \ \ }
\startm
\m{\vdash}\m{(}\m{\varphi}\m{\rightarrow}\m{\psi}\m{)}
\endm

\setbox\startprefix=\hbox{\tt \ \ min\ \$e\ }
\setbox\contprefix=\hbox{\tt \ \ \ \ \ \ \ \ \ }
\startm
\m{\vdash}\m{\varphi}
\endm

\setbox\startprefix=\hbox{\tt \ \ ax-mp\ \$a\ }
\setbox\contprefix=\hbox{\tt \ \ \ \ \ \ \ \ \ \ \ }
\startm
\m{\vdash}\m{\psi}
\endm


\needspace{7\baselineskip}
\subsection{Axioms of Predicate Calculus with Equality---Tarski's S2}\index{axioms of predicate calculus}

\needspace{3\baselineskip}
\noindent Rule of Generalization.\index{rule of generalization}

\setbox\startprefix=\hbox{\tt \ \ ax-g.1\ \$e\ }
\setbox\contprefix=\hbox{\tt \ \ \ \ \ \ \ \ \ \ \ \ }
\startm
\m{\vdash}\m{\varphi}
\endm

\setbox\startprefix=\hbox{\tt \ \ ax-gen\ \$a\ }
\setbox\contprefix=\hbox{\tt \ \ \ \ \ \ \ \ \ \ \ \ }
\startm
\m{\vdash}\m{\forall}\m{x}\m{\varphi}
\endm

\needspace{2\baselineskip}
\noindent Axiom of Quantified Implication.

\setbox\startprefix=\hbox{\tt \ \ ax-4\ \$a\ }
\setbox\contprefix=\hbox{\tt \ \ \ \ \ \ \ \ \ \ }
\startm
\m{\vdash}\m{(}\m{\forall}\m{x}\m{(}\m{\forall}\m{x}\m{\varphi}\m{\rightarrow}\m{
\psi}\m{)}\m{\rightarrow}\m{(}\m{\forall}\m{x}\m{\varphi}\m{\rightarrow}\m{
\forall}\m{x}\m{\psi}\m{)}\m{)}
\endm

\needspace{3\baselineskip}
\noindent Axiom of Distinctness.

% Aka: Add $d x ph $.
\setbox\startprefix=\hbox{\tt \ \ ax-5\ \$a\ }
\setbox\contprefix=\hbox{\tt \ \ \ \ \ \ \ \ \ \ }
\startm
\m{\vdash}\m{(}\m{\varphi}\m{\rightarrow}\m{\forall}\m{x}\m{\varphi}\m{)}\m{where}\m{ }\m{\$d}\m{ }\m{x}\m{ }\m{\varphi}\m{ }\m{(}\m{x}\m{ }\m{does}\m{ }\m{not}\m{ }\m{occur}\m{ }\m{in}\m{ }\m{\varphi}\m{)}
\endm

\needspace{2\baselineskip}
\noindent Axiom of Existence.

\setbox\startprefix=\hbox{\tt \ \ ax-6\ \$a\ }
\setbox\contprefix=\hbox{\tt \ \ \ \ \ \ \ \ \ \ }
\startm
\m{\vdash}\m{(}\m{\forall}\m{x}\m{(}\m{x}\m{=}\m{y}\m{\rightarrow}\m{\forall}
\m{x}\m{\varphi}\m{)}\m{\rightarrow}\m{\varphi}\m{)}
\endm

\needspace{2\baselineskip}
\noindent Axiom of Equality.

\setbox\startprefix=\hbox{\tt \ \ ax-7\ \$a\ }
\setbox\contprefix=\hbox{\tt \ \ \ \ \ \ \ \ \ \ }
\startm
\m{\vdash}\m{(}\m{x}\m{=}\m{y}\m{\rightarrow}\m{(}\m{x}\m{=}\m{z}\m{
\rightarrow}\m{y}\m{=}\m{z}\m{)}\m{)}
\endm

\needspace{2\baselineskip}
\noindent Axiom of Left Equality for Binary Predicate.

\setbox\startprefix=\hbox{\tt \ \ ax-8\ \$a\ }
\setbox\contprefix=\hbox{\tt \ \ \ \ \ \ \ \ \ \ \ }
\startm
\m{\vdash}\m{(}\m{x}\m{=}\m{y}\m{\rightarrow}\m{(}\m{x}\m{\in}\m{z}\m{
\rightarrow}\m{y}\m{\in}\m{z}\m{)}\m{)}
\endm

\needspace{2\baselineskip}
\noindent Axiom of Right Equality for Binary Predicate.

\setbox\startprefix=\hbox{\tt \ \ ax-9\ \$a\ }
\setbox\contprefix=\hbox{\tt \ \ \ \ \ \ \ \ \ \ \ }
\startm
\m{\vdash}\m{(}\m{x}\m{=}\m{y}\m{\rightarrow}\m{(}\m{z}\m{\in}\m{x}\m{
\rightarrow}\m{z}\m{\in}\m{y}\m{)}\m{)}
\endm


\needspace{4\baselineskip}
\subsection{Axioms of Predicate Calculus with Equality---Auxiliary}\index{axioms of predicate calculus - auxiliary}

\needspace{2\baselineskip}
\noindent Axiom of Quantified Negation.

\setbox\startprefix=\hbox{\tt \ \ ax-10\ \$a\ }
\setbox\contprefix=\hbox{\tt \ \ \ \ \ \ \ \ \ \ }
\startm
\m{\vdash}\m{(}\m{\lnot}\m{\forall}\m{x}\m{\lnot}\m{\forall}\m{x}\m{\varphi}\m{
\rightarrow}\m{\varphi}\m{)}
\endm

\needspace{2\baselineskip}
\noindent Axiom of Quantifier Commutation.

\setbox\startprefix=\hbox{\tt \ \ ax-11\ \$a\ }
\setbox\contprefix=\hbox{\tt \ \ \ \ \ \ \ \ \ \ }
\startm
\m{\vdash}\m{(}\m{\forall}\m{x}\m{\forall}\m{y}\m{\varphi}\m{\rightarrow}\m{
\forall}\m{y}\m{\forall}\m{x}\m{\varphi}\m{)}
\endm

\needspace{3\baselineskip}
\noindent Axiom of Substitution.

\setbox\startprefix=\hbox{\tt \ \ ax-12\ \$a\ }
\setbox\contprefix=\hbox{\tt \ \ \ \ \ \ \ \ \ \ \ }
\startm
\m{\vdash}\m{(}\m{\lnot}\m{\forall}\m{x}\m{\,x}\m{=}\m{y}\m{\rightarrow}\m{(}
\m{x}\m{=}\m{y}\m{\rightarrow}\m{(}\m{\varphi}\m{\rightarrow}\m{\forall}\m{x}\m{(}
\m{x}\m{=}\m{y}\m{\rightarrow}\m{\varphi}\m{)}\m{)}\m{)}\m{)}
\endm

\needspace{3\baselineskip}
\noindent Axiom of Quantified Equality.

\setbox\startprefix=\hbox{\tt \ \ ax-13\ \$a\ }
\setbox\contprefix=\hbox{\tt \ \ \ \ \ \ \ \ \ \ \ }
\startm
\m{\vdash}\m{(}\m{\lnot}\m{\forall}\m{z}\m{\,z}\m{=}\m{x}\m{\rightarrow}\m{(}
\m{\lnot}\m{\forall}\m{z}\m{\,z}\m{=}\m{y}\m{\rightarrow}\m{(}\m{x}\m{=}\m{y}
\m{\rightarrow}\m{\forall}\m{z}\m{\,x}\m{=}\m{y}\m{)}\m{)}\m{)}
\endm

% \noindent Axiom of Quantifier Substitution
%
% \setbox\startprefix=\hbox{\tt \ \ ax-c11n\ \$a\ }
% \setbox\contprefix=\hbox{\tt \ \ \ \ \ \ \ \ \ \ \ }
% \startm
% \m{\vdash}\m{(}\m{\forall}\m{x}\m{\,x}\m{=}\m{y}\m{\rightarrow}\m{(}\m{\forall}
% \m{x}\m{\varphi}\m{\rightarrow}\m{\forall}\m{y}\m{\varphi}\m{)}\m{)}
% \endm
%
% \noindent Axiom of Distinct Variables. (This axiom requires
% that two individual variables
% be distinct\index{\texttt{\$d} statement}\index{distinct
% variables}.)
%
% \setbox\startprefix=\hbox{\tt \ \ \ \ \ \ \ \ \$d\ }
% \setbox\contprefix=\hbox{\tt \ \ \ \ \ \ \ \ \ \ \ }
% \startm
% \m{x}\m{\,}\m{y}
% \endm
%
% \setbox\startprefix=\hbox{\tt \ \ ax-c16\ \$a\ }
% \setbox\contprefix=\hbox{\tt \ \ \ \ \ \ \ \ \ \ \ }
% \startm
% \m{\vdash}\m{(}\m{\forall}\m{x}\m{\,x}\m{=}\m{y}\m{\rightarrow}\m{(}\m{\varphi}\m{
% \rightarrow}\m{\forall}\m{x}\m{\varphi}\m{)}\m{)}
% \endm

% \noindent Axiom of Quantifier Introduction (2).  (This axiom requires
% that the individual variable not occur in the
% wff\index{\texttt{\$d} statement}\index{distinct variables}.)
%
% \setbox\startprefix=\hbox{\tt \ \ \ \ \ \ \ \ \$d\ }
% \setbox\contprefix=\hbox{\tt \ \ \ \ \ \ \ \ \ \ \ }
% \startm
% \m{x}\m{\,}\m{\varphi}
% \endm
% \setbox\startprefix=\hbox{\tt \ \ ax-5\ \$a\ }
% \setbox\contprefix=\hbox{\tt \ \ \ \ \ \ \ \ \ \ \ }
% \startm
% \m{\vdash}\m{(}\m{\varphi}\m{\rightarrow}\m{\forall}\m{x}\m{\varphi}\m{)}
% \endm

\subsection{Set Theory}\label{mmsettheoryaxioms}

In order to make the axioms of set theory\index{axioms of set theory} a little
more compact, there are several definitions from logic that we make use of
implicitly, namely, ``logical {\sc and},''\index{conjunction ($\wedge$)}
\index{logical {\sc and} ($\wedge$)} ``logical equivalence,''\index{logical
equivalence ($\leftrightarrow$)}\index{biconditional ($\leftrightarrow$)} and
``there exists.''\index{existential quantifier ($\exists$)}

\begin{center}\begin{tabular}{rcl}
  $( \varphi \wedge \psi )$ &\mbox{stands for}& $\neg ( \varphi
     \rightarrow \neg \psi )$\\
  $( \varphi \leftrightarrow \psi )$& \mbox{stands
     for}& $( ( \varphi \rightarrow \psi ) \wedge
     ( \psi \rightarrow \varphi ) )$\\
  $\exists x \,\varphi$ &\mbox{stands for}& $\neg \forall x \neg \varphi$
\end{tabular}\end{center}

In addition, the axioms of set theory require that all variables be
dis\-tinct,\index{distinct variables}\footnote{Set theory axioms can be
devised so that {\em no} variables are required to be distinct,
provided we replace \texttt{ax-c16} with an axiom stating that ``at
least two things exist,'' thus
making \texttt{ax-5} the only other axiom requiring the
\texttt{\$d} statement.  These axioms are unconventional and are not
presented here, but they can be found on the \url{http://metamath.org}
web site.  See also the Comment on
p.~\pageref{nodd}.}\index{\texttt{\$d} statement} thus we also assume:
\begin{center}
  \texttt{\$d }$x\,y\,z\,w$
\end{center}

\needspace{2\baselineskip}
\noindent Axiom of Extensionality.\index{Axiom of Extensionality}

\setbox\startprefix=\hbox{\tt \ \ ax-ext\ \$a\ }
\setbox\contprefix=\hbox{\tt \ \ \ \ \ \ \ \ \ \ \ \ }
\startm
\m{\vdash}\m{(}\m{\forall}\m{x}\m{(}\m{x}\m{\in}\m{y}\m{\leftrightarrow}\m{x}
\m{\in}\m{z}\m{)}\m{\rightarrow}\m{y}\m{=}\m{z}\m{)}
\endm

\needspace{3\baselineskip}
\noindent Axiom of Replacement.\index{Axiom of Replacement}

\setbox\startprefix=\hbox{\tt \ \ ax-rep\ \$a\ }
\setbox\contprefix=\hbox{\tt \ \ \ \ \ \ \ \ \ \ \ \ }
\startm
\m{\vdash}\m{(}\m{\forall}\m{w}\m{\exists}\m{y}\m{\forall}\m{z}\m{(}\m{%
\forall}\m{y}\m{\varphi}\m{\rightarrow}\m{z}\m{=}\m{y}\m{)}\m{\rightarrow}\m{%
\exists}\m{y}\m{\forall}\m{z}\m{(}\m{z}\m{\in}\m{y}\m{\leftrightarrow}\m{%
\exists}\m{w}\m{(}\m{w}\m{\in}\m{x}\m{\wedge}\m{\forall}\m{y}\m{\varphi}\m{)}%
\m{)}\m{)}
\endm

\needspace{2\baselineskip}
\noindent Axiom of Union.\index{Axiom of Union}

\setbox\startprefix=\hbox{\tt \ \ ax-un\ \$a\ }
\setbox\contprefix=\hbox{\tt \ \ \ \ \ \ \ \ \ \ \ }
\startm
\m{\vdash}\m{\exists}\m{x}\m{\forall}\m{y}\m{(}\m{\exists}\m{x}\m{(}\m{y}\m{
\in}\m{x}\m{\wedge}\m{x}\m{\in}\m{z}\m{)}\m{\rightarrow}\m{y}\m{\in}\m{x}\m{)}
\endm

\needspace{2\baselineskip}
\noindent Axiom of Power Sets.\index{Axiom of Power Sets}

\setbox\startprefix=\hbox{\tt \ \ ax-pow\ \$a\ }
\setbox\contprefix=\hbox{\tt \ \ \ \ \ \ \ \ \ \ \ \ }
\startm
\m{\vdash}\m{\exists}\m{x}\m{\forall}\m{y}\m{(}\m{\forall}\m{x}\m{(}\m{x}\m{
\in}\m{y}\m{\rightarrow}\m{x}\m{\in}\m{z}\m{)}\m{\rightarrow}\m{y}\m{\in}\m{x}
\m{)}
\endm

\needspace{3\baselineskip}
\noindent Axiom of Regularity.\index{Axiom of Regularity}

\setbox\startprefix=\hbox{\tt \ \ ax-reg\ \$a\ }
\setbox\contprefix=\hbox{\tt \ \ \ \ \ \ \ \ \ \ \ \ }
\startm
\m{\vdash}\m{(}\m{\exists}\m{x}\m{\,x}\m{\in}\m{y}\m{\rightarrow}\m{\exists}
\m{x}\m{(}\m{x}\m{\in}\m{y}\m{\wedge}\m{\forall}\m{z}\m{(}\m{z}\m{\in}\m{x}\m{
\rightarrow}\m{\lnot}\m{z}\m{\in}\m{y}\m{)}\m{)}\m{)}
\endm

\needspace{3\baselineskip}
\noindent Axiom of Infinity.\index{Axiom of Infinity}

\setbox\startprefix=\hbox{\tt \ \ ax-inf\ \$a\ }
\setbox\contprefix=\hbox{\tt \ \ \ \ \ \ \ \ \ \ \ \ \ \ \ }
\startm
\m{\vdash}\m{\exists}\m{x}\m{(}\m{y}\m{\in}\m{x}\m{\wedge}\m{\forall}\m{y}%
\m{(}\m{y}\m{\in}\m{x}\m{\rightarrow}\m{\exists}\m{z}\m{(}\m{y}\m{\in}\m{z}\m{%
\wedge}\m{z}\m{\in}\m{x}\m{)}\m{)}\m{)}
\endm

\needspace{4\baselineskip}
\noindent Axiom of Choice.\index{Axiom of Choice}

\setbox\startprefix=\hbox{\tt \ \ ax-ac\ \$a\ }
\setbox\contprefix=\hbox{\tt \ \ \ \ \ \ \ \ \ \ \ \ \ \ }
\startm
\m{\vdash}\m{\exists}\m{x}\m{\forall}\m{y}\m{\forall}\m{z}\m{(}\m{(}\m{y}\m{%
\in}\m{z}\m{\wedge}\m{z}\m{\in}\m{w}\m{)}\m{\rightarrow}\m{\exists}\m{w}\m{%
\forall}\m{y}\m{(}\m{\exists}\m{w}\m{(}\m{(}\m{y}\m{\in}\m{z}\m{\wedge}\m{z}%
\m{\in}\m{w}\m{)}\m{\wedge}\m{(}\m{y}\m{\in}\m{w}\m{\wedge}\m{w}\m{\in}\m{x}%
\m{)}\m{)}\m{\leftrightarrow}\m{y}\m{=}\m{w}\m{)}\m{)}
\endm

\subsection{That's It}

There you have it, the axioms for (essentially) all of mathematics!
Wonder at them and stare at them in awe.  Put a copy in your wallet, and
you will carry in your pocket the encoding for all theorems ever proved
and that ever will be proved, from the most mundane to the most
profound.

\section{A Hierarchy of Definitions}\label{hierarchy}

The axioms in the previous section in principle embody everything that can be
done within standard mathematics.  However, it is impractical to accomplish
very much by using them directly, for even simple concepts (from a human
perspective) can involve extremely long, incomprehensible formulas.
Mathematics is made practical by introducing definitions\index{definition}.
Definitions usually introduce new symbols, or at least new relationships among
existing symbols, to abbreviate more complex formulas.  An important
requirement for a definition is that there exist a straightforward
(algorithmic) method for eliminating the abbreviation by expanding it into the
more primitive symbol string that it represents.  Some
important definitions included in
the file \texttt{set.mm} are listed in this section for reference, and also to
give you a feel for why something like $\omega$\index{omega ($\omega$)} (the
set of natural numbers\index{natural number} 0, 1, 2,\ldots) becomes very
complicated when completely expanded into primitive symbols.

What is the motivation for definitions, aside from allowing complicated
expressions to be expressed more simply?  In the case of  $\omega$, one goal is
to provide a basis for the theory of natural numbers.\index{natural number}
Before set theory was invented, a set of axioms for arithmetic, called Peano's
postulates\index{Peano's postulates}, was devised and shown to have the
properties one expects for natural numbers.  Now anyone can postulate a
set of axioms, but if the axioms are inconsistent contradictions can be derived
from them.  Once a contradiction is derived, anything can be trivially
proved, including
all the facts of arithmetic and their negations.  To ensure that an
axiom system is at least as reliable as the axioms for set theory, we can
define sets and operations on those sets that satisfy the new axioms. In the
\texttt{set.mm} Metamath database, we prove that the elements of $\omega$ satisfy
Peano's postulates, and it's a long and hard journey to get there directly
from the axioms of set theory.  But the result is confidence in the
foundations of arithmetic.  And there is another advantage:  we now have all
the tools of set theory at our disposal for manipulating objects that obey the
axioms for arithmetic.

What are the criteria we use for definitions?  First, and of utmost importance,
the definition should not be {\em creative}\index{creative
definition}\index{definition!creative}, that
is it should not allow an expression that previously qualified as a wff but
was not provable, to become provable.   Second, the definition should be {\em
eliminable}\index{definition!eliminability}, that is, there should exist an
algorithmic method for converting any expression using the definition into
a logically equivalent expression that previously qualified as a wff.

In almost all cases below, definitions connect two expressions with either
$\leftrightarrow$ or $=$.  Eliminating\footnote{Here we mean the
elimination that a human might do in his or her head.  To eliminate them as
part of a Metamath proof we would invoke one of a number of
theorems that deal with transitivity of equivalence or equality; there are
many such examples in the proofs in \texttt{set.mm}.} such a definition is a
simple matter of substituting the expression on the left-hand side ({\em
definiendum}\index{definiendum} or thing being defined) with the equivalent,
more primitive expression on the right-hand side ({\em
definiens}\index{definiens} or definition).

Often a definition has variables on the right-hand side which do not appear on
the left-hand side; these are called {\em dummy variables}.\index{dummy
variable!in definitions}  In this case, any
allowable substitution (such as a new, distinct
variable) can be used when the definition is eliminated.  Dummy variables may
be used only if they are {\em effectively bound}\index{effectively bound
variable}, meaning that the definition will remain logically equivalent upon
any substitution of a dummy variable with any other {\em qualifying
expression}\index{qualifying expression}, i.e.\ any symbol string (such as
another variable) that
meets the restrictions on the dummy variable imposed by \texttt{\$d} and
\texttt{\$f} statements.  For example, we could define a constant $\perp$
(inverted tee, meaning logical ``false'') as $( \varphi \wedge \lnot \varphi
)$, i.e.\ ``phi and not phi.''  Here $\varphi$ is effectively bound because the
definition remains logically equivalent when we replace $\varphi$ with any
other wff.  (It is actually \texttt{df-fal}
in \texttt{set.mm}, which defines $\perp$.)

There are two cases where eliminating definitions is a little more
complex.  These cases are the definitions \texttt{df-bi} and
\texttt{df-cleq}.  The first stretches the concept of a definition a
little, as in effect it ``defines a definition;'' however, it meets our
requirements for a definition in that it is eliminable and does not
strengthen the language.  Theorem \texttt{bii} shows the substitution
needed to eliminate the $\leftrightarrow$\index{logical equivalence
($\leftrightarrow$)}\index{biconditional ($\leftrightarrow$)} symbol.

Definition \texttt{df-cleq}\index{equality ($=$)} extends the usage of
the equality symbol to include ``classes''\index{class} in set theory.  The
reason it is potentially problematic is that it can lead to statements which
do not follow from logic alone but presuppose the Axiom of
Extensionality\index{Axiom of Extensionality}, so we include this axiom
as a hypothesis for the definition.  We could have made \texttt{df-cleq} directly
eliminable by introducing a new equality symbol, but have chosen not to do so
in keeping with standard textbook practice.  Definitions such as \texttt{df-cleq}
that extend the meaning of existing symbols must be introduced carefully so
that they do not lead to contradictions.  Definition \texttt{df-clel} also
extends the meaning of an existing symbol ($\in$); while it doesn't strengthen
the language like \texttt{df-cleq}, this is not obvious and it must also be
subject to the same scrutiny.

Exercise:  Study how the wff $x\in\omega$, meaning ``$x$ is a natural
number,'' could be expanded in terms of primitive symbols, starting with the
definitions \texttt{df-clel} on p.~\pageref{dfclel} and \texttt{df-om} on
p.~\pageref{dfom} and working your way back.  Don't bother to work out the
details; just make sure that you understand how you could do it in principle.
The answer is shown in the footnote on p.~\pageref{expandom}.  If you
actually do work it out, you won't get exactly the same answer because we used
a few simplifications such as discarding occurrences of $\lnot\lnot$ (double
negation).

In the definitions below, we have placed the {\sc ascii} Metamath source
below each of the formulas to help you become familiar with the
notation in the database.  For simplicity, the necessary \texttt{\$f}
and \texttt{\$d} statements are not shown.  If you are in doubt, use the
\texttt{show statement}\index{\texttt{show statement} command} command
in the Metamath program to see the full statement.
A selection of this notation is summarized in Appendix~\ref{ASCII}.

To understand the motivation for these definitions, you should consult the
references indicated:  Takeuti and Zaring \cite{Takeuti}\index{Takeuti, Gaisi},
Quine \cite{Quine}\index{Quine, Willard Van Orman}, Bell and Machover
\cite{Bell}\index{Bell, J. L.}, and Enderton \cite{Enderton}\index{Enderton,
Herbert B.}.  Our list of definitions is provided more for reference than as a
learning aid.  However, by looking at a few of them you can gain a feel for
how the hierarchy is built up.  The definitions are a representative sample of
the 130 or so in \texttt{set.mm}, but they are complete with respect to the
theorem examples we will present in Section~\ref{sometheorems}.  Also, some are
slightly different from, but logically equivalent to, the ones in \texttt{set.mm}
(some of which have been revised over time to shorten them, for example).

\subsection{Definitions for Propositional Calculus}\label{metadefprop}

The symbols $\varphi$, $\psi$, and $\chi$ represent wffs.

Our first definition introduces the biconditional
connective\footnote{The term ``connective'' is informally used to mean a
symbol that is placed between two variables or adjacent to a variable,
whereas a mathematical ``constant'' usually indicates a symbol such as
the number 0 that may replace a variable or metavariable.  From
Metamath's point of view, there is no distinction between a connective
and a constant; both are constants in the Metamath
language.}\index{connective}\index{constant} (also called logical
equivalence)\index{logical equivalence
($\leftrightarrow$)}\index{biconditional ($\leftrightarrow$)}.  Unlike
most traditional developments, we have chosen not to have a separate
symbol such as ``Df.'' to mean ``is defined as.''  Instead, we will use
the biconditional connective for this purpose, as it lets us use
logic to manipulate definitions directly.  Here we state the properties
of the biconditional connective with a carefully crafted \texttt{\$a}
statement, which effectively uses the biconditional connective to define
itself.  The $\leftrightarrow$ symbol can be eliminated from a formula
using theorem \texttt{bii}, which is derived later.

\vskip 2ex
\noindent Define the biconditional connective.\label{df-bi}

\vskip 0.5ex
\setbox\startprefix=\hbox{\tt \ \ df-bi\ \$a\ }
\setbox\contprefix=\hbox{\tt \ \ \ \ \ \ \ \ \ \ \ }
\startm
\m{\vdash}\m{\lnot}\m{(}\m{(}\m{(}\m{\varphi}\m{\leftrightarrow}\m{\psi}\m{)}%
\m{\rightarrow}\m{\lnot}\m{(}\m{(}\m{\varphi}\m{\rightarrow}\m{\psi}\m{)}\m{%
\rightarrow}\m{\lnot}\m{(}\m{\psi}\m{\rightarrow}\m{\varphi}\m{)}\m{)}\m{)}\m{%
\rightarrow}\m{\lnot}\m{(}\m{\lnot}\m{(}\m{(}\m{\varphi}\m{\rightarrow}\m{%
\psi}\m{)}\m{\rightarrow}\m{\lnot}\m{(}\m{\psi}\m{\rightarrow}\m{\varphi}\m{)}%
\m{)}\m{\rightarrow}\m{(}\m{\varphi}\m{\leftrightarrow}\m{\psi}\m{)}\m{)}\m{)}
\endm
\begin{mmraw}%
|- -. ( ( ( ph <-> ps ) -> -. ( ( ph -> ps ) ->
-. ( ps -> ph ) ) ) -> -. ( -. ( ( ph -> ps ) -> -. (
ps -> ph ) ) -> ( ph <-> ps ) ) ) \$.
\end{mmraw}

\noindent This theorem relates the biconditional connective to primitive
connectives and can be used to eliminate the $\leftrightarrow$ symbol from any
wff.

\vskip 0.5ex
\setbox\startprefix=\hbox{\tt \ \ bii\ \$p\ }
\setbox\contprefix=\hbox{\tt \ \ \ \ \ \ \ \ \ }
\startm
\m{\vdash}\m{(}\m{(}\m{\varphi}\m{\leftrightarrow}\m{\psi}\m{)}\m{\leftrightarrow}
\m{\lnot}\m{(}\m{(}\m{\varphi}\m{\rightarrow}\m{\psi}\m{)}\m{\rightarrow}\m{\lnot}
\m{(}\m{\psi}\m{\rightarrow}\m{\varphi}\m{)}\m{)}\m{)}
\endm
\begin{mmraw}%
|- ( ( ph <-> ps ) <-> -. ( ( ph -> ps ) -> -. ( ps -> ph ) ) ) \$= ... \$.
\end{mmraw}

\noindent Define disjunction ({\sc or}).\index{disjunction ($\vee$)}%
\index{logical or (vee)@logical {\sc or} ($\vee$)}%
\index{df-or@\texttt{df-or}}\label{df-or}

\vskip 0.5ex
\setbox\startprefix=\hbox{\tt \ \ df-or\ \$a\ }
\setbox\contprefix=\hbox{\tt \ \ \ \ \ \ \ \ \ \ \ }
\startm
\m{\vdash}\m{(}\m{(}\m{\varphi}\m{\vee}\m{\psi}\m{)}\m{\leftrightarrow}\m{(}\m{
\lnot}\m{\varphi}\m{\rightarrow}\m{\psi}\m{)}\m{)}
\endm
\begin{mmraw}%
|- ( ( ph \TOR ps ) <-> ( -. ph -> ps ) ) \$.
\end{mmraw}

\noindent Define conjunction ({\sc and}).\index{conjunction ($\wedge$)}%
\index{logical {\sc and} ($\wedge$)}%
\index{df-an@\texttt{df-an}}\label{df-an}

\vskip 0.5ex
\setbox\startprefix=\hbox{\tt \ \ df-an\ \$a\ }
\setbox\contprefix=\hbox{\tt \ \ \ \ \ \ \ \ \ \ \ }
\startm
\m{\vdash}\m{(}\m{(}\m{\varphi}\m{\wedge}\m{\psi}\m{)}\m{\leftrightarrow}\m{\lnot}
\m{(}\m{\varphi}\m{\rightarrow}\m{\lnot}\m{\psi}\m{)}\m{)}
\endm
\begin{mmraw}%
|- ( ( ph \TAND ps ) <-> -. ( ph -> -. ps ) ) \$.
\end{mmraw}

\noindent Define disjunction ({\sc or}) of 3 wffs.%
\index{df-3or@\texttt{df-3or}}\label{df-3or}

\vskip 0.5ex
\setbox\startprefix=\hbox{\tt \ \ df-3or\ \$a\ }
\setbox\contprefix=\hbox{\tt \ \ \ \ \ \ \ \ \ \ \ \ }
\startm
\m{\vdash}\m{(}\m{(}\m{\varphi}\m{\vee}\m{\psi}\m{\vee}\m{\chi}\m{)}\m{
\leftrightarrow}\m{(}\m{(}\m{\varphi}\m{\vee}\m{\psi}\m{)}\m{\vee}\m{\chi}\m{)}
\m{)}
\endm
\begin{mmraw}%
|- ( ( ph \TOR ps \TOR ch ) <-> ( ( ph \TOR ps ) \TOR ch ) ) \$.
\end{mmraw}

\noindent Define conjunction ({\sc and}) of 3 wffs.%
\index{df-3an}\label{df-3an}

\vskip 0.5ex
\setbox\startprefix=\hbox{\tt \ \ df-3an\ \$a\ }
\setbox\contprefix=\hbox{\tt \ \ \ \ \ \ \ \ \ \ \ \ }
\startm
\m{\vdash}\m{(}\m{(}\m{\varphi}\m{\wedge}\m{\psi}\m{\wedge}\m{\chi}\m{)}\m{
\leftrightarrow}\m{(}\m{(}\m{\varphi}\m{\wedge}\m{\psi}\m{)}\m{\wedge}\m{\chi}
\m{)}\m{)}
\endm

\begin{mmraw}%
|- ( ( ph \TAND ps \TAND ch ) <-> ( ( ph \TAND ps ) \TAND ch ) ) \$.
\end{mmraw}

\subsection{Definitions for Predicate Calculus}\label{metadefpred}

The symbols $x$, $y$, and $z$ represent individual variables of predicate
calculus.  In this section, they are not necessarily distinct unless it is
explicitly
mentioned.

\vskip 2ex
\noindent Define existential quantification.
The expression $\exists x \varphi$ means
``there exists an $x$ where $\varphi$ is true.''\index{existential quantifier
($\exists$)}\label{df-ex}

\vskip 0.5ex
\setbox\startprefix=\hbox{\tt \ \ df-ex\ \$a\ }
\setbox\contprefix=\hbox{\tt \ \ \ \ \ \ \ \ \ \ \ }
\startm
\m{\vdash}\m{(}\m{\exists}\m{x}\m{\varphi}\m{\leftrightarrow}\m{\lnot}\m{\forall}
\m{x}\m{\lnot}\m{\varphi}\m{)}
\endm
\begin{mmraw}%
|- ( E. x ph <-> -. A. x -. ph ) \$.
\end{mmraw}

\noindent Define proper substitution.\index{proper
substitution}\index{substitution!proper}\label{df-sb}
In our notation, we use $[ y / x ] \varphi$ to mean ``the wff that
results when $y$ is properly substituted for $x$ in the wff
$\varphi$.''\footnote{
This can also be described
as substituting $x$ with $y$, $y$ properly replaces $x$, or
$x$ is properly replaced by $y$.}
% This is elsb4, though it currently says: ( [ x / y ] z e. y <-> z e. x )
For example,
$[ y / x ] z \in x$ is the same as $z \in y$.
One way to remember this notation is to notice that it looks like division
and recall that $( y / x ) \cdot x $ is $y$ (when $x \neq 0$).
The notation is different from the notation $\varphi ( x | y )$
that is sometimes used, because the latter notation is ambiguous for us:
for example, we don't know whether $\lnot \varphi ( x | y )$ is to be
interpreted as $\lnot ( \varphi ( x | y ) )$ or
$( \lnot \varphi ) ( x | y )$.\footnote{Because of the way
we initially defined wffs, this is the case
with any postfix connective\index{postfix connective} (one occurring after the
symbols being connected) or infix connective\index{infix connective} (one
occurring between the symbols being connected).  Metamath does not have a
built-in notion of operator binding strength that could eliminate the
ambiguity.  The initial parenthesis effectively provides a prefix
connective\index{prefix connective} to eliminate ambiguity.  Some conventions,
such as Polish notation\index{Polish notation} used in the 1930's and 1940's
by Polish logicians, use only prefix connectives and thus allow the total
elimination of parentheses, at the expense of readability.  In Metamath we
could actually redefine all notation to be Polish if we wanted to without
having to change any proofs!}  Other texts often use $\varphi(y)$ to indicate
our $[ y / x ] \varphi$, but this notation is even more ambiguous since there is
no explicit indication of what is being substituted.
Note that this
definition is valid even when
$x$ and $y$ are the same variable.  The first conjunct is a ``trick'' used to
achieve this property, making the definition look somewhat peculiar at
first.

\vskip 0.5ex
\setbox\startprefix=\hbox{\tt \ \ df-sb\ \$a\ }
\setbox\contprefix=\hbox{\tt \ \ \ \ \ \ \ \ \ \ \ }
\startm
\m{\vdash}\m{(}\m{[}\m{y}\m{/}\m{x}\m{]}\m{\varphi}\m{\leftrightarrow}\m{(}%
\m{(}\m{x}\m{=}\m{y}\m{\rightarrow}\m{\varphi}\m{)}\m{\wedge}\m{\exists}\m{x}%
\m{(}\m{x}\m{=}\m{y}\m{\wedge}\m{\varphi}\m{)}\m{)}\m{)}
\endm
\begin{mmraw}%
|- ( [ y / x ] ph <-> ( ( x = y -> ph ) \TAND E. x ( x = y \TAND ph ) ) ) \$.
\end{mmraw}


\noindent Define existential uniqueness\index{existential uniqueness
quantifier ($\exists "!$)} (``there exists exactly one'').  Note that $y$ is a
variable distinct from $x$ and not occurring in $\varphi$.\label{df-eu}

\vskip 0.5ex
\setbox\startprefix=\hbox{\tt \ \ df-eu\ \$a\ }
\setbox\contprefix=\hbox{\tt \ \ \ \ \ \ \ \ \ \ \ }
\startm
\m{\vdash}\m{(}\m{\exists}\m{{!}}\m{x}\m{\varphi}\m{\leftrightarrow}\m{\exists}
\m{y}\m{\forall}\m{x}\m{(}\m{\varphi}\m{\leftrightarrow}\m{x}\m{=}\m{y}\m{)}\m{)}
\endm

\begin{mmraw}%
|- ( E! x ph <-> E. y A. x ( ph <-> x = y ) ) \$.
\end{mmraw}

\subsection{Definitions for Set Theory}\label{setdefinitions}

The symbols $x$, $y$, $z$, and $w$ represent individual variables of
predicate calculus, which in set theory are understood to be sets.
However, using only the constructs shown so far would be very inconvenient.

To make set theory more practical, we introduce the notion of a ``class.''
A class\index{class} is either a set variable (such as $x$) or an
expression of the form $\{ x | \varphi\}$ (called an ``abstraction
class''\index{abstraction class}\index{class abstraction}).  Note that
sets (i.e.\ individual variables) always exist (this is a theorem of
logic, namely $\exists y \, y = x$ for any set $x$), whereas classes may
or may not exist (i.e.\ $\exists y \, y = A$ may or may not be true).
If a class does not exist it is called a ``proper class.''\index{proper
class}\index{class!proper} Definitions \texttt{df-clab},
\texttt{df-cleq}, and \texttt{df-clel} can be used to convert an
expression containing classes into one containing only set variables and
wff metavariables.

The symbols $A$, $B$, $C$, $D$, $ F$, $G$, and $R$ are metavariables that range
over classes.  A class metavariable $A$ may be eliminated from a wff by
replacing it with $\{ x|\varphi\}$ where neither $x$ nor $\varphi$ occur in
the wff.

The theory of classes can be shown to be an eliminable and conservative
extension of set theory. The \textbf{eliminability}
property shows that for every
formula in the extended language we can build a logically equivalent
formula in the basic language; so that even if the extended language
provides more ease to convey and formulate mathematical ideas for set
theory, its expressive power does not in fact strengthen the basic
language's expressive power.
The \textbf{conservation} property shows that for
every proof of a formula of the basic language in the extended system
we can build another proof of the same formula in the basic system;
so that, concerning theorems on sets only, the deductive powers of
the extended system and of the basic system are identical. Together,
these properties mean that the extended language can be treated as a
definitional extension that is \textbf{sound}.

A rigorous justification, which we will not give here, can be found in
Levy \cite[pp.~357-366]{Levy} supplementing his informal introduction to class
theory on pp.~7-17. Two other good treatments of class theory are provided
by Quine \cite[pp.~15-21]{Quine}\index{Quine, Willard Van Orman}
and also \cite[pp.~10-14]{Takeuti}\index{Takeuti, Gaisi}.
Quine's exposition (he calls them virtual classes)
is nicely written and very readable.

In the rest of this
section, individual variables are always assumed to be distinct from
each other unless otherwise indicated.  In addition, dummy variables on the
right-hand side of a definition do not occur in the class and wff
metavariables in the definition.

The definitions we present here are a partial but self-contained
collection selected from several hundred that appear in the current
\texttt{set.mm} database.  They are adequate for a basic development of
elementary set theory.

\vskip 2ex
\noindent Define the abstraction class.\index{abstraction class}\index{class
abstraction}\label{df-clab}  $x$ and $y$
need not be distinct.  Definition 2.1 of Quine, p.~16.  This definition may
seem puzzling since it is shorter than the expression being defined and does not
buy us anything in terms of brevity.  The reason we introduce this definition
is because it fits in neatly with the extension of the $\in$ connective
provided by \texttt{df-clel}.

\vskip 0.5ex
\setbox\startprefix=\hbox{\tt \ \ df-clab\ \$a\ }
\setbox\contprefix=\hbox{\tt \ \ \ \ \ \ \ \ \ \ \ \ \ }
\startm
\m{\vdash}\m{(}\m{x}\m{\in}\m{\{}\m{y}\m{|}\m{\varphi}\m{\}}\m{%
\leftrightarrow}\m{[}\m{x}\m{/}\m{y}\m{]}\m{\varphi}\m{)}
\endm
\begin{mmraw}%
|- ( x e. \{ y | ph \} <-> [ x / y ] ph ) \$.
\end{mmraw}

\noindent Define the equality connective between classes\index{class
equality}\label{df-cleq}.  See Quine or Chapter 4 of Takeuti and Zaring for its
justification and methods for eliminating it.  This is an example of a
somewhat ``dangerous'' definition, because it extends the use of the
existing equality symbol rather than introducing a new symbol, allowing
us to make statements in the original language that may not be true.
For example, it permits us to deduce $y = z \leftrightarrow \forall x (
x \in y \leftrightarrow x \in z )$ which is not a theorem of logic but
rather presupposes the Axiom of Extensionality,\index{Axiom of
Extensionality} which we include as a hypothesis so that we can know
when this axiom is assumed in a proof (with the \texttt{show
trace{\char`\_}back} command).  We could avoid the danger by introducing
another symbol, say $\eqcirc$, in place of $=$; this
would also have the advantage of making elimination of the definition
straightforward and would eliminate the need for Extensionality as a
hypothesis.  We would then also have the advantage of being able to
identify exactly where Extensionality truly comes into play.  One of our
theorems would be $x \eqcirc y \leftrightarrow x = y$
by invoking Extensionality.  However in keeping with standard practice
we retain the ``dangerous'' definition.

\vskip 0.5ex
\setbox\startprefix=\hbox{\tt \ \ df-cleq.1\ \$e\ }
\setbox\contprefix=\hbox{\tt \ \ \ \ \ \ \ \ \ \ \ \ \ \ \ }
\startm
\m{\vdash}\m{(}\m{\forall}\m{x}\m{(}\m{x}\m{\in}\m{y}\m{\leftrightarrow}\m{x}
\m{\in}\m{z}\m{)}\m{\rightarrow}\m{y}\m{=}\m{z}\m{)}
\endm
\setbox\startprefix=\hbox{\tt \ \ df-cleq\ \$a\ }
\setbox\contprefix=\hbox{\tt \ \ \ \ \ \ \ \ \ \ \ \ \ }
\startm
\m{\vdash}\m{(}\m{A}\m{=}\m{B}\m{\leftrightarrow}\m{\forall}\m{x}\m{(}\m{x}\m{
\in}\m{A}\m{\leftrightarrow}\m{x}\m{\in}\m{B}\m{)}\m{)}
\endm
% We need to reset the startprefix and contprefix.
\setbox\startprefix=\hbox{\tt \ \ df-cleq.1\ \$e\ }
\setbox\contprefix=\hbox{\tt \ \ \ \ \ \ \ \ \ \ \ \ \ \ \ }
\begin{mmraw}%
|- ( A. x ( x e. y <-> x e. z ) -> y = z ) \$.
\end{mmraw}
\setbox\startprefix=\hbox{\tt \ \ df-cleq\ \$a\ }
\setbox\contprefix=\hbox{\tt \ \ \ \ \ \ \ \ \ \ \ \ \ }
\begin{mmraw}%
|- ( A = B <-> A. x ( x e. A <-> x e. B ) ) \$.
\end{mmraw}

\noindent Define the membership connective between classes\index{class
membership}.  Theorem 6.3 of Quine, p.~41, which we adopt as a definition.
Note that it extends the use of the existing membership symbol, but unlike
{\tt df-cleq} it does not extend the set of valid wffs of logic when the class
metavariables are replaced with set variables.\label{dfclel}\label{df-clel}

\vskip 0.5ex
\setbox\startprefix=\hbox{\tt \ \ df-clel\ \$a\ }
\setbox\contprefix=\hbox{\tt \ \ \ \ \ \ \ \ \ \ \ \ \ }
\startm
\m{\vdash}\m{(}\m{A}\m{\in}\m{B}\m{\leftrightarrow}\m{\exists}\m{x}\m{(}\m{x}
\m{=}\m{A}\m{\wedge}\m{x}\m{\in}\m{B}\m{)}\m{)}
\endm
\begin{mmraw}%
|- ( A e. B <-> E. x ( x = A \TAND x e. B ) ) \$.?
\end{mmraw}

\noindent Define inequality.

\vskip 0.5ex
\setbox\startprefix=\hbox{\tt \ \ df-ne\ \$a\ }
\setbox\contprefix=\hbox{\tt \ \ \ \ \ \ \ \ \ \ \ }
\startm
\m{\vdash}\m{(}\m{A}\m{\ne}\m{B}\m{\leftrightarrow}\m{\lnot}\m{A}\m{=}\m{B}%
\m{)}
\endm
\begin{mmraw}%
|- ( A =/= B <-> -. A = B ) \$.
\end{mmraw}

\noindent Define restricted universal quantification.\index{universal
quantifier ($\forall$)!restricted}  Enderton, p.~22.

\vskip 0.5ex
\setbox\startprefix=\hbox{\tt \ \ df-ral\ \$a\ }
\setbox\contprefix=\hbox{\tt \ \ \ \ \ \ \ \ \ \ \ \ }
\startm
\m{\vdash}\m{(}\m{\forall}\m{x}\m{\in}\m{A}\m{\varphi}\m{\leftrightarrow}\m{%
\forall}\m{x}\m{(}\m{x}\m{\in}\m{A}\m{\rightarrow}\m{\varphi}\m{)}\m{)}
\endm
\begin{mmraw}%
|- ( A. x e. A ph <-> A. x ( x e. A -> ph ) ) \$.
\end{mmraw}

\noindent Define restricted existential quantification.\index{existential
quantifier ($\exists$)!restricted}  Enderton, p.~22.

\vskip 0.5ex
\setbox\startprefix=\hbox{\tt \ \ df-rex\ \$a\ }
\setbox\contprefix=\hbox{\tt \ \ \ \ \ \ \ \ \ \ \ \ }
\startm
\m{\vdash}\m{(}\m{\exists}\m{x}\m{\in}\m{A}\m{\varphi}\m{\leftrightarrow}\m{%
\exists}\m{x}\m{(}\m{x}\m{\in}\m{A}\m{\wedge}\m{\varphi}\m{)}\m{)}
\endm
\begin{mmraw}%
|- ( E. x e. A ph <-> E. x ( x e. A \TAND ph ) ) \$.
\end{mmraw}

\noindent Define the universal class\index{universal class ($V$)}.  Definition
5.20, p.~21, of Takeuti and Zaring.\label{df-v}

\vskip 0.5ex
\setbox\startprefix=\hbox{\tt \ \ df-v\ \$a\ }
\setbox\contprefix=\hbox{\tt \ \ \ \ \ \ \ \ \ \ }
\startm
\m{\vdash}\m{{\rm V}}\m{=}\m{\{}\m{x}\m{|}\m{x}\m{=}\m{x}\m{\}}
\endm
\begin{mmraw}%
|- {\char`\_}V = \{ x | x = x \} \$.
\end{mmraw}

\noindent Define the subclass\index{subclass}\index{subset} relationship
between two classes (called the subset relation if the classes are sets i.e.\
are not proper).  Definition 5.9 of Takeuti and Zaring, p.~17.\label{df-ss}

\vskip 0.5ex
\setbox\startprefix=\hbox{\tt \ \ df-ss\ \$a\ }
\setbox\contprefix=\hbox{\tt \ \ \ \ \ \ \ \ \ \ \ }
\startm
\m{\vdash}\m{(}\m{A}\m{\subseteq}\m{B}\m{\leftrightarrow}\m{\forall}\m{x}\m{(}
\m{x}\m{\in}\m{A}\m{\rightarrow}\m{x}\m{\in}\m{B}\m{)}\m{)}
\endm
\begin{mmraw}%
|- ( A C\_ B <-> A. x ( x e. A -> x e. B ) ) \$.
\end{mmraw}

\noindent Define the union\index{union} of two classes.  Definition 5.6 of Takeuti and Zaring,
p.~16.\label{df-un}

\vskip 0.5ex
\setbox\startprefix=\hbox{\tt \ \ df-un\ \$a\ }
\setbox\contprefix=\hbox{\tt \ \ \ \ \ \ \ \ \ \ \ }
\startm
\m{\vdash}\m{(}\m{A}\m{\cup}\m{B}\m{)}\m{=}\m{\{}\m{x}\m{|}\m{(}\m{x}\m{\in}
\m{A}\m{\vee}\m{x}\m{\in}\m{B}\m{)}\m{\}}
\endm
\begin{mmraw}%
( A u. B ) = \{ x | ( x e. A \TOR x e. B ) \} \$.
\end{mmraw}

\noindent Define the intersection\index{intersection} of two classes.  Definition 5.6 of
Takeuti and Zaring, p.~16.\label{df-in}

\vskip 0.5ex
\setbox\startprefix=\hbox{\tt \ \ df-in\ \$a\ }
\setbox\contprefix=\hbox{\tt \ \ \ \ \ \ \ \ \ \ \ }
\startm
\m{\vdash}\m{(}\m{A}\m{\cap}\m{B}\m{)}\m{=}\m{\{}\m{x}\m{|}\m{(}\m{x}\m{\in}
\m{A}\m{\wedge}\m{x}\m{\in}\m{B}\m{)}\m{\}}
\endm
% Caret ^ requires special treatment
\begin{mmraw}%
|- ( A i\^{}i B ) = \{ x | ( x e. A \TAND x e. B ) \} \$.
\end{mmraw}

\noindent Define class difference\index{class difference}\index{set difference}.
Definition 5.12 of Takeuti and Zaring, p.~20.  Several notations are used in
the literature; we chose the $\setminus$ convention instead of a minus sign to
reserve the latter for later use in, e.g., arithmetic.\label{df-dif}

\vskip 0.5ex
\setbox\startprefix=\hbox{\tt \ \ df-dif\ \$a\ }
\setbox\contprefix=\hbox{\tt \ \ \ \ \ \ \ \ \ \ \ \ }
\startm
\m{\vdash}\m{(}\m{A}\m{\setminus}\m{B}\m{)}\m{=}\m{\{}\m{x}\m{|}\m{(}\m{x}\m{
\in}\m{A}\m{\wedge}\m{\lnot}\m{x}\m{\in}\m{B}\m{)}\m{\}}
\endm
\begin{mmraw}%
( A \SLASH B ) = \{ x | ( x e. A \TAND -. x e. B ) \} \$.
\end{mmraw}

\noindent Define the empty or null set\index{empty set}\index{null set}.
Compare  Definition 5.14 of Takeuti and Zaring, p.~20.\label{df-nul}

\vskip 0.5ex
\setbox\startprefix=\hbox{\tt \ \ df-nul\ \$a\ }
\setbox\contprefix=\hbox{\tt \ \ \ \ \ \ \ \ \ \ }
\startm
\m{\vdash}\m{\varnothing}\m{=}\m{(}\m{{\rm V}}\m{\setminus}\m{{\rm V}}\m{)}
\endm
\begin{mmraw}%
|- (/) = ( {\char`\_}V \SLASH {\char`\_}V ) \$.
\end{mmraw}

\noindent Define power class\index{power set}\index{power class}.  Definition 5.10 of
Takeuti and Zaring, p.~17, but we also let it apply to proper classes.  (Note
that \verb$~P$ is the symbol for calligraphic P, the tilde
suggesting ``curly;'' see Appendix~\ref{ASCII}.)\label{df-pw}

\vskip 0.5ex
\setbox\startprefix=\hbox{\tt \ \ df-pw\ \$a\ }
\setbox\contprefix=\hbox{\tt \ \ \ \ \ \ \ \ \ \ \ }
\startm
\m{\vdash}\m{{\cal P}}\m{A}\m{=}\m{\{}\m{x}\m{|}\m{x}\m{\subseteq}\m{A}\m{\}}
\endm
% Special incantation required to put ~ into the text
\begin{mmraw}%
|- \char`\~P~A = \{ x | x C\_ A \} \$.
\end{mmraw}

\noindent Define the singleton of a class\index{singleton}.  Definition 7.1 of
Quine, p.~48.  It is well-defined for proper classes, although
it is not very meaningful in this case, where it evaluates to the empty
set.

\vskip 0.5ex
\setbox\startprefix=\hbox{\tt \ \ df-sn\ \$a\ }
\setbox\contprefix=\hbox{\tt \ \ \ \ \ \ \ \ \ \ \ }
\startm
\m{\vdash}\m{\{}\m{A}\m{\}}\m{=}\m{\{}\m{x}\m{|}\m{x}\m{=}\m{A}\m{\}}
\endm
\begin{mmraw}%
|- \{ A \} = \{ x | x = A \} \$.
\end{mmraw}%

\noindent Define an unordered pair of classes\index{unordered pair}\index{pair}.  Definition
7.1 of Quine, p.~48.

\vskip 0.5ex
\setbox\startprefix=\hbox{\tt \ \ df-pr\ \$a\ }
\setbox\contprefix=\hbox{\tt \ \ \ \ \ \ \ \ \ \ \ }
\startm
\m{\vdash}\m{\{}\m{A}\m{,}\m{B}\m{\}}\m{=}\m{(}\m{\{}\m{A}\m{\}}\m{\cup}\m{\{}
\m{B}\m{\}}\m{)}
\endm
\begin{mmraw}%
|- \{ A , B \} = ( \{ A \} u. \{ B \} ) \$.
\end{mmraw}

\noindent Define an unordered triple of classes\index{unordered triple}.  Definition of
Enderton, p.~19.

\vskip 0.5ex
\setbox\startprefix=\hbox{\tt \ \ df-tp\ \$a\ }
\setbox\contprefix=\hbox{\tt \ \ \ \ \ \ \ \ \ \ \ }
\startm
\m{\vdash}\m{\{}\m{A}\m{,}\m{B}\m{,}\m{C}\m{\}}\m{=}\m{(}\m{\{}\m{A}\m{,}\m{B}
\m{\}}\m{\cup}\m{\{}\m{C}\m{\}}\m{)}
\endm
\begin{mmraw}%
|- \{ A , B , C \} = ( \{ A , B \} u. \{ C \} ) \$.
\end{mmraw}%

\noindent Kuratowski's\index{Kuratowski, Kazimierz} ordered pair\index{ordered
pair} definition.  Definition 9.1 of Quine, p.~58. For proper classes it is
not meaningful but is well-defined for convenience.  (Note that \verb$<.$
stands for $\langle$ whereas \verb$<$ stands for $<$, and similarly for
\verb$>.$\,.)\label{df-op}

\vskip 0.5ex
\setbox\startprefix=\hbox{\tt \ \ df-op\ \$a\ }
\setbox\contprefix=\hbox{\tt \ \ \ \ \ \ \ \ \ \ \ }
\startm
\m{\vdash}\m{\langle}\m{A}\m{,}\m{B}\m{\rangle}\m{=}\m{\{}\m{\{}\m{A}\m{\}}
\m{,}\m{\{}\m{A}\m{,}\m{B}\m{\}}\m{\}}
\endm
\begin{mmraw}%
|- <. A , B >. = \{ \{ A \} , \{ A , B \} \} \$.
\end{mmraw}

\noindent Define the union of a class\index{union}.  Definition 5.5, p.~16,
of Takeuti and Zaring.

\vskip 0.5ex
\setbox\startprefix=\hbox{\tt \ \ df-uni\ \$a\ }
\setbox\contprefix=\hbox{\tt \ \ \ \ \ \ \ \ \ \ \ \ }
\startm
\m{\vdash}\m{\bigcup}\m{A}\m{=}\m{\{}\m{x}\m{|}\m{\exists}\m{y}\m{(}\m{x}\m{
\in}\m{y}\m{\wedge}\m{y}\m{\in}\m{A}\m{)}\m{\}}
\endm
\begin{mmraw}%
|- U. A = \{ x | E. y ( x e. y \TAND y e. A ) \} \$.
\end{mmraw}

\noindent Define the intersection\index{intersection} of a class.  Definition 7.35,
p.~44, of Takeuti and Zaring.

\vskip 0.5ex
\setbox\startprefix=\hbox{\tt \ \ df-int\ \$a\ }
\setbox\contprefix=\hbox{\tt \ \ \ \ \ \ \ \ \ \ \ \ }
\startm
\m{\vdash}\m{\bigcap}\m{A}\m{=}\m{\{}\m{x}\m{|}\m{\forall}\m{y}\m{(}\m{y}\m{
\in}\m{A}\m{\rightarrow}\m{x}\m{\in}\m{y}\m{)}\m{\}}
\endm
\begin{mmraw}%
|- |\^{}| A = \{ x | A. y ( y e. A -> x e. y ) \} \$.
\end{mmraw}

\noindent Define a transitive class\index{transitive class}\index{transitive
set}.  This should not be confused with a transitive relation, which is a different
concept.  Definition from p.~71 of Enderton, extended to classes.

\vskip 0.5ex
\setbox\startprefix=\hbox{\tt \ \ df-tr\ \$a\ }
\setbox\contprefix=\hbox{\tt \ \ \ \ \ \ \ \ \ \ \ }
\startm
\m{\vdash}\m{(}\m{\mbox{\rm Tr}}\m{A}\m{\leftrightarrow}\m{\bigcup}\m{A}\m{
\subseteq}\m{A}\m{)}
\endm
\begin{mmraw}%
|- ( Tr A <-> U. A C\_ A ) \$.
\end{mmraw}

\noindent Define a notation for a general binary relation\index{binary
relation}.  Definition 6.18, p.~29, of Takeuti and Zaring, generalized to
arbitrary classes.  This definition is well-defined, although not very
meaningful, when classes $A$ and/or $B$ are proper.\label{dfbr}  The lack of
parentheses (or any other connective) creates no ambiguity since we are defining
an atomic wff.

\vskip 0.5ex
\setbox\startprefix=\hbox{\tt \ \ df-br\ \$a\ }
\setbox\contprefix=\hbox{\tt \ \ \ \ \ \ \ \ \ \ \ }
\startm
\m{\vdash}\m{(}\m{A}\m{\,R}\m{\,B}\m{\leftrightarrow}\m{\langle}\m{A}\m{,}\m{B}
\m{\rangle}\m{\in}\m{R}\m{)}
\endm
\begin{mmraw}%
|- ( A R B <-> <. A , B >. e. R ) \$.
\end{mmraw}

\noindent Define an abstraction class of ordered pairs\index{abstraction
class!of ordered
pairs}.  A special case of Definition 4.16, p.~14, of Takeuti and Zaring.
Note that $ z $ must be distinct from $ x $ and $ y $,
and $ z $ must not occur in $\varphi$, but $ x $ and $ y $ may be
identical and may appear in $\varphi$.

\vskip 0.5ex
\setbox\startprefix=\hbox{\tt \ \ df-opab\ \$a\ }
\setbox\contprefix=\hbox{\tt \ \ \ \ \ \ \ \ \ \ \ \ \ }
\startm
\m{\vdash}\m{\{}\m{\langle}\m{x}\m{,}\m{y}\m{\rangle}\m{|}\m{\varphi}\m{\}}\m{=}
\m{\{}\m{z}\m{|}\m{\exists}\m{x}\m{\exists}\m{y}\m{(}\m{z}\m{=}\m{\langle}\m{x}
\m{,}\m{y}\m{\rangle}\m{\wedge}\m{\varphi}\m{)}\m{\}}
\endm

\begin{mmraw}%
|- \{ <. x , y >. | ph \} = \{ z | E. x E. y ( z =
<. x , y >. /\ ph ) \} \$.
\end{mmraw}

\noindent Define the epsilon relation\index{epsilon relation}.  Similar to Definition
6.22, p.~30, of Takeuti and Zaring.

\vskip 0.5ex
\setbox\startprefix=\hbox{\tt \ \ df-eprel\ \$a\ }
\setbox\contprefix=\hbox{\tt \ \ \ \ \ \ \ \ \ \ \ \ \ \ }
\startm
\m{\vdash}\m{{\rm E}}\m{=}\m{\{}\m{\langle}\m{x}\m{,}\m{y}\m{\rangle}\m{|}\m{x}\m{
\in}\m{y}\m{\}}
\endm
\begin{mmraw}%
|- \_E = \{ <. x , y >. | x e. y \} \$.
\end{mmraw}

\noindent Define a founded relation\index{founded relation}.  $R$ is a founded
relation on $A$ iff\index{iff} (if and only if) each nonempty subset of $A$
has an ``$R$-minimal element.''  Similar to Definition 6.21, p.~30, of
Takeuti and Zaring.

\vskip 0.5ex
\setbox\startprefix=\hbox{\tt \ \ df-fr\ \$a\ }
\setbox\contprefix=\hbox{\tt \ \ \ \ \ \ \ \ \ \ \ }
\startm
\m{\vdash}\m{(}\m{R}\m{\,\mbox{\rm Fr}}\m{\,A}\m{\leftrightarrow}\m{\forall}\m{x}
\m{(}\m{(}\m{x}\m{\subseteq}\m{A}\m{\wedge}\m{\lnot}\m{x}\m{=}\m{\varnothing}
\m{)}\m{\rightarrow}\m{\exists}\m{y}\m{(}\m{y}\m{\in}\m{x}\m{\wedge}\m{(}\m{x}
\m{\cap}\m{\{}\m{z}\m{|}\m{z}\m{\,R}\m{\,y}\m{\}}\m{)}\m{=}\m{\varnothing}\m{)}
\m{)}\m{)}
\endm
\begin{mmraw}%
|- ( R Fr A <-> A. x ( ( x C\_ A \TAND -. x = (/) ) ->
E. y ( y e. x \TAND ( x i\^{}i \{ z | z R y \} ) = (/) ) ) ) \$.
\end{mmraw}

\noindent Define a well-ordering\index{well-ordering}.  $R$ is a well-ordering of $A$ iff
it is founded on $A$ and the elements of $A$ are pairwise $R$-comparable.
Similar to Definition 6.24(2), p.~30, of Takeuti and Zaring.

\vskip 0.5ex
\setbox\startprefix=\hbox{\tt \ \ df-we\ \$a\ }
\setbox\contprefix=\hbox{\tt \ \ \ \ \ \ \ \ \ \ \ }
\startm
\m{\vdash}\m{(}\m{R}\m{\,\mbox{\rm We}}\m{\,A}\m{\leftrightarrow}\m{(}\m{R}\m{\,
\mbox{\rm Fr}}\m{\,A}\m{\wedge}\m{\forall}\m{x}\m{\forall}\m{y}\m{(}\m{(}\m{x}\m{
\in}\m{A}\m{\wedge}\m{y}\m{\in}\m{A}\m{)}\m{\rightarrow}\m{(}\m{x}\m{\,R}\m{\,y}
\m{\vee}\m{x}\m{=}\m{y}\m{\vee}\m{y}\m{\,R}\m{\,x}\m{)}\m{)}\m{)}\m{)}
\endm
\begin{mmraw}%
( R We A <-> ( R Fr A \TAND A. x A. y ( ( x e.
A \TAND y e. A ) -> ( x R y \TOR x = y \TOR y R x ) ) ) ) \$.
\end{mmraw}

\noindent Define the ordinal predicate\index{ordinal predicate}, which is true for a class
that is transitive and is well-ordered by the epsilon relation.  Similar to
definition on p.~468, Bell and Machover.

\vskip 0.5ex
\setbox\startprefix=\hbox{\tt \ \ df-ord\ \$a\ }
\setbox\contprefix=\hbox{\tt \ \ \ \ \ \ \ \ \ \ \ \ }
\startm
\m{\vdash}\m{(}\m{\mbox{\rm Ord}}\m{\,A}\m{\leftrightarrow}\m{(}
\m{\mbox{\rm Tr}}\m{\,A}\m{\wedge}\m{E}\m{\,\mbox{\rm We}}\m{\,A}\m{)}\m{)}
\endm
\begin{mmraw}%
|- ( Ord A <-> ( Tr A \TAND E We A ) ) \$.
\end{mmraw}

\noindent Define the class of all ordinal numbers\index{ordinal number}.  An ordinal number is
a set that satisfies the ordinal predicate.  Definition 7.11 of Takeuti and
Zaring, p.~38.

\vskip 0.5ex
\setbox\startprefix=\hbox{\tt \ \ df-on\ \$a\ }
\setbox\contprefix=\hbox{\tt \ \ \ \ \ \ \ \ \ \ \ }
\startm
\m{\vdash}\m{\,\mbox{\rm On}}\m{=}\m{\{}\m{x}\m{|}\m{\mbox{\rm Ord}}\m{\,x}
\m{\}}
\endm
\begin{mmraw}%
|- On = \{ x | Ord x \} \$.
\end{mmraw}

\noindent Define the limit ordinal predicate\index{limit ordinal}, which is true for a
non-empty ordinal that is not a successor (i.e.\ that is the union of itself).
Compare Bell and Machover, p.~471 and Exercise (1), p.~42 of Takeuti and
Zaring.

\vskip 0.5ex
\setbox\startprefix=\hbox{\tt \ \ df-lim\ \$a\ }
\setbox\contprefix=\hbox{\tt \ \ \ \ \ \ \ \ \ \ \ \ }
\startm
\m{\vdash}\m{(}\m{\mbox{\rm Lim}}\m{\,A}\m{\leftrightarrow}\m{(}\m{\mbox{
\rm Ord}}\m{\,A}\m{\wedge}\m{\lnot}\m{A}\m{=}\m{\varnothing}\m{\wedge}\m{A}
\m{=}\m{\bigcup}\m{A}\m{)}\m{)}
\endm
\begin{mmraw}%
|- ( Lim A <-> ( Ord A \TAND -. A = (/) \TAND A = U. A ) ) \$.
\end{mmraw}

\noindent Define the successor\index{successor} of a class.  Definition 7.22 of Takeuti
and Zaring, p.~41.  Our definition is a generalization to classes, although it
is meaningless when classes are proper.

\vskip 0.5ex
\setbox\startprefix=\hbox{\tt \ \ df-suc\ \$a\ }
\setbox\contprefix=\hbox{\tt \ \ \ \ \ \ \ \ \ \ \ \ }
\startm
\m{\vdash}\m{\,\mbox{\rm suc}}\m{\,A}\m{=}\m{(}\m{A}\m{\cup}\m{\{}\m{A}\m{\}}
\m{)}
\endm
\begin{mmraw}%
|- suc A = ( A u. \{ A \} ) \$.
\end{mmraw}

\noindent Define the class of natural numbers\index{natural number}\index{omega
($\omega$)}.  Compare Bell and Machover, p.~471.\label{dfom}

\vskip 0.5ex
\setbox\startprefix=\hbox{\tt \ \ df-om\ \$a\ }
\setbox\contprefix=\hbox{\tt \ \ \ \ \ \ \ \ \ \ \ }
\startm
\m{\vdash}\m{\omega}\m{=}\m{\{}\m{x}\m{|}\m{(}\m{\mbox{\rm Ord}}\m{\,x}\m{
\wedge}\m{\forall}\m{y}\m{(}\m{\mbox{\rm Lim}}\m{\,y}\m{\rightarrow}\m{x}\m{
\in}\m{y}\m{)}\m{)}\m{\}}
\endm
\begin{mmraw}%
|- om = \{ x | ( Ord x \TAND A. y ( Lim y -> x e. y ) ) \} \$.
\end{mmraw}

\noindent Define the Cartesian product (also called the
cross product)\index{Cartesian product}\index{cross product}
of two classes.  Definition 9.11 of Quine, p.~64.

\vskip 0.5ex
\setbox\startprefix=\hbox{\tt \ \ df-xp\ \$a\ }
\setbox\contprefix=\hbox{\tt \ \ \ \ \ \ \ \ \ \ \ }
\startm
\m{\vdash}\m{(}\m{A}\m{\times}\m{B}\m{)}\m{=}\m{\{}\m{\langle}\m{x}\m{,}\m{y}
\m{\rangle}\m{|}\m{(}\m{x}\m{\in}\m{A}\m{\wedge}\m{y}\m{\in}\m{B}\m{)}\m{\}}
\endm
\begin{mmraw}%
|- ( A X. B ) = \{ <. x , y >. | ( x e. A \TAND y e. B) \} \$.
\end{mmraw}

\noindent Define a relation\index{relation}.  Definition 6.4(1) of Takeuti and
Zaring, p.~23.

\vskip 0.5ex
\setbox\startprefix=\hbox{\tt \ \ df-rel\ \$a\ }
\setbox\contprefix=\hbox{\tt \ \ \ \ \ \ \ \ \ \ \ \ }
\startm
\m{\vdash}\m{(}\m{\mbox{\rm Rel}}\m{\,A}\m{\leftrightarrow}\m{A}\m{\subseteq}
\m{(}\m{{\rm V}}\m{\times}\m{{\rm V}}\m{)}\m{)}
\endm
\begin{mmraw}%
|- ( Rel A <-> A C\_ ( {\char`\_}V X. {\char`\_}V ) ) \$.
\end{mmraw}

\noindent Define the domain\index{domain} of a class.  Definition 6.5(1) of
Takeuti and Zaring, p.~24.

\vskip 0.5ex
\setbox\startprefix=\hbox{\tt \ \ df-dm\ \$a\ }
\setbox\contprefix=\hbox{\tt \ \ \ \ \ \ \ \ \ \ \ }
\startm
\m{\vdash}\m{\,\mbox{\rm dom}}\m{A}\m{=}\m{\{}\m{x}\m{|}\m{\exists}\m{y}\m{
\langle}\m{x}\m{,}\m{y}\m{\rangle}\m{\in}\m{A}\m{\}}
\endm
\begin{mmraw}%
|- dom A = \{ x | E. y <. x , y >. e. A \} \$.
\end{mmraw}

\noindent Define the range\index{range} of a class.  Definition 6.5(2) of
Takeuti and Zaring, p.~24.

\vskip 0.5ex
\setbox\startprefix=\hbox{\tt \ \ df-rn\ \$a\ }
\setbox\contprefix=\hbox{\tt \ \ \ \ \ \ \ \ \ \ \ }
\startm
\m{\vdash}\m{\,\mbox{\rm ran}}\m{A}\m{=}\m{\{}\m{y}\m{|}\m{\exists}\m{x}\m{
\langle}\m{x}\m{,}\m{y}\m{\rangle}\m{\in}\m{A}\m{\}}
\endm
\begin{mmraw}%
|- ran A = \{ y | E. x <. x , y >. e. A \} \$.
\end{mmraw}

\noindent Define the restriction\index{restriction} of a class.  Definition
6.6(1) of Takeuti and Zaring, p.~24.

\vskip 0.5ex
\setbox\startprefix=\hbox{\tt \ \ df-res\ \$a\ }
\setbox\contprefix=\hbox{\tt \ \ \ \ \ \ \ \ \ \ \ \ }
\startm
\m{\vdash}\m{(}\m{A}\m{\restriction}\m{B}\m{)}\m{=}\m{(}\m{A}\m{\cap}\m{(}\m{B}
\m{\times}\m{{\rm V}}\m{)}\m{)}
\endm
\begin{mmraw}%
|- ( A |` B ) = ( A i\^{}i ( B X. {\char`\_}V ) ) \$.
\end{mmraw}

\noindent Define the image\index{image} of a class.  Definition 6.6(2) of
Takeuti and Zaring, p.~24.

\vskip 0.5ex
\setbox\startprefix=\hbox{\tt \ \ df-ima\ \$a\ }
\setbox\contprefix=\hbox{\tt \ \ \ \ \ \ \ \ \ \ \ \ }
\startm
\m{\vdash}\m{(}\m{A}\m{``}\m{B}\m{)}\m{=}\m{\,\mbox{\rm ran}}\m{\,(}\m{A}\m{
\restriction}\m{B}\m{)}
\endm
\begin{mmraw}%
|- ( A " B ) = ran ( A |` B ) \$.
\end{mmraw}

\noindent Define the composition\index{composition} of two classes.  Definition 6.6(3) of
Takeuti and Zaring, p.~24.

\vskip 0.5ex
\setbox\startprefix=\hbox{\tt \ \ df-co\ \$a\ }
\setbox\contprefix=\hbox{\tt \ \ \ \ \ \ \ \ \ \ \ \ }
\startm
\m{\vdash}\m{(}\m{A}\m{\circ}\m{B}\m{)}\m{=}\m{\{}\m{\langle}\m{x}\m{,}\m{y}\m{
\rangle}\m{|}\m{\exists}\m{z}\m{(}\m{\langle}\m{x}\m{,}\m{z}\m{\rangle}\m{\in}
\m{B}\m{\wedge}\m{\langle}\m{z}\m{,}\m{y}\m{\rangle}\m{\in}\m{A}\m{)}\m{\}}
\endm
\begin{mmraw}%
|- ( A o. B ) = \{ <. x , y >. | E. z ( <. x , z
>. e. B \TAND <. z , y >. e. A ) \} \$.
\end{mmraw}

\noindent Define a function\index{function}.  Definition 6.4(4) of Takeuti and
Zaring, p.~24.

\vskip 0.5ex
\setbox\startprefix=\hbox{\tt \ \ df-fun\ \$a\ }
\setbox\contprefix=\hbox{\tt \ \ \ \ \ \ \ \ \ \ \ \ }
\startm
\m{\vdash}\m{(}\m{\mbox{\rm Fun}}\m{\,A}\m{\leftrightarrow}\m{(}
\m{\mbox{\rm Rel}}\m{\,A}\m{\wedge}
\m{\forall}\m{x}\m{\exists}\m{z}\m{\forall}\m{y}\m{(}
\m{\langle}\m{x}\m{,}\m{y}\m{\rangle}\m{\in}\m{A}\m{\rightarrow}\m{y}\m{=}\m{z}
\m{)}\m{)}\m{)}
\endm
\begin{mmraw}%
|- ( Fun A <-> ( Rel A /\ A. x E. z A. y ( <. x
   , y >. e. A -> y = z ) ) ) \$.
\end{mmraw}

\noindent Define a function with domain.  Definition 6.15(1) of Takeuti and
Zaring, p.~27.

\vskip 0.5ex
\setbox\startprefix=\hbox{\tt \ \ df-fn\ \$a\ }
\setbox\contprefix=\hbox{\tt \ \ \ \ \ \ \ \ \ \ \ }
\startm
\m{\vdash}\m{(}\m{A}\m{\,\mbox{\rm Fn}}\m{\,B}\m{\leftrightarrow}\m{(}
\m{\mbox{\rm Fun}}\m{\,A}\m{\wedge}\m{\mbox{\rm dom}}\m{\,A}\m{=}\m{B}\m{)}
\m{)}
\endm
\begin{mmraw}%
|- ( A Fn B <-> ( Fun A \TAND dom A = B ) ) \$.
\end{mmraw}

\noindent Define a function with domain and co-domain.  Definition 6.15(3)
of Takeuti and Zaring, p.~27.

\vskip 0.5ex
\setbox\startprefix=\hbox{\tt \ \ df-f\ \$a\ }
\setbox\contprefix=\hbox{\tt \ \ \ \ \ \ \ \ \ \ }
\startm
\m{\vdash}\m{(}\m{F}\m{:}\m{A}\m{\longrightarrow}\m{B}\m{
\leftrightarrow}\m{(}\m{F}\m{\,\mbox{\rm Fn}}\m{\,A}\m{\wedge}\m{
\mbox{\rm ran}}\m{\,F}\m{\subseteq}\m{B}\m{)}\m{)}
\endm
\begin{mmraw}%
|- ( F : A --> B <-> ( F Fn A \TAND ran F C\_ B ) ) \$.
\end{mmraw}

\noindent Define a one-to-one function\index{one-to-one function}.  Compare
Definition 6.15(5) of Takeuti and Zaring, p.~27.

\vskip 0.5ex
\setbox\startprefix=\hbox{\tt \ \ df-f1\ \$a\ }
\setbox\contprefix=\hbox{\tt \ \ \ \ \ \ \ \ \ \ \ }
\startm
\m{\vdash}\m{(}\m{F}\m{:}\m{A}\m{
\raisebox{.5ex}{${\textstyle{\:}_{\mbox{\footnotesize\rm
1\tt -\rm 1}}}\atop{\textstyle{
\longrightarrow}\atop{\textstyle{}^{\mbox{\footnotesize\rm {\ }}}}}$}
}\m{B}
\m{\leftrightarrow}\m{(}\m{F}\m{:}\m{A}\m{\longrightarrow}\m{B}
\m{\wedge}\m{\forall}\m{y}\m{\exists}\m{z}\m{\forall}\m{x}\m{(}\m{\langle}\m{x}
\m{,}\m{y}\m{\rangle}\m{\in}\m{F}\m{\rightarrow}\m{x}\m{=}\m{z}\m{)}\m{)}\m{)}
\endm
\begin{mmraw}%
|- ( F : A -1-1-> B <-> ( F : A --> B \TAND
   A. y E. z A. x ( <. x , y >. e. F -> x = z ) ) ) \$.
\end{mmraw}

\noindent Define an onto function\index{onto function}.  Definition 6.15(4) of Takeuti and
Zaring, p.~27.

\vskip 0.5ex
\setbox\startprefix=\hbox{\tt \ \ df-fo\ \$a\ }
\setbox\contprefix=\hbox{\tt \ \ \ \ \ \ \ \ \ \ \ }
\startm
\m{\vdash}\m{(}\m{F}\m{:}\m{A}\m{
\raisebox{.5ex}{${\textstyle{\:}_{\mbox{\footnotesize\rm
{\ }}}}\atop{\textstyle{
\longrightarrow}\atop{\textstyle{}^{\mbox{\footnotesize\rm onto}}}}$}
}\m{B}
\m{\leftrightarrow}\m{(}\m{F}\m{\,\mbox{\rm Fn}}\m{\,A}\m{\wedge}
\m{\mbox{\rm ran}}\m{\,F}\m{=}\m{B}\m{)}\m{)}
\endm
\begin{mmraw}%
|- ( F : A -onto-> B <-> ( F Fn A /\ ran F = B ) ) \$.
\end{mmraw}

\noindent Define a one-to-one, onto function.  Compare Definition 6.15(6) of
Takeuti and Zaring, p.~27.

\vskip 0.5ex
\setbox\startprefix=\hbox{\tt \ \ df-f1o\ \$a\ }
\setbox\contprefix=\hbox{\tt \ \ \ \ \ \ \ \ \ \ \ \ }
\startm
\m{\vdash}\m{(}\m{F}\m{:}\m{A}
\m{
\raisebox{.5ex}{${\textstyle{\:}_{\mbox{\footnotesize\rm
1\tt -\rm 1}}}\atop{\textstyle{
\longrightarrow}\atop{\textstyle{}^{\mbox{\footnotesize\rm onto}}}}$}
}
\m{B}
\m{\leftrightarrow}\m{(}\m{F}\m{:}\m{A}
\m{
\raisebox{.5ex}{${\textstyle{\:}_{\mbox{\footnotesize\rm
1\tt -\rm 1}}}\atop{\textstyle{
\longrightarrow}\atop{\textstyle{}^{\mbox{\footnotesize\rm {\ }}}}}$}
}
\m{B}\m{\wedge}\m{F}\m{:}\m{A}
\m{
\raisebox{.5ex}{${\textstyle{\:}_{\mbox{\footnotesize\rm
{\ }}}}\atop{\textstyle{
\longrightarrow}\atop{\textstyle{}^{\mbox{\footnotesize\rm onto}}}}$}
}
\m{B}\m{)}\m{)}
\endm
\begin{mmraw}%
|- ( F : A -1-1-onto-> B <-> ( F : A -1-1-> B? \TAND F : A -onto-> B ) ) \$.?
\end{mmraw}

\noindent Define the value of a function\index{function value}.  This
definition applies to any class and evaluates to the empty set when it is not
meaningful. Note that $ F`A$ means the same thing as the more familiar $ F(A)$
notation for a function's value at $A$.  The $ F`A$ notation is common in
formal set theory.\label{df-fv}

\vskip 0.5ex
\setbox\startprefix=\hbox{\tt \ \ df-fv\ \$a\ }
\setbox\contprefix=\hbox{\tt \ \ \ \ \ \ \ \ \ \ \ }
\startm
\m{\vdash}\m{(}\m{F}\m{`}\m{A}\m{)}\m{=}\m{\bigcup}\m{\{}\m{x}\m{|}\m{(}\m{F}%
\m{``}\m{\{}\m{A}\m{\}}\m{)}\m{=}\m{\{}\m{x}\m{\}}\m{\}}
\endm
\begin{mmraw}%
|- ( F ` A ) = U. \{ x | ( F " \{ A \} ) = \{ x \} \} \$.
\end{mmraw}

\noindent Define the result of an operation.\index{operation}  Here, $F$ is
     an operation on two
     values (such as $+$ for real numbers).   This is defined for proper
     classes $A$ and $B$ even though not meaningful in that case.  However,
     the definition can be meaningful when $F$ is a proper class.\label{dfopr}

\vskip 0.5ex
\setbox\startprefix=\hbox{\tt \ \ df-opr\ \$a\ }
\setbox\contprefix=\hbox{\tt \ \ \ \ \ \ \ \ \ \ \ \ }
\startm
\m{\vdash}\m{(}\m{A}\m{\,F}\m{\,B}\m{)}\m{=}\m{(}\m{F}\m{`}\m{\langle}\m{A}%
\m{,}\m{B}\m{\rangle}\m{)}
\endm
\begin{mmraw}%
|- ( A F B ) = ( F ` <. A , B >. ) \$.
\end{mmraw}

\section{Tricks of the Trade}\label{tricks}

In the \texttt{set.mm}\index{set theory database (\texttt{set.mm})} database our goal
was usually to conform to modern notation.  However in some cases the
relationship to standard textbook language may be obscured by several
unconventional devices we used to simplify the development and to take
advantage of the Metamath language.  In this section we will describe some
common conventions used in \texttt{set.mm}.

\begin{itemize}
\item
The turnstile\index{turnstile ({$\,\vdash$})} symbol, $\vdash$, meaning ``it
is provable that,'' is the first token of all assertions and hypotheses that
aren't syntax constructions.  This is a standard convention in logic.  (We
mentioned this earlier, but this symbol is bothersome to some people without a
logic background.  It has no deeper meaning but just provides us with a way to
distinguish syntax constructions from ordinary mathematical statements.)

\item
A hypothesis of the form

\vskip 1ex
\setbox\startprefix=\hbox{\tt \ \ \ \ \ \ \ \ \ \$e\ }
\setbox\contprefix=\hbox{\tt \ \ \ \ \ \ \ \ \ \ }
\startm
\m{\vdash}\m{(}\m{\varphi}\m{\rightarrow}\m{\forall}\m{x}\m{\varphi}\m{)}
\endm
\vskip 1ex

should be read ``assume variable $x$ is (effectively) not free in wff
$\varphi$.''\index{effectively not free}
Literally, this says ``assume it is provable that $\varphi \rightarrow \forall
x\, \varphi$.''  This device lets us avoid the complexities associated with
the standard treatment of free and bound variables.
%% Uncomment this when uncommenting section {formalspec} below
The footnote on p.~\pageref{effectivelybound} discusses this further.

\item
A statement of one of the forms

\vskip 1ex
\setbox\startprefix=\hbox{\tt \ \ \ \ \ \ \ \ \ \$a\ }
\setbox\contprefix=\hbox{\tt \ \ \ \ \ \ \ \ \ \ }
\startm
\m{\vdash}\m{(}\m{\lnot}\m{\forall}\m{x}\m{\,x}\m{=}\m{y}\m{\rightarrow}
\m{\ldots}\m{)}
\endm
\setbox\startprefix=\hbox{\tt \ \ \ \ \ \ \ \ \ \$p\ }
\setbox\contprefix=\hbox{\tt \ \ \ \ \ \ \ \ \ \ }
\startm
\m{\vdash}\m{(}\m{\lnot}\m{\forall}\m{x}\m{\,x}\m{=}\m{y}\m{\rightarrow}
\m{\ldots}\m{)}
\endm
\vskip 1ex

should be read ``if $x$ and $y$ are distinct variables, then...''  This
antecedent provides us with a technical device to avoid the need for the
\texttt{\$d} statement early in our development of predicate calculus,
permitting symbol manipulations to be as conceptually simple as those in
propositional calculus.  However, the \texttt{\$d} statement eventually
becomes a requirement, and after that this device is rarely used.

\item
The statement

\vskip 1ex
\setbox\startprefix=\hbox{\tt \ \ \ \ \ \ \ \ \ \$d\ }
\setbox\contprefix=\hbox{\tt \ \ \ \ \ \ \ \ \ \ }
\startm
\m{x}\m{\,y}
\endm
\vskip 1ex

should be read ``assume $x$ and $y$ are distinct variables.''

\item
The statement

\vskip 1ex
\setbox\startprefix=\hbox{\tt \ \ \ \ \ \ \ \ \ \$d\ }
\setbox\contprefix=\hbox{\tt \ \ \ \ \ \ \ \ \ \ }
\startm
\m{x}\m{\,\varphi}
\endm
\vskip 1ex

should be read ``assume $x$ does not occur in $\varphi$.''

\item
The statement

\vskip 1ex
\setbox\startprefix=\hbox{\tt \ \ \ \ \ \ \ \ \ \$d\ }
\setbox\contprefix=\hbox{\tt \ \ \ \ \ \ \ \ \ \ }
\startm
\m{x}\m{\,A}
\endm
\vskip 1ex

should be read ``assume variable $x$ does not occur in class $A$.''

\item
The restriction and hypothesis group

\vskip 1ex
\setbox\startprefix=\hbox{\tt \ \ \ \ \ \ \ \ \ \$d\ }
\setbox\contprefix=\hbox{\tt \ \ \ \ \ \ \ \ \ \ }
\startm
\m{x}\m{\,A}
\endm
\setbox\startprefix=\hbox{\tt \ \ \ \ \ \ \ \ \ \$d\ }
\setbox\contprefix=\hbox{\tt \ \ \ \ \ \ \ \ \ \ }
\startm
\m{x}\m{\,\psi}
\endm
\setbox\startprefix=\hbox{\tt \ \ \ \ \ \ \ \ \ \$e\ }
\setbox\contprefix=\hbox{\tt \ \ \ \ \ \ \ \ \ \ }
\startm
\m{\vdash}\m{(}\m{x}\m{=}\m{A}\m{\rightarrow}\m{(}\m{\varphi}\m{\leftrightarrow}
\m{\psi}\m{)}\m{)}
\endm
\vskip 1ex

is frequently used in place of explicit substitution, meaning ``assume
$\psi$ results from the proper substitution of $A$ for $x$ in
$\varphi$.''  Sometimes ``\texttt{\$e} $\vdash ( \psi \rightarrow
\forall x \, \psi )$'' is used instead of ``\texttt{\$d} $x\, \psi $,''
which requires only that $x$ be effectively not free in $\varphi$ but
not necessarily absent from it.  The use of implicit
substitution\index{substitution!implicit} is partly a matter of personal
style, although it may make proofs somewhat shorter than would be the
case with explicit substitution.

\item
The hypothesis


\vskip 1ex
\setbox\startprefix=\hbox{\tt \ \ \ \ \ \ \ \ \ \$e\ }
\setbox\contprefix=\hbox{\tt \ \ \ \ \ \ \ \ \ \ }
\startm
\m{\vdash}\m{A}\m{\in}\m{{\rm V}}
\endm
\vskip 1ex

should be read ``assume class $A$ is a set (i.e.\ exists).''
This is a convenient convention used by Quine.

\item
The restriction and hypothesis

\vskip 1ex
\setbox\startprefix=\hbox{\tt \ \ \ \ \ \ \ \ \ \$d\ }
\setbox\contprefix=\hbox{\tt \ \ \ \ \ \ \ \ \ \ }
\startm
\m{x}\m{\,y}
\endm
\setbox\startprefix=\hbox{\tt \ \ \ \ \ \ \ \ \ \$e\ }
\setbox\contprefix=\hbox{\tt \ \ \ \ \ \ \ \ \ \ }
\startm
\m{\vdash}\m{(}\m{y}\m{\in}\m{A}\m{\rightarrow}\m{\forall}\m{x}\m{\,y}
\m{\in}\m{A}\m{)}
\endm
\vskip 1ex

should be read ``assume variable $x$ is
(effectively) not free in class $A$.''

\end{itemize}

\section{A Theorem Sampler}\label{sometheorems}

In this section we list some of the more important theorems that are proved in
the \texttt{set.mm} database, and they illustrate the kinds of things that can be
done with Metamath.  While all of these facts are well-known results,
Metamath offers the advantage of easily allowing you to trace their
derivation back to axioms.  Our intent here is not to try to explain the
details or motivation; for this we refer you to the textbooks that are
mentioned in the descriptions.  (The \texttt{set.mm} file has bibliographic
references for the text references.)  Their proofs often embody important
concepts you may wish to explore with the Metamath program (see
Section~\ref{exploring}).  All the symbols that are used here are defined in
Section~\ref{hierarchy}.  For brevity we haven't included the \texttt{\$d}
restrictions or \texttt{\$f} hypotheses for these theorems; when you are
uncertain consult the \texttt{set.mm} database.

We start with \texttt{syl} (principle of the syllogism).
In \textit{Principia Mathematica}
Whitehead and Russell call this ``the principle of the
syllogism... because... the syllogism in Barbara is derived from them''
\cite[quote after Theorem *2.06 p.~101]{PM}.
Some authors call this law a ``hypothetical syllogism.''
At the time of this writing it is the most commonly referenced assertion in the
\texttt{set.mm} database.\footnote{
The Metamath program command \texttt{show usage}
shows the number of references.
}

\vskip 2ex
\noindent Theorem syl (principle of the syllogism)\index{Syllogism}%
\index{\texttt{syl}}\label{syl}.

\vskip 0.5ex
\setbox\startprefix=\hbox{\tt \ \ syl.1\ \$e\ }
\setbox\contprefix=\hbox{\tt \ \ \ \ \ \ \ \ \ \ \ }
\startm
\m{\vdash}\m{(}\m{\varphi}\m{ \rightarrow }\m{\psi}\m{)}
\endm
\setbox\startprefix=\hbox{\tt \ \ syl.2\ \$e\ }
\setbox\contprefix=\hbox{\tt \ \ \ \ \ \ \ \ \ \ \ }
\startm
\m{\vdash}\m{(}\m{\psi}\m{ \rightarrow }\m{\chi}\m{)}
\endm
\setbox\startprefix=\hbox{\tt \ \ syl\ \$p\ }
\setbox\contprefix=\hbox{\tt \ \ \ \ \ \ \ \ \ }
\startm
\m{\vdash}\m{(}\m{\varphi}\m{ \rightarrow }\m{\chi}\m{)}
\endm
\vskip 2ex

The following theorem is not very deep but provides us with a notational device
that is frequently used.  It allows us to use the expression ``$A \in V$'' as
a compact way of saying that class $A$ exists, i.e.\ is a set.

\vskip 2ex
\noindent Two ways to say ``$A$ is a set'':  $A$ is a member of the universe
$V$ if and only if $A$ exists (i.e.\ there exists a set equal to $A$).
Theorem 6.9 of Quine, p. 43.

\vskip 0.5ex
\setbox\startprefix=\hbox{\tt \ \ isset\ \$p\ }
\setbox\contprefix=\hbox{\tt \ \ \ \ \ \ \ \ \ \ \ }
\startm
\m{\vdash}\m{(}\m{A}\m{\in}\m{{\rm V}}\m{\leftrightarrow}\m{\exists}\m{x}\m{\,x}\m{=}
\m{A}\m{)}
\endm
\vskip 1ex

Next we prove the axioms of standard ZF set theory that were missing from our
axiom system.  From our point of view they are theorems since they
can be derived from the other axioms.

\vskip 2ex
\noindent Axiom of Separation\index{Axiom of Separation}
(Aussonderung)\index{Aussonderung} proved from the other axioms of ZF set
theory.  Compare Exercise 4 of Takeuti and Zaring, p.~22.

\vskip 0.5ex
\setbox\startprefix=\hbox{\tt \ \ inex1.1\ \$e\ }
\setbox\contprefix=\hbox{\tt \ \ \ \ \ \ \ \ \ \ \ \ \ \ \ }
\startm
\m{\vdash}\m{A}\m{\in}\m{{\rm V}}
\endm
\setbox\startprefix=\hbox{\tt \ \ inex\ \$p\ }
\setbox\contprefix=\hbox{\tt \ \ \ \ \ \ \ \ \ \ \ \ \ }
\startm
\m{\vdash}\m{(}\m{A}\m{\cap}\m{B}\m{)}\m{\in}\m{{\rm V}}
\endm
\vskip 1ex

\noindent Axiom of the Null Set\index{Axiom of the Null Set} proved from the
other axioms of ZF set theory. Corollary 5.16 of Takeuti and Zaring, p.~20.

\vskip 0.5ex
\setbox\startprefix=\hbox{\tt \ \ 0ex\ \$p\ }
\setbox\contprefix=\hbox{\tt \ \ \ \ \ \ \ \ \ \ \ \ }
\startm
\m{\vdash}\m{\varnothing}\m{\in}\m{{\rm V}}
\endm
\vskip 1ex

\noindent The Axiom of Pairing\index{Axiom of Pairing} proved from the other
axioms of ZF set theory.  Theorem 7.13 of Quine, p.~51.
\vskip 0.5ex
\setbox\startprefix=\hbox{\tt \ \ prex\ \$p\ }
\setbox\contprefix=\hbox{\tt \ \ \ \ \ \ \ \ \ \ \ \ \ \ }
\startm
\m{\vdash}\m{\{}\m{A}\m{,}\m{B}\m{\}}\m{\in}\m{{\rm V}}
\endm
\vskip 2ex

Next we will list some famous or important theorems that are proved in
the \texttt{set.mm} database.  None of them except \texttt{omex}
require the Axiom of Infinity, as you can verify with the \texttt{show
trace{\char`\_}back} Metamath command.

\vskip 2ex
\noindent The resolution of Russell's paradox\index{Russell's paradox}.  There
exists no set containing the set of all sets which are not members of
themselves.  Proposition 4.14 of Takeuti and Zaring, p.~14.

\vskip 0.5ex
\setbox\startprefix=\hbox{\tt \ \ ru\ \$p\ }
\setbox\contprefix=\hbox{\tt \ \ \ \ \ \ \ \ }
\startm
\m{\vdash}\m{\lnot}\m{\exists}\m{x}\m{\,x}\m{=}\m{\{}\m{y}\m{|}\m{\lnot}\m{y}
\m{\in}\m{y}\m{\}}
\endm
\vskip 1ex

\noindent Cantor's theorem\index{Cantor's theorem}.  No set can be mapped onto
its power set.  Compare Theorem 6B(b) of Enderton, p.~132.

\vskip 0.5ex
\setbox\startprefix=\hbox{\tt \ \ canth.1\ \$e\ }
\setbox\contprefix=\hbox{\tt \ \ \ \ \ \ \ \ \ \ \ \ \ }
\startm
\m{\vdash}\m{A}\m{\in}\m{{\rm V}}
\endm
\setbox\startprefix=\hbox{\tt \ \ canth\ \$p\ }
\setbox\contprefix=\hbox{\tt \ \ \ \ \ \ \ \ \ \ \ }
\startm
\m{\vdash}\m{\lnot}\m{F}\m{:}\m{A}\m{\raisebox{.5ex}{${\textstyle{\:}_{
\mbox{\footnotesize\rm {\ }}}}\atop{\textstyle{\longrightarrow}\atop{
\textstyle{}^{\mbox{\footnotesize\rm onto}}}}$}}\m{{\cal P}}\m{A}
\endm
\vskip 1ex

\noindent The Burali-Forti paradox\index{Burali-Forti paradox}.  No set
contains all ordinal numbers. Enderton, p.~194.  (Burali-Forti was one person,
not two.)

\vskip 0.5ex
\setbox\startprefix=\hbox{\tt \ \ onprc\ \$p\ }
\setbox\contprefix=\hbox{\tt \ \ \ \ \ \ \ \ \ \ \ \ }
\startm
\m{\vdash}\m{\lnot}\m{\mbox{\rm On}}\m{\in}\m{{\rm V}}
\endm
\vskip 1ex

\noindent Peano's postulates\index{Peano's postulates} for arithmetic.
Proposition 7.30 of Takeuti and Zaring, pp.~42--43.  The objects being
described are the members of $\omega$ i.e.\ the natural numbers 0, 1,
2,\ldots.  The successor\index{successor} operation suc means ``plus
one.''  \texttt{peano1} says that 0 (which is defined as the empty set)
is a natural number.  \texttt{peano2} says that if $A$ is a natural
number, so is $A+1$.  \texttt{peano3} says that 0 is not the successor
of any natural number.  \texttt{peano4} says that two natural numbers
are equal if and only if their successors are equal.  \texttt{peano5} is
essentially the same as mathematical induction.

\vskip 1ex
\setbox\startprefix=\hbox{\tt \ \ peano1\ \$p\ }
\setbox\contprefix=\hbox{\tt \ \ \ \ \ \ \ \ \ \ \ \ }
\startm
\m{\vdash}\m{\varnothing}\m{\in}\m{\omega}
\endm
\vskip 1.5ex

\setbox\startprefix=\hbox{\tt \ \ peano2\ \$p\ }
\setbox\contprefix=\hbox{\tt \ \ \ \ \ \ \ \ \ \ \ \ }
\startm
\m{\vdash}\m{(}\m{A}\m{\in}\m{\omega}\m{\rightarrow}\m{{\rm suc}}\m{A}\m{\in}%
\m{\omega}\m{)}
\endm
\vskip 1.5ex

\setbox\startprefix=\hbox{\tt \ \ peano3\ \$p\ }
\setbox\contprefix=\hbox{\tt \ \ \ \ \ \ \ \ \ \ \ \ }
\startm
\m{\vdash}\m{(}\m{A}\m{\in}\m{\omega}\m{\rightarrow}\m{\lnot}\m{{\rm suc}}%
\m{A}\m{=}\m{\varnothing}\m{)}
\endm
\vskip 1.5ex

\setbox\startprefix=\hbox{\tt \ \ peano4\ \$p\ }
\setbox\contprefix=\hbox{\tt \ \ \ \ \ \ \ \ \ \ \ \ }
\startm
\m{\vdash}\m{(}\m{(}\m{A}\m{\in}\m{\omega}\m{\wedge}\m{B}\m{\in}\m{\omega}%
\m{)}\m{\rightarrow}\m{(}\m{{\rm suc}}\m{A}\m{=}\m{{\rm suc}}\m{B}\m{%
\leftrightarrow}\m{A}\m{=}\m{B}\m{)}\m{)}
\endm
\vskip 1.5ex

\setbox\startprefix=\hbox{\tt \ \ peano5\ \$p\ }
\setbox\contprefix=\hbox{\tt \ \ \ \ \ \ \ \ \ \ \ \ }
\startm
\m{\vdash}\m{(}\m{(}\m{\varnothing}\m{\in}\m{A}\m{\wedge}\m{\forall}\m{x}\m{%
\in}\m{\omega}\m{(}\m{x}\m{\in}\m{A}\m{\rightarrow}\m{{\rm suc}}\m{x}\m{\in}%
\m{A}\m{)}\m{)}\m{\rightarrow}\m{\omega}\m{\subseteq}\m{A}\m{)}
\endm
\vskip 1.5ex

\noindent Finite Induction (mathematical induction).\index{finite
induction}\index{mathematical induction} The first hypothesis is the
basis and the second is the induction hypothesis.  Theorem Schema 22 of
Suppes, p.~136.

\vskip 0.5ex
\setbox\startprefix=\hbox{\tt \ \ findes.1\ \$e\ }
\setbox\contprefix=\hbox{\tt \ \ \ \ \ \ \ \ \ \ \ \ \ \ }
\startm
\m{\vdash}\m{[}\m{\varnothing}\m{/}\m{x}\m{]}\m{\varphi}
\endm
\setbox\startprefix=\hbox{\tt \ \ findes.2\ \$e\ }
\setbox\contprefix=\hbox{\tt \ \ \ \ \ \ \ \ \ \ \ \ \ \ }
\startm
\m{\vdash}\m{(}\m{x}\m{\in}\m{\omega}\m{\rightarrow}\m{(}\m{\varphi}\m{%
\rightarrow}\m{[}\m{{\rm suc}}\m{x}\m{/}\m{x}\m{]}\m{\varphi}\m{)}\m{)}
\endm
\setbox\startprefix=\hbox{\tt \ \ findes\ \$p\ }
\setbox\contprefix=\hbox{\tt \ \ \ \ \ \ \ \ \ \ \ \ }
\startm
\m{\vdash}\m{(}\m{x}\m{\in}\m{\omega}\m{\rightarrow}\m{\varphi}\m{)}
\endm
\vskip 1ex

\noindent Transfinite Induction with explicit substitution.  The first
hypothesis is the basis, the second is the induction hypothesis for
successors, and the third is the induction hypothesis for limit
ordinals.  Theorem Schema 4 of Suppes, p. 197.

\vskip 0.5ex
\setbox\startprefix=\hbox{\tt \ \ tfindes.1\ \$e\ }
\setbox\contprefix=\hbox{\tt \ \ \ \ \ \ \ \ \ \ \ \ \ \ \ }
\startm
\m{\vdash}\m{[}\m{\varnothing}\m{/}\m{x}\m{]}\m{\varphi}
\endm
\setbox\startprefix=\hbox{\tt \ \ tfindes.2\ \$e\ }
\setbox\contprefix=\hbox{\tt \ \ \ \ \ \ \ \ \ \ \ \ \ \ \ }
\startm
\m{\vdash}\m{(}\m{x}\m{\in}\m{{\rm On}}\m{\rightarrow}\m{(}\m{\varphi}\m{%
\rightarrow}\m{[}\m{{\rm suc}}\m{x}\m{/}\m{x}\m{]}\m{\varphi}\m{)}\m{)}
\endm
\setbox\startprefix=\hbox{\tt \ \ tfindes.3\ \$e\ }
\setbox\contprefix=\hbox{\tt \ \ \ \ \ \ \ \ \ \ \ \ \ \ \ }
\startm
\m{\vdash}\m{(}\m{{\rm Lim}}\m{y}\m{\rightarrow}\m{(}\m{\forall}\m{x}\m{\in}%
\m{y}\m{\varphi}\m{\rightarrow}\m{[}\m{y}\m{/}\m{x}\m{]}\m{\varphi}\m{)}\m{)}
\endm
\setbox\startprefix=\hbox{\tt \ \ tfindes\ \$p\ }
\setbox\contprefix=\hbox{\tt \ \ \ \ \ \ \ \ \ \ \ \ \ }
\startm
\m{\vdash}\m{(}\m{x}\m{\in}\m{{\rm On}}\m{\rightarrow}\m{\varphi}\m{)}
\endm
\vskip 1ex

\noindent Principle of Transfinite Recursion.\index{transfinite
recursion} Theorem 7.41 of Takeuti and Zaring, p.~47.  Transfinite
recursion is the key theorem that allows arithmetic of ordinals to be
rigorously defined, and has many other important uses as well.
Hypotheses \texttt{tfr.1} and \texttt{tfr.2} specify a certain (proper)
class $ F$.  The complicated definition of $ F$ is not important in
itself; what is important is that there be such an $ F$ with the
required properties, and we show this by displaying $ F$ explicitly.
\texttt{tfr1} states that $ F$ is a function whose domain is the set of
ordinal numbers.  \texttt{tfr2} states that any value of $ F$ is
completely determined by its previous values and the values of an
auxiliary function, $G$.  \texttt{tfr3} states that $ F$ is unique,
i.e.\ it is the only function that satisfies \texttt{tfr1} and
\texttt{tfr2}.  Note that $ f$ is an individual variable like $x$ and
$y$; it is just a mnemonic to remind us that $A$ is a collection of
functions.

\vskip 0.5ex
\setbox\startprefix=\hbox{\tt \ \ tfr.1\ \$e\ }
\setbox\contprefix=\hbox{\tt \ \ \ \ \ \ \ \ \ \ \ }
\startm
\m{\vdash}\m{A}\m{=}\m{\{}\m{f}\m{|}\m{\exists}\m{x}\m{\in}\m{{\rm On}}\m{(}%
\m{f}\m{{\rm Fn}}\m{x}\m{\wedge}\m{\forall}\m{y}\m{\in}\m{x}\m{(}\m{f}\m{`}%
\m{y}\m{)}\m{=}\m{(}\m{G}\m{`}\m{(}\m{f}\m{\restriction}\m{y}\m{)}\m{)}\m{)}%
\m{\}}
\endm
\setbox\startprefix=\hbox{\tt \ \ tfr.2\ \$e\ }
\setbox\contprefix=\hbox{\tt \ \ \ \ \ \ \ \ \ \ \ }
\startm
\m{\vdash}\m{F}\m{=}\m{\bigcup}\m{A}
\endm
\setbox\startprefix=\hbox{\tt \ \ tfr1\ \$p\ }
\setbox\contprefix=\hbox{\tt \ \ \ \ \ \ \ \ \ \ }
\startm
\m{\vdash}\m{F}\m{{\rm Fn}}\m{{\rm On}}
\endm
\setbox\startprefix=\hbox{\tt \ \ tfr2\ \$p\ }
\setbox\contprefix=\hbox{\tt \ \ \ \ \ \ \ \ \ \ }
\startm
\m{\vdash}\m{(}\m{z}\m{\in}\m{{\rm On}}\m{\rightarrow}\m{(}\m{F}\m{`}\m{z}%
\m{)}\m{=}\m{(}\m{G}\m{`}\m{(}\m{F}\m{\restriction}\m{z}\m{)}\m{)}\m{)}
\endm
\setbox\startprefix=\hbox{\tt \ \ tfr3\ \$p\ }
\setbox\contprefix=\hbox{\tt \ \ \ \ \ \ \ \ \ \ }
\startm
\m{\vdash}\m{(}\m{(}\m{B}\m{{\rm Fn}}\m{{\rm On}}\m{\wedge}\m{\forall}\m{x}\m{%
\in}\m{{\rm On}}\m{(}\m{B}\m{`}\m{x}\m{)}\m{=}\m{(}\m{G}\m{`}\m{(}\m{B}\m{%
\restriction}\m{x}\m{)}\m{)}\m{)}\m{\rightarrow}\m{B}\m{=}\m{F}\m{)}
\endm
\vskip 1ex

\noindent The existence of omega (the class of natural numbers).\index{natural
number}\index{omega ($\omega$)}\index{Axiom of Infinity}  Axiom 7 of Takeuti
and Zaring, p.~43.  (This is the only theorem in this section requiring the
Axiom of Infinity.)

\vskip 0.5ex
\setbox\startprefix=\hbox{\tt \
\ omex\ \$p\ }
\setbox\contprefix=\hbox{\tt \ \ \ \ \ \ \ \ \ \ }
\startm
\m{\vdash}\m{\omega}\m{\in}\m{{\rm V}}
\endm
%\vskip 2ex


\section{Axioms for Real and Complex Numbers}\label{real}
\index{real and complex numbers!axioms for}

This section presents the axioms for real and complex numbers.  Analysis
textbooks implicitly or explicitly use these axioms or their equivalents
as their starting point.  In the database \texttt{set.mm}, we define real
and complex numbers as (rather complicated) specific sets and derive these
axioms as {\em theorems} from the axioms of ZF set theory, using a method
called Dedekind cuts.  We omit the details of this construction, which you can
follow if you wish using the \texttt{set.mm} database in conjunction with the
textbooks referenced therein.

Once we prove those theorems, we then restate these proven theorems as axioms.
This lets us easily identify which axioms are needed for a particular complex number proof, without the obfuscation of the set theory used to derive them.
As a result,
the construction is actually unimportant other
than to show that sets exist that satisfy the axioms, and thus that the axioms
are consistent if ZF set theory is consistent.  When working with real numbers
you can think of them as being the actual sets resulting
from the construction (for definiteness), or you can
think of them as otherwise unspecified sets that happen to satisfy the axioms.
The derivation is not easy, but the fact that it works is quite remarkable
and lends support to the idea that ZFC set theory is all we need to
provide a foundation for essentially all of mathematics.

For the axioms we are given (or postulate) 8 classes:  $\mathbb{C}$ (the
set of complex numbers), $\mathbb{R}$ (the set of real numbers, a subset
of $\mathbb{C}$), $0$ (zero), $1$ (one), $i$ (square root of
$-1$), $+$ (plus), $\cdot$ (times), and
$<_{\mathbb{R}}$ (less than for just the real numbers).
Subtraction and division are defined terms and are not part of the
axioms; for their definitions see \texttt{set.mm}.

Note that the notation $(A+B)$ (and similarly $(A\cdot B)$) specifies a class
called an {\em operation},\index{operation} and is the function value of the
class $+$ at ordered pair $\langle A,B \rangle$.  An operation is defined by
statement \texttt{df-opr} on p.~\pageref{dfopr}.
The notation $A <_{\mathbb{R}} B$ specifies a
wff called a {\em binary relation}\index{binary relation} and means $\langle A,B \rangle \in \,<_{\mathbb{R}}$, as defined by statement \texttt{df-br} on p.~\pageref{dfbr}.

Our set of 8 given classes is assumed to satisfy the following 22 axioms
(in the axioms listed below, $<$ really means $<_{\mathbb{R}}$).

\vskip 2ex

\noindent 1. The real numbers are a subset of the complex numbers.

%\vskip 0.5ex
\setbox\startprefix=\hbox{\tt \ \ ax-resscn\ \$p\ }
\setbox\contprefix=\hbox{\tt \ \ \ \ \ \ \ \ \ \ \ \ \ \ }
\startm
\m{\vdash}\m{\mathbb{R}}\m{\subseteq}\m{\mathbb{C}}
\endm
%\vskip 1ex

\noindent 2. One is a complex number.

%\vskip 0.5ex
\setbox\startprefix=\hbox{\tt \ \ ax-1cn\ \$p\ }
\setbox\contprefix=\hbox{\tt \ \ \ \ \ \ \ \ \ \ \ }
\startm
\m{\vdash}\m{1}\m{\in}\m{\mathbb{C}}
\endm
%\vskip 1ex

\noindent 3. The imaginary number $i$ is a complex number.

%\vskip 0.5ex
\setbox\startprefix=\hbox{\tt \ \ ax-icn\ \$p\ }
\setbox\contprefix=\hbox{\tt \ \ \ \ \ \ \ \ \ \ \ }
\startm
\m{\vdash}\m{i}\m{\in}\m{\mathbb{C}}
\endm
%\vskip 1ex

\noindent 4. Complex numbers are closed under addition.

%\vskip 0.5ex
\setbox\startprefix=\hbox{\tt \ \ ax-addcl\ \$p\ }
\setbox\contprefix=\hbox{\tt \ \ \ \ \ \ \ \ \ \ \ \ \ }
\startm
\m{\vdash}\m{(}\m{(}\m{A}\m{\in}\m{\mathbb{C}}\m{\wedge}\m{B}\m{\in}\m{\mathbb{C}}%
\m{)}\m{\rightarrow}\m{(}\m{A}\m{+}\m{B}\m{)}\m{\in}\m{\mathbb{C}}\m{)}
\endm
%\vskip 1ex

\noindent 5. Real numbers are closed under addition.

%\vskip 0.5ex
\setbox\startprefix=\hbox{\tt \ \ ax-addrcl\ \$p\ }
\setbox\contprefix=\hbox{\tt \ \ \ \ \ \ \ \ \ \ \ \ \ \ }
\startm
\m{\vdash}\m{(}\m{(}\m{A}\m{\in}\m{\mathbb{R}}\m{\wedge}\m{B}\m{\in}\m{\mathbb{R}}%
\m{)}\m{\rightarrow}\m{(}\m{A}\m{+}\m{B}\m{)}\m{\in}\m{\mathbb{R}}\m{)}
\endm
%\vskip 1ex

\noindent 6. Complex numbers are closed under multiplication.

%\vskip 0.5ex
\setbox\startprefix=\hbox{\tt \ \ ax-mulcl\ \$p\ }
\setbox\contprefix=\hbox{\tt \ \ \ \ \ \ \ \ \ \ \ \ \ }
\startm
\m{\vdash}\m{(}\m{(}\m{A}\m{\in}\m{\mathbb{C}}\m{\wedge}\m{B}\m{\in}\m{\mathbb{C}}%
\m{)}\m{\rightarrow}\m{(}\m{A}\m{\cdot}\m{B}\m{)}\m{\in}\m{\mathbb{C}}\m{)}
\endm
%\vskip 1ex

\noindent 7. Real numbers are closed under multiplication.

%\vskip 0.5ex
\setbox\startprefix=\hbox{\tt \ \ ax-mulrcl\ \$p\ }
\setbox\contprefix=\hbox{\tt \ \ \ \ \ \ \ \ \ \ \ \ \ \ }
\startm
\m{\vdash}\m{(}\m{(}\m{A}\m{\in}\m{\mathbb{R}}\m{\wedge}\m{B}\m{\in}\m{\mathbb{R}}%
\m{)}\m{\rightarrow}\m{(}\m{A}\m{\cdot}\m{B}\m{)}\m{\in}\m{\mathbb{R}}\m{)}
\endm
%\vskip 1ex

\noindent 8. Multiplication of complex numbers is commutative.

%\vskip 0.5ex
\setbox\startprefix=\hbox{\tt \ \ ax-mulcom\ \$p\ }
\setbox\contprefix=\hbox{\tt \ \ \ \ \ \ \ \ \ \ \ \ \ \ }
\startm
\m{\vdash}\m{(}\m{(}\m{A}\m{\in}\m{\mathbb{C}}\m{\wedge}\m{B}\m{\in}\m{\mathbb{C}}%
\m{)}\m{\rightarrow}\m{(}\m{A}\m{\cdot}\m{B}\m{)}\m{=}\m{(}\m{B}\m{\cdot}\m{A}%
\m{)}\m{)}
\endm
%\vskip 1ex

\noindent 9. Addition of complex numbers is associative.

%\vskip 0.5ex
\setbox\startprefix=\hbox{\tt \ \ ax-addass\ \$p\ }
\setbox\contprefix=\hbox{\tt \ \ \ \ \ \ \ \ \ \ \ \ \ \ }
\startm
\m{\vdash}\m{(}\m{(}\m{A}\m{\in}\m{\mathbb{C}}\m{\wedge}\m{B}\m{\in}\m{\mathbb{C}}%
\m{\wedge}\m{C}\m{\in}\m{\mathbb{C}}\m{)}\m{\rightarrow}\m{(}\m{(}\m{A}\m{+}%
\m{B}\m{)}\m{+}\m{C}\m{)}\m{=}\m{(}\m{A}\m{+}\m{(}\m{B}\m{+}\m{C}\m{)}\m{)}%
\m{)}
\endm
%\vskip 1ex

\noindent 10. Multiplication of complex numbers is associative.

%\vskip 0.5ex
\setbox\startprefix=\hbox{\tt \ \ ax-mulass\ \$p\ }
\setbox\contprefix=\hbox{\tt \ \ \ \ \ \ \ \ \ \ \ \ \ \ }
\startm
\m{\vdash}\m{(}\m{(}\m{A}\m{\in}\m{\mathbb{C}}\m{\wedge}\m{B}\m{\in}\m{\mathbb{C}}%
\m{\wedge}\m{C}\m{\in}\m{\mathbb{C}}\m{)}\m{\rightarrow}\m{(}\m{(}\m{A}\m{\cdot}%
\m{B}\m{)}\m{\cdot}\m{C}\m{)}\m{=}\m{(}\m{A}\m{\cdot}\m{(}\m{B}\m{\cdot}\m{C}%
\m{)}\m{)}\m{)}
\endm
%\vskip 1ex

\noindent 11. Multiplication distributes over addition for complex numbers.

%\vskip 0.5ex
\setbox\startprefix=\hbox{\tt \ \ ax-distr\ \$p\ }
\setbox\contprefix=\hbox{\tt \ \ \ \ \ \ \ \ \ \ \ \ \ }
\startm
\m{\vdash}\m{(}\m{(}\m{A}\m{\in}\m{\mathbb{C}}\m{\wedge}\m{B}\m{\in}\m{\mathbb{C}}%
\m{\wedge}\m{C}\m{\in}\m{\mathbb{C}}\m{)}\m{\rightarrow}\m{(}\m{A}\m{\cdot}\m{(}%
\m{B}\m{+}\m{C}\m{)}\m{)}\m{=}\m{(}\m{(}\m{A}\m{\cdot}\m{B}\m{)}\m{+}\m{(}%
\m{A}\m{\cdot}\m{C}\m{)}\m{)}\m{)}
\endm
%\vskip 1ex

\noindent 12. The square of $i$ equals $-1$ (expressed as $i$-squared plus 1 is
0).

%\vskip 0.5ex
\setbox\startprefix=\hbox{\tt \ \ ax-i2m1\ \$p\ }
\setbox\contprefix=\hbox{\tt \ \ \ \ \ \ \ \ \ \ \ \ }
\startm
\m{\vdash}\m{(}\m{(}\m{i}\m{\cdot}\m{i}\m{)}\m{+}\m{1}\m{)}\m{=}\m{0}
\endm
%\vskip 1ex

\noindent 13. One and zero are distinct.

%\vskip 0.5ex
\setbox\startprefix=\hbox{\tt \ \ ax-1ne0\ \$p\ }
\setbox\contprefix=\hbox{\tt \ \ \ \ \ \ \ \ \ \ \ \ }
\startm
\m{\vdash}\m{1}\m{\ne}\m{0}
\endm
%\vskip 1ex

\noindent 14. One is an identity element for real multiplication.

%\vskip 0.5ex
\setbox\startprefix=\hbox{\tt \ \ ax-1rid\ \$p\ }
\setbox\contprefix=\hbox{\tt \ \ \ \ \ \ \ \ \ \ \ }
\startm
\m{\vdash}\m{(}\m{A}\m{\in}\m{\mathbb{R}}\m{\rightarrow}\m{(}\m{A}\m{\cdot}\m{1}%
\m{)}\m{=}\m{A}\m{)}
\endm
%\vskip 1ex

\noindent 15. Every real number has a negative.

%\vskip 0.5ex
\setbox\startprefix=\hbox{\tt \ \ ax-rnegex\ \$p\ }
\setbox\contprefix=\hbox{\tt \ \ \ \ \ \ \ \ \ \ \ \ \ \ }
\startm
\m{\vdash}\m{(}\m{A}\m{\in}\m{\mathbb{R}}\m{\rightarrow}\m{\exists}\m{x}\m{\in}%
\m{\mathbb{R}}\m{(}\m{A}\m{+}\m{x}\m{)}\m{=}\m{0}\m{)}
\endm
%\vskip 1ex

\noindent 16. Every nonzero real number has a reciprocal.

%\vskip 0.5ex
\setbox\startprefix=\hbox{\tt \ \ ax-rrecex\ \$p\ }
\setbox\contprefix=\hbox{\tt \ \ \ \ \ \ \ \ \ \ \ \ \ \ }
\startm
\m{\vdash}\m{(}\m{A}\m{\in}\m{\mathbb{R}}\m{\rightarrow}\m{(}\m{A}\m{\ne}\m{0}%
\m{\rightarrow}\m{\exists}\m{x}\m{\in}\m{\mathbb{R}}\m{(}\m{A}\m{\cdot}%
\m{x}\m{)}\m{=}\m{1}\m{)}\m{)}
\endm
%\vskip 1ex

\noindent 17. A complex number can be expressed in terms of two reals.

%\vskip 0.5ex
\setbox\startprefix=\hbox{\tt \ \ ax-cnre\ \$p\ }
\setbox\contprefix=\hbox{\tt \ \ \ \ \ \ \ \ \ \ \ \ }
\startm
\m{\vdash}\m{(}\m{A}\m{\in}\m{\mathbb{C}}\m{\rightarrow}\m{\exists}\m{x}\m{\in}%
\m{\mathbb{R}}\m{\exists}\m{y}\m{\in}\m{\mathbb{R}}\m{A}\m{=}\m{(}\m{x}\m{+}\m{(}%
\m{y}\m{\cdot}\m{i}\m{)}\m{)}\m{)}
\endm
%\vskip 1ex

\noindent 18. Ordering on reals satisfies strict trichotomy.

%\vskip 0.5ex
\setbox\startprefix=\hbox{\tt \ \ ax-pre-lttri\ \$p\ }
\setbox\contprefix=\hbox{\tt \ \ \ \ \ \ \ \ \ \ \ \ \ }
\startm
\m{\vdash}\m{(}\m{(}\m{A}\m{\in}\m{\mathbb{R}}\m{\wedge}\m{B}\m{\in}\m{\mathbb{R}}%
\m{)}\m{\rightarrow}\m{(}\m{A}\m{<}\m{B}\m{\leftrightarrow}\m{\lnot}\m{(}\m{A}%
\m{=}\m{B}\m{\vee}\m{B}\m{<}\m{A}\m{)}\m{)}\m{)}
\endm
%\vskip 1ex

\noindent 19. Ordering on reals is transitive.

%\vskip 0.5ex
\setbox\startprefix=\hbox{\tt \ \ ax-pre-lttrn\ \$p\ }
\setbox\contprefix=\hbox{\tt \ \ \ \ \ \ \ \ \ \ \ \ \ }
\startm
\m{\vdash}\m{(}\m{(}\m{A}\m{\in}\m{\mathbb{R}}\m{\wedge}\m{B}\m{\in}\m{\mathbb{R}}%
\m{\wedge}\m{C}\m{\in}\m{\mathbb{R}}\m{)}\m{\rightarrow}\m{(}\m{(}\m{A}\m{<}%
\m{B}\m{\wedge}\m{B}\m{<}\m{C}\m{)}\m{\rightarrow}\m{A}\m{<}\m{C}\m{)}\m{)}
\endm
%\vskip 1ex

\noindent 20. Ordering on reals is preserved after addition to both sides.

%\vskip 0.5ex
\setbox\startprefix=\hbox{\tt \ \ ax-pre-ltadd\ \$p\ }
\setbox\contprefix=\hbox{\tt \ \ \ \ \ \ \ \ \ \ \ \ \ }
\startm
\m{\vdash}\m{(}\m{(}\m{A}\m{\in}\m{\mathbb{R}}\m{\wedge}\m{B}\m{\in}\m{\mathbb{R}}%
\m{\wedge}\m{C}\m{\in}\m{\mathbb{R}}\m{)}\m{\rightarrow}\m{(}\m{A}\m{<}\m{B}\m{%
\rightarrow}\m{(}\m{C}\m{+}\m{A}\m{)}\m{<}\m{(}\m{C}\m{+}\m{B}\m{)}\m{)}\m{)}
\endm
%\vskip 1ex

\noindent 21. The product of two positive reals is positive.

%\vskip 0.5ex
\setbox\startprefix=\hbox{\tt \ \ ax-pre-mulgt0\ \$p\ }
\setbox\contprefix=\hbox{\tt \ \ \ \ \ \ \ \ \ \ \ \ \ \ }
\startm
\m{\vdash}\m{(}\m{(}\m{A}\m{\in}\m{\mathbb{R}}\m{\wedge}\m{B}\m{\in}\m{\mathbb{R}}%
\m{)}\m{\rightarrow}\m{(}\m{(}\m{0}\m{<}\m{A}\m{\wedge}\m{0}%
\m{<}\m{B}\m{)}\m{\rightarrow}\m{0}\m{<}\m{(}\m{A}\m{\cdot}\m{B}\m{)}%
\m{)}\m{)}
\endm
%\vskip 1ex

\noindent 22. A non-empty, bounded-above set of reals has a supremum.

%\vskip 0.5ex
\setbox\startprefix=\hbox{\tt \ \ ax-pre-sup\ \$p\ }
\setbox\contprefix=\hbox{\tt \ \ \ \ \ \ \ \ \ \ \ }
\startm
\m{\vdash}\m{(}\m{(}\m{A}\m{\subseteq}\m{\mathbb{R}}\m{\wedge}\m{A}\m{\ne}\m{%
\varnothing}\m{\wedge}\m{\exists}\m{x}\m{\in}\m{\mathbb{R}}\m{\forall}\m{y}\m{%
\in}\m{A}\m{\,y}\m{<}\m{x}\m{)}\m{\rightarrow}\m{\exists}\m{x}\m{\in}\m{%
\mathbb{R}}\m{(}\m{\forall}\m{y}\m{\in}\m{A}\m{\lnot}\m{x}\m{<}\m{y}\m{\wedge}\m{%
\forall}\m{y}\m{\in}\m{\mathbb{R}}\m{(}\m{y}\m{<}\m{x}\m{\rightarrow}\m{\exists}%
\m{z}\m{\in}\m{A}\m{\,y}\m{<}\m{z}\m{)}\m{)}\m{)}
\endm

% NOTE: The \m{...} expressions above could be represented as
% $ \vdash ( ( A \subseteq \mathbb{R} \wedge A \ne \varnothing \wedge \exists x \in \mathbb{R} \forall y \in A \,y < x ) \rightarrow \exists x \in \mathbb{R} ( \forall y \in A \lnot x < y \wedge \forall y \in \mathbb{R} ( y < x \rightarrow \exists z \in A \,y < z ) ) ) $

\vskip 2ex

This completes the set of axioms for real and complex numbers.  You may
wish to look at how subtraction, division, and decimal numbers
are defined in \texttt{set.mm}, and for fun look at the proof of $2+
2 = 4$ (theorem \texttt{2p2e4} in \texttt{set.mm})
as discussed in section \ref{2p2e4}.

In \texttt{set.mm} we define the non-negative integers $\mathbb{N}$, the integers
$\mathbb{Z}$, and the rationals $\mathbb{Q}$ as subsets of $\mathbb{R}$.  This leads
to the nice inclusion $\mathbb{N} \subseteq \mathbb{Z} \subseteq \mathbb{Q} \subseteq
\mathbb{R} \subseteq \mathbb{C}$, giving us a uniform framework in which, for
example, a property such as commutativity of complex number addition
automatically applies to integers.  The natural numbers $\mathbb{N}$
are different from the set $\omega$ we defined earlier, but both satisfy
Peano's postulates.

\subsubsection{Complex Number Axioms in Analysis Texts}

Most analysis texts construct complex numbers as ordered pairs of reals,
leading to construction-dependent properties that satisfy these axioms
but are not stated in their pure form.  (This is also done in
\texttt{set.mm} but our axioms are extracted from that construction.)
Other texts will simply state that $\mathbb{R}$ is a ``complete ordered
subfield of $\mathbb{C}$,'' leading to redundant axioms when this phrase
is completely expanded out.  In fact I have not seen a text with the
axioms in the explicit form above.
None of these axioms is unique individually, but this carefully worked out
collection of axioms is the result of years of work
by the Metamath community.

\subsubsection{Eliminating Unnecessary Complex Number Axioms}

We once had more axioms for real and complex numbers, but over years of time
we (the Metamath community)
have found ways to eliminate them (by proving them from other axioms)
or weaken them (by making weaker claims without reducing what
can be proved).
In particular, here are statements that used to be complex number
axioms but have since been formally proven (with Metamath) to be redundant:

\begin{itemize}
\item
  $\mathbb{C} \in V$.
  At one time this was listed as a ``complex number axiom''
  However, this is not properly speaking a complex number axiom,
  and in any case its proof uses axioms of set theory.
  Proven redundant by Mario Carneiro\index{Carneiro, Mario} on
  17-Nov-2014 (see \texttt{axcnex}).
\item
  $((A \in \mathbb{C} \land B \in \mathbb{C}$) $\rightarrow$
  $(A + B) = (B + A))$.
  Proved redundant by Eric Schmidt\index{Schmidt, Eric} on 19-Jun-2012,
  and formalized by Scott Fenton\index{Fenton, Scott} on 3-Jan-2013
  (see \texttt{addcom}).
\item
  $(A \in \mathbb{C} \rightarrow (A + 0) = A)$.
  Proved redundant by Eric Schmidt on 19-Jun-2012,
  and formalized by Scott Fenton on 3-Jan-2013
  (see \texttt{addid1}).
\item
  $(A \in \mathbb{C} \rightarrow \exists x \in \mathbb{C} (A + x) = 0)$.
  Proved redundant by Eric Schmidt and formalized on 21-May-2007
  (see \texttt{cnegex}).
\item
  $((A \in \mathbb{C} \land A \ne 0) \rightarrow \exists x \in \mathbb{C} (A \cdot x) = 1)$.
  Proved redundant by Eric Schmidt and formalized on 22-May-2007
  (see \texttt{recex}).
\item
  $0 \in \mathbb{R}$.
  Proved redundant by Eric Schmidt on 19-Feb-2005 and formalized 21-May-2007
  (see \texttt{0re}).
\end{itemize}

We could eliminate 0 as an axiomatic object by defining it as
$( ( i \cdot i ) + 1 )$
and replacing it with this expression throughout the axioms. If this
is done, axiom ax-i2m1 becomes redundant. However, the remaining axioms
would become longer and less intuitive.

Eric Schmidt's paper analyzing this axiom system \cite{Schmidt}
presented a proof that these remaining axioms,
with the possible exception of ax-mulcom, are independent of the others.
It is currently an open question if ax-mulcom is independent of the others.

\section{Two Plus Two Equals Four}\label{2p2e4}

Here is a proof that $2 + 2 = 4$, as proven in the theorem \texttt{2p2e4}
in the database \texttt{set.mm}.
This is a useful demonstration of what a Metamath proof can look like.
This proof may have more steps than you're used to, but each step is rigorously
proven all the way back to the axioms of logic and set theory.
This display was originally generated by the Metamath program
as an {\sc HTML} file.

In the table showing the proof ``Step'' is the sequential step number,
while its associated ``Expression'' is an expression that we have proved.
``Ref'' is the name of a theorem or axiom that justifies that expression,
and ``Hyp'' refers to previous steps (if any) that the theorem or axiom
needs so that we can use it.  Expressions are indented further than
the expressions that depend on them to show their interdependencies.

\begin{table}[!htbp]
\caption{Two plus two equals four}
\begin{tabular}{lllll}
\textbf{Step} & \textbf{Hyp} & \textbf{Ref} & \textbf{Expression} & \\
1  &       & df-2    & $ \; \; \vdash 2 = 1 + 1$  & \\
2  & 1     & oveq2i  & $ \; \vdash (2 + 2) = (2 + (1 + 1))$ & \\
3  &       & df-4    & $ \; \; \vdash 4 = (3 + 1)$ & \\
4  &       & df-3    & $ \; \; \; \vdash 3 = (2 + 1)$ & \\
5  & 4     & oveq1i  & $ \; \; \vdash (3 + 1) = ((2 + 1) + 1)$ & \\
6  &       & 2cn     & $ \; \; \; \vdash 2 \in \mathbb{C}$ & \\
7  &       & ax-1cn  & $ \; \; \; \vdash 1 \in \mathbb{C}$ & \\
8  & 6,7,7 & addassi & $ \; \; \vdash ((2 + 1) + 1) = (2 + (1 + 1))$ & \\
9  & 3,5,8 & 3eqtri  & $ \; \vdash 4 = (2 + (1 + 1))$ & \\
10 & 2,9   & eqtr4i  & $ \vdash (2 + 2) = 4$ & \\
\end{tabular}
\end{table}

Step 1 says that we can assert that $2 = 1 + 1$ because it is
justified by \texttt{df-2}.
What is \texttt{df-2}?
It is simply the definition of $2$, which in our system is defined as being
equal to $1 + 1$.  This shows how we can use definitions in proofs.

Look at Step 2 of the proof. In the Ref column, we see that it references
a previously proved theorem, \texttt{oveq2i}.
It turns out that
theorem \texttt{oveq2i} requires a
hypothesis, and in the Hyp column of Step 2 we indicate that Step 1 will
satisfy (match) this hypothesis.
If we looked at \texttt{oveq2i}
we would find that it proves that given some hypothesis
$A = B$, we can prove that $( C F A ) = ( C F B )$.
If we use \texttt{oveq2i} and apply step 1's result as the hypothesis,
that will mean that $A = 2$ and $B = ( 1 + 1 )$ within this use of
\texttt{oveq2i}.
We are free to select any value of $C$ and $F$ (subject to syntax constraints),
so we are free to select $C = 2$ and $F = +$,
producing our desired result,
$ (2 + 2) = (2 + (1 + 1))$.

Step 2 is an example of substitution.
In the end, every step in every proof uses only this one substitution rule.
All the rules of logic, and all the axioms, are expressed so that
they can be used via this one substitution rule.
So once you master substitution, you can master every Metamath proof,
no exceptions.

Each step is clear and can be immediately checked.
In the {\sc HTML} display you can even click on each reference to see why it is
justified, making it easy to see why the proof works.

\section{Deduction}\label{deduction}

Strictly speaking,
a deduction (also called an inference) is a kind of statement that needs
some hypotheses to be true in order for its conclusion to be true.
A theorem, on the other hand, has no hypotheses.
Informally we often call both of them theorems, but in this section we
will stick to the strict definitions.

It sometimes happens that we have proved a deduction of the form
$\varphi \Rightarrow \psi$\index{$\Rightarrow$}
(given hypothesis $\varphi$ we can prove $\psi$)
and we want to then prove a theorem of the form
$\varphi \rightarrow \psi$.

Converting a deduction (which uses a hypothesis) into a theorem
(which does not) is not as simple as you might think.
The deduction says, ``if we can prove $\varphi$ then we can prove $\psi$,''
which is in some sense weaker than saying
``$\varphi$ implies $\psi$.''
There is no axiom of logic that permits us to directly obtain the theorem
given the deduction.\footnote{
The conversion of a deduction to a theorem does not even hold in general
for quantum propositional calculus,
which is a weak subset of classical propositional calculus.
It has been shown that adding the Standard Deduction Theorem (discussed below)
to quantum propositional calculus turns it into classical
propositional calculus!
}

This is in contrast to going the other way.
If we have the theorem ($\varphi \rightarrow \psi$),
it is easy to recover the deduction
($\varphi \Rightarrow \psi$)
using modus ponens\index{modus ponens}
(\texttt{ax-mp}; see section \ref{axmp}).

In the following subsections we first discuss the standard deduction theorem
(the traditional but awkward way to convert deductions into theorems) and
the weak deduction theorem (a limited version of the standard deduction
theorem that is easier to use and was once widely used in
\texttt{set.mm}\index{set theory database (\texttt{set.mm})}).
In section \ref{deductionstyle} we discuss
deduction style, the newer approach we now recommend in most cases.
Deduction style uses ``deduction form,'' a form that
prefixes each hypothesis (other than definitions) and the
conclusion with a universal antecedent (``$\varphi \rightarrow$'').
Deduction style is widely used in \texttt{set.mm},
so it is useful to understand it and \textit{why} it is widely used.
Section \ref{naturaldeduction}
briefly discusses our approach for using natural deduction
within \texttt{set.mm},
as that approach is deeply related to deduction style.
We conclude with a summary of the strengths of
our approach, which we believe are compelling.

\subsection{The Standard Deduction Theorem}\label{standarddeductiontheorem}

It is possible to make use of information
contained in the deduction or its proof to assist us with the proof of
the related theorem.
In traditional logic books, there is a metatheorem called the
Deduction Theorem\index{Deduction Theorem}\index{Standard Deduction Theorem},
discovered independently by Herbrand and Tarski around 1930.
The Deduction Theorem, which we often call the Standard Deduction Theorem,
provides an algorithm for constructing a proof of a theorem from the
proof of its corresponding deduction. See, for example,
\cite[p.~56]{Margaris}\index{Margaris, Angelo}.
To construct a proof for a theorem, the
algorithm looks at each step in the proof of the original deduction and
rewrites the step with several steps wherein the hypothesis is eliminated
and becomes an antecedent.

In ordinary mathematics, no one actually carries out the algorithm,
because (in its most basic form) it involves an exponential explosion of
the number of proof steps as more hypotheses are eliminated. Instead,
the Standard Deduction Theorem is invoked simply to claim that it can
be done in principle, without actually doing it.
What's more, the algorithm is not as simple as it might first appear
when applying it rigorously.
There is a subtle restriction on the Standard Deduction Theorem
that must be taken into account involving the axiom of generalization
when working with predicate calculus (see the literature for more detail).

One of the goals of Metamath is to let you plainly see, with as few
underlying concepts as possible, how mathematics can be derived directly
from the axioms, and not indirectly according to some hidden rules
buried inside a program or understood only by logicians. If we added
the Standard Deduction Theorem to the language and proof verifier,
that would greatly complicate both and largely defeat Metamath's goal
of simplicity. In principle, we could show direct proofs by expanding
out the proof steps generated by the algorithm of the Standard Deduction
Theorem, but that is not feasible in practice because the number of proof
steps quickly becomes huge, even astronomical.
Since the algorithm of the Standard Deduction Theorem is driven by the proof,
we would have to go through that proof
all over again---starting from axioms---in order to obtain the theorem form.
In terms of proof length, there would be no savings over just
proving the theorem directly instead of first proving the deduction form.

\subsection{Weak Deduction Theorem}\label{weakdeductiontheorem}

We have developed
a more efficient method for proving a theorem from a deduction
that can be used instead of the Standard Deduction Theorem
in many (but not all) cases.
We call this more efficient method the
Weak Deduction Theorem\index{Weak Deduction Theorem}.\footnote{
There is also an unrelated ``Weak Deduction Theorem''
in the field of relevance logic, so to avoid confusion we could call
ours the ``Weak Deduction Theorem for Classical Logic.''}
Unlike the Standard Deduction Theorem, the Weak Deduction Theorem produces the
theorem directly from a special substitution instance of the deduction,
using a small, fixed number of steps roughly proportional to the length
of the final theorem.

If you come to a proof referencing the Weak Deduction Theorem
\texttt{dedth} (or one of its variants \texttt{dedthxx}),
here is how to follow the proof without getting into the details:
just click on the theorem referenced in the step
just before the reference to \texttt{dedth} and ignore everything else.
Theorem \texttt{dedth} simply turns a hypothesis into an antecedent
(i.e. the hypothesis followed by $\rightarrow$
is placed in front of the assertion, and the hypothesis
itself is eliminated) given certain conditions.

The Weak Deduction Theorem
eliminates a hypothesis $\varphi$, making it become an antecedent.
It does this by proving an expression
$ \varphi \rightarrow \psi $ given two hypotheses:
(1)
$ ( A = {\rm if} ( \varphi , A , B ) \rightarrow ( \varphi \leftrightarrow \chi ) ) $
and
(2) $\chi$.
Note that it requires that a proof exists for $\varphi$ when the class variable
$A$ is replaced with a specific class $B$. The hypothesis $\chi$
should be assigned to the inference.
You can see the details of the proof of the Weak Deduction Theorem
in theorem \texttt{dedth}.

The Weak Deduction Theorem
is probably easier to understand by studying proofs that make use of it.
For example, let's look at the proof of \texttt{renegcl}, which proves that
$ \vdash ( A \in \mathbb{R} \rightarrow - A \in \mathbb{R} )$:

\needspace{4\baselineskip}
\begin{longtabu} {l l l X}
\textbf{Step} & \textbf{Hyp} & \textbf{Ref} & \textbf{Expression} \\
  1 &  & negeq &
  $\vdash$ $($ $A$ $=$ ${\rm if}$ $($ $A$ $\in$ $\mathbb{R}$ $,$ $A$ $,$ $1$ $)$ $\rightarrow$
  $\textrm{-}$ $A$ $=$ $\textrm{-}$ ${\rm if}$ $($ $A$ $\in$ $\mathbb{R}$
  $,$ $A$ $,$ $1$ $)$ $)$ \\
 2 & 1 & eleq1d &
    $\vdash$ $($ $A$ $=$ ${\rm if}$ $($ $A$ $\in$ $\mathbb{R}$ $,$ $A$ $,$ $1$ $)$ $\rightarrow$ $($
    $\textrm{-}$ $A$ $\in$ $\mathbb{R}$ $\leftrightarrow$
    $\textrm{-}$ ${\rm if}$ $($ $A$ $\in$ $\mathbb{R}$ $,$ $A$ $,$ $1$ $)$ $\in$
    $\mathbb{R}$ $)$ $)$ \\
 3 &  & 1re & $\vdash 1 \in \mathbb{R}$ \\
 4 & 3 & elimel &
   $\vdash {\rm if} ( A \in \mathbb{R} , A , 1 ) \in \mathbb{R}$ \\
 5 & 4 & renegcli &
   $\vdash \textrm{-} {\rm if} ( A \in \mathbb{R} , A , 1 ) \in \mathbb{R}$ \\
 6 & 2,5 & dedth &
   $\vdash ( A \in \mathbb{R} \rightarrow \textrm{-} A \in \mathbb{R}$ ) \\
\end{longtabu}

The somewhat strange-looking steps in \texttt{renegcl} before step 5 are
technical stuff that makes this magic work, and they can be ignored
for a quick overview of the proof. To continue following the ``important''
part of the proof of \texttt{renegcl},
you can look at the reference to \texttt{renegcli} at step 5.

That said, let's briefly look at how
\texttt{renegcl} uses the
Weak Deduction Theorem (\texttt{dedth}) to do its job,
in case you want to do something similar or want understand it more deeply.
Let's work backwards in the proof of \texttt{renegcl}.
Step 6 applies \texttt{dedth} to produce our goal result
$ \vdash ( A \in \mathbb{R} \rightarrow\, - A \in \mathbb{R} )$.
This requires on the one hand the (substituted) deduction
\texttt{renegcli} in step 5.
By itself \texttt{renegcli} proves the deduction
$ \vdash A \in \mathbb{R} \Rightarrow\, \vdash - A \in \mathbb{R}$;
this is the deduction form we are trying to turn into theorem form,
and thus
\texttt{renegcli} has a separate hypothesis that must be fulfilled.
To fulfill the hypothesis of the invocation of
\texttt{renegcli} in step 5, it is eventually
reduced to the already proven theorem $1 \in \mathbb{R}$ in step 3.
Step 4 connects steps 3 and 5; step 4 invokes
\texttt{elimel}, a special case of \texttt{elimhyp} that eliminates
a membership hypothesis for the weak deduction theorem.
On the other hand, the equivalence of the conclusion of
\texttt{renegcl}
$( - A \in \mathbb{R} )$ and the substituted conclusion of
\texttt{renegcli} must be proven, which is done in steps 2 and 1.

The weak deduction theorem has limitations.
In particular, we must be able to prove a special case of the deduction's
hypothesis as a stand-alone theorem.
For example, we used $1 \in \mathbb{R}$ in step 3 of \texttt{renegcl}.

We used to use the weak deduction theorem
extensively within \texttt{set.mm}.
However, we now recommend applying ``deduction style''
instead in most cases, as deduction style is
often an easier and clearer approach.
Therefore, we will now describe deduction style.

\subsection{Deduction Style}\label{deductionstyle}

We now prefer to write assertions in ``deduction form''
instead of writing a proof that would require use of the standard or
weak deduction theorem.
We call this appraoch
``deduction style.''\index{deduction style}

It will be easier to explain this by first defining some terms:

\begin{itemize}
\item \textbf{closed form}\index{closed form}\index{forms!closed}:
A kind of assertion (theorem) with no hypotheses.
Typically its label has no special suffix.
An example is \texttt{unss}, which states:
$\vdash ( ( A \subseteq C \wedge B \subseteq C ) \leftrightarrow ( A \cup B )
\subseteq C )\label{eq:unss}$
\item \textbf{deduction form}\index{deduction form}\index{forms!deduction}:
A kind of assertion with one or more hypotheses
where the conclusion is an implication with
a wff variable as the antecedent (usually $\varphi$), and every hypothesis
(\$e statement)
is either (1) an implication with the same antecedent as the conclusion or
(2) a definition.
A definition
can be for a class variable (this is a class variable followed by ``='')
or a wff variable (this is a wff variable followed by $\leftrightarrow$);
class variable definitions are more common.
In practice, a proof
in deduction form will also contain many steps that are implications
where the antecedent is either that wff variable (normally $\varphi$)
or is
a conjunction (...$\land$...) including that wff variable ($\varphi$).
If an assertion is in deduction form, and other forms are also available,
then we suffix its label with ``d.''
An example is \texttt{unssd}, which states\footnote{
For brevity we show here (and in other places)
a $\&$\index{$\&$} between hypotheses\index{hypotheses}
and a $\Rightarrow$\index{$\Rightarrow$}\index{conclusion}
between the hypotheses and the conclusion.
This notation is technically not part of the Metamath language, but is
instead a convenient abbreviation to show both the hypotheses and conclusion.}:
$\vdash ( \varphi \rightarrow A \subseteq C )\quad\&\quad \vdash ( \varphi
    \rightarrow B \subseteq C )\quad\Rightarrow\quad \vdash ( \varphi
    \rightarrow ( A \cup B ) \subseteq C )\label{eq:unssd}$
\item \textbf{inference form}\index{inference form}\index{forms!inference}:
A kind of assertion with one or more hypotheses that is not in deduction form
(e.g., there is no common antecedent).
If an assertion is in inference form, and other forms are also available,
then we suffix its label with ``i.''
An example is \texttt{unssi}, which states:
$\vdash A \subseteq C\quad\&\quad \vdash B \subseteq C\quad\Rightarrow\quad
    \vdash ( A \cup B ) \subseteq C\label{eq:unssi}$
\end{itemize}

When using deduction style we express an assertion in deduction form.
This form prefixes each hypothesis (other than definitions) and the
conclusion with a universal antecedent (``$\varphi \rightarrow$'').
The antecedent (e.g., $\varphi$)
mimics the context handled in the deduction theorem, eliminating
the need to directly use the deduction theorem.

Once you have an assertion in deduction form, you can easily convert it
to inference form or closed form:

\begin{itemize}
\item To
prove some assertion Ti in inference form, given assertion Td in deduction
form, there is a simple mechanical process you can use. First take each
Ti hypothesis and insert a \texttt{T.} $\rightarrow$ prefix (``true implies'')
using \texttt{a1i}. You
can then use the existing assertion Td to prove the resulting conclusion
with a \texttt{T.} $\rightarrow$ prefix.
Finally, you can remove that prefix using \texttt{trud},
resulting in the conclusion you wanted to prove.
\item To
prove some assertion T in closed form, given assertion Td in deduction
form, there is another simple mechanical process you can use. First,
select an expression that is the conjunction (...$\land$...) of all of the
consequents of every hypothesis of Td. Next, prove that this expression
implies each of the separate hypotheses of Td in turn by eliminating
conjuncts (there are a variety of proven assertions to do this, including
\texttt{simpl},
\texttt{simpr},
\texttt{3simpa},
\texttt{3simpb},
\texttt{3simpc},
\texttt{simp1},
\texttt{simp2},
and
\texttt{simp3}).
If the
expression has nested conjunctions, inner conjuncts can be broken out by
chaining the above theorems with \texttt{syl}
(see section \ref{syl}).\footnote{
There are actually many theorems
(labeled simp* such as \texttt{simp333}) that break out inner conjuncts in one
step, but rather than learning them you can just use the chaining we
just described to prove them, and then let the Metamath program command
\texttt{minimize{\char`\_}with}\index{\texttt{minimize{\char`\_}with} command}
figure out the right ones needed to collapse them.}
As your final step, you can then apply the already-proven assertion Td
(which is in deduction form), proving assertion T in closed form.
\end{itemize}

We can also easily convert any assertion T in closed form to its related
assertion Ti in inference form by applying
modus ponens\index{modus ponens} (see section \ref{axmp}).

The deduction form antecedent can also be used to represent the context
necessary to support natural deduction systems, so we will now
discuss natural deduction.

\subsection{Natural Deduction}\label{naturaldeduction}

Natural deduction\index{natural deduction}
(ND) systems, as such, were originally introduced in
1934 by two logicians working independently: Ja\'skowski and Gentzen. ND
systems are supposed to reconstruct, in a formally proper way, traditional
ways of mathematical reasoning (such as conditional proof, indirect proof,
and proof by cases). As reconstructions they were naturally influenced
by previous work, and many specific ND systems and notations have been
developed since their original work.

There are many ND variants, but
Indrzejczak \cite[p.~31-32]{Indrzejczak}\index{Indrzejczak, Andrzej}
suggests that any natural deductive system must satisfy at
least these three criteria:

\begin{itemize}
\item ``There are some means for entering assumptions into a proof and
also for eliminating them. Usually it requires some bookkeeping devices
for indicating the scope of an assumption, and showing that a part of
a proof depending on eliminated assumption is discharged.
\item There are no (or, at least, a very limited set of) axioms, because
their role is taken over by the set of primitive rules for introduction
and elimination of logical constants which means that elementary
inferences instead of formulae are taken as primitive.
\item (A genuine) ND system admits a lot of freedom in proof construction
and possibility of applying several proof search strategies, like
conditional proof, proof by cases, proof by reductio ad absurdum etc.''
\end{itemize}

The Metamath Proof Explorer (MPE) as defined in \texttt{set.mm}
is fundamentally a Hilbert-style system.
That is, MPE is based on a larger number of axioms (compared
to natural deduction systems), a very small set of rules of inference
(modus ponens), and the context is not changed by the rules of inference
in the middle of a proof. That said, MPE proofs can be developed using
the natural deduction (ND) approach as originally developed by Ja\'skowski
and Gentzen.

The most common and recommended approach for applying ND in MPE is to use
deduction form\index{deduction form}%
\index{forms!deduction}
and apply the MPE proven assertions that are equivalent to ND rules.
For example, MPE's \texttt{jca} is equivalent to ND rule $\land$-I
(and-insertion).
We maintain a list of equivalences that you may consult.
This approach for applying an ND approach within MPE relies on Metamath's
wff metavariables in an essential way, and is described in more detail
in the presentation ``Natural Deductions in the Metamath Proof Language''
by Mario Carneiro \cite{CarneiroND}\index{Carneiro, Mario}.

In this style many steps are an implication, whose antecedent mimics
the context ($\Gamma$) of most ND systems. To add an assumption, simply add
it to the implication antecedent (typically using
\texttt{simpr}),
and use that
new antecedent for all later claims in the same scope. If you wish to
use an assertion in an ND hypothesis scope that is outside the current
ND hypothesis scope, modify the assertion so that the ND hypothesis
assumption is added to its antecedent (typically using \texttt{adantr}). Most
proof steps will be proved using rules that have hypotheses and results
of the form $\varphi \rightarrow$ ...

An example may make this clearer.
Let's show theorem 5.5 of
\cite[p.~18]{Clemente}\index{Clemente Laboreo, Daniel}
along with a line by line translation using the usual
translation of natural deduction (ND) in the Metamath Proof Explorer
(MPE) notation (this is proof \texttt{ex-natded5.5}).
The proof's original goal was to prove
$\lnot \psi$ given two hypotheses,
$( \psi \rightarrow \chi )$ and $ \lnot \chi$.
We will translate these statements into MPE deduction form
by prefixing them all with $\varphi \rightarrow$.
As a result, in MPE the goal is stated as
$( \varphi \rightarrow \lnot \psi )$, and the two hypotheses are stated as
$( \varphi \rightarrow ( \psi \rightarrow \chi ) )$ and
$( \varphi \rightarrow \lnot \chi )$.

The following table shows the proof in Fitch natural deduction style
and its MPE equivalent.
The \textit{\#} column shows the original numbering,
\textit{MPE\#} shows the number in the equivalent MPE proof
(which we will show later),
\textit{ND Expression} shows the original proof claim in ND notation,
and \textit{MPE Translation} shows its translation into MPE
as discussed in this section.
The final columns show the rationale in ND and MPE respectively.

\needspace{4\baselineskip}
{\setlength{\extrarowsep}{4pt} % Keep rows from being too close together
\begin{longtabu}   { @{} c c X X X X }
\textbf{\#} & \textbf{MPE\#} & \textbf{ND Ex\-pres\-sion} &
\textbf{MPE Trans\-lation} & \textbf{ND Ration\-ale} &
\textbf{MPE Ra\-tio\-nale} \\
\endhead

1 & 2;3 &
$( \psi \rightarrow \chi )$ &
$( \varphi \rightarrow ( \psi \rightarrow \chi ) )$ &
Given &
\$e; \texttt{adantr} to put in ND hypothesis \\

2 & 5 &
$ \lnot \chi$ &
$( \varphi \rightarrow \lnot \chi )$ &
Given &
\$e; \texttt{adantr} to put in ND hypothesis \\

3 & 1 &
... $\vert$ $\psi$ &
$( \varphi \rightarrow \psi )$ &
ND hypothesis assumption &
\texttt{simpr} \\

4 & 4 &
... $\chi$ &
$( ( \varphi \land \psi ) \rightarrow \chi )$ &
$\rightarrow$\,E 1,3 &
\texttt{mpd} 1,3 \\

5 & 6 &
... $\lnot \chi$ &
$( ( \varphi \land \psi ) \rightarrow \lnot \chi )$ &
IT 2 &
\texttt{adantr} 5 \\

6 & 7 &
$\lnot \psi$ &
$( \varphi \rightarrow \lnot \psi )$ &
$\land$\,I 3,4,5 &
\texttt{pm2.65da} 4,6 \\

\end{longtabu}
}


The original used Latin letters; we have replaced them with Greek letters
to follow Metamath naming conventions and so that it is easier to follow
the Metamath translation. The Metamath line-for-line translation of
this natural deduction approach precedes every line with an antecedent
including $\varphi$ and uses the Metamath equivalents of the natural deduction
rules. To add an assumption, the antecedent is modified to include it
(typically by using \texttt{adantr};
\texttt{simpr} is useful when you want to
depend directly on the new assumption, as is shown here).

In Metamath we can represent the two given statements as these hypotheses:

\needspace{2\baselineskip}
\begin{itemize}
\item ex-natded5.5.1 $\vdash ( \varphi \rightarrow ( \psi \rightarrow \chi ) )$
\item ex-natded5.5.2 $\vdash ( \varphi \rightarrow \lnot \chi )$
\end{itemize}

\needspace{4\baselineskip}
Here is the proof in Metamath as a line-by-line translation:

\begin{longtabu}   { l l l X }
\textbf{Step} & \textbf{Hyp} & \textbf{Ref} & \textbf{Ex\-pres\-sion} \\
\endhead
1 & & simpr & $\vdash ( ( \varphi \land \psi ) \rightarrow \psi )$ \\
2 & & ex-natded5.5.1 &
  $\vdash ( \varphi \rightarrow ( \psi \rightarrow \chi ) )$ \\
3 & 2 & adantr &
 $\vdash ( ( \varphi \land \psi ) \rightarrow ( \psi \rightarrow \chi ) )$ \\
4 & 1, 3 & mpd &
 $\vdash ( ( \varphi \land \psi ) \rightarrow \chi ) $ \\
5 & & ex-natded5.5.2 &
 $\vdash ( \varphi \rightarrow \lnot \chi )$ \\
6 & 5 & adantr &
 $\vdash ( ( \varphi \land \psi ) \rightarrow \lnot \chi )$ \\
7 & 4, 6 & pm2.65da &
 $\vdash ( \varphi \rightarrow \lnot \psi )$ \\
\end{longtabu}

Only using specific natural deduction rules directly can lead to very
long proofs, for exactly the same reason that only using axioms directly
in Hilbert-style proofs can lead to very long proofs.
If the goal is short and clear proofs,
then it is better to reuse already-proven assertions
in deduction form than to start from scratch each time
and using only basic natural deduction rules.

\subsection{Strengths of Our Approach}

As far as we know there is nothing else in the literature like either the
weak deduction theorem or Mario Carneiro\index{Carneiro, Mario}'s
natural deduction method.
In order to
transform a hypothesis into an antecedent, the literature's standard
``Deduction Theorem''\index{Deduction Theorem}\index{Standard Deduction Theorem}
requires metalogic outside of the notions provided
by the axiom system. We instead generally prefer to use Mario Carneiro's
natural deduction method, then use the weak deduction theorem in cases
where that is difficult to apply, and only then use the full standard
deduction theorem as a last resort.

The weak deduction theorem\index{Weak Deduction Theorem}
does not require any additional metalogic
but converts an inference directly into a closed form theorem, with
a rigorous proof that uses only the axiom system. Unlike the standard
Deduction Theorem, there is no implicit external justification that we
have to trust in order to use it.

Mario Carneiro's natural deduction\index{natural deduction}
method also does not require any new metalogical
notions. It avoids the Deduction Theorem's metalogic by prefixing the
hypotheses and conclusion of every would-be inference with a universal
antecedent (``$\varphi \rightarrow$'') from the very start.

We think it is impressive and satisfying that we can do so much in a
practical sense without stepping outside of our Hilbert-style axiom system.
Of course our axiomatization, which is in the form of schemes,
contains a metalogic of its own that we exploit. But this metalogic
is relatively simple, and for our Deduction Theorem alternatives,
we primarily use just the direct substitution of expressions for
metavariables.

\begin{sloppy}
\section{Exploring the Set The\-o\-ry Data\-base}\label{exploring}
\end{sloppy}
% NOTE: All examples performed in this section are
% recorded wtih "set width 61" % on set.mm as of 2019-05-28
% commit c1e7849557661260f77cfdf0f97ac4354fbb4f4d.

At this point you may wish to study the \texttt{set.mm}\index{set theory
database (\texttt{set.mm})} file in more detail.  Pay particular
attention to the assumptions needed to define wffs\index{well-formed
formula (wff)} (which are not included above), the variable types
(\texttt{\$f}\index{\texttt{\$f} statement} statements), and the
definitions that are introduced.  Start with some simple theorems in
propositional calculus, making sure you understand in detail each step
of a proof.  Once you get past the first few proofs and become familiar
with the Metamath language, any part of the \texttt{set.mm} database
will be as easy to follow, step by step, as any other part---you won't
have to undergo a ``quantum leap'' in mathematical sophistication to be
able to follow a deep proof in set theory.

Next, you may want to explore how concepts such as natural numbers are
defined and described.  This is probably best done in conjunction with
standard set theory textbooks, which can help give you a higher-level
understanding.  The \texttt{set.mm} database provides references that will get
you started.  From there, you will be on your way towards a very deep,
rigorous understanding of abstract mathematics.

The Metamath\index{Metamath} program can help you peruse a Metamath data\-base,
wheth\-er you are trying to figure out how a certain step follows in a proof or
just have a general curiosity.  We will go through some examples of the
commands, using the \texttt{set.mm}\index{set theory database (\texttt{set.mm})}
database provided with the Metamath software.  These should help get you
started.  See Chapter~\ref{commands} for a more detailed description of
the commands.  Note that we have included the full spelling of all commands to
prevent ambiguity with future commands.  In practice you may type just the
characters needed to specify each command keyword\index{command keyword}
unambiguously, often just one or two characters per keyword, and you don't
need to type them in upper case.

First run the Metamath program as described earlier.  You should see the
\verb/MM>/ prompt.  Read in the \texttt{set.mm} file:\index{\texttt{read}
command}

\begin{verbatim}
MM> read set.mm
Reading source file "set.mm"... 34554442 bytes
34554442 bytes were read into the source buffer.
The source has 155711 statements; 2254 are $a and 32250 are $p.
No errors were found.  However, proofs were not checked.
Type VERIFY PROOF * if you want to check them.
\end{verbatim}

As with most examples in this book, what you will see
will be slightly different because we are continuously
improving our databases (including \texttt{set.mm}).

Let's check the database integrity.  This may take a minute or two to run if
your computer is slow.

\begin{verbatim}
MM> verify proof *
0 10%  20%  30%  40%  50%  60%  70%  80%  90% 100%
..................................................
All proofs in the database were verified in 2.84 s.
\end{verbatim}

No errors were reported, so every proof is correct.

You need to know the names (labels) of theorems before you can look at them.
Often just examining the database file(s) with a text editor is the best
approach.  In \texttt{set.mm} there are many detailed comments, especially near
the beginning, that can help guide you. The \texttt{search} command in the
Metamath program is also handy.  The \texttt{comments} qualifier will list the
statements whose associated comment (the one immediately before it) contain a
string you give it.  For example, if you are studying Enderton's {\em Elements
of Set Theory} \cite{Enderton}\index{Enderton, Herbert B.} you may want to see
the references to it in the database.  The search string \texttt{enderton} is not
case sensitive.  (This will not show you all the database theorems that are in
Enderton's book because there is usually only one citation for a given
theorem, which may appear in several textbooks.)\index{\texttt{search}
command}

\begin{verbatim}
MM> search * "enderton" / comments
12067 unineq $p "... Exercise 20 of [Enderton] p. 32 and ..."
12459 undif2 $p "...Corollary 6K of [Enderton] p. 144. (C..."
12953 df-tp $a "...s. Definition of [Enderton] p. 19. (Co..."
13689 unissb $p ".... Exercise 5 of [Enderton] p. 26 and ..."
\end{verbatim}
\begin{center}
(etc.)
\end{center}

Or you may want to see what theorems have something to do with
conjunction (logical {\sc and}).  The quotes around the search
string are optional when there's no ambiguity.\index{\texttt{search}
command}

\begin{verbatim}
MM> search * conjunction / comments
120 a1d $p "...be replaced with a conjunction ( ~ df-an )..."
662 df-bi $a "...viated form after conjunction is introdu..."
1319 wa $a "...ff definition to include conjunction ('and')."
1321 df-an $a "Define conjunction (logical 'and'). Defini..."
1420 imnan $p "...tion in terms of conjunction. (Contribu..."
\end{verbatim}
\begin{center}
(etc.)
\end{center}


Now we will start to look at some details.  Let's look at the first
axiom of propositional calculus
(we could use \texttt{sh st} to abbreviate
\texttt{show statement}).\index{\texttt{show statement} command}

\begin{verbatim}
MM> show statement ax-1/full
Statement 19 is located on line 881 of the file "set.mm".
"Axiom _Simp_.  Axiom A1 of [Margaris] p. 49.  One of the 3
axioms of propositional calculus.  The 3 axioms are also
        ...
19 ax-1 $a |- ( ph -> ( ps -> ph ) ) $.
Its mandatory hypotheses in RPN order are:
  wph $f wff ph $.
  wps $f wff ps $.
The statement and its hypotheses require the variables:  ph
      ps
The variables it contains are:  ph ps


Statement 49 is located on line 11182 of the file "set.mm".
Its statement number for HTML pages is 6.
"Axiom _Simp_.  Axiom A1 of [Margaris] p. 49.  One of the 3
axioms of propositional calculus.  The 3 axioms are also
given as Definition 2.1 of [Hamilton] p. 28.
...
49 ax-1 $a |- ( ph -> ( ps -> ph ) ) $.
Its mandatory hypotheses in RPN order are:
  wph $f wff ph $.
  wps $f wff ps $.
The statement and its hypotheses require the variables:
  ph ps
The variables it contains are:  ph ps
\end{verbatim}

Compare this to \texttt{ax-1} on p.~\pageref{ax1}.  You can see that the
symbol \texttt{ph} is the {\sc ascii} notation for $\varphi$, etc.  To
see the mathematical symbols for any expression you may typeset it in
\LaTeX\ (type \texttt{help tex} for instructions)\index{latex@{\LaTeX}}
or, easier, just use a text editor to look at the comments where symbols
are first introduced in \texttt{set.mm}.  The hypotheses \texttt{wph}
and \texttt{wps} required by \texttt{ax-1} mean that variables
\texttt{ph} and \texttt{ps} must be wffs.

Next we'll pick a simple theorem of propositional calculus, the Principle of
Identity, which is proved directly from the axioms.  We'll look at the
statement then its proof.\index{\texttt{show statement}
command}

\begin{verbatim}
MM> show statement id1/full
Statement 116 is located on line 11371 of the file "set.mm".
Its statement number for HTML pages is 22.
"Principle of identity.  Theorem *2.08 of [WhiteheadRussell]
p. 101.  This version is proved directly from the axioms for
demonstration purposes.
...
116 id1 $p |- ( ph -> ph ) $= ... $.
Its mandatory hypotheses in RPN order are:
  wph $f wff ph $.
Its optional hypotheses are:  wps wch wth wta wet
      wze wsi wrh wmu wla wka
The statement and its hypotheses require the variables:  ph
These additional variables are allowed in its proof:
      ps ch th ta et ze si rh mu la ka
The variables it contains are:  ph
\end{verbatim}

The optional variables\index{optional variable} \texttt{ps}, \texttt{ch}, etc.\ are
available for use in a proof of this statement if we wish, and were we to do
so we would make use of optional hypotheses \texttt{wps}, \texttt{wch}, etc.  (See
Section~\ref{dollaref} for the meaning of ``optional
hypothesis.''\index{optional hypothesis}) The reason these show up in the
statement display is that statement \texttt{id1} happens to be in their scope
(see Section~\ref{scoping} for the definition of ``scope''\index{scope}), but
in fact in propositional calculus we will never make use of optional
hypotheses or variables.  This becomes important after quantifiers are
introduced, where ``dummy'' variables\index{dummy variable} are often needed
in the middle of a proof.

Let's look at the proof of statement \texttt{id1}.  We'll use the
\texttt{show proof} command, which by default suppresses the
``non-essential'' steps that construct the wffs.\index{\texttt{show proof}
command}
We will display the proof in ``lemmon' format (a non-indented format
with explicit previous step number references) and renumber the
displayed steps:

\begin{verbatim}
MM> show proof id1 /lemmon/renumber
1 ax-1           $a |- ( ph -> ( ph -> ph ) )
2 ax-1           $a |- ( ph -> ( ( ph -> ph ) -> ph ) )
3 ax-2           $a |- ( ( ph -> ( ( ph -> ph ) -> ph ) ) ->
                     ( ( ph -> ( ph -> ph ) ) -> ( ph -> ph )
                                                          ) )
4 2,3 ax-mp      $a |- ( ( ph -> ( ph -> ph ) ) -> ( ph -> ph
                                                          ) )
5 1,4 ax-mp      $a |- ( ph -> ph )
\end{verbatim}

If you have read Section~\ref{trialrun}, you'll know how to interpret this
proof.  Step~2, for example, is an application of axiom \texttt{ax-1}.  This
proof is identical to the one in Hamilton's {\em Logic for Mathematicians}
\cite[p.~32]{Hamilton}\index{Hamilton, Alan G.}.

You may want to look at what
substitutions\index{substitution!variable}\index{variable substitution} are
made into \texttt{ax-1} to arrive at step~2.  The command to do this needs to
know the ``real'' step number, so we'll display the proof again without
the \texttt{renumber} qualifier.\index{\texttt{show proof}
command}

\begin{verbatim}
MM> show proof id1 /lemmon
 9 ax-1          $a |- ( ph -> ( ph -> ph ) )
20 ax-1          $a |- ( ph -> ( ( ph -> ph ) -> ph ) )
24 ax-2          $a |- ( ( ph -> ( ( ph -> ph ) -> ph ) ) ->
                     ( ( ph -> ( ph -> ph ) ) -> ( ph -> ph )
                                                          ) )
25 20,24 ax-mp   $a |- ( ( ph -> ( ph -> ph ) ) -> ( ph -> ph
                                                          ) )
26 9,25 ax-mp    $a |- ( ph -> ph )
\end{verbatim}

The ``real'' step number is 20.  Let's look at its details.

\begin{verbatim}
MM> show proof id1 /detailed_step 20
Proof step 20:  min=ax-1 $a |- ( ph -> ( ( ph -> ph ) -> ph )
  )
This step assigns source "ax-1" ($a) to target "min" ($e).
The source assertion requires the hypotheses "wph" ($f, step
18) and "wps" ($f, step 19).  The parent assertion of the
target hypothesis is "ax-mp" ($a, step 25).
The source assertion before substitution was:
    ax-1 $a |- ( ph -> ( ps -> ph ) )
The following substitutions were made to the source
assertion:
    Variable  Substituted with
     ph        ph
     ps        ( ph -> ph )
The target hypothesis before substitution was:
    min $e |- ph
The following substitution was made to the target hypothesis:
    Variable  Substituted with
     ph        ( ph -> ( ( ph -> ph ) -> ph ) )
\end{verbatim}

This shows the substitutions\index{substitution!variable}\index{variable
substitution} made to the variables in \texttt{ax-1}.  References are made to
steps 18 and 19 which are not shown in our proof display.  To see these steps,
you can display the proof with the \texttt{all} qualifier.

Let's look at a slightly more advanced proof of propositional calculus.  Note
that \verb+/\+ is the symbol for $\wedge$ (logical {\sc and}, also
called conjunction).\index{conjunction ($\wedge$)}
\index{logical {\sc and} ($\wedge$)}

\begin{verbatim}
MM> show statement prth/full
Statement 1791 is located on line 15503 of the file "set.mm".
Its statement number for HTML pages is 559.
"Conjoin antecedents and consequents of two premises.  This
is the closed theorem form of ~ anim12d .  Theorem *3.47 of
[WhiteheadRussell] p. 113.  It was proved by Leibniz,
and it evidently pleased him enough to call it
_praeclarum theorema_ (splendid theorem).
...
1791 prth $p |- ( ( ( ph -> ps ) /\ ( ch -> th ) ) -> ( ( ph
      /\ ch ) -> ( ps /\ th ) ) ) $= ... $.
Its mandatory hypotheses in RPN order are:
  wph $f wff ph $.
  wps $f wff ps $.
  wch $f wff ch $.
  wth $f wff th $.
Its optional hypotheses are:  wta wet wze wsi wrh wmu wla wka
The statement and its hypotheses require the variables:  ph
      ps ch th
These additional variables are allowed in its proof:  ta et
      ze si rh mu la ka
The variables it contains are:  ph ps ch th


MM> show proof prth /lemmon/renumber
1 simpl          $p |- ( ( ( ph -> ps ) /\ ( ch -> th ) ) ->
                                               ( ph -> ps ) )
2 simpr          $p |- ( ( ( ph -> ps ) /\ ( ch -> th ) ) ->
                                               ( ch -> th ) )
3 1,2 anim12d    $p |- ( ( ( ph -> ps ) /\ ( ch -> th ) ) ->
                           ( ( ph /\ ch ) -> ( ps /\ th ) ) )
\end{verbatim}

There are references to a lot of unfamiliar statements.  To see what they are,
you may type the following:

\begin{verbatim}
MM> show proof prth /statement_summary
Summary of statements used in the proof of "prth":

Statement simpl is located on line 14748 of the file
"set.mm".
"Elimination of a conjunct.  Theorem *3.26 (Simp) of
[WhiteheadRussell] p. 112. ..."
  simpl $p |- ( ( ph /\ ps ) -> ph ) $= ... $.

Statement simpr is located on line 14777 of the file
"set.mm".
"Elimination of a conjunct.  Theorem *3.27 (Simp) of
[WhiteheadRussell] ..."
  simpr $p |- ( ( ph /\ ps ) -> ps ) $= ... $.

Statement anim12d is located on line 15445 of the file
"set.mm".
"Conjoin antecedents and consequents in a deduction.
..."
  anim12d.1 $e |- ( ph -> ( ps -> ch ) ) $.
  anim12d.2 $e |- ( ph -> ( th -> ta ) ) $.
  anim12d $p |- ( ph -> ( ( ps /\ th ) -> ( ch /\ ta ) ) )
      $= ... $.
\end{verbatim}
\begin{center}
(etc.)
\end{center}

Of course you can look at each of these statements and their proofs, and
so on, back to the axioms of propositional calculus if you wish.

The \texttt{search} command is useful for finding statements when you
know all or part of their contents.  The following example finds all
statements containing \verb@ph -> ps@ followed by \verb@ch -> th@.  The
\verb@$*@ is a wildcard that matches anything; the \texttt{\$} before the
\verb$*$ prevents conflicts with math symbol token names.  The \verb@*@ after
\texttt{SEARCH} is also a wildcard that in this case means ``match any label.''
\index{\texttt{search} command}

% I'm omitting this one, since readers are unlikely to see it:
% 1096 bisymOLD $p |- ( ( ( ph -> ps ) -> ( ch -> th ) ) -> ( (
%   ( ps -> ph ) -> ( th -> ch ) ) -> ( ( ph <-> ps ) -> ( ch
%    <-> th ) ) ) )
\begin{verbatim}
MM> search * "ph -> ps $* ch -> th"
1791 prth $p |- ( ( ( ph -> ps ) /\ ( ch -> th ) ) -> ( ( ph
    /\ ch ) -> ( ps /\ th ) ) )
2455 pm3.48 $p |- ( ( ( ph -> ps ) /\ ( ch -> th ) ) -> ( (
    ph \/ ch ) -> ( ps \/ th ) ) )
117859 pm11.71 $p |- ( ( E. x ph /\ E. y ch ) -> ( ( A. x (
    ph -> ps ) /\ A. y ( ch -> th ) ) <-> A. x A. y ( ( ph /\
    ch ) -> ( ps /\ th ) ) ) )
\end{verbatim}

Three statements, \texttt{prth}, \texttt{pm3.48},
 and \texttt{pm11.71}, were found to match.

To see what axioms\index{axiom} and definitions\index{definition}
\texttt{prth} ultimately depends on for its proof, you can have the
program backtrack through the hierarchy\index{hierarchy} of theorems and
definitions.\index{\texttt{show trace{\char`\_}back} command}

\begin{verbatim}
MM> show trace_back prth /essential/axioms
Statement "prth" assumes the following axioms ($a
statements):
  ax-1 ax-2 ax-3 ax-mp df-bi df-an
\end{verbatim}

Note that the 3 axioms of propositional calculus and the modus ponens rule are
needed (as expected); in addition, there are a couple of definitions that are used
along the way.  Note that Metamath makes no distinction\index{axiom vs.\
definition} between axioms\index{axiom} and definitions\index{definition}.  In
\texttt{set.mm} they have been distinguished artificially by prefixing their
labels\index{labels in \texttt{set.mm}} with \texttt{ax-} and \texttt{df-}
respectively.  For example, \texttt{df-an} defines conjunction (logical {\sc
and}), which is represented by the symbol \verb+/\+.
Section~\ref{definitions} discusses the philosophy of definitions, and the
Metamath language takes a particularly simple, conservative approach by using
the \texttt{\$a}\index{\texttt{\$a} statement} statement for both axioms and
definitions.

You can also have the program compute how many steps a proof
has\index{proof length} if we were to follow it all the way back to
\texttt{\$a} statements.

\begin{verbatim}
MM> show trace_back prth /essential/count_steps
The statement's actual proof has 3 steps.  Backtracking, a
total of 79 different subtheorems are used.  The statement
and subtheorems have a total of 274 actual steps.  If
subtheorems used only once were eliminated, there would be a
total of 38 subtheorems, and the statement and subtheorems
would have a total of 185 steps.  The proof would have 28349
steps if fully expanded back to axiom references.  The
maximum path length is 38.  A longest path is:  prth <-
anim12d <- syl2and <- sylan2d <- ancomsd <- ancom <- pm3.22
<- pm3.21 <- pm3.2 <- ex <- sylbir <- biimpri <- bicomi <-
bicom1 <- bi2 <- dfbi1 <- impbii <- bi3 <- simprim <- impi <-
con1i <- nsyl2 <- mt3d <- con1d <- notnot1 <- con2i <- nsyl3
<- mt2d <- con2d <- notnot2 <- pm2.18d <- pm2.18 <- pm2.21 <-
pm2.21d <- a1d <- syl <- mpd <- a2i <- a2i.1 .
\end{verbatim}

This tells us that we would have to inspect 274 steps if we want to
verify the proof completely starting from the axioms.  A few more
statistics are also shown.  There are one or more paths back to axioms
that are the longest; this command ferrets out one of them and shows it
to you.  There may be a sense in which the longest path length is
related to how ``deep'' the theorem is.

We might also be curious about what proofs depend on the theorem
\texttt{prth}.  If it is never used later on, we could eliminate it as
redundant if it has no intrinsic interest by itself.\index{\texttt{show
usage} command}

% I decided to show the OLD values here.
\begin{verbatim}
MM> show usage prth
Statement "prth" is directly referenced in the proofs of 18
statements:
  mo3 moOLD 2mo 2moOLD euind reuind reuss2 reusv3i opelopabt
  wemaplem2 rexanre rlimcn2 o1of2 o1rlimmul 2sqlem6 spanuni
  heicant pm11.71
\end{verbatim}

Thus \texttt{prth} is directly used by 18 proofs.
We can use the \texttt{/recursive} qualifier to include indirect use:

\begin{verbatim}
MM> show usage prth /recursive
Statement "prth" directly or indirectly affects the proofs of
24214 statements:
  mo3 mo mo3OLD eu2 moOLD eu2OLD eu3OLD mo4f mo4 eu4 mopick
...
\end{verbatim}

\subsection{A Note on the ``Compact'' Proof Format}

The present version of Metamath\index{Metamath!limitations of
version 0.07.30} will display proofs in a ``compact''\index{compact proof}
format whenever the proof is stored in compressed format in the database.  It
may be be slightly confusing unless you know how to interpret it.  (A
future version may eliminate this format from displays.)  For example,
if you display the complete proof of theorem \texttt{id1} it will start
off as follows:

\begin{verbatim}
MM> show proof id1 /lemmon/all
 1 wph           $f wff ph
 2 wph           $f wff ph
 3 wph           $f wff ph
 4 2,3 wi    @4: $a wff ( ph -> ph )
 5 1,4 wi    @5: $a wff ( ph -> ( ph -> ph ) )
 6 @4            $a wff ( ph -> ph )
\end{verbatim}

\begin{center}
{etc.}
\end{center}

Step 4 has a ``local label,''\index{local label} \texttt{@4}, assigned to it.
Later on, at step 6, the label \texttt{@1} is referenced instead of
displaying the explicit proof for that step.  This technique takes advantage
of the fact that steps in a proof often repeat, especially during the
construction of wffs.  The compact format reduces the number of steps in the
proof display and may actually be preferred by some people, which is one
reason that it is still displayed by the program.

If you want to see the normal format with the ``true'' step numbers, you can
use the following workaround:\index{\texttt{save proof} command}

\begin{verbatim}
MM> save proof id1 /normal
The proof of "id1" has been reformatted and saved internally.
Remember to use WRITE SOURCE to save it permanently.
MM> show proof id1 /lemmon/all
 1 wph           $f wff ph
 2 wph           $f wff ph
 3 wph           $f wff ph
 4 2,3 wi        $a wff ( ph -> ph )
 5 1,4 wi        $a wff ( ph -> ( ph -> ph ) )
 6 wph           $f wff ph
 7 wph           $f wff ph
 8 6,7 wi        $a wff ( ph -> ph )
\end{verbatim}

\begin{center}
{etc.}
\end{center}

Note that the original 6 steps are now 8 steps.  However, the format is
now the same as that described in Chapter~\ref{using}.

\chapter{The Metamath Language}
\label{languagespec}

\begin{quote}
  {\em Thus mathematics may be defined as the subject in which we never know
what we are talking about, nor whether what we are saying is true.}
    \flushright\sc  Bertrand Russell\footnote{\cite[p.~84]{Russell2}.}\\
\end{quote}\index{Russell, Bertrand}

Probably the most striking feature of the Metamath language is its almost
complete absence of hard-wired syntax. Metamath\index{Metamath} does not
understand any mathematics or logic other than that needed to construct finite
sequences of symbols according to a small set of simple, built-in rules.  The
only rule it uses in a proof is the substitution of an expression (symbol
sequence) for a variable, subject to a simple constraint to prevent
bound-variable clashes.  The primitive notions built into Metamath involve the
simple manipulation of finite objects (symbols) that we as humans can easily
visualize and that computers can easily deal with.  They seem to be just
about the simplest notions possible that are required to do standard
mathematics.

This chapter serves as a reference manual for the Metamath\index{Metamath}
language. It covers the tedious technical details of the language, some of
which you may wish to skim in a first reading.  On the other hand, you should
pay close attention to the defined terms in {\bf boldface}; they have precise
meanings that are important to keep in mind for later understanding.  It may
be best to first become familiar with the examples in Chapter~\ref{using} to
gain some motivation for the language.

%% Uncomment this when uncommenting section {formalspec} below
If you have some knowledge of set theory, you may wish to study this
chapter in conjunction with the formal set-theoretical description of the
Metamath language in Appendix~\ref{formalspec}.

We will use the name ``Metamath''\index{Metamath} to mean either the Metamath
computer language or the Metamath software associated with the computer
language.  We will not distinguish these two when the context is clear.

The next section contains the complete specification of the Metamath
language.
It serves as an
authoritative reference and presents the syntax in enough detail to
write a parser\index{parsing Metamath} and proof verifier.  The
specification is terse and it is probably hard to learn the language
directly from it, but we include it here for those impatient people who
prefer to see everything up front before looking at verbose expository
material.  Later sections explain this material and provide examples.
We will repeat the definitions in those sections, and you may skip the
next section at first reading and proceed to Section~\ref{tut1}
(p.~\pageref{tut1}).

\section{Specification of the Metamath Language}\label{spec}
\index{Metamath!specification}

\begin{quote}
  {\em Sometimes one has to say difficult things, but one ought to say
them as simply as one knows how.}
    \flushright\sc  G. H. Hardy\footnote{As quoted in
    \cite{deMillo}, p.~273.}\\
\end{quote}\index{Hardy, G. H.}

\subsection{Preliminaries}\label{spec1}

% Space is technically a printable character, so we'll word things
% carefully so it's unambiguous.
A Metamath {\bf database}\index{database} is built up from a top-level source
file together with any source files that are brought in through file inclusion
commands (see below).  The only characters that are allowed to appear in a
Metamath source file are the 94 non-whitespace printable {\sc
ascii}\index{ascii@{\sc ascii}} characters, which are digits, upper and lower
case letters, and the following 32 special
characters\index{special characters}:\label{spec1chars}

\begin{verbatim}
! " # $ % & ' ( ) * + , - . / :
; < = > ? @ [ \ ] ^ _ ` { | } ~
\end{verbatim}
plus the following characters which are the ``white space'' characters:
space (a printable character),
tab, carriage return, line feed, and form feed.\label{whitespace}
We will use \texttt{typewriter}
font to display the printable characters.

A Metamath database consists of a sequence of three kinds of {\bf
tokens}\index{token} separated by {\bf white space}\index{white space}
(which is any sequence of one or more white space characters).  The set
of {\bf keyword}\index{keyword} tokens is \texttt{\$\char`\{},
\texttt{\$\char`\}}, \texttt{\$c}, \texttt{\$v}, \texttt{\$f},
\texttt{\$e}, \texttt{\$d}, \texttt{\$a}, \texttt{\$p}, \texttt{\$.},
\texttt{\$=}, \texttt{\$(}, \texttt{\$)}, \texttt{\$[}, and
\texttt{\$]}.  The last four are called {\bf auxiliary}\index{auxiliary
keyword} or preprocessing keywords.  A {\bf label}\index{label} token
consists of any combination of letters, digits, and the characters
hyphen, underscore, and period.  A {\bf math symbol}\index{math symbol}
token may consist of any combination of the 92 printable standard {\sc
ascii} characters other than space or \texttt{\$}~. All tokens are
case-sensitive.

\subsection{Preprocessing}

The token \texttt{\$(} begins a {\bf comment} and
\texttt{\$)} ends a comment.\index{\texttt{\$(}
and \texttt{\$)} auxiliary keywords}\index{comment}
Comments may contain any of
the 94 non-whitespace printable characters and white space,
except they may not contain the
2-character sequences \texttt{\$(} or \texttt{\$)} (comments do not nest).
Comments are ignored (treated
like white space) for the purpose of parsing, e.g.,
\texttt{\$( \$[ \$)} is a comment.
See p.~\pageref{mathcomments} for comment typesetting conventions; these
conventions may be ignored for the purpose of parsing.

A {\bf file inclusion command} consists of \texttt{\$[} followed by a file name
followed by \texttt{\$]}.\index{\texttt{\$[} and \texttt{\$]} auxiliary
keywords}\index{included file}\index{file inclusion}
It is only allowed in the outermost scope (i.e., not between
\texttt{\$\char`\{} and \texttt{\$\char`\}})
and must not be inside a statement (e.g., it may not occur
between the label of a \texttt{\$a} statement and its \texttt{\$.}).
The file name may not
contain a \texttt{\$} or white space.  The file must exist.
The case-sensitivity
of its name follows the conventions of the operating system.  The contents of
the file replace the inclusion command.
Included files may include other files.
Only the first reference to a given file is included; any later
references to the same file (whether in the top-level file or in included
files) cause the inclusion command to be ignored (treated like white space).
A verifier may assume that file names with different strings
refer to different files for the purpose of ignoring later references.
A file self-reference is ignored, as is any reference to the top-level file
(to avoid loops).
Included files may not include a \texttt{\$(} without a matching \texttt{\$)},
may not include a \texttt{\$[} without a matching \texttt{\$]}, and may
not include incomplete statements (e.g., a \texttt{\$a} without a matching
\texttt{\$.}).
It is currently unspecified if path references are relative to the process'
current directory or the file's containing directory, so databases should
avoid using pathname separators (e.g., ``/'') in file names.

Like all tokens, the \texttt{\$(}, \texttt{\$)}, \texttt{\$[}, and \texttt{\$]} keywords
must be surrounded by white space.

\subsection{Basic Syntax}

After preprocessing, a database will consist of a sequence of {\bf
statements}.
These are the scoping statements \texttt{\$\char`\{} and
\texttt{\$\char`\}}, along with the \texttt{\$c}, \texttt{\$v},
\texttt{\$f}, \texttt{\$e}, \texttt{\$d}, \texttt{\$a}, and \texttt{\$p}
statements.

A {\bf scoping statement}\index{scoping statement} consists only of its
keyword, \texttt{\$\char`\{} or \texttt{\$\char`\}}.
A \texttt{\$\char`\{} begins a {\bf
block}\index{block} and a matching \texttt{\$\char`\}} ends the block.
Every \texttt{\$\char`\{}
must have a matching \texttt{\$\char`\}}.
Defining it recursively, we say a block
contains a sequence of zero or more tokens other
than \texttt{\$\char`\{} and \texttt{\$\char`\}} and
possibly other blocks.  There is an {\bf outermost
block}\index{block!outermost} not bracketed by \texttt{\$\char`\{} \texttt{\$\char`\}}; the end
of the outermost block is the end of the database.

% LaTeX bug? can't do \bf\tt

A {\bf \$v} or {\bf \$c statement}\index{\texttt{\$v} statement}\index{\texttt{\$c}
statement} consists of the keyword token \texttt{\$v} or \texttt{\$c} respectively,
followed by one or more math symbols,
% The word "token" is used to distinguish "$." from the sentence-ending period.
followed by the \texttt{\$.}\ token.
These
statements {\bf declare}\index{declaration} the math symbols to be {\bf
variables}\index{variable!Metamath} or {\bf constants}\index{constant}
respectively. The same math symbol may not occur twice in a given \texttt{\$v} or
\texttt{\$c} statement.

%c%A math symbol becomes an {\bf active}\index{active math symbol}
%c%when declared and stays active until the end of the block in which it is
%c%declared.  A math symbol may not be declared a second time while it is active,
%c%but it may be declared again after it becomes inactive.

A math symbol becomes {\bf active}\index{active math symbol} when declared
and stays active until the end of the block in which it is declared.  A
variable may not be declared a second time while it is active, but it
may be declared again (as a variable, but not as a constant) after it
becomes inactive.  A constant must be declared in the outermost block and may
not be declared a second time.\footnote{The rules for redeclaration may
become more general in the future;
see p.~\pageref{redeclarationf}.}\index{redeclaration of symbols}

A {\bf \$f statement}\index{\texttt{\$f} statement} consists of a label,
followed by \texttt{\$f}, followed by its typecode (an active constant),
followed by an
active variable, followed by the \texttt{\$.}\ token.  A {\bf \$e
statement}\index{\texttt{\$e} statement} consists of a label, followed
by \texttt{\$e}, followed by its typecode (an active constant),
followed by zero or more
active math symbols, followed by the \texttt{\$.}\ token.  A {\bf
hypothesis}\index{hypothesis} is a \texttt{\$f} or \texttt{\$e}
statement.
The type declared by a \texttt{\$f} statement for a given label
is global even if the variable is not
(e.g., a database may not have \texttt{wff P} in one local scope
nd \texttt{class P} in another).

A {\bf simple \$d statement}\index{\texttt{\$d} statement!simple}
consists of \texttt{\$d}, followed by two different active variables,
followed by the \texttt{\$.}\ token.  A {\bf compound \$d
statement}\index{\texttt{\$d} statement!compound} consists of
\texttt{\$d}, followed by three or more variables (all different),
followed by the \texttt{\$.}\ token.  The order of the variables in a
\texttt{\$d} statement is unimportant.  A compound \texttt{\$d}
statement is equivalent to a set of simple \texttt{\$d} statements, one
for each possible pair of variables occurring in the compound
\texttt{\$d} statement.  Henceforth in this specification we shall
assume all \texttt{\$d} statements are simple.  A \texttt{\$d} statement
is also called a {\bf disjoint} (or {\bf distinct}) {\bf variable
restriction}.\index{disjoint-variable restriction}

A {\bf \$a statement}\index{\texttt{\$a} statement} consists of a label,
followed by \texttt{\$a}, followed by its typecode (an active constant),
followed by
zero or more active math symbols, followed by the \texttt{\$.}\ token.  A {\bf
\$p statement}\index{\texttt{\$p} statement} consists of a label,
followed by \texttt{\$p}, followed by its typecode (an active constant),
followed by
zero or more active math symbols, followed by \texttt{\$=}, followed by
a sequence of labels, followed by the \texttt{\$.}\ token.  An {\bf
assertion}\index{assertion} is a \texttt{\$a} or \texttt{\$p} statement.

A \texttt{\$f}, \texttt{\$e}, or \texttt{\$d} statement is {\bf active}\index{active
statement} from the place it occurs until the end of the block it occurs in.
A \texttt{\$a} or \texttt{\$p} statement is {\bf active} from the place it occurs
through the end of the database.
There may not be two active \texttt{\$f} statements containing the same
variable.  Each variable in a \texttt{\$e}, \texttt{\$a}, or
\texttt{\$p} statement must exist in an active \texttt{\$f}
statement.\footnote{This requirement can greatly simplify the
unification algorithm (substitution calculation) required by proof
verification.}

%The label that begins each \texttt{\$f}, \texttt{\$e}, \texttt{\$a}, and
%\texttt{\$p} statement must be unique.
Each label token must be unique, and
no label token may match any math symbol
token.\label{namespace}\footnote{This
restriction was added on June 24, 2006.
It is not theoretically necessary but is imposed to make it easier to
write certain parsers.}

The set of {\bf mandatory variables}\index{mandatory variable} associated with
an assertion is the set of (zero or more) variables in the assertion and in any
active \texttt{\$e} statements.  The (possibly empty) set of {\bf mandatory
hypotheses}\index{mandatory hypothesis} is the set of all active \texttt{\$f}
statements containing mandatory variables, together with all active \texttt{\$e}
statements.
The set of {\bf mandatory {\bf \$d} statements}\index{mandatory
disjoint-variable restriction} associated with an assertion are those active
\texttt{\$d} statements whose variables are both among the assertion's
mandatory variables.

\subsection{Proof Verification}\label{spec4}

The sequence of labels between the \texttt{\$=} and \texttt{\$.}\ tokens
in a \texttt{\$p} statement is a {\bf proof}.\index{proof!Metamath} Each
label in a proof must be the label of an active statement other than the
\texttt{\$p} statement itself; thus a label must refer either to an
active hypothesis of the \texttt{\$p} statement or to an earlier
assertion.

An {\bf expression}\index{expression} is a sequence of math symbols. A {\bf
substitution map}\index{substitution map} associates a set of variables with a
set of expressions.  It is acceptable for a variable to be mapped to an
expression containing it.  A {\bf
substitution}\index{substitution!variable}\index{variable substitution} is the
simultaneous replacement of all variables in one or more expressions with the
expressions that the variables map to.

A proof is scanned in order of its label sequence.  If the label refers to an
active hypothesis, the expression in the hypothesis is pushed onto a
stack.\index{stack}\index{RPN stack}  If the label refers to an assertion, a
(unique) substitution must exist that, when made to the mandatory hypotheses
of the referenced assertion, causes them to match the topmost (i.e.\ most
recent) entries of the stack, in order of occurrence of the mandatory
hypotheses, with the topmost stack entry matching the last mandatory
hypothesis of the referenced assertion.  As many stack entries as there are
mandatory hypotheses are then popped from the stack.  The same substitution is
made to the referenced assertion, and the result is pushed onto the stack.
After the last label in the proof is processed, the stack must have a single
entry that matches the expression in the \texttt{\$p} statement containing the
proof.

%c%{\footnotesize\begin{quotation}\index{redeclaration of symbols}
%c%{{\em Comment.}\label{spec4comment} Whenever a math symbol token occurs in a
%c%{\texttt{\$c} or \texttt{\$v} statement, it is considered to designate a distinct new
%c%{symbol, even if the same token was previously declared (and is now inactive).
%c%{Thus a math token declared as a constant in two different blocks is considered
%c%{to designate two distinct constants (even though they have the same name).
%c%{The two constants will not match in a proof that references both blocks.
%c%{However, a proof referencing both blocks is acceptable as long as it doesn't
%c%{require that the constants match.  Similarly, a token declared to be a
%c%{constant for a referenced assertion will not match the same token declared to
%c%{be a variable for the \texttt{\$p} statement containing the proof.  In the case
%c%{of a token declared to be a variable for a referenced assertion, this is not
%c%{an issue since the variable can be substituted with whatever expression is
%c%{needed to achieve the required match.
%c%{\end{quotation}}
%c2%A proof may reference an assertion that contains or whose hypotheses contain a
%c2%constant that is not active for the \texttt{\$p} statement containing the proof.
%c2%However, the final result of the proof may not contain that constant. A proof
%c2%may also reference an assertion that contains or whose hypotheses contain a
%c2%variable that is not active for the \texttt{\$p} statement containing the proof.
%c2%That variable, of course, will be substituted with whatever expression is
%c2%needed to achieve the required match.

A proof may contain a \texttt{?}\ in place of a label to indicate an unknown step
(Section~\ref{unknown}).  A proof verifier may ignore any proof containing
\texttt{?}\ but should warn the user that the proof is incomplete.

A {\bf compressed proof}\index{compressed proof}\index{proof!compressed} is an
alternate proof notation described in Appen\-dix~\ref{compressed}; also see
references to ``compressed proof'' in the Index.  Compressed proofs are a
Metamath language extension which a complete proof verifier should be able to
parse and verify.

\subsubsection{Verifying Disjoint Variable Restrictions}

Each substitution made in a proof must be checked to verify that any
disjoint variable restrictions are satisfied, as follows.

If two variables replaced by a substitution exist in a mandatory \texttt{\$d}
statement\index{\texttt{\$d} statement} of the assertion referenced, the two
expressions resulting from the substitution must satisfy the following
conditions.  First, the two expressions must have no variables in common.
Second, each possible pair of variables, one from each expression, must exist
in an active \texttt{\$d} statement of the \texttt{\$p} statement containing the
proof.

\vskip 1ex

This ends the specification of the Metamath language;
see Appendix \ref{BNF} for its syntax in
Extended Backus--Naur Form (EBNF)\index{Extended Backus--Naur Form}\index{EBNF}.

\section{The Basic Keywords}\label{tut1}

Our expository material begins here.

Like most computer languages, Metamath\index{Metamath} takes its input from
one or more {\bf source files}\index{source file} which contain characters
expressed in the standard {\sc ascii} (American Standard Code for Information
Interchange)\index{ascii@{\sc ascii}} code for computers.  A source file
consists of a series of {\bf tokens}\index{token}, which are strings of
non-whitespace
printable characters (from the set of 94 shown on p.~\pageref{spec1chars})
separated by {\bf white space}\index{white space} (spaces, tabs, carriage
returns, line feeds, and form feeds). Any string consisting only of these
characters is treated the same as a single space.  The non-whitespace printable
characters\index{printable character} that Metamath recognizes are the 94
characters on standard {\sc ascii} keyboards.

Metamath has the ability to join several files together to form its
input (Section~\ref{include}).  We call the aggregate contents of all
the files after they have been joined together a {\bf
database}\index{database} to distinguish it from an individual source
file.  The tokens in a database consist of {\bf
keywords}\index{keyword}, which are built into the language, together
with two kinds of user-defined tokens called {\bf labels}\index{label}
and {\bf math symbols}\index{math symbol}.  (Often we will simply say
{\bf symbol}\index{symbol} instead of math symbol for brevity).  The set
of {\bf basic keywords}\index{basic keyword} is
\texttt{\$c}\index{\texttt{\$c} statement},
\texttt{\$v}\index{\texttt{\$v} statement},
\texttt{\$e}\index{\texttt{\$e} statement},
\texttt{\$f}\index{\texttt{\$f} statement},
\texttt{\$d}\index{\texttt{\$d} statement},
\texttt{\$a}\index{\texttt{\$a} statement},
\texttt{\$p}\index{\texttt{\$p} statement},
\texttt{\$=}\index{\texttt{\$=} keyword},
\texttt{\$.}\index{\texttt{\$.}\ keyword},
\texttt{\$\char`\{}\index{\texttt{\$\char`\{} and \texttt{\$\char`\}}
keywords}, and \texttt{\$\char`\}}.  This is the complete set of
syntactical elements of what we call the {\bf basic
language}\index{basic language} of Metamath, and with them you can
express all of the mathematics that were intended by the design of
Metamath.  You should make it a point to become very familiar with them.
Table~\ref{basickeywords} lists the basic keywords along with a brief
description of their functions.  For now, this description will give you
only a vague notion of what the keywords are for; later we will describe
the keywords in detail.


\begin{table}[htp] \caption{Summary of the basic Metamath
keywords} \label{basickeywords}
\begin{center}
\begin{tabular}{|p{4pc}|l|}
\hline
\em \centering Keyword&\em Description\\
\hline
\hline
\centering
   \texttt{\$c}&Constant symbol declaration\\
\hline
\centering
   \texttt{\$v}&Variable symbol declaration\\
\hline
\centering
   \texttt{\$d}&Disjoint variable restriction\\
\hline
\centering
   \texttt{\$f}&Variable-type (``floating'') hypothesis\\
\hline
\centering
   \texttt{\$e}&Logical (``essential'') hypothesis\\
\hline
\centering
   \texttt{\$a}&Axiomatic assertion\\
\hline
\centering
   \texttt{\$p}&Provable assertion\\
\hline
\centering
   \texttt{\$=}&Start of proof in \texttt{\$p} statement\\
\hline
\centering
   \texttt{\$.}&End of the above statement types\\
\hline
\centering
   \texttt{\$\char`\{}&Start of block\\
\hline
\centering
   \texttt{\$\char`\}}&End of block\\
\hline
\end{tabular}
\end{center}
\end{table}

%For LaTeX bug(?) where it puts tables on blank page instead of btwn text
%May have to adjust if text changes
%\newpage

There are some additional keywords, called {\bf auxiliary
keywords}\index{auxiliary keyword} that help make Metamath\index{Metamath}
more practical. These are part of the {\bf extended language}\index{extended
language}. They provide you with a means to put comments into a Metamath
source file\index{source file} and reference other source files.  We will
introduce these in later sections. Table~\ref{otherkeywords} summarizes them
so that you can recognize them now if you want to peruse some source
files while learning the basic keywords.


\begin{table}[htp] \caption{Auxiliary Metamath
keywords} \label{otherkeywords}
\begin{center}
\begin{tabular}{|p{4pc}|l|}
\hline
\em \centering Keyword&\em Description\\
\hline
\hline
\centering
   \texttt{\$(}&Start of comment\\
\hline
\centering
   \texttt{\$)}&End of comment\\
\hline
\centering
   \texttt{\$[}&Start of included source file name\\
\hline
\centering
   \texttt{\$]}&End of included source file name\\
\hline
\end{tabular}
\end{center}
\end{table}
\index{\texttt{\$(} and \texttt{\$)} auxiliary keywords}
\index{\texttt{\$[} and \texttt{\$]} auxiliary keywords}


Unlike those in some computer languages, the keywords\index{keyword} are short
two-character sequences rather than English-like words.  While this may make
them slightly more difficult to remember at first, their brevity allows
them to blend in with the mathematics being described, not
distract from it, like punctuation marks.


\subsection{User-Defined Tokens}\label{dollardollar}\index{token}

As you may have noticed, all keywords\index{keyword} begin with the \texttt{\$}
character.  This mundane monetary symbol is not ordinarily used in higher
mathematics (outside of grant proposals), so we have appropriated it to
distinguish the Metamath\index{Metamath} keywords from ordinary mathematical
symbols. The \texttt{\$} character is thus considered special and may not be
used as a character in a user-defined token.  All tokens and keywords are
case-sensitive; for example, \texttt{n} is considered to be a different character
from \texttt{N}.  Case-sensitivity makes the available {\sc ascii} character set
as rich as possible.

\subsubsection{Math Symbol Tokens}\index{token}

Math symbols\index{math symbol} are tokens used to represent the symbols
that appear in ordinary mathematical formulas.  They may consist of any
combination of the 93 non-whitespace printable {\sc ascii} characters other than
\texttt{\$}~. Some examples are \texttt{x}, \texttt{+}, \texttt{(},
\texttt{|-}, \verb$!%@?&$, and \texttt{bounded}.  For readability, it is
best to try to make these look as similar to actual mathematical symbols
as possible, within the constraints of the {\sc ascii} character set, in
order to make the resulting mathematical expressions more readable.

In the Metamath\index{Metamath} language, you express ordinary
mathematical formulas and statements as sequences of math symbols such
as \texttt{2 + 2 = 4} (five symbols, all constants).\footnote{To
eliminate ambiguity with other expressions, this is expressed in the set
theory database \texttt{set.mm} as \texttt{|- ( 2 + 2
 ) = 4 }, whose \LaTeX\ equivalent is $\vdash
(2+2)=4$.  The \,$\vdash$ means ``is a theorem'' and the
parentheses allow explicit associative grouping.}\index{turnstile
({$\,\vdash$})} They may even be English
sentences, as in \texttt{E is closed and bounded} (five symbols)---here
\texttt{E} would be a variable and the other four symbols constants.  In
principle, a Metamath database could be constructed to work with almost
any unambiguous English-language mathematical statement, but as a
practical matter the definitions needed to provide for all possible
syntax variations would be cumbersome and distracting and possibly have
subtle pitfalls accidentally built in.  We generally recommend that you
express mathematical statements with compact standard mathematical
symbols whenever possible and put their English-language descriptions in
comments.  Axioms\index{axiom} and definitions\index{definition}
(\texttt{\$a}\index{\texttt{\$a} statement} statements) are the only
places where Metamath will not detect an error, and doing this will help
reduce the number of definitions needed.

You are free to use any tokens\index{token} you like for math
symbols\index{math symbol}.  Appendix~\ref{ASCII} recommends token names to
use for symbols in set theory, and we suggest you adopt these in order to be
able to include the \texttt{set.mm} set theory database in your database.  For
printouts, you can convert the tokens in a database
to standard mathematical symbols with the \LaTeX\ typesetting program.  The
Metamath command \texttt{open tex} {\em filename}\index{\texttt{open tex} command}
produces output that can be read by \LaTeX.\index{latex@{\LaTeX}}
The correspondence
between tokens and the actual symbols is made by \texttt{latexdef}
statements inside a special database comment tagged
with \texttt{\$t}.\index{\texttt{\$t} comment}\index{typesetting comment}
  You can edit
this comment to change the definitions or add new ones.
Appendix~\ref{ASCII} describes how to do this in more detail.

% White space\index{white space} is normally used to separate math
% symbol\index{math symbol} tokens, but they may be juxtaposed without white
% space in \texttt{\$d}\index{\texttt{\$d} statement}, \texttt{\$e}\index{\texttt{\$e}
% statement}, \texttt{\$f}\index{\texttt{\$f} statement}, \texttt{\$a}\index{\texttt{\$a}
% statement}, and \texttt{\$p}\index{\texttt{\$p} statement} statements when no
% ambiguity will result.  Specifically, Metamath parses the math symbol sequence
% in one of these statements in the following manner:  when the math symbol
% sequence has been broken up into tokens\index{token} up to a given character,
% the next token is the longest string of characters that could constitute a
% math symbol that is active\index{active
% math symbol} at that point.  (See Section~\ref{scoping} for the
% definition of an active math symbol.)  For example, if \texttt{-}, \texttt{>}, and
% \texttt{->} are the only active math symbols, the juxtaposition \texttt{>-} will be
% interpreted as the two symbols \texttt{>} and \texttt{-}, whereas \texttt{->} will
% always be interpreted as that single symbol.\footnote{For better readability we
% recommend a white space between each token.  This also makes searching for a
% symbol easier to do with an editor.  Omission of optional white space is useful
% for reducing typing when assigning an expression to a temporary
% variable\index{temporary variable} with the \texttt{let variable} Metamath
% program command.}\index{\texttt{let variable} command}
%
% Keywords\index{keyword} may be placed next to math symbols without white
% space\index{white space} between them.\footnote{Again, we do not recommend
% this for readability.}
%
% The math symbols\index{math symbol} in \texttt{\$c}\index{\texttt{\$c} statement}
% and \texttt{\$v}\index{\texttt{\$v} statement} statements must always be separated
% by white space\index{white
% space}, for the obvious reason that these statements define the names
% of the symbols.
%
% Math symbols referred to in comments (see Section~\ref{comments}) must also be
% separated by white space.  This allows you to make comments about symbols that
% are not yet active\index{active
% math symbol}.  (The ``math mode'' feature of comments is also a quick and
% easy way to obtain word processing text with embedded mathematical symbols,
% independently of the main purpose of Metamath; the way to do this is described
% in Section~\ref{comments})

\subsubsection{Label Tokens}\index{token}\index{label}

Label tokens are used to identify Metamath\index{Metamath} statements for
later reference. Label tokens may contain only letters, digits, and the three
characters period, hyphen, and underscore:
\begin{verbatim}
. - _
\end{verbatim}

A label is {\bf declared}\index{label declaration} by placing it immediately
before the keyword of the statement it identifies.  For example, the label
\texttt{axiom.1} might be declared as follows:
\begin{verbatim}
axiom.1 $a |- x = x $.
\end{verbatim}

Each \texttt{\$e}\index{\texttt{\$e} statement},
\texttt{\$f}\index{\texttt{\$f} statement},
\texttt{\$a}\index{\texttt{\$a} statement}, and
\texttt{\$p}\index{\texttt{\$p} statement} statement in a database must
have a label declared for it.  No other statement types may have label
declarations.  Every label must be unique.

A label (and the statement it identifies) is {\bf referenced}\index{label
reference} by including the label between the \texttt{\$=}\index{\texttt{\$=}
keyword} and \texttt{\$.}\index{\texttt{\$.}\ keyword}\ keywords in a \texttt{\$p}
statement.  The sequence of labels\index{label sequence} between these two
keywords is called a {\bf proof}\index{proof}.  An example of a statement with
a proof that we will encounter later (Section~\ref{proof}) is
\begin{verbatim}
wnew $p wff ( s -> ( r -> p ) )
     $= ws wr wp w2 w2 $.
\end{verbatim}

You don't have to know what this means just yet, but you should know that the
label \texttt{wnew} is declared by this \texttt{\$p} statement and that the labels
\texttt{ws}, \texttt{wr}, \texttt{wp}, and \texttt{w2} are assumed to have been declared
earlier in the database and are referenced here.

\subsection{Constants and Variables}
\index{constant}
\index{variable}

An {\bf expression}\index{expression} is any sequence of math
symbols, possibly empty.

The basic Metamath\index{Metamath} language\index{basic language} has two
kinds of math symbols\index{math symbol}:  {\bf constants}\index{constant} and
{\bf variables}\index{variable}.  In a Metamath proof, a constant may not be
substituted with any expression.  A variable can be
substituted\index{substitution!variable}\index{variable substitution} with any
expression.  This sequence may include other variables and may even include
the variable being substituted.  This substitution takes place when proofs are
verified, and it will be described in Section~\ref{proof}.  The \texttt{\$f}
statement (described later in Section~\ref{dollaref}) is used to specify the
{\bf type} of a variable (i.e.\ what kind of
variable it is)\index{variable type}\index{type} and
give it a meaning typically
associated with a ``metavariable''\index{metavariable}\footnote{A metavariable
is a variable that ranges over the syntactical elements of the object language
being discussed; for example, one metavariable might represent a variable of
the object language and another metavariable might represent a formula in the
object language.} in ordinary mathematics; for example, a variable may be
specified to be a wff or well-formed formula (in logic), a set (in set
theory), or a non-negative integer (in number theory).

%\subsection{The \texttt{\$c} and \texttt{\$v} Declaration Statements}
\subsection{The \texttt{\$c} and \texttt{\$v} Declaration Statements}
\index{\texttt{\$c} statement}
\index{constant declaration}
\index{\texttt{\$v} statement}
\index{variable declaration}

Constants are introduced or {\bf declared}\index{constant declaration}
with \texttt{\$c}\index{\texttt{\$c} statement} statements, and
variables are declared\index{variable declaration} with
\texttt{\$v}\index{\texttt{\$v} statement} statements.  A {\bf simple}
declaration\index{simple declaration} statement introduces a single
constant or variable.  Its syntax is one of the following:
\begin{center}
  \texttt{\$c} {\em math-symbol} \texttt{\$.}\\
  \texttt{\$v} {\em math-symbol} \texttt{\$.}
\end{center}
The notation {\em math-symbol} means any math symbol token\index{token}.

Some examples of simple declaration statements are:
\begin{center}
  \texttt{\$c + \$.}\\
  \texttt{\$c -> \$.}\\
  \texttt{\$c ( \$.}\\
  \texttt{\$v x \$.}\\
  \texttt{\$v y2 \$.}
\end{center}

The characters in a math symbol\index{math symbol} being declared are
irrelevant to Meta\-math; for example, we could declare a right parenthesis to
be a variable,
\begin{center}
  \texttt{\$v ) \$.}\\
\end{center}
although this would be unconventional.

A {\bf compound} declaration\index{compound declaration} statement is a
shorthand for declaring several symbols at once.  Its syntax is one of the
following:
\begin{center}
  \texttt{\$c} {\em math-symbol}\ \,$\cdots$\ {\em math-symbol} \texttt{\$.}\\
  \texttt{\$v} {\em math-symbol}\ \,$\cdots$\ {\em math-symbol} \texttt{\$.}
\end{center}\index{\texttt{\$c} statement}
Here, the ellipsis (\ldots) means any number of {\em math-symbol}\,s.

An example of a compound declaration statement is:
\begin{center}
  \texttt{\$v x y mu \$.}\\
\end{center}
This is equivalent to the three simple declaration statements
\begin{center}
  \texttt{\$v x \$.}\\
  \texttt{\$v y \$.}\\
  \texttt{\$v mu \$.}\\
\end{center}
\index{\texttt{\$v} statement}

There are certain rules on where in the database math symbols may be declared,
what sections of the database are aware of them (i.e.\ where they are
``active''), and when they may be declared more than once.  These will be
discussed in Section~\ref{scoping} and specifically on
p.~\pageref{redeclaration}.

\subsection{The \texttt{\$d} Statement}\label{dollard}
\index{\texttt{\$d} statement}

The \texttt{\$d} statement is called a {\bf disjoint-variable restriction}.  The
syntax of the {\bf simple} version of this statement is
\begin{center}
  \texttt{\$d} {\em variable variable} \texttt{\$.}
\end{center}
where each {\em variable} is a previously declared variable and the two {\em
variable}\,s are different.  (More specifically, each  {\em variable} must be
an {\bf active} variable\index{active math symbol}, which means there must be
a previous \texttt{\$v} statement whose {\bf scope}\index{scope} includes the
\texttt{\$d} statement.  These terms will be defined when we discuss scoping
statements in Section~\ref{scoping}.)

In ordinary mathematics, formulas may arise that are true if the variables in
them are distinct\index{distinct variables}, but become false when those
variables are made identical. For example, the formula in logic $\exists x\,x
\neq y$, which means ``for a given $y$, there exists an $x$ that is not equal
to $y$,'' is true in most mathematical theories (namely all non-trivial
theories\index{non-trivial theory}, i.e.\ those that describe more than one
individual, such as arithmetic).  However, if we substitute $y$ with $x$, we
obtain $\exists x\,x \neq x$, which is always false, as it means ``there
exists something that is not equal to itself.''\footnote{If you are a
logician, you will recognize this as the improper substitution\index{proper
substitution}\index{substitution!proper} of a free variable\index{free
variable} with a bound variable\index{bound variable}.  Metamath makes no
inherent distinction between free and bound variables; instead, you let
Metamath know what substitutions are permissible by using \texttt{\$d} statements
in the right way in your axiom system.}\index{free vs.\ bound variable}  The
\texttt{\$d} statement allows you to specify a restriction that forbids the
substitution of one variable with another.  In
this case, we would use the statement
\begin{center}
  \texttt{\$d x y \$.}
\end{center}\index{\texttt{\$d} statement}
to specify this restriction.

The order in which the variables appear in a \texttt{\$d} statement is not
important.  We could also use
\begin{center}
  \texttt{\$d y x \$.}
\end{center}

The \texttt{\$d} statement is actually more general than this, as the
``disjoint''\index{disjoint variables} in its name suggests.  The full meaning
is that if any substitution is made to its two variables (during the
course of a proof that references a \texttt{\$a} or \texttt{\$p} statement
associated with the \texttt{\$d}), the two expressions that result from the
substitution must have no variables in common.  In addition, each possible
pair of variables, one from each expression, must be in a \texttt{\$d} statement
associated with the statement being proved.  (This requirement forces the
statement being proved to ``inherit'' the original disjoint variable
restriction.)

For example, suppose \texttt{u} is a variable.  If the restriction
\begin{center}
  \texttt{\$d A B \$.}
\end{center}
has been specified for a theorem referenced in a
proof, we may not substitute \texttt{A} with \mbox{\tt a + u} and
\texttt{B} with \mbox{\tt b + u} because these two symbol sequences have the
variable \texttt{u} in common.  Furthermore, if \texttt{a} and \texttt{b} are
variables, we may not substitute \texttt{A} with \texttt{a} and \texttt{B} with \texttt{b}
unless we have also specified \texttt{\$d a b} for the theorem being proved; in
other words, the \texttt{\$d} property associated with a pair of variables must
be effectively preserved after substitution.

The \texttt{\$d}\index{\texttt{\$d} statement} statement does {\em not} mean ``the
two variables may not be substituted with the same thing,'' as you might think
at first.  For example, substituting each of \texttt{A} and \texttt{B} in the above
example with identical symbol sequences consisting only of constants does not
cause a disjoint variable conflict, because two symbol sequences have no
variables in common (since they have no variables, period).  Similarly, a
conflict will not occur by substituting the two variables in a \texttt{\$d}
statement with the empty symbol sequence\index{empty substitution}.

The \texttt{\$d} statement does not have a direct counterpart in
ordinary mathematics, partly because the variables\index{variable} of
Metamath are not really the same as the variables\index{variable!in
ordinary mathematics} of ordinary mathematics but rather are
metavariables\index{metavariable} ranging over them (as well as over
other kinds of symbols and groups of symbols).  Depending on the
situation, we may informally interpret the \texttt{\$d} statement in
different ways.  Suppose, for example, that \texttt{x} and \texttt{y}
are variables ranging over numbers (more precisely, that \texttt{x} and
\texttt{y} are metavariables ranging over variables that range over
numbers), and that \texttt{ph} ($\varphi$) and \texttt{ps} ($\psi$) are
variables (more precisely, metavariables) ranging over formulas.  We can
make the following interpretations that correspond to the informal
language of ordinary mathematics:
\begin{quote}
\begin{tabbing}
\texttt{\$d x y \$.} means ``assume $x$ and $y$ are
distinct variables.''\\
\texttt{\$d x ph \$.} means ``assume $x$ does not
occur in $\varphi$.''\\
\texttt{\$d ph ps \$.} \=means ``assume $\varphi$ and
$\psi$ have no variables\\ \>in common.''
\end{tabbing}
\end{quote}\index{\texttt{\$d} statement}

\subsubsection{Compound \texttt{\$d} Statements}

The {\bf compound} version of the \texttt{\$d} statement is a shorthand for
specifying several variables whose substitutions must be pairwise disjoint.
Its syntax is:
\begin{center}
  \texttt{\$d} {\em variable}\ \,$\cdots$\ {\em variable} \texttt{\$.}
\end{center}\index{\texttt{\$d} statement}
Here, {\em variable} represents the token of a previously declared
variable (specifically, an active variable) and all {\em variable}\,s are
different.  The compound \texttt{\$d}
statement is internally broken up by Metamath into one simple \texttt{\$d}
statement for each possible pair of variables in the original \texttt{\$d}
statement.  For example,
\begin{center}
  \texttt{\$d w x y z \$.}
\end{center}
is equivalent to
\begin{center}
  \texttt{\$d w x \$.}\\
  \texttt{\$d w y \$.}\\
  \texttt{\$d w z \$.}\\
  \texttt{\$d x y \$.}\\
  \texttt{\$d x z \$.}\\
  \texttt{\$d y z \$.}
\end{center}

Two or more simple \texttt{\$d} statements specifying the same variable pair are
internally combined into a single \texttt{\$d} statement.  Thus the set of three
statements
\begin{center}
  \texttt{\$d x y \$.}
  \texttt{\$d x y \$.}
  \texttt{\$d y x \$.}
\end{center}
is equivalent to
\begin{center}
  \texttt{\$d x y \$.}
\end{center}

Similarly, compound \texttt{\$d} statements, after being internally broken up,
internally have their common variable pairs combined.  For example the
set of statements
\begin{center}
  \texttt{\$d x y A \$.}
  \texttt{\$d x y B \$.}
\end{center}
is equivalent to
\begin{center}
  \texttt{\$d x y \$.}
  \texttt{\$d x A \$.}
  \texttt{\$d y A \$.}
  \texttt{\$d x y \$.}
  \texttt{\$d x B \$.}
  \texttt{\$d y B \$.}
\end{center}
which is equivalent to
\begin{center}
  \texttt{\$d x y \$.}
  \texttt{\$d x A \$.}
  \texttt{\$d y A \$.}
  \texttt{\$d x B \$.}
  \texttt{\$d y B \$.}
\end{center}

Metamath\index{Metamath} automatically verifies that all \texttt{\$d}
restrictions are met whenever it verifies proofs.  \texttt{\$d} statements are
never referenced directly in proofs (this is why they do not have
labels\index{label}), but Metamath is always aware of which ones must be
satisfied (i.e.\ are active) and will notify you with an error message if any
violation occurs.

To illustrate how Metamath detects a missing \texttt{\$d}
statement, we will look at the following example from the
\texttt{set.mm} database.

\begin{verbatim}
$d x z $.  $d y z $.
$( Theorem to add distinct quantifier to atomic formula. $)
ax17eq $p |- ( x = y -> A. z x = y ) $=...
\end{verbatim}

This statement has the obvious requirement that $z$ must be
distinct\index{distinct variables} from $x$ in theorem \texttt{ax17eq} that
states $x=y \rightarrow \forall z \, x=y$ (well, obvious if you're a logician,
for otherwise we could conclude  $x=y \rightarrow \forall x \, x=y$, which is
false when the free variables $x$ and $y$ are equal.).

Let's look at what happens if we edit the database to comment out this
requirement.

\begin{verbatim}
$( $d x z $. $) $d y z $.
$( Theorem to add distinct quantifier to atomic formula. $)
ax17eq $p |- ( x = y -> A. z x = y ) $=...
\end{verbatim}

When it tries to verify the proof, Metamath will tell you that \texttt{x} and
\texttt{z} must be disjoint, because one of its steps references an axiom or
theorem that has this requirement.

\begin{verbatim}
MM> verify proof ax17eq
ax17eq ?Error at statement 1918, label "ax17eq", type "$p":
      vz wal wi vx vy vz ax-13 vx vy weq vz vx ax-c16 vx vy
                                               ^^^^^
There is a disjoint variable ($d) violation at proof step 29.
Assertion "ax-c16" requires that variables "x" and "y" be
disjoint.  But "x" was substituted with "z" and "y" was
substituted with "x".  The assertion being proved, "ax17eq",
does not require that variables "z" and "x" be disjoint.
\end{verbatim}

We can see the substitutions into \texttt{ax-c16} with the following command.

\begin{verbatim}
MM> show proof ax17eq / detailed_step 29
Proof step 29:  pm2.61dd.2=ax-c16 $a |- ( A. z z = x -> ( x =
  y -> A. z x = y ) )
This step assigns source "ax-c16" ($a) to target "pm2.61dd.2"
($e).  The source assertion requires the hypotheses "wph"
($f, step 26), "vx" ($f, step 27), and "vy" ($f, step 28).
The parent assertion of the target hypothesis is "pm2.61dd"
($p, step 36).
The source assertion before substitution was:
    ax-c16 $a |- ( A. x x = y -> ( ph -> A. x ph ) )
The following substitutions were made to the source
assertion:
    Variable  Substituted with
     x         z
     y         x
     ph        x = y
The target hypothesis before substitution was:
    pm2.61dd.2 $e |- ( ph -> ch )
The following substitutions were made to the target
hypothesis:
    Variable  Substituted with
     ph        A. z z = x
     ch        ( x = y -> A. z x = y )
\end{verbatim}

The disjoint variable restrictions of \texttt{ax-c16} can be seen from the
\texttt{show state\-ment} command.  The line that begins ``\texttt{Its mandatory
dis\-joint var\-i\-able pairs are:}\ldots'' lists any \texttt{\$d} variable
pairs in brackets.

\begin{verbatim}
MM> show statement ax-c16/full
Statement 3033 is located on line 9338 of the file "set.mm".
"Axiom of Distinct Variables. ..."
  ax-c16 $a |- ( A. x x = y -> ( ph -> A. x ph ) ) $.
Its mandatory hypotheses in RPN order are:
  wph $f wff ph $.
  vx $f set x $.
  vy $f set y $.
Its mandatory disjoint variable pairs are:  <x,y>
The statement and its hypotheses require the variables:  x y
      ph
The variables it contains are:  x y ph
\end{verbatim}

Since Metamath will always detect when \texttt{\$d}\index{\texttt{\$d} statement}
statements are needed for a proof, you don't have to worry too much about
forgetting to put one in; it can always be added if you see the error message
above.  If you put in unnecessary \texttt{\$d} statements, the worst that could
happen is that your theorem might not be as general as it could be, and this
may limit its use later on.

On the other hand, when you introduce axioms (\texttt{\$a}\index{\texttt{\$a}
statement} statements), you must be very careful to properly specify the
necessary associated \texttt{\$d} statements since Metamath has no way of knowing
whether your axioms are correct.  For example, Metamath would have no idea
that \texttt{ax-c16}, which we are telling it is an axiom of logic, would lead to
contradictions if we omitted its associated \texttt{\$d} statement.

% This was previously a comment in footnote-sized type, but it can be
% hard to read this much text in a small size.
% As a result, it's been changed to normally-sized text.
\label{nodd}
You may wonder if it is possible to develop standard
mathematics in the Metamath language without the \texttt{\$d}\index{\texttt{\$d}
statement} statement, since it seems like a nuisance that complicates proof
verification. The \texttt{\$d} statement is not needed in certain subsets of
mathematics such as propositional calculus.  However, dummy
variables\index{dummy variable!eliminating} and their associated \texttt{\$d}
statements are impossible to avoid in proofs in standard first-order logic as
well as in the variant used in \texttt{set.mm}.  In fact, there is no upper bound to
the number of dummy variables that might be needed in a proof of a theorem of
first-order logic containing 3 or more variables, as shown by H.\
Andr\'{e}ka\index{Andr{\'{e}}ka, H.} \cite{Nemeti}.  A first-order system that
avoids them entirely is given in \cite{Megill}\index{Megill, Norman}; the
trick there is simply to embed harmlessly the necessary dummy variables into a
theorem being proved so that they aren't ``dummy'' anymore, then interpret the
resulting longer theorem so as to ignore the embedded dummy variables.  If
this interests you, the system in \texttt{set.mm} obtained from \texttt{ax-1}
through \texttt{ax-c14} in \texttt{set.mm}, and deleting \texttt{ax-c16} and \texttt{ax-5},
requires no \texttt{\$d} statements but is logically complete in the sense
described in \cite{Megill}.  This means it can prove any theorem of
first-order logic as long as we add to the theorem an antecedent that embeds
dummy and any other variables that must be distinct.  In a similar fashion,
axioms for set theory can be devised that
do not require distinct variable
provisos\index{Set theory without distinct variable provisos},
as explained at
\url{http://us.metamath.org/mpeuni/mmzfcnd.html}.
Together, these in principle allow all of
mathematics to be developed under Metamath without a \texttt{\$d} statement,
although the length of the resulting theorems will grow as more and
more dummy variables become required in their proofs.

\subsection{The \texttt{\$f}
and \texttt{\$e} Statements}\label{dollaref}
\index{\texttt{\$e} statement}
\index{\texttt{\$f} statement}
\index{floating hypothesis}
\index{essential hypothesis}
\index{variable-type hypothesis}
\index{logical hypothesis}
\index{hypothesis}

Metamath has two kinds of hypo\-theses, the \texttt{\$f}\index{\texttt{\$f}
statement} or {\bf variable-type} hypothesis and the \texttt{\$e} or {\bf logical}
hypo\-the\-sis.\index{\texttt{\$d} statement}\footnote{Strictly speaking, the
\texttt{\$d} statement is also a hypothesis, but it is never directly referenced
in a proof, so we call it a restriction rather than a hypothesis to lessen
confusion.  The checking for violations of \texttt{\$d} restrictions is automatic
and built into Metamath's proof-checking algorithm.} The letters \texttt{f} and
\texttt{e} stand for ``floating''\index{floating hypothesis} (roughly meaning
used only if relevant) and ``essential''\index{essential hypothesis} (meaning
always used) respectively, for reasons that will become apparent
when we discuss frames in
Section~\ref{frames} and scoping in Section~\ref{scoping}. The syntax of these
are as follows:
\begin{center}
  {\em label} \texttt{\$f} {\em typecode} {\em variable} \texttt{\$.}\\
  {\em label} \texttt{\$e} {\em typecode}
      {\em math-symbol}\ \,$\cdots$\ {\em math-symbol} \texttt{\$.}\\
\end{center}
\index{\texttt{\$e} statement}
\index{\texttt{\$f} statement}
A hypothesis must have a {\em label}\index{label}.  The expression in a
\texttt{\$e} hypothesis consists of a typecode (an active constant math symbol)
followed by a sequence
of zero or more math symbols. Each math symbol (including {\em constant}
and {\em variable}) must be a previously declared constant or variable.  (In
addition, each math symbol must be active, which will be covered when we
discuss scoping statements in Section~\ref{scoping}.)  You use a \texttt{\$f}
hypothesis to specify the
nature or {\bf type}\index{variable type}\index{type} of a variable (such as ``let $x$ be an
integer'') and use a \texttt{\$e} hypothesis to express a logical truth (such as
``assume $x$ is prime'') that must be established in order for an assertion
requiring it to also be true.

A variable must have its type specified in a \texttt{\$f} statement before
it may be used in a \texttt{\$e}, \texttt{\$a}, or \texttt{\$p}
statement.  There may be only one (active) \texttt{\$f} statement for a
given variable.  (``Active'' is defined in Section~\ref{scoping}.)

In ordinary mathematics, theorems\index{theorem} are often expressed in the
form ``Assume $P$; then $Q$,'' where $Q$ is a statement that you can derive
if you start with statement $P$.\index{free variable}\footnote{A stronger
version of a theorem like this would be the {\em single} formula $P\rightarrow
Q$ ($P$ implies $Q$) from which the weaker version above follows by the rule
of modus ponens in logic.  We are not discussing this stronger form here.  In
the weaker form, we are saying only that if we can {\em prove} $P$, then we can
{\em prove} $Q$.  In a logician's language, if $x$ is the only free variable
in $P$ and $Q$, the stronger form is equivalent to $\forall x ( P \rightarrow
Q)$ (for all $x$, $P$ implies $Q$), whereas the weaker form is equivalent to
$\forall x P \rightarrow \forall x Q$. The stronger form implies the weaker,
but not vice-versa.  To be precise, the weaker form of the theorem is more
properly called an ``inference'' rather than a theorem.}\index{inference}
In the
Metamath\index{Metamath} language, you would express mathematical statement
$P$ as a hypothesis (a \texttt{\$e} Metamath language statement in this case) and
statement $Q$ as a provable assertion (a \texttt{\$p}\index{\texttt{\$p} statement}
statement).

Some examples of hypotheses you might encounter in logic and set theory are
\begin{center}
  \texttt{stmt1 \$f wff P \$.}\\
  \texttt{stmt2 \$f set x \$.}\\
  \texttt{stmt3 \$e |- ( P -> Q ) \$.}
\end{center}
\index{\texttt{\$e} statement}
\index{\texttt{\$f} statement}
Informally, these would be read, ``Let $P$ be a well-formed-formula,'' ``Let
$x$ be an (individual) variable,'' and ``Assume we have proved $P \rightarrow
Q$.''  The turnstile symbol \,$\vdash$\index{turnstile ({$\,\vdash$})} is
commonly used in logic texts to mean ``a proof exists for.''

To summarize:
\begin{itemize}
\item A \texttt{\$f} hypothesis tells Metamath the type or kind of its variable.
It is analogous to a variable declaration in a computer language that
tells the compiler that a variable is an integer or a floating-point
number.
\item The \texttt{\$e} hypothesis corresponds to what you would usually call a
``hypothesis'' in ordinary mathematics.
\end{itemize}

Before an assertion\index{assertion} (\texttt{\$a} or \texttt{\$p} statement) can be
referenced in a proof, all of its associated \texttt{\$f} and \texttt{\$e} hypotheses
(i.e.\ those \texttt{\$e} hypotheses that are active) must be satisfied (i.e.
established by the proof).  The meaning of ``associated'' (which we will call
{\bf mandatory} in Section~\ref{frames}) will become clear when we discuss
scoping later.

Note that after any \texttt{\$f}, \texttt{\$e},
\texttt{\$a}, or \texttt{\$p} token there is a required
\textit{typecode}\index{typecode}.
The typecode is a constant used to enforce types of expressions.
This will become clearer once we learn more about
assertions (\texttt{\$a} and \texttt{\$p} statements).
An example may also clarify their purpose.
In the
\texttt{set.mm}\index{set theory database (\texttt{set.mm})}%
\index{Metamath Proof Explorer}
database,
the following typecodes are used:

\begin{itemize}
\item \texttt{wff} :
  Well-formed formula (wff) symbol
  (read: ``the following symbol sequence is a wff'').
% The *textual* typecode for turnstile is "|-", but when read it's a little
% confusing, so I intentionally display the mathematical symbol here instead
% (I think it's clearer in this context).
\item \texttt{$\vdash$} :
  Turnstile (read: ``the following symbol sequence is provable'' or
  ``a proof exists for'').
\item \texttt{setvar} :
  Individual set variable type (read: ``the following is an
  individual set variable'').
  Note that this is \textit{not} the type of an arbitrary set expression,
  instead, it is used to ensure that there is only a single symbol used
  after quantifiers like for-all ($\forall$) and there-exists ($\exists$).
\item \texttt{class} :
  An expression that is a syntactically valid class expression.
  All valid set expressions are also valid class expression, so expressions
  of sets normally have the \texttt{class} typecode.
  Use the \texttt{class} typecode,
  \textit{not} the \texttt{setvar} typecode,
  for the type of set expressions unless you are specifically identifying
  a single set variable.
\end{itemize}

\subsection{Assertions (\texttt{\$a} and \texttt{\$p} Statements)}
\index{\texttt{\$a} statement}
\index{\texttt{\$p} statement}\index{assertion}\index{axiomatic assertion}
\index{provable assertion}

There are two types of assertions, \texttt{\$a}\index{\texttt{\$a} statement}
statements ({\bf axiomatic assertions}) and \texttt{\$p} statements ({\bf
provable assertions}).  Their syntax is as follows:
\begin{center}
  {\em label} \texttt{\$a} {\em typecode} {\em math-symbol} \ldots
         {\em math-symbol} \texttt{\$.}\\
  {\em label} \texttt{\$p} {\em typecode} {\em math-symbol} \ldots
        {\em math-symbol} \texttt{\$=} {\em proof} \texttt{\$.}
\end{center}
\index{\texttt{\$a} statement}
\index{\texttt{\$p} statement}
\index{\texttt{\$=} keyword}
An assertion always requires a {\em label}\index{label}. The expression in an
assertion consists of a typecode (an active constant)
followed by a sequence of zero
or more math symbols.  Each math symbol, including any {\em constant}, must be a
previously declared constant or variable.  (In addition, each math symbol
must be active, which will be covered when we discuss scoping statements in
Section~\ref{scoping}.)

A \texttt{\$a} statement is usually a definition of syntax (for example, if $P$
and $Q$ are wffs then so is $(P\to Q)$), an axiom\index{axiom} of ordinary
mathematics (for example, $x=x$), or a definition\index{definition} of
ordinary mathematics (for example, $x\ne y$ means $\lnot x=y$). A \texttt{\$p}
statement is a claim that a certain combination of math symbols follows from
previous assertions and is accompanied by a proof that demonstrates it.

Assertions can also be referenced in (later) proofs in order to derive new
assertions from them. The label of an assertion is used to refer to it in a
proof. Section~\ref{proof} will describe the proof in detail.

Assertions also provide the primary means for communicating the mathematical
results in the database to people.  Proofs (when conveniently displayed)
communicate to people how the results were arrived at.

\subsubsection{The \texttt{\$a} Statement}
\index{\texttt{\$a} statement}

Axiomatic assertions (\texttt{\$a} statements) represent the starting points from
which other assertions (\texttt{\$p}\index{\texttt{\$p} statement} statements) are
derived.  Their most obvious use is for specifying ordinary mathematical
axioms\index{axiom}, but they are also used for two other purposes.

First, Metamath\index{Metamath} needs to know the syntax of symbol
sequences that constitute valid mathematical statements.  A Metamath
proof must be broken down into much more detail than ordinary
mathematical proofs that you may be used to thinking of (even the
``complete'' proofs of formal logic\index{formal logic}).  This is one
of the things that makes Metamath a general-purpose language,
independent of any system of logic or even syntax.  If you want to use a
substitution instance of an assertion as a step in a proof, you must
first prove that the substitution is syntactically correct (or if you
prefer, you must ``construct'' it), showing for example that the
expression you are substituting for a wff metavariable is a valid wff.
The \texttt{\$a}\index{\texttt{\$a} statement} statement is used to
specify those combinations of symbols that are considered syntactically
valid, such as the legal forms of wffs.

Second, \texttt{\$a} statements are used to specify what are ordinarily thought of
as definitions, i.e.\ new combinations of symbols that abbreviate other
combinations of symbols.  Metamath makes no distinction\index{axiom vs.\
definition} between axioms\index{axiom} and definitions\index{definition}.
Indeed, it has been argued that such distinction should not be made even in
ordinary mathematics; see Section~\ref{definitions}, which discusses the
philosophy of definitions.  Section~\ref{hierarchy} discusses some
technical requirements for definitions.  In \texttt{set.mm} we adopt the
convention of prefixing axiom labels with \texttt{ax-} and definition labels with
\texttt{df-}\index{label}.

The results that can be derived with the Metamath language are only as good as
the \texttt{\$a}\index{\texttt{\$a} statement} statements used as their starting
point.  We cannot stress this too strongly.  For example, Metamath will
not prevent you from specifying $x\neq x$ as an axiom of logic.  It is
essential that you scrutinize all \texttt{\$a} statements with great care.
Because they are a source of potential pitfalls, it is best not to add new
ones (usually new definitions) casually; rather you should carefully evaluate
each one's necessity and advantages.

Once you have in place all of the basic axioms\index{axiom} and
rules\index{rule} of a mathematical theory, the only \texttt{\$a} statements that
you will be adding will be what are ordinarily called definitions.  In
principle, definitions should be in some sense eliminable from the language of
a theory according to some convention (usually involving logical equivalence
or equality).  The most common convention is that any formula that was
syntactically valid but not provable before the definition was introduced will
not become provable after the definition is introduced.  In an ideal world,
definitions should not be present at all if one is to have absolute confidence
in a mathematical result.  However, they are necessary to make
mathematics practical, for otherwise the resulting formulas would be
extremely long and incomprehensible.  Since the nature of definitions (in the
most general sense) does not permit them to automatically be verified as
``proper,''\index{proper definition}\index{definition!proper} the judgment of
the mathematician is required to ensure it.  (In \texttt{set.mm} effort was made
to make almost all definitions directly eliminable and thus minimize the need
for such judgment.)

If you are not a mathematician, it may be best not to add or change any
\texttt{\$a}\index{\texttt{\$a} statement} statements but instead use
the mathematical language already provided in standard databases.  This
way Metamath will not allow you to make a mistake (i.e.\ prove a false
result).


\subsection{Frames}\label{frames}

We now introduce the concept of a collection of related Metamath statements
called a frame.  Every assertion (\texttt{\$a} or \texttt{\$p} statement) in the database has
an associated frame.

A {\bf frame}\index{frame} is a sequence of \texttt{\$d}, \texttt{\$f},
and \texttt{\$e} statements (zero or more of each) followed by one
\texttt{\$a} or \texttt{\$p} statement, subject to certain conditions we
will describe.  For simplicity we will assume that all math symbol
tokens used are declared at the beginning of the database with
\texttt{\$c} and \texttt{\$v} statements (which are not properly part of
a frame).  Also for simplicity we will assume there are only simple
\texttt{\$d} statements (those with only two variables) and imagine any
compound \texttt{\$d} statements (those with more than two variables) as
broken up into simple ones.

A frame groups together those hypotheses (and \texttt{\$d} statements) relevant
to an assertion (\texttt{\$a} or \texttt{\$p} statement).  The statements in a frame
may or may not be physically adjacent in a database; we will cover
this in our discussion of scoping statements
in Section~\ref{scoping}.

A frame has the following properties:
\begin{enumerate}
 \item The set of variables contained in its \texttt{\$f} statements must
be identical to the set of variables contained in its \texttt{\$e},
\texttt{\$a}, and/or \texttt{\$p} statements.  In other words, each
variable in a \texttt{\$e}, \texttt{\$a}, or \texttt{\$p} statement must
have an associated ``variable type'' defined for it in a \texttt{\$f}
statement.
  \item No two \texttt{\$f} statements may contain the same variable.
  \item Any \texttt{\$f} statement
must occur before a \texttt{\$e} statement in which its variable occurs.
\end{enumerate}

The first property determines the set of variables occurring in a frame.
These are the {\bf mandatory
variables}\index{mandatory variable} of the frame.  The second property
tells us there must be only one type specified for a variable.
The last property is not a theoretical requirement but it
makes parsing of the database easier.

For our examples, we assume our database has the following declarations:

\begin{verbatim}
$v P Q R $.
$c -> ( ) |- wff $.
\end{verbatim}

The following sequence of statements, describing the modus ponens inference
rule, is an example of a frame:

\begin{verbatim}
wp  $f wff P $.
wq  $f wff Q $.
maj $e |- ( P -> Q ) $.
min $e |- P $.
mp  $a |- Q $.
\end{verbatim}

The following sequence of statements is not a frame because \texttt{R} does not
occur in the \texttt{\$e}'s or the \texttt{\$a}:

\begin{verbatim}
wp  $f wff P $.
wq  $f wff Q $.
wr  $f wff R $.
maj $e |- ( P -> Q ) $.
min $e |- P $.
mp  $a |- Q $.
\end{verbatim}

The following sequence of statements is not a frame because \texttt{Q} does not
occur in a \texttt{\$f}:

\begin{verbatim}
wp  $f wff P $.
maj $e |- ( P -> Q ) $.
min $e |- P $.
mp  $a |- Q $.
\end{verbatim}

The following sequence of statements is not a frame because the \texttt{\$a} statement is
not the last one:

\begin{verbatim}
wp  $f wff P $.
wq  $f wff Q $.
maj $e |- ( P -> Q ) $.
mp  $a |- Q $.
min $e |- P $.
\end{verbatim}

Associated with a frame is a sequence of {\bf mandatory
hypotheses}\index{mandatory hypothesis}.  This is simply the set of all
\texttt{\$f} and \texttt{\$e} statements in the frame, in the order they
appear.  A frame can be referenced in a later proof using the label of
the \texttt{\$a} or \texttt{\$p} assertion statement, and the proof
makes an assignment to each mandatory hypothesis in the order in which
it appears.  This means the order of the hypotheses, once chosen, must
not be changed so as not to affect later proofs referencing the frame's
assertion statement.  (The Metamath proof verifier will, of course, flag
an error if a proof becomes incorrect by doing this.)  Since proofs make
use of ``Reverse Polish notation,'' described in Section~\ref{proof}, we
call this order the {\bf RPN order}\index{RPN order} of the hypotheses.

Note that \texttt{\$d} statements are not part of the set of mandatory
hypotheses, and their order doesn't matter (as long as they satisfy the
fourth property for a frame described above).  The \texttt{\$d}
statements specify restrictions on variables that must be satisfied (and
are checked by the proof verifier) when expressions are substituted for
them in a proof, and the \texttt{\$d} statements themselves are never
referenced directly in a proof.

A frame with a \texttt{\$p} (provable) statement requires a proof as part of the
\texttt{\$p} statement.  Sometimes in a proof we want to make use of temporary or
dummy variables\index{dummy variable} that do not occur in the \texttt{\$p}
statement or its mandatory hypotheses.  To accommodate this we define an {\bf
extended frame}\index{extended frame} as a frame together with zero or more
\texttt{\$d} and \texttt{\$f} statements that reference variables not among the
mandatory variables of the frame.  Any new variables referenced are called the
{\bf optional variables}\index{optional variable} of the extended frame. If a
\texttt{\$f} statement references an optional variable it is called an {\bf
optional hypothesis}\index{optional hypothesis}, and if one or both of the
variables in a \texttt{\$d} statement are optional variables it is called an {\bf
optional disjoint-variable restriction}\index{optional disjoint-variable
restriction}.  Properties 2, 3, and 4 for a frame also apply to an extended
frame.

The concept of optional variables is not meaningful for frames with \texttt{\$a}
statements, since those statements have no proofs that might make use of them.
There is no restriction on including optional hypotheses in the extended frame
for a \texttt{\$a} statement, but they serve no purpose.

The following set of statements is an example of an extended frame, which
contains an optional variable \texttt{R} and an optional hypothesis \texttt{wr}.  In
this example, we suppose the rule of modus ponens is not an axiom but is
derived as a theorem from earlier statements (we omit its presumed proof).
Variable \texttt{R} may be used in its proof if desired (although this would
probably have no advantage in propositional calculus).  Note that the sequence
of mandatory hypotheses in RPN order is still \texttt{wp}, \texttt{wq}, \texttt{maj},
\texttt{min} (i.e.\ \texttt{wr} is omitted), and this sequence is still assumed
whenever the assertion \texttt{mp} is referenced in a subsequent proof.

\begin{verbatim}
wp  $f wff P $.
wq  $f wff Q $.
wr  $f wff R $.
maj $e |- ( P -> Q ) $.
min $e |- P $.
mp  $p |- Q $= ... $.
\end{verbatim}

Every frame is an extended frame, but not every extended frame is a frame, as
this example shows.  The underlying frame for an extended frame is
obtained by simply removing all statements containing optional variables.
Any proof referencing an assertion will ignore any extensions to its
frame, which means we may add or delete optional hypotheses at will without
affecting subsequent proofs.

The conceptually simplest way of organizing a Metamath database is as a
sequence of extended frames.  The scoping statements
\texttt{\$\char`\{}\index{\texttt{\$\char`\{} and \texttt{\$\char`\}}
keywords} and \texttt{\$\char`\}} can be used to delimit the start and
end of an extended frame, leading to the following possible structure for a
database.  \label{framelist}

\vskip 2ex
\setbox\startprefix=\hbox{\tt \ \ \ \ \ \ \ \ }
\setbox\contprefix=\hbox{}
\startm
\m{\mbox{(\texttt{\$v} {\em and} \texttt{\$c}\,{\em statements})}}
\endm
\startm
\m{\mbox{\texttt{\$\char`\{}}}
\endm
\startm
\m{\mbox{\texttt{\ \ } {\em extended frame}}}
\endm
\startm
\m{\mbox{\texttt{\$\char`\}}}}
\endm
\startm
\m{\mbox{\texttt{\$\char`\{}}}
\endm
\startm
\m{\mbox{\texttt{\ \ } {\em extended frame}}}
\endm
\startm
\m{\mbox{\texttt{\$\char`\}}}}
\endm
\startm
\m{\mbox{\texttt{\ \ \ \ \ \ \ \ \ }}\vdots}
\endm
\vskip 2ex

In practice, this structure is inconvenient because we have to repeat
any \texttt{\$f}, \texttt{\$e}, and \texttt{\$d} statements over and
over again rather than stating them once for use by several assertions.
The scoping statements, which we will discuss next, allow this to be
done.  In principle, any Metamath database can be converted to the above
format, and the above format is the most convenient to use when studying
a Metamath database as a formal system%
%% Uncomment this when uncommenting section {formalspec} below
   (Appendix \ref{formalspec})%
.
In fact, Metamath internally converts the database to the above format.
The command \texttt{show statement} in the Metamath program will show
you the contents of the frame for any \texttt{\$a} or \texttt{\$p}
statement, as well as its extension in the case of a \texttt{\$p}
statement.

%c%(provided that all ``local'' variables and constants with limited scope have
%c%unique names),

During our discussion of scoping statements, it may be helpful to
think in terms of the equivalent sequence of frames that will result when
the database is parsed.  Scoping (other than the limited
use above to delimit frames) is not a theoretical requirement for
Metamath but makes it more convenient.


\subsection{Scoping Statements (\texttt{\$\{} and \texttt{\$\}})}\label{scoping}
\index{\texttt{\$\char`\{} and \texttt{\$\char`\}} keywords}\index{scoping statement}

%c%Some Metamath statements may be needed only temporarily to
%c%serve a specific purpose, and after we're done with them we would like to
%c%disregard or ignore them.  For example, when we're finished using a variable,
%c%we might want to
%c%we might want to free up the token\index{token} used to name it so that the
%c%token can be used for other purposes later on, such as a different kind of
%c%variable or even a constant.  In the terminology of computer programming, we
%c%might want to let some symbol declarations be ``local'' rather than ``global.''
%c%\index{local symbol}\index{global symbol}

The {\bf scoping} statements, \texttt{\$\char`\{} ({\bf start of block}) and \texttt{\$\char`\}}
({\bf end of block})\index{block}, provide a means for controlling the portion
of a database over which certain statement types are recognized.  The
syntax of a scoping statement is very simple; it just consists of the
statement's keyword:
\begin{center}
\texttt{\$\char`\{}\\
\texttt{\$\char`\}}
\end{center}
\index{\texttt{\$\char`\{} and \texttt{\$\char`\}} keywords}

For example, consider the following database where we have stripped out
all tokens except the scoping statement keywords.  For the purpose of the
discussion, we have added subscripts to the scoping statements; these subscripts
do not appear in the actual database.
\[
 \mbox{\tt \ \$\char`\{}_1
 \mbox{\tt \ \$\char`\{}_2
 \mbox{\tt \ \$\char`\}}_2
 \mbox{\tt \ \$\char`\{}_3
 \mbox{\tt \ \$\char`\{}_4
 \mbox{\tt \ \$\char`\}}_4
 \mbox{\tt \ \$\char`\}}_3
 \mbox{\tt \ \$\char`\}}_1
\]
Each \texttt{\$\char`\{} statement in this example is said to be {\bf
matched} with the \texttt{\$\char`\}} statement that has the same
subscript.  Each pair of matched scoping statements defines a region of
the database called a {\bf block}.\index{block} Blocks can be {\bf
nested}\index{nested block} inside other blocks; in the example, the
block defined by $\mbox{\tt \$\char`\{}_4$ and $\mbox{\tt \$\char`\}}_4$
is nested inside the block defined by $\mbox{\tt \$\char`\{}_3$ and
$\mbox{\tt \$\char`\}}_3$ as well as inside the block defined by
$\mbox{\tt \$\char`\{}_1$ and $\mbox{\tt \$\char`\}}_1$.  In general, a
block may be empty, it may contain only non-scoping
statements,\footnote{Those statements other than \texttt{\$\char`\{} and
\texttt{\$\char`\}}.}\index{non-scoping statement} or it may contain any
mixture of other blocks and non-scoping statements.  (This is called a
``recursive'' definition\index{recursive definition} of a block.)

Associated with each block is a number called its {\bf nesting
level}\index{nesting level} that indicates how deeply the block is nested.
The nesting levels of the blocks in our example are as follows:
\[
  \underbrace{
    \mbox{\tt \ }
    \underbrace{
     \mbox{\tt \$\char`\{\ }
     \underbrace{
       \mbox{\tt \$\char`\{\ }
       \mbox{\tt \$\char`\}}
     }_{2}
     \mbox{\tt \ }
     \underbrace{
       \mbox{\tt \$\char`\{\ }
       \underbrace{
         \mbox{\tt \$\char`\{\ }
         \mbox{\tt \$\char`\}}
       }_{3}
       \mbox{\tt \ \$\char`\}}
     }_{2}
     \mbox{\tt \ \$\char`\}}
   }_{1}
   \mbox{\tt \ }
 }_{0}
\]
\index{\texttt{\$\char`\{} and \texttt{\$\char`\}} keywords}
The entire database is considered to be one big block (the {\bf outermost}
block) with a nesting level of 0.  The outermost block is {\em not} bracketed
by scoping statements.\footnote{The language was designed this way so that
several source files can be joined together more easily.}\index{outermost
block}

All non-scoping Metamath statements become recognized or {\bf
active}\index{active statement} at the place where they appear.\footnote{To
keep things slightly simpler, we do not bother to define the concept of
``active'' for the scoping statements.}  Certain of these statement types
become inactive at the end of the block in which they appear; these statement
types are:
\begin{center}
  \texttt{\$c}, \texttt{\$v}, \texttt{\$d}, \texttt{\$e}, and \texttt{\$f}.
%  \texttt{\$v}, \texttt{\$f}, \texttt{\$e}, and \texttt{\$d}.
\end{center}
\index{\texttt{\$c} statement}
\index{\texttt{\$d} statement}
\index{\texttt{\$e} statement}
\index{\texttt{\$f} statement}
\index{\texttt{\$v} statement}
The other statement types remain active forever (i.e.\ through the end of the
database); they are:
\begin{center}
  \texttt{\$a} and \texttt{\$p}.
%  \texttt{\$c}, \texttt{\$a}, and \texttt{\$p}.
\end{center}
\index{\texttt{\$a} statement}
\index{\texttt{\$p} statement}
Any statement (of these 7 types) located in the outermost
block\index{outermost block} will remain active through the end of the
database and thus are effectively ``global'' statements.\index{global
statement}

All \texttt{\$c} statements must be placed in the outermost block.  Since they are
therefore always global, they could be considered as belonging to both of the
above categories.

The {\bf scope}\index{scope} of a statement is the set of statements that
recognize it as active.

%c%The concept of ``active'' is also defined for math symbols\index{math
%c%symbol}.  Math symbols (constants\index{constant} and
%c%variables\index{variable}) become {\bf active}\index{active
%c%math symbol} in the \texttt{\$c}\index{\texttt{\$c}
%c%statement} and \texttt{\$v}\index{\texttt{\$v} statement} statements that
%c%declare them.  They become inactive when their declaration statements become
%c%inactive.

The concept of ``active'' is also defined for math symbols\index{math
symbol}.  Math symbols (constants\index{constant} and
variables\index{variable}) become {\bf active}\index{active math symbol}
in the \texttt{\$c}\index{\texttt{\$c} statement} and
\texttt{\$v}\index{\texttt{\$v} statement} statements that declare them.
A variable becomes inactive when its declaration statement becomes
inactive.  Because all \texttt{\$c} statements must be in the outermost
block, a constant will never become inactive after it is declared.

\subsubsection{Redeclaration of Math Symbols}
\index{redeclaration of symbols}\label{redeclaration}

%c%A math symbol may not be declared a second time while it is active, but it may
%c%be declared again after it becomes inactive.

A variable may not be declared a second time while it is active, but it may be
declared again after it becomes inactive.  This provides a convenient way to
introduce ``local'' variables,\index{local variable} i.e.\ temporary variables
for use in the frame of an assertion or in a proof without keeping them around
forever.  A previously declared variable may not be redeclared as a constant.

A constant may not be redeclared.  And, as mentioned above, constants must be
declared in the outermost block.

The reason variables may have limited scope but not constants is that an
assertion (\texttt{\$a} or \texttt{\$p} statement) remains available for use in
proofs through the end of the database.  Variables in an assertion's frame may
be substituted with whatever is needed in a proof step that references the
assertion, whereas constants remain fixed and may not be substituted with
anything.  The particular token used for a variable in an assertion's frame is
irrelevant when the assertion is referenced in a proof, and it doesn't matter
if that token is not available outside of the referenced assertion's frame.
Constants, however, must be globally fixed.

In the present version of the Metamath language, there is no theoretical
benefit for the feature allowing variables to be active for limited scopes
rather than global. It is just a convenience that allows them, for example, to
be locally grouped together with their corresponding \texttt{\$f} variable-type
declarations.\footnote{A future version\label{redeclarationf} of the Metamath
language may extend the language to include true local variables as well as
local constants.  I have not decided upon the best way to do this or even if
it should be done at all; the main issue is whether the benefits offset
greater confusion in learning the language.  In any case, the rules for
the present language were chosen to let existing databases be compatible with
any future language extensions.}

%c%If you declare a math symbol more than once, internally Metamath considers it a
%c%new distinct symbol, even though it has the same name.  If you are unaware of
%c%this, you may find that what you think are correct proofs are incorrectly
%c%rejected as invalid, because Metamath may tell you that a constant you
%c%previously declared does not match a newly declared math symbol with the same
%c%name.  For details on this subtle point, see the Comment on
%c%p.~\pageref{spec4comment}.  This is done purposely to allow temporary
%c%constants to be introduced while developing a subtheory, then allow their math
%c%symbol tokens to be reused later on; in general they will not refer to the
%c%same thing.  In practice, you would not ordinarily reuse the names of
%c%constants because it would tend to be confusing to the reader.  The reuse of
%c%names of variables, on the other hand, is something that is often useful to do
%c%(for example it is done frequently in \texttt{set.mm}).  Since variables in an
%c%assertion referenced in a proof can be substituted as needed to achieve a
%c%symbol match, this is not an issue.

% (This section covers a somewhat advanced topic you may want to skip
% at first reading.)
%
% Under certain circumstances, math symbol\index{math symbol}
% tokens\index{token} may be redeclared (i.e.\ the token
% may appear in more than
% one \texttt{\$c}\index{\texttt{\$c} statement} or \texttt{\$v}\index{\texttt{\$v}
% statement} statement).  You might want to do this say, to make temporary use
% of a variable name without having to worry about its affect elsewhere,
% somewhat analogous to declaring a local variable in a standard computer
% language.  Understanding what goes on when math symbol tokens are redeclared
% is a little tricky to understand at first, since it requires that we
% distinguish the token itself from the math symbol that it names.  It will help
% if we first take a peek at the internal workings of the
% Metamath\index{Metamath} program.
%
% Metamath reserves a memory location for each occurrence of a
% token\index{token} in a declaration statement (\texttt{\$c}\index{\texttt{\$c}
% statement} or \texttt{\$v}\index{\texttt{\$v} statement}).  If a given token appears
% in more than one declaration statement, it will refer to more than one memory
% locations.  A math symbol\index{math symbol} may be thought of as being one of
% these memory locations rather than as the token itself.  Only one of the
% memory locations associated with a given token may be active at any one time.
% The math symbol (memory location) that gets looked up when the token appears
% in a non-declaration statement is the one that happens to be active at that
% time.
%
% We now look at the rules for the redeclaration\index{redeclaration of symbols}
% of math symbol tokens.
% \begin{itemize}
% \item A math symbol token may not be declared twice in the
% same block.\footnote{While there is no theoretical reason for disallowing
% this, it was decided in the design of Metamath that allowing it would offer no
% advantage and might cause confusion.}
% \item An inactive math symbol may always be
% redeclared.
% \item  An active math symbol may be redeclared in a different (i.e.\
% inner) block\index{block} from the one it became active in.
% \end{itemize}
%
% When a math symbol token is redeclared, it conceptually refers to a different
% math symbol, just as it would be if it were called a different name.  In
% addition, the original math symbol that it referred to, if it was active,
% temporarily becomes inactive.  At the end of the block in which the
% redeclaration occurred, the new math symbol\index{math symbol} becomes
% inactive and the original symbol becomes active again.  This concept is
% illustrated in the following example, where the symbol \texttt{e} is
% ordinarily a constant (say Euler's constant, 2.71828...) but
% temporarily we want to use it as a ``local'' variable, say as a coefficient
% in the equation $a x^4 + b x^3 + c x^2 + d x + e$:
% \[
%   \mbox{\tt \$\char`\{\ \$c e \$.}
%   \underbrace{
%     \ \ldots\ %
%     \mbox{\tt \$\char`\{}\ \ldots\ %
%   }_{\mbox{\rm region A}}
%   \mbox{\tt \$v e \$.}
%   \underbrace{
%     \mbox{\ \ \ \ldots\ \ \ }
%   }_{\mbox{\rm region B}}
%   \mbox{\tt \$\char`\}}
%   \underbrace{
%     \mbox{\ \ \ \ldots\ \ \ }
%   }_{\mbox{\rm region C}}
%   \mbox{\tt \$\char`\}}
% \]
% \index{\texttt{\$\char`\{} and \texttt{\$\char`\}} keywords}
% In region A, the token \texttt{e} refers to a constant.  It is redeclared as a
% variable in region B, and any reference to it in this region will refer to this
% variable.  In region C, the redeclaration becomes inactive, and the original
% declaration becomes active again.  In region C, the token \texttt{x} refers to the
% original constant.
%
% As a practical matter, overuse of math symbol\index{math symbol}
% redeclarations\index{redeclaration of symbols} can be confusing (even though
% it is well-defined) and is best avoided when possible.  Here are some good
% general guidelines you can follow.  Usually, you should declare all
% constants\index{constant} in the outermost block\index{outermost block},
% especially if they are general-purpose (such as the token \verb$A.$, meaning
% $\forall$ or ``for all'').  This will make them ``globally'' active (although
% as in the example above local redeclarations will temporarily make them
% inactive.)  Most or all variables\index{variable}, on the other hand, could be
% declared in inner blocks, so that the token for them can be used later for a
% different type of variable or a constant.  (The names of the variables you
% choose are not used when you refer to an assertion\index{assertion} in a
% proof, whereas constants must match exactly.  A locally declared constant will
% not match a globally declared constant in a proof, even if they use the same
% token, because Metamath internally considers them to be different math
% symbols.)  To avoid confusion, you should generally avoid redeclaring active
% variables.  If you must redeclare them, do so at the beginning of a block.
% The temporary declaration of constants in inner blocks might be occasionally
% appropriate when you make use of a temporary definition to prove lemmas
% leading to a main result that does not make direct use of the definition.
% This way, you will not clutter up your database with a large number of
% seldom-used global constant symbols.  You might want to note that while
% inactive constants may not appear directly in an assertion (a \texttt{\$a}\index{\texttt{\$a}
% statement} or \texttt{\$p}\index{\texttt{\$p} statement}
% statement), they may be indirectly used in the proof of a \texttt{\$p} statement
% so long as they do not appear in the final math symbol sequence constructed by
% the proof.  In the end, you will have to use your best judgment, taking into
% account standard mathematical usage of the symbols as well as consideration
% for the reader of your work.
%
% \subsubsection{Reuse of Labels}\index{reuse of labels}\index{label}
%
% The \texttt{\$e}\index{\texttt{\$e} statement}, \texttt{\$f}\index{\texttt{\$f}
% statement}, \texttt{\$a}\index{\texttt{\$a} statement}, and
% \texttt{\$p}\index{\texttt{\$p}
% statement} statement types require labels, which allow them to be
% referenced later inside proofs.  A label is considered {\bf
% active}\index{active label} when the statement it is associated with is
% active.  The token\index{token} for a label may be reused
% (redeclared)\index{redeclaration of labels} provided that it is not being used
% for a currently active label.  (Unlike the tokens for math symbols, active
% label tokens may not be redeclared in an inner scope.)  Note that the labels
% of \texttt{\$a} and \texttt{\$p} statements can never be reused after these
% statements appear, because these statements remain active through the end of
% the database.
%
% You might find the reuse of labels a convenient way to have standard names for
% temporary hypotheses, such as \texttt{h1}, \texttt{h2}, etc.  This way you don't have
% to invent unique names for each of them, and in some cases it may be less
% confusing to the reader (although in other cases it might be more confusing, if
% the hypothesis is located far away from the assertion that uses
% it).\footnote{The current implementation requires that all labels, even
% inactive ones, be unique.}

\subsubsection{Frames Revisited}\index{frames and scoping statements}

Now that we have covered scoping, we will look at how an arbitrary
Metamath database can be converted to the simple sequence of extended
frames described on p.~\pageref{framelist}.  This is also how Metamath
stores the database internally when it reads in the database
source.\label{frameconvert} The method is simple.  First, we collect all
constant and variable (\texttt{\$c} and \texttt{\$v}) declarations in
the database, ignoring duplicate declarations of the same variable in
different scopes.  We then put our collected \texttt{\$c} and
\texttt{\$v} declarations at the beginning of the database, so that
their scope is the entire database.  Next, for each assertion in the
database, we determine its frame and extended frame.  The extended frame
is simply the \texttt{\$f}, \texttt{\$e}, and \texttt{\$d} statements
that are active.  The frame is the extended frame with all optional
hypotheses removed.

An equivalent way of saying this is that the extended frame of an assertion
is the collection of all \texttt{\$f}, \texttt{\$e}, and \texttt{\$d} statements
whose scope includes the assertion.
The \texttt{\$f} and \texttt{\$e} statements
occur in the order they appear
(order is irrelevant for \texttt{\$d} statements).

%c%, renaming any
%c%redeclared variables as needed so that all of them have unique names.  (The
%c%exact renaming convention is unimportant.  You might imagine renaming
%c%different declarations of math symbol \texttt{a} as \texttt{a\$1}, \texttt{a\$2}, etc.\
%c%which would prevent any conflicts since \texttt{\$} is not a legal character in a
%c%math symbol token.)

\section{The Anatomy of a Proof} \label{proof}
\index{proof!Metamath, description of}

Each provable assertion (\texttt{\$p}\index{\texttt{\$p} statement} statement) in a
database must include a {\bf proof}\index{proof}.  The proof is located
between the \texttt{\$=}\index{\texttt{\$=} keyword} and \texttt{\$.}\ keywords in the
\texttt{\$p} statement.

In the basic Metamath language\index{basic language}, a proof is a
sequence of statement labels.  This label sequence\index{label sequence}
serves as a set of instructions that the Metamath program uses to
construct a series of math symbol sequences.  The construction must
ultimately result in the math symbol sequence contained between the
\texttt{\$p}\index{\texttt{\$p} statement} and
\texttt{\$=}\index{\texttt{\$=} keyword} keywords of the \texttt{\$p}
statement.  Otherwise, the Metamath program will consider the proof
incorrect, and it will notify you with an appropriate error message when
you ask it to verify the proof.\footnote{To make the loading faster, the
Metamath program does not automatically verify proofs when you
\texttt{read} in a database unless you use the \texttt{/verify}
qualifier.  After a database has been read in, you may use the
\texttt{verify proof *} command to verify proofs.}\index{\texttt{verify
proof} command} Each label in a proof is said to {\bf
reference}\index{label reference} its corresponding statement.

Associated with any assertion\index{assertion} (\texttt{\$p} or
\texttt{\$a}\index{\texttt{\$a} statement} statement) is a set of
hypotheses (\texttt{\$f}\index{\texttt{\$f} statement} or
\texttt{\$e}\index{\texttt{\$e} statement} statements) that are active
with respect to that assertion.  Some are mandatory and the others are
optional.  You should review these concepts if necessary.

Each label\index{label} in a proof must be either the label of a
previous assertion (\texttt{\$a}\index{\texttt{\$a} statement} or
\texttt{\$p}\index{\texttt{\$p} statement} statement) or the label of an
active hypothesis (\texttt{\$e} or \texttt{\$f}\index{\texttt{\$f}
statement} statement) of the \texttt{\$p} statement containing the
proof.  Hypothesis labels may reference both the
mandatory\index{mandatory hypothesis} and the optional hypotheses of the
\texttt{\$p} statement.

The label sequence in a proof specifies a construction in {\bf reverse Polish
notation}\index{reverse Polish notation (RPN)} (RPN).  You may be familiar
with RPN if you have used older
Hewlett--Packard or similar hand-held calculators.
In the calculator analogy, a hypothesis label\index{hypothesis label} is like
a number and an assertion label\index{assertion label} is like an operation
(more precisely, an $n$-ary operation when the
assertion has $n$ \texttt{\$e}-hypotheses).
On an RPN calculator, an operation takes one or more previous numbers in an
input sequence, performs a calculation on them, and replaces those numbers and
itself with the result of the calculation.  For example, the input sequence
$2,3,+$ on an RPN calculator results in $5$, and the input sequence
$2,3,5,{\times},+$ results in $2,15,+$ which results in $17$.

Understanding how RPN is processed involves the concept of a {\bf
stack}\index{stack}\index{RPN stack}, which can be thought of as a set of
temporary memory locations that hold intermediate results.  When Metamath
encounters a hypothesis label it places or {\bf pushes}\index{push} the math
symbol sequence of the hypothesis onto the stack.  When Metamath encounters an
assertion label, it associates the most recent stack entries with the {\em
mandatory} hypotheses\index{mandatory hypothesis} of the assertion, in the
order where the most recent stack entry is associated with the last mandatory
hypothesis of the assertion.  It then determines what
substitutions\index{substitution!variable}\index{variable substitution} have
to be made into the variables of the assertion's mandatory hypotheses to make
them identical to the associated stack entries.  It then makes those same
substitutions into the assertion itself.  Finally, Metamath removes or {\bf
pops}\index{pop} the matched hypotheses from the stack and pushes the
substituted assertion onto the stack.

For the purpose of matching the mandatory hypothesis to the most recent stack
entries, whether a hypothesis is a \texttt{\$e} or \texttt{\$f} statement is
irrelevant.  The only important thing is that a set of
substitutions\footnote{In the Metamath spec (Section~\ref{spec}), we use the
singular term ``substitution'' to refer to the set of substitutions we talk
about here.} exist that allow a match (and if they don't, the proof verifier
will let you know with an error message).  The Metamath language is specified
in such a way that if a set of substitutions exists, it will be unique.
Specifically, the requirement that each variable have a type specified for it
with a \texttt{\$f} statement ensures the uniqueness.

We will illustrate this with an example.
Consider the following Metamath source file:
\begin{verbatim}
$c ( ) -> wff $.
$v p q r s $.
wp $f wff p $.
wq $f wff q $.
wr $f wff r $.
ws $f wff s $.
w2 $a wff ( p -> q ) $.
wnew $p wff ( s -> ( r -> p ) ) $= ws wr wp w2 w2 $.
\end{verbatim}
This Metamath source example shows the definition and ``proof'' (i.e.,
construction) of a well-formed formula (wff)\index{well-formed formula (wff)}
in propositional calculus.  (You may wish to type this example into a file to
experiment with the Metamath program.)  The first two statements declare
(introduce the names of) four constants and four variables.  The next four
statements specify the variable types, namely that
each variable is assumed to be a wff.  Statement \texttt{w2} defines (postulates)
a way to produce a new wff, \texttt{( p -> q )}, from two given wffs \texttt{p} and
\texttt{q}. The mandatory hypotheses of \texttt{w2} are \texttt{wp} and \texttt{wq}.
Statement \texttt{wnew} claims that \texttt{( s -> ( r -> p ) )} is a wff given
three wffs \texttt{s}, \texttt{r}, and \texttt{p}.  More precisely, \texttt{wnew} claims
that the sequence of ten symbols \texttt{wff ( s -> ( r -> p ) )} is provable from
previous assertions and the hypotheses of \texttt{wnew}.  Metamath does not know
or care what a wff is, and as far as it is concerned
the typecode \texttt{wff} is just an
arbitrary constant symbol in a math symbol sequence.  The mandatory hypotheses
of \texttt{wnew} are \texttt{wp}, \texttt{wr}, and \texttt{ws}; \texttt{wq} is an optional
hypothesis.  In our particular proof, the optional hypothesis is not
referenced, but in general, any combination of active (i.e.\ optional and
mandatory) hypotheses could be referenced.  The proof of statement \texttt{wnew}
is the sequence of five labels starting with \texttt{ws} (step~1) and ending with
\texttt{w2} (step~5).

When Metamath verifies the proof, it scans the proof from left to right.  We
will examine what happens at each step of the proof.  The stack starts off
empty.  At step 1, Metamath looks up label \texttt{ws} and determines that it is a
hypothesis, so it pushes the symbol sequence of statement \texttt{ws} onto the
stack:

\begin{center}\begin{tabular}{|l|l|}\hline
{Stack location} & {Contents} \\ \hline \hline
1 & \texttt{wff s} \\ \hline
\end{tabular}\end{center}

Metamath sees that the labels \texttt{wr} and \texttt{wp} in steps~2 and 3 are also
hypotheses, so it pushes them onto the stack.  After step~3, the stack looks
like
this:

\begin{center}\begin{tabular}{|l|l|}\hline
{Stack location} & {Contents} \\ \hline \hline
3 & \texttt{wff p} \\ \hline
2 & \texttt{wff r} \\ \hline
1 & \texttt{wff s} \\ \hline
\end{tabular}\end{center}

At step 4, Metamath sees that label \texttt{w2} is an assertion, so it must do
some processing.  First, it associates the mandatory hypotheses of \texttt{w2},
which are \texttt{wp} and \texttt{wq}, with stack locations~2 and 3, {\em in that
order}. Metamath determines that the only possible way
to make hypothesis \texttt{wp} match (become identical to) stack location~2 and
\texttt{wq} match stack location 3 is to substitute variable \texttt{p} with \texttt{r}
and \texttt{q} with \texttt{p}.  Metamath makes these substitutions into \texttt{w2} and
obtains the symbol sequence \texttt{wff ( r -> p )}.  It removes the hypotheses
from stack locations~2 and 3, then places the result into stack location~2:

\begin{center}\begin{tabular}{|l|l|}\hline
{Stack location} & {Contents} \\ \hline \hline
2 & \texttt{wff ( r -> p )} \\ \hline
1 & \texttt{wff s} \\ \hline
\end{tabular}\end{center}

At step 5, Metamath sees that label \texttt{w2} is an assertion, so it must again
do some processing.  First, it matches the mandatory hypotheses of \texttt{w2},
which are \texttt{wp} and \texttt{wq}, to stack locations 1 and 2.
Metamath determines that the only possible way to make the
hypotheses match is to substitute variable \texttt{p} with \texttt{s} and \texttt{q} with
\texttt{( r -> p )}.  Metamath makes these substitutions into \texttt{w2} and obtains
the symbol
sequence \texttt{wff ( s -> ( r -> p ) )}.  It removes stack
locations 1 and 2, then places the result into stack location~1:

\begin{center}\begin{tabular}{|l|l|}\hline
{Stack location} & {Contents} \\ \hline \hline
1 & \texttt{wff ( s -> ( r -> p ) )} \\ \hline
\end{tabular}\end{center}

After Metamath finishes processing the proof, it checks to see that the
stack contains exactly one element and that this element is
the same as the math symbol sequence in the
\texttt{\$p}\index{\texttt{\$p} statement} statement.  This is the case for our
proof of \texttt{wnew},
so we have proved \texttt{wnew} successfully.  If the result
differs, Metamath will notify you with an error message.  An error message
will also result if the stack contains more than one entry at the end of the
proof, or if the stack did not contain enough entries at any point in the
proof to match all of the mandatory hypotheses\index{mandatory hypothesis} of
an assertion.  Finally, Metamath will notify you with an error message if no
substitution is possible that will make a referenced assertion's hypothesis
match the
stack entries.  You may want to experiment with the different kinds of errors
that Metamath will detect by making some small changes in the proof of our
example.

Metamath's proof notation was designed primarily to express proofs in a
relatively compact manner, not for readability by humans.  Metamath can display
proofs in a number of different ways with the \texttt{show proof}\index{\texttt{show
proof} command} command.  The
\texttt{/lemmon} qualifier displays it in a format that is easier to read when the
proofs are short, and you saw examples of its use in Chapter~\ref{using}.  For
longer proofs, it is useful to see the tree structure of the proof.  A tree
structure is displayed when the \texttt{/lemmon} qualifier is omitted.  You will
probably find this display more convenient as you get used to it. The tree
display of the proof in our example looks like
this:\label{treeproof}\index{tree-style proof}\index{proof!tree-style}
\begin{verbatim}
1     wp=ws    $f wff s
2        wp=wr    $f wff r
3        wq=wp    $f wff p
4     wq=w2    $a wff ( r -> p )
5  wnew=w2  $a wff ( s -> ( r -> p ) )
\end{verbatim}
The number to the left of each line is the step number.  Following it is a
{\bf hypothesis association}\index{hypothesis association}, consisting of two
labels\index{label} separated by \texttt{=}.  To the left of the \texttt{=} (except
in the last step) is the label of a hypothesis of an assertion referenced
later in the proof; here, steps 1 and 4 are the hypothesis associations for
the assertion \texttt{w2} that is referenced in step 5.  A hypothesis association
is indented one level more than the assertion that uses it, so it is easy to
find the corresponding assertion by moving directly down until the indentation
level decreases to one less than where you started from.  To the right of each
\texttt{=} is the proof step label for that proof step.  The statement keyword of
the proof step label is listed next, followed by the content of the top of the
stack (the most recent stack entry) as it exists after that proof step is
processed.  With a little practice, you should have no trouble reading proofs
displayed in this format.

Metamath proofs include the syntax construction of a formula.
In standard mathematics, this kind of
construction is not considered a proper part of the proof at all, and it
certainly becomes rather boring after a while.
Therefore,
by default the \texttt{show proof}\index{\texttt{show proof}
command} command does not show the syntax construction.
Historically \texttt{show proof} command
\textit{did} show the syntax construction, and you needed to add the
\texttt{/essential} option to hide, them, but today
\texttt{/essential} is the default and you need to use
\texttt{/all} to see the syntax constructions.

When verifying a proof, Metamath will check that no mandatory
\texttt{\$d}\index{\texttt{\$d} statement}\index{mandatory \texttt{\$d}
statement} statement of an assertion referenced in a proof is violated
when substitutions\index{substitution!variable}\index{variable
substitution} are made to the variables in the assertion.  For details
see Section~\ref{spec4} or \ref{dollard}.

\subsection{The Concept of Unification} \label{unify}

During the course of verifying a proof, when Metamath\index{Metamath}
encounters an assertion label\index{assertion label}, it associates the
mandatory hypotheses\index{mandatory hypothesis} of the assertion with the top
entries of the RPN stack\index{stack}\index{RPN stack}.  Metamath then
determines what substitutions\index{substitution!variable}\index{variable
substitution} it must make to the variables in the assertion's mandatory
hypotheses in order for these hypotheses to become identical to their
corresponding stack entries.  This process is called {\bf
unification}\index{unification}.  (We also informally use the term
``unification'' to refer to a set of substitutions that results from the
process, as in ``two unifications are possible.'')  After the substitutions
are made, the hypotheses are said to be {\bf unified}.

If no such substitutions are possible, Metamath will consider the proof
incorrect and notify you with an error message.
% (deleted 3/10/07, per suggestion of Mel O'Cat:)
% The syntax of the
% Metamath language ensures that if a set of substitutions exists, it
% will be unique.

The general algorithm for unification described in the literature is
somewhat complex.
However, in the case of Metamath it is intentionally trivial.
Mandatory hypotheses must be
pushed on the proof stack in the order in which they appear.
In addition, each variable must have its type specified
with a \texttt{\$f} hypothesis before it is used
and that each \texttt{\$f} hypothesis
have the restricted syntax of a typecode (a constant) followed by a variable.
The typecode in the \texttt{\$f} hypothesis must match the first symbol of
the corresponding RPN stack entry (which will also be a constant), so
the only possible match for the variable in the \texttt{\$f} hypothesis is
the sequence of symbols in the stack entry after the initial constant.

In the Proof Assistant\index{Proof Assistant}, a more general unification
algorithm is used.  While a proof is being developed, sometimes not enough
information is available to determine a unique unification.  In this case
Metamath will ask you to pick the correct one.\index{ambiguous
unification}\index{unification!ambiguous}

\section{Extensions to the Metamath Language}\index{extended
language}

\subsection{Comments in the Metamath Language}\label{comments}
\index{markup notation}
\index{comments!markup notation}

The commenting feature allows you to annotate the contents of
a database.  Just as with most
computer languages, comments are ignored for the purpose of interpreting the
contents of the database. Comments effectively act as
additional white space\index{white
space} between tokens
when a database is parsed.

A comment may be placed at the beginning, end, or
between any two tokens\index{token} in a source file.

Comments have the following syntax:
\begin{center}
 \texttt{\$(} {\em text} \texttt{\$)}
\end{center}
Here,\index{\texttt{\$(} and \texttt{\$)} auxiliary
keywords}\index{comment} {\em text} is a string, possibly empty, of any
characters in Metamath's character set (p.~\pageref{spec1chars}), except
that the character strings \texttt{\$(} and \texttt{\$)} may not appear
in {\em text}.  Thus nested comments are not
permitted:\footnote{Computer languages have differing standards for
nested comments, and rather than picking one it was felt simplest not to
allow them at all, at least in the current version of
Metamath\index{Metamath!limitations of version 0.07.30}.} Metamath will
complain if you give it
\begin{center}
 \texttt{\$( This is a \$( nested \$) comment.\ \$)}
\end{center}
To compensate for this non-nesting behavior, I often change all \texttt{\$}'s
to \texttt{@}'s in sections of Metamath code I wish to comment out.

The Metamath program supports a number of markup mechanisms and conventions
to generate good-looking results in \LaTeX\ and {\sc html},
as discussed below.
These markup features have to do only with how the comments are typeset,
and have no effect on how Metamath verifies the proofs in the database.
The improper
use of them may result in incorrectly typeset output, but no Metamath
error messages will result during the \texttt{read} and \texttt{verify
proof} commands.  (However, the \texttt{write
theorem\texttt{\char`\_}list} command
will check for markup errors as a side-effect of its
{\sc html} generation.)
Section~\ref{texout} has instructions for creating \LaTeX\ output, and
section~\ref{htmlout} has instructions for creating
{\sc html}\index{HTML} output.

\subsubsection{Headings}\label{commentheadings}

If the \texttt{\$(} is immediately followed by a new line
starting with a heading marker, it is a header.
This can start with:

\begin{itemize}
 \item[] \texttt{\#\#\#\#} - major part header
 \item[] \texttt{\#*\#*} - section header
 \item[] \texttt{=-=-} - subsection header
 \item[] \texttt{-.-.} - subsubsection header
\end{itemize}

The line following the marker line
will be used for the table of contents entry, after trimming spaces.
The next line should be another (closing) matching marker line.
Any text after that
but before the closing \texttt{\$}, such as an extended description of the
section, will be included on the \texttt{mmtheoremsNNN.html} page.

For more information, run
\texttt{help write theorem\char`\_list}.

\subsubsection{Math mode}
\label{mathcomments}
\index{\texttt{`} inside comments}
\index{\texttt{\char`\~} inside comments}
\index{math mode}

Inside of comments, a string of tokens\index{token} enclosed in
grave accents\index{grave accent (\texttt{`})} (\texttt{`}) will be converted
to standard mathematical symbols during
{\sc HTML}\index{HTML} or \LaTeX\ output
typesetting,\index{latex@{\LaTeX}} according to the information in the
special \texttt{\$t}\index{\texttt{\$t} comment}\index{typesetting
comment} comment in the database
(see section~\ref{tcomment} for information about the typesetting
comment, and Appendix~\ref{ASCII} to see examples of its results).

The first grave accent\index{grave accent (\texttt{`})} \texttt{`}
causes the output processor to enter {\bf math mode}\index{math mode}
and the second one exits it.
In this
mode, the characters following the \texttt{`} are interpreted as a
sequence of math symbol tokens separated by white space\index{white
space}.  The tokens are looked up in the \texttt{\$t}
comment\index{\texttt{\$t} comment}\index{typesetting comment} and if
found, they will be replaced by the standard mathematical symbols that
they correspond to before being placed in the typeset output file.  If
not found, the symbol will be output as is and a warning will be issued.
The tokens do not have to be active in the database, although a warning
will be issued if they are not declared with \texttt{\$c} or
\texttt{\$v} statements.

Two consecutive
grave accents \texttt{``} are treated as a single actual grave accent
(both inside and outside of math mode) and will not cause the output
processor to enter or exit math mode.

Here is an example of its use\index{Pierce's axiom}:
\begin{center}
\texttt{\$( Pierce's axiom, ` ( ( ph -> ps ) -> ph ) -> ph ` ,\\
         is not very intuitive. \$)}
\end{center}
becomes
\begin{center}
   \texttt{\$(} Pierce's axiom, $((\varphi \rightarrow \psi)\rightarrow
\varphi)\rightarrow \varphi$, is not very intuitive. \texttt{\$)}
\end{center}

Note that the math symbol tokens\index{token} must be surrounded by white
space\index{white space}.
%, since there is no context that allows ambiguity to be
%resolved, as is the case with math symbol sequences in some of the Metamath
%statements.
White space should also surround the \texttt{`}
delimiters.

The math mode feature also gives you a quick and easy way to generate
text containing mathematical symbols, independently of the intended
purpose of Metamath.\index{Metamath!using as a math editor} To do this,
simply create your text with grave accents surrounding your formulas,
after making sure that your math symbols are mapped to \LaTeX\ symbols
as described in Appendix~\ref{ASCII}.  It is easier if you start with a
database with predefined symbols such as \texttt{set.mm}.  Use your
grave-quoted math string to replace an existing comment, then typeset
the statement corresponding to that comment following the instructions
from the \texttt{help tex} command in the Metamath program.  You will
then probably want to edit the resulting file with a text editor to fine
tune it to your exact needs.

\subsubsection{Label Mode}\index{label mode}

Outside of math mode, a tilde\index{tilde (\texttt{\char`\~})} \verb/~/
indicates to Metamath's\index{Metamath} output processor that the
token\index{token} that follows (i.e.\ the characters up to the next
white space\index{white space}) represents a statement label or URL.
This formatting mode is called {\bf label mode}\index{label mode}.
If a literal tilde
is desired (outside of math mode) instead of label mode,
use two tildes in a row to represent it.

When generating a \LaTeX\ output file,
the following token will be formatted in \texttt{typewriter}
font, and the tilde removed, to make it stand out from the rest of the text.
This formatting will be applied to all characters after the
tilde up to the first white space\index{white space}.
Whether
or not the token is an actual statement label is not checked, and the
token does not have to have the correct syntax for a label; no error
messages will be produced.  The only effect of the label mode on the
output is that typewriter font will be used for the tokens that are
placed in the \LaTeX\ output file.

When generating {\sc html},
the tokens after the tilde {\em must} be a URL (either http: or https:)
or a valid label.
Error messages will be issued during that output if they aren't.
A hyperlink will be generated to that URL or label.

\subsubsection{Link to bibliographical reference}\index{citation}%
\index{link to bibliographical reference}

Bibliographical references are handled specially when generating
{\sc html} if formatted specially.
Text in the form \texttt{[}{\em author}\texttt{]}
is considered a link to a bibliographical reference.
See \texttt{help html} and \texttt{help write
bibliography} in the Metamath program for more
information.
% \index{\texttt{\char`\[}\ldots\texttt{]} inside comments}
See also Sections~\ref{tcomment} and \ref{wrbib}.

The \texttt{[}{\em author}\texttt{]} notation will also create an entry in
the bibliography cross-reference file generated by \texttt{write
bibliography} (Section~\ref{wrbib}) for {\sc HTML}.
For this to work properly, the
surrounding comment must be formatted as follows:
\begin{quote}
    {\em keyword} {\em label} {\em noise-word}
     \texttt{[}{\em author}\texttt{] p.} {\em number}
\end{quote}
for example
\begin{verbatim}
     Theorem 5.2 of [Monk] p. 223
\end{verbatim}
The {\em keyword} is not case sensitive and must be one of the following:
\begin{verbatim}
     theorem lemma definition compare proposition corollary
     axiom rule remark exercise problem notation example
     property figure postulate equation scheme chapter
\end{verbatim}
The optional {\em label} may consist of more than one
(non-{\em keyword} and non-{\em noise-word}) word.
The optional {\em noise-word} is one of:
\begin{verbatim}
     of in from on
\end{verbatim}
and is  ignored when the cross-reference file is created.  The
\texttt{write
biblio\-graphy} command will perform error checking to verify the
above format.\index{error checking}

\subsubsection{Parentheticals}\label{parentheticals}

The end of a comment may include one or more parenthicals, that is,
statements enclosed in parentheses.
The Metamath program looks for certain parentheticals and can issue
warnings based on them.
They are:

\begin{itemize}
 \item[] \texttt{(Contributed by }
   \textit{NAME}\texttt{,} \textit{DATE}\texttt{.)} -
   document the original contributor's name and the date it was created.
 \item[] \texttt{(Revised by }
   \textit{NAME}\texttt{,} \textit{DATE}\texttt{.)} -
   document the contributor's name and creation date
   that resulted in significant revision
   (not just an automated minimization or shortening).
 \item[] \texttt{(Proof shortened by }
   \textit{NAME}\texttt{,} \textit{DATE}\texttt{.)} -
   document the contributor's name and date that developed a significant
   shortening of the proof (not just an automated minimization).
 \item[] \texttt{(Proof modification is discouraged.)} -
   Note that this proof should normally not be modified.
 \item[] \texttt{(New usage is discouraged.)} -
   Note that this assertion should normally not be used.
\end{itemize}

The \textit{DATE} must be in form YYYY-MMM-DD, where MMM is the
English abbreviation of that month.

\subsubsection{Other markup}\label{othermarkup}
\index{markup notation}

There are other markup notations for generating good-looking results
beyond math mode and label mode:

\begin{itemize}
 \item[]
         \texttt{\char`\_} (underscore)\index{\texttt{\char`\_} inside comments} -
             Italicize text starting from
              {\em space}\texttt{\char`\_}{\em non-space} (i.e.\ \texttt{\char`\_}
              with a space before it and a non-space character after it) until
             the next
             {\em non-space}\texttt{\char`\_}{\em space}.  Normal
             punctuation (e.g.\ a trailing
             comma or period) is ignored when determining {\em space}.
 \item[]
         \texttt{\char`\_} (underscore) - {\em
         non-space}\texttt{\char`\_}{\em non-space-string}, where
          {\em non-space-string} is a string of non-space characters,
         will make {\em non-space-string} become a subscript.
 \item[]
         \texttt{<HTML>}...\texttt{</HTML>} - do not convert
         ``\texttt{<}'' and ``\texttt{>}''
         in the enclosed text when generating {\sc HTML},
         otherwise process markup normally. This allows direct insertion
         of {\sc html} commands.
 \item[]
       ``\texttt{\&}ref\texttt{;}'' - insert an {\sc HTML}
         character reference.
         This is how to insert arbitrary Unicode characters
         (such as accented characters).  Currently only directly supported
         when generating {\sc HTML}.
\end{itemize}

It is recommended that spaces surround any \texttt{\char`\~} and
\texttt{`} tokens in the comment and that a space follow the {\em label}
after a \texttt{\char`\~} token.  This will make global substitutions
to change labels and symbol names much easier and also eliminate any
future chance of ambiguity.  Spaces around these tokens are automatically
removed in the final output to conform with normal rules of punctuation;
for example, a space between a trailing \texttt{`} and a left parenthesis
will be removed.

A good way to become familiar with the markup notation is to look at
the extensive examples in the \texttt{set.mm} database.

\subsection{The Typesetting Comment (\texttt{\$t})}\label{tcomment}

The typesetting comment \texttt{\$t} in the input database file
provides the information necessary to produce good-looking results.
It provides \LaTeX\ and {\sc html}
definitions for math symbols,
as well supporting as some
customization of the generated web page.
If you add a new token to a database, you should also
update the \texttt{\$t} comment information if you want to eventually
create output in \LaTeX\ or {\sc HTML}.
See the
\texttt{set.mm}\index{set theory database (\texttt{set.mm})} database
file for an extensive example of a \texttt{\$t} comment illustrating
many of the features described below.

Programs that do not need to generate good-looking presentation results,
such as programs that only verify Metamath databases,
can completely ignore typesetting comments
and just treat them as normal comments.
Even the Metamath program only consults the
\texttt{\$t} comment information when it needs to generate typeset output
in \LaTeX\ or {\sc HTML}
(e.g., when you open a \LaTeX\ output file with the \texttt{open tex} command).

We will first discuss the syntax of typesetting comments, and then
briefly discuss how this can be used within the Metamath program.

\subsubsection{Typesetting Comment Syntax Overview}

The typesetting comment is identified by the token
\texttt{\$t}\index{\texttt{\$t} comment}\index{typesetting comment} in
the comment, and the typesetting comment ends at the matching
\texttt{\$)}:
\[
  \mbox{\tt \$(\ }
  \mbox{\tt \$t\ }
  \underbrace{
    \mbox{\tt \ \ \ \ \ \ \ \ \ \ \ }
    \cdots
    \mbox{\tt \ \ \ \ \ \ \ \ \ \ \ }
  }_{\mbox{Typesetting definitions go here}}
  \mbox{\tt \ \$)}
\]

There must be one or more white space characters, and only white space
characters, between the \texttt{\$(} that starts the comment
and the \texttt{\$t} symbol,
and the \texttt{\$t} must be followed by one
or more white space characters
(see section \ref{whitespace} for the definition of white space characters).
The typesetting comment continues until the comment end token \texttt{\$)}
(which must be preceded by one or more white space characters).

In version 0.07.30\index{Metamath!limitations of version 0.07.30} of the
Metamath program, there may be only one \texttt{\$t} comment in a
database.  This restriction may be lifted in the future to allow
many \texttt{\$t} comments in a database.

Between the \texttt{\$t} symbol (and its following white space) and the
comment end token \texttt{\$)} (and its preceding white space)
is a sequence of one or more typesetting definitions, where
each definition has the form
\textit{definition-type arg arg ... ;}.
Each of the zero or more \textit{arg} values
can be either a typesetting data or a keyword
(what keywords are allowed, and where, depends on the specific
\textit{definition-type}).
The \textit{definition-type}, and each argument \textit{arg},
are separated by one or more white space characters.
Every definition ends in an unquoted semicolon;
white space is not required before the terminating semicolon of a definition.
Each definition should start on a new line.\footnote{This
restriction of the current version of Metamath
(0.07.30)\index{Metamath!limitations of version 0.07.30} may be removed
in a future version, but you should do it anyway for readability.}

For example, this typesetting definition:
\begin{center}
 \verb$latexdef "C_" as "\subseteq";$
\end{center}
defines the token \verb$C_$ as the \LaTeX\ symbol $\subseteq$ (which means
``subset'').

Typesetting data is a sequence of one or more quoted strings
(if there is more than one, they are connected by \texttt{\char`\+}).
Often a single quoted string is used to provide data for a definition, using
either double (\texttt{\char`\"}) or single (\texttt{'}) quotation marks.
However,
{\em a quoted string (enclosed in quotation marks) may not include
line breaks.}
A quoted string
may include a quotation mark that matches the enclosing quotes by repeating
the quotation mark twice.  Here are some examples:

\begin{tabu}   { l l }
\textbf{Example} & \textbf{Meaning} \\
\texttt{\char`\"a\char`\"\char`\"b\char`\"} & \texttt{a\char`\"b} \\
\texttt{'c''d'} & \texttt{c'd} \\
\texttt{\char`\"e''f\char`\"} & \texttt{e''f} \\
\texttt{'g\char`\"\char`\"h'} & \texttt{g\char`\"\char`\"h} \\
\end{tabu}

Finally, a long quoted string
may be broken up into multiple quoted strings (considered, as a whole,
a single quoted string) and joined with \texttt{\char`\+}.
You can even use multiple lines as long as a
'+' is at the end of every line except the last one.
The \texttt{\char`\+} should be preceded and followed by at least one
white space character.
Thus, for example,
\begin{center}
 \texttt{\char`\"ab\char`\"\ \char`\+\ \char`\"cd\char`\"
    \ \char`\+\ \\ 'ef'}
\end{center}
is the same as
\begin{center}
 \texttt{\char`\"abcdef\char`\"}
\end{center}

{\sc c}-style comments \texttt{/*}\ldots\texttt{*/} are also supported.

In practice, whenever you add a new math token you will often want to add
typesetting definitions using
\texttt{latexdef}, \texttt{htmldef}, and
\texttt{althtmldef}, as described below.
That way, they will all be up to date.
Of course, whether or not you want to use all three definitions will
depend on how the database is intended to be used.

Below we discuss the different possible \textit{definition-kind} options.
We will show data surrounded by double quotes (in practice they can also use
single quotes and/or be a sequence joined by \texttt{+}s).
We will use specific names for the \textit{data} to make clear what
the data is used for, such as
{\em math-token} (for a Metamath math token,
{\em latex-string} (for string to be placed in a \LaTeX\ stream),
{\em {\sc html}-code} (for {\sc html} code),
and {\em filename} (for a filename).

\subsubsection{Typesetting Comment - \LaTeX}

The syntax for a \LaTeX\ definition is:
\begin{center}
 \texttt{latexdef "}{\em math-token}\texttt{" as "}{\em latex-string}\texttt{";}
\end{center}
\index{latex definitions@\LaTeX\ definitions}%
\index{\texttt{latexdef} statement}

The {\em token-string} and {\em latex-string} are the data
(character strings) for
the token and the \LaTeX\ definition of the token, respectively,

These \LaTeX\ definitions are used by the Metamath program
when it is asked to product \LaTeX output using
the \texttt{write tex} command.

\subsubsection{Typesetting Comment - {\sc html}}

The key kinds of {\sc HTML} definitions have the following syntax:

\vskip 1ex
    \texttt{htmldef "}{\em math-token}\texttt{" as "}{\em
    {\sc html}-code}\texttt{";}\index{\texttt{htmldef} statement}
                    \ \ \ \ \ \ldots

    \texttt{althtmldef "}{\em math-token}\texttt{" as "}{\em
{\sc html}-code}\texttt{";}\index{\texttt{althtmldef} statement}

                    \ \ \ \ \ \ldots

Note that in {\sc HTML} there are two possible definitions for math tokens.
This feature is useful when
an alternate representation of symbols is desired, for example one that
uses Unicode entities and another uses {\sc gif} images.

There are many other typesetting definitions that can control {\sc HTML}.
These include:

\vskip 1ex

    \texttt{htmldef "}{\em math-token}\texttt{" as "}{\em {\sc
    html}-code}\texttt{";}

    \texttt{htmltitle "}{\em {\sc html}-code}\texttt{";}%
\index{\texttt{htmltitle} statement}

    \texttt{htmlhome "}{\em {\sc html}-code}\texttt{";}%
\index{\texttt{htmlhome} statement}

    \texttt{htmlvarcolor "}{\em {\sc html}-code}\texttt{";}%
\index{\texttt{htmlvarcolor} statement}

    \texttt{htmlbibliography "}{\em filename}\texttt{";}%
\index{\texttt{htmlbibliography} statement}

\vskip 1ex

\noindent The \texttt{htmltitle} is the {\sc html} code for a common
title, such as ``Metamath Proof Explorer.''  The \texttt{htmlhome} is
code for a link back to the home page.  The \texttt{htmlvarcolor} is
code for a color key that appears at the bottom of each proof.  The file
specified by {\em filename} is an {\sc html} file that is assumed to
have a \texttt{<A NAME=}\ldots\texttt{>} tag for each bibiographic
reference in the database comments.  For example, if
\texttt{[Monk]}\index{\texttt{\char`\[}\ldots\texttt{]} inside comments}
occurs in the comment for a theorem, then \texttt{<A NAME='Monk'>} must
be present in the file; if not, a warning message is given.

Associated with
\texttt{althtmldef}
are the statements
\vskip 1ex

    \texttt{htmldir "}{\em
      directoryname}\texttt{";}\index{\texttt{htmldir} statement}

    \texttt{althtmldir "}{\em
     directoryname}\texttt{";}\index{\texttt{althtmldir} statement}

\vskip 1ex
\noindent giving the directories of the {\sc gif} and Unicode versions
respectively; their purpose is to provide cross-linking between the
two versions in the generated web pages.

When two different types of pages need to be produced from a single
database, such as the Hilbert Space Explorer that extends the Metamath
Proof Explorer, ``extended'' variables may be declared in the
\texttt{\$t} comment:
\vskip 1ex

    \texttt{exthtmltitle "}{\em {\sc html}-code}\texttt{";}%
\index{\texttt{exthtmltitle} statement}

    \texttt{exthtmlhome "}{\em {\sc html}-code}\texttt{";}%
\index{\texttt{exthtmlhome} statement}

    \texttt{exthtmlbibliography "}{\em filename}\texttt{";}%
\index{\texttt{exthtmlbibliography} statement}

\vskip 1ex
\noindent When these are declared, you also must declare
\vskip 1ex

    \texttt{exthtmllabel "}{\em label}\texttt{";}%
\index{\texttt{exthtmllabel} statement}

\vskip 1ex \noindent that identifies the database statement where the
``extended'' section of the database starts (in our example, where the
Hilbert Space Explorer starts).  During the generation of web pages for
that starting statement and the statements after it, the {\sc html} code
assigned to \texttt{exthtmltitle} and \texttt{exthtmlhome} is used
instead of that assigned to \texttt{htmltitle} and \texttt{htmlhome},
respectively.

\begin{sloppy}
\subsection{Additional Information Com\-ment (\texttt{\$j})} \label{jcomment}
\end{sloppy}

The additional information comment, aka the
\texttt{\$j}\index{\texttt{\$j} comment}\index{additional information comment}
comment,
provides a way to add additional structured information that can
be optionally parsed by systems.

The additional information comment is parsed the same way as the
typesetting comment (\texttt{\$t}) (see section \ref{tcomment}).
That is,
the additional information comment begins with the token
\texttt{\$j} within a comment,
and continues until the comment close \texttt{\$)}.
Within an additional information comment is a sequence of one or more
commands of the form \texttt{command arg arg ... ;}
where each of the zero or more \texttt{arg} values
can be either a quoted string or a keyword.
Note that every command ends in an unquoted semicolon.
If a verifier is parsing an additional information comment, but
doesn't recognize a particular command, it must skip the command
by finding the end of the command (an unquoted semicolon).

A database may have 0 or more additional information comments.
Note, however, that a verifier may ignore these comments entirely or only
process certain commands in an additional information comment.
The \texttt{mmj2} verifier supports many commands in additional information
comments.
We encourage systems that process additional information comments
to coordinate so that they will use the same command for the same effect.

Examples of additional information comments with various commands
(from the \texttt{set.mm} database) are:

\begin{itemize}
   \item Define the syntax and logical typecodes,
     and declare that our grammar is
     unambiguous (verifiable using the KLR parser, with compositing depth 5).
\begin{verbatim}
  $( $j
    syntax 'wff';
    syntax '|-' as 'wff';
    unambiguous 'klr 5';
  $)
\end{verbatim}

   \item Register $\lnot$ and $\rightarrow$ as primitive expressions
           (lacking definitions).
\begin{verbatim}
  $( $j primitive 'wn' 'wi'; $)
\end{verbatim}

   \item There is a special justification for \texttt{df-bi}.
\begin{verbatim}
  $( $j justification 'bijust' for 'df-bi'; $)
\end{verbatim}

   \item Register $\leftrightarrow$ as an equality for its type (wff).
\begin{verbatim}
  $( $j
    equality 'wb' from 'biid' 'bicomi' 'bitri';
    definition 'dfbi1' for 'wb';
  $)
\end{verbatim}

   \item Theorem \texttt{notbii} is the congruence law for negation.
\begin{verbatim}
  $( $j congruence 'notbii'; $)
\end{verbatim}

   \item Add \texttt{setvar} as a typecode.
\begin{verbatim}
  $( $j syntax 'setvar'; $)
\end{verbatim}

   \item Register $=$ as an equality for its type (\texttt{class}).
\begin{verbatim}
  $( $j equality 'wceq' from 'eqid' 'eqcomi' 'eqtri'; $)
\end{verbatim}

\end{itemize}


\subsection{Including Other Files in a Metamath Source File} \label{include}
\index{\texttt{\$[} and \texttt{\$]} auxiliary keywords}

The keywords \texttt{\$[} and \texttt{\$]} specify a file to be
included\index{included file}\index{file inclusion} at that point in a
Metamath\index{Metamath} source file\index{source file}.  The syntax for
including a file is as follows:
\begin{center}
\texttt{\$[} {\em file-name} \texttt{\$]}
\end{center}

The {\em file-name} should be a single token\index{token} with the same syntax
as a math symbol (i.e., all 93 non-whitespace
printable characters other than \texttt{\$} are
allowed, subject to the file-naming limitations of your operating system).
Comments may appear between the \texttt{\$[} and \texttt{\$]} keywords.  Included
files may include other files, which may in turn include other files, and so
on.

For example, suppose you want to use the set theory database as the starting
point for your own theory.  The first line in your file could be
\begin{center}
\texttt{\$[ set.mm \$]}
\end{center} All of the information (axioms, theorems,
etc.) in \texttt{set.mm} and any files that {\em it} includes will become
available for you to reference in your file. This can help make your work more
modular. A drawback to including files is that if you change the name of a
symbol or the label of a statement, you must also remember to update any
references in any file that includes it.


The naming conventions for included files are the same as those of your
operating system.\footnote{On the Macintosh, prior to Mac OS X,
 a colon is used to separate disk
and folder names from your file name.  For example, {\em volume}\texttt{:}{\em
file-name} refers to the root directory, {\em volume}\texttt{:}{\em
folder-name}\texttt{:}{\em file-name} refers to a folder in root, and {\em
volume}\texttt{:}{\em folder-name}\texttt{:}\ldots\texttt{:}{\em file-name} refers to a
deeper folder.  A simple {\em file-name} refers to a file in the folder from
which you launch the Metamath application.  Under Mac OS X and later,
the Metamath program is run under the Terminal application, which
conforms to Unix naming conventions.}\index{Macintosh file
names}\index{file names!Macintosh}\label{includef} For compatibility among
operating systems, you should keep the file names as simple as possible.  A
good convention to use is {\em file}\texttt{.mm} where {\em file} is eight
characters or less, in lower case.

There is no limit to the nesting depth of included files.  One thing that you
should be aware of is that if two included files themselves include a common
third file, only the {\em first} reference to this common file will be read
in.  This allows you to include two or more files that build on a common
starting file without having to worry about label and symbol conflicts that
would occur if the common file were read in more than once.  (In fact, if a
file includes itself, the self-reference will be ignored, although of course
it would not make any sense to do that.)  This feature also means, however,
that if you try to include a common file in several inner blocks, the result
might not be what you expect, since only the first reference will be replaced
with the included file (unlike the include statement in most other computer
languages).  Thus you would normally include common files only in the
outermost block\index{outermost block}.

\subsection{Compressed Proof Format}\label{compressed1}\index{compressed
proof}\index{proof!compressed}

The proof notation presented in Section~\ref{proof} is called a
{\bf normal proof}\index{normal proof}\index{proof!normal} and in principle is
sufficient to express any proof.  However, proofs often contain steps and
subproofs that are identical.  This is particularly true in typical
Metamath\index{Metamath} applications, because Metamath requires that the math
symbol sequence (usually containing a formula) at each step be separately
constructed, that is, built up piece by piece. As a result, a lot of
repetition often results.  The {\bf compressed proof} format allows Metamath
to take advantage of this redundancy to shorten proofs.

The specification for the compressed proof format is given in
Appen\-dix~\ref{compressed}.

Normally you need not concern yourself with the details of the compressed
proof format, since the Metamath program will allow you to convert from
the normal format to the compressed format with ease, and will also
automatically convert from the compressed format when proofs are displayed.
The overall structure of the compressed format is as follows:
\begin{center}
  \texttt{\$= ( } {\em label-list} \texttt{) } {\em compressed-proof\ }\ \texttt{\$.}
\end{center}
\index{\texttt{\$=} keyword}
The first \texttt{(} serves as a flag to Metamath that a compressed proof
follows.  The {\em label-list} includes all statements referred to by the
proof except the mandatory hypotheses\index{mandatory hypothesis}.  The {\em
compressed-proof} is a compact encoding of the proof, using upper-case
letters, and can be thought of as a large integer in base 26.  White
space\index{white space} inside a {\em compressed-proof} is
optional and is ignored.

It is important to note that the order of the mandatory hypotheses of
the statement being proved must not be changed if the compressed proof
format is used, otherwise the proof will become incorrect.  The reason
for this is that the mandatory hypotheses are not mentioned explicitly
in the compressed proof in order to make the compression more efficient.
If you wish to change the order of mandatory hypotheses, you must first
convert the proof back to normal format using the \texttt{save proof
{\em statement} /normal}\index{\texttt{save proof} command} command.
Later, you can go back to compressed format with \texttt{save proof {\em
statement} /compressed}.

During error checking with the \texttt{verify proof} command, an error
found in a compressed proof may point to a character in {\em
compressed-proof}, which may not be very meaningful to you.  In this
case, try to \texttt{save proof /normal} first, then do the
\texttt{verify proof} again.  In general, it is best to make sure a
proof is correct before saving it in compressed format, because severe
errors are less likely to be recoverable than in normal format.

\subsection{Specifying Unknown Proofs or Subproofs}\label{unknown}

In a proof under development, any step or subproof that is not yet known
may be represented with a single \texttt{?}.  For the purposes of
parsing the proof, the \texttt{?}\ \index{\texttt{]}@\texttt{?}\ inside
proofs} will push a single entry onto the RPN stack just as if it were a
hypothesis.  While developing a proof with the Proof
Assistant\index{Proof Assistant}, a partially developed proof may be
saved with the \texttt{save new{\char`\_}proof}\index{\texttt{save
new{\char`\_}proof} command} command, and \texttt{?}'s will be placed at
the appropriate places.

All \texttt{\$p}\index{\texttt{\$p} statement} statements must have
proofs, even if they are entirely unknown.  Before creating a proof with
the Proof Assistant, you should specify a completely unknown proof as
follows:
\begin{center}
  {\em label} \texttt{\$p} {\em statement} \texttt{\$= ?\ \$.}
\end{center}
\index{\texttt{\$=} keyword}
\index{\texttt{]}@\texttt{?}\ inside proofs}

The \texttt{verify proof}\index{\texttt{verify proof} command} command
will check the known portions of a partial proof for errors, but will
warn you that the statement has not been proved.

Note that partially developed proofs may be saved in compressed format
if desired.  In this case, you will see one or more \texttt{?}'s in the
{\em compressed-proof} part.\index{compressed
proof}\index{proof!compressed}

\section{Axioms vs.\ Definitions}\label{definitions}

The \textit{basic}
Metamath\index{Metamath} language and program
make no distinction\index{axiom vs.\
definition} between axioms\index{axiom} and
definitions.\index{definition} The \texttt{\$a}\index{\texttt{\$a}
statement} statement is used for both.  At first, this may seem
puzzling.  In the minds of many mathematicians, the distinction is
clear, even obvious, and hardly worth discussing.  A definition is
considered to be merely an abbreviation that can be replaced by the
expression for which it stands; although unless one actually does this,
to be precise then one should say that a theorem\index{theorem} is a
consequence of the axioms {\em and} the definitions that are used in the
formulation of the theorem \cite[p.~20]{Behnke}.\index{Behnke, H.}

\subsection{What is a Definition?}

What is a definition?  In its simplest form, a definition introduces a new
symbol and provides an unambiguous rule to transform an expression containing
the new symbol to one without it.  The concept of a ``proper
definition''\index{proper definition}\index{definition!proper} (as opposed to
a creative definition)\index{creative definition}\index{definition!creative}
that is usually agreed upon is (1) the definition should not strengthen the
language and (2) any symbols introduced by the definition should be eliminable
from the language \cite{Nemesszeghy}\index{Nemesszeghy, E. Z.}.  In other
words, they are mere typographical conveniences that do not belong to the
system and are theoretically superfluous.  This may seem obvious, but in fact
the nature of definitions can be subtle, sometimes requiring difficult
metatheorems to establish that they are not creative.

A more conservative stance was taken by logician S.
Le\'{s}niewski.\index{Le\'{s}niewski, S.}
\begin{quote}
Le\'{s}niewski
regards definitions as theses of the system.  In this respect they do
not differ either from the axioms or from theorems, i.e.\ from the
theses added to the system on the basis of the rule of substitution or
the rule of detachment [modus ponens].  Once definitions have been
accepted as theses of the system, it becomes necessary to consider them
as true propositions in the same sense in which axioms are true
\cite{Lejewski}.
\end{quote}\index{Lejewski, Czeslaw}

Let us look at some simple examples of definitions in propositional
calculus.  Consider the definition of logical {\sc or}
(disjunction):\index{disjunction ($\vee$)} ``$P\vee Q$ denotes $\neg P
\rightarrow Q$ (not $P$ implies $Q$).''  It is very easy to recognize a
statement making use of this definition, because it introduces the new
symbol $\vee$ that did not previously exist in the language.  It is easy
to see that no new theorems of the original language will result from
this definition.

Next, consider a definition that eliminates parentheses:  ``$P
\rightarrow Q\rightarrow R$ denotes $P\rightarrow (Q \rightarrow R)$.''
This is more subtle, because no new symbols are introduced.  The reason
this definition is considered proper is that no new symbol sequences
that are valid wffs (well-formed formulas)\index{well-formed formula
(wff)} in the original language will result from the definition, since
``$P \rightarrow Q\rightarrow R$'' is not a wff in the original
language.  Here, we implicitly make use of the fact that there is a
decision procedure that allows us to determine whether or not a symbol
sequence is a wff, and this fact allows us to use symbol sequences that
are not wffs to represent other things (such as wffs) by means of the
definition.  However, to justify the definition as not being creative we
need to prove that ``$P \rightarrow Q\rightarrow R$'' is in fact not a
wff in the original language, and this is more difficult than in the
case where we simply introduce a new symbol.

%Now let's take this reasoning to an extreme.  Propositional calculus is a
%decidable theory,\footnote{This means that a mechanical algorithm exists to
%determine whether or not a wff is a theorem.} so in principle we could make use
%of symbol sequences that are not theorems to represent other things (say, to
%encode actual theorems in a more compact way).  For example, let us extend the
%language by defining a wff ``$P$'' in the extended language as the theorem
%``$P\rightarrow P$''\footnote{This is one of the first theorems proved in the
%Metamath database \texttt{set.mm}.}\index{set
%theory database (\texttt{set.mm})} in the original language whenever ``$P$'' is
%not a theorem in the original language.  In the extended language, any wff
%``$Q$'' thus represents a theorem; to find out what theorem (in the original
%language) ``$Q$'' represents, we determine whether ``$Q$'' is a theorem in the
%original language (before the definition was introduced).  If so, we're done; if
%not, we replace ``$Q$'' by ``$Q\rightarrow Q$'' to eliminate the definition.
%This definition is therefore eliminable, and it does not ``strengthen'' the
%language because any wff that is not a theorem is not in the set of statements
%provable in the original language and thus is available for use by definitions.
%
%Of course, a definition such as this would render practically useless the
%communication of theorems of propositional calculus; but
%this is just a human shortcoming, since we can't always easily discern what is
%and is not a theorem by inspection.  In fact, the extended theory with this
%definition has no more and no less information than the original theory; it just
%expresses certain theorems of the form ``$P\rightarrow P$''
%in a more compact way.
%
%The point here is that what constitutes a proper definition is a matter of
%judgment about whether a symbol sequence can easily be recognized by a human
%as invalid in some sense (for example, not a wff); if so, the symbol sequence
%can be appropriated for use by a definition in order to make the extended
%language more compact.  Metamath\index{Metamath} lacks the ability to make this
%judgment, since as far as Metamath is concerned the definition of a wff, for
%example, is arbitrary.  You define for Metamath how wffs\index{well-formed
%formula (wff)} are constructed according to your own preferred style.  The
%concept of a wff may not even exist in a given formal system\index{formal
%system}.  Metamath treats all definitions as if they were new axioms, and it
%is up to the human mathematician to judge whether the definition is ``proper''
%'\index{proper definition}\index{definition!proper} in some agreed-upon way.

What constitutes a definition\index{definition} versus\index{axiom vs.\
definition} an axiom\index{axiom} is sometimes arbitrary in mathematical
literature.  For example, the connectives $\vee$ ({\sc or}), $\wedge$
({\sc and}), and $\leftrightarrow$ (equivalent to) in propositional
calculus are usually considered defined symbols that can be used as
abbreviations for expressions containing the ``primitive'' connectives
$\rightarrow$ and $\neg$.  This is the way we treat them in the standard
logic and set theory database \texttt{set.mm}\index{set theory database
(\texttt{set.mm})}.  However, the first three connectives can also be
considered ``primitive,'' and axiom systems have been devised that treat
all of them as such.  For example,
\cite[p.~35]{Goodstein}\index{Goodstein, R. L.} presents one with 15
axioms, some of which in fact coincide with what we have chosen to call
definitions in \texttt{set.mm}.  In certain subsets of classical
propositional calculus, such as the intuitionist
fragment\index{intuitionism}, it can be shown that one cannot make do
with just $\rightarrow$ and $\neg$ but must treat additional connectives
as primitive in order for the system to make sense.\footnote{Two nice
systems that make the transition from intuitionistic and other weak
fragments to classical logic just by adding axioms are given in
\cite{Robinsont}\index{Robinson, T. Thacher}.}

\subsection{The Approach to Definitions in \texttt{set.mm}}

In set theory, recursive definitions define a newly introduced symbol in
terms of itself.  The justification of recursive definitions, using
several ``recursion theorems,'' is usually one of the first
sophisticated proofs a student encounters when learning set theory, and
there is a significant amount of implicit metalogic behind a recursive
definition even though the definition itself is typically simple to
state.  It is, however, possible to substitute one kind of complexity
for another.  We can eliminate the need for metalogical justification by
defining the operation directly with an explicit (but complicated)
expression, then deriving the recursive definition directly as a
theorem, using a recursion theorem ``in reverse.''  We do this in
\texttt{set.mm}, as follows.

In \texttt{set.mm} our goal was to introduce almost all definitions in
the form of two expressions connected by either $\leftrightarrow$ or
$=$, where the thing being defined does not appear on the right hand
side.  Quine calls this form ``a genuine or direct definition'' \cite[p.
174]{Quine}\index{Quine, Willard Van Orman}, which makes the definitions
very easy to eliminate and the metalogic\index{metalogic} needed to
justify them as simple as possible.

We achieved this goal in almost all cases in \texttt{set.mm}.
Sometimes this makes the definitions more complex and less
intuitive.  For example, the traditional way to define addition of
natural numbers is to define an operation called {\em
successor}\index{successor} (which means ``plus one'' and is denoted by
``${\rm suc}$''), then define addition recursively\index{recursive
definition} with the two definitions $n + 0 = n$ and $m + {\rm suc}\,n =
{\rm suc} (m + n)$.  Although this definition seems simple and obvious,
the method to eliminate the definition is not obvious:  in the second
part of the definition, addition is defined in terms of itself.  By
eliminating the definition, we don't mean repeatedly applying it to
specific $m$ and $n$ but rather showing the explicit, closed-form
set-theoretical expression that $m + n$ represents, that will work for
any $m$ and $n$ and that does not have a $+$ sign on its right-hand
side.  For a recursive definition like this not to be circular
(creative), there are some hidden, underlying assumptions we must make,
for example that the natural numbers have a certain kind of order.  In
\texttt{set.mm} we chose to start with the direct (though complex and
nonintuitive) definition then derive from it the standard recursive
definition.\footnote{The closed-form definition used in \texttt{set.mm}
for the addition operation on ordinals\index{ordinal
addition}\index{addition!of ordinals} (of which natural numbers are a
subset) is

\setbox\startprefix=\hbox{\tt \ \ df-oadd\ \$a\ }
\setbox\contprefix=\hbox{\tt \ \ \ \ \ \ \ \ \ \ \ \ \ }
\startm
\m{\vdash}\m{+_o}\m{=}\m{\{}\m{\langle}\m{\langle}\m{x}\m{,}\m{y}\m{\rangle}%
\m{,}\m{z}\m{\rangle}\m{|}\m{(}\m{(}\m{x}\m{\in}\m{{\rm On}}\m{\wedge}\m{y}\m{%
\in}\m{{\rm On}}\m{)}\m{\wedge}\m{z}\m{=}\m{(}\m{{\rm rec}}\m{(}\m{\{}\m{%
\langle}\m{w}\m{,}\m{v}\m{\rangle}\m{|}\m{v}\m{=}\m{{\rm suc}}\m{w}\m{\}}\m{,}%
\m{x}\m{)}\m{`}\m{y}\m{)}\m{)}\m{\}}
\endm

\noindent Here, the abstraction class of nested ordered pairs\index{abstraction
class!of nested ordered pairs} is defined by \texttt{df-oprab} in \texttt{set.mm},
and ${\rm rec}$ is a ``recursion operator''\index{recursion operator} with
the definition

\setbox\startprefix=\hbox{\tt \ \ df-rfg\ \$a\ }
\setbox\contprefix=\hbox{\tt \ \ \ \ \ \ \ \ \ \ \ \ }
\startm
\m{\vdash}\m{{\rm rec}}\m{(}\m{F}\m{,}\m{A}\m{)}\m{=}\m{\bigcup}\m{\{}\m{f}%
\m{|}\m{\exists}\m{x}\m{(}\m{x}\m{\in}\m{{\rm On}}\m{\wedge}\m{(}\m{f}\m{{%
\rm Fn}}\m{x}\m{\wedge}\m{\forall}\m{y}\m{(}\m{y}\m{\in}\m{x}\m{\rightarrow}%
\m{(}\m{f}\m{`}\m{y}\m{)}\m{=}\m{(}\m{\{}\m{\langle}\m{g}\m{,}\m{z}\m{\rangle}%
\m{|}\m{(}\m{(}\m{g}\m{=}\m{\varnothing}\m{\wedge}\m{z}\m{=}\m{A}\m{)}\m{\vee}%
\m{(}\m{\lnot}\m{(}\m{g}\m{=}\m{\varnothing}\m{\vee}\m{{\rm Lim}}\m{{\rm dom}}%
\m{g}\m{)}\m{\wedge}\m{z}\m{=}\m{(}\m{F}\m{`}\m{(}\m{g}\m{`}\m{\bigcup}\m{{%
\rm dom}}\m{g}\m{)}\m{)}\m{)}\m{\vee}\m{(}\m{{\rm Lim}}\m{{\rm dom}}\m{g}\m{%
\wedge}\m{z}\m{=}\m{\bigcup}\m{{\rm ran}}\m{g}\m{)}\m{)}\m{\}}\m{`}\m{(}\m{f}%
\m{\restriction}\m{y}\m{)}\m{)}\m{)}\m{)}\m{)}\m{\}}
\endm

\noindent which can be further broken down with definitions shown in
Section~\ref{setdefinitions}.  You may be surprised at the complexity of what
seems like such a simple notion.  From these definitions the simpler, more
intuitive recursive definition is derived as a set of theorems.}\index{natural
number}\index{addition}\index{recursive definition}\index{ordinal addition}
The end result is the same, but we completely eliminate the rather complex
metalogic that justifies the recursive definition.  (For a mathematician,
recursive definitions are more efficient and intuitive than
direct ones once the metalogic has been learned or possibly just accepted as
correct.  However, it was felt that direct definition in \texttt{set.mm}
maximizes rigor by minimizing metalogic.  It can be eliminated effortlessly,
something difficult to do with a recursive definition.)

\subsection{Adding Constraints on Definitions}

The basic Metamath language and the Metamath program do
not have any built-in constraints on definitions, since they are just
\$a statements.

However, nothing prevents a verification system from verifying
additional rules to impose further limitations on definitions.
For example, the \texttt{mmj2}\index{mmj2} program
supports various kinds of
additional information comments (see section \ref{jcomment}).
One of their uses is to optionally verify additional constraints,
including constraints to verify that definitions meet certain
requirements.
These additional checks are required by the
continuous integration (CI)\index{continuous integration (CI)}
checks of the
\texttt{set.mm}\index{set theory database (\texttt{set.mm})}%
\index{Metamath Proof Explorer}
database.
This approach enables us to optionally impose additional requirements
on definitions if we wish, without necessarily imposing those rules on
all databases or requiring all verification systems to implement them.
In addition, this allows us to impose specialized constraints tailored
to one database while not requiring other systems to implement
those specialized constraints.

We impose two constraints on the
\texttt{set.mm}\index{set theory database (\texttt{set.mm})}%
\index{Metamath Proof Explorer} database
via the \texttt{mmj2}\index{mmj2} program that are worth discussing here:
a parse check and a definition soundness check.

% On February 11, 2019 8:32:32 PM EST, saueran@oregonstate.edu wrote:
% The following addition to the end of set.mm is accepted by the mmj2
% parser and definition checker and the metamath verifier(at least it was
% when I checked, you should check it too), and creates a contradiction by
% proving the theorem |- ph.
% ${
% wleftp $a wff ( ( ph ) $.
% wbothp $a wff ( ph ) $.
% df-leftp $a |- ( ( ( ph ) <-> -. ph ) $.
% df-bothp $a |- ( ( ph ) <-> ph ) $.
% anything $p |- ph $=
%   ( wbothp wn wi wleftp df-leftp biimpi df-bothp mpbir mpbi simplim ax-mp)
%   ABZAMACZDZCZMOEZOCQAEZNDZRNAFGSHIOFJMNKLAHJ $.
% $}
%
% This particular problem is countered by enabling, within mmj2,
% SetParser,mmj.verify.LRParser

First,
we enable a parse check in \texttt{mmj2} (through its
\texttt{SetParser} command) that requires that all new definitions
must \textit{not} create an ambiguous parse for a KLR(5) parser.
This prevents some errors such as definitions with imbalanced parentheses.

Second, we run a definition soundness check specific to
\texttt{set.mm} or databases similar to it.
(through the \texttt{definitionCheck} macro).
Some \texttt{\$a} statements (including all ax-* statemnets)
are excluded from these checks, as they will
always fail this simple check,
but they are appropriate for most definitions.
This check imposes a set of additional rules:

\begin{enumerate}

\item New definitions must be introduced using $=$ or $\leftrightarrow$.

\item No \texttt{\$a} statement introduced before this one is allowed to use the
symbol being defined in this definition, and the definition is not
allowed to use itself (except once, in the definiendum).

\item Every variable in the definiens should not be distinct

\item Every dummy variable in the definiendum
are required to be distinct from each other and from variables in
the definiendum.
To determine this, the system will look for a "justification" theorem
in the database, and if it is not there, attempt to briefly prove
$( \varphi \rightarrow \forall x \varphi )$  for each dummy variable x.

\item Every dummy variable should be a set variable,
unless there is a justification theorem available.

\item Every dummy variable must be bound
(if the system cannot determine this a justification theorem must be
provided).

\end{enumerate}

\subsection{Summary of Approach to Definitions}

In short, when being rigorous it turns out that
definitions can be subtle, sometimes requiring difficult
metatheorems to establish that they are not creative.

Instead of building such complications into the Metamath language itself,
the basic Metmath language and program simply treat traditional
axioms and definitions as the same kind of \texttt{\$a} statement.
We have then built various tools to enable people to
verify additional conditions as their creators believe is appropriate
for those specific databases, without complicating the Metamath foundations.

\chapter{The Metamath Program}\label{commands}

This chapter provides a reference manual for the
Metamath program.\index{Metamath!commands}

Current instructions for obtaining and installing the Metamath program
can be found at the \url{http://metamath.org} web site.
For Windows, there is a pre-compiled version called
\texttt{metamath.exe}.  For Unix, Linux, and Mac OS X
(which we will refer to collectively as ``Unix''), the Metamath program
can be compiled from its source code with the command
\begin{verbatim}
gcc *.c -o metamath
\end{verbatim}
using the \texttt{gcc} {\sc c} compiler available on those systems.

In the command syntax descriptions below, fields enclosed in square
brackets [\ ] are optional.  File names may be optionally enclosed in
single or double quotes.  This is useful if the file name contains
spaces or
slashes (\texttt{/}), such as in Unix path names, \index{Unix file
names}\index{file names!Unix} that might be confused with Metamath
command qualifiers.\index{Unix file names}\index{file names!Unix}

\section{Invoking Metamath}

Unix, Linux, and Mac OS X
have a command-line interface called the {\em
bash shell}.  (In Mac OS X, select the Terminal application from
Applications/Utilities.)  To invoke Metamath from the bash shell prompt,
assuming that the Metamath program is in the current directory, type
\begin{verbatim}
bash$ ./metamath
\end{verbatim}

To invoke Metamath from a Windows DOS or Command Prompt, assuming that
the Metamath program is in the current directory (or in a directory
included in the Path system environment variable), type
\begin{verbatim}
C:\metamath>metamath
\end{verbatim}

To use command-line arguments at invocation, the command-line arguments
should be a list of Metamath commands, surrounded by quotes if they
contain spaces.  In Windows, the surrounding quotes must be double (not
single) quotes.  For example, to read the database file \texttt{set.mm},
verify all proofs, and exit the program, type (under Unix)
\begin{verbatim}
bash$ ./metamath 'read set.mm' 'verify proof *' exit
\end{verbatim}
Note that in Unix, any directory path with \texttt{/}'s must be
surrounded by quotes so Metamath will not interpret the \texttt{/} as a
command qualifier.  So if \texttt{set.mm} is in the \texttt{/tmp}
directory, use for the above example
\begin{verbatim}
bash$ ./metamath 'read "/tmp/set.mm"' 'verify proof *' exit
\end{verbatim}

For convenience, if the command-line has one argument and no spaces in
the argument, the command is implicitly assumed to be \texttt{read}.  In
this one special case, \texttt{/}'s are not interpreted as command
qualifiers, so you don't need quotes around a Unix file name.  Thus
\begin{verbatim}
bash$ ./metamath /tmp/set.mm
\end{verbatim}
and
\begin{verbatim}
bash$ ./metamath "read '/tmp/set.mm'"
\end{verbatim}
are equivalent.


\section{Controlling Metamath}

The Metamath program was first developed on a {\sc vax/vms} system, and
some aspects of its command line behavior reflect this heritage.
Hopefully you will find it reasonably user-friendly once you get used to
it.

Each command line is a sequence of English-like words separated by
spaces, as in \texttt{show settings}.  Command words are not case
sensitive, and only as many letters are needed as are necessary to
eliminate ambiguity; for example, \texttt{sh se} would work for the
command \texttt{show settings}.  In some cases arguments such as file
names, statement labels, or symbol names are required; these are
case-sensitive (although file names may not be on some operating
systems).

A command line is entered by typing it in then pressing the {\em return}
({\em enter}) key.  To find out what commands are available, type
\texttt{?} at the \texttt{MM>} prompt.  To find out the choices at any
point in a command, press {\em return} and you will be prompted for
them.  The default choice (the one selected if you just press {\em
return}) is shown in brackets (\texttt{<>}).

You may also type \texttt{?} in place of a command word to force
Metamath to tell you what the choices are.  The \texttt{?} method won't
work, though, if a non-keyword argument such as a file name is expected
at that point, because the program will think that \texttt{?} is the
value of the argument.

Some commands have one or more optional qualifiers which modify the
behavior of the command.  Qualifiers are preceded by a slash
(\texttt{/}), such as in \texttt{read set.mm / verify}.  Spaces are
optional around the \texttt{/}.  If you need to use a space or
slash in a command
argument, as in a Unix file name, put single or double quotes around the
command argument.

The \texttt{open log} command will save everything you see on the
screen and is useful to help you recover should something go wrong in a
proof, or if you want to document a bug.

If a command responds with more than a screenful, you will be
prompted to \texttt{<return> to continue, Q to quit, or S to scroll to
end}.  \texttt{Q} or \texttt{q} (not case-sensitive) will complete the
command internally but will suppress further output until the next
\texttt{MM>} prompt.  \texttt{s} will suppress further pausing until the
next \texttt{MM>} prompt.  After the first screen, you are also
presented with the choice of \texttt{b} to go back a screenful.  Note
that \texttt{b} may also be entered at the \texttt{MM>} prompt
immediately after a command to scroll back through the output of that
command.

A command line enclosed in quotes is executed by your operating system.
See Section~\ref{oscmd}.

{\em Warning:} Pressing {\sc ctrl-c} will abort the Metamath program
unconditionally.  This means any unsaved work will be lost.


\subsection{\texttt{exit} Command}\index{\texttt{exit} command}

Syntax:  \texttt{exit} [\texttt{/force}]

This command exits from Metamath.  If there have been changes to the
source with the \texttt{save proof} or \texttt{save new{\char`\_}proof}
commands, you will be given an opportunity to \texttt{write source} to
permanently save the changes.

In Proof Assistant\index{Proof Assistant} mode, the \texttt{exit} command will
return to the \verb/MM>/ prompt. If there were changes to the proof, you will
be given an opportunity to \texttt{save new{\char`\_}proof}.

The \texttt{quit} command is a synonym for \texttt{exit}.

Optional qualifier:
    \texttt{/force} - Do not prompt if changes were not saved.  This qualifier is
        useful in \texttt{submit} command files (Section~\ref{sbmt})
        to ensure predictable behavior.





\subsection{\texttt{open log} Command}\index{\texttt{open log} command}
Syntax:  \texttt{open log} {\em file-name}

This command will open a log file that will store everything you see on
the screen.  It is useful to help recovery from a mistake in a long Proof
Assistant session, or to document bugs.\index{Metamath!bugs}

The log file can be closed with \texttt{close log}.  It will automatically be
closed upon exiting Metamath.



\subsection{\texttt{close log} Command}\index{\texttt{close log} command}
Syntax:  \texttt{close log}

The \texttt{close log} command closes a log file if one is open.  See
also \texttt{open log}.




\subsection{\texttt{submit} Command}\index{\texttt{submit} command}\label{sbmt}
Syntax:  \texttt{submit} {\em filename}

This command causes further command lines to be taken from the specified
file.  Note that any line beginning with an exclamation point (\texttt{!}) is
treated as a comment (i.e.\ ignored).  Also note that the scrolling
of the screen output is continuous, so you may want to open a log file
(see \texttt{open log}) to record the results that fly by on the screen.
After the lines in the file are exhausted, Metamath returns to its
normal user interface mode.

Currently, the \texttt{submit} command is not recursive.  In other words,
\texttt{submit} commands are not allowed inside a command file.


\subsection{\texttt{erase} Command}\index{\texttt{erase} command}
Syntax:  \texttt{erase}

This command will reset Metamath to its starting state, deleting any
data\-base that was \texttt{read} in.
 If there have been changes to the
source with the \texttt{save proof} or \texttt{save new{\char`\_}proof}
commands, you will be given an opportunity to \texttt{write source} to
permanently save the changes.



\subsection{\texttt{set echo} Command}\index{\texttt{set echo} command}
Syntax:  \texttt{set echo on} or \texttt{set echo off}

The \texttt{set echo on} command will cause command lines to be echoed with any
abbreviations expanded.  While learning the Metamath commands, this
feature will show you the exact command that your abbreviated input
corresponds to.



\subsection{\texttt{set scroll} Command}\index{\texttt{set scroll} command}
Syntax:  \texttt{set scroll prompted} or \texttt{set scroll continuous}

The Metamath command line interface starts off in the \texttt{prompted} mode,
which means that you will be prompted to continue or quit after each
full screen in a long listing.  In \texttt{continuous} mode, long listings will be
scrolled without pausing.

% LaTeX bug? (1) \texttt{\_} puts out different character than
% \texttt{{\char`\_}}
%  = \verb$_$  (2) \texttt{{\char`\_}} puts out garbage in \subsection
%  argument
\subsection{\texttt{set width} Command}\index{\texttt{set
width} command}
Syntax:  \texttt{set width} {\em number}

Metamath assumes the width of your screen is 79 characters (chosen
because the Command Prompt in Windows XP has a wrapping bug at column
80).  If your screen is wider or narrower, this command allows you to
change this default screen width.  A larger width is advantageous for
logging proofs to an output file to be printed on a wide printer.  A
smaller width may be necessary on some terminals; in this case, the
wrapping of the information messages may sometimes seem somewhat
unnatural, however.  In \LaTeX\index{latex@{\LaTeX}!characters per line},
there is normally a maximum of 61 characters per line with typewriter
font.  (The examples in this book were produced with 61 characters per
line.)

\subsection{\texttt{set height} Command}\index{\texttt{set
height} command}
Syntax:  \texttt{set height} {\em number}

Metamath assumes your screen height is 24 lines of characters.  If your
screen is taller or shorter, this command lets you to change the number
of lines at which the display pauses and prompts you to continue.

\subsection{\texttt{beep} Command}\index{\texttt{beep} command}

Syntax:  \texttt{beep}

This command will produce a beep.  By typing it ahead after a
long-running command has started, it will alert you that the command is
finished.  For convenience, \texttt{b} is an abbreviation for
\texttt{beep}.

Note:  If \texttt{b} is typed at the \texttt{MM>} prompt immediately
after the end of a multiple-page display paged with ``\texttt{Press
<return> for more}...'' prompts, then the \texttt{b} will back up to the
previous page rather than perform the \texttt{beep} command.
In that case you must type the unabbreviated \texttt{beep} form
of the command.

\subsection{\texttt{more} Command}\index{\texttt{more} command}

Syntax:  \texttt{more} {\em filename}

This command will display the contents of an {\sc ascii} file on your
screen.  (This command is provided for convenience but is not very
powerful.  See Section~\ref{oscmd} to invoke your operating system's
command to do this, such as the \texttt{more} command in Unix.)

\subsection{Operating System Commands}\index{operating system
command}\label{oscmd}

A line enclosed in single or double quotes will be executed by your
computer's operating system if it has a command line interface.  For
example, on a {\sc vax/vms} system,
\verb/MM> 'dir'/
will print disk directory contents.  Note that this feature will not
work on the Macintosh prior to Mac OS X, which does not have a command
line interface.

For your convenience, the trailing quote is optional.

\subsection{Size Limitations in Metamath}

In general, there are no fixed, predefined limits\index{Metamath!memory
limits} on how many labels, tokens\index{token}, statements, etc.\ that
you may have in a database file.  The Metamath program uses 32-bit
variables (64-bit on 64-bit CPUs) as indices for almost all internal
arrays, which are allocated dynamically as needed.



\section{Reading and Writing Files}

The following commands create new files:  the \texttt{open} commands;
the \texttt{write} commands; the \texttt{/html},
\texttt{/alt{\char`\_}html}, \texttt{/brief{\char`\_}html},
\texttt{/brief{\char`\_}alt{\char`\_}html} qualifiers of \texttt{show
statement}, and \texttt{midi}.  The following commands append to files
previously opened:  the \texttt{/tex} qualifier of \texttt{show proof}
and \texttt{show new{\char`\_}proof}; the \texttt{/tex} and
\texttt{/simple{\char`\_}tex} qualifiers of \texttt{show statement}; the
\texttt{close} commands; and all screen dialog between \texttt{open log}
and \texttt{close log}.

The commands that create new files will not overwrite an existing {\em
filename} but will rename the existing one to {\em
filename}\texttt{{\char`\~}1}.  An existing {\em
filename}\texttt{{\char`\~}1} is renamed {\em
filename}\texttt{{\char`\~}2}, etc.\ up to {\em
filename}\texttt{{\char`\~}9}.  An existing {\em
filename}\texttt{{\char`\~}9} is deleted.  This makes recovery from
mistakes easier but also will clutter up your directory, so occasionally
you may want to clean up (delete) these \texttt{{\char`\~}}$n$ files.


\subsection{\texttt{read} Command}\index{\texttt{read} command}
Syntax:  \texttt{read} {\em file-name} [\texttt{/verify}]

This command will read in a Metamath language source file and any included
files.  Normally it will be the first thing you do when entering Metamath.
Statement syntax is checked, but proof syntax is not checked.
Note that the file name may be enclosed in single or double quotes;
this is useful if the file name contains slashes, as might be the case
under Unix.

If you are getting an ``\texttt{?Expected VERIFY}'' error
when trying to read a Unix file name with slashes, you probably haven't
quoted it.\index{Unix file names}\index{file names!Unix}

If you are prompted for the file name (by pressing {\em return}
 after \texttt{read})
you should {\em not} put quotes around it, even if it is a Unix file name
with slashes.

Optional command qualifier:

    \texttt{/verify} - Verify all proofs as the database is read in.  This
         qualifier will slow down reading in the file.  See \texttt{verify
         proof} for more information on file error-checking.

See also \texttt{erase}.



\subsection{\texttt{write source} Command}\index{\texttt{write source} command}
Syntax:  \texttt{write source} {\em filename}
[\texttt{/rewrap}]
[\texttt{/split}]
[\texttt{/keep\_includes}]
[\texttt{/no\_versioning}]

This command will write the contents of a Metamath\index{database}
database into a file.\index{source file}

Optional command qualifiers:

\texttt{/rewrap} -
Reformats statements and comments according to the
convention used in the set.mm database.
It unwraps the
lines in the comment before each \texttt{\$a} and \texttt{\$p} statement,
then it
rewraps the line.  You should compare the output to the original
to make sure that the desired effect results; if not, go back to
the original source.  The wrapped line length honors the
\texttt{set width}
parameter currently in effect.  Note:  Text
enclosed in \texttt{<HTML>}...\texttt{</HTML>} tags is not modified by the
\texttt{/rewrap} qualifier.
Proofs themselves are not reformatted;
use \texttt{save proof * / compressed} to do that.
An isolated tilde (\~{}) is always kept on the same line as the following
symbol, so you can find all comment references to a symbol by
searching for \~{} followed by a space and that symbol
(this is useful for finding cross-references).
Incidentally, \texttt{save proof} also honors the \texttt{set width}
parameter currently in effect.

\texttt{/split} - Files included in the source using the expression
\$[ \textit{inclfile} \$] will be
written into separate files instead of being included in a single output
file.  The name of each separately written included file will be the
\textit{inclfile} argument of its inclusion command.

\texttt{/keep\_includes} - If a source file has includes but is written as a
single file by omitting \texttt{/split}, by default the included files will
be deleted (actually just renamed with a \char`\~1 suffix unless
\texttt{/no\_versioning} is specified) to prevent the possibly confusing
source duplication in both the output file and the included file.
The \texttt{/keep\_includes} qualifier will prevent this deletion.

\texttt{/no\_versioning} - Backup files suffixed with \char`\~1 are not created.


\section{Showing Status and Statements}



\subsection{\texttt{show settings} Command}\index{\texttt{show settings} command}
Syntax:  \texttt{show settings}

This command shows the state of various parameters.

\subsection{\texttt{show memory} Command}\index{\texttt{show memory} command}
Syntax:  \texttt{show memory}

This command shows the available memory left.  It is not meaningful
on most modern operating systems,
which have virtual memory.\index{Metamath!memory usage}


\subsection{\texttt{show labels} Command}\index{\texttt{show labels} command}
Syntax:  \texttt{show labels} {\em label-match} [\texttt{/all}]
   [\texttt{/linear}]

This command shows the labels of \texttt{\$a} and \texttt{\$p}
statements that match {\em label-match}.  A \verb$*$ in {label-match}
matches zero or more characters.  For example, \verb$*abc*def$ will match all
labels containing \verb$abc$ and ending with \verb$def$.

Optional command qualifiers:

   \texttt{/all} - Include matches for \texttt{\$e} and \texttt{\$f}
   statement labels.

   \texttt{/linear} - Display only one label per line.  This can be useful for
       building scripts in conjunction with the utilities under the
       \texttt{tools}\index{\texttt{tools} command} command.



\subsection{\texttt{show statement} Command}\index{\texttt{show statement} command}
Syntax:  \texttt{show statement} {\em label-match} [{\em qualifiers} (see below)]

This command provides information about a statement.  Only statements
that have labels (\texttt{\$f}\index{\texttt{\$f} statement},
\texttt{\$e}\index{\texttt{\$e} statement},
\texttt{\$a}\index{\texttt{\$a} statement}, and
\texttt{\$p}\index{\texttt{\$p} statement}) may be specified.
If {\em label-match}
contains wildcard (\verb$*$) characters, all matching statements will be
displayed in the order they occur in the database.

Optional qualifiers (only one qualifier at a time is allowed):

    \texttt{/comment} - This qualifier includes the comment that immediately
       precedes the statement.

    \texttt{/full} - Show complete information about each statement,
       and show all
       statements matching {\em label} (including \texttt{\$e}
       and \texttt{\$f} statements).

    \texttt{/tex} - This qualifier will write the statement information to the
       \LaTeX\ file previously opened with \texttt{open tex}.  See
       Section~\ref{texout}.

    \texttt{/simple{\char`\_}tex} - The same as \texttt{/tex}, except that
       \LaTeX\ macros are not used for formatting equations, allowing easier
       manual edits of the output for slide presentations, etc.

    \texttt{/html}\index{html generation@{\sc html} generation},
       \texttt{/alt{\char`\_}html}, \texttt{/brief{\char`\_}html},
       \texttt{/brief{\char`\_}alt{\char`\_}html} -
       These qualifiers invoke a special mode of
       \texttt{show statement} that
       creates a web page for the statement.  They may not be used with
       any other qualifier.  See Section~\ref{htmlout} or
       \texttt{help html} in the program.


\subsection{\texttt{search} Command}\index{\texttt{search} command}
Syntax:  search {\em label-match}
\texttt{"}{\em symbol-match}{\tt}" [\texttt{/all}] [\texttt{/comments}]
[\texttt{/join}]

This command searches all \texttt{\$a} and \texttt{\$p} statements
matching {\em label-match} for occurrences of {\em symbol-match}.  A
\verb@*@ in {\em label-match} matches any label character.  A \verb@$*@
in {\em symbol-match} matches any sequence of symbols.  The symbols in
{\em symbol-match} must be separated by white space.  The quotes
surrounding {\em symbol-match} may be single or double quotes.  For
example, \texttt{search b}\verb@* "-> $* ch"@ will list all statements
whose labels begin with \texttt{b} and contain the symbols \verb@->@ and
\texttt{ch} surrounding any symbol sequence (including no symbol
sequence).  The wildcards \texttt{?} and \texttt{\$?} are also available
to match individual characters in labels and symbols respectively; see
\texttt{help search} in the Metamath program for details on their usage.

Optional command qualifiers:

    \texttt{/all} - Also search \texttt{\$e} and \texttt{\$f} statements.

    \texttt{/comments} - Search the comment that immediately precedes each
        label-matched statement for {\em symbol-match}.  In this case
        {\em symbol-match} is an arbitrary, non-case-sensitive character
        string.  Quotes around {\em symbol-match} are optional if there
        is no ambiguity.

    \texttt{/join} - In the case of a \texttt{\$a} or \texttt{\$p} statement,
	prepend its \texttt{\$e}
	hypotheses for searching. The
	\texttt{/join} qualifier has no effect in \texttt{/comments} mode.

\section{Displaying and Verifying Proofs}


\subsection{\texttt{show proof} Command}\index{\texttt{show proof} command}
Syntax:  \texttt{show proof} {\em label-match} [{\em qualifiers} (see below)]

This command displays the proof of the specified
\texttt{\$p}\index{\texttt{\$p} statement} statement in various formats.
The {\em label-match} may contain wildcard (\verb@$*@) characters to match
multiple statements.  Without any qualifiers, only the logical steps
will be shown (i.e.\ syntax construction steps will be omitted), in an
indented format.

Most of the time, you will use
    \texttt{show proof} {\em label}
to see just the proof steps corresponding to logical inferences.

Optional command qualifiers:

    \texttt{/essential} - The proof tree is trimmed of all
        \texttt{\$f}\index{\texttt{\$f} statement} hypotheses before
        being displayed.  (This is the default, and it is redundant to
        specify it.)

    \texttt{/all} - the proof tree is not trimmed of all \texttt{\$f} hypotheses before
        being displayed.  \texttt{/essential} and \texttt{/all} are mutually exclusive.

    \texttt{/from{\char`\_}step} {\em step} - The display starts at the specified
        step.  If
        this qualifier is omitted, the display starts at the first step.

    \texttt{/to{\char`\_}step} {\em step} - The display ends at the specified
        step.  If this
        qualifier is omitted, the display ends at the last step.

    \texttt{/tree{\char`\_}depth} {\em number} - Only
         steps at less than the specified proof
        tree depth are displayed.  Sometimes useful for obtaining an overview of
        the proof.

    \texttt{/reverse} - The steps are displayed in reverse order.

    \texttt{/renumber} - When used with \texttt{/essential}, the steps are renumbered
        to correspond only to the essential steps.

    \texttt{/tex} - The proof is converted to \LaTeX\ and\index{latex@{\LaTeX}}
        stored in the file opened
        with \texttt{open tex}.  See Section~\ref{texout} or
        \texttt{help tex} in the program.

    \texttt{/lemmon} - The proof is displayed in a non-indented format known
        as Lemmon style, with explicit previous step number references.
        If this qualifier is omitted, steps are indented in a tree format.

    \texttt{/start{\char`\_}column} {\em number} - Overrides the default column
        (16)
        at which the formula display starts in a Lemmon-style display.  May be
        used only in conjunction with \texttt{/lemmon}.

    \texttt{/normal} - The proof is displayed in normal format suitable for
        inclusion in a Metamath source file.  May not be used with any other
        qualifier.

    \texttt{/compressed} - The proof is displayed in compressed format
        suitable for inclusion in a Metamath source file.  May not be used with
        any other qualifier.

    \texttt{/statement{\char`\_}summary} - Summarizes all statements (like a
        brief \texttt{show statement})
        used by the proof.  It may not be used with any other qualifier
        except \texttt{/essential}.

    \texttt{/detailed{\char`\_}step} {\em step} - Shows the details of what is
        happening at
        a specific proof step.  May not be used with any other qualifier.
        The {\em step} is the step number shown when displaying a
        proof without the \texttt{/renumber} qualifier.


\subsection{\texttt{show usage} Command}\index{\texttt{show usage} command}
Syntax:  \texttt{show usage} {\em label-match} [\texttt{/recursive}]

This command lists the statements whose proofs make direct reference to
the statement specified.

Optional command qualifier:

    \texttt{/recursive} - Also include statements whose proofs ultimately
        depend on the statement specified.



\subsection{\texttt{show trace\_back} Command}\index{\texttt{show
       trace{\char`\_}back} command}
Syntax:  \texttt{show trace{\char`\_}back} {\em label-match} [\texttt{/essential}] [\texttt{/axioms}]
    [\texttt{/tree}] [\texttt{/depth} {\em number}]

This command lists all statements that the proof of the \texttt{\$p}
statement(s) specified by {\em label-match} depends on.

Optional command qualifiers:

    \texttt{/essential} - Restrict the trace-back to \texttt{\$e}
        \index{\texttt{\$e} statement} hypotheses of proof trees.

    \texttt{/axioms} - List only the axioms that the proof ultimately depends on.

    \texttt{/tree} - Display the trace-back in an indented tree format.

    \texttt{/depth} {\em number} - Restrict the \texttt{/tree} trace-back to the
        specified indentation depth.

    \texttt{/count{\char`\_}steps} - Count the number of steps the proof has
       all the way back to axioms.  If \texttt{/essential} is specified,
       expansions of variable-type hypotheses (syntax constructions) are not counted.

\subsection{\texttt{verify proof} Command}\index{\texttt{verify proof} command}
Syntax:  \texttt{verify proof} {\em label-match} [\texttt{/syntax{\char`\_}only}]

This command verifies the proofs of the specified statements.  {\em
label-match} may contain wild card characters (\texttt{*}) to verify
more than one proof; for example \verb/*abc*def/ will match all labels
containing \texttt{abc} and ending with \texttt{def}.
The command \texttt{verify proof *} will verify all proofs in the database.

Optional command qualifier:

    \texttt{/syntax{\char`\_}only} - This qualifier will perform a check of syntax
        and RPN
        stack violations only.  It will not verify that the proof is
        correct.  This qualifier is useful for quickly determining which
        proofs are incomplete (i.e.\ are under development and have \texttt{?}'s
        in them).

{\em Note:} \texttt{read}, followed by \texttt{verify proof *}, will ensure
 the database is free
from errors in the Metamath language but will not check the markup notation
in comments.
You can also check the markup notation by running \texttt{verify markup *}
as discussed in Section~\ref{verifymarkup}; see also the discussion
on generating {\sc HTML} in Section~\ref{htmlout}.

\subsection{\texttt{verify markup} Command}\index{\texttt{verify markup} command}\label{verifymarkup}
Syntax:  \texttt{verify markup} {\em label-match}
[\texttt{/date{\char`\_}skip}]
[\texttt{/top{\char`\_}date{\char`\_}skip}]
[\texttt{/file{\char`\_}skip}]
[\texttt{/verbose}]

This command checks comment markup and other informal conventions we have
adopted.  It error-checks the latexdef, htmldef, and althtmldef statements
in the \texttt{\$t} statement of a Metamath source file.\index{error checking}
It error-checks any \texttt{`}...\texttt{`},
\texttt{\char`\~}~\textit{label},
and bibliographic markups in statement descriptions.
It checks that
\texttt{\$p} and \texttt{\$a} statements
have the same content when their labels start with
``ax'' and ``ax-'' respectively but are otherwise identical, for example
ax4 and ax-4.
It also verifies the date consistency of ``(Contributed by...),''
``(Revised by...),'' and ``(Proof shortened by...)'' tags in the comment
above each \texttt{\$a} and \texttt{\$p} statement.

Optional command qualifiers:

    \texttt{/date{\char`\_}skip} - This qualifier will
        skip date consistency checking,
        which is usually not required for databases other than
	\texttt{set.mm}.

    \texttt{/top{\char`\_}date{\char`\_}skip} - This qualifier will check date consistency except
        that the version date at the top of the database file will not
        be checked.  Only one of
        \texttt{/date{\char`\_}skip} and
        \texttt{/top{\char`\_}date{\char`\_}skip} may be
        specified.

    \texttt{/file{\char`\_}skip} - This qualifier will skip checks that require
        external files to be present, such as checking GIF existence and
        bibliographic links to mmset.html or equivalent.  It is useful
        for doing a quick check from a directory without these files.

    \texttt{/verbose} - Provides more information.  Currently it provides a list
        of axXXX vs. ax-XXX matches.

\subsection{\texttt{save proof} Command}\index{\texttt{save proof} command}
Syntax:  \texttt{save proof} {\em label-match} [\texttt{/normal}]
   [\texttt{/compressed}]

The \texttt{save proof} command will reformat a proof in one of two formats and
replace the existing proof in the source buffer\index{source
buffer}.  It is useful for
converting between proof formats.  Note that a proof will not be
permanently saved until a \texttt{write source} command is issued.

Optional command qualifiers:

    \texttt{/normal} - The proof is saved in the normal format (i.e., as a
        sequence
        of labels, which is the defined format of the basic Metamath
        language).\index{basic language}  This is the default format that
        is used if a qualifier
        is omitted.

    \texttt{/compressed} - The proof is saved in the compressed format which
        reduces storage requirements for a database.
        See Appendix~\ref{compressed}.




\section{Creating Proofs}\label{pfcommands}\index{Proof Assistant}

Before using the Proof Assistant, you must add a \texttt{\$p} to your
source file (using a text editor) containing the statement you want to
prove.  Its proof should consist of a single \texttt{?}, meaning
``unknown step.''  Example:
\begin{verbatim}
equid $p x = x $= ? $.
\end{verbatim}

To enter the Proof assistant, type \texttt{prove} {\em label}, e.g.
\texttt{prove equid}.  Metamath will respond with the \texttt{MM-PA>}
prompt.

Proofs are created working backwards from the statement being proved,
primarily using a series of \texttt{assign} commands.  A proof is
complete when all steps are assigned to statements and all steps are
unified and completely known.  During the creation of a proof, Metamath
will allow only operations that are legal based on what is known up to
that point.  For example, it will not allow an \texttt{assign} of a
statement that cannot be unified with the unknown proof step being
assigned.

{\em Important:}
The Proof Assistant is
{\em not} a tool to help you discover proofs.  It is just a tool to help
you add them to the database.  For a tutorial read
Section~\ref{frstprf}.
To practice using the Proof Assistant, you may
want to \texttt{prove} an existing theorem, then delete all steps with
\texttt{delete all}, then re-create it with the Proof Assistant while
looking at its proof display (before deletion).
You might want to figure out your first few proofs completely
and write them down by hand, before using the Proof Assistant, though
not everyone finds that effective.

{\em Important:}
The \texttt{undo} command if very helpful when entering a proof, because
it allows you to undo a previously-entered step.
In addition, we suggest that you
keep track of your work with a log file (\texttt{open
log}) and save it frequently (\texttt{save new{\char`\_}proof},
\texttt{write source}).
You can use \texttt{delete} to reverse an \texttt{assign}.
You can also do \texttt{delete floating{\char`\_}hypotheses}, then
\texttt{initialize all}, then \texttt{unify all /interactive} to
reinitialize bad unifications made accidentally or by bad
\texttt{assign}s.  You cannot reverse a \texttt{delete} except by
a relevant \texttt{undo} or using
\texttt{exit /force} then reentering the Proof Assistant to recover from
the last \texttt{save new{\char`\_}proof}.

The following commands are available in the Proof Assistant (at the
\texttt{MM-PA>} prompt) to help you create your proof.  See the
individual commands for more detail.

\begin{itemize}
\item[]
    \texttt{show new{\char`\_}proof} [\texttt{/all},...] - Displays the
        proof in progress.  You will use this command a lot; see \texttt{help
        show new{\char`\_}proof} to become familiar with its qualifiers.  The
        qualifiers \texttt{/unknown} and \texttt{/not{\char`\_}unified} are
        useful for seeing the work remaining to be done.  The combination
        \texttt{/all/unknown} is useful for identifying dummy variables that must be
        assigned, or attempts to use illegal syntax, when \texttt{improve all}
        is unable to complete the syntax constructions.  Unknown variables are
        shown as \texttt{\$1}, \texttt{\$2},...
\item[]
    \texttt{assign} {\em step} {\em label} - Assigns an unknown {\em step}
        number with the statement
        specified by {\em label}.
\item[]
    \texttt{let variable} {\em variable}
        \texttt{= "}{\em symbol sequence}\texttt{"}
          - Forces a symbol
        sequence to replace an unknown variable (such as \texttt{\$1}) in a proof.
        It is useful
        for helping difficult unifications, and it is necessary when you have
        dummy variables that eventually must be assigned a name.
\item[]
    \texttt{let step} {\em step} \texttt{= "}{\em symbol sequence}\texttt{"} -
          Forces a symbol sequence
        to replace the contents of a proof step, provided it can be
        unified with the existing step contents.  (I rarely use this.)
\item[]
    \texttt{unify step} {\em step} (or \texttt{unify all}) - Unifies
        the source and target of
        a step.  If you specify a specific step, you will be prompted
        to select among the unifications that are possible.  If you
        specify \texttt{all}, all steps with unique unifications, but only
        those steps, will be
        unified.  \texttt{unify all /interactive} goes through all non-unified
        steps.
\item[]
    \texttt{initialize} {\em step} (or \texttt{all}) - De-unifies the target and source of
        a step (or all steps), as well as the hypotheses of the source,
        and makes all variables in the source unknown.  Useful to recover from
        an \texttt{assign} or \texttt{let} mistake that
        resulted in incorrect unifications.
\item[]
    \texttt{delete} {\em step} (or \texttt{all} or \texttt{floating{\char`\_}hypotheses}) -
        Deletes the specified
        step(s).  \texttt{delete floating{\char`\_}hypotheses}, then \texttt{initialize all}, then
        \texttt{unify all /interactive} is useful for recovering from mistakes
        where incorrect unifications assigned wrong math symbol strings to
        variables.
\item[]
    \texttt{improve} {\em step} (or \texttt{all}) -
      Automatically creates a proof for steps (with no unknown variables)
      whose proof requires no statements with \texttt{\$e} hypotheses.  Useful
      for filling in proofs of \texttt{\$f} hypotheses.  The \texttt{/depth}
      qualifier will also try statements whose \texttt{\$e} hypotheses contain
      no new variables.  {\em Warning:} Save your work (with \texttt{save
      new{\char`\_}proof} then \texttt{write source}) before using
      \texttt{/depth = 2} or greater, since the search time grows
      exponentially and may never terminate in a reasonable time, and you
      cannot interrupt the search.  I have found that it is rare for
      \texttt{/depth = 3} or greater to be useful.
 \item[]
    \texttt{save new{\char`\_}proof} - Saves the proof in progress in the program's
        internal database buffer.  To save it permanently into the database file,
        use \texttt{write source} after
        \texttt{save new{\char`\_}proof}.  To revert to the last
        \texttt{save new{\char`\_}proof},
        \texttt{exit /force} from the Proof Assistant then re-enter the Proof
        Assistant.
 \item[]
    \texttt{match step} {\em step} (or \texttt{match all}) - Shows what
        statements are
        possibilities for the \texttt{assign} statement. (This command
        is not very
        useful in its present form and hopefully will be improved
        eventually.  In the meantime, use the \texttt{search} statement for
        candidates matching specific math token combinations.)
 \item[]
 \texttt{minimize{\char`\_}with}\index{\texttt{minimize{\char`\_}with} command}
% 3/10/07 Note: line-breaking the above results in duplicate index entries
     - After a proof is complete, this command will attempt
        to match other database theorems to the proof to see if the proof
        size can be reduced as a result.  See \texttt{help
        minimize{\char`\_}with} in the
        Metamath program for its usage.
 \item[]
 \texttt{undo}\index{\texttt{undo} command}
    - Undo the effect of a proof-changing command (all but the
      \texttt{show} and \texttt{save} commands above).
 \item[]
 \texttt{redo}\index{\texttt{redo} command}
    - Reverse the previous \texttt{undo}.
\end{itemize}

The following commands set parameters that may be relevant to your proof.
Consult the individual \texttt{help set}... commands.
\begin{itemize}
   \item[] \texttt{set unification{\char`\_}timeout}
 \item[]
    \texttt{set search{\char`\_}limit}
  \item[]
    \texttt{set empty{\char`\_}substitution} - note that default is \texttt{off}
\end{itemize}

Type \texttt{exit} to exit the \texttt{MM-PA>}
 prompt and get back to the \texttt{MM>} prompt.
Another \texttt{exit} will then get you out of Metamath.



\subsection{\texttt{prove} Command}\index{\texttt{prove} command}
Syntax:  \texttt{prove} {\em label}

This command will enter the Proof Assistant\index{Proof Assistant}, which will
allow you to create or edit the proof of the specified statement.
The command-line prompt will change from \texttt{MM>} to \texttt{MM-PA>}.

Note:  In the present version of
Metamath\index{Metamath!limitations of version 0.07.30}, the Proof
Assistant does not verify that \texttt{\$d}\index{\texttt{\$d}
statement} restrictions are met as a proof is being built.  After you
have completed a proof, you should type \texttt{save new{\char`\_}proof}
followed by \texttt{verify proof} {\em label} (where {\em label} is the
statement you are proving with the \texttt{prove} command) to verify the
\texttt{\$d} restrictions.

See also: \texttt{exit}

\subsection{\texttt{set unification\_timeout} Command}\index{\texttt{set
unification{\char`\_}timeout} command}
Syntax:  \texttt{set unification{\char`\_}timeout} {\em number}

(This command is available outside the Proof Assistant but affects the
Proof Assistant\index{Proof Assistant} only.)

Sometimes the Proof Assistant will inform you that a unification
time-out occurred.  This may happen when you try to \texttt{unify}
formulas with many temporary variables\index{temporary variable}
(\texttt{\$1}, \texttt{\$2}, etc.), since the time to compute all possible
unifications may grow exponentially with the number of variables.  If
you want Metamath to try harder (and you're willing to wait longer) you
may increase this parameter.  \texttt{show settings} will show you the
current value.

\subsection{\texttt{set empty\_substitution} Command}\index{\texttt{set
empty{\char`\_}substitution} command}
% These long names can't break well in narrow mode, and even "sloppy"
% is not enough. Work around this by not demanding justification.
\begin{flushleft}
Syntax:  \texttt{set empty{\char`\_}substitution on} or \texttt{set
empty{\char`\_}substitution off}
\end{flushleft}

(This command is available outside the Proof Assistant but affects the
Proof Assistant\index{Proof Assistant} only.)

The Metamath language allows variables to be
substituted\index{substitution!variable}\index{variable substitution}
with empty symbol sequences\index{empty substitution}.  However, in many
formal systems\index{formal system} this will never happen in a valid
proof.  Allowing for this possibility increases the likelihood of
ambiguous unifications\index{ambiguous
unification}\index{unification!ambiguous} during proof creation.
The default is that
empty substitutions are not allowed; for formal systems requiring them,
you must \texttt{set empty{\char`\_}substitution on}.
(An example where this must be \texttt{on}
would be a system that implements a Deduction Rule and in
which deductions from empty assumption lists would be permissible.  The
MIU-system\index{MIU-system} described in Appendix~\ref{MIU} is another
example.)
Note that empty substitutions are
always permissible in proof verification (VERIFY PROOF...) outside the
Proof Assistant.  (See the MIU system in the Metamath book for an example
of a system needing empty substitutions; another example would be a
system that implements a Deduction Rule and in which deductions from
empty assumption lists would be permissible.)

It is better to leave this \texttt{off} when working with \texttt{set.mm}.
Note that this command does not affect the way proofs are verified with
the \texttt{verify proof} command.  Outside of the Proof Assistant,
substitution of empty sequences for math symbols is always allowed.

\subsection{\texttt{set search\_limit} Command}\index{\texttt{set
search{\char`\_}limit} command} Syntax:  \texttt{set search{\char`\_}limit} {\em
number}

(This command is available outside the Proof Assistant but affects the
Proof Assistant\index{Proof Assistant} only.)

This command sets a parameter that determines when the \texttt{improve} command
in Proof Assistant mode gives up.  If you want \texttt{improve} to search harder,
you may increase it.  The \texttt{show settings} command tells you its current
value.


\subsection{\texttt{show new\_proof} Command}\index{\texttt{show
new{\char`\_}proof} command}
Syntax:  \texttt{show new{\char`\_}proof} [{\em
qualifiers} (see below)]

This command (available only in Proof Assistant mode) displays the proof
in progress.  It is identical to the \texttt{show proof} command, except that
there is no statement argument (since it is the statement being proved) and
the following qualifiers are not available:

    \texttt{/statement{\char`\_}summary}

    \texttt{/detailed{\char`\_}step}

Also, the following additional qualifiers are available:

    \texttt{/unknown} - Shows only steps that have no statement assigned.

    \texttt{/not{\char`\_}unified} - Shows only steps that have not been unified.

Note that \texttt{/essential}, \texttt{/depth}, \texttt{/unknown}, and
\texttt{/not{\char`\_}unified} may be
used in any combination; each of them effectively filters out additional
steps from the proof display.

See also:  \texttt{show proof}






\subsection{\texttt{assign} Command}\index{\texttt{assign} command}
Syntax:   \texttt{assign} {\em step} {\em label} [\texttt{/no{\char`\_}unify}]

   and:   \texttt{assign first} {\em label}

   and:   \texttt{assign last} {\em label}


This command, available in the Proof Assistant only, assigns an unknown
step (one with \texttt{?} in the \texttt{show new{\char`\_}proof}
listing) with the statement specified by {\em label}.  The assignment
will not be allowed if the statement cannot be unified with the step.

If \texttt{last} is specified instead of {\em step} number, the last
step that is shown by \texttt{show new{\char`\_}proof /unknown} will be
used.  This can be useful for building a proof with a command file (see
\texttt{help submit}).  It also makes building proofs faster when you know
the assignment for the last step.

If \texttt{first} is specified instead of {\em step} number, the first
step that is shown by \texttt{show new{\char`\_}proof /unknown} will be
used.

If {\em step} is zero or negative, the -{\em step}th from last unknown
step, as shown by \texttt{show new{\char`\_}proof /unknown}, will be
used.  \texttt{assign -1} {\em label} will assign the penultimate
unknown step, \texttt{assign -2} {\em label} the antepenultimate, and
\texttt{assign 0} {\em label} is the same as \texttt{assign last} {\em
label}.

Optional command qualifier:

    \texttt{/no{\char`\_}unify} - do not prompt user to select a unification if there is
        more than one possibility.  This is useful for noninteractive
        command files.  Later, the user can \texttt{unify all /interactive}.
        (The assignment will still be automatically unified if there is only
        one possibility and will be refused if unification is not possible.)



\subsection{\texttt{match} Command}\index{\texttt{match} command}
Syntax:  \texttt{match step} {\em step} [\texttt{/max{\char`\_}essential{\char`\_}hyp}
{\em number}]

    and:  \texttt{match all} [\texttt{/essential}]
          [\texttt{/max{\char`\_}essential{\char`\_}hyp} {\em number}]

This command, available in the Proof Assistant only, shows what
statements can be unified with the specified step(s).  {\em Note:} In
its current form, this command is not very useful because of the large
number of matches it reports.  I rarely use it.
It may be enhanced in the future.

Optional command qualifiers:

    \texttt{/max{\char`\_}essential{\char`\_}hyp} {\em number} - filters out
        of the list any statements
        with more than the specified number of
        \texttt{\$e}\index{\texttt{\$e} statement} hypotheses

    \texttt{/essential{\char`\_}only} - in the \texttt{match all} statement, only
        the steps that
        would be listed in the \texttt{show new{\char`\_}proof /essential} display are
        matched.



\subsection{\texttt{let} Command}\index{\texttt{let} command}
Syntax: \texttt{let variable} {\em variable} = \verb/"/{\em symbol-sequence}\verb/"/

  and: \texttt{let step} {\em step} = \verb/"/{\em symbol-sequence}\verb/"/

These commands, available in the Proof Assistant\index{Proof Assistant}
only, assign a temporary variable\index{temporary variable} or unknown
step with a specific symbol sequence.  They are useful in the middle of
creating a proof, when you know what should be in the proof step but the
unification algorithm doesn't yet have enough information to completely
specify the temporary variables.  A ``temporary variable'' is one that
has the form \texttt{\$}{\em nn} in the proof display, such as
\texttt{\$1}, \texttt{\$2}, etc.  The {\em symbol-sequence} may contain
other unknown variables if desired.  Examples:

    \verb/let variable $32 = "A = B"/

    \verb/let variable $32 = "A = $35"/

    \verb/let step 10 = '|- x = x'/

    \verb/let step -2 = "|- ( $7 = ph )"/

Any symbol sequence will be accepted for the \texttt{let variable}
command.  Only those symbol sequences that can be unified with the step
will be accepted for \texttt{let step}.

The \texttt{let} commands ``zap'' the proof with information that can
only be verified when the proof is built up further.  If you make an
error, the command sequence \texttt{delete
floating{\char`\_}hypotheses}, \texttt{initialize all}, and
\texttt{unify all /interactive} will undo a bad \texttt{let} assignment.

If {\em step} is zero or negative, the -{\em step}th from last unknown
step, as shown by \texttt{show new{\char`\_}proof /unknown}, will be
used.  The command \texttt{let step 0} = \verb/"/{\em
symbol-sequence}\verb/"/ will use the last unknown step, \texttt{let
step -1} = \verb/"/{\em symbol-sequence}\verb/"/ the penultimate, etc.
If {\em step} is positive, \texttt{let step} may be used to assign known
(in the sense of having previously been assigned a label with
\texttt{assign}) as well as unknown steps.

Either single or double quotes can surround the {\em symbol-sequence} as
long as they are different from any quotes inside a {\em
symbol-sequence}.  If {\em symbol-sequence} contains both kinds of
quotes, see the instructions at the end of \texttt{help let} in the
Metamath program.


\subsection{\texttt{unify} Command}\index{\texttt{unify} command}
Syntax:  \texttt{unify step} {\em step}

      and:   \texttt{unify all} [\texttt{/interactive}]

These commands, available in the Proof Assistant only, unify the source
and target of the specified step(s). If you specify a specific step, you
will be prompted to select among the unifications that are possible.  If
you specify \texttt{all}, only those steps with unique unifications will be
unified.

Optional command qualifier for \texttt{unify all}:

    \texttt{/interactive} - You will be prompted to select among the
        unifications
        that are possible for any steps that do not have unique
        unifications.  (Otherwise \texttt{unify all} will bypass these.)

See also \texttt{set unification{\char`\_}timeout}.  The default is
100000, but increasing it to 1000000 can help difficult cases.  Manually
assigning some or all of the unknown variables with the \texttt{let
variable} command also helps difficult cases.



\subsection{\texttt{initialize} Command}\index{\texttt{initialize} command}
Syntax:  \texttt{initialize step} {\em step}

    and: \texttt{initialize all}

These commands, available in the Proof Assistant\index{Proof Assistant}
only, ``de-unify'' the target and source of a step (or all steps), as
well as the hypotheses of the source, and makes all variables in the
source and the source's hypotheses unknown.  This command is useful to
help recover from incorrect unifications that resulted from an incorrect
\texttt{assign}, \texttt{let}, or unification choice.  Part or all of
the command sequence \texttt{delete floating{\char`\_}hypotheses},
\texttt{initialize all}, and \texttt{unify all /interactive} will recover
from incorrect unifications.

See also:  \texttt{unify} and \texttt{delete}



\subsection{\texttt{delete} Command}\index{\texttt{delete} command}
Syntax:  \texttt{delete step} {\em step}

   and:      \texttt{delete all} -- {\em Warning: dangerous!}

   and:      \texttt{delete floating{\char`\_}hypotheses}

These commands are available in the Proof Assistant only.  The
\texttt{delete step} command deletes the proof tree section that
branches off of the specified step and makes the step become unknown.
\texttt{delete all} is equivalent to \texttt{delete step} {\em step}
where {\em step} is the last step in the proof (i.e.\ the beginning of
the proof tree).

In most cases the \texttt{undo} command is the best way to undo
a previous step.
An alternative is to salvage your last \texttt{save
new{\char`\_}proof} by exiting and reentering the Proof Assistant.
For this to work, keep a log file open to record your work
and to do \texttt{save new{\char`\_}proof} frequently, especially before
\texttt{delete}.

\texttt{delete floating{\char`\_}hypotheses} will delete all sections of
the proof that branch off of \texttt{\$f}\index{\texttt{\$f} statement}
statements.  It is sometimes useful to do this before an
\texttt{initialize} command to recover from an error.  Note that once a
proof step with a \texttt{\$f} hypothesis as the target is completely
known, the \texttt{improve} command can usually fill in the proof for
that step.  Unlike the deletion of logical steps, \texttt{delete
floating{\char`\_}hypotheses} is a relatively safe command that is
usually easy to recover from.



\subsection{\texttt{improve} Command}\index{\texttt{improve} command}
\label{improve}
Syntax:  \texttt{improve} {\em step} [\texttt{/depth} {\em number}]
                                               [\texttt{/no{\char`\_}distinct}]

   and:   \texttt{improve first} [\texttt{/depth} {\em number}]
                                              [\texttt{/no{\char`\_}distinct}]

   and:   \texttt{improve last} [\texttt{/depth} {\em number}]
                                              [\texttt{/no{\char`\_}distinct}]

   and:   \texttt{improve all} [\texttt{/depth} {\em number}]
                                              [\texttt{/no{\char`\_}distinct}]

These commands, available in the Proof Assistant\index{Proof Assistant}
only, try to find proofs automatically for unknown steps whose symbol
sequences are completely known.  They are primarily useful for filling in
proofs of \texttt{\$f}\index{\texttt{\$f} statement} hypotheses.  The
search will be restricted to statements having no
\texttt{\$e}\index{\texttt{\$e} statement} hypotheses.

\begin{sloppypar} % narrow
Note:  If memory is limited, \texttt{improve all} on a large proof may
overflow memory.  If you use \texttt{set unification{\char`\_}timeout 1}
before \texttt{improve all}, there will usually be sufficient
improvement to easily recover and completely \texttt{improve} the proof
later on a larger computer.  Warning:  Once memory has overflowed, there
is no recovery.  If in doubt, save the intermediate proof (\texttt{save
new{\char`\_}proof} then \texttt{write source}) before \texttt{improve
all}.
\end{sloppypar}

If \texttt{last} is specified instead of {\em step} number, the last
step that is shown by \texttt{show new{\char`\_}proof /unknown} will be
used.

If \texttt{first} is specified instead of {\em step} number, the first
step that is shown by \texttt{show new{\char`\_}proof /unknown} will be
used.

If {\em step} is zero or negative, the -{\em step}th from last unknown
step, as shown by \texttt{show new{\char`\_}proof /unknown}, will be
used.  \texttt{improve -1} will use the penultimate
unknown step, \texttt{improve -2} {\em label} the antepenultimate, and
\texttt{improve 0} is the same as \texttt{improve last}.

Optional command qualifier:

    \texttt{/depth} {\em number} - This qualifier will cause the search
        to include
        statements with \texttt{\$e} hypotheses (but no new variables in
        the \texttt{\$e}
        hypotheses), provided that the backtracking has not exceeded the
        specified depth. {\em Warning:}  Try \texttt{/depth 1},
        then \texttt{2}, then \texttt{3}, etc.
        in sequence because of possible exponential blowups.  Save your
        work before trying \texttt{/depth} greater than \texttt{1}!

    \texttt{/no{\char`\_}distinct} - Skip trial statements that have
        \texttt{\$d}\index{\texttt{\$d} statement} requirements.
        This qualifier will prevent assignments that might violate \texttt{\$d}
        requirements but it also could miss possible legal assignments.

See also: \texttt{set search{\char`\_}limit}

\subsection{\texttt{save new\_proof} Command}\index{\texttt{save
new{\char`\_}proof} command}
Syntax:  \texttt{save new{\char`\_}proof} {\em label} [\texttt{/normal}]
   [\texttt{/compressed}]

The \texttt{save new{\char`\_}proof} command is available in the Proof
Assistant only.  It saves the proof in progress in the source
buffer\index{source buffer}.  \texttt{save new{\char`\_}proof} may be
used to save a completed proof, or it may be used to save a proof in
progress in order to work on it later.  If an incomplete proof is saved,
any user assignments with \texttt{let step} or \texttt{let variable}
will be lost, as will any ambiguous unifications\index{ambiguous
unification}\index{unification!ambiguous} that were resolved manually.
To help make recovery easier, it can be helpful to \texttt{improve all}
before \texttt{save new{\char`\_}proof} so that the incomplete proof
will have as much information as possible.

Note that the proof will not be permanently saved until a \texttt{write
source} command is issued.

Optional command qualifiers:

    \texttt{/normal} - The proof is saved in the normal format (i.e., as a
        sequence of labels, which is the defined format of the basic Metamath
        language).\index{basic language}  This is the default format that
        is used if a qualifier is omitted.

    \texttt{/compressed} - The proof is saved in the compressed format, which
        reduces storage requirements for a database.
        (See Appendix~\ref{compressed}.)


\section{Creating \LaTeX\ Output}\label{texout}\index{latex@{\LaTeX}}

You can generate \LaTeX\ output given the
information in a database.
The database must already include the necessary typesetting information
(see section \ref{tcomment} for how to provide this information).

The \texttt{show statement} and \texttt{show proof} commands each have a
special \texttt{/tex} command qualifier that produces \LaTeX\ output.
(The \texttt{show statement} command also has the
\texttt{/simple{\char`\_}tex} qualifier for output that is easier to
edit by hand.)  Before you can use them, you must open a \LaTeX\ file to
which to send their output.  A typical complete session will use this
sequence of Metamath commands:

\begin{verbatim}
read set.mm
open tex example.tex
show statement a1i /tex
show proof a1i /all/lemmon/renumber/tex
show statement uneq2 /tex
show proof uneq2 /all/lemmon/renumber/tex
close tex
\end{verbatim}

See Section~\ref{mathcomments} for information on comment markup and
Appendix~\ref{ASCII} for information on how math symbol translation is
specified.

To format and print the \LaTeX\ source, you will need the \LaTeX\
program, which is standard on most Linux installations and available for
Windows.  On Linux, in order to create a {\sc pdf} file, you will
typically type at the shell prompt
\begin{verbatim}
$ pdflatex example.tex
\end{verbatim}

\subsection{\texttt{open tex} Command}\index{\texttt{open tex} command}
Syntax:  \texttt{open tex} {\em file-name} [\texttt{/no{\char`\_}header}]

This command opens a file for writing \LaTeX\
source\index{latex@{\LaTeX}} and writes a \LaTeX\ header to the file.
\LaTeX\ source can be written with the \texttt{show proof}, \texttt{show
new{\char`\_}proof}, and \texttt{show statement} commands using the
\texttt{/tex} qualifier.

The mapping to \LaTeX\ symbols is defined in a special comment
containing a \texttt{\$t} token, described in Appendix~\ref{ASCII}.

There is an optional command qualifier:

    \texttt{/no{\char`\_}header} - This qualifier prevents a standard
        \LaTeX\ header and trailer
        from being included with the output \LaTeX\ code.


\subsection{\texttt{close tex} Command}\index{\texttt{close tex} command}
Syntax:  \texttt{close tex}

This command writes a trailer to any \LaTeX\ file\index{latex@{\LaTeX}}
that was opened with \texttt{open tex} (unless
\texttt{/no{\char`\_}header} was used with \texttt{open tex}) and closes
the \LaTeX\ file.


\section{Creating {\sc HTML} Output}\label{htmlout}

You can generate {\sc html} web pages given the
information in a database.
The database must already include the necessary typesetting information
(see section \ref{tcomment} for how to provide this information).
The ability to produce {\sc html} web pages was added in Metamath version
0.07.30.

To create an {\sc html} output file(s) for \texttt{\$a} or \texttt{\$p}
statement(s), use
\begin{quote}
    \texttt{show statement} {\em label-match} \texttt{/html}
\end{quote}
The output file will be named {\em label-match}\texttt{.html}
for each match.  When {\em
label-match} has wildcard (\texttt{*}) characters, all statements with
matching labels will have {\sc html} files produced for them.  Also,
when {\em label-match} has a wildcard (\texttt{*}) character, two additional
files, \texttt{mmdefinitions.html} and \texttt{mmascii.html} will be
produced.  To produce {\em only} these two additional files, you can use
\texttt{?*}, which will not match any statement label, in place of {\em
label-match}.

There are three other qualifiers for \texttt{show statement} that also
generate {\sc HTML} code.  These are \texttt{/alt{\char`\_}html},
\texttt{/brief{\char`\_}html}, and
\texttt{/brief{\char`\_}alt{\char`\_}html}, and are described in the
next section.

The command
\begin{quote}
    \texttt{show statement} {\em label-match} \texttt{/alt{\char`\_}html}
\end{quote}
does the same as \texttt{show statement} {\em label-match} \texttt{/html},
except that the {\sc html} code for the symbols is taken from
\texttt{althtmldef} statements instead of \texttt{htmldef} statements in
the \texttt{\$t} comment.

The command
\begin{verbatim}
show statement * /brief_html
\end{verbatim}
invokes a special mode that just produces definition and theorem lists
accompanied by their symbol strings, in a format suitable for copying and
pasting into another web page (such as the tutorial pages on the
Metamath web site).

Finally, the command
\begin{verbatim}
show statement * /brief_alt_html
\end{verbatim}
does the same as \texttt{show statement * / brief{\char`\_}html}
for the alternate {\sc html}
symbol representation.

A statement's comment can include a special notation that provides a
certain amount of control over the {\sc HTML} version of the comment.  See
Section~\ref{mathcomments} (p.~\pageref{mathcomments}) for the comment
markup features.

The \texttt{write theorem{\char`\_}list} and \texttt{write bibliography}
commands, which are described below, provide as a side effect complete
error checking for all of the features described in this
section.\index{error checking}

\subsection{\texttt{write theorem\_list}
Command}\index{\texttt{write theorem{\char`\_}list} command}
Syntax:  \texttt{write theorem{\char`\_}list}
[\texttt{/theorems{\char`\_}per{\char`\_}page} {\em number}]

This command writes a list of all of the \texttt{\$a} and \texttt{\$p}
statements in the database into a web page file
 called \texttt{mmtheorems.html}.
When additional files are needed, they are called
\texttt{mmtheorems2.html}, \texttt{mmtheorems3.html}, etc.

Optional command qualifier:

    \texttt{/theorems{\char`\_}per{\char`\_}page} {\em number} -
 This qualifier specifies the number of statements to
        write per web page.  The default is 100.

{\em Note:} In version 0.07.30\index{Metamath!limitations of version
0.07.30} of Metamath, the ``Nearby theorems'' links on the individual
web pages presuppose 100 theorems per page when linking to the theorem
list pages.  Therefore the \texttt{/theorems{\char`\_}per{\char`\_}page}
qualifier, if it specifies a number other than 100, will cause the
individual web pages to be out of sync and should not be used to
generate the main theorem list for the web site.  This is expected to be
fixed in a future version.


\subsection{\texttt{write bibliography}\label{wrbib}
Command}\index{\texttt{write bibliography} command}
Syntax:  \texttt{write bibliography} {\em filename}

This command reads an existing {\sc html} bibliographic cross-reference
file, normally called \texttt{mmbiblio.html}, and updates it per the
bibliographic links in the database comments.  The file is updated
between the {\sc html} comment lines \texttt{<!--
{\char`\#}START{\char`\#} -->} and \texttt{<!-- {\char`\#}END{\char`\#}
-->}.  The original input file is renamed to {\em
filename}\texttt{{\char`\~}1}.

A bibliographic reference is indicated with the reference name
in brackets, such as  \texttt{Theorem 3.1 of
[Monk] p.\ 22}.
See Section~\ref{htmlout} (p.~\pageref{htmlout}) for
syntax details.


\subsection{\texttt{write recent\_additions}
Command}\index{\texttt{write recent{\char`\_}additions} command}
Syntax:  \texttt{write recent{\char`\_}additions} {\em filename}
[\texttt{/limit} {\em number}]

This command reads an existing ``Recent Additions'' {\sc html} file,
normally called \texttt{mmrecent.html}, and updates it with the
descriptions of the most recently added theorems to the database.
 The file is updated between
the {\sc html} comment lines \texttt{<!-- {\char`\#}START{\char`\#} -->}
and \texttt{<!-- {\char`\#}END{\char`\#} -->}.  The original input file
is renamed to {\em filename}\texttt{{\char`\~}1}.

Optional command qualifier:

    \texttt{/limit} {\em number} -
 This qualifier specifies the number of most recent theorems to
   write to the output file.  The default is 100.


\section{Text File Utilities}

\subsection{\texttt{tools} Command}\index{\texttt{tools} command}
Syntax:  \texttt{tools}

This command invokes an easy-to-use, general purpose utility for
manipulating the contents of {\sc ascii} text files.  Upon typing
\texttt{tools}, the command-line prompt will change to \texttt{TOOLS>}
until you type \texttt{exit}.  The \texttt{tools} commands can be used
to perform simple, global edits on an input/output file,
such as making a character string substitution on each line, adding a
string to each line, and so on.  A typical use of this utility is
to build a \texttt{submit} input file to perform a common operation on a
list of statements obtained from \texttt{show label} or \texttt{show
usage}.

The actions of most of the \texttt{tools} commands can also be
performed with equivalent (and more powerful) Unix shell commands, and
some users may find those more efficient.  But for Windows users or
users not comfortable with Unix, \texttt{tools} provides an
easy-to-learn alternative that is adequate for most of the
script-building tasks needed to use the Metamath program effectively.

\subsection{\texttt{help} Command (in \texttt{tools})}
Syntax:  \texttt{help}

The \texttt{help} command lists the commands available in the
\texttt{tools} utility, along with a brief description.  Each command,
in turn, has its own help, such as \texttt{help add}.  As with
Metamath's \texttt{MM>} prompt, a complete command can be entered at
once, or just the command word can be typed, causing the program to
prompt for each argument.

\vskip 1ex
\noindent Line-by-line editing commands:

  \texttt{add} - Add a specified string to each line in a file.

  \texttt{clean} - Trim spaces and tabs on each line in a file; convert
         characters.

  \texttt{delete} - Delete a section of each line in a file.

  \texttt{insert} - Insert a string at a specified column in each line of
        a file.

  \texttt{substitute} - Make a simple substitution on each line of the file.

  \texttt{tag} - Like \texttt{add}, but restricted to a range of lines.

  \texttt{swap} - Swap the two halves of each line in a file.

\vskip 1ex
\noindent Other file-processing commands:

  \texttt{break} - Break up (tokenize) a file into a list of tokens (one per
        line).

  \texttt{build} - Build a file with multiple tokens per line from a list.

  \texttt{count} - Count the occurrences in a file of a specified string.

  \texttt{number} - Create a list of numbers.

  \texttt{parallel} - Put two files in parallel.

  \texttt{reverse} - Reverse the order of the lines in a file.

  \texttt{right} - Right-justify lines in a file (useful before sorting
         numbers).

%  \texttt{tag} - Tag edit updates in a program for revision control.

  \texttt{sort} - Sort the lines in a file with key starting at
         specified string.

  \texttt{match} - Extract lines containing (or not) a specified string.

  \texttt{unduplicate} - Eliminate duplicate occurrences of lines in a file.

  \texttt{duplicate} - Extract first occurrence of any line occurring
         more than

   \ \ \    once in a file, discarding lines occurring exactly once.

  \texttt{unique} - Extract lines occurring exactly once in a file.

  \texttt{type} (10 lines) - Display the first few lines in a file.
                  Similar to Unix \texttt{head}.

  \texttt{copy} - Similar to Unix \texttt{cat} but safe (same input
         and output file allowed).

  \texttt{submit} - Run a script containing \texttt{tools} commands.

\vskip 1ex

\noindent Note:
  \texttt{unduplicate}, \texttt{duplicate}, and \texttt{unique} also
 sort the lines as a side effect.


\subsection{Using \texttt{tools} to Build Metamath \texttt{submit}
Scripts}

The \texttt{break} command is typically used to break up a series of
statement labels, such as the output of Metamath's \texttt{show usage},
into one label per line.  The other \texttt{tools} commands can then be
used to add strings before and after each statement label to specify
commands to be performed on the statement.  The \texttt{parallel}
command is useful when a statement label must be mentioned more than
once on a line.

Very often a \texttt{submit} script for Metamath will require multiple
command lines for each statement being processed.  For example, you may
want to enter the Proof Assistant, \texttt{minimize{\char`\_}with} your
latest theorem, \texttt{save} the new proof, and \texttt{exit} the Proof
Assistant.  To accomplish this, you can build a file with these four
commands for each statement on a single line, separating each command
with a designated character such as \texttt{@}.  Then at the end you can
\texttt{substitute} each \texttt{@} with \texttt{{\char`\\}n} to break
up the lines into individual command lines (see \texttt{help
substitute}).


\subsection{Example of a \texttt{tools} Session}

To give you a quick feel for the \texttt{tools} utility, we show a
simple session where we create a file \texttt{n.txt} with 3 lines, add
strings before and after each line, and display the lines on the screen.
You can experiment with the various commands to gain experience with the
\texttt{tools} utility.

\begin{verbatim}
MM> tools
Entering the Text Tools utilities.
Type HELP for help, EXIT to exit.
TOOLS> number
Output file <n.tmp>? n.txt
First number <1>?
Last number <10>? 3
Increment <1>?
TOOLS> add
Input/output file? n.txt
String to add to beginning of each line <>? This is line
String to add to end of each line <>? .
The file n.txt has 3 lines; 3 were changed.
First change is on line 1:
This is line 1.
TOOLS> type n.txt
This is line 1.
This is line 2.
This is line 3.
TOOLS> exit
Exiting the Text Tools.
Type EXIT again to exit Metamath.
MM>
\end{verbatim}



\appendix
\chapter{Sample Representations}
\label{ASCII}

This Appendix provides a sample of {\sc ASCII} representations,
their corresponding traditional mathematical symbols,
and a discussion of their meanings
in the \texttt{set.mm} database.
These are provided in order of appearance.
This is only a partial list, and new definitions are routinely added.
A complete list is available at \url{http://metamath.org}.

These {\sc ASCII} representations, along
with information on how to display them,
are defined in the \texttt{set.mm} database file inside
a special comment called a \texttt{\$t} {\em
comment}\index{\texttt{\$t} comment} or {\em typesetting
comment.}\index{typesetting comment}
A typesetting comment
is indicated by the appearance of the
two-character string \texttt{\$t} at the beginning of the comment.
For more information,
see Section~\ref{tcomment}, p.~\pageref{tcomment}.

In the following table the ``{\sc ASCII}'' column shows the {\sc ASCII}
representation,
``Symbol'' shows the mathematical symbolic display
that corresponds to that {\sc ASCII} representation, ``Labels'' shows
the key label(s) that define the representation, and
``Description'' provides a description about the symbol.
As usual, ``iff'' is short for ``if and only if.''\index{iff}
In most cases the ``{\sc ASCII}'' column only shows
the key token, but it will sometimes show a sequence of tokens
if that is necessary for clarity.

{\setlength{\extrarowsep}{4pt} % Keep rows from being too close together
\begin{longtabu}   { @{} c c l X }
\textbf{ASCII} & \textbf{Symbol} & \textbf{Labels} & \textbf{Description} \\
\endhead
\texttt{|-} & $\vdash$ & &
  ``It is provable that...'' \\
\texttt{ph} & $\varphi$ & \texttt{wph} &
  The wff (boolean) variable phi,
  conventionally the first wff variable. \\
\texttt{ps} & $\psi$ & \texttt{wps} &
  The wff (boolean) variable psi,
  conventionally the second wff variable. \\
\texttt{ch} & $\chi$ & \texttt{wch} &
  The wff (boolean) variable chi,
  conventionally the third wff variable. \\
\texttt{-.} & $\lnot$ & \texttt{wn} &
  Logical not. E.g., if $\varphi$ is true, then $\lnot \varphi$ is false. \\
\texttt{->} & $\rightarrow$ & \texttt{wi} &
  Implies, also known as material implication.
  In classical logic the expression $\varphi \rightarrow \psi$ is true
  if either $\varphi$ is false or $\psi$ is true (or both), that is,
  $\varphi \rightarrow \psi$ has the same meaning as
  $\lnot \varphi \lor \psi$ (as proven in theorem \texttt{imor}). \\
\texttt{<->} & $\leftrightarrow$ &
  \hyperref[df-bi]{\texttt{df-bi}} &
  Biconditional (aka is-equals for boolean values).
  $\varphi \leftrightarrow \psi$ is true iff
  $\varphi$ and $\psi$ have the same value. \\
\texttt{\char`\\/} & $\lor$ &
  \makecell[tl]{{\hyperref[df-or]{\texttt{df-or}}}, \\
	         \hyperref[df-3or]{\texttt{df-3or}}} &
  Disjunction (logical ``or''). $\varphi \lor \psi$ is true iff
  $\varphi$, $\psi$, or both are true. \\
\texttt{/\char`\\} & $\land$ &
  \makecell[tl]{{\hyperref[df-an]{\texttt{df-an}}}, \\
                 \hyperref[df-3an]{\texttt{df-3an}}} &
  Conjunction (logical ``and''). $\varphi \land \psi$ is true iff
  both $\varphi$ and $\psi$ are true. \\
\texttt{A.} & $\forall$ &
  \texttt{wal} &
  For all; the wff $\forall x \varphi$ is true iff
  $\varphi$ is true for all values of $x$. \\
\texttt{E.} & $\exists$ &
  \hyperref[df-ex]{\texttt{df-ex}} &
  There exists; the wff
  $\exists x \varphi$ is true iff
  there is at least one $x$ where $\varphi$ is true. \\
\texttt{[ y / x ]} & $[ y / x ]$ &
  \hyperref[df-sb]{\texttt{df-sb}} &
  The wff $[ y / x ] \varphi$ produces
  the result when $y$ is properly substituted for $x$ in $\varphi$
  ($y$ replaces $x$).
  % This is elsb4
  % ( [ x / y ] z e. y <-> z e. x )
  For example,
  $[ x / y ] z \in y$ is the same as $z \in x$. \\
\texttt{E!} & $\exists !$ &
  \hyperref[df-eu]{\texttt{df-eu}} &
  There exists exactly one;
  $\exists ! x \varphi$ is true iff
  there is at least one $x$ where $\varphi$ is true. \\
\texttt{\{ y | phi \}}  & $ \{ y | \varphi \}$ &
  \hyperref[df-clab]{\texttt{df-clab}} &
  The class of all sets where $\varphi$ is true. \\
\texttt{=} & $ = $ &
  \hyperref[df-cleq]{\texttt{df-cleq}} &
  Class equality; $A = B$ iff $A$ equals $B$. \\
\texttt{e.} & $ \in $ &
  \hyperref[df-clel]{\texttt{df-clel}} &
  Class membership; $A \in B$ if $A$ is a member of $B$. \\
\texttt{{\char`\_}V} & {\rm V} &
  \hyperref[df-v]{\texttt{df-v}} &
  Class of all sets (not itself a set). \\
\texttt{C\_} & $ \subseteq $ &
  \hyperref[df-ss]{\texttt{df-ss}} &
  Subclass (subset); $A \subseteq B$ is true iff
  $A$ is a subclass of $B$. \\
\texttt{u.} & $ \cup $ &
  \hyperref[df-un]{\texttt{df-un}} &
  $A \cup B$ is the union of classes $A$ and $B$. \\
\texttt{i^i} & $ \cap $ &
  \hyperref[df-in]{\texttt{df-in}} &
  $A \cap B$ is the intersection of classes $A$ and $B$. \\
\texttt{\char`\\} & $ \setminus $ &
  \hyperref[df-dif]{\texttt{df-dif}} &
  $A \setminus B$ (set difference)
  is the class of all sets in $A$ except for those in $B$. \\
\texttt{(/)} & $ \varnothing $ &
  \hyperref[df-nul]{\texttt{df-nul}} &
  $ \varnothing $ is the empty set (aka null set). \\
\texttt{\char`\~P} & $ \cal P $ &
  \hyperref[df-pw]{\texttt{df-pw}} &
  Power class. \\
\texttt{<.\ A , B >.} & $\langle A , B \rangle$ &
  \hyperref[df-op]{\texttt{df-op}} &
  The ordered pair $\langle A , B \rangle$. \\
\texttt{( F ` A )} & $ ( F ` A ) $ &
  \hyperref[df-fv]{\texttt{df-fv}} &
  The value of function $F$ when applied to $A$. \\
\texttt{\_i} & $ i $ &
  \texttt{df-i} &
  The square root of negative one. \\
\texttt{x.} & $ \cdot $ &
  \texttt{df-mul} &
  Complex number multiplication; $2~\cdot~3~=~6$. \\
\texttt{CC} & $ \mathbb{C} $ &
  \texttt{df-c} &
  The set of complex numbers. \\
\texttt{RR} & $ \mathbb{R} $ &
  \texttt{df-r} &
  The set of real numbers. \\
\end{longtabu}
} % end of extrarowsep

\chapter{Compressed Proofs}
\label{compressed}\index{compressed proof}\index{proof!compressed}

The proofs in the \texttt{set.mm} set theory database are stored in compressed
format for efficiency.  Normally you needn't concern yourself with the
compressed format, since you can display it with the usual proof display tools
in the Metamath program (\texttt{show proof}\ldots) or convert it to the normal
RPN proof format described in Section~\ref{proof} (with \texttt{save proof}
{\em label} \texttt{/normal}).  However for sake of completeness we describe the
format here and show how it maps to the normal RPN proof format.

A compressed proof, located between \texttt{\$=} and \texttt{\$.}\ keywords, consists
of a left parenthesis, a sequence of statement labels, a right parenthesis,
and a sequence of upper-case letters \texttt{A} through \texttt{Z} (with optional
white space between them).  White space must surround the parentheses
and the labels.  The left parenthesis tells Metamath that a
compressed proof follows.  (A normal RPN proof consists of just a sequence of
labels, and a parenthesis is not a legal character in a label.)

The sequence of upper-case letters corresponds to a sequence of integers
with the following mapping.  Each integer corresponds to a proof step as
described later.
\begin{center}
  \texttt{A} = 1 \\
  \texttt{B} = 2 \\
   \ldots \\
  \texttt{T} = 20 \\
  \texttt{UA} = 21 \\
  \texttt{UB} = 22 \\
   \ldots \\
  \texttt{UT} = 40 \\
  \texttt{VA} = 41 \\
  \texttt{VB} = 42 \\
   \ldots \\
  \texttt{YT} = 120 \\
  \texttt{UUA} = 121 \\
   \ldots \\
  \texttt{YYT} = 620 \\
  \texttt{UUUA} = 621 \\
   etc.
\end{center}

In other words, \texttt{A} through \texttt{T} represent the
least-significant digit in base 20, and \texttt{U} through \texttt{Y}
represent zero or more most-significant digits in base 5, where the
digits start counting at 1 instead of the usual 0. With this scheme, we
don't need white space between these ``numbers.''

(In the design of the compressed proof format, only upper-case letters,
as opposed to say all non-whitespace printable {\sc ascii} characters other than
%\texttt{\$}, was chosen to make the compressed proof a little less
%displeasing to the eye, at the expense of a typical 20\% compression
\texttt{\$}, were chosen so as not to collide with most text editor
searches, at the expense of a typical 20\% compression
loss.  The base 5/base 20 grouping, as opposed to say base 6/base 19,
was chosen by experimentally determining the grouping that resulted in
best typical compression.)

The letter \texttt{Z} identifies (tags) a proof step that is identical to one
that occurs later on in the proof; it helps shorten the proof by not requiring
that identical proof steps be proved over and over again (which happens often
when building wff's).  The \texttt{Z} is placed immediately after the
least-significant digit (letters \texttt{A} through \texttt{T}) that ends the integer
corresponding to the step to later be referenced.

The integers that the upper-case letters correspond to are mapped to labels as
follows.  If the statement being proved has $m$ mandatory hypotheses, integers
1 through $m$ correspond to the labels of these hypotheses in the order shown
by the \texttt{show statement ... / full} command, i.e., the RPN order\index{RPN
order} of the mandatory
hypotheses.  Integers $m+1$ through $m+n$ correspond to the labels enclosed in
the parentheses of the compressed proof, in the order that they appear, where
$n$ is the number of those labels.  Integers $m+n+1$ on up don't directly
correspond to statement labels but point to proof steps identified with the
letter \texttt{Z}, so that these proof steps can be referenced later in the
proof.  Integer $m+n+1$ corresponds to the first step tagged with a \texttt{Z},
$m+n+2$ to the second step tagged with a \texttt{Z}, etc.  When the compressed
proof is converted to a normal proof, the entire subproof of a step tagged
with \texttt{Z} replaces the reference to that step.

For efficiency, Metamath works with compressed proofs directly, without
converting them internally to normal proofs.  In addition to the usual
error-checking, an error message is given if (1) a label in the label list in
parentheses does not refer to a previous \texttt{\$p} or \texttt{\$a} statement or a
non-mandatory hypothesis of the statement being proved and (2) a proof step
tagged with \texttt{Z} is referenced before the step tagged with the \texttt{Z}.

Just as in a normal proof under development (Section~\ref{unknown}), any step
or subproof that is not yet known may be represented with a single \texttt{?}.
White space does not have to appear between the \texttt{?}\ and the upper-case
letters (or other \texttt{?}'s) representing the remainder of the proof.

% April 1, 2004 Appendix C has been added back in with corrections.
%
% May 20, 2003 Appendix C was removed for now because there was a problem found
% by Bob Solovay
%
% Also, removed earlier \ref{formalspec} 's (3 cases above)
%
% Bob Solovay wrote on 30 Nov 2002:
%%%%%%%%%%%%% (start of email comment )
%      3. My next set of comments concern appendix C. I read this before I
% read Chapter 4. So I first noted that the system as presented in the
% Appendix lacked a certain formal property that I thought desirable. I
% then came up with a revised formal system that had this property. Upon
% reading Chapter 4, I noticed that the revised system was closer to the
% treatment in Chapter 4 than the system in Appendix C.
%
%         First a very minor correction:
%
%         On page 142 line 2: The condition that V(e) != V(f) should only be
% required of e, f in T such that e != f.
%
%         Here is a natural property [transitivity] that one would like
% the formal system to have:
%
%         Let Gamma be a set of statements. Suppose that the statement Phi
% is provable from Gamma and that the statement Psi is provable from Gamma
% \cup {Phi}. Then Psi is provable from Gamma.
%
%         I shall present an example to show that this property does not
% hold for the formal systems of Appendix C:
%
%         I write the example in metamath style:
%
% $c A B C D E $.
% $v x y
%
% ${
% tx $f A x $.
% ty $f B y $.
% ax1 $a C x y $.
% $}
%
% ${
% tx $f A x $.
% ty $f B y $.
% ax2-h1 $e C x y $.
% ax2 $a D y $.
% $}
%
% ${
% ty $f B y $.
% ax3-h1 $e D y $.
% ax3 $a E y $.
% $}
%
% $(These three axioms are Gamma $)
%
% ${
% tx $f A x $.
% ty $f B y $.
% Phi $p D y $=
% tx ty tx ty ax1 ax2 $.
% $}
%
% ${
% ty $f B y $.
% Psi $p E y $=
% ty ty Phi ax3 $.
% $}
%
%
% I omit the formal proofs of the following claims. [I will be glad to
% supply them upon request.]
%
% 1) Psi is not provable from Gamma;
%
% 2) Psi is provable from Gamma + Phi.
%
% Here "provable" refers to the formalism of Appendix C.
%
% The trouble of course is that Psi is lacking the variable declaration
%
% $f Ax $.
%
% In the Metamath system there is no trouble proving Psi. I attach a
% metamath file that shows this and which has been checked by the
% metamath program.
%
% I next want to indicate how I think the treatment in Appendix C should
% be revised so as to conform more closely to the metamath system of the
% main text. The revised system *does* have the transitivity property.
%
% We want to give revised definitions of "statement" and
% "provable". [cf. sections C.2.4. and C.2.5] Our new definitions will
% use the definitions given in Appendix C. So we take the following
% tack. We refer to the original notions as o-statement and o-provable. And
% we refer to the notions we are defining as n-statement and n-provable.
%
%         A n-statement is an o-statement in which the only variables
% that appear in the T component are mandatory.
%
%         To any o-statement we can associate its reduct which is a
% n-statement by dropping all the elements of T or D which contain
% non-mandatory variables.
%
%         An n-statement gamma is n-provable if there is an o-statement
% gamma' which has gamma as its reduct andf such that gamma' is
% o-provable.
%
%         It seems to me [though I am not completely sure on this point]
% that n-provability corresponds to metamath provability as discussed
% say in Chapter 4.
%
%         Attached to this letter is the metamath proof of Phi and Psi
% from Gamma discussed above.
%
%         I am still brooding over the question of whether metamath
% correctly formalizes set-theory. No doubt I will have some questions
% re this after my thoughts become clearer.
%%%%%%%%%%%%%%%% (end of email comment)

%%%%%%%%%%%%%%%% (start of 2nd email comment from Bob Solovay 1-Apr-04)
%
%         I hope that Appendix C is the one that gives a "formal" treatment
% of Metamath. At any rate, thats the appendix I want to comment on.
%
%         I'm going to suggest two changes in the definition.
%
%         First change (in the definition of statement): Require that the
% sets D, T, and E be finite.
%
%         Probably things are fine as you give them. But in the applications
% to the main metamath system they will always be finite, and its useful in
% thinking about things [at least for me] to stick to the finite case.
%
%         Second change:
%
%         First let me give an approximate description. Remove the dummy
% variables from the statement. Instead, include them in the proof.
%
%         More formally: Require that T consists of type declarations only
% for mandatory variables. Require that all the pairs in D consist of
% mandatory variables.
%
%         At the start of a proof we are allowed to declare a finite number
% of dummy variables [provided that none of them appear in any of the
% statements in E \cup {A}. We have to supply type declarations for all the
% dummy variables. We are allowed to add new $d statements referring to
% either the mandatory or dummy variables. But we require that no new $d
% statement references only mandatory variables.
%
%         I find this way of doing things more conceptual than the treatment
% in Appendix C. But the change [which I will use implicitly in later
% letters about doing Peano] is mainly aesthetic. I definitely claim that my
% results on doing Peano all apply to Metamath as it is presented in your
% book.
%
%         --Bob
%
%%%%%%%%%%%%%%%% (end of 2nd email comment)

%%
%% When uncommenting the below, also uncomment references above to {formalspec}
%%
\chapter{Metamath's Formal System}\label{formalspec}\index{Metamath!as a formal
system}

\section{Introduction}

\begin{quote}
  {\em Perfection is when there is no longer anything more to take away.}
    \flushright\sc Antoine de
     Saint-Exupery\footnote{\cite[p.~3-25]{Campbell}.}\\
\end{quote}\index{de Saint-Exupery, Antoine}

This appendix describes the theory behind the Metamath language in an abstract
way intended for mathematicians.  Specifically, we construct two
set-theo\-ret\-i\-cal objects:  a ``formal system'' (roughly, a set of syntax
rules, axioms, and logical rules) and its ``universe'' (roughly, the set of
theorems derivable in the formal system).  The Metamath computer language
provides us with a way to describe specific formal systems and, with the aid of
a proof provided by the user, to verify that given theorems
belong to their universes.

To understand this appendix, you need a basic knowledge of informal set theory.
It should be sufficient to understand, for example, Ch.\ 1 of Munkres' {\em
Topology} \cite{Munkres}\index{Munkres, James R.} or the
introductory set theory chapter
in many textbooks that introduce abstract mathematics. (Note that there are
minor notational differences among authors; e.g.\ Munkres uses $\subset$ instead
of our $\subseteq$ for ``subset.''  We use ``included in'' to mean ``a subset
of,'' and ``belongs to'' or ``is contained in'' to mean ``is an element of.'')
What we call a ``formal'' description here, unlike earlier, is actually an
informal description in the ordinary language of mathematicians.  However we
provide sufficient detail so that a mathematician could easily formalize it,
even in the language of Metamath itself if desired.  To understand the logic
examples at the end of this appendix, familiarity with an introductory book on
mathematical logic would be helpful.

\section{The Formal Description}

\subsection[Preliminaries]{Preliminaries\protect\footnotemark}%
\footnotetext{This section is taken mostly verbatim
from Tarski \cite[p.~63]{Tarski1965}\index{Tarski, Alfred}.}

By $\omega$ we denote the set of all natural numbers (non-negative integers).
Each natural number $n$ is identified with the set of all smaller numbers: $n =
\{ m | m < n \}$.  The formula $m < n$ is thus equivalent to the condition: $m
\in n$ and $m,n \in \omega$. In particular, 0 is the number zero and at the
same time the empty set $\varnothing$, $1=\{0\}$, $2=\{0,1\}$, etc. ${}^B A$
denotes the set of all functions on $B$ to $A$ (i.e.\ with domain $B$ and range
included in $A$).  The members of ${}^\omega A$ are what are called {\em simple
infinite sequences},\index{simple infinite sequence}
with all {\em terms}\index{term} in $A$.  In case $n \in \omega$, the
members of ${}^n A$ are referred to as {\em finite $n$-termed
sequences},\index{finite $n$-termed
sequence} again
with terms in $A$.  The consecutive terms (function values) of a finite or
infinite sequence $f$ are denoted by $f_0, f_1, \ldots ,f_n,\ldots$.  Every
finite sequence $f \in \bigcup _{n \in \omega} {}^n A$ uniquely determines the
number $n$ such that $f \in {}^n A$; $n$ is called the {\em
length}\index{length of a sequence ({$"|\ "|$})} of $f$ and
is denoted by $|f|$.  $\langle a \rangle$ is the sequence $f$ with $|f|=1$ and
$f_0=a$; $\langle a,b \rangle$ is the sequence $f$ with $|f|=2$, $f_0=a$,
$f_1=b$; etc.  Given two finite sequences $f$ and $g$, we denote by $f\frown g$
their {\em concatenation},\index{concatenation} i.e., the
finite sequence $h$ determined by the
conditions:
\begin{eqnarray*}
& |h| = |f|+|g|;&  \\
& h_n = f_n & \mbox{\ for\ } n < |f|;  \\
& h_{|f|+n} = g_n & \mbox{\ for\ } n < |g|.
\end{eqnarray*}

\subsection{Constants, Variables, and Expressions}

A formal system has a set of {\em symbols}\index{symbol!in
a formal system} denoted
by $\mbox{\em SM}$.  A
precise set-theo\-ret\-i\-cal definition of this set is unimportant; a symbol
could be considered a primitive or atomic element if we wish.  We assume this
set is divided into two disjoint subsets:  a set $\mbox{\em CN}$ of {\em
constants}\index{constant!in a formal system} and a set $\mbox{\em VR}$ of
{\em variables}.\index{variable!in a formal system}  $\mbox{\em CN}$ and
$\mbox{\em VR}$ are each assumed to consist of countably many symbols which
may be arranged in finite or simple infinite sequences $c_0, c_1, \ldots$ and
$v_0, v_1, \ldots$ respectively, without repeating terms.  We will represent
arbitrary symbols by metavariables $\alpha$, $\beta$, etc.

{\footnotesize\begin{quotation}
{\em Comment.} The variables $v_0, v_1, \ldots$ of our formal system
correspond to what are usually considered ``metavariables'' in
descriptions of specific formal systems in the literature.  Typically,
when describing a specific formal system a book will postulate a set of
primitive objects called variables, then proceed to describe their
properties using metavariables that range over them, never mentioning
again the actual variables themselves.  Our formal system does not
mention these primitive variable objects at all but deals directly with
metavariables, as its primitive objects, from the start.  This is a
subtle but key distinction you should keep in mind, and it makes our
definition of ``formal system'' somewhat different from that typically
found in the literature.  (So, the $\alpha$, $\beta$, etc.\ above are
actually ``metametavariables'' when used to represent $v_0, v_1,
\ldots$.)
\end{quotation}}

Finite sequences all terms of which are symbols are called {\em
expressions}.\index{expression!in a formal system}  $\mbox{\em EX}$ is
the set of all expressions; thus
\begin{displaymath}
\mbox{\em EX} = \bigcup _{n \in \omega} {}^n \mbox{\em SM}.
\end{displaymath}

A {\em constant-prefixed expression}\index{constant-prefixed expression}
is an expression of non-zero length
whose first term is a constant.  We denote the set of all constant-prefixed
expressions by $\mbox{\em EX}_C = \{ e \in \mbox{\em EX} | ( |e| > 0 \wedge
e_0 \in \mbox{\em CN} ) \}$.

A {\em constant-variable pair}\index{constant-variable pair}
is an expression of length 2 whose first term
is a constant and whose second term is a variable.  We denote the set of all
constant-variable pairs by $\mbox{\em EX}_2 = \{ e \in \mbox{\em EX}_C | ( |e|
= 2 \wedge e_1 \in \mbox{\em VR} ) \}$.


{\footnotesize\begin{quotation}
{\em Relationship to Metamath.} In general, the set $\mbox{\em SM}$
corresponds to the set of declared math symbols in a Metamath database, the
set $\mbox{\em CN}$ to those declared with \texttt{\$c} statements, and the set
$\mbox{\em VR}$ to those declared with \texttt{\$v} statements.  Of course a
Metamath database can only have a finite number of math symbols, whereas
formal systems in general can have an infinite number, although the number of
Metamath math symbols available is in principle unlimited.

The set $\mbox{\em EX}_C$ corresponds to the set of permissible expressions
for \texttt{\$e}, \texttt{\$a}, and \texttt{\$p} statements.  The set $\mbox{\em EX}_2$
corresponds to the set of permissible expressions for \texttt{\$f} statements.
\end{quotation}}

We denote by ${\cal V}(e)$ the set of all variables in an expression $e \in
\mbox{\em EX}$, i.e.\ the set of all $\alpha \in \mbox{\em VR}$ such that
$\alpha = e_n$ for some $n < |e|$.  We also denote (with abuse of notation) by
${\cal V}(E)$ the set of all variables in a collection of expressions $E
\subseteq \mbox{\em EX}$, i.e.\ $\bigcup _{e \in E} {\cal V}(e)$.


\subsection{Substitution}

Given a function $F$ from $\mbox{\em VR}$ to
$\mbox{\em EX}$, we
denote by $\sigma_{F}$ or just $\sigma$ the function from $\mbox{\em EX}$ to
$\mbox{\em EX}$ defined recursively for nonempty sequences by
\begin{eqnarray*}
& \sigma(<\alpha>) = F(\alpha) & \mbox{for\ } \alpha \in \mbox{\em VR}; \\
& \sigma(<\alpha>) = <\alpha> & \mbox{for\ } \alpha \not\in \mbox{\em VR}; \\
& \sigma(g \frown h) = \sigma(g) \frown
    \sigma(h) & \mbox{for\ } g,h \in \mbox{\em EX}.
\end{eqnarray*}
We also define $\sigma(\varnothing)=\varnothing$.  We call $\sigma$ a {\em
simultaneous substitution}\index{substitution!variable}\index{variable
substitution} (or just {\em substitution}) with {\em substitution
map}\index{substitution map} $F$.

We also denote (with abuse of notation) by $\sigma(E)$ a substitution on a
collection of expressions $E \subseteq \mbox{\em EX}$, i.e.\ the set $\{
\sigma(e) | e \in E \}$.  The collection $\sigma(E)$ may of course contain
fewer expressions than $E$ because duplicate expressions could result from the
substitution.

\subsection{Statements}

We denote by $\mbox{\em DV}$ the set of all
unordered pairs $\{\alpha, \beta \} \subseteq \mbox{\em VR}$ such that $\alpha
\neq \beta$.  $\mbox{\em DV}$ stands for ``distinct variables.''

A {\em pre-statement}\index{pre-statement!in a formal system} is a
quadruple $\langle D,T,H,A \rangle$ such that
$D\subseteq \mbox{\em DV}$, $T\subseteq \mbox{\em EX}_2$, $H\subseteq
\mbox{\em EX}_C$ and $H$ is finite,
$A\in \mbox{\em EX}_C$, ${\cal V}(H\cup\{A\}) \subseteq
{\cal V}(T)$, and $\forall e,f\in T {\ } {\cal V}(e) \neq {\cal V}(f)$ (or
equivalently, $e_1 \ne f_1$) whenever $e \neq f$. The terms of the quadruple are called {\em
distinct-variable restrictions},\index{disjoint-variable restriction!in a
formal system} {\em variable-type hypotheses},\index{variable-type
hypothesis!in a formal system} {\em logical hypotheses},\index{logical
hypothesis!in a formal system} and the {\em assertion}\index{assertion!in a
formal system} respectively.  We denote by $T_M$ ({\em mandatory variable-type
hypotheses}\index{mandatory variable-type hypothesis!in a formal system}) the
subset of $T$ such that ${\cal V}(T_M) ={\cal V}(H \cup \{A\})$.  We denote by
$D_M=\{\{\alpha,\beta\}\in D|\{\alpha,\beta\}\subseteq {\cal V}(T_M)\}$ the
{\em mandatory distinct-variable restrictions}\index{mandatory
disjoint-variable restriction!in a formal system} of the pre-statement.
The set
of {\em mandatory hypotheses}\index{mandatory hypothesis!in a formal system}
is $T_M\cup H$.  We call the quadruple $\langle D_M,T_M,H,A \rangle$
the {\em reduct}\index{reduct!in a formal system} of
the pre-statement $\langle D,T,H,A \rangle$.

A {\em statement} is the reduct of some pre-statement\index{statement!in a
formal system}.  A statement is therefore a special kind of pre-statement;
in particular, a statement is the reduct of itself.

{\footnotesize\begin{quotation}
{\em Comment.}  $T$ is a set of expressions, each of length 2, that associate
a set of constants (``variable types'') with a set of variables.  The
condition ${\cal V}(H\cup\{A\}) \subseteq {\cal V}(T) $
means that each variable occurring in a statement's logical
hypotheses or assertion must have an associated variable-type hypothesis or
``type declaration,'' in  analogy to a computer programming language, where a
variable must be declared to be say, a string or an integer.  The requirement
that $\forall e,f\in T \, e_1 \ne f_1$ for $e\neq f$
means that each variable must be
associated with a unique constant designating its variable type; e.g., a
variable might be a ``wff'' or a ``set'' but not both.

Distinct-variable restrictions are used to specify what variable substitutions
are permissible to make for the statement to remain valid.  For example, in
the theorem scheme of set theory $\lnot\forall x\,x=y$ we may not substitute
the same variable for both $x$ and $y$.  On the other hand, the theorem scheme
$x=y\to y=x$ does not require that $x$ and $y$ be distinct, so we do not
require a distinct-variable restriction, although having one
would cause no harm other than making the scheme less general.

A mandatory variable-type hypothesis is one whose variable exists in a logical
hypothesis or the assertion.  A provable pre-statement
(defined below) may require
non-mandatory variable-type hypotheses that effectively introduce ``dummy''
variables for use in its proof.  Any number of dummy variables might
be required by a specific proof; indeed, it has been shown by H.\
Andr\'{e}ka\index{Andr{\'{e}}ka, H.} \cite{Nemeti} that there is no finite
upper bound to the number of dummy variables needed to prove an arbitrary
theorem in first-order logic (with equality) having a fixed number $n>2$ of
individual variables.  (See also the Comment on p.~\pageref{nodd}.)
For this reason we do not set a finite size bound on the collections $D$ and
$T$, although in an actual application (Metamath database) these will of
course be finite, increased to whatever size is necessary as more
proofs are added.
\end{quotation}}

{\footnotesize\begin{quotation}
{\em Relationship to Metamath.} A pre-statement of a formal system
corresponds to an extended frame in a Metamath database
(Section~\ref{frames}).  The collections $D$, $T$, and $H$ correspond
respectively to the \texttt{\$d}, \texttt{\$f}, and \texttt{\$e}
statement collections in an extended frame.  The expression $A$
corresponds to the \texttt{\$a} (or \texttt{\$p}) statement in an
extended frame.

A statement of a formal system corresponds to a frame in a Metamath
database.
\end{quotation}}

\subsection{Formal Systems}

A {\em formal system}\index{formal system} is a
triple $\langle \mbox{\em CN},\mbox{\em
VR},\Gamma\rangle$ where $\Gamma$ is a set of statements.  The members of
$\Gamma$ are called {\em axiomatic statements}.\index{axiomatic
statement!in a formal system}  Sometimes we will refer to a
formal system by just $\Gamma$ when $\mbox{\em CN}$ and $\mbox{\em VR}$ are
understood.

Given a formal system $\Gamma$, the {\em closure}\index{closure}\footnote{This
definition of closure incorporates a simplification due to
Josh Purinton.\index{Purinton, Josh}.} of a
pre-statement
$\langle D,T,H,A \rangle$ is the smallest set $C$ of expressions
such that:
%\begin{enumerate}
%  \item $T\cup H\subseteq C$; and
%  \item If for some axiomatic statement
%    $\langle D_M',T_M',H',A' \rangle \in \Gamma_A$, for
%    some $E \subseteq C$, some $F \subseteq C-T$ (where ``-'' denotes
%    set difference), and some substitution
%    $\sigma$ we have
%    \begin{enumerate}
%       \item $\sigma(T_M') = E$ (where, as above, the $M$ denotes the
%           mandatory variable-type hypotheses of $T^A$);
%       \item $\sigma(H') = F$;
%       \item for all $\{\alpha,\beta\}\in D^A$ and $\subseteq
%         {\cal V}(T_M')$, for all $\gamma\in {\cal V}(\sigma(\langle \alpha
%         \rangle))$, and for all $\delta\in  {\cal V}(\sigma(\langle \beta
%         \rangle))$, we have $\{\gamma, \delta\} \in D$;
%   \end{enumerate}
%   then $\sigma(A') \in C$.
%\end{enumerate}
\begin{list}{}{\itemsep 0.0pt}
  \item[1.] $T\cup H\subseteq C$; and
  \item[2.] If for some axiomatic statement
    $\langle D_M',T_M',H',A' \rangle \in
       \Gamma$ and for some substitution
    $\sigma$ we have
    \begin{enumerate}
       \item[a.] $\sigma(T_M' \cup H') \subseteq C$; and
       \item[b.] for all $\{\alpha,\beta\}\in D_M'$, for all $\gamma\in
         {\cal V}(\sigma(\langle \alpha
         \rangle))$, and for all $\delta\in  {\cal V}(\sigma(\langle \beta
         \rangle))$, we have $\{\gamma, \delta\} \in D$;
   \end{enumerate}
   then $\sigma(A') \in C$.
\end{list}
A pre-statement $\langle D,T,H,A
\rangle$ is {\em provable}\index{provable statement!in a formal
system} if $A\in C$ i.e.\ if its assertion belongs to its
closure.  A statement is {\em provable} if it is
the reduct of a provable pre-statement.
The {\em universe}\index{universe of a formal system}
of a formal system is
the collection of all of its provable statements.  Note that the
set of axiomatic statements $\Gamma$ in a formal system is a subset of its
universe.

{\footnotesize\begin{quotation}
{\em Comment.} The first condition in the definition of closure simply says
that the hypotheses of the pre-statement are in its closure.

Condition 2(a) says that a substitution exists that makes the
mandatory hypotheses of an axiomatic statement exactly match some members of
the closure.  This is what we explicitly demonstrate in a Metamath language
proof.

%Conditions 2(a) and 2(b) say that a substitution exists that makes the
%(mandatory) hypotheses of an axiomatic statement exactly match some members of
%the closure.  This is what we explicitly demonstrate with a Metamath language
%proof.
%
%The set of expressions $F$ in condition 2(b) excludes the variable-type
%hypotheses; this is done because non-mandatory variable-type hypotheses are
%effectively ``dropped'' as irrelevant whereas logical hypotheses must be
%retained to achieve a consistent logical system.

Condition 2(b) describes how distinct-variable restrictions in the axiomatic
statement must be met.  It means that after a substitution for two variables
that must be distinct, the resulting two expressions must either contain no
variables, or if they do, they may not have variables in common, and each pair
of any variables they do have, with one variable from each expression, must be
specified as distinct in the original statement.
\end{quotation}}

{\footnotesize\begin{quotation}
{\em Relationship to Metamath.} Axiomatic statements
 and provable statements in a formal
system correspond to the frames for \texttt{\$a} and \texttt{\$p} statements
respectively in a Metamath database.  The set of axiomatic statements is a
subset of the set of provable statements in a formal system, although in a
Metamath database a \texttt{\$a} statement is distinguished by not having a
proof.  A Metamath language proof for a \texttt{\$p} statement tells the computer
how to explicitly construct a series of members of the closure ultimately
leading to a demonstration that the assertion
being proved is in the closure.  The actual closure typically contains
an infinite number of expressions.  A formal system itself does not have
an explicit object called a ``proof'' but rather the existence of a proof
is implied indirectly by membership of an assertion in a provable
statement's closure.  We do this to make the formal system easier
to describe in the language of set theory.

We also note that once established as provable, a statement may be considered
to acquire the same status as an axiomatic statement, because if the set of
axiomatic statements is extended with a provable statement, the universe of
the formal system remains unchanged (provided that $\mbox{\em VR}$ is
infinite).
In practice, this means we can build a hierarchy of provable statements to
more efficiently establish additional provable statements.  This is
what we do in Metamath when we allow proofs to reference previous
\texttt{\$p} statements as well as previous \texttt{\$a} statements.
\end{quotation}}

\section{Examples of Formal Systems}

{\footnotesize\begin{quotation}
{\em Relationship to Metamath.} The examples in this section, except Example~2,
are for the most part exact equivalents of the development in the set
theory database \texttt{set.mm}.  You may want to compare Examples~1, 3, and 5
to Section~\ref{metaaxioms}, Example 4 to Sections~\ref{metadefprop} and
\ref{metadefpred}, and Example 6 to
Section~\ref{setdefinitions}.\label{exampleref}
\end{quotation}}

\subsection{Example~1---Propositional Calculus}\index{propositional calculus}

Classical propositional calculus can be described by the following formal
system.  We assume the set of variables is infinite.  Rather than denoting the
constants and variables by $c_0, c_1, \ldots$ and $v_0, v_1, \ldots$, for
readability we will instead use more conventional symbols, with the
understanding of course that they denote distinct primitive objects.
Also for readability we may omit commas between successive terms of a
sequence; thus $\langle \mbox{wff\ } \varphi\rangle$ denotes
$\langle \mbox{wff}, \varphi\rangle$.

Let
\begin{itemize}
  \item[] $\mbox{\em CN}=\{\mbox{wff}, \vdash, \to, \lnot, (,)\}$
  \item[] $\mbox{\em VR}=\{\varphi,\psi,\chi,\ldots\}$
  \item[] $T = \{\langle \mbox{wff\ } \varphi\rangle,
             \langle \mbox{wff\ } \psi\rangle,
             \langle \mbox{wff\ } \chi\rangle,\ldots\}$, i.e.\ those
             expressions of length 2 whose first member is $\mbox{\rm wff}$
             and whose second member belongs to $\mbox{\em VR}$.\footnote{For
convenience we let $T$ be an infinite set; the definition of a statement
permits this in principle.  Since a Metamath source file has a finite size, in
practice we must of course use appropriate finite subsets of this $T$,
specifically ones containing at least the mandatory variable-type
hypotheses.  Similarly, in the source file we introduce new variables as
required, with the understanding that a potentially infinite number of
them are available.}
\noindent Then $\Gamma$ consists of the axiomatic statements that
are the reducts of the following pre-statements:
    \begin{itemize}
      \item[] $\langle\varnothing,T,\varnothing,
               \langle \mbox{wff\ }(\varphi\to\psi)\rangle\rangle$
      \item[] $\langle\varnothing,T,\varnothing,
               \langle \mbox{wff\ }\lnot\varphi\rangle\rangle$
      \item[] $\langle\varnothing,T,\varnothing,
               \langle \vdash(\varphi\to(\psi\to\varphi))
               \rangle\rangle$
      \item[] $\langle\varnothing,T,
               \varnothing,
               \langle \vdash((\varphi\to(\psi\to\chi))\to
               ((\varphi\to\psi)\to(\varphi\to\chi)))
               \rangle\rangle$
      \item[] $\langle\varnothing,T,
               \varnothing,
               \langle \vdash((\lnot\varphi\to\lnot\psi)\to
               (\psi\to\varphi))\rangle\rangle$
      \item[] $\langle\varnothing,T,
               \{\langle\vdash(\varphi\to\psi)\rangle,
                 \langle\vdash\varphi\rangle\},
               \langle\vdash\psi\rangle\rangle$
    \end{itemize}
\end{itemize}

(For example, the reduct of $\langle\varnothing,T,\varnothing,
               \langle \mbox{wff\ }(\varphi\to\psi)\rangle\rangle$
is
\begin{itemize}
\item[] $\langle\varnothing,
\{\langle \mbox{wff\ } \varphi\rangle,
             \langle \mbox{wff\ } \psi\rangle\},
             \varnothing,
               \langle \mbox{wff\ }(\varphi\to\psi)\rangle\rangle$,
\end{itemize}
which is the first axiomatic statement.)

We call the members of $\mbox{\em VR}$ {\em wff variables} or (in the context
of first-order logic which we will describe shortly) {\em wff metavariables}.
Note that the symbols $\phi$, $\psi$, etc.\ denote actual specific members of
$\mbox{\em VR}$; they are not metavariables of our expository language (which
we denote with $\alpha$, $\beta$, etc.) but are instead (meta)constant symbols
(members of $\mbox{\em SM}$) from the point of view of our expository
language.  The equivalent system of propositional calculus described in
\cite{Tarski1965} also uses the symbols $\phi$, $\psi$, etc.\ to denote wff
metavariables, but in \cite{Tarski1965} unlike here those are metavariables of
the expository language and not primitive symbols of the formal system.

The first two statements define wffs: if $\varphi$ and $\psi$ are wffs, so is
$(\varphi \to \psi)$; if $\varphi$ is a wff, so is $\lnot\varphi$. The next
three are the axioms of propositional calculus: if $\varphi$ and $\psi$ are
wffs, then $\vdash (\varphi \to (\psi \to \varphi))$ is an (axiomatic)
theorem; etc. The
last is the rule of modus ponens: if $\varphi$ and $\psi$ are wffs, and
$\vdash (\varphi\to\psi)$ and $\vdash \varphi$ are theorems, then $\vdash
\psi$ is a theorem.

The correspondence to ordinary propositional calculus is as follows.  We
consider only provable statements of the form $\langle\varnothing,
T,\varnothing,A\rangle$ with $T$ defined as above.  The first term of the
assertion $A$ of any such statement is either ``wff'' or ``$\vdash$''.  A
statement for which the first term is ``wff'' is a {\em wff} of propositional
calculus, and one where the first term is ``$\vdash$'' is a {\em
theorem (scheme)} of propositional calculus.

The universe of this formal system also contains many other provable
statements.  Those with distinct-variable restrictions are irrelevant because
propositional calculus has no constraints on substitutions.  Those that have
logical hypotheses we call {\em inferences}\index{inference} when
the logical hypotheses are of the form
$\langle\vdash\rangle\frown w$ where $w$ is a wff (with the leading constant
term ``wff'' removed).  Inferences (other than the modus ponens rule) are not a
proper part of propositional calculus but are convenient to use when building a
hierarchy of provable statements.  A provable statement with a nonsense
hypothesis such as $\langle \to,\vdash,\lnot\rangle$, and this same expression
as its assertion, we consider irrelevant; no use can be made of it in
proving theorems, since there is no way to eliminate the nonsense hypothesis.

{\footnotesize\begin{quotation}
{\em Comment.} Our use of parentheses in the definition of a wff illustrates
how axiomatic statements should be carefully stated in a way that
ties in unambiguously with the substitutions allowed by the formal system.
There are many ways we could have defined wffs---for example, Polish
prefix notation would have allowed us to omit parentheses entirely, at
the expense of readability---but we must define them in a way that is
unambiguous.  For example, if we had omitted parentheses from the
definition of $(\varphi\to \psi)$, the wff $\lnot\varphi\to \psi$ could
be interpreted as either $\lnot(\varphi\to\psi)$ or $(\lnot\varphi\to\psi)$
and would have allowed us to prove nonsense.  Note that there is no
concept of operator binding precedence built into our formal system.
\end{quotation}}

\begin{sloppy}
\subsection{Example~2---Predicate Calculus with Equality}\index{predicate
calculus}
\end{sloppy}

Here we extend Example~1 to include predicate calculus with equality,
illustrating the use of distinct-variable restrictions.  This system is the
same as Tarski's system $\mathfrak{S}_2$ in \cite{Tarski1965} (except that the
axioms of propositional calculus are different but equivalent, and a redundant
axiom is omitted).  We extend $\mbox{\em CN}$ with the constants
$\{\mbox{var},\forall,=\}$.  We extend $\mbox{\em VR}$ with an infinite set of
{\em individual metavariables}\index{individual
metavariable} $\{x,y,z,\ldots\}$ and denote this subset
$\mbox{\em Vr}$.

We also join to $\mbox{\em CN}$ a possibly infinite set $\mbox{\em Pr}$ of {\em
predicates} $\{R,S,\ldots\}$.  We associate with $\mbox{\em Pr}$ a function
$\mbox{rnk}$ from $\mbox{\em Pr}$ to $\omega$, and for $\alpha\in \mbox{\em
Pr}$ we call $\mbox{rnk}(\alpha)$ the {\em rank} of the predicate $\alpha$,
which is simply the number of ``arguments'' that the predicate has.  (Most
applications of predicate calculus will have a finite number of predicates;
for example, set theory has the single two-argument or binary predicate $\in$,
which is usually written with its arguments surrounding the predicate symbol
rather than with the prefix notation we will use for the general case.)  As a
device to facilitate our discussion, we will let $\mbox{\em Vs}$ be any fixed
one-to-one function from $\omega$ to $\mbox{\em Vr}$; thus $\mbox{\em Vs}$ is
any simple infinite sequence of individual metavariables with no repeating
terms.

In this example we will not include the function symbols that are often part of
formalizations of predicate calculus.  Using metalogical arguments that are
beyond the scope of our discussion, it can be shown that our formalization is
equivalent when functions are introduced via appropriate definitions.

We extend the set $T$ defined in Example~1 with the expressions
$\{\langle \mbox{var\ } x\rangle,$ $ \langle \mbox{var\ } y\rangle, \langle
\mbox{var\ } z\rangle,\ldots\}$.  We extend the $\Gamma$ above
with the axiomatic statements that are the reducts of the following
pre-statements:
\begin{list}{}{\itemsep 0.0pt}
      \item[] $\langle\varnothing,T,\varnothing,
               \langle \mbox{wff\ }\forall x\,\varphi\rangle\rangle$
      \item[] $\langle\varnothing,T,\varnothing,
               \langle \mbox{wff\ }x=y\rangle\rangle$
      \item[] $\langle\varnothing,T,
               \{\langle\vdash\varphi\rangle\},
               \langle\vdash\forall x\,\varphi\rangle\rangle$
      \item[] $\langle\varnothing,T,\varnothing,
               \langle \vdash((\forall x(\varphi\to\psi)
                  \to(\forall x\,\varphi\to\forall x\,\psi))
               \rangle\rangle$
      \item[] $\langle\{\{x,\varphi\}\},T,\varnothing,
               \langle \vdash(\varphi\to\forall x\,\varphi)
               \rangle\rangle$
      \item[] $\langle\{\{x,y\}\},T,\varnothing,
               \langle \vdash\lnot\forall x\lnot x=y
               \rangle\rangle$
      \item[] $\langle\varnothing,T,\varnothing,
               \langle \vdash(x=z
                  \to(x=y\to z=y))
               \rangle\rangle$
      \item[] $\langle\varnothing,T,\varnothing,
               \langle \vdash(y=z
                  \to(x=y\to x=z))
               \rangle\rangle$
\end{list}
These are the axioms not involving predicate symbols. The first two statements
extend the definition of a wff.  The third is the rule of generalization.  The
fifth states, in effect, ``For a wff $\varphi$ and variable $x$,
$\vdash(\varphi\to\forall x\,\varphi)$, provided that $x$ does not occur in
$\varphi$.''  The sixth states ``For variables $x$ and $y$,
$\vdash\lnot\forall x\lnot x = y$, provided that $x$ and $y$ are distinct.''
(This proviso is not necessary but was included by Tarski to
weaken the axiom and still show that the system is logically complete.)

Finally, for each predicate symbol $\alpha\in \mbox{\em Pr}$, we add to
$\Gamma$ an axiomatic statement, extending the definition of wff,
that is the reduct of the following pre-statement:
\begin{displaymath}
    \langle\varnothing,T,\varnothing,
            \langle \mbox{wff},\alpha\rangle\
            \frown \mbox{\em Vs}\restriction\mbox{rnk}(\alpha)\rangle
\end{displaymath}
and for each $\alpha\in \mbox{\em Pr}$ and each $n < \mbox{rnk}(\alpha)$
we add to $\Gamma$ an equality axiom that is the reduct of the
following pre-statement:
\begin{eqnarray*}
    \lefteqn{\langle\varnothing,T,\varnothing,
            \langle
      \vdash,(,\mbox{\em Vs}_n,=,\mbox{\em Vs}_{\mbox{rnk}(\alpha)},\to,
            (,\alpha\rangle\frown \mbox{\em Vs}\restriction\mbox{rnk}(\alpha)} \\
  & & \frown
            \langle\to,\alpha\rangle\frown \mbox{\em Vs}\restriction n\frown
            \langle \mbox{\em Vs}_{\mbox{rnk}(\alpha)}\rangle \\
 & & \frown
            \mbox{\em Vs}\restriction(\mbox{rnk}(\alpha)\setminus(n+1))\frown
            \langle),)\rangle\rangle
\end{eqnarray*}
where $\restriction$ denotes function domain restriction and $\setminus$
denotes set difference.  Recall that a subscript on $\mbox{\em Vs}$
denotes one of its terms.  (In the above two axiom sets commas are placed
between successive terms of sequences to prevent ambiguity, and if you examine
them with care you will be able to distinguish those parentheses that denote
constant symbols from those of our expository language that delimit function
arguments.  Although it might have been better to use boldface for our
primitive symbols, unfortunately boldface was not available for all characters
on the \LaTeX\ system used to typeset this text.)  These seemingly forbidding
axioms can be understood by analogy to concatenation of substrings in a
computer language.  They are actually relatively simple for each specific case
and will become clearer by looking at the special case of a binary predicate
$\alpha = R$ where $\mbox{rnk}(R)=2$.  Letting $\mbox{\em Vs}$ be the sequence
$\langle x,y,z,\ldots\rangle$, the axioms we would add to $\Gamma$ for this
case would be the wff extension and two equality axioms that are the
reducts of the pre-statements:
\begin{list}{}{\itemsep 0.0pt}
      \item[] $\langle\varnothing,T,\varnothing,
               \langle \mbox{wff\ }R x y\rangle\rangle$
      \item[] $\langle\varnothing,T,\varnothing,
               \langle \vdash(x=z
                  \to(R x y \to R z y))
               \rangle\rangle$
      \item[] $\langle\varnothing,T,\varnothing,
               \langle \vdash(y=z
                  \to(R x y \to R x z))
               \rangle\rangle$
\end{list}
Study these carefully to see how the general axioms above evaluate to
them.  In practice, typically only a few special cases such as this would be
needed, and in any case the Metamath language will only permit us to describe
a finite number of predicates, as opposed to the infinite number permitted by
the formal system.  (If an infinite number should be needed for some reason,
we could not define the formal system directly in the Metamath language but
could instead define it metalogically under set theory as we
do in this appendix, and only the underlying set theory, with its single
binary predicate, would be defined directly in the Metamath language.)


{\footnotesize\begin{quotation}
{\em Comment.}  As we noted earlier, the specific variables denoted by the
symbols $x,y,z,\ldots\in \mbox{\em Vr}\subseteq \mbox{\em VR}\subseteq
\mbox{\em SM}$ in Example~2 are not the actual variables of ordinary predicate
calculus but should be thought of as metavariables ranging over them.  For
example, a distinct-variable restriction would be meaningless for actual
variables of ordinary predicate calculus since two different actual variables
are by definition distinct.  And when we talk about an arbitrary
representative $\alpha\in \mbox{\em Vr}$, $\alpha$ is a metavariable (in our
expository language) that ranges over metavariables (which are primitives of
our formal system) each of which ranges over the actual individual variables
of predicate calculus (which are never mentioned in our formal system).

The constant called ``var'' above is called \texttt{setvar} in the
\texttt{set.mm} database file, but it means the same thing.  I felt
that ``var'' is a more meaningful name in the context of predicate
calculus, whose use is not limited to set theory.  For consistency we
stick with the name ``var'' throughout this Appendix, even after set
theory is introduced.
\end{quotation}}

\subsection{Free Variables and Proper Substitution}\index{free variable}
\index{proper substitution}\index{substitution!proper}

Typical representations of mathematical axioms use concepts such
as ``free variable,'' ``bound variable,'' and ``proper substitution''
as primitive notions.
A free variable is a variable that
is not a parameter of any container expression.
A bound variable is the opposite of a free variable; it is a
a variable that has been bound in a container expression.
For example, in the expression $\forall x \varphi$ (for all $x$, $\varphi$
is true), the variable $x$
is bound within the for-all ($\forall$) expression.
It is possible to change one variable to another, and that process is called
``proper substitution.''
In most books, proper substitution has a somewhat complicated recursive
definition with multiple cases based on the occurrences of free and
bound variables.
You may consult
\cite[ch.\ 3--4]{Hamilton}\index{Hamilton, Alan G.} (as well as
many other texts) for more formal details about these terms.

Using these concepts as \texttt{primitives} creates complications
for computer implementations.

In the system of Example~2, there are no primitive notions of free variable
and proper substitution.  Tarski \cite{Tarski1965} shows that this system is
logically equivalent to the more typical textbook systems that do have these
primitive notions, if we introduce these notions with appropriate definitions
and metalogic.  We could also define axioms for such systems directly,
although the recursive definitions of free variable and proper substitution
would be messy and awkward to work with.  Instead, we mention two devices that
can be used in practice to mimic these notions.  (1) Instead of introducing
special notation to express (as a logical hypothesis) ``where $x$ is not free
in $\varphi$'' we can use the logical hypothesis $\vdash(\varphi\to\forall
x\,\varphi)$.\label{effectivelybound}\index{effectively
not free}\footnote{This is a slightly weaker requirement than ``where $x$ is
not free in $\varphi$.''  If we let $\varphi$ be $x=x$, we have the theorem
$(x=x\to\forall x\,x=x)$ which satisfies the hypothesis, even though $x$ is
free in $x=x$ .  In a case like this we say that $x$ is {\em effectively not
free}\index{effectively not free} in $x=x$, since $x=x$ is logically
equivalent to $\forall x\,x=x$ in which $x$ is bound.} (2) It can be shown
that the wff $((x=y\to\varphi)\wedge\exists x(x=y\wedge\varphi))$ (with the
usual definitions of $\wedge$ and $\exists$; see Example~4 below) is logically
equivalent to ``the wff that results from proper substitution of $y$ for $x$
in $\varphi$.''  This works whether or not $x$ and $y$ are distinct.

\subsection{Metalogical Completeness}\index{metalogical completeness}

In the system of Example~2, the
following are provable pre-statements (and their reducts are
provable statements):
\begin{eqnarray*}
      & \langle\{\{x,y\}\},T,\varnothing,
               \langle \vdash\lnot\forall x\lnot x=y
               \rangle\rangle & \\
     &  \langle\varnothing,T,\varnothing,
               \langle \vdash\lnot\forall x\lnot x=x
               \rangle\rangle &
\end{eqnarray*}
whereas the following pre-statement is not to my knowledge provable (but
in any case we will pretend it's not for sake of illustration):
\begin{eqnarray*}
     &  \langle\varnothing,T,\varnothing,
               \langle \vdash\lnot\forall x\lnot x=y
               \rangle\rangle &
\end{eqnarray*}
In other words, we can prove ``$\lnot\forall x\lnot x=y$ where $x$ and $y$ are
distinct'' and separately prove ``$\lnot\forall x\lnot x=x$'', but we can't
prove the combined general case ``$\lnot\forall x\lnot x=y$'' that has no
proviso.  Now this does not compromise logical completeness, because the
variables are really metavariables and the two provable cases together cover
all possible cases.  The third case can be considered a metatheorem whose
direct proof, using the system of Example~2, lies outside the capability of the
formal system.

Also, in the system of Example~2 the following pre-statement is not to my
knowledge provable (again, a conjecture that we will pretend to be the case):
\begin{eqnarray*}
     & \langle\varnothing,T,\varnothing,
               \langle \vdash(\forall x\, \varphi\to\varphi)
               \rangle\rangle &
\end{eqnarray*}
Instead, we can only prove specific cases of $\varphi$ involving individual
metavariables, and by induction on formula length, prove as a metatheorem
outside of our formal system the general statement above.  The details of this
proof are found in \cite{Kalish}.

There does, however, exist a system of predicate calculus in which all such
``simple metatheorems'' as those above can be proved directly, and we present
it in Example~3. A {\em simple metatheorem}\index{simple metatheorem}
is any statement of the formal
system of Example~2 where all distinct variable restrictions consist of either
two individual metavariables or an individual metavariable and a wff
metavariable, and which is provable by combining cases outside the system as
above.  A system is {\em metalogically complete}\index{metalogical
completeness} if all of its simple
metatheorems are (directly) provable statements. The precise definition of
``simple metatheorem'' and the proof of the ``metalogical completeness'' of
Example~3 is found in Remark 9.6 and Theorem 9.7 of \cite{Megill}.\index{Megill,
Norman}

\begin{sloppy}
\subsection{Example~3---Metalogically Complete Predicate
Calculus with
Equality}
\end{sloppy}

For simplicity we will assume there is one binary predicate $R$;
this system suffices for set theory, where the $R$ is of course the $\in$
predicate.  We label the axioms as they appear in \cite{Megill}.  This
system is logically equivalent to that of Example~2 (when the latter is
restricted to this single binary predicate) but is also metalogically
complete.\index{metalogical completeness}

Let
\begin{itemize}
  \item[] $\mbox{\em CN}=\{\mbox{wff}, \mbox{var}, \vdash, \to, \lnot, (,),\forall,=,R\}$.
  \item[] $\mbox{\em VR}=\{\varphi,\psi,\chi,\ldots\}\cup\{x,y,z,\ldots\}$.
  \item[] $T = \{\langle \mbox{wff\ } \varphi\rangle,
             \langle \mbox{wff\ } \psi\rangle,
             \langle \mbox{wff\ } \chi\rangle,\ldots\}\cup
       \{\langle \mbox{var\ } x\rangle, \langle \mbox{var\ } y\rangle, \langle
       \mbox{var\ }z\rangle,\ldots\}$.

\noindent Then
  $\Gamma$ consists of the reducts of the following pre-statements:
    \begin{itemize}
      \item[] $\langle\varnothing,T,\varnothing,
               \langle \mbox{wff\ }(\varphi\to\psi)\rangle\rangle$
      \item[] $\langle\varnothing,T,\varnothing,
               \langle \mbox{wff\ }\lnot\varphi\rangle\rangle$
      \item[] $\langle\varnothing,T,\varnothing,
               \langle \mbox{wff\ }\forall x\,\varphi\rangle\rangle$
      \item[] $\langle\varnothing,T,\varnothing,
               \langle \mbox{wff\ }x=y\rangle\rangle$
      \item[] $\langle\varnothing,T,\varnothing,
               \langle \mbox{wff\ }Rxy\rangle\rangle$
      \item[(C1$'$)] $\langle\varnothing,T,\varnothing,
               \langle \vdash(\varphi\to(\psi\to\varphi))
               \rangle\rangle$
      \item[(C2$'$)] $\langle\varnothing,T,
               \varnothing,
               \langle \vdash((\varphi\to(\psi\to\chi))\to
               ((\varphi\to\psi)\to(\varphi\to\chi)))
               \rangle\rangle$
      \item[(C3$'$)] $\langle\varnothing,T,
               \varnothing,
               \langle \vdash((\lnot\varphi\to\lnot\psi)\to
               (\psi\to\varphi))\rangle\rangle$
      \item[(C4$'$)] $\langle\varnothing,T,
               \varnothing,
               \langle \vdash(\forall x(\forall x\,\varphi\to\psi)\to
                 (\forall x\,\varphi\to\forall x\,\psi))\rangle\rangle$
      \item[(C5$'$)] $\langle\varnothing,T,
               \varnothing,
               \langle \vdash(\forall x\,\varphi\to\varphi)\rangle\rangle$
      \item[(C6$'$)] $\langle\varnothing,T,
               \varnothing,
               \langle \vdash(\forall x\forall y\,\varphi\to
                 \forall y\forall x\,\varphi)\rangle\rangle$
      \item[(C7$'$)] $\langle\varnothing,T,
               \varnothing,
               \langle \vdash(\lnot\varphi\to\forall x\lnot\forall x\,\varphi
                 )\rangle\rangle$
      \item[(C8$'$)] $\langle\varnothing,T,
               \varnothing,
               \langle \vdash(x=y\to(x=z\to y=z))\rangle\rangle$
      \item[(C9$'$)] $\langle\varnothing,T,
               \varnothing,
               \langle \vdash(\lnot\forall x\, x=y\to(\lnot\forall x\, x=z\to
                 (y=z\to\forall x\, y=z)))\rangle\rangle$
      \item[(C10$'$)] $\langle\varnothing,T,
               \varnothing,
               \langle \vdash(\forall x(x=y\to\forall x\,\varphi)\to
                 \varphi))\rangle\rangle$
      \item[(C11$'$)] $\langle\varnothing,T,
               \varnothing,
               \langle \vdash(\forall x\, x=y\to(\forall x\,\varphi
               \to\forall y\,\varphi))\rangle\rangle$
      \item[(C12$'$)] $\langle\varnothing,T,
               \varnothing,
               \langle \vdash(x=y\to(Rxz\to Ryz))\rangle\rangle$
      \item[(C13$'$)] $\langle\varnothing,T,
               \varnothing,
               \langle \vdash(x=y\to(Rzx\to Rzy))\rangle\rangle$
      \item[(C15$'$)] $\langle\varnothing,T,
               \varnothing,
               \langle \vdash(\lnot\forall x\, x=y\to(x=y\to(\varphi
                 \to\forall x(x=y\to\varphi))))\rangle\rangle$
      \item[(C16$'$)] $\langle\{\{x,y\}\},T,
               \varnothing,
               \langle \vdash(\forall x\, x=y\to(\varphi\to\forall x\,\varphi)
                 )\rangle\rangle$
      \item[(C5)] $\langle\{\{x,\varphi\}\},T,\varnothing,
               \langle \vdash(\varphi\to\forall x\,\varphi)
               \rangle\rangle$
      \item[(MP)] $\langle\varnothing,T,
               \{\langle\vdash(\varphi\to\psi)\rangle,
                 \langle\vdash\varphi\rangle\},
               \langle\vdash\psi\rangle\rangle$
      \item[(Gen)] $\langle\varnothing,T,
               \{\langle\vdash\varphi\rangle\},
               \langle\vdash\forall x\,\varphi\rangle\rangle$
    \end{itemize}
\end{itemize}

While it is known that these axioms are ``metalogically complete,'' it is
not known whether they are independent (i.e.\ none is
redundant) in the metalogical sense; specifically, whether any axiom (possibly
with additional non-mandatory distinct-variable restrictions, for use with any
dummy variables in its proof) is provable from the others.  Note that
metalogical independence is a weaker requirement than independence in the
usual logical sense.  Not all of the above axioms are logically independent:
for example, C9$'$ can be proved as a metatheorem from the others, outside the
formal system, by combining the possible cases of distinct variables.

\subsection{Example~4---Adding Definitions}\index{definition}
There are several ways to add definitions to a formal system.  Probably the
most proper way is to consider definitions not as part of the formal system at
all but rather as abbreviations that are part of the expository metalogic
outside the formal system.  For convenience, though, we may use the formal
system itself to incorporate definitions, adding them as axiomatic extensions
to the system.  This could be done by adding a constant representing the
concept ``is defined as'' along with axioms for it. But there is a nicer way,
at least in this writer's opinion, that introduces definitions as direct
extensions to the language rather than as extralogical primitive notions.  We
introduce additional logical connectives and provide axioms for them.  For
systems of logic such as Examples 1 through 3, the additional axioms must be
conservative in the sense that no wff of the original system that was not a
theorem (when the initial term ``wff'' is replaced by ``$\vdash$'' of course)
becomes a theorem of the extended system.  In this example we extend Example~3
(or 2) with standard abbreviations of logic.

We extend $\mbox{\em CN}$ of Example~3 with new constants $\{\leftrightarrow,
\wedge,\vee,\exists\}$, corresponding to logical equivalence,\index{logical
equivalence ($\leftrightarrow$)}\index{biconditional ($\leftrightarrow$)}
conjunction,\index{conjunction ($\wedge$)} disjunction,\index{disjunction
($\vee$)} and the existential quantifier.\index{existential quantifier
($\exists$)}  We extend $\Gamma$ with the axiomatic statements that are
the reducts of the following pre-statements:
\begin{list}{}{\itemsep 0.0pt}
      \item[] $\langle\varnothing,T,\varnothing,
               \langle \mbox{wff\ }(\varphi\leftrightarrow\psi)\rangle\rangle$
      \item[] $\langle\varnothing,T,\varnothing,
               \langle \mbox{wff\ }(\varphi\vee\psi)\rangle\rangle$
      \item[] $\langle\varnothing,T,\varnothing,
               \langle \mbox{wff\ }(\varphi\wedge\psi)\rangle\rangle$
      \item[] $\langle\varnothing,T,\varnothing,
               \langle \mbox{wff\ }\exists x\, \varphi\rangle\rangle$
  \item[] $\langle\varnothing,T,\varnothing,
     \langle\vdash ( ( \varphi \leftrightarrow \psi ) \to
     ( \varphi \to \psi ) )\rangle\rangle$
  \item[] $\langle\varnothing,T,\varnothing,
     \langle\vdash ((\varphi\leftrightarrow\psi)\to
    (\psi\to\varphi))\rangle\rangle$
  \item[] $\langle\varnothing,T,\varnothing,
     \langle\vdash ((\varphi\to\psi)\to(
     (\psi\to\varphi)\to(\varphi
     \leftrightarrow\psi)))\rangle\rangle$
  \item[] $\langle\varnothing,T,\varnothing,
     \langle\vdash (( \varphi \wedge \psi ) \leftrightarrow\neg ( \varphi
     \to \neg \psi )) \rangle\rangle$
  \item[] $\langle\varnothing,T,\varnothing,
     \langle\vdash (( \varphi \vee \psi ) \leftrightarrow (\neg \varphi
     \to \psi )) \rangle\rangle$
  \item[] $\langle\varnothing,T,\varnothing,
     \langle\vdash (\exists x \,\varphi\leftrightarrow
     \lnot \forall x \lnot \varphi)\rangle\rangle$
\end{list}
The first three logical axioms (statements containing ``$\vdash$'') introduce
and effectively define logical equivalence, ``$\leftrightarrow$''.  The last
three use ``$\leftrightarrow$'' to effectively mean ``is defined as.''

\subsection{Example~5---ZFC Set Theory}\index{ZFC set theory}

Here we add to the system of Example~4 the axioms of Zermelo--Fraenkel set
theory with Choice.  For convenience we make use of the
definitions in Example~4.

In the $\mbox{\em CN}$ of Example~4 (which extends Example~3), we replace the symbol $R$
with the symbol $\in$.
More explicitly, we remove from $\Gamma$ of Example~4 the three
axiomatic statements containing $R$ and replace them with the
reducts of the following:
\begin{list}{}{\itemsep 0.0pt}
      \item[] $\langle\varnothing,T,\varnothing,
               \langle \mbox{wff\ }x\in y\rangle\rangle$
      \item[] $\langle\varnothing,T,
               \varnothing,
               \langle \vdash(x=y\to(x\in z\to y\in z))\rangle\rangle$
      \item[] $\langle\varnothing,T,
               \varnothing,
               \langle \vdash(x=y\to(z\in x\to z\in y))\rangle\rangle$
\end{list}
Letting $D=\{\{\alpha,\beta\}\in \mbox{\em DV}\,|\alpha,\beta\in \mbox{\em
Vr}\}$ (in other words all individual variables must be distinct), we extend
$\Gamma$ with the ZFC axioms, called
\index{Axiom of Extensionality}
\index{Axiom of Replacement}
\index{Axiom of Union}
\index{Axiom of Power Sets}
\index{Axiom of Regularity}
\index{Axiom of Infinity}
\index{Axiom of Choice}
Extensionality, Replacement, Union, Power
Set, Regularity, Infinity, and Choice, that are the reducts of:
\begin{list}{}{\itemsep 0.0pt}
      \item[Ext] $\langle D,T,
               \varnothing,
               \langle\vdash (\forall x(x\in y\leftrightarrow x \in z)\to y
               =z) \rangle\rangle$
      \item[Rep] $\langle D,T,
               \varnothing,
               \langle\vdash\exists x ( \exists y \forall z (\varphi \to z = y
                        ) \to
                        \forall z ( z \in x \leftrightarrow \exists x ( x \in
                        y \wedge \forall y\,\varphi ) ) )\rangle\rangle$
      \item[Un] $\langle D,T,
               \varnothing,
               \langle\vdash \exists x \forall y ( \exists x ( y \in x \wedge
               x \in z ) \to y \in x ) \rangle\rangle$
      \item[Pow] $\langle D,T,
               \varnothing,
               \langle\vdash \exists x \forall y ( \forall x ( x \in y \to x
               \in z ) \to y \in x ) \rangle\rangle$
      \item[Reg] $\langle D,T,
               \varnothing,
               \langle\vdash (  x \in y \to
                 \exists x ( x \in y \wedge \forall z ( z \in x \to \lnot z
                \in y ) ) ) \rangle\rangle$
      \item[Inf] $\langle D,T,
               \varnothing,
               \langle\vdash \exists x(y\in x\wedge\forall y(y\in
               x\to
               \exists z(y \in z\wedge z\in x))) \rangle\rangle$
      \item[AC] $\langle D,T,
               \varnothing,
               \langle\vdash \exists x \forall y \forall z ( ( y \in z
               \wedge z \in w ) \to \exists w \forall y ( \exists w
              ( ( y \in z \wedge z \in w ) \wedge ( y \in w \wedge w \in x
              ) ) \leftrightarrow y = w ) ) \rangle\rangle$
\end{list}

\subsection{Example~6---Class Notation in Set Theory}\label{class}

A powerful device that makes set theory easier (and that we have
been using all along in our informal expository language) is {\em class
abstraction notation}.\index{class abstraction}\index{abstraction class}  The
definitions we introduce are rigorously justified
as conservative by Takeuti and Zaring \cite{Takeuti}\index{Takeuti, Gaisi} or
Quine \cite{Quine}\index{Quine, Willard Van Orman}.  The key idea is to
introduce the notation $\{x|\mbox{---}\}$ which means ``the class of all $x$
such that ---'' for abstraction classes and introduce (meta)variables that
range over them.  An abstraction class may or may not be a set, depending on
whether it exists (as a set).  A class that does not exist is
called a {\em proper class}.\index{proper class}\index{class!proper}

To illustrate the use of abstraction classes we will provide some examples
of definitions that make use of them:  the empty set, class union, and
unordered pair.  Many other such definitions can be found in the
Metamath set theory database,
\texttt{set.mm}.\index{set theory database (\texttt{set.mm})}

% We intentionally break up the sequence of math symbols here
% because otherwise the overlong line goes beyond the page in narrow mode.
We extend $\mbox{\em CN}$ of Example~5 with new symbols $\{$
$\mbox{class},$ $\{,$ $|,$ $\},$ $\varnothing,$ $\cup,$ $,$ $\}$
where the inner braces and last comma are
constant symbols. (As before,
our dual use of some mathematical symbols for both our expository
language and as primitives of the formal system should be clear from context.)

We extend $\mbox{\em VR}$ of Example~5 with a set of {\em class
variables}\index{class variable}
$\{A,B,C,\ldots\}$. We extend the $T$ of Example~5 with $\{\langle
\mbox{class\ } A\rangle, \langle \mbox{class\ }B\rangle, \langle \mbox{class\ }
C\rangle,\ldots\}$.

To
introduce our definitions,
we add to $\Gamma$ of Example~5 the axiomatic statements
that are the reducts of the following pre-statements:
\begin{list}{}{\itemsep 0.0pt}
      \item[] $\langle\varnothing,T,\varnothing,
               \langle \mbox{class\ }x\rangle\rangle$
      \item[] $\langle\varnothing,T,\varnothing,
               \langle \mbox{class\ }\{x|\varphi\}\rangle\rangle$
      \item[] $\langle\varnothing,T,\varnothing,
               \langle \mbox{wff\ }A=B\rangle\rangle$
      \item[] $\langle\varnothing,T,\varnothing,
               \langle \mbox{wff\ }A\in B\rangle\rangle$
      \item[Ab] $\langle\varnothing,T,\varnothing,
               \langle \vdash ( y \in \{ x |\varphi\} \leftrightarrow
                  ( ( x = y \to\varphi) \wedge \exists x ( x = y
                  \wedge\varphi) ))
               \rangle\rangle$
      \item[Eq] $\langle\{\{x,A\},\{x,B\}\},T,\varnothing,
               \langle \vdash ( A = B \leftrightarrow
               \forall x ( x \in A \leftrightarrow x \in B ) )
               \rangle\rangle$
      \item[El] $\langle\{\{x,A\},\{x,B\}\},T,\varnothing,
               \langle \vdash ( A \in B \leftrightarrow \exists x
               ( x = A \wedge x \in B ) )
               \rangle\rangle$
\end{list}
Here we say that an individual variable is a class; $\{x|\varphi\}$ is a
class; and we extend the definition of a wff to include class equality and
membership.  Axiom Ab defines membership of a variable in a class abstraction;
the right-hand side can be read as ``the wff that results from proper
substitution of $y$ for $x$ in $\varphi$.''\footnote{Note that this definition
makes unnecessary the introduction of a separate notation similar to
$\varphi(x|y)$ for proper substitution, although we may choose to do so to be
conventional.  Incidentally, $\varphi(x|y)$ as it stands would be ambiguous in
the formal systems of our examples, since we wouldn't know whether
$\lnot\varphi(x|y)$ meant $\lnot(\varphi(x|y))$ or $(\lnot\varphi)(x|y)$.
Instead, we would have to use an unambiguous variant such as $(\varphi\,
x|y)$.}  Axioms Eq and El extend the meaning of the existing equality and
membership connectives.  This is potentially dangerous and requires careful
justification.  For example, from Eq we can derive the Axiom of Extensionality
with predicate logic alone; thus in principle we should include the Axiom of
Extensionality as a logical hypothesis.  However we do not bother to do this
since we have already presupposed that axiom earlier. The distinct variable
restrictions should be read ``where $x$ does not occur in $A$ or $B$.''  We
typically do this when the right-hand side of a definition involves an
individual variable not in the expression being defined; it is done so that
the right-hand side remains independent of the particular ``dummy'' variable
we use.

We continue to add to $\Gamma$ the following definitions
(i.e. the reducts of the following pre-statements) for empty
set,\index{empty set} class union,\index{union} and unordered
pair.\index{unordered pair}  They should be self-explanatory.  Analogous to our
use of ``$\leftrightarrow$'' to define new wffs in Example~4, we use ``$=$''
to define new abstraction terms, and both may be read informally as ``is
defined as'' in this context.
\begin{list}{}{\itemsep 0.0pt}
      \item[] $\langle\varnothing,T,\varnothing,
               \langle \mbox{class\ }\varnothing\rangle\rangle$
      \item[] $\langle\varnothing,T,\varnothing,
               \langle \vdash \varnothing = \{ x | \lnot x = x \}
               \rangle\rangle$
      \item[] $\langle\varnothing,T,\varnothing,
               \langle \mbox{class\ }(A\cup B)\rangle\rangle$
      \item[] $\langle\{\{x,A\},\{x,B\}\},T,\varnothing,
               \langle \vdash ( A \cup B ) = \{ x | ( x \in A \vee x \in B ) \}
               \rangle\rangle$
      \item[] $\langle\varnothing,T,\varnothing,
               \langle \mbox{class\ }\{A,B\}\rangle\rangle$
      \item[] $\langle\{\{x,A\},\{x,B\}\},T,\varnothing,
               \langle \vdash \{ A , B \} = \{ x | ( x = A \vee x = B ) \}
               \rangle\rangle$
\end{list}

\section{Metamath as a Formal System}\label{theorymm}

This section presupposes a familiarity with the Metamath computer language.

Our theory describes formal systems and their universes.  The Metamath
language provides a way of representing these set-theoretical objects to
a computer.  A Metamath database, being a finite set of {\sc ascii}
characters, can usually describe only a subset of a formal system and
its universe, which are typically infinite.  However the database can
contain as large a finite subset of the formal system and its universe
as we wish.  (Of course a Metamath set theory database can, in
principle, indirectly describe an entire infinite formal system by
formalizing the expository language in this Appendix.)

For purpose of our discussion, we assume the Metamath database
is in the simple form described on p.~\pageref{framelist},
consisting of all constant and variable declarations at the beginning,
followed by a sequence of extended frames each
delimited by \texttt{\$\char`\{} and \texttt{\$\char`\}}.  Any Metamath database can
be converted to this form, as described on p.~\pageref{frameconvert}.

The math symbol tokens of a Metamath source file, which are declared
with \texttt{\$c} and \texttt{\$v} statements, are names we assign to
representatives of $\mbox{\em CN}$ and $\mbox{\em VR}$.  For
definiteness we could assume that the first math symbol declared as a
variable corresponds to $v_0$, the second to $v_1$, etc., although the
exact correspondence we choose is not important.

In the Metamath language, each \texttt{\$d}, \texttt{\$f}, and
 \texttt{\$e} source
statement in an extended frame (Section~\ref{frames})
corresponds respectively to a member of the
collections $D$, $T$, and $H$ in a formal system statement $\langle
D_M,T_M,H,A\rangle$.  The math symbol strings following these Metamath keywords
correspond to a variable pair (in the case of \texttt{\$d}) or an expression (for
the other two keywords). The math symbol string following a \texttt{\$a} source
statement corresponds to expression $A$ in an axiomatic statement of the
formal system; the one following a \texttt{\$p} source statement corresponds to
$A$ in a provable statement that is not axiomatic.  In other words, each
extended frame in a Metamath database corresponds to
a pre-statement of the formal system, and a frame corresponds to
a statement of the formal system.  (Don't confuse the two meanings of
``statement'' here.  A statement of the formal system corresponds to the
several statements in a Metamath database that may constitute a
frame.)

In order for the computer to verify that a formal system statement is
provable, each \texttt{\$p} source statement is accompanied by a proof.
However, the proof does not correspond to anything in the formal system
but is simply a way of communicating to the computer the information
needed for its verification.  The proof tells the computer {\em how to
construct} specific members of closure of the formal system
pre-statement corresponding to the extended frame of the \texttt{\$p}
statement.  The final result of the construction is the member of the
closure that matches the \texttt{\$p} statement.  The abstract formal
system, on the other hand, is concerned only with the {\em existence} of
members of the closure.

As mentioned on p.~\pageref{exampleref}, Examples 1 and 3--6 in the
previous Section parallel the development of logic and set theory in the
Metamath database
\texttt{set.mm}.\index{set theory database (\texttt{set.mm})} You may
find it instructive to compare them.


\chapter{The MIU System}
\label{MIU}
\index{formal system}
\index{MIU-system}

The following is a listing of the file \texttt{miu.mm}.  It is self-explanatory.

%%%%%%%%%%%%%%%%%%%%%%%%%%%%%%%%%%%%%%%%%%%%%%%%%%%%%%%%%%%%

\begin{verbatim}
$( The MIU-system:  A simple formal system $)

$( Note:  This formal system is unusual in that it allows
empty wffs.  To work with a proof, you must type
SET EMPTY_SUBSTITUTION ON before using the PROVE command.
By default, this is OFF in order to reduce the number of
ambiguous unification possibilities that have to be selected
during the construction of a proof.  $)

$(
Hofstadter's MIU-system is a simple example of a formal
system that illustrates some concepts of Metamath.  See
Douglas R. Hofstadter, _Goedel, Escher, Bach:  An Eternal
Golden Braid_ (Vintage Books, New York, 1979), pp. 33ff. for
a description of the MIU-system.

The system has 3 constant symbols, M, I, and U.  The sole
axiom of the system is MI. There are 4 rules:
     Rule I:  If you possess a string whose last letter is I,
     you can add on a U at the end.
     Rule II:  Suppose you have Mx.  Then you may add Mxx to
     your collection.
     Rule III:  If III occurs in one of the strings in your
     collection, you may make a new string with U in place
     of III.
     Rule IV:  If UU occurs inside one of your strings, you
     can drop it.
Unfortunately, Rules III and IV do not have unique results:
strings could have more than one occurrence of III or UU.
This requires that we introduce the concept of an "MIU
well-formed formula" or wff, which allows us to construct
unique symbol sequences to which Rules III and IV can be
applied.
$)

$( First, we declare the constant symbols of the language.
Note that we need two symbols to distinguish the assertion
that a sequence is a wff from the assertion that it is a
theorem; we have arbitrarily chosen "wff" and "|-". $)
      $c M I U |- wff $. $( Declare constants $)

$( Next, we declare some variables. $)
     $v x y $.

$( Throughout our theory, we shall assume that these
variables represent wffs. $)
 wx   $f wff x $.
 wy   $f wff y $.

$( Define MIU-wffs.  We allow the empty sequence to be a
wff. $)

$( The empty sequence is a wff. $)
 we   $a wff $.
$( "M" after any wff is a wff. $)
 wM   $a wff x M $.
$( "I" after any wff is a wff. $)
 wI   $a wff x I $.
$( "U" after any wff is a wff. $)
 wU   $a wff x U $.

$( Assert the axiom. $)
 ax   $a |- M I $.

$( Assert the rules. $)
 ${
   Ia   $e |- x I $.
$( Given any theorem ending with "I", it remains a theorem
if "U" is added after it.  (We distinguish the label I_
from the math symbol I to conform to the 24-Jun-2006
Metamath spec.) $)
   I_    $a |- x I U $.
 $}
 ${
IIa  $e |- M x $.
$( Given any theorem starting with "M", it remains a theorem
if the part after the "M" is added again after it. $)
   II   $a |- M x x $.
 $}
 ${
   IIIa $e |- x I I I y $.
$( Given any theorem with "III" in the middle, it remains a
theorem if the "III" is replaced with "U". $)
   III  $a |- x U y $.
 $}
 ${
   IVa  $e |- x U U y $.
$( Given any theorem with "UU" in the middle, it remains a
theorem if the "UU" is deleted. $)
   IV   $a |- x y $.
  $}

$( Now we prove the theorem MUIIU.  You may be interested in
comparing this proof with that of Hofstadter (pp. 35 - 36).
$)
 theorem1  $p |- M U I I U $=
      we wM wU wI we wI wU we wU wI wU we wM we wI wU we wM
      wI wI wI we wI wI we wI ax II II I_ III II IV $.
\end{verbatim}\index{well-formed formula (wff)}

The \texttt{show proof /lemmon/renumber} command
yields the following display.  It is very similar
to the one in \cite[pp.~35--36]{Hofstadter}.\index{Hofstadter, Douglas R.}

\begin{verbatim}
1 ax             $a |- M I
2 1 II           $a |- M I I
3 2 II           $a |- M I I I I
4 3 I_           $a |- M I I I I U
5 4 III          $a |- M U I U
6 5 II           $a |- M U I U U I U
7 6 IV           $a |- M U I I U
\end{verbatim}

We note that Hofstadter's ``MU-puzzle,'' which asks whether
MU is a theorem of the MIU-system, cannot be answered using
the system above because the MU-puzzle is a question {\em
about} the system.  To prove the answer to the MU-puzzle,
a much more elaborate system is needed, namely one that
models the MIU-system within set theory.  (Incidentally, the
answer to the MU-puzzle is no.)

\chapter{Metamath Language EBNF}%
\label{BNF}%
\index{Metamath Language EBNF}

The following is a formal description of the basic Metamath language syntax
(with compressed proofs and support for unknown proof steps).
It is defined using the
Extended Backus--Naur Form (EBNF)\index{Extended Backus--Naur Form}\index{EBNF}
notation from W3C\index{W3C}
\textit{Extensible Markup Language (XML) 1.0 (Fifth Edition)}
(W3C Recommendation 26 November 2008) at
\url{https://www.w3.org/TR/xml/#sec-notation}.

The \texttt{database}
rule is processed until the end of the file (\texttt{EOF}).
The rules eventually require reading whitespace-separated tokens.
A token has an upper-case definition (see below)
or is a string constant in a non-token (such as \texttt{'\$a'}).
We intend for this to be correct, but if there is a conflict the
rules of section \ref{spec} govern. That section also discusses
non-syntax restrictions not shown here
(e.g., that each new label token
defined in a \texttt{hypothesis-stmt} or \texttt{assert-stmt}
must be unique).

\begin{verbatim}
database ::= outermost-scope-stmt*

outermost-scope-stmt ::=
  include-stmt | constant-stmt | stmt

/* File inclusion command; process file as a database.
   Databases should NOT have a comment in the filename. */
include-stmt ::= '$[' filename '$]'

/* Constant symbols declaration. */
constant-stmt ::= '$c' constant+ '$.'

/* A normal statement can occur in any scope. */
stmt ::= block | variable-stmt | disjoint-stmt |
  hypothesis-stmt | assert-stmt

/* A block. You can have 0 statements in a block. */
block ::= '${' stmt* '$}'

/* Variable symbols declaration. */
variable-stmt ::= '$v' variable+ '$.'

/* Disjoint variables. Simple disjoint statements have
   2 variables, i.e., "variable*" is empty for them. */
disjoint-stmt ::= '$d' variable variable variable* '$.'

hypothesis-stmt ::= floating-stmt | essential-stmt

/* Floating (variable-type) hypothesis. */
floating-stmt ::= LABEL '$f' typecode variable '$.'

/* Essential (logical) hypothesis. */
essential-stmt ::= LABEL '$e' typecode MATH-SYMBOL* '$.'

assert-stmt ::= axiom-stmt | provable-stmt

/* Axiomatic assertion. */
axiom-stmt ::= LABEL '$a' typecode MATH-SYMBOL* '$.'

/* Provable assertion. */
provable-stmt ::= LABEL '$p' typecode MATH-SYMBOL*
  '$=' proof '$.'

/* A proof. Proofs may be interspersed by comments.
   If '?' is in a proof it's an "incomplete" proof. */
proof ::= uncompressed-proof | compressed-proof
uncompressed-proof ::= (LABEL | '?')+
compressed-proof ::= '(' LABEL* ')' COMPRESSED-PROOF-BLOCK+

typecode ::= constant

filename ::= MATH-SYMBOL /* No whitespace or '$' */
constant ::= MATH-SYMBOL
variable ::= MATH-SYMBOL
\end{verbatim}

\needspace{2\baselineskip}
A \texttt{frame} is a sequence of 0 or more
\texttt{disjoint-{\allowbreak}stmt} and
\texttt{hypotheses-{\allowbreak}stmt} statements
(possibly interleaved with other non-\texttt{assert-stmt} statements)
followed by one \texttt{assert-stmt}.

\needspace{3\baselineskip}
Here are the rules for lexical processing (tokenization) beyond
the constant tokens shown above.
By convention these tokenization rules have upper-case names.
Every token is read for the longest possible length.
Whitespace-separated tokens are read sequentially;
note that the separating whitespace and \texttt{\$(} ... \texttt{\$)}
comments are skipped.

If a token definition uses another token definition, the whole thing
is considered a single token.
A pattern that is only part of a full token has a name beginning
with an underscore (``\_'').
An implementation could tokenize many tokens as a
\texttt{PRINTABLE-SEQUENCE}
and then check if it meets the more specific rule shown here.

Comments do not nest, and both \texttt{\$(} and \texttt{\$)}
have to be surrounded
by at least one whitespace character (\texttt{\_WHITECHAR}).
Technically comments end without consuming the trailing
\texttt{\_WHITECHAR}, but the trailing
\texttt{\_WHITECHAR} gets ignored anyway so we ignore that detail here.
Metamath language processors
are not required to support \texttt{\$)} followed
immediately by a bare end-of-file, because the closing
comment symbol is supposed to be followed by a
\texttt{\_WHITECHAR} such as a newline.

\begin{verbatim}
PRINTABLE-SEQUENCE ::= _PRINTABLE-CHARACTER+

MATH-SYMBOL ::= (_PRINTABLE-CHARACTER - '$')+

/* ASCII non-whitespace printable characters */
_PRINTABLE-CHARACTER ::= [#x21-#x7e]

LABEL ::= ( _LETTER-OR-DIGIT | '.' | '-' | '_' )+

_LETTER-OR-DIGIT ::= [A-Za-z0-9]

COMPRESSED-PROOF-BLOCK ::= ([A-Z] | '?')+

/* Define whitespace between tokens. The -> SKIP
   means that when whitespace is seen, it is
   skipped and we simply read again. */
WHITESPACE ::= (_WHITECHAR+ | _COMMENT) -> SKIP

/* Comments. $( ... $) and do not nest. */
_COMMENT ::= '$(' (_WHITECHAR+ (PRINTABLE-SEQUENCE - '$)')*
  _WHITECHAR+ '$)' _WHITECHAR

/* Whitespace: (' ' | '\t' | '\r' | '\n' | '\f') */
_WHITECHAR ::= [#x20#x09#x0d#x0a#x0c]
\end{verbatim}
% This EBNF was developed as a collaboration between
% David A. Wheeler\index{Wheeler, David A.},
% Mario Carneiro\index{Carneiro, Mario}, and
% Benoit Jubin\index{Jubin, Benoit}, inspired by a request
% (and a lot of initial work) by Benoit Jubin.
%
% \chapter{Disclaimer and Trademarks}
%
% Information in this document is subject to change without notice and does not
% represent a commitment on the part of Norman Megill.
% \vspace{2ex}
%
% \noindent Norman D. Megill makes no warranties, either express or implied,
% regarding the Metamath computer software package.
%
% \vspace{2ex}
%
% \noindent Any trademarks mentioned in this book are the property of
% their respective owners.  The name ``Metamath'' is a trademark of
% Norman Megill.
%
\cleardoublepage
\phantomsection  % fixes the link anchor
\addcontentsline{toc}{chapter}{\bibname}

\bibliography{metamath}
%\input{metamath.bbl}

\raggedright
\cleardoublepage
\phantomsection % fixes the link anchor
\addcontentsline{toc}{chapter}{\indexname}
%\printindex   ??
\input{metamath.ind}

\end{document}



\end{document}



\end{document}



\raggedright
\cleardoublepage
\phantomsection % fixes the link anchor
\addcontentsline{toc}{chapter}{\indexname}
%\printindex   ??
% metamath.tex - Version of 24-May-2019
% If you change the date above, also change the "Printed date" below.
% SPDX-License-Identifier: CC0-1.0
%
%                              PUBLIC DOMAIN
%
% This file (specifically, the version of this file with the above date)
% has been released into the Public Domain per the
% Creative Commons CC0 1.0 Universal (CC0 1.0) Public Domain Dedication
% https://creativecommons.org/publicdomain/zero/1.0/
%
% The public domain release applies worldwide.  In case this is not
% legally possible, the right is granted to use the work for any purpose,
% without any conditions, unless such conditions are required by law.
%
% Several short, attributed quotations from copyrighted works
% appear in this file under the ``fair use'' provision of Section 107 of
% the United States Copyright Act (Title 17 of the {\em United States
% Code}).  The public-domain status of this file is not applicable to
% those quotations.
%
% Norman Megill - email: nm(at)alum(dot)mit(dot)edu
%
% David A. Wheeler also donates his improvements to this file to the
% public domain per the CC0.  He works at the Institute for Defense Analyses
% (IDA), but IDA has agreed that this Metamath work is outside its "lane"
% and is not a work by IDA.  This was specifically confirmed by
% Margaret E. Myers (Division Director of the Information Technology
% and Systems Division) on 2019-05-24 and by Ben Lindorf (General Counsel)
% on 2019-05-22.

% This file, 'metamath.tex', is self-contained with everything needed to
% generate the the PDF file 'metamath.pdf' (the _Metamath_ book) on
% standard LaTeX 2e installations.  The auxiliary files are embedded with
% "filecontents" commands.  To generate metamath.pdf file, run these
% commands under Linux or Cygwin in the directory that contains
% 'metamath.tex':
%
%   rm -f realref.sty metamath.bib
%   touch metamath.ind
%   pdflatex metamath
%   pdflatex metamath
%   bibtex metamath
%   makeindex metamath
%   pdflatex metamath
%   pdflatex metamath
%
% The warnings that occur in the initial runs of pdflatex can be ignored.
% For the final run,
%
%   egrep -i 'error|warn' metamath.log
%
% should show exactly these 5 warnings:
%
%   LaTeX Warning: File `realref.sty' already exists on the system.
%   LaTeX Warning: File `metamath.bib' already exists on the system.
%   LaTeX Font Warning: Font shape `OMS/cmtt/m/n' undefined
%   LaTeX Font Warning: Font shape `OMS/cmtt/bx/n' undefined
%   LaTeX Font Warning: Some font shapes were not available, defaults
%       substituted.
%
% Search for "Uncomment" below if you want to suppress hyperlink boxes
% in the PDF output file
%
% TYPOGRAPHICAL NOTES:
% * It is customary to use an en dash (--) to "connect" names of different
%   people (and to denote ranges), and use a hyphen (-) for a
%   single compound name. Examples of connected multiple people are
%   Zermelo--Fraenkel, Schr\"{o}der--Bernstein, Tarski--Grothendieck,
%   Hewlett--Packard, and Backus--Naur.  Examples of a single person with
%   a compound name include Levi-Civita, Mittag-Leffler, and Burali-Forti.
% * Use non-breaking spaces after page abbreviations, e.g.,
%   p.~\pageref{note2002}.
%
% --------------------------- Start of realref.sty -----------------------------
\begin{filecontents}{realref.sty}
% Save the following as realref.sty.
% You can then use it with \usepackage{realref}
%
% This has \pageref jumping to the page on which the ref appears,
% \ref jumping to the point of the anchor, and \sectionref
% jumping to the start of section.
%
% Author:  Anthony Williams
%          Software Engineer
%          Nortel Networks Optical Components Ltd
% Date:    9 Nov 2001 (posted to comp.text.tex)
%
% The following declaration was made by Anthony Williams on
% 24 Jul 2006 (private email to Norman Megill):
%
%   ``I hereby donate the code for realref.sty posted on the
%   comp.text.tex newsgroup on 9th November 2001, accessible from
%   http://groups.google.com/group/comp.text.tex/msg/5a0e1cc13ea7fbb2
%   to the public domain.''
%
\ProvidesPackage{realref}
\RequirePackage[plainpages=false,pdfpagelabels=true]{hyperref}
\def\realref@anchorname{}
\AtBeginDocument{%
% ensure every label is a possible hyperlink target
\let\realref@oldrefstepcounter\refstepcounter%
\DeclareRobustCommand{\refstepcounter}[1]{\realref@oldrefstepcounter{#1}
\edef\realref@anchorname{\string #1.\@currentlabel}%
}%
\let\realref@oldlabel\label%
\DeclareRobustCommand{\label}[1]{\realref@oldlabel{#1}\hypertarget{#1}{}%
\@bsphack\protected@write\@auxout{}{%
    \string\expandafter\gdef\protect\csname
    page@num.#1\string\endcsname{\thepage}%
    \string\expandafter\gdef\protect\csname
    ref@num.#1\string\endcsname{\@currentlabel}%
    \string\expandafter\gdef\protect\csname
    sectionref@name.#1\string\endcsname{\realref@anchorname}%
}\@esphack}%
\DeclareRobustCommand\pageref[1]{{\edef\a{\csname
            page@num.#1\endcsname}\expandafter\hyperlink{page.\a}{\a}}}%
\DeclareRobustCommand\ref[1]{{\edef\a{\csname
            ref@num.#1\endcsname}\hyperlink{#1}{\a}}}%
\DeclareRobustCommand\sectionref[1]{{\edef\a{\csname
            ref@num.#1\endcsname}\edef\b{\csname
            sectionref@name.#1\endcsname}\hyperlink{\b}{\a}}}%
}
\end{filecontents}
% ---------------------------- End of realref.sty ------------------------------

% --------------------------- Start of metamath.bib -----------------------------
\begin{filecontents}{metamath.bib}
@book{Albers, editor = "Donald J. Albers and G. L. Alexanderson",
  title = "Mathematical People",
  publisher = "Contemporary Books, Inc.",
  address = "Chicago",
  note = "[QA28.M37]",
  year = 1985 }
@book{Anderson, author = "Alan Ross Anderson and Nuel D. Belnap",
  title = "Entailment",
  publisher = "Princeton University Press",
  address = "Princeton",
  volume = 1,
  note = "[QA9.A634 1975 v.1]",
  year = 1975}
@book{Barrow, author = "John D. Barrow",
  title = "Theories of Everything:  The Quest for Ultimate Explanation",
  publisher = "Oxford University Press",
  address = "Oxford",
  note = "[Q175.B225]",
  year = 1991 }
@book{Behnke,
  editor = "H. Behnke and F. Backmann and K. Fladt and W. S{\"{u}}ss",
  title = "Fundamentals of Mathematics",
  volume = "I",
  publisher = "The MIT Press",
  address = "Cambridge, Massachusetts",
  note = "[QA37.2.B413]",
  year = 1974 }
@book{Bell, author = "J. L. Bell and M. Machover",
  title = "A Course in Mathematical Logic",
  publisher = "North-Holland",
  address = "Amsterdam",
  note = "[QA9.B3953]",
  year = 1977 }
@inproceedings{Blass, author = "Andrea Blass",
  title = "The Interaction Between Category Theory and Set Theory",
  pages = "5--29",
  booktitle = "Mathematical Applications of Category Theory (Proceedings
     of the Special Session on Mathematical Applications
     Category Theory, 89th Annual Meeting of the American Mathematical
     Society, held in Denver, Colorado January 5--9, 1983)",
  editor = "John Walter Gray",
  year = 1983,
  note = "[QA169.A47 1983]",
  publisher = "American Mathematical Society",
  address = "Providence, Rhode Island"}
@proceedings{Bledsoe, editor = "W. W. Bledsoe and D. W. Loveland",
  title = "Automated Theorem Proving:  After 25 Years (Proceedings
     of the Special Session on Automatic Theorem Proving,
     89th Annual Meeting of the American Mathematical
     Society, held in Denver, Colorado January 5--9, 1983)",
  year = 1983,
  note = "[QA76.9.A96.S64 1983]",
  publisher = "American Mathematical Society",
  address = "Providence, Rhode Island" }
@book{Boolos, author = "George S. Boolos and Richard C. Jeffrey",
  title = "Computability and Log\-ic",
  publisher = "Cambridge University Press",
  edition = "third",
  address = "Cambridge",
  note = "[QA9.59.B66 1989]",
  year = 1989 }
@book{Campbell, author = "John Campbell",
  title = "Programmer's Progress",
  publisher = "White Star Software",
  address = "Box 51623, Palo Alto, CA 94303",
  year = 1991 }
@article{DBLP:journals/corr/Carneiro14,
  author    = {Mario Carneiro},
  title     = {Conversion of {HOL} Light proofs into Metamath},
  journal   = {CoRR},
  volume    = {abs/1412.8091},
  year      = {2014},
  url       = {http://arxiv.org/abs/1412.8091},
  archivePrefix = {arXiv},
  eprint    = {1412.8091},
  timestamp = {Mon, 13 Aug 2018 16:47:05 +0200},
  biburl    = {https://dblp.org/rec/bib/journals/corr/Carneiro14},
  bibsource = {dblp computer science bibliography, https://dblp.org}
}
@article{CarneiroND,
  author    = {Mario Carneiro},
  title     = {Natural Deductions in the Metamath Proof Language},
  url       = {http://us.metamath.org/ocat/natded.pdf},
  year      = 2014
}
@inproceedings{Chou, author = "Shang-Ching Chou",
  title = "Proving Elementary Geometry Theorems Using {W}u's Algorithm",
  pages = "243--286",
  booktitle = "Automated Theorem Proving:  After 25 Years (Proceedings
     of the Special Session on Automatic Theorem Proving,
     89th Annual Meeting of the American Mathematical
     Society, held in Denver, Colorado January 5--9, 1983)",
  editor = "W. W. Bledsoe and D. W. Loveland",
  year = 1983,
  note = "[QA76.9.A96.S64 1983]",
  publisher = "American Mathematical Society",
  address = "Providence, Rhode Island" }
@book{Clemente, author = "Daniel Clemente Laboreo",
  title = "Introduction to natural deduction",
  year = 2014,
  url = "http://www.danielclemente.com/logica/dn.en.pdf" }
@incollection{Courant, author = "Richard Courant and Herbert Robbins",
  title = "Topology",
  pages = "573--590",
  booktitle = "The World of Mathematics, Volume One",
  editor = "James R. Newman",
  publisher = "Simon and Schuster",
  address = "New York",
  note = "[QA3.W67 1988]",
  year = 1956 }
@book{Curry, author = "Haskell B. Curry",
  title = "Foundations of Mathematical Logic",
  publisher = "Dover Publications, Inc.",
  address = "New York",
  note = "[QA9.C976 1977]",
  year = 1977 }
@book{Davis, author = "Philip J. Davis and Reuben Hersh",
  title = "The Mathematical Experience",
  publisher = "Birkh{\"{a}}user Boston",
  address = "Boston",
  note = "[QA8.4.D37 1982]",
  year = 1981 }
@incollection{deMillo,
  author = "Richard de Millo and Richard Lipton and Alan Perlis",
  title = "Social Processes and Proofs of Theorems and Programs",
  pages = "267--285",
  booktitle = "New Directions in the Philosophy of Mathematics",
  editor = "Thomas Tymoczko",
  publisher = "Birkh{\"{a}}user Boston, Inc.",
  address = "Boston",
  note = "[QA8.6.N48 1986]",
  year = 1986 }
@book{Edwards, author = "Robert E. Edwards",
  title = "A Formal Background to Mathematics",
  publisher = "Springer-Verlag",
  address = "New York",
  note = "[QA37.2.E38 v.1a]",
  year = 1979 }
@book{Enderton, author = "Herbert B. Enderton",
  title = "Elements of Set Theory",
  publisher = "Academic Press, Inc.",
  address = "San Diego",
  note = "[QA248.E5]",
  year = 1977 }
@book{Goodstein, author = "R. L. Goodstein",
  title = "Development of Mathematical Logic",
  publisher = "Springer-Verlag New York Inc.",
  address = "New York",
  note = "[QA9.G6554]",
  year = 1971 }
@book{Guillen, author = "Michael Guillen",
  title = "Bridges to Infinity",
  publisher = "Jeremy P. Tarcher, Inc.",
  address = "Los Angeles",
  note = "[QA93.G8]",
  year = 1983 }
@book{Hamilton, author = "Alan G. Hamilton",
  title = "Logic for Mathematicians",
  edition = "revised",
  publisher = "Cambridge University Press",
  address = "Cambridge",
  note = "[QA9.H298]",
  year = 1988 }
@unpublished{Harrison, author = "John Robert Harrison",
  title = "Metatheory and Reflection in Theorem Proving:
    A Survey and Critique",
  note = "Technical Report
    CRC-053.
    SRI Cambridge,
    Millers Yard, Cambridge, UK,
    1995.
    Available on the Web as
{\verb+http:+}\-{\verb+//www.cl.cam.ac.uk/users/jrh/papers/reflect.html+}"}
@TECHREPORT{Harrison-thesis,
        author          = "John Robert Harrison",
        title           = "Theorem Proving with the Real Numbers",
        institution   = "University of Cambridge Computer
                         Lab\-o\-ra\-to\-ry",
        address         = "New Museums Site, Pembroke Street, Cambridge,
                           CB2 3QG, UK",
        year            = 1996,
        number          = 408,
        type            = "Technical Report",
        note            = "Author's PhD thesis,
   available on the Web at
{\verb+http:+}\-{\verb+//www.cl.cam.ac.uk+}\-{\verb+/users+}\-{\verb+/jrh+}%
\-{\verb+/papers+}\-{\verb+/thesis.html+}"}
@book{Herrlich, author = "Horst Herrlich and George E. Strecker",
  title = "Category Theory:  An Introduction",
  publisher = "Allyn and Bacon Inc.",
  address = "Boston",
  note = "[QA169.H567]",
  year = 1973 }
@article{Hindley, author = "J. Roger Hindley and David Meredith",
  title = "Principal Type-Schemes and Condensed Detachment",
  journal = "The Journal of Symbolic Logic",
  volume = 55,
  year = 1990,
  note = "[QA.J87]",
  pages = "90--105" }
@book{Hofstadter, author = "Douglas R. Hofstadter",
  title = "G{\"{o}}del, Escher, Bach",
  publisher = "Basic Books, Inc.",
  address = "New York",
  note = "[QA9.H63 1980]",
  year = 1979 }
@article{Indrzejczak, author= "Andrzej Indrzejczak",
  title = "Natural Deduction, Hybrid Systems and Modal Logic",
  journal = "Trends in Logic",
  volume = 30,
  publisher = "Springer",
  year = 2010 }
@article{Kalish, author = "D. Kalish and R. Montague",
  title = "On {T}arski's Formalization of Predicate Logic with Identity",
  journal = "Archiv f{\"{u}}r Mathematische Logik und Grundlagenfor\-schung",
  volume = 7,
  year = 1965,
  note = "[QA.A673]",
  pages = "81--101" }
@article{Kalman, author = "J. A. Kalman",
  title = "Condensed Detachment as a Rule of Inference",
  journal = "Studia Logica",
  volume = 42,
  number = 4,
  year = 1983,
  note = "[B18.P6.S933]",
  pages = "443-451" }
@book{Kline, author = "Morris Kline",
  title = "Mathematical Thought from Ancient to Modern Times",
  publisher = "Oxford University Press",
  address = "New York",
  note = "[QA21.K516 1990 v.3]",
  year = 1972 }
@book{Klinel, author = "Morris Kline",
  title = "Mathematics, The Loss of Certainty",
  publisher = "Oxford University Press",
  address = "New York",
  note = "[QA21.K525]",
  year = 1980 }
@book{Kramer, author = "Edna E. Kramer",
  title = "The Nature and Growth of Modern Mathematics",
  publisher = "Princeton University Press",
  address = "Princeton, New Jersey",
  note = "[QA93.K89 1981]",
  year = 1981 }
@article{Knill, author = "Oliver Knill",
  title = "Some Fundamental Theorems in Mathematics",
  year = "2018",
  url = "https://arxiv.org/abs/1807.08416" }
@book{Landau, author = "Edmund Landau",
  title = "Foundations of Analysis",
  publisher = "Chelsea Publishing Company",
  address = "New York",
  edition = "second",
  note = "[QA241.L2541 1960]",
  year = 1960 }
@article{Leblanc, author = "Hugues Leblanc",
  title = "On {M}eyer and {L}ambert's Quantificational Calculus {FQ}",
  journal = "The Journal of Symbolic Logic",
  volume = 33,
  year = 1968,
  note = "[QA.J87]",
  pages = "275--280" }
@article{Lejewski, author = "Czeslaw Lejewski",
  title = "On Implicational Definitions",
  journal = "Studia Logica",
  volume = 8,
  year = 1958,
  note = "[B18.P6.S933]",
  pages = "189--208" }
@book{Levy, author = "Azriel Levy",
  title = "Basic Set Theory",
  publisher = "Dover Publications",
  address = "Mineola, NY",
  year = "2002"
}
@book{Margaris, author = "Angelo Margaris",
  title = "First Order Mathematical Logic",
  publisher = "Blaisdell Publishing Company",
  address = "Waltham, Massachusetts",
  note = "[QA9.M327]",
  year = 1967}
@book{Manin, author = "Yu I. Manin",
  title = "A Course in Mathematical Logic",
  publisher = "Springer-Verlag",
  address = "New York",
  note = "[QA9.M29613]",
  year = "1977" }
@article{Mathias, author = "Adrian R. D. Mathias",
  title = "A Term of Length 4,523,659,424,929",
  journal = "Synthese",
  volume = 133,
  year = 2002,
  note = "[Q.S993]",
  pages = "75--86" }
@article{Megill, author = "Norman D. Megill",
  title = "A Finitely Axiomatized Formalization of Predicate Calculus
     with Equality",
  journal = "Notre Dame Journal of Formal Logic",
  volume = 36,
  year = 1995,
  note = "[QA.N914]",
  pages = "435--453" }
@unpublished{Megillc, author = "Norman D. Megill",
  title = "A Shorter Equivalent of the Axiom of Choice",
  month = "June",
  note = "Unpublished",
  year = 1991 }
@article{MegillBunder, author = "Norman D. Megill and Martin W.
    Bunder",
  title = "Weaker {D}-Complete Logics",
  journal = "Journal of the IGPL",
  volume = 4,
  year = 1996,
  pages = "215--225",
  note = "Available on the Web at
{\verb+http:+}\-{\verb+//www.mpi-sb.mpg.de+}\-{\verb+/igpl+}%
\-{\verb+/Journal+}\-{\verb+/V4-2+}\-{\verb+/#Megill+}"}
}
@book{Mendelson, author = "Elliott Mendelson",
  title = "Introduction to Mathematical Logic",
  edition = "second",
  publisher = "D. Van Nostrand Company, Inc.",
  address = "New York",
  note = "[QA9.M537 1979]",
  year = 1979 }
@article{Meredith, author = "David Meredith",
  title = "In Memoriam {C}arew {A}rthur {M}eredith (1904-1976)",
  journal = "Notre Dame Journal of Formal Logic",
  volume = 18,
  year = 1977,
  note = "[QA.N914]",
  pages = "513--516" }
@article{CAMeredith, author = "C. A. Meredith",
  title = "Single Axioms for the Systems ({C},{N}), ({C},{O}) and ({A},{N})
      of the Two-Valued Propositional Calculus",
  journal = "The Journal of Computing Systems",
  volume = 3,
  year = 1953,
  pages = "155--164" }
@article{Monk, author = "J. Donald Monk",
  title = "Provability With Finitely Many Variables",
  journal = "The Journal of Symbolic Logic",
  volume = 27,
  year = 1971,
  note = "[QA.J87]",
  pages = "353--358" }
@article{Monks, author = "J. Donald Monk",
  title = "Substitutionless Predicate Logic With Identity",
  journal = "Archiv f{\"{u}}r Mathematische Logik und Grundlagenfor\-schung",
  volume = 7,
  year = 1965,
  pages = "103--121" }
  %% Took out this from above to prevent LaTeX underfull warning:
  % note = "[QA.A673]",
@book{Moore, author = "A. W. Moore",
  title = "The Infinite",
  publisher = "Routledge",
  address = "New York",
  note = "[BD411.M59]",
  year = 1989}
@book{Munkres, author = "James R. Munkres",
  title = "Topology: A First Course",
  publisher = "Prentice-Hall, Inc.",
  address = "Englewood Cliffs, New Jersey",
  note = "[QA611.M82]",
  year = 1975}
@article{Nemesszeghy, author = "E. Z. Nemesszeghy and E. A. Nemesszeghy",
  title = "On Strongly Creative Definitions:  A Reply to {V}. {F}. {R}ickey",
  journal = "Logique et Analyse (N.\ S.)",
  year = 1977,
  volume = 20,
  note = "[BC.L832]",
  pages = "111--115" }
@unpublished{Nemeti, author = "N{\'{e}}meti, I.",
  title = "Algebraizations of Quantifier Logics, an Overview",
  note = "Version 11.4, preprint, Mathematical Institute, Budapest,
    1994.  A shortened version without proofs appeared in
    ``Algebraizations of quantifier logics, an introductory overview,''
   {\em Studia Logica}, 50:485--569, 1991 [B18.P6.S933]"}
@article{Pavicic, author = "M. Pavi{\v{c}}i{\'{c}}",
  title = "A New Axiomatization of Unified Quantum Logic",
  journal = "International Journal of Theoretical Physics",
  year = 1992,
  volume = 31,
  note = "[QC.I626]",
  pages = "1753 --1766" }
@book{Penrose, author = "Roger Penrose",
  title = "The Emperor's New Mind",
  publisher = "Oxford University Press",
  address = "New York",
  note = "[Q335.P415]",
  year = 1989 }
@book{PetersonI, author = "Ivars Peterson",
  title = "The Mathematical Tourist",
  publisher = "W. H. Freeman and Company",
  address = "New York",
  note = "[QA93.P475]",
  year = 1988 }
@article{Peterson, author = "Jeremy George Peterson",
  title = "An automatic theorem prover for substitution and detachment systems",
  journal = "Notre Dame Journal of Formal Logic",
  volume = 19,
  year = 1978,
  note = "[QA.N914]",
  pages = "119--122" }
@book{Quine, author = "Willard Van Orman Quine",
  title = "Set Theory and Its Logic",
  edition = "revised",
  publisher = "The Belknap Press of Harvard University Press",
  address = "Cambridge, Massachusetts",
  note = "[QA248.Q7 1969]",
  year = 1969 }
@article{Robinson, author = "J. A. Robinson",
  title = "A Machine-Oriented Logic Based on the Resolution Principle",
  journal = "Journal of the Association for Computing Machinery",
  year = 1965,
  volume = 12,
  pages = "23--41" }
@article{RobinsonT, author = "T. Thacher Robinson",
  title = "Independence of Two Nice Sets of Axioms for the Propositional
    Calculus",
  journal = "The Journal of Symbolic Logic",
  volume = 33,
  year = 1968,
  note = "[QA.J87]",
  pages = "265--270" }
@book{Rucker, author = "Rudy Rucker",
  title = "Infinity and the Mind:  The Science and Philosophy of the
    Infinite",
  publisher = "Bantam Books, Inc.",
  address = "New York",
  note = "[QA9.R79 1982]",
  year = 1982 }
@book{Russell, author = "Bertrand Russell",
  title = "Mysticism and Logic, and Other Essays",
  publisher = "Barnes \& Noble Books",
  address = "Totowa, New Jersey",
  note = "[B1649.R963.M9 1981]",
  year = 1981 }
@article{Russell2, author = "Bertrand Russell",
  title = "Recent Work on the Principles of Mathematics",
  journal = "International Monthly",
  volume = 4,
  year = 1901,
  pages = "84"}
@article{Schmidt, author = "Eric Schmidt",
  title = "Reductions in Norman Megill's axiom system for complex numbers",
  url = "http://us.metamath.org/downloads/schmidt-cnaxioms.pdf",
  year = "2012" }
@book{Shoenfield, author = "Joseph R. Shoenfield",
  title = "Mathematical Logic",
  publisher = "Addison-Wesley Publishing Company, Inc.",
  address = "Reading, Massachusetts",
  year = 1967,
  note = "[QA9.S52]" }
@book{Smullyan, author = "Raymond M. Smullyan",
  title = "Theory of Formal Systems",
  publisher = "Princeton University Press",
  address = "Princeton, New Jersey",
  year = 1961,
  note = "[QA248.5.S55]" }
@book{Solow, author = "Daniel Solow",
  title = "How to Read and Do Proofs:  An Introduction to Mathematical
    Thought Process",
  publisher = "John Wiley \& Sons",
  address = "New York",
  year = 1982,
  note = "[QA9.S577]" }
@book{Stark, author = "Harold M. Stark",
  title = "An Introduction to Number Theory",
  publisher = "Markham Publishing Company",
  address = "Chicago",
  note = "[QA241.S72 1978]",
  year = 1970 }
@article{Swart, author = "E. R. Swart",
  title = "The Philosophical Implications of the Four-Color Problem",
  journal = "American Mathematical Monthly",
  year = 1980,
  volume = 87,
  month = "November",
  note = "[QA.A5125]",
  pages = "697--707" }
@book{Szpiro, author = "George G. Szpiro",
  title = "Poincar{\'{e}}'s Prize: The Hundred-Year Quest to Solve One
    of Math's Greatest Puzzles",
  publisher = "Penguin Books Ltd",
  address = "London",
  note = "[QA43.S985 2007]",
  year = 2007}
@book{Takeuti, author = "Gaisi Takeuti and Wilson M. Zaring",
  title = "Introduction to Axiomatic Set Theory",
  edition = "second",
  publisher = "Springer-Verlag New York Inc.",
  address = "New York",
  note = "[QA248.T136 1982]",
  year = 1982}
@inproceedings{Tarski, author = "Alfred Tarski",
  title = "What is Elementary Geometry",
  pages = "16--29",
  booktitle = "The Axiomatic Method, with Special Reference to Geometry and
     Physics (Proceedings of an International Symposium held at the University
     of California, Berkeley, December 26, 1957 --- January 4, 1958)",
  editor = "Leon Henkin and Patrick Suppes and Alfred Tarski",
  year = 1959,
  publisher = "North-Holland Publishing Company",
  address = "Amsterdam"}
@article{Tarski1965, author = "Alfred Tarski",
  title = "A Simplified Formalization of Predicate Logic with Identity",
  journal = "Archiv f{\"{u}}r Mathematische Logik und Grundlagenforschung",
  volume = 7,
  year = 1965,
  note = "[QA.A673]",
  pages = "61--79" }
@book{Tymoczko,
  title = "New Directions in the Philosophy of Mathematics",
  editor = "Thomas Tymoczko",
  publisher = "Birkh{\"{a}}user Boston, Inc.",
  address = "Boston",
  note = "[QA8.6.N48 1986]",
  year = 1986 }
@incollection{Wang,
  author = "Hao Wang",
  title = "Theory and Practice in Mathematics",
  pages = "129--152",
  booktitle = "New Directions in the Philosophy of Mathematics",
  editor = "Thomas Tymoczko",
  publisher = "Birkh{\"{a}}user Boston, Inc.",
  address = "Boston",
  note = "[QA8.6.N48 1986]",
  year = 1986 }
@manual{Webster,
  title = "Webster's New Collegiate Dictionary",
  organization = "G. \& C. Merriam Co.",
  address = "Springfield, Massachusetts",
  note = "[PE1628.W4M4 1977]",
  year = 1977 }
@manual{Whitehead, author = "Alfred North Whitehead",
  title = "An Introduction to Mathematics",
  year = 1911 }
@book{PM, author = "Alfred North Whitehead and Bertrand Russell",
  title = "Principia Mathematica",
  edition = "second",
  publisher = "Cambridge University Press",
  address = "Cambridge",
  year = "1927",
  note = "(3 vols.) [QA9.W592 1927]" }
@article{DBLP:journals/corr/Whalen16,
  author    = {Daniel Whalen},
  title     = {Holophrasm: a neural Automated Theorem Prover for higher-order logic},
  journal   = {CoRR},
  volume    = {abs/1608.02644},
  year      = {2016},
  url       = {http://arxiv.org/abs/1608.02644},
  archivePrefix = {arXiv},
  eprint    = {1608.02644},
  timestamp = {Mon, 13 Aug 2018 16:46:19 +0200},
  biburl    = {https://dblp.org/rec/bib/journals/corr/Whalen16},
  bibsource = {dblp computer science bibliography, https://dblp.org} }
@article{Wiedijk-revisited,
  author = {Freek Wiedijk},
  title = {The QED Manifesto Revisited},
  year = {2007},
  url = {http://mizar.org/trybulec65/8.pdf} }
@book{Wolfram,
  author = "Stephen Wolfram",
  title = "Mathematica:  A System for Doing Mathematics by Computer",
  edition = "second",
  publisher = "Addison-Wesley Publishing Co.",
  address = "Redwood City, California",
  note = "[QA76.95.W65 1991]",
  year = 1991 }
@book{Wos, author = "Larry Wos and Ross Overbeek and Ewing Lusk and Jim Boyle",
  title = "Automated Reasoning:  Introduction and Applications",
  edition = "second",
  publisher = "McGraw-Hill, Inc.",
  address = "New York",
  note = "[QA76.9.A96.A93 1992]",
  year = 1992 }

%
%
%[1] Church, Alonzo, Introduction to Mathematical Logic,
% Volume 1, Princeton University Press, Princeton, N. J., 1956.
%
%[2] Cohen, Paul J., Set Theory and the Continuum Hypothesis,
% W. A. Benjamin, Inc., Reading, Mass., 1966.
%
%[3] Hamilton, Alan G., Logic for Mathematicians, Cambridge
% University Press,
% Cambridge, 1988.

%[6] Kleene, Stephen Cole, Introduction to Metamathematics, D.  Van
% Nostrand Company, Inc., Princeton (1952).

%[13] Tarski, Alfred, "A simplified formalization of predicate
% logic with identity," Archiv fur Mathematische Logik und
% Grundlagenforschung, vol. 7 (1965), pp. 61-79.

%[14] Tarski, Alfred and Steven Givant, A Formalization of Set
% Theory Without Variables, American Mathematical Society Colloquium
% Publications, vol. 41, American Mathematical Society,
% Providence, R. I., 1987.

%[15] Zeman, J. J., Modal Logic, Oxford University Press, Oxford, 1973.
\end{filecontents}
% --------------------------- End of metamath.bib -----------------------------


%Book: Metamath
%Author:  Norman Megill Email:  nm at alum.mit.edu
%Author:  David A. Wheeler Email:  dwheeler at dwheeler.com

% A book template example
% http://www.stsci.edu/ftp/software/tex/bookstuff/book.template

\documentclass[leqno]{book} % LaTeX 2e. 10pt. Use [leqno,12pt] for 12pt
% hyperref 2002/05/27 v6.72r  (couldn't get pagebackref to work)
\usepackage[plainpages=false,pdfpagelabels=true]{hyperref}

\usepackage{needspace}     % Enable control over page breaks
\usepackage{breqn}         % automatic equation breaking
\usepackage{microtype}     % microtypography, reduces hyphenation

% Packages for flexible tables.  We need to be able to
% wrap text within a cell (with automatically-determined widths) AND
% split a table automatically across multiple pages.
% * "tabularx" wraps text in cells but only 1 page
% * "longtable" goes across pages but by itself is incompatible with tabularx
% * "ltxtable" combines longtable and tabularx, but table contents
%    must be in a separate file.
% * "ltablex" combines tabularx and longtable - must install specially
% * "booktabs" is recommended as a way to improve the look of tables,
%   but doesn't add these capabilities.
% * "tabu" much more capable and seems to be recommended. So use that.

\usepackage{makecell}      % Enable forced line splits within a table cell
\usepackage{longtable}     % Enable multi-page tables
\usepackage{tabu}          % Multi-page tables with wrapped text in a cell

% You can find more Tex packages using commands like:
% tlmgr search --file tabu.sty
% find /usr/share/texmf-dist/ -name '*tab*'
%
%%%%%%%%%%%%%%%%%%%%%%%%%%%%%%%%%%%%%%%%%%%%%%%%%%%%%%%%%%%%%%%%%%%%%%%%%%%%
% Uncomment the next 3 lines to suppress boxes and colors on the hyperlinks
%%%%%%%%%%%%%%%%%%%%%%%%%%%%%%%%%%%%%%%%%%%%%%%%%%%%%%%%%%%%%%%%%%%%%%%%%%%%
%\hypersetup{
%colorlinks,citecolor=black,filecolor=black,linkcolor=black,urlcolor=black
%}
%
\usepackage{realref}

% Restarting page numbers: try?
%   \printglossary
%   \cleardoublepage
%   \pagenumbering{arabic}
%   \setcounter{page}{1}    ???needed
%   \include{chap1}

% not used:
% \def\R2Lurl#1#2{\mbox{\href{#1}\texttt{#2}}}

\usepackage{amssymb}

% Version 1 of book: margins: t=.4, b=.2, ll=.4, rr=.55
% \usepackage{anysize}
% % \papersize{<height>}{<width>}
% % \marginsize{<left>}{<right>}{<top>}{<bottom>}
% \papersize{9in}{6in}
% % l/r 0.6124-0.6170 works t/b 0.2418-0.3411 = 192pp. 0.2926-03118=exact
% \marginsize{0.7147in}{0.5147in}{0.4012in}{0.2012in}

\usepackage{anysize}
% \papersize{<height>}{<width>}
% \marginsize{<left>}{<right>}{<top>}{<bottom>}
\papersize{9in}{6in}
% l/r 0.85in&0.6431-0.6539 works t/b ?-?
%\marginsize{0.85in}{0.6485in}{0.55in}{0.35in}
\marginsize{0.8in}{0.65in}{0.5in}{0.3in}

% \usepackage[papersize={3.6in,4.8in},hmargin=0.1in,vmargin={0.1in,0.1in}]{geometry}  % page geometry
\usepackage{special-settings}

\raggedbottom
\makeindex

\begin{document}
% Discourage page widows and orphans:
\clubpenalty=300
\widowpenalty=300

%%%%%%% load in AMS fonts %%%%%%% % LaTeX 2.09 - obsolete in LaTeX 2e
%\input{amssym.def}
%\input{amssym.tex}
%\input{c:/texmf/tex/plain/amsfonts/amssym.def}
%\input{c:/texmf/tex/plain/amsfonts/amssym.tex}

\bibliographystyle{plain}
\pagenumbering{roman}
\pagestyle{headings}

\thispagestyle{empty}

\hfill
\vfill

\begin{center}
{\LARGE\bf Metamath} \\
\vspace{1ex}
{\large A Computer Language for Mathematical Proofs} \\
\vspace{7ex}
{\large Norman Megill} \\
\vspace{7ex}
with extensive revisions by \\
\vspace{1ex}
{\large David A. Wheeler} \\
\vspace{7ex}
% Printed date. If changing the date below, also fix the date at the beginning.
2019-05-24
\end{center}

\vfill
\hfill

\newpage
\thispagestyle{empty}

\hfill
\vfill

\begin{center}
$\sim$\ {\sc Public Domain}\ $\sim$

\vspace{2ex}
This book (including its later revisions)
has been released into the Public Domain by Norman Megill per the
Creative Commons CC0 1.0 Universal (CC0 1.0) Public Domain Dedication
(\url{https://creativecommons.org/publicdomain/zero/1.0/}).
David A. Wheeler has done the same.
The public domain release applies worldwide.  In case this is not
legally possible, the right is granted to use the work for any purpose,
without any conditions, unless such conditions are required by law.

\vspace{3ex}
Several short, attributed quotations from copyrighted works
appear in this book under the ``fair use'' provision of Section 107 of
the United States Copyright Act (Title 17 of the {\em United States
Code}).  The public-domain status of this book is not applicable to
those quotations.

\vspace{3ex}
Any trademarks used in this book are the property of their owners.

% QA76.9.L63.M??

% \vspace{1ex}
%
% \vspace{1ex}
% {\small Permission is granted to make and distribute verbatim copies of this
% book
% provided the copyright notice and this
% permission notice are preserved on all copies.}
%
% \vspace{1ex}
% {\small Permission is granted to copy and distribute modified versions of this
% book under the conditions for verbatim copying, provided that the
% entire
% resulting derived work is distributed under the terms of a permission
% notice
% identical to this one.}
%
% \vspace{1ex}
% {\small Permission is granted to copy and distribute translations of this
% book into another language, under the above conditions for modified
% versions,
% except that this permission notice may be stated in a translation
% approved by the
% author.}
%
% \vspace{1ex}
% %{\small   For a copy of the \LaTeX\ source files for this book, contact
% %the author.} \\
% \ \\
% \ \\

\vspace{7ex}
% ISBN: 1-4116-3724-0 \\
ISBN: 978-1-4116-3724-5 \\
{\ } \\
Lulu Press \\
Morrisville, North Carolina\\
USA


\hfill
\vfill

Norman Megill\\ 93 Bridge St., Lexington, MA 02421 \\
E-mail address: \texttt{nm{\char`\@}alum.mit.edu} \\
\vspace{7ex}
David A. Wheeler \\
E-mail address: \texttt{dwheeler{\char`\@}dwheeler.com} \\
% See notes added at end of Preface for revision history. \\
% For current information on the Metamath software see \\
\vspace{7ex}
\url{http://metamath.org}
\end{center}

% \newpage
% \thispagestyle{empty}
%
% \hfill
% \vfill
%
% \begin{center}
% {\it To my son Robin Dwight Megill}
% \end{center}
%
% \vfill
% \hfill
%
% \newpage

\tableofcontents
%\listoftables

\chapter*{Preface}
\markboth{PREFACE}{PREFACE}
\addcontentsline{toc}{section}{Preface}


% (For current information, see the notes added at the
% end of this preface on p.~\pageref{note2002}.)

\subsubsection{Overview}

Metamath\index{Metamath} is a computer language and an associated computer
program for archiving, verifying, and studying mathematical proofs at a very
detailed level.  The Metamath language incorporates no mathematics per se but
treats all mathematical statements as mere sequences of symbols.  You provide
Metamath with certain special sequences (axioms) that tell it what rules
of inference are allowed.  Metamath is not limited to any specific field of
mathematics.  The Metamath language is simple and robust, with an
almost total absence of hard-wired syntax, and
I\footnote{Unless otherwise noted, the words
``I,'' ``me,'' and ``my'' refer to Norman Megill\index{Megill, Norman}, while
``we,'' ``us,'' and ``our'' refer to Norman Megill and
David A. Wheeler\index{Wheeler, David A.}.}
believe that it
provides about the simplest possible framework that allows essentially all of
mathematics to be expressed with absolute rigor.

% index test
%\newcommand{\nn}[1]{#1n}
%\index{aaa@bbb}
%\index{abc!def}
%\index{abd|see{qqq}}
%\index{abe|nn}
%\index{abf|emph}
%\index{abg|(}
%\index{abg|)}

Using the Metamath language, you can build formal or mathematical
systems\index{formal system}\footnote{A formal or mathematical system consists
of a collection of symbols (such as $2$, $4$, $+$ and $=$), syntax rules that
describe how symbols may be combined to form a legal expression (called a
well-formed formula or {\em wff}, pronounced ``whiff''), some starting wffs
called axioms, and inference rules that describe how theorems may be derived
(proved) from the axioms.  A theorem is a mathematical fact such as $2+2=4$.
Strictly speaking, even an obvious fact such as this must be proved from
axioms to be formally acceptable to a mathematician.}\index{theorem}
\index{axiom}\index{rule}\index{well-formed formula (wff)} that involve
inferences from axioms.  Although a database is provided
that includes a recommended set of axioms for standard mathematics, if you
wish you can supply your own symbols, syntax, axioms, rules, and definitions.

The name ``Metamath'' was chosen to suggest that the language provides a
means for {\em describing} mathematics rather than {\em being} the
mathematics itself.  Actually in some sense any mathematical language is
metamathematical.  Symbols written on paper, or stored in a computer,
are not mathematics itself but rather a way of expressing mathematics.
For example ``7'' and ``VII'' are symbols for denoting the number seven
in Arabic and Roman numerals; neither {\em is} the number seven.

If you are able to understand and write computer programs, you should be able
to follow abstract mathematics with the aid of Metamath.  Used in conjunction
with standard textbooks, Metamath can guide you step-by-step towards an
understanding of abstract mathematics from a very rigorous viewpoint, even if
you have no formal abstract mathematics background.  By using a single,
consistent notation to express proofs, once you grasp its basic concepts
Metamath provides you with the ability to immediately follow and dissect
proofs even in totally unfamiliar areas.

Of course, just being able follow a proof will not necessarily give you an
intuitive familiarity with mathematics.  Memorizing the rules of chess does not
give you the ability to appreciate the game of a master, and knowing how the
notes on a musical score map to piano keys does not give you the ability to
hear in your head how it would sound.  But each of these can be a first step.

Metamath allows you to explore proofs in the sense that you can see the
theorem referenced at any step expanded in as much detail as you want, right
down to the underlying axioms of logic and set theory (in the case of the set
theory database provided).  While Metamath will not replace the higher-level
understanding that can only be acquired through exercises and hard work, being
able to see how gaps in a proof are filled in can give you increased
confidence that can speed up the learning process and save you time when you
get stuck.

The Metamath language breaks down a mathematical proof into its tiniest
possible parts.  These can be pieced together, like interlocking
pieces in a puzzle, only in a way that produces correct and absolutely rigorous
mathematics.

The nature of Metamath\index{Metamath} enforces very precise mathematical
thinking, similar to that involved in writing a computer program.  A crucial
difference, though, is that once a proof is verified (by the Metamath program)
to be correct, it is definitely correct; it can never have a hidden
``bug.''\index{computer program bugs}  After getting used to the kind of rigor
and accuracy provided by Metamath, you might even be tempted to
adopt the attitude that a proof should never be considered correct until it
has been verified by a computer, just as you would not completely trust a
manual calculation until you have verified it on a
calculator.

My goal
for Metamath was a system for describing and verifying
mathematics that is completely universal yet conceptually as simple as
possible.  In approaching mathematics from an axiomatic, formal viewpoint, I
wanted Metamath to be able to handle almost any mathematical system, not
necessarily with ease, but at least in principle and hopefully in practice. I
wanted it to verify proofs with absolute rigor, and for this reason Metamath
is what might be thought of as a ``compile-only'' language rather than an
algorithmic or Turing-machine language (Pascal, C, Prolog, Mathematica,
etc.).  In other words, a database written in the Metamath
language doesn't ``do'' anything; it merely exhibits mathematical knowledge
and permits this knowledge to be verified as being correct.  A program in an
algorithmic language can potentially have hidden bugs\index{computer program
bugs} as well as possibly being hard to understand.  But each token in a
Metamath database must be consistent with the database's earlier
contents according to simple, fixed rules.
If a database is verified
to be correct,\footnote{This includes
verification that a sequential list of proof steps results in the specified
theorem.} then the mathematical content is correct if the
verifier is correct and the axioms are correct.
The verification program could be incorrect, but the verification algorithm
is relatively simple (making it unlikely to be implemented incorrectly
by the Metamath program),
and there are over a dozen Metamath database verifiers
written by different people in different programming languages
(so these different verifiers can act as multiple reviewers of a database).
The most-used Metamath database, the Metamath Proof Explorer
(aka \texttt{set.mm}\index{set theory database (\texttt{set.mm})}%
\index{Metamath Proof Explorer}),
is currently verified by four different Metamath verifiers written by
four different people in four different languages, including the
original Metamath program described in this book.
The only ``bugs'' that can exist are in the statement of the axioms,
for example if the axioms are inconsistent (a famous problem shown to be
unsolvable by G\"{o}del's incompleteness theorem\index{G\"{o}del's
incompleteness theorem}).
However, real mathematical systems have very few axioms, and these can
be carefully studied.
All of this provides extraordinarily high confidence that the verified database
is in fact correct.

The Metamath program
doesn't prove theorems automatically but is designed to verify proofs
that you supply to it.
The underlying Metamath language is completely general and has no built-in,
preconceived notions about your formal system\index{formal system}, its logic
or its syntax.
For constructing proofs, the Metamath program has a Proof Assistant\index{Proof
Assistant} which helps you fill in some of a proof step's details, shows you
what choices you have at any step, and verifies the proof as you build it; but
you are still expected to provide the proof.

There are many other programs that can process or generate information
in the Metamath language, and more continue to be written.
This is in part because the Metamath language itself is very simple
and intentionally easy to automatically process.
Some programs, such as \texttt{mmj2}\index{mmj2}, include a proof assistant
that can automate some steps beyond what the Metamath program can do.
Mario Carneiro has developed an algorithm for converting proofs from
the OpenTheory interchange format, which can be translated to and from
any of the HOL family of proof languages (HOL4, HOL Light, ProofPower,
and Isabelle), into the
Metamath language.\cite{DBLP:journals/corr/Carneiro14}\index{Carneiro, Mario}.
Daniel Whalen has developed Holophrasm, which can automatically
prove many Metamath proofs using
machine learning\index{machine learning}\index{artificial intelligence}
approaches
(including multiple neural networks\index{neural networks})\cite{DBLP:journals/corr/Whalen16}\index{Whalen, Daniel}.
However,
a discussion of these other programs is beyond the scope of this book.

Like most computer languages, the Metamath\index{Metamath} language uses the
standard ({\sc ascii}) characters on a computer keyboard, so it cannot
directly represent many of the special symbols that mathematicians use.  A
useful feature of the Metamath program is its ability to convert its notation
into the \LaTeX\ typesetting language.\index{latex@{\LaTeX}}  This feature
lets you convert the {\sc ascii} tokens you've defined into standard
mathematical symbols, so you end up with symbols and formulas you are familiar
with instead of somewhat cryptic {\sc ascii} representations of them.
The Metamath program can also generate HTML\index{HTML}, making it easy
to view results on the web and to see related information by using
hypertext links.

Metamath is probably conceptually different from anything you've seen
before and some aspects may take some getting used to.  This book will
help you decide whether Metamath suits your specific needs.

\subsubsection{Setting Your Expectations}
It is important for you to understand what Metamath\index{Metamath} is and is
not.  As mentioned, the Metamath program
is {\em not} an automated theorem prover but
rather a proof verifier.  Developing a database can be tedious, hard work,
especially if you want to make the proofs as short as possible, but it becomes
easier as you build up a collection of useful theorems.  The purpose of
Metamath is simply to document existing mathematics in an absolutely rigorous,
computer-verifiable way, not to aid directly in the creation of new
mathematics.  It also is not a magic solution for learning abstract
mathematics, although it may be helpful to be able to actually see the implied
rigor behind what you are learning from textbooks, as well as providing hints
to work out proofs that you are stumped on.

As of this writing, a sizable set theory database has been developed to
provide a foundation for many fields of mathematics, but much more work would
be required to develop useful databases for specific fields.

Metamath\index{Metamath} ``knows no math;'' it just provides a framework in
which to express mathematics.  Its language is very small.  You can define two
kinds of symbols, constants\index{constant} and variables\index{variable}.
The only thing Metamath knows how to do is to substitute strings of symbols
for the variables\index{substitution!variable}\index{variable substitution} in
an expression based on instructions you provide it in a proof, subject to
certain constraints you specify for the variables.  Even the decimal
representation of a number is merely a string of certain constants (digits)
which together, in a specific context, correspond to whatever mathematical
object you choose to define for it; unlike other computer languages, there is
no actual number stored inside the computer.  In a proof, you in effect
instruct Metamath what symbol substitutions to make in previous axioms or
theorems and join a sequence of them together to result in the desired
theorem.  This kind of symbol manipulation captures the essence of mathematics
at a preaxiomatic level.

\subsubsection{Metamath and Mathematical Literature}

In advanced mathematical literature, proofs are usually presented in the form
of short outlines that often only an expert can follow.  This is partly out of
a desire for brevity, but it would also be unwise (even if it were practical)
to present proofs in complete formal detail, since the overall picture would
be lost.\index{formal proof}

A solution I envision\label{envision} that would allow mathematics to remain
acceptable to the expert, yet increase its accessibility to non-specialists,
consists of a combination of the traditional short, informal proof in print
accompanied by a complete formal proof stored in a computer database.  In an
analogy with a computer program, the informal proof is like a set of comments
that describe the overall reasoning and content of the proof, whereas the
computer database is like the actual program and provides a means for anyone,
even a non-expert, to follow the proof in as much detail as desired, exploring
it back through layers of theorems (like subroutines that call other
subroutines) all the way back to the axioms of the theory.  In addition, the
computer database would have the advantage of providing absolute assurance
that the proof is correct, since each step can be verified automatically.

There are several other approaches besides Metamath to a project such
as this.  Section~\ref{proofverifiers} discusses some of these.

To us, a noble goal would be a database with hundreds of thousands of
theorems and their computer-verifiable proofs, encompassing a significant
fraction of known mathematics and available for instant access.
These would be fully verified by multiple independently-implemented verifiers,
to provide extremely high confidence that the proofs are completely correct.
The database would allow people to investigate whatever details they were
interested in, so that they could confirm whatever portions they wished.
Whether or not Metamath is an appropriate choice remains to be seen, but in
principle we believe it is sufficient.

\subsubsection{Formalism}

Over the past fifty years, a group of French mathematicians working
collectively under the pseudonym of Bourbaki\index{Bourbaki, Nicolas} have
co-authored a series of monographs that attempt to rigorously and
consistently formalize large bodies of mathematics from foundations.  On the
one hand, certainly such an effort has its merits; on the other hand, the
Bourbaki project has been criticized for its ``scholasticism'' and
``hyperaxiomatics'' that hide the intuitive steps that lead to the results
\cite[p.~191]{Barrow}\index{Barrow, John D.}.

Metamath unabashedly carries this philosophy to its extreme and no doubt is
subject to the same kind of criticism.  Nonetheless I think that in
conjunction with conventional approaches to mathematics Metamath can serve a
useful purpose.  The Bourbaki approach is essentially pedagogic, requiring the
reader to become intimately familiar with each detail in a very large
hierarchy before he or she can proceed to the next step.  The difference with
Metamath is that the ``reader'' (user) knows that all details are contained in
its computer database, available as needed; it does not demand that the user
know everything but conveniently makes available those portions that are of
interest.  As the body of all mathematical knowledge grows larger and larger,
no one individual can have a thorough grasp of its entirety.  Metamath
can finalize and put to rest any questions about the validity of any part of it
and can make any part of it accessible, in principle, to a non-specialist.

\subsubsection{A Personal Note}
Why did I develop Metamath\index{Metamath}?  I enjoy abstract mathematics, but
I sometimes get lost in a barrage of definitions and start to lose confidence
that my proofs are correct.  Or I reach a point where I lose sight of how
anything I'm doing relates to the axioms that a theory is based on and am
sometimes suspicious that there may be some overlooked implicit axiom
accidentally introduced along the way (as happened historically with Euclidean
geometry\index{Euclidean geometry}, whose omission of Pasch's
axiom\index{Pasch's axiom} went unnoticed for 2000 years
\cite[p.~160]{Davis}!). I'm also somewhat lazy and wish to avoid the effort
involved in re-verifying the gaps in informal proofs ``left to the reader;'' I
prefer to figure them out just once and not have to go through the same
frustration a year from now when I've forgotten what I did.  Metamath provides
better recovery of my efforts than scraps of paper that I can't
decipher anymore.  But mostly I find very appealing the idea of rigorously
archiving mathematical knowledge in a computer database, providing precision,
certainty, and elimination of human error.

\subsubsection{Note on Bibliography and Index}

The Bibliography usually includes the Library of Congress classification
for a work to make it easier for you to find it in on a university
library shelf.  The Index has author references to pages where their works
are cited, even though the authors' names may not appear on those pages.

\subsubsection{Acknowledgments}

Acknowledgments are first due to my wife, Deborah (who passed away on
September 4, 1998), for critiquing the manu\-script but most of all for
her patience and support.  I also wish to thank Joe Wright, Richard
Becker, Clarke Evans, Buddha Buck, and Jeremy Henty for helpful
comments.  Any errors, omissions, and other shortcomings are of course
my responsibility.

\subsubsection{Note Added June 22, 2005}\label{note2002}

The original, unpublished version of this book was written in 1997 and
distributed via the web.  The present edition has been updated to
reflect the current Metamath program and databases, as well as more
current {\sc url}s for Internet sites.  Thanks to Josh
Purinton\index{Purinton, Josh}, One Hand
Clapping, Mel L.\ O'Cat, and Roy F. Longton for pointing out
typographical and other errors.  I have also benefitted from numerous
discussions with Raph Levien\index{Levien, Raph}, who has extended
Metamath's philosophy of rigor to result in his {\em
Ghilbert}\index{Ghilbert} proof language (\url{http://ghilbert.org}).

Robert (Bob) Solovay\index{Solovay, Robert} communicated a new result of
A.~R.~D.~Mathias on the system of Bourbaki, and the text has been
updated accordingly (p.~\pageref{bourbaki}).

Bob also pointed out a clarification of the literature regarding
category theory and inaccessible cardinals\index{category
theory}\index{cardinal, inaccessible} (p.~\pageref{categoryth}),
and a misleading statement was removed from the text.  Specifically,
contrary to a statement in previous editions, it is possible to express
``There is a proper class of inaccessible cardinals'' in the language of
ZFC.  This can be done as follows:  ``For every set $x$ there is an
inaccessible cardinal $\kappa$ such that $\kappa$ is not in $x$.''
Bob writes:\footnote{Private communication, Nov.~30, 2002.}
\begin{quotation}
     This axiom is how Grothendieck presents category theory.  To each
inaccessible cardinal $\kappa$ one associates a Grothendieck universe
\index{Grothendieck, Alexander} $U(\kappa)$.  $U(\kappa)$ consists of
those sets which lie in a transitive set of cardinality less than
$\kappa$.  Instead of the ``category of all groups,'' one works relative
to a universe [considering the category of groups of cardinality less
than $\kappa$].  Now the category whose objects are all categories
``relative to the universe $U(\kappa)$'' will be a category not
relative to this universe but to the next universe.

     All of the things category theorists like to do can be done in this
framework.  The only controversial point is whether the Grothen\-dieck
axiom is too strong for the needs of category theorists.  Mac Lane
\index{Mac Lane, Saunders} argues that ``one universe is enough'' and
Feferman\index{Feferman, Solomon} has argued that one can get by with
ordinary ZFC.  I don't find Feferman's arguments persuasive.  Mac Lane
may be right, but when I think about category theory I do it \`{a} la
Grothendieck.

        By the way Mizar\index{Mizar} adds the axiom ``there is a proper
class of inaccessibles'' precisely so as to do category theory.
\end{quotation}

The most current information on the Metamath program and databases can
always be found at \url{http://metamath.org}.


\subsubsection{Note Added June 24, 2006}\label{note2006}

The Metamath spec was restricted slightly to make parsers easier to
write.  See the footnote on p.~\pageref{namespace}.

%\subsubsection{Note Added July 24, 2006}\label{note2006b}
\subsubsection{Note Added March 10, 2007}\label{note2006b}

I am grateful to Anthony Williams\index{Williams, Anthony} for writing
the \LaTeX\ package called {\tt realref.sty} and contributing it to the
public domain.  This package allows the internal hyperlinks in a {\sc
pdf} file to anchor to specific page numbers instead of just section
titles, making the navigation of the {\sc pdf} file for this book much
more pleasant and ``logical.''

A typographical error found by Martin Kiselkov was corrected.
A confusing remark about unification was deleted per suggestion of
Mel O'Cat.

\subsubsection{Note Added May 27, 2009}\label{note2009}

Several typos found by Kim Sparre were corrected.  A note was added that
the Poincar\'{e} conjecture has been proved (p.~\pageref{poincare}).

\subsubsection{Note Added Nov. 17, 2014}\label{note2014}

The statement of the Schr\"{o}der--Bernstein theorem was corrected in
Section~\ref{trust}.  Thanks to Bob Solovay for pointing out the error.

\subsubsection{Note Added May 25, 2016}\label{note2016}

Thanks to Jerry James for correcting 16 typos.

\subsubsection{Note Added February 25, 2019}\label{note201902}

David A. Wheeler\index{Wheeler, David A.}
made a large number of improvements and updates,
in coordination with Norman Megill.
The predicate calculus axioms were renumbered, and the text makes
it clear that they are based on Tarski's system S2;
the one slight deviation in axiom ax-6 is explained and justified.
The real and complex number axioms were modified to be consistent with
\texttt{set.mm}\index{set theory database (\texttt{set.mm})}%
\index{Metamath Proof Explorer}.
Long-awaited specification changes ``1--8'' were made,
which clarified previously ambiguous points.
Some errors in the text involving \texttt{\$f} and
\texttt{\$d} statements were corrected (the spec was correct, but
the in-book explanations unintentionally contradicted the spec).
We now have a system for automatically generating narrow PDFs,
so that those with smartphones can have easy access to the current
version of this document.
A new section on deduction was added;
it discusses the standard deduction theorem,
the weak deduction theorem,
deduction style, and natural deduction.
Many minor corrections (too numerous to list here) were also made.

\subsubsection{Note Added March 7, 2019}\label{note201903}

This added a description of the Matamath language syntax in
Extended Backus--Naur Form (EBNF)\index{Extended Backus--Naur Form}\index{EBNF}
in Appendix \ref{BNF}, added a brief explanation about typecodes,
inserted more examples in the deduction section,
and added a variety of smaller improvements.

\subsubsection{Note Added April 7, 2019}\label{note201904}

This version clarified the proper substitution notation, improved the
discussion on the weak deduction theorem and natural deduction,
documented the \texttt{undo} command, updated the information on
\texttt{write source}, changed the typecode
from \texttt{set} to \texttt{setvar} to be consistent with the current
version of \texttt{set.mm}, added more documentation about comment markup
(e.g., documented how to create headings), and clarified the
differences between various assertion forms (in particular deduction form).

\subsubsection{Note Added May 20, 2019}\label{note201905}

This version fixes various small issues and added specific examples
of forms (deduction form, inference form, and closed form).

\chapter{Introduction}
\pagenumbering{arabic}

\begin{quotation}
  {\em {\em I.M.:}  No, no.  There's nothing subjective about it!  Everybody
knows what a proof is.  Just read some books, take courses from a competent
mathematician, and you'll catch on.

{\em Student:}  Are you sure?

{\em I.M.:}  Well---it is possible that you won't, if you don't have any
aptitude for it.  That can happen, too.

{\em Student:}  Then {\em you} decide what a proof is, and if I don't learn
to decide in the same way, you decide I don't have any aptitude.

{\em I.M.:}  If not me, then who?}
    \flushright\sc  ``The Ideal Mathematician''
    \index{Davis, Phillip J.}
    \footnote{\cite{Davis}, p.~40.}\\
\end{quotation}

In the past century, brilliant mathematicians have discovered almost
unimaginably profound results that rank among the crowning intellectual
achievements of mankind.  However, there is a sense in which modern abstract
mathematics is behind the times, stuck in an era before computers existed.
While no one disputes the remarkable results that have been achieved,
communicating these results in a precise way to the uninitiated is virtually
impossible.  To describe these results, a terse informal language is used which
despite its elegance is very difficult to learn.  This informal language is not
imprecise, far from it, but rather it often has omitted detail
and symbols with hidden context that are
implicitly understood by an expert but few others.  Extremely complex technical
meanings are associated with innocent-sounding English words such as
``compact'' and ``measurable'' that barely hint at what is actually being
said.  Anyone who does not keep the precise technical meaning constantly in
mind is bound to fail, and acquiring the ability to do this can be achieved
only through much practice and hard work.  Only the few who complete the
painful learning experience can join the small in-group of pure
mathematicians.  The informal language effectively cuts off the true nature of
their knowledge from most everyone else.

Metamath\index{Metamath} makes abstract mathematics more concrete.  It allows
a computer to keep track of the complexity associated with each word or symbol
with absolute rigor.  You can explore this complexity at your leisure, to
whatever degree you desire.  Whether or not you believe that concepts such as
infinity actually ``exist'' outside of the mind, Metamath lets you get to the
foundation for what's really being said.

Metamath also enables completely rigorous and thorough proof verification.
Its language is simple enough so that you
don't have to rely on the authority of experts but can verify the results
yourself, step by step.  If you want to attempt to derive your own results,
Metamath will not let you make a mistake in reasoning.
Even professional mathematicians make mistakes; Metamath makes it possible
to thoroughly verify that proofs are correct.

Metamath\index{Metamath} is a computer language and an associated computer
program for archiving, verifying, and studying mathematical proofs at a very
detailed level.
The Metamath language
describes formal\index{formal system} mathematical
systems and expresses proofs of theorems in those systems.  Such a language
is called a metalanguage\index{metalanguage} by mathematicians.
The Metamath program is a computer program that verifies
proofs expressed in the Metamath language.
The Metamath program does not have the built-in
ability to make logical inferences; it just makes a series of symbol
substitutions according to instructions given to it in a proof
and verifies that the result matches the expected theorem.  It makes logical
inferences based only on rules of logic that are contained in a set of
axioms\index{axiom}, or first principles, that you provide to it as the
starting point for proofs.

The complete specification of the Metamath language is only four pages long
(Section~\ref{spec}, p.~\pageref{spec}).  Its simplicity may at first make you
wonder how it can do much of anything at all.  But in fact the kinds of
symbol manipulations it performs are the ones that are implicitly done in all
mathematical systems at the lowest level.  You can learn it relatively quickly
and have complete confidence in any mathematical proof that it verifies.  On
the other hand, it is powerful and general enough so that virtually any
mathematical theory, from the most basic to the deeply abstract, can be
described with it.

Although in principle Metamath can be used with any
kind of mathematics, it is best suited for abstract or ``pure'' mathematics
that is mostly concerned with theorems and their proofs, as opposed to the
kind of mathematics that deals with the practical manipulation of numbers.
Examples of branches of pure mathematics are logic\index{logic},\footnote{Logic
is the study of statements that are universally true regardless of the objects
being described by the statements.  An example is the statement, ``if $P$
implies $Q$, then either $P$ is false or $Q$ is true.''} set theory\index{set
theory},\footnote{Set theory is the study of general-purpose mathematical objects called
``sets,'' and from it essentially all of mathematics can be derived.  For
example, numbers can be defined as specific sets, and their properties
can be explored using the tools of set theory.} number theory\index{number
theory},\footnote{Number theory deals with the properties of positive and
negative integers (whole numbers).} group theory\index{group
theory},\footnote{Group theory studies the properties of mathematical objects
called groups that obey a simple set of axioms and have properties of symmetry
that make them useful in many other fields.} abstract algebra\index{abstract
algebra},\footnote{Abstract algebra includes group theory and also studies
groups with additional properties that qualify them as ``rings'' and
``fields.''  The set of real numbers is a familiar example of a field.},
analysis\index{analysis} \index{real and complex numbers}\footnote{Analysis is
the study of real and complex numbers.} and
topology\index{topology}.\footnote{One area studied by topology are properties
that remain unchanged when geometrical objects undergo stretching
deformations; for example a doughnut and a coffee cup each have one hole (the
cup's hole is in its handle) and are thus considered topologically
equivalent.  In general, though, topology is the study of abstract
mathematical objects that obey a certain (surprisingly simple) set of axioms.
See, for example, Munkres \cite{Munkres}\index{Munkres, James R.}.} Even in
physics, Metamath could be applied to certain branches that make use of
abstract mathematics, such as quantum logic\index{quantum logic} (used to study
aspects of quantum mechanics\index{quantum mechanics}).

On the other hand, Metamath\index{Metamath} is less suited to applications
that deal primarily with intensive numeric computations.  Metamath does not
have any built-in representation of numbers\index{Metamath!representation of
numbers}; instead, a specific string of symbols (digits) must be syntactically
constructed as part of any proof in which an ordinary number is used.  For
this reason, numbers in Metamath are best limited to specific constants that
arise during the course of a theorem or its proof.  Numbers are only a tiny
part of the world of abstract mathematics.  The exclusion of built-in numbers
was a conscious decision to help achieve Metamath's simplicity, and there are
other software tools if you have different mathematical needs.
If you wish to quickly solve algebraic problems, the computer algebra
programs\index{computer algebra system} {\sc
macsyma}\index{macsyma@{\sc macsyma}}, Mathematica\index{Mathematica}, and
Maple\index{Maple} are specifically suited to handling numbers and
algebra efficiently.
If you wish to simply calculate numeric or matrix expressions easily,
tools such as Octave\index{Octave} may be a better choice.

After learning Metamath's basic statement types, any
tech\-ni\-cal\-ly ori\-ent\-ed person, mathematician or not, can
immediately trace
any theorem proved in the language as far back as he or she wants, all the way
to the axioms on which the theorem is based.  This ability suggests a
non-traditional way of learning about pure mathematics.  Used in conjunction
with traditional methods, Metamath could make pure mathematics accessible to
people who are not sufficiently skilled to figure out the implicit detail in
ordinary textbook proofs.  Once you learn the axioms of a theory, you can have
complete confidence that everything you need to understand a proof you are
studying is all there, at your beck and call, allowing you to focus in on any
proof step you don't understand in as much depth as you need, without worrying
about getting stuck on a step you can't figure out.\footnote{On the other
hand, writing proofs in the Metamath language is challenging, requiring
a degree of rigor far in excess of that normally taught to students.  In a
classroom setting, I doubt that writing Metamath proofs would ever replace
traditional homework exercises involving informal proofs, because the time
needed to work out the details would not allow a course to
cover much material.  For students who have trouble grasping the implied rigor
in traditional material, writing a few simple proofs in the Metamath language
might help clarify fuzzy thought processes.  Although somewhat difficult at
first, it eventually becomes fun to do, like solving a puzzle, because of the
instant feedback provided by the computer.}

Metamath\index{Metamath} is probably unlike anything you have
encountered before.  In this first chapter we will look at the philosophy and
use of computers in mathematics in order to better understand the motivation
behind Metamath.  The material in this chapter is not required in order to use
Metamath.  You may skip it if you are impatient, but I hope you will find it
educational and enjoyable.  If you want to start experimenting with the
Metamath program right away, proceed directly to Chapter~\ref{using}
(p.~\pageref{using}).  To
learn the Metamath language, skim Chapter~\ref{using} then proceed to
Chapter~\ref{languagespec} (p.~\pageref{languagespec}).

\section{Mathematics as a Computer Language}

\begin{quote}
  {\em The study of mathematics is apt to commence in
dis\-ap\-point\-ment.\ldots \\
We are told that by its aid the stars are weighted
and the billions of molecules in a drop of water are counted.  Yet, like the
ghost of Hamlet's father, this great science eludes the efforts of our mental
weapons to grasp it.}
  \flushright\sc  Alfred North Whitehead\footnote{\cite{Whitehead}, ch.\ 1.}\\
\end{quote}\index{Whitehead, Alfred North}

\subsection{Is Mathematics ``User-Friendly''?}

Suppose you have no formal training in abstract mathematics.  But popular
books you've read offer tempting glimpses of this world filled with profound
ideas that have stirred the human spirit.  You are not satisfied with the
informal, watered-down descriptions you've read but feel it is important to
grasp the underlying mathematics itself to understand its true meaning. It's
not practical to go back to school to learn it, though; you don't want to
dedicate years of your life to it.  There are many important things in life,
and you have to set priorities for what's important to you.  What would happen
if you tried to pursue it on your own, in your spare time?

After all, you were able to learn a computer programming language such as
Pascal on your own without too much difficulty, even though you had no formal
training in computers.  You don't claim to be an expert in software design,
but you can write a passable program when necessary to suit your needs.  Even
more important, you know that you can look at anyone else's Pascal program, no
matter how complex, and with enough patience figure out exactly how it works,
even though you are not a specialist.  Pascal allows you do anything that a
computer can do, at least in principle.  Thus you know you have the ability,
in principle, to follow anything that a computer program can do:  you just
have to break it down into small enough pieces.

Here's an imaginary scenario of what might happen if you na\-ive\-ly a\-dopted
this same view of abstract mathematics and tried to pick it up on your own, in
a period of time comparable to, say, learning a computer programming
language.

\subsubsection{A Non-Mathematician's Quest for Truth}

\begin{quote}
  {\em \ldots my daughters have been studying (chemistry) for several
se\-mes\-ters, think they have learned differential and integral calculus in
school, and yet even today don't know why $x\cdot y=y\cdot x$ is true.}
  \flushright\sc  Edmund Landau\footnote{\cite{Landau}, p.~vi.}\\
\end{quote}\index{Landau, Edmund}

\begin{quote}
  {\em Minus times minus is plus,\\
The reason for this we need not discuss.}
  \flushright\sc W.\ H.\ Auden\footnote{As quoted in \cite{Guillen}, p.~64.}\\
\end{quote}\index{Auden, W.\ H.}\index{Guillen, Michael}

We'll suppose you are a technically oriented professional, perhaps an engineer, a
computer programmer, or a physicist, but probably not a mathematician.  You
consider yourself reasonably intelligent.  You did well in school, learning a
variety of methods and techniques in practical mathematics such as calculus and
differential equations.  But rarely did your courses get into anything
resembling modern abstract mathematics, and proofs were something that appeared
only occasionally in your textbooks, a kind of necessary evil that was
supposed to convince you of a certain key result.  Most of your
homework consisted of exercises that gave you practice in the techniques, and
you were hardly ever asked to come up with a proof of your own.

You find yourself curious about advanced, abstract mathematics.  You are
driven by an inner conviction that it is important to understand and
appreciate some of the most profound knowledge discovered by mankind.  But it
seems very hard to learn, something that only certain gifted longhairs can
access and understand.  You are frustrated that it seems forever cut off from
you.

Eventually your curiosity drives you to do something about it.
You set for yourself a goal of ``really'' understanding mathematics:  not just
how to manipulate equations in algebra or calculus according to cookbook
rules, but rather to gain a deep understanding of where those rules come from.
In fact, you're not thinking about this kind of ordinary mathematics at all,
but about a much more abstract, ethereal realm of pure mathematics, where
famous results such as G\"{o}del's incompleteness theorem\index{G\"{o}del's
incompleteness theorem} and Cantor's different kinds of infinities
reside.

You have probably read a number of popular books, with titles like {\em
Infinity and the Mind} \cite{Rucker}\index{Rucker, Rudy}, on topics such as
these.  You found them inspiring but at the same time somewhat
unsatisfactory.  They gave you a general idea of what these results are about,
but if someone asked you to prove them, you wouldn't have the faintest idea of
where to begin.   Sure, you could give the same overall outline that you
learned from the popular books; and in a general sort of way, you do have an
understanding.  But deep down inside, you know that there is a rigor that is
missing, that probably there are many subtle steps and pitfalls along the way,
and ultimately it seems you have to place your trust in the experts in the
field.  You don't like this; you want to be able to verify these results for
yourself.

So where do you go next?  As a first step, you decide to look up some of the
original papers on the theorems you are curious about, or better, obtain some
standard textbooks in the field.  You look up a theorem you want to
understand.  Sure enough, it's there, but it's expressed with strange
terms and odd symbols that mean absolutely nothing to you.  It might as well be written in
a foreign language you've never seen before, whose symbols are totally alien.
You look at the proof, and you haven't the foggiest notion what each step
means, much less how one step follows from another.  Well, obviously you have
a lot to learn if you want to understand this stuff.

You feel that you could probably understand it by
going back to college for another three to six years and getting a math
degree.  But that does not fit in with your career and the other things in
your life and would serve no practical purpose.  You decide to seek a quicker
path.  You figure you'll just trace your way back to the beginning, step by
step, as you would do with a computer program, until you understand it.  But
you quickly find that this is not possible, since you can't even understand
enough to know what you have to trace back to.

Maybe a different approach is in order---maybe you should start at the
beginning and work your way up.  First, you read the introduction to the book
to find out what the prerequisites are.  In a similar fashion, you trace your
way back through two or three more books, finally arriving at one that seems
to start at a beginning:  it lists the axioms of arithmetic.  ``Aha!'' you
naively think, ``This must be the starting point, the source of all mathematical
knowledge.'' Or at least the starting point for mathematics dealing with
numbers; you have to start somewhere and have no idea what the starting point
for other mathematics would be.  But the word ``axioms'' looks promising.  So
you eagerly read along and work through some elementary exercises at the
beginning of the book.  You feel vaguely bothered:  these
don't seem like axioms at all, at least not in the sense that you want to
think of axioms.  Axioms imply a starting point from which everything else can
be built up, according to precise rules specified in the axiom system.  Even
though you can understand the first few proofs in an informal way,
and are able to do some of the
exercises, it's hard to pin down precisely what the
rules are.   Sure, each step seems to follow logically from the others, but
exactly what does that mean?  Is the ``logic'' just a matter of common sense,
something vague that we all understand but can never quite state precisely?

You've spent a number of years, off and on, programming computers, and you
know that in the case of computer languages there is no question of what the
rules are---they are precise and crystal clear.  If you follow them, your
program will work, and if you don't, it won't.  No matter how complex a
program, it can always be broken down into simpler and simpler pieces, until
you can ultimately identify the bits that are moved around to perform a
specific function.  Some programs might require a lot of perseverance to
accomplish this, but if you focus on a specific portion of it, you don't even
necessarily have to know how the rest of it works. Shouldn't there be an
analogy in mathematics?

You decide to apply the ultimate test:  you ask yourself how a computer could
verify or ensure that the steps in these proofs follow from one another.
Certainly mathematics must be at least as precisely defined as a computer
language, if not more so; after all, computer science itself is based on it.
If you can get a computer to verify these proofs, then you should also be
able, in principle, to understand them yourself in a very crystal clear,
precise way.

You're in for a surprise:  you can conceive of no way to convert the
proofs, which are in English, to a form that the computer can understand.
The proofs are filled with phrases such as ``assume there exists a unique
$x$\ldots'' and ``given any $y$, let $z$ be the number such that\ldots''  This
isn't the kind of logic you are used to in computer programming, where
everything, even arithmetic, reduces to Boolean ones and zeroes if you care to
break it down sufficiently.  Even though you think you understand the proofs,
there seems to be some kind of higher reasoning involved rather than precise
rules that define how you manipulate the symbols in the axioms.  Whatever it
is, it just isn't obvious how you would express it to a computer, and the more
you think about it, the more puzzled and confused you get, to the point where
you even wonder whether {\em you} really understand it.  There's a lot more to
these axioms of arithmetic than meets the eye.

Nobody ever talked about this in school in your applied math and engineering
courses.  You just learned the rules they gave you, not quite understanding
how or why they worked, sometimes vaguely suspicious or uncertain of them, and
through homework problems and osmosis learned how to present solutions that
satisfied the instructor and earned you an ``A.''  Rarely did you actually
``prove'' anything in a rigorous way, and the math majors who did do stuff
like that seemed to be in a different world.

Of course, there are computer algebra programs that can do mathematics, and
rather impressively.  They can instantly solve the integrals that you
struggled with in freshman calculus, and do much, much more.  But when you
look at these programs, what you see is a big collection of algorithms and
techniques that evolved and were added to over time, along with some basic
software that manipulates symbols.  Each algorithm that is built in is the
result of someone's theorem whose proof is omitted; you just have to trust the
person who proved it and the person who programmed it in and hope there are no
bugs.\index{computer program bugs}  Somehow this doesn't seem to be the
essence of mathematics.  Although computer algebra systems can generate
theorems with amazing speed, they can't actually prove a single one of them.

After some puzzlement, you revisit some popular books on what mathematics is
all about.  Somewhere you read that all of mathematics is actually derived
from something called ``set theory.''  This is a little confusing, because
nowhere in the book that presented the axioms of arithmetic was there any
mention of set theory, or if there was, it seemed to be just a tool that helps
you describe things better---the set of even numbers, that sort of thing.  If
set theory is the basis for all mathematics, then why are additional axioms
needed for arithmetic?

Something is wrong but you're not sure what.  One of your friends is a pure
mathematician.  He knows he is unable to communicate to you what he does for a
living and seems to have little interest in trying.  You do know that for him,
proofs are what mathematics is all about. You ask him what a proof is, and he
essentially tells you that, while of course it's based on logic, really it's
something you learn by doing it over and over until you pick it up.  He refers
you to a book, {\em How to Read and Do Proofs} \cite{Solow}.\index{Solow,
Daniel}  Although this book helps you understand traditional informal proofs,
there is still something missing you can't seem to pin down yet.

You ask your friend how you would go about having a computer verify a proof.
At first he seems puzzled by the question; why would you want to do that?
Then he says it's not something that would make any sense to do, but he's
heard that you'd have to break the proof down into thousands or even millions
of individual steps to do such a thing, because the reasoning involved is at
such a high level of abstraction.  He says that maybe it's something you could
do up to a point, but the computer would be completely impractical once you
get into any meaningful mathematics.  There, the only way you can verify a
proof is by hand, and you can only acquire the ability to do this by
specializing in the field for a couple of years in grad school.  Anyway, he
thinks it all has to do with set theory, although he has never taken a formal
course in set theory but just learned what he needed as he went along.

You are intrigued and amazed.  Apparently a mathematician can grasp as a
single concept something that would take a computer a thousand or a million
steps to verify, and have complete confidence in it.  Each one of these
thousand or million steps must be absolutely correct, or else the whole proof
is meaningless.  If you added a million numbers by hand, would you trust the
result?  How do you really know that all these steps are correct, that there
isn't some subtle pitfall in one of these million steps, like a bug in a
computer program?\index{computer program bugs}  After all, you've read that
famous mathematicians have occasionally made mistakes, and you certainly know
you've made your share on your math homework problems in school.

You recall the analogy with a computer program.  Sure, you can understand what
a large computer program such as a word processor does, as a single high-level
concept or a small set of such concepts, but your ability to understand it in
no way ensures that the program is correct and doesn't have hidden bugs.  Even
if you wrote the program yourself you can't really know this; most large
programs that you've written have had bugs that crop up at some later date, no
matter how careful you tried to be while writing them.

OK, so now it seems the reason you can't figure out how to make a
computer verify proofs is because each step really corresponds to a
million small steps.  Well, you say, a computer can do a million
calculations in a second, so maybe it's still practical to do.  Now the
puzzle becomes how to figure out what the million steps are that each
English-language step corresponds to.  Your mathematician friend hasn't
a clue, but suggests that maybe you would find the answer by studying
set theory.  Actually, your friend thinks you're a little off the wall
for even wondering such a thing.  For him, this is not what mathematics
is all about.

The subject of set theory keeps popping up, so you decide it's
time to look it up.

You decide to start off on a careful footing, so you start reading a couple of
very elementary books on set theory.  A lot of it seems pretty obvious, like
intersections, subsets, and Venn diagrams.  You thumb through one of the
books; nowhere is anything about axioms mentioned. The other book relegates to
an appendix a brief discussion that mentions a set of axioms called
``Zermelo--Fraenkel set theory''\index{Zermelo--Fraenkel set theory} and states
them in English.  You look at them and have no idea what they really mean or
what you can do with them.  The comments in this appendix say that the purpose
of mentioning them is to expose you to the idea, but imply that they are not
necessary for basic understanding and that they are really the subject matter
of advanced treatments where fine points such as a certain paradox (Russell's
paradox\index{Russell's paradox}\footnote{Russell's paradox assumes that there
exists a set $S$ that is a collection of all sets that don't contain
themselves.  Now, either $S$ contains itself or it doesn't.  If it contains
itself, it contradicts its definition.  But if it doesn't contain itself, it
also contradicts its definition.  Russell's paradox is resolved in ZF set
theory by denying that such a set $S$ exists.}) are resolved.  Wait a
minute---shouldn't the axioms be a starting point, not an ending point?  If
there are paradoxes that arise without the axioms, how do you know you won't
stumble across one accidentally when using the informal approach?

And nowhere do these books describe how ``all of mathematics can be
derived from set theory'' which by now you've heard a few times.

You find a more advanced book on set theory.  This one actually lists the
axioms of ZF set theory in plain English on page one.  {\em Now} you think
your quest has ended and you've finally found the source of all mathematical
knowledge; you just have to understand what it means.  Here, in one place, is
the basis for all of mathematics!  You stare at the axioms in awe, puzzle over
them, memorize them, hoping that if you just meditate on them long enough they
will become clear.  Of course, you haven't the slightest idea how the rest of
mathematics is ``derived'' from them; in particular, if these are the axioms
of mathematics, then why do arithmetic, group theory, and so on need their own
axioms?

You start reading this advanced book carefully, pondering the meaning of every
word, because by now you're really determined to get to the bottom of this.
The first thing the book does is explain how the axioms came about, which was
to resolve Russell's paradox.\index{Russell's paradox}  In fact that seems to
be the main purpose of their existence; that they supposedly can be used to
derive all of mathematics seems irrelevant and is not even mentioned.  Well,
you go on.  You hope the book will explain to you clearly, step by step, how
to derive things from the axioms.  After all, this is the starting point of
mathematics, like a book that explains the basics of a computer programming
language.  But something is missing.  You find you can't even understand the
first proof or do the first exercise.  Symbols such as $\exists$ and $\forall$
permeate the page without any mention of where they came from or how to
manipulate them; the author assumes you are totally familiar with them and
doesn't even tell you what they mean.  By now you know that $\exists$ means
``there exists'' and $\forall$ means ``for all,'' but shouldn't the rules for
manipulating these symbols be part of the axioms?  You still have no idea
how you could even describe the axioms to a computer.

Certainly there is something much different here from the technical
literature you're used to reading.  A computer language manual almost
always explains very clearly what all the symbols mean, precisely what
they do, and the rules used for combining them, and you work your way up
from there.

After glancing at four or five other such books, you come to the realization
that there is another whole field of study that you need just to get to the
point at which you can understand the axioms of set theory.  The field is
called ``logic.''  In fact, some of the books did recommend it as a
prerequisite, but it just didn't sink in.  You assumed logic was, well, just
logic, something that a person with common sense intuitively understood.  Why
waste your time reading boring treatises on symbolic logic, the manipulation
of 1's and 0's that computers do, when you already know that?  But this is a
different kind of logic, quite alien to you.  The subject of {\sc nand} and
{\sc nor} gates is not even touched upon or in any case has to do with only a
very small part of this field.

So your quest continues.  Skimming through the first couple of introductory
books, you get a general idea of what logic is about and what quantifiers
(``for all,'' ``there exists'') mean, but you find their examples somewhat
trivial and mildly annoying (``all dogs are animals,'' ``some animals are
dogs,'' and such).  But all you want to know is what the rules are for
manipulating the symbols so you can apply them to set theory.  Some formulas
describing the relationships among quantifiers ($\exists$ and $\forall$) are
listed in tables, along with some verbal reasoning to justify them.
Presumably, if you want to find out if a formula is correct, you go through
this same kind of mental reasoning process, possibly using images of dogs and
animals. Intuitively, the formulas seem to make sense.  But when you ask
yourself, ``What are the rules I need to get a computer to figure out whether
this formula is correct?'', you still don't know.  Certainly you don't ask the
computer to imagine dogs and animals.

You look at some more advanced logic books.  Many of them have an introductory
chapter summarizing set theory, which turns out to be a prerequisite.  You
need logic to understand set theory, but it seems you also need set theory to
understand logic!  These books jump right into proving rather advanced
theorems about logic, without offering the faintest clue about where the logic
came from that allows them to prove these theorems.

Luckily, you come across an elementary book of logic that, halfway through,
after the usual truth tables and metaphors, presents in a clear, precise way
what you've been looking for all along: the axioms!  They're divided into
propositional calculus (also called sentential logic) and predicate calculus
(also called first-order logic),\index{first-order logic} with rules so simple
and crystal clear that now you can finally program a computer to understand
them.  Indeed, they're no harder than learning how to play a game of chess.
As far as what you seem to need is concerned, the whole book could have been
written in five pages!

{\em Now} you think you've found the ultimate source of mathematical
truth.  So---the axioms of mathematics consist of these axioms of logic,
together with the axioms of ZF set theory. (By now you've also been able to
figure out how to translate the ZF axioms from English into the
actual symbols of logic which you can now manipulate according to
precise, easy-to-understand rules.)

Of course, you still don't understand how ``all of mathematics can be
derived from set theory,'' but maybe this will reveal itself in due
course.

You eagerly set out to program the axioms and rules into a computer and start
to look at the theorems you will have to prove as the logic is developed.  All
sorts of important theorems start popping up:  the deduction
theorem,\index{deduction theorem} the substitution theorem,\index{substitution
theorem} the completeness theorem of propositional calculus,\index{first-order
logic!completeness} the completeness theorem of predicate calculus.  Uh-oh,
there seems to be trouble.  They all get harder and harder, and not one of
them can be derived with the axioms and rules of logic you've just been
handed.  Instead, they all require ``metalogic'' for their proofs, a kind of
mixture of logic and set theory that allows you to prove things {\em about}
the axioms and theorems of logic rather than {\em with} them.

You plow ahead anyway.  A month later, you've spent much of your
free time getting the computer to verify proofs in propositional calculus.
You've programmed in the axioms, but you've also had to program in the
deduction theorem, the substitution theorem, and the completeness theorem of
propositional calculus, which by now you've resigned yourself to treating as
rather complex additional axioms, since they can't be proved from the axioms
you were given.  You can now get the computer to verify and even generate
complete, rigorous, formal proofs\index{formal proof}.  Never mind that they
may have 100,000 steps---at least now you can have complete, absolute
confidence in them.  Unfortunately, the only theorems you have proved are
pretty trivial and you can easily verify them in a few minutes with truth
tables, if not by inspection.

It looks like your mathematician friend was right.  Getting the computer to do
serious mathematics with this kind of rigor seems almost hopeless.  Even
worse, it seems that the further along you get, the more ``axioms'' you have
to add, as each new theorem seems to involve additional ``metamathematical''
reasoning that hasn't been formalized, and none of it can be derived from the
axioms of logic.  Not only do the proofs keep growing exponentially as you get
further along, but the program to verify them keeps getting bigger and bigger
as you program in more ``metatheorems.''\index{metatheorem}\footnote{A
metatheorem is usually a statement that is too general to be directly provable
in a theory.  For example, ``if $n_1$, $n_2$, and $n_3$ are integers, then
$n_1+n_2+n_3$ is an integer'' is a theorem of number theory.  But ``for any
integer $k > 1$, if $n_1, \ldots, n_k$ are integers, then $n_1+\ldots +n_k$ is
an integer'' is a metatheorem, in other words a family of theorems, one for
every $k$.  The reason it is not a theorem is that the general sum $n_1+\ldots
+n_k$ (as a function of $k$) is not an operation that can be defined directly
in number theory.} The bugs\index{computer program bugs} that have cropped up
so far have already made you start to lose faith in the rigor you seem to have
achieved, and you know it's just going to get worse as your program gets larger.

\subsection{Mathematics and the Non-Specialist}

\begin{quote}
  {\em A real proof is not checkable by a machine, or even by any mathematician
not privy to the gestalt, the mode of thought of the particular field of
mathematics in which the proof is located.}
  \flushright\sc  Davis and Hersh\index{Davis, Phillip J.}
  \footnote{\cite{Davis}, p.~354.}\\
\end{quote}

The bulk of abstract or theoretical mathematics is ordinarily outside
the reach of anyone but a few specialists in each field who have completed
the necessary difficult internship in order to enter its coterie.  The
typical intelligent layperson has no reasonable hope of understanding much of
it, nor even the specialist mathematician of understanding other fields.  It
is like a foreign language that has no dictionary to look up the translation;
the only way you can learn it is by living in the country for a few years.  It
is argued that the effort involved in learning a specialty is a necessary
process for acquiring a deep understanding.  Of course, this is almost certainly
true if one is to make significant contributions to a field; in particular,
``doing'' proofs is probably the most important part of a mathematician's
training.  But is it also necessary to deny outsiders access to it?  Is it
necessary that abstract mathematics be so hard for a layperson to grasp?

A computer normally is of no help whatsoever.  Most published proofs are
actually just series of hints written in an informal style that requires
considerable knowledge of the field to understand.  These are the ``real
proofs'' referred to by Davis and Hersh.\index{informal proof}  There is an
implicit understanding that, in principle, such a proof could be converted to
a complete formal proof\index{formal proof}.  However, it is said that no one
would ever attempt such a conversion, even if they could, because that would
presumably require millions of steps (Section~\ref{dream}).  Unfortunately the
informal style automatically excludes the understanding of the proof
by anyone who hasn't gone through the necessary apprenticeship. The
best that the intelligent layperson can do is to read popular books about deep
and famous results; while this can be helpful, it can also be misleading, and
the lack of detail usually leaves the reader with no ability whatsoever to
explore any aspect of the field being described.

The statements of theorems often use sophisticated notation that makes them
inaccessible to the non-specialist.  For a non-specialist who wants to achieve
a deeper understanding of a proof, the process of tracing definitions and
lemmas back through their hierarchy\index{hierarchy} quickly becomes confusing
and discouraging.  Textbooks are usually written to train mathematicians or to
communicate to people who are already mathematicians, and large gaps in proofs
are often left as exercises to the reader who is left at an impasse if he or
she becomes stuck.

I believe that eventually computers will enable non-specialists and even
intelligent laypersons to follow almost any mathematical proof in any field.
Metamath is an attempt in that direction.  If all of mathematics were as
easily accessible as a computer programming language, I could envision
computer programmers and hobbyists who otherwise lack mathematical
sophistication exploring and being amazed by the world of theorems and proofs
in obscure specialties, perhaps even coming up with results of their own.  A
tremendous advantage would be that anyone could experiment with conjectures in
any field---the computer would offer instant feedback as to whether
an inference step was correct.

Mathematicians sometimes have to put up with the annoyance of
cranks\index{cranks} who lack a fundamental understanding of mathematics but
insist that their ``proofs'' of, say, Fermat's Last Theorem\index{Fermat's
Last Theorem} be taken seriously.  I think part of the problem is that these
people are misled by informal mathematical language, treating it as if they
were reading ordinary expository English and failing to appreciate the
implicit underlying rigor.  Such cranks are rare in the field of computers,
because computer languages are much more explicit, and ultimately the proof is
in whether a computer program works or not.  With easily accessible
computer-based abstract mathematics, a mathematician could say to a crank,
``don't bother me until you've demonstrated your claim on the computer!''

% 22-May-04 nm
% Attempt to move De Millo quote so it doesn't separate from attribution
% CHANGE THIS NUMBER (AND ELIMINATE IF POSSIBLE) WHEN ABOVE TEXT CHANGES
\vspace{-0.5em}

\subsection{An Impossible Dream?}\label{dream}

\begin{quote}
  {\em Even quite basic theorems would demand almost unbelievably vast
  books to display their proofs.}
    \flushright\sc  Robert E. Edwards\footnote{\cite{Edwards}, p.~68.}\\
\end{quote}\index{Edwards, Robert E.}

\begin{quote}
  {\em Oh, of course no one ever really {\em does} it.  It would take
  forever!  You just show that you could do it, that's sufficient.}
    \flushright\sc  ``The Ideal Mathematician''
    \index{Davis, Phillip J.}\footnote{\cite{Davis},
p.~40.}\\
\end{quote}

\begin{quote}
  {\em There is a theorem in the primitive notation of set theory that
  corresponds to the arithmetic theorem `$1000+2000=3000$'.  The formula
  would be forbiddingly long\ldots even if [one] knows the definitions
  and is asked to simplify the long formula according to them, chances are
  he will make errors and arrive at some incorrect result.}
    \flushright\sc  Hao Wang\footnote{\cite{Wang}, p.~140.}\\
\end{quote}\index{Wang, Hao}

% 22-May-04 nm
% Attempt to move De Millo quote so it doesn't separate from attribution
% CHANGE THIS NUMBER (AND ELIMINATE IF POSSIBLE) WHEN ABOVE TEXT CHANGES
\vspace{-0.5em}

\begin{quote}
  {\em The {\em Principia Mathematica} was the crowning achievement of the
  formalists.  It was also the deathblow of the formalist view.\ldots
  {[Rus\-sell]} failed, in three enormous volumes, to get beyond the elementary
  facts of arithmetic.  He showed what can be done in principle and what
  cannot be done in practice.  If the mathematical process were really
  one of strict, logical progression, we would still be counting our
  fingers.\ldots
  One theoretician estimates\ldots that a demonstration of one of
  Ramanujan's conjectures assuming set theory and elementary analysis would
  take about two thousand pages; the length of a deduction from first principles
  is nearly in\-con\-ceiv\-a\-ble\ldots The probabilists argue that\ldots any
  very long proof can at best be viewed as only probably correct\ldots}
  \flushright\sc Richard de Millo et. al.\footnote{\cite{deMillo}, pp.~269,
  271.}\\
\end{quote}\index{de Millo, Richard}

A number of writers have conveyed the impression that the kind of absolute
rigor provided by Metamath\index{Metamath} is an impossible dream, suggesting
that a complete, formal verification\index{formal proof} of a typical theorem
would take millions of steps in untold volumes of books.  Even if it could be
done, the thinking sometimes goes, all meaning would be lost in such a
monstrous, tedious verification.\index{informal proof}\index{proof length}

These writers assume, however, that in order to achieve the kind of complete
formal verification they desire one must break down a proof into individual
primitive steps that make direct reference to the axioms.  This is
not necessary.  There is no reason not to make use of previously proved
theorems rather than proving them over and over.

Just as important, definitions\index{definition} can be introduced along
the way, allowing very complex formulas to be represented with few
symbols.  Not doing this can lead to absurdly long formulas.  For
example, the mere statement of
G\"{o}del's incompleteness theorem\index{G\"{o}del's
incompleteness theorem}, which can be expressed with a small number of
defined symbols, would require about 20,000 primitive symbols to express
it.\index{Boolos, George S.}\footnote{George S.\ Boolos, lecture at
Massachusetts Institute of Technology, spring 1990.} An extreme example
is Bourbaki's\label{bourbaki} language for set theory, which requires
4,523,659,424,929 symbols plus 1,179,618,517,981 disambiguatory links
(lines connecting symbol pairs, usually drawn below or above the
formula) to express the number
``one'' \cite{Mathias}.\index{Mathias, Adrian R. D.}\index{Bourbaki,
Nicolas}
% http://www.dpmms.cam.ac.uk/~ardm/

A hierarchy\index{hierarchy} of theorems and definitions permits an
exponential growth in the formula sizes and primitive proof steps to be
described with only a linear growth in the number of symbols used.  Of course,
this is how ordinary informal mathematics is normally done anyway, but with
Metamath\index{Metamath} it can be done with absolute rigor and precision.

\subsection{Beauty}


\begin{quote}
  {\em No one shall be able to drive us from the paradise that Cantor has
created for us.}
   \flushright\sc  David Hilbert\footnote{As quoted in \cite{Moore}, p.~131.}\\
\end{quote}\index{Hilbert, David}

\needspace{3\baselineskip}
\begin{quote}
  {\em Mathematics possesses not only truth, but some supreme beauty ---a
  beauty cold and austere, like that of a sculpture.}
    \flushright\sc  Bertrand
    Russell\footnote{\cite{Russell}.}\\
\end{quote}\index{Russell, Bertrand}

\begin{quote}
  {\em Euclid alone has looked on Beauty bare.}
  \flushright\sc Edna Millay\footnote{As quoted in \cite{Davis}, p.~150.}\\
\end{quote}\index{Millay, Edna}

For most people, abstract mathematics is distant, strange, and
incomprehensible.  Many popular books have tried to convey some of the sense
of beauty in famous theorems.  But even an intelligent layperson is left with
only a general idea of what a theorem is about and is hardly given the tools
needed to make use of it.  Traditionally, it is only after years of arduous
study that one can grasp the concepts needed for deep understanding.
Metamath\index{Metamath} allows you to approach the proof of the theorem from
a quite different perspective, peeling apart the formulas and definitions
layer by layer until an entirely different kind of understanding is achieved.
Every step of the proof is there, pieced together with absolute precision and
instantly available for inspection through a microscope with a magnification
as powerful as you desire.

A proof in itself can be considered an object of beauty.  Constructing an
elegant proof is an art.  Once a famous theorem has been proved, often
considerable effort is made to find simpler and more easily understood
proofs.  Creating and communicating elegant proofs is a major concern of
mathematicians.  Metamath is one way of providing a common language for
archiving and preserving this information.

The length of a proof can, to a certain extent, be considered an
objective measure of its ``beauty,'' since shorter proofs are usually
considered more elegant.  In the set theory database
\texttt{set.mm}\index{set theory database (\texttt{set.mm})}%
\index{Metamath Proof Explorer}
provided with Metamath, one goal was to make all proofs as short as possible.

\needspace{4\baselineskip}
\subsection{Simplicity}

\begin{quote}
  {\em God made man simple; man's complex problems are of his own
  devising.}
    \flushright\sc Eccles. 7:29\footnote{Jerusalem Bible.}\\
\end{quote}\index{Bible}

\needspace{3\baselineskip}
\begin{quote}
  {\em God made integers, all else is the work of man.}
    \flushright\sc Leopold Kronecker\footnote{{\em Jahresbericht
	der Deutschen Mathematiker-Vereinigung }, vol. 2, p. 19.}\\
\end{quote}\index{Kronecker, Leopold}

\needspace{3\baselineskip}
\begin{quote}
  {\em For what is clear and easily comprehended attracts; the
  complicated repels.}
    \flushright\sc David Hilbert\footnote{As quoted in \cite{deMillo},
p.~273.}\\
\end{quote}\index{Hilbert, David}

The Metamath\index{Metamath} language is simple and Spartan.  Metamath treats
all mathematical expressions as simple sequences of symbols, devoid of meaning.
The higher-level or ``metamathematical'' notions underlying Metamath are about
as simple as they could possibly be.  Each individual step in a proof involves
a single basic concept, the substitution of an expression for a variable, so
that in principle almost anyone, whether mathematician or not, can
completely understand how it was arrived at.

In one of its most basic applications, Metamath\index{Metamath} can be used to
develop the foundations of mathematics\index{foundations of mathematics} from
the very beginning.  This is done in the set theory database that is provided
with the Metamath package and is the subject matter
of Chapter~\ref{fol}. Any language (a metalanguage\index{metalanguage})
used to describe mathematics (an object language\index{object language}) must
have a mathematical content of its own, but it is desirable to keep this
content down to a bare minimum, namely that needed to make use of the
inference rules specified by the axioms.  With any metalanguage there is a
``chicken and egg'' problem somewhat like circular reasoning:  you must assume
the validity of the mathematics of the metalanguage in order to prove the
validity of the mathematics of the object language.  The mathematical content
of Metamath itself is quite limited.  Like the rules of a game of chess, the
essential concepts are simple enough so that virtually anyone should be able to
understand them (although that in itself will not let you play like
a master).  The symbols that Metamath manipulates do not in themselves
have any intrinsic meaning.  Your interpretation of the axioms that you supply
to Metamath is what gives them meaning.  Metamath is an attempt to strip down
mathematical thought to its bare essence and show you exactly how the symbols
are manipulated.

Philosophers and logicians, with various motivations, have often thought it
important to study ``weak'' fragments of logic\index{weak logic}
\cite{Anderson}\index{Anderson, Alan Ross} \cite{MegillBunder}\index{Megill,
Norman}\index{Bunder, Martin}, other unconventional systems of logic (such as
``modal'' logic\index{modal logic} \cite[ch.\ 27]{Boolos}\index{Boolos, George
S.}), and quantum logic\index{quantum logic} in physics
\cite{Pavicic}\index{Pavi{\v{c}}i{\'{c}}, M.}.  Metamath\index{Metamath}
provides a framework in which such systems can be expressed, with an absolute
precision that makes all underlying metamathematical assumptions rigorous and
crystal clear.

Some schools of philosophical thought, for example
intuitionism\index{intuitionism} and constructivism\index{constructivism},
demand that the notions underlying any mathematical system be as simple and
concrete as possible.  Metamath should meet the requirements of these
philosophies.  Metamath must be taught the symbols, axioms\index{axiom}, and
rules\index{rule} for a specific theory, from the skeptical (such as
intuitionism\index{intuitionism}\footnote{Intuitionism does not accept the law
of excluded middle (``either something is true or it is not true'').  See
\cite[p.~xi]{Tymoczko}\index{Tymoczko, Thomas} for discussion and references
on this topic.  Consider the theorem, ``There exist irrational numbers $a$ and
$b$ such that $a^b$ is rational.''  An intuitionist would reject the following
proof:  If $\sqrt{2}^{\sqrt{2}}$ is rational, we are done.  Otherwise, let
$a=\sqrt{2}^{\sqrt{2}}$ and $b=\sqrt{2}$. Then $a^b=2$, which is rational.})
to the bold (such as the axiom of choice in set theory\footnote{The axiom of
choice\index{Axiom of Choice} asserts that given any collection of pairwise
disjoint nonempty sets, there exists a set that has exactly one element in
common with each set of the collection.  It is used to prove many important
theorems in standard mathematics.  Some philosophers object to it because it
asserts the existence of a set without specifying what the set contains
\cite[p.~154]{Enderton}\index{Enderton, Herbert B.}.  In one foundation for
mathematics due to Quine\index{Quine, Willard Van Orman}, that has not been
otherwise shown to be inconsistent, the axiom of choice turns out to be false
\cite[p.~23]{Curry}\index{Curry, Haskell B.}.  The \texttt{show
trace{\char`\_}back} command of the Metamath program allows you to find out
whether the axiom of choice, or any other axiom, was assumed by a
proof.}\index{\texttt{show trace{\char`\_}back} command}).

The simplicity of the Metamath language lets the algorithm (computer program)
that verifies the validity of a Metamath proof be straightforward and
robust.  You can have confidence that the theorems it verifies really can be
derived from your axioms.

\subsection{Rigor}

\begin{quote}
  {\em Rigor became a goal with the Greeks\ldots But the efforts to
  pursue rigor to the utmost have led to an impasse in which there is
  no longer any agreement on what it really means.  Mathematics remains
  alive and vital, but only on a pragmatic basis.}
    \flushright\sc  Morris Kline\footnote{\cite{Kline}, p.~1209.}\\
\end{quote}\index{Kline, Morris}

Kline refers to a much deeper kind of rigor than that which we will discuss in
this section.  G\"{o}del's incompleteness theorem\index{G\"{o}del's
incompleteness theorem} showed that it is impossible to achieve absolute rigor
in standard mathematics because we can never prove that mathematics is
consistent (free from contradictions).\index{consistent theory}  If
mathematics is consistent, we will never know it, but must rely on faith.  If
mathematics is inconsistent, the best we can hope for is that some clever
future mathematician will discover the inconsistency.  In this case, the
axioms would probably be revised slightly to eliminate the inconsistency, as
was done in the case of Russell's paradox,\index{Russell's paradox} but the
bulk of mathematics would probably not be affected by such a discovery.
Russell's paradox, for example, did not affect most of the remarkable results
achieved by 19th-century and earlier mathematicians.  It mainly invalidated
some of Gottlob Frege's\index{Frege, Gottlob} work on the foundations of
mathematics in the late 1800's; in fact Frege's work inspired Russell's
discovery.  Despite the paradox, Frege's work contains important concepts that
have significantly influenced modern logic.  Kline's {\em Mathematics, The
Loss of Certainty} \cite{Klinel}\index{Kline, Morris} has an interesting
discussion of this topic.

What {\em can} be achieved with absolute certainty\index{certainty} is the
knowledge that if we assume the axioms are consistent and true, then the
results derived from them are true.  Part of the beauty of mathematics is that
it is the one area of human endeavor where absolute certainty can be achieved
in this sense.  A mathematical truth will remain such for eternity.  However,
our actual knowledge of whether a particular statement is a mathematical truth
is only as certain as the correctness of the proof that establishes it.  If
the proof of a statement is questionable or vague, we can't have absolute
confidence in the truth that the statement claims.

Let us look at some traditional ways of expressing proofs.

Except in the field of formal logic\index{formal logic}, almost all
traditional proofs in mathematics are really not proofs at all, but rather
proof outlines or hints as to how to go about constructing the proof.  Many
gaps\index{gaps in proofs} are left for the reader to fill in. There are
several reasons for this.  First, it is usually assumed in mathematical
literature that the person reading the proof is a mathematician familiar with
the specialty being described, and that the missing steps are obvious to such
a reader or at least that the reader is capable of filling them in.  This
attitude is fine for professional mathematicians in the specialty, but
unfortunately it often has the drawback of cutting off the rest of the world,
including mathematicians in other specialties, from understanding the proof.
We discussed one possible resolution to this on p.~\pageref{envision}.
Second, it is often assumed that a complete formal proof\index{formal proof}
would require countless millions of symbols (Section~\ref{dream}). This might
be true if the proof were to be expressed directly in terms of the axioms of
logic and set theory,\index{set theory} but it is usually not true if we allow
ourselves a hierarchy\index{hierarchy} of definitions and theorems to build
upon, using a notation that allows us to introduce new symbols, definitions,
and theorems in a precisely specified way.

Even in formal logic,\index{formal logic} formal proofs\index{formal proof}
that are considered complete still contain hidden or implicit information.
For example, a ``proof'' is usually defined as a sequence of
wffs,\index{well-formed formula (wff)}\footnote{A {\em wff} or well-formed
formula is a mathematical expression (string of symbols) constructed according
to some precise rules.  A formal mathematical system\index{formal system}
contains (1) the rules for constructing syntactically correct
wffs,\index{syntax rules} (2) a list of starting wffs called
axioms,\index{axiom} and (3) one or more rules prescribing how to derive new
wffs, called theorems\index{theorem}, from the axioms or previously derived
theorems.  An example of such a system is contained in
Metamath's\index{Metamath} set theory database, which defines a formal
system\index{formal system} from which all of standard mathematics can be
derived.  Section~\ref{startf} steps you through a complete example of a formal
system, and you may want to skim it now if you are unfamiliar with the
concept.} each of which is an axiom or follows from a rule applied to previous
wffs in the sequence.  The implicit part of the proof is the algorithm by
which a sequence of symbols is verified to be a valid wff, given the
definition of a wff.  The algorithm in this case is rather simple, but for a
computer to verify the proof,\index{automated proof verification} it must have
the algorithm built into its verification program.\footnote{It is possible, of
course, to specify wff construction syntax outside of the program itself
with a suitable input language (the Metamath language being an example), but
some proof-verification or theorem-proving programs lack the ability to extend
wff syntax in such a fashion.} If one deals exclusively with axioms and
elementary wffs, it is straightforward to implement such an algorithm.  But as
more and more definitions are added to the theory in order to make the
expression of wffs more compact, the algorithm becomes more and more
complicated.  A computer program that implements the algorithm becomes larger
and harder to understand as each definition is introduced, and thus more prone
to bugs.\index{computer program bugs}  The larger the program, the
more suspicious the mathematician may be about
the validity of its algorithms.  This is especially true because
computer programs are inherently hard to follow to begin with, and few people
enjoy verifying them manually in detail.

Metamath\index{Metamath} takes a different approach.  Metamath's ``knowledge''
is limited to the ability to substitute variables for expressions, subject to
some simple constraints.  Once the basic algorithm of Metamath is assumed to
be debugged, and perhaps independently confirmed, it
can be trusted once and for all.  The information that Metamath needs to
``understand'' mathematics is contained entirely in the body of knowledge
presented to Metamath.  Any errors in reasoning can only be errors in the
axioms or definitions contained in this body of knowledge.  As a
``constructive'' language\index{constructive language} Metamath has no
conditional branches or loops like the ones that make computer programs hard
to decipher; instead, the language can only build new sequences of symbols
from earlier sequences  of symbols.

The simplicity of the rules that underlie Metamath not only makes Metamath
easy to learn but also gives Metamath a great deal of flexibility. For
example, Metamath is not limited to describing standard first-order
logic\index{first-order logic}; higher-order logics\index{higher-order logic}
and fragments of logic\index{weak logic} can be described just as easily.
Metamath gives you the freedom to define whatever wff notation you prefer; it
has no built-in conception of the syntax of a wff.\index{well-formed formula
(wff)}  With suitable axioms and definitions, Metamath can even describe and
prove things about itself.\index{Metamath!self-description}  (John
Harrison\index{Harrison, John} discusses the ``reflection''
principle\index{reflection principle} involved in self-descriptive systems in
\cite{Harrison}.)

The flexibility of Metamath requires that its proofs specify a lot of detail,
much more than in an ordinary ``formal'' proof.\index{formal proof}  For
example, in an ordinary formal proof, a single step consists of displaying the
wff that constitutes that step.  In order for a computer program to verify
that the step is acceptable, it first must verify that the symbol sequence
being displayed is an acceptable wff.\index{automated proof verification} Most
proof verifiers have at least basic wff syntax built into their programs.
Metamath has no hard-wired knowledge of what constitutes a wff built into it;
instead every wff must be explicitly constructed based on rules defining wffs
that are present in a database.  Thus a single step in an ordinary formal
proof may be correspond to many steps in a Metamath proof. Despite the larger
number of steps, though, this does not mean that a Metamath proof must be
significantly larger than an ordinary formal proof. The reason is that since
we have constructed the wff from scratch, we know what the wff is, so there is
no reason to display it.  We only need to refer to a sequence of statements
that construct it.  In a sense, the display of the wff in an ordinary formal
proof is an implicit proof of its own validity as a wff; Metamath just makes
the proof explicit. (Section~\ref{proof} describes Metamath's proof notation.)

\section{Computers and Mathematicians}

\begin{quote}
  {\em The computer is important, but not to mathematics.}
    \flushright\sc  Paul Halmos\footnote{As quoted in \cite{Albers}, p.~121.}\\
\end{quote}\index{Halmos, Paul}

Pure mathematicians have traditionally been indifferent to computers, even to
the point of disdain.\index{computers and pure mathematics}  Computer science
itself is sometimes considered to fall in the mundane realm of ``applied''
mathematics, perhaps essential for the real world but intellectually unexciting
to those who seek the deepest truths in mathematics.  Perhaps a reason for this
attitude towards computers is that there is little or no computer software that
meets their needs, and there may be a general feeling that such software could
not even exist.  On the one hand, there are the practical computer algebra
systems, which can perform amazing symbolic manipulations in algebra and
calculus,\index{computer algebra system} yet can't prove the simplest
existence theorem, if the idea of a proof is present at all.  On the other
hand, there are specialized automated theorem provers that technically speaking
may generate correct proofs.\index{automated theorem proving}  But sometimes
their specialized input notation may be cryptic and their output perceived to
be long, inelegant, incomprehensible proofs.    The output
may be viewed with suspicion, since the program that generates it tends to be
very large, and its size increases the potential for bugs\index{computer
program bugs}.  Such a proof may be considered trustworthy only if
independently verified and ``understood'' by a human, but no one wants to
waste their time on such a boring, unrewarding chore.



\needspace{4\baselineskip}
\subsection{Trusting the Computer}

\begin{quote}
  {\em \ldots I continue to find the quasi-empirical interpretation of
  computer proofs to be the more plausible.\ldots Since not
  everything that claims to be a computer proof can be
  accepted as valid, what are the mathematical criteria for acceptable
  computer proofs?}
    \flushright\sc  Thomas Tymoczko\footnote{\cite{Tymoczko}, p.~245.}\\
\end{quote}\index{Tymoczko, Thomas}

In some cases, computers have been essential tools for proving famous
theorems.  But if a proof is so long and obscure that it can be verified in a
practical way only with a computer, it is vaguely felt to be suspicious.  For
example, proving the famous four-color theorem\index{four-color
theorem}\index{proof length} (``a map needs no more than four colors to
prevent any two adjacent countries from having the same color'') can presently
only be done with the aid of a very complex computer program which originally
required 1200 hours of computer time. There has been considerable debate about
whether such a proof can be trusted and whether such a proof is ``real''
mathematics \cite{Swart}\index{Swart, E. R.}.\index{trusting computers}

However, under normal circumstances even a skeptical mathematician would have a
great deal of confidence in the result of multiplying two numbers on a pocket
calculator, even though the precise details of what goes on are hidden from its
user.  Even the verification on a supercomputer that a huge number is prime is
trusted, especially if there is independent verification; no one bothers to
debate the philosophical significance of its ``proof,'' even though the actual
proof would be so large that it would be completely impractical to ever write
it down on paper.  It seems that if the algorithm used by the computer is
simple enough to be readily understood, then the computer can be trusted.

Metamath\index{Metamath} adopts this philosophy.  The simplicity of its
language makes it easy to learn, and because of its simplicity one can have
essentially absolute confidence that a proof is correct. All axioms, rules, and
definitions are available for inspection at any time because they are defined
by the user; there are no hidden or built-in rules that may be prone to subtle
bugs\index{computer program bugs}.  The basic algorithm at the heart of
Metamath is simple and fixed, and it can be assumed to be bug-free and robust
with a degree of confidence approaching certainty.
Independently written implementations of the Metamath verifier
can reduce any residual doubt on the part of a skeptic even further;
there are now over a dozen such implementations, written by many people.

\subsection{Trusting the Mathematician}\label{trust}

\begin{quote}
  {\em There is no Algebraist nor Mathematician so expert in his science, as
  to place entire confidence in any truth immediately upon his discovery of it,
  or regard it as any thing, but a mere probability.  Every time he runs over
  his proofs, his confidence encreases; but still more by the approbation of
  his friends; and is rais'd to its utmost perfection by the universal assent
  and applauses of the learned world.}
  \flushright\sc David Hume\footnote{{\em A Treatise of Human Nature}, as
  quoted in \cite{deMillo}, p.~267.}\\
\end{quote}\index{Hume, David}

\begin{quote}
  {\em Stanislaw Ulam estimates that mathematicians publish 200,000 theorems
  every year.  A number of these are subsequently contradicted or otherwise
  disallowed, others are thrown into doubt, and most are ignored.}
  \flushright\sc Richard de Millo et. al.\footnote{\cite{deMillo}, p.~269.}\\
\end{quote}\index{Ulam, Stanislaw}

Whether or not the computer can be trusted, humans  of course will occasionally
err. Only the most memorable proofs get independently verified, and of these
only a handful of truly great ones achieve the status of being ``known''
mathematical truths that are used without giving a second thought to their
correctness.

There are many famous examples of incorrect theorems and proofs in
mathematical literature.\index{errors in proofs}

\begin{itemize}
\item There have been thousands of purported proofs of Fermat's Last
Theorem\index{Fermat's Last Theorem} (``no integer solutions exist to $x^n +
y^n = z^n$ for $n > 2$''), by amateurs, cranks, and well-regarded
mathematicians \cite[p.~5]{Stark}\index{Stark, Harold M}.  Fermat wrote a note
in his copy of Bachet's {\em Diophantus} that he found ``a truly marvelous
proof of this theorem but this margin is too narrow to contain it''
\cite[p.~507]{Kramer}.  A recent, much publicized proof by Yoichi
Miyaoka\index{Miyaoka, Yoichi} was shown to be incorrect ({\em Science News},
April 9, 1988, p.~230).  The theorem was finally proved by Andrew
Wiles\index{Wiles, Andrew} ({\em Science News}, July 3, 1993, p.~5), but it
initially had some gaps and took over a year after its announcement to be
checked thoroughly by experts.  On Oct. 25, 1994, Wiles announced that the last
gap found in his proof had been filled in.
  \item In 1882, M. Pasch discovered that an axiom was omitted from Euclid's
formulation of geometry\index{Euclidean geometry}; without it, the proofs of
important theorems of Euclid are not valid.  Pasch's axiom\index{Pasch's
axiom} states that a line that intersects one side of a triangle must also
intersect another side, provided that it does not touch any of the triangle's
vertices.  The omission of Pasch's axiom went unnoticed for 2000
years \cite[p.~160]{Davis}, in spite of (one presumes) the thousands of
students, instructors, and mathematicians who studied Euclid.
  \item The first published proof of the famous Schr\"{o}der--Bernstein
theorem\index{Schr\"{o}der--Bernstein theorem} in set theory was incorrect
\cite[p.~148]{Enderton}\index{Enderton, Herbert B.}.  This theorem states
that if there exists a 1-to-1 function\footnote{A {\em set}\index{set} is any
collection of objects. A {\em function}\index{function} or {\em
mapping}\index{mapping} is a rule that assigns to each element of one set
(called the function's {\em domain}\index{domain}) an element from another
set.} from set $A$ into set $B$ and vice-versa, then sets $A$ and $B$ have
a 1-to-1 correspondence.  Although it sounds simple and obvious,
the standard proof is quite long and complex.
  \item In the early 1900's, Hilbert\index{Hilbert, David} published a
purported proof of the continuum hypothesis\index{continuum hypothesis}, which
was eventually established as unprovable by Cohen\index{Cohen, Paul} in 1963
\cite[p.~166]{Enderton}.  The continuum hypothesis states that no
infinity\index{infinity} (``transfinite cardinal number'')\index{cardinal,
transfinite} exists whose size (or ``cardinality''\index{cardinality}) is
between the size of the set of integers and the size of the set of real
numbers.  This hypothesis originated with German mathematician Georg
Cantor\index{Cantor, Georg} in the late 1800's, and his inability to prove it
is said to have contributed to mental illness that afflicted him in his later
years.
  \item An incorrect proof of the four-color theorem\index{four-color theorem}
was published by Kempe\index{Kempe, A. B.} in 1879
\cite[p.~582]{Courant}\index{Courant, Richard}; it stood for 11 years before
its flaw was discovered.  This theorem states that any map can be colored
using only four colors, so that no two adjacent countries have the same
color.  In 1976 the theorem was finally proved by the famous computer-assisted
proof of Haken, Appel, and Koch \cite{Swart}\index{Appel, K.}\index{Haken,
W.}\index{Koch, K.}.  Or at least it seems that way.  Mathematician
H.~S.~M.~Coxeter\index{Coxeter, H. S. M.} has doubts \cite[p.~58]{Davis}:  ``I
have a feeling that that is an untidy kind of use of the computers, and the more
you correspond with Haken and Appel, the more shaky you seem to be.''
  \item Many false ``proofs'' of the Poincar\'{e}
conjecture\index{Poincar\'{e} conjecture} have been proposed over the years.
This conjecture states that any object that mathematically behaves like a
three-dimensional sphere is a three-dimensional sphere topologically,
regardless of how it is distorted.  In March 1986, mathematicians Colin
Rourke\index{Rourke, Colin} and Eduardo R\^{e}go\index{R\^{e}go, Eduardo}
caused  a stir in the mathematical community by announcing that they had found
a proof; in November of that year the proof was found to be false \cite[p.
218]{PetersonI}.  It was finally proved in 2003 by Grigory Perelman
\label{poincare}\index{Szpiro, George}\index{Perelman, Grigory}\cite{Szpiro}.
 \end{itemize}

Many counterexamples to ``theorems'' in recent mathematical
literature related to Clifford algebras\index{Clifford algebras}
 have been found by Pertti
Lounesto (who passed away in 2002).\index{Lounesto, Pertti}
See the web page \url{http://mathforum.org/library/view/4933.html}.
% http://users.tkk.fi/~ppuska/mirror/Lounesto/counterexamples.htm

One of the purposes of Metamath\index{Metamath} is to allow proofs to be
expressed with absolute precision.  Developing a proof in the Metamath
language can be challenging, because Metamath will not permit even the
tiniest mistake.\index{errors in proofs}  But once the proof is created, its
correctness can be trusted immediately, without having to depend on the
process of peer review for confirmation.

\section{The Use of Computers in Mathematics}

\subsection{Computer Algebra Systems}

For the most part, you will find that Metamath\index{Metamath} is not a
practical tool for manipulating numbers.  (Even proving that $2 + 2 = 4$, if
you start with set theory, can be quite complex!)  Several commercial
mathematics packages are quite good at arithmetic, algebra, and calculus, and
as practical tools they are invaluable.\index{computer algebra system} But
they have no notion of proof, and cannot understand statements starting with
``there exists such and such...''.

Software packages such as Mathematica \cite{Wolfram}\index{Mathematica} do not
concern themselves with proofs but instead work directly with known results.
These packages primarily emphasize heuristic rules such as the substitution of
equals for equals to achieve simpler expressions or expressions in a different
form.  Starting with a rich collection of built-in rules and algorithms, users
can add to the collection by means of a powerful programming language.
However, results such as, say, the existence of a certain abstract object
without displaying the actual object cannot be expressed (directly) in their
languages.  The idea of a proof from a small set of axioms is absent.  Instead
this software simply assumes that each fact or rule you add to the built-in
collection of algorithms is valid.  One way to view the software is as a large
collection of axioms from which the software, with certain goals, attempts to
derive new theorems, for example equating a complex expression with a simpler
equivalent. But the terms ``theorem''\index{theorem} and
``proof,''\index{proof} for example, are not even mentioned in the index of
the user's manual for Mathematica.\index{Mathematica and proofs}  What is also
unsatisfactory from a philosophical point of view is that there is no way to
ensure the validity of the results other than by trusting the writer of each
application module or tediously checking each module by hand, similar to
checking a computer program for bugs.\index{computer program
bugs}\footnote{Two examples illustrate why the knowledge database of computer
algebra systems should sometimes be regarded with a certain caution.  If you
ask Mathematica (version 3.0) to \texttt{Solve[x\^{ }n + y\^{ }n == z\^{ }n , n]}
it will respond with \texttt{\{\{n-\char`\>-2\}, \{n-\char`\>-1\},
\{n-\char`\>1\}, \{n-\char`\>2\}\}}. In other words, Mathematica seems to
``know'' that Fermat's Last Theorem\index{Fermat's Last Theorem} is true!  (At
the time this version of Mathematica was released this fact was unknown.)  If
you ask Maple\index{Maple} to \texttt{solve(x\^{ }2 = 2\^{ }x)} then
\texttt{simplify(\{"\})}, it returns the solution set \texttt{\{2, 4\}}, apparently
unaware that 0.7666647\ldots is also a solution.} While of course extremely
valuable in applied mathematics, computer algebra systems tend to be of little
interest to the theoretical mathematician except as aids for exploring certain
specific problems.

Because of possible bugs, trusting the output of a computer algebra system for
use as theorems in a proof-verifier would defeat the latter's goal of rigor.
On the other hand, a fact such that a certain relatively large number is
prime, while easy for a computer algebra system to derive, might have a long,
tedious proof that could overwhelm a proof-verifier. One approach for linking
computer algebra systems to a proof-verifier while retaining the advantages of
both is to add a hypothesis to each such theorem indicating its source.  For
example, a constant {\sc maple} could indicate the theorem came from the Maple
package, and would mean ``assuming Maple is consistent, then\ldots''  This and
many other topics concerning the formalization of mathematics are discussed in
John Harrison's\index{Harrison, John} very interesting
PhD thesis~\cite{Harrison-thesis}.

\subsection{Automated Theorem Provers}\label{theoremprovers}

A mathematical theory is ``decidable''\index{decidable theory} if a mechanical
method or algorithm exists that is guaranteed to determine whether or not a
particular formula is a theorem.  Among the few theories that are decidable is
elementary geometry,\index{Euclidean geometry} as was shown by a classic
result of logician Alfred Tarski\index{Tarski, Alfred} in 1948
\cite{Tarski}.\footnote{Tarski's result actually applies to a subset of the
geometry discussed in elementary textbooks.  This subset includes most of what
would be considered elementary geometry but it is not powerful enough to
express, among other things, the notions of the circumference and area of a
circle.  Extending the theory in a way that includes notions such as these
makes the theory undecidable, as was also shown by Tarski.  Tarski's algorithm
is far too inefficient to implement practically on a computer.  A practical
algorithm for proving a smaller subset of geometry theorems (those not
involving concepts of ``order'' or ``continuity'') was discovered by Chinese
mathematician Wu Wen-ts\"{u}n in 1977 \cite{Chou}\index{Chou,
Shang-Ching}.}\index{Wen-ts{\"{u}}n, Wu}  But most theories, including
elementary arithmetic, are undecidable.  This fact contributes to keeping
mathematics alive and well, since many mathematicians believe
that they will never be
replaced by computers (if they believe Roger Penrose's argument that a
computer can never replace the brain \cite{Penrose}\index{Penrose, Roger}).
In fact,  elementary geometry is often considered a ``dead'' field
for the simple reason that it is decidable.

On the other hand, the undecidability of a theory does not mean that one cannot
use a computer to search for proofs, providing one is willing to give up if a
proof is not found after a reasonable amount of time.  The field of automated
theorem proving\index{automated theorem proving} specializes in pursuing such
computer searches.  Among the more successful results to date are those based
on an algorithm known as Robinson's resolution principle
\cite{Robinson}\index{Robinson's resolution principle}.

Automated theorem provers can be excellent tools for those willing to learn
how to use them.  But they are not widely used in mainstream pure
mathematics, even though they could probably be useful in many areas.  There
are several reasons for this.  Probably most important, the main goal in pure
mathematics is to arrive at results that are considered to be deep or
important; proving them is essential but secondary.  Usually, an automated
theorem prover cannot assist in this main goal, and by the time the main goal
is achieved, the mathematician may have already figured out the proof as a
by-product.  There is also a notational problem.  Mathematicians are used to
using very compact syntax where one or two symbols (heavily dependent on
context) can represent very complex concepts; this is part of the
hierarchy\index{hierarchy} they have built up to tackle difficult problems.  A
theorem prover on the other hand might require that a theorem be expressed in
``first-order logic,''\index{first-order logic} which is the logic on which
most of mathematics is ultimately based but which is not ordinarily used
directly because expressions can become very long.  Some automated theorem
provers are experimental programs, limited in their use to very specialized
areas, and the goal of many is simply research into the nature of automated
theorem proving itself.  Finally, much research remains to be done to enable
them to prove very deep theorems.  One significant result was a
computer proof by Larry Wos\index{Wos, Larry} and colleagues that every Robbins
algebra\index{Robbins algebra} is a Boolean  algebra\index{Boolean algebra}
({\em New York Times}, Dec. 10, 1996).\footnote{In 1933, E.~V.\
Huntington\index{Huntington, E. V.}
presented the following axiom system for
Boolean algebra with a unary operation $n$ and a binary operation $+$:
\begin{center}
    $x + y = y + x$ \\
    $(x + y) + z = x + (y + z)$ \\
    $n(n(x) + y) + n(n(x) + n(y)) = x$
\end{center}
Herbert Robbins\index{Robbins, Herbert}, a student of Huntington, conjectured
that the last equation can be replaced with a simpler one:
\begin{center}
    $n(n(x + y) + n(x + n(y))) = x$
\end{center}
Robbins and Huntington could not find a proof.  The problem was
later studied unsuccessfully by Tarski\index{Tarski, Alfred} and his
students, and it remained an unsolved problem until a
computer found the proof in 1996.  For more information on
the Robbins algebra problem see \cite{Wos}.}

How does Metamath\index{Metamath} relate to automated theorem provers?  A
theorem prover is primarily concerned with one theorem at a time (perhaps
tapping into a small database of known theorems) whereas Metamath is more like
a theorem archiving system, storing both the theorem and its proof in a
database for access and verification.  Metamath is one answer to ``what do you
do with the output of a theorem prover?''  and could be viewed as the
next step in the process.  Automated theorem provers could be useful tools for
helping develop its database.
Note that very long, automatically
generated proofs can make your database fat and ugly and cause Metamath's proof
verification to take a long time to run.  Unless you have a particularly good
program that generates very concise proofs, it might be best to consider the
use of automatically generated proofs as a quick-and-dirty approach, to be
manually rewritten at some later date.

The program {\sc otter}\index{otter@{\sc otter}}\footnote{\url{http://www.cs.unm.edu/\~mccune/otter/}.}, later succeeded by
prover9\index{prover9}\footnote{\url{https://www.cs.unm.edu/~mccune/mace4/}.},
have been historically influential.
The E prover\index{E prover}\footnote{\url{https://github.com/eprover/eprover}.}
is a maintained automated theorem prover
for full first-order logic with equality.
There are many other automated theorem provers as well.

If you want to combine automated theorem provers with Metamath
consider investigating
the book {\em Automated Reasoning:  Introduction and Applications}
\cite{Wos}\index{Wos, Larry}.  This book discusses
how to use {\sc otter} in a way that can
not only able to generate
relatively efficient proofs, it can even be instructed to search for
shorter proofs.  The effective use of {\sc otter} (and similar tools)
does require a certain
amount of experience, skill, and patience.  The axiom system used in the
\texttt{set.mm}\index{set theory database (\texttt{set.mm})} set theory
database can be expressed to {\sc otter} using a method described in
\cite{Megill}.\index{Megill, Norman}\footnote{To use those axioms with
{\sc otter}, they must be restated in a way that eliminates the need for
``dummy variables.''\index{dummy variable!eliminating} See the Comment
on p.~\pageref{nodd}.} When successful, this method tends to generate
short and clever proofs, but my experiments with it indicate that the
method will find proofs within a reasonable time only for relatively
easy theorems.  It is still fun to experiment with.

Reference \cite{Bledsoe}\index{Bledsoe, W. W.} surveys a number of approaches
people have explored in the field of automated theorem proving\index{automated
theorem proving}.

\subsection{Interactive Theorem Provers}\label{interactivetheoremprovers}

Finding proofs completely automatically is difficult, so there
are some interactive theorem provers that allow a human to guide the
computer to find a proof.
Examples include
HOL Light\index{HOL light}%
\footnote{\url{https://www.cl.cam.ac.uk/~jrh13/hol-light/}.},
Isabelle\index{Isabelle}%
\footnote{\url{http://www.cl.cam.ac.uk/Research/HVG/Isabelle}.},
{\sc hol}\index{hol@{\sc hol}}%
\footnote{\url{https://hol-theorem-prover.org/}.},
and
Coq\index{Coq}\footnote{\url{https://coq.inria.fr/}.}.

A major difference between most of these tools and Metamath is that the
``proofs'' are actually programs that guide the program to find a proof,
and not the proof itself.
For example, an Isabelle/HOL proof might apply a step
\texttt{apply (blast dest: rearrange reduction)}. The \texttt{blast}
instruction applies
an automatic tableux prover and returns if it found a sequence of proof
steps that work... but the sequence is not considered part of the proof.

A good overview of
higher-level proof verification languages (such as {\sc lcf}\index{lcf@{\sc
lcf}} and {\sc hol}\index{hol@{\sc hol}})
is given in \cite{Harrison}.  All of these languages are fundamentally
different from Metamath in that much of the mathematical foundational
knowledge is embedded in the underlying proof-verification program, rather
than placed directly in the database that is being verified.
These can have a steep learning curve for those without a mathematical
background.  For example, one usually must have a fair understanding of
mathematical logic in order to follow their proofs.

\subsection{Proof Verifiers}\label{proofverifiers}

A proof verifier is a program that doesn't generate proofs but instead
verifies proofs that you give it.  Many proof verifiers have limited built-in
automated proof capabilities, such as figuring out simple logical inferences
(while still being guided by a person who provides the overall proof).  Metamath
has no built-in automated proof capability other than the limited
capability of its Proof Assistant.

Proof-verification languages are not used as frequently as they might be.
Pure mathematicians are more concerned with producing new results, and such
detail and rigor would interfere with that goal.  The use of computers in pure
mathematics is primarily focused on automated theorem provers (not verifiers),
again with the ultimate goal of aiding the creation of new mathematics.
Automated theorem provers are usually concerned with attacking one theorem at
time rather than making a large, organized database easily available to the
user.  Metamath is one way to help close this gap.

By itself Metamath is a mostly a proof verifier.
This does not mean that other approaches can't be used; the difference
is that in Metamath, the results of various provers must be recorded
step-by-step so that they can be verified.

Another proof-verification language is Mizar,\index{Mizar} which can display
its proofs in the informal language that mathematicians are accustomed to.
Information on the Mizar language is available at \url{http://mizar.org}.

For the working mathematician, Mizar is an excellent tool for rigorously
documenting proofs. Mizar typesets its proofs in the informal English used by
mathematicians (and, while fine for them, are just as inscrutable by
laypersons!). A price paid for Mizar is a relatively steep learning curve of a
couple of weeks.  Several mathematicians are actively formalizing different
areas of mathematics using Mizar and publishing the proofs in a dedicated
journal. Unfortunately the task of formalizing mathematics is still looked
down upon to a certain extent since it doesn't involve the creation of ``new''
mathematics.

The closest system to Metamath is
the {\em Ghilbert}\index{Ghilbert} proof language (\url{http://ghilbert.org})
system developed by
Raph Levien\index{Levien, Raph}.
Ghilbert is a formal proof checker heavily inspired by Metamath.
Ghilbert statements are s-expressions (a la Lisp), which is easy
for computers to parse but many people find them hard to read.
There are a number of differences in their specific constructs, but
there is at least one tool to translate some Metamath materials into Ghilbert.
As of 2019 the Ghilbert community is smaller and less active than the
Metamath community.
That said, the Metamath and Ghilbert communities overlap, and fruitful
conversations between them have occurred many times over the years.

\subsection{Creating a Database of Formalized Mathematics}\label{mathdatabase}

Besides Metamath, there are several other ongoing projects with the goal of
formalizing mathematics into computer-verifiable databases.
Understanding some history will help.

The {\sc qed}\index{qed project@{\sc qed} project}%
\footnote{\url{http://www-unix.mcs.anl.gov/qed}.}
project arose in 1993 and its goals were outlined in the
{\sc qed} manifesto.
The {\sc qed} manifesto was
a proposal for a computer-based database of all mathematical knowledge,
strictly formalized and with all proofs having been checked automatically.
The project had a conference in 1994 and another in 1995;
there was also a ``twenty years of the {\sc qed} manifesto'' workshop
in 2014.
Its ideals are regularly reraised.

In a 2007 paper, Freek Wiedijk identified two reasons
for the failure of the {\sc qed} project as originally envisioned:%
\cite{Wiedijk-revisited}\index{Wiedijk, Freek}

\begin{itemize}
\item Very few people are working on formalization of mathematics. There is no compelling application for fully mechanized mathematics.
\item Formalized mathematics does not yet resemble traditional mathematics. This is partly due to the complexity of mathematical notation, and partly to the limitations of existing theorem provers and proof assistants.
\end{itemize}

But this did not end the dream of
formalizing mathematics into computer-verifiable databases.
The problems that led to the {\sc qed} manifesto are still with us,
even though the challenges were harder than originally considered.
What has happened instead is that various independent projects have
worked towards formalizing mathematics into computer-verifiable databases,
each simultaneously competing and cooperating with each other.

A concrete way to see this is
Freek Wiedijk's ``Formalizing 100 Theorems'' list%
\footnote{\url{http://www.cs.ru.nl/\%7Efreek/100/}.}
which shows the progress different systems have made on a challenge list
of 100 mathematical theorems.%
\footnote{ This is not the only list of ``interesting'' theorems.
Another interesting list was posted by Oliver Knill's list
\cite{Knill}\index{Knill, Oliver}.}
The top systems as of February 2019
(in order of the number of challenges completed) are
HOL Light, Isabelle, Metamath, Coq, and Mizar.

The Metamath 100%
\footnote{\url{http://us.metamath.org/mm\_100.html}}
page (maintained by David A. Wheeler\index{Wheeler, David A.})
shows the progress of Metamath (specifically its \texttt{set.mm} database)
against this challenge list maintained by Freek Wiedijk.
The Metamath \texttt{set.mm} database
has made a lot of progress over the years,
in part because working to prove those challenge theorems required
defining various terms and proving their properties as a prerequisite.
Here are just a few of the many statements that have been
formally proven with Metamath:

% The entries of this cause the narrow display to break poorly,
% since the short amount of text means LaTeX doesn't get a lot to work with
% and the itemize format gives it even *less* margin than usual.
% No one will mind if we make just this list flushleft, since this list
% will be internally consistent.
\begin{flushleft}
\begin{itemize}
\item 1. The Irrationality of the Square Root of 2
  (\texttt{sqr2irr}, by Norman Megill, 2001-08-20)
\item 2. The Fundamental Theorem of Algebra
  (\texttt{fta}, by Mario Carneiro, 2014-09-15)
\item 22. The Non-Denumerability of the Continuum
  (\texttt{ruc}, by Norman Megill, 2004-08-13)
\item 54. The Konigsberg Bridge Problem
  (\texttt{konigsberg}, by Mario Carneiro, 2015-04-16)
\item 83. The Friendship Theorem
  (\texttt{friendship}, by Alexander W. van der Vekens, 2018-10-09)
\end{itemize}
\end{flushleft}

We thank all of those who have developed at least one of the Metamath 100
proofs, and we particularly thank
Mario Carneiro\index{Carneiro, Mario}
who has contributed the most Metamath 100 proofs as of 2019.
The Metamath 100 page shows the list of all people who have contributed a
proof, and links to graphs and charts showing progress over time.
We encourage others to work on proving theorems not yet proven in Metamath,
since doing so improves the work as a whole.

Each of the math formalization systems (including Metamath)
has different strengths and weaknesses, depending on what you value.
Key aspects that differentiate Metamath from the other top systems are:

\begin{itemize}
\item Metamath is not tied to any particular set of axioms.
\item Metamath can show every step of every proof, no exceptions.
  Most other provers only assert that a proof can be found, and do not
  show every step. This also makes verification fast, because
  the system does not need to rediscover proof details.
\item The Metamath verifier has been re-implemented in many different
  programming languages, so verification can be done by multiple
  implementations.  In particular, the
  \texttt{set.mm}\index{set theory database (\texttt{set.mm})}%
  \index{Metamath Proof Explorer} database is verified by
  four different verifiers
  written in four different languages by four different authors.
  This greatly reduces the risk of accepting an invalid
  proof due to an error in the verifier.
\item Proofs stay proven.  In some systems, changes to the system's
  syntax or how a tactic works causes proofs to fail in later versions,
  causing older work to become essentially lost.
  Metamath's language is
  extremely small and fixed, so once a proof is added to a database,
  the database can be rechecked with later versions of the Metamath program
  and with other verifiers of Metamath databases.
  If an axiom or key definition needs to be changed, it is easy to
  manipulate the database as a whole to handle the change
  without touching the underlying verifier.
  Since re-verification of an entire database takes seconds, there
  is never a reason to delay complete verification.
  This aspect is especially compelling if your
  goal is to have a long-term database of proofs.
\item Licensing is generous.  The main Metamath databases are released to
  the public domain, and the main Metamath program is open source software
  under a standard, widely-used license.
\item Substitutions are easy to understand, even by those who are not
  professional mathematicians.
\end{itemize}

Of course, other systems may have advantages over Metamath
that are more compelling, depending on what you value.
In any case, we hope this helps you understand Metamath
within a wider context.

\subsection{In Summary}\label{computers-summary}

To summarize our discussions of computers and mathematics, computer algebra
systems can be viewed as theorem generators focusing on a narrow realm of
mathematics (numbers and their properties), automated theorem provers as proof
generators for specific theorems in a much broader realm covered by a built-in
formal system such as first-order logic, interactive theorem
provers require human guidance, proof verifiers verify proofs but
historically they have been
restricted to first-order logic.
Metamath, in contrast,
is a proof verifier and documenter whose realm is essentially unlimited.

\section{Mathematics and Metamath}

\subsection{Standard Mathematics}

There are a number of ways that Metamath\index{Metamath} can be used with
standard mathematics.  The most satisfying way philosophically is to start at
the very beginning, and develop the desired mathematics from the axioms of
logic and set theory.\index{set theory}  This is the approach taken in the
\texttt{set.mm}\index{set theory database (\texttt{set.mm})}%
\index{Metamath Proof Explorer}
database (also known as the Metamath Proof Explorer).
Among other things, this database builds up to the
axioms of real and complex numbers\index{analysis}\index{real and complex
numbers} (see Section~\ref{real}), and a standard development of analysis, for
example, could start at that point, using it as a basis.   Besides this
philosophical advantage, there are practical advantages to having all of the
tools of set theory available in the supporting infrastructure.

On the other hand, you may wish to start with the standard axioms of a
mathematical theory without going through the set theoretical proofs of those
axioms.  You will need mathematical logic to make inferences, but if you wish
you can simply introduce theorems\index{theorem} of logic as
``axioms''\index{axiom} wherever you need them, with the implicit assumption
that in principle they can be proved, if they are obvious to you.  If you
choose this approach, you will probably want to review the notation used in
\texttt{set.mm}\index{set theory database (\texttt{set.mm})} so that your own
notation will be consistent with it.

\subsection{Other Formal Systems}
\index{formal system}

Unlike some programs, Metamath\index{Metamath} is not limited to any specific
area of mathematics, nor committed to any particular mathematical philosophy
such as classical logic versus intuitionism, nor limited, say, to expressions
in first-order logic.  Although the database \texttt{set.mm}
describes standard logic and set theory, Meta\-math
is actually a general-purpose language for describing a wide variety of formal
systems.\index{formal system}  Non-standard systems such as modal
logic,\index{modal logic} intuitionist logic\index{intuitionism}, higher-order
logic\index{higher-order logic}, quantum logic\index{quantum logic}, and
category theory\index{category theory} can all be described with the Metamath
language.  You define the symbols you prefer and tell Metamath the axioms and
rules you want to start from, and Metamath will verify any inferences you make
from those axioms and rules.  A simple example of a non-standard formal system
is Hofstadter's\index{Hofstadter, Douglas R.} MIU system,\index{MIU-system}
whose Metamath description is presented in Appendix~\ref{MIU}.

This is not hypothetical.
The largest Metamath database is
\texttt{set.mm}\index{set theory database (\texttt{set.mm}}%
\index{Metamath Proof Explorer}), aka the Metamath Proof Explorer,
which uses the most common axioms for mathematical foundations
(specifically classical logic combined with Zermelo--Fraenkel
set theory\index{Zermelo--Fraenkel set theory} with the Axiom of Choice).
But other Metamath databases are available:

\begin{itemize}
\item The database
  \texttt{iset.mm}\index{intuitionistic logic database (\texttt{iset.mm})},
  aka the
  Intuitionistic Logic Explorer\index{Intuitionistic Logic Explorer},
  uses intuitionistic logic (a constructivist point of view)
  instead of classical logic.
\item The database
  \texttt{nf.mm}\index{New Foundations database (\texttt{nf.mm})},
  aka the
  New Foundations Explorer\index{New Foundations Explorer},
  constructs mathematics from scratch,
  starting from Quine's New Foundations (NF) set theory axioms.
\item The database
  \texttt{hol.mm}\index{Higher-order Logic database (\texttt{hol.mm})},
  aka the
  Higher-Order Logic (HOL) Explorer\index{Higher-Order Logic (HOL) Explorer},
  starts with HOL (also called simple type theory) and derives
  equivalents to ZFC axioms, connecting the two approaches.
\end{itemize}

Since the days of David Hilbert,\index{Hilbert, David} mathematicians have
been concerned with the fact that the metalanguage\index{metalanguage} used to
describe mathematics may be stronger than the mathematics being described.
Metamath\index{Metamath}'s underlying finitary\index{finitary proof},
constructive nature provides a good philosophical basis for studying even the
weakest logics.\index{weak logic}

The usual treatment of many non-standard formal systems\index{formal
system} uses model theory\index{model theory} or proof theory\index{proof
theory} to describe these systems; these theories, in turn, are based on
standard set theory.  In other words, a non-standard formal system is defined
as a set with certain properties, and standard set theory is used to derive
additional properties of this set.  The standard set theory database provided
with Metamath can be used for this purpose, and when used this way
the development of a special
axiom system for the non-standard formal system becomes unnecessary.  The
model- or proof-theoretic approach often allows you to prove much deeper
results with less effort.

Metamath supports both approaches.  You can define the non-standard
formal system directly, or define the non-standard formal system as
a set with certain properties, whichever you find most helpful.

%\section{Additional Remarks}

\subsection{Metamath and Its Philosophy}

Closely related to Metamath\index{Metamath} is a philosophy or way of looking
at mathematics. This philosophy is related to the formalist
philosophy\index{formalism} of Hilbert\index{Hilbert, David} and his followers
\cite[pp.~1203--1208]{Kline}\index{Kline, Morris}
\cite[p.~6]{Behnke}\index{Behnke, H.}. In this philosophy, mathematics is
viewed as nothing more than a set of rules that manipulate symbols, together
with the consequences of those rules.  While the mathematics being described
may be complex, the rules used to describe it (the
``metamathematics''\index{metamathematics}) should be as simple as possible.
In particular, proofs should be restricted to dealing with concrete objects
(the symbols we write on paper rather than the abstract concepts they
represent) in a constructive manner; these are called ``finitary''
proofs\index{finitary proof} \cite[pp.~2--3]{Shoenfield}\index{Shoenfield,
Joseph R.}.

Whether or not you find Metamath interesting or useful will in part depend on
the appeal you find in its philosophy, and this appeal will probably depend on
your particular goals with respect to mathematics.  For example, if you are a
pure mathematician at the forefront of discovering new mathematical knowledge,
you will probably find that the rigid formality of Metamath stifles your
creativity.  On the other hand, we would argue that once this knowledge is
discovered, there are advantages to documenting it in a standard format that
will make it accessible to others.  Sixty years from now, your field may be
dormant, and as Davis and Hersh put it, your ``writings would become less
translatable than those of the Maya'' \cite[p.~37]{Davis}\index{Davis, Phillip
J.}.


\subsection{A History of the Approach Behind Metamath}

Probably the one work that has had the most motivating influence on
Metamath\index{Metamath} is Whitehead and Russell's monumental {\em Principia
Mathematica} \cite{PM}\index{Whitehead, Alfred North}\index{Russell,
Bertrand}\index{principia mathematica@{\em Principia Mathematica}}, whose aim
was to deduce all of mathematics from a small number of primitive ideas, in a
very explicit way that in principle anyone could understand and follow.  While
this work was tremendously influential in its time, from a modern perspective
it suffers from several drawbacks.  Both its notation and its underlying
axioms are now considered dated and are no longer used.  From our point of
view, its development is not really as accessible as we would like to see; for
practical reasons, proofs become more and more sketchy as its mathematics
progresses, and working them out in fine detail requires a degree of
mathematical skill and patience that many people don't have.  There are
numerous small errors, which is understandable given the tedious, technical
nature of the proofs and the lack of a computer to verify the details.
However, even today {\em Principia Mathematica} stands out as the work closest
in spirit to Metamath.  It remains a mind-boggling work, and one can't help
but be amazed at seeing ``$1+1=2$'' finally appear on page 83 of Volume II
(Theorem *110.643).

The origin of the proof notation used by Metamath dates back to the 1950's,
when the logician C.~A.~Meredith expressed his proofs in a compact notation
called ``condensed detachment''\index{condensed detachment}
\cite{Hindley}\index{Hindley, J. Roger} \cite{Kalman}\index{Kalman, J. A.}
\cite{Meredith}\index{Meredith, C. A.} \cite{Peterson}\index{Peterson, Jeremy
George}.  This notation allows proofs to be communicated unambiguously by
merely referencing the axiom\index{axiom}, rule\index{rule}, or
theorem\index{theorem} used at each step, without explicitly indicating the
substitutions\index{substitution!variable}\index{variable substitution} that
have to be made to the variables in that axiom, rule, or theorem.  Ordinarily,
condensed detachment is more or less limited to propositional
calculus\index{propositional calculus}.  The concept has been extended to
first-order logic\index{first-order logic} in \cite{Megill}\index{Megill,
Norman}, making it is easy to write a small computer program to verify proofs
of simple first-order logic theorems.\index{condensed detachment!and
first-order logic}

A key concept behind the notation of condensed detachment is called
``unification,''\index{unification} which is an algorithm for determining what
substitutions\index{substitution!variable}\index{variable substitution} to
variables have to be made to make two expressions match each other.
Unification was first precisely defined by the logician J.~A.~Robinson, who
used it in the development of a powerful
theorem-proving technique called the ``resolution principle''
\cite{Robinson}\index{Robinson's resolution principle}. Metamath does not make
use of the resolution principle, which is intended for systems of first-order
logic.\index{first-order logic}  Metamath's use is not restricted to
first-order logic, and as we have mentioned it does not automatically discover
proofs.  However, unification is a key idea behind Metamath's proof
notation, and Metamath makes use of a very simple version of it
(Section~\ref{unify}).

\subsection{Metamath and First-Order Logic}

First-order logic\index{first-order logic} is the supporting structure
for standard mathematics.  On top of it is set theory, which contains
the axioms from which virtually all of mathematics can be derived---a
remarkable fact.\index{category
theory}\index{cardinal, inaccessible}\label{categoryth}\footnote{An exception seems
to be category theory.  There are several schools of thought on whether
category theory is derivable from set theory.  At a minimum, it appears
that an additional axiom is needed that asserts the existence of an
``inaccessible cardinal'' (a type of infinity so large that standard set
theory can't prove or deny that it exists).
%
%%%% (I took this out that was in previous editions:)
% But it is also argued that not just one but a ``proper class'' of them
% is needed, and the existence of proper classes is impossible in standard
% set theory.  (A proper class is a collection of sets so huge that no set
% can contain it as an element.  Proper classes can lead to
% inconsistencies such as ``Russell's paradox.''  The axioms of standard
% set theory are devised so as to deny the existence of proper classes.)
%
For more information, see
\cite[pp.~328--331]{Herrlich}\index{Herrlich, Horst} and
\cite{Blass}\index{Blass, Andrea}.}

One of the things that makes Metamath\index{Metamath} more practical for
first-order theories is a set of axioms for first-order logic designed
specifically with Metamath's approach in mind.  These are included in
the database \texttt{set.mm}\index{set theory database (\texttt{set.mm})}.
See Chapter~\ref{fol} for a detailed
description; the axioms are shown in Section~\ref{metaaxioms}.  While
logically equivalent to standard axiom systems, our axiom system breaks
up the standard axioms into smaller pieces such that from them, you can
directly derive what in other systems can only be derived as higher-level
``metatheorems.''\index{metatheorem}  In other words, it is more powerful than
the standard axioms from a metalogical point of view.  A rigorous
justification for this system and its ``metalogical
completeness''\index{metalogical completeness} is found in
\cite{Megill}\index{Megill, Norman}.  The system is closely related to a
system developed by Monk\index{Monk, J. Donald} and Tarski\index{Tarski,
Alfred} in 1965 \cite{Monks}.

For example, the formula $\exists x \, x = y $ (given $y$, there exists some
$x$ equal to it) is a theorem of logic,\footnote{Specifically, it is a theorem
of those systems of logic that assume non-empty domains.  It is not a theorem
of more general systems that include the empty domain\index{empty domain}, in
which nothing exists, period!  Such systems are called ``free
logics.''\index{free logic} For a discussion of these systems, see
\cite{Leblanc}\index{Leblanc, Hugues}.  Since our use for logic is as a basis
for set theory, which has a non-empty domain, it is more convenient (and more
traditional) to use a less general system.  An interesting curiosity is that,
using a free logic as a basis for Zermelo--Fraenkel set
theory\index{Zermelo--Fraenkel set theory} (with the redundant Axiom of the
Null Set omitted),\index{Axiom of the Null Set} we cannot even prove the
existence of a single set without assuming the axiom of infinity!\index{Axiom
of Infinity}} whether or not $x$ and $y$ are distinct variables\index{distinct
variables}.  In many systems of logic, we would have to prove two theorems to
arrive at this result.  First we would prove ``$\exists x \, x = x $,'' then
we would separately prove ``$\exists x \, x = y $, where $x$ and $y$ are
distinct variables.''  We would then combine these two special cases ``outside
of the system'' (i.e.\ in our heads) to be able to claim, ``$\exists x \, x =
y $, regardless of whether $x$ and $y$ are distinct.''  In other words, the
combination of the two special cases is a
metatheorem.  In the system of logic
used in Metamath's set theory\index{set theory database (\texttt{set.mm})}
database, the axioms of logic are broken down into small pieces that allow
them to be reassembled in such a way that theorems such as these can be proved
directly.

Breaking down the axioms in this way makes them look peculiar and not very
intuitive at first, but rest assured that they are correct and complete.  Their
correctness is ensured because they are theorem schemes of standard first-order
logic (which you can easily verify if you are a logician).  Their completeness
follows from the fact that we explicitly derive the standard axioms of
first-order logic as theorems.  Deriving the standard axioms is somewhat
tricky, but once we're there, we have at our disposal a system that is less
awkward to work with in formal proofs\index{formal proof}.  In technical terms
that logicians understand, we eliminate the cumbersome concepts of ``free
variable,''\index{free variable} ``bound variable,''\index{bound variable} and
``proper substitution''\index{proper substitution}\index{substitution!proper}
as primitive notions.  These concepts are present in our system but are
defined in terms of concepts expressed by the axioms and can be eliminated in
principle.  In standard systems, these concepts are really like additional,
implicit axioms\index{implicit axiom} that are somewhat complex and cannot be
eliminated.

The traditional approach to logic, wherein free variables and proper
substitution is defined, is also possible to do directly in the Metamath
language.  However, the notation tends to become awkward, and there are
disadvantages:  for example, extending the definition of a wff with a
definition is awkward, because the free variable and proper substitution
concepts have to have their definitions also extended.  Our choice of
axioms for \texttt{set.mm} is to a certain extent a matter of style, in
an attempt to achieve overall simplicity, but you should also be aware
that the traditional approach is possible as well if you should choose
it.

\chapter{Using the Metamath Program}
\label{using}

\section{Installation}

The way that you install Metamath\index{Metamath!installation} on your
computer system will vary for different computers.  Current
instructions are provided with the Metamath program download at
\url{http://metamath.org}.  In general, the installation is simple.
There is one file containing the Metamath program itself.  This file is
usually called \texttt{metamath} or \texttt{metamath.}{\em xxx} where
{\em xxx} is the convention (such as \texttt{exe}) for an executable
program on your operating system.  There are several additional files
containing samples of the Metamath language, all ending with
\texttt{.mm}.  The file \texttt{set.mm}\index{set theory database
(\texttt{set.mm})} contains logic and set theory and can be used as a
starting point for other areas of mathematics.

You will also need a text editor\index{text editor} capable of editing plain
{\sc ascii}\footnote{American Standard Code for Information Interchange.} text
in order to prepare your input files.\index{ascii@{\sc ascii}}  Most computers
have this capability built in.  Note that plain text is not necessarily the
default for some word processing programs\index{word processor}, especially if
they can handle different fonts; for example, with Microsoft Word\index{Word
(Microsoft)}, you must save the file in the format ``Text Only With Line
Breaks'' to get a plain text\index{plain text} file.\footnote{It is
recommended that all lines in a Metamath source file be 79 characters or less
in length for compatibility among different computer terminals.  When creating
a source file on an editor such as Word, select a monospaced
font\index{monospaced font} such as Courier\index{Courier font} or
Monaco\index{Monaco font} to make this easier to achieve.  Better yet,
just use a plain text editor such as Notepad.}

On some computer systems, Metamath does not have the capability to print
its output directly; instead, you send its output to a file (using the
\texttt{open} commands described later).  The way you print this output
file depends on your computer.\index{printers} Some computers have a
print command, whereas with others, you may have to read the file into
an editor and print it from there.

If you want to print your Metamath source files with typeset formulas
containing standard mathematical symbols, you will need the \LaTeX\
typesetting program\index{latex@{\LaTeX}}, which is widely and freely
available for most operating systems.  It runs natively on Unix and
Linux, and can be installed on Windows as part of the free Cygwin
package (\url{http://cygwin.com}).

You can also produce {\sc html}\footnote{HyperText Markup Language.}
web pages.  The {\tt help html} command in the Metamath program will
assist you with this feature.

\section{Your First Formal System}\label{start}
\subsection{From Nothing to Zero}\label{startf}

To give you a feel for what the Metamath\index{Metamath} language looks like,
we will take a look at a very simple example from formal number
theory\index{number theory}.  This example is taken from
Mendelson\index{Mendelson, Elliot} \cite[p. 123]{Mendelson}.\footnote{To keep
the example simple, we have changed the formalism slightly, and what we call
axioms\index{axiom} are strictly speaking theorems\index{theorem} in
\cite{Mendelson}.}  We will look at a small subset of this theory, namely that
part needed for the first number theory theorem proved in \cite{Mendelson}.

First we will look at a standard formal proof\index{formal proof} for the
example we have picked, then we will look at the Metamath version.  If you
have never been exposed to formal proofs, the notation may seem to be such
overkill to express such simple notions that you may wonder if you are missing
something.  You aren't.  The concepts involved are in fact very simple, and a
detailed breakdown in this fashion is necessary to express the proof in a way
that can be verified mechanically.  And as you will see, Metamath breaks the
proof down into even finer pieces so that the mechanical verification process
can be about as simple as possible.

Before we can introduce the axioms\index{axiom} of the theory, we must define
the syntax rules for forming legal expressions\index{syntax rules}
(combinations of symbols) with which those axioms can be used. The number 0 is
a {\bf term}\index{term}; and if $ t$ and $r$ are terms, so is $(t+r)$. Here,
$ t$ and $r$ are ``metavariables''\index{metavariable} ranging over terms; they
themselves do not appear as symbols in an actual term.  Some examples of
actual terms are $(0 + 0)$ and $((0+0)+0)$.  (Note that our theory describes
only the number zero and sums of zeroes.  Of course, not much can be done with
such a trivial theory, but remember that we have picked a very small subset of
complete number theory for our example.  The important thing for you to focus
on is our definitions that describe how symbols are combined to form valid
expressions, and not on the content or meaning of those expressions.) If $ t$
and $r$ are terms, an expression of the form $ t=r$ is a {\bf wff}
(well-formed formula)\index{well-formed formula (wff)}; and if $P$ and $Q$ are
wffs, so is $(P\rightarrow Q)$ (which means ``$P$ implies
$Q$''\index{implication ($\rightarrow$)} or ``if $P$ then $Q$'').
Here $P$ and $Q$ are metavariables ranging over wffs.  Examples of actual
wffs are $0=0$, $(0+0)=0$, $(0=0 \rightarrow (0+0)=0)$, and $(0=0\rightarrow
(0=0\rightarrow 0=(0+0)))$.  (Our notation makes use of more parentheses than
are customary, but the elimination of ambiguity this way simplifies our
example by avoiding the need to define operator precedence\index{operator
precedence}.)

The {\bf axioms}\index{axiom} of our theory are all wffs of the following
form, where $ t$, $r$, and $s$ are any terms:

%Latex p. 92
\renewcommand{\theequation}{A\arabic{equation}}

\begin{equation}
(t=r\rightarrow (t=s\rightarrow r=s))
\end{equation}
\begin{equation}
(t+0)=t
\end{equation}

Note that there are an infinite number of axioms since there are an infinite
number of possible terms.  A1 and A2 are properly called ``axiom
schemes,''\index{axiom scheme} but we will refer to them as ``axioms'' for
brevity.

An axiom is a {\bf theorem}; and if $P$ and $(P\rightarrow Q)$ are theorems
(where $P$ and $Q$ are wffs), then $Q$ is also a theorem.\index{theorem}  The
second part of this definition is called the modus ponens (MP) rule of
inference\index{inference rule}\index{modus ponens}.  It allows us to obtain
new theorems from old ones.

The {\bf proof}\index{proof} of a theorem is a sequence of one or more
theorems, each of which is either an axiom or the result of modus ponens
applied to two previous theorems in the sequence, and the last of which is the
theorem being proved.

The theorem we will prove for our example is very simple:  $ t=t$.  The proof of
our theorem follows.  Study it carefully until you feel sure you
understand it.\label{zeroproof}

% Use tabu so that lines will wrap automatically as needed.
\begin{tabu} { l X X }
1. & $(t+0)=t$ & (by axiom A2) \\
2. & $(t+0)=t$ & (by axiom A2) \\
3. & $((t+0)=t \rightarrow ((t+0)=t\rightarrow t=t))$ & (by axiom A1) \\
4. & $((t+0)=t\rightarrow t=t)$ & (by MP applied to steps 2 and 3) \\
5. & $t=t$ & (by MP applied to steps 1 and 4) \\
\end{tabu}

(You may wonder why step 1 is repeated twice.  This is not necessary in the
formal language we have defined, but in Metamath's ``reverse Polish
notation''\index{reverse Polish notation (RPN)} for proofs, a previous step
can be referred to only once.  The repetition of step~1 here will enable you
to see more clearly the correspondence of this proof with the
Metamath\index{Metamath} version on p.~\pageref{demoproof}.)

Our theorem is more properly called a ``theorem scheme,''\index{theorem
scheme} for it represents an infinite number of theorems, one for each
possible term $ t$.  Two examples of actual theorems would be $0=0$ and
$(0+0)=(0+0)$.  Rarely do we prove actual theorems, since by proving schemes
we can prove an infinite number of theorems in one fell swoop.  Similarly, our
proof should really be called a ``proof scheme.''\index{proof scheme}  To
obtain an actual proof, pick an actual term to use in place of $ t$, and
substitute it for $ t$ throughout the proof.

Let's discuss what we have done here.  The axioms\index{axiom} of our theory,
A1 and A2, are trivial and obvious.  Everyone knows that adding zero to
something doesn't change it, and also that if two things are equal to a third,
then they are equal to each other. In fact, stating the trivial and obvious is
a goal to strive for in any axiomatic system.  From trivial and obvious truths
that everyone agrees upon, we can prove results that are not so obvious yet
have absolute faith in them.  If we trust the axioms and the rules, we must,
by definition, trust the consequences of those axioms and rules, if logic is
to mean anything at all.

Our rule of inference\index{rule}, modus ponens\index{modus ponens}, is also
pretty obvious once you understand what it means.  If we prove a fact $P$, and
we also prove that $P$ implies $Q$, then $Q$ necessarily follows as a new
fact.  The rule provides us with a means for obtaining new facts (i.e.\
theorems\index{theorem}) from old ones.

The theorem that we have proved, $ t=t$, is so fundamental that you may wonder
why it isn't one of the axioms\index{axiom}.  In some axiom systems of
arithmetic, it {\em is} an axiom.  The choice of axioms in a theory is to some
extent arbitrary and even an art form, constrained only by the requirement
that any two equivalent axiom systems be able to derive each other as
theorems.  We could imagine that the inventor of our axiom system originally
included $ t=t$ as an axiom, then discovered that it could be derived as a
theorem from the other axioms.  Because of this, it was not necessary to
keep it as an axiom.  By eliminating it, the final set of axioms became
that much simpler.

Unless you have worked with formal proofs\index{formal proof} before, it
probably wasn't apparent to you that $ t=t$ could be derived from our two
axioms until you saw the proof. While you certainly believe that $ t=t$ is
true, you might not be able to convince an imaginary skeptic who believes only
in our two axioms until you produce the proof.  Formal proofs such as this are
hard to come up with when you first start working with them, but after you get
used to them they can become interesting and fun.  Once you understand the
idea behind formal proofs you will have grasped the fundamental principle that
underlies all of mathematics.  As the mathematics becomes more sophisticated,
its proofs become more challenging, but ultimately they all can be broken down
into individual steps as simple as the ones in our proof above.

Mendelson's\index{Mendelson, Elliot} book, from which our example was taken,
contains a number of detailed formal proofs such as these, and you may be
interested in looking it up.  The book is intended for mathematicians,
however, and most of it is rather advanced.  Popular literature describing
formal proofs\index{formal proof} include \cite[p.~296]{Rucker}\index{Rucker,
Rudy} and \cite[pp.~204--230]{Hofstadter}\index{Hofstadter, Douglas R.}.

\subsection{Converting It to Metamath}\label{convert}

Formal proofs\index{formal proof} such as the one in our example break down
logical reasoning into small, precise steps that leave little doubt that the
results follow from the axioms\index{axiom}.  You might think that this would
be the finest breakdown we can achieve in mathematics.  However, there is more
to the proof than meets the eye. Although our axioms were rather simple, a lot
of verbiage was needed before we could even state them:  we needed to define
``term,'' ``wff,'' and so on.  In addition, there are a number of implied
rules that we haven't even mentioned. For example, how do we know that step 3
of our proof follows from axiom A1? There is some hidden reasoning involved in
determining this.  Axiom A1 has two occurrences of the letter $ t$.  One of
the implied rules states that whatever we substitute for $ t$ must be a legal
term\index{term}.\footnote{Some authors make this implied rule explicit by
stating, ``only expressions of the above form are terms,'' after defining
``term.''}  The expression $ t+0$ is pretty obviously a legal term whenever $
t$ is, but suppose we wanted to substitute a huge term with thousands of
symbols?  Certainly a lot of work would be involved in determining that it
really is a term, but in ordinary formal proofs all of this work would be
considered a single ``step.''

To express our axiom system in the Metamath\index{Metamath} language, we must
describe this auxiliary information in addition to the axioms themselves.
Metamath does not know what a ``term'' or a ``wff''\index{well-formed formula
(wff)} is.  In Metamath, the specification of the ways in which we can combine
symbols to obtain terms and wffs are like little axioms in themselves.  These
auxiliary axioms are expressed in the same notation as the ``real''
axioms\index{axiom}, and Metamath does not distinguish between the two.  The
distinction is made by you, i.e.\ by the way in which you interpret the
notation you have chosen to express these two kinds of axioms.

The Metamath language breaks down mathematical proofs into tiny pieces, much
more so than in ordinary formal proofs\index{formal proof}.  If a single
step\index{proof step} involves the
substitution\index{substitution!variable}\index{variable substitution} of a
complex term for one of its variables, Metamath must see this single step
broken down into many small steps.  This fine-grained breakdown is what gives
Metamath generality and flexibility as it lets it not be limited to any
particular mathematical notation.

Metamath's proof notation is not, in itself, intended to be read by humans but
rather is in a compact format intended for a machine.  The Metamath program
will convert this notation to a form you can understand, using the \texttt{show
proof}\index{\texttt{show proof} command} command.  You can tell the program what
level of detail of the proof you want to look at.  You may want to look at
just the logical inference steps that correspond
to ordinary formal proof steps,
or you may want to see the fine-grained steps that prove that an expression is
a term.

Here, without further ado, is our example converted to the
Metamath\index{Metamath} language:\index{metavariable}\label{demo0}

\begin{verbatim}
$( Declare the constant symbols we will use $)
    $c 0 + = -> ( ) term wff |- $.
$( Declare the metavariables we will use $)
    $v t r s P Q $.
$( Specify properties of the metavariables $)
    tt $f term t $.
    tr $f term r $.
    ts $f term s $.
    wp $f wff P $.
    wq $f wff Q $.
$( Define "term" and "wff" $)
    tze $a term 0 $.
    tpl $a term ( t + r ) $.
    weq $a wff t = r $.
    wim $a wff ( P -> Q ) $.
$( State the axioms $)
    a1 $a |- ( t = r -> ( t = s -> r = s ) ) $.
    a2 $a |- ( t + 0 ) = t $.
$( Define the modus ponens inference rule $)
    ${
       min $e |- P $.
       maj $e |- ( P -> Q ) $.
       mp  $a |- Q $.
    $}
$( Prove a theorem $)
    th1 $p |- t = t $=
  $( Here is its proof: $)
       tt tze tpl tt weq tt tt weq tt a2 tt tze tpl
       tt weq tt tze tpl tt weq tt tt weq wim tt a2
       tt tze tpl tt tt a1 mp mp
     $.
\end{verbatim}\index{metavariable}

A ``database''\index{database} is a set of one or more {\sc ascii} source
files.  Here's a brief description of this Metamath\index{Metamath} database
(which consists of this single source file), so that you can understand in
general terms what is going on.  To understand the source file in detail, you
should read Chapter~\ref{languagespec}.

The database is a sequence of ``tokens,''\index{token} which are normally
separated by spaces or line breaks.  The only tokens that are built into
the Metamath language are those beginning with \texttt{\$}.  These tokens
are called ``keywords.''\index{keyword}  All other tokens are
user-defined, and their names are arbitrary.

As you might have guessed, the Metamath token \texttt{\$(}\index{\texttt{\$(} and
\texttt{\$)} auxiliary keywords} starts a comment and \texttt{\$)} ends a comment.

The Metamath tokens \texttt{\$c}\index{\texttt{\$c} statement},
\texttt{\$v}\index{\texttt{\$v} statement},
\texttt{\$e}\index{\texttt{\$e} statement},
\texttt{\$f}\index{\texttt{\$f} statement},
\texttt{\$a}\index{\texttt{\$a} statement}, and
\texttt{\$p}\index{\texttt{\$p} statement} specify ``statements'' that
end with \texttt{\$.}\,.\index{\texttt{\$.}\ keyword}

The Metamath tokens \texttt{\$c} and \texttt{\$v} each declare\index{constant
declaration}\index{variable declaration} a list of user-defined tokens, called
``math symbols,''\index{math symbol} that the database will reference later
on.  All of the math symbols they define you have seen earlier except the
turnstile symbol \texttt{|-} ($\vdash$)\index{turnstile ({$\,\vdash$})}, which is
commonly used by logicians to mean ``a proof exists for.''  For us
the turnstile is just a
convenient symbol that distinguishes expressions that are axioms\index{axiom}
or theorems\index{theorem} from expressions that are terms or wffs.

The \texttt{\$c} statement declares ``constants''\index{constant} and
the \texttt{\$v} statement declares
``variables''\index{variable}\index{constant declaration}\index{variable
declaration} (or more precisely, metavariables\index{metavariable}).  A
variable may be substituted\index{substitution!variable}\index{variable
substitution} with sequences of math symbols whereas a constant may not
be substituted with anything.

It may seem redundant to require both \texttt{\$c}\index{\texttt{\$c} statement} and
\texttt{\$v}\index{\texttt{\$v} statement} statements (since any math
symbol\index{math symbol} not specified with a \texttt{\$c} statement could be
presumed to be a variable), but this provides for better error checking and
also allows math symbols to be redeclared\index{redeclaration of symbols}
(Section~\ref{scoping}).

The token \texttt{\$f}\index{\texttt{\$f} statement} specifies a
statement called a ``variable-type hypothesis'' (also called a
``floating hypothesis'') and \texttt{\$e}\index{\texttt{\$e} statement}
specifies a ``logical hypothesis'' (also called an ``essential
hypothesis'').\index{hypothesis}\index{variable-type
hypothesis}\index{logical hypothesis}\index{floating
hypothesis}\index{essential hypothesis} The token
\texttt{\$a}\index{\texttt{\$a} statement} specifies an ``axiomatic
assertion,''\index{axiomatic assertion} and
\texttt{\$p}\index{\texttt{\$p} statement} specifies a ``provable
assertion.''\index{provable assertion} To the left of each occurrence of
these four tokens is a ``label''\index{label} that identifies the
hypothesis or assertion for later reference.  For example, the label of
the first axiomatic assertion is \texttt{tze}.  A \texttt{\$f} statement
must contain exactly two math symbols, a constant followed by a
variable.  The \texttt{\$e}, \texttt{\$a}, and \texttt{\$p} statements
each start with a constant followed by, in general, an arbitrary
sequence of math symbols.

Associated with each assertion\index{assertion} is a set of hypotheses
that must be satisfied in order for the assertion to be used in a proof.
These are called the ``mandatory hypotheses''\index{mandatory
hypothesis} of the assertion.  Among those hypotheses whose ``scope''
(described below) includes the assertion, \texttt{\$e} hypotheses are
always mandatory and \texttt{\$f}\index{\texttt{\$f} statement}
hypotheses are mandatory when they share their variable with the
assertion or its \texttt{\$e} hypotheses.  The exact rules for
determining which hypotheses are mandatory are described in detail in
Sections~\ref{frames} and \ref{scoping}.  For example, the mandatory
hypotheses of assertion \texttt{tpl} are \texttt{tt} and \texttt{tr},
whereas assertion \texttt{tze} has no mandatory hypotheses because it
contains no variables and has no \texttt{\$e}\index{\texttt{\$e}
statement} hypothesis.  Metamath's \texttt{show statement}
command\index{\texttt{show statement} command}, described in the next
section, will show you a statement's mandatory hypotheses.

Sometimes we need to make a hypothesis relevant to only certain
assertions.  The set of statements to which a hypothesis is relevant is
called its ``scope.''  The Metamath brackets,
\texttt{\$\char`\{}\index{\texttt{\$\char`\{} and \texttt{\$\char`\}}
keywords} and \texttt{\$\char`\}}, define a ``block''\index{block} that
delimits the scope of any hypothesis contained between them.  The
assertion \texttt{mp} has mandatory hypotheses \texttt{wp}, \texttt{wq},
\texttt{min}, and \texttt{maj}.  The only mandatory hypothesis of
\texttt{th1}, on the other hand, is \texttt{tt}, since \texttt{th1}
occurs outside of the block containing \texttt{min} and \texttt{maj}.

Note that \texttt{\$\char`\{} and \texttt{\$\char`\}} do not affect the
scope of assertions (\texttt{\$a} and \texttt{\$p}).  Assertions are always
available to be referenced by any later proof in the source file.

Each provable assertion (\texttt{\$p}\index{\texttt{\$p} statement}
statement) has two parts.  The first part is the
assertion\index{assertion} itself, which is a sequence of math
symbol\index{math symbol} tokens placed between the \texttt{\$p} token
and a \texttt{\$=}\index{\texttt{\$=} keyword} token.  The second part
is a ``proof,'' which is a list of label tokens placed between the
\texttt{\$=} token and the \texttt{\$.}\index{\texttt{\$.}\ keyword}\
token that ends the statement.\footnote{If you've looked at the
\texttt{set.mm} database, you may have noticed another notation used for
proofs.  The other notation is called ``compressed.''\index{compressed
proof}\index{proof!compressed} It reduces the amount of space needed to
store a proof in the database and is described in
Appendix~\ref{compressed}.  In the example above, we use
``normal''\index{normal proof}\index{proof!normal} notation.} The proof
acts as a series of instructions to the Metamath program, telling it how
to build up the sequence of math symbols contained in the assertion part of
the \texttt{\$p} statement, making use of the hypotheses of the
\texttt{\$p} statement and previous assertions.  The construction takes
place according to precise rules.  If the list of labels in the proof
causes these rules to be violated, or if the final sequence that results
does not match the assertion, the Metamath program will notify you with
an error message.

If you are familiar with reverse Polish notation (RPN), which is sometimes used
on pocket calculators, here in a nutshell is how a proof works.  Each
hypothesis label\index{hypothesis label} in the proof is pushed\index{push}
onto the RPN stack\index{stack}\index{RPN stack} as it is encountered. Each
assertion label\index{assertion label} pops\index{pop} off the stack as many
entries as the referenced assertion has mandatory hypotheses.  Variable
substitutions\index{substitution!variable}\index{variable substitution} are
computed which, when made to the referenced assertion's mandatory hypotheses,
cause these hypotheses to match the stack entries. These same substitutions
are then made to the variables in the referenced assertion itself, which is
then pushed onto the stack.  At the end of the proof, there should be one
stack entry, namely the assertion being proved.  This process is explained in
detail in Section~\ref{proof}.

Metamath's proof notation is not very readable for humans, but it allows the
proof to be stored compactly in a file.  The Metamath\index{Metamath} program
has proof display features that let you see what's going on in a more
readable way, as you will see in the next section.

The rules used in verifying a proof are not based on any built-in syntax of the
symbol sequence in an assertion\index{assertion} nor on any built-in meanings
attached to specific symbol names.  They are based strictly on symbol
matching:  constants\index{constant} must match themselves, and
variables\index{variable} may be replaced with anything that allows a match to
occur.  For example, instead of \texttt{term}, \texttt{0}, and \verb$|-$ we could
have just as well used \texttt{yellow}, \texttt{zero}, and \texttt{provable}, as long
as we did so consistently throughout the database.  Also, we could have used
\texttt{is provable} (two tokens) instead of \verb$|-$ (one token) throughout the
database.  In each of these cases, the proof would be exactly the same.  The
independence of proofs and notation means that you have a lot of flexibility to
change the notation you use without having to change any proofs.

\section{A Trial Run}\label{trialrun}

Now you are ready to try out the Metamath\index{Metamath} program.

On all computer systems, Metamath has a standard ``command line
interface'' (CLI)\index{command line interface (CLI)} that allows you to
interact with it.  You supply commands to the CLI by typing them on the
keyboard and pressing your keyboard's {\em return} key after each line
you enter.  The CLI is designed to be easy to use and has built-in help
features.

The first thing you should do is to use a text editor to create a file
called \texttt{demo0.mm} and type into it the Metamath source shown on
p.~\pageref{demo0}.  Actually, this file is included with your Metamath
software package, so check that first.  If you type it in, make sure
that you save it in the form of ``plain {\sc ascii} text with line
breaks.''  Most word processors will have this feature.

Next you must run the Metamath program.  Depending on your computer
system and how Metamath is installed, this could range from clicking the
mouse on the Metamath icon to typing \texttt{run metamath} to typing
simply \texttt{metamath}.  (Metamath's {\tt help invoke} command describes
alternate ways of invoking the Metamath program.)

When you first enter Metamath\index{Metamath}, it will be at the CLI, waiting
for your input. You will see something like the following on your screen:
\begin{verbatim}
Metamath - Version 0.172 25-Jan-2019
Type HELP for help, EXIT to exit.
MM>
\end{verbatim}
The \texttt{MM>} prompt means that Metamath is waiting for a command.
Command keywords\index{command keyword} are not case sensitive;
we will use lower-case commands in our examples.
The version number and its release date will probably be different on your
system from the one we show above.

The first thing that you need to do is to read in your
database:\index{\texttt{read} command}\footnote{If a directory path is
needed on Unix,\index{Unix file names}\index{file names!Unix} you should
enclose the path/file name in quotes to prevent Metamath from thinking
that the \texttt{/} in the path name is a command qualifier, e.g.,
\texttt{read \char`\"db/set.mm\char`\"}.  Quotes are optional when there
is no ambiguity.}
\begin{verbatim}
MM> read demo0.mm
\end{verbatim}
Remember to press the {\em return} key after entering this command.  If
you omit the file name, Metamath will prompt you for one.   The syntax for
specifying a Macintosh file name path is given in a footnote on
p.~\pageref{includef}.\index{Macintosh file names}\index{file
names!Macintosh}

If there are any syntax errors in the database, Metamath will let you know
when it reads in the file.  The one thing that Metamath does not check when
reading in a database is that all proofs are correct, because this would
slow it down too much.  It is a good idea to periodically verify the proofs in
a database you are making changes to.  To do this, use the following command
(and do it for your \texttt{demo0.mm} file now).  Note that the \texttt{*} is a
``wild card'' meaning all proofs in the file.\index{\texttt{verify proof} command}
\begin{verbatim}
MM> verify proof *
\end{verbatim}
Metamath will report any proofs that are incorrect.

It is often useful to save the information that the Metamath program displays
on the screen. You can save everything that happens on the screen by opening a
log file. You may want to do this before you read in a database so that you
can examine any errors later on.  To open a log file, type
\begin{verbatim}
MM> open log abc.log
\end{verbatim}
This will open a file called \texttt{abc.log}, and everything that appears on the
screen from this point on will be stored in this file.  The name of the log file
is arbitrary. To close the log file, type
\begin{verbatim}
MM> close log
\end{verbatim}

Several commands let you examine what's inside your database.
Section~\ref{exploring} has an overview of some useful ones.  The
\texttt{show labels} command lets you see what statement
labels\index{label} exist.  A \texttt{*} matches any combination of
characters, and \texttt{t*} refers to all labels starting with the
letter \texttt{t}.\index{\texttt{show labels} command} The \texttt{/all}
is a ``command qualifier''\index{command qualifier} that tells Metamath
to include labels of hypotheses.  (To see the syntax explained, type
\texttt{help show labels}.)  Type
\begin{verbatim}
MM> show labels t* /all
\end{verbatim}
Metamath will respond with
\begin{verbatim}
The statement number, label, and type are shown.
3 tt $f       4 tr $f       5 ts $f       8 tze $a
9 tpl $a      19 th1 $p
\end{verbatim}

You can use the \texttt{show statement} command to get information about a
particular statement.\index{\texttt{show statement} command}
For example, you can get information about the statement with label \texttt{mp}
by typing
\begin{verbatim}
MM> show statement mp /full
\end{verbatim}
Metamath will respond with
\begin{verbatim}
Statement 17 is located on line 43 of the file
"demo0.mm".
"Define the modus ponens inference rule"
17 mp $a |- Q $.
Its mandatory hypotheses in RPN order are:
  wp $f wff P $.
  wq $f wff Q $.
  min $e |- P $.
  maj $e |- ( P -> Q ) $.
The statement and its hypotheses require the
      variables:  Q P
The variables it contains are:  Q P
\end{verbatim}
The mandatory hypotheses\index{mandatory hypothesis} and their
order\index{RPN order} are
useful to know when you are trying to understand or debug a proof.

Now you are ready to look at what's really inside our proof.  First, here is
how to look at every step in the proof---not just the ones corresponding to an
ordinary formal proof\index{formal proof}, but also the ones that build up the
formulas that appear in each ordinary formal proof step.\index{\texttt{show
proof} command}
\begin{verbatim}
MM> show proof th1 /lemmon /all
\end{verbatim}

This will display the proof on the screen in the following format:
\begin{verbatim}
 1 tt            $f term t
 2 tze           $a term 0
 3 1,2 tpl       $a term ( t + 0 )
 4 tt            $f term t
 5 3,4 weq       $a wff ( t + 0 ) = t
 6 tt            $f term t
 7 tt            $f term t
 8 6,7 weq       $a wff t = t
 9 tt            $f term t
10 9 a2          $a |- ( t + 0 ) = t
11 tt            $f term t
12 tze           $a term 0
13 11,12 tpl     $a term ( t + 0 )
14 tt            $f term t
15 13,14 weq     $a wff ( t + 0 ) = t
16 tt            $f term t
17 tze           $a term 0
18 16,17 tpl     $a term ( t + 0 )
19 tt            $f term t
20 18,19 weq     $a wff ( t + 0 ) = t
21 tt            $f term t
22 tt            $f term t
23 21,22 weq     $a wff t = t
24 20,23 wim     $a wff ( ( t + 0 ) = t -> t = t )
25 tt            $f term t
26 25 a2         $a |- ( t + 0 ) = t
27 tt            $f term t
28 tze           $a term 0
29 27,28 tpl     $a term ( t + 0 )
30 tt            $f term t
31 tt            $f term t
32 29,30,31 a1   $a |- ( ( t + 0 ) = t -> ( ( t + 0 )
                                     = t -> t = t ) )
33 15,24,26,32 mp  $a |- ( ( t + 0 ) = t -> t = t )
34 5,8,10,33 mp  $a |- t = t
\end{verbatim}

The \texttt{/lemmon} command qualifier specifies what is known as a Lemmon-style
display\index{Lemmon-style proof}\index{proof!Lemmon-style}.  Omitting the
\texttt{/lemmon} qualifier results in a tree-style proof (see
p.~\pageref{treeproof} for an example) that is somewhat less explicit but
easier to follow once you get used to it.\index{tree-style
proof}\index{proof!tree-style}

The first number on each line is the step
number of the proof.  Any numbers that follow are step numbers assigned to the
hypotheses of the statement referenced by that step.  Next is the label of
the statement referenced by the step.  The statement type of the statement
referenced comes next, followed by the math symbol\index{math symbol} string
constructed by the proof up to that step.

The last step, 34, contains the statement that is being proved.

Looking at a small piece of the proof, notice that steps 3 and 4 have
established that
\texttt{( t + 0 )} and \texttt{t} are \texttt{term}\,s, and step 5 makes use of steps 3 and
4 to establish that \texttt{( t + 0 ) = t} is a \texttt{wff}.  Let Metamath
itself tell us in detail what is happening in step 5.  Note that the
``target hypothesis'' refers to where step 5 is eventually used, i.e., in step
34.
\begin{verbatim}
MM> show proof th1 /detailed_step 5
Proof step 5:  wp=weq $a wff ( t + 0 ) = t
This step assigns source "weq" ($a) to target "wp"
($f).  The source assertion requires the hypotheses
"tt" ($f, step 3) and "tr" ($f, step 4).  The parent
assertion of the target hypothesis is "mp" ($a,
step 34).
The source assertion before substitution was:
    weq $a wff t = r
The following substitutions were made to the source
assertion:
    Variable  Substituted with
     t         ( t + 0 )
     r         t
The target hypothesis before substitution was:
    wp $f wff P
The following substitution was made to the target
hypothesis:
    Variable  Substituted with
     P         ( t + 0 ) = t
\end{verbatim}

The full proof just shown is useful to understand what is going on in detail.
However, most of the time you will just be interested in
the ``essential'' or logical steps of a proof, i.e.\ those steps
that correspond to an
ordinary formal proof\index{formal proof}.  If you type
\begin{verbatim}
MM> show proof th1 /lemmon /renumber
\end{verbatim}
you will see\label{demoproof}
\begin{verbatim}
1 a2             $a |- ( t + 0 ) = t
2 a2             $a |- ( t + 0 ) = t
3 a1             $a |- ( ( t + 0 ) = t -> ( ( t + 0 )
                                     = t -> t = t ) )
4 2,3 mp         $a |- ( ( t + 0 ) = t -> t = t )
5 1,4 mp         $a |- t = t
\end{verbatim}
Compare this to the formal proof on p.~\pageref{zeroproof} and
notice the resemblance.
By default Metamath
does not show \texttt{\$f}\index{\texttt{\$f}
statement} hypotheses and everything branching off of them in the proof tree
when the proof is displayed; this makes the proof look more like an ordinary
mathematical proof, which does not normally incorporate the explicit
construction of expressions.
This is called the ``essential'' view
(at one time you had to add the
\texttt{/essential} qualifier in the \texttt{show proof}
command to get this view, but this is now the default).
You can could use the \texttt{/all} qualifier in the \texttt{show
proof} command to also show the explicit construction of expressions.
The \texttt{/renumber} qualifier means to renumber
the steps to correspond only to what is displayed.\index{\texttt{show proof}
command}

To exit Metamath, type\index{\texttt{exit} command}
\begin{verbatim}
MM> exit
\end{verbatim}

\subsection{Some Hints for Using the Command Line Interface}

We will conclude this quick introduction to Metamath\index{Metamath} with some
helpful hints on how to navigate your way through the commands.
\index{command line interface (CLI)}

When you type commands into Metamath's CLI, you only have to type as many
characters of a command keyword\index{command keyword} as are needed to make
it unambiguous.  If you type too few characters, Metamath will tell you what
the choices are.  In the case of the \texttt{read} command, only the \texttt{r} is
needed to specify it unambiguously, so you could have typed\index{\texttt{read}
command}
\begin{verbatim}
MM> r demo0.mm
\end{verbatim}
instead of
\begin{verbatim}
MM> read demo0.mm
\end{verbatim}
In our description, we always show the full command words.  When using the
Metamath CLI commands in a command file (to be read with the \texttt{submit}
command)\index{\texttt{submit} command}, it is good practice to use
the unabbreviated command to ensure your instructions will not become ambiguous
if more commands are added to the Metamath program in the future.

The command keywords\index{command
keyword} are not case sensitive; you may type either \texttt{read} or
\texttt{ReAd}.  File names may or may not be case sensitive, depending on your
computer's operating system.  Metamath label\index{label} and math
symbol\index{math symbol} tokens\index{token} are case-sensitive.

The \texttt{help} command\index{\texttt{help} command} will provide you
with a list of topics you can get help on.  You can then type
\texttt{help} {\em topic} to get help on that topic.

If you are uncertain of a command's spelling, just type as many characters
as you remember of the command.  If you have not typed enough characters to
specify it unambiguously, Metamath will tell you what choices you have.

\begin{verbatim}
MM> show s
         ^
?Ambiguous keyword - please specify SETTINGS,
STATEMENT, or SOURCE.
\end{verbatim}

If you don't know what argument to use as part of a command, type a
\texttt{?}\index{\texttt{]}@\texttt{?}\ in command lines}\ at the
argument position.  Metamath will tell you what it expected there.

\begin{verbatim}
MM> show ?
         ^
?Expected SETTINGS, LABELS, STATEMENT, SOURCE, PROOF,
MEMORY, TRACE_BACK, or USAGE.
\end{verbatim}

Finally, you may type just the first word or words of a command followed
by {\em return}.  Metamath will prompt you for the remaining part of the
command, showing you the choices at each step.  For example, instead of
typing \texttt{show statement th1 /full} you could interact in the
following manner:
\begin{verbatim}
MM> show
SETTINGS, LABELS, STATEMENT, SOURCE, PROOF,
MEMORY, TRACE_BACK, or USAGE <SETTINGS>? st
What is the statement label <th1>?
/ or nothing <nothing>? /
TEX, COMMENT_ONLY, or FULL <TEX>? f
/ or nothing <nothing>?
19 th1 $p |- t = t $= ... $.
\end{verbatim}
After each \texttt{?}\ in this mode, you must give Metamath the
information it requests.  Sometimes Metamath gives you a list of choices
with the default choice indicated by brackets \texttt{< > }. Pressing
{\em return} after the \texttt{?}\ will select the default choice.
Answering anything else will override the default.  Note that the
\texttt{/} in command qualifiers is considered a separate
token\index{token} by the parser, and this is why it is asked for
separately.

\section{Your First Proof}\label{frstprf}

Proofs are developed with the aid of the Proof Assistant\index{Proof
Assistant}.  We will now show you how the proof of theorem \texttt{th1}
was built.  So that you can repeat these steps, we will first have the
Proof Assistant erase the proof in Metamath's source buffer\index{source
buffer}, then reconstruct it.  (The source buffer is the place in memory
where Metamath stores the information in the database when it is
\texttt{read}\index{\texttt{read} command} in.  New or modified proofs
are kept in the source buffer until a \texttt{write source}
command\index{\texttt{write source} command} is issued.)  In practice, you
would place a \texttt{?}\index{\texttt{]}@\texttt{?}\ inside proofs}\
between \texttt{\$=}\index{\texttt{\$=} keyword} and
\texttt{\$.}\index{\texttt{\$.}\ keyword}\ in the database to indicate
to Metamath\index{Metamath} that the proof is unknown, and that would be
your starting point.  Whenever the \texttt{verify proof} command encounters
a proof with a \texttt{?}\ in place of a proof step, the statement is
identified as not proved.

When I first started creating Metamath proofs, I would write down
on a piece of paper the complete
formal proof\index{formal proof} as it would appear
in a \texttt{show proof} command\index{\texttt{show proof} command}; see
the display of \texttt{show proof th1 /lemmon /re\-num\-ber} above as an
example.  After you get used to using the Proof Assistant\index{Proof
Assistant} you may get to a point where you can ``see'' the proof in your mind
and let the Proof Assistant guide you in filling in the details, at least for
simpler proofs, but until you gain that experience you may find it very useful
to write down all the details in advance.
Otherwise you may waste a lot of time as you let it take you down a wrong path.
However, others do not find this approach as helpful.
For example, Thomas Brendan Leahy\index{Leahy, Thomas Brendan}
finds that it is more helpful to him to interactively
work backward from a machine-readable statement.
David A. Wheeler\index{Wheeler, David A.}
writes down a general approach, but develops the proof
interactively by switching between
working forwards (from hypotheses and facts likely to be useful) and
backwards (from the goal) until the forwards and backwards approaches meet.
In the end, use whatever approach works for you.

A proof is developed with the Proof Assistant by working backwards, starting
with the theorem\index{theorem} to be proved, and assigning each unknown step
with a theorem or hypothesis until no more unknown steps remain.  The Proof
Assistant will not let you make an assignment unless it can be ``unified''
with the unknown step.  This means that a
substitution\index{substitution!variable}\index{variable substitution} of
variables exists that will make the assignment match the unknown step.  On the
other hand, in the middle of a proof, when working backwards, often more than
one unification\index{unification} (set of substitutions) is possible, since
there is not enough information available at that point to uniquely establish
it.  In this case you can tell Metamath which unification to choose, or you
can continue to assign unknown steps until enough information is available to
make the unification unique.

We will assume you have entered Metamath and read in the database as described
above.  The following dialog shows how the proof was developed.  For more
details on what some of the commands do, refer to Section~\ref{pfcommands}.
\index{\texttt{prove} command}

\begin{verbatim}
MM> prove th1
Entering the Proof Assistant.  Type HELP for help, EXIT
to exit.  You will be working on the proof of statement th1:
  $p |- t = t
Note:  The proof you are starting with is already complete.
MM-PA>
\end{verbatim}

The \verb/MM-PA>/ prompt means we are inside the Proof
Assistant.\index{Proof Assistant} Most of the regular Metamath commands
(\texttt{show statement}, etc.) are still available if you need them.

\begin{verbatim}
MM-PA> delete all
The entire proof was deleted.
\end{verbatim}

We have deleted the whole proof so we can start from scratch.

\begin{verbatim}
MM-PA> show new_proof/lemmon/all
1 ?              $? |- t = t
\end{verbatim}

The \texttt{show new{\char`\_}proof} command\index{\texttt{show
new{\char`\_}proof} command} is like \texttt{show proof} except that we
don't specify a statement; instead, the proof we're working on is
displayed.

\begin{verbatim}
MM-PA> assign 1 mp
To undo the assignment, DELETE STEP 5 and INITIALIZE, UNIFY
if needed.
3   min=?  $? |- $2
4   maj=?  $? |- ( $2 -> t = t )
\end{verbatim}

The \texttt{assign} command\index{\texttt{assign} command} above means
``assign step 1 with the statement whose label is \texttt{mp}.''  Note
that step renumbering will constantly occur as you assign steps in the
middle of a proof; in general all steps from the step you assign until
the end of the proof will get moved up.  In this case, what used to be
step 1 is now step 5, because the (partial) proof now has five steps:
the four hypotheses of the \texttt{mp} statement and the \texttt{mp}
statement itself.  Let's look at all the steps in our partial proof:

\begin{verbatim}
MM-PA> show new_proof/lemmon/all
1 ?              $? wff $2
2 ?              $? wff t = t
3 ?              $? |- $2
4 ?              $? |- ( $2 -> t = t )
5 1,2,3,4 mp     $a |- t = t
\end{verbatim}

The symbol \texttt{\$2} is a temporary variable\index{temporary
variable} that represents a symbol sequence not yet known.  In the final
proof, all temporary variables will be eliminated.  The general format
for a temporary variable is \texttt{\$} followed by an integer.  Note
that \texttt{\$} is not a legal character in a math symbol (see
Section~\ref{dollardollar}, p.~\pageref{dollardollar}), so there will
never be a naming conflict between real symbols and temporary variables.

Unknown steps 1 and 2 are constructions of the two wffs used by the
modus ponens rule.  As you will see at the end of this section, the
Proof Assistant\index{Proof Assistant} can usually figure these steps
out by itself, and we will not have to worry about them.  Therefore from
here on we will display only the ``essential'' hypotheses, i.e.\ those
steps that correspond to traditional formal proofs\index{formal proof}.

\begin{verbatim}
MM-PA> show new_proof/lemmon
3 ?              $? |- $2
4 ?              $? |- ( $2 -> t = t )
5 3,4 mp         $a |- t = t
\end{verbatim}

Unknown steps 3 and 4 are the ones we must focus on.  They correspond to the
minor and major premises of the modus ponens rule.  We will assign them as
follows.  Notice that because of the step renumbering that takes place
after an assignment, it is advantageous to assign unknown steps in reverse
order, because earlier steps will not get renumbered.

\begin{verbatim}
MM-PA> assign 4 mp
To undo the assignment, DELETE STEP 8 and INITIALIZE, UNIFY
if needed.
3   min=?  $? |- $2
6     min=?  $? |- $4
7     maj=?  $? |- ( $4 -> ( $2 -> t = t ) )
\end{verbatim}

We are now going to describe an obscure feature that you will probably
never use but should be aware of.  The Metamath language allows empty
symbol sequences to be substituted for variables, but in most formal
systems this feature is never used.  One of the few examples where is it
used is the MIU-system\index{MIU-system} described in
Appendix~\ref{MIU}.  But such systems are rare, and by default this
feature is turned off in the Proof Assistant.  (It is always allowed for
{\tt verify proof}.)  Let us turn it on and see what
happens.\index{\texttt{set empty{\char`\_}substitution} command}

\begin{verbatim}
MM-PA> set empty_substitution on
Substitutions with empty symbol sequences is now allowed.
\end{verbatim}

With this feature enabled, more unifications will be
ambiguous\index{ambiguous unification}\index{unification!ambiguous} in
the middle of a proof, because
substitution\index{substitution!variable}\index{variable substitution}
of variables with empty symbol sequences will become an additional
possibility.  Let's see what happens when we make our next assignment.

\begin{verbatim}
MM-PA> assign 3 a2
There are 2 possible unifications.  Please select the correct
    one or Q if you want to UNIFY later.
Unify:  |- $6
 with:  |- ( $9 + 0 ) = $9
Unification #1 of 2 (weight = 7):
  Replace "$6" with "( + 0 ) ="
  Replace "$9" with ""
  Accept (A), reject (R), or quit (Q) <A>? r
\end{verbatim}

The first choice presented is the wrong one.  If we had selected it,
temporary variable \texttt{\$6} would have been assigned a truncated
wff, and temporary variable \texttt{\$9} would have been assigned an
empty sequence (which is not allowed in our system).  With this choice,
eventually we would reach a point where we would get stuck because
we would end up with steps impossible to prove.  (You may want to
try it.)  We typed \texttt{r} to reject the choice.

\begin{verbatim}
Unification #2 of 2 (weight = 21):
  Replace "$6" with "( $9 + 0 ) = $9"
  Accept (A), reject (R), or quit (Q) <A>? q
To undo the assignment, DELETE STEP 4 and INITIALIZE, UNIFY
if needed.
 7     min=?  $? |- $8
 8     maj=?  $? |- ( $8 -> ( $6 -> t = t ) )
\end{verbatim}

The second choice is correct, and normally we would type \texttt{a}
to accept it.  But instead we typed \texttt{q} to show what will happen:
it will leave the step with an unknown unification, which can be
seen as follows:

\begin{verbatim}
MM-PA> show new_proof/not_unified
 4   min    $a |- $6
        =a2  = |- ( $9 + 0 ) = $9
\end{verbatim}

Later we can unify this with the \texttt{unify}
\texttt{all/interactive} command.

The important point to remember is that occasionally you will be
presented with several unification choices while entering a proof, when
the program determines that there is not enough information yet to make
an unambiguous choice automatically (and this can happen even with
\texttt{set empty{\char`\_}substitution} turned off).  Usually it is
obvious by inspection which choice is correct, since incorrect ones will
tend to be meaningless fragments of wffs.  In addition, the correct
choice will usually be the first one presented, unlike our example
above.

Enough of this digression.  Let us go back to the default setting.

\begin{verbatim}
MM-PA> set empty_substitution off
The ability to substitute empty expressions for variables
has been turned off.  Note that this may make the Proof
Assistant too restrictive in some cases.
\end{verbatim}

If we delete the proof, start over, and get to the point where
we digressed above, there will no longer be an ambiguous unification.

\begin{verbatim}
MM-PA> assign 3 a2
To undo the assignment, DELETE STEP 4 and INITIALIZE, UNIFY
if needed.
 7     min=?  $? |- $4
 8     maj=?  $? |- ( $4 -> ( ( $5 + 0 ) = $5 -> t = t ) )
\end{verbatim}

Let us look at our proof so far, and continue.

\begin{verbatim}
MM-PA> show new_proof/lemmon
 4 a2            $a |- ( $5 + 0 ) = $5
 7 ?             $? |- $4
 8 ?             $? |- ( $4 -> ( ( $5 + 0 ) = $5 -> t = t ) )
 9 7,8 mp        $a |- ( ( $5 + 0 ) = $5 -> t = t )
10 4,9 mp        $a |- t = t
MM-PA> assign 8 a1
To undo the assignment, DELETE STEP 11 and INITIALIZE, UNIFY
if needed.
 7     min=?  $? |- ( t + 0 ) = t
MM-PA> assign 7 a2
To undo the assignment, DELETE STEP 8 and INITIALIZE, UNIFY
if needed.
MM-PA> show new_proof/lemmon
 4 a2            $a |- ( t + 0 ) = t
 8 a2            $a |- ( t + 0 ) = t
12 a1            $a |- ( ( t + 0 ) = t -> ( ( t + 0 ) = t ->
                                                    t = t ) )
13 8,12 mp       $a |- ( ( t + 0 ) = t -> t = t )
14 4,13 mp       $a |- t = t
\end{verbatim}

Now all temporary variables and unknown steps have been eliminated from the
``essential'' part of the proof.  When this is achieved, the Proof
Assistant\index{Proof Assistant} can usually figure out the rest of the proof
automatically.  (Note that the \texttt{improve} command can occasionally be
useful for filling in essential steps as well, but it only tries to make use
of statements that introduce no new variables in their hypotheses, which is
not the case for \texttt{mp}. Also it will not try to improve steps containing
temporary variables.)  Let's look at the complete proof, then run
the \texttt{improve} command, then look at it again.

\begin{verbatim}
MM-PA> show new_proof/lemmon/all
 1 ?             $? wff ( t + 0 ) = t
 2 ?             $? wff t = t
 3 ?             $? term t
 4 3 a2          $a |- ( t + 0 ) = t
 5 ?             $? wff ( t + 0 ) = t
 6 ?             $? wff ( ( t + 0 ) = t -> t = t )
 7 ?             $? term t
 8 7 a2          $a |- ( t + 0 ) = t
 9 ?             $? term ( t + 0 )
10 ?             $? term t
11 ?             $? term t
12 9,10,11 a1    $a |- ( ( t + 0 ) = t -> ( ( t + 0 ) = t ->
                                                    t = t ) )
13 5,6,8,12 mp   $a |- ( ( t + 0 ) = t -> t = t )
14 1,2,4,13 mp   $a |- t = t
\end{verbatim}

\begin{verbatim}
MM-PA> improve all
A proof of length 1 was found for step 11.
A proof of length 1 was found for step 10.
A proof of length 3 was found for step 9.
A proof of length 1 was found for step 7.
A proof of length 9 was found for step 6.
A proof of length 5 was found for step 5.
A proof of length 1 was found for step 3.
A proof of length 3 was found for step 2.
A proof of length 5 was found for step 1.
Steps 1 and above have been renumbered.
CONGRATULATIONS!  The proof is complete.  Use SAVE
NEW_PROOF to save it.  Note:  The Proof Assistant does
not detect $d violations.  After saving the proof, you
should verify it with VERIFY PROOF.
\end{verbatim}

The \texttt{save new{\char`\_}proof} command\index{\texttt{save
new{\char`\_}proof} command} will save the proof in the database.  Here
we will just display it in a form that can be clipped out of a log file
and inserted manually into the database source file with a text
editor.\index{normal proof}\index{proof!normal}

\begin{verbatim}
MM-PA> show new_proof/normal
---------Clip out the proof below this line:
      tt tze tpl tt weq tt tt weq tt a2 tt tze tpl tt weq
      tt tze tpl tt weq tt tt weq wim tt a2 tt tze tpl tt
      tt a1 mp mp $.
---------The proof of 'th1' to clip out ends above this line.
\end{verbatim}

There is another proof format called ``compressed''\index{compressed
proof}\index{proof!compressed} that you will see in databases.  It is
not important to understand how it is encoded but only to recognize it
when you see it.  Its only purpose is to reduce storage requirements for
large proofs.  A compressed proof can always be converted to a normal
one and vice-versa, and the Metamath \texttt{show proof}
commands\index{\texttt{show proof} command} work equally well with
compressed proofs.  The compressed proof format is described in
Appendix~\ref{compressed}.

\begin{verbatim}
MM-PA> show new_proof/compressed
---------Clip out the proof below this line:
      ( tze tpl weq a2 wim a1 mp ) ABCZADZAADZAEZJJKFLIA
      AGHH $.
---------The proof of 'th1' to clip out ends above this line.
\end{verbatim}

Now we will exit the Proof Assistant.  Since we made changes to the proof,
it will warn us that we have not saved it.  In this case, we don't care.

\begin{verbatim}
MM-PA> exit
Warning:  You have not saved changes to the proof.
Do you want to EXIT anyway (Y, N) <N>? y
Exiting the Proof Assistant.
Type EXIT again to exit Metamath.
\end{verbatim}

The Proof Assistant\index{Proof Assistant} has several other commands
that can help you while creating proofs.  See Section~\ref{pfcommands}
for a list of them.

A command that is often useful is \texttt{minimize{\char`\_}with
*/brief}, which tries to shorten the proof.  It can make the process
more efficient by letting you write a somewhat ``sloppy'' proof then
clean up some of the fine details of optimization for you (although it
can't perform miracles such as restructuring the overall proof).

\section{A Note About Editing a Data\-base File}

Once your source file contains proofs, there are some restrictions on
how you can edit it so that the proofs remain valid.  Pay particular
attention to these rules, since otherwise you can lose a lot of work.
It is a good idea to periodically verify all proofs with \texttt{verify
proof *} to ensure their integrity.

If your file contains only normal (as opposed to compressed) proofs, the
main rule is that you may not change the order of the mandatory
hypotheses\index{mandatory hypothesis} of any statement referenced in a
later proof.  For example, if you swap the order of the major and minor
premise in the modus ponens rule, all proofs making use of that rule
will become incorrect.  The \texttt{show statement}
command\index{\texttt{show statement} command} will show you the
mandatory hypotheses of a statement and their order.

If a statement has a compressed proof, you also must not change the
order of {\em its} mandatory hypotheses.  The compressed proof format
makes use of this information as part of the compression technique.
Note that swapping the names of two variables in a theorem will change
the order of its mandatory hypotheses.

The safest way to edit a statement, say \texttt{mytheorem}, is to
duplicate it then rename the original to \texttt{mytheoremOLD}
throughout the database.  Once the edited version is re-proved, all
statements referencing \texttt{mytheoremOLD} can be updated in the Proof
Assistant using \texttt{minimize{\char`\_}with
mytheorem
/allow{\char`\_}growth}.\index{\texttt{minimize{\char`\_}with} command}
% 3/10/07 Note: line-breaking the above results in duplicate index entries

\chapter{Abstract Mathematics Revealed}\label{fol}

\section{Logic and Set Theory}\label{logicandsettheory}

\begin{quote}
  {\em Set theory can be viewed as a form of exact theology.}
  \flushright\sc  Rudy Rucker\footnote{\cite{Barrow}, p.~31.}\\
\end{quote}\index{Rucker, Rudy}

Despite its seeming complexity, all of standard mathematics, no matter how
deep or abstract, can amazingly enough be derived from a relatively small set
of axioms\index{axiom} or first principles. The development of these axioms is
among the most impressive and important accomplishments of mathematics in the
20th century. Ultimately, these axioms can be broken down into a set of rules
for manipulating symbols that any technically oriented person can follow.

We will not spend much time trying to convey a deep, higher-level
understanding of the meaning of the axioms. This kind of understanding
requires some mathematical sophistication as well as an understanding of the
philosophy underlying the foundations of mathematics and typically develops
over time as you work with mathematics.  Our goal, instead, is to give you the
immediate ability to follow how theorems\index{theorem} are derived from the
axioms and from other theorems.  This will be similar to learning the syntax
of a computer language, which lets you follow the details in a program but
does not necessarily give you the ability to write non-trivial programs on
your own, an ability that comes with practice. For now don't be alarmed by
abstract-sounding names of the axioms; just focus on the rules for
manipulating the symbols, which follow the simple conventions of the
Metamath\index{Metamath} language.

The axioms that underlie all of standard mathematics consist of axioms of logic
and axioms of set theory. The axioms of logic are divided into two
subcategories, propositional calculus\index{propositional calculus} (sometimes
called sentential logic\index{sentential logic}) and predicate calculus
(sometimes called first-order logic\index{first-order logic}\index{quantifier
theory}\index{predicate calculus} or quantifier theory).  Propositional
calculus is a prerequisite for predicate calculus, and predicate calculus is a
prerequisite for set theory.  The version of set theory most commonly used is
Zermelo--Fraenkel set theory\index{Zermelo--Fraenkel set theory}\index{set theory}
with the axiom of choice,
often abbreviated as ZFC\index{ZFC}.

Here in a nutshell is what the axioms are all about in an informal way. The
connection between this description and symbols we will show you won't be
immediately apparent and in principle needn't ever be.  Our description just
tries to summarize what mathematicians think about when they work with the
axioms.

Logic is a set of rules that allow us determine truths given other truths.
Put another way,
logic is more or less the translation of what we would consider common sense
into a rigorous set of axioms.\index{axioms of logic}  Suppose $\varphi$,
$\psi$, and $\chi$ (the Greek letters phi, psi, and chi) represent statements
that are either true or false, and $x$ is a variable\index{variable!in predicate
calculus} ranging over some group of mathematical objects (sets, integers,
real numbers, etc.). In mathematics, a ``statement'' really means a formula,
and $\psi$ could be for example ``$x = 2$.''
Propositional calculus\index{propositional calculus}
allows us to use variables that are either true or false
and make deductions such as
``if $\varphi$ implies $\psi$ and $\psi$ implies $\chi$, then $\varphi$
implies $\chi$.''
Predicate calculus\index{predicate calculus}
extends propositional calculus by also allowing us
to discuss statements about objects (not just true and false values), including
statements about ``all'' or ``at least one'' object.
For example, predicate calculus allows to say,
``if $\varphi$ is true for all $x$, then $\varphi$ is true for some $x$.''
The logic used in \texttt{set.mm} is standard classical logic
(as opposed to other logic systems like intuitionistic logic).

Set theory\index{set theory} has to do with the manipulation of objects and
collections of objects, specifically the abstract, imaginary objects that
mathematics deals with, such as numbers. Everything that is claimed to exist
in mathematics is considered to be a set.  A set called the empty
set\index{empty set} contains nothing.  We represent the empty set by
$\varnothing$.  Many sets can be built up from the empty set.  There is a set
represented by $\{\varnothing\}$ that contains the empty set, another set
represented by $\{\varnothing,\{\varnothing\}\}$ that contains this set as
well as the empty set, another set represented by $\{\{\varnothing\}\}$ that
contains just the set that contains the empty set, and so on ad infinitum. All
mathematical objects, no matter how complex, are defined as being identical to
certain sets: the integer\index{integer} 0 is defined as the empty set, the
integer 1 is defined as $\{\varnothing\}$, the integer 2 is defined as
$\{\varnothing,\{\varnothing\}\}$.  (How these definitions were chosen doesn't
matter now, but the idea behind it is that these sets have the properties we
expect of integers once suitable operations are defined.)  Mathematical
operations, such as addition, are defined in terms of operations on
sets---their union\index{set union}, intersection\index{set intersection}, and
so on---operations you may have used in elementary school when you worked
with groups of apples and oranges.

With a leap of faith, the axioms also postulate the existence of infinite
sets\index{infinite set}, such as the set of all non-negative integers ($0, 1,
2,\ldots$, also called ``natural numbers''\index{natural number}).  This set
can't be represented with the brace notation\index{brace notation} we just
showed you, but requires a more complicated notation called ``class
abstraction.''\index{class abstraction}\index{abstraction class}  For
example, the infinite set $\{ x |
\mbox{``$x$ is a natural number''} \} $ means the ``set of all objects $x$
such that $x$ is a natural number'' i.e.\ the set of natural numbers; here,
``$x$ is a natural number'' is a rather complicated formula when broken down
into the primitive symbols.\label{expandom}\footnote{The statement ``$x$ is a
natural number'' is formally expressed as ``$x \in \omega$,'' where $\in$
(stylized epsilon) means ``is in'' or ``is an element of'' and $\omega$
(omega) means ``the set of natural numbers.''  When ``$x\in\omega$'' is
completely expanded in terms of the primitive symbols of set theory, the
result is  $\lnot$ $($ $\lnot$ $($ $\forall$ $z$ $($ $\lnot$ $\forall$ $w$ $($
$z$ $\in$ $w$ $\rightarrow$ $\lnot$ $w$ $\in$ $x$ $)$ $\rightarrow$ $z$ $\in$
$x$ $)$ $\rightarrow$ $($ $\forall$ $z$ $($ $\lnot$ $($ $\forall$ $w$ $($ $w$
$\in$ $z$ $\rightarrow$ $w$ $\in$ $x$ $)$ $\rightarrow$ $\forall$ $w$ $\lnot$
$w$ $\in$ $z$ $)$ $\rightarrow$ $\lnot$ $\forall$ $w$ $($ $w$ $\in$ $z$
$\rightarrow$ $\lnot$ $\forall$ $v$ $($ $v$ $\in$ $z$ $\rightarrow$ $\lnot$
$v$ $\in$ $w$ $)$ $)$ $)$ $\rightarrow$ $\lnot$ $\forall$ $z$ $\forall$ $w$
$($ $\lnot$ $($ $z$ $\in$ $x$ $\rightarrow$ $\lnot$ $w$ $\in$ $x$ $)$
$\rightarrow$ $($ $\lnot$ $z$ $\in$ $w$ $\rightarrow$ $($ $\lnot$ $z$ $=$ $w$
$\rightarrow$ $w$ $\in$ $z$ $)$ $)$ $)$ $)$ $)$ $\rightarrow$ $\lnot$
$\forall$ $y$ $($ $\lnot$ $($ $\lnot$ $($ $\forall$ $z$ $($ $\lnot$ $\forall$
$w$ $($ $z$ $\in$ $w$ $\rightarrow$ $\lnot$ $w$ $\in$ $y$ $)$ $\rightarrow$
$z$ $\in$ $y$ $)$ $\rightarrow$ $($ $\forall$ $z$ $($ $\lnot$ $($ $\forall$
$w$ $($ $w$ $\in$ $z$ $\rightarrow$ $w$ $\in$ $y$ $)$ $\rightarrow$ $\forall$
$w$ $\lnot$ $w$ $\in$ $z$ $)$ $\rightarrow$ $\lnot$ $\forall$ $w$ $($ $w$
$\in$ $z$ $\rightarrow$ $\lnot$ $\forall$ $v$ $($ $v$ $\in$ $z$ $\rightarrow$
$\lnot$ $v$ $\in$ $w$ $)$ $)$ $)$ $\rightarrow$ $\lnot$ $\forall$ $z$
$\forall$ $w$ $($ $\lnot$ $($ $z$ $\in$ $y$ $\rightarrow$ $\lnot$ $w$ $\in$
$y$ $)$ $\rightarrow$ $($ $\lnot$ $z$ $\in$ $w$ $\rightarrow$ $($ $\lnot$ $z$
$=$ $w$ $\rightarrow$ $w$ $\in$ $z$ $)$ $)$ $)$ $)$ $\rightarrow$ $($
$\forall$ $z$ $\lnot$ $z$ $\in$ $y$ $\rightarrow$ $\lnot$ $\forall$ $w$ $($
$\lnot$ $($ $w$ $\in$ $y$ $\rightarrow$ $\lnot$ $\forall$ $z$ $($ $w$ $\in$
$z$ $\rightarrow$ $\lnot$ $z$ $\in$ $y$ $)$ $)$ $\rightarrow$ $\lnot$ $($
$\lnot$ $\forall$ $z$ $($ $w$ $\in$ $z$ $\rightarrow$ $\lnot$ $z$ $\in$ $y$
$)$ $\rightarrow$ $w$ $\in$ $y$ $)$ $)$ $)$ $)$ $\rightarrow$ $x$ $\in$ $y$
$)$ $)$ $)$. Section~\ref{hierarchy} shows the hierarchy of definitions that
leads up to this expression.}\index{stylized epsilon ($\in$)}\index{omega
($\omega$)}  Actually, the primitive symbols don't even include the brace
notation.  The brace notation is a high-level definition, which you can find in
Section~\ref{hierarchy}.

Interestingly, the arithmetic of integers\index{integer} and
rationals\index{rational number} can be developed without appealing to the
existence of an infinite set, whereas the arithmetic of real
numbers\index{real number} requires it.

Each variable\index{variable!in set theory} in the axioms of set theory
represents an arbitrary set, and the axioms specify the legal kinds of things
you can do with these variables at a very primitive level.

Now, you may think that numbers and arithmetic are a lot more intuitive and
fundamental than sets and therefore should be the foundation of mathematics.
What is really the case is that you've dealt with numbers all your life and
are comfortable with a few rules for manipulating them such as addition and
multiplication.  Those rules only cover a small portion of what can be done
with numbers and only a very tiny fraction of the rest of mathematics.  If you
look at any elementary book on number theory, you will quickly become lost if
these are the only rules that you know.  Even though such books may present a
list of ``axioms''\index{axiom} for arithmetic, the ability to use the axioms
and to understand proofs of theorems\index{theorem} (facts) about numbers
requires an implicit mathematical talent that frustrates many people
from studying abstract mathematics.  The kind of mathematics that most people
know limits them to the practical, everyday usage of blindly manipulating
numbers and formulas, without any understanding of why those rules are correct
nor any ability to go any further.  For example, do you know why multiplying
two negative numbers yields a positive number?  Starting with set theory, you
will also start off blindly manipulating symbols according to the rules we give
you, but with the advantage that these rules will allow you, in principle, to
access {\em all} of mathematics, not just a tiny part of it.

Of course, concrete examples are often helpful in the learning process. For
example, you can verify that $2\cdot 3=3 \cdot 2$ by actually grouping
objects and can easily ``see'' how it generalizes to $x\cdot y = y\cdot x$,
even though you might not be able to rigorously prove it.  Similarly, in set
theory it can be helpful to understand how the axioms of set theory apply to
(and are correct for) small finite collections of objects.  You should be aware
that in set theory intuition can be misleading for infinite collections, and
rigorous proofs become more important.  For example, while $x\cdot y = y\cdot
x$ is correct for finite ordinals (which are the natural numbers), it is not
usually true for infinite ordinals.

\section{The Axioms for All of Mathematics}

In this section\index{axioms for mathematics}, we will show you the axioms
for all of standard mathematics (i.e.\ logic and set theory) as they are
traditionally presented.  The traditional presentation is useful for someone
with the mathematical experience needed to correctly manipulate high-level
abstract concepts.  For someone without this talent, knowing how to actually
make use of these axioms can be difficult.  The purpose of this section is to
allow you to see how the version of the axioms used in the standard
Metamath\index{Metamath} database \texttt{set.mm}\index{set
theory database (\texttt{set.mm})} relates to  the typical version
in textbooks, and also to give you an informal feel for them.

\subsection{Propositional Calculus}

Propositional calculus\index{propositional calculus} concerns itself with
statements that can be interpreted as either true or false.  Some examples of
statements (outside of mathematics) that are either true or false are ``It is
raining today'' and ``The United States has a female president.'' In
mathematics, as we mentioned, statements are really formulas.

In propositional calculus, we don't care what the statements are.  We also
treat a logical combination of statements, such as ``It is raining today and
the United States has a female president,'' no differently from a single
statement.  Statements and their combinations are called well-formed formulas
(wffs)\index{well-formed formula (wff)}.  We define wffs only in terms of
other wffs and don't define what a ``starting'' wff is.  As is common practice
in the literature, we use Greek letters to represent wffs.

Specifically, suppose $\varphi$ and $\psi$ are wffs.  Then the combinations
$\varphi\rightarrow\psi$ (``$\varphi$ implies $\psi$,'' also read ``if
$\varphi$ then $\psi$'')\index{implication ($\rightarrow$)} and $\lnot\varphi$
(``not $\varphi$'')\index{negation ($\lnot$)} are also wffs.

The three axioms of propositional calculus\index{axioms of propositional
calculus} are all wffs of the following form:\footnote{A remarkable result of
C.~A.~Meredith\index{Meredith, C. A.} squeezes these three axioms into the
single axiom $((((\varphi\rightarrow \psi)\rightarrow(\neg \chi\rightarrow\neg
\theta))\rightarrow \chi )\rightarrow \tau)\rightarrow((\tau\rightarrow
\varphi)\rightarrow(\theta\rightarrow \varphi))$ \cite{CAMeredith},
which is believed to be the shortest possible.}
\begin{center}
     $\varphi\rightarrow(\psi\rightarrow \varphi)$\\

     $(\varphi\rightarrow (\psi\rightarrow \chi))\rightarrow
((\varphi\rightarrow  \psi)\rightarrow (\varphi\rightarrow \chi))$\\

     $(\neg \varphi\rightarrow \neg\psi)\rightarrow (\psi\rightarrow
\varphi)$
\end{center}

These three axioms are widely used.
They are attributed to Jan {\L}ukasiewicz
(pronounced woo-kah-SHAY-vitch) and was popularized by Alonzo Church,
who called it system P2. (Thanks to Ted Ulrich for this information.)

There are an infinite number of axioms, one for each possible
wff\index{well-formed formula (wff)} of the above form.  (For this reason,
axioms such as the above are often called ``axiom schemes.''\index{axiom
scheme})  Each Greek letter in the axioms may be substituted with a more
complex wff to result in another axiom.  For example, substituting
$\neg(\varphi\rightarrow\chi)$ for $\varphi$ in the first axiom yields
$\neg(\varphi\rightarrow\chi)\rightarrow(\psi\rightarrow
\neg(\varphi\rightarrow\chi))$, which is still an axiom.

To deduce new true statements (theorems\index{theorem}) from the axioms, a
rule\index{rule} called ``modus ponens''\index{modus ponens} is used.  This
rule states that if the wff $\varphi$ is an axiom or a theorem, and the wff
$\varphi\rightarrow\psi$ is an axiom or a theorem, then the wff $\psi$ is also
a theorem\index{theorem}.

As a non-mathematical example of modus ponens, suppose we have proved (or
taken as an axiom) ``Bob is a man'' and separately have proved (or taken as
an axiom) ``If Bob is a man, then Bob is a human.''  Using the rule of modus
ponens, we can logically deduce, ``Bob is a human.''

From Metamath's\index{Metamath} point of view, the axioms and the rule of
modus ponens just define a mechanical means for deducing new true statements
from existing true statements, and that is the complete content of
propositional calculus as far as Metamath is concerned.  You can read a logic
textbook to gain a better understanding of their meaning, or you can just let
their meaning slowly become apparent to you after you use them for a while.

It is actually rather easy to check to see if a formula is a theorem of
propositional calculus.  Theorems of propositional calculus are also called
``tautologies.''\index{tautology}  The technique to check whether a formula is
a tautology is called the ``truth table method,''\index{truth table} and it
works like this.  A wff $\varphi\rightarrow\psi$ is false whenever $\varphi$ is true
and $\psi$ is false.  Otherwise it is true.  A wff $\lnot\varphi$ is false
whenever $\varphi$ is true and false otherwise. To verify a tautology such as
$\varphi\rightarrow(\psi\rightarrow \varphi)$, you break it down into sub-wffs and
construct a truth table that accounts for all possible combinations of true
and false assigned to the wff metavariables:
\begin{center}\begin{tabular}{|c|c|c|c|}\hline
\mbox{$\varphi$} & \mbox{$\psi$} & \mbox{$\psi\rightarrow\varphi$}
    & \mbox{$\varphi\rightarrow(\psi\rightarrow \varphi)$} \\ \hline \hline
              T   &  T    &      T       &        T    \\ \hline
              T   &  F    &      T       &        T    \\ \hline
              F   &  T    &      F       &        T    \\ \hline
              F   &  F    &      T       &        T    \\ \hline
\end{tabular}\end{center}
If all entries in the last column are true, the formula is a tautology.

Now, the truth table method doesn't tell you how to prove the tautology from
the axioms, but only that a proof exists.  Finding an actual proof (especially
one that is short and elegant) can be challenging.  Methods do exist for
automatically generating proofs in propositional calculus, but the proofs that
result can sometimes be very long.  In the Metamath \texttt{set.mm}\index{set
theory database (\texttt{set.mm})} database, most
or all proofs were created manually.

Section \ref{metadefprop} discusses various definitions
that make propositional calculus easier to use.
For example, we define:

\begin{itemize}
\item $\varphi \vee \psi$
  is true if either $\varphi$ or $\psi$ (or both) are true
  (this is disjunction\index{disjunction ($\vee$)}
  aka logical {\sc or}\index{logical {\sc or} ($\vee$)}).

\item $\varphi \wedge \psi$
  is true if both $\varphi$ and $\psi$ are true
  (this is conjunction\index{conjunction ($\wedge$)}
  aka logical {\sc and}\index{logical {\sc and} ($\wedge$)}).

\item $\varphi \leftrightarrow \psi$
  is true if $\varphi$ and $\psi$ have the same value, that is,
  they are both true or both false
  (this is the biconditional\index{biconditional ($\leftrightarrow$)}).
\end{itemize}

\subsection{Predicate Calculus}

Predicate calculus\index{predicate calculus} introduces the concept of
``individual variables,''\index{variable!in predicate calculus}\index{individual
variable} which
we will usually just call ``variables.''
These variables can represent something other than true or false (wffs),
and will always represent sets when we get to set theory.  There are also
three new symbols $\forall$\index{universal quantifier ($\forall$)},
$=$\index{equality ($=$)}, and $\in$\index{stylized epsilon ($\in$)},
read ``for all,'' ``equals,'' and ``is an element of''
respectively.  We will represent variables with the letters $x$, $y$, $z$, and
$w$, as is common practice in the literature.
For example, $\forall x \varphi$ means ``for all possible values of
$x$, $\varphi$ is true.''

In predicate calculus, we extend the definition of a wff\index{well-formed
formula (wff)}.  If $\varphi$ is a wff and $x$ and $y$ are variables, then
$\forall x \, \varphi$, $x=y$, and $x\in y$ are wffs. Note that these three new
types of wffs can be considered ``starting'' wffs from which we can build
other wffs with $\rightarrow$ and $\neg$ .  The concept of a starting wff was
absent in propositional calculus.  But starting wff or not, all we are really
concerned with is whether our wffs are correctly constructed according to
these mechanical rules.

A quick aside:
To prevent confusion, it might be best at this point to think of the variables
of Metamath\index{Metamath} as ``metavariables,''\index{metavariable} because
they are not quite the same as the variables we are introducing here.  A
(meta)variable in Metamath can be a wff or an individual variable, as well
as many other things; in general, it represents a kind of place holder for an
unspecified sequence of math symbols\index{math symbol}.

Unlike propositional calculus, no decision procedure\index{decision procedure}
analogous to the truth table method exists (nor theoretically can exist) that
will definitely determine whether a formula is a theorem of predicate
calculus.  Much of the work in the field of automated theorem
proving\index{automated theorem proving} has been dedicated to coming up with
clever heuristics for proving theorems of predicate calculus, but they can
never be guaranteed to work always.

Section \ref{metadefpred} discusses various definitions
that make predicate calculus easier to use.
For example, we define
$\exists x \varphi$ to mean
``there exists at least one possible value of $x$ where $\varphi$ is true.''

We now turn to looking at how predicate calculus can be formally
represented.

\subsubsection{Common Axioms}

There is a new rule of inference in predicate calculus:  if $\varphi$ is
an axiom or a theorem, then $\forall x \,\varphi$ is also a
theorem\index{theorem}.  This is called the rule of
``generalization.''\index{rule of generalization}
This is easily represented in Metamath.

In standard texts of logic, there are often two axioms of predicate
calculus\index{axioms of predicate calculus}:
\begin{center}
  $\forall x \,\varphi ( x ) \rightarrow \varphi ( y )$,
      where ``$y$ is properly substituted for $x$.''\\
  $\forall x ( \varphi \rightarrow \psi )\rightarrow ( \varphi \rightarrow
    \forall x\, \psi )$,
    where ``$x$ is not free in $\varphi$.''
\end{center}

Now at first glance, this seems simple:  just two axioms.  However,
conditional clauses are attached to each axiom describing requirements that
may seem puzzling to you.  In addition, the first axiom puts a variable symbol
in parentheses after each wff, seemingly violating our definition of a
wff\index{well-formed formula (wff)}; this is just an informal way of
referring to some arbitrary variable that may occur in the wff.  The
conditional clauses do, of course, have a precise meaning, but as it turns out
the precise meaning is somewhat complicated and awkward to formalize in a
way that a computer can handle easily.  Unlike propositional calculus, a
certain amount of mathematical sophistication and practice is needed to be
able to easily grasp and manipulate these concepts correctly.

Predicate calculus may be presented with or without axioms for
equality\index{axioms of equality}\index{equality ($=$)}. We will require the
axioms of equality as a prerequisite for the version of set theory we will
use.  The axioms for equality, when included, are often represented using these
two axioms:
\begin{center}
$x=x$\\ \ \\
$x=y\rightarrow (\varphi(x,x)\rightarrow\varphi(x,y))$ where ``$\varphi(x,y)$
   arises from $\varphi(x,x)$ by replacing some, but not necessarily all,
   free\index{free variable}
   occurrences of $x$ by $y$,\\ provided that $y$ is free for $x$
   in $\varphi(x,x)$.'' \end{center}
% (Mendelson p. 95)
The first equality axiom is simple, but again,
the condition on the second one is
somewhat awkward to implement on a computer.

\subsubsection{Tarski System S2}

Of course, we are not the first to notice the complications of these
predicate calculus axioms when being rigorous.

Well-known logician Alfred Tarski published in 1965
a system he called system S2\cite[p.~77]{Tarski1965}.
Tarski's system is \textit{exactly equivalent} to the traditional textbook
formalization, but (by clever use of equality axioms) it eliminates the
latter's primitive notions of ``proper substitution'' and ``free variable,''
replacing them with direct substitution and the notion of a variable
not occurring in a formula (which we express with distinct variable
constraints).

In advocating his system, Tarski wrote, ``The relatively complicated
character of [free variables and proper substitution] is a source
of certain inconveniences of both practical and theoretical nature;
this is clearly experienced both in teaching an elementary course of
mathematical logic and in formalizing the syntax of predicate logic for
some theoretical purposes''\cite[p.~61]{Tarski1965}\index{Tarski, Alfred}.

\subsubsection{Developing a Metamath Representation}

The standard textbook axioms of predicate calculus are somewhat
cumbersome to implement on a computer because of the complex notions of
``free variable''\index{free variable} and ``proper
substitution.''\index{proper substitution}\index{substitution!proper}
While it is possible to use the Metamath\index{Metamath} language to
implement these concepts, we have chosen not to implement them
as primitive constructs in the
\texttt{set.mm} set theory database.  Instead, we have eliminated them
within the axioms
by carefully crafting the axioms so as to avoid them,
building on Tarski's system S2.  This makes it
easy for a beginner to follow the steps in a proof without knowing any
advanced concepts other than the simple concept of
replacing\index{substitution!variable}\index{variable substitution}
variables with expressions.

In order to develop the concepts of free variable and proper
substitution from the axioms, we use an additional
Metamath statement type called ``disjoint variable
restriction''\index{disjoint variables} that we have not encountered
before.  In the context of the axioms, the statement \texttt{\$d} $ x\,
y$\index{\texttt{\$d} statement} simply means that $x$ and $y$ must be
distinct\index{distinct variables}, i.e.\ they may not be simultaneously
substituted\index{substitution!variable}\index{variable substitution}
with the same variable.  The statement \texttt{\$d} $ x\, \varphi$ means
variable $x$ must not occur in wff $\varphi$.  For the precise
definition of \texttt{\$d}, see Section~\ref{dollard}.

\subsubsection{Metamath representation}

The Metamath axiom system for predicate calculus
defined in set.mm uses Tarski's system S2.
As noted above, this has a different representation
than the traditional textbook formalization,
but it is \textit{exactly equivalent} to the textbook formalization,
and it is \textit{much} easier to work with.
This is reproduced as system S3 in Section 6 of
Megill's formalization \cite{Megill}\index{Megill, Norman}.

There is one exception, Tarski's axiom of existence,
which we label as axiom ax-6.
In the case of ax-6, Tarski's version is weaker because it includes a
distinct variable proviso. If we wish, we can also weaken our version
in this way and still have a metalogically complete system. Theorem
ax6 shows this by deriving, in the presence of the other axioms, our
ax-6 from Tarski's weaker version ax6v. However, we chose the stronger
version for our system because it is simpler to state and easier to use.

Tarski's system was designed for proving specific theorems rather than
more general theorem schemes. However, theorem schemes are much more
efficient than specific theorems for building a body of mathematical
knowledge, since they can be reused with different instances as
needed. While Tarski does derive some theorem schemes from his axioms,
their proofs require concepts that are ``outside'' of the system, such as
induction on formula length. The verification of such proofs is difficult
to automate in a proof verifier. (Specifically, Tarski treats the formulas
of his system as set-theoretical objects. In order to verify the proofs
of his theorem schemes, a proof verifier would need a significant amount
of set theory built into it.)

The Metamath axiom system for predicate calculus extends
Tarski's system to eliminate this difficulty. The additional
``auxilliary'' axiom
schemes (as we will call them in this section; see below) endow Tarski's
system with a nice property we call
metalogical completeness \cite[Remark 9.6]{Megill}\index{Megill, Norman}.
As a result, we can prove any theorem scheme
expressable in the ``simple metalogic'' of Tarski's system by using
only Metamath's direct substitution rule applied to the axiom system
(and no other metalogical or set-theoretical notions ``outside'' of the
system). Simple metalogic consists of schemes containing wff metavariables
(with no arguments) and/or set (also called ``individual'') metavariables,
accompanied by optional provisos each stating that two specified set
metavariables must be distinct or that a specified set metavariable may
not occur in a specified wff metavariable. Metamath's logic and set theory
axiom and rule schemes are all examples of simple metalogic. The schemes
of traditional predicate calculus with equality are examples which are
not simple metalogic, because they use wff metavariables with arguments
and have ``free for'' and ``not free in'' side conditions.

A rigorous justification for this system, using an older but
exactly equivalent set of axioms, can be
found in \cite{Megill}\index{Megill, Norman}.

This allows us to
take a different approach in the Metamath\index{Metamath} database
\texttt{set.mm}\index{set theory database (\texttt{set.mm})}.  We do not
directly use the primitive notions of ``free variable''\index{free variable}
and ``proper substitution''\index{proper
substitution}\index{substitution!proper} at all as primitive constructs.
Instead, we use a set
of axioms that are almost as simple to manipulate as those of
propositional calculus.  Our axiom system avoids complex primitive
notions by effectively embedding the complexity into the axioms
themselves.  As a result, we will end up with a larger number of axioms,
but they are ideally suited for a computer language such as Metamath.
(Section~\ref{metaaxioms} shows these axioms.)

We will not elaborate further
on the ``free variable'' and ``proper substitution''
concepts here.  You may consult
\cite[ch.\ 3--4]{Hamilton}\index{Hamilton, Alan G.} (as well as
many other books) for a precise explanation
of these concepts.  If you intend to do serious mathematical work, it is wise
to become familiar with the traditional textbook approach; even though the
concepts embedded in their axioms require a higher level of sophistication,
they can be more practical to deal with on an everyday, informal basis.  Even
if you are just developing Metamath proofs, familiarity with the traditional
approach can help you arrive at a proof outline much faster, which you can
then convert to the detail required by Metamath.

We do develop proper substitution rules later on, but in set.mm
they are defined as derived constructs; they are not primitives.

You should also note that our system of predicate calculus is specifically
tailored for set theory; thus there are only two specific predicates $=$ and
$\in$ and no functions\index{function!in predicate calculus}
or constants\index{constant!in predicate calculus} unlike more general systems.
We later add these.

\subsection{Set Theory}

Traditional Zermelo--Fraenkel set theory\index{Zermelo--Fraenkel set
theory}\index{set theory} with the Axiom of Choice
has 10 axioms, which can be expressed in the
language of predicate calculus.  In this section, we will list only the
names and brief English descriptions of these axioms, since we will give
you the precise formulas used by the Metamath\index{Metamath} set theory
database \texttt{set.mm} later on.

In the descriptions of the axioms, we assume that $x$, $y$, $z$, $w$, and $v$
represent sets.  These are the same as the variables\index{variable!in set
theory} in our predicate calculus system above, except that now we informally
think of the variables as ranging over sets.  Note that the terms
``object,''\index{object} ``set,''\index{set} ``element,''\index{element}
``collection,''\index{collection} and ``family''\index{family} are synonymous,
as are ``is an element of,'' ``is a member of,''\index{member} ``is contained
in,'' and ``belongs to.''  The different terms are used for convenience; for
example, ``a collection of sets'' is less confusing than ``a set of sets.''
A set $x$ is said to be a ``subset''\index{subset} of $y$ if every element of
$x$ is also an element of $y$; we also say $x$ is ``included in''
$y$.

The axioms are very general and apply to almost any conceivable mathematical
object, and this level of abstraction can be overwhelming at first.  To gain an
intuitive feel, it can be helpful to draw a picture illustrating the concept;
for example, a circle containing dots could represent a collection of sets,
and a smaller circle drawn inside the circle could represent a subset.
Overlapping circles can illustrate intersection and union.  Circles that
illustrate the concepts of set theory are frequently used in elementary
textbooks and are called Venn diagrams\index{Venn diagram}.\index{axioms of
set theory}

1. Axiom of Extensionality:  Two sets are identical if they contain the same
   elements.\index{Axiom of Extensionality}

2. Axiom of Pairing:  The set $\{ x , y \}$ exists.\index{Axiom of Pairing}

3. Axiom of Power Sets:  The power set of a set (the collection of all of
   its subsets) exists.  For example, the power set of $\{x,y\}$ is
   $\{\varnothing,\{x\},\{y\},\{x,y\}\}$ and it exists.\index{Axiom
of Power Sets}

4. Axiom of the Null Set:  The empty set $\varnothing$ exists.\index{Axiom of
the Null Set}

5. Axiom of Union:  The union of a set (the set containing the elements of
   its members) exists.  For example, the union of $\{\{x,y\},\{z\}\}$ is
 $\{x,y,z\}$ and
   it exists.\index{Axiom of Union}

6. Axiom of Regularity:  Roughly, no set can contain itself, nor can there
   be membership ``loops,'' such as a set being an
   element of one of its members.\index{Axiom of Regularity}

7. Axiom of Infinity:  An infinite set exists.  An example of an infinite
   set is the set of all
   integers.\index{Axiom of Infinity}

8. Axiom of Separation:  The set exists that is obtained by restricting $x$
   with some property.  For example, if the set of all integers exists,
   then the set of all even integers exists.\index{Axiom of Separation}

9. Axiom of Replacement:  The range of a function whose domain is restricted
   to the elements of a set $x$, is also a set.  For example, there
   is a function
   from integers (the function's domain) to their squares (its
   range).  If we
   restrict the domain to even integers, its range will become the set of
   squares of even integers, so this axiom asserts that the set of
    squares of even numbers exists.  Technical note:  In general, the
   ``function'' need not be a set but can be a proper class.
   \index{Axiom of Replacement}

10. Axiom of Choice:  Let $x$ be a set whose members are pairwise
  disjoint\index{disjoint sets} (i.e,
  whose members contain no elements in common).  Then there exists another
  set containing one element from each member of $x$.  For
  example, if $x$ is
  $\{\{y,z\},\{w,v\}\}$, where $y$, $z$, $w$, and $v$ are
  different sets, then a set such as $\{z,w\}$
  exists (but the axiom doesn't tell
  us which one).  (Actually the Axiom
  of Choice is redundant if the set $x$, as in this example, has a finite
  number of elements.)\index{Axiom of Choice}

The Axiom of Choice is usually considered an extension of ZF set theory rather
than a proper part of it.  It is sometimes considered philosophically
controversial because it specifies the existence of a set without specifying
what the set is. Constructive logics, including intuitionistic logic,
do not accept the axiom of choice.
Since there is some lingering controversy, we often prefer proofs that do
not use the axiom of choice (where there is a known alternative), and
in some cases we will use weaker axioms than the full axiom of choice.
That said, the axiom of choice is a powerful and widely-accepted tool,
so we do use it when needed.
ZF set theory that includes the Axiom of Choice is
called Zermelo--Fraenkel set theory with choice (ZFC\index{ZFC set theory}).

When expressed symbolically, the Axiom of Separation and the Axiom of
Replacement contain wff symbols and therefore each represent infinitely many
axioms, one for each possible wff. For this reason, they are often called
axiom schemes\index{axiom scheme}\index{well-formed formula (wff)}.

It turns out that the Axiom of the Null Set, the Axiom of Pairing, and the
Axiom of Separation can be derived from the other axioms and are therefore
unnecessary, although they tend to be included in standard texts for various
reasons (historical, philosophical, and possibly because some authors may not
know this).  In the Metamath\index{Metamath} set theory database, these
redundant axioms are derived from the other ones instead of truly
being considered axioms.
This is in keeping with our general goal of minimizing the number of
axioms we must depend on.

\subsection{Other Axioms}

Above we qualified the phrase "all of mathematics" with "essentially."
The main important missing piece is the ability to do category theory,
which requires huge sets (inaccessible cardinals) larger than those
postulated by the ZFC axioms. The Tarski--Grothendieck Axiom postulates
the existence of such sets.
Note that this is the same axiom used by Mizar for supporting
category theory.
The Tarski--Grothendieck axiom
can be viewed as a very strong replacement of the Axiom of Infinity,
the Axiom of Choice, and the Axiom of Power Sets.
The \texttt{set.mm} database includes this axiom; see the database
for details about it.
Again, we only use this axiom when we need to.
You are only likely to encounter or use this axiom if you are doing
category theory, since its use is highly specialized,
so we will not list the Tarsky-Grothendieck axiom
in the short list of axioms below.

Can there be even more axioms?
Of course.
G\"{o}del showed that no finite set of axioms or axiom schemes can completely
describe any consistent theory strong enough to include arithmetic.
But practically speaking, the ones above are the accepted foundation that
almost all mathematicians explicitly or implicitly base their work on.

\section{The Axioms in the Metamath Language}\label{metaaxioms}

Here we list the axioms as they appear in
\texttt{set.mm}\index{set theory database (\texttt{set.mm})} so you can
look them up there easily.  Incidentally, the \texttt{show statement
/tex} command\index{\texttt{show statement} command} was used to
typeset them.

%macros from show statement /tex
\newbox\mlinebox
\newbox\mtrialbox
\newbox\startprefix  % Prefix for first line of a formula
\newbox\contprefix  % Prefix for continuation line of a formula
\def\startm{  % Initialize formula line
  \setbox\mlinebox=\hbox{\unhcopy\startprefix}
}
\def\m#1{  % Add a symbol to the formula
  \setbox\mtrialbox=\hbox{\unhcopy\mlinebox $\,#1$}
  \ifdim\wd\mtrialbox>\hsize
    \box\mlinebox
    \setbox\mlinebox=\hbox{\unhcopy\contprefix $\,#1$}
  \else
    \setbox\mlinebox=\hbox{\unhbox\mtrialbox}
  \fi
}
\def\endm{  % Output the last line of a formula
  \box\mlinebox
}

% \SLASH for \ , \TOR for \/ (text OR), \TAND for /\ (text and)
% This embeds a following forced space to force the space.
\newcommand\SLASH{\char`\\~}
\newcommand\TOR{\char`\\/~}
\newcommand\TAND{/\char`\\~}
%
% Macro to output metamath raw text.
% This assumes \startprefix and \contprefix are set.
% NOTE: "\" is tricky to escape, use \SLASH, \TOR, and \TAND inside.
% Any use of "$ { ~ ^" must be escaped; ~ and ^ must be escaped specially.
% We escape { and } for consistency.
% For more about how this macro written, see:
% https://stackoverflow.com/questions/4073674/
% how-to-disable-indentation-in-particular-section-in-latex/4075706
% Use frenchspacing, or "e." will get an extra space after it.
\newlength\mystoreparindent
\newlength\mystorehangindent
\newenvironment{mmraw}{%
\setlength{\mystoreparindent}{\the\parindent}
\setlength{\mystorehangindent}{\the\hangindent}
\setlength{\parindent}{0pt} % TODO - we'll put in the \startprefix instead
\setlength{\hangindent}{\wd\the\contprefix}
\begin{flushleft}
\begin{frenchspacing}
\begin{tt}
{\unhcopy\startprefix}%
}{%
\end{tt}
\end{frenchspacing}
\end{flushleft}
\setlength{\parindent}{\mystoreparindent}
\setlength{\hangindent}{\mystorehangindent}
\vskip 1ex
}

\needspace{5\baselineskip}
\subsection{Propositional Calculus}\label{propcalc}\index{axioms of
propositional calculus}

\needspace{2\baselineskip}
Axiom of Simplification.\label{ax1}

\setbox\startprefix=\hbox{\tt \ \ ax-1\ \$a\ }
\setbox\contprefix=\hbox{\tt \ \ \ \ \ \ \ \ \ \ }
\startm
\m{\vdash}\m{(}\m{\varphi}\m{\rightarrow}\m{(}\m{\psi}\m{\rightarrow}\m{\varphi}\m{)}
\m{)}
\endm

\needspace{3\baselineskip}
\noindent Axiom of Distribution.

\setbox\startprefix=\hbox{\tt \ \ ax-2\ \$a\ }
\setbox\contprefix=\hbox{\tt \ \ \ \ \ \ \ \ \ \ }
\startm
\m{\vdash}\m{(}\m{(}\m{\varphi}\m{\rightarrow}\m{(}\m{\psi}\m{\rightarrow}\m{\chi}
\m{)}\m{)}\m{\rightarrow}\m{(}\m{(}\m{\varphi}\m{\rightarrow}\m{\psi}\m{)}\m{
\rightarrow}\m{(}\m{\varphi}\m{\rightarrow}\m{\chi}\m{)}\m{)}\m{)}
\endm

\needspace{2\baselineskip}
\noindent Axiom of Contraposition.

\setbox\startprefix=\hbox{\tt \ \ ax-3\ \$a\ }
\setbox\contprefix=\hbox{\tt \ \ \ \ \ \ \ \ \ \ }
\startm
\m{\vdash}\m{(}\m{(}\m{\lnot}\m{\varphi}\m{\rightarrow}\m{\lnot}\m{\psi}\m{)}\m{
\rightarrow}\m{(}\m{\psi}\m{\rightarrow}\m{\varphi}\m{)}\m{)}
\endm


\needspace{4\baselineskip}
\noindent Rule of Modus Ponens.\label{axmp}\index{modus ponens}

\setbox\startprefix=\hbox{\tt \ \ maj\ \$e\ }
\setbox\contprefix=\hbox{\tt \ \ \ \ \ \ \ \ \ }
\startm
\m{\vdash}\m{(}\m{\varphi}\m{\rightarrow}\m{\psi}\m{)}
\endm

\setbox\startprefix=\hbox{\tt \ \ min\ \$e\ }
\setbox\contprefix=\hbox{\tt \ \ \ \ \ \ \ \ \ }
\startm
\m{\vdash}\m{\varphi}
\endm

\setbox\startprefix=\hbox{\tt \ \ ax-mp\ \$a\ }
\setbox\contprefix=\hbox{\tt \ \ \ \ \ \ \ \ \ \ \ }
\startm
\m{\vdash}\m{\psi}
\endm


\needspace{7\baselineskip}
\subsection{Axioms of Predicate Calculus with Equality---Tarski's S2}\index{axioms of predicate calculus}

\needspace{3\baselineskip}
\noindent Rule of Generalization.\index{rule of generalization}

\setbox\startprefix=\hbox{\tt \ \ ax-g.1\ \$e\ }
\setbox\contprefix=\hbox{\tt \ \ \ \ \ \ \ \ \ \ \ \ }
\startm
\m{\vdash}\m{\varphi}
\endm

\setbox\startprefix=\hbox{\tt \ \ ax-gen\ \$a\ }
\setbox\contprefix=\hbox{\tt \ \ \ \ \ \ \ \ \ \ \ \ }
\startm
\m{\vdash}\m{\forall}\m{x}\m{\varphi}
\endm

\needspace{2\baselineskip}
\noindent Axiom of Quantified Implication.

\setbox\startprefix=\hbox{\tt \ \ ax-4\ \$a\ }
\setbox\contprefix=\hbox{\tt \ \ \ \ \ \ \ \ \ \ }
\startm
\m{\vdash}\m{(}\m{\forall}\m{x}\m{(}\m{\forall}\m{x}\m{\varphi}\m{\rightarrow}\m{
\psi}\m{)}\m{\rightarrow}\m{(}\m{\forall}\m{x}\m{\varphi}\m{\rightarrow}\m{
\forall}\m{x}\m{\psi}\m{)}\m{)}
\endm

\needspace{3\baselineskip}
\noindent Axiom of Distinctness.

% Aka: Add $d x ph $.
\setbox\startprefix=\hbox{\tt \ \ ax-5\ \$a\ }
\setbox\contprefix=\hbox{\tt \ \ \ \ \ \ \ \ \ \ }
\startm
\m{\vdash}\m{(}\m{\varphi}\m{\rightarrow}\m{\forall}\m{x}\m{\varphi}\m{)}\m{where}\m{ }\m{\$d}\m{ }\m{x}\m{ }\m{\varphi}\m{ }\m{(}\m{x}\m{ }\m{does}\m{ }\m{not}\m{ }\m{occur}\m{ }\m{in}\m{ }\m{\varphi}\m{)}
\endm

\needspace{2\baselineskip}
\noindent Axiom of Existence.

\setbox\startprefix=\hbox{\tt \ \ ax-6\ \$a\ }
\setbox\contprefix=\hbox{\tt \ \ \ \ \ \ \ \ \ \ }
\startm
\m{\vdash}\m{(}\m{\forall}\m{x}\m{(}\m{x}\m{=}\m{y}\m{\rightarrow}\m{\forall}
\m{x}\m{\varphi}\m{)}\m{\rightarrow}\m{\varphi}\m{)}
\endm

\needspace{2\baselineskip}
\noindent Axiom of Equality.

\setbox\startprefix=\hbox{\tt \ \ ax-7\ \$a\ }
\setbox\contprefix=\hbox{\tt \ \ \ \ \ \ \ \ \ \ }
\startm
\m{\vdash}\m{(}\m{x}\m{=}\m{y}\m{\rightarrow}\m{(}\m{x}\m{=}\m{z}\m{
\rightarrow}\m{y}\m{=}\m{z}\m{)}\m{)}
\endm

\needspace{2\baselineskip}
\noindent Axiom of Left Equality for Binary Predicate.

\setbox\startprefix=\hbox{\tt \ \ ax-8\ \$a\ }
\setbox\contprefix=\hbox{\tt \ \ \ \ \ \ \ \ \ \ \ }
\startm
\m{\vdash}\m{(}\m{x}\m{=}\m{y}\m{\rightarrow}\m{(}\m{x}\m{\in}\m{z}\m{
\rightarrow}\m{y}\m{\in}\m{z}\m{)}\m{)}
\endm

\needspace{2\baselineskip}
\noindent Axiom of Right Equality for Binary Predicate.

\setbox\startprefix=\hbox{\tt \ \ ax-9\ \$a\ }
\setbox\contprefix=\hbox{\tt \ \ \ \ \ \ \ \ \ \ \ }
\startm
\m{\vdash}\m{(}\m{x}\m{=}\m{y}\m{\rightarrow}\m{(}\m{z}\m{\in}\m{x}\m{
\rightarrow}\m{z}\m{\in}\m{y}\m{)}\m{)}
\endm


\needspace{4\baselineskip}
\subsection{Axioms of Predicate Calculus with Equality---Auxiliary}\index{axioms of predicate calculus - auxiliary}

\needspace{2\baselineskip}
\noindent Axiom of Quantified Negation.

\setbox\startprefix=\hbox{\tt \ \ ax-10\ \$a\ }
\setbox\contprefix=\hbox{\tt \ \ \ \ \ \ \ \ \ \ }
\startm
\m{\vdash}\m{(}\m{\lnot}\m{\forall}\m{x}\m{\lnot}\m{\forall}\m{x}\m{\varphi}\m{
\rightarrow}\m{\varphi}\m{)}
\endm

\needspace{2\baselineskip}
\noindent Axiom of Quantifier Commutation.

\setbox\startprefix=\hbox{\tt \ \ ax-11\ \$a\ }
\setbox\contprefix=\hbox{\tt \ \ \ \ \ \ \ \ \ \ }
\startm
\m{\vdash}\m{(}\m{\forall}\m{x}\m{\forall}\m{y}\m{\varphi}\m{\rightarrow}\m{
\forall}\m{y}\m{\forall}\m{x}\m{\varphi}\m{)}
\endm

\needspace{3\baselineskip}
\noindent Axiom of Substitution.

\setbox\startprefix=\hbox{\tt \ \ ax-12\ \$a\ }
\setbox\contprefix=\hbox{\tt \ \ \ \ \ \ \ \ \ \ \ }
\startm
\m{\vdash}\m{(}\m{\lnot}\m{\forall}\m{x}\m{\,x}\m{=}\m{y}\m{\rightarrow}\m{(}
\m{x}\m{=}\m{y}\m{\rightarrow}\m{(}\m{\varphi}\m{\rightarrow}\m{\forall}\m{x}\m{(}
\m{x}\m{=}\m{y}\m{\rightarrow}\m{\varphi}\m{)}\m{)}\m{)}\m{)}
\endm

\needspace{3\baselineskip}
\noindent Axiom of Quantified Equality.

\setbox\startprefix=\hbox{\tt \ \ ax-13\ \$a\ }
\setbox\contprefix=\hbox{\tt \ \ \ \ \ \ \ \ \ \ \ }
\startm
\m{\vdash}\m{(}\m{\lnot}\m{\forall}\m{z}\m{\,z}\m{=}\m{x}\m{\rightarrow}\m{(}
\m{\lnot}\m{\forall}\m{z}\m{\,z}\m{=}\m{y}\m{\rightarrow}\m{(}\m{x}\m{=}\m{y}
\m{\rightarrow}\m{\forall}\m{z}\m{\,x}\m{=}\m{y}\m{)}\m{)}\m{)}
\endm

% \noindent Axiom of Quantifier Substitution
%
% \setbox\startprefix=\hbox{\tt \ \ ax-c11n\ \$a\ }
% \setbox\contprefix=\hbox{\tt \ \ \ \ \ \ \ \ \ \ \ }
% \startm
% \m{\vdash}\m{(}\m{\forall}\m{x}\m{\,x}\m{=}\m{y}\m{\rightarrow}\m{(}\m{\forall}
% \m{x}\m{\varphi}\m{\rightarrow}\m{\forall}\m{y}\m{\varphi}\m{)}\m{)}
% \endm
%
% \noindent Axiom of Distinct Variables. (This axiom requires
% that two individual variables
% be distinct\index{\texttt{\$d} statement}\index{distinct
% variables}.)
%
% \setbox\startprefix=\hbox{\tt \ \ \ \ \ \ \ \ \$d\ }
% \setbox\contprefix=\hbox{\tt \ \ \ \ \ \ \ \ \ \ \ }
% \startm
% \m{x}\m{\,}\m{y}
% \endm
%
% \setbox\startprefix=\hbox{\tt \ \ ax-c16\ \$a\ }
% \setbox\contprefix=\hbox{\tt \ \ \ \ \ \ \ \ \ \ \ }
% \startm
% \m{\vdash}\m{(}\m{\forall}\m{x}\m{\,x}\m{=}\m{y}\m{\rightarrow}\m{(}\m{\varphi}\m{
% \rightarrow}\m{\forall}\m{x}\m{\varphi}\m{)}\m{)}
% \endm

% \noindent Axiom of Quantifier Introduction (2).  (This axiom requires
% that the individual variable not occur in the
% wff\index{\texttt{\$d} statement}\index{distinct variables}.)
%
% \setbox\startprefix=\hbox{\tt \ \ \ \ \ \ \ \ \$d\ }
% \setbox\contprefix=\hbox{\tt \ \ \ \ \ \ \ \ \ \ \ }
% \startm
% \m{x}\m{\,}\m{\varphi}
% \endm
% \setbox\startprefix=\hbox{\tt \ \ ax-5\ \$a\ }
% \setbox\contprefix=\hbox{\tt \ \ \ \ \ \ \ \ \ \ \ }
% \startm
% \m{\vdash}\m{(}\m{\varphi}\m{\rightarrow}\m{\forall}\m{x}\m{\varphi}\m{)}
% \endm

\subsection{Set Theory}\label{mmsettheoryaxioms}

In order to make the axioms of set theory\index{axioms of set theory} a little
more compact, there are several definitions from logic that we make use of
implicitly, namely, ``logical {\sc and},''\index{conjunction ($\wedge$)}
\index{logical {\sc and} ($\wedge$)} ``logical equivalence,''\index{logical
equivalence ($\leftrightarrow$)}\index{biconditional ($\leftrightarrow$)} and
``there exists.''\index{existential quantifier ($\exists$)}

\begin{center}\begin{tabular}{rcl}
  $( \varphi \wedge \psi )$ &\mbox{stands for}& $\neg ( \varphi
     \rightarrow \neg \psi )$\\
  $( \varphi \leftrightarrow \psi )$& \mbox{stands
     for}& $( ( \varphi \rightarrow \psi ) \wedge
     ( \psi \rightarrow \varphi ) )$\\
  $\exists x \,\varphi$ &\mbox{stands for}& $\neg \forall x \neg \varphi$
\end{tabular}\end{center}

In addition, the axioms of set theory require that all variables be
dis\-tinct,\index{distinct variables}\footnote{Set theory axioms can be
devised so that {\em no} variables are required to be distinct,
provided we replace \texttt{ax-c16} with an axiom stating that ``at
least two things exist,'' thus
making \texttt{ax-5} the only other axiom requiring the
\texttt{\$d} statement.  These axioms are unconventional and are not
presented here, but they can be found on the \url{http://metamath.org}
web site.  See also the Comment on
p.~\pageref{nodd}.}\index{\texttt{\$d} statement} thus we also assume:
\begin{center}
  \texttt{\$d }$x\,y\,z\,w$
\end{center}

\needspace{2\baselineskip}
\noindent Axiom of Extensionality.\index{Axiom of Extensionality}

\setbox\startprefix=\hbox{\tt \ \ ax-ext\ \$a\ }
\setbox\contprefix=\hbox{\tt \ \ \ \ \ \ \ \ \ \ \ \ }
\startm
\m{\vdash}\m{(}\m{\forall}\m{x}\m{(}\m{x}\m{\in}\m{y}\m{\leftrightarrow}\m{x}
\m{\in}\m{z}\m{)}\m{\rightarrow}\m{y}\m{=}\m{z}\m{)}
\endm

\needspace{3\baselineskip}
\noindent Axiom of Replacement.\index{Axiom of Replacement}

\setbox\startprefix=\hbox{\tt \ \ ax-rep\ \$a\ }
\setbox\contprefix=\hbox{\tt \ \ \ \ \ \ \ \ \ \ \ \ }
\startm
\m{\vdash}\m{(}\m{\forall}\m{w}\m{\exists}\m{y}\m{\forall}\m{z}\m{(}\m{%
\forall}\m{y}\m{\varphi}\m{\rightarrow}\m{z}\m{=}\m{y}\m{)}\m{\rightarrow}\m{%
\exists}\m{y}\m{\forall}\m{z}\m{(}\m{z}\m{\in}\m{y}\m{\leftrightarrow}\m{%
\exists}\m{w}\m{(}\m{w}\m{\in}\m{x}\m{\wedge}\m{\forall}\m{y}\m{\varphi}\m{)}%
\m{)}\m{)}
\endm

\needspace{2\baselineskip}
\noindent Axiom of Union.\index{Axiom of Union}

\setbox\startprefix=\hbox{\tt \ \ ax-un\ \$a\ }
\setbox\contprefix=\hbox{\tt \ \ \ \ \ \ \ \ \ \ \ }
\startm
\m{\vdash}\m{\exists}\m{x}\m{\forall}\m{y}\m{(}\m{\exists}\m{x}\m{(}\m{y}\m{
\in}\m{x}\m{\wedge}\m{x}\m{\in}\m{z}\m{)}\m{\rightarrow}\m{y}\m{\in}\m{x}\m{)}
\endm

\needspace{2\baselineskip}
\noindent Axiom of Power Sets.\index{Axiom of Power Sets}

\setbox\startprefix=\hbox{\tt \ \ ax-pow\ \$a\ }
\setbox\contprefix=\hbox{\tt \ \ \ \ \ \ \ \ \ \ \ \ }
\startm
\m{\vdash}\m{\exists}\m{x}\m{\forall}\m{y}\m{(}\m{\forall}\m{x}\m{(}\m{x}\m{
\in}\m{y}\m{\rightarrow}\m{x}\m{\in}\m{z}\m{)}\m{\rightarrow}\m{y}\m{\in}\m{x}
\m{)}
\endm

\needspace{3\baselineskip}
\noindent Axiom of Regularity.\index{Axiom of Regularity}

\setbox\startprefix=\hbox{\tt \ \ ax-reg\ \$a\ }
\setbox\contprefix=\hbox{\tt \ \ \ \ \ \ \ \ \ \ \ \ }
\startm
\m{\vdash}\m{(}\m{\exists}\m{x}\m{\,x}\m{\in}\m{y}\m{\rightarrow}\m{\exists}
\m{x}\m{(}\m{x}\m{\in}\m{y}\m{\wedge}\m{\forall}\m{z}\m{(}\m{z}\m{\in}\m{x}\m{
\rightarrow}\m{\lnot}\m{z}\m{\in}\m{y}\m{)}\m{)}\m{)}
\endm

\needspace{3\baselineskip}
\noindent Axiom of Infinity.\index{Axiom of Infinity}

\setbox\startprefix=\hbox{\tt \ \ ax-inf\ \$a\ }
\setbox\contprefix=\hbox{\tt \ \ \ \ \ \ \ \ \ \ \ \ \ \ \ }
\startm
\m{\vdash}\m{\exists}\m{x}\m{(}\m{y}\m{\in}\m{x}\m{\wedge}\m{\forall}\m{y}%
\m{(}\m{y}\m{\in}\m{x}\m{\rightarrow}\m{\exists}\m{z}\m{(}\m{y}\m{\in}\m{z}\m{%
\wedge}\m{z}\m{\in}\m{x}\m{)}\m{)}\m{)}
\endm

\needspace{4\baselineskip}
\noindent Axiom of Choice.\index{Axiom of Choice}

\setbox\startprefix=\hbox{\tt \ \ ax-ac\ \$a\ }
\setbox\contprefix=\hbox{\tt \ \ \ \ \ \ \ \ \ \ \ \ \ \ }
\startm
\m{\vdash}\m{\exists}\m{x}\m{\forall}\m{y}\m{\forall}\m{z}\m{(}\m{(}\m{y}\m{%
\in}\m{z}\m{\wedge}\m{z}\m{\in}\m{w}\m{)}\m{\rightarrow}\m{\exists}\m{w}\m{%
\forall}\m{y}\m{(}\m{\exists}\m{w}\m{(}\m{(}\m{y}\m{\in}\m{z}\m{\wedge}\m{z}%
\m{\in}\m{w}\m{)}\m{\wedge}\m{(}\m{y}\m{\in}\m{w}\m{\wedge}\m{w}\m{\in}\m{x}%
\m{)}\m{)}\m{\leftrightarrow}\m{y}\m{=}\m{w}\m{)}\m{)}
\endm

\subsection{That's It}

There you have it, the axioms for (essentially) all of mathematics!
Wonder at them and stare at them in awe.  Put a copy in your wallet, and
you will carry in your pocket the encoding for all theorems ever proved
and that ever will be proved, from the most mundane to the most
profound.

\section{A Hierarchy of Definitions}\label{hierarchy}

The axioms in the previous section in principle embody everything that can be
done within standard mathematics.  However, it is impractical to accomplish
very much by using them directly, for even simple concepts (from a human
perspective) can involve extremely long, incomprehensible formulas.
Mathematics is made practical by introducing definitions\index{definition}.
Definitions usually introduce new symbols, or at least new relationships among
existing symbols, to abbreviate more complex formulas.  An important
requirement for a definition is that there exist a straightforward
(algorithmic) method for eliminating the abbreviation by expanding it into the
more primitive symbol string that it represents.  Some
important definitions included in
the file \texttt{set.mm} are listed in this section for reference, and also to
give you a feel for why something like $\omega$\index{omega ($\omega$)} (the
set of natural numbers\index{natural number} 0, 1, 2,\ldots) becomes very
complicated when completely expanded into primitive symbols.

What is the motivation for definitions, aside from allowing complicated
expressions to be expressed more simply?  In the case of  $\omega$, one goal is
to provide a basis for the theory of natural numbers.\index{natural number}
Before set theory was invented, a set of axioms for arithmetic, called Peano's
postulates\index{Peano's postulates}, was devised and shown to have the
properties one expects for natural numbers.  Now anyone can postulate a
set of axioms, but if the axioms are inconsistent contradictions can be derived
from them.  Once a contradiction is derived, anything can be trivially
proved, including
all the facts of arithmetic and their negations.  To ensure that an
axiom system is at least as reliable as the axioms for set theory, we can
define sets and operations on those sets that satisfy the new axioms. In the
\texttt{set.mm} Metamath database, we prove that the elements of $\omega$ satisfy
Peano's postulates, and it's a long and hard journey to get there directly
from the axioms of set theory.  But the result is confidence in the
foundations of arithmetic.  And there is another advantage:  we now have all
the tools of set theory at our disposal for manipulating objects that obey the
axioms for arithmetic.

What are the criteria we use for definitions?  First, and of utmost importance,
the definition should not be {\em creative}\index{creative
definition}\index{definition!creative}, that
is it should not allow an expression that previously qualified as a wff but
was not provable, to become provable.   Second, the definition should be {\em
eliminable}\index{definition!eliminability}, that is, there should exist an
algorithmic method for converting any expression using the definition into
a logically equivalent expression that previously qualified as a wff.

In almost all cases below, definitions connect two expressions with either
$\leftrightarrow$ or $=$.  Eliminating\footnote{Here we mean the
elimination that a human might do in his or her head.  To eliminate them as
part of a Metamath proof we would invoke one of a number of
theorems that deal with transitivity of equivalence or equality; there are
many such examples in the proofs in \texttt{set.mm}.} such a definition is a
simple matter of substituting the expression on the left-hand side ({\em
definiendum}\index{definiendum} or thing being defined) with the equivalent,
more primitive expression on the right-hand side ({\em
definiens}\index{definiens} or definition).

Often a definition has variables on the right-hand side which do not appear on
the left-hand side; these are called {\em dummy variables}.\index{dummy
variable!in definitions}  In this case, any
allowable substitution (such as a new, distinct
variable) can be used when the definition is eliminated.  Dummy variables may
be used only if they are {\em effectively bound}\index{effectively bound
variable}, meaning that the definition will remain logically equivalent upon
any substitution of a dummy variable with any other {\em qualifying
expression}\index{qualifying expression}, i.e.\ any symbol string (such as
another variable) that
meets the restrictions on the dummy variable imposed by \texttt{\$d} and
\texttt{\$f} statements.  For example, we could define a constant $\perp$
(inverted tee, meaning logical ``false'') as $( \varphi \wedge \lnot \varphi
)$, i.e.\ ``phi and not phi.''  Here $\varphi$ is effectively bound because the
definition remains logically equivalent when we replace $\varphi$ with any
other wff.  (It is actually \texttt{df-fal}
in \texttt{set.mm}, which defines $\perp$.)

There are two cases where eliminating definitions is a little more
complex.  These cases are the definitions \texttt{df-bi} and
\texttt{df-cleq}.  The first stretches the concept of a definition a
little, as in effect it ``defines a definition;'' however, it meets our
requirements for a definition in that it is eliminable and does not
strengthen the language.  Theorem \texttt{bii} shows the substitution
needed to eliminate the $\leftrightarrow$\index{logical equivalence
($\leftrightarrow$)}\index{biconditional ($\leftrightarrow$)} symbol.

Definition \texttt{df-cleq}\index{equality ($=$)} extends the usage of
the equality symbol to include ``classes''\index{class} in set theory.  The
reason it is potentially problematic is that it can lead to statements which
do not follow from logic alone but presuppose the Axiom of
Extensionality\index{Axiom of Extensionality}, so we include this axiom
as a hypothesis for the definition.  We could have made \texttt{df-cleq} directly
eliminable by introducing a new equality symbol, but have chosen not to do so
in keeping with standard textbook practice.  Definitions such as \texttt{df-cleq}
that extend the meaning of existing symbols must be introduced carefully so
that they do not lead to contradictions.  Definition \texttt{df-clel} also
extends the meaning of an existing symbol ($\in$); while it doesn't strengthen
the language like \texttt{df-cleq}, this is not obvious and it must also be
subject to the same scrutiny.

Exercise:  Study how the wff $x\in\omega$, meaning ``$x$ is a natural
number,'' could be expanded in terms of primitive symbols, starting with the
definitions \texttt{df-clel} on p.~\pageref{dfclel} and \texttt{df-om} on
p.~\pageref{dfom} and working your way back.  Don't bother to work out the
details; just make sure that you understand how you could do it in principle.
The answer is shown in the footnote on p.~\pageref{expandom}.  If you
actually do work it out, you won't get exactly the same answer because we used
a few simplifications such as discarding occurrences of $\lnot\lnot$ (double
negation).

In the definitions below, we have placed the {\sc ascii} Metamath source
below each of the formulas to help you become familiar with the
notation in the database.  For simplicity, the necessary \texttt{\$f}
and \texttt{\$d} statements are not shown.  If you are in doubt, use the
\texttt{show statement}\index{\texttt{show statement} command} command
in the Metamath program to see the full statement.
A selection of this notation is summarized in Appendix~\ref{ASCII}.

To understand the motivation for these definitions, you should consult the
references indicated:  Takeuti and Zaring \cite{Takeuti}\index{Takeuti, Gaisi},
Quine \cite{Quine}\index{Quine, Willard Van Orman}, Bell and Machover
\cite{Bell}\index{Bell, J. L.}, and Enderton \cite{Enderton}\index{Enderton,
Herbert B.}.  Our list of definitions is provided more for reference than as a
learning aid.  However, by looking at a few of them you can gain a feel for
how the hierarchy is built up.  The definitions are a representative sample of
the 130 or so in \texttt{set.mm}, but they are complete with respect to the
theorem examples we will present in Section~\ref{sometheorems}.  Also, some are
slightly different from, but logically equivalent to, the ones in \texttt{set.mm}
(some of which have been revised over time to shorten them, for example).

\subsection{Definitions for Propositional Calculus}\label{metadefprop}

The symbols $\varphi$, $\psi$, and $\chi$ represent wffs.

Our first definition introduces the biconditional
connective\footnote{The term ``connective'' is informally used to mean a
symbol that is placed between two variables or adjacent to a variable,
whereas a mathematical ``constant'' usually indicates a symbol such as
the number 0 that may replace a variable or metavariable.  From
Metamath's point of view, there is no distinction between a connective
and a constant; both are constants in the Metamath
language.}\index{connective}\index{constant} (also called logical
equivalence)\index{logical equivalence
($\leftrightarrow$)}\index{biconditional ($\leftrightarrow$)}.  Unlike
most traditional developments, we have chosen not to have a separate
symbol such as ``Df.'' to mean ``is defined as.''  Instead, we will use
the biconditional connective for this purpose, as it lets us use
logic to manipulate definitions directly.  Here we state the properties
of the biconditional connective with a carefully crafted \texttt{\$a}
statement, which effectively uses the biconditional connective to define
itself.  The $\leftrightarrow$ symbol can be eliminated from a formula
using theorem \texttt{bii}, which is derived later.

\vskip 2ex
\noindent Define the biconditional connective.\label{df-bi}

\vskip 0.5ex
\setbox\startprefix=\hbox{\tt \ \ df-bi\ \$a\ }
\setbox\contprefix=\hbox{\tt \ \ \ \ \ \ \ \ \ \ \ }
\startm
\m{\vdash}\m{\lnot}\m{(}\m{(}\m{(}\m{\varphi}\m{\leftrightarrow}\m{\psi}\m{)}%
\m{\rightarrow}\m{\lnot}\m{(}\m{(}\m{\varphi}\m{\rightarrow}\m{\psi}\m{)}\m{%
\rightarrow}\m{\lnot}\m{(}\m{\psi}\m{\rightarrow}\m{\varphi}\m{)}\m{)}\m{)}\m{%
\rightarrow}\m{\lnot}\m{(}\m{\lnot}\m{(}\m{(}\m{\varphi}\m{\rightarrow}\m{%
\psi}\m{)}\m{\rightarrow}\m{\lnot}\m{(}\m{\psi}\m{\rightarrow}\m{\varphi}\m{)}%
\m{)}\m{\rightarrow}\m{(}\m{\varphi}\m{\leftrightarrow}\m{\psi}\m{)}\m{)}\m{)}
\endm
\begin{mmraw}%
|- -. ( ( ( ph <-> ps ) -> -. ( ( ph -> ps ) ->
-. ( ps -> ph ) ) ) -> -. ( -. ( ( ph -> ps ) -> -. (
ps -> ph ) ) -> ( ph <-> ps ) ) ) \$.
\end{mmraw}

\noindent This theorem relates the biconditional connective to primitive
connectives and can be used to eliminate the $\leftrightarrow$ symbol from any
wff.

\vskip 0.5ex
\setbox\startprefix=\hbox{\tt \ \ bii\ \$p\ }
\setbox\contprefix=\hbox{\tt \ \ \ \ \ \ \ \ \ }
\startm
\m{\vdash}\m{(}\m{(}\m{\varphi}\m{\leftrightarrow}\m{\psi}\m{)}\m{\leftrightarrow}
\m{\lnot}\m{(}\m{(}\m{\varphi}\m{\rightarrow}\m{\psi}\m{)}\m{\rightarrow}\m{\lnot}
\m{(}\m{\psi}\m{\rightarrow}\m{\varphi}\m{)}\m{)}\m{)}
\endm
\begin{mmraw}%
|- ( ( ph <-> ps ) <-> -. ( ( ph -> ps ) -> -. ( ps -> ph ) ) ) \$= ... \$.
\end{mmraw}

\noindent Define disjunction ({\sc or}).\index{disjunction ($\vee$)}%
\index{logical or (vee)@logical {\sc or} ($\vee$)}%
\index{df-or@\texttt{df-or}}\label{df-or}

\vskip 0.5ex
\setbox\startprefix=\hbox{\tt \ \ df-or\ \$a\ }
\setbox\contprefix=\hbox{\tt \ \ \ \ \ \ \ \ \ \ \ }
\startm
\m{\vdash}\m{(}\m{(}\m{\varphi}\m{\vee}\m{\psi}\m{)}\m{\leftrightarrow}\m{(}\m{
\lnot}\m{\varphi}\m{\rightarrow}\m{\psi}\m{)}\m{)}
\endm
\begin{mmraw}%
|- ( ( ph \TOR ps ) <-> ( -. ph -> ps ) ) \$.
\end{mmraw}

\noindent Define conjunction ({\sc and}).\index{conjunction ($\wedge$)}%
\index{logical {\sc and} ($\wedge$)}%
\index{df-an@\texttt{df-an}}\label{df-an}

\vskip 0.5ex
\setbox\startprefix=\hbox{\tt \ \ df-an\ \$a\ }
\setbox\contprefix=\hbox{\tt \ \ \ \ \ \ \ \ \ \ \ }
\startm
\m{\vdash}\m{(}\m{(}\m{\varphi}\m{\wedge}\m{\psi}\m{)}\m{\leftrightarrow}\m{\lnot}
\m{(}\m{\varphi}\m{\rightarrow}\m{\lnot}\m{\psi}\m{)}\m{)}
\endm
\begin{mmraw}%
|- ( ( ph \TAND ps ) <-> -. ( ph -> -. ps ) ) \$.
\end{mmraw}

\noindent Define disjunction ({\sc or}) of 3 wffs.%
\index{df-3or@\texttt{df-3or}}\label{df-3or}

\vskip 0.5ex
\setbox\startprefix=\hbox{\tt \ \ df-3or\ \$a\ }
\setbox\contprefix=\hbox{\tt \ \ \ \ \ \ \ \ \ \ \ \ }
\startm
\m{\vdash}\m{(}\m{(}\m{\varphi}\m{\vee}\m{\psi}\m{\vee}\m{\chi}\m{)}\m{
\leftrightarrow}\m{(}\m{(}\m{\varphi}\m{\vee}\m{\psi}\m{)}\m{\vee}\m{\chi}\m{)}
\m{)}
\endm
\begin{mmraw}%
|- ( ( ph \TOR ps \TOR ch ) <-> ( ( ph \TOR ps ) \TOR ch ) ) \$.
\end{mmraw}

\noindent Define conjunction ({\sc and}) of 3 wffs.%
\index{df-3an}\label{df-3an}

\vskip 0.5ex
\setbox\startprefix=\hbox{\tt \ \ df-3an\ \$a\ }
\setbox\contprefix=\hbox{\tt \ \ \ \ \ \ \ \ \ \ \ \ }
\startm
\m{\vdash}\m{(}\m{(}\m{\varphi}\m{\wedge}\m{\psi}\m{\wedge}\m{\chi}\m{)}\m{
\leftrightarrow}\m{(}\m{(}\m{\varphi}\m{\wedge}\m{\psi}\m{)}\m{\wedge}\m{\chi}
\m{)}\m{)}
\endm

\begin{mmraw}%
|- ( ( ph \TAND ps \TAND ch ) <-> ( ( ph \TAND ps ) \TAND ch ) ) \$.
\end{mmraw}

\subsection{Definitions for Predicate Calculus}\label{metadefpred}

The symbols $x$, $y$, and $z$ represent individual variables of predicate
calculus.  In this section, they are not necessarily distinct unless it is
explicitly
mentioned.

\vskip 2ex
\noindent Define existential quantification.
The expression $\exists x \varphi$ means
``there exists an $x$ where $\varphi$ is true.''\index{existential quantifier
($\exists$)}\label{df-ex}

\vskip 0.5ex
\setbox\startprefix=\hbox{\tt \ \ df-ex\ \$a\ }
\setbox\contprefix=\hbox{\tt \ \ \ \ \ \ \ \ \ \ \ }
\startm
\m{\vdash}\m{(}\m{\exists}\m{x}\m{\varphi}\m{\leftrightarrow}\m{\lnot}\m{\forall}
\m{x}\m{\lnot}\m{\varphi}\m{)}
\endm
\begin{mmraw}%
|- ( E. x ph <-> -. A. x -. ph ) \$.
\end{mmraw}

\noindent Define proper substitution.\index{proper
substitution}\index{substitution!proper}\label{df-sb}
In our notation, we use $[ y / x ] \varphi$ to mean ``the wff that
results when $y$ is properly substituted for $x$ in the wff
$\varphi$.''\footnote{
This can also be described
as substituting $x$ with $y$, $y$ properly replaces $x$, or
$x$ is properly replaced by $y$.}
% This is elsb4, though it currently says: ( [ x / y ] z e. y <-> z e. x )
For example,
$[ y / x ] z \in x$ is the same as $z \in y$.
One way to remember this notation is to notice that it looks like division
and recall that $( y / x ) \cdot x $ is $y$ (when $x \neq 0$).
The notation is different from the notation $\varphi ( x | y )$
that is sometimes used, because the latter notation is ambiguous for us:
for example, we don't know whether $\lnot \varphi ( x | y )$ is to be
interpreted as $\lnot ( \varphi ( x | y ) )$ or
$( \lnot \varphi ) ( x | y )$.\footnote{Because of the way
we initially defined wffs, this is the case
with any postfix connective\index{postfix connective} (one occurring after the
symbols being connected) or infix connective\index{infix connective} (one
occurring between the symbols being connected).  Metamath does not have a
built-in notion of operator binding strength that could eliminate the
ambiguity.  The initial parenthesis effectively provides a prefix
connective\index{prefix connective} to eliminate ambiguity.  Some conventions,
such as Polish notation\index{Polish notation} used in the 1930's and 1940's
by Polish logicians, use only prefix connectives and thus allow the total
elimination of parentheses, at the expense of readability.  In Metamath we
could actually redefine all notation to be Polish if we wanted to without
having to change any proofs!}  Other texts often use $\varphi(y)$ to indicate
our $[ y / x ] \varphi$, but this notation is even more ambiguous since there is
no explicit indication of what is being substituted.
Note that this
definition is valid even when
$x$ and $y$ are the same variable.  The first conjunct is a ``trick'' used to
achieve this property, making the definition look somewhat peculiar at
first.

\vskip 0.5ex
\setbox\startprefix=\hbox{\tt \ \ df-sb\ \$a\ }
\setbox\contprefix=\hbox{\tt \ \ \ \ \ \ \ \ \ \ \ }
\startm
\m{\vdash}\m{(}\m{[}\m{y}\m{/}\m{x}\m{]}\m{\varphi}\m{\leftrightarrow}\m{(}%
\m{(}\m{x}\m{=}\m{y}\m{\rightarrow}\m{\varphi}\m{)}\m{\wedge}\m{\exists}\m{x}%
\m{(}\m{x}\m{=}\m{y}\m{\wedge}\m{\varphi}\m{)}\m{)}\m{)}
\endm
\begin{mmraw}%
|- ( [ y / x ] ph <-> ( ( x = y -> ph ) \TAND E. x ( x = y \TAND ph ) ) ) \$.
\end{mmraw}


\noindent Define existential uniqueness\index{existential uniqueness
quantifier ($\exists "!$)} (``there exists exactly one'').  Note that $y$ is a
variable distinct from $x$ and not occurring in $\varphi$.\label{df-eu}

\vskip 0.5ex
\setbox\startprefix=\hbox{\tt \ \ df-eu\ \$a\ }
\setbox\contprefix=\hbox{\tt \ \ \ \ \ \ \ \ \ \ \ }
\startm
\m{\vdash}\m{(}\m{\exists}\m{{!}}\m{x}\m{\varphi}\m{\leftrightarrow}\m{\exists}
\m{y}\m{\forall}\m{x}\m{(}\m{\varphi}\m{\leftrightarrow}\m{x}\m{=}\m{y}\m{)}\m{)}
\endm

\begin{mmraw}%
|- ( E! x ph <-> E. y A. x ( ph <-> x = y ) ) \$.
\end{mmraw}

\subsection{Definitions for Set Theory}\label{setdefinitions}

The symbols $x$, $y$, $z$, and $w$ represent individual variables of
predicate calculus, which in set theory are understood to be sets.
However, using only the constructs shown so far would be very inconvenient.

To make set theory more practical, we introduce the notion of a ``class.''
A class\index{class} is either a set variable (such as $x$) or an
expression of the form $\{ x | \varphi\}$ (called an ``abstraction
class''\index{abstraction class}\index{class abstraction}).  Note that
sets (i.e.\ individual variables) always exist (this is a theorem of
logic, namely $\exists y \, y = x$ for any set $x$), whereas classes may
or may not exist (i.e.\ $\exists y \, y = A$ may or may not be true).
If a class does not exist it is called a ``proper class.''\index{proper
class}\index{class!proper} Definitions \texttt{df-clab},
\texttt{df-cleq}, and \texttt{df-clel} can be used to convert an
expression containing classes into one containing only set variables and
wff metavariables.

The symbols $A$, $B$, $C$, $D$, $ F$, $G$, and $R$ are metavariables that range
over classes.  A class metavariable $A$ may be eliminated from a wff by
replacing it with $\{ x|\varphi\}$ where neither $x$ nor $\varphi$ occur in
the wff.

The theory of classes can be shown to be an eliminable and conservative
extension of set theory. The \textbf{eliminability}
property shows that for every
formula in the extended language we can build a logically equivalent
formula in the basic language; so that even if the extended language
provides more ease to convey and formulate mathematical ideas for set
theory, its expressive power does not in fact strengthen the basic
language's expressive power.
The \textbf{conservation} property shows that for
every proof of a formula of the basic language in the extended system
we can build another proof of the same formula in the basic system;
so that, concerning theorems on sets only, the deductive powers of
the extended system and of the basic system are identical. Together,
these properties mean that the extended language can be treated as a
definitional extension that is \textbf{sound}.

A rigorous justification, which we will not give here, can be found in
Levy \cite[pp.~357-366]{Levy} supplementing his informal introduction to class
theory on pp.~7-17. Two other good treatments of class theory are provided
by Quine \cite[pp.~15-21]{Quine}\index{Quine, Willard Van Orman}
and also \cite[pp.~10-14]{Takeuti}\index{Takeuti, Gaisi}.
Quine's exposition (he calls them virtual classes)
is nicely written and very readable.

In the rest of this
section, individual variables are always assumed to be distinct from
each other unless otherwise indicated.  In addition, dummy variables on the
right-hand side of a definition do not occur in the class and wff
metavariables in the definition.

The definitions we present here are a partial but self-contained
collection selected from several hundred that appear in the current
\texttt{set.mm} database.  They are adequate for a basic development of
elementary set theory.

\vskip 2ex
\noindent Define the abstraction class.\index{abstraction class}\index{class
abstraction}\label{df-clab}  $x$ and $y$
need not be distinct.  Definition 2.1 of Quine, p.~16.  This definition may
seem puzzling since it is shorter than the expression being defined and does not
buy us anything in terms of brevity.  The reason we introduce this definition
is because it fits in neatly with the extension of the $\in$ connective
provided by \texttt{df-clel}.

\vskip 0.5ex
\setbox\startprefix=\hbox{\tt \ \ df-clab\ \$a\ }
\setbox\contprefix=\hbox{\tt \ \ \ \ \ \ \ \ \ \ \ \ \ }
\startm
\m{\vdash}\m{(}\m{x}\m{\in}\m{\{}\m{y}\m{|}\m{\varphi}\m{\}}\m{%
\leftrightarrow}\m{[}\m{x}\m{/}\m{y}\m{]}\m{\varphi}\m{)}
\endm
\begin{mmraw}%
|- ( x e. \{ y | ph \} <-> [ x / y ] ph ) \$.
\end{mmraw}

\noindent Define the equality connective between classes\index{class
equality}\label{df-cleq}.  See Quine or Chapter 4 of Takeuti and Zaring for its
justification and methods for eliminating it.  This is an example of a
somewhat ``dangerous'' definition, because it extends the use of the
existing equality symbol rather than introducing a new symbol, allowing
us to make statements in the original language that may not be true.
For example, it permits us to deduce $y = z \leftrightarrow \forall x (
x \in y \leftrightarrow x \in z )$ which is not a theorem of logic but
rather presupposes the Axiom of Extensionality,\index{Axiom of
Extensionality} which we include as a hypothesis so that we can know
when this axiom is assumed in a proof (with the \texttt{show
trace{\char`\_}back} command).  We could avoid the danger by introducing
another symbol, say $\eqcirc$, in place of $=$; this
would also have the advantage of making elimination of the definition
straightforward and would eliminate the need for Extensionality as a
hypothesis.  We would then also have the advantage of being able to
identify exactly where Extensionality truly comes into play.  One of our
theorems would be $x \eqcirc y \leftrightarrow x = y$
by invoking Extensionality.  However in keeping with standard practice
we retain the ``dangerous'' definition.

\vskip 0.5ex
\setbox\startprefix=\hbox{\tt \ \ df-cleq.1\ \$e\ }
\setbox\contprefix=\hbox{\tt \ \ \ \ \ \ \ \ \ \ \ \ \ \ \ }
\startm
\m{\vdash}\m{(}\m{\forall}\m{x}\m{(}\m{x}\m{\in}\m{y}\m{\leftrightarrow}\m{x}
\m{\in}\m{z}\m{)}\m{\rightarrow}\m{y}\m{=}\m{z}\m{)}
\endm
\setbox\startprefix=\hbox{\tt \ \ df-cleq\ \$a\ }
\setbox\contprefix=\hbox{\tt \ \ \ \ \ \ \ \ \ \ \ \ \ }
\startm
\m{\vdash}\m{(}\m{A}\m{=}\m{B}\m{\leftrightarrow}\m{\forall}\m{x}\m{(}\m{x}\m{
\in}\m{A}\m{\leftrightarrow}\m{x}\m{\in}\m{B}\m{)}\m{)}
\endm
% We need to reset the startprefix and contprefix.
\setbox\startprefix=\hbox{\tt \ \ df-cleq.1\ \$e\ }
\setbox\contprefix=\hbox{\tt \ \ \ \ \ \ \ \ \ \ \ \ \ \ \ }
\begin{mmraw}%
|- ( A. x ( x e. y <-> x e. z ) -> y = z ) \$.
\end{mmraw}
\setbox\startprefix=\hbox{\tt \ \ df-cleq\ \$a\ }
\setbox\contprefix=\hbox{\tt \ \ \ \ \ \ \ \ \ \ \ \ \ }
\begin{mmraw}%
|- ( A = B <-> A. x ( x e. A <-> x e. B ) ) \$.
\end{mmraw}

\noindent Define the membership connective between classes\index{class
membership}.  Theorem 6.3 of Quine, p.~41, which we adopt as a definition.
Note that it extends the use of the existing membership symbol, but unlike
{\tt df-cleq} it does not extend the set of valid wffs of logic when the class
metavariables are replaced with set variables.\label{dfclel}\label{df-clel}

\vskip 0.5ex
\setbox\startprefix=\hbox{\tt \ \ df-clel\ \$a\ }
\setbox\contprefix=\hbox{\tt \ \ \ \ \ \ \ \ \ \ \ \ \ }
\startm
\m{\vdash}\m{(}\m{A}\m{\in}\m{B}\m{\leftrightarrow}\m{\exists}\m{x}\m{(}\m{x}
\m{=}\m{A}\m{\wedge}\m{x}\m{\in}\m{B}\m{)}\m{)}
\endm
\begin{mmraw}%
|- ( A e. B <-> E. x ( x = A \TAND x e. B ) ) \$.?
\end{mmraw}

\noindent Define inequality.

\vskip 0.5ex
\setbox\startprefix=\hbox{\tt \ \ df-ne\ \$a\ }
\setbox\contprefix=\hbox{\tt \ \ \ \ \ \ \ \ \ \ \ }
\startm
\m{\vdash}\m{(}\m{A}\m{\ne}\m{B}\m{\leftrightarrow}\m{\lnot}\m{A}\m{=}\m{B}%
\m{)}
\endm
\begin{mmraw}%
|- ( A =/= B <-> -. A = B ) \$.
\end{mmraw}

\noindent Define restricted universal quantification.\index{universal
quantifier ($\forall$)!restricted}  Enderton, p.~22.

\vskip 0.5ex
\setbox\startprefix=\hbox{\tt \ \ df-ral\ \$a\ }
\setbox\contprefix=\hbox{\tt \ \ \ \ \ \ \ \ \ \ \ \ }
\startm
\m{\vdash}\m{(}\m{\forall}\m{x}\m{\in}\m{A}\m{\varphi}\m{\leftrightarrow}\m{%
\forall}\m{x}\m{(}\m{x}\m{\in}\m{A}\m{\rightarrow}\m{\varphi}\m{)}\m{)}
\endm
\begin{mmraw}%
|- ( A. x e. A ph <-> A. x ( x e. A -> ph ) ) \$.
\end{mmraw}

\noindent Define restricted existential quantification.\index{existential
quantifier ($\exists$)!restricted}  Enderton, p.~22.

\vskip 0.5ex
\setbox\startprefix=\hbox{\tt \ \ df-rex\ \$a\ }
\setbox\contprefix=\hbox{\tt \ \ \ \ \ \ \ \ \ \ \ \ }
\startm
\m{\vdash}\m{(}\m{\exists}\m{x}\m{\in}\m{A}\m{\varphi}\m{\leftrightarrow}\m{%
\exists}\m{x}\m{(}\m{x}\m{\in}\m{A}\m{\wedge}\m{\varphi}\m{)}\m{)}
\endm
\begin{mmraw}%
|- ( E. x e. A ph <-> E. x ( x e. A \TAND ph ) ) \$.
\end{mmraw}

\noindent Define the universal class\index{universal class ($V$)}.  Definition
5.20, p.~21, of Takeuti and Zaring.\label{df-v}

\vskip 0.5ex
\setbox\startprefix=\hbox{\tt \ \ df-v\ \$a\ }
\setbox\contprefix=\hbox{\tt \ \ \ \ \ \ \ \ \ \ }
\startm
\m{\vdash}\m{{\rm V}}\m{=}\m{\{}\m{x}\m{|}\m{x}\m{=}\m{x}\m{\}}
\endm
\begin{mmraw}%
|- {\char`\_}V = \{ x | x = x \} \$.
\end{mmraw}

\noindent Define the subclass\index{subclass}\index{subset} relationship
between two classes (called the subset relation if the classes are sets i.e.\
are not proper).  Definition 5.9 of Takeuti and Zaring, p.~17.\label{df-ss}

\vskip 0.5ex
\setbox\startprefix=\hbox{\tt \ \ df-ss\ \$a\ }
\setbox\contprefix=\hbox{\tt \ \ \ \ \ \ \ \ \ \ \ }
\startm
\m{\vdash}\m{(}\m{A}\m{\subseteq}\m{B}\m{\leftrightarrow}\m{\forall}\m{x}\m{(}
\m{x}\m{\in}\m{A}\m{\rightarrow}\m{x}\m{\in}\m{B}\m{)}\m{)}
\endm
\begin{mmraw}%
|- ( A C\_ B <-> A. x ( x e. A -> x e. B ) ) \$.
\end{mmraw}

\noindent Define the union\index{union} of two classes.  Definition 5.6 of Takeuti and Zaring,
p.~16.\label{df-un}

\vskip 0.5ex
\setbox\startprefix=\hbox{\tt \ \ df-un\ \$a\ }
\setbox\contprefix=\hbox{\tt \ \ \ \ \ \ \ \ \ \ \ }
\startm
\m{\vdash}\m{(}\m{A}\m{\cup}\m{B}\m{)}\m{=}\m{\{}\m{x}\m{|}\m{(}\m{x}\m{\in}
\m{A}\m{\vee}\m{x}\m{\in}\m{B}\m{)}\m{\}}
\endm
\begin{mmraw}%
( A u. B ) = \{ x | ( x e. A \TOR x e. B ) \} \$.
\end{mmraw}

\noindent Define the intersection\index{intersection} of two classes.  Definition 5.6 of
Takeuti and Zaring, p.~16.\label{df-in}

\vskip 0.5ex
\setbox\startprefix=\hbox{\tt \ \ df-in\ \$a\ }
\setbox\contprefix=\hbox{\tt \ \ \ \ \ \ \ \ \ \ \ }
\startm
\m{\vdash}\m{(}\m{A}\m{\cap}\m{B}\m{)}\m{=}\m{\{}\m{x}\m{|}\m{(}\m{x}\m{\in}
\m{A}\m{\wedge}\m{x}\m{\in}\m{B}\m{)}\m{\}}
\endm
% Caret ^ requires special treatment
\begin{mmraw}%
|- ( A i\^{}i B ) = \{ x | ( x e. A \TAND x e. B ) \} \$.
\end{mmraw}

\noindent Define class difference\index{class difference}\index{set difference}.
Definition 5.12 of Takeuti and Zaring, p.~20.  Several notations are used in
the literature; we chose the $\setminus$ convention instead of a minus sign to
reserve the latter for later use in, e.g., arithmetic.\label{df-dif}

\vskip 0.5ex
\setbox\startprefix=\hbox{\tt \ \ df-dif\ \$a\ }
\setbox\contprefix=\hbox{\tt \ \ \ \ \ \ \ \ \ \ \ \ }
\startm
\m{\vdash}\m{(}\m{A}\m{\setminus}\m{B}\m{)}\m{=}\m{\{}\m{x}\m{|}\m{(}\m{x}\m{
\in}\m{A}\m{\wedge}\m{\lnot}\m{x}\m{\in}\m{B}\m{)}\m{\}}
\endm
\begin{mmraw}%
( A \SLASH B ) = \{ x | ( x e. A \TAND -. x e. B ) \} \$.
\end{mmraw}

\noindent Define the empty or null set\index{empty set}\index{null set}.
Compare  Definition 5.14 of Takeuti and Zaring, p.~20.\label{df-nul}

\vskip 0.5ex
\setbox\startprefix=\hbox{\tt \ \ df-nul\ \$a\ }
\setbox\contprefix=\hbox{\tt \ \ \ \ \ \ \ \ \ \ }
\startm
\m{\vdash}\m{\varnothing}\m{=}\m{(}\m{{\rm V}}\m{\setminus}\m{{\rm V}}\m{)}
\endm
\begin{mmraw}%
|- (/) = ( {\char`\_}V \SLASH {\char`\_}V ) \$.
\end{mmraw}

\noindent Define power class\index{power set}\index{power class}.  Definition 5.10 of
Takeuti and Zaring, p.~17, but we also let it apply to proper classes.  (Note
that \verb$~P$ is the symbol for calligraphic P, the tilde
suggesting ``curly;'' see Appendix~\ref{ASCII}.)\label{df-pw}

\vskip 0.5ex
\setbox\startprefix=\hbox{\tt \ \ df-pw\ \$a\ }
\setbox\contprefix=\hbox{\tt \ \ \ \ \ \ \ \ \ \ \ }
\startm
\m{\vdash}\m{{\cal P}}\m{A}\m{=}\m{\{}\m{x}\m{|}\m{x}\m{\subseteq}\m{A}\m{\}}
\endm
% Special incantation required to put ~ into the text
\begin{mmraw}%
|- \char`\~P~A = \{ x | x C\_ A \} \$.
\end{mmraw}

\noindent Define the singleton of a class\index{singleton}.  Definition 7.1 of
Quine, p.~48.  It is well-defined for proper classes, although
it is not very meaningful in this case, where it evaluates to the empty
set.

\vskip 0.5ex
\setbox\startprefix=\hbox{\tt \ \ df-sn\ \$a\ }
\setbox\contprefix=\hbox{\tt \ \ \ \ \ \ \ \ \ \ \ }
\startm
\m{\vdash}\m{\{}\m{A}\m{\}}\m{=}\m{\{}\m{x}\m{|}\m{x}\m{=}\m{A}\m{\}}
\endm
\begin{mmraw}%
|- \{ A \} = \{ x | x = A \} \$.
\end{mmraw}%

\noindent Define an unordered pair of classes\index{unordered pair}\index{pair}.  Definition
7.1 of Quine, p.~48.

\vskip 0.5ex
\setbox\startprefix=\hbox{\tt \ \ df-pr\ \$a\ }
\setbox\contprefix=\hbox{\tt \ \ \ \ \ \ \ \ \ \ \ }
\startm
\m{\vdash}\m{\{}\m{A}\m{,}\m{B}\m{\}}\m{=}\m{(}\m{\{}\m{A}\m{\}}\m{\cup}\m{\{}
\m{B}\m{\}}\m{)}
\endm
\begin{mmraw}%
|- \{ A , B \} = ( \{ A \} u. \{ B \} ) \$.
\end{mmraw}

\noindent Define an unordered triple of classes\index{unordered triple}.  Definition of
Enderton, p.~19.

\vskip 0.5ex
\setbox\startprefix=\hbox{\tt \ \ df-tp\ \$a\ }
\setbox\contprefix=\hbox{\tt \ \ \ \ \ \ \ \ \ \ \ }
\startm
\m{\vdash}\m{\{}\m{A}\m{,}\m{B}\m{,}\m{C}\m{\}}\m{=}\m{(}\m{\{}\m{A}\m{,}\m{B}
\m{\}}\m{\cup}\m{\{}\m{C}\m{\}}\m{)}
\endm
\begin{mmraw}%
|- \{ A , B , C \} = ( \{ A , B \} u. \{ C \} ) \$.
\end{mmraw}%

\noindent Kuratowski's\index{Kuratowski, Kazimierz} ordered pair\index{ordered
pair} definition.  Definition 9.1 of Quine, p.~58. For proper classes it is
not meaningful but is well-defined for convenience.  (Note that \verb$<.$
stands for $\langle$ whereas \verb$<$ stands for $<$, and similarly for
\verb$>.$\,.)\label{df-op}

\vskip 0.5ex
\setbox\startprefix=\hbox{\tt \ \ df-op\ \$a\ }
\setbox\contprefix=\hbox{\tt \ \ \ \ \ \ \ \ \ \ \ }
\startm
\m{\vdash}\m{\langle}\m{A}\m{,}\m{B}\m{\rangle}\m{=}\m{\{}\m{\{}\m{A}\m{\}}
\m{,}\m{\{}\m{A}\m{,}\m{B}\m{\}}\m{\}}
\endm
\begin{mmraw}%
|- <. A , B >. = \{ \{ A \} , \{ A , B \} \} \$.
\end{mmraw}

\noindent Define the union of a class\index{union}.  Definition 5.5, p.~16,
of Takeuti and Zaring.

\vskip 0.5ex
\setbox\startprefix=\hbox{\tt \ \ df-uni\ \$a\ }
\setbox\contprefix=\hbox{\tt \ \ \ \ \ \ \ \ \ \ \ \ }
\startm
\m{\vdash}\m{\bigcup}\m{A}\m{=}\m{\{}\m{x}\m{|}\m{\exists}\m{y}\m{(}\m{x}\m{
\in}\m{y}\m{\wedge}\m{y}\m{\in}\m{A}\m{)}\m{\}}
\endm
\begin{mmraw}%
|- U. A = \{ x | E. y ( x e. y \TAND y e. A ) \} \$.
\end{mmraw}

\noindent Define the intersection\index{intersection} of a class.  Definition 7.35,
p.~44, of Takeuti and Zaring.

\vskip 0.5ex
\setbox\startprefix=\hbox{\tt \ \ df-int\ \$a\ }
\setbox\contprefix=\hbox{\tt \ \ \ \ \ \ \ \ \ \ \ \ }
\startm
\m{\vdash}\m{\bigcap}\m{A}\m{=}\m{\{}\m{x}\m{|}\m{\forall}\m{y}\m{(}\m{y}\m{
\in}\m{A}\m{\rightarrow}\m{x}\m{\in}\m{y}\m{)}\m{\}}
\endm
\begin{mmraw}%
|- |\^{}| A = \{ x | A. y ( y e. A -> x e. y ) \} \$.
\end{mmraw}

\noindent Define a transitive class\index{transitive class}\index{transitive
set}.  This should not be confused with a transitive relation, which is a different
concept.  Definition from p.~71 of Enderton, extended to classes.

\vskip 0.5ex
\setbox\startprefix=\hbox{\tt \ \ df-tr\ \$a\ }
\setbox\contprefix=\hbox{\tt \ \ \ \ \ \ \ \ \ \ \ }
\startm
\m{\vdash}\m{(}\m{\mbox{\rm Tr}}\m{A}\m{\leftrightarrow}\m{\bigcup}\m{A}\m{
\subseteq}\m{A}\m{)}
\endm
\begin{mmraw}%
|- ( Tr A <-> U. A C\_ A ) \$.
\end{mmraw}

\noindent Define a notation for a general binary relation\index{binary
relation}.  Definition 6.18, p.~29, of Takeuti and Zaring, generalized to
arbitrary classes.  This definition is well-defined, although not very
meaningful, when classes $A$ and/or $B$ are proper.\label{dfbr}  The lack of
parentheses (or any other connective) creates no ambiguity since we are defining
an atomic wff.

\vskip 0.5ex
\setbox\startprefix=\hbox{\tt \ \ df-br\ \$a\ }
\setbox\contprefix=\hbox{\tt \ \ \ \ \ \ \ \ \ \ \ }
\startm
\m{\vdash}\m{(}\m{A}\m{\,R}\m{\,B}\m{\leftrightarrow}\m{\langle}\m{A}\m{,}\m{B}
\m{\rangle}\m{\in}\m{R}\m{)}
\endm
\begin{mmraw}%
|- ( A R B <-> <. A , B >. e. R ) \$.
\end{mmraw}

\noindent Define an abstraction class of ordered pairs\index{abstraction
class!of ordered
pairs}.  A special case of Definition 4.16, p.~14, of Takeuti and Zaring.
Note that $ z $ must be distinct from $ x $ and $ y $,
and $ z $ must not occur in $\varphi$, but $ x $ and $ y $ may be
identical and may appear in $\varphi$.

\vskip 0.5ex
\setbox\startprefix=\hbox{\tt \ \ df-opab\ \$a\ }
\setbox\contprefix=\hbox{\tt \ \ \ \ \ \ \ \ \ \ \ \ \ }
\startm
\m{\vdash}\m{\{}\m{\langle}\m{x}\m{,}\m{y}\m{\rangle}\m{|}\m{\varphi}\m{\}}\m{=}
\m{\{}\m{z}\m{|}\m{\exists}\m{x}\m{\exists}\m{y}\m{(}\m{z}\m{=}\m{\langle}\m{x}
\m{,}\m{y}\m{\rangle}\m{\wedge}\m{\varphi}\m{)}\m{\}}
\endm

\begin{mmraw}%
|- \{ <. x , y >. | ph \} = \{ z | E. x E. y ( z =
<. x , y >. /\ ph ) \} \$.
\end{mmraw}

\noindent Define the epsilon relation\index{epsilon relation}.  Similar to Definition
6.22, p.~30, of Takeuti and Zaring.

\vskip 0.5ex
\setbox\startprefix=\hbox{\tt \ \ df-eprel\ \$a\ }
\setbox\contprefix=\hbox{\tt \ \ \ \ \ \ \ \ \ \ \ \ \ \ }
\startm
\m{\vdash}\m{{\rm E}}\m{=}\m{\{}\m{\langle}\m{x}\m{,}\m{y}\m{\rangle}\m{|}\m{x}\m{
\in}\m{y}\m{\}}
\endm
\begin{mmraw}%
|- \_E = \{ <. x , y >. | x e. y \} \$.
\end{mmraw}

\noindent Define a founded relation\index{founded relation}.  $R$ is a founded
relation on $A$ iff\index{iff} (if and only if) each nonempty subset of $A$
has an ``$R$-minimal element.''  Similar to Definition 6.21, p.~30, of
Takeuti and Zaring.

\vskip 0.5ex
\setbox\startprefix=\hbox{\tt \ \ df-fr\ \$a\ }
\setbox\contprefix=\hbox{\tt \ \ \ \ \ \ \ \ \ \ \ }
\startm
\m{\vdash}\m{(}\m{R}\m{\,\mbox{\rm Fr}}\m{\,A}\m{\leftrightarrow}\m{\forall}\m{x}
\m{(}\m{(}\m{x}\m{\subseteq}\m{A}\m{\wedge}\m{\lnot}\m{x}\m{=}\m{\varnothing}
\m{)}\m{\rightarrow}\m{\exists}\m{y}\m{(}\m{y}\m{\in}\m{x}\m{\wedge}\m{(}\m{x}
\m{\cap}\m{\{}\m{z}\m{|}\m{z}\m{\,R}\m{\,y}\m{\}}\m{)}\m{=}\m{\varnothing}\m{)}
\m{)}\m{)}
\endm
\begin{mmraw}%
|- ( R Fr A <-> A. x ( ( x C\_ A \TAND -. x = (/) ) ->
E. y ( y e. x \TAND ( x i\^{}i \{ z | z R y \} ) = (/) ) ) ) \$.
\end{mmraw}

\noindent Define a well-ordering\index{well-ordering}.  $R$ is a well-ordering of $A$ iff
it is founded on $A$ and the elements of $A$ are pairwise $R$-comparable.
Similar to Definition 6.24(2), p.~30, of Takeuti and Zaring.

\vskip 0.5ex
\setbox\startprefix=\hbox{\tt \ \ df-we\ \$a\ }
\setbox\contprefix=\hbox{\tt \ \ \ \ \ \ \ \ \ \ \ }
\startm
\m{\vdash}\m{(}\m{R}\m{\,\mbox{\rm We}}\m{\,A}\m{\leftrightarrow}\m{(}\m{R}\m{\,
\mbox{\rm Fr}}\m{\,A}\m{\wedge}\m{\forall}\m{x}\m{\forall}\m{y}\m{(}\m{(}\m{x}\m{
\in}\m{A}\m{\wedge}\m{y}\m{\in}\m{A}\m{)}\m{\rightarrow}\m{(}\m{x}\m{\,R}\m{\,y}
\m{\vee}\m{x}\m{=}\m{y}\m{\vee}\m{y}\m{\,R}\m{\,x}\m{)}\m{)}\m{)}\m{)}
\endm
\begin{mmraw}%
( R We A <-> ( R Fr A \TAND A. x A. y ( ( x e.
A \TAND y e. A ) -> ( x R y \TOR x = y \TOR y R x ) ) ) ) \$.
\end{mmraw}

\noindent Define the ordinal predicate\index{ordinal predicate}, which is true for a class
that is transitive and is well-ordered by the epsilon relation.  Similar to
definition on p.~468, Bell and Machover.

\vskip 0.5ex
\setbox\startprefix=\hbox{\tt \ \ df-ord\ \$a\ }
\setbox\contprefix=\hbox{\tt \ \ \ \ \ \ \ \ \ \ \ \ }
\startm
\m{\vdash}\m{(}\m{\mbox{\rm Ord}}\m{\,A}\m{\leftrightarrow}\m{(}
\m{\mbox{\rm Tr}}\m{\,A}\m{\wedge}\m{E}\m{\,\mbox{\rm We}}\m{\,A}\m{)}\m{)}
\endm
\begin{mmraw}%
|- ( Ord A <-> ( Tr A \TAND E We A ) ) \$.
\end{mmraw}

\noindent Define the class of all ordinal numbers\index{ordinal number}.  An ordinal number is
a set that satisfies the ordinal predicate.  Definition 7.11 of Takeuti and
Zaring, p.~38.

\vskip 0.5ex
\setbox\startprefix=\hbox{\tt \ \ df-on\ \$a\ }
\setbox\contprefix=\hbox{\tt \ \ \ \ \ \ \ \ \ \ \ }
\startm
\m{\vdash}\m{\,\mbox{\rm On}}\m{=}\m{\{}\m{x}\m{|}\m{\mbox{\rm Ord}}\m{\,x}
\m{\}}
\endm
\begin{mmraw}%
|- On = \{ x | Ord x \} \$.
\end{mmraw}

\noindent Define the limit ordinal predicate\index{limit ordinal}, which is true for a
non-empty ordinal that is not a successor (i.e.\ that is the union of itself).
Compare Bell and Machover, p.~471 and Exercise (1), p.~42 of Takeuti and
Zaring.

\vskip 0.5ex
\setbox\startprefix=\hbox{\tt \ \ df-lim\ \$a\ }
\setbox\contprefix=\hbox{\tt \ \ \ \ \ \ \ \ \ \ \ \ }
\startm
\m{\vdash}\m{(}\m{\mbox{\rm Lim}}\m{\,A}\m{\leftrightarrow}\m{(}\m{\mbox{
\rm Ord}}\m{\,A}\m{\wedge}\m{\lnot}\m{A}\m{=}\m{\varnothing}\m{\wedge}\m{A}
\m{=}\m{\bigcup}\m{A}\m{)}\m{)}
\endm
\begin{mmraw}%
|- ( Lim A <-> ( Ord A \TAND -. A = (/) \TAND A = U. A ) ) \$.
\end{mmraw}

\noindent Define the successor\index{successor} of a class.  Definition 7.22 of Takeuti
and Zaring, p.~41.  Our definition is a generalization to classes, although it
is meaningless when classes are proper.

\vskip 0.5ex
\setbox\startprefix=\hbox{\tt \ \ df-suc\ \$a\ }
\setbox\contprefix=\hbox{\tt \ \ \ \ \ \ \ \ \ \ \ \ }
\startm
\m{\vdash}\m{\,\mbox{\rm suc}}\m{\,A}\m{=}\m{(}\m{A}\m{\cup}\m{\{}\m{A}\m{\}}
\m{)}
\endm
\begin{mmraw}%
|- suc A = ( A u. \{ A \} ) \$.
\end{mmraw}

\noindent Define the class of natural numbers\index{natural number}\index{omega
($\omega$)}.  Compare Bell and Machover, p.~471.\label{dfom}

\vskip 0.5ex
\setbox\startprefix=\hbox{\tt \ \ df-om\ \$a\ }
\setbox\contprefix=\hbox{\tt \ \ \ \ \ \ \ \ \ \ \ }
\startm
\m{\vdash}\m{\omega}\m{=}\m{\{}\m{x}\m{|}\m{(}\m{\mbox{\rm Ord}}\m{\,x}\m{
\wedge}\m{\forall}\m{y}\m{(}\m{\mbox{\rm Lim}}\m{\,y}\m{\rightarrow}\m{x}\m{
\in}\m{y}\m{)}\m{)}\m{\}}
\endm
\begin{mmraw}%
|- om = \{ x | ( Ord x \TAND A. y ( Lim y -> x e. y ) ) \} \$.
\end{mmraw}

\noindent Define the Cartesian product (also called the
cross product)\index{Cartesian product}\index{cross product}
of two classes.  Definition 9.11 of Quine, p.~64.

\vskip 0.5ex
\setbox\startprefix=\hbox{\tt \ \ df-xp\ \$a\ }
\setbox\contprefix=\hbox{\tt \ \ \ \ \ \ \ \ \ \ \ }
\startm
\m{\vdash}\m{(}\m{A}\m{\times}\m{B}\m{)}\m{=}\m{\{}\m{\langle}\m{x}\m{,}\m{y}
\m{\rangle}\m{|}\m{(}\m{x}\m{\in}\m{A}\m{\wedge}\m{y}\m{\in}\m{B}\m{)}\m{\}}
\endm
\begin{mmraw}%
|- ( A X. B ) = \{ <. x , y >. | ( x e. A \TAND y e. B) \} \$.
\end{mmraw}

\noindent Define a relation\index{relation}.  Definition 6.4(1) of Takeuti and
Zaring, p.~23.

\vskip 0.5ex
\setbox\startprefix=\hbox{\tt \ \ df-rel\ \$a\ }
\setbox\contprefix=\hbox{\tt \ \ \ \ \ \ \ \ \ \ \ \ }
\startm
\m{\vdash}\m{(}\m{\mbox{\rm Rel}}\m{\,A}\m{\leftrightarrow}\m{A}\m{\subseteq}
\m{(}\m{{\rm V}}\m{\times}\m{{\rm V}}\m{)}\m{)}
\endm
\begin{mmraw}%
|- ( Rel A <-> A C\_ ( {\char`\_}V X. {\char`\_}V ) ) \$.
\end{mmraw}

\noindent Define the domain\index{domain} of a class.  Definition 6.5(1) of
Takeuti and Zaring, p.~24.

\vskip 0.5ex
\setbox\startprefix=\hbox{\tt \ \ df-dm\ \$a\ }
\setbox\contprefix=\hbox{\tt \ \ \ \ \ \ \ \ \ \ \ }
\startm
\m{\vdash}\m{\,\mbox{\rm dom}}\m{A}\m{=}\m{\{}\m{x}\m{|}\m{\exists}\m{y}\m{
\langle}\m{x}\m{,}\m{y}\m{\rangle}\m{\in}\m{A}\m{\}}
\endm
\begin{mmraw}%
|- dom A = \{ x | E. y <. x , y >. e. A \} \$.
\end{mmraw}

\noindent Define the range\index{range} of a class.  Definition 6.5(2) of
Takeuti and Zaring, p.~24.

\vskip 0.5ex
\setbox\startprefix=\hbox{\tt \ \ df-rn\ \$a\ }
\setbox\contprefix=\hbox{\tt \ \ \ \ \ \ \ \ \ \ \ }
\startm
\m{\vdash}\m{\,\mbox{\rm ran}}\m{A}\m{=}\m{\{}\m{y}\m{|}\m{\exists}\m{x}\m{
\langle}\m{x}\m{,}\m{y}\m{\rangle}\m{\in}\m{A}\m{\}}
\endm
\begin{mmraw}%
|- ran A = \{ y | E. x <. x , y >. e. A \} \$.
\end{mmraw}

\noindent Define the restriction\index{restriction} of a class.  Definition
6.6(1) of Takeuti and Zaring, p.~24.

\vskip 0.5ex
\setbox\startprefix=\hbox{\tt \ \ df-res\ \$a\ }
\setbox\contprefix=\hbox{\tt \ \ \ \ \ \ \ \ \ \ \ \ }
\startm
\m{\vdash}\m{(}\m{A}\m{\restriction}\m{B}\m{)}\m{=}\m{(}\m{A}\m{\cap}\m{(}\m{B}
\m{\times}\m{{\rm V}}\m{)}\m{)}
\endm
\begin{mmraw}%
|- ( A |` B ) = ( A i\^{}i ( B X. {\char`\_}V ) ) \$.
\end{mmraw}

\noindent Define the image\index{image} of a class.  Definition 6.6(2) of
Takeuti and Zaring, p.~24.

\vskip 0.5ex
\setbox\startprefix=\hbox{\tt \ \ df-ima\ \$a\ }
\setbox\contprefix=\hbox{\tt \ \ \ \ \ \ \ \ \ \ \ \ }
\startm
\m{\vdash}\m{(}\m{A}\m{``}\m{B}\m{)}\m{=}\m{\,\mbox{\rm ran}}\m{\,(}\m{A}\m{
\restriction}\m{B}\m{)}
\endm
\begin{mmraw}%
|- ( A " B ) = ran ( A |` B ) \$.
\end{mmraw}

\noindent Define the composition\index{composition} of two classes.  Definition 6.6(3) of
Takeuti and Zaring, p.~24.

\vskip 0.5ex
\setbox\startprefix=\hbox{\tt \ \ df-co\ \$a\ }
\setbox\contprefix=\hbox{\tt \ \ \ \ \ \ \ \ \ \ \ \ }
\startm
\m{\vdash}\m{(}\m{A}\m{\circ}\m{B}\m{)}\m{=}\m{\{}\m{\langle}\m{x}\m{,}\m{y}\m{
\rangle}\m{|}\m{\exists}\m{z}\m{(}\m{\langle}\m{x}\m{,}\m{z}\m{\rangle}\m{\in}
\m{B}\m{\wedge}\m{\langle}\m{z}\m{,}\m{y}\m{\rangle}\m{\in}\m{A}\m{)}\m{\}}
\endm
\begin{mmraw}%
|- ( A o. B ) = \{ <. x , y >. | E. z ( <. x , z
>. e. B \TAND <. z , y >. e. A ) \} \$.
\end{mmraw}

\noindent Define a function\index{function}.  Definition 6.4(4) of Takeuti and
Zaring, p.~24.

\vskip 0.5ex
\setbox\startprefix=\hbox{\tt \ \ df-fun\ \$a\ }
\setbox\contprefix=\hbox{\tt \ \ \ \ \ \ \ \ \ \ \ \ }
\startm
\m{\vdash}\m{(}\m{\mbox{\rm Fun}}\m{\,A}\m{\leftrightarrow}\m{(}
\m{\mbox{\rm Rel}}\m{\,A}\m{\wedge}
\m{\forall}\m{x}\m{\exists}\m{z}\m{\forall}\m{y}\m{(}
\m{\langle}\m{x}\m{,}\m{y}\m{\rangle}\m{\in}\m{A}\m{\rightarrow}\m{y}\m{=}\m{z}
\m{)}\m{)}\m{)}
\endm
\begin{mmraw}%
|- ( Fun A <-> ( Rel A /\ A. x E. z A. y ( <. x
   , y >. e. A -> y = z ) ) ) \$.
\end{mmraw}

\noindent Define a function with domain.  Definition 6.15(1) of Takeuti and
Zaring, p.~27.

\vskip 0.5ex
\setbox\startprefix=\hbox{\tt \ \ df-fn\ \$a\ }
\setbox\contprefix=\hbox{\tt \ \ \ \ \ \ \ \ \ \ \ }
\startm
\m{\vdash}\m{(}\m{A}\m{\,\mbox{\rm Fn}}\m{\,B}\m{\leftrightarrow}\m{(}
\m{\mbox{\rm Fun}}\m{\,A}\m{\wedge}\m{\mbox{\rm dom}}\m{\,A}\m{=}\m{B}\m{)}
\m{)}
\endm
\begin{mmraw}%
|- ( A Fn B <-> ( Fun A \TAND dom A = B ) ) \$.
\end{mmraw}

\noindent Define a function with domain and co-domain.  Definition 6.15(3)
of Takeuti and Zaring, p.~27.

\vskip 0.5ex
\setbox\startprefix=\hbox{\tt \ \ df-f\ \$a\ }
\setbox\contprefix=\hbox{\tt \ \ \ \ \ \ \ \ \ \ }
\startm
\m{\vdash}\m{(}\m{F}\m{:}\m{A}\m{\longrightarrow}\m{B}\m{
\leftrightarrow}\m{(}\m{F}\m{\,\mbox{\rm Fn}}\m{\,A}\m{\wedge}\m{
\mbox{\rm ran}}\m{\,F}\m{\subseteq}\m{B}\m{)}\m{)}
\endm
\begin{mmraw}%
|- ( F : A --> B <-> ( F Fn A \TAND ran F C\_ B ) ) \$.
\end{mmraw}

\noindent Define a one-to-one function\index{one-to-one function}.  Compare
Definition 6.15(5) of Takeuti and Zaring, p.~27.

\vskip 0.5ex
\setbox\startprefix=\hbox{\tt \ \ df-f1\ \$a\ }
\setbox\contprefix=\hbox{\tt \ \ \ \ \ \ \ \ \ \ \ }
\startm
\m{\vdash}\m{(}\m{F}\m{:}\m{A}\m{
\raisebox{.5ex}{${\textstyle{\:}_{\mbox{\footnotesize\rm
1\tt -\rm 1}}}\atop{\textstyle{
\longrightarrow}\atop{\textstyle{}^{\mbox{\footnotesize\rm {\ }}}}}$}
}\m{B}
\m{\leftrightarrow}\m{(}\m{F}\m{:}\m{A}\m{\longrightarrow}\m{B}
\m{\wedge}\m{\forall}\m{y}\m{\exists}\m{z}\m{\forall}\m{x}\m{(}\m{\langle}\m{x}
\m{,}\m{y}\m{\rangle}\m{\in}\m{F}\m{\rightarrow}\m{x}\m{=}\m{z}\m{)}\m{)}\m{)}
\endm
\begin{mmraw}%
|- ( F : A -1-1-> B <-> ( F : A --> B \TAND
   A. y E. z A. x ( <. x , y >. e. F -> x = z ) ) ) \$.
\end{mmraw}

\noindent Define an onto function\index{onto function}.  Definition 6.15(4) of Takeuti and
Zaring, p.~27.

\vskip 0.5ex
\setbox\startprefix=\hbox{\tt \ \ df-fo\ \$a\ }
\setbox\contprefix=\hbox{\tt \ \ \ \ \ \ \ \ \ \ \ }
\startm
\m{\vdash}\m{(}\m{F}\m{:}\m{A}\m{
\raisebox{.5ex}{${\textstyle{\:}_{\mbox{\footnotesize\rm
{\ }}}}\atop{\textstyle{
\longrightarrow}\atop{\textstyle{}^{\mbox{\footnotesize\rm onto}}}}$}
}\m{B}
\m{\leftrightarrow}\m{(}\m{F}\m{\,\mbox{\rm Fn}}\m{\,A}\m{\wedge}
\m{\mbox{\rm ran}}\m{\,F}\m{=}\m{B}\m{)}\m{)}
\endm
\begin{mmraw}%
|- ( F : A -onto-> B <-> ( F Fn A /\ ran F = B ) ) \$.
\end{mmraw}

\noindent Define a one-to-one, onto function.  Compare Definition 6.15(6) of
Takeuti and Zaring, p.~27.

\vskip 0.5ex
\setbox\startprefix=\hbox{\tt \ \ df-f1o\ \$a\ }
\setbox\contprefix=\hbox{\tt \ \ \ \ \ \ \ \ \ \ \ \ }
\startm
\m{\vdash}\m{(}\m{F}\m{:}\m{A}
\m{
\raisebox{.5ex}{${\textstyle{\:}_{\mbox{\footnotesize\rm
1\tt -\rm 1}}}\atop{\textstyle{
\longrightarrow}\atop{\textstyle{}^{\mbox{\footnotesize\rm onto}}}}$}
}
\m{B}
\m{\leftrightarrow}\m{(}\m{F}\m{:}\m{A}
\m{
\raisebox{.5ex}{${\textstyle{\:}_{\mbox{\footnotesize\rm
1\tt -\rm 1}}}\atop{\textstyle{
\longrightarrow}\atop{\textstyle{}^{\mbox{\footnotesize\rm {\ }}}}}$}
}
\m{B}\m{\wedge}\m{F}\m{:}\m{A}
\m{
\raisebox{.5ex}{${\textstyle{\:}_{\mbox{\footnotesize\rm
{\ }}}}\atop{\textstyle{
\longrightarrow}\atop{\textstyle{}^{\mbox{\footnotesize\rm onto}}}}$}
}
\m{B}\m{)}\m{)}
\endm
\begin{mmraw}%
|- ( F : A -1-1-onto-> B <-> ( F : A -1-1-> B? \TAND F : A -onto-> B ) ) \$.?
\end{mmraw}

\noindent Define the value of a function\index{function value}.  This
definition applies to any class and evaluates to the empty set when it is not
meaningful. Note that $ F`A$ means the same thing as the more familiar $ F(A)$
notation for a function's value at $A$.  The $ F`A$ notation is common in
formal set theory.\label{df-fv}

\vskip 0.5ex
\setbox\startprefix=\hbox{\tt \ \ df-fv\ \$a\ }
\setbox\contprefix=\hbox{\tt \ \ \ \ \ \ \ \ \ \ \ }
\startm
\m{\vdash}\m{(}\m{F}\m{`}\m{A}\m{)}\m{=}\m{\bigcup}\m{\{}\m{x}\m{|}\m{(}\m{F}%
\m{``}\m{\{}\m{A}\m{\}}\m{)}\m{=}\m{\{}\m{x}\m{\}}\m{\}}
\endm
\begin{mmraw}%
|- ( F ` A ) = U. \{ x | ( F " \{ A \} ) = \{ x \} \} \$.
\end{mmraw}

\noindent Define the result of an operation.\index{operation}  Here, $F$ is
     an operation on two
     values (such as $+$ for real numbers).   This is defined for proper
     classes $A$ and $B$ even though not meaningful in that case.  However,
     the definition can be meaningful when $F$ is a proper class.\label{dfopr}

\vskip 0.5ex
\setbox\startprefix=\hbox{\tt \ \ df-opr\ \$a\ }
\setbox\contprefix=\hbox{\tt \ \ \ \ \ \ \ \ \ \ \ \ }
\startm
\m{\vdash}\m{(}\m{A}\m{\,F}\m{\,B}\m{)}\m{=}\m{(}\m{F}\m{`}\m{\langle}\m{A}%
\m{,}\m{B}\m{\rangle}\m{)}
\endm
\begin{mmraw}%
|- ( A F B ) = ( F ` <. A , B >. ) \$.
\end{mmraw}

\section{Tricks of the Trade}\label{tricks}

In the \texttt{set.mm}\index{set theory database (\texttt{set.mm})} database our goal
was usually to conform to modern notation.  However in some cases the
relationship to standard textbook language may be obscured by several
unconventional devices we used to simplify the development and to take
advantage of the Metamath language.  In this section we will describe some
common conventions used in \texttt{set.mm}.

\begin{itemize}
\item
The turnstile\index{turnstile ({$\,\vdash$})} symbol, $\vdash$, meaning ``it
is provable that,'' is the first token of all assertions and hypotheses that
aren't syntax constructions.  This is a standard convention in logic.  (We
mentioned this earlier, but this symbol is bothersome to some people without a
logic background.  It has no deeper meaning but just provides us with a way to
distinguish syntax constructions from ordinary mathematical statements.)

\item
A hypothesis of the form

\vskip 1ex
\setbox\startprefix=\hbox{\tt \ \ \ \ \ \ \ \ \ \$e\ }
\setbox\contprefix=\hbox{\tt \ \ \ \ \ \ \ \ \ \ }
\startm
\m{\vdash}\m{(}\m{\varphi}\m{\rightarrow}\m{\forall}\m{x}\m{\varphi}\m{)}
\endm
\vskip 1ex

should be read ``assume variable $x$ is (effectively) not free in wff
$\varphi$.''\index{effectively not free}
Literally, this says ``assume it is provable that $\varphi \rightarrow \forall
x\, \varphi$.''  This device lets us avoid the complexities associated with
the standard treatment of free and bound variables.
%% Uncomment this when uncommenting section {formalspec} below
The footnote on p.~\pageref{effectivelybound} discusses this further.

\item
A statement of one of the forms

\vskip 1ex
\setbox\startprefix=\hbox{\tt \ \ \ \ \ \ \ \ \ \$a\ }
\setbox\contprefix=\hbox{\tt \ \ \ \ \ \ \ \ \ \ }
\startm
\m{\vdash}\m{(}\m{\lnot}\m{\forall}\m{x}\m{\,x}\m{=}\m{y}\m{\rightarrow}
\m{\ldots}\m{)}
\endm
\setbox\startprefix=\hbox{\tt \ \ \ \ \ \ \ \ \ \$p\ }
\setbox\contprefix=\hbox{\tt \ \ \ \ \ \ \ \ \ \ }
\startm
\m{\vdash}\m{(}\m{\lnot}\m{\forall}\m{x}\m{\,x}\m{=}\m{y}\m{\rightarrow}
\m{\ldots}\m{)}
\endm
\vskip 1ex

should be read ``if $x$ and $y$ are distinct variables, then...''  This
antecedent provides us with a technical device to avoid the need for the
\texttt{\$d} statement early in our development of predicate calculus,
permitting symbol manipulations to be as conceptually simple as those in
propositional calculus.  However, the \texttt{\$d} statement eventually
becomes a requirement, and after that this device is rarely used.

\item
The statement

\vskip 1ex
\setbox\startprefix=\hbox{\tt \ \ \ \ \ \ \ \ \ \$d\ }
\setbox\contprefix=\hbox{\tt \ \ \ \ \ \ \ \ \ \ }
\startm
\m{x}\m{\,y}
\endm
\vskip 1ex

should be read ``assume $x$ and $y$ are distinct variables.''

\item
The statement

\vskip 1ex
\setbox\startprefix=\hbox{\tt \ \ \ \ \ \ \ \ \ \$d\ }
\setbox\contprefix=\hbox{\tt \ \ \ \ \ \ \ \ \ \ }
\startm
\m{x}\m{\,\varphi}
\endm
\vskip 1ex

should be read ``assume $x$ does not occur in $\varphi$.''

\item
The statement

\vskip 1ex
\setbox\startprefix=\hbox{\tt \ \ \ \ \ \ \ \ \ \$d\ }
\setbox\contprefix=\hbox{\tt \ \ \ \ \ \ \ \ \ \ }
\startm
\m{x}\m{\,A}
\endm
\vskip 1ex

should be read ``assume variable $x$ does not occur in class $A$.''

\item
The restriction and hypothesis group

\vskip 1ex
\setbox\startprefix=\hbox{\tt \ \ \ \ \ \ \ \ \ \$d\ }
\setbox\contprefix=\hbox{\tt \ \ \ \ \ \ \ \ \ \ }
\startm
\m{x}\m{\,A}
\endm
\setbox\startprefix=\hbox{\tt \ \ \ \ \ \ \ \ \ \$d\ }
\setbox\contprefix=\hbox{\tt \ \ \ \ \ \ \ \ \ \ }
\startm
\m{x}\m{\,\psi}
\endm
\setbox\startprefix=\hbox{\tt \ \ \ \ \ \ \ \ \ \$e\ }
\setbox\contprefix=\hbox{\tt \ \ \ \ \ \ \ \ \ \ }
\startm
\m{\vdash}\m{(}\m{x}\m{=}\m{A}\m{\rightarrow}\m{(}\m{\varphi}\m{\leftrightarrow}
\m{\psi}\m{)}\m{)}
\endm
\vskip 1ex

is frequently used in place of explicit substitution, meaning ``assume
$\psi$ results from the proper substitution of $A$ for $x$ in
$\varphi$.''  Sometimes ``\texttt{\$e} $\vdash ( \psi \rightarrow
\forall x \, \psi )$'' is used instead of ``\texttt{\$d} $x\, \psi $,''
which requires only that $x$ be effectively not free in $\varphi$ but
not necessarily absent from it.  The use of implicit
substitution\index{substitution!implicit} is partly a matter of personal
style, although it may make proofs somewhat shorter than would be the
case with explicit substitution.

\item
The hypothesis


\vskip 1ex
\setbox\startprefix=\hbox{\tt \ \ \ \ \ \ \ \ \ \$e\ }
\setbox\contprefix=\hbox{\tt \ \ \ \ \ \ \ \ \ \ }
\startm
\m{\vdash}\m{A}\m{\in}\m{{\rm V}}
\endm
\vskip 1ex

should be read ``assume class $A$ is a set (i.e.\ exists).''
This is a convenient convention used by Quine.

\item
The restriction and hypothesis

\vskip 1ex
\setbox\startprefix=\hbox{\tt \ \ \ \ \ \ \ \ \ \$d\ }
\setbox\contprefix=\hbox{\tt \ \ \ \ \ \ \ \ \ \ }
\startm
\m{x}\m{\,y}
\endm
\setbox\startprefix=\hbox{\tt \ \ \ \ \ \ \ \ \ \$e\ }
\setbox\contprefix=\hbox{\tt \ \ \ \ \ \ \ \ \ \ }
\startm
\m{\vdash}\m{(}\m{y}\m{\in}\m{A}\m{\rightarrow}\m{\forall}\m{x}\m{\,y}
\m{\in}\m{A}\m{)}
\endm
\vskip 1ex

should be read ``assume variable $x$ is
(effectively) not free in class $A$.''

\end{itemize}

\section{A Theorem Sampler}\label{sometheorems}

In this section we list some of the more important theorems that are proved in
the \texttt{set.mm} database, and they illustrate the kinds of things that can be
done with Metamath.  While all of these facts are well-known results,
Metamath offers the advantage of easily allowing you to trace their
derivation back to axioms.  Our intent here is not to try to explain the
details or motivation; for this we refer you to the textbooks that are
mentioned in the descriptions.  (The \texttt{set.mm} file has bibliographic
references for the text references.)  Their proofs often embody important
concepts you may wish to explore with the Metamath program (see
Section~\ref{exploring}).  All the symbols that are used here are defined in
Section~\ref{hierarchy}.  For brevity we haven't included the \texttt{\$d}
restrictions or \texttt{\$f} hypotheses for these theorems; when you are
uncertain consult the \texttt{set.mm} database.

We start with \texttt{syl} (principle of the syllogism).
In \textit{Principia Mathematica}
Whitehead and Russell call this ``the principle of the
syllogism... because... the syllogism in Barbara is derived from them''
\cite[quote after Theorem *2.06 p.~101]{PM}.
Some authors call this law a ``hypothetical syllogism.''
At the time of this writing it is the most commonly referenced assertion in the
\texttt{set.mm} database.\footnote{
The Metamath program command \texttt{show usage}
shows the number of references.
}

\vskip 2ex
\noindent Theorem syl (principle of the syllogism)\index{Syllogism}%
\index{\texttt{syl}}\label{syl}.

\vskip 0.5ex
\setbox\startprefix=\hbox{\tt \ \ syl.1\ \$e\ }
\setbox\contprefix=\hbox{\tt \ \ \ \ \ \ \ \ \ \ \ }
\startm
\m{\vdash}\m{(}\m{\varphi}\m{ \rightarrow }\m{\psi}\m{)}
\endm
\setbox\startprefix=\hbox{\tt \ \ syl.2\ \$e\ }
\setbox\contprefix=\hbox{\tt \ \ \ \ \ \ \ \ \ \ \ }
\startm
\m{\vdash}\m{(}\m{\psi}\m{ \rightarrow }\m{\chi}\m{)}
\endm
\setbox\startprefix=\hbox{\tt \ \ syl\ \$p\ }
\setbox\contprefix=\hbox{\tt \ \ \ \ \ \ \ \ \ }
\startm
\m{\vdash}\m{(}\m{\varphi}\m{ \rightarrow }\m{\chi}\m{)}
\endm
\vskip 2ex

The following theorem is not very deep but provides us with a notational device
that is frequently used.  It allows us to use the expression ``$A \in V$'' as
a compact way of saying that class $A$ exists, i.e.\ is a set.

\vskip 2ex
\noindent Two ways to say ``$A$ is a set'':  $A$ is a member of the universe
$V$ if and only if $A$ exists (i.e.\ there exists a set equal to $A$).
Theorem 6.9 of Quine, p. 43.

\vskip 0.5ex
\setbox\startprefix=\hbox{\tt \ \ isset\ \$p\ }
\setbox\contprefix=\hbox{\tt \ \ \ \ \ \ \ \ \ \ \ }
\startm
\m{\vdash}\m{(}\m{A}\m{\in}\m{{\rm V}}\m{\leftrightarrow}\m{\exists}\m{x}\m{\,x}\m{=}
\m{A}\m{)}
\endm
\vskip 1ex

Next we prove the axioms of standard ZF set theory that were missing from our
axiom system.  From our point of view they are theorems since they
can be derived from the other axioms.

\vskip 2ex
\noindent Axiom of Separation\index{Axiom of Separation}
(Aussonderung)\index{Aussonderung} proved from the other axioms of ZF set
theory.  Compare Exercise 4 of Takeuti and Zaring, p.~22.

\vskip 0.5ex
\setbox\startprefix=\hbox{\tt \ \ inex1.1\ \$e\ }
\setbox\contprefix=\hbox{\tt \ \ \ \ \ \ \ \ \ \ \ \ \ \ \ }
\startm
\m{\vdash}\m{A}\m{\in}\m{{\rm V}}
\endm
\setbox\startprefix=\hbox{\tt \ \ inex\ \$p\ }
\setbox\contprefix=\hbox{\tt \ \ \ \ \ \ \ \ \ \ \ \ \ }
\startm
\m{\vdash}\m{(}\m{A}\m{\cap}\m{B}\m{)}\m{\in}\m{{\rm V}}
\endm
\vskip 1ex

\noindent Axiom of the Null Set\index{Axiom of the Null Set} proved from the
other axioms of ZF set theory. Corollary 5.16 of Takeuti and Zaring, p.~20.

\vskip 0.5ex
\setbox\startprefix=\hbox{\tt \ \ 0ex\ \$p\ }
\setbox\contprefix=\hbox{\tt \ \ \ \ \ \ \ \ \ \ \ \ }
\startm
\m{\vdash}\m{\varnothing}\m{\in}\m{{\rm V}}
\endm
\vskip 1ex

\noindent The Axiom of Pairing\index{Axiom of Pairing} proved from the other
axioms of ZF set theory.  Theorem 7.13 of Quine, p.~51.
\vskip 0.5ex
\setbox\startprefix=\hbox{\tt \ \ prex\ \$p\ }
\setbox\contprefix=\hbox{\tt \ \ \ \ \ \ \ \ \ \ \ \ \ \ }
\startm
\m{\vdash}\m{\{}\m{A}\m{,}\m{B}\m{\}}\m{\in}\m{{\rm V}}
\endm
\vskip 2ex

Next we will list some famous or important theorems that are proved in
the \texttt{set.mm} database.  None of them except \texttt{omex}
require the Axiom of Infinity, as you can verify with the \texttt{show
trace{\char`\_}back} Metamath command.

\vskip 2ex
\noindent The resolution of Russell's paradox\index{Russell's paradox}.  There
exists no set containing the set of all sets which are not members of
themselves.  Proposition 4.14 of Takeuti and Zaring, p.~14.

\vskip 0.5ex
\setbox\startprefix=\hbox{\tt \ \ ru\ \$p\ }
\setbox\contprefix=\hbox{\tt \ \ \ \ \ \ \ \ }
\startm
\m{\vdash}\m{\lnot}\m{\exists}\m{x}\m{\,x}\m{=}\m{\{}\m{y}\m{|}\m{\lnot}\m{y}
\m{\in}\m{y}\m{\}}
\endm
\vskip 1ex

\noindent Cantor's theorem\index{Cantor's theorem}.  No set can be mapped onto
its power set.  Compare Theorem 6B(b) of Enderton, p.~132.

\vskip 0.5ex
\setbox\startprefix=\hbox{\tt \ \ canth.1\ \$e\ }
\setbox\contprefix=\hbox{\tt \ \ \ \ \ \ \ \ \ \ \ \ \ }
\startm
\m{\vdash}\m{A}\m{\in}\m{{\rm V}}
\endm
\setbox\startprefix=\hbox{\tt \ \ canth\ \$p\ }
\setbox\contprefix=\hbox{\tt \ \ \ \ \ \ \ \ \ \ \ }
\startm
\m{\vdash}\m{\lnot}\m{F}\m{:}\m{A}\m{\raisebox{.5ex}{${\textstyle{\:}_{
\mbox{\footnotesize\rm {\ }}}}\atop{\textstyle{\longrightarrow}\atop{
\textstyle{}^{\mbox{\footnotesize\rm onto}}}}$}}\m{{\cal P}}\m{A}
\endm
\vskip 1ex

\noindent The Burali-Forti paradox\index{Burali-Forti paradox}.  No set
contains all ordinal numbers. Enderton, p.~194.  (Burali-Forti was one person,
not two.)

\vskip 0.5ex
\setbox\startprefix=\hbox{\tt \ \ onprc\ \$p\ }
\setbox\contprefix=\hbox{\tt \ \ \ \ \ \ \ \ \ \ \ \ }
\startm
\m{\vdash}\m{\lnot}\m{\mbox{\rm On}}\m{\in}\m{{\rm V}}
\endm
\vskip 1ex

\noindent Peano's postulates\index{Peano's postulates} for arithmetic.
Proposition 7.30 of Takeuti and Zaring, pp.~42--43.  The objects being
described are the members of $\omega$ i.e.\ the natural numbers 0, 1,
2,\ldots.  The successor\index{successor} operation suc means ``plus
one.''  \texttt{peano1} says that 0 (which is defined as the empty set)
is a natural number.  \texttt{peano2} says that if $A$ is a natural
number, so is $A+1$.  \texttt{peano3} says that 0 is not the successor
of any natural number.  \texttt{peano4} says that two natural numbers
are equal if and only if their successors are equal.  \texttt{peano5} is
essentially the same as mathematical induction.

\vskip 1ex
\setbox\startprefix=\hbox{\tt \ \ peano1\ \$p\ }
\setbox\contprefix=\hbox{\tt \ \ \ \ \ \ \ \ \ \ \ \ }
\startm
\m{\vdash}\m{\varnothing}\m{\in}\m{\omega}
\endm
\vskip 1.5ex

\setbox\startprefix=\hbox{\tt \ \ peano2\ \$p\ }
\setbox\contprefix=\hbox{\tt \ \ \ \ \ \ \ \ \ \ \ \ }
\startm
\m{\vdash}\m{(}\m{A}\m{\in}\m{\omega}\m{\rightarrow}\m{{\rm suc}}\m{A}\m{\in}%
\m{\omega}\m{)}
\endm
\vskip 1.5ex

\setbox\startprefix=\hbox{\tt \ \ peano3\ \$p\ }
\setbox\contprefix=\hbox{\tt \ \ \ \ \ \ \ \ \ \ \ \ }
\startm
\m{\vdash}\m{(}\m{A}\m{\in}\m{\omega}\m{\rightarrow}\m{\lnot}\m{{\rm suc}}%
\m{A}\m{=}\m{\varnothing}\m{)}
\endm
\vskip 1.5ex

\setbox\startprefix=\hbox{\tt \ \ peano4\ \$p\ }
\setbox\contprefix=\hbox{\tt \ \ \ \ \ \ \ \ \ \ \ \ }
\startm
\m{\vdash}\m{(}\m{(}\m{A}\m{\in}\m{\omega}\m{\wedge}\m{B}\m{\in}\m{\omega}%
\m{)}\m{\rightarrow}\m{(}\m{{\rm suc}}\m{A}\m{=}\m{{\rm suc}}\m{B}\m{%
\leftrightarrow}\m{A}\m{=}\m{B}\m{)}\m{)}
\endm
\vskip 1.5ex

\setbox\startprefix=\hbox{\tt \ \ peano5\ \$p\ }
\setbox\contprefix=\hbox{\tt \ \ \ \ \ \ \ \ \ \ \ \ }
\startm
\m{\vdash}\m{(}\m{(}\m{\varnothing}\m{\in}\m{A}\m{\wedge}\m{\forall}\m{x}\m{%
\in}\m{\omega}\m{(}\m{x}\m{\in}\m{A}\m{\rightarrow}\m{{\rm suc}}\m{x}\m{\in}%
\m{A}\m{)}\m{)}\m{\rightarrow}\m{\omega}\m{\subseteq}\m{A}\m{)}
\endm
\vskip 1.5ex

\noindent Finite Induction (mathematical induction).\index{finite
induction}\index{mathematical induction} The first hypothesis is the
basis and the second is the induction hypothesis.  Theorem Schema 22 of
Suppes, p.~136.

\vskip 0.5ex
\setbox\startprefix=\hbox{\tt \ \ findes.1\ \$e\ }
\setbox\contprefix=\hbox{\tt \ \ \ \ \ \ \ \ \ \ \ \ \ \ }
\startm
\m{\vdash}\m{[}\m{\varnothing}\m{/}\m{x}\m{]}\m{\varphi}
\endm
\setbox\startprefix=\hbox{\tt \ \ findes.2\ \$e\ }
\setbox\contprefix=\hbox{\tt \ \ \ \ \ \ \ \ \ \ \ \ \ \ }
\startm
\m{\vdash}\m{(}\m{x}\m{\in}\m{\omega}\m{\rightarrow}\m{(}\m{\varphi}\m{%
\rightarrow}\m{[}\m{{\rm suc}}\m{x}\m{/}\m{x}\m{]}\m{\varphi}\m{)}\m{)}
\endm
\setbox\startprefix=\hbox{\tt \ \ findes\ \$p\ }
\setbox\contprefix=\hbox{\tt \ \ \ \ \ \ \ \ \ \ \ \ }
\startm
\m{\vdash}\m{(}\m{x}\m{\in}\m{\omega}\m{\rightarrow}\m{\varphi}\m{)}
\endm
\vskip 1ex

\noindent Transfinite Induction with explicit substitution.  The first
hypothesis is the basis, the second is the induction hypothesis for
successors, and the third is the induction hypothesis for limit
ordinals.  Theorem Schema 4 of Suppes, p. 197.

\vskip 0.5ex
\setbox\startprefix=\hbox{\tt \ \ tfindes.1\ \$e\ }
\setbox\contprefix=\hbox{\tt \ \ \ \ \ \ \ \ \ \ \ \ \ \ \ }
\startm
\m{\vdash}\m{[}\m{\varnothing}\m{/}\m{x}\m{]}\m{\varphi}
\endm
\setbox\startprefix=\hbox{\tt \ \ tfindes.2\ \$e\ }
\setbox\contprefix=\hbox{\tt \ \ \ \ \ \ \ \ \ \ \ \ \ \ \ }
\startm
\m{\vdash}\m{(}\m{x}\m{\in}\m{{\rm On}}\m{\rightarrow}\m{(}\m{\varphi}\m{%
\rightarrow}\m{[}\m{{\rm suc}}\m{x}\m{/}\m{x}\m{]}\m{\varphi}\m{)}\m{)}
\endm
\setbox\startprefix=\hbox{\tt \ \ tfindes.3\ \$e\ }
\setbox\contprefix=\hbox{\tt \ \ \ \ \ \ \ \ \ \ \ \ \ \ \ }
\startm
\m{\vdash}\m{(}\m{{\rm Lim}}\m{y}\m{\rightarrow}\m{(}\m{\forall}\m{x}\m{\in}%
\m{y}\m{\varphi}\m{\rightarrow}\m{[}\m{y}\m{/}\m{x}\m{]}\m{\varphi}\m{)}\m{)}
\endm
\setbox\startprefix=\hbox{\tt \ \ tfindes\ \$p\ }
\setbox\contprefix=\hbox{\tt \ \ \ \ \ \ \ \ \ \ \ \ \ }
\startm
\m{\vdash}\m{(}\m{x}\m{\in}\m{{\rm On}}\m{\rightarrow}\m{\varphi}\m{)}
\endm
\vskip 1ex

\noindent Principle of Transfinite Recursion.\index{transfinite
recursion} Theorem 7.41 of Takeuti and Zaring, p.~47.  Transfinite
recursion is the key theorem that allows arithmetic of ordinals to be
rigorously defined, and has many other important uses as well.
Hypotheses \texttt{tfr.1} and \texttt{tfr.2} specify a certain (proper)
class $ F$.  The complicated definition of $ F$ is not important in
itself; what is important is that there be such an $ F$ with the
required properties, and we show this by displaying $ F$ explicitly.
\texttt{tfr1} states that $ F$ is a function whose domain is the set of
ordinal numbers.  \texttt{tfr2} states that any value of $ F$ is
completely determined by its previous values and the values of an
auxiliary function, $G$.  \texttt{tfr3} states that $ F$ is unique,
i.e.\ it is the only function that satisfies \texttt{tfr1} and
\texttt{tfr2}.  Note that $ f$ is an individual variable like $x$ and
$y$; it is just a mnemonic to remind us that $A$ is a collection of
functions.

\vskip 0.5ex
\setbox\startprefix=\hbox{\tt \ \ tfr.1\ \$e\ }
\setbox\contprefix=\hbox{\tt \ \ \ \ \ \ \ \ \ \ \ }
\startm
\m{\vdash}\m{A}\m{=}\m{\{}\m{f}\m{|}\m{\exists}\m{x}\m{\in}\m{{\rm On}}\m{(}%
\m{f}\m{{\rm Fn}}\m{x}\m{\wedge}\m{\forall}\m{y}\m{\in}\m{x}\m{(}\m{f}\m{`}%
\m{y}\m{)}\m{=}\m{(}\m{G}\m{`}\m{(}\m{f}\m{\restriction}\m{y}\m{)}\m{)}\m{)}%
\m{\}}
\endm
\setbox\startprefix=\hbox{\tt \ \ tfr.2\ \$e\ }
\setbox\contprefix=\hbox{\tt \ \ \ \ \ \ \ \ \ \ \ }
\startm
\m{\vdash}\m{F}\m{=}\m{\bigcup}\m{A}
\endm
\setbox\startprefix=\hbox{\tt \ \ tfr1\ \$p\ }
\setbox\contprefix=\hbox{\tt \ \ \ \ \ \ \ \ \ \ }
\startm
\m{\vdash}\m{F}\m{{\rm Fn}}\m{{\rm On}}
\endm
\setbox\startprefix=\hbox{\tt \ \ tfr2\ \$p\ }
\setbox\contprefix=\hbox{\tt \ \ \ \ \ \ \ \ \ \ }
\startm
\m{\vdash}\m{(}\m{z}\m{\in}\m{{\rm On}}\m{\rightarrow}\m{(}\m{F}\m{`}\m{z}%
\m{)}\m{=}\m{(}\m{G}\m{`}\m{(}\m{F}\m{\restriction}\m{z}\m{)}\m{)}\m{)}
\endm
\setbox\startprefix=\hbox{\tt \ \ tfr3\ \$p\ }
\setbox\contprefix=\hbox{\tt \ \ \ \ \ \ \ \ \ \ }
\startm
\m{\vdash}\m{(}\m{(}\m{B}\m{{\rm Fn}}\m{{\rm On}}\m{\wedge}\m{\forall}\m{x}\m{%
\in}\m{{\rm On}}\m{(}\m{B}\m{`}\m{x}\m{)}\m{=}\m{(}\m{G}\m{`}\m{(}\m{B}\m{%
\restriction}\m{x}\m{)}\m{)}\m{)}\m{\rightarrow}\m{B}\m{=}\m{F}\m{)}
\endm
\vskip 1ex

\noindent The existence of omega (the class of natural numbers).\index{natural
number}\index{omega ($\omega$)}\index{Axiom of Infinity}  Axiom 7 of Takeuti
and Zaring, p.~43.  (This is the only theorem in this section requiring the
Axiom of Infinity.)

\vskip 0.5ex
\setbox\startprefix=\hbox{\tt \
\ omex\ \$p\ }
\setbox\contprefix=\hbox{\tt \ \ \ \ \ \ \ \ \ \ }
\startm
\m{\vdash}\m{\omega}\m{\in}\m{{\rm V}}
\endm
%\vskip 2ex


\section{Axioms for Real and Complex Numbers}\label{real}
\index{real and complex numbers!axioms for}

This section presents the axioms for real and complex numbers.  Analysis
textbooks implicitly or explicitly use these axioms or their equivalents
as their starting point.  In the database \texttt{set.mm}, we define real
and complex numbers as (rather complicated) specific sets and derive these
axioms as {\em theorems} from the axioms of ZF set theory, using a method
called Dedekind cuts.  We omit the details of this construction, which you can
follow if you wish using the \texttt{set.mm} database in conjunction with the
textbooks referenced therein.

Once we prove those theorems, we then restate these proven theorems as axioms.
This lets us easily identify which axioms are needed for a particular complex number proof, without the obfuscation of the set theory used to derive them.
As a result,
the construction is actually unimportant other
than to show that sets exist that satisfy the axioms, and thus that the axioms
are consistent if ZF set theory is consistent.  When working with real numbers
you can think of them as being the actual sets resulting
from the construction (for definiteness), or you can
think of them as otherwise unspecified sets that happen to satisfy the axioms.
The derivation is not easy, but the fact that it works is quite remarkable
and lends support to the idea that ZFC set theory is all we need to
provide a foundation for essentially all of mathematics.

For the axioms we are given (or postulate) 8 classes:  $\mathbb{C}$ (the
set of complex numbers), $\mathbb{R}$ (the set of real numbers, a subset
of $\mathbb{C}$), $0$ (zero), $1$ (one), $i$ (square root of
$-1$), $+$ (plus), $\cdot$ (times), and
$<_{\mathbb{R}}$ (less than for just the real numbers).
Subtraction and division are defined terms and are not part of the
axioms; for their definitions see \texttt{set.mm}.

Note that the notation $(A+B)$ (and similarly $(A\cdot B)$) specifies a class
called an {\em operation},\index{operation} and is the function value of the
class $+$ at ordered pair $\langle A,B \rangle$.  An operation is defined by
statement \texttt{df-opr} on p.~\pageref{dfopr}.
The notation $A <_{\mathbb{R}} B$ specifies a
wff called a {\em binary relation}\index{binary relation} and means $\langle A,B \rangle \in \,<_{\mathbb{R}}$, as defined by statement \texttt{df-br} on p.~\pageref{dfbr}.

Our set of 8 given classes is assumed to satisfy the following 22 axioms
(in the axioms listed below, $<$ really means $<_{\mathbb{R}}$).

\vskip 2ex

\noindent 1. The real numbers are a subset of the complex numbers.

%\vskip 0.5ex
\setbox\startprefix=\hbox{\tt \ \ ax-resscn\ \$p\ }
\setbox\contprefix=\hbox{\tt \ \ \ \ \ \ \ \ \ \ \ \ \ \ }
\startm
\m{\vdash}\m{\mathbb{R}}\m{\subseteq}\m{\mathbb{C}}
\endm
%\vskip 1ex

\noindent 2. One is a complex number.

%\vskip 0.5ex
\setbox\startprefix=\hbox{\tt \ \ ax-1cn\ \$p\ }
\setbox\contprefix=\hbox{\tt \ \ \ \ \ \ \ \ \ \ \ }
\startm
\m{\vdash}\m{1}\m{\in}\m{\mathbb{C}}
\endm
%\vskip 1ex

\noindent 3. The imaginary number $i$ is a complex number.

%\vskip 0.5ex
\setbox\startprefix=\hbox{\tt \ \ ax-icn\ \$p\ }
\setbox\contprefix=\hbox{\tt \ \ \ \ \ \ \ \ \ \ \ }
\startm
\m{\vdash}\m{i}\m{\in}\m{\mathbb{C}}
\endm
%\vskip 1ex

\noindent 4. Complex numbers are closed under addition.

%\vskip 0.5ex
\setbox\startprefix=\hbox{\tt \ \ ax-addcl\ \$p\ }
\setbox\contprefix=\hbox{\tt \ \ \ \ \ \ \ \ \ \ \ \ \ }
\startm
\m{\vdash}\m{(}\m{(}\m{A}\m{\in}\m{\mathbb{C}}\m{\wedge}\m{B}\m{\in}\m{\mathbb{C}}%
\m{)}\m{\rightarrow}\m{(}\m{A}\m{+}\m{B}\m{)}\m{\in}\m{\mathbb{C}}\m{)}
\endm
%\vskip 1ex

\noindent 5. Real numbers are closed under addition.

%\vskip 0.5ex
\setbox\startprefix=\hbox{\tt \ \ ax-addrcl\ \$p\ }
\setbox\contprefix=\hbox{\tt \ \ \ \ \ \ \ \ \ \ \ \ \ \ }
\startm
\m{\vdash}\m{(}\m{(}\m{A}\m{\in}\m{\mathbb{R}}\m{\wedge}\m{B}\m{\in}\m{\mathbb{R}}%
\m{)}\m{\rightarrow}\m{(}\m{A}\m{+}\m{B}\m{)}\m{\in}\m{\mathbb{R}}\m{)}
\endm
%\vskip 1ex

\noindent 6. Complex numbers are closed under multiplication.

%\vskip 0.5ex
\setbox\startprefix=\hbox{\tt \ \ ax-mulcl\ \$p\ }
\setbox\contprefix=\hbox{\tt \ \ \ \ \ \ \ \ \ \ \ \ \ }
\startm
\m{\vdash}\m{(}\m{(}\m{A}\m{\in}\m{\mathbb{C}}\m{\wedge}\m{B}\m{\in}\m{\mathbb{C}}%
\m{)}\m{\rightarrow}\m{(}\m{A}\m{\cdot}\m{B}\m{)}\m{\in}\m{\mathbb{C}}\m{)}
\endm
%\vskip 1ex

\noindent 7. Real numbers are closed under multiplication.

%\vskip 0.5ex
\setbox\startprefix=\hbox{\tt \ \ ax-mulrcl\ \$p\ }
\setbox\contprefix=\hbox{\tt \ \ \ \ \ \ \ \ \ \ \ \ \ \ }
\startm
\m{\vdash}\m{(}\m{(}\m{A}\m{\in}\m{\mathbb{R}}\m{\wedge}\m{B}\m{\in}\m{\mathbb{R}}%
\m{)}\m{\rightarrow}\m{(}\m{A}\m{\cdot}\m{B}\m{)}\m{\in}\m{\mathbb{R}}\m{)}
\endm
%\vskip 1ex

\noindent 8. Multiplication of complex numbers is commutative.

%\vskip 0.5ex
\setbox\startprefix=\hbox{\tt \ \ ax-mulcom\ \$p\ }
\setbox\contprefix=\hbox{\tt \ \ \ \ \ \ \ \ \ \ \ \ \ \ }
\startm
\m{\vdash}\m{(}\m{(}\m{A}\m{\in}\m{\mathbb{C}}\m{\wedge}\m{B}\m{\in}\m{\mathbb{C}}%
\m{)}\m{\rightarrow}\m{(}\m{A}\m{\cdot}\m{B}\m{)}\m{=}\m{(}\m{B}\m{\cdot}\m{A}%
\m{)}\m{)}
\endm
%\vskip 1ex

\noindent 9. Addition of complex numbers is associative.

%\vskip 0.5ex
\setbox\startprefix=\hbox{\tt \ \ ax-addass\ \$p\ }
\setbox\contprefix=\hbox{\tt \ \ \ \ \ \ \ \ \ \ \ \ \ \ }
\startm
\m{\vdash}\m{(}\m{(}\m{A}\m{\in}\m{\mathbb{C}}\m{\wedge}\m{B}\m{\in}\m{\mathbb{C}}%
\m{\wedge}\m{C}\m{\in}\m{\mathbb{C}}\m{)}\m{\rightarrow}\m{(}\m{(}\m{A}\m{+}%
\m{B}\m{)}\m{+}\m{C}\m{)}\m{=}\m{(}\m{A}\m{+}\m{(}\m{B}\m{+}\m{C}\m{)}\m{)}%
\m{)}
\endm
%\vskip 1ex

\noindent 10. Multiplication of complex numbers is associative.

%\vskip 0.5ex
\setbox\startprefix=\hbox{\tt \ \ ax-mulass\ \$p\ }
\setbox\contprefix=\hbox{\tt \ \ \ \ \ \ \ \ \ \ \ \ \ \ }
\startm
\m{\vdash}\m{(}\m{(}\m{A}\m{\in}\m{\mathbb{C}}\m{\wedge}\m{B}\m{\in}\m{\mathbb{C}}%
\m{\wedge}\m{C}\m{\in}\m{\mathbb{C}}\m{)}\m{\rightarrow}\m{(}\m{(}\m{A}\m{\cdot}%
\m{B}\m{)}\m{\cdot}\m{C}\m{)}\m{=}\m{(}\m{A}\m{\cdot}\m{(}\m{B}\m{\cdot}\m{C}%
\m{)}\m{)}\m{)}
\endm
%\vskip 1ex

\noindent 11. Multiplication distributes over addition for complex numbers.

%\vskip 0.5ex
\setbox\startprefix=\hbox{\tt \ \ ax-distr\ \$p\ }
\setbox\contprefix=\hbox{\tt \ \ \ \ \ \ \ \ \ \ \ \ \ }
\startm
\m{\vdash}\m{(}\m{(}\m{A}\m{\in}\m{\mathbb{C}}\m{\wedge}\m{B}\m{\in}\m{\mathbb{C}}%
\m{\wedge}\m{C}\m{\in}\m{\mathbb{C}}\m{)}\m{\rightarrow}\m{(}\m{A}\m{\cdot}\m{(}%
\m{B}\m{+}\m{C}\m{)}\m{)}\m{=}\m{(}\m{(}\m{A}\m{\cdot}\m{B}\m{)}\m{+}\m{(}%
\m{A}\m{\cdot}\m{C}\m{)}\m{)}\m{)}
\endm
%\vskip 1ex

\noindent 12. The square of $i$ equals $-1$ (expressed as $i$-squared plus 1 is
0).

%\vskip 0.5ex
\setbox\startprefix=\hbox{\tt \ \ ax-i2m1\ \$p\ }
\setbox\contprefix=\hbox{\tt \ \ \ \ \ \ \ \ \ \ \ \ }
\startm
\m{\vdash}\m{(}\m{(}\m{i}\m{\cdot}\m{i}\m{)}\m{+}\m{1}\m{)}\m{=}\m{0}
\endm
%\vskip 1ex

\noindent 13. One and zero are distinct.

%\vskip 0.5ex
\setbox\startprefix=\hbox{\tt \ \ ax-1ne0\ \$p\ }
\setbox\contprefix=\hbox{\tt \ \ \ \ \ \ \ \ \ \ \ \ }
\startm
\m{\vdash}\m{1}\m{\ne}\m{0}
\endm
%\vskip 1ex

\noindent 14. One is an identity element for real multiplication.

%\vskip 0.5ex
\setbox\startprefix=\hbox{\tt \ \ ax-1rid\ \$p\ }
\setbox\contprefix=\hbox{\tt \ \ \ \ \ \ \ \ \ \ \ }
\startm
\m{\vdash}\m{(}\m{A}\m{\in}\m{\mathbb{R}}\m{\rightarrow}\m{(}\m{A}\m{\cdot}\m{1}%
\m{)}\m{=}\m{A}\m{)}
\endm
%\vskip 1ex

\noindent 15. Every real number has a negative.

%\vskip 0.5ex
\setbox\startprefix=\hbox{\tt \ \ ax-rnegex\ \$p\ }
\setbox\contprefix=\hbox{\tt \ \ \ \ \ \ \ \ \ \ \ \ \ \ }
\startm
\m{\vdash}\m{(}\m{A}\m{\in}\m{\mathbb{R}}\m{\rightarrow}\m{\exists}\m{x}\m{\in}%
\m{\mathbb{R}}\m{(}\m{A}\m{+}\m{x}\m{)}\m{=}\m{0}\m{)}
\endm
%\vskip 1ex

\noindent 16. Every nonzero real number has a reciprocal.

%\vskip 0.5ex
\setbox\startprefix=\hbox{\tt \ \ ax-rrecex\ \$p\ }
\setbox\contprefix=\hbox{\tt \ \ \ \ \ \ \ \ \ \ \ \ \ \ }
\startm
\m{\vdash}\m{(}\m{A}\m{\in}\m{\mathbb{R}}\m{\rightarrow}\m{(}\m{A}\m{\ne}\m{0}%
\m{\rightarrow}\m{\exists}\m{x}\m{\in}\m{\mathbb{R}}\m{(}\m{A}\m{\cdot}%
\m{x}\m{)}\m{=}\m{1}\m{)}\m{)}
\endm
%\vskip 1ex

\noindent 17. A complex number can be expressed in terms of two reals.

%\vskip 0.5ex
\setbox\startprefix=\hbox{\tt \ \ ax-cnre\ \$p\ }
\setbox\contprefix=\hbox{\tt \ \ \ \ \ \ \ \ \ \ \ \ }
\startm
\m{\vdash}\m{(}\m{A}\m{\in}\m{\mathbb{C}}\m{\rightarrow}\m{\exists}\m{x}\m{\in}%
\m{\mathbb{R}}\m{\exists}\m{y}\m{\in}\m{\mathbb{R}}\m{A}\m{=}\m{(}\m{x}\m{+}\m{(}%
\m{y}\m{\cdot}\m{i}\m{)}\m{)}\m{)}
\endm
%\vskip 1ex

\noindent 18. Ordering on reals satisfies strict trichotomy.

%\vskip 0.5ex
\setbox\startprefix=\hbox{\tt \ \ ax-pre-lttri\ \$p\ }
\setbox\contprefix=\hbox{\tt \ \ \ \ \ \ \ \ \ \ \ \ \ }
\startm
\m{\vdash}\m{(}\m{(}\m{A}\m{\in}\m{\mathbb{R}}\m{\wedge}\m{B}\m{\in}\m{\mathbb{R}}%
\m{)}\m{\rightarrow}\m{(}\m{A}\m{<}\m{B}\m{\leftrightarrow}\m{\lnot}\m{(}\m{A}%
\m{=}\m{B}\m{\vee}\m{B}\m{<}\m{A}\m{)}\m{)}\m{)}
\endm
%\vskip 1ex

\noindent 19. Ordering on reals is transitive.

%\vskip 0.5ex
\setbox\startprefix=\hbox{\tt \ \ ax-pre-lttrn\ \$p\ }
\setbox\contprefix=\hbox{\tt \ \ \ \ \ \ \ \ \ \ \ \ \ }
\startm
\m{\vdash}\m{(}\m{(}\m{A}\m{\in}\m{\mathbb{R}}\m{\wedge}\m{B}\m{\in}\m{\mathbb{R}}%
\m{\wedge}\m{C}\m{\in}\m{\mathbb{R}}\m{)}\m{\rightarrow}\m{(}\m{(}\m{A}\m{<}%
\m{B}\m{\wedge}\m{B}\m{<}\m{C}\m{)}\m{\rightarrow}\m{A}\m{<}\m{C}\m{)}\m{)}
\endm
%\vskip 1ex

\noindent 20. Ordering on reals is preserved after addition to both sides.

%\vskip 0.5ex
\setbox\startprefix=\hbox{\tt \ \ ax-pre-ltadd\ \$p\ }
\setbox\contprefix=\hbox{\tt \ \ \ \ \ \ \ \ \ \ \ \ \ }
\startm
\m{\vdash}\m{(}\m{(}\m{A}\m{\in}\m{\mathbb{R}}\m{\wedge}\m{B}\m{\in}\m{\mathbb{R}}%
\m{\wedge}\m{C}\m{\in}\m{\mathbb{R}}\m{)}\m{\rightarrow}\m{(}\m{A}\m{<}\m{B}\m{%
\rightarrow}\m{(}\m{C}\m{+}\m{A}\m{)}\m{<}\m{(}\m{C}\m{+}\m{B}\m{)}\m{)}\m{)}
\endm
%\vskip 1ex

\noindent 21. The product of two positive reals is positive.

%\vskip 0.5ex
\setbox\startprefix=\hbox{\tt \ \ ax-pre-mulgt0\ \$p\ }
\setbox\contprefix=\hbox{\tt \ \ \ \ \ \ \ \ \ \ \ \ \ \ }
\startm
\m{\vdash}\m{(}\m{(}\m{A}\m{\in}\m{\mathbb{R}}\m{\wedge}\m{B}\m{\in}\m{\mathbb{R}}%
\m{)}\m{\rightarrow}\m{(}\m{(}\m{0}\m{<}\m{A}\m{\wedge}\m{0}%
\m{<}\m{B}\m{)}\m{\rightarrow}\m{0}\m{<}\m{(}\m{A}\m{\cdot}\m{B}\m{)}%
\m{)}\m{)}
\endm
%\vskip 1ex

\noindent 22. A non-empty, bounded-above set of reals has a supremum.

%\vskip 0.5ex
\setbox\startprefix=\hbox{\tt \ \ ax-pre-sup\ \$p\ }
\setbox\contprefix=\hbox{\tt \ \ \ \ \ \ \ \ \ \ \ }
\startm
\m{\vdash}\m{(}\m{(}\m{A}\m{\subseteq}\m{\mathbb{R}}\m{\wedge}\m{A}\m{\ne}\m{%
\varnothing}\m{\wedge}\m{\exists}\m{x}\m{\in}\m{\mathbb{R}}\m{\forall}\m{y}\m{%
\in}\m{A}\m{\,y}\m{<}\m{x}\m{)}\m{\rightarrow}\m{\exists}\m{x}\m{\in}\m{%
\mathbb{R}}\m{(}\m{\forall}\m{y}\m{\in}\m{A}\m{\lnot}\m{x}\m{<}\m{y}\m{\wedge}\m{%
\forall}\m{y}\m{\in}\m{\mathbb{R}}\m{(}\m{y}\m{<}\m{x}\m{\rightarrow}\m{\exists}%
\m{z}\m{\in}\m{A}\m{\,y}\m{<}\m{z}\m{)}\m{)}\m{)}
\endm

% NOTE: The \m{...} expressions above could be represented as
% $ \vdash ( ( A \subseteq \mathbb{R} \wedge A \ne \varnothing \wedge \exists x \in \mathbb{R} \forall y \in A \,y < x ) \rightarrow \exists x \in \mathbb{R} ( \forall y \in A \lnot x < y \wedge \forall y \in \mathbb{R} ( y < x \rightarrow \exists z \in A \,y < z ) ) ) $

\vskip 2ex

This completes the set of axioms for real and complex numbers.  You may
wish to look at how subtraction, division, and decimal numbers
are defined in \texttt{set.mm}, and for fun look at the proof of $2+
2 = 4$ (theorem \texttt{2p2e4} in \texttt{set.mm})
as discussed in section \ref{2p2e4}.

In \texttt{set.mm} we define the non-negative integers $\mathbb{N}$, the integers
$\mathbb{Z}$, and the rationals $\mathbb{Q}$ as subsets of $\mathbb{R}$.  This leads
to the nice inclusion $\mathbb{N} \subseteq \mathbb{Z} \subseteq \mathbb{Q} \subseteq
\mathbb{R} \subseteq \mathbb{C}$, giving us a uniform framework in which, for
example, a property such as commutativity of complex number addition
automatically applies to integers.  The natural numbers $\mathbb{N}$
are different from the set $\omega$ we defined earlier, but both satisfy
Peano's postulates.

\subsubsection{Complex Number Axioms in Analysis Texts}

Most analysis texts construct complex numbers as ordered pairs of reals,
leading to construction-dependent properties that satisfy these axioms
but are not stated in their pure form.  (This is also done in
\texttt{set.mm} but our axioms are extracted from that construction.)
Other texts will simply state that $\mathbb{R}$ is a ``complete ordered
subfield of $\mathbb{C}$,'' leading to redundant axioms when this phrase
is completely expanded out.  In fact I have not seen a text with the
axioms in the explicit form above.
None of these axioms is unique individually, but this carefully worked out
collection of axioms is the result of years of work
by the Metamath community.

\subsubsection{Eliminating Unnecessary Complex Number Axioms}

We once had more axioms for real and complex numbers, but over years of time
we (the Metamath community)
have found ways to eliminate them (by proving them from other axioms)
or weaken them (by making weaker claims without reducing what
can be proved).
In particular, here are statements that used to be complex number
axioms but have since been formally proven (with Metamath) to be redundant:

\begin{itemize}
\item
  $\mathbb{C} \in V$.
  At one time this was listed as a ``complex number axiom''
  However, this is not properly speaking a complex number axiom,
  and in any case its proof uses axioms of set theory.
  Proven redundant by Mario Carneiro\index{Carneiro, Mario} on
  17-Nov-2014 (see \texttt{axcnex}).
\item
  $((A \in \mathbb{C} \land B \in \mathbb{C}$) $\rightarrow$
  $(A + B) = (B + A))$.
  Proved redundant by Eric Schmidt\index{Schmidt, Eric} on 19-Jun-2012,
  and formalized by Scott Fenton\index{Fenton, Scott} on 3-Jan-2013
  (see \texttt{addcom}).
\item
  $(A \in \mathbb{C} \rightarrow (A + 0) = A)$.
  Proved redundant by Eric Schmidt on 19-Jun-2012,
  and formalized by Scott Fenton on 3-Jan-2013
  (see \texttt{addid1}).
\item
  $(A \in \mathbb{C} \rightarrow \exists x \in \mathbb{C} (A + x) = 0)$.
  Proved redundant by Eric Schmidt and formalized on 21-May-2007
  (see \texttt{cnegex}).
\item
  $((A \in \mathbb{C} \land A \ne 0) \rightarrow \exists x \in \mathbb{C} (A \cdot x) = 1)$.
  Proved redundant by Eric Schmidt and formalized on 22-May-2007
  (see \texttt{recex}).
\item
  $0 \in \mathbb{R}$.
  Proved redundant by Eric Schmidt on 19-Feb-2005 and formalized 21-May-2007
  (see \texttt{0re}).
\end{itemize}

We could eliminate 0 as an axiomatic object by defining it as
$( ( i \cdot i ) + 1 )$
and replacing it with this expression throughout the axioms. If this
is done, axiom ax-i2m1 becomes redundant. However, the remaining axioms
would become longer and less intuitive.

Eric Schmidt's paper analyzing this axiom system \cite{Schmidt}
presented a proof that these remaining axioms,
with the possible exception of ax-mulcom, are independent of the others.
It is currently an open question if ax-mulcom is independent of the others.

\section{Two Plus Two Equals Four}\label{2p2e4}

Here is a proof that $2 + 2 = 4$, as proven in the theorem \texttt{2p2e4}
in the database \texttt{set.mm}.
This is a useful demonstration of what a Metamath proof can look like.
This proof may have more steps than you're used to, but each step is rigorously
proven all the way back to the axioms of logic and set theory.
This display was originally generated by the Metamath program
as an {\sc HTML} file.

In the table showing the proof ``Step'' is the sequential step number,
while its associated ``Expression'' is an expression that we have proved.
``Ref'' is the name of a theorem or axiom that justifies that expression,
and ``Hyp'' refers to previous steps (if any) that the theorem or axiom
needs so that we can use it.  Expressions are indented further than
the expressions that depend on them to show their interdependencies.

\begin{table}[!htbp]
\caption{Two plus two equals four}
\begin{tabular}{lllll}
\textbf{Step} & \textbf{Hyp} & \textbf{Ref} & \textbf{Expression} & \\
1  &       & df-2    & $ \; \; \vdash 2 = 1 + 1$  & \\
2  & 1     & oveq2i  & $ \; \vdash (2 + 2) = (2 + (1 + 1))$ & \\
3  &       & df-4    & $ \; \; \vdash 4 = (3 + 1)$ & \\
4  &       & df-3    & $ \; \; \; \vdash 3 = (2 + 1)$ & \\
5  & 4     & oveq1i  & $ \; \; \vdash (3 + 1) = ((2 + 1) + 1)$ & \\
6  &       & 2cn     & $ \; \; \; \vdash 2 \in \mathbb{C}$ & \\
7  &       & ax-1cn  & $ \; \; \; \vdash 1 \in \mathbb{C}$ & \\
8  & 6,7,7 & addassi & $ \; \; \vdash ((2 + 1) + 1) = (2 + (1 + 1))$ & \\
9  & 3,5,8 & 3eqtri  & $ \; \vdash 4 = (2 + (1 + 1))$ & \\
10 & 2,9   & eqtr4i  & $ \vdash (2 + 2) = 4$ & \\
\end{tabular}
\end{table}

Step 1 says that we can assert that $2 = 1 + 1$ because it is
justified by \texttt{df-2}.
What is \texttt{df-2}?
It is simply the definition of $2$, which in our system is defined as being
equal to $1 + 1$.  This shows how we can use definitions in proofs.

Look at Step 2 of the proof. In the Ref column, we see that it references
a previously proved theorem, \texttt{oveq2i}.
It turns out that
theorem \texttt{oveq2i} requires a
hypothesis, and in the Hyp column of Step 2 we indicate that Step 1 will
satisfy (match) this hypothesis.
If we looked at \texttt{oveq2i}
we would find that it proves that given some hypothesis
$A = B$, we can prove that $( C F A ) = ( C F B )$.
If we use \texttt{oveq2i} and apply step 1's result as the hypothesis,
that will mean that $A = 2$ and $B = ( 1 + 1 )$ within this use of
\texttt{oveq2i}.
We are free to select any value of $C$ and $F$ (subject to syntax constraints),
so we are free to select $C = 2$ and $F = +$,
producing our desired result,
$ (2 + 2) = (2 + (1 + 1))$.

Step 2 is an example of substitution.
In the end, every step in every proof uses only this one substitution rule.
All the rules of logic, and all the axioms, are expressed so that
they can be used via this one substitution rule.
So once you master substitution, you can master every Metamath proof,
no exceptions.

Each step is clear and can be immediately checked.
In the {\sc HTML} display you can even click on each reference to see why it is
justified, making it easy to see why the proof works.

\section{Deduction}\label{deduction}

Strictly speaking,
a deduction (also called an inference) is a kind of statement that needs
some hypotheses to be true in order for its conclusion to be true.
A theorem, on the other hand, has no hypotheses.
Informally we often call both of them theorems, but in this section we
will stick to the strict definitions.

It sometimes happens that we have proved a deduction of the form
$\varphi \Rightarrow \psi$\index{$\Rightarrow$}
(given hypothesis $\varphi$ we can prove $\psi$)
and we want to then prove a theorem of the form
$\varphi \rightarrow \psi$.

Converting a deduction (which uses a hypothesis) into a theorem
(which does not) is not as simple as you might think.
The deduction says, ``if we can prove $\varphi$ then we can prove $\psi$,''
which is in some sense weaker than saying
``$\varphi$ implies $\psi$.''
There is no axiom of logic that permits us to directly obtain the theorem
given the deduction.\footnote{
The conversion of a deduction to a theorem does not even hold in general
for quantum propositional calculus,
which is a weak subset of classical propositional calculus.
It has been shown that adding the Standard Deduction Theorem (discussed below)
to quantum propositional calculus turns it into classical
propositional calculus!
}

This is in contrast to going the other way.
If we have the theorem ($\varphi \rightarrow \psi$),
it is easy to recover the deduction
($\varphi \Rightarrow \psi$)
using modus ponens\index{modus ponens}
(\texttt{ax-mp}; see section \ref{axmp}).

In the following subsections we first discuss the standard deduction theorem
(the traditional but awkward way to convert deductions into theorems) and
the weak deduction theorem (a limited version of the standard deduction
theorem that is easier to use and was once widely used in
\texttt{set.mm}\index{set theory database (\texttt{set.mm})}).
In section \ref{deductionstyle} we discuss
deduction style, the newer approach we now recommend in most cases.
Deduction style uses ``deduction form,'' a form that
prefixes each hypothesis (other than definitions) and the
conclusion with a universal antecedent (``$\varphi \rightarrow$'').
Deduction style is widely used in \texttt{set.mm},
so it is useful to understand it and \textit{why} it is widely used.
Section \ref{naturaldeduction}
briefly discusses our approach for using natural deduction
within \texttt{set.mm},
as that approach is deeply related to deduction style.
We conclude with a summary of the strengths of
our approach, which we believe are compelling.

\subsection{The Standard Deduction Theorem}\label{standarddeductiontheorem}

It is possible to make use of information
contained in the deduction or its proof to assist us with the proof of
the related theorem.
In traditional logic books, there is a metatheorem called the
Deduction Theorem\index{Deduction Theorem}\index{Standard Deduction Theorem},
discovered independently by Herbrand and Tarski around 1930.
The Deduction Theorem, which we often call the Standard Deduction Theorem,
provides an algorithm for constructing a proof of a theorem from the
proof of its corresponding deduction. See, for example,
\cite[p.~56]{Margaris}\index{Margaris, Angelo}.
To construct a proof for a theorem, the
algorithm looks at each step in the proof of the original deduction and
rewrites the step with several steps wherein the hypothesis is eliminated
and becomes an antecedent.

In ordinary mathematics, no one actually carries out the algorithm,
because (in its most basic form) it involves an exponential explosion of
the number of proof steps as more hypotheses are eliminated. Instead,
the Standard Deduction Theorem is invoked simply to claim that it can
be done in principle, without actually doing it.
What's more, the algorithm is not as simple as it might first appear
when applying it rigorously.
There is a subtle restriction on the Standard Deduction Theorem
that must be taken into account involving the axiom of generalization
when working with predicate calculus (see the literature for more detail).

One of the goals of Metamath is to let you plainly see, with as few
underlying concepts as possible, how mathematics can be derived directly
from the axioms, and not indirectly according to some hidden rules
buried inside a program or understood only by logicians. If we added
the Standard Deduction Theorem to the language and proof verifier,
that would greatly complicate both and largely defeat Metamath's goal
of simplicity. In principle, we could show direct proofs by expanding
out the proof steps generated by the algorithm of the Standard Deduction
Theorem, but that is not feasible in practice because the number of proof
steps quickly becomes huge, even astronomical.
Since the algorithm of the Standard Deduction Theorem is driven by the proof,
we would have to go through that proof
all over again---starting from axioms---in order to obtain the theorem form.
In terms of proof length, there would be no savings over just
proving the theorem directly instead of first proving the deduction form.

\subsection{Weak Deduction Theorem}\label{weakdeductiontheorem}

We have developed
a more efficient method for proving a theorem from a deduction
that can be used instead of the Standard Deduction Theorem
in many (but not all) cases.
We call this more efficient method the
Weak Deduction Theorem\index{Weak Deduction Theorem}.\footnote{
There is also an unrelated ``Weak Deduction Theorem''
in the field of relevance logic, so to avoid confusion we could call
ours the ``Weak Deduction Theorem for Classical Logic.''}
Unlike the Standard Deduction Theorem, the Weak Deduction Theorem produces the
theorem directly from a special substitution instance of the deduction,
using a small, fixed number of steps roughly proportional to the length
of the final theorem.

If you come to a proof referencing the Weak Deduction Theorem
\texttt{dedth} (or one of its variants \texttt{dedthxx}),
here is how to follow the proof without getting into the details:
just click on the theorem referenced in the step
just before the reference to \texttt{dedth} and ignore everything else.
Theorem \texttt{dedth} simply turns a hypothesis into an antecedent
(i.e. the hypothesis followed by $\rightarrow$
is placed in front of the assertion, and the hypothesis
itself is eliminated) given certain conditions.

The Weak Deduction Theorem
eliminates a hypothesis $\varphi$, making it become an antecedent.
It does this by proving an expression
$ \varphi \rightarrow \psi $ given two hypotheses:
(1)
$ ( A = {\rm if} ( \varphi , A , B ) \rightarrow ( \varphi \leftrightarrow \chi ) ) $
and
(2) $\chi$.
Note that it requires that a proof exists for $\varphi$ when the class variable
$A$ is replaced with a specific class $B$. The hypothesis $\chi$
should be assigned to the inference.
You can see the details of the proof of the Weak Deduction Theorem
in theorem \texttt{dedth}.

The Weak Deduction Theorem
is probably easier to understand by studying proofs that make use of it.
For example, let's look at the proof of \texttt{renegcl}, which proves that
$ \vdash ( A \in \mathbb{R} \rightarrow - A \in \mathbb{R} )$:

\needspace{4\baselineskip}
\begin{longtabu} {l l l X}
\textbf{Step} & \textbf{Hyp} & \textbf{Ref} & \textbf{Expression} \\
  1 &  & negeq &
  $\vdash$ $($ $A$ $=$ ${\rm if}$ $($ $A$ $\in$ $\mathbb{R}$ $,$ $A$ $,$ $1$ $)$ $\rightarrow$
  $\textrm{-}$ $A$ $=$ $\textrm{-}$ ${\rm if}$ $($ $A$ $\in$ $\mathbb{R}$
  $,$ $A$ $,$ $1$ $)$ $)$ \\
 2 & 1 & eleq1d &
    $\vdash$ $($ $A$ $=$ ${\rm if}$ $($ $A$ $\in$ $\mathbb{R}$ $,$ $A$ $,$ $1$ $)$ $\rightarrow$ $($
    $\textrm{-}$ $A$ $\in$ $\mathbb{R}$ $\leftrightarrow$
    $\textrm{-}$ ${\rm if}$ $($ $A$ $\in$ $\mathbb{R}$ $,$ $A$ $,$ $1$ $)$ $\in$
    $\mathbb{R}$ $)$ $)$ \\
 3 &  & 1re & $\vdash 1 \in \mathbb{R}$ \\
 4 & 3 & elimel &
   $\vdash {\rm if} ( A \in \mathbb{R} , A , 1 ) \in \mathbb{R}$ \\
 5 & 4 & renegcli &
   $\vdash \textrm{-} {\rm if} ( A \in \mathbb{R} , A , 1 ) \in \mathbb{R}$ \\
 6 & 2,5 & dedth &
   $\vdash ( A \in \mathbb{R} \rightarrow \textrm{-} A \in \mathbb{R}$ ) \\
\end{longtabu}

The somewhat strange-looking steps in \texttt{renegcl} before step 5 are
technical stuff that makes this magic work, and they can be ignored
for a quick overview of the proof. To continue following the ``important''
part of the proof of \texttt{renegcl},
you can look at the reference to \texttt{renegcli} at step 5.

That said, let's briefly look at how
\texttt{renegcl} uses the
Weak Deduction Theorem (\texttt{dedth}) to do its job,
in case you want to do something similar or want understand it more deeply.
Let's work backwards in the proof of \texttt{renegcl}.
Step 6 applies \texttt{dedth} to produce our goal result
$ \vdash ( A \in \mathbb{R} \rightarrow\, - A \in \mathbb{R} )$.
This requires on the one hand the (substituted) deduction
\texttt{renegcli} in step 5.
By itself \texttt{renegcli} proves the deduction
$ \vdash A \in \mathbb{R} \Rightarrow\, \vdash - A \in \mathbb{R}$;
this is the deduction form we are trying to turn into theorem form,
and thus
\texttt{renegcli} has a separate hypothesis that must be fulfilled.
To fulfill the hypothesis of the invocation of
\texttt{renegcli} in step 5, it is eventually
reduced to the already proven theorem $1 \in \mathbb{R}$ in step 3.
Step 4 connects steps 3 and 5; step 4 invokes
\texttt{elimel}, a special case of \texttt{elimhyp} that eliminates
a membership hypothesis for the weak deduction theorem.
On the other hand, the equivalence of the conclusion of
\texttt{renegcl}
$( - A \in \mathbb{R} )$ and the substituted conclusion of
\texttt{renegcli} must be proven, which is done in steps 2 and 1.

The weak deduction theorem has limitations.
In particular, we must be able to prove a special case of the deduction's
hypothesis as a stand-alone theorem.
For example, we used $1 \in \mathbb{R}$ in step 3 of \texttt{renegcl}.

We used to use the weak deduction theorem
extensively within \texttt{set.mm}.
However, we now recommend applying ``deduction style''
instead in most cases, as deduction style is
often an easier and clearer approach.
Therefore, we will now describe deduction style.

\subsection{Deduction Style}\label{deductionstyle}

We now prefer to write assertions in ``deduction form''
instead of writing a proof that would require use of the standard or
weak deduction theorem.
We call this appraoch
``deduction style.''\index{deduction style}

It will be easier to explain this by first defining some terms:

\begin{itemize}
\item \textbf{closed form}\index{closed form}\index{forms!closed}:
A kind of assertion (theorem) with no hypotheses.
Typically its label has no special suffix.
An example is \texttt{unss}, which states:
$\vdash ( ( A \subseteq C \wedge B \subseteq C ) \leftrightarrow ( A \cup B )
\subseteq C )\label{eq:unss}$
\item \textbf{deduction form}\index{deduction form}\index{forms!deduction}:
A kind of assertion with one or more hypotheses
where the conclusion is an implication with
a wff variable as the antecedent (usually $\varphi$), and every hypothesis
(\$e statement)
is either (1) an implication with the same antecedent as the conclusion or
(2) a definition.
A definition
can be for a class variable (this is a class variable followed by ``='')
or a wff variable (this is a wff variable followed by $\leftrightarrow$);
class variable definitions are more common.
In practice, a proof
in deduction form will also contain many steps that are implications
where the antecedent is either that wff variable (normally $\varphi$)
or is
a conjunction (...$\land$...) including that wff variable ($\varphi$).
If an assertion is in deduction form, and other forms are also available,
then we suffix its label with ``d.''
An example is \texttt{unssd}, which states\footnote{
For brevity we show here (and in other places)
a $\&$\index{$\&$} between hypotheses\index{hypotheses}
and a $\Rightarrow$\index{$\Rightarrow$}\index{conclusion}
between the hypotheses and the conclusion.
This notation is technically not part of the Metamath language, but is
instead a convenient abbreviation to show both the hypotheses and conclusion.}:
$\vdash ( \varphi \rightarrow A \subseteq C )\quad\&\quad \vdash ( \varphi
    \rightarrow B \subseteq C )\quad\Rightarrow\quad \vdash ( \varphi
    \rightarrow ( A \cup B ) \subseteq C )\label{eq:unssd}$
\item \textbf{inference form}\index{inference form}\index{forms!inference}:
A kind of assertion with one or more hypotheses that is not in deduction form
(e.g., there is no common antecedent).
If an assertion is in inference form, and other forms are also available,
then we suffix its label with ``i.''
An example is \texttt{unssi}, which states:
$\vdash A \subseteq C\quad\&\quad \vdash B \subseteq C\quad\Rightarrow\quad
    \vdash ( A \cup B ) \subseteq C\label{eq:unssi}$
\end{itemize}

When using deduction style we express an assertion in deduction form.
This form prefixes each hypothesis (other than definitions) and the
conclusion with a universal antecedent (``$\varphi \rightarrow$'').
The antecedent (e.g., $\varphi$)
mimics the context handled in the deduction theorem, eliminating
the need to directly use the deduction theorem.

Once you have an assertion in deduction form, you can easily convert it
to inference form or closed form:

\begin{itemize}
\item To
prove some assertion Ti in inference form, given assertion Td in deduction
form, there is a simple mechanical process you can use. First take each
Ti hypothesis and insert a \texttt{T.} $\rightarrow$ prefix (``true implies'')
using \texttt{a1i}. You
can then use the existing assertion Td to prove the resulting conclusion
with a \texttt{T.} $\rightarrow$ prefix.
Finally, you can remove that prefix using \texttt{trud},
resulting in the conclusion you wanted to prove.
\item To
prove some assertion T in closed form, given assertion Td in deduction
form, there is another simple mechanical process you can use. First,
select an expression that is the conjunction (...$\land$...) of all of the
consequents of every hypothesis of Td. Next, prove that this expression
implies each of the separate hypotheses of Td in turn by eliminating
conjuncts (there are a variety of proven assertions to do this, including
\texttt{simpl},
\texttt{simpr},
\texttt{3simpa},
\texttt{3simpb},
\texttt{3simpc},
\texttt{simp1},
\texttt{simp2},
and
\texttt{simp3}).
If the
expression has nested conjunctions, inner conjuncts can be broken out by
chaining the above theorems with \texttt{syl}
(see section \ref{syl}).\footnote{
There are actually many theorems
(labeled simp* such as \texttt{simp333}) that break out inner conjuncts in one
step, but rather than learning them you can just use the chaining we
just described to prove them, and then let the Metamath program command
\texttt{minimize{\char`\_}with}\index{\texttt{minimize{\char`\_}with} command}
figure out the right ones needed to collapse them.}
As your final step, you can then apply the already-proven assertion Td
(which is in deduction form), proving assertion T in closed form.
\end{itemize}

We can also easily convert any assertion T in closed form to its related
assertion Ti in inference form by applying
modus ponens\index{modus ponens} (see section \ref{axmp}).

The deduction form antecedent can also be used to represent the context
necessary to support natural deduction systems, so we will now
discuss natural deduction.

\subsection{Natural Deduction}\label{naturaldeduction}

Natural deduction\index{natural deduction}
(ND) systems, as such, were originally introduced in
1934 by two logicians working independently: Ja\'skowski and Gentzen. ND
systems are supposed to reconstruct, in a formally proper way, traditional
ways of mathematical reasoning (such as conditional proof, indirect proof,
and proof by cases). As reconstructions they were naturally influenced
by previous work, and many specific ND systems and notations have been
developed since their original work.

There are many ND variants, but
Indrzejczak \cite[p.~31-32]{Indrzejczak}\index{Indrzejczak, Andrzej}
suggests that any natural deductive system must satisfy at
least these three criteria:

\begin{itemize}
\item ``There are some means for entering assumptions into a proof and
also for eliminating them. Usually it requires some bookkeeping devices
for indicating the scope of an assumption, and showing that a part of
a proof depending on eliminated assumption is discharged.
\item There are no (or, at least, a very limited set of) axioms, because
their role is taken over by the set of primitive rules for introduction
and elimination of logical constants which means that elementary
inferences instead of formulae are taken as primitive.
\item (A genuine) ND system admits a lot of freedom in proof construction
and possibility of applying several proof search strategies, like
conditional proof, proof by cases, proof by reductio ad absurdum etc.''
\end{itemize}

The Metamath Proof Explorer (MPE) as defined in \texttt{set.mm}
is fundamentally a Hilbert-style system.
That is, MPE is based on a larger number of axioms (compared
to natural deduction systems), a very small set of rules of inference
(modus ponens), and the context is not changed by the rules of inference
in the middle of a proof. That said, MPE proofs can be developed using
the natural deduction (ND) approach as originally developed by Ja\'skowski
and Gentzen.

The most common and recommended approach for applying ND in MPE is to use
deduction form\index{deduction form}%
\index{forms!deduction}
and apply the MPE proven assertions that are equivalent to ND rules.
For example, MPE's \texttt{jca} is equivalent to ND rule $\land$-I
(and-insertion).
We maintain a list of equivalences that you may consult.
This approach for applying an ND approach within MPE relies on Metamath's
wff metavariables in an essential way, and is described in more detail
in the presentation ``Natural Deductions in the Metamath Proof Language''
by Mario Carneiro \cite{CarneiroND}\index{Carneiro, Mario}.

In this style many steps are an implication, whose antecedent mimics
the context ($\Gamma$) of most ND systems. To add an assumption, simply add
it to the implication antecedent (typically using
\texttt{simpr}),
and use that
new antecedent for all later claims in the same scope. If you wish to
use an assertion in an ND hypothesis scope that is outside the current
ND hypothesis scope, modify the assertion so that the ND hypothesis
assumption is added to its antecedent (typically using \texttt{adantr}). Most
proof steps will be proved using rules that have hypotheses and results
of the form $\varphi \rightarrow$ ...

An example may make this clearer.
Let's show theorem 5.5 of
\cite[p.~18]{Clemente}\index{Clemente Laboreo, Daniel}
along with a line by line translation using the usual
translation of natural deduction (ND) in the Metamath Proof Explorer
(MPE) notation (this is proof \texttt{ex-natded5.5}).
The proof's original goal was to prove
$\lnot \psi$ given two hypotheses,
$( \psi \rightarrow \chi )$ and $ \lnot \chi$.
We will translate these statements into MPE deduction form
by prefixing them all with $\varphi \rightarrow$.
As a result, in MPE the goal is stated as
$( \varphi \rightarrow \lnot \psi )$, and the two hypotheses are stated as
$( \varphi \rightarrow ( \psi \rightarrow \chi ) )$ and
$( \varphi \rightarrow \lnot \chi )$.

The following table shows the proof in Fitch natural deduction style
and its MPE equivalent.
The \textit{\#} column shows the original numbering,
\textit{MPE\#} shows the number in the equivalent MPE proof
(which we will show later),
\textit{ND Expression} shows the original proof claim in ND notation,
and \textit{MPE Translation} shows its translation into MPE
as discussed in this section.
The final columns show the rationale in ND and MPE respectively.

\needspace{4\baselineskip}
{\setlength{\extrarowsep}{4pt} % Keep rows from being too close together
\begin{longtabu}   { @{} c c X X X X }
\textbf{\#} & \textbf{MPE\#} & \textbf{ND Ex\-pres\-sion} &
\textbf{MPE Trans\-lation} & \textbf{ND Ration\-ale} &
\textbf{MPE Ra\-tio\-nale} \\
\endhead

1 & 2;3 &
$( \psi \rightarrow \chi )$ &
$( \varphi \rightarrow ( \psi \rightarrow \chi ) )$ &
Given &
\$e; \texttt{adantr} to put in ND hypothesis \\

2 & 5 &
$ \lnot \chi$ &
$( \varphi \rightarrow \lnot \chi )$ &
Given &
\$e; \texttt{adantr} to put in ND hypothesis \\

3 & 1 &
... $\vert$ $\psi$ &
$( \varphi \rightarrow \psi )$ &
ND hypothesis assumption &
\texttt{simpr} \\

4 & 4 &
... $\chi$ &
$( ( \varphi \land \psi ) \rightarrow \chi )$ &
$\rightarrow$\,E 1,3 &
\texttt{mpd} 1,3 \\

5 & 6 &
... $\lnot \chi$ &
$( ( \varphi \land \psi ) \rightarrow \lnot \chi )$ &
IT 2 &
\texttt{adantr} 5 \\

6 & 7 &
$\lnot \psi$ &
$( \varphi \rightarrow \lnot \psi )$ &
$\land$\,I 3,4,5 &
\texttt{pm2.65da} 4,6 \\

\end{longtabu}
}


The original used Latin letters; we have replaced them with Greek letters
to follow Metamath naming conventions and so that it is easier to follow
the Metamath translation. The Metamath line-for-line translation of
this natural deduction approach precedes every line with an antecedent
including $\varphi$ and uses the Metamath equivalents of the natural deduction
rules. To add an assumption, the antecedent is modified to include it
(typically by using \texttt{adantr};
\texttt{simpr} is useful when you want to
depend directly on the new assumption, as is shown here).

In Metamath we can represent the two given statements as these hypotheses:

\needspace{2\baselineskip}
\begin{itemize}
\item ex-natded5.5.1 $\vdash ( \varphi \rightarrow ( \psi \rightarrow \chi ) )$
\item ex-natded5.5.2 $\vdash ( \varphi \rightarrow \lnot \chi )$
\end{itemize}

\needspace{4\baselineskip}
Here is the proof in Metamath as a line-by-line translation:

\begin{longtabu}   { l l l X }
\textbf{Step} & \textbf{Hyp} & \textbf{Ref} & \textbf{Ex\-pres\-sion} \\
\endhead
1 & & simpr & $\vdash ( ( \varphi \land \psi ) \rightarrow \psi )$ \\
2 & & ex-natded5.5.1 &
  $\vdash ( \varphi \rightarrow ( \psi \rightarrow \chi ) )$ \\
3 & 2 & adantr &
 $\vdash ( ( \varphi \land \psi ) \rightarrow ( \psi \rightarrow \chi ) )$ \\
4 & 1, 3 & mpd &
 $\vdash ( ( \varphi \land \psi ) \rightarrow \chi ) $ \\
5 & & ex-natded5.5.2 &
 $\vdash ( \varphi \rightarrow \lnot \chi )$ \\
6 & 5 & adantr &
 $\vdash ( ( \varphi \land \psi ) \rightarrow \lnot \chi )$ \\
7 & 4, 6 & pm2.65da &
 $\vdash ( \varphi \rightarrow \lnot \psi )$ \\
\end{longtabu}

Only using specific natural deduction rules directly can lead to very
long proofs, for exactly the same reason that only using axioms directly
in Hilbert-style proofs can lead to very long proofs.
If the goal is short and clear proofs,
then it is better to reuse already-proven assertions
in deduction form than to start from scratch each time
and using only basic natural deduction rules.

\subsection{Strengths of Our Approach}

As far as we know there is nothing else in the literature like either the
weak deduction theorem or Mario Carneiro\index{Carneiro, Mario}'s
natural deduction method.
In order to
transform a hypothesis into an antecedent, the literature's standard
``Deduction Theorem''\index{Deduction Theorem}\index{Standard Deduction Theorem}
requires metalogic outside of the notions provided
by the axiom system. We instead generally prefer to use Mario Carneiro's
natural deduction method, then use the weak deduction theorem in cases
where that is difficult to apply, and only then use the full standard
deduction theorem as a last resort.

The weak deduction theorem\index{Weak Deduction Theorem}
does not require any additional metalogic
but converts an inference directly into a closed form theorem, with
a rigorous proof that uses only the axiom system. Unlike the standard
Deduction Theorem, there is no implicit external justification that we
have to trust in order to use it.

Mario Carneiro's natural deduction\index{natural deduction}
method also does not require any new metalogical
notions. It avoids the Deduction Theorem's metalogic by prefixing the
hypotheses and conclusion of every would-be inference with a universal
antecedent (``$\varphi \rightarrow$'') from the very start.

We think it is impressive and satisfying that we can do so much in a
practical sense without stepping outside of our Hilbert-style axiom system.
Of course our axiomatization, which is in the form of schemes,
contains a metalogic of its own that we exploit. But this metalogic
is relatively simple, and for our Deduction Theorem alternatives,
we primarily use just the direct substitution of expressions for
metavariables.

\begin{sloppy}
\section{Exploring the Set The\-o\-ry Data\-base}\label{exploring}
\end{sloppy}
% NOTE: All examples performed in this section are
% recorded wtih "set width 61" % on set.mm as of 2019-05-28
% commit c1e7849557661260f77cfdf0f97ac4354fbb4f4d.

At this point you may wish to study the \texttt{set.mm}\index{set theory
database (\texttt{set.mm})} file in more detail.  Pay particular
attention to the assumptions needed to define wffs\index{well-formed
formula (wff)} (which are not included above), the variable types
(\texttt{\$f}\index{\texttt{\$f} statement} statements), and the
definitions that are introduced.  Start with some simple theorems in
propositional calculus, making sure you understand in detail each step
of a proof.  Once you get past the first few proofs and become familiar
with the Metamath language, any part of the \texttt{set.mm} database
will be as easy to follow, step by step, as any other part---you won't
have to undergo a ``quantum leap'' in mathematical sophistication to be
able to follow a deep proof in set theory.

Next, you may want to explore how concepts such as natural numbers are
defined and described.  This is probably best done in conjunction with
standard set theory textbooks, which can help give you a higher-level
understanding.  The \texttt{set.mm} database provides references that will get
you started.  From there, you will be on your way towards a very deep,
rigorous understanding of abstract mathematics.

The Metamath\index{Metamath} program can help you peruse a Metamath data\-base,
wheth\-er you are trying to figure out how a certain step follows in a proof or
just have a general curiosity.  We will go through some examples of the
commands, using the \texttt{set.mm}\index{set theory database (\texttt{set.mm})}
database provided with the Metamath software.  These should help get you
started.  See Chapter~\ref{commands} for a more detailed description of
the commands.  Note that we have included the full spelling of all commands to
prevent ambiguity with future commands.  In practice you may type just the
characters needed to specify each command keyword\index{command keyword}
unambiguously, often just one or two characters per keyword, and you don't
need to type them in upper case.

First run the Metamath program as described earlier.  You should see the
\verb/MM>/ prompt.  Read in the \texttt{set.mm} file:\index{\texttt{read}
command}

\begin{verbatim}
MM> read set.mm
Reading source file "set.mm"... 34554442 bytes
34554442 bytes were read into the source buffer.
The source has 155711 statements; 2254 are $a and 32250 are $p.
No errors were found.  However, proofs were not checked.
Type VERIFY PROOF * if you want to check them.
\end{verbatim}

As with most examples in this book, what you will see
will be slightly different because we are continuously
improving our databases (including \texttt{set.mm}).

Let's check the database integrity.  This may take a minute or two to run if
your computer is slow.

\begin{verbatim}
MM> verify proof *
0 10%  20%  30%  40%  50%  60%  70%  80%  90% 100%
..................................................
All proofs in the database were verified in 2.84 s.
\end{verbatim}

No errors were reported, so every proof is correct.

You need to know the names (labels) of theorems before you can look at them.
Often just examining the database file(s) with a text editor is the best
approach.  In \texttt{set.mm} there are many detailed comments, especially near
the beginning, that can help guide you. The \texttt{search} command in the
Metamath program is also handy.  The \texttt{comments} qualifier will list the
statements whose associated comment (the one immediately before it) contain a
string you give it.  For example, if you are studying Enderton's {\em Elements
of Set Theory} \cite{Enderton}\index{Enderton, Herbert B.} you may want to see
the references to it in the database.  The search string \texttt{enderton} is not
case sensitive.  (This will not show you all the database theorems that are in
Enderton's book because there is usually only one citation for a given
theorem, which may appear in several textbooks.)\index{\texttt{search}
command}

\begin{verbatim}
MM> search * "enderton" / comments
12067 unineq $p "... Exercise 20 of [Enderton] p. 32 and ..."
12459 undif2 $p "...Corollary 6K of [Enderton] p. 144. (C..."
12953 df-tp $a "...s. Definition of [Enderton] p. 19. (Co..."
13689 unissb $p ".... Exercise 5 of [Enderton] p. 26 and ..."
\end{verbatim}
\begin{center}
(etc.)
\end{center}

Or you may want to see what theorems have something to do with
conjunction (logical {\sc and}).  The quotes around the search
string are optional when there's no ambiguity.\index{\texttt{search}
command}

\begin{verbatim}
MM> search * conjunction / comments
120 a1d $p "...be replaced with a conjunction ( ~ df-an )..."
662 df-bi $a "...viated form after conjunction is introdu..."
1319 wa $a "...ff definition to include conjunction ('and')."
1321 df-an $a "Define conjunction (logical 'and'). Defini..."
1420 imnan $p "...tion in terms of conjunction. (Contribu..."
\end{verbatim}
\begin{center}
(etc.)
\end{center}


Now we will start to look at some details.  Let's look at the first
axiom of propositional calculus
(we could use \texttt{sh st} to abbreviate
\texttt{show statement}).\index{\texttt{show statement} command}

\begin{verbatim}
MM> show statement ax-1/full
Statement 19 is located on line 881 of the file "set.mm".
"Axiom _Simp_.  Axiom A1 of [Margaris] p. 49.  One of the 3
axioms of propositional calculus.  The 3 axioms are also
        ...
19 ax-1 $a |- ( ph -> ( ps -> ph ) ) $.
Its mandatory hypotheses in RPN order are:
  wph $f wff ph $.
  wps $f wff ps $.
The statement and its hypotheses require the variables:  ph
      ps
The variables it contains are:  ph ps


Statement 49 is located on line 11182 of the file "set.mm".
Its statement number for HTML pages is 6.
"Axiom _Simp_.  Axiom A1 of [Margaris] p. 49.  One of the 3
axioms of propositional calculus.  The 3 axioms are also
given as Definition 2.1 of [Hamilton] p. 28.
...
49 ax-1 $a |- ( ph -> ( ps -> ph ) ) $.
Its mandatory hypotheses in RPN order are:
  wph $f wff ph $.
  wps $f wff ps $.
The statement and its hypotheses require the variables:
  ph ps
The variables it contains are:  ph ps
\end{verbatim}

Compare this to \texttt{ax-1} on p.~\pageref{ax1}.  You can see that the
symbol \texttt{ph} is the {\sc ascii} notation for $\varphi$, etc.  To
see the mathematical symbols for any expression you may typeset it in
\LaTeX\ (type \texttt{help tex} for instructions)\index{latex@{\LaTeX}}
or, easier, just use a text editor to look at the comments where symbols
are first introduced in \texttt{set.mm}.  The hypotheses \texttt{wph}
and \texttt{wps} required by \texttt{ax-1} mean that variables
\texttt{ph} and \texttt{ps} must be wffs.

Next we'll pick a simple theorem of propositional calculus, the Principle of
Identity, which is proved directly from the axioms.  We'll look at the
statement then its proof.\index{\texttt{show statement}
command}

\begin{verbatim}
MM> show statement id1/full
Statement 116 is located on line 11371 of the file "set.mm".
Its statement number for HTML pages is 22.
"Principle of identity.  Theorem *2.08 of [WhiteheadRussell]
p. 101.  This version is proved directly from the axioms for
demonstration purposes.
...
116 id1 $p |- ( ph -> ph ) $= ... $.
Its mandatory hypotheses in RPN order are:
  wph $f wff ph $.
Its optional hypotheses are:  wps wch wth wta wet
      wze wsi wrh wmu wla wka
The statement and its hypotheses require the variables:  ph
These additional variables are allowed in its proof:
      ps ch th ta et ze si rh mu la ka
The variables it contains are:  ph
\end{verbatim}

The optional variables\index{optional variable} \texttt{ps}, \texttt{ch}, etc.\ are
available for use in a proof of this statement if we wish, and were we to do
so we would make use of optional hypotheses \texttt{wps}, \texttt{wch}, etc.  (See
Section~\ref{dollaref} for the meaning of ``optional
hypothesis.''\index{optional hypothesis}) The reason these show up in the
statement display is that statement \texttt{id1} happens to be in their scope
(see Section~\ref{scoping} for the definition of ``scope''\index{scope}), but
in fact in propositional calculus we will never make use of optional
hypotheses or variables.  This becomes important after quantifiers are
introduced, where ``dummy'' variables\index{dummy variable} are often needed
in the middle of a proof.

Let's look at the proof of statement \texttt{id1}.  We'll use the
\texttt{show proof} command, which by default suppresses the
``non-essential'' steps that construct the wffs.\index{\texttt{show proof}
command}
We will display the proof in ``lemmon' format (a non-indented format
with explicit previous step number references) and renumber the
displayed steps:

\begin{verbatim}
MM> show proof id1 /lemmon/renumber
1 ax-1           $a |- ( ph -> ( ph -> ph ) )
2 ax-1           $a |- ( ph -> ( ( ph -> ph ) -> ph ) )
3 ax-2           $a |- ( ( ph -> ( ( ph -> ph ) -> ph ) ) ->
                     ( ( ph -> ( ph -> ph ) ) -> ( ph -> ph )
                                                          ) )
4 2,3 ax-mp      $a |- ( ( ph -> ( ph -> ph ) ) -> ( ph -> ph
                                                          ) )
5 1,4 ax-mp      $a |- ( ph -> ph )
\end{verbatim}

If you have read Section~\ref{trialrun}, you'll know how to interpret this
proof.  Step~2, for example, is an application of axiom \texttt{ax-1}.  This
proof is identical to the one in Hamilton's {\em Logic for Mathematicians}
\cite[p.~32]{Hamilton}\index{Hamilton, Alan G.}.

You may want to look at what
substitutions\index{substitution!variable}\index{variable substitution} are
made into \texttt{ax-1} to arrive at step~2.  The command to do this needs to
know the ``real'' step number, so we'll display the proof again without
the \texttt{renumber} qualifier.\index{\texttt{show proof}
command}

\begin{verbatim}
MM> show proof id1 /lemmon
 9 ax-1          $a |- ( ph -> ( ph -> ph ) )
20 ax-1          $a |- ( ph -> ( ( ph -> ph ) -> ph ) )
24 ax-2          $a |- ( ( ph -> ( ( ph -> ph ) -> ph ) ) ->
                     ( ( ph -> ( ph -> ph ) ) -> ( ph -> ph )
                                                          ) )
25 20,24 ax-mp   $a |- ( ( ph -> ( ph -> ph ) ) -> ( ph -> ph
                                                          ) )
26 9,25 ax-mp    $a |- ( ph -> ph )
\end{verbatim}

The ``real'' step number is 20.  Let's look at its details.

\begin{verbatim}
MM> show proof id1 /detailed_step 20
Proof step 20:  min=ax-1 $a |- ( ph -> ( ( ph -> ph ) -> ph )
  )
This step assigns source "ax-1" ($a) to target "min" ($e).
The source assertion requires the hypotheses "wph" ($f, step
18) and "wps" ($f, step 19).  The parent assertion of the
target hypothesis is "ax-mp" ($a, step 25).
The source assertion before substitution was:
    ax-1 $a |- ( ph -> ( ps -> ph ) )
The following substitutions were made to the source
assertion:
    Variable  Substituted with
     ph        ph
     ps        ( ph -> ph )
The target hypothesis before substitution was:
    min $e |- ph
The following substitution was made to the target hypothesis:
    Variable  Substituted with
     ph        ( ph -> ( ( ph -> ph ) -> ph ) )
\end{verbatim}

This shows the substitutions\index{substitution!variable}\index{variable
substitution} made to the variables in \texttt{ax-1}.  References are made to
steps 18 and 19 which are not shown in our proof display.  To see these steps,
you can display the proof with the \texttt{all} qualifier.

Let's look at a slightly more advanced proof of propositional calculus.  Note
that \verb+/\+ is the symbol for $\wedge$ (logical {\sc and}, also
called conjunction).\index{conjunction ($\wedge$)}
\index{logical {\sc and} ($\wedge$)}

\begin{verbatim}
MM> show statement prth/full
Statement 1791 is located on line 15503 of the file "set.mm".
Its statement number for HTML pages is 559.
"Conjoin antecedents and consequents of two premises.  This
is the closed theorem form of ~ anim12d .  Theorem *3.47 of
[WhiteheadRussell] p. 113.  It was proved by Leibniz,
and it evidently pleased him enough to call it
_praeclarum theorema_ (splendid theorem).
...
1791 prth $p |- ( ( ( ph -> ps ) /\ ( ch -> th ) ) -> ( ( ph
      /\ ch ) -> ( ps /\ th ) ) ) $= ... $.
Its mandatory hypotheses in RPN order are:
  wph $f wff ph $.
  wps $f wff ps $.
  wch $f wff ch $.
  wth $f wff th $.
Its optional hypotheses are:  wta wet wze wsi wrh wmu wla wka
The statement and its hypotheses require the variables:  ph
      ps ch th
These additional variables are allowed in its proof:  ta et
      ze si rh mu la ka
The variables it contains are:  ph ps ch th


MM> show proof prth /lemmon/renumber
1 simpl          $p |- ( ( ( ph -> ps ) /\ ( ch -> th ) ) ->
                                               ( ph -> ps ) )
2 simpr          $p |- ( ( ( ph -> ps ) /\ ( ch -> th ) ) ->
                                               ( ch -> th ) )
3 1,2 anim12d    $p |- ( ( ( ph -> ps ) /\ ( ch -> th ) ) ->
                           ( ( ph /\ ch ) -> ( ps /\ th ) ) )
\end{verbatim}

There are references to a lot of unfamiliar statements.  To see what they are,
you may type the following:

\begin{verbatim}
MM> show proof prth /statement_summary
Summary of statements used in the proof of "prth":

Statement simpl is located on line 14748 of the file
"set.mm".
"Elimination of a conjunct.  Theorem *3.26 (Simp) of
[WhiteheadRussell] p. 112. ..."
  simpl $p |- ( ( ph /\ ps ) -> ph ) $= ... $.

Statement simpr is located on line 14777 of the file
"set.mm".
"Elimination of a conjunct.  Theorem *3.27 (Simp) of
[WhiteheadRussell] ..."
  simpr $p |- ( ( ph /\ ps ) -> ps ) $= ... $.

Statement anim12d is located on line 15445 of the file
"set.mm".
"Conjoin antecedents and consequents in a deduction.
..."
  anim12d.1 $e |- ( ph -> ( ps -> ch ) ) $.
  anim12d.2 $e |- ( ph -> ( th -> ta ) ) $.
  anim12d $p |- ( ph -> ( ( ps /\ th ) -> ( ch /\ ta ) ) )
      $= ... $.
\end{verbatim}
\begin{center}
(etc.)
\end{center}

Of course you can look at each of these statements and their proofs, and
so on, back to the axioms of propositional calculus if you wish.

The \texttt{search} command is useful for finding statements when you
know all or part of their contents.  The following example finds all
statements containing \verb@ph -> ps@ followed by \verb@ch -> th@.  The
\verb@$*@ is a wildcard that matches anything; the \texttt{\$} before the
\verb$*$ prevents conflicts with math symbol token names.  The \verb@*@ after
\texttt{SEARCH} is also a wildcard that in this case means ``match any label.''
\index{\texttt{search} command}

% I'm omitting this one, since readers are unlikely to see it:
% 1096 bisymOLD $p |- ( ( ( ph -> ps ) -> ( ch -> th ) ) -> ( (
%   ( ps -> ph ) -> ( th -> ch ) ) -> ( ( ph <-> ps ) -> ( ch
%    <-> th ) ) ) )
\begin{verbatim}
MM> search * "ph -> ps $* ch -> th"
1791 prth $p |- ( ( ( ph -> ps ) /\ ( ch -> th ) ) -> ( ( ph
    /\ ch ) -> ( ps /\ th ) ) )
2455 pm3.48 $p |- ( ( ( ph -> ps ) /\ ( ch -> th ) ) -> ( (
    ph \/ ch ) -> ( ps \/ th ) ) )
117859 pm11.71 $p |- ( ( E. x ph /\ E. y ch ) -> ( ( A. x (
    ph -> ps ) /\ A. y ( ch -> th ) ) <-> A. x A. y ( ( ph /\
    ch ) -> ( ps /\ th ) ) ) )
\end{verbatim}

Three statements, \texttt{prth}, \texttt{pm3.48},
 and \texttt{pm11.71}, were found to match.

To see what axioms\index{axiom} and definitions\index{definition}
\texttt{prth} ultimately depends on for its proof, you can have the
program backtrack through the hierarchy\index{hierarchy} of theorems and
definitions.\index{\texttt{show trace{\char`\_}back} command}

\begin{verbatim}
MM> show trace_back prth /essential/axioms
Statement "prth" assumes the following axioms ($a
statements):
  ax-1 ax-2 ax-3 ax-mp df-bi df-an
\end{verbatim}

Note that the 3 axioms of propositional calculus and the modus ponens rule are
needed (as expected); in addition, there are a couple of definitions that are used
along the way.  Note that Metamath makes no distinction\index{axiom vs.\
definition} between axioms\index{axiom} and definitions\index{definition}.  In
\texttt{set.mm} they have been distinguished artificially by prefixing their
labels\index{labels in \texttt{set.mm}} with \texttt{ax-} and \texttt{df-}
respectively.  For example, \texttt{df-an} defines conjunction (logical {\sc
and}), which is represented by the symbol \verb+/\+.
Section~\ref{definitions} discusses the philosophy of definitions, and the
Metamath language takes a particularly simple, conservative approach by using
the \texttt{\$a}\index{\texttt{\$a} statement} statement for both axioms and
definitions.

You can also have the program compute how many steps a proof
has\index{proof length} if we were to follow it all the way back to
\texttt{\$a} statements.

\begin{verbatim}
MM> show trace_back prth /essential/count_steps
The statement's actual proof has 3 steps.  Backtracking, a
total of 79 different subtheorems are used.  The statement
and subtheorems have a total of 274 actual steps.  If
subtheorems used only once were eliminated, there would be a
total of 38 subtheorems, and the statement and subtheorems
would have a total of 185 steps.  The proof would have 28349
steps if fully expanded back to axiom references.  The
maximum path length is 38.  A longest path is:  prth <-
anim12d <- syl2and <- sylan2d <- ancomsd <- ancom <- pm3.22
<- pm3.21 <- pm3.2 <- ex <- sylbir <- biimpri <- bicomi <-
bicom1 <- bi2 <- dfbi1 <- impbii <- bi3 <- simprim <- impi <-
con1i <- nsyl2 <- mt3d <- con1d <- notnot1 <- con2i <- nsyl3
<- mt2d <- con2d <- notnot2 <- pm2.18d <- pm2.18 <- pm2.21 <-
pm2.21d <- a1d <- syl <- mpd <- a2i <- a2i.1 .
\end{verbatim}

This tells us that we would have to inspect 274 steps if we want to
verify the proof completely starting from the axioms.  A few more
statistics are also shown.  There are one or more paths back to axioms
that are the longest; this command ferrets out one of them and shows it
to you.  There may be a sense in which the longest path length is
related to how ``deep'' the theorem is.

We might also be curious about what proofs depend on the theorem
\texttt{prth}.  If it is never used later on, we could eliminate it as
redundant if it has no intrinsic interest by itself.\index{\texttt{show
usage} command}

% I decided to show the OLD values here.
\begin{verbatim}
MM> show usage prth
Statement "prth" is directly referenced in the proofs of 18
statements:
  mo3 moOLD 2mo 2moOLD euind reuind reuss2 reusv3i opelopabt
  wemaplem2 rexanre rlimcn2 o1of2 o1rlimmul 2sqlem6 spanuni
  heicant pm11.71
\end{verbatim}

Thus \texttt{prth} is directly used by 18 proofs.
We can use the \texttt{/recursive} qualifier to include indirect use:

\begin{verbatim}
MM> show usage prth /recursive
Statement "prth" directly or indirectly affects the proofs of
24214 statements:
  mo3 mo mo3OLD eu2 moOLD eu2OLD eu3OLD mo4f mo4 eu4 mopick
...
\end{verbatim}

\subsection{A Note on the ``Compact'' Proof Format}

The present version of Metamath\index{Metamath!limitations of
version 0.07.30} will display proofs in a ``compact''\index{compact proof}
format whenever the proof is stored in compressed format in the database.  It
may be be slightly confusing unless you know how to interpret it.  (A
future version may eliminate this format from displays.)  For example,
if you display the complete proof of theorem \texttt{id1} it will start
off as follows:

\begin{verbatim}
MM> show proof id1 /lemmon/all
 1 wph           $f wff ph
 2 wph           $f wff ph
 3 wph           $f wff ph
 4 2,3 wi    @4: $a wff ( ph -> ph )
 5 1,4 wi    @5: $a wff ( ph -> ( ph -> ph ) )
 6 @4            $a wff ( ph -> ph )
\end{verbatim}

\begin{center}
{etc.}
\end{center}

Step 4 has a ``local label,''\index{local label} \texttt{@4}, assigned to it.
Later on, at step 6, the label \texttt{@1} is referenced instead of
displaying the explicit proof for that step.  This technique takes advantage
of the fact that steps in a proof often repeat, especially during the
construction of wffs.  The compact format reduces the number of steps in the
proof display and may actually be preferred by some people, which is one
reason that it is still displayed by the program.

If you want to see the normal format with the ``true'' step numbers, you can
use the following workaround:\index{\texttt{save proof} command}

\begin{verbatim}
MM> save proof id1 /normal
The proof of "id1" has been reformatted and saved internally.
Remember to use WRITE SOURCE to save it permanently.
MM> show proof id1 /lemmon/all
 1 wph           $f wff ph
 2 wph           $f wff ph
 3 wph           $f wff ph
 4 2,3 wi        $a wff ( ph -> ph )
 5 1,4 wi        $a wff ( ph -> ( ph -> ph ) )
 6 wph           $f wff ph
 7 wph           $f wff ph
 8 6,7 wi        $a wff ( ph -> ph )
\end{verbatim}

\begin{center}
{etc.}
\end{center}

Note that the original 6 steps are now 8 steps.  However, the format is
now the same as that described in Chapter~\ref{using}.

\chapter{The Metamath Language}
\label{languagespec}

\begin{quote}
  {\em Thus mathematics may be defined as the subject in which we never know
what we are talking about, nor whether what we are saying is true.}
    \flushright\sc  Bertrand Russell\footnote{\cite[p.~84]{Russell2}.}\\
\end{quote}\index{Russell, Bertrand}

Probably the most striking feature of the Metamath language is its almost
complete absence of hard-wired syntax. Metamath\index{Metamath} does not
understand any mathematics or logic other than that needed to construct finite
sequences of symbols according to a small set of simple, built-in rules.  The
only rule it uses in a proof is the substitution of an expression (symbol
sequence) for a variable, subject to a simple constraint to prevent
bound-variable clashes.  The primitive notions built into Metamath involve the
simple manipulation of finite objects (symbols) that we as humans can easily
visualize and that computers can easily deal with.  They seem to be just
about the simplest notions possible that are required to do standard
mathematics.

This chapter serves as a reference manual for the Metamath\index{Metamath}
language. It covers the tedious technical details of the language, some of
which you may wish to skim in a first reading.  On the other hand, you should
pay close attention to the defined terms in {\bf boldface}; they have precise
meanings that are important to keep in mind for later understanding.  It may
be best to first become familiar with the examples in Chapter~\ref{using} to
gain some motivation for the language.

%% Uncomment this when uncommenting section {formalspec} below
If you have some knowledge of set theory, you may wish to study this
chapter in conjunction with the formal set-theoretical description of the
Metamath language in Appendix~\ref{formalspec}.

We will use the name ``Metamath''\index{Metamath} to mean either the Metamath
computer language or the Metamath software associated with the computer
language.  We will not distinguish these two when the context is clear.

The next section contains the complete specification of the Metamath
language.
It serves as an
authoritative reference and presents the syntax in enough detail to
write a parser\index{parsing Metamath} and proof verifier.  The
specification is terse and it is probably hard to learn the language
directly from it, but we include it here for those impatient people who
prefer to see everything up front before looking at verbose expository
material.  Later sections explain this material and provide examples.
We will repeat the definitions in those sections, and you may skip the
next section at first reading and proceed to Section~\ref{tut1}
(p.~\pageref{tut1}).

\section{Specification of the Metamath Language}\label{spec}
\index{Metamath!specification}

\begin{quote}
  {\em Sometimes one has to say difficult things, but one ought to say
them as simply as one knows how.}
    \flushright\sc  G. H. Hardy\footnote{As quoted in
    \cite{deMillo}, p.~273.}\\
\end{quote}\index{Hardy, G. H.}

\subsection{Preliminaries}\label{spec1}

% Space is technically a printable character, so we'll word things
% carefully so it's unambiguous.
A Metamath {\bf database}\index{database} is built up from a top-level source
file together with any source files that are brought in through file inclusion
commands (see below).  The only characters that are allowed to appear in a
Metamath source file are the 94 non-whitespace printable {\sc
ascii}\index{ascii@{\sc ascii}} characters, which are digits, upper and lower
case letters, and the following 32 special
characters\index{special characters}:\label{spec1chars}

\begin{verbatim}
! " # $ % & ' ( ) * + , - . / :
; < = > ? @ [ \ ] ^ _ ` { | } ~
\end{verbatim}
plus the following characters which are the ``white space'' characters:
space (a printable character),
tab, carriage return, line feed, and form feed.\label{whitespace}
We will use \texttt{typewriter}
font to display the printable characters.

A Metamath database consists of a sequence of three kinds of {\bf
tokens}\index{token} separated by {\bf white space}\index{white space}
(which is any sequence of one or more white space characters).  The set
of {\bf keyword}\index{keyword} tokens is \texttt{\$\char`\{},
\texttt{\$\char`\}}, \texttt{\$c}, \texttt{\$v}, \texttt{\$f},
\texttt{\$e}, \texttt{\$d}, \texttt{\$a}, \texttt{\$p}, \texttt{\$.},
\texttt{\$=}, \texttt{\$(}, \texttt{\$)}, \texttt{\$[}, and
\texttt{\$]}.  The last four are called {\bf auxiliary}\index{auxiliary
keyword} or preprocessing keywords.  A {\bf label}\index{label} token
consists of any combination of letters, digits, and the characters
hyphen, underscore, and period.  A {\bf math symbol}\index{math symbol}
token may consist of any combination of the 92 printable standard {\sc
ascii} characters other than space or \texttt{\$}~. All tokens are
case-sensitive.

\subsection{Preprocessing}

The token \texttt{\$(} begins a {\bf comment} and
\texttt{\$)} ends a comment.\index{\texttt{\$(}
and \texttt{\$)} auxiliary keywords}\index{comment}
Comments may contain any of
the 94 non-whitespace printable characters and white space,
except they may not contain the
2-character sequences \texttt{\$(} or \texttt{\$)} (comments do not nest).
Comments are ignored (treated
like white space) for the purpose of parsing, e.g.,
\texttt{\$( \$[ \$)} is a comment.
See p.~\pageref{mathcomments} for comment typesetting conventions; these
conventions may be ignored for the purpose of parsing.

A {\bf file inclusion command} consists of \texttt{\$[} followed by a file name
followed by \texttt{\$]}.\index{\texttt{\$[} and \texttt{\$]} auxiliary
keywords}\index{included file}\index{file inclusion}
It is only allowed in the outermost scope (i.e., not between
\texttt{\$\char`\{} and \texttt{\$\char`\}})
and must not be inside a statement (e.g., it may not occur
between the label of a \texttt{\$a} statement and its \texttt{\$.}).
The file name may not
contain a \texttt{\$} or white space.  The file must exist.
The case-sensitivity
of its name follows the conventions of the operating system.  The contents of
the file replace the inclusion command.
Included files may include other files.
Only the first reference to a given file is included; any later
references to the same file (whether in the top-level file or in included
files) cause the inclusion command to be ignored (treated like white space).
A verifier may assume that file names with different strings
refer to different files for the purpose of ignoring later references.
A file self-reference is ignored, as is any reference to the top-level file
(to avoid loops).
Included files may not include a \texttt{\$(} without a matching \texttt{\$)},
may not include a \texttt{\$[} without a matching \texttt{\$]}, and may
not include incomplete statements (e.g., a \texttt{\$a} without a matching
\texttt{\$.}).
It is currently unspecified if path references are relative to the process'
current directory or the file's containing directory, so databases should
avoid using pathname separators (e.g., ``/'') in file names.

Like all tokens, the \texttt{\$(}, \texttt{\$)}, \texttt{\$[}, and \texttt{\$]} keywords
must be surrounded by white space.

\subsection{Basic Syntax}

After preprocessing, a database will consist of a sequence of {\bf
statements}.
These are the scoping statements \texttt{\$\char`\{} and
\texttt{\$\char`\}}, along with the \texttt{\$c}, \texttt{\$v},
\texttt{\$f}, \texttt{\$e}, \texttt{\$d}, \texttt{\$a}, and \texttt{\$p}
statements.

A {\bf scoping statement}\index{scoping statement} consists only of its
keyword, \texttt{\$\char`\{} or \texttt{\$\char`\}}.
A \texttt{\$\char`\{} begins a {\bf
block}\index{block} and a matching \texttt{\$\char`\}} ends the block.
Every \texttt{\$\char`\{}
must have a matching \texttt{\$\char`\}}.
Defining it recursively, we say a block
contains a sequence of zero or more tokens other
than \texttt{\$\char`\{} and \texttt{\$\char`\}} and
possibly other blocks.  There is an {\bf outermost
block}\index{block!outermost} not bracketed by \texttt{\$\char`\{} \texttt{\$\char`\}}; the end
of the outermost block is the end of the database.

% LaTeX bug? can't do \bf\tt

A {\bf \$v} or {\bf \$c statement}\index{\texttt{\$v} statement}\index{\texttt{\$c}
statement} consists of the keyword token \texttt{\$v} or \texttt{\$c} respectively,
followed by one or more math symbols,
% The word "token" is used to distinguish "$." from the sentence-ending period.
followed by the \texttt{\$.}\ token.
These
statements {\bf declare}\index{declaration} the math symbols to be {\bf
variables}\index{variable!Metamath} or {\bf constants}\index{constant}
respectively. The same math symbol may not occur twice in a given \texttt{\$v} or
\texttt{\$c} statement.

%c%A math symbol becomes an {\bf active}\index{active math symbol}
%c%when declared and stays active until the end of the block in which it is
%c%declared.  A math symbol may not be declared a second time while it is active,
%c%but it may be declared again after it becomes inactive.

A math symbol becomes {\bf active}\index{active math symbol} when declared
and stays active until the end of the block in which it is declared.  A
variable may not be declared a second time while it is active, but it
may be declared again (as a variable, but not as a constant) after it
becomes inactive.  A constant must be declared in the outermost block and may
not be declared a second time.\footnote{The rules for redeclaration may
become more general in the future;
see p.~\pageref{redeclarationf}.}\index{redeclaration of symbols}

A {\bf \$f statement}\index{\texttt{\$f} statement} consists of a label,
followed by \texttt{\$f}, followed by its typecode (an active constant),
followed by an
active variable, followed by the \texttt{\$.}\ token.  A {\bf \$e
statement}\index{\texttt{\$e} statement} consists of a label, followed
by \texttt{\$e}, followed by its typecode (an active constant),
followed by zero or more
active math symbols, followed by the \texttt{\$.}\ token.  A {\bf
hypothesis}\index{hypothesis} is a \texttt{\$f} or \texttt{\$e}
statement.
The type declared by a \texttt{\$f} statement for a given label
is global even if the variable is not
(e.g., a database may not have \texttt{wff P} in one local scope
nd \texttt{class P} in another).

A {\bf simple \$d statement}\index{\texttt{\$d} statement!simple}
consists of \texttt{\$d}, followed by two different active variables,
followed by the \texttt{\$.}\ token.  A {\bf compound \$d
statement}\index{\texttt{\$d} statement!compound} consists of
\texttt{\$d}, followed by three or more variables (all different),
followed by the \texttt{\$.}\ token.  The order of the variables in a
\texttt{\$d} statement is unimportant.  A compound \texttt{\$d}
statement is equivalent to a set of simple \texttt{\$d} statements, one
for each possible pair of variables occurring in the compound
\texttt{\$d} statement.  Henceforth in this specification we shall
assume all \texttt{\$d} statements are simple.  A \texttt{\$d} statement
is also called a {\bf disjoint} (or {\bf distinct}) {\bf variable
restriction}.\index{disjoint-variable restriction}

A {\bf \$a statement}\index{\texttt{\$a} statement} consists of a label,
followed by \texttt{\$a}, followed by its typecode (an active constant),
followed by
zero or more active math symbols, followed by the \texttt{\$.}\ token.  A {\bf
\$p statement}\index{\texttt{\$p} statement} consists of a label,
followed by \texttt{\$p}, followed by its typecode (an active constant),
followed by
zero or more active math symbols, followed by \texttt{\$=}, followed by
a sequence of labels, followed by the \texttt{\$.}\ token.  An {\bf
assertion}\index{assertion} is a \texttt{\$a} or \texttt{\$p} statement.

A \texttt{\$f}, \texttt{\$e}, or \texttt{\$d} statement is {\bf active}\index{active
statement} from the place it occurs until the end of the block it occurs in.
A \texttt{\$a} or \texttt{\$p} statement is {\bf active} from the place it occurs
through the end of the database.
There may not be two active \texttt{\$f} statements containing the same
variable.  Each variable in a \texttt{\$e}, \texttt{\$a}, or
\texttt{\$p} statement must exist in an active \texttt{\$f}
statement.\footnote{This requirement can greatly simplify the
unification algorithm (substitution calculation) required by proof
verification.}

%The label that begins each \texttt{\$f}, \texttt{\$e}, \texttt{\$a}, and
%\texttt{\$p} statement must be unique.
Each label token must be unique, and
no label token may match any math symbol
token.\label{namespace}\footnote{This
restriction was added on June 24, 2006.
It is not theoretically necessary but is imposed to make it easier to
write certain parsers.}

The set of {\bf mandatory variables}\index{mandatory variable} associated with
an assertion is the set of (zero or more) variables in the assertion and in any
active \texttt{\$e} statements.  The (possibly empty) set of {\bf mandatory
hypotheses}\index{mandatory hypothesis} is the set of all active \texttt{\$f}
statements containing mandatory variables, together with all active \texttt{\$e}
statements.
The set of {\bf mandatory {\bf \$d} statements}\index{mandatory
disjoint-variable restriction} associated with an assertion are those active
\texttt{\$d} statements whose variables are both among the assertion's
mandatory variables.

\subsection{Proof Verification}\label{spec4}

The sequence of labels between the \texttt{\$=} and \texttt{\$.}\ tokens
in a \texttt{\$p} statement is a {\bf proof}.\index{proof!Metamath} Each
label in a proof must be the label of an active statement other than the
\texttt{\$p} statement itself; thus a label must refer either to an
active hypothesis of the \texttt{\$p} statement or to an earlier
assertion.

An {\bf expression}\index{expression} is a sequence of math symbols. A {\bf
substitution map}\index{substitution map} associates a set of variables with a
set of expressions.  It is acceptable for a variable to be mapped to an
expression containing it.  A {\bf
substitution}\index{substitution!variable}\index{variable substitution} is the
simultaneous replacement of all variables in one or more expressions with the
expressions that the variables map to.

A proof is scanned in order of its label sequence.  If the label refers to an
active hypothesis, the expression in the hypothesis is pushed onto a
stack.\index{stack}\index{RPN stack}  If the label refers to an assertion, a
(unique) substitution must exist that, when made to the mandatory hypotheses
of the referenced assertion, causes them to match the topmost (i.e.\ most
recent) entries of the stack, in order of occurrence of the mandatory
hypotheses, with the topmost stack entry matching the last mandatory
hypothesis of the referenced assertion.  As many stack entries as there are
mandatory hypotheses are then popped from the stack.  The same substitution is
made to the referenced assertion, and the result is pushed onto the stack.
After the last label in the proof is processed, the stack must have a single
entry that matches the expression in the \texttt{\$p} statement containing the
proof.

%c%{\footnotesize\begin{quotation}\index{redeclaration of symbols}
%c%{{\em Comment.}\label{spec4comment} Whenever a math symbol token occurs in a
%c%{\texttt{\$c} or \texttt{\$v} statement, it is considered to designate a distinct new
%c%{symbol, even if the same token was previously declared (and is now inactive).
%c%{Thus a math token declared as a constant in two different blocks is considered
%c%{to designate two distinct constants (even though they have the same name).
%c%{The two constants will not match in a proof that references both blocks.
%c%{However, a proof referencing both blocks is acceptable as long as it doesn't
%c%{require that the constants match.  Similarly, a token declared to be a
%c%{constant for a referenced assertion will not match the same token declared to
%c%{be a variable for the \texttt{\$p} statement containing the proof.  In the case
%c%{of a token declared to be a variable for a referenced assertion, this is not
%c%{an issue since the variable can be substituted with whatever expression is
%c%{needed to achieve the required match.
%c%{\end{quotation}}
%c2%A proof may reference an assertion that contains or whose hypotheses contain a
%c2%constant that is not active for the \texttt{\$p} statement containing the proof.
%c2%However, the final result of the proof may not contain that constant. A proof
%c2%may also reference an assertion that contains or whose hypotheses contain a
%c2%variable that is not active for the \texttt{\$p} statement containing the proof.
%c2%That variable, of course, will be substituted with whatever expression is
%c2%needed to achieve the required match.

A proof may contain a \texttt{?}\ in place of a label to indicate an unknown step
(Section~\ref{unknown}).  A proof verifier may ignore any proof containing
\texttt{?}\ but should warn the user that the proof is incomplete.

A {\bf compressed proof}\index{compressed proof}\index{proof!compressed} is an
alternate proof notation described in Appen\-dix~\ref{compressed}; also see
references to ``compressed proof'' in the Index.  Compressed proofs are a
Metamath language extension which a complete proof verifier should be able to
parse and verify.

\subsubsection{Verifying Disjoint Variable Restrictions}

Each substitution made in a proof must be checked to verify that any
disjoint variable restrictions are satisfied, as follows.

If two variables replaced by a substitution exist in a mandatory \texttt{\$d}
statement\index{\texttt{\$d} statement} of the assertion referenced, the two
expressions resulting from the substitution must satisfy the following
conditions.  First, the two expressions must have no variables in common.
Second, each possible pair of variables, one from each expression, must exist
in an active \texttt{\$d} statement of the \texttt{\$p} statement containing the
proof.

\vskip 1ex

This ends the specification of the Metamath language;
see Appendix \ref{BNF} for its syntax in
Extended Backus--Naur Form (EBNF)\index{Extended Backus--Naur Form}\index{EBNF}.

\section{The Basic Keywords}\label{tut1}

Our expository material begins here.

Like most computer languages, Metamath\index{Metamath} takes its input from
one or more {\bf source files}\index{source file} which contain characters
expressed in the standard {\sc ascii} (American Standard Code for Information
Interchange)\index{ascii@{\sc ascii}} code for computers.  A source file
consists of a series of {\bf tokens}\index{token}, which are strings of
non-whitespace
printable characters (from the set of 94 shown on p.~\pageref{spec1chars})
separated by {\bf white space}\index{white space} (spaces, tabs, carriage
returns, line feeds, and form feeds). Any string consisting only of these
characters is treated the same as a single space.  The non-whitespace printable
characters\index{printable character} that Metamath recognizes are the 94
characters on standard {\sc ascii} keyboards.

Metamath has the ability to join several files together to form its
input (Section~\ref{include}).  We call the aggregate contents of all
the files after they have been joined together a {\bf
database}\index{database} to distinguish it from an individual source
file.  The tokens in a database consist of {\bf
keywords}\index{keyword}, which are built into the language, together
with two kinds of user-defined tokens called {\bf labels}\index{label}
and {\bf math symbols}\index{math symbol}.  (Often we will simply say
{\bf symbol}\index{symbol} instead of math symbol for brevity).  The set
of {\bf basic keywords}\index{basic keyword} is
\texttt{\$c}\index{\texttt{\$c} statement},
\texttt{\$v}\index{\texttt{\$v} statement},
\texttt{\$e}\index{\texttt{\$e} statement},
\texttt{\$f}\index{\texttt{\$f} statement},
\texttt{\$d}\index{\texttt{\$d} statement},
\texttt{\$a}\index{\texttt{\$a} statement},
\texttt{\$p}\index{\texttt{\$p} statement},
\texttt{\$=}\index{\texttt{\$=} keyword},
\texttt{\$.}\index{\texttt{\$.}\ keyword},
\texttt{\$\char`\{}\index{\texttt{\$\char`\{} and \texttt{\$\char`\}}
keywords}, and \texttt{\$\char`\}}.  This is the complete set of
syntactical elements of what we call the {\bf basic
language}\index{basic language} of Metamath, and with them you can
express all of the mathematics that were intended by the design of
Metamath.  You should make it a point to become very familiar with them.
Table~\ref{basickeywords} lists the basic keywords along with a brief
description of their functions.  For now, this description will give you
only a vague notion of what the keywords are for; later we will describe
the keywords in detail.


\begin{table}[htp] \caption{Summary of the basic Metamath
keywords} \label{basickeywords}
\begin{center}
\begin{tabular}{|p{4pc}|l|}
\hline
\em \centering Keyword&\em Description\\
\hline
\hline
\centering
   \texttt{\$c}&Constant symbol declaration\\
\hline
\centering
   \texttt{\$v}&Variable symbol declaration\\
\hline
\centering
   \texttt{\$d}&Disjoint variable restriction\\
\hline
\centering
   \texttt{\$f}&Variable-type (``floating'') hypothesis\\
\hline
\centering
   \texttt{\$e}&Logical (``essential'') hypothesis\\
\hline
\centering
   \texttt{\$a}&Axiomatic assertion\\
\hline
\centering
   \texttt{\$p}&Provable assertion\\
\hline
\centering
   \texttt{\$=}&Start of proof in \texttt{\$p} statement\\
\hline
\centering
   \texttt{\$.}&End of the above statement types\\
\hline
\centering
   \texttt{\$\char`\{}&Start of block\\
\hline
\centering
   \texttt{\$\char`\}}&End of block\\
\hline
\end{tabular}
\end{center}
\end{table}

%For LaTeX bug(?) where it puts tables on blank page instead of btwn text
%May have to adjust if text changes
%\newpage

There are some additional keywords, called {\bf auxiliary
keywords}\index{auxiliary keyword} that help make Metamath\index{Metamath}
more practical. These are part of the {\bf extended language}\index{extended
language}. They provide you with a means to put comments into a Metamath
source file\index{source file} and reference other source files.  We will
introduce these in later sections. Table~\ref{otherkeywords} summarizes them
so that you can recognize them now if you want to peruse some source
files while learning the basic keywords.


\begin{table}[htp] \caption{Auxiliary Metamath
keywords} \label{otherkeywords}
\begin{center}
\begin{tabular}{|p{4pc}|l|}
\hline
\em \centering Keyword&\em Description\\
\hline
\hline
\centering
   \texttt{\$(}&Start of comment\\
\hline
\centering
   \texttt{\$)}&End of comment\\
\hline
\centering
   \texttt{\$[}&Start of included source file name\\
\hline
\centering
   \texttt{\$]}&End of included source file name\\
\hline
\end{tabular}
\end{center}
\end{table}
\index{\texttt{\$(} and \texttt{\$)} auxiliary keywords}
\index{\texttt{\$[} and \texttt{\$]} auxiliary keywords}


Unlike those in some computer languages, the keywords\index{keyword} are short
two-character sequences rather than English-like words.  While this may make
them slightly more difficult to remember at first, their brevity allows
them to blend in with the mathematics being described, not
distract from it, like punctuation marks.


\subsection{User-Defined Tokens}\label{dollardollar}\index{token}

As you may have noticed, all keywords\index{keyword} begin with the \texttt{\$}
character.  This mundane monetary symbol is not ordinarily used in higher
mathematics (outside of grant proposals), so we have appropriated it to
distinguish the Metamath\index{Metamath} keywords from ordinary mathematical
symbols. The \texttt{\$} character is thus considered special and may not be
used as a character in a user-defined token.  All tokens and keywords are
case-sensitive; for example, \texttt{n} is considered to be a different character
from \texttt{N}.  Case-sensitivity makes the available {\sc ascii} character set
as rich as possible.

\subsubsection{Math Symbol Tokens}\index{token}

Math symbols\index{math symbol} are tokens used to represent the symbols
that appear in ordinary mathematical formulas.  They may consist of any
combination of the 93 non-whitespace printable {\sc ascii} characters other than
\texttt{\$}~. Some examples are \texttt{x}, \texttt{+}, \texttt{(},
\texttt{|-}, \verb$!%@?&$, and \texttt{bounded}.  For readability, it is
best to try to make these look as similar to actual mathematical symbols
as possible, within the constraints of the {\sc ascii} character set, in
order to make the resulting mathematical expressions more readable.

In the Metamath\index{Metamath} language, you express ordinary
mathematical formulas and statements as sequences of math symbols such
as \texttt{2 + 2 = 4} (five symbols, all constants).\footnote{To
eliminate ambiguity with other expressions, this is expressed in the set
theory database \texttt{set.mm} as \texttt{|- ( 2 + 2
 ) = 4 }, whose \LaTeX\ equivalent is $\vdash
(2+2)=4$.  The \,$\vdash$ means ``is a theorem'' and the
parentheses allow explicit associative grouping.}\index{turnstile
({$\,\vdash$})} They may even be English
sentences, as in \texttt{E is closed and bounded} (five symbols)---here
\texttt{E} would be a variable and the other four symbols constants.  In
principle, a Metamath database could be constructed to work with almost
any unambiguous English-language mathematical statement, but as a
practical matter the definitions needed to provide for all possible
syntax variations would be cumbersome and distracting and possibly have
subtle pitfalls accidentally built in.  We generally recommend that you
express mathematical statements with compact standard mathematical
symbols whenever possible and put their English-language descriptions in
comments.  Axioms\index{axiom} and definitions\index{definition}
(\texttt{\$a}\index{\texttt{\$a} statement} statements) are the only
places where Metamath will not detect an error, and doing this will help
reduce the number of definitions needed.

You are free to use any tokens\index{token} you like for math
symbols\index{math symbol}.  Appendix~\ref{ASCII} recommends token names to
use for symbols in set theory, and we suggest you adopt these in order to be
able to include the \texttt{set.mm} set theory database in your database.  For
printouts, you can convert the tokens in a database
to standard mathematical symbols with the \LaTeX\ typesetting program.  The
Metamath command \texttt{open tex} {\em filename}\index{\texttt{open tex} command}
produces output that can be read by \LaTeX.\index{latex@{\LaTeX}}
The correspondence
between tokens and the actual symbols is made by \texttt{latexdef}
statements inside a special database comment tagged
with \texttt{\$t}.\index{\texttt{\$t} comment}\index{typesetting comment}
  You can edit
this comment to change the definitions or add new ones.
Appendix~\ref{ASCII} describes how to do this in more detail.

% White space\index{white space} is normally used to separate math
% symbol\index{math symbol} tokens, but they may be juxtaposed without white
% space in \texttt{\$d}\index{\texttt{\$d} statement}, \texttt{\$e}\index{\texttt{\$e}
% statement}, \texttt{\$f}\index{\texttt{\$f} statement}, \texttt{\$a}\index{\texttt{\$a}
% statement}, and \texttt{\$p}\index{\texttt{\$p} statement} statements when no
% ambiguity will result.  Specifically, Metamath parses the math symbol sequence
% in one of these statements in the following manner:  when the math symbol
% sequence has been broken up into tokens\index{token} up to a given character,
% the next token is the longest string of characters that could constitute a
% math symbol that is active\index{active
% math symbol} at that point.  (See Section~\ref{scoping} for the
% definition of an active math symbol.)  For example, if \texttt{-}, \texttt{>}, and
% \texttt{->} are the only active math symbols, the juxtaposition \texttt{>-} will be
% interpreted as the two symbols \texttt{>} and \texttt{-}, whereas \texttt{->} will
% always be interpreted as that single symbol.\footnote{For better readability we
% recommend a white space between each token.  This also makes searching for a
% symbol easier to do with an editor.  Omission of optional white space is useful
% for reducing typing when assigning an expression to a temporary
% variable\index{temporary variable} with the \texttt{let variable} Metamath
% program command.}\index{\texttt{let variable} command}
%
% Keywords\index{keyword} may be placed next to math symbols without white
% space\index{white space} between them.\footnote{Again, we do not recommend
% this for readability.}
%
% The math symbols\index{math symbol} in \texttt{\$c}\index{\texttt{\$c} statement}
% and \texttt{\$v}\index{\texttt{\$v} statement} statements must always be separated
% by white space\index{white
% space}, for the obvious reason that these statements define the names
% of the symbols.
%
% Math symbols referred to in comments (see Section~\ref{comments}) must also be
% separated by white space.  This allows you to make comments about symbols that
% are not yet active\index{active
% math symbol}.  (The ``math mode'' feature of comments is also a quick and
% easy way to obtain word processing text with embedded mathematical symbols,
% independently of the main purpose of Metamath; the way to do this is described
% in Section~\ref{comments})

\subsubsection{Label Tokens}\index{token}\index{label}

Label tokens are used to identify Metamath\index{Metamath} statements for
later reference. Label tokens may contain only letters, digits, and the three
characters period, hyphen, and underscore:
\begin{verbatim}
. - _
\end{verbatim}

A label is {\bf declared}\index{label declaration} by placing it immediately
before the keyword of the statement it identifies.  For example, the label
\texttt{axiom.1} might be declared as follows:
\begin{verbatim}
axiom.1 $a |- x = x $.
\end{verbatim}

Each \texttt{\$e}\index{\texttt{\$e} statement},
\texttt{\$f}\index{\texttt{\$f} statement},
\texttt{\$a}\index{\texttt{\$a} statement}, and
\texttt{\$p}\index{\texttt{\$p} statement} statement in a database must
have a label declared for it.  No other statement types may have label
declarations.  Every label must be unique.

A label (and the statement it identifies) is {\bf referenced}\index{label
reference} by including the label between the \texttt{\$=}\index{\texttt{\$=}
keyword} and \texttt{\$.}\index{\texttt{\$.}\ keyword}\ keywords in a \texttt{\$p}
statement.  The sequence of labels\index{label sequence} between these two
keywords is called a {\bf proof}\index{proof}.  An example of a statement with
a proof that we will encounter later (Section~\ref{proof}) is
\begin{verbatim}
wnew $p wff ( s -> ( r -> p ) )
     $= ws wr wp w2 w2 $.
\end{verbatim}

You don't have to know what this means just yet, but you should know that the
label \texttt{wnew} is declared by this \texttt{\$p} statement and that the labels
\texttt{ws}, \texttt{wr}, \texttt{wp}, and \texttt{w2} are assumed to have been declared
earlier in the database and are referenced here.

\subsection{Constants and Variables}
\index{constant}
\index{variable}

An {\bf expression}\index{expression} is any sequence of math
symbols, possibly empty.

The basic Metamath\index{Metamath} language\index{basic language} has two
kinds of math symbols\index{math symbol}:  {\bf constants}\index{constant} and
{\bf variables}\index{variable}.  In a Metamath proof, a constant may not be
substituted with any expression.  A variable can be
substituted\index{substitution!variable}\index{variable substitution} with any
expression.  This sequence may include other variables and may even include
the variable being substituted.  This substitution takes place when proofs are
verified, and it will be described in Section~\ref{proof}.  The \texttt{\$f}
statement (described later in Section~\ref{dollaref}) is used to specify the
{\bf type} of a variable (i.e.\ what kind of
variable it is)\index{variable type}\index{type} and
give it a meaning typically
associated with a ``metavariable''\index{metavariable}\footnote{A metavariable
is a variable that ranges over the syntactical elements of the object language
being discussed; for example, one metavariable might represent a variable of
the object language and another metavariable might represent a formula in the
object language.} in ordinary mathematics; for example, a variable may be
specified to be a wff or well-formed formula (in logic), a set (in set
theory), or a non-negative integer (in number theory).

%\subsection{The \texttt{\$c} and \texttt{\$v} Declaration Statements}
\subsection{The \texttt{\$c} and \texttt{\$v} Declaration Statements}
\index{\texttt{\$c} statement}
\index{constant declaration}
\index{\texttt{\$v} statement}
\index{variable declaration}

Constants are introduced or {\bf declared}\index{constant declaration}
with \texttt{\$c}\index{\texttt{\$c} statement} statements, and
variables are declared\index{variable declaration} with
\texttt{\$v}\index{\texttt{\$v} statement} statements.  A {\bf simple}
declaration\index{simple declaration} statement introduces a single
constant or variable.  Its syntax is one of the following:
\begin{center}
  \texttt{\$c} {\em math-symbol} \texttt{\$.}\\
  \texttt{\$v} {\em math-symbol} \texttt{\$.}
\end{center}
The notation {\em math-symbol} means any math symbol token\index{token}.

Some examples of simple declaration statements are:
\begin{center}
  \texttt{\$c + \$.}\\
  \texttt{\$c -> \$.}\\
  \texttt{\$c ( \$.}\\
  \texttt{\$v x \$.}\\
  \texttt{\$v y2 \$.}
\end{center}

The characters in a math symbol\index{math symbol} being declared are
irrelevant to Meta\-math; for example, we could declare a right parenthesis to
be a variable,
\begin{center}
  \texttt{\$v ) \$.}\\
\end{center}
although this would be unconventional.

A {\bf compound} declaration\index{compound declaration} statement is a
shorthand for declaring several symbols at once.  Its syntax is one of the
following:
\begin{center}
  \texttt{\$c} {\em math-symbol}\ \,$\cdots$\ {\em math-symbol} \texttt{\$.}\\
  \texttt{\$v} {\em math-symbol}\ \,$\cdots$\ {\em math-symbol} \texttt{\$.}
\end{center}\index{\texttt{\$c} statement}
Here, the ellipsis (\ldots) means any number of {\em math-symbol}\,s.

An example of a compound declaration statement is:
\begin{center}
  \texttt{\$v x y mu \$.}\\
\end{center}
This is equivalent to the three simple declaration statements
\begin{center}
  \texttt{\$v x \$.}\\
  \texttt{\$v y \$.}\\
  \texttt{\$v mu \$.}\\
\end{center}
\index{\texttt{\$v} statement}

There are certain rules on where in the database math symbols may be declared,
what sections of the database are aware of them (i.e.\ where they are
``active''), and when they may be declared more than once.  These will be
discussed in Section~\ref{scoping} and specifically on
p.~\pageref{redeclaration}.

\subsection{The \texttt{\$d} Statement}\label{dollard}
\index{\texttt{\$d} statement}

The \texttt{\$d} statement is called a {\bf disjoint-variable restriction}.  The
syntax of the {\bf simple} version of this statement is
\begin{center}
  \texttt{\$d} {\em variable variable} \texttt{\$.}
\end{center}
where each {\em variable} is a previously declared variable and the two {\em
variable}\,s are different.  (More specifically, each  {\em variable} must be
an {\bf active} variable\index{active math symbol}, which means there must be
a previous \texttt{\$v} statement whose {\bf scope}\index{scope} includes the
\texttt{\$d} statement.  These terms will be defined when we discuss scoping
statements in Section~\ref{scoping}.)

In ordinary mathematics, formulas may arise that are true if the variables in
them are distinct\index{distinct variables}, but become false when those
variables are made identical. For example, the formula in logic $\exists x\,x
\neq y$, which means ``for a given $y$, there exists an $x$ that is not equal
to $y$,'' is true in most mathematical theories (namely all non-trivial
theories\index{non-trivial theory}, i.e.\ those that describe more than one
individual, such as arithmetic).  However, if we substitute $y$ with $x$, we
obtain $\exists x\,x \neq x$, which is always false, as it means ``there
exists something that is not equal to itself.''\footnote{If you are a
logician, you will recognize this as the improper substitution\index{proper
substitution}\index{substitution!proper} of a free variable\index{free
variable} with a bound variable\index{bound variable}.  Metamath makes no
inherent distinction between free and bound variables; instead, you let
Metamath know what substitutions are permissible by using \texttt{\$d} statements
in the right way in your axiom system.}\index{free vs.\ bound variable}  The
\texttt{\$d} statement allows you to specify a restriction that forbids the
substitution of one variable with another.  In
this case, we would use the statement
\begin{center}
  \texttt{\$d x y \$.}
\end{center}\index{\texttt{\$d} statement}
to specify this restriction.

The order in which the variables appear in a \texttt{\$d} statement is not
important.  We could also use
\begin{center}
  \texttt{\$d y x \$.}
\end{center}

The \texttt{\$d} statement is actually more general than this, as the
``disjoint''\index{disjoint variables} in its name suggests.  The full meaning
is that if any substitution is made to its two variables (during the
course of a proof that references a \texttt{\$a} or \texttt{\$p} statement
associated with the \texttt{\$d}), the two expressions that result from the
substitution must have no variables in common.  In addition, each possible
pair of variables, one from each expression, must be in a \texttt{\$d} statement
associated with the statement being proved.  (This requirement forces the
statement being proved to ``inherit'' the original disjoint variable
restriction.)

For example, suppose \texttt{u} is a variable.  If the restriction
\begin{center}
  \texttt{\$d A B \$.}
\end{center}
has been specified for a theorem referenced in a
proof, we may not substitute \texttt{A} with \mbox{\tt a + u} and
\texttt{B} with \mbox{\tt b + u} because these two symbol sequences have the
variable \texttt{u} in common.  Furthermore, if \texttt{a} and \texttt{b} are
variables, we may not substitute \texttt{A} with \texttt{a} and \texttt{B} with \texttt{b}
unless we have also specified \texttt{\$d a b} for the theorem being proved; in
other words, the \texttt{\$d} property associated with a pair of variables must
be effectively preserved after substitution.

The \texttt{\$d}\index{\texttt{\$d} statement} statement does {\em not} mean ``the
two variables may not be substituted with the same thing,'' as you might think
at first.  For example, substituting each of \texttt{A} and \texttt{B} in the above
example with identical symbol sequences consisting only of constants does not
cause a disjoint variable conflict, because two symbol sequences have no
variables in common (since they have no variables, period).  Similarly, a
conflict will not occur by substituting the two variables in a \texttt{\$d}
statement with the empty symbol sequence\index{empty substitution}.

The \texttt{\$d} statement does not have a direct counterpart in
ordinary mathematics, partly because the variables\index{variable} of
Metamath are not really the same as the variables\index{variable!in
ordinary mathematics} of ordinary mathematics but rather are
metavariables\index{metavariable} ranging over them (as well as over
other kinds of symbols and groups of symbols).  Depending on the
situation, we may informally interpret the \texttt{\$d} statement in
different ways.  Suppose, for example, that \texttt{x} and \texttt{y}
are variables ranging over numbers (more precisely, that \texttt{x} and
\texttt{y} are metavariables ranging over variables that range over
numbers), and that \texttt{ph} ($\varphi$) and \texttt{ps} ($\psi$) are
variables (more precisely, metavariables) ranging over formulas.  We can
make the following interpretations that correspond to the informal
language of ordinary mathematics:
\begin{quote}
\begin{tabbing}
\texttt{\$d x y \$.} means ``assume $x$ and $y$ are
distinct variables.''\\
\texttt{\$d x ph \$.} means ``assume $x$ does not
occur in $\varphi$.''\\
\texttt{\$d ph ps \$.} \=means ``assume $\varphi$ and
$\psi$ have no variables\\ \>in common.''
\end{tabbing}
\end{quote}\index{\texttt{\$d} statement}

\subsubsection{Compound \texttt{\$d} Statements}

The {\bf compound} version of the \texttt{\$d} statement is a shorthand for
specifying several variables whose substitutions must be pairwise disjoint.
Its syntax is:
\begin{center}
  \texttt{\$d} {\em variable}\ \,$\cdots$\ {\em variable} \texttt{\$.}
\end{center}\index{\texttt{\$d} statement}
Here, {\em variable} represents the token of a previously declared
variable (specifically, an active variable) and all {\em variable}\,s are
different.  The compound \texttt{\$d}
statement is internally broken up by Metamath into one simple \texttt{\$d}
statement for each possible pair of variables in the original \texttt{\$d}
statement.  For example,
\begin{center}
  \texttt{\$d w x y z \$.}
\end{center}
is equivalent to
\begin{center}
  \texttt{\$d w x \$.}\\
  \texttt{\$d w y \$.}\\
  \texttt{\$d w z \$.}\\
  \texttt{\$d x y \$.}\\
  \texttt{\$d x z \$.}\\
  \texttt{\$d y z \$.}
\end{center}

Two or more simple \texttt{\$d} statements specifying the same variable pair are
internally combined into a single \texttt{\$d} statement.  Thus the set of three
statements
\begin{center}
  \texttt{\$d x y \$.}
  \texttt{\$d x y \$.}
  \texttt{\$d y x \$.}
\end{center}
is equivalent to
\begin{center}
  \texttt{\$d x y \$.}
\end{center}

Similarly, compound \texttt{\$d} statements, after being internally broken up,
internally have their common variable pairs combined.  For example the
set of statements
\begin{center}
  \texttt{\$d x y A \$.}
  \texttt{\$d x y B \$.}
\end{center}
is equivalent to
\begin{center}
  \texttt{\$d x y \$.}
  \texttt{\$d x A \$.}
  \texttt{\$d y A \$.}
  \texttt{\$d x y \$.}
  \texttt{\$d x B \$.}
  \texttt{\$d y B \$.}
\end{center}
which is equivalent to
\begin{center}
  \texttt{\$d x y \$.}
  \texttt{\$d x A \$.}
  \texttt{\$d y A \$.}
  \texttt{\$d x B \$.}
  \texttt{\$d y B \$.}
\end{center}

Metamath\index{Metamath} automatically verifies that all \texttt{\$d}
restrictions are met whenever it verifies proofs.  \texttt{\$d} statements are
never referenced directly in proofs (this is why they do not have
labels\index{label}), but Metamath is always aware of which ones must be
satisfied (i.e.\ are active) and will notify you with an error message if any
violation occurs.

To illustrate how Metamath detects a missing \texttt{\$d}
statement, we will look at the following example from the
\texttt{set.mm} database.

\begin{verbatim}
$d x z $.  $d y z $.
$( Theorem to add distinct quantifier to atomic formula. $)
ax17eq $p |- ( x = y -> A. z x = y ) $=...
\end{verbatim}

This statement has the obvious requirement that $z$ must be
distinct\index{distinct variables} from $x$ in theorem \texttt{ax17eq} that
states $x=y \rightarrow \forall z \, x=y$ (well, obvious if you're a logician,
for otherwise we could conclude  $x=y \rightarrow \forall x \, x=y$, which is
false when the free variables $x$ and $y$ are equal.).

Let's look at what happens if we edit the database to comment out this
requirement.

\begin{verbatim}
$( $d x z $. $) $d y z $.
$( Theorem to add distinct quantifier to atomic formula. $)
ax17eq $p |- ( x = y -> A. z x = y ) $=...
\end{verbatim}

When it tries to verify the proof, Metamath will tell you that \texttt{x} and
\texttt{z} must be disjoint, because one of its steps references an axiom or
theorem that has this requirement.

\begin{verbatim}
MM> verify proof ax17eq
ax17eq ?Error at statement 1918, label "ax17eq", type "$p":
      vz wal wi vx vy vz ax-13 vx vy weq vz vx ax-c16 vx vy
                                               ^^^^^
There is a disjoint variable ($d) violation at proof step 29.
Assertion "ax-c16" requires that variables "x" and "y" be
disjoint.  But "x" was substituted with "z" and "y" was
substituted with "x".  The assertion being proved, "ax17eq",
does not require that variables "z" and "x" be disjoint.
\end{verbatim}

We can see the substitutions into \texttt{ax-c16} with the following command.

\begin{verbatim}
MM> show proof ax17eq / detailed_step 29
Proof step 29:  pm2.61dd.2=ax-c16 $a |- ( A. z z = x -> ( x =
  y -> A. z x = y ) )
This step assigns source "ax-c16" ($a) to target "pm2.61dd.2"
($e).  The source assertion requires the hypotheses "wph"
($f, step 26), "vx" ($f, step 27), and "vy" ($f, step 28).
The parent assertion of the target hypothesis is "pm2.61dd"
($p, step 36).
The source assertion before substitution was:
    ax-c16 $a |- ( A. x x = y -> ( ph -> A. x ph ) )
The following substitutions were made to the source
assertion:
    Variable  Substituted with
     x         z
     y         x
     ph        x = y
The target hypothesis before substitution was:
    pm2.61dd.2 $e |- ( ph -> ch )
The following substitutions were made to the target
hypothesis:
    Variable  Substituted with
     ph        A. z z = x
     ch        ( x = y -> A. z x = y )
\end{verbatim}

The disjoint variable restrictions of \texttt{ax-c16} can be seen from the
\texttt{show state\-ment} command.  The line that begins ``\texttt{Its mandatory
dis\-joint var\-i\-able pairs are:}\ldots'' lists any \texttt{\$d} variable
pairs in brackets.

\begin{verbatim}
MM> show statement ax-c16/full
Statement 3033 is located on line 9338 of the file "set.mm".
"Axiom of Distinct Variables. ..."
  ax-c16 $a |- ( A. x x = y -> ( ph -> A. x ph ) ) $.
Its mandatory hypotheses in RPN order are:
  wph $f wff ph $.
  vx $f set x $.
  vy $f set y $.
Its mandatory disjoint variable pairs are:  <x,y>
The statement and its hypotheses require the variables:  x y
      ph
The variables it contains are:  x y ph
\end{verbatim}

Since Metamath will always detect when \texttt{\$d}\index{\texttt{\$d} statement}
statements are needed for a proof, you don't have to worry too much about
forgetting to put one in; it can always be added if you see the error message
above.  If you put in unnecessary \texttt{\$d} statements, the worst that could
happen is that your theorem might not be as general as it could be, and this
may limit its use later on.

On the other hand, when you introduce axioms (\texttt{\$a}\index{\texttt{\$a}
statement} statements), you must be very careful to properly specify the
necessary associated \texttt{\$d} statements since Metamath has no way of knowing
whether your axioms are correct.  For example, Metamath would have no idea
that \texttt{ax-c16}, which we are telling it is an axiom of logic, would lead to
contradictions if we omitted its associated \texttt{\$d} statement.

% This was previously a comment in footnote-sized type, but it can be
% hard to read this much text in a small size.
% As a result, it's been changed to normally-sized text.
\label{nodd}
You may wonder if it is possible to develop standard
mathematics in the Metamath language without the \texttt{\$d}\index{\texttt{\$d}
statement} statement, since it seems like a nuisance that complicates proof
verification. The \texttt{\$d} statement is not needed in certain subsets of
mathematics such as propositional calculus.  However, dummy
variables\index{dummy variable!eliminating} and their associated \texttt{\$d}
statements are impossible to avoid in proofs in standard first-order logic as
well as in the variant used in \texttt{set.mm}.  In fact, there is no upper bound to
the number of dummy variables that might be needed in a proof of a theorem of
first-order logic containing 3 or more variables, as shown by H.\
Andr\'{e}ka\index{Andr{\'{e}}ka, H.} \cite{Nemeti}.  A first-order system that
avoids them entirely is given in \cite{Megill}\index{Megill, Norman}; the
trick there is simply to embed harmlessly the necessary dummy variables into a
theorem being proved so that they aren't ``dummy'' anymore, then interpret the
resulting longer theorem so as to ignore the embedded dummy variables.  If
this interests you, the system in \texttt{set.mm} obtained from \texttt{ax-1}
through \texttt{ax-c14} in \texttt{set.mm}, and deleting \texttt{ax-c16} and \texttt{ax-5},
requires no \texttt{\$d} statements but is logically complete in the sense
described in \cite{Megill}.  This means it can prove any theorem of
first-order logic as long as we add to the theorem an antecedent that embeds
dummy and any other variables that must be distinct.  In a similar fashion,
axioms for set theory can be devised that
do not require distinct variable
provisos\index{Set theory without distinct variable provisos},
as explained at
\url{http://us.metamath.org/mpeuni/mmzfcnd.html}.
Together, these in principle allow all of
mathematics to be developed under Metamath without a \texttt{\$d} statement,
although the length of the resulting theorems will grow as more and
more dummy variables become required in their proofs.

\subsection{The \texttt{\$f}
and \texttt{\$e} Statements}\label{dollaref}
\index{\texttt{\$e} statement}
\index{\texttt{\$f} statement}
\index{floating hypothesis}
\index{essential hypothesis}
\index{variable-type hypothesis}
\index{logical hypothesis}
\index{hypothesis}

Metamath has two kinds of hypo\-theses, the \texttt{\$f}\index{\texttt{\$f}
statement} or {\bf variable-type} hypothesis and the \texttt{\$e} or {\bf logical}
hypo\-the\-sis.\index{\texttt{\$d} statement}\footnote{Strictly speaking, the
\texttt{\$d} statement is also a hypothesis, but it is never directly referenced
in a proof, so we call it a restriction rather than a hypothesis to lessen
confusion.  The checking for violations of \texttt{\$d} restrictions is automatic
and built into Metamath's proof-checking algorithm.} The letters \texttt{f} and
\texttt{e} stand for ``floating''\index{floating hypothesis} (roughly meaning
used only if relevant) and ``essential''\index{essential hypothesis} (meaning
always used) respectively, for reasons that will become apparent
when we discuss frames in
Section~\ref{frames} and scoping in Section~\ref{scoping}. The syntax of these
are as follows:
\begin{center}
  {\em label} \texttt{\$f} {\em typecode} {\em variable} \texttt{\$.}\\
  {\em label} \texttt{\$e} {\em typecode}
      {\em math-symbol}\ \,$\cdots$\ {\em math-symbol} \texttt{\$.}\\
\end{center}
\index{\texttt{\$e} statement}
\index{\texttt{\$f} statement}
A hypothesis must have a {\em label}\index{label}.  The expression in a
\texttt{\$e} hypothesis consists of a typecode (an active constant math symbol)
followed by a sequence
of zero or more math symbols. Each math symbol (including {\em constant}
and {\em variable}) must be a previously declared constant or variable.  (In
addition, each math symbol must be active, which will be covered when we
discuss scoping statements in Section~\ref{scoping}.)  You use a \texttt{\$f}
hypothesis to specify the
nature or {\bf type}\index{variable type}\index{type} of a variable (such as ``let $x$ be an
integer'') and use a \texttt{\$e} hypothesis to express a logical truth (such as
``assume $x$ is prime'') that must be established in order for an assertion
requiring it to also be true.

A variable must have its type specified in a \texttt{\$f} statement before
it may be used in a \texttt{\$e}, \texttt{\$a}, or \texttt{\$p}
statement.  There may be only one (active) \texttt{\$f} statement for a
given variable.  (``Active'' is defined in Section~\ref{scoping}.)

In ordinary mathematics, theorems\index{theorem} are often expressed in the
form ``Assume $P$; then $Q$,'' where $Q$ is a statement that you can derive
if you start with statement $P$.\index{free variable}\footnote{A stronger
version of a theorem like this would be the {\em single} formula $P\rightarrow
Q$ ($P$ implies $Q$) from which the weaker version above follows by the rule
of modus ponens in logic.  We are not discussing this stronger form here.  In
the weaker form, we are saying only that if we can {\em prove} $P$, then we can
{\em prove} $Q$.  In a logician's language, if $x$ is the only free variable
in $P$ and $Q$, the stronger form is equivalent to $\forall x ( P \rightarrow
Q)$ (for all $x$, $P$ implies $Q$), whereas the weaker form is equivalent to
$\forall x P \rightarrow \forall x Q$. The stronger form implies the weaker,
but not vice-versa.  To be precise, the weaker form of the theorem is more
properly called an ``inference'' rather than a theorem.}\index{inference}
In the
Metamath\index{Metamath} language, you would express mathematical statement
$P$ as a hypothesis (a \texttt{\$e} Metamath language statement in this case) and
statement $Q$ as a provable assertion (a \texttt{\$p}\index{\texttt{\$p} statement}
statement).

Some examples of hypotheses you might encounter in logic and set theory are
\begin{center}
  \texttt{stmt1 \$f wff P \$.}\\
  \texttt{stmt2 \$f set x \$.}\\
  \texttt{stmt3 \$e |- ( P -> Q ) \$.}
\end{center}
\index{\texttt{\$e} statement}
\index{\texttt{\$f} statement}
Informally, these would be read, ``Let $P$ be a well-formed-formula,'' ``Let
$x$ be an (individual) variable,'' and ``Assume we have proved $P \rightarrow
Q$.''  The turnstile symbol \,$\vdash$\index{turnstile ({$\,\vdash$})} is
commonly used in logic texts to mean ``a proof exists for.''

To summarize:
\begin{itemize}
\item A \texttt{\$f} hypothesis tells Metamath the type or kind of its variable.
It is analogous to a variable declaration in a computer language that
tells the compiler that a variable is an integer or a floating-point
number.
\item The \texttt{\$e} hypothesis corresponds to what you would usually call a
``hypothesis'' in ordinary mathematics.
\end{itemize}

Before an assertion\index{assertion} (\texttt{\$a} or \texttt{\$p} statement) can be
referenced in a proof, all of its associated \texttt{\$f} and \texttt{\$e} hypotheses
(i.e.\ those \texttt{\$e} hypotheses that are active) must be satisfied (i.e.
established by the proof).  The meaning of ``associated'' (which we will call
{\bf mandatory} in Section~\ref{frames}) will become clear when we discuss
scoping later.

Note that after any \texttt{\$f}, \texttt{\$e},
\texttt{\$a}, or \texttt{\$p} token there is a required
\textit{typecode}\index{typecode}.
The typecode is a constant used to enforce types of expressions.
This will become clearer once we learn more about
assertions (\texttt{\$a} and \texttt{\$p} statements).
An example may also clarify their purpose.
In the
\texttt{set.mm}\index{set theory database (\texttt{set.mm})}%
\index{Metamath Proof Explorer}
database,
the following typecodes are used:

\begin{itemize}
\item \texttt{wff} :
  Well-formed formula (wff) symbol
  (read: ``the following symbol sequence is a wff'').
% The *textual* typecode for turnstile is "|-", but when read it's a little
% confusing, so I intentionally display the mathematical symbol here instead
% (I think it's clearer in this context).
\item \texttt{$\vdash$} :
  Turnstile (read: ``the following symbol sequence is provable'' or
  ``a proof exists for'').
\item \texttt{setvar} :
  Individual set variable type (read: ``the following is an
  individual set variable'').
  Note that this is \textit{not} the type of an arbitrary set expression,
  instead, it is used to ensure that there is only a single symbol used
  after quantifiers like for-all ($\forall$) and there-exists ($\exists$).
\item \texttt{class} :
  An expression that is a syntactically valid class expression.
  All valid set expressions are also valid class expression, so expressions
  of sets normally have the \texttt{class} typecode.
  Use the \texttt{class} typecode,
  \textit{not} the \texttt{setvar} typecode,
  for the type of set expressions unless you are specifically identifying
  a single set variable.
\end{itemize}

\subsection{Assertions (\texttt{\$a} and \texttt{\$p} Statements)}
\index{\texttt{\$a} statement}
\index{\texttt{\$p} statement}\index{assertion}\index{axiomatic assertion}
\index{provable assertion}

There are two types of assertions, \texttt{\$a}\index{\texttt{\$a} statement}
statements ({\bf axiomatic assertions}) and \texttt{\$p} statements ({\bf
provable assertions}).  Their syntax is as follows:
\begin{center}
  {\em label} \texttt{\$a} {\em typecode} {\em math-symbol} \ldots
         {\em math-symbol} \texttt{\$.}\\
  {\em label} \texttt{\$p} {\em typecode} {\em math-symbol} \ldots
        {\em math-symbol} \texttt{\$=} {\em proof} \texttt{\$.}
\end{center}
\index{\texttt{\$a} statement}
\index{\texttt{\$p} statement}
\index{\texttt{\$=} keyword}
An assertion always requires a {\em label}\index{label}. The expression in an
assertion consists of a typecode (an active constant)
followed by a sequence of zero
or more math symbols.  Each math symbol, including any {\em constant}, must be a
previously declared constant or variable.  (In addition, each math symbol
must be active, which will be covered when we discuss scoping statements in
Section~\ref{scoping}.)

A \texttt{\$a} statement is usually a definition of syntax (for example, if $P$
and $Q$ are wffs then so is $(P\to Q)$), an axiom\index{axiom} of ordinary
mathematics (for example, $x=x$), or a definition\index{definition} of
ordinary mathematics (for example, $x\ne y$ means $\lnot x=y$). A \texttt{\$p}
statement is a claim that a certain combination of math symbols follows from
previous assertions and is accompanied by a proof that demonstrates it.

Assertions can also be referenced in (later) proofs in order to derive new
assertions from them. The label of an assertion is used to refer to it in a
proof. Section~\ref{proof} will describe the proof in detail.

Assertions also provide the primary means for communicating the mathematical
results in the database to people.  Proofs (when conveniently displayed)
communicate to people how the results were arrived at.

\subsubsection{The \texttt{\$a} Statement}
\index{\texttt{\$a} statement}

Axiomatic assertions (\texttt{\$a} statements) represent the starting points from
which other assertions (\texttt{\$p}\index{\texttt{\$p} statement} statements) are
derived.  Their most obvious use is for specifying ordinary mathematical
axioms\index{axiom}, but they are also used for two other purposes.

First, Metamath\index{Metamath} needs to know the syntax of symbol
sequences that constitute valid mathematical statements.  A Metamath
proof must be broken down into much more detail than ordinary
mathematical proofs that you may be used to thinking of (even the
``complete'' proofs of formal logic\index{formal logic}).  This is one
of the things that makes Metamath a general-purpose language,
independent of any system of logic or even syntax.  If you want to use a
substitution instance of an assertion as a step in a proof, you must
first prove that the substitution is syntactically correct (or if you
prefer, you must ``construct'' it), showing for example that the
expression you are substituting for a wff metavariable is a valid wff.
The \texttt{\$a}\index{\texttt{\$a} statement} statement is used to
specify those combinations of symbols that are considered syntactically
valid, such as the legal forms of wffs.

Second, \texttt{\$a} statements are used to specify what are ordinarily thought of
as definitions, i.e.\ new combinations of symbols that abbreviate other
combinations of symbols.  Metamath makes no distinction\index{axiom vs.\
definition} between axioms\index{axiom} and definitions\index{definition}.
Indeed, it has been argued that such distinction should not be made even in
ordinary mathematics; see Section~\ref{definitions}, which discusses the
philosophy of definitions.  Section~\ref{hierarchy} discusses some
technical requirements for definitions.  In \texttt{set.mm} we adopt the
convention of prefixing axiom labels with \texttt{ax-} and definition labels with
\texttt{df-}\index{label}.

The results that can be derived with the Metamath language are only as good as
the \texttt{\$a}\index{\texttt{\$a} statement} statements used as their starting
point.  We cannot stress this too strongly.  For example, Metamath will
not prevent you from specifying $x\neq x$ as an axiom of logic.  It is
essential that you scrutinize all \texttt{\$a} statements with great care.
Because they are a source of potential pitfalls, it is best not to add new
ones (usually new definitions) casually; rather you should carefully evaluate
each one's necessity and advantages.

Once you have in place all of the basic axioms\index{axiom} and
rules\index{rule} of a mathematical theory, the only \texttt{\$a} statements that
you will be adding will be what are ordinarily called definitions.  In
principle, definitions should be in some sense eliminable from the language of
a theory according to some convention (usually involving logical equivalence
or equality).  The most common convention is that any formula that was
syntactically valid but not provable before the definition was introduced will
not become provable after the definition is introduced.  In an ideal world,
definitions should not be present at all if one is to have absolute confidence
in a mathematical result.  However, they are necessary to make
mathematics practical, for otherwise the resulting formulas would be
extremely long and incomprehensible.  Since the nature of definitions (in the
most general sense) does not permit them to automatically be verified as
``proper,''\index{proper definition}\index{definition!proper} the judgment of
the mathematician is required to ensure it.  (In \texttt{set.mm} effort was made
to make almost all definitions directly eliminable and thus minimize the need
for such judgment.)

If you are not a mathematician, it may be best not to add or change any
\texttt{\$a}\index{\texttt{\$a} statement} statements but instead use
the mathematical language already provided in standard databases.  This
way Metamath will not allow you to make a mistake (i.e.\ prove a false
result).


\subsection{Frames}\label{frames}

We now introduce the concept of a collection of related Metamath statements
called a frame.  Every assertion (\texttt{\$a} or \texttt{\$p} statement) in the database has
an associated frame.

A {\bf frame}\index{frame} is a sequence of \texttt{\$d}, \texttt{\$f},
and \texttt{\$e} statements (zero or more of each) followed by one
\texttt{\$a} or \texttt{\$p} statement, subject to certain conditions we
will describe.  For simplicity we will assume that all math symbol
tokens used are declared at the beginning of the database with
\texttt{\$c} and \texttt{\$v} statements (which are not properly part of
a frame).  Also for simplicity we will assume there are only simple
\texttt{\$d} statements (those with only two variables) and imagine any
compound \texttt{\$d} statements (those with more than two variables) as
broken up into simple ones.

A frame groups together those hypotheses (and \texttt{\$d} statements) relevant
to an assertion (\texttt{\$a} or \texttt{\$p} statement).  The statements in a frame
may or may not be physically adjacent in a database; we will cover
this in our discussion of scoping statements
in Section~\ref{scoping}.

A frame has the following properties:
\begin{enumerate}
 \item The set of variables contained in its \texttt{\$f} statements must
be identical to the set of variables contained in its \texttt{\$e},
\texttt{\$a}, and/or \texttt{\$p} statements.  In other words, each
variable in a \texttt{\$e}, \texttt{\$a}, or \texttt{\$p} statement must
have an associated ``variable type'' defined for it in a \texttt{\$f}
statement.
  \item No two \texttt{\$f} statements may contain the same variable.
  \item Any \texttt{\$f} statement
must occur before a \texttt{\$e} statement in which its variable occurs.
\end{enumerate}

The first property determines the set of variables occurring in a frame.
These are the {\bf mandatory
variables}\index{mandatory variable} of the frame.  The second property
tells us there must be only one type specified for a variable.
The last property is not a theoretical requirement but it
makes parsing of the database easier.

For our examples, we assume our database has the following declarations:

\begin{verbatim}
$v P Q R $.
$c -> ( ) |- wff $.
\end{verbatim}

The following sequence of statements, describing the modus ponens inference
rule, is an example of a frame:

\begin{verbatim}
wp  $f wff P $.
wq  $f wff Q $.
maj $e |- ( P -> Q ) $.
min $e |- P $.
mp  $a |- Q $.
\end{verbatim}

The following sequence of statements is not a frame because \texttt{R} does not
occur in the \texttt{\$e}'s or the \texttt{\$a}:

\begin{verbatim}
wp  $f wff P $.
wq  $f wff Q $.
wr  $f wff R $.
maj $e |- ( P -> Q ) $.
min $e |- P $.
mp  $a |- Q $.
\end{verbatim}

The following sequence of statements is not a frame because \texttt{Q} does not
occur in a \texttt{\$f}:

\begin{verbatim}
wp  $f wff P $.
maj $e |- ( P -> Q ) $.
min $e |- P $.
mp  $a |- Q $.
\end{verbatim}

The following sequence of statements is not a frame because the \texttt{\$a} statement is
not the last one:

\begin{verbatim}
wp  $f wff P $.
wq  $f wff Q $.
maj $e |- ( P -> Q ) $.
mp  $a |- Q $.
min $e |- P $.
\end{verbatim}

Associated with a frame is a sequence of {\bf mandatory
hypotheses}\index{mandatory hypothesis}.  This is simply the set of all
\texttt{\$f} and \texttt{\$e} statements in the frame, in the order they
appear.  A frame can be referenced in a later proof using the label of
the \texttt{\$a} or \texttt{\$p} assertion statement, and the proof
makes an assignment to each mandatory hypothesis in the order in which
it appears.  This means the order of the hypotheses, once chosen, must
not be changed so as not to affect later proofs referencing the frame's
assertion statement.  (The Metamath proof verifier will, of course, flag
an error if a proof becomes incorrect by doing this.)  Since proofs make
use of ``Reverse Polish notation,'' described in Section~\ref{proof}, we
call this order the {\bf RPN order}\index{RPN order} of the hypotheses.

Note that \texttt{\$d} statements are not part of the set of mandatory
hypotheses, and their order doesn't matter (as long as they satisfy the
fourth property for a frame described above).  The \texttt{\$d}
statements specify restrictions on variables that must be satisfied (and
are checked by the proof verifier) when expressions are substituted for
them in a proof, and the \texttt{\$d} statements themselves are never
referenced directly in a proof.

A frame with a \texttt{\$p} (provable) statement requires a proof as part of the
\texttt{\$p} statement.  Sometimes in a proof we want to make use of temporary or
dummy variables\index{dummy variable} that do not occur in the \texttt{\$p}
statement or its mandatory hypotheses.  To accommodate this we define an {\bf
extended frame}\index{extended frame} as a frame together with zero or more
\texttt{\$d} and \texttt{\$f} statements that reference variables not among the
mandatory variables of the frame.  Any new variables referenced are called the
{\bf optional variables}\index{optional variable} of the extended frame. If a
\texttt{\$f} statement references an optional variable it is called an {\bf
optional hypothesis}\index{optional hypothesis}, and if one or both of the
variables in a \texttt{\$d} statement are optional variables it is called an {\bf
optional disjoint-variable restriction}\index{optional disjoint-variable
restriction}.  Properties 2, 3, and 4 for a frame also apply to an extended
frame.

The concept of optional variables is not meaningful for frames with \texttt{\$a}
statements, since those statements have no proofs that might make use of them.
There is no restriction on including optional hypotheses in the extended frame
for a \texttt{\$a} statement, but they serve no purpose.

The following set of statements is an example of an extended frame, which
contains an optional variable \texttt{R} and an optional hypothesis \texttt{wr}.  In
this example, we suppose the rule of modus ponens is not an axiom but is
derived as a theorem from earlier statements (we omit its presumed proof).
Variable \texttt{R} may be used in its proof if desired (although this would
probably have no advantage in propositional calculus).  Note that the sequence
of mandatory hypotheses in RPN order is still \texttt{wp}, \texttt{wq}, \texttt{maj},
\texttt{min} (i.e.\ \texttt{wr} is omitted), and this sequence is still assumed
whenever the assertion \texttt{mp} is referenced in a subsequent proof.

\begin{verbatim}
wp  $f wff P $.
wq  $f wff Q $.
wr  $f wff R $.
maj $e |- ( P -> Q ) $.
min $e |- P $.
mp  $p |- Q $= ... $.
\end{verbatim}

Every frame is an extended frame, but not every extended frame is a frame, as
this example shows.  The underlying frame for an extended frame is
obtained by simply removing all statements containing optional variables.
Any proof referencing an assertion will ignore any extensions to its
frame, which means we may add or delete optional hypotheses at will without
affecting subsequent proofs.

The conceptually simplest way of organizing a Metamath database is as a
sequence of extended frames.  The scoping statements
\texttt{\$\char`\{}\index{\texttt{\$\char`\{} and \texttt{\$\char`\}}
keywords} and \texttt{\$\char`\}} can be used to delimit the start and
end of an extended frame, leading to the following possible structure for a
database.  \label{framelist}

\vskip 2ex
\setbox\startprefix=\hbox{\tt \ \ \ \ \ \ \ \ }
\setbox\contprefix=\hbox{}
\startm
\m{\mbox{(\texttt{\$v} {\em and} \texttt{\$c}\,{\em statements})}}
\endm
\startm
\m{\mbox{\texttt{\$\char`\{}}}
\endm
\startm
\m{\mbox{\texttt{\ \ } {\em extended frame}}}
\endm
\startm
\m{\mbox{\texttt{\$\char`\}}}}
\endm
\startm
\m{\mbox{\texttt{\$\char`\{}}}
\endm
\startm
\m{\mbox{\texttt{\ \ } {\em extended frame}}}
\endm
\startm
\m{\mbox{\texttt{\$\char`\}}}}
\endm
\startm
\m{\mbox{\texttt{\ \ \ \ \ \ \ \ \ }}\vdots}
\endm
\vskip 2ex

In practice, this structure is inconvenient because we have to repeat
any \texttt{\$f}, \texttt{\$e}, and \texttt{\$d} statements over and
over again rather than stating them once for use by several assertions.
The scoping statements, which we will discuss next, allow this to be
done.  In principle, any Metamath database can be converted to the above
format, and the above format is the most convenient to use when studying
a Metamath database as a formal system%
%% Uncomment this when uncommenting section {formalspec} below
   (Appendix \ref{formalspec})%
.
In fact, Metamath internally converts the database to the above format.
The command \texttt{show statement} in the Metamath program will show
you the contents of the frame for any \texttt{\$a} or \texttt{\$p}
statement, as well as its extension in the case of a \texttt{\$p}
statement.

%c%(provided that all ``local'' variables and constants with limited scope have
%c%unique names),

During our discussion of scoping statements, it may be helpful to
think in terms of the equivalent sequence of frames that will result when
the database is parsed.  Scoping (other than the limited
use above to delimit frames) is not a theoretical requirement for
Metamath but makes it more convenient.


\subsection{Scoping Statements (\texttt{\$\{} and \texttt{\$\}})}\label{scoping}
\index{\texttt{\$\char`\{} and \texttt{\$\char`\}} keywords}\index{scoping statement}

%c%Some Metamath statements may be needed only temporarily to
%c%serve a specific purpose, and after we're done with them we would like to
%c%disregard or ignore them.  For example, when we're finished using a variable,
%c%we might want to
%c%we might want to free up the token\index{token} used to name it so that the
%c%token can be used for other purposes later on, such as a different kind of
%c%variable or even a constant.  In the terminology of computer programming, we
%c%might want to let some symbol declarations be ``local'' rather than ``global.''
%c%\index{local symbol}\index{global symbol}

The {\bf scoping} statements, \texttt{\$\char`\{} ({\bf start of block}) and \texttt{\$\char`\}}
({\bf end of block})\index{block}, provide a means for controlling the portion
of a database over which certain statement types are recognized.  The
syntax of a scoping statement is very simple; it just consists of the
statement's keyword:
\begin{center}
\texttt{\$\char`\{}\\
\texttt{\$\char`\}}
\end{center}
\index{\texttt{\$\char`\{} and \texttt{\$\char`\}} keywords}

For example, consider the following database where we have stripped out
all tokens except the scoping statement keywords.  For the purpose of the
discussion, we have added subscripts to the scoping statements; these subscripts
do not appear in the actual database.
\[
 \mbox{\tt \ \$\char`\{}_1
 \mbox{\tt \ \$\char`\{}_2
 \mbox{\tt \ \$\char`\}}_2
 \mbox{\tt \ \$\char`\{}_3
 \mbox{\tt \ \$\char`\{}_4
 \mbox{\tt \ \$\char`\}}_4
 \mbox{\tt \ \$\char`\}}_3
 \mbox{\tt \ \$\char`\}}_1
\]
Each \texttt{\$\char`\{} statement in this example is said to be {\bf
matched} with the \texttt{\$\char`\}} statement that has the same
subscript.  Each pair of matched scoping statements defines a region of
the database called a {\bf block}.\index{block} Blocks can be {\bf
nested}\index{nested block} inside other blocks; in the example, the
block defined by $\mbox{\tt \$\char`\{}_4$ and $\mbox{\tt \$\char`\}}_4$
is nested inside the block defined by $\mbox{\tt \$\char`\{}_3$ and
$\mbox{\tt \$\char`\}}_3$ as well as inside the block defined by
$\mbox{\tt \$\char`\{}_1$ and $\mbox{\tt \$\char`\}}_1$.  In general, a
block may be empty, it may contain only non-scoping
statements,\footnote{Those statements other than \texttt{\$\char`\{} and
\texttt{\$\char`\}}.}\index{non-scoping statement} or it may contain any
mixture of other blocks and non-scoping statements.  (This is called a
``recursive'' definition\index{recursive definition} of a block.)

Associated with each block is a number called its {\bf nesting
level}\index{nesting level} that indicates how deeply the block is nested.
The nesting levels of the blocks in our example are as follows:
\[
  \underbrace{
    \mbox{\tt \ }
    \underbrace{
     \mbox{\tt \$\char`\{\ }
     \underbrace{
       \mbox{\tt \$\char`\{\ }
       \mbox{\tt \$\char`\}}
     }_{2}
     \mbox{\tt \ }
     \underbrace{
       \mbox{\tt \$\char`\{\ }
       \underbrace{
         \mbox{\tt \$\char`\{\ }
         \mbox{\tt \$\char`\}}
       }_{3}
       \mbox{\tt \ \$\char`\}}
     }_{2}
     \mbox{\tt \ \$\char`\}}
   }_{1}
   \mbox{\tt \ }
 }_{0}
\]
\index{\texttt{\$\char`\{} and \texttt{\$\char`\}} keywords}
The entire database is considered to be one big block (the {\bf outermost}
block) with a nesting level of 0.  The outermost block is {\em not} bracketed
by scoping statements.\footnote{The language was designed this way so that
several source files can be joined together more easily.}\index{outermost
block}

All non-scoping Metamath statements become recognized or {\bf
active}\index{active statement} at the place where they appear.\footnote{To
keep things slightly simpler, we do not bother to define the concept of
``active'' for the scoping statements.}  Certain of these statement types
become inactive at the end of the block in which they appear; these statement
types are:
\begin{center}
  \texttt{\$c}, \texttt{\$v}, \texttt{\$d}, \texttt{\$e}, and \texttt{\$f}.
%  \texttt{\$v}, \texttt{\$f}, \texttt{\$e}, and \texttt{\$d}.
\end{center}
\index{\texttt{\$c} statement}
\index{\texttt{\$d} statement}
\index{\texttt{\$e} statement}
\index{\texttt{\$f} statement}
\index{\texttt{\$v} statement}
The other statement types remain active forever (i.e.\ through the end of the
database); they are:
\begin{center}
  \texttt{\$a} and \texttt{\$p}.
%  \texttt{\$c}, \texttt{\$a}, and \texttt{\$p}.
\end{center}
\index{\texttt{\$a} statement}
\index{\texttt{\$p} statement}
Any statement (of these 7 types) located in the outermost
block\index{outermost block} will remain active through the end of the
database and thus are effectively ``global'' statements.\index{global
statement}

All \texttt{\$c} statements must be placed in the outermost block.  Since they are
therefore always global, they could be considered as belonging to both of the
above categories.

The {\bf scope}\index{scope} of a statement is the set of statements that
recognize it as active.

%c%The concept of ``active'' is also defined for math symbols\index{math
%c%symbol}.  Math symbols (constants\index{constant} and
%c%variables\index{variable}) become {\bf active}\index{active
%c%math symbol} in the \texttt{\$c}\index{\texttt{\$c}
%c%statement} and \texttt{\$v}\index{\texttt{\$v} statement} statements that
%c%declare them.  They become inactive when their declaration statements become
%c%inactive.

The concept of ``active'' is also defined for math symbols\index{math
symbol}.  Math symbols (constants\index{constant} and
variables\index{variable}) become {\bf active}\index{active math symbol}
in the \texttt{\$c}\index{\texttt{\$c} statement} and
\texttt{\$v}\index{\texttt{\$v} statement} statements that declare them.
A variable becomes inactive when its declaration statement becomes
inactive.  Because all \texttt{\$c} statements must be in the outermost
block, a constant will never become inactive after it is declared.

\subsubsection{Redeclaration of Math Symbols}
\index{redeclaration of symbols}\label{redeclaration}

%c%A math symbol may not be declared a second time while it is active, but it may
%c%be declared again after it becomes inactive.

A variable may not be declared a second time while it is active, but it may be
declared again after it becomes inactive.  This provides a convenient way to
introduce ``local'' variables,\index{local variable} i.e.\ temporary variables
for use in the frame of an assertion or in a proof without keeping them around
forever.  A previously declared variable may not be redeclared as a constant.

A constant may not be redeclared.  And, as mentioned above, constants must be
declared in the outermost block.

The reason variables may have limited scope but not constants is that an
assertion (\texttt{\$a} or \texttt{\$p} statement) remains available for use in
proofs through the end of the database.  Variables in an assertion's frame may
be substituted with whatever is needed in a proof step that references the
assertion, whereas constants remain fixed and may not be substituted with
anything.  The particular token used for a variable in an assertion's frame is
irrelevant when the assertion is referenced in a proof, and it doesn't matter
if that token is not available outside of the referenced assertion's frame.
Constants, however, must be globally fixed.

In the present version of the Metamath language, there is no theoretical
benefit for the feature allowing variables to be active for limited scopes
rather than global. It is just a convenience that allows them, for example, to
be locally grouped together with their corresponding \texttt{\$f} variable-type
declarations.\footnote{A future version\label{redeclarationf} of the Metamath
language may extend the language to include true local variables as well as
local constants.  I have not decided upon the best way to do this or even if
it should be done at all; the main issue is whether the benefits offset
greater confusion in learning the language.  In any case, the rules for
the present language were chosen to let existing databases be compatible with
any future language extensions.}

%c%If you declare a math symbol more than once, internally Metamath considers it a
%c%new distinct symbol, even though it has the same name.  If you are unaware of
%c%this, you may find that what you think are correct proofs are incorrectly
%c%rejected as invalid, because Metamath may tell you that a constant you
%c%previously declared does not match a newly declared math symbol with the same
%c%name.  For details on this subtle point, see the Comment on
%c%p.~\pageref{spec4comment}.  This is done purposely to allow temporary
%c%constants to be introduced while developing a subtheory, then allow their math
%c%symbol tokens to be reused later on; in general they will not refer to the
%c%same thing.  In practice, you would not ordinarily reuse the names of
%c%constants because it would tend to be confusing to the reader.  The reuse of
%c%names of variables, on the other hand, is something that is often useful to do
%c%(for example it is done frequently in \texttt{set.mm}).  Since variables in an
%c%assertion referenced in a proof can be substituted as needed to achieve a
%c%symbol match, this is not an issue.

% (This section covers a somewhat advanced topic you may want to skip
% at first reading.)
%
% Under certain circumstances, math symbol\index{math symbol}
% tokens\index{token} may be redeclared (i.e.\ the token
% may appear in more than
% one \texttt{\$c}\index{\texttt{\$c} statement} or \texttt{\$v}\index{\texttt{\$v}
% statement} statement).  You might want to do this say, to make temporary use
% of a variable name without having to worry about its affect elsewhere,
% somewhat analogous to declaring a local variable in a standard computer
% language.  Understanding what goes on when math symbol tokens are redeclared
% is a little tricky to understand at first, since it requires that we
% distinguish the token itself from the math symbol that it names.  It will help
% if we first take a peek at the internal workings of the
% Metamath\index{Metamath} program.
%
% Metamath reserves a memory location for each occurrence of a
% token\index{token} in a declaration statement (\texttt{\$c}\index{\texttt{\$c}
% statement} or \texttt{\$v}\index{\texttt{\$v} statement}).  If a given token appears
% in more than one declaration statement, it will refer to more than one memory
% locations.  A math symbol\index{math symbol} may be thought of as being one of
% these memory locations rather than as the token itself.  Only one of the
% memory locations associated with a given token may be active at any one time.
% The math symbol (memory location) that gets looked up when the token appears
% in a non-declaration statement is the one that happens to be active at that
% time.
%
% We now look at the rules for the redeclaration\index{redeclaration of symbols}
% of math symbol tokens.
% \begin{itemize}
% \item A math symbol token may not be declared twice in the
% same block.\footnote{While there is no theoretical reason for disallowing
% this, it was decided in the design of Metamath that allowing it would offer no
% advantage and might cause confusion.}
% \item An inactive math symbol may always be
% redeclared.
% \item  An active math symbol may be redeclared in a different (i.e.\
% inner) block\index{block} from the one it became active in.
% \end{itemize}
%
% When a math symbol token is redeclared, it conceptually refers to a different
% math symbol, just as it would be if it were called a different name.  In
% addition, the original math symbol that it referred to, if it was active,
% temporarily becomes inactive.  At the end of the block in which the
% redeclaration occurred, the new math symbol\index{math symbol} becomes
% inactive and the original symbol becomes active again.  This concept is
% illustrated in the following example, where the symbol \texttt{e} is
% ordinarily a constant (say Euler's constant, 2.71828...) but
% temporarily we want to use it as a ``local'' variable, say as a coefficient
% in the equation $a x^4 + b x^3 + c x^2 + d x + e$:
% \[
%   \mbox{\tt \$\char`\{\ \$c e \$.}
%   \underbrace{
%     \ \ldots\ %
%     \mbox{\tt \$\char`\{}\ \ldots\ %
%   }_{\mbox{\rm region A}}
%   \mbox{\tt \$v e \$.}
%   \underbrace{
%     \mbox{\ \ \ \ldots\ \ \ }
%   }_{\mbox{\rm region B}}
%   \mbox{\tt \$\char`\}}
%   \underbrace{
%     \mbox{\ \ \ \ldots\ \ \ }
%   }_{\mbox{\rm region C}}
%   \mbox{\tt \$\char`\}}
% \]
% \index{\texttt{\$\char`\{} and \texttt{\$\char`\}} keywords}
% In region A, the token \texttt{e} refers to a constant.  It is redeclared as a
% variable in region B, and any reference to it in this region will refer to this
% variable.  In region C, the redeclaration becomes inactive, and the original
% declaration becomes active again.  In region C, the token \texttt{x} refers to the
% original constant.
%
% As a practical matter, overuse of math symbol\index{math symbol}
% redeclarations\index{redeclaration of symbols} can be confusing (even though
% it is well-defined) and is best avoided when possible.  Here are some good
% general guidelines you can follow.  Usually, you should declare all
% constants\index{constant} in the outermost block\index{outermost block},
% especially if they are general-purpose (such as the token \verb$A.$, meaning
% $\forall$ or ``for all'').  This will make them ``globally'' active (although
% as in the example above local redeclarations will temporarily make them
% inactive.)  Most or all variables\index{variable}, on the other hand, could be
% declared in inner blocks, so that the token for them can be used later for a
% different type of variable or a constant.  (The names of the variables you
% choose are not used when you refer to an assertion\index{assertion} in a
% proof, whereas constants must match exactly.  A locally declared constant will
% not match a globally declared constant in a proof, even if they use the same
% token, because Metamath internally considers them to be different math
% symbols.)  To avoid confusion, you should generally avoid redeclaring active
% variables.  If you must redeclare them, do so at the beginning of a block.
% The temporary declaration of constants in inner blocks might be occasionally
% appropriate when you make use of a temporary definition to prove lemmas
% leading to a main result that does not make direct use of the definition.
% This way, you will not clutter up your database with a large number of
% seldom-used global constant symbols.  You might want to note that while
% inactive constants may not appear directly in an assertion (a \texttt{\$a}\index{\texttt{\$a}
% statement} or \texttt{\$p}\index{\texttt{\$p} statement}
% statement), they may be indirectly used in the proof of a \texttt{\$p} statement
% so long as they do not appear in the final math symbol sequence constructed by
% the proof.  In the end, you will have to use your best judgment, taking into
% account standard mathematical usage of the symbols as well as consideration
% for the reader of your work.
%
% \subsubsection{Reuse of Labels}\index{reuse of labels}\index{label}
%
% The \texttt{\$e}\index{\texttt{\$e} statement}, \texttt{\$f}\index{\texttt{\$f}
% statement}, \texttt{\$a}\index{\texttt{\$a} statement}, and
% \texttt{\$p}\index{\texttt{\$p}
% statement} statement types require labels, which allow them to be
% referenced later inside proofs.  A label is considered {\bf
% active}\index{active label} when the statement it is associated with is
% active.  The token\index{token} for a label may be reused
% (redeclared)\index{redeclaration of labels} provided that it is not being used
% for a currently active label.  (Unlike the tokens for math symbols, active
% label tokens may not be redeclared in an inner scope.)  Note that the labels
% of \texttt{\$a} and \texttt{\$p} statements can never be reused after these
% statements appear, because these statements remain active through the end of
% the database.
%
% You might find the reuse of labels a convenient way to have standard names for
% temporary hypotheses, such as \texttt{h1}, \texttt{h2}, etc.  This way you don't have
% to invent unique names for each of them, and in some cases it may be less
% confusing to the reader (although in other cases it might be more confusing, if
% the hypothesis is located far away from the assertion that uses
% it).\footnote{The current implementation requires that all labels, even
% inactive ones, be unique.}

\subsubsection{Frames Revisited}\index{frames and scoping statements}

Now that we have covered scoping, we will look at how an arbitrary
Metamath database can be converted to the simple sequence of extended
frames described on p.~\pageref{framelist}.  This is also how Metamath
stores the database internally when it reads in the database
source.\label{frameconvert} The method is simple.  First, we collect all
constant and variable (\texttt{\$c} and \texttt{\$v}) declarations in
the database, ignoring duplicate declarations of the same variable in
different scopes.  We then put our collected \texttt{\$c} and
\texttt{\$v} declarations at the beginning of the database, so that
their scope is the entire database.  Next, for each assertion in the
database, we determine its frame and extended frame.  The extended frame
is simply the \texttt{\$f}, \texttt{\$e}, and \texttt{\$d} statements
that are active.  The frame is the extended frame with all optional
hypotheses removed.

An equivalent way of saying this is that the extended frame of an assertion
is the collection of all \texttt{\$f}, \texttt{\$e}, and \texttt{\$d} statements
whose scope includes the assertion.
The \texttt{\$f} and \texttt{\$e} statements
occur in the order they appear
(order is irrelevant for \texttt{\$d} statements).

%c%, renaming any
%c%redeclared variables as needed so that all of them have unique names.  (The
%c%exact renaming convention is unimportant.  You might imagine renaming
%c%different declarations of math symbol \texttt{a} as \texttt{a\$1}, \texttt{a\$2}, etc.\
%c%which would prevent any conflicts since \texttt{\$} is not a legal character in a
%c%math symbol token.)

\section{The Anatomy of a Proof} \label{proof}
\index{proof!Metamath, description of}

Each provable assertion (\texttt{\$p}\index{\texttt{\$p} statement} statement) in a
database must include a {\bf proof}\index{proof}.  The proof is located
between the \texttt{\$=}\index{\texttt{\$=} keyword} and \texttt{\$.}\ keywords in the
\texttt{\$p} statement.

In the basic Metamath language\index{basic language}, a proof is a
sequence of statement labels.  This label sequence\index{label sequence}
serves as a set of instructions that the Metamath program uses to
construct a series of math symbol sequences.  The construction must
ultimately result in the math symbol sequence contained between the
\texttt{\$p}\index{\texttt{\$p} statement} and
\texttt{\$=}\index{\texttt{\$=} keyword} keywords of the \texttt{\$p}
statement.  Otherwise, the Metamath program will consider the proof
incorrect, and it will notify you with an appropriate error message when
you ask it to verify the proof.\footnote{To make the loading faster, the
Metamath program does not automatically verify proofs when you
\texttt{read} in a database unless you use the \texttt{/verify}
qualifier.  After a database has been read in, you may use the
\texttt{verify proof *} command to verify proofs.}\index{\texttt{verify
proof} command} Each label in a proof is said to {\bf
reference}\index{label reference} its corresponding statement.

Associated with any assertion\index{assertion} (\texttt{\$p} or
\texttt{\$a}\index{\texttt{\$a} statement} statement) is a set of
hypotheses (\texttt{\$f}\index{\texttt{\$f} statement} or
\texttt{\$e}\index{\texttt{\$e} statement} statements) that are active
with respect to that assertion.  Some are mandatory and the others are
optional.  You should review these concepts if necessary.

Each label\index{label} in a proof must be either the label of a
previous assertion (\texttt{\$a}\index{\texttt{\$a} statement} or
\texttt{\$p}\index{\texttt{\$p} statement} statement) or the label of an
active hypothesis (\texttt{\$e} or \texttt{\$f}\index{\texttt{\$f}
statement} statement) of the \texttt{\$p} statement containing the
proof.  Hypothesis labels may reference both the
mandatory\index{mandatory hypothesis} and the optional hypotheses of the
\texttt{\$p} statement.

The label sequence in a proof specifies a construction in {\bf reverse Polish
notation}\index{reverse Polish notation (RPN)} (RPN).  You may be familiar
with RPN if you have used older
Hewlett--Packard or similar hand-held calculators.
In the calculator analogy, a hypothesis label\index{hypothesis label} is like
a number and an assertion label\index{assertion label} is like an operation
(more precisely, an $n$-ary operation when the
assertion has $n$ \texttt{\$e}-hypotheses).
On an RPN calculator, an operation takes one or more previous numbers in an
input sequence, performs a calculation on them, and replaces those numbers and
itself with the result of the calculation.  For example, the input sequence
$2,3,+$ on an RPN calculator results in $5$, and the input sequence
$2,3,5,{\times},+$ results in $2,15,+$ which results in $17$.

Understanding how RPN is processed involves the concept of a {\bf
stack}\index{stack}\index{RPN stack}, which can be thought of as a set of
temporary memory locations that hold intermediate results.  When Metamath
encounters a hypothesis label it places or {\bf pushes}\index{push} the math
symbol sequence of the hypothesis onto the stack.  When Metamath encounters an
assertion label, it associates the most recent stack entries with the {\em
mandatory} hypotheses\index{mandatory hypothesis} of the assertion, in the
order where the most recent stack entry is associated with the last mandatory
hypothesis of the assertion.  It then determines what
substitutions\index{substitution!variable}\index{variable substitution} have
to be made into the variables of the assertion's mandatory hypotheses to make
them identical to the associated stack entries.  It then makes those same
substitutions into the assertion itself.  Finally, Metamath removes or {\bf
pops}\index{pop} the matched hypotheses from the stack and pushes the
substituted assertion onto the stack.

For the purpose of matching the mandatory hypothesis to the most recent stack
entries, whether a hypothesis is a \texttt{\$e} or \texttt{\$f} statement is
irrelevant.  The only important thing is that a set of
substitutions\footnote{In the Metamath spec (Section~\ref{spec}), we use the
singular term ``substitution'' to refer to the set of substitutions we talk
about here.} exist that allow a match (and if they don't, the proof verifier
will let you know with an error message).  The Metamath language is specified
in such a way that if a set of substitutions exists, it will be unique.
Specifically, the requirement that each variable have a type specified for it
with a \texttt{\$f} statement ensures the uniqueness.

We will illustrate this with an example.
Consider the following Metamath source file:
\begin{verbatim}
$c ( ) -> wff $.
$v p q r s $.
wp $f wff p $.
wq $f wff q $.
wr $f wff r $.
ws $f wff s $.
w2 $a wff ( p -> q ) $.
wnew $p wff ( s -> ( r -> p ) ) $= ws wr wp w2 w2 $.
\end{verbatim}
This Metamath source example shows the definition and ``proof'' (i.e.,
construction) of a well-formed formula (wff)\index{well-formed formula (wff)}
in propositional calculus.  (You may wish to type this example into a file to
experiment with the Metamath program.)  The first two statements declare
(introduce the names of) four constants and four variables.  The next four
statements specify the variable types, namely that
each variable is assumed to be a wff.  Statement \texttt{w2} defines (postulates)
a way to produce a new wff, \texttt{( p -> q )}, from two given wffs \texttt{p} and
\texttt{q}. The mandatory hypotheses of \texttt{w2} are \texttt{wp} and \texttt{wq}.
Statement \texttt{wnew} claims that \texttt{( s -> ( r -> p ) )} is a wff given
three wffs \texttt{s}, \texttt{r}, and \texttt{p}.  More precisely, \texttt{wnew} claims
that the sequence of ten symbols \texttt{wff ( s -> ( r -> p ) )} is provable from
previous assertions and the hypotheses of \texttt{wnew}.  Metamath does not know
or care what a wff is, and as far as it is concerned
the typecode \texttt{wff} is just an
arbitrary constant symbol in a math symbol sequence.  The mandatory hypotheses
of \texttt{wnew} are \texttt{wp}, \texttt{wr}, and \texttt{ws}; \texttt{wq} is an optional
hypothesis.  In our particular proof, the optional hypothesis is not
referenced, but in general, any combination of active (i.e.\ optional and
mandatory) hypotheses could be referenced.  The proof of statement \texttt{wnew}
is the sequence of five labels starting with \texttt{ws} (step~1) and ending with
\texttt{w2} (step~5).

When Metamath verifies the proof, it scans the proof from left to right.  We
will examine what happens at each step of the proof.  The stack starts off
empty.  At step 1, Metamath looks up label \texttt{ws} and determines that it is a
hypothesis, so it pushes the symbol sequence of statement \texttt{ws} onto the
stack:

\begin{center}\begin{tabular}{|l|l|}\hline
{Stack location} & {Contents} \\ \hline \hline
1 & \texttt{wff s} \\ \hline
\end{tabular}\end{center}

Metamath sees that the labels \texttt{wr} and \texttt{wp} in steps~2 and 3 are also
hypotheses, so it pushes them onto the stack.  After step~3, the stack looks
like
this:

\begin{center}\begin{tabular}{|l|l|}\hline
{Stack location} & {Contents} \\ \hline \hline
3 & \texttt{wff p} \\ \hline
2 & \texttt{wff r} \\ \hline
1 & \texttt{wff s} \\ \hline
\end{tabular}\end{center}

At step 4, Metamath sees that label \texttt{w2} is an assertion, so it must do
some processing.  First, it associates the mandatory hypotheses of \texttt{w2},
which are \texttt{wp} and \texttt{wq}, with stack locations~2 and 3, {\em in that
order}. Metamath determines that the only possible way
to make hypothesis \texttt{wp} match (become identical to) stack location~2 and
\texttt{wq} match stack location 3 is to substitute variable \texttt{p} with \texttt{r}
and \texttt{q} with \texttt{p}.  Metamath makes these substitutions into \texttt{w2} and
obtains the symbol sequence \texttt{wff ( r -> p )}.  It removes the hypotheses
from stack locations~2 and 3, then places the result into stack location~2:

\begin{center}\begin{tabular}{|l|l|}\hline
{Stack location} & {Contents} \\ \hline \hline
2 & \texttt{wff ( r -> p )} \\ \hline
1 & \texttt{wff s} \\ \hline
\end{tabular}\end{center}

At step 5, Metamath sees that label \texttt{w2} is an assertion, so it must again
do some processing.  First, it matches the mandatory hypotheses of \texttt{w2},
which are \texttt{wp} and \texttt{wq}, to stack locations 1 and 2.
Metamath determines that the only possible way to make the
hypotheses match is to substitute variable \texttt{p} with \texttt{s} and \texttt{q} with
\texttt{( r -> p )}.  Metamath makes these substitutions into \texttt{w2} and obtains
the symbol
sequence \texttt{wff ( s -> ( r -> p ) )}.  It removes stack
locations 1 and 2, then places the result into stack location~1:

\begin{center}\begin{tabular}{|l|l|}\hline
{Stack location} & {Contents} \\ \hline \hline
1 & \texttt{wff ( s -> ( r -> p ) )} \\ \hline
\end{tabular}\end{center}

After Metamath finishes processing the proof, it checks to see that the
stack contains exactly one element and that this element is
the same as the math symbol sequence in the
\texttt{\$p}\index{\texttt{\$p} statement} statement.  This is the case for our
proof of \texttt{wnew},
so we have proved \texttt{wnew} successfully.  If the result
differs, Metamath will notify you with an error message.  An error message
will also result if the stack contains more than one entry at the end of the
proof, or if the stack did not contain enough entries at any point in the
proof to match all of the mandatory hypotheses\index{mandatory hypothesis} of
an assertion.  Finally, Metamath will notify you with an error message if no
substitution is possible that will make a referenced assertion's hypothesis
match the
stack entries.  You may want to experiment with the different kinds of errors
that Metamath will detect by making some small changes in the proof of our
example.

Metamath's proof notation was designed primarily to express proofs in a
relatively compact manner, not for readability by humans.  Metamath can display
proofs in a number of different ways with the \texttt{show proof}\index{\texttt{show
proof} command} command.  The
\texttt{/lemmon} qualifier displays it in a format that is easier to read when the
proofs are short, and you saw examples of its use in Chapter~\ref{using}.  For
longer proofs, it is useful to see the tree structure of the proof.  A tree
structure is displayed when the \texttt{/lemmon} qualifier is omitted.  You will
probably find this display more convenient as you get used to it. The tree
display of the proof in our example looks like
this:\label{treeproof}\index{tree-style proof}\index{proof!tree-style}
\begin{verbatim}
1     wp=ws    $f wff s
2        wp=wr    $f wff r
3        wq=wp    $f wff p
4     wq=w2    $a wff ( r -> p )
5  wnew=w2  $a wff ( s -> ( r -> p ) )
\end{verbatim}
The number to the left of each line is the step number.  Following it is a
{\bf hypothesis association}\index{hypothesis association}, consisting of two
labels\index{label} separated by \texttt{=}.  To the left of the \texttt{=} (except
in the last step) is the label of a hypothesis of an assertion referenced
later in the proof; here, steps 1 and 4 are the hypothesis associations for
the assertion \texttt{w2} that is referenced in step 5.  A hypothesis association
is indented one level more than the assertion that uses it, so it is easy to
find the corresponding assertion by moving directly down until the indentation
level decreases to one less than where you started from.  To the right of each
\texttt{=} is the proof step label for that proof step.  The statement keyword of
the proof step label is listed next, followed by the content of the top of the
stack (the most recent stack entry) as it exists after that proof step is
processed.  With a little practice, you should have no trouble reading proofs
displayed in this format.

Metamath proofs include the syntax construction of a formula.
In standard mathematics, this kind of
construction is not considered a proper part of the proof at all, and it
certainly becomes rather boring after a while.
Therefore,
by default the \texttt{show proof}\index{\texttt{show proof}
command} command does not show the syntax construction.
Historically \texttt{show proof} command
\textit{did} show the syntax construction, and you needed to add the
\texttt{/essential} option to hide, them, but today
\texttt{/essential} is the default and you need to use
\texttt{/all} to see the syntax constructions.

When verifying a proof, Metamath will check that no mandatory
\texttt{\$d}\index{\texttt{\$d} statement}\index{mandatory \texttt{\$d}
statement} statement of an assertion referenced in a proof is violated
when substitutions\index{substitution!variable}\index{variable
substitution} are made to the variables in the assertion.  For details
see Section~\ref{spec4} or \ref{dollard}.

\subsection{The Concept of Unification} \label{unify}

During the course of verifying a proof, when Metamath\index{Metamath}
encounters an assertion label\index{assertion label}, it associates the
mandatory hypotheses\index{mandatory hypothesis} of the assertion with the top
entries of the RPN stack\index{stack}\index{RPN stack}.  Metamath then
determines what substitutions\index{substitution!variable}\index{variable
substitution} it must make to the variables in the assertion's mandatory
hypotheses in order for these hypotheses to become identical to their
corresponding stack entries.  This process is called {\bf
unification}\index{unification}.  (We also informally use the term
``unification'' to refer to a set of substitutions that results from the
process, as in ``two unifications are possible.'')  After the substitutions
are made, the hypotheses are said to be {\bf unified}.

If no such substitutions are possible, Metamath will consider the proof
incorrect and notify you with an error message.
% (deleted 3/10/07, per suggestion of Mel O'Cat:)
% The syntax of the
% Metamath language ensures that if a set of substitutions exists, it
% will be unique.

The general algorithm for unification described in the literature is
somewhat complex.
However, in the case of Metamath it is intentionally trivial.
Mandatory hypotheses must be
pushed on the proof stack in the order in which they appear.
In addition, each variable must have its type specified
with a \texttt{\$f} hypothesis before it is used
and that each \texttt{\$f} hypothesis
have the restricted syntax of a typecode (a constant) followed by a variable.
The typecode in the \texttt{\$f} hypothesis must match the first symbol of
the corresponding RPN stack entry (which will also be a constant), so
the only possible match for the variable in the \texttt{\$f} hypothesis is
the sequence of symbols in the stack entry after the initial constant.

In the Proof Assistant\index{Proof Assistant}, a more general unification
algorithm is used.  While a proof is being developed, sometimes not enough
information is available to determine a unique unification.  In this case
Metamath will ask you to pick the correct one.\index{ambiguous
unification}\index{unification!ambiguous}

\section{Extensions to the Metamath Language}\index{extended
language}

\subsection{Comments in the Metamath Language}\label{comments}
\index{markup notation}
\index{comments!markup notation}

The commenting feature allows you to annotate the contents of
a database.  Just as with most
computer languages, comments are ignored for the purpose of interpreting the
contents of the database. Comments effectively act as
additional white space\index{white
space} between tokens
when a database is parsed.

A comment may be placed at the beginning, end, or
between any two tokens\index{token} in a source file.

Comments have the following syntax:
\begin{center}
 \texttt{\$(} {\em text} \texttt{\$)}
\end{center}
Here,\index{\texttt{\$(} and \texttt{\$)} auxiliary
keywords}\index{comment} {\em text} is a string, possibly empty, of any
characters in Metamath's character set (p.~\pageref{spec1chars}), except
that the character strings \texttt{\$(} and \texttt{\$)} may not appear
in {\em text}.  Thus nested comments are not
permitted:\footnote{Computer languages have differing standards for
nested comments, and rather than picking one it was felt simplest not to
allow them at all, at least in the current version of
Metamath\index{Metamath!limitations of version 0.07.30}.} Metamath will
complain if you give it
\begin{center}
 \texttt{\$( This is a \$( nested \$) comment.\ \$)}
\end{center}
To compensate for this non-nesting behavior, I often change all \texttt{\$}'s
to \texttt{@}'s in sections of Metamath code I wish to comment out.

The Metamath program supports a number of markup mechanisms and conventions
to generate good-looking results in \LaTeX\ and {\sc html},
as discussed below.
These markup features have to do only with how the comments are typeset,
and have no effect on how Metamath verifies the proofs in the database.
The improper
use of them may result in incorrectly typeset output, but no Metamath
error messages will result during the \texttt{read} and \texttt{verify
proof} commands.  (However, the \texttt{write
theorem\texttt{\char`\_}list} command
will check for markup errors as a side-effect of its
{\sc html} generation.)
Section~\ref{texout} has instructions for creating \LaTeX\ output, and
section~\ref{htmlout} has instructions for creating
{\sc html}\index{HTML} output.

\subsubsection{Headings}\label{commentheadings}

If the \texttt{\$(} is immediately followed by a new line
starting with a heading marker, it is a header.
This can start with:

\begin{itemize}
 \item[] \texttt{\#\#\#\#} - major part header
 \item[] \texttt{\#*\#*} - section header
 \item[] \texttt{=-=-} - subsection header
 \item[] \texttt{-.-.} - subsubsection header
\end{itemize}

The line following the marker line
will be used for the table of contents entry, after trimming spaces.
The next line should be another (closing) matching marker line.
Any text after that
but before the closing \texttt{\$}, such as an extended description of the
section, will be included on the \texttt{mmtheoremsNNN.html} page.

For more information, run
\texttt{help write theorem\char`\_list}.

\subsubsection{Math mode}
\label{mathcomments}
\index{\texttt{`} inside comments}
\index{\texttt{\char`\~} inside comments}
\index{math mode}

Inside of comments, a string of tokens\index{token} enclosed in
grave accents\index{grave accent (\texttt{`})} (\texttt{`}) will be converted
to standard mathematical symbols during
{\sc HTML}\index{HTML} or \LaTeX\ output
typesetting,\index{latex@{\LaTeX}} according to the information in the
special \texttt{\$t}\index{\texttt{\$t} comment}\index{typesetting
comment} comment in the database
(see section~\ref{tcomment} for information about the typesetting
comment, and Appendix~\ref{ASCII} to see examples of its results).

The first grave accent\index{grave accent (\texttt{`})} \texttt{`}
causes the output processor to enter {\bf math mode}\index{math mode}
and the second one exits it.
In this
mode, the characters following the \texttt{`} are interpreted as a
sequence of math symbol tokens separated by white space\index{white
space}.  The tokens are looked up in the \texttt{\$t}
comment\index{\texttt{\$t} comment}\index{typesetting comment} and if
found, they will be replaced by the standard mathematical symbols that
they correspond to before being placed in the typeset output file.  If
not found, the symbol will be output as is and a warning will be issued.
The tokens do not have to be active in the database, although a warning
will be issued if they are not declared with \texttt{\$c} or
\texttt{\$v} statements.

Two consecutive
grave accents \texttt{``} are treated as a single actual grave accent
(both inside and outside of math mode) and will not cause the output
processor to enter or exit math mode.

Here is an example of its use\index{Pierce's axiom}:
\begin{center}
\texttt{\$( Pierce's axiom, ` ( ( ph -> ps ) -> ph ) -> ph ` ,\\
         is not very intuitive. \$)}
\end{center}
becomes
\begin{center}
   \texttt{\$(} Pierce's axiom, $((\varphi \rightarrow \psi)\rightarrow
\varphi)\rightarrow \varphi$, is not very intuitive. \texttt{\$)}
\end{center}

Note that the math symbol tokens\index{token} must be surrounded by white
space\index{white space}.
%, since there is no context that allows ambiguity to be
%resolved, as is the case with math symbol sequences in some of the Metamath
%statements.
White space should also surround the \texttt{`}
delimiters.

The math mode feature also gives you a quick and easy way to generate
text containing mathematical symbols, independently of the intended
purpose of Metamath.\index{Metamath!using as a math editor} To do this,
simply create your text with grave accents surrounding your formulas,
after making sure that your math symbols are mapped to \LaTeX\ symbols
as described in Appendix~\ref{ASCII}.  It is easier if you start with a
database with predefined symbols such as \texttt{set.mm}.  Use your
grave-quoted math string to replace an existing comment, then typeset
the statement corresponding to that comment following the instructions
from the \texttt{help tex} command in the Metamath program.  You will
then probably want to edit the resulting file with a text editor to fine
tune it to your exact needs.

\subsubsection{Label Mode}\index{label mode}

Outside of math mode, a tilde\index{tilde (\texttt{\char`\~})} \verb/~/
indicates to Metamath's\index{Metamath} output processor that the
token\index{token} that follows (i.e.\ the characters up to the next
white space\index{white space}) represents a statement label or URL.
This formatting mode is called {\bf label mode}\index{label mode}.
If a literal tilde
is desired (outside of math mode) instead of label mode,
use two tildes in a row to represent it.

When generating a \LaTeX\ output file,
the following token will be formatted in \texttt{typewriter}
font, and the tilde removed, to make it stand out from the rest of the text.
This formatting will be applied to all characters after the
tilde up to the first white space\index{white space}.
Whether
or not the token is an actual statement label is not checked, and the
token does not have to have the correct syntax for a label; no error
messages will be produced.  The only effect of the label mode on the
output is that typewriter font will be used for the tokens that are
placed in the \LaTeX\ output file.

When generating {\sc html},
the tokens after the tilde {\em must} be a URL (either http: or https:)
or a valid label.
Error messages will be issued during that output if they aren't.
A hyperlink will be generated to that URL or label.

\subsubsection{Link to bibliographical reference}\index{citation}%
\index{link to bibliographical reference}

Bibliographical references are handled specially when generating
{\sc html} if formatted specially.
Text in the form \texttt{[}{\em author}\texttt{]}
is considered a link to a bibliographical reference.
See \texttt{help html} and \texttt{help write
bibliography} in the Metamath program for more
information.
% \index{\texttt{\char`\[}\ldots\texttt{]} inside comments}
See also Sections~\ref{tcomment} and \ref{wrbib}.

The \texttt{[}{\em author}\texttt{]} notation will also create an entry in
the bibliography cross-reference file generated by \texttt{write
bibliography} (Section~\ref{wrbib}) for {\sc HTML}.
For this to work properly, the
surrounding comment must be formatted as follows:
\begin{quote}
    {\em keyword} {\em label} {\em noise-word}
     \texttt{[}{\em author}\texttt{] p.} {\em number}
\end{quote}
for example
\begin{verbatim}
     Theorem 5.2 of [Monk] p. 223
\end{verbatim}
The {\em keyword} is not case sensitive and must be one of the following:
\begin{verbatim}
     theorem lemma definition compare proposition corollary
     axiom rule remark exercise problem notation example
     property figure postulate equation scheme chapter
\end{verbatim}
The optional {\em label} may consist of more than one
(non-{\em keyword} and non-{\em noise-word}) word.
The optional {\em noise-word} is one of:
\begin{verbatim}
     of in from on
\end{verbatim}
and is  ignored when the cross-reference file is created.  The
\texttt{write
biblio\-graphy} command will perform error checking to verify the
above format.\index{error checking}

\subsubsection{Parentheticals}\label{parentheticals}

The end of a comment may include one or more parenthicals, that is,
statements enclosed in parentheses.
The Metamath program looks for certain parentheticals and can issue
warnings based on them.
They are:

\begin{itemize}
 \item[] \texttt{(Contributed by }
   \textit{NAME}\texttt{,} \textit{DATE}\texttt{.)} -
   document the original contributor's name and the date it was created.
 \item[] \texttt{(Revised by }
   \textit{NAME}\texttt{,} \textit{DATE}\texttt{.)} -
   document the contributor's name and creation date
   that resulted in significant revision
   (not just an automated minimization or shortening).
 \item[] \texttt{(Proof shortened by }
   \textit{NAME}\texttt{,} \textit{DATE}\texttt{.)} -
   document the contributor's name and date that developed a significant
   shortening of the proof (not just an automated minimization).
 \item[] \texttt{(Proof modification is discouraged.)} -
   Note that this proof should normally not be modified.
 \item[] \texttt{(New usage is discouraged.)} -
   Note that this assertion should normally not be used.
\end{itemize}

The \textit{DATE} must be in form YYYY-MMM-DD, where MMM is the
English abbreviation of that month.

\subsubsection{Other markup}\label{othermarkup}
\index{markup notation}

There are other markup notations for generating good-looking results
beyond math mode and label mode:

\begin{itemize}
 \item[]
         \texttt{\char`\_} (underscore)\index{\texttt{\char`\_} inside comments} -
             Italicize text starting from
              {\em space}\texttt{\char`\_}{\em non-space} (i.e.\ \texttt{\char`\_}
              with a space before it and a non-space character after it) until
             the next
             {\em non-space}\texttt{\char`\_}{\em space}.  Normal
             punctuation (e.g.\ a trailing
             comma or period) is ignored when determining {\em space}.
 \item[]
         \texttt{\char`\_} (underscore) - {\em
         non-space}\texttt{\char`\_}{\em non-space-string}, where
          {\em non-space-string} is a string of non-space characters,
         will make {\em non-space-string} become a subscript.
 \item[]
         \texttt{<HTML>}...\texttt{</HTML>} - do not convert
         ``\texttt{<}'' and ``\texttt{>}''
         in the enclosed text when generating {\sc HTML},
         otherwise process markup normally. This allows direct insertion
         of {\sc html} commands.
 \item[]
       ``\texttt{\&}ref\texttt{;}'' - insert an {\sc HTML}
         character reference.
         This is how to insert arbitrary Unicode characters
         (such as accented characters).  Currently only directly supported
         when generating {\sc HTML}.
\end{itemize}

It is recommended that spaces surround any \texttt{\char`\~} and
\texttt{`} tokens in the comment and that a space follow the {\em label}
after a \texttt{\char`\~} token.  This will make global substitutions
to change labels and symbol names much easier and also eliminate any
future chance of ambiguity.  Spaces around these tokens are automatically
removed in the final output to conform with normal rules of punctuation;
for example, a space between a trailing \texttt{`} and a left parenthesis
will be removed.

A good way to become familiar with the markup notation is to look at
the extensive examples in the \texttt{set.mm} database.

\subsection{The Typesetting Comment (\texttt{\$t})}\label{tcomment}

The typesetting comment \texttt{\$t} in the input database file
provides the information necessary to produce good-looking results.
It provides \LaTeX\ and {\sc html}
definitions for math symbols,
as well supporting as some
customization of the generated web page.
If you add a new token to a database, you should also
update the \texttt{\$t} comment information if you want to eventually
create output in \LaTeX\ or {\sc HTML}.
See the
\texttt{set.mm}\index{set theory database (\texttt{set.mm})} database
file for an extensive example of a \texttt{\$t} comment illustrating
many of the features described below.

Programs that do not need to generate good-looking presentation results,
such as programs that only verify Metamath databases,
can completely ignore typesetting comments
and just treat them as normal comments.
Even the Metamath program only consults the
\texttt{\$t} comment information when it needs to generate typeset output
in \LaTeX\ or {\sc HTML}
(e.g., when you open a \LaTeX\ output file with the \texttt{open tex} command).

We will first discuss the syntax of typesetting comments, and then
briefly discuss how this can be used within the Metamath program.

\subsubsection{Typesetting Comment Syntax Overview}

The typesetting comment is identified by the token
\texttt{\$t}\index{\texttt{\$t} comment}\index{typesetting comment} in
the comment, and the typesetting comment ends at the matching
\texttt{\$)}:
\[
  \mbox{\tt \$(\ }
  \mbox{\tt \$t\ }
  \underbrace{
    \mbox{\tt \ \ \ \ \ \ \ \ \ \ \ }
    \cdots
    \mbox{\tt \ \ \ \ \ \ \ \ \ \ \ }
  }_{\mbox{Typesetting definitions go here}}
  \mbox{\tt \ \$)}
\]

There must be one or more white space characters, and only white space
characters, between the \texttt{\$(} that starts the comment
and the \texttt{\$t} symbol,
and the \texttt{\$t} must be followed by one
or more white space characters
(see section \ref{whitespace} for the definition of white space characters).
The typesetting comment continues until the comment end token \texttt{\$)}
(which must be preceded by one or more white space characters).

In version 0.07.30\index{Metamath!limitations of version 0.07.30} of the
Metamath program, there may be only one \texttt{\$t} comment in a
database.  This restriction may be lifted in the future to allow
many \texttt{\$t} comments in a database.

Between the \texttt{\$t} symbol (and its following white space) and the
comment end token \texttt{\$)} (and its preceding white space)
is a sequence of one or more typesetting definitions, where
each definition has the form
\textit{definition-type arg arg ... ;}.
Each of the zero or more \textit{arg} values
can be either a typesetting data or a keyword
(what keywords are allowed, and where, depends on the specific
\textit{definition-type}).
The \textit{definition-type}, and each argument \textit{arg},
are separated by one or more white space characters.
Every definition ends in an unquoted semicolon;
white space is not required before the terminating semicolon of a definition.
Each definition should start on a new line.\footnote{This
restriction of the current version of Metamath
(0.07.30)\index{Metamath!limitations of version 0.07.30} may be removed
in a future version, but you should do it anyway for readability.}

For example, this typesetting definition:
\begin{center}
 \verb$latexdef "C_" as "\subseteq";$
\end{center}
defines the token \verb$C_$ as the \LaTeX\ symbol $\subseteq$ (which means
``subset'').

Typesetting data is a sequence of one or more quoted strings
(if there is more than one, they are connected by \texttt{\char`\+}).
Often a single quoted string is used to provide data for a definition, using
either double (\texttt{\char`\"}) or single (\texttt{'}) quotation marks.
However,
{\em a quoted string (enclosed in quotation marks) may not include
line breaks.}
A quoted string
may include a quotation mark that matches the enclosing quotes by repeating
the quotation mark twice.  Here are some examples:

\begin{tabu}   { l l }
\textbf{Example} & \textbf{Meaning} \\
\texttt{\char`\"a\char`\"\char`\"b\char`\"} & \texttt{a\char`\"b} \\
\texttt{'c''d'} & \texttt{c'd} \\
\texttt{\char`\"e''f\char`\"} & \texttt{e''f} \\
\texttt{'g\char`\"\char`\"h'} & \texttt{g\char`\"\char`\"h} \\
\end{tabu}

Finally, a long quoted string
may be broken up into multiple quoted strings (considered, as a whole,
a single quoted string) and joined with \texttt{\char`\+}.
You can even use multiple lines as long as a
'+' is at the end of every line except the last one.
The \texttt{\char`\+} should be preceded and followed by at least one
white space character.
Thus, for example,
\begin{center}
 \texttt{\char`\"ab\char`\"\ \char`\+\ \char`\"cd\char`\"
    \ \char`\+\ \\ 'ef'}
\end{center}
is the same as
\begin{center}
 \texttt{\char`\"abcdef\char`\"}
\end{center}

{\sc c}-style comments \texttt{/*}\ldots\texttt{*/} are also supported.

In practice, whenever you add a new math token you will often want to add
typesetting definitions using
\texttt{latexdef}, \texttt{htmldef}, and
\texttt{althtmldef}, as described below.
That way, they will all be up to date.
Of course, whether or not you want to use all three definitions will
depend on how the database is intended to be used.

Below we discuss the different possible \textit{definition-kind} options.
We will show data surrounded by double quotes (in practice they can also use
single quotes and/or be a sequence joined by \texttt{+}s).
We will use specific names for the \textit{data} to make clear what
the data is used for, such as
{\em math-token} (for a Metamath math token,
{\em latex-string} (for string to be placed in a \LaTeX\ stream),
{\em {\sc html}-code} (for {\sc html} code),
and {\em filename} (for a filename).

\subsubsection{Typesetting Comment - \LaTeX}

The syntax for a \LaTeX\ definition is:
\begin{center}
 \texttt{latexdef "}{\em math-token}\texttt{" as "}{\em latex-string}\texttt{";}
\end{center}
\index{latex definitions@\LaTeX\ definitions}%
\index{\texttt{latexdef} statement}

The {\em token-string} and {\em latex-string} are the data
(character strings) for
the token and the \LaTeX\ definition of the token, respectively,

These \LaTeX\ definitions are used by the Metamath program
when it is asked to product \LaTeX output using
the \texttt{write tex} command.

\subsubsection{Typesetting Comment - {\sc html}}

The key kinds of {\sc HTML} definitions have the following syntax:

\vskip 1ex
    \texttt{htmldef "}{\em math-token}\texttt{" as "}{\em
    {\sc html}-code}\texttt{";}\index{\texttt{htmldef} statement}
                    \ \ \ \ \ \ldots

    \texttt{althtmldef "}{\em math-token}\texttt{" as "}{\em
{\sc html}-code}\texttt{";}\index{\texttt{althtmldef} statement}

                    \ \ \ \ \ \ldots

Note that in {\sc HTML} there are two possible definitions for math tokens.
This feature is useful when
an alternate representation of symbols is desired, for example one that
uses Unicode entities and another uses {\sc gif} images.

There are many other typesetting definitions that can control {\sc HTML}.
These include:

\vskip 1ex

    \texttt{htmldef "}{\em math-token}\texttt{" as "}{\em {\sc
    html}-code}\texttt{";}

    \texttt{htmltitle "}{\em {\sc html}-code}\texttt{";}%
\index{\texttt{htmltitle} statement}

    \texttt{htmlhome "}{\em {\sc html}-code}\texttt{";}%
\index{\texttt{htmlhome} statement}

    \texttt{htmlvarcolor "}{\em {\sc html}-code}\texttt{";}%
\index{\texttt{htmlvarcolor} statement}

    \texttt{htmlbibliography "}{\em filename}\texttt{";}%
\index{\texttt{htmlbibliography} statement}

\vskip 1ex

\noindent The \texttt{htmltitle} is the {\sc html} code for a common
title, such as ``Metamath Proof Explorer.''  The \texttt{htmlhome} is
code for a link back to the home page.  The \texttt{htmlvarcolor} is
code for a color key that appears at the bottom of each proof.  The file
specified by {\em filename} is an {\sc html} file that is assumed to
have a \texttt{<A NAME=}\ldots\texttt{>} tag for each bibiographic
reference in the database comments.  For example, if
\texttt{[Monk]}\index{\texttt{\char`\[}\ldots\texttt{]} inside comments}
occurs in the comment for a theorem, then \texttt{<A NAME='Monk'>} must
be present in the file; if not, a warning message is given.

Associated with
\texttt{althtmldef}
are the statements
\vskip 1ex

    \texttt{htmldir "}{\em
      directoryname}\texttt{";}\index{\texttt{htmldir} statement}

    \texttt{althtmldir "}{\em
     directoryname}\texttt{";}\index{\texttt{althtmldir} statement}

\vskip 1ex
\noindent giving the directories of the {\sc gif} and Unicode versions
respectively; their purpose is to provide cross-linking between the
two versions in the generated web pages.

When two different types of pages need to be produced from a single
database, such as the Hilbert Space Explorer that extends the Metamath
Proof Explorer, ``extended'' variables may be declared in the
\texttt{\$t} comment:
\vskip 1ex

    \texttt{exthtmltitle "}{\em {\sc html}-code}\texttt{";}%
\index{\texttt{exthtmltitle} statement}

    \texttt{exthtmlhome "}{\em {\sc html}-code}\texttt{";}%
\index{\texttt{exthtmlhome} statement}

    \texttt{exthtmlbibliography "}{\em filename}\texttt{";}%
\index{\texttt{exthtmlbibliography} statement}

\vskip 1ex
\noindent When these are declared, you also must declare
\vskip 1ex

    \texttt{exthtmllabel "}{\em label}\texttt{";}%
\index{\texttt{exthtmllabel} statement}

\vskip 1ex \noindent that identifies the database statement where the
``extended'' section of the database starts (in our example, where the
Hilbert Space Explorer starts).  During the generation of web pages for
that starting statement and the statements after it, the {\sc html} code
assigned to \texttt{exthtmltitle} and \texttt{exthtmlhome} is used
instead of that assigned to \texttt{htmltitle} and \texttt{htmlhome},
respectively.

\begin{sloppy}
\subsection{Additional Information Com\-ment (\texttt{\$j})} \label{jcomment}
\end{sloppy}

The additional information comment, aka the
\texttt{\$j}\index{\texttt{\$j} comment}\index{additional information comment}
comment,
provides a way to add additional structured information that can
be optionally parsed by systems.

The additional information comment is parsed the same way as the
typesetting comment (\texttt{\$t}) (see section \ref{tcomment}).
That is,
the additional information comment begins with the token
\texttt{\$j} within a comment,
and continues until the comment close \texttt{\$)}.
Within an additional information comment is a sequence of one or more
commands of the form \texttt{command arg arg ... ;}
where each of the zero or more \texttt{arg} values
can be either a quoted string or a keyword.
Note that every command ends in an unquoted semicolon.
If a verifier is parsing an additional information comment, but
doesn't recognize a particular command, it must skip the command
by finding the end of the command (an unquoted semicolon).

A database may have 0 or more additional information comments.
Note, however, that a verifier may ignore these comments entirely or only
process certain commands in an additional information comment.
The \texttt{mmj2} verifier supports many commands in additional information
comments.
We encourage systems that process additional information comments
to coordinate so that they will use the same command for the same effect.

Examples of additional information comments with various commands
(from the \texttt{set.mm} database) are:

\begin{itemize}
   \item Define the syntax and logical typecodes,
     and declare that our grammar is
     unambiguous (verifiable using the KLR parser, with compositing depth 5).
\begin{verbatim}
  $( $j
    syntax 'wff';
    syntax '|-' as 'wff';
    unambiguous 'klr 5';
  $)
\end{verbatim}

   \item Register $\lnot$ and $\rightarrow$ as primitive expressions
           (lacking definitions).
\begin{verbatim}
  $( $j primitive 'wn' 'wi'; $)
\end{verbatim}

   \item There is a special justification for \texttt{df-bi}.
\begin{verbatim}
  $( $j justification 'bijust' for 'df-bi'; $)
\end{verbatim}

   \item Register $\leftrightarrow$ as an equality for its type (wff).
\begin{verbatim}
  $( $j
    equality 'wb' from 'biid' 'bicomi' 'bitri';
    definition 'dfbi1' for 'wb';
  $)
\end{verbatim}

   \item Theorem \texttt{notbii} is the congruence law for negation.
\begin{verbatim}
  $( $j congruence 'notbii'; $)
\end{verbatim}

   \item Add \texttt{setvar} as a typecode.
\begin{verbatim}
  $( $j syntax 'setvar'; $)
\end{verbatim}

   \item Register $=$ as an equality for its type (\texttt{class}).
\begin{verbatim}
  $( $j equality 'wceq' from 'eqid' 'eqcomi' 'eqtri'; $)
\end{verbatim}

\end{itemize}


\subsection{Including Other Files in a Metamath Source File} \label{include}
\index{\texttt{\$[} and \texttt{\$]} auxiliary keywords}

The keywords \texttt{\$[} and \texttt{\$]} specify a file to be
included\index{included file}\index{file inclusion} at that point in a
Metamath\index{Metamath} source file\index{source file}.  The syntax for
including a file is as follows:
\begin{center}
\texttt{\$[} {\em file-name} \texttt{\$]}
\end{center}

The {\em file-name} should be a single token\index{token} with the same syntax
as a math symbol (i.e., all 93 non-whitespace
printable characters other than \texttt{\$} are
allowed, subject to the file-naming limitations of your operating system).
Comments may appear between the \texttt{\$[} and \texttt{\$]} keywords.  Included
files may include other files, which may in turn include other files, and so
on.

For example, suppose you want to use the set theory database as the starting
point for your own theory.  The first line in your file could be
\begin{center}
\texttt{\$[ set.mm \$]}
\end{center} All of the information (axioms, theorems,
etc.) in \texttt{set.mm} and any files that {\em it} includes will become
available for you to reference in your file. This can help make your work more
modular. A drawback to including files is that if you change the name of a
symbol or the label of a statement, you must also remember to update any
references in any file that includes it.


The naming conventions for included files are the same as those of your
operating system.\footnote{On the Macintosh, prior to Mac OS X,
 a colon is used to separate disk
and folder names from your file name.  For example, {\em volume}\texttt{:}{\em
file-name} refers to the root directory, {\em volume}\texttt{:}{\em
folder-name}\texttt{:}{\em file-name} refers to a folder in root, and {\em
volume}\texttt{:}{\em folder-name}\texttt{:}\ldots\texttt{:}{\em file-name} refers to a
deeper folder.  A simple {\em file-name} refers to a file in the folder from
which you launch the Metamath application.  Under Mac OS X and later,
the Metamath program is run under the Terminal application, which
conforms to Unix naming conventions.}\index{Macintosh file
names}\index{file names!Macintosh}\label{includef} For compatibility among
operating systems, you should keep the file names as simple as possible.  A
good convention to use is {\em file}\texttt{.mm} where {\em file} is eight
characters or less, in lower case.

There is no limit to the nesting depth of included files.  One thing that you
should be aware of is that if two included files themselves include a common
third file, only the {\em first} reference to this common file will be read
in.  This allows you to include two or more files that build on a common
starting file without having to worry about label and symbol conflicts that
would occur if the common file were read in more than once.  (In fact, if a
file includes itself, the self-reference will be ignored, although of course
it would not make any sense to do that.)  This feature also means, however,
that if you try to include a common file in several inner blocks, the result
might not be what you expect, since only the first reference will be replaced
with the included file (unlike the include statement in most other computer
languages).  Thus you would normally include common files only in the
outermost block\index{outermost block}.

\subsection{Compressed Proof Format}\label{compressed1}\index{compressed
proof}\index{proof!compressed}

The proof notation presented in Section~\ref{proof} is called a
{\bf normal proof}\index{normal proof}\index{proof!normal} and in principle is
sufficient to express any proof.  However, proofs often contain steps and
subproofs that are identical.  This is particularly true in typical
Metamath\index{Metamath} applications, because Metamath requires that the math
symbol sequence (usually containing a formula) at each step be separately
constructed, that is, built up piece by piece. As a result, a lot of
repetition often results.  The {\bf compressed proof} format allows Metamath
to take advantage of this redundancy to shorten proofs.

The specification for the compressed proof format is given in
Appen\-dix~\ref{compressed}.

Normally you need not concern yourself with the details of the compressed
proof format, since the Metamath program will allow you to convert from
the normal format to the compressed format with ease, and will also
automatically convert from the compressed format when proofs are displayed.
The overall structure of the compressed format is as follows:
\begin{center}
  \texttt{\$= ( } {\em label-list} \texttt{) } {\em compressed-proof\ }\ \texttt{\$.}
\end{center}
\index{\texttt{\$=} keyword}
The first \texttt{(} serves as a flag to Metamath that a compressed proof
follows.  The {\em label-list} includes all statements referred to by the
proof except the mandatory hypotheses\index{mandatory hypothesis}.  The {\em
compressed-proof} is a compact encoding of the proof, using upper-case
letters, and can be thought of as a large integer in base 26.  White
space\index{white space} inside a {\em compressed-proof} is
optional and is ignored.

It is important to note that the order of the mandatory hypotheses of
the statement being proved must not be changed if the compressed proof
format is used, otherwise the proof will become incorrect.  The reason
for this is that the mandatory hypotheses are not mentioned explicitly
in the compressed proof in order to make the compression more efficient.
If you wish to change the order of mandatory hypotheses, you must first
convert the proof back to normal format using the \texttt{save proof
{\em statement} /normal}\index{\texttt{save proof} command} command.
Later, you can go back to compressed format with \texttt{save proof {\em
statement} /compressed}.

During error checking with the \texttt{verify proof} command, an error
found in a compressed proof may point to a character in {\em
compressed-proof}, which may not be very meaningful to you.  In this
case, try to \texttt{save proof /normal} first, then do the
\texttt{verify proof} again.  In general, it is best to make sure a
proof is correct before saving it in compressed format, because severe
errors are less likely to be recoverable than in normal format.

\subsection{Specifying Unknown Proofs or Subproofs}\label{unknown}

In a proof under development, any step or subproof that is not yet known
may be represented with a single \texttt{?}.  For the purposes of
parsing the proof, the \texttt{?}\ \index{\texttt{]}@\texttt{?}\ inside
proofs} will push a single entry onto the RPN stack just as if it were a
hypothesis.  While developing a proof with the Proof
Assistant\index{Proof Assistant}, a partially developed proof may be
saved with the \texttt{save new{\char`\_}proof}\index{\texttt{save
new{\char`\_}proof} command} command, and \texttt{?}'s will be placed at
the appropriate places.

All \texttt{\$p}\index{\texttt{\$p} statement} statements must have
proofs, even if they are entirely unknown.  Before creating a proof with
the Proof Assistant, you should specify a completely unknown proof as
follows:
\begin{center}
  {\em label} \texttt{\$p} {\em statement} \texttt{\$= ?\ \$.}
\end{center}
\index{\texttt{\$=} keyword}
\index{\texttt{]}@\texttt{?}\ inside proofs}

The \texttt{verify proof}\index{\texttt{verify proof} command} command
will check the known portions of a partial proof for errors, but will
warn you that the statement has not been proved.

Note that partially developed proofs may be saved in compressed format
if desired.  In this case, you will see one or more \texttt{?}'s in the
{\em compressed-proof} part.\index{compressed
proof}\index{proof!compressed}

\section{Axioms vs.\ Definitions}\label{definitions}

The \textit{basic}
Metamath\index{Metamath} language and program
make no distinction\index{axiom vs.\
definition} between axioms\index{axiom} and
definitions.\index{definition} The \texttt{\$a}\index{\texttt{\$a}
statement} statement is used for both.  At first, this may seem
puzzling.  In the minds of many mathematicians, the distinction is
clear, even obvious, and hardly worth discussing.  A definition is
considered to be merely an abbreviation that can be replaced by the
expression for which it stands; although unless one actually does this,
to be precise then one should say that a theorem\index{theorem} is a
consequence of the axioms {\em and} the definitions that are used in the
formulation of the theorem \cite[p.~20]{Behnke}.\index{Behnke, H.}

\subsection{What is a Definition?}

What is a definition?  In its simplest form, a definition introduces a new
symbol and provides an unambiguous rule to transform an expression containing
the new symbol to one without it.  The concept of a ``proper
definition''\index{proper definition}\index{definition!proper} (as opposed to
a creative definition)\index{creative definition}\index{definition!creative}
that is usually agreed upon is (1) the definition should not strengthen the
language and (2) any symbols introduced by the definition should be eliminable
from the language \cite{Nemesszeghy}\index{Nemesszeghy, E. Z.}.  In other
words, they are mere typographical conveniences that do not belong to the
system and are theoretically superfluous.  This may seem obvious, but in fact
the nature of definitions can be subtle, sometimes requiring difficult
metatheorems to establish that they are not creative.

A more conservative stance was taken by logician S.
Le\'{s}niewski.\index{Le\'{s}niewski, S.}
\begin{quote}
Le\'{s}niewski
regards definitions as theses of the system.  In this respect they do
not differ either from the axioms or from theorems, i.e.\ from the
theses added to the system on the basis of the rule of substitution or
the rule of detachment [modus ponens].  Once definitions have been
accepted as theses of the system, it becomes necessary to consider them
as true propositions in the same sense in which axioms are true
\cite{Lejewski}.
\end{quote}\index{Lejewski, Czeslaw}

Let us look at some simple examples of definitions in propositional
calculus.  Consider the definition of logical {\sc or}
(disjunction):\index{disjunction ($\vee$)} ``$P\vee Q$ denotes $\neg P
\rightarrow Q$ (not $P$ implies $Q$).''  It is very easy to recognize a
statement making use of this definition, because it introduces the new
symbol $\vee$ that did not previously exist in the language.  It is easy
to see that no new theorems of the original language will result from
this definition.

Next, consider a definition that eliminates parentheses:  ``$P
\rightarrow Q\rightarrow R$ denotes $P\rightarrow (Q \rightarrow R)$.''
This is more subtle, because no new symbols are introduced.  The reason
this definition is considered proper is that no new symbol sequences
that are valid wffs (well-formed formulas)\index{well-formed formula
(wff)} in the original language will result from the definition, since
``$P \rightarrow Q\rightarrow R$'' is not a wff in the original
language.  Here, we implicitly make use of the fact that there is a
decision procedure that allows us to determine whether or not a symbol
sequence is a wff, and this fact allows us to use symbol sequences that
are not wffs to represent other things (such as wffs) by means of the
definition.  However, to justify the definition as not being creative we
need to prove that ``$P \rightarrow Q\rightarrow R$'' is in fact not a
wff in the original language, and this is more difficult than in the
case where we simply introduce a new symbol.

%Now let's take this reasoning to an extreme.  Propositional calculus is a
%decidable theory,\footnote{This means that a mechanical algorithm exists to
%determine whether or not a wff is a theorem.} so in principle we could make use
%of symbol sequences that are not theorems to represent other things (say, to
%encode actual theorems in a more compact way).  For example, let us extend the
%language by defining a wff ``$P$'' in the extended language as the theorem
%``$P\rightarrow P$''\footnote{This is one of the first theorems proved in the
%Metamath database \texttt{set.mm}.}\index{set
%theory database (\texttt{set.mm})} in the original language whenever ``$P$'' is
%not a theorem in the original language.  In the extended language, any wff
%``$Q$'' thus represents a theorem; to find out what theorem (in the original
%language) ``$Q$'' represents, we determine whether ``$Q$'' is a theorem in the
%original language (before the definition was introduced).  If so, we're done; if
%not, we replace ``$Q$'' by ``$Q\rightarrow Q$'' to eliminate the definition.
%This definition is therefore eliminable, and it does not ``strengthen'' the
%language because any wff that is not a theorem is not in the set of statements
%provable in the original language and thus is available for use by definitions.
%
%Of course, a definition such as this would render practically useless the
%communication of theorems of propositional calculus; but
%this is just a human shortcoming, since we can't always easily discern what is
%and is not a theorem by inspection.  In fact, the extended theory with this
%definition has no more and no less information than the original theory; it just
%expresses certain theorems of the form ``$P\rightarrow P$''
%in a more compact way.
%
%The point here is that what constitutes a proper definition is a matter of
%judgment about whether a symbol sequence can easily be recognized by a human
%as invalid in some sense (for example, not a wff); if so, the symbol sequence
%can be appropriated for use by a definition in order to make the extended
%language more compact.  Metamath\index{Metamath} lacks the ability to make this
%judgment, since as far as Metamath is concerned the definition of a wff, for
%example, is arbitrary.  You define for Metamath how wffs\index{well-formed
%formula (wff)} are constructed according to your own preferred style.  The
%concept of a wff may not even exist in a given formal system\index{formal
%system}.  Metamath treats all definitions as if they were new axioms, and it
%is up to the human mathematician to judge whether the definition is ``proper''
%'\index{proper definition}\index{definition!proper} in some agreed-upon way.

What constitutes a definition\index{definition} versus\index{axiom vs.\
definition} an axiom\index{axiom} is sometimes arbitrary in mathematical
literature.  For example, the connectives $\vee$ ({\sc or}), $\wedge$
({\sc and}), and $\leftrightarrow$ (equivalent to) in propositional
calculus are usually considered defined symbols that can be used as
abbreviations for expressions containing the ``primitive'' connectives
$\rightarrow$ and $\neg$.  This is the way we treat them in the standard
logic and set theory database \texttt{set.mm}\index{set theory database
(\texttt{set.mm})}.  However, the first three connectives can also be
considered ``primitive,'' and axiom systems have been devised that treat
all of them as such.  For example,
\cite[p.~35]{Goodstein}\index{Goodstein, R. L.} presents one with 15
axioms, some of which in fact coincide with what we have chosen to call
definitions in \texttt{set.mm}.  In certain subsets of classical
propositional calculus, such as the intuitionist
fragment\index{intuitionism}, it can be shown that one cannot make do
with just $\rightarrow$ and $\neg$ but must treat additional connectives
as primitive in order for the system to make sense.\footnote{Two nice
systems that make the transition from intuitionistic and other weak
fragments to classical logic just by adding axioms are given in
\cite{Robinsont}\index{Robinson, T. Thacher}.}

\subsection{The Approach to Definitions in \texttt{set.mm}}

In set theory, recursive definitions define a newly introduced symbol in
terms of itself.  The justification of recursive definitions, using
several ``recursion theorems,'' is usually one of the first
sophisticated proofs a student encounters when learning set theory, and
there is a significant amount of implicit metalogic behind a recursive
definition even though the definition itself is typically simple to
state.  It is, however, possible to substitute one kind of complexity
for another.  We can eliminate the need for metalogical justification by
defining the operation directly with an explicit (but complicated)
expression, then deriving the recursive definition directly as a
theorem, using a recursion theorem ``in reverse.''  We do this in
\texttt{set.mm}, as follows.

In \texttt{set.mm} our goal was to introduce almost all definitions in
the form of two expressions connected by either $\leftrightarrow$ or
$=$, where the thing being defined does not appear on the right hand
side.  Quine calls this form ``a genuine or direct definition'' \cite[p.
174]{Quine}\index{Quine, Willard Van Orman}, which makes the definitions
very easy to eliminate and the metalogic\index{metalogic} needed to
justify them as simple as possible.

We achieved this goal in almost all cases in \texttt{set.mm}.
Sometimes this makes the definitions more complex and less
intuitive.  For example, the traditional way to define addition of
natural numbers is to define an operation called {\em
successor}\index{successor} (which means ``plus one'' and is denoted by
``${\rm suc}$''), then define addition recursively\index{recursive
definition} with the two definitions $n + 0 = n$ and $m + {\rm suc}\,n =
{\rm suc} (m + n)$.  Although this definition seems simple and obvious,
the method to eliminate the definition is not obvious:  in the second
part of the definition, addition is defined in terms of itself.  By
eliminating the definition, we don't mean repeatedly applying it to
specific $m$ and $n$ but rather showing the explicit, closed-form
set-theoretical expression that $m + n$ represents, that will work for
any $m$ and $n$ and that does not have a $+$ sign on its right-hand
side.  For a recursive definition like this not to be circular
(creative), there are some hidden, underlying assumptions we must make,
for example that the natural numbers have a certain kind of order.  In
\texttt{set.mm} we chose to start with the direct (though complex and
nonintuitive) definition then derive from it the standard recursive
definition.\footnote{The closed-form definition used in \texttt{set.mm}
for the addition operation on ordinals\index{ordinal
addition}\index{addition!of ordinals} (of which natural numbers are a
subset) is

\setbox\startprefix=\hbox{\tt \ \ df-oadd\ \$a\ }
\setbox\contprefix=\hbox{\tt \ \ \ \ \ \ \ \ \ \ \ \ \ }
\startm
\m{\vdash}\m{+_o}\m{=}\m{\{}\m{\langle}\m{\langle}\m{x}\m{,}\m{y}\m{\rangle}%
\m{,}\m{z}\m{\rangle}\m{|}\m{(}\m{(}\m{x}\m{\in}\m{{\rm On}}\m{\wedge}\m{y}\m{%
\in}\m{{\rm On}}\m{)}\m{\wedge}\m{z}\m{=}\m{(}\m{{\rm rec}}\m{(}\m{\{}\m{%
\langle}\m{w}\m{,}\m{v}\m{\rangle}\m{|}\m{v}\m{=}\m{{\rm suc}}\m{w}\m{\}}\m{,}%
\m{x}\m{)}\m{`}\m{y}\m{)}\m{)}\m{\}}
\endm

\noindent Here, the abstraction class of nested ordered pairs\index{abstraction
class!of nested ordered pairs} is defined by \texttt{df-oprab} in \texttt{set.mm},
and ${\rm rec}$ is a ``recursion operator''\index{recursion operator} with
the definition

\setbox\startprefix=\hbox{\tt \ \ df-rfg\ \$a\ }
\setbox\contprefix=\hbox{\tt \ \ \ \ \ \ \ \ \ \ \ \ }
\startm
\m{\vdash}\m{{\rm rec}}\m{(}\m{F}\m{,}\m{A}\m{)}\m{=}\m{\bigcup}\m{\{}\m{f}%
\m{|}\m{\exists}\m{x}\m{(}\m{x}\m{\in}\m{{\rm On}}\m{\wedge}\m{(}\m{f}\m{{%
\rm Fn}}\m{x}\m{\wedge}\m{\forall}\m{y}\m{(}\m{y}\m{\in}\m{x}\m{\rightarrow}%
\m{(}\m{f}\m{`}\m{y}\m{)}\m{=}\m{(}\m{\{}\m{\langle}\m{g}\m{,}\m{z}\m{\rangle}%
\m{|}\m{(}\m{(}\m{g}\m{=}\m{\varnothing}\m{\wedge}\m{z}\m{=}\m{A}\m{)}\m{\vee}%
\m{(}\m{\lnot}\m{(}\m{g}\m{=}\m{\varnothing}\m{\vee}\m{{\rm Lim}}\m{{\rm dom}}%
\m{g}\m{)}\m{\wedge}\m{z}\m{=}\m{(}\m{F}\m{`}\m{(}\m{g}\m{`}\m{\bigcup}\m{{%
\rm dom}}\m{g}\m{)}\m{)}\m{)}\m{\vee}\m{(}\m{{\rm Lim}}\m{{\rm dom}}\m{g}\m{%
\wedge}\m{z}\m{=}\m{\bigcup}\m{{\rm ran}}\m{g}\m{)}\m{)}\m{\}}\m{`}\m{(}\m{f}%
\m{\restriction}\m{y}\m{)}\m{)}\m{)}\m{)}\m{)}\m{\}}
\endm

\noindent which can be further broken down with definitions shown in
Section~\ref{setdefinitions}.  You may be surprised at the complexity of what
seems like such a simple notion.  From these definitions the simpler, more
intuitive recursive definition is derived as a set of theorems.}\index{natural
number}\index{addition}\index{recursive definition}\index{ordinal addition}
The end result is the same, but we completely eliminate the rather complex
metalogic that justifies the recursive definition.  (For a mathematician,
recursive definitions are more efficient and intuitive than
direct ones once the metalogic has been learned or possibly just accepted as
correct.  However, it was felt that direct definition in \texttt{set.mm}
maximizes rigor by minimizing metalogic.  It can be eliminated effortlessly,
something difficult to do with a recursive definition.)

\subsection{Adding Constraints on Definitions}

The basic Metamath language and the Metamath program do
not have any built-in constraints on definitions, since they are just
\$a statements.

However, nothing prevents a verification system from verifying
additional rules to impose further limitations on definitions.
For example, the \texttt{mmj2}\index{mmj2} program
supports various kinds of
additional information comments (see section \ref{jcomment}).
One of their uses is to optionally verify additional constraints,
including constraints to verify that definitions meet certain
requirements.
These additional checks are required by the
continuous integration (CI)\index{continuous integration (CI)}
checks of the
\texttt{set.mm}\index{set theory database (\texttt{set.mm})}%
\index{Metamath Proof Explorer}
database.
This approach enables us to optionally impose additional requirements
on definitions if we wish, without necessarily imposing those rules on
all databases or requiring all verification systems to implement them.
In addition, this allows us to impose specialized constraints tailored
to one database while not requiring other systems to implement
those specialized constraints.

We impose two constraints on the
\texttt{set.mm}\index{set theory database (\texttt{set.mm})}%
\index{Metamath Proof Explorer} database
via the \texttt{mmj2}\index{mmj2} program that are worth discussing here:
a parse check and a definition soundness check.

% On February 11, 2019 8:32:32 PM EST, saueran@oregonstate.edu wrote:
% The following addition to the end of set.mm is accepted by the mmj2
% parser and definition checker and the metamath verifier(at least it was
% when I checked, you should check it too), and creates a contradiction by
% proving the theorem |- ph.
% ${
% wleftp $a wff ( ( ph ) $.
% wbothp $a wff ( ph ) $.
% df-leftp $a |- ( ( ( ph ) <-> -. ph ) $.
% df-bothp $a |- ( ( ph ) <-> ph ) $.
% anything $p |- ph $=
%   ( wbothp wn wi wleftp df-leftp biimpi df-bothp mpbir mpbi simplim ax-mp)
%   ABZAMACZDZCZMOEZOCQAEZNDZRNAFGSHIOFJMNKLAHJ $.
% $}
%
% This particular problem is countered by enabling, within mmj2,
% SetParser,mmj.verify.LRParser

First,
we enable a parse check in \texttt{mmj2} (through its
\texttt{SetParser} command) that requires that all new definitions
must \textit{not} create an ambiguous parse for a KLR(5) parser.
This prevents some errors such as definitions with imbalanced parentheses.

Second, we run a definition soundness check specific to
\texttt{set.mm} or databases similar to it.
(through the \texttt{definitionCheck} macro).
Some \texttt{\$a} statements (including all ax-* statemnets)
are excluded from these checks, as they will
always fail this simple check,
but they are appropriate for most definitions.
This check imposes a set of additional rules:

\begin{enumerate}

\item New definitions must be introduced using $=$ or $\leftrightarrow$.

\item No \texttt{\$a} statement introduced before this one is allowed to use the
symbol being defined in this definition, and the definition is not
allowed to use itself (except once, in the definiendum).

\item Every variable in the definiens should not be distinct

\item Every dummy variable in the definiendum
are required to be distinct from each other and from variables in
the definiendum.
To determine this, the system will look for a "justification" theorem
in the database, and if it is not there, attempt to briefly prove
$( \varphi \rightarrow \forall x \varphi )$  for each dummy variable x.

\item Every dummy variable should be a set variable,
unless there is a justification theorem available.

\item Every dummy variable must be bound
(if the system cannot determine this a justification theorem must be
provided).

\end{enumerate}

\subsection{Summary of Approach to Definitions}

In short, when being rigorous it turns out that
definitions can be subtle, sometimes requiring difficult
metatheorems to establish that they are not creative.

Instead of building such complications into the Metamath language itself,
the basic Metmath language and program simply treat traditional
axioms and definitions as the same kind of \texttt{\$a} statement.
We have then built various tools to enable people to
verify additional conditions as their creators believe is appropriate
for those specific databases, without complicating the Metamath foundations.

\chapter{The Metamath Program}\label{commands}

This chapter provides a reference manual for the
Metamath program.\index{Metamath!commands}

Current instructions for obtaining and installing the Metamath program
can be found at the \url{http://metamath.org} web site.
For Windows, there is a pre-compiled version called
\texttt{metamath.exe}.  For Unix, Linux, and Mac OS X
(which we will refer to collectively as ``Unix''), the Metamath program
can be compiled from its source code with the command
\begin{verbatim}
gcc *.c -o metamath
\end{verbatim}
using the \texttt{gcc} {\sc c} compiler available on those systems.

In the command syntax descriptions below, fields enclosed in square
brackets [\ ] are optional.  File names may be optionally enclosed in
single or double quotes.  This is useful if the file name contains
spaces or
slashes (\texttt{/}), such as in Unix path names, \index{Unix file
names}\index{file names!Unix} that might be confused with Metamath
command qualifiers.\index{Unix file names}\index{file names!Unix}

\section{Invoking Metamath}

Unix, Linux, and Mac OS X
have a command-line interface called the {\em
bash shell}.  (In Mac OS X, select the Terminal application from
Applications/Utilities.)  To invoke Metamath from the bash shell prompt,
assuming that the Metamath program is in the current directory, type
\begin{verbatim}
bash$ ./metamath
\end{verbatim}

To invoke Metamath from a Windows DOS or Command Prompt, assuming that
the Metamath program is in the current directory (or in a directory
included in the Path system environment variable), type
\begin{verbatim}
C:\metamath>metamath
\end{verbatim}

To use command-line arguments at invocation, the command-line arguments
should be a list of Metamath commands, surrounded by quotes if they
contain spaces.  In Windows, the surrounding quotes must be double (not
single) quotes.  For example, to read the database file \texttt{set.mm},
verify all proofs, and exit the program, type (under Unix)
\begin{verbatim}
bash$ ./metamath 'read set.mm' 'verify proof *' exit
\end{verbatim}
Note that in Unix, any directory path with \texttt{/}'s must be
surrounded by quotes so Metamath will not interpret the \texttt{/} as a
command qualifier.  So if \texttt{set.mm} is in the \texttt{/tmp}
directory, use for the above example
\begin{verbatim}
bash$ ./metamath 'read "/tmp/set.mm"' 'verify proof *' exit
\end{verbatim}

For convenience, if the command-line has one argument and no spaces in
the argument, the command is implicitly assumed to be \texttt{read}.  In
this one special case, \texttt{/}'s are not interpreted as command
qualifiers, so you don't need quotes around a Unix file name.  Thus
\begin{verbatim}
bash$ ./metamath /tmp/set.mm
\end{verbatim}
and
\begin{verbatim}
bash$ ./metamath "read '/tmp/set.mm'"
\end{verbatim}
are equivalent.


\section{Controlling Metamath}

The Metamath program was first developed on a {\sc vax/vms} system, and
some aspects of its command line behavior reflect this heritage.
Hopefully you will find it reasonably user-friendly once you get used to
it.

Each command line is a sequence of English-like words separated by
spaces, as in \texttt{show settings}.  Command words are not case
sensitive, and only as many letters are needed as are necessary to
eliminate ambiguity; for example, \texttt{sh se} would work for the
command \texttt{show settings}.  In some cases arguments such as file
names, statement labels, or symbol names are required; these are
case-sensitive (although file names may not be on some operating
systems).

A command line is entered by typing it in then pressing the {\em return}
({\em enter}) key.  To find out what commands are available, type
\texttt{?} at the \texttt{MM>} prompt.  To find out the choices at any
point in a command, press {\em return} and you will be prompted for
them.  The default choice (the one selected if you just press {\em
return}) is shown in brackets (\texttt{<>}).

You may also type \texttt{?} in place of a command word to force
Metamath to tell you what the choices are.  The \texttt{?} method won't
work, though, if a non-keyword argument such as a file name is expected
at that point, because the program will think that \texttt{?} is the
value of the argument.

Some commands have one or more optional qualifiers which modify the
behavior of the command.  Qualifiers are preceded by a slash
(\texttt{/}), such as in \texttt{read set.mm / verify}.  Spaces are
optional around the \texttt{/}.  If you need to use a space or
slash in a command
argument, as in a Unix file name, put single or double quotes around the
command argument.

The \texttt{open log} command will save everything you see on the
screen and is useful to help you recover should something go wrong in a
proof, or if you want to document a bug.

If a command responds with more than a screenful, you will be
prompted to \texttt{<return> to continue, Q to quit, or S to scroll to
end}.  \texttt{Q} or \texttt{q} (not case-sensitive) will complete the
command internally but will suppress further output until the next
\texttt{MM>} prompt.  \texttt{s} will suppress further pausing until the
next \texttt{MM>} prompt.  After the first screen, you are also
presented with the choice of \texttt{b} to go back a screenful.  Note
that \texttt{b} may also be entered at the \texttt{MM>} prompt
immediately after a command to scroll back through the output of that
command.

A command line enclosed in quotes is executed by your operating system.
See Section~\ref{oscmd}.

{\em Warning:} Pressing {\sc ctrl-c} will abort the Metamath program
unconditionally.  This means any unsaved work will be lost.


\subsection{\texttt{exit} Command}\index{\texttt{exit} command}

Syntax:  \texttt{exit} [\texttt{/force}]

This command exits from Metamath.  If there have been changes to the
source with the \texttt{save proof} or \texttt{save new{\char`\_}proof}
commands, you will be given an opportunity to \texttt{write source} to
permanently save the changes.

In Proof Assistant\index{Proof Assistant} mode, the \texttt{exit} command will
return to the \verb/MM>/ prompt. If there were changes to the proof, you will
be given an opportunity to \texttt{save new{\char`\_}proof}.

The \texttt{quit} command is a synonym for \texttt{exit}.

Optional qualifier:
    \texttt{/force} - Do not prompt if changes were not saved.  This qualifier is
        useful in \texttt{submit} command files (Section~\ref{sbmt})
        to ensure predictable behavior.





\subsection{\texttt{open log} Command}\index{\texttt{open log} command}
Syntax:  \texttt{open log} {\em file-name}

This command will open a log file that will store everything you see on
the screen.  It is useful to help recovery from a mistake in a long Proof
Assistant session, or to document bugs.\index{Metamath!bugs}

The log file can be closed with \texttt{close log}.  It will automatically be
closed upon exiting Metamath.



\subsection{\texttt{close log} Command}\index{\texttt{close log} command}
Syntax:  \texttt{close log}

The \texttt{close log} command closes a log file if one is open.  See
also \texttt{open log}.




\subsection{\texttt{submit} Command}\index{\texttt{submit} command}\label{sbmt}
Syntax:  \texttt{submit} {\em filename}

This command causes further command lines to be taken from the specified
file.  Note that any line beginning with an exclamation point (\texttt{!}) is
treated as a comment (i.e.\ ignored).  Also note that the scrolling
of the screen output is continuous, so you may want to open a log file
(see \texttt{open log}) to record the results that fly by on the screen.
After the lines in the file are exhausted, Metamath returns to its
normal user interface mode.

Currently, the \texttt{submit} command is not recursive.  In other words,
\texttt{submit} commands are not allowed inside a command file.


\subsection{\texttt{erase} Command}\index{\texttt{erase} command}
Syntax:  \texttt{erase}

This command will reset Metamath to its starting state, deleting any
data\-base that was \texttt{read} in.
 If there have been changes to the
source with the \texttt{save proof} or \texttt{save new{\char`\_}proof}
commands, you will be given an opportunity to \texttt{write source} to
permanently save the changes.



\subsection{\texttt{set echo} Command}\index{\texttt{set echo} command}
Syntax:  \texttt{set echo on} or \texttt{set echo off}

The \texttt{set echo on} command will cause command lines to be echoed with any
abbreviations expanded.  While learning the Metamath commands, this
feature will show you the exact command that your abbreviated input
corresponds to.



\subsection{\texttt{set scroll} Command}\index{\texttt{set scroll} command}
Syntax:  \texttt{set scroll prompted} or \texttt{set scroll continuous}

The Metamath command line interface starts off in the \texttt{prompted} mode,
which means that you will be prompted to continue or quit after each
full screen in a long listing.  In \texttt{continuous} mode, long listings will be
scrolled without pausing.

% LaTeX bug? (1) \texttt{\_} puts out different character than
% \texttt{{\char`\_}}
%  = \verb$_$  (2) \texttt{{\char`\_}} puts out garbage in \subsection
%  argument
\subsection{\texttt{set width} Command}\index{\texttt{set
width} command}
Syntax:  \texttt{set width} {\em number}

Metamath assumes the width of your screen is 79 characters (chosen
because the Command Prompt in Windows XP has a wrapping bug at column
80).  If your screen is wider or narrower, this command allows you to
change this default screen width.  A larger width is advantageous for
logging proofs to an output file to be printed on a wide printer.  A
smaller width may be necessary on some terminals; in this case, the
wrapping of the information messages may sometimes seem somewhat
unnatural, however.  In \LaTeX\index{latex@{\LaTeX}!characters per line},
there is normally a maximum of 61 characters per line with typewriter
font.  (The examples in this book were produced with 61 characters per
line.)

\subsection{\texttt{set height} Command}\index{\texttt{set
height} command}
Syntax:  \texttt{set height} {\em number}

Metamath assumes your screen height is 24 lines of characters.  If your
screen is taller or shorter, this command lets you to change the number
of lines at which the display pauses and prompts you to continue.

\subsection{\texttt{beep} Command}\index{\texttt{beep} command}

Syntax:  \texttt{beep}

This command will produce a beep.  By typing it ahead after a
long-running command has started, it will alert you that the command is
finished.  For convenience, \texttt{b} is an abbreviation for
\texttt{beep}.

Note:  If \texttt{b} is typed at the \texttt{MM>} prompt immediately
after the end of a multiple-page display paged with ``\texttt{Press
<return> for more}...'' prompts, then the \texttt{b} will back up to the
previous page rather than perform the \texttt{beep} command.
In that case you must type the unabbreviated \texttt{beep} form
of the command.

\subsection{\texttt{more} Command}\index{\texttt{more} command}

Syntax:  \texttt{more} {\em filename}

This command will display the contents of an {\sc ascii} file on your
screen.  (This command is provided for convenience but is not very
powerful.  See Section~\ref{oscmd} to invoke your operating system's
command to do this, such as the \texttt{more} command in Unix.)

\subsection{Operating System Commands}\index{operating system
command}\label{oscmd}

A line enclosed in single or double quotes will be executed by your
computer's operating system if it has a command line interface.  For
example, on a {\sc vax/vms} system,
\verb/MM> 'dir'/
will print disk directory contents.  Note that this feature will not
work on the Macintosh prior to Mac OS X, which does not have a command
line interface.

For your convenience, the trailing quote is optional.

\subsection{Size Limitations in Metamath}

In general, there are no fixed, predefined limits\index{Metamath!memory
limits} on how many labels, tokens\index{token}, statements, etc.\ that
you may have in a database file.  The Metamath program uses 32-bit
variables (64-bit on 64-bit CPUs) as indices for almost all internal
arrays, which are allocated dynamically as needed.



\section{Reading and Writing Files}

The following commands create new files:  the \texttt{open} commands;
the \texttt{write} commands; the \texttt{/html},
\texttt{/alt{\char`\_}html}, \texttt{/brief{\char`\_}html},
\texttt{/brief{\char`\_}alt{\char`\_}html} qualifiers of \texttt{show
statement}, and \texttt{midi}.  The following commands append to files
previously opened:  the \texttt{/tex} qualifier of \texttt{show proof}
and \texttt{show new{\char`\_}proof}; the \texttt{/tex} and
\texttt{/simple{\char`\_}tex} qualifiers of \texttt{show statement}; the
\texttt{close} commands; and all screen dialog between \texttt{open log}
and \texttt{close log}.

The commands that create new files will not overwrite an existing {\em
filename} but will rename the existing one to {\em
filename}\texttt{{\char`\~}1}.  An existing {\em
filename}\texttt{{\char`\~}1} is renamed {\em
filename}\texttt{{\char`\~}2}, etc.\ up to {\em
filename}\texttt{{\char`\~}9}.  An existing {\em
filename}\texttt{{\char`\~}9} is deleted.  This makes recovery from
mistakes easier but also will clutter up your directory, so occasionally
you may want to clean up (delete) these \texttt{{\char`\~}}$n$ files.


\subsection{\texttt{read} Command}\index{\texttt{read} command}
Syntax:  \texttt{read} {\em file-name} [\texttt{/verify}]

This command will read in a Metamath language source file and any included
files.  Normally it will be the first thing you do when entering Metamath.
Statement syntax is checked, but proof syntax is not checked.
Note that the file name may be enclosed in single or double quotes;
this is useful if the file name contains slashes, as might be the case
under Unix.

If you are getting an ``\texttt{?Expected VERIFY}'' error
when trying to read a Unix file name with slashes, you probably haven't
quoted it.\index{Unix file names}\index{file names!Unix}

If you are prompted for the file name (by pressing {\em return}
 after \texttt{read})
you should {\em not} put quotes around it, even if it is a Unix file name
with slashes.

Optional command qualifier:

    \texttt{/verify} - Verify all proofs as the database is read in.  This
         qualifier will slow down reading in the file.  See \texttt{verify
         proof} for more information on file error-checking.

See also \texttt{erase}.



\subsection{\texttt{write source} Command}\index{\texttt{write source} command}
Syntax:  \texttt{write source} {\em filename}
[\texttt{/rewrap}]
[\texttt{/split}]
[\texttt{/keep\_includes}]
[\texttt{/no\_versioning}]

This command will write the contents of a Metamath\index{database}
database into a file.\index{source file}

Optional command qualifiers:

\texttt{/rewrap} -
Reformats statements and comments according to the
convention used in the set.mm database.
It unwraps the
lines in the comment before each \texttt{\$a} and \texttt{\$p} statement,
then it
rewraps the line.  You should compare the output to the original
to make sure that the desired effect results; if not, go back to
the original source.  The wrapped line length honors the
\texttt{set width}
parameter currently in effect.  Note:  Text
enclosed in \texttt{<HTML>}...\texttt{</HTML>} tags is not modified by the
\texttt{/rewrap} qualifier.
Proofs themselves are not reformatted;
use \texttt{save proof * / compressed} to do that.
An isolated tilde (\~{}) is always kept on the same line as the following
symbol, so you can find all comment references to a symbol by
searching for \~{} followed by a space and that symbol
(this is useful for finding cross-references).
Incidentally, \texttt{save proof} also honors the \texttt{set width}
parameter currently in effect.

\texttt{/split} - Files included in the source using the expression
\$[ \textit{inclfile} \$] will be
written into separate files instead of being included in a single output
file.  The name of each separately written included file will be the
\textit{inclfile} argument of its inclusion command.

\texttt{/keep\_includes} - If a source file has includes but is written as a
single file by omitting \texttt{/split}, by default the included files will
be deleted (actually just renamed with a \char`\~1 suffix unless
\texttt{/no\_versioning} is specified) to prevent the possibly confusing
source duplication in both the output file and the included file.
The \texttt{/keep\_includes} qualifier will prevent this deletion.

\texttt{/no\_versioning} - Backup files suffixed with \char`\~1 are not created.


\section{Showing Status and Statements}



\subsection{\texttt{show settings} Command}\index{\texttt{show settings} command}
Syntax:  \texttt{show settings}

This command shows the state of various parameters.

\subsection{\texttt{show memory} Command}\index{\texttt{show memory} command}
Syntax:  \texttt{show memory}

This command shows the available memory left.  It is not meaningful
on most modern operating systems,
which have virtual memory.\index{Metamath!memory usage}


\subsection{\texttt{show labels} Command}\index{\texttt{show labels} command}
Syntax:  \texttt{show labels} {\em label-match} [\texttt{/all}]
   [\texttt{/linear}]

This command shows the labels of \texttt{\$a} and \texttt{\$p}
statements that match {\em label-match}.  A \verb$*$ in {label-match}
matches zero or more characters.  For example, \verb$*abc*def$ will match all
labels containing \verb$abc$ and ending with \verb$def$.

Optional command qualifiers:

   \texttt{/all} - Include matches for \texttt{\$e} and \texttt{\$f}
   statement labels.

   \texttt{/linear} - Display only one label per line.  This can be useful for
       building scripts in conjunction with the utilities under the
       \texttt{tools}\index{\texttt{tools} command} command.



\subsection{\texttt{show statement} Command}\index{\texttt{show statement} command}
Syntax:  \texttt{show statement} {\em label-match} [{\em qualifiers} (see below)]

This command provides information about a statement.  Only statements
that have labels (\texttt{\$f}\index{\texttt{\$f} statement},
\texttt{\$e}\index{\texttt{\$e} statement},
\texttt{\$a}\index{\texttt{\$a} statement}, and
\texttt{\$p}\index{\texttt{\$p} statement}) may be specified.
If {\em label-match}
contains wildcard (\verb$*$) characters, all matching statements will be
displayed in the order they occur in the database.

Optional qualifiers (only one qualifier at a time is allowed):

    \texttt{/comment} - This qualifier includes the comment that immediately
       precedes the statement.

    \texttt{/full} - Show complete information about each statement,
       and show all
       statements matching {\em label} (including \texttt{\$e}
       and \texttt{\$f} statements).

    \texttt{/tex} - This qualifier will write the statement information to the
       \LaTeX\ file previously opened with \texttt{open tex}.  See
       Section~\ref{texout}.

    \texttt{/simple{\char`\_}tex} - The same as \texttt{/tex}, except that
       \LaTeX\ macros are not used for formatting equations, allowing easier
       manual edits of the output for slide presentations, etc.

    \texttt{/html}\index{html generation@{\sc html} generation},
       \texttt{/alt{\char`\_}html}, \texttt{/brief{\char`\_}html},
       \texttt{/brief{\char`\_}alt{\char`\_}html} -
       These qualifiers invoke a special mode of
       \texttt{show statement} that
       creates a web page for the statement.  They may not be used with
       any other qualifier.  See Section~\ref{htmlout} or
       \texttt{help html} in the program.


\subsection{\texttt{search} Command}\index{\texttt{search} command}
Syntax:  search {\em label-match}
\texttt{"}{\em symbol-match}{\tt}" [\texttt{/all}] [\texttt{/comments}]
[\texttt{/join}]

This command searches all \texttt{\$a} and \texttt{\$p} statements
matching {\em label-match} for occurrences of {\em symbol-match}.  A
\verb@*@ in {\em label-match} matches any label character.  A \verb@$*@
in {\em symbol-match} matches any sequence of symbols.  The symbols in
{\em symbol-match} must be separated by white space.  The quotes
surrounding {\em symbol-match} may be single or double quotes.  For
example, \texttt{search b}\verb@* "-> $* ch"@ will list all statements
whose labels begin with \texttt{b} and contain the symbols \verb@->@ and
\texttt{ch} surrounding any symbol sequence (including no symbol
sequence).  The wildcards \texttt{?} and \texttt{\$?} are also available
to match individual characters in labels and symbols respectively; see
\texttt{help search} in the Metamath program for details on their usage.

Optional command qualifiers:

    \texttt{/all} - Also search \texttt{\$e} and \texttt{\$f} statements.

    \texttt{/comments} - Search the comment that immediately precedes each
        label-matched statement for {\em symbol-match}.  In this case
        {\em symbol-match} is an arbitrary, non-case-sensitive character
        string.  Quotes around {\em symbol-match} are optional if there
        is no ambiguity.

    \texttt{/join} - In the case of a \texttt{\$a} or \texttt{\$p} statement,
	prepend its \texttt{\$e}
	hypotheses for searching. The
	\texttt{/join} qualifier has no effect in \texttt{/comments} mode.

\section{Displaying and Verifying Proofs}


\subsection{\texttt{show proof} Command}\index{\texttt{show proof} command}
Syntax:  \texttt{show proof} {\em label-match} [{\em qualifiers} (see below)]

This command displays the proof of the specified
\texttt{\$p}\index{\texttt{\$p} statement} statement in various formats.
The {\em label-match} may contain wildcard (\verb@$*@) characters to match
multiple statements.  Without any qualifiers, only the logical steps
will be shown (i.e.\ syntax construction steps will be omitted), in an
indented format.

Most of the time, you will use
    \texttt{show proof} {\em label}
to see just the proof steps corresponding to logical inferences.

Optional command qualifiers:

    \texttt{/essential} - The proof tree is trimmed of all
        \texttt{\$f}\index{\texttt{\$f} statement} hypotheses before
        being displayed.  (This is the default, and it is redundant to
        specify it.)

    \texttt{/all} - the proof tree is not trimmed of all \texttt{\$f} hypotheses before
        being displayed.  \texttt{/essential} and \texttt{/all} are mutually exclusive.

    \texttt{/from{\char`\_}step} {\em step} - The display starts at the specified
        step.  If
        this qualifier is omitted, the display starts at the first step.

    \texttt{/to{\char`\_}step} {\em step} - The display ends at the specified
        step.  If this
        qualifier is omitted, the display ends at the last step.

    \texttt{/tree{\char`\_}depth} {\em number} - Only
         steps at less than the specified proof
        tree depth are displayed.  Sometimes useful for obtaining an overview of
        the proof.

    \texttt{/reverse} - The steps are displayed in reverse order.

    \texttt{/renumber} - When used with \texttt{/essential}, the steps are renumbered
        to correspond only to the essential steps.

    \texttt{/tex} - The proof is converted to \LaTeX\ and\index{latex@{\LaTeX}}
        stored in the file opened
        with \texttt{open tex}.  See Section~\ref{texout} or
        \texttt{help tex} in the program.

    \texttt{/lemmon} - The proof is displayed in a non-indented format known
        as Lemmon style, with explicit previous step number references.
        If this qualifier is omitted, steps are indented in a tree format.

    \texttt{/start{\char`\_}column} {\em number} - Overrides the default column
        (16)
        at which the formula display starts in a Lemmon-style display.  May be
        used only in conjunction with \texttt{/lemmon}.

    \texttt{/normal} - The proof is displayed in normal format suitable for
        inclusion in a Metamath source file.  May not be used with any other
        qualifier.

    \texttt{/compressed} - The proof is displayed in compressed format
        suitable for inclusion in a Metamath source file.  May not be used with
        any other qualifier.

    \texttt{/statement{\char`\_}summary} - Summarizes all statements (like a
        brief \texttt{show statement})
        used by the proof.  It may not be used with any other qualifier
        except \texttt{/essential}.

    \texttt{/detailed{\char`\_}step} {\em step} - Shows the details of what is
        happening at
        a specific proof step.  May not be used with any other qualifier.
        The {\em step} is the step number shown when displaying a
        proof without the \texttt{/renumber} qualifier.


\subsection{\texttt{show usage} Command}\index{\texttt{show usage} command}
Syntax:  \texttt{show usage} {\em label-match} [\texttt{/recursive}]

This command lists the statements whose proofs make direct reference to
the statement specified.

Optional command qualifier:

    \texttt{/recursive} - Also include statements whose proofs ultimately
        depend on the statement specified.



\subsection{\texttt{show trace\_back} Command}\index{\texttt{show
       trace{\char`\_}back} command}
Syntax:  \texttt{show trace{\char`\_}back} {\em label-match} [\texttt{/essential}] [\texttt{/axioms}]
    [\texttt{/tree}] [\texttt{/depth} {\em number}]

This command lists all statements that the proof of the \texttt{\$p}
statement(s) specified by {\em label-match} depends on.

Optional command qualifiers:

    \texttt{/essential} - Restrict the trace-back to \texttt{\$e}
        \index{\texttt{\$e} statement} hypotheses of proof trees.

    \texttt{/axioms} - List only the axioms that the proof ultimately depends on.

    \texttt{/tree} - Display the trace-back in an indented tree format.

    \texttt{/depth} {\em number} - Restrict the \texttt{/tree} trace-back to the
        specified indentation depth.

    \texttt{/count{\char`\_}steps} - Count the number of steps the proof has
       all the way back to axioms.  If \texttt{/essential} is specified,
       expansions of variable-type hypotheses (syntax constructions) are not counted.

\subsection{\texttt{verify proof} Command}\index{\texttt{verify proof} command}
Syntax:  \texttt{verify proof} {\em label-match} [\texttt{/syntax{\char`\_}only}]

This command verifies the proofs of the specified statements.  {\em
label-match} may contain wild card characters (\texttt{*}) to verify
more than one proof; for example \verb/*abc*def/ will match all labels
containing \texttt{abc} and ending with \texttt{def}.
The command \texttt{verify proof *} will verify all proofs in the database.

Optional command qualifier:

    \texttt{/syntax{\char`\_}only} - This qualifier will perform a check of syntax
        and RPN
        stack violations only.  It will not verify that the proof is
        correct.  This qualifier is useful for quickly determining which
        proofs are incomplete (i.e.\ are under development and have \texttt{?}'s
        in them).

{\em Note:} \texttt{read}, followed by \texttt{verify proof *}, will ensure
 the database is free
from errors in the Metamath language but will not check the markup notation
in comments.
You can also check the markup notation by running \texttt{verify markup *}
as discussed in Section~\ref{verifymarkup}; see also the discussion
on generating {\sc HTML} in Section~\ref{htmlout}.

\subsection{\texttt{verify markup} Command}\index{\texttt{verify markup} command}\label{verifymarkup}
Syntax:  \texttt{verify markup} {\em label-match}
[\texttt{/date{\char`\_}skip}]
[\texttt{/top{\char`\_}date{\char`\_}skip}]
[\texttt{/file{\char`\_}skip}]
[\texttt{/verbose}]

This command checks comment markup and other informal conventions we have
adopted.  It error-checks the latexdef, htmldef, and althtmldef statements
in the \texttt{\$t} statement of a Metamath source file.\index{error checking}
It error-checks any \texttt{`}...\texttt{`},
\texttt{\char`\~}~\textit{label},
and bibliographic markups in statement descriptions.
It checks that
\texttt{\$p} and \texttt{\$a} statements
have the same content when their labels start with
``ax'' and ``ax-'' respectively but are otherwise identical, for example
ax4 and ax-4.
It also verifies the date consistency of ``(Contributed by...),''
``(Revised by...),'' and ``(Proof shortened by...)'' tags in the comment
above each \texttt{\$a} and \texttt{\$p} statement.

Optional command qualifiers:

    \texttt{/date{\char`\_}skip} - This qualifier will
        skip date consistency checking,
        which is usually not required for databases other than
	\texttt{set.mm}.

    \texttt{/top{\char`\_}date{\char`\_}skip} - This qualifier will check date consistency except
        that the version date at the top of the database file will not
        be checked.  Only one of
        \texttt{/date{\char`\_}skip} and
        \texttt{/top{\char`\_}date{\char`\_}skip} may be
        specified.

    \texttt{/file{\char`\_}skip} - This qualifier will skip checks that require
        external files to be present, such as checking GIF existence and
        bibliographic links to mmset.html or equivalent.  It is useful
        for doing a quick check from a directory without these files.

    \texttt{/verbose} - Provides more information.  Currently it provides a list
        of axXXX vs. ax-XXX matches.

\subsection{\texttt{save proof} Command}\index{\texttt{save proof} command}
Syntax:  \texttt{save proof} {\em label-match} [\texttt{/normal}]
   [\texttt{/compressed}]

The \texttt{save proof} command will reformat a proof in one of two formats and
replace the existing proof in the source buffer\index{source
buffer}.  It is useful for
converting between proof formats.  Note that a proof will not be
permanently saved until a \texttt{write source} command is issued.

Optional command qualifiers:

    \texttt{/normal} - The proof is saved in the normal format (i.e., as a
        sequence
        of labels, which is the defined format of the basic Metamath
        language).\index{basic language}  This is the default format that
        is used if a qualifier
        is omitted.

    \texttt{/compressed} - The proof is saved in the compressed format which
        reduces storage requirements for a database.
        See Appendix~\ref{compressed}.




\section{Creating Proofs}\label{pfcommands}\index{Proof Assistant}

Before using the Proof Assistant, you must add a \texttt{\$p} to your
source file (using a text editor) containing the statement you want to
prove.  Its proof should consist of a single \texttt{?}, meaning
``unknown step.''  Example:
\begin{verbatim}
equid $p x = x $= ? $.
\end{verbatim}

To enter the Proof assistant, type \texttt{prove} {\em label}, e.g.
\texttt{prove equid}.  Metamath will respond with the \texttt{MM-PA>}
prompt.

Proofs are created working backwards from the statement being proved,
primarily using a series of \texttt{assign} commands.  A proof is
complete when all steps are assigned to statements and all steps are
unified and completely known.  During the creation of a proof, Metamath
will allow only operations that are legal based on what is known up to
that point.  For example, it will not allow an \texttt{assign} of a
statement that cannot be unified with the unknown proof step being
assigned.

{\em Important:}
The Proof Assistant is
{\em not} a tool to help you discover proofs.  It is just a tool to help
you add them to the database.  For a tutorial read
Section~\ref{frstprf}.
To practice using the Proof Assistant, you may
want to \texttt{prove} an existing theorem, then delete all steps with
\texttt{delete all}, then re-create it with the Proof Assistant while
looking at its proof display (before deletion).
You might want to figure out your first few proofs completely
and write them down by hand, before using the Proof Assistant, though
not everyone finds that effective.

{\em Important:}
The \texttt{undo} command if very helpful when entering a proof, because
it allows you to undo a previously-entered step.
In addition, we suggest that you
keep track of your work with a log file (\texttt{open
log}) and save it frequently (\texttt{save new{\char`\_}proof},
\texttt{write source}).
You can use \texttt{delete} to reverse an \texttt{assign}.
You can also do \texttt{delete floating{\char`\_}hypotheses}, then
\texttt{initialize all}, then \texttt{unify all /interactive} to
reinitialize bad unifications made accidentally or by bad
\texttt{assign}s.  You cannot reverse a \texttt{delete} except by
a relevant \texttt{undo} or using
\texttt{exit /force} then reentering the Proof Assistant to recover from
the last \texttt{save new{\char`\_}proof}.

The following commands are available in the Proof Assistant (at the
\texttt{MM-PA>} prompt) to help you create your proof.  See the
individual commands for more detail.

\begin{itemize}
\item[]
    \texttt{show new{\char`\_}proof} [\texttt{/all},...] - Displays the
        proof in progress.  You will use this command a lot; see \texttt{help
        show new{\char`\_}proof} to become familiar with its qualifiers.  The
        qualifiers \texttt{/unknown} and \texttt{/not{\char`\_}unified} are
        useful for seeing the work remaining to be done.  The combination
        \texttt{/all/unknown} is useful for identifying dummy variables that must be
        assigned, or attempts to use illegal syntax, when \texttt{improve all}
        is unable to complete the syntax constructions.  Unknown variables are
        shown as \texttt{\$1}, \texttt{\$2},...
\item[]
    \texttt{assign} {\em step} {\em label} - Assigns an unknown {\em step}
        number with the statement
        specified by {\em label}.
\item[]
    \texttt{let variable} {\em variable}
        \texttt{= "}{\em symbol sequence}\texttt{"}
          - Forces a symbol
        sequence to replace an unknown variable (such as \texttt{\$1}) in a proof.
        It is useful
        for helping difficult unifications, and it is necessary when you have
        dummy variables that eventually must be assigned a name.
\item[]
    \texttt{let step} {\em step} \texttt{= "}{\em symbol sequence}\texttt{"} -
          Forces a symbol sequence
        to replace the contents of a proof step, provided it can be
        unified with the existing step contents.  (I rarely use this.)
\item[]
    \texttt{unify step} {\em step} (or \texttt{unify all}) - Unifies
        the source and target of
        a step.  If you specify a specific step, you will be prompted
        to select among the unifications that are possible.  If you
        specify \texttt{all}, all steps with unique unifications, but only
        those steps, will be
        unified.  \texttt{unify all /interactive} goes through all non-unified
        steps.
\item[]
    \texttt{initialize} {\em step} (or \texttt{all}) - De-unifies the target and source of
        a step (or all steps), as well as the hypotheses of the source,
        and makes all variables in the source unknown.  Useful to recover from
        an \texttt{assign} or \texttt{let} mistake that
        resulted in incorrect unifications.
\item[]
    \texttt{delete} {\em step} (or \texttt{all} or \texttt{floating{\char`\_}hypotheses}) -
        Deletes the specified
        step(s).  \texttt{delete floating{\char`\_}hypotheses}, then \texttt{initialize all}, then
        \texttt{unify all /interactive} is useful for recovering from mistakes
        where incorrect unifications assigned wrong math symbol strings to
        variables.
\item[]
    \texttt{improve} {\em step} (or \texttt{all}) -
      Automatically creates a proof for steps (with no unknown variables)
      whose proof requires no statements with \texttt{\$e} hypotheses.  Useful
      for filling in proofs of \texttt{\$f} hypotheses.  The \texttt{/depth}
      qualifier will also try statements whose \texttt{\$e} hypotheses contain
      no new variables.  {\em Warning:} Save your work (with \texttt{save
      new{\char`\_}proof} then \texttt{write source}) before using
      \texttt{/depth = 2} or greater, since the search time grows
      exponentially and may never terminate in a reasonable time, and you
      cannot interrupt the search.  I have found that it is rare for
      \texttt{/depth = 3} or greater to be useful.
 \item[]
    \texttt{save new{\char`\_}proof} - Saves the proof in progress in the program's
        internal database buffer.  To save it permanently into the database file,
        use \texttt{write source} after
        \texttt{save new{\char`\_}proof}.  To revert to the last
        \texttt{save new{\char`\_}proof},
        \texttt{exit /force} from the Proof Assistant then re-enter the Proof
        Assistant.
 \item[]
    \texttt{match step} {\em step} (or \texttt{match all}) - Shows what
        statements are
        possibilities for the \texttt{assign} statement. (This command
        is not very
        useful in its present form and hopefully will be improved
        eventually.  In the meantime, use the \texttt{search} statement for
        candidates matching specific math token combinations.)
 \item[]
 \texttt{minimize{\char`\_}with}\index{\texttt{minimize{\char`\_}with} command}
% 3/10/07 Note: line-breaking the above results in duplicate index entries
     - After a proof is complete, this command will attempt
        to match other database theorems to the proof to see if the proof
        size can be reduced as a result.  See \texttt{help
        minimize{\char`\_}with} in the
        Metamath program for its usage.
 \item[]
 \texttt{undo}\index{\texttt{undo} command}
    - Undo the effect of a proof-changing command (all but the
      \texttt{show} and \texttt{save} commands above).
 \item[]
 \texttt{redo}\index{\texttt{redo} command}
    - Reverse the previous \texttt{undo}.
\end{itemize}

The following commands set parameters that may be relevant to your proof.
Consult the individual \texttt{help set}... commands.
\begin{itemize}
   \item[] \texttt{set unification{\char`\_}timeout}
 \item[]
    \texttt{set search{\char`\_}limit}
  \item[]
    \texttt{set empty{\char`\_}substitution} - note that default is \texttt{off}
\end{itemize}

Type \texttt{exit} to exit the \texttt{MM-PA>}
 prompt and get back to the \texttt{MM>} prompt.
Another \texttt{exit} will then get you out of Metamath.



\subsection{\texttt{prove} Command}\index{\texttt{prove} command}
Syntax:  \texttt{prove} {\em label}

This command will enter the Proof Assistant\index{Proof Assistant}, which will
allow you to create or edit the proof of the specified statement.
The command-line prompt will change from \texttt{MM>} to \texttt{MM-PA>}.

Note:  In the present version of
Metamath\index{Metamath!limitations of version 0.07.30}, the Proof
Assistant does not verify that \texttt{\$d}\index{\texttt{\$d}
statement} restrictions are met as a proof is being built.  After you
have completed a proof, you should type \texttt{save new{\char`\_}proof}
followed by \texttt{verify proof} {\em label} (where {\em label} is the
statement you are proving with the \texttt{prove} command) to verify the
\texttt{\$d} restrictions.

See also: \texttt{exit}

\subsection{\texttt{set unification\_timeout} Command}\index{\texttt{set
unification{\char`\_}timeout} command}
Syntax:  \texttt{set unification{\char`\_}timeout} {\em number}

(This command is available outside the Proof Assistant but affects the
Proof Assistant\index{Proof Assistant} only.)

Sometimes the Proof Assistant will inform you that a unification
time-out occurred.  This may happen when you try to \texttt{unify}
formulas with many temporary variables\index{temporary variable}
(\texttt{\$1}, \texttt{\$2}, etc.), since the time to compute all possible
unifications may grow exponentially with the number of variables.  If
you want Metamath to try harder (and you're willing to wait longer) you
may increase this parameter.  \texttt{show settings} will show you the
current value.

\subsection{\texttt{set empty\_substitution} Command}\index{\texttt{set
empty{\char`\_}substitution} command}
% These long names can't break well in narrow mode, and even "sloppy"
% is not enough. Work around this by not demanding justification.
\begin{flushleft}
Syntax:  \texttt{set empty{\char`\_}substitution on} or \texttt{set
empty{\char`\_}substitution off}
\end{flushleft}

(This command is available outside the Proof Assistant but affects the
Proof Assistant\index{Proof Assistant} only.)

The Metamath language allows variables to be
substituted\index{substitution!variable}\index{variable substitution}
with empty symbol sequences\index{empty substitution}.  However, in many
formal systems\index{formal system} this will never happen in a valid
proof.  Allowing for this possibility increases the likelihood of
ambiguous unifications\index{ambiguous
unification}\index{unification!ambiguous} during proof creation.
The default is that
empty substitutions are not allowed; for formal systems requiring them,
you must \texttt{set empty{\char`\_}substitution on}.
(An example where this must be \texttt{on}
would be a system that implements a Deduction Rule and in
which deductions from empty assumption lists would be permissible.  The
MIU-system\index{MIU-system} described in Appendix~\ref{MIU} is another
example.)
Note that empty substitutions are
always permissible in proof verification (VERIFY PROOF...) outside the
Proof Assistant.  (See the MIU system in the Metamath book for an example
of a system needing empty substitutions; another example would be a
system that implements a Deduction Rule and in which deductions from
empty assumption lists would be permissible.)

It is better to leave this \texttt{off} when working with \texttt{set.mm}.
Note that this command does not affect the way proofs are verified with
the \texttt{verify proof} command.  Outside of the Proof Assistant,
substitution of empty sequences for math symbols is always allowed.

\subsection{\texttt{set search\_limit} Command}\index{\texttt{set
search{\char`\_}limit} command} Syntax:  \texttt{set search{\char`\_}limit} {\em
number}

(This command is available outside the Proof Assistant but affects the
Proof Assistant\index{Proof Assistant} only.)

This command sets a parameter that determines when the \texttt{improve} command
in Proof Assistant mode gives up.  If you want \texttt{improve} to search harder,
you may increase it.  The \texttt{show settings} command tells you its current
value.


\subsection{\texttt{show new\_proof} Command}\index{\texttt{show
new{\char`\_}proof} command}
Syntax:  \texttt{show new{\char`\_}proof} [{\em
qualifiers} (see below)]

This command (available only in Proof Assistant mode) displays the proof
in progress.  It is identical to the \texttt{show proof} command, except that
there is no statement argument (since it is the statement being proved) and
the following qualifiers are not available:

    \texttt{/statement{\char`\_}summary}

    \texttt{/detailed{\char`\_}step}

Also, the following additional qualifiers are available:

    \texttt{/unknown} - Shows only steps that have no statement assigned.

    \texttt{/not{\char`\_}unified} - Shows only steps that have not been unified.

Note that \texttt{/essential}, \texttt{/depth}, \texttt{/unknown}, and
\texttt{/not{\char`\_}unified} may be
used in any combination; each of them effectively filters out additional
steps from the proof display.

See also:  \texttt{show proof}






\subsection{\texttt{assign} Command}\index{\texttt{assign} command}
Syntax:   \texttt{assign} {\em step} {\em label} [\texttt{/no{\char`\_}unify}]

   and:   \texttt{assign first} {\em label}

   and:   \texttt{assign last} {\em label}


This command, available in the Proof Assistant only, assigns an unknown
step (one with \texttt{?} in the \texttt{show new{\char`\_}proof}
listing) with the statement specified by {\em label}.  The assignment
will not be allowed if the statement cannot be unified with the step.

If \texttt{last} is specified instead of {\em step} number, the last
step that is shown by \texttt{show new{\char`\_}proof /unknown} will be
used.  This can be useful for building a proof with a command file (see
\texttt{help submit}).  It also makes building proofs faster when you know
the assignment for the last step.

If \texttt{first} is specified instead of {\em step} number, the first
step that is shown by \texttt{show new{\char`\_}proof /unknown} will be
used.

If {\em step} is zero or negative, the -{\em step}th from last unknown
step, as shown by \texttt{show new{\char`\_}proof /unknown}, will be
used.  \texttt{assign -1} {\em label} will assign the penultimate
unknown step, \texttt{assign -2} {\em label} the antepenultimate, and
\texttt{assign 0} {\em label} is the same as \texttt{assign last} {\em
label}.

Optional command qualifier:

    \texttt{/no{\char`\_}unify} - do not prompt user to select a unification if there is
        more than one possibility.  This is useful for noninteractive
        command files.  Later, the user can \texttt{unify all /interactive}.
        (The assignment will still be automatically unified if there is only
        one possibility and will be refused if unification is not possible.)



\subsection{\texttt{match} Command}\index{\texttt{match} command}
Syntax:  \texttt{match step} {\em step} [\texttt{/max{\char`\_}essential{\char`\_}hyp}
{\em number}]

    and:  \texttt{match all} [\texttt{/essential}]
          [\texttt{/max{\char`\_}essential{\char`\_}hyp} {\em number}]

This command, available in the Proof Assistant only, shows what
statements can be unified with the specified step(s).  {\em Note:} In
its current form, this command is not very useful because of the large
number of matches it reports.  I rarely use it.
It may be enhanced in the future.

Optional command qualifiers:

    \texttt{/max{\char`\_}essential{\char`\_}hyp} {\em number} - filters out
        of the list any statements
        with more than the specified number of
        \texttt{\$e}\index{\texttt{\$e} statement} hypotheses

    \texttt{/essential{\char`\_}only} - in the \texttt{match all} statement, only
        the steps that
        would be listed in the \texttt{show new{\char`\_}proof /essential} display are
        matched.



\subsection{\texttt{let} Command}\index{\texttt{let} command}
Syntax: \texttt{let variable} {\em variable} = \verb/"/{\em symbol-sequence}\verb/"/

  and: \texttt{let step} {\em step} = \verb/"/{\em symbol-sequence}\verb/"/

These commands, available in the Proof Assistant\index{Proof Assistant}
only, assign a temporary variable\index{temporary variable} or unknown
step with a specific symbol sequence.  They are useful in the middle of
creating a proof, when you know what should be in the proof step but the
unification algorithm doesn't yet have enough information to completely
specify the temporary variables.  A ``temporary variable'' is one that
has the form \texttt{\$}{\em nn} in the proof display, such as
\texttt{\$1}, \texttt{\$2}, etc.  The {\em symbol-sequence} may contain
other unknown variables if desired.  Examples:

    \verb/let variable $32 = "A = B"/

    \verb/let variable $32 = "A = $35"/

    \verb/let step 10 = '|- x = x'/

    \verb/let step -2 = "|- ( $7 = ph )"/

Any symbol sequence will be accepted for the \texttt{let variable}
command.  Only those symbol sequences that can be unified with the step
will be accepted for \texttt{let step}.

The \texttt{let} commands ``zap'' the proof with information that can
only be verified when the proof is built up further.  If you make an
error, the command sequence \texttt{delete
floating{\char`\_}hypotheses}, \texttt{initialize all}, and
\texttt{unify all /interactive} will undo a bad \texttt{let} assignment.

If {\em step} is zero or negative, the -{\em step}th from last unknown
step, as shown by \texttt{show new{\char`\_}proof /unknown}, will be
used.  The command \texttt{let step 0} = \verb/"/{\em
symbol-sequence}\verb/"/ will use the last unknown step, \texttt{let
step -1} = \verb/"/{\em symbol-sequence}\verb/"/ the penultimate, etc.
If {\em step} is positive, \texttt{let step} may be used to assign known
(in the sense of having previously been assigned a label with
\texttt{assign}) as well as unknown steps.

Either single or double quotes can surround the {\em symbol-sequence} as
long as they are different from any quotes inside a {\em
symbol-sequence}.  If {\em symbol-sequence} contains both kinds of
quotes, see the instructions at the end of \texttt{help let} in the
Metamath program.


\subsection{\texttt{unify} Command}\index{\texttt{unify} command}
Syntax:  \texttt{unify step} {\em step}

      and:   \texttt{unify all} [\texttt{/interactive}]

These commands, available in the Proof Assistant only, unify the source
and target of the specified step(s). If you specify a specific step, you
will be prompted to select among the unifications that are possible.  If
you specify \texttt{all}, only those steps with unique unifications will be
unified.

Optional command qualifier for \texttt{unify all}:

    \texttt{/interactive} - You will be prompted to select among the
        unifications
        that are possible for any steps that do not have unique
        unifications.  (Otherwise \texttt{unify all} will bypass these.)

See also \texttt{set unification{\char`\_}timeout}.  The default is
100000, but increasing it to 1000000 can help difficult cases.  Manually
assigning some or all of the unknown variables with the \texttt{let
variable} command also helps difficult cases.



\subsection{\texttt{initialize} Command}\index{\texttt{initialize} command}
Syntax:  \texttt{initialize step} {\em step}

    and: \texttt{initialize all}

These commands, available in the Proof Assistant\index{Proof Assistant}
only, ``de-unify'' the target and source of a step (or all steps), as
well as the hypotheses of the source, and makes all variables in the
source and the source's hypotheses unknown.  This command is useful to
help recover from incorrect unifications that resulted from an incorrect
\texttt{assign}, \texttt{let}, or unification choice.  Part or all of
the command sequence \texttt{delete floating{\char`\_}hypotheses},
\texttt{initialize all}, and \texttt{unify all /interactive} will recover
from incorrect unifications.

See also:  \texttt{unify} and \texttt{delete}



\subsection{\texttt{delete} Command}\index{\texttt{delete} command}
Syntax:  \texttt{delete step} {\em step}

   and:      \texttt{delete all} -- {\em Warning: dangerous!}

   and:      \texttt{delete floating{\char`\_}hypotheses}

These commands are available in the Proof Assistant only.  The
\texttt{delete step} command deletes the proof tree section that
branches off of the specified step and makes the step become unknown.
\texttt{delete all} is equivalent to \texttt{delete step} {\em step}
where {\em step} is the last step in the proof (i.e.\ the beginning of
the proof tree).

In most cases the \texttt{undo} command is the best way to undo
a previous step.
An alternative is to salvage your last \texttt{save
new{\char`\_}proof} by exiting and reentering the Proof Assistant.
For this to work, keep a log file open to record your work
and to do \texttt{save new{\char`\_}proof} frequently, especially before
\texttt{delete}.

\texttt{delete floating{\char`\_}hypotheses} will delete all sections of
the proof that branch off of \texttt{\$f}\index{\texttt{\$f} statement}
statements.  It is sometimes useful to do this before an
\texttt{initialize} command to recover from an error.  Note that once a
proof step with a \texttt{\$f} hypothesis as the target is completely
known, the \texttt{improve} command can usually fill in the proof for
that step.  Unlike the deletion of logical steps, \texttt{delete
floating{\char`\_}hypotheses} is a relatively safe command that is
usually easy to recover from.



\subsection{\texttt{improve} Command}\index{\texttt{improve} command}
\label{improve}
Syntax:  \texttt{improve} {\em step} [\texttt{/depth} {\em number}]
                                               [\texttt{/no{\char`\_}distinct}]

   and:   \texttt{improve first} [\texttt{/depth} {\em number}]
                                              [\texttt{/no{\char`\_}distinct}]

   and:   \texttt{improve last} [\texttt{/depth} {\em number}]
                                              [\texttt{/no{\char`\_}distinct}]

   and:   \texttt{improve all} [\texttt{/depth} {\em number}]
                                              [\texttt{/no{\char`\_}distinct}]

These commands, available in the Proof Assistant\index{Proof Assistant}
only, try to find proofs automatically for unknown steps whose symbol
sequences are completely known.  They are primarily useful for filling in
proofs of \texttt{\$f}\index{\texttt{\$f} statement} hypotheses.  The
search will be restricted to statements having no
\texttt{\$e}\index{\texttt{\$e} statement} hypotheses.

\begin{sloppypar} % narrow
Note:  If memory is limited, \texttt{improve all} on a large proof may
overflow memory.  If you use \texttt{set unification{\char`\_}timeout 1}
before \texttt{improve all}, there will usually be sufficient
improvement to easily recover and completely \texttt{improve} the proof
later on a larger computer.  Warning:  Once memory has overflowed, there
is no recovery.  If in doubt, save the intermediate proof (\texttt{save
new{\char`\_}proof} then \texttt{write source}) before \texttt{improve
all}.
\end{sloppypar}

If \texttt{last} is specified instead of {\em step} number, the last
step that is shown by \texttt{show new{\char`\_}proof /unknown} will be
used.

If \texttt{first} is specified instead of {\em step} number, the first
step that is shown by \texttt{show new{\char`\_}proof /unknown} will be
used.

If {\em step} is zero or negative, the -{\em step}th from last unknown
step, as shown by \texttt{show new{\char`\_}proof /unknown}, will be
used.  \texttt{improve -1} will use the penultimate
unknown step, \texttt{improve -2} {\em label} the antepenultimate, and
\texttt{improve 0} is the same as \texttt{improve last}.

Optional command qualifier:

    \texttt{/depth} {\em number} - This qualifier will cause the search
        to include
        statements with \texttt{\$e} hypotheses (but no new variables in
        the \texttt{\$e}
        hypotheses), provided that the backtracking has not exceeded the
        specified depth. {\em Warning:}  Try \texttt{/depth 1},
        then \texttt{2}, then \texttt{3}, etc.
        in sequence because of possible exponential blowups.  Save your
        work before trying \texttt{/depth} greater than \texttt{1}!

    \texttt{/no{\char`\_}distinct} - Skip trial statements that have
        \texttt{\$d}\index{\texttt{\$d} statement} requirements.
        This qualifier will prevent assignments that might violate \texttt{\$d}
        requirements but it also could miss possible legal assignments.

See also: \texttt{set search{\char`\_}limit}

\subsection{\texttt{save new\_proof} Command}\index{\texttt{save
new{\char`\_}proof} command}
Syntax:  \texttt{save new{\char`\_}proof} {\em label} [\texttt{/normal}]
   [\texttt{/compressed}]

The \texttt{save new{\char`\_}proof} command is available in the Proof
Assistant only.  It saves the proof in progress in the source
buffer\index{source buffer}.  \texttt{save new{\char`\_}proof} may be
used to save a completed proof, or it may be used to save a proof in
progress in order to work on it later.  If an incomplete proof is saved,
any user assignments with \texttt{let step} or \texttt{let variable}
will be lost, as will any ambiguous unifications\index{ambiguous
unification}\index{unification!ambiguous} that were resolved manually.
To help make recovery easier, it can be helpful to \texttt{improve all}
before \texttt{save new{\char`\_}proof} so that the incomplete proof
will have as much information as possible.

Note that the proof will not be permanently saved until a \texttt{write
source} command is issued.

Optional command qualifiers:

    \texttt{/normal} - The proof is saved in the normal format (i.e., as a
        sequence of labels, which is the defined format of the basic Metamath
        language).\index{basic language}  This is the default format that
        is used if a qualifier is omitted.

    \texttt{/compressed} - The proof is saved in the compressed format, which
        reduces storage requirements for a database.
        (See Appendix~\ref{compressed}.)


\section{Creating \LaTeX\ Output}\label{texout}\index{latex@{\LaTeX}}

You can generate \LaTeX\ output given the
information in a database.
The database must already include the necessary typesetting information
(see section \ref{tcomment} for how to provide this information).

The \texttt{show statement} and \texttt{show proof} commands each have a
special \texttt{/tex} command qualifier that produces \LaTeX\ output.
(The \texttt{show statement} command also has the
\texttt{/simple{\char`\_}tex} qualifier for output that is easier to
edit by hand.)  Before you can use them, you must open a \LaTeX\ file to
which to send their output.  A typical complete session will use this
sequence of Metamath commands:

\begin{verbatim}
read set.mm
open tex example.tex
show statement a1i /tex
show proof a1i /all/lemmon/renumber/tex
show statement uneq2 /tex
show proof uneq2 /all/lemmon/renumber/tex
close tex
\end{verbatim}

See Section~\ref{mathcomments} for information on comment markup and
Appendix~\ref{ASCII} for information on how math symbol translation is
specified.

To format and print the \LaTeX\ source, you will need the \LaTeX\
program, which is standard on most Linux installations and available for
Windows.  On Linux, in order to create a {\sc pdf} file, you will
typically type at the shell prompt
\begin{verbatim}
$ pdflatex example.tex
\end{verbatim}

\subsection{\texttt{open tex} Command}\index{\texttt{open tex} command}
Syntax:  \texttt{open tex} {\em file-name} [\texttt{/no{\char`\_}header}]

This command opens a file for writing \LaTeX\
source\index{latex@{\LaTeX}} and writes a \LaTeX\ header to the file.
\LaTeX\ source can be written with the \texttt{show proof}, \texttt{show
new{\char`\_}proof}, and \texttt{show statement} commands using the
\texttt{/tex} qualifier.

The mapping to \LaTeX\ symbols is defined in a special comment
containing a \texttt{\$t} token, described in Appendix~\ref{ASCII}.

There is an optional command qualifier:

    \texttt{/no{\char`\_}header} - This qualifier prevents a standard
        \LaTeX\ header and trailer
        from being included with the output \LaTeX\ code.


\subsection{\texttt{close tex} Command}\index{\texttt{close tex} command}
Syntax:  \texttt{close tex}

This command writes a trailer to any \LaTeX\ file\index{latex@{\LaTeX}}
that was opened with \texttt{open tex} (unless
\texttt{/no{\char`\_}header} was used with \texttt{open tex}) and closes
the \LaTeX\ file.


\section{Creating {\sc HTML} Output}\label{htmlout}

You can generate {\sc html} web pages given the
information in a database.
The database must already include the necessary typesetting information
(see section \ref{tcomment} for how to provide this information).
The ability to produce {\sc html} web pages was added in Metamath version
0.07.30.

To create an {\sc html} output file(s) for \texttt{\$a} or \texttt{\$p}
statement(s), use
\begin{quote}
    \texttt{show statement} {\em label-match} \texttt{/html}
\end{quote}
The output file will be named {\em label-match}\texttt{.html}
for each match.  When {\em
label-match} has wildcard (\texttt{*}) characters, all statements with
matching labels will have {\sc html} files produced for them.  Also,
when {\em label-match} has a wildcard (\texttt{*}) character, two additional
files, \texttt{mmdefinitions.html} and \texttt{mmascii.html} will be
produced.  To produce {\em only} these two additional files, you can use
\texttt{?*}, which will not match any statement label, in place of {\em
label-match}.

There are three other qualifiers for \texttt{show statement} that also
generate {\sc HTML} code.  These are \texttt{/alt{\char`\_}html},
\texttt{/brief{\char`\_}html}, and
\texttt{/brief{\char`\_}alt{\char`\_}html}, and are described in the
next section.

The command
\begin{quote}
    \texttt{show statement} {\em label-match} \texttt{/alt{\char`\_}html}
\end{quote}
does the same as \texttt{show statement} {\em label-match} \texttt{/html},
except that the {\sc html} code for the symbols is taken from
\texttt{althtmldef} statements instead of \texttt{htmldef} statements in
the \texttt{\$t} comment.

The command
\begin{verbatim}
show statement * /brief_html
\end{verbatim}
invokes a special mode that just produces definition and theorem lists
accompanied by their symbol strings, in a format suitable for copying and
pasting into another web page (such as the tutorial pages on the
Metamath web site).

Finally, the command
\begin{verbatim}
show statement * /brief_alt_html
\end{verbatim}
does the same as \texttt{show statement * / brief{\char`\_}html}
for the alternate {\sc html}
symbol representation.

A statement's comment can include a special notation that provides a
certain amount of control over the {\sc HTML} version of the comment.  See
Section~\ref{mathcomments} (p.~\pageref{mathcomments}) for the comment
markup features.

The \texttt{write theorem{\char`\_}list} and \texttt{write bibliography}
commands, which are described below, provide as a side effect complete
error checking for all of the features described in this
section.\index{error checking}

\subsection{\texttt{write theorem\_list}
Command}\index{\texttt{write theorem{\char`\_}list} command}
Syntax:  \texttt{write theorem{\char`\_}list}
[\texttt{/theorems{\char`\_}per{\char`\_}page} {\em number}]

This command writes a list of all of the \texttt{\$a} and \texttt{\$p}
statements in the database into a web page file
 called \texttt{mmtheorems.html}.
When additional files are needed, they are called
\texttt{mmtheorems2.html}, \texttt{mmtheorems3.html}, etc.

Optional command qualifier:

    \texttt{/theorems{\char`\_}per{\char`\_}page} {\em number} -
 This qualifier specifies the number of statements to
        write per web page.  The default is 100.

{\em Note:} In version 0.07.30\index{Metamath!limitations of version
0.07.30} of Metamath, the ``Nearby theorems'' links on the individual
web pages presuppose 100 theorems per page when linking to the theorem
list pages.  Therefore the \texttt{/theorems{\char`\_}per{\char`\_}page}
qualifier, if it specifies a number other than 100, will cause the
individual web pages to be out of sync and should not be used to
generate the main theorem list for the web site.  This is expected to be
fixed in a future version.


\subsection{\texttt{write bibliography}\label{wrbib}
Command}\index{\texttt{write bibliography} command}
Syntax:  \texttt{write bibliography} {\em filename}

This command reads an existing {\sc html} bibliographic cross-reference
file, normally called \texttt{mmbiblio.html}, and updates it per the
bibliographic links in the database comments.  The file is updated
between the {\sc html} comment lines \texttt{<!--
{\char`\#}START{\char`\#} -->} and \texttt{<!-- {\char`\#}END{\char`\#}
-->}.  The original input file is renamed to {\em
filename}\texttt{{\char`\~}1}.

A bibliographic reference is indicated with the reference name
in brackets, such as  \texttt{Theorem 3.1 of
[Monk] p.\ 22}.
See Section~\ref{htmlout} (p.~\pageref{htmlout}) for
syntax details.


\subsection{\texttt{write recent\_additions}
Command}\index{\texttt{write recent{\char`\_}additions} command}
Syntax:  \texttt{write recent{\char`\_}additions} {\em filename}
[\texttt{/limit} {\em number}]

This command reads an existing ``Recent Additions'' {\sc html} file,
normally called \texttt{mmrecent.html}, and updates it with the
descriptions of the most recently added theorems to the database.
 The file is updated between
the {\sc html} comment lines \texttt{<!-- {\char`\#}START{\char`\#} -->}
and \texttt{<!-- {\char`\#}END{\char`\#} -->}.  The original input file
is renamed to {\em filename}\texttt{{\char`\~}1}.

Optional command qualifier:

    \texttt{/limit} {\em number} -
 This qualifier specifies the number of most recent theorems to
   write to the output file.  The default is 100.


\section{Text File Utilities}

\subsection{\texttt{tools} Command}\index{\texttt{tools} command}
Syntax:  \texttt{tools}

This command invokes an easy-to-use, general purpose utility for
manipulating the contents of {\sc ascii} text files.  Upon typing
\texttt{tools}, the command-line prompt will change to \texttt{TOOLS>}
until you type \texttt{exit}.  The \texttt{tools} commands can be used
to perform simple, global edits on an input/output file,
such as making a character string substitution on each line, adding a
string to each line, and so on.  A typical use of this utility is
to build a \texttt{submit} input file to perform a common operation on a
list of statements obtained from \texttt{show label} or \texttt{show
usage}.

The actions of most of the \texttt{tools} commands can also be
performed with equivalent (and more powerful) Unix shell commands, and
some users may find those more efficient.  But for Windows users or
users not comfortable with Unix, \texttt{tools} provides an
easy-to-learn alternative that is adequate for most of the
script-building tasks needed to use the Metamath program effectively.

\subsection{\texttt{help} Command (in \texttt{tools})}
Syntax:  \texttt{help}

The \texttt{help} command lists the commands available in the
\texttt{tools} utility, along with a brief description.  Each command,
in turn, has its own help, such as \texttt{help add}.  As with
Metamath's \texttt{MM>} prompt, a complete command can be entered at
once, or just the command word can be typed, causing the program to
prompt for each argument.

\vskip 1ex
\noindent Line-by-line editing commands:

  \texttt{add} - Add a specified string to each line in a file.

  \texttt{clean} - Trim spaces and tabs on each line in a file; convert
         characters.

  \texttt{delete} - Delete a section of each line in a file.

  \texttt{insert} - Insert a string at a specified column in each line of
        a file.

  \texttt{substitute} - Make a simple substitution on each line of the file.

  \texttt{tag} - Like \texttt{add}, but restricted to a range of lines.

  \texttt{swap} - Swap the two halves of each line in a file.

\vskip 1ex
\noindent Other file-processing commands:

  \texttt{break} - Break up (tokenize) a file into a list of tokens (one per
        line).

  \texttt{build} - Build a file with multiple tokens per line from a list.

  \texttt{count} - Count the occurrences in a file of a specified string.

  \texttt{number} - Create a list of numbers.

  \texttt{parallel} - Put two files in parallel.

  \texttt{reverse} - Reverse the order of the lines in a file.

  \texttt{right} - Right-justify lines in a file (useful before sorting
         numbers).

%  \texttt{tag} - Tag edit updates in a program for revision control.

  \texttt{sort} - Sort the lines in a file with key starting at
         specified string.

  \texttt{match} - Extract lines containing (or not) a specified string.

  \texttt{unduplicate} - Eliminate duplicate occurrences of lines in a file.

  \texttt{duplicate} - Extract first occurrence of any line occurring
         more than

   \ \ \    once in a file, discarding lines occurring exactly once.

  \texttt{unique} - Extract lines occurring exactly once in a file.

  \texttt{type} (10 lines) - Display the first few lines in a file.
                  Similar to Unix \texttt{head}.

  \texttt{copy} - Similar to Unix \texttt{cat} but safe (same input
         and output file allowed).

  \texttt{submit} - Run a script containing \texttt{tools} commands.

\vskip 1ex

\noindent Note:
  \texttt{unduplicate}, \texttt{duplicate}, and \texttt{unique} also
 sort the lines as a side effect.


\subsection{Using \texttt{tools} to Build Metamath \texttt{submit}
Scripts}

The \texttt{break} command is typically used to break up a series of
statement labels, such as the output of Metamath's \texttt{show usage},
into one label per line.  The other \texttt{tools} commands can then be
used to add strings before and after each statement label to specify
commands to be performed on the statement.  The \texttt{parallel}
command is useful when a statement label must be mentioned more than
once on a line.

Very often a \texttt{submit} script for Metamath will require multiple
command lines for each statement being processed.  For example, you may
want to enter the Proof Assistant, \texttt{minimize{\char`\_}with} your
latest theorem, \texttt{save} the new proof, and \texttt{exit} the Proof
Assistant.  To accomplish this, you can build a file with these four
commands for each statement on a single line, separating each command
with a designated character such as \texttt{@}.  Then at the end you can
\texttt{substitute} each \texttt{@} with \texttt{{\char`\\}n} to break
up the lines into individual command lines (see \texttt{help
substitute}).


\subsection{Example of a \texttt{tools} Session}

To give you a quick feel for the \texttt{tools} utility, we show a
simple session where we create a file \texttt{n.txt} with 3 lines, add
strings before and after each line, and display the lines on the screen.
You can experiment with the various commands to gain experience with the
\texttt{tools} utility.

\begin{verbatim}
MM> tools
Entering the Text Tools utilities.
Type HELP for help, EXIT to exit.
TOOLS> number
Output file <n.tmp>? n.txt
First number <1>?
Last number <10>? 3
Increment <1>?
TOOLS> add
Input/output file? n.txt
String to add to beginning of each line <>? This is line
String to add to end of each line <>? .
The file n.txt has 3 lines; 3 were changed.
First change is on line 1:
This is line 1.
TOOLS> type n.txt
This is line 1.
This is line 2.
This is line 3.
TOOLS> exit
Exiting the Text Tools.
Type EXIT again to exit Metamath.
MM>
\end{verbatim}



\appendix
\chapter{Sample Representations}
\label{ASCII}

This Appendix provides a sample of {\sc ASCII} representations,
their corresponding traditional mathematical symbols,
and a discussion of their meanings
in the \texttt{set.mm} database.
These are provided in order of appearance.
This is only a partial list, and new definitions are routinely added.
A complete list is available at \url{http://metamath.org}.

These {\sc ASCII} representations, along
with information on how to display them,
are defined in the \texttt{set.mm} database file inside
a special comment called a \texttt{\$t} {\em
comment}\index{\texttt{\$t} comment} or {\em typesetting
comment.}\index{typesetting comment}
A typesetting comment
is indicated by the appearance of the
two-character string \texttt{\$t} at the beginning of the comment.
For more information,
see Section~\ref{tcomment}, p.~\pageref{tcomment}.

In the following table the ``{\sc ASCII}'' column shows the {\sc ASCII}
representation,
``Symbol'' shows the mathematical symbolic display
that corresponds to that {\sc ASCII} representation, ``Labels'' shows
the key label(s) that define the representation, and
``Description'' provides a description about the symbol.
As usual, ``iff'' is short for ``if and only if.''\index{iff}
In most cases the ``{\sc ASCII}'' column only shows
the key token, but it will sometimes show a sequence of tokens
if that is necessary for clarity.

{\setlength{\extrarowsep}{4pt} % Keep rows from being too close together
\begin{longtabu}   { @{} c c l X }
\textbf{ASCII} & \textbf{Symbol} & \textbf{Labels} & \textbf{Description} \\
\endhead
\texttt{|-} & $\vdash$ & &
  ``It is provable that...'' \\
\texttt{ph} & $\varphi$ & \texttt{wph} &
  The wff (boolean) variable phi,
  conventionally the first wff variable. \\
\texttt{ps} & $\psi$ & \texttt{wps} &
  The wff (boolean) variable psi,
  conventionally the second wff variable. \\
\texttt{ch} & $\chi$ & \texttt{wch} &
  The wff (boolean) variable chi,
  conventionally the third wff variable. \\
\texttt{-.} & $\lnot$ & \texttt{wn} &
  Logical not. E.g., if $\varphi$ is true, then $\lnot \varphi$ is false. \\
\texttt{->} & $\rightarrow$ & \texttt{wi} &
  Implies, also known as material implication.
  In classical logic the expression $\varphi \rightarrow \psi$ is true
  if either $\varphi$ is false or $\psi$ is true (or both), that is,
  $\varphi \rightarrow \psi$ has the same meaning as
  $\lnot \varphi \lor \psi$ (as proven in theorem \texttt{imor}). \\
\texttt{<->} & $\leftrightarrow$ &
  \hyperref[df-bi]{\texttt{df-bi}} &
  Biconditional (aka is-equals for boolean values).
  $\varphi \leftrightarrow \psi$ is true iff
  $\varphi$ and $\psi$ have the same value. \\
\texttt{\char`\\/} & $\lor$ &
  \makecell[tl]{{\hyperref[df-or]{\texttt{df-or}}}, \\
	         \hyperref[df-3or]{\texttt{df-3or}}} &
  Disjunction (logical ``or''). $\varphi \lor \psi$ is true iff
  $\varphi$, $\psi$, or both are true. \\
\texttt{/\char`\\} & $\land$ &
  \makecell[tl]{{\hyperref[df-an]{\texttt{df-an}}}, \\
                 \hyperref[df-3an]{\texttt{df-3an}}} &
  Conjunction (logical ``and''). $\varphi \land \psi$ is true iff
  both $\varphi$ and $\psi$ are true. \\
\texttt{A.} & $\forall$ &
  \texttt{wal} &
  For all; the wff $\forall x \varphi$ is true iff
  $\varphi$ is true for all values of $x$. \\
\texttt{E.} & $\exists$ &
  \hyperref[df-ex]{\texttt{df-ex}} &
  There exists; the wff
  $\exists x \varphi$ is true iff
  there is at least one $x$ where $\varphi$ is true. \\
\texttt{[ y / x ]} & $[ y / x ]$ &
  \hyperref[df-sb]{\texttt{df-sb}} &
  The wff $[ y / x ] \varphi$ produces
  the result when $y$ is properly substituted for $x$ in $\varphi$
  ($y$ replaces $x$).
  % This is elsb4
  % ( [ x / y ] z e. y <-> z e. x )
  For example,
  $[ x / y ] z \in y$ is the same as $z \in x$. \\
\texttt{E!} & $\exists !$ &
  \hyperref[df-eu]{\texttt{df-eu}} &
  There exists exactly one;
  $\exists ! x \varphi$ is true iff
  there is at least one $x$ where $\varphi$ is true. \\
\texttt{\{ y | phi \}}  & $ \{ y | \varphi \}$ &
  \hyperref[df-clab]{\texttt{df-clab}} &
  The class of all sets where $\varphi$ is true. \\
\texttt{=} & $ = $ &
  \hyperref[df-cleq]{\texttt{df-cleq}} &
  Class equality; $A = B$ iff $A$ equals $B$. \\
\texttt{e.} & $ \in $ &
  \hyperref[df-clel]{\texttt{df-clel}} &
  Class membership; $A \in B$ if $A$ is a member of $B$. \\
\texttt{{\char`\_}V} & {\rm V} &
  \hyperref[df-v]{\texttt{df-v}} &
  Class of all sets (not itself a set). \\
\texttt{C\_} & $ \subseteq $ &
  \hyperref[df-ss]{\texttt{df-ss}} &
  Subclass (subset); $A \subseteq B$ is true iff
  $A$ is a subclass of $B$. \\
\texttt{u.} & $ \cup $ &
  \hyperref[df-un]{\texttt{df-un}} &
  $A \cup B$ is the union of classes $A$ and $B$. \\
\texttt{i^i} & $ \cap $ &
  \hyperref[df-in]{\texttt{df-in}} &
  $A \cap B$ is the intersection of classes $A$ and $B$. \\
\texttt{\char`\\} & $ \setminus $ &
  \hyperref[df-dif]{\texttt{df-dif}} &
  $A \setminus B$ (set difference)
  is the class of all sets in $A$ except for those in $B$. \\
\texttt{(/)} & $ \varnothing $ &
  \hyperref[df-nul]{\texttt{df-nul}} &
  $ \varnothing $ is the empty set (aka null set). \\
\texttt{\char`\~P} & $ \cal P $ &
  \hyperref[df-pw]{\texttt{df-pw}} &
  Power class. \\
\texttt{<.\ A , B >.} & $\langle A , B \rangle$ &
  \hyperref[df-op]{\texttt{df-op}} &
  The ordered pair $\langle A , B \rangle$. \\
\texttt{( F ` A )} & $ ( F ` A ) $ &
  \hyperref[df-fv]{\texttt{df-fv}} &
  The value of function $F$ when applied to $A$. \\
\texttt{\_i} & $ i $ &
  \texttt{df-i} &
  The square root of negative one. \\
\texttt{x.} & $ \cdot $ &
  \texttt{df-mul} &
  Complex number multiplication; $2~\cdot~3~=~6$. \\
\texttt{CC} & $ \mathbb{C} $ &
  \texttt{df-c} &
  The set of complex numbers. \\
\texttt{RR} & $ \mathbb{R} $ &
  \texttt{df-r} &
  The set of real numbers. \\
\end{longtabu}
} % end of extrarowsep

\chapter{Compressed Proofs}
\label{compressed}\index{compressed proof}\index{proof!compressed}

The proofs in the \texttt{set.mm} set theory database are stored in compressed
format for efficiency.  Normally you needn't concern yourself with the
compressed format, since you can display it with the usual proof display tools
in the Metamath program (\texttt{show proof}\ldots) or convert it to the normal
RPN proof format described in Section~\ref{proof} (with \texttt{save proof}
{\em label} \texttt{/normal}).  However for sake of completeness we describe the
format here and show how it maps to the normal RPN proof format.

A compressed proof, located between \texttt{\$=} and \texttt{\$.}\ keywords, consists
of a left parenthesis, a sequence of statement labels, a right parenthesis,
and a sequence of upper-case letters \texttt{A} through \texttt{Z} (with optional
white space between them).  White space must surround the parentheses
and the labels.  The left parenthesis tells Metamath that a
compressed proof follows.  (A normal RPN proof consists of just a sequence of
labels, and a parenthesis is not a legal character in a label.)

The sequence of upper-case letters corresponds to a sequence of integers
with the following mapping.  Each integer corresponds to a proof step as
described later.
\begin{center}
  \texttt{A} = 1 \\
  \texttt{B} = 2 \\
   \ldots \\
  \texttt{T} = 20 \\
  \texttt{UA} = 21 \\
  \texttt{UB} = 22 \\
   \ldots \\
  \texttt{UT} = 40 \\
  \texttt{VA} = 41 \\
  \texttt{VB} = 42 \\
   \ldots \\
  \texttt{YT} = 120 \\
  \texttt{UUA} = 121 \\
   \ldots \\
  \texttt{YYT} = 620 \\
  \texttt{UUUA} = 621 \\
   etc.
\end{center}

In other words, \texttt{A} through \texttt{T} represent the
least-significant digit in base 20, and \texttt{U} through \texttt{Y}
represent zero or more most-significant digits in base 5, where the
digits start counting at 1 instead of the usual 0. With this scheme, we
don't need white space between these ``numbers.''

(In the design of the compressed proof format, only upper-case letters,
as opposed to say all non-whitespace printable {\sc ascii} characters other than
%\texttt{\$}, was chosen to make the compressed proof a little less
%displeasing to the eye, at the expense of a typical 20\% compression
\texttt{\$}, were chosen so as not to collide with most text editor
searches, at the expense of a typical 20\% compression
loss.  The base 5/base 20 grouping, as opposed to say base 6/base 19,
was chosen by experimentally determining the grouping that resulted in
best typical compression.)

The letter \texttt{Z} identifies (tags) a proof step that is identical to one
that occurs later on in the proof; it helps shorten the proof by not requiring
that identical proof steps be proved over and over again (which happens often
when building wff's).  The \texttt{Z} is placed immediately after the
least-significant digit (letters \texttt{A} through \texttt{T}) that ends the integer
corresponding to the step to later be referenced.

The integers that the upper-case letters correspond to are mapped to labels as
follows.  If the statement being proved has $m$ mandatory hypotheses, integers
1 through $m$ correspond to the labels of these hypotheses in the order shown
by the \texttt{show statement ... / full} command, i.e., the RPN order\index{RPN
order} of the mandatory
hypotheses.  Integers $m+1$ through $m+n$ correspond to the labels enclosed in
the parentheses of the compressed proof, in the order that they appear, where
$n$ is the number of those labels.  Integers $m+n+1$ on up don't directly
correspond to statement labels but point to proof steps identified with the
letter \texttt{Z}, so that these proof steps can be referenced later in the
proof.  Integer $m+n+1$ corresponds to the first step tagged with a \texttt{Z},
$m+n+2$ to the second step tagged with a \texttt{Z}, etc.  When the compressed
proof is converted to a normal proof, the entire subproof of a step tagged
with \texttt{Z} replaces the reference to that step.

For efficiency, Metamath works with compressed proofs directly, without
converting them internally to normal proofs.  In addition to the usual
error-checking, an error message is given if (1) a label in the label list in
parentheses does not refer to a previous \texttt{\$p} or \texttt{\$a} statement or a
non-mandatory hypothesis of the statement being proved and (2) a proof step
tagged with \texttt{Z} is referenced before the step tagged with the \texttt{Z}.

Just as in a normal proof under development (Section~\ref{unknown}), any step
or subproof that is not yet known may be represented with a single \texttt{?}.
White space does not have to appear between the \texttt{?}\ and the upper-case
letters (or other \texttt{?}'s) representing the remainder of the proof.

% April 1, 2004 Appendix C has been added back in with corrections.
%
% May 20, 2003 Appendix C was removed for now because there was a problem found
% by Bob Solovay
%
% Also, removed earlier \ref{formalspec} 's (3 cases above)
%
% Bob Solovay wrote on 30 Nov 2002:
%%%%%%%%%%%%% (start of email comment )
%      3. My next set of comments concern appendix C. I read this before I
% read Chapter 4. So I first noted that the system as presented in the
% Appendix lacked a certain formal property that I thought desirable. I
% then came up with a revised formal system that had this property. Upon
% reading Chapter 4, I noticed that the revised system was closer to the
% treatment in Chapter 4 than the system in Appendix C.
%
%         First a very minor correction:
%
%         On page 142 line 2: The condition that V(e) != V(f) should only be
% required of e, f in T such that e != f.
%
%         Here is a natural property [transitivity] that one would like
% the formal system to have:
%
%         Let Gamma be a set of statements. Suppose that the statement Phi
% is provable from Gamma and that the statement Psi is provable from Gamma
% \cup {Phi}. Then Psi is provable from Gamma.
%
%         I shall present an example to show that this property does not
% hold for the formal systems of Appendix C:
%
%         I write the example in metamath style:
%
% $c A B C D E $.
% $v x y
%
% ${
% tx $f A x $.
% ty $f B y $.
% ax1 $a C x y $.
% $}
%
% ${
% tx $f A x $.
% ty $f B y $.
% ax2-h1 $e C x y $.
% ax2 $a D y $.
% $}
%
% ${
% ty $f B y $.
% ax3-h1 $e D y $.
% ax3 $a E y $.
% $}
%
% $(These three axioms are Gamma $)
%
% ${
% tx $f A x $.
% ty $f B y $.
% Phi $p D y $=
% tx ty tx ty ax1 ax2 $.
% $}
%
% ${
% ty $f B y $.
% Psi $p E y $=
% ty ty Phi ax3 $.
% $}
%
%
% I omit the formal proofs of the following claims. [I will be glad to
% supply them upon request.]
%
% 1) Psi is not provable from Gamma;
%
% 2) Psi is provable from Gamma + Phi.
%
% Here "provable" refers to the formalism of Appendix C.
%
% The trouble of course is that Psi is lacking the variable declaration
%
% $f Ax $.
%
% In the Metamath system there is no trouble proving Psi. I attach a
% metamath file that shows this and which has been checked by the
% metamath program.
%
% I next want to indicate how I think the treatment in Appendix C should
% be revised so as to conform more closely to the metamath system of the
% main text. The revised system *does* have the transitivity property.
%
% We want to give revised definitions of "statement" and
% "provable". [cf. sections C.2.4. and C.2.5] Our new definitions will
% use the definitions given in Appendix C. So we take the following
% tack. We refer to the original notions as o-statement and o-provable. And
% we refer to the notions we are defining as n-statement and n-provable.
%
%         A n-statement is an o-statement in which the only variables
% that appear in the T component are mandatory.
%
%         To any o-statement we can associate its reduct which is a
% n-statement by dropping all the elements of T or D which contain
% non-mandatory variables.
%
%         An n-statement gamma is n-provable if there is an o-statement
% gamma' which has gamma as its reduct andf such that gamma' is
% o-provable.
%
%         It seems to me [though I am not completely sure on this point]
% that n-provability corresponds to metamath provability as discussed
% say in Chapter 4.
%
%         Attached to this letter is the metamath proof of Phi and Psi
% from Gamma discussed above.
%
%         I am still brooding over the question of whether metamath
% correctly formalizes set-theory. No doubt I will have some questions
% re this after my thoughts become clearer.
%%%%%%%%%%%%%%%% (end of email comment)

%%%%%%%%%%%%%%%% (start of 2nd email comment from Bob Solovay 1-Apr-04)
%
%         I hope that Appendix C is the one that gives a "formal" treatment
% of Metamath. At any rate, thats the appendix I want to comment on.
%
%         I'm going to suggest two changes in the definition.
%
%         First change (in the definition of statement): Require that the
% sets D, T, and E be finite.
%
%         Probably things are fine as you give them. But in the applications
% to the main metamath system they will always be finite, and its useful in
% thinking about things [at least for me] to stick to the finite case.
%
%         Second change:
%
%         First let me give an approximate description. Remove the dummy
% variables from the statement. Instead, include them in the proof.
%
%         More formally: Require that T consists of type declarations only
% for mandatory variables. Require that all the pairs in D consist of
% mandatory variables.
%
%         At the start of a proof we are allowed to declare a finite number
% of dummy variables [provided that none of them appear in any of the
% statements in E \cup {A}. We have to supply type declarations for all the
% dummy variables. We are allowed to add new $d statements referring to
% either the mandatory or dummy variables. But we require that no new $d
% statement references only mandatory variables.
%
%         I find this way of doing things more conceptual than the treatment
% in Appendix C. But the change [which I will use implicitly in later
% letters about doing Peano] is mainly aesthetic. I definitely claim that my
% results on doing Peano all apply to Metamath as it is presented in your
% book.
%
%         --Bob
%
%%%%%%%%%%%%%%%% (end of 2nd email comment)

%%
%% When uncommenting the below, also uncomment references above to {formalspec}
%%
\chapter{Metamath's Formal System}\label{formalspec}\index{Metamath!as a formal
system}

\section{Introduction}

\begin{quote}
  {\em Perfection is when there is no longer anything more to take away.}
    \flushright\sc Antoine de
     Saint-Exupery\footnote{\cite[p.~3-25]{Campbell}.}\\
\end{quote}\index{de Saint-Exupery, Antoine}

This appendix describes the theory behind the Metamath language in an abstract
way intended for mathematicians.  Specifically, we construct two
set-theo\-ret\-i\-cal objects:  a ``formal system'' (roughly, a set of syntax
rules, axioms, and logical rules) and its ``universe'' (roughly, the set of
theorems derivable in the formal system).  The Metamath computer language
provides us with a way to describe specific formal systems and, with the aid of
a proof provided by the user, to verify that given theorems
belong to their universes.

To understand this appendix, you need a basic knowledge of informal set theory.
It should be sufficient to understand, for example, Ch.\ 1 of Munkres' {\em
Topology} \cite{Munkres}\index{Munkres, James R.} or the
introductory set theory chapter
in many textbooks that introduce abstract mathematics. (Note that there are
minor notational differences among authors; e.g.\ Munkres uses $\subset$ instead
of our $\subseteq$ for ``subset.''  We use ``included in'' to mean ``a subset
of,'' and ``belongs to'' or ``is contained in'' to mean ``is an element of.'')
What we call a ``formal'' description here, unlike earlier, is actually an
informal description in the ordinary language of mathematicians.  However we
provide sufficient detail so that a mathematician could easily formalize it,
even in the language of Metamath itself if desired.  To understand the logic
examples at the end of this appendix, familiarity with an introductory book on
mathematical logic would be helpful.

\section{The Formal Description}

\subsection[Preliminaries]{Preliminaries\protect\footnotemark}%
\footnotetext{This section is taken mostly verbatim
from Tarski \cite[p.~63]{Tarski1965}\index{Tarski, Alfred}.}

By $\omega$ we denote the set of all natural numbers (non-negative integers).
Each natural number $n$ is identified with the set of all smaller numbers: $n =
\{ m | m < n \}$.  The formula $m < n$ is thus equivalent to the condition: $m
\in n$ and $m,n \in \omega$. In particular, 0 is the number zero and at the
same time the empty set $\varnothing$, $1=\{0\}$, $2=\{0,1\}$, etc. ${}^B A$
denotes the set of all functions on $B$ to $A$ (i.e.\ with domain $B$ and range
included in $A$).  The members of ${}^\omega A$ are what are called {\em simple
infinite sequences},\index{simple infinite sequence}
with all {\em terms}\index{term} in $A$.  In case $n \in \omega$, the
members of ${}^n A$ are referred to as {\em finite $n$-termed
sequences},\index{finite $n$-termed
sequence} again
with terms in $A$.  The consecutive terms (function values) of a finite or
infinite sequence $f$ are denoted by $f_0, f_1, \ldots ,f_n,\ldots$.  Every
finite sequence $f \in \bigcup _{n \in \omega} {}^n A$ uniquely determines the
number $n$ such that $f \in {}^n A$; $n$ is called the {\em
length}\index{length of a sequence ({$"|\ "|$})} of $f$ and
is denoted by $|f|$.  $\langle a \rangle$ is the sequence $f$ with $|f|=1$ and
$f_0=a$; $\langle a,b \rangle$ is the sequence $f$ with $|f|=2$, $f_0=a$,
$f_1=b$; etc.  Given two finite sequences $f$ and $g$, we denote by $f\frown g$
their {\em concatenation},\index{concatenation} i.e., the
finite sequence $h$ determined by the
conditions:
\begin{eqnarray*}
& |h| = |f|+|g|;&  \\
& h_n = f_n & \mbox{\ for\ } n < |f|;  \\
& h_{|f|+n} = g_n & \mbox{\ for\ } n < |g|.
\end{eqnarray*}

\subsection{Constants, Variables, and Expressions}

A formal system has a set of {\em symbols}\index{symbol!in
a formal system} denoted
by $\mbox{\em SM}$.  A
precise set-theo\-ret\-i\-cal definition of this set is unimportant; a symbol
could be considered a primitive or atomic element if we wish.  We assume this
set is divided into two disjoint subsets:  a set $\mbox{\em CN}$ of {\em
constants}\index{constant!in a formal system} and a set $\mbox{\em VR}$ of
{\em variables}.\index{variable!in a formal system}  $\mbox{\em CN}$ and
$\mbox{\em VR}$ are each assumed to consist of countably many symbols which
may be arranged in finite or simple infinite sequences $c_0, c_1, \ldots$ and
$v_0, v_1, \ldots$ respectively, without repeating terms.  We will represent
arbitrary symbols by metavariables $\alpha$, $\beta$, etc.

{\footnotesize\begin{quotation}
{\em Comment.} The variables $v_0, v_1, \ldots$ of our formal system
correspond to what are usually considered ``metavariables'' in
descriptions of specific formal systems in the literature.  Typically,
when describing a specific formal system a book will postulate a set of
primitive objects called variables, then proceed to describe their
properties using metavariables that range over them, never mentioning
again the actual variables themselves.  Our formal system does not
mention these primitive variable objects at all but deals directly with
metavariables, as its primitive objects, from the start.  This is a
subtle but key distinction you should keep in mind, and it makes our
definition of ``formal system'' somewhat different from that typically
found in the literature.  (So, the $\alpha$, $\beta$, etc.\ above are
actually ``metametavariables'' when used to represent $v_0, v_1,
\ldots$.)
\end{quotation}}

Finite sequences all terms of which are symbols are called {\em
expressions}.\index{expression!in a formal system}  $\mbox{\em EX}$ is
the set of all expressions; thus
\begin{displaymath}
\mbox{\em EX} = \bigcup _{n \in \omega} {}^n \mbox{\em SM}.
\end{displaymath}

A {\em constant-prefixed expression}\index{constant-prefixed expression}
is an expression of non-zero length
whose first term is a constant.  We denote the set of all constant-prefixed
expressions by $\mbox{\em EX}_C = \{ e \in \mbox{\em EX} | ( |e| > 0 \wedge
e_0 \in \mbox{\em CN} ) \}$.

A {\em constant-variable pair}\index{constant-variable pair}
is an expression of length 2 whose first term
is a constant and whose second term is a variable.  We denote the set of all
constant-variable pairs by $\mbox{\em EX}_2 = \{ e \in \mbox{\em EX}_C | ( |e|
= 2 \wedge e_1 \in \mbox{\em VR} ) \}$.


{\footnotesize\begin{quotation}
{\em Relationship to Metamath.} In general, the set $\mbox{\em SM}$
corresponds to the set of declared math symbols in a Metamath database, the
set $\mbox{\em CN}$ to those declared with \texttt{\$c} statements, and the set
$\mbox{\em VR}$ to those declared with \texttt{\$v} statements.  Of course a
Metamath database can only have a finite number of math symbols, whereas
formal systems in general can have an infinite number, although the number of
Metamath math symbols available is in principle unlimited.

The set $\mbox{\em EX}_C$ corresponds to the set of permissible expressions
for \texttt{\$e}, \texttt{\$a}, and \texttt{\$p} statements.  The set $\mbox{\em EX}_2$
corresponds to the set of permissible expressions for \texttt{\$f} statements.
\end{quotation}}

We denote by ${\cal V}(e)$ the set of all variables in an expression $e \in
\mbox{\em EX}$, i.e.\ the set of all $\alpha \in \mbox{\em VR}$ such that
$\alpha = e_n$ for some $n < |e|$.  We also denote (with abuse of notation) by
${\cal V}(E)$ the set of all variables in a collection of expressions $E
\subseteq \mbox{\em EX}$, i.e.\ $\bigcup _{e \in E} {\cal V}(e)$.


\subsection{Substitution}

Given a function $F$ from $\mbox{\em VR}$ to
$\mbox{\em EX}$, we
denote by $\sigma_{F}$ or just $\sigma$ the function from $\mbox{\em EX}$ to
$\mbox{\em EX}$ defined recursively for nonempty sequences by
\begin{eqnarray*}
& \sigma(<\alpha>) = F(\alpha) & \mbox{for\ } \alpha \in \mbox{\em VR}; \\
& \sigma(<\alpha>) = <\alpha> & \mbox{for\ } \alpha \not\in \mbox{\em VR}; \\
& \sigma(g \frown h) = \sigma(g) \frown
    \sigma(h) & \mbox{for\ } g,h \in \mbox{\em EX}.
\end{eqnarray*}
We also define $\sigma(\varnothing)=\varnothing$.  We call $\sigma$ a {\em
simultaneous substitution}\index{substitution!variable}\index{variable
substitution} (or just {\em substitution}) with {\em substitution
map}\index{substitution map} $F$.

We also denote (with abuse of notation) by $\sigma(E)$ a substitution on a
collection of expressions $E \subseteq \mbox{\em EX}$, i.e.\ the set $\{
\sigma(e) | e \in E \}$.  The collection $\sigma(E)$ may of course contain
fewer expressions than $E$ because duplicate expressions could result from the
substitution.

\subsection{Statements}

We denote by $\mbox{\em DV}$ the set of all
unordered pairs $\{\alpha, \beta \} \subseteq \mbox{\em VR}$ such that $\alpha
\neq \beta$.  $\mbox{\em DV}$ stands for ``distinct variables.''

A {\em pre-statement}\index{pre-statement!in a formal system} is a
quadruple $\langle D,T,H,A \rangle$ such that
$D\subseteq \mbox{\em DV}$, $T\subseteq \mbox{\em EX}_2$, $H\subseteq
\mbox{\em EX}_C$ and $H$ is finite,
$A\in \mbox{\em EX}_C$, ${\cal V}(H\cup\{A\}) \subseteq
{\cal V}(T)$, and $\forall e,f\in T {\ } {\cal V}(e) \neq {\cal V}(f)$ (or
equivalently, $e_1 \ne f_1$) whenever $e \neq f$. The terms of the quadruple are called {\em
distinct-variable restrictions},\index{disjoint-variable restriction!in a
formal system} {\em variable-type hypotheses},\index{variable-type
hypothesis!in a formal system} {\em logical hypotheses},\index{logical
hypothesis!in a formal system} and the {\em assertion}\index{assertion!in a
formal system} respectively.  We denote by $T_M$ ({\em mandatory variable-type
hypotheses}\index{mandatory variable-type hypothesis!in a formal system}) the
subset of $T$ such that ${\cal V}(T_M) ={\cal V}(H \cup \{A\})$.  We denote by
$D_M=\{\{\alpha,\beta\}\in D|\{\alpha,\beta\}\subseteq {\cal V}(T_M)\}$ the
{\em mandatory distinct-variable restrictions}\index{mandatory
disjoint-variable restriction!in a formal system} of the pre-statement.
The set
of {\em mandatory hypotheses}\index{mandatory hypothesis!in a formal system}
is $T_M\cup H$.  We call the quadruple $\langle D_M,T_M,H,A \rangle$
the {\em reduct}\index{reduct!in a formal system} of
the pre-statement $\langle D,T,H,A \rangle$.

A {\em statement} is the reduct of some pre-statement\index{statement!in a
formal system}.  A statement is therefore a special kind of pre-statement;
in particular, a statement is the reduct of itself.

{\footnotesize\begin{quotation}
{\em Comment.}  $T$ is a set of expressions, each of length 2, that associate
a set of constants (``variable types'') with a set of variables.  The
condition ${\cal V}(H\cup\{A\}) \subseteq {\cal V}(T) $
means that each variable occurring in a statement's logical
hypotheses or assertion must have an associated variable-type hypothesis or
``type declaration,'' in  analogy to a computer programming language, where a
variable must be declared to be say, a string or an integer.  The requirement
that $\forall e,f\in T \, e_1 \ne f_1$ for $e\neq f$
means that each variable must be
associated with a unique constant designating its variable type; e.g., a
variable might be a ``wff'' or a ``set'' but not both.

Distinct-variable restrictions are used to specify what variable substitutions
are permissible to make for the statement to remain valid.  For example, in
the theorem scheme of set theory $\lnot\forall x\,x=y$ we may not substitute
the same variable for both $x$ and $y$.  On the other hand, the theorem scheme
$x=y\to y=x$ does not require that $x$ and $y$ be distinct, so we do not
require a distinct-variable restriction, although having one
would cause no harm other than making the scheme less general.

A mandatory variable-type hypothesis is one whose variable exists in a logical
hypothesis or the assertion.  A provable pre-statement
(defined below) may require
non-mandatory variable-type hypotheses that effectively introduce ``dummy''
variables for use in its proof.  Any number of dummy variables might
be required by a specific proof; indeed, it has been shown by H.\
Andr\'{e}ka\index{Andr{\'{e}}ka, H.} \cite{Nemeti} that there is no finite
upper bound to the number of dummy variables needed to prove an arbitrary
theorem in first-order logic (with equality) having a fixed number $n>2$ of
individual variables.  (See also the Comment on p.~\pageref{nodd}.)
For this reason we do not set a finite size bound on the collections $D$ and
$T$, although in an actual application (Metamath database) these will of
course be finite, increased to whatever size is necessary as more
proofs are added.
\end{quotation}}

{\footnotesize\begin{quotation}
{\em Relationship to Metamath.} A pre-statement of a formal system
corresponds to an extended frame in a Metamath database
(Section~\ref{frames}).  The collections $D$, $T$, and $H$ correspond
respectively to the \texttt{\$d}, \texttt{\$f}, and \texttt{\$e}
statement collections in an extended frame.  The expression $A$
corresponds to the \texttt{\$a} (or \texttt{\$p}) statement in an
extended frame.

A statement of a formal system corresponds to a frame in a Metamath
database.
\end{quotation}}

\subsection{Formal Systems}

A {\em formal system}\index{formal system} is a
triple $\langle \mbox{\em CN},\mbox{\em
VR},\Gamma\rangle$ where $\Gamma$ is a set of statements.  The members of
$\Gamma$ are called {\em axiomatic statements}.\index{axiomatic
statement!in a formal system}  Sometimes we will refer to a
formal system by just $\Gamma$ when $\mbox{\em CN}$ and $\mbox{\em VR}$ are
understood.

Given a formal system $\Gamma$, the {\em closure}\index{closure}\footnote{This
definition of closure incorporates a simplification due to
Josh Purinton.\index{Purinton, Josh}.} of a
pre-statement
$\langle D,T,H,A \rangle$ is the smallest set $C$ of expressions
such that:
%\begin{enumerate}
%  \item $T\cup H\subseteq C$; and
%  \item If for some axiomatic statement
%    $\langle D_M',T_M',H',A' \rangle \in \Gamma_A$, for
%    some $E \subseteq C$, some $F \subseteq C-T$ (where ``-'' denotes
%    set difference), and some substitution
%    $\sigma$ we have
%    \begin{enumerate}
%       \item $\sigma(T_M') = E$ (where, as above, the $M$ denotes the
%           mandatory variable-type hypotheses of $T^A$);
%       \item $\sigma(H') = F$;
%       \item for all $\{\alpha,\beta\}\in D^A$ and $\subseteq
%         {\cal V}(T_M')$, for all $\gamma\in {\cal V}(\sigma(\langle \alpha
%         \rangle))$, and for all $\delta\in  {\cal V}(\sigma(\langle \beta
%         \rangle))$, we have $\{\gamma, \delta\} \in D$;
%   \end{enumerate}
%   then $\sigma(A') \in C$.
%\end{enumerate}
\begin{list}{}{\itemsep 0.0pt}
  \item[1.] $T\cup H\subseteq C$; and
  \item[2.] If for some axiomatic statement
    $\langle D_M',T_M',H',A' \rangle \in
       \Gamma$ and for some substitution
    $\sigma$ we have
    \begin{enumerate}
       \item[a.] $\sigma(T_M' \cup H') \subseteq C$; and
       \item[b.] for all $\{\alpha,\beta\}\in D_M'$, for all $\gamma\in
         {\cal V}(\sigma(\langle \alpha
         \rangle))$, and for all $\delta\in  {\cal V}(\sigma(\langle \beta
         \rangle))$, we have $\{\gamma, \delta\} \in D$;
   \end{enumerate}
   then $\sigma(A') \in C$.
\end{list}
A pre-statement $\langle D,T,H,A
\rangle$ is {\em provable}\index{provable statement!in a formal
system} if $A\in C$ i.e.\ if its assertion belongs to its
closure.  A statement is {\em provable} if it is
the reduct of a provable pre-statement.
The {\em universe}\index{universe of a formal system}
of a formal system is
the collection of all of its provable statements.  Note that the
set of axiomatic statements $\Gamma$ in a formal system is a subset of its
universe.

{\footnotesize\begin{quotation}
{\em Comment.} The first condition in the definition of closure simply says
that the hypotheses of the pre-statement are in its closure.

Condition 2(a) says that a substitution exists that makes the
mandatory hypotheses of an axiomatic statement exactly match some members of
the closure.  This is what we explicitly demonstrate in a Metamath language
proof.

%Conditions 2(a) and 2(b) say that a substitution exists that makes the
%(mandatory) hypotheses of an axiomatic statement exactly match some members of
%the closure.  This is what we explicitly demonstrate with a Metamath language
%proof.
%
%The set of expressions $F$ in condition 2(b) excludes the variable-type
%hypotheses; this is done because non-mandatory variable-type hypotheses are
%effectively ``dropped'' as irrelevant whereas logical hypotheses must be
%retained to achieve a consistent logical system.

Condition 2(b) describes how distinct-variable restrictions in the axiomatic
statement must be met.  It means that after a substitution for two variables
that must be distinct, the resulting two expressions must either contain no
variables, or if they do, they may not have variables in common, and each pair
of any variables they do have, with one variable from each expression, must be
specified as distinct in the original statement.
\end{quotation}}

{\footnotesize\begin{quotation}
{\em Relationship to Metamath.} Axiomatic statements
 and provable statements in a formal
system correspond to the frames for \texttt{\$a} and \texttt{\$p} statements
respectively in a Metamath database.  The set of axiomatic statements is a
subset of the set of provable statements in a formal system, although in a
Metamath database a \texttt{\$a} statement is distinguished by not having a
proof.  A Metamath language proof for a \texttt{\$p} statement tells the computer
how to explicitly construct a series of members of the closure ultimately
leading to a demonstration that the assertion
being proved is in the closure.  The actual closure typically contains
an infinite number of expressions.  A formal system itself does not have
an explicit object called a ``proof'' but rather the existence of a proof
is implied indirectly by membership of an assertion in a provable
statement's closure.  We do this to make the formal system easier
to describe in the language of set theory.

We also note that once established as provable, a statement may be considered
to acquire the same status as an axiomatic statement, because if the set of
axiomatic statements is extended with a provable statement, the universe of
the formal system remains unchanged (provided that $\mbox{\em VR}$ is
infinite).
In practice, this means we can build a hierarchy of provable statements to
more efficiently establish additional provable statements.  This is
what we do in Metamath when we allow proofs to reference previous
\texttt{\$p} statements as well as previous \texttt{\$a} statements.
\end{quotation}}

\section{Examples of Formal Systems}

{\footnotesize\begin{quotation}
{\em Relationship to Metamath.} The examples in this section, except Example~2,
are for the most part exact equivalents of the development in the set
theory database \texttt{set.mm}.  You may want to compare Examples~1, 3, and 5
to Section~\ref{metaaxioms}, Example 4 to Sections~\ref{metadefprop} and
\ref{metadefpred}, and Example 6 to
Section~\ref{setdefinitions}.\label{exampleref}
\end{quotation}}

\subsection{Example~1---Propositional Calculus}\index{propositional calculus}

Classical propositional calculus can be described by the following formal
system.  We assume the set of variables is infinite.  Rather than denoting the
constants and variables by $c_0, c_1, \ldots$ and $v_0, v_1, \ldots$, for
readability we will instead use more conventional symbols, with the
understanding of course that they denote distinct primitive objects.
Also for readability we may omit commas between successive terms of a
sequence; thus $\langle \mbox{wff\ } \varphi\rangle$ denotes
$\langle \mbox{wff}, \varphi\rangle$.

Let
\begin{itemize}
  \item[] $\mbox{\em CN}=\{\mbox{wff}, \vdash, \to, \lnot, (,)\}$
  \item[] $\mbox{\em VR}=\{\varphi,\psi,\chi,\ldots\}$
  \item[] $T = \{\langle \mbox{wff\ } \varphi\rangle,
             \langle \mbox{wff\ } \psi\rangle,
             \langle \mbox{wff\ } \chi\rangle,\ldots\}$, i.e.\ those
             expressions of length 2 whose first member is $\mbox{\rm wff}$
             and whose second member belongs to $\mbox{\em VR}$.\footnote{For
convenience we let $T$ be an infinite set; the definition of a statement
permits this in principle.  Since a Metamath source file has a finite size, in
practice we must of course use appropriate finite subsets of this $T$,
specifically ones containing at least the mandatory variable-type
hypotheses.  Similarly, in the source file we introduce new variables as
required, with the understanding that a potentially infinite number of
them are available.}
\noindent Then $\Gamma$ consists of the axiomatic statements that
are the reducts of the following pre-statements:
    \begin{itemize}
      \item[] $\langle\varnothing,T,\varnothing,
               \langle \mbox{wff\ }(\varphi\to\psi)\rangle\rangle$
      \item[] $\langle\varnothing,T,\varnothing,
               \langle \mbox{wff\ }\lnot\varphi\rangle\rangle$
      \item[] $\langle\varnothing,T,\varnothing,
               \langle \vdash(\varphi\to(\psi\to\varphi))
               \rangle\rangle$
      \item[] $\langle\varnothing,T,
               \varnothing,
               \langle \vdash((\varphi\to(\psi\to\chi))\to
               ((\varphi\to\psi)\to(\varphi\to\chi)))
               \rangle\rangle$
      \item[] $\langle\varnothing,T,
               \varnothing,
               \langle \vdash((\lnot\varphi\to\lnot\psi)\to
               (\psi\to\varphi))\rangle\rangle$
      \item[] $\langle\varnothing,T,
               \{\langle\vdash(\varphi\to\psi)\rangle,
                 \langle\vdash\varphi\rangle\},
               \langle\vdash\psi\rangle\rangle$
    \end{itemize}
\end{itemize}

(For example, the reduct of $\langle\varnothing,T,\varnothing,
               \langle \mbox{wff\ }(\varphi\to\psi)\rangle\rangle$
is
\begin{itemize}
\item[] $\langle\varnothing,
\{\langle \mbox{wff\ } \varphi\rangle,
             \langle \mbox{wff\ } \psi\rangle\},
             \varnothing,
               \langle \mbox{wff\ }(\varphi\to\psi)\rangle\rangle$,
\end{itemize}
which is the first axiomatic statement.)

We call the members of $\mbox{\em VR}$ {\em wff variables} or (in the context
of first-order logic which we will describe shortly) {\em wff metavariables}.
Note that the symbols $\phi$, $\psi$, etc.\ denote actual specific members of
$\mbox{\em VR}$; they are not metavariables of our expository language (which
we denote with $\alpha$, $\beta$, etc.) but are instead (meta)constant symbols
(members of $\mbox{\em SM}$) from the point of view of our expository
language.  The equivalent system of propositional calculus described in
\cite{Tarski1965} also uses the symbols $\phi$, $\psi$, etc.\ to denote wff
metavariables, but in \cite{Tarski1965} unlike here those are metavariables of
the expository language and not primitive symbols of the formal system.

The first two statements define wffs: if $\varphi$ and $\psi$ are wffs, so is
$(\varphi \to \psi)$; if $\varphi$ is a wff, so is $\lnot\varphi$. The next
three are the axioms of propositional calculus: if $\varphi$ and $\psi$ are
wffs, then $\vdash (\varphi \to (\psi \to \varphi))$ is an (axiomatic)
theorem; etc. The
last is the rule of modus ponens: if $\varphi$ and $\psi$ are wffs, and
$\vdash (\varphi\to\psi)$ and $\vdash \varphi$ are theorems, then $\vdash
\psi$ is a theorem.

The correspondence to ordinary propositional calculus is as follows.  We
consider only provable statements of the form $\langle\varnothing,
T,\varnothing,A\rangle$ with $T$ defined as above.  The first term of the
assertion $A$ of any such statement is either ``wff'' or ``$\vdash$''.  A
statement for which the first term is ``wff'' is a {\em wff} of propositional
calculus, and one where the first term is ``$\vdash$'' is a {\em
theorem (scheme)} of propositional calculus.

The universe of this formal system also contains many other provable
statements.  Those with distinct-variable restrictions are irrelevant because
propositional calculus has no constraints on substitutions.  Those that have
logical hypotheses we call {\em inferences}\index{inference} when
the logical hypotheses are of the form
$\langle\vdash\rangle\frown w$ where $w$ is a wff (with the leading constant
term ``wff'' removed).  Inferences (other than the modus ponens rule) are not a
proper part of propositional calculus but are convenient to use when building a
hierarchy of provable statements.  A provable statement with a nonsense
hypothesis such as $\langle \to,\vdash,\lnot\rangle$, and this same expression
as its assertion, we consider irrelevant; no use can be made of it in
proving theorems, since there is no way to eliminate the nonsense hypothesis.

{\footnotesize\begin{quotation}
{\em Comment.} Our use of parentheses in the definition of a wff illustrates
how axiomatic statements should be carefully stated in a way that
ties in unambiguously with the substitutions allowed by the formal system.
There are many ways we could have defined wffs---for example, Polish
prefix notation would have allowed us to omit parentheses entirely, at
the expense of readability---but we must define them in a way that is
unambiguous.  For example, if we had omitted parentheses from the
definition of $(\varphi\to \psi)$, the wff $\lnot\varphi\to \psi$ could
be interpreted as either $\lnot(\varphi\to\psi)$ or $(\lnot\varphi\to\psi)$
and would have allowed us to prove nonsense.  Note that there is no
concept of operator binding precedence built into our formal system.
\end{quotation}}

\begin{sloppy}
\subsection{Example~2---Predicate Calculus with Equality}\index{predicate
calculus}
\end{sloppy}

Here we extend Example~1 to include predicate calculus with equality,
illustrating the use of distinct-variable restrictions.  This system is the
same as Tarski's system $\mathfrak{S}_2$ in \cite{Tarski1965} (except that the
axioms of propositional calculus are different but equivalent, and a redundant
axiom is omitted).  We extend $\mbox{\em CN}$ with the constants
$\{\mbox{var},\forall,=\}$.  We extend $\mbox{\em VR}$ with an infinite set of
{\em individual metavariables}\index{individual
metavariable} $\{x,y,z,\ldots\}$ and denote this subset
$\mbox{\em Vr}$.

We also join to $\mbox{\em CN}$ a possibly infinite set $\mbox{\em Pr}$ of {\em
predicates} $\{R,S,\ldots\}$.  We associate with $\mbox{\em Pr}$ a function
$\mbox{rnk}$ from $\mbox{\em Pr}$ to $\omega$, and for $\alpha\in \mbox{\em
Pr}$ we call $\mbox{rnk}(\alpha)$ the {\em rank} of the predicate $\alpha$,
which is simply the number of ``arguments'' that the predicate has.  (Most
applications of predicate calculus will have a finite number of predicates;
for example, set theory has the single two-argument or binary predicate $\in$,
which is usually written with its arguments surrounding the predicate symbol
rather than with the prefix notation we will use for the general case.)  As a
device to facilitate our discussion, we will let $\mbox{\em Vs}$ be any fixed
one-to-one function from $\omega$ to $\mbox{\em Vr}$; thus $\mbox{\em Vs}$ is
any simple infinite sequence of individual metavariables with no repeating
terms.

In this example we will not include the function symbols that are often part of
formalizations of predicate calculus.  Using metalogical arguments that are
beyond the scope of our discussion, it can be shown that our formalization is
equivalent when functions are introduced via appropriate definitions.

We extend the set $T$ defined in Example~1 with the expressions
$\{\langle \mbox{var\ } x\rangle,$ $ \langle \mbox{var\ } y\rangle, \langle
\mbox{var\ } z\rangle,\ldots\}$.  We extend the $\Gamma$ above
with the axiomatic statements that are the reducts of the following
pre-statements:
\begin{list}{}{\itemsep 0.0pt}
      \item[] $\langle\varnothing,T,\varnothing,
               \langle \mbox{wff\ }\forall x\,\varphi\rangle\rangle$
      \item[] $\langle\varnothing,T,\varnothing,
               \langle \mbox{wff\ }x=y\rangle\rangle$
      \item[] $\langle\varnothing,T,
               \{\langle\vdash\varphi\rangle\},
               \langle\vdash\forall x\,\varphi\rangle\rangle$
      \item[] $\langle\varnothing,T,\varnothing,
               \langle \vdash((\forall x(\varphi\to\psi)
                  \to(\forall x\,\varphi\to\forall x\,\psi))
               \rangle\rangle$
      \item[] $\langle\{\{x,\varphi\}\},T,\varnothing,
               \langle \vdash(\varphi\to\forall x\,\varphi)
               \rangle\rangle$
      \item[] $\langle\{\{x,y\}\},T,\varnothing,
               \langle \vdash\lnot\forall x\lnot x=y
               \rangle\rangle$
      \item[] $\langle\varnothing,T,\varnothing,
               \langle \vdash(x=z
                  \to(x=y\to z=y))
               \rangle\rangle$
      \item[] $\langle\varnothing,T,\varnothing,
               \langle \vdash(y=z
                  \to(x=y\to x=z))
               \rangle\rangle$
\end{list}
These are the axioms not involving predicate symbols. The first two statements
extend the definition of a wff.  The third is the rule of generalization.  The
fifth states, in effect, ``For a wff $\varphi$ and variable $x$,
$\vdash(\varphi\to\forall x\,\varphi)$, provided that $x$ does not occur in
$\varphi$.''  The sixth states ``For variables $x$ and $y$,
$\vdash\lnot\forall x\lnot x = y$, provided that $x$ and $y$ are distinct.''
(This proviso is not necessary but was included by Tarski to
weaken the axiom and still show that the system is logically complete.)

Finally, for each predicate symbol $\alpha\in \mbox{\em Pr}$, we add to
$\Gamma$ an axiomatic statement, extending the definition of wff,
that is the reduct of the following pre-statement:
\begin{displaymath}
    \langle\varnothing,T,\varnothing,
            \langle \mbox{wff},\alpha\rangle\
            \frown \mbox{\em Vs}\restriction\mbox{rnk}(\alpha)\rangle
\end{displaymath}
and for each $\alpha\in \mbox{\em Pr}$ and each $n < \mbox{rnk}(\alpha)$
we add to $\Gamma$ an equality axiom that is the reduct of the
following pre-statement:
\begin{eqnarray*}
    \lefteqn{\langle\varnothing,T,\varnothing,
            \langle
      \vdash,(,\mbox{\em Vs}_n,=,\mbox{\em Vs}_{\mbox{rnk}(\alpha)},\to,
            (,\alpha\rangle\frown \mbox{\em Vs}\restriction\mbox{rnk}(\alpha)} \\
  & & \frown
            \langle\to,\alpha\rangle\frown \mbox{\em Vs}\restriction n\frown
            \langle \mbox{\em Vs}_{\mbox{rnk}(\alpha)}\rangle \\
 & & \frown
            \mbox{\em Vs}\restriction(\mbox{rnk}(\alpha)\setminus(n+1))\frown
            \langle),)\rangle\rangle
\end{eqnarray*}
where $\restriction$ denotes function domain restriction and $\setminus$
denotes set difference.  Recall that a subscript on $\mbox{\em Vs}$
denotes one of its terms.  (In the above two axiom sets commas are placed
between successive terms of sequences to prevent ambiguity, and if you examine
them with care you will be able to distinguish those parentheses that denote
constant symbols from those of our expository language that delimit function
arguments.  Although it might have been better to use boldface for our
primitive symbols, unfortunately boldface was not available for all characters
on the \LaTeX\ system used to typeset this text.)  These seemingly forbidding
axioms can be understood by analogy to concatenation of substrings in a
computer language.  They are actually relatively simple for each specific case
and will become clearer by looking at the special case of a binary predicate
$\alpha = R$ where $\mbox{rnk}(R)=2$.  Letting $\mbox{\em Vs}$ be the sequence
$\langle x,y,z,\ldots\rangle$, the axioms we would add to $\Gamma$ for this
case would be the wff extension and two equality axioms that are the
reducts of the pre-statements:
\begin{list}{}{\itemsep 0.0pt}
      \item[] $\langle\varnothing,T,\varnothing,
               \langle \mbox{wff\ }R x y\rangle\rangle$
      \item[] $\langle\varnothing,T,\varnothing,
               \langle \vdash(x=z
                  \to(R x y \to R z y))
               \rangle\rangle$
      \item[] $\langle\varnothing,T,\varnothing,
               \langle \vdash(y=z
                  \to(R x y \to R x z))
               \rangle\rangle$
\end{list}
Study these carefully to see how the general axioms above evaluate to
them.  In practice, typically only a few special cases such as this would be
needed, and in any case the Metamath language will only permit us to describe
a finite number of predicates, as opposed to the infinite number permitted by
the formal system.  (If an infinite number should be needed for some reason,
we could not define the formal system directly in the Metamath language but
could instead define it metalogically under set theory as we
do in this appendix, and only the underlying set theory, with its single
binary predicate, would be defined directly in the Metamath language.)


{\footnotesize\begin{quotation}
{\em Comment.}  As we noted earlier, the specific variables denoted by the
symbols $x,y,z,\ldots\in \mbox{\em Vr}\subseteq \mbox{\em VR}\subseteq
\mbox{\em SM}$ in Example~2 are not the actual variables of ordinary predicate
calculus but should be thought of as metavariables ranging over them.  For
example, a distinct-variable restriction would be meaningless for actual
variables of ordinary predicate calculus since two different actual variables
are by definition distinct.  And when we talk about an arbitrary
representative $\alpha\in \mbox{\em Vr}$, $\alpha$ is a metavariable (in our
expository language) that ranges over metavariables (which are primitives of
our formal system) each of which ranges over the actual individual variables
of predicate calculus (which are never mentioned in our formal system).

The constant called ``var'' above is called \texttt{setvar} in the
\texttt{set.mm} database file, but it means the same thing.  I felt
that ``var'' is a more meaningful name in the context of predicate
calculus, whose use is not limited to set theory.  For consistency we
stick with the name ``var'' throughout this Appendix, even after set
theory is introduced.
\end{quotation}}

\subsection{Free Variables and Proper Substitution}\index{free variable}
\index{proper substitution}\index{substitution!proper}

Typical representations of mathematical axioms use concepts such
as ``free variable,'' ``bound variable,'' and ``proper substitution''
as primitive notions.
A free variable is a variable that
is not a parameter of any container expression.
A bound variable is the opposite of a free variable; it is a
a variable that has been bound in a container expression.
For example, in the expression $\forall x \varphi$ (for all $x$, $\varphi$
is true), the variable $x$
is bound within the for-all ($\forall$) expression.
It is possible to change one variable to another, and that process is called
``proper substitution.''
In most books, proper substitution has a somewhat complicated recursive
definition with multiple cases based on the occurrences of free and
bound variables.
You may consult
\cite[ch.\ 3--4]{Hamilton}\index{Hamilton, Alan G.} (as well as
many other texts) for more formal details about these terms.

Using these concepts as \texttt{primitives} creates complications
for computer implementations.

In the system of Example~2, there are no primitive notions of free variable
and proper substitution.  Tarski \cite{Tarski1965} shows that this system is
logically equivalent to the more typical textbook systems that do have these
primitive notions, if we introduce these notions with appropriate definitions
and metalogic.  We could also define axioms for such systems directly,
although the recursive definitions of free variable and proper substitution
would be messy and awkward to work with.  Instead, we mention two devices that
can be used in practice to mimic these notions.  (1) Instead of introducing
special notation to express (as a logical hypothesis) ``where $x$ is not free
in $\varphi$'' we can use the logical hypothesis $\vdash(\varphi\to\forall
x\,\varphi)$.\label{effectivelybound}\index{effectively
not free}\footnote{This is a slightly weaker requirement than ``where $x$ is
not free in $\varphi$.''  If we let $\varphi$ be $x=x$, we have the theorem
$(x=x\to\forall x\,x=x)$ which satisfies the hypothesis, even though $x$ is
free in $x=x$ .  In a case like this we say that $x$ is {\em effectively not
free}\index{effectively not free} in $x=x$, since $x=x$ is logically
equivalent to $\forall x\,x=x$ in which $x$ is bound.} (2) It can be shown
that the wff $((x=y\to\varphi)\wedge\exists x(x=y\wedge\varphi))$ (with the
usual definitions of $\wedge$ and $\exists$; see Example~4 below) is logically
equivalent to ``the wff that results from proper substitution of $y$ for $x$
in $\varphi$.''  This works whether or not $x$ and $y$ are distinct.

\subsection{Metalogical Completeness}\index{metalogical completeness}

In the system of Example~2, the
following are provable pre-statements (and their reducts are
provable statements):
\begin{eqnarray*}
      & \langle\{\{x,y\}\},T,\varnothing,
               \langle \vdash\lnot\forall x\lnot x=y
               \rangle\rangle & \\
     &  \langle\varnothing,T,\varnothing,
               \langle \vdash\lnot\forall x\lnot x=x
               \rangle\rangle &
\end{eqnarray*}
whereas the following pre-statement is not to my knowledge provable (but
in any case we will pretend it's not for sake of illustration):
\begin{eqnarray*}
     &  \langle\varnothing,T,\varnothing,
               \langle \vdash\lnot\forall x\lnot x=y
               \rangle\rangle &
\end{eqnarray*}
In other words, we can prove ``$\lnot\forall x\lnot x=y$ where $x$ and $y$ are
distinct'' and separately prove ``$\lnot\forall x\lnot x=x$'', but we can't
prove the combined general case ``$\lnot\forall x\lnot x=y$'' that has no
proviso.  Now this does not compromise logical completeness, because the
variables are really metavariables and the two provable cases together cover
all possible cases.  The third case can be considered a metatheorem whose
direct proof, using the system of Example~2, lies outside the capability of the
formal system.

Also, in the system of Example~2 the following pre-statement is not to my
knowledge provable (again, a conjecture that we will pretend to be the case):
\begin{eqnarray*}
     & \langle\varnothing,T,\varnothing,
               \langle \vdash(\forall x\, \varphi\to\varphi)
               \rangle\rangle &
\end{eqnarray*}
Instead, we can only prove specific cases of $\varphi$ involving individual
metavariables, and by induction on formula length, prove as a metatheorem
outside of our formal system the general statement above.  The details of this
proof are found in \cite{Kalish}.

There does, however, exist a system of predicate calculus in which all such
``simple metatheorems'' as those above can be proved directly, and we present
it in Example~3. A {\em simple metatheorem}\index{simple metatheorem}
is any statement of the formal
system of Example~2 where all distinct variable restrictions consist of either
two individual metavariables or an individual metavariable and a wff
metavariable, and which is provable by combining cases outside the system as
above.  A system is {\em metalogically complete}\index{metalogical
completeness} if all of its simple
metatheorems are (directly) provable statements. The precise definition of
``simple metatheorem'' and the proof of the ``metalogical completeness'' of
Example~3 is found in Remark 9.6 and Theorem 9.7 of \cite{Megill}.\index{Megill,
Norman}

\begin{sloppy}
\subsection{Example~3---Metalogically Complete Predicate
Calculus with
Equality}
\end{sloppy}

For simplicity we will assume there is one binary predicate $R$;
this system suffices for set theory, where the $R$ is of course the $\in$
predicate.  We label the axioms as they appear in \cite{Megill}.  This
system is logically equivalent to that of Example~2 (when the latter is
restricted to this single binary predicate) but is also metalogically
complete.\index{metalogical completeness}

Let
\begin{itemize}
  \item[] $\mbox{\em CN}=\{\mbox{wff}, \mbox{var}, \vdash, \to, \lnot, (,),\forall,=,R\}$.
  \item[] $\mbox{\em VR}=\{\varphi,\psi,\chi,\ldots\}\cup\{x,y,z,\ldots\}$.
  \item[] $T = \{\langle \mbox{wff\ } \varphi\rangle,
             \langle \mbox{wff\ } \psi\rangle,
             \langle \mbox{wff\ } \chi\rangle,\ldots\}\cup
       \{\langle \mbox{var\ } x\rangle, \langle \mbox{var\ } y\rangle, \langle
       \mbox{var\ }z\rangle,\ldots\}$.

\noindent Then
  $\Gamma$ consists of the reducts of the following pre-statements:
    \begin{itemize}
      \item[] $\langle\varnothing,T,\varnothing,
               \langle \mbox{wff\ }(\varphi\to\psi)\rangle\rangle$
      \item[] $\langle\varnothing,T,\varnothing,
               \langle \mbox{wff\ }\lnot\varphi\rangle\rangle$
      \item[] $\langle\varnothing,T,\varnothing,
               \langle \mbox{wff\ }\forall x\,\varphi\rangle\rangle$
      \item[] $\langle\varnothing,T,\varnothing,
               \langle \mbox{wff\ }x=y\rangle\rangle$
      \item[] $\langle\varnothing,T,\varnothing,
               \langle \mbox{wff\ }Rxy\rangle\rangle$
      \item[(C1$'$)] $\langle\varnothing,T,\varnothing,
               \langle \vdash(\varphi\to(\psi\to\varphi))
               \rangle\rangle$
      \item[(C2$'$)] $\langle\varnothing,T,
               \varnothing,
               \langle \vdash((\varphi\to(\psi\to\chi))\to
               ((\varphi\to\psi)\to(\varphi\to\chi)))
               \rangle\rangle$
      \item[(C3$'$)] $\langle\varnothing,T,
               \varnothing,
               \langle \vdash((\lnot\varphi\to\lnot\psi)\to
               (\psi\to\varphi))\rangle\rangle$
      \item[(C4$'$)] $\langle\varnothing,T,
               \varnothing,
               \langle \vdash(\forall x(\forall x\,\varphi\to\psi)\to
                 (\forall x\,\varphi\to\forall x\,\psi))\rangle\rangle$
      \item[(C5$'$)] $\langle\varnothing,T,
               \varnothing,
               \langle \vdash(\forall x\,\varphi\to\varphi)\rangle\rangle$
      \item[(C6$'$)] $\langle\varnothing,T,
               \varnothing,
               \langle \vdash(\forall x\forall y\,\varphi\to
                 \forall y\forall x\,\varphi)\rangle\rangle$
      \item[(C7$'$)] $\langle\varnothing,T,
               \varnothing,
               \langle \vdash(\lnot\varphi\to\forall x\lnot\forall x\,\varphi
                 )\rangle\rangle$
      \item[(C8$'$)] $\langle\varnothing,T,
               \varnothing,
               \langle \vdash(x=y\to(x=z\to y=z))\rangle\rangle$
      \item[(C9$'$)] $\langle\varnothing,T,
               \varnothing,
               \langle \vdash(\lnot\forall x\, x=y\to(\lnot\forall x\, x=z\to
                 (y=z\to\forall x\, y=z)))\rangle\rangle$
      \item[(C10$'$)] $\langle\varnothing,T,
               \varnothing,
               \langle \vdash(\forall x(x=y\to\forall x\,\varphi)\to
                 \varphi))\rangle\rangle$
      \item[(C11$'$)] $\langle\varnothing,T,
               \varnothing,
               \langle \vdash(\forall x\, x=y\to(\forall x\,\varphi
               \to\forall y\,\varphi))\rangle\rangle$
      \item[(C12$'$)] $\langle\varnothing,T,
               \varnothing,
               \langle \vdash(x=y\to(Rxz\to Ryz))\rangle\rangle$
      \item[(C13$'$)] $\langle\varnothing,T,
               \varnothing,
               \langle \vdash(x=y\to(Rzx\to Rzy))\rangle\rangle$
      \item[(C15$'$)] $\langle\varnothing,T,
               \varnothing,
               \langle \vdash(\lnot\forall x\, x=y\to(x=y\to(\varphi
                 \to\forall x(x=y\to\varphi))))\rangle\rangle$
      \item[(C16$'$)] $\langle\{\{x,y\}\},T,
               \varnothing,
               \langle \vdash(\forall x\, x=y\to(\varphi\to\forall x\,\varphi)
                 )\rangle\rangle$
      \item[(C5)] $\langle\{\{x,\varphi\}\},T,\varnothing,
               \langle \vdash(\varphi\to\forall x\,\varphi)
               \rangle\rangle$
      \item[(MP)] $\langle\varnothing,T,
               \{\langle\vdash(\varphi\to\psi)\rangle,
                 \langle\vdash\varphi\rangle\},
               \langle\vdash\psi\rangle\rangle$
      \item[(Gen)] $\langle\varnothing,T,
               \{\langle\vdash\varphi\rangle\},
               \langle\vdash\forall x\,\varphi\rangle\rangle$
    \end{itemize}
\end{itemize}

While it is known that these axioms are ``metalogically complete,'' it is
not known whether they are independent (i.e.\ none is
redundant) in the metalogical sense; specifically, whether any axiom (possibly
with additional non-mandatory distinct-variable restrictions, for use with any
dummy variables in its proof) is provable from the others.  Note that
metalogical independence is a weaker requirement than independence in the
usual logical sense.  Not all of the above axioms are logically independent:
for example, C9$'$ can be proved as a metatheorem from the others, outside the
formal system, by combining the possible cases of distinct variables.

\subsection{Example~4---Adding Definitions}\index{definition}
There are several ways to add definitions to a formal system.  Probably the
most proper way is to consider definitions not as part of the formal system at
all but rather as abbreviations that are part of the expository metalogic
outside the formal system.  For convenience, though, we may use the formal
system itself to incorporate definitions, adding them as axiomatic extensions
to the system.  This could be done by adding a constant representing the
concept ``is defined as'' along with axioms for it. But there is a nicer way,
at least in this writer's opinion, that introduces definitions as direct
extensions to the language rather than as extralogical primitive notions.  We
introduce additional logical connectives and provide axioms for them.  For
systems of logic such as Examples 1 through 3, the additional axioms must be
conservative in the sense that no wff of the original system that was not a
theorem (when the initial term ``wff'' is replaced by ``$\vdash$'' of course)
becomes a theorem of the extended system.  In this example we extend Example~3
(or 2) with standard abbreviations of logic.

We extend $\mbox{\em CN}$ of Example~3 with new constants $\{\leftrightarrow,
\wedge,\vee,\exists\}$, corresponding to logical equivalence,\index{logical
equivalence ($\leftrightarrow$)}\index{biconditional ($\leftrightarrow$)}
conjunction,\index{conjunction ($\wedge$)} disjunction,\index{disjunction
($\vee$)} and the existential quantifier.\index{existential quantifier
($\exists$)}  We extend $\Gamma$ with the axiomatic statements that are
the reducts of the following pre-statements:
\begin{list}{}{\itemsep 0.0pt}
      \item[] $\langle\varnothing,T,\varnothing,
               \langle \mbox{wff\ }(\varphi\leftrightarrow\psi)\rangle\rangle$
      \item[] $\langle\varnothing,T,\varnothing,
               \langle \mbox{wff\ }(\varphi\vee\psi)\rangle\rangle$
      \item[] $\langle\varnothing,T,\varnothing,
               \langle \mbox{wff\ }(\varphi\wedge\psi)\rangle\rangle$
      \item[] $\langle\varnothing,T,\varnothing,
               \langle \mbox{wff\ }\exists x\, \varphi\rangle\rangle$
  \item[] $\langle\varnothing,T,\varnothing,
     \langle\vdash ( ( \varphi \leftrightarrow \psi ) \to
     ( \varphi \to \psi ) )\rangle\rangle$
  \item[] $\langle\varnothing,T,\varnothing,
     \langle\vdash ((\varphi\leftrightarrow\psi)\to
    (\psi\to\varphi))\rangle\rangle$
  \item[] $\langle\varnothing,T,\varnothing,
     \langle\vdash ((\varphi\to\psi)\to(
     (\psi\to\varphi)\to(\varphi
     \leftrightarrow\psi)))\rangle\rangle$
  \item[] $\langle\varnothing,T,\varnothing,
     \langle\vdash (( \varphi \wedge \psi ) \leftrightarrow\neg ( \varphi
     \to \neg \psi )) \rangle\rangle$
  \item[] $\langle\varnothing,T,\varnothing,
     \langle\vdash (( \varphi \vee \psi ) \leftrightarrow (\neg \varphi
     \to \psi )) \rangle\rangle$
  \item[] $\langle\varnothing,T,\varnothing,
     \langle\vdash (\exists x \,\varphi\leftrightarrow
     \lnot \forall x \lnot \varphi)\rangle\rangle$
\end{list}
The first three logical axioms (statements containing ``$\vdash$'') introduce
and effectively define logical equivalence, ``$\leftrightarrow$''.  The last
three use ``$\leftrightarrow$'' to effectively mean ``is defined as.''

\subsection{Example~5---ZFC Set Theory}\index{ZFC set theory}

Here we add to the system of Example~4 the axioms of Zermelo--Fraenkel set
theory with Choice.  For convenience we make use of the
definitions in Example~4.

In the $\mbox{\em CN}$ of Example~4 (which extends Example~3), we replace the symbol $R$
with the symbol $\in$.
More explicitly, we remove from $\Gamma$ of Example~4 the three
axiomatic statements containing $R$ and replace them with the
reducts of the following:
\begin{list}{}{\itemsep 0.0pt}
      \item[] $\langle\varnothing,T,\varnothing,
               \langle \mbox{wff\ }x\in y\rangle\rangle$
      \item[] $\langle\varnothing,T,
               \varnothing,
               \langle \vdash(x=y\to(x\in z\to y\in z))\rangle\rangle$
      \item[] $\langle\varnothing,T,
               \varnothing,
               \langle \vdash(x=y\to(z\in x\to z\in y))\rangle\rangle$
\end{list}
Letting $D=\{\{\alpha,\beta\}\in \mbox{\em DV}\,|\alpha,\beta\in \mbox{\em
Vr}\}$ (in other words all individual variables must be distinct), we extend
$\Gamma$ with the ZFC axioms, called
\index{Axiom of Extensionality}
\index{Axiom of Replacement}
\index{Axiom of Union}
\index{Axiom of Power Sets}
\index{Axiom of Regularity}
\index{Axiom of Infinity}
\index{Axiom of Choice}
Extensionality, Replacement, Union, Power
Set, Regularity, Infinity, and Choice, that are the reducts of:
\begin{list}{}{\itemsep 0.0pt}
      \item[Ext] $\langle D,T,
               \varnothing,
               \langle\vdash (\forall x(x\in y\leftrightarrow x \in z)\to y
               =z) \rangle\rangle$
      \item[Rep] $\langle D,T,
               \varnothing,
               \langle\vdash\exists x ( \exists y \forall z (\varphi \to z = y
                        ) \to
                        \forall z ( z \in x \leftrightarrow \exists x ( x \in
                        y \wedge \forall y\,\varphi ) ) )\rangle\rangle$
      \item[Un] $\langle D,T,
               \varnothing,
               \langle\vdash \exists x \forall y ( \exists x ( y \in x \wedge
               x \in z ) \to y \in x ) \rangle\rangle$
      \item[Pow] $\langle D,T,
               \varnothing,
               \langle\vdash \exists x \forall y ( \forall x ( x \in y \to x
               \in z ) \to y \in x ) \rangle\rangle$
      \item[Reg] $\langle D,T,
               \varnothing,
               \langle\vdash (  x \in y \to
                 \exists x ( x \in y \wedge \forall z ( z \in x \to \lnot z
                \in y ) ) ) \rangle\rangle$
      \item[Inf] $\langle D,T,
               \varnothing,
               \langle\vdash \exists x(y\in x\wedge\forall y(y\in
               x\to
               \exists z(y \in z\wedge z\in x))) \rangle\rangle$
      \item[AC] $\langle D,T,
               \varnothing,
               \langle\vdash \exists x \forall y \forall z ( ( y \in z
               \wedge z \in w ) \to \exists w \forall y ( \exists w
              ( ( y \in z \wedge z \in w ) \wedge ( y \in w \wedge w \in x
              ) ) \leftrightarrow y = w ) ) \rangle\rangle$
\end{list}

\subsection{Example~6---Class Notation in Set Theory}\label{class}

A powerful device that makes set theory easier (and that we have
been using all along in our informal expository language) is {\em class
abstraction notation}.\index{class abstraction}\index{abstraction class}  The
definitions we introduce are rigorously justified
as conservative by Takeuti and Zaring \cite{Takeuti}\index{Takeuti, Gaisi} or
Quine \cite{Quine}\index{Quine, Willard Van Orman}.  The key idea is to
introduce the notation $\{x|\mbox{---}\}$ which means ``the class of all $x$
such that ---'' for abstraction classes and introduce (meta)variables that
range over them.  An abstraction class may or may not be a set, depending on
whether it exists (as a set).  A class that does not exist is
called a {\em proper class}.\index{proper class}\index{class!proper}

To illustrate the use of abstraction classes we will provide some examples
of definitions that make use of them:  the empty set, class union, and
unordered pair.  Many other such definitions can be found in the
Metamath set theory database,
\texttt{set.mm}.\index{set theory database (\texttt{set.mm})}

% We intentionally break up the sequence of math symbols here
% because otherwise the overlong line goes beyond the page in narrow mode.
We extend $\mbox{\em CN}$ of Example~5 with new symbols $\{$
$\mbox{class},$ $\{,$ $|,$ $\},$ $\varnothing,$ $\cup,$ $,$ $\}$
where the inner braces and last comma are
constant symbols. (As before,
our dual use of some mathematical symbols for both our expository
language and as primitives of the formal system should be clear from context.)

We extend $\mbox{\em VR}$ of Example~5 with a set of {\em class
variables}\index{class variable}
$\{A,B,C,\ldots\}$. We extend the $T$ of Example~5 with $\{\langle
\mbox{class\ } A\rangle, \langle \mbox{class\ }B\rangle, \langle \mbox{class\ }
C\rangle,\ldots\}$.

To
introduce our definitions,
we add to $\Gamma$ of Example~5 the axiomatic statements
that are the reducts of the following pre-statements:
\begin{list}{}{\itemsep 0.0pt}
      \item[] $\langle\varnothing,T,\varnothing,
               \langle \mbox{class\ }x\rangle\rangle$
      \item[] $\langle\varnothing,T,\varnothing,
               \langle \mbox{class\ }\{x|\varphi\}\rangle\rangle$
      \item[] $\langle\varnothing,T,\varnothing,
               \langle \mbox{wff\ }A=B\rangle\rangle$
      \item[] $\langle\varnothing,T,\varnothing,
               \langle \mbox{wff\ }A\in B\rangle\rangle$
      \item[Ab] $\langle\varnothing,T,\varnothing,
               \langle \vdash ( y \in \{ x |\varphi\} \leftrightarrow
                  ( ( x = y \to\varphi) \wedge \exists x ( x = y
                  \wedge\varphi) ))
               \rangle\rangle$
      \item[Eq] $\langle\{\{x,A\},\{x,B\}\},T,\varnothing,
               \langle \vdash ( A = B \leftrightarrow
               \forall x ( x \in A \leftrightarrow x \in B ) )
               \rangle\rangle$
      \item[El] $\langle\{\{x,A\},\{x,B\}\},T,\varnothing,
               \langle \vdash ( A \in B \leftrightarrow \exists x
               ( x = A \wedge x \in B ) )
               \rangle\rangle$
\end{list}
Here we say that an individual variable is a class; $\{x|\varphi\}$ is a
class; and we extend the definition of a wff to include class equality and
membership.  Axiom Ab defines membership of a variable in a class abstraction;
the right-hand side can be read as ``the wff that results from proper
substitution of $y$ for $x$ in $\varphi$.''\footnote{Note that this definition
makes unnecessary the introduction of a separate notation similar to
$\varphi(x|y)$ for proper substitution, although we may choose to do so to be
conventional.  Incidentally, $\varphi(x|y)$ as it stands would be ambiguous in
the formal systems of our examples, since we wouldn't know whether
$\lnot\varphi(x|y)$ meant $\lnot(\varphi(x|y))$ or $(\lnot\varphi)(x|y)$.
Instead, we would have to use an unambiguous variant such as $(\varphi\,
x|y)$.}  Axioms Eq and El extend the meaning of the existing equality and
membership connectives.  This is potentially dangerous and requires careful
justification.  For example, from Eq we can derive the Axiom of Extensionality
with predicate logic alone; thus in principle we should include the Axiom of
Extensionality as a logical hypothesis.  However we do not bother to do this
since we have already presupposed that axiom earlier. The distinct variable
restrictions should be read ``where $x$ does not occur in $A$ or $B$.''  We
typically do this when the right-hand side of a definition involves an
individual variable not in the expression being defined; it is done so that
the right-hand side remains independent of the particular ``dummy'' variable
we use.

We continue to add to $\Gamma$ the following definitions
(i.e. the reducts of the following pre-statements) for empty
set,\index{empty set} class union,\index{union} and unordered
pair.\index{unordered pair}  They should be self-explanatory.  Analogous to our
use of ``$\leftrightarrow$'' to define new wffs in Example~4, we use ``$=$''
to define new abstraction terms, and both may be read informally as ``is
defined as'' in this context.
\begin{list}{}{\itemsep 0.0pt}
      \item[] $\langle\varnothing,T,\varnothing,
               \langle \mbox{class\ }\varnothing\rangle\rangle$
      \item[] $\langle\varnothing,T,\varnothing,
               \langle \vdash \varnothing = \{ x | \lnot x = x \}
               \rangle\rangle$
      \item[] $\langle\varnothing,T,\varnothing,
               \langle \mbox{class\ }(A\cup B)\rangle\rangle$
      \item[] $\langle\{\{x,A\},\{x,B\}\},T,\varnothing,
               \langle \vdash ( A \cup B ) = \{ x | ( x \in A \vee x \in B ) \}
               \rangle\rangle$
      \item[] $\langle\varnothing,T,\varnothing,
               \langle \mbox{class\ }\{A,B\}\rangle\rangle$
      \item[] $\langle\{\{x,A\},\{x,B\}\},T,\varnothing,
               \langle \vdash \{ A , B \} = \{ x | ( x = A \vee x = B ) \}
               \rangle\rangle$
\end{list}

\section{Metamath as a Formal System}\label{theorymm}

This section presupposes a familiarity with the Metamath computer language.

Our theory describes formal systems and their universes.  The Metamath
language provides a way of representing these set-theoretical objects to
a computer.  A Metamath database, being a finite set of {\sc ascii}
characters, can usually describe only a subset of a formal system and
its universe, which are typically infinite.  However the database can
contain as large a finite subset of the formal system and its universe
as we wish.  (Of course a Metamath set theory database can, in
principle, indirectly describe an entire infinite formal system by
formalizing the expository language in this Appendix.)

For purpose of our discussion, we assume the Metamath database
is in the simple form described on p.~\pageref{framelist},
consisting of all constant and variable declarations at the beginning,
followed by a sequence of extended frames each
delimited by \texttt{\$\char`\{} and \texttt{\$\char`\}}.  Any Metamath database can
be converted to this form, as described on p.~\pageref{frameconvert}.

The math symbol tokens of a Metamath source file, which are declared
with \texttt{\$c} and \texttt{\$v} statements, are names we assign to
representatives of $\mbox{\em CN}$ and $\mbox{\em VR}$.  For
definiteness we could assume that the first math symbol declared as a
variable corresponds to $v_0$, the second to $v_1$, etc., although the
exact correspondence we choose is not important.

In the Metamath language, each \texttt{\$d}, \texttt{\$f}, and
 \texttt{\$e} source
statement in an extended frame (Section~\ref{frames})
corresponds respectively to a member of the
collections $D$, $T$, and $H$ in a formal system statement $\langle
D_M,T_M,H,A\rangle$.  The math symbol strings following these Metamath keywords
correspond to a variable pair (in the case of \texttt{\$d}) or an expression (for
the other two keywords). The math symbol string following a \texttt{\$a} source
statement corresponds to expression $A$ in an axiomatic statement of the
formal system; the one following a \texttt{\$p} source statement corresponds to
$A$ in a provable statement that is not axiomatic.  In other words, each
extended frame in a Metamath database corresponds to
a pre-statement of the formal system, and a frame corresponds to
a statement of the formal system.  (Don't confuse the two meanings of
``statement'' here.  A statement of the formal system corresponds to the
several statements in a Metamath database that may constitute a
frame.)

In order for the computer to verify that a formal system statement is
provable, each \texttt{\$p} source statement is accompanied by a proof.
However, the proof does not correspond to anything in the formal system
but is simply a way of communicating to the computer the information
needed for its verification.  The proof tells the computer {\em how to
construct} specific members of closure of the formal system
pre-statement corresponding to the extended frame of the \texttt{\$p}
statement.  The final result of the construction is the member of the
closure that matches the \texttt{\$p} statement.  The abstract formal
system, on the other hand, is concerned only with the {\em existence} of
members of the closure.

As mentioned on p.~\pageref{exampleref}, Examples 1 and 3--6 in the
previous Section parallel the development of logic and set theory in the
Metamath database
\texttt{set.mm}.\index{set theory database (\texttt{set.mm})} You may
find it instructive to compare them.


\chapter{The MIU System}
\label{MIU}
\index{formal system}
\index{MIU-system}

The following is a listing of the file \texttt{miu.mm}.  It is self-explanatory.

%%%%%%%%%%%%%%%%%%%%%%%%%%%%%%%%%%%%%%%%%%%%%%%%%%%%%%%%%%%%

\begin{verbatim}
$( The MIU-system:  A simple formal system $)

$( Note:  This formal system is unusual in that it allows
empty wffs.  To work with a proof, you must type
SET EMPTY_SUBSTITUTION ON before using the PROVE command.
By default, this is OFF in order to reduce the number of
ambiguous unification possibilities that have to be selected
during the construction of a proof.  $)

$(
Hofstadter's MIU-system is a simple example of a formal
system that illustrates some concepts of Metamath.  See
Douglas R. Hofstadter, _Goedel, Escher, Bach:  An Eternal
Golden Braid_ (Vintage Books, New York, 1979), pp. 33ff. for
a description of the MIU-system.

The system has 3 constant symbols, M, I, and U.  The sole
axiom of the system is MI. There are 4 rules:
     Rule I:  If you possess a string whose last letter is I,
     you can add on a U at the end.
     Rule II:  Suppose you have Mx.  Then you may add Mxx to
     your collection.
     Rule III:  If III occurs in one of the strings in your
     collection, you may make a new string with U in place
     of III.
     Rule IV:  If UU occurs inside one of your strings, you
     can drop it.
Unfortunately, Rules III and IV do not have unique results:
strings could have more than one occurrence of III or UU.
This requires that we introduce the concept of an "MIU
well-formed formula" or wff, which allows us to construct
unique symbol sequences to which Rules III and IV can be
applied.
$)

$( First, we declare the constant symbols of the language.
Note that we need two symbols to distinguish the assertion
that a sequence is a wff from the assertion that it is a
theorem; we have arbitrarily chosen "wff" and "|-". $)
      $c M I U |- wff $. $( Declare constants $)

$( Next, we declare some variables. $)
     $v x y $.

$( Throughout our theory, we shall assume that these
variables represent wffs. $)
 wx   $f wff x $.
 wy   $f wff y $.

$( Define MIU-wffs.  We allow the empty sequence to be a
wff. $)

$( The empty sequence is a wff. $)
 we   $a wff $.
$( "M" after any wff is a wff. $)
 wM   $a wff x M $.
$( "I" after any wff is a wff. $)
 wI   $a wff x I $.
$( "U" after any wff is a wff. $)
 wU   $a wff x U $.

$( Assert the axiom. $)
 ax   $a |- M I $.

$( Assert the rules. $)
 ${
   Ia   $e |- x I $.
$( Given any theorem ending with "I", it remains a theorem
if "U" is added after it.  (We distinguish the label I_
from the math symbol I to conform to the 24-Jun-2006
Metamath spec.) $)
   I_    $a |- x I U $.
 $}
 ${
IIa  $e |- M x $.
$( Given any theorem starting with "M", it remains a theorem
if the part after the "M" is added again after it. $)
   II   $a |- M x x $.
 $}
 ${
   IIIa $e |- x I I I y $.
$( Given any theorem with "III" in the middle, it remains a
theorem if the "III" is replaced with "U". $)
   III  $a |- x U y $.
 $}
 ${
   IVa  $e |- x U U y $.
$( Given any theorem with "UU" in the middle, it remains a
theorem if the "UU" is deleted. $)
   IV   $a |- x y $.
  $}

$( Now we prove the theorem MUIIU.  You may be interested in
comparing this proof with that of Hofstadter (pp. 35 - 36).
$)
 theorem1  $p |- M U I I U $=
      we wM wU wI we wI wU we wU wI wU we wM we wI wU we wM
      wI wI wI we wI wI we wI ax II II I_ III II IV $.
\end{verbatim}\index{well-formed formula (wff)}

The \texttt{show proof /lemmon/renumber} command
yields the following display.  It is very similar
to the one in \cite[pp.~35--36]{Hofstadter}.\index{Hofstadter, Douglas R.}

\begin{verbatim}
1 ax             $a |- M I
2 1 II           $a |- M I I
3 2 II           $a |- M I I I I
4 3 I_           $a |- M I I I I U
5 4 III          $a |- M U I U
6 5 II           $a |- M U I U U I U
7 6 IV           $a |- M U I I U
\end{verbatim}

We note that Hofstadter's ``MU-puzzle,'' which asks whether
MU is a theorem of the MIU-system, cannot be answered using
the system above because the MU-puzzle is a question {\em
about} the system.  To prove the answer to the MU-puzzle,
a much more elaborate system is needed, namely one that
models the MIU-system within set theory.  (Incidentally, the
answer to the MU-puzzle is no.)

\chapter{Metamath Language EBNF}%
\label{BNF}%
\index{Metamath Language EBNF}

The following is a formal description of the basic Metamath language syntax
(with compressed proofs and support for unknown proof steps).
It is defined using the
Extended Backus--Naur Form (EBNF)\index{Extended Backus--Naur Form}\index{EBNF}
notation from W3C\index{W3C}
\textit{Extensible Markup Language (XML) 1.0 (Fifth Edition)}
(W3C Recommendation 26 November 2008) at
\url{https://www.w3.org/TR/xml/#sec-notation}.

The \texttt{database}
rule is processed until the end of the file (\texttt{EOF}).
The rules eventually require reading whitespace-separated tokens.
A token has an upper-case definition (see below)
or is a string constant in a non-token (such as \texttt{'\$a'}).
We intend for this to be correct, but if there is a conflict the
rules of section \ref{spec} govern. That section also discusses
non-syntax restrictions not shown here
(e.g., that each new label token
defined in a \texttt{hypothesis-stmt} or \texttt{assert-stmt}
must be unique).

\begin{verbatim}
database ::= outermost-scope-stmt*

outermost-scope-stmt ::=
  include-stmt | constant-stmt | stmt

/* File inclusion command; process file as a database.
   Databases should NOT have a comment in the filename. */
include-stmt ::= '$[' filename '$]'

/* Constant symbols declaration. */
constant-stmt ::= '$c' constant+ '$.'

/* A normal statement can occur in any scope. */
stmt ::= block | variable-stmt | disjoint-stmt |
  hypothesis-stmt | assert-stmt

/* A block. You can have 0 statements in a block. */
block ::= '${' stmt* '$}'

/* Variable symbols declaration. */
variable-stmt ::= '$v' variable+ '$.'

/* Disjoint variables. Simple disjoint statements have
   2 variables, i.e., "variable*" is empty for them. */
disjoint-stmt ::= '$d' variable variable variable* '$.'

hypothesis-stmt ::= floating-stmt | essential-stmt

/* Floating (variable-type) hypothesis. */
floating-stmt ::= LABEL '$f' typecode variable '$.'

/* Essential (logical) hypothesis. */
essential-stmt ::= LABEL '$e' typecode MATH-SYMBOL* '$.'

assert-stmt ::= axiom-stmt | provable-stmt

/* Axiomatic assertion. */
axiom-stmt ::= LABEL '$a' typecode MATH-SYMBOL* '$.'

/* Provable assertion. */
provable-stmt ::= LABEL '$p' typecode MATH-SYMBOL*
  '$=' proof '$.'

/* A proof. Proofs may be interspersed by comments.
   If '?' is in a proof it's an "incomplete" proof. */
proof ::= uncompressed-proof | compressed-proof
uncompressed-proof ::= (LABEL | '?')+
compressed-proof ::= '(' LABEL* ')' COMPRESSED-PROOF-BLOCK+

typecode ::= constant

filename ::= MATH-SYMBOL /* No whitespace or '$' */
constant ::= MATH-SYMBOL
variable ::= MATH-SYMBOL
\end{verbatim}

\needspace{2\baselineskip}
A \texttt{frame} is a sequence of 0 or more
\texttt{disjoint-{\allowbreak}stmt} and
\texttt{hypotheses-{\allowbreak}stmt} statements
(possibly interleaved with other non-\texttt{assert-stmt} statements)
followed by one \texttt{assert-stmt}.

\needspace{3\baselineskip}
Here are the rules for lexical processing (tokenization) beyond
the constant tokens shown above.
By convention these tokenization rules have upper-case names.
Every token is read for the longest possible length.
Whitespace-separated tokens are read sequentially;
note that the separating whitespace and \texttt{\$(} ... \texttt{\$)}
comments are skipped.

If a token definition uses another token definition, the whole thing
is considered a single token.
A pattern that is only part of a full token has a name beginning
with an underscore (``\_'').
An implementation could tokenize many tokens as a
\texttt{PRINTABLE-SEQUENCE}
and then check if it meets the more specific rule shown here.

Comments do not nest, and both \texttt{\$(} and \texttt{\$)}
have to be surrounded
by at least one whitespace character (\texttt{\_WHITECHAR}).
Technically comments end without consuming the trailing
\texttt{\_WHITECHAR}, but the trailing
\texttt{\_WHITECHAR} gets ignored anyway so we ignore that detail here.
Metamath language processors
are not required to support \texttt{\$)} followed
immediately by a bare end-of-file, because the closing
comment symbol is supposed to be followed by a
\texttt{\_WHITECHAR} such as a newline.

\begin{verbatim}
PRINTABLE-SEQUENCE ::= _PRINTABLE-CHARACTER+

MATH-SYMBOL ::= (_PRINTABLE-CHARACTER - '$')+

/* ASCII non-whitespace printable characters */
_PRINTABLE-CHARACTER ::= [#x21-#x7e]

LABEL ::= ( _LETTER-OR-DIGIT | '.' | '-' | '_' )+

_LETTER-OR-DIGIT ::= [A-Za-z0-9]

COMPRESSED-PROOF-BLOCK ::= ([A-Z] | '?')+

/* Define whitespace between tokens. The -> SKIP
   means that when whitespace is seen, it is
   skipped and we simply read again. */
WHITESPACE ::= (_WHITECHAR+ | _COMMENT) -> SKIP

/* Comments. $( ... $) and do not nest. */
_COMMENT ::= '$(' (_WHITECHAR+ (PRINTABLE-SEQUENCE - '$)')*
  _WHITECHAR+ '$)' _WHITECHAR

/* Whitespace: (' ' | '\t' | '\r' | '\n' | '\f') */
_WHITECHAR ::= [#x20#x09#x0d#x0a#x0c]
\end{verbatim}
% This EBNF was developed as a collaboration between
% David A. Wheeler\index{Wheeler, David A.},
% Mario Carneiro\index{Carneiro, Mario}, and
% Benoit Jubin\index{Jubin, Benoit}, inspired by a request
% (and a lot of initial work) by Benoit Jubin.
%
% \chapter{Disclaimer and Trademarks}
%
% Information in this document is subject to change without notice and does not
% represent a commitment on the part of Norman Megill.
% \vspace{2ex}
%
% \noindent Norman D. Megill makes no warranties, either express or implied,
% regarding the Metamath computer software package.
%
% \vspace{2ex}
%
% \noindent Any trademarks mentioned in this book are the property of
% their respective owners.  The name ``Metamath'' is a trademark of
% Norman Megill.
%
\cleardoublepage
\phantomsection  % fixes the link anchor
\addcontentsline{toc}{chapter}{\bibname}

\bibliography{metamath}
%% metamath.tex - Version of 24-May-2019
% If you change the date above, also change the "Printed date" below.
% SPDX-License-Identifier: CC0-1.0
%
%                              PUBLIC DOMAIN
%
% This file (specifically, the version of this file with the above date)
% has been released into the Public Domain per the
% Creative Commons CC0 1.0 Universal (CC0 1.0) Public Domain Dedication
% https://creativecommons.org/publicdomain/zero/1.0/
%
% The public domain release applies worldwide.  In case this is not
% legally possible, the right is granted to use the work for any purpose,
% without any conditions, unless such conditions are required by law.
%
% Several short, attributed quotations from copyrighted works
% appear in this file under the ``fair use'' provision of Section 107 of
% the United States Copyright Act (Title 17 of the {\em United States
% Code}).  The public-domain status of this file is not applicable to
% those quotations.
%
% Norman Megill - email: nm(at)alum(dot)mit(dot)edu
%
% David A. Wheeler also donates his improvements to this file to the
% public domain per the CC0.  He works at the Institute for Defense Analyses
% (IDA), but IDA has agreed that this Metamath work is outside its "lane"
% and is not a work by IDA.  This was specifically confirmed by
% Margaret E. Myers (Division Director of the Information Technology
% and Systems Division) on 2019-05-24 and by Ben Lindorf (General Counsel)
% on 2019-05-22.

% This file, 'metamath.tex', is self-contained with everything needed to
% generate the the PDF file 'metamath.pdf' (the _Metamath_ book) on
% standard LaTeX 2e installations.  The auxiliary files are embedded with
% "filecontents" commands.  To generate metamath.pdf file, run these
% commands under Linux or Cygwin in the directory that contains
% 'metamath.tex':
%
%   rm -f realref.sty metamath.bib
%   touch metamath.ind
%   pdflatex metamath
%   pdflatex metamath
%   bibtex metamath
%   makeindex metamath
%   pdflatex metamath
%   pdflatex metamath
%
% The warnings that occur in the initial runs of pdflatex can be ignored.
% For the final run,
%
%   egrep -i 'error|warn' metamath.log
%
% should show exactly these 5 warnings:
%
%   LaTeX Warning: File `realref.sty' already exists on the system.
%   LaTeX Warning: File `metamath.bib' already exists on the system.
%   LaTeX Font Warning: Font shape `OMS/cmtt/m/n' undefined
%   LaTeX Font Warning: Font shape `OMS/cmtt/bx/n' undefined
%   LaTeX Font Warning: Some font shapes were not available, defaults
%       substituted.
%
% Search for "Uncomment" below if you want to suppress hyperlink boxes
% in the PDF output file
%
% TYPOGRAPHICAL NOTES:
% * It is customary to use an en dash (--) to "connect" names of different
%   people (and to denote ranges), and use a hyphen (-) for a
%   single compound name. Examples of connected multiple people are
%   Zermelo--Fraenkel, Schr\"{o}der--Bernstein, Tarski--Grothendieck,
%   Hewlett--Packard, and Backus--Naur.  Examples of a single person with
%   a compound name include Levi-Civita, Mittag-Leffler, and Burali-Forti.
% * Use non-breaking spaces after page abbreviations, e.g.,
%   p.~\pageref{note2002}.
%
% --------------------------- Start of realref.sty -----------------------------
\begin{filecontents}{realref.sty}
% Save the following as realref.sty.
% You can then use it with \usepackage{realref}
%
% This has \pageref jumping to the page on which the ref appears,
% \ref jumping to the point of the anchor, and \sectionref
% jumping to the start of section.
%
% Author:  Anthony Williams
%          Software Engineer
%          Nortel Networks Optical Components Ltd
% Date:    9 Nov 2001 (posted to comp.text.tex)
%
% The following declaration was made by Anthony Williams on
% 24 Jul 2006 (private email to Norman Megill):
%
%   ``I hereby donate the code for realref.sty posted on the
%   comp.text.tex newsgroup on 9th November 2001, accessible from
%   http://groups.google.com/group/comp.text.tex/msg/5a0e1cc13ea7fbb2
%   to the public domain.''
%
\ProvidesPackage{realref}
\RequirePackage[plainpages=false,pdfpagelabels=true]{hyperref}
\def\realref@anchorname{}
\AtBeginDocument{%
% ensure every label is a possible hyperlink target
\let\realref@oldrefstepcounter\refstepcounter%
\DeclareRobustCommand{\refstepcounter}[1]{\realref@oldrefstepcounter{#1}
\edef\realref@anchorname{\string #1.\@currentlabel}%
}%
\let\realref@oldlabel\label%
\DeclareRobustCommand{\label}[1]{\realref@oldlabel{#1}\hypertarget{#1}{}%
\@bsphack\protected@write\@auxout{}{%
    \string\expandafter\gdef\protect\csname
    page@num.#1\string\endcsname{\thepage}%
    \string\expandafter\gdef\protect\csname
    ref@num.#1\string\endcsname{\@currentlabel}%
    \string\expandafter\gdef\protect\csname
    sectionref@name.#1\string\endcsname{\realref@anchorname}%
}\@esphack}%
\DeclareRobustCommand\pageref[1]{{\edef\a{\csname
            page@num.#1\endcsname}\expandafter\hyperlink{page.\a}{\a}}}%
\DeclareRobustCommand\ref[1]{{\edef\a{\csname
            ref@num.#1\endcsname}\hyperlink{#1}{\a}}}%
\DeclareRobustCommand\sectionref[1]{{\edef\a{\csname
            ref@num.#1\endcsname}\edef\b{\csname
            sectionref@name.#1\endcsname}\hyperlink{\b}{\a}}}%
}
\end{filecontents}
% ---------------------------- End of realref.sty ------------------------------

% --------------------------- Start of metamath.bib -----------------------------
\begin{filecontents}{metamath.bib}
@book{Albers, editor = "Donald J. Albers and G. L. Alexanderson",
  title = "Mathematical People",
  publisher = "Contemporary Books, Inc.",
  address = "Chicago",
  note = "[QA28.M37]",
  year = 1985 }
@book{Anderson, author = "Alan Ross Anderson and Nuel D. Belnap",
  title = "Entailment",
  publisher = "Princeton University Press",
  address = "Princeton",
  volume = 1,
  note = "[QA9.A634 1975 v.1]",
  year = 1975}
@book{Barrow, author = "John D. Barrow",
  title = "Theories of Everything:  The Quest for Ultimate Explanation",
  publisher = "Oxford University Press",
  address = "Oxford",
  note = "[Q175.B225]",
  year = 1991 }
@book{Behnke,
  editor = "H. Behnke and F. Backmann and K. Fladt and W. S{\"{u}}ss",
  title = "Fundamentals of Mathematics",
  volume = "I",
  publisher = "The MIT Press",
  address = "Cambridge, Massachusetts",
  note = "[QA37.2.B413]",
  year = 1974 }
@book{Bell, author = "J. L. Bell and M. Machover",
  title = "A Course in Mathematical Logic",
  publisher = "North-Holland",
  address = "Amsterdam",
  note = "[QA9.B3953]",
  year = 1977 }
@inproceedings{Blass, author = "Andrea Blass",
  title = "The Interaction Between Category Theory and Set Theory",
  pages = "5--29",
  booktitle = "Mathematical Applications of Category Theory (Proceedings
     of the Special Session on Mathematical Applications
     Category Theory, 89th Annual Meeting of the American Mathematical
     Society, held in Denver, Colorado January 5--9, 1983)",
  editor = "John Walter Gray",
  year = 1983,
  note = "[QA169.A47 1983]",
  publisher = "American Mathematical Society",
  address = "Providence, Rhode Island"}
@proceedings{Bledsoe, editor = "W. W. Bledsoe and D. W. Loveland",
  title = "Automated Theorem Proving:  After 25 Years (Proceedings
     of the Special Session on Automatic Theorem Proving,
     89th Annual Meeting of the American Mathematical
     Society, held in Denver, Colorado January 5--9, 1983)",
  year = 1983,
  note = "[QA76.9.A96.S64 1983]",
  publisher = "American Mathematical Society",
  address = "Providence, Rhode Island" }
@book{Boolos, author = "George S. Boolos and Richard C. Jeffrey",
  title = "Computability and Log\-ic",
  publisher = "Cambridge University Press",
  edition = "third",
  address = "Cambridge",
  note = "[QA9.59.B66 1989]",
  year = 1989 }
@book{Campbell, author = "John Campbell",
  title = "Programmer's Progress",
  publisher = "White Star Software",
  address = "Box 51623, Palo Alto, CA 94303",
  year = 1991 }
@article{DBLP:journals/corr/Carneiro14,
  author    = {Mario Carneiro},
  title     = {Conversion of {HOL} Light proofs into Metamath},
  journal   = {CoRR},
  volume    = {abs/1412.8091},
  year      = {2014},
  url       = {http://arxiv.org/abs/1412.8091},
  archivePrefix = {arXiv},
  eprint    = {1412.8091},
  timestamp = {Mon, 13 Aug 2018 16:47:05 +0200},
  biburl    = {https://dblp.org/rec/bib/journals/corr/Carneiro14},
  bibsource = {dblp computer science bibliography, https://dblp.org}
}
@article{CarneiroND,
  author    = {Mario Carneiro},
  title     = {Natural Deductions in the Metamath Proof Language},
  url       = {http://us.metamath.org/ocat/natded.pdf},
  year      = 2014
}
@inproceedings{Chou, author = "Shang-Ching Chou",
  title = "Proving Elementary Geometry Theorems Using {W}u's Algorithm",
  pages = "243--286",
  booktitle = "Automated Theorem Proving:  After 25 Years (Proceedings
     of the Special Session on Automatic Theorem Proving,
     89th Annual Meeting of the American Mathematical
     Society, held in Denver, Colorado January 5--9, 1983)",
  editor = "W. W. Bledsoe and D. W. Loveland",
  year = 1983,
  note = "[QA76.9.A96.S64 1983]",
  publisher = "American Mathematical Society",
  address = "Providence, Rhode Island" }
@book{Clemente, author = "Daniel Clemente Laboreo",
  title = "Introduction to natural deduction",
  year = 2014,
  url = "http://www.danielclemente.com/logica/dn.en.pdf" }
@incollection{Courant, author = "Richard Courant and Herbert Robbins",
  title = "Topology",
  pages = "573--590",
  booktitle = "The World of Mathematics, Volume One",
  editor = "James R. Newman",
  publisher = "Simon and Schuster",
  address = "New York",
  note = "[QA3.W67 1988]",
  year = 1956 }
@book{Curry, author = "Haskell B. Curry",
  title = "Foundations of Mathematical Logic",
  publisher = "Dover Publications, Inc.",
  address = "New York",
  note = "[QA9.C976 1977]",
  year = 1977 }
@book{Davis, author = "Philip J. Davis and Reuben Hersh",
  title = "The Mathematical Experience",
  publisher = "Birkh{\"{a}}user Boston",
  address = "Boston",
  note = "[QA8.4.D37 1982]",
  year = 1981 }
@incollection{deMillo,
  author = "Richard de Millo and Richard Lipton and Alan Perlis",
  title = "Social Processes and Proofs of Theorems and Programs",
  pages = "267--285",
  booktitle = "New Directions in the Philosophy of Mathematics",
  editor = "Thomas Tymoczko",
  publisher = "Birkh{\"{a}}user Boston, Inc.",
  address = "Boston",
  note = "[QA8.6.N48 1986]",
  year = 1986 }
@book{Edwards, author = "Robert E. Edwards",
  title = "A Formal Background to Mathematics",
  publisher = "Springer-Verlag",
  address = "New York",
  note = "[QA37.2.E38 v.1a]",
  year = 1979 }
@book{Enderton, author = "Herbert B. Enderton",
  title = "Elements of Set Theory",
  publisher = "Academic Press, Inc.",
  address = "San Diego",
  note = "[QA248.E5]",
  year = 1977 }
@book{Goodstein, author = "R. L. Goodstein",
  title = "Development of Mathematical Logic",
  publisher = "Springer-Verlag New York Inc.",
  address = "New York",
  note = "[QA9.G6554]",
  year = 1971 }
@book{Guillen, author = "Michael Guillen",
  title = "Bridges to Infinity",
  publisher = "Jeremy P. Tarcher, Inc.",
  address = "Los Angeles",
  note = "[QA93.G8]",
  year = 1983 }
@book{Hamilton, author = "Alan G. Hamilton",
  title = "Logic for Mathematicians",
  edition = "revised",
  publisher = "Cambridge University Press",
  address = "Cambridge",
  note = "[QA9.H298]",
  year = 1988 }
@unpublished{Harrison, author = "John Robert Harrison",
  title = "Metatheory and Reflection in Theorem Proving:
    A Survey and Critique",
  note = "Technical Report
    CRC-053.
    SRI Cambridge,
    Millers Yard, Cambridge, UK,
    1995.
    Available on the Web as
{\verb+http:+}\-{\verb+//www.cl.cam.ac.uk/users/jrh/papers/reflect.html+}"}
@TECHREPORT{Harrison-thesis,
        author          = "John Robert Harrison",
        title           = "Theorem Proving with the Real Numbers",
        institution   = "University of Cambridge Computer
                         Lab\-o\-ra\-to\-ry",
        address         = "New Museums Site, Pembroke Street, Cambridge,
                           CB2 3QG, UK",
        year            = 1996,
        number          = 408,
        type            = "Technical Report",
        note            = "Author's PhD thesis,
   available on the Web at
{\verb+http:+}\-{\verb+//www.cl.cam.ac.uk+}\-{\verb+/users+}\-{\verb+/jrh+}%
\-{\verb+/papers+}\-{\verb+/thesis.html+}"}
@book{Herrlich, author = "Horst Herrlich and George E. Strecker",
  title = "Category Theory:  An Introduction",
  publisher = "Allyn and Bacon Inc.",
  address = "Boston",
  note = "[QA169.H567]",
  year = 1973 }
@article{Hindley, author = "J. Roger Hindley and David Meredith",
  title = "Principal Type-Schemes and Condensed Detachment",
  journal = "The Journal of Symbolic Logic",
  volume = 55,
  year = 1990,
  note = "[QA.J87]",
  pages = "90--105" }
@book{Hofstadter, author = "Douglas R. Hofstadter",
  title = "G{\"{o}}del, Escher, Bach",
  publisher = "Basic Books, Inc.",
  address = "New York",
  note = "[QA9.H63 1980]",
  year = 1979 }
@article{Indrzejczak, author= "Andrzej Indrzejczak",
  title = "Natural Deduction, Hybrid Systems and Modal Logic",
  journal = "Trends in Logic",
  volume = 30,
  publisher = "Springer",
  year = 2010 }
@article{Kalish, author = "D. Kalish and R. Montague",
  title = "On {T}arski's Formalization of Predicate Logic with Identity",
  journal = "Archiv f{\"{u}}r Mathematische Logik und Grundlagenfor\-schung",
  volume = 7,
  year = 1965,
  note = "[QA.A673]",
  pages = "81--101" }
@article{Kalman, author = "J. A. Kalman",
  title = "Condensed Detachment as a Rule of Inference",
  journal = "Studia Logica",
  volume = 42,
  number = 4,
  year = 1983,
  note = "[B18.P6.S933]",
  pages = "443-451" }
@book{Kline, author = "Morris Kline",
  title = "Mathematical Thought from Ancient to Modern Times",
  publisher = "Oxford University Press",
  address = "New York",
  note = "[QA21.K516 1990 v.3]",
  year = 1972 }
@book{Klinel, author = "Morris Kline",
  title = "Mathematics, The Loss of Certainty",
  publisher = "Oxford University Press",
  address = "New York",
  note = "[QA21.K525]",
  year = 1980 }
@book{Kramer, author = "Edna E. Kramer",
  title = "The Nature and Growth of Modern Mathematics",
  publisher = "Princeton University Press",
  address = "Princeton, New Jersey",
  note = "[QA93.K89 1981]",
  year = 1981 }
@article{Knill, author = "Oliver Knill",
  title = "Some Fundamental Theorems in Mathematics",
  year = "2018",
  url = "https://arxiv.org/abs/1807.08416" }
@book{Landau, author = "Edmund Landau",
  title = "Foundations of Analysis",
  publisher = "Chelsea Publishing Company",
  address = "New York",
  edition = "second",
  note = "[QA241.L2541 1960]",
  year = 1960 }
@article{Leblanc, author = "Hugues Leblanc",
  title = "On {M}eyer and {L}ambert's Quantificational Calculus {FQ}",
  journal = "The Journal of Symbolic Logic",
  volume = 33,
  year = 1968,
  note = "[QA.J87]",
  pages = "275--280" }
@article{Lejewski, author = "Czeslaw Lejewski",
  title = "On Implicational Definitions",
  journal = "Studia Logica",
  volume = 8,
  year = 1958,
  note = "[B18.P6.S933]",
  pages = "189--208" }
@book{Levy, author = "Azriel Levy",
  title = "Basic Set Theory",
  publisher = "Dover Publications",
  address = "Mineola, NY",
  year = "2002"
}
@book{Margaris, author = "Angelo Margaris",
  title = "First Order Mathematical Logic",
  publisher = "Blaisdell Publishing Company",
  address = "Waltham, Massachusetts",
  note = "[QA9.M327]",
  year = 1967}
@book{Manin, author = "Yu I. Manin",
  title = "A Course in Mathematical Logic",
  publisher = "Springer-Verlag",
  address = "New York",
  note = "[QA9.M29613]",
  year = "1977" }
@article{Mathias, author = "Adrian R. D. Mathias",
  title = "A Term of Length 4,523,659,424,929",
  journal = "Synthese",
  volume = 133,
  year = 2002,
  note = "[Q.S993]",
  pages = "75--86" }
@article{Megill, author = "Norman D. Megill",
  title = "A Finitely Axiomatized Formalization of Predicate Calculus
     with Equality",
  journal = "Notre Dame Journal of Formal Logic",
  volume = 36,
  year = 1995,
  note = "[QA.N914]",
  pages = "435--453" }
@unpublished{Megillc, author = "Norman D. Megill",
  title = "A Shorter Equivalent of the Axiom of Choice",
  month = "June",
  note = "Unpublished",
  year = 1991 }
@article{MegillBunder, author = "Norman D. Megill and Martin W.
    Bunder",
  title = "Weaker {D}-Complete Logics",
  journal = "Journal of the IGPL",
  volume = 4,
  year = 1996,
  pages = "215--225",
  note = "Available on the Web at
{\verb+http:+}\-{\verb+//www.mpi-sb.mpg.de+}\-{\verb+/igpl+}%
\-{\verb+/Journal+}\-{\verb+/V4-2+}\-{\verb+/#Megill+}"}
}
@book{Mendelson, author = "Elliott Mendelson",
  title = "Introduction to Mathematical Logic",
  edition = "second",
  publisher = "D. Van Nostrand Company, Inc.",
  address = "New York",
  note = "[QA9.M537 1979]",
  year = 1979 }
@article{Meredith, author = "David Meredith",
  title = "In Memoriam {C}arew {A}rthur {M}eredith (1904-1976)",
  journal = "Notre Dame Journal of Formal Logic",
  volume = 18,
  year = 1977,
  note = "[QA.N914]",
  pages = "513--516" }
@article{CAMeredith, author = "C. A. Meredith",
  title = "Single Axioms for the Systems ({C},{N}), ({C},{O}) and ({A},{N})
      of the Two-Valued Propositional Calculus",
  journal = "The Journal of Computing Systems",
  volume = 3,
  year = 1953,
  pages = "155--164" }
@article{Monk, author = "J. Donald Monk",
  title = "Provability With Finitely Many Variables",
  journal = "The Journal of Symbolic Logic",
  volume = 27,
  year = 1971,
  note = "[QA.J87]",
  pages = "353--358" }
@article{Monks, author = "J. Donald Monk",
  title = "Substitutionless Predicate Logic With Identity",
  journal = "Archiv f{\"{u}}r Mathematische Logik und Grundlagenfor\-schung",
  volume = 7,
  year = 1965,
  pages = "103--121" }
  %% Took out this from above to prevent LaTeX underfull warning:
  % note = "[QA.A673]",
@book{Moore, author = "A. W. Moore",
  title = "The Infinite",
  publisher = "Routledge",
  address = "New York",
  note = "[BD411.M59]",
  year = 1989}
@book{Munkres, author = "James R. Munkres",
  title = "Topology: A First Course",
  publisher = "Prentice-Hall, Inc.",
  address = "Englewood Cliffs, New Jersey",
  note = "[QA611.M82]",
  year = 1975}
@article{Nemesszeghy, author = "E. Z. Nemesszeghy and E. A. Nemesszeghy",
  title = "On Strongly Creative Definitions:  A Reply to {V}. {F}. {R}ickey",
  journal = "Logique et Analyse (N.\ S.)",
  year = 1977,
  volume = 20,
  note = "[BC.L832]",
  pages = "111--115" }
@unpublished{Nemeti, author = "N{\'{e}}meti, I.",
  title = "Algebraizations of Quantifier Logics, an Overview",
  note = "Version 11.4, preprint, Mathematical Institute, Budapest,
    1994.  A shortened version without proofs appeared in
    ``Algebraizations of quantifier logics, an introductory overview,''
   {\em Studia Logica}, 50:485--569, 1991 [B18.P6.S933]"}
@article{Pavicic, author = "M. Pavi{\v{c}}i{\'{c}}",
  title = "A New Axiomatization of Unified Quantum Logic",
  journal = "International Journal of Theoretical Physics",
  year = 1992,
  volume = 31,
  note = "[QC.I626]",
  pages = "1753 --1766" }
@book{Penrose, author = "Roger Penrose",
  title = "The Emperor's New Mind",
  publisher = "Oxford University Press",
  address = "New York",
  note = "[Q335.P415]",
  year = 1989 }
@book{PetersonI, author = "Ivars Peterson",
  title = "The Mathematical Tourist",
  publisher = "W. H. Freeman and Company",
  address = "New York",
  note = "[QA93.P475]",
  year = 1988 }
@article{Peterson, author = "Jeremy George Peterson",
  title = "An automatic theorem prover for substitution and detachment systems",
  journal = "Notre Dame Journal of Formal Logic",
  volume = 19,
  year = 1978,
  note = "[QA.N914]",
  pages = "119--122" }
@book{Quine, author = "Willard Van Orman Quine",
  title = "Set Theory and Its Logic",
  edition = "revised",
  publisher = "The Belknap Press of Harvard University Press",
  address = "Cambridge, Massachusetts",
  note = "[QA248.Q7 1969]",
  year = 1969 }
@article{Robinson, author = "J. A. Robinson",
  title = "A Machine-Oriented Logic Based on the Resolution Principle",
  journal = "Journal of the Association for Computing Machinery",
  year = 1965,
  volume = 12,
  pages = "23--41" }
@article{RobinsonT, author = "T. Thacher Robinson",
  title = "Independence of Two Nice Sets of Axioms for the Propositional
    Calculus",
  journal = "The Journal of Symbolic Logic",
  volume = 33,
  year = 1968,
  note = "[QA.J87]",
  pages = "265--270" }
@book{Rucker, author = "Rudy Rucker",
  title = "Infinity and the Mind:  The Science and Philosophy of the
    Infinite",
  publisher = "Bantam Books, Inc.",
  address = "New York",
  note = "[QA9.R79 1982]",
  year = 1982 }
@book{Russell, author = "Bertrand Russell",
  title = "Mysticism and Logic, and Other Essays",
  publisher = "Barnes \& Noble Books",
  address = "Totowa, New Jersey",
  note = "[B1649.R963.M9 1981]",
  year = 1981 }
@article{Russell2, author = "Bertrand Russell",
  title = "Recent Work on the Principles of Mathematics",
  journal = "International Monthly",
  volume = 4,
  year = 1901,
  pages = "84"}
@article{Schmidt, author = "Eric Schmidt",
  title = "Reductions in Norman Megill's axiom system for complex numbers",
  url = "http://us.metamath.org/downloads/schmidt-cnaxioms.pdf",
  year = "2012" }
@book{Shoenfield, author = "Joseph R. Shoenfield",
  title = "Mathematical Logic",
  publisher = "Addison-Wesley Publishing Company, Inc.",
  address = "Reading, Massachusetts",
  year = 1967,
  note = "[QA9.S52]" }
@book{Smullyan, author = "Raymond M. Smullyan",
  title = "Theory of Formal Systems",
  publisher = "Princeton University Press",
  address = "Princeton, New Jersey",
  year = 1961,
  note = "[QA248.5.S55]" }
@book{Solow, author = "Daniel Solow",
  title = "How to Read and Do Proofs:  An Introduction to Mathematical
    Thought Process",
  publisher = "John Wiley \& Sons",
  address = "New York",
  year = 1982,
  note = "[QA9.S577]" }
@book{Stark, author = "Harold M. Stark",
  title = "An Introduction to Number Theory",
  publisher = "Markham Publishing Company",
  address = "Chicago",
  note = "[QA241.S72 1978]",
  year = 1970 }
@article{Swart, author = "E. R. Swart",
  title = "The Philosophical Implications of the Four-Color Problem",
  journal = "American Mathematical Monthly",
  year = 1980,
  volume = 87,
  month = "November",
  note = "[QA.A5125]",
  pages = "697--707" }
@book{Szpiro, author = "George G. Szpiro",
  title = "Poincar{\'{e}}'s Prize: The Hundred-Year Quest to Solve One
    of Math's Greatest Puzzles",
  publisher = "Penguin Books Ltd",
  address = "London",
  note = "[QA43.S985 2007]",
  year = 2007}
@book{Takeuti, author = "Gaisi Takeuti and Wilson M. Zaring",
  title = "Introduction to Axiomatic Set Theory",
  edition = "second",
  publisher = "Springer-Verlag New York Inc.",
  address = "New York",
  note = "[QA248.T136 1982]",
  year = 1982}
@inproceedings{Tarski, author = "Alfred Tarski",
  title = "What is Elementary Geometry",
  pages = "16--29",
  booktitle = "The Axiomatic Method, with Special Reference to Geometry and
     Physics (Proceedings of an International Symposium held at the University
     of California, Berkeley, December 26, 1957 --- January 4, 1958)",
  editor = "Leon Henkin and Patrick Suppes and Alfred Tarski",
  year = 1959,
  publisher = "North-Holland Publishing Company",
  address = "Amsterdam"}
@article{Tarski1965, author = "Alfred Tarski",
  title = "A Simplified Formalization of Predicate Logic with Identity",
  journal = "Archiv f{\"{u}}r Mathematische Logik und Grundlagenforschung",
  volume = 7,
  year = 1965,
  note = "[QA.A673]",
  pages = "61--79" }
@book{Tymoczko,
  title = "New Directions in the Philosophy of Mathematics",
  editor = "Thomas Tymoczko",
  publisher = "Birkh{\"{a}}user Boston, Inc.",
  address = "Boston",
  note = "[QA8.6.N48 1986]",
  year = 1986 }
@incollection{Wang,
  author = "Hao Wang",
  title = "Theory and Practice in Mathematics",
  pages = "129--152",
  booktitle = "New Directions in the Philosophy of Mathematics",
  editor = "Thomas Tymoczko",
  publisher = "Birkh{\"{a}}user Boston, Inc.",
  address = "Boston",
  note = "[QA8.6.N48 1986]",
  year = 1986 }
@manual{Webster,
  title = "Webster's New Collegiate Dictionary",
  organization = "G. \& C. Merriam Co.",
  address = "Springfield, Massachusetts",
  note = "[PE1628.W4M4 1977]",
  year = 1977 }
@manual{Whitehead, author = "Alfred North Whitehead",
  title = "An Introduction to Mathematics",
  year = 1911 }
@book{PM, author = "Alfred North Whitehead and Bertrand Russell",
  title = "Principia Mathematica",
  edition = "second",
  publisher = "Cambridge University Press",
  address = "Cambridge",
  year = "1927",
  note = "(3 vols.) [QA9.W592 1927]" }
@article{DBLP:journals/corr/Whalen16,
  author    = {Daniel Whalen},
  title     = {Holophrasm: a neural Automated Theorem Prover for higher-order logic},
  journal   = {CoRR},
  volume    = {abs/1608.02644},
  year      = {2016},
  url       = {http://arxiv.org/abs/1608.02644},
  archivePrefix = {arXiv},
  eprint    = {1608.02644},
  timestamp = {Mon, 13 Aug 2018 16:46:19 +0200},
  biburl    = {https://dblp.org/rec/bib/journals/corr/Whalen16},
  bibsource = {dblp computer science bibliography, https://dblp.org} }
@article{Wiedijk-revisited,
  author = {Freek Wiedijk},
  title = {The QED Manifesto Revisited},
  year = {2007},
  url = {http://mizar.org/trybulec65/8.pdf} }
@book{Wolfram,
  author = "Stephen Wolfram",
  title = "Mathematica:  A System for Doing Mathematics by Computer",
  edition = "second",
  publisher = "Addison-Wesley Publishing Co.",
  address = "Redwood City, California",
  note = "[QA76.95.W65 1991]",
  year = 1991 }
@book{Wos, author = "Larry Wos and Ross Overbeek and Ewing Lusk and Jim Boyle",
  title = "Automated Reasoning:  Introduction and Applications",
  edition = "second",
  publisher = "McGraw-Hill, Inc.",
  address = "New York",
  note = "[QA76.9.A96.A93 1992]",
  year = 1992 }

%
%
%[1] Church, Alonzo, Introduction to Mathematical Logic,
% Volume 1, Princeton University Press, Princeton, N. J., 1956.
%
%[2] Cohen, Paul J., Set Theory and the Continuum Hypothesis,
% W. A. Benjamin, Inc., Reading, Mass., 1966.
%
%[3] Hamilton, Alan G., Logic for Mathematicians, Cambridge
% University Press,
% Cambridge, 1988.

%[6] Kleene, Stephen Cole, Introduction to Metamathematics, D.  Van
% Nostrand Company, Inc., Princeton (1952).

%[13] Tarski, Alfred, "A simplified formalization of predicate
% logic with identity," Archiv fur Mathematische Logik und
% Grundlagenforschung, vol. 7 (1965), pp. 61-79.

%[14] Tarski, Alfred and Steven Givant, A Formalization of Set
% Theory Without Variables, American Mathematical Society Colloquium
% Publications, vol. 41, American Mathematical Society,
% Providence, R. I., 1987.

%[15] Zeman, J. J., Modal Logic, Oxford University Press, Oxford, 1973.
\end{filecontents}
% --------------------------- End of metamath.bib -----------------------------


%Book: Metamath
%Author:  Norman Megill Email:  nm at alum.mit.edu
%Author:  David A. Wheeler Email:  dwheeler at dwheeler.com

% A book template example
% http://www.stsci.edu/ftp/software/tex/bookstuff/book.template

\documentclass[leqno]{book} % LaTeX 2e. 10pt. Use [leqno,12pt] for 12pt
% hyperref 2002/05/27 v6.72r  (couldn't get pagebackref to work)
\usepackage[plainpages=false,pdfpagelabels=true]{hyperref}

\usepackage{needspace}     % Enable control over page breaks
\usepackage{breqn}         % automatic equation breaking
\usepackage{microtype}     % microtypography, reduces hyphenation

% Packages for flexible tables.  We need to be able to
% wrap text within a cell (with automatically-determined widths) AND
% split a table automatically across multiple pages.
% * "tabularx" wraps text in cells but only 1 page
% * "longtable" goes across pages but by itself is incompatible with tabularx
% * "ltxtable" combines longtable and tabularx, but table contents
%    must be in a separate file.
% * "ltablex" combines tabularx and longtable - must install specially
% * "booktabs" is recommended as a way to improve the look of tables,
%   but doesn't add these capabilities.
% * "tabu" much more capable and seems to be recommended. So use that.

\usepackage{makecell}      % Enable forced line splits within a table cell
\usepackage{longtable}     % Enable multi-page tables
\usepackage{tabu}          % Multi-page tables with wrapped text in a cell

% You can find more Tex packages using commands like:
% tlmgr search --file tabu.sty
% find /usr/share/texmf-dist/ -name '*tab*'
%
%%%%%%%%%%%%%%%%%%%%%%%%%%%%%%%%%%%%%%%%%%%%%%%%%%%%%%%%%%%%%%%%%%%%%%%%%%%%
% Uncomment the next 3 lines to suppress boxes and colors on the hyperlinks
%%%%%%%%%%%%%%%%%%%%%%%%%%%%%%%%%%%%%%%%%%%%%%%%%%%%%%%%%%%%%%%%%%%%%%%%%%%%
%\hypersetup{
%colorlinks,citecolor=black,filecolor=black,linkcolor=black,urlcolor=black
%}
%
\usepackage{realref}

% Restarting page numbers: try?
%   \printglossary
%   \cleardoublepage
%   \pagenumbering{arabic}
%   \setcounter{page}{1}    ???needed
%   \include{chap1}

% not used:
% \def\R2Lurl#1#2{\mbox{\href{#1}\texttt{#2}}}

\usepackage{amssymb}

% Version 1 of book: margins: t=.4, b=.2, ll=.4, rr=.55
% \usepackage{anysize}
% % \papersize{<height>}{<width>}
% % \marginsize{<left>}{<right>}{<top>}{<bottom>}
% \papersize{9in}{6in}
% % l/r 0.6124-0.6170 works t/b 0.2418-0.3411 = 192pp. 0.2926-03118=exact
% \marginsize{0.7147in}{0.5147in}{0.4012in}{0.2012in}

\usepackage{anysize}
% \papersize{<height>}{<width>}
% \marginsize{<left>}{<right>}{<top>}{<bottom>}
\papersize{9in}{6in}
% l/r 0.85in&0.6431-0.6539 works t/b ?-?
%\marginsize{0.85in}{0.6485in}{0.55in}{0.35in}
\marginsize{0.8in}{0.65in}{0.5in}{0.3in}

% \usepackage[papersize={3.6in,4.8in},hmargin=0.1in,vmargin={0.1in,0.1in}]{geometry}  % page geometry
\usepackage{special-settings}

\raggedbottom
\makeindex

\begin{document}
% Discourage page widows and orphans:
\clubpenalty=300
\widowpenalty=300

%%%%%%% load in AMS fonts %%%%%%% % LaTeX 2.09 - obsolete in LaTeX 2e
%\input{amssym.def}
%\input{amssym.tex}
%\input{c:/texmf/tex/plain/amsfonts/amssym.def}
%\input{c:/texmf/tex/plain/amsfonts/amssym.tex}

\bibliographystyle{plain}
\pagenumbering{roman}
\pagestyle{headings}

\thispagestyle{empty}

\hfill
\vfill

\begin{center}
{\LARGE\bf Metamath} \\
\vspace{1ex}
{\large A Computer Language for Mathematical Proofs} \\
\vspace{7ex}
{\large Norman Megill} \\
\vspace{7ex}
with extensive revisions by \\
\vspace{1ex}
{\large David A. Wheeler} \\
\vspace{7ex}
% Printed date. If changing the date below, also fix the date at the beginning.
2019-05-24
\end{center}

\vfill
\hfill

\newpage
\thispagestyle{empty}

\hfill
\vfill

\begin{center}
$\sim$\ {\sc Public Domain}\ $\sim$

\vspace{2ex}
This book (including its later revisions)
has been released into the Public Domain by Norman Megill per the
Creative Commons CC0 1.0 Universal (CC0 1.0) Public Domain Dedication
(\url{https://creativecommons.org/publicdomain/zero/1.0/}).
David A. Wheeler has done the same.
The public domain release applies worldwide.  In case this is not
legally possible, the right is granted to use the work for any purpose,
without any conditions, unless such conditions are required by law.

\vspace{3ex}
Several short, attributed quotations from copyrighted works
appear in this book under the ``fair use'' provision of Section 107 of
the United States Copyright Act (Title 17 of the {\em United States
Code}).  The public-domain status of this book is not applicable to
those quotations.

\vspace{3ex}
Any trademarks used in this book are the property of their owners.

% QA76.9.L63.M??

% \vspace{1ex}
%
% \vspace{1ex}
% {\small Permission is granted to make and distribute verbatim copies of this
% book
% provided the copyright notice and this
% permission notice are preserved on all copies.}
%
% \vspace{1ex}
% {\small Permission is granted to copy and distribute modified versions of this
% book under the conditions for verbatim copying, provided that the
% entire
% resulting derived work is distributed under the terms of a permission
% notice
% identical to this one.}
%
% \vspace{1ex}
% {\small Permission is granted to copy and distribute translations of this
% book into another language, under the above conditions for modified
% versions,
% except that this permission notice may be stated in a translation
% approved by the
% author.}
%
% \vspace{1ex}
% %{\small   For a copy of the \LaTeX\ source files for this book, contact
% %the author.} \\
% \ \\
% \ \\

\vspace{7ex}
% ISBN: 1-4116-3724-0 \\
ISBN: 978-1-4116-3724-5 \\
{\ } \\
Lulu Press \\
Morrisville, North Carolina\\
USA


\hfill
\vfill

Norman Megill\\ 93 Bridge St., Lexington, MA 02421 \\
E-mail address: \texttt{nm{\char`\@}alum.mit.edu} \\
\vspace{7ex}
David A. Wheeler \\
E-mail address: \texttt{dwheeler{\char`\@}dwheeler.com} \\
% See notes added at end of Preface for revision history. \\
% For current information on the Metamath software see \\
\vspace{7ex}
\url{http://metamath.org}
\end{center}

% \newpage
% \thispagestyle{empty}
%
% \hfill
% \vfill
%
% \begin{center}
% {\it To my son Robin Dwight Megill}
% \end{center}
%
% \vfill
% \hfill
%
% \newpage

\tableofcontents
%\listoftables

\chapter*{Preface}
\markboth{PREFACE}{PREFACE}
\addcontentsline{toc}{section}{Preface}


% (For current information, see the notes added at the
% end of this preface on p.~\pageref{note2002}.)

\subsubsection{Overview}

Metamath\index{Metamath} is a computer language and an associated computer
program for archiving, verifying, and studying mathematical proofs at a very
detailed level.  The Metamath language incorporates no mathematics per se but
treats all mathematical statements as mere sequences of symbols.  You provide
Metamath with certain special sequences (axioms) that tell it what rules
of inference are allowed.  Metamath is not limited to any specific field of
mathematics.  The Metamath language is simple and robust, with an
almost total absence of hard-wired syntax, and
I\footnote{Unless otherwise noted, the words
``I,'' ``me,'' and ``my'' refer to Norman Megill\index{Megill, Norman}, while
``we,'' ``us,'' and ``our'' refer to Norman Megill and
David A. Wheeler\index{Wheeler, David A.}.}
believe that it
provides about the simplest possible framework that allows essentially all of
mathematics to be expressed with absolute rigor.

% index test
%\newcommand{\nn}[1]{#1n}
%\index{aaa@bbb}
%\index{abc!def}
%\index{abd|see{qqq}}
%\index{abe|nn}
%\index{abf|emph}
%\index{abg|(}
%\index{abg|)}

Using the Metamath language, you can build formal or mathematical
systems\index{formal system}\footnote{A formal or mathematical system consists
of a collection of symbols (such as $2$, $4$, $+$ and $=$), syntax rules that
describe how symbols may be combined to form a legal expression (called a
well-formed formula or {\em wff}, pronounced ``whiff''), some starting wffs
called axioms, and inference rules that describe how theorems may be derived
(proved) from the axioms.  A theorem is a mathematical fact such as $2+2=4$.
Strictly speaking, even an obvious fact such as this must be proved from
axioms to be formally acceptable to a mathematician.}\index{theorem}
\index{axiom}\index{rule}\index{well-formed formula (wff)} that involve
inferences from axioms.  Although a database is provided
that includes a recommended set of axioms for standard mathematics, if you
wish you can supply your own symbols, syntax, axioms, rules, and definitions.

The name ``Metamath'' was chosen to suggest that the language provides a
means for {\em describing} mathematics rather than {\em being} the
mathematics itself.  Actually in some sense any mathematical language is
metamathematical.  Symbols written on paper, or stored in a computer,
are not mathematics itself but rather a way of expressing mathematics.
For example ``7'' and ``VII'' are symbols for denoting the number seven
in Arabic and Roman numerals; neither {\em is} the number seven.

If you are able to understand and write computer programs, you should be able
to follow abstract mathematics with the aid of Metamath.  Used in conjunction
with standard textbooks, Metamath can guide you step-by-step towards an
understanding of abstract mathematics from a very rigorous viewpoint, even if
you have no formal abstract mathematics background.  By using a single,
consistent notation to express proofs, once you grasp its basic concepts
Metamath provides you with the ability to immediately follow and dissect
proofs even in totally unfamiliar areas.

Of course, just being able follow a proof will not necessarily give you an
intuitive familiarity with mathematics.  Memorizing the rules of chess does not
give you the ability to appreciate the game of a master, and knowing how the
notes on a musical score map to piano keys does not give you the ability to
hear in your head how it would sound.  But each of these can be a first step.

Metamath allows you to explore proofs in the sense that you can see the
theorem referenced at any step expanded in as much detail as you want, right
down to the underlying axioms of logic and set theory (in the case of the set
theory database provided).  While Metamath will not replace the higher-level
understanding that can only be acquired through exercises and hard work, being
able to see how gaps in a proof are filled in can give you increased
confidence that can speed up the learning process and save you time when you
get stuck.

The Metamath language breaks down a mathematical proof into its tiniest
possible parts.  These can be pieced together, like interlocking
pieces in a puzzle, only in a way that produces correct and absolutely rigorous
mathematics.

The nature of Metamath\index{Metamath} enforces very precise mathematical
thinking, similar to that involved in writing a computer program.  A crucial
difference, though, is that once a proof is verified (by the Metamath program)
to be correct, it is definitely correct; it can never have a hidden
``bug.''\index{computer program bugs}  After getting used to the kind of rigor
and accuracy provided by Metamath, you might even be tempted to
adopt the attitude that a proof should never be considered correct until it
has been verified by a computer, just as you would not completely trust a
manual calculation until you have verified it on a
calculator.

My goal
for Metamath was a system for describing and verifying
mathematics that is completely universal yet conceptually as simple as
possible.  In approaching mathematics from an axiomatic, formal viewpoint, I
wanted Metamath to be able to handle almost any mathematical system, not
necessarily with ease, but at least in principle and hopefully in practice. I
wanted it to verify proofs with absolute rigor, and for this reason Metamath
is what might be thought of as a ``compile-only'' language rather than an
algorithmic or Turing-machine language (Pascal, C, Prolog, Mathematica,
etc.).  In other words, a database written in the Metamath
language doesn't ``do'' anything; it merely exhibits mathematical knowledge
and permits this knowledge to be verified as being correct.  A program in an
algorithmic language can potentially have hidden bugs\index{computer program
bugs} as well as possibly being hard to understand.  But each token in a
Metamath database must be consistent with the database's earlier
contents according to simple, fixed rules.
If a database is verified
to be correct,\footnote{This includes
verification that a sequential list of proof steps results in the specified
theorem.} then the mathematical content is correct if the
verifier is correct and the axioms are correct.
The verification program could be incorrect, but the verification algorithm
is relatively simple (making it unlikely to be implemented incorrectly
by the Metamath program),
and there are over a dozen Metamath database verifiers
written by different people in different programming languages
(so these different verifiers can act as multiple reviewers of a database).
The most-used Metamath database, the Metamath Proof Explorer
(aka \texttt{set.mm}\index{set theory database (\texttt{set.mm})}%
\index{Metamath Proof Explorer}),
is currently verified by four different Metamath verifiers written by
four different people in four different languages, including the
original Metamath program described in this book.
The only ``bugs'' that can exist are in the statement of the axioms,
for example if the axioms are inconsistent (a famous problem shown to be
unsolvable by G\"{o}del's incompleteness theorem\index{G\"{o}del's
incompleteness theorem}).
However, real mathematical systems have very few axioms, and these can
be carefully studied.
All of this provides extraordinarily high confidence that the verified database
is in fact correct.

The Metamath program
doesn't prove theorems automatically but is designed to verify proofs
that you supply to it.
The underlying Metamath language is completely general and has no built-in,
preconceived notions about your formal system\index{formal system}, its logic
or its syntax.
For constructing proofs, the Metamath program has a Proof Assistant\index{Proof
Assistant} which helps you fill in some of a proof step's details, shows you
what choices you have at any step, and verifies the proof as you build it; but
you are still expected to provide the proof.

There are many other programs that can process or generate information
in the Metamath language, and more continue to be written.
This is in part because the Metamath language itself is very simple
and intentionally easy to automatically process.
Some programs, such as \texttt{mmj2}\index{mmj2}, include a proof assistant
that can automate some steps beyond what the Metamath program can do.
Mario Carneiro has developed an algorithm for converting proofs from
the OpenTheory interchange format, which can be translated to and from
any of the HOL family of proof languages (HOL4, HOL Light, ProofPower,
and Isabelle), into the
Metamath language.\cite{DBLP:journals/corr/Carneiro14}\index{Carneiro, Mario}.
Daniel Whalen has developed Holophrasm, which can automatically
prove many Metamath proofs using
machine learning\index{machine learning}\index{artificial intelligence}
approaches
(including multiple neural networks\index{neural networks})\cite{DBLP:journals/corr/Whalen16}\index{Whalen, Daniel}.
However,
a discussion of these other programs is beyond the scope of this book.

Like most computer languages, the Metamath\index{Metamath} language uses the
standard ({\sc ascii}) characters on a computer keyboard, so it cannot
directly represent many of the special symbols that mathematicians use.  A
useful feature of the Metamath program is its ability to convert its notation
into the \LaTeX\ typesetting language.\index{latex@{\LaTeX}}  This feature
lets you convert the {\sc ascii} tokens you've defined into standard
mathematical symbols, so you end up with symbols and formulas you are familiar
with instead of somewhat cryptic {\sc ascii} representations of them.
The Metamath program can also generate HTML\index{HTML}, making it easy
to view results on the web and to see related information by using
hypertext links.

Metamath is probably conceptually different from anything you've seen
before and some aspects may take some getting used to.  This book will
help you decide whether Metamath suits your specific needs.

\subsubsection{Setting Your Expectations}
It is important for you to understand what Metamath\index{Metamath} is and is
not.  As mentioned, the Metamath program
is {\em not} an automated theorem prover but
rather a proof verifier.  Developing a database can be tedious, hard work,
especially if you want to make the proofs as short as possible, but it becomes
easier as you build up a collection of useful theorems.  The purpose of
Metamath is simply to document existing mathematics in an absolutely rigorous,
computer-verifiable way, not to aid directly in the creation of new
mathematics.  It also is not a magic solution for learning abstract
mathematics, although it may be helpful to be able to actually see the implied
rigor behind what you are learning from textbooks, as well as providing hints
to work out proofs that you are stumped on.

As of this writing, a sizable set theory database has been developed to
provide a foundation for many fields of mathematics, but much more work would
be required to develop useful databases for specific fields.

Metamath\index{Metamath} ``knows no math;'' it just provides a framework in
which to express mathematics.  Its language is very small.  You can define two
kinds of symbols, constants\index{constant} and variables\index{variable}.
The only thing Metamath knows how to do is to substitute strings of symbols
for the variables\index{substitution!variable}\index{variable substitution} in
an expression based on instructions you provide it in a proof, subject to
certain constraints you specify for the variables.  Even the decimal
representation of a number is merely a string of certain constants (digits)
which together, in a specific context, correspond to whatever mathematical
object you choose to define for it; unlike other computer languages, there is
no actual number stored inside the computer.  In a proof, you in effect
instruct Metamath what symbol substitutions to make in previous axioms or
theorems and join a sequence of them together to result in the desired
theorem.  This kind of symbol manipulation captures the essence of mathematics
at a preaxiomatic level.

\subsubsection{Metamath and Mathematical Literature}

In advanced mathematical literature, proofs are usually presented in the form
of short outlines that often only an expert can follow.  This is partly out of
a desire for brevity, but it would also be unwise (even if it were practical)
to present proofs in complete formal detail, since the overall picture would
be lost.\index{formal proof}

A solution I envision\label{envision} that would allow mathematics to remain
acceptable to the expert, yet increase its accessibility to non-specialists,
consists of a combination of the traditional short, informal proof in print
accompanied by a complete formal proof stored in a computer database.  In an
analogy with a computer program, the informal proof is like a set of comments
that describe the overall reasoning and content of the proof, whereas the
computer database is like the actual program and provides a means for anyone,
even a non-expert, to follow the proof in as much detail as desired, exploring
it back through layers of theorems (like subroutines that call other
subroutines) all the way back to the axioms of the theory.  In addition, the
computer database would have the advantage of providing absolute assurance
that the proof is correct, since each step can be verified automatically.

There are several other approaches besides Metamath to a project such
as this.  Section~\ref{proofverifiers} discusses some of these.

To us, a noble goal would be a database with hundreds of thousands of
theorems and their computer-verifiable proofs, encompassing a significant
fraction of known mathematics and available for instant access.
These would be fully verified by multiple independently-implemented verifiers,
to provide extremely high confidence that the proofs are completely correct.
The database would allow people to investigate whatever details they were
interested in, so that they could confirm whatever portions they wished.
Whether or not Metamath is an appropriate choice remains to be seen, but in
principle we believe it is sufficient.

\subsubsection{Formalism}

Over the past fifty years, a group of French mathematicians working
collectively under the pseudonym of Bourbaki\index{Bourbaki, Nicolas} have
co-authored a series of monographs that attempt to rigorously and
consistently formalize large bodies of mathematics from foundations.  On the
one hand, certainly such an effort has its merits; on the other hand, the
Bourbaki project has been criticized for its ``scholasticism'' and
``hyperaxiomatics'' that hide the intuitive steps that lead to the results
\cite[p.~191]{Barrow}\index{Barrow, John D.}.

Metamath unabashedly carries this philosophy to its extreme and no doubt is
subject to the same kind of criticism.  Nonetheless I think that in
conjunction with conventional approaches to mathematics Metamath can serve a
useful purpose.  The Bourbaki approach is essentially pedagogic, requiring the
reader to become intimately familiar with each detail in a very large
hierarchy before he or she can proceed to the next step.  The difference with
Metamath is that the ``reader'' (user) knows that all details are contained in
its computer database, available as needed; it does not demand that the user
know everything but conveniently makes available those portions that are of
interest.  As the body of all mathematical knowledge grows larger and larger,
no one individual can have a thorough grasp of its entirety.  Metamath
can finalize and put to rest any questions about the validity of any part of it
and can make any part of it accessible, in principle, to a non-specialist.

\subsubsection{A Personal Note}
Why did I develop Metamath\index{Metamath}?  I enjoy abstract mathematics, but
I sometimes get lost in a barrage of definitions and start to lose confidence
that my proofs are correct.  Or I reach a point where I lose sight of how
anything I'm doing relates to the axioms that a theory is based on and am
sometimes suspicious that there may be some overlooked implicit axiom
accidentally introduced along the way (as happened historically with Euclidean
geometry\index{Euclidean geometry}, whose omission of Pasch's
axiom\index{Pasch's axiom} went unnoticed for 2000 years
\cite[p.~160]{Davis}!). I'm also somewhat lazy and wish to avoid the effort
involved in re-verifying the gaps in informal proofs ``left to the reader;'' I
prefer to figure them out just once and not have to go through the same
frustration a year from now when I've forgotten what I did.  Metamath provides
better recovery of my efforts than scraps of paper that I can't
decipher anymore.  But mostly I find very appealing the idea of rigorously
archiving mathematical knowledge in a computer database, providing precision,
certainty, and elimination of human error.

\subsubsection{Note on Bibliography and Index}

The Bibliography usually includes the Library of Congress classification
for a work to make it easier for you to find it in on a university
library shelf.  The Index has author references to pages where their works
are cited, even though the authors' names may not appear on those pages.

\subsubsection{Acknowledgments}

Acknowledgments are first due to my wife, Deborah (who passed away on
September 4, 1998), for critiquing the manu\-script but most of all for
her patience and support.  I also wish to thank Joe Wright, Richard
Becker, Clarke Evans, Buddha Buck, and Jeremy Henty for helpful
comments.  Any errors, omissions, and other shortcomings are of course
my responsibility.

\subsubsection{Note Added June 22, 2005}\label{note2002}

The original, unpublished version of this book was written in 1997 and
distributed via the web.  The present edition has been updated to
reflect the current Metamath program and databases, as well as more
current {\sc url}s for Internet sites.  Thanks to Josh
Purinton\index{Purinton, Josh}, One Hand
Clapping, Mel L.\ O'Cat, and Roy F. Longton for pointing out
typographical and other errors.  I have also benefitted from numerous
discussions with Raph Levien\index{Levien, Raph}, who has extended
Metamath's philosophy of rigor to result in his {\em
Ghilbert}\index{Ghilbert} proof language (\url{http://ghilbert.org}).

Robert (Bob) Solovay\index{Solovay, Robert} communicated a new result of
A.~R.~D.~Mathias on the system of Bourbaki, and the text has been
updated accordingly (p.~\pageref{bourbaki}).

Bob also pointed out a clarification of the literature regarding
category theory and inaccessible cardinals\index{category
theory}\index{cardinal, inaccessible} (p.~\pageref{categoryth}),
and a misleading statement was removed from the text.  Specifically,
contrary to a statement in previous editions, it is possible to express
``There is a proper class of inaccessible cardinals'' in the language of
ZFC.  This can be done as follows:  ``For every set $x$ there is an
inaccessible cardinal $\kappa$ such that $\kappa$ is not in $x$.''
Bob writes:\footnote{Private communication, Nov.~30, 2002.}
\begin{quotation}
     This axiom is how Grothendieck presents category theory.  To each
inaccessible cardinal $\kappa$ one associates a Grothendieck universe
\index{Grothendieck, Alexander} $U(\kappa)$.  $U(\kappa)$ consists of
those sets which lie in a transitive set of cardinality less than
$\kappa$.  Instead of the ``category of all groups,'' one works relative
to a universe [considering the category of groups of cardinality less
than $\kappa$].  Now the category whose objects are all categories
``relative to the universe $U(\kappa)$'' will be a category not
relative to this universe but to the next universe.

     All of the things category theorists like to do can be done in this
framework.  The only controversial point is whether the Grothen\-dieck
axiom is too strong for the needs of category theorists.  Mac Lane
\index{Mac Lane, Saunders} argues that ``one universe is enough'' and
Feferman\index{Feferman, Solomon} has argued that one can get by with
ordinary ZFC.  I don't find Feferman's arguments persuasive.  Mac Lane
may be right, but when I think about category theory I do it \`{a} la
Grothendieck.

        By the way Mizar\index{Mizar} adds the axiom ``there is a proper
class of inaccessibles'' precisely so as to do category theory.
\end{quotation}

The most current information on the Metamath program and databases can
always be found at \url{http://metamath.org}.


\subsubsection{Note Added June 24, 2006}\label{note2006}

The Metamath spec was restricted slightly to make parsers easier to
write.  See the footnote on p.~\pageref{namespace}.

%\subsubsection{Note Added July 24, 2006}\label{note2006b}
\subsubsection{Note Added March 10, 2007}\label{note2006b}

I am grateful to Anthony Williams\index{Williams, Anthony} for writing
the \LaTeX\ package called {\tt realref.sty} and contributing it to the
public domain.  This package allows the internal hyperlinks in a {\sc
pdf} file to anchor to specific page numbers instead of just section
titles, making the navigation of the {\sc pdf} file for this book much
more pleasant and ``logical.''

A typographical error found by Martin Kiselkov was corrected.
A confusing remark about unification was deleted per suggestion of
Mel O'Cat.

\subsubsection{Note Added May 27, 2009}\label{note2009}

Several typos found by Kim Sparre were corrected.  A note was added that
the Poincar\'{e} conjecture has been proved (p.~\pageref{poincare}).

\subsubsection{Note Added Nov. 17, 2014}\label{note2014}

The statement of the Schr\"{o}der--Bernstein theorem was corrected in
Section~\ref{trust}.  Thanks to Bob Solovay for pointing out the error.

\subsubsection{Note Added May 25, 2016}\label{note2016}

Thanks to Jerry James for correcting 16 typos.

\subsubsection{Note Added February 25, 2019}\label{note201902}

David A. Wheeler\index{Wheeler, David A.}
made a large number of improvements and updates,
in coordination with Norman Megill.
The predicate calculus axioms were renumbered, and the text makes
it clear that they are based on Tarski's system S2;
the one slight deviation in axiom ax-6 is explained and justified.
The real and complex number axioms were modified to be consistent with
\texttt{set.mm}\index{set theory database (\texttt{set.mm})}%
\index{Metamath Proof Explorer}.
Long-awaited specification changes ``1--8'' were made,
which clarified previously ambiguous points.
Some errors in the text involving \texttt{\$f} and
\texttt{\$d} statements were corrected (the spec was correct, but
the in-book explanations unintentionally contradicted the spec).
We now have a system for automatically generating narrow PDFs,
so that those with smartphones can have easy access to the current
version of this document.
A new section on deduction was added;
it discusses the standard deduction theorem,
the weak deduction theorem,
deduction style, and natural deduction.
Many minor corrections (too numerous to list here) were also made.

\subsubsection{Note Added March 7, 2019}\label{note201903}

This added a description of the Matamath language syntax in
Extended Backus--Naur Form (EBNF)\index{Extended Backus--Naur Form}\index{EBNF}
in Appendix \ref{BNF}, added a brief explanation about typecodes,
inserted more examples in the deduction section,
and added a variety of smaller improvements.

\subsubsection{Note Added April 7, 2019}\label{note201904}

This version clarified the proper substitution notation, improved the
discussion on the weak deduction theorem and natural deduction,
documented the \texttt{undo} command, updated the information on
\texttt{write source}, changed the typecode
from \texttt{set} to \texttt{setvar} to be consistent with the current
version of \texttt{set.mm}, added more documentation about comment markup
(e.g., documented how to create headings), and clarified the
differences between various assertion forms (in particular deduction form).

\subsubsection{Note Added May 20, 2019}\label{note201905}

This version fixes various small issues and added specific examples
of forms (deduction form, inference form, and closed form).

\chapter{Introduction}
\pagenumbering{arabic}

\begin{quotation}
  {\em {\em I.M.:}  No, no.  There's nothing subjective about it!  Everybody
knows what a proof is.  Just read some books, take courses from a competent
mathematician, and you'll catch on.

{\em Student:}  Are you sure?

{\em I.M.:}  Well---it is possible that you won't, if you don't have any
aptitude for it.  That can happen, too.

{\em Student:}  Then {\em you} decide what a proof is, and if I don't learn
to decide in the same way, you decide I don't have any aptitude.

{\em I.M.:}  If not me, then who?}
    \flushright\sc  ``The Ideal Mathematician''
    \index{Davis, Phillip J.}
    \footnote{\cite{Davis}, p.~40.}\\
\end{quotation}

In the past century, brilliant mathematicians have discovered almost
unimaginably profound results that rank among the crowning intellectual
achievements of mankind.  However, there is a sense in which modern abstract
mathematics is behind the times, stuck in an era before computers existed.
While no one disputes the remarkable results that have been achieved,
communicating these results in a precise way to the uninitiated is virtually
impossible.  To describe these results, a terse informal language is used which
despite its elegance is very difficult to learn.  This informal language is not
imprecise, far from it, but rather it often has omitted detail
and symbols with hidden context that are
implicitly understood by an expert but few others.  Extremely complex technical
meanings are associated with innocent-sounding English words such as
``compact'' and ``measurable'' that barely hint at what is actually being
said.  Anyone who does not keep the precise technical meaning constantly in
mind is bound to fail, and acquiring the ability to do this can be achieved
only through much practice and hard work.  Only the few who complete the
painful learning experience can join the small in-group of pure
mathematicians.  The informal language effectively cuts off the true nature of
their knowledge from most everyone else.

Metamath\index{Metamath} makes abstract mathematics more concrete.  It allows
a computer to keep track of the complexity associated with each word or symbol
with absolute rigor.  You can explore this complexity at your leisure, to
whatever degree you desire.  Whether or not you believe that concepts such as
infinity actually ``exist'' outside of the mind, Metamath lets you get to the
foundation for what's really being said.

Metamath also enables completely rigorous and thorough proof verification.
Its language is simple enough so that you
don't have to rely on the authority of experts but can verify the results
yourself, step by step.  If you want to attempt to derive your own results,
Metamath will not let you make a mistake in reasoning.
Even professional mathematicians make mistakes; Metamath makes it possible
to thoroughly verify that proofs are correct.

Metamath\index{Metamath} is a computer language and an associated computer
program for archiving, verifying, and studying mathematical proofs at a very
detailed level.
The Metamath language
describes formal\index{formal system} mathematical
systems and expresses proofs of theorems in those systems.  Such a language
is called a metalanguage\index{metalanguage} by mathematicians.
The Metamath program is a computer program that verifies
proofs expressed in the Metamath language.
The Metamath program does not have the built-in
ability to make logical inferences; it just makes a series of symbol
substitutions according to instructions given to it in a proof
and verifies that the result matches the expected theorem.  It makes logical
inferences based only on rules of logic that are contained in a set of
axioms\index{axiom}, or first principles, that you provide to it as the
starting point for proofs.

The complete specification of the Metamath language is only four pages long
(Section~\ref{spec}, p.~\pageref{spec}).  Its simplicity may at first make you
wonder how it can do much of anything at all.  But in fact the kinds of
symbol manipulations it performs are the ones that are implicitly done in all
mathematical systems at the lowest level.  You can learn it relatively quickly
and have complete confidence in any mathematical proof that it verifies.  On
the other hand, it is powerful and general enough so that virtually any
mathematical theory, from the most basic to the deeply abstract, can be
described with it.

Although in principle Metamath can be used with any
kind of mathematics, it is best suited for abstract or ``pure'' mathematics
that is mostly concerned with theorems and their proofs, as opposed to the
kind of mathematics that deals with the practical manipulation of numbers.
Examples of branches of pure mathematics are logic\index{logic},\footnote{Logic
is the study of statements that are universally true regardless of the objects
being described by the statements.  An example is the statement, ``if $P$
implies $Q$, then either $P$ is false or $Q$ is true.''} set theory\index{set
theory},\footnote{Set theory is the study of general-purpose mathematical objects called
``sets,'' and from it essentially all of mathematics can be derived.  For
example, numbers can be defined as specific sets, and their properties
can be explored using the tools of set theory.} number theory\index{number
theory},\footnote{Number theory deals with the properties of positive and
negative integers (whole numbers).} group theory\index{group
theory},\footnote{Group theory studies the properties of mathematical objects
called groups that obey a simple set of axioms and have properties of symmetry
that make them useful in many other fields.} abstract algebra\index{abstract
algebra},\footnote{Abstract algebra includes group theory and also studies
groups with additional properties that qualify them as ``rings'' and
``fields.''  The set of real numbers is a familiar example of a field.},
analysis\index{analysis} \index{real and complex numbers}\footnote{Analysis is
the study of real and complex numbers.} and
topology\index{topology}.\footnote{One area studied by topology are properties
that remain unchanged when geometrical objects undergo stretching
deformations; for example a doughnut and a coffee cup each have one hole (the
cup's hole is in its handle) and are thus considered topologically
equivalent.  In general, though, topology is the study of abstract
mathematical objects that obey a certain (surprisingly simple) set of axioms.
See, for example, Munkres \cite{Munkres}\index{Munkres, James R.}.} Even in
physics, Metamath could be applied to certain branches that make use of
abstract mathematics, such as quantum logic\index{quantum logic} (used to study
aspects of quantum mechanics\index{quantum mechanics}).

On the other hand, Metamath\index{Metamath} is less suited to applications
that deal primarily with intensive numeric computations.  Metamath does not
have any built-in representation of numbers\index{Metamath!representation of
numbers}; instead, a specific string of symbols (digits) must be syntactically
constructed as part of any proof in which an ordinary number is used.  For
this reason, numbers in Metamath are best limited to specific constants that
arise during the course of a theorem or its proof.  Numbers are only a tiny
part of the world of abstract mathematics.  The exclusion of built-in numbers
was a conscious decision to help achieve Metamath's simplicity, and there are
other software tools if you have different mathematical needs.
If you wish to quickly solve algebraic problems, the computer algebra
programs\index{computer algebra system} {\sc
macsyma}\index{macsyma@{\sc macsyma}}, Mathematica\index{Mathematica}, and
Maple\index{Maple} are specifically suited to handling numbers and
algebra efficiently.
If you wish to simply calculate numeric or matrix expressions easily,
tools such as Octave\index{Octave} may be a better choice.

After learning Metamath's basic statement types, any
tech\-ni\-cal\-ly ori\-ent\-ed person, mathematician or not, can
immediately trace
any theorem proved in the language as far back as he or she wants, all the way
to the axioms on which the theorem is based.  This ability suggests a
non-traditional way of learning about pure mathematics.  Used in conjunction
with traditional methods, Metamath could make pure mathematics accessible to
people who are not sufficiently skilled to figure out the implicit detail in
ordinary textbook proofs.  Once you learn the axioms of a theory, you can have
complete confidence that everything you need to understand a proof you are
studying is all there, at your beck and call, allowing you to focus in on any
proof step you don't understand in as much depth as you need, without worrying
about getting stuck on a step you can't figure out.\footnote{On the other
hand, writing proofs in the Metamath language is challenging, requiring
a degree of rigor far in excess of that normally taught to students.  In a
classroom setting, I doubt that writing Metamath proofs would ever replace
traditional homework exercises involving informal proofs, because the time
needed to work out the details would not allow a course to
cover much material.  For students who have trouble grasping the implied rigor
in traditional material, writing a few simple proofs in the Metamath language
might help clarify fuzzy thought processes.  Although somewhat difficult at
first, it eventually becomes fun to do, like solving a puzzle, because of the
instant feedback provided by the computer.}

Metamath\index{Metamath} is probably unlike anything you have
encountered before.  In this first chapter we will look at the philosophy and
use of computers in mathematics in order to better understand the motivation
behind Metamath.  The material in this chapter is not required in order to use
Metamath.  You may skip it if you are impatient, but I hope you will find it
educational and enjoyable.  If you want to start experimenting with the
Metamath program right away, proceed directly to Chapter~\ref{using}
(p.~\pageref{using}).  To
learn the Metamath language, skim Chapter~\ref{using} then proceed to
Chapter~\ref{languagespec} (p.~\pageref{languagespec}).

\section{Mathematics as a Computer Language}

\begin{quote}
  {\em The study of mathematics is apt to commence in
dis\-ap\-point\-ment.\ldots \\
We are told that by its aid the stars are weighted
and the billions of molecules in a drop of water are counted.  Yet, like the
ghost of Hamlet's father, this great science eludes the efforts of our mental
weapons to grasp it.}
  \flushright\sc  Alfred North Whitehead\footnote{\cite{Whitehead}, ch.\ 1.}\\
\end{quote}\index{Whitehead, Alfred North}

\subsection{Is Mathematics ``User-Friendly''?}

Suppose you have no formal training in abstract mathematics.  But popular
books you've read offer tempting glimpses of this world filled with profound
ideas that have stirred the human spirit.  You are not satisfied with the
informal, watered-down descriptions you've read but feel it is important to
grasp the underlying mathematics itself to understand its true meaning. It's
not practical to go back to school to learn it, though; you don't want to
dedicate years of your life to it.  There are many important things in life,
and you have to set priorities for what's important to you.  What would happen
if you tried to pursue it on your own, in your spare time?

After all, you were able to learn a computer programming language such as
Pascal on your own without too much difficulty, even though you had no formal
training in computers.  You don't claim to be an expert in software design,
but you can write a passable program when necessary to suit your needs.  Even
more important, you know that you can look at anyone else's Pascal program, no
matter how complex, and with enough patience figure out exactly how it works,
even though you are not a specialist.  Pascal allows you do anything that a
computer can do, at least in principle.  Thus you know you have the ability,
in principle, to follow anything that a computer program can do:  you just
have to break it down into small enough pieces.

Here's an imaginary scenario of what might happen if you na\-ive\-ly a\-dopted
this same view of abstract mathematics and tried to pick it up on your own, in
a period of time comparable to, say, learning a computer programming
language.

\subsubsection{A Non-Mathematician's Quest for Truth}

\begin{quote}
  {\em \ldots my daughters have been studying (chemistry) for several
se\-mes\-ters, think they have learned differential and integral calculus in
school, and yet even today don't know why $x\cdot y=y\cdot x$ is true.}
  \flushright\sc  Edmund Landau\footnote{\cite{Landau}, p.~vi.}\\
\end{quote}\index{Landau, Edmund}

\begin{quote}
  {\em Minus times minus is plus,\\
The reason for this we need not discuss.}
  \flushright\sc W.\ H.\ Auden\footnote{As quoted in \cite{Guillen}, p.~64.}\\
\end{quote}\index{Auden, W.\ H.}\index{Guillen, Michael}

We'll suppose you are a technically oriented professional, perhaps an engineer, a
computer programmer, or a physicist, but probably not a mathematician.  You
consider yourself reasonably intelligent.  You did well in school, learning a
variety of methods and techniques in practical mathematics such as calculus and
differential equations.  But rarely did your courses get into anything
resembling modern abstract mathematics, and proofs were something that appeared
only occasionally in your textbooks, a kind of necessary evil that was
supposed to convince you of a certain key result.  Most of your
homework consisted of exercises that gave you practice in the techniques, and
you were hardly ever asked to come up with a proof of your own.

You find yourself curious about advanced, abstract mathematics.  You are
driven by an inner conviction that it is important to understand and
appreciate some of the most profound knowledge discovered by mankind.  But it
seems very hard to learn, something that only certain gifted longhairs can
access and understand.  You are frustrated that it seems forever cut off from
you.

Eventually your curiosity drives you to do something about it.
You set for yourself a goal of ``really'' understanding mathematics:  not just
how to manipulate equations in algebra or calculus according to cookbook
rules, but rather to gain a deep understanding of where those rules come from.
In fact, you're not thinking about this kind of ordinary mathematics at all,
but about a much more abstract, ethereal realm of pure mathematics, where
famous results such as G\"{o}del's incompleteness theorem\index{G\"{o}del's
incompleteness theorem} and Cantor's different kinds of infinities
reside.

You have probably read a number of popular books, with titles like {\em
Infinity and the Mind} \cite{Rucker}\index{Rucker, Rudy}, on topics such as
these.  You found them inspiring but at the same time somewhat
unsatisfactory.  They gave you a general idea of what these results are about,
but if someone asked you to prove them, you wouldn't have the faintest idea of
where to begin.   Sure, you could give the same overall outline that you
learned from the popular books; and in a general sort of way, you do have an
understanding.  But deep down inside, you know that there is a rigor that is
missing, that probably there are many subtle steps and pitfalls along the way,
and ultimately it seems you have to place your trust in the experts in the
field.  You don't like this; you want to be able to verify these results for
yourself.

So where do you go next?  As a first step, you decide to look up some of the
original papers on the theorems you are curious about, or better, obtain some
standard textbooks in the field.  You look up a theorem you want to
understand.  Sure enough, it's there, but it's expressed with strange
terms and odd symbols that mean absolutely nothing to you.  It might as well be written in
a foreign language you've never seen before, whose symbols are totally alien.
You look at the proof, and you haven't the foggiest notion what each step
means, much less how one step follows from another.  Well, obviously you have
a lot to learn if you want to understand this stuff.

You feel that you could probably understand it by
going back to college for another three to six years and getting a math
degree.  But that does not fit in with your career and the other things in
your life and would serve no practical purpose.  You decide to seek a quicker
path.  You figure you'll just trace your way back to the beginning, step by
step, as you would do with a computer program, until you understand it.  But
you quickly find that this is not possible, since you can't even understand
enough to know what you have to trace back to.

Maybe a different approach is in order---maybe you should start at the
beginning and work your way up.  First, you read the introduction to the book
to find out what the prerequisites are.  In a similar fashion, you trace your
way back through two or three more books, finally arriving at one that seems
to start at a beginning:  it lists the axioms of arithmetic.  ``Aha!'' you
naively think, ``This must be the starting point, the source of all mathematical
knowledge.'' Or at least the starting point for mathematics dealing with
numbers; you have to start somewhere and have no idea what the starting point
for other mathematics would be.  But the word ``axioms'' looks promising.  So
you eagerly read along and work through some elementary exercises at the
beginning of the book.  You feel vaguely bothered:  these
don't seem like axioms at all, at least not in the sense that you want to
think of axioms.  Axioms imply a starting point from which everything else can
be built up, according to precise rules specified in the axiom system.  Even
though you can understand the first few proofs in an informal way,
and are able to do some of the
exercises, it's hard to pin down precisely what the
rules are.   Sure, each step seems to follow logically from the others, but
exactly what does that mean?  Is the ``logic'' just a matter of common sense,
something vague that we all understand but can never quite state precisely?

You've spent a number of years, off and on, programming computers, and you
know that in the case of computer languages there is no question of what the
rules are---they are precise and crystal clear.  If you follow them, your
program will work, and if you don't, it won't.  No matter how complex a
program, it can always be broken down into simpler and simpler pieces, until
you can ultimately identify the bits that are moved around to perform a
specific function.  Some programs might require a lot of perseverance to
accomplish this, but if you focus on a specific portion of it, you don't even
necessarily have to know how the rest of it works. Shouldn't there be an
analogy in mathematics?

You decide to apply the ultimate test:  you ask yourself how a computer could
verify or ensure that the steps in these proofs follow from one another.
Certainly mathematics must be at least as precisely defined as a computer
language, if not more so; after all, computer science itself is based on it.
If you can get a computer to verify these proofs, then you should also be
able, in principle, to understand them yourself in a very crystal clear,
precise way.

You're in for a surprise:  you can conceive of no way to convert the
proofs, which are in English, to a form that the computer can understand.
The proofs are filled with phrases such as ``assume there exists a unique
$x$\ldots'' and ``given any $y$, let $z$ be the number such that\ldots''  This
isn't the kind of logic you are used to in computer programming, where
everything, even arithmetic, reduces to Boolean ones and zeroes if you care to
break it down sufficiently.  Even though you think you understand the proofs,
there seems to be some kind of higher reasoning involved rather than precise
rules that define how you manipulate the symbols in the axioms.  Whatever it
is, it just isn't obvious how you would express it to a computer, and the more
you think about it, the more puzzled and confused you get, to the point where
you even wonder whether {\em you} really understand it.  There's a lot more to
these axioms of arithmetic than meets the eye.

Nobody ever talked about this in school in your applied math and engineering
courses.  You just learned the rules they gave you, not quite understanding
how or why they worked, sometimes vaguely suspicious or uncertain of them, and
through homework problems and osmosis learned how to present solutions that
satisfied the instructor and earned you an ``A.''  Rarely did you actually
``prove'' anything in a rigorous way, and the math majors who did do stuff
like that seemed to be in a different world.

Of course, there are computer algebra programs that can do mathematics, and
rather impressively.  They can instantly solve the integrals that you
struggled with in freshman calculus, and do much, much more.  But when you
look at these programs, what you see is a big collection of algorithms and
techniques that evolved and were added to over time, along with some basic
software that manipulates symbols.  Each algorithm that is built in is the
result of someone's theorem whose proof is omitted; you just have to trust the
person who proved it and the person who programmed it in and hope there are no
bugs.\index{computer program bugs}  Somehow this doesn't seem to be the
essence of mathematics.  Although computer algebra systems can generate
theorems with amazing speed, they can't actually prove a single one of them.

After some puzzlement, you revisit some popular books on what mathematics is
all about.  Somewhere you read that all of mathematics is actually derived
from something called ``set theory.''  This is a little confusing, because
nowhere in the book that presented the axioms of arithmetic was there any
mention of set theory, or if there was, it seemed to be just a tool that helps
you describe things better---the set of even numbers, that sort of thing.  If
set theory is the basis for all mathematics, then why are additional axioms
needed for arithmetic?

Something is wrong but you're not sure what.  One of your friends is a pure
mathematician.  He knows he is unable to communicate to you what he does for a
living and seems to have little interest in trying.  You do know that for him,
proofs are what mathematics is all about. You ask him what a proof is, and he
essentially tells you that, while of course it's based on logic, really it's
something you learn by doing it over and over until you pick it up.  He refers
you to a book, {\em How to Read and Do Proofs} \cite{Solow}.\index{Solow,
Daniel}  Although this book helps you understand traditional informal proofs,
there is still something missing you can't seem to pin down yet.

You ask your friend how you would go about having a computer verify a proof.
At first he seems puzzled by the question; why would you want to do that?
Then he says it's not something that would make any sense to do, but he's
heard that you'd have to break the proof down into thousands or even millions
of individual steps to do such a thing, because the reasoning involved is at
such a high level of abstraction.  He says that maybe it's something you could
do up to a point, but the computer would be completely impractical once you
get into any meaningful mathematics.  There, the only way you can verify a
proof is by hand, and you can only acquire the ability to do this by
specializing in the field for a couple of years in grad school.  Anyway, he
thinks it all has to do with set theory, although he has never taken a formal
course in set theory but just learned what he needed as he went along.

You are intrigued and amazed.  Apparently a mathematician can grasp as a
single concept something that would take a computer a thousand or a million
steps to verify, and have complete confidence in it.  Each one of these
thousand or million steps must be absolutely correct, or else the whole proof
is meaningless.  If you added a million numbers by hand, would you trust the
result?  How do you really know that all these steps are correct, that there
isn't some subtle pitfall in one of these million steps, like a bug in a
computer program?\index{computer program bugs}  After all, you've read that
famous mathematicians have occasionally made mistakes, and you certainly know
you've made your share on your math homework problems in school.

You recall the analogy with a computer program.  Sure, you can understand what
a large computer program such as a word processor does, as a single high-level
concept or a small set of such concepts, but your ability to understand it in
no way ensures that the program is correct and doesn't have hidden bugs.  Even
if you wrote the program yourself you can't really know this; most large
programs that you've written have had bugs that crop up at some later date, no
matter how careful you tried to be while writing them.

OK, so now it seems the reason you can't figure out how to make a
computer verify proofs is because each step really corresponds to a
million small steps.  Well, you say, a computer can do a million
calculations in a second, so maybe it's still practical to do.  Now the
puzzle becomes how to figure out what the million steps are that each
English-language step corresponds to.  Your mathematician friend hasn't
a clue, but suggests that maybe you would find the answer by studying
set theory.  Actually, your friend thinks you're a little off the wall
for even wondering such a thing.  For him, this is not what mathematics
is all about.

The subject of set theory keeps popping up, so you decide it's
time to look it up.

You decide to start off on a careful footing, so you start reading a couple of
very elementary books on set theory.  A lot of it seems pretty obvious, like
intersections, subsets, and Venn diagrams.  You thumb through one of the
books; nowhere is anything about axioms mentioned. The other book relegates to
an appendix a brief discussion that mentions a set of axioms called
``Zermelo--Fraenkel set theory''\index{Zermelo--Fraenkel set theory} and states
them in English.  You look at them and have no idea what they really mean or
what you can do with them.  The comments in this appendix say that the purpose
of mentioning them is to expose you to the idea, but imply that they are not
necessary for basic understanding and that they are really the subject matter
of advanced treatments where fine points such as a certain paradox (Russell's
paradox\index{Russell's paradox}\footnote{Russell's paradox assumes that there
exists a set $S$ that is a collection of all sets that don't contain
themselves.  Now, either $S$ contains itself or it doesn't.  If it contains
itself, it contradicts its definition.  But if it doesn't contain itself, it
also contradicts its definition.  Russell's paradox is resolved in ZF set
theory by denying that such a set $S$ exists.}) are resolved.  Wait a
minute---shouldn't the axioms be a starting point, not an ending point?  If
there are paradoxes that arise without the axioms, how do you know you won't
stumble across one accidentally when using the informal approach?

And nowhere do these books describe how ``all of mathematics can be
derived from set theory'' which by now you've heard a few times.

You find a more advanced book on set theory.  This one actually lists the
axioms of ZF set theory in plain English on page one.  {\em Now} you think
your quest has ended and you've finally found the source of all mathematical
knowledge; you just have to understand what it means.  Here, in one place, is
the basis for all of mathematics!  You stare at the axioms in awe, puzzle over
them, memorize them, hoping that if you just meditate on them long enough they
will become clear.  Of course, you haven't the slightest idea how the rest of
mathematics is ``derived'' from them; in particular, if these are the axioms
of mathematics, then why do arithmetic, group theory, and so on need their own
axioms?

You start reading this advanced book carefully, pondering the meaning of every
word, because by now you're really determined to get to the bottom of this.
The first thing the book does is explain how the axioms came about, which was
to resolve Russell's paradox.\index{Russell's paradox}  In fact that seems to
be the main purpose of their existence; that they supposedly can be used to
derive all of mathematics seems irrelevant and is not even mentioned.  Well,
you go on.  You hope the book will explain to you clearly, step by step, how
to derive things from the axioms.  After all, this is the starting point of
mathematics, like a book that explains the basics of a computer programming
language.  But something is missing.  You find you can't even understand the
first proof or do the first exercise.  Symbols such as $\exists$ and $\forall$
permeate the page without any mention of where they came from or how to
manipulate them; the author assumes you are totally familiar with them and
doesn't even tell you what they mean.  By now you know that $\exists$ means
``there exists'' and $\forall$ means ``for all,'' but shouldn't the rules for
manipulating these symbols be part of the axioms?  You still have no idea
how you could even describe the axioms to a computer.

Certainly there is something much different here from the technical
literature you're used to reading.  A computer language manual almost
always explains very clearly what all the symbols mean, precisely what
they do, and the rules used for combining them, and you work your way up
from there.

After glancing at four or five other such books, you come to the realization
that there is another whole field of study that you need just to get to the
point at which you can understand the axioms of set theory.  The field is
called ``logic.''  In fact, some of the books did recommend it as a
prerequisite, but it just didn't sink in.  You assumed logic was, well, just
logic, something that a person with common sense intuitively understood.  Why
waste your time reading boring treatises on symbolic logic, the manipulation
of 1's and 0's that computers do, when you already know that?  But this is a
different kind of logic, quite alien to you.  The subject of {\sc nand} and
{\sc nor} gates is not even touched upon or in any case has to do with only a
very small part of this field.

So your quest continues.  Skimming through the first couple of introductory
books, you get a general idea of what logic is about and what quantifiers
(``for all,'' ``there exists'') mean, but you find their examples somewhat
trivial and mildly annoying (``all dogs are animals,'' ``some animals are
dogs,'' and such).  But all you want to know is what the rules are for
manipulating the symbols so you can apply them to set theory.  Some formulas
describing the relationships among quantifiers ($\exists$ and $\forall$) are
listed in tables, along with some verbal reasoning to justify them.
Presumably, if you want to find out if a formula is correct, you go through
this same kind of mental reasoning process, possibly using images of dogs and
animals. Intuitively, the formulas seem to make sense.  But when you ask
yourself, ``What are the rules I need to get a computer to figure out whether
this formula is correct?'', you still don't know.  Certainly you don't ask the
computer to imagine dogs and animals.

You look at some more advanced logic books.  Many of them have an introductory
chapter summarizing set theory, which turns out to be a prerequisite.  You
need logic to understand set theory, but it seems you also need set theory to
understand logic!  These books jump right into proving rather advanced
theorems about logic, without offering the faintest clue about where the logic
came from that allows them to prove these theorems.

Luckily, you come across an elementary book of logic that, halfway through,
after the usual truth tables and metaphors, presents in a clear, precise way
what you've been looking for all along: the axioms!  They're divided into
propositional calculus (also called sentential logic) and predicate calculus
(also called first-order logic),\index{first-order logic} with rules so simple
and crystal clear that now you can finally program a computer to understand
them.  Indeed, they're no harder than learning how to play a game of chess.
As far as what you seem to need is concerned, the whole book could have been
written in five pages!

{\em Now} you think you've found the ultimate source of mathematical
truth.  So---the axioms of mathematics consist of these axioms of logic,
together with the axioms of ZF set theory. (By now you've also been able to
figure out how to translate the ZF axioms from English into the
actual symbols of logic which you can now manipulate according to
precise, easy-to-understand rules.)

Of course, you still don't understand how ``all of mathematics can be
derived from set theory,'' but maybe this will reveal itself in due
course.

You eagerly set out to program the axioms and rules into a computer and start
to look at the theorems you will have to prove as the logic is developed.  All
sorts of important theorems start popping up:  the deduction
theorem,\index{deduction theorem} the substitution theorem,\index{substitution
theorem} the completeness theorem of propositional calculus,\index{first-order
logic!completeness} the completeness theorem of predicate calculus.  Uh-oh,
there seems to be trouble.  They all get harder and harder, and not one of
them can be derived with the axioms and rules of logic you've just been
handed.  Instead, they all require ``metalogic'' for their proofs, a kind of
mixture of logic and set theory that allows you to prove things {\em about}
the axioms and theorems of logic rather than {\em with} them.

You plow ahead anyway.  A month later, you've spent much of your
free time getting the computer to verify proofs in propositional calculus.
You've programmed in the axioms, but you've also had to program in the
deduction theorem, the substitution theorem, and the completeness theorem of
propositional calculus, which by now you've resigned yourself to treating as
rather complex additional axioms, since they can't be proved from the axioms
you were given.  You can now get the computer to verify and even generate
complete, rigorous, formal proofs\index{formal proof}.  Never mind that they
may have 100,000 steps---at least now you can have complete, absolute
confidence in them.  Unfortunately, the only theorems you have proved are
pretty trivial and you can easily verify them in a few minutes with truth
tables, if not by inspection.

It looks like your mathematician friend was right.  Getting the computer to do
serious mathematics with this kind of rigor seems almost hopeless.  Even
worse, it seems that the further along you get, the more ``axioms'' you have
to add, as each new theorem seems to involve additional ``metamathematical''
reasoning that hasn't been formalized, and none of it can be derived from the
axioms of logic.  Not only do the proofs keep growing exponentially as you get
further along, but the program to verify them keeps getting bigger and bigger
as you program in more ``metatheorems.''\index{metatheorem}\footnote{A
metatheorem is usually a statement that is too general to be directly provable
in a theory.  For example, ``if $n_1$, $n_2$, and $n_3$ are integers, then
$n_1+n_2+n_3$ is an integer'' is a theorem of number theory.  But ``for any
integer $k > 1$, if $n_1, \ldots, n_k$ are integers, then $n_1+\ldots +n_k$ is
an integer'' is a metatheorem, in other words a family of theorems, one for
every $k$.  The reason it is not a theorem is that the general sum $n_1+\ldots
+n_k$ (as a function of $k$) is not an operation that can be defined directly
in number theory.} The bugs\index{computer program bugs} that have cropped up
so far have already made you start to lose faith in the rigor you seem to have
achieved, and you know it's just going to get worse as your program gets larger.

\subsection{Mathematics and the Non-Specialist}

\begin{quote}
  {\em A real proof is not checkable by a machine, or even by any mathematician
not privy to the gestalt, the mode of thought of the particular field of
mathematics in which the proof is located.}
  \flushright\sc  Davis and Hersh\index{Davis, Phillip J.}
  \footnote{\cite{Davis}, p.~354.}\\
\end{quote}

The bulk of abstract or theoretical mathematics is ordinarily outside
the reach of anyone but a few specialists in each field who have completed
the necessary difficult internship in order to enter its coterie.  The
typical intelligent layperson has no reasonable hope of understanding much of
it, nor even the specialist mathematician of understanding other fields.  It
is like a foreign language that has no dictionary to look up the translation;
the only way you can learn it is by living in the country for a few years.  It
is argued that the effort involved in learning a specialty is a necessary
process for acquiring a deep understanding.  Of course, this is almost certainly
true if one is to make significant contributions to a field; in particular,
``doing'' proofs is probably the most important part of a mathematician's
training.  But is it also necessary to deny outsiders access to it?  Is it
necessary that abstract mathematics be so hard for a layperson to grasp?

A computer normally is of no help whatsoever.  Most published proofs are
actually just series of hints written in an informal style that requires
considerable knowledge of the field to understand.  These are the ``real
proofs'' referred to by Davis and Hersh.\index{informal proof}  There is an
implicit understanding that, in principle, such a proof could be converted to
a complete formal proof\index{formal proof}.  However, it is said that no one
would ever attempt such a conversion, even if they could, because that would
presumably require millions of steps (Section~\ref{dream}).  Unfortunately the
informal style automatically excludes the understanding of the proof
by anyone who hasn't gone through the necessary apprenticeship. The
best that the intelligent layperson can do is to read popular books about deep
and famous results; while this can be helpful, it can also be misleading, and
the lack of detail usually leaves the reader with no ability whatsoever to
explore any aspect of the field being described.

The statements of theorems often use sophisticated notation that makes them
inaccessible to the non-specialist.  For a non-specialist who wants to achieve
a deeper understanding of a proof, the process of tracing definitions and
lemmas back through their hierarchy\index{hierarchy} quickly becomes confusing
and discouraging.  Textbooks are usually written to train mathematicians or to
communicate to people who are already mathematicians, and large gaps in proofs
are often left as exercises to the reader who is left at an impasse if he or
she becomes stuck.

I believe that eventually computers will enable non-specialists and even
intelligent laypersons to follow almost any mathematical proof in any field.
Metamath is an attempt in that direction.  If all of mathematics were as
easily accessible as a computer programming language, I could envision
computer programmers and hobbyists who otherwise lack mathematical
sophistication exploring and being amazed by the world of theorems and proofs
in obscure specialties, perhaps even coming up with results of their own.  A
tremendous advantage would be that anyone could experiment with conjectures in
any field---the computer would offer instant feedback as to whether
an inference step was correct.

Mathematicians sometimes have to put up with the annoyance of
cranks\index{cranks} who lack a fundamental understanding of mathematics but
insist that their ``proofs'' of, say, Fermat's Last Theorem\index{Fermat's
Last Theorem} be taken seriously.  I think part of the problem is that these
people are misled by informal mathematical language, treating it as if they
were reading ordinary expository English and failing to appreciate the
implicit underlying rigor.  Such cranks are rare in the field of computers,
because computer languages are much more explicit, and ultimately the proof is
in whether a computer program works or not.  With easily accessible
computer-based abstract mathematics, a mathematician could say to a crank,
``don't bother me until you've demonstrated your claim on the computer!''

% 22-May-04 nm
% Attempt to move De Millo quote so it doesn't separate from attribution
% CHANGE THIS NUMBER (AND ELIMINATE IF POSSIBLE) WHEN ABOVE TEXT CHANGES
\vspace{-0.5em}

\subsection{An Impossible Dream?}\label{dream}

\begin{quote}
  {\em Even quite basic theorems would demand almost unbelievably vast
  books to display their proofs.}
    \flushright\sc  Robert E. Edwards\footnote{\cite{Edwards}, p.~68.}\\
\end{quote}\index{Edwards, Robert E.}

\begin{quote}
  {\em Oh, of course no one ever really {\em does} it.  It would take
  forever!  You just show that you could do it, that's sufficient.}
    \flushright\sc  ``The Ideal Mathematician''
    \index{Davis, Phillip J.}\footnote{\cite{Davis},
p.~40.}\\
\end{quote}

\begin{quote}
  {\em There is a theorem in the primitive notation of set theory that
  corresponds to the arithmetic theorem `$1000+2000=3000$'.  The formula
  would be forbiddingly long\ldots even if [one] knows the definitions
  and is asked to simplify the long formula according to them, chances are
  he will make errors and arrive at some incorrect result.}
    \flushright\sc  Hao Wang\footnote{\cite{Wang}, p.~140.}\\
\end{quote}\index{Wang, Hao}

% 22-May-04 nm
% Attempt to move De Millo quote so it doesn't separate from attribution
% CHANGE THIS NUMBER (AND ELIMINATE IF POSSIBLE) WHEN ABOVE TEXT CHANGES
\vspace{-0.5em}

\begin{quote}
  {\em The {\em Principia Mathematica} was the crowning achievement of the
  formalists.  It was also the deathblow of the formalist view.\ldots
  {[Rus\-sell]} failed, in three enormous volumes, to get beyond the elementary
  facts of arithmetic.  He showed what can be done in principle and what
  cannot be done in practice.  If the mathematical process were really
  one of strict, logical progression, we would still be counting our
  fingers.\ldots
  One theoretician estimates\ldots that a demonstration of one of
  Ramanujan's conjectures assuming set theory and elementary analysis would
  take about two thousand pages; the length of a deduction from first principles
  is nearly in\-con\-ceiv\-a\-ble\ldots The probabilists argue that\ldots any
  very long proof can at best be viewed as only probably correct\ldots}
  \flushright\sc Richard de Millo et. al.\footnote{\cite{deMillo}, pp.~269,
  271.}\\
\end{quote}\index{de Millo, Richard}

A number of writers have conveyed the impression that the kind of absolute
rigor provided by Metamath\index{Metamath} is an impossible dream, suggesting
that a complete, formal verification\index{formal proof} of a typical theorem
would take millions of steps in untold volumes of books.  Even if it could be
done, the thinking sometimes goes, all meaning would be lost in such a
monstrous, tedious verification.\index{informal proof}\index{proof length}

These writers assume, however, that in order to achieve the kind of complete
formal verification they desire one must break down a proof into individual
primitive steps that make direct reference to the axioms.  This is
not necessary.  There is no reason not to make use of previously proved
theorems rather than proving them over and over.

Just as important, definitions\index{definition} can be introduced along
the way, allowing very complex formulas to be represented with few
symbols.  Not doing this can lead to absurdly long formulas.  For
example, the mere statement of
G\"{o}del's incompleteness theorem\index{G\"{o}del's
incompleteness theorem}, which can be expressed with a small number of
defined symbols, would require about 20,000 primitive symbols to express
it.\index{Boolos, George S.}\footnote{George S.\ Boolos, lecture at
Massachusetts Institute of Technology, spring 1990.} An extreme example
is Bourbaki's\label{bourbaki} language for set theory, which requires
4,523,659,424,929 symbols plus 1,179,618,517,981 disambiguatory links
(lines connecting symbol pairs, usually drawn below or above the
formula) to express the number
``one'' \cite{Mathias}.\index{Mathias, Adrian R. D.}\index{Bourbaki,
Nicolas}
% http://www.dpmms.cam.ac.uk/~ardm/

A hierarchy\index{hierarchy} of theorems and definitions permits an
exponential growth in the formula sizes and primitive proof steps to be
described with only a linear growth in the number of symbols used.  Of course,
this is how ordinary informal mathematics is normally done anyway, but with
Metamath\index{Metamath} it can be done with absolute rigor and precision.

\subsection{Beauty}


\begin{quote}
  {\em No one shall be able to drive us from the paradise that Cantor has
created for us.}
   \flushright\sc  David Hilbert\footnote{As quoted in \cite{Moore}, p.~131.}\\
\end{quote}\index{Hilbert, David}

\needspace{3\baselineskip}
\begin{quote}
  {\em Mathematics possesses not only truth, but some supreme beauty ---a
  beauty cold and austere, like that of a sculpture.}
    \flushright\sc  Bertrand
    Russell\footnote{\cite{Russell}.}\\
\end{quote}\index{Russell, Bertrand}

\begin{quote}
  {\em Euclid alone has looked on Beauty bare.}
  \flushright\sc Edna Millay\footnote{As quoted in \cite{Davis}, p.~150.}\\
\end{quote}\index{Millay, Edna}

For most people, abstract mathematics is distant, strange, and
incomprehensible.  Many popular books have tried to convey some of the sense
of beauty in famous theorems.  But even an intelligent layperson is left with
only a general idea of what a theorem is about and is hardly given the tools
needed to make use of it.  Traditionally, it is only after years of arduous
study that one can grasp the concepts needed for deep understanding.
Metamath\index{Metamath} allows you to approach the proof of the theorem from
a quite different perspective, peeling apart the formulas and definitions
layer by layer until an entirely different kind of understanding is achieved.
Every step of the proof is there, pieced together with absolute precision and
instantly available for inspection through a microscope with a magnification
as powerful as you desire.

A proof in itself can be considered an object of beauty.  Constructing an
elegant proof is an art.  Once a famous theorem has been proved, often
considerable effort is made to find simpler and more easily understood
proofs.  Creating and communicating elegant proofs is a major concern of
mathematicians.  Metamath is one way of providing a common language for
archiving and preserving this information.

The length of a proof can, to a certain extent, be considered an
objective measure of its ``beauty,'' since shorter proofs are usually
considered more elegant.  In the set theory database
\texttt{set.mm}\index{set theory database (\texttt{set.mm})}%
\index{Metamath Proof Explorer}
provided with Metamath, one goal was to make all proofs as short as possible.

\needspace{4\baselineskip}
\subsection{Simplicity}

\begin{quote}
  {\em God made man simple; man's complex problems are of his own
  devising.}
    \flushright\sc Eccles. 7:29\footnote{Jerusalem Bible.}\\
\end{quote}\index{Bible}

\needspace{3\baselineskip}
\begin{quote}
  {\em God made integers, all else is the work of man.}
    \flushright\sc Leopold Kronecker\footnote{{\em Jahresbericht
	der Deutschen Mathematiker-Vereinigung }, vol. 2, p. 19.}\\
\end{quote}\index{Kronecker, Leopold}

\needspace{3\baselineskip}
\begin{quote}
  {\em For what is clear and easily comprehended attracts; the
  complicated repels.}
    \flushright\sc David Hilbert\footnote{As quoted in \cite{deMillo},
p.~273.}\\
\end{quote}\index{Hilbert, David}

The Metamath\index{Metamath} language is simple and Spartan.  Metamath treats
all mathematical expressions as simple sequences of symbols, devoid of meaning.
The higher-level or ``metamathematical'' notions underlying Metamath are about
as simple as they could possibly be.  Each individual step in a proof involves
a single basic concept, the substitution of an expression for a variable, so
that in principle almost anyone, whether mathematician or not, can
completely understand how it was arrived at.

In one of its most basic applications, Metamath\index{Metamath} can be used to
develop the foundations of mathematics\index{foundations of mathematics} from
the very beginning.  This is done in the set theory database that is provided
with the Metamath package and is the subject matter
of Chapter~\ref{fol}. Any language (a metalanguage\index{metalanguage})
used to describe mathematics (an object language\index{object language}) must
have a mathematical content of its own, but it is desirable to keep this
content down to a bare minimum, namely that needed to make use of the
inference rules specified by the axioms.  With any metalanguage there is a
``chicken and egg'' problem somewhat like circular reasoning:  you must assume
the validity of the mathematics of the metalanguage in order to prove the
validity of the mathematics of the object language.  The mathematical content
of Metamath itself is quite limited.  Like the rules of a game of chess, the
essential concepts are simple enough so that virtually anyone should be able to
understand them (although that in itself will not let you play like
a master).  The symbols that Metamath manipulates do not in themselves
have any intrinsic meaning.  Your interpretation of the axioms that you supply
to Metamath is what gives them meaning.  Metamath is an attempt to strip down
mathematical thought to its bare essence and show you exactly how the symbols
are manipulated.

Philosophers and logicians, with various motivations, have often thought it
important to study ``weak'' fragments of logic\index{weak logic}
\cite{Anderson}\index{Anderson, Alan Ross} \cite{MegillBunder}\index{Megill,
Norman}\index{Bunder, Martin}, other unconventional systems of logic (such as
``modal'' logic\index{modal logic} \cite[ch.\ 27]{Boolos}\index{Boolos, George
S.}), and quantum logic\index{quantum logic} in physics
\cite{Pavicic}\index{Pavi{\v{c}}i{\'{c}}, M.}.  Metamath\index{Metamath}
provides a framework in which such systems can be expressed, with an absolute
precision that makes all underlying metamathematical assumptions rigorous and
crystal clear.

Some schools of philosophical thought, for example
intuitionism\index{intuitionism} and constructivism\index{constructivism},
demand that the notions underlying any mathematical system be as simple and
concrete as possible.  Metamath should meet the requirements of these
philosophies.  Metamath must be taught the symbols, axioms\index{axiom}, and
rules\index{rule} for a specific theory, from the skeptical (such as
intuitionism\index{intuitionism}\footnote{Intuitionism does not accept the law
of excluded middle (``either something is true or it is not true'').  See
\cite[p.~xi]{Tymoczko}\index{Tymoczko, Thomas} for discussion and references
on this topic.  Consider the theorem, ``There exist irrational numbers $a$ and
$b$ such that $a^b$ is rational.''  An intuitionist would reject the following
proof:  If $\sqrt{2}^{\sqrt{2}}$ is rational, we are done.  Otherwise, let
$a=\sqrt{2}^{\sqrt{2}}$ and $b=\sqrt{2}$. Then $a^b=2$, which is rational.})
to the bold (such as the axiom of choice in set theory\footnote{The axiom of
choice\index{Axiom of Choice} asserts that given any collection of pairwise
disjoint nonempty sets, there exists a set that has exactly one element in
common with each set of the collection.  It is used to prove many important
theorems in standard mathematics.  Some philosophers object to it because it
asserts the existence of a set without specifying what the set contains
\cite[p.~154]{Enderton}\index{Enderton, Herbert B.}.  In one foundation for
mathematics due to Quine\index{Quine, Willard Van Orman}, that has not been
otherwise shown to be inconsistent, the axiom of choice turns out to be false
\cite[p.~23]{Curry}\index{Curry, Haskell B.}.  The \texttt{show
trace{\char`\_}back} command of the Metamath program allows you to find out
whether the axiom of choice, or any other axiom, was assumed by a
proof.}\index{\texttt{show trace{\char`\_}back} command}).

The simplicity of the Metamath language lets the algorithm (computer program)
that verifies the validity of a Metamath proof be straightforward and
robust.  You can have confidence that the theorems it verifies really can be
derived from your axioms.

\subsection{Rigor}

\begin{quote}
  {\em Rigor became a goal with the Greeks\ldots But the efforts to
  pursue rigor to the utmost have led to an impasse in which there is
  no longer any agreement on what it really means.  Mathematics remains
  alive and vital, but only on a pragmatic basis.}
    \flushright\sc  Morris Kline\footnote{\cite{Kline}, p.~1209.}\\
\end{quote}\index{Kline, Morris}

Kline refers to a much deeper kind of rigor than that which we will discuss in
this section.  G\"{o}del's incompleteness theorem\index{G\"{o}del's
incompleteness theorem} showed that it is impossible to achieve absolute rigor
in standard mathematics because we can never prove that mathematics is
consistent (free from contradictions).\index{consistent theory}  If
mathematics is consistent, we will never know it, but must rely on faith.  If
mathematics is inconsistent, the best we can hope for is that some clever
future mathematician will discover the inconsistency.  In this case, the
axioms would probably be revised slightly to eliminate the inconsistency, as
was done in the case of Russell's paradox,\index{Russell's paradox} but the
bulk of mathematics would probably not be affected by such a discovery.
Russell's paradox, for example, did not affect most of the remarkable results
achieved by 19th-century and earlier mathematicians.  It mainly invalidated
some of Gottlob Frege's\index{Frege, Gottlob} work on the foundations of
mathematics in the late 1800's; in fact Frege's work inspired Russell's
discovery.  Despite the paradox, Frege's work contains important concepts that
have significantly influenced modern logic.  Kline's {\em Mathematics, The
Loss of Certainty} \cite{Klinel}\index{Kline, Morris} has an interesting
discussion of this topic.

What {\em can} be achieved with absolute certainty\index{certainty} is the
knowledge that if we assume the axioms are consistent and true, then the
results derived from them are true.  Part of the beauty of mathematics is that
it is the one area of human endeavor where absolute certainty can be achieved
in this sense.  A mathematical truth will remain such for eternity.  However,
our actual knowledge of whether a particular statement is a mathematical truth
is only as certain as the correctness of the proof that establishes it.  If
the proof of a statement is questionable or vague, we can't have absolute
confidence in the truth that the statement claims.

Let us look at some traditional ways of expressing proofs.

Except in the field of formal logic\index{formal logic}, almost all
traditional proofs in mathematics are really not proofs at all, but rather
proof outlines or hints as to how to go about constructing the proof.  Many
gaps\index{gaps in proofs} are left for the reader to fill in. There are
several reasons for this.  First, it is usually assumed in mathematical
literature that the person reading the proof is a mathematician familiar with
the specialty being described, and that the missing steps are obvious to such
a reader or at least that the reader is capable of filling them in.  This
attitude is fine for professional mathematicians in the specialty, but
unfortunately it often has the drawback of cutting off the rest of the world,
including mathematicians in other specialties, from understanding the proof.
We discussed one possible resolution to this on p.~\pageref{envision}.
Second, it is often assumed that a complete formal proof\index{formal proof}
would require countless millions of symbols (Section~\ref{dream}). This might
be true if the proof were to be expressed directly in terms of the axioms of
logic and set theory,\index{set theory} but it is usually not true if we allow
ourselves a hierarchy\index{hierarchy} of definitions and theorems to build
upon, using a notation that allows us to introduce new symbols, definitions,
and theorems in a precisely specified way.

Even in formal logic,\index{formal logic} formal proofs\index{formal proof}
that are considered complete still contain hidden or implicit information.
For example, a ``proof'' is usually defined as a sequence of
wffs,\index{well-formed formula (wff)}\footnote{A {\em wff} or well-formed
formula is a mathematical expression (string of symbols) constructed according
to some precise rules.  A formal mathematical system\index{formal system}
contains (1) the rules for constructing syntactically correct
wffs,\index{syntax rules} (2) a list of starting wffs called
axioms,\index{axiom} and (3) one or more rules prescribing how to derive new
wffs, called theorems\index{theorem}, from the axioms or previously derived
theorems.  An example of such a system is contained in
Metamath's\index{Metamath} set theory database, which defines a formal
system\index{formal system} from which all of standard mathematics can be
derived.  Section~\ref{startf} steps you through a complete example of a formal
system, and you may want to skim it now if you are unfamiliar with the
concept.} each of which is an axiom or follows from a rule applied to previous
wffs in the sequence.  The implicit part of the proof is the algorithm by
which a sequence of symbols is verified to be a valid wff, given the
definition of a wff.  The algorithm in this case is rather simple, but for a
computer to verify the proof,\index{automated proof verification} it must have
the algorithm built into its verification program.\footnote{It is possible, of
course, to specify wff construction syntax outside of the program itself
with a suitable input language (the Metamath language being an example), but
some proof-verification or theorem-proving programs lack the ability to extend
wff syntax in such a fashion.} If one deals exclusively with axioms and
elementary wffs, it is straightforward to implement such an algorithm.  But as
more and more definitions are added to the theory in order to make the
expression of wffs more compact, the algorithm becomes more and more
complicated.  A computer program that implements the algorithm becomes larger
and harder to understand as each definition is introduced, and thus more prone
to bugs.\index{computer program bugs}  The larger the program, the
more suspicious the mathematician may be about
the validity of its algorithms.  This is especially true because
computer programs are inherently hard to follow to begin with, and few people
enjoy verifying them manually in detail.

Metamath\index{Metamath} takes a different approach.  Metamath's ``knowledge''
is limited to the ability to substitute variables for expressions, subject to
some simple constraints.  Once the basic algorithm of Metamath is assumed to
be debugged, and perhaps independently confirmed, it
can be trusted once and for all.  The information that Metamath needs to
``understand'' mathematics is contained entirely in the body of knowledge
presented to Metamath.  Any errors in reasoning can only be errors in the
axioms or definitions contained in this body of knowledge.  As a
``constructive'' language\index{constructive language} Metamath has no
conditional branches or loops like the ones that make computer programs hard
to decipher; instead, the language can only build new sequences of symbols
from earlier sequences  of symbols.

The simplicity of the rules that underlie Metamath not only makes Metamath
easy to learn but also gives Metamath a great deal of flexibility. For
example, Metamath is not limited to describing standard first-order
logic\index{first-order logic}; higher-order logics\index{higher-order logic}
and fragments of logic\index{weak logic} can be described just as easily.
Metamath gives you the freedom to define whatever wff notation you prefer; it
has no built-in conception of the syntax of a wff.\index{well-formed formula
(wff)}  With suitable axioms and definitions, Metamath can even describe and
prove things about itself.\index{Metamath!self-description}  (John
Harrison\index{Harrison, John} discusses the ``reflection''
principle\index{reflection principle} involved in self-descriptive systems in
\cite{Harrison}.)

The flexibility of Metamath requires that its proofs specify a lot of detail,
much more than in an ordinary ``formal'' proof.\index{formal proof}  For
example, in an ordinary formal proof, a single step consists of displaying the
wff that constitutes that step.  In order for a computer program to verify
that the step is acceptable, it first must verify that the symbol sequence
being displayed is an acceptable wff.\index{automated proof verification} Most
proof verifiers have at least basic wff syntax built into their programs.
Metamath has no hard-wired knowledge of what constitutes a wff built into it;
instead every wff must be explicitly constructed based on rules defining wffs
that are present in a database.  Thus a single step in an ordinary formal
proof may be correspond to many steps in a Metamath proof. Despite the larger
number of steps, though, this does not mean that a Metamath proof must be
significantly larger than an ordinary formal proof. The reason is that since
we have constructed the wff from scratch, we know what the wff is, so there is
no reason to display it.  We only need to refer to a sequence of statements
that construct it.  In a sense, the display of the wff in an ordinary formal
proof is an implicit proof of its own validity as a wff; Metamath just makes
the proof explicit. (Section~\ref{proof} describes Metamath's proof notation.)

\section{Computers and Mathematicians}

\begin{quote}
  {\em The computer is important, but not to mathematics.}
    \flushright\sc  Paul Halmos\footnote{As quoted in \cite{Albers}, p.~121.}\\
\end{quote}\index{Halmos, Paul}

Pure mathematicians have traditionally been indifferent to computers, even to
the point of disdain.\index{computers and pure mathematics}  Computer science
itself is sometimes considered to fall in the mundane realm of ``applied''
mathematics, perhaps essential for the real world but intellectually unexciting
to those who seek the deepest truths in mathematics.  Perhaps a reason for this
attitude towards computers is that there is little or no computer software that
meets their needs, and there may be a general feeling that such software could
not even exist.  On the one hand, there are the practical computer algebra
systems, which can perform amazing symbolic manipulations in algebra and
calculus,\index{computer algebra system} yet can't prove the simplest
existence theorem, if the idea of a proof is present at all.  On the other
hand, there are specialized automated theorem provers that technically speaking
may generate correct proofs.\index{automated theorem proving}  But sometimes
their specialized input notation may be cryptic and their output perceived to
be long, inelegant, incomprehensible proofs.    The output
may be viewed with suspicion, since the program that generates it tends to be
very large, and its size increases the potential for bugs\index{computer
program bugs}.  Such a proof may be considered trustworthy only if
independently verified and ``understood'' by a human, but no one wants to
waste their time on such a boring, unrewarding chore.



\needspace{4\baselineskip}
\subsection{Trusting the Computer}

\begin{quote}
  {\em \ldots I continue to find the quasi-empirical interpretation of
  computer proofs to be the more plausible.\ldots Since not
  everything that claims to be a computer proof can be
  accepted as valid, what are the mathematical criteria for acceptable
  computer proofs?}
    \flushright\sc  Thomas Tymoczko\footnote{\cite{Tymoczko}, p.~245.}\\
\end{quote}\index{Tymoczko, Thomas}

In some cases, computers have been essential tools for proving famous
theorems.  But if a proof is so long and obscure that it can be verified in a
practical way only with a computer, it is vaguely felt to be suspicious.  For
example, proving the famous four-color theorem\index{four-color
theorem}\index{proof length} (``a map needs no more than four colors to
prevent any two adjacent countries from having the same color'') can presently
only be done with the aid of a very complex computer program which originally
required 1200 hours of computer time. There has been considerable debate about
whether such a proof can be trusted and whether such a proof is ``real''
mathematics \cite{Swart}\index{Swart, E. R.}.\index{trusting computers}

However, under normal circumstances even a skeptical mathematician would have a
great deal of confidence in the result of multiplying two numbers on a pocket
calculator, even though the precise details of what goes on are hidden from its
user.  Even the verification on a supercomputer that a huge number is prime is
trusted, especially if there is independent verification; no one bothers to
debate the philosophical significance of its ``proof,'' even though the actual
proof would be so large that it would be completely impractical to ever write
it down on paper.  It seems that if the algorithm used by the computer is
simple enough to be readily understood, then the computer can be trusted.

Metamath\index{Metamath} adopts this philosophy.  The simplicity of its
language makes it easy to learn, and because of its simplicity one can have
essentially absolute confidence that a proof is correct. All axioms, rules, and
definitions are available for inspection at any time because they are defined
by the user; there are no hidden or built-in rules that may be prone to subtle
bugs\index{computer program bugs}.  The basic algorithm at the heart of
Metamath is simple and fixed, and it can be assumed to be bug-free and robust
with a degree of confidence approaching certainty.
Independently written implementations of the Metamath verifier
can reduce any residual doubt on the part of a skeptic even further;
there are now over a dozen such implementations, written by many people.

\subsection{Trusting the Mathematician}\label{trust}

\begin{quote}
  {\em There is no Algebraist nor Mathematician so expert in his science, as
  to place entire confidence in any truth immediately upon his discovery of it,
  or regard it as any thing, but a mere probability.  Every time he runs over
  his proofs, his confidence encreases; but still more by the approbation of
  his friends; and is rais'd to its utmost perfection by the universal assent
  and applauses of the learned world.}
  \flushright\sc David Hume\footnote{{\em A Treatise of Human Nature}, as
  quoted in \cite{deMillo}, p.~267.}\\
\end{quote}\index{Hume, David}

\begin{quote}
  {\em Stanislaw Ulam estimates that mathematicians publish 200,000 theorems
  every year.  A number of these are subsequently contradicted or otherwise
  disallowed, others are thrown into doubt, and most are ignored.}
  \flushright\sc Richard de Millo et. al.\footnote{\cite{deMillo}, p.~269.}\\
\end{quote}\index{Ulam, Stanislaw}

Whether or not the computer can be trusted, humans  of course will occasionally
err. Only the most memorable proofs get independently verified, and of these
only a handful of truly great ones achieve the status of being ``known''
mathematical truths that are used without giving a second thought to their
correctness.

There are many famous examples of incorrect theorems and proofs in
mathematical literature.\index{errors in proofs}

\begin{itemize}
\item There have been thousands of purported proofs of Fermat's Last
Theorem\index{Fermat's Last Theorem} (``no integer solutions exist to $x^n +
y^n = z^n$ for $n > 2$''), by amateurs, cranks, and well-regarded
mathematicians \cite[p.~5]{Stark}\index{Stark, Harold M}.  Fermat wrote a note
in his copy of Bachet's {\em Diophantus} that he found ``a truly marvelous
proof of this theorem but this margin is too narrow to contain it''
\cite[p.~507]{Kramer}.  A recent, much publicized proof by Yoichi
Miyaoka\index{Miyaoka, Yoichi} was shown to be incorrect ({\em Science News},
April 9, 1988, p.~230).  The theorem was finally proved by Andrew
Wiles\index{Wiles, Andrew} ({\em Science News}, July 3, 1993, p.~5), but it
initially had some gaps and took over a year after its announcement to be
checked thoroughly by experts.  On Oct. 25, 1994, Wiles announced that the last
gap found in his proof had been filled in.
  \item In 1882, M. Pasch discovered that an axiom was omitted from Euclid's
formulation of geometry\index{Euclidean geometry}; without it, the proofs of
important theorems of Euclid are not valid.  Pasch's axiom\index{Pasch's
axiom} states that a line that intersects one side of a triangle must also
intersect another side, provided that it does not touch any of the triangle's
vertices.  The omission of Pasch's axiom went unnoticed for 2000
years \cite[p.~160]{Davis}, in spite of (one presumes) the thousands of
students, instructors, and mathematicians who studied Euclid.
  \item The first published proof of the famous Schr\"{o}der--Bernstein
theorem\index{Schr\"{o}der--Bernstein theorem} in set theory was incorrect
\cite[p.~148]{Enderton}\index{Enderton, Herbert B.}.  This theorem states
that if there exists a 1-to-1 function\footnote{A {\em set}\index{set} is any
collection of objects. A {\em function}\index{function} or {\em
mapping}\index{mapping} is a rule that assigns to each element of one set
(called the function's {\em domain}\index{domain}) an element from another
set.} from set $A$ into set $B$ and vice-versa, then sets $A$ and $B$ have
a 1-to-1 correspondence.  Although it sounds simple and obvious,
the standard proof is quite long and complex.
  \item In the early 1900's, Hilbert\index{Hilbert, David} published a
purported proof of the continuum hypothesis\index{continuum hypothesis}, which
was eventually established as unprovable by Cohen\index{Cohen, Paul} in 1963
\cite[p.~166]{Enderton}.  The continuum hypothesis states that no
infinity\index{infinity} (``transfinite cardinal number'')\index{cardinal,
transfinite} exists whose size (or ``cardinality''\index{cardinality}) is
between the size of the set of integers and the size of the set of real
numbers.  This hypothesis originated with German mathematician Georg
Cantor\index{Cantor, Georg} in the late 1800's, and his inability to prove it
is said to have contributed to mental illness that afflicted him in his later
years.
  \item An incorrect proof of the four-color theorem\index{four-color theorem}
was published by Kempe\index{Kempe, A. B.} in 1879
\cite[p.~582]{Courant}\index{Courant, Richard}; it stood for 11 years before
its flaw was discovered.  This theorem states that any map can be colored
using only four colors, so that no two adjacent countries have the same
color.  In 1976 the theorem was finally proved by the famous computer-assisted
proof of Haken, Appel, and Koch \cite{Swart}\index{Appel, K.}\index{Haken,
W.}\index{Koch, K.}.  Or at least it seems that way.  Mathematician
H.~S.~M.~Coxeter\index{Coxeter, H. S. M.} has doubts \cite[p.~58]{Davis}:  ``I
have a feeling that that is an untidy kind of use of the computers, and the more
you correspond with Haken and Appel, the more shaky you seem to be.''
  \item Many false ``proofs'' of the Poincar\'{e}
conjecture\index{Poincar\'{e} conjecture} have been proposed over the years.
This conjecture states that any object that mathematically behaves like a
three-dimensional sphere is a three-dimensional sphere topologically,
regardless of how it is distorted.  In March 1986, mathematicians Colin
Rourke\index{Rourke, Colin} and Eduardo R\^{e}go\index{R\^{e}go, Eduardo}
caused  a stir in the mathematical community by announcing that they had found
a proof; in November of that year the proof was found to be false \cite[p.
218]{PetersonI}.  It was finally proved in 2003 by Grigory Perelman
\label{poincare}\index{Szpiro, George}\index{Perelman, Grigory}\cite{Szpiro}.
 \end{itemize}

Many counterexamples to ``theorems'' in recent mathematical
literature related to Clifford algebras\index{Clifford algebras}
 have been found by Pertti
Lounesto (who passed away in 2002).\index{Lounesto, Pertti}
See the web page \url{http://mathforum.org/library/view/4933.html}.
% http://users.tkk.fi/~ppuska/mirror/Lounesto/counterexamples.htm

One of the purposes of Metamath\index{Metamath} is to allow proofs to be
expressed with absolute precision.  Developing a proof in the Metamath
language can be challenging, because Metamath will not permit even the
tiniest mistake.\index{errors in proofs}  But once the proof is created, its
correctness can be trusted immediately, without having to depend on the
process of peer review for confirmation.

\section{The Use of Computers in Mathematics}

\subsection{Computer Algebra Systems}

For the most part, you will find that Metamath\index{Metamath} is not a
practical tool for manipulating numbers.  (Even proving that $2 + 2 = 4$, if
you start with set theory, can be quite complex!)  Several commercial
mathematics packages are quite good at arithmetic, algebra, and calculus, and
as practical tools they are invaluable.\index{computer algebra system} But
they have no notion of proof, and cannot understand statements starting with
``there exists such and such...''.

Software packages such as Mathematica \cite{Wolfram}\index{Mathematica} do not
concern themselves with proofs but instead work directly with known results.
These packages primarily emphasize heuristic rules such as the substitution of
equals for equals to achieve simpler expressions or expressions in a different
form.  Starting with a rich collection of built-in rules and algorithms, users
can add to the collection by means of a powerful programming language.
However, results such as, say, the existence of a certain abstract object
without displaying the actual object cannot be expressed (directly) in their
languages.  The idea of a proof from a small set of axioms is absent.  Instead
this software simply assumes that each fact or rule you add to the built-in
collection of algorithms is valid.  One way to view the software is as a large
collection of axioms from which the software, with certain goals, attempts to
derive new theorems, for example equating a complex expression with a simpler
equivalent. But the terms ``theorem''\index{theorem} and
``proof,''\index{proof} for example, are not even mentioned in the index of
the user's manual for Mathematica.\index{Mathematica and proofs}  What is also
unsatisfactory from a philosophical point of view is that there is no way to
ensure the validity of the results other than by trusting the writer of each
application module or tediously checking each module by hand, similar to
checking a computer program for bugs.\index{computer program
bugs}\footnote{Two examples illustrate why the knowledge database of computer
algebra systems should sometimes be regarded with a certain caution.  If you
ask Mathematica (version 3.0) to \texttt{Solve[x\^{ }n + y\^{ }n == z\^{ }n , n]}
it will respond with \texttt{\{\{n-\char`\>-2\}, \{n-\char`\>-1\},
\{n-\char`\>1\}, \{n-\char`\>2\}\}}. In other words, Mathematica seems to
``know'' that Fermat's Last Theorem\index{Fermat's Last Theorem} is true!  (At
the time this version of Mathematica was released this fact was unknown.)  If
you ask Maple\index{Maple} to \texttt{solve(x\^{ }2 = 2\^{ }x)} then
\texttt{simplify(\{"\})}, it returns the solution set \texttt{\{2, 4\}}, apparently
unaware that 0.7666647\ldots is also a solution.} While of course extremely
valuable in applied mathematics, computer algebra systems tend to be of little
interest to the theoretical mathematician except as aids for exploring certain
specific problems.

Because of possible bugs, trusting the output of a computer algebra system for
use as theorems in a proof-verifier would defeat the latter's goal of rigor.
On the other hand, a fact such that a certain relatively large number is
prime, while easy for a computer algebra system to derive, might have a long,
tedious proof that could overwhelm a proof-verifier. One approach for linking
computer algebra systems to a proof-verifier while retaining the advantages of
both is to add a hypothesis to each such theorem indicating its source.  For
example, a constant {\sc maple} could indicate the theorem came from the Maple
package, and would mean ``assuming Maple is consistent, then\ldots''  This and
many other topics concerning the formalization of mathematics are discussed in
John Harrison's\index{Harrison, John} very interesting
PhD thesis~\cite{Harrison-thesis}.

\subsection{Automated Theorem Provers}\label{theoremprovers}

A mathematical theory is ``decidable''\index{decidable theory} if a mechanical
method or algorithm exists that is guaranteed to determine whether or not a
particular formula is a theorem.  Among the few theories that are decidable is
elementary geometry,\index{Euclidean geometry} as was shown by a classic
result of logician Alfred Tarski\index{Tarski, Alfred} in 1948
\cite{Tarski}.\footnote{Tarski's result actually applies to a subset of the
geometry discussed in elementary textbooks.  This subset includes most of what
would be considered elementary geometry but it is not powerful enough to
express, among other things, the notions of the circumference and area of a
circle.  Extending the theory in a way that includes notions such as these
makes the theory undecidable, as was also shown by Tarski.  Tarski's algorithm
is far too inefficient to implement practically on a computer.  A practical
algorithm for proving a smaller subset of geometry theorems (those not
involving concepts of ``order'' or ``continuity'') was discovered by Chinese
mathematician Wu Wen-ts\"{u}n in 1977 \cite{Chou}\index{Chou,
Shang-Ching}.}\index{Wen-ts{\"{u}}n, Wu}  But most theories, including
elementary arithmetic, are undecidable.  This fact contributes to keeping
mathematics alive and well, since many mathematicians believe
that they will never be
replaced by computers (if they believe Roger Penrose's argument that a
computer can never replace the brain \cite{Penrose}\index{Penrose, Roger}).
In fact,  elementary geometry is often considered a ``dead'' field
for the simple reason that it is decidable.

On the other hand, the undecidability of a theory does not mean that one cannot
use a computer to search for proofs, providing one is willing to give up if a
proof is not found after a reasonable amount of time.  The field of automated
theorem proving\index{automated theorem proving} specializes in pursuing such
computer searches.  Among the more successful results to date are those based
on an algorithm known as Robinson's resolution principle
\cite{Robinson}\index{Robinson's resolution principle}.

Automated theorem provers can be excellent tools for those willing to learn
how to use them.  But they are not widely used in mainstream pure
mathematics, even though they could probably be useful in many areas.  There
are several reasons for this.  Probably most important, the main goal in pure
mathematics is to arrive at results that are considered to be deep or
important; proving them is essential but secondary.  Usually, an automated
theorem prover cannot assist in this main goal, and by the time the main goal
is achieved, the mathematician may have already figured out the proof as a
by-product.  There is also a notational problem.  Mathematicians are used to
using very compact syntax where one or two symbols (heavily dependent on
context) can represent very complex concepts; this is part of the
hierarchy\index{hierarchy} they have built up to tackle difficult problems.  A
theorem prover on the other hand might require that a theorem be expressed in
``first-order logic,''\index{first-order logic} which is the logic on which
most of mathematics is ultimately based but which is not ordinarily used
directly because expressions can become very long.  Some automated theorem
provers are experimental programs, limited in their use to very specialized
areas, and the goal of many is simply research into the nature of automated
theorem proving itself.  Finally, much research remains to be done to enable
them to prove very deep theorems.  One significant result was a
computer proof by Larry Wos\index{Wos, Larry} and colleagues that every Robbins
algebra\index{Robbins algebra} is a Boolean  algebra\index{Boolean algebra}
({\em New York Times}, Dec. 10, 1996).\footnote{In 1933, E.~V.\
Huntington\index{Huntington, E. V.}
presented the following axiom system for
Boolean algebra with a unary operation $n$ and a binary operation $+$:
\begin{center}
    $x + y = y + x$ \\
    $(x + y) + z = x + (y + z)$ \\
    $n(n(x) + y) + n(n(x) + n(y)) = x$
\end{center}
Herbert Robbins\index{Robbins, Herbert}, a student of Huntington, conjectured
that the last equation can be replaced with a simpler one:
\begin{center}
    $n(n(x + y) + n(x + n(y))) = x$
\end{center}
Robbins and Huntington could not find a proof.  The problem was
later studied unsuccessfully by Tarski\index{Tarski, Alfred} and his
students, and it remained an unsolved problem until a
computer found the proof in 1996.  For more information on
the Robbins algebra problem see \cite{Wos}.}

How does Metamath\index{Metamath} relate to automated theorem provers?  A
theorem prover is primarily concerned with one theorem at a time (perhaps
tapping into a small database of known theorems) whereas Metamath is more like
a theorem archiving system, storing both the theorem and its proof in a
database for access and verification.  Metamath is one answer to ``what do you
do with the output of a theorem prover?''  and could be viewed as the
next step in the process.  Automated theorem provers could be useful tools for
helping develop its database.
Note that very long, automatically
generated proofs can make your database fat and ugly and cause Metamath's proof
verification to take a long time to run.  Unless you have a particularly good
program that generates very concise proofs, it might be best to consider the
use of automatically generated proofs as a quick-and-dirty approach, to be
manually rewritten at some later date.

The program {\sc otter}\index{otter@{\sc otter}}\footnote{\url{http://www.cs.unm.edu/\~mccune/otter/}.}, later succeeded by
prover9\index{prover9}\footnote{\url{https://www.cs.unm.edu/~mccune/mace4/}.},
have been historically influential.
The E prover\index{E prover}\footnote{\url{https://github.com/eprover/eprover}.}
is a maintained automated theorem prover
for full first-order logic with equality.
There are many other automated theorem provers as well.

If you want to combine automated theorem provers with Metamath
consider investigating
the book {\em Automated Reasoning:  Introduction and Applications}
\cite{Wos}\index{Wos, Larry}.  This book discusses
how to use {\sc otter} in a way that can
not only able to generate
relatively efficient proofs, it can even be instructed to search for
shorter proofs.  The effective use of {\sc otter} (and similar tools)
does require a certain
amount of experience, skill, and patience.  The axiom system used in the
\texttt{set.mm}\index{set theory database (\texttt{set.mm})} set theory
database can be expressed to {\sc otter} using a method described in
\cite{Megill}.\index{Megill, Norman}\footnote{To use those axioms with
{\sc otter}, they must be restated in a way that eliminates the need for
``dummy variables.''\index{dummy variable!eliminating} See the Comment
on p.~\pageref{nodd}.} When successful, this method tends to generate
short and clever proofs, but my experiments with it indicate that the
method will find proofs within a reasonable time only for relatively
easy theorems.  It is still fun to experiment with.

Reference \cite{Bledsoe}\index{Bledsoe, W. W.} surveys a number of approaches
people have explored in the field of automated theorem proving\index{automated
theorem proving}.

\subsection{Interactive Theorem Provers}\label{interactivetheoremprovers}

Finding proofs completely automatically is difficult, so there
are some interactive theorem provers that allow a human to guide the
computer to find a proof.
Examples include
HOL Light\index{HOL light}%
\footnote{\url{https://www.cl.cam.ac.uk/~jrh13/hol-light/}.},
Isabelle\index{Isabelle}%
\footnote{\url{http://www.cl.cam.ac.uk/Research/HVG/Isabelle}.},
{\sc hol}\index{hol@{\sc hol}}%
\footnote{\url{https://hol-theorem-prover.org/}.},
and
Coq\index{Coq}\footnote{\url{https://coq.inria.fr/}.}.

A major difference between most of these tools and Metamath is that the
``proofs'' are actually programs that guide the program to find a proof,
and not the proof itself.
For example, an Isabelle/HOL proof might apply a step
\texttt{apply (blast dest: rearrange reduction)}. The \texttt{blast}
instruction applies
an automatic tableux prover and returns if it found a sequence of proof
steps that work... but the sequence is not considered part of the proof.

A good overview of
higher-level proof verification languages (such as {\sc lcf}\index{lcf@{\sc
lcf}} and {\sc hol}\index{hol@{\sc hol}})
is given in \cite{Harrison}.  All of these languages are fundamentally
different from Metamath in that much of the mathematical foundational
knowledge is embedded in the underlying proof-verification program, rather
than placed directly in the database that is being verified.
These can have a steep learning curve for those without a mathematical
background.  For example, one usually must have a fair understanding of
mathematical logic in order to follow their proofs.

\subsection{Proof Verifiers}\label{proofverifiers}

A proof verifier is a program that doesn't generate proofs but instead
verifies proofs that you give it.  Many proof verifiers have limited built-in
automated proof capabilities, such as figuring out simple logical inferences
(while still being guided by a person who provides the overall proof).  Metamath
has no built-in automated proof capability other than the limited
capability of its Proof Assistant.

Proof-verification languages are not used as frequently as they might be.
Pure mathematicians are more concerned with producing new results, and such
detail and rigor would interfere with that goal.  The use of computers in pure
mathematics is primarily focused on automated theorem provers (not verifiers),
again with the ultimate goal of aiding the creation of new mathematics.
Automated theorem provers are usually concerned with attacking one theorem at
time rather than making a large, organized database easily available to the
user.  Metamath is one way to help close this gap.

By itself Metamath is a mostly a proof verifier.
This does not mean that other approaches can't be used; the difference
is that in Metamath, the results of various provers must be recorded
step-by-step so that they can be verified.

Another proof-verification language is Mizar,\index{Mizar} which can display
its proofs in the informal language that mathematicians are accustomed to.
Information on the Mizar language is available at \url{http://mizar.org}.

For the working mathematician, Mizar is an excellent tool for rigorously
documenting proofs. Mizar typesets its proofs in the informal English used by
mathematicians (and, while fine for them, are just as inscrutable by
laypersons!). A price paid for Mizar is a relatively steep learning curve of a
couple of weeks.  Several mathematicians are actively formalizing different
areas of mathematics using Mizar and publishing the proofs in a dedicated
journal. Unfortunately the task of formalizing mathematics is still looked
down upon to a certain extent since it doesn't involve the creation of ``new''
mathematics.

The closest system to Metamath is
the {\em Ghilbert}\index{Ghilbert} proof language (\url{http://ghilbert.org})
system developed by
Raph Levien\index{Levien, Raph}.
Ghilbert is a formal proof checker heavily inspired by Metamath.
Ghilbert statements are s-expressions (a la Lisp), which is easy
for computers to parse but many people find them hard to read.
There are a number of differences in their specific constructs, but
there is at least one tool to translate some Metamath materials into Ghilbert.
As of 2019 the Ghilbert community is smaller and less active than the
Metamath community.
That said, the Metamath and Ghilbert communities overlap, and fruitful
conversations between them have occurred many times over the years.

\subsection{Creating a Database of Formalized Mathematics}\label{mathdatabase}

Besides Metamath, there are several other ongoing projects with the goal of
formalizing mathematics into computer-verifiable databases.
Understanding some history will help.

The {\sc qed}\index{qed project@{\sc qed} project}%
\footnote{\url{http://www-unix.mcs.anl.gov/qed}.}
project arose in 1993 and its goals were outlined in the
{\sc qed} manifesto.
The {\sc qed} manifesto was
a proposal for a computer-based database of all mathematical knowledge,
strictly formalized and with all proofs having been checked automatically.
The project had a conference in 1994 and another in 1995;
there was also a ``twenty years of the {\sc qed} manifesto'' workshop
in 2014.
Its ideals are regularly reraised.

In a 2007 paper, Freek Wiedijk identified two reasons
for the failure of the {\sc qed} project as originally envisioned:%
\cite{Wiedijk-revisited}\index{Wiedijk, Freek}

\begin{itemize}
\item Very few people are working on formalization of mathematics. There is no compelling application for fully mechanized mathematics.
\item Formalized mathematics does not yet resemble traditional mathematics. This is partly due to the complexity of mathematical notation, and partly to the limitations of existing theorem provers and proof assistants.
\end{itemize}

But this did not end the dream of
formalizing mathematics into computer-verifiable databases.
The problems that led to the {\sc qed} manifesto are still with us,
even though the challenges were harder than originally considered.
What has happened instead is that various independent projects have
worked towards formalizing mathematics into computer-verifiable databases,
each simultaneously competing and cooperating with each other.

A concrete way to see this is
Freek Wiedijk's ``Formalizing 100 Theorems'' list%
\footnote{\url{http://www.cs.ru.nl/\%7Efreek/100/}.}
which shows the progress different systems have made on a challenge list
of 100 mathematical theorems.%
\footnote{ This is not the only list of ``interesting'' theorems.
Another interesting list was posted by Oliver Knill's list
\cite{Knill}\index{Knill, Oliver}.}
The top systems as of February 2019
(in order of the number of challenges completed) are
HOL Light, Isabelle, Metamath, Coq, and Mizar.

The Metamath 100%
\footnote{\url{http://us.metamath.org/mm\_100.html}}
page (maintained by David A. Wheeler\index{Wheeler, David A.})
shows the progress of Metamath (specifically its \texttt{set.mm} database)
against this challenge list maintained by Freek Wiedijk.
The Metamath \texttt{set.mm} database
has made a lot of progress over the years,
in part because working to prove those challenge theorems required
defining various terms and proving their properties as a prerequisite.
Here are just a few of the many statements that have been
formally proven with Metamath:

% The entries of this cause the narrow display to break poorly,
% since the short amount of text means LaTeX doesn't get a lot to work with
% and the itemize format gives it even *less* margin than usual.
% No one will mind if we make just this list flushleft, since this list
% will be internally consistent.
\begin{flushleft}
\begin{itemize}
\item 1. The Irrationality of the Square Root of 2
  (\texttt{sqr2irr}, by Norman Megill, 2001-08-20)
\item 2. The Fundamental Theorem of Algebra
  (\texttt{fta}, by Mario Carneiro, 2014-09-15)
\item 22. The Non-Denumerability of the Continuum
  (\texttt{ruc}, by Norman Megill, 2004-08-13)
\item 54. The Konigsberg Bridge Problem
  (\texttt{konigsberg}, by Mario Carneiro, 2015-04-16)
\item 83. The Friendship Theorem
  (\texttt{friendship}, by Alexander W. van der Vekens, 2018-10-09)
\end{itemize}
\end{flushleft}

We thank all of those who have developed at least one of the Metamath 100
proofs, and we particularly thank
Mario Carneiro\index{Carneiro, Mario}
who has contributed the most Metamath 100 proofs as of 2019.
The Metamath 100 page shows the list of all people who have contributed a
proof, and links to graphs and charts showing progress over time.
We encourage others to work on proving theorems not yet proven in Metamath,
since doing so improves the work as a whole.

Each of the math formalization systems (including Metamath)
has different strengths and weaknesses, depending on what you value.
Key aspects that differentiate Metamath from the other top systems are:

\begin{itemize}
\item Metamath is not tied to any particular set of axioms.
\item Metamath can show every step of every proof, no exceptions.
  Most other provers only assert that a proof can be found, and do not
  show every step. This also makes verification fast, because
  the system does not need to rediscover proof details.
\item The Metamath verifier has been re-implemented in many different
  programming languages, so verification can be done by multiple
  implementations.  In particular, the
  \texttt{set.mm}\index{set theory database (\texttt{set.mm})}%
  \index{Metamath Proof Explorer} database is verified by
  four different verifiers
  written in four different languages by four different authors.
  This greatly reduces the risk of accepting an invalid
  proof due to an error in the verifier.
\item Proofs stay proven.  In some systems, changes to the system's
  syntax or how a tactic works causes proofs to fail in later versions,
  causing older work to become essentially lost.
  Metamath's language is
  extremely small and fixed, so once a proof is added to a database,
  the database can be rechecked with later versions of the Metamath program
  and with other verifiers of Metamath databases.
  If an axiom or key definition needs to be changed, it is easy to
  manipulate the database as a whole to handle the change
  without touching the underlying verifier.
  Since re-verification of an entire database takes seconds, there
  is never a reason to delay complete verification.
  This aspect is especially compelling if your
  goal is to have a long-term database of proofs.
\item Licensing is generous.  The main Metamath databases are released to
  the public domain, and the main Metamath program is open source software
  under a standard, widely-used license.
\item Substitutions are easy to understand, even by those who are not
  professional mathematicians.
\end{itemize}

Of course, other systems may have advantages over Metamath
that are more compelling, depending on what you value.
In any case, we hope this helps you understand Metamath
within a wider context.

\subsection{In Summary}\label{computers-summary}

To summarize our discussions of computers and mathematics, computer algebra
systems can be viewed as theorem generators focusing on a narrow realm of
mathematics (numbers and their properties), automated theorem provers as proof
generators for specific theorems in a much broader realm covered by a built-in
formal system such as first-order logic, interactive theorem
provers require human guidance, proof verifiers verify proofs but
historically they have been
restricted to first-order logic.
Metamath, in contrast,
is a proof verifier and documenter whose realm is essentially unlimited.

\section{Mathematics and Metamath}

\subsection{Standard Mathematics}

There are a number of ways that Metamath\index{Metamath} can be used with
standard mathematics.  The most satisfying way philosophically is to start at
the very beginning, and develop the desired mathematics from the axioms of
logic and set theory.\index{set theory}  This is the approach taken in the
\texttt{set.mm}\index{set theory database (\texttt{set.mm})}%
\index{Metamath Proof Explorer}
database (also known as the Metamath Proof Explorer).
Among other things, this database builds up to the
axioms of real and complex numbers\index{analysis}\index{real and complex
numbers} (see Section~\ref{real}), and a standard development of analysis, for
example, could start at that point, using it as a basis.   Besides this
philosophical advantage, there are practical advantages to having all of the
tools of set theory available in the supporting infrastructure.

On the other hand, you may wish to start with the standard axioms of a
mathematical theory without going through the set theoretical proofs of those
axioms.  You will need mathematical logic to make inferences, but if you wish
you can simply introduce theorems\index{theorem} of logic as
``axioms''\index{axiom} wherever you need them, with the implicit assumption
that in principle they can be proved, if they are obvious to you.  If you
choose this approach, you will probably want to review the notation used in
\texttt{set.mm}\index{set theory database (\texttt{set.mm})} so that your own
notation will be consistent with it.

\subsection{Other Formal Systems}
\index{formal system}

Unlike some programs, Metamath\index{Metamath} is not limited to any specific
area of mathematics, nor committed to any particular mathematical philosophy
such as classical logic versus intuitionism, nor limited, say, to expressions
in first-order logic.  Although the database \texttt{set.mm}
describes standard logic and set theory, Meta\-math
is actually a general-purpose language for describing a wide variety of formal
systems.\index{formal system}  Non-standard systems such as modal
logic,\index{modal logic} intuitionist logic\index{intuitionism}, higher-order
logic\index{higher-order logic}, quantum logic\index{quantum logic}, and
category theory\index{category theory} can all be described with the Metamath
language.  You define the symbols you prefer and tell Metamath the axioms and
rules you want to start from, and Metamath will verify any inferences you make
from those axioms and rules.  A simple example of a non-standard formal system
is Hofstadter's\index{Hofstadter, Douglas R.} MIU system,\index{MIU-system}
whose Metamath description is presented in Appendix~\ref{MIU}.

This is not hypothetical.
The largest Metamath database is
\texttt{set.mm}\index{set theory database (\texttt{set.mm}}%
\index{Metamath Proof Explorer}), aka the Metamath Proof Explorer,
which uses the most common axioms for mathematical foundations
(specifically classical logic combined with Zermelo--Fraenkel
set theory\index{Zermelo--Fraenkel set theory} with the Axiom of Choice).
But other Metamath databases are available:

\begin{itemize}
\item The database
  \texttt{iset.mm}\index{intuitionistic logic database (\texttt{iset.mm})},
  aka the
  Intuitionistic Logic Explorer\index{Intuitionistic Logic Explorer},
  uses intuitionistic logic (a constructivist point of view)
  instead of classical logic.
\item The database
  \texttt{nf.mm}\index{New Foundations database (\texttt{nf.mm})},
  aka the
  New Foundations Explorer\index{New Foundations Explorer},
  constructs mathematics from scratch,
  starting from Quine's New Foundations (NF) set theory axioms.
\item The database
  \texttt{hol.mm}\index{Higher-order Logic database (\texttt{hol.mm})},
  aka the
  Higher-Order Logic (HOL) Explorer\index{Higher-Order Logic (HOL) Explorer},
  starts with HOL (also called simple type theory) and derives
  equivalents to ZFC axioms, connecting the two approaches.
\end{itemize}

Since the days of David Hilbert,\index{Hilbert, David} mathematicians have
been concerned with the fact that the metalanguage\index{metalanguage} used to
describe mathematics may be stronger than the mathematics being described.
Metamath\index{Metamath}'s underlying finitary\index{finitary proof},
constructive nature provides a good philosophical basis for studying even the
weakest logics.\index{weak logic}

The usual treatment of many non-standard formal systems\index{formal
system} uses model theory\index{model theory} or proof theory\index{proof
theory} to describe these systems; these theories, in turn, are based on
standard set theory.  In other words, a non-standard formal system is defined
as a set with certain properties, and standard set theory is used to derive
additional properties of this set.  The standard set theory database provided
with Metamath can be used for this purpose, and when used this way
the development of a special
axiom system for the non-standard formal system becomes unnecessary.  The
model- or proof-theoretic approach often allows you to prove much deeper
results with less effort.

Metamath supports both approaches.  You can define the non-standard
formal system directly, or define the non-standard formal system as
a set with certain properties, whichever you find most helpful.

%\section{Additional Remarks}

\subsection{Metamath and Its Philosophy}

Closely related to Metamath\index{Metamath} is a philosophy or way of looking
at mathematics. This philosophy is related to the formalist
philosophy\index{formalism} of Hilbert\index{Hilbert, David} and his followers
\cite[pp.~1203--1208]{Kline}\index{Kline, Morris}
\cite[p.~6]{Behnke}\index{Behnke, H.}. In this philosophy, mathematics is
viewed as nothing more than a set of rules that manipulate symbols, together
with the consequences of those rules.  While the mathematics being described
may be complex, the rules used to describe it (the
``metamathematics''\index{metamathematics}) should be as simple as possible.
In particular, proofs should be restricted to dealing with concrete objects
(the symbols we write on paper rather than the abstract concepts they
represent) in a constructive manner; these are called ``finitary''
proofs\index{finitary proof} \cite[pp.~2--3]{Shoenfield}\index{Shoenfield,
Joseph R.}.

Whether or not you find Metamath interesting or useful will in part depend on
the appeal you find in its philosophy, and this appeal will probably depend on
your particular goals with respect to mathematics.  For example, if you are a
pure mathematician at the forefront of discovering new mathematical knowledge,
you will probably find that the rigid formality of Metamath stifles your
creativity.  On the other hand, we would argue that once this knowledge is
discovered, there are advantages to documenting it in a standard format that
will make it accessible to others.  Sixty years from now, your field may be
dormant, and as Davis and Hersh put it, your ``writings would become less
translatable than those of the Maya'' \cite[p.~37]{Davis}\index{Davis, Phillip
J.}.


\subsection{A History of the Approach Behind Metamath}

Probably the one work that has had the most motivating influence on
Metamath\index{Metamath} is Whitehead and Russell's monumental {\em Principia
Mathematica} \cite{PM}\index{Whitehead, Alfred North}\index{Russell,
Bertrand}\index{principia mathematica@{\em Principia Mathematica}}, whose aim
was to deduce all of mathematics from a small number of primitive ideas, in a
very explicit way that in principle anyone could understand and follow.  While
this work was tremendously influential in its time, from a modern perspective
it suffers from several drawbacks.  Both its notation and its underlying
axioms are now considered dated and are no longer used.  From our point of
view, its development is not really as accessible as we would like to see; for
practical reasons, proofs become more and more sketchy as its mathematics
progresses, and working them out in fine detail requires a degree of
mathematical skill and patience that many people don't have.  There are
numerous small errors, which is understandable given the tedious, technical
nature of the proofs and the lack of a computer to verify the details.
However, even today {\em Principia Mathematica} stands out as the work closest
in spirit to Metamath.  It remains a mind-boggling work, and one can't help
but be amazed at seeing ``$1+1=2$'' finally appear on page 83 of Volume II
(Theorem *110.643).

The origin of the proof notation used by Metamath dates back to the 1950's,
when the logician C.~A.~Meredith expressed his proofs in a compact notation
called ``condensed detachment''\index{condensed detachment}
\cite{Hindley}\index{Hindley, J. Roger} \cite{Kalman}\index{Kalman, J. A.}
\cite{Meredith}\index{Meredith, C. A.} \cite{Peterson}\index{Peterson, Jeremy
George}.  This notation allows proofs to be communicated unambiguously by
merely referencing the axiom\index{axiom}, rule\index{rule}, or
theorem\index{theorem} used at each step, without explicitly indicating the
substitutions\index{substitution!variable}\index{variable substitution} that
have to be made to the variables in that axiom, rule, or theorem.  Ordinarily,
condensed detachment is more or less limited to propositional
calculus\index{propositional calculus}.  The concept has been extended to
first-order logic\index{first-order logic} in \cite{Megill}\index{Megill,
Norman}, making it is easy to write a small computer program to verify proofs
of simple first-order logic theorems.\index{condensed detachment!and
first-order logic}

A key concept behind the notation of condensed detachment is called
``unification,''\index{unification} which is an algorithm for determining what
substitutions\index{substitution!variable}\index{variable substitution} to
variables have to be made to make two expressions match each other.
Unification was first precisely defined by the logician J.~A.~Robinson, who
used it in the development of a powerful
theorem-proving technique called the ``resolution principle''
\cite{Robinson}\index{Robinson's resolution principle}. Metamath does not make
use of the resolution principle, which is intended for systems of first-order
logic.\index{first-order logic}  Metamath's use is not restricted to
first-order logic, and as we have mentioned it does not automatically discover
proofs.  However, unification is a key idea behind Metamath's proof
notation, and Metamath makes use of a very simple version of it
(Section~\ref{unify}).

\subsection{Metamath and First-Order Logic}

First-order logic\index{first-order logic} is the supporting structure
for standard mathematics.  On top of it is set theory, which contains
the axioms from which virtually all of mathematics can be derived---a
remarkable fact.\index{category
theory}\index{cardinal, inaccessible}\label{categoryth}\footnote{An exception seems
to be category theory.  There are several schools of thought on whether
category theory is derivable from set theory.  At a minimum, it appears
that an additional axiom is needed that asserts the existence of an
``inaccessible cardinal'' (a type of infinity so large that standard set
theory can't prove or deny that it exists).
%
%%%% (I took this out that was in previous editions:)
% But it is also argued that not just one but a ``proper class'' of them
% is needed, and the existence of proper classes is impossible in standard
% set theory.  (A proper class is a collection of sets so huge that no set
% can contain it as an element.  Proper classes can lead to
% inconsistencies such as ``Russell's paradox.''  The axioms of standard
% set theory are devised so as to deny the existence of proper classes.)
%
For more information, see
\cite[pp.~328--331]{Herrlich}\index{Herrlich, Horst} and
\cite{Blass}\index{Blass, Andrea}.}

One of the things that makes Metamath\index{Metamath} more practical for
first-order theories is a set of axioms for first-order logic designed
specifically with Metamath's approach in mind.  These are included in
the database \texttt{set.mm}\index{set theory database (\texttt{set.mm})}.
See Chapter~\ref{fol} for a detailed
description; the axioms are shown in Section~\ref{metaaxioms}.  While
logically equivalent to standard axiom systems, our axiom system breaks
up the standard axioms into smaller pieces such that from them, you can
directly derive what in other systems can only be derived as higher-level
``metatheorems.''\index{metatheorem}  In other words, it is more powerful than
the standard axioms from a metalogical point of view.  A rigorous
justification for this system and its ``metalogical
completeness''\index{metalogical completeness} is found in
\cite{Megill}\index{Megill, Norman}.  The system is closely related to a
system developed by Monk\index{Monk, J. Donald} and Tarski\index{Tarski,
Alfred} in 1965 \cite{Monks}.

For example, the formula $\exists x \, x = y $ (given $y$, there exists some
$x$ equal to it) is a theorem of logic,\footnote{Specifically, it is a theorem
of those systems of logic that assume non-empty domains.  It is not a theorem
of more general systems that include the empty domain\index{empty domain}, in
which nothing exists, period!  Such systems are called ``free
logics.''\index{free logic} For a discussion of these systems, see
\cite{Leblanc}\index{Leblanc, Hugues}.  Since our use for logic is as a basis
for set theory, which has a non-empty domain, it is more convenient (and more
traditional) to use a less general system.  An interesting curiosity is that,
using a free logic as a basis for Zermelo--Fraenkel set
theory\index{Zermelo--Fraenkel set theory} (with the redundant Axiom of the
Null Set omitted),\index{Axiom of the Null Set} we cannot even prove the
existence of a single set without assuming the axiom of infinity!\index{Axiom
of Infinity}} whether or not $x$ and $y$ are distinct variables\index{distinct
variables}.  In many systems of logic, we would have to prove two theorems to
arrive at this result.  First we would prove ``$\exists x \, x = x $,'' then
we would separately prove ``$\exists x \, x = y $, where $x$ and $y$ are
distinct variables.''  We would then combine these two special cases ``outside
of the system'' (i.e.\ in our heads) to be able to claim, ``$\exists x \, x =
y $, regardless of whether $x$ and $y$ are distinct.''  In other words, the
combination of the two special cases is a
metatheorem.  In the system of logic
used in Metamath's set theory\index{set theory database (\texttt{set.mm})}
database, the axioms of logic are broken down into small pieces that allow
them to be reassembled in such a way that theorems such as these can be proved
directly.

Breaking down the axioms in this way makes them look peculiar and not very
intuitive at first, but rest assured that they are correct and complete.  Their
correctness is ensured because they are theorem schemes of standard first-order
logic (which you can easily verify if you are a logician).  Their completeness
follows from the fact that we explicitly derive the standard axioms of
first-order logic as theorems.  Deriving the standard axioms is somewhat
tricky, but once we're there, we have at our disposal a system that is less
awkward to work with in formal proofs\index{formal proof}.  In technical terms
that logicians understand, we eliminate the cumbersome concepts of ``free
variable,''\index{free variable} ``bound variable,''\index{bound variable} and
``proper substitution''\index{proper substitution}\index{substitution!proper}
as primitive notions.  These concepts are present in our system but are
defined in terms of concepts expressed by the axioms and can be eliminated in
principle.  In standard systems, these concepts are really like additional,
implicit axioms\index{implicit axiom} that are somewhat complex and cannot be
eliminated.

The traditional approach to logic, wherein free variables and proper
substitution is defined, is also possible to do directly in the Metamath
language.  However, the notation tends to become awkward, and there are
disadvantages:  for example, extending the definition of a wff with a
definition is awkward, because the free variable and proper substitution
concepts have to have their definitions also extended.  Our choice of
axioms for \texttt{set.mm} is to a certain extent a matter of style, in
an attempt to achieve overall simplicity, but you should also be aware
that the traditional approach is possible as well if you should choose
it.

\chapter{Using the Metamath Program}
\label{using}

\section{Installation}

The way that you install Metamath\index{Metamath!installation} on your
computer system will vary for different computers.  Current
instructions are provided with the Metamath program download at
\url{http://metamath.org}.  In general, the installation is simple.
There is one file containing the Metamath program itself.  This file is
usually called \texttt{metamath} or \texttt{metamath.}{\em xxx} where
{\em xxx} is the convention (such as \texttt{exe}) for an executable
program on your operating system.  There are several additional files
containing samples of the Metamath language, all ending with
\texttt{.mm}.  The file \texttt{set.mm}\index{set theory database
(\texttt{set.mm})} contains logic and set theory and can be used as a
starting point for other areas of mathematics.

You will also need a text editor\index{text editor} capable of editing plain
{\sc ascii}\footnote{American Standard Code for Information Interchange.} text
in order to prepare your input files.\index{ascii@{\sc ascii}}  Most computers
have this capability built in.  Note that plain text is not necessarily the
default for some word processing programs\index{word processor}, especially if
they can handle different fonts; for example, with Microsoft Word\index{Word
(Microsoft)}, you must save the file in the format ``Text Only With Line
Breaks'' to get a plain text\index{plain text} file.\footnote{It is
recommended that all lines in a Metamath source file be 79 characters or less
in length for compatibility among different computer terminals.  When creating
a source file on an editor such as Word, select a monospaced
font\index{monospaced font} such as Courier\index{Courier font} or
Monaco\index{Monaco font} to make this easier to achieve.  Better yet,
just use a plain text editor such as Notepad.}

On some computer systems, Metamath does not have the capability to print
its output directly; instead, you send its output to a file (using the
\texttt{open} commands described later).  The way you print this output
file depends on your computer.\index{printers} Some computers have a
print command, whereas with others, you may have to read the file into
an editor and print it from there.

If you want to print your Metamath source files with typeset formulas
containing standard mathematical symbols, you will need the \LaTeX\
typesetting program\index{latex@{\LaTeX}}, which is widely and freely
available for most operating systems.  It runs natively on Unix and
Linux, and can be installed on Windows as part of the free Cygwin
package (\url{http://cygwin.com}).

You can also produce {\sc html}\footnote{HyperText Markup Language.}
web pages.  The {\tt help html} command in the Metamath program will
assist you with this feature.

\section{Your First Formal System}\label{start}
\subsection{From Nothing to Zero}\label{startf}

To give you a feel for what the Metamath\index{Metamath} language looks like,
we will take a look at a very simple example from formal number
theory\index{number theory}.  This example is taken from
Mendelson\index{Mendelson, Elliot} \cite[p. 123]{Mendelson}.\footnote{To keep
the example simple, we have changed the formalism slightly, and what we call
axioms\index{axiom} are strictly speaking theorems\index{theorem} in
\cite{Mendelson}.}  We will look at a small subset of this theory, namely that
part needed for the first number theory theorem proved in \cite{Mendelson}.

First we will look at a standard formal proof\index{formal proof} for the
example we have picked, then we will look at the Metamath version.  If you
have never been exposed to formal proofs, the notation may seem to be such
overkill to express such simple notions that you may wonder if you are missing
something.  You aren't.  The concepts involved are in fact very simple, and a
detailed breakdown in this fashion is necessary to express the proof in a way
that can be verified mechanically.  And as you will see, Metamath breaks the
proof down into even finer pieces so that the mechanical verification process
can be about as simple as possible.

Before we can introduce the axioms\index{axiom} of the theory, we must define
the syntax rules for forming legal expressions\index{syntax rules}
(combinations of symbols) with which those axioms can be used. The number 0 is
a {\bf term}\index{term}; and if $ t$ and $r$ are terms, so is $(t+r)$. Here,
$ t$ and $r$ are ``metavariables''\index{metavariable} ranging over terms; they
themselves do not appear as symbols in an actual term.  Some examples of
actual terms are $(0 + 0)$ and $((0+0)+0)$.  (Note that our theory describes
only the number zero and sums of zeroes.  Of course, not much can be done with
such a trivial theory, but remember that we have picked a very small subset of
complete number theory for our example.  The important thing for you to focus
on is our definitions that describe how symbols are combined to form valid
expressions, and not on the content or meaning of those expressions.) If $ t$
and $r$ are terms, an expression of the form $ t=r$ is a {\bf wff}
(well-formed formula)\index{well-formed formula (wff)}; and if $P$ and $Q$ are
wffs, so is $(P\rightarrow Q)$ (which means ``$P$ implies
$Q$''\index{implication ($\rightarrow$)} or ``if $P$ then $Q$'').
Here $P$ and $Q$ are metavariables ranging over wffs.  Examples of actual
wffs are $0=0$, $(0+0)=0$, $(0=0 \rightarrow (0+0)=0)$, and $(0=0\rightarrow
(0=0\rightarrow 0=(0+0)))$.  (Our notation makes use of more parentheses than
are customary, but the elimination of ambiguity this way simplifies our
example by avoiding the need to define operator precedence\index{operator
precedence}.)

The {\bf axioms}\index{axiom} of our theory are all wffs of the following
form, where $ t$, $r$, and $s$ are any terms:

%Latex p. 92
\renewcommand{\theequation}{A\arabic{equation}}

\begin{equation}
(t=r\rightarrow (t=s\rightarrow r=s))
\end{equation}
\begin{equation}
(t+0)=t
\end{equation}

Note that there are an infinite number of axioms since there are an infinite
number of possible terms.  A1 and A2 are properly called ``axiom
schemes,''\index{axiom scheme} but we will refer to them as ``axioms'' for
brevity.

An axiom is a {\bf theorem}; and if $P$ and $(P\rightarrow Q)$ are theorems
(where $P$ and $Q$ are wffs), then $Q$ is also a theorem.\index{theorem}  The
second part of this definition is called the modus ponens (MP) rule of
inference\index{inference rule}\index{modus ponens}.  It allows us to obtain
new theorems from old ones.

The {\bf proof}\index{proof} of a theorem is a sequence of one or more
theorems, each of which is either an axiom or the result of modus ponens
applied to two previous theorems in the sequence, and the last of which is the
theorem being proved.

The theorem we will prove for our example is very simple:  $ t=t$.  The proof of
our theorem follows.  Study it carefully until you feel sure you
understand it.\label{zeroproof}

% Use tabu so that lines will wrap automatically as needed.
\begin{tabu} { l X X }
1. & $(t+0)=t$ & (by axiom A2) \\
2. & $(t+0)=t$ & (by axiom A2) \\
3. & $((t+0)=t \rightarrow ((t+0)=t\rightarrow t=t))$ & (by axiom A1) \\
4. & $((t+0)=t\rightarrow t=t)$ & (by MP applied to steps 2 and 3) \\
5. & $t=t$ & (by MP applied to steps 1 and 4) \\
\end{tabu}

(You may wonder why step 1 is repeated twice.  This is not necessary in the
formal language we have defined, but in Metamath's ``reverse Polish
notation''\index{reverse Polish notation (RPN)} for proofs, a previous step
can be referred to only once.  The repetition of step~1 here will enable you
to see more clearly the correspondence of this proof with the
Metamath\index{Metamath} version on p.~\pageref{demoproof}.)

Our theorem is more properly called a ``theorem scheme,''\index{theorem
scheme} for it represents an infinite number of theorems, one for each
possible term $ t$.  Two examples of actual theorems would be $0=0$ and
$(0+0)=(0+0)$.  Rarely do we prove actual theorems, since by proving schemes
we can prove an infinite number of theorems in one fell swoop.  Similarly, our
proof should really be called a ``proof scheme.''\index{proof scheme}  To
obtain an actual proof, pick an actual term to use in place of $ t$, and
substitute it for $ t$ throughout the proof.

Let's discuss what we have done here.  The axioms\index{axiom} of our theory,
A1 and A2, are trivial and obvious.  Everyone knows that adding zero to
something doesn't change it, and also that if two things are equal to a third,
then they are equal to each other. In fact, stating the trivial and obvious is
a goal to strive for in any axiomatic system.  From trivial and obvious truths
that everyone agrees upon, we can prove results that are not so obvious yet
have absolute faith in them.  If we trust the axioms and the rules, we must,
by definition, trust the consequences of those axioms and rules, if logic is
to mean anything at all.

Our rule of inference\index{rule}, modus ponens\index{modus ponens}, is also
pretty obvious once you understand what it means.  If we prove a fact $P$, and
we also prove that $P$ implies $Q$, then $Q$ necessarily follows as a new
fact.  The rule provides us with a means for obtaining new facts (i.e.\
theorems\index{theorem}) from old ones.

The theorem that we have proved, $ t=t$, is so fundamental that you may wonder
why it isn't one of the axioms\index{axiom}.  In some axiom systems of
arithmetic, it {\em is} an axiom.  The choice of axioms in a theory is to some
extent arbitrary and even an art form, constrained only by the requirement
that any two equivalent axiom systems be able to derive each other as
theorems.  We could imagine that the inventor of our axiom system originally
included $ t=t$ as an axiom, then discovered that it could be derived as a
theorem from the other axioms.  Because of this, it was not necessary to
keep it as an axiom.  By eliminating it, the final set of axioms became
that much simpler.

Unless you have worked with formal proofs\index{formal proof} before, it
probably wasn't apparent to you that $ t=t$ could be derived from our two
axioms until you saw the proof. While you certainly believe that $ t=t$ is
true, you might not be able to convince an imaginary skeptic who believes only
in our two axioms until you produce the proof.  Formal proofs such as this are
hard to come up with when you first start working with them, but after you get
used to them they can become interesting and fun.  Once you understand the
idea behind formal proofs you will have grasped the fundamental principle that
underlies all of mathematics.  As the mathematics becomes more sophisticated,
its proofs become more challenging, but ultimately they all can be broken down
into individual steps as simple as the ones in our proof above.

Mendelson's\index{Mendelson, Elliot} book, from which our example was taken,
contains a number of detailed formal proofs such as these, and you may be
interested in looking it up.  The book is intended for mathematicians,
however, and most of it is rather advanced.  Popular literature describing
formal proofs\index{formal proof} include \cite[p.~296]{Rucker}\index{Rucker,
Rudy} and \cite[pp.~204--230]{Hofstadter}\index{Hofstadter, Douglas R.}.

\subsection{Converting It to Metamath}\label{convert}

Formal proofs\index{formal proof} such as the one in our example break down
logical reasoning into small, precise steps that leave little doubt that the
results follow from the axioms\index{axiom}.  You might think that this would
be the finest breakdown we can achieve in mathematics.  However, there is more
to the proof than meets the eye. Although our axioms were rather simple, a lot
of verbiage was needed before we could even state them:  we needed to define
``term,'' ``wff,'' and so on.  In addition, there are a number of implied
rules that we haven't even mentioned. For example, how do we know that step 3
of our proof follows from axiom A1? There is some hidden reasoning involved in
determining this.  Axiom A1 has two occurrences of the letter $ t$.  One of
the implied rules states that whatever we substitute for $ t$ must be a legal
term\index{term}.\footnote{Some authors make this implied rule explicit by
stating, ``only expressions of the above form are terms,'' after defining
``term.''}  The expression $ t+0$ is pretty obviously a legal term whenever $
t$ is, but suppose we wanted to substitute a huge term with thousands of
symbols?  Certainly a lot of work would be involved in determining that it
really is a term, but in ordinary formal proofs all of this work would be
considered a single ``step.''

To express our axiom system in the Metamath\index{Metamath} language, we must
describe this auxiliary information in addition to the axioms themselves.
Metamath does not know what a ``term'' or a ``wff''\index{well-formed formula
(wff)} is.  In Metamath, the specification of the ways in which we can combine
symbols to obtain terms and wffs are like little axioms in themselves.  These
auxiliary axioms are expressed in the same notation as the ``real''
axioms\index{axiom}, and Metamath does not distinguish between the two.  The
distinction is made by you, i.e.\ by the way in which you interpret the
notation you have chosen to express these two kinds of axioms.

The Metamath language breaks down mathematical proofs into tiny pieces, much
more so than in ordinary formal proofs\index{formal proof}.  If a single
step\index{proof step} involves the
substitution\index{substitution!variable}\index{variable substitution} of a
complex term for one of its variables, Metamath must see this single step
broken down into many small steps.  This fine-grained breakdown is what gives
Metamath generality and flexibility as it lets it not be limited to any
particular mathematical notation.

Metamath's proof notation is not, in itself, intended to be read by humans but
rather is in a compact format intended for a machine.  The Metamath program
will convert this notation to a form you can understand, using the \texttt{show
proof}\index{\texttt{show proof} command} command.  You can tell the program what
level of detail of the proof you want to look at.  You may want to look at
just the logical inference steps that correspond
to ordinary formal proof steps,
or you may want to see the fine-grained steps that prove that an expression is
a term.

Here, without further ado, is our example converted to the
Metamath\index{Metamath} language:\index{metavariable}\label{demo0}

\begin{verbatim}
$( Declare the constant symbols we will use $)
    $c 0 + = -> ( ) term wff |- $.
$( Declare the metavariables we will use $)
    $v t r s P Q $.
$( Specify properties of the metavariables $)
    tt $f term t $.
    tr $f term r $.
    ts $f term s $.
    wp $f wff P $.
    wq $f wff Q $.
$( Define "term" and "wff" $)
    tze $a term 0 $.
    tpl $a term ( t + r ) $.
    weq $a wff t = r $.
    wim $a wff ( P -> Q ) $.
$( State the axioms $)
    a1 $a |- ( t = r -> ( t = s -> r = s ) ) $.
    a2 $a |- ( t + 0 ) = t $.
$( Define the modus ponens inference rule $)
    ${
       min $e |- P $.
       maj $e |- ( P -> Q ) $.
       mp  $a |- Q $.
    $}
$( Prove a theorem $)
    th1 $p |- t = t $=
  $( Here is its proof: $)
       tt tze tpl tt weq tt tt weq tt a2 tt tze tpl
       tt weq tt tze tpl tt weq tt tt weq wim tt a2
       tt tze tpl tt tt a1 mp mp
     $.
\end{verbatim}\index{metavariable}

A ``database''\index{database} is a set of one or more {\sc ascii} source
files.  Here's a brief description of this Metamath\index{Metamath} database
(which consists of this single source file), so that you can understand in
general terms what is going on.  To understand the source file in detail, you
should read Chapter~\ref{languagespec}.

The database is a sequence of ``tokens,''\index{token} which are normally
separated by spaces or line breaks.  The only tokens that are built into
the Metamath language are those beginning with \texttt{\$}.  These tokens
are called ``keywords.''\index{keyword}  All other tokens are
user-defined, and their names are arbitrary.

As you might have guessed, the Metamath token \texttt{\$(}\index{\texttt{\$(} and
\texttt{\$)} auxiliary keywords} starts a comment and \texttt{\$)} ends a comment.

The Metamath tokens \texttt{\$c}\index{\texttt{\$c} statement},
\texttt{\$v}\index{\texttt{\$v} statement},
\texttt{\$e}\index{\texttt{\$e} statement},
\texttt{\$f}\index{\texttt{\$f} statement},
\texttt{\$a}\index{\texttt{\$a} statement}, and
\texttt{\$p}\index{\texttt{\$p} statement} specify ``statements'' that
end with \texttt{\$.}\,.\index{\texttt{\$.}\ keyword}

The Metamath tokens \texttt{\$c} and \texttt{\$v} each declare\index{constant
declaration}\index{variable declaration} a list of user-defined tokens, called
``math symbols,''\index{math symbol} that the database will reference later
on.  All of the math symbols they define you have seen earlier except the
turnstile symbol \texttt{|-} ($\vdash$)\index{turnstile ({$\,\vdash$})}, which is
commonly used by logicians to mean ``a proof exists for.''  For us
the turnstile is just a
convenient symbol that distinguishes expressions that are axioms\index{axiom}
or theorems\index{theorem} from expressions that are terms or wffs.

The \texttt{\$c} statement declares ``constants''\index{constant} and
the \texttt{\$v} statement declares
``variables''\index{variable}\index{constant declaration}\index{variable
declaration} (or more precisely, metavariables\index{metavariable}).  A
variable may be substituted\index{substitution!variable}\index{variable
substitution} with sequences of math symbols whereas a constant may not
be substituted with anything.

It may seem redundant to require both \texttt{\$c}\index{\texttt{\$c} statement} and
\texttt{\$v}\index{\texttt{\$v} statement} statements (since any math
symbol\index{math symbol} not specified with a \texttt{\$c} statement could be
presumed to be a variable), but this provides for better error checking and
also allows math symbols to be redeclared\index{redeclaration of symbols}
(Section~\ref{scoping}).

The token \texttt{\$f}\index{\texttt{\$f} statement} specifies a
statement called a ``variable-type hypothesis'' (also called a
``floating hypothesis'') and \texttt{\$e}\index{\texttt{\$e} statement}
specifies a ``logical hypothesis'' (also called an ``essential
hypothesis'').\index{hypothesis}\index{variable-type
hypothesis}\index{logical hypothesis}\index{floating
hypothesis}\index{essential hypothesis} The token
\texttt{\$a}\index{\texttt{\$a} statement} specifies an ``axiomatic
assertion,''\index{axiomatic assertion} and
\texttt{\$p}\index{\texttt{\$p} statement} specifies a ``provable
assertion.''\index{provable assertion} To the left of each occurrence of
these four tokens is a ``label''\index{label} that identifies the
hypothesis or assertion for later reference.  For example, the label of
the first axiomatic assertion is \texttt{tze}.  A \texttt{\$f} statement
must contain exactly two math symbols, a constant followed by a
variable.  The \texttt{\$e}, \texttt{\$a}, and \texttt{\$p} statements
each start with a constant followed by, in general, an arbitrary
sequence of math symbols.

Associated with each assertion\index{assertion} is a set of hypotheses
that must be satisfied in order for the assertion to be used in a proof.
These are called the ``mandatory hypotheses''\index{mandatory
hypothesis} of the assertion.  Among those hypotheses whose ``scope''
(described below) includes the assertion, \texttt{\$e} hypotheses are
always mandatory and \texttt{\$f}\index{\texttt{\$f} statement}
hypotheses are mandatory when they share their variable with the
assertion or its \texttt{\$e} hypotheses.  The exact rules for
determining which hypotheses are mandatory are described in detail in
Sections~\ref{frames} and \ref{scoping}.  For example, the mandatory
hypotheses of assertion \texttt{tpl} are \texttt{tt} and \texttt{tr},
whereas assertion \texttt{tze} has no mandatory hypotheses because it
contains no variables and has no \texttt{\$e}\index{\texttt{\$e}
statement} hypothesis.  Metamath's \texttt{show statement}
command\index{\texttt{show statement} command}, described in the next
section, will show you a statement's mandatory hypotheses.

Sometimes we need to make a hypothesis relevant to only certain
assertions.  The set of statements to which a hypothesis is relevant is
called its ``scope.''  The Metamath brackets,
\texttt{\$\char`\{}\index{\texttt{\$\char`\{} and \texttt{\$\char`\}}
keywords} and \texttt{\$\char`\}}, define a ``block''\index{block} that
delimits the scope of any hypothesis contained between them.  The
assertion \texttt{mp} has mandatory hypotheses \texttt{wp}, \texttt{wq},
\texttt{min}, and \texttt{maj}.  The only mandatory hypothesis of
\texttt{th1}, on the other hand, is \texttt{tt}, since \texttt{th1}
occurs outside of the block containing \texttt{min} and \texttt{maj}.

Note that \texttt{\$\char`\{} and \texttt{\$\char`\}} do not affect the
scope of assertions (\texttt{\$a} and \texttt{\$p}).  Assertions are always
available to be referenced by any later proof in the source file.

Each provable assertion (\texttt{\$p}\index{\texttt{\$p} statement}
statement) has two parts.  The first part is the
assertion\index{assertion} itself, which is a sequence of math
symbol\index{math symbol} tokens placed between the \texttt{\$p} token
and a \texttt{\$=}\index{\texttt{\$=} keyword} token.  The second part
is a ``proof,'' which is a list of label tokens placed between the
\texttt{\$=} token and the \texttt{\$.}\index{\texttt{\$.}\ keyword}\
token that ends the statement.\footnote{If you've looked at the
\texttt{set.mm} database, you may have noticed another notation used for
proofs.  The other notation is called ``compressed.''\index{compressed
proof}\index{proof!compressed} It reduces the amount of space needed to
store a proof in the database and is described in
Appendix~\ref{compressed}.  In the example above, we use
``normal''\index{normal proof}\index{proof!normal} notation.} The proof
acts as a series of instructions to the Metamath program, telling it how
to build up the sequence of math symbols contained in the assertion part of
the \texttt{\$p} statement, making use of the hypotheses of the
\texttt{\$p} statement and previous assertions.  The construction takes
place according to precise rules.  If the list of labels in the proof
causes these rules to be violated, or if the final sequence that results
does not match the assertion, the Metamath program will notify you with
an error message.

If you are familiar with reverse Polish notation (RPN), which is sometimes used
on pocket calculators, here in a nutshell is how a proof works.  Each
hypothesis label\index{hypothesis label} in the proof is pushed\index{push}
onto the RPN stack\index{stack}\index{RPN stack} as it is encountered. Each
assertion label\index{assertion label} pops\index{pop} off the stack as many
entries as the referenced assertion has mandatory hypotheses.  Variable
substitutions\index{substitution!variable}\index{variable substitution} are
computed which, when made to the referenced assertion's mandatory hypotheses,
cause these hypotheses to match the stack entries. These same substitutions
are then made to the variables in the referenced assertion itself, which is
then pushed onto the stack.  At the end of the proof, there should be one
stack entry, namely the assertion being proved.  This process is explained in
detail in Section~\ref{proof}.

Metamath's proof notation is not very readable for humans, but it allows the
proof to be stored compactly in a file.  The Metamath\index{Metamath} program
has proof display features that let you see what's going on in a more
readable way, as you will see in the next section.

The rules used in verifying a proof are not based on any built-in syntax of the
symbol sequence in an assertion\index{assertion} nor on any built-in meanings
attached to specific symbol names.  They are based strictly on symbol
matching:  constants\index{constant} must match themselves, and
variables\index{variable} may be replaced with anything that allows a match to
occur.  For example, instead of \texttt{term}, \texttt{0}, and \verb$|-$ we could
have just as well used \texttt{yellow}, \texttt{zero}, and \texttt{provable}, as long
as we did so consistently throughout the database.  Also, we could have used
\texttt{is provable} (two tokens) instead of \verb$|-$ (one token) throughout the
database.  In each of these cases, the proof would be exactly the same.  The
independence of proofs and notation means that you have a lot of flexibility to
change the notation you use without having to change any proofs.

\section{A Trial Run}\label{trialrun}

Now you are ready to try out the Metamath\index{Metamath} program.

On all computer systems, Metamath has a standard ``command line
interface'' (CLI)\index{command line interface (CLI)} that allows you to
interact with it.  You supply commands to the CLI by typing them on the
keyboard and pressing your keyboard's {\em return} key after each line
you enter.  The CLI is designed to be easy to use and has built-in help
features.

The first thing you should do is to use a text editor to create a file
called \texttt{demo0.mm} and type into it the Metamath source shown on
p.~\pageref{demo0}.  Actually, this file is included with your Metamath
software package, so check that first.  If you type it in, make sure
that you save it in the form of ``plain {\sc ascii} text with line
breaks.''  Most word processors will have this feature.

Next you must run the Metamath program.  Depending on your computer
system and how Metamath is installed, this could range from clicking the
mouse on the Metamath icon to typing \texttt{run metamath} to typing
simply \texttt{metamath}.  (Metamath's {\tt help invoke} command describes
alternate ways of invoking the Metamath program.)

When you first enter Metamath\index{Metamath}, it will be at the CLI, waiting
for your input. You will see something like the following on your screen:
\begin{verbatim}
Metamath - Version 0.172 25-Jan-2019
Type HELP for help, EXIT to exit.
MM>
\end{verbatim}
The \texttt{MM>} prompt means that Metamath is waiting for a command.
Command keywords\index{command keyword} are not case sensitive;
we will use lower-case commands in our examples.
The version number and its release date will probably be different on your
system from the one we show above.

The first thing that you need to do is to read in your
database:\index{\texttt{read} command}\footnote{If a directory path is
needed on Unix,\index{Unix file names}\index{file names!Unix} you should
enclose the path/file name in quotes to prevent Metamath from thinking
that the \texttt{/} in the path name is a command qualifier, e.g.,
\texttt{read \char`\"db/set.mm\char`\"}.  Quotes are optional when there
is no ambiguity.}
\begin{verbatim}
MM> read demo0.mm
\end{verbatim}
Remember to press the {\em return} key after entering this command.  If
you omit the file name, Metamath will prompt you for one.   The syntax for
specifying a Macintosh file name path is given in a footnote on
p.~\pageref{includef}.\index{Macintosh file names}\index{file
names!Macintosh}

If there are any syntax errors in the database, Metamath will let you know
when it reads in the file.  The one thing that Metamath does not check when
reading in a database is that all proofs are correct, because this would
slow it down too much.  It is a good idea to periodically verify the proofs in
a database you are making changes to.  To do this, use the following command
(and do it for your \texttt{demo0.mm} file now).  Note that the \texttt{*} is a
``wild card'' meaning all proofs in the file.\index{\texttt{verify proof} command}
\begin{verbatim}
MM> verify proof *
\end{verbatim}
Metamath will report any proofs that are incorrect.

It is often useful to save the information that the Metamath program displays
on the screen. You can save everything that happens on the screen by opening a
log file. You may want to do this before you read in a database so that you
can examine any errors later on.  To open a log file, type
\begin{verbatim}
MM> open log abc.log
\end{verbatim}
This will open a file called \texttt{abc.log}, and everything that appears on the
screen from this point on will be stored in this file.  The name of the log file
is arbitrary. To close the log file, type
\begin{verbatim}
MM> close log
\end{verbatim}

Several commands let you examine what's inside your database.
Section~\ref{exploring} has an overview of some useful ones.  The
\texttt{show labels} command lets you see what statement
labels\index{label} exist.  A \texttt{*} matches any combination of
characters, and \texttt{t*} refers to all labels starting with the
letter \texttt{t}.\index{\texttt{show labels} command} The \texttt{/all}
is a ``command qualifier''\index{command qualifier} that tells Metamath
to include labels of hypotheses.  (To see the syntax explained, type
\texttt{help show labels}.)  Type
\begin{verbatim}
MM> show labels t* /all
\end{verbatim}
Metamath will respond with
\begin{verbatim}
The statement number, label, and type are shown.
3 tt $f       4 tr $f       5 ts $f       8 tze $a
9 tpl $a      19 th1 $p
\end{verbatim}

You can use the \texttt{show statement} command to get information about a
particular statement.\index{\texttt{show statement} command}
For example, you can get information about the statement with label \texttt{mp}
by typing
\begin{verbatim}
MM> show statement mp /full
\end{verbatim}
Metamath will respond with
\begin{verbatim}
Statement 17 is located on line 43 of the file
"demo0.mm".
"Define the modus ponens inference rule"
17 mp $a |- Q $.
Its mandatory hypotheses in RPN order are:
  wp $f wff P $.
  wq $f wff Q $.
  min $e |- P $.
  maj $e |- ( P -> Q ) $.
The statement and its hypotheses require the
      variables:  Q P
The variables it contains are:  Q P
\end{verbatim}
The mandatory hypotheses\index{mandatory hypothesis} and their
order\index{RPN order} are
useful to know when you are trying to understand or debug a proof.

Now you are ready to look at what's really inside our proof.  First, here is
how to look at every step in the proof---not just the ones corresponding to an
ordinary formal proof\index{formal proof}, but also the ones that build up the
formulas that appear in each ordinary formal proof step.\index{\texttt{show
proof} command}
\begin{verbatim}
MM> show proof th1 /lemmon /all
\end{verbatim}

This will display the proof on the screen in the following format:
\begin{verbatim}
 1 tt            $f term t
 2 tze           $a term 0
 3 1,2 tpl       $a term ( t + 0 )
 4 tt            $f term t
 5 3,4 weq       $a wff ( t + 0 ) = t
 6 tt            $f term t
 7 tt            $f term t
 8 6,7 weq       $a wff t = t
 9 tt            $f term t
10 9 a2          $a |- ( t + 0 ) = t
11 tt            $f term t
12 tze           $a term 0
13 11,12 tpl     $a term ( t + 0 )
14 tt            $f term t
15 13,14 weq     $a wff ( t + 0 ) = t
16 tt            $f term t
17 tze           $a term 0
18 16,17 tpl     $a term ( t + 0 )
19 tt            $f term t
20 18,19 weq     $a wff ( t + 0 ) = t
21 tt            $f term t
22 tt            $f term t
23 21,22 weq     $a wff t = t
24 20,23 wim     $a wff ( ( t + 0 ) = t -> t = t )
25 tt            $f term t
26 25 a2         $a |- ( t + 0 ) = t
27 tt            $f term t
28 tze           $a term 0
29 27,28 tpl     $a term ( t + 0 )
30 tt            $f term t
31 tt            $f term t
32 29,30,31 a1   $a |- ( ( t + 0 ) = t -> ( ( t + 0 )
                                     = t -> t = t ) )
33 15,24,26,32 mp  $a |- ( ( t + 0 ) = t -> t = t )
34 5,8,10,33 mp  $a |- t = t
\end{verbatim}

The \texttt{/lemmon} command qualifier specifies what is known as a Lemmon-style
display\index{Lemmon-style proof}\index{proof!Lemmon-style}.  Omitting the
\texttt{/lemmon} qualifier results in a tree-style proof (see
p.~\pageref{treeproof} for an example) that is somewhat less explicit but
easier to follow once you get used to it.\index{tree-style
proof}\index{proof!tree-style}

The first number on each line is the step
number of the proof.  Any numbers that follow are step numbers assigned to the
hypotheses of the statement referenced by that step.  Next is the label of
the statement referenced by the step.  The statement type of the statement
referenced comes next, followed by the math symbol\index{math symbol} string
constructed by the proof up to that step.

The last step, 34, contains the statement that is being proved.

Looking at a small piece of the proof, notice that steps 3 and 4 have
established that
\texttt{( t + 0 )} and \texttt{t} are \texttt{term}\,s, and step 5 makes use of steps 3 and
4 to establish that \texttt{( t + 0 ) = t} is a \texttt{wff}.  Let Metamath
itself tell us in detail what is happening in step 5.  Note that the
``target hypothesis'' refers to where step 5 is eventually used, i.e., in step
34.
\begin{verbatim}
MM> show proof th1 /detailed_step 5
Proof step 5:  wp=weq $a wff ( t + 0 ) = t
This step assigns source "weq" ($a) to target "wp"
($f).  The source assertion requires the hypotheses
"tt" ($f, step 3) and "tr" ($f, step 4).  The parent
assertion of the target hypothesis is "mp" ($a,
step 34).
The source assertion before substitution was:
    weq $a wff t = r
The following substitutions were made to the source
assertion:
    Variable  Substituted with
     t         ( t + 0 )
     r         t
The target hypothesis before substitution was:
    wp $f wff P
The following substitution was made to the target
hypothesis:
    Variable  Substituted with
     P         ( t + 0 ) = t
\end{verbatim}

The full proof just shown is useful to understand what is going on in detail.
However, most of the time you will just be interested in
the ``essential'' or logical steps of a proof, i.e.\ those steps
that correspond to an
ordinary formal proof\index{formal proof}.  If you type
\begin{verbatim}
MM> show proof th1 /lemmon /renumber
\end{verbatim}
you will see\label{demoproof}
\begin{verbatim}
1 a2             $a |- ( t + 0 ) = t
2 a2             $a |- ( t + 0 ) = t
3 a1             $a |- ( ( t + 0 ) = t -> ( ( t + 0 )
                                     = t -> t = t ) )
4 2,3 mp         $a |- ( ( t + 0 ) = t -> t = t )
5 1,4 mp         $a |- t = t
\end{verbatim}
Compare this to the formal proof on p.~\pageref{zeroproof} and
notice the resemblance.
By default Metamath
does not show \texttt{\$f}\index{\texttt{\$f}
statement} hypotheses and everything branching off of them in the proof tree
when the proof is displayed; this makes the proof look more like an ordinary
mathematical proof, which does not normally incorporate the explicit
construction of expressions.
This is called the ``essential'' view
(at one time you had to add the
\texttt{/essential} qualifier in the \texttt{show proof}
command to get this view, but this is now the default).
You can could use the \texttt{/all} qualifier in the \texttt{show
proof} command to also show the explicit construction of expressions.
The \texttt{/renumber} qualifier means to renumber
the steps to correspond only to what is displayed.\index{\texttt{show proof}
command}

To exit Metamath, type\index{\texttt{exit} command}
\begin{verbatim}
MM> exit
\end{verbatim}

\subsection{Some Hints for Using the Command Line Interface}

We will conclude this quick introduction to Metamath\index{Metamath} with some
helpful hints on how to navigate your way through the commands.
\index{command line interface (CLI)}

When you type commands into Metamath's CLI, you only have to type as many
characters of a command keyword\index{command keyword} as are needed to make
it unambiguous.  If you type too few characters, Metamath will tell you what
the choices are.  In the case of the \texttt{read} command, only the \texttt{r} is
needed to specify it unambiguously, so you could have typed\index{\texttt{read}
command}
\begin{verbatim}
MM> r demo0.mm
\end{verbatim}
instead of
\begin{verbatim}
MM> read demo0.mm
\end{verbatim}
In our description, we always show the full command words.  When using the
Metamath CLI commands in a command file (to be read with the \texttt{submit}
command)\index{\texttt{submit} command}, it is good practice to use
the unabbreviated command to ensure your instructions will not become ambiguous
if more commands are added to the Metamath program in the future.

The command keywords\index{command
keyword} are not case sensitive; you may type either \texttt{read} or
\texttt{ReAd}.  File names may or may not be case sensitive, depending on your
computer's operating system.  Metamath label\index{label} and math
symbol\index{math symbol} tokens\index{token} are case-sensitive.

The \texttt{help} command\index{\texttt{help} command} will provide you
with a list of topics you can get help on.  You can then type
\texttt{help} {\em topic} to get help on that topic.

If you are uncertain of a command's spelling, just type as many characters
as you remember of the command.  If you have not typed enough characters to
specify it unambiguously, Metamath will tell you what choices you have.

\begin{verbatim}
MM> show s
         ^
?Ambiguous keyword - please specify SETTINGS,
STATEMENT, or SOURCE.
\end{verbatim}

If you don't know what argument to use as part of a command, type a
\texttt{?}\index{\texttt{]}@\texttt{?}\ in command lines}\ at the
argument position.  Metamath will tell you what it expected there.

\begin{verbatim}
MM> show ?
         ^
?Expected SETTINGS, LABELS, STATEMENT, SOURCE, PROOF,
MEMORY, TRACE_BACK, or USAGE.
\end{verbatim}

Finally, you may type just the first word or words of a command followed
by {\em return}.  Metamath will prompt you for the remaining part of the
command, showing you the choices at each step.  For example, instead of
typing \texttt{show statement th1 /full} you could interact in the
following manner:
\begin{verbatim}
MM> show
SETTINGS, LABELS, STATEMENT, SOURCE, PROOF,
MEMORY, TRACE_BACK, or USAGE <SETTINGS>? st
What is the statement label <th1>?
/ or nothing <nothing>? /
TEX, COMMENT_ONLY, or FULL <TEX>? f
/ or nothing <nothing>?
19 th1 $p |- t = t $= ... $.
\end{verbatim}
After each \texttt{?}\ in this mode, you must give Metamath the
information it requests.  Sometimes Metamath gives you a list of choices
with the default choice indicated by brackets \texttt{< > }. Pressing
{\em return} after the \texttt{?}\ will select the default choice.
Answering anything else will override the default.  Note that the
\texttt{/} in command qualifiers is considered a separate
token\index{token} by the parser, and this is why it is asked for
separately.

\section{Your First Proof}\label{frstprf}

Proofs are developed with the aid of the Proof Assistant\index{Proof
Assistant}.  We will now show you how the proof of theorem \texttt{th1}
was built.  So that you can repeat these steps, we will first have the
Proof Assistant erase the proof in Metamath's source buffer\index{source
buffer}, then reconstruct it.  (The source buffer is the place in memory
where Metamath stores the information in the database when it is
\texttt{read}\index{\texttt{read} command} in.  New or modified proofs
are kept in the source buffer until a \texttt{write source}
command\index{\texttt{write source} command} is issued.)  In practice, you
would place a \texttt{?}\index{\texttt{]}@\texttt{?}\ inside proofs}\
between \texttt{\$=}\index{\texttt{\$=} keyword} and
\texttt{\$.}\index{\texttt{\$.}\ keyword}\ in the database to indicate
to Metamath\index{Metamath} that the proof is unknown, and that would be
your starting point.  Whenever the \texttt{verify proof} command encounters
a proof with a \texttt{?}\ in place of a proof step, the statement is
identified as not proved.

When I first started creating Metamath proofs, I would write down
on a piece of paper the complete
formal proof\index{formal proof} as it would appear
in a \texttt{show proof} command\index{\texttt{show proof} command}; see
the display of \texttt{show proof th1 /lemmon /re\-num\-ber} above as an
example.  After you get used to using the Proof Assistant\index{Proof
Assistant} you may get to a point where you can ``see'' the proof in your mind
and let the Proof Assistant guide you in filling in the details, at least for
simpler proofs, but until you gain that experience you may find it very useful
to write down all the details in advance.
Otherwise you may waste a lot of time as you let it take you down a wrong path.
However, others do not find this approach as helpful.
For example, Thomas Brendan Leahy\index{Leahy, Thomas Brendan}
finds that it is more helpful to him to interactively
work backward from a machine-readable statement.
David A. Wheeler\index{Wheeler, David A.}
writes down a general approach, but develops the proof
interactively by switching between
working forwards (from hypotheses and facts likely to be useful) and
backwards (from the goal) until the forwards and backwards approaches meet.
In the end, use whatever approach works for you.

A proof is developed with the Proof Assistant by working backwards, starting
with the theorem\index{theorem} to be proved, and assigning each unknown step
with a theorem or hypothesis until no more unknown steps remain.  The Proof
Assistant will not let you make an assignment unless it can be ``unified''
with the unknown step.  This means that a
substitution\index{substitution!variable}\index{variable substitution} of
variables exists that will make the assignment match the unknown step.  On the
other hand, in the middle of a proof, when working backwards, often more than
one unification\index{unification} (set of substitutions) is possible, since
there is not enough information available at that point to uniquely establish
it.  In this case you can tell Metamath which unification to choose, or you
can continue to assign unknown steps until enough information is available to
make the unification unique.

We will assume you have entered Metamath and read in the database as described
above.  The following dialog shows how the proof was developed.  For more
details on what some of the commands do, refer to Section~\ref{pfcommands}.
\index{\texttt{prove} command}

\begin{verbatim}
MM> prove th1
Entering the Proof Assistant.  Type HELP for help, EXIT
to exit.  You will be working on the proof of statement th1:
  $p |- t = t
Note:  The proof you are starting with is already complete.
MM-PA>
\end{verbatim}

The \verb/MM-PA>/ prompt means we are inside the Proof
Assistant.\index{Proof Assistant} Most of the regular Metamath commands
(\texttt{show statement}, etc.) are still available if you need them.

\begin{verbatim}
MM-PA> delete all
The entire proof was deleted.
\end{verbatim}

We have deleted the whole proof so we can start from scratch.

\begin{verbatim}
MM-PA> show new_proof/lemmon/all
1 ?              $? |- t = t
\end{verbatim}

The \texttt{show new{\char`\_}proof} command\index{\texttt{show
new{\char`\_}proof} command} is like \texttt{show proof} except that we
don't specify a statement; instead, the proof we're working on is
displayed.

\begin{verbatim}
MM-PA> assign 1 mp
To undo the assignment, DELETE STEP 5 and INITIALIZE, UNIFY
if needed.
3   min=?  $? |- $2
4   maj=?  $? |- ( $2 -> t = t )
\end{verbatim}

The \texttt{assign} command\index{\texttt{assign} command} above means
``assign step 1 with the statement whose label is \texttt{mp}.''  Note
that step renumbering will constantly occur as you assign steps in the
middle of a proof; in general all steps from the step you assign until
the end of the proof will get moved up.  In this case, what used to be
step 1 is now step 5, because the (partial) proof now has five steps:
the four hypotheses of the \texttt{mp} statement and the \texttt{mp}
statement itself.  Let's look at all the steps in our partial proof:

\begin{verbatim}
MM-PA> show new_proof/lemmon/all
1 ?              $? wff $2
2 ?              $? wff t = t
3 ?              $? |- $2
4 ?              $? |- ( $2 -> t = t )
5 1,2,3,4 mp     $a |- t = t
\end{verbatim}

The symbol \texttt{\$2} is a temporary variable\index{temporary
variable} that represents a symbol sequence not yet known.  In the final
proof, all temporary variables will be eliminated.  The general format
for a temporary variable is \texttt{\$} followed by an integer.  Note
that \texttt{\$} is not a legal character in a math symbol (see
Section~\ref{dollardollar}, p.~\pageref{dollardollar}), so there will
never be a naming conflict between real symbols and temporary variables.

Unknown steps 1 and 2 are constructions of the two wffs used by the
modus ponens rule.  As you will see at the end of this section, the
Proof Assistant\index{Proof Assistant} can usually figure these steps
out by itself, and we will not have to worry about them.  Therefore from
here on we will display only the ``essential'' hypotheses, i.e.\ those
steps that correspond to traditional formal proofs\index{formal proof}.

\begin{verbatim}
MM-PA> show new_proof/lemmon
3 ?              $? |- $2
4 ?              $? |- ( $2 -> t = t )
5 3,4 mp         $a |- t = t
\end{verbatim}

Unknown steps 3 and 4 are the ones we must focus on.  They correspond to the
minor and major premises of the modus ponens rule.  We will assign them as
follows.  Notice that because of the step renumbering that takes place
after an assignment, it is advantageous to assign unknown steps in reverse
order, because earlier steps will not get renumbered.

\begin{verbatim}
MM-PA> assign 4 mp
To undo the assignment, DELETE STEP 8 and INITIALIZE, UNIFY
if needed.
3   min=?  $? |- $2
6     min=?  $? |- $4
7     maj=?  $? |- ( $4 -> ( $2 -> t = t ) )
\end{verbatim}

We are now going to describe an obscure feature that you will probably
never use but should be aware of.  The Metamath language allows empty
symbol sequences to be substituted for variables, but in most formal
systems this feature is never used.  One of the few examples where is it
used is the MIU-system\index{MIU-system} described in
Appendix~\ref{MIU}.  But such systems are rare, and by default this
feature is turned off in the Proof Assistant.  (It is always allowed for
{\tt verify proof}.)  Let us turn it on and see what
happens.\index{\texttt{set empty{\char`\_}substitution} command}

\begin{verbatim}
MM-PA> set empty_substitution on
Substitutions with empty symbol sequences is now allowed.
\end{verbatim}

With this feature enabled, more unifications will be
ambiguous\index{ambiguous unification}\index{unification!ambiguous} in
the middle of a proof, because
substitution\index{substitution!variable}\index{variable substitution}
of variables with empty symbol sequences will become an additional
possibility.  Let's see what happens when we make our next assignment.

\begin{verbatim}
MM-PA> assign 3 a2
There are 2 possible unifications.  Please select the correct
    one or Q if you want to UNIFY later.
Unify:  |- $6
 with:  |- ( $9 + 0 ) = $9
Unification #1 of 2 (weight = 7):
  Replace "$6" with "( + 0 ) ="
  Replace "$9" with ""
  Accept (A), reject (R), or quit (Q) <A>? r
\end{verbatim}

The first choice presented is the wrong one.  If we had selected it,
temporary variable \texttt{\$6} would have been assigned a truncated
wff, and temporary variable \texttt{\$9} would have been assigned an
empty sequence (which is not allowed in our system).  With this choice,
eventually we would reach a point where we would get stuck because
we would end up with steps impossible to prove.  (You may want to
try it.)  We typed \texttt{r} to reject the choice.

\begin{verbatim}
Unification #2 of 2 (weight = 21):
  Replace "$6" with "( $9 + 0 ) = $9"
  Accept (A), reject (R), or quit (Q) <A>? q
To undo the assignment, DELETE STEP 4 and INITIALIZE, UNIFY
if needed.
 7     min=?  $? |- $8
 8     maj=?  $? |- ( $8 -> ( $6 -> t = t ) )
\end{verbatim}

The second choice is correct, and normally we would type \texttt{a}
to accept it.  But instead we typed \texttt{q} to show what will happen:
it will leave the step with an unknown unification, which can be
seen as follows:

\begin{verbatim}
MM-PA> show new_proof/not_unified
 4   min    $a |- $6
        =a2  = |- ( $9 + 0 ) = $9
\end{verbatim}

Later we can unify this with the \texttt{unify}
\texttt{all/interactive} command.

The important point to remember is that occasionally you will be
presented with several unification choices while entering a proof, when
the program determines that there is not enough information yet to make
an unambiguous choice automatically (and this can happen even with
\texttt{set empty{\char`\_}substitution} turned off).  Usually it is
obvious by inspection which choice is correct, since incorrect ones will
tend to be meaningless fragments of wffs.  In addition, the correct
choice will usually be the first one presented, unlike our example
above.

Enough of this digression.  Let us go back to the default setting.

\begin{verbatim}
MM-PA> set empty_substitution off
The ability to substitute empty expressions for variables
has been turned off.  Note that this may make the Proof
Assistant too restrictive in some cases.
\end{verbatim}

If we delete the proof, start over, and get to the point where
we digressed above, there will no longer be an ambiguous unification.

\begin{verbatim}
MM-PA> assign 3 a2
To undo the assignment, DELETE STEP 4 and INITIALIZE, UNIFY
if needed.
 7     min=?  $? |- $4
 8     maj=?  $? |- ( $4 -> ( ( $5 + 0 ) = $5 -> t = t ) )
\end{verbatim}

Let us look at our proof so far, and continue.

\begin{verbatim}
MM-PA> show new_proof/lemmon
 4 a2            $a |- ( $5 + 0 ) = $5
 7 ?             $? |- $4
 8 ?             $? |- ( $4 -> ( ( $5 + 0 ) = $5 -> t = t ) )
 9 7,8 mp        $a |- ( ( $5 + 0 ) = $5 -> t = t )
10 4,9 mp        $a |- t = t
MM-PA> assign 8 a1
To undo the assignment, DELETE STEP 11 and INITIALIZE, UNIFY
if needed.
 7     min=?  $? |- ( t + 0 ) = t
MM-PA> assign 7 a2
To undo the assignment, DELETE STEP 8 and INITIALIZE, UNIFY
if needed.
MM-PA> show new_proof/lemmon
 4 a2            $a |- ( t + 0 ) = t
 8 a2            $a |- ( t + 0 ) = t
12 a1            $a |- ( ( t + 0 ) = t -> ( ( t + 0 ) = t ->
                                                    t = t ) )
13 8,12 mp       $a |- ( ( t + 0 ) = t -> t = t )
14 4,13 mp       $a |- t = t
\end{verbatim}

Now all temporary variables and unknown steps have been eliminated from the
``essential'' part of the proof.  When this is achieved, the Proof
Assistant\index{Proof Assistant} can usually figure out the rest of the proof
automatically.  (Note that the \texttt{improve} command can occasionally be
useful for filling in essential steps as well, but it only tries to make use
of statements that introduce no new variables in their hypotheses, which is
not the case for \texttt{mp}. Also it will not try to improve steps containing
temporary variables.)  Let's look at the complete proof, then run
the \texttt{improve} command, then look at it again.

\begin{verbatim}
MM-PA> show new_proof/lemmon/all
 1 ?             $? wff ( t + 0 ) = t
 2 ?             $? wff t = t
 3 ?             $? term t
 4 3 a2          $a |- ( t + 0 ) = t
 5 ?             $? wff ( t + 0 ) = t
 6 ?             $? wff ( ( t + 0 ) = t -> t = t )
 7 ?             $? term t
 8 7 a2          $a |- ( t + 0 ) = t
 9 ?             $? term ( t + 0 )
10 ?             $? term t
11 ?             $? term t
12 9,10,11 a1    $a |- ( ( t + 0 ) = t -> ( ( t + 0 ) = t ->
                                                    t = t ) )
13 5,6,8,12 mp   $a |- ( ( t + 0 ) = t -> t = t )
14 1,2,4,13 mp   $a |- t = t
\end{verbatim}

\begin{verbatim}
MM-PA> improve all
A proof of length 1 was found for step 11.
A proof of length 1 was found for step 10.
A proof of length 3 was found for step 9.
A proof of length 1 was found for step 7.
A proof of length 9 was found for step 6.
A proof of length 5 was found for step 5.
A proof of length 1 was found for step 3.
A proof of length 3 was found for step 2.
A proof of length 5 was found for step 1.
Steps 1 and above have been renumbered.
CONGRATULATIONS!  The proof is complete.  Use SAVE
NEW_PROOF to save it.  Note:  The Proof Assistant does
not detect $d violations.  After saving the proof, you
should verify it with VERIFY PROOF.
\end{verbatim}

The \texttt{save new{\char`\_}proof} command\index{\texttt{save
new{\char`\_}proof} command} will save the proof in the database.  Here
we will just display it in a form that can be clipped out of a log file
and inserted manually into the database source file with a text
editor.\index{normal proof}\index{proof!normal}

\begin{verbatim}
MM-PA> show new_proof/normal
---------Clip out the proof below this line:
      tt tze tpl tt weq tt tt weq tt a2 tt tze tpl tt weq
      tt tze tpl tt weq tt tt weq wim tt a2 tt tze tpl tt
      tt a1 mp mp $.
---------The proof of 'th1' to clip out ends above this line.
\end{verbatim}

There is another proof format called ``compressed''\index{compressed
proof}\index{proof!compressed} that you will see in databases.  It is
not important to understand how it is encoded but only to recognize it
when you see it.  Its only purpose is to reduce storage requirements for
large proofs.  A compressed proof can always be converted to a normal
one and vice-versa, and the Metamath \texttt{show proof}
commands\index{\texttt{show proof} command} work equally well with
compressed proofs.  The compressed proof format is described in
Appendix~\ref{compressed}.

\begin{verbatim}
MM-PA> show new_proof/compressed
---------Clip out the proof below this line:
      ( tze tpl weq a2 wim a1 mp ) ABCZADZAADZAEZJJKFLIA
      AGHH $.
---------The proof of 'th1' to clip out ends above this line.
\end{verbatim}

Now we will exit the Proof Assistant.  Since we made changes to the proof,
it will warn us that we have not saved it.  In this case, we don't care.

\begin{verbatim}
MM-PA> exit
Warning:  You have not saved changes to the proof.
Do you want to EXIT anyway (Y, N) <N>? y
Exiting the Proof Assistant.
Type EXIT again to exit Metamath.
\end{verbatim}

The Proof Assistant\index{Proof Assistant} has several other commands
that can help you while creating proofs.  See Section~\ref{pfcommands}
for a list of them.

A command that is often useful is \texttt{minimize{\char`\_}with
*/brief}, which tries to shorten the proof.  It can make the process
more efficient by letting you write a somewhat ``sloppy'' proof then
clean up some of the fine details of optimization for you (although it
can't perform miracles such as restructuring the overall proof).

\section{A Note About Editing a Data\-base File}

Once your source file contains proofs, there are some restrictions on
how you can edit it so that the proofs remain valid.  Pay particular
attention to these rules, since otherwise you can lose a lot of work.
It is a good idea to periodically verify all proofs with \texttt{verify
proof *} to ensure their integrity.

If your file contains only normal (as opposed to compressed) proofs, the
main rule is that you may not change the order of the mandatory
hypotheses\index{mandatory hypothesis} of any statement referenced in a
later proof.  For example, if you swap the order of the major and minor
premise in the modus ponens rule, all proofs making use of that rule
will become incorrect.  The \texttt{show statement}
command\index{\texttt{show statement} command} will show you the
mandatory hypotheses of a statement and their order.

If a statement has a compressed proof, you also must not change the
order of {\em its} mandatory hypotheses.  The compressed proof format
makes use of this information as part of the compression technique.
Note that swapping the names of two variables in a theorem will change
the order of its mandatory hypotheses.

The safest way to edit a statement, say \texttt{mytheorem}, is to
duplicate it then rename the original to \texttt{mytheoremOLD}
throughout the database.  Once the edited version is re-proved, all
statements referencing \texttt{mytheoremOLD} can be updated in the Proof
Assistant using \texttt{minimize{\char`\_}with
mytheorem
/allow{\char`\_}growth}.\index{\texttt{minimize{\char`\_}with} command}
% 3/10/07 Note: line-breaking the above results in duplicate index entries

\chapter{Abstract Mathematics Revealed}\label{fol}

\section{Logic and Set Theory}\label{logicandsettheory}

\begin{quote}
  {\em Set theory can be viewed as a form of exact theology.}
  \flushright\sc  Rudy Rucker\footnote{\cite{Barrow}, p.~31.}\\
\end{quote}\index{Rucker, Rudy}

Despite its seeming complexity, all of standard mathematics, no matter how
deep or abstract, can amazingly enough be derived from a relatively small set
of axioms\index{axiom} or first principles. The development of these axioms is
among the most impressive and important accomplishments of mathematics in the
20th century. Ultimately, these axioms can be broken down into a set of rules
for manipulating symbols that any technically oriented person can follow.

We will not spend much time trying to convey a deep, higher-level
understanding of the meaning of the axioms. This kind of understanding
requires some mathematical sophistication as well as an understanding of the
philosophy underlying the foundations of mathematics and typically develops
over time as you work with mathematics.  Our goal, instead, is to give you the
immediate ability to follow how theorems\index{theorem} are derived from the
axioms and from other theorems.  This will be similar to learning the syntax
of a computer language, which lets you follow the details in a program but
does not necessarily give you the ability to write non-trivial programs on
your own, an ability that comes with practice. For now don't be alarmed by
abstract-sounding names of the axioms; just focus on the rules for
manipulating the symbols, which follow the simple conventions of the
Metamath\index{Metamath} language.

The axioms that underlie all of standard mathematics consist of axioms of logic
and axioms of set theory. The axioms of logic are divided into two
subcategories, propositional calculus\index{propositional calculus} (sometimes
called sentential logic\index{sentential logic}) and predicate calculus
(sometimes called first-order logic\index{first-order logic}\index{quantifier
theory}\index{predicate calculus} or quantifier theory).  Propositional
calculus is a prerequisite for predicate calculus, and predicate calculus is a
prerequisite for set theory.  The version of set theory most commonly used is
Zermelo--Fraenkel set theory\index{Zermelo--Fraenkel set theory}\index{set theory}
with the axiom of choice,
often abbreviated as ZFC\index{ZFC}.

Here in a nutshell is what the axioms are all about in an informal way. The
connection between this description and symbols we will show you won't be
immediately apparent and in principle needn't ever be.  Our description just
tries to summarize what mathematicians think about when they work with the
axioms.

Logic is a set of rules that allow us determine truths given other truths.
Put another way,
logic is more or less the translation of what we would consider common sense
into a rigorous set of axioms.\index{axioms of logic}  Suppose $\varphi$,
$\psi$, and $\chi$ (the Greek letters phi, psi, and chi) represent statements
that are either true or false, and $x$ is a variable\index{variable!in predicate
calculus} ranging over some group of mathematical objects (sets, integers,
real numbers, etc.). In mathematics, a ``statement'' really means a formula,
and $\psi$ could be for example ``$x = 2$.''
Propositional calculus\index{propositional calculus}
allows us to use variables that are either true or false
and make deductions such as
``if $\varphi$ implies $\psi$ and $\psi$ implies $\chi$, then $\varphi$
implies $\chi$.''
Predicate calculus\index{predicate calculus}
extends propositional calculus by also allowing us
to discuss statements about objects (not just true and false values), including
statements about ``all'' or ``at least one'' object.
For example, predicate calculus allows to say,
``if $\varphi$ is true for all $x$, then $\varphi$ is true for some $x$.''
The logic used in \texttt{set.mm} is standard classical logic
(as opposed to other logic systems like intuitionistic logic).

Set theory\index{set theory} has to do with the manipulation of objects and
collections of objects, specifically the abstract, imaginary objects that
mathematics deals with, such as numbers. Everything that is claimed to exist
in mathematics is considered to be a set.  A set called the empty
set\index{empty set} contains nothing.  We represent the empty set by
$\varnothing$.  Many sets can be built up from the empty set.  There is a set
represented by $\{\varnothing\}$ that contains the empty set, another set
represented by $\{\varnothing,\{\varnothing\}\}$ that contains this set as
well as the empty set, another set represented by $\{\{\varnothing\}\}$ that
contains just the set that contains the empty set, and so on ad infinitum. All
mathematical objects, no matter how complex, are defined as being identical to
certain sets: the integer\index{integer} 0 is defined as the empty set, the
integer 1 is defined as $\{\varnothing\}$, the integer 2 is defined as
$\{\varnothing,\{\varnothing\}\}$.  (How these definitions were chosen doesn't
matter now, but the idea behind it is that these sets have the properties we
expect of integers once suitable operations are defined.)  Mathematical
operations, such as addition, are defined in terms of operations on
sets---their union\index{set union}, intersection\index{set intersection}, and
so on---operations you may have used in elementary school when you worked
with groups of apples and oranges.

With a leap of faith, the axioms also postulate the existence of infinite
sets\index{infinite set}, such as the set of all non-negative integers ($0, 1,
2,\ldots$, also called ``natural numbers''\index{natural number}).  This set
can't be represented with the brace notation\index{brace notation} we just
showed you, but requires a more complicated notation called ``class
abstraction.''\index{class abstraction}\index{abstraction class}  For
example, the infinite set $\{ x |
\mbox{``$x$ is a natural number''} \} $ means the ``set of all objects $x$
such that $x$ is a natural number'' i.e.\ the set of natural numbers; here,
``$x$ is a natural number'' is a rather complicated formula when broken down
into the primitive symbols.\label{expandom}\footnote{The statement ``$x$ is a
natural number'' is formally expressed as ``$x \in \omega$,'' where $\in$
(stylized epsilon) means ``is in'' or ``is an element of'' and $\omega$
(omega) means ``the set of natural numbers.''  When ``$x\in\omega$'' is
completely expanded in terms of the primitive symbols of set theory, the
result is  $\lnot$ $($ $\lnot$ $($ $\forall$ $z$ $($ $\lnot$ $\forall$ $w$ $($
$z$ $\in$ $w$ $\rightarrow$ $\lnot$ $w$ $\in$ $x$ $)$ $\rightarrow$ $z$ $\in$
$x$ $)$ $\rightarrow$ $($ $\forall$ $z$ $($ $\lnot$ $($ $\forall$ $w$ $($ $w$
$\in$ $z$ $\rightarrow$ $w$ $\in$ $x$ $)$ $\rightarrow$ $\forall$ $w$ $\lnot$
$w$ $\in$ $z$ $)$ $\rightarrow$ $\lnot$ $\forall$ $w$ $($ $w$ $\in$ $z$
$\rightarrow$ $\lnot$ $\forall$ $v$ $($ $v$ $\in$ $z$ $\rightarrow$ $\lnot$
$v$ $\in$ $w$ $)$ $)$ $)$ $\rightarrow$ $\lnot$ $\forall$ $z$ $\forall$ $w$
$($ $\lnot$ $($ $z$ $\in$ $x$ $\rightarrow$ $\lnot$ $w$ $\in$ $x$ $)$
$\rightarrow$ $($ $\lnot$ $z$ $\in$ $w$ $\rightarrow$ $($ $\lnot$ $z$ $=$ $w$
$\rightarrow$ $w$ $\in$ $z$ $)$ $)$ $)$ $)$ $)$ $\rightarrow$ $\lnot$
$\forall$ $y$ $($ $\lnot$ $($ $\lnot$ $($ $\forall$ $z$ $($ $\lnot$ $\forall$
$w$ $($ $z$ $\in$ $w$ $\rightarrow$ $\lnot$ $w$ $\in$ $y$ $)$ $\rightarrow$
$z$ $\in$ $y$ $)$ $\rightarrow$ $($ $\forall$ $z$ $($ $\lnot$ $($ $\forall$
$w$ $($ $w$ $\in$ $z$ $\rightarrow$ $w$ $\in$ $y$ $)$ $\rightarrow$ $\forall$
$w$ $\lnot$ $w$ $\in$ $z$ $)$ $\rightarrow$ $\lnot$ $\forall$ $w$ $($ $w$
$\in$ $z$ $\rightarrow$ $\lnot$ $\forall$ $v$ $($ $v$ $\in$ $z$ $\rightarrow$
$\lnot$ $v$ $\in$ $w$ $)$ $)$ $)$ $\rightarrow$ $\lnot$ $\forall$ $z$
$\forall$ $w$ $($ $\lnot$ $($ $z$ $\in$ $y$ $\rightarrow$ $\lnot$ $w$ $\in$
$y$ $)$ $\rightarrow$ $($ $\lnot$ $z$ $\in$ $w$ $\rightarrow$ $($ $\lnot$ $z$
$=$ $w$ $\rightarrow$ $w$ $\in$ $z$ $)$ $)$ $)$ $)$ $\rightarrow$ $($
$\forall$ $z$ $\lnot$ $z$ $\in$ $y$ $\rightarrow$ $\lnot$ $\forall$ $w$ $($
$\lnot$ $($ $w$ $\in$ $y$ $\rightarrow$ $\lnot$ $\forall$ $z$ $($ $w$ $\in$
$z$ $\rightarrow$ $\lnot$ $z$ $\in$ $y$ $)$ $)$ $\rightarrow$ $\lnot$ $($
$\lnot$ $\forall$ $z$ $($ $w$ $\in$ $z$ $\rightarrow$ $\lnot$ $z$ $\in$ $y$
$)$ $\rightarrow$ $w$ $\in$ $y$ $)$ $)$ $)$ $)$ $\rightarrow$ $x$ $\in$ $y$
$)$ $)$ $)$. Section~\ref{hierarchy} shows the hierarchy of definitions that
leads up to this expression.}\index{stylized epsilon ($\in$)}\index{omega
($\omega$)}  Actually, the primitive symbols don't even include the brace
notation.  The brace notation is a high-level definition, which you can find in
Section~\ref{hierarchy}.

Interestingly, the arithmetic of integers\index{integer} and
rationals\index{rational number} can be developed without appealing to the
existence of an infinite set, whereas the arithmetic of real
numbers\index{real number} requires it.

Each variable\index{variable!in set theory} in the axioms of set theory
represents an arbitrary set, and the axioms specify the legal kinds of things
you can do with these variables at a very primitive level.

Now, you may think that numbers and arithmetic are a lot more intuitive and
fundamental than sets and therefore should be the foundation of mathematics.
What is really the case is that you've dealt with numbers all your life and
are comfortable with a few rules for manipulating them such as addition and
multiplication.  Those rules only cover a small portion of what can be done
with numbers and only a very tiny fraction of the rest of mathematics.  If you
look at any elementary book on number theory, you will quickly become lost if
these are the only rules that you know.  Even though such books may present a
list of ``axioms''\index{axiom} for arithmetic, the ability to use the axioms
and to understand proofs of theorems\index{theorem} (facts) about numbers
requires an implicit mathematical talent that frustrates many people
from studying abstract mathematics.  The kind of mathematics that most people
know limits them to the practical, everyday usage of blindly manipulating
numbers and formulas, without any understanding of why those rules are correct
nor any ability to go any further.  For example, do you know why multiplying
two negative numbers yields a positive number?  Starting with set theory, you
will also start off blindly manipulating symbols according to the rules we give
you, but with the advantage that these rules will allow you, in principle, to
access {\em all} of mathematics, not just a tiny part of it.

Of course, concrete examples are often helpful in the learning process. For
example, you can verify that $2\cdot 3=3 \cdot 2$ by actually grouping
objects and can easily ``see'' how it generalizes to $x\cdot y = y\cdot x$,
even though you might not be able to rigorously prove it.  Similarly, in set
theory it can be helpful to understand how the axioms of set theory apply to
(and are correct for) small finite collections of objects.  You should be aware
that in set theory intuition can be misleading for infinite collections, and
rigorous proofs become more important.  For example, while $x\cdot y = y\cdot
x$ is correct for finite ordinals (which are the natural numbers), it is not
usually true for infinite ordinals.

\section{The Axioms for All of Mathematics}

In this section\index{axioms for mathematics}, we will show you the axioms
for all of standard mathematics (i.e.\ logic and set theory) as they are
traditionally presented.  The traditional presentation is useful for someone
with the mathematical experience needed to correctly manipulate high-level
abstract concepts.  For someone without this talent, knowing how to actually
make use of these axioms can be difficult.  The purpose of this section is to
allow you to see how the version of the axioms used in the standard
Metamath\index{Metamath} database \texttt{set.mm}\index{set
theory database (\texttt{set.mm})} relates to  the typical version
in textbooks, and also to give you an informal feel for them.

\subsection{Propositional Calculus}

Propositional calculus\index{propositional calculus} concerns itself with
statements that can be interpreted as either true or false.  Some examples of
statements (outside of mathematics) that are either true or false are ``It is
raining today'' and ``The United States has a female president.'' In
mathematics, as we mentioned, statements are really formulas.

In propositional calculus, we don't care what the statements are.  We also
treat a logical combination of statements, such as ``It is raining today and
the United States has a female president,'' no differently from a single
statement.  Statements and their combinations are called well-formed formulas
(wffs)\index{well-formed formula (wff)}.  We define wffs only in terms of
other wffs and don't define what a ``starting'' wff is.  As is common practice
in the literature, we use Greek letters to represent wffs.

Specifically, suppose $\varphi$ and $\psi$ are wffs.  Then the combinations
$\varphi\rightarrow\psi$ (``$\varphi$ implies $\psi$,'' also read ``if
$\varphi$ then $\psi$'')\index{implication ($\rightarrow$)} and $\lnot\varphi$
(``not $\varphi$'')\index{negation ($\lnot$)} are also wffs.

The three axioms of propositional calculus\index{axioms of propositional
calculus} are all wffs of the following form:\footnote{A remarkable result of
C.~A.~Meredith\index{Meredith, C. A.} squeezes these three axioms into the
single axiom $((((\varphi\rightarrow \psi)\rightarrow(\neg \chi\rightarrow\neg
\theta))\rightarrow \chi )\rightarrow \tau)\rightarrow((\tau\rightarrow
\varphi)\rightarrow(\theta\rightarrow \varphi))$ \cite{CAMeredith},
which is believed to be the shortest possible.}
\begin{center}
     $\varphi\rightarrow(\psi\rightarrow \varphi)$\\

     $(\varphi\rightarrow (\psi\rightarrow \chi))\rightarrow
((\varphi\rightarrow  \psi)\rightarrow (\varphi\rightarrow \chi))$\\

     $(\neg \varphi\rightarrow \neg\psi)\rightarrow (\psi\rightarrow
\varphi)$
\end{center}

These three axioms are widely used.
They are attributed to Jan {\L}ukasiewicz
(pronounced woo-kah-SHAY-vitch) and was popularized by Alonzo Church,
who called it system P2. (Thanks to Ted Ulrich for this information.)

There are an infinite number of axioms, one for each possible
wff\index{well-formed formula (wff)} of the above form.  (For this reason,
axioms such as the above are often called ``axiom schemes.''\index{axiom
scheme})  Each Greek letter in the axioms may be substituted with a more
complex wff to result in another axiom.  For example, substituting
$\neg(\varphi\rightarrow\chi)$ for $\varphi$ in the first axiom yields
$\neg(\varphi\rightarrow\chi)\rightarrow(\psi\rightarrow
\neg(\varphi\rightarrow\chi))$, which is still an axiom.

To deduce new true statements (theorems\index{theorem}) from the axioms, a
rule\index{rule} called ``modus ponens''\index{modus ponens} is used.  This
rule states that if the wff $\varphi$ is an axiom or a theorem, and the wff
$\varphi\rightarrow\psi$ is an axiom or a theorem, then the wff $\psi$ is also
a theorem\index{theorem}.

As a non-mathematical example of modus ponens, suppose we have proved (or
taken as an axiom) ``Bob is a man'' and separately have proved (or taken as
an axiom) ``If Bob is a man, then Bob is a human.''  Using the rule of modus
ponens, we can logically deduce, ``Bob is a human.''

From Metamath's\index{Metamath} point of view, the axioms and the rule of
modus ponens just define a mechanical means for deducing new true statements
from existing true statements, and that is the complete content of
propositional calculus as far as Metamath is concerned.  You can read a logic
textbook to gain a better understanding of their meaning, or you can just let
their meaning slowly become apparent to you after you use them for a while.

It is actually rather easy to check to see if a formula is a theorem of
propositional calculus.  Theorems of propositional calculus are also called
``tautologies.''\index{tautology}  The technique to check whether a formula is
a tautology is called the ``truth table method,''\index{truth table} and it
works like this.  A wff $\varphi\rightarrow\psi$ is false whenever $\varphi$ is true
and $\psi$ is false.  Otherwise it is true.  A wff $\lnot\varphi$ is false
whenever $\varphi$ is true and false otherwise. To verify a tautology such as
$\varphi\rightarrow(\psi\rightarrow \varphi)$, you break it down into sub-wffs and
construct a truth table that accounts for all possible combinations of true
and false assigned to the wff metavariables:
\begin{center}\begin{tabular}{|c|c|c|c|}\hline
\mbox{$\varphi$} & \mbox{$\psi$} & \mbox{$\psi\rightarrow\varphi$}
    & \mbox{$\varphi\rightarrow(\psi\rightarrow \varphi)$} \\ \hline \hline
              T   &  T    &      T       &        T    \\ \hline
              T   &  F    &      T       &        T    \\ \hline
              F   &  T    &      F       &        T    \\ \hline
              F   &  F    &      T       &        T    \\ \hline
\end{tabular}\end{center}
If all entries in the last column are true, the formula is a tautology.

Now, the truth table method doesn't tell you how to prove the tautology from
the axioms, but only that a proof exists.  Finding an actual proof (especially
one that is short and elegant) can be challenging.  Methods do exist for
automatically generating proofs in propositional calculus, but the proofs that
result can sometimes be very long.  In the Metamath \texttt{set.mm}\index{set
theory database (\texttt{set.mm})} database, most
or all proofs were created manually.

Section \ref{metadefprop} discusses various definitions
that make propositional calculus easier to use.
For example, we define:

\begin{itemize}
\item $\varphi \vee \psi$
  is true if either $\varphi$ or $\psi$ (or both) are true
  (this is disjunction\index{disjunction ($\vee$)}
  aka logical {\sc or}\index{logical {\sc or} ($\vee$)}).

\item $\varphi \wedge \psi$
  is true if both $\varphi$ and $\psi$ are true
  (this is conjunction\index{conjunction ($\wedge$)}
  aka logical {\sc and}\index{logical {\sc and} ($\wedge$)}).

\item $\varphi \leftrightarrow \psi$
  is true if $\varphi$ and $\psi$ have the same value, that is,
  they are both true or both false
  (this is the biconditional\index{biconditional ($\leftrightarrow$)}).
\end{itemize}

\subsection{Predicate Calculus}

Predicate calculus\index{predicate calculus} introduces the concept of
``individual variables,''\index{variable!in predicate calculus}\index{individual
variable} which
we will usually just call ``variables.''
These variables can represent something other than true or false (wffs),
and will always represent sets when we get to set theory.  There are also
three new symbols $\forall$\index{universal quantifier ($\forall$)},
$=$\index{equality ($=$)}, and $\in$\index{stylized epsilon ($\in$)},
read ``for all,'' ``equals,'' and ``is an element of''
respectively.  We will represent variables with the letters $x$, $y$, $z$, and
$w$, as is common practice in the literature.
For example, $\forall x \varphi$ means ``for all possible values of
$x$, $\varphi$ is true.''

In predicate calculus, we extend the definition of a wff\index{well-formed
formula (wff)}.  If $\varphi$ is a wff and $x$ and $y$ are variables, then
$\forall x \, \varphi$, $x=y$, and $x\in y$ are wffs. Note that these three new
types of wffs can be considered ``starting'' wffs from which we can build
other wffs with $\rightarrow$ and $\neg$ .  The concept of a starting wff was
absent in propositional calculus.  But starting wff or not, all we are really
concerned with is whether our wffs are correctly constructed according to
these mechanical rules.

A quick aside:
To prevent confusion, it might be best at this point to think of the variables
of Metamath\index{Metamath} as ``metavariables,''\index{metavariable} because
they are not quite the same as the variables we are introducing here.  A
(meta)variable in Metamath can be a wff or an individual variable, as well
as many other things; in general, it represents a kind of place holder for an
unspecified sequence of math symbols\index{math symbol}.

Unlike propositional calculus, no decision procedure\index{decision procedure}
analogous to the truth table method exists (nor theoretically can exist) that
will definitely determine whether a formula is a theorem of predicate
calculus.  Much of the work in the field of automated theorem
proving\index{automated theorem proving} has been dedicated to coming up with
clever heuristics for proving theorems of predicate calculus, but they can
never be guaranteed to work always.

Section \ref{metadefpred} discusses various definitions
that make predicate calculus easier to use.
For example, we define
$\exists x \varphi$ to mean
``there exists at least one possible value of $x$ where $\varphi$ is true.''

We now turn to looking at how predicate calculus can be formally
represented.

\subsubsection{Common Axioms}

There is a new rule of inference in predicate calculus:  if $\varphi$ is
an axiom or a theorem, then $\forall x \,\varphi$ is also a
theorem\index{theorem}.  This is called the rule of
``generalization.''\index{rule of generalization}
This is easily represented in Metamath.

In standard texts of logic, there are often two axioms of predicate
calculus\index{axioms of predicate calculus}:
\begin{center}
  $\forall x \,\varphi ( x ) \rightarrow \varphi ( y )$,
      where ``$y$ is properly substituted for $x$.''\\
  $\forall x ( \varphi \rightarrow \psi )\rightarrow ( \varphi \rightarrow
    \forall x\, \psi )$,
    where ``$x$ is not free in $\varphi$.''
\end{center}

Now at first glance, this seems simple:  just two axioms.  However,
conditional clauses are attached to each axiom describing requirements that
may seem puzzling to you.  In addition, the first axiom puts a variable symbol
in parentheses after each wff, seemingly violating our definition of a
wff\index{well-formed formula (wff)}; this is just an informal way of
referring to some arbitrary variable that may occur in the wff.  The
conditional clauses do, of course, have a precise meaning, but as it turns out
the precise meaning is somewhat complicated and awkward to formalize in a
way that a computer can handle easily.  Unlike propositional calculus, a
certain amount of mathematical sophistication and practice is needed to be
able to easily grasp and manipulate these concepts correctly.

Predicate calculus may be presented with or without axioms for
equality\index{axioms of equality}\index{equality ($=$)}. We will require the
axioms of equality as a prerequisite for the version of set theory we will
use.  The axioms for equality, when included, are often represented using these
two axioms:
\begin{center}
$x=x$\\ \ \\
$x=y\rightarrow (\varphi(x,x)\rightarrow\varphi(x,y))$ where ``$\varphi(x,y)$
   arises from $\varphi(x,x)$ by replacing some, but not necessarily all,
   free\index{free variable}
   occurrences of $x$ by $y$,\\ provided that $y$ is free for $x$
   in $\varphi(x,x)$.'' \end{center}
% (Mendelson p. 95)
The first equality axiom is simple, but again,
the condition on the second one is
somewhat awkward to implement on a computer.

\subsubsection{Tarski System S2}

Of course, we are not the first to notice the complications of these
predicate calculus axioms when being rigorous.

Well-known logician Alfred Tarski published in 1965
a system he called system S2\cite[p.~77]{Tarski1965}.
Tarski's system is \textit{exactly equivalent} to the traditional textbook
formalization, but (by clever use of equality axioms) it eliminates the
latter's primitive notions of ``proper substitution'' and ``free variable,''
replacing them with direct substitution and the notion of a variable
not occurring in a formula (which we express with distinct variable
constraints).

In advocating his system, Tarski wrote, ``The relatively complicated
character of [free variables and proper substitution] is a source
of certain inconveniences of both practical and theoretical nature;
this is clearly experienced both in teaching an elementary course of
mathematical logic and in formalizing the syntax of predicate logic for
some theoretical purposes''\cite[p.~61]{Tarski1965}\index{Tarski, Alfred}.

\subsubsection{Developing a Metamath Representation}

The standard textbook axioms of predicate calculus are somewhat
cumbersome to implement on a computer because of the complex notions of
``free variable''\index{free variable} and ``proper
substitution.''\index{proper substitution}\index{substitution!proper}
While it is possible to use the Metamath\index{Metamath} language to
implement these concepts, we have chosen not to implement them
as primitive constructs in the
\texttt{set.mm} set theory database.  Instead, we have eliminated them
within the axioms
by carefully crafting the axioms so as to avoid them,
building on Tarski's system S2.  This makes it
easy for a beginner to follow the steps in a proof without knowing any
advanced concepts other than the simple concept of
replacing\index{substitution!variable}\index{variable substitution}
variables with expressions.

In order to develop the concepts of free variable and proper
substitution from the axioms, we use an additional
Metamath statement type called ``disjoint variable
restriction''\index{disjoint variables} that we have not encountered
before.  In the context of the axioms, the statement \texttt{\$d} $ x\,
y$\index{\texttt{\$d} statement} simply means that $x$ and $y$ must be
distinct\index{distinct variables}, i.e.\ they may not be simultaneously
substituted\index{substitution!variable}\index{variable substitution}
with the same variable.  The statement \texttt{\$d} $ x\, \varphi$ means
variable $x$ must not occur in wff $\varphi$.  For the precise
definition of \texttt{\$d}, see Section~\ref{dollard}.

\subsubsection{Metamath representation}

The Metamath axiom system for predicate calculus
defined in set.mm uses Tarski's system S2.
As noted above, this has a different representation
than the traditional textbook formalization,
but it is \textit{exactly equivalent} to the textbook formalization,
and it is \textit{much} easier to work with.
This is reproduced as system S3 in Section 6 of
Megill's formalization \cite{Megill}\index{Megill, Norman}.

There is one exception, Tarski's axiom of existence,
which we label as axiom ax-6.
In the case of ax-6, Tarski's version is weaker because it includes a
distinct variable proviso. If we wish, we can also weaken our version
in this way and still have a metalogically complete system. Theorem
ax6 shows this by deriving, in the presence of the other axioms, our
ax-6 from Tarski's weaker version ax6v. However, we chose the stronger
version for our system because it is simpler to state and easier to use.

Tarski's system was designed for proving specific theorems rather than
more general theorem schemes. However, theorem schemes are much more
efficient than specific theorems for building a body of mathematical
knowledge, since they can be reused with different instances as
needed. While Tarski does derive some theorem schemes from his axioms,
their proofs require concepts that are ``outside'' of the system, such as
induction on formula length. The verification of such proofs is difficult
to automate in a proof verifier. (Specifically, Tarski treats the formulas
of his system as set-theoretical objects. In order to verify the proofs
of his theorem schemes, a proof verifier would need a significant amount
of set theory built into it.)

The Metamath axiom system for predicate calculus extends
Tarski's system to eliminate this difficulty. The additional
``auxilliary'' axiom
schemes (as we will call them in this section; see below) endow Tarski's
system with a nice property we call
metalogical completeness \cite[Remark 9.6]{Megill}\index{Megill, Norman}.
As a result, we can prove any theorem scheme
expressable in the ``simple metalogic'' of Tarski's system by using
only Metamath's direct substitution rule applied to the axiom system
(and no other metalogical or set-theoretical notions ``outside'' of the
system). Simple metalogic consists of schemes containing wff metavariables
(with no arguments) and/or set (also called ``individual'') metavariables,
accompanied by optional provisos each stating that two specified set
metavariables must be distinct or that a specified set metavariable may
not occur in a specified wff metavariable. Metamath's logic and set theory
axiom and rule schemes are all examples of simple metalogic. The schemes
of traditional predicate calculus with equality are examples which are
not simple metalogic, because they use wff metavariables with arguments
and have ``free for'' and ``not free in'' side conditions.

A rigorous justification for this system, using an older but
exactly equivalent set of axioms, can be
found in \cite{Megill}\index{Megill, Norman}.

This allows us to
take a different approach in the Metamath\index{Metamath} database
\texttt{set.mm}\index{set theory database (\texttt{set.mm})}.  We do not
directly use the primitive notions of ``free variable''\index{free variable}
and ``proper substitution''\index{proper
substitution}\index{substitution!proper} at all as primitive constructs.
Instead, we use a set
of axioms that are almost as simple to manipulate as those of
propositional calculus.  Our axiom system avoids complex primitive
notions by effectively embedding the complexity into the axioms
themselves.  As a result, we will end up with a larger number of axioms,
but they are ideally suited for a computer language such as Metamath.
(Section~\ref{metaaxioms} shows these axioms.)

We will not elaborate further
on the ``free variable'' and ``proper substitution''
concepts here.  You may consult
\cite[ch.\ 3--4]{Hamilton}\index{Hamilton, Alan G.} (as well as
many other books) for a precise explanation
of these concepts.  If you intend to do serious mathematical work, it is wise
to become familiar with the traditional textbook approach; even though the
concepts embedded in their axioms require a higher level of sophistication,
they can be more practical to deal with on an everyday, informal basis.  Even
if you are just developing Metamath proofs, familiarity with the traditional
approach can help you arrive at a proof outline much faster, which you can
then convert to the detail required by Metamath.

We do develop proper substitution rules later on, but in set.mm
they are defined as derived constructs; they are not primitives.

You should also note that our system of predicate calculus is specifically
tailored for set theory; thus there are only two specific predicates $=$ and
$\in$ and no functions\index{function!in predicate calculus}
or constants\index{constant!in predicate calculus} unlike more general systems.
We later add these.

\subsection{Set Theory}

Traditional Zermelo--Fraenkel set theory\index{Zermelo--Fraenkel set
theory}\index{set theory} with the Axiom of Choice
has 10 axioms, which can be expressed in the
language of predicate calculus.  In this section, we will list only the
names and brief English descriptions of these axioms, since we will give
you the precise formulas used by the Metamath\index{Metamath} set theory
database \texttt{set.mm} later on.

In the descriptions of the axioms, we assume that $x$, $y$, $z$, $w$, and $v$
represent sets.  These are the same as the variables\index{variable!in set
theory} in our predicate calculus system above, except that now we informally
think of the variables as ranging over sets.  Note that the terms
``object,''\index{object} ``set,''\index{set} ``element,''\index{element}
``collection,''\index{collection} and ``family''\index{family} are synonymous,
as are ``is an element of,'' ``is a member of,''\index{member} ``is contained
in,'' and ``belongs to.''  The different terms are used for convenience; for
example, ``a collection of sets'' is less confusing than ``a set of sets.''
A set $x$ is said to be a ``subset''\index{subset} of $y$ if every element of
$x$ is also an element of $y$; we also say $x$ is ``included in''
$y$.

The axioms are very general and apply to almost any conceivable mathematical
object, and this level of abstraction can be overwhelming at first.  To gain an
intuitive feel, it can be helpful to draw a picture illustrating the concept;
for example, a circle containing dots could represent a collection of sets,
and a smaller circle drawn inside the circle could represent a subset.
Overlapping circles can illustrate intersection and union.  Circles that
illustrate the concepts of set theory are frequently used in elementary
textbooks and are called Venn diagrams\index{Venn diagram}.\index{axioms of
set theory}

1. Axiom of Extensionality:  Two sets are identical if they contain the same
   elements.\index{Axiom of Extensionality}

2. Axiom of Pairing:  The set $\{ x , y \}$ exists.\index{Axiom of Pairing}

3. Axiom of Power Sets:  The power set of a set (the collection of all of
   its subsets) exists.  For example, the power set of $\{x,y\}$ is
   $\{\varnothing,\{x\},\{y\},\{x,y\}\}$ and it exists.\index{Axiom
of Power Sets}

4. Axiom of the Null Set:  The empty set $\varnothing$ exists.\index{Axiom of
the Null Set}

5. Axiom of Union:  The union of a set (the set containing the elements of
   its members) exists.  For example, the union of $\{\{x,y\},\{z\}\}$ is
 $\{x,y,z\}$ and
   it exists.\index{Axiom of Union}

6. Axiom of Regularity:  Roughly, no set can contain itself, nor can there
   be membership ``loops,'' such as a set being an
   element of one of its members.\index{Axiom of Regularity}

7. Axiom of Infinity:  An infinite set exists.  An example of an infinite
   set is the set of all
   integers.\index{Axiom of Infinity}

8. Axiom of Separation:  The set exists that is obtained by restricting $x$
   with some property.  For example, if the set of all integers exists,
   then the set of all even integers exists.\index{Axiom of Separation}

9. Axiom of Replacement:  The range of a function whose domain is restricted
   to the elements of a set $x$, is also a set.  For example, there
   is a function
   from integers (the function's domain) to their squares (its
   range).  If we
   restrict the domain to even integers, its range will become the set of
   squares of even integers, so this axiom asserts that the set of
    squares of even numbers exists.  Technical note:  In general, the
   ``function'' need not be a set but can be a proper class.
   \index{Axiom of Replacement}

10. Axiom of Choice:  Let $x$ be a set whose members are pairwise
  disjoint\index{disjoint sets} (i.e,
  whose members contain no elements in common).  Then there exists another
  set containing one element from each member of $x$.  For
  example, if $x$ is
  $\{\{y,z\},\{w,v\}\}$, where $y$, $z$, $w$, and $v$ are
  different sets, then a set such as $\{z,w\}$
  exists (but the axiom doesn't tell
  us which one).  (Actually the Axiom
  of Choice is redundant if the set $x$, as in this example, has a finite
  number of elements.)\index{Axiom of Choice}

The Axiom of Choice is usually considered an extension of ZF set theory rather
than a proper part of it.  It is sometimes considered philosophically
controversial because it specifies the existence of a set without specifying
what the set is. Constructive logics, including intuitionistic logic,
do not accept the axiom of choice.
Since there is some lingering controversy, we often prefer proofs that do
not use the axiom of choice (where there is a known alternative), and
in some cases we will use weaker axioms than the full axiom of choice.
That said, the axiom of choice is a powerful and widely-accepted tool,
so we do use it when needed.
ZF set theory that includes the Axiom of Choice is
called Zermelo--Fraenkel set theory with choice (ZFC\index{ZFC set theory}).

When expressed symbolically, the Axiom of Separation and the Axiom of
Replacement contain wff symbols and therefore each represent infinitely many
axioms, one for each possible wff. For this reason, they are often called
axiom schemes\index{axiom scheme}\index{well-formed formula (wff)}.

It turns out that the Axiom of the Null Set, the Axiom of Pairing, and the
Axiom of Separation can be derived from the other axioms and are therefore
unnecessary, although they tend to be included in standard texts for various
reasons (historical, philosophical, and possibly because some authors may not
know this).  In the Metamath\index{Metamath} set theory database, these
redundant axioms are derived from the other ones instead of truly
being considered axioms.
This is in keeping with our general goal of minimizing the number of
axioms we must depend on.

\subsection{Other Axioms}

Above we qualified the phrase "all of mathematics" with "essentially."
The main important missing piece is the ability to do category theory,
which requires huge sets (inaccessible cardinals) larger than those
postulated by the ZFC axioms. The Tarski--Grothendieck Axiom postulates
the existence of such sets.
Note that this is the same axiom used by Mizar for supporting
category theory.
The Tarski--Grothendieck axiom
can be viewed as a very strong replacement of the Axiom of Infinity,
the Axiom of Choice, and the Axiom of Power Sets.
The \texttt{set.mm} database includes this axiom; see the database
for details about it.
Again, we only use this axiom when we need to.
You are only likely to encounter or use this axiom if you are doing
category theory, since its use is highly specialized,
so we will not list the Tarsky-Grothendieck axiom
in the short list of axioms below.

Can there be even more axioms?
Of course.
G\"{o}del showed that no finite set of axioms or axiom schemes can completely
describe any consistent theory strong enough to include arithmetic.
But practically speaking, the ones above are the accepted foundation that
almost all mathematicians explicitly or implicitly base their work on.

\section{The Axioms in the Metamath Language}\label{metaaxioms}

Here we list the axioms as they appear in
\texttt{set.mm}\index{set theory database (\texttt{set.mm})} so you can
look them up there easily.  Incidentally, the \texttt{show statement
/tex} command\index{\texttt{show statement} command} was used to
typeset them.

%macros from show statement /tex
\newbox\mlinebox
\newbox\mtrialbox
\newbox\startprefix  % Prefix for first line of a formula
\newbox\contprefix  % Prefix for continuation line of a formula
\def\startm{  % Initialize formula line
  \setbox\mlinebox=\hbox{\unhcopy\startprefix}
}
\def\m#1{  % Add a symbol to the formula
  \setbox\mtrialbox=\hbox{\unhcopy\mlinebox $\,#1$}
  \ifdim\wd\mtrialbox>\hsize
    \box\mlinebox
    \setbox\mlinebox=\hbox{\unhcopy\contprefix $\,#1$}
  \else
    \setbox\mlinebox=\hbox{\unhbox\mtrialbox}
  \fi
}
\def\endm{  % Output the last line of a formula
  \box\mlinebox
}

% \SLASH for \ , \TOR for \/ (text OR), \TAND for /\ (text and)
% This embeds a following forced space to force the space.
\newcommand\SLASH{\char`\\~}
\newcommand\TOR{\char`\\/~}
\newcommand\TAND{/\char`\\~}
%
% Macro to output metamath raw text.
% This assumes \startprefix and \contprefix are set.
% NOTE: "\" is tricky to escape, use \SLASH, \TOR, and \TAND inside.
% Any use of "$ { ~ ^" must be escaped; ~ and ^ must be escaped specially.
% We escape { and } for consistency.
% For more about how this macro written, see:
% https://stackoverflow.com/questions/4073674/
% how-to-disable-indentation-in-particular-section-in-latex/4075706
% Use frenchspacing, or "e." will get an extra space after it.
\newlength\mystoreparindent
\newlength\mystorehangindent
\newenvironment{mmraw}{%
\setlength{\mystoreparindent}{\the\parindent}
\setlength{\mystorehangindent}{\the\hangindent}
\setlength{\parindent}{0pt} % TODO - we'll put in the \startprefix instead
\setlength{\hangindent}{\wd\the\contprefix}
\begin{flushleft}
\begin{frenchspacing}
\begin{tt}
{\unhcopy\startprefix}%
}{%
\end{tt}
\end{frenchspacing}
\end{flushleft}
\setlength{\parindent}{\mystoreparindent}
\setlength{\hangindent}{\mystorehangindent}
\vskip 1ex
}

\needspace{5\baselineskip}
\subsection{Propositional Calculus}\label{propcalc}\index{axioms of
propositional calculus}

\needspace{2\baselineskip}
Axiom of Simplification.\label{ax1}

\setbox\startprefix=\hbox{\tt \ \ ax-1\ \$a\ }
\setbox\contprefix=\hbox{\tt \ \ \ \ \ \ \ \ \ \ }
\startm
\m{\vdash}\m{(}\m{\varphi}\m{\rightarrow}\m{(}\m{\psi}\m{\rightarrow}\m{\varphi}\m{)}
\m{)}
\endm

\needspace{3\baselineskip}
\noindent Axiom of Distribution.

\setbox\startprefix=\hbox{\tt \ \ ax-2\ \$a\ }
\setbox\contprefix=\hbox{\tt \ \ \ \ \ \ \ \ \ \ }
\startm
\m{\vdash}\m{(}\m{(}\m{\varphi}\m{\rightarrow}\m{(}\m{\psi}\m{\rightarrow}\m{\chi}
\m{)}\m{)}\m{\rightarrow}\m{(}\m{(}\m{\varphi}\m{\rightarrow}\m{\psi}\m{)}\m{
\rightarrow}\m{(}\m{\varphi}\m{\rightarrow}\m{\chi}\m{)}\m{)}\m{)}
\endm

\needspace{2\baselineskip}
\noindent Axiom of Contraposition.

\setbox\startprefix=\hbox{\tt \ \ ax-3\ \$a\ }
\setbox\contprefix=\hbox{\tt \ \ \ \ \ \ \ \ \ \ }
\startm
\m{\vdash}\m{(}\m{(}\m{\lnot}\m{\varphi}\m{\rightarrow}\m{\lnot}\m{\psi}\m{)}\m{
\rightarrow}\m{(}\m{\psi}\m{\rightarrow}\m{\varphi}\m{)}\m{)}
\endm


\needspace{4\baselineskip}
\noindent Rule of Modus Ponens.\label{axmp}\index{modus ponens}

\setbox\startprefix=\hbox{\tt \ \ maj\ \$e\ }
\setbox\contprefix=\hbox{\tt \ \ \ \ \ \ \ \ \ }
\startm
\m{\vdash}\m{(}\m{\varphi}\m{\rightarrow}\m{\psi}\m{)}
\endm

\setbox\startprefix=\hbox{\tt \ \ min\ \$e\ }
\setbox\contprefix=\hbox{\tt \ \ \ \ \ \ \ \ \ }
\startm
\m{\vdash}\m{\varphi}
\endm

\setbox\startprefix=\hbox{\tt \ \ ax-mp\ \$a\ }
\setbox\contprefix=\hbox{\tt \ \ \ \ \ \ \ \ \ \ \ }
\startm
\m{\vdash}\m{\psi}
\endm


\needspace{7\baselineskip}
\subsection{Axioms of Predicate Calculus with Equality---Tarski's S2}\index{axioms of predicate calculus}

\needspace{3\baselineskip}
\noindent Rule of Generalization.\index{rule of generalization}

\setbox\startprefix=\hbox{\tt \ \ ax-g.1\ \$e\ }
\setbox\contprefix=\hbox{\tt \ \ \ \ \ \ \ \ \ \ \ \ }
\startm
\m{\vdash}\m{\varphi}
\endm

\setbox\startprefix=\hbox{\tt \ \ ax-gen\ \$a\ }
\setbox\contprefix=\hbox{\tt \ \ \ \ \ \ \ \ \ \ \ \ }
\startm
\m{\vdash}\m{\forall}\m{x}\m{\varphi}
\endm

\needspace{2\baselineskip}
\noindent Axiom of Quantified Implication.

\setbox\startprefix=\hbox{\tt \ \ ax-4\ \$a\ }
\setbox\contprefix=\hbox{\tt \ \ \ \ \ \ \ \ \ \ }
\startm
\m{\vdash}\m{(}\m{\forall}\m{x}\m{(}\m{\forall}\m{x}\m{\varphi}\m{\rightarrow}\m{
\psi}\m{)}\m{\rightarrow}\m{(}\m{\forall}\m{x}\m{\varphi}\m{\rightarrow}\m{
\forall}\m{x}\m{\psi}\m{)}\m{)}
\endm

\needspace{3\baselineskip}
\noindent Axiom of Distinctness.

% Aka: Add $d x ph $.
\setbox\startprefix=\hbox{\tt \ \ ax-5\ \$a\ }
\setbox\contprefix=\hbox{\tt \ \ \ \ \ \ \ \ \ \ }
\startm
\m{\vdash}\m{(}\m{\varphi}\m{\rightarrow}\m{\forall}\m{x}\m{\varphi}\m{)}\m{where}\m{ }\m{\$d}\m{ }\m{x}\m{ }\m{\varphi}\m{ }\m{(}\m{x}\m{ }\m{does}\m{ }\m{not}\m{ }\m{occur}\m{ }\m{in}\m{ }\m{\varphi}\m{)}
\endm

\needspace{2\baselineskip}
\noindent Axiom of Existence.

\setbox\startprefix=\hbox{\tt \ \ ax-6\ \$a\ }
\setbox\contprefix=\hbox{\tt \ \ \ \ \ \ \ \ \ \ }
\startm
\m{\vdash}\m{(}\m{\forall}\m{x}\m{(}\m{x}\m{=}\m{y}\m{\rightarrow}\m{\forall}
\m{x}\m{\varphi}\m{)}\m{\rightarrow}\m{\varphi}\m{)}
\endm

\needspace{2\baselineskip}
\noindent Axiom of Equality.

\setbox\startprefix=\hbox{\tt \ \ ax-7\ \$a\ }
\setbox\contprefix=\hbox{\tt \ \ \ \ \ \ \ \ \ \ }
\startm
\m{\vdash}\m{(}\m{x}\m{=}\m{y}\m{\rightarrow}\m{(}\m{x}\m{=}\m{z}\m{
\rightarrow}\m{y}\m{=}\m{z}\m{)}\m{)}
\endm

\needspace{2\baselineskip}
\noindent Axiom of Left Equality for Binary Predicate.

\setbox\startprefix=\hbox{\tt \ \ ax-8\ \$a\ }
\setbox\contprefix=\hbox{\tt \ \ \ \ \ \ \ \ \ \ \ }
\startm
\m{\vdash}\m{(}\m{x}\m{=}\m{y}\m{\rightarrow}\m{(}\m{x}\m{\in}\m{z}\m{
\rightarrow}\m{y}\m{\in}\m{z}\m{)}\m{)}
\endm

\needspace{2\baselineskip}
\noindent Axiom of Right Equality for Binary Predicate.

\setbox\startprefix=\hbox{\tt \ \ ax-9\ \$a\ }
\setbox\contprefix=\hbox{\tt \ \ \ \ \ \ \ \ \ \ \ }
\startm
\m{\vdash}\m{(}\m{x}\m{=}\m{y}\m{\rightarrow}\m{(}\m{z}\m{\in}\m{x}\m{
\rightarrow}\m{z}\m{\in}\m{y}\m{)}\m{)}
\endm


\needspace{4\baselineskip}
\subsection{Axioms of Predicate Calculus with Equality---Auxiliary}\index{axioms of predicate calculus - auxiliary}

\needspace{2\baselineskip}
\noindent Axiom of Quantified Negation.

\setbox\startprefix=\hbox{\tt \ \ ax-10\ \$a\ }
\setbox\contprefix=\hbox{\tt \ \ \ \ \ \ \ \ \ \ }
\startm
\m{\vdash}\m{(}\m{\lnot}\m{\forall}\m{x}\m{\lnot}\m{\forall}\m{x}\m{\varphi}\m{
\rightarrow}\m{\varphi}\m{)}
\endm

\needspace{2\baselineskip}
\noindent Axiom of Quantifier Commutation.

\setbox\startprefix=\hbox{\tt \ \ ax-11\ \$a\ }
\setbox\contprefix=\hbox{\tt \ \ \ \ \ \ \ \ \ \ }
\startm
\m{\vdash}\m{(}\m{\forall}\m{x}\m{\forall}\m{y}\m{\varphi}\m{\rightarrow}\m{
\forall}\m{y}\m{\forall}\m{x}\m{\varphi}\m{)}
\endm

\needspace{3\baselineskip}
\noindent Axiom of Substitution.

\setbox\startprefix=\hbox{\tt \ \ ax-12\ \$a\ }
\setbox\contprefix=\hbox{\tt \ \ \ \ \ \ \ \ \ \ \ }
\startm
\m{\vdash}\m{(}\m{\lnot}\m{\forall}\m{x}\m{\,x}\m{=}\m{y}\m{\rightarrow}\m{(}
\m{x}\m{=}\m{y}\m{\rightarrow}\m{(}\m{\varphi}\m{\rightarrow}\m{\forall}\m{x}\m{(}
\m{x}\m{=}\m{y}\m{\rightarrow}\m{\varphi}\m{)}\m{)}\m{)}\m{)}
\endm

\needspace{3\baselineskip}
\noindent Axiom of Quantified Equality.

\setbox\startprefix=\hbox{\tt \ \ ax-13\ \$a\ }
\setbox\contprefix=\hbox{\tt \ \ \ \ \ \ \ \ \ \ \ }
\startm
\m{\vdash}\m{(}\m{\lnot}\m{\forall}\m{z}\m{\,z}\m{=}\m{x}\m{\rightarrow}\m{(}
\m{\lnot}\m{\forall}\m{z}\m{\,z}\m{=}\m{y}\m{\rightarrow}\m{(}\m{x}\m{=}\m{y}
\m{\rightarrow}\m{\forall}\m{z}\m{\,x}\m{=}\m{y}\m{)}\m{)}\m{)}
\endm

% \noindent Axiom of Quantifier Substitution
%
% \setbox\startprefix=\hbox{\tt \ \ ax-c11n\ \$a\ }
% \setbox\contprefix=\hbox{\tt \ \ \ \ \ \ \ \ \ \ \ }
% \startm
% \m{\vdash}\m{(}\m{\forall}\m{x}\m{\,x}\m{=}\m{y}\m{\rightarrow}\m{(}\m{\forall}
% \m{x}\m{\varphi}\m{\rightarrow}\m{\forall}\m{y}\m{\varphi}\m{)}\m{)}
% \endm
%
% \noindent Axiom of Distinct Variables. (This axiom requires
% that two individual variables
% be distinct\index{\texttt{\$d} statement}\index{distinct
% variables}.)
%
% \setbox\startprefix=\hbox{\tt \ \ \ \ \ \ \ \ \$d\ }
% \setbox\contprefix=\hbox{\tt \ \ \ \ \ \ \ \ \ \ \ }
% \startm
% \m{x}\m{\,}\m{y}
% \endm
%
% \setbox\startprefix=\hbox{\tt \ \ ax-c16\ \$a\ }
% \setbox\contprefix=\hbox{\tt \ \ \ \ \ \ \ \ \ \ \ }
% \startm
% \m{\vdash}\m{(}\m{\forall}\m{x}\m{\,x}\m{=}\m{y}\m{\rightarrow}\m{(}\m{\varphi}\m{
% \rightarrow}\m{\forall}\m{x}\m{\varphi}\m{)}\m{)}
% \endm

% \noindent Axiom of Quantifier Introduction (2).  (This axiom requires
% that the individual variable not occur in the
% wff\index{\texttt{\$d} statement}\index{distinct variables}.)
%
% \setbox\startprefix=\hbox{\tt \ \ \ \ \ \ \ \ \$d\ }
% \setbox\contprefix=\hbox{\tt \ \ \ \ \ \ \ \ \ \ \ }
% \startm
% \m{x}\m{\,}\m{\varphi}
% \endm
% \setbox\startprefix=\hbox{\tt \ \ ax-5\ \$a\ }
% \setbox\contprefix=\hbox{\tt \ \ \ \ \ \ \ \ \ \ \ }
% \startm
% \m{\vdash}\m{(}\m{\varphi}\m{\rightarrow}\m{\forall}\m{x}\m{\varphi}\m{)}
% \endm

\subsection{Set Theory}\label{mmsettheoryaxioms}

In order to make the axioms of set theory\index{axioms of set theory} a little
more compact, there are several definitions from logic that we make use of
implicitly, namely, ``logical {\sc and},''\index{conjunction ($\wedge$)}
\index{logical {\sc and} ($\wedge$)} ``logical equivalence,''\index{logical
equivalence ($\leftrightarrow$)}\index{biconditional ($\leftrightarrow$)} and
``there exists.''\index{existential quantifier ($\exists$)}

\begin{center}\begin{tabular}{rcl}
  $( \varphi \wedge \psi )$ &\mbox{stands for}& $\neg ( \varphi
     \rightarrow \neg \psi )$\\
  $( \varphi \leftrightarrow \psi )$& \mbox{stands
     for}& $( ( \varphi \rightarrow \psi ) \wedge
     ( \psi \rightarrow \varphi ) )$\\
  $\exists x \,\varphi$ &\mbox{stands for}& $\neg \forall x \neg \varphi$
\end{tabular}\end{center}

In addition, the axioms of set theory require that all variables be
dis\-tinct,\index{distinct variables}\footnote{Set theory axioms can be
devised so that {\em no} variables are required to be distinct,
provided we replace \texttt{ax-c16} with an axiom stating that ``at
least two things exist,'' thus
making \texttt{ax-5} the only other axiom requiring the
\texttt{\$d} statement.  These axioms are unconventional and are not
presented here, but they can be found on the \url{http://metamath.org}
web site.  See also the Comment on
p.~\pageref{nodd}.}\index{\texttt{\$d} statement} thus we also assume:
\begin{center}
  \texttt{\$d }$x\,y\,z\,w$
\end{center}

\needspace{2\baselineskip}
\noindent Axiom of Extensionality.\index{Axiom of Extensionality}

\setbox\startprefix=\hbox{\tt \ \ ax-ext\ \$a\ }
\setbox\contprefix=\hbox{\tt \ \ \ \ \ \ \ \ \ \ \ \ }
\startm
\m{\vdash}\m{(}\m{\forall}\m{x}\m{(}\m{x}\m{\in}\m{y}\m{\leftrightarrow}\m{x}
\m{\in}\m{z}\m{)}\m{\rightarrow}\m{y}\m{=}\m{z}\m{)}
\endm

\needspace{3\baselineskip}
\noindent Axiom of Replacement.\index{Axiom of Replacement}

\setbox\startprefix=\hbox{\tt \ \ ax-rep\ \$a\ }
\setbox\contprefix=\hbox{\tt \ \ \ \ \ \ \ \ \ \ \ \ }
\startm
\m{\vdash}\m{(}\m{\forall}\m{w}\m{\exists}\m{y}\m{\forall}\m{z}\m{(}\m{%
\forall}\m{y}\m{\varphi}\m{\rightarrow}\m{z}\m{=}\m{y}\m{)}\m{\rightarrow}\m{%
\exists}\m{y}\m{\forall}\m{z}\m{(}\m{z}\m{\in}\m{y}\m{\leftrightarrow}\m{%
\exists}\m{w}\m{(}\m{w}\m{\in}\m{x}\m{\wedge}\m{\forall}\m{y}\m{\varphi}\m{)}%
\m{)}\m{)}
\endm

\needspace{2\baselineskip}
\noindent Axiom of Union.\index{Axiom of Union}

\setbox\startprefix=\hbox{\tt \ \ ax-un\ \$a\ }
\setbox\contprefix=\hbox{\tt \ \ \ \ \ \ \ \ \ \ \ }
\startm
\m{\vdash}\m{\exists}\m{x}\m{\forall}\m{y}\m{(}\m{\exists}\m{x}\m{(}\m{y}\m{
\in}\m{x}\m{\wedge}\m{x}\m{\in}\m{z}\m{)}\m{\rightarrow}\m{y}\m{\in}\m{x}\m{)}
\endm

\needspace{2\baselineskip}
\noindent Axiom of Power Sets.\index{Axiom of Power Sets}

\setbox\startprefix=\hbox{\tt \ \ ax-pow\ \$a\ }
\setbox\contprefix=\hbox{\tt \ \ \ \ \ \ \ \ \ \ \ \ }
\startm
\m{\vdash}\m{\exists}\m{x}\m{\forall}\m{y}\m{(}\m{\forall}\m{x}\m{(}\m{x}\m{
\in}\m{y}\m{\rightarrow}\m{x}\m{\in}\m{z}\m{)}\m{\rightarrow}\m{y}\m{\in}\m{x}
\m{)}
\endm

\needspace{3\baselineskip}
\noindent Axiom of Regularity.\index{Axiom of Regularity}

\setbox\startprefix=\hbox{\tt \ \ ax-reg\ \$a\ }
\setbox\contprefix=\hbox{\tt \ \ \ \ \ \ \ \ \ \ \ \ }
\startm
\m{\vdash}\m{(}\m{\exists}\m{x}\m{\,x}\m{\in}\m{y}\m{\rightarrow}\m{\exists}
\m{x}\m{(}\m{x}\m{\in}\m{y}\m{\wedge}\m{\forall}\m{z}\m{(}\m{z}\m{\in}\m{x}\m{
\rightarrow}\m{\lnot}\m{z}\m{\in}\m{y}\m{)}\m{)}\m{)}
\endm

\needspace{3\baselineskip}
\noindent Axiom of Infinity.\index{Axiom of Infinity}

\setbox\startprefix=\hbox{\tt \ \ ax-inf\ \$a\ }
\setbox\contprefix=\hbox{\tt \ \ \ \ \ \ \ \ \ \ \ \ \ \ \ }
\startm
\m{\vdash}\m{\exists}\m{x}\m{(}\m{y}\m{\in}\m{x}\m{\wedge}\m{\forall}\m{y}%
\m{(}\m{y}\m{\in}\m{x}\m{\rightarrow}\m{\exists}\m{z}\m{(}\m{y}\m{\in}\m{z}\m{%
\wedge}\m{z}\m{\in}\m{x}\m{)}\m{)}\m{)}
\endm

\needspace{4\baselineskip}
\noindent Axiom of Choice.\index{Axiom of Choice}

\setbox\startprefix=\hbox{\tt \ \ ax-ac\ \$a\ }
\setbox\contprefix=\hbox{\tt \ \ \ \ \ \ \ \ \ \ \ \ \ \ }
\startm
\m{\vdash}\m{\exists}\m{x}\m{\forall}\m{y}\m{\forall}\m{z}\m{(}\m{(}\m{y}\m{%
\in}\m{z}\m{\wedge}\m{z}\m{\in}\m{w}\m{)}\m{\rightarrow}\m{\exists}\m{w}\m{%
\forall}\m{y}\m{(}\m{\exists}\m{w}\m{(}\m{(}\m{y}\m{\in}\m{z}\m{\wedge}\m{z}%
\m{\in}\m{w}\m{)}\m{\wedge}\m{(}\m{y}\m{\in}\m{w}\m{\wedge}\m{w}\m{\in}\m{x}%
\m{)}\m{)}\m{\leftrightarrow}\m{y}\m{=}\m{w}\m{)}\m{)}
\endm

\subsection{That's It}

There you have it, the axioms for (essentially) all of mathematics!
Wonder at them and stare at them in awe.  Put a copy in your wallet, and
you will carry in your pocket the encoding for all theorems ever proved
and that ever will be proved, from the most mundane to the most
profound.

\section{A Hierarchy of Definitions}\label{hierarchy}

The axioms in the previous section in principle embody everything that can be
done within standard mathematics.  However, it is impractical to accomplish
very much by using them directly, for even simple concepts (from a human
perspective) can involve extremely long, incomprehensible formulas.
Mathematics is made practical by introducing definitions\index{definition}.
Definitions usually introduce new symbols, or at least new relationships among
existing symbols, to abbreviate more complex formulas.  An important
requirement for a definition is that there exist a straightforward
(algorithmic) method for eliminating the abbreviation by expanding it into the
more primitive symbol string that it represents.  Some
important definitions included in
the file \texttt{set.mm} are listed in this section for reference, and also to
give you a feel for why something like $\omega$\index{omega ($\omega$)} (the
set of natural numbers\index{natural number} 0, 1, 2,\ldots) becomes very
complicated when completely expanded into primitive symbols.

What is the motivation for definitions, aside from allowing complicated
expressions to be expressed more simply?  In the case of  $\omega$, one goal is
to provide a basis for the theory of natural numbers.\index{natural number}
Before set theory was invented, a set of axioms for arithmetic, called Peano's
postulates\index{Peano's postulates}, was devised and shown to have the
properties one expects for natural numbers.  Now anyone can postulate a
set of axioms, but if the axioms are inconsistent contradictions can be derived
from them.  Once a contradiction is derived, anything can be trivially
proved, including
all the facts of arithmetic and their negations.  To ensure that an
axiom system is at least as reliable as the axioms for set theory, we can
define sets and operations on those sets that satisfy the new axioms. In the
\texttt{set.mm} Metamath database, we prove that the elements of $\omega$ satisfy
Peano's postulates, and it's a long and hard journey to get there directly
from the axioms of set theory.  But the result is confidence in the
foundations of arithmetic.  And there is another advantage:  we now have all
the tools of set theory at our disposal for manipulating objects that obey the
axioms for arithmetic.

What are the criteria we use for definitions?  First, and of utmost importance,
the definition should not be {\em creative}\index{creative
definition}\index{definition!creative}, that
is it should not allow an expression that previously qualified as a wff but
was not provable, to become provable.   Second, the definition should be {\em
eliminable}\index{definition!eliminability}, that is, there should exist an
algorithmic method for converting any expression using the definition into
a logically equivalent expression that previously qualified as a wff.

In almost all cases below, definitions connect two expressions with either
$\leftrightarrow$ or $=$.  Eliminating\footnote{Here we mean the
elimination that a human might do in his or her head.  To eliminate them as
part of a Metamath proof we would invoke one of a number of
theorems that deal with transitivity of equivalence or equality; there are
many such examples in the proofs in \texttt{set.mm}.} such a definition is a
simple matter of substituting the expression on the left-hand side ({\em
definiendum}\index{definiendum} or thing being defined) with the equivalent,
more primitive expression on the right-hand side ({\em
definiens}\index{definiens} or definition).

Often a definition has variables on the right-hand side which do not appear on
the left-hand side; these are called {\em dummy variables}.\index{dummy
variable!in definitions}  In this case, any
allowable substitution (such as a new, distinct
variable) can be used when the definition is eliminated.  Dummy variables may
be used only if they are {\em effectively bound}\index{effectively bound
variable}, meaning that the definition will remain logically equivalent upon
any substitution of a dummy variable with any other {\em qualifying
expression}\index{qualifying expression}, i.e.\ any symbol string (such as
another variable) that
meets the restrictions on the dummy variable imposed by \texttt{\$d} and
\texttt{\$f} statements.  For example, we could define a constant $\perp$
(inverted tee, meaning logical ``false'') as $( \varphi \wedge \lnot \varphi
)$, i.e.\ ``phi and not phi.''  Here $\varphi$ is effectively bound because the
definition remains logically equivalent when we replace $\varphi$ with any
other wff.  (It is actually \texttt{df-fal}
in \texttt{set.mm}, which defines $\perp$.)

There are two cases where eliminating definitions is a little more
complex.  These cases are the definitions \texttt{df-bi} and
\texttt{df-cleq}.  The first stretches the concept of a definition a
little, as in effect it ``defines a definition;'' however, it meets our
requirements for a definition in that it is eliminable and does not
strengthen the language.  Theorem \texttt{bii} shows the substitution
needed to eliminate the $\leftrightarrow$\index{logical equivalence
($\leftrightarrow$)}\index{biconditional ($\leftrightarrow$)} symbol.

Definition \texttt{df-cleq}\index{equality ($=$)} extends the usage of
the equality symbol to include ``classes''\index{class} in set theory.  The
reason it is potentially problematic is that it can lead to statements which
do not follow from logic alone but presuppose the Axiom of
Extensionality\index{Axiom of Extensionality}, so we include this axiom
as a hypothesis for the definition.  We could have made \texttt{df-cleq} directly
eliminable by introducing a new equality symbol, but have chosen not to do so
in keeping with standard textbook practice.  Definitions such as \texttt{df-cleq}
that extend the meaning of existing symbols must be introduced carefully so
that they do not lead to contradictions.  Definition \texttt{df-clel} also
extends the meaning of an existing symbol ($\in$); while it doesn't strengthen
the language like \texttt{df-cleq}, this is not obvious and it must also be
subject to the same scrutiny.

Exercise:  Study how the wff $x\in\omega$, meaning ``$x$ is a natural
number,'' could be expanded in terms of primitive symbols, starting with the
definitions \texttt{df-clel} on p.~\pageref{dfclel} and \texttt{df-om} on
p.~\pageref{dfom} and working your way back.  Don't bother to work out the
details; just make sure that you understand how you could do it in principle.
The answer is shown in the footnote on p.~\pageref{expandom}.  If you
actually do work it out, you won't get exactly the same answer because we used
a few simplifications such as discarding occurrences of $\lnot\lnot$ (double
negation).

In the definitions below, we have placed the {\sc ascii} Metamath source
below each of the formulas to help you become familiar with the
notation in the database.  For simplicity, the necessary \texttt{\$f}
and \texttt{\$d} statements are not shown.  If you are in doubt, use the
\texttt{show statement}\index{\texttt{show statement} command} command
in the Metamath program to see the full statement.
A selection of this notation is summarized in Appendix~\ref{ASCII}.

To understand the motivation for these definitions, you should consult the
references indicated:  Takeuti and Zaring \cite{Takeuti}\index{Takeuti, Gaisi},
Quine \cite{Quine}\index{Quine, Willard Van Orman}, Bell and Machover
\cite{Bell}\index{Bell, J. L.}, and Enderton \cite{Enderton}\index{Enderton,
Herbert B.}.  Our list of definitions is provided more for reference than as a
learning aid.  However, by looking at a few of them you can gain a feel for
how the hierarchy is built up.  The definitions are a representative sample of
the 130 or so in \texttt{set.mm}, but they are complete with respect to the
theorem examples we will present in Section~\ref{sometheorems}.  Also, some are
slightly different from, but logically equivalent to, the ones in \texttt{set.mm}
(some of which have been revised over time to shorten them, for example).

\subsection{Definitions for Propositional Calculus}\label{metadefprop}

The symbols $\varphi$, $\psi$, and $\chi$ represent wffs.

Our first definition introduces the biconditional
connective\footnote{The term ``connective'' is informally used to mean a
symbol that is placed between two variables or adjacent to a variable,
whereas a mathematical ``constant'' usually indicates a symbol such as
the number 0 that may replace a variable or metavariable.  From
Metamath's point of view, there is no distinction between a connective
and a constant; both are constants in the Metamath
language.}\index{connective}\index{constant} (also called logical
equivalence)\index{logical equivalence
($\leftrightarrow$)}\index{biconditional ($\leftrightarrow$)}.  Unlike
most traditional developments, we have chosen not to have a separate
symbol such as ``Df.'' to mean ``is defined as.''  Instead, we will use
the biconditional connective for this purpose, as it lets us use
logic to manipulate definitions directly.  Here we state the properties
of the biconditional connective with a carefully crafted \texttt{\$a}
statement, which effectively uses the biconditional connective to define
itself.  The $\leftrightarrow$ symbol can be eliminated from a formula
using theorem \texttt{bii}, which is derived later.

\vskip 2ex
\noindent Define the biconditional connective.\label{df-bi}

\vskip 0.5ex
\setbox\startprefix=\hbox{\tt \ \ df-bi\ \$a\ }
\setbox\contprefix=\hbox{\tt \ \ \ \ \ \ \ \ \ \ \ }
\startm
\m{\vdash}\m{\lnot}\m{(}\m{(}\m{(}\m{\varphi}\m{\leftrightarrow}\m{\psi}\m{)}%
\m{\rightarrow}\m{\lnot}\m{(}\m{(}\m{\varphi}\m{\rightarrow}\m{\psi}\m{)}\m{%
\rightarrow}\m{\lnot}\m{(}\m{\psi}\m{\rightarrow}\m{\varphi}\m{)}\m{)}\m{)}\m{%
\rightarrow}\m{\lnot}\m{(}\m{\lnot}\m{(}\m{(}\m{\varphi}\m{\rightarrow}\m{%
\psi}\m{)}\m{\rightarrow}\m{\lnot}\m{(}\m{\psi}\m{\rightarrow}\m{\varphi}\m{)}%
\m{)}\m{\rightarrow}\m{(}\m{\varphi}\m{\leftrightarrow}\m{\psi}\m{)}\m{)}\m{)}
\endm
\begin{mmraw}%
|- -. ( ( ( ph <-> ps ) -> -. ( ( ph -> ps ) ->
-. ( ps -> ph ) ) ) -> -. ( -. ( ( ph -> ps ) -> -. (
ps -> ph ) ) -> ( ph <-> ps ) ) ) \$.
\end{mmraw}

\noindent This theorem relates the biconditional connective to primitive
connectives and can be used to eliminate the $\leftrightarrow$ symbol from any
wff.

\vskip 0.5ex
\setbox\startprefix=\hbox{\tt \ \ bii\ \$p\ }
\setbox\contprefix=\hbox{\tt \ \ \ \ \ \ \ \ \ }
\startm
\m{\vdash}\m{(}\m{(}\m{\varphi}\m{\leftrightarrow}\m{\psi}\m{)}\m{\leftrightarrow}
\m{\lnot}\m{(}\m{(}\m{\varphi}\m{\rightarrow}\m{\psi}\m{)}\m{\rightarrow}\m{\lnot}
\m{(}\m{\psi}\m{\rightarrow}\m{\varphi}\m{)}\m{)}\m{)}
\endm
\begin{mmraw}%
|- ( ( ph <-> ps ) <-> -. ( ( ph -> ps ) -> -. ( ps -> ph ) ) ) \$= ... \$.
\end{mmraw}

\noindent Define disjunction ({\sc or}).\index{disjunction ($\vee$)}%
\index{logical or (vee)@logical {\sc or} ($\vee$)}%
\index{df-or@\texttt{df-or}}\label{df-or}

\vskip 0.5ex
\setbox\startprefix=\hbox{\tt \ \ df-or\ \$a\ }
\setbox\contprefix=\hbox{\tt \ \ \ \ \ \ \ \ \ \ \ }
\startm
\m{\vdash}\m{(}\m{(}\m{\varphi}\m{\vee}\m{\psi}\m{)}\m{\leftrightarrow}\m{(}\m{
\lnot}\m{\varphi}\m{\rightarrow}\m{\psi}\m{)}\m{)}
\endm
\begin{mmraw}%
|- ( ( ph \TOR ps ) <-> ( -. ph -> ps ) ) \$.
\end{mmraw}

\noindent Define conjunction ({\sc and}).\index{conjunction ($\wedge$)}%
\index{logical {\sc and} ($\wedge$)}%
\index{df-an@\texttt{df-an}}\label{df-an}

\vskip 0.5ex
\setbox\startprefix=\hbox{\tt \ \ df-an\ \$a\ }
\setbox\contprefix=\hbox{\tt \ \ \ \ \ \ \ \ \ \ \ }
\startm
\m{\vdash}\m{(}\m{(}\m{\varphi}\m{\wedge}\m{\psi}\m{)}\m{\leftrightarrow}\m{\lnot}
\m{(}\m{\varphi}\m{\rightarrow}\m{\lnot}\m{\psi}\m{)}\m{)}
\endm
\begin{mmraw}%
|- ( ( ph \TAND ps ) <-> -. ( ph -> -. ps ) ) \$.
\end{mmraw}

\noindent Define disjunction ({\sc or}) of 3 wffs.%
\index{df-3or@\texttt{df-3or}}\label{df-3or}

\vskip 0.5ex
\setbox\startprefix=\hbox{\tt \ \ df-3or\ \$a\ }
\setbox\contprefix=\hbox{\tt \ \ \ \ \ \ \ \ \ \ \ \ }
\startm
\m{\vdash}\m{(}\m{(}\m{\varphi}\m{\vee}\m{\psi}\m{\vee}\m{\chi}\m{)}\m{
\leftrightarrow}\m{(}\m{(}\m{\varphi}\m{\vee}\m{\psi}\m{)}\m{\vee}\m{\chi}\m{)}
\m{)}
\endm
\begin{mmraw}%
|- ( ( ph \TOR ps \TOR ch ) <-> ( ( ph \TOR ps ) \TOR ch ) ) \$.
\end{mmraw}

\noindent Define conjunction ({\sc and}) of 3 wffs.%
\index{df-3an}\label{df-3an}

\vskip 0.5ex
\setbox\startprefix=\hbox{\tt \ \ df-3an\ \$a\ }
\setbox\contprefix=\hbox{\tt \ \ \ \ \ \ \ \ \ \ \ \ }
\startm
\m{\vdash}\m{(}\m{(}\m{\varphi}\m{\wedge}\m{\psi}\m{\wedge}\m{\chi}\m{)}\m{
\leftrightarrow}\m{(}\m{(}\m{\varphi}\m{\wedge}\m{\psi}\m{)}\m{\wedge}\m{\chi}
\m{)}\m{)}
\endm

\begin{mmraw}%
|- ( ( ph \TAND ps \TAND ch ) <-> ( ( ph \TAND ps ) \TAND ch ) ) \$.
\end{mmraw}

\subsection{Definitions for Predicate Calculus}\label{metadefpred}

The symbols $x$, $y$, and $z$ represent individual variables of predicate
calculus.  In this section, they are not necessarily distinct unless it is
explicitly
mentioned.

\vskip 2ex
\noindent Define existential quantification.
The expression $\exists x \varphi$ means
``there exists an $x$ where $\varphi$ is true.''\index{existential quantifier
($\exists$)}\label{df-ex}

\vskip 0.5ex
\setbox\startprefix=\hbox{\tt \ \ df-ex\ \$a\ }
\setbox\contprefix=\hbox{\tt \ \ \ \ \ \ \ \ \ \ \ }
\startm
\m{\vdash}\m{(}\m{\exists}\m{x}\m{\varphi}\m{\leftrightarrow}\m{\lnot}\m{\forall}
\m{x}\m{\lnot}\m{\varphi}\m{)}
\endm
\begin{mmraw}%
|- ( E. x ph <-> -. A. x -. ph ) \$.
\end{mmraw}

\noindent Define proper substitution.\index{proper
substitution}\index{substitution!proper}\label{df-sb}
In our notation, we use $[ y / x ] \varphi$ to mean ``the wff that
results when $y$ is properly substituted for $x$ in the wff
$\varphi$.''\footnote{
This can also be described
as substituting $x$ with $y$, $y$ properly replaces $x$, or
$x$ is properly replaced by $y$.}
% This is elsb4, though it currently says: ( [ x / y ] z e. y <-> z e. x )
For example,
$[ y / x ] z \in x$ is the same as $z \in y$.
One way to remember this notation is to notice that it looks like division
and recall that $( y / x ) \cdot x $ is $y$ (when $x \neq 0$).
The notation is different from the notation $\varphi ( x | y )$
that is sometimes used, because the latter notation is ambiguous for us:
for example, we don't know whether $\lnot \varphi ( x | y )$ is to be
interpreted as $\lnot ( \varphi ( x | y ) )$ or
$( \lnot \varphi ) ( x | y )$.\footnote{Because of the way
we initially defined wffs, this is the case
with any postfix connective\index{postfix connective} (one occurring after the
symbols being connected) or infix connective\index{infix connective} (one
occurring between the symbols being connected).  Metamath does not have a
built-in notion of operator binding strength that could eliminate the
ambiguity.  The initial parenthesis effectively provides a prefix
connective\index{prefix connective} to eliminate ambiguity.  Some conventions,
such as Polish notation\index{Polish notation} used in the 1930's and 1940's
by Polish logicians, use only prefix connectives and thus allow the total
elimination of parentheses, at the expense of readability.  In Metamath we
could actually redefine all notation to be Polish if we wanted to without
having to change any proofs!}  Other texts often use $\varphi(y)$ to indicate
our $[ y / x ] \varphi$, but this notation is even more ambiguous since there is
no explicit indication of what is being substituted.
Note that this
definition is valid even when
$x$ and $y$ are the same variable.  The first conjunct is a ``trick'' used to
achieve this property, making the definition look somewhat peculiar at
first.

\vskip 0.5ex
\setbox\startprefix=\hbox{\tt \ \ df-sb\ \$a\ }
\setbox\contprefix=\hbox{\tt \ \ \ \ \ \ \ \ \ \ \ }
\startm
\m{\vdash}\m{(}\m{[}\m{y}\m{/}\m{x}\m{]}\m{\varphi}\m{\leftrightarrow}\m{(}%
\m{(}\m{x}\m{=}\m{y}\m{\rightarrow}\m{\varphi}\m{)}\m{\wedge}\m{\exists}\m{x}%
\m{(}\m{x}\m{=}\m{y}\m{\wedge}\m{\varphi}\m{)}\m{)}\m{)}
\endm
\begin{mmraw}%
|- ( [ y / x ] ph <-> ( ( x = y -> ph ) \TAND E. x ( x = y \TAND ph ) ) ) \$.
\end{mmraw}


\noindent Define existential uniqueness\index{existential uniqueness
quantifier ($\exists "!$)} (``there exists exactly one'').  Note that $y$ is a
variable distinct from $x$ and not occurring in $\varphi$.\label{df-eu}

\vskip 0.5ex
\setbox\startprefix=\hbox{\tt \ \ df-eu\ \$a\ }
\setbox\contprefix=\hbox{\tt \ \ \ \ \ \ \ \ \ \ \ }
\startm
\m{\vdash}\m{(}\m{\exists}\m{{!}}\m{x}\m{\varphi}\m{\leftrightarrow}\m{\exists}
\m{y}\m{\forall}\m{x}\m{(}\m{\varphi}\m{\leftrightarrow}\m{x}\m{=}\m{y}\m{)}\m{)}
\endm

\begin{mmraw}%
|- ( E! x ph <-> E. y A. x ( ph <-> x = y ) ) \$.
\end{mmraw}

\subsection{Definitions for Set Theory}\label{setdefinitions}

The symbols $x$, $y$, $z$, and $w$ represent individual variables of
predicate calculus, which in set theory are understood to be sets.
However, using only the constructs shown so far would be very inconvenient.

To make set theory more practical, we introduce the notion of a ``class.''
A class\index{class} is either a set variable (such as $x$) or an
expression of the form $\{ x | \varphi\}$ (called an ``abstraction
class''\index{abstraction class}\index{class abstraction}).  Note that
sets (i.e.\ individual variables) always exist (this is a theorem of
logic, namely $\exists y \, y = x$ for any set $x$), whereas classes may
or may not exist (i.e.\ $\exists y \, y = A$ may or may not be true).
If a class does not exist it is called a ``proper class.''\index{proper
class}\index{class!proper} Definitions \texttt{df-clab},
\texttt{df-cleq}, and \texttt{df-clel} can be used to convert an
expression containing classes into one containing only set variables and
wff metavariables.

The symbols $A$, $B$, $C$, $D$, $ F$, $G$, and $R$ are metavariables that range
over classes.  A class metavariable $A$ may be eliminated from a wff by
replacing it with $\{ x|\varphi\}$ where neither $x$ nor $\varphi$ occur in
the wff.

The theory of classes can be shown to be an eliminable and conservative
extension of set theory. The \textbf{eliminability}
property shows that for every
formula in the extended language we can build a logically equivalent
formula in the basic language; so that even if the extended language
provides more ease to convey and formulate mathematical ideas for set
theory, its expressive power does not in fact strengthen the basic
language's expressive power.
The \textbf{conservation} property shows that for
every proof of a formula of the basic language in the extended system
we can build another proof of the same formula in the basic system;
so that, concerning theorems on sets only, the deductive powers of
the extended system and of the basic system are identical. Together,
these properties mean that the extended language can be treated as a
definitional extension that is \textbf{sound}.

A rigorous justification, which we will not give here, can be found in
Levy \cite[pp.~357-366]{Levy} supplementing his informal introduction to class
theory on pp.~7-17. Two other good treatments of class theory are provided
by Quine \cite[pp.~15-21]{Quine}\index{Quine, Willard Van Orman}
and also \cite[pp.~10-14]{Takeuti}\index{Takeuti, Gaisi}.
Quine's exposition (he calls them virtual classes)
is nicely written and very readable.

In the rest of this
section, individual variables are always assumed to be distinct from
each other unless otherwise indicated.  In addition, dummy variables on the
right-hand side of a definition do not occur in the class and wff
metavariables in the definition.

The definitions we present here are a partial but self-contained
collection selected from several hundred that appear in the current
\texttt{set.mm} database.  They are adequate for a basic development of
elementary set theory.

\vskip 2ex
\noindent Define the abstraction class.\index{abstraction class}\index{class
abstraction}\label{df-clab}  $x$ and $y$
need not be distinct.  Definition 2.1 of Quine, p.~16.  This definition may
seem puzzling since it is shorter than the expression being defined and does not
buy us anything in terms of brevity.  The reason we introduce this definition
is because it fits in neatly with the extension of the $\in$ connective
provided by \texttt{df-clel}.

\vskip 0.5ex
\setbox\startprefix=\hbox{\tt \ \ df-clab\ \$a\ }
\setbox\contprefix=\hbox{\tt \ \ \ \ \ \ \ \ \ \ \ \ \ }
\startm
\m{\vdash}\m{(}\m{x}\m{\in}\m{\{}\m{y}\m{|}\m{\varphi}\m{\}}\m{%
\leftrightarrow}\m{[}\m{x}\m{/}\m{y}\m{]}\m{\varphi}\m{)}
\endm
\begin{mmraw}%
|- ( x e. \{ y | ph \} <-> [ x / y ] ph ) \$.
\end{mmraw}

\noindent Define the equality connective between classes\index{class
equality}\label{df-cleq}.  See Quine or Chapter 4 of Takeuti and Zaring for its
justification and methods for eliminating it.  This is an example of a
somewhat ``dangerous'' definition, because it extends the use of the
existing equality symbol rather than introducing a new symbol, allowing
us to make statements in the original language that may not be true.
For example, it permits us to deduce $y = z \leftrightarrow \forall x (
x \in y \leftrightarrow x \in z )$ which is not a theorem of logic but
rather presupposes the Axiom of Extensionality,\index{Axiom of
Extensionality} which we include as a hypothesis so that we can know
when this axiom is assumed in a proof (with the \texttt{show
trace{\char`\_}back} command).  We could avoid the danger by introducing
another symbol, say $\eqcirc$, in place of $=$; this
would also have the advantage of making elimination of the definition
straightforward and would eliminate the need for Extensionality as a
hypothesis.  We would then also have the advantage of being able to
identify exactly where Extensionality truly comes into play.  One of our
theorems would be $x \eqcirc y \leftrightarrow x = y$
by invoking Extensionality.  However in keeping with standard practice
we retain the ``dangerous'' definition.

\vskip 0.5ex
\setbox\startprefix=\hbox{\tt \ \ df-cleq.1\ \$e\ }
\setbox\contprefix=\hbox{\tt \ \ \ \ \ \ \ \ \ \ \ \ \ \ \ }
\startm
\m{\vdash}\m{(}\m{\forall}\m{x}\m{(}\m{x}\m{\in}\m{y}\m{\leftrightarrow}\m{x}
\m{\in}\m{z}\m{)}\m{\rightarrow}\m{y}\m{=}\m{z}\m{)}
\endm
\setbox\startprefix=\hbox{\tt \ \ df-cleq\ \$a\ }
\setbox\contprefix=\hbox{\tt \ \ \ \ \ \ \ \ \ \ \ \ \ }
\startm
\m{\vdash}\m{(}\m{A}\m{=}\m{B}\m{\leftrightarrow}\m{\forall}\m{x}\m{(}\m{x}\m{
\in}\m{A}\m{\leftrightarrow}\m{x}\m{\in}\m{B}\m{)}\m{)}
\endm
% We need to reset the startprefix and contprefix.
\setbox\startprefix=\hbox{\tt \ \ df-cleq.1\ \$e\ }
\setbox\contprefix=\hbox{\tt \ \ \ \ \ \ \ \ \ \ \ \ \ \ \ }
\begin{mmraw}%
|- ( A. x ( x e. y <-> x e. z ) -> y = z ) \$.
\end{mmraw}
\setbox\startprefix=\hbox{\tt \ \ df-cleq\ \$a\ }
\setbox\contprefix=\hbox{\tt \ \ \ \ \ \ \ \ \ \ \ \ \ }
\begin{mmraw}%
|- ( A = B <-> A. x ( x e. A <-> x e. B ) ) \$.
\end{mmraw}

\noindent Define the membership connective between classes\index{class
membership}.  Theorem 6.3 of Quine, p.~41, which we adopt as a definition.
Note that it extends the use of the existing membership symbol, but unlike
{\tt df-cleq} it does not extend the set of valid wffs of logic when the class
metavariables are replaced with set variables.\label{dfclel}\label{df-clel}

\vskip 0.5ex
\setbox\startprefix=\hbox{\tt \ \ df-clel\ \$a\ }
\setbox\contprefix=\hbox{\tt \ \ \ \ \ \ \ \ \ \ \ \ \ }
\startm
\m{\vdash}\m{(}\m{A}\m{\in}\m{B}\m{\leftrightarrow}\m{\exists}\m{x}\m{(}\m{x}
\m{=}\m{A}\m{\wedge}\m{x}\m{\in}\m{B}\m{)}\m{)}
\endm
\begin{mmraw}%
|- ( A e. B <-> E. x ( x = A \TAND x e. B ) ) \$.?
\end{mmraw}

\noindent Define inequality.

\vskip 0.5ex
\setbox\startprefix=\hbox{\tt \ \ df-ne\ \$a\ }
\setbox\contprefix=\hbox{\tt \ \ \ \ \ \ \ \ \ \ \ }
\startm
\m{\vdash}\m{(}\m{A}\m{\ne}\m{B}\m{\leftrightarrow}\m{\lnot}\m{A}\m{=}\m{B}%
\m{)}
\endm
\begin{mmraw}%
|- ( A =/= B <-> -. A = B ) \$.
\end{mmraw}

\noindent Define restricted universal quantification.\index{universal
quantifier ($\forall$)!restricted}  Enderton, p.~22.

\vskip 0.5ex
\setbox\startprefix=\hbox{\tt \ \ df-ral\ \$a\ }
\setbox\contprefix=\hbox{\tt \ \ \ \ \ \ \ \ \ \ \ \ }
\startm
\m{\vdash}\m{(}\m{\forall}\m{x}\m{\in}\m{A}\m{\varphi}\m{\leftrightarrow}\m{%
\forall}\m{x}\m{(}\m{x}\m{\in}\m{A}\m{\rightarrow}\m{\varphi}\m{)}\m{)}
\endm
\begin{mmraw}%
|- ( A. x e. A ph <-> A. x ( x e. A -> ph ) ) \$.
\end{mmraw}

\noindent Define restricted existential quantification.\index{existential
quantifier ($\exists$)!restricted}  Enderton, p.~22.

\vskip 0.5ex
\setbox\startprefix=\hbox{\tt \ \ df-rex\ \$a\ }
\setbox\contprefix=\hbox{\tt \ \ \ \ \ \ \ \ \ \ \ \ }
\startm
\m{\vdash}\m{(}\m{\exists}\m{x}\m{\in}\m{A}\m{\varphi}\m{\leftrightarrow}\m{%
\exists}\m{x}\m{(}\m{x}\m{\in}\m{A}\m{\wedge}\m{\varphi}\m{)}\m{)}
\endm
\begin{mmraw}%
|- ( E. x e. A ph <-> E. x ( x e. A \TAND ph ) ) \$.
\end{mmraw}

\noindent Define the universal class\index{universal class ($V$)}.  Definition
5.20, p.~21, of Takeuti and Zaring.\label{df-v}

\vskip 0.5ex
\setbox\startprefix=\hbox{\tt \ \ df-v\ \$a\ }
\setbox\contprefix=\hbox{\tt \ \ \ \ \ \ \ \ \ \ }
\startm
\m{\vdash}\m{{\rm V}}\m{=}\m{\{}\m{x}\m{|}\m{x}\m{=}\m{x}\m{\}}
\endm
\begin{mmraw}%
|- {\char`\_}V = \{ x | x = x \} \$.
\end{mmraw}

\noindent Define the subclass\index{subclass}\index{subset} relationship
between two classes (called the subset relation if the classes are sets i.e.\
are not proper).  Definition 5.9 of Takeuti and Zaring, p.~17.\label{df-ss}

\vskip 0.5ex
\setbox\startprefix=\hbox{\tt \ \ df-ss\ \$a\ }
\setbox\contprefix=\hbox{\tt \ \ \ \ \ \ \ \ \ \ \ }
\startm
\m{\vdash}\m{(}\m{A}\m{\subseteq}\m{B}\m{\leftrightarrow}\m{\forall}\m{x}\m{(}
\m{x}\m{\in}\m{A}\m{\rightarrow}\m{x}\m{\in}\m{B}\m{)}\m{)}
\endm
\begin{mmraw}%
|- ( A C\_ B <-> A. x ( x e. A -> x e. B ) ) \$.
\end{mmraw}

\noindent Define the union\index{union} of two classes.  Definition 5.6 of Takeuti and Zaring,
p.~16.\label{df-un}

\vskip 0.5ex
\setbox\startprefix=\hbox{\tt \ \ df-un\ \$a\ }
\setbox\contprefix=\hbox{\tt \ \ \ \ \ \ \ \ \ \ \ }
\startm
\m{\vdash}\m{(}\m{A}\m{\cup}\m{B}\m{)}\m{=}\m{\{}\m{x}\m{|}\m{(}\m{x}\m{\in}
\m{A}\m{\vee}\m{x}\m{\in}\m{B}\m{)}\m{\}}
\endm
\begin{mmraw}%
( A u. B ) = \{ x | ( x e. A \TOR x e. B ) \} \$.
\end{mmraw}

\noindent Define the intersection\index{intersection} of two classes.  Definition 5.6 of
Takeuti and Zaring, p.~16.\label{df-in}

\vskip 0.5ex
\setbox\startprefix=\hbox{\tt \ \ df-in\ \$a\ }
\setbox\contprefix=\hbox{\tt \ \ \ \ \ \ \ \ \ \ \ }
\startm
\m{\vdash}\m{(}\m{A}\m{\cap}\m{B}\m{)}\m{=}\m{\{}\m{x}\m{|}\m{(}\m{x}\m{\in}
\m{A}\m{\wedge}\m{x}\m{\in}\m{B}\m{)}\m{\}}
\endm
% Caret ^ requires special treatment
\begin{mmraw}%
|- ( A i\^{}i B ) = \{ x | ( x e. A \TAND x e. B ) \} \$.
\end{mmraw}

\noindent Define class difference\index{class difference}\index{set difference}.
Definition 5.12 of Takeuti and Zaring, p.~20.  Several notations are used in
the literature; we chose the $\setminus$ convention instead of a minus sign to
reserve the latter for later use in, e.g., arithmetic.\label{df-dif}

\vskip 0.5ex
\setbox\startprefix=\hbox{\tt \ \ df-dif\ \$a\ }
\setbox\contprefix=\hbox{\tt \ \ \ \ \ \ \ \ \ \ \ \ }
\startm
\m{\vdash}\m{(}\m{A}\m{\setminus}\m{B}\m{)}\m{=}\m{\{}\m{x}\m{|}\m{(}\m{x}\m{
\in}\m{A}\m{\wedge}\m{\lnot}\m{x}\m{\in}\m{B}\m{)}\m{\}}
\endm
\begin{mmraw}%
( A \SLASH B ) = \{ x | ( x e. A \TAND -. x e. B ) \} \$.
\end{mmraw}

\noindent Define the empty or null set\index{empty set}\index{null set}.
Compare  Definition 5.14 of Takeuti and Zaring, p.~20.\label{df-nul}

\vskip 0.5ex
\setbox\startprefix=\hbox{\tt \ \ df-nul\ \$a\ }
\setbox\contprefix=\hbox{\tt \ \ \ \ \ \ \ \ \ \ }
\startm
\m{\vdash}\m{\varnothing}\m{=}\m{(}\m{{\rm V}}\m{\setminus}\m{{\rm V}}\m{)}
\endm
\begin{mmraw}%
|- (/) = ( {\char`\_}V \SLASH {\char`\_}V ) \$.
\end{mmraw}

\noindent Define power class\index{power set}\index{power class}.  Definition 5.10 of
Takeuti and Zaring, p.~17, but we also let it apply to proper classes.  (Note
that \verb$~P$ is the symbol for calligraphic P, the tilde
suggesting ``curly;'' see Appendix~\ref{ASCII}.)\label{df-pw}

\vskip 0.5ex
\setbox\startprefix=\hbox{\tt \ \ df-pw\ \$a\ }
\setbox\contprefix=\hbox{\tt \ \ \ \ \ \ \ \ \ \ \ }
\startm
\m{\vdash}\m{{\cal P}}\m{A}\m{=}\m{\{}\m{x}\m{|}\m{x}\m{\subseteq}\m{A}\m{\}}
\endm
% Special incantation required to put ~ into the text
\begin{mmraw}%
|- \char`\~P~A = \{ x | x C\_ A \} \$.
\end{mmraw}

\noindent Define the singleton of a class\index{singleton}.  Definition 7.1 of
Quine, p.~48.  It is well-defined for proper classes, although
it is not very meaningful in this case, where it evaluates to the empty
set.

\vskip 0.5ex
\setbox\startprefix=\hbox{\tt \ \ df-sn\ \$a\ }
\setbox\contprefix=\hbox{\tt \ \ \ \ \ \ \ \ \ \ \ }
\startm
\m{\vdash}\m{\{}\m{A}\m{\}}\m{=}\m{\{}\m{x}\m{|}\m{x}\m{=}\m{A}\m{\}}
\endm
\begin{mmraw}%
|- \{ A \} = \{ x | x = A \} \$.
\end{mmraw}%

\noindent Define an unordered pair of classes\index{unordered pair}\index{pair}.  Definition
7.1 of Quine, p.~48.

\vskip 0.5ex
\setbox\startprefix=\hbox{\tt \ \ df-pr\ \$a\ }
\setbox\contprefix=\hbox{\tt \ \ \ \ \ \ \ \ \ \ \ }
\startm
\m{\vdash}\m{\{}\m{A}\m{,}\m{B}\m{\}}\m{=}\m{(}\m{\{}\m{A}\m{\}}\m{\cup}\m{\{}
\m{B}\m{\}}\m{)}
\endm
\begin{mmraw}%
|- \{ A , B \} = ( \{ A \} u. \{ B \} ) \$.
\end{mmraw}

\noindent Define an unordered triple of classes\index{unordered triple}.  Definition of
Enderton, p.~19.

\vskip 0.5ex
\setbox\startprefix=\hbox{\tt \ \ df-tp\ \$a\ }
\setbox\contprefix=\hbox{\tt \ \ \ \ \ \ \ \ \ \ \ }
\startm
\m{\vdash}\m{\{}\m{A}\m{,}\m{B}\m{,}\m{C}\m{\}}\m{=}\m{(}\m{\{}\m{A}\m{,}\m{B}
\m{\}}\m{\cup}\m{\{}\m{C}\m{\}}\m{)}
\endm
\begin{mmraw}%
|- \{ A , B , C \} = ( \{ A , B \} u. \{ C \} ) \$.
\end{mmraw}%

\noindent Kuratowski's\index{Kuratowski, Kazimierz} ordered pair\index{ordered
pair} definition.  Definition 9.1 of Quine, p.~58. For proper classes it is
not meaningful but is well-defined for convenience.  (Note that \verb$<.$
stands for $\langle$ whereas \verb$<$ stands for $<$, and similarly for
\verb$>.$\,.)\label{df-op}

\vskip 0.5ex
\setbox\startprefix=\hbox{\tt \ \ df-op\ \$a\ }
\setbox\contprefix=\hbox{\tt \ \ \ \ \ \ \ \ \ \ \ }
\startm
\m{\vdash}\m{\langle}\m{A}\m{,}\m{B}\m{\rangle}\m{=}\m{\{}\m{\{}\m{A}\m{\}}
\m{,}\m{\{}\m{A}\m{,}\m{B}\m{\}}\m{\}}
\endm
\begin{mmraw}%
|- <. A , B >. = \{ \{ A \} , \{ A , B \} \} \$.
\end{mmraw}

\noindent Define the union of a class\index{union}.  Definition 5.5, p.~16,
of Takeuti and Zaring.

\vskip 0.5ex
\setbox\startprefix=\hbox{\tt \ \ df-uni\ \$a\ }
\setbox\contprefix=\hbox{\tt \ \ \ \ \ \ \ \ \ \ \ \ }
\startm
\m{\vdash}\m{\bigcup}\m{A}\m{=}\m{\{}\m{x}\m{|}\m{\exists}\m{y}\m{(}\m{x}\m{
\in}\m{y}\m{\wedge}\m{y}\m{\in}\m{A}\m{)}\m{\}}
\endm
\begin{mmraw}%
|- U. A = \{ x | E. y ( x e. y \TAND y e. A ) \} \$.
\end{mmraw}

\noindent Define the intersection\index{intersection} of a class.  Definition 7.35,
p.~44, of Takeuti and Zaring.

\vskip 0.5ex
\setbox\startprefix=\hbox{\tt \ \ df-int\ \$a\ }
\setbox\contprefix=\hbox{\tt \ \ \ \ \ \ \ \ \ \ \ \ }
\startm
\m{\vdash}\m{\bigcap}\m{A}\m{=}\m{\{}\m{x}\m{|}\m{\forall}\m{y}\m{(}\m{y}\m{
\in}\m{A}\m{\rightarrow}\m{x}\m{\in}\m{y}\m{)}\m{\}}
\endm
\begin{mmraw}%
|- |\^{}| A = \{ x | A. y ( y e. A -> x e. y ) \} \$.
\end{mmraw}

\noindent Define a transitive class\index{transitive class}\index{transitive
set}.  This should not be confused with a transitive relation, which is a different
concept.  Definition from p.~71 of Enderton, extended to classes.

\vskip 0.5ex
\setbox\startprefix=\hbox{\tt \ \ df-tr\ \$a\ }
\setbox\contprefix=\hbox{\tt \ \ \ \ \ \ \ \ \ \ \ }
\startm
\m{\vdash}\m{(}\m{\mbox{\rm Tr}}\m{A}\m{\leftrightarrow}\m{\bigcup}\m{A}\m{
\subseteq}\m{A}\m{)}
\endm
\begin{mmraw}%
|- ( Tr A <-> U. A C\_ A ) \$.
\end{mmraw}

\noindent Define a notation for a general binary relation\index{binary
relation}.  Definition 6.18, p.~29, of Takeuti and Zaring, generalized to
arbitrary classes.  This definition is well-defined, although not very
meaningful, when classes $A$ and/or $B$ are proper.\label{dfbr}  The lack of
parentheses (or any other connective) creates no ambiguity since we are defining
an atomic wff.

\vskip 0.5ex
\setbox\startprefix=\hbox{\tt \ \ df-br\ \$a\ }
\setbox\contprefix=\hbox{\tt \ \ \ \ \ \ \ \ \ \ \ }
\startm
\m{\vdash}\m{(}\m{A}\m{\,R}\m{\,B}\m{\leftrightarrow}\m{\langle}\m{A}\m{,}\m{B}
\m{\rangle}\m{\in}\m{R}\m{)}
\endm
\begin{mmraw}%
|- ( A R B <-> <. A , B >. e. R ) \$.
\end{mmraw}

\noindent Define an abstraction class of ordered pairs\index{abstraction
class!of ordered
pairs}.  A special case of Definition 4.16, p.~14, of Takeuti and Zaring.
Note that $ z $ must be distinct from $ x $ and $ y $,
and $ z $ must not occur in $\varphi$, but $ x $ and $ y $ may be
identical and may appear in $\varphi$.

\vskip 0.5ex
\setbox\startprefix=\hbox{\tt \ \ df-opab\ \$a\ }
\setbox\contprefix=\hbox{\tt \ \ \ \ \ \ \ \ \ \ \ \ \ }
\startm
\m{\vdash}\m{\{}\m{\langle}\m{x}\m{,}\m{y}\m{\rangle}\m{|}\m{\varphi}\m{\}}\m{=}
\m{\{}\m{z}\m{|}\m{\exists}\m{x}\m{\exists}\m{y}\m{(}\m{z}\m{=}\m{\langle}\m{x}
\m{,}\m{y}\m{\rangle}\m{\wedge}\m{\varphi}\m{)}\m{\}}
\endm

\begin{mmraw}%
|- \{ <. x , y >. | ph \} = \{ z | E. x E. y ( z =
<. x , y >. /\ ph ) \} \$.
\end{mmraw}

\noindent Define the epsilon relation\index{epsilon relation}.  Similar to Definition
6.22, p.~30, of Takeuti and Zaring.

\vskip 0.5ex
\setbox\startprefix=\hbox{\tt \ \ df-eprel\ \$a\ }
\setbox\contprefix=\hbox{\tt \ \ \ \ \ \ \ \ \ \ \ \ \ \ }
\startm
\m{\vdash}\m{{\rm E}}\m{=}\m{\{}\m{\langle}\m{x}\m{,}\m{y}\m{\rangle}\m{|}\m{x}\m{
\in}\m{y}\m{\}}
\endm
\begin{mmraw}%
|- \_E = \{ <. x , y >. | x e. y \} \$.
\end{mmraw}

\noindent Define a founded relation\index{founded relation}.  $R$ is a founded
relation on $A$ iff\index{iff} (if and only if) each nonempty subset of $A$
has an ``$R$-minimal element.''  Similar to Definition 6.21, p.~30, of
Takeuti and Zaring.

\vskip 0.5ex
\setbox\startprefix=\hbox{\tt \ \ df-fr\ \$a\ }
\setbox\contprefix=\hbox{\tt \ \ \ \ \ \ \ \ \ \ \ }
\startm
\m{\vdash}\m{(}\m{R}\m{\,\mbox{\rm Fr}}\m{\,A}\m{\leftrightarrow}\m{\forall}\m{x}
\m{(}\m{(}\m{x}\m{\subseteq}\m{A}\m{\wedge}\m{\lnot}\m{x}\m{=}\m{\varnothing}
\m{)}\m{\rightarrow}\m{\exists}\m{y}\m{(}\m{y}\m{\in}\m{x}\m{\wedge}\m{(}\m{x}
\m{\cap}\m{\{}\m{z}\m{|}\m{z}\m{\,R}\m{\,y}\m{\}}\m{)}\m{=}\m{\varnothing}\m{)}
\m{)}\m{)}
\endm
\begin{mmraw}%
|- ( R Fr A <-> A. x ( ( x C\_ A \TAND -. x = (/) ) ->
E. y ( y e. x \TAND ( x i\^{}i \{ z | z R y \} ) = (/) ) ) ) \$.
\end{mmraw}

\noindent Define a well-ordering\index{well-ordering}.  $R$ is a well-ordering of $A$ iff
it is founded on $A$ and the elements of $A$ are pairwise $R$-comparable.
Similar to Definition 6.24(2), p.~30, of Takeuti and Zaring.

\vskip 0.5ex
\setbox\startprefix=\hbox{\tt \ \ df-we\ \$a\ }
\setbox\contprefix=\hbox{\tt \ \ \ \ \ \ \ \ \ \ \ }
\startm
\m{\vdash}\m{(}\m{R}\m{\,\mbox{\rm We}}\m{\,A}\m{\leftrightarrow}\m{(}\m{R}\m{\,
\mbox{\rm Fr}}\m{\,A}\m{\wedge}\m{\forall}\m{x}\m{\forall}\m{y}\m{(}\m{(}\m{x}\m{
\in}\m{A}\m{\wedge}\m{y}\m{\in}\m{A}\m{)}\m{\rightarrow}\m{(}\m{x}\m{\,R}\m{\,y}
\m{\vee}\m{x}\m{=}\m{y}\m{\vee}\m{y}\m{\,R}\m{\,x}\m{)}\m{)}\m{)}\m{)}
\endm
\begin{mmraw}%
( R We A <-> ( R Fr A \TAND A. x A. y ( ( x e.
A \TAND y e. A ) -> ( x R y \TOR x = y \TOR y R x ) ) ) ) \$.
\end{mmraw}

\noindent Define the ordinal predicate\index{ordinal predicate}, which is true for a class
that is transitive and is well-ordered by the epsilon relation.  Similar to
definition on p.~468, Bell and Machover.

\vskip 0.5ex
\setbox\startprefix=\hbox{\tt \ \ df-ord\ \$a\ }
\setbox\contprefix=\hbox{\tt \ \ \ \ \ \ \ \ \ \ \ \ }
\startm
\m{\vdash}\m{(}\m{\mbox{\rm Ord}}\m{\,A}\m{\leftrightarrow}\m{(}
\m{\mbox{\rm Tr}}\m{\,A}\m{\wedge}\m{E}\m{\,\mbox{\rm We}}\m{\,A}\m{)}\m{)}
\endm
\begin{mmraw}%
|- ( Ord A <-> ( Tr A \TAND E We A ) ) \$.
\end{mmraw}

\noindent Define the class of all ordinal numbers\index{ordinal number}.  An ordinal number is
a set that satisfies the ordinal predicate.  Definition 7.11 of Takeuti and
Zaring, p.~38.

\vskip 0.5ex
\setbox\startprefix=\hbox{\tt \ \ df-on\ \$a\ }
\setbox\contprefix=\hbox{\tt \ \ \ \ \ \ \ \ \ \ \ }
\startm
\m{\vdash}\m{\,\mbox{\rm On}}\m{=}\m{\{}\m{x}\m{|}\m{\mbox{\rm Ord}}\m{\,x}
\m{\}}
\endm
\begin{mmraw}%
|- On = \{ x | Ord x \} \$.
\end{mmraw}

\noindent Define the limit ordinal predicate\index{limit ordinal}, which is true for a
non-empty ordinal that is not a successor (i.e.\ that is the union of itself).
Compare Bell and Machover, p.~471 and Exercise (1), p.~42 of Takeuti and
Zaring.

\vskip 0.5ex
\setbox\startprefix=\hbox{\tt \ \ df-lim\ \$a\ }
\setbox\contprefix=\hbox{\tt \ \ \ \ \ \ \ \ \ \ \ \ }
\startm
\m{\vdash}\m{(}\m{\mbox{\rm Lim}}\m{\,A}\m{\leftrightarrow}\m{(}\m{\mbox{
\rm Ord}}\m{\,A}\m{\wedge}\m{\lnot}\m{A}\m{=}\m{\varnothing}\m{\wedge}\m{A}
\m{=}\m{\bigcup}\m{A}\m{)}\m{)}
\endm
\begin{mmraw}%
|- ( Lim A <-> ( Ord A \TAND -. A = (/) \TAND A = U. A ) ) \$.
\end{mmraw}

\noindent Define the successor\index{successor} of a class.  Definition 7.22 of Takeuti
and Zaring, p.~41.  Our definition is a generalization to classes, although it
is meaningless when classes are proper.

\vskip 0.5ex
\setbox\startprefix=\hbox{\tt \ \ df-suc\ \$a\ }
\setbox\contprefix=\hbox{\tt \ \ \ \ \ \ \ \ \ \ \ \ }
\startm
\m{\vdash}\m{\,\mbox{\rm suc}}\m{\,A}\m{=}\m{(}\m{A}\m{\cup}\m{\{}\m{A}\m{\}}
\m{)}
\endm
\begin{mmraw}%
|- suc A = ( A u. \{ A \} ) \$.
\end{mmraw}

\noindent Define the class of natural numbers\index{natural number}\index{omega
($\omega$)}.  Compare Bell and Machover, p.~471.\label{dfom}

\vskip 0.5ex
\setbox\startprefix=\hbox{\tt \ \ df-om\ \$a\ }
\setbox\contprefix=\hbox{\tt \ \ \ \ \ \ \ \ \ \ \ }
\startm
\m{\vdash}\m{\omega}\m{=}\m{\{}\m{x}\m{|}\m{(}\m{\mbox{\rm Ord}}\m{\,x}\m{
\wedge}\m{\forall}\m{y}\m{(}\m{\mbox{\rm Lim}}\m{\,y}\m{\rightarrow}\m{x}\m{
\in}\m{y}\m{)}\m{)}\m{\}}
\endm
\begin{mmraw}%
|- om = \{ x | ( Ord x \TAND A. y ( Lim y -> x e. y ) ) \} \$.
\end{mmraw}

\noindent Define the Cartesian product (also called the
cross product)\index{Cartesian product}\index{cross product}
of two classes.  Definition 9.11 of Quine, p.~64.

\vskip 0.5ex
\setbox\startprefix=\hbox{\tt \ \ df-xp\ \$a\ }
\setbox\contprefix=\hbox{\tt \ \ \ \ \ \ \ \ \ \ \ }
\startm
\m{\vdash}\m{(}\m{A}\m{\times}\m{B}\m{)}\m{=}\m{\{}\m{\langle}\m{x}\m{,}\m{y}
\m{\rangle}\m{|}\m{(}\m{x}\m{\in}\m{A}\m{\wedge}\m{y}\m{\in}\m{B}\m{)}\m{\}}
\endm
\begin{mmraw}%
|- ( A X. B ) = \{ <. x , y >. | ( x e. A \TAND y e. B) \} \$.
\end{mmraw}

\noindent Define a relation\index{relation}.  Definition 6.4(1) of Takeuti and
Zaring, p.~23.

\vskip 0.5ex
\setbox\startprefix=\hbox{\tt \ \ df-rel\ \$a\ }
\setbox\contprefix=\hbox{\tt \ \ \ \ \ \ \ \ \ \ \ \ }
\startm
\m{\vdash}\m{(}\m{\mbox{\rm Rel}}\m{\,A}\m{\leftrightarrow}\m{A}\m{\subseteq}
\m{(}\m{{\rm V}}\m{\times}\m{{\rm V}}\m{)}\m{)}
\endm
\begin{mmraw}%
|- ( Rel A <-> A C\_ ( {\char`\_}V X. {\char`\_}V ) ) \$.
\end{mmraw}

\noindent Define the domain\index{domain} of a class.  Definition 6.5(1) of
Takeuti and Zaring, p.~24.

\vskip 0.5ex
\setbox\startprefix=\hbox{\tt \ \ df-dm\ \$a\ }
\setbox\contprefix=\hbox{\tt \ \ \ \ \ \ \ \ \ \ \ }
\startm
\m{\vdash}\m{\,\mbox{\rm dom}}\m{A}\m{=}\m{\{}\m{x}\m{|}\m{\exists}\m{y}\m{
\langle}\m{x}\m{,}\m{y}\m{\rangle}\m{\in}\m{A}\m{\}}
\endm
\begin{mmraw}%
|- dom A = \{ x | E. y <. x , y >. e. A \} \$.
\end{mmraw}

\noindent Define the range\index{range} of a class.  Definition 6.5(2) of
Takeuti and Zaring, p.~24.

\vskip 0.5ex
\setbox\startprefix=\hbox{\tt \ \ df-rn\ \$a\ }
\setbox\contprefix=\hbox{\tt \ \ \ \ \ \ \ \ \ \ \ }
\startm
\m{\vdash}\m{\,\mbox{\rm ran}}\m{A}\m{=}\m{\{}\m{y}\m{|}\m{\exists}\m{x}\m{
\langle}\m{x}\m{,}\m{y}\m{\rangle}\m{\in}\m{A}\m{\}}
\endm
\begin{mmraw}%
|- ran A = \{ y | E. x <. x , y >. e. A \} \$.
\end{mmraw}

\noindent Define the restriction\index{restriction} of a class.  Definition
6.6(1) of Takeuti and Zaring, p.~24.

\vskip 0.5ex
\setbox\startprefix=\hbox{\tt \ \ df-res\ \$a\ }
\setbox\contprefix=\hbox{\tt \ \ \ \ \ \ \ \ \ \ \ \ }
\startm
\m{\vdash}\m{(}\m{A}\m{\restriction}\m{B}\m{)}\m{=}\m{(}\m{A}\m{\cap}\m{(}\m{B}
\m{\times}\m{{\rm V}}\m{)}\m{)}
\endm
\begin{mmraw}%
|- ( A |` B ) = ( A i\^{}i ( B X. {\char`\_}V ) ) \$.
\end{mmraw}

\noindent Define the image\index{image} of a class.  Definition 6.6(2) of
Takeuti and Zaring, p.~24.

\vskip 0.5ex
\setbox\startprefix=\hbox{\tt \ \ df-ima\ \$a\ }
\setbox\contprefix=\hbox{\tt \ \ \ \ \ \ \ \ \ \ \ \ }
\startm
\m{\vdash}\m{(}\m{A}\m{``}\m{B}\m{)}\m{=}\m{\,\mbox{\rm ran}}\m{\,(}\m{A}\m{
\restriction}\m{B}\m{)}
\endm
\begin{mmraw}%
|- ( A " B ) = ran ( A |` B ) \$.
\end{mmraw}

\noindent Define the composition\index{composition} of two classes.  Definition 6.6(3) of
Takeuti and Zaring, p.~24.

\vskip 0.5ex
\setbox\startprefix=\hbox{\tt \ \ df-co\ \$a\ }
\setbox\contprefix=\hbox{\tt \ \ \ \ \ \ \ \ \ \ \ \ }
\startm
\m{\vdash}\m{(}\m{A}\m{\circ}\m{B}\m{)}\m{=}\m{\{}\m{\langle}\m{x}\m{,}\m{y}\m{
\rangle}\m{|}\m{\exists}\m{z}\m{(}\m{\langle}\m{x}\m{,}\m{z}\m{\rangle}\m{\in}
\m{B}\m{\wedge}\m{\langle}\m{z}\m{,}\m{y}\m{\rangle}\m{\in}\m{A}\m{)}\m{\}}
\endm
\begin{mmraw}%
|- ( A o. B ) = \{ <. x , y >. | E. z ( <. x , z
>. e. B \TAND <. z , y >. e. A ) \} \$.
\end{mmraw}

\noindent Define a function\index{function}.  Definition 6.4(4) of Takeuti and
Zaring, p.~24.

\vskip 0.5ex
\setbox\startprefix=\hbox{\tt \ \ df-fun\ \$a\ }
\setbox\contprefix=\hbox{\tt \ \ \ \ \ \ \ \ \ \ \ \ }
\startm
\m{\vdash}\m{(}\m{\mbox{\rm Fun}}\m{\,A}\m{\leftrightarrow}\m{(}
\m{\mbox{\rm Rel}}\m{\,A}\m{\wedge}
\m{\forall}\m{x}\m{\exists}\m{z}\m{\forall}\m{y}\m{(}
\m{\langle}\m{x}\m{,}\m{y}\m{\rangle}\m{\in}\m{A}\m{\rightarrow}\m{y}\m{=}\m{z}
\m{)}\m{)}\m{)}
\endm
\begin{mmraw}%
|- ( Fun A <-> ( Rel A /\ A. x E. z A. y ( <. x
   , y >. e. A -> y = z ) ) ) \$.
\end{mmraw}

\noindent Define a function with domain.  Definition 6.15(1) of Takeuti and
Zaring, p.~27.

\vskip 0.5ex
\setbox\startprefix=\hbox{\tt \ \ df-fn\ \$a\ }
\setbox\contprefix=\hbox{\tt \ \ \ \ \ \ \ \ \ \ \ }
\startm
\m{\vdash}\m{(}\m{A}\m{\,\mbox{\rm Fn}}\m{\,B}\m{\leftrightarrow}\m{(}
\m{\mbox{\rm Fun}}\m{\,A}\m{\wedge}\m{\mbox{\rm dom}}\m{\,A}\m{=}\m{B}\m{)}
\m{)}
\endm
\begin{mmraw}%
|- ( A Fn B <-> ( Fun A \TAND dom A = B ) ) \$.
\end{mmraw}

\noindent Define a function with domain and co-domain.  Definition 6.15(3)
of Takeuti and Zaring, p.~27.

\vskip 0.5ex
\setbox\startprefix=\hbox{\tt \ \ df-f\ \$a\ }
\setbox\contprefix=\hbox{\tt \ \ \ \ \ \ \ \ \ \ }
\startm
\m{\vdash}\m{(}\m{F}\m{:}\m{A}\m{\longrightarrow}\m{B}\m{
\leftrightarrow}\m{(}\m{F}\m{\,\mbox{\rm Fn}}\m{\,A}\m{\wedge}\m{
\mbox{\rm ran}}\m{\,F}\m{\subseteq}\m{B}\m{)}\m{)}
\endm
\begin{mmraw}%
|- ( F : A --> B <-> ( F Fn A \TAND ran F C\_ B ) ) \$.
\end{mmraw}

\noindent Define a one-to-one function\index{one-to-one function}.  Compare
Definition 6.15(5) of Takeuti and Zaring, p.~27.

\vskip 0.5ex
\setbox\startprefix=\hbox{\tt \ \ df-f1\ \$a\ }
\setbox\contprefix=\hbox{\tt \ \ \ \ \ \ \ \ \ \ \ }
\startm
\m{\vdash}\m{(}\m{F}\m{:}\m{A}\m{
\raisebox{.5ex}{${\textstyle{\:}_{\mbox{\footnotesize\rm
1\tt -\rm 1}}}\atop{\textstyle{
\longrightarrow}\atop{\textstyle{}^{\mbox{\footnotesize\rm {\ }}}}}$}
}\m{B}
\m{\leftrightarrow}\m{(}\m{F}\m{:}\m{A}\m{\longrightarrow}\m{B}
\m{\wedge}\m{\forall}\m{y}\m{\exists}\m{z}\m{\forall}\m{x}\m{(}\m{\langle}\m{x}
\m{,}\m{y}\m{\rangle}\m{\in}\m{F}\m{\rightarrow}\m{x}\m{=}\m{z}\m{)}\m{)}\m{)}
\endm
\begin{mmraw}%
|- ( F : A -1-1-> B <-> ( F : A --> B \TAND
   A. y E. z A. x ( <. x , y >. e. F -> x = z ) ) ) \$.
\end{mmraw}

\noindent Define an onto function\index{onto function}.  Definition 6.15(4) of Takeuti and
Zaring, p.~27.

\vskip 0.5ex
\setbox\startprefix=\hbox{\tt \ \ df-fo\ \$a\ }
\setbox\contprefix=\hbox{\tt \ \ \ \ \ \ \ \ \ \ \ }
\startm
\m{\vdash}\m{(}\m{F}\m{:}\m{A}\m{
\raisebox{.5ex}{${\textstyle{\:}_{\mbox{\footnotesize\rm
{\ }}}}\atop{\textstyle{
\longrightarrow}\atop{\textstyle{}^{\mbox{\footnotesize\rm onto}}}}$}
}\m{B}
\m{\leftrightarrow}\m{(}\m{F}\m{\,\mbox{\rm Fn}}\m{\,A}\m{\wedge}
\m{\mbox{\rm ran}}\m{\,F}\m{=}\m{B}\m{)}\m{)}
\endm
\begin{mmraw}%
|- ( F : A -onto-> B <-> ( F Fn A /\ ran F = B ) ) \$.
\end{mmraw}

\noindent Define a one-to-one, onto function.  Compare Definition 6.15(6) of
Takeuti and Zaring, p.~27.

\vskip 0.5ex
\setbox\startprefix=\hbox{\tt \ \ df-f1o\ \$a\ }
\setbox\contprefix=\hbox{\tt \ \ \ \ \ \ \ \ \ \ \ \ }
\startm
\m{\vdash}\m{(}\m{F}\m{:}\m{A}
\m{
\raisebox{.5ex}{${\textstyle{\:}_{\mbox{\footnotesize\rm
1\tt -\rm 1}}}\atop{\textstyle{
\longrightarrow}\atop{\textstyle{}^{\mbox{\footnotesize\rm onto}}}}$}
}
\m{B}
\m{\leftrightarrow}\m{(}\m{F}\m{:}\m{A}
\m{
\raisebox{.5ex}{${\textstyle{\:}_{\mbox{\footnotesize\rm
1\tt -\rm 1}}}\atop{\textstyle{
\longrightarrow}\atop{\textstyle{}^{\mbox{\footnotesize\rm {\ }}}}}$}
}
\m{B}\m{\wedge}\m{F}\m{:}\m{A}
\m{
\raisebox{.5ex}{${\textstyle{\:}_{\mbox{\footnotesize\rm
{\ }}}}\atop{\textstyle{
\longrightarrow}\atop{\textstyle{}^{\mbox{\footnotesize\rm onto}}}}$}
}
\m{B}\m{)}\m{)}
\endm
\begin{mmraw}%
|- ( F : A -1-1-onto-> B <-> ( F : A -1-1-> B? \TAND F : A -onto-> B ) ) \$.?
\end{mmraw}

\noindent Define the value of a function\index{function value}.  This
definition applies to any class and evaluates to the empty set when it is not
meaningful. Note that $ F`A$ means the same thing as the more familiar $ F(A)$
notation for a function's value at $A$.  The $ F`A$ notation is common in
formal set theory.\label{df-fv}

\vskip 0.5ex
\setbox\startprefix=\hbox{\tt \ \ df-fv\ \$a\ }
\setbox\contprefix=\hbox{\tt \ \ \ \ \ \ \ \ \ \ \ }
\startm
\m{\vdash}\m{(}\m{F}\m{`}\m{A}\m{)}\m{=}\m{\bigcup}\m{\{}\m{x}\m{|}\m{(}\m{F}%
\m{``}\m{\{}\m{A}\m{\}}\m{)}\m{=}\m{\{}\m{x}\m{\}}\m{\}}
\endm
\begin{mmraw}%
|- ( F ` A ) = U. \{ x | ( F " \{ A \} ) = \{ x \} \} \$.
\end{mmraw}

\noindent Define the result of an operation.\index{operation}  Here, $F$ is
     an operation on two
     values (such as $+$ for real numbers).   This is defined for proper
     classes $A$ and $B$ even though not meaningful in that case.  However,
     the definition can be meaningful when $F$ is a proper class.\label{dfopr}

\vskip 0.5ex
\setbox\startprefix=\hbox{\tt \ \ df-opr\ \$a\ }
\setbox\contprefix=\hbox{\tt \ \ \ \ \ \ \ \ \ \ \ \ }
\startm
\m{\vdash}\m{(}\m{A}\m{\,F}\m{\,B}\m{)}\m{=}\m{(}\m{F}\m{`}\m{\langle}\m{A}%
\m{,}\m{B}\m{\rangle}\m{)}
\endm
\begin{mmraw}%
|- ( A F B ) = ( F ` <. A , B >. ) \$.
\end{mmraw}

\section{Tricks of the Trade}\label{tricks}

In the \texttt{set.mm}\index{set theory database (\texttt{set.mm})} database our goal
was usually to conform to modern notation.  However in some cases the
relationship to standard textbook language may be obscured by several
unconventional devices we used to simplify the development and to take
advantage of the Metamath language.  In this section we will describe some
common conventions used in \texttt{set.mm}.

\begin{itemize}
\item
The turnstile\index{turnstile ({$\,\vdash$})} symbol, $\vdash$, meaning ``it
is provable that,'' is the first token of all assertions and hypotheses that
aren't syntax constructions.  This is a standard convention in logic.  (We
mentioned this earlier, but this symbol is bothersome to some people without a
logic background.  It has no deeper meaning but just provides us with a way to
distinguish syntax constructions from ordinary mathematical statements.)

\item
A hypothesis of the form

\vskip 1ex
\setbox\startprefix=\hbox{\tt \ \ \ \ \ \ \ \ \ \$e\ }
\setbox\contprefix=\hbox{\tt \ \ \ \ \ \ \ \ \ \ }
\startm
\m{\vdash}\m{(}\m{\varphi}\m{\rightarrow}\m{\forall}\m{x}\m{\varphi}\m{)}
\endm
\vskip 1ex

should be read ``assume variable $x$ is (effectively) not free in wff
$\varphi$.''\index{effectively not free}
Literally, this says ``assume it is provable that $\varphi \rightarrow \forall
x\, \varphi$.''  This device lets us avoid the complexities associated with
the standard treatment of free and bound variables.
%% Uncomment this when uncommenting section {formalspec} below
The footnote on p.~\pageref{effectivelybound} discusses this further.

\item
A statement of one of the forms

\vskip 1ex
\setbox\startprefix=\hbox{\tt \ \ \ \ \ \ \ \ \ \$a\ }
\setbox\contprefix=\hbox{\tt \ \ \ \ \ \ \ \ \ \ }
\startm
\m{\vdash}\m{(}\m{\lnot}\m{\forall}\m{x}\m{\,x}\m{=}\m{y}\m{\rightarrow}
\m{\ldots}\m{)}
\endm
\setbox\startprefix=\hbox{\tt \ \ \ \ \ \ \ \ \ \$p\ }
\setbox\contprefix=\hbox{\tt \ \ \ \ \ \ \ \ \ \ }
\startm
\m{\vdash}\m{(}\m{\lnot}\m{\forall}\m{x}\m{\,x}\m{=}\m{y}\m{\rightarrow}
\m{\ldots}\m{)}
\endm
\vskip 1ex

should be read ``if $x$ and $y$ are distinct variables, then...''  This
antecedent provides us with a technical device to avoid the need for the
\texttt{\$d} statement early in our development of predicate calculus,
permitting symbol manipulations to be as conceptually simple as those in
propositional calculus.  However, the \texttt{\$d} statement eventually
becomes a requirement, and after that this device is rarely used.

\item
The statement

\vskip 1ex
\setbox\startprefix=\hbox{\tt \ \ \ \ \ \ \ \ \ \$d\ }
\setbox\contprefix=\hbox{\tt \ \ \ \ \ \ \ \ \ \ }
\startm
\m{x}\m{\,y}
\endm
\vskip 1ex

should be read ``assume $x$ and $y$ are distinct variables.''

\item
The statement

\vskip 1ex
\setbox\startprefix=\hbox{\tt \ \ \ \ \ \ \ \ \ \$d\ }
\setbox\contprefix=\hbox{\tt \ \ \ \ \ \ \ \ \ \ }
\startm
\m{x}\m{\,\varphi}
\endm
\vskip 1ex

should be read ``assume $x$ does not occur in $\varphi$.''

\item
The statement

\vskip 1ex
\setbox\startprefix=\hbox{\tt \ \ \ \ \ \ \ \ \ \$d\ }
\setbox\contprefix=\hbox{\tt \ \ \ \ \ \ \ \ \ \ }
\startm
\m{x}\m{\,A}
\endm
\vskip 1ex

should be read ``assume variable $x$ does not occur in class $A$.''

\item
The restriction and hypothesis group

\vskip 1ex
\setbox\startprefix=\hbox{\tt \ \ \ \ \ \ \ \ \ \$d\ }
\setbox\contprefix=\hbox{\tt \ \ \ \ \ \ \ \ \ \ }
\startm
\m{x}\m{\,A}
\endm
\setbox\startprefix=\hbox{\tt \ \ \ \ \ \ \ \ \ \$d\ }
\setbox\contprefix=\hbox{\tt \ \ \ \ \ \ \ \ \ \ }
\startm
\m{x}\m{\,\psi}
\endm
\setbox\startprefix=\hbox{\tt \ \ \ \ \ \ \ \ \ \$e\ }
\setbox\contprefix=\hbox{\tt \ \ \ \ \ \ \ \ \ \ }
\startm
\m{\vdash}\m{(}\m{x}\m{=}\m{A}\m{\rightarrow}\m{(}\m{\varphi}\m{\leftrightarrow}
\m{\psi}\m{)}\m{)}
\endm
\vskip 1ex

is frequently used in place of explicit substitution, meaning ``assume
$\psi$ results from the proper substitution of $A$ for $x$ in
$\varphi$.''  Sometimes ``\texttt{\$e} $\vdash ( \psi \rightarrow
\forall x \, \psi )$'' is used instead of ``\texttt{\$d} $x\, \psi $,''
which requires only that $x$ be effectively not free in $\varphi$ but
not necessarily absent from it.  The use of implicit
substitution\index{substitution!implicit} is partly a matter of personal
style, although it may make proofs somewhat shorter than would be the
case with explicit substitution.

\item
The hypothesis


\vskip 1ex
\setbox\startprefix=\hbox{\tt \ \ \ \ \ \ \ \ \ \$e\ }
\setbox\contprefix=\hbox{\tt \ \ \ \ \ \ \ \ \ \ }
\startm
\m{\vdash}\m{A}\m{\in}\m{{\rm V}}
\endm
\vskip 1ex

should be read ``assume class $A$ is a set (i.e.\ exists).''
This is a convenient convention used by Quine.

\item
The restriction and hypothesis

\vskip 1ex
\setbox\startprefix=\hbox{\tt \ \ \ \ \ \ \ \ \ \$d\ }
\setbox\contprefix=\hbox{\tt \ \ \ \ \ \ \ \ \ \ }
\startm
\m{x}\m{\,y}
\endm
\setbox\startprefix=\hbox{\tt \ \ \ \ \ \ \ \ \ \$e\ }
\setbox\contprefix=\hbox{\tt \ \ \ \ \ \ \ \ \ \ }
\startm
\m{\vdash}\m{(}\m{y}\m{\in}\m{A}\m{\rightarrow}\m{\forall}\m{x}\m{\,y}
\m{\in}\m{A}\m{)}
\endm
\vskip 1ex

should be read ``assume variable $x$ is
(effectively) not free in class $A$.''

\end{itemize}

\section{A Theorem Sampler}\label{sometheorems}

In this section we list some of the more important theorems that are proved in
the \texttt{set.mm} database, and they illustrate the kinds of things that can be
done with Metamath.  While all of these facts are well-known results,
Metamath offers the advantage of easily allowing you to trace their
derivation back to axioms.  Our intent here is not to try to explain the
details or motivation; for this we refer you to the textbooks that are
mentioned in the descriptions.  (The \texttt{set.mm} file has bibliographic
references for the text references.)  Their proofs often embody important
concepts you may wish to explore with the Metamath program (see
Section~\ref{exploring}).  All the symbols that are used here are defined in
Section~\ref{hierarchy}.  For brevity we haven't included the \texttt{\$d}
restrictions or \texttt{\$f} hypotheses for these theorems; when you are
uncertain consult the \texttt{set.mm} database.

We start with \texttt{syl} (principle of the syllogism).
In \textit{Principia Mathematica}
Whitehead and Russell call this ``the principle of the
syllogism... because... the syllogism in Barbara is derived from them''
\cite[quote after Theorem *2.06 p.~101]{PM}.
Some authors call this law a ``hypothetical syllogism.''
At the time of this writing it is the most commonly referenced assertion in the
\texttt{set.mm} database.\footnote{
The Metamath program command \texttt{show usage}
shows the number of references.
}

\vskip 2ex
\noindent Theorem syl (principle of the syllogism)\index{Syllogism}%
\index{\texttt{syl}}\label{syl}.

\vskip 0.5ex
\setbox\startprefix=\hbox{\tt \ \ syl.1\ \$e\ }
\setbox\contprefix=\hbox{\tt \ \ \ \ \ \ \ \ \ \ \ }
\startm
\m{\vdash}\m{(}\m{\varphi}\m{ \rightarrow }\m{\psi}\m{)}
\endm
\setbox\startprefix=\hbox{\tt \ \ syl.2\ \$e\ }
\setbox\contprefix=\hbox{\tt \ \ \ \ \ \ \ \ \ \ \ }
\startm
\m{\vdash}\m{(}\m{\psi}\m{ \rightarrow }\m{\chi}\m{)}
\endm
\setbox\startprefix=\hbox{\tt \ \ syl\ \$p\ }
\setbox\contprefix=\hbox{\tt \ \ \ \ \ \ \ \ \ }
\startm
\m{\vdash}\m{(}\m{\varphi}\m{ \rightarrow }\m{\chi}\m{)}
\endm
\vskip 2ex

The following theorem is not very deep but provides us with a notational device
that is frequently used.  It allows us to use the expression ``$A \in V$'' as
a compact way of saying that class $A$ exists, i.e.\ is a set.

\vskip 2ex
\noindent Two ways to say ``$A$ is a set'':  $A$ is a member of the universe
$V$ if and only if $A$ exists (i.e.\ there exists a set equal to $A$).
Theorem 6.9 of Quine, p. 43.

\vskip 0.5ex
\setbox\startprefix=\hbox{\tt \ \ isset\ \$p\ }
\setbox\contprefix=\hbox{\tt \ \ \ \ \ \ \ \ \ \ \ }
\startm
\m{\vdash}\m{(}\m{A}\m{\in}\m{{\rm V}}\m{\leftrightarrow}\m{\exists}\m{x}\m{\,x}\m{=}
\m{A}\m{)}
\endm
\vskip 1ex

Next we prove the axioms of standard ZF set theory that were missing from our
axiom system.  From our point of view they are theorems since they
can be derived from the other axioms.

\vskip 2ex
\noindent Axiom of Separation\index{Axiom of Separation}
(Aussonderung)\index{Aussonderung} proved from the other axioms of ZF set
theory.  Compare Exercise 4 of Takeuti and Zaring, p.~22.

\vskip 0.5ex
\setbox\startprefix=\hbox{\tt \ \ inex1.1\ \$e\ }
\setbox\contprefix=\hbox{\tt \ \ \ \ \ \ \ \ \ \ \ \ \ \ \ }
\startm
\m{\vdash}\m{A}\m{\in}\m{{\rm V}}
\endm
\setbox\startprefix=\hbox{\tt \ \ inex\ \$p\ }
\setbox\contprefix=\hbox{\tt \ \ \ \ \ \ \ \ \ \ \ \ \ }
\startm
\m{\vdash}\m{(}\m{A}\m{\cap}\m{B}\m{)}\m{\in}\m{{\rm V}}
\endm
\vskip 1ex

\noindent Axiom of the Null Set\index{Axiom of the Null Set} proved from the
other axioms of ZF set theory. Corollary 5.16 of Takeuti and Zaring, p.~20.

\vskip 0.5ex
\setbox\startprefix=\hbox{\tt \ \ 0ex\ \$p\ }
\setbox\contprefix=\hbox{\tt \ \ \ \ \ \ \ \ \ \ \ \ }
\startm
\m{\vdash}\m{\varnothing}\m{\in}\m{{\rm V}}
\endm
\vskip 1ex

\noindent The Axiom of Pairing\index{Axiom of Pairing} proved from the other
axioms of ZF set theory.  Theorem 7.13 of Quine, p.~51.
\vskip 0.5ex
\setbox\startprefix=\hbox{\tt \ \ prex\ \$p\ }
\setbox\contprefix=\hbox{\tt \ \ \ \ \ \ \ \ \ \ \ \ \ \ }
\startm
\m{\vdash}\m{\{}\m{A}\m{,}\m{B}\m{\}}\m{\in}\m{{\rm V}}
\endm
\vskip 2ex

Next we will list some famous or important theorems that are proved in
the \texttt{set.mm} database.  None of them except \texttt{omex}
require the Axiom of Infinity, as you can verify with the \texttt{show
trace{\char`\_}back} Metamath command.

\vskip 2ex
\noindent The resolution of Russell's paradox\index{Russell's paradox}.  There
exists no set containing the set of all sets which are not members of
themselves.  Proposition 4.14 of Takeuti and Zaring, p.~14.

\vskip 0.5ex
\setbox\startprefix=\hbox{\tt \ \ ru\ \$p\ }
\setbox\contprefix=\hbox{\tt \ \ \ \ \ \ \ \ }
\startm
\m{\vdash}\m{\lnot}\m{\exists}\m{x}\m{\,x}\m{=}\m{\{}\m{y}\m{|}\m{\lnot}\m{y}
\m{\in}\m{y}\m{\}}
\endm
\vskip 1ex

\noindent Cantor's theorem\index{Cantor's theorem}.  No set can be mapped onto
its power set.  Compare Theorem 6B(b) of Enderton, p.~132.

\vskip 0.5ex
\setbox\startprefix=\hbox{\tt \ \ canth.1\ \$e\ }
\setbox\contprefix=\hbox{\tt \ \ \ \ \ \ \ \ \ \ \ \ \ }
\startm
\m{\vdash}\m{A}\m{\in}\m{{\rm V}}
\endm
\setbox\startprefix=\hbox{\tt \ \ canth\ \$p\ }
\setbox\contprefix=\hbox{\tt \ \ \ \ \ \ \ \ \ \ \ }
\startm
\m{\vdash}\m{\lnot}\m{F}\m{:}\m{A}\m{\raisebox{.5ex}{${\textstyle{\:}_{
\mbox{\footnotesize\rm {\ }}}}\atop{\textstyle{\longrightarrow}\atop{
\textstyle{}^{\mbox{\footnotesize\rm onto}}}}$}}\m{{\cal P}}\m{A}
\endm
\vskip 1ex

\noindent The Burali-Forti paradox\index{Burali-Forti paradox}.  No set
contains all ordinal numbers. Enderton, p.~194.  (Burali-Forti was one person,
not two.)

\vskip 0.5ex
\setbox\startprefix=\hbox{\tt \ \ onprc\ \$p\ }
\setbox\contprefix=\hbox{\tt \ \ \ \ \ \ \ \ \ \ \ \ }
\startm
\m{\vdash}\m{\lnot}\m{\mbox{\rm On}}\m{\in}\m{{\rm V}}
\endm
\vskip 1ex

\noindent Peano's postulates\index{Peano's postulates} for arithmetic.
Proposition 7.30 of Takeuti and Zaring, pp.~42--43.  The objects being
described are the members of $\omega$ i.e.\ the natural numbers 0, 1,
2,\ldots.  The successor\index{successor} operation suc means ``plus
one.''  \texttt{peano1} says that 0 (which is defined as the empty set)
is a natural number.  \texttt{peano2} says that if $A$ is a natural
number, so is $A+1$.  \texttt{peano3} says that 0 is not the successor
of any natural number.  \texttt{peano4} says that two natural numbers
are equal if and only if their successors are equal.  \texttt{peano5} is
essentially the same as mathematical induction.

\vskip 1ex
\setbox\startprefix=\hbox{\tt \ \ peano1\ \$p\ }
\setbox\contprefix=\hbox{\tt \ \ \ \ \ \ \ \ \ \ \ \ }
\startm
\m{\vdash}\m{\varnothing}\m{\in}\m{\omega}
\endm
\vskip 1.5ex

\setbox\startprefix=\hbox{\tt \ \ peano2\ \$p\ }
\setbox\contprefix=\hbox{\tt \ \ \ \ \ \ \ \ \ \ \ \ }
\startm
\m{\vdash}\m{(}\m{A}\m{\in}\m{\omega}\m{\rightarrow}\m{{\rm suc}}\m{A}\m{\in}%
\m{\omega}\m{)}
\endm
\vskip 1.5ex

\setbox\startprefix=\hbox{\tt \ \ peano3\ \$p\ }
\setbox\contprefix=\hbox{\tt \ \ \ \ \ \ \ \ \ \ \ \ }
\startm
\m{\vdash}\m{(}\m{A}\m{\in}\m{\omega}\m{\rightarrow}\m{\lnot}\m{{\rm suc}}%
\m{A}\m{=}\m{\varnothing}\m{)}
\endm
\vskip 1.5ex

\setbox\startprefix=\hbox{\tt \ \ peano4\ \$p\ }
\setbox\contprefix=\hbox{\tt \ \ \ \ \ \ \ \ \ \ \ \ }
\startm
\m{\vdash}\m{(}\m{(}\m{A}\m{\in}\m{\omega}\m{\wedge}\m{B}\m{\in}\m{\omega}%
\m{)}\m{\rightarrow}\m{(}\m{{\rm suc}}\m{A}\m{=}\m{{\rm suc}}\m{B}\m{%
\leftrightarrow}\m{A}\m{=}\m{B}\m{)}\m{)}
\endm
\vskip 1.5ex

\setbox\startprefix=\hbox{\tt \ \ peano5\ \$p\ }
\setbox\contprefix=\hbox{\tt \ \ \ \ \ \ \ \ \ \ \ \ }
\startm
\m{\vdash}\m{(}\m{(}\m{\varnothing}\m{\in}\m{A}\m{\wedge}\m{\forall}\m{x}\m{%
\in}\m{\omega}\m{(}\m{x}\m{\in}\m{A}\m{\rightarrow}\m{{\rm suc}}\m{x}\m{\in}%
\m{A}\m{)}\m{)}\m{\rightarrow}\m{\omega}\m{\subseteq}\m{A}\m{)}
\endm
\vskip 1.5ex

\noindent Finite Induction (mathematical induction).\index{finite
induction}\index{mathematical induction} The first hypothesis is the
basis and the second is the induction hypothesis.  Theorem Schema 22 of
Suppes, p.~136.

\vskip 0.5ex
\setbox\startprefix=\hbox{\tt \ \ findes.1\ \$e\ }
\setbox\contprefix=\hbox{\tt \ \ \ \ \ \ \ \ \ \ \ \ \ \ }
\startm
\m{\vdash}\m{[}\m{\varnothing}\m{/}\m{x}\m{]}\m{\varphi}
\endm
\setbox\startprefix=\hbox{\tt \ \ findes.2\ \$e\ }
\setbox\contprefix=\hbox{\tt \ \ \ \ \ \ \ \ \ \ \ \ \ \ }
\startm
\m{\vdash}\m{(}\m{x}\m{\in}\m{\omega}\m{\rightarrow}\m{(}\m{\varphi}\m{%
\rightarrow}\m{[}\m{{\rm suc}}\m{x}\m{/}\m{x}\m{]}\m{\varphi}\m{)}\m{)}
\endm
\setbox\startprefix=\hbox{\tt \ \ findes\ \$p\ }
\setbox\contprefix=\hbox{\tt \ \ \ \ \ \ \ \ \ \ \ \ }
\startm
\m{\vdash}\m{(}\m{x}\m{\in}\m{\omega}\m{\rightarrow}\m{\varphi}\m{)}
\endm
\vskip 1ex

\noindent Transfinite Induction with explicit substitution.  The first
hypothesis is the basis, the second is the induction hypothesis for
successors, and the third is the induction hypothesis for limit
ordinals.  Theorem Schema 4 of Suppes, p. 197.

\vskip 0.5ex
\setbox\startprefix=\hbox{\tt \ \ tfindes.1\ \$e\ }
\setbox\contprefix=\hbox{\tt \ \ \ \ \ \ \ \ \ \ \ \ \ \ \ }
\startm
\m{\vdash}\m{[}\m{\varnothing}\m{/}\m{x}\m{]}\m{\varphi}
\endm
\setbox\startprefix=\hbox{\tt \ \ tfindes.2\ \$e\ }
\setbox\contprefix=\hbox{\tt \ \ \ \ \ \ \ \ \ \ \ \ \ \ \ }
\startm
\m{\vdash}\m{(}\m{x}\m{\in}\m{{\rm On}}\m{\rightarrow}\m{(}\m{\varphi}\m{%
\rightarrow}\m{[}\m{{\rm suc}}\m{x}\m{/}\m{x}\m{]}\m{\varphi}\m{)}\m{)}
\endm
\setbox\startprefix=\hbox{\tt \ \ tfindes.3\ \$e\ }
\setbox\contprefix=\hbox{\tt \ \ \ \ \ \ \ \ \ \ \ \ \ \ \ }
\startm
\m{\vdash}\m{(}\m{{\rm Lim}}\m{y}\m{\rightarrow}\m{(}\m{\forall}\m{x}\m{\in}%
\m{y}\m{\varphi}\m{\rightarrow}\m{[}\m{y}\m{/}\m{x}\m{]}\m{\varphi}\m{)}\m{)}
\endm
\setbox\startprefix=\hbox{\tt \ \ tfindes\ \$p\ }
\setbox\contprefix=\hbox{\tt \ \ \ \ \ \ \ \ \ \ \ \ \ }
\startm
\m{\vdash}\m{(}\m{x}\m{\in}\m{{\rm On}}\m{\rightarrow}\m{\varphi}\m{)}
\endm
\vskip 1ex

\noindent Principle of Transfinite Recursion.\index{transfinite
recursion} Theorem 7.41 of Takeuti and Zaring, p.~47.  Transfinite
recursion is the key theorem that allows arithmetic of ordinals to be
rigorously defined, and has many other important uses as well.
Hypotheses \texttt{tfr.1} and \texttt{tfr.2} specify a certain (proper)
class $ F$.  The complicated definition of $ F$ is not important in
itself; what is important is that there be such an $ F$ with the
required properties, and we show this by displaying $ F$ explicitly.
\texttt{tfr1} states that $ F$ is a function whose domain is the set of
ordinal numbers.  \texttt{tfr2} states that any value of $ F$ is
completely determined by its previous values and the values of an
auxiliary function, $G$.  \texttt{tfr3} states that $ F$ is unique,
i.e.\ it is the only function that satisfies \texttt{tfr1} and
\texttt{tfr2}.  Note that $ f$ is an individual variable like $x$ and
$y$; it is just a mnemonic to remind us that $A$ is a collection of
functions.

\vskip 0.5ex
\setbox\startprefix=\hbox{\tt \ \ tfr.1\ \$e\ }
\setbox\contprefix=\hbox{\tt \ \ \ \ \ \ \ \ \ \ \ }
\startm
\m{\vdash}\m{A}\m{=}\m{\{}\m{f}\m{|}\m{\exists}\m{x}\m{\in}\m{{\rm On}}\m{(}%
\m{f}\m{{\rm Fn}}\m{x}\m{\wedge}\m{\forall}\m{y}\m{\in}\m{x}\m{(}\m{f}\m{`}%
\m{y}\m{)}\m{=}\m{(}\m{G}\m{`}\m{(}\m{f}\m{\restriction}\m{y}\m{)}\m{)}\m{)}%
\m{\}}
\endm
\setbox\startprefix=\hbox{\tt \ \ tfr.2\ \$e\ }
\setbox\contprefix=\hbox{\tt \ \ \ \ \ \ \ \ \ \ \ }
\startm
\m{\vdash}\m{F}\m{=}\m{\bigcup}\m{A}
\endm
\setbox\startprefix=\hbox{\tt \ \ tfr1\ \$p\ }
\setbox\contprefix=\hbox{\tt \ \ \ \ \ \ \ \ \ \ }
\startm
\m{\vdash}\m{F}\m{{\rm Fn}}\m{{\rm On}}
\endm
\setbox\startprefix=\hbox{\tt \ \ tfr2\ \$p\ }
\setbox\contprefix=\hbox{\tt \ \ \ \ \ \ \ \ \ \ }
\startm
\m{\vdash}\m{(}\m{z}\m{\in}\m{{\rm On}}\m{\rightarrow}\m{(}\m{F}\m{`}\m{z}%
\m{)}\m{=}\m{(}\m{G}\m{`}\m{(}\m{F}\m{\restriction}\m{z}\m{)}\m{)}\m{)}
\endm
\setbox\startprefix=\hbox{\tt \ \ tfr3\ \$p\ }
\setbox\contprefix=\hbox{\tt \ \ \ \ \ \ \ \ \ \ }
\startm
\m{\vdash}\m{(}\m{(}\m{B}\m{{\rm Fn}}\m{{\rm On}}\m{\wedge}\m{\forall}\m{x}\m{%
\in}\m{{\rm On}}\m{(}\m{B}\m{`}\m{x}\m{)}\m{=}\m{(}\m{G}\m{`}\m{(}\m{B}\m{%
\restriction}\m{x}\m{)}\m{)}\m{)}\m{\rightarrow}\m{B}\m{=}\m{F}\m{)}
\endm
\vskip 1ex

\noindent The existence of omega (the class of natural numbers).\index{natural
number}\index{omega ($\omega$)}\index{Axiom of Infinity}  Axiom 7 of Takeuti
and Zaring, p.~43.  (This is the only theorem in this section requiring the
Axiom of Infinity.)

\vskip 0.5ex
\setbox\startprefix=\hbox{\tt \
\ omex\ \$p\ }
\setbox\contprefix=\hbox{\tt \ \ \ \ \ \ \ \ \ \ }
\startm
\m{\vdash}\m{\omega}\m{\in}\m{{\rm V}}
\endm
%\vskip 2ex


\section{Axioms for Real and Complex Numbers}\label{real}
\index{real and complex numbers!axioms for}

This section presents the axioms for real and complex numbers.  Analysis
textbooks implicitly or explicitly use these axioms or their equivalents
as their starting point.  In the database \texttt{set.mm}, we define real
and complex numbers as (rather complicated) specific sets and derive these
axioms as {\em theorems} from the axioms of ZF set theory, using a method
called Dedekind cuts.  We omit the details of this construction, which you can
follow if you wish using the \texttt{set.mm} database in conjunction with the
textbooks referenced therein.

Once we prove those theorems, we then restate these proven theorems as axioms.
This lets us easily identify which axioms are needed for a particular complex number proof, without the obfuscation of the set theory used to derive them.
As a result,
the construction is actually unimportant other
than to show that sets exist that satisfy the axioms, and thus that the axioms
are consistent if ZF set theory is consistent.  When working with real numbers
you can think of them as being the actual sets resulting
from the construction (for definiteness), or you can
think of them as otherwise unspecified sets that happen to satisfy the axioms.
The derivation is not easy, but the fact that it works is quite remarkable
and lends support to the idea that ZFC set theory is all we need to
provide a foundation for essentially all of mathematics.

For the axioms we are given (or postulate) 8 classes:  $\mathbb{C}$ (the
set of complex numbers), $\mathbb{R}$ (the set of real numbers, a subset
of $\mathbb{C}$), $0$ (zero), $1$ (one), $i$ (square root of
$-1$), $+$ (plus), $\cdot$ (times), and
$<_{\mathbb{R}}$ (less than for just the real numbers).
Subtraction and division are defined terms and are not part of the
axioms; for their definitions see \texttt{set.mm}.

Note that the notation $(A+B)$ (and similarly $(A\cdot B)$) specifies a class
called an {\em operation},\index{operation} and is the function value of the
class $+$ at ordered pair $\langle A,B \rangle$.  An operation is defined by
statement \texttt{df-opr} on p.~\pageref{dfopr}.
The notation $A <_{\mathbb{R}} B$ specifies a
wff called a {\em binary relation}\index{binary relation} and means $\langle A,B \rangle \in \,<_{\mathbb{R}}$, as defined by statement \texttt{df-br} on p.~\pageref{dfbr}.

Our set of 8 given classes is assumed to satisfy the following 22 axioms
(in the axioms listed below, $<$ really means $<_{\mathbb{R}}$).

\vskip 2ex

\noindent 1. The real numbers are a subset of the complex numbers.

%\vskip 0.5ex
\setbox\startprefix=\hbox{\tt \ \ ax-resscn\ \$p\ }
\setbox\contprefix=\hbox{\tt \ \ \ \ \ \ \ \ \ \ \ \ \ \ }
\startm
\m{\vdash}\m{\mathbb{R}}\m{\subseteq}\m{\mathbb{C}}
\endm
%\vskip 1ex

\noindent 2. One is a complex number.

%\vskip 0.5ex
\setbox\startprefix=\hbox{\tt \ \ ax-1cn\ \$p\ }
\setbox\contprefix=\hbox{\tt \ \ \ \ \ \ \ \ \ \ \ }
\startm
\m{\vdash}\m{1}\m{\in}\m{\mathbb{C}}
\endm
%\vskip 1ex

\noindent 3. The imaginary number $i$ is a complex number.

%\vskip 0.5ex
\setbox\startprefix=\hbox{\tt \ \ ax-icn\ \$p\ }
\setbox\contprefix=\hbox{\tt \ \ \ \ \ \ \ \ \ \ \ }
\startm
\m{\vdash}\m{i}\m{\in}\m{\mathbb{C}}
\endm
%\vskip 1ex

\noindent 4. Complex numbers are closed under addition.

%\vskip 0.5ex
\setbox\startprefix=\hbox{\tt \ \ ax-addcl\ \$p\ }
\setbox\contprefix=\hbox{\tt \ \ \ \ \ \ \ \ \ \ \ \ \ }
\startm
\m{\vdash}\m{(}\m{(}\m{A}\m{\in}\m{\mathbb{C}}\m{\wedge}\m{B}\m{\in}\m{\mathbb{C}}%
\m{)}\m{\rightarrow}\m{(}\m{A}\m{+}\m{B}\m{)}\m{\in}\m{\mathbb{C}}\m{)}
\endm
%\vskip 1ex

\noindent 5. Real numbers are closed under addition.

%\vskip 0.5ex
\setbox\startprefix=\hbox{\tt \ \ ax-addrcl\ \$p\ }
\setbox\contprefix=\hbox{\tt \ \ \ \ \ \ \ \ \ \ \ \ \ \ }
\startm
\m{\vdash}\m{(}\m{(}\m{A}\m{\in}\m{\mathbb{R}}\m{\wedge}\m{B}\m{\in}\m{\mathbb{R}}%
\m{)}\m{\rightarrow}\m{(}\m{A}\m{+}\m{B}\m{)}\m{\in}\m{\mathbb{R}}\m{)}
\endm
%\vskip 1ex

\noindent 6. Complex numbers are closed under multiplication.

%\vskip 0.5ex
\setbox\startprefix=\hbox{\tt \ \ ax-mulcl\ \$p\ }
\setbox\contprefix=\hbox{\tt \ \ \ \ \ \ \ \ \ \ \ \ \ }
\startm
\m{\vdash}\m{(}\m{(}\m{A}\m{\in}\m{\mathbb{C}}\m{\wedge}\m{B}\m{\in}\m{\mathbb{C}}%
\m{)}\m{\rightarrow}\m{(}\m{A}\m{\cdot}\m{B}\m{)}\m{\in}\m{\mathbb{C}}\m{)}
\endm
%\vskip 1ex

\noindent 7. Real numbers are closed under multiplication.

%\vskip 0.5ex
\setbox\startprefix=\hbox{\tt \ \ ax-mulrcl\ \$p\ }
\setbox\contprefix=\hbox{\tt \ \ \ \ \ \ \ \ \ \ \ \ \ \ }
\startm
\m{\vdash}\m{(}\m{(}\m{A}\m{\in}\m{\mathbb{R}}\m{\wedge}\m{B}\m{\in}\m{\mathbb{R}}%
\m{)}\m{\rightarrow}\m{(}\m{A}\m{\cdot}\m{B}\m{)}\m{\in}\m{\mathbb{R}}\m{)}
\endm
%\vskip 1ex

\noindent 8. Multiplication of complex numbers is commutative.

%\vskip 0.5ex
\setbox\startprefix=\hbox{\tt \ \ ax-mulcom\ \$p\ }
\setbox\contprefix=\hbox{\tt \ \ \ \ \ \ \ \ \ \ \ \ \ \ }
\startm
\m{\vdash}\m{(}\m{(}\m{A}\m{\in}\m{\mathbb{C}}\m{\wedge}\m{B}\m{\in}\m{\mathbb{C}}%
\m{)}\m{\rightarrow}\m{(}\m{A}\m{\cdot}\m{B}\m{)}\m{=}\m{(}\m{B}\m{\cdot}\m{A}%
\m{)}\m{)}
\endm
%\vskip 1ex

\noindent 9. Addition of complex numbers is associative.

%\vskip 0.5ex
\setbox\startprefix=\hbox{\tt \ \ ax-addass\ \$p\ }
\setbox\contprefix=\hbox{\tt \ \ \ \ \ \ \ \ \ \ \ \ \ \ }
\startm
\m{\vdash}\m{(}\m{(}\m{A}\m{\in}\m{\mathbb{C}}\m{\wedge}\m{B}\m{\in}\m{\mathbb{C}}%
\m{\wedge}\m{C}\m{\in}\m{\mathbb{C}}\m{)}\m{\rightarrow}\m{(}\m{(}\m{A}\m{+}%
\m{B}\m{)}\m{+}\m{C}\m{)}\m{=}\m{(}\m{A}\m{+}\m{(}\m{B}\m{+}\m{C}\m{)}\m{)}%
\m{)}
\endm
%\vskip 1ex

\noindent 10. Multiplication of complex numbers is associative.

%\vskip 0.5ex
\setbox\startprefix=\hbox{\tt \ \ ax-mulass\ \$p\ }
\setbox\contprefix=\hbox{\tt \ \ \ \ \ \ \ \ \ \ \ \ \ \ }
\startm
\m{\vdash}\m{(}\m{(}\m{A}\m{\in}\m{\mathbb{C}}\m{\wedge}\m{B}\m{\in}\m{\mathbb{C}}%
\m{\wedge}\m{C}\m{\in}\m{\mathbb{C}}\m{)}\m{\rightarrow}\m{(}\m{(}\m{A}\m{\cdot}%
\m{B}\m{)}\m{\cdot}\m{C}\m{)}\m{=}\m{(}\m{A}\m{\cdot}\m{(}\m{B}\m{\cdot}\m{C}%
\m{)}\m{)}\m{)}
\endm
%\vskip 1ex

\noindent 11. Multiplication distributes over addition for complex numbers.

%\vskip 0.5ex
\setbox\startprefix=\hbox{\tt \ \ ax-distr\ \$p\ }
\setbox\contprefix=\hbox{\tt \ \ \ \ \ \ \ \ \ \ \ \ \ }
\startm
\m{\vdash}\m{(}\m{(}\m{A}\m{\in}\m{\mathbb{C}}\m{\wedge}\m{B}\m{\in}\m{\mathbb{C}}%
\m{\wedge}\m{C}\m{\in}\m{\mathbb{C}}\m{)}\m{\rightarrow}\m{(}\m{A}\m{\cdot}\m{(}%
\m{B}\m{+}\m{C}\m{)}\m{)}\m{=}\m{(}\m{(}\m{A}\m{\cdot}\m{B}\m{)}\m{+}\m{(}%
\m{A}\m{\cdot}\m{C}\m{)}\m{)}\m{)}
\endm
%\vskip 1ex

\noindent 12. The square of $i$ equals $-1$ (expressed as $i$-squared plus 1 is
0).

%\vskip 0.5ex
\setbox\startprefix=\hbox{\tt \ \ ax-i2m1\ \$p\ }
\setbox\contprefix=\hbox{\tt \ \ \ \ \ \ \ \ \ \ \ \ }
\startm
\m{\vdash}\m{(}\m{(}\m{i}\m{\cdot}\m{i}\m{)}\m{+}\m{1}\m{)}\m{=}\m{0}
\endm
%\vskip 1ex

\noindent 13. One and zero are distinct.

%\vskip 0.5ex
\setbox\startprefix=\hbox{\tt \ \ ax-1ne0\ \$p\ }
\setbox\contprefix=\hbox{\tt \ \ \ \ \ \ \ \ \ \ \ \ }
\startm
\m{\vdash}\m{1}\m{\ne}\m{0}
\endm
%\vskip 1ex

\noindent 14. One is an identity element for real multiplication.

%\vskip 0.5ex
\setbox\startprefix=\hbox{\tt \ \ ax-1rid\ \$p\ }
\setbox\contprefix=\hbox{\tt \ \ \ \ \ \ \ \ \ \ \ }
\startm
\m{\vdash}\m{(}\m{A}\m{\in}\m{\mathbb{R}}\m{\rightarrow}\m{(}\m{A}\m{\cdot}\m{1}%
\m{)}\m{=}\m{A}\m{)}
\endm
%\vskip 1ex

\noindent 15. Every real number has a negative.

%\vskip 0.5ex
\setbox\startprefix=\hbox{\tt \ \ ax-rnegex\ \$p\ }
\setbox\contprefix=\hbox{\tt \ \ \ \ \ \ \ \ \ \ \ \ \ \ }
\startm
\m{\vdash}\m{(}\m{A}\m{\in}\m{\mathbb{R}}\m{\rightarrow}\m{\exists}\m{x}\m{\in}%
\m{\mathbb{R}}\m{(}\m{A}\m{+}\m{x}\m{)}\m{=}\m{0}\m{)}
\endm
%\vskip 1ex

\noindent 16. Every nonzero real number has a reciprocal.

%\vskip 0.5ex
\setbox\startprefix=\hbox{\tt \ \ ax-rrecex\ \$p\ }
\setbox\contprefix=\hbox{\tt \ \ \ \ \ \ \ \ \ \ \ \ \ \ }
\startm
\m{\vdash}\m{(}\m{A}\m{\in}\m{\mathbb{R}}\m{\rightarrow}\m{(}\m{A}\m{\ne}\m{0}%
\m{\rightarrow}\m{\exists}\m{x}\m{\in}\m{\mathbb{R}}\m{(}\m{A}\m{\cdot}%
\m{x}\m{)}\m{=}\m{1}\m{)}\m{)}
\endm
%\vskip 1ex

\noindent 17. A complex number can be expressed in terms of two reals.

%\vskip 0.5ex
\setbox\startprefix=\hbox{\tt \ \ ax-cnre\ \$p\ }
\setbox\contprefix=\hbox{\tt \ \ \ \ \ \ \ \ \ \ \ \ }
\startm
\m{\vdash}\m{(}\m{A}\m{\in}\m{\mathbb{C}}\m{\rightarrow}\m{\exists}\m{x}\m{\in}%
\m{\mathbb{R}}\m{\exists}\m{y}\m{\in}\m{\mathbb{R}}\m{A}\m{=}\m{(}\m{x}\m{+}\m{(}%
\m{y}\m{\cdot}\m{i}\m{)}\m{)}\m{)}
\endm
%\vskip 1ex

\noindent 18. Ordering on reals satisfies strict trichotomy.

%\vskip 0.5ex
\setbox\startprefix=\hbox{\tt \ \ ax-pre-lttri\ \$p\ }
\setbox\contprefix=\hbox{\tt \ \ \ \ \ \ \ \ \ \ \ \ \ }
\startm
\m{\vdash}\m{(}\m{(}\m{A}\m{\in}\m{\mathbb{R}}\m{\wedge}\m{B}\m{\in}\m{\mathbb{R}}%
\m{)}\m{\rightarrow}\m{(}\m{A}\m{<}\m{B}\m{\leftrightarrow}\m{\lnot}\m{(}\m{A}%
\m{=}\m{B}\m{\vee}\m{B}\m{<}\m{A}\m{)}\m{)}\m{)}
\endm
%\vskip 1ex

\noindent 19. Ordering on reals is transitive.

%\vskip 0.5ex
\setbox\startprefix=\hbox{\tt \ \ ax-pre-lttrn\ \$p\ }
\setbox\contprefix=\hbox{\tt \ \ \ \ \ \ \ \ \ \ \ \ \ }
\startm
\m{\vdash}\m{(}\m{(}\m{A}\m{\in}\m{\mathbb{R}}\m{\wedge}\m{B}\m{\in}\m{\mathbb{R}}%
\m{\wedge}\m{C}\m{\in}\m{\mathbb{R}}\m{)}\m{\rightarrow}\m{(}\m{(}\m{A}\m{<}%
\m{B}\m{\wedge}\m{B}\m{<}\m{C}\m{)}\m{\rightarrow}\m{A}\m{<}\m{C}\m{)}\m{)}
\endm
%\vskip 1ex

\noindent 20. Ordering on reals is preserved after addition to both sides.

%\vskip 0.5ex
\setbox\startprefix=\hbox{\tt \ \ ax-pre-ltadd\ \$p\ }
\setbox\contprefix=\hbox{\tt \ \ \ \ \ \ \ \ \ \ \ \ \ }
\startm
\m{\vdash}\m{(}\m{(}\m{A}\m{\in}\m{\mathbb{R}}\m{\wedge}\m{B}\m{\in}\m{\mathbb{R}}%
\m{\wedge}\m{C}\m{\in}\m{\mathbb{R}}\m{)}\m{\rightarrow}\m{(}\m{A}\m{<}\m{B}\m{%
\rightarrow}\m{(}\m{C}\m{+}\m{A}\m{)}\m{<}\m{(}\m{C}\m{+}\m{B}\m{)}\m{)}\m{)}
\endm
%\vskip 1ex

\noindent 21. The product of two positive reals is positive.

%\vskip 0.5ex
\setbox\startprefix=\hbox{\tt \ \ ax-pre-mulgt0\ \$p\ }
\setbox\contprefix=\hbox{\tt \ \ \ \ \ \ \ \ \ \ \ \ \ \ }
\startm
\m{\vdash}\m{(}\m{(}\m{A}\m{\in}\m{\mathbb{R}}\m{\wedge}\m{B}\m{\in}\m{\mathbb{R}}%
\m{)}\m{\rightarrow}\m{(}\m{(}\m{0}\m{<}\m{A}\m{\wedge}\m{0}%
\m{<}\m{B}\m{)}\m{\rightarrow}\m{0}\m{<}\m{(}\m{A}\m{\cdot}\m{B}\m{)}%
\m{)}\m{)}
\endm
%\vskip 1ex

\noindent 22. A non-empty, bounded-above set of reals has a supremum.

%\vskip 0.5ex
\setbox\startprefix=\hbox{\tt \ \ ax-pre-sup\ \$p\ }
\setbox\contprefix=\hbox{\tt \ \ \ \ \ \ \ \ \ \ \ }
\startm
\m{\vdash}\m{(}\m{(}\m{A}\m{\subseteq}\m{\mathbb{R}}\m{\wedge}\m{A}\m{\ne}\m{%
\varnothing}\m{\wedge}\m{\exists}\m{x}\m{\in}\m{\mathbb{R}}\m{\forall}\m{y}\m{%
\in}\m{A}\m{\,y}\m{<}\m{x}\m{)}\m{\rightarrow}\m{\exists}\m{x}\m{\in}\m{%
\mathbb{R}}\m{(}\m{\forall}\m{y}\m{\in}\m{A}\m{\lnot}\m{x}\m{<}\m{y}\m{\wedge}\m{%
\forall}\m{y}\m{\in}\m{\mathbb{R}}\m{(}\m{y}\m{<}\m{x}\m{\rightarrow}\m{\exists}%
\m{z}\m{\in}\m{A}\m{\,y}\m{<}\m{z}\m{)}\m{)}\m{)}
\endm

% NOTE: The \m{...} expressions above could be represented as
% $ \vdash ( ( A \subseteq \mathbb{R} \wedge A \ne \varnothing \wedge \exists x \in \mathbb{R} \forall y \in A \,y < x ) \rightarrow \exists x \in \mathbb{R} ( \forall y \in A \lnot x < y \wedge \forall y \in \mathbb{R} ( y < x \rightarrow \exists z \in A \,y < z ) ) ) $

\vskip 2ex

This completes the set of axioms for real and complex numbers.  You may
wish to look at how subtraction, division, and decimal numbers
are defined in \texttt{set.mm}, and for fun look at the proof of $2+
2 = 4$ (theorem \texttt{2p2e4} in \texttt{set.mm})
as discussed in section \ref{2p2e4}.

In \texttt{set.mm} we define the non-negative integers $\mathbb{N}$, the integers
$\mathbb{Z}$, and the rationals $\mathbb{Q}$ as subsets of $\mathbb{R}$.  This leads
to the nice inclusion $\mathbb{N} \subseteq \mathbb{Z} \subseteq \mathbb{Q} \subseteq
\mathbb{R} \subseteq \mathbb{C}$, giving us a uniform framework in which, for
example, a property such as commutativity of complex number addition
automatically applies to integers.  The natural numbers $\mathbb{N}$
are different from the set $\omega$ we defined earlier, but both satisfy
Peano's postulates.

\subsubsection{Complex Number Axioms in Analysis Texts}

Most analysis texts construct complex numbers as ordered pairs of reals,
leading to construction-dependent properties that satisfy these axioms
but are not stated in their pure form.  (This is also done in
\texttt{set.mm} but our axioms are extracted from that construction.)
Other texts will simply state that $\mathbb{R}$ is a ``complete ordered
subfield of $\mathbb{C}$,'' leading to redundant axioms when this phrase
is completely expanded out.  In fact I have not seen a text with the
axioms in the explicit form above.
None of these axioms is unique individually, but this carefully worked out
collection of axioms is the result of years of work
by the Metamath community.

\subsubsection{Eliminating Unnecessary Complex Number Axioms}

We once had more axioms for real and complex numbers, but over years of time
we (the Metamath community)
have found ways to eliminate them (by proving them from other axioms)
or weaken them (by making weaker claims without reducing what
can be proved).
In particular, here are statements that used to be complex number
axioms but have since been formally proven (with Metamath) to be redundant:

\begin{itemize}
\item
  $\mathbb{C} \in V$.
  At one time this was listed as a ``complex number axiom''
  However, this is not properly speaking a complex number axiom,
  and in any case its proof uses axioms of set theory.
  Proven redundant by Mario Carneiro\index{Carneiro, Mario} on
  17-Nov-2014 (see \texttt{axcnex}).
\item
  $((A \in \mathbb{C} \land B \in \mathbb{C}$) $\rightarrow$
  $(A + B) = (B + A))$.
  Proved redundant by Eric Schmidt\index{Schmidt, Eric} on 19-Jun-2012,
  and formalized by Scott Fenton\index{Fenton, Scott} on 3-Jan-2013
  (see \texttt{addcom}).
\item
  $(A \in \mathbb{C} \rightarrow (A + 0) = A)$.
  Proved redundant by Eric Schmidt on 19-Jun-2012,
  and formalized by Scott Fenton on 3-Jan-2013
  (see \texttt{addid1}).
\item
  $(A \in \mathbb{C} \rightarrow \exists x \in \mathbb{C} (A + x) = 0)$.
  Proved redundant by Eric Schmidt and formalized on 21-May-2007
  (see \texttt{cnegex}).
\item
  $((A \in \mathbb{C} \land A \ne 0) \rightarrow \exists x \in \mathbb{C} (A \cdot x) = 1)$.
  Proved redundant by Eric Schmidt and formalized on 22-May-2007
  (see \texttt{recex}).
\item
  $0 \in \mathbb{R}$.
  Proved redundant by Eric Schmidt on 19-Feb-2005 and formalized 21-May-2007
  (see \texttt{0re}).
\end{itemize}

We could eliminate 0 as an axiomatic object by defining it as
$( ( i \cdot i ) + 1 )$
and replacing it with this expression throughout the axioms. If this
is done, axiom ax-i2m1 becomes redundant. However, the remaining axioms
would become longer and less intuitive.

Eric Schmidt's paper analyzing this axiom system \cite{Schmidt}
presented a proof that these remaining axioms,
with the possible exception of ax-mulcom, are independent of the others.
It is currently an open question if ax-mulcom is independent of the others.

\section{Two Plus Two Equals Four}\label{2p2e4}

Here is a proof that $2 + 2 = 4$, as proven in the theorem \texttt{2p2e4}
in the database \texttt{set.mm}.
This is a useful demonstration of what a Metamath proof can look like.
This proof may have more steps than you're used to, but each step is rigorously
proven all the way back to the axioms of logic and set theory.
This display was originally generated by the Metamath program
as an {\sc HTML} file.

In the table showing the proof ``Step'' is the sequential step number,
while its associated ``Expression'' is an expression that we have proved.
``Ref'' is the name of a theorem or axiom that justifies that expression,
and ``Hyp'' refers to previous steps (if any) that the theorem or axiom
needs so that we can use it.  Expressions are indented further than
the expressions that depend on them to show their interdependencies.

\begin{table}[!htbp]
\caption{Two plus two equals four}
\begin{tabular}{lllll}
\textbf{Step} & \textbf{Hyp} & \textbf{Ref} & \textbf{Expression} & \\
1  &       & df-2    & $ \; \; \vdash 2 = 1 + 1$  & \\
2  & 1     & oveq2i  & $ \; \vdash (2 + 2) = (2 + (1 + 1))$ & \\
3  &       & df-4    & $ \; \; \vdash 4 = (3 + 1)$ & \\
4  &       & df-3    & $ \; \; \; \vdash 3 = (2 + 1)$ & \\
5  & 4     & oveq1i  & $ \; \; \vdash (3 + 1) = ((2 + 1) + 1)$ & \\
6  &       & 2cn     & $ \; \; \; \vdash 2 \in \mathbb{C}$ & \\
7  &       & ax-1cn  & $ \; \; \; \vdash 1 \in \mathbb{C}$ & \\
8  & 6,7,7 & addassi & $ \; \; \vdash ((2 + 1) + 1) = (2 + (1 + 1))$ & \\
9  & 3,5,8 & 3eqtri  & $ \; \vdash 4 = (2 + (1 + 1))$ & \\
10 & 2,9   & eqtr4i  & $ \vdash (2 + 2) = 4$ & \\
\end{tabular}
\end{table}

Step 1 says that we can assert that $2 = 1 + 1$ because it is
justified by \texttt{df-2}.
What is \texttt{df-2}?
It is simply the definition of $2$, which in our system is defined as being
equal to $1 + 1$.  This shows how we can use definitions in proofs.

Look at Step 2 of the proof. In the Ref column, we see that it references
a previously proved theorem, \texttt{oveq2i}.
It turns out that
theorem \texttt{oveq2i} requires a
hypothesis, and in the Hyp column of Step 2 we indicate that Step 1 will
satisfy (match) this hypothesis.
If we looked at \texttt{oveq2i}
we would find that it proves that given some hypothesis
$A = B$, we can prove that $( C F A ) = ( C F B )$.
If we use \texttt{oveq2i} and apply step 1's result as the hypothesis,
that will mean that $A = 2$ and $B = ( 1 + 1 )$ within this use of
\texttt{oveq2i}.
We are free to select any value of $C$ and $F$ (subject to syntax constraints),
so we are free to select $C = 2$ and $F = +$,
producing our desired result,
$ (2 + 2) = (2 + (1 + 1))$.

Step 2 is an example of substitution.
In the end, every step in every proof uses only this one substitution rule.
All the rules of logic, and all the axioms, are expressed so that
they can be used via this one substitution rule.
So once you master substitution, you can master every Metamath proof,
no exceptions.

Each step is clear and can be immediately checked.
In the {\sc HTML} display you can even click on each reference to see why it is
justified, making it easy to see why the proof works.

\section{Deduction}\label{deduction}

Strictly speaking,
a deduction (also called an inference) is a kind of statement that needs
some hypotheses to be true in order for its conclusion to be true.
A theorem, on the other hand, has no hypotheses.
Informally we often call both of them theorems, but in this section we
will stick to the strict definitions.

It sometimes happens that we have proved a deduction of the form
$\varphi \Rightarrow \psi$\index{$\Rightarrow$}
(given hypothesis $\varphi$ we can prove $\psi$)
and we want to then prove a theorem of the form
$\varphi \rightarrow \psi$.

Converting a deduction (which uses a hypothesis) into a theorem
(which does not) is not as simple as you might think.
The deduction says, ``if we can prove $\varphi$ then we can prove $\psi$,''
which is in some sense weaker than saying
``$\varphi$ implies $\psi$.''
There is no axiom of logic that permits us to directly obtain the theorem
given the deduction.\footnote{
The conversion of a deduction to a theorem does not even hold in general
for quantum propositional calculus,
which is a weak subset of classical propositional calculus.
It has been shown that adding the Standard Deduction Theorem (discussed below)
to quantum propositional calculus turns it into classical
propositional calculus!
}

This is in contrast to going the other way.
If we have the theorem ($\varphi \rightarrow \psi$),
it is easy to recover the deduction
($\varphi \Rightarrow \psi$)
using modus ponens\index{modus ponens}
(\texttt{ax-mp}; see section \ref{axmp}).

In the following subsections we first discuss the standard deduction theorem
(the traditional but awkward way to convert deductions into theorems) and
the weak deduction theorem (a limited version of the standard deduction
theorem that is easier to use and was once widely used in
\texttt{set.mm}\index{set theory database (\texttt{set.mm})}).
In section \ref{deductionstyle} we discuss
deduction style, the newer approach we now recommend in most cases.
Deduction style uses ``deduction form,'' a form that
prefixes each hypothesis (other than definitions) and the
conclusion with a universal antecedent (``$\varphi \rightarrow$'').
Deduction style is widely used in \texttt{set.mm},
so it is useful to understand it and \textit{why} it is widely used.
Section \ref{naturaldeduction}
briefly discusses our approach for using natural deduction
within \texttt{set.mm},
as that approach is deeply related to deduction style.
We conclude with a summary of the strengths of
our approach, which we believe are compelling.

\subsection{The Standard Deduction Theorem}\label{standarddeductiontheorem}

It is possible to make use of information
contained in the deduction or its proof to assist us with the proof of
the related theorem.
In traditional logic books, there is a metatheorem called the
Deduction Theorem\index{Deduction Theorem}\index{Standard Deduction Theorem},
discovered independently by Herbrand and Tarski around 1930.
The Deduction Theorem, which we often call the Standard Deduction Theorem,
provides an algorithm for constructing a proof of a theorem from the
proof of its corresponding deduction. See, for example,
\cite[p.~56]{Margaris}\index{Margaris, Angelo}.
To construct a proof for a theorem, the
algorithm looks at each step in the proof of the original deduction and
rewrites the step with several steps wherein the hypothesis is eliminated
and becomes an antecedent.

In ordinary mathematics, no one actually carries out the algorithm,
because (in its most basic form) it involves an exponential explosion of
the number of proof steps as more hypotheses are eliminated. Instead,
the Standard Deduction Theorem is invoked simply to claim that it can
be done in principle, without actually doing it.
What's more, the algorithm is not as simple as it might first appear
when applying it rigorously.
There is a subtle restriction on the Standard Deduction Theorem
that must be taken into account involving the axiom of generalization
when working with predicate calculus (see the literature for more detail).

One of the goals of Metamath is to let you plainly see, with as few
underlying concepts as possible, how mathematics can be derived directly
from the axioms, and not indirectly according to some hidden rules
buried inside a program or understood only by logicians. If we added
the Standard Deduction Theorem to the language and proof verifier,
that would greatly complicate both and largely defeat Metamath's goal
of simplicity. In principle, we could show direct proofs by expanding
out the proof steps generated by the algorithm of the Standard Deduction
Theorem, but that is not feasible in practice because the number of proof
steps quickly becomes huge, even astronomical.
Since the algorithm of the Standard Deduction Theorem is driven by the proof,
we would have to go through that proof
all over again---starting from axioms---in order to obtain the theorem form.
In terms of proof length, there would be no savings over just
proving the theorem directly instead of first proving the deduction form.

\subsection{Weak Deduction Theorem}\label{weakdeductiontheorem}

We have developed
a more efficient method for proving a theorem from a deduction
that can be used instead of the Standard Deduction Theorem
in many (but not all) cases.
We call this more efficient method the
Weak Deduction Theorem\index{Weak Deduction Theorem}.\footnote{
There is also an unrelated ``Weak Deduction Theorem''
in the field of relevance logic, so to avoid confusion we could call
ours the ``Weak Deduction Theorem for Classical Logic.''}
Unlike the Standard Deduction Theorem, the Weak Deduction Theorem produces the
theorem directly from a special substitution instance of the deduction,
using a small, fixed number of steps roughly proportional to the length
of the final theorem.

If you come to a proof referencing the Weak Deduction Theorem
\texttt{dedth} (or one of its variants \texttt{dedthxx}),
here is how to follow the proof without getting into the details:
just click on the theorem referenced in the step
just before the reference to \texttt{dedth} and ignore everything else.
Theorem \texttt{dedth} simply turns a hypothesis into an antecedent
(i.e. the hypothesis followed by $\rightarrow$
is placed in front of the assertion, and the hypothesis
itself is eliminated) given certain conditions.

The Weak Deduction Theorem
eliminates a hypothesis $\varphi$, making it become an antecedent.
It does this by proving an expression
$ \varphi \rightarrow \psi $ given two hypotheses:
(1)
$ ( A = {\rm if} ( \varphi , A , B ) \rightarrow ( \varphi \leftrightarrow \chi ) ) $
and
(2) $\chi$.
Note that it requires that a proof exists for $\varphi$ when the class variable
$A$ is replaced with a specific class $B$. The hypothesis $\chi$
should be assigned to the inference.
You can see the details of the proof of the Weak Deduction Theorem
in theorem \texttt{dedth}.

The Weak Deduction Theorem
is probably easier to understand by studying proofs that make use of it.
For example, let's look at the proof of \texttt{renegcl}, which proves that
$ \vdash ( A \in \mathbb{R} \rightarrow - A \in \mathbb{R} )$:

\needspace{4\baselineskip}
\begin{longtabu} {l l l X}
\textbf{Step} & \textbf{Hyp} & \textbf{Ref} & \textbf{Expression} \\
  1 &  & negeq &
  $\vdash$ $($ $A$ $=$ ${\rm if}$ $($ $A$ $\in$ $\mathbb{R}$ $,$ $A$ $,$ $1$ $)$ $\rightarrow$
  $\textrm{-}$ $A$ $=$ $\textrm{-}$ ${\rm if}$ $($ $A$ $\in$ $\mathbb{R}$
  $,$ $A$ $,$ $1$ $)$ $)$ \\
 2 & 1 & eleq1d &
    $\vdash$ $($ $A$ $=$ ${\rm if}$ $($ $A$ $\in$ $\mathbb{R}$ $,$ $A$ $,$ $1$ $)$ $\rightarrow$ $($
    $\textrm{-}$ $A$ $\in$ $\mathbb{R}$ $\leftrightarrow$
    $\textrm{-}$ ${\rm if}$ $($ $A$ $\in$ $\mathbb{R}$ $,$ $A$ $,$ $1$ $)$ $\in$
    $\mathbb{R}$ $)$ $)$ \\
 3 &  & 1re & $\vdash 1 \in \mathbb{R}$ \\
 4 & 3 & elimel &
   $\vdash {\rm if} ( A \in \mathbb{R} , A , 1 ) \in \mathbb{R}$ \\
 5 & 4 & renegcli &
   $\vdash \textrm{-} {\rm if} ( A \in \mathbb{R} , A , 1 ) \in \mathbb{R}$ \\
 6 & 2,5 & dedth &
   $\vdash ( A \in \mathbb{R} \rightarrow \textrm{-} A \in \mathbb{R}$ ) \\
\end{longtabu}

The somewhat strange-looking steps in \texttt{renegcl} before step 5 are
technical stuff that makes this magic work, and they can be ignored
for a quick overview of the proof. To continue following the ``important''
part of the proof of \texttt{renegcl},
you can look at the reference to \texttt{renegcli} at step 5.

That said, let's briefly look at how
\texttt{renegcl} uses the
Weak Deduction Theorem (\texttt{dedth}) to do its job,
in case you want to do something similar or want understand it more deeply.
Let's work backwards in the proof of \texttt{renegcl}.
Step 6 applies \texttt{dedth} to produce our goal result
$ \vdash ( A \in \mathbb{R} \rightarrow\, - A \in \mathbb{R} )$.
This requires on the one hand the (substituted) deduction
\texttt{renegcli} in step 5.
By itself \texttt{renegcli} proves the deduction
$ \vdash A \in \mathbb{R} \Rightarrow\, \vdash - A \in \mathbb{R}$;
this is the deduction form we are trying to turn into theorem form,
and thus
\texttt{renegcli} has a separate hypothesis that must be fulfilled.
To fulfill the hypothesis of the invocation of
\texttt{renegcli} in step 5, it is eventually
reduced to the already proven theorem $1 \in \mathbb{R}$ in step 3.
Step 4 connects steps 3 and 5; step 4 invokes
\texttt{elimel}, a special case of \texttt{elimhyp} that eliminates
a membership hypothesis for the weak deduction theorem.
On the other hand, the equivalence of the conclusion of
\texttt{renegcl}
$( - A \in \mathbb{R} )$ and the substituted conclusion of
\texttt{renegcli} must be proven, which is done in steps 2 and 1.

The weak deduction theorem has limitations.
In particular, we must be able to prove a special case of the deduction's
hypothesis as a stand-alone theorem.
For example, we used $1 \in \mathbb{R}$ in step 3 of \texttt{renegcl}.

We used to use the weak deduction theorem
extensively within \texttt{set.mm}.
However, we now recommend applying ``deduction style''
instead in most cases, as deduction style is
often an easier and clearer approach.
Therefore, we will now describe deduction style.

\subsection{Deduction Style}\label{deductionstyle}

We now prefer to write assertions in ``deduction form''
instead of writing a proof that would require use of the standard or
weak deduction theorem.
We call this appraoch
``deduction style.''\index{deduction style}

It will be easier to explain this by first defining some terms:

\begin{itemize}
\item \textbf{closed form}\index{closed form}\index{forms!closed}:
A kind of assertion (theorem) with no hypotheses.
Typically its label has no special suffix.
An example is \texttt{unss}, which states:
$\vdash ( ( A \subseteq C \wedge B \subseteq C ) \leftrightarrow ( A \cup B )
\subseteq C )\label{eq:unss}$
\item \textbf{deduction form}\index{deduction form}\index{forms!deduction}:
A kind of assertion with one or more hypotheses
where the conclusion is an implication with
a wff variable as the antecedent (usually $\varphi$), and every hypothesis
(\$e statement)
is either (1) an implication with the same antecedent as the conclusion or
(2) a definition.
A definition
can be for a class variable (this is a class variable followed by ``='')
or a wff variable (this is a wff variable followed by $\leftrightarrow$);
class variable definitions are more common.
In practice, a proof
in deduction form will also contain many steps that are implications
where the antecedent is either that wff variable (normally $\varphi$)
or is
a conjunction (...$\land$...) including that wff variable ($\varphi$).
If an assertion is in deduction form, and other forms are also available,
then we suffix its label with ``d.''
An example is \texttt{unssd}, which states\footnote{
For brevity we show here (and in other places)
a $\&$\index{$\&$} between hypotheses\index{hypotheses}
and a $\Rightarrow$\index{$\Rightarrow$}\index{conclusion}
between the hypotheses and the conclusion.
This notation is technically not part of the Metamath language, but is
instead a convenient abbreviation to show both the hypotheses and conclusion.}:
$\vdash ( \varphi \rightarrow A \subseteq C )\quad\&\quad \vdash ( \varphi
    \rightarrow B \subseteq C )\quad\Rightarrow\quad \vdash ( \varphi
    \rightarrow ( A \cup B ) \subseteq C )\label{eq:unssd}$
\item \textbf{inference form}\index{inference form}\index{forms!inference}:
A kind of assertion with one or more hypotheses that is not in deduction form
(e.g., there is no common antecedent).
If an assertion is in inference form, and other forms are also available,
then we suffix its label with ``i.''
An example is \texttt{unssi}, which states:
$\vdash A \subseteq C\quad\&\quad \vdash B \subseteq C\quad\Rightarrow\quad
    \vdash ( A \cup B ) \subseteq C\label{eq:unssi}$
\end{itemize}

When using deduction style we express an assertion in deduction form.
This form prefixes each hypothesis (other than definitions) and the
conclusion with a universal antecedent (``$\varphi \rightarrow$'').
The antecedent (e.g., $\varphi$)
mimics the context handled in the deduction theorem, eliminating
the need to directly use the deduction theorem.

Once you have an assertion in deduction form, you can easily convert it
to inference form or closed form:

\begin{itemize}
\item To
prove some assertion Ti in inference form, given assertion Td in deduction
form, there is a simple mechanical process you can use. First take each
Ti hypothesis and insert a \texttt{T.} $\rightarrow$ prefix (``true implies'')
using \texttt{a1i}. You
can then use the existing assertion Td to prove the resulting conclusion
with a \texttt{T.} $\rightarrow$ prefix.
Finally, you can remove that prefix using \texttt{trud},
resulting in the conclusion you wanted to prove.
\item To
prove some assertion T in closed form, given assertion Td in deduction
form, there is another simple mechanical process you can use. First,
select an expression that is the conjunction (...$\land$...) of all of the
consequents of every hypothesis of Td. Next, prove that this expression
implies each of the separate hypotheses of Td in turn by eliminating
conjuncts (there are a variety of proven assertions to do this, including
\texttt{simpl},
\texttt{simpr},
\texttt{3simpa},
\texttt{3simpb},
\texttt{3simpc},
\texttt{simp1},
\texttt{simp2},
and
\texttt{simp3}).
If the
expression has nested conjunctions, inner conjuncts can be broken out by
chaining the above theorems with \texttt{syl}
(see section \ref{syl}).\footnote{
There are actually many theorems
(labeled simp* such as \texttt{simp333}) that break out inner conjuncts in one
step, but rather than learning them you can just use the chaining we
just described to prove them, and then let the Metamath program command
\texttt{minimize{\char`\_}with}\index{\texttt{minimize{\char`\_}with} command}
figure out the right ones needed to collapse them.}
As your final step, you can then apply the already-proven assertion Td
(which is in deduction form), proving assertion T in closed form.
\end{itemize}

We can also easily convert any assertion T in closed form to its related
assertion Ti in inference form by applying
modus ponens\index{modus ponens} (see section \ref{axmp}).

The deduction form antecedent can also be used to represent the context
necessary to support natural deduction systems, so we will now
discuss natural deduction.

\subsection{Natural Deduction}\label{naturaldeduction}

Natural deduction\index{natural deduction}
(ND) systems, as such, were originally introduced in
1934 by two logicians working independently: Ja\'skowski and Gentzen. ND
systems are supposed to reconstruct, in a formally proper way, traditional
ways of mathematical reasoning (such as conditional proof, indirect proof,
and proof by cases). As reconstructions they were naturally influenced
by previous work, and many specific ND systems and notations have been
developed since their original work.

There are many ND variants, but
Indrzejczak \cite[p.~31-32]{Indrzejczak}\index{Indrzejczak, Andrzej}
suggests that any natural deductive system must satisfy at
least these three criteria:

\begin{itemize}
\item ``There are some means for entering assumptions into a proof and
also for eliminating them. Usually it requires some bookkeeping devices
for indicating the scope of an assumption, and showing that a part of
a proof depending on eliminated assumption is discharged.
\item There are no (or, at least, a very limited set of) axioms, because
their role is taken over by the set of primitive rules for introduction
and elimination of logical constants which means that elementary
inferences instead of formulae are taken as primitive.
\item (A genuine) ND system admits a lot of freedom in proof construction
and possibility of applying several proof search strategies, like
conditional proof, proof by cases, proof by reductio ad absurdum etc.''
\end{itemize}

The Metamath Proof Explorer (MPE) as defined in \texttt{set.mm}
is fundamentally a Hilbert-style system.
That is, MPE is based on a larger number of axioms (compared
to natural deduction systems), a very small set of rules of inference
(modus ponens), and the context is not changed by the rules of inference
in the middle of a proof. That said, MPE proofs can be developed using
the natural deduction (ND) approach as originally developed by Ja\'skowski
and Gentzen.

The most common and recommended approach for applying ND in MPE is to use
deduction form\index{deduction form}%
\index{forms!deduction}
and apply the MPE proven assertions that are equivalent to ND rules.
For example, MPE's \texttt{jca} is equivalent to ND rule $\land$-I
(and-insertion).
We maintain a list of equivalences that you may consult.
This approach for applying an ND approach within MPE relies on Metamath's
wff metavariables in an essential way, and is described in more detail
in the presentation ``Natural Deductions in the Metamath Proof Language''
by Mario Carneiro \cite{CarneiroND}\index{Carneiro, Mario}.

In this style many steps are an implication, whose antecedent mimics
the context ($\Gamma$) of most ND systems. To add an assumption, simply add
it to the implication antecedent (typically using
\texttt{simpr}),
and use that
new antecedent for all later claims in the same scope. If you wish to
use an assertion in an ND hypothesis scope that is outside the current
ND hypothesis scope, modify the assertion so that the ND hypothesis
assumption is added to its antecedent (typically using \texttt{adantr}). Most
proof steps will be proved using rules that have hypotheses and results
of the form $\varphi \rightarrow$ ...

An example may make this clearer.
Let's show theorem 5.5 of
\cite[p.~18]{Clemente}\index{Clemente Laboreo, Daniel}
along with a line by line translation using the usual
translation of natural deduction (ND) in the Metamath Proof Explorer
(MPE) notation (this is proof \texttt{ex-natded5.5}).
The proof's original goal was to prove
$\lnot \psi$ given two hypotheses,
$( \psi \rightarrow \chi )$ and $ \lnot \chi$.
We will translate these statements into MPE deduction form
by prefixing them all with $\varphi \rightarrow$.
As a result, in MPE the goal is stated as
$( \varphi \rightarrow \lnot \psi )$, and the two hypotheses are stated as
$( \varphi \rightarrow ( \psi \rightarrow \chi ) )$ and
$( \varphi \rightarrow \lnot \chi )$.

The following table shows the proof in Fitch natural deduction style
and its MPE equivalent.
The \textit{\#} column shows the original numbering,
\textit{MPE\#} shows the number in the equivalent MPE proof
(which we will show later),
\textit{ND Expression} shows the original proof claim in ND notation,
and \textit{MPE Translation} shows its translation into MPE
as discussed in this section.
The final columns show the rationale in ND and MPE respectively.

\needspace{4\baselineskip}
{\setlength{\extrarowsep}{4pt} % Keep rows from being too close together
\begin{longtabu}   { @{} c c X X X X }
\textbf{\#} & \textbf{MPE\#} & \textbf{ND Ex\-pres\-sion} &
\textbf{MPE Trans\-lation} & \textbf{ND Ration\-ale} &
\textbf{MPE Ra\-tio\-nale} \\
\endhead

1 & 2;3 &
$( \psi \rightarrow \chi )$ &
$( \varphi \rightarrow ( \psi \rightarrow \chi ) )$ &
Given &
\$e; \texttt{adantr} to put in ND hypothesis \\

2 & 5 &
$ \lnot \chi$ &
$( \varphi \rightarrow \lnot \chi )$ &
Given &
\$e; \texttt{adantr} to put in ND hypothesis \\

3 & 1 &
... $\vert$ $\psi$ &
$( \varphi \rightarrow \psi )$ &
ND hypothesis assumption &
\texttt{simpr} \\

4 & 4 &
... $\chi$ &
$( ( \varphi \land \psi ) \rightarrow \chi )$ &
$\rightarrow$\,E 1,3 &
\texttt{mpd} 1,3 \\

5 & 6 &
... $\lnot \chi$ &
$( ( \varphi \land \psi ) \rightarrow \lnot \chi )$ &
IT 2 &
\texttt{adantr} 5 \\

6 & 7 &
$\lnot \psi$ &
$( \varphi \rightarrow \lnot \psi )$ &
$\land$\,I 3,4,5 &
\texttt{pm2.65da} 4,6 \\

\end{longtabu}
}


The original used Latin letters; we have replaced them with Greek letters
to follow Metamath naming conventions and so that it is easier to follow
the Metamath translation. The Metamath line-for-line translation of
this natural deduction approach precedes every line with an antecedent
including $\varphi$ and uses the Metamath equivalents of the natural deduction
rules. To add an assumption, the antecedent is modified to include it
(typically by using \texttt{adantr};
\texttt{simpr} is useful when you want to
depend directly on the new assumption, as is shown here).

In Metamath we can represent the two given statements as these hypotheses:

\needspace{2\baselineskip}
\begin{itemize}
\item ex-natded5.5.1 $\vdash ( \varphi \rightarrow ( \psi \rightarrow \chi ) )$
\item ex-natded5.5.2 $\vdash ( \varphi \rightarrow \lnot \chi )$
\end{itemize}

\needspace{4\baselineskip}
Here is the proof in Metamath as a line-by-line translation:

\begin{longtabu}   { l l l X }
\textbf{Step} & \textbf{Hyp} & \textbf{Ref} & \textbf{Ex\-pres\-sion} \\
\endhead
1 & & simpr & $\vdash ( ( \varphi \land \psi ) \rightarrow \psi )$ \\
2 & & ex-natded5.5.1 &
  $\vdash ( \varphi \rightarrow ( \psi \rightarrow \chi ) )$ \\
3 & 2 & adantr &
 $\vdash ( ( \varphi \land \psi ) \rightarrow ( \psi \rightarrow \chi ) )$ \\
4 & 1, 3 & mpd &
 $\vdash ( ( \varphi \land \psi ) \rightarrow \chi ) $ \\
5 & & ex-natded5.5.2 &
 $\vdash ( \varphi \rightarrow \lnot \chi )$ \\
6 & 5 & adantr &
 $\vdash ( ( \varphi \land \psi ) \rightarrow \lnot \chi )$ \\
7 & 4, 6 & pm2.65da &
 $\vdash ( \varphi \rightarrow \lnot \psi )$ \\
\end{longtabu}

Only using specific natural deduction rules directly can lead to very
long proofs, for exactly the same reason that only using axioms directly
in Hilbert-style proofs can lead to very long proofs.
If the goal is short and clear proofs,
then it is better to reuse already-proven assertions
in deduction form than to start from scratch each time
and using only basic natural deduction rules.

\subsection{Strengths of Our Approach}

As far as we know there is nothing else in the literature like either the
weak deduction theorem or Mario Carneiro\index{Carneiro, Mario}'s
natural deduction method.
In order to
transform a hypothesis into an antecedent, the literature's standard
``Deduction Theorem''\index{Deduction Theorem}\index{Standard Deduction Theorem}
requires metalogic outside of the notions provided
by the axiom system. We instead generally prefer to use Mario Carneiro's
natural deduction method, then use the weak deduction theorem in cases
where that is difficult to apply, and only then use the full standard
deduction theorem as a last resort.

The weak deduction theorem\index{Weak Deduction Theorem}
does not require any additional metalogic
but converts an inference directly into a closed form theorem, with
a rigorous proof that uses only the axiom system. Unlike the standard
Deduction Theorem, there is no implicit external justification that we
have to trust in order to use it.

Mario Carneiro's natural deduction\index{natural deduction}
method also does not require any new metalogical
notions. It avoids the Deduction Theorem's metalogic by prefixing the
hypotheses and conclusion of every would-be inference with a universal
antecedent (``$\varphi \rightarrow$'') from the very start.

We think it is impressive and satisfying that we can do so much in a
practical sense without stepping outside of our Hilbert-style axiom system.
Of course our axiomatization, which is in the form of schemes,
contains a metalogic of its own that we exploit. But this metalogic
is relatively simple, and for our Deduction Theorem alternatives,
we primarily use just the direct substitution of expressions for
metavariables.

\begin{sloppy}
\section{Exploring the Set The\-o\-ry Data\-base}\label{exploring}
\end{sloppy}
% NOTE: All examples performed in this section are
% recorded wtih "set width 61" % on set.mm as of 2019-05-28
% commit c1e7849557661260f77cfdf0f97ac4354fbb4f4d.

At this point you may wish to study the \texttt{set.mm}\index{set theory
database (\texttt{set.mm})} file in more detail.  Pay particular
attention to the assumptions needed to define wffs\index{well-formed
formula (wff)} (which are not included above), the variable types
(\texttt{\$f}\index{\texttt{\$f} statement} statements), and the
definitions that are introduced.  Start with some simple theorems in
propositional calculus, making sure you understand in detail each step
of a proof.  Once you get past the first few proofs and become familiar
with the Metamath language, any part of the \texttt{set.mm} database
will be as easy to follow, step by step, as any other part---you won't
have to undergo a ``quantum leap'' in mathematical sophistication to be
able to follow a deep proof in set theory.

Next, you may want to explore how concepts such as natural numbers are
defined and described.  This is probably best done in conjunction with
standard set theory textbooks, which can help give you a higher-level
understanding.  The \texttt{set.mm} database provides references that will get
you started.  From there, you will be on your way towards a very deep,
rigorous understanding of abstract mathematics.

The Metamath\index{Metamath} program can help you peruse a Metamath data\-base,
wheth\-er you are trying to figure out how a certain step follows in a proof or
just have a general curiosity.  We will go through some examples of the
commands, using the \texttt{set.mm}\index{set theory database (\texttt{set.mm})}
database provided with the Metamath software.  These should help get you
started.  See Chapter~\ref{commands} for a more detailed description of
the commands.  Note that we have included the full spelling of all commands to
prevent ambiguity with future commands.  In practice you may type just the
characters needed to specify each command keyword\index{command keyword}
unambiguously, often just one or two characters per keyword, and you don't
need to type them in upper case.

First run the Metamath program as described earlier.  You should see the
\verb/MM>/ prompt.  Read in the \texttt{set.mm} file:\index{\texttt{read}
command}

\begin{verbatim}
MM> read set.mm
Reading source file "set.mm"... 34554442 bytes
34554442 bytes were read into the source buffer.
The source has 155711 statements; 2254 are $a and 32250 are $p.
No errors were found.  However, proofs were not checked.
Type VERIFY PROOF * if you want to check them.
\end{verbatim}

As with most examples in this book, what you will see
will be slightly different because we are continuously
improving our databases (including \texttt{set.mm}).

Let's check the database integrity.  This may take a minute or two to run if
your computer is slow.

\begin{verbatim}
MM> verify proof *
0 10%  20%  30%  40%  50%  60%  70%  80%  90% 100%
..................................................
All proofs in the database were verified in 2.84 s.
\end{verbatim}

No errors were reported, so every proof is correct.

You need to know the names (labels) of theorems before you can look at them.
Often just examining the database file(s) with a text editor is the best
approach.  In \texttt{set.mm} there are many detailed comments, especially near
the beginning, that can help guide you. The \texttt{search} command in the
Metamath program is also handy.  The \texttt{comments} qualifier will list the
statements whose associated comment (the one immediately before it) contain a
string you give it.  For example, if you are studying Enderton's {\em Elements
of Set Theory} \cite{Enderton}\index{Enderton, Herbert B.} you may want to see
the references to it in the database.  The search string \texttt{enderton} is not
case sensitive.  (This will not show you all the database theorems that are in
Enderton's book because there is usually only one citation for a given
theorem, which may appear in several textbooks.)\index{\texttt{search}
command}

\begin{verbatim}
MM> search * "enderton" / comments
12067 unineq $p "... Exercise 20 of [Enderton] p. 32 and ..."
12459 undif2 $p "...Corollary 6K of [Enderton] p. 144. (C..."
12953 df-tp $a "...s. Definition of [Enderton] p. 19. (Co..."
13689 unissb $p ".... Exercise 5 of [Enderton] p. 26 and ..."
\end{verbatim}
\begin{center}
(etc.)
\end{center}

Or you may want to see what theorems have something to do with
conjunction (logical {\sc and}).  The quotes around the search
string are optional when there's no ambiguity.\index{\texttt{search}
command}

\begin{verbatim}
MM> search * conjunction / comments
120 a1d $p "...be replaced with a conjunction ( ~ df-an )..."
662 df-bi $a "...viated form after conjunction is introdu..."
1319 wa $a "...ff definition to include conjunction ('and')."
1321 df-an $a "Define conjunction (logical 'and'). Defini..."
1420 imnan $p "...tion in terms of conjunction. (Contribu..."
\end{verbatim}
\begin{center}
(etc.)
\end{center}


Now we will start to look at some details.  Let's look at the first
axiom of propositional calculus
(we could use \texttt{sh st} to abbreviate
\texttt{show statement}).\index{\texttt{show statement} command}

\begin{verbatim}
MM> show statement ax-1/full
Statement 19 is located on line 881 of the file "set.mm".
"Axiom _Simp_.  Axiom A1 of [Margaris] p. 49.  One of the 3
axioms of propositional calculus.  The 3 axioms are also
        ...
19 ax-1 $a |- ( ph -> ( ps -> ph ) ) $.
Its mandatory hypotheses in RPN order are:
  wph $f wff ph $.
  wps $f wff ps $.
The statement and its hypotheses require the variables:  ph
      ps
The variables it contains are:  ph ps


Statement 49 is located on line 11182 of the file "set.mm".
Its statement number for HTML pages is 6.
"Axiom _Simp_.  Axiom A1 of [Margaris] p. 49.  One of the 3
axioms of propositional calculus.  The 3 axioms are also
given as Definition 2.1 of [Hamilton] p. 28.
...
49 ax-1 $a |- ( ph -> ( ps -> ph ) ) $.
Its mandatory hypotheses in RPN order are:
  wph $f wff ph $.
  wps $f wff ps $.
The statement and its hypotheses require the variables:
  ph ps
The variables it contains are:  ph ps
\end{verbatim}

Compare this to \texttt{ax-1} on p.~\pageref{ax1}.  You can see that the
symbol \texttt{ph} is the {\sc ascii} notation for $\varphi$, etc.  To
see the mathematical symbols for any expression you may typeset it in
\LaTeX\ (type \texttt{help tex} for instructions)\index{latex@{\LaTeX}}
or, easier, just use a text editor to look at the comments where symbols
are first introduced in \texttt{set.mm}.  The hypotheses \texttt{wph}
and \texttt{wps} required by \texttt{ax-1} mean that variables
\texttt{ph} and \texttt{ps} must be wffs.

Next we'll pick a simple theorem of propositional calculus, the Principle of
Identity, which is proved directly from the axioms.  We'll look at the
statement then its proof.\index{\texttt{show statement}
command}

\begin{verbatim}
MM> show statement id1/full
Statement 116 is located on line 11371 of the file "set.mm".
Its statement number for HTML pages is 22.
"Principle of identity.  Theorem *2.08 of [WhiteheadRussell]
p. 101.  This version is proved directly from the axioms for
demonstration purposes.
...
116 id1 $p |- ( ph -> ph ) $= ... $.
Its mandatory hypotheses in RPN order are:
  wph $f wff ph $.
Its optional hypotheses are:  wps wch wth wta wet
      wze wsi wrh wmu wla wka
The statement and its hypotheses require the variables:  ph
These additional variables are allowed in its proof:
      ps ch th ta et ze si rh mu la ka
The variables it contains are:  ph
\end{verbatim}

The optional variables\index{optional variable} \texttt{ps}, \texttt{ch}, etc.\ are
available for use in a proof of this statement if we wish, and were we to do
so we would make use of optional hypotheses \texttt{wps}, \texttt{wch}, etc.  (See
Section~\ref{dollaref} for the meaning of ``optional
hypothesis.''\index{optional hypothesis}) The reason these show up in the
statement display is that statement \texttt{id1} happens to be in their scope
(see Section~\ref{scoping} for the definition of ``scope''\index{scope}), but
in fact in propositional calculus we will never make use of optional
hypotheses or variables.  This becomes important after quantifiers are
introduced, where ``dummy'' variables\index{dummy variable} are often needed
in the middle of a proof.

Let's look at the proof of statement \texttt{id1}.  We'll use the
\texttt{show proof} command, which by default suppresses the
``non-essential'' steps that construct the wffs.\index{\texttt{show proof}
command}
We will display the proof in ``lemmon' format (a non-indented format
with explicit previous step number references) and renumber the
displayed steps:

\begin{verbatim}
MM> show proof id1 /lemmon/renumber
1 ax-1           $a |- ( ph -> ( ph -> ph ) )
2 ax-1           $a |- ( ph -> ( ( ph -> ph ) -> ph ) )
3 ax-2           $a |- ( ( ph -> ( ( ph -> ph ) -> ph ) ) ->
                     ( ( ph -> ( ph -> ph ) ) -> ( ph -> ph )
                                                          ) )
4 2,3 ax-mp      $a |- ( ( ph -> ( ph -> ph ) ) -> ( ph -> ph
                                                          ) )
5 1,4 ax-mp      $a |- ( ph -> ph )
\end{verbatim}

If you have read Section~\ref{trialrun}, you'll know how to interpret this
proof.  Step~2, for example, is an application of axiom \texttt{ax-1}.  This
proof is identical to the one in Hamilton's {\em Logic for Mathematicians}
\cite[p.~32]{Hamilton}\index{Hamilton, Alan G.}.

You may want to look at what
substitutions\index{substitution!variable}\index{variable substitution} are
made into \texttt{ax-1} to arrive at step~2.  The command to do this needs to
know the ``real'' step number, so we'll display the proof again without
the \texttt{renumber} qualifier.\index{\texttt{show proof}
command}

\begin{verbatim}
MM> show proof id1 /lemmon
 9 ax-1          $a |- ( ph -> ( ph -> ph ) )
20 ax-1          $a |- ( ph -> ( ( ph -> ph ) -> ph ) )
24 ax-2          $a |- ( ( ph -> ( ( ph -> ph ) -> ph ) ) ->
                     ( ( ph -> ( ph -> ph ) ) -> ( ph -> ph )
                                                          ) )
25 20,24 ax-mp   $a |- ( ( ph -> ( ph -> ph ) ) -> ( ph -> ph
                                                          ) )
26 9,25 ax-mp    $a |- ( ph -> ph )
\end{verbatim}

The ``real'' step number is 20.  Let's look at its details.

\begin{verbatim}
MM> show proof id1 /detailed_step 20
Proof step 20:  min=ax-1 $a |- ( ph -> ( ( ph -> ph ) -> ph )
  )
This step assigns source "ax-1" ($a) to target "min" ($e).
The source assertion requires the hypotheses "wph" ($f, step
18) and "wps" ($f, step 19).  The parent assertion of the
target hypothesis is "ax-mp" ($a, step 25).
The source assertion before substitution was:
    ax-1 $a |- ( ph -> ( ps -> ph ) )
The following substitutions were made to the source
assertion:
    Variable  Substituted with
     ph        ph
     ps        ( ph -> ph )
The target hypothesis before substitution was:
    min $e |- ph
The following substitution was made to the target hypothesis:
    Variable  Substituted with
     ph        ( ph -> ( ( ph -> ph ) -> ph ) )
\end{verbatim}

This shows the substitutions\index{substitution!variable}\index{variable
substitution} made to the variables in \texttt{ax-1}.  References are made to
steps 18 and 19 which are not shown in our proof display.  To see these steps,
you can display the proof with the \texttt{all} qualifier.

Let's look at a slightly more advanced proof of propositional calculus.  Note
that \verb+/\+ is the symbol for $\wedge$ (logical {\sc and}, also
called conjunction).\index{conjunction ($\wedge$)}
\index{logical {\sc and} ($\wedge$)}

\begin{verbatim}
MM> show statement prth/full
Statement 1791 is located on line 15503 of the file "set.mm".
Its statement number for HTML pages is 559.
"Conjoin antecedents and consequents of two premises.  This
is the closed theorem form of ~ anim12d .  Theorem *3.47 of
[WhiteheadRussell] p. 113.  It was proved by Leibniz,
and it evidently pleased him enough to call it
_praeclarum theorema_ (splendid theorem).
...
1791 prth $p |- ( ( ( ph -> ps ) /\ ( ch -> th ) ) -> ( ( ph
      /\ ch ) -> ( ps /\ th ) ) ) $= ... $.
Its mandatory hypotheses in RPN order are:
  wph $f wff ph $.
  wps $f wff ps $.
  wch $f wff ch $.
  wth $f wff th $.
Its optional hypotheses are:  wta wet wze wsi wrh wmu wla wka
The statement and its hypotheses require the variables:  ph
      ps ch th
These additional variables are allowed in its proof:  ta et
      ze si rh mu la ka
The variables it contains are:  ph ps ch th


MM> show proof prth /lemmon/renumber
1 simpl          $p |- ( ( ( ph -> ps ) /\ ( ch -> th ) ) ->
                                               ( ph -> ps ) )
2 simpr          $p |- ( ( ( ph -> ps ) /\ ( ch -> th ) ) ->
                                               ( ch -> th ) )
3 1,2 anim12d    $p |- ( ( ( ph -> ps ) /\ ( ch -> th ) ) ->
                           ( ( ph /\ ch ) -> ( ps /\ th ) ) )
\end{verbatim}

There are references to a lot of unfamiliar statements.  To see what they are,
you may type the following:

\begin{verbatim}
MM> show proof prth /statement_summary
Summary of statements used in the proof of "prth":

Statement simpl is located on line 14748 of the file
"set.mm".
"Elimination of a conjunct.  Theorem *3.26 (Simp) of
[WhiteheadRussell] p. 112. ..."
  simpl $p |- ( ( ph /\ ps ) -> ph ) $= ... $.

Statement simpr is located on line 14777 of the file
"set.mm".
"Elimination of a conjunct.  Theorem *3.27 (Simp) of
[WhiteheadRussell] ..."
  simpr $p |- ( ( ph /\ ps ) -> ps ) $= ... $.

Statement anim12d is located on line 15445 of the file
"set.mm".
"Conjoin antecedents and consequents in a deduction.
..."
  anim12d.1 $e |- ( ph -> ( ps -> ch ) ) $.
  anim12d.2 $e |- ( ph -> ( th -> ta ) ) $.
  anim12d $p |- ( ph -> ( ( ps /\ th ) -> ( ch /\ ta ) ) )
      $= ... $.
\end{verbatim}
\begin{center}
(etc.)
\end{center}

Of course you can look at each of these statements and their proofs, and
so on, back to the axioms of propositional calculus if you wish.

The \texttt{search} command is useful for finding statements when you
know all or part of their contents.  The following example finds all
statements containing \verb@ph -> ps@ followed by \verb@ch -> th@.  The
\verb@$*@ is a wildcard that matches anything; the \texttt{\$} before the
\verb$*$ prevents conflicts with math symbol token names.  The \verb@*@ after
\texttt{SEARCH} is also a wildcard that in this case means ``match any label.''
\index{\texttt{search} command}

% I'm omitting this one, since readers are unlikely to see it:
% 1096 bisymOLD $p |- ( ( ( ph -> ps ) -> ( ch -> th ) ) -> ( (
%   ( ps -> ph ) -> ( th -> ch ) ) -> ( ( ph <-> ps ) -> ( ch
%    <-> th ) ) ) )
\begin{verbatim}
MM> search * "ph -> ps $* ch -> th"
1791 prth $p |- ( ( ( ph -> ps ) /\ ( ch -> th ) ) -> ( ( ph
    /\ ch ) -> ( ps /\ th ) ) )
2455 pm3.48 $p |- ( ( ( ph -> ps ) /\ ( ch -> th ) ) -> ( (
    ph \/ ch ) -> ( ps \/ th ) ) )
117859 pm11.71 $p |- ( ( E. x ph /\ E. y ch ) -> ( ( A. x (
    ph -> ps ) /\ A. y ( ch -> th ) ) <-> A. x A. y ( ( ph /\
    ch ) -> ( ps /\ th ) ) ) )
\end{verbatim}

Three statements, \texttt{prth}, \texttt{pm3.48},
 and \texttt{pm11.71}, were found to match.

To see what axioms\index{axiom} and definitions\index{definition}
\texttt{prth} ultimately depends on for its proof, you can have the
program backtrack through the hierarchy\index{hierarchy} of theorems and
definitions.\index{\texttt{show trace{\char`\_}back} command}

\begin{verbatim}
MM> show trace_back prth /essential/axioms
Statement "prth" assumes the following axioms ($a
statements):
  ax-1 ax-2 ax-3 ax-mp df-bi df-an
\end{verbatim}

Note that the 3 axioms of propositional calculus and the modus ponens rule are
needed (as expected); in addition, there are a couple of definitions that are used
along the way.  Note that Metamath makes no distinction\index{axiom vs.\
definition} between axioms\index{axiom} and definitions\index{definition}.  In
\texttt{set.mm} they have been distinguished artificially by prefixing their
labels\index{labels in \texttt{set.mm}} with \texttt{ax-} and \texttt{df-}
respectively.  For example, \texttt{df-an} defines conjunction (logical {\sc
and}), which is represented by the symbol \verb+/\+.
Section~\ref{definitions} discusses the philosophy of definitions, and the
Metamath language takes a particularly simple, conservative approach by using
the \texttt{\$a}\index{\texttt{\$a} statement} statement for both axioms and
definitions.

You can also have the program compute how many steps a proof
has\index{proof length} if we were to follow it all the way back to
\texttt{\$a} statements.

\begin{verbatim}
MM> show trace_back prth /essential/count_steps
The statement's actual proof has 3 steps.  Backtracking, a
total of 79 different subtheorems are used.  The statement
and subtheorems have a total of 274 actual steps.  If
subtheorems used only once were eliminated, there would be a
total of 38 subtheorems, and the statement and subtheorems
would have a total of 185 steps.  The proof would have 28349
steps if fully expanded back to axiom references.  The
maximum path length is 38.  A longest path is:  prth <-
anim12d <- syl2and <- sylan2d <- ancomsd <- ancom <- pm3.22
<- pm3.21 <- pm3.2 <- ex <- sylbir <- biimpri <- bicomi <-
bicom1 <- bi2 <- dfbi1 <- impbii <- bi3 <- simprim <- impi <-
con1i <- nsyl2 <- mt3d <- con1d <- notnot1 <- con2i <- nsyl3
<- mt2d <- con2d <- notnot2 <- pm2.18d <- pm2.18 <- pm2.21 <-
pm2.21d <- a1d <- syl <- mpd <- a2i <- a2i.1 .
\end{verbatim}

This tells us that we would have to inspect 274 steps if we want to
verify the proof completely starting from the axioms.  A few more
statistics are also shown.  There are one or more paths back to axioms
that are the longest; this command ferrets out one of them and shows it
to you.  There may be a sense in which the longest path length is
related to how ``deep'' the theorem is.

We might also be curious about what proofs depend on the theorem
\texttt{prth}.  If it is never used later on, we could eliminate it as
redundant if it has no intrinsic interest by itself.\index{\texttt{show
usage} command}

% I decided to show the OLD values here.
\begin{verbatim}
MM> show usage prth
Statement "prth" is directly referenced in the proofs of 18
statements:
  mo3 moOLD 2mo 2moOLD euind reuind reuss2 reusv3i opelopabt
  wemaplem2 rexanre rlimcn2 o1of2 o1rlimmul 2sqlem6 spanuni
  heicant pm11.71
\end{verbatim}

Thus \texttt{prth} is directly used by 18 proofs.
We can use the \texttt{/recursive} qualifier to include indirect use:

\begin{verbatim}
MM> show usage prth /recursive
Statement "prth" directly or indirectly affects the proofs of
24214 statements:
  mo3 mo mo3OLD eu2 moOLD eu2OLD eu3OLD mo4f mo4 eu4 mopick
...
\end{verbatim}

\subsection{A Note on the ``Compact'' Proof Format}

The present version of Metamath\index{Metamath!limitations of
version 0.07.30} will display proofs in a ``compact''\index{compact proof}
format whenever the proof is stored in compressed format in the database.  It
may be be slightly confusing unless you know how to interpret it.  (A
future version may eliminate this format from displays.)  For example,
if you display the complete proof of theorem \texttt{id1} it will start
off as follows:

\begin{verbatim}
MM> show proof id1 /lemmon/all
 1 wph           $f wff ph
 2 wph           $f wff ph
 3 wph           $f wff ph
 4 2,3 wi    @4: $a wff ( ph -> ph )
 5 1,4 wi    @5: $a wff ( ph -> ( ph -> ph ) )
 6 @4            $a wff ( ph -> ph )
\end{verbatim}

\begin{center}
{etc.}
\end{center}

Step 4 has a ``local label,''\index{local label} \texttt{@4}, assigned to it.
Later on, at step 6, the label \texttt{@1} is referenced instead of
displaying the explicit proof for that step.  This technique takes advantage
of the fact that steps in a proof often repeat, especially during the
construction of wffs.  The compact format reduces the number of steps in the
proof display and may actually be preferred by some people, which is one
reason that it is still displayed by the program.

If you want to see the normal format with the ``true'' step numbers, you can
use the following workaround:\index{\texttt{save proof} command}

\begin{verbatim}
MM> save proof id1 /normal
The proof of "id1" has been reformatted and saved internally.
Remember to use WRITE SOURCE to save it permanently.
MM> show proof id1 /lemmon/all
 1 wph           $f wff ph
 2 wph           $f wff ph
 3 wph           $f wff ph
 4 2,3 wi        $a wff ( ph -> ph )
 5 1,4 wi        $a wff ( ph -> ( ph -> ph ) )
 6 wph           $f wff ph
 7 wph           $f wff ph
 8 6,7 wi        $a wff ( ph -> ph )
\end{verbatim}

\begin{center}
{etc.}
\end{center}

Note that the original 6 steps are now 8 steps.  However, the format is
now the same as that described in Chapter~\ref{using}.

\chapter{The Metamath Language}
\label{languagespec}

\begin{quote}
  {\em Thus mathematics may be defined as the subject in which we never know
what we are talking about, nor whether what we are saying is true.}
    \flushright\sc  Bertrand Russell\footnote{\cite[p.~84]{Russell2}.}\\
\end{quote}\index{Russell, Bertrand}

Probably the most striking feature of the Metamath language is its almost
complete absence of hard-wired syntax. Metamath\index{Metamath} does not
understand any mathematics or logic other than that needed to construct finite
sequences of symbols according to a small set of simple, built-in rules.  The
only rule it uses in a proof is the substitution of an expression (symbol
sequence) for a variable, subject to a simple constraint to prevent
bound-variable clashes.  The primitive notions built into Metamath involve the
simple manipulation of finite objects (symbols) that we as humans can easily
visualize and that computers can easily deal with.  They seem to be just
about the simplest notions possible that are required to do standard
mathematics.

This chapter serves as a reference manual for the Metamath\index{Metamath}
language. It covers the tedious technical details of the language, some of
which you may wish to skim in a first reading.  On the other hand, you should
pay close attention to the defined terms in {\bf boldface}; they have precise
meanings that are important to keep in mind for later understanding.  It may
be best to first become familiar with the examples in Chapter~\ref{using} to
gain some motivation for the language.

%% Uncomment this when uncommenting section {formalspec} below
If you have some knowledge of set theory, you may wish to study this
chapter in conjunction with the formal set-theoretical description of the
Metamath language in Appendix~\ref{formalspec}.

We will use the name ``Metamath''\index{Metamath} to mean either the Metamath
computer language or the Metamath software associated with the computer
language.  We will not distinguish these two when the context is clear.

The next section contains the complete specification of the Metamath
language.
It serves as an
authoritative reference and presents the syntax in enough detail to
write a parser\index{parsing Metamath} and proof verifier.  The
specification is terse and it is probably hard to learn the language
directly from it, but we include it here for those impatient people who
prefer to see everything up front before looking at verbose expository
material.  Later sections explain this material and provide examples.
We will repeat the definitions in those sections, and you may skip the
next section at first reading and proceed to Section~\ref{tut1}
(p.~\pageref{tut1}).

\section{Specification of the Metamath Language}\label{spec}
\index{Metamath!specification}

\begin{quote}
  {\em Sometimes one has to say difficult things, but one ought to say
them as simply as one knows how.}
    \flushright\sc  G. H. Hardy\footnote{As quoted in
    \cite{deMillo}, p.~273.}\\
\end{quote}\index{Hardy, G. H.}

\subsection{Preliminaries}\label{spec1}

% Space is technically a printable character, so we'll word things
% carefully so it's unambiguous.
A Metamath {\bf database}\index{database} is built up from a top-level source
file together with any source files that are brought in through file inclusion
commands (see below).  The only characters that are allowed to appear in a
Metamath source file are the 94 non-whitespace printable {\sc
ascii}\index{ascii@{\sc ascii}} characters, which are digits, upper and lower
case letters, and the following 32 special
characters\index{special characters}:\label{spec1chars}

\begin{verbatim}
! " # $ % & ' ( ) * + , - . / :
; < = > ? @ [ \ ] ^ _ ` { | } ~
\end{verbatim}
plus the following characters which are the ``white space'' characters:
space (a printable character),
tab, carriage return, line feed, and form feed.\label{whitespace}
We will use \texttt{typewriter}
font to display the printable characters.

A Metamath database consists of a sequence of three kinds of {\bf
tokens}\index{token} separated by {\bf white space}\index{white space}
(which is any sequence of one or more white space characters).  The set
of {\bf keyword}\index{keyword} tokens is \texttt{\$\char`\{},
\texttt{\$\char`\}}, \texttt{\$c}, \texttt{\$v}, \texttt{\$f},
\texttt{\$e}, \texttt{\$d}, \texttt{\$a}, \texttt{\$p}, \texttt{\$.},
\texttt{\$=}, \texttt{\$(}, \texttt{\$)}, \texttt{\$[}, and
\texttt{\$]}.  The last four are called {\bf auxiliary}\index{auxiliary
keyword} or preprocessing keywords.  A {\bf label}\index{label} token
consists of any combination of letters, digits, and the characters
hyphen, underscore, and period.  A {\bf math symbol}\index{math symbol}
token may consist of any combination of the 92 printable standard {\sc
ascii} characters other than space or \texttt{\$}~. All tokens are
case-sensitive.

\subsection{Preprocessing}

The token \texttt{\$(} begins a {\bf comment} and
\texttt{\$)} ends a comment.\index{\texttt{\$(}
and \texttt{\$)} auxiliary keywords}\index{comment}
Comments may contain any of
the 94 non-whitespace printable characters and white space,
except they may not contain the
2-character sequences \texttt{\$(} or \texttt{\$)} (comments do not nest).
Comments are ignored (treated
like white space) for the purpose of parsing, e.g.,
\texttt{\$( \$[ \$)} is a comment.
See p.~\pageref{mathcomments} for comment typesetting conventions; these
conventions may be ignored for the purpose of parsing.

A {\bf file inclusion command} consists of \texttt{\$[} followed by a file name
followed by \texttt{\$]}.\index{\texttt{\$[} and \texttt{\$]} auxiliary
keywords}\index{included file}\index{file inclusion}
It is only allowed in the outermost scope (i.e., not between
\texttt{\$\char`\{} and \texttt{\$\char`\}})
and must not be inside a statement (e.g., it may not occur
between the label of a \texttt{\$a} statement and its \texttt{\$.}).
The file name may not
contain a \texttt{\$} or white space.  The file must exist.
The case-sensitivity
of its name follows the conventions of the operating system.  The contents of
the file replace the inclusion command.
Included files may include other files.
Only the first reference to a given file is included; any later
references to the same file (whether in the top-level file or in included
files) cause the inclusion command to be ignored (treated like white space).
A verifier may assume that file names with different strings
refer to different files for the purpose of ignoring later references.
A file self-reference is ignored, as is any reference to the top-level file
(to avoid loops).
Included files may not include a \texttt{\$(} without a matching \texttt{\$)},
may not include a \texttt{\$[} without a matching \texttt{\$]}, and may
not include incomplete statements (e.g., a \texttt{\$a} without a matching
\texttt{\$.}).
It is currently unspecified if path references are relative to the process'
current directory or the file's containing directory, so databases should
avoid using pathname separators (e.g., ``/'') in file names.

Like all tokens, the \texttt{\$(}, \texttt{\$)}, \texttt{\$[}, and \texttt{\$]} keywords
must be surrounded by white space.

\subsection{Basic Syntax}

After preprocessing, a database will consist of a sequence of {\bf
statements}.
These are the scoping statements \texttt{\$\char`\{} and
\texttt{\$\char`\}}, along with the \texttt{\$c}, \texttt{\$v},
\texttt{\$f}, \texttt{\$e}, \texttt{\$d}, \texttt{\$a}, and \texttt{\$p}
statements.

A {\bf scoping statement}\index{scoping statement} consists only of its
keyword, \texttt{\$\char`\{} or \texttt{\$\char`\}}.
A \texttt{\$\char`\{} begins a {\bf
block}\index{block} and a matching \texttt{\$\char`\}} ends the block.
Every \texttt{\$\char`\{}
must have a matching \texttt{\$\char`\}}.
Defining it recursively, we say a block
contains a sequence of zero or more tokens other
than \texttt{\$\char`\{} and \texttt{\$\char`\}} and
possibly other blocks.  There is an {\bf outermost
block}\index{block!outermost} not bracketed by \texttt{\$\char`\{} \texttt{\$\char`\}}; the end
of the outermost block is the end of the database.

% LaTeX bug? can't do \bf\tt

A {\bf \$v} or {\bf \$c statement}\index{\texttt{\$v} statement}\index{\texttt{\$c}
statement} consists of the keyword token \texttt{\$v} or \texttt{\$c} respectively,
followed by one or more math symbols,
% The word "token" is used to distinguish "$." from the sentence-ending period.
followed by the \texttt{\$.}\ token.
These
statements {\bf declare}\index{declaration} the math symbols to be {\bf
variables}\index{variable!Metamath} or {\bf constants}\index{constant}
respectively. The same math symbol may not occur twice in a given \texttt{\$v} or
\texttt{\$c} statement.

%c%A math symbol becomes an {\bf active}\index{active math symbol}
%c%when declared and stays active until the end of the block in which it is
%c%declared.  A math symbol may not be declared a second time while it is active,
%c%but it may be declared again after it becomes inactive.

A math symbol becomes {\bf active}\index{active math symbol} when declared
and stays active until the end of the block in which it is declared.  A
variable may not be declared a second time while it is active, but it
may be declared again (as a variable, but not as a constant) after it
becomes inactive.  A constant must be declared in the outermost block and may
not be declared a second time.\footnote{The rules for redeclaration may
become more general in the future;
see p.~\pageref{redeclarationf}.}\index{redeclaration of symbols}

A {\bf \$f statement}\index{\texttt{\$f} statement} consists of a label,
followed by \texttt{\$f}, followed by its typecode (an active constant),
followed by an
active variable, followed by the \texttt{\$.}\ token.  A {\bf \$e
statement}\index{\texttt{\$e} statement} consists of a label, followed
by \texttt{\$e}, followed by its typecode (an active constant),
followed by zero or more
active math symbols, followed by the \texttt{\$.}\ token.  A {\bf
hypothesis}\index{hypothesis} is a \texttt{\$f} or \texttt{\$e}
statement.
The type declared by a \texttt{\$f} statement for a given label
is global even if the variable is not
(e.g., a database may not have \texttt{wff P} in one local scope
nd \texttt{class P} in another).

A {\bf simple \$d statement}\index{\texttt{\$d} statement!simple}
consists of \texttt{\$d}, followed by two different active variables,
followed by the \texttt{\$.}\ token.  A {\bf compound \$d
statement}\index{\texttt{\$d} statement!compound} consists of
\texttt{\$d}, followed by three or more variables (all different),
followed by the \texttt{\$.}\ token.  The order of the variables in a
\texttt{\$d} statement is unimportant.  A compound \texttt{\$d}
statement is equivalent to a set of simple \texttt{\$d} statements, one
for each possible pair of variables occurring in the compound
\texttt{\$d} statement.  Henceforth in this specification we shall
assume all \texttt{\$d} statements are simple.  A \texttt{\$d} statement
is also called a {\bf disjoint} (or {\bf distinct}) {\bf variable
restriction}.\index{disjoint-variable restriction}

A {\bf \$a statement}\index{\texttt{\$a} statement} consists of a label,
followed by \texttt{\$a}, followed by its typecode (an active constant),
followed by
zero or more active math symbols, followed by the \texttt{\$.}\ token.  A {\bf
\$p statement}\index{\texttt{\$p} statement} consists of a label,
followed by \texttt{\$p}, followed by its typecode (an active constant),
followed by
zero or more active math symbols, followed by \texttt{\$=}, followed by
a sequence of labels, followed by the \texttt{\$.}\ token.  An {\bf
assertion}\index{assertion} is a \texttt{\$a} or \texttt{\$p} statement.

A \texttt{\$f}, \texttt{\$e}, or \texttt{\$d} statement is {\bf active}\index{active
statement} from the place it occurs until the end of the block it occurs in.
A \texttt{\$a} or \texttt{\$p} statement is {\bf active} from the place it occurs
through the end of the database.
There may not be two active \texttt{\$f} statements containing the same
variable.  Each variable in a \texttt{\$e}, \texttt{\$a}, or
\texttt{\$p} statement must exist in an active \texttt{\$f}
statement.\footnote{This requirement can greatly simplify the
unification algorithm (substitution calculation) required by proof
verification.}

%The label that begins each \texttt{\$f}, \texttt{\$e}, \texttt{\$a}, and
%\texttt{\$p} statement must be unique.
Each label token must be unique, and
no label token may match any math symbol
token.\label{namespace}\footnote{This
restriction was added on June 24, 2006.
It is not theoretically necessary but is imposed to make it easier to
write certain parsers.}

The set of {\bf mandatory variables}\index{mandatory variable} associated with
an assertion is the set of (zero or more) variables in the assertion and in any
active \texttt{\$e} statements.  The (possibly empty) set of {\bf mandatory
hypotheses}\index{mandatory hypothesis} is the set of all active \texttt{\$f}
statements containing mandatory variables, together with all active \texttt{\$e}
statements.
The set of {\bf mandatory {\bf \$d} statements}\index{mandatory
disjoint-variable restriction} associated with an assertion are those active
\texttt{\$d} statements whose variables are both among the assertion's
mandatory variables.

\subsection{Proof Verification}\label{spec4}

The sequence of labels between the \texttt{\$=} and \texttt{\$.}\ tokens
in a \texttt{\$p} statement is a {\bf proof}.\index{proof!Metamath} Each
label in a proof must be the label of an active statement other than the
\texttt{\$p} statement itself; thus a label must refer either to an
active hypothesis of the \texttt{\$p} statement or to an earlier
assertion.

An {\bf expression}\index{expression} is a sequence of math symbols. A {\bf
substitution map}\index{substitution map} associates a set of variables with a
set of expressions.  It is acceptable for a variable to be mapped to an
expression containing it.  A {\bf
substitution}\index{substitution!variable}\index{variable substitution} is the
simultaneous replacement of all variables in one or more expressions with the
expressions that the variables map to.

A proof is scanned in order of its label sequence.  If the label refers to an
active hypothesis, the expression in the hypothesis is pushed onto a
stack.\index{stack}\index{RPN stack}  If the label refers to an assertion, a
(unique) substitution must exist that, when made to the mandatory hypotheses
of the referenced assertion, causes them to match the topmost (i.e.\ most
recent) entries of the stack, in order of occurrence of the mandatory
hypotheses, with the topmost stack entry matching the last mandatory
hypothesis of the referenced assertion.  As many stack entries as there are
mandatory hypotheses are then popped from the stack.  The same substitution is
made to the referenced assertion, and the result is pushed onto the stack.
After the last label in the proof is processed, the stack must have a single
entry that matches the expression in the \texttt{\$p} statement containing the
proof.

%c%{\footnotesize\begin{quotation}\index{redeclaration of symbols}
%c%{{\em Comment.}\label{spec4comment} Whenever a math symbol token occurs in a
%c%{\texttt{\$c} or \texttt{\$v} statement, it is considered to designate a distinct new
%c%{symbol, even if the same token was previously declared (and is now inactive).
%c%{Thus a math token declared as a constant in two different blocks is considered
%c%{to designate two distinct constants (even though they have the same name).
%c%{The two constants will not match in a proof that references both blocks.
%c%{However, a proof referencing both blocks is acceptable as long as it doesn't
%c%{require that the constants match.  Similarly, a token declared to be a
%c%{constant for a referenced assertion will not match the same token declared to
%c%{be a variable for the \texttt{\$p} statement containing the proof.  In the case
%c%{of a token declared to be a variable for a referenced assertion, this is not
%c%{an issue since the variable can be substituted with whatever expression is
%c%{needed to achieve the required match.
%c%{\end{quotation}}
%c2%A proof may reference an assertion that contains or whose hypotheses contain a
%c2%constant that is not active for the \texttt{\$p} statement containing the proof.
%c2%However, the final result of the proof may not contain that constant. A proof
%c2%may also reference an assertion that contains or whose hypotheses contain a
%c2%variable that is not active for the \texttt{\$p} statement containing the proof.
%c2%That variable, of course, will be substituted with whatever expression is
%c2%needed to achieve the required match.

A proof may contain a \texttt{?}\ in place of a label to indicate an unknown step
(Section~\ref{unknown}).  A proof verifier may ignore any proof containing
\texttt{?}\ but should warn the user that the proof is incomplete.

A {\bf compressed proof}\index{compressed proof}\index{proof!compressed} is an
alternate proof notation described in Appen\-dix~\ref{compressed}; also see
references to ``compressed proof'' in the Index.  Compressed proofs are a
Metamath language extension which a complete proof verifier should be able to
parse and verify.

\subsubsection{Verifying Disjoint Variable Restrictions}

Each substitution made in a proof must be checked to verify that any
disjoint variable restrictions are satisfied, as follows.

If two variables replaced by a substitution exist in a mandatory \texttt{\$d}
statement\index{\texttt{\$d} statement} of the assertion referenced, the two
expressions resulting from the substitution must satisfy the following
conditions.  First, the two expressions must have no variables in common.
Second, each possible pair of variables, one from each expression, must exist
in an active \texttt{\$d} statement of the \texttt{\$p} statement containing the
proof.

\vskip 1ex

This ends the specification of the Metamath language;
see Appendix \ref{BNF} for its syntax in
Extended Backus--Naur Form (EBNF)\index{Extended Backus--Naur Form}\index{EBNF}.

\section{The Basic Keywords}\label{tut1}

Our expository material begins here.

Like most computer languages, Metamath\index{Metamath} takes its input from
one or more {\bf source files}\index{source file} which contain characters
expressed in the standard {\sc ascii} (American Standard Code for Information
Interchange)\index{ascii@{\sc ascii}} code for computers.  A source file
consists of a series of {\bf tokens}\index{token}, which are strings of
non-whitespace
printable characters (from the set of 94 shown on p.~\pageref{spec1chars})
separated by {\bf white space}\index{white space} (spaces, tabs, carriage
returns, line feeds, and form feeds). Any string consisting only of these
characters is treated the same as a single space.  The non-whitespace printable
characters\index{printable character} that Metamath recognizes are the 94
characters on standard {\sc ascii} keyboards.

Metamath has the ability to join several files together to form its
input (Section~\ref{include}).  We call the aggregate contents of all
the files after they have been joined together a {\bf
database}\index{database} to distinguish it from an individual source
file.  The tokens in a database consist of {\bf
keywords}\index{keyword}, which are built into the language, together
with two kinds of user-defined tokens called {\bf labels}\index{label}
and {\bf math symbols}\index{math symbol}.  (Often we will simply say
{\bf symbol}\index{symbol} instead of math symbol for brevity).  The set
of {\bf basic keywords}\index{basic keyword} is
\texttt{\$c}\index{\texttt{\$c} statement},
\texttt{\$v}\index{\texttt{\$v} statement},
\texttt{\$e}\index{\texttt{\$e} statement},
\texttt{\$f}\index{\texttt{\$f} statement},
\texttt{\$d}\index{\texttt{\$d} statement},
\texttt{\$a}\index{\texttt{\$a} statement},
\texttt{\$p}\index{\texttt{\$p} statement},
\texttt{\$=}\index{\texttt{\$=} keyword},
\texttt{\$.}\index{\texttt{\$.}\ keyword},
\texttt{\$\char`\{}\index{\texttt{\$\char`\{} and \texttt{\$\char`\}}
keywords}, and \texttt{\$\char`\}}.  This is the complete set of
syntactical elements of what we call the {\bf basic
language}\index{basic language} of Metamath, and with them you can
express all of the mathematics that were intended by the design of
Metamath.  You should make it a point to become very familiar with them.
Table~\ref{basickeywords} lists the basic keywords along with a brief
description of their functions.  For now, this description will give you
only a vague notion of what the keywords are for; later we will describe
the keywords in detail.


\begin{table}[htp] \caption{Summary of the basic Metamath
keywords} \label{basickeywords}
\begin{center}
\begin{tabular}{|p{4pc}|l|}
\hline
\em \centering Keyword&\em Description\\
\hline
\hline
\centering
   \texttt{\$c}&Constant symbol declaration\\
\hline
\centering
   \texttt{\$v}&Variable symbol declaration\\
\hline
\centering
   \texttt{\$d}&Disjoint variable restriction\\
\hline
\centering
   \texttt{\$f}&Variable-type (``floating'') hypothesis\\
\hline
\centering
   \texttt{\$e}&Logical (``essential'') hypothesis\\
\hline
\centering
   \texttt{\$a}&Axiomatic assertion\\
\hline
\centering
   \texttt{\$p}&Provable assertion\\
\hline
\centering
   \texttt{\$=}&Start of proof in \texttt{\$p} statement\\
\hline
\centering
   \texttt{\$.}&End of the above statement types\\
\hline
\centering
   \texttt{\$\char`\{}&Start of block\\
\hline
\centering
   \texttt{\$\char`\}}&End of block\\
\hline
\end{tabular}
\end{center}
\end{table}

%For LaTeX bug(?) where it puts tables on blank page instead of btwn text
%May have to adjust if text changes
%\newpage

There are some additional keywords, called {\bf auxiliary
keywords}\index{auxiliary keyword} that help make Metamath\index{Metamath}
more practical. These are part of the {\bf extended language}\index{extended
language}. They provide you with a means to put comments into a Metamath
source file\index{source file} and reference other source files.  We will
introduce these in later sections. Table~\ref{otherkeywords} summarizes them
so that you can recognize them now if you want to peruse some source
files while learning the basic keywords.


\begin{table}[htp] \caption{Auxiliary Metamath
keywords} \label{otherkeywords}
\begin{center}
\begin{tabular}{|p{4pc}|l|}
\hline
\em \centering Keyword&\em Description\\
\hline
\hline
\centering
   \texttt{\$(}&Start of comment\\
\hline
\centering
   \texttt{\$)}&End of comment\\
\hline
\centering
   \texttt{\$[}&Start of included source file name\\
\hline
\centering
   \texttt{\$]}&End of included source file name\\
\hline
\end{tabular}
\end{center}
\end{table}
\index{\texttt{\$(} and \texttt{\$)} auxiliary keywords}
\index{\texttt{\$[} and \texttt{\$]} auxiliary keywords}


Unlike those in some computer languages, the keywords\index{keyword} are short
two-character sequences rather than English-like words.  While this may make
them slightly more difficult to remember at first, their brevity allows
them to blend in with the mathematics being described, not
distract from it, like punctuation marks.


\subsection{User-Defined Tokens}\label{dollardollar}\index{token}

As you may have noticed, all keywords\index{keyword} begin with the \texttt{\$}
character.  This mundane monetary symbol is not ordinarily used in higher
mathematics (outside of grant proposals), so we have appropriated it to
distinguish the Metamath\index{Metamath} keywords from ordinary mathematical
symbols. The \texttt{\$} character is thus considered special and may not be
used as a character in a user-defined token.  All tokens and keywords are
case-sensitive; for example, \texttt{n} is considered to be a different character
from \texttt{N}.  Case-sensitivity makes the available {\sc ascii} character set
as rich as possible.

\subsubsection{Math Symbol Tokens}\index{token}

Math symbols\index{math symbol} are tokens used to represent the symbols
that appear in ordinary mathematical formulas.  They may consist of any
combination of the 93 non-whitespace printable {\sc ascii} characters other than
\texttt{\$}~. Some examples are \texttt{x}, \texttt{+}, \texttt{(},
\texttt{|-}, \verb$!%@?&$, and \texttt{bounded}.  For readability, it is
best to try to make these look as similar to actual mathematical symbols
as possible, within the constraints of the {\sc ascii} character set, in
order to make the resulting mathematical expressions more readable.

In the Metamath\index{Metamath} language, you express ordinary
mathematical formulas and statements as sequences of math symbols such
as \texttt{2 + 2 = 4} (five symbols, all constants).\footnote{To
eliminate ambiguity with other expressions, this is expressed in the set
theory database \texttt{set.mm} as \texttt{|- ( 2 + 2
 ) = 4 }, whose \LaTeX\ equivalent is $\vdash
(2+2)=4$.  The \,$\vdash$ means ``is a theorem'' and the
parentheses allow explicit associative grouping.}\index{turnstile
({$\,\vdash$})} They may even be English
sentences, as in \texttt{E is closed and bounded} (five symbols)---here
\texttt{E} would be a variable and the other four symbols constants.  In
principle, a Metamath database could be constructed to work with almost
any unambiguous English-language mathematical statement, but as a
practical matter the definitions needed to provide for all possible
syntax variations would be cumbersome and distracting and possibly have
subtle pitfalls accidentally built in.  We generally recommend that you
express mathematical statements with compact standard mathematical
symbols whenever possible and put their English-language descriptions in
comments.  Axioms\index{axiom} and definitions\index{definition}
(\texttt{\$a}\index{\texttt{\$a} statement} statements) are the only
places where Metamath will not detect an error, and doing this will help
reduce the number of definitions needed.

You are free to use any tokens\index{token} you like for math
symbols\index{math symbol}.  Appendix~\ref{ASCII} recommends token names to
use for symbols in set theory, and we suggest you adopt these in order to be
able to include the \texttt{set.mm} set theory database in your database.  For
printouts, you can convert the tokens in a database
to standard mathematical symbols with the \LaTeX\ typesetting program.  The
Metamath command \texttt{open tex} {\em filename}\index{\texttt{open tex} command}
produces output that can be read by \LaTeX.\index{latex@{\LaTeX}}
The correspondence
between tokens and the actual symbols is made by \texttt{latexdef}
statements inside a special database comment tagged
with \texttt{\$t}.\index{\texttt{\$t} comment}\index{typesetting comment}
  You can edit
this comment to change the definitions or add new ones.
Appendix~\ref{ASCII} describes how to do this in more detail.

% White space\index{white space} is normally used to separate math
% symbol\index{math symbol} tokens, but they may be juxtaposed without white
% space in \texttt{\$d}\index{\texttt{\$d} statement}, \texttt{\$e}\index{\texttt{\$e}
% statement}, \texttt{\$f}\index{\texttt{\$f} statement}, \texttt{\$a}\index{\texttt{\$a}
% statement}, and \texttt{\$p}\index{\texttt{\$p} statement} statements when no
% ambiguity will result.  Specifically, Metamath parses the math symbol sequence
% in one of these statements in the following manner:  when the math symbol
% sequence has been broken up into tokens\index{token} up to a given character,
% the next token is the longest string of characters that could constitute a
% math symbol that is active\index{active
% math symbol} at that point.  (See Section~\ref{scoping} for the
% definition of an active math symbol.)  For example, if \texttt{-}, \texttt{>}, and
% \texttt{->} are the only active math symbols, the juxtaposition \texttt{>-} will be
% interpreted as the two symbols \texttt{>} and \texttt{-}, whereas \texttt{->} will
% always be interpreted as that single symbol.\footnote{For better readability we
% recommend a white space between each token.  This also makes searching for a
% symbol easier to do with an editor.  Omission of optional white space is useful
% for reducing typing when assigning an expression to a temporary
% variable\index{temporary variable} with the \texttt{let variable} Metamath
% program command.}\index{\texttt{let variable} command}
%
% Keywords\index{keyword} may be placed next to math symbols without white
% space\index{white space} between them.\footnote{Again, we do not recommend
% this for readability.}
%
% The math symbols\index{math symbol} in \texttt{\$c}\index{\texttt{\$c} statement}
% and \texttt{\$v}\index{\texttt{\$v} statement} statements must always be separated
% by white space\index{white
% space}, for the obvious reason that these statements define the names
% of the symbols.
%
% Math symbols referred to in comments (see Section~\ref{comments}) must also be
% separated by white space.  This allows you to make comments about symbols that
% are not yet active\index{active
% math symbol}.  (The ``math mode'' feature of comments is also a quick and
% easy way to obtain word processing text with embedded mathematical symbols,
% independently of the main purpose of Metamath; the way to do this is described
% in Section~\ref{comments})

\subsubsection{Label Tokens}\index{token}\index{label}

Label tokens are used to identify Metamath\index{Metamath} statements for
later reference. Label tokens may contain only letters, digits, and the three
characters period, hyphen, and underscore:
\begin{verbatim}
. - _
\end{verbatim}

A label is {\bf declared}\index{label declaration} by placing it immediately
before the keyword of the statement it identifies.  For example, the label
\texttt{axiom.1} might be declared as follows:
\begin{verbatim}
axiom.1 $a |- x = x $.
\end{verbatim}

Each \texttt{\$e}\index{\texttt{\$e} statement},
\texttt{\$f}\index{\texttt{\$f} statement},
\texttt{\$a}\index{\texttt{\$a} statement}, and
\texttt{\$p}\index{\texttt{\$p} statement} statement in a database must
have a label declared for it.  No other statement types may have label
declarations.  Every label must be unique.

A label (and the statement it identifies) is {\bf referenced}\index{label
reference} by including the label between the \texttt{\$=}\index{\texttt{\$=}
keyword} and \texttt{\$.}\index{\texttt{\$.}\ keyword}\ keywords in a \texttt{\$p}
statement.  The sequence of labels\index{label sequence} between these two
keywords is called a {\bf proof}\index{proof}.  An example of a statement with
a proof that we will encounter later (Section~\ref{proof}) is
\begin{verbatim}
wnew $p wff ( s -> ( r -> p ) )
     $= ws wr wp w2 w2 $.
\end{verbatim}

You don't have to know what this means just yet, but you should know that the
label \texttt{wnew} is declared by this \texttt{\$p} statement and that the labels
\texttt{ws}, \texttt{wr}, \texttt{wp}, and \texttt{w2} are assumed to have been declared
earlier in the database and are referenced here.

\subsection{Constants and Variables}
\index{constant}
\index{variable}

An {\bf expression}\index{expression} is any sequence of math
symbols, possibly empty.

The basic Metamath\index{Metamath} language\index{basic language} has two
kinds of math symbols\index{math symbol}:  {\bf constants}\index{constant} and
{\bf variables}\index{variable}.  In a Metamath proof, a constant may not be
substituted with any expression.  A variable can be
substituted\index{substitution!variable}\index{variable substitution} with any
expression.  This sequence may include other variables and may even include
the variable being substituted.  This substitution takes place when proofs are
verified, and it will be described in Section~\ref{proof}.  The \texttt{\$f}
statement (described later in Section~\ref{dollaref}) is used to specify the
{\bf type} of a variable (i.e.\ what kind of
variable it is)\index{variable type}\index{type} and
give it a meaning typically
associated with a ``metavariable''\index{metavariable}\footnote{A metavariable
is a variable that ranges over the syntactical elements of the object language
being discussed; for example, one metavariable might represent a variable of
the object language and another metavariable might represent a formula in the
object language.} in ordinary mathematics; for example, a variable may be
specified to be a wff or well-formed formula (in logic), a set (in set
theory), or a non-negative integer (in number theory).

%\subsection{The \texttt{\$c} and \texttt{\$v} Declaration Statements}
\subsection{The \texttt{\$c} and \texttt{\$v} Declaration Statements}
\index{\texttt{\$c} statement}
\index{constant declaration}
\index{\texttt{\$v} statement}
\index{variable declaration}

Constants are introduced or {\bf declared}\index{constant declaration}
with \texttt{\$c}\index{\texttt{\$c} statement} statements, and
variables are declared\index{variable declaration} with
\texttt{\$v}\index{\texttt{\$v} statement} statements.  A {\bf simple}
declaration\index{simple declaration} statement introduces a single
constant or variable.  Its syntax is one of the following:
\begin{center}
  \texttt{\$c} {\em math-symbol} \texttt{\$.}\\
  \texttt{\$v} {\em math-symbol} \texttt{\$.}
\end{center}
The notation {\em math-symbol} means any math symbol token\index{token}.

Some examples of simple declaration statements are:
\begin{center}
  \texttt{\$c + \$.}\\
  \texttt{\$c -> \$.}\\
  \texttt{\$c ( \$.}\\
  \texttt{\$v x \$.}\\
  \texttt{\$v y2 \$.}
\end{center}

The characters in a math symbol\index{math symbol} being declared are
irrelevant to Meta\-math; for example, we could declare a right parenthesis to
be a variable,
\begin{center}
  \texttt{\$v ) \$.}\\
\end{center}
although this would be unconventional.

A {\bf compound} declaration\index{compound declaration} statement is a
shorthand for declaring several symbols at once.  Its syntax is one of the
following:
\begin{center}
  \texttt{\$c} {\em math-symbol}\ \,$\cdots$\ {\em math-symbol} \texttt{\$.}\\
  \texttt{\$v} {\em math-symbol}\ \,$\cdots$\ {\em math-symbol} \texttt{\$.}
\end{center}\index{\texttt{\$c} statement}
Here, the ellipsis (\ldots) means any number of {\em math-symbol}\,s.

An example of a compound declaration statement is:
\begin{center}
  \texttt{\$v x y mu \$.}\\
\end{center}
This is equivalent to the three simple declaration statements
\begin{center}
  \texttt{\$v x \$.}\\
  \texttt{\$v y \$.}\\
  \texttt{\$v mu \$.}\\
\end{center}
\index{\texttt{\$v} statement}

There are certain rules on where in the database math symbols may be declared,
what sections of the database are aware of them (i.e.\ where they are
``active''), and when they may be declared more than once.  These will be
discussed in Section~\ref{scoping} and specifically on
p.~\pageref{redeclaration}.

\subsection{The \texttt{\$d} Statement}\label{dollard}
\index{\texttt{\$d} statement}

The \texttt{\$d} statement is called a {\bf disjoint-variable restriction}.  The
syntax of the {\bf simple} version of this statement is
\begin{center}
  \texttt{\$d} {\em variable variable} \texttt{\$.}
\end{center}
where each {\em variable} is a previously declared variable and the two {\em
variable}\,s are different.  (More specifically, each  {\em variable} must be
an {\bf active} variable\index{active math symbol}, which means there must be
a previous \texttt{\$v} statement whose {\bf scope}\index{scope} includes the
\texttt{\$d} statement.  These terms will be defined when we discuss scoping
statements in Section~\ref{scoping}.)

In ordinary mathematics, formulas may arise that are true if the variables in
them are distinct\index{distinct variables}, but become false when those
variables are made identical. For example, the formula in logic $\exists x\,x
\neq y$, which means ``for a given $y$, there exists an $x$ that is not equal
to $y$,'' is true in most mathematical theories (namely all non-trivial
theories\index{non-trivial theory}, i.e.\ those that describe more than one
individual, such as arithmetic).  However, if we substitute $y$ with $x$, we
obtain $\exists x\,x \neq x$, which is always false, as it means ``there
exists something that is not equal to itself.''\footnote{If you are a
logician, you will recognize this as the improper substitution\index{proper
substitution}\index{substitution!proper} of a free variable\index{free
variable} with a bound variable\index{bound variable}.  Metamath makes no
inherent distinction between free and bound variables; instead, you let
Metamath know what substitutions are permissible by using \texttt{\$d} statements
in the right way in your axiom system.}\index{free vs.\ bound variable}  The
\texttt{\$d} statement allows you to specify a restriction that forbids the
substitution of one variable with another.  In
this case, we would use the statement
\begin{center}
  \texttt{\$d x y \$.}
\end{center}\index{\texttt{\$d} statement}
to specify this restriction.

The order in which the variables appear in a \texttt{\$d} statement is not
important.  We could also use
\begin{center}
  \texttt{\$d y x \$.}
\end{center}

The \texttt{\$d} statement is actually more general than this, as the
``disjoint''\index{disjoint variables} in its name suggests.  The full meaning
is that if any substitution is made to its two variables (during the
course of a proof that references a \texttt{\$a} or \texttt{\$p} statement
associated with the \texttt{\$d}), the two expressions that result from the
substitution must have no variables in common.  In addition, each possible
pair of variables, one from each expression, must be in a \texttt{\$d} statement
associated with the statement being proved.  (This requirement forces the
statement being proved to ``inherit'' the original disjoint variable
restriction.)

For example, suppose \texttt{u} is a variable.  If the restriction
\begin{center}
  \texttt{\$d A B \$.}
\end{center}
has been specified for a theorem referenced in a
proof, we may not substitute \texttt{A} with \mbox{\tt a + u} and
\texttt{B} with \mbox{\tt b + u} because these two symbol sequences have the
variable \texttt{u} in common.  Furthermore, if \texttt{a} and \texttt{b} are
variables, we may not substitute \texttt{A} with \texttt{a} and \texttt{B} with \texttt{b}
unless we have also specified \texttt{\$d a b} for the theorem being proved; in
other words, the \texttt{\$d} property associated with a pair of variables must
be effectively preserved after substitution.

The \texttt{\$d}\index{\texttt{\$d} statement} statement does {\em not} mean ``the
two variables may not be substituted with the same thing,'' as you might think
at first.  For example, substituting each of \texttt{A} and \texttt{B} in the above
example with identical symbol sequences consisting only of constants does not
cause a disjoint variable conflict, because two symbol sequences have no
variables in common (since they have no variables, period).  Similarly, a
conflict will not occur by substituting the two variables in a \texttt{\$d}
statement with the empty symbol sequence\index{empty substitution}.

The \texttt{\$d} statement does not have a direct counterpart in
ordinary mathematics, partly because the variables\index{variable} of
Metamath are not really the same as the variables\index{variable!in
ordinary mathematics} of ordinary mathematics but rather are
metavariables\index{metavariable} ranging over them (as well as over
other kinds of symbols and groups of symbols).  Depending on the
situation, we may informally interpret the \texttt{\$d} statement in
different ways.  Suppose, for example, that \texttt{x} and \texttt{y}
are variables ranging over numbers (more precisely, that \texttt{x} and
\texttt{y} are metavariables ranging over variables that range over
numbers), and that \texttt{ph} ($\varphi$) and \texttt{ps} ($\psi$) are
variables (more precisely, metavariables) ranging over formulas.  We can
make the following interpretations that correspond to the informal
language of ordinary mathematics:
\begin{quote}
\begin{tabbing}
\texttt{\$d x y \$.} means ``assume $x$ and $y$ are
distinct variables.''\\
\texttt{\$d x ph \$.} means ``assume $x$ does not
occur in $\varphi$.''\\
\texttt{\$d ph ps \$.} \=means ``assume $\varphi$ and
$\psi$ have no variables\\ \>in common.''
\end{tabbing}
\end{quote}\index{\texttt{\$d} statement}

\subsubsection{Compound \texttt{\$d} Statements}

The {\bf compound} version of the \texttt{\$d} statement is a shorthand for
specifying several variables whose substitutions must be pairwise disjoint.
Its syntax is:
\begin{center}
  \texttt{\$d} {\em variable}\ \,$\cdots$\ {\em variable} \texttt{\$.}
\end{center}\index{\texttt{\$d} statement}
Here, {\em variable} represents the token of a previously declared
variable (specifically, an active variable) and all {\em variable}\,s are
different.  The compound \texttt{\$d}
statement is internally broken up by Metamath into one simple \texttt{\$d}
statement for each possible pair of variables in the original \texttt{\$d}
statement.  For example,
\begin{center}
  \texttt{\$d w x y z \$.}
\end{center}
is equivalent to
\begin{center}
  \texttt{\$d w x \$.}\\
  \texttt{\$d w y \$.}\\
  \texttt{\$d w z \$.}\\
  \texttt{\$d x y \$.}\\
  \texttt{\$d x z \$.}\\
  \texttt{\$d y z \$.}
\end{center}

Two or more simple \texttt{\$d} statements specifying the same variable pair are
internally combined into a single \texttt{\$d} statement.  Thus the set of three
statements
\begin{center}
  \texttt{\$d x y \$.}
  \texttt{\$d x y \$.}
  \texttt{\$d y x \$.}
\end{center}
is equivalent to
\begin{center}
  \texttt{\$d x y \$.}
\end{center}

Similarly, compound \texttt{\$d} statements, after being internally broken up,
internally have their common variable pairs combined.  For example the
set of statements
\begin{center}
  \texttt{\$d x y A \$.}
  \texttt{\$d x y B \$.}
\end{center}
is equivalent to
\begin{center}
  \texttt{\$d x y \$.}
  \texttt{\$d x A \$.}
  \texttt{\$d y A \$.}
  \texttt{\$d x y \$.}
  \texttt{\$d x B \$.}
  \texttt{\$d y B \$.}
\end{center}
which is equivalent to
\begin{center}
  \texttt{\$d x y \$.}
  \texttt{\$d x A \$.}
  \texttt{\$d y A \$.}
  \texttt{\$d x B \$.}
  \texttt{\$d y B \$.}
\end{center}

Metamath\index{Metamath} automatically verifies that all \texttt{\$d}
restrictions are met whenever it verifies proofs.  \texttt{\$d} statements are
never referenced directly in proofs (this is why they do not have
labels\index{label}), but Metamath is always aware of which ones must be
satisfied (i.e.\ are active) and will notify you with an error message if any
violation occurs.

To illustrate how Metamath detects a missing \texttt{\$d}
statement, we will look at the following example from the
\texttt{set.mm} database.

\begin{verbatim}
$d x z $.  $d y z $.
$( Theorem to add distinct quantifier to atomic formula. $)
ax17eq $p |- ( x = y -> A. z x = y ) $=...
\end{verbatim}

This statement has the obvious requirement that $z$ must be
distinct\index{distinct variables} from $x$ in theorem \texttt{ax17eq} that
states $x=y \rightarrow \forall z \, x=y$ (well, obvious if you're a logician,
for otherwise we could conclude  $x=y \rightarrow \forall x \, x=y$, which is
false when the free variables $x$ and $y$ are equal.).

Let's look at what happens if we edit the database to comment out this
requirement.

\begin{verbatim}
$( $d x z $. $) $d y z $.
$( Theorem to add distinct quantifier to atomic formula. $)
ax17eq $p |- ( x = y -> A. z x = y ) $=...
\end{verbatim}

When it tries to verify the proof, Metamath will tell you that \texttt{x} and
\texttt{z} must be disjoint, because one of its steps references an axiom or
theorem that has this requirement.

\begin{verbatim}
MM> verify proof ax17eq
ax17eq ?Error at statement 1918, label "ax17eq", type "$p":
      vz wal wi vx vy vz ax-13 vx vy weq vz vx ax-c16 vx vy
                                               ^^^^^
There is a disjoint variable ($d) violation at proof step 29.
Assertion "ax-c16" requires that variables "x" and "y" be
disjoint.  But "x" was substituted with "z" and "y" was
substituted with "x".  The assertion being proved, "ax17eq",
does not require that variables "z" and "x" be disjoint.
\end{verbatim}

We can see the substitutions into \texttt{ax-c16} with the following command.

\begin{verbatim}
MM> show proof ax17eq / detailed_step 29
Proof step 29:  pm2.61dd.2=ax-c16 $a |- ( A. z z = x -> ( x =
  y -> A. z x = y ) )
This step assigns source "ax-c16" ($a) to target "pm2.61dd.2"
($e).  The source assertion requires the hypotheses "wph"
($f, step 26), "vx" ($f, step 27), and "vy" ($f, step 28).
The parent assertion of the target hypothesis is "pm2.61dd"
($p, step 36).
The source assertion before substitution was:
    ax-c16 $a |- ( A. x x = y -> ( ph -> A. x ph ) )
The following substitutions were made to the source
assertion:
    Variable  Substituted with
     x         z
     y         x
     ph        x = y
The target hypothesis before substitution was:
    pm2.61dd.2 $e |- ( ph -> ch )
The following substitutions were made to the target
hypothesis:
    Variable  Substituted with
     ph        A. z z = x
     ch        ( x = y -> A. z x = y )
\end{verbatim}

The disjoint variable restrictions of \texttt{ax-c16} can be seen from the
\texttt{show state\-ment} command.  The line that begins ``\texttt{Its mandatory
dis\-joint var\-i\-able pairs are:}\ldots'' lists any \texttt{\$d} variable
pairs in brackets.

\begin{verbatim}
MM> show statement ax-c16/full
Statement 3033 is located on line 9338 of the file "set.mm".
"Axiom of Distinct Variables. ..."
  ax-c16 $a |- ( A. x x = y -> ( ph -> A. x ph ) ) $.
Its mandatory hypotheses in RPN order are:
  wph $f wff ph $.
  vx $f set x $.
  vy $f set y $.
Its mandatory disjoint variable pairs are:  <x,y>
The statement and its hypotheses require the variables:  x y
      ph
The variables it contains are:  x y ph
\end{verbatim}

Since Metamath will always detect when \texttt{\$d}\index{\texttt{\$d} statement}
statements are needed for a proof, you don't have to worry too much about
forgetting to put one in; it can always be added if you see the error message
above.  If you put in unnecessary \texttt{\$d} statements, the worst that could
happen is that your theorem might not be as general as it could be, and this
may limit its use later on.

On the other hand, when you introduce axioms (\texttt{\$a}\index{\texttt{\$a}
statement} statements), you must be very careful to properly specify the
necessary associated \texttt{\$d} statements since Metamath has no way of knowing
whether your axioms are correct.  For example, Metamath would have no idea
that \texttt{ax-c16}, which we are telling it is an axiom of logic, would lead to
contradictions if we omitted its associated \texttt{\$d} statement.

% This was previously a comment in footnote-sized type, but it can be
% hard to read this much text in a small size.
% As a result, it's been changed to normally-sized text.
\label{nodd}
You may wonder if it is possible to develop standard
mathematics in the Metamath language without the \texttt{\$d}\index{\texttt{\$d}
statement} statement, since it seems like a nuisance that complicates proof
verification. The \texttt{\$d} statement is not needed in certain subsets of
mathematics such as propositional calculus.  However, dummy
variables\index{dummy variable!eliminating} and their associated \texttt{\$d}
statements are impossible to avoid in proofs in standard first-order logic as
well as in the variant used in \texttt{set.mm}.  In fact, there is no upper bound to
the number of dummy variables that might be needed in a proof of a theorem of
first-order logic containing 3 or more variables, as shown by H.\
Andr\'{e}ka\index{Andr{\'{e}}ka, H.} \cite{Nemeti}.  A first-order system that
avoids them entirely is given in \cite{Megill}\index{Megill, Norman}; the
trick there is simply to embed harmlessly the necessary dummy variables into a
theorem being proved so that they aren't ``dummy'' anymore, then interpret the
resulting longer theorem so as to ignore the embedded dummy variables.  If
this interests you, the system in \texttt{set.mm} obtained from \texttt{ax-1}
through \texttt{ax-c14} in \texttt{set.mm}, and deleting \texttt{ax-c16} and \texttt{ax-5},
requires no \texttt{\$d} statements but is logically complete in the sense
described in \cite{Megill}.  This means it can prove any theorem of
first-order logic as long as we add to the theorem an antecedent that embeds
dummy and any other variables that must be distinct.  In a similar fashion,
axioms for set theory can be devised that
do not require distinct variable
provisos\index{Set theory without distinct variable provisos},
as explained at
\url{http://us.metamath.org/mpeuni/mmzfcnd.html}.
Together, these in principle allow all of
mathematics to be developed under Metamath without a \texttt{\$d} statement,
although the length of the resulting theorems will grow as more and
more dummy variables become required in their proofs.

\subsection{The \texttt{\$f}
and \texttt{\$e} Statements}\label{dollaref}
\index{\texttt{\$e} statement}
\index{\texttt{\$f} statement}
\index{floating hypothesis}
\index{essential hypothesis}
\index{variable-type hypothesis}
\index{logical hypothesis}
\index{hypothesis}

Metamath has two kinds of hypo\-theses, the \texttt{\$f}\index{\texttt{\$f}
statement} or {\bf variable-type} hypothesis and the \texttt{\$e} or {\bf logical}
hypo\-the\-sis.\index{\texttt{\$d} statement}\footnote{Strictly speaking, the
\texttt{\$d} statement is also a hypothesis, but it is never directly referenced
in a proof, so we call it a restriction rather than a hypothesis to lessen
confusion.  The checking for violations of \texttt{\$d} restrictions is automatic
and built into Metamath's proof-checking algorithm.} The letters \texttt{f} and
\texttt{e} stand for ``floating''\index{floating hypothesis} (roughly meaning
used only if relevant) and ``essential''\index{essential hypothesis} (meaning
always used) respectively, for reasons that will become apparent
when we discuss frames in
Section~\ref{frames} and scoping in Section~\ref{scoping}. The syntax of these
are as follows:
\begin{center}
  {\em label} \texttt{\$f} {\em typecode} {\em variable} \texttt{\$.}\\
  {\em label} \texttt{\$e} {\em typecode}
      {\em math-symbol}\ \,$\cdots$\ {\em math-symbol} \texttt{\$.}\\
\end{center}
\index{\texttt{\$e} statement}
\index{\texttt{\$f} statement}
A hypothesis must have a {\em label}\index{label}.  The expression in a
\texttt{\$e} hypothesis consists of a typecode (an active constant math symbol)
followed by a sequence
of zero or more math symbols. Each math symbol (including {\em constant}
and {\em variable}) must be a previously declared constant or variable.  (In
addition, each math symbol must be active, which will be covered when we
discuss scoping statements in Section~\ref{scoping}.)  You use a \texttt{\$f}
hypothesis to specify the
nature or {\bf type}\index{variable type}\index{type} of a variable (such as ``let $x$ be an
integer'') and use a \texttt{\$e} hypothesis to express a logical truth (such as
``assume $x$ is prime'') that must be established in order for an assertion
requiring it to also be true.

A variable must have its type specified in a \texttt{\$f} statement before
it may be used in a \texttt{\$e}, \texttt{\$a}, or \texttt{\$p}
statement.  There may be only one (active) \texttt{\$f} statement for a
given variable.  (``Active'' is defined in Section~\ref{scoping}.)

In ordinary mathematics, theorems\index{theorem} are often expressed in the
form ``Assume $P$; then $Q$,'' where $Q$ is a statement that you can derive
if you start with statement $P$.\index{free variable}\footnote{A stronger
version of a theorem like this would be the {\em single} formula $P\rightarrow
Q$ ($P$ implies $Q$) from which the weaker version above follows by the rule
of modus ponens in logic.  We are not discussing this stronger form here.  In
the weaker form, we are saying only that if we can {\em prove} $P$, then we can
{\em prove} $Q$.  In a logician's language, if $x$ is the only free variable
in $P$ and $Q$, the stronger form is equivalent to $\forall x ( P \rightarrow
Q)$ (for all $x$, $P$ implies $Q$), whereas the weaker form is equivalent to
$\forall x P \rightarrow \forall x Q$. The stronger form implies the weaker,
but not vice-versa.  To be precise, the weaker form of the theorem is more
properly called an ``inference'' rather than a theorem.}\index{inference}
In the
Metamath\index{Metamath} language, you would express mathematical statement
$P$ as a hypothesis (a \texttt{\$e} Metamath language statement in this case) and
statement $Q$ as a provable assertion (a \texttt{\$p}\index{\texttt{\$p} statement}
statement).

Some examples of hypotheses you might encounter in logic and set theory are
\begin{center}
  \texttt{stmt1 \$f wff P \$.}\\
  \texttt{stmt2 \$f set x \$.}\\
  \texttt{stmt3 \$e |- ( P -> Q ) \$.}
\end{center}
\index{\texttt{\$e} statement}
\index{\texttt{\$f} statement}
Informally, these would be read, ``Let $P$ be a well-formed-formula,'' ``Let
$x$ be an (individual) variable,'' and ``Assume we have proved $P \rightarrow
Q$.''  The turnstile symbol \,$\vdash$\index{turnstile ({$\,\vdash$})} is
commonly used in logic texts to mean ``a proof exists for.''

To summarize:
\begin{itemize}
\item A \texttt{\$f} hypothesis tells Metamath the type or kind of its variable.
It is analogous to a variable declaration in a computer language that
tells the compiler that a variable is an integer or a floating-point
number.
\item The \texttt{\$e} hypothesis corresponds to what you would usually call a
``hypothesis'' in ordinary mathematics.
\end{itemize}

Before an assertion\index{assertion} (\texttt{\$a} or \texttt{\$p} statement) can be
referenced in a proof, all of its associated \texttt{\$f} and \texttt{\$e} hypotheses
(i.e.\ those \texttt{\$e} hypotheses that are active) must be satisfied (i.e.
established by the proof).  The meaning of ``associated'' (which we will call
{\bf mandatory} in Section~\ref{frames}) will become clear when we discuss
scoping later.

Note that after any \texttt{\$f}, \texttt{\$e},
\texttt{\$a}, or \texttt{\$p} token there is a required
\textit{typecode}\index{typecode}.
The typecode is a constant used to enforce types of expressions.
This will become clearer once we learn more about
assertions (\texttt{\$a} and \texttt{\$p} statements).
An example may also clarify their purpose.
In the
\texttt{set.mm}\index{set theory database (\texttt{set.mm})}%
\index{Metamath Proof Explorer}
database,
the following typecodes are used:

\begin{itemize}
\item \texttt{wff} :
  Well-formed formula (wff) symbol
  (read: ``the following symbol sequence is a wff'').
% The *textual* typecode for turnstile is "|-", but when read it's a little
% confusing, so I intentionally display the mathematical symbol here instead
% (I think it's clearer in this context).
\item \texttt{$\vdash$} :
  Turnstile (read: ``the following symbol sequence is provable'' or
  ``a proof exists for'').
\item \texttt{setvar} :
  Individual set variable type (read: ``the following is an
  individual set variable'').
  Note that this is \textit{not} the type of an arbitrary set expression,
  instead, it is used to ensure that there is only a single symbol used
  after quantifiers like for-all ($\forall$) and there-exists ($\exists$).
\item \texttt{class} :
  An expression that is a syntactically valid class expression.
  All valid set expressions are also valid class expression, so expressions
  of sets normally have the \texttt{class} typecode.
  Use the \texttt{class} typecode,
  \textit{not} the \texttt{setvar} typecode,
  for the type of set expressions unless you are specifically identifying
  a single set variable.
\end{itemize}

\subsection{Assertions (\texttt{\$a} and \texttt{\$p} Statements)}
\index{\texttt{\$a} statement}
\index{\texttt{\$p} statement}\index{assertion}\index{axiomatic assertion}
\index{provable assertion}

There are two types of assertions, \texttt{\$a}\index{\texttt{\$a} statement}
statements ({\bf axiomatic assertions}) and \texttt{\$p} statements ({\bf
provable assertions}).  Their syntax is as follows:
\begin{center}
  {\em label} \texttt{\$a} {\em typecode} {\em math-symbol} \ldots
         {\em math-symbol} \texttt{\$.}\\
  {\em label} \texttt{\$p} {\em typecode} {\em math-symbol} \ldots
        {\em math-symbol} \texttt{\$=} {\em proof} \texttt{\$.}
\end{center}
\index{\texttt{\$a} statement}
\index{\texttt{\$p} statement}
\index{\texttt{\$=} keyword}
An assertion always requires a {\em label}\index{label}. The expression in an
assertion consists of a typecode (an active constant)
followed by a sequence of zero
or more math symbols.  Each math symbol, including any {\em constant}, must be a
previously declared constant or variable.  (In addition, each math symbol
must be active, which will be covered when we discuss scoping statements in
Section~\ref{scoping}.)

A \texttt{\$a} statement is usually a definition of syntax (for example, if $P$
and $Q$ are wffs then so is $(P\to Q)$), an axiom\index{axiom} of ordinary
mathematics (for example, $x=x$), or a definition\index{definition} of
ordinary mathematics (for example, $x\ne y$ means $\lnot x=y$). A \texttt{\$p}
statement is a claim that a certain combination of math symbols follows from
previous assertions and is accompanied by a proof that demonstrates it.

Assertions can also be referenced in (later) proofs in order to derive new
assertions from them. The label of an assertion is used to refer to it in a
proof. Section~\ref{proof} will describe the proof in detail.

Assertions also provide the primary means for communicating the mathematical
results in the database to people.  Proofs (when conveniently displayed)
communicate to people how the results were arrived at.

\subsubsection{The \texttt{\$a} Statement}
\index{\texttt{\$a} statement}

Axiomatic assertions (\texttt{\$a} statements) represent the starting points from
which other assertions (\texttt{\$p}\index{\texttt{\$p} statement} statements) are
derived.  Their most obvious use is for specifying ordinary mathematical
axioms\index{axiom}, but they are also used for two other purposes.

First, Metamath\index{Metamath} needs to know the syntax of symbol
sequences that constitute valid mathematical statements.  A Metamath
proof must be broken down into much more detail than ordinary
mathematical proofs that you may be used to thinking of (even the
``complete'' proofs of formal logic\index{formal logic}).  This is one
of the things that makes Metamath a general-purpose language,
independent of any system of logic or even syntax.  If you want to use a
substitution instance of an assertion as a step in a proof, you must
first prove that the substitution is syntactically correct (or if you
prefer, you must ``construct'' it), showing for example that the
expression you are substituting for a wff metavariable is a valid wff.
The \texttt{\$a}\index{\texttt{\$a} statement} statement is used to
specify those combinations of symbols that are considered syntactically
valid, such as the legal forms of wffs.

Second, \texttt{\$a} statements are used to specify what are ordinarily thought of
as definitions, i.e.\ new combinations of symbols that abbreviate other
combinations of symbols.  Metamath makes no distinction\index{axiom vs.\
definition} between axioms\index{axiom} and definitions\index{definition}.
Indeed, it has been argued that such distinction should not be made even in
ordinary mathematics; see Section~\ref{definitions}, which discusses the
philosophy of definitions.  Section~\ref{hierarchy} discusses some
technical requirements for definitions.  In \texttt{set.mm} we adopt the
convention of prefixing axiom labels with \texttt{ax-} and definition labels with
\texttt{df-}\index{label}.

The results that can be derived with the Metamath language are only as good as
the \texttt{\$a}\index{\texttt{\$a} statement} statements used as their starting
point.  We cannot stress this too strongly.  For example, Metamath will
not prevent you from specifying $x\neq x$ as an axiom of logic.  It is
essential that you scrutinize all \texttt{\$a} statements with great care.
Because they are a source of potential pitfalls, it is best not to add new
ones (usually new definitions) casually; rather you should carefully evaluate
each one's necessity and advantages.

Once you have in place all of the basic axioms\index{axiom} and
rules\index{rule} of a mathematical theory, the only \texttt{\$a} statements that
you will be adding will be what are ordinarily called definitions.  In
principle, definitions should be in some sense eliminable from the language of
a theory according to some convention (usually involving logical equivalence
or equality).  The most common convention is that any formula that was
syntactically valid but not provable before the definition was introduced will
not become provable after the definition is introduced.  In an ideal world,
definitions should not be present at all if one is to have absolute confidence
in a mathematical result.  However, they are necessary to make
mathematics practical, for otherwise the resulting formulas would be
extremely long and incomprehensible.  Since the nature of definitions (in the
most general sense) does not permit them to automatically be verified as
``proper,''\index{proper definition}\index{definition!proper} the judgment of
the mathematician is required to ensure it.  (In \texttt{set.mm} effort was made
to make almost all definitions directly eliminable and thus minimize the need
for such judgment.)

If you are not a mathematician, it may be best not to add or change any
\texttt{\$a}\index{\texttt{\$a} statement} statements but instead use
the mathematical language already provided in standard databases.  This
way Metamath will not allow you to make a mistake (i.e.\ prove a false
result).


\subsection{Frames}\label{frames}

We now introduce the concept of a collection of related Metamath statements
called a frame.  Every assertion (\texttt{\$a} or \texttt{\$p} statement) in the database has
an associated frame.

A {\bf frame}\index{frame} is a sequence of \texttt{\$d}, \texttt{\$f},
and \texttt{\$e} statements (zero or more of each) followed by one
\texttt{\$a} or \texttt{\$p} statement, subject to certain conditions we
will describe.  For simplicity we will assume that all math symbol
tokens used are declared at the beginning of the database with
\texttt{\$c} and \texttt{\$v} statements (which are not properly part of
a frame).  Also for simplicity we will assume there are only simple
\texttt{\$d} statements (those with only two variables) and imagine any
compound \texttt{\$d} statements (those with more than two variables) as
broken up into simple ones.

A frame groups together those hypotheses (and \texttt{\$d} statements) relevant
to an assertion (\texttt{\$a} or \texttt{\$p} statement).  The statements in a frame
may or may not be physically adjacent in a database; we will cover
this in our discussion of scoping statements
in Section~\ref{scoping}.

A frame has the following properties:
\begin{enumerate}
 \item The set of variables contained in its \texttt{\$f} statements must
be identical to the set of variables contained in its \texttt{\$e},
\texttt{\$a}, and/or \texttt{\$p} statements.  In other words, each
variable in a \texttt{\$e}, \texttt{\$a}, or \texttt{\$p} statement must
have an associated ``variable type'' defined for it in a \texttt{\$f}
statement.
  \item No two \texttt{\$f} statements may contain the same variable.
  \item Any \texttt{\$f} statement
must occur before a \texttt{\$e} statement in which its variable occurs.
\end{enumerate}

The first property determines the set of variables occurring in a frame.
These are the {\bf mandatory
variables}\index{mandatory variable} of the frame.  The second property
tells us there must be only one type specified for a variable.
The last property is not a theoretical requirement but it
makes parsing of the database easier.

For our examples, we assume our database has the following declarations:

\begin{verbatim}
$v P Q R $.
$c -> ( ) |- wff $.
\end{verbatim}

The following sequence of statements, describing the modus ponens inference
rule, is an example of a frame:

\begin{verbatim}
wp  $f wff P $.
wq  $f wff Q $.
maj $e |- ( P -> Q ) $.
min $e |- P $.
mp  $a |- Q $.
\end{verbatim}

The following sequence of statements is not a frame because \texttt{R} does not
occur in the \texttt{\$e}'s or the \texttt{\$a}:

\begin{verbatim}
wp  $f wff P $.
wq  $f wff Q $.
wr  $f wff R $.
maj $e |- ( P -> Q ) $.
min $e |- P $.
mp  $a |- Q $.
\end{verbatim}

The following sequence of statements is not a frame because \texttt{Q} does not
occur in a \texttt{\$f}:

\begin{verbatim}
wp  $f wff P $.
maj $e |- ( P -> Q ) $.
min $e |- P $.
mp  $a |- Q $.
\end{verbatim}

The following sequence of statements is not a frame because the \texttt{\$a} statement is
not the last one:

\begin{verbatim}
wp  $f wff P $.
wq  $f wff Q $.
maj $e |- ( P -> Q ) $.
mp  $a |- Q $.
min $e |- P $.
\end{verbatim}

Associated with a frame is a sequence of {\bf mandatory
hypotheses}\index{mandatory hypothesis}.  This is simply the set of all
\texttt{\$f} and \texttt{\$e} statements in the frame, in the order they
appear.  A frame can be referenced in a later proof using the label of
the \texttt{\$a} or \texttt{\$p} assertion statement, and the proof
makes an assignment to each mandatory hypothesis in the order in which
it appears.  This means the order of the hypotheses, once chosen, must
not be changed so as not to affect later proofs referencing the frame's
assertion statement.  (The Metamath proof verifier will, of course, flag
an error if a proof becomes incorrect by doing this.)  Since proofs make
use of ``Reverse Polish notation,'' described in Section~\ref{proof}, we
call this order the {\bf RPN order}\index{RPN order} of the hypotheses.

Note that \texttt{\$d} statements are not part of the set of mandatory
hypotheses, and their order doesn't matter (as long as they satisfy the
fourth property for a frame described above).  The \texttt{\$d}
statements specify restrictions on variables that must be satisfied (and
are checked by the proof verifier) when expressions are substituted for
them in a proof, and the \texttt{\$d} statements themselves are never
referenced directly in a proof.

A frame with a \texttt{\$p} (provable) statement requires a proof as part of the
\texttt{\$p} statement.  Sometimes in a proof we want to make use of temporary or
dummy variables\index{dummy variable} that do not occur in the \texttt{\$p}
statement or its mandatory hypotheses.  To accommodate this we define an {\bf
extended frame}\index{extended frame} as a frame together with zero or more
\texttt{\$d} and \texttt{\$f} statements that reference variables not among the
mandatory variables of the frame.  Any new variables referenced are called the
{\bf optional variables}\index{optional variable} of the extended frame. If a
\texttt{\$f} statement references an optional variable it is called an {\bf
optional hypothesis}\index{optional hypothesis}, and if one or both of the
variables in a \texttt{\$d} statement are optional variables it is called an {\bf
optional disjoint-variable restriction}\index{optional disjoint-variable
restriction}.  Properties 2, 3, and 4 for a frame also apply to an extended
frame.

The concept of optional variables is not meaningful for frames with \texttt{\$a}
statements, since those statements have no proofs that might make use of them.
There is no restriction on including optional hypotheses in the extended frame
for a \texttt{\$a} statement, but they serve no purpose.

The following set of statements is an example of an extended frame, which
contains an optional variable \texttt{R} and an optional hypothesis \texttt{wr}.  In
this example, we suppose the rule of modus ponens is not an axiom but is
derived as a theorem from earlier statements (we omit its presumed proof).
Variable \texttt{R} may be used in its proof if desired (although this would
probably have no advantage in propositional calculus).  Note that the sequence
of mandatory hypotheses in RPN order is still \texttt{wp}, \texttt{wq}, \texttt{maj},
\texttt{min} (i.e.\ \texttt{wr} is omitted), and this sequence is still assumed
whenever the assertion \texttt{mp} is referenced in a subsequent proof.

\begin{verbatim}
wp  $f wff P $.
wq  $f wff Q $.
wr  $f wff R $.
maj $e |- ( P -> Q ) $.
min $e |- P $.
mp  $p |- Q $= ... $.
\end{verbatim}

Every frame is an extended frame, but not every extended frame is a frame, as
this example shows.  The underlying frame for an extended frame is
obtained by simply removing all statements containing optional variables.
Any proof referencing an assertion will ignore any extensions to its
frame, which means we may add or delete optional hypotheses at will without
affecting subsequent proofs.

The conceptually simplest way of organizing a Metamath database is as a
sequence of extended frames.  The scoping statements
\texttt{\$\char`\{}\index{\texttt{\$\char`\{} and \texttt{\$\char`\}}
keywords} and \texttt{\$\char`\}} can be used to delimit the start and
end of an extended frame, leading to the following possible structure for a
database.  \label{framelist}

\vskip 2ex
\setbox\startprefix=\hbox{\tt \ \ \ \ \ \ \ \ }
\setbox\contprefix=\hbox{}
\startm
\m{\mbox{(\texttt{\$v} {\em and} \texttt{\$c}\,{\em statements})}}
\endm
\startm
\m{\mbox{\texttt{\$\char`\{}}}
\endm
\startm
\m{\mbox{\texttt{\ \ } {\em extended frame}}}
\endm
\startm
\m{\mbox{\texttt{\$\char`\}}}}
\endm
\startm
\m{\mbox{\texttt{\$\char`\{}}}
\endm
\startm
\m{\mbox{\texttt{\ \ } {\em extended frame}}}
\endm
\startm
\m{\mbox{\texttt{\$\char`\}}}}
\endm
\startm
\m{\mbox{\texttt{\ \ \ \ \ \ \ \ \ }}\vdots}
\endm
\vskip 2ex

In practice, this structure is inconvenient because we have to repeat
any \texttt{\$f}, \texttt{\$e}, and \texttt{\$d} statements over and
over again rather than stating them once for use by several assertions.
The scoping statements, which we will discuss next, allow this to be
done.  In principle, any Metamath database can be converted to the above
format, and the above format is the most convenient to use when studying
a Metamath database as a formal system%
%% Uncomment this when uncommenting section {formalspec} below
   (Appendix \ref{formalspec})%
.
In fact, Metamath internally converts the database to the above format.
The command \texttt{show statement} in the Metamath program will show
you the contents of the frame for any \texttt{\$a} or \texttt{\$p}
statement, as well as its extension in the case of a \texttt{\$p}
statement.

%c%(provided that all ``local'' variables and constants with limited scope have
%c%unique names),

During our discussion of scoping statements, it may be helpful to
think in terms of the equivalent sequence of frames that will result when
the database is parsed.  Scoping (other than the limited
use above to delimit frames) is not a theoretical requirement for
Metamath but makes it more convenient.


\subsection{Scoping Statements (\texttt{\$\{} and \texttt{\$\}})}\label{scoping}
\index{\texttt{\$\char`\{} and \texttt{\$\char`\}} keywords}\index{scoping statement}

%c%Some Metamath statements may be needed only temporarily to
%c%serve a specific purpose, and after we're done with them we would like to
%c%disregard or ignore them.  For example, when we're finished using a variable,
%c%we might want to
%c%we might want to free up the token\index{token} used to name it so that the
%c%token can be used for other purposes later on, such as a different kind of
%c%variable or even a constant.  In the terminology of computer programming, we
%c%might want to let some symbol declarations be ``local'' rather than ``global.''
%c%\index{local symbol}\index{global symbol}

The {\bf scoping} statements, \texttt{\$\char`\{} ({\bf start of block}) and \texttt{\$\char`\}}
({\bf end of block})\index{block}, provide a means for controlling the portion
of a database over which certain statement types are recognized.  The
syntax of a scoping statement is very simple; it just consists of the
statement's keyword:
\begin{center}
\texttt{\$\char`\{}\\
\texttt{\$\char`\}}
\end{center}
\index{\texttt{\$\char`\{} and \texttt{\$\char`\}} keywords}

For example, consider the following database where we have stripped out
all tokens except the scoping statement keywords.  For the purpose of the
discussion, we have added subscripts to the scoping statements; these subscripts
do not appear in the actual database.
\[
 \mbox{\tt \ \$\char`\{}_1
 \mbox{\tt \ \$\char`\{}_2
 \mbox{\tt \ \$\char`\}}_2
 \mbox{\tt \ \$\char`\{}_3
 \mbox{\tt \ \$\char`\{}_4
 \mbox{\tt \ \$\char`\}}_4
 \mbox{\tt \ \$\char`\}}_3
 \mbox{\tt \ \$\char`\}}_1
\]
Each \texttt{\$\char`\{} statement in this example is said to be {\bf
matched} with the \texttt{\$\char`\}} statement that has the same
subscript.  Each pair of matched scoping statements defines a region of
the database called a {\bf block}.\index{block} Blocks can be {\bf
nested}\index{nested block} inside other blocks; in the example, the
block defined by $\mbox{\tt \$\char`\{}_4$ and $\mbox{\tt \$\char`\}}_4$
is nested inside the block defined by $\mbox{\tt \$\char`\{}_3$ and
$\mbox{\tt \$\char`\}}_3$ as well as inside the block defined by
$\mbox{\tt \$\char`\{}_1$ and $\mbox{\tt \$\char`\}}_1$.  In general, a
block may be empty, it may contain only non-scoping
statements,\footnote{Those statements other than \texttt{\$\char`\{} and
\texttt{\$\char`\}}.}\index{non-scoping statement} or it may contain any
mixture of other blocks and non-scoping statements.  (This is called a
``recursive'' definition\index{recursive definition} of a block.)

Associated with each block is a number called its {\bf nesting
level}\index{nesting level} that indicates how deeply the block is nested.
The nesting levels of the blocks in our example are as follows:
\[
  \underbrace{
    \mbox{\tt \ }
    \underbrace{
     \mbox{\tt \$\char`\{\ }
     \underbrace{
       \mbox{\tt \$\char`\{\ }
       \mbox{\tt \$\char`\}}
     }_{2}
     \mbox{\tt \ }
     \underbrace{
       \mbox{\tt \$\char`\{\ }
       \underbrace{
         \mbox{\tt \$\char`\{\ }
         \mbox{\tt \$\char`\}}
       }_{3}
       \mbox{\tt \ \$\char`\}}
     }_{2}
     \mbox{\tt \ \$\char`\}}
   }_{1}
   \mbox{\tt \ }
 }_{0}
\]
\index{\texttt{\$\char`\{} and \texttt{\$\char`\}} keywords}
The entire database is considered to be one big block (the {\bf outermost}
block) with a nesting level of 0.  The outermost block is {\em not} bracketed
by scoping statements.\footnote{The language was designed this way so that
several source files can be joined together more easily.}\index{outermost
block}

All non-scoping Metamath statements become recognized or {\bf
active}\index{active statement} at the place where they appear.\footnote{To
keep things slightly simpler, we do not bother to define the concept of
``active'' for the scoping statements.}  Certain of these statement types
become inactive at the end of the block in which they appear; these statement
types are:
\begin{center}
  \texttt{\$c}, \texttt{\$v}, \texttt{\$d}, \texttt{\$e}, and \texttt{\$f}.
%  \texttt{\$v}, \texttt{\$f}, \texttt{\$e}, and \texttt{\$d}.
\end{center}
\index{\texttt{\$c} statement}
\index{\texttt{\$d} statement}
\index{\texttt{\$e} statement}
\index{\texttt{\$f} statement}
\index{\texttt{\$v} statement}
The other statement types remain active forever (i.e.\ through the end of the
database); they are:
\begin{center}
  \texttt{\$a} and \texttt{\$p}.
%  \texttt{\$c}, \texttt{\$a}, and \texttt{\$p}.
\end{center}
\index{\texttt{\$a} statement}
\index{\texttt{\$p} statement}
Any statement (of these 7 types) located in the outermost
block\index{outermost block} will remain active through the end of the
database and thus are effectively ``global'' statements.\index{global
statement}

All \texttt{\$c} statements must be placed in the outermost block.  Since they are
therefore always global, they could be considered as belonging to both of the
above categories.

The {\bf scope}\index{scope} of a statement is the set of statements that
recognize it as active.

%c%The concept of ``active'' is also defined for math symbols\index{math
%c%symbol}.  Math symbols (constants\index{constant} and
%c%variables\index{variable}) become {\bf active}\index{active
%c%math symbol} in the \texttt{\$c}\index{\texttt{\$c}
%c%statement} and \texttt{\$v}\index{\texttt{\$v} statement} statements that
%c%declare them.  They become inactive when their declaration statements become
%c%inactive.

The concept of ``active'' is also defined for math symbols\index{math
symbol}.  Math symbols (constants\index{constant} and
variables\index{variable}) become {\bf active}\index{active math symbol}
in the \texttt{\$c}\index{\texttt{\$c} statement} and
\texttt{\$v}\index{\texttt{\$v} statement} statements that declare them.
A variable becomes inactive when its declaration statement becomes
inactive.  Because all \texttt{\$c} statements must be in the outermost
block, a constant will never become inactive after it is declared.

\subsubsection{Redeclaration of Math Symbols}
\index{redeclaration of symbols}\label{redeclaration}

%c%A math symbol may not be declared a second time while it is active, but it may
%c%be declared again after it becomes inactive.

A variable may not be declared a second time while it is active, but it may be
declared again after it becomes inactive.  This provides a convenient way to
introduce ``local'' variables,\index{local variable} i.e.\ temporary variables
for use in the frame of an assertion or in a proof without keeping them around
forever.  A previously declared variable may not be redeclared as a constant.

A constant may not be redeclared.  And, as mentioned above, constants must be
declared in the outermost block.

The reason variables may have limited scope but not constants is that an
assertion (\texttt{\$a} or \texttt{\$p} statement) remains available for use in
proofs through the end of the database.  Variables in an assertion's frame may
be substituted with whatever is needed in a proof step that references the
assertion, whereas constants remain fixed and may not be substituted with
anything.  The particular token used for a variable in an assertion's frame is
irrelevant when the assertion is referenced in a proof, and it doesn't matter
if that token is not available outside of the referenced assertion's frame.
Constants, however, must be globally fixed.

In the present version of the Metamath language, there is no theoretical
benefit for the feature allowing variables to be active for limited scopes
rather than global. It is just a convenience that allows them, for example, to
be locally grouped together with their corresponding \texttt{\$f} variable-type
declarations.\footnote{A future version\label{redeclarationf} of the Metamath
language may extend the language to include true local variables as well as
local constants.  I have not decided upon the best way to do this or even if
it should be done at all; the main issue is whether the benefits offset
greater confusion in learning the language.  In any case, the rules for
the present language were chosen to let existing databases be compatible with
any future language extensions.}

%c%If you declare a math symbol more than once, internally Metamath considers it a
%c%new distinct symbol, even though it has the same name.  If you are unaware of
%c%this, you may find that what you think are correct proofs are incorrectly
%c%rejected as invalid, because Metamath may tell you that a constant you
%c%previously declared does not match a newly declared math symbol with the same
%c%name.  For details on this subtle point, see the Comment on
%c%p.~\pageref{spec4comment}.  This is done purposely to allow temporary
%c%constants to be introduced while developing a subtheory, then allow their math
%c%symbol tokens to be reused later on; in general they will not refer to the
%c%same thing.  In practice, you would not ordinarily reuse the names of
%c%constants because it would tend to be confusing to the reader.  The reuse of
%c%names of variables, on the other hand, is something that is often useful to do
%c%(for example it is done frequently in \texttt{set.mm}).  Since variables in an
%c%assertion referenced in a proof can be substituted as needed to achieve a
%c%symbol match, this is not an issue.

% (This section covers a somewhat advanced topic you may want to skip
% at first reading.)
%
% Under certain circumstances, math symbol\index{math symbol}
% tokens\index{token} may be redeclared (i.e.\ the token
% may appear in more than
% one \texttt{\$c}\index{\texttt{\$c} statement} or \texttt{\$v}\index{\texttt{\$v}
% statement} statement).  You might want to do this say, to make temporary use
% of a variable name without having to worry about its affect elsewhere,
% somewhat analogous to declaring a local variable in a standard computer
% language.  Understanding what goes on when math symbol tokens are redeclared
% is a little tricky to understand at first, since it requires that we
% distinguish the token itself from the math symbol that it names.  It will help
% if we first take a peek at the internal workings of the
% Metamath\index{Metamath} program.
%
% Metamath reserves a memory location for each occurrence of a
% token\index{token} in a declaration statement (\texttt{\$c}\index{\texttt{\$c}
% statement} or \texttt{\$v}\index{\texttt{\$v} statement}).  If a given token appears
% in more than one declaration statement, it will refer to more than one memory
% locations.  A math symbol\index{math symbol} may be thought of as being one of
% these memory locations rather than as the token itself.  Only one of the
% memory locations associated with a given token may be active at any one time.
% The math symbol (memory location) that gets looked up when the token appears
% in a non-declaration statement is the one that happens to be active at that
% time.
%
% We now look at the rules for the redeclaration\index{redeclaration of symbols}
% of math symbol tokens.
% \begin{itemize}
% \item A math symbol token may not be declared twice in the
% same block.\footnote{While there is no theoretical reason for disallowing
% this, it was decided in the design of Metamath that allowing it would offer no
% advantage and might cause confusion.}
% \item An inactive math symbol may always be
% redeclared.
% \item  An active math symbol may be redeclared in a different (i.e.\
% inner) block\index{block} from the one it became active in.
% \end{itemize}
%
% When a math symbol token is redeclared, it conceptually refers to a different
% math symbol, just as it would be if it were called a different name.  In
% addition, the original math symbol that it referred to, if it was active,
% temporarily becomes inactive.  At the end of the block in which the
% redeclaration occurred, the new math symbol\index{math symbol} becomes
% inactive and the original symbol becomes active again.  This concept is
% illustrated in the following example, where the symbol \texttt{e} is
% ordinarily a constant (say Euler's constant, 2.71828...) but
% temporarily we want to use it as a ``local'' variable, say as a coefficient
% in the equation $a x^4 + b x^3 + c x^2 + d x + e$:
% \[
%   \mbox{\tt \$\char`\{\ \$c e \$.}
%   \underbrace{
%     \ \ldots\ %
%     \mbox{\tt \$\char`\{}\ \ldots\ %
%   }_{\mbox{\rm region A}}
%   \mbox{\tt \$v e \$.}
%   \underbrace{
%     \mbox{\ \ \ \ldots\ \ \ }
%   }_{\mbox{\rm region B}}
%   \mbox{\tt \$\char`\}}
%   \underbrace{
%     \mbox{\ \ \ \ldots\ \ \ }
%   }_{\mbox{\rm region C}}
%   \mbox{\tt \$\char`\}}
% \]
% \index{\texttt{\$\char`\{} and \texttt{\$\char`\}} keywords}
% In region A, the token \texttt{e} refers to a constant.  It is redeclared as a
% variable in region B, and any reference to it in this region will refer to this
% variable.  In region C, the redeclaration becomes inactive, and the original
% declaration becomes active again.  In region C, the token \texttt{x} refers to the
% original constant.
%
% As a practical matter, overuse of math symbol\index{math symbol}
% redeclarations\index{redeclaration of symbols} can be confusing (even though
% it is well-defined) and is best avoided when possible.  Here are some good
% general guidelines you can follow.  Usually, you should declare all
% constants\index{constant} in the outermost block\index{outermost block},
% especially if they are general-purpose (such as the token \verb$A.$, meaning
% $\forall$ or ``for all'').  This will make them ``globally'' active (although
% as in the example above local redeclarations will temporarily make them
% inactive.)  Most or all variables\index{variable}, on the other hand, could be
% declared in inner blocks, so that the token for them can be used later for a
% different type of variable or a constant.  (The names of the variables you
% choose are not used when you refer to an assertion\index{assertion} in a
% proof, whereas constants must match exactly.  A locally declared constant will
% not match a globally declared constant in a proof, even if they use the same
% token, because Metamath internally considers them to be different math
% symbols.)  To avoid confusion, you should generally avoid redeclaring active
% variables.  If you must redeclare them, do so at the beginning of a block.
% The temporary declaration of constants in inner blocks might be occasionally
% appropriate when you make use of a temporary definition to prove lemmas
% leading to a main result that does not make direct use of the definition.
% This way, you will not clutter up your database with a large number of
% seldom-used global constant symbols.  You might want to note that while
% inactive constants may not appear directly in an assertion (a \texttt{\$a}\index{\texttt{\$a}
% statement} or \texttt{\$p}\index{\texttt{\$p} statement}
% statement), they may be indirectly used in the proof of a \texttt{\$p} statement
% so long as they do not appear in the final math symbol sequence constructed by
% the proof.  In the end, you will have to use your best judgment, taking into
% account standard mathematical usage of the symbols as well as consideration
% for the reader of your work.
%
% \subsubsection{Reuse of Labels}\index{reuse of labels}\index{label}
%
% The \texttt{\$e}\index{\texttt{\$e} statement}, \texttt{\$f}\index{\texttt{\$f}
% statement}, \texttt{\$a}\index{\texttt{\$a} statement}, and
% \texttt{\$p}\index{\texttt{\$p}
% statement} statement types require labels, which allow them to be
% referenced later inside proofs.  A label is considered {\bf
% active}\index{active label} when the statement it is associated with is
% active.  The token\index{token} for a label may be reused
% (redeclared)\index{redeclaration of labels} provided that it is not being used
% for a currently active label.  (Unlike the tokens for math symbols, active
% label tokens may not be redeclared in an inner scope.)  Note that the labels
% of \texttt{\$a} and \texttt{\$p} statements can never be reused after these
% statements appear, because these statements remain active through the end of
% the database.
%
% You might find the reuse of labels a convenient way to have standard names for
% temporary hypotheses, such as \texttt{h1}, \texttt{h2}, etc.  This way you don't have
% to invent unique names for each of them, and in some cases it may be less
% confusing to the reader (although in other cases it might be more confusing, if
% the hypothesis is located far away from the assertion that uses
% it).\footnote{The current implementation requires that all labels, even
% inactive ones, be unique.}

\subsubsection{Frames Revisited}\index{frames and scoping statements}

Now that we have covered scoping, we will look at how an arbitrary
Metamath database can be converted to the simple sequence of extended
frames described on p.~\pageref{framelist}.  This is also how Metamath
stores the database internally when it reads in the database
source.\label{frameconvert} The method is simple.  First, we collect all
constant and variable (\texttt{\$c} and \texttt{\$v}) declarations in
the database, ignoring duplicate declarations of the same variable in
different scopes.  We then put our collected \texttt{\$c} and
\texttt{\$v} declarations at the beginning of the database, so that
their scope is the entire database.  Next, for each assertion in the
database, we determine its frame and extended frame.  The extended frame
is simply the \texttt{\$f}, \texttt{\$e}, and \texttt{\$d} statements
that are active.  The frame is the extended frame with all optional
hypotheses removed.

An equivalent way of saying this is that the extended frame of an assertion
is the collection of all \texttt{\$f}, \texttt{\$e}, and \texttt{\$d} statements
whose scope includes the assertion.
The \texttt{\$f} and \texttt{\$e} statements
occur in the order they appear
(order is irrelevant for \texttt{\$d} statements).

%c%, renaming any
%c%redeclared variables as needed so that all of them have unique names.  (The
%c%exact renaming convention is unimportant.  You might imagine renaming
%c%different declarations of math symbol \texttt{a} as \texttt{a\$1}, \texttt{a\$2}, etc.\
%c%which would prevent any conflicts since \texttt{\$} is not a legal character in a
%c%math symbol token.)

\section{The Anatomy of a Proof} \label{proof}
\index{proof!Metamath, description of}

Each provable assertion (\texttt{\$p}\index{\texttt{\$p} statement} statement) in a
database must include a {\bf proof}\index{proof}.  The proof is located
between the \texttt{\$=}\index{\texttt{\$=} keyword} and \texttt{\$.}\ keywords in the
\texttt{\$p} statement.

In the basic Metamath language\index{basic language}, a proof is a
sequence of statement labels.  This label sequence\index{label sequence}
serves as a set of instructions that the Metamath program uses to
construct a series of math symbol sequences.  The construction must
ultimately result in the math symbol sequence contained between the
\texttt{\$p}\index{\texttt{\$p} statement} and
\texttt{\$=}\index{\texttt{\$=} keyword} keywords of the \texttt{\$p}
statement.  Otherwise, the Metamath program will consider the proof
incorrect, and it will notify you with an appropriate error message when
you ask it to verify the proof.\footnote{To make the loading faster, the
Metamath program does not automatically verify proofs when you
\texttt{read} in a database unless you use the \texttt{/verify}
qualifier.  After a database has been read in, you may use the
\texttt{verify proof *} command to verify proofs.}\index{\texttt{verify
proof} command} Each label in a proof is said to {\bf
reference}\index{label reference} its corresponding statement.

Associated with any assertion\index{assertion} (\texttt{\$p} or
\texttt{\$a}\index{\texttt{\$a} statement} statement) is a set of
hypotheses (\texttt{\$f}\index{\texttt{\$f} statement} or
\texttt{\$e}\index{\texttt{\$e} statement} statements) that are active
with respect to that assertion.  Some are mandatory and the others are
optional.  You should review these concepts if necessary.

Each label\index{label} in a proof must be either the label of a
previous assertion (\texttt{\$a}\index{\texttt{\$a} statement} or
\texttt{\$p}\index{\texttt{\$p} statement} statement) or the label of an
active hypothesis (\texttt{\$e} or \texttt{\$f}\index{\texttt{\$f}
statement} statement) of the \texttt{\$p} statement containing the
proof.  Hypothesis labels may reference both the
mandatory\index{mandatory hypothesis} and the optional hypotheses of the
\texttt{\$p} statement.

The label sequence in a proof specifies a construction in {\bf reverse Polish
notation}\index{reverse Polish notation (RPN)} (RPN).  You may be familiar
with RPN if you have used older
Hewlett--Packard or similar hand-held calculators.
In the calculator analogy, a hypothesis label\index{hypothesis label} is like
a number and an assertion label\index{assertion label} is like an operation
(more precisely, an $n$-ary operation when the
assertion has $n$ \texttt{\$e}-hypotheses).
On an RPN calculator, an operation takes one or more previous numbers in an
input sequence, performs a calculation on them, and replaces those numbers and
itself with the result of the calculation.  For example, the input sequence
$2,3,+$ on an RPN calculator results in $5$, and the input sequence
$2,3,5,{\times},+$ results in $2,15,+$ which results in $17$.

Understanding how RPN is processed involves the concept of a {\bf
stack}\index{stack}\index{RPN stack}, which can be thought of as a set of
temporary memory locations that hold intermediate results.  When Metamath
encounters a hypothesis label it places or {\bf pushes}\index{push} the math
symbol sequence of the hypothesis onto the stack.  When Metamath encounters an
assertion label, it associates the most recent stack entries with the {\em
mandatory} hypotheses\index{mandatory hypothesis} of the assertion, in the
order where the most recent stack entry is associated with the last mandatory
hypothesis of the assertion.  It then determines what
substitutions\index{substitution!variable}\index{variable substitution} have
to be made into the variables of the assertion's mandatory hypotheses to make
them identical to the associated stack entries.  It then makes those same
substitutions into the assertion itself.  Finally, Metamath removes or {\bf
pops}\index{pop} the matched hypotheses from the stack and pushes the
substituted assertion onto the stack.

For the purpose of matching the mandatory hypothesis to the most recent stack
entries, whether a hypothesis is a \texttt{\$e} or \texttt{\$f} statement is
irrelevant.  The only important thing is that a set of
substitutions\footnote{In the Metamath spec (Section~\ref{spec}), we use the
singular term ``substitution'' to refer to the set of substitutions we talk
about here.} exist that allow a match (and if they don't, the proof verifier
will let you know with an error message).  The Metamath language is specified
in such a way that if a set of substitutions exists, it will be unique.
Specifically, the requirement that each variable have a type specified for it
with a \texttt{\$f} statement ensures the uniqueness.

We will illustrate this with an example.
Consider the following Metamath source file:
\begin{verbatim}
$c ( ) -> wff $.
$v p q r s $.
wp $f wff p $.
wq $f wff q $.
wr $f wff r $.
ws $f wff s $.
w2 $a wff ( p -> q ) $.
wnew $p wff ( s -> ( r -> p ) ) $= ws wr wp w2 w2 $.
\end{verbatim}
This Metamath source example shows the definition and ``proof'' (i.e.,
construction) of a well-formed formula (wff)\index{well-formed formula (wff)}
in propositional calculus.  (You may wish to type this example into a file to
experiment with the Metamath program.)  The first two statements declare
(introduce the names of) four constants and four variables.  The next four
statements specify the variable types, namely that
each variable is assumed to be a wff.  Statement \texttt{w2} defines (postulates)
a way to produce a new wff, \texttt{( p -> q )}, from two given wffs \texttt{p} and
\texttt{q}. The mandatory hypotheses of \texttt{w2} are \texttt{wp} and \texttt{wq}.
Statement \texttt{wnew} claims that \texttt{( s -> ( r -> p ) )} is a wff given
three wffs \texttt{s}, \texttt{r}, and \texttt{p}.  More precisely, \texttt{wnew} claims
that the sequence of ten symbols \texttt{wff ( s -> ( r -> p ) )} is provable from
previous assertions and the hypotheses of \texttt{wnew}.  Metamath does not know
or care what a wff is, and as far as it is concerned
the typecode \texttt{wff} is just an
arbitrary constant symbol in a math symbol sequence.  The mandatory hypotheses
of \texttt{wnew} are \texttt{wp}, \texttt{wr}, and \texttt{ws}; \texttt{wq} is an optional
hypothesis.  In our particular proof, the optional hypothesis is not
referenced, but in general, any combination of active (i.e.\ optional and
mandatory) hypotheses could be referenced.  The proof of statement \texttt{wnew}
is the sequence of five labels starting with \texttt{ws} (step~1) and ending with
\texttt{w2} (step~5).

When Metamath verifies the proof, it scans the proof from left to right.  We
will examine what happens at each step of the proof.  The stack starts off
empty.  At step 1, Metamath looks up label \texttt{ws} and determines that it is a
hypothesis, so it pushes the symbol sequence of statement \texttt{ws} onto the
stack:

\begin{center}\begin{tabular}{|l|l|}\hline
{Stack location} & {Contents} \\ \hline \hline
1 & \texttt{wff s} \\ \hline
\end{tabular}\end{center}

Metamath sees that the labels \texttt{wr} and \texttt{wp} in steps~2 and 3 are also
hypotheses, so it pushes them onto the stack.  After step~3, the stack looks
like
this:

\begin{center}\begin{tabular}{|l|l|}\hline
{Stack location} & {Contents} \\ \hline \hline
3 & \texttt{wff p} \\ \hline
2 & \texttt{wff r} \\ \hline
1 & \texttt{wff s} \\ \hline
\end{tabular}\end{center}

At step 4, Metamath sees that label \texttt{w2} is an assertion, so it must do
some processing.  First, it associates the mandatory hypotheses of \texttt{w2},
which are \texttt{wp} and \texttt{wq}, with stack locations~2 and 3, {\em in that
order}. Metamath determines that the only possible way
to make hypothesis \texttt{wp} match (become identical to) stack location~2 and
\texttt{wq} match stack location 3 is to substitute variable \texttt{p} with \texttt{r}
and \texttt{q} with \texttt{p}.  Metamath makes these substitutions into \texttt{w2} and
obtains the symbol sequence \texttt{wff ( r -> p )}.  It removes the hypotheses
from stack locations~2 and 3, then places the result into stack location~2:

\begin{center}\begin{tabular}{|l|l|}\hline
{Stack location} & {Contents} \\ \hline \hline
2 & \texttt{wff ( r -> p )} \\ \hline
1 & \texttt{wff s} \\ \hline
\end{tabular}\end{center}

At step 5, Metamath sees that label \texttt{w2} is an assertion, so it must again
do some processing.  First, it matches the mandatory hypotheses of \texttt{w2},
which are \texttt{wp} and \texttt{wq}, to stack locations 1 and 2.
Metamath determines that the only possible way to make the
hypotheses match is to substitute variable \texttt{p} with \texttt{s} and \texttt{q} with
\texttt{( r -> p )}.  Metamath makes these substitutions into \texttt{w2} and obtains
the symbol
sequence \texttt{wff ( s -> ( r -> p ) )}.  It removes stack
locations 1 and 2, then places the result into stack location~1:

\begin{center}\begin{tabular}{|l|l|}\hline
{Stack location} & {Contents} \\ \hline \hline
1 & \texttt{wff ( s -> ( r -> p ) )} \\ \hline
\end{tabular}\end{center}

After Metamath finishes processing the proof, it checks to see that the
stack contains exactly one element and that this element is
the same as the math symbol sequence in the
\texttt{\$p}\index{\texttt{\$p} statement} statement.  This is the case for our
proof of \texttt{wnew},
so we have proved \texttt{wnew} successfully.  If the result
differs, Metamath will notify you with an error message.  An error message
will also result if the stack contains more than one entry at the end of the
proof, or if the stack did not contain enough entries at any point in the
proof to match all of the mandatory hypotheses\index{mandatory hypothesis} of
an assertion.  Finally, Metamath will notify you with an error message if no
substitution is possible that will make a referenced assertion's hypothesis
match the
stack entries.  You may want to experiment with the different kinds of errors
that Metamath will detect by making some small changes in the proof of our
example.

Metamath's proof notation was designed primarily to express proofs in a
relatively compact manner, not for readability by humans.  Metamath can display
proofs in a number of different ways with the \texttt{show proof}\index{\texttt{show
proof} command} command.  The
\texttt{/lemmon} qualifier displays it in a format that is easier to read when the
proofs are short, and you saw examples of its use in Chapter~\ref{using}.  For
longer proofs, it is useful to see the tree structure of the proof.  A tree
structure is displayed when the \texttt{/lemmon} qualifier is omitted.  You will
probably find this display more convenient as you get used to it. The tree
display of the proof in our example looks like
this:\label{treeproof}\index{tree-style proof}\index{proof!tree-style}
\begin{verbatim}
1     wp=ws    $f wff s
2        wp=wr    $f wff r
3        wq=wp    $f wff p
4     wq=w2    $a wff ( r -> p )
5  wnew=w2  $a wff ( s -> ( r -> p ) )
\end{verbatim}
The number to the left of each line is the step number.  Following it is a
{\bf hypothesis association}\index{hypothesis association}, consisting of two
labels\index{label} separated by \texttt{=}.  To the left of the \texttt{=} (except
in the last step) is the label of a hypothesis of an assertion referenced
later in the proof; here, steps 1 and 4 are the hypothesis associations for
the assertion \texttt{w2} that is referenced in step 5.  A hypothesis association
is indented one level more than the assertion that uses it, so it is easy to
find the corresponding assertion by moving directly down until the indentation
level decreases to one less than where you started from.  To the right of each
\texttt{=} is the proof step label for that proof step.  The statement keyword of
the proof step label is listed next, followed by the content of the top of the
stack (the most recent stack entry) as it exists after that proof step is
processed.  With a little practice, you should have no trouble reading proofs
displayed in this format.

Metamath proofs include the syntax construction of a formula.
In standard mathematics, this kind of
construction is not considered a proper part of the proof at all, and it
certainly becomes rather boring after a while.
Therefore,
by default the \texttt{show proof}\index{\texttt{show proof}
command} command does not show the syntax construction.
Historically \texttt{show proof} command
\textit{did} show the syntax construction, and you needed to add the
\texttt{/essential} option to hide, them, but today
\texttt{/essential} is the default and you need to use
\texttt{/all} to see the syntax constructions.

When verifying a proof, Metamath will check that no mandatory
\texttt{\$d}\index{\texttt{\$d} statement}\index{mandatory \texttt{\$d}
statement} statement of an assertion referenced in a proof is violated
when substitutions\index{substitution!variable}\index{variable
substitution} are made to the variables in the assertion.  For details
see Section~\ref{spec4} or \ref{dollard}.

\subsection{The Concept of Unification} \label{unify}

During the course of verifying a proof, when Metamath\index{Metamath}
encounters an assertion label\index{assertion label}, it associates the
mandatory hypotheses\index{mandatory hypothesis} of the assertion with the top
entries of the RPN stack\index{stack}\index{RPN stack}.  Metamath then
determines what substitutions\index{substitution!variable}\index{variable
substitution} it must make to the variables in the assertion's mandatory
hypotheses in order for these hypotheses to become identical to their
corresponding stack entries.  This process is called {\bf
unification}\index{unification}.  (We also informally use the term
``unification'' to refer to a set of substitutions that results from the
process, as in ``two unifications are possible.'')  After the substitutions
are made, the hypotheses are said to be {\bf unified}.

If no such substitutions are possible, Metamath will consider the proof
incorrect and notify you with an error message.
% (deleted 3/10/07, per suggestion of Mel O'Cat:)
% The syntax of the
% Metamath language ensures that if a set of substitutions exists, it
% will be unique.

The general algorithm for unification described in the literature is
somewhat complex.
However, in the case of Metamath it is intentionally trivial.
Mandatory hypotheses must be
pushed on the proof stack in the order in which they appear.
In addition, each variable must have its type specified
with a \texttt{\$f} hypothesis before it is used
and that each \texttt{\$f} hypothesis
have the restricted syntax of a typecode (a constant) followed by a variable.
The typecode in the \texttt{\$f} hypothesis must match the first symbol of
the corresponding RPN stack entry (which will also be a constant), so
the only possible match for the variable in the \texttt{\$f} hypothesis is
the sequence of symbols in the stack entry after the initial constant.

In the Proof Assistant\index{Proof Assistant}, a more general unification
algorithm is used.  While a proof is being developed, sometimes not enough
information is available to determine a unique unification.  In this case
Metamath will ask you to pick the correct one.\index{ambiguous
unification}\index{unification!ambiguous}

\section{Extensions to the Metamath Language}\index{extended
language}

\subsection{Comments in the Metamath Language}\label{comments}
\index{markup notation}
\index{comments!markup notation}

The commenting feature allows you to annotate the contents of
a database.  Just as with most
computer languages, comments are ignored for the purpose of interpreting the
contents of the database. Comments effectively act as
additional white space\index{white
space} between tokens
when a database is parsed.

A comment may be placed at the beginning, end, or
between any two tokens\index{token} in a source file.

Comments have the following syntax:
\begin{center}
 \texttt{\$(} {\em text} \texttt{\$)}
\end{center}
Here,\index{\texttt{\$(} and \texttt{\$)} auxiliary
keywords}\index{comment} {\em text} is a string, possibly empty, of any
characters in Metamath's character set (p.~\pageref{spec1chars}), except
that the character strings \texttt{\$(} and \texttt{\$)} may not appear
in {\em text}.  Thus nested comments are not
permitted:\footnote{Computer languages have differing standards for
nested comments, and rather than picking one it was felt simplest not to
allow them at all, at least in the current version of
Metamath\index{Metamath!limitations of version 0.07.30}.} Metamath will
complain if you give it
\begin{center}
 \texttt{\$( This is a \$( nested \$) comment.\ \$)}
\end{center}
To compensate for this non-nesting behavior, I often change all \texttt{\$}'s
to \texttt{@}'s in sections of Metamath code I wish to comment out.

The Metamath program supports a number of markup mechanisms and conventions
to generate good-looking results in \LaTeX\ and {\sc html},
as discussed below.
These markup features have to do only with how the comments are typeset,
and have no effect on how Metamath verifies the proofs in the database.
The improper
use of them may result in incorrectly typeset output, but no Metamath
error messages will result during the \texttt{read} and \texttt{verify
proof} commands.  (However, the \texttt{write
theorem\texttt{\char`\_}list} command
will check for markup errors as a side-effect of its
{\sc html} generation.)
Section~\ref{texout} has instructions for creating \LaTeX\ output, and
section~\ref{htmlout} has instructions for creating
{\sc html}\index{HTML} output.

\subsubsection{Headings}\label{commentheadings}

If the \texttt{\$(} is immediately followed by a new line
starting with a heading marker, it is a header.
This can start with:

\begin{itemize}
 \item[] \texttt{\#\#\#\#} - major part header
 \item[] \texttt{\#*\#*} - section header
 \item[] \texttt{=-=-} - subsection header
 \item[] \texttt{-.-.} - subsubsection header
\end{itemize}

The line following the marker line
will be used for the table of contents entry, after trimming spaces.
The next line should be another (closing) matching marker line.
Any text after that
but before the closing \texttt{\$}, such as an extended description of the
section, will be included on the \texttt{mmtheoremsNNN.html} page.

For more information, run
\texttt{help write theorem\char`\_list}.

\subsubsection{Math mode}
\label{mathcomments}
\index{\texttt{`} inside comments}
\index{\texttt{\char`\~} inside comments}
\index{math mode}

Inside of comments, a string of tokens\index{token} enclosed in
grave accents\index{grave accent (\texttt{`})} (\texttt{`}) will be converted
to standard mathematical symbols during
{\sc HTML}\index{HTML} or \LaTeX\ output
typesetting,\index{latex@{\LaTeX}} according to the information in the
special \texttt{\$t}\index{\texttt{\$t} comment}\index{typesetting
comment} comment in the database
(see section~\ref{tcomment} for information about the typesetting
comment, and Appendix~\ref{ASCII} to see examples of its results).

The first grave accent\index{grave accent (\texttt{`})} \texttt{`}
causes the output processor to enter {\bf math mode}\index{math mode}
and the second one exits it.
In this
mode, the characters following the \texttt{`} are interpreted as a
sequence of math symbol tokens separated by white space\index{white
space}.  The tokens are looked up in the \texttt{\$t}
comment\index{\texttt{\$t} comment}\index{typesetting comment} and if
found, they will be replaced by the standard mathematical symbols that
they correspond to before being placed in the typeset output file.  If
not found, the symbol will be output as is and a warning will be issued.
The tokens do not have to be active in the database, although a warning
will be issued if they are not declared with \texttt{\$c} or
\texttt{\$v} statements.

Two consecutive
grave accents \texttt{``} are treated as a single actual grave accent
(both inside and outside of math mode) and will not cause the output
processor to enter or exit math mode.

Here is an example of its use\index{Pierce's axiom}:
\begin{center}
\texttt{\$( Pierce's axiom, ` ( ( ph -> ps ) -> ph ) -> ph ` ,\\
         is not very intuitive. \$)}
\end{center}
becomes
\begin{center}
   \texttt{\$(} Pierce's axiom, $((\varphi \rightarrow \psi)\rightarrow
\varphi)\rightarrow \varphi$, is not very intuitive. \texttt{\$)}
\end{center}

Note that the math symbol tokens\index{token} must be surrounded by white
space\index{white space}.
%, since there is no context that allows ambiguity to be
%resolved, as is the case with math symbol sequences in some of the Metamath
%statements.
White space should also surround the \texttt{`}
delimiters.

The math mode feature also gives you a quick and easy way to generate
text containing mathematical symbols, independently of the intended
purpose of Metamath.\index{Metamath!using as a math editor} To do this,
simply create your text with grave accents surrounding your formulas,
after making sure that your math symbols are mapped to \LaTeX\ symbols
as described in Appendix~\ref{ASCII}.  It is easier if you start with a
database with predefined symbols such as \texttt{set.mm}.  Use your
grave-quoted math string to replace an existing comment, then typeset
the statement corresponding to that comment following the instructions
from the \texttt{help tex} command in the Metamath program.  You will
then probably want to edit the resulting file with a text editor to fine
tune it to your exact needs.

\subsubsection{Label Mode}\index{label mode}

Outside of math mode, a tilde\index{tilde (\texttt{\char`\~})} \verb/~/
indicates to Metamath's\index{Metamath} output processor that the
token\index{token} that follows (i.e.\ the characters up to the next
white space\index{white space}) represents a statement label or URL.
This formatting mode is called {\bf label mode}\index{label mode}.
If a literal tilde
is desired (outside of math mode) instead of label mode,
use two tildes in a row to represent it.

When generating a \LaTeX\ output file,
the following token will be formatted in \texttt{typewriter}
font, and the tilde removed, to make it stand out from the rest of the text.
This formatting will be applied to all characters after the
tilde up to the first white space\index{white space}.
Whether
or not the token is an actual statement label is not checked, and the
token does not have to have the correct syntax for a label; no error
messages will be produced.  The only effect of the label mode on the
output is that typewriter font will be used for the tokens that are
placed in the \LaTeX\ output file.

When generating {\sc html},
the tokens after the tilde {\em must} be a URL (either http: or https:)
or a valid label.
Error messages will be issued during that output if they aren't.
A hyperlink will be generated to that URL or label.

\subsubsection{Link to bibliographical reference}\index{citation}%
\index{link to bibliographical reference}

Bibliographical references are handled specially when generating
{\sc html} if formatted specially.
Text in the form \texttt{[}{\em author}\texttt{]}
is considered a link to a bibliographical reference.
See \texttt{help html} and \texttt{help write
bibliography} in the Metamath program for more
information.
% \index{\texttt{\char`\[}\ldots\texttt{]} inside comments}
See also Sections~\ref{tcomment} and \ref{wrbib}.

The \texttt{[}{\em author}\texttt{]} notation will also create an entry in
the bibliography cross-reference file generated by \texttt{write
bibliography} (Section~\ref{wrbib}) for {\sc HTML}.
For this to work properly, the
surrounding comment must be formatted as follows:
\begin{quote}
    {\em keyword} {\em label} {\em noise-word}
     \texttt{[}{\em author}\texttt{] p.} {\em number}
\end{quote}
for example
\begin{verbatim}
     Theorem 5.2 of [Monk] p. 223
\end{verbatim}
The {\em keyword} is not case sensitive and must be one of the following:
\begin{verbatim}
     theorem lemma definition compare proposition corollary
     axiom rule remark exercise problem notation example
     property figure postulate equation scheme chapter
\end{verbatim}
The optional {\em label} may consist of more than one
(non-{\em keyword} and non-{\em noise-word}) word.
The optional {\em noise-word} is one of:
\begin{verbatim}
     of in from on
\end{verbatim}
and is  ignored when the cross-reference file is created.  The
\texttt{write
biblio\-graphy} command will perform error checking to verify the
above format.\index{error checking}

\subsubsection{Parentheticals}\label{parentheticals}

The end of a comment may include one or more parenthicals, that is,
statements enclosed in parentheses.
The Metamath program looks for certain parentheticals and can issue
warnings based on them.
They are:

\begin{itemize}
 \item[] \texttt{(Contributed by }
   \textit{NAME}\texttt{,} \textit{DATE}\texttt{.)} -
   document the original contributor's name and the date it was created.
 \item[] \texttt{(Revised by }
   \textit{NAME}\texttt{,} \textit{DATE}\texttt{.)} -
   document the contributor's name and creation date
   that resulted in significant revision
   (not just an automated minimization or shortening).
 \item[] \texttt{(Proof shortened by }
   \textit{NAME}\texttt{,} \textit{DATE}\texttt{.)} -
   document the contributor's name and date that developed a significant
   shortening of the proof (not just an automated minimization).
 \item[] \texttt{(Proof modification is discouraged.)} -
   Note that this proof should normally not be modified.
 \item[] \texttt{(New usage is discouraged.)} -
   Note that this assertion should normally not be used.
\end{itemize}

The \textit{DATE} must be in form YYYY-MMM-DD, where MMM is the
English abbreviation of that month.

\subsubsection{Other markup}\label{othermarkup}
\index{markup notation}

There are other markup notations for generating good-looking results
beyond math mode and label mode:

\begin{itemize}
 \item[]
         \texttt{\char`\_} (underscore)\index{\texttt{\char`\_} inside comments} -
             Italicize text starting from
              {\em space}\texttt{\char`\_}{\em non-space} (i.e.\ \texttt{\char`\_}
              with a space before it and a non-space character after it) until
             the next
             {\em non-space}\texttt{\char`\_}{\em space}.  Normal
             punctuation (e.g.\ a trailing
             comma or period) is ignored when determining {\em space}.
 \item[]
         \texttt{\char`\_} (underscore) - {\em
         non-space}\texttt{\char`\_}{\em non-space-string}, where
          {\em non-space-string} is a string of non-space characters,
         will make {\em non-space-string} become a subscript.
 \item[]
         \texttt{<HTML>}...\texttt{</HTML>} - do not convert
         ``\texttt{<}'' and ``\texttt{>}''
         in the enclosed text when generating {\sc HTML},
         otherwise process markup normally. This allows direct insertion
         of {\sc html} commands.
 \item[]
       ``\texttt{\&}ref\texttt{;}'' - insert an {\sc HTML}
         character reference.
         This is how to insert arbitrary Unicode characters
         (such as accented characters).  Currently only directly supported
         when generating {\sc HTML}.
\end{itemize}

It is recommended that spaces surround any \texttt{\char`\~} and
\texttt{`} tokens in the comment and that a space follow the {\em label}
after a \texttt{\char`\~} token.  This will make global substitutions
to change labels and symbol names much easier and also eliminate any
future chance of ambiguity.  Spaces around these tokens are automatically
removed in the final output to conform with normal rules of punctuation;
for example, a space between a trailing \texttt{`} and a left parenthesis
will be removed.

A good way to become familiar with the markup notation is to look at
the extensive examples in the \texttt{set.mm} database.

\subsection{The Typesetting Comment (\texttt{\$t})}\label{tcomment}

The typesetting comment \texttt{\$t} in the input database file
provides the information necessary to produce good-looking results.
It provides \LaTeX\ and {\sc html}
definitions for math symbols,
as well supporting as some
customization of the generated web page.
If you add a new token to a database, you should also
update the \texttt{\$t} comment information if you want to eventually
create output in \LaTeX\ or {\sc HTML}.
See the
\texttt{set.mm}\index{set theory database (\texttt{set.mm})} database
file for an extensive example of a \texttt{\$t} comment illustrating
many of the features described below.

Programs that do not need to generate good-looking presentation results,
such as programs that only verify Metamath databases,
can completely ignore typesetting comments
and just treat them as normal comments.
Even the Metamath program only consults the
\texttt{\$t} comment information when it needs to generate typeset output
in \LaTeX\ or {\sc HTML}
(e.g., when you open a \LaTeX\ output file with the \texttt{open tex} command).

We will first discuss the syntax of typesetting comments, and then
briefly discuss how this can be used within the Metamath program.

\subsubsection{Typesetting Comment Syntax Overview}

The typesetting comment is identified by the token
\texttt{\$t}\index{\texttt{\$t} comment}\index{typesetting comment} in
the comment, and the typesetting comment ends at the matching
\texttt{\$)}:
\[
  \mbox{\tt \$(\ }
  \mbox{\tt \$t\ }
  \underbrace{
    \mbox{\tt \ \ \ \ \ \ \ \ \ \ \ }
    \cdots
    \mbox{\tt \ \ \ \ \ \ \ \ \ \ \ }
  }_{\mbox{Typesetting definitions go here}}
  \mbox{\tt \ \$)}
\]

There must be one or more white space characters, and only white space
characters, between the \texttt{\$(} that starts the comment
and the \texttt{\$t} symbol,
and the \texttt{\$t} must be followed by one
or more white space characters
(see section \ref{whitespace} for the definition of white space characters).
The typesetting comment continues until the comment end token \texttt{\$)}
(which must be preceded by one or more white space characters).

In version 0.07.30\index{Metamath!limitations of version 0.07.30} of the
Metamath program, there may be only one \texttt{\$t} comment in a
database.  This restriction may be lifted in the future to allow
many \texttt{\$t} comments in a database.

Between the \texttt{\$t} symbol (and its following white space) and the
comment end token \texttt{\$)} (and its preceding white space)
is a sequence of one or more typesetting definitions, where
each definition has the form
\textit{definition-type arg arg ... ;}.
Each of the zero or more \textit{arg} values
can be either a typesetting data or a keyword
(what keywords are allowed, and where, depends on the specific
\textit{definition-type}).
The \textit{definition-type}, and each argument \textit{arg},
are separated by one or more white space characters.
Every definition ends in an unquoted semicolon;
white space is not required before the terminating semicolon of a definition.
Each definition should start on a new line.\footnote{This
restriction of the current version of Metamath
(0.07.30)\index{Metamath!limitations of version 0.07.30} may be removed
in a future version, but you should do it anyway for readability.}

For example, this typesetting definition:
\begin{center}
 \verb$latexdef "C_" as "\subseteq";$
\end{center}
defines the token \verb$C_$ as the \LaTeX\ symbol $\subseteq$ (which means
``subset'').

Typesetting data is a sequence of one or more quoted strings
(if there is more than one, they are connected by \texttt{\char`\+}).
Often a single quoted string is used to provide data for a definition, using
either double (\texttt{\char`\"}) or single (\texttt{'}) quotation marks.
However,
{\em a quoted string (enclosed in quotation marks) may not include
line breaks.}
A quoted string
may include a quotation mark that matches the enclosing quotes by repeating
the quotation mark twice.  Here are some examples:

\begin{tabu}   { l l }
\textbf{Example} & \textbf{Meaning} \\
\texttt{\char`\"a\char`\"\char`\"b\char`\"} & \texttt{a\char`\"b} \\
\texttt{'c''d'} & \texttt{c'd} \\
\texttt{\char`\"e''f\char`\"} & \texttt{e''f} \\
\texttt{'g\char`\"\char`\"h'} & \texttt{g\char`\"\char`\"h} \\
\end{tabu}

Finally, a long quoted string
may be broken up into multiple quoted strings (considered, as a whole,
a single quoted string) and joined with \texttt{\char`\+}.
You can even use multiple lines as long as a
'+' is at the end of every line except the last one.
The \texttt{\char`\+} should be preceded and followed by at least one
white space character.
Thus, for example,
\begin{center}
 \texttt{\char`\"ab\char`\"\ \char`\+\ \char`\"cd\char`\"
    \ \char`\+\ \\ 'ef'}
\end{center}
is the same as
\begin{center}
 \texttt{\char`\"abcdef\char`\"}
\end{center}

{\sc c}-style comments \texttt{/*}\ldots\texttt{*/} are also supported.

In practice, whenever you add a new math token you will often want to add
typesetting definitions using
\texttt{latexdef}, \texttt{htmldef}, and
\texttt{althtmldef}, as described below.
That way, they will all be up to date.
Of course, whether or not you want to use all three definitions will
depend on how the database is intended to be used.

Below we discuss the different possible \textit{definition-kind} options.
We will show data surrounded by double quotes (in practice they can also use
single quotes and/or be a sequence joined by \texttt{+}s).
We will use specific names for the \textit{data} to make clear what
the data is used for, such as
{\em math-token} (for a Metamath math token,
{\em latex-string} (for string to be placed in a \LaTeX\ stream),
{\em {\sc html}-code} (for {\sc html} code),
and {\em filename} (for a filename).

\subsubsection{Typesetting Comment - \LaTeX}

The syntax for a \LaTeX\ definition is:
\begin{center}
 \texttt{latexdef "}{\em math-token}\texttt{" as "}{\em latex-string}\texttt{";}
\end{center}
\index{latex definitions@\LaTeX\ definitions}%
\index{\texttt{latexdef} statement}

The {\em token-string} and {\em latex-string} are the data
(character strings) for
the token and the \LaTeX\ definition of the token, respectively,

These \LaTeX\ definitions are used by the Metamath program
when it is asked to product \LaTeX output using
the \texttt{write tex} command.

\subsubsection{Typesetting Comment - {\sc html}}

The key kinds of {\sc HTML} definitions have the following syntax:

\vskip 1ex
    \texttt{htmldef "}{\em math-token}\texttt{" as "}{\em
    {\sc html}-code}\texttt{";}\index{\texttt{htmldef} statement}
                    \ \ \ \ \ \ldots

    \texttt{althtmldef "}{\em math-token}\texttt{" as "}{\em
{\sc html}-code}\texttt{";}\index{\texttt{althtmldef} statement}

                    \ \ \ \ \ \ldots

Note that in {\sc HTML} there are two possible definitions for math tokens.
This feature is useful when
an alternate representation of symbols is desired, for example one that
uses Unicode entities and another uses {\sc gif} images.

There are many other typesetting definitions that can control {\sc HTML}.
These include:

\vskip 1ex

    \texttt{htmldef "}{\em math-token}\texttt{" as "}{\em {\sc
    html}-code}\texttt{";}

    \texttt{htmltitle "}{\em {\sc html}-code}\texttt{";}%
\index{\texttt{htmltitle} statement}

    \texttt{htmlhome "}{\em {\sc html}-code}\texttt{";}%
\index{\texttt{htmlhome} statement}

    \texttt{htmlvarcolor "}{\em {\sc html}-code}\texttt{";}%
\index{\texttt{htmlvarcolor} statement}

    \texttt{htmlbibliography "}{\em filename}\texttt{";}%
\index{\texttt{htmlbibliography} statement}

\vskip 1ex

\noindent The \texttt{htmltitle} is the {\sc html} code for a common
title, such as ``Metamath Proof Explorer.''  The \texttt{htmlhome} is
code for a link back to the home page.  The \texttt{htmlvarcolor} is
code for a color key that appears at the bottom of each proof.  The file
specified by {\em filename} is an {\sc html} file that is assumed to
have a \texttt{<A NAME=}\ldots\texttt{>} tag for each bibiographic
reference in the database comments.  For example, if
\texttt{[Monk]}\index{\texttt{\char`\[}\ldots\texttt{]} inside comments}
occurs in the comment for a theorem, then \texttt{<A NAME='Monk'>} must
be present in the file; if not, a warning message is given.

Associated with
\texttt{althtmldef}
are the statements
\vskip 1ex

    \texttt{htmldir "}{\em
      directoryname}\texttt{";}\index{\texttt{htmldir} statement}

    \texttt{althtmldir "}{\em
     directoryname}\texttt{";}\index{\texttt{althtmldir} statement}

\vskip 1ex
\noindent giving the directories of the {\sc gif} and Unicode versions
respectively; their purpose is to provide cross-linking between the
two versions in the generated web pages.

When two different types of pages need to be produced from a single
database, such as the Hilbert Space Explorer that extends the Metamath
Proof Explorer, ``extended'' variables may be declared in the
\texttt{\$t} comment:
\vskip 1ex

    \texttt{exthtmltitle "}{\em {\sc html}-code}\texttt{";}%
\index{\texttt{exthtmltitle} statement}

    \texttt{exthtmlhome "}{\em {\sc html}-code}\texttt{";}%
\index{\texttt{exthtmlhome} statement}

    \texttt{exthtmlbibliography "}{\em filename}\texttt{";}%
\index{\texttt{exthtmlbibliography} statement}

\vskip 1ex
\noindent When these are declared, you also must declare
\vskip 1ex

    \texttt{exthtmllabel "}{\em label}\texttt{";}%
\index{\texttt{exthtmllabel} statement}

\vskip 1ex \noindent that identifies the database statement where the
``extended'' section of the database starts (in our example, where the
Hilbert Space Explorer starts).  During the generation of web pages for
that starting statement and the statements after it, the {\sc html} code
assigned to \texttt{exthtmltitle} and \texttt{exthtmlhome} is used
instead of that assigned to \texttt{htmltitle} and \texttt{htmlhome},
respectively.

\begin{sloppy}
\subsection{Additional Information Com\-ment (\texttt{\$j})} \label{jcomment}
\end{sloppy}

The additional information comment, aka the
\texttt{\$j}\index{\texttt{\$j} comment}\index{additional information comment}
comment,
provides a way to add additional structured information that can
be optionally parsed by systems.

The additional information comment is parsed the same way as the
typesetting comment (\texttt{\$t}) (see section \ref{tcomment}).
That is,
the additional information comment begins with the token
\texttt{\$j} within a comment,
and continues until the comment close \texttt{\$)}.
Within an additional information comment is a sequence of one or more
commands of the form \texttt{command arg arg ... ;}
where each of the zero or more \texttt{arg} values
can be either a quoted string or a keyword.
Note that every command ends in an unquoted semicolon.
If a verifier is parsing an additional information comment, but
doesn't recognize a particular command, it must skip the command
by finding the end of the command (an unquoted semicolon).

A database may have 0 or more additional information comments.
Note, however, that a verifier may ignore these comments entirely or only
process certain commands in an additional information comment.
The \texttt{mmj2} verifier supports many commands in additional information
comments.
We encourage systems that process additional information comments
to coordinate so that they will use the same command for the same effect.

Examples of additional information comments with various commands
(from the \texttt{set.mm} database) are:

\begin{itemize}
   \item Define the syntax and logical typecodes,
     and declare that our grammar is
     unambiguous (verifiable using the KLR parser, with compositing depth 5).
\begin{verbatim}
  $( $j
    syntax 'wff';
    syntax '|-' as 'wff';
    unambiguous 'klr 5';
  $)
\end{verbatim}

   \item Register $\lnot$ and $\rightarrow$ as primitive expressions
           (lacking definitions).
\begin{verbatim}
  $( $j primitive 'wn' 'wi'; $)
\end{verbatim}

   \item There is a special justification for \texttt{df-bi}.
\begin{verbatim}
  $( $j justification 'bijust' for 'df-bi'; $)
\end{verbatim}

   \item Register $\leftrightarrow$ as an equality for its type (wff).
\begin{verbatim}
  $( $j
    equality 'wb' from 'biid' 'bicomi' 'bitri';
    definition 'dfbi1' for 'wb';
  $)
\end{verbatim}

   \item Theorem \texttt{notbii} is the congruence law for negation.
\begin{verbatim}
  $( $j congruence 'notbii'; $)
\end{verbatim}

   \item Add \texttt{setvar} as a typecode.
\begin{verbatim}
  $( $j syntax 'setvar'; $)
\end{verbatim}

   \item Register $=$ as an equality for its type (\texttt{class}).
\begin{verbatim}
  $( $j equality 'wceq' from 'eqid' 'eqcomi' 'eqtri'; $)
\end{verbatim}

\end{itemize}


\subsection{Including Other Files in a Metamath Source File} \label{include}
\index{\texttt{\$[} and \texttt{\$]} auxiliary keywords}

The keywords \texttt{\$[} and \texttt{\$]} specify a file to be
included\index{included file}\index{file inclusion} at that point in a
Metamath\index{Metamath} source file\index{source file}.  The syntax for
including a file is as follows:
\begin{center}
\texttt{\$[} {\em file-name} \texttt{\$]}
\end{center}

The {\em file-name} should be a single token\index{token} with the same syntax
as a math symbol (i.e., all 93 non-whitespace
printable characters other than \texttt{\$} are
allowed, subject to the file-naming limitations of your operating system).
Comments may appear between the \texttt{\$[} and \texttt{\$]} keywords.  Included
files may include other files, which may in turn include other files, and so
on.

For example, suppose you want to use the set theory database as the starting
point for your own theory.  The first line in your file could be
\begin{center}
\texttt{\$[ set.mm \$]}
\end{center} All of the information (axioms, theorems,
etc.) in \texttt{set.mm} and any files that {\em it} includes will become
available for you to reference in your file. This can help make your work more
modular. A drawback to including files is that if you change the name of a
symbol or the label of a statement, you must also remember to update any
references in any file that includes it.


The naming conventions for included files are the same as those of your
operating system.\footnote{On the Macintosh, prior to Mac OS X,
 a colon is used to separate disk
and folder names from your file name.  For example, {\em volume}\texttt{:}{\em
file-name} refers to the root directory, {\em volume}\texttt{:}{\em
folder-name}\texttt{:}{\em file-name} refers to a folder in root, and {\em
volume}\texttt{:}{\em folder-name}\texttt{:}\ldots\texttt{:}{\em file-name} refers to a
deeper folder.  A simple {\em file-name} refers to a file in the folder from
which you launch the Metamath application.  Under Mac OS X and later,
the Metamath program is run under the Terminal application, which
conforms to Unix naming conventions.}\index{Macintosh file
names}\index{file names!Macintosh}\label{includef} For compatibility among
operating systems, you should keep the file names as simple as possible.  A
good convention to use is {\em file}\texttt{.mm} where {\em file} is eight
characters or less, in lower case.

There is no limit to the nesting depth of included files.  One thing that you
should be aware of is that if two included files themselves include a common
third file, only the {\em first} reference to this common file will be read
in.  This allows you to include two or more files that build on a common
starting file without having to worry about label and symbol conflicts that
would occur if the common file were read in more than once.  (In fact, if a
file includes itself, the self-reference will be ignored, although of course
it would not make any sense to do that.)  This feature also means, however,
that if you try to include a common file in several inner blocks, the result
might not be what you expect, since only the first reference will be replaced
with the included file (unlike the include statement in most other computer
languages).  Thus you would normally include common files only in the
outermost block\index{outermost block}.

\subsection{Compressed Proof Format}\label{compressed1}\index{compressed
proof}\index{proof!compressed}

The proof notation presented in Section~\ref{proof} is called a
{\bf normal proof}\index{normal proof}\index{proof!normal} and in principle is
sufficient to express any proof.  However, proofs often contain steps and
subproofs that are identical.  This is particularly true in typical
Metamath\index{Metamath} applications, because Metamath requires that the math
symbol sequence (usually containing a formula) at each step be separately
constructed, that is, built up piece by piece. As a result, a lot of
repetition often results.  The {\bf compressed proof} format allows Metamath
to take advantage of this redundancy to shorten proofs.

The specification for the compressed proof format is given in
Appen\-dix~\ref{compressed}.

Normally you need not concern yourself with the details of the compressed
proof format, since the Metamath program will allow you to convert from
the normal format to the compressed format with ease, and will also
automatically convert from the compressed format when proofs are displayed.
The overall structure of the compressed format is as follows:
\begin{center}
  \texttt{\$= ( } {\em label-list} \texttt{) } {\em compressed-proof\ }\ \texttt{\$.}
\end{center}
\index{\texttt{\$=} keyword}
The first \texttt{(} serves as a flag to Metamath that a compressed proof
follows.  The {\em label-list} includes all statements referred to by the
proof except the mandatory hypotheses\index{mandatory hypothesis}.  The {\em
compressed-proof} is a compact encoding of the proof, using upper-case
letters, and can be thought of as a large integer in base 26.  White
space\index{white space} inside a {\em compressed-proof} is
optional and is ignored.

It is important to note that the order of the mandatory hypotheses of
the statement being proved must not be changed if the compressed proof
format is used, otherwise the proof will become incorrect.  The reason
for this is that the mandatory hypotheses are not mentioned explicitly
in the compressed proof in order to make the compression more efficient.
If you wish to change the order of mandatory hypotheses, you must first
convert the proof back to normal format using the \texttt{save proof
{\em statement} /normal}\index{\texttt{save proof} command} command.
Later, you can go back to compressed format with \texttt{save proof {\em
statement} /compressed}.

During error checking with the \texttt{verify proof} command, an error
found in a compressed proof may point to a character in {\em
compressed-proof}, which may not be very meaningful to you.  In this
case, try to \texttt{save proof /normal} first, then do the
\texttt{verify proof} again.  In general, it is best to make sure a
proof is correct before saving it in compressed format, because severe
errors are less likely to be recoverable than in normal format.

\subsection{Specifying Unknown Proofs or Subproofs}\label{unknown}

In a proof under development, any step or subproof that is not yet known
may be represented with a single \texttt{?}.  For the purposes of
parsing the proof, the \texttt{?}\ \index{\texttt{]}@\texttt{?}\ inside
proofs} will push a single entry onto the RPN stack just as if it were a
hypothesis.  While developing a proof with the Proof
Assistant\index{Proof Assistant}, a partially developed proof may be
saved with the \texttt{save new{\char`\_}proof}\index{\texttt{save
new{\char`\_}proof} command} command, and \texttt{?}'s will be placed at
the appropriate places.

All \texttt{\$p}\index{\texttt{\$p} statement} statements must have
proofs, even if they are entirely unknown.  Before creating a proof with
the Proof Assistant, you should specify a completely unknown proof as
follows:
\begin{center}
  {\em label} \texttt{\$p} {\em statement} \texttt{\$= ?\ \$.}
\end{center}
\index{\texttt{\$=} keyword}
\index{\texttt{]}@\texttt{?}\ inside proofs}

The \texttt{verify proof}\index{\texttt{verify proof} command} command
will check the known portions of a partial proof for errors, but will
warn you that the statement has not been proved.

Note that partially developed proofs may be saved in compressed format
if desired.  In this case, you will see one or more \texttt{?}'s in the
{\em compressed-proof} part.\index{compressed
proof}\index{proof!compressed}

\section{Axioms vs.\ Definitions}\label{definitions}

The \textit{basic}
Metamath\index{Metamath} language and program
make no distinction\index{axiom vs.\
definition} between axioms\index{axiom} and
definitions.\index{definition} The \texttt{\$a}\index{\texttt{\$a}
statement} statement is used for both.  At first, this may seem
puzzling.  In the minds of many mathematicians, the distinction is
clear, even obvious, and hardly worth discussing.  A definition is
considered to be merely an abbreviation that can be replaced by the
expression for which it stands; although unless one actually does this,
to be precise then one should say that a theorem\index{theorem} is a
consequence of the axioms {\em and} the definitions that are used in the
formulation of the theorem \cite[p.~20]{Behnke}.\index{Behnke, H.}

\subsection{What is a Definition?}

What is a definition?  In its simplest form, a definition introduces a new
symbol and provides an unambiguous rule to transform an expression containing
the new symbol to one without it.  The concept of a ``proper
definition''\index{proper definition}\index{definition!proper} (as opposed to
a creative definition)\index{creative definition}\index{definition!creative}
that is usually agreed upon is (1) the definition should not strengthen the
language and (2) any symbols introduced by the definition should be eliminable
from the language \cite{Nemesszeghy}\index{Nemesszeghy, E. Z.}.  In other
words, they are mere typographical conveniences that do not belong to the
system and are theoretically superfluous.  This may seem obvious, but in fact
the nature of definitions can be subtle, sometimes requiring difficult
metatheorems to establish that they are not creative.

A more conservative stance was taken by logician S.
Le\'{s}niewski.\index{Le\'{s}niewski, S.}
\begin{quote}
Le\'{s}niewski
regards definitions as theses of the system.  In this respect they do
not differ either from the axioms or from theorems, i.e.\ from the
theses added to the system on the basis of the rule of substitution or
the rule of detachment [modus ponens].  Once definitions have been
accepted as theses of the system, it becomes necessary to consider them
as true propositions in the same sense in which axioms are true
\cite{Lejewski}.
\end{quote}\index{Lejewski, Czeslaw}

Let us look at some simple examples of definitions in propositional
calculus.  Consider the definition of logical {\sc or}
(disjunction):\index{disjunction ($\vee$)} ``$P\vee Q$ denotes $\neg P
\rightarrow Q$ (not $P$ implies $Q$).''  It is very easy to recognize a
statement making use of this definition, because it introduces the new
symbol $\vee$ that did not previously exist in the language.  It is easy
to see that no new theorems of the original language will result from
this definition.

Next, consider a definition that eliminates parentheses:  ``$P
\rightarrow Q\rightarrow R$ denotes $P\rightarrow (Q \rightarrow R)$.''
This is more subtle, because no new symbols are introduced.  The reason
this definition is considered proper is that no new symbol sequences
that are valid wffs (well-formed formulas)\index{well-formed formula
(wff)} in the original language will result from the definition, since
``$P \rightarrow Q\rightarrow R$'' is not a wff in the original
language.  Here, we implicitly make use of the fact that there is a
decision procedure that allows us to determine whether or not a symbol
sequence is a wff, and this fact allows us to use symbol sequences that
are not wffs to represent other things (such as wffs) by means of the
definition.  However, to justify the definition as not being creative we
need to prove that ``$P \rightarrow Q\rightarrow R$'' is in fact not a
wff in the original language, and this is more difficult than in the
case where we simply introduce a new symbol.

%Now let's take this reasoning to an extreme.  Propositional calculus is a
%decidable theory,\footnote{This means that a mechanical algorithm exists to
%determine whether or not a wff is a theorem.} so in principle we could make use
%of symbol sequences that are not theorems to represent other things (say, to
%encode actual theorems in a more compact way).  For example, let us extend the
%language by defining a wff ``$P$'' in the extended language as the theorem
%``$P\rightarrow P$''\footnote{This is one of the first theorems proved in the
%Metamath database \texttt{set.mm}.}\index{set
%theory database (\texttt{set.mm})} in the original language whenever ``$P$'' is
%not a theorem in the original language.  In the extended language, any wff
%``$Q$'' thus represents a theorem; to find out what theorem (in the original
%language) ``$Q$'' represents, we determine whether ``$Q$'' is a theorem in the
%original language (before the definition was introduced).  If so, we're done; if
%not, we replace ``$Q$'' by ``$Q\rightarrow Q$'' to eliminate the definition.
%This definition is therefore eliminable, and it does not ``strengthen'' the
%language because any wff that is not a theorem is not in the set of statements
%provable in the original language and thus is available for use by definitions.
%
%Of course, a definition such as this would render practically useless the
%communication of theorems of propositional calculus; but
%this is just a human shortcoming, since we can't always easily discern what is
%and is not a theorem by inspection.  In fact, the extended theory with this
%definition has no more and no less information than the original theory; it just
%expresses certain theorems of the form ``$P\rightarrow P$''
%in a more compact way.
%
%The point here is that what constitutes a proper definition is a matter of
%judgment about whether a symbol sequence can easily be recognized by a human
%as invalid in some sense (for example, not a wff); if so, the symbol sequence
%can be appropriated for use by a definition in order to make the extended
%language more compact.  Metamath\index{Metamath} lacks the ability to make this
%judgment, since as far as Metamath is concerned the definition of a wff, for
%example, is arbitrary.  You define for Metamath how wffs\index{well-formed
%formula (wff)} are constructed according to your own preferred style.  The
%concept of a wff may not even exist in a given formal system\index{formal
%system}.  Metamath treats all definitions as if they were new axioms, and it
%is up to the human mathematician to judge whether the definition is ``proper''
%'\index{proper definition}\index{definition!proper} in some agreed-upon way.

What constitutes a definition\index{definition} versus\index{axiom vs.\
definition} an axiom\index{axiom} is sometimes arbitrary in mathematical
literature.  For example, the connectives $\vee$ ({\sc or}), $\wedge$
({\sc and}), and $\leftrightarrow$ (equivalent to) in propositional
calculus are usually considered defined symbols that can be used as
abbreviations for expressions containing the ``primitive'' connectives
$\rightarrow$ and $\neg$.  This is the way we treat them in the standard
logic and set theory database \texttt{set.mm}\index{set theory database
(\texttt{set.mm})}.  However, the first three connectives can also be
considered ``primitive,'' and axiom systems have been devised that treat
all of them as such.  For example,
\cite[p.~35]{Goodstein}\index{Goodstein, R. L.} presents one with 15
axioms, some of which in fact coincide with what we have chosen to call
definitions in \texttt{set.mm}.  In certain subsets of classical
propositional calculus, such as the intuitionist
fragment\index{intuitionism}, it can be shown that one cannot make do
with just $\rightarrow$ and $\neg$ but must treat additional connectives
as primitive in order for the system to make sense.\footnote{Two nice
systems that make the transition from intuitionistic and other weak
fragments to classical logic just by adding axioms are given in
\cite{Robinsont}\index{Robinson, T. Thacher}.}

\subsection{The Approach to Definitions in \texttt{set.mm}}

In set theory, recursive definitions define a newly introduced symbol in
terms of itself.  The justification of recursive definitions, using
several ``recursion theorems,'' is usually one of the first
sophisticated proofs a student encounters when learning set theory, and
there is a significant amount of implicit metalogic behind a recursive
definition even though the definition itself is typically simple to
state.  It is, however, possible to substitute one kind of complexity
for another.  We can eliminate the need for metalogical justification by
defining the operation directly with an explicit (but complicated)
expression, then deriving the recursive definition directly as a
theorem, using a recursion theorem ``in reverse.''  We do this in
\texttt{set.mm}, as follows.

In \texttt{set.mm} our goal was to introduce almost all definitions in
the form of two expressions connected by either $\leftrightarrow$ or
$=$, where the thing being defined does not appear on the right hand
side.  Quine calls this form ``a genuine or direct definition'' \cite[p.
174]{Quine}\index{Quine, Willard Van Orman}, which makes the definitions
very easy to eliminate and the metalogic\index{metalogic} needed to
justify them as simple as possible.

We achieved this goal in almost all cases in \texttt{set.mm}.
Sometimes this makes the definitions more complex and less
intuitive.  For example, the traditional way to define addition of
natural numbers is to define an operation called {\em
successor}\index{successor} (which means ``plus one'' and is denoted by
``${\rm suc}$''), then define addition recursively\index{recursive
definition} with the two definitions $n + 0 = n$ and $m + {\rm suc}\,n =
{\rm suc} (m + n)$.  Although this definition seems simple and obvious,
the method to eliminate the definition is not obvious:  in the second
part of the definition, addition is defined in terms of itself.  By
eliminating the definition, we don't mean repeatedly applying it to
specific $m$ and $n$ but rather showing the explicit, closed-form
set-theoretical expression that $m + n$ represents, that will work for
any $m$ and $n$ and that does not have a $+$ sign on its right-hand
side.  For a recursive definition like this not to be circular
(creative), there are some hidden, underlying assumptions we must make,
for example that the natural numbers have a certain kind of order.  In
\texttt{set.mm} we chose to start with the direct (though complex and
nonintuitive) definition then derive from it the standard recursive
definition.\footnote{The closed-form definition used in \texttt{set.mm}
for the addition operation on ordinals\index{ordinal
addition}\index{addition!of ordinals} (of which natural numbers are a
subset) is

\setbox\startprefix=\hbox{\tt \ \ df-oadd\ \$a\ }
\setbox\contprefix=\hbox{\tt \ \ \ \ \ \ \ \ \ \ \ \ \ }
\startm
\m{\vdash}\m{+_o}\m{=}\m{\{}\m{\langle}\m{\langle}\m{x}\m{,}\m{y}\m{\rangle}%
\m{,}\m{z}\m{\rangle}\m{|}\m{(}\m{(}\m{x}\m{\in}\m{{\rm On}}\m{\wedge}\m{y}\m{%
\in}\m{{\rm On}}\m{)}\m{\wedge}\m{z}\m{=}\m{(}\m{{\rm rec}}\m{(}\m{\{}\m{%
\langle}\m{w}\m{,}\m{v}\m{\rangle}\m{|}\m{v}\m{=}\m{{\rm suc}}\m{w}\m{\}}\m{,}%
\m{x}\m{)}\m{`}\m{y}\m{)}\m{)}\m{\}}
\endm

\noindent Here, the abstraction class of nested ordered pairs\index{abstraction
class!of nested ordered pairs} is defined by \texttt{df-oprab} in \texttt{set.mm},
and ${\rm rec}$ is a ``recursion operator''\index{recursion operator} with
the definition

\setbox\startprefix=\hbox{\tt \ \ df-rfg\ \$a\ }
\setbox\contprefix=\hbox{\tt \ \ \ \ \ \ \ \ \ \ \ \ }
\startm
\m{\vdash}\m{{\rm rec}}\m{(}\m{F}\m{,}\m{A}\m{)}\m{=}\m{\bigcup}\m{\{}\m{f}%
\m{|}\m{\exists}\m{x}\m{(}\m{x}\m{\in}\m{{\rm On}}\m{\wedge}\m{(}\m{f}\m{{%
\rm Fn}}\m{x}\m{\wedge}\m{\forall}\m{y}\m{(}\m{y}\m{\in}\m{x}\m{\rightarrow}%
\m{(}\m{f}\m{`}\m{y}\m{)}\m{=}\m{(}\m{\{}\m{\langle}\m{g}\m{,}\m{z}\m{\rangle}%
\m{|}\m{(}\m{(}\m{g}\m{=}\m{\varnothing}\m{\wedge}\m{z}\m{=}\m{A}\m{)}\m{\vee}%
\m{(}\m{\lnot}\m{(}\m{g}\m{=}\m{\varnothing}\m{\vee}\m{{\rm Lim}}\m{{\rm dom}}%
\m{g}\m{)}\m{\wedge}\m{z}\m{=}\m{(}\m{F}\m{`}\m{(}\m{g}\m{`}\m{\bigcup}\m{{%
\rm dom}}\m{g}\m{)}\m{)}\m{)}\m{\vee}\m{(}\m{{\rm Lim}}\m{{\rm dom}}\m{g}\m{%
\wedge}\m{z}\m{=}\m{\bigcup}\m{{\rm ran}}\m{g}\m{)}\m{)}\m{\}}\m{`}\m{(}\m{f}%
\m{\restriction}\m{y}\m{)}\m{)}\m{)}\m{)}\m{)}\m{\}}
\endm

\noindent which can be further broken down with definitions shown in
Section~\ref{setdefinitions}.  You may be surprised at the complexity of what
seems like such a simple notion.  From these definitions the simpler, more
intuitive recursive definition is derived as a set of theorems.}\index{natural
number}\index{addition}\index{recursive definition}\index{ordinal addition}
The end result is the same, but we completely eliminate the rather complex
metalogic that justifies the recursive definition.  (For a mathematician,
recursive definitions are more efficient and intuitive than
direct ones once the metalogic has been learned or possibly just accepted as
correct.  However, it was felt that direct definition in \texttt{set.mm}
maximizes rigor by minimizing metalogic.  It can be eliminated effortlessly,
something difficult to do with a recursive definition.)

\subsection{Adding Constraints on Definitions}

The basic Metamath language and the Metamath program do
not have any built-in constraints on definitions, since they are just
\$a statements.

However, nothing prevents a verification system from verifying
additional rules to impose further limitations on definitions.
For example, the \texttt{mmj2}\index{mmj2} program
supports various kinds of
additional information comments (see section \ref{jcomment}).
One of their uses is to optionally verify additional constraints,
including constraints to verify that definitions meet certain
requirements.
These additional checks are required by the
continuous integration (CI)\index{continuous integration (CI)}
checks of the
\texttt{set.mm}\index{set theory database (\texttt{set.mm})}%
\index{Metamath Proof Explorer}
database.
This approach enables us to optionally impose additional requirements
on definitions if we wish, without necessarily imposing those rules on
all databases or requiring all verification systems to implement them.
In addition, this allows us to impose specialized constraints tailored
to one database while not requiring other systems to implement
those specialized constraints.

We impose two constraints on the
\texttt{set.mm}\index{set theory database (\texttt{set.mm})}%
\index{Metamath Proof Explorer} database
via the \texttt{mmj2}\index{mmj2} program that are worth discussing here:
a parse check and a definition soundness check.

% On February 11, 2019 8:32:32 PM EST, saueran@oregonstate.edu wrote:
% The following addition to the end of set.mm is accepted by the mmj2
% parser and definition checker and the metamath verifier(at least it was
% when I checked, you should check it too), and creates a contradiction by
% proving the theorem |- ph.
% ${
% wleftp $a wff ( ( ph ) $.
% wbothp $a wff ( ph ) $.
% df-leftp $a |- ( ( ( ph ) <-> -. ph ) $.
% df-bothp $a |- ( ( ph ) <-> ph ) $.
% anything $p |- ph $=
%   ( wbothp wn wi wleftp df-leftp biimpi df-bothp mpbir mpbi simplim ax-mp)
%   ABZAMACZDZCZMOEZOCQAEZNDZRNAFGSHIOFJMNKLAHJ $.
% $}
%
% This particular problem is countered by enabling, within mmj2,
% SetParser,mmj.verify.LRParser

First,
we enable a parse check in \texttt{mmj2} (through its
\texttt{SetParser} command) that requires that all new definitions
must \textit{not} create an ambiguous parse for a KLR(5) parser.
This prevents some errors such as definitions with imbalanced parentheses.

Second, we run a definition soundness check specific to
\texttt{set.mm} or databases similar to it.
(through the \texttt{definitionCheck} macro).
Some \texttt{\$a} statements (including all ax-* statemnets)
are excluded from these checks, as they will
always fail this simple check,
but they are appropriate for most definitions.
This check imposes a set of additional rules:

\begin{enumerate}

\item New definitions must be introduced using $=$ or $\leftrightarrow$.

\item No \texttt{\$a} statement introduced before this one is allowed to use the
symbol being defined in this definition, and the definition is not
allowed to use itself (except once, in the definiendum).

\item Every variable in the definiens should not be distinct

\item Every dummy variable in the definiendum
are required to be distinct from each other and from variables in
the definiendum.
To determine this, the system will look for a "justification" theorem
in the database, and if it is not there, attempt to briefly prove
$( \varphi \rightarrow \forall x \varphi )$  for each dummy variable x.

\item Every dummy variable should be a set variable,
unless there is a justification theorem available.

\item Every dummy variable must be bound
(if the system cannot determine this a justification theorem must be
provided).

\end{enumerate}

\subsection{Summary of Approach to Definitions}

In short, when being rigorous it turns out that
definitions can be subtle, sometimes requiring difficult
metatheorems to establish that they are not creative.

Instead of building such complications into the Metamath language itself,
the basic Metmath language and program simply treat traditional
axioms and definitions as the same kind of \texttt{\$a} statement.
We have then built various tools to enable people to
verify additional conditions as their creators believe is appropriate
for those specific databases, without complicating the Metamath foundations.

\chapter{The Metamath Program}\label{commands}

This chapter provides a reference manual for the
Metamath program.\index{Metamath!commands}

Current instructions for obtaining and installing the Metamath program
can be found at the \url{http://metamath.org} web site.
For Windows, there is a pre-compiled version called
\texttt{metamath.exe}.  For Unix, Linux, and Mac OS X
(which we will refer to collectively as ``Unix''), the Metamath program
can be compiled from its source code with the command
\begin{verbatim}
gcc *.c -o metamath
\end{verbatim}
using the \texttt{gcc} {\sc c} compiler available on those systems.

In the command syntax descriptions below, fields enclosed in square
brackets [\ ] are optional.  File names may be optionally enclosed in
single or double quotes.  This is useful if the file name contains
spaces or
slashes (\texttt{/}), such as in Unix path names, \index{Unix file
names}\index{file names!Unix} that might be confused with Metamath
command qualifiers.\index{Unix file names}\index{file names!Unix}

\section{Invoking Metamath}

Unix, Linux, and Mac OS X
have a command-line interface called the {\em
bash shell}.  (In Mac OS X, select the Terminal application from
Applications/Utilities.)  To invoke Metamath from the bash shell prompt,
assuming that the Metamath program is in the current directory, type
\begin{verbatim}
bash$ ./metamath
\end{verbatim}

To invoke Metamath from a Windows DOS or Command Prompt, assuming that
the Metamath program is in the current directory (or in a directory
included in the Path system environment variable), type
\begin{verbatim}
C:\metamath>metamath
\end{verbatim}

To use command-line arguments at invocation, the command-line arguments
should be a list of Metamath commands, surrounded by quotes if they
contain spaces.  In Windows, the surrounding quotes must be double (not
single) quotes.  For example, to read the database file \texttt{set.mm},
verify all proofs, and exit the program, type (under Unix)
\begin{verbatim}
bash$ ./metamath 'read set.mm' 'verify proof *' exit
\end{verbatim}
Note that in Unix, any directory path with \texttt{/}'s must be
surrounded by quotes so Metamath will not interpret the \texttt{/} as a
command qualifier.  So if \texttt{set.mm} is in the \texttt{/tmp}
directory, use for the above example
\begin{verbatim}
bash$ ./metamath 'read "/tmp/set.mm"' 'verify proof *' exit
\end{verbatim}

For convenience, if the command-line has one argument and no spaces in
the argument, the command is implicitly assumed to be \texttt{read}.  In
this one special case, \texttt{/}'s are not interpreted as command
qualifiers, so you don't need quotes around a Unix file name.  Thus
\begin{verbatim}
bash$ ./metamath /tmp/set.mm
\end{verbatim}
and
\begin{verbatim}
bash$ ./metamath "read '/tmp/set.mm'"
\end{verbatim}
are equivalent.


\section{Controlling Metamath}

The Metamath program was first developed on a {\sc vax/vms} system, and
some aspects of its command line behavior reflect this heritage.
Hopefully you will find it reasonably user-friendly once you get used to
it.

Each command line is a sequence of English-like words separated by
spaces, as in \texttt{show settings}.  Command words are not case
sensitive, and only as many letters are needed as are necessary to
eliminate ambiguity; for example, \texttt{sh se} would work for the
command \texttt{show settings}.  In some cases arguments such as file
names, statement labels, or symbol names are required; these are
case-sensitive (although file names may not be on some operating
systems).

A command line is entered by typing it in then pressing the {\em return}
({\em enter}) key.  To find out what commands are available, type
\texttt{?} at the \texttt{MM>} prompt.  To find out the choices at any
point in a command, press {\em return} and you will be prompted for
them.  The default choice (the one selected if you just press {\em
return}) is shown in brackets (\texttt{<>}).

You may also type \texttt{?} in place of a command word to force
Metamath to tell you what the choices are.  The \texttt{?} method won't
work, though, if a non-keyword argument such as a file name is expected
at that point, because the program will think that \texttt{?} is the
value of the argument.

Some commands have one or more optional qualifiers which modify the
behavior of the command.  Qualifiers are preceded by a slash
(\texttt{/}), such as in \texttt{read set.mm / verify}.  Spaces are
optional around the \texttt{/}.  If you need to use a space or
slash in a command
argument, as in a Unix file name, put single or double quotes around the
command argument.

The \texttt{open log} command will save everything you see on the
screen and is useful to help you recover should something go wrong in a
proof, or if you want to document a bug.

If a command responds with more than a screenful, you will be
prompted to \texttt{<return> to continue, Q to quit, or S to scroll to
end}.  \texttt{Q} or \texttt{q} (not case-sensitive) will complete the
command internally but will suppress further output until the next
\texttt{MM>} prompt.  \texttt{s} will suppress further pausing until the
next \texttt{MM>} prompt.  After the first screen, you are also
presented with the choice of \texttt{b} to go back a screenful.  Note
that \texttt{b} may also be entered at the \texttt{MM>} prompt
immediately after a command to scroll back through the output of that
command.

A command line enclosed in quotes is executed by your operating system.
See Section~\ref{oscmd}.

{\em Warning:} Pressing {\sc ctrl-c} will abort the Metamath program
unconditionally.  This means any unsaved work will be lost.


\subsection{\texttt{exit} Command}\index{\texttt{exit} command}

Syntax:  \texttt{exit} [\texttt{/force}]

This command exits from Metamath.  If there have been changes to the
source with the \texttt{save proof} or \texttt{save new{\char`\_}proof}
commands, you will be given an opportunity to \texttt{write source} to
permanently save the changes.

In Proof Assistant\index{Proof Assistant} mode, the \texttt{exit} command will
return to the \verb/MM>/ prompt. If there were changes to the proof, you will
be given an opportunity to \texttt{save new{\char`\_}proof}.

The \texttt{quit} command is a synonym for \texttt{exit}.

Optional qualifier:
    \texttt{/force} - Do not prompt if changes were not saved.  This qualifier is
        useful in \texttt{submit} command files (Section~\ref{sbmt})
        to ensure predictable behavior.





\subsection{\texttt{open log} Command}\index{\texttt{open log} command}
Syntax:  \texttt{open log} {\em file-name}

This command will open a log file that will store everything you see on
the screen.  It is useful to help recovery from a mistake in a long Proof
Assistant session, or to document bugs.\index{Metamath!bugs}

The log file can be closed with \texttt{close log}.  It will automatically be
closed upon exiting Metamath.



\subsection{\texttt{close log} Command}\index{\texttt{close log} command}
Syntax:  \texttt{close log}

The \texttt{close log} command closes a log file if one is open.  See
also \texttt{open log}.




\subsection{\texttt{submit} Command}\index{\texttt{submit} command}\label{sbmt}
Syntax:  \texttt{submit} {\em filename}

This command causes further command lines to be taken from the specified
file.  Note that any line beginning with an exclamation point (\texttt{!}) is
treated as a comment (i.e.\ ignored).  Also note that the scrolling
of the screen output is continuous, so you may want to open a log file
(see \texttt{open log}) to record the results that fly by on the screen.
After the lines in the file are exhausted, Metamath returns to its
normal user interface mode.

Currently, the \texttt{submit} command is not recursive.  In other words,
\texttt{submit} commands are not allowed inside a command file.


\subsection{\texttt{erase} Command}\index{\texttt{erase} command}
Syntax:  \texttt{erase}

This command will reset Metamath to its starting state, deleting any
data\-base that was \texttt{read} in.
 If there have been changes to the
source with the \texttt{save proof} or \texttt{save new{\char`\_}proof}
commands, you will be given an opportunity to \texttt{write source} to
permanently save the changes.



\subsection{\texttt{set echo} Command}\index{\texttt{set echo} command}
Syntax:  \texttt{set echo on} or \texttt{set echo off}

The \texttt{set echo on} command will cause command lines to be echoed with any
abbreviations expanded.  While learning the Metamath commands, this
feature will show you the exact command that your abbreviated input
corresponds to.



\subsection{\texttt{set scroll} Command}\index{\texttt{set scroll} command}
Syntax:  \texttt{set scroll prompted} or \texttt{set scroll continuous}

The Metamath command line interface starts off in the \texttt{prompted} mode,
which means that you will be prompted to continue or quit after each
full screen in a long listing.  In \texttt{continuous} mode, long listings will be
scrolled without pausing.

% LaTeX bug? (1) \texttt{\_} puts out different character than
% \texttt{{\char`\_}}
%  = \verb$_$  (2) \texttt{{\char`\_}} puts out garbage in \subsection
%  argument
\subsection{\texttt{set width} Command}\index{\texttt{set
width} command}
Syntax:  \texttt{set width} {\em number}

Metamath assumes the width of your screen is 79 characters (chosen
because the Command Prompt in Windows XP has a wrapping bug at column
80).  If your screen is wider or narrower, this command allows you to
change this default screen width.  A larger width is advantageous for
logging proofs to an output file to be printed on a wide printer.  A
smaller width may be necessary on some terminals; in this case, the
wrapping of the information messages may sometimes seem somewhat
unnatural, however.  In \LaTeX\index{latex@{\LaTeX}!characters per line},
there is normally a maximum of 61 characters per line with typewriter
font.  (The examples in this book were produced with 61 characters per
line.)

\subsection{\texttt{set height} Command}\index{\texttt{set
height} command}
Syntax:  \texttt{set height} {\em number}

Metamath assumes your screen height is 24 lines of characters.  If your
screen is taller or shorter, this command lets you to change the number
of lines at which the display pauses and prompts you to continue.

\subsection{\texttt{beep} Command}\index{\texttt{beep} command}

Syntax:  \texttt{beep}

This command will produce a beep.  By typing it ahead after a
long-running command has started, it will alert you that the command is
finished.  For convenience, \texttt{b} is an abbreviation for
\texttt{beep}.

Note:  If \texttt{b} is typed at the \texttt{MM>} prompt immediately
after the end of a multiple-page display paged with ``\texttt{Press
<return> for more}...'' prompts, then the \texttt{b} will back up to the
previous page rather than perform the \texttt{beep} command.
In that case you must type the unabbreviated \texttt{beep} form
of the command.

\subsection{\texttt{more} Command}\index{\texttt{more} command}

Syntax:  \texttt{more} {\em filename}

This command will display the contents of an {\sc ascii} file on your
screen.  (This command is provided for convenience but is not very
powerful.  See Section~\ref{oscmd} to invoke your operating system's
command to do this, such as the \texttt{more} command in Unix.)

\subsection{Operating System Commands}\index{operating system
command}\label{oscmd}

A line enclosed in single or double quotes will be executed by your
computer's operating system if it has a command line interface.  For
example, on a {\sc vax/vms} system,
\verb/MM> 'dir'/
will print disk directory contents.  Note that this feature will not
work on the Macintosh prior to Mac OS X, which does not have a command
line interface.

For your convenience, the trailing quote is optional.

\subsection{Size Limitations in Metamath}

In general, there are no fixed, predefined limits\index{Metamath!memory
limits} on how many labels, tokens\index{token}, statements, etc.\ that
you may have in a database file.  The Metamath program uses 32-bit
variables (64-bit on 64-bit CPUs) as indices for almost all internal
arrays, which are allocated dynamically as needed.



\section{Reading and Writing Files}

The following commands create new files:  the \texttt{open} commands;
the \texttt{write} commands; the \texttt{/html},
\texttt{/alt{\char`\_}html}, \texttt{/brief{\char`\_}html},
\texttt{/brief{\char`\_}alt{\char`\_}html} qualifiers of \texttt{show
statement}, and \texttt{midi}.  The following commands append to files
previously opened:  the \texttt{/tex} qualifier of \texttt{show proof}
and \texttt{show new{\char`\_}proof}; the \texttt{/tex} and
\texttt{/simple{\char`\_}tex} qualifiers of \texttt{show statement}; the
\texttt{close} commands; and all screen dialog between \texttt{open log}
and \texttt{close log}.

The commands that create new files will not overwrite an existing {\em
filename} but will rename the existing one to {\em
filename}\texttt{{\char`\~}1}.  An existing {\em
filename}\texttt{{\char`\~}1} is renamed {\em
filename}\texttt{{\char`\~}2}, etc.\ up to {\em
filename}\texttt{{\char`\~}9}.  An existing {\em
filename}\texttt{{\char`\~}9} is deleted.  This makes recovery from
mistakes easier but also will clutter up your directory, so occasionally
you may want to clean up (delete) these \texttt{{\char`\~}}$n$ files.


\subsection{\texttt{read} Command}\index{\texttt{read} command}
Syntax:  \texttt{read} {\em file-name} [\texttt{/verify}]

This command will read in a Metamath language source file and any included
files.  Normally it will be the first thing you do when entering Metamath.
Statement syntax is checked, but proof syntax is not checked.
Note that the file name may be enclosed in single or double quotes;
this is useful if the file name contains slashes, as might be the case
under Unix.

If you are getting an ``\texttt{?Expected VERIFY}'' error
when trying to read a Unix file name with slashes, you probably haven't
quoted it.\index{Unix file names}\index{file names!Unix}

If you are prompted for the file name (by pressing {\em return}
 after \texttt{read})
you should {\em not} put quotes around it, even if it is a Unix file name
with slashes.

Optional command qualifier:

    \texttt{/verify} - Verify all proofs as the database is read in.  This
         qualifier will slow down reading in the file.  See \texttt{verify
         proof} for more information on file error-checking.

See also \texttt{erase}.



\subsection{\texttt{write source} Command}\index{\texttt{write source} command}
Syntax:  \texttt{write source} {\em filename}
[\texttt{/rewrap}]
[\texttt{/split}]
[\texttt{/keep\_includes}]
[\texttt{/no\_versioning}]

This command will write the contents of a Metamath\index{database}
database into a file.\index{source file}

Optional command qualifiers:

\texttt{/rewrap} -
Reformats statements and comments according to the
convention used in the set.mm database.
It unwraps the
lines in the comment before each \texttt{\$a} and \texttt{\$p} statement,
then it
rewraps the line.  You should compare the output to the original
to make sure that the desired effect results; if not, go back to
the original source.  The wrapped line length honors the
\texttt{set width}
parameter currently in effect.  Note:  Text
enclosed in \texttt{<HTML>}...\texttt{</HTML>} tags is not modified by the
\texttt{/rewrap} qualifier.
Proofs themselves are not reformatted;
use \texttt{save proof * / compressed} to do that.
An isolated tilde (\~{}) is always kept on the same line as the following
symbol, so you can find all comment references to a symbol by
searching for \~{} followed by a space and that symbol
(this is useful for finding cross-references).
Incidentally, \texttt{save proof} also honors the \texttt{set width}
parameter currently in effect.

\texttt{/split} - Files included in the source using the expression
\$[ \textit{inclfile} \$] will be
written into separate files instead of being included in a single output
file.  The name of each separately written included file will be the
\textit{inclfile} argument of its inclusion command.

\texttt{/keep\_includes} - If a source file has includes but is written as a
single file by omitting \texttt{/split}, by default the included files will
be deleted (actually just renamed with a \char`\~1 suffix unless
\texttt{/no\_versioning} is specified) to prevent the possibly confusing
source duplication in both the output file and the included file.
The \texttt{/keep\_includes} qualifier will prevent this deletion.

\texttt{/no\_versioning} - Backup files suffixed with \char`\~1 are not created.


\section{Showing Status and Statements}



\subsection{\texttt{show settings} Command}\index{\texttt{show settings} command}
Syntax:  \texttt{show settings}

This command shows the state of various parameters.

\subsection{\texttt{show memory} Command}\index{\texttt{show memory} command}
Syntax:  \texttt{show memory}

This command shows the available memory left.  It is not meaningful
on most modern operating systems,
which have virtual memory.\index{Metamath!memory usage}


\subsection{\texttt{show labels} Command}\index{\texttt{show labels} command}
Syntax:  \texttt{show labels} {\em label-match} [\texttt{/all}]
   [\texttt{/linear}]

This command shows the labels of \texttt{\$a} and \texttt{\$p}
statements that match {\em label-match}.  A \verb$*$ in {label-match}
matches zero or more characters.  For example, \verb$*abc*def$ will match all
labels containing \verb$abc$ and ending with \verb$def$.

Optional command qualifiers:

   \texttt{/all} - Include matches for \texttt{\$e} and \texttt{\$f}
   statement labels.

   \texttt{/linear} - Display only one label per line.  This can be useful for
       building scripts in conjunction with the utilities under the
       \texttt{tools}\index{\texttt{tools} command} command.



\subsection{\texttt{show statement} Command}\index{\texttt{show statement} command}
Syntax:  \texttt{show statement} {\em label-match} [{\em qualifiers} (see below)]

This command provides information about a statement.  Only statements
that have labels (\texttt{\$f}\index{\texttt{\$f} statement},
\texttt{\$e}\index{\texttt{\$e} statement},
\texttt{\$a}\index{\texttt{\$a} statement}, and
\texttt{\$p}\index{\texttt{\$p} statement}) may be specified.
If {\em label-match}
contains wildcard (\verb$*$) characters, all matching statements will be
displayed in the order they occur in the database.

Optional qualifiers (only one qualifier at a time is allowed):

    \texttt{/comment} - This qualifier includes the comment that immediately
       precedes the statement.

    \texttt{/full} - Show complete information about each statement,
       and show all
       statements matching {\em label} (including \texttt{\$e}
       and \texttt{\$f} statements).

    \texttt{/tex} - This qualifier will write the statement information to the
       \LaTeX\ file previously opened with \texttt{open tex}.  See
       Section~\ref{texout}.

    \texttt{/simple{\char`\_}tex} - The same as \texttt{/tex}, except that
       \LaTeX\ macros are not used for formatting equations, allowing easier
       manual edits of the output for slide presentations, etc.

    \texttt{/html}\index{html generation@{\sc html} generation},
       \texttt{/alt{\char`\_}html}, \texttt{/brief{\char`\_}html},
       \texttt{/brief{\char`\_}alt{\char`\_}html} -
       These qualifiers invoke a special mode of
       \texttt{show statement} that
       creates a web page for the statement.  They may not be used with
       any other qualifier.  See Section~\ref{htmlout} or
       \texttt{help html} in the program.


\subsection{\texttt{search} Command}\index{\texttt{search} command}
Syntax:  search {\em label-match}
\texttt{"}{\em symbol-match}{\tt}" [\texttt{/all}] [\texttt{/comments}]
[\texttt{/join}]

This command searches all \texttt{\$a} and \texttt{\$p} statements
matching {\em label-match} for occurrences of {\em symbol-match}.  A
\verb@*@ in {\em label-match} matches any label character.  A \verb@$*@
in {\em symbol-match} matches any sequence of symbols.  The symbols in
{\em symbol-match} must be separated by white space.  The quotes
surrounding {\em symbol-match} may be single or double quotes.  For
example, \texttt{search b}\verb@* "-> $* ch"@ will list all statements
whose labels begin with \texttt{b} and contain the symbols \verb@->@ and
\texttt{ch} surrounding any symbol sequence (including no symbol
sequence).  The wildcards \texttt{?} and \texttt{\$?} are also available
to match individual characters in labels and symbols respectively; see
\texttt{help search} in the Metamath program for details on their usage.

Optional command qualifiers:

    \texttt{/all} - Also search \texttt{\$e} and \texttt{\$f} statements.

    \texttt{/comments} - Search the comment that immediately precedes each
        label-matched statement for {\em symbol-match}.  In this case
        {\em symbol-match} is an arbitrary, non-case-sensitive character
        string.  Quotes around {\em symbol-match} are optional if there
        is no ambiguity.

    \texttt{/join} - In the case of a \texttt{\$a} or \texttt{\$p} statement,
	prepend its \texttt{\$e}
	hypotheses for searching. The
	\texttt{/join} qualifier has no effect in \texttt{/comments} mode.

\section{Displaying and Verifying Proofs}


\subsection{\texttt{show proof} Command}\index{\texttt{show proof} command}
Syntax:  \texttt{show proof} {\em label-match} [{\em qualifiers} (see below)]

This command displays the proof of the specified
\texttt{\$p}\index{\texttt{\$p} statement} statement in various formats.
The {\em label-match} may contain wildcard (\verb@$*@) characters to match
multiple statements.  Without any qualifiers, only the logical steps
will be shown (i.e.\ syntax construction steps will be omitted), in an
indented format.

Most of the time, you will use
    \texttt{show proof} {\em label}
to see just the proof steps corresponding to logical inferences.

Optional command qualifiers:

    \texttt{/essential} - The proof tree is trimmed of all
        \texttt{\$f}\index{\texttt{\$f} statement} hypotheses before
        being displayed.  (This is the default, and it is redundant to
        specify it.)

    \texttt{/all} - the proof tree is not trimmed of all \texttt{\$f} hypotheses before
        being displayed.  \texttt{/essential} and \texttt{/all} are mutually exclusive.

    \texttt{/from{\char`\_}step} {\em step} - The display starts at the specified
        step.  If
        this qualifier is omitted, the display starts at the first step.

    \texttt{/to{\char`\_}step} {\em step} - The display ends at the specified
        step.  If this
        qualifier is omitted, the display ends at the last step.

    \texttt{/tree{\char`\_}depth} {\em number} - Only
         steps at less than the specified proof
        tree depth are displayed.  Sometimes useful for obtaining an overview of
        the proof.

    \texttt{/reverse} - The steps are displayed in reverse order.

    \texttt{/renumber} - When used with \texttt{/essential}, the steps are renumbered
        to correspond only to the essential steps.

    \texttt{/tex} - The proof is converted to \LaTeX\ and\index{latex@{\LaTeX}}
        stored in the file opened
        with \texttt{open tex}.  See Section~\ref{texout} or
        \texttt{help tex} in the program.

    \texttt{/lemmon} - The proof is displayed in a non-indented format known
        as Lemmon style, with explicit previous step number references.
        If this qualifier is omitted, steps are indented in a tree format.

    \texttt{/start{\char`\_}column} {\em number} - Overrides the default column
        (16)
        at which the formula display starts in a Lemmon-style display.  May be
        used only in conjunction with \texttt{/lemmon}.

    \texttt{/normal} - The proof is displayed in normal format suitable for
        inclusion in a Metamath source file.  May not be used with any other
        qualifier.

    \texttt{/compressed} - The proof is displayed in compressed format
        suitable for inclusion in a Metamath source file.  May not be used with
        any other qualifier.

    \texttt{/statement{\char`\_}summary} - Summarizes all statements (like a
        brief \texttt{show statement})
        used by the proof.  It may not be used with any other qualifier
        except \texttt{/essential}.

    \texttt{/detailed{\char`\_}step} {\em step} - Shows the details of what is
        happening at
        a specific proof step.  May not be used with any other qualifier.
        The {\em step} is the step number shown when displaying a
        proof without the \texttt{/renumber} qualifier.


\subsection{\texttt{show usage} Command}\index{\texttt{show usage} command}
Syntax:  \texttt{show usage} {\em label-match} [\texttt{/recursive}]

This command lists the statements whose proofs make direct reference to
the statement specified.

Optional command qualifier:

    \texttt{/recursive} - Also include statements whose proofs ultimately
        depend on the statement specified.



\subsection{\texttt{show trace\_back} Command}\index{\texttt{show
       trace{\char`\_}back} command}
Syntax:  \texttt{show trace{\char`\_}back} {\em label-match} [\texttt{/essential}] [\texttt{/axioms}]
    [\texttt{/tree}] [\texttt{/depth} {\em number}]

This command lists all statements that the proof of the \texttt{\$p}
statement(s) specified by {\em label-match} depends on.

Optional command qualifiers:

    \texttt{/essential} - Restrict the trace-back to \texttt{\$e}
        \index{\texttt{\$e} statement} hypotheses of proof trees.

    \texttt{/axioms} - List only the axioms that the proof ultimately depends on.

    \texttt{/tree} - Display the trace-back in an indented tree format.

    \texttt{/depth} {\em number} - Restrict the \texttt{/tree} trace-back to the
        specified indentation depth.

    \texttt{/count{\char`\_}steps} - Count the number of steps the proof has
       all the way back to axioms.  If \texttt{/essential} is specified,
       expansions of variable-type hypotheses (syntax constructions) are not counted.

\subsection{\texttt{verify proof} Command}\index{\texttt{verify proof} command}
Syntax:  \texttt{verify proof} {\em label-match} [\texttt{/syntax{\char`\_}only}]

This command verifies the proofs of the specified statements.  {\em
label-match} may contain wild card characters (\texttt{*}) to verify
more than one proof; for example \verb/*abc*def/ will match all labels
containing \texttt{abc} and ending with \texttt{def}.
The command \texttt{verify proof *} will verify all proofs in the database.

Optional command qualifier:

    \texttt{/syntax{\char`\_}only} - This qualifier will perform a check of syntax
        and RPN
        stack violations only.  It will not verify that the proof is
        correct.  This qualifier is useful for quickly determining which
        proofs are incomplete (i.e.\ are under development and have \texttt{?}'s
        in them).

{\em Note:} \texttt{read}, followed by \texttt{verify proof *}, will ensure
 the database is free
from errors in the Metamath language but will not check the markup notation
in comments.
You can also check the markup notation by running \texttt{verify markup *}
as discussed in Section~\ref{verifymarkup}; see also the discussion
on generating {\sc HTML} in Section~\ref{htmlout}.

\subsection{\texttt{verify markup} Command}\index{\texttt{verify markup} command}\label{verifymarkup}
Syntax:  \texttt{verify markup} {\em label-match}
[\texttt{/date{\char`\_}skip}]
[\texttt{/top{\char`\_}date{\char`\_}skip}]
[\texttt{/file{\char`\_}skip}]
[\texttt{/verbose}]

This command checks comment markup and other informal conventions we have
adopted.  It error-checks the latexdef, htmldef, and althtmldef statements
in the \texttt{\$t} statement of a Metamath source file.\index{error checking}
It error-checks any \texttt{`}...\texttt{`},
\texttt{\char`\~}~\textit{label},
and bibliographic markups in statement descriptions.
It checks that
\texttt{\$p} and \texttt{\$a} statements
have the same content when their labels start with
``ax'' and ``ax-'' respectively but are otherwise identical, for example
ax4 and ax-4.
It also verifies the date consistency of ``(Contributed by...),''
``(Revised by...),'' and ``(Proof shortened by...)'' tags in the comment
above each \texttt{\$a} and \texttt{\$p} statement.

Optional command qualifiers:

    \texttt{/date{\char`\_}skip} - This qualifier will
        skip date consistency checking,
        which is usually not required for databases other than
	\texttt{set.mm}.

    \texttt{/top{\char`\_}date{\char`\_}skip} - This qualifier will check date consistency except
        that the version date at the top of the database file will not
        be checked.  Only one of
        \texttt{/date{\char`\_}skip} and
        \texttt{/top{\char`\_}date{\char`\_}skip} may be
        specified.

    \texttt{/file{\char`\_}skip} - This qualifier will skip checks that require
        external files to be present, such as checking GIF existence and
        bibliographic links to mmset.html or equivalent.  It is useful
        for doing a quick check from a directory without these files.

    \texttt{/verbose} - Provides more information.  Currently it provides a list
        of axXXX vs. ax-XXX matches.

\subsection{\texttt{save proof} Command}\index{\texttt{save proof} command}
Syntax:  \texttt{save proof} {\em label-match} [\texttt{/normal}]
   [\texttt{/compressed}]

The \texttt{save proof} command will reformat a proof in one of two formats and
replace the existing proof in the source buffer\index{source
buffer}.  It is useful for
converting between proof formats.  Note that a proof will not be
permanently saved until a \texttt{write source} command is issued.

Optional command qualifiers:

    \texttt{/normal} - The proof is saved in the normal format (i.e., as a
        sequence
        of labels, which is the defined format of the basic Metamath
        language).\index{basic language}  This is the default format that
        is used if a qualifier
        is omitted.

    \texttt{/compressed} - The proof is saved in the compressed format which
        reduces storage requirements for a database.
        See Appendix~\ref{compressed}.




\section{Creating Proofs}\label{pfcommands}\index{Proof Assistant}

Before using the Proof Assistant, you must add a \texttt{\$p} to your
source file (using a text editor) containing the statement you want to
prove.  Its proof should consist of a single \texttt{?}, meaning
``unknown step.''  Example:
\begin{verbatim}
equid $p x = x $= ? $.
\end{verbatim}

To enter the Proof assistant, type \texttt{prove} {\em label}, e.g.
\texttt{prove equid}.  Metamath will respond with the \texttt{MM-PA>}
prompt.

Proofs are created working backwards from the statement being proved,
primarily using a series of \texttt{assign} commands.  A proof is
complete when all steps are assigned to statements and all steps are
unified and completely known.  During the creation of a proof, Metamath
will allow only operations that are legal based on what is known up to
that point.  For example, it will not allow an \texttt{assign} of a
statement that cannot be unified with the unknown proof step being
assigned.

{\em Important:}
The Proof Assistant is
{\em not} a tool to help you discover proofs.  It is just a tool to help
you add them to the database.  For a tutorial read
Section~\ref{frstprf}.
To practice using the Proof Assistant, you may
want to \texttt{prove} an existing theorem, then delete all steps with
\texttt{delete all}, then re-create it with the Proof Assistant while
looking at its proof display (before deletion).
You might want to figure out your first few proofs completely
and write them down by hand, before using the Proof Assistant, though
not everyone finds that effective.

{\em Important:}
The \texttt{undo} command if very helpful when entering a proof, because
it allows you to undo a previously-entered step.
In addition, we suggest that you
keep track of your work with a log file (\texttt{open
log}) and save it frequently (\texttt{save new{\char`\_}proof},
\texttt{write source}).
You can use \texttt{delete} to reverse an \texttt{assign}.
You can also do \texttt{delete floating{\char`\_}hypotheses}, then
\texttt{initialize all}, then \texttt{unify all /interactive} to
reinitialize bad unifications made accidentally or by bad
\texttt{assign}s.  You cannot reverse a \texttt{delete} except by
a relevant \texttt{undo} or using
\texttt{exit /force} then reentering the Proof Assistant to recover from
the last \texttt{save new{\char`\_}proof}.

The following commands are available in the Proof Assistant (at the
\texttt{MM-PA>} prompt) to help you create your proof.  See the
individual commands for more detail.

\begin{itemize}
\item[]
    \texttt{show new{\char`\_}proof} [\texttt{/all},...] - Displays the
        proof in progress.  You will use this command a lot; see \texttt{help
        show new{\char`\_}proof} to become familiar with its qualifiers.  The
        qualifiers \texttt{/unknown} and \texttt{/not{\char`\_}unified} are
        useful for seeing the work remaining to be done.  The combination
        \texttt{/all/unknown} is useful for identifying dummy variables that must be
        assigned, or attempts to use illegal syntax, when \texttt{improve all}
        is unable to complete the syntax constructions.  Unknown variables are
        shown as \texttt{\$1}, \texttt{\$2},...
\item[]
    \texttt{assign} {\em step} {\em label} - Assigns an unknown {\em step}
        number with the statement
        specified by {\em label}.
\item[]
    \texttt{let variable} {\em variable}
        \texttt{= "}{\em symbol sequence}\texttt{"}
          - Forces a symbol
        sequence to replace an unknown variable (such as \texttt{\$1}) in a proof.
        It is useful
        for helping difficult unifications, and it is necessary when you have
        dummy variables that eventually must be assigned a name.
\item[]
    \texttt{let step} {\em step} \texttt{= "}{\em symbol sequence}\texttt{"} -
          Forces a symbol sequence
        to replace the contents of a proof step, provided it can be
        unified with the existing step contents.  (I rarely use this.)
\item[]
    \texttt{unify step} {\em step} (or \texttt{unify all}) - Unifies
        the source and target of
        a step.  If you specify a specific step, you will be prompted
        to select among the unifications that are possible.  If you
        specify \texttt{all}, all steps with unique unifications, but only
        those steps, will be
        unified.  \texttt{unify all /interactive} goes through all non-unified
        steps.
\item[]
    \texttt{initialize} {\em step} (or \texttt{all}) - De-unifies the target and source of
        a step (or all steps), as well as the hypotheses of the source,
        and makes all variables in the source unknown.  Useful to recover from
        an \texttt{assign} or \texttt{let} mistake that
        resulted in incorrect unifications.
\item[]
    \texttt{delete} {\em step} (or \texttt{all} or \texttt{floating{\char`\_}hypotheses}) -
        Deletes the specified
        step(s).  \texttt{delete floating{\char`\_}hypotheses}, then \texttt{initialize all}, then
        \texttt{unify all /interactive} is useful for recovering from mistakes
        where incorrect unifications assigned wrong math symbol strings to
        variables.
\item[]
    \texttt{improve} {\em step} (or \texttt{all}) -
      Automatically creates a proof for steps (with no unknown variables)
      whose proof requires no statements with \texttt{\$e} hypotheses.  Useful
      for filling in proofs of \texttt{\$f} hypotheses.  The \texttt{/depth}
      qualifier will also try statements whose \texttt{\$e} hypotheses contain
      no new variables.  {\em Warning:} Save your work (with \texttt{save
      new{\char`\_}proof} then \texttt{write source}) before using
      \texttt{/depth = 2} or greater, since the search time grows
      exponentially and may never terminate in a reasonable time, and you
      cannot interrupt the search.  I have found that it is rare for
      \texttt{/depth = 3} or greater to be useful.
 \item[]
    \texttt{save new{\char`\_}proof} - Saves the proof in progress in the program's
        internal database buffer.  To save it permanently into the database file,
        use \texttt{write source} after
        \texttt{save new{\char`\_}proof}.  To revert to the last
        \texttt{save new{\char`\_}proof},
        \texttt{exit /force} from the Proof Assistant then re-enter the Proof
        Assistant.
 \item[]
    \texttt{match step} {\em step} (or \texttt{match all}) - Shows what
        statements are
        possibilities for the \texttt{assign} statement. (This command
        is not very
        useful in its present form and hopefully will be improved
        eventually.  In the meantime, use the \texttt{search} statement for
        candidates matching specific math token combinations.)
 \item[]
 \texttt{minimize{\char`\_}with}\index{\texttt{minimize{\char`\_}with} command}
% 3/10/07 Note: line-breaking the above results in duplicate index entries
     - After a proof is complete, this command will attempt
        to match other database theorems to the proof to see if the proof
        size can be reduced as a result.  See \texttt{help
        minimize{\char`\_}with} in the
        Metamath program for its usage.
 \item[]
 \texttt{undo}\index{\texttt{undo} command}
    - Undo the effect of a proof-changing command (all but the
      \texttt{show} and \texttt{save} commands above).
 \item[]
 \texttt{redo}\index{\texttt{redo} command}
    - Reverse the previous \texttt{undo}.
\end{itemize}

The following commands set parameters that may be relevant to your proof.
Consult the individual \texttt{help set}... commands.
\begin{itemize}
   \item[] \texttt{set unification{\char`\_}timeout}
 \item[]
    \texttt{set search{\char`\_}limit}
  \item[]
    \texttt{set empty{\char`\_}substitution} - note that default is \texttt{off}
\end{itemize}

Type \texttt{exit} to exit the \texttt{MM-PA>}
 prompt and get back to the \texttt{MM>} prompt.
Another \texttt{exit} will then get you out of Metamath.



\subsection{\texttt{prove} Command}\index{\texttt{prove} command}
Syntax:  \texttt{prove} {\em label}

This command will enter the Proof Assistant\index{Proof Assistant}, which will
allow you to create or edit the proof of the specified statement.
The command-line prompt will change from \texttt{MM>} to \texttt{MM-PA>}.

Note:  In the present version of
Metamath\index{Metamath!limitations of version 0.07.30}, the Proof
Assistant does not verify that \texttt{\$d}\index{\texttt{\$d}
statement} restrictions are met as a proof is being built.  After you
have completed a proof, you should type \texttt{save new{\char`\_}proof}
followed by \texttt{verify proof} {\em label} (where {\em label} is the
statement you are proving with the \texttt{prove} command) to verify the
\texttt{\$d} restrictions.

See also: \texttt{exit}

\subsection{\texttt{set unification\_timeout} Command}\index{\texttt{set
unification{\char`\_}timeout} command}
Syntax:  \texttt{set unification{\char`\_}timeout} {\em number}

(This command is available outside the Proof Assistant but affects the
Proof Assistant\index{Proof Assistant} only.)

Sometimes the Proof Assistant will inform you that a unification
time-out occurred.  This may happen when you try to \texttt{unify}
formulas with many temporary variables\index{temporary variable}
(\texttt{\$1}, \texttt{\$2}, etc.), since the time to compute all possible
unifications may grow exponentially with the number of variables.  If
you want Metamath to try harder (and you're willing to wait longer) you
may increase this parameter.  \texttt{show settings} will show you the
current value.

\subsection{\texttt{set empty\_substitution} Command}\index{\texttt{set
empty{\char`\_}substitution} command}
% These long names can't break well in narrow mode, and even "sloppy"
% is not enough. Work around this by not demanding justification.
\begin{flushleft}
Syntax:  \texttt{set empty{\char`\_}substitution on} or \texttt{set
empty{\char`\_}substitution off}
\end{flushleft}

(This command is available outside the Proof Assistant but affects the
Proof Assistant\index{Proof Assistant} only.)

The Metamath language allows variables to be
substituted\index{substitution!variable}\index{variable substitution}
with empty symbol sequences\index{empty substitution}.  However, in many
formal systems\index{formal system} this will never happen in a valid
proof.  Allowing for this possibility increases the likelihood of
ambiguous unifications\index{ambiguous
unification}\index{unification!ambiguous} during proof creation.
The default is that
empty substitutions are not allowed; for formal systems requiring them,
you must \texttt{set empty{\char`\_}substitution on}.
(An example where this must be \texttt{on}
would be a system that implements a Deduction Rule and in
which deductions from empty assumption lists would be permissible.  The
MIU-system\index{MIU-system} described in Appendix~\ref{MIU} is another
example.)
Note that empty substitutions are
always permissible in proof verification (VERIFY PROOF...) outside the
Proof Assistant.  (See the MIU system in the Metamath book for an example
of a system needing empty substitutions; another example would be a
system that implements a Deduction Rule and in which deductions from
empty assumption lists would be permissible.)

It is better to leave this \texttt{off} when working with \texttt{set.mm}.
Note that this command does not affect the way proofs are verified with
the \texttt{verify proof} command.  Outside of the Proof Assistant,
substitution of empty sequences for math symbols is always allowed.

\subsection{\texttt{set search\_limit} Command}\index{\texttt{set
search{\char`\_}limit} command} Syntax:  \texttt{set search{\char`\_}limit} {\em
number}

(This command is available outside the Proof Assistant but affects the
Proof Assistant\index{Proof Assistant} only.)

This command sets a parameter that determines when the \texttt{improve} command
in Proof Assistant mode gives up.  If you want \texttt{improve} to search harder,
you may increase it.  The \texttt{show settings} command tells you its current
value.


\subsection{\texttt{show new\_proof} Command}\index{\texttt{show
new{\char`\_}proof} command}
Syntax:  \texttt{show new{\char`\_}proof} [{\em
qualifiers} (see below)]

This command (available only in Proof Assistant mode) displays the proof
in progress.  It is identical to the \texttt{show proof} command, except that
there is no statement argument (since it is the statement being proved) and
the following qualifiers are not available:

    \texttt{/statement{\char`\_}summary}

    \texttt{/detailed{\char`\_}step}

Also, the following additional qualifiers are available:

    \texttt{/unknown} - Shows only steps that have no statement assigned.

    \texttt{/not{\char`\_}unified} - Shows only steps that have not been unified.

Note that \texttt{/essential}, \texttt{/depth}, \texttt{/unknown}, and
\texttt{/not{\char`\_}unified} may be
used in any combination; each of them effectively filters out additional
steps from the proof display.

See also:  \texttt{show proof}






\subsection{\texttt{assign} Command}\index{\texttt{assign} command}
Syntax:   \texttt{assign} {\em step} {\em label} [\texttt{/no{\char`\_}unify}]

   and:   \texttt{assign first} {\em label}

   and:   \texttt{assign last} {\em label}


This command, available in the Proof Assistant only, assigns an unknown
step (one with \texttt{?} in the \texttt{show new{\char`\_}proof}
listing) with the statement specified by {\em label}.  The assignment
will not be allowed if the statement cannot be unified with the step.

If \texttt{last} is specified instead of {\em step} number, the last
step that is shown by \texttt{show new{\char`\_}proof /unknown} will be
used.  This can be useful for building a proof with a command file (see
\texttt{help submit}).  It also makes building proofs faster when you know
the assignment for the last step.

If \texttt{first} is specified instead of {\em step} number, the first
step that is shown by \texttt{show new{\char`\_}proof /unknown} will be
used.

If {\em step} is zero or negative, the -{\em step}th from last unknown
step, as shown by \texttt{show new{\char`\_}proof /unknown}, will be
used.  \texttt{assign -1} {\em label} will assign the penultimate
unknown step, \texttt{assign -2} {\em label} the antepenultimate, and
\texttt{assign 0} {\em label} is the same as \texttt{assign last} {\em
label}.

Optional command qualifier:

    \texttt{/no{\char`\_}unify} - do not prompt user to select a unification if there is
        more than one possibility.  This is useful for noninteractive
        command files.  Later, the user can \texttt{unify all /interactive}.
        (The assignment will still be automatically unified if there is only
        one possibility and will be refused if unification is not possible.)



\subsection{\texttt{match} Command}\index{\texttt{match} command}
Syntax:  \texttt{match step} {\em step} [\texttt{/max{\char`\_}essential{\char`\_}hyp}
{\em number}]

    and:  \texttt{match all} [\texttt{/essential}]
          [\texttt{/max{\char`\_}essential{\char`\_}hyp} {\em number}]

This command, available in the Proof Assistant only, shows what
statements can be unified with the specified step(s).  {\em Note:} In
its current form, this command is not very useful because of the large
number of matches it reports.  I rarely use it.
It may be enhanced in the future.

Optional command qualifiers:

    \texttt{/max{\char`\_}essential{\char`\_}hyp} {\em number} - filters out
        of the list any statements
        with more than the specified number of
        \texttt{\$e}\index{\texttt{\$e} statement} hypotheses

    \texttt{/essential{\char`\_}only} - in the \texttt{match all} statement, only
        the steps that
        would be listed in the \texttt{show new{\char`\_}proof /essential} display are
        matched.



\subsection{\texttt{let} Command}\index{\texttt{let} command}
Syntax: \texttt{let variable} {\em variable} = \verb/"/{\em symbol-sequence}\verb/"/

  and: \texttt{let step} {\em step} = \verb/"/{\em symbol-sequence}\verb/"/

These commands, available in the Proof Assistant\index{Proof Assistant}
only, assign a temporary variable\index{temporary variable} or unknown
step with a specific symbol sequence.  They are useful in the middle of
creating a proof, when you know what should be in the proof step but the
unification algorithm doesn't yet have enough information to completely
specify the temporary variables.  A ``temporary variable'' is one that
has the form \texttt{\$}{\em nn} in the proof display, such as
\texttt{\$1}, \texttt{\$2}, etc.  The {\em symbol-sequence} may contain
other unknown variables if desired.  Examples:

    \verb/let variable $32 = "A = B"/

    \verb/let variable $32 = "A = $35"/

    \verb/let step 10 = '|- x = x'/

    \verb/let step -2 = "|- ( $7 = ph )"/

Any symbol sequence will be accepted for the \texttt{let variable}
command.  Only those symbol sequences that can be unified with the step
will be accepted for \texttt{let step}.

The \texttt{let} commands ``zap'' the proof with information that can
only be verified when the proof is built up further.  If you make an
error, the command sequence \texttt{delete
floating{\char`\_}hypotheses}, \texttt{initialize all}, and
\texttt{unify all /interactive} will undo a bad \texttt{let} assignment.

If {\em step} is zero or negative, the -{\em step}th from last unknown
step, as shown by \texttt{show new{\char`\_}proof /unknown}, will be
used.  The command \texttt{let step 0} = \verb/"/{\em
symbol-sequence}\verb/"/ will use the last unknown step, \texttt{let
step -1} = \verb/"/{\em symbol-sequence}\verb/"/ the penultimate, etc.
If {\em step} is positive, \texttt{let step} may be used to assign known
(in the sense of having previously been assigned a label with
\texttt{assign}) as well as unknown steps.

Either single or double quotes can surround the {\em symbol-sequence} as
long as they are different from any quotes inside a {\em
symbol-sequence}.  If {\em symbol-sequence} contains both kinds of
quotes, see the instructions at the end of \texttt{help let} in the
Metamath program.


\subsection{\texttt{unify} Command}\index{\texttt{unify} command}
Syntax:  \texttt{unify step} {\em step}

      and:   \texttt{unify all} [\texttt{/interactive}]

These commands, available in the Proof Assistant only, unify the source
and target of the specified step(s). If you specify a specific step, you
will be prompted to select among the unifications that are possible.  If
you specify \texttt{all}, only those steps with unique unifications will be
unified.

Optional command qualifier for \texttt{unify all}:

    \texttt{/interactive} - You will be prompted to select among the
        unifications
        that are possible for any steps that do not have unique
        unifications.  (Otherwise \texttt{unify all} will bypass these.)

See also \texttt{set unification{\char`\_}timeout}.  The default is
100000, but increasing it to 1000000 can help difficult cases.  Manually
assigning some or all of the unknown variables with the \texttt{let
variable} command also helps difficult cases.



\subsection{\texttt{initialize} Command}\index{\texttt{initialize} command}
Syntax:  \texttt{initialize step} {\em step}

    and: \texttt{initialize all}

These commands, available in the Proof Assistant\index{Proof Assistant}
only, ``de-unify'' the target and source of a step (or all steps), as
well as the hypotheses of the source, and makes all variables in the
source and the source's hypotheses unknown.  This command is useful to
help recover from incorrect unifications that resulted from an incorrect
\texttt{assign}, \texttt{let}, or unification choice.  Part or all of
the command sequence \texttt{delete floating{\char`\_}hypotheses},
\texttt{initialize all}, and \texttt{unify all /interactive} will recover
from incorrect unifications.

See also:  \texttt{unify} and \texttt{delete}



\subsection{\texttt{delete} Command}\index{\texttt{delete} command}
Syntax:  \texttt{delete step} {\em step}

   and:      \texttt{delete all} -- {\em Warning: dangerous!}

   and:      \texttt{delete floating{\char`\_}hypotheses}

These commands are available in the Proof Assistant only.  The
\texttt{delete step} command deletes the proof tree section that
branches off of the specified step and makes the step become unknown.
\texttt{delete all} is equivalent to \texttt{delete step} {\em step}
where {\em step} is the last step in the proof (i.e.\ the beginning of
the proof tree).

In most cases the \texttt{undo} command is the best way to undo
a previous step.
An alternative is to salvage your last \texttt{save
new{\char`\_}proof} by exiting and reentering the Proof Assistant.
For this to work, keep a log file open to record your work
and to do \texttt{save new{\char`\_}proof} frequently, especially before
\texttt{delete}.

\texttt{delete floating{\char`\_}hypotheses} will delete all sections of
the proof that branch off of \texttt{\$f}\index{\texttt{\$f} statement}
statements.  It is sometimes useful to do this before an
\texttt{initialize} command to recover from an error.  Note that once a
proof step with a \texttt{\$f} hypothesis as the target is completely
known, the \texttt{improve} command can usually fill in the proof for
that step.  Unlike the deletion of logical steps, \texttt{delete
floating{\char`\_}hypotheses} is a relatively safe command that is
usually easy to recover from.



\subsection{\texttt{improve} Command}\index{\texttt{improve} command}
\label{improve}
Syntax:  \texttt{improve} {\em step} [\texttt{/depth} {\em number}]
                                               [\texttt{/no{\char`\_}distinct}]

   and:   \texttt{improve first} [\texttt{/depth} {\em number}]
                                              [\texttt{/no{\char`\_}distinct}]

   and:   \texttt{improve last} [\texttt{/depth} {\em number}]
                                              [\texttt{/no{\char`\_}distinct}]

   and:   \texttt{improve all} [\texttt{/depth} {\em number}]
                                              [\texttt{/no{\char`\_}distinct}]

These commands, available in the Proof Assistant\index{Proof Assistant}
only, try to find proofs automatically for unknown steps whose symbol
sequences are completely known.  They are primarily useful for filling in
proofs of \texttt{\$f}\index{\texttt{\$f} statement} hypotheses.  The
search will be restricted to statements having no
\texttt{\$e}\index{\texttt{\$e} statement} hypotheses.

\begin{sloppypar} % narrow
Note:  If memory is limited, \texttt{improve all} on a large proof may
overflow memory.  If you use \texttt{set unification{\char`\_}timeout 1}
before \texttt{improve all}, there will usually be sufficient
improvement to easily recover and completely \texttt{improve} the proof
later on a larger computer.  Warning:  Once memory has overflowed, there
is no recovery.  If in doubt, save the intermediate proof (\texttt{save
new{\char`\_}proof} then \texttt{write source}) before \texttt{improve
all}.
\end{sloppypar}

If \texttt{last} is specified instead of {\em step} number, the last
step that is shown by \texttt{show new{\char`\_}proof /unknown} will be
used.

If \texttt{first} is specified instead of {\em step} number, the first
step that is shown by \texttt{show new{\char`\_}proof /unknown} will be
used.

If {\em step} is zero or negative, the -{\em step}th from last unknown
step, as shown by \texttt{show new{\char`\_}proof /unknown}, will be
used.  \texttt{improve -1} will use the penultimate
unknown step, \texttt{improve -2} {\em label} the antepenultimate, and
\texttt{improve 0} is the same as \texttt{improve last}.

Optional command qualifier:

    \texttt{/depth} {\em number} - This qualifier will cause the search
        to include
        statements with \texttt{\$e} hypotheses (but no new variables in
        the \texttt{\$e}
        hypotheses), provided that the backtracking has not exceeded the
        specified depth. {\em Warning:}  Try \texttt{/depth 1},
        then \texttt{2}, then \texttt{3}, etc.
        in sequence because of possible exponential blowups.  Save your
        work before trying \texttt{/depth} greater than \texttt{1}!

    \texttt{/no{\char`\_}distinct} - Skip trial statements that have
        \texttt{\$d}\index{\texttt{\$d} statement} requirements.
        This qualifier will prevent assignments that might violate \texttt{\$d}
        requirements but it also could miss possible legal assignments.

See also: \texttt{set search{\char`\_}limit}

\subsection{\texttt{save new\_proof} Command}\index{\texttt{save
new{\char`\_}proof} command}
Syntax:  \texttt{save new{\char`\_}proof} {\em label} [\texttt{/normal}]
   [\texttt{/compressed}]

The \texttt{save new{\char`\_}proof} command is available in the Proof
Assistant only.  It saves the proof in progress in the source
buffer\index{source buffer}.  \texttt{save new{\char`\_}proof} may be
used to save a completed proof, or it may be used to save a proof in
progress in order to work on it later.  If an incomplete proof is saved,
any user assignments with \texttt{let step} or \texttt{let variable}
will be lost, as will any ambiguous unifications\index{ambiguous
unification}\index{unification!ambiguous} that were resolved manually.
To help make recovery easier, it can be helpful to \texttt{improve all}
before \texttt{save new{\char`\_}proof} so that the incomplete proof
will have as much information as possible.

Note that the proof will not be permanently saved until a \texttt{write
source} command is issued.

Optional command qualifiers:

    \texttt{/normal} - The proof is saved in the normal format (i.e., as a
        sequence of labels, which is the defined format of the basic Metamath
        language).\index{basic language}  This is the default format that
        is used if a qualifier is omitted.

    \texttt{/compressed} - The proof is saved in the compressed format, which
        reduces storage requirements for a database.
        (See Appendix~\ref{compressed}.)


\section{Creating \LaTeX\ Output}\label{texout}\index{latex@{\LaTeX}}

You can generate \LaTeX\ output given the
information in a database.
The database must already include the necessary typesetting information
(see section \ref{tcomment} for how to provide this information).

The \texttt{show statement} and \texttt{show proof} commands each have a
special \texttt{/tex} command qualifier that produces \LaTeX\ output.
(The \texttt{show statement} command also has the
\texttt{/simple{\char`\_}tex} qualifier for output that is easier to
edit by hand.)  Before you can use them, you must open a \LaTeX\ file to
which to send their output.  A typical complete session will use this
sequence of Metamath commands:

\begin{verbatim}
read set.mm
open tex example.tex
show statement a1i /tex
show proof a1i /all/lemmon/renumber/tex
show statement uneq2 /tex
show proof uneq2 /all/lemmon/renumber/tex
close tex
\end{verbatim}

See Section~\ref{mathcomments} for information on comment markup and
Appendix~\ref{ASCII} for information on how math symbol translation is
specified.

To format and print the \LaTeX\ source, you will need the \LaTeX\
program, which is standard on most Linux installations and available for
Windows.  On Linux, in order to create a {\sc pdf} file, you will
typically type at the shell prompt
\begin{verbatim}
$ pdflatex example.tex
\end{verbatim}

\subsection{\texttt{open tex} Command}\index{\texttt{open tex} command}
Syntax:  \texttt{open tex} {\em file-name} [\texttt{/no{\char`\_}header}]

This command opens a file for writing \LaTeX\
source\index{latex@{\LaTeX}} and writes a \LaTeX\ header to the file.
\LaTeX\ source can be written with the \texttt{show proof}, \texttt{show
new{\char`\_}proof}, and \texttt{show statement} commands using the
\texttt{/tex} qualifier.

The mapping to \LaTeX\ symbols is defined in a special comment
containing a \texttt{\$t} token, described in Appendix~\ref{ASCII}.

There is an optional command qualifier:

    \texttt{/no{\char`\_}header} - This qualifier prevents a standard
        \LaTeX\ header and trailer
        from being included with the output \LaTeX\ code.


\subsection{\texttt{close tex} Command}\index{\texttt{close tex} command}
Syntax:  \texttt{close tex}

This command writes a trailer to any \LaTeX\ file\index{latex@{\LaTeX}}
that was opened with \texttt{open tex} (unless
\texttt{/no{\char`\_}header} was used with \texttt{open tex}) and closes
the \LaTeX\ file.


\section{Creating {\sc HTML} Output}\label{htmlout}

You can generate {\sc html} web pages given the
information in a database.
The database must already include the necessary typesetting information
(see section \ref{tcomment} for how to provide this information).
The ability to produce {\sc html} web pages was added in Metamath version
0.07.30.

To create an {\sc html} output file(s) for \texttt{\$a} or \texttt{\$p}
statement(s), use
\begin{quote}
    \texttt{show statement} {\em label-match} \texttt{/html}
\end{quote}
The output file will be named {\em label-match}\texttt{.html}
for each match.  When {\em
label-match} has wildcard (\texttt{*}) characters, all statements with
matching labels will have {\sc html} files produced for them.  Also,
when {\em label-match} has a wildcard (\texttt{*}) character, two additional
files, \texttt{mmdefinitions.html} and \texttt{mmascii.html} will be
produced.  To produce {\em only} these two additional files, you can use
\texttt{?*}, which will not match any statement label, in place of {\em
label-match}.

There are three other qualifiers for \texttt{show statement} that also
generate {\sc HTML} code.  These are \texttt{/alt{\char`\_}html},
\texttt{/brief{\char`\_}html}, and
\texttt{/brief{\char`\_}alt{\char`\_}html}, and are described in the
next section.

The command
\begin{quote}
    \texttt{show statement} {\em label-match} \texttt{/alt{\char`\_}html}
\end{quote}
does the same as \texttt{show statement} {\em label-match} \texttt{/html},
except that the {\sc html} code for the symbols is taken from
\texttt{althtmldef} statements instead of \texttt{htmldef} statements in
the \texttt{\$t} comment.

The command
\begin{verbatim}
show statement * /brief_html
\end{verbatim}
invokes a special mode that just produces definition and theorem lists
accompanied by their symbol strings, in a format suitable for copying and
pasting into another web page (such as the tutorial pages on the
Metamath web site).

Finally, the command
\begin{verbatim}
show statement * /brief_alt_html
\end{verbatim}
does the same as \texttt{show statement * / brief{\char`\_}html}
for the alternate {\sc html}
symbol representation.

A statement's comment can include a special notation that provides a
certain amount of control over the {\sc HTML} version of the comment.  See
Section~\ref{mathcomments} (p.~\pageref{mathcomments}) for the comment
markup features.

The \texttt{write theorem{\char`\_}list} and \texttt{write bibliography}
commands, which are described below, provide as a side effect complete
error checking for all of the features described in this
section.\index{error checking}

\subsection{\texttt{write theorem\_list}
Command}\index{\texttt{write theorem{\char`\_}list} command}
Syntax:  \texttt{write theorem{\char`\_}list}
[\texttt{/theorems{\char`\_}per{\char`\_}page} {\em number}]

This command writes a list of all of the \texttt{\$a} and \texttt{\$p}
statements in the database into a web page file
 called \texttt{mmtheorems.html}.
When additional files are needed, they are called
\texttt{mmtheorems2.html}, \texttt{mmtheorems3.html}, etc.

Optional command qualifier:

    \texttt{/theorems{\char`\_}per{\char`\_}page} {\em number} -
 This qualifier specifies the number of statements to
        write per web page.  The default is 100.

{\em Note:} In version 0.07.30\index{Metamath!limitations of version
0.07.30} of Metamath, the ``Nearby theorems'' links on the individual
web pages presuppose 100 theorems per page when linking to the theorem
list pages.  Therefore the \texttt{/theorems{\char`\_}per{\char`\_}page}
qualifier, if it specifies a number other than 100, will cause the
individual web pages to be out of sync and should not be used to
generate the main theorem list for the web site.  This is expected to be
fixed in a future version.


\subsection{\texttt{write bibliography}\label{wrbib}
Command}\index{\texttt{write bibliography} command}
Syntax:  \texttt{write bibliography} {\em filename}

This command reads an existing {\sc html} bibliographic cross-reference
file, normally called \texttt{mmbiblio.html}, and updates it per the
bibliographic links in the database comments.  The file is updated
between the {\sc html} comment lines \texttt{<!--
{\char`\#}START{\char`\#} -->} and \texttt{<!-- {\char`\#}END{\char`\#}
-->}.  The original input file is renamed to {\em
filename}\texttt{{\char`\~}1}.

A bibliographic reference is indicated with the reference name
in brackets, such as  \texttt{Theorem 3.1 of
[Monk] p.\ 22}.
See Section~\ref{htmlout} (p.~\pageref{htmlout}) for
syntax details.


\subsection{\texttt{write recent\_additions}
Command}\index{\texttt{write recent{\char`\_}additions} command}
Syntax:  \texttt{write recent{\char`\_}additions} {\em filename}
[\texttt{/limit} {\em number}]

This command reads an existing ``Recent Additions'' {\sc html} file,
normally called \texttt{mmrecent.html}, and updates it with the
descriptions of the most recently added theorems to the database.
 The file is updated between
the {\sc html} comment lines \texttt{<!-- {\char`\#}START{\char`\#} -->}
and \texttt{<!-- {\char`\#}END{\char`\#} -->}.  The original input file
is renamed to {\em filename}\texttt{{\char`\~}1}.

Optional command qualifier:

    \texttt{/limit} {\em number} -
 This qualifier specifies the number of most recent theorems to
   write to the output file.  The default is 100.


\section{Text File Utilities}

\subsection{\texttt{tools} Command}\index{\texttt{tools} command}
Syntax:  \texttt{tools}

This command invokes an easy-to-use, general purpose utility for
manipulating the contents of {\sc ascii} text files.  Upon typing
\texttt{tools}, the command-line prompt will change to \texttt{TOOLS>}
until you type \texttt{exit}.  The \texttt{tools} commands can be used
to perform simple, global edits on an input/output file,
such as making a character string substitution on each line, adding a
string to each line, and so on.  A typical use of this utility is
to build a \texttt{submit} input file to perform a common operation on a
list of statements obtained from \texttt{show label} or \texttt{show
usage}.

The actions of most of the \texttt{tools} commands can also be
performed with equivalent (and more powerful) Unix shell commands, and
some users may find those more efficient.  But for Windows users or
users not comfortable with Unix, \texttt{tools} provides an
easy-to-learn alternative that is adequate for most of the
script-building tasks needed to use the Metamath program effectively.

\subsection{\texttt{help} Command (in \texttt{tools})}
Syntax:  \texttt{help}

The \texttt{help} command lists the commands available in the
\texttt{tools} utility, along with a brief description.  Each command,
in turn, has its own help, such as \texttt{help add}.  As with
Metamath's \texttt{MM>} prompt, a complete command can be entered at
once, or just the command word can be typed, causing the program to
prompt for each argument.

\vskip 1ex
\noindent Line-by-line editing commands:

  \texttt{add} - Add a specified string to each line in a file.

  \texttt{clean} - Trim spaces and tabs on each line in a file; convert
         characters.

  \texttt{delete} - Delete a section of each line in a file.

  \texttt{insert} - Insert a string at a specified column in each line of
        a file.

  \texttt{substitute} - Make a simple substitution on each line of the file.

  \texttt{tag} - Like \texttt{add}, but restricted to a range of lines.

  \texttt{swap} - Swap the two halves of each line in a file.

\vskip 1ex
\noindent Other file-processing commands:

  \texttt{break} - Break up (tokenize) a file into a list of tokens (one per
        line).

  \texttt{build} - Build a file with multiple tokens per line from a list.

  \texttt{count} - Count the occurrences in a file of a specified string.

  \texttt{number} - Create a list of numbers.

  \texttt{parallel} - Put two files in parallel.

  \texttt{reverse} - Reverse the order of the lines in a file.

  \texttt{right} - Right-justify lines in a file (useful before sorting
         numbers).

%  \texttt{tag} - Tag edit updates in a program for revision control.

  \texttt{sort} - Sort the lines in a file with key starting at
         specified string.

  \texttt{match} - Extract lines containing (or not) a specified string.

  \texttt{unduplicate} - Eliminate duplicate occurrences of lines in a file.

  \texttt{duplicate} - Extract first occurrence of any line occurring
         more than

   \ \ \    once in a file, discarding lines occurring exactly once.

  \texttt{unique} - Extract lines occurring exactly once in a file.

  \texttt{type} (10 lines) - Display the first few lines in a file.
                  Similar to Unix \texttt{head}.

  \texttt{copy} - Similar to Unix \texttt{cat} but safe (same input
         and output file allowed).

  \texttt{submit} - Run a script containing \texttt{tools} commands.

\vskip 1ex

\noindent Note:
  \texttt{unduplicate}, \texttt{duplicate}, and \texttt{unique} also
 sort the lines as a side effect.


\subsection{Using \texttt{tools} to Build Metamath \texttt{submit}
Scripts}

The \texttt{break} command is typically used to break up a series of
statement labels, such as the output of Metamath's \texttt{show usage},
into one label per line.  The other \texttt{tools} commands can then be
used to add strings before and after each statement label to specify
commands to be performed on the statement.  The \texttt{parallel}
command is useful when a statement label must be mentioned more than
once on a line.

Very often a \texttt{submit} script for Metamath will require multiple
command lines for each statement being processed.  For example, you may
want to enter the Proof Assistant, \texttt{minimize{\char`\_}with} your
latest theorem, \texttt{save} the new proof, and \texttt{exit} the Proof
Assistant.  To accomplish this, you can build a file with these four
commands for each statement on a single line, separating each command
with a designated character such as \texttt{@}.  Then at the end you can
\texttt{substitute} each \texttt{@} with \texttt{{\char`\\}n} to break
up the lines into individual command lines (see \texttt{help
substitute}).


\subsection{Example of a \texttt{tools} Session}

To give you a quick feel for the \texttt{tools} utility, we show a
simple session where we create a file \texttt{n.txt} with 3 lines, add
strings before and after each line, and display the lines on the screen.
You can experiment with the various commands to gain experience with the
\texttt{tools} utility.

\begin{verbatim}
MM> tools
Entering the Text Tools utilities.
Type HELP for help, EXIT to exit.
TOOLS> number
Output file <n.tmp>? n.txt
First number <1>?
Last number <10>? 3
Increment <1>?
TOOLS> add
Input/output file? n.txt
String to add to beginning of each line <>? This is line
String to add to end of each line <>? .
The file n.txt has 3 lines; 3 were changed.
First change is on line 1:
This is line 1.
TOOLS> type n.txt
This is line 1.
This is line 2.
This is line 3.
TOOLS> exit
Exiting the Text Tools.
Type EXIT again to exit Metamath.
MM>
\end{verbatim}



\appendix
\chapter{Sample Representations}
\label{ASCII}

This Appendix provides a sample of {\sc ASCII} representations,
their corresponding traditional mathematical symbols,
and a discussion of their meanings
in the \texttt{set.mm} database.
These are provided in order of appearance.
This is only a partial list, and new definitions are routinely added.
A complete list is available at \url{http://metamath.org}.

These {\sc ASCII} representations, along
with information on how to display them,
are defined in the \texttt{set.mm} database file inside
a special comment called a \texttt{\$t} {\em
comment}\index{\texttt{\$t} comment} or {\em typesetting
comment.}\index{typesetting comment}
A typesetting comment
is indicated by the appearance of the
two-character string \texttt{\$t} at the beginning of the comment.
For more information,
see Section~\ref{tcomment}, p.~\pageref{tcomment}.

In the following table the ``{\sc ASCII}'' column shows the {\sc ASCII}
representation,
``Symbol'' shows the mathematical symbolic display
that corresponds to that {\sc ASCII} representation, ``Labels'' shows
the key label(s) that define the representation, and
``Description'' provides a description about the symbol.
As usual, ``iff'' is short for ``if and only if.''\index{iff}
In most cases the ``{\sc ASCII}'' column only shows
the key token, but it will sometimes show a sequence of tokens
if that is necessary for clarity.

{\setlength{\extrarowsep}{4pt} % Keep rows from being too close together
\begin{longtabu}   { @{} c c l X }
\textbf{ASCII} & \textbf{Symbol} & \textbf{Labels} & \textbf{Description} \\
\endhead
\texttt{|-} & $\vdash$ & &
  ``It is provable that...'' \\
\texttt{ph} & $\varphi$ & \texttt{wph} &
  The wff (boolean) variable phi,
  conventionally the first wff variable. \\
\texttt{ps} & $\psi$ & \texttt{wps} &
  The wff (boolean) variable psi,
  conventionally the second wff variable. \\
\texttt{ch} & $\chi$ & \texttt{wch} &
  The wff (boolean) variable chi,
  conventionally the third wff variable. \\
\texttt{-.} & $\lnot$ & \texttt{wn} &
  Logical not. E.g., if $\varphi$ is true, then $\lnot \varphi$ is false. \\
\texttt{->} & $\rightarrow$ & \texttt{wi} &
  Implies, also known as material implication.
  In classical logic the expression $\varphi \rightarrow \psi$ is true
  if either $\varphi$ is false or $\psi$ is true (or both), that is,
  $\varphi \rightarrow \psi$ has the same meaning as
  $\lnot \varphi \lor \psi$ (as proven in theorem \texttt{imor}). \\
\texttt{<->} & $\leftrightarrow$ &
  \hyperref[df-bi]{\texttt{df-bi}} &
  Biconditional (aka is-equals for boolean values).
  $\varphi \leftrightarrow \psi$ is true iff
  $\varphi$ and $\psi$ have the same value. \\
\texttt{\char`\\/} & $\lor$ &
  \makecell[tl]{{\hyperref[df-or]{\texttt{df-or}}}, \\
	         \hyperref[df-3or]{\texttt{df-3or}}} &
  Disjunction (logical ``or''). $\varphi \lor \psi$ is true iff
  $\varphi$, $\psi$, or both are true. \\
\texttt{/\char`\\} & $\land$ &
  \makecell[tl]{{\hyperref[df-an]{\texttt{df-an}}}, \\
                 \hyperref[df-3an]{\texttt{df-3an}}} &
  Conjunction (logical ``and''). $\varphi \land \psi$ is true iff
  both $\varphi$ and $\psi$ are true. \\
\texttt{A.} & $\forall$ &
  \texttt{wal} &
  For all; the wff $\forall x \varphi$ is true iff
  $\varphi$ is true for all values of $x$. \\
\texttt{E.} & $\exists$ &
  \hyperref[df-ex]{\texttt{df-ex}} &
  There exists; the wff
  $\exists x \varphi$ is true iff
  there is at least one $x$ where $\varphi$ is true. \\
\texttt{[ y / x ]} & $[ y / x ]$ &
  \hyperref[df-sb]{\texttt{df-sb}} &
  The wff $[ y / x ] \varphi$ produces
  the result when $y$ is properly substituted for $x$ in $\varphi$
  ($y$ replaces $x$).
  % This is elsb4
  % ( [ x / y ] z e. y <-> z e. x )
  For example,
  $[ x / y ] z \in y$ is the same as $z \in x$. \\
\texttt{E!} & $\exists !$ &
  \hyperref[df-eu]{\texttt{df-eu}} &
  There exists exactly one;
  $\exists ! x \varphi$ is true iff
  there is at least one $x$ where $\varphi$ is true. \\
\texttt{\{ y | phi \}}  & $ \{ y | \varphi \}$ &
  \hyperref[df-clab]{\texttt{df-clab}} &
  The class of all sets where $\varphi$ is true. \\
\texttt{=} & $ = $ &
  \hyperref[df-cleq]{\texttt{df-cleq}} &
  Class equality; $A = B$ iff $A$ equals $B$. \\
\texttt{e.} & $ \in $ &
  \hyperref[df-clel]{\texttt{df-clel}} &
  Class membership; $A \in B$ if $A$ is a member of $B$. \\
\texttt{{\char`\_}V} & {\rm V} &
  \hyperref[df-v]{\texttt{df-v}} &
  Class of all sets (not itself a set). \\
\texttt{C\_} & $ \subseteq $ &
  \hyperref[df-ss]{\texttt{df-ss}} &
  Subclass (subset); $A \subseteq B$ is true iff
  $A$ is a subclass of $B$. \\
\texttt{u.} & $ \cup $ &
  \hyperref[df-un]{\texttt{df-un}} &
  $A \cup B$ is the union of classes $A$ and $B$. \\
\texttt{i^i} & $ \cap $ &
  \hyperref[df-in]{\texttt{df-in}} &
  $A \cap B$ is the intersection of classes $A$ and $B$. \\
\texttt{\char`\\} & $ \setminus $ &
  \hyperref[df-dif]{\texttt{df-dif}} &
  $A \setminus B$ (set difference)
  is the class of all sets in $A$ except for those in $B$. \\
\texttt{(/)} & $ \varnothing $ &
  \hyperref[df-nul]{\texttt{df-nul}} &
  $ \varnothing $ is the empty set (aka null set). \\
\texttt{\char`\~P} & $ \cal P $ &
  \hyperref[df-pw]{\texttt{df-pw}} &
  Power class. \\
\texttt{<.\ A , B >.} & $\langle A , B \rangle$ &
  \hyperref[df-op]{\texttt{df-op}} &
  The ordered pair $\langle A , B \rangle$. \\
\texttt{( F ` A )} & $ ( F ` A ) $ &
  \hyperref[df-fv]{\texttt{df-fv}} &
  The value of function $F$ when applied to $A$. \\
\texttt{\_i} & $ i $ &
  \texttt{df-i} &
  The square root of negative one. \\
\texttt{x.} & $ \cdot $ &
  \texttt{df-mul} &
  Complex number multiplication; $2~\cdot~3~=~6$. \\
\texttt{CC} & $ \mathbb{C} $ &
  \texttt{df-c} &
  The set of complex numbers. \\
\texttt{RR} & $ \mathbb{R} $ &
  \texttt{df-r} &
  The set of real numbers. \\
\end{longtabu}
} % end of extrarowsep

\chapter{Compressed Proofs}
\label{compressed}\index{compressed proof}\index{proof!compressed}

The proofs in the \texttt{set.mm} set theory database are stored in compressed
format for efficiency.  Normally you needn't concern yourself with the
compressed format, since you can display it with the usual proof display tools
in the Metamath program (\texttt{show proof}\ldots) or convert it to the normal
RPN proof format described in Section~\ref{proof} (with \texttt{save proof}
{\em label} \texttt{/normal}).  However for sake of completeness we describe the
format here and show how it maps to the normal RPN proof format.

A compressed proof, located between \texttt{\$=} and \texttt{\$.}\ keywords, consists
of a left parenthesis, a sequence of statement labels, a right parenthesis,
and a sequence of upper-case letters \texttt{A} through \texttt{Z} (with optional
white space between them).  White space must surround the parentheses
and the labels.  The left parenthesis tells Metamath that a
compressed proof follows.  (A normal RPN proof consists of just a sequence of
labels, and a parenthesis is not a legal character in a label.)

The sequence of upper-case letters corresponds to a sequence of integers
with the following mapping.  Each integer corresponds to a proof step as
described later.
\begin{center}
  \texttt{A} = 1 \\
  \texttt{B} = 2 \\
   \ldots \\
  \texttt{T} = 20 \\
  \texttt{UA} = 21 \\
  \texttt{UB} = 22 \\
   \ldots \\
  \texttt{UT} = 40 \\
  \texttt{VA} = 41 \\
  \texttt{VB} = 42 \\
   \ldots \\
  \texttt{YT} = 120 \\
  \texttt{UUA} = 121 \\
   \ldots \\
  \texttt{YYT} = 620 \\
  \texttt{UUUA} = 621 \\
   etc.
\end{center}

In other words, \texttt{A} through \texttt{T} represent the
least-significant digit in base 20, and \texttt{U} through \texttt{Y}
represent zero or more most-significant digits in base 5, where the
digits start counting at 1 instead of the usual 0. With this scheme, we
don't need white space between these ``numbers.''

(In the design of the compressed proof format, only upper-case letters,
as opposed to say all non-whitespace printable {\sc ascii} characters other than
%\texttt{\$}, was chosen to make the compressed proof a little less
%displeasing to the eye, at the expense of a typical 20\% compression
\texttt{\$}, were chosen so as not to collide with most text editor
searches, at the expense of a typical 20\% compression
loss.  The base 5/base 20 grouping, as opposed to say base 6/base 19,
was chosen by experimentally determining the grouping that resulted in
best typical compression.)

The letter \texttt{Z} identifies (tags) a proof step that is identical to one
that occurs later on in the proof; it helps shorten the proof by not requiring
that identical proof steps be proved over and over again (which happens often
when building wff's).  The \texttt{Z} is placed immediately after the
least-significant digit (letters \texttt{A} through \texttt{T}) that ends the integer
corresponding to the step to later be referenced.

The integers that the upper-case letters correspond to are mapped to labels as
follows.  If the statement being proved has $m$ mandatory hypotheses, integers
1 through $m$ correspond to the labels of these hypotheses in the order shown
by the \texttt{show statement ... / full} command, i.e., the RPN order\index{RPN
order} of the mandatory
hypotheses.  Integers $m+1$ through $m+n$ correspond to the labels enclosed in
the parentheses of the compressed proof, in the order that they appear, where
$n$ is the number of those labels.  Integers $m+n+1$ on up don't directly
correspond to statement labels but point to proof steps identified with the
letter \texttt{Z}, so that these proof steps can be referenced later in the
proof.  Integer $m+n+1$ corresponds to the first step tagged with a \texttt{Z},
$m+n+2$ to the second step tagged with a \texttt{Z}, etc.  When the compressed
proof is converted to a normal proof, the entire subproof of a step tagged
with \texttt{Z} replaces the reference to that step.

For efficiency, Metamath works with compressed proofs directly, without
converting them internally to normal proofs.  In addition to the usual
error-checking, an error message is given if (1) a label in the label list in
parentheses does not refer to a previous \texttt{\$p} or \texttt{\$a} statement or a
non-mandatory hypothesis of the statement being proved and (2) a proof step
tagged with \texttt{Z} is referenced before the step tagged with the \texttt{Z}.

Just as in a normal proof under development (Section~\ref{unknown}), any step
or subproof that is not yet known may be represented with a single \texttt{?}.
White space does not have to appear between the \texttt{?}\ and the upper-case
letters (or other \texttt{?}'s) representing the remainder of the proof.

% April 1, 2004 Appendix C has been added back in with corrections.
%
% May 20, 2003 Appendix C was removed for now because there was a problem found
% by Bob Solovay
%
% Also, removed earlier \ref{formalspec} 's (3 cases above)
%
% Bob Solovay wrote on 30 Nov 2002:
%%%%%%%%%%%%% (start of email comment )
%      3. My next set of comments concern appendix C. I read this before I
% read Chapter 4. So I first noted that the system as presented in the
% Appendix lacked a certain formal property that I thought desirable. I
% then came up with a revised formal system that had this property. Upon
% reading Chapter 4, I noticed that the revised system was closer to the
% treatment in Chapter 4 than the system in Appendix C.
%
%         First a very minor correction:
%
%         On page 142 line 2: The condition that V(e) != V(f) should only be
% required of e, f in T such that e != f.
%
%         Here is a natural property [transitivity] that one would like
% the formal system to have:
%
%         Let Gamma be a set of statements. Suppose that the statement Phi
% is provable from Gamma and that the statement Psi is provable from Gamma
% \cup {Phi}. Then Psi is provable from Gamma.
%
%         I shall present an example to show that this property does not
% hold for the formal systems of Appendix C:
%
%         I write the example in metamath style:
%
% $c A B C D E $.
% $v x y
%
% ${
% tx $f A x $.
% ty $f B y $.
% ax1 $a C x y $.
% $}
%
% ${
% tx $f A x $.
% ty $f B y $.
% ax2-h1 $e C x y $.
% ax2 $a D y $.
% $}
%
% ${
% ty $f B y $.
% ax3-h1 $e D y $.
% ax3 $a E y $.
% $}
%
% $(These three axioms are Gamma $)
%
% ${
% tx $f A x $.
% ty $f B y $.
% Phi $p D y $=
% tx ty tx ty ax1 ax2 $.
% $}
%
% ${
% ty $f B y $.
% Psi $p E y $=
% ty ty Phi ax3 $.
% $}
%
%
% I omit the formal proofs of the following claims. [I will be glad to
% supply them upon request.]
%
% 1) Psi is not provable from Gamma;
%
% 2) Psi is provable from Gamma + Phi.
%
% Here "provable" refers to the formalism of Appendix C.
%
% The trouble of course is that Psi is lacking the variable declaration
%
% $f Ax $.
%
% In the Metamath system there is no trouble proving Psi. I attach a
% metamath file that shows this and which has been checked by the
% metamath program.
%
% I next want to indicate how I think the treatment in Appendix C should
% be revised so as to conform more closely to the metamath system of the
% main text. The revised system *does* have the transitivity property.
%
% We want to give revised definitions of "statement" and
% "provable". [cf. sections C.2.4. and C.2.5] Our new definitions will
% use the definitions given in Appendix C. So we take the following
% tack. We refer to the original notions as o-statement and o-provable. And
% we refer to the notions we are defining as n-statement and n-provable.
%
%         A n-statement is an o-statement in which the only variables
% that appear in the T component are mandatory.
%
%         To any o-statement we can associate its reduct which is a
% n-statement by dropping all the elements of T or D which contain
% non-mandatory variables.
%
%         An n-statement gamma is n-provable if there is an o-statement
% gamma' which has gamma as its reduct andf such that gamma' is
% o-provable.
%
%         It seems to me [though I am not completely sure on this point]
% that n-provability corresponds to metamath provability as discussed
% say in Chapter 4.
%
%         Attached to this letter is the metamath proof of Phi and Psi
% from Gamma discussed above.
%
%         I am still brooding over the question of whether metamath
% correctly formalizes set-theory. No doubt I will have some questions
% re this after my thoughts become clearer.
%%%%%%%%%%%%%%%% (end of email comment)

%%%%%%%%%%%%%%%% (start of 2nd email comment from Bob Solovay 1-Apr-04)
%
%         I hope that Appendix C is the one that gives a "formal" treatment
% of Metamath. At any rate, thats the appendix I want to comment on.
%
%         I'm going to suggest two changes in the definition.
%
%         First change (in the definition of statement): Require that the
% sets D, T, and E be finite.
%
%         Probably things are fine as you give them. But in the applications
% to the main metamath system they will always be finite, and its useful in
% thinking about things [at least for me] to stick to the finite case.
%
%         Second change:
%
%         First let me give an approximate description. Remove the dummy
% variables from the statement. Instead, include them in the proof.
%
%         More formally: Require that T consists of type declarations only
% for mandatory variables. Require that all the pairs in D consist of
% mandatory variables.
%
%         At the start of a proof we are allowed to declare a finite number
% of dummy variables [provided that none of them appear in any of the
% statements in E \cup {A}. We have to supply type declarations for all the
% dummy variables. We are allowed to add new $d statements referring to
% either the mandatory or dummy variables. But we require that no new $d
% statement references only mandatory variables.
%
%         I find this way of doing things more conceptual than the treatment
% in Appendix C. But the change [which I will use implicitly in later
% letters about doing Peano] is mainly aesthetic. I definitely claim that my
% results on doing Peano all apply to Metamath as it is presented in your
% book.
%
%         --Bob
%
%%%%%%%%%%%%%%%% (end of 2nd email comment)

%%
%% When uncommenting the below, also uncomment references above to {formalspec}
%%
\chapter{Metamath's Formal System}\label{formalspec}\index{Metamath!as a formal
system}

\section{Introduction}

\begin{quote}
  {\em Perfection is when there is no longer anything more to take away.}
    \flushright\sc Antoine de
     Saint-Exupery\footnote{\cite[p.~3-25]{Campbell}.}\\
\end{quote}\index{de Saint-Exupery, Antoine}

This appendix describes the theory behind the Metamath language in an abstract
way intended for mathematicians.  Specifically, we construct two
set-theo\-ret\-i\-cal objects:  a ``formal system'' (roughly, a set of syntax
rules, axioms, and logical rules) and its ``universe'' (roughly, the set of
theorems derivable in the formal system).  The Metamath computer language
provides us with a way to describe specific formal systems and, with the aid of
a proof provided by the user, to verify that given theorems
belong to their universes.

To understand this appendix, you need a basic knowledge of informal set theory.
It should be sufficient to understand, for example, Ch.\ 1 of Munkres' {\em
Topology} \cite{Munkres}\index{Munkres, James R.} or the
introductory set theory chapter
in many textbooks that introduce abstract mathematics. (Note that there are
minor notational differences among authors; e.g.\ Munkres uses $\subset$ instead
of our $\subseteq$ for ``subset.''  We use ``included in'' to mean ``a subset
of,'' and ``belongs to'' or ``is contained in'' to mean ``is an element of.'')
What we call a ``formal'' description here, unlike earlier, is actually an
informal description in the ordinary language of mathematicians.  However we
provide sufficient detail so that a mathematician could easily formalize it,
even in the language of Metamath itself if desired.  To understand the logic
examples at the end of this appendix, familiarity with an introductory book on
mathematical logic would be helpful.

\section{The Formal Description}

\subsection[Preliminaries]{Preliminaries\protect\footnotemark}%
\footnotetext{This section is taken mostly verbatim
from Tarski \cite[p.~63]{Tarski1965}\index{Tarski, Alfred}.}

By $\omega$ we denote the set of all natural numbers (non-negative integers).
Each natural number $n$ is identified with the set of all smaller numbers: $n =
\{ m | m < n \}$.  The formula $m < n$ is thus equivalent to the condition: $m
\in n$ and $m,n \in \omega$. In particular, 0 is the number zero and at the
same time the empty set $\varnothing$, $1=\{0\}$, $2=\{0,1\}$, etc. ${}^B A$
denotes the set of all functions on $B$ to $A$ (i.e.\ with domain $B$ and range
included in $A$).  The members of ${}^\omega A$ are what are called {\em simple
infinite sequences},\index{simple infinite sequence}
with all {\em terms}\index{term} in $A$.  In case $n \in \omega$, the
members of ${}^n A$ are referred to as {\em finite $n$-termed
sequences},\index{finite $n$-termed
sequence} again
with terms in $A$.  The consecutive terms (function values) of a finite or
infinite sequence $f$ are denoted by $f_0, f_1, \ldots ,f_n,\ldots$.  Every
finite sequence $f \in \bigcup _{n \in \omega} {}^n A$ uniquely determines the
number $n$ such that $f \in {}^n A$; $n$ is called the {\em
length}\index{length of a sequence ({$"|\ "|$})} of $f$ and
is denoted by $|f|$.  $\langle a \rangle$ is the sequence $f$ with $|f|=1$ and
$f_0=a$; $\langle a,b \rangle$ is the sequence $f$ with $|f|=2$, $f_0=a$,
$f_1=b$; etc.  Given two finite sequences $f$ and $g$, we denote by $f\frown g$
their {\em concatenation},\index{concatenation} i.e., the
finite sequence $h$ determined by the
conditions:
\begin{eqnarray*}
& |h| = |f|+|g|;&  \\
& h_n = f_n & \mbox{\ for\ } n < |f|;  \\
& h_{|f|+n} = g_n & \mbox{\ for\ } n < |g|.
\end{eqnarray*}

\subsection{Constants, Variables, and Expressions}

A formal system has a set of {\em symbols}\index{symbol!in
a formal system} denoted
by $\mbox{\em SM}$.  A
precise set-theo\-ret\-i\-cal definition of this set is unimportant; a symbol
could be considered a primitive or atomic element if we wish.  We assume this
set is divided into two disjoint subsets:  a set $\mbox{\em CN}$ of {\em
constants}\index{constant!in a formal system} and a set $\mbox{\em VR}$ of
{\em variables}.\index{variable!in a formal system}  $\mbox{\em CN}$ and
$\mbox{\em VR}$ are each assumed to consist of countably many symbols which
may be arranged in finite or simple infinite sequences $c_0, c_1, \ldots$ and
$v_0, v_1, \ldots$ respectively, without repeating terms.  We will represent
arbitrary symbols by metavariables $\alpha$, $\beta$, etc.

{\footnotesize\begin{quotation}
{\em Comment.} The variables $v_0, v_1, \ldots$ of our formal system
correspond to what are usually considered ``metavariables'' in
descriptions of specific formal systems in the literature.  Typically,
when describing a specific formal system a book will postulate a set of
primitive objects called variables, then proceed to describe their
properties using metavariables that range over them, never mentioning
again the actual variables themselves.  Our formal system does not
mention these primitive variable objects at all but deals directly with
metavariables, as its primitive objects, from the start.  This is a
subtle but key distinction you should keep in mind, and it makes our
definition of ``formal system'' somewhat different from that typically
found in the literature.  (So, the $\alpha$, $\beta$, etc.\ above are
actually ``metametavariables'' when used to represent $v_0, v_1,
\ldots$.)
\end{quotation}}

Finite sequences all terms of which are symbols are called {\em
expressions}.\index{expression!in a formal system}  $\mbox{\em EX}$ is
the set of all expressions; thus
\begin{displaymath}
\mbox{\em EX} = \bigcup _{n \in \omega} {}^n \mbox{\em SM}.
\end{displaymath}

A {\em constant-prefixed expression}\index{constant-prefixed expression}
is an expression of non-zero length
whose first term is a constant.  We denote the set of all constant-prefixed
expressions by $\mbox{\em EX}_C = \{ e \in \mbox{\em EX} | ( |e| > 0 \wedge
e_0 \in \mbox{\em CN} ) \}$.

A {\em constant-variable pair}\index{constant-variable pair}
is an expression of length 2 whose first term
is a constant and whose second term is a variable.  We denote the set of all
constant-variable pairs by $\mbox{\em EX}_2 = \{ e \in \mbox{\em EX}_C | ( |e|
= 2 \wedge e_1 \in \mbox{\em VR} ) \}$.


{\footnotesize\begin{quotation}
{\em Relationship to Metamath.} In general, the set $\mbox{\em SM}$
corresponds to the set of declared math symbols in a Metamath database, the
set $\mbox{\em CN}$ to those declared with \texttt{\$c} statements, and the set
$\mbox{\em VR}$ to those declared with \texttt{\$v} statements.  Of course a
Metamath database can only have a finite number of math symbols, whereas
formal systems in general can have an infinite number, although the number of
Metamath math symbols available is in principle unlimited.

The set $\mbox{\em EX}_C$ corresponds to the set of permissible expressions
for \texttt{\$e}, \texttt{\$a}, and \texttt{\$p} statements.  The set $\mbox{\em EX}_2$
corresponds to the set of permissible expressions for \texttt{\$f} statements.
\end{quotation}}

We denote by ${\cal V}(e)$ the set of all variables in an expression $e \in
\mbox{\em EX}$, i.e.\ the set of all $\alpha \in \mbox{\em VR}$ such that
$\alpha = e_n$ for some $n < |e|$.  We also denote (with abuse of notation) by
${\cal V}(E)$ the set of all variables in a collection of expressions $E
\subseteq \mbox{\em EX}$, i.e.\ $\bigcup _{e \in E} {\cal V}(e)$.


\subsection{Substitution}

Given a function $F$ from $\mbox{\em VR}$ to
$\mbox{\em EX}$, we
denote by $\sigma_{F}$ or just $\sigma$ the function from $\mbox{\em EX}$ to
$\mbox{\em EX}$ defined recursively for nonempty sequences by
\begin{eqnarray*}
& \sigma(<\alpha>) = F(\alpha) & \mbox{for\ } \alpha \in \mbox{\em VR}; \\
& \sigma(<\alpha>) = <\alpha> & \mbox{for\ } \alpha \not\in \mbox{\em VR}; \\
& \sigma(g \frown h) = \sigma(g) \frown
    \sigma(h) & \mbox{for\ } g,h \in \mbox{\em EX}.
\end{eqnarray*}
We also define $\sigma(\varnothing)=\varnothing$.  We call $\sigma$ a {\em
simultaneous substitution}\index{substitution!variable}\index{variable
substitution} (or just {\em substitution}) with {\em substitution
map}\index{substitution map} $F$.

We also denote (with abuse of notation) by $\sigma(E)$ a substitution on a
collection of expressions $E \subseteq \mbox{\em EX}$, i.e.\ the set $\{
\sigma(e) | e \in E \}$.  The collection $\sigma(E)$ may of course contain
fewer expressions than $E$ because duplicate expressions could result from the
substitution.

\subsection{Statements}

We denote by $\mbox{\em DV}$ the set of all
unordered pairs $\{\alpha, \beta \} \subseteq \mbox{\em VR}$ such that $\alpha
\neq \beta$.  $\mbox{\em DV}$ stands for ``distinct variables.''

A {\em pre-statement}\index{pre-statement!in a formal system} is a
quadruple $\langle D,T,H,A \rangle$ such that
$D\subseteq \mbox{\em DV}$, $T\subseteq \mbox{\em EX}_2$, $H\subseteq
\mbox{\em EX}_C$ and $H$ is finite,
$A\in \mbox{\em EX}_C$, ${\cal V}(H\cup\{A\}) \subseteq
{\cal V}(T)$, and $\forall e,f\in T {\ } {\cal V}(e) \neq {\cal V}(f)$ (or
equivalently, $e_1 \ne f_1$) whenever $e \neq f$. The terms of the quadruple are called {\em
distinct-variable restrictions},\index{disjoint-variable restriction!in a
formal system} {\em variable-type hypotheses},\index{variable-type
hypothesis!in a formal system} {\em logical hypotheses},\index{logical
hypothesis!in a formal system} and the {\em assertion}\index{assertion!in a
formal system} respectively.  We denote by $T_M$ ({\em mandatory variable-type
hypotheses}\index{mandatory variable-type hypothesis!in a formal system}) the
subset of $T$ such that ${\cal V}(T_M) ={\cal V}(H \cup \{A\})$.  We denote by
$D_M=\{\{\alpha,\beta\}\in D|\{\alpha,\beta\}\subseteq {\cal V}(T_M)\}$ the
{\em mandatory distinct-variable restrictions}\index{mandatory
disjoint-variable restriction!in a formal system} of the pre-statement.
The set
of {\em mandatory hypotheses}\index{mandatory hypothesis!in a formal system}
is $T_M\cup H$.  We call the quadruple $\langle D_M,T_M,H,A \rangle$
the {\em reduct}\index{reduct!in a formal system} of
the pre-statement $\langle D,T,H,A \rangle$.

A {\em statement} is the reduct of some pre-statement\index{statement!in a
formal system}.  A statement is therefore a special kind of pre-statement;
in particular, a statement is the reduct of itself.

{\footnotesize\begin{quotation}
{\em Comment.}  $T$ is a set of expressions, each of length 2, that associate
a set of constants (``variable types'') with a set of variables.  The
condition ${\cal V}(H\cup\{A\}) \subseteq {\cal V}(T) $
means that each variable occurring in a statement's logical
hypotheses or assertion must have an associated variable-type hypothesis or
``type declaration,'' in  analogy to a computer programming language, where a
variable must be declared to be say, a string or an integer.  The requirement
that $\forall e,f\in T \, e_1 \ne f_1$ for $e\neq f$
means that each variable must be
associated with a unique constant designating its variable type; e.g., a
variable might be a ``wff'' or a ``set'' but not both.

Distinct-variable restrictions are used to specify what variable substitutions
are permissible to make for the statement to remain valid.  For example, in
the theorem scheme of set theory $\lnot\forall x\,x=y$ we may not substitute
the same variable for both $x$ and $y$.  On the other hand, the theorem scheme
$x=y\to y=x$ does not require that $x$ and $y$ be distinct, so we do not
require a distinct-variable restriction, although having one
would cause no harm other than making the scheme less general.

A mandatory variable-type hypothesis is one whose variable exists in a logical
hypothesis or the assertion.  A provable pre-statement
(defined below) may require
non-mandatory variable-type hypotheses that effectively introduce ``dummy''
variables for use in its proof.  Any number of dummy variables might
be required by a specific proof; indeed, it has been shown by H.\
Andr\'{e}ka\index{Andr{\'{e}}ka, H.} \cite{Nemeti} that there is no finite
upper bound to the number of dummy variables needed to prove an arbitrary
theorem in first-order logic (with equality) having a fixed number $n>2$ of
individual variables.  (See also the Comment on p.~\pageref{nodd}.)
For this reason we do not set a finite size bound on the collections $D$ and
$T$, although in an actual application (Metamath database) these will of
course be finite, increased to whatever size is necessary as more
proofs are added.
\end{quotation}}

{\footnotesize\begin{quotation}
{\em Relationship to Metamath.} A pre-statement of a formal system
corresponds to an extended frame in a Metamath database
(Section~\ref{frames}).  The collections $D$, $T$, and $H$ correspond
respectively to the \texttt{\$d}, \texttt{\$f}, and \texttt{\$e}
statement collections in an extended frame.  The expression $A$
corresponds to the \texttt{\$a} (or \texttt{\$p}) statement in an
extended frame.

A statement of a formal system corresponds to a frame in a Metamath
database.
\end{quotation}}

\subsection{Formal Systems}

A {\em formal system}\index{formal system} is a
triple $\langle \mbox{\em CN},\mbox{\em
VR},\Gamma\rangle$ where $\Gamma$ is a set of statements.  The members of
$\Gamma$ are called {\em axiomatic statements}.\index{axiomatic
statement!in a formal system}  Sometimes we will refer to a
formal system by just $\Gamma$ when $\mbox{\em CN}$ and $\mbox{\em VR}$ are
understood.

Given a formal system $\Gamma$, the {\em closure}\index{closure}\footnote{This
definition of closure incorporates a simplification due to
Josh Purinton.\index{Purinton, Josh}.} of a
pre-statement
$\langle D,T,H,A \rangle$ is the smallest set $C$ of expressions
such that:
%\begin{enumerate}
%  \item $T\cup H\subseteq C$; and
%  \item If for some axiomatic statement
%    $\langle D_M',T_M',H',A' \rangle \in \Gamma_A$, for
%    some $E \subseteq C$, some $F \subseteq C-T$ (where ``-'' denotes
%    set difference), and some substitution
%    $\sigma$ we have
%    \begin{enumerate}
%       \item $\sigma(T_M') = E$ (where, as above, the $M$ denotes the
%           mandatory variable-type hypotheses of $T^A$);
%       \item $\sigma(H') = F$;
%       \item for all $\{\alpha,\beta\}\in D^A$ and $\subseteq
%         {\cal V}(T_M')$, for all $\gamma\in {\cal V}(\sigma(\langle \alpha
%         \rangle))$, and for all $\delta\in  {\cal V}(\sigma(\langle \beta
%         \rangle))$, we have $\{\gamma, \delta\} \in D$;
%   \end{enumerate}
%   then $\sigma(A') \in C$.
%\end{enumerate}
\begin{list}{}{\itemsep 0.0pt}
  \item[1.] $T\cup H\subseteq C$; and
  \item[2.] If for some axiomatic statement
    $\langle D_M',T_M',H',A' \rangle \in
       \Gamma$ and for some substitution
    $\sigma$ we have
    \begin{enumerate}
       \item[a.] $\sigma(T_M' \cup H') \subseteq C$; and
       \item[b.] for all $\{\alpha,\beta\}\in D_M'$, for all $\gamma\in
         {\cal V}(\sigma(\langle \alpha
         \rangle))$, and for all $\delta\in  {\cal V}(\sigma(\langle \beta
         \rangle))$, we have $\{\gamma, \delta\} \in D$;
   \end{enumerate}
   then $\sigma(A') \in C$.
\end{list}
A pre-statement $\langle D,T,H,A
\rangle$ is {\em provable}\index{provable statement!in a formal
system} if $A\in C$ i.e.\ if its assertion belongs to its
closure.  A statement is {\em provable} if it is
the reduct of a provable pre-statement.
The {\em universe}\index{universe of a formal system}
of a formal system is
the collection of all of its provable statements.  Note that the
set of axiomatic statements $\Gamma$ in a formal system is a subset of its
universe.

{\footnotesize\begin{quotation}
{\em Comment.} The first condition in the definition of closure simply says
that the hypotheses of the pre-statement are in its closure.

Condition 2(a) says that a substitution exists that makes the
mandatory hypotheses of an axiomatic statement exactly match some members of
the closure.  This is what we explicitly demonstrate in a Metamath language
proof.

%Conditions 2(a) and 2(b) say that a substitution exists that makes the
%(mandatory) hypotheses of an axiomatic statement exactly match some members of
%the closure.  This is what we explicitly demonstrate with a Metamath language
%proof.
%
%The set of expressions $F$ in condition 2(b) excludes the variable-type
%hypotheses; this is done because non-mandatory variable-type hypotheses are
%effectively ``dropped'' as irrelevant whereas logical hypotheses must be
%retained to achieve a consistent logical system.

Condition 2(b) describes how distinct-variable restrictions in the axiomatic
statement must be met.  It means that after a substitution for two variables
that must be distinct, the resulting two expressions must either contain no
variables, or if they do, they may not have variables in common, and each pair
of any variables they do have, with one variable from each expression, must be
specified as distinct in the original statement.
\end{quotation}}

{\footnotesize\begin{quotation}
{\em Relationship to Metamath.} Axiomatic statements
 and provable statements in a formal
system correspond to the frames for \texttt{\$a} and \texttt{\$p} statements
respectively in a Metamath database.  The set of axiomatic statements is a
subset of the set of provable statements in a formal system, although in a
Metamath database a \texttt{\$a} statement is distinguished by not having a
proof.  A Metamath language proof for a \texttt{\$p} statement tells the computer
how to explicitly construct a series of members of the closure ultimately
leading to a demonstration that the assertion
being proved is in the closure.  The actual closure typically contains
an infinite number of expressions.  A formal system itself does not have
an explicit object called a ``proof'' but rather the existence of a proof
is implied indirectly by membership of an assertion in a provable
statement's closure.  We do this to make the formal system easier
to describe in the language of set theory.

We also note that once established as provable, a statement may be considered
to acquire the same status as an axiomatic statement, because if the set of
axiomatic statements is extended with a provable statement, the universe of
the formal system remains unchanged (provided that $\mbox{\em VR}$ is
infinite).
In practice, this means we can build a hierarchy of provable statements to
more efficiently establish additional provable statements.  This is
what we do in Metamath when we allow proofs to reference previous
\texttt{\$p} statements as well as previous \texttt{\$a} statements.
\end{quotation}}

\section{Examples of Formal Systems}

{\footnotesize\begin{quotation}
{\em Relationship to Metamath.} The examples in this section, except Example~2,
are for the most part exact equivalents of the development in the set
theory database \texttt{set.mm}.  You may want to compare Examples~1, 3, and 5
to Section~\ref{metaaxioms}, Example 4 to Sections~\ref{metadefprop} and
\ref{metadefpred}, and Example 6 to
Section~\ref{setdefinitions}.\label{exampleref}
\end{quotation}}

\subsection{Example~1---Propositional Calculus}\index{propositional calculus}

Classical propositional calculus can be described by the following formal
system.  We assume the set of variables is infinite.  Rather than denoting the
constants and variables by $c_0, c_1, \ldots$ and $v_0, v_1, \ldots$, for
readability we will instead use more conventional symbols, with the
understanding of course that they denote distinct primitive objects.
Also for readability we may omit commas between successive terms of a
sequence; thus $\langle \mbox{wff\ } \varphi\rangle$ denotes
$\langle \mbox{wff}, \varphi\rangle$.

Let
\begin{itemize}
  \item[] $\mbox{\em CN}=\{\mbox{wff}, \vdash, \to, \lnot, (,)\}$
  \item[] $\mbox{\em VR}=\{\varphi,\psi,\chi,\ldots\}$
  \item[] $T = \{\langle \mbox{wff\ } \varphi\rangle,
             \langle \mbox{wff\ } \psi\rangle,
             \langle \mbox{wff\ } \chi\rangle,\ldots\}$, i.e.\ those
             expressions of length 2 whose first member is $\mbox{\rm wff}$
             and whose second member belongs to $\mbox{\em VR}$.\footnote{For
convenience we let $T$ be an infinite set; the definition of a statement
permits this in principle.  Since a Metamath source file has a finite size, in
practice we must of course use appropriate finite subsets of this $T$,
specifically ones containing at least the mandatory variable-type
hypotheses.  Similarly, in the source file we introduce new variables as
required, with the understanding that a potentially infinite number of
them are available.}
\noindent Then $\Gamma$ consists of the axiomatic statements that
are the reducts of the following pre-statements:
    \begin{itemize}
      \item[] $\langle\varnothing,T,\varnothing,
               \langle \mbox{wff\ }(\varphi\to\psi)\rangle\rangle$
      \item[] $\langle\varnothing,T,\varnothing,
               \langle \mbox{wff\ }\lnot\varphi\rangle\rangle$
      \item[] $\langle\varnothing,T,\varnothing,
               \langle \vdash(\varphi\to(\psi\to\varphi))
               \rangle\rangle$
      \item[] $\langle\varnothing,T,
               \varnothing,
               \langle \vdash((\varphi\to(\psi\to\chi))\to
               ((\varphi\to\psi)\to(\varphi\to\chi)))
               \rangle\rangle$
      \item[] $\langle\varnothing,T,
               \varnothing,
               \langle \vdash((\lnot\varphi\to\lnot\psi)\to
               (\psi\to\varphi))\rangle\rangle$
      \item[] $\langle\varnothing,T,
               \{\langle\vdash(\varphi\to\psi)\rangle,
                 \langle\vdash\varphi\rangle\},
               \langle\vdash\psi\rangle\rangle$
    \end{itemize}
\end{itemize}

(For example, the reduct of $\langle\varnothing,T,\varnothing,
               \langle \mbox{wff\ }(\varphi\to\psi)\rangle\rangle$
is
\begin{itemize}
\item[] $\langle\varnothing,
\{\langle \mbox{wff\ } \varphi\rangle,
             \langle \mbox{wff\ } \psi\rangle\},
             \varnothing,
               \langle \mbox{wff\ }(\varphi\to\psi)\rangle\rangle$,
\end{itemize}
which is the first axiomatic statement.)

We call the members of $\mbox{\em VR}$ {\em wff variables} or (in the context
of first-order logic which we will describe shortly) {\em wff metavariables}.
Note that the symbols $\phi$, $\psi$, etc.\ denote actual specific members of
$\mbox{\em VR}$; they are not metavariables of our expository language (which
we denote with $\alpha$, $\beta$, etc.) but are instead (meta)constant symbols
(members of $\mbox{\em SM}$) from the point of view of our expository
language.  The equivalent system of propositional calculus described in
\cite{Tarski1965} also uses the symbols $\phi$, $\psi$, etc.\ to denote wff
metavariables, but in \cite{Tarski1965} unlike here those are metavariables of
the expository language and not primitive symbols of the formal system.

The first two statements define wffs: if $\varphi$ and $\psi$ are wffs, so is
$(\varphi \to \psi)$; if $\varphi$ is a wff, so is $\lnot\varphi$. The next
three are the axioms of propositional calculus: if $\varphi$ and $\psi$ are
wffs, then $\vdash (\varphi \to (\psi \to \varphi))$ is an (axiomatic)
theorem; etc. The
last is the rule of modus ponens: if $\varphi$ and $\psi$ are wffs, and
$\vdash (\varphi\to\psi)$ and $\vdash \varphi$ are theorems, then $\vdash
\psi$ is a theorem.

The correspondence to ordinary propositional calculus is as follows.  We
consider only provable statements of the form $\langle\varnothing,
T,\varnothing,A\rangle$ with $T$ defined as above.  The first term of the
assertion $A$ of any such statement is either ``wff'' or ``$\vdash$''.  A
statement for which the first term is ``wff'' is a {\em wff} of propositional
calculus, and one where the first term is ``$\vdash$'' is a {\em
theorem (scheme)} of propositional calculus.

The universe of this formal system also contains many other provable
statements.  Those with distinct-variable restrictions are irrelevant because
propositional calculus has no constraints on substitutions.  Those that have
logical hypotheses we call {\em inferences}\index{inference} when
the logical hypotheses are of the form
$\langle\vdash\rangle\frown w$ where $w$ is a wff (with the leading constant
term ``wff'' removed).  Inferences (other than the modus ponens rule) are not a
proper part of propositional calculus but are convenient to use when building a
hierarchy of provable statements.  A provable statement with a nonsense
hypothesis such as $\langle \to,\vdash,\lnot\rangle$, and this same expression
as its assertion, we consider irrelevant; no use can be made of it in
proving theorems, since there is no way to eliminate the nonsense hypothesis.

{\footnotesize\begin{quotation}
{\em Comment.} Our use of parentheses in the definition of a wff illustrates
how axiomatic statements should be carefully stated in a way that
ties in unambiguously with the substitutions allowed by the formal system.
There are many ways we could have defined wffs---for example, Polish
prefix notation would have allowed us to omit parentheses entirely, at
the expense of readability---but we must define them in a way that is
unambiguous.  For example, if we had omitted parentheses from the
definition of $(\varphi\to \psi)$, the wff $\lnot\varphi\to \psi$ could
be interpreted as either $\lnot(\varphi\to\psi)$ or $(\lnot\varphi\to\psi)$
and would have allowed us to prove nonsense.  Note that there is no
concept of operator binding precedence built into our formal system.
\end{quotation}}

\begin{sloppy}
\subsection{Example~2---Predicate Calculus with Equality}\index{predicate
calculus}
\end{sloppy}

Here we extend Example~1 to include predicate calculus with equality,
illustrating the use of distinct-variable restrictions.  This system is the
same as Tarski's system $\mathfrak{S}_2$ in \cite{Tarski1965} (except that the
axioms of propositional calculus are different but equivalent, and a redundant
axiom is omitted).  We extend $\mbox{\em CN}$ with the constants
$\{\mbox{var},\forall,=\}$.  We extend $\mbox{\em VR}$ with an infinite set of
{\em individual metavariables}\index{individual
metavariable} $\{x,y,z,\ldots\}$ and denote this subset
$\mbox{\em Vr}$.

We also join to $\mbox{\em CN}$ a possibly infinite set $\mbox{\em Pr}$ of {\em
predicates} $\{R,S,\ldots\}$.  We associate with $\mbox{\em Pr}$ a function
$\mbox{rnk}$ from $\mbox{\em Pr}$ to $\omega$, and for $\alpha\in \mbox{\em
Pr}$ we call $\mbox{rnk}(\alpha)$ the {\em rank} of the predicate $\alpha$,
which is simply the number of ``arguments'' that the predicate has.  (Most
applications of predicate calculus will have a finite number of predicates;
for example, set theory has the single two-argument or binary predicate $\in$,
which is usually written with its arguments surrounding the predicate symbol
rather than with the prefix notation we will use for the general case.)  As a
device to facilitate our discussion, we will let $\mbox{\em Vs}$ be any fixed
one-to-one function from $\omega$ to $\mbox{\em Vr}$; thus $\mbox{\em Vs}$ is
any simple infinite sequence of individual metavariables with no repeating
terms.

In this example we will not include the function symbols that are often part of
formalizations of predicate calculus.  Using metalogical arguments that are
beyond the scope of our discussion, it can be shown that our formalization is
equivalent when functions are introduced via appropriate definitions.

We extend the set $T$ defined in Example~1 with the expressions
$\{\langle \mbox{var\ } x\rangle,$ $ \langle \mbox{var\ } y\rangle, \langle
\mbox{var\ } z\rangle,\ldots\}$.  We extend the $\Gamma$ above
with the axiomatic statements that are the reducts of the following
pre-statements:
\begin{list}{}{\itemsep 0.0pt}
      \item[] $\langle\varnothing,T,\varnothing,
               \langle \mbox{wff\ }\forall x\,\varphi\rangle\rangle$
      \item[] $\langle\varnothing,T,\varnothing,
               \langle \mbox{wff\ }x=y\rangle\rangle$
      \item[] $\langle\varnothing,T,
               \{\langle\vdash\varphi\rangle\},
               \langle\vdash\forall x\,\varphi\rangle\rangle$
      \item[] $\langle\varnothing,T,\varnothing,
               \langle \vdash((\forall x(\varphi\to\psi)
                  \to(\forall x\,\varphi\to\forall x\,\psi))
               \rangle\rangle$
      \item[] $\langle\{\{x,\varphi\}\},T,\varnothing,
               \langle \vdash(\varphi\to\forall x\,\varphi)
               \rangle\rangle$
      \item[] $\langle\{\{x,y\}\},T,\varnothing,
               \langle \vdash\lnot\forall x\lnot x=y
               \rangle\rangle$
      \item[] $\langle\varnothing,T,\varnothing,
               \langle \vdash(x=z
                  \to(x=y\to z=y))
               \rangle\rangle$
      \item[] $\langle\varnothing,T,\varnothing,
               \langle \vdash(y=z
                  \to(x=y\to x=z))
               \rangle\rangle$
\end{list}
These are the axioms not involving predicate symbols. The first two statements
extend the definition of a wff.  The third is the rule of generalization.  The
fifth states, in effect, ``For a wff $\varphi$ and variable $x$,
$\vdash(\varphi\to\forall x\,\varphi)$, provided that $x$ does not occur in
$\varphi$.''  The sixth states ``For variables $x$ and $y$,
$\vdash\lnot\forall x\lnot x = y$, provided that $x$ and $y$ are distinct.''
(This proviso is not necessary but was included by Tarski to
weaken the axiom and still show that the system is logically complete.)

Finally, for each predicate symbol $\alpha\in \mbox{\em Pr}$, we add to
$\Gamma$ an axiomatic statement, extending the definition of wff,
that is the reduct of the following pre-statement:
\begin{displaymath}
    \langle\varnothing,T,\varnothing,
            \langle \mbox{wff},\alpha\rangle\
            \frown \mbox{\em Vs}\restriction\mbox{rnk}(\alpha)\rangle
\end{displaymath}
and for each $\alpha\in \mbox{\em Pr}$ and each $n < \mbox{rnk}(\alpha)$
we add to $\Gamma$ an equality axiom that is the reduct of the
following pre-statement:
\begin{eqnarray*}
    \lefteqn{\langle\varnothing,T,\varnothing,
            \langle
      \vdash,(,\mbox{\em Vs}_n,=,\mbox{\em Vs}_{\mbox{rnk}(\alpha)},\to,
            (,\alpha\rangle\frown \mbox{\em Vs}\restriction\mbox{rnk}(\alpha)} \\
  & & \frown
            \langle\to,\alpha\rangle\frown \mbox{\em Vs}\restriction n\frown
            \langle \mbox{\em Vs}_{\mbox{rnk}(\alpha)}\rangle \\
 & & \frown
            \mbox{\em Vs}\restriction(\mbox{rnk}(\alpha)\setminus(n+1))\frown
            \langle),)\rangle\rangle
\end{eqnarray*}
where $\restriction$ denotes function domain restriction and $\setminus$
denotes set difference.  Recall that a subscript on $\mbox{\em Vs}$
denotes one of its terms.  (In the above two axiom sets commas are placed
between successive terms of sequences to prevent ambiguity, and if you examine
them with care you will be able to distinguish those parentheses that denote
constant symbols from those of our expository language that delimit function
arguments.  Although it might have been better to use boldface for our
primitive symbols, unfortunately boldface was not available for all characters
on the \LaTeX\ system used to typeset this text.)  These seemingly forbidding
axioms can be understood by analogy to concatenation of substrings in a
computer language.  They are actually relatively simple for each specific case
and will become clearer by looking at the special case of a binary predicate
$\alpha = R$ where $\mbox{rnk}(R)=2$.  Letting $\mbox{\em Vs}$ be the sequence
$\langle x,y,z,\ldots\rangle$, the axioms we would add to $\Gamma$ for this
case would be the wff extension and two equality axioms that are the
reducts of the pre-statements:
\begin{list}{}{\itemsep 0.0pt}
      \item[] $\langle\varnothing,T,\varnothing,
               \langle \mbox{wff\ }R x y\rangle\rangle$
      \item[] $\langle\varnothing,T,\varnothing,
               \langle \vdash(x=z
                  \to(R x y \to R z y))
               \rangle\rangle$
      \item[] $\langle\varnothing,T,\varnothing,
               \langle \vdash(y=z
                  \to(R x y \to R x z))
               \rangle\rangle$
\end{list}
Study these carefully to see how the general axioms above evaluate to
them.  In practice, typically only a few special cases such as this would be
needed, and in any case the Metamath language will only permit us to describe
a finite number of predicates, as opposed to the infinite number permitted by
the formal system.  (If an infinite number should be needed for some reason,
we could not define the formal system directly in the Metamath language but
could instead define it metalogically under set theory as we
do in this appendix, and only the underlying set theory, with its single
binary predicate, would be defined directly in the Metamath language.)


{\footnotesize\begin{quotation}
{\em Comment.}  As we noted earlier, the specific variables denoted by the
symbols $x,y,z,\ldots\in \mbox{\em Vr}\subseteq \mbox{\em VR}\subseteq
\mbox{\em SM}$ in Example~2 are not the actual variables of ordinary predicate
calculus but should be thought of as metavariables ranging over them.  For
example, a distinct-variable restriction would be meaningless for actual
variables of ordinary predicate calculus since two different actual variables
are by definition distinct.  And when we talk about an arbitrary
representative $\alpha\in \mbox{\em Vr}$, $\alpha$ is a metavariable (in our
expository language) that ranges over metavariables (which are primitives of
our formal system) each of which ranges over the actual individual variables
of predicate calculus (which are never mentioned in our formal system).

The constant called ``var'' above is called \texttt{setvar} in the
\texttt{set.mm} database file, but it means the same thing.  I felt
that ``var'' is a more meaningful name in the context of predicate
calculus, whose use is not limited to set theory.  For consistency we
stick with the name ``var'' throughout this Appendix, even after set
theory is introduced.
\end{quotation}}

\subsection{Free Variables and Proper Substitution}\index{free variable}
\index{proper substitution}\index{substitution!proper}

Typical representations of mathematical axioms use concepts such
as ``free variable,'' ``bound variable,'' and ``proper substitution''
as primitive notions.
A free variable is a variable that
is not a parameter of any container expression.
A bound variable is the opposite of a free variable; it is a
a variable that has been bound in a container expression.
For example, in the expression $\forall x \varphi$ (for all $x$, $\varphi$
is true), the variable $x$
is bound within the for-all ($\forall$) expression.
It is possible to change one variable to another, and that process is called
``proper substitution.''
In most books, proper substitution has a somewhat complicated recursive
definition with multiple cases based on the occurrences of free and
bound variables.
You may consult
\cite[ch.\ 3--4]{Hamilton}\index{Hamilton, Alan G.} (as well as
many other texts) for more formal details about these terms.

Using these concepts as \texttt{primitives} creates complications
for computer implementations.

In the system of Example~2, there are no primitive notions of free variable
and proper substitution.  Tarski \cite{Tarski1965} shows that this system is
logically equivalent to the more typical textbook systems that do have these
primitive notions, if we introduce these notions with appropriate definitions
and metalogic.  We could also define axioms for such systems directly,
although the recursive definitions of free variable and proper substitution
would be messy and awkward to work with.  Instead, we mention two devices that
can be used in practice to mimic these notions.  (1) Instead of introducing
special notation to express (as a logical hypothesis) ``where $x$ is not free
in $\varphi$'' we can use the logical hypothesis $\vdash(\varphi\to\forall
x\,\varphi)$.\label{effectivelybound}\index{effectively
not free}\footnote{This is a slightly weaker requirement than ``where $x$ is
not free in $\varphi$.''  If we let $\varphi$ be $x=x$, we have the theorem
$(x=x\to\forall x\,x=x)$ which satisfies the hypothesis, even though $x$ is
free in $x=x$ .  In a case like this we say that $x$ is {\em effectively not
free}\index{effectively not free} in $x=x$, since $x=x$ is logically
equivalent to $\forall x\,x=x$ in which $x$ is bound.} (2) It can be shown
that the wff $((x=y\to\varphi)\wedge\exists x(x=y\wedge\varphi))$ (with the
usual definitions of $\wedge$ and $\exists$; see Example~4 below) is logically
equivalent to ``the wff that results from proper substitution of $y$ for $x$
in $\varphi$.''  This works whether or not $x$ and $y$ are distinct.

\subsection{Metalogical Completeness}\index{metalogical completeness}

In the system of Example~2, the
following are provable pre-statements (and their reducts are
provable statements):
\begin{eqnarray*}
      & \langle\{\{x,y\}\},T,\varnothing,
               \langle \vdash\lnot\forall x\lnot x=y
               \rangle\rangle & \\
     &  \langle\varnothing,T,\varnothing,
               \langle \vdash\lnot\forall x\lnot x=x
               \rangle\rangle &
\end{eqnarray*}
whereas the following pre-statement is not to my knowledge provable (but
in any case we will pretend it's not for sake of illustration):
\begin{eqnarray*}
     &  \langle\varnothing,T,\varnothing,
               \langle \vdash\lnot\forall x\lnot x=y
               \rangle\rangle &
\end{eqnarray*}
In other words, we can prove ``$\lnot\forall x\lnot x=y$ where $x$ and $y$ are
distinct'' and separately prove ``$\lnot\forall x\lnot x=x$'', but we can't
prove the combined general case ``$\lnot\forall x\lnot x=y$'' that has no
proviso.  Now this does not compromise logical completeness, because the
variables are really metavariables and the two provable cases together cover
all possible cases.  The third case can be considered a metatheorem whose
direct proof, using the system of Example~2, lies outside the capability of the
formal system.

Also, in the system of Example~2 the following pre-statement is not to my
knowledge provable (again, a conjecture that we will pretend to be the case):
\begin{eqnarray*}
     & \langle\varnothing,T,\varnothing,
               \langle \vdash(\forall x\, \varphi\to\varphi)
               \rangle\rangle &
\end{eqnarray*}
Instead, we can only prove specific cases of $\varphi$ involving individual
metavariables, and by induction on formula length, prove as a metatheorem
outside of our formal system the general statement above.  The details of this
proof are found in \cite{Kalish}.

There does, however, exist a system of predicate calculus in which all such
``simple metatheorems'' as those above can be proved directly, and we present
it in Example~3. A {\em simple metatheorem}\index{simple metatheorem}
is any statement of the formal
system of Example~2 where all distinct variable restrictions consist of either
two individual metavariables or an individual metavariable and a wff
metavariable, and which is provable by combining cases outside the system as
above.  A system is {\em metalogically complete}\index{metalogical
completeness} if all of its simple
metatheorems are (directly) provable statements. The precise definition of
``simple metatheorem'' and the proof of the ``metalogical completeness'' of
Example~3 is found in Remark 9.6 and Theorem 9.7 of \cite{Megill}.\index{Megill,
Norman}

\begin{sloppy}
\subsection{Example~3---Metalogically Complete Predicate
Calculus with
Equality}
\end{sloppy}

For simplicity we will assume there is one binary predicate $R$;
this system suffices for set theory, where the $R$ is of course the $\in$
predicate.  We label the axioms as they appear in \cite{Megill}.  This
system is logically equivalent to that of Example~2 (when the latter is
restricted to this single binary predicate) but is also metalogically
complete.\index{metalogical completeness}

Let
\begin{itemize}
  \item[] $\mbox{\em CN}=\{\mbox{wff}, \mbox{var}, \vdash, \to, \lnot, (,),\forall,=,R\}$.
  \item[] $\mbox{\em VR}=\{\varphi,\psi,\chi,\ldots\}\cup\{x,y,z,\ldots\}$.
  \item[] $T = \{\langle \mbox{wff\ } \varphi\rangle,
             \langle \mbox{wff\ } \psi\rangle,
             \langle \mbox{wff\ } \chi\rangle,\ldots\}\cup
       \{\langle \mbox{var\ } x\rangle, \langle \mbox{var\ } y\rangle, \langle
       \mbox{var\ }z\rangle,\ldots\}$.

\noindent Then
  $\Gamma$ consists of the reducts of the following pre-statements:
    \begin{itemize}
      \item[] $\langle\varnothing,T,\varnothing,
               \langle \mbox{wff\ }(\varphi\to\psi)\rangle\rangle$
      \item[] $\langle\varnothing,T,\varnothing,
               \langle \mbox{wff\ }\lnot\varphi\rangle\rangle$
      \item[] $\langle\varnothing,T,\varnothing,
               \langle \mbox{wff\ }\forall x\,\varphi\rangle\rangle$
      \item[] $\langle\varnothing,T,\varnothing,
               \langle \mbox{wff\ }x=y\rangle\rangle$
      \item[] $\langle\varnothing,T,\varnothing,
               \langle \mbox{wff\ }Rxy\rangle\rangle$
      \item[(C1$'$)] $\langle\varnothing,T,\varnothing,
               \langle \vdash(\varphi\to(\psi\to\varphi))
               \rangle\rangle$
      \item[(C2$'$)] $\langle\varnothing,T,
               \varnothing,
               \langle \vdash((\varphi\to(\psi\to\chi))\to
               ((\varphi\to\psi)\to(\varphi\to\chi)))
               \rangle\rangle$
      \item[(C3$'$)] $\langle\varnothing,T,
               \varnothing,
               \langle \vdash((\lnot\varphi\to\lnot\psi)\to
               (\psi\to\varphi))\rangle\rangle$
      \item[(C4$'$)] $\langle\varnothing,T,
               \varnothing,
               \langle \vdash(\forall x(\forall x\,\varphi\to\psi)\to
                 (\forall x\,\varphi\to\forall x\,\psi))\rangle\rangle$
      \item[(C5$'$)] $\langle\varnothing,T,
               \varnothing,
               \langle \vdash(\forall x\,\varphi\to\varphi)\rangle\rangle$
      \item[(C6$'$)] $\langle\varnothing,T,
               \varnothing,
               \langle \vdash(\forall x\forall y\,\varphi\to
                 \forall y\forall x\,\varphi)\rangle\rangle$
      \item[(C7$'$)] $\langle\varnothing,T,
               \varnothing,
               \langle \vdash(\lnot\varphi\to\forall x\lnot\forall x\,\varphi
                 )\rangle\rangle$
      \item[(C8$'$)] $\langle\varnothing,T,
               \varnothing,
               \langle \vdash(x=y\to(x=z\to y=z))\rangle\rangle$
      \item[(C9$'$)] $\langle\varnothing,T,
               \varnothing,
               \langle \vdash(\lnot\forall x\, x=y\to(\lnot\forall x\, x=z\to
                 (y=z\to\forall x\, y=z)))\rangle\rangle$
      \item[(C10$'$)] $\langle\varnothing,T,
               \varnothing,
               \langle \vdash(\forall x(x=y\to\forall x\,\varphi)\to
                 \varphi))\rangle\rangle$
      \item[(C11$'$)] $\langle\varnothing,T,
               \varnothing,
               \langle \vdash(\forall x\, x=y\to(\forall x\,\varphi
               \to\forall y\,\varphi))\rangle\rangle$
      \item[(C12$'$)] $\langle\varnothing,T,
               \varnothing,
               \langle \vdash(x=y\to(Rxz\to Ryz))\rangle\rangle$
      \item[(C13$'$)] $\langle\varnothing,T,
               \varnothing,
               \langle \vdash(x=y\to(Rzx\to Rzy))\rangle\rangle$
      \item[(C15$'$)] $\langle\varnothing,T,
               \varnothing,
               \langle \vdash(\lnot\forall x\, x=y\to(x=y\to(\varphi
                 \to\forall x(x=y\to\varphi))))\rangle\rangle$
      \item[(C16$'$)] $\langle\{\{x,y\}\},T,
               \varnothing,
               \langle \vdash(\forall x\, x=y\to(\varphi\to\forall x\,\varphi)
                 )\rangle\rangle$
      \item[(C5)] $\langle\{\{x,\varphi\}\},T,\varnothing,
               \langle \vdash(\varphi\to\forall x\,\varphi)
               \rangle\rangle$
      \item[(MP)] $\langle\varnothing,T,
               \{\langle\vdash(\varphi\to\psi)\rangle,
                 \langle\vdash\varphi\rangle\},
               \langle\vdash\psi\rangle\rangle$
      \item[(Gen)] $\langle\varnothing,T,
               \{\langle\vdash\varphi\rangle\},
               \langle\vdash\forall x\,\varphi\rangle\rangle$
    \end{itemize}
\end{itemize}

While it is known that these axioms are ``metalogically complete,'' it is
not known whether they are independent (i.e.\ none is
redundant) in the metalogical sense; specifically, whether any axiom (possibly
with additional non-mandatory distinct-variable restrictions, for use with any
dummy variables in its proof) is provable from the others.  Note that
metalogical independence is a weaker requirement than independence in the
usual logical sense.  Not all of the above axioms are logically independent:
for example, C9$'$ can be proved as a metatheorem from the others, outside the
formal system, by combining the possible cases of distinct variables.

\subsection{Example~4---Adding Definitions}\index{definition}
There are several ways to add definitions to a formal system.  Probably the
most proper way is to consider definitions not as part of the formal system at
all but rather as abbreviations that are part of the expository metalogic
outside the formal system.  For convenience, though, we may use the formal
system itself to incorporate definitions, adding them as axiomatic extensions
to the system.  This could be done by adding a constant representing the
concept ``is defined as'' along with axioms for it. But there is a nicer way,
at least in this writer's opinion, that introduces definitions as direct
extensions to the language rather than as extralogical primitive notions.  We
introduce additional logical connectives and provide axioms for them.  For
systems of logic such as Examples 1 through 3, the additional axioms must be
conservative in the sense that no wff of the original system that was not a
theorem (when the initial term ``wff'' is replaced by ``$\vdash$'' of course)
becomes a theorem of the extended system.  In this example we extend Example~3
(or 2) with standard abbreviations of logic.

We extend $\mbox{\em CN}$ of Example~3 with new constants $\{\leftrightarrow,
\wedge,\vee,\exists\}$, corresponding to logical equivalence,\index{logical
equivalence ($\leftrightarrow$)}\index{biconditional ($\leftrightarrow$)}
conjunction,\index{conjunction ($\wedge$)} disjunction,\index{disjunction
($\vee$)} and the existential quantifier.\index{existential quantifier
($\exists$)}  We extend $\Gamma$ with the axiomatic statements that are
the reducts of the following pre-statements:
\begin{list}{}{\itemsep 0.0pt}
      \item[] $\langle\varnothing,T,\varnothing,
               \langle \mbox{wff\ }(\varphi\leftrightarrow\psi)\rangle\rangle$
      \item[] $\langle\varnothing,T,\varnothing,
               \langle \mbox{wff\ }(\varphi\vee\psi)\rangle\rangle$
      \item[] $\langle\varnothing,T,\varnothing,
               \langle \mbox{wff\ }(\varphi\wedge\psi)\rangle\rangle$
      \item[] $\langle\varnothing,T,\varnothing,
               \langle \mbox{wff\ }\exists x\, \varphi\rangle\rangle$
  \item[] $\langle\varnothing,T,\varnothing,
     \langle\vdash ( ( \varphi \leftrightarrow \psi ) \to
     ( \varphi \to \psi ) )\rangle\rangle$
  \item[] $\langle\varnothing,T,\varnothing,
     \langle\vdash ((\varphi\leftrightarrow\psi)\to
    (\psi\to\varphi))\rangle\rangle$
  \item[] $\langle\varnothing,T,\varnothing,
     \langle\vdash ((\varphi\to\psi)\to(
     (\psi\to\varphi)\to(\varphi
     \leftrightarrow\psi)))\rangle\rangle$
  \item[] $\langle\varnothing,T,\varnothing,
     \langle\vdash (( \varphi \wedge \psi ) \leftrightarrow\neg ( \varphi
     \to \neg \psi )) \rangle\rangle$
  \item[] $\langle\varnothing,T,\varnothing,
     \langle\vdash (( \varphi \vee \psi ) \leftrightarrow (\neg \varphi
     \to \psi )) \rangle\rangle$
  \item[] $\langle\varnothing,T,\varnothing,
     \langle\vdash (\exists x \,\varphi\leftrightarrow
     \lnot \forall x \lnot \varphi)\rangle\rangle$
\end{list}
The first three logical axioms (statements containing ``$\vdash$'') introduce
and effectively define logical equivalence, ``$\leftrightarrow$''.  The last
three use ``$\leftrightarrow$'' to effectively mean ``is defined as.''

\subsection{Example~5---ZFC Set Theory}\index{ZFC set theory}

Here we add to the system of Example~4 the axioms of Zermelo--Fraenkel set
theory with Choice.  For convenience we make use of the
definitions in Example~4.

In the $\mbox{\em CN}$ of Example~4 (which extends Example~3), we replace the symbol $R$
with the symbol $\in$.
More explicitly, we remove from $\Gamma$ of Example~4 the three
axiomatic statements containing $R$ and replace them with the
reducts of the following:
\begin{list}{}{\itemsep 0.0pt}
      \item[] $\langle\varnothing,T,\varnothing,
               \langle \mbox{wff\ }x\in y\rangle\rangle$
      \item[] $\langle\varnothing,T,
               \varnothing,
               \langle \vdash(x=y\to(x\in z\to y\in z))\rangle\rangle$
      \item[] $\langle\varnothing,T,
               \varnothing,
               \langle \vdash(x=y\to(z\in x\to z\in y))\rangle\rangle$
\end{list}
Letting $D=\{\{\alpha,\beta\}\in \mbox{\em DV}\,|\alpha,\beta\in \mbox{\em
Vr}\}$ (in other words all individual variables must be distinct), we extend
$\Gamma$ with the ZFC axioms, called
\index{Axiom of Extensionality}
\index{Axiom of Replacement}
\index{Axiom of Union}
\index{Axiom of Power Sets}
\index{Axiom of Regularity}
\index{Axiom of Infinity}
\index{Axiom of Choice}
Extensionality, Replacement, Union, Power
Set, Regularity, Infinity, and Choice, that are the reducts of:
\begin{list}{}{\itemsep 0.0pt}
      \item[Ext] $\langle D,T,
               \varnothing,
               \langle\vdash (\forall x(x\in y\leftrightarrow x \in z)\to y
               =z) \rangle\rangle$
      \item[Rep] $\langle D,T,
               \varnothing,
               \langle\vdash\exists x ( \exists y \forall z (\varphi \to z = y
                        ) \to
                        \forall z ( z \in x \leftrightarrow \exists x ( x \in
                        y \wedge \forall y\,\varphi ) ) )\rangle\rangle$
      \item[Un] $\langle D,T,
               \varnothing,
               \langle\vdash \exists x \forall y ( \exists x ( y \in x \wedge
               x \in z ) \to y \in x ) \rangle\rangle$
      \item[Pow] $\langle D,T,
               \varnothing,
               \langle\vdash \exists x \forall y ( \forall x ( x \in y \to x
               \in z ) \to y \in x ) \rangle\rangle$
      \item[Reg] $\langle D,T,
               \varnothing,
               \langle\vdash (  x \in y \to
                 \exists x ( x \in y \wedge \forall z ( z \in x \to \lnot z
                \in y ) ) ) \rangle\rangle$
      \item[Inf] $\langle D,T,
               \varnothing,
               \langle\vdash \exists x(y\in x\wedge\forall y(y\in
               x\to
               \exists z(y \in z\wedge z\in x))) \rangle\rangle$
      \item[AC] $\langle D,T,
               \varnothing,
               \langle\vdash \exists x \forall y \forall z ( ( y \in z
               \wedge z \in w ) \to \exists w \forall y ( \exists w
              ( ( y \in z \wedge z \in w ) \wedge ( y \in w \wedge w \in x
              ) ) \leftrightarrow y = w ) ) \rangle\rangle$
\end{list}

\subsection{Example~6---Class Notation in Set Theory}\label{class}

A powerful device that makes set theory easier (and that we have
been using all along in our informal expository language) is {\em class
abstraction notation}.\index{class abstraction}\index{abstraction class}  The
definitions we introduce are rigorously justified
as conservative by Takeuti and Zaring \cite{Takeuti}\index{Takeuti, Gaisi} or
Quine \cite{Quine}\index{Quine, Willard Van Orman}.  The key idea is to
introduce the notation $\{x|\mbox{---}\}$ which means ``the class of all $x$
such that ---'' for abstraction classes and introduce (meta)variables that
range over them.  An abstraction class may or may not be a set, depending on
whether it exists (as a set).  A class that does not exist is
called a {\em proper class}.\index{proper class}\index{class!proper}

To illustrate the use of abstraction classes we will provide some examples
of definitions that make use of them:  the empty set, class union, and
unordered pair.  Many other such definitions can be found in the
Metamath set theory database,
\texttt{set.mm}.\index{set theory database (\texttt{set.mm})}

% We intentionally break up the sequence of math symbols here
% because otherwise the overlong line goes beyond the page in narrow mode.
We extend $\mbox{\em CN}$ of Example~5 with new symbols $\{$
$\mbox{class},$ $\{,$ $|,$ $\},$ $\varnothing,$ $\cup,$ $,$ $\}$
where the inner braces and last comma are
constant symbols. (As before,
our dual use of some mathematical symbols for both our expository
language and as primitives of the formal system should be clear from context.)

We extend $\mbox{\em VR}$ of Example~5 with a set of {\em class
variables}\index{class variable}
$\{A,B,C,\ldots\}$. We extend the $T$ of Example~5 with $\{\langle
\mbox{class\ } A\rangle, \langle \mbox{class\ }B\rangle, \langle \mbox{class\ }
C\rangle,\ldots\}$.

To
introduce our definitions,
we add to $\Gamma$ of Example~5 the axiomatic statements
that are the reducts of the following pre-statements:
\begin{list}{}{\itemsep 0.0pt}
      \item[] $\langle\varnothing,T,\varnothing,
               \langle \mbox{class\ }x\rangle\rangle$
      \item[] $\langle\varnothing,T,\varnothing,
               \langle \mbox{class\ }\{x|\varphi\}\rangle\rangle$
      \item[] $\langle\varnothing,T,\varnothing,
               \langle \mbox{wff\ }A=B\rangle\rangle$
      \item[] $\langle\varnothing,T,\varnothing,
               \langle \mbox{wff\ }A\in B\rangle\rangle$
      \item[Ab] $\langle\varnothing,T,\varnothing,
               \langle \vdash ( y \in \{ x |\varphi\} \leftrightarrow
                  ( ( x = y \to\varphi) \wedge \exists x ( x = y
                  \wedge\varphi) ))
               \rangle\rangle$
      \item[Eq] $\langle\{\{x,A\},\{x,B\}\},T,\varnothing,
               \langle \vdash ( A = B \leftrightarrow
               \forall x ( x \in A \leftrightarrow x \in B ) )
               \rangle\rangle$
      \item[El] $\langle\{\{x,A\},\{x,B\}\},T,\varnothing,
               \langle \vdash ( A \in B \leftrightarrow \exists x
               ( x = A \wedge x \in B ) )
               \rangle\rangle$
\end{list}
Here we say that an individual variable is a class; $\{x|\varphi\}$ is a
class; and we extend the definition of a wff to include class equality and
membership.  Axiom Ab defines membership of a variable in a class abstraction;
the right-hand side can be read as ``the wff that results from proper
substitution of $y$ for $x$ in $\varphi$.''\footnote{Note that this definition
makes unnecessary the introduction of a separate notation similar to
$\varphi(x|y)$ for proper substitution, although we may choose to do so to be
conventional.  Incidentally, $\varphi(x|y)$ as it stands would be ambiguous in
the formal systems of our examples, since we wouldn't know whether
$\lnot\varphi(x|y)$ meant $\lnot(\varphi(x|y))$ or $(\lnot\varphi)(x|y)$.
Instead, we would have to use an unambiguous variant such as $(\varphi\,
x|y)$.}  Axioms Eq and El extend the meaning of the existing equality and
membership connectives.  This is potentially dangerous and requires careful
justification.  For example, from Eq we can derive the Axiom of Extensionality
with predicate logic alone; thus in principle we should include the Axiom of
Extensionality as a logical hypothesis.  However we do not bother to do this
since we have already presupposed that axiom earlier. The distinct variable
restrictions should be read ``where $x$ does not occur in $A$ or $B$.''  We
typically do this when the right-hand side of a definition involves an
individual variable not in the expression being defined; it is done so that
the right-hand side remains independent of the particular ``dummy'' variable
we use.

We continue to add to $\Gamma$ the following definitions
(i.e. the reducts of the following pre-statements) for empty
set,\index{empty set} class union,\index{union} and unordered
pair.\index{unordered pair}  They should be self-explanatory.  Analogous to our
use of ``$\leftrightarrow$'' to define new wffs in Example~4, we use ``$=$''
to define new abstraction terms, and both may be read informally as ``is
defined as'' in this context.
\begin{list}{}{\itemsep 0.0pt}
      \item[] $\langle\varnothing,T,\varnothing,
               \langle \mbox{class\ }\varnothing\rangle\rangle$
      \item[] $\langle\varnothing,T,\varnothing,
               \langle \vdash \varnothing = \{ x | \lnot x = x \}
               \rangle\rangle$
      \item[] $\langle\varnothing,T,\varnothing,
               \langle \mbox{class\ }(A\cup B)\rangle\rangle$
      \item[] $\langle\{\{x,A\},\{x,B\}\},T,\varnothing,
               \langle \vdash ( A \cup B ) = \{ x | ( x \in A \vee x \in B ) \}
               \rangle\rangle$
      \item[] $\langle\varnothing,T,\varnothing,
               \langle \mbox{class\ }\{A,B\}\rangle\rangle$
      \item[] $\langle\{\{x,A\},\{x,B\}\},T,\varnothing,
               \langle \vdash \{ A , B \} = \{ x | ( x = A \vee x = B ) \}
               \rangle\rangle$
\end{list}

\section{Metamath as a Formal System}\label{theorymm}

This section presupposes a familiarity with the Metamath computer language.

Our theory describes formal systems and their universes.  The Metamath
language provides a way of representing these set-theoretical objects to
a computer.  A Metamath database, being a finite set of {\sc ascii}
characters, can usually describe only a subset of a formal system and
its universe, which are typically infinite.  However the database can
contain as large a finite subset of the formal system and its universe
as we wish.  (Of course a Metamath set theory database can, in
principle, indirectly describe an entire infinite formal system by
formalizing the expository language in this Appendix.)

For purpose of our discussion, we assume the Metamath database
is in the simple form described on p.~\pageref{framelist},
consisting of all constant and variable declarations at the beginning,
followed by a sequence of extended frames each
delimited by \texttt{\$\char`\{} and \texttt{\$\char`\}}.  Any Metamath database can
be converted to this form, as described on p.~\pageref{frameconvert}.

The math symbol tokens of a Metamath source file, which are declared
with \texttt{\$c} and \texttt{\$v} statements, are names we assign to
representatives of $\mbox{\em CN}$ and $\mbox{\em VR}$.  For
definiteness we could assume that the first math symbol declared as a
variable corresponds to $v_0$, the second to $v_1$, etc., although the
exact correspondence we choose is not important.

In the Metamath language, each \texttt{\$d}, \texttt{\$f}, and
 \texttt{\$e} source
statement in an extended frame (Section~\ref{frames})
corresponds respectively to a member of the
collections $D$, $T$, and $H$ in a formal system statement $\langle
D_M,T_M,H,A\rangle$.  The math symbol strings following these Metamath keywords
correspond to a variable pair (in the case of \texttt{\$d}) or an expression (for
the other two keywords). The math symbol string following a \texttt{\$a} source
statement corresponds to expression $A$ in an axiomatic statement of the
formal system; the one following a \texttt{\$p} source statement corresponds to
$A$ in a provable statement that is not axiomatic.  In other words, each
extended frame in a Metamath database corresponds to
a pre-statement of the formal system, and a frame corresponds to
a statement of the formal system.  (Don't confuse the two meanings of
``statement'' here.  A statement of the formal system corresponds to the
several statements in a Metamath database that may constitute a
frame.)

In order for the computer to verify that a formal system statement is
provable, each \texttt{\$p} source statement is accompanied by a proof.
However, the proof does not correspond to anything in the formal system
but is simply a way of communicating to the computer the information
needed for its verification.  The proof tells the computer {\em how to
construct} specific members of closure of the formal system
pre-statement corresponding to the extended frame of the \texttt{\$p}
statement.  The final result of the construction is the member of the
closure that matches the \texttt{\$p} statement.  The abstract formal
system, on the other hand, is concerned only with the {\em existence} of
members of the closure.

As mentioned on p.~\pageref{exampleref}, Examples 1 and 3--6 in the
previous Section parallel the development of logic and set theory in the
Metamath database
\texttt{set.mm}.\index{set theory database (\texttt{set.mm})} You may
find it instructive to compare them.


\chapter{The MIU System}
\label{MIU}
\index{formal system}
\index{MIU-system}

The following is a listing of the file \texttt{miu.mm}.  It is self-explanatory.

%%%%%%%%%%%%%%%%%%%%%%%%%%%%%%%%%%%%%%%%%%%%%%%%%%%%%%%%%%%%

\begin{verbatim}
$( The MIU-system:  A simple formal system $)

$( Note:  This formal system is unusual in that it allows
empty wffs.  To work with a proof, you must type
SET EMPTY_SUBSTITUTION ON before using the PROVE command.
By default, this is OFF in order to reduce the number of
ambiguous unification possibilities that have to be selected
during the construction of a proof.  $)

$(
Hofstadter's MIU-system is a simple example of a formal
system that illustrates some concepts of Metamath.  See
Douglas R. Hofstadter, _Goedel, Escher, Bach:  An Eternal
Golden Braid_ (Vintage Books, New York, 1979), pp. 33ff. for
a description of the MIU-system.

The system has 3 constant symbols, M, I, and U.  The sole
axiom of the system is MI. There are 4 rules:
     Rule I:  If you possess a string whose last letter is I,
     you can add on a U at the end.
     Rule II:  Suppose you have Mx.  Then you may add Mxx to
     your collection.
     Rule III:  If III occurs in one of the strings in your
     collection, you may make a new string with U in place
     of III.
     Rule IV:  If UU occurs inside one of your strings, you
     can drop it.
Unfortunately, Rules III and IV do not have unique results:
strings could have more than one occurrence of III or UU.
This requires that we introduce the concept of an "MIU
well-formed formula" or wff, which allows us to construct
unique symbol sequences to which Rules III and IV can be
applied.
$)

$( First, we declare the constant symbols of the language.
Note that we need two symbols to distinguish the assertion
that a sequence is a wff from the assertion that it is a
theorem; we have arbitrarily chosen "wff" and "|-". $)
      $c M I U |- wff $. $( Declare constants $)

$( Next, we declare some variables. $)
     $v x y $.

$( Throughout our theory, we shall assume that these
variables represent wffs. $)
 wx   $f wff x $.
 wy   $f wff y $.

$( Define MIU-wffs.  We allow the empty sequence to be a
wff. $)

$( The empty sequence is a wff. $)
 we   $a wff $.
$( "M" after any wff is a wff. $)
 wM   $a wff x M $.
$( "I" after any wff is a wff. $)
 wI   $a wff x I $.
$( "U" after any wff is a wff. $)
 wU   $a wff x U $.

$( Assert the axiom. $)
 ax   $a |- M I $.

$( Assert the rules. $)
 ${
   Ia   $e |- x I $.
$( Given any theorem ending with "I", it remains a theorem
if "U" is added after it.  (We distinguish the label I_
from the math symbol I to conform to the 24-Jun-2006
Metamath spec.) $)
   I_    $a |- x I U $.
 $}
 ${
IIa  $e |- M x $.
$( Given any theorem starting with "M", it remains a theorem
if the part after the "M" is added again after it. $)
   II   $a |- M x x $.
 $}
 ${
   IIIa $e |- x I I I y $.
$( Given any theorem with "III" in the middle, it remains a
theorem if the "III" is replaced with "U". $)
   III  $a |- x U y $.
 $}
 ${
   IVa  $e |- x U U y $.
$( Given any theorem with "UU" in the middle, it remains a
theorem if the "UU" is deleted. $)
   IV   $a |- x y $.
  $}

$( Now we prove the theorem MUIIU.  You may be interested in
comparing this proof with that of Hofstadter (pp. 35 - 36).
$)
 theorem1  $p |- M U I I U $=
      we wM wU wI we wI wU we wU wI wU we wM we wI wU we wM
      wI wI wI we wI wI we wI ax II II I_ III II IV $.
\end{verbatim}\index{well-formed formula (wff)}

The \texttt{show proof /lemmon/renumber} command
yields the following display.  It is very similar
to the one in \cite[pp.~35--36]{Hofstadter}.\index{Hofstadter, Douglas R.}

\begin{verbatim}
1 ax             $a |- M I
2 1 II           $a |- M I I
3 2 II           $a |- M I I I I
4 3 I_           $a |- M I I I I U
5 4 III          $a |- M U I U
6 5 II           $a |- M U I U U I U
7 6 IV           $a |- M U I I U
\end{verbatim}

We note that Hofstadter's ``MU-puzzle,'' which asks whether
MU is a theorem of the MIU-system, cannot be answered using
the system above because the MU-puzzle is a question {\em
about} the system.  To prove the answer to the MU-puzzle,
a much more elaborate system is needed, namely one that
models the MIU-system within set theory.  (Incidentally, the
answer to the MU-puzzle is no.)

\chapter{Metamath Language EBNF}%
\label{BNF}%
\index{Metamath Language EBNF}

The following is a formal description of the basic Metamath language syntax
(with compressed proofs and support for unknown proof steps).
It is defined using the
Extended Backus--Naur Form (EBNF)\index{Extended Backus--Naur Form}\index{EBNF}
notation from W3C\index{W3C}
\textit{Extensible Markup Language (XML) 1.0 (Fifth Edition)}
(W3C Recommendation 26 November 2008) at
\url{https://www.w3.org/TR/xml/#sec-notation}.

The \texttt{database}
rule is processed until the end of the file (\texttt{EOF}).
The rules eventually require reading whitespace-separated tokens.
A token has an upper-case definition (see below)
or is a string constant in a non-token (such as \texttt{'\$a'}).
We intend for this to be correct, but if there is a conflict the
rules of section \ref{spec} govern. That section also discusses
non-syntax restrictions not shown here
(e.g., that each new label token
defined in a \texttt{hypothesis-stmt} or \texttt{assert-stmt}
must be unique).

\begin{verbatim}
database ::= outermost-scope-stmt*

outermost-scope-stmt ::=
  include-stmt | constant-stmt | stmt

/* File inclusion command; process file as a database.
   Databases should NOT have a comment in the filename. */
include-stmt ::= '$[' filename '$]'

/* Constant symbols declaration. */
constant-stmt ::= '$c' constant+ '$.'

/* A normal statement can occur in any scope. */
stmt ::= block | variable-stmt | disjoint-stmt |
  hypothesis-stmt | assert-stmt

/* A block. You can have 0 statements in a block. */
block ::= '${' stmt* '$}'

/* Variable symbols declaration. */
variable-stmt ::= '$v' variable+ '$.'

/* Disjoint variables. Simple disjoint statements have
   2 variables, i.e., "variable*" is empty for them. */
disjoint-stmt ::= '$d' variable variable variable* '$.'

hypothesis-stmt ::= floating-stmt | essential-stmt

/* Floating (variable-type) hypothesis. */
floating-stmt ::= LABEL '$f' typecode variable '$.'

/* Essential (logical) hypothesis. */
essential-stmt ::= LABEL '$e' typecode MATH-SYMBOL* '$.'

assert-stmt ::= axiom-stmt | provable-stmt

/* Axiomatic assertion. */
axiom-stmt ::= LABEL '$a' typecode MATH-SYMBOL* '$.'

/* Provable assertion. */
provable-stmt ::= LABEL '$p' typecode MATH-SYMBOL*
  '$=' proof '$.'

/* A proof. Proofs may be interspersed by comments.
   If '?' is in a proof it's an "incomplete" proof. */
proof ::= uncompressed-proof | compressed-proof
uncompressed-proof ::= (LABEL | '?')+
compressed-proof ::= '(' LABEL* ')' COMPRESSED-PROOF-BLOCK+

typecode ::= constant

filename ::= MATH-SYMBOL /* No whitespace or '$' */
constant ::= MATH-SYMBOL
variable ::= MATH-SYMBOL
\end{verbatim}

\needspace{2\baselineskip}
A \texttt{frame} is a sequence of 0 or more
\texttt{disjoint-{\allowbreak}stmt} and
\texttt{hypotheses-{\allowbreak}stmt} statements
(possibly interleaved with other non-\texttt{assert-stmt} statements)
followed by one \texttt{assert-stmt}.

\needspace{3\baselineskip}
Here are the rules for lexical processing (tokenization) beyond
the constant tokens shown above.
By convention these tokenization rules have upper-case names.
Every token is read for the longest possible length.
Whitespace-separated tokens are read sequentially;
note that the separating whitespace and \texttt{\$(} ... \texttt{\$)}
comments are skipped.

If a token definition uses another token definition, the whole thing
is considered a single token.
A pattern that is only part of a full token has a name beginning
with an underscore (``\_'').
An implementation could tokenize many tokens as a
\texttt{PRINTABLE-SEQUENCE}
and then check if it meets the more specific rule shown here.

Comments do not nest, and both \texttt{\$(} and \texttt{\$)}
have to be surrounded
by at least one whitespace character (\texttt{\_WHITECHAR}).
Technically comments end without consuming the trailing
\texttt{\_WHITECHAR}, but the trailing
\texttt{\_WHITECHAR} gets ignored anyway so we ignore that detail here.
Metamath language processors
are not required to support \texttt{\$)} followed
immediately by a bare end-of-file, because the closing
comment symbol is supposed to be followed by a
\texttt{\_WHITECHAR} such as a newline.

\begin{verbatim}
PRINTABLE-SEQUENCE ::= _PRINTABLE-CHARACTER+

MATH-SYMBOL ::= (_PRINTABLE-CHARACTER - '$')+

/* ASCII non-whitespace printable characters */
_PRINTABLE-CHARACTER ::= [#x21-#x7e]

LABEL ::= ( _LETTER-OR-DIGIT | '.' | '-' | '_' )+

_LETTER-OR-DIGIT ::= [A-Za-z0-9]

COMPRESSED-PROOF-BLOCK ::= ([A-Z] | '?')+

/* Define whitespace between tokens. The -> SKIP
   means that when whitespace is seen, it is
   skipped and we simply read again. */
WHITESPACE ::= (_WHITECHAR+ | _COMMENT) -> SKIP

/* Comments. $( ... $) and do not nest. */
_COMMENT ::= '$(' (_WHITECHAR+ (PRINTABLE-SEQUENCE - '$)')*
  _WHITECHAR+ '$)' _WHITECHAR

/* Whitespace: (' ' | '\t' | '\r' | '\n' | '\f') */
_WHITECHAR ::= [#x20#x09#x0d#x0a#x0c]
\end{verbatim}
% This EBNF was developed as a collaboration between
% David A. Wheeler\index{Wheeler, David A.},
% Mario Carneiro\index{Carneiro, Mario}, and
% Benoit Jubin\index{Jubin, Benoit}, inspired by a request
% (and a lot of initial work) by Benoit Jubin.
%
% \chapter{Disclaimer and Trademarks}
%
% Information in this document is subject to change without notice and does not
% represent a commitment on the part of Norman Megill.
% \vspace{2ex}
%
% \noindent Norman D. Megill makes no warranties, either express or implied,
% regarding the Metamath computer software package.
%
% \vspace{2ex}
%
% \noindent Any trademarks mentioned in this book are the property of
% their respective owners.  The name ``Metamath'' is a trademark of
% Norman Megill.
%
\cleardoublepage
\phantomsection  % fixes the link anchor
\addcontentsline{toc}{chapter}{\bibname}

\bibliography{metamath}
%% metamath.tex - Version of 24-May-2019
% If you change the date above, also change the "Printed date" below.
% SPDX-License-Identifier: CC0-1.0
%
%                              PUBLIC DOMAIN
%
% This file (specifically, the version of this file with the above date)
% has been released into the Public Domain per the
% Creative Commons CC0 1.0 Universal (CC0 1.0) Public Domain Dedication
% https://creativecommons.org/publicdomain/zero/1.0/
%
% The public domain release applies worldwide.  In case this is not
% legally possible, the right is granted to use the work for any purpose,
% without any conditions, unless such conditions are required by law.
%
% Several short, attributed quotations from copyrighted works
% appear in this file under the ``fair use'' provision of Section 107 of
% the United States Copyright Act (Title 17 of the {\em United States
% Code}).  The public-domain status of this file is not applicable to
% those quotations.
%
% Norman Megill - email: nm(at)alum(dot)mit(dot)edu
%
% David A. Wheeler also donates his improvements to this file to the
% public domain per the CC0.  He works at the Institute for Defense Analyses
% (IDA), but IDA has agreed that this Metamath work is outside its "lane"
% and is not a work by IDA.  This was specifically confirmed by
% Margaret E. Myers (Division Director of the Information Technology
% and Systems Division) on 2019-05-24 and by Ben Lindorf (General Counsel)
% on 2019-05-22.

% This file, 'metamath.tex', is self-contained with everything needed to
% generate the the PDF file 'metamath.pdf' (the _Metamath_ book) on
% standard LaTeX 2e installations.  The auxiliary files are embedded with
% "filecontents" commands.  To generate metamath.pdf file, run these
% commands under Linux or Cygwin in the directory that contains
% 'metamath.tex':
%
%   rm -f realref.sty metamath.bib
%   touch metamath.ind
%   pdflatex metamath
%   pdflatex metamath
%   bibtex metamath
%   makeindex metamath
%   pdflatex metamath
%   pdflatex metamath
%
% The warnings that occur in the initial runs of pdflatex can be ignored.
% For the final run,
%
%   egrep -i 'error|warn' metamath.log
%
% should show exactly these 5 warnings:
%
%   LaTeX Warning: File `realref.sty' already exists on the system.
%   LaTeX Warning: File `metamath.bib' already exists on the system.
%   LaTeX Font Warning: Font shape `OMS/cmtt/m/n' undefined
%   LaTeX Font Warning: Font shape `OMS/cmtt/bx/n' undefined
%   LaTeX Font Warning: Some font shapes were not available, defaults
%       substituted.
%
% Search for "Uncomment" below if you want to suppress hyperlink boxes
% in the PDF output file
%
% TYPOGRAPHICAL NOTES:
% * It is customary to use an en dash (--) to "connect" names of different
%   people (and to denote ranges), and use a hyphen (-) for a
%   single compound name. Examples of connected multiple people are
%   Zermelo--Fraenkel, Schr\"{o}der--Bernstein, Tarski--Grothendieck,
%   Hewlett--Packard, and Backus--Naur.  Examples of a single person with
%   a compound name include Levi-Civita, Mittag-Leffler, and Burali-Forti.
% * Use non-breaking spaces after page abbreviations, e.g.,
%   p.~\pageref{note2002}.
%
% --------------------------- Start of realref.sty -----------------------------
\begin{filecontents}{realref.sty}
% Save the following as realref.sty.
% You can then use it with \usepackage{realref}
%
% This has \pageref jumping to the page on which the ref appears,
% \ref jumping to the point of the anchor, and \sectionref
% jumping to the start of section.
%
% Author:  Anthony Williams
%          Software Engineer
%          Nortel Networks Optical Components Ltd
% Date:    9 Nov 2001 (posted to comp.text.tex)
%
% The following declaration was made by Anthony Williams on
% 24 Jul 2006 (private email to Norman Megill):
%
%   ``I hereby donate the code for realref.sty posted on the
%   comp.text.tex newsgroup on 9th November 2001, accessible from
%   http://groups.google.com/group/comp.text.tex/msg/5a0e1cc13ea7fbb2
%   to the public domain.''
%
\ProvidesPackage{realref}
\RequirePackage[plainpages=false,pdfpagelabels=true]{hyperref}
\def\realref@anchorname{}
\AtBeginDocument{%
% ensure every label is a possible hyperlink target
\let\realref@oldrefstepcounter\refstepcounter%
\DeclareRobustCommand{\refstepcounter}[1]{\realref@oldrefstepcounter{#1}
\edef\realref@anchorname{\string #1.\@currentlabel}%
}%
\let\realref@oldlabel\label%
\DeclareRobustCommand{\label}[1]{\realref@oldlabel{#1}\hypertarget{#1}{}%
\@bsphack\protected@write\@auxout{}{%
    \string\expandafter\gdef\protect\csname
    page@num.#1\string\endcsname{\thepage}%
    \string\expandafter\gdef\protect\csname
    ref@num.#1\string\endcsname{\@currentlabel}%
    \string\expandafter\gdef\protect\csname
    sectionref@name.#1\string\endcsname{\realref@anchorname}%
}\@esphack}%
\DeclareRobustCommand\pageref[1]{{\edef\a{\csname
            page@num.#1\endcsname}\expandafter\hyperlink{page.\a}{\a}}}%
\DeclareRobustCommand\ref[1]{{\edef\a{\csname
            ref@num.#1\endcsname}\hyperlink{#1}{\a}}}%
\DeclareRobustCommand\sectionref[1]{{\edef\a{\csname
            ref@num.#1\endcsname}\edef\b{\csname
            sectionref@name.#1\endcsname}\hyperlink{\b}{\a}}}%
}
\end{filecontents}
% ---------------------------- End of realref.sty ------------------------------

% --------------------------- Start of metamath.bib -----------------------------
\begin{filecontents}{metamath.bib}
@book{Albers, editor = "Donald J. Albers and G. L. Alexanderson",
  title = "Mathematical People",
  publisher = "Contemporary Books, Inc.",
  address = "Chicago",
  note = "[QA28.M37]",
  year = 1985 }
@book{Anderson, author = "Alan Ross Anderson and Nuel D. Belnap",
  title = "Entailment",
  publisher = "Princeton University Press",
  address = "Princeton",
  volume = 1,
  note = "[QA9.A634 1975 v.1]",
  year = 1975}
@book{Barrow, author = "John D. Barrow",
  title = "Theories of Everything:  The Quest for Ultimate Explanation",
  publisher = "Oxford University Press",
  address = "Oxford",
  note = "[Q175.B225]",
  year = 1991 }
@book{Behnke,
  editor = "H. Behnke and F. Backmann and K. Fladt and W. S{\"{u}}ss",
  title = "Fundamentals of Mathematics",
  volume = "I",
  publisher = "The MIT Press",
  address = "Cambridge, Massachusetts",
  note = "[QA37.2.B413]",
  year = 1974 }
@book{Bell, author = "J. L. Bell and M. Machover",
  title = "A Course in Mathematical Logic",
  publisher = "North-Holland",
  address = "Amsterdam",
  note = "[QA9.B3953]",
  year = 1977 }
@inproceedings{Blass, author = "Andrea Blass",
  title = "The Interaction Between Category Theory and Set Theory",
  pages = "5--29",
  booktitle = "Mathematical Applications of Category Theory (Proceedings
     of the Special Session on Mathematical Applications
     Category Theory, 89th Annual Meeting of the American Mathematical
     Society, held in Denver, Colorado January 5--9, 1983)",
  editor = "John Walter Gray",
  year = 1983,
  note = "[QA169.A47 1983]",
  publisher = "American Mathematical Society",
  address = "Providence, Rhode Island"}
@proceedings{Bledsoe, editor = "W. W. Bledsoe and D. W. Loveland",
  title = "Automated Theorem Proving:  After 25 Years (Proceedings
     of the Special Session on Automatic Theorem Proving,
     89th Annual Meeting of the American Mathematical
     Society, held in Denver, Colorado January 5--9, 1983)",
  year = 1983,
  note = "[QA76.9.A96.S64 1983]",
  publisher = "American Mathematical Society",
  address = "Providence, Rhode Island" }
@book{Boolos, author = "George S. Boolos and Richard C. Jeffrey",
  title = "Computability and Log\-ic",
  publisher = "Cambridge University Press",
  edition = "third",
  address = "Cambridge",
  note = "[QA9.59.B66 1989]",
  year = 1989 }
@book{Campbell, author = "John Campbell",
  title = "Programmer's Progress",
  publisher = "White Star Software",
  address = "Box 51623, Palo Alto, CA 94303",
  year = 1991 }
@article{DBLP:journals/corr/Carneiro14,
  author    = {Mario Carneiro},
  title     = {Conversion of {HOL} Light proofs into Metamath},
  journal   = {CoRR},
  volume    = {abs/1412.8091},
  year      = {2014},
  url       = {http://arxiv.org/abs/1412.8091},
  archivePrefix = {arXiv},
  eprint    = {1412.8091},
  timestamp = {Mon, 13 Aug 2018 16:47:05 +0200},
  biburl    = {https://dblp.org/rec/bib/journals/corr/Carneiro14},
  bibsource = {dblp computer science bibliography, https://dblp.org}
}
@article{CarneiroND,
  author    = {Mario Carneiro},
  title     = {Natural Deductions in the Metamath Proof Language},
  url       = {http://us.metamath.org/ocat/natded.pdf},
  year      = 2014
}
@inproceedings{Chou, author = "Shang-Ching Chou",
  title = "Proving Elementary Geometry Theorems Using {W}u's Algorithm",
  pages = "243--286",
  booktitle = "Automated Theorem Proving:  After 25 Years (Proceedings
     of the Special Session on Automatic Theorem Proving,
     89th Annual Meeting of the American Mathematical
     Society, held in Denver, Colorado January 5--9, 1983)",
  editor = "W. W. Bledsoe and D. W. Loveland",
  year = 1983,
  note = "[QA76.9.A96.S64 1983]",
  publisher = "American Mathematical Society",
  address = "Providence, Rhode Island" }
@book{Clemente, author = "Daniel Clemente Laboreo",
  title = "Introduction to natural deduction",
  year = 2014,
  url = "http://www.danielclemente.com/logica/dn.en.pdf" }
@incollection{Courant, author = "Richard Courant and Herbert Robbins",
  title = "Topology",
  pages = "573--590",
  booktitle = "The World of Mathematics, Volume One",
  editor = "James R. Newman",
  publisher = "Simon and Schuster",
  address = "New York",
  note = "[QA3.W67 1988]",
  year = 1956 }
@book{Curry, author = "Haskell B. Curry",
  title = "Foundations of Mathematical Logic",
  publisher = "Dover Publications, Inc.",
  address = "New York",
  note = "[QA9.C976 1977]",
  year = 1977 }
@book{Davis, author = "Philip J. Davis and Reuben Hersh",
  title = "The Mathematical Experience",
  publisher = "Birkh{\"{a}}user Boston",
  address = "Boston",
  note = "[QA8.4.D37 1982]",
  year = 1981 }
@incollection{deMillo,
  author = "Richard de Millo and Richard Lipton and Alan Perlis",
  title = "Social Processes and Proofs of Theorems and Programs",
  pages = "267--285",
  booktitle = "New Directions in the Philosophy of Mathematics",
  editor = "Thomas Tymoczko",
  publisher = "Birkh{\"{a}}user Boston, Inc.",
  address = "Boston",
  note = "[QA8.6.N48 1986]",
  year = 1986 }
@book{Edwards, author = "Robert E. Edwards",
  title = "A Formal Background to Mathematics",
  publisher = "Springer-Verlag",
  address = "New York",
  note = "[QA37.2.E38 v.1a]",
  year = 1979 }
@book{Enderton, author = "Herbert B. Enderton",
  title = "Elements of Set Theory",
  publisher = "Academic Press, Inc.",
  address = "San Diego",
  note = "[QA248.E5]",
  year = 1977 }
@book{Goodstein, author = "R. L. Goodstein",
  title = "Development of Mathematical Logic",
  publisher = "Springer-Verlag New York Inc.",
  address = "New York",
  note = "[QA9.G6554]",
  year = 1971 }
@book{Guillen, author = "Michael Guillen",
  title = "Bridges to Infinity",
  publisher = "Jeremy P. Tarcher, Inc.",
  address = "Los Angeles",
  note = "[QA93.G8]",
  year = 1983 }
@book{Hamilton, author = "Alan G. Hamilton",
  title = "Logic for Mathematicians",
  edition = "revised",
  publisher = "Cambridge University Press",
  address = "Cambridge",
  note = "[QA9.H298]",
  year = 1988 }
@unpublished{Harrison, author = "John Robert Harrison",
  title = "Metatheory and Reflection in Theorem Proving:
    A Survey and Critique",
  note = "Technical Report
    CRC-053.
    SRI Cambridge,
    Millers Yard, Cambridge, UK,
    1995.
    Available on the Web as
{\verb+http:+}\-{\verb+//www.cl.cam.ac.uk/users/jrh/papers/reflect.html+}"}
@TECHREPORT{Harrison-thesis,
        author          = "John Robert Harrison",
        title           = "Theorem Proving with the Real Numbers",
        institution   = "University of Cambridge Computer
                         Lab\-o\-ra\-to\-ry",
        address         = "New Museums Site, Pembroke Street, Cambridge,
                           CB2 3QG, UK",
        year            = 1996,
        number          = 408,
        type            = "Technical Report",
        note            = "Author's PhD thesis,
   available on the Web at
{\verb+http:+}\-{\verb+//www.cl.cam.ac.uk+}\-{\verb+/users+}\-{\verb+/jrh+}%
\-{\verb+/papers+}\-{\verb+/thesis.html+}"}
@book{Herrlich, author = "Horst Herrlich and George E. Strecker",
  title = "Category Theory:  An Introduction",
  publisher = "Allyn and Bacon Inc.",
  address = "Boston",
  note = "[QA169.H567]",
  year = 1973 }
@article{Hindley, author = "J. Roger Hindley and David Meredith",
  title = "Principal Type-Schemes and Condensed Detachment",
  journal = "The Journal of Symbolic Logic",
  volume = 55,
  year = 1990,
  note = "[QA.J87]",
  pages = "90--105" }
@book{Hofstadter, author = "Douglas R. Hofstadter",
  title = "G{\"{o}}del, Escher, Bach",
  publisher = "Basic Books, Inc.",
  address = "New York",
  note = "[QA9.H63 1980]",
  year = 1979 }
@article{Indrzejczak, author= "Andrzej Indrzejczak",
  title = "Natural Deduction, Hybrid Systems and Modal Logic",
  journal = "Trends in Logic",
  volume = 30,
  publisher = "Springer",
  year = 2010 }
@article{Kalish, author = "D. Kalish and R. Montague",
  title = "On {T}arski's Formalization of Predicate Logic with Identity",
  journal = "Archiv f{\"{u}}r Mathematische Logik und Grundlagenfor\-schung",
  volume = 7,
  year = 1965,
  note = "[QA.A673]",
  pages = "81--101" }
@article{Kalman, author = "J. A. Kalman",
  title = "Condensed Detachment as a Rule of Inference",
  journal = "Studia Logica",
  volume = 42,
  number = 4,
  year = 1983,
  note = "[B18.P6.S933]",
  pages = "443-451" }
@book{Kline, author = "Morris Kline",
  title = "Mathematical Thought from Ancient to Modern Times",
  publisher = "Oxford University Press",
  address = "New York",
  note = "[QA21.K516 1990 v.3]",
  year = 1972 }
@book{Klinel, author = "Morris Kline",
  title = "Mathematics, The Loss of Certainty",
  publisher = "Oxford University Press",
  address = "New York",
  note = "[QA21.K525]",
  year = 1980 }
@book{Kramer, author = "Edna E. Kramer",
  title = "The Nature and Growth of Modern Mathematics",
  publisher = "Princeton University Press",
  address = "Princeton, New Jersey",
  note = "[QA93.K89 1981]",
  year = 1981 }
@article{Knill, author = "Oliver Knill",
  title = "Some Fundamental Theorems in Mathematics",
  year = "2018",
  url = "https://arxiv.org/abs/1807.08416" }
@book{Landau, author = "Edmund Landau",
  title = "Foundations of Analysis",
  publisher = "Chelsea Publishing Company",
  address = "New York",
  edition = "second",
  note = "[QA241.L2541 1960]",
  year = 1960 }
@article{Leblanc, author = "Hugues Leblanc",
  title = "On {M}eyer and {L}ambert's Quantificational Calculus {FQ}",
  journal = "The Journal of Symbolic Logic",
  volume = 33,
  year = 1968,
  note = "[QA.J87]",
  pages = "275--280" }
@article{Lejewski, author = "Czeslaw Lejewski",
  title = "On Implicational Definitions",
  journal = "Studia Logica",
  volume = 8,
  year = 1958,
  note = "[B18.P6.S933]",
  pages = "189--208" }
@book{Levy, author = "Azriel Levy",
  title = "Basic Set Theory",
  publisher = "Dover Publications",
  address = "Mineola, NY",
  year = "2002"
}
@book{Margaris, author = "Angelo Margaris",
  title = "First Order Mathematical Logic",
  publisher = "Blaisdell Publishing Company",
  address = "Waltham, Massachusetts",
  note = "[QA9.M327]",
  year = 1967}
@book{Manin, author = "Yu I. Manin",
  title = "A Course in Mathematical Logic",
  publisher = "Springer-Verlag",
  address = "New York",
  note = "[QA9.M29613]",
  year = "1977" }
@article{Mathias, author = "Adrian R. D. Mathias",
  title = "A Term of Length 4,523,659,424,929",
  journal = "Synthese",
  volume = 133,
  year = 2002,
  note = "[Q.S993]",
  pages = "75--86" }
@article{Megill, author = "Norman D. Megill",
  title = "A Finitely Axiomatized Formalization of Predicate Calculus
     with Equality",
  journal = "Notre Dame Journal of Formal Logic",
  volume = 36,
  year = 1995,
  note = "[QA.N914]",
  pages = "435--453" }
@unpublished{Megillc, author = "Norman D. Megill",
  title = "A Shorter Equivalent of the Axiom of Choice",
  month = "June",
  note = "Unpublished",
  year = 1991 }
@article{MegillBunder, author = "Norman D. Megill and Martin W.
    Bunder",
  title = "Weaker {D}-Complete Logics",
  journal = "Journal of the IGPL",
  volume = 4,
  year = 1996,
  pages = "215--225",
  note = "Available on the Web at
{\verb+http:+}\-{\verb+//www.mpi-sb.mpg.de+}\-{\verb+/igpl+}%
\-{\verb+/Journal+}\-{\verb+/V4-2+}\-{\verb+/#Megill+}"}
}
@book{Mendelson, author = "Elliott Mendelson",
  title = "Introduction to Mathematical Logic",
  edition = "second",
  publisher = "D. Van Nostrand Company, Inc.",
  address = "New York",
  note = "[QA9.M537 1979]",
  year = 1979 }
@article{Meredith, author = "David Meredith",
  title = "In Memoriam {C}arew {A}rthur {M}eredith (1904-1976)",
  journal = "Notre Dame Journal of Formal Logic",
  volume = 18,
  year = 1977,
  note = "[QA.N914]",
  pages = "513--516" }
@article{CAMeredith, author = "C. A. Meredith",
  title = "Single Axioms for the Systems ({C},{N}), ({C},{O}) and ({A},{N})
      of the Two-Valued Propositional Calculus",
  journal = "The Journal of Computing Systems",
  volume = 3,
  year = 1953,
  pages = "155--164" }
@article{Monk, author = "J. Donald Monk",
  title = "Provability With Finitely Many Variables",
  journal = "The Journal of Symbolic Logic",
  volume = 27,
  year = 1971,
  note = "[QA.J87]",
  pages = "353--358" }
@article{Monks, author = "J. Donald Monk",
  title = "Substitutionless Predicate Logic With Identity",
  journal = "Archiv f{\"{u}}r Mathematische Logik und Grundlagenfor\-schung",
  volume = 7,
  year = 1965,
  pages = "103--121" }
  %% Took out this from above to prevent LaTeX underfull warning:
  % note = "[QA.A673]",
@book{Moore, author = "A. W. Moore",
  title = "The Infinite",
  publisher = "Routledge",
  address = "New York",
  note = "[BD411.M59]",
  year = 1989}
@book{Munkres, author = "James R. Munkres",
  title = "Topology: A First Course",
  publisher = "Prentice-Hall, Inc.",
  address = "Englewood Cliffs, New Jersey",
  note = "[QA611.M82]",
  year = 1975}
@article{Nemesszeghy, author = "E. Z. Nemesszeghy and E. A. Nemesszeghy",
  title = "On Strongly Creative Definitions:  A Reply to {V}. {F}. {R}ickey",
  journal = "Logique et Analyse (N.\ S.)",
  year = 1977,
  volume = 20,
  note = "[BC.L832]",
  pages = "111--115" }
@unpublished{Nemeti, author = "N{\'{e}}meti, I.",
  title = "Algebraizations of Quantifier Logics, an Overview",
  note = "Version 11.4, preprint, Mathematical Institute, Budapest,
    1994.  A shortened version without proofs appeared in
    ``Algebraizations of quantifier logics, an introductory overview,''
   {\em Studia Logica}, 50:485--569, 1991 [B18.P6.S933]"}
@article{Pavicic, author = "M. Pavi{\v{c}}i{\'{c}}",
  title = "A New Axiomatization of Unified Quantum Logic",
  journal = "International Journal of Theoretical Physics",
  year = 1992,
  volume = 31,
  note = "[QC.I626]",
  pages = "1753 --1766" }
@book{Penrose, author = "Roger Penrose",
  title = "The Emperor's New Mind",
  publisher = "Oxford University Press",
  address = "New York",
  note = "[Q335.P415]",
  year = 1989 }
@book{PetersonI, author = "Ivars Peterson",
  title = "The Mathematical Tourist",
  publisher = "W. H. Freeman and Company",
  address = "New York",
  note = "[QA93.P475]",
  year = 1988 }
@article{Peterson, author = "Jeremy George Peterson",
  title = "An automatic theorem prover for substitution and detachment systems",
  journal = "Notre Dame Journal of Formal Logic",
  volume = 19,
  year = 1978,
  note = "[QA.N914]",
  pages = "119--122" }
@book{Quine, author = "Willard Van Orman Quine",
  title = "Set Theory and Its Logic",
  edition = "revised",
  publisher = "The Belknap Press of Harvard University Press",
  address = "Cambridge, Massachusetts",
  note = "[QA248.Q7 1969]",
  year = 1969 }
@article{Robinson, author = "J. A. Robinson",
  title = "A Machine-Oriented Logic Based on the Resolution Principle",
  journal = "Journal of the Association for Computing Machinery",
  year = 1965,
  volume = 12,
  pages = "23--41" }
@article{RobinsonT, author = "T. Thacher Robinson",
  title = "Independence of Two Nice Sets of Axioms for the Propositional
    Calculus",
  journal = "The Journal of Symbolic Logic",
  volume = 33,
  year = 1968,
  note = "[QA.J87]",
  pages = "265--270" }
@book{Rucker, author = "Rudy Rucker",
  title = "Infinity and the Mind:  The Science and Philosophy of the
    Infinite",
  publisher = "Bantam Books, Inc.",
  address = "New York",
  note = "[QA9.R79 1982]",
  year = 1982 }
@book{Russell, author = "Bertrand Russell",
  title = "Mysticism and Logic, and Other Essays",
  publisher = "Barnes \& Noble Books",
  address = "Totowa, New Jersey",
  note = "[B1649.R963.M9 1981]",
  year = 1981 }
@article{Russell2, author = "Bertrand Russell",
  title = "Recent Work on the Principles of Mathematics",
  journal = "International Monthly",
  volume = 4,
  year = 1901,
  pages = "84"}
@article{Schmidt, author = "Eric Schmidt",
  title = "Reductions in Norman Megill's axiom system for complex numbers",
  url = "http://us.metamath.org/downloads/schmidt-cnaxioms.pdf",
  year = "2012" }
@book{Shoenfield, author = "Joseph R. Shoenfield",
  title = "Mathematical Logic",
  publisher = "Addison-Wesley Publishing Company, Inc.",
  address = "Reading, Massachusetts",
  year = 1967,
  note = "[QA9.S52]" }
@book{Smullyan, author = "Raymond M. Smullyan",
  title = "Theory of Formal Systems",
  publisher = "Princeton University Press",
  address = "Princeton, New Jersey",
  year = 1961,
  note = "[QA248.5.S55]" }
@book{Solow, author = "Daniel Solow",
  title = "How to Read and Do Proofs:  An Introduction to Mathematical
    Thought Process",
  publisher = "John Wiley \& Sons",
  address = "New York",
  year = 1982,
  note = "[QA9.S577]" }
@book{Stark, author = "Harold M. Stark",
  title = "An Introduction to Number Theory",
  publisher = "Markham Publishing Company",
  address = "Chicago",
  note = "[QA241.S72 1978]",
  year = 1970 }
@article{Swart, author = "E. R. Swart",
  title = "The Philosophical Implications of the Four-Color Problem",
  journal = "American Mathematical Monthly",
  year = 1980,
  volume = 87,
  month = "November",
  note = "[QA.A5125]",
  pages = "697--707" }
@book{Szpiro, author = "George G. Szpiro",
  title = "Poincar{\'{e}}'s Prize: The Hundred-Year Quest to Solve One
    of Math's Greatest Puzzles",
  publisher = "Penguin Books Ltd",
  address = "London",
  note = "[QA43.S985 2007]",
  year = 2007}
@book{Takeuti, author = "Gaisi Takeuti and Wilson M. Zaring",
  title = "Introduction to Axiomatic Set Theory",
  edition = "second",
  publisher = "Springer-Verlag New York Inc.",
  address = "New York",
  note = "[QA248.T136 1982]",
  year = 1982}
@inproceedings{Tarski, author = "Alfred Tarski",
  title = "What is Elementary Geometry",
  pages = "16--29",
  booktitle = "The Axiomatic Method, with Special Reference to Geometry and
     Physics (Proceedings of an International Symposium held at the University
     of California, Berkeley, December 26, 1957 --- January 4, 1958)",
  editor = "Leon Henkin and Patrick Suppes and Alfred Tarski",
  year = 1959,
  publisher = "North-Holland Publishing Company",
  address = "Amsterdam"}
@article{Tarski1965, author = "Alfred Tarski",
  title = "A Simplified Formalization of Predicate Logic with Identity",
  journal = "Archiv f{\"{u}}r Mathematische Logik und Grundlagenforschung",
  volume = 7,
  year = 1965,
  note = "[QA.A673]",
  pages = "61--79" }
@book{Tymoczko,
  title = "New Directions in the Philosophy of Mathematics",
  editor = "Thomas Tymoczko",
  publisher = "Birkh{\"{a}}user Boston, Inc.",
  address = "Boston",
  note = "[QA8.6.N48 1986]",
  year = 1986 }
@incollection{Wang,
  author = "Hao Wang",
  title = "Theory and Practice in Mathematics",
  pages = "129--152",
  booktitle = "New Directions in the Philosophy of Mathematics",
  editor = "Thomas Tymoczko",
  publisher = "Birkh{\"{a}}user Boston, Inc.",
  address = "Boston",
  note = "[QA8.6.N48 1986]",
  year = 1986 }
@manual{Webster,
  title = "Webster's New Collegiate Dictionary",
  organization = "G. \& C. Merriam Co.",
  address = "Springfield, Massachusetts",
  note = "[PE1628.W4M4 1977]",
  year = 1977 }
@manual{Whitehead, author = "Alfred North Whitehead",
  title = "An Introduction to Mathematics",
  year = 1911 }
@book{PM, author = "Alfred North Whitehead and Bertrand Russell",
  title = "Principia Mathematica",
  edition = "second",
  publisher = "Cambridge University Press",
  address = "Cambridge",
  year = "1927",
  note = "(3 vols.) [QA9.W592 1927]" }
@article{DBLP:journals/corr/Whalen16,
  author    = {Daniel Whalen},
  title     = {Holophrasm: a neural Automated Theorem Prover for higher-order logic},
  journal   = {CoRR},
  volume    = {abs/1608.02644},
  year      = {2016},
  url       = {http://arxiv.org/abs/1608.02644},
  archivePrefix = {arXiv},
  eprint    = {1608.02644},
  timestamp = {Mon, 13 Aug 2018 16:46:19 +0200},
  biburl    = {https://dblp.org/rec/bib/journals/corr/Whalen16},
  bibsource = {dblp computer science bibliography, https://dblp.org} }
@article{Wiedijk-revisited,
  author = {Freek Wiedijk},
  title = {The QED Manifesto Revisited},
  year = {2007},
  url = {http://mizar.org/trybulec65/8.pdf} }
@book{Wolfram,
  author = "Stephen Wolfram",
  title = "Mathematica:  A System for Doing Mathematics by Computer",
  edition = "second",
  publisher = "Addison-Wesley Publishing Co.",
  address = "Redwood City, California",
  note = "[QA76.95.W65 1991]",
  year = 1991 }
@book{Wos, author = "Larry Wos and Ross Overbeek and Ewing Lusk and Jim Boyle",
  title = "Automated Reasoning:  Introduction and Applications",
  edition = "second",
  publisher = "McGraw-Hill, Inc.",
  address = "New York",
  note = "[QA76.9.A96.A93 1992]",
  year = 1992 }

%
%
%[1] Church, Alonzo, Introduction to Mathematical Logic,
% Volume 1, Princeton University Press, Princeton, N. J., 1956.
%
%[2] Cohen, Paul J., Set Theory and the Continuum Hypothesis,
% W. A. Benjamin, Inc., Reading, Mass., 1966.
%
%[3] Hamilton, Alan G., Logic for Mathematicians, Cambridge
% University Press,
% Cambridge, 1988.

%[6] Kleene, Stephen Cole, Introduction to Metamathematics, D.  Van
% Nostrand Company, Inc., Princeton (1952).

%[13] Tarski, Alfred, "A simplified formalization of predicate
% logic with identity," Archiv fur Mathematische Logik und
% Grundlagenforschung, vol. 7 (1965), pp. 61-79.

%[14] Tarski, Alfred and Steven Givant, A Formalization of Set
% Theory Without Variables, American Mathematical Society Colloquium
% Publications, vol. 41, American Mathematical Society,
% Providence, R. I., 1987.

%[15] Zeman, J. J., Modal Logic, Oxford University Press, Oxford, 1973.
\end{filecontents}
% --------------------------- End of metamath.bib -----------------------------


%Book: Metamath
%Author:  Norman Megill Email:  nm at alum.mit.edu
%Author:  David A. Wheeler Email:  dwheeler at dwheeler.com

% A book template example
% http://www.stsci.edu/ftp/software/tex/bookstuff/book.template

\documentclass[leqno]{book} % LaTeX 2e. 10pt. Use [leqno,12pt] for 12pt
% hyperref 2002/05/27 v6.72r  (couldn't get pagebackref to work)
\usepackage[plainpages=false,pdfpagelabels=true]{hyperref}

\usepackage{needspace}     % Enable control over page breaks
\usepackage{breqn}         % automatic equation breaking
\usepackage{microtype}     % microtypography, reduces hyphenation

% Packages for flexible tables.  We need to be able to
% wrap text within a cell (with automatically-determined widths) AND
% split a table automatically across multiple pages.
% * "tabularx" wraps text in cells but only 1 page
% * "longtable" goes across pages but by itself is incompatible with tabularx
% * "ltxtable" combines longtable and tabularx, but table contents
%    must be in a separate file.
% * "ltablex" combines tabularx and longtable - must install specially
% * "booktabs" is recommended as a way to improve the look of tables,
%   but doesn't add these capabilities.
% * "tabu" much more capable and seems to be recommended. So use that.

\usepackage{makecell}      % Enable forced line splits within a table cell
\usepackage{longtable}     % Enable multi-page tables
\usepackage{tabu}          % Multi-page tables with wrapped text in a cell

% You can find more Tex packages using commands like:
% tlmgr search --file tabu.sty
% find /usr/share/texmf-dist/ -name '*tab*'
%
%%%%%%%%%%%%%%%%%%%%%%%%%%%%%%%%%%%%%%%%%%%%%%%%%%%%%%%%%%%%%%%%%%%%%%%%%%%%
% Uncomment the next 3 lines to suppress boxes and colors on the hyperlinks
%%%%%%%%%%%%%%%%%%%%%%%%%%%%%%%%%%%%%%%%%%%%%%%%%%%%%%%%%%%%%%%%%%%%%%%%%%%%
%\hypersetup{
%colorlinks,citecolor=black,filecolor=black,linkcolor=black,urlcolor=black
%}
%
\usepackage{realref}

% Restarting page numbers: try?
%   \printglossary
%   \cleardoublepage
%   \pagenumbering{arabic}
%   \setcounter{page}{1}    ???needed
%   \include{chap1}

% not used:
% \def\R2Lurl#1#2{\mbox{\href{#1}\texttt{#2}}}

\usepackage{amssymb}

% Version 1 of book: margins: t=.4, b=.2, ll=.4, rr=.55
% \usepackage{anysize}
% % \papersize{<height>}{<width>}
% % \marginsize{<left>}{<right>}{<top>}{<bottom>}
% \papersize{9in}{6in}
% % l/r 0.6124-0.6170 works t/b 0.2418-0.3411 = 192pp. 0.2926-03118=exact
% \marginsize{0.7147in}{0.5147in}{0.4012in}{0.2012in}

\usepackage{anysize}
% \papersize{<height>}{<width>}
% \marginsize{<left>}{<right>}{<top>}{<bottom>}
\papersize{9in}{6in}
% l/r 0.85in&0.6431-0.6539 works t/b ?-?
%\marginsize{0.85in}{0.6485in}{0.55in}{0.35in}
\marginsize{0.8in}{0.65in}{0.5in}{0.3in}

% \usepackage[papersize={3.6in,4.8in},hmargin=0.1in,vmargin={0.1in,0.1in}]{geometry}  % page geometry
\usepackage{special-settings}

\raggedbottom
\makeindex

\begin{document}
% Discourage page widows and orphans:
\clubpenalty=300
\widowpenalty=300

%%%%%%% load in AMS fonts %%%%%%% % LaTeX 2.09 - obsolete in LaTeX 2e
%\input{amssym.def}
%\input{amssym.tex}
%\input{c:/texmf/tex/plain/amsfonts/amssym.def}
%\input{c:/texmf/tex/plain/amsfonts/amssym.tex}

\bibliographystyle{plain}
\pagenumbering{roman}
\pagestyle{headings}

\thispagestyle{empty}

\hfill
\vfill

\begin{center}
{\LARGE\bf Metamath} \\
\vspace{1ex}
{\large A Computer Language for Mathematical Proofs} \\
\vspace{7ex}
{\large Norman Megill} \\
\vspace{7ex}
with extensive revisions by \\
\vspace{1ex}
{\large David A. Wheeler} \\
\vspace{7ex}
% Printed date. If changing the date below, also fix the date at the beginning.
2019-05-24
\end{center}

\vfill
\hfill

\newpage
\thispagestyle{empty}

\hfill
\vfill

\begin{center}
$\sim$\ {\sc Public Domain}\ $\sim$

\vspace{2ex}
This book (including its later revisions)
has been released into the Public Domain by Norman Megill per the
Creative Commons CC0 1.0 Universal (CC0 1.0) Public Domain Dedication
(\url{https://creativecommons.org/publicdomain/zero/1.0/}).
David A. Wheeler has done the same.
The public domain release applies worldwide.  In case this is not
legally possible, the right is granted to use the work for any purpose,
without any conditions, unless such conditions are required by law.

\vspace{3ex}
Several short, attributed quotations from copyrighted works
appear in this book under the ``fair use'' provision of Section 107 of
the United States Copyright Act (Title 17 of the {\em United States
Code}).  The public-domain status of this book is not applicable to
those quotations.

\vspace{3ex}
Any trademarks used in this book are the property of their owners.

% QA76.9.L63.M??

% \vspace{1ex}
%
% \vspace{1ex}
% {\small Permission is granted to make and distribute verbatim copies of this
% book
% provided the copyright notice and this
% permission notice are preserved on all copies.}
%
% \vspace{1ex}
% {\small Permission is granted to copy and distribute modified versions of this
% book under the conditions for verbatim copying, provided that the
% entire
% resulting derived work is distributed under the terms of a permission
% notice
% identical to this one.}
%
% \vspace{1ex}
% {\small Permission is granted to copy and distribute translations of this
% book into another language, under the above conditions for modified
% versions,
% except that this permission notice may be stated in a translation
% approved by the
% author.}
%
% \vspace{1ex}
% %{\small   For a copy of the \LaTeX\ source files for this book, contact
% %the author.} \\
% \ \\
% \ \\

\vspace{7ex}
% ISBN: 1-4116-3724-0 \\
ISBN: 978-1-4116-3724-5 \\
{\ } \\
Lulu Press \\
Morrisville, North Carolina\\
USA


\hfill
\vfill

Norman Megill\\ 93 Bridge St., Lexington, MA 02421 \\
E-mail address: \texttt{nm{\char`\@}alum.mit.edu} \\
\vspace{7ex}
David A. Wheeler \\
E-mail address: \texttt{dwheeler{\char`\@}dwheeler.com} \\
% See notes added at end of Preface for revision history. \\
% For current information on the Metamath software see \\
\vspace{7ex}
\url{http://metamath.org}
\end{center}

% \newpage
% \thispagestyle{empty}
%
% \hfill
% \vfill
%
% \begin{center}
% {\it To my son Robin Dwight Megill}
% \end{center}
%
% \vfill
% \hfill
%
% \newpage

\tableofcontents
%\listoftables

\chapter*{Preface}
\markboth{PREFACE}{PREFACE}
\addcontentsline{toc}{section}{Preface}


% (For current information, see the notes added at the
% end of this preface on p.~\pageref{note2002}.)

\subsubsection{Overview}

Metamath\index{Metamath} is a computer language and an associated computer
program for archiving, verifying, and studying mathematical proofs at a very
detailed level.  The Metamath language incorporates no mathematics per se but
treats all mathematical statements as mere sequences of symbols.  You provide
Metamath with certain special sequences (axioms) that tell it what rules
of inference are allowed.  Metamath is not limited to any specific field of
mathematics.  The Metamath language is simple and robust, with an
almost total absence of hard-wired syntax, and
I\footnote{Unless otherwise noted, the words
``I,'' ``me,'' and ``my'' refer to Norman Megill\index{Megill, Norman}, while
``we,'' ``us,'' and ``our'' refer to Norman Megill and
David A. Wheeler\index{Wheeler, David A.}.}
believe that it
provides about the simplest possible framework that allows essentially all of
mathematics to be expressed with absolute rigor.

% index test
%\newcommand{\nn}[1]{#1n}
%\index{aaa@bbb}
%\index{abc!def}
%\index{abd|see{qqq}}
%\index{abe|nn}
%\index{abf|emph}
%\index{abg|(}
%\index{abg|)}

Using the Metamath language, you can build formal or mathematical
systems\index{formal system}\footnote{A formal or mathematical system consists
of a collection of symbols (such as $2$, $4$, $+$ and $=$), syntax rules that
describe how symbols may be combined to form a legal expression (called a
well-formed formula or {\em wff}, pronounced ``whiff''), some starting wffs
called axioms, and inference rules that describe how theorems may be derived
(proved) from the axioms.  A theorem is a mathematical fact such as $2+2=4$.
Strictly speaking, even an obvious fact such as this must be proved from
axioms to be formally acceptable to a mathematician.}\index{theorem}
\index{axiom}\index{rule}\index{well-formed formula (wff)} that involve
inferences from axioms.  Although a database is provided
that includes a recommended set of axioms for standard mathematics, if you
wish you can supply your own symbols, syntax, axioms, rules, and definitions.

The name ``Metamath'' was chosen to suggest that the language provides a
means for {\em describing} mathematics rather than {\em being} the
mathematics itself.  Actually in some sense any mathematical language is
metamathematical.  Symbols written on paper, or stored in a computer,
are not mathematics itself but rather a way of expressing mathematics.
For example ``7'' and ``VII'' are symbols for denoting the number seven
in Arabic and Roman numerals; neither {\em is} the number seven.

If you are able to understand and write computer programs, you should be able
to follow abstract mathematics with the aid of Metamath.  Used in conjunction
with standard textbooks, Metamath can guide you step-by-step towards an
understanding of abstract mathematics from a very rigorous viewpoint, even if
you have no formal abstract mathematics background.  By using a single,
consistent notation to express proofs, once you grasp its basic concepts
Metamath provides you with the ability to immediately follow and dissect
proofs even in totally unfamiliar areas.

Of course, just being able follow a proof will not necessarily give you an
intuitive familiarity with mathematics.  Memorizing the rules of chess does not
give you the ability to appreciate the game of a master, and knowing how the
notes on a musical score map to piano keys does not give you the ability to
hear in your head how it would sound.  But each of these can be a first step.

Metamath allows you to explore proofs in the sense that you can see the
theorem referenced at any step expanded in as much detail as you want, right
down to the underlying axioms of logic and set theory (in the case of the set
theory database provided).  While Metamath will not replace the higher-level
understanding that can only be acquired through exercises and hard work, being
able to see how gaps in a proof are filled in can give you increased
confidence that can speed up the learning process and save you time when you
get stuck.

The Metamath language breaks down a mathematical proof into its tiniest
possible parts.  These can be pieced together, like interlocking
pieces in a puzzle, only in a way that produces correct and absolutely rigorous
mathematics.

The nature of Metamath\index{Metamath} enforces very precise mathematical
thinking, similar to that involved in writing a computer program.  A crucial
difference, though, is that once a proof is verified (by the Metamath program)
to be correct, it is definitely correct; it can never have a hidden
``bug.''\index{computer program bugs}  After getting used to the kind of rigor
and accuracy provided by Metamath, you might even be tempted to
adopt the attitude that a proof should never be considered correct until it
has been verified by a computer, just as you would not completely trust a
manual calculation until you have verified it on a
calculator.

My goal
for Metamath was a system for describing and verifying
mathematics that is completely universal yet conceptually as simple as
possible.  In approaching mathematics from an axiomatic, formal viewpoint, I
wanted Metamath to be able to handle almost any mathematical system, not
necessarily with ease, but at least in principle and hopefully in practice. I
wanted it to verify proofs with absolute rigor, and for this reason Metamath
is what might be thought of as a ``compile-only'' language rather than an
algorithmic or Turing-machine language (Pascal, C, Prolog, Mathematica,
etc.).  In other words, a database written in the Metamath
language doesn't ``do'' anything; it merely exhibits mathematical knowledge
and permits this knowledge to be verified as being correct.  A program in an
algorithmic language can potentially have hidden bugs\index{computer program
bugs} as well as possibly being hard to understand.  But each token in a
Metamath database must be consistent with the database's earlier
contents according to simple, fixed rules.
If a database is verified
to be correct,\footnote{This includes
verification that a sequential list of proof steps results in the specified
theorem.} then the mathematical content is correct if the
verifier is correct and the axioms are correct.
The verification program could be incorrect, but the verification algorithm
is relatively simple (making it unlikely to be implemented incorrectly
by the Metamath program),
and there are over a dozen Metamath database verifiers
written by different people in different programming languages
(so these different verifiers can act as multiple reviewers of a database).
The most-used Metamath database, the Metamath Proof Explorer
(aka \texttt{set.mm}\index{set theory database (\texttt{set.mm})}%
\index{Metamath Proof Explorer}),
is currently verified by four different Metamath verifiers written by
four different people in four different languages, including the
original Metamath program described in this book.
The only ``bugs'' that can exist are in the statement of the axioms,
for example if the axioms are inconsistent (a famous problem shown to be
unsolvable by G\"{o}del's incompleteness theorem\index{G\"{o}del's
incompleteness theorem}).
However, real mathematical systems have very few axioms, and these can
be carefully studied.
All of this provides extraordinarily high confidence that the verified database
is in fact correct.

The Metamath program
doesn't prove theorems automatically but is designed to verify proofs
that you supply to it.
The underlying Metamath language is completely general and has no built-in,
preconceived notions about your formal system\index{formal system}, its logic
or its syntax.
For constructing proofs, the Metamath program has a Proof Assistant\index{Proof
Assistant} which helps you fill in some of a proof step's details, shows you
what choices you have at any step, and verifies the proof as you build it; but
you are still expected to provide the proof.

There are many other programs that can process or generate information
in the Metamath language, and more continue to be written.
This is in part because the Metamath language itself is very simple
and intentionally easy to automatically process.
Some programs, such as \texttt{mmj2}\index{mmj2}, include a proof assistant
that can automate some steps beyond what the Metamath program can do.
Mario Carneiro has developed an algorithm for converting proofs from
the OpenTheory interchange format, which can be translated to and from
any of the HOL family of proof languages (HOL4, HOL Light, ProofPower,
and Isabelle), into the
Metamath language.\cite{DBLP:journals/corr/Carneiro14}\index{Carneiro, Mario}.
Daniel Whalen has developed Holophrasm, which can automatically
prove many Metamath proofs using
machine learning\index{machine learning}\index{artificial intelligence}
approaches
(including multiple neural networks\index{neural networks})\cite{DBLP:journals/corr/Whalen16}\index{Whalen, Daniel}.
However,
a discussion of these other programs is beyond the scope of this book.

Like most computer languages, the Metamath\index{Metamath} language uses the
standard ({\sc ascii}) characters on a computer keyboard, so it cannot
directly represent many of the special symbols that mathematicians use.  A
useful feature of the Metamath program is its ability to convert its notation
into the \LaTeX\ typesetting language.\index{latex@{\LaTeX}}  This feature
lets you convert the {\sc ascii} tokens you've defined into standard
mathematical symbols, so you end up with symbols and formulas you are familiar
with instead of somewhat cryptic {\sc ascii} representations of them.
The Metamath program can also generate HTML\index{HTML}, making it easy
to view results on the web and to see related information by using
hypertext links.

Metamath is probably conceptually different from anything you've seen
before and some aspects may take some getting used to.  This book will
help you decide whether Metamath suits your specific needs.

\subsubsection{Setting Your Expectations}
It is important for you to understand what Metamath\index{Metamath} is and is
not.  As mentioned, the Metamath program
is {\em not} an automated theorem prover but
rather a proof verifier.  Developing a database can be tedious, hard work,
especially if you want to make the proofs as short as possible, but it becomes
easier as you build up a collection of useful theorems.  The purpose of
Metamath is simply to document existing mathematics in an absolutely rigorous,
computer-verifiable way, not to aid directly in the creation of new
mathematics.  It also is not a magic solution for learning abstract
mathematics, although it may be helpful to be able to actually see the implied
rigor behind what you are learning from textbooks, as well as providing hints
to work out proofs that you are stumped on.

As of this writing, a sizable set theory database has been developed to
provide a foundation for many fields of mathematics, but much more work would
be required to develop useful databases for specific fields.

Metamath\index{Metamath} ``knows no math;'' it just provides a framework in
which to express mathematics.  Its language is very small.  You can define two
kinds of symbols, constants\index{constant} and variables\index{variable}.
The only thing Metamath knows how to do is to substitute strings of symbols
for the variables\index{substitution!variable}\index{variable substitution} in
an expression based on instructions you provide it in a proof, subject to
certain constraints you specify for the variables.  Even the decimal
representation of a number is merely a string of certain constants (digits)
which together, in a specific context, correspond to whatever mathematical
object you choose to define for it; unlike other computer languages, there is
no actual number stored inside the computer.  In a proof, you in effect
instruct Metamath what symbol substitutions to make in previous axioms or
theorems and join a sequence of them together to result in the desired
theorem.  This kind of symbol manipulation captures the essence of mathematics
at a preaxiomatic level.

\subsubsection{Metamath and Mathematical Literature}

In advanced mathematical literature, proofs are usually presented in the form
of short outlines that often only an expert can follow.  This is partly out of
a desire for brevity, but it would also be unwise (even if it were practical)
to present proofs in complete formal detail, since the overall picture would
be lost.\index{formal proof}

A solution I envision\label{envision} that would allow mathematics to remain
acceptable to the expert, yet increase its accessibility to non-specialists,
consists of a combination of the traditional short, informal proof in print
accompanied by a complete formal proof stored in a computer database.  In an
analogy with a computer program, the informal proof is like a set of comments
that describe the overall reasoning and content of the proof, whereas the
computer database is like the actual program and provides a means for anyone,
even a non-expert, to follow the proof in as much detail as desired, exploring
it back through layers of theorems (like subroutines that call other
subroutines) all the way back to the axioms of the theory.  In addition, the
computer database would have the advantage of providing absolute assurance
that the proof is correct, since each step can be verified automatically.

There are several other approaches besides Metamath to a project such
as this.  Section~\ref{proofverifiers} discusses some of these.

To us, a noble goal would be a database with hundreds of thousands of
theorems and their computer-verifiable proofs, encompassing a significant
fraction of known mathematics and available for instant access.
These would be fully verified by multiple independently-implemented verifiers,
to provide extremely high confidence that the proofs are completely correct.
The database would allow people to investigate whatever details they were
interested in, so that they could confirm whatever portions they wished.
Whether or not Metamath is an appropriate choice remains to be seen, but in
principle we believe it is sufficient.

\subsubsection{Formalism}

Over the past fifty years, a group of French mathematicians working
collectively under the pseudonym of Bourbaki\index{Bourbaki, Nicolas} have
co-authored a series of monographs that attempt to rigorously and
consistently formalize large bodies of mathematics from foundations.  On the
one hand, certainly such an effort has its merits; on the other hand, the
Bourbaki project has been criticized for its ``scholasticism'' and
``hyperaxiomatics'' that hide the intuitive steps that lead to the results
\cite[p.~191]{Barrow}\index{Barrow, John D.}.

Metamath unabashedly carries this philosophy to its extreme and no doubt is
subject to the same kind of criticism.  Nonetheless I think that in
conjunction with conventional approaches to mathematics Metamath can serve a
useful purpose.  The Bourbaki approach is essentially pedagogic, requiring the
reader to become intimately familiar with each detail in a very large
hierarchy before he or she can proceed to the next step.  The difference with
Metamath is that the ``reader'' (user) knows that all details are contained in
its computer database, available as needed; it does not demand that the user
know everything but conveniently makes available those portions that are of
interest.  As the body of all mathematical knowledge grows larger and larger,
no one individual can have a thorough grasp of its entirety.  Metamath
can finalize and put to rest any questions about the validity of any part of it
and can make any part of it accessible, in principle, to a non-specialist.

\subsubsection{A Personal Note}
Why did I develop Metamath\index{Metamath}?  I enjoy abstract mathematics, but
I sometimes get lost in a barrage of definitions and start to lose confidence
that my proofs are correct.  Or I reach a point where I lose sight of how
anything I'm doing relates to the axioms that a theory is based on and am
sometimes suspicious that there may be some overlooked implicit axiom
accidentally introduced along the way (as happened historically with Euclidean
geometry\index{Euclidean geometry}, whose omission of Pasch's
axiom\index{Pasch's axiom} went unnoticed for 2000 years
\cite[p.~160]{Davis}!). I'm also somewhat lazy and wish to avoid the effort
involved in re-verifying the gaps in informal proofs ``left to the reader;'' I
prefer to figure them out just once and not have to go through the same
frustration a year from now when I've forgotten what I did.  Metamath provides
better recovery of my efforts than scraps of paper that I can't
decipher anymore.  But mostly I find very appealing the idea of rigorously
archiving mathematical knowledge in a computer database, providing precision,
certainty, and elimination of human error.

\subsubsection{Note on Bibliography and Index}

The Bibliography usually includes the Library of Congress classification
for a work to make it easier for you to find it in on a university
library shelf.  The Index has author references to pages where their works
are cited, even though the authors' names may not appear on those pages.

\subsubsection{Acknowledgments}

Acknowledgments are first due to my wife, Deborah (who passed away on
September 4, 1998), for critiquing the manu\-script but most of all for
her patience and support.  I also wish to thank Joe Wright, Richard
Becker, Clarke Evans, Buddha Buck, and Jeremy Henty for helpful
comments.  Any errors, omissions, and other shortcomings are of course
my responsibility.

\subsubsection{Note Added June 22, 2005}\label{note2002}

The original, unpublished version of this book was written in 1997 and
distributed via the web.  The present edition has been updated to
reflect the current Metamath program and databases, as well as more
current {\sc url}s for Internet sites.  Thanks to Josh
Purinton\index{Purinton, Josh}, One Hand
Clapping, Mel L.\ O'Cat, and Roy F. Longton for pointing out
typographical and other errors.  I have also benefitted from numerous
discussions with Raph Levien\index{Levien, Raph}, who has extended
Metamath's philosophy of rigor to result in his {\em
Ghilbert}\index{Ghilbert} proof language (\url{http://ghilbert.org}).

Robert (Bob) Solovay\index{Solovay, Robert} communicated a new result of
A.~R.~D.~Mathias on the system of Bourbaki, and the text has been
updated accordingly (p.~\pageref{bourbaki}).

Bob also pointed out a clarification of the literature regarding
category theory and inaccessible cardinals\index{category
theory}\index{cardinal, inaccessible} (p.~\pageref{categoryth}),
and a misleading statement was removed from the text.  Specifically,
contrary to a statement in previous editions, it is possible to express
``There is a proper class of inaccessible cardinals'' in the language of
ZFC.  This can be done as follows:  ``For every set $x$ there is an
inaccessible cardinal $\kappa$ such that $\kappa$ is not in $x$.''
Bob writes:\footnote{Private communication, Nov.~30, 2002.}
\begin{quotation}
     This axiom is how Grothendieck presents category theory.  To each
inaccessible cardinal $\kappa$ one associates a Grothendieck universe
\index{Grothendieck, Alexander} $U(\kappa)$.  $U(\kappa)$ consists of
those sets which lie in a transitive set of cardinality less than
$\kappa$.  Instead of the ``category of all groups,'' one works relative
to a universe [considering the category of groups of cardinality less
than $\kappa$].  Now the category whose objects are all categories
``relative to the universe $U(\kappa)$'' will be a category not
relative to this universe but to the next universe.

     All of the things category theorists like to do can be done in this
framework.  The only controversial point is whether the Grothen\-dieck
axiom is too strong for the needs of category theorists.  Mac Lane
\index{Mac Lane, Saunders} argues that ``one universe is enough'' and
Feferman\index{Feferman, Solomon} has argued that one can get by with
ordinary ZFC.  I don't find Feferman's arguments persuasive.  Mac Lane
may be right, but when I think about category theory I do it \`{a} la
Grothendieck.

        By the way Mizar\index{Mizar} adds the axiom ``there is a proper
class of inaccessibles'' precisely so as to do category theory.
\end{quotation}

The most current information on the Metamath program and databases can
always be found at \url{http://metamath.org}.


\subsubsection{Note Added June 24, 2006}\label{note2006}

The Metamath spec was restricted slightly to make parsers easier to
write.  See the footnote on p.~\pageref{namespace}.

%\subsubsection{Note Added July 24, 2006}\label{note2006b}
\subsubsection{Note Added March 10, 2007}\label{note2006b}

I am grateful to Anthony Williams\index{Williams, Anthony} for writing
the \LaTeX\ package called {\tt realref.sty} and contributing it to the
public domain.  This package allows the internal hyperlinks in a {\sc
pdf} file to anchor to specific page numbers instead of just section
titles, making the navigation of the {\sc pdf} file for this book much
more pleasant and ``logical.''

A typographical error found by Martin Kiselkov was corrected.
A confusing remark about unification was deleted per suggestion of
Mel O'Cat.

\subsubsection{Note Added May 27, 2009}\label{note2009}

Several typos found by Kim Sparre were corrected.  A note was added that
the Poincar\'{e} conjecture has been proved (p.~\pageref{poincare}).

\subsubsection{Note Added Nov. 17, 2014}\label{note2014}

The statement of the Schr\"{o}der--Bernstein theorem was corrected in
Section~\ref{trust}.  Thanks to Bob Solovay for pointing out the error.

\subsubsection{Note Added May 25, 2016}\label{note2016}

Thanks to Jerry James for correcting 16 typos.

\subsubsection{Note Added February 25, 2019}\label{note201902}

David A. Wheeler\index{Wheeler, David A.}
made a large number of improvements and updates,
in coordination with Norman Megill.
The predicate calculus axioms were renumbered, and the text makes
it clear that they are based on Tarski's system S2;
the one slight deviation in axiom ax-6 is explained and justified.
The real and complex number axioms were modified to be consistent with
\texttt{set.mm}\index{set theory database (\texttt{set.mm})}%
\index{Metamath Proof Explorer}.
Long-awaited specification changes ``1--8'' were made,
which clarified previously ambiguous points.
Some errors in the text involving \texttt{\$f} and
\texttt{\$d} statements were corrected (the spec was correct, but
the in-book explanations unintentionally contradicted the spec).
We now have a system for automatically generating narrow PDFs,
so that those with smartphones can have easy access to the current
version of this document.
A new section on deduction was added;
it discusses the standard deduction theorem,
the weak deduction theorem,
deduction style, and natural deduction.
Many minor corrections (too numerous to list here) were also made.

\subsubsection{Note Added March 7, 2019}\label{note201903}

This added a description of the Matamath language syntax in
Extended Backus--Naur Form (EBNF)\index{Extended Backus--Naur Form}\index{EBNF}
in Appendix \ref{BNF}, added a brief explanation about typecodes,
inserted more examples in the deduction section,
and added a variety of smaller improvements.

\subsubsection{Note Added April 7, 2019}\label{note201904}

This version clarified the proper substitution notation, improved the
discussion on the weak deduction theorem and natural deduction,
documented the \texttt{undo} command, updated the information on
\texttt{write source}, changed the typecode
from \texttt{set} to \texttt{setvar} to be consistent with the current
version of \texttt{set.mm}, added more documentation about comment markup
(e.g., documented how to create headings), and clarified the
differences between various assertion forms (in particular deduction form).

\subsubsection{Note Added May 20, 2019}\label{note201905}

This version fixes various small issues and added specific examples
of forms (deduction form, inference form, and closed form).

\chapter{Introduction}
\pagenumbering{arabic}

\begin{quotation}
  {\em {\em I.M.:}  No, no.  There's nothing subjective about it!  Everybody
knows what a proof is.  Just read some books, take courses from a competent
mathematician, and you'll catch on.

{\em Student:}  Are you sure?

{\em I.M.:}  Well---it is possible that you won't, if you don't have any
aptitude for it.  That can happen, too.

{\em Student:}  Then {\em you} decide what a proof is, and if I don't learn
to decide in the same way, you decide I don't have any aptitude.

{\em I.M.:}  If not me, then who?}
    \flushright\sc  ``The Ideal Mathematician''
    \index{Davis, Phillip J.}
    \footnote{\cite{Davis}, p.~40.}\\
\end{quotation}

In the past century, brilliant mathematicians have discovered almost
unimaginably profound results that rank among the crowning intellectual
achievements of mankind.  However, there is a sense in which modern abstract
mathematics is behind the times, stuck in an era before computers existed.
While no one disputes the remarkable results that have been achieved,
communicating these results in a precise way to the uninitiated is virtually
impossible.  To describe these results, a terse informal language is used which
despite its elegance is very difficult to learn.  This informal language is not
imprecise, far from it, but rather it often has omitted detail
and symbols with hidden context that are
implicitly understood by an expert but few others.  Extremely complex technical
meanings are associated with innocent-sounding English words such as
``compact'' and ``measurable'' that barely hint at what is actually being
said.  Anyone who does not keep the precise technical meaning constantly in
mind is bound to fail, and acquiring the ability to do this can be achieved
only through much practice and hard work.  Only the few who complete the
painful learning experience can join the small in-group of pure
mathematicians.  The informal language effectively cuts off the true nature of
their knowledge from most everyone else.

Metamath\index{Metamath} makes abstract mathematics more concrete.  It allows
a computer to keep track of the complexity associated with each word or symbol
with absolute rigor.  You can explore this complexity at your leisure, to
whatever degree you desire.  Whether or not you believe that concepts such as
infinity actually ``exist'' outside of the mind, Metamath lets you get to the
foundation for what's really being said.

Metamath also enables completely rigorous and thorough proof verification.
Its language is simple enough so that you
don't have to rely on the authority of experts but can verify the results
yourself, step by step.  If you want to attempt to derive your own results,
Metamath will not let you make a mistake in reasoning.
Even professional mathematicians make mistakes; Metamath makes it possible
to thoroughly verify that proofs are correct.

Metamath\index{Metamath} is a computer language and an associated computer
program for archiving, verifying, and studying mathematical proofs at a very
detailed level.
The Metamath language
describes formal\index{formal system} mathematical
systems and expresses proofs of theorems in those systems.  Such a language
is called a metalanguage\index{metalanguage} by mathematicians.
The Metamath program is a computer program that verifies
proofs expressed in the Metamath language.
The Metamath program does not have the built-in
ability to make logical inferences; it just makes a series of symbol
substitutions according to instructions given to it in a proof
and verifies that the result matches the expected theorem.  It makes logical
inferences based only on rules of logic that are contained in a set of
axioms\index{axiom}, or first principles, that you provide to it as the
starting point for proofs.

The complete specification of the Metamath language is only four pages long
(Section~\ref{spec}, p.~\pageref{spec}).  Its simplicity may at first make you
wonder how it can do much of anything at all.  But in fact the kinds of
symbol manipulations it performs are the ones that are implicitly done in all
mathematical systems at the lowest level.  You can learn it relatively quickly
and have complete confidence in any mathematical proof that it verifies.  On
the other hand, it is powerful and general enough so that virtually any
mathematical theory, from the most basic to the deeply abstract, can be
described with it.

Although in principle Metamath can be used with any
kind of mathematics, it is best suited for abstract or ``pure'' mathematics
that is mostly concerned with theorems and their proofs, as opposed to the
kind of mathematics that deals with the practical manipulation of numbers.
Examples of branches of pure mathematics are logic\index{logic},\footnote{Logic
is the study of statements that are universally true regardless of the objects
being described by the statements.  An example is the statement, ``if $P$
implies $Q$, then either $P$ is false or $Q$ is true.''} set theory\index{set
theory},\footnote{Set theory is the study of general-purpose mathematical objects called
``sets,'' and from it essentially all of mathematics can be derived.  For
example, numbers can be defined as specific sets, and their properties
can be explored using the tools of set theory.} number theory\index{number
theory},\footnote{Number theory deals with the properties of positive and
negative integers (whole numbers).} group theory\index{group
theory},\footnote{Group theory studies the properties of mathematical objects
called groups that obey a simple set of axioms and have properties of symmetry
that make them useful in many other fields.} abstract algebra\index{abstract
algebra},\footnote{Abstract algebra includes group theory and also studies
groups with additional properties that qualify them as ``rings'' and
``fields.''  The set of real numbers is a familiar example of a field.},
analysis\index{analysis} \index{real and complex numbers}\footnote{Analysis is
the study of real and complex numbers.} and
topology\index{topology}.\footnote{One area studied by topology are properties
that remain unchanged when geometrical objects undergo stretching
deformations; for example a doughnut and a coffee cup each have one hole (the
cup's hole is in its handle) and are thus considered topologically
equivalent.  In general, though, topology is the study of abstract
mathematical objects that obey a certain (surprisingly simple) set of axioms.
See, for example, Munkres \cite{Munkres}\index{Munkres, James R.}.} Even in
physics, Metamath could be applied to certain branches that make use of
abstract mathematics, such as quantum logic\index{quantum logic} (used to study
aspects of quantum mechanics\index{quantum mechanics}).

On the other hand, Metamath\index{Metamath} is less suited to applications
that deal primarily with intensive numeric computations.  Metamath does not
have any built-in representation of numbers\index{Metamath!representation of
numbers}; instead, a specific string of symbols (digits) must be syntactically
constructed as part of any proof in which an ordinary number is used.  For
this reason, numbers in Metamath are best limited to specific constants that
arise during the course of a theorem or its proof.  Numbers are only a tiny
part of the world of abstract mathematics.  The exclusion of built-in numbers
was a conscious decision to help achieve Metamath's simplicity, and there are
other software tools if you have different mathematical needs.
If you wish to quickly solve algebraic problems, the computer algebra
programs\index{computer algebra system} {\sc
macsyma}\index{macsyma@{\sc macsyma}}, Mathematica\index{Mathematica}, and
Maple\index{Maple} are specifically suited to handling numbers and
algebra efficiently.
If you wish to simply calculate numeric or matrix expressions easily,
tools such as Octave\index{Octave} may be a better choice.

After learning Metamath's basic statement types, any
tech\-ni\-cal\-ly ori\-ent\-ed person, mathematician or not, can
immediately trace
any theorem proved in the language as far back as he or she wants, all the way
to the axioms on which the theorem is based.  This ability suggests a
non-traditional way of learning about pure mathematics.  Used in conjunction
with traditional methods, Metamath could make pure mathematics accessible to
people who are not sufficiently skilled to figure out the implicit detail in
ordinary textbook proofs.  Once you learn the axioms of a theory, you can have
complete confidence that everything you need to understand a proof you are
studying is all there, at your beck and call, allowing you to focus in on any
proof step you don't understand in as much depth as you need, without worrying
about getting stuck on a step you can't figure out.\footnote{On the other
hand, writing proofs in the Metamath language is challenging, requiring
a degree of rigor far in excess of that normally taught to students.  In a
classroom setting, I doubt that writing Metamath proofs would ever replace
traditional homework exercises involving informal proofs, because the time
needed to work out the details would not allow a course to
cover much material.  For students who have trouble grasping the implied rigor
in traditional material, writing a few simple proofs in the Metamath language
might help clarify fuzzy thought processes.  Although somewhat difficult at
first, it eventually becomes fun to do, like solving a puzzle, because of the
instant feedback provided by the computer.}

Metamath\index{Metamath} is probably unlike anything you have
encountered before.  In this first chapter we will look at the philosophy and
use of computers in mathematics in order to better understand the motivation
behind Metamath.  The material in this chapter is not required in order to use
Metamath.  You may skip it if you are impatient, but I hope you will find it
educational and enjoyable.  If you want to start experimenting with the
Metamath program right away, proceed directly to Chapter~\ref{using}
(p.~\pageref{using}).  To
learn the Metamath language, skim Chapter~\ref{using} then proceed to
Chapter~\ref{languagespec} (p.~\pageref{languagespec}).

\section{Mathematics as a Computer Language}

\begin{quote}
  {\em The study of mathematics is apt to commence in
dis\-ap\-point\-ment.\ldots \\
We are told that by its aid the stars are weighted
and the billions of molecules in a drop of water are counted.  Yet, like the
ghost of Hamlet's father, this great science eludes the efforts of our mental
weapons to grasp it.}
  \flushright\sc  Alfred North Whitehead\footnote{\cite{Whitehead}, ch.\ 1.}\\
\end{quote}\index{Whitehead, Alfred North}

\subsection{Is Mathematics ``User-Friendly''?}

Suppose you have no formal training in abstract mathematics.  But popular
books you've read offer tempting glimpses of this world filled with profound
ideas that have stirred the human spirit.  You are not satisfied with the
informal, watered-down descriptions you've read but feel it is important to
grasp the underlying mathematics itself to understand its true meaning. It's
not practical to go back to school to learn it, though; you don't want to
dedicate years of your life to it.  There are many important things in life,
and you have to set priorities for what's important to you.  What would happen
if you tried to pursue it on your own, in your spare time?

After all, you were able to learn a computer programming language such as
Pascal on your own without too much difficulty, even though you had no formal
training in computers.  You don't claim to be an expert in software design,
but you can write a passable program when necessary to suit your needs.  Even
more important, you know that you can look at anyone else's Pascal program, no
matter how complex, and with enough patience figure out exactly how it works,
even though you are not a specialist.  Pascal allows you do anything that a
computer can do, at least in principle.  Thus you know you have the ability,
in principle, to follow anything that a computer program can do:  you just
have to break it down into small enough pieces.

Here's an imaginary scenario of what might happen if you na\-ive\-ly a\-dopted
this same view of abstract mathematics and tried to pick it up on your own, in
a period of time comparable to, say, learning a computer programming
language.

\subsubsection{A Non-Mathematician's Quest for Truth}

\begin{quote}
  {\em \ldots my daughters have been studying (chemistry) for several
se\-mes\-ters, think they have learned differential and integral calculus in
school, and yet even today don't know why $x\cdot y=y\cdot x$ is true.}
  \flushright\sc  Edmund Landau\footnote{\cite{Landau}, p.~vi.}\\
\end{quote}\index{Landau, Edmund}

\begin{quote}
  {\em Minus times minus is plus,\\
The reason for this we need not discuss.}
  \flushright\sc W.\ H.\ Auden\footnote{As quoted in \cite{Guillen}, p.~64.}\\
\end{quote}\index{Auden, W.\ H.}\index{Guillen, Michael}

We'll suppose you are a technically oriented professional, perhaps an engineer, a
computer programmer, or a physicist, but probably not a mathematician.  You
consider yourself reasonably intelligent.  You did well in school, learning a
variety of methods and techniques in practical mathematics such as calculus and
differential equations.  But rarely did your courses get into anything
resembling modern abstract mathematics, and proofs were something that appeared
only occasionally in your textbooks, a kind of necessary evil that was
supposed to convince you of a certain key result.  Most of your
homework consisted of exercises that gave you practice in the techniques, and
you were hardly ever asked to come up with a proof of your own.

You find yourself curious about advanced, abstract mathematics.  You are
driven by an inner conviction that it is important to understand and
appreciate some of the most profound knowledge discovered by mankind.  But it
seems very hard to learn, something that only certain gifted longhairs can
access and understand.  You are frustrated that it seems forever cut off from
you.

Eventually your curiosity drives you to do something about it.
You set for yourself a goal of ``really'' understanding mathematics:  not just
how to manipulate equations in algebra or calculus according to cookbook
rules, but rather to gain a deep understanding of where those rules come from.
In fact, you're not thinking about this kind of ordinary mathematics at all,
but about a much more abstract, ethereal realm of pure mathematics, where
famous results such as G\"{o}del's incompleteness theorem\index{G\"{o}del's
incompleteness theorem} and Cantor's different kinds of infinities
reside.

You have probably read a number of popular books, with titles like {\em
Infinity and the Mind} \cite{Rucker}\index{Rucker, Rudy}, on topics such as
these.  You found them inspiring but at the same time somewhat
unsatisfactory.  They gave you a general idea of what these results are about,
but if someone asked you to prove them, you wouldn't have the faintest idea of
where to begin.   Sure, you could give the same overall outline that you
learned from the popular books; and in a general sort of way, you do have an
understanding.  But deep down inside, you know that there is a rigor that is
missing, that probably there are many subtle steps and pitfalls along the way,
and ultimately it seems you have to place your trust in the experts in the
field.  You don't like this; you want to be able to verify these results for
yourself.

So where do you go next?  As a first step, you decide to look up some of the
original papers on the theorems you are curious about, or better, obtain some
standard textbooks in the field.  You look up a theorem you want to
understand.  Sure enough, it's there, but it's expressed with strange
terms and odd symbols that mean absolutely nothing to you.  It might as well be written in
a foreign language you've never seen before, whose symbols are totally alien.
You look at the proof, and you haven't the foggiest notion what each step
means, much less how one step follows from another.  Well, obviously you have
a lot to learn if you want to understand this stuff.

You feel that you could probably understand it by
going back to college for another three to six years and getting a math
degree.  But that does not fit in with your career and the other things in
your life and would serve no practical purpose.  You decide to seek a quicker
path.  You figure you'll just trace your way back to the beginning, step by
step, as you would do with a computer program, until you understand it.  But
you quickly find that this is not possible, since you can't even understand
enough to know what you have to trace back to.

Maybe a different approach is in order---maybe you should start at the
beginning and work your way up.  First, you read the introduction to the book
to find out what the prerequisites are.  In a similar fashion, you trace your
way back through two or three more books, finally arriving at one that seems
to start at a beginning:  it lists the axioms of arithmetic.  ``Aha!'' you
naively think, ``This must be the starting point, the source of all mathematical
knowledge.'' Or at least the starting point for mathematics dealing with
numbers; you have to start somewhere and have no idea what the starting point
for other mathematics would be.  But the word ``axioms'' looks promising.  So
you eagerly read along and work through some elementary exercises at the
beginning of the book.  You feel vaguely bothered:  these
don't seem like axioms at all, at least not in the sense that you want to
think of axioms.  Axioms imply a starting point from which everything else can
be built up, according to precise rules specified in the axiom system.  Even
though you can understand the first few proofs in an informal way,
and are able to do some of the
exercises, it's hard to pin down precisely what the
rules are.   Sure, each step seems to follow logically from the others, but
exactly what does that mean?  Is the ``logic'' just a matter of common sense,
something vague that we all understand but can never quite state precisely?

You've spent a number of years, off and on, programming computers, and you
know that in the case of computer languages there is no question of what the
rules are---they are precise and crystal clear.  If you follow them, your
program will work, and if you don't, it won't.  No matter how complex a
program, it can always be broken down into simpler and simpler pieces, until
you can ultimately identify the bits that are moved around to perform a
specific function.  Some programs might require a lot of perseverance to
accomplish this, but if you focus on a specific portion of it, you don't even
necessarily have to know how the rest of it works. Shouldn't there be an
analogy in mathematics?

You decide to apply the ultimate test:  you ask yourself how a computer could
verify or ensure that the steps in these proofs follow from one another.
Certainly mathematics must be at least as precisely defined as a computer
language, if not more so; after all, computer science itself is based on it.
If you can get a computer to verify these proofs, then you should also be
able, in principle, to understand them yourself in a very crystal clear,
precise way.

You're in for a surprise:  you can conceive of no way to convert the
proofs, which are in English, to a form that the computer can understand.
The proofs are filled with phrases such as ``assume there exists a unique
$x$\ldots'' and ``given any $y$, let $z$ be the number such that\ldots''  This
isn't the kind of logic you are used to in computer programming, where
everything, even arithmetic, reduces to Boolean ones and zeroes if you care to
break it down sufficiently.  Even though you think you understand the proofs,
there seems to be some kind of higher reasoning involved rather than precise
rules that define how you manipulate the symbols in the axioms.  Whatever it
is, it just isn't obvious how you would express it to a computer, and the more
you think about it, the more puzzled and confused you get, to the point where
you even wonder whether {\em you} really understand it.  There's a lot more to
these axioms of arithmetic than meets the eye.

Nobody ever talked about this in school in your applied math and engineering
courses.  You just learned the rules they gave you, not quite understanding
how or why they worked, sometimes vaguely suspicious or uncertain of them, and
through homework problems and osmosis learned how to present solutions that
satisfied the instructor and earned you an ``A.''  Rarely did you actually
``prove'' anything in a rigorous way, and the math majors who did do stuff
like that seemed to be in a different world.

Of course, there are computer algebra programs that can do mathematics, and
rather impressively.  They can instantly solve the integrals that you
struggled with in freshman calculus, and do much, much more.  But when you
look at these programs, what you see is a big collection of algorithms and
techniques that evolved and were added to over time, along with some basic
software that manipulates symbols.  Each algorithm that is built in is the
result of someone's theorem whose proof is omitted; you just have to trust the
person who proved it and the person who programmed it in and hope there are no
bugs.\index{computer program bugs}  Somehow this doesn't seem to be the
essence of mathematics.  Although computer algebra systems can generate
theorems with amazing speed, they can't actually prove a single one of them.

After some puzzlement, you revisit some popular books on what mathematics is
all about.  Somewhere you read that all of mathematics is actually derived
from something called ``set theory.''  This is a little confusing, because
nowhere in the book that presented the axioms of arithmetic was there any
mention of set theory, or if there was, it seemed to be just a tool that helps
you describe things better---the set of even numbers, that sort of thing.  If
set theory is the basis for all mathematics, then why are additional axioms
needed for arithmetic?

Something is wrong but you're not sure what.  One of your friends is a pure
mathematician.  He knows he is unable to communicate to you what he does for a
living and seems to have little interest in trying.  You do know that for him,
proofs are what mathematics is all about. You ask him what a proof is, and he
essentially tells you that, while of course it's based on logic, really it's
something you learn by doing it over and over until you pick it up.  He refers
you to a book, {\em How to Read and Do Proofs} \cite{Solow}.\index{Solow,
Daniel}  Although this book helps you understand traditional informal proofs,
there is still something missing you can't seem to pin down yet.

You ask your friend how you would go about having a computer verify a proof.
At first he seems puzzled by the question; why would you want to do that?
Then he says it's not something that would make any sense to do, but he's
heard that you'd have to break the proof down into thousands or even millions
of individual steps to do such a thing, because the reasoning involved is at
such a high level of abstraction.  He says that maybe it's something you could
do up to a point, but the computer would be completely impractical once you
get into any meaningful mathematics.  There, the only way you can verify a
proof is by hand, and you can only acquire the ability to do this by
specializing in the field for a couple of years in grad school.  Anyway, he
thinks it all has to do with set theory, although he has never taken a formal
course in set theory but just learned what he needed as he went along.

You are intrigued and amazed.  Apparently a mathematician can grasp as a
single concept something that would take a computer a thousand or a million
steps to verify, and have complete confidence in it.  Each one of these
thousand or million steps must be absolutely correct, or else the whole proof
is meaningless.  If you added a million numbers by hand, would you trust the
result?  How do you really know that all these steps are correct, that there
isn't some subtle pitfall in one of these million steps, like a bug in a
computer program?\index{computer program bugs}  After all, you've read that
famous mathematicians have occasionally made mistakes, and you certainly know
you've made your share on your math homework problems in school.

You recall the analogy with a computer program.  Sure, you can understand what
a large computer program such as a word processor does, as a single high-level
concept or a small set of such concepts, but your ability to understand it in
no way ensures that the program is correct and doesn't have hidden bugs.  Even
if you wrote the program yourself you can't really know this; most large
programs that you've written have had bugs that crop up at some later date, no
matter how careful you tried to be while writing them.

OK, so now it seems the reason you can't figure out how to make a
computer verify proofs is because each step really corresponds to a
million small steps.  Well, you say, a computer can do a million
calculations in a second, so maybe it's still practical to do.  Now the
puzzle becomes how to figure out what the million steps are that each
English-language step corresponds to.  Your mathematician friend hasn't
a clue, but suggests that maybe you would find the answer by studying
set theory.  Actually, your friend thinks you're a little off the wall
for even wondering such a thing.  For him, this is not what mathematics
is all about.

The subject of set theory keeps popping up, so you decide it's
time to look it up.

You decide to start off on a careful footing, so you start reading a couple of
very elementary books on set theory.  A lot of it seems pretty obvious, like
intersections, subsets, and Venn diagrams.  You thumb through one of the
books; nowhere is anything about axioms mentioned. The other book relegates to
an appendix a brief discussion that mentions a set of axioms called
``Zermelo--Fraenkel set theory''\index{Zermelo--Fraenkel set theory} and states
them in English.  You look at them and have no idea what they really mean or
what you can do with them.  The comments in this appendix say that the purpose
of mentioning them is to expose you to the idea, but imply that they are not
necessary for basic understanding and that they are really the subject matter
of advanced treatments where fine points such as a certain paradox (Russell's
paradox\index{Russell's paradox}\footnote{Russell's paradox assumes that there
exists a set $S$ that is a collection of all sets that don't contain
themselves.  Now, either $S$ contains itself or it doesn't.  If it contains
itself, it contradicts its definition.  But if it doesn't contain itself, it
also contradicts its definition.  Russell's paradox is resolved in ZF set
theory by denying that such a set $S$ exists.}) are resolved.  Wait a
minute---shouldn't the axioms be a starting point, not an ending point?  If
there are paradoxes that arise without the axioms, how do you know you won't
stumble across one accidentally when using the informal approach?

And nowhere do these books describe how ``all of mathematics can be
derived from set theory'' which by now you've heard a few times.

You find a more advanced book on set theory.  This one actually lists the
axioms of ZF set theory in plain English on page one.  {\em Now} you think
your quest has ended and you've finally found the source of all mathematical
knowledge; you just have to understand what it means.  Here, in one place, is
the basis for all of mathematics!  You stare at the axioms in awe, puzzle over
them, memorize them, hoping that if you just meditate on them long enough they
will become clear.  Of course, you haven't the slightest idea how the rest of
mathematics is ``derived'' from them; in particular, if these are the axioms
of mathematics, then why do arithmetic, group theory, and so on need their own
axioms?

You start reading this advanced book carefully, pondering the meaning of every
word, because by now you're really determined to get to the bottom of this.
The first thing the book does is explain how the axioms came about, which was
to resolve Russell's paradox.\index{Russell's paradox}  In fact that seems to
be the main purpose of their existence; that they supposedly can be used to
derive all of mathematics seems irrelevant and is not even mentioned.  Well,
you go on.  You hope the book will explain to you clearly, step by step, how
to derive things from the axioms.  After all, this is the starting point of
mathematics, like a book that explains the basics of a computer programming
language.  But something is missing.  You find you can't even understand the
first proof or do the first exercise.  Symbols such as $\exists$ and $\forall$
permeate the page without any mention of where they came from or how to
manipulate them; the author assumes you are totally familiar with them and
doesn't even tell you what they mean.  By now you know that $\exists$ means
``there exists'' and $\forall$ means ``for all,'' but shouldn't the rules for
manipulating these symbols be part of the axioms?  You still have no idea
how you could even describe the axioms to a computer.

Certainly there is something much different here from the technical
literature you're used to reading.  A computer language manual almost
always explains very clearly what all the symbols mean, precisely what
they do, and the rules used for combining them, and you work your way up
from there.

After glancing at four or five other such books, you come to the realization
that there is another whole field of study that you need just to get to the
point at which you can understand the axioms of set theory.  The field is
called ``logic.''  In fact, some of the books did recommend it as a
prerequisite, but it just didn't sink in.  You assumed logic was, well, just
logic, something that a person with common sense intuitively understood.  Why
waste your time reading boring treatises on symbolic logic, the manipulation
of 1's and 0's that computers do, when you already know that?  But this is a
different kind of logic, quite alien to you.  The subject of {\sc nand} and
{\sc nor} gates is not even touched upon or in any case has to do with only a
very small part of this field.

So your quest continues.  Skimming through the first couple of introductory
books, you get a general idea of what logic is about and what quantifiers
(``for all,'' ``there exists'') mean, but you find their examples somewhat
trivial and mildly annoying (``all dogs are animals,'' ``some animals are
dogs,'' and such).  But all you want to know is what the rules are for
manipulating the symbols so you can apply them to set theory.  Some formulas
describing the relationships among quantifiers ($\exists$ and $\forall$) are
listed in tables, along with some verbal reasoning to justify them.
Presumably, if you want to find out if a formula is correct, you go through
this same kind of mental reasoning process, possibly using images of dogs and
animals. Intuitively, the formulas seem to make sense.  But when you ask
yourself, ``What are the rules I need to get a computer to figure out whether
this formula is correct?'', you still don't know.  Certainly you don't ask the
computer to imagine dogs and animals.

You look at some more advanced logic books.  Many of them have an introductory
chapter summarizing set theory, which turns out to be a prerequisite.  You
need logic to understand set theory, but it seems you also need set theory to
understand logic!  These books jump right into proving rather advanced
theorems about logic, without offering the faintest clue about where the logic
came from that allows them to prove these theorems.

Luckily, you come across an elementary book of logic that, halfway through,
after the usual truth tables and metaphors, presents in a clear, precise way
what you've been looking for all along: the axioms!  They're divided into
propositional calculus (also called sentential logic) and predicate calculus
(also called first-order logic),\index{first-order logic} with rules so simple
and crystal clear that now you can finally program a computer to understand
them.  Indeed, they're no harder than learning how to play a game of chess.
As far as what you seem to need is concerned, the whole book could have been
written in five pages!

{\em Now} you think you've found the ultimate source of mathematical
truth.  So---the axioms of mathematics consist of these axioms of logic,
together with the axioms of ZF set theory. (By now you've also been able to
figure out how to translate the ZF axioms from English into the
actual symbols of logic which you can now manipulate according to
precise, easy-to-understand rules.)

Of course, you still don't understand how ``all of mathematics can be
derived from set theory,'' but maybe this will reveal itself in due
course.

You eagerly set out to program the axioms and rules into a computer and start
to look at the theorems you will have to prove as the logic is developed.  All
sorts of important theorems start popping up:  the deduction
theorem,\index{deduction theorem} the substitution theorem,\index{substitution
theorem} the completeness theorem of propositional calculus,\index{first-order
logic!completeness} the completeness theorem of predicate calculus.  Uh-oh,
there seems to be trouble.  They all get harder and harder, and not one of
them can be derived with the axioms and rules of logic you've just been
handed.  Instead, they all require ``metalogic'' for their proofs, a kind of
mixture of logic and set theory that allows you to prove things {\em about}
the axioms and theorems of logic rather than {\em with} them.

You plow ahead anyway.  A month later, you've spent much of your
free time getting the computer to verify proofs in propositional calculus.
You've programmed in the axioms, but you've also had to program in the
deduction theorem, the substitution theorem, and the completeness theorem of
propositional calculus, which by now you've resigned yourself to treating as
rather complex additional axioms, since they can't be proved from the axioms
you were given.  You can now get the computer to verify and even generate
complete, rigorous, formal proofs\index{formal proof}.  Never mind that they
may have 100,000 steps---at least now you can have complete, absolute
confidence in them.  Unfortunately, the only theorems you have proved are
pretty trivial and you can easily verify them in a few minutes with truth
tables, if not by inspection.

It looks like your mathematician friend was right.  Getting the computer to do
serious mathematics with this kind of rigor seems almost hopeless.  Even
worse, it seems that the further along you get, the more ``axioms'' you have
to add, as each new theorem seems to involve additional ``metamathematical''
reasoning that hasn't been formalized, and none of it can be derived from the
axioms of logic.  Not only do the proofs keep growing exponentially as you get
further along, but the program to verify them keeps getting bigger and bigger
as you program in more ``metatheorems.''\index{metatheorem}\footnote{A
metatheorem is usually a statement that is too general to be directly provable
in a theory.  For example, ``if $n_1$, $n_2$, and $n_3$ are integers, then
$n_1+n_2+n_3$ is an integer'' is a theorem of number theory.  But ``for any
integer $k > 1$, if $n_1, \ldots, n_k$ are integers, then $n_1+\ldots +n_k$ is
an integer'' is a metatheorem, in other words a family of theorems, one for
every $k$.  The reason it is not a theorem is that the general sum $n_1+\ldots
+n_k$ (as a function of $k$) is not an operation that can be defined directly
in number theory.} The bugs\index{computer program bugs} that have cropped up
so far have already made you start to lose faith in the rigor you seem to have
achieved, and you know it's just going to get worse as your program gets larger.

\subsection{Mathematics and the Non-Specialist}

\begin{quote}
  {\em A real proof is not checkable by a machine, or even by any mathematician
not privy to the gestalt, the mode of thought of the particular field of
mathematics in which the proof is located.}
  \flushright\sc  Davis and Hersh\index{Davis, Phillip J.}
  \footnote{\cite{Davis}, p.~354.}\\
\end{quote}

The bulk of abstract or theoretical mathematics is ordinarily outside
the reach of anyone but a few specialists in each field who have completed
the necessary difficult internship in order to enter its coterie.  The
typical intelligent layperson has no reasonable hope of understanding much of
it, nor even the specialist mathematician of understanding other fields.  It
is like a foreign language that has no dictionary to look up the translation;
the only way you can learn it is by living in the country for a few years.  It
is argued that the effort involved in learning a specialty is a necessary
process for acquiring a deep understanding.  Of course, this is almost certainly
true if one is to make significant contributions to a field; in particular,
``doing'' proofs is probably the most important part of a mathematician's
training.  But is it also necessary to deny outsiders access to it?  Is it
necessary that abstract mathematics be so hard for a layperson to grasp?

A computer normally is of no help whatsoever.  Most published proofs are
actually just series of hints written in an informal style that requires
considerable knowledge of the field to understand.  These are the ``real
proofs'' referred to by Davis and Hersh.\index{informal proof}  There is an
implicit understanding that, in principle, such a proof could be converted to
a complete formal proof\index{formal proof}.  However, it is said that no one
would ever attempt such a conversion, even if they could, because that would
presumably require millions of steps (Section~\ref{dream}).  Unfortunately the
informal style automatically excludes the understanding of the proof
by anyone who hasn't gone through the necessary apprenticeship. The
best that the intelligent layperson can do is to read popular books about deep
and famous results; while this can be helpful, it can also be misleading, and
the lack of detail usually leaves the reader with no ability whatsoever to
explore any aspect of the field being described.

The statements of theorems often use sophisticated notation that makes them
inaccessible to the non-specialist.  For a non-specialist who wants to achieve
a deeper understanding of a proof, the process of tracing definitions and
lemmas back through their hierarchy\index{hierarchy} quickly becomes confusing
and discouraging.  Textbooks are usually written to train mathematicians or to
communicate to people who are already mathematicians, and large gaps in proofs
are often left as exercises to the reader who is left at an impasse if he or
she becomes stuck.

I believe that eventually computers will enable non-specialists and even
intelligent laypersons to follow almost any mathematical proof in any field.
Metamath is an attempt in that direction.  If all of mathematics were as
easily accessible as a computer programming language, I could envision
computer programmers and hobbyists who otherwise lack mathematical
sophistication exploring and being amazed by the world of theorems and proofs
in obscure specialties, perhaps even coming up with results of their own.  A
tremendous advantage would be that anyone could experiment with conjectures in
any field---the computer would offer instant feedback as to whether
an inference step was correct.

Mathematicians sometimes have to put up with the annoyance of
cranks\index{cranks} who lack a fundamental understanding of mathematics but
insist that their ``proofs'' of, say, Fermat's Last Theorem\index{Fermat's
Last Theorem} be taken seriously.  I think part of the problem is that these
people are misled by informal mathematical language, treating it as if they
were reading ordinary expository English and failing to appreciate the
implicit underlying rigor.  Such cranks are rare in the field of computers,
because computer languages are much more explicit, and ultimately the proof is
in whether a computer program works or not.  With easily accessible
computer-based abstract mathematics, a mathematician could say to a crank,
``don't bother me until you've demonstrated your claim on the computer!''

% 22-May-04 nm
% Attempt to move De Millo quote so it doesn't separate from attribution
% CHANGE THIS NUMBER (AND ELIMINATE IF POSSIBLE) WHEN ABOVE TEXT CHANGES
\vspace{-0.5em}

\subsection{An Impossible Dream?}\label{dream}

\begin{quote}
  {\em Even quite basic theorems would demand almost unbelievably vast
  books to display their proofs.}
    \flushright\sc  Robert E. Edwards\footnote{\cite{Edwards}, p.~68.}\\
\end{quote}\index{Edwards, Robert E.}

\begin{quote}
  {\em Oh, of course no one ever really {\em does} it.  It would take
  forever!  You just show that you could do it, that's sufficient.}
    \flushright\sc  ``The Ideal Mathematician''
    \index{Davis, Phillip J.}\footnote{\cite{Davis},
p.~40.}\\
\end{quote}

\begin{quote}
  {\em There is a theorem in the primitive notation of set theory that
  corresponds to the arithmetic theorem `$1000+2000=3000$'.  The formula
  would be forbiddingly long\ldots even if [one] knows the definitions
  and is asked to simplify the long formula according to them, chances are
  he will make errors and arrive at some incorrect result.}
    \flushright\sc  Hao Wang\footnote{\cite{Wang}, p.~140.}\\
\end{quote}\index{Wang, Hao}

% 22-May-04 nm
% Attempt to move De Millo quote so it doesn't separate from attribution
% CHANGE THIS NUMBER (AND ELIMINATE IF POSSIBLE) WHEN ABOVE TEXT CHANGES
\vspace{-0.5em}

\begin{quote}
  {\em The {\em Principia Mathematica} was the crowning achievement of the
  formalists.  It was also the deathblow of the formalist view.\ldots
  {[Rus\-sell]} failed, in three enormous volumes, to get beyond the elementary
  facts of arithmetic.  He showed what can be done in principle and what
  cannot be done in practice.  If the mathematical process were really
  one of strict, logical progression, we would still be counting our
  fingers.\ldots
  One theoretician estimates\ldots that a demonstration of one of
  Ramanujan's conjectures assuming set theory and elementary analysis would
  take about two thousand pages; the length of a deduction from first principles
  is nearly in\-con\-ceiv\-a\-ble\ldots The probabilists argue that\ldots any
  very long proof can at best be viewed as only probably correct\ldots}
  \flushright\sc Richard de Millo et. al.\footnote{\cite{deMillo}, pp.~269,
  271.}\\
\end{quote}\index{de Millo, Richard}

A number of writers have conveyed the impression that the kind of absolute
rigor provided by Metamath\index{Metamath} is an impossible dream, suggesting
that a complete, formal verification\index{formal proof} of a typical theorem
would take millions of steps in untold volumes of books.  Even if it could be
done, the thinking sometimes goes, all meaning would be lost in such a
monstrous, tedious verification.\index{informal proof}\index{proof length}

These writers assume, however, that in order to achieve the kind of complete
formal verification they desire one must break down a proof into individual
primitive steps that make direct reference to the axioms.  This is
not necessary.  There is no reason not to make use of previously proved
theorems rather than proving them over and over.

Just as important, definitions\index{definition} can be introduced along
the way, allowing very complex formulas to be represented with few
symbols.  Not doing this can lead to absurdly long formulas.  For
example, the mere statement of
G\"{o}del's incompleteness theorem\index{G\"{o}del's
incompleteness theorem}, which can be expressed with a small number of
defined symbols, would require about 20,000 primitive symbols to express
it.\index{Boolos, George S.}\footnote{George S.\ Boolos, lecture at
Massachusetts Institute of Technology, spring 1990.} An extreme example
is Bourbaki's\label{bourbaki} language for set theory, which requires
4,523,659,424,929 symbols plus 1,179,618,517,981 disambiguatory links
(lines connecting symbol pairs, usually drawn below or above the
formula) to express the number
``one'' \cite{Mathias}.\index{Mathias, Adrian R. D.}\index{Bourbaki,
Nicolas}
% http://www.dpmms.cam.ac.uk/~ardm/

A hierarchy\index{hierarchy} of theorems and definitions permits an
exponential growth in the formula sizes and primitive proof steps to be
described with only a linear growth in the number of symbols used.  Of course,
this is how ordinary informal mathematics is normally done anyway, but with
Metamath\index{Metamath} it can be done with absolute rigor and precision.

\subsection{Beauty}


\begin{quote}
  {\em No one shall be able to drive us from the paradise that Cantor has
created for us.}
   \flushright\sc  David Hilbert\footnote{As quoted in \cite{Moore}, p.~131.}\\
\end{quote}\index{Hilbert, David}

\needspace{3\baselineskip}
\begin{quote}
  {\em Mathematics possesses not only truth, but some supreme beauty ---a
  beauty cold and austere, like that of a sculpture.}
    \flushright\sc  Bertrand
    Russell\footnote{\cite{Russell}.}\\
\end{quote}\index{Russell, Bertrand}

\begin{quote}
  {\em Euclid alone has looked on Beauty bare.}
  \flushright\sc Edna Millay\footnote{As quoted in \cite{Davis}, p.~150.}\\
\end{quote}\index{Millay, Edna}

For most people, abstract mathematics is distant, strange, and
incomprehensible.  Many popular books have tried to convey some of the sense
of beauty in famous theorems.  But even an intelligent layperson is left with
only a general idea of what a theorem is about and is hardly given the tools
needed to make use of it.  Traditionally, it is only after years of arduous
study that one can grasp the concepts needed for deep understanding.
Metamath\index{Metamath} allows you to approach the proof of the theorem from
a quite different perspective, peeling apart the formulas and definitions
layer by layer until an entirely different kind of understanding is achieved.
Every step of the proof is there, pieced together with absolute precision and
instantly available for inspection through a microscope with a magnification
as powerful as you desire.

A proof in itself can be considered an object of beauty.  Constructing an
elegant proof is an art.  Once a famous theorem has been proved, often
considerable effort is made to find simpler and more easily understood
proofs.  Creating and communicating elegant proofs is a major concern of
mathematicians.  Metamath is one way of providing a common language for
archiving and preserving this information.

The length of a proof can, to a certain extent, be considered an
objective measure of its ``beauty,'' since shorter proofs are usually
considered more elegant.  In the set theory database
\texttt{set.mm}\index{set theory database (\texttt{set.mm})}%
\index{Metamath Proof Explorer}
provided with Metamath, one goal was to make all proofs as short as possible.

\needspace{4\baselineskip}
\subsection{Simplicity}

\begin{quote}
  {\em God made man simple; man's complex problems are of his own
  devising.}
    \flushright\sc Eccles. 7:29\footnote{Jerusalem Bible.}\\
\end{quote}\index{Bible}

\needspace{3\baselineskip}
\begin{quote}
  {\em God made integers, all else is the work of man.}
    \flushright\sc Leopold Kronecker\footnote{{\em Jahresbericht
	der Deutschen Mathematiker-Vereinigung }, vol. 2, p. 19.}\\
\end{quote}\index{Kronecker, Leopold}

\needspace{3\baselineskip}
\begin{quote}
  {\em For what is clear and easily comprehended attracts; the
  complicated repels.}
    \flushright\sc David Hilbert\footnote{As quoted in \cite{deMillo},
p.~273.}\\
\end{quote}\index{Hilbert, David}

The Metamath\index{Metamath} language is simple and Spartan.  Metamath treats
all mathematical expressions as simple sequences of symbols, devoid of meaning.
The higher-level or ``metamathematical'' notions underlying Metamath are about
as simple as they could possibly be.  Each individual step in a proof involves
a single basic concept, the substitution of an expression for a variable, so
that in principle almost anyone, whether mathematician or not, can
completely understand how it was arrived at.

In one of its most basic applications, Metamath\index{Metamath} can be used to
develop the foundations of mathematics\index{foundations of mathematics} from
the very beginning.  This is done in the set theory database that is provided
with the Metamath package and is the subject matter
of Chapter~\ref{fol}. Any language (a metalanguage\index{metalanguage})
used to describe mathematics (an object language\index{object language}) must
have a mathematical content of its own, but it is desirable to keep this
content down to a bare minimum, namely that needed to make use of the
inference rules specified by the axioms.  With any metalanguage there is a
``chicken and egg'' problem somewhat like circular reasoning:  you must assume
the validity of the mathematics of the metalanguage in order to prove the
validity of the mathematics of the object language.  The mathematical content
of Metamath itself is quite limited.  Like the rules of a game of chess, the
essential concepts are simple enough so that virtually anyone should be able to
understand them (although that in itself will not let you play like
a master).  The symbols that Metamath manipulates do not in themselves
have any intrinsic meaning.  Your interpretation of the axioms that you supply
to Metamath is what gives them meaning.  Metamath is an attempt to strip down
mathematical thought to its bare essence and show you exactly how the symbols
are manipulated.

Philosophers and logicians, with various motivations, have often thought it
important to study ``weak'' fragments of logic\index{weak logic}
\cite{Anderson}\index{Anderson, Alan Ross} \cite{MegillBunder}\index{Megill,
Norman}\index{Bunder, Martin}, other unconventional systems of logic (such as
``modal'' logic\index{modal logic} \cite[ch.\ 27]{Boolos}\index{Boolos, George
S.}), and quantum logic\index{quantum logic} in physics
\cite{Pavicic}\index{Pavi{\v{c}}i{\'{c}}, M.}.  Metamath\index{Metamath}
provides a framework in which such systems can be expressed, with an absolute
precision that makes all underlying metamathematical assumptions rigorous and
crystal clear.

Some schools of philosophical thought, for example
intuitionism\index{intuitionism} and constructivism\index{constructivism},
demand that the notions underlying any mathematical system be as simple and
concrete as possible.  Metamath should meet the requirements of these
philosophies.  Metamath must be taught the symbols, axioms\index{axiom}, and
rules\index{rule} for a specific theory, from the skeptical (such as
intuitionism\index{intuitionism}\footnote{Intuitionism does not accept the law
of excluded middle (``either something is true or it is not true'').  See
\cite[p.~xi]{Tymoczko}\index{Tymoczko, Thomas} for discussion and references
on this topic.  Consider the theorem, ``There exist irrational numbers $a$ and
$b$ such that $a^b$ is rational.''  An intuitionist would reject the following
proof:  If $\sqrt{2}^{\sqrt{2}}$ is rational, we are done.  Otherwise, let
$a=\sqrt{2}^{\sqrt{2}}$ and $b=\sqrt{2}$. Then $a^b=2$, which is rational.})
to the bold (such as the axiom of choice in set theory\footnote{The axiom of
choice\index{Axiom of Choice} asserts that given any collection of pairwise
disjoint nonempty sets, there exists a set that has exactly one element in
common with each set of the collection.  It is used to prove many important
theorems in standard mathematics.  Some philosophers object to it because it
asserts the existence of a set without specifying what the set contains
\cite[p.~154]{Enderton}\index{Enderton, Herbert B.}.  In one foundation for
mathematics due to Quine\index{Quine, Willard Van Orman}, that has not been
otherwise shown to be inconsistent, the axiom of choice turns out to be false
\cite[p.~23]{Curry}\index{Curry, Haskell B.}.  The \texttt{show
trace{\char`\_}back} command of the Metamath program allows you to find out
whether the axiom of choice, or any other axiom, was assumed by a
proof.}\index{\texttt{show trace{\char`\_}back} command}).

The simplicity of the Metamath language lets the algorithm (computer program)
that verifies the validity of a Metamath proof be straightforward and
robust.  You can have confidence that the theorems it verifies really can be
derived from your axioms.

\subsection{Rigor}

\begin{quote}
  {\em Rigor became a goal with the Greeks\ldots But the efforts to
  pursue rigor to the utmost have led to an impasse in which there is
  no longer any agreement on what it really means.  Mathematics remains
  alive and vital, but only on a pragmatic basis.}
    \flushright\sc  Morris Kline\footnote{\cite{Kline}, p.~1209.}\\
\end{quote}\index{Kline, Morris}

Kline refers to a much deeper kind of rigor than that which we will discuss in
this section.  G\"{o}del's incompleteness theorem\index{G\"{o}del's
incompleteness theorem} showed that it is impossible to achieve absolute rigor
in standard mathematics because we can never prove that mathematics is
consistent (free from contradictions).\index{consistent theory}  If
mathematics is consistent, we will never know it, but must rely on faith.  If
mathematics is inconsistent, the best we can hope for is that some clever
future mathematician will discover the inconsistency.  In this case, the
axioms would probably be revised slightly to eliminate the inconsistency, as
was done in the case of Russell's paradox,\index{Russell's paradox} but the
bulk of mathematics would probably not be affected by such a discovery.
Russell's paradox, for example, did not affect most of the remarkable results
achieved by 19th-century and earlier mathematicians.  It mainly invalidated
some of Gottlob Frege's\index{Frege, Gottlob} work on the foundations of
mathematics in the late 1800's; in fact Frege's work inspired Russell's
discovery.  Despite the paradox, Frege's work contains important concepts that
have significantly influenced modern logic.  Kline's {\em Mathematics, The
Loss of Certainty} \cite{Klinel}\index{Kline, Morris} has an interesting
discussion of this topic.

What {\em can} be achieved with absolute certainty\index{certainty} is the
knowledge that if we assume the axioms are consistent and true, then the
results derived from them are true.  Part of the beauty of mathematics is that
it is the one area of human endeavor where absolute certainty can be achieved
in this sense.  A mathematical truth will remain such for eternity.  However,
our actual knowledge of whether a particular statement is a mathematical truth
is only as certain as the correctness of the proof that establishes it.  If
the proof of a statement is questionable or vague, we can't have absolute
confidence in the truth that the statement claims.

Let us look at some traditional ways of expressing proofs.

Except in the field of formal logic\index{formal logic}, almost all
traditional proofs in mathematics are really not proofs at all, but rather
proof outlines or hints as to how to go about constructing the proof.  Many
gaps\index{gaps in proofs} are left for the reader to fill in. There are
several reasons for this.  First, it is usually assumed in mathematical
literature that the person reading the proof is a mathematician familiar with
the specialty being described, and that the missing steps are obvious to such
a reader or at least that the reader is capable of filling them in.  This
attitude is fine for professional mathematicians in the specialty, but
unfortunately it often has the drawback of cutting off the rest of the world,
including mathematicians in other specialties, from understanding the proof.
We discussed one possible resolution to this on p.~\pageref{envision}.
Second, it is often assumed that a complete formal proof\index{formal proof}
would require countless millions of symbols (Section~\ref{dream}). This might
be true if the proof were to be expressed directly in terms of the axioms of
logic and set theory,\index{set theory} but it is usually not true if we allow
ourselves a hierarchy\index{hierarchy} of definitions and theorems to build
upon, using a notation that allows us to introduce new symbols, definitions,
and theorems in a precisely specified way.

Even in formal logic,\index{formal logic} formal proofs\index{formal proof}
that are considered complete still contain hidden or implicit information.
For example, a ``proof'' is usually defined as a sequence of
wffs,\index{well-formed formula (wff)}\footnote{A {\em wff} or well-formed
formula is a mathematical expression (string of symbols) constructed according
to some precise rules.  A formal mathematical system\index{formal system}
contains (1) the rules for constructing syntactically correct
wffs,\index{syntax rules} (2) a list of starting wffs called
axioms,\index{axiom} and (3) one or more rules prescribing how to derive new
wffs, called theorems\index{theorem}, from the axioms or previously derived
theorems.  An example of such a system is contained in
Metamath's\index{Metamath} set theory database, which defines a formal
system\index{formal system} from which all of standard mathematics can be
derived.  Section~\ref{startf} steps you through a complete example of a formal
system, and you may want to skim it now if you are unfamiliar with the
concept.} each of which is an axiom or follows from a rule applied to previous
wffs in the sequence.  The implicit part of the proof is the algorithm by
which a sequence of symbols is verified to be a valid wff, given the
definition of a wff.  The algorithm in this case is rather simple, but for a
computer to verify the proof,\index{automated proof verification} it must have
the algorithm built into its verification program.\footnote{It is possible, of
course, to specify wff construction syntax outside of the program itself
with a suitable input language (the Metamath language being an example), but
some proof-verification or theorem-proving programs lack the ability to extend
wff syntax in such a fashion.} If one deals exclusively with axioms and
elementary wffs, it is straightforward to implement such an algorithm.  But as
more and more definitions are added to the theory in order to make the
expression of wffs more compact, the algorithm becomes more and more
complicated.  A computer program that implements the algorithm becomes larger
and harder to understand as each definition is introduced, and thus more prone
to bugs.\index{computer program bugs}  The larger the program, the
more suspicious the mathematician may be about
the validity of its algorithms.  This is especially true because
computer programs are inherently hard to follow to begin with, and few people
enjoy verifying them manually in detail.

Metamath\index{Metamath} takes a different approach.  Metamath's ``knowledge''
is limited to the ability to substitute variables for expressions, subject to
some simple constraints.  Once the basic algorithm of Metamath is assumed to
be debugged, and perhaps independently confirmed, it
can be trusted once and for all.  The information that Metamath needs to
``understand'' mathematics is contained entirely in the body of knowledge
presented to Metamath.  Any errors in reasoning can only be errors in the
axioms or definitions contained in this body of knowledge.  As a
``constructive'' language\index{constructive language} Metamath has no
conditional branches or loops like the ones that make computer programs hard
to decipher; instead, the language can only build new sequences of symbols
from earlier sequences  of symbols.

The simplicity of the rules that underlie Metamath not only makes Metamath
easy to learn but also gives Metamath a great deal of flexibility. For
example, Metamath is not limited to describing standard first-order
logic\index{first-order logic}; higher-order logics\index{higher-order logic}
and fragments of logic\index{weak logic} can be described just as easily.
Metamath gives you the freedom to define whatever wff notation you prefer; it
has no built-in conception of the syntax of a wff.\index{well-formed formula
(wff)}  With suitable axioms and definitions, Metamath can even describe and
prove things about itself.\index{Metamath!self-description}  (John
Harrison\index{Harrison, John} discusses the ``reflection''
principle\index{reflection principle} involved in self-descriptive systems in
\cite{Harrison}.)

The flexibility of Metamath requires that its proofs specify a lot of detail,
much more than in an ordinary ``formal'' proof.\index{formal proof}  For
example, in an ordinary formal proof, a single step consists of displaying the
wff that constitutes that step.  In order for a computer program to verify
that the step is acceptable, it first must verify that the symbol sequence
being displayed is an acceptable wff.\index{automated proof verification} Most
proof verifiers have at least basic wff syntax built into their programs.
Metamath has no hard-wired knowledge of what constitutes a wff built into it;
instead every wff must be explicitly constructed based on rules defining wffs
that are present in a database.  Thus a single step in an ordinary formal
proof may be correspond to many steps in a Metamath proof. Despite the larger
number of steps, though, this does not mean that a Metamath proof must be
significantly larger than an ordinary formal proof. The reason is that since
we have constructed the wff from scratch, we know what the wff is, so there is
no reason to display it.  We only need to refer to a sequence of statements
that construct it.  In a sense, the display of the wff in an ordinary formal
proof is an implicit proof of its own validity as a wff; Metamath just makes
the proof explicit. (Section~\ref{proof} describes Metamath's proof notation.)

\section{Computers and Mathematicians}

\begin{quote}
  {\em The computer is important, but not to mathematics.}
    \flushright\sc  Paul Halmos\footnote{As quoted in \cite{Albers}, p.~121.}\\
\end{quote}\index{Halmos, Paul}

Pure mathematicians have traditionally been indifferent to computers, even to
the point of disdain.\index{computers and pure mathematics}  Computer science
itself is sometimes considered to fall in the mundane realm of ``applied''
mathematics, perhaps essential for the real world but intellectually unexciting
to those who seek the deepest truths in mathematics.  Perhaps a reason for this
attitude towards computers is that there is little or no computer software that
meets their needs, and there may be a general feeling that such software could
not even exist.  On the one hand, there are the practical computer algebra
systems, which can perform amazing symbolic manipulations in algebra and
calculus,\index{computer algebra system} yet can't prove the simplest
existence theorem, if the idea of a proof is present at all.  On the other
hand, there are specialized automated theorem provers that technically speaking
may generate correct proofs.\index{automated theorem proving}  But sometimes
their specialized input notation may be cryptic and their output perceived to
be long, inelegant, incomprehensible proofs.    The output
may be viewed with suspicion, since the program that generates it tends to be
very large, and its size increases the potential for bugs\index{computer
program bugs}.  Such a proof may be considered trustworthy only if
independently verified and ``understood'' by a human, but no one wants to
waste their time on such a boring, unrewarding chore.



\needspace{4\baselineskip}
\subsection{Trusting the Computer}

\begin{quote}
  {\em \ldots I continue to find the quasi-empirical interpretation of
  computer proofs to be the more plausible.\ldots Since not
  everything that claims to be a computer proof can be
  accepted as valid, what are the mathematical criteria for acceptable
  computer proofs?}
    \flushright\sc  Thomas Tymoczko\footnote{\cite{Tymoczko}, p.~245.}\\
\end{quote}\index{Tymoczko, Thomas}

In some cases, computers have been essential tools for proving famous
theorems.  But if a proof is so long and obscure that it can be verified in a
practical way only with a computer, it is vaguely felt to be suspicious.  For
example, proving the famous four-color theorem\index{four-color
theorem}\index{proof length} (``a map needs no more than four colors to
prevent any two adjacent countries from having the same color'') can presently
only be done with the aid of a very complex computer program which originally
required 1200 hours of computer time. There has been considerable debate about
whether such a proof can be trusted and whether such a proof is ``real''
mathematics \cite{Swart}\index{Swart, E. R.}.\index{trusting computers}

However, under normal circumstances even a skeptical mathematician would have a
great deal of confidence in the result of multiplying two numbers on a pocket
calculator, even though the precise details of what goes on are hidden from its
user.  Even the verification on a supercomputer that a huge number is prime is
trusted, especially if there is independent verification; no one bothers to
debate the philosophical significance of its ``proof,'' even though the actual
proof would be so large that it would be completely impractical to ever write
it down on paper.  It seems that if the algorithm used by the computer is
simple enough to be readily understood, then the computer can be trusted.

Metamath\index{Metamath} adopts this philosophy.  The simplicity of its
language makes it easy to learn, and because of its simplicity one can have
essentially absolute confidence that a proof is correct. All axioms, rules, and
definitions are available for inspection at any time because they are defined
by the user; there are no hidden or built-in rules that may be prone to subtle
bugs\index{computer program bugs}.  The basic algorithm at the heart of
Metamath is simple and fixed, and it can be assumed to be bug-free and robust
with a degree of confidence approaching certainty.
Independently written implementations of the Metamath verifier
can reduce any residual doubt on the part of a skeptic even further;
there are now over a dozen such implementations, written by many people.

\subsection{Trusting the Mathematician}\label{trust}

\begin{quote}
  {\em There is no Algebraist nor Mathematician so expert in his science, as
  to place entire confidence in any truth immediately upon his discovery of it,
  or regard it as any thing, but a mere probability.  Every time he runs over
  his proofs, his confidence encreases; but still more by the approbation of
  his friends; and is rais'd to its utmost perfection by the universal assent
  and applauses of the learned world.}
  \flushright\sc David Hume\footnote{{\em A Treatise of Human Nature}, as
  quoted in \cite{deMillo}, p.~267.}\\
\end{quote}\index{Hume, David}

\begin{quote}
  {\em Stanislaw Ulam estimates that mathematicians publish 200,000 theorems
  every year.  A number of these are subsequently contradicted or otherwise
  disallowed, others are thrown into doubt, and most are ignored.}
  \flushright\sc Richard de Millo et. al.\footnote{\cite{deMillo}, p.~269.}\\
\end{quote}\index{Ulam, Stanislaw}

Whether or not the computer can be trusted, humans  of course will occasionally
err. Only the most memorable proofs get independently verified, and of these
only a handful of truly great ones achieve the status of being ``known''
mathematical truths that are used without giving a second thought to their
correctness.

There are many famous examples of incorrect theorems and proofs in
mathematical literature.\index{errors in proofs}

\begin{itemize}
\item There have been thousands of purported proofs of Fermat's Last
Theorem\index{Fermat's Last Theorem} (``no integer solutions exist to $x^n +
y^n = z^n$ for $n > 2$''), by amateurs, cranks, and well-regarded
mathematicians \cite[p.~5]{Stark}\index{Stark, Harold M}.  Fermat wrote a note
in his copy of Bachet's {\em Diophantus} that he found ``a truly marvelous
proof of this theorem but this margin is too narrow to contain it''
\cite[p.~507]{Kramer}.  A recent, much publicized proof by Yoichi
Miyaoka\index{Miyaoka, Yoichi} was shown to be incorrect ({\em Science News},
April 9, 1988, p.~230).  The theorem was finally proved by Andrew
Wiles\index{Wiles, Andrew} ({\em Science News}, July 3, 1993, p.~5), but it
initially had some gaps and took over a year after its announcement to be
checked thoroughly by experts.  On Oct. 25, 1994, Wiles announced that the last
gap found in his proof had been filled in.
  \item In 1882, M. Pasch discovered that an axiom was omitted from Euclid's
formulation of geometry\index{Euclidean geometry}; without it, the proofs of
important theorems of Euclid are not valid.  Pasch's axiom\index{Pasch's
axiom} states that a line that intersects one side of a triangle must also
intersect another side, provided that it does not touch any of the triangle's
vertices.  The omission of Pasch's axiom went unnoticed for 2000
years \cite[p.~160]{Davis}, in spite of (one presumes) the thousands of
students, instructors, and mathematicians who studied Euclid.
  \item The first published proof of the famous Schr\"{o}der--Bernstein
theorem\index{Schr\"{o}der--Bernstein theorem} in set theory was incorrect
\cite[p.~148]{Enderton}\index{Enderton, Herbert B.}.  This theorem states
that if there exists a 1-to-1 function\footnote{A {\em set}\index{set} is any
collection of objects. A {\em function}\index{function} or {\em
mapping}\index{mapping} is a rule that assigns to each element of one set
(called the function's {\em domain}\index{domain}) an element from another
set.} from set $A$ into set $B$ and vice-versa, then sets $A$ and $B$ have
a 1-to-1 correspondence.  Although it sounds simple and obvious,
the standard proof is quite long and complex.
  \item In the early 1900's, Hilbert\index{Hilbert, David} published a
purported proof of the continuum hypothesis\index{continuum hypothesis}, which
was eventually established as unprovable by Cohen\index{Cohen, Paul} in 1963
\cite[p.~166]{Enderton}.  The continuum hypothesis states that no
infinity\index{infinity} (``transfinite cardinal number'')\index{cardinal,
transfinite} exists whose size (or ``cardinality''\index{cardinality}) is
between the size of the set of integers and the size of the set of real
numbers.  This hypothesis originated with German mathematician Georg
Cantor\index{Cantor, Georg} in the late 1800's, and his inability to prove it
is said to have contributed to mental illness that afflicted him in his later
years.
  \item An incorrect proof of the four-color theorem\index{four-color theorem}
was published by Kempe\index{Kempe, A. B.} in 1879
\cite[p.~582]{Courant}\index{Courant, Richard}; it stood for 11 years before
its flaw was discovered.  This theorem states that any map can be colored
using only four colors, so that no two adjacent countries have the same
color.  In 1976 the theorem was finally proved by the famous computer-assisted
proof of Haken, Appel, and Koch \cite{Swart}\index{Appel, K.}\index{Haken,
W.}\index{Koch, K.}.  Or at least it seems that way.  Mathematician
H.~S.~M.~Coxeter\index{Coxeter, H. S. M.} has doubts \cite[p.~58]{Davis}:  ``I
have a feeling that that is an untidy kind of use of the computers, and the more
you correspond with Haken and Appel, the more shaky you seem to be.''
  \item Many false ``proofs'' of the Poincar\'{e}
conjecture\index{Poincar\'{e} conjecture} have been proposed over the years.
This conjecture states that any object that mathematically behaves like a
three-dimensional sphere is a three-dimensional sphere topologically,
regardless of how it is distorted.  In March 1986, mathematicians Colin
Rourke\index{Rourke, Colin} and Eduardo R\^{e}go\index{R\^{e}go, Eduardo}
caused  a stir in the mathematical community by announcing that they had found
a proof; in November of that year the proof was found to be false \cite[p.
218]{PetersonI}.  It was finally proved in 2003 by Grigory Perelman
\label{poincare}\index{Szpiro, George}\index{Perelman, Grigory}\cite{Szpiro}.
 \end{itemize}

Many counterexamples to ``theorems'' in recent mathematical
literature related to Clifford algebras\index{Clifford algebras}
 have been found by Pertti
Lounesto (who passed away in 2002).\index{Lounesto, Pertti}
See the web page \url{http://mathforum.org/library/view/4933.html}.
% http://users.tkk.fi/~ppuska/mirror/Lounesto/counterexamples.htm

One of the purposes of Metamath\index{Metamath} is to allow proofs to be
expressed with absolute precision.  Developing a proof in the Metamath
language can be challenging, because Metamath will not permit even the
tiniest mistake.\index{errors in proofs}  But once the proof is created, its
correctness can be trusted immediately, without having to depend on the
process of peer review for confirmation.

\section{The Use of Computers in Mathematics}

\subsection{Computer Algebra Systems}

For the most part, you will find that Metamath\index{Metamath} is not a
practical tool for manipulating numbers.  (Even proving that $2 + 2 = 4$, if
you start with set theory, can be quite complex!)  Several commercial
mathematics packages are quite good at arithmetic, algebra, and calculus, and
as practical tools they are invaluable.\index{computer algebra system} But
they have no notion of proof, and cannot understand statements starting with
``there exists such and such...''.

Software packages such as Mathematica \cite{Wolfram}\index{Mathematica} do not
concern themselves with proofs but instead work directly with known results.
These packages primarily emphasize heuristic rules such as the substitution of
equals for equals to achieve simpler expressions or expressions in a different
form.  Starting with a rich collection of built-in rules and algorithms, users
can add to the collection by means of a powerful programming language.
However, results such as, say, the existence of a certain abstract object
without displaying the actual object cannot be expressed (directly) in their
languages.  The idea of a proof from a small set of axioms is absent.  Instead
this software simply assumes that each fact or rule you add to the built-in
collection of algorithms is valid.  One way to view the software is as a large
collection of axioms from which the software, with certain goals, attempts to
derive new theorems, for example equating a complex expression with a simpler
equivalent. But the terms ``theorem''\index{theorem} and
``proof,''\index{proof} for example, are not even mentioned in the index of
the user's manual for Mathematica.\index{Mathematica and proofs}  What is also
unsatisfactory from a philosophical point of view is that there is no way to
ensure the validity of the results other than by trusting the writer of each
application module or tediously checking each module by hand, similar to
checking a computer program for bugs.\index{computer program
bugs}\footnote{Two examples illustrate why the knowledge database of computer
algebra systems should sometimes be regarded with a certain caution.  If you
ask Mathematica (version 3.0) to \texttt{Solve[x\^{ }n + y\^{ }n == z\^{ }n , n]}
it will respond with \texttt{\{\{n-\char`\>-2\}, \{n-\char`\>-1\},
\{n-\char`\>1\}, \{n-\char`\>2\}\}}. In other words, Mathematica seems to
``know'' that Fermat's Last Theorem\index{Fermat's Last Theorem} is true!  (At
the time this version of Mathematica was released this fact was unknown.)  If
you ask Maple\index{Maple} to \texttt{solve(x\^{ }2 = 2\^{ }x)} then
\texttt{simplify(\{"\})}, it returns the solution set \texttt{\{2, 4\}}, apparently
unaware that 0.7666647\ldots is also a solution.} While of course extremely
valuable in applied mathematics, computer algebra systems tend to be of little
interest to the theoretical mathematician except as aids for exploring certain
specific problems.

Because of possible bugs, trusting the output of a computer algebra system for
use as theorems in a proof-verifier would defeat the latter's goal of rigor.
On the other hand, a fact such that a certain relatively large number is
prime, while easy for a computer algebra system to derive, might have a long,
tedious proof that could overwhelm a proof-verifier. One approach for linking
computer algebra systems to a proof-verifier while retaining the advantages of
both is to add a hypothesis to each such theorem indicating its source.  For
example, a constant {\sc maple} could indicate the theorem came from the Maple
package, and would mean ``assuming Maple is consistent, then\ldots''  This and
many other topics concerning the formalization of mathematics are discussed in
John Harrison's\index{Harrison, John} very interesting
PhD thesis~\cite{Harrison-thesis}.

\subsection{Automated Theorem Provers}\label{theoremprovers}

A mathematical theory is ``decidable''\index{decidable theory} if a mechanical
method or algorithm exists that is guaranteed to determine whether or not a
particular formula is a theorem.  Among the few theories that are decidable is
elementary geometry,\index{Euclidean geometry} as was shown by a classic
result of logician Alfred Tarski\index{Tarski, Alfred} in 1948
\cite{Tarski}.\footnote{Tarski's result actually applies to a subset of the
geometry discussed in elementary textbooks.  This subset includes most of what
would be considered elementary geometry but it is not powerful enough to
express, among other things, the notions of the circumference and area of a
circle.  Extending the theory in a way that includes notions such as these
makes the theory undecidable, as was also shown by Tarski.  Tarski's algorithm
is far too inefficient to implement practically on a computer.  A practical
algorithm for proving a smaller subset of geometry theorems (those not
involving concepts of ``order'' or ``continuity'') was discovered by Chinese
mathematician Wu Wen-ts\"{u}n in 1977 \cite{Chou}\index{Chou,
Shang-Ching}.}\index{Wen-ts{\"{u}}n, Wu}  But most theories, including
elementary arithmetic, are undecidable.  This fact contributes to keeping
mathematics alive and well, since many mathematicians believe
that they will never be
replaced by computers (if they believe Roger Penrose's argument that a
computer can never replace the brain \cite{Penrose}\index{Penrose, Roger}).
In fact,  elementary geometry is often considered a ``dead'' field
for the simple reason that it is decidable.

On the other hand, the undecidability of a theory does not mean that one cannot
use a computer to search for proofs, providing one is willing to give up if a
proof is not found after a reasonable amount of time.  The field of automated
theorem proving\index{automated theorem proving} specializes in pursuing such
computer searches.  Among the more successful results to date are those based
on an algorithm known as Robinson's resolution principle
\cite{Robinson}\index{Robinson's resolution principle}.

Automated theorem provers can be excellent tools for those willing to learn
how to use them.  But they are not widely used in mainstream pure
mathematics, even though they could probably be useful in many areas.  There
are several reasons for this.  Probably most important, the main goal in pure
mathematics is to arrive at results that are considered to be deep or
important; proving them is essential but secondary.  Usually, an automated
theorem prover cannot assist in this main goal, and by the time the main goal
is achieved, the mathematician may have already figured out the proof as a
by-product.  There is also a notational problem.  Mathematicians are used to
using very compact syntax where one or two symbols (heavily dependent on
context) can represent very complex concepts; this is part of the
hierarchy\index{hierarchy} they have built up to tackle difficult problems.  A
theorem prover on the other hand might require that a theorem be expressed in
``first-order logic,''\index{first-order logic} which is the logic on which
most of mathematics is ultimately based but which is not ordinarily used
directly because expressions can become very long.  Some automated theorem
provers are experimental programs, limited in their use to very specialized
areas, and the goal of many is simply research into the nature of automated
theorem proving itself.  Finally, much research remains to be done to enable
them to prove very deep theorems.  One significant result was a
computer proof by Larry Wos\index{Wos, Larry} and colleagues that every Robbins
algebra\index{Robbins algebra} is a Boolean  algebra\index{Boolean algebra}
({\em New York Times}, Dec. 10, 1996).\footnote{In 1933, E.~V.\
Huntington\index{Huntington, E. V.}
presented the following axiom system for
Boolean algebra with a unary operation $n$ and a binary operation $+$:
\begin{center}
    $x + y = y + x$ \\
    $(x + y) + z = x + (y + z)$ \\
    $n(n(x) + y) + n(n(x) + n(y)) = x$
\end{center}
Herbert Robbins\index{Robbins, Herbert}, a student of Huntington, conjectured
that the last equation can be replaced with a simpler one:
\begin{center}
    $n(n(x + y) + n(x + n(y))) = x$
\end{center}
Robbins and Huntington could not find a proof.  The problem was
later studied unsuccessfully by Tarski\index{Tarski, Alfred} and his
students, and it remained an unsolved problem until a
computer found the proof in 1996.  For more information on
the Robbins algebra problem see \cite{Wos}.}

How does Metamath\index{Metamath} relate to automated theorem provers?  A
theorem prover is primarily concerned with one theorem at a time (perhaps
tapping into a small database of known theorems) whereas Metamath is more like
a theorem archiving system, storing both the theorem and its proof in a
database for access and verification.  Metamath is one answer to ``what do you
do with the output of a theorem prover?''  and could be viewed as the
next step in the process.  Automated theorem provers could be useful tools for
helping develop its database.
Note that very long, automatically
generated proofs can make your database fat and ugly and cause Metamath's proof
verification to take a long time to run.  Unless you have a particularly good
program that generates very concise proofs, it might be best to consider the
use of automatically generated proofs as a quick-and-dirty approach, to be
manually rewritten at some later date.

The program {\sc otter}\index{otter@{\sc otter}}\footnote{\url{http://www.cs.unm.edu/\~mccune/otter/}.}, later succeeded by
prover9\index{prover9}\footnote{\url{https://www.cs.unm.edu/~mccune/mace4/}.},
have been historically influential.
The E prover\index{E prover}\footnote{\url{https://github.com/eprover/eprover}.}
is a maintained automated theorem prover
for full first-order logic with equality.
There are many other automated theorem provers as well.

If you want to combine automated theorem provers with Metamath
consider investigating
the book {\em Automated Reasoning:  Introduction and Applications}
\cite{Wos}\index{Wos, Larry}.  This book discusses
how to use {\sc otter} in a way that can
not only able to generate
relatively efficient proofs, it can even be instructed to search for
shorter proofs.  The effective use of {\sc otter} (and similar tools)
does require a certain
amount of experience, skill, and patience.  The axiom system used in the
\texttt{set.mm}\index{set theory database (\texttt{set.mm})} set theory
database can be expressed to {\sc otter} using a method described in
\cite{Megill}.\index{Megill, Norman}\footnote{To use those axioms with
{\sc otter}, they must be restated in a way that eliminates the need for
``dummy variables.''\index{dummy variable!eliminating} See the Comment
on p.~\pageref{nodd}.} When successful, this method tends to generate
short and clever proofs, but my experiments with it indicate that the
method will find proofs within a reasonable time only for relatively
easy theorems.  It is still fun to experiment with.

Reference \cite{Bledsoe}\index{Bledsoe, W. W.} surveys a number of approaches
people have explored in the field of automated theorem proving\index{automated
theorem proving}.

\subsection{Interactive Theorem Provers}\label{interactivetheoremprovers}

Finding proofs completely automatically is difficult, so there
are some interactive theorem provers that allow a human to guide the
computer to find a proof.
Examples include
HOL Light\index{HOL light}%
\footnote{\url{https://www.cl.cam.ac.uk/~jrh13/hol-light/}.},
Isabelle\index{Isabelle}%
\footnote{\url{http://www.cl.cam.ac.uk/Research/HVG/Isabelle}.},
{\sc hol}\index{hol@{\sc hol}}%
\footnote{\url{https://hol-theorem-prover.org/}.},
and
Coq\index{Coq}\footnote{\url{https://coq.inria.fr/}.}.

A major difference between most of these tools and Metamath is that the
``proofs'' are actually programs that guide the program to find a proof,
and not the proof itself.
For example, an Isabelle/HOL proof might apply a step
\texttt{apply (blast dest: rearrange reduction)}. The \texttt{blast}
instruction applies
an automatic tableux prover and returns if it found a sequence of proof
steps that work... but the sequence is not considered part of the proof.

A good overview of
higher-level proof verification languages (such as {\sc lcf}\index{lcf@{\sc
lcf}} and {\sc hol}\index{hol@{\sc hol}})
is given in \cite{Harrison}.  All of these languages are fundamentally
different from Metamath in that much of the mathematical foundational
knowledge is embedded in the underlying proof-verification program, rather
than placed directly in the database that is being verified.
These can have a steep learning curve for those without a mathematical
background.  For example, one usually must have a fair understanding of
mathematical logic in order to follow their proofs.

\subsection{Proof Verifiers}\label{proofverifiers}

A proof verifier is a program that doesn't generate proofs but instead
verifies proofs that you give it.  Many proof verifiers have limited built-in
automated proof capabilities, such as figuring out simple logical inferences
(while still being guided by a person who provides the overall proof).  Metamath
has no built-in automated proof capability other than the limited
capability of its Proof Assistant.

Proof-verification languages are not used as frequently as they might be.
Pure mathematicians are more concerned with producing new results, and such
detail and rigor would interfere with that goal.  The use of computers in pure
mathematics is primarily focused on automated theorem provers (not verifiers),
again with the ultimate goal of aiding the creation of new mathematics.
Automated theorem provers are usually concerned with attacking one theorem at
time rather than making a large, organized database easily available to the
user.  Metamath is one way to help close this gap.

By itself Metamath is a mostly a proof verifier.
This does not mean that other approaches can't be used; the difference
is that in Metamath, the results of various provers must be recorded
step-by-step so that they can be verified.

Another proof-verification language is Mizar,\index{Mizar} which can display
its proofs in the informal language that mathematicians are accustomed to.
Information on the Mizar language is available at \url{http://mizar.org}.

For the working mathematician, Mizar is an excellent tool for rigorously
documenting proofs. Mizar typesets its proofs in the informal English used by
mathematicians (and, while fine for them, are just as inscrutable by
laypersons!). A price paid for Mizar is a relatively steep learning curve of a
couple of weeks.  Several mathematicians are actively formalizing different
areas of mathematics using Mizar and publishing the proofs in a dedicated
journal. Unfortunately the task of formalizing mathematics is still looked
down upon to a certain extent since it doesn't involve the creation of ``new''
mathematics.

The closest system to Metamath is
the {\em Ghilbert}\index{Ghilbert} proof language (\url{http://ghilbert.org})
system developed by
Raph Levien\index{Levien, Raph}.
Ghilbert is a formal proof checker heavily inspired by Metamath.
Ghilbert statements are s-expressions (a la Lisp), which is easy
for computers to parse but many people find them hard to read.
There are a number of differences in their specific constructs, but
there is at least one tool to translate some Metamath materials into Ghilbert.
As of 2019 the Ghilbert community is smaller and less active than the
Metamath community.
That said, the Metamath and Ghilbert communities overlap, and fruitful
conversations between them have occurred many times over the years.

\subsection{Creating a Database of Formalized Mathematics}\label{mathdatabase}

Besides Metamath, there are several other ongoing projects with the goal of
formalizing mathematics into computer-verifiable databases.
Understanding some history will help.

The {\sc qed}\index{qed project@{\sc qed} project}%
\footnote{\url{http://www-unix.mcs.anl.gov/qed}.}
project arose in 1993 and its goals were outlined in the
{\sc qed} manifesto.
The {\sc qed} manifesto was
a proposal for a computer-based database of all mathematical knowledge,
strictly formalized and with all proofs having been checked automatically.
The project had a conference in 1994 and another in 1995;
there was also a ``twenty years of the {\sc qed} manifesto'' workshop
in 2014.
Its ideals are regularly reraised.

In a 2007 paper, Freek Wiedijk identified two reasons
for the failure of the {\sc qed} project as originally envisioned:%
\cite{Wiedijk-revisited}\index{Wiedijk, Freek}

\begin{itemize}
\item Very few people are working on formalization of mathematics. There is no compelling application for fully mechanized mathematics.
\item Formalized mathematics does not yet resemble traditional mathematics. This is partly due to the complexity of mathematical notation, and partly to the limitations of existing theorem provers and proof assistants.
\end{itemize}

But this did not end the dream of
formalizing mathematics into computer-verifiable databases.
The problems that led to the {\sc qed} manifesto are still with us,
even though the challenges were harder than originally considered.
What has happened instead is that various independent projects have
worked towards formalizing mathematics into computer-verifiable databases,
each simultaneously competing and cooperating with each other.

A concrete way to see this is
Freek Wiedijk's ``Formalizing 100 Theorems'' list%
\footnote{\url{http://www.cs.ru.nl/\%7Efreek/100/}.}
which shows the progress different systems have made on a challenge list
of 100 mathematical theorems.%
\footnote{ This is not the only list of ``interesting'' theorems.
Another interesting list was posted by Oliver Knill's list
\cite{Knill}\index{Knill, Oliver}.}
The top systems as of February 2019
(in order of the number of challenges completed) are
HOL Light, Isabelle, Metamath, Coq, and Mizar.

The Metamath 100%
\footnote{\url{http://us.metamath.org/mm\_100.html}}
page (maintained by David A. Wheeler\index{Wheeler, David A.})
shows the progress of Metamath (specifically its \texttt{set.mm} database)
against this challenge list maintained by Freek Wiedijk.
The Metamath \texttt{set.mm} database
has made a lot of progress over the years,
in part because working to prove those challenge theorems required
defining various terms and proving their properties as a prerequisite.
Here are just a few of the many statements that have been
formally proven with Metamath:

% The entries of this cause the narrow display to break poorly,
% since the short amount of text means LaTeX doesn't get a lot to work with
% and the itemize format gives it even *less* margin than usual.
% No one will mind if we make just this list flushleft, since this list
% will be internally consistent.
\begin{flushleft}
\begin{itemize}
\item 1. The Irrationality of the Square Root of 2
  (\texttt{sqr2irr}, by Norman Megill, 2001-08-20)
\item 2. The Fundamental Theorem of Algebra
  (\texttt{fta}, by Mario Carneiro, 2014-09-15)
\item 22. The Non-Denumerability of the Continuum
  (\texttt{ruc}, by Norman Megill, 2004-08-13)
\item 54. The Konigsberg Bridge Problem
  (\texttt{konigsberg}, by Mario Carneiro, 2015-04-16)
\item 83. The Friendship Theorem
  (\texttt{friendship}, by Alexander W. van der Vekens, 2018-10-09)
\end{itemize}
\end{flushleft}

We thank all of those who have developed at least one of the Metamath 100
proofs, and we particularly thank
Mario Carneiro\index{Carneiro, Mario}
who has contributed the most Metamath 100 proofs as of 2019.
The Metamath 100 page shows the list of all people who have contributed a
proof, and links to graphs and charts showing progress over time.
We encourage others to work on proving theorems not yet proven in Metamath,
since doing so improves the work as a whole.

Each of the math formalization systems (including Metamath)
has different strengths and weaknesses, depending on what you value.
Key aspects that differentiate Metamath from the other top systems are:

\begin{itemize}
\item Metamath is not tied to any particular set of axioms.
\item Metamath can show every step of every proof, no exceptions.
  Most other provers only assert that a proof can be found, and do not
  show every step. This also makes verification fast, because
  the system does not need to rediscover proof details.
\item The Metamath verifier has been re-implemented in many different
  programming languages, so verification can be done by multiple
  implementations.  In particular, the
  \texttt{set.mm}\index{set theory database (\texttt{set.mm})}%
  \index{Metamath Proof Explorer} database is verified by
  four different verifiers
  written in four different languages by four different authors.
  This greatly reduces the risk of accepting an invalid
  proof due to an error in the verifier.
\item Proofs stay proven.  In some systems, changes to the system's
  syntax or how a tactic works causes proofs to fail in later versions,
  causing older work to become essentially lost.
  Metamath's language is
  extremely small and fixed, so once a proof is added to a database,
  the database can be rechecked with later versions of the Metamath program
  and with other verifiers of Metamath databases.
  If an axiom or key definition needs to be changed, it is easy to
  manipulate the database as a whole to handle the change
  without touching the underlying verifier.
  Since re-verification of an entire database takes seconds, there
  is never a reason to delay complete verification.
  This aspect is especially compelling if your
  goal is to have a long-term database of proofs.
\item Licensing is generous.  The main Metamath databases are released to
  the public domain, and the main Metamath program is open source software
  under a standard, widely-used license.
\item Substitutions are easy to understand, even by those who are not
  professional mathematicians.
\end{itemize}

Of course, other systems may have advantages over Metamath
that are more compelling, depending on what you value.
In any case, we hope this helps you understand Metamath
within a wider context.

\subsection{In Summary}\label{computers-summary}

To summarize our discussions of computers and mathematics, computer algebra
systems can be viewed as theorem generators focusing on a narrow realm of
mathematics (numbers and their properties), automated theorem provers as proof
generators for specific theorems in a much broader realm covered by a built-in
formal system such as first-order logic, interactive theorem
provers require human guidance, proof verifiers verify proofs but
historically they have been
restricted to first-order logic.
Metamath, in contrast,
is a proof verifier and documenter whose realm is essentially unlimited.

\section{Mathematics and Metamath}

\subsection{Standard Mathematics}

There are a number of ways that Metamath\index{Metamath} can be used with
standard mathematics.  The most satisfying way philosophically is to start at
the very beginning, and develop the desired mathematics from the axioms of
logic and set theory.\index{set theory}  This is the approach taken in the
\texttt{set.mm}\index{set theory database (\texttt{set.mm})}%
\index{Metamath Proof Explorer}
database (also known as the Metamath Proof Explorer).
Among other things, this database builds up to the
axioms of real and complex numbers\index{analysis}\index{real and complex
numbers} (see Section~\ref{real}), and a standard development of analysis, for
example, could start at that point, using it as a basis.   Besides this
philosophical advantage, there are practical advantages to having all of the
tools of set theory available in the supporting infrastructure.

On the other hand, you may wish to start with the standard axioms of a
mathematical theory without going through the set theoretical proofs of those
axioms.  You will need mathematical logic to make inferences, but if you wish
you can simply introduce theorems\index{theorem} of logic as
``axioms''\index{axiom} wherever you need them, with the implicit assumption
that in principle they can be proved, if they are obvious to you.  If you
choose this approach, you will probably want to review the notation used in
\texttt{set.mm}\index{set theory database (\texttt{set.mm})} so that your own
notation will be consistent with it.

\subsection{Other Formal Systems}
\index{formal system}

Unlike some programs, Metamath\index{Metamath} is not limited to any specific
area of mathematics, nor committed to any particular mathematical philosophy
such as classical logic versus intuitionism, nor limited, say, to expressions
in first-order logic.  Although the database \texttt{set.mm}
describes standard logic and set theory, Meta\-math
is actually a general-purpose language for describing a wide variety of formal
systems.\index{formal system}  Non-standard systems such as modal
logic,\index{modal logic} intuitionist logic\index{intuitionism}, higher-order
logic\index{higher-order logic}, quantum logic\index{quantum logic}, and
category theory\index{category theory} can all be described with the Metamath
language.  You define the symbols you prefer and tell Metamath the axioms and
rules you want to start from, and Metamath will verify any inferences you make
from those axioms and rules.  A simple example of a non-standard formal system
is Hofstadter's\index{Hofstadter, Douglas R.} MIU system,\index{MIU-system}
whose Metamath description is presented in Appendix~\ref{MIU}.

This is not hypothetical.
The largest Metamath database is
\texttt{set.mm}\index{set theory database (\texttt{set.mm}}%
\index{Metamath Proof Explorer}), aka the Metamath Proof Explorer,
which uses the most common axioms for mathematical foundations
(specifically classical logic combined with Zermelo--Fraenkel
set theory\index{Zermelo--Fraenkel set theory} with the Axiom of Choice).
But other Metamath databases are available:

\begin{itemize}
\item The database
  \texttt{iset.mm}\index{intuitionistic logic database (\texttt{iset.mm})},
  aka the
  Intuitionistic Logic Explorer\index{Intuitionistic Logic Explorer},
  uses intuitionistic logic (a constructivist point of view)
  instead of classical logic.
\item The database
  \texttt{nf.mm}\index{New Foundations database (\texttt{nf.mm})},
  aka the
  New Foundations Explorer\index{New Foundations Explorer},
  constructs mathematics from scratch,
  starting from Quine's New Foundations (NF) set theory axioms.
\item The database
  \texttt{hol.mm}\index{Higher-order Logic database (\texttt{hol.mm})},
  aka the
  Higher-Order Logic (HOL) Explorer\index{Higher-Order Logic (HOL) Explorer},
  starts with HOL (also called simple type theory) and derives
  equivalents to ZFC axioms, connecting the two approaches.
\end{itemize}

Since the days of David Hilbert,\index{Hilbert, David} mathematicians have
been concerned with the fact that the metalanguage\index{metalanguage} used to
describe mathematics may be stronger than the mathematics being described.
Metamath\index{Metamath}'s underlying finitary\index{finitary proof},
constructive nature provides a good philosophical basis for studying even the
weakest logics.\index{weak logic}

The usual treatment of many non-standard formal systems\index{formal
system} uses model theory\index{model theory} or proof theory\index{proof
theory} to describe these systems; these theories, in turn, are based on
standard set theory.  In other words, a non-standard formal system is defined
as a set with certain properties, and standard set theory is used to derive
additional properties of this set.  The standard set theory database provided
with Metamath can be used for this purpose, and when used this way
the development of a special
axiom system for the non-standard formal system becomes unnecessary.  The
model- or proof-theoretic approach often allows you to prove much deeper
results with less effort.

Metamath supports both approaches.  You can define the non-standard
formal system directly, or define the non-standard formal system as
a set with certain properties, whichever you find most helpful.

%\section{Additional Remarks}

\subsection{Metamath and Its Philosophy}

Closely related to Metamath\index{Metamath} is a philosophy or way of looking
at mathematics. This philosophy is related to the formalist
philosophy\index{formalism} of Hilbert\index{Hilbert, David} and his followers
\cite[pp.~1203--1208]{Kline}\index{Kline, Morris}
\cite[p.~6]{Behnke}\index{Behnke, H.}. In this philosophy, mathematics is
viewed as nothing more than a set of rules that manipulate symbols, together
with the consequences of those rules.  While the mathematics being described
may be complex, the rules used to describe it (the
``metamathematics''\index{metamathematics}) should be as simple as possible.
In particular, proofs should be restricted to dealing with concrete objects
(the symbols we write on paper rather than the abstract concepts they
represent) in a constructive manner; these are called ``finitary''
proofs\index{finitary proof} \cite[pp.~2--3]{Shoenfield}\index{Shoenfield,
Joseph R.}.

Whether or not you find Metamath interesting or useful will in part depend on
the appeal you find in its philosophy, and this appeal will probably depend on
your particular goals with respect to mathematics.  For example, if you are a
pure mathematician at the forefront of discovering new mathematical knowledge,
you will probably find that the rigid formality of Metamath stifles your
creativity.  On the other hand, we would argue that once this knowledge is
discovered, there are advantages to documenting it in a standard format that
will make it accessible to others.  Sixty years from now, your field may be
dormant, and as Davis and Hersh put it, your ``writings would become less
translatable than those of the Maya'' \cite[p.~37]{Davis}\index{Davis, Phillip
J.}.


\subsection{A History of the Approach Behind Metamath}

Probably the one work that has had the most motivating influence on
Metamath\index{Metamath} is Whitehead and Russell's monumental {\em Principia
Mathematica} \cite{PM}\index{Whitehead, Alfred North}\index{Russell,
Bertrand}\index{principia mathematica@{\em Principia Mathematica}}, whose aim
was to deduce all of mathematics from a small number of primitive ideas, in a
very explicit way that in principle anyone could understand and follow.  While
this work was tremendously influential in its time, from a modern perspective
it suffers from several drawbacks.  Both its notation and its underlying
axioms are now considered dated and are no longer used.  From our point of
view, its development is not really as accessible as we would like to see; for
practical reasons, proofs become more and more sketchy as its mathematics
progresses, and working them out in fine detail requires a degree of
mathematical skill and patience that many people don't have.  There are
numerous small errors, which is understandable given the tedious, technical
nature of the proofs and the lack of a computer to verify the details.
However, even today {\em Principia Mathematica} stands out as the work closest
in spirit to Metamath.  It remains a mind-boggling work, and one can't help
but be amazed at seeing ``$1+1=2$'' finally appear on page 83 of Volume II
(Theorem *110.643).

The origin of the proof notation used by Metamath dates back to the 1950's,
when the logician C.~A.~Meredith expressed his proofs in a compact notation
called ``condensed detachment''\index{condensed detachment}
\cite{Hindley}\index{Hindley, J. Roger} \cite{Kalman}\index{Kalman, J. A.}
\cite{Meredith}\index{Meredith, C. A.} \cite{Peterson}\index{Peterson, Jeremy
George}.  This notation allows proofs to be communicated unambiguously by
merely referencing the axiom\index{axiom}, rule\index{rule}, or
theorem\index{theorem} used at each step, without explicitly indicating the
substitutions\index{substitution!variable}\index{variable substitution} that
have to be made to the variables in that axiom, rule, or theorem.  Ordinarily,
condensed detachment is more or less limited to propositional
calculus\index{propositional calculus}.  The concept has been extended to
first-order logic\index{first-order logic} in \cite{Megill}\index{Megill,
Norman}, making it is easy to write a small computer program to verify proofs
of simple first-order logic theorems.\index{condensed detachment!and
first-order logic}

A key concept behind the notation of condensed detachment is called
``unification,''\index{unification} which is an algorithm for determining what
substitutions\index{substitution!variable}\index{variable substitution} to
variables have to be made to make two expressions match each other.
Unification was first precisely defined by the logician J.~A.~Robinson, who
used it in the development of a powerful
theorem-proving technique called the ``resolution principle''
\cite{Robinson}\index{Robinson's resolution principle}. Metamath does not make
use of the resolution principle, which is intended for systems of first-order
logic.\index{first-order logic}  Metamath's use is not restricted to
first-order logic, and as we have mentioned it does not automatically discover
proofs.  However, unification is a key idea behind Metamath's proof
notation, and Metamath makes use of a very simple version of it
(Section~\ref{unify}).

\subsection{Metamath and First-Order Logic}

First-order logic\index{first-order logic} is the supporting structure
for standard mathematics.  On top of it is set theory, which contains
the axioms from which virtually all of mathematics can be derived---a
remarkable fact.\index{category
theory}\index{cardinal, inaccessible}\label{categoryth}\footnote{An exception seems
to be category theory.  There are several schools of thought on whether
category theory is derivable from set theory.  At a minimum, it appears
that an additional axiom is needed that asserts the existence of an
``inaccessible cardinal'' (a type of infinity so large that standard set
theory can't prove or deny that it exists).
%
%%%% (I took this out that was in previous editions:)
% But it is also argued that not just one but a ``proper class'' of them
% is needed, and the existence of proper classes is impossible in standard
% set theory.  (A proper class is a collection of sets so huge that no set
% can contain it as an element.  Proper classes can lead to
% inconsistencies such as ``Russell's paradox.''  The axioms of standard
% set theory are devised so as to deny the existence of proper classes.)
%
For more information, see
\cite[pp.~328--331]{Herrlich}\index{Herrlich, Horst} and
\cite{Blass}\index{Blass, Andrea}.}

One of the things that makes Metamath\index{Metamath} more practical for
first-order theories is a set of axioms for first-order logic designed
specifically with Metamath's approach in mind.  These are included in
the database \texttt{set.mm}\index{set theory database (\texttt{set.mm})}.
See Chapter~\ref{fol} for a detailed
description; the axioms are shown in Section~\ref{metaaxioms}.  While
logically equivalent to standard axiom systems, our axiom system breaks
up the standard axioms into smaller pieces such that from them, you can
directly derive what in other systems can only be derived as higher-level
``metatheorems.''\index{metatheorem}  In other words, it is more powerful than
the standard axioms from a metalogical point of view.  A rigorous
justification for this system and its ``metalogical
completeness''\index{metalogical completeness} is found in
\cite{Megill}\index{Megill, Norman}.  The system is closely related to a
system developed by Monk\index{Monk, J. Donald} and Tarski\index{Tarski,
Alfred} in 1965 \cite{Monks}.

For example, the formula $\exists x \, x = y $ (given $y$, there exists some
$x$ equal to it) is a theorem of logic,\footnote{Specifically, it is a theorem
of those systems of logic that assume non-empty domains.  It is not a theorem
of more general systems that include the empty domain\index{empty domain}, in
which nothing exists, period!  Such systems are called ``free
logics.''\index{free logic} For a discussion of these systems, see
\cite{Leblanc}\index{Leblanc, Hugues}.  Since our use for logic is as a basis
for set theory, which has a non-empty domain, it is more convenient (and more
traditional) to use a less general system.  An interesting curiosity is that,
using a free logic as a basis for Zermelo--Fraenkel set
theory\index{Zermelo--Fraenkel set theory} (with the redundant Axiom of the
Null Set omitted),\index{Axiom of the Null Set} we cannot even prove the
existence of a single set without assuming the axiom of infinity!\index{Axiom
of Infinity}} whether or not $x$ and $y$ are distinct variables\index{distinct
variables}.  In many systems of logic, we would have to prove two theorems to
arrive at this result.  First we would prove ``$\exists x \, x = x $,'' then
we would separately prove ``$\exists x \, x = y $, where $x$ and $y$ are
distinct variables.''  We would then combine these two special cases ``outside
of the system'' (i.e.\ in our heads) to be able to claim, ``$\exists x \, x =
y $, regardless of whether $x$ and $y$ are distinct.''  In other words, the
combination of the two special cases is a
metatheorem.  In the system of logic
used in Metamath's set theory\index{set theory database (\texttt{set.mm})}
database, the axioms of logic are broken down into small pieces that allow
them to be reassembled in such a way that theorems such as these can be proved
directly.

Breaking down the axioms in this way makes them look peculiar and not very
intuitive at first, but rest assured that they are correct and complete.  Their
correctness is ensured because they are theorem schemes of standard first-order
logic (which you can easily verify if you are a logician).  Their completeness
follows from the fact that we explicitly derive the standard axioms of
first-order logic as theorems.  Deriving the standard axioms is somewhat
tricky, but once we're there, we have at our disposal a system that is less
awkward to work with in formal proofs\index{formal proof}.  In technical terms
that logicians understand, we eliminate the cumbersome concepts of ``free
variable,''\index{free variable} ``bound variable,''\index{bound variable} and
``proper substitution''\index{proper substitution}\index{substitution!proper}
as primitive notions.  These concepts are present in our system but are
defined in terms of concepts expressed by the axioms and can be eliminated in
principle.  In standard systems, these concepts are really like additional,
implicit axioms\index{implicit axiom} that are somewhat complex and cannot be
eliminated.

The traditional approach to logic, wherein free variables and proper
substitution is defined, is also possible to do directly in the Metamath
language.  However, the notation tends to become awkward, and there are
disadvantages:  for example, extending the definition of a wff with a
definition is awkward, because the free variable and proper substitution
concepts have to have their definitions also extended.  Our choice of
axioms for \texttt{set.mm} is to a certain extent a matter of style, in
an attempt to achieve overall simplicity, but you should also be aware
that the traditional approach is possible as well if you should choose
it.

\chapter{Using the Metamath Program}
\label{using}

\section{Installation}

The way that you install Metamath\index{Metamath!installation} on your
computer system will vary for different computers.  Current
instructions are provided with the Metamath program download at
\url{http://metamath.org}.  In general, the installation is simple.
There is one file containing the Metamath program itself.  This file is
usually called \texttt{metamath} or \texttt{metamath.}{\em xxx} where
{\em xxx} is the convention (such as \texttt{exe}) for an executable
program on your operating system.  There are several additional files
containing samples of the Metamath language, all ending with
\texttt{.mm}.  The file \texttt{set.mm}\index{set theory database
(\texttt{set.mm})} contains logic and set theory and can be used as a
starting point for other areas of mathematics.

You will also need a text editor\index{text editor} capable of editing plain
{\sc ascii}\footnote{American Standard Code for Information Interchange.} text
in order to prepare your input files.\index{ascii@{\sc ascii}}  Most computers
have this capability built in.  Note that plain text is not necessarily the
default for some word processing programs\index{word processor}, especially if
they can handle different fonts; for example, with Microsoft Word\index{Word
(Microsoft)}, you must save the file in the format ``Text Only With Line
Breaks'' to get a plain text\index{plain text} file.\footnote{It is
recommended that all lines in a Metamath source file be 79 characters or less
in length for compatibility among different computer terminals.  When creating
a source file on an editor such as Word, select a monospaced
font\index{monospaced font} such as Courier\index{Courier font} or
Monaco\index{Monaco font} to make this easier to achieve.  Better yet,
just use a plain text editor such as Notepad.}

On some computer systems, Metamath does not have the capability to print
its output directly; instead, you send its output to a file (using the
\texttt{open} commands described later).  The way you print this output
file depends on your computer.\index{printers} Some computers have a
print command, whereas with others, you may have to read the file into
an editor and print it from there.

If you want to print your Metamath source files with typeset formulas
containing standard mathematical symbols, you will need the \LaTeX\
typesetting program\index{latex@{\LaTeX}}, which is widely and freely
available for most operating systems.  It runs natively on Unix and
Linux, and can be installed on Windows as part of the free Cygwin
package (\url{http://cygwin.com}).

You can also produce {\sc html}\footnote{HyperText Markup Language.}
web pages.  The {\tt help html} command in the Metamath program will
assist you with this feature.

\section{Your First Formal System}\label{start}
\subsection{From Nothing to Zero}\label{startf}

To give you a feel for what the Metamath\index{Metamath} language looks like,
we will take a look at a very simple example from formal number
theory\index{number theory}.  This example is taken from
Mendelson\index{Mendelson, Elliot} \cite[p. 123]{Mendelson}.\footnote{To keep
the example simple, we have changed the formalism slightly, and what we call
axioms\index{axiom} are strictly speaking theorems\index{theorem} in
\cite{Mendelson}.}  We will look at a small subset of this theory, namely that
part needed for the first number theory theorem proved in \cite{Mendelson}.

First we will look at a standard formal proof\index{formal proof} for the
example we have picked, then we will look at the Metamath version.  If you
have never been exposed to formal proofs, the notation may seem to be such
overkill to express such simple notions that you may wonder if you are missing
something.  You aren't.  The concepts involved are in fact very simple, and a
detailed breakdown in this fashion is necessary to express the proof in a way
that can be verified mechanically.  And as you will see, Metamath breaks the
proof down into even finer pieces so that the mechanical verification process
can be about as simple as possible.

Before we can introduce the axioms\index{axiom} of the theory, we must define
the syntax rules for forming legal expressions\index{syntax rules}
(combinations of symbols) with which those axioms can be used. The number 0 is
a {\bf term}\index{term}; and if $ t$ and $r$ are terms, so is $(t+r)$. Here,
$ t$ and $r$ are ``metavariables''\index{metavariable} ranging over terms; they
themselves do not appear as symbols in an actual term.  Some examples of
actual terms are $(0 + 0)$ and $((0+0)+0)$.  (Note that our theory describes
only the number zero and sums of zeroes.  Of course, not much can be done with
such a trivial theory, but remember that we have picked a very small subset of
complete number theory for our example.  The important thing for you to focus
on is our definitions that describe how symbols are combined to form valid
expressions, and not on the content or meaning of those expressions.) If $ t$
and $r$ are terms, an expression of the form $ t=r$ is a {\bf wff}
(well-formed formula)\index{well-formed formula (wff)}; and if $P$ and $Q$ are
wffs, so is $(P\rightarrow Q)$ (which means ``$P$ implies
$Q$''\index{implication ($\rightarrow$)} or ``if $P$ then $Q$'').
Here $P$ and $Q$ are metavariables ranging over wffs.  Examples of actual
wffs are $0=0$, $(0+0)=0$, $(0=0 \rightarrow (0+0)=0)$, and $(0=0\rightarrow
(0=0\rightarrow 0=(0+0)))$.  (Our notation makes use of more parentheses than
are customary, but the elimination of ambiguity this way simplifies our
example by avoiding the need to define operator precedence\index{operator
precedence}.)

The {\bf axioms}\index{axiom} of our theory are all wffs of the following
form, where $ t$, $r$, and $s$ are any terms:

%Latex p. 92
\renewcommand{\theequation}{A\arabic{equation}}

\begin{equation}
(t=r\rightarrow (t=s\rightarrow r=s))
\end{equation}
\begin{equation}
(t+0)=t
\end{equation}

Note that there are an infinite number of axioms since there are an infinite
number of possible terms.  A1 and A2 are properly called ``axiom
schemes,''\index{axiom scheme} but we will refer to them as ``axioms'' for
brevity.

An axiom is a {\bf theorem}; and if $P$ and $(P\rightarrow Q)$ are theorems
(where $P$ and $Q$ are wffs), then $Q$ is also a theorem.\index{theorem}  The
second part of this definition is called the modus ponens (MP) rule of
inference\index{inference rule}\index{modus ponens}.  It allows us to obtain
new theorems from old ones.

The {\bf proof}\index{proof} of a theorem is a sequence of one or more
theorems, each of which is either an axiom or the result of modus ponens
applied to two previous theorems in the sequence, and the last of which is the
theorem being proved.

The theorem we will prove for our example is very simple:  $ t=t$.  The proof of
our theorem follows.  Study it carefully until you feel sure you
understand it.\label{zeroproof}

% Use tabu so that lines will wrap automatically as needed.
\begin{tabu} { l X X }
1. & $(t+0)=t$ & (by axiom A2) \\
2. & $(t+0)=t$ & (by axiom A2) \\
3. & $((t+0)=t \rightarrow ((t+0)=t\rightarrow t=t))$ & (by axiom A1) \\
4. & $((t+0)=t\rightarrow t=t)$ & (by MP applied to steps 2 and 3) \\
5. & $t=t$ & (by MP applied to steps 1 and 4) \\
\end{tabu}

(You may wonder why step 1 is repeated twice.  This is not necessary in the
formal language we have defined, but in Metamath's ``reverse Polish
notation''\index{reverse Polish notation (RPN)} for proofs, a previous step
can be referred to only once.  The repetition of step~1 here will enable you
to see more clearly the correspondence of this proof with the
Metamath\index{Metamath} version on p.~\pageref{demoproof}.)

Our theorem is more properly called a ``theorem scheme,''\index{theorem
scheme} for it represents an infinite number of theorems, one for each
possible term $ t$.  Two examples of actual theorems would be $0=0$ and
$(0+0)=(0+0)$.  Rarely do we prove actual theorems, since by proving schemes
we can prove an infinite number of theorems in one fell swoop.  Similarly, our
proof should really be called a ``proof scheme.''\index{proof scheme}  To
obtain an actual proof, pick an actual term to use in place of $ t$, and
substitute it for $ t$ throughout the proof.

Let's discuss what we have done here.  The axioms\index{axiom} of our theory,
A1 and A2, are trivial and obvious.  Everyone knows that adding zero to
something doesn't change it, and also that if two things are equal to a third,
then they are equal to each other. In fact, stating the trivial and obvious is
a goal to strive for in any axiomatic system.  From trivial and obvious truths
that everyone agrees upon, we can prove results that are not so obvious yet
have absolute faith in them.  If we trust the axioms and the rules, we must,
by definition, trust the consequences of those axioms and rules, if logic is
to mean anything at all.

Our rule of inference\index{rule}, modus ponens\index{modus ponens}, is also
pretty obvious once you understand what it means.  If we prove a fact $P$, and
we also prove that $P$ implies $Q$, then $Q$ necessarily follows as a new
fact.  The rule provides us with a means for obtaining new facts (i.e.\
theorems\index{theorem}) from old ones.

The theorem that we have proved, $ t=t$, is so fundamental that you may wonder
why it isn't one of the axioms\index{axiom}.  In some axiom systems of
arithmetic, it {\em is} an axiom.  The choice of axioms in a theory is to some
extent arbitrary and even an art form, constrained only by the requirement
that any two equivalent axiom systems be able to derive each other as
theorems.  We could imagine that the inventor of our axiom system originally
included $ t=t$ as an axiom, then discovered that it could be derived as a
theorem from the other axioms.  Because of this, it was not necessary to
keep it as an axiom.  By eliminating it, the final set of axioms became
that much simpler.

Unless you have worked with formal proofs\index{formal proof} before, it
probably wasn't apparent to you that $ t=t$ could be derived from our two
axioms until you saw the proof. While you certainly believe that $ t=t$ is
true, you might not be able to convince an imaginary skeptic who believes only
in our two axioms until you produce the proof.  Formal proofs such as this are
hard to come up with when you first start working with them, but after you get
used to them they can become interesting and fun.  Once you understand the
idea behind formal proofs you will have grasped the fundamental principle that
underlies all of mathematics.  As the mathematics becomes more sophisticated,
its proofs become more challenging, but ultimately they all can be broken down
into individual steps as simple as the ones in our proof above.

Mendelson's\index{Mendelson, Elliot} book, from which our example was taken,
contains a number of detailed formal proofs such as these, and you may be
interested in looking it up.  The book is intended for mathematicians,
however, and most of it is rather advanced.  Popular literature describing
formal proofs\index{formal proof} include \cite[p.~296]{Rucker}\index{Rucker,
Rudy} and \cite[pp.~204--230]{Hofstadter}\index{Hofstadter, Douglas R.}.

\subsection{Converting It to Metamath}\label{convert}

Formal proofs\index{formal proof} such as the one in our example break down
logical reasoning into small, precise steps that leave little doubt that the
results follow from the axioms\index{axiom}.  You might think that this would
be the finest breakdown we can achieve in mathematics.  However, there is more
to the proof than meets the eye. Although our axioms were rather simple, a lot
of verbiage was needed before we could even state them:  we needed to define
``term,'' ``wff,'' and so on.  In addition, there are a number of implied
rules that we haven't even mentioned. For example, how do we know that step 3
of our proof follows from axiom A1? There is some hidden reasoning involved in
determining this.  Axiom A1 has two occurrences of the letter $ t$.  One of
the implied rules states that whatever we substitute for $ t$ must be a legal
term\index{term}.\footnote{Some authors make this implied rule explicit by
stating, ``only expressions of the above form are terms,'' after defining
``term.''}  The expression $ t+0$ is pretty obviously a legal term whenever $
t$ is, but suppose we wanted to substitute a huge term with thousands of
symbols?  Certainly a lot of work would be involved in determining that it
really is a term, but in ordinary formal proofs all of this work would be
considered a single ``step.''

To express our axiom system in the Metamath\index{Metamath} language, we must
describe this auxiliary information in addition to the axioms themselves.
Metamath does not know what a ``term'' or a ``wff''\index{well-formed formula
(wff)} is.  In Metamath, the specification of the ways in which we can combine
symbols to obtain terms and wffs are like little axioms in themselves.  These
auxiliary axioms are expressed in the same notation as the ``real''
axioms\index{axiom}, and Metamath does not distinguish between the two.  The
distinction is made by you, i.e.\ by the way in which you interpret the
notation you have chosen to express these two kinds of axioms.

The Metamath language breaks down mathematical proofs into tiny pieces, much
more so than in ordinary formal proofs\index{formal proof}.  If a single
step\index{proof step} involves the
substitution\index{substitution!variable}\index{variable substitution} of a
complex term for one of its variables, Metamath must see this single step
broken down into many small steps.  This fine-grained breakdown is what gives
Metamath generality and flexibility as it lets it not be limited to any
particular mathematical notation.

Metamath's proof notation is not, in itself, intended to be read by humans but
rather is in a compact format intended for a machine.  The Metamath program
will convert this notation to a form you can understand, using the \texttt{show
proof}\index{\texttt{show proof} command} command.  You can tell the program what
level of detail of the proof you want to look at.  You may want to look at
just the logical inference steps that correspond
to ordinary formal proof steps,
or you may want to see the fine-grained steps that prove that an expression is
a term.

Here, without further ado, is our example converted to the
Metamath\index{Metamath} language:\index{metavariable}\label{demo0}

\begin{verbatim}
$( Declare the constant symbols we will use $)
    $c 0 + = -> ( ) term wff |- $.
$( Declare the metavariables we will use $)
    $v t r s P Q $.
$( Specify properties of the metavariables $)
    tt $f term t $.
    tr $f term r $.
    ts $f term s $.
    wp $f wff P $.
    wq $f wff Q $.
$( Define "term" and "wff" $)
    tze $a term 0 $.
    tpl $a term ( t + r ) $.
    weq $a wff t = r $.
    wim $a wff ( P -> Q ) $.
$( State the axioms $)
    a1 $a |- ( t = r -> ( t = s -> r = s ) ) $.
    a2 $a |- ( t + 0 ) = t $.
$( Define the modus ponens inference rule $)
    ${
       min $e |- P $.
       maj $e |- ( P -> Q ) $.
       mp  $a |- Q $.
    $}
$( Prove a theorem $)
    th1 $p |- t = t $=
  $( Here is its proof: $)
       tt tze tpl tt weq tt tt weq tt a2 tt tze tpl
       tt weq tt tze tpl tt weq tt tt weq wim tt a2
       tt tze tpl tt tt a1 mp mp
     $.
\end{verbatim}\index{metavariable}

A ``database''\index{database} is a set of one or more {\sc ascii} source
files.  Here's a brief description of this Metamath\index{Metamath} database
(which consists of this single source file), so that you can understand in
general terms what is going on.  To understand the source file in detail, you
should read Chapter~\ref{languagespec}.

The database is a sequence of ``tokens,''\index{token} which are normally
separated by spaces or line breaks.  The only tokens that are built into
the Metamath language are those beginning with \texttt{\$}.  These tokens
are called ``keywords.''\index{keyword}  All other tokens are
user-defined, and their names are arbitrary.

As you might have guessed, the Metamath token \texttt{\$(}\index{\texttt{\$(} and
\texttt{\$)} auxiliary keywords} starts a comment and \texttt{\$)} ends a comment.

The Metamath tokens \texttt{\$c}\index{\texttt{\$c} statement},
\texttt{\$v}\index{\texttt{\$v} statement},
\texttt{\$e}\index{\texttt{\$e} statement},
\texttt{\$f}\index{\texttt{\$f} statement},
\texttt{\$a}\index{\texttt{\$a} statement}, and
\texttt{\$p}\index{\texttt{\$p} statement} specify ``statements'' that
end with \texttt{\$.}\,.\index{\texttt{\$.}\ keyword}

The Metamath tokens \texttt{\$c} and \texttt{\$v} each declare\index{constant
declaration}\index{variable declaration} a list of user-defined tokens, called
``math symbols,''\index{math symbol} that the database will reference later
on.  All of the math symbols they define you have seen earlier except the
turnstile symbol \texttt{|-} ($\vdash$)\index{turnstile ({$\,\vdash$})}, which is
commonly used by logicians to mean ``a proof exists for.''  For us
the turnstile is just a
convenient symbol that distinguishes expressions that are axioms\index{axiom}
or theorems\index{theorem} from expressions that are terms or wffs.

The \texttt{\$c} statement declares ``constants''\index{constant} and
the \texttt{\$v} statement declares
``variables''\index{variable}\index{constant declaration}\index{variable
declaration} (or more precisely, metavariables\index{metavariable}).  A
variable may be substituted\index{substitution!variable}\index{variable
substitution} with sequences of math symbols whereas a constant may not
be substituted with anything.

It may seem redundant to require both \texttt{\$c}\index{\texttt{\$c} statement} and
\texttt{\$v}\index{\texttt{\$v} statement} statements (since any math
symbol\index{math symbol} not specified with a \texttt{\$c} statement could be
presumed to be a variable), but this provides for better error checking and
also allows math symbols to be redeclared\index{redeclaration of symbols}
(Section~\ref{scoping}).

The token \texttt{\$f}\index{\texttt{\$f} statement} specifies a
statement called a ``variable-type hypothesis'' (also called a
``floating hypothesis'') and \texttt{\$e}\index{\texttt{\$e} statement}
specifies a ``logical hypothesis'' (also called an ``essential
hypothesis'').\index{hypothesis}\index{variable-type
hypothesis}\index{logical hypothesis}\index{floating
hypothesis}\index{essential hypothesis} The token
\texttt{\$a}\index{\texttt{\$a} statement} specifies an ``axiomatic
assertion,''\index{axiomatic assertion} and
\texttt{\$p}\index{\texttt{\$p} statement} specifies a ``provable
assertion.''\index{provable assertion} To the left of each occurrence of
these four tokens is a ``label''\index{label} that identifies the
hypothesis or assertion for later reference.  For example, the label of
the first axiomatic assertion is \texttt{tze}.  A \texttt{\$f} statement
must contain exactly two math symbols, a constant followed by a
variable.  The \texttt{\$e}, \texttt{\$a}, and \texttt{\$p} statements
each start with a constant followed by, in general, an arbitrary
sequence of math symbols.

Associated with each assertion\index{assertion} is a set of hypotheses
that must be satisfied in order for the assertion to be used in a proof.
These are called the ``mandatory hypotheses''\index{mandatory
hypothesis} of the assertion.  Among those hypotheses whose ``scope''
(described below) includes the assertion, \texttt{\$e} hypotheses are
always mandatory and \texttt{\$f}\index{\texttt{\$f} statement}
hypotheses are mandatory when they share their variable with the
assertion or its \texttt{\$e} hypotheses.  The exact rules for
determining which hypotheses are mandatory are described in detail in
Sections~\ref{frames} and \ref{scoping}.  For example, the mandatory
hypotheses of assertion \texttt{tpl} are \texttt{tt} and \texttt{tr},
whereas assertion \texttt{tze} has no mandatory hypotheses because it
contains no variables and has no \texttt{\$e}\index{\texttt{\$e}
statement} hypothesis.  Metamath's \texttt{show statement}
command\index{\texttt{show statement} command}, described in the next
section, will show you a statement's mandatory hypotheses.

Sometimes we need to make a hypothesis relevant to only certain
assertions.  The set of statements to which a hypothesis is relevant is
called its ``scope.''  The Metamath brackets,
\texttt{\$\char`\{}\index{\texttt{\$\char`\{} and \texttt{\$\char`\}}
keywords} and \texttt{\$\char`\}}, define a ``block''\index{block} that
delimits the scope of any hypothesis contained between them.  The
assertion \texttt{mp} has mandatory hypotheses \texttt{wp}, \texttt{wq},
\texttt{min}, and \texttt{maj}.  The only mandatory hypothesis of
\texttt{th1}, on the other hand, is \texttt{tt}, since \texttt{th1}
occurs outside of the block containing \texttt{min} and \texttt{maj}.

Note that \texttt{\$\char`\{} and \texttt{\$\char`\}} do not affect the
scope of assertions (\texttt{\$a} and \texttt{\$p}).  Assertions are always
available to be referenced by any later proof in the source file.

Each provable assertion (\texttt{\$p}\index{\texttt{\$p} statement}
statement) has two parts.  The first part is the
assertion\index{assertion} itself, which is a sequence of math
symbol\index{math symbol} tokens placed between the \texttt{\$p} token
and a \texttt{\$=}\index{\texttt{\$=} keyword} token.  The second part
is a ``proof,'' which is a list of label tokens placed between the
\texttt{\$=} token and the \texttt{\$.}\index{\texttt{\$.}\ keyword}\
token that ends the statement.\footnote{If you've looked at the
\texttt{set.mm} database, you may have noticed another notation used for
proofs.  The other notation is called ``compressed.''\index{compressed
proof}\index{proof!compressed} It reduces the amount of space needed to
store a proof in the database and is described in
Appendix~\ref{compressed}.  In the example above, we use
``normal''\index{normal proof}\index{proof!normal} notation.} The proof
acts as a series of instructions to the Metamath program, telling it how
to build up the sequence of math symbols contained in the assertion part of
the \texttt{\$p} statement, making use of the hypotheses of the
\texttt{\$p} statement and previous assertions.  The construction takes
place according to precise rules.  If the list of labels in the proof
causes these rules to be violated, or if the final sequence that results
does not match the assertion, the Metamath program will notify you with
an error message.

If you are familiar with reverse Polish notation (RPN), which is sometimes used
on pocket calculators, here in a nutshell is how a proof works.  Each
hypothesis label\index{hypothesis label} in the proof is pushed\index{push}
onto the RPN stack\index{stack}\index{RPN stack} as it is encountered. Each
assertion label\index{assertion label} pops\index{pop} off the stack as many
entries as the referenced assertion has mandatory hypotheses.  Variable
substitutions\index{substitution!variable}\index{variable substitution} are
computed which, when made to the referenced assertion's mandatory hypotheses,
cause these hypotheses to match the stack entries. These same substitutions
are then made to the variables in the referenced assertion itself, which is
then pushed onto the stack.  At the end of the proof, there should be one
stack entry, namely the assertion being proved.  This process is explained in
detail in Section~\ref{proof}.

Metamath's proof notation is not very readable for humans, but it allows the
proof to be stored compactly in a file.  The Metamath\index{Metamath} program
has proof display features that let you see what's going on in a more
readable way, as you will see in the next section.

The rules used in verifying a proof are not based on any built-in syntax of the
symbol sequence in an assertion\index{assertion} nor on any built-in meanings
attached to specific symbol names.  They are based strictly on symbol
matching:  constants\index{constant} must match themselves, and
variables\index{variable} may be replaced with anything that allows a match to
occur.  For example, instead of \texttt{term}, \texttt{0}, and \verb$|-$ we could
have just as well used \texttt{yellow}, \texttt{zero}, and \texttt{provable}, as long
as we did so consistently throughout the database.  Also, we could have used
\texttt{is provable} (two tokens) instead of \verb$|-$ (one token) throughout the
database.  In each of these cases, the proof would be exactly the same.  The
independence of proofs and notation means that you have a lot of flexibility to
change the notation you use without having to change any proofs.

\section{A Trial Run}\label{trialrun}

Now you are ready to try out the Metamath\index{Metamath} program.

On all computer systems, Metamath has a standard ``command line
interface'' (CLI)\index{command line interface (CLI)} that allows you to
interact with it.  You supply commands to the CLI by typing them on the
keyboard and pressing your keyboard's {\em return} key after each line
you enter.  The CLI is designed to be easy to use and has built-in help
features.

The first thing you should do is to use a text editor to create a file
called \texttt{demo0.mm} and type into it the Metamath source shown on
p.~\pageref{demo0}.  Actually, this file is included with your Metamath
software package, so check that first.  If you type it in, make sure
that you save it in the form of ``plain {\sc ascii} text with line
breaks.''  Most word processors will have this feature.

Next you must run the Metamath program.  Depending on your computer
system and how Metamath is installed, this could range from clicking the
mouse on the Metamath icon to typing \texttt{run metamath} to typing
simply \texttt{metamath}.  (Metamath's {\tt help invoke} command describes
alternate ways of invoking the Metamath program.)

When you first enter Metamath\index{Metamath}, it will be at the CLI, waiting
for your input. You will see something like the following on your screen:
\begin{verbatim}
Metamath - Version 0.172 25-Jan-2019
Type HELP for help, EXIT to exit.
MM>
\end{verbatim}
The \texttt{MM>} prompt means that Metamath is waiting for a command.
Command keywords\index{command keyword} are not case sensitive;
we will use lower-case commands in our examples.
The version number and its release date will probably be different on your
system from the one we show above.

The first thing that you need to do is to read in your
database:\index{\texttt{read} command}\footnote{If a directory path is
needed on Unix,\index{Unix file names}\index{file names!Unix} you should
enclose the path/file name in quotes to prevent Metamath from thinking
that the \texttt{/} in the path name is a command qualifier, e.g.,
\texttt{read \char`\"db/set.mm\char`\"}.  Quotes are optional when there
is no ambiguity.}
\begin{verbatim}
MM> read demo0.mm
\end{verbatim}
Remember to press the {\em return} key after entering this command.  If
you omit the file name, Metamath will prompt you for one.   The syntax for
specifying a Macintosh file name path is given in a footnote on
p.~\pageref{includef}.\index{Macintosh file names}\index{file
names!Macintosh}

If there are any syntax errors in the database, Metamath will let you know
when it reads in the file.  The one thing that Metamath does not check when
reading in a database is that all proofs are correct, because this would
slow it down too much.  It is a good idea to periodically verify the proofs in
a database you are making changes to.  To do this, use the following command
(and do it for your \texttt{demo0.mm} file now).  Note that the \texttt{*} is a
``wild card'' meaning all proofs in the file.\index{\texttt{verify proof} command}
\begin{verbatim}
MM> verify proof *
\end{verbatim}
Metamath will report any proofs that are incorrect.

It is often useful to save the information that the Metamath program displays
on the screen. You can save everything that happens on the screen by opening a
log file. You may want to do this before you read in a database so that you
can examine any errors later on.  To open a log file, type
\begin{verbatim}
MM> open log abc.log
\end{verbatim}
This will open a file called \texttt{abc.log}, and everything that appears on the
screen from this point on will be stored in this file.  The name of the log file
is arbitrary. To close the log file, type
\begin{verbatim}
MM> close log
\end{verbatim}

Several commands let you examine what's inside your database.
Section~\ref{exploring} has an overview of some useful ones.  The
\texttt{show labels} command lets you see what statement
labels\index{label} exist.  A \texttt{*} matches any combination of
characters, and \texttt{t*} refers to all labels starting with the
letter \texttt{t}.\index{\texttt{show labels} command} The \texttt{/all}
is a ``command qualifier''\index{command qualifier} that tells Metamath
to include labels of hypotheses.  (To see the syntax explained, type
\texttt{help show labels}.)  Type
\begin{verbatim}
MM> show labels t* /all
\end{verbatim}
Metamath will respond with
\begin{verbatim}
The statement number, label, and type are shown.
3 tt $f       4 tr $f       5 ts $f       8 tze $a
9 tpl $a      19 th1 $p
\end{verbatim}

You can use the \texttt{show statement} command to get information about a
particular statement.\index{\texttt{show statement} command}
For example, you can get information about the statement with label \texttt{mp}
by typing
\begin{verbatim}
MM> show statement mp /full
\end{verbatim}
Metamath will respond with
\begin{verbatim}
Statement 17 is located on line 43 of the file
"demo0.mm".
"Define the modus ponens inference rule"
17 mp $a |- Q $.
Its mandatory hypotheses in RPN order are:
  wp $f wff P $.
  wq $f wff Q $.
  min $e |- P $.
  maj $e |- ( P -> Q ) $.
The statement and its hypotheses require the
      variables:  Q P
The variables it contains are:  Q P
\end{verbatim}
The mandatory hypotheses\index{mandatory hypothesis} and their
order\index{RPN order} are
useful to know when you are trying to understand or debug a proof.

Now you are ready to look at what's really inside our proof.  First, here is
how to look at every step in the proof---not just the ones corresponding to an
ordinary formal proof\index{formal proof}, but also the ones that build up the
formulas that appear in each ordinary formal proof step.\index{\texttt{show
proof} command}
\begin{verbatim}
MM> show proof th1 /lemmon /all
\end{verbatim}

This will display the proof on the screen in the following format:
\begin{verbatim}
 1 tt            $f term t
 2 tze           $a term 0
 3 1,2 tpl       $a term ( t + 0 )
 4 tt            $f term t
 5 3,4 weq       $a wff ( t + 0 ) = t
 6 tt            $f term t
 7 tt            $f term t
 8 6,7 weq       $a wff t = t
 9 tt            $f term t
10 9 a2          $a |- ( t + 0 ) = t
11 tt            $f term t
12 tze           $a term 0
13 11,12 tpl     $a term ( t + 0 )
14 tt            $f term t
15 13,14 weq     $a wff ( t + 0 ) = t
16 tt            $f term t
17 tze           $a term 0
18 16,17 tpl     $a term ( t + 0 )
19 tt            $f term t
20 18,19 weq     $a wff ( t + 0 ) = t
21 tt            $f term t
22 tt            $f term t
23 21,22 weq     $a wff t = t
24 20,23 wim     $a wff ( ( t + 0 ) = t -> t = t )
25 tt            $f term t
26 25 a2         $a |- ( t + 0 ) = t
27 tt            $f term t
28 tze           $a term 0
29 27,28 tpl     $a term ( t + 0 )
30 tt            $f term t
31 tt            $f term t
32 29,30,31 a1   $a |- ( ( t + 0 ) = t -> ( ( t + 0 )
                                     = t -> t = t ) )
33 15,24,26,32 mp  $a |- ( ( t + 0 ) = t -> t = t )
34 5,8,10,33 mp  $a |- t = t
\end{verbatim}

The \texttt{/lemmon} command qualifier specifies what is known as a Lemmon-style
display\index{Lemmon-style proof}\index{proof!Lemmon-style}.  Omitting the
\texttt{/lemmon} qualifier results in a tree-style proof (see
p.~\pageref{treeproof} for an example) that is somewhat less explicit but
easier to follow once you get used to it.\index{tree-style
proof}\index{proof!tree-style}

The first number on each line is the step
number of the proof.  Any numbers that follow are step numbers assigned to the
hypotheses of the statement referenced by that step.  Next is the label of
the statement referenced by the step.  The statement type of the statement
referenced comes next, followed by the math symbol\index{math symbol} string
constructed by the proof up to that step.

The last step, 34, contains the statement that is being proved.

Looking at a small piece of the proof, notice that steps 3 and 4 have
established that
\texttt{( t + 0 )} and \texttt{t} are \texttt{term}\,s, and step 5 makes use of steps 3 and
4 to establish that \texttt{( t + 0 ) = t} is a \texttt{wff}.  Let Metamath
itself tell us in detail what is happening in step 5.  Note that the
``target hypothesis'' refers to where step 5 is eventually used, i.e., in step
34.
\begin{verbatim}
MM> show proof th1 /detailed_step 5
Proof step 5:  wp=weq $a wff ( t + 0 ) = t
This step assigns source "weq" ($a) to target "wp"
($f).  The source assertion requires the hypotheses
"tt" ($f, step 3) and "tr" ($f, step 4).  The parent
assertion of the target hypothesis is "mp" ($a,
step 34).
The source assertion before substitution was:
    weq $a wff t = r
The following substitutions were made to the source
assertion:
    Variable  Substituted with
     t         ( t + 0 )
     r         t
The target hypothesis before substitution was:
    wp $f wff P
The following substitution was made to the target
hypothesis:
    Variable  Substituted with
     P         ( t + 0 ) = t
\end{verbatim}

The full proof just shown is useful to understand what is going on in detail.
However, most of the time you will just be interested in
the ``essential'' or logical steps of a proof, i.e.\ those steps
that correspond to an
ordinary formal proof\index{formal proof}.  If you type
\begin{verbatim}
MM> show proof th1 /lemmon /renumber
\end{verbatim}
you will see\label{demoproof}
\begin{verbatim}
1 a2             $a |- ( t + 0 ) = t
2 a2             $a |- ( t + 0 ) = t
3 a1             $a |- ( ( t + 0 ) = t -> ( ( t + 0 )
                                     = t -> t = t ) )
4 2,3 mp         $a |- ( ( t + 0 ) = t -> t = t )
5 1,4 mp         $a |- t = t
\end{verbatim}
Compare this to the formal proof on p.~\pageref{zeroproof} and
notice the resemblance.
By default Metamath
does not show \texttt{\$f}\index{\texttt{\$f}
statement} hypotheses and everything branching off of them in the proof tree
when the proof is displayed; this makes the proof look more like an ordinary
mathematical proof, which does not normally incorporate the explicit
construction of expressions.
This is called the ``essential'' view
(at one time you had to add the
\texttt{/essential} qualifier in the \texttt{show proof}
command to get this view, but this is now the default).
You can could use the \texttt{/all} qualifier in the \texttt{show
proof} command to also show the explicit construction of expressions.
The \texttt{/renumber} qualifier means to renumber
the steps to correspond only to what is displayed.\index{\texttt{show proof}
command}

To exit Metamath, type\index{\texttt{exit} command}
\begin{verbatim}
MM> exit
\end{verbatim}

\subsection{Some Hints for Using the Command Line Interface}

We will conclude this quick introduction to Metamath\index{Metamath} with some
helpful hints on how to navigate your way through the commands.
\index{command line interface (CLI)}

When you type commands into Metamath's CLI, you only have to type as many
characters of a command keyword\index{command keyword} as are needed to make
it unambiguous.  If you type too few characters, Metamath will tell you what
the choices are.  In the case of the \texttt{read} command, only the \texttt{r} is
needed to specify it unambiguously, so you could have typed\index{\texttt{read}
command}
\begin{verbatim}
MM> r demo0.mm
\end{verbatim}
instead of
\begin{verbatim}
MM> read demo0.mm
\end{verbatim}
In our description, we always show the full command words.  When using the
Metamath CLI commands in a command file (to be read with the \texttt{submit}
command)\index{\texttt{submit} command}, it is good practice to use
the unabbreviated command to ensure your instructions will not become ambiguous
if more commands are added to the Metamath program in the future.

The command keywords\index{command
keyword} are not case sensitive; you may type either \texttt{read} or
\texttt{ReAd}.  File names may or may not be case sensitive, depending on your
computer's operating system.  Metamath label\index{label} and math
symbol\index{math symbol} tokens\index{token} are case-sensitive.

The \texttt{help} command\index{\texttt{help} command} will provide you
with a list of topics you can get help on.  You can then type
\texttt{help} {\em topic} to get help on that topic.

If you are uncertain of a command's spelling, just type as many characters
as you remember of the command.  If you have not typed enough characters to
specify it unambiguously, Metamath will tell you what choices you have.

\begin{verbatim}
MM> show s
         ^
?Ambiguous keyword - please specify SETTINGS,
STATEMENT, or SOURCE.
\end{verbatim}

If you don't know what argument to use as part of a command, type a
\texttt{?}\index{\texttt{]}@\texttt{?}\ in command lines}\ at the
argument position.  Metamath will tell you what it expected there.

\begin{verbatim}
MM> show ?
         ^
?Expected SETTINGS, LABELS, STATEMENT, SOURCE, PROOF,
MEMORY, TRACE_BACK, or USAGE.
\end{verbatim}

Finally, you may type just the first word or words of a command followed
by {\em return}.  Metamath will prompt you for the remaining part of the
command, showing you the choices at each step.  For example, instead of
typing \texttt{show statement th1 /full} you could interact in the
following manner:
\begin{verbatim}
MM> show
SETTINGS, LABELS, STATEMENT, SOURCE, PROOF,
MEMORY, TRACE_BACK, or USAGE <SETTINGS>? st
What is the statement label <th1>?
/ or nothing <nothing>? /
TEX, COMMENT_ONLY, or FULL <TEX>? f
/ or nothing <nothing>?
19 th1 $p |- t = t $= ... $.
\end{verbatim}
After each \texttt{?}\ in this mode, you must give Metamath the
information it requests.  Sometimes Metamath gives you a list of choices
with the default choice indicated by brackets \texttt{< > }. Pressing
{\em return} after the \texttt{?}\ will select the default choice.
Answering anything else will override the default.  Note that the
\texttt{/} in command qualifiers is considered a separate
token\index{token} by the parser, and this is why it is asked for
separately.

\section{Your First Proof}\label{frstprf}

Proofs are developed with the aid of the Proof Assistant\index{Proof
Assistant}.  We will now show you how the proof of theorem \texttt{th1}
was built.  So that you can repeat these steps, we will first have the
Proof Assistant erase the proof in Metamath's source buffer\index{source
buffer}, then reconstruct it.  (The source buffer is the place in memory
where Metamath stores the information in the database when it is
\texttt{read}\index{\texttt{read} command} in.  New or modified proofs
are kept in the source buffer until a \texttt{write source}
command\index{\texttt{write source} command} is issued.)  In practice, you
would place a \texttt{?}\index{\texttt{]}@\texttt{?}\ inside proofs}\
between \texttt{\$=}\index{\texttt{\$=} keyword} and
\texttt{\$.}\index{\texttt{\$.}\ keyword}\ in the database to indicate
to Metamath\index{Metamath} that the proof is unknown, and that would be
your starting point.  Whenever the \texttt{verify proof} command encounters
a proof with a \texttt{?}\ in place of a proof step, the statement is
identified as not proved.

When I first started creating Metamath proofs, I would write down
on a piece of paper the complete
formal proof\index{formal proof} as it would appear
in a \texttt{show proof} command\index{\texttt{show proof} command}; see
the display of \texttt{show proof th1 /lemmon /re\-num\-ber} above as an
example.  After you get used to using the Proof Assistant\index{Proof
Assistant} you may get to a point where you can ``see'' the proof in your mind
and let the Proof Assistant guide you in filling in the details, at least for
simpler proofs, but until you gain that experience you may find it very useful
to write down all the details in advance.
Otherwise you may waste a lot of time as you let it take you down a wrong path.
However, others do not find this approach as helpful.
For example, Thomas Brendan Leahy\index{Leahy, Thomas Brendan}
finds that it is more helpful to him to interactively
work backward from a machine-readable statement.
David A. Wheeler\index{Wheeler, David A.}
writes down a general approach, but develops the proof
interactively by switching between
working forwards (from hypotheses and facts likely to be useful) and
backwards (from the goal) until the forwards and backwards approaches meet.
In the end, use whatever approach works for you.

A proof is developed with the Proof Assistant by working backwards, starting
with the theorem\index{theorem} to be proved, and assigning each unknown step
with a theorem or hypothesis until no more unknown steps remain.  The Proof
Assistant will not let you make an assignment unless it can be ``unified''
with the unknown step.  This means that a
substitution\index{substitution!variable}\index{variable substitution} of
variables exists that will make the assignment match the unknown step.  On the
other hand, in the middle of a proof, when working backwards, often more than
one unification\index{unification} (set of substitutions) is possible, since
there is not enough information available at that point to uniquely establish
it.  In this case you can tell Metamath which unification to choose, or you
can continue to assign unknown steps until enough information is available to
make the unification unique.

We will assume you have entered Metamath and read in the database as described
above.  The following dialog shows how the proof was developed.  For more
details on what some of the commands do, refer to Section~\ref{pfcommands}.
\index{\texttt{prove} command}

\begin{verbatim}
MM> prove th1
Entering the Proof Assistant.  Type HELP for help, EXIT
to exit.  You will be working on the proof of statement th1:
  $p |- t = t
Note:  The proof you are starting with is already complete.
MM-PA>
\end{verbatim}

The \verb/MM-PA>/ prompt means we are inside the Proof
Assistant.\index{Proof Assistant} Most of the regular Metamath commands
(\texttt{show statement}, etc.) are still available if you need them.

\begin{verbatim}
MM-PA> delete all
The entire proof was deleted.
\end{verbatim}

We have deleted the whole proof so we can start from scratch.

\begin{verbatim}
MM-PA> show new_proof/lemmon/all
1 ?              $? |- t = t
\end{verbatim}

The \texttt{show new{\char`\_}proof} command\index{\texttt{show
new{\char`\_}proof} command} is like \texttt{show proof} except that we
don't specify a statement; instead, the proof we're working on is
displayed.

\begin{verbatim}
MM-PA> assign 1 mp
To undo the assignment, DELETE STEP 5 and INITIALIZE, UNIFY
if needed.
3   min=?  $? |- $2
4   maj=?  $? |- ( $2 -> t = t )
\end{verbatim}

The \texttt{assign} command\index{\texttt{assign} command} above means
``assign step 1 with the statement whose label is \texttt{mp}.''  Note
that step renumbering will constantly occur as you assign steps in the
middle of a proof; in general all steps from the step you assign until
the end of the proof will get moved up.  In this case, what used to be
step 1 is now step 5, because the (partial) proof now has five steps:
the four hypotheses of the \texttt{mp} statement and the \texttt{mp}
statement itself.  Let's look at all the steps in our partial proof:

\begin{verbatim}
MM-PA> show new_proof/lemmon/all
1 ?              $? wff $2
2 ?              $? wff t = t
3 ?              $? |- $2
4 ?              $? |- ( $2 -> t = t )
5 1,2,3,4 mp     $a |- t = t
\end{verbatim}

The symbol \texttt{\$2} is a temporary variable\index{temporary
variable} that represents a symbol sequence not yet known.  In the final
proof, all temporary variables will be eliminated.  The general format
for a temporary variable is \texttt{\$} followed by an integer.  Note
that \texttt{\$} is not a legal character in a math symbol (see
Section~\ref{dollardollar}, p.~\pageref{dollardollar}), so there will
never be a naming conflict between real symbols and temporary variables.

Unknown steps 1 and 2 are constructions of the two wffs used by the
modus ponens rule.  As you will see at the end of this section, the
Proof Assistant\index{Proof Assistant} can usually figure these steps
out by itself, and we will not have to worry about them.  Therefore from
here on we will display only the ``essential'' hypotheses, i.e.\ those
steps that correspond to traditional formal proofs\index{formal proof}.

\begin{verbatim}
MM-PA> show new_proof/lemmon
3 ?              $? |- $2
4 ?              $? |- ( $2 -> t = t )
5 3,4 mp         $a |- t = t
\end{verbatim}

Unknown steps 3 and 4 are the ones we must focus on.  They correspond to the
minor and major premises of the modus ponens rule.  We will assign them as
follows.  Notice that because of the step renumbering that takes place
after an assignment, it is advantageous to assign unknown steps in reverse
order, because earlier steps will not get renumbered.

\begin{verbatim}
MM-PA> assign 4 mp
To undo the assignment, DELETE STEP 8 and INITIALIZE, UNIFY
if needed.
3   min=?  $? |- $2
6     min=?  $? |- $4
7     maj=?  $? |- ( $4 -> ( $2 -> t = t ) )
\end{verbatim}

We are now going to describe an obscure feature that you will probably
never use but should be aware of.  The Metamath language allows empty
symbol sequences to be substituted for variables, but in most formal
systems this feature is never used.  One of the few examples where is it
used is the MIU-system\index{MIU-system} described in
Appendix~\ref{MIU}.  But such systems are rare, and by default this
feature is turned off in the Proof Assistant.  (It is always allowed for
{\tt verify proof}.)  Let us turn it on and see what
happens.\index{\texttt{set empty{\char`\_}substitution} command}

\begin{verbatim}
MM-PA> set empty_substitution on
Substitutions with empty symbol sequences is now allowed.
\end{verbatim}

With this feature enabled, more unifications will be
ambiguous\index{ambiguous unification}\index{unification!ambiguous} in
the middle of a proof, because
substitution\index{substitution!variable}\index{variable substitution}
of variables with empty symbol sequences will become an additional
possibility.  Let's see what happens when we make our next assignment.

\begin{verbatim}
MM-PA> assign 3 a2
There are 2 possible unifications.  Please select the correct
    one or Q if you want to UNIFY later.
Unify:  |- $6
 with:  |- ( $9 + 0 ) = $9
Unification #1 of 2 (weight = 7):
  Replace "$6" with "( + 0 ) ="
  Replace "$9" with ""
  Accept (A), reject (R), or quit (Q) <A>? r
\end{verbatim}

The first choice presented is the wrong one.  If we had selected it,
temporary variable \texttt{\$6} would have been assigned a truncated
wff, and temporary variable \texttt{\$9} would have been assigned an
empty sequence (which is not allowed in our system).  With this choice,
eventually we would reach a point where we would get stuck because
we would end up with steps impossible to prove.  (You may want to
try it.)  We typed \texttt{r} to reject the choice.

\begin{verbatim}
Unification #2 of 2 (weight = 21):
  Replace "$6" with "( $9 + 0 ) = $9"
  Accept (A), reject (R), or quit (Q) <A>? q
To undo the assignment, DELETE STEP 4 and INITIALIZE, UNIFY
if needed.
 7     min=?  $? |- $8
 8     maj=?  $? |- ( $8 -> ( $6 -> t = t ) )
\end{verbatim}

The second choice is correct, and normally we would type \texttt{a}
to accept it.  But instead we typed \texttt{q} to show what will happen:
it will leave the step with an unknown unification, which can be
seen as follows:

\begin{verbatim}
MM-PA> show new_proof/not_unified
 4   min    $a |- $6
        =a2  = |- ( $9 + 0 ) = $9
\end{verbatim}

Later we can unify this with the \texttt{unify}
\texttt{all/interactive} command.

The important point to remember is that occasionally you will be
presented with several unification choices while entering a proof, when
the program determines that there is not enough information yet to make
an unambiguous choice automatically (and this can happen even with
\texttt{set empty{\char`\_}substitution} turned off).  Usually it is
obvious by inspection which choice is correct, since incorrect ones will
tend to be meaningless fragments of wffs.  In addition, the correct
choice will usually be the first one presented, unlike our example
above.

Enough of this digression.  Let us go back to the default setting.

\begin{verbatim}
MM-PA> set empty_substitution off
The ability to substitute empty expressions for variables
has been turned off.  Note that this may make the Proof
Assistant too restrictive in some cases.
\end{verbatim}

If we delete the proof, start over, and get to the point where
we digressed above, there will no longer be an ambiguous unification.

\begin{verbatim}
MM-PA> assign 3 a2
To undo the assignment, DELETE STEP 4 and INITIALIZE, UNIFY
if needed.
 7     min=?  $? |- $4
 8     maj=?  $? |- ( $4 -> ( ( $5 + 0 ) = $5 -> t = t ) )
\end{verbatim}

Let us look at our proof so far, and continue.

\begin{verbatim}
MM-PA> show new_proof/lemmon
 4 a2            $a |- ( $5 + 0 ) = $5
 7 ?             $? |- $4
 8 ?             $? |- ( $4 -> ( ( $5 + 0 ) = $5 -> t = t ) )
 9 7,8 mp        $a |- ( ( $5 + 0 ) = $5 -> t = t )
10 4,9 mp        $a |- t = t
MM-PA> assign 8 a1
To undo the assignment, DELETE STEP 11 and INITIALIZE, UNIFY
if needed.
 7     min=?  $? |- ( t + 0 ) = t
MM-PA> assign 7 a2
To undo the assignment, DELETE STEP 8 and INITIALIZE, UNIFY
if needed.
MM-PA> show new_proof/lemmon
 4 a2            $a |- ( t + 0 ) = t
 8 a2            $a |- ( t + 0 ) = t
12 a1            $a |- ( ( t + 0 ) = t -> ( ( t + 0 ) = t ->
                                                    t = t ) )
13 8,12 mp       $a |- ( ( t + 0 ) = t -> t = t )
14 4,13 mp       $a |- t = t
\end{verbatim}

Now all temporary variables and unknown steps have been eliminated from the
``essential'' part of the proof.  When this is achieved, the Proof
Assistant\index{Proof Assistant} can usually figure out the rest of the proof
automatically.  (Note that the \texttt{improve} command can occasionally be
useful for filling in essential steps as well, but it only tries to make use
of statements that introduce no new variables in their hypotheses, which is
not the case for \texttt{mp}. Also it will not try to improve steps containing
temporary variables.)  Let's look at the complete proof, then run
the \texttt{improve} command, then look at it again.

\begin{verbatim}
MM-PA> show new_proof/lemmon/all
 1 ?             $? wff ( t + 0 ) = t
 2 ?             $? wff t = t
 3 ?             $? term t
 4 3 a2          $a |- ( t + 0 ) = t
 5 ?             $? wff ( t + 0 ) = t
 6 ?             $? wff ( ( t + 0 ) = t -> t = t )
 7 ?             $? term t
 8 7 a2          $a |- ( t + 0 ) = t
 9 ?             $? term ( t + 0 )
10 ?             $? term t
11 ?             $? term t
12 9,10,11 a1    $a |- ( ( t + 0 ) = t -> ( ( t + 0 ) = t ->
                                                    t = t ) )
13 5,6,8,12 mp   $a |- ( ( t + 0 ) = t -> t = t )
14 1,2,4,13 mp   $a |- t = t
\end{verbatim}

\begin{verbatim}
MM-PA> improve all
A proof of length 1 was found for step 11.
A proof of length 1 was found for step 10.
A proof of length 3 was found for step 9.
A proof of length 1 was found for step 7.
A proof of length 9 was found for step 6.
A proof of length 5 was found for step 5.
A proof of length 1 was found for step 3.
A proof of length 3 was found for step 2.
A proof of length 5 was found for step 1.
Steps 1 and above have been renumbered.
CONGRATULATIONS!  The proof is complete.  Use SAVE
NEW_PROOF to save it.  Note:  The Proof Assistant does
not detect $d violations.  After saving the proof, you
should verify it with VERIFY PROOF.
\end{verbatim}

The \texttt{save new{\char`\_}proof} command\index{\texttt{save
new{\char`\_}proof} command} will save the proof in the database.  Here
we will just display it in a form that can be clipped out of a log file
and inserted manually into the database source file with a text
editor.\index{normal proof}\index{proof!normal}

\begin{verbatim}
MM-PA> show new_proof/normal
---------Clip out the proof below this line:
      tt tze tpl tt weq tt tt weq tt a2 tt tze tpl tt weq
      tt tze tpl tt weq tt tt weq wim tt a2 tt tze tpl tt
      tt a1 mp mp $.
---------The proof of 'th1' to clip out ends above this line.
\end{verbatim}

There is another proof format called ``compressed''\index{compressed
proof}\index{proof!compressed} that you will see in databases.  It is
not important to understand how it is encoded but only to recognize it
when you see it.  Its only purpose is to reduce storage requirements for
large proofs.  A compressed proof can always be converted to a normal
one and vice-versa, and the Metamath \texttt{show proof}
commands\index{\texttt{show proof} command} work equally well with
compressed proofs.  The compressed proof format is described in
Appendix~\ref{compressed}.

\begin{verbatim}
MM-PA> show new_proof/compressed
---------Clip out the proof below this line:
      ( tze tpl weq a2 wim a1 mp ) ABCZADZAADZAEZJJKFLIA
      AGHH $.
---------The proof of 'th1' to clip out ends above this line.
\end{verbatim}

Now we will exit the Proof Assistant.  Since we made changes to the proof,
it will warn us that we have not saved it.  In this case, we don't care.

\begin{verbatim}
MM-PA> exit
Warning:  You have not saved changes to the proof.
Do you want to EXIT anyway (Y, N) <N>? y
Exiting the Proof Assistant.
Type EXIT again to exit Metamath.
\end{verbatim}

The Proof Assistant\index{Proof Assistant} has several other commands
that can help you while creating proofs.  See Section~\ref{pfcommands}
for a list of them.

A command that is often useful is \texttt{minimize{\char`\_}with
*/brief}, which tries to shorten the proof.  It can make the process
more efficient by letting you write a somewhat ``sloppy'' proof then
clean up some of the fine details of optimization for you (although it
can't perform miracles such as restructuring the overall proof).

\section{A Note About Editing a Data\-base File}

Once your source file contains proofs, there are some restrictions on
how you can edit it so that the proofs remain valid.  Pay particular
attention to these rules, since otherwise you can lose a lot of work.
It is a good idea to periodically verify all proofs with \texttt{verify
proof *} to ensure their integrity.

If your file contains only normal (as opposed to compressed) proofs, the
main rule is that you may not change the order of the mandatory
hypotheses\index{mandatory hypothesis} of any statement referenced in a
later proof.  For example, if you swap the order of the major and minor
premise in the modus ponens rule, all proofs making use of that rule
will become incorrect.  The \texttt{show statement}
command\index{\texttt{show statement} command} will show you the
mandatory hypotheses of a statement and their order.

If a statement has a compressed proof, you also must not change the
order of {\em its} mandatory hypotheses.  The compressed proof format
makes use of this information as part of the compression technique.
Note that swapping the names of two variables in a theorem will change
the order of its mandatory hypotheses.

The safest way to edit a statement, say \texttt{mytheorem}, is to
duplicate it then rename the original to \texttt{mytheoremOLD}
throughout the database.  Once the edited version is re-proved, all
statements referencing \texttt{mytheoremOLD} can be updated in the Proof
Assistant using \texttt{minimize{\char`\_}with
mytheorem
/allow{\char`\_}growth}.\index{\texttt{minimize{\char`\_}with} command}
% 3/10/07 Note: line-breaking the above results in duplicate index entries

\chapter{Abstract Mathematics Revealed}\label{fol}

\section{Logic and Set Theory}\label{logicandsettheory}

\begin{quote}
  {\em Set theory can be viewed as a form of exact theology.}
  \flushright\sc  Rudy Rucker\footnote{\cite{Barrow}, p.~31.}\\
\end{quote}\index{Rucker, Rudy}

Despite its seeming complexity, all of standard mathematics, no matter how
deep or abstract, can amazingly enough be derived from a relatively small set
of axioms\index{axiom} or first principles. The development of these axioms is
among the most impressive and important accomplishments of mathematics in the
20th century. Ultimately, these axioms can be broken down into a set of rules
for manipulating symbols that any technically oriented person can follow.

We will not spend much time trying to convey a deep, higher-level
understanding of the meaning of the axioms. This kind of understanding
requires some mathematical sophistication as well as an understanding of the
philosophy underlying the foundations of mathematics and typically develops
over time as you work with mathematics.  Our goal, instead, is to give you the
immediate ability to follow how theorems\index{theorem} are derived from the
axioms and from other theorems.  This will be similar to learning the syntax
of a computer language, which lets you follow the details in a program but
does not necessarily give you the ability to write non-trivial programs on
your own, an ability that comes with practice. For now don't be alarmed by
abstract-sounding names of the axioms; just focus on the rules for
manipulating the symbols, which follow the simple conventions of the
Metamath\index{Metamath} language.

The axioms that underlie all of standard mathematics consist of axioms of logic
and axioms of set theory. The axioms of logic are divided into two
subcategories, propositional calculus\index{propositional calculus} (sometimes
called sentential logic\index{sentential logic}) and predicate calculus
(sometimes called first-order logic\index{first-order logic}\index{quantifier
theory}\index{predicate calculus} or quantifier theory).  Propositional
calculus is a prerequisite for predicate calculus, and predicate calculus is a
prerequisite for set theory.  The version of set theory most commonly used is
Zermelo--Fraenkel set theory\index{Zermelo--Fraenkel set theory}\index{set theory}
with the axiom of choice,
often abbreviated as ZFC\index{ZFC}.

Here in a nutshell is what the axioms are all about in an informal way. The
connection between this description and symbols we will show you won't be
immediately apparent and in principle needn't ever be.  Our description just
tries to summarize what mathematicians think about when they work with the
axioms.

Logic is a set of rules that allow us determine truths given other truths.
Put another way,
logic is more or less the translation of what we would consider common sense
into a rigorous set of axioms.\index{axioms of logic}  Suppose $\varphi$,
$\psi$, and $\chi$ (the Greek letters phi, psi, and chi) represent statements
that are either true or false, and $x$ is a variable\index{variable!in predicate
calculus} ranging over some group of mathematical objects (sets, integers,
real numbers, etc.). In mathematics, a ``statement'' really means a formula,
and $\psi$ could be for example ``$x = 2$.''
Propositional calculus\index{propositional calculus}
allows us to use variables that are either true or false
and make deductions such as
``if $\varphi$ implies $\psi$ and $\psi$ implies $\chi$, then $\varphi$
implies $\chi$.''
Predicate calculus\index{predicate calculus}
extends propositional calculus by also allowing us
to discuss statements about objects (not just true and false values), including
statements about ``all'' or ``at least one'' object.
For example, predicate calculus allows to say,
``if $\varphi$ is true for all $x$, then $\varphi$ is true for some $x$.''
The logic used in \texttt{set.mm} is standard classical logic
(as opposed to other logic systems like intuitionistic logic).

Set theory\index{set theory} has to do with the manipulation of objects and
collections of objects, specifically the abstract, imaginary objects that
mathematics deals with, such as numbers. Everything that is claimed to exist
in mathematics is considered to be a set.  A set called the empty
set\index{empty set} contains nothing.  We represent the empty set by
$\varnothing$.  Many sets can be built up from the empty set.  There is a set
represented by $\{\varnothing\}$ that contains the empty set, another set
represented by $\{\varnothing,\{\varnothing\}\}$ that contains this set as
well as the empty set, another set represented by $\{\{\varnothing\}\}$ that
contains just the set that contains the empty set, and so on ad infinitum. All
mathematical objects, no matter how complex, are defined as being identical to
certain sets: the integer\index{integer} 0 is defined as the empty set, the
integer 1 is defined as $\{\varnothing\}$, the integer 2 is defined as
$\{\varnothing,\{\varnothing\}\}$.  (How these definitions were chosen doesn't
matter now, but the idea behind it is that these sets have the properties we
expect of integers once suitable operations are defined.)  Mathematical
operations, such as addition, are defined in terms of operations on
sets---their union\index{set union}, intersection\index{set intersection}, and
so on---operations you may have used in elementary school when you worked
with groups of apples and oranges.

With a leap of faith, the axioms also postulate the existence of infinite
sets\index{infinite set}, such as the set of all non-negative integers ($0, 1,
2,\ldots$, also called ``natural numbers''\index{natural number}).  This set
can't be represented with the brace notation\index{brace notation} we just
showed you, but requires a more complicated notation called ``class
abstraction.''\index{class abstraction}\index{abstraction class}  For
example, the infinite set $\{ x |
\mbox{``$x$ is a natural number''} \} $ means the ``set of all objects $x$
such that $x$ is a natural number'' i.e.\ the set of natural numbers; here,
``$x$ is a natural number'' is a rather complicated formula when broken down
into the primitive symbols.\label{expandom}\footnote{The statement ``$x$ is a
natural number'' is formally expressed as ``$x \in \omega$,'' where $\in$
(stylized epsilon) means ``is in'' or ``is an element of'' and $\omega$
(omega) means ``the set of natural numbers.''  When ``$x\in\omega$'' is
completely expanded in terms of the primitive symbols of set theory, the
result is  $\lnot$ $($ $\lnot$ $($ $\forall$ $z$ $($ $\lnot$ $\forall$ $w$ $($
$z$ $\in$ $w$ $\rightarrow$ $\lnot$ $w$ $\in$ $x$ $)$ $\rightarrow$ $z$ $\in$
$x$ $)$ $\rightarrow$ $($ $\forall$ $z$ $($ $\lnot$ $($ $\forall$ $w$ $($ $w$
$\in$ $z$ $\rightarrow$ $w$ $\in$ $x$ $)$ $\rightarrow$ $\forall$ $w$ $\lnot$
$w$ $\in$ $z$ $)$ $\rightarrow$ $\lnot$ $\forall$ $w$ $($ $w$ $\in$ $z$
$\rightarrow$ $\lnot$ $\forall$ $v$ $($ $v$ $\in$ $z$ $\rightarrow$ $\lnot$
$v$ $\in$ $w$ $)$ $)$ $)$ $\rightarrow$ $\lnot$ $\forall$ $z$ $\forall$ $w$
$($ $\lnot$ $($ $z$ $\in$ $x$ $\rightarrow$ $\lnot$ $w$ $\in$ $x$ $)$
$\rightarrow$ $($ $\lnot$ $z$ $\in$ $w$ $\rightarrow$ $($ $\lnot$ $z$ $=$ $w$
$\rightarrow$ $w$ $\in$ $z$ $)$ $)$ $)$ $)$ $)$ $\rightarrow$ $\lnot$
$\forall$ $y$ $($ $\lnot$ $($ $\lnot$ $($ $\forall$ $z$ $($ $\lnot$ $\forall$
$w$ $($ $z$ $\in$ $w$ $\rightarrow$ $\lnot$ $w$ $\in$ $y$ $)$ $\rightarrow$
$z$ $\in$ $y$ $)$ $\rightarrow$ $($ $\forall$ $z$ $($ $\lnot$ $($ $\forall$
$w$ $($ $w$ $\in$ $z$ $\rightarrow$ $w$ $\in$ $y$ $)$ $\rightarrow$ $\forall$
$w$ $\lnot$ $w$ $\in$ $z$ $)$ $\rightarrow$ $\lnot$ $\forall$ $w$ $($ $w$
$\in$ $z$ $\rightarrow$ $\lnot$ $\forall$ $v$ $($ $v$ $\in$ $z$ $\rightarrow$
$\lnot$ $v$ $\in$ $w$ $)$ $)$ $)$ $\rightarrow$ $\lnot$ $\forall$ $z$
$\forall$ $w$ $($ $\lnot$ $($ $z$ $\in$ $y$ $\rightarrow$ $\lnot$ $w$ $\in$
$y$ $)$ $\rightarrow$ $($ $\lnot$ $z$ $\in$ $w$ $\rightarrow$ $($ $\lnot$ $z$
$=$ $w$ $\rightarrow$ $w$ $\in$ $z$ $)$ $)$ $)$ $)$ $\rightarrow$ $($
$\forall$ $z$ $\lnot$ $z$ $\in$ $y$ $\rightarrow$ $\lnot$ $\forall$ $w$ $($
$\lnot$ $($ $w$ $\in$ $y$ $\rightarrow$ $\lnot$ $\forall$ $z$ $($ $w$ $\in$
$z$ $\rightarrow$ $\lnot$ $z$ $\in$ $y$ $)$ $)$ $\rightarrow$ $\lnot$ $($
$\lnot$ $\forall$ $z$ $($ $w$ $\in$ $z$ $\rightarrow$ $\lnot$ $z$ $\in$ $y$
$)$ $\rightarrow$ $w$ $\in$ $y$ $)$ $)$ $)$ $)$ $\rightarrow$ $x$ $\in$ $y$
$)$ $)$ $)$. Section~\ref{hierarchy} shows the hierarchy of definitions that
leads up to this expression.}\index{stylized epsilon ($\in$)}\index{omega
($\omega$)}  Actually, the primitive symbols don't even include the brace
notation.  The brace notation is a high-level definition, which you can find in
Section~\ref{hierarchy}.

Interestingly, the arithmetic of integers\index{integer} and
rationals\index{rational number} can be developed without appealing to the
existence of an infinite set, whereas the arithmetic of real
numbers\index{real number} requires it.

Each variable\index{variable!in set theory} in the axioms of set theory
represents an arbitrary set, and the axioms specify the legal kinds of things
you can do with these variables at a very primitive level.

Now, you may think that numbers and arithmetic are a lot more intuitive and
fundamental than sets and therefore should be the foundation of mathematics.
What is really the case is that you've dealt with numbers all your life and
are comfortable with a few rules for manipulating them such as addition and
multiplication.  Those rules only cover a small portion of what can be done
with numbers and only a very tiny fraction of the rest of mathematics.  If you
look at any elementary book on number theory, you will quickly become lost if
these are the only rules that you know.  Even though such books may present a
list of ``axioms''\index{axiom} for arithmetic, the ability to use the axioms
and to understand proofs of theorems\index{theorem} (facts) about numbers
requires an implicit mathematical talent that frustrates many people
from studying abstract mathematics.  The kind of mathematics that most people
know limits them to the practical, everyday usage of blindly manipulating
numbers and formulas, without any understanding of why those rules are correct
nor any ability to go any further.  For example, do you know why multiplying
two negative numbers yields a positive number?  Starting with set theory, you
will also start off blindly manipulating symbols according to the rules we give
you, but with the advantage that these rules will allow you, in principle, to
access {\em all} of mathematics, not just a tiny part of it.

Of course, concrete examples are often helpful in the learning process. For
example, you can verify that $2\cdot 3=3 \cdot 2$ by actually grouping
objects and can easily ``see'' how it generalizes to $x\cdot y = y\cdot x$,
even though you might not be able to rigorously prove it.  Similarly, in set
theory it can be helpful to understand how the axioms of set theory apply to
(and are correct for) small finite collections of objects.  You should be aware
that in set theory intuition can be misleading for infinite collections, and
rigorous proofs become more important.  For example, while $x\cdot y = y\cdot
x$ is correct for finite ordinals (which are the natural numbers), it is not
usually true for infinite ordinals.

\section{The Axioms for All of Mathematics}

In this section\index{axioms for mathematics}, we will show you the axioms
for all of standard mathematics (i.e.\ logic and set theory) as they are
traditionally presented.  The traditional presentation is useful for someone
with the mathematical experience needed to correctly manipulate high-level
abstract concepts.  For someone without this talent, knowing how to actually
make use of these axioms can be difficult.  The purpose of this section is to
allow you to see how the version of the axioms used in the standard
Metamath\index{Metamath} database \texttt{set.mm}\index{set
theory database (\texttt{set.mm})} relates to  the typical version
in textbooks, and also to give you an informal feel for them.

\subsection{Propositional Calculus}

Propositional calculus\index{propositional calculus} concerns itself with
statements that can be interpreted as either true or false.  Some examples of
statements (outside of mathematics) that are either true or false are ``It is
raining today'' and ``The United States has a female president.'' In
mathematics, as we mentioned, statements are really formulas.

In propositional calculus, we don't care what the statements are.  We also
treat a logical combination of statements, such as ``It is raining today and
the United States has a female president,'' no differently from a single
statement.  Statements and their combinations are called well-formed formulas
(wffs)\index{well-formed formula (wff)}.  We define wffs only in terms of
other wffs and don't define what a ``starting'' wff is.  As is common practice
in the literature, we use Greek letters to represent wffs.

Specifically, suppose $\varphi$ and $\psi$ are wffs.  Then the combinations
$\varphi\rightarrow\psi$ (``$\varphi$ implies $\psi$,'' also read ``if
$\varphi$ then $\psi$'')\index{implication ($\rightarrow$)} and $\lnot\varphi$
(``not $\varphi$'')\index{negation ($\lnot$)} are also wffs.

The three axioms of propositional calculus\index{axioms of propositional
calculus} are all wffs of the following form:\footnote{A remarkable result of
C.~A.~Meredith\index{Meredith, C. A.} squeezes these three axioms into the
single axiom $((((\varphi\rightarrow \psi)\rightarrow(\neg \chi\rightarrow\neg
\theta))\rightarrow \chi )\rightarrow \tau)\rightarrow((\tau\rightarrow
\varphi)\rightarrow(\theta\rightarrow \varphi))$ \cite{CAMeredith},
which is believed to be the shortest possible.}
\begin{center}
     $\varphi\rightarrow(\psi\rightarrow \varphi)$\\

     $(\varphi\rightarrow (\psi\rightarrow \chi))\rightarrow
((\varphi\rightarrow  \psi)\rightarrow (\varphi\rightarrow \chi))$\\

     $(\neg \varphi\rightarrow \neg\psi)\rightarrow (\psi\rightarrow
\varphi)$
\end{center}

These three axioms are widely used.
They are attributed to Jan {\L}ukasiewicz
(pronounced woo-kah-SHAY-vitch) and was popularized by Alonzo Church,
who called it system P2. (Thanks to Ted Ulrich for this information.)

There are an infinite number of axioms, one for each possible
wff\index{well-formed formula (wff)} of the above form.  (For this reason,
axioms such as the above are often called ``axiom schemes.''\index{axiom
scheme})  Each Greek letter in the axioms may be substituted with a more
complex wff to result in another axiom.  For example, substituting
$\neg(\varphi\rightarrow\chi)$ for $\varphi$ in the first axiom yields
$\neg(\varphi\rightarrow\chi)\rightarrow(\psi\rightarrow
\neg(\varphi\rightarrow\chi))$, which is still an axiom.

To deduce new true statements (theorems\index{theorem}) from the axioms, a
rule\index{rule} called ``modus ponens''\index{modus ponens} is used.  This
rule states that if the wff $\varphi$ is an axiom or a theorem, and the wff
$\varphi\rightarrow\psi$ is an axiom or a theorem, then the wff $\psi$ is also
a theorem\index{theorem}.

As a non-mathematical example of modus ponens, suppose we have proved (or
taken as an axiom) ``Bob is a man'' and separately have proved (or taken as
an axiom) ``If Bob is a man, then Bob is a human.''  Using the rule of modus
ponens, we can logically deduce, ``Bob is a human.''

From Metamath's\index{Metamath} point of view, the axioms and the rule of
modus ponens just define a mechanical means for deducing new true statements
from existing true statements, and that is the complete content of
propositional calculus as far as Metamath is concerned.  You can read a logic
textbook to gain a better understanding of their meaning, or you can just let
their meaning slowly become apparent to you after you use them for a while.

It is actually rather easy to check to see if a formula is a theorem of
propositional calculus.  Theorems of propositional calculus are also called
``tautologies.''\index{tautology}  The technique to check whether a formula is
a tautology is called the ``truth table method,''\index{truth table} and it
works like this.  A wff $\varphi\rightarrow\psi$ is false whenever $\varphi$ is true
and $\psi$ is false.  Otherwise it is true.  A wff $\lnot\varphi$ is false
whenever $\varphi$ is true and false otherwise. To verify a tautology such as
$\varphi\rightarrow(\psi\rightarrow \varphi)$, you break it down into sub-wffs and
construct a truth table that accounts for all possible combinations of true
and false assigned to the wff metavariables:
\begin{center}\begin{tabular}{|c|c|c|c|}\hline
\mbox{$\varphi$} & \mbox{$\psi$} & \mbox{$\psi\rightarrow\varphi$}
    & \mbox{$\varphi\rightarrow(\psi\rightarrow \varphi)$} \\ \hline \hline
              T   &  T    &      T       &        T    \\ \hline
              T   &  F    &      T       &        T    \\ \hline
              F   &  T    &      F       &        T    \\ \hline
              F   &  F    &      T       &        T    \\ \hline
\end{tabular}\end{center}
If all entries in the last column are true, the formula is a tautology.

Now, the truth table method doesn't tell you how to prove the tautology from
the axioms, but only that a proof exists.  Finding an actual proof (especially
one that is short and elegant) can be challenging.  Methods do exist for
automatically generating proofs in propositional calculus, but the proofs that
result can sometimes be very long.  In the Metamath \texttt{set.mm}\index{set
theory database (\texttt{set.mm})} database, most
or all proofs were created manually.

Section \ref{metadefprop} discusses various definitions
that make propositional calculus easier to use.
For example, we define:

\begin{itemize}
\item $\varphi \vee \psi$
  is true if either $\varphi$ or $\psi$ (or both) are true
  (this is disjunction\index{disjunction ($\vee$)}
  aka logical {\sc or}\index{logical {\sc or} ($\vee$)}).

\item $\varphi \wedge \psi$
  is true if both $\varphi$ and $\psi$ are true
  (this is conjunction\index{conjunction ($\wedge$)}
  aka logical {\sc and}\index{logical {\sc and} ($\wedge$)}).

\item $\varphi \leftrightarrow \psi$
  is true if $\varphi$ and $\psi$ have the same value, that is,
  they are both true or both false
  (this is the biconditional\index{biconditional ($\leftrightarrow$)}).
\end{itemize}

\subsection{Predicate Calculus}

Predicate calculus\index{predicate calculus} introduces the concept of
``individual variables,''\index{variable!in predicate calculus}\index{individual
variable} which
we will usually just call ``variables.''
These variables can represent something other than true or false (wffs),
and will always represent sets when we get to set theory.  There are also
three new symbols $\forall$\index{universal quantifier ($\forall$)},
$=$\index{equality ($=$)}, and $\in$\index{stylized epsilon ($\in$)},
read ``for all,'' ``equals,'' and ``is an element of''
respectively.  We will represent variables with the letters $x$, $y$, $z$, and
$w$, as is common practice in the literature.
For example, $\forall x \varphi$ means ``for all possible values of
$x$, $\varphi$ is true.''

In predicate calculus, we extend the definition of a wff\index{well-formed
formula (wff)}.  If $\varphi$ is a wff and $x$ and $y$ are variables, then
$\forall x \, \varphi$, $x=y$, and $x\in y$ are wffs. Note that these three new
types of wffs can be considered ``starting'' wffs from which we can build
other wffs with $\rightarrow$ and $\neg$ .  The concept of a starting wff was
absent in propositional calculus.  But starting wff or not, all we are really
concerned with is whether our wffs are correctly constructed according to
these mechanical rules.

A quick aside:
To prevent confusion, it might be best at this point to think of the variables
of Metamath\index{Metamath} as ``metavariables,''\index{metavariable} because
they are not quite the same as the variables we are introducing here.  A
(meta)variable in Metamath can be a wff or an individual variable, as well
as many other things; in general, it represents a kind of place holder for an
unspecified sequence of math symbols\index{math symbol}.

Unlike propositional calculus, no decision procedure\index{decision procedure}
analogous to the truth table method exists (nor theoretically can exist) that
will definitely determine whether a formula is a theorem of predicate
calculus.  Much of the work in the field of automated theorem
proving\index{automated theorem proving} has been dedicated to coming up with
clever heuristics for proving theorems of predicate calculus, but they can
never be guaranteed to work always.

Section \ref{metadefpred} discusses various definitions
that make predicate calculus easier to use.
For example, we define
$\exists x \varphi$ to mean
``there exists at least one possible value of $x$ where $\varphi$ is true.''

We now turn to looking at how predicate calculus can be formally
represented.

\subsubsection{Common Axioms}

There is a new rule of inference in predicate calculus:  if $\varphi$ is
an axiom or a theorem, then $\forall x \,\varphi$ is also a
theorem\index{theorem}.  This is called the rule of
``generalization.''\index{rule of generalization}
This is easily represented in Metamath.

In standard texts of logic, there are often two axioms of predicate
calculus\index{axioms of predicate calculus}:
\begin{center}
  $\forall x \,\varphi ( x ) \rightarrow \varphi ( y )$,
      where ``$y$ is properly substituted for $x$.''\\
  $\forall x ( \varphi \rightarrow \psi )\rightarrow ( \varphi \rightarrow
    \forall x\, \psi )$,
    where ``$x$ is not free in $\varphi$.''
\end{center}

Now at first glance, this seems simple:  just two axioms.  However,
conditional clauses are attached to each axiom describing requirements that
may seem puzzling to you.  In addition, the first axiom puts a variable symbol
in parentheses after each wff, seemingly violating our definition of a
wff\index{well-formed formula (wff)}; this is just an informal way of
referring to some arbitrary variable that may occur in the wff.  The
conditional clauses do, of course, have a precise meaning, but as it turns out
the precise meaning is somewhat complicated and awkward to formalize in a
way that a computer can handle easily.  Unlike propositional calculus, a
certain amount of mathematical sophistication and practice is needed to be
able to easily grasp and manipulate these concepts correctly.

Predicate calculus may be presented with or without axioms for
equality\index{axioms of equality}\index{equality ($=$)}. We will require the
axioms of equality as a prerequisite for the version of set theory we will
use.  The axioms for equality, when included, are often represented using these
two axioms:
\begin{center}
$x=x$\\ \ \\
$x=y\rightarrow (\varphi(x,x)\rightarrow\varphi(x,y))$ where ``$\varphi(x,y)$
   arises from $\varphi(x,x)$ by replacing some, but not necessarily all,
   free\index{free variable}
   occurrences of $x$ by $y$,\\ provided that $y$ is free for $x$
   in $\varphi(x,x)$.'' \end{center}
% (Mendelson p. 95)
The first equality axiom is simple, but again,
the condition on the second one is
somewhat awkward to implement on a computer.

\subsubsection{Tarski System S2}

Of course, we are not the first to notice the complications of these
predicate calculus axioms when being rigorous.

Well-known logician Alfred Tarski published in 1965
a system he called system S2\cite[p.~77]{Tarski1965}.
Tarski's system is \textit{exactly equivalent} to the traditional textbook
formalization, but (by clever use of equality axioms) it eliminates the
latter's primitive notions of ``proper substitution'' and ``free variable,''
replacing them with direct substitution and the notion of a variable
not occurring in a formula (which we express with distinct variable
constraints).

In advocating his system, Tarski wrote, ``The relatively complicated
character of [free variables and proper substitution] is a source
of certain inconveniences of both practical and theoretical nature;
this is clearly experienced both in teaching an elementary course of
mathematical logic and in formalizing the syntax of predicate logic for
some theoretical purposes''\cite[p.~61]{Tarski1965}\index{Tarski, Alfred}.

\subsubsection{Developing a Metamath Representation}

The standard textbook axioms of predicate calculus are somewhat
cumbersome to implement on a computer because of the complex notions of
``free variable''\index{free variable} and ``proper
substitution.''\index{proper substitution}\index{substitution!proper}
While it is possible to use the Metamath\index{Metamath} language to
implement these concepts, we have chosen not to implement them
as primitive constructs in the
\texttt{set.mm} set theory database.  Instead, we have eliminated them
within the axioms
by carefully crafting the axioms so as to avoid them,
building on Tarski's system S2.  This makes it
easy for a beginner to follow the steps in a proof without knowing any
advanced concepts other than the simple concept of
replacing\index{substitution!variable}\index{variable substitution}
variables with expressions.

In order to develop the concepts of free variable and proper
substitution from the axioms, we use an additional
Metamath statement type called ``disjoint variable
restriction''\index{disjoint variables} that we have not encountered
before.  In the context of the axioms, the statement \texttt{\$d} $ x\,
y$\index{\texttt{\$d} statement} simply means that $x$ and $y$ must be
distinct\index{distinct variables}, i.e.\ they may not be simultaneously
substituted\index{substitution!variable}\index{variable substitution}
with the same variable.  The statement \texttt{\$d} $ x\, \varphi$ means
variable $x$ must not occur in wff $\varphi$.  For the precise
definition of \texttt{\$d}, see Section~\ref{dollard}.

\subsubsection{Metamath representation}

The Metamath axiom system for predicate calculus
defined in set.mm uses Tarski's system S2.
As noted above, this has a different representation
than the traditional textbook formalization,
but it is \textit{exactly equivalent} to the textbook formalization,
and it is \textit{much} easier to work with.
This is reproduced as system S3 in Section 6 of
Megill's formalization \cite{Megill}\index{Megill, Norman}.

There is one exception, Tarski's axiom of existence,
which we label as axiom ax-6.
In the case of ax-6, Tarski's version is weaker because it includes a
distinct variable proviso. If we wish, we can also weaken our version
in this way and still have a metalogically complete system. Theorem
ax6 shows this by deriving, in the presence of the other axioms, our
ax-6 from Tarski's weaker version ax6v. However, we chose the stronger
version for our system because it is simpler to state and easier to use.

Tarski's system was designed for proving specific theorems rather than
more general theorem schemes. However, theorem schemes are much more
efficient than specific theorems for building a body of mathematical
knowledge, since they can be reused with different instances as
needed. While Tarski does derive some theorem schemes from his axioms,
their proofs require concepts that are ``outside'' of the system, such as
induction on formula length. The verification of such proofs is difficult
to automate in a proof verifier. (Specifically, Tarski treats the formulas
of his system as set-theoretical objects. In order to verify the proofs
of his theorem schemes, a proof verifier would need a significant amount
of set theory built into it.)

The Metamath axiom system for predicate calculus extends
Tarski's system to eliminate this difficulty. The additional
``auxilliary'' axiom
schemes (as we will call them in this section; see below) endow Tarski's
system with a nice property we call
metalogical completeness \cite[Remark 9.6]{Megill}\index{Megill, Norman}.
As a result, we can prove any theorem scheme
expressable in the ``simple metalogic'' of Tarski's system by using
only Metamath's direct substitution rule applied to the axiom system
(and no other metalogical or set-theoretical notions ``outside'' of the
system). Simple metalogic consists of schemes containing wff metavariables
(with no arguments) and/or set (also called ``individual'') metavariables,
accompanied by optional provisos each stating that two specified set
metavariables must be distinct or that a specified set metavariable may
not occur in a specified wff metavariable. Metamath's logic and set theory
axiom and rule schemes are all examples of simple metalogic. The schemes
of traditional predicate calculus with equality are examples which are
not simple metalogic, because they use wff metavariables with arguments
and have ``free for'' and ``not free in'' side conditions.

A rigorous justification for this system, using an older but
exactly equivalent set of axioms, can be
found in \cite{Megill}\index{Megill, Norman}.

This allows us to
take a different approach in the Metamath\index{Metamath} database
\texttt{set.mm}\index{set theory database (\texttt{set.mm})}.  We do not
directly use the primitive notions of ``free variable''\index{free variable}
and ``proper substitution''\index{proper
substitution}\index{substitution!proper} at all as primitive constructs.
Instead, we use a set
of axioms that are almost as simple to manipulate as those of
propositional calculus.  Our axiom system avoids complex primitive
notions by effectively embedding the complexity into the axioms
themselves.  As a result, we will end up with a larger number of axioms,
but they are ideally suited for a computer language such as Metamath.
(Section~\ref{metaaxioms} shows these axioms.)

We will not elaborate further
on the ``free variable'' and ``proper substitution''
concepts here.  You may consult
\cite[ch.\ 3--4]{Hamilton}\index{Hamilton, Alan G.} (as well as
many other books) for a precise explanation
of these concepts.  If you intend to do serious mathematical work, it is wise
to become familiar with the traditional textbook approach; even though the
concepts embedded in their axioms require a higher level of sophistication,
they can be more practical to deal with on an everyday, informal basis.  Even
if you are just developing Metamath proofs, familiarity with the traditional
approach can help you arrive at a proof outline much faster, which you can
then convert to the detail required by Metamath.

We do develop proper substitution rules later on, but in set.mm
they are defined as derived constructs; they are not primitives.

You should also note that our system of predicate calculus is specifically
tailored for set theory; thus there are only two specific predicates $=$ and
$\in$ and no functions\index{function!in predicate calculus}
or constants\index{constant!in predicate calculus} unlike more general systems.
We later add these.

\subsection{Set Theory}

Traditional Zermelo--Fraenkel set theory\index{Zermelo--Fraenkel set
theory}\index{set theory} with the Axiom of Choice
has 10 axioms, which can be expressed in the
language of predicate calculus.  In this section, we will list only the
names and brief English descriptions of these axioms, since we will give
you the precise formulas used by the Metamath\index{Metamath} set theory
database \texttt{set.mm} later on.

In the descriptions of the axioms, we assume that $x$, $y$, $z$, $w$, and $v$
represent sets.  These are the same as the variables\index{variable!in set
theory} in our predicate calculus system above, except that now we informally
think of the variables as ranging over sets.  Note that the terms
``object,''\index{object} ``set,''\index{set} ``element,''\index{element}
``collection,''\index{collection} and ``family''\index{family} are synonymous,
as are ``is an element of,'' ``is a member of,''\index{member} ``is contained
in,'' and ``belongs to.''  The different terms are used for convenience; for
example, ``a collection of sets'' is less confusing than ``a set of sets.''
A set $x$ is said to be a ``subset''\index{subset} of $y$ if every element of
$x$ is also an element of $y$; we also say $x$ is ``included in''
$y$.

The axioms are very general and apply to almost any conceivable mathematical
object, and this level of abstraction can be overwhelming at first.  To gain an
intuitive feel, it can be helpful to draw a picture illustrating the concept;
for example, a circle containing dots could represent a collection of sets,
and a smaller circle drawn inside the circle could represent a subset.
Overlapping circles can illustrate intersection and union.  Circles that
illustrate the concepts of set theory are frequently used in elementary
textbooks and are called Venn diagrams\index{Venn diagram}.\index{axioms of
set theory}

1. Axiom of Extensionality:  Two sets are identical if they contain the same
   elements.\index{Axiom of Extensionality}

2. Axiom of Pairing:  The set $\{ x , y \}$ exists.\index{Axiom of Pairing}

3. Axiom of Power Sets:  The power set of a set (the collection of all of
   its subsets) exists.  For example, the power set of $\{x,y\}$ is
   $\{\varnothing,\{x\},\{y\},\{x,y\}\}$ and it exists.\index{Axiom
of Power Sets}

4. Axiom of the Null Set:  The empty set $\varnothing$ exists.\index{Axiom of
the Null Set}

5. Axiom of Union:  The union of a set (the set containing the elements of
   its members) exists.  For example, the union of $\{\{x,y\},\{z\}\}$ is
 $\{x,y,z\}$ and
   it exists.\index{Axiom of Union}

6. Axiom of Regularity:  Roughly, no set can contain itself, nor can there
   be membership ``loops,'' such as a set being an
   element of one of its members.\index{Axiom of Regularity}

7. Axiom of Infinity:  An infinite set exists.  An example of an infinite
   set is the set of all
   integers.\index{Axiom of Infinity}

8. Axiom of Separation:  The set exists that is obtained by restricting $x$
   with some property.  For example, if the set of all integers exists,
   then the set of all even integers exists.\index{Axiom of Separation}

9. Axiom of Replacement:  The range of a function whose domain is restricted
   to the elements of a set $x$, is also a set.  For example, there
   is a function
   from integers (the function's domain) to their squares (its
   range).  If we
   restrict the domain to even integers, its range will become the set of
   squares of even integers, so this axiom asserts that the set of
    squares of even numbers exists.  Technical note:  In general, the
   ``function'' need not be a set but can be a proper class.
   \index{Axiom of Replacement}

10. Axiom of Choice:  Let $x$ be a set whose members are pairwise
  disjoint\index{disjoint sets} (i.e,
  whose members contain no elements in common).  Then there exists another
  set containing one element from each member of $x$.  For
  example, if $x$ is
  $\{\{y,z\},\{w,v\}\}$, where $y$, $z$, $w$, and $v$ are
  different sets, then a set such as $\{z,w\}$
  exists (but the axiom doesn't tell
  us which one).  (Actually the Axiom
  of Choice is redundant if the set $x$, as in this example, has a finite
  number of elements.)\index{Axiom of Choice}

The Axiom of Choice is usually considered an extension of ZF set theory rather
than a proper part of it.  It is sometimes considered philosophically
controversial because it specifies the existence of a set without specifying
what the set is. Constructive logics, including intuitionistic logic,
do not accept the axiom of choice.
Since there is some lingering controversy, we often prefer proofs that do
not use the axiom of choice (where there is a known alternative), and
in some cases we will use weaker axioms than the full axiom of choice.
That said, the axiom of choice is a powerful and widely-accepted tool,
so we do use it when needed.
ZF set theory that includes the Axiom of Choice is
called Zermelo--Fraenkel set theory with choice (ZFC\index{ZFC set theory}).

When expressed symbolically, the Axiom of Separation and the Axiom of
Replacement contain wff symbols and therefore each represent infinitely many
axioms, one for each possible wff. For this reason, they are often called
axiom schemes\index{axiom scheme}\index{well-formed formula (wff)}.

It turns out that the Axiom of the Null Set, the Axiom of Pairing, and the
Axiom of Separation can be derived from the other axioms and are therefore
unnecessary, although they tend to be included in standard texts for various
reasons (historical, philosophical, and possibly because some authors may not
know this).  In the Metamath\index{Metamath} set theory database, these
redundant axioms are derived from the other ones instead of truly
being considered axioms.
This is in keeping with our general goal of minimizing the number of
axioms we must depend on.

\subsection{Other Axioms}

Above we qualified the phrase "all of mathematics" with "essentially."
The main important missing piece is the ability to do category theory,
which requires huge sets (inaccessible cardinals) larger than those
postulated by the ZFC axioms. The Tarski--Grothendieck Axiom postulates
the existence of such sets.
Note that this is the same axiom used by Mizar for supporting
category theory.
The Tarski--Grothendieck axiom
can be viewed as a very strong replacement of the Axiom of Infinity,
the Axiom of Choice, and the Axiom of Power Sets.
The \texttt{set.mm} database includes this axiom; see the database
for details about it.
Again, we only use this axiom when we need to.
You are only likely to encounter or use this axiom if you are doing
category theory, since its use is highly specialized,
so we will not list the Tarsky-Grothendieck axiom
in the short list of axioms below.

Can there be even more axioms?
Of course.
G\"{o}del showed that no finite set of axioms or axiom schemes can completely
describe any consistent theory strong enough to include arithmetic.
But practically speaking, the ones above are the accepted foundation that
almost all mathematicians explicitly or implicitly base their work on.

\section{The Axioms in the Metamath Language}\label{metaaxioms}

Here we list the axioms as they appear in
\texttt{set.mm}\index{set theory database (\texttt{set.mm})} so you can
look them up there easily.  Incidentally, the \texttt{show statement
/tex} command\index{\texttt{show statement} command} was used to
typeset them.

%macros from show statement /tex
\newbox\mlinebox
\newbox\mtrialbox
\newbox\startprefix  % Prefix for first line of a formula
\newbox\contprefix  % Prefix for continuation line of a formula
\def\startm{  % Initialize formula line
  \setbox\mlinebox=\hbox{\unhcopy\startprefix}
}
\def\m#1{  % Add a symbol to the formula
  \setbox\mtrialbox=\hbox{\unhcopy\mlinebox $\,#1$}
  \ifdim\wd\mtrialbox>\hsize
    \box\mlinebox
    \setbox\mlinebox=\hbox{\unhcopy\contprefix $\,#1$}
  \else
    \setbox\mlinebox=\hbox{\unhbox\mtrialbox}
  \fi
}
\def\endm{  % Output the last line of a formula
  \box\mlinebox
}

% \SLASH for \ , \TOR for \/ (text OR), \TAND for /\ (text and)
% This embeds a following forced space to force the space.
\newcommand\SLASH{\char`\\~}
\newcommand\TOR{\char`\\/~}
\newcommand\TAND{/\char`\\~}
%
% Macro to output metamath raw text.
% This assumes \startprefix and \contprefix are set.
% NOTE: "\" is tricky to escape, use \SLASH, \TOR, and \TAND inside.
% Any use of "$ { ~ ^" must be escaped; ~ and ^ must be escaped specially.
% We escape { and } for consistency.
% For more about how this macro written, see:
% https://stackoverflow.com/questions/4073674/
% how-to-disable-indentation-in-particular-section-in-latex/4075706
% Use frenchspacing, or "e." will get an extra space after it.
\newlength\mystoreparindent
\newlength\mystorehangindent
\newenvironment{mmraw}{%
\setlength{\mystoreparindent}{\the\parindent}
\setlength{\mystorehangindent}{\the\hangindent}
\setlength{\parindent}{0pt} % TODO - we'll put in the \startprefix instead
\setlength{\hangindent}{\wd\the\contprefix}
\begin{flushleft}
\begin{frenchspacing}
\begin{tt}
{\unhcopy\startprefix}%
}{%
\end{tt}
\end{frenchspacing}
\end{flushleft}
\setlength{\parindent}{\mystoreparindent}
\setlength{\hangindent}{\mystorehangindent}
\vskip 1ex
}

\needspace{5\baselineskip}
\subsection{Propositional Calculus}\label{propcalc}\index{axioms of
propositional calculus}

\needspace{2\baselineskip}
Axiom of Simplification.\label{ax1}

\setbox\startprefix=\hbox{\tt \ \ ax-1\ \$a\ }
\setbox\contprefix=\hbox{\tt \ \ \ \ \ \ \ \ \ \ }
\startm
\m{\vdash}\m{(}\m{\varphi}\m{\rightarrow}\m{(}\m{\psi}\m{\rightarrow}\m{\varphi}\m{)}
\m{)}
\endm

\needspace{3\baselineskip}
\noindent Axiom of Distribution.

\setbox\startprefix=\hbox{\tt \ \ ax-2\ \$a\ }
\setbox\contprefix=\hbox{\tt \ \ \ \ \ \ \ \ \ \ }
\startm
\m{\vdash}\m{(}\m{(}\m{\varphi}\m{\rightarrow}\m{(}\m{\psi}\m{\rightarrow}\m{\chi}
\m{)}\m{)}\m{\rightarrow}\m{(}\m{(}\m{\varphi}\m{\rightarrow}\m{\psi}\m{)}\m{
\rightarrow}\m{(}\m{\varphi}\m{\rightarrow}\m{\chi}\m{)}\m{)}\m{)}
\endm

\needspace{2\baselineskip}
\noindent Axiom of Contraposition.

\setbox\startprefix=\hbox{\tt \ \ ax-3\ \$a\ }
\setbox\contprefix=\hbox{\tt \ \ \ \ \ \ \ \ \ \ }
\startm
\m{\vdash}\m{(}\m{(}\m{\lnot}\m{\varphi}\m{\rightarrow}\m{\lnot}\m{\psi}\m{)}\m{
\rightarrow}\m{(}\m{\psi}\m{\rightarrow}\m{\varphi}\m{)}\m{)}
\endm


\needspace{4\baselineskip}
\noindent Rule of Modus Ponens.\label{axmp}\index{modus ponens}

\setbox\startprefix=\hbox{\tt \ \ maj\ \$e\ }
\setbox\contprefix=\hbox{\tt \ \ \ \ \ \ \ \ \ }
\startm
\m{\vdash}\m{(}\m{\varphi}\m{\rightarrow}\m{\psi}\m{)}
\endm

\setbox\startprefix=\hbox{\tt \ \ min\ \$e\ }
\setbox\contprefix=\hbox{\tt \ \ \ \ \ \ \ \ \ }
\startm
\m{\vdash}\m{\varphi}
\endm

\setbox\startprefix=\hbox{\tt \ \ ax-mp\ \$a\ }
\setbox\contprefix=\hbox{\tt \ \ \ \ \ \ \ \ \ \ \ }
\startm
\m{\vdash}\m{\psi}
\endm


\needspace{7\baselineskip}
\subsection{Axioms of Predicate Calculus with Equality---Tarski's S2}\index{axioms of predicate calculus}

\needspace{3\baselineskip}
\noindent Rule of Generalization.\index{rule of generalization}

\setbox\startprefix=\hbox{\tt \ \ ax-g.1\ \$e\ }
\setbox\contprefix=\hbox{\tt \ \ \ \ \ \ \ \ \ \ \ \ }
\startm
\m{\vdash}\m{\varphi}
\endm

\setbox\startprefix=\hbox{\tt \ \ ax-gen\ \$a\ }
\setbox\contprefix=\hbox{\tt \ \ \ \ \ \ \ \ \ \ \ \ }
\startm
\m{\vdash}\m{\forall}\m{x}\m{\varphi}
\endm

\needspace{2\baselineskip}
\noindent Axiom of Quantified Implication.

\setbox\startprefix=\hbox{\tt \ \ ax-4\ \$a\ }
\setbox\contprefix=\hbox{\tt \ \ \ \ \ \ \ \ \ \ }
\startm
\m{\vdash}\m{(}\m{\forall}\m{x}\m{(}\m{\forall}\m{x}\m{\varphi}\m{\rightarrow}\m{
\psi}\m{)}\m{\rightarrow}\m{(}\m{\forall}\m{x}\m{\varphi}\m{\rightarrow}\m{
\forall}\m{x}\m{\psi}\m{)}\m{)}
\endm

\needspace{3\baselineskip}
\noindent Axiom of Distinctness.

% Aka: Add $d x ph $.
\setbox\startprefix=\hbox{\tt \ \ ax-5\ \$a\ }
\setbox\contprefix=\hbox{\tt \ \ \ \ \ \ \ \ \ \ }
\startm
\m{\vdash}\m{(}\m{\varphi}\m{\rightarrow}\m{\forall}\m{x}\m{\varphi}\m{)}\m{where}\m{ }\m{\$d}\m{ }\m{x}\m{ }\m{\varphi}\m{ }\m{(}\m{x}\m{ }\m{does}\m{ }\m{not}\m{ }\m{occur}\m{ }\m{in}\m{ }\m{\varphi}\m{)}
\endm

\needspace{2\baselineskip}
\noindent Axiom of Existence.

\setbox\startprefix=\hbox{\tt \ \ ax-6\ \$a\ }
\setbox\contprefix=\hbox{\tt \ \ \ \ \ \ \ \ \ \ }
\startm
\m{\vdash}\m{(}\m{\forall}\m{x}\m{(}\m{x}\m{=}\m{y}\m{\rightarrow}\m{\forall}
\m{x}\m{\varphi}\m{)}\m{\rightarrow}\m{\varphi}\m{)}
\endm

\needspace{2\baselineskip}
\noindent Axiom of Equality.

\setbox\startprefix=\hbox{\tt \ \ ax-7\ \$a\ }
\setbox\contprefix=\hbox{\tt \ \ \ \ \ \ \ \ \ \ }
\startm
\m{\vdash}\m{(}\m{x}\m{=}\m{y}\m{\rightarrow}\m{(}\m{x}\m{=}\m{z}\m{
\rightarrow}\m{y}\m{=}\m{z}\m{)}\m{)}
\endm

\needspace{2\baselineskip}
\noindent Axiom of Left Equality for Binary Predicate.

\setbox\startprefix=\hbox{\tt \ \ ax-8\ \$a\ }
\setbox\contprefix=\hbox{\tt \ \ \ \ \ \ \ \ \ \ \ }
\startm
\m{\vdash}\m{(}\m{x}\m{=}\m{y}\m{\rightarrow}\m{(}\m{x}\m{\in}\m{z}\m{
\rightarrow}\m{y}\m{\in}\m{z}\m{)}\m{)}
\endm

\needspace{2\baselineskip}
\noindent Axiom of Right Equality for Binary Predicate.

\setbox\startprefix=\hbox{\tt \ \ ax-9\ \$a\ }
\setbox\contprefix=\hbox{\tt \ \ \ \ \ \ \ \ \ \ \ }
\startm
\m{\vdash}\m{(}\m{x}\m{=}\m{y}\m{\rightarrow}\m{(}\m{z}\m{\in}\m{x}\m{
\rightarrow}\m{z}\m{\in}\m{y}\m{)}\m{)}
\endm


\needspace{4\baselineskip}
\subsection{Axioms of Predicate Calculus with Equality---Auxiliary}\index{axioms of predicate calculus - auxiliary}

\needspace{2\baselineskip}
\noindent Axiom of Quantified Negation.

\setbox\startprefix=\hbox{\tt \ \ ax-10\ \$a\ }
\setbox\contprefix=\hbox{\tt \ \ \ \ \ \ \ \ \ \ }
\startm
\m{\vdash}\m{(}\m{\lnot}\m{\forall}\m{x}\m{\lnot}\m{\forall}\m{x}\m{\varphi}\m{
\rightarrow}\m{\varphi}\m{)}
\endm

\needspace{2\baselineskip}
\noindent Axiom of Quantifier Commutation.

\setbox\startprefix=\hbox{\tt \ \ ax-11\ \$a\ }
\setbox\contprefix=\hbox{\tt \ \ \ \ \ \ \ \ \ \ }
\startm
\m{\vdash}\m{(}\m{\forall}\m{x}\m{\forall}\m{y}\m{\varphi}\m{\rightarrow}\m{
\forall}\m{y}\m{\forall}\m{x}\m{\varphi}\m{)}
\endm

\needspace{3\baselineskip}
\noindent Axiom of Substitution.

\setbox\startprefix=\hbox{\tt \ \ ax-12\ \$a\ }
\setbox\contprefix=\hbox{\tt \ \ \ \ \ \ \ \ \ \ \ }
\startm
\m{\vdash}\m{(}\m{\lnot}\m{\forall}\m{x}\m{\,x}\m{=}\m{y}\m{\rightarrow}\m{(}
\m{x}\m{=}\m{y}\m{\rightarrow}\m{(}\m{\varphi}\m{\rightarrow}\m{\forall}\m{x}\m{(}
\m{x}\m{=}\m{y}\m{\rightarrow}\m{\varphi}\m{)}\m{)}\m{)}\m{)}
\endm

\needspace{3\baselineskip}
\noindent Axiom of Quantified Equality.

\setbox\startprefix=\hbox{\tt \ \ ax-13\ \$a\ }
\setbox\contprefix=\hbox{\tt \ \ \ \ \ \ \ \ \ \ \ }
\startm
\m{\vdash}\m{(}\m{\lnot}\m{\forall}\m{z}\m{\,z}\m{=}\m{x}\m{\rightarrow}\m{(}
\m{\lnot}\m{\forall}\m{z}\m{\,z}\m{=}\m{y}\m{\rightarrow}\m{(}\m{x}\m{=}\m{y}
\m{\rightarrow}\m{\forall}\m{z}\m{\,x}\m{=}\m{y}\m{)}\m{)}\m{)}
\endm

% \noindent Axiom of Quantifier Substitution
%
% \setbox\startprefix=\hbox{\tt \ \ ax-c11n\ \$a\ }
% \setbox\contprefix=\hbox{\tt \ \ \ \ \ \ \ \ \ \ \ }
% \startm
% \m{\vdash}\m{(}\m{\forall}\m{x}\m{\,x}\m{=}\m{y}\m{\rightarrow}\m{(}\m{\forall}
% \m{x}\m{\varphi}\m{\rightarrow}\m{\forall}\m{y}\m{\varphi}\m{)}\m{)}
% \endm
%
% \noindent Axiom of Distinct Variables. (This axiom requires
% that two individual variables
% be distinct\index{\texttt{\$d} statement}\index{distinct
% variables}.)
%
% \setbox\startprefix=\hbox{\tt \ \ \ \ \ \ \ \ \$d\ }
% \setbox\contprefix=\hbox{\tt \ \ \ \ \ \ \ \ \ \ \ }
% \startm
% \m{x}\m{\,}\m{y}
% \endm
%
% \setbox\startprefix=\hbox{\tt \ \ ax-c16\ \$a\ }
% \setbox\contprefix=\hbox{\tt \ \ \ \ \ \ \ \ \ \ \ }
% \startm
% \m{\vdash}\m{(}\m{\forall}\m{x}\m{\,x}\m{=}\m{y}\m{\rightarrow}\m{(}\m{\varphi}\m{
% \rightarrow}\m{\forall}\m{x}\m{\varphi}\m{)}\m{)}
% \endm

% \noindent Axiom of Quantifier Introduction (2).  (This axiom requires
% that the individual variable not occur in the
% wff\index{\texttt{\$d} statement}\index{distinct variables}.)
%
% \setbox\startprefix=\hbox{\tt \ \ \ \ \ \ \ \ \$d\ }
% \setbox\contprefix=\hbox{\tt \ \ \ \ \ \ \ \ \ \ \ }
% \startm
% \m{x}\m{\,}\m{\varphi}
% \endm
% \setbox\startprefix=\hbox{\tt \ \ ax-5\ \$a\ }
% \setbox\contprefix=\hbox{\tt \ \ \ \ \ \ \ \ \ \ \ }
% \startm
% \m{\vdash}\m{(}\m{\varphi}\m{\rightarrow}\m{\forall}\m{x}\m{\varphi}\m{)}
% \endm

\subsection{Set Theory}\label{mmsettheoryaxioms}

In order to make the axioms of set theory\index{axioms of set theory} a little
more compact, there are several definitions from logic that we make use of
implicitly, namely, ``logical {\sc and},''\index{conjunction ($\wedge$)}
\index{logical {\sc and} ($\wedge$)} ``logical equivalence,''\index{logical
equivalence ($\leftrightarrow$)}\index{biconditional ($\leftrightarrow$)} and
``there exists.''\index{existential quantifier ($\exists$)}

\begin{center}\begin{tabular}{rcl}
  $( \varphi \wedge \psi )$ &\mbox{stands for}& $\neg ( \varphi
     \rightarrow \neg \psi )$\\
  $( \varphi \leftrightarrow \psi )$& \mbox{stands
     for}& $( ( \varphi \rightarrow \psi ) \wedge
     ( \psi \rightarrow \varphi ) )$\\
  $\exists x \,\varphi$ &\mbox{stands for}& $\neg \forall x \neg \varphi$
\end{tabular}\end{center}

In addition, the axioms of set theory require that all variables be
dis\-tinct,\index{distinct variables}\footnote{Set theory axioms can be
devised so that {\em no} variables are required to be distinct,
provided we replace \texttt{ax-c16} with an axiom stating that ``at
least two things exist,'' thus
making \texttt{ax-5} the only other axiom requiring the
\texttt{\$d} statement.  These axioms are unconventional and are not
presented here, but they can be found on the \url{http://metamath.org}
web site.  See also the Comment on
p.~\pageref{nodd}.}\index{\texttt{\$d} statement} thus we also assume:
\begin{center}
  \texttt{\$d }$x\,y\,z\,w$
\end{center}

\needspace{2\baselineskip}
\noindent Axiom of Extensionality.\index{Axiom of Extensionality}

\setbox\startprefix=\hbox{\tt \ \ ax-ext\ \$a\ }
\setbox\contprefix=\hbox{\tt \ \ \ \ \ \ \ \ \ \ \ \ }
\startm
\m{\vdash}\m{(}\m{\forall}\m{x}\m{(}\m{x}\m{\in}\m{y}\m{\leftrightarrow}\m{x}
\m{\in}\m{z}\m{)}\m{\rightarrow}\m{y}\m{=}\m{z}\m{)}
\endm

\needspace{3\baselineskip}
\noindent Axiom of Replacement.\index{Axiom of Replacement}

\setbox\startprefix=\hbox{\tt \ \ ax-rep\ \$a\ }
\setbox\contprefix=\hbox{\tt \ \ \ \ \ \ \ \ \ \ \ \ }
\startm
\m{\vdash}\m{(}\m{\forall}\m{w}\m{\exists}\m{y}\m{\forall}\m{z}\m{(}\m{%
\forall}\m{y}\m{\varphi}\m{\rightarrow}\m{z}\m{=}\m{y}\m{)}\m{\rightarrow}\m{%
\exists}\m{y}\m{\forall}\m{z}\m{(}\m{z}\m{\in}\m{y}\m{\leftrightarrow}\m{%
\exists}\m{w}\m{(}\m{w}\m{\in}\m{x}\m{\wedge}\m{\forall}\m{y}\m{\varphi}\m{)}%
\m{)}\m{)}
\endm

\needspace{2\baselineskip}
\noindent Axiom of Union.\index{Axiom of Union}

\setbox\startprefix=\hbox{\tt \ \ ax-un\ \$a\ }
\setbox\contprefix=\hbox{\tt \ \ \ \ \ \ \ \ \ \ \ }
\startm
\m{\vdash}\m{\exists}\m{x}\m{\forall}\m{y}\m{(}\m{\exists}\m{x}\m{(}\m{y}\m{
\in}\m{x}\m{\wedge}\m{x}\m{\in}\m{z}\m{)}\m{\rightarrow}\m{y}\m{\in}\m{x}\m{)}
\endm

\needspace{2\baselineskip}
\noindent Axiom of Power Sets.\index{Axiom of Power Sets}

\setbox\startprefix=\hbox{\tt \ \ ax-pow\ \$a\ }
\setbox\contprefix=\hbox{\tt \ \ \ \ \ \ \ \ \ \ \ \ }
\startm
\m{\vdash}\m{\exists}\m{x}\m{\forall}\m{y}\m{(}\m{\forall}\m{x}\m{(}\m{x}\m{
\in}\m{y}\m{\rightarrow}\m{x}\m{\in}\m{z}\m{)}\m{\rightarrow}\m{y}\m{\in}\m{x}
\m{)}
\endm

\needspace{3\baselineskip}
\noindent Axiom of Regularity.\index{Axiom of Regularity}

\setbox\startprefix=\hbox{\tt \ \ ax-reg\ \$a\ }
\setbox\contprefix=\hbox{\tt \ \ \ \ \ \ \ \ \ \ \ \ }
\startm
\m{\vdash}\m{(}\m{\exists}\m{x}\m{\,x}\m{\in}\m{y}\m{\rightarrow}\m{\exists}
\m{x}\m{(}\m{x}\m{\in}\m{y}\m{\wedge}\m{\forall}\m{z}\m{(}\m{z}\m{\in}\m{x}\m{
\rightarrow}\m{\lnot}\m{z}\m{\in}\m{y}\m{)}\m{)}\m{)}
\endm

\needspace{3\baselineskip}
\noindent Axiom of Infinity.\index{Axiom of Infinity}

\setbox\startprefix=\hbox{\tt \ \ ax-inf\ \$a\ }
\setbox\contprefix=\hbox{\tt \ \ \ \ \ \ \ \ \ \ \ \ \ \ \ }
\startm
\m{\vdash}\m{\exists}\m{x}\m{(}\m{y}\m{\in}\m{x}\m{\wedge}\m{\forall}\m{y}%
\m{(}\m{y}\m{\in}\m{x}\m{\rightarrow}\m{\exists}\m{z}\m{(}\m{y}\m{\in}\m{z}\m{%
\wedge}\m{z}\m{\in}\m{x}\m{)}\m{)}\m{)}
\endm

\needspace{4\baselineskip}
\noindent Axiom of Choice.\index{Axiom of Choice}

\setbox\startprefix=\hbox{\tt \ \ ax-ac\ \$a\ }
\setbox\contprefix=\hbox{\tt \ \ \ \ \ \ \ \ \ \ \ \ \ \ }
\startm
\m{\vdash}\m{\exists}\m{x}\m{\forall}\m{y}\m{\forall}\m{z}\m{(}\m{(}\m{y}\m{%
\in}\m{z}\m{\wedge}\m{z}\m{\in}\m{w}\m{)}\m{\rightarrow}\m{\exists}\m{w}\m{%
\forall}\m{y}\m{(}\m{\exists}\m{w}\m{(}\m{(}\m{y}\m{\in}\m{z}\m{\wedge}\m{z}%
\m{\in}\m{w}\m{)}\m{\wedge}\m{(}\m{y}\m{\in}\m{w}\m{\wedge}\m{w}\m{\in}\m{x}%
\m{)}\m{)}\m{\leftrightarrow}\m{y}\m{=}\m{w}\m{)}\m{)}
\endm

\subsection{That's It}

There you have it, the axioms for (essentially) all of mathematics!
Wonder at them and stare at them in awe.  Put a copy in your wallet, and
you will carry in your pocket the encoding for all theorems ever proved
and that ever will be proved, from the most mundane to the most
profound.

\section{A Hierarchy of Definitions}\label{hierarchy}

The axioms in the previous section in principle embody everything that can be
done within standard mathematics.  However, it is impractical to accomplish
very much by using them directly, for even simple concepts (from a human
perspective) can involve extremely long, incomprehensible formulas.
Mathematics is made practical by introducing definitions\index{definition}.
Definitions usually introduce new symbols, or at least new relationships among
existing symbols, to abbreviate more complex formulas.  An important
requirement for a definition is that there exist a straightforward
(algorithmic) method for eliminating the abbreviation by expanding it into the
more primitive symbol string that it represents.  Some
important definitions included in
the file \texttt{set.mm} are listed in this section for reference, and also to
give you a feel for why something like $\omega$\index{omega ($\omega$)} (the
set of natural numbers\index{natural number} 0, 1, 2,\ldots) becomes very
complicated when completely expanded into primitive symbols.

What is the motivation for definitions, aside from allowing complicated
expressions to be expressed more simply?  In the case of  $\omega$, one goal is
to provide a basis for the theory of natural numbers.\index{natural number}
Before set theory was invented, a set of axioms for arithmetic, called Peano's
postulates\index{Peano's postulates}, was devised and shown to have the
properties one expects for natural numbers.  Now anyone can postulate a
set of axioms, but if the axioms are inconsistent contradictions can be derived
from them.  Once a contradiction is derived, anything can be trivially
proved, including
all the facts of arithmetic and their negations.  To ensure that an
axiom system is at least as reliable as the axioms for set theory, we can
define sets and operations on those sets that satisfy the new axioms. In the
\texttt{set.mm} Metamath database, we prove that the elements of $\omega$ satisfy
Peano's postulates, and it's a long and hard journey to get there directly
from the axioms of set theory.  But the result is confidence in the
foundations of arithmetic.  And there is another advantage:  we now have all
the tools of set theory at our disposal for manipulating objects that obey the
axioms for arithmetic.

What are the criteria we use for definitions?  First, and of utmost importance,
the definition should not be {\em creative}\index{creative
definition}\index{definition!creative}, that
is it should not allow an expression that previously qualified as a wff but
was not provable, to become provable.   Second, the definition should be {\em
eliminable}\index{definition!eliminability}, that is, there should exist an
algorithmic method for converting any expression using the definition into
a logically equivalent expression that previously qualified as a wff.

In almost all cases below, definitions connect two expressions with either
$\leftrightarrow$ or $=$.  Eliminating\footnote{Here we mean the
elimination that a human might do in his or her head.  To eliminate them as
part of a Metamath proof we would invoke one of a number of
theorems that deal with transitivity of equivalence or equality; there are
many such examples in the proofs in \texttt{set.mm}.} such a definition is a
simple matter of substituting the expression on the left-hand side ({\em
definiendum}\index{definiendum} or thing being defined) with the equivalent,
more primitive expression on the right-hand side ({\em
definiens}\index{definiens} or definition).

Often a definition has variables on the right-hand side which do not appear on
the left-hand side; these are called {\em dummy variables}.\index{dummy
variable!in definitions}  In this case, any
allowable substitution (such as a new, distinct
variable) can be used when the definition is eliminated.  Dummy variables may
be used only if they are {\em effectively bound}\index{effectively bound
variable}, meaning that the definition will remain logically equivalent upon
any substitution of a dummy variable with any other {\em qualifying
expression}\index{qualifying expression}, i.e.\ any symbol string (such as
another variable) that
meets the restrictions on the dummy variable imposed by \texttt{\$d} and
\texttt{\$f} statements.  For example, we could define a constant $\perp$
(inverted tee, meaning logical ``false'') as $( \varphi \wedge \lnot \varphi
)$, i.e.\ ``phi and not phi.''  Here $\varphi$ is effectively bound because the
definition remains logically equivalent when we replace $\varphi$ with any
other wff.  (It is actually \texttt{df-fal}
in \texttt{set.mm}, which defines $\perp$.)

There are two cases where eliminating definitions is a little more
complex.  These cases are the definitions \texttt{df-bi} and
\texttt{df-cleq}.  The first stretches the concept of a definition a
little, as in effect it ``defines a definition;'' however, it meets our
requirements for a definition in that it is eliminable and does not
strengthen the language.  Theorem \texttt{bii} shows the substitution
needed to eliminate the $\leftrightarrow$\index{logical equivalence
($\leftrightarrow$)}\index{biconditional ($\leftrightarrow$)} symbol.

Definition \texttt{df-cleq}\index{equality ($=$)} extends the usage of
the equality symbol to include ``classes''\index{class} in set theory.  The
reason it is potentially problematic is that it can lead to statements which
do not follow from logic alone but presuppose the Axiom of
Extensionality\index{Axiom of Extensionality}, so we include this axiom
as a hypothesis for the definition.  We could have made \texttt{df-cleq} directly
eliminable by introducing a new equality symbol, but have chosen not to do so
in keeping with standard textbook practice.  Definitions such as \texttt{df-cleq}
that extend the meaning of existing symbols must be introduced carefully so
that they do not lead to contradictions.  Definition \texttt{df-clel} also
extends the meaning of an existing symbol ($\in$); while it doesn't strengthen
the language like \texttt{df-cleq}, this is not obvious and it must also be
subject to the same scrutiny.

Exercise:  Study how the wff $x\in\omega$, meaning ``$x$ is a natural
number,'' could be expanded in terms of primitive symbols, starting with the
definitions \texttt{df-clel} on p.~\pageref{dfclel} and \texttt{df-om} on
p.~\pageref{dfom} and working your way back.  Don't bother to work out the
details; just make sure that you understand how you could do it in principle.
The answer is shown in the footnote on p.~\pageref{expandom}.  If you
actually do work it out, you won't get exactly the same answer because we used
a few simplifications such as discarding occurrences of $\lnot\lnot$ (double
negation).

In the definitions below, we have placed the {\sc ascii} Metamath source
below each of the formulas to help you become familiar with the
notation in the database.  For simplicity, the necessary \texttt{\$f}
and \texttt{\$d} statements are not shown.  If you are in doubt, use the
\texttt{show statement}\index{\texttt{show statement} command} command
in the Metamath program to see the full statement.
A selection of this notation is summarized in Appendix~\ref{ASCII}.

To understand the motivation for these definitions, you should consult the
references indicated:  Takeuti and Zaring \cite{Takeuti}\index{Takeuti, Gaisi},
Quine \cite{Quine}\index{Quine, Willard Van Orman}, Bell and Machover
\cite{Bell}\index{Bell, J. L.}, and Enderton \cite{Enderton}\index{Enderton,
Herbert B.}.  Our list of definitions is provided more for reference than as a
learning aid.  However, by looking at a few of them you can gain a feel for
how the hierarchy is built up.  The definitions are a representative sample of
the 130 or so in \texttt{set.mm}, but they are complete with respect to the
theorem examples we will present in Section~\ref{sometheorems}.  Also, some are
slightly different from, but logically equivalent to, the ones in \texttt{set.mm}
(some of which have been revised over time to shorten them, for example).

\subsection{Definitions for Propositional Calculus}\label{metadefprop}

The symbols $\varphi$, $\psi$, and $\chi$ represent wffs.

Our first definition introduces the biconditional
connective\footnote{The term ``connective'' is informally used to mean a
symbol that is placed between two variables or adjacent to a variable,
whereas a mathematical ``constant'' usually indicates a symbol such as
the number 0 that may replace a variable or metavariable.  From
Metamath's point of view, there is no distinction between a connective
and a constant; both are constants in the Metamath
language.}\index{connective}\index{constant} (also called logical
equivalence)\index{logical equivalence
($\leftrightarrow$)}\index{biconditional ($\leftrightarrow$)}.  Unlike
most traditional developments, we have chosen not to have a separate
symbol such as ``Df.'' to mean ``is defined as.''  Instead, we will use
the biconditional connective for this purpose, as it lets us use
logic to manipulate definitions directly.  Here we state the properties
of the biconditional connective with a carefully crafted \texttt{\$a}
statement, which effectively uses the biconditional connective to define
itself.  The $\leftrightarrow$ symbol can be eliminated from a formula
using theorem \texttt{bii}, which is derived later.

\vskip 2ex
\noindent Define the biconditional connective.\label{df-bi}

\vskip 0.5ex
\setbox\startprefix=\hbox{\tt \ \ df-bi\ \$a\ }
\setbox\contprefix=\hbox{\tt \ \ \ \ \ \ \ \ \ \ \ }
\startm
\m{\vdash}\m{\lnot}\m{(}\m{(}\m{(}\m{\varphi}\m{\leftrightarrow}\m{\psi}\m{)}%
\m{\rightarrow}\m{\lnot}\m{(}\m{(}\m{\varphi}\m{\rightarrow}\m{\psi}\m{)}\m{%
\rightarrow}\m{\lnot}\m{(}\m{\psi}\m{\rightarrow}\m{\varphi}\m{)}\m{)}\m{)}\m{%
\rightarrow}\m{\lnot}\m{(}\m{\lnot}\m{(}\m{(}\m{\varphi}\m{\rightarrow}\m{%
\psi}\m{)}\m{\rightarrow}\m{\lnot}\m{(}\m{\psi}\m{\rightarrow}\m{\varphi}\m{)}%
\m{)}\m{\rightarrow}\m{(}\m{\varphi}\m{\leftrightarrow}\m{\psi}\m{)}\m{)}\m{)}
\endm
\begin{mmraw}%
|- -. ( ( ( ph <-> ps ) -> -. ( ( ph -> ps ) ->
-. ( ps -> ph ) ) ) -> -. ( -. ( ( ph -> ps ) -> -. (
ps -> ph ) ) -> ( ph <-> ps ) ) ) \$.
\end{mmraw}

\noindent This theorem relates the biconditional connective to primitive
connectives and can be used to eliminate the $\leftrightarrow$ symbol from any
wff.

\vskip 0.5ex
\setbox\startprefix=\hbox{\tt \ \ bii\ \$p\ }
\setbox\contprefix=\hbox{\tt \ \ \ \ \ \ \ \ \ }
\startm
\m{\vdash}\m{(}\m{(}\m{\varphi}\m{\leftrightarrow}\m{\psi}\m{)}\m{\leftrightarrow}
\m{\lnot}\m{(}\m{(}\m{\varphi}\m{\rightarrow}\m{\psi}\m{)}\m{\rightarrow}\m{\lnot}
\m{(}\m{\psi}\m{\rightarrow}\m{\varphi}\m{)}\m{)}\m{)}
\endm
\begin{mmraw}%
|- ( ( ph <-> ps ) <-> -. ( ( ph -> ps ) -> -. ( ps -> ph ) ) ) \$= ... \$.
\end{mmraw}

\noindent Define disjunction ({\sc or}).\index{disjunction ($\vee$)}%
\index{logical or (vee)@logical {\sc or} ($\vee$)}%
\index{df-or@\texttt{df-or}}\label{df-or}

\vskip 0.5ex
\setbox\startprefix=\hbox{\tt \ \ df-or\ \$a\ }
\setbox\contprefix=\hbox{\tt \ \ \ \ \ \ \ \ \ \ \ }
\startm
\m{\vdash}\m{(}\m{(}\m{\varphi}\m{\vee}\m{\psi}\m{)}\m{\leftrightarrow}\m{(}\m{
\lnot}\m{\varphi}\m{\rightarrow}\m{\psi}\m{)}\m{)}
\endm
\begin{mmraw}%
|- ( ( ph \TOR ps ) <-> ( -. ph -> ps ) ) \$.
\end{mmraw}

\noindent Define conjunction ({\sc and}).\index{conjunction ($\wedge$)}%
\index{logical {\sc and} ($\wedge$)}%
\index{df-an@\texttt{df-an}}\label{df-an}

\vskip 0.5ex
\setbox\startprefix=\hbox{\tt \ \ df-an\ \$a\ }
\setbox\contprefix=\hbox{\tt \ \ \ \ \ \ \ \ \ \ \ }
\startm
\m{\vdash}\m{(}\m{(}\m{\varphi}\m{\wedge}\m{\psi}\m{)}\m{\leftrightarrow}\m{\lnot}
\m{(}\m{\varphi}\m{\rightarrow}\m{\lnot}\m{\psi}\m{)}\m{)}
\endm
\begin{mmraw}%
|- ( ( ph \TAND ps ) <-> -. ( ph -> -. ps ) ) \$.
\end{mmraw}

\noindent Define disjunction ({\sc or}) of 3 wffs.%
\index{df-3or@\texttt{df-3or}}\label{df-3or}

\vskip 0.5ex
\setbox\startprefix=\hbox{\tt \ \ df-3or\ \$a\ }
\setbox\contprefix=\hbox{\tt \ \ \ \ \ \ \ \ \ \ \ \ }
\startm
\m{\vdash}\m{(}\m{(}\m{\varphi}\m{\vee}\m{\psi}\m{\vee}\m{\chi}\m{)}\m{
\leftrightarrow}\m{(}\m{(}\m{\varphi}\m{\vee}\m{\psi}\m{)}\m{\vee}\m{\chi}\m{)}
\m{)}
\endm
\begin{mmraw}%
|- ( ( ph \TOR ps \TOR ch ) <-> ( ( ph \TOR ps ) \TOR ch ) ) \$.
\end{mmraw}

\noindent Define conjunction ({\sc and}) of 3 wffs.%
\index{df-3an}\label{df-3an}

\vskip 0.5ex
\setbox\startprefix=\hbox{\tt \ \ df-3an\ \$a\ }
\setbox\contprefix=\hbox{\tt \ \ \ \ \ \ \ \ \ \ \ \ }
\startm
\m{\vdash}\m{(}\m{(}\m{\varphi}\m{\wedge}\m{\psi}\m{\wedge}\m{\chi}\m{)}\m{
\leftrightarrow}\m{(}\m{(}\m{\varphi}\m{\wedge}\m{\psi}\m{)}\m{\wedge}\m{\chi}
\m{)}\m{)}
\endm

\begin{mmraw}%
|- ( ( ph \TAND ps \TAND ch ) <-> ( ( ph \TAND ps ) \TAND ch ) ) \$.
\end{mmraw}

\subsection{Definitions for Predicate Calculus}\label{metadefpred}

The symbols $x$, $y$, and $z$ represent individual variables of predicate
calculus.  In this section, they are not necessarily distinct unless it is
explicitly
mentioned.

\vskip 2ex
\noindent Define existential quantification.
The expression $\exists x \varphi$ means
``there exists an $x$ where $\varphi$ is true.''\index{existential quantifier
($\exists$)}\label{df-ex}

\vskip 0.5ex
\setbox\startprefix=\hbox{\tt \ \ df-ex\ \$a\ }
\setbox\contprefix=\hbox{\tt \ \ \ \ \ \ \ \ \ \ \ }
\startm
\m{\vdash}\m{(}\m{\exists}\m{x}\m{\varphi}\m{\leftrightarrow}\m{\lnot}\m{\forall}
\m{x}\m{\lnot}\m{\varphi}\m{)}
\endm
\begin{mmraw}%
|- ( E. x ph <-> -. A. x -. ph ) \$.
\end{mmraw}

\noindent Define proper substitution.\index{proper
substitution}\index{substitution!proper}\label{df-sb}
In our notation, we use $[ y / x ] \varphi$ to mean ``the wff that
results when $y$ is properly substituted for $x$ in the wff
$\varphi$.''\footnote{
This can also be described
as substituting $x$ with $y$, $y$ properly replaces $x$, or
$x$ is properly replaced by $y$.}
% This is elsb4, though it currently says: ( [ x / y ] z e. y <-> z e. x )
For example,
$[ y / x ] z \in x$ is the same as $z \in y$.
One way to remember this notation is to notice that it looks like division
and recall that $( y / x ) \cdot x $ is $y$ (when $x \neq 0$).
The notation is different from the notation $\varphi ( x | y )$
that is sometimes used, because the latter notation is ambiguous for us:
for example, we don't know whether $\lnot \varphi ( x | y )$ is to be
interpreted as $\lnot ( \varphi ( x | y ) )$ or
$( \lnot \varphi ) ( x | y )$.\footnote{Because of the way
we initially defined wffs, this is the case
with any postfix connective\index{postfix connective} (one occurring after the
symbols being connected) or infix connective\index{infix connective} (one
occurring between the symbols being connected).  Metamath does not have a
built-in notion of operator binding strength that could eliminate the
ambiguity.  The initial parenthesis effectively provides a prefix
connective\index{prefix connective} to eliminate ambiguity.  Some conventions,
such as Polish notation\index{Polish notation} used in the 1930's and 1940's
by Polish logicians, use only prefix connectives and thus allow the total
elimination of parentheses, at the expense of readability.  In Metamath we
could actually redefine all notation to be Polish if we wanted to without
having to change any proofs!}  Other texts often use $\varphi(y)$ to indicate
our $[ y / x ] \varphi$, but this notation is even more ambiguous since there is
no explicit indication of what is being substituted.
Note that this
definition is valid even when
$x$ and $y$ are the same variable.  The first conjunct is a ``trick'' used to
achieve this property, making the definition look somewhat peculiar at
first.

\vskip 0.5ex
\setbox\startprefix=\hbox{\tt \ \ df-sb\ \$a\ }
\setbox\contprefix=\hbox{\tt \ \ \ \ \ \ \ \ \ \ \ }
\startm
\m{\vdash}\m{(}\m{[}\m{y}\m{/}\m{x}\m{]}\m{\varphi}\m{\leftrightarrow}\m{(}%
\m{(}\m{x}\m{=}\m{y}\m{\rightarrow}\m{\varphi}\m{)}\m{\wedge}\m{\exists}\m{x}%
\m{(}\m{x}\m{=}\m{y}\m{\wedge}\m{\varphi}\m{)}\m{)}\m{)}
\endm
\begin{mmraw}%
|- ( [ y / x ] ph <-> ( ( x = y -> ph ) \TAND E. x ( x = y \TAND ph ) ) ) \$.
\end{mmraw}


\noindent Define existential uniqueness\index{existential uniqueness
quantifier ($\exists "!$)} (``there exists exactly one'').  Note that $y$ is a
variable distinct from $x$ and not occurring in $\varphi$.\label{df-eu}

\vskip 0.5ex
\setbox\startprefix=\hbox{\tt \ \ df-eu\ \$a\ }
\setbox\contprefix=\hbox{\tt \ \ \ \ \ \ \ \ \ \ \ }
\startm
\m{\vdash}\m{(}\m{\exists}\m{{!}}\m{x}\m{\varphi}\m{\leftrightarrow}\m{\exists}
\m{y}\m{\forall}\m{x}\m{(}\m{\varphi}\m{\leftrightarrow}\m{x}\m{=}\m{y}\m{)}\m{)}
\endm

\begin{mmraw}%
|- ( E! x ph <-> E. y A. x ( ph <-> x = y ) ) \$.
\end{mmraw}

\subsection{Definitions for Set Theory}\label{setdefinitions}

The symbols $x$, $y$, $z$, and $w$ represent individual variables of
predicate calculus, which in set theory are understood to be sets.
However, using only the constructs shown so far would be very inconvenient.

To make set theory more practical, we introduce the notion of a ``class.''
A class\index{class} is either a set variable (such as $x$) or an
expression of the form $\{ x | \varphi\}$ (called an ``abstraction
class''\index{abstraction class}\index{class abstraction}).  Note that
sets (i.e.\ individual variables) always exist (this is a theorem of
logic, namely $\exists y \, y = x$ for any set $x$), whereas classes may
or may not exist (i.e.\ $\exists y \, y = A$ may or may not be true).
If a class does not exist it is called a ``proper class.''\index{proper
class}\index{class!proper} Definitions \texttt{df-clab},
\texttt{df-cleq}, and \texttt{df-clel} can be used to convert an
expression containing classes into one containing only set variables and
wff metavariables.

The symbols $A$, $B$, $C$, $D$, $ F$, $G$, and $R$ are metavariables that range
over classes.  A class metavariable $A$ may be eliminated from a wff by
replacing it with $\{ x|\varphi\}$ where neither $x$ nor $\varphi$ occur in
the wff.

The theory of classes can be shown to be an eliminable and conservative
extension of set theory. The \textbf{eliminability}
property shows that for every
formula in the extended language we can build a logically equivalent
formula in the basic language; so that even if the extended language
provides more ease to convey and formulate mathematical ideas for set
theory, its expressive power does not in fact strengthen the basic
language's expressive power.
The \textbf{conservation} property shows that for
every proof of a formula of the basic language in the extended system
we can build another proof of the same formula in the basic system;
so that, concerning theorems on sets only, the deductive powers of
the extended system and of the basic system are identical. Together,
these properties mean that the extended language can be treated as a
definitional extension that is \textbf{sound}.

A rigorous justification, which we will not give here, can be found in
Levy \cite[pp.~357-366]{Levy} supplementing his informal introduction to class
theory on pp.~7-17. Two other good treatments of class theory are provided
by Quine \cite[pp.~15-21]{Quine}\index{Quine, Willard Van Orman}
and also \cite[pp.~10-14]{Takeuti}\index{Takeuti, Gaisi}.
Quine's exposition (he calls them virtual classes)
is nicely written and very readable.

In the rest of this
section, individual variables are always assumed to be distinct from
each other unless otherwise indicated.  In addition, dummy variables on the
right-hand side of a definition do not occur in the class and wff
metavariables in the definition.

The definitions we present here are a partial but self-contained
collection selected from several hundred that appear in the current
\texttt{set.mm} database.  They are adequate for a basic development of
elementary set theory.

\vskip 2ex
\noindent Define the abstraction class.\index{abstraction class}\index{class
abstraction}\label{df-clab}  $x$ and $y$
need not be distinct.  Definition 2.1 of Quine, p.~16.  This definition may
seem puzzling since it is shorter than the expression being defined and does not
buy us anything in terms of brevity.  The reason we introduce this definition
is because it fits in neatly with the extension of the $\in$ connective
provided by \texttt{df-clel}.

\vskip 0.5ex
\setbox\startprefix=\hbox{\tt \ \ df-clab\ \$a\ }
\setbox\contprefix=\hbox{\tt \ \ \ \ \ \ \ \ \ \ \ \ \ }
\startm
\m{\vdash}\m{(}\m{x}\m{\in}\m{\{}\m{y}\m{|}\m{\varphi}\m{\}}\m{%
\leftrightarrow}\m{[}\m{x}\m{/}\m{y}\m{]}\m{\varphi}\m{)}
\endm
\begin{mmraw}%
|- ( x e. \{ y | ph \} <-> [ x / y ] ph ) \$.
\end{mmraw}

\noindent Define the equality connective between classes\index{class
equality}\label{df-cleq}.  See Quine or Chapter 4 of Takeuti and Zaring for its
justification and methods for eliminating it.  This is an example of a
somewhat ``dangerous'' definition, because it extends the use of the
existing equality symbol rather than introducing a new symbol, allowing
us to make statements in the original language that may not be true.
For example, it permits us to deduce $y = z \leftrightarrow \forall x (
x \in y \leftrightarrow x \in z )$ which is not a theorem of logic but
rather presupposes the Axiom of Extensionality,\index{Axiom of
Extensionality} which we include as a hypothesis so that we can know
when this axiom is assumed in a proof (with the \texttt{show
trace{\char`\_}back} command).  We could avoid the danger by introducing
another symbol, say $\eqcirc$, in place of $=$; this
would also have the advantage of making elimination of the definition
straightforward and would eliminate the need for Extensionality as a
hypothesis.  We would then also have the advantage of being able to
identify exactly where Extensionality truly comes into play.  One of our
theorems would be $x \eqcirc y \leftrightarrow x = y$
by invoking Extensionality.  However in keeping with standard practice
we retain the ``dangerous'' definition.

\vskip 0.5ex
\setbox\startprefix=\hbox{\tt \ \ df-cleq.1\ \$e\ }
\setbox\contprefix=\hbox{\tt \ \ \ \ \ \ \ \ \ \ \ \ \ \ \ }
\startm
\m{\vdash}\m{(}\m{\forall}\m{x}\m{(}\m{x}\m{\in}\m{y}\m{\leftrightarrow}\m{x}
\m{\in}\m{z}\m{)}\m{\rightarrow}\m{y}\m{=}\m{z}\m{)}
\endm
\setbox\startprefix=\hbox{\tt \ \ df-cleq\ \$a\ }
\setbox\contprefix=\hbox{\tt \ \ \ \ \ \ \ \ \ \ \ \ \ }
\startm
\m{\vdash}\m{(}\m{A}\m{=}\m{B}\m{\leftrightarrow}\m{\forall}\m{x}\m{(}\m{x}\m{
\in}\m{A}\m{\leftrightarrow}\m{x}\m{\in}\m{B}\m{)}\m{)}
\endm
% We need to reset the startprefix and contprefix.
\setbox\startprefix=\hbox{\tt \ \ df-cleq.1\ \$e\ }
\setbox\contprefix=\hbox{\tt \ \ \ \ \ \ \ \ \ \ \ \ \ \ \ }
\begin{mmraw}%
|- ( A. x ( x e. y <-> x e. z ) -> y = z ) \$.
\end{mmraw}
\setbox\startprefix=\hbox{\tt \ \ df-cleq\ \$a\ }
\setbox\contprefix=\hbox{\tt \ \ \ \ \ \ \ \ \ \ \ \ \ }
\begin{mmraw}%
|- ( A = B <-> A. x ( x e. A <-> x e. B ) ) \$.
\end{mmraw}

\noindent Define the membership connective between classes\index{class
membership}.  Theorem 6.3 of Quine, p.~41, which we adopt as a definition.
Note that it extends the use of the existing membership symbol, but unlike
{\tt df-cleq} it does not extend the set of valid wffs of logic when the class
metavariables are replaced with set variables.\label{dfclel}\label{df-clel}

\vskip 0.5ex
\setbox\startprefix=\hbox{\tt \ \ df-clel\ \$a\ }
\setbox\contprefix=\hbox{\tt \ \ \ \ \ \ \ \ \ \ \ \ \ }
\startm
\m{\vdash}\m{(}\m{A}\m{\in}\m{B}\m{\leftrightarrow}\m{\exists}\m{x}\m{(}\m{x}
\m{=}\m{A}\m{\wedge}\m{x}\m{\in}\m{B}\m{)}\m{)}
\endm
\begin{mmraw}%
|- ( A e. B <-> E. x ( x = A \TAND x e. B ) ) \$.?
\end{mmraw}

\noindent Define inequality.

\vskip 0.5ex
\setbox\startprefix=\hbox{\tt \ \ df-ne\ \$a\ }
\setbox\contprefix=\hbox{\tt \ \ \ \ \ \ \ \ \ \ \ }
\startm
\m{\vdash}\m{(}\m{A}\m{\ne}\m{B}\m{\leftrightarrow}\m{\lnot}\m{A}\m{=}\m{B}%
\m{)}
\endm
\begin{mmraw}%
|- ( A =/= B <-> -. A = B ) \$.
\end{mmraw}

\noindent Define restricted universal quantification.\index{universal
quantifier ($\forall$)!restricted}  Enderton, p.~22.

\vskip 0.5ex
\setbox\startprefix=\hbox{\tt \ \ df-ral\ \$a\ }
\setbox\contprefix=\hbox{\tt \ \ \ \ \ \ \ \ \ \ \ \ }
\startm
\m{\vdash}\m{(}\m{\forall}\m{x}\m{\in}\m{A}\m{\varphi}\m{\leftrightarrow}\m{%
\forall}\m{x}\m{(}\m{x}\m{\in}\m{A}\m{\rightarrow}\m{\varphi}\m{)}\m{)}
\endm
\begin{mmraw}%
|- ( A. x e. A ph <-> A. x ( x e. A -> ph ) ) \$.
\end{mmraw}

\noindent Define restricted existential quantification.\index{existential
quantifier ($\exists$)!restricted}  Enderton, p.~22.

\vskip 0.5ex
\setbox\startprefix=\hbox{\tt \ \ df-rex\ \$a\ }
\setbox\contprefix=\hbox{\tt \ \ \ \ \ \ \ \ \ \ \ \ }
\startm
\m{\vdash}\m{(}\m{\exists}\m{x}\m{\in}\m{A}\m{\varphi}\m{\leftrightarrow}\m{%
\exists}\m{x}\m{(}\m{x}\m{\in}\m{A}\m{\wedge}\m{\varphi}\m{)}\m{)}
\endm
\begin{mmraw}%
|- ( E. x e. A ph <-> E. x ( x e. A \TAND ph ) ) \$.
\end{mmraw}

\noindent Define the universal class\index{universal class ($V$)}.  Definition
5.20, p.~21, of Takeuti and Zaring.\label{df-v}

\vskip 0.5ex
\setbox\startprefix=\hbox{\tt \ \ df-v\ \$a\ }
\setbox\contprefix=\hbox{\tt \ \ \ \ \ \ \ \ \ \ }
\startm
\m{\vdash}\m{{\rm V}}\m{=}\m{\{}\m{x}\m{|}\m{x}\m{=}\m{x}\m{\}}
\endm
\begin{mmraw}%
|- {\char`\_}V = \{ x | x = x \} \$.
\end{mmraw}

\noindent Define the subclass\index{subclass}\index{subset} relationship
between two classes (called the subset relation if the classes are sets i.e.\
are not proper).  Definition 5.9 of Takeuti and Zaring, p.~17.\label{df-ss}

\vskip 0.5ex
\setbox\startprefix=\hbox{\tt \ \ df-ss\ \$a\ }
\setbox\contprefix=\hbox{\tt \ \ \ \ \ \ \ \ \ \ \ }
\startm
\m{\vdash}\m{(}\m{A}\m{\subseteq}\m{B}\m{\leftrightarrow}\m{\forall}\m{x}\m{(}
\m{x}\m{\in}\m{A}\m{\rightarrow}\m{x}\m{\in}\m{B}\m{)}\m{)}
\endm
\begin{mmraw}%
|- ( A C\_ B <-> A. x ( x e. A -> x e. B ) ) \$.
\end{mmraw}

\noindent Define the union\index{union} of two classes.  Definition 5.6 of Takeuti and Zaring,
p.~16.\label{df-un}

\vskip 0.5ex
\setbox\startprefix=\hbox{\tt \ \ df-un\ \$a\ }
\setbox\contprefix=\hbox{\tt \ \ \ \ \ \ \ \ \ \ \ }
\startm
\m{\vdash}\m{(}\m{A}\m{\cup}\m{B}\m{)}\m{=}\m{\{}\m{x}\m{|}\m{(}\m{x}\m{\in}
\m{A}\m{\vee}\m{x}\m{\in}\m{B}\m{)}\m{\}}
\endm
\begin{mmraw}%
( A u. B ) = \{ x | ( x e. A \TOR x e. B ) \} \$.
\end{mmraw}

\noindent Define the intersection\index{intersection} of two classes.  Definition 5.6 of
Takeuti and Zaring, p.~16.\label{df-in}

\vskip 0.5ex
\setbox\startprefix=\hbox{\tt \ \ df-in\ \$a\ }
\setbox\contprefix=\hbox{\tt \ \ \ \ \ \ \ \ \ \ \ }
\startm
\m{\vdash}\m{(}\m{A}\m{\cap}\m{B}\m{)}\m{=}\m{\{}\m{x}\m{|}\m{(}\m{x}\m{\in}
\m{A}\m{\wedge}\m{x}\m{\in}\m{B}\m{)}\m{\}}
\endm
% Caret ^ requires special treatment
\begin{mmraw}%
|- ( A i\^{}i B ) = \{ x | ( x e. A \TAND x e. B ) \} \$.
\end{mmraw}

\noindent Define class difference\index{class difference}\index{set difference}.
Definition 5.12 of Takeuti and Zaring, p.~20.  Several notations are used in
the literature; we chose the $\setminus$ convention instead of a minus sign to
reserve the latter for later use in, e.g., arithmetic.\label{df-dif}

\vskip 0.5ex
\setbox\startprefix=\hbox{\tt \ \ df-dif\ \$a\ }
\setbox\contprefix=\hbox{\tt \ \ \ \ \ \ \ \ \ \ \ \ }
\startm
\m{\vdash}\m{(}\m{A}\m{\setminus}\m{B}\m{)}\m{=}\m{\{}\m{x}\m{|}\m{(}\m{x}\m{
\in}\m{A}\m{\wedge}\m{\lnot}\m{x}\m{\in}\m{B}\m{)}\m{\}}
\endm
\begin{mmraw}%
( A \SLASH B ) = \{ x | ( x e. A \TAND -. x e. B ) \} \$.
\end{mmraw}

\noindent Define the empty or null set\index{empty set}\index{null set}.
Compare  Definition 5.14 of Takeuti and Zaring, p.~20.\label{df-nul}

\vskip 0.5ex
\setbox\startprefix=\hbox{\tt \ \ df-nul\ \$a\ }
\setbox\contprefix=\hbox{\tt \ \ \ \ \ \ \ \ \ \ }
\startm
\m{\vdash}\m{\varnothing}\m{=}\m{(}\m{{\rm V}}\m{\setminus}\m{{\rm V}}\m{)}
\endm
\begin{mmraw}%
|- (/) = ( {\char`\_}V \SLASH {\char`\_}V ) \$.
\end{mmraw}

\noindent Define power class\index{power set}\index{power class}.  Definition 5.10 of
Takeuti and Zaring, p.~17, but we also let it apply to proper classes.  (Note
that \verb$~P$ is the symbol for calligraphic P, the tilde
suggesting ``curly;'' see Appendix~\ref{ASCII}.)\label{df-pw}

\vskip 0.5ex
\setbox\startprefix=\hbox{\tt \ \ df-pw\ \$a\ }
\setbox\contprefix=\hbox{\tt \ \ \ \ \ \ \ \ \ \ \ }
\startm
\m{\vdash}\m{{\cal P}}\m{A}\m{=}\m{\{}\m{x}\m{|}\m{x}\m{\subseteq}\m{A}\m{\}}
\endm
% Special incantation required to put ~ into the text
\begin{mmraw}%
|- \char`\~P~A = \{ x | x C\_ A \} \$.
\end{mmraw}

\noindent Define the singleton of a class\index{singleton}.  Definition 7.1 of
Quine, p.~48.  It is well-defined for proper classes, although
it is not very meaningful in this case, where it evaluates to the empty
set.

\vskip 0.5ex
\setbox\startprefix=\hbox{\tt \ \ df-sn\ \$a\ }
\setbox\contprefix=\hbox{\tt \ \ \ \ \ \ \ \ \ \ \ }
\startm
\m{\vdash}\m{\{}\m{A}\m{\}}\m{=}\m{\{}\m{x}\m{|}\m{x}\m{=}\m{A}\m{\}}
\endm
\begin{mmraw}%
|- \{ A \} = \{ x | x = A \} \$.
\end{mmraw}%

\noindent Define an unordered pair of classes\index{unordered pair}\index{pair}.  Definition
7.1 of Quine, p.~48.

\vskip 0.5ex
\setbox\startprefix=\hbox{\tt \ \ df-pr\ \$a\ }
\setbox\contprefix=\hbox{\tt \ \ \ \ \ \ \ \ \ \ \ }
\startm
\m{\vdash}\m{\{}\m{A}\m{,}\m{B}\m{\}}\m{=}\m{(}\m{\{}\m{A}\m{\}}\m{\cup}\m{\{}
\m{B}\m{\}}\m{)}
\endm
\begin{mmraw}%
|- \{ A , B \} = ( \{ A \} u. \{ B \} ) \$.
\end{mmraw}

\noindent Define an unordered triple of classes\index{unordered triple}.  Definition of
Enderton, p.~19.

\vskip 0.5ex
\setbox\startprefix=\hbox{\tt \ \ df-tp\ \$a\ }
\setbox\contprefix=\hbox{\tt \ \ \ \ \ \ \ \ \ \ \ }
\startm
\m{\vdash}\m{\{}\m{A}\m{,}\m{B}\m{,}\m{C}\m{\}}\m{=}\m{(}\m{\{}\m{A}\m{,}\m{B}
\m{\}}\m{\cup}\m{\{}\m{C}\m{\}}\m{)}
\endm
\begin{mmraw}%
|- \{ A , B , C \} = ( \{ A , B \} u. \{ C \} ) \$.
\end{mmraw}%

\noindent Kuratowski's\index{Kuratowski, Kazimierz} ordered pair\index{ordered
pair} definition.  Definition 9.1 of Quine, p.~58. For proper classes it is
not meaningful but is well-defined for convenience.  (Note that \verb$<.$
stands for $\langle$ whereas \verb$<$ stands for $<$, and similarly for
\verb$>.$\,.)\label{df-op}

\vskip 0.5ex
\setbox\startprefix=\hbox{\tt \ \ df-op\ \$a\ }
\setbox\contprefix=\hbox{\tt \ \ \ \ \ \ \ \ \ \ \ }
\startm
\m{\vdash}\m{\langle}\m{A}\m{,}\m{B}\m{\rangle}\m{=}\m{\{}\m{\{}\m{A}\m{\}}
\m{,}\m{\{}\m{A}\m{,}\m{B}\m{\}}\m{\}}
\endm
\begin{mmraw}%
|- <. A , B >. = \{ \{ A \} , \{ A , B \} \} \$.
\end{mmraw}

\noindent Define the union of a class\index{union}.  Definition 5.5, p.~16,
of Takeuti and Zaring.

\vskip 0.5ex
\setbox\startprefix=\hbox{\tt \ \ df-uni\ \$a\ }
\setbox\contprefix=\hbox{\tt \ \ \ \ \ \ \ \ \ \ \ \ }
\startm
\m{\vdash}\m{\bigcup}\m{A}\m{=}\m{\{}\m{x}\m{|}\m{\exists}\m{y}\m{(}\m{x}\m{
\in}\m{y}\m{\wedge}\m{y}\m{\in}\m{A}\m{)}\m{\}}
\endm
\begin{mmraw}%
|- U. A = \{ x | E. y ( x e. y \TAND y e. A ) \} \$.
\end{mmraw}

\noindent Define the intersection\index{intersection} of a class.  Definition 7.35,
p.~44, of Takeuti and Zaring.

\vskip 0.5ex
\setbox\startprefix=\hbox{\tt \ \ df-int\ \$a\ }
\setbox\contprefix=\hbox{\tt \ \ \ \ \ \ \ \ \ \ \ \ }
\startm
\m{\vdash}\m{\bigcap}\m{A}\m{=}\m{\{}\m{x}\m{|}\m{\forall}\m{y}\m{(}\m{y}\m{
\in}\m{A}\m{\rightarrow}\m{x}\m{\in}\m{y}\m{)}\m{\}}
\endm
\begin{mmraw}%
|- |\^{}| A = \{ x | A. y ( y e. A -> x e. y ) \} \$.
\end{mmraw}

\noindent Define a transitive class\index{transitive class}\index{transitive
set}.  This should not be confused with a transitive relation, which is a different
concept.  Definition from p.~71 of Enderton, extended to classes.

\vskip 0.5ex
\setbox\startprefix=\hbox{\tt \ \ df-tr\ \$a\ }
\setbox\contprefix=\hbox{\tt \ \ \ \ \ \ \ \ \ \ \ }
\startm
\m{\vdash}\m{(}\m{\mbox{\rm Tr}}\m{A}\m{\leftrightarrow}\m{\bigcup}\m{A}\m{
\subseteq}\m{A}\m{)}
\endm
\begin{mmraw}%
|- ( Tr A <-> U. A C\_ A ) \$.
\end{mmraw}

\noindent Define a notation for a general binary relation\index{binary
relation}.  Definition 6.18, p.~29, of Takeuti and Zaring, generalized to
arbitrary classes.  This definition is well-defined, although not very
meaningful, when classes $A$ and/or $B$ are proper.\label{dfbr}  The lack of
parentheses (or any other connective) creates no ambiguity since we are defining
an atomic wff.

\vskip 0.5ex
\setbox\startprefix=\hbox{\tt \ \ df-br\ \$a\ }
\setbox\contprefix=\hbox{\tt \ \ \ \ \ \ \ \ \ \ \ }
\startm
\m{\vdash}\m{(}\m{A}\m{\,R}\m{\,B}\m{\leftrightarrow}\m{\langle}\m{A}\m{,}\m{B}
\m{\rangle}\m{\in}\m{R}\m{)}
\endm
\begin{mmraw}%
|- ( A R B <-> <. A , B >. e. R ) \$.
\end{mmraw}

\noindent Define an abstraction class of ordered pairs\index{abstraction
class!of ordered
pairs}.  A special case of Definition 4.16, p.~14, of Takeuti and Zaring.
Note that $ z $ must be distinct from $ x $ and $ y $,
and $ z $ must not occur in $\varphi$, but $ x $ and $ y $ may be
identical and may appear in $\varphi$.

\vskip 0.5ex
\setbox\startprefix=\hbox{\tt \ \ df-opab\ \$a\ }
\setbox\contprefix=\hbox{\tt \ \ \ \ \ \ \ \ \ \ \ \ \ }
\startm
\m{\vdash}\m{\{}\m{\langle}\m{x}\m{,}\m{y}\m{\rangle}\m{|}\m{\varphi}\m{\}}\m{=}
\m{\{}\m{z}\m{|}\m{\exists}\m{x}\m{\exists}\m{y}\m{(}\m{z}\m{=}\m{\langle}\m{x}
\m{,}\m{y}\m{\rangle}\m{\wedge}\m{\varphi}\m{)}\m{\}}
\endm

\begin{mmraw}%
|- \{ <. x , y >. | ph \} = \{ z | E. x E. y ( z =
<. x , y >. /\ ph ) \} \$.
\end{mmraw}

\noindent Define the epsilon relation\index{epsilon relation}.  Similar to Definition
6.22, p.~30, of Takeuti and Zaring.

\vskip 0.5ex
\setbox\startprefix=\hbox{\tt \ \ df-eprel\ \$a\ }
\setbox\contprefix=\hbox{\tt \ \ \ \ \ \ \ \ \ \ \ \ \ \ }
\startm
\m{\vdash}\m{{\rm E}}\m{=}\m{\{}\m{\langle}\m{x}\m{,}\m{y}\m{\rangle}\m{|}\m{x}\m{
\in}\m{y}\m{\}}
\endm
\begin{mmraw}%
|- \_E = \{ <. x , y >. | x e. y \} \$.
\end{mmraw}

\noindent Define a founded relation\index{founded relation}.  $R$ is a founded
relation on $A$ iff\index{iff} (if and only if) each nonempty subset of $A$
has an ``$R$-minimal element.''  Similar to Definition 6.21, p.~30, of
Takeuti and Zaring.

\vskip 0.5ex
\setbox\startprefix=\hbox{\tt \ \ df-fr\ \$a\ }
\setbox\contprefix=\hbox{\tt \ \ \ \ \ \ \ \ \ \ \ }
\startm
\m{\vdash}\m{(}\m{R}\m{\,\mbox{\rm Fr}}\m{\,A}\m{\leftrightarrow}\m{\forall}\m{x}
\m{(}\m{(}\m{x}\m{\subseteq}\m{A}\m{\wedge}\m{\lnot}\m{x}\m{=}\m{\varnothing}
\m{)}\m{\rightarrow}\m{\exists}\m{y}\m{(}\m{y}\m{\in}\m{x}\m{\wedge}\m{(}\m{x}
\m{\cap}\m{\{}\m{z}\m{|}\m{z}\m{\,R}\m{\,y}\m{\}}\m{)}\m{=}\m{\varnothing}\m{)}
\m{)}\m{)}
\endm
\begin{mmraw}%
|- ( R Fr A <-> A. x ( ( x C\_ A \TAND -. x = (/) ) ->
E. y ( y e. x \TAND ( x i\^{}i \{ z | z R y \} ) = (/) ) ) ) \$.
\end{mmraw}

\noindent Define a well-ordering\index{well-ordering}.  $R$ is a well-ordering of $A$ iff
it is founded on $A$ and the elements of $A$ are pairwise $R$-comparable.
Similar to Definition 6.24(2), p.~30, of Takeuti and Zaring.

\vskip 0.5ex
\setbox\startprefix=\hbox{\tt \ \ df-we\ \$a\ }
\setbox\contprefix=\hbox{\tt \ \ \ \ \ \ \ \ \ \ \ }
\startm
\m{\vdash}\m{(}\m{R}\m{\,\mbox{\rm We}}\m{\,A}\m{\leftrightarrow}\m{(}\m{R}\m{\,
\mbox{\rm Fr}}\m{\,A}\m{\wedge}\m{\forall}\m{x}\m{\forall}\m{y}\m{(}\m{(}\m{x}\m{
\in}\m{A}\m{\wedge}\m{y}\m{\in}\m{A}\m{)}\m{\rightarrow}\m{(}\m{x}\m{\,R}\m{\,y}
\m{\vee}\m{x}\m{=}\m{y}\m{\vee}\m{y}\m{\,R}\m{\,x}\m{)}\m{)}\m{)}\m{)}
\endm
\begin{mmraw}%
( R We A <-> ( R Fr A \TAND A. x A. y ( ( x e.
A \TAND y e. A ) -> ( x R y \TOR x = y \TOR y R x ) ) ) ) \$.
\end{mmraw}

\noindent Define the ordinal predicate\index{ordinal predicate}, which is true for a class
that is transitive and is well-ordered by the epsilon relation.  Similar to
definition on p.~468, Bell and Machover.

\vskip 0.5ex
\setbox\startprefix=\hbox{\tt \ \ df-ord\ \$a\ }
\setbox\contprefix=\hbox{\tt \ \ \ \ \ \ \ \ \ \ \ \ }
\startm
\m{\vdash}\m{(}\m{\mbox{\rm Ord}}\m{\,A}\m{\leftrightarrow}\m{(}
\m{\mbox{\rm Tr}}\m{\,A}\m{\wedge}\m{E}\m{\,\mbox{\rm We}}\m{\,A}\m{)}\m{)}
\endm
\begin{mmraw}%
|- ( Ord A <-> ( Tr A \TAND E We A ) ) \$.
\end{mmraw}

\noindent Define the class of all ordinal numbers\index{ordinal number}.  An ordinal number is
a set that satisfies the ordinal predicate.  Definition 7.11 of Takeuti and
Zaring, p.~38.

\vskip 0.5ex
\setbox\startprefix=\hbox{\tt \ \ df-on\ \$a\ }
\setbox\contprefix=\hbox{\tt \ \ \ \ \ \ \ \ \ \ \ }
\startm
\m{\vdash}\m{\,\mbox{\rm On}}\m{=}\m{\{}\m{x}\m{|}\m{\mbox{\rm Ord}}\m{\,x}
\m{\}}
\endm
\begin{mmraw}%
|- On = \{ x | Ord x \} \$.
\end{mmraw}

\noindent Define the limit ordinal predicate\index{limit ordinal}, which is true for a
non-empty ordinal that is not a successor (i.e.\ that is the union of itself).
Compare Bell and Machover, p.~471 and Exercise (1), p.~42 of Takeuti and
Zaring.

\vskip 0.5ex
\setbox\startprefix=\hbox{\tt \ \ df-lim\ \$a\ }
\setbox\contprefix=\hbox{\tt \ \ \ \ \ \ \ \ \ \ \ \ }
\startm
\m{\vdash}\m{(}\m{\mbox{\rm Lim}}\m{\,A}\m{\leftrightarrow}\m{(}\m{\mbox{
\rm Ord}}\m{\,A}\m{\wedge}\m{\lnot}\m{A}\m{=}\m{\varnothing}\m{\wedge}\m{A}
\m{=}\m{\bigcup}\m{A}\m{)}\m{)}
\endm
\begin{mmraw}%
|- ( Lim A <-> ( Ord A \TAND -. A = (/) \TAND A = U. A ) ) \$.
\end{mmraw}

\noindent Define the successor\index{successor} of a class.  Definition 7.22 of Takeuti
and Zaring, p.~41.  Our definition is a generalization to classes, although it
is meaningless when classes are proper.

\vskip 0.5ex
\setbox\startprefix=\hbox{\tt \ \ df-suc\ \$a\ }
\setbox\contprefix=\hbox{\tt \ \ \ \ \ \ \ \ \ \ \ \ }
\startm
\m{\vdash}\m{\,\mbox{\rm suc}}\m{\,A}\m{=}\m{(}\m{A}\m{\cup}\m{\{}\m{A}\m{\}}
\m{)}
\endm
\begin{mmraw}%
|- suc A = ( A u. \{ A \} ) \$.
\end{mmraw}

\noindent Define the class of natural numbers\index{natural number}\index{omega
($\omega$)}.  Compare Bell and Machover, p.~471.\label{dfom}

\vskip 0.5ex
\setbox\startprefix=\hbox{\tt \ \ df-om\ \$a\ }
\setbox\contprefix=\hbox{\tt \ \ \ \ \ \ \ \ \ \ \ }
\startm
\m{\vdash}\m{\omega}\m{=}\m{\{}\m{x}\m{|}\m{(}\m{\mbox{\rm Ord}}\m{\,x}\m{
\wedge}\m{\forall}\m{y}\m{(}\m{\mbox{\rm Lim}}\m{\,y}\m{\rightarrow}\m{x}\m{
\in}\m{y}\m{)}\m{)}\m{\}}
\endm
\begin{mmraw}%
|- om = \{ x | ( Ord x \TAND A. y ( Lim y -> x e. y ) ) \} \$.
\end{mmraw}

\noindent Define the Cartesian product (also called the
cross product)\index{Cartesian product}\index{cross product}
of two classes.  Definition 9.11 of Quine, p.~64.

\vskip 0.5ex
\setbox\startprefix=\hbox{\tt \ \ df-xp\ \$a\ }
\setbox\contprefix=\hbox{\tt \ \ \ \ \ \ \ \ \ \ \ }
\startm
\m{\vdash}\m{(}\m{A}\m{\times}\m{B}\m{)}\m{=}\m{\{}\m{\langle}\m{x}\m{,}\m{y}
\m{\rangle}\m{|}\m{(}\m{x}\m{\in}\m{A}\m{\wedge}\m{y}\m{\in}\m{B}\m{)}\m{\}}
\endm
\begin{mmraw}%
|- ( A X. B ) = \{ <. x , y >. | ( x e. A \TAND y e. B) \} \$.
\end{mmraw}

\noindent Define a relation\index{relation}.  Definition 6.4(1) of Takeuti and
Zaring, p.~23.

\vskip 0.5ex
\setbox\startprefix=\hbox{\tt \ \ df-rel\ \$a\ }
\setbox\contprefix=\hbox{\tt \ \ \ \ \ \ \ \ \ \ \ \ }
\startm
\m{\vdash}\m{(}\m{\mbox{\rm Rel}}\m{\,A}\m{\leftrightarrow}\m{A}\m{\subseteq}
\m{(}\m{{\rm V}}\m{\times}\m{{\rm V}}\m{)}\m{)}
\endm
\begin{mmraw}%
|- ( Rel A <-> A C\_ ( {\char`\_}V X. {\char`\_}V ) ) \$.
\end{mmraw}

\noindent Define the domain\index{domain} of a class.  Definition 6.5(1) of
Takeuti and Zaring, p.~24.

\vskip 0.5ex
\setbox\startprefix=\hbox{\tt \ \ df-dm\ \$a\ }
\setbox\contprefix=\hbox{\tt \ \ \ \ \ \ \ \ \ \ \ }
\startm
\m{\vdash}\m{\,\mbox{\rm dom}}\m{A}\m{=}\m{\{}\m{x}\m{|}\m{\exists}\m{y}\m{
\langle}\m{x}\m{,}\m{y}\m{\rangle}\m{\in}\m{A}\m{\}}
\endm
\begin{mmraw}%
|- dom A = \{ x | E. y <. x , y >. e. A \} \$.
\end{mmraw}

\noindent Define the range\index{range} of a class.  Definition 6.5(2) of
Takeuti and Zaring, p.~24.

\vskip 0.5ex
\setbox\startprefix=\hbox{\tt \ \ df-rn\ \$a\ }
\setbox\contprefix=\hbox{\tt \ \ \ \ \ \ \ \ \ \ \ }
\startm
\m{\vdash}\m{\,\mbox{\rm ran}}\m{A}\m{=}\m{\{}\m{y}\m{|}\m{\exists}\m{x}\m{
\langle}\m{x}\m{,}\m{y}\m{\rangle}\m{\in}\m{A}\m{\}}
\endm
\begin{mmraw}%
|- ran A = \{ y | E. x <. x , y >. e. A \} \$.
\end{mmraw}

\noindent Define the restriction\index{restriction} of a class.  Definition
6.6(1) of Takeuti and Zaring, p.~24.

\vskip 0.5ex
\setbox\startprefix=\hbox{\tt \ \ df-res\ \$a\ }
\setbox\contprefix=\hbox{\tt \ \ \ \ \ \ \ \ \ \ \ \ }
\startm
\m{\vdash}\m{(}\m{A}\m{\restriction}\m{B}\m{)}\m{=}\m{(}\m{A}\m{\cap}\m{(}\m{B}
\m{\times}\m{{\rm V}}\m{)}\m{)}
\endm
\begin{mmraw}%
|- ( A |` B ) = ( A i\^{}i ( B X. {\char`\_}V ) ) \$.
\end{mmraw}

\noindent Define the image\index{image} of a class.  Definition 6.6(2) of
Takeuti and Zaring, p.~24.

\vskip 0.5ex
\setbox\startprefix=\hbox{\tt \ \ df-ima\ \$a\ }
\setbox\contprefix=\hbox{\tt \ \ \ \ \ \ \ \ \ \ \ \ }
\startm
\m{\vdash}\m{(}\m{A}\m{``}\m{B}\m{)}\m{=}\m{\,\mbox{\rm ran}}\m{\,(}\m{A}\m{
\restriction}\m{B}\m{)}
\endm
\begin{mmraw}%
|- ( A " B ) = ran ( A |` B ) \$.
\end{mmraw}

\noindent Define the composition\index{composition} of two classes.  Definition 6.6(3) of
Takeuti and Zaring, p.~24.

\vskip 0.5ex
\setbox\startprefix=\hbox{\tt \ \ df-co\ \$a\ }
\setbox\contprefix=\hbox{\tt \ \ \ \ \ \ \ \ \ \ \ \ }
\startm
\m{\vdash}\m{(}\m{A}\m{\circ}\m{B}\m{)}\m{=}\m{\{}\m{\langle}\m{x}\m{,}\m{y}\m{
\rangle}\m{|}\m{\exists}\m{z}\m{(}\m{\langle}\m{x}\m{,}\m{z}\m{\rangle}\m{\in}
\m{B}\m{\wedge}\m{\langle}\m{z}\m{,}\m{y}\m{\rangle}\m{\in}\m{A}\m{)}\m{\}}
\endm
\begin{mmraw}%
|- ( A o. B ) = \{ <. x , y >. | E. z ( <. x , z
>. e. B \TAND <. z , y >. e. A ) \} \$.
\end{mmraw}

\noindent Define a function\index{function}.  Definition 6.4(4) of Takeuti and
Zaring, p.~24.

\vskip 0.5ex
\setbox\startprefix=\hbox{\tt \ \ df-fun\ \$a\ }
\setbox\contprefix=\hbox{\tt \ \ \ \ \ \ \ \ \ \ \ \ }
\startm
\m{\vdash}\m{(}\m{\mbox{\rm Fun}}\m{\,A}\m{\leftrightarrow}\m{(}
\m{\mbox{\rm Rel}}\m{\,A}\m{\wedge}
\m{\forall}\m{x}\m{\exists}\m{z}\m{\forall}\m{y}\m{(}
\m{\langle}\m{x}\m{,}\m{y}\m{\rangle}\m{\in}\m{A}\m{\rightarrow}\m{y}\m{=}\m{z}
\m{)}\m{)}\m{)}
\endm
\begin{mmraw}%
|- ( Fun A <-> ( Rel A /\ A. x E. z A. y ( <. x
   , y >. e. A -> y = z ) ) ) \$.
\end{mmraw}

\noindent Define a function with domain.  Definition 6.15(1) of Takeuti and
Zaring, p.~27.

\vskip 0.5ex
\setbox\startprefix=\hbox{\tt \ \ df-fn\ \$a\ }
\setbox\contprefix=\hbox{\tt \ \ \ \ \ \ \ \ \ \ \ }
\startm
\m{\vdash}\m{(}\m{A}\m{\,\mbox{\rm Fn}}\m{\,B}\m{\leftrightarrow}\m{(}
\m{\mbox{\rm Fun}}\m{\,A}\m{\wedge}\m{\mbox{\rm dom}}\m{\,A}\m{=}\m{B}\m{)}
\m{)}
\endm
\begin{mmraw}%
|- ( A Fn B <-> ( Fun A \TAND dom A = B ) ) \$.
\end{mmraw}

\noindent Define a function with domain and co-domain.  Definition 6.15(3)
of Takeuti and Zaring, p.~27.

\vskip 0.5ex
\setbox\startprefix=\hbox{\tt \ \ df-f\ \$a\ }
\setbox\contprefix=\hbox{\tt \ \ \ \ \ \ \ \ \ \ }
\startm
\m{\vdash}\m{(}\m{F}\m{:}\m{A}\m{\longrightarrow}\m{B}\m{
\leftrightarrow}\m{(}\m{F}\m{\,\mbox{\rm Fn}}\m{\,A}\m{\wedge}\m{
\mbox{\rm ran}}\m{\,F}\m{\subseteq}\m{B}\m{)}\m{)}
\endm
\begin{mmraw}%
|- ( F : A --> B <-> ( F Fn A \TAND ran F C\_ B ) ) \$.
\end{mmraw}

\noindent Define a one-to-one function\index{one-to-one function}.  Compare
Definition 6.15(5) of Takeuti and Zaring, p.~27.

\vskip 0.5ex
\setbox\startprefix=\hbox{\tt \ \ df-f1\ \$a\ }
\setbox\contprefix=\hbox{\tt \ \ \ \ \ \ \ \ \ \ \ }
\startm
\m{\vdash}\m{(}\m{F}\m{:}\m{A}\m{
\raisebox{.5ex}{${\textstyle{\:}_{\mbox{\footnotesize\rm
1\tt -\rm 1}}}\atop{\textstyle{
\longrightarrow}\atop{\textstyle{}^{\mbox{\footnotesize\rm {\ }}}}}$}
}\m{B}
\m{\leftrightarrow}\m{(}\m{F}\m{:}\m{A}\m{\longrightarrow}\m{B}
\m{\wedge}\m{\forall}\m{y}\m{\exists}\m{z}\m{\forall}\m{x}\m{(}\m{\langle}\m{x}
\m{,}\m{y}\m{\rangle}\m{\in}\m{F}\m{\rightarrow}\m{x}\m{=}\m{z}\m{)}\m{)}\m{)}
\endm
\begin{mmraw}%
|- ( F : A -1-1-> B <-> ( F : A --> B \TAND
   A. y E. z A. x ( <. x , y >. e. F -> x = z ) ) ) \$.
\end{mmraw}

\noindent Define an onto function\index{onto function}.  Definition 6.15(4) of Takeuti and
Zaring, p.~27.

\vskip 0.5ex
\setbox\startprefix=\hbox{\tt \ \ df-fo\ \$a\ }
\setbox\contprefix=\hbox{\tt \ \ \ \ \ \ \ \ \ \ \ }
\startm
\m{\vdash}\m{(}\m{F}\m{:}\m{A}\m{
\raisebox{.5ex}{${\textstyle{\:}_{\mbox{\footnotesize\rm
{\ }}}}\atop{\textstyle{
\longrightarrow}\atop{\textstyle{}^{\mbox{\footnotesize\rm onto}}}}$}
}\m{B}
\m{\leftrightarrow}\m{(}\m{F}\m{\,\mbox{\rm Fn}}\m{\,A}\m{\wedge}
\m{\mbox{\rm ran}}\m{\,F}\m{=}\m{B}\m{)}\m{)}
\endm
\begin{mmraw}%
|- ( F : A -onto-> B <-> ( F Fn A /\ ran F = B ) ) \$.
\end{mmraw}

\noindent Define a one-to-one, onto function.  Compare Definition 6.15(6) of
Takeuti and Zaring, p.~27.

\vskip 0.5ex
\setbox\startprefix=\hbox{\tt \ \ df-f1o\ \$a\ }
\setbox\contprefix=\hbox{\tt \ \ \ \ \ \ \ \ \ \ \ \ }
\startm
\m{\vdash}\m{(}\m{F}\m{:}\m{A}
\m{
\raisebox{.5ex}{${\textstyle{\:}_{\mbox{\footnotesize\rm
1\tt -\rm 1}}}\atop{\textstyle{
\longrightarrow}\atop{\textstyle{}^{\mbox{\footnotesize\rm onto}}}}$}
}
\m{B}
\m{\leftrightarrow}\m{(}\m{F}\m{:}\m{A}
\m{
\raisebox{.5ex}{${\textstyle{\:}_{\mbox{\footnotesize\rm
1\tt -\rm 1}}}\atop{\textstyle{
\longrightarrow}\atop{\textstyle{}^{\mbox{\footnotesize\rm {\ }}}}}$}
}
\m{B}\m{\wedge}\m{F}\m{:}\m{A}
\m{
\raisebox{.5ex}{${\textstyle{\:}_{\mbox{\footnotesize\rm
{\ }}}}\atop{\textstyle{
\longrightarrow}\atop{\textstyle{}^{\mbox{\footnotesize\rm onto}}}}$}
}
\m{B}\m{)}\m{)}
\endm
\begin{mmraw}%
|- ( F : A -1-1-onto-> B <-> ( F : A -1-1-> B? \TAND F : A -onto-> B ) ) \$.?
\end{mmraw}

\noindent Define the value of a function\index{function value}.  This
definition applies to any class and evaluates to the empty set when it is not
meaningful. Note that $ F`A$ means the same thing as the more familiar $ F(A)$
notation for a function's value at $A$.  The $ F`A$ notation is common in
formal set theory.\label{df-fv}

\vskip 0.5ex
\setbox\startprefix=\hbox{\tt \ \ df-fv\ \$a\ }
\setbox\contprefix=\hbox{\tt \ \ \ \ \ \ \ \ \ \ \ }
\startm
\m{\vdash}\m{(}\m{F}\m{`}\m{A}\m{)}\m{=}\m{\bigcup}\m{\{}\m{x}\m{|}\m{(}\m{F}%
\m{``}\m{\{}\m{A}\m{\}}\m{)}\m{=}\m{\{}\m{x}\m{\}}\m{\}}
\endm
\begin{mmraw}%
|- ( F ` A ) = U. \{ x | ( F " \{ A \} ) = \{ x \} \} \$.
\end{mmraw}

\noindent Define the result of an operation.\index{operation}  Here, $F$ is
     an operation on two
     values (such as $+$ for real numbers).   This is defined for proper
     classes $A$ and $B$ even though not meaningful in that case.  However,
     the definition can be meaningful when $F$ is a proper class.\label{dfopr}

\vskip 0.5ex
\setbox\startprefix=\hbox{\tt \ \ df-opr\ \$a\ }
\setbox\contprefix=\hbox{\tt \ \ \ \ \ \ \ \ \ \ \ \ }
\startm
\m{\vdash}\m{(}\m{A}\m{\,F}\m{\,B}\m{)}\m{=}\m{(}\m{F}\m{`}\m{\langle}\m{A}%
\m{,}\m{B}\m{\rangle}\m{)}
\endm
\begin{mmraw}%
|- ( A F B ) = ( F ` <. A , B >. ) \$.
\end{mmraw}

\section{Tricks of the Trade}\label{tricks}

In the \texttt{set.mm}\index{set theory database (\texttt{set.mm})} database our goal
was usually to conform to modern notation.  However in some cases the
relationship to standard textbook language may be obscured by several
unconventional devices we used to simplify the development and to take
advantage of the Metamath language.  In this section we will describe some
common conventions used in \texttt{set.mm}.

\begin{itemize}
\item
The turnstile\index{turnstile ({$\,\vdash$})} symbol, $\vdash$, meaning ``it
is provable that,'' is the first token of all assertions and hypotheses that
aren't syntax constructions.  This is a standard convention in logic.  (We
mentioned this earlier, but this symbol is bothersome to some people without a
logic background.  It has no deeper meaning but just provides us with a way to
distinguish syntax constructions from ordinary mathematical statements.)

\item
A hypothesis of the form

\vskip 1ex
\setbox\startprefix=\hbox{\tt \ \ \ \ \ \ \ \ \ \$e\ }
\setbox\contprefix=\hbox{\tt \ \ \ \ \ \ \ \ \ \ }
\startm
\m{\vdash}\m{(}\m{\varphi}\m{\rightarrow}\m{\forall}\m{x}\m{\varphi}\m{)}
\endm
\vskip 1ex

should be read ``assume variable $x$ is (effectively) not free in wff
$\varphi$.''\index{effectively not free}
Literally, this says ``assume it is provable that $\varphi \rightarrow \forall
x\, \varphi$.''  This device lets us avoid the complexities associated with
the standard treatment of free and bound variables.
%% Uncomment this when uncommenting section {formalspec} below
The footnote on p.~\pageref{effectivelybound} discusses this further.

\item
A statement of one of the forms

\vskip 1ex
\setbox\startprefix=\hbox{\tt \ \ \ \ \ \ \ \ \ \$a\ }
\setbox\contprefix=\hbox{\tt \ \ \ \ \ \ \ \ \ \ }
\startm
\m{\vdash}\m{(}\m{\lnot}\m{\forall}\m{x}\m{\,x}\m{=}\m{y}\m{\rightarrow}
\m{\ldots}\m{)}
\endm
\setbox\startprefix=\hbox{\tt \ \ \ \ \ \ \ \ \ \$p\ }
\setbox\contprefix=\hbox{\tt \ \ \ \ \ \ \ \ \ \ }
\startm
\m{\vdash}\m{(}\m{\lnot}\m{\forall}\m{x}\m{\,x}\m{=}\m{y}\m{\rightarrow}
\m{\ldots}\m{)}
\endm
\vskip 1ex

should be read ``if $x$ and $y$ are distinct variables, then...''  This
antecedent provides us with a technical device to avoid the need for the
\texttt{\$d} statement early in our development of predicate calculus,
permitting symbol manipulations to be as conceptually simple as those in
propositional calculus.  However, the \texttt{\$d} statement eventually
becomes a requirement, and after that this device is rarely used.

\item
The statement

\vskip 1ex
\setbox\startprefix=\hbox{\tt \ \ \ \ \ \ \ \ \ \$d\ }
\setbox\contprefix=\hbox{\tt \ \ \ \ \ \ \ \ \ \ }
\startm
\m{x}\m{\,y}
\endm
\vskip 1ex

should be read ``assume $x$ and $y$ are distinct variables.''

\item
The statement

\vskip 1ex
\setbox\startprefix=\hbox{\tt \ \ \ \ \ \ \ \ \ \$d\ }
\setbox\contprefix=\hbox{\tt \ \ \ \ \ \ \ \ \ \ }
\startm
\m{x}\m{\,\varphi}
\endm
\vskip 1ex

should be read ``assume $x$ does not occur in $\varphi$.''

\item
The statement

\vskip 1ex
\setbox\startprefix=\hbox{\tt \ \ \ \ \ \ \ \ \ \$d\ }
\setbox\contprefix=\hbox{\tt \ \ \ \ \ \ \ \ \ \ }
\startm
\m{x}\m{\,A}
\endm
\vskip 1ex

should be read ``assume variable $x$ does not occur in class $A$.''

\item
The restriction and hypothesis group

\vskip 1ex
\setbox\startprefix=\hbox{\tt \ \ \ \ \ \ \ \ \ \$d\ }
\setbox\contprefix=\hbox{\tt \ \ \ \ \ \ \ \ \ \ }
\startm
\m{x}\m{\,A}
\endm
\setbox\startprefix=\hbox{\tt \ \ \ \ \ \ \ \ \ \$d\ }
\setbox\contprefix=\hbox{\tt \ \ \ \ \ \ \ \ \ \ }
\startm
\m{x}\m{\,\psi}
\endm
\setbox\startprefix=\hbox{\tt \ \ \ \ \ \ \ \ \ \$e\ }
\setbox\contprefix=\hbox{\tt \ \ \ \ \ \ \ \ \ \ }
\startm
\m{\vdash}\m{(}\m{x}\m{=}\m{A}\m{\rightarrow}\m{(}\m{\varphi}\m{\leftrightarrow}
\m{\psi}\m{)}\m{)}
\endm
\vskip 1ex

is frequently used in place of explicit substitution, meaning ``assume
$\psi$ results from the proper substitution of $A$ for $x$ in
$\varphi$.''  Sometimes ``\texttt{\$e} $\vdash ( \psi \rightarrow
\forall x \, \psi )$'' is used instead of ``\texttt{\$d} $x\, \psi $,''
which requires only that $x$ be effectively not free in $\varphi$ but
not necessarily absent from it.  The use of implicit
substitution\index{substitution!implicit} is partly a matter of personal
style, although it may make proofs somewhat shorter than would be the
case with explicit substitution.

\item
The hypothesis


\vskip 1ex
\setbox\startprefix=\hbox{\tt \ \ \ \ \ \ \ \ \ \$e\ }
\setbox\contprefix=\hbox{\tt \ \ \ \ \ \ \ \ \ \ }
\startm
\m{\vdash}\m{A}\m{\in}\m{{\rm V}}
\endm
\vskip 1ex

should be read ``assume class $A$ is a set (i.e.\ exists).''
This is a convenient convention used by Quine.

\item
The restriction and hypothesis

\vskip 1ex
\setbox\startprefix=\hbox{\tt \ \ \ \ \ \ \ \ \ \$d\ }
\setbox\contprefix=\hbox{\tt \ \ \ \ \ \ \ \ \ \ }
\startm
\m{x}\m{\,y}
\endm
\setbox\startprefix=\hbox{\tt \ \ \ \ \ \ \ \ \ \$e\ }
\setbox\contprefix=\hbox{\tt \ \ \ \ \ \ \ \ \ \ }
\startm
\m{\vdash}\m{(}\m{y}\m{\in}\m{A}\m{\rightarrow}\m{\forall}\m{x}\m{\,y}
\m{\in}\m{A}\m{)}
\endm
\vskip 1ex

should be read ``assume variable $x$ is
(effectively) not free in class $A$.''

\end{itemize}

\section{A Theorem Sampler}\label{sometheorems}

In this section we list some of the more important theorems that are proved in
the \texttt{set.mm} database, and they illustrate the kinds of things that can be
done with Metamath.  While all of these facts are well-known results,
Metamath offers the advantage of easily allowing you to trace their
derivation back to axioms.  Our intent here is not to try to explain the
details or motivation; for this we refer you to the textbooks that are
mentioned in the descriptions.  (The \texttt{set.mm} file has bibliographic
references for the text references.)  Their proofs often embody important
concepts you may wish to explore with the Metamath program (see
Section~\ref{exploring}).  All the symbols that are used here are defined in
Section~\ref{hierarchy}.  For brevity we haven't included the \texttt{\$d}
restrictions or \texttt{\$f} hypotheses for these theorems; when you are
uncertain consult the \texttt{set.mm} database.

We start with \texttt{syl} (principle of the syllogism).
In \textit{Principia Mathematica}
Whitehead and Russell call this ``the principle of the
syllogism... because... the syllogism in Barbara is derived from them''
\cite[quote after Theorem *2.06 p.~101]{PM}.
Some authors call this law a ``hypothetical syllogism.''
At the time of this writing it is the most commonly referenced assertion in the
\texttt{set.mm} database.\footnote{
The Metamath program command \texttt{show usage}
shows the number of references.
}

\vskip 2ex
\noindent Theorem syl (principle of the syllogism)\index{Syllogism}%
\index{\texttt{syl}}\label{syl}.

\vskip 0.5ex
\setbox\startprefix=\hbox{\tt \ \ syl.1\ \$e\ }
\setbox\contprefix=\hbox{\tt \ \ \ \ \ \ \ \ \ \ \ }
\startm
\m{\vdash}\m{(}\m{\varphi}\m{ \rightarrow }\m{\psi}\m{)}
\endm
\setbox\startprefix=\hbox{\tt \ \ syl.2\ \$e\ }
\setbox\contprefix=\hbox{\tt \ \ \ \ \ \ \ \ \ \ \ }
\startm
\m{\vdash}\m{(}\m{\psi}\m{ \rightarrow }\m{\chi}\m{)}
\endm
\setbox\startprefix=\hbox{\tt \ \ syl\ \$p\ }
\setbox\contprefix=\hbox{\tt \ \ \ \ \ \ \ \ \ }
\startm
\m{\vdash}\m{(}\m{\varphi}\m{ \rightarrow }\m{\chi}\m{)}
\endm
\vskip 2ex

The following theorem is not very deep but provides us with a notational device
that is frequently used.  It allows us to use the expression ``$A \in V$'' as
a compact way of saying that class $A$ exists, i.e.\ is a set.

\vskip 2ex
\noindent Two ways to say ``$A$ is a set'':  $A$ is a member of the universe
$V$ if and only if $A$ exists (i.e.\ there exists a set equal to $A$).
Theorem 6.9 of Quine, p. 43.

\vskip 0.5ex
\setbox\startprefix=\hbox{\tt \ \ isset\ \$p\ }
\setbox\contprefix=\hbox{\tt \ \ \ \ \ \ \ \ \ \ \ }
\startm
\m{\vdash}\m{(}\m{A}\m{\in}\m{{\rm V}}\m{\leftrightarrow}\m{\exists}\m{x}\m{\,x}\m{=}
\m{A}\m{)}
\endm
\vskip 1ex

Next we prove the axioms of standard ZF set theory that were missing from our
axiom system.  From our point of view they are theorems since they
can be derived from the other axioms.

\vskip 2ex
\noindent Axiom of Separation\index{Axiom of Separation}
(Aussonderung)\index{Aussonderung} proved from the other axioms of ZF set
theory.  Compare Exercise 4 of Takeuti and Zaring, p.~22.

\vskip 0.5ex
\setbox\startprefix=\hbox{\tt \ \ inex1.1\ \$e\ }
\setbox\contprefix=\hbox{\tt \ \ \ \ \ \ \ \ \ \ \ \ \ \ \ }
\startm
\m{\vdash}\m{A}\m{\in}\m{{\rm V}}
\endm
\setbox\startprefix=\hbox{\tt \ \ inex\ \$p\ }
\setbox\contprefix=\hbox{\tt \ \ \ \ \ \ \ \ \ \ \ \ \ }
\startm
\m{\vdash}\m{(}\m{A}\m{\cap}\m{B}\m{)}\m{\in}\m{{\rm V}}
\endm
\vskip 1ex

\noindent Axiom of the Null Set\index{Axiom of the Null Set} proved from the
other axioms of ZF set theory. Corollary 5.16 of Takeuti and Zaring, p.~20.

\vskip 0.5ex
\setbox\startprefix=\hbox{\tt \ \ 0ex\ \$p\ }
\setbox\contprefix=\hbox{\tt \ \ \ \ \ \ \ \ \ \ \ \ }
\startm
\m{\vdash}\m{\varnothing}\m{\in}\m{{\rm V}}
\endm
\vskip 1ex

\noindent The Axiom of Pairing\index{Axiom of Pairing} proved from the other
axioms of ZF set theory.  Theorem 7.13 of Quine, p.~51.
\vskip 0.5ex
\setbox\startprefix=\hbox{\tt \ \ prex\ \$p\ }
\setbox\contprefix=\hbox{\tt \ \ \ \ \ \ \ \ \ \ \ \ \ \ }
\startm
\m{\vdash}\m{\{}\m{A}\m{,}\m{B}\m{\}}\m{\in}\m{{\rm V}}
\endm
\vskip 2ex

Next we will list some famous or important theorems that are proved in
the \texttt{set.mm} database.  None of them except \texttt{omex}
require the Axiom of Infinity, as you can verify with the \texttt{show
trace{\char`\_}back} Metamath command.

\vskip 2ex
\noindent The resolution of Russell's paradox\index{Russell's paradox}.  There
exists no set containing the set of all sets which are not members of
themselves.  Proposition 4.14 of Takeuti and Zaring, p.~14.

\vskip 0.5ex
\setbox\startprefix=\hbox{\tt \ \ ru\ \$p\ }
\setbox\contprefix=\hbox{\tt \ \ \ \ \ \ \ \ }
\startm
\m{\vdash}\m{\lnot}\m{\exists}\m{x}\m{\,x}\m{=}\m{\{}\m{y}\m{|}\m{\lnot}\m{y}
\m{\in}\m{y}\m{\}}
\endm
\vskip 1ex

\noindent Cantor's theorem\index{Cantor's theorem}.  No set can be mapped onto
its power set.  Compare Theorem 6B(b) of Enderton, p.~132.

\vskip 0.5ex
\setbox\startprefix=\hbox{\tt \ \ canth.1\ \$e\ }
\setbox\contprefix=\hbox{\tt \ \ \ \ \ \ \ \ \ \ \ \ \ }
\startm
\m{\vdash}\m{A}\m{\in}\m{{\rm V}}
\endm
\setbox\startprefix=\hbox{\tt \ \ canth\ \$p\ }
\setbox\contprefix=\hbox{\tt \ \ \ \ \ \ \ \ \ \ \ }
\startm
\m{\vdash}\m{\lnot}\m{F}\m{:}\m{A}\m{\raisebox{.5ex}{${\textstyle{\:}_{
\mbox{\footnotesize\rm {\ }}}}\atop{\textstyle{\longrightarrow}\atop{
\textstyle{}^{\mbox{\footnotesize\rm onto}}}}$}}\m{{\cal P}}\m{A}
\endm
\vskip 1ex

\noindent The Burali-Forti paradox\index{Burali-Forti paradox}.  No set
contains all ordinal numbers. Enderton, p.~194.  (Burali-Forti was one person,
not two.)

\vskip 0.5ex
\setbox\startprefix=\hbox{\tt \ \ onprc\ \$p\ }
\setbox\contprefix=\hbox{\tt \ \ \ \ \ \ \ \ \ \ \ \ }
\startm
\m{\vdash}\m{\lnot}\m{\mbox{\rm On}}\m{\in}\m{{\rm V}}
\endm
\vskip 1ex

\noindent Peano's postulates\index{Peano's postulates} for arithmetic.
Proposition 7.30 of Takeuti and Zaring, pp.~42--43.  The objects being
described are the members of $\omega$ i.e.\ the natural numbers 0, 1,
2,\ldots.  The successor\index{successor} operation suc means ``plus
one.''  \texttt{peano1} says that 0 (which is defined as the empty set)
is a natural number.  \texttt{peano2} says that if $A$ is a natural
number, so is $A+1$.  \texttt{peano3} says that 0 is not the successor
of any natural number.  \texttt{peano4} says that two natural numbers
are equal if and only if their successors are equal.  \texttt{peano5} is
essentially the same as mathematical induction.

\vskip 1ex
\setbox\startprefix=\hbox{\tt \ \ peano1\ \$p\ }
\setbox\contprefix=\hbox{\tt \ \ \ \ \ \ \ \ \ \ \ \ }
\startm
\m{\vdash}\m{\varnothing}\m{\in}\m{\omega}
\endm
\vskip 1.5ex

\setbox\startprefix=\hbox{\tt \ \ peano2\ \$p\ }
\setbox\contprefix=\hbox{\tt \ \ \ \ \ \ \ \ \ \ \ \ }
\startm
\m{\vdash}\m{(}\m{A}\m{\in}\m{\omega}\m{\rightarrow}\m{{\rm suc}}\m{A}\m{\in}%
\m{\omega}\m{)}
\endm
\vskip 1.5ex

\setbox\startprefix=\hbox{\tt \ \ peano3\ \$p\ }
\setbox\contprefix=\hbox{\tt \ \ \ \ \ \ \ \ \ \ \ \ }
\startm
\m{\vdash}\m{(}\m{A}\m{\in}\m{\omega}\m{\rightarrow}\m{\lnot}\m{{\rm suc}}%
\m{A}\m{=}\m{\varnothing}\m{)}
\endm
\vskip 1.5ex

\setbox\startprefix=\hbox{\tt \ \ peano4\ \$p\ }
\setbox\contprefix=\hbox{\tt \ \ \ \ \ \ \ \ \ \ \ \ }
\startm
\m{\vdash}\m{(}\m{(}\m{A}\m{\in}\m{\omega}\m{\wedge}\m{B}\m{\in}\m{\omega}%
\m{)}\m{\rightarrow}\m{(}\m{{\rm suc}}\m{A}\m{=}\m{{\rm suc}}\m{B}\m{%
\leftrightarrow}\m{A}\m{=}\m{B}\m{)}\m{)}
\endm
\vskip 1.5ex

\setbox\startprefix=\hbox{\tt \ \ peano5\ \$p\ }
\setbox\contprefix=\hbox{\tt \ \ \ \ \ \ \ \ \ \ \ \ }
\startm
\m{\vdash}\m{(}\m{(}\m{\varnothing}\m{\in}\m{A}\m{\wedge}\m{\forall}\m{x}\m{%
\in}\m{\omega}\m{(}\m{x}\m{\in}\m{A}\m{\rightarrow}\m{{\rm suc}}\m{x}\m{\in}%
\m{A}\m{)}\m{)}\m{\rightarrow}\m{\omega}\m{\subseteq}\m{A}\m{)}
\endm
\vskip 1.5ex

\noindent Finite Induction (mathematical induction).\index{finite
induction}\index{mathematical induction} The first hypothesis is the
basis and the second is the induction hypothesis.  Theorem Schema 22 of
Suppes, p.~136.

\vskip 0.5ex
\setbox\startprefix=\hbox{\tt \ \ findes.1\ \$e\ }
\setbox\contprefix=\hbox{\tt \ \ \ \ \ \ \ \ \ \ \ \ \ \ }
\startm
\m{\vdash}\m{[}\m{\varnothing}\m{/}\m{x}\m{]}\m{\varphi}
\endm
\setbox\startprefix=\hbox{\tt \ \ findes.2\ \$e\ }
\setbox\contprefix=\hbox{\tt \ \ \ \ \ \ \ \ \ \ \ \ \ \ }
\startm
\m{\vdash}\m{(}\m{x}\m{\in}\m{\omega}\m{\rightarrow}\m{(}\m{\varphi}\m{%
\rightarrow}\m{[}\m{{\rm suc}}\m{x}\m{/}\m{x}\m{]}\m{\varphi}\m{)}\m{)}
\endm
\setbox\startprefix=\hbox{\tt \ \ findes\ \$p\ }
\setbox\contprefix=\hbox{\tt \ \ \ \ \ \ \ \ \ \ \ \ }
\startm
\m{\vdash}\m{(}\m{x}\m{\in}\m{\omega}\m{\rightarrow}\m{\varphi}\m{)}
\endm
\vskip 1ex

\noindent Transfinite Induction with explicit substitution.  The first
hypothesis is the basis, the second is the induction hypothesis for
successors, and the third is the induction hypothesis for limit
ordinals.  Theorem Schema 4 of Suppes, p. 197.

\vskip 0.5ex
\setbox\startprefix=\hbox{\tt \ \ tfindes.1\ \$e\ }
\setbox\contprefix=\hbox{\tt \ \ \ \ \ \ \ \ \ \ \ \ \ \ \ }
\startm
\m{\vdash}\m{[}\m{\varnothing}\m{/}\m{x}\m{]}\m{\varphi}
\endm
\setbox\startprefix=\hbox{\tt \ \ tfindes.2\ \$e\ }
\setbox\contprefix=\hbox{\tt \ \ \ \ \ \ \ \ \ \ \ \ \ \ \ }
\startm
\m{\vdash}\m{(}\m{x}\m{\in}\m{{\rm On}}\m{\rightarrow}\m{(}\m{\varphi}\m{%
\rightarrow}\m{[}\m{{\rm suc}}\m{x}\m{/}\m{x}\m{]}\m{\varphi}\m{)}\m{)}
\endm
\setbox\startprefix=\hbox{\tt \ \ tfindes.3\ \$e\ }
\setbox\contprefix=\hbox{\tt \ \ \ \ \ \ \ \ \ \ \ \ \ \ \ }
\startm
\m{\vdash}\m{(}\m{{\rm Lim}}\m{y}\m{\rightarrow}\m{(}\m{\forall}\m{x}\m{\in}%
\m{y}\m{\varphi}\m{\rightarrow}\m{[}\m{y}\m{/}\m{x}\m{]}\m{\varphi}\m{)}\m{)}
\endm
\setbox\startprefix=\hbox{\tt \ \ tfindes\ \$p\ }
\setbox\contprefix=\hbox{\tt \ \ \ \ \ \ \ \ \ \ \ \ \ }
\startm
\m{\vdash}\m{(}\m{x}\m{\in}\m{{\rm On}}\m{\rightarrow}\m{\varphi}\m{)}
\endm
\vskip 1ex

\noindent Principle of Transfinite Recursion.\index{transfinite
recursion} Theorem 7.41 of Takeuti and Zaring, p.~47.  Transfinite
recursion is the key theorem that allows arithmetic of ordinals to be
rigorously defined, and has many other important uses as well.
Hypotheses \texttt{tfr.1} and \texttt{tfr.2} specify a certain (proper)
class $ F$.  The complicated definition of $ F$ is not important in
itself; what is important is that there be such an $ F$ with the
required properties, and we show this by displaying $ F$ explicitly.
\texttt{tfr1} states that $ F$ is a function whose domain is the set of
ordinal numbers.  \texttt{tfr2} states that any value of $ F$ is
completely determined by its previous values and the values of an
auxiliary function, $G$.  \texttt{tfr3} states that $ F$ is unique,
i.e.\ it is the only function that satisfies \texttt{tfr1} and
\texttt{tfr2}.  Note that $ f$ is an individual variable like $x$ and
$y$; it is just a mnemonic to remind us that $A$ is a collection of
functions.

\vskip 0.5ex
\setbox\startprefix=\hbox{\tt \ \ tfr.1\ \$e\ }
\setbox\contprefix=\hbox{\tt \ \ \ \ \ \ \ \ \ \ \ }
\startm
\m{\vdash}\m{A}\m{=}\m{\{}\m{f}\m{|}\m{\exists}\m{x}\m{\in}\m{{\rm On}}\m{(}%
\m{f}\m{{\rm Fn}}\m{x}\m{\wedge}\m{\forall}\m{y}\m{\in}\m{x}\m{(}\m{f}\m{`}%
\m{y}\m{)}\m{=}\m{(}\m{G}\m{`}\m{(}\m{f}\m{\restriction}\m{y}\m{)}\m{)}\m{)}%
\m{\}}
\endm
\setbox\startprefix=\hbox{\tt \ \ tfr.2\ \$e\ }
\setbox\contprefix=\hbox{\tt \ \ \ \ \ \ \ \ \ \ \ }
\startm
\m{\vdash}\m{F}\m{=}\m{\bigcup}\m{A}
\endm
\setbox\startprefix=\hbox{\tt \ \ tfr1\ \$p\ }
\setbox\contprefix=\hbox{\tt \ \ \ \ \ \ \ \ \ \ }
\startm
\m{\vdash}\m{F}\m{{\rm Fn}}\m{{\rm On}}
\endm
\setbox\startprefix=\hbox{\tt \ \ tfr2\ \$p\ }
\setbox\contprefix=\hbox{\tt \ \ \ \ \ \ \ \ \ \ }
\startm
\m{\vdash}\m{(}\m{z}\m{\in}\m{{\rm On}}\m{\rightarrow}\m{(}\m{F}\m{`}\m{z}%
\m{)}\m{=}\m{(}\m{G}\m{`}\m{(}\m{F}\m{\restriction}\m{z}\m{)}\m{)}\m{)}
\endm
\setbox\startprefix=\hbox{\tt \ \ tfr3\ \$p\ }
\setbox\contprefix=\hbox{\tt \ \ \ \ \ \ \ \ \ \ }
\startm
\m{\vdash}\m{(}\m{(}\m{B}\m{{\rm Fn}}\m{{\rm On}}\m{\wedge}\m{\forall}\m{x}\m{%
\in}\m{{\rm On}}\m{(}\m{B}\m{`}\m{x}\m{)}\m{=}\m{(}\m{G}\m{`}\m{(}\m{B}\m{%
\restriction}\m{x}\m{)}\m{)}\m{)}\m{\rightarrow}\m{B}\m{=}\m{F}\m{)}
\endm
\vskip 1ex

\noindent The existence of omega (the class of natural numbers).\index{natural
number}\index{omega ($\omega$)}\index{Axiom of Infinity}  Axiom 7 of Takeuti
and Zaring, p.~43.  (This is the only theorem in this section requiring the
Axiom of Infinity.)

\vskip 0.5ex
\setbox\startprefix=\hbox{\tt \
\ omex\ \$p\ }
\setbox\contprefix=\hbox{\tt \ \ \ \ \ \ \ \ \ \ }
\startm
\m{\vdash}\m{\omega}\m{\in}\m{{\rm V}}
\endm
%\vskip 2ex


\section{Axioms for Real and Complex Numbers}\label{real}
\index{real and complex numbers!axioms for}

This section presents the axioms for real and complex numbers.  Analysis
textbooks implicitly or explicitly use these axioms or their equivalents
as their starting point.  In the database \texttt{set.mm}, we define real
and complex numbers as (rather complicated) specific sets and derive these
axioms as {\em theorems} from the axioms of ZF set theory, using a method
called Dedekind cuts.  We omit the details of this construction, which you can
follow if you wish using the \texttt{set.mm} database in conjunction with the
textbooks referenced therein.

Once we prove those theorems, we then restate these proven theorems as axioms.
This lets us easily identify which axioms are needed for a particular complex number proof, without the obfuscation of the set theory used to derive them.
As a result,
the construction is actually unimportant other
than to show that sets exist that satisfy the axioms, and thus that the axioms
are consistent if ZF set theory is consistent.  When working with real numbers
you can think of them as being the actual sets resulting
from the construction (for definiteness), or you can
think of them as otherwise unspecified sets that happen to satisfy the axioms.
The derivation is not easy, but the fact that it works is quite remarkable
and lends support to the idea that ZFC set theory is all we need to
provide a foundation for essentially all of mathematics.

For the axioms we are given (or postulate) 8 classes:  $\mathbb{C}$ (the
set of complex numbers), $\mathbb{R}$ (the set of real numbers, a subset
of $\mathbb{C}$), $0$ (zero), $1$ (one), $i$ (square root of
$-1$), $+$ (plus), $\cdot$ (times), and
$<_{\mathbb{R}}$ (less than for just the real numbers).
Subtraction and division are defined terms and are not part of the
axioms; for their definitions see \texttt{set.mm}.

Note that the notation $(A+B)$ (and similarly $(A\cdot B)$) specifies a class
called an {\em operation},\index{operation} and is the function value of the
class $+$ at ordered pair $\langle A,B \rangle$.  An operation is defined by
statement \texttt{df-opr} on p.~\pageref{dfopr}.
The notation $A <_{\mathbb{R}} B$ specifies a
wff called a {\em binary relation}\index{binary relation} and means $\langle A,B \rangle \in \,<_{\mathbb{R}}$, as defined by statement \texttt{df-br} on p.~\pageref{dfbr}.

Our set of 8 given classes is assumed to satisfy the following 22 axioms
(in the axioms listed below, $<$ really means $<_{\mathbb{R}}$).

\vskip 2ex

\noindent 1. The real numbers are a subset of the complex numbers.

%\vskip 0.5ex
\setbox\startprefix=\hbox{\tt \ \ ax-resscn\ \$p\ }
\setbox\contprefix=\hbox{\tt \ \ \ \ \ \ \ \ \ \ \ \ \ \ }
\startm
\m{\vdash}\m{\mathbb{R}}\m{\subseteq}\m{\mathbb{C}}
\endm
%\vskip 1ex

\noindent 2. One is a complex number.

%\vskip 0.5ex
\setbox\startprefix=\hbox{\tt \ \ ax-1cn\ \$p\ }
\setbox\contprefix=\hbox{\tt \ \ \ \ \ \ \ \ \ \ \ }
\startm
\m{\vdash}\m{1}\m{\in}\m{\mathbb{C}}
\endm
%\vskip 1ex

\noindent 3. The imaginary number $i$ is a complex number.

%\vskip 0.5ex
\setbox\startprefix=\hbox{\tt \ \ ax-icn\ \$p\ }
\setbox\contprefix=\hbox{\tt \ \ \ \ \ \ \ \ \ \ \ }
\startm
\m{\vdash}\m{i}\m{\in}\m{\mathbb{C}}
\endm
%\vskip 1ex

\noindent 4. Complex numbers are closed under addition.

%\vskip 0.5ex
\setbox\startprefix=\hbox{\tt \ \ ax-addcl\ \$p\ }
\setbox\contprefix=\hbox{\tt \ \ \ \ \ \ \ \ \ \ \ \ \ }
\startm
\m{\vdash}\m{(}\m{(}\m{A}\m{\in}\m{\mathbb{C}}\m{\wedge}\m{B}\m{\in}\m{\mathbb{C}}%
\m{)}\m{\rightarrow}\m{(}\m{A}\m{+}\m{B}\m{)}\m{\in}\m{\mathbb{C}}\m{)}
\endm
%\vskip 1ex

\noindent 5. Real numbers are closed under addition.

%\vskip 0.5ex
\setbox\startprefix=\hbox{\tt \ \ ax-addrcl\ \$p\ }
\setbox\contprefix=\hbox{\tt \ \ \ \ \ \ \ \ \ \ \ \ \ \ }
\startm
\m{\vdash}\m{(}\m{(}\m{A}\m{\in}\m{\mathbb{R}}\m{\wedge}\m{B}\m{\in}\m{\mathbb{R}}%
\m{)}\m{\rightarrow}\m{(}\m{A}\m{+}\m{B}\m{)}\m{\in}\m{\mathbb{R}}\m{)}
\endm
%\vskip 1ex

\noindent 6. Complex numbers are closed under multiplication.

%\vskip 0.5ex
\setbox\startprefix=\hbox{\tt \ \ ax-mulcl\ \$p\ }
\setbox\contprefix=\hbox{\tt \ \ \ \ \ \ \ \ \ \ \ \ \ }
\startm
\m{\vdash}\m{(}\m{(}\m{A}\m{\in}\m{\mathbb{C}}\m{\wedge}\m{B}\m{\in}\m{\mathbb{C}}%
\m{)}\m{\rightarrow}\m{(}\m{A}\m{\cdot}\m{B}\m{)}\m{\in}\m{\mathbb{C}}\m{)}
\endm
%\vskip 1ex

\noindent 7. Real numbers are closed under multiplication.

%\vskip 0.5ex
\setbox\startprefix=\hbox{\tt \ \ ax-mulrcl\ \$p\ }
\setbox\contprefix=\hbox{\tt \ \ \ \ \ \ \ \ \ \ \ \ \ \ }
\startm
\m{\vdash}\m{(}\m{(}\m{A}\m{\in}\m{\mathbb{R}}\m{\wedge}\m{B}\m{\in}\m{\mathbb{R}}%
\m{)}\m{\rightarrow}\m{(}\m{A}\m{\cdot}\m{B}\m{)}\m{\in}\m{\mathbb{R}}\m{)}
\endm
%\vskip 1ex

\noindent 8. Multiplication of complex numbers is commutative.

%\vskip 0.5ex
\setbox\startprefix=\hbox{\tt \ \ ax-mulcom\ \$p\ }
\setbox\contprefix=\hbox{\tt \ \ \ \ \ \ \ \ \ \ \ \ \ \ }
\startm
\m{\vdash}\m{(}\m{(}\m{A}\m{\in}\m{\mathbb{C}}\m{\wedge}\m{B}\m{\in}\m{\mathbb{C}}%
\m{)}\m{\rightarrow}\m{(}\m{A}\m{\cdot}\m{B}\m{)}\m{=}\m{(}\m{B}\m{\cdot}\m{A}%
\m{)}\m{)}
\endm
%\vskip 1ex

\noindent 9. Addition of complex numbers is associative.

%\vskip 0.5ex
\setbox\startprefix=\hbox{\tt \ \ ax-addass\ \$p\ }
\setbox\contprefix=\hbox{\tt \ \ \ \ \ \ \ \ \ \ \ \ \ \ }
\startm
\m{\vdash}\m{(}\m{(}\m{A}\m{\in}\m{\mathbb{C}}\m{\wedge}\m{B}\m{\in}\m{\mathbb{C}}%
\m{\wedge}\m{C}\m{\in}\m{\mathbb{C}}\m{)}\m{\rightarrow}\m{(}\m{(}\m{A}\m{+}%
\m{B}\m{)}\m{+}\m{C}\m{)}\m{=}\m{(}\m{A}\m{+}\m{(}\m{B}\m{+}\m{C}\m{)}\m{)}%
\m{)}
\endm
%\vskip 1ex

\noindent 10. Multiplication of complex numbers is associative.

%\vskip 0.5ex
\setbox\startprefix=\hbox{\tt \ \ ax-mulass\ \$p\ }
\setbox\contprefix=\hbox{\tt \ \ \ \ \ \ \ \ \ \ \ \ \ \ }
\startm
\m{\vdash}\m{(}\m{(}\m{A}\m{\in}\m{\mathbb{C}}\m{\wedge}\m{B}\m{\in}\m{\mathbb{C}}%
\m{\wedge}\m{C}\m{\in}\m{\mathbb{C}}\m{)}\m{\rightarrow}\m{(}\m{(}\m{A}\m{\cdot}%
\m{B}\m{)}\m{\cdot}\m{C}\m{)}\m{=}\m{(}\m{A}\m{\cdot}\m{(}\m{B}\m{\cdot}\m{C}%
\m{)}\m{)}\m{)}
\endm
%\vskip 1ex

\noindent 11. Multiplication distributes over addition for complex numbers.

%\vskip 0.5ex
\setbox\startprefix=\hbox{\tt \ \ ax-distr\ \$p\ }
\setbox\contprefix=\hbox{\tt \ \ \ \ \ \ \ \ \ \ \ \ \ }
\startm
\m{\vdash}\m{(}\m{(}\m{A}\m{\in}\m{\mathbb{C}}\m{\wedge}\m{B}\m{\in}\m{\mathbb{C}}%
\m{\wedge}\m{C}\m{\in}\m{\mathbb{C}}\m{)}\m{\rightarrow}\m{(}\m{A}\m{\cdot}\m{(}%
\m{B}\m{+}\m{C}\m{)}\m{)}\m{=}\m{(}\m{(}\m{A}\m{\cdot}\m{B}\m{)}\m{+}\m{(}%
\m{A}\m{\cdot}\m{C}\m{)}\m{)}\m{)}
\endm
%\vskip 1ex

\noindent 12. The square of $i$ equals $-1$ (expressed as $i$-squared plus 1 is
0).

%\vskip 0.5ex
\setbox\startprefix=\hbox{\tt \ \ ax-i2m1\ \$p\ }
\setbox\contprefix=\hbox{\tt \ \ \ \ \ \ \ \ \ \ \ \ }
\startm
\m{\vdash}\m{(}\m{(}\m{i}\m{\cdot}\m{i}\m{)}\m{+}\m{1}\m{)}\m{=}\m{0}
\endm
%\vskip 1ex

\noindent 13. One and zero are distinct.

%\vskip 0.5ex
\setbox\startprefix=\hbox{\tt \ \ ax-1ne0\ \$p\ }
\setbox\contprefix=\hbox{\tt \ \ \ \ \ \ \ \ \ \ \ \ }
\startm
\m{\vdash}\m{1}\m{\ne}\m{0}
\endm
%\vskip 1ex

\noindent 14. One is an identity element for real multiplication.

%\vskip 0.5ex
\setbox\startprefix=\hbox{\tt \ \ ax-1rid\ \$p\ }
\setbox\contprefix=\hbox{\tt \ \ \ \ \ \ \ \ \ \ \ }
\startm
\m{\vdash}\m{(}\m{A}\m{\in}\m{\mathbb{R}}\m{\rightarrow}\m{(}\m{A}\m{\cdot}\m{1}%
\m{)}\m{=}\m{A}\m{)}
\endm
%\vskip 1ex

\noindent 15. Every real number has a negative.

%\vskip 0.5ex
\setbox\startprefix=\hbox{\tt \ \ ax-rnegex\ \$p\ }
\setbox\contprefix=\hbox{\tt \ \ \ \ \ \ \ \ \ \ \ \ \ \ }
\startm
\m{\vdash}\m{(}\m{A}\m{\in}\m{\mathbb{R}}\m{\rightarrow}\m{\exists}\m{x}\m{\in}%
\m{\mathbb{R}}\m{(}\m{A}\m{+}\m{x}\m{)}\m{=}\m{0}\m{)}
\endm
%\vskip 1ex

\noindent 16. Every nonzero real number has a reciprocal.

%\vskip 0.5ex
\setbox\startprefix=\hbox{\tt \ \ ax-rrecex\ \$p\ }
\setbox\contprefix=\hbox{\tt \ \ \ \ \ \ \ \ \ \ \ \ \ \ }
\startm
\m{\vdash}\m{(}\m{A}\m{\in}\m{\mathbb{R}}\m{\rightarrow}\m{(}\m{A}\m{\ne}\m{0}%
\m{\rightarrow}\m{\exists}\m{x}\m{\in}\m{\mathbb{R}}\m{(}\m{A}\m{\cdot}%
\m{x}\m{)}\m{=}\m{1}\m{)}\m{)}
\endm
%\vskip 1ex

\noindent 17. A complex number can be expressed in terms of two reals.

%\vskip 0.5ex
\setbox\startprefix=\hbox{\tt \ \ ax-cnre\ \$p\ }
\setbox\contprefix=\hbox{\tt \ \ \ \ \ \ \ \ \ \ \ \ }
\startm
\m{\vdash}\m{(}\m{A}\m{\in}\m{\mathbb{C}}\m{\rightarrow}\m{\exists}\m{x}\m{\in}%
\m{\mathbb{R}}\m{\exists}\m{y}\m{\in}\m{\mathbb{R}}\m{A}\m{=}\m{(}\m{x}\m{+}\m{(}%
\m{y}\m{\cdot}\m{i}\m{)}\m{)}\m{)}
\endm
%\vskip 1ex

\noindent 18. Ordering on reals satisfies strict trichotomy.

%\vskip 0.5ex
\setbox\startprefix=\hbox{\tt \ \ ax-pre-lttri\ \$p\ }
\setbox\contprefix=\hbox{\tt \ \ \ \ \ \ \ \ \ \ \ \ \ }
\startm
\m{\vdash}\m{(}\m{(}\m{A}\m{\in}\m{\mathbb{R}}\m{\wedge}\m{B}\m{\in}\m{\mathbb{R}}%
\m{)}\m{\rightarrow}\m{(}\m{A}\m{<}\m{B}\m{\leftrightarrow}\m{\lnot}\m{(}\m{A}%
\m{=}\m{B}\m{\vee}\m{B}\m{<}\m{A}\m{)}\m{)}\m{)}
\endm
%\vskip 1ex

\noindent 19. Ordering on reals is transitive.

%\vskip 0.5ex
\setbox\startprefix=\hbox{\tt \ \ ax-pre-lttrn\ \$p\ }
\setbox\contprefix=\hbox{\tt \ \ \ \ \ \ \ \ \ \ \ \ \ }
\startm
\m{\vdash}\m{(}\m{(}\m{A}\m{\in}\m{\mathbb{R}}\m{\wedge}\m{B}\m{\in}\m{\mathbb{R}}%
\m{\wedge}\m{C}\m{\in}\m{\mathbb{R}}\m{)}\m{\rightarrow}\m{(}\m{(}\m{A}\m{<}%
\m{B}\m{\wedge}\m{B}\m{<}\m{C}\m{)}\m{\rightarrow}\m{A}\m{<}\m{C}\m{)}\m{)}
\endm
%\vskip 1ex

\noindent 20. Ordering on reals is preserved after addition to both sides.

%\vskip 0.5ex
\setbox\startprefix=\hbox{\tt \ \ ax-pre-ltadd\ \$p\ }
\setbox\contprefix=\hbox{\tt \ \ \ \ \ \ \ \ \ \ \ \ \ }
\startm
\m{\vdash}\m{(}\m{(}\m{A}\m{\in}\m{\mathbb{R}}\m{\wedge}\m{B}\m{\in}\m{\mathbb{R}}%
\m{\wedge}\m{C}\m{\in}\m{\mathbb{R}}\m{)}\m{\rightarrow}\m{(}\m{A}\m{<}\m{B}\m{%
\rightarrow}\m{(}\m{C}\m{+}\m{A}\m{)}\m{<}\m{(}\m{C}\m{+}\m{B}\m{)}\m{)}\m{)}
\endm
%\vskip 1ex

\noindent 21. The product of two positive reals is positive.

%\vskip 0.5ex
\setbox\startprefix=\hbox{\tt \ \ ax-pre-mulgt0\ \$p\ }
\setbox\contprefix=\hbox{\tt \ \ \ \ \ \ \ \ \ \ \ \ \ \ }
\startm
\m{\vdash}\m{(}\m{(}\m{A}\m{\in}\m{\mathbb{R}}\m{\wedge}\m{B}\m{\in}\m{\mathbb{R}}%
\m{)}\m{\rightarrow}\m{(}\m{(}\m{0}\m{<}\m{A}\m{\wedge}\m{0}%
\m{<}\m{B}\m{)}\m{\rightarrow}\m{0}\m{<}\m{(}\m{A}\m{\cdot}\m{B}\m{)}%
\m{)}\m{)}
\endm
%\vskip 1ex

\noindent 22. A non-empty, bounded-above set of reals has a supremum.

%\vskip 0.5ex
\setbox\startprefix=\hbox{\tt \ \ ax-pre-sup\ \$p\ }
\setbox\contprefix=\hbox{\tt \ \ \ \ \ \ \ \ \ \ \ }
\startm
\m{\vdash}\m{(}\m{(}\m{A}\m{\subseteq}\m{\mathbb{R}}\m{\wedge}\m{A}\m{\ne}\m{%
\varnothing}\m{\wedge}\m{\exists}\m{x}\m{\in}\m{\mathbb{R}}\m{\forall}\m{y}\m{%
\in}\m{A}\m{\,y}\m{<}\m{x}\m{)}\m{\rightarrow}\m{\exists}\m{x}\m{\in}\m{%
\mathbb{R}}\m{(}\m{\forall}\m{y}\m{\in}\m{A}\m{\lnot}\m{x}\m{<}\m{y}\m{\wedge}\m{%
\forall}\m{y}\m{\in}\m{\mathbb{R}}\m{(}\m{y}\m{<}\m{x}\m{\rightarrow}\m{\exists}%
\m{z}\m{\in}\m{A}\m{\,y}\m{<}\m{z}\m{)}\m{)}\m{)}
\endm

% NOTE: The \m{...} expressions above could be represented as
% $ \vdash ( ( A \subseteq \mathbb{R} \wedge A \ne \varnothing \wedge \exists x \in \mathbb{R} \forall y \in A \,y < x ) \rightarrow \exists x \in \mathbb{R} ( \forall y \in A \lnot x < y \wedge \forall y \in \mathbb{R} ( y < x \rightarrow \exists z \in A \,y < z ) ) ) $

\vskip 2ex

This completes the set of axioms for real and complex numbers.  You may
wish to look at how subtraction, division, and decimal numbers
are defined in \texttt{set.mm}, and for fun look at the proof of $2+
2 = 4$ (theorem \texttt{2p2e4} in \texttt{set.mm})
as discussed in section \ref{2p2e4}.

In \texttt{set.mm} we define the non-negative integers $\mathbb{N}$, the integers
$\mathbb{Z}$, and the rationals $\mathbb{Q}$ as subsets of $\mathbb{R}$.  This leads
to the nice inclusion $\mathbb{N} \subseteq \mathbb{Z} \subseteq \mathbb{Q} \subseteq
\mathbb{R} \subseteq \mathbb{C}$, giving us a uniform framework in which, for
example, a property such as commutativity of complex number addition
automatically applies to integers.  The natural numbers $\mathbb{N}$
are different from the set $\omega$ we defined earlier, but both satisfy
Peano's postulates.

\subsubsection{Complex Number Axioms in Analysis Texts}

Most analysis texts construct complex numbers as ordered pairs of reals,
leading to construction-dependent properties that satisfy these axioms
but are not stated in their pure form.  (This is also done in
\texttt{set.mm} but our axioms are extracted from that construction.)
Other texts will simply state that $\mathbb{R}$ is a ``complete ordered
subfield of $\mathbb{C}$,'' leading to redundant axioms when this phrase
is completely expanded out.  In fact I have not seen a text with the
axioms in the explicit form above.
None of these axioms is unique individually, but this carefully worked out
collection of axioms is the result of years of work
by the Metamath community.

\subsubsection{Eliminating Unnecessary Complex Number Axioms}

We once had more axioms for real and complex numbers, but over years of time
we (the Metamath community)
have found ways to eliminate them (by proving them from other axioms)
or weaken them (by making weaker claims without reducing what
can be proved).
In particular, here are statements that used to be complex number
axioms but have since been formally proven (with Metamath) to be redundant:

\begin{itemize}
\item
  $\mathbb{C} \in V$.
  At one time this was listed as a ``complex number axiom''
  However, this is not properly speaking a complex number axiom,
  and in any case its proof uses axioms of set theory.
  Proven redundant by Mario Carneiro\index{Carneiro, Mario} on
  17-Nov-2014 (see \texttt{axcnex}).
\item
  $((A \in \mathbb{C} \land B \in \mathbb{C}$) $\rightarrow$
  $(A + B) = (B + A))$.
  Proved redundant by Eric Schmidt\index{Schmidt, Eric} on 19-Jun-2012,
  and formalized by Scott Fenton\index{Fenton, Scott} on 3-Jan-2013
  (see \texttt{addcom}).
\item
  $(A \in \mathbb{C} \rightarrow (A + 0) = A)$.
  Proved redundant by Eric Schmidt on 19-Jun-2012,
  and formalized by Scott Fenton on 3-Jan-2013
  (see \texttt{addid1}).
\item
  $(A \in \mathbb{C} \rightarrow \exists x \in \mathbb{C} (A + x) = 0)$.
  Proved redundant by Eric Schmidt and formalized on 21-May-2007
  (see \texttt{cnegex}).
\item
  $((A \in \mathbb{C} \land A \ne 0) \rightarrow \exists x \in \mathbb{C} (A \cdot x) = 1)$.
  Proved redundant by Eric Schmidt and formalized on 22-May-2007
  (see \texttt{recex}).
\item
  $0 \in \mathbb{R}$.
  Proved redundant by Eric Schmidt on 19-Feb-2005 and formalized 21-May-2007
  (see \texttt{0re}).
\end{itemize}

We could eliminate 0 as an axiomatic object by defining it as
$( ( i \cdot i ) + 1 )$
and replacing it with this expression throughout the axioms. If this
is done, axiom ax-i2m1 becomes redundant. However, the remaining axioms
would become longer and less intuitive.

Eric Schmidt's paper analyzing this axiom system \cite{Schmidt}
presented a proof that these remaining axioms,
with the possible exception of ax-mulcom, are independent of the others.
It is currently an open question if ax-mulcom is independent of the others.

\section{Two Plus Two Equals Four}\label{2p2e4}

Here is a proof that $2 + 2 = 4$, as proven in the theorem \texttt{2p2e4}
in the database \texttt{set.mm}.
This is a useful demonstration of what a Metamath proof can look like.
This proof may have more steps than you're used to, but each step is rigorously
proven all the way back to the axioms of logic and set theory.
This display was originally generated by the Metamath program
as an {\sc HTML} file.

In the table showing the proof ``Step'' is the sequential step number,
while its associated ``Expression'' is an expression that we have proved.
``Ref'' is the name of a theorem or axiom that justifies that expression,
and ``Hyp'' refers to previous steps (if any) that the theorem or axiom
needs so that we can use it.  Expressions are indented further than
the expressions that depend on them to show their interdependencies.

\begin{table}[!htbp]
\caption{Two plus two equals four}
\begin{tabular}{lllll}
\textbf{Step} & \textbf{Hyp} & \textbf{Ref} & \textbf{Expression} & \\
1  &       & df-2    & $ \; \; \vdash 2 = 1 + 1$  & \\
2  & 1     & oveq2i  & $ \; \vdash (2 + 2) = (2 + (1 + 1))$ & \\
3  &       & df-4    & $ \; \; \vdash 4 = (3 + 1)$ & \\
4  &       & df-3    & $ \; \; \; \vdash 3 = (2 + 1)$ & \\
5  & 4     & oveq1i  & $ \; \; \vdash (3 + 1) = ((2 + 1) + 1)$ & \\
6  &       & 2cn     & $ \; \; \; \vdash 2 \in \mathbb{C}$ & \\
7  &       & ax-1cn  & $ \; \; \; \vdash 1 \in \mathbb{C}$ & \\
8  & 6,7,7 & addassi & $ \; \; \vdash ((2 + 1) + 1) = (2 + (1 + 1))$ & \\
9  & 3,5,8 & 3eqtri  & $ \; \vdash 4 = (2 + (1 + 1))$ & \\
10 & 2,9   & eqtr4i  & $ \vdash (2 + 2) = 4$ & \\
\end{tabular}
\end{table}

Step 1 says that we can assert that $2 = 1 + 1$ because it is
justified by \texttt{df-2}.
What is \texttt{df-2}?
It is simply the definition of $2$, which in our system is defined as being
equal to $1 + 1$.  This shows how we can use definitions in proofs.

Look at Step 2 of the proof. In the Ref column, we see that it references
a previously proved theorem, \texttt{oveq2i}.
It turns out that
theorem \texttt{oveq2i} requires a
hypothesis, and in the Hyp column of Step 2 we indicate that Step 1 will
satisfy (match) this hypothesis.
If we looked at \texttt{oveq2i}
we would find that it proves that given some hypothesis
$A = B$, we can prove that $( C F A ) = ( C F B )$.
If we use \texttt{oveq2i} and apply step 1's result as the hypothesis,
that will mean that $A = 2$ and $B = ( 1 + 1 )$ within this use of
\texttt{oveq2i}.
We are free to select any value of $C$ and $F$ (subject to syntax constraints),
so we are free to select $C = 2$ and $F = +$,
producing our desired result,
$ (2 + 2) = (2 + (1 + 1))$.

Step 2 is an example of substitution.
In the end, every step in every proof uses only this one substitution rule.
All the rules of logic, and all the axioms, are expressed so that
they can be used via this one substitution rule.
So once you master substitution, you can master every Metamath proof,
no exceptions.

Each step is clear and can be immediately checked.
In the {\sc HTML} display you can even click on each reference to see why it is
justified, making it easy to see why the proof works.

\section{Deduction}\label{deduction}

Strictly speaking,
a deduction (also called an inference) is a kind of statement that needs
some hypotheses to be true in order for its conclusion to be true.
A theorem, on the other hand, has no hypotheses.
Informally we often call both of them theorems, but in this section we
will stick to the strict definitions.

It sometimes happens that we have proved a deduction of the form
$\varphi \Rightarrow \psi$\index{$\Rightarrow$}
(given hypothesis $\varphi$ we can prove $\psi$)
and we want to then prove a theorem of the form
$\varphi \rightarrow \psi$.

Converting a deduction (which uses a hypothesis) into a theorem
(which does not) is not as simple as you might think.
The deduction says, ``if we can prove $\varphi$ then we can prove $\psi$,''
which is in some sense weaker than saying
``$\varphi$ implies $\psi$.''
There is no axiom of logic that permits us to directly obtain the theorem
given the deduction.\footnote{
The conversion of a deduction to a theorem does not even hold in general
for quantum propositional calculus,
which is a weak subset of classical propositional calculus.
It has been shown that adding the Standard Deduction Theorem (discussed below)
to quantum propositional calculus turns it into classical
propositional calculus!
}

This is in contrast to going the other way.
If we have the theorem ($\varphi \rightarrow \psi$),
it is easy to recover the deduction
($\varphi \Rightarrow \psi$)
using modus ponens\index{modus ponens}
(\texttt{ax-mp}; see section \ref{axmp}).

In the following subsections we first discuss the standard deduction theorem
(the traditional but awkward way to convert deductions into theorems) and
the weak deduction theorem (a limited version of the standard deduction
theorem that is easier to use and was once widely used in
\texttt{set.mm}\index{set theory database (\texttt{set.mm})}).
In section \ref{deductionstyle} we discuss
deduction style, the newer approach we now recommend in most cases.
Deduction style uses ``deduction form,'' a form that
prefixes each hypothesis (other than definitions) and the
conclusion with a universal antecedent (``$\varphi \rightarrow$'').
Deduction style is widely used in \texttt{set.mm},
so it is useful to understand it and \textit{why} it is widely used.
Section \ref{naturaldeduction}
briefly discusses our approach for using natural deduction
within \texttt{set.mm},
as that approach is deeply related to deduction style.
We conclude with a summary of the strengths of
our approach, which we believe are compelling.

\subsection{The Standard Deduction Theorem}\label{standarddeductiontheorem}

It is possible to make use of information
contained in the deduction or its proof to assist us with the proof of
the related theorem.
In traditional logic books, there is a metatheorem called the
Deduction Theorem\index{Deduction Theorem}\index{Standard Deduction Theorem},
discovered independently by Herbrand and Tarski around 1930.
The Deduction Theorem, which we often call the Standard Deduction Theorem,
provides an algorithm for constructing a proof of a theorem from the
proof of its corresponding deduction. See, for example,
\cite[p.~56]{Margaris}\index{Margaris, Angelo}.
To construct a proof for a theorem, the
algorithm looks at each step in the proof of the original deduction and
rewrites the step with several steps wherein the hypothesis is eliminated
and becomes an antecedent.

In ordinary mathematics, no one actually carries out the algorithm,
because (in its most basic form) it involves an exponential explosion of
the number of proof steps as more hypotheses are eliminated. Instead,
the Standard Deduction Theorem is invoked simply to claim that it can
be done in principle, without actually doing it.
What's more, the algorithm is not as simple as it might first appear
when applying it rigorously.
There is a subtle restriction on the Standard Deduction Theorem
that must be taken into account involving the axiom of generalization
when working with predicate calculus (see the literature for more detail).

One of the goals of Metamath is to let you plainly see, with as few
underlying concepts as possible, how mathematics can be derived directly
from the axioms, and not indirectly according to some hidden rules
buried inside a program or understood only by logicians. If we added
the Standard Deduction Theorem to the language and proof verifier,
that would greatly complicate both and largely defeat Metamath's goal
of simplicity. In principle, we could show direct proofs by expanding
out the proof steps generated by the algorithm of the Standard Deduction
Theorem, but that is not feasible in practice because the number of proof
steps quickly becomes huge, even astronomical.
Since the algorithm of the Standard Deduction Theorem is driven by the proof,
we would have to go through that proof
all over again---starting from axioms---in order to obtain the theorem form.
In terms of proof length, there would be no savings over just
proving the theorem directly instead of first proving the deduction form.

\subsection{Weak Deduction Theorem}\label{weakdeductiontheorem}

We have developed
a more efficient method for proving a theorem from a deduction
that can be used instead of the Standard Deduction Theorem
in many (but not all) cases.
We call this more efficient method the
Weak Deduction Theorem\index{Weak Deduction Theorem}.\footnote{
There is also an unrelated ``Weak Deduction Theorem''
in the field of relevance logic, so to avoid confusion we could call
ours the ``Weak Deduction Theorem for Classical Logic.''}
Unlike the Standard Deduction Theorem, the Weak Deduction Theorem produces the
theorem directly from a special substitution instance of the deduction,
using a small, fixed number of steps roughly proportional to the length
of the final theorem.

If you come to a proof referencing the Weak Deduction Theorem
\texttt{dedth} (or one of its variants \texttt{dedthxx}),
here is how to follow the proof without getting into the details:
just click on the theorem referenced in the step
just before the reference to \texttt{dedth} and ignore everything else.
Theorem \texttt{dedth} simply turns a hypothesis into an antecedent
(i.e. the hypothesis followed by $\rightarrow$
is placed in front of the assertion, and the hypothesis
itself is eliminated) given certain conditions.

The Weak Deduction Theorem
eliminates a hypothesis $\varphi$, making it become an antecedent.
It does this by proving an expression
$ \varphi \rightarrow \psi $ given two hypotheses:
(1)
$ ( A = {\rm if} ( \varphi , A , B ) \rightarrow ( \varphi \leftrightarrow \chi ) ) $
and
(2) $\chi$.
Note that it requires that a proof exists for $\varphi$ when the class variable
$A$ is replaced with a specific class $B$. The hypothesis $\chi$
should be assigned to the inference.
You can see the details of the proof of the Weak Deduction Theorem
in theorem \texttt{dedth}.

The Weak Deduction Theorem
is probably easier to understand by studying proofs that make use of it.
For example, let's look at the proof of \texttt{renegcl}, which proves that
$ \vdash ( A \in \mathbb{R} \rightarrow - A \in \mathbb{R} )$:

\needspace{4\baselineskip}
\begin{longtabu} {l l l X}
\textbf{Step} & \textbf{Hyp} & \textbf{Ref} & \textbf{Expression} \\
  1 &  & negeq &
  $\vdash$ $($ $A$ $=$ ${\rm if}$ $($ $A$ $\in$ $\mathbb{R}$ $,$ $A$ $,$ $1$ $)$ $\rightarrow$
  $\textrm{-}$ $A$ $=$ $\textrm{-}$ ${\rm if}$ $($ $A$ $\in$ $\mathbb{R}$
  $,$ $A$ $,$ $1$ $)$ $)$ \\
 2 & 1 & eleq1d &
    $\vdash$ $($ $A$ $=$ ${\rm if}$ $($ $A$ $\in$ $\mathbb{R}$ $,$ $A$ $,$ $1$ $)$ $\rightarrow$ $($
    $\textrm{-}$ $A$ $\in$ $\mathbb{R}$ $\leftrightarrow$
    $\textrm{-}$ ${\rm if}$ $($ $A$ $\in$ $\mathbb{R}$ $,$ $A$ $,$ $1$ $)$ $\in$
    $\mathbb{R}$ $)$ $)$ \\
 3 &  & 1re & $\vdash 1 \in \mathbb{R}$ \\
 4 & 3 & elimel &
   $\vdash {\rm if} ( A \in \mathbb{R} , A , 1 ) \in \mathbb{R}$ \\
 5 & 4 & renegcli &
   $\vdash \textrm{-} {\rm if} ( A \in \mathbb{R} , A , 1 ) \in \mathbb{R}$ \\
 6 & 2,5 & dedth &
   $\vdash ( A \in \mathbb{R} \rightarrow \textrm{-} A \in \mathbb{R}$ ) \\
\end{longtabu}

The somewhat strange-looking steps in \texttt{renegcl} before step 5 are
technical stuff that makes this magic work, and they can be ignored
for a quick overview of the proof. To continue following the ``important''
part of the proof of \texttt{renegcl},
you can look at the reference to \texttt{renegcli} at step 5.

That said, let's briefly look at how
\texttt{renegcl} uses the
Weak Deduction Theorem (\texttt{dedth}) to do its job,
in case you want to do something similar or want understand it more deeply.
Let's work backwards in the proof of \texttt{renegcl}.
Step 6 applies \texttt{dedth} to produce our goal result
$ \vdash ( A \in \mathbb{R} \rightarrow\, - A \in \mathbb{R} )$.
This requires on the one hand the (substituted) deduction
\texttt{renegcli} in step 5.
By itself \texttt{renegcli} proves the deduction
$ \vdash A \in \mathbb{R} \Rightarrow\, \vdash - A \in \mathbb{R}$;
this is the deduction form we are trying to turn into theorem form,
and thus
\texttt{renegcli} has a separate hypothesis that must be fulfilled.
To fulfill the hypothesis of the invocation of
\texttt{renegcli} in step 5, it is eventually
reduced to the already proven theorem $1 \in \mathbb{R}$ in step 3.
Step 4 connects steps 3 and 5; step 4 invokes
\texttt{elimel}, a special case of \texttt{elimhyp} that eliminates
a membership hypothesis for the weak deduction theorem.
On the other hand, the equivalence of the conclusion of
\texttt{renegcl}
$( - A \in \mathbb{R} )$ and the substituted conclusion of
\texttt{renegcli} must be proven, which is done in steps 2 and 1.

The weak deduction theorem has limitations.
In particular, we must be able to prove a special case of the deduction's
hypothesis as a stand-alone theorem.
For example, we used $1 \in \mathbb{R}$ in step 3 of \texttt{renegcl}.

We used to use the weak deduction theorem
extensively within \texttt{set.mm}.
However, we now recommend applying ``deduction style''
instead in most cases, as deduction style is
often an easier and clearer approach.
Therefore, we will now describe deduction style.

\subsection{Deduction Style}\label{deductionstyle}

We now prefer to write assertions in ``deduction form''
instead of writing a proof that would require use of the standard or
weak deduction theorem.
We call this appraoch
``deduction style.''\index{deduction style}

It will be easier to explain this by first defining some terms:

\begin{itemize}
\item \textbf{closed form}\index{closed form}\index{forms!closed}:
A kind of assertion (theorem) with no hypotheses.
Typically its label has no special suffix.
An example is \texttt{unss}, which states:
$\vdash ( ( A \subseteq C \wedge B \subseteq C ) \leftrightarrow ( A \cup B )
\subseteq C )\label{eq:unss}$
\item \textbf{deduction form}\index{deduction form}\index{forms!deduction}:
A kind of assertion with one or more hypotheses
where the conclusion is an implication with
a wff variable as the antecedent (usually $\varphi$), and every hypothesis
(\$e statement)
is either (1) an implication with the same antecedent as the conclusion or
(2) a definition.
A definition
can be for a class variable (this is a class variable followed by ``='')
or a wff variable (this is a wff variable followed by $\leftrightarrow$);
class variable definitions are more common.
In practice, a proof
in deduction form will also contain many steps that are implications
where the antecedent is either that wff variable (normally $\varphi$)
or is
a conjunction (...$\land$...) including that wff variable ($\varphi$).
If an assertion is in deduction form, and other forms are also available,
then we suffix its label with ``d.''
An example is \texttt{unssd}, which states\footnote{
For brevity we show here (and in other places)
a $\&$\index{$\&$} between hypotheses\index{hypotheses}
and a $\Rightarrow$\index{$\Rightarrow$}\index{conclusion}
between the hypotheses and the conclusion.
This notation is technically not part of the Metamath language, but is
instead a convenient abbreviation to show both the hypotheses and conclusion.}:
$\vdash ( \varphi \rightarrow A \subseteq C )\quad\&\quad \vdash ( \varphi
    \rightarrow B \subseteq C )\quad\Rightarrow\quad \vdash ( \varphi
    \rightarrow ( A \cup B ) \subseteq C )\label{eq:unssd}$
\item \textbf{inference form}\index{inference form}\index{forms!inference}:
A kind of assertion with one or more hypotheses that is not in deduction form
(e.g., there is no common antecedent).
If an assertion is in inference form, and other forms are also available,
then we suffix its label with ``i.''
An example is \texttt{unssi}, which states:
$\vdash A \subseteq C\quad\&\quad \vdash B \subseteq C\quad\Rightarrow\quad
    \vdash ( A \cup B ) \subseteq C\label{eq:unssi}$
\end{itemize}

When using deduction style we express an assertion in deduction form.
This form prefixes each hypothesis (other than definitions) and the
conclusion with a universal antecedent (``$\varphi \rightarrow$'').
The antecedent (e.g., $\varphi$)
mimics the context handled in the deduction theorem, eliminating
the need to directly use the deduction theorem.

Once you have an assertion in deduction form, you can easily convert it
to inference form or closed form:

\begin{itemize}
\item To
prove some assertion Ti in inference form, given assertion Td in deduction
form, there is a simple mechanical process you can use. First take each
Ti hypothesis and insert a \texttt{T.} $\rightarrow$ prefix (``true implies'')
using \texttt{a1i}. You
can then use the existing assertion Td to prove the resulting conclusion
with a \texttt{T.} $\rightarrow$ prefix.
Finally, you can remove that prefix using \texttt{trud},
resulting in the conclusion you wanted to prove.
\item To
prove some assertion T in closed form, given assertion Td in deduction
form, there is another simple mechanical process you can use. First,
select an expression that is the conjunction (...$\land$...) of all of the
consequents of every hypothesis of Td. Next, prove that this expression
implies each of the separate hypotheses of Td in turn by eliminating
conjuncts (there are a variety of proven assertions to do this, including
\texttt{simpl},
\texttt{simpr},
\texttt{3simpa},
\texttt{3simpb},
\texttt{3simpc},
\texttt{simp1},
\texttt{simp2},
and
\texttt{simp3}).
If the
expression has nested conjunctions, inner conjuncts can be broken out by
chaining the above theorems with \texttt{syl}
(see section \ref{syl}).\footnote{
There are actually many theorems
(labeled simp* such as \texttt{simp333}) that break out inner conjuncts in one
step, but rather than learning them you can just use the chaining we
just described to prove them, and then let the Metamath program command
\texttt{minimize{\char`\_}with}\index{\texttt{minimize{\char`\_}with} command}
figure out the right ones needed to collapse them.}
As your final step, you can then apply the already-proven assertion Td
(which is in deduction form), proving assertion T in closed form.
\end{itemize}

We can also easily convert any assertion T in closed form to its related
assertion Ti in inference form by applying
modus ponens\index{modus ponens} (see section \ref{axmp}).

The deduction form antecedent can also be used to represent the context
necessary to support natural deduction systems, so we will now
discuss natural deduction.

\subsection{Natural Deduction}\label{naturaldeduction}

Natural deduction\index{natural deduction}
(ND) systems, as such, were originally introduced in
1934 by two logicians working independently: Ja\'skowski and Gentzen. ND
systems are supposed to reconstruct, in a formally proper way, traditional
ways of mathematical reasoning (such as conditional proof, indirect proof,
and proof by cases). As reconstructions they were naturally influenced
by previous work, and many specific ND systems and notations have been
developed since their original work.

There are many ND variants, but
Indrzejczak \cite[p.~31-32]{Indrzejczak}\index{Indrzejczak, Andrzej}
suggests that any natural deductive system must satisfy at
least these three criteria:

\begin{itemize}
\item ``There are some means for entering assumptions into a proof and
also for eliminating them. Usually it requires some bookkeeping devices
for indicating the scope of an assumption, and showing that a part of
a proof depending on eliminated assumption is discharged.
\item There are no (or, at least, a very limited set of) axioms, because
their role is taken over by the set of primitive rules for introduction
and elimination of logical constants which means that elementary
inferences instead of formulae are taken as primitive.
\item (A genuine) ND system admits a lot of freedom in proof construction
and possibility of applying several proof search strategies, like
conditional proof, proof by cases, proof by reductio ad absurdum etc.''
\end{itemize}

The Metamath Proof Explorer (MPE) as defined in \texttt{set.mm}
is fundamentally a Hilbert-style system.
That is, MPE is based on a larger number of axioms (compared
to natural deduction systems), a very small set of rules of inference
(modus ponens), and the context is not changed by the rules of inference
in the middle of a proof. That said, MPE proofs can be developed using
the natural deduction (ND) approach as originally developed by Ja\'skowski
and Gentzen.

The most common and recommended approach for applying ND in MPE is to use
deduction form\index{deduction form}%
\index{forms!deduction}
and apply the MPE proven assertions that are equivalent to ND rules.
For example, MPE's \texttt{jca} is equivalent to ND rule $\land$-I
(and-insertion).
We maintain a list of equivalences that you may consult.
This approach for applying an ND approach within MPE relies on Metamath's
wff metavariables in an essential way, and is described in more detail
in the presentation ``Natural Deductions in the Metamath Proof Language''
by Mario Carneiro \cite{CarneiroND}\index{Carneiro, Mario}.

In this style many steps are an implication, whose antecedent mimics
the context ($\Gamma$) of most ND systems. To add an assumption, simply add
it to the implication antecedent (typically using
\texttt{simpr}),
and use that
new antecedent for all later claims in the same scope. If you wish to
use an assertion in an ND hypothesis scope that is outside the current
ND hypothesis scope, modify the assertion so that the ND hypothesis
assumption is added to its antecedent (typically using \texttt{adantr}). Most
proof steps will be proved using rules that have hypotheses and results
of the form $\varphi \rightarrow$ ...

An example may make this clearer.
Let's show theorem 5.5 of
\cite[p.~18]{Clemente}\index{Clemente Laboreo, Daniel}
along with a line by line translation using the usual
translation of natural deduction (ND) in the Metamath Proof Explorer
(MPE) notation (this is proof \texttt{ex-natded5.5}).
The proof's original goal was to prove
$\lnot \psi$ given two hypotheses,
$( \psi \rightarrow \chi )$ and $ \lnot \chi$.
We will translate these statements into MPE deduction form
by prefixing them all with $\varphi \rightarrow$.
As a result, in MPE the goal is stated as
$( \varphi \rightarrow \lnot \psi )$, and the two hypotheses are stated as
$( \varphi \rightarrow ( \psi \rightarrow \chi ) )$ and
$( \varphi \rightarrow \lnot \chi )$.

The following table shows the proof in Fitch natural deduction style
and its MPE equivalent.
The \textit{\#} column shows the original numbering,
\textit{MPE\#} shows the number in the equivalent MPE proof
(which we will show later),
\textit{ND Expression} shows the original proof claim in ND notation,
and \textit{MPE Translation} shows its translation into MPE
as discussed in this section.
The final columns show the rationale in ND and MPE respectively.

\needspace{4\baselineskip}
{\setlength{\extrarowsep}{4pt} % Keep rows from being too close together
\begin{longtabu}   { @{} c c X X X X }
\textbf{\#} & \textbf{MPE\#} & \textbf{ND Ex\-pres\-sion} &
\textbf{MPE Trans\-lation} & \textbf{ND Ration\-ale} &
\textbf{MPE Ra\-tio\-nale} \\
\endhead

1 & 2;3 &
$( \psi \rightarrow \chi )$ &
$( \varphi \rightarrow ( \psi \rightarrow \chi ) )$ &
Given &
\$e; \texttt{adantr} to put in ND hypothesis \\

2 & 5 &
$ \lnot \chi$ &
$( \varphi \rightarrow \lnot \chi )$ &
Given &
\$e; \texttt{adantr} to put in ND hypothesis \\

3 & 1 &
... $\vert$ $\psi$ &
$( \varphi \rightarrow \psi )$ &
ND hypothesis assumption &
\texttt{simpr} \\

4 & 4 &
... $\chi$ &
$( ( \varphi \land \psi ) \rightarrow \chi )$ &
$\rightarrow$\,E 1,3 &
\texttt{mpd} 1,3 \\

5 & 6 &
... $\lnot \chi$ &
$( ( \varphi \land \psi ) \rightarrow \lnot \chi )$ &
IT 2 &
\texttt{adantr} 5 \\

6 & 7 &
$\lnot \psi$ &
$( \varphi \rightarrow \lnot \psi )$ &
$\land$\,I 3,4,5 &
\texttt{pm2.65da} 4,6 \\

\end{longtabu}
}


The original used Latin letters; we have replaced them with Greek letters
to follow Metamath naming conventions and so that it is easier to follow
the Metamath translation. The Metamath line-for-line translation of
this natural deduction approach precedes every line with an antecedent
including $\varphi$ and uses the Metamath equivalents of the natural deduction
rules. To add an assumption, the antecedent is modified to include it
(typically by using \texttt{adantr};
\texttt{simpr} is useful when you want to
depend directly on the new assumption, as is shown here).

In Metamath we can represent the two given statements as these hypotheses:

\needspace{2\baselineskip}
\begin{itemize}
\item ex-natded5.5.1 $\vdash ( \varphi \rightarrow ( \psi \rightarrow \chi ) )$
\item ex-natded5.5.2 $\vdash ( \varphi \rightarrow \lnot \chi )$
\end{itemize}

\needspace{4\baselineskip}
Here is the proof in Metamath as a line-by-line translation:

\begin{longtabu}   { l l l X }
\textbf{Step} & \textbf{Hyp} & \textbf{Ref} & \textbf{Ex\-pres\-sion} \\
\endhead
1 & & simpr & $\vdash ( ( \varphi \land \psi ) \rightarrow \psi )$ \\
2 & & ex-natded5.5.1 &
  $\vdash ( \varphi \rightarrow ( \psi \rightarrow \chi ) )$ \\
3 & 2 & adantr &
 $\vdash ( ( \varphi \land \psi ) \rightarrow ( \psi \rightarrow \chi ) )$ \\
4 & 1, 3 & mpd &
 $\vdash ( ( \varphi \land \psi ) \rightarrow \chi ) $ \\
5 & & ex-natded5.5.2 &
 $\vdash ( \varphi \rightarrow \lnot \chi )$ \\
6 & 5 & adantr &
 $\vdash ( ( \varphi \land \psi ) \rightarrow \lnot \chi )$ \\
7 & 4, 6 & pm2.65da &
 $\vdash ( \varphi \rightarrow \lnot \psi )$ \\
\end{longtabu}

Only using specific natural deduction rules directly can lead to very
long proofs, for exactly the same reason that only using axioms directly
in Hilbert-style proofs can lead to very long proofs.
If the goal is short and clear proofs,
then it is better to reuse already-proven assertions
in deduction form than to start from scratch each time
and using only basic natural deduction rules.

\subsection{Strengths of Our Approach}

As far as we know there is nothing else in the literature like either the
weak deduction theorem or Mario Carneiro\index{Carneiro, Mario}'s
natural deduction method.
In order to
transform a hypothesis into an antecedent, the literature's standard
``Deduction Theorem''\index{Deduction Theorem}\index{Standard Deduction Theorem}
requires metalogic outside of the notions provided
by the axiom system. We instead generally prefer to use Mario Carneiro's
natural deduction method, then use the weak deduction theorem in cases
where that is difficult to apply, and only then use the full standard
deduction theorem as a last resort.

The weak deduction theorem\index{Weak Deduction Theorem}
does not require any additional metalogic
but converts an inference directly into a closed form theorem, with
a rigorous proof that uses only the axiom system. Unlike the standard
Deduction Theorem, there is no implicit external justification that we
have to trust in order to use it.

Mario Carneiro's natural deduction\index{natural deduction}
method also does not require any new metalogical
notions. It avoids the Deduction Theorem's metalogic by prefixing the
hypotheses and conclusion of every would-be inference with a universal
antecedent (``$\varphi \rightarrow$'') from the very start.

We think it is impressive and satisfying that we can do so much in a
practical sense without stepping outside of our Hilbert-style axiom system.
Of course our axiomatization, which is in the form of schemes,
contains a metalogic of its own that we exploit. But this metalogic
is relatively simple, and for our Deduction Theorem alternatives,
we primarily use just the direct substitution of expressions for
metavariables.

\begin{sloppy}
\section{Exploring the Set The\-o\-ry Data\-base}\label{exploring}
\end{sloppy}
% NOTE: All examples performed in this section are
% recorded wtih "set width 61" % on set.mm as of 2019-05-28
% commit c1e7849557661260f77cfdf0f97ac4354fbb4f4d.

At this point you may wish to study the \texttt{set.mm}\index{set theory
database (\texttt{set.mm})} file in more detail.  Pay particular
attention to the assumptions needed to define wffs\index{well-formed
formula (wff)} (which are not included above), the variable types
(\texttt{\$f}\index{\texttt{\$f} statement} statements), and the
definitions that are introduced.  Start with some simple theorems in
propositional calculus, making sure you understand in detail each step
of a proof.  Once you get past the first few proofs and become familiar
with the Metamath language, any part of the \texttt{set.mm} database
will be as easy to follow, step by step, as any other part---you won't
have to undergo a ``quantum leap'' in mathematical sophistication to be
able to follow a deep proof in set theory.

Next, you may want to explore how concepts such as natural numbers are
defined and described.  This is probably best done in conjunction with
standard set theory textbooks, which can help give you a higher-level
understanding.  The \texttt{set.mm} database provides references that will get
you started.  From there, you will be on your way towards a very deep,
rigorous understanding of abstract mathematics.

The Metamath\index{Metamath} program can help you peruse a Metamath data\-base,
wheth\-er you are trying to figure out how a certain step follows in a proof or
just have a general curiosity.  We will go through some examples of the
commands, using the \texttt{set.mm}\index{set theory database (\texttt{set.mm})}
database provided with the Metamath software.  These should help get you
started.  See Chapter~\ref{commands} for a more detailed description of
the commands.  Note that we have included the full spelling of all commands to
prevent ambiguity with future commands.  In practice you may type just the
characters needed to specify each command keyword\index{command keyword}
unambiguously, often just one or two characters per keyword, and you don't
need to type them in upper case.

First run the Metamath program as described earlier.  You should see the
\verb/MM>/ prompt.  Read in the \texttt{set.mm} file:\index{\texttt{read}
command}

\begin{verbatim}
MM> read set.mm
Reading source file "set.mm"... 34554442 bytes
34554442 bytes were read into the source buffer.
The source has 155711 statements; 2254 are $a and 32250 are $p.
No errors were found.  However, proofs were not checked.
Type VERIFY PROOF * if you want to check them.
\end{verbatim}

As with most examples in this book, what you will see
will be slightly different because we are continuously
improving our databases (including \texttt{set.mm}).

Let's check the database integrity.  This may take a minute or two to run if
your computer is slow.

\begin{verbatim}
MM> verify proof *
0 10%  20%  30%  40%  50%  60%  70%  80%  90% 100%
..................................................
All proofs in the database were verified in 2.84 s.
\end{verbatim}

No errors were reported, so every proof is correct.

You need to know the names (labels) of theorems before you can look at them.
Often just examining the database file(s) with a text editor is the best
approach.  In \texttt{set.mm} there are many detailed comments, especially near
the beginning, that can help guide you. The \texttt{search} command in the
Metamath program is also handy.  The \texttt{comments} qualifier will list the
statements whose associated comment (the one immediately before it) contain a
string you give it.  For example, if you are studying Enderton's {\em Elements
of Set Theory} \cite{Enderton}\index{Enderton, Herbert B.} you may want to see
the references to it in the database.  The search string \texttt{enderton} is not
case sensitive.  (This will not show you all the database theorems that are in
Enderton's book because there is usually only one citation for a given
theorem, which may appear in several textbooks.)\index{\texttt{search}
command}

\begin{verbatim}
MM> search * "enderton" / comments
12067 unineq $p "... Exercise 20 of [Enderton] p. 32 and ..."
12459 undif2 $p "...Corollary 6K of [Enderton] p. 144. (C..."
12953 df-tp $a "...s. Definition of [Enderton] p. 19. (Co..."
13689 unissb $p ".... Exercise 5 of [Enderton] p. 26 and ..."
\end{verbatim}
\begin{center}
(etc.)
\end{center}

Or you may want to see what theorems have something to do with
conjunction (logical {\sc and}).  The quotes around the search
string are optional when there's no ambiguity.\index{\texttt{search}
command}

\begin{verbatim}
MM> search * conjunction / comments
120 a1d $p "...be replaced with a conjunction ( ~ df-an )..."
662 df-bi $a "...viated form after conjunction is introdu..."
1319 wa $a "...ff definition to include conjunction ('and')."
1321 df-an $a "Define conjunction (logical 'and'). Defini..."
1420 imnan $p "...tion in terms of conjunction. (Contribu..."
\end{verbatim}
\begin{center}
(etc.)
\end{center}


Now we will start to look at some details.  Let's look at the first
axiom of propositional calculus
(we could use \texttt{sh st} to abbreviate
\texttt{show statement}).\index{\texttt{show statement} command}

\begin{verbatim}
MM> show statement ax-1/full
Statement 19 is located on line 881 of the file "set.mm".
"Axiom _Simp_.  Axiom A1 of [Margaris] p. 49.  One of the 3
axioms of propositional calculus.  The 3 axioms are also
        ...
19 ax-1 $a |- ( ph -> ( ps -> ph ) ) $.
Its mandatory hypotheses in RPN order are:
  wph $f wff ph $.
  wps $f wff ps $.
The statement and its hypotheses require the variables:  ph
      ps
The variables it contains are:  ph ps


Statement 49 is located on line 11182 of the file "set.mm".
Its statement number for HTML pages is 6.
"Axiom _Simp_.  Axiom A1 of [Margaris] p. 49.  One of the 3
axioms of propositional calculus.  The 3 axioms are also
given as Definition 2.1 of [Hamilton] p. 28.
...
49 ax-1 $a |- ( ph -> ( ps -> ph ) ) $.
Its mandatory hypotheses in RPN order are:
  wph $f wff ph $.
  wps $f wff ps $.
The statement and its hypotheses require the variables:
  ph ps
The variables it contains are:  ph ps
\end{verbatim}

Compare this to \texttt{ax-1} on p.~\pageref{ax1}.  You can see that the
symbol \texttt{ph} is the {\sc ascii} notation for $\varphi$, etc.  To
see the mathematical symbols for any expression you may typeset it in
\LaTeX\ (type \texttt{help tex} for instructions)\index{latex@{\LaTeX}}
or, easier, just use a text editor to look at the comments where symbols
are first introduced in \texttt{set.mm}.  The hypotheses \texttt{wph}
and \texttt{wps} required by \texttt{ax-1} mean that variables
\texttt{ph} and \texttt{ps} must be wffs.

Next we'll pick a simple theorem of propositional calculus, the Principle of
Identity, which is proved directly from the axioms.  We'll look at the
statement then its proof.\index{\texttt{show statement}
command}

\begin{verbatim}
MM> show statement id1/full
Statement 116 is located on line 11371 of the file "set.mm".
Its statement number for HTML pages is 22.
"Principle of identity.  Theorem *2.08 of [WhiteheadRussell]
p. 101.  This version is proved directly from the axioms for
demonstration purposes.
...
116 id1 $p |- ( ph -> ph ) $= ... $.
Its mandatory hypotheses in RPN order are:
  wph $f wff ph $.
Its optional hypotheses are:  wps wch wth wta wet
      wze wsi wrh wmu wla wka
The statement and its hypotheses require the variables:  ph
These additional variables are allowed in its proof:
      ps ch th ta et ze si rh mu la ka
The variables it contains are:  ph
\end{verbatim}

The optional variables\index{optional variable} \texttt{ps}, \texttt{ch}, etc.\ are
available for use in a proof of this statement if we wish, and were we to do
so we would make use of optional hypotheses \texttt{wps}, \texttt{wch}, etc.  (See
Section~\ref{dollaref} for the meaning of ``optional
hypothesis.''\index{optional hypothesis}) The reason these show up in the
statement display is that statement \texttt{id1} happens to be in their scope
(see Section~\ref{scoping} for the definition of ``scope''\index{scope}), but
in fact in propositional calculus we will never make use of optional
hypotheses or variables.  This becomes important after quantifiers are
introduced, where ``dummy'' variables\index{dummy variable} are often needed
in the middle of a proof.

Let's look at the proof of statement \texttt{id1}.  We'll use the
\texttt{show proof} command, which by default suppresses the
``non-essential'' steps that construct the wffs.\index{\texttt{show proof}
command}
We will display the proof in ``lemmon' format (a non-indented format
with explicit previous step number references) and renumber the
displayed steps:

\begin{verbatim}
MM> show proof id1 /lemmon/renumber
1 ax-1           $a |- ( ph -> ( ph -> ph ) )
2 ax-1           $a |- ( ph -> ( ( ph -> ph ) -> ph ) )
3 ax-2           $a |- ( ( ph -> ( ( ph -> ph ) -> ph ) ) ->
                     ( ( ph -> ( ph -> ph ) ) -> ( ph -> ph )
                                                          ) )
4 2,3 ax-mp      $a |- ( ( ph -> ( ph -> ph ) ) -> ( ph -> ph
                                                          ) )
5 1,4 ax-mp      $a |- ( ph -> ph )
\end{verbatim}

If you have read Section~\ref{trialrun}, you'll know how to interpret this
proof.  Step~2, for example, is an application of axiom \texttt{ax-1}.  This
proof is identical to the one in Hamilton's {\em Logic for Mathematicians}
\cite[p.~32]{Hamilton}\index{Hamilton, Alan G.}.

You may want to look at what
substitutions\index{substitution!variable}\index{variable substitution} are
made into \texttt{ax-1} to arrive at step~2.  The command to do this needs to
know the ``real'' step number, so we'll display the proof again without
the \texttt{renumber} qualifier.\index{\texttt{show proof}
command}

\begin{verbatim}
MM> show proof id1 /lemmon
 9 ax-1          $a |- ( ph -> ( ph -> ph ) )
20 ax-1          $a |- ( ph -> ( ( ph -> ph ) -> ph ) )
24 ax-2          $a |- ( ( ph -> ( ( ph -> ph ) -> ph ) ) ->
                     ( ( ph -> ( ph -> ph ) ) -> ( ph -> ph )
                                                          ) )
25 20,24 ax-mp   $a |- ( ( ph -> ( ph -> ph ) ) -> ( ph -> ph
                                                          ) )
26 9,25 ax-mp    $a |- ( ph -> ph )
\end{verbatim}

The ``real'' step number is 20.  Let's look at its details.

\begin{verbatim}
MM> show proof id1 /detailed_step 20
Proof step 20:  min=ax-1 $a |- ( ph -> ( ( ph -> ph ) -> ph )
  )
This step assigns source "ax-1" ($a) to target "min" ($e).
The source assertion requires the hypotheses "wph" ($f, step
18) and "wps" ($f, step 19).  The parent assertion of the
target hypothesis is "ax-mp" ($a, step 25).
The source assertion before substitution was:
    ax-1 $a |- ( ph -> ( ps -> ph ) )
The following substitutions were made to the source
assertion:
    Variable  Substituted with
     ph        ph
     ps        ( ph -> ph )
The target hypothesis before substitution was:
    min $e |- ph
The following substitution was made to the target hypothesis:
    Variable  Substituted with
     ph        ( ph -> ( ( ph -> ph ) -> ph ) )
\end{verbatim}

This shows the substitutions\index{substitution!variable}\index{variable
substitution} made to the variables in \texttt{ax-1}.  References are made to
steps 18 and 19 which are not shown in our proof display.  To see these steps,
you can display the proof with the \texttt{all} qualifier.

Let's look at a slightly more advanced proof of propositional calculus.  Note
that \verb+/\+ is the symbol for $\wedge$ (logical {\sc and}, also
called conjunction).\index{conjunction ($\wedge$)}
\index{logical {\sc and} ($\wedge$)}

\begin{verbatim}
MM> show statement prth/full
Statement 1791 is located on line 15503 of the file "set.mm".
Its statement number for HTML pages is 559.
"Conjoin antecedents and consequents of two premises.  This
is the closed theorem form of ~ anim12d .  Theorem *3.47 of
[WhiteheadRussell] p. 113.  It was proved by Leibniz,
and it evidently pleased him enough to call it
_praeclarum theorema_ (splendid theorem).
...
1791 prth $p |- ( ( ( ph -> ps ) /\ ( ch -> th ) ) -> ( ( ph
      /\ ch ) -> ( ps /\ th ) ) ) $= ... $.
Its mandatory hypotheses in RPN order are:
  wph $f wff ph $.
  wps $f wff ps $.
  wch $f wff ch $.
  wth $f wff th $.
Its optional hypotheses are:  wta wet wze wsi wrh wmu wla wka
The statement and its hypotheses require the variables:  ph
      ps ch th
These additional variables are allowed in its proof:  ta et
      ze si rh mu la ka
The variables it contains are:  ph ps ch th


MM> show proof prth /lemmon/renumber
1 simpl          $p |- ( ( ( ph -> ps ) /\ ( ch -> th ) ) ->
                                               ( ph -> ps ) )
2 simpr          $p |- ( ( ( ph -> ps ) /\ ( ch -> th ) ) ->
                                               ( ch -> th ) )
3 1,2 anim12d    $p |- ( ( ( ph -> ps ) /\ ( ch -> th ) ) ->
                           ( ( ph /\ ch ) -> ( ps /\ th ) ) )
\end{verbatim}

There are references to a lot of unfamiliar statements.  To see what they are,
you may type the following:

\begin{verbatim}
MM> show proof prth /statement_summary
Summary of statements used in the proof of "prth":

Statement simpl is located on line 14748 of the file
"set.mm".
"Elimination of a conjunct.  Theorem *3.26 (Simp) of
[WhiteheadRussell] p. 112. ..."
  simpl $p |- ( ( ph /\ ps ) -> ph ) $= ... $.

Statement simpr is located on line 14777 of the file
"set.mm".
"Elimination of a conjunct.  Theorem *3.27 (Simp) of
[WhiteheadRussell] ..."
  simpr $p |- ( ( ph /\ ps ) -> ps ) $= ... $.

Statement anim12d is located on line 15445 of the file
"set.mm".
"Conjoin antecedents and consequents in a deduction.
..."
  anim12d.1 $e |- ( ph -> ( ps -> ch ) ) $.
  anim12d.2 $e |- ( ph -> ( th -> ta ) ) $.
  anim12d $p |- ( ph -> ( ( ps /\ th ) -> ( ch /\ ta ) ) )
      $= ... $.
\end{verbatim}
\begin{center}
(etc.)
\end{center}

Of course you can look at each of these statements and their proofs, and
so on, back to the axioms of propositional calculus if you wish.

The \texttt{search} command is useful for finding statements when you
know all or part of their contents.  The following example finds all
statements containing \verb@ph -> ps@ followed by \verb@ch -> th@.  The
\verb@$*@ is a wildcard that matches anything; the \texttt{\$} before the
\verb$*$ prevents conflicts with math symbol token names.  The \verb@*@ after
\texttt{SEARCH} is also a wildcard that in this case means ``match any label.''
\index{\texttt{search} command}

% I'm omitting this one, since readers are unlikely to see it:
% 1096 bisymOLD $p |- ( ( ( ph -> ps ) -> ( ch -> th ) ) -> ( (
%   ( ps -> ph ) -> ( th -> ch ) ) -> ( ( ph <-> ps ) -> ( ch
%    <-> th ) ) ) )
\begin{verbatim}
MM> search * "ph -> ps $* ch -> th"
1791 prth $p |- ( ( ( ph -> ps ) /\ ( ch -> th ) ) -> ( ( ph
    /\ ch ) -> ( ps /\ th ) ) )
2455 pm3.48 $p |- ( ( ( ph -> ps ) /\ ( ch -> th ) ) -> ( (
    ph \/ ch ) -> ( ps \/ th ) ) )
117859 pm11.71 $p |- ( ( E. x ph /\ E. y ch ) -> ( ( A. x (
    ph -> ps ) /\ A. y ( ch -> th ) ) <-> A. x A. y ( ( ph /\
    ch ) -> ( ps /\ th ) ) ) )
\end{verbatim}

Three statements, \texttt{prth}, \texttt{pm3.48},
 and \texttt{pm11.71}, were found to match.

To see what axioms\index{axiom} and definitions\index{definition}
\texttt{prth} ultimately depends on for its proof, you can have the
program backtrack through the hierarchy\index{hierarchy} of theorems and
definitions.\index{\texttt{show trace{\char`\_}back} command}

\begin{verbatim}
MM> show trace_back prth /essential/axioms
Statement "prth" assumes the following axioms ($a
statements):
  ax-1 ax-2 ax-3 ax-mp df-bi df-an
\end{verbatim}

Note that the 3 axioms of propositional calculus and the modus ponens rule are
needed (as expected); in addition, there are a couple of definitions that are used
along the way.  Note that Metamath makes no distinction\index{axiom vs.\
definition} between axioms\index{axiom} and definitions\index{definition}.  In
\texttt{set.mm} they have been distinguished artificially by prefixing their
labels\index{labels in \texttt{set.mm}} with \texttt{ax-} and \texttt{df-}
respectively.  For example, \texttt{df-an} defines conjunction (logical {\sc
and}), which is represented by the symbol \verb+/\+.
Section~\ref{definitions} discusses the philosophy of definitions, and the
Metamath language takes a particularly simple, conservative approach by using
the \texttt{\$a}\index{\texttt{\$a} statement} statement for both axioms and
definitions.

You can also have the program compute how many steps a proof
has\index{proof length} if we were to follow it all the way back to
\texttt{\$a} statements.

\begin{verbatim}
MM> show trace_back prth /essential/count_steps
The statement's actual proof has 3 steps.  Backtracking, a
total of 79 different subtheorems are used.  The statement
and subtheorems have a total of 274 actual steps.  If
subtheorems used only once were eliminated, there would be a
total of 38 subtheorems, and the statement and subtheorems
would have a total of 185 steps.  The proof would have 28349
steps if fully expanded back to axiom references.  The
maximum path length is 38.  A longest path is:  prth <-
anim12d <- syl2and <- sylan2d <- ancomsd <- ancom <- pm3.22
<- pm3.21 <- pm3.2 <- ex <- sylbir <- biimpri <- bicomi <-
bicom1 <- bi2 <- dfbi1 <- impbii <- bi3 <- simprim <- impi <-
con1i <- nsyl2 <- mt3d <- con1d <- notnot1 <- con2i <- nsyl3
<- mt2d <- con2d <- notnot2 <- pm2.18d <- pm2.18 <- pm2.21 <-
pm2.21d <- a1d <- syl <- mpd <- a2i <- a2i.1 .
\end{verbatim}

This tells us that we would have to inspect 274 steps if we want to
verify the proof completely starting from the axioms.  A few more
statistics are also shown.  There are one or more paths back to axioms
that are the longest; this command ferrets out one of them and shows it
to you.  There may be a sense in which the longest path length is
related to how ``deep'' the theorem is.

We might also be curious about what proofs depend on the theorem
\texttt{prth}.  If it is never used later on, we could eliminate it as
redundant if it has no intrinsic interest by itself.\index{\texttt{show
usage} command}

% I decided to show the OLD values here.
\begin{verbatim}
MM> show usage prth
Statement "prth" is directly referenced in the proofs of 18
statements:
  mo3 moOLD 2mo 2moOLD euind reuind reuss2 reusv3i opelopabt
  wemaplem2 rexanre rlimcn2 o1of2 o1rlimmul 2sqlem6 spanuni
  heicant pm11.71
\end{verbatim}

Thus \texttt{prth} is directly used by 18 proofs.
We can use the \texttt{/recursive} qualifier to include indirect use:

\begin{verbatim}
MM> show usage prth /recursive
Statement "prth" directly or indirectly affects the proofs of
24214 statements:
  mo3 mo mo3OLD eu2 moOLD eu2OLD eu3OLD mo4f mo4 eu4 mopick
...
\end{verbatim}

\subsection{A Note on the ``Compact'' Proof Format}

The present version of Metamath\index{Metamath!limitations of
version 0.07.30} will display proofs in a ``compact''\index{compact proof}
format whenever the proof is stored in compressed format in the database.  It
may be be slightly confusing unless you know how to interpret it.  (A
future version may eliminate this format from displays.)  For example,
if you display the complete proof of theorem \texttt{id1} it will start
off as follows:

\begin{verbatim}
MM> show proof id1 /lemmon/all
 1 wph           $f wff ph
 2 wph           $f wff ph
 3 wph           $f wff ph
 4 2,3 wi    @4: $a wff ( ph -> ph )
 5 1,4 wi    @5: $a wff ( ph -> ( ph -> ph ) )
 6 @4            $a wff ( ph -> ph )
\end{verbatim}

\begin{center}
{etc.}
\end{center}

Step 4 has a ``local label,''\index{local label} \texttt{@4}, assigned to it.
Later on, at step 6, the label \texttt{@1} is referenced instead of
displaying the explicit proof for that step.  This technique takes advantage
of the fact that steps in a proof often repeat, especially during the
construction of wffs.  The compact format reduces the number of steps in the
proof display and may actually be preferred by some people, which is one
reason that it is still displayed by the program.

If you want to see the normal format with the ``true'' step numbers, you can
use the following workaround:\index{\texttt{save proof} command}

\begin{verbatim}
MM> save proof id1 /normal
The proof of "id1" has been reformatted and saved internally.
Remember to use WRITE SOURCE to save it permanently.
MM> show proof id1 /lemmon/all
 1 wph           $f wff ph
 2 wph           $f wff ph
 3 wph           $f wff ph
 4 2,3 wi        $a wff ( ph -> ph )
 5 1,4 wi        $a wff ( ph -> ( ph -> ph ) )
 6 wph           $f wff ph
 7 wph           $f wff ph
 8 6,7 wi        $a wff ( ph -> ph )
\end{verbatim}

\begin{center}
{etc.}
\end{center}

Note that the original 6 steps are now 8 steps.  However, the format is
now the same as that described in Chapter~\ref{using}.

\chapter{The Metamath Language}
\label{languagespec}

\begin{quote}
  {\em Thus mathematics may be defined as the subject in which we never know
what we are talking about, nor whether what we are saying is true.}
    \flushright\sc  Bertrand Russell\footnote{\cite[p.~84]{Russell2}.}\\
\end{quote}\index{Russell, Bertrand}

Probably the most striking feature of the Metamath language is its almost
complete absence of hard-wired syntax. Metamath\index{Metamath} does not
understand any mathematics or logic other than that needed to construct finite
sequences of symbols according to a small set of simple, built-in rules.  The
only rule it uses in a proof is the substitution of an expression (symbol
sequence) for a variable, subject to a simple constraint to prevent
bound-variable clashes.  The primitive notions built into Metamath involve the
simple manipulation of finite objects (symbols) that we as humans can easily
visualize and that computers can easily deal with.  They seem to be just
about the simplest notions possible that are required to do standard
mathematics.

This chapter serves as a reference manual for the Metamath\index{Metamath}
language. It covers the tedious technical details of the language, some of
which you may wish to skim in a first reading.  On the other hand, you should
pay close attention to the defined terms in {\bf boldface}; they have precise
meanings that are important to keep in mind for later understanding.  It may
be best to first become familiar with the examples in Chapter~\ref{using} to
gain some motivation for the language.

%% Uncomment this when uncommenting section {formalspec} below
If you have some knowledge of set theory, you may wish to study this
chapter in conjunction with the formal set-theoretical description of the
Metamath language in Appendix~\ref{formalspec}.

We will use the name ``Metamath''\index{Metamath} to mean either the Metamath
computer language or the Metamath software associated with the computer
language.  We will not distinguish these two when the context is clear.

The next section contains the complete specification of the Metamath
language.
It serves as an
authoritative reference and presents the syntax in enough detail to
write a parser\index{parsing Metamath} and proof verifier.  The
specification is terse and it is probably hard to learn the language
directly from it, but we include it here for those impatient people who
prefer to see everything up front before looking at verbose expository
material.  Later sections explain this material and provide examples.
We will repeat the definitions in those sections, and you may skip the
next section at first reading and proceed to Section~\ref{tut1}
(p.~\pageref{tut1}).

\section{Specification of the Metamath Language}\label{spec}
\index{Metamath!specification}

\begin{quote}
  {\em Sometimes one has to say difficult things, but one ought to say
them as simply as one knows how.}
    \flushright\sc  G. H. Hardy\footnote{As quoted in
    \cite{deMillo}, p.~273.}\\
\end{quote}\index{Hardy, G. H.}

\subsection{Preliminaries}\label{spec1}

% Space is technically a printable character, so we'll word things
% carefully so it's unambiguous.
A Metamath {\bf database}\index{database} is built up from a top-level source
file together with any source files that are brought in through file inclusion
commands (see below).  The only characters that are allowed to appear in a
Metamath source file are the 94 non-whitespace printable {\sc
ascii}\index{ascii@{\sc ascii}} characters, which are digits, upper and lower
case letters, and the following 32 special
characters\index{special characters}:\label{spec1chars}

\begin{verbatim}
! " # $ % & ' ( ) * + , - . / :
; < = > ? @ [ \ ] ^ _ ` { | } ~
\end{verbatim}
plus the following characters which are the ``white space'' characters:
space (a printable character),
tab, carriage return, line feed, and form feed.\label{whitespace}
We will use \texttt{typewriter}
font to display the printable characters.

A Metamath database consists of a sequence of three kinds of {\bf
tokens}\index{token} separated by {\bf white space}\index{white space}
(which is any sequence of one or more white space characters).  The set
of {\bf keyword}\index{keyword} tokens is \texttt{\$\char`\{},
\texttt{\$\char`\}}, \texttt{\$c}, \texttt{\$v}, \texttt{\$f},
\texttt{\$e}, \texttt{\$d}, \texttt{\$a}, \texttt{\$p}, \texttt{\$.},
\texttt{\$=}, \texttt{\$(}, \texttt{\$)}, \texttt{\$[}, and
\texttt{\$]}.  The last four are called {\bf auxiliary}\index{auxiliary
keyword} or preprocessing keywords.  A {\bf label}\index{label} token
consists of any combination of letters, digits, and the characters
hyphen, underscore, and period.  A {\bf math symbol}\index{math symbol}
token may consist of any combination of the 92 printable standard {\sc
ascii} characters other than space or \texttt{\$}~. All tokens are
case-sensitive.

\subsection{Preprocessing}

The token \texttt{\$(} begins a {\bf comment} and
\texttt{\$)} ends a comment.\index{\texttt{\$(}
and \texttt{\$)} auxiliary keywords}\index{comment}
Comments may contain any of
the 94 non-whitespace printable characters and white space,
except they may not contain the
2-character sequences \texttt{\$(} or \texttt{\$)} (comments do not nest).
Comments are ignored (treated
like white space) for the purpose of parsing, e.g.,
\texttt{\$( \$[ \$)} is a comment.
See p.~\pageref{mathcomments} for comment typesetting conventions; these
conventions may be ignored for the purpose of parsing.

A {\bf file inclusion command} consists of \texttt{\$[} followed by a file name
followed by \texttt{\$]}.\index{\texttt{\$[} and \texttt{\$]} auxiliary
keywords}\index{included file}\index{file inclusion}
It is only allowed in the outermost scope (i.e., not between
\texttt{\$\char`\{} and \texttt{\$\char`\}})
and must not be inside a statement (e.g., it may not occur
between the label of a \texttt{\$a} statement and its \texttt{\$.}).
The file name may not
contain a \texttt{\$} or white space.  The file must exist.
The case-sensitivity
of its name follows the conventions of the operating system.  The contents of
the file replace the inclusion command.
Included files may include other files.
Only the first reference to a given file is included; any later
references to the same file (whether in the top-level file or in included
files) cause the inclusion command to be ignored (treated like white space).
A verifier may assume that file names with different strings
refer to different files for the purpose of ignoring later references.
A file self-reference is ignored, as is any reference to the top-level file
(to avoid loops).
Included files may not include a \texttt{\$(} without a matching \texttt{\$)},
may not include a \texttt{\$[} without a matching \texttt{\$]}, and may
not include incomplete statements (e.g., a \texttt{\$a} without a matching
\texttt{\$.}).
It is currently unspecified if path references are relative to the process'
current directory or the file's containing directory, so databases should
avoid using pathname separators (e.g., ``/'') in file names.

Like all tokens, the \texttt{\$(}, \texttt{\$)}, \texttt{\$[}, and \texttt{\$]} keywords
must be surrounded by white space.

\subsection{Basic Syntax}

After preprocessing, a database will consist of a sequence of {\bf
statements}.
These are the scoping statements \texttt{\$\char`\{} and
\texttt{\$\char`\}}, along with the \texttt{\$c}, \texttt{\$v},
\texttt{\$f}, \texttt{\$e}, \texttt{\$d}, \texttt{\$a}, and \texttt{\$p}
statements.

A {\bf scoping statement}\index{scoping statement} consists only of its
keyword, \texttt{\$\char`\{} or \texttt{\$\char`\}}.
A \texttt{\$\char`\{} begins a {\bf
block}\index{block} and a matching \texttt{\$\char`\}} ends the block.
Every \texttt{\$\char`\{}
must have a matching \texttt{\$\char`\}}.
Defining it recursively, we say a block
contains a sequence of zero or more tokens other
than \texttt{\$\char`\{} and \texttt{\$\char`\}} and
possibly other blocks.  There is an {\bf outermost
block}\index{block!outermost} not bracketed by \texttt{\$\char`\{} \texttt{\$\char`\}}; the end
of the outermost block is the end of the database.

% LaTeX bug? can't do \bf\tt

A {\bf \$v} or {\bf \$c statement}\index{\texttt{\$v} statement}\index{\texttt{\$c}
statement} consists of the keyword token \texttt{\$v} or \texttt{\$c} respectively,
followed by one or more math symbols,
% The word "token" is used to distinguish "$." from the sentence-ending period.
followed by the \texttt{\$.}\ token.
These
statements {\bf declare}\index{declaration} the math symbols to be {\bf
variables}\index{variable!Metamath} or {\bf constants}\index{constant}
respectively. The same math symbol may not occur twice in a given \texttt{\$v} or
\texttt{\$c} statement.

%c%A math symbol becomes an {\bf active}\index{active math symbol}
%c%when declared and stays active until the end of the block in which it is
%c%declared.  A math symbol may not be declared a second time while it is active,
%c%but it may be declared again after it becomes inactive.

A math symbol becomes {\bf active}\index{active math symbol} when declared
and stays active until the end of the block in which it is declared.  A
variable may not be declared a second time while it is active, but it
may be declared again (as a variable, but not as a constant) after it
becomes inactive.  A constant must be declared in the outermost block and may
not be declared a second time.\footnote{The rules for redeclaration may
become more general in the future;
see p.~\pageref{redeclarationf}.}\index{redeclaration of symbols}

A {\bf \$f statement}\index{\texttt{\$f} statement} consists of a label,
followed by \texttt{\$f}, followed by its typecode (an active constant),
followed by an
active variable, followed by the \texttt{\$.}\ token.  A {\bf \$e
statement}\index{\texttt{\$e} statement} consists of a label, followed
by \texttt{\$e}, followed by its typecode (an active constant),
followed by zero or more
active math symbols, followed by the \texttt{\$.}\ token.  A {\bf
hypothesis}\index{hypothesis} is a \texttt{\$f} or \texttt{\$e}
statement.
The type declared by a \texttt{\$f} statement for a given label
is global even if the variable is not
(e.g., a database may not have \texttt{wff P} in one local scope
nd \texttt{class P} in another).

A {\bf simple \$d statement}\index{\texttt{\$d} statement!simple}
consists of \texttt{\$d}, followed by two different active variables,
followed by the \texttt{\$.}\ token.  A {\bf compound \$d
statement}\index{\texttt{\$d} statement!compound} consists of
\texttt{\$d}, followed by three or more variables (all different),
followed by the \texttt{\$.}\ token.  The order of the variables in a
\texttt{\$d} statement is unimportant.  A compound \texttt{\$d}
statement is equivalent to a set of simple \texttt{\$d} statements, one
for each possible pair of variables occurring in the compound
\texttt{\$d} statement.  Henceforth in this specification we shall
assume all \texttt{\$d} statements are simple.  A \texttt{\$d} statement
is also called a {\bf disjoint} (or {\bf distinct}) {\bf variable
restriction}.\index{disjoint-variable restriction}

A {\bf \$a statement}\index{\texttt{\$a} statement} consists of a label,
followed by \texttt{\$a}, followed by its typecode (an active constant),
followed by
zero or more active math symbols, followed by the \texttt{\$.}\ token.  A {\bf
\$p statement}\index{\texttt{\$p} statement} consists of a label,
followed by \texttt{\$p}, followed by its typecode (an active constant),
followed by
zero or more active math symbols, followed by \texttt{\$=}, followed by
a sequence of labels, followed by the \texttt{\$.}\ token.  An {\bf
assertion}\index{assertion} is a \texttt{\$a} or \texttt{\$p} statement.

A \texttt{\$f}, \texttt{\$e}, or \texttt{\$d} statement is {\bf active}\index{active
statement} from the place it occurs until the end of the block it occurs in.
A \texttt{\$a} or \texttt{\$p} statement is {\bf active} from the place it occurs
through the end of the database.
There may not be two active \texttt{\$f} statements containing the same
variable.  Each variable in a \texttt{\$e}, \texttt{\$a}, or
\texttt{\$p} statement must exist in an active \texttt{\$f}
statement.\footnote{This requirement can greatly simplify the
unification algorithm (substitution calculation) required by proof
verification.}

%The label that begins each \texttt{\$f}, \texttt{\$e}, \texttt{\$a}, and
%\texttt{\$p} statement must be unique.
Each label token must be unique, and
no label token may match any math symbol
token.\label{namespace}\footnote{This
restriction was added on June 24, 2006.
It is not theoretically necessary but is imposed to make it easier to
write certain parsers.}

The set of {\bf mandatory variables}\index{mandatory variable} associated with
an assertion is the set of (zero or more) variables in the assertion and in any
active \texttt{\$e} statements.  The (possibly empty) set of {\bf mandatory
hypotheses}\index{mandatory hypothesis} is the set of all active \texttt{\$f}
statements containing mandatory variables, together with all active \texttt{\$e}
statements.
The set of {\bf mandatory {\bf \$d} statements}\index{mandatory
disjoint-variable restriction} associated with an assertion are those active
\texttt{\$d} statements whose variables are both among the assertion's
mandatory variables.

\subsection{Proof Verification}\label{spec4}

The sequence of labels between the \texttt{\$=} and \texttt{\$.}\ tokens
in a \texttt{\$p} statement is a {\bf proof}.\index{proof!Metamath} Each
label in a proof must be the label of an active statement other than the
\texttt{\$p} statement itself; thus a label must refer either to an
active hypothesis of the \texttt{\$p} statement or to an earlier
assertion.

An {\bf expression}\index{expression} is a sequence of math symbols. A {\bf
substitution map}\index{substitution map} associates a set of variables with a
set of expressions.  It is acceptable for a variable to be mapped to an
expression containing it.  A {\bf
substitution}\index{substitution!variable}\index{variable substitution} is the
simultaneous replacement of all variables in one or more expressions with the
expressions that the variables map to.

A proof is scanned in order of its label sequence.  If the label refers to an
active hypothesis, the expression in the hypothesis is pushed onto a
stack.\index{stack}\index{RPN stack}  If the label refers to an assertion, a
(unique) substitution must exist that, when made to the mandatory hypotheses
of the referenced assertion, causes them to match the topmost (i.e.\ most
recent) entries of the stack, in order of occurrence of the mandatory
hypotheses, with the topmost stack entry matching the last mandatory
hypothesis of the referenced assertion.  As many stack entries as there are
mandatory hypotheses are then popped from the stack.  The same substitution is
made to the referenced assertion, and the result is pushed onto the stack.
After the last label in the proof is processed, the stack must have a single
entry that matches the expression in the \texttt{\$p} statement containing the
proof.

%c%{\footnotesize\begin{quotation}\index{redeclaration of symbols}
%c%{{\em Comment.}\label{spec4comment} Whenever a math symbol token occurs in a
%c%{\texttt{\$c} or \texttt{\$v} statement, it is considered to designate a distinct new
%c%{symbol, even if the same token was previously declared (and is now inactive).
%c%{Thus a math token declared as a constant in two different blocks is considered
%c%{to designate two distinct constants (even though they have the same name).
%c%{The two constants will not match in a proof that references both blocks.
%c%{However, a proof referencing both blocks is acceptable as long as it doesn't
%c%{require that the constants match.  Similarly, a token declared to be a
%c%{constant for a referenced assertion will not match the same token declared to
%c%{be a variable for the \texttt{\$p} statement containing the proof.  In the case
%c%{of a token declared to be a variable for a referenced assertion, this is not
%c%{an issue since the variable can be substituted with whatever expression is
%c%{needed to achieve the required match.
%c%{\end{quotation}}
%c2%A proof may reference an assertion that contains or whose hypotheses contain a
%c2%constant that is not active for the \texttt{\$p} statement containing the proof.
%c2%However, the final result of the proof may not contain that constant. A proof
%c2%may also reference an assertion that contains or whose hypotheses contain a
%c2%variable that is not active for the \texttt{\$p} statement containing the proof.
%c2%That variable, of course, will be substituted with whatever expression is
%c2%needed to achieve the required match.

A proof may contain a \texttt{?}\ in place of a label to indicate an unknown step
(Section~\ref{unknown}).  A proof verifier may ignore any proof containing
\texttt{?}\ but should warn the user that the proof is incomplete.

A {\bf compressed proof}\index{compressed proof}\index{proof!compressed} is an
alternate proof notation described in Appen\-dix~\ref{compressed}; also see
references to ``compressed proof'' in the Index.  Compressed proofs are a
Metamath language extension which a complete proof verifier should be able to
parse and verify.

\subsubsection{Verifying Disjoint Variable Restrictions}

Each substitution made in a proof must be checked to verify that any
disjoint variable restrictions are satisfied, as follows.

If two variables replaced by a substitution exist in a mandatory \texttt{\$d}
statement\index{\texttt{\$d} statement} of the assertion referenced, the two
expressions resulting from the substitution must satisfy the following
conditions.  First, the two expressions must have no variables in common.
Second, each possible pair of variables, one from each expression, must exist
in an active \texttt{\$d} statement of the \texttt{\$p} statement containing the
proof.

\vskip 1ex

This ends the specification of the Metamath language;
see Appendix \ref{BNF} for its syntax in
Extended Backus--Naur Form (EBNF)\index{Extended Backus--Naur Form}\index{EBNF}.

\section{The Basic Keywords}\label{tut1}

Our expository material begins here.

Like most computer languages, Metamath\index{Metamath} takes its input from
one or more {\bf source files}\index{source file} which contain characters
expressed in the standard {\sc ascii} (American Standard Code for Information
Interchange)\index{ascii@{\sc ascii}} code for computers.  A source file
consists of a series of {\bf tokens}\index{token}, which are strings of
non-whitespace
printable characters (from the set of 94 shown on p.~\pageref{spec1chars})
separated by {\bf white space}\index{white space} (spaces, tabs, carriage
returns, line feeds, and form feeds). Any string consisting only of these
characters is treated the same as a single space.  The non-whitespace printable
characters\index{printable character} that Metamath recognizes are the 94
characters on standard {\sc ascii} keyboards.

Metamath has the ability to join several files together to form its
input (Section~\ref{include}).  We call the aggregate contents of all
the files after they have been joined together a {\bf
database}\index{database} to distinguish it from an individual source
file.  The tokens in a database consist of {\bf
keywords}\index{keyword}, which are built into the language, together
with two kinds of user-defined tokens called {\bf labels}\index{label}
and {\bf math symbols}\index{math symbol}.  (Often we will simply say
{\bf symbol}\index{symbol} instead of math symbol for brevity).  The set
of {\bf basic keywords}\index{basic keyword} is
\texttt{\$c}\index{\texttt{\$c} statement},
\texttt{\$v}\index{\texttt{\$v} statement},
\texttt{\$e}\index{\texttt{\$e} statement},
\texttt{\$f}\index{\texttt{\$f} statement},
\texttt{\$d}\index{\texttt{\$d} statement},
\texttt{\$a}\index{\texttt{\$a} statement},
\texttt{\$p}\index{\texttt{\$p} statement},
\texttt{\$=}\index{\texttt{\$=} keyword},
\texttt{\$.}\index{\texttt{\$.}\ keyword},
\texttt{\$\char`\{}\index{\texttt{\$\char`\{} and \texttt{\$\char`\}}
keywords}, and \texttt{\$\char`\}}.  This is the complete set of
syntactical elements of what we call the {\bf basic
language}\index{basic language} of Metamath, and with them you can
express all of the mathematics that were intended by the design of
Metamath.  You should make it a point to become very familiar with them.
Table~\ref{basickeywords} lists the basic keywords along with a brief
description of their functions.  For now, this description will give you
only a vague notion of what the keywords are for; later we will describe
the keywords in detail.


\begin{table}[htp] \caption{Summary of the basic Metamath
keywords} \label{basickeywords}
\begin{center}
\begin{tabular}{|p{4pc}|l|}
\hline
\em \centering Keyword&\em Description\\
\hline
\hline
\centering
   \texttt{\$c}&Constant symbol declaration\\
\hline
\centering
   \texttt{\$v}&Variable symbol declaration\\
\hline
\centering
   \texttt{\$d}&Disjoint variable restriction\\
\hline
\centering
   \texttt{\$f}&Variable-type (``floating'') hypothesis\\
\hline
\centering
   \texttt{\$e}&Logical (``essential'') hypothesis\\
\hline
\centering
   \texttt{\$a}&Axiomatic assertion\\
\hline
\centering
   \texttt{\$p}&Provable assertion\\
\hline
\centering
   \texttt{\$=}&Start of proof in \texttt{\$p} statement\\
\hline
\centering
   \texttt{\$.}&End of the above statement types\\
\hline
\centering
   \texttt{\$\char`\{}&Start of block\\
\hline
\centering
   \texttt{\$\char`\}}&End of block\\
\hline
\end{tabular}
\end{center}
\end{table}

%For LaTeX bug(?) where it puts tables on blank page instead of btwn text
%May have to adjust if text changes
%\newpage

There are some additional keywords, called {\bf auxiliary
keywords}\index{auxiliary keyword} that help make Metamath\index{Metamath}
more practical. These are part of the {\bf extended language}\index{extended
language}. They provide you with a means to put comments into a Metamath
source file\index{source file} and reference other source files.  We will
introduce these in later sections. Table~\ref{otherkeywords} summarizes them
so that you can recognize them now if you want to peruse some source
files while learning the basic keywords.


\begin{table}[htp] \caption{Auxiliary Metamath
keywords} \label{otherkeywords}
\begin{center}
\begin{tabular}{|p{4pc}|l|}
\hline
\em \centering Keyword&\em Description\\
\hline
\hline
\centering
   \texttt{\$(}&Start of comment\\
\hline
\centering
   \texttt{\$)}&End of comment\\
\hline
\centering
   \texttt{\$[}&Start of included source file name\\
\hline
\centering
   \texttt{\$]}&End of included source file name\\
\hline
\end{tabular}
\end{center}
\end{table}
\index{\texttt{\$(} and \texttt{\$)} auxiliary keywords}
\index{\texttt{\$[} and \texttt{\$]} auxiliary keywords}


Unlike those in some computer languages, the keywords\index{keyword} are short
two-character sequences rather than English-like words.  While this may make
them slightly more difficult to remember at first, their brevity allows
them to blend in with the mathematics being described, not
distract from it, like punctuation marks.


\subsection{User-Defined Tokens}\label{dollardollar}\index{token}

As you may have noticed, all keywords\index{keyword} begin with the \texttt{\$}
character.  This mundane monetary symbol is not ordinarily used in higher
mathematics (outside of grant proposals), so we have appropriated it to
distinguish the Metamath\index{Metamath} keywords from ordinary mathematical
symbols. The \texttt{\$} character is thus considered special and may not be
used as a character in a user-defined token.  All tokens and keywords are
case-sensitive; for example, \texttt{n} is considered to be a different character
from \texttt{N}.  Case-sensitivity makes the available {\sc ascii} character set
as rich as possible.

\subsubsection{Math Symbol Tokens}\index{token}

Math symbols\index{math symbol} are tokens used to represent the symbols
that appear in ordinary mathematical formulas.  They may consist of any
combination of the 93 non-whitespace printable {\sc ascii} characters other than
\texttt{\$}~. Some examples are \texttt{x}, \texttt{+}, \texttt{(},
\texttt{|-}, \verb$!%@?&$, and \texttt{bounded}.  For readability, it is
best to try to make these look as similar to actual mathematical symbols
as possible, within the constraints of the {\sc ascii} character set, in
order to make the resulting mathematical expressions more readable.

In the Metamath\index{Metamath} language, you express ordinary
mathematical formulas and statements as sequences of math symbols such
as \texttt{2 + 2 = 4} (five symbols, all constants).\footnote{To
eliminate ambiguity with other expressions, this is expressed in the set
theory database \texttt{set.mm} as \texttt{|- ( 2 + 2
 ) = 4 }, whose \LaTeX\ equivalent is $\vdash
(2+2)=4$.  The \,$\vdash$ means ``is a theorem'' and the
parentheses allow explicit associative grouping.}\index{turnstile
({$\,\vdash$})} They may even be English
sentences, as in \texttt{E is closed and bounded} (five symbols)---here
\texttt{E} would be a variable and the other four symbols constants.  In
principle, a Metamath database could be constructed to work with almost
any unambiguous English-language mathematical statement, but as a
practical matter the definitions needed to provide for all possible
syntax variations would be cumbersome and distracting and possibly have
subtle pitfalls accidentally built in.  We generally recommend that you
express mathematical statements with compact standard mathematical
symbols whenever possible and put their English-language descriptions in
comments.  Axioms\index{axiom} and definitions\index{definition}
(\texttt{\$a}\index{\texttt{\$a} statement} statements) are the only
places where Metamath will not detect an error, and doing this will help
reduce the number of definitions needed.

You are free to use any tokens\index{token} you like for math
symbols\index{math symbol}.  Appendix~\ref{ASCII} recommends token names to
use for symbols in set theory, and we suggest you adopt these in order to be
able to include the \texttt{set.mm} set theory database in your database.  For
printouts, you can convert the tokens in a database
to standard mathematical symbols with the \LaTeX\ typesetting program.  The
Metamath command \texttt{open tex} {\em filename}\index{\texttt{open tex} command}
produces output that can be read by \LaTeX.\index{latex@{\LaTeX}}
The correspondence
between tokens and the actual symbols is made by \texttt{latexdef}
statements inside a special database comment tagged
with \texttt{\$t}.\index{\texttt{\$t} comment}\index{typesetting comment}
  You can edit
this comment to change the definitions or add new ones.
Appendix~\ref{ASCII} describes how to do this in more detail.

% White space\index{white space} is normally used to separate math
% symbol\index{math symbol} tokens, but they may be juxtaposed without white
% space in \texttt{\$d}\index{\texttt{\$d} statement}, \texttt{\$e}\index{\texttt{\$e}
% statement}, \texttt{\$f}\index{\texttt{\$f} statement}, \texttt{\$a}\index{\texttt{\$a}
% statement}, and \texttt{\$p}\index{\texttt{\$p} statement} statements when no
% ambiguity will result.  Specifically, Metamath parses the math symbol sequence
% in one of these statements in the following manner:  when the math symbol
% sequence has been broken up into tokens\index{token} up to a given character,
% the next token is the longest string of characters that could constitute a
% math symbol that is active\index{active
% math symbol} at that point.  (See Section~\ref{scoping} for the
% definition of an active math symbol.)  For example, if \texttt{-}, \texttt{>}, and
% \texttt{->} are the only active math symbols, the juxtaposition \texttt{>-} will be
% interpreted as the two symbols \texttt{>} and \texttt{-}, whereas \texttt{->} will
% always be interpreted as that single symbol.\footnote{For better readability we
% recommend a white space between each token.  This also makes searching for a
% symbol easier to do with an editor.  Omission of optional white space is useful
% for reducing typing when assigning an expression to a temporary
% variable\index{temporary variable} with the \texttt{let variable} Metamath
% program command.}\index{\texttt{let variable} command}
%
% Keywords\index{keyword} may be placed next to math symbols without white
% space\index{white space} between them.\footnote{Again, we do not recommend
% this for readability.}
%
% The math symbols\index{math symbol} in \texttt{\$c}\index{\texttt{\$c} statement}
% and \texttt{\$v}\index{\texttt{\$v} statement} statements must always be separated
% by white space\index{white
% space}, for the obvious reason that these statements define the names
% of the symbols.
%
% Math symbols referred to in comments (see Section~\ref{comments}) must also be
% separated by white space.  This allows you to make comments about symbols that
% are not yet active\index{active
% math symbol}.  (The ``math mode'' feature of comments is also a quick and
% easy way to obtain word processing text with embedded mathematical symbols,
% independently of the main purpose of Metamath; the way to do this is described
% in Section~\ref{comments})

\subsubsection{Label Tokens}\index{token}\index{label}

Label tokens are used to identify Metamath\index{Metamath} statements for
later reference. Label tokens may contain only letters, digits, and the three
characters period, hyphen, and underscore:
\begin{verbatim}
. - _
\end{verbatim}

A label is {\bf declared}\index{label declaration} by placing it immediately
before the keyword of the statement it identifies.  For example, the label
\texttt{axiom.1} might be declared as follows:
\begin{verbatim}
axiom.1 $a |- x = x $.
\end{verbatim}

Each \texttt{\$e}\index{\texttt{\$e} statement},
\texttt{\$f}\index{\texttt{\$f} statement},
\texttt{\$a}\index{\texttt{\$a} statement}, and
\texttt{\$p}\index{\texttt{\$p} statement} statement in a database must
have a label declared for it.  No other statement types may have label
declarations.  Every label must be unique.

A label (and the statement it identifies) is {\bf referenced}\index{label
reference} by including the label between the \texttt{\$=}\index{\texttt{\$=}
keyword} and \texttt{\$.}\index{\texttt{\$.}\ keyword}\ keywords in a \texttt{\$p}
statement.  The sequence of labels\index{label sequence} between these two
keywords is called a {\bf proof}\index{proof}.  An example of a statement with
a proof that we will encounter later (Section~\ref{proof}) is
\begin{verbatim}
wnew $p wff ( s -> ( r -> p ) )
     $= ws wr wp w2 w2 $.
\end{verbatim}

You don't have to know what this means just yet, but you should know that the
label \texttt{wnew} is declared by this \texttt{\$p} statement and that the labels
\texttt{ws}, \texttt{wr}, \texttt{wp}, and \texttt{w2} are assumed to have been declared
earlier in the database and are referenced here.

\subsection{Constants and Variables}
\index{constant}
\index{variable}

An {\bf expression}\index{expression} is any sequence of math
symbols, possibly empty.

The basic Metamath\index{Metamath} language\index{basic language} has two
kinds of math symbols\index{math symbol}:  {\bf constants}\index{constant} and
{\bf variables}\index{variable}.  In a Metamath proof, a constant may not be
substituted with any expression.  A variable can be
substituted\index{substitution!variable}\index{variable substitution} with any
expression.  This sequence may include other variables and may even include
the variable being substituted.  This substitution takes place when proofs are
verified, and it will be described in Section~\ref{proof}.  The \texttt{\$f}
statement (described later in Section~\ref{dollaref}) is used to specify the
{\bf type} of a variable (i.e.\ what kind of
variable it is)\index{variable type}\index{type} and
give it a meaning typically
associated with a ``metavariable''\index{metavariable}\footnote{A metavariable
is a variable that ranges over the syntactical elements of the object language
being discussed; for example, one metavariable might represent a variable of
the object language and another metavariable might represent a formula in the
object language.} in ordinary mathematics; for example, a variable may be
specified to be a wff or well-formed formula (in logic), a set (in set
theory), or a non-negative integer (in number theory).

%\subsection{The \texttt{\$c} and \texttt{\$v} Declaration Statements}
\subsection{The \texttt{\$c} and \texttt{\$v} Declaration Statements}
\index{\texttt{\$c} statement}
\index{constant declaration}
\index{\texttt{\$v} statement}
\index{variable declaration}

Constants are introduced or {\bf declared}\index{constant declaration}
with \texttt{\$c}\index{\texttt{\$c} statement} statements, and
variables are declared\index{variable declaration} with
\texttt{\$v}\index{\texttt{\$v} statement} statements.  A {\bf simple}
declaration\index{simple declaration} statement introduces a single
constant or variable.  Its syntax is one of the following:
\begin{center}
  \texttt{\$c} {\em math-symbol} \texttt{\$.}\\
  \texttt{\$v} {\em math-symbol} \texttt{\$.}
\end{center}
The notation {\em math-symbol} means any math symbol token\index{token}.

Some examples of simple declaration statements are:
\begin{center}
  \texttt{\$c + \$.}\\
  \texttt{\$c -> \$.}\\
  \texttt{\$c ( \$.}\\
  \texttt{\$v x \$.}\\
  \texttt{\$v y2 \$.}
\end{center}

The characters in a math symbol\index{math symbol} being declared are
irrelevant to Meta\-math; for example, we could declare a right parenthesis to
be a variable,
\begin{center}
  \texttt{\$v ) \$.}\\
\end{center}
although this would be unconventional.

A {\bf compound} declaration\index{compound declaration} statement is a
shorthand for declaring several symbols at once.  Its syntax is one of the
following:
\begin{center}
  \texttt{\$c} {\em math-symbol}\ \,$\cdots$\ {\em math-symbol} \texttt{\$.}\\
  \texttt{\$v} {\em math-symbol}\ \,$\cdots$\ {\em math-symbol} \texttt{\$.}
\end{center}\index{\texttt{\$c} statement}
Here, the ellipsis (\ldots) means any number of {\em math-symbol}\,s.

An example of a compound declaration statement is:
\begin{center}
  \texttt{\$v x y mu \$.}\\
\end{center}
This is equivalent to the three simple declaration statements
\begin{center}
  \texttt{\$v x \$.}\\
  \texttt{\$v y \$.}\\
  \texttt{\$v mu \$.}\\
\end{center}
\index{\texttt{\$v} statement}

There are certain rules on where in the database math symbols may be declared,
what sections of the database are aware of them (i.e.\ where they are
``active''), and when they may be declared more than once.  These will be
discussed in Section~\ref{scoping} and specifically on
p.~\pageref{redeclaration}.

\subsection{The \texttt{\$d} Statement}\label{dollard}
\index{\texttt{\$d} statement}

The \texttt{\$d} statement is called a {\bf disjoint-variable restriction}.  The
syntax of the {\bf simple} version of this statement is
\begin{center}
  \texttt{\$d} {\em variable variable} \texttt{\$.}
\end{center}
where each {\em variable} is a previously declared variable and the two {\em
variable}\,s are different.  (More specifically, each  {\em variable} must be
an {\bf active} variable\index{active math symbol}, which means there must be
a previous \texttt{\$v} statement whose {\bf scope}\index{scope} includes the
\texttt{\$d} statement.  These terms will be defined when we discuss scoping
statements in Section~\ref{scoping}.)

In ordinary mathematics, formulas may arise that are true if the variables in
them are distinct\index{distinct variables}, but become false when those
variables are made identical. For example, the formula in logic $\exists x\,x
\neq y$, which means ``for a given $y$, there exists an $x$ that is not equal
to $y$,'' is true in most mathematical theories (namely all non-trivial
theories\index{non-trivial theory}, i.e.\ those that describe more than one
individual, such as arithmetic).  However, if we substitute $y$ with $x$, we
obtain $\exists x\,x \neq x$, which is always false, as it means ``there
exists something that is not equal to itself.''\footnote{If you are a
logician, you will recognize this as the improper substitution\index{proper
substitution}\index{substitution!proper} of a free variable\index{free
variable} with a bound variable\index{bound variable}.  Metamath makes no
inherent distinction between free and bound variables; instead, you let
Metamath know what substitutions are permissible by using \texttt{\$d} statements
in the right way in your axiom system.}\index{free vs.\ bound variable}  The
\texttt{\$d} statement allows you to specify a restriction that forbids the
substitution of one variable with another.  In
this case, we would use the statement
\begin{center}
  \texttt{\$d x y \$.}
\end{center}\index{\texttt{\$d} statement}
to specify this restriction.

The order in which the variables appear in a \texttt{\$d} statement is not
important.  We could also use
\begin{center}
  \texttt{\$d y x \$.}
\end{center}

The \texttt{\$d} statement is actually more general than this, as the
``disjoint''\index{disjoint variables} in its name suggests.  The full meaning
is that if any substitution is made to its two variables (during the
course of a proof that references a \texttt{\$a} or \texttt{\$p} statement
associated with the \texttt{\$d}), the two expressions that result from the
substitution must have no variables in common.  In addition, each possible
pair of variables, one from each expression, must be in a \texttt{\$d} statement
associated with the statement being proved.  (This requirement forces the
statement being proved to ``inherit'' the original disjoint variable
restriction.)

For example, suppose \texttt{u} is a variable.  If the restriction
\begin{center}
  \texttt{\$d A B \$.}
\end{center}
has been specified for a theorem referenced in a
proof, we may not substitute \texttt{A} with \mbox{\tt a + u} and
\texttt{B} with \mbox{\tt b + u} because these two symbol sequences have the
variable \texttt{u} in common.  Furthermore, if \texttt{a} and \texttt{b} are
variables, we may not substitute \texttt{A} with \texttt{a} and \texttt{B} with \texttt{b}
unless we have also specified \texttt{\$d a b} for the theorem being proved; in
other words, the \texttt{\$d} property associated with a pair of variables must
be effectively preserved after substitution.

The \texttt{\$d}\index{\texttt{\$d} statement} statement does {\em not} mean ``the
two variables may not be substituted with the same thing,'' as you might think
at first.  For example, substituting each of \texttt{A} and \texttt{B} in the above
example with identical symbol sequences consisting only of constants does not
cause a disjoint variable conflict, because two symbol sequences have no
variables in common (since they have no variables, period).  Similarly, a
conflict will not occur by substituting the two variables in a \texttt{\$d}
statement with the empty symbol sequence\index{empty substitution}.

The \texttt{\$d} statement does not have a direct counterpart in
ordinary mathematics, partly because the variables\index{variable} of
Metamath are not really the same as the variables\index{variable!in
ordinary mathematics} of ordinary mathematics but rather are
metavariables\index{metavariable} ranging over them (as well as over
other kinds of symbols and groups of symbols).  Depending on the
situation, we may informally interpret the \texttt{\$d} statement in
different ways.  Suppose, for example, that \texttt{x} and \texttt{y}
are variables ranging over numbers (more precisely, that \texttt{x} and
\texttt{y} are metavariables ranging over variables that range over
numbers), and that \texttt{ph} ($\varphi$) and \texttt{ps} ($\psi$) are
variables (more precisely, metavariables) ranging over formulas.  We can
make the following interpretations that correspond to the informal
language of ordinary mathematics:
\begin{quote}
\begin{tabbing}
\texttt{\$d x y \$.} means ``assume $x$ and $y$ are
distinct variables.''\\
\texttt{\$d x ph \$.} means ``assume $x$ does not
occur in $\varphi$.''\\
\texttt{\$d ph ps \$.} \=means ``assume $\varphi$ and
$\psi$ have no variables\\ \>in common.''
\end{tabbing}
\end{quote}\index{\texttt{\$d} statement}

\subsubsection{Compound \texttt{\$d} Statements}

The {\bf compound} version of the \texttt{\$d} statement is a shorthand for
specifying several variables whose substitutions must be pairwise disjoint.
Its syntax is:
\begin{center}
  \texttt{\$d} {\em variable}\ \,$\cdots$\ {\em variable} \texttt{\$.}
\end{center}\index{\texttt{\$d} statement}
Here, {\em variable} represents the token of a previously declared
variable (specifically, an active variable) and all {\em variable}\,s are
different.  The compound \texttt{\$d}
statement is internally broken up by Metamath into one simple \texttt{\$d}
statement for each possible pair of variables in the original \texttt{\$d}
statement.  For example,
\begin{center}
  \texttt{\$d w x y z \$.}
\end{center}
is equivalent to
\begin{center}
  \texttt{\$d w x \$.}\\
  \texttt{\$d w y \$.}\\
  \texttt{\$d w z \$.}\\
  \texttt{\$d x y \$.}\\
  \texttt{\$d x z \$.}\\
  \texttt{\$d y z \$.}
\end{center}

Two or more simple \texttt{\$d} statements specifying the same variable pair are
internally combined into a single \texttt{\$d} statement.  Thus the set of three
statements
\begin{center}
  \texttt{\$d x y \$.}
  \texttt{\$d x y \$.}
  \texttt{\$d y x \$.}
\end{center}
is equivalent to
\begin{center}
  \texttt{\$d x y \$.}
\end{center}

Similarly, compound \texttt{\$d} statements, after being internally broken up,
internally have their common variable pairs combined.  For example the
set of statements
\begin{center}
  \texttt{\$d x y A \$.}
  \texttt{\$d x y B \$.}
\end{center}
is equivalent to
\begin{center}
  \texttt{\$d x y \$.}
  \texttt{\$d x A \$.}
  \texttt{\$d y A \$.}
  \texttt{\$d x y \$.}
  \texttt{\$d x B \$.}
  \texttt{\$d y B \$.}
\end{center}
which is equivalent to
\begin{center}
  \texttt{\$d x y \$.}
  \texttt{\$d x A \$.}
  \texttt{\$d y A \$.}
  \texttt{\$d x B \$.}
  \texttt{\$d y B \$.}
\end{center}

Metamath\index{Metamath} automatically verifies that all \texttt{\$d}
restrictions are met whenever it verifies proofs.  \texttt{\$d} statements are
never referenced directly in proofs (this is why they do not have
labels\index{label}), but Metamath is always aware of which ones must be
satisfied (i.e.\ are active) and will notify you with an error message if any
violation occurs.

To illustrate how Metamath detects a missing \texttt{\$d}
statement, we will look at the following example from the
\texttt{set.mm} database.

\begin{verbatim}
$d x z $.  $d y z $.
$( Theorem to add distinct quantifier to atomic formula. $)
ax17eq $p |- ( x = y -> A. z x = y ) $=...
\end{verbatim}

This statement has the obvious requirement that $z$ must be
distinct\index{distinct variables} from $x$ in theorem \texttt{ax17eq} that
states $x=y \rightarrow \forall z \, x=y$ (well, obvious if you're a logician,
for otherwise we could conclude  $x=y \rightarrow \forall x \, x=y$, which is
false when the free variables $x$ and $y$ are equal.).

Let's look at what happens if we edit the database to comment out this
requirement.

\begin{verbatim}
$( $d x z $. $) $d y z $.
$( Theorem to add distinct quantifier to atomic formula. $)
ax17eq $p |- ( x = y -> A. z x = y ) $=...
\end{verbatim}

When it tries to verify the proof, Metamath will tell you that \texttt{x} and
\texttt{z} must be disjoint, because one of its steps references an axiom or
theorem that has this requirement.

\begin{verbatim}
MM> verify proof ax17eq
ax17eq ?Error at statement 1918, label "ax17eq", type "$p":
      vz wal wi vx vy vz ax-13 vx vy weq vz vx ax-c16 vx vy
                                               ^^^^^
There is a disjoint variable ($d) violation at proof step 29.
Assertion "ax-c16" requires that variables "x" and "y" be
disjoint.  But "x" was substituted with "z" and "y" was
substituted with "x".  The assertion being proved, "ax17eq",
does not require that variables "z" and "x" be disjoint.
\end{verbatim}

We can see the substitutions into \texttt{ax-c16} with the following command.

\begin{verbatim}
MM> show proof ax17eq / detailed_step 29
Proof step 29:  pm2.61dd.2=ax-c16 $a |- ( A. z z = x -> ( x =
  y -> A. z x = y ) )
This step assigns source "ax-c16" ($a) to target "pm2.61dd.2"
($e).  The source assertion requires the hypotheses "wph"
($f, step 26), "vx" ($f, step 27), and "vy" ($f, step 28).
The parent assertion of the target hypothesis is "pm2.61dd"
($p, step 36).
The source assertion before substitution was:
    ax-c16 $a |- ( A. x x = y -> ( ph -> A. x ph ) )
The following substitutions were made to the source
assertion:
    Variable  Substituted with
     x         z
     y         x
     ph        x = y
The target hypothesis before substitution was:
    pm2.61dd.2 $e |- ( ph -> ch )
The following substitutions were made to the target
hypothesis:
    Variable  Substituted with
     ph        A. z z = x
     ch        ( x = y -> A. z x = y )
\end{verbatim}

The disjoint variable restrictions of \texttt{ax-c16} can be seen from the
\texttt{show state\-ment} command.  The line that begins ``\texttt{Its mandatory
dis\-joint var\-i\-able pairs are:}\ldots'' lists any \texttt{\$d} variable
pairs in brackets.

\begin{verbatim}
MM> show statement ax-c16/full
Statement 3033 is located on line 9338 of the file "set.mm".
"Axiom of Distinct Variables. ..."
  ax-c16 $a |- ( A. x x = y -> ( ph -> A. x ph ) ) $.
Its mandatory hypotheses in RPN order are:
  wph $f wff ph $.
  vx $f set x $.
  vy $f set y $.
Its mandatory disjoint variable pairs are:  <x,y>
The statement and its hypotheses require the variables:  x y
      ph
The variables it contains are:  x y ph
\end{verbatim}

Since Metamath will always detect when \texttt{\$d}\index{\texttt{\$d} statement}
statements are needed for a proof, you don't have to worry too much about
forgetting to put one in; it can always be added if you see the error message
above.  If you put in unnecessary \texttt{\$d} statements, the worst that could
happen is that your theorem might not be as general as it could be, and this
may limit its use later on.

On the other hand, when you introduce axioms (\texttt{\$a}\index{\texttt{\$a}
statement} statements), you must be very careful to properly specify the
necessary associated \texttt{\$d} statements since Metamath has no way of knowing
whether your axioms are correct.  For example, Metamath would have no idea
that \texttt{ax-c16}, which we are telling it is an axiom of logic, would lead to
contradictions if we omitted its associated \texttt{\$d} statement.

% This was previously a comment in footnote-sized type, but it can be
% hard to read this much text in a small size.
% As a result, it's been changed to normally-sized text.
\label{nodd}
You may wonder if it is possible to develop standard
mathematics in the Metamath language without the \texttt{\$d}\index{\texttt{\$d}
statement} statement, since it seems like a nuisance that complicates proof
verification. The \texttt{\$d} statement is not needed in certain subsets of
mathematics such as propositional calculus.  However, dummy
variables\index{dummy variable!eliminating} and their associated \texttt{\$d}
statements are impossible to avoid in proofs in standard first-order logic as
well as in the variant used in \texttt{set.mm}.  In fact, there is no upper bound to
the number of dummy variables that might be needed in a proof of a theorem of
first-order logic containing 3 or more variables, as shown by H.\
Andr\'{e}ka\index{Andr{\'{e}}ka, H.} \cite{Nemeti}.  A first-order system that
avoids them entirely is given in \cite{Megill}\index{Megill, Norman}; the
trick there is simply to embed harmlessly the necessary dummy variables into a
theorem being proved so that they aren't ``dummy'' anymore, then interpret the
resulting longer theorem so as to ignore the embedded dummy variables.  If
this interests you, the system in \texttt{set.mm} obtained from \texttt{ax-1}
through \texttt{ax-c14} in \texttt{set.mm}, and deleting \texttt{ax-c16} and \texttt{ax-5},
requires no \texttt{\$d} statements but is logically complete in the sense
described in \cite{Megill}.  This means it can prove any theorem of
first-order logic as long as we add to the theorem an antecedent that embeds
dummy and any other variables that must be distinct.  In a similar fashion,
axioms for set theory can be devised that
do not require distinct variable
provisos\index{Set theory without distinct variable provisos},
as explained at
\url{http://us.metamath.org/mpeuni/mmzfcnd.html}.
Together, these in principle allow all of
mathematics to be developed under Metamath without a \texttt{\$d} statement,
although the length of the resulting theorems will grow as more and
more dummy variables become required in their proofs.

\subsection{The \texttt{\$f}
and \texttt{\$e} Statements}\label{dollaref}
\index{\texttt{\$e} statement}
\index{\texttt{\$f} statement}
\index{floating hypothesis}
\index{essential hypothesis}
\index{variable-type hypothesis}
\index{logical hypothesis}
\index{hypothesis}

Metamath has two kinds of hypo\-theses, the \texttt{\$f}\index{\texttt{\$f}
statement} or {\bf variable-type} hypothesis and the \texttt{\$e} or {\bf logical}
hypo\-the\-sis.\index{\texttt{\$d} statement}\footnote{Strictly speaking, the
\texttt{\$d} statement is also a hypothesis, but it is never directly referenced
in a proof, so we call it a restriction rather than a hypothesis to lessen
confusion.  The checking for violations of \texttt{\$d} restrictions is automatic
and built into Metamath's proof-checking algorithm.} The letters \texttt{f} and
\texttt{e} stand for ``floating''\index{floating hypothesis} (roughly meaning
used only if relevant) and ``essential''\index{essential hypothesis} (meaning
always used) respectively, for reasons that will become apparent
when we discuss frames in
Section~\ref{frames} and scoping in Section~\ref{scoping}. The syntax of these
are as follows:
\begin{center}
  {\em label} \texttt{\$f} {\em typecode} {\em variable} \texttt{\$.}\\
  {\em label} \texttt{\$e} {\em typecode}
      {\em math-symbol}\ \,$\cdots$\ {\em math-symbol} \texttt{\$.}\\
\end{center}
\index{\texttt{\$e} statement}
\index{\texttt{\$f} statement}
A hypothesis must have a {\em label}\index{label}.  The expression in a
\texttt{\$e} hypothesis consists of a typecode (an active constant math symbol)
followed by a sequence
of zero or more math symbols. Each math symbol (including {\em constant}
and {\em variable}) must be a previously declared constant or variable.  (In
addition, each math symbol must be active, which will be covered when we
discuss scoping statements in Section~\ref{scoping}.)  You use a \texttt{\$f}
hypothesis to specify the
nature or {\bf type}\index{variable type}\index{type} of a variable (such as ``let $x$ be an
integer'') and use a \texttt{\$e} hypothesis to express a logical truth (such as
``assume $x$ is prime'') that must be established in order for an assertion
requiring it to also be true.

A variable must have its type specified in a \texttt{\$f} statement before
it may be used in a \texttt{\$e}, \texttt{\$a}, or \texttt{\$p}
statement.  There may be only one (active) \texttt{\$f} statement for a
given variable.  (``Active'' is defined in Section~\ref{scoping}.)

In ordinary mathematics, theorems\index{theorem} are often expressed in the
form ``Assume $P$; then $Q$,'' where $Q$ is a statement that you can derive
if you start with statement $P$.\index{free variable}\footnote{A stronger
version of a theorem like this would be the {\em single} formula $P\rightarrow
Q$ ($P$ implies $Q$) from which the weaker version above follows by the rule
of modus ponens in logic.  We are not discussing this stronger form here.  In
the weaker form, we are saying only that if we can {\em prove} $P$, then we can
{\em prove} $Q$.  In a logician's language, if $x$ is the only free variable
in $P$ and $Q$, the stronger form is equivalent to $\forall x ( P \rightarrow
Q)$ (for all $x$, $P$ implies $Q$), whereas the weaker form is equivalent to
$\forall x P \rightarrow \forall x Q$. The stronger form implies the weaker,
but not vice-versa.  To be precise, the weaker form of the theorem is more
properly called an ``inference'' rather than a theorem.}\index{inference}
In the
Metamath\index{Metamath} language, you would express mathematical statement
$P$ as a hypothesis (a \texttt{\$e} Metamath language statement in this case) and
statement $Q$ as a provable assertion (a \texttt{\$p}\index{\texttt{\$p} statement}
statement).

Some examples of hypotheses you might encounter in logic and set theory are
\begin{center}
  \texttt{stmt1 \$f wff P \$.}\\
  \texttt{stmt2 \$f set x \$.}\\
  \texttt{stmt3 \$e |- ( P -> Q ) \$.}
\end{center}
\index{\texttt{\$e} statement}
\index{\texttt{\$f} statement}
Informally, these would be read, ``Let $P$ be a well-formed-formula,'' ``Let
$x$ be an (individual) variable,'' and ``Assume we have proved $P \rightarrow
Q$.''  The turnstile symbol \,$\vdash$\index{turnstile ({$\,\vdash$})} is
commonly used in logic texts to mean ``a proof exists for.''

To summarize:
\begin{itemize}
\item A \texttt{\$f} hypothesis tells Metamath the type or kind of its variable.
It is analogous to a variable declaration in a computer language that
tells the compiler that a variable is an integer or a floating-point
number.
\item The \texttt{\$e} hypothesis corresponds to what you would usually call a
``hypothesis'' in ordinary mathematics.
\end{itemize}

Before an assertion\index{assertion} (\texttt{\$a} or \texttt{\$p} statement) can be
referenced in a proof, all of its associated \texttt{\$f} and \texttt{\$e} hypotheses
(i.e.\ those \texttt{\$e} hypotheses that are active) must be satisfied (i.e.
established by the proof).  The meaning of ``associated'' (which we will call
{\bf mandatory} in Section~\ref{frames}) will become clear when we discuss
scoping later.

Note that after any \texttt{\$f}, \texttt{\$e},
\texttt{\$a}, or \texttt{\$p} token there is a required
\textit{typecode}\index{typecode}.
The typecode is a constant used to enforce types of expressions.
This will become clearer once we learn more about
assertions (\texttt{\$a} and \texttt{\$p} statements).
An example may also clarify their purpose.
In the
\texttt{set.mm}\index{set theory database (\texttt{set.mm})}%
\index{Metamath Proof Explorer}
database,
the following typecodes are used:

\begin{itemize}
\item \texttt{wff} :
  Well-formed formula (wff) symbol
  (read: ``the following symbol sequence is a wff'').
% The *textual* typecode for turnstile is "|-", but when read it's a little
% confusing, so I intentionally display the mathematical symbol here instead
% (I think it's clearer in this context).
\item \texttt{$\vdash$} :
  Turnstile (read: ``the following symbol sequence is provable'' or
  ``a proof exists for'').
\item \texttt{setvar} :
  Individual set variable type (read: ``the following is an
  individual set variable'').
  Note that this is \textit{not} the type of an arbitrary set expression,
  instead, it is used to ensure that there is only a single symbol used
  after quantifiers like for-all ($\forall$) and there-exists ($\exists$).
\item \texttt{class} :
  An expression that is a syntactically valid class expression.
  All valid set expressions are also valid class expression, so expressions
  of sets normally have the \texttt{class} typecode.
  Use the \texttt{class} typecode,
  \textit{not} the \texttt{setvar} typecode,
  for the type of set expressions unless you are specifically identifying
  a single set variable.
\end{itemize}

\subsection{Assertions (\texttt{\$a} and \texttt{\$p} Statements)}
\index{\texttt{\$a} statement}
\index{\texttt{\$p} statement}\index{assertion}\index{axiomatic assertion}
\index{provable assertion}

There are two types of assertions, \texttt{\$a}\index{\texttt{\$a} statement}
statements ({\bf axiomatic assertions}) and \texttt{\$p} statements ({\bf
provable assertions}).  Their syntax is as follows:
\begin{center}
  {\em label} \texttt{\$a} {\em typecode} {\em math-symbol} \ldots
         {\em math-symbol} \texttt{\$.}\\
  {\em label} \texttt{\$p} {\em typecode} {\em math-symbol} \ldots
        {\em math-symbol} \texttt{\$=} {\em proof} \texttt{\$.}
\end{center}
\index{\texttt{\$a} statement}
\index{\texttt{\$p} statement}
\index{\texttt{\$=} keyword}
An assertion always requires a {\em label}\index{label}. The expression in an
assertion consists of a typecode (an active constant)
followed by a sequence of zero
or more math symbols.  Each math symbol, including any {\em constant}, must be a
previously declared constant or variable.  (In addition, each math symbol
must be active, which will be covered when we discuss scoping statements in
Section~\ref{scoping}.)

A \texttt{\$a} statement is usually a definition of syntax (for example, if $P$
and $Q$ are wffs then so is $(P\to Q)$), an axiom\index{axiom} of ordinary
mathematics (for example, $x=x$), or a definition\index{definition} of
ordinary mathematics (for example, $x\ne y$ means $\lnot x=y$). A \texttt{\$p}
statement is a claim that a certain combination of math symbols follows from
previous assertions and is accompanied by a proof that demonstrates it.

Assertions can also be referenced in (later) proofs in order to derive new
assertions from them. The label of an assertion is used to refer to it in a
proof. Section~\ref{proof} will describe the proof in detail.

Assertions also provide the primary means for communicating the mathematical
results in the database to people.  Proofs (when conveniently displayed)
communicate to people how the results were arrived at.

\subsubsection{The \texttt{\$a} Statement}
\index{\texttt{\$a} statement}

Axiomatic assertions (\texttt{\$a} statements) represent the starting points from
which other assertions (\texttt{\$p}\index{\texttt{\$p} statement} statements) are
derived.  Their most obvious use is for specifying ordinary mathematical
axioms\index{axiom}, but they are also used for two other purposes.

First, Metamath\index{Metamath} needs to know the syntax of symbol
sequences that constitute valid mathematical statements.  A Metamath
proof must be broken down into much more detail than ordinary
mathematical proofs that you may be used to thinking of (even the
``complete'' proofs of formal logic\index{formal logic}).  This is one
of the things that makes Metamath a general-purpose language,
independent of any system of logic or even syntax.  If you want to use a
substitution instance of an assertion as a step in a proof, you must
first prove that the substitution is syntactically correct (or if you
prefer, you must ``construct'' it), showing for example that the
expression you are substituting for a wff metavariable is a valid wff.
The \texttt{\$a}\index{\texttt{\$a} statement} statement is used to
specify those combinations of symbols that are considered syntactically
valid, such as the legal forms of wffs.

Second, \texttt{\$a} statements are used to specify what are ordinarily thought of
as definitions, i.e.\ new combinations of symbols that abbreviate other
combinations of symbols.  Metamath makes no distinction\index{axiom vs.\
definition} between axioms\index{axiom} and definitions\index{definition}.
Indeed, it has been argued that such distinction should not be made even in
ordinary mathematics; see Section~\ref{definitions}, which discusses the
philosophy of definitions.  Section~\ref{hierarchy} discusses some
technical requirements for definitions.  In \texttt{set.mm} we adopt the
convention of prefixing axiom labels with \texttt{ax-} and definition labels with
\texttt{df-}\index{label}.

The results that can be derived with the Metamath language are only as good as
the \texttt{\$a}\index{\texttt{\$a} statement} statements used as their starting
point.  We cannot stress this too strongly.  For example, Metamath will
not prevent you from specifying $x\neq x$ as an axiom of logic.  It is
essential that you scrutinize all \texttt{\$a} statements with great care.
Because they are a source of potential pitfalls, it is best not to add new
ones (usually new definitions) casually; rather you should carefully evaluate
each one's necessity and advantages.

Once you have in place all of the basic axioms\index{axiom} and
rules\index{rule} of a mathematical theory, the only \texttt{\$a} statements that
you will be adding will be what are ordinarily called definitions.  In
principle, definitions should be in some sense eliminable from the language of
a theory according to some convention (usually involving logical equivalence
or equality).  The most common convention is that any formula that was
syntactically valid but not provable before the definition was introduced will
not become provable after the definition is introduced.  In an ideal world,
definitions should not be present at all if one is to have absolute confidence
in a mathematical result.  However, they are necessary to make
mathematics practical, for otherwise the resulting formulas would be
extremely long and incomprehensible.  Since the nature of definitions (in the
most general sense) does not permit them to automatically be verified as
``proper,''\index{proper definition}\index{definition!proper} the judgment of
the mathematician is required to ensure it.  (In \texttt{set.mm} effort was made
to make almost all definitions directly eliminable and thus minimize the need
for such judgment.)

If you are not a mathematician, it may be best not to add or change any
\texttt{\$a}\index{\texttt{\$a} statement} statements but instead use
the mathematical language already provided in standard databases.  This
way Metamath will not allow you to make a mistake (i.e.\ prove a false
result).


\subsection{Frames}\label{frames}

We now introduce the concept of a collection of related Metamath statements
called a frame.  Every assertion (\texttt{\$a} or \texttt{\$p} statement) in the database has
an associated frame.

A {\bf frame}\index{frame} is a sequence of \texttt{\$d}, \texttt{\$f},
and \texttt{\$e} statements (zero or more of each) followed by one
\texttt{\$a} or \texttt{\$p} statement, subject to certain conditions we
will describe.  For simplicity we will assume that all math symbol
tokens used are declared at the beginning of the database with
\texttt{\$c} and \texttt{\$v} statements (which are not properly part of
a frame).  Also for simplicity we will assume there are only simple
\texttt{\$d} statements (those with only two variables) and imagine any
compound \texttt{\$d} statements (those with more than two variables) as
broken up into simple ones.

A frame groups together those hypotheses (and \texttt{\$d} statements) relevant
to an assertion (\texttt{\$a} or \texttt{\$p} statement).  The statements in a frame
may or may not be physically adjacent in a database; we will cover
this in our discussion of scoping statements
in Section~\ref{scoping}.

A frame has the following properties:
\begin{enumerate}
 \item The set of variables contained in its \texttt{\$f} statements must
be identical to the set of variables contained in its \texttt{\$e},
\texttt{\$a}, and/or \texttt{\$p} statements.  In other words, each
variable in a \texttt{\$e}, \texttt{\$a}, or \texttt{\$p} statement must
have an associated ``variable type'' defined for it in a \texttt{\$f}
statement.
  \item No two \texttt{\$f} statements may contain the same variable.
  \item Any \texttt{\$f} statement
must occur before a \texttt{\$e} statement in which its variable occurs.
\end{enumerate}

The first property determines the set of variables occurring in a frame.
These are the {\bf mandatory
variables}\index{mandatory variable} of the frame.  The second property
tells us there must be only one type specified for a variable.
The last property is not a theoretical requirement but it
makes parsing of the database easier.

For our examples, we assume our database has the following declarations:

\begin{verbatim}
$v P Q R $.
$c -> ( ) |- wff $.
\end{verbatim}

The following sequence of statements, describing the modus ponens inference
rule, is an example of a frame:

\begin{verbatim}
wp  $f wff P $.
wq  $f wff Q $.
maj $e |- ( P -> Q ) $.
min $e |- P $.
mp  $a |- Q $.
\end{verbatim}

The following sequence of statements is not a frame because \texttt{R} does not
occur in the \texttt{\$e}'s or the \texttt{\$a}:

\begin{verbatim}
wp  $f wff P $.
wq  $f wff Q $.
wr  $f wff R $.
maj $e |- ( P -> Q ) $.
min $e |- P $.
mp  $a |- Q $.
\end{verbatim}

The following sequence of statements is not a frame because \texttt{Q} does not
occur in a \texttt{\$f}:

\begin{verbatim}
wp  $f wff P $.
maj $e |- ( P -> Q ) $.
min $e |- P $.
mp  $a |- Q $.
\end{verbatim}

The following sequence of statements is not a frame because the \texttt{\$a} statement is
not the last one:

\begin{verbatim}
wp  $f wff P $.
wq  $f wff Q $.
maj $e |- ( P -> Q ) $.
mp  $a |- Q $.
min $e |- P $.
\end{verbatim}

Associated with a frame is a sequence of {\bf mandatory
hypotheses}\index{mandatory hypothesis}.  This is simply the set of all
\texttt{\$f} and \texttt{\$e} statements in the frame, in the order they
appear.  A frame can be referenced in a later proof using the label of
the \texttt{\$a} or \texttt{\$p} assertion statement, and the proof
makes an assignment to each mandatory hypothesis in the order in which
it appears.  This means the order of the hypotheses, once chosen, must
not be changed so as not to affect later proofs referencing the frame's
assertion statement.  (The Metamath proof verifier will, of course, flag
an error if a proof becomes incorrect by doing this.)  Since proofs make
use of ``Reverse Polish notation,'' described in Section~\ref{proof}, we
call this order the {\bf RPN order}\index{RPN order} of the hypotheses.

Note that \texttt{\$d} statements are not part of the set of mandatory
hypotheses, and their order doesn't matter (as long as they satisfy the
fourth property for a frame described above).  The \texttt{\$d}
statements specify restrictions on variables that must be satisfied (and
are checked by the proof verifier) when expressions are substituted for
them in a proof, and the \texttt{\$d} statements themselves are never
referenced directly in a proof.

A frame with a \texttt{\$p} (provable) statement requires a proof as part of the
\texttt{\$p} statement.  Sometimes in a proof we want to make use of temporary or
dummy variables\index{dummy variable} that do not occur in the \texttt{\$p}
statement or its mandatory hypotheses.  To accommodate this we define an {\bf
extended frame}\index{extended frame} as a frame together with zero or more
\texttt{\$d} and \texttt{\$f} statements that reference variables not among the
mandatory variables of the frame.  Any new variables referenced are called the
{\bf optional variables}\index{optional variable} of the extended frame. If a
\texttt{\$f} statement references an optional variable it is called an {\bf
optional hypothesis}\index{optional hypothesis}, and if one or both of the
variables in a \texttt{\$d} statement are optional variables it is called an {\bf
optional disjoint-variable restriction}\index{optional disjoint-variable
restriction}.  Properties 2, 3, and 4 for a frame also apply to an extended
frame.

The concept of optional variables is not meaningful for frames with \texttt{\$a}
statements, since those statements have no proofs that might make use of them.
There is no restriction on including optional hypotheses in the extended frame
for a \texttt{\$a} statement, but they serve no purpose.

The following set of statements is an example of an extended frame, which
contains an optional variable \texttt{R} and an optional hypothesis \texttt{wr}.  In
this example, we suppose the rule of modus ponens is not an axiom but is
derived as a theorem from earlier statements (we omit its presumed proof).
Variable \texttt{R} may be used in its proof if desired (although this would
probably have no advantage in propositional calculus).  Note that the sequence
of mandatory hypotheses in RPN order is still \texttt{wp}, \texttt{wq}, \texttt{maj},
\texttt{min} (i.e.\ \texttt{wr} is omitted), and this sequence is still assumed
whenever the assertion \texttt{mp} is referenced in a subsequent proof.

\begin{verbatim}
wp  $f wff P $.
wq  $f wff Q $.
wr  $f wff R $.
maj $e |- ( P -> Q ) $.
min $e |- P $.
mp  $p |- Q $= ... $.
\end{verbatim}

Every frame is an extended frame, but not every extended frame is a frame, as
this example shows.  The underlying frame for an extended frame is
obtained by simply removing all statements containing optional variables.
Any proof referencing an assertion will ignore any extensions to its
frame, which means we may add or delete optional hypotheses at will without
affecting subsequent proofs.

The conceptually simplest way of organizing a Metamath database is as a
sequence of extended frames.  The scoping statements
\texttt{\$\char`\{}\index{\texttt{\$\char`\{} and \texttt{\$\char`\}}
keywords} and \texttt{\$\char`\}} can be used to delimit the start and
end of an extended frame, leading to the following possible structure for a
database.  \label{framelist}

\vskip 2ex
\setbox\startprefix=\hbox{\tt \ \ \ \ \ \ \ \ }
\setbox\contprefix=\hbox{}
\startm
\m{\mbox{(\texttt{\$v} {\em and} \texttt{\$c}\,{\em statements})}}
\endm
\startm
\m{\mbox{\texttt{\$\char`\{}}}
\endm
\startm
\m{\mbox{\texttt{\ \ } {\em extended frame}}}
\endm
\startm
\m{\mbox{\texttt{\$\char`\}}}}
\endm
\startm
\m{\mbox{\texttt{\$\char`\{}}}
\endm
\startm
\m{\mbox{\texttt{\ \ } {\em extended frame}}}
\endm
\startm
\m{\mbox{\texttt{\$\char`\}}}}
\endm
\startm
\m{\mbox{\texttt{\ \ \ \ \ \ \ \ \ }}\vdots}
\endm
\vskip 2ex

In practice, this structure is inconvenient because we have to repeat
any \texttt{\$f}, \texttt{\$e}, and \texttt{\$d} statements over and
over again rather than stating them once for use by several assertions.
The scoping statements, which we will discuss next, allow this to be
done.  In principle, any Metamath database can be converted to the above
format, and the above format is the most convenient to use when studying
a Metamath database as a formal system%
%% Uncomment this when uncommenting section {formalspec} below
   (Appendix \ref{formalspec})%
.
In fact, Metamath internally converts the database to the above format.
The command \texttt{show statement} in the Metamath program will show
you the contents of the frame for any \texttt{\$a} or \texttt{\$p}
statement, as well as its extension in the case of a \texttt{\$p}
statement.

%c%(provided that all ``local'' variables and constants with limited scope have
%c%unique names),

During our discussion of scoping statements, it may be helpful to
think in terms of the equivalent sequence of frames that will result when
the database is parsed.  Scoping (other than the limited
use above to delimit frames) is not a theoretical requirement for
Metamath but makes it more convenient.


\subsection{Scoping Statements (\texttt{\$\{} and \texttt{\$\}})}\label{scoping}
\index{\texttt{\$\char`\{} and \texttt{\$\char`\}} keywords}\index{scoping statement}

%c%Some Metamath statements may be needed only temporarily to
%c%serve a specific purpose, and after we're done with them we would like to
%c%disregard or ignore them.  For example, when we're finished using a variable,
%c%we might want to
%c%we might want to free up the token\index{token} used to name it so that the
%c%token can be used for other purposes later on, such as a different kind of
%c%variable or even a constant.  In the terminology of computer programming, we
%c%might want to let some symbol declarations be ``local'' rather than ``global.''
%c%\index{local symbol}\index{global symbol}

The {\bf scoping} statements, \texttt{\$\char`\{} ({\bf start of block}) and \texttt{\$\char`\}}
({\bf end of block})\index{block}, provide a means for controlling the portion
of a database over which certain statement types are recognized.  The
syntax of a scoping statement is very simple; it just consists of the
statement's keyword:
\begin{center}
\texttt{\$\char`\{}\\
\texttt{\$\char`\}}
\end{center}
\index{\texttt{\$\char`\{} and \texttt{\$\char`\}} keywords}

For example, consider the following database where we have stripped out
all tokens except the scoping statement keywords.  For the purpose of the
discussion, we have added subscripts to the scoping statements; these subscripts
do not appear in the actual database.
\[
 \mbox{\tt \ \$\char`\{}_1
 \mbox{\tt \ \$\char`\{}_2
 \mbox{\tt \ \$\char`\}}_2
 \mbox{\tt \ \$\char`\{}_3
 \mbox{\tt \ \$\char`\{}_4
 \mbox{\tt \ \$\char`\}}_4
 \mbox{\tt \ \$\char`\}}_3
 \mbox{\tt \ \$\char`\}}_1
\]
Each \texttt{\$\char`\{} statement in this example is said to be {\bf
matched} with the \texttt{\$\char`\}} statement that has the same
subscript.  Each pair of matched scoping statements defines a region of
the database called a {\bf block}.\index{block} Blocks can be {\bf
nested}\index{nested block} inside other blocks; in the example, the
block defined by $\mbox{\tt \$\char`\{}_4$ and $\mbox{\tt \$\char`\}}_4$
is nested inside the block defined by $\mbox{\tt \$\char`\{}_3$ and
$\mbox{\tt \$\char`\}}_3$ as well as inside the block defined by
$\mbox{\tt \$\char`\{}_1$ and $\mbox{\tt \$\char`\}}_1$.  In general, a
block may be empty, it may contain only non-scoping
statements,\footnote{Those statements other than \texttt{\$\char`\{} and
\texttt{\$\char`\}}.}\index{non-scoping statement} or it may contain any
mixture of other blocks and non-scoping statements.  (This is called a
``recursive'' definition\index{recursive definition} of a block.)

Associated with each block is a number called its {\bf nesting
level}\index{nesting level} that indicates how deeply the block is nested.
The nesting levels of the blocks in our example are as follows:
\[
  \underbrace{
    \mbox{\tt \ }
    \underbrace{
     \mbox{\tt \$\char`\{\ }
     \underbrace{
       \mbox{\tt \$\char`\{\ }
       \mbox{\tt \$\char`\}}
     }_{2}
     \mbox{\tt \ }
     \underbrace{
       \mbox{\tt \$\char`\{\ }
       \underbrace{
         \mbox{\tt \$\char`\{\ }
         \mbox{\tt \$\char`\}}
       }_{3}
       \mbox{\tt \ \$\char`\}}
     }_{2}
     \mbox{\tt \ \$\char`\}}
   }_{1}
   \mbox{\tt \ }
 }_{0}
\]
\index{\texttt{\$\char`\{} and \texttt{\$\char`\}} keywords}
The entire database is considered to be one big block (the {\bf outermost}
block) with a nesting level of 0.  The outermost block is {\em not} bracketed
by scoping statements.\footnote{The language was designed this way so that
several source files can be joined together more easily.}\index{outermost
block}

All non-scoping Metamath statements become recognized or {\bf
active}\index{active statement} at the place where they appear.\footnote{To
keep things slightly simpler, we do not bother to define the concept of
``active'' for the scoping statements.}  Certain of these statement types
become inactive at the end of the block in which they appear; these statement
types are:
\begin{center}
  \texttt{\$c}, \texttt{\$v}, \texttt{\$d}, \texttt{\$e}, and \texttt{\$f}.
%  \texttt{\$v}, \texttt{\$f}, \texttt{\$e}, and \texttt{\$d}.
\end{center}
\index{\texttt{\$c} statement}
\index{\texttt{\$d} statement}
\index{\texttt{\$e} statement}
\index{\texttt{\$f} statement}
\index{\texttt{\$v} statement}
The other statement types remain active forever (i.e.\ through the end of the
database); they are:
\begin{center}
  \texttt{\$a} and \texttt{\$p}.
%  \texttt{\$c}, \texttt{\$a}, and \texttt{\$p}.
\end{center}
\index{\texttt{\$a} statement}
\index{\texttt{\$p} statement}
Any statement (of these 7 types) located in the outermost
block\index{outermost block} will remain active through the end of the
database and thus are effectively ``global'' statements.\index{global
statement}

All \texttt{\$c} statements must be placed in the outermost block.  Since they are
therefore always global, they could be considered as belonging to both of the
above categories.

The {\bf scope}\index{scope} of a statement is the set of statements that
recognize it as active.

%c%The concept of ``active'' is also defined for math symbols\index{math
%c%symbol}.  Math symbols (constants\index{constant} and
%c%variables\index{variable}) become {\bf active}\index{active
%c%math symbol} in the \texttt{\$c}\index{\texttt{\$c}
%c%statement} and \texttt{\$v}\index{\texttt{\$v} statement} statements that
%c%declare them.  They become inactive when their declaration statements become
%c%inactive.

The concept of ``active'' is also defined for math symbols\index{math
symbol}.  Math symbols (constants\index{constant} and
variables\index{variable}) become {\bf active}\index{active math symbol}
in the \texttt{\$c}\index{\texttt{\$c} statement} and
\texttt{\$v}\index{\texttt{\$v} statement} statements that declare them.
A variable becomes inactive when its declaration statement becomes
inactive.  Because all \texttt{\$c} statements must be in the outermost
block, a constant will never become inactive after it is declared.

\subsubsection{Redeclaration of Math Symbols}
\index{redeclaration of symbols}\label{redeclaration}

%c%A math symbol may not be declared a second time while it is active, but it may
%c%be declared again after it becomes inactive.

A variable may not be declared a second time while it is active, but it may be
declared again after it becomes inactive.  This provides a convenient way to
introduce ``local'' variables,\index{local variable} i.e.\ temporary variables
for use in the frame of an assertion or in a proof without keeping them around
forever.  A previously declared variable may not be redeclared as a constant.

A constant may not be redeclared.  And, as mentioned above, constants must be
declared in the outermost block.

The reason variables may have limited scope but not constants is that an
assertion (\texttt{\$a} or \texttt{\$p} statement) remains available for use in
proofs through the end of the database.  Variables in an assertion's frame may
be substituted with whatever is needed in a proof step that references the
assertion, whereas constants remain fixed and may not be substituted with
anything.  The particular token used for a variable in an assertion's frame is
irrelevant when the assertion is referenced in a proof, and it doesn't matter
if that token is not available outside of the referenced assertion's frame.
Constants, however, must be globally fixed.

In the present version of the Metamath language, there is no theoretical
benefit for the feature allowing variables to be active for limited scopes
rather than global. It is just a convenience that allows them, for example, to
be locally grouped together with their corresponding \texttt{\$f} variable-type
declarations.\footnote{A future version\label{redeclarationf} of the Metamath
language may extend the language to include true local variables as well as
local constants.  I have not decided upon the best way to do this or even if
it should be done at all; the main issue is whether the benefits offset
greater confusion in learning the language.  In any case, the rules for
the present language were chosen to let existing databases be compatible with
any future language extensions.}

%c%If you declare a math symbol more than once, internally Metamath considers it a
%c%new distinct symbol, even though it has the same name.  If you are unaware of
%c%this, you may find that what you think are correct proofs are incorrectly
%c%rejected as invalid, because Metamath may tell you that a constant you
%c%previously declared does not match a newly declared math symbol with the same
%c%name.  For details on this subtle point, see the Comment on
%c%p.~\pageref{spec4comment}.  This is done purposely to allow temporary
%c%constants to be introduced while developing a subtheory, then allow their math
%c%symbol tokens to be reused later on; in general they will not refer to the
%c%same thing.  In practice, you would not ordinarily reuse the names of
%c%constants because it would tend to be confusing to the reader.  The reuse of
%c%names of variables, on the other hand, is something that is often useful to do
%c%(for example it is done frequently in \texttt{set.mm}).  Since variables in an
%c%assertion referenced in a proof can be substituted as needed to achieve a
%c%symbol match, this is not an issue.

% (This section covers a somewhat advanced topic you may want to skip
% at first reading.)
%
% Under certain circumstances, math symbol\index{math symbol}
% tokens\index{token} may be redeclared (i.e.\ the token
% may appear in more than
% one \texttt{\$c}\index{\texttt{\$c} statement} or \texttt{\$v}\index{\texttt{\$v}
% statement} statement).  You might want to do this say, to make temporary use
% of a variable name without having to worry about its affect elsewhere,
% somewhat analogous to declaring a local variable in a standard computer
% language.  Understanding what goes on when math symbol tokens are redeclared
% is a little tricky to understand at first, since it requires that we
% distinguish the token itself from the math symbol that it names.  It will help
% if we first take a peek at the internal workings of the
% Metamath\index{Metamath} program.
%
% Metamath reserves a memory location for each occurrence of a
% token\index{token} in a declaration statement (\texttt{\$c}\index{\texttt{\$c}
% statement} or \texttt{\$v}\index{\texttt{\$v} statement}).  If a given token appears
% in more than one declaration statement, it will refer to more than one memory
% locations.  A math symbol\index{math symbol} may be thought of as being one of
% these memory locations rather than as the token itself.  Only one of the
% memory locations associated with a given token may be active at any one time.
% The math symbol (memory location) that gets looked up when the token appears
% in a non-declaration statement is the one that happens to be active at that
% time.
%
% We now look at the rules for the redeclaration\index{redeclaration of symbols}
% of math symbol tokens.
% \begin{itemize}
% \item A math symbol token may not be declared twice in the
% same block.\footnote{While there is no theoretical reason for disallowing
% this, it was decided in the design of Metamath that allowing it would offer no
% advantage and might cause confusion.}
% \item An inactive math symbol may always be
% redeclared.
% \item  An active math symbol may be redeclared in a different (i.e.\
% inner) block\index{block} from the one it became active in.
% \end{itemize}
%
% When a math symbol token is redeclared, it conceptually refers to a different
% math symbol, just as it would be if it were called a different name.  In
% addition, the original math symbol that it referred to, if it was active,
% temporarily becomes inactive.  At the end of the block in which the
% redeclaration occurred, the new math symbol\index{math symbol} becomes
% inactive and the original symbol becomes active again.  This concept is
% illustrated in the following example, where the symbol \texttt{e} is
% ordinarily a constant (say Euler's constant, 2.71828...) but
% temporarily we want to use it as a ``local'' variable, say as a coefficient
% in the equation $a x^4 + b x^3 + c x^2 + d x + e$:
% \[
%   \mbox{\tt \$\char`\{\ \$c e \$.}
%   \underbrace{
%     \ \ldots\ %
%     \mbox{\tt \$\char`\{}\ \ldots\ %
%   }_{\mbox{\rm region A}}
%   \mbox{\tt \$v e \$.}
%   \underbrace{
%     \mbox{\ \ \ \ldots\ \ \ }
%   }_{\mbox{\rm region B}}
%   \mbox{\tt \$\char`\}}
%   \underbrace{
%     \mbox{\ \ \ \ldots\ \ \ }
%   }_{\mbox{\rm region C}}
%   \mbox{\tt \$\char`\}}
% \]
% \index{\texttt{\$\char`\{} and \texttt{\$\char`\}} keywords}
% In region A, the token \texttt{e} refers to a constant.  It is redeclared as a
% variable in region B, and any reference to it in this region will refer to this
% variable.  In region C, the redeclaration becomes inactive, and the original
% declaration becomes active again.  In region C, the token \texttt{x} refers to the
% original constant.
%
% As a practical matter, overuse of math symbol\index{math symbol}
% redeclarations\index{redeclaration of symbols} can be confusing (even though
% it is well-defined) and is best avoided when possible.  Here are some good
% general guidelines you can follow.  Usually, you should declare all
% constants\index{constant} in the outermost block\index{outermost block},
% especially if they are general-purpose (such as the token \verb$A.$, meaning
% $\forall$ or ``for all'').  This will make them ``globally'' active (although
% as in the example above local redeclarations will temporarily make them
% inactive.)  Most or all variables\index{variable}, on the other hand, could be
% declared in inner blocks, so that the token for them can be used later for a
% different type of variable or a constant.  (The names of the variables you
% choose are not used when you refer to an assertion\index{assertion} in a
% proof, whereas constants must match exactly.  A locally declared constant will
% not match a globally declared constant in a proof, even if they use the same
% token, because Metamath internally considers them to be different math
% symbols.)  To avoid confusion, you should generally avoid redeclaring active
% variables.  If you must redeclare them, do so at the beginning of a block.
% The temporary declaration of constants in inner blocks might be occasionally
% appropriate when you make use of a temporary definition to prove lemmas
% leading to a main result that does not make direct use of the definition.
% This way, you will not clutter up your database with a large number of
% seldom-used global constant symbols.  You might want to note that while
% inactive constants may not appear directly in an assertion (a \texttt{\$a}\index{\texttt{\$a}
% statement} or \texttt{\$p}\index{\texttt{\$p} statement}
% statement), they may be indirectly used in the proof of a \texttt{\$p} statement
% so long as they do not appear in the final math symbol sequence constructed by
% the proof.  In the end, you will have to use your best judgment, taking into
% account standard mathematical usage of the symbols as well as consideration
% for the reader of your work.
%
% \subsubsection{Reuse of Labels}\index{reuse of labels}\index{label}
%
% The \texttt{\$e}\index{\texttt{\$e} statement}, \texttt{\$f}\index{\texttt{\$f}
% statement}, \texttt{\$a}\index{\texttt{\$a} statement}, and
% \texttt{\$p}\index{\texttt{\$p}
% statement} statement types require labels, which allow them to be
% referenced later inside proofs.  A label is considered {\bf
% active}\index{active label} when the statement it is associated with is
% active.  The token\index{token} for a label may be reused
% (redeclared)\index{redeclaration of labels} provided that it is not being used
% for a currently active label.  (Unlike the tokens for math symbols, active
% label tokens may not be redeclared in an inner scope.)  Note that the labels
% of \texttt{\$a} and \texttt{\$p} statements can never be reused after these
% statements appear, because these statements remain active through the end of
% the database.
%
% You might find the reuse of labels a convenient way to have standard names for
% temporary hypotheses, such as \texttt{h1}, \texttt{h2}, etc.  This way you don't have
% to invent unique names for each of them, and in some cases it may be less
% confusing to the reader (although in other cases it might be more confusing, if
% the hypothesis is located far away from the assertion that uses
% it).\footnote{The current implementation requires that all labels, even
% inactive ones, be unique.}

\subsubsection{Frames Revisited}\index{frames and scoping statements}

Now that we have covered scoping, we will look at how an arbitrary
Metamath database can be converted to the simple sequence of extended
frames described on p.~\pageref{framelist}.  This is also how Metamath
stores the database internally when it reads in the database
source.\label{frameconvert} The method is simple.  First, we collect all
constant and variable (\texttt{\$c} and \texttt{\$v}) declarations in
the database, ignoring duplicate declarations of the same variable in
different scopes.  We then put our collected \texttt{\$c} and
\texttt{\$v} declarations at the beginning of the database, so that
their scope is the entire database.  Next, for each assertion in the
database, we determine its frame and extended frame.  The extended frame
is simply the \texttt{\$f}, \texttt{\$e}, and \texttt{\$d} statements
that are active.  The frame is the extended frame with all optional
hypotheses removed.

An equivalent way of saying this is that the extended frame of an assertion
is the collection of all \texttt{\$f}, \texttt{\$e}, and \texttt{\$d} statements
whose scope includes the assertion.
The \texttt{\$f} and \texttt{\$e} statements
occur in the order they appear
(order is irrelevant for \texttt{\$d} statements).

%c%, renaming any
%c%redeclared variables as needed so that all of them have unique names.  (The
%c%exact renaming convention is unimportant.  You might imagine renaming
%c%different declarations of math symbol \texttt{a} as \texttt{a\$1}, \texttt{a\$2}, etc.\
%c%which would prevent any conflicts since \texttt{\$} is not a legal character in a
%c%math symbol token.)

\section{The Anatomy of a Proof} \label{proof}
\index{proof!Metamath, description of}

Each provable assertion (\texttt{\$p}\index{\texttt{\$p} statement} statement) in a
database must include a {\bf proof}\index{proof}.  The proof is located
between the \texttt{\$=}\index{\texttt{\$=} keyword} and \texttt{\$.}\ keywords in the
\texttt{\$p} statement.

In the basic Metamath language\index{basic language}, a proof is a
sequence of statement labels.  This label sequence\index{label sequence}
serves as a set of instructions that the Metamath program uses to
construct a series of math symbol sequences.  The construction must
ultimately result in the math symbol sequence contained between the
\texttt{\$p}\index{\texttt{\$p} statement} and
\texttt{\$=}\index{\texttt{\$=} keyword} keywords of the \texttt{\$p}
statement.  Otherwise, the Metamath program will consider the proof
incorrect, and it will notify you with an appropriate error message when
you ask it to verify the proof.\footnote{To make the loading faster, the
Metamath program does not automatically verify proofs when you
\texttt{read} in a database unless you use the \texttt{/verify}
qualifier.  After a database has been read in, you may use the
\texttt{verify proof *} command to verify proofs.}\index{\texttt{verify
proof} command} Each label in a proof is said to {\bf
reference}\index{label reference} its corresponding statement.

Associated with any assertion\index{assertion} (\texttt{\$p} or
\texttt{\$a}\index{\texttt{\$a} statement} statement) is a set of
hypotheses (\texttt{\$f}\index{\texttt{\$f} statement} or
\texttt{\$e}\index{\texttt{\$e} statement} statements) that are active
with respect to that assertion.  Some are mandatory and the others are
optional.  You should review these concepts if necessary.

Each label\index{label} in a proof must be either the label of a
previous assertion (\texttt{\$a}\index{\texttt{\$a} statement} or
\texttt{\$p}\index{\texttt{\$p} statement} statement) or the label of an
active hypothesis (\texttt{\$e} or \texttt{\$f}\index{\texttt{\$f}
statement} statement) of the \texttt{\$p} statement containing the
proof.  Hypothesis labels may reference both the
mandatory\index{mandatory hypothesis} and the optional hypotheses of the
\texttt{\$p} statement.

The label sequence in a proof specifies a construction in {\bf reverse Polish
notation}\index{reverse Polish notation (RPN)} (RPN).  You may be familiar
with RPN if you have used older
Hewlett--Packard or similar hand-held calculators.
In the calculator analogy, a hypothesis label\index{hypothesis label} is like
a number and an assertion label\index{assertion label} is like an operation
(more precisely, an $n$-ary operation when the
assertion has $n$ \texttt{\$e}-hypotheses).
On an RPN calculator, an operation takes one or more previous numbers in an
input sequence, performs a calculation on them, and replaces those numbers and
itself with the result of the calculation.  For example, the input sequence
$2,3,+$ on an RPN calculator results in $5$, and the input sequence
$2,3,5,{\times},+$ results in $2,15,+$ which results in $17$.

Understanding how RPN is processed involves the concept of a {\bf
stack}\index{stack}\index{RPN stack}, which can be thought of as a set of
temporary memory locations that hold intermediate results.  When Metamath
encounters a hypothesis label it places or {\bf pushes}\index{push} the math
symbol sequence of the hypothesis onto the stack.  When Metamath encounters an
assertion label, it associates the most recent stack entries with the {\em
mandatory} hypotheses\index{mandatory hypothesis} of the assertion, in the
order where the most recent stack entry is associated with the last mandatory
hypothesis of the assertion.  It then determines what
substitutions\index{substitution!variable}\index{variable substitution} have
to be made into the variables of the assertion's mandatory hypotheses to make
them identical to the associated stack entries.  It then makes those same
substitutions into the assertion itself.  Finally, Metamath removes or {\bf
pops}\index{pop} the matched hypotheses from the stack and pushes the
substituted assertion onto the stack.

For the purpose of matching the mandatory hypothesis to the most recent stack
entries, whether a hypothesis is a \texttt{\$e} or \texttt{\$f} statement is
irrelevant.  The only important thing is that a set of
substitutions\footnote{In the Metamath spec (Section~\ref{spec}), we use the
singular term ``substitution'' to refer to the set of substitutions we talk
about here.} exist that allow a match (and if they don't, the proof verifier
will let you know with an error message).  The Metamath language is specified
in such a way that if a set of substitutions exists, it will be unique.
Specifically, the requirement that each variable have a type specified for it
with a \texttt{\$f} statement ensures the uniqueness.

We will illustrate this with an example.
Consider the following Metamath source file:
\begin{verbatim}
$c ( ) -> wff $.
$v p q r s $.
wp $f wff p $.
wq $f wff q $.
wr $f wff r $.
ws $f wff s $.
w2 $a wff ( p -> q ) $.
wnew $p wff ( s -> ( r -> p ) ) $= ws wr wp w2 w2 $.
\end{verbatim}
This Metamath source example shows the definition and ``proof'' (i.e.,
construction) of a well-formed formula (wff)\index{well-formed formula (wff)}
in propositional calculus.  (You may wish to type this example into a file to
experiment with the Metamath program.)  The first two statements declare
(introduce the names of) four constants and four variables.  The next four
statements specify the variable types, namely that
each variable is assumed to be a wff.  Statement \texttt{w2} defines (postulates)
a way to produce a new wff, \texttt{( p -> q )}, from two given wffs \texttt{p} and
\texttt{q}. The mandatory hypotheses of \texttt{w2} are \texttt{wp} and \texttt{wq}.
Statement \texttt{wnew} claims that \texttt{( s -> ( r -> p ) )} is a wff given
three wffs \texttt{s}, \texttt{r}, and \texttt{p}.  More precisely, \texttt{wnew} claims
that the sequence of ten symbols \texttt{wff ( s -> ( r -> p ) )} is provable from
previous assertions and the hypotheses of \texttt{wnew}.  Metamath does not know
or care what a wff is, and as far as it is concerned
the typecode \texttt{wff} is just an
arbitrary constant symbol in a math symbol sequence.  The mandatory hypotheses
of \texttt{wnew} are \texttt{wp}, \texttt{wr}, and \texttt{ws}; \texttt{wq} is an optional
hypothesis.  In our particular proof, the optional hypothesis is not
referenced, but in general, any combination of active (i.e.\ optional and
mandatory) hypotheses could be referenced.  The proof of statement \texttt{wnew}
is the sequence of five labels starting with \texttt{ws} (step~1) and ending with
\texttt{w2} (step~5).

When Metamath verifies the proof, it scans the proof from left to right.  We
will examine what happens at each step of the proof.  The stack starts off
empty.  At step 1, Metamath looks up label \texttt{ws} and determines that it is a
hypothesis, so it pushes the symbol sequence of statement \texttt{ws} onto the
stack:

\begin{center}\begin{tabular}{|l|l|}\hline
{Stack location} & {Contents} \\ \hline \hline
1 & \texttt{wff s} \\ \hline
\end{tabular}\end{center}

Metamath sees that the labels \texttt{wr} and \texttt{wp} in steps~2 and 3 are also
hypotheses, so it pushes them onto the stack.  After step~3, the stack looks
like
this:

\begin{center}\begin{tabular}{|l|l|}\hline
{Stack location} & {Contents} \\ \hline \hline
3 & \texttt{wff p} \\ \hline
2 & \texttt{wff r} \\ \hline
1 & \texttt{wff s} \\ \hline
\end{tabular}\end{center}

At step 4, Metamath sees that label \texttt{w2} is an assertion, so it must do
some processing.  First, it associates the mandatory hypotheses of \texttt{w2},
which are \texttt{wp} and \texttt{wq}, with stack locations~2 and 3, {\em in that
order}. Metamath determines that the only possible way
to make hypothesis \texttt{wp} match (become identical to) stack location~2 and
\texttt{wq} match stack location 3 is to substitute variable \texttt{p} with \texttt{r}
and \texttt{q} with \texttt{p}.  Metamath makes these substitutions into \texttt{w2} and
obtains the symbol sequence \texttt{wff ( r -> p )}.  It removes the hypotheses
from stack locations~2 and 3, then places the result into stack location~2:

\begin{center}\begin{tabular}{|l|l|}\hline
{Stack location} & {Contents} \\ \hline \hline
2 & \texttt{wff ( r -> p )} \\ \hline
1 & \texttt{wff s} \\ \hline
\end{tabular}\end{center}

At step 5, Metamath sees that label \texttt{w2} is an assertion, so it must again
do some processing.  First, it matches the mandatory hypotheses of \texttt{w2},
which are \texttt{wp} and \texttt{wq}, to stack locations 1 and 2.
Metamath determines that the only possible way to make the
hypotheses match is to substitute variable \texttt{p} with \texttt{s} and \texttt{q} with
\texttt{( r -> p )}.  Metamath makes these substitutions into \texttt{w2} and obtains
the symbol
sequence \texttt{wff ( s -> ( r -> p ) )}.  It removes stack
locations 1 and 2, then places the result into stack location~1:

\begin{center}\begin{tabular}{|l|l|}\hline
{Stack location} & {Contents} \\ \hline \hline
1 & \texttt{wff ( s -> ( r -> p ) )} \\ \hline
\end{tabular}\end{center}

After Metamath finishes processing the proof, it checks to see that the
stack contains exactly one element and that this element is
the same as the math symbol sequence in the
\texttt{\$p}\index{\texttt{\$p} statement} statement.  This is the case for our
proof of \texttt{wnew},
so we have proved \texttt{wnew} successfully.  If the result
differs, Metamath will notify you with an error message.  An error message
will also result if the stack contains more than one entry at the end of the
proof, or if the stack did not contain enough entries at any point in the
proof to match all of the mandatory hypotheses\index{mandatory hypothesis} of
an assertion.  Finally, Metamath will notify you with an error message if no
substitution is possible that will make a referenced assertion's hypothesis
match the
stack entries.  You may want to experiment with the different kinds of errors
that Metamath will detect by making some small changes in the proof of our
example.

Metamath's proof notation was designed primarily to express proofs in a
relatively compact manner, not for readability by humans.  Metamath can display
proofs in a number of different ways with the \texttt{show proof}\index{\texttt{show
proof} command} command.  The
\texttt{/lemmon} qualifier displays it in a format that is easier to read when the
proofs are short, and you saw examples of its use in Chapter~\ref{using}.  For
longer proofs, it is useful to see the tree structure of the proof.  A tree
structure is displayed when the \texttt{/lemmon} qualifier is omitted.  You will
probably find this display more convenient as you get used to it. The tree
display of the proof in our example looks like
this:\label{treeproof}\index{tree-style proof}\index{proof!tree-style}
\begin{verbatim}
1     wp=ws    $f wff s
2        wp=wr    $f wff r
3        wq=wp    $f wff p
4     wq=w2    $a wff ( r -> p )
5  wnew=w2  $a wff ( s -> ( r -> p ) )
\end{verbatim}
The number to the left of each line is the step number.  Following it is a
{\bf hypothesis association}\index{hypothesis association}, consisting of two
labels\index{label} separated by \texttt{=}.  To the left of the \texttt{=} (except
in the last step) is the label of a hypothesis of an assertion referenced
later in the proof; here, steps 1 and 4 are the hypothesis associations for
the assertion \texttt{w2} that is referenced in step 5.  A hypothesis association
is indented one level more than the assertion that uses it, so it is easy to
find the corresponding assertion by moving directly down until the indentation
level decreases to one less than where you started from.  To the right of each
\texttt{=} is the proof step label for that proof step.  The statement keyword of
the proof step label is listed next, followed by the content of the top of the
stack (the most recent stack entry) as it exists after that proof step is
processed.  With a little practice, you should have no trouble reading proofs
displayed in this format.

Metamath proofs include the syntax construction of a formula.
In standard mathematics, this kind of
construction is not considered a proper part of the proof at all, and it
certainly becomes rather boring after a while.
Therefore,
by default the \texttt{show proof}\index{\texttt{show proof}
command} command does not show the syntax construction.
Historically \texttt{show proof} command
\textit{did} show the syntax construction, and you needed to add the
\texttt{/essential} option to hide, them, but today
\texttt{/essential} is the default and you need to use
\texttt{/all} to see the syntax constructions.

When verifying a proof, Metamath will check that no mandatory
\texttt{\$d}\index{\texttt{\$d} statement}\index{mandatory \texttt{\$d}
statement} statement of an assertion referenced in a proof is violated
when substitutions\index{substitution!variable}\index{variable
substitution} are made to the variables in the assertion.  For details
see Section~\ref{spec4} or \ref{dollard}.

\subsection{The Concept of Unification} \label{unify}

During the course of verifying a proof, when Metamath\index{Metamath}
encounters an assertion label\index{assertion label}, it associates the
mandatory hypotheses\index{mandatory hypothesis} of the assertion with the top
entries of the RPN stack\index{stack}\index{RPN stack}.  Metamath then
determines what substitutions\index{substitution!variable}\index{variable
substitution} it must make to the variables in the assertion's mandatory
hypotheses in order for these hypotheses to become identical to their
corresponding stack entries.  This process is called {\bf
unification}\index{unification}.  (We also informally use the term
``unification'' to refer to a set of substitutions that results from the
process, as in ``two unifications are possible.'')  After the substitutions
are made, the hypotheses are said to be {\bf unified}.

If no such substitutions are possible, Metamath will consider the proof
incorrect and notify you with an error message.
% (deleted 3/10/07, per suggestion of Mel O'Cat:)
% The syntax of the
% Metamath language ensures that if a set of substitutions exists, it
% will be unique.

The general algorithm for unification described in the literature is
somewhat complex.
However, in the case of Metamath it is intentionally trivial.
Mandatory hypotheses must be
pushed on the proof stack in the order in which they appear.
In addition, each variable must have its type specified
with a \texttt{\$f} hypothesis before it is used
and that each \texttt{\$f} hypothesis
have the restricted syntax of a typecode (a constant) followed by a variable.
The typecode in the \texttt{\$f} hypothesis must match the first symbol of
the corresponding RPN stack entry (which will also be a constant), so
the only possible match for the variable in the \texttt{\$f} hypothesis is
the sequence of symbols in the stack entry after the initial constant.

In the Proof Assistant\index{Proof Assistant}, a more general unification
algorithm is used.  While a proof is being developed, sometimes not enough
information is available to determine a unique unification.  In this case
Metamath will ask you to pick the correct one.\index{ambiguous
unification}\index{unification!ambiguous}

\section{Extensions to the Metamath Language}\index{extended
language}

\subsection{Comments in the Metamath Language}\label{comments}
\index{markup notation}
\index{comments!markup notation}

The commenting feature allows you to annotate the contents of
a database.  Just as with most
computer languages, comments are ignored for the purpose of interpreting the
contents of the database. Comments effectively act as
additional white space\index{white
space} between tokens
when a database is parsed.

A comment may be placed at the beginning, end, or
between any two tokens\index{token} in a source file.

Comments have the following syntax:
\begin{center}
 \texttt{\$(} {\em text} \texttt{\$)}
\end{center}
Here,\index{\texttt{\$(} and \texttt{\$)} auxiliary
keywords}\index{comment} {\em text} is a string, possibly empty, of any
characters in Metamath's character set (p.~\pageref{spec1chars}), except
that the character strings \texttt{\$(} and \texttt{\$)} may not appear
in {\em text}.  Thus nested comments are not
permitted:\footnote{Computer languages have differing standards for
nested comments, and rather than picking one it was felt simplest not to
allow them at all, at least in the current version of
Metamath\index{Metamath!limitations of version 0.07.30}.} Metamath will
complain if you give it
\begin{center}
 \texttt{\$( This is a \$( nested \$) comment.\ \$)}
\end{center}
To compensate for this non-nesting behavior, I often change all \texttt{\$}'s
to \texttt{@}'s in sections of Metamath code I wish to comment out.

The Metamath program supports a number of markup mechanisms and conventions
to generate good-looking results in \LaTeX\ and {\sc html},
as discussed below.
These markup features have to do only with how the comments are typeset,
and have no effect on how Metamath verifies the proofs in the database.
The improper
use of them may result in incorrectly typeset output, but no Metamath
error messages will result during the \texttt{read} and \texttt{verify
proof} commands.  (However, the \texttt{write
theorem\texttt{\char`\_}list} command
will check for markup errors as a side-effect of its
{\sc html} generation.)
Section~\ref{texout} has instructions for creating \LaTeX\ output, and
section~\ref{htmlout} has instructions for creating
{\sc html}\index{HTML} output.

\subsubsection{Headings}\label{commentheadings}

If the \texttt{\$(} is immediately followed by a new line
starting with a heading marker, it is a header.
This can start with:

\begin{itemize}
 \item[] \texttt{\#\#\#\#} - major part header
 \item[] \texttt{\#*\#*} - section header
 \item[] \texttt{=-=-} - subsection header
 \item[] \texttt{-.-.} - subsubsection header
\end{itemize}

The line following the marker line
will be used for the table of contents entry, after trimming spaces.
The next line should be another (closing) matching marker line.
Any text after that
but before the closing \texttt{\$}, such as an extended description of the
section, will be included on the \texttt{mmtheoremsNNN.html} page.

For more information, run
\texttt{help write theorem\char`\_list}.

\subsubsection{Math mode}
\label{mathcomments}
\index{\texttt{`} inside comments}
\index{\texttt{\char`\~} inside comments}
\index{math mode}

Inside of comments, a string of tokens\index{token} enclosed in
grave accents\index{grave accent (\texttt{`})} (\texttt{`}) will be converted
to standard mathematical symbols during
{\sc HTML}\index{HTML} or \LaTeX\ output
typesetting,\index{latex@{\LaTeX}} according to the information in the
special \texttt{\$t}\index{\texttt{\$t} comment}\index{typesetting
comment} comment in the database
(see section~\ref{tcomment} for information about the typesetting
comment, and Appendix~\ref{ASCII} to see examples of its results).

The first grave accent\index{grave accent (\texttt{`})} \texttt{`}
causes the output processor to enter {\bf math mode}\index{math mode}
and the second one exits it.
In this
mode, the characters following the \texttt{`} are interpreted as a
sequence of math symbol tokens separated by white space\index{white
space}.  The tokens are looked up in the \texttt{\$t}
comment\index{\texttt{\$t} comment}\index{typesetting comment} and if
found, they will be replaced by the standard mathematical symbols that
they correspond to before being placed in the typeset output file.  If
not found, the symbol will be output as is and a warning will be issued.
The tokens do not have to be active in the database, although a warning
will be issued if they are not declared with \texttt{\$c} or
\texttt{\$v} statements.

Two consecutive
grave accents \texttt{``} are treated as a single actual grave accent
(both inside and outside of math mode) and will not cause the output
processor to enter or exit math mode.

Here is an example of its use\index{Pierce's axiom}:
\begin{center}
\texttt{\$( Pierce's axiom, ` ( ( ph -> ps ) -> ph ) -> ph ` ,\\
         is not very intuitive. \$)}
\end{center}
becomes
\begin{center}
   \texttt{\$(} Pierce's axiom, $((\varphi \rightarrow \psi)\rightarrow
\varphi)\rightarrow \varphi$, is not very intuitive. \texttt{\$)}
\end{center}

Note that the math symbol tokens\index{token} must be surrounded by white
space\index{white space}.
%, since there is no context that allows ambiguity to be
%resolved, as is the case with math symbol sequences in some of the Metamath
%statements.
White space should also surround the \texttt{`}
delimiters.

The math mode feature also gives you a quick and easy way to generate
text containing mathematical symbols, independently of the intended
purpose of Metamath.\index{Metamath!using as a math editor} To do this,
simply create your text with grave accents surrounding your formulas,
after making sure that your math symbols are mapped to \LaTeX\ symbols
as described in Appendix~\ref{ASCII}.  It is easier if you start with a
database with predefined symbols such as \texttt{set.mm}.  Use your
grave-quoted math string to replace an existing comment, then typeset
the statement corresponding to that comment following the instructions
from the \texttt{help tex} command in the Metamath program.  You will
then probably want to edit the resulting file with a text editor to fine
tune it to your exact needs.

\subsubsection{Label Mode}\index{label mode}

Outside of math mode, a tilde\index{tilde (\texttt{\char`\~})} \verb/~/
indicates to Metamath's\index{Metamath} output processor that the
token\index{token} that follows (i.e.\ the characters up to the next
white space\index{white space}) represents a statement label or URL.
This formatting mode is called {\bf label mode}\index{label mode}.
If a literal tilde
is desired (outside of math mode) instead of label mode,
use two tildes in a row to represent it.

When generating a \LaTeX\ output file,
the following token will be formatted in \texttt{typewriter}
font, and the tilde removed, to make it stand out from the rest of the text.
This formatting will be applied to all characters after the
tilde up to the first white space\index{white space}.
Whether
or not the token is an actual statement label is not checked, and the
token does not have to have the correct syntax for a label; no error
messages will be produced.  The only effect of the label mode on the
output is that typewriter font will be used for the tokens that are
placed in the \LaTeX\ output file.

When generating {\sc html},
the tokens after the tilde {\em must} be a URL (either http: or https:)
or a valid label.
Error messages will be issued during that output if they aren't.
A hyperlink will be generated to that URL or label.

\subsubsection{Link to bibliographical reference}\index{citation}%
\index{link to bibliographical reference}

Bibliographical references are handled specially when generating
{\sc html} if formatted specially.
Text in the form \texttt{[}{\em author}\texttt{]}
is considered a link to a bibliographical reference.
See \texttt{help html} and \texttt{help write
bibliography} in the Metamath program for more
information.
% \index{\texttt{\char`\[}\ldots\texttt{]} inside comments}
See also Sections~\ref{tcomment} and \ref{wrbib}.

The \texttt{[}{\em author}\texttt{]} notation will also create an entry in
the bibliography cross-reference file generated by \texttt{write
bibliography} (Section~\ref{wrbib}) for {\sc HTML}.
For this to work properly, the
surrounding comment must be formatted as follows:
\begin{quote}
    {\em keyword} {\em label} {\em noise-word}
     \texttt{[}{\em author}\texttt{] p.} {\em number}
\end{quote}
for example
\begin{verbatim}
     Theorem 5.2 of [Monk] p. 223
\end{verbatim}
The {\em keyword} is not case sensitive and must be one of the following:
\begin{verbatim}
     theorem lemma definition compare proposition corollary
     axiom rule remark exercise problem notation example
     property figure postulate equation scheme chapter
\end{verbatim}
The optional {\em label} may consist of more than one
(non-{\em keyword} and non-{\em noise-word}) word.
The optional {\em noise-word} is one of:
\begin{verbatim}
     of in from on
\end{verbatim}
and is  ignored when the cross-reference file is created.  The
\texttt{write
biblio\-graphy} command will perform error checking to verify the
above format.\index{error checking}

\subsubsection{Parentheticals}\label{parentheticals}

The end of a comment may include one or more parenthicals, that is,
statements enclosed in parentheses.
The Metamath program looks for certain parentheticals and can issue
warnings based on them.
They are:

\begin{itemize}
 \item[] \texttt{(Contributed by }
   \textit{NAME}\texttt{,} \textit{DATE}\texttt{.)} -
   document the original contributor's name and the date it was created.
 \item[] \texttt{(Revised by }
   \textit{NAME}\texttt{,} \textit{DATE}\texttt{.)} -
   document the contributor's name and creation date
   that resulted in significant revision
   (not just an automated minimization or shortening).
 \item[] \texttt{(Proof shortened by }
   \textit{NAME}\texttt{,} \textit{DATE}\texttt{.)} -
   document the contributor's name and date that developed a significant
   shortening of the proof (not just an automated minimization).
 \item[] \texttt{(Proof modification is discouraged.)} -
   Note that this proof should normally not be modified.
 \item[] \texttt{(New usage is discouraged.)} -
   Note that this assertion should normally not be used.
\end{itemize}

The \textit{DATE} must be in form YYYY-MMM-DD, where MMM is the
English abbreviation of that month.

\subsubsection{Other markup}\label{othermarkup}
\index{markup notation}

There are other markup notations for generating good-looking results
beyond math mode and label mode:

\begin{itemize}
 \item[]
         \texttt{\char`\_} (underscore)\index{\texttt{\char`\_} inside comments} -
             Italicize text starting from
              {\em space}\texttt{\char`\_}{\em non-space} (i.e.\ \texttt{\char`\_}
              with a space before it and a non-space character after it) until
             the next
             {\em non-space}\texttt{\char`\_}{\em space}.  Normal
             punctuation (e.g.\ a trailing
             comma or period) is ignored when determining {\em space}.
 \item[]
         \texttt{\char`\_} (underscore) - {\em
         non-space}\texttt{\char`\_}{\em non-space-string}, where
          {\em non-space-string} is a string of non-space characters,
         will make {\em non-space-string} become a subscript.
 \item[]
         \texttt{<HTML>}...\texttt{</HTML>} - do not convert
         ``\texttt{<}'' and ``\texttt{>}''
         in the enclosed text when generating {\sc HTML},
         otherwise process markup normally. This allows direct insertion
         of {\sc html} commands.
 \item[]
       ``\texttt{\&}ref\texttt{;}'' - insert an {\sc HTML}
         character reference.
         This is how to insert arbitrary Unicode characters
         (such as accented characters).  Currently only directly supported
         when generating {\sc HTML}.
\end{itemize}

It is recommended that spaces surround any \texttt{\char`\~} and
\texttt{`} tokens in the comment and that a space follow the {\em label}
after a \texttt{\char`\~} token.  This will make global substitutions
to change labels and symbol names much easier and also eliminate any
future chance of ambiguity.  Spaces around these tokens are automatically
removed in the final output to conform with normal rules of punctuation;
for example, a space between a trailing \texttt{`} and a left parenthesis
will be removed.

A good way to become familiar with the markup notation is to look at
the extensive examples in the \texttt{set.mm} database.

\subsection{The Typesetting Comment (\texttt{\$t})}\label{tcomment}

The typesetting comment \texttt{\$t} in the input database file
provides the information necessary to produce good-looking results.
It provides \LaTeX\ and {\sc html}
definitions for math symbols,
as well supporting as some
customization of the generated web page.
If you add a new token to a database, you should also
update the \texttt{\$t} comment information if you want to eventually
create output in \LaTeX\ or {\sc HTML}.
See the
\texttt{set.mm}\index{set theory database (\texttt{set.mm})} database
file for an extensive example of a \texttt{\$t} comment illustrating
many of the features described below.

Programs that do not need to generate good-looking presentation results,
such as programs that only verify Metamath databases,
can completely ignore typesetting comments
and just treat them as normal comments.
Even the Metamath program only consults the
\texttt{\$t} comment information when it needs to generate typeset output
in \LaTeX\ or {\sc HTML}
(e.g., when you open a \LaTeX\ output file with the \texttt{open tex} command).

We will first discuss the syntax of typesetting comments, and then
briefly discuss how this can be used within the Metamath program.

\subsubsection{Typesetting Comment Syntax Overview}

The typesetting comment is identified by the token
\texttt{\$t}\index{\texttt{\$t} comment}\index{typesetting comment} in
the comment, and the typesetting comment ends at the matching
\texttt{\$)}:
\[
  \mbox{\tt \$(\ }
  \mbox{\tt \$t\ }
  \underbrace{
    \mbox{\tt \ \ \ \ \ \ \ \ \ \ \ }
    \cdots
    \mbox{\tt \ \ \ \ \ \ \ \ \ \ \ }
  }_{\mbox{Typesetting definitions go here}}
  \mbox{\tt \ \$)}
\]

There must be one or more white space characters, and only white space
characters, between the \texttt{\$(} that starts the comment
and the \texttt{\$t} symbol,
and the \texttt{\$t} must be followed by one
or more white space characters
(see section \ref{whitespace} for the definition of white space characters).
The typesetting comment continues until the comment end token \texttt{\$)}
(which must be preceded by one or more white space characters).

In version 0.07.30\index{Metamath!limitations of version 0.07.30} of the
Metamath program, there may be only one \texttt{\$t} comment in a
database.  This restriction may be lifted in the future to allow
many \texttt{\$t} comments in a database.

Between the \texttt{\$t} symbol (and its following white space) and the
comment end token \texttt{\$)} (and its preceding white space)
is a sequence of one or more typesetting definitions, where
each definition has the form
\textit{definition-type arg arg ... ;}.
Each of the zero or more \textit{arg} values
can be either a typesetting data or a keyword
(what keywords are allowed, and where, depends on the specific
\textit{definition-type}).
The \textit{definition-type}, and each argument \textit{arg},
are separated by one or more white space characters.
Every definition ends in an unquoted semicolon;
white space is not required before the terminating semicolon of a definition.
Each definition should start on a new line.\footnote{This
restriction of the current version of Metamath
(0.07.30)\index{Metamath!limitations of version 0.07.30} may be removed
in a future version, but you should do it anyway for readability.}

For example, this typesetting definition:
\begin{center}
 \verb$latexdef "C_" as "\subseteq";$
\end{center}
defines the token \verb$C_$ as the \LaTeX\ symbol $\subseteq$ (which means
``subset'').

Typesetting data is a sequence of one or more quoted strings
(if there is more than one, they are connected by \texttt{\char`\+}).
Often a single quoted string is used to provide data for a definition, using
either double (\texttt{\char`\"}) or single (\texttt{'}) quotation marks.
However,
{\em a quoted string (enclosed in quotation marks) may not include
line breaks.}
A quoted string
may include a quotation mark that matches the enclosing quotes by repeating
the quotation mark twice.  Here are some examples:

\begin{tabu}   { l l }
\textbf{Example} & \textbf{Meaning} \\
\texttt{\char`\"a\char`\"\char`\"b\char`\"} & \texttt{a\char`\"b} \\
\texttt{'c''d'} & \texttt{c'd} \\
\texttt{\char`\"e''f\char`\"} & \texttt{e''f} \\
\texttt{'g\char`\"\char`\"h'} & \texttt{g\char`\"\char`\"h} \\
\end{tabu}

Finally, a long quoted string
may be broken up into multiple quoted strings (considered, as a whole,
a single quoted string) and joined with \texttt{\char`\+}.
You can even use multiple lines as long as a
'+' is at the end of every line except the last one.
The \texttt{\char`\+} should be preceded and followed by at least one
white space character.
Thus, for example,
\begin{center}
 \texttt{\char`\"ab\char`\"\ \char`\+\ \char`\"cd\char`\"
    \ \char`\+\ \\ 'ef'}
\end{center}
is the same as
\begin{center}
 \texttt{\char`\"abcdef\char`\"}
\end{center}

{\sc c}-style comments \texttt{/*}\ldots\texttt{*/} are also supported.

In practice, whenever you add a new math token you will often want to add
typesetting definitions using
\texttt{latexdef}, \texttt{htmldef}, and
\texttt{althtmldef}, as described below.
That way, they will all be up to date.
Of course, whether or not you want to use all three definitions will
depend on how the database is intended to be used.

Below we discuss the different possible \textit{definition-kind} options.
We will show data surrounded by double quotes (in practice they can also use
single quotes and/or be a sequence joined by \texttt{+}s).
We will use specific names for the \textit{data} to make clear what
the data is used for, such as
{\em math-token} (for a Metamath math token,
{\em latex-string} (for string to be placed in a \LaTeX\ stream),
{\em {\sc html}-code} (for {\sc html} code),
and {\em filename} (for a filename).

\subsubsection{Typesetting Comment - \LaTeX}

The syntax for a \LaTeX\ definition is:
\begin{center}
 \texttt{latexdef "}{\em math-token}\texttt{" as "}{\em latex-string}\texttt{";}
\end{center}
\index{latex definitions@\LaTeX\ definitions}%
\index{\texttt{latexdef} statement}

The {\em token-string} and {\em latex-string} are the data
(character strings) for
the token and the \LaTeX\ definition of the token, respectively,

These \LaTeX\ definitions are used by the Metamath program
when it is asked to product \LaTeX output using
the \texttt{write tex} command.

\subsubsection{Typesetting Comment - {\sc html}}

The key kinds of {\sc HTML} definitions have the following syntax:

\vskip 1ex
    \texttt{htmldef "}{\em math-token}\texttt{" as "}{\em
    {\sc html}-code}\texttt{";}\index{\texttt{htmldef} statement}
                    \ \ \ \ \ \ldots

    \texttt{althtmldef "}{\em math-token}\texttt{" as "}{\em
{\sc html}-code}\texttt{";}\index{\texttt{althtmldef} statement}

                    \ \ \ \ \ \ldots

Note that in {\sc HTML} there are two possible definitions for math tokens.
This feature is useful when
an alternate representation of symbols is desired, for example one that
uses Unicode entities and another uses {\sc gif} images.

There are many other typesetting definitions that can control {\sc HTML}.
These include:

\vskip 1ex

    \texttt{htmldef "}{\em math-token}\texttt{" as "}{\em {\sc
    html}-code}\texttt{";}

    \texttt{htmltitle "}{\em {\sc html}-code}\texttt{";}%
\index{\texttt{htmltitle} statement}

    \texttt{htmlhome "}{\em {\sc html}-code}\texttt{";}%
\index{\texttt{htmlhome} statement}

    \texttt{htmlvarcolor "}{\em {\sc html}-code}\texttt{";}%
\index{\texttt{htmlvarcolor} statement}

    \texttt{htmlbibliography "}{\em filename}\texttt{";}%
\index{\texttt{htmlbibliography} statement}

\vskip 1ex

\noindent The \texttt{htmltitle} is the {\sc html} code for a common
title, such as ``Metamath Proof Explorer.''  The \texttt{htmlhome} is
code for a link back to the home page.  The \texttt{htmlvarcolor} is
code for a color key that appears at the bottom of each proof.  The file
specified by {\em filename} is an {\sc html} file that is assumed to
have a \texttt{<A NAME=}\ldots\texttt{>} tag for each bibiographic
reference in the database comments.  For example, if
\texttt{[Monk]}\index{\texttt{\char`\[}\ldots\texttt{]} inside comments}
occurs in the comment for a theorem, then \texttt{<A NAME='Monk'>} must
be present in the file; if not, a warning message is given.

Associated with
\texttt{althtmldef}
are the statements
\vskip 1ex

    \texttt{htmldir "}{\em
      directoryname}\texttt{";}\index{\texttt{htmldir} statement}

    \texttt{althtmldir "}{\em
     directoryname}\texttt{";}\index{\texttt{althtmldir} statement}

\vskip 1ex
\noindent giving the directories of the {\sc gif} and Unicode versions
respectively; their purpose is to provide cross-linking between the
two versions in the generated web pages.

When two different types of pages need to be produced from a single
database, such as the Hilbert Space Explorer that extends the Metamath
Proof Explorer, ``extended'' variables may be declared in the
\texttt{\$t} comment:
\vskip 1ex

    \texttt{exthtmltitle "}{\em {\sc html}-code}\texttt{";}%
\index{\texttt{exthtmltitle} statement}

    \texttt{exthtmlhome "}{\em {\sc html}-code}\texttt{";}%
\index{\texttt{exthtmlhome} statement}

    \texttt{exthtmlbibliography "}{\em filename}\texttt{";}%
\index{\texttt{exthtmlbibliography} statement}

\vskip 1ex
\noindent When these are declared, you also must declare
\vskip 1ex

    \texttt{exthtmllabel "}{\em label}\texttt{";}%
\index{\texttt{exthtmllabel} statement}

\vskip 1ex \noindent that identifies the database statement where the
``extended'' section of the database starts (in our example, where the
Hilbert Space Explorer starts).  During the generation of web pages for
that starting statement and the statements after it, the {\sc html} code
assigned to \texttt{exthtmltitle} and \texttt{exthtmlhome} is used
instead of that assigned to \texttt{htmltitle} and \texttt{htmlhome},
respectively.

\begin{sloppy}
\subsection{Additional Information Com\-ment (\texttt{\$j})} \label{jcomment}
\end{sloppy}

The additional information comment, aka the
\texttt{\$j}\index{\texttt{\$j} comment}\index{additional information comment}
comment,
provides a way to add additional structured information that can
be optionally parsed by systems.

The additional information comment is parsed the same way as the
typesetting comment (\texttt{\$t}) (see section \ref{tcomment}).
That is,
the additional information comment begins with the token
\texttt{\$j} within a comment,
and continues until the comment close \texttt{\$)}.
Within an additional information comment is a sequence of one or more
commands of the form \texttt{command arg arg ... ;}
where each of the zero or more \texttt{arg} values
can be either a quoted string or a keyword.
Note that every command ends in an unquoted semicolon.
If a verifier is parsing an additional information comment, but
doesn't recognize a particular command, it must skip the command
by finding the end of the command (an unquoted semicolon).

A database may have 0 or more additional information comments.
Note, however, that a verifier may ignore these comments entirely or only
process certain commands in an additional information comment.
The \texttt{mmj2} verifier supports many commands in additional information
comments.
We encourage systems that process additional information comments
to coordinate so that they will use the same command for the same effect.

Examples of additional information comments with various commands
(from the \texttt{set.mm} database) are:

\begin{itemize}
   \item Define the syntax and logical typecodes,
     and declare that our grammar is
     unambiguous (verifiable using the KLR parser, with compositing depth 5).
\begin{verbatim}
  $( $j
    syntax 'wff';
    syntax '|-' as 'wff';
    unambiguous 'klr 5';
  $)
\end{verbatim}

   \item Register $\lnot$ and $\rightarrow$ as primitive expressions
           (lacking definitions).
\begin{verbatim}
  $( $j primitive 'wn' 'wi'; $)
\end{verbatim}

   \item There is a special justification for \texttt{df-bi}.
\begin{verbatim}
  $( $j justification 'bijust' for 'df-bi'; $)
\end{verbatim}

   \item Register $\leftrightarrow$ as an equality for its type (wff).
\begin{verbatim}
  $( $j
    equality 'wb' from 'biid' 'bicomi' 'bitri';
    definition 'dfbi1' for 'wb';
  $)
\end{verbatim}

   \item Theorem \texttt{notbii} is the congruence law for negation.
\begin{verbatim}
  $( $j congruence 'notbii'; $)
\end{verbatim}

   \item Add \texttt{setvar} as a typecode.
\begin{verbatim}
  $( $j syntax 'setvar'; $)
\end{verbatim}

   \item Register $=$ as an equality for its type (\texttt{class}).
\begin{verbatim}
  $( $j equality 'wceq' from 'eqid' 'eqcomi' 'eqtri'; $)
\end{verbatim}

\end{itemize}


\subsection{Including Other Files in a Metamath Source File} \label{include}
\index{\texttt{\$[} and \texttt{\$]} auxiliary keywords}

The keywords \texttt{\$[} and \texttt{\$]} specify a file to be
included\index{included file}\index{file inclusion} at that point in a
Metamath\index{Metamath} source file\index{source file}.  The syntax for
including a file is as follows:
\begin{center}
\texttt{\$[} {\em file-name} \texttt{\$]}
\end{center}

The {\em file-name} should be a single token\index{token} with the same syntax
as a math symbol (i.e., all 93 non-whitespace
printable characters other than \texttt{\$} are
allowed, subject to the file-naming limitations of your operating system).
Comments may appear between the \texttt{\$[} and \texttt{\$]} keywords.  Included
files may include other files, which may in turn include other files, and so
on.

For example, suppose you want to use the set theory database as the starting
point for your own theory.  The first line in your file could be
\begin{center}
\texttt{\$[ set.mm \$]}
\end{center} All of the information (axioms, theorems,
etc.) in \texttt{set.mm} and any files that {\em it} includes will become
available for you to reference in your file. This can help make your work more
modular. A drawback to including files is that if you change the name of a
symbol or the label of a statement, you must also remember to update any
references in any file that includes it.


The naming conventions for included files are the same as those of your
operating system.\footnote{On the Macintosh, prior to Mac OS X,
 a colon is used to separate disk
and folder names from your file name.  For example, {\em volume}\texttt{:}{\em
file-name} refers to the root directory, {\em volume}\texttt{:}{\em
folder-name}\texttt{:}{\em file-name} refers to a folder in root, and {\em
volume}\texttt{:}{\em folder-name}\texttt{:}\ldots\texttt{:}{\em file-name} refers to a
deeper folder.  A simple {\em file-name} refers to a file in the folder from
which you launch the Metamath application.  Under Mac OS X and later,
the Metamath program is run under the Terminal application, which
conforms to Unix naming conventions.}\index{Macintosh file
names}\index{file names!Macintosh}\label{includef} For compatibility among
operating systems, you should keep the file names as simple as possible.  A
good convention to use is {\em file}\texttt{.mm} where {\em file} is eight
characters or less, in lower case.

There is no limit to the nesting depth of included files.  One thing that you
should be aware of is that if two included files themselves include a common
third file, only the {\em first} reference to this common file will be read
in.  This allows you to include two or more files that build on a common
starting file without having to worry about label and symbol conflicts that
would occur if the common file were read in more than once.  (In fact, if a
file includes itself, the self-reference will be ignored, although of course
it would not make any sense to do that.)  This feature also means, however,
that if you try to include a common file in several inner blocks, the result
might not be what you expect, since only the first reference will be replaced
with the included file (unlike the include statement in most other computer
languages).  Thus you would normally include common files only in the
outermost block\index{outermost block}.

\subsection{Compressed Proof Format}\label{compressed1}\index{compressed
proof}\index{proof!compressed}

The proof notation presented in Section~\ref{proof} is called a
{\bf normal proof}\index{normal proof}\index{proof!normal} and in principle is
sufficient to express any proof.  However, proofs often contain steps and
subproofs that are identical.  This is particularly true in typical
Metamath\index{Metamath} applications, because Metamath requires that the math
symbol sequence (usually containing a formula) at each step be separately
constructed, that is, built up piece by piece. As a result, a lot of
repetition often results.  The {\bf compressed proof} format allows Metamath
to take advantage of this redundancy to shorten proofs.

The specification for the compressed proof format is given in
Appen\-dix~\ref{compressed}.

Normally you need not concern yourself with the details of the compressed
proof format, since the Metamath program will allow you to convert from
the normal format to the compressed format with ease, and will also
automatically convert from the compressed format when proofs are displayed.
The overall structure of the compressed format is as follows:
\begin{center}
  \texttt{\$= ( } {\em label-list} \texttt{) } {\em compressed-proof\ }\ \texttt{\$.}
\end{center}
\index{\texttt{\$=} keyword}
The first \texttt{(} serves as a flag to Metamath that a compressed proof
follows.  The {\em label-list} includes all statements referred to by the
proof except the mandatory hypotheses\index{mandatory hypothesis}.  The {\em
compressed-proof} is a compact encoding of the proof, using upper-case
letters, and can be thought of as a large integer in base 26.  White
space\index{white space} inside a {\em compressed-proof} is
optional and is ignored.

It is important to note that the order of the mandatory hypotheses of
the statement being proved must not be changed if the compressed proof
format is used, otherwise the proof will become incorrect.  The reason
for this is that the mandatory hypotheses are not mentioned explicitly
in the compressed proof in order to make the compression more efficient.
If you wish to change the order of mandatory hypotheses, you must first
convert the proof back to normal format using the \texttt{save proof
{\em statement} /normal}\index{\texttt{save proof} command} command.
Later, you can go back to compressed format with \texttt{save proof {\em
statement} /compressed}.

During error checking with the \texttt{verify proof} command, an error
found in a compressed proof may point to a character in {\em
compressed-proof}, which may not be very meaningful to you.  In this
case, try to \texttt{save proof /normal} first, then do the
\texttt{verify proof} again.  In general, it is best to make sure a
proof is correct before saving it in compressed format, because severe
errors are less likely to be recoverable than in normal format.

\subsection{Specifying Unknown Proofs or Subproofs}\label{unknown}

In a proof under development, any step or subproof that is not yet known
may be represented with a single \texttt{?}.  For the purposes of
parsing the proof, the \texttt{?}\ \index{\texttt{]}@\texttt{?}\ inside
proofs} will push a single entry onto the RPN stack just as if it were a
hypothesis.  While developing a proof with the Proof
Assistant\index{Proof Assistant}, a partially developed proof may be
saved with the \texttt{save new{\char`\_}proof}\index{\texttt{save
new{\char`\_}proof} command} command, and \texttt{?}'s will be placed at
the appropriate places.

All \texttt{\$p}\index{\texttt{\$p} statement} statements must have
proofs, even if they are entirely unknown.  Before creating a proof with
the Proof Assistant, you should specify a completely unknown proof as
follows:
\begin{center}
  {\em label} \texttt{\$p} {\em statement} \texttt{\$= ?\ \$.}
\end{center}
\index{\texttt{\$=} keyword}
\index{\texttt{]}@\texttt{?}\ inside proofs}

The \texttt{verify proof}\index{\texttt{verify proof} command} command
will check the known portions of a partial proof for errors, but will
warn you that the statement has not been proved.

Note that partially developed proofs may be saved in compressed format
if desired.  In this case, you will see one or more \texttt{?}'s in the
{\em compressed-proof} part.\index{compressed
proof}\index{proof!compressed}

\section{Axioms vs.\ Definitions}\label{definitions}

The \textit{basic}
Metamath\index{Metamath} language and program
make no distinction\index{axiom vs.\
definition} between axioms\index{axiom} and
definitions.\index{definition} The \texttt{\$a}\index{\texttt{\$a}
statement} statement is used for both.  At first, this may seem
puzzling.  In the minds of many mathematicians, the distinction is
clear, even obvious, and hardly worth discussing.  A definition is
considered to be merely an abbreviation that can be replaced by the
expression for which it stands; although unless one actually does this,
to be precise then one should say that a theorem\index{theorem} is a
consequence of the axioms {\em and} the definitions that are used in the
formulation of the theorem \cite[p.~20]{Behnke}.\index{Behnke, H.}

\subsection{What is a Definition?}

What is a definition?  In its simplest form, a definition introduces a new
symbol and provides an unambiguous rule to transform an expression containing
the new symbol to one without it.  The concept of a ``proper
definition''\index{proper definition}\index{definition!proper} (as opposed to
a creative definition)\index{creative definition}\index{definition!creative}
that is usually agreed upon is (1) the definition should not strengthen the
language and (2) any symbols introduced by the definition should be eliminable
from the language \cite{Nemesszeghy}\index{Nemesszeghy, E. Z.}.  In other
words, they are mere typographical conveniences that do not belong to the
system and are theoretically superfluous.  This may seem obvious, but in fact
the nature of definitions can be subtle, sometimes requiring difficult
metatheorems to establish that they are not creative.

A more conservative stance was taken by logician S.
Le\'{s}niewski.\index{Le\'{s}niewski, S.}
\begin{quote}
Le\'{s}niewski
regards definitions as theses of the system.  In this respect they do
not differ either from the axioms or from theorems, i.e.\ from the
theses added to the system on the basis of the rule of substitution or
the rule of detachment [modus ponens].  Once definitions have been
accepted as theses of the system, it becomes necessary to consider them
as true propositions in the same sense in which axioms are true
\cite{Lejewski}.
\end{quote}\index{Lejewski, Czeslaw}

Let us look at some simple examples of definitions in propositional
calculus.  Consider the definition of logical {\sc or}
(disjunction):\index{disjunction ($\vee$)} ``$P\vee Q$ denotes $\neg P
\rightarrow Q$ (not $P$ implies $Q$).''  It is very easy to recognize a
statement making use of this definition, because it introduces the new
symbol $\vee$ that did not previously exist in the language.  It is easy
to see that no new theorems of the original language will result from
this definition.

Next, consider a definition that eliminates parentheses:  ``$P
\rightarrow Q\rightarrow R$ denotes $P\rightarrow (Q \rightarrow R)$.''
This is more subtle, because no new symbols are introduced.  The reason
this definition is considered proper is that no new symbol sequences
that are valid wffs (well-formed formulas)\index{well-formed formula
(wff)} in the original language will result from the definition, since
``$P \rightarrow Q\rightarrow R$'' is not a wff in the original
language.  Here, we implicitly make use of the fact that there is a
decision procedure that allows us to determine whether or not a symbol
sequence is a wff, and this fact allows us to use symbol sequences that
are not wffs to represent other things (such as wffs) by means of the
definition.  However, to justify the definition as not being creative we
need to prove that ``$P \rightarrow Q\rightarrow R$'' is in fact not a
wff in the original language, and this is more difficult than in the
case where we simply introduce a new symbol.

%Now let's take this reasoning to an extreme.  Propositional calculus is a
%decidable theory,\footnote{This means that a mechanical algorithm exists to
%determine whether or not a wff is a theorem.} so in principle we could make use
%of symbol sequences that are not theorems to represent other things (say, to
%encode actual theorems in a more compact way).  For example, let us extend the
%language by defining a wff ``$P$'' in the extended language as the theorem
%``$P\rightarrow P$''\footnote{This is one of the first theorems proved in the
%Metamath database \texttt{set.mm}.}\index{set
%theory database (\texttt{set.mm})} in the original language whenever ``$P$'' is
%not a theorem in the original language.  In the extended language, any wff
%``$Q$'' thus represents a theorem; to find out what theorem (in the original
%language) ``$Q$'' represents, we determine whether ``$Q$'' is a theorem in the
%original language (before the definition was introduced).  If so, we're done; if
%not, we replace ``$Q$'' by ``$Q\rightarrow Q$'' to eliminate the definition.
%This definition is therefore eliminable, and it does not ``strengthen'' the
%language because any wff that is not a theorem is not in the set of statements
%provable in the original language and thus is available for use by definitions.
%
%Of course, a definition such as this would render practically useless the
%communication of theorems of propositional calculus; but
%this is just a human shortcoming, since we can't always easily discern what is
%and is not a theorem by inspection.  In fact, the extended theory with this
%definition has no more and no less information than the original theory; it just
%expresses certain theorems of the form ``$P\rightarrow P$''
%in a more compact way.
%
%The point here is that what constitutes a proper definition is a matter of
%judgment about whether a symbol sequence can easily be recognized by a human
%as invalid in some sense (for example, not a wff); if so, the symbol sequence
%can be appropriated for use by a definition in order to make the extended
%language more compact.  Metamath\index{Metamath} lacks the ability to make this
%judgment, since as far as Metamath is concerned the definition of a wff, for
%example, is arbitrary.  You define for Metamath how wffs\index{well-formed
%formula (wff)} are constructed according to your own preferred style.  The
%concept of a wff may not even exist in a given formal system\index{formal
%system}.  Metamath treats all definitions as if they were new axioms, and it
%is up to the human mathematician to judge whether the definition is ``proper''
%'\index{proper definition}\index{definition!proper} in some agreed-upon way.

What constitutes a definition\index{definition} versus\index{axiom vs.\
definition} an axiom\index{axiom} is sometimes arbitrary in mathematical
literature.  For example, the connectives $\vee$ ({\sc or}), $\wedge$
({\sc and}), and $\leftrightarrow$ (equivalent to) in propositional
calculus are usually considered defined symbols that can be used as
abbreviations for expressions containing the ``primitive'' connectives
$\rightarrow$ and $\neg$.  This is the way we treat them in the standard
logic and set theory database \texttt{set.mm}\index{set theory database
(\texttt{set.mm})}.  However, the first three connectives can also be
considered ``primitive,'' and axiom systems have been devised that treat
all of them as such.  For example,
\cite[p.~35]{Goodstein}\index{Goodstein, R. L.} presents one with 15
axioms, some of which in fact coincide with what we have chosen to call
definitions in \texttt{set.mm}.  In certain subsets of classical
propositional calculus, such as the intuitionist
fragment\index{intuitionism}, it can be shown that one cannot make do
with just $\rightarrow$ and $\neg$ but must treat additional connectives
as primitive in order for the system to make sense.\footnote{Two nice
systems that make the transition from intuitionistic and other weak
fragments to classical logic just by adding axioms are given in
\cite{Robinsont}\index{Robinson, T. Thacher}.}

\subsection{The Approach to Definitions in \texttt{set.mm}}

In set theory, recursive definitions define a newly introduced symbol in
terms of itself.  The justification of recursive definitions, using
several ``recursion theorems,'' is usually one of the first
sophisticated proofs a student encounters when learning set theory, and
there is a significant amount of implicit metalogic behind a recursive
definition even though the definition itself is typically simple to
state.  It is, however, possible to substitute one kind of complexity
for another.  We can eliminate the need for metalogical justification by
defining the operation directly with an explicit (but complicated)
expression, then deriving the recursive definition directly as a
theorem, using a recursion theorem ``in reverse.''  We do this in
\texttt{set.mm}, as follows.

In \texttt{set.mm} our goal was to introduce almost all definitions in
the form of two expressions connected by either $\leftrightarrow$ or
$=$, where the thing being defined does not appear on the right hand
side.  Quine calls this form ``a genuine or direct definition'' \cite[p.
174]{Quine}\index{Quine, Willard Van Orman}, which makes the definitions
very easy to eliminate and the metalogic\index{metalogic} needed to
justify them as simple as possible.

We achieved this goal in almost all cases in \texttt{set.mm}.
Sometimes this makes the definitions more complex and less
intuitive.  For example, the traditional way to define addition of
natural numbers is to define an operation called {\em
successor}\index{successor} (which means ``plus one'' and is denoted by
``${\rm suc}$''), then define addition recursively\index{recursive
definition} with the two definitions $n + 0 = n$ and $m + {\rm suc}\,n =
{\rm suc} (m + n)$.  Although this definition seems simple and obvious,
the method to eliminate the definition is not obvious:  in the second
part of the definition, addition is defined in terms of itself.  By
eliminating the definition, we don't mean repeatedly applying it to
specific $m$ and $n$ but rather showing the explicit, closed-form
set-theoretical expression that $m + n$ represents, that will work for
any $m$ and $n$ and that does not have a $+$ sign on its right-hand
side.  For a recursive definition like this not to be circular
(creative), there are some hidden, underlying assumptions we must make,
for example that the natural numbers have a certain kind of order.  In
\texttt{set.mm} we chose to start with the direct (though complex and
nonintuitive) definition then derive from it the standard recursive
definition.\footnote{The closed-form definition used in \texttt{set.mm}
for the addition operation on ordinals\index{ordinal
addition}\index{addition!of ordinals} (of which natural numbers are a
subset) is

\setbox\startprefix=\hbox{\tt \ \ df-oadd\ \$a\ }
\setbox\contprefix=\hbox{\tt \ \ \ \ \ \ \ \ \ \ \ \ \ }
\startm
\m{\vdash}\m{+_o}\m{=}\m{\{}\m{\langle}\m{\langle}\m{x}\m{,}\m{y}\m{\rangle}%
\m{,}\m{z}\m{\rangle}\m{|}\m{(}\m{(}\m{x}\m{\in}\m{{\rm On}}\m{\wedge}\m{y}\m{%
\in}\m{{\rm On}}\m{)}\m{\wedge}\m{z}\m{=}\m{(}\m{{\rm rec}}\m{(}\m{\{}\m{%
\langle}\m{w}\m{,}\m{v}\m{\rangle}\m{|}\m{v}\m{=}\m{{\rm suc}}\m{w}\m{\}}\m{,}%
\m{x}\m{)}\m{`}\m{y}\m{)}\m{)}\m{\}}
\endm

\noindent Here, the abstraction class of nested ordered pairs\index{abstraction
class!of nested ordered pairs} is defined by \texttt{df-oprab} in \texttt{set.mm},
and ${\rm rec}$ is a ``recursion operator''\index{recursion operator} with
the definition

\setbox\startprefix=\hbox{\tt \ \ df-rfg\ \$a\ }
\setbox\contprefix=\hbox{\tt \ \ \ \ \ \ \ \ \ \ \ \ }
\startm
\m{\vdash}\m{{\rm rec}}\m{(}\m{F}\m{,}\m{A}\m{)}\m{=}\m{\bigcup}\m{\{}\m{f}%
\m{|}\m{\exists}\m{x}\m{(}\m{x}\m{\in}\m{{\rm On}}\m{\wedge}\m{(}\m{f}\m{{%
\rm Fn}}\m{x}\m{\wedge}\m{\forall}\m{y}\m{(}\m{y}\m{\in}\m{x}\m{\rightarrow}%
\m{(}\m{f}\m{`}\m{y}\m{)}\m{=}\m{(}\m{\{}\m{\langle}\m{g}\m{,}\m{z}\m{\rangle}%
\m{|}\m{(}\m{(}\m{g}\m{=}\m{\varnothing}\m{\wedge}\m{z}\m{=}\m{A}\m{)}\m{\vee}%
\m{(}\m{\lnot}\m{(}\m{g}\m{=}\m{\varnothing}\m{\vee}\m{{\rm Lim}}\m{{\rm dom}}%
\m{g}\m{)}\m{\wedge}\m{z}\m{=}\m{(}\m{F}\m{`}\m{(}\m{g}\m{`}\m{\bigcup}\m{{%
\rm dom}}\m{g}\m{)}\m{)}\m{)}\m{\vee}\m{(}\m{{\rm Lim}}\m{{\rm dom}}\m{g}\m{%
\wedge}\m{z}\m{=}\m{\bigcup}\m{{\rm ran}}\m{g}\m{)}\m{)}\m{\}}\m{`}\m{(}\m{f}%
\m{\restriction}\m{y}\m{)}\m{)}\m{)}\m{)}\m{)}\m{\}}
\endm

\noindent which can be further broken down with definitions shown in
Section~\ref{setdefinitions}.  You may be surprised at the complexity of what
seems like such a simple notion.  From these definitions the simpler, more
intuitive recursive definition is derived as a set of theorems.}\index{natural
number}\index{addition}\index{recursive definition}\index{ordinal addition}
The end result is the same, but we completely eliminate the rather complex
metalogic that justifies the recursive definition.  (For a mathematician,
recursive definitions are more efficient and intuitive than
direct ones once the metalogic has been learned or possibly just accepted as
correct.  However, it was felt that direct definition in \texttt{set.mm}
maximizes rigor by minimizing metalogic.  It can be eliminated effortlessly,
something difficult to do with a recursive definition.)

\subsection{Adding Constraints on Definitions}

The basic Metamath language and the Metamath program do
not have any built-in constraints on definitions, since they are just
\$a statements.

However, nothing prevents a verification system from verifying
additional rules to impose further limitations on definitions.
For example, the \texttt{mmj2}\index{mmj2} program
supports various kinds of
additional information comments (see section \ref{jcomment}).
One of their uses is to optionally verify additional constraints,
including constraints to verify that definitions meet certain
requirements.
These additional checks are required by the
continuous integration (CI)\index{continuous integration (CI)}
checks of the
\texttt{set.mm}\index{set theory database (\texttt{set.mm})}%
\index{Metamath Proof Explorer}
database.
This approach enables us to optionally impose additional requirements
on definitions if we wish, without necessarily imposing those rules on
all databases or requiring all verification systems to implement them.
In addition, this allows us to impose specialized constraints tailored
to one database while not requiring other systems to implement
those specialized constraints.

We impose two constraints on the
\texttt{set.mm}\index{set theory database (\texttt{set.mm})}%
\index{Metamath Proof Explorer} database
via the \texttt{mmj2}\index{mmj2} program that are worth discussing here:
a parse check and a definition soundness check.

% On February 11, 2019 8:32:32 PM EST, saueran@oregonstate.edu wrote:
% The following addition to the end of set.mm is accepted by the mmj2
% parser and definition checker and the metamath verifier(at least it was
% when I checked, you should check it too), and creates a contradiction by
% proving the theorem |- ph.
% ${
% wleftp $a wff ( ( ph ) $.
% wbothp $a wff ( ph ) $.
% df-leftp $a |- ( ( ( ph ) <-> -. ph ) $.
% df-bothp $a |- ( ( ph ) <-> ph ) $.
% anything $p |- ph $=
%   ( wbothp wn wi wleftp df-leftp biimpi df-bothp mpbir mpbi simplim ax-mp)
%   ABZAMACZDZCZMOEZOCQAEZNDZRNAFGSHIOFJMNKLAHJ $.
% $}
%
% This particular problem is countered by enabling, within mmj2,
% SetParser,mmj.verify.LRParser

First,
we enable a parse check in \texttt{mmj2} (through its
\texttt{SetParser} command) that requires that all new definitions
must \textit{not} create an ambiguous parse for a KLR(5) parser.
This prevents some errors such as definitions with imbalanced parentheses.

Second, we run a definition soundness check specific to
\texttt{set.mm} or databases similar to it.
(through the \texttt{definitionCheck} macro).
Some \texttt{\$a} statements (including all ax-* statemnets)
are excluded from these checks, as they will
always fail this simple check,
but they are appropriate for most definitions.
This check imposes a set of additional rules:

\begin{enumerate}

\item New definitions must be introduced using $=$ or $\leftrightarrow$.

\item No \texttt{\$a} statement introduced before this one is allowed to use the
symbol being defined in this definition, and the definition is not
allowed to use itself (except once, in the definiendum).

\item Every variable in the definiens should not be distinct

\item Every dummy variable in the definiendum
are required to be distinct from each other and from variables in
the definiendum.
To determine this, the system will look for a "justification" theorem
in the database, and if it is not there, attempt to briefly prove
$( \varphi \rightarrow \forall x \varphi )$  for each dummy variable x.

\item Every dummy variable should be a set variable,
unless there is a justification theorem available.

\item Every dummy variable must be bound
(if the system cannot determine this a justification theorem must be
provided).

\end{enumerate}

\subsection{Summary of Approach to Definitions}

In short, when being rigorous it turns out that
definitions can be subtle, sometimes requiring difficult
metatheorems to establish that they are not creative.

Instead of building such complications into the Metamath language itself,
the basic Metmath language and program simply treat traditional
axioms and definitions as the same kind of \texttt{\$a} statement.
We have then built various tools to enable people to
verify additional conditions as their creators believe is appropriate
for those specific databases, without complicating the Metamath foundations.

\chapter{The Metamath Program}\label{commands}

This chapter provides a reference manual for the
Metamath program.\index{Metamath!commands}

Current instructions for obtaining and installing the Metamath program
can be found at the \url{http://metamath.org} web site.
For Windows, there is a pre-compiled version called
\texttt{metamath.exe}.  For Unix, Linux, and Mac OS X
(which we will refer to collectively as ``Unix''), the Metamath program
can be compiled from its source code with the command
\begin{verbatim}
gcc *.c -o metamath
\end{verbatim}
using the \texttt{gcc} {\sc c} compiler available on those systems.

In the command syntax descriptions below, fields enclosed in square
brackets [\ ] are optional.  File names may be optionally enclosed in
single or double quotes.  This is useful if the file name contains
spaces or
slashes (\texttt{/}), such as in Unix path names, \index{Unix file
names}\index{file names!Unix} that might be confused with Metamath
command qualifiers.\index{Unix file names}\index{file names!Unix}

\section{Invoking Metamath}

Unix, Linux, and Mac OS X
have a command-line interface called the {\em
bash shell}.  (In Mac OS X, select the Terminal application from
Applications/Utilities.)  To invoke Metamath from the bash shell prompt,
assuming that the Metamath program is in the current directory, type
\begin{verbatim}
bash$ ./metamath
\end{verbatim}

To invoke Metamath from a Windows DOS or Command Prompt, assuming that
the Metamath program is in the current directory (or in a directory
included in the Path system environment variable), type
\begin{verbatim}
C:\metamath>metamath
\end{verbatim}

To use command-line arguments at invocation, the command-line arguments
should be a list of Metamath commands, surrounded by quotes if they
contain spaces.  In Windows, the surrounding quotes must be double (not
single) quotes.  For example, to read the database file \texttt{set.mm},
verify all proofs, and exit the program, type (under Unix)
\begin{verbatim}
bash$ ./metamath 'read set.mm' 'verify proof *' exit
\end{verbatim}
Note that in Unix, any directory path with \texttt{/}'s must be
surrounded by quotes so Metamath will not interpret the \texttt{/} as a
command qualifier.  So if \texttt{set.mm} is in the \texttt{/tmp}
directory, use for the above example
\begin{verbatim}
bash$ ./metamath 'read "/tmp/set.mm"' 'verify proof *' exit
\end{verbatim}

For convenience, if the command-line has one argument and no spaces in
the argument, the command is implicitly assumed to be \texttt{read}.  In
this one special case, \texttt{/}'s are not interpreted as command
qualifiers, so you don't need quotes around a Unix file name.  Thus
\begin{verbatim}
bash$ ./metamath /tmp/set.mm
\end{verbatim}
and
\begin{verbatim}
bash$ ./metamath "read '/tmp/set.mm'"
\end{verbatim}
are equivalent.


\section{Controlling Metamath}

The Metamath program was first developed on a {\sc vax/vms} system, and
some aspects of its command line behavior reflect this heritage.
Hopefully you will find it reasonably user-friendly once you get used to
it.

Each command line is a sequence of English-like words separated by
spaces, as in \texttt{show settings}.  Command words are not case
sensitive, and only as many letters are needed as are necessary to
eliminate ambiguity; for example, \texttt{sh se} would work for the
command \texttt{show settings}.  In some cases arguments such as file
names, statement labels, or symbol names are required; these are
case-sensitive (although file names may not be on some operating
systems).

A command line is entered by typing it in then pressing the {\em return}
({\em enter}) key.  To find out what commands are available, type
\texttt{?} at the \texttt{MM>} prompt.  To find out the choices at any
point in a command, press {\em return} and you will be prompted for
them.  The default choice (the one selected if you just press {\em
return}) is shown in brackets (\texttt{<>}).

You may also type \texttt{?} in place of a command word to force
Metamath to tell you what the choices are.  The \texttt{?} method won't
work, though, if a non-keyword argument such as a file name is expected
at that point, because the program will think that \texttt{?} is the
value of the argument.

Some commands have one or more optional qualifiers which modify the
behavior of the command.  Qualifiers are preceded by a slash
(\texttt{/}), such as in \texttt{read set.mm / verify}.  Spaces are
optional around the \texttt{/}.  If you need to use a space or
slash in a command
argument, as in a Unix file name, put single or double quotes around the
command argument.

The \texttt{open log} command will save everything you see on the
screen and is useful to help you recover should something go wrong in a
proof, or if you want to document a bug.

If a command responds with more than a screenful, you will be
prompted to \texttt{<return> to continue, Q to quit, or S to scroll to
end}.  \texttt{Q} or \texttt{q} (not case-sensitive) will complete the
command internally but will suppress further output until the next
\texttt{MM>} prompt.  \texttt{s} will suppress further pausing until the
next \texttt{MM>} prompt.  After the first screen, you are also
presented with the choice of \texttt{b} to go back a screenful.  Note
that \texttt{b} may also be entered at the \texttt{MM>} prompt
immediately after a command to scroll back through the output of that
command.

A command line enclosed in quotes is executed by your operating system.
See Section~\ref{oscmd}.

{\em Warning:} Pressing {\sc ctrl-c} will abort the Metamath program
unconditionally.  This means any unsaved work will be lost.


\subsection{\texttt{exit} Command}\index{\texttt{exit} command}

Syntax:  \texttt{exit} [\texttt{/force}]

This command exits from Metamath.  If there have been changes to the
source with the \texttt{save proof} or \texttt{save new{\char`\_}proof}
commands, you will be given an opportunity to \texttt{write source} to
permanently save the changes.

In Proof Assistant\index{Proof Assistant} mode, the \texttt{exit} command will
return to the \verb/MM>/ prompt. If there were changes to the proof, you will
be given an opportunity to \texttt{save new{\char`\_}proof}.

The \texttt{quit} command is a synonym for \texttt{exit}.

Optional qualifier:
    \texttt{/force} - Do not prompt if changes were not saved.  This qualifier is
        useful in \texttt{submit} command files (Section~\ref{sbmt})
        to ensure predictable behavior.





\subsection{\texttt{open log} Command}\index{\texttt{open log} command}
Syntax:  \texttt{open log} {\em file-name}

This command will open a log file that will store everything you see on
the screen.  It is useful to help recovery from a mistake in a long Proof
Assistant session, or to document bugs.\index{Metamath!bugs}

The log file can be closed with \texttt{close log}.  It will automatically be
closed upon exiting Metamath.



\subsection{\texttt{close log} Command}\index{\texttt{close log} command}
Syntax:  \texttt{close log}

The \texttt{close log} command closes a log file if one is open.  See
also \texttt{open log}.




\subsection{\texttt{submit} Command}\index{\texttt{submit} command}\label{sbmt}
Syntax:  \texttt{submit} {\em filename}

This command causes further command lines to be taken from the specified
file.  Note that any line beginning with an exclamation point (\texttt{!}) is
treated as a comment (i.e.\ ignored).  Also note that the scrolling
of the screen output is continuous, so you may want to open a log file
(see \texttt{open log}) to record the results that fly by on the screen.
After the lines in the file are exhausted, Metamath returns to its
normal user interface mode.

Currently, the \texttt{submit} command is not recursive.  In other words,
\texttt{submit} commands are not allowed inside a command file.


\subsection{\texttt{erase} Command}\index{\texttt{erase} command}
Syntax:  \texttt{erase}

This command will reset Metamath to its starting state, deleting any
data\-base that was \texttt{read} in.
 If there have been changes to the
source with the \texttt{save proof} or \texttt{save new{\char`\_}proof}
commands, you will be given an opportunity to \texttt{write source} to
permanently save the changes.



\subsection{\texttt{set echo} Command}\index{\texttt{set echo} command}
Syntax:  \texttt{set echo on} or \texttt{set echo off}

The \texttt{set echo on} command will cause command lines to be echoed with any
abbreviations expanded.  While learning the Metamath commands, this
feature will show you the exact command that your abbreviated input
corresponds to.



\subsection{\texttt{set scroll} Command}\index{\texttt{set scroll} command}
Syntax:  \texttt{set scroll prompted} or \texttt{set scroll continuous}

The Metamath command line interface starts off in the \texttt{prompted} mode,
which means that you will be prompted to continue or quit after each
full screen in a long listing.  In \texttt{continuous} mode, long listings will be
scrolled without pausing.

% LaTeX bug? (1) \texttt{\_} puts out different character than
% \texttt{{\char`\_}}
%  = \verb$_$  (2) \texttt{{\char`\_}} puts out garbage in \subsection
%  argument
\subsection{\texttt{set width} Command}\index{\texttt{set
width} command}
Syntax:  \texttt{set width} {\em number}

Metamath assumes the width of your screen is 79 characters (chosen
because the Command Prompt in Windows XP has a wrapping bug at column
80).  If your screen is wider or narrower, this command allows you to
change this default screen width.  A larger width is advantageous for
logging proofs to an output file to be printed on a wide printer.  A
smaller width may be necessary on some terminals; in this case, the
wrapping of the information messages may sometimes seem somewhat
unnatural, however.  In \LaTeX\index{latex@{\LaTeX}!characters per line},
there is normally a maximum of 61 characters per line with typewriter
font.  (The examples in this book were produced with 61 characters per
line.)

\subsection{\texttt{set height} Command}\index{\texttt{set
height} command}
Syntax:  \texttt{set height} {\em number}

Metamath assumes your screen height is 24 lines of characters.  If your
screen is taller or shorter, this command lets you to change the number
of lines at which the display pauses and prompts you to continue.

\subsection{\texttt{beep} Command}\index{\texttt{beep} command}

Syntax:  \texttt{beep}

This command will produce a beep.  By typing it ahead after a
long-running command has started, it will alert you that the command is
finished.  For convenience, \texttt{b} is an abbreviation for
\texttt{beep}.

Note:  If \texttt{b} is typed at the \texttt{MM>} prompt immediately
after the end of a multiple-page display paged with ``\texttt{Press
<return> for more}...'' prompts, then the \texttt{b} will back up to the
previous page rather than perform the \texttt{beep} command.
In that case you must type the unabbreviated \texttt{beep} form
of the command.

\subsection{\texttt{more} Command}\index{\texttt{more} command}

Syntax:  \texttt{more} {\em filename}

This command will display the contents of an {\sc ascii} file on your
screen.  (This command is provided for convenience but is not very
powerful.  See Section~\ref{oscmd} to invoke your operating system's
command to do this, such as the \texttt{more} command in Unix.)

\subsection{Operating System Commands}\index{operating system
command}\label{oscmd}

A line enclosed in single or double quotes will be executed by your
computer's operating system if it has a command line interface.  For
example, on a {\sc vax/vms} system,
\verb/MM> 'dir'/
will print disk directory contents.  Note that this feature will not
work on the Macintosh prior to Mac OS X, which does not have a command
line interface.

For your convenience, the trailing quote is optional.

\subsection{Size Limitations in Metamath}

In general, there are no fixed, predefined limits\index{Metamath!memory
limits} on how many labels, tokens\index{token}, statements, etc.\ that
you may have in a database file.  The Metamath program uses 32-bit
variables (64-bit on 64-bit CPUs) as indices for almost all internal
arrays, which are allocated dynamically as needed.



\section{Reading and Writing Files}

The following commands create new files:  the \texttt{open} commands;
the \texttt{write} commands; the \texttt{/html},
\texttt{/alt{\char`\_}html}, \texttt{/brief{\char`\_}html},
\texttt{/brief{\char`\_}alt{\char`\_}html} qualifiers of \texttt{show
statement}, and \texttt{midi}.  The following commands append to files
previously opened:  the \texttt{/tex} qualifier of \texttt{show proof}
and \texttt{show new{\char`\_}proof}; the \texttt{/tex} and
\texttt{/simple{\char`\_}tex} qualifiers of \texttt{show statement}; the
\texttt{close} commands; and all screen dialog between \texttt{open log}
and \texttt{close log}.

The commands that create new files will not overwrite an existing {\em
filename} but will rename the existing one to {\em
filename}\texttt{{\char`\~}1}.  An existing {\em
filename}\texttt{{\char`\~}1} is renamed {\em
filename}\texttt{{\char`\~}2}, etc.\ up to {\em
filename}\texttt{{\char`\~}9}.  An existing {\em
filename}\texttt{{\char`\~}9} is deleted.  This makes recovery from
mistakes easier but also will clutter up your directory, so occasionally
you may want to clean up (delete) these \texttt{{\char`\~}}$n$ files.


\subsection{\texttt{read} Command}\index{\texttt{read} command}
Syntax:  \texttt{read} {\em file-name} [\texttt{/verify}]

This command will read in a Metamath language source file and any included
files.  Normally it will be the first thing you do when entering Metamath.
Statement syntax is checked, but proof syntax is not checked.
Note that the file name may be enclosed in single or double quotes;
this is useful if the file name contains slashes, as might be the case
under Unix.

If you are getting an ``\texttt{?Expected VERIFY}'' error
when trying to read a Unix file name with slashes, you probably haven't
quoted it.\index{Unix file names}\index{file names!Unix}

If you are prompted for the file name (by pressing {\em return}
 after \texttt{read})
you should {\em not} put quotes around it, even if it is a Unix file name
with slashes.

Optional command qualifier:

    \texttt{/verify} - Verify all proofs as the database is read in.  This
         qualifier will slow down reading in the file.  See \texttt{verify
         proof} for more information on file error-checking.

See also \texttt{erase}.



\subsection{\texttt{write source} Command}\index{\texttt{write source} command}
Syntax:  \texttt{write source} {\em filename}
[\texttt{/rewrap}]
[\texttt{/split}]
[\texttt{/keep\_includes}]
[\texttt{/no\_versioning}]

This command will write the contents of a Metamath\index{database}
database into a file.\index{source file}

Optional command qualifiers:

\texttt{/rewrap} -
Reformats statements and comments according to the
convention used in the set.mm database.
It unwraps the
lines in the comment before each \texttt{\$a} and \texttt{\$p} statement,
then it
rewraps the line.  You should compare the output to the original
to make sure that the desired effect results; if not, go back to
the original source.  The wrapped line length honors the
\texttt{set width}
parameter currently in effect.  Note:  Text
enclosed in \texttt{<HTML>}...\texttt{</HTML>} tags is not modified by the
\texttt{/rewrap} qualifier.
Proofs themselves are not reformatted;
use \texttt{save proof * / compressed} to do that.
An isolated tilde (\~{}) is always kept on the same line as the following
symbol, so you can find all comment references to a symbol by
searching for \~{} followed by a space and that symbol
(this is useful for finding cross-references).
Incidentally, \texttt{save proof} also honors the \texttt{set width}
parameter currently in effect.

\texttt{/split} - Files included in the source using the expression
\$[ \textit{inclfile} \$] will be
written into separate files instead of being included in a single output
file.  The name of each separately written included file will be the
\textit{inclfile} argument of its inclusion command.

\texttt{/keep\_includes} - If a source file has includes but is written as a
single file by omitting \texttt{/split}, by default the included files will
be deleted (actually just renamed with a \char`\~1 suffix unless
\texttt{/no\_versioning} is specified) to prevent the possibly confusing
source duplication in both the output file and the included file.
The \texttt{/keep\_includes} qualifier will prevent this deletion.

\texttt{/no\_versioning} - Backup files suffixed with \char`\~1 are not created.


\section{Showing Status and Statements}



\subsection{\texttt{show settings} Command}\index{\texttt{show settings} command}
Syntax:  \texttt{show settings}

This command shows the state of various parameters.

\subsection{\texttt{show memory} Command}\index{\texttt{show memory} command}
Syntax:  \texttt{show memory}

This command shows the available memory left.  It is not meaningful
on most modern operating systems,
which have virtual memory.\index{Metamath!memory usage}


\subsection{\texttt{show labels} Command}\index{\texttt{show labels} command}
Syntax:  \texttt{show labels} {\em label-match} [\texttt{/all}]
   [\texttt{/linear}]

This command shows the labels of \texttt{\$a} and \texttt{\$p}
statements that match {\em label-match}.  A \verb$*$ in {label-match}
matches zero or more characters.  For example, \verb$*abc*def$ will match all
labels containing \verb$abc$ and ending with \verb$def$.

Optional command qualifiers:

   \texttt{/all} - Include matches for \texttt{\$e} and \texttt{\$f}
   statement labels.

   \texttt{/linear} - Display only one label per line.  This can be useful for
       building scripts in conjunction with the utilities under the
       \texttt{tools}\index{\texttt{tools} command} command.



\subsection{\texttt{show statement} Command}\index{\texttt{show statement} command}
Syntax:  \texttt{show statement} {\em label-match} [{\em qualifiers} (see below)]

This command provides information about a statement.  Only statements
that have labels (\texttt{\$f}\index{\texttt{\$f} statement},
\texttt{\$e}\index{\texttt{\$e} statement},
\texttt{\$a}\index{\texttt{\$a} statement}, and
\texttt{\$p}\index{\texttt{\$p} statement}) may be specified.
If {\em label-match}
contains wildcard (\verb$*$) characters, all matching statements will be
displayed in the order they occur in the database.

Optional qualifiers (only one qualifier at a time is allowed):

    \texttt{/comment} - This qualifier includes the comment that immediately
       precedes the statement.

    \texttt{/full} - Show complete information about each statement,
       and show all
       statements matching {\em label} (including \texttt{\$e}
       and \texttt{\$f} statements).

    \texttt{/tex} - This qualifier will write the statement information to the
       \LaTeX\ file previously opened with \texttt{open tex}.  See
       Section~\ref{texout}.

    \texttt{/simple{\char`\_}tex} - The same as \texttt{/tex}, except that
       \LaTeX\ macros are not used for formatting equations, allowing easier
       manual edits of the output for slide presentations, etc.

    \texttt{/html}\index{html generation@{\sc html} generation},
       \texttt{/alt{\char`\_}html}, \texttt{/brief{\char`\_}html},
       \texttt{/brief{\char`\_}alt{\char`\_}html} -
       These qualifiers invoke a special mode of
       \texttt{show statement} that
       creates a web page for the statement.  They may not be used with
       any other qualifier.  See Section~\ref{htmlout} or
       \texttt{help html} in the program.


\subsection{\texttt{search} Command}\index{\texttt{search} command}
Syntax:  search {\em label-match}
\texttt{"}{\em symbol-match}{\tt}" [\texttt{/all}] [\texttt{/comments}]
[\texttt{/join}]

This command searches all \texttt{\$a} and \texttt{\$p} statements
matching {\em label-match} for occurrences of {\em symbol-match}.  A
\verb@*@ in {\em label-match} matches any label character.  A \verb@$*@
in {\em symbol-match} matches any sequence of symbols.  The symbols in
{\em symbol-match} must be separated by white space.  The quotes
surrounding {\em symbol-match} may be single or double quotes.  For
example, \texttt{search b}\verb@* "-> $* ch"@ will list all statements
whose labels begin with \texttt{b} and contain the symbols \verb@->@ and
\texttt{ch} surrounding any symbol sequence (including no symbol
sequence).  The wildcards \texttt{?} and \texttt{\$?} are also available
to match individual characters in labels and symbols respectively; see
\texttt{help search} in the Metamath program for details on their usage.

Optional command qualifiers:

    \texttt{/all} - Also search \texttt{\$e} and \texttt{\$f} statements.

    \texttt{/comments} - Search the comment that immediately precedes each
        label-matched statement for {\em symbol-match}.  In this case
        {\em symbol-match} is an arbitrary, non-case-sensitive character
        string.  Quotes around {\em symbol-match} are optional if there
        is no ambiguity.

    \texttt{/join} - In the case of a \texttt{\$a} or \texttt{\$p} statement,
	prepend its \texttt{\$e}
	hypotheses for searching. The
	\texttt{/join} qualifier has no effect in \texttt{/comments} mode.

\section{Displaying and Verifying Proofs}


\subsection{\texttt{show proof} Command}\index{\texttt{show proof} command}
Syntax:  \texttt{show proof} {\em label-match} [{\em qualifiers} (see below)]

This command displays the proof of the specified
\texttt{\$p}\index{\texttt{\$p} statement} statement in various formats.
The {\em label-match} may contain wildcard (\verb@$*@) characters to match
multiple statements.  Without any qualifiers, only the logical steps
will be shown (i.e.\ syntax construction steps will be omitted), in an
indented format.

Most of the time, you will use
    \texttt{show proof} {\em label}
to see just the proof steps corresponding to logical inferences.

Optional command qualifiers:

    \texttt{/essential} - The proof tree is trimmed of all
        \texttt{\$f}\index{\texttt{\$f} statement} hypotheses before
        being displayed.  (This is the default, and it is redundant to
        specify it.)

    \texttt{/all} - the proof tree is not trimmed of all \texttt{\$f} hypotheses before
        being displayed.  \texttt{/essential} and \texttt{/all} are mutually exclusive.

    \texttt{/from{\char`\_}step} {\em step} - The display starts at the specified
        step.  If
        this qualifier is omitted, the display starts at the first step.

    \texttt{/to{\char`\_}step} {\em step} - The display ends at the specified
        step.  If this
        qualifier is omitted, the display ends at the last step.

    \texttt{/tree{\char`\_}depth} {\em number} - Only
         steps at less than the specified proof
        tree depth are displayed.  Sometimes useful for obtaining an overview of
        the proof.

    \texttt{/reverse} - The steps are displayed in reverse order.

    \texttt{/renumber} - When used with \texttt{/essential}, the steps are renumbered
        to correspond only to the essential steps.

    \texttt{/tex} - The proof is converted to \LaTeX\ and\index{latex@{\LaTeX}}
        stored in the file opened
        with \texttt{open tex}.  See Section~\ref{texout} or
        \texttt{help tex} in the program.

    \texttt{/lemmon} - The proof is displayed in a non-indented format known
        as Lemmon style, with explicit previous step number references.
        If this qualifier is omitted, steps are indented in a tree format.

    \texttt{/start{\char`\_}column} {\em number} - Overrides the default column
        (16)
        at which the formula display starts in a Lemmon-style display.  May be
        used only in conjunction with \texttt{/lemmon}.

    \texttt{/normal} - The proof is displayed in normal format suitable for
        inclusion in a Metamath source file.  May not be used with any other
        qualifier.

    \texttt{/compressed} - The proof is displayed in compressed format
        suitable for inclusion in a Metamath source file.  May not be used with
        any other qualifier.

    \texttt{/statement{\char`\_}summary} - Summarizes all statements (like a
        brief \texttt{show statement})
        used by the proof.  It may not be used with any other qualifier
        except \texttt{/essential}.

    \texttt{/detailed{\char`\_}step} {\em step} - Shows the details of what is
        happening at
        a specific proof step.  May not be used with any other qualifier.
        The {\em step} is the step number shown when displaying a
        proof without the \texttt{/renumber} qualifier.


\subsection{\texttt{show usage} Command}\index{\texttt{show usage} command}
Syntax:  \texttt{show usage} {\em label-match} [\texttt{/recursive}]

This command lists the statements whose proofs make direct reference to
the statement specified.

Optional command qualifier:

    \texttt{/recursive} - Also include statements whose proofs ultimately
        depend on the statement specified.



\subsection{\texttt{show trace\_back} Command}\index{\texttt{show
       trace{\char`\_}back} command}
Syntax:  \texttt{show trace{\char`\_}back} {\em label-match} [\texttt{/essential}] [\texttt{/axioms}]
    [\texttt{/tree}] [\texttt{/depth} {\em number}]

This command lists all statements that the proof of the \texttt{\$p}
statement(s) specified by {\em label-match} depends on.

Optional command qualifiers:

    \texttt{/essential} - Restrict the trace-back to \texttt{\$e}
        \index{\texttt{\$e} statement} hypotheses of proof trees.

    \texttt{/axioms} - List only the axioms that the proof ultimately depends on.

    \texttt{/tree} - Display the trace-back in an indented tree format.

    \texttt{/depth} {\em number} - Restrict the \texttt{/tree} trace-back to the
        specified indentation depth.

    \texttt{/count{\char`\_}steps} - Count the number of steps the proof has
       all the way back to axioms.  If \texttt{/essential} is specified,
       expansions of variable-type hypotheses (syntax constructions) are not counted.

\subsection{\texttt{verify proof} Command}\index{\texttt{verify proof} command}
Syntax:  \texttt{verify proof} {\em label-match} [\texttt{/syntax{\char`\_}only}]

This command verifies the proofs of the specified statements.  {\em
label-match} may contain wild card characters (\texttt{*}) to verify
more than one proof; for example \verb/*abc*def/ will match all labels
containing \texttt{abc} and ending with \texttt{def}.
The command \texttt{verify proof *} will verify all proofs in the database.

Optional command qualifier:

    \texttt{/syntax{\char`\_}only} - This qualifier will perform a check of syntax
        and RPN
        stack violations only.  It will not verify that the proof is
        correct.  This qualifier is useful for quickly determining which
        proofs are incomplete (i.e.\ are under development and have \texttt{?}'s
        in them).

{\em Note:} \texttt{read}, followed by \texttt{verify proof *}, will ensure
 the database is free
from errors in the Metamath language but will not check the markup notation
in comments.
You can also check the markup notation by running \texttt{verify markup *}
as discussed in Section~\ref{verifymarkup}; see also the discussion
on generating {\sc HTML} in Section~\ref{htmlout}.

\subsection{\texttt{verify markup} Command}\index{\texttt{verify markup} command}\label{verifymarkup}
Syntax:  \texttt{verify markup} {\em label-match}
[\texttt{/date{\char`\_}skip}]
[\texttt{/top{\char`\_}date{\char`\_}skip}]
[\texttt{/file{\char`\_}skip}]
[\texttt{/verbose}]

This command checks comment markup and other informal conventions we have
adopted.  It error-checks the latexdef, htmldef, and althtmldef statements
in the \texttt{\$t} statement of a Metamath source file.\index{error checking}
It error-checks any \texttt{`}...\texttt{`},
\texttt{\char`\~}~\textit{label},
and bibliographic markups in statement descriptions.
It checks that
\texttt{\$p} and \texttt{\$a} statements
have the same content when their labels start with
``ax'' and ``ax-'' respectively but are otherwise identical, for example
ax4 and ax-4.
It also verifies the date consistency of ``(Contributed by...),''
``(Revised by...),'' and ``(Proof shortened by...)'' tags in the comment
above each \texttt{\$a} and \texttt{\$p} statement.

Optional command qualifiers:

    \texttt{/date{\char`\_}skip} - This qualifier will
        skip date consistency checking,
        which is usually not required for databases other than
	\texttt{set.mm}.

    \texttt{/top{\char`\_}date{\char`\_}skip} - This qualifier will check date consistency except
        that the version date at the top of the database file will not
        be checked.  Only one of
        \texttt{/date{\char`\_}skip} and
        \texttt{/top{\char`\_}date{\char`\_}skip} may be
        specified.

    \texttt{/file{\char`\_}skip} - This qualifier will skip checks that require
        external files to be present, such as checking GIF existence and
        bibliographic links to mmset.html or equivalent.  It is useful
        for doing a quick check from a directory without these files.

    \texttt{/verbose} - Provides more information.  Currently it provides a list
        of axXXX vs. ax-XXX matches.

\subsection{\texttt{save proof} Command}\index{\texttt{save proof} command}
Syntax:  \texttt{save proof} {\em label-match} [\texttt{/normal}]
   [\texttt{/compressed}]

The \texttt{save proof} command will reformat a proof in one of two formats and
replace the existing proof in the source buffer\index{source
buffer}.  It is useful for
converting between proof formats.  Note that a proof will not be
permanently saved until a \texttt{write source} command is issued.

Optional command qualifiers:

    \texttt{/normal} - The proof is saved in the normal format (i.e., as a
        sequence
        of labels, which is the defined format of the basic Metamath
        language).\index{basic language}  This is the default format that
        is used if a qualifier
        is omitted.

    \texttt{/compressed} - The proof is saved in the compressed format which
        reduces storage requirements for a database.
        See Appendix~\ref{compressed}.




\section{Creating Proofs}\label{pfcommands}\index{Proof Assistant}

Before using the Proof Assistant, you must add a \texttt{\$p} to your
source file (using a text editor) containing the statement you want to
prove.  Its proof should consist of a single \texttt{?}, meaning
``unknown step.''  Example:
\begin{verbatim}
equid $p x = x $= ? $.
\end{verbatim}

To enter the Proof assistant, type \texttt{prove} {\em label}, e.g.
\texttt{prove equid}.  Metamath will respond with the \texttt{MM-PA>}
prompt.

Proofs are created working backwards from the statement being proved,
primarily using a series of \texttt{assign} commands.  A proof is
complete when all steps are assigned to statements and all steps are
unified and completely known.  During the creation of a proof, Metamath
will allow only operations that are legal based on what is known up to
that point.  For example, it will not allow an \texttt{assign} of a
statement that cannot be unified with the unknown proof step being
assigned.

{\em Important:}
The Proof Assistant is
{\em not} a tool to help you discover proofs.  It is just a tool to help
you add them to the database.  For a tutorial read
Section~\ref{frstprf}.
To practice using the Proof Assistant, you may
want to \texttt{prove} an existing theorem, then delete all steps with
\texttt{delete all}, then re-create it with the Proof Assistant while
looking at its proof display (before deletion).
You might want to figure out your first few proofs completely
and write them down by hand, before using the Proof Assistant, though
not everyone finds that effective.

{\em Important:}
The \texttt{undo} command if very helpful when entering a proof, because
it allows you to undo a previously-entered step.
In addition, we suggest that you
keep track of your work with a log file (\texttt{open
log}) and save it frequently (\texttt{save new{\char`\_}proof},
\texttt{write source}).
You can use \texttt{delete} to reverse an \texttt{assign}.
You can also do \texttt{delete floating{\char`\_}hypotheses}, then
\texttt{initialize all}, then \texttt{unify all /interactive} to
reinitialize bad unifications made accidentally or by bad
\texttt{assign}s.  You cannot reverse a \texttt{delete} except by
a relevant \texttt{undo} or using
\texttt{exit /force} then reentering the Proof Assistant to recover from
the last \texttt{save new{\char`\_}proof}.

The following commands are available in the Proof Assistant (at the
\texttt{MM-PA>} prompt) to help you create your proof.  See the
individual commands for more detail.

\begin{itemize}
\item[]
    \texttt{show new{\char`\_}proof} [\texttt{/all},...] - Displays the
        proof in progress.  You will use this command a lot; see \texttt{help
        show new{\char`\_}proof} to become familiar with its qualifiers.  The
        qualifiers \texttt{/unknown} and \texttt{/not{\char`\_}unified} are
        useful for seeing the work remaining to be done.  The combination
        \texttt{/all/unknown} is useful for identifying dummy variables that must be
        assigned, or attempts to use illegal syntax, when \texttt{improve all}
        is unable to complete the syntax constructions.  Unknown variables are
        shown as \texttt{\$1}, \texttt{\$2},...
\item[]
    \texttt{assign} {\em step} {\em label} - Assigns an unknown {\em step}
        number with the statement
        specified by {\em label}.
\item[]
    \texttt{let variable} {\em variable}
        \texttt{= "}{\em symbol sequence}\texttt{"}
          - Forces a symbol
        sequence to replace an unknown variable (such as \texttt{\$1}) in a proof.
        It is useful
        for helping difficult unifications, and it is necessary when you have
        dummy variables that eventually must be assigned a name.
\item[]
    \texttt{let step} {\em step} \texttt{= "}{\em symbol sequence}\texttt{"} -
          Forces a symbol sequence
        to replace the contents of a proof step, provided it can be
        unified with the existing step contents.  (I rarely use this.)
\item[]
    \texttt{unify step} {\em step} (or \texttt{unify all}) - Unifies
        the source and target of
        a step.  If you specify a specific step, you will be prompted
        to select among the unifications that are possible.  If you
        specify \texttt{all}, all steps with unique unifications, but only
        those steps, will be
        unified.  \texttt{unify all /interactive} goes through all non-unified
        steps.
\item[]
    \texttt{initialize} {\em step} (or \texttt{all}) - De-unifies the target and source of
        a step (or all steps), as well as the hypotheses of the source,
        and makes all variables in the source unknown.  Useful to recover from
        an \texttt{assign} or \texttt{let} mistake that
        resulted in incorrect unifications.
\item[]
    \texttt{delete} {\em step} (or \texttt{all} or \texttt{floating{\char`\_}hypotheses}) -
        Deletes the specified
        step(s).  \texttt{delete floating{\char`\_}hypotheses}, then \texttt{initialize all}, then
        \texttt{unify all /interactive} is useful for recovering from mistakes
        where incorrect unifications assigned wrong math symbol strings to
        variables.
\item[]
    \texttt{improve} {\em step} (or \texttt{all}) -
      Automatically creates a proof for steps (with no unknown variables)
      whose proof requires no statements with \texttt{\$e} hypotheses.  Useful
      for filling in proofs of \texttt{\$f} hypotheses.  The \texttt{/depth}
      qualifier will also try statements whose \texttt{\$e} hypotheses contain
      no new variables.  {\em Warning:} Save your work (with \texttt{save
      new{\char`\_}proof} then \texttt{write source}) before using
      \texttt{/depth = 2} or greater, since the search time grows
      exponentially and may never terminate in a reasonable time, and you
      cannot interrupt the search.  I have found that it is rare for
      \texttt{/depth = 3} or greater to be useful.
 \item[]
    \texttt{save new{\char`\_}proof} - Saves the proof in progress in the program's
        internal database buffer.  To save it permanently into the database file,
        use \texttt{write source} after
        \texttt{save new{\char`\_}proof}.  To revert to the last
        \texttt{save new{\char`\_}proof},
        \texttt{exit /force} from the Proof Assistant then re-enter the Proof
        Assistant.
 \item[]
    \texttt{match step} {\em step} (or \texttt{match all}) - Shows what
        statements are
        possibilities for the \texttt{assign} statement. (This command
        is not very
        useful in its present form and hopefully will be improved
        eventually.  In the meantime, use the \texttt{search} statement for
        candidates matching specific math token combinations.)
 \item[]
 \texttt{minimize{\char`\_}with}\index{\texttt{minimize{\char`\_}with} command}
% 3/10/07 Note: line-breaking the above results in duplicate index entries
     - After a proof is complete, this command will attempt
        to match other database theorems to the proof to see if the proof
        size can be reduced as a result.  See \texttt{help
        minimize{\char`\_}with} in the
        Metamath program for its usage.
 \item[]
 \texttt{undo}\index{\texttt{undo} command}
    - Undo the effect of a proof-changing command (all but the
      \texttt{show} and \texttt{save} commands above).
 \item[]
 \texttt{redo}\index{\texttt{redo} command}
    - Reverse the previous \texttt{undo}.
\end{itemize}

The following commands set parameters that may be relevant to your proof.
Consult the individual \texttt{help set}... commands.
\begin{itemize}
   \item[] \texttt{set unification{\char`\_}timeout}
 \item[]
    \texttt{set search{\char`\_}limit}
  \item[]
    \texttt{set empty{\char`\_}substitution} - note that default is \texttt{off}
\end{itemize}

Type \texttt{exit} to exit the \texttt{MM-PA>}
 prompt and get back to the \texttt{MM>} prompt.
Another \texttt{exit} will then get you out of Metamath.



\subsection{\texttt{prove} Command}\index{\texttt{prove} command}
Syntax:  \texttt{prove} {\em label}

This command will enter the Proof Assistant\index{Proof Assistant}, which will
allow you to create or edit the proof of the specified statement.
The command-line prompt will change from \texttt{MM>} to \texttt{MM-PA>}.

Note:  In the present version of
Metamath\index{Metamath!limitations of version 0.07.30}, the Proof
Assistant does not verify that \texttt{\$d}\index{\texttt{\$d}
statement} restrictions are met as a proof is being built.  After you
have completed a proof, you should type \texttt{save new{\char`\_}proof}
followed by \texttt{verify proof} {\em label} (where {\em label} is the
statement you are proving with the \texttt{prove} command) to verify the
\texttt{\$d} restrictions.

See also: \texttt{exit}

\subsection{\texttt{set unification\_timeout} Command}\index{\texttt{set
unification{\char`\_}timeout} command}
Syntax:  \texttt{set unification{\char`\_}timeout} {\em number}

(This command is available outside the Proof Assistant but affects the
Proof Assistant\index{Proof Assistant} only.)

Sometimes the Proof Assistant will inform you that a unification
time-out occurred.  This may happen when you try to \texttt{unify}
formulas with many temporary variables\index{temporary variable}
(\texttt{\$1}, \texttt{\$2}, etc.), since the time to compute all possible
unifications may grow exponentially with the number of variables.  If
you want Metamath to try harder (and you're willing to wait longer) you
may increase this parameter.  \texttt{show settings} will show you the
current value.

\subsection{\texttt{set empty\_substitution} Command}\index{\texttt{set
empty{\char`\_}substitution} command}
% These long names can't break well in narrow mode, and even "sloppy"
% is not enough. Work around this by not demanding justification.
\begin{flushleft}
Syntax:  \texttt{set empty{\char`\_}substitution on} or \texttt{set
empty{\char`\_}substitution off}
\end{flushleft}

(This command is available outside the Proof Assistant but affects the
Proof Assistant\index{Proof Assistant} only.)

The Metamath language allows variables to be
substituted\index{substitution!variable}\index{variable substitution}
with empty symbol sequences\index{empty substitution}.  However, in many
formal systems\index{formal system} this will never happen in a valid
proof.  Allowing for this possibility increases the likelihood of
ambiguous unifications\index{ambiguous
unification}\index{unification!ambiguous} during proof creation.
The default is that
empty substitutions are not allowed; for formal systems requiring them,
you must \texttt{set empty{\char`\_}substitution on}.
(An example where this must be \texttt{on}
would be a system that implements a Deduction Rule and in
which deductions from empty assumption lists would be permissible.  The
MIU-system\index{MIU-system} described in Appendix~\ref{MIU} is another
example.)
Note that empty substitutions are
always permissible in proof verification (VERIFY PROOF...) outside the
Proof Assistant.  (See the MIU system in the Metamath book for an example
of a system needing empty substitutions; another example would be a
system that implements a Deduction Rule and in which deductions from
empty assumption lists would be permissible.)

It is better to leave this \texttt{off} when working with \texttt{set.mm}.
Note that this command does not affect the way proofs are verified with
the \texttt{verify proof} command.  Outside of the Proof Assistant,
substitution of empty sequences for math symbols is always allowed.

\subsection{\texttt{set search\_limit} Command}\index{\texttt{set
search{\char`\_}limit} command} Syntax:  \texttt{set search{\char`\_}limit} {\em
number}

(This command is available outside the Proof Assistant but affects the
Proof Assistant\index{Proof Assistant} only.)

This command sets a parameter that determines when the \texttt{improve} command
in Proof Assistant mode gives up.  If you want \texttt{improve} to search harder,
you may increase it.  The \texttt{show settings} command tells you its current
value.


\subsection{\texttt{show new\_proof} Command}\index{\texttt{show
new{\char`\_}proof} command}
Syntax:  \texttt{show new{\char`\_}proof} [{\em
qualifiers} (see below)]

This command (available only in Proof Assistant mode) displays the proof
in progress.  It is identical to the \texttt{show proof} command, except that
there is no statement argument (since it is the statement being proved) and
the following qualifiers are not available:

    \texttt{/statement{\char`\_}summary}

    \texttt{/detailed{\char`\_}step}

Also, the following additional qualifiers are available:

    \texttt{/unknown} - Shows only steps that have no statement assigned.

    \texttt{/not{\char`\_}unified} - Shows only steps that have not been unified.

Note that \texttt{/essential}, \texttt{/depth}, \texttt{/unknown}, and
\texttt{/not{\char`\_}unified} may be
used in any combination; each of them effectively filters out additional
steps from the proof display.

See also:  \texttt{show proof}






\subsection{\texttt{assign} Command}\index{\texttt{assign} command}
Syntax:   \texttt{assign} {\em step} {\em label} [\texttt{/no{\char`\_}unify}]

   and:   \texttt{assign first} {\em label}

   and:   \texttt{assign last} {\em label}


This command, available in the Proof Assistant only, assigns an unknown
step (one with \texttt{?} in the \texttt{show new{\char`\_}proof}
listing) with the statement specified by {\em label}.  The assignment
will not be allowed if the statement cannot be unified with the step.

If \texttt{last} is specified instead of {\em step} number, the last
step that is shown by \texttt{show new{\char`\_}proof /unknown} will be
used.  This can be useful for building a proof with a command file (see
\texttt{help submit}).  It also makes building proofs faster when you know
the assignment for the last step.

If \texttt{first} is specified instead of {\em step} number, the first
step that is shown by \texttt{show new{\char`\_}proof /unknown} will be
used.

If {\em step} is zero or negative, the -{\em step}th from last unknown
step, as shown by \texttt{show new{\char`\_}proof /unknown}, will be
used.  \texttt{assign -1} {\em label} will assign the penultimate
unknown step, \texttt{assign -2} {\em label} the antepenultimate, and
\texttt{assign 0} {\em label} is the same as \texttt{assign last} {\em
label}.

Optional command qualifier:

    \texttt{/no{\char`\_}unify} - do not prompt user to select a unification if there is
        more than one possibility.  This is useful for noninteractive
        command files.  Later, the user can \texttt{unify all /interactive}.
        (The assignment will still be automatically unified if there is only
        one possibility and will be refused if unification is not possible.)



\subsection{\texttt{match} Command}\index{\texttt{match} command}
Syntax:  \texttt{match step} {\em step} [\texttt{/max{\char`\_}essential{\char`\_}hyp}
{\em number}]

    and:  \texttt{match all} [\texttt{/essential}]
          [\texttt{/max{\char`\_}essential{\char`\_}hyp} {\em number}]

This command, available in the Proof Assistant only, shows what
statements can be unified with the specified step(s).  {\em Note:} In
its current form, this command is not very useful because of the large
number of matches it reports.  I rarely use it.
It may be enhanced in the future.

Optional command qualifiers:

    \texttt{/max{\char`\_}essential{\char`\_}hyp} {\em number} - filters out
        of the list any statements
        with more than the specified number of
        \texttt{\$e}\index{\texttt{\$e} statement} hypotheses

    \texttt{/essential{\char`\_}only} - in the \texttt{match all} statement, only
        the steps that
        would be listed in the \texttt{show new{\char`\_}proof /essential} display are
        matched.



\subsection{\texttt{let} Command}\index{\texttt{let} command}
Syntax: \texttt{let variable} {\em variable} = \verb/"/{\em symbol-sequence}\verb/"/

  and: \texttt{let step} {\em step} = \verb/"/{\em symbol-sequence}\verb/"/

These commands, available in the Proof Assistant\index{Proof Assistant}
only, assign a temporary variable\index{temporary variable} or unknown
step with a specific symbol sequence.  They are useful in the middle of
creating a proof, when you know what should be in the proof step but the
unification algorithm doesn't yet have enough information to completely
specify the temporary variables.  A ``temporary variable'' is one that
has the form \texttt{\$}{\em nn} in the proof display, such as
\texttt{\$1}, \texttt{\$2}, etc.  The {\em symbol-sequence} may contain
other unknown variables if desired.  Examples:

    \verb/let variable $32 = "A = B"/

    \verb/let variable $32 = "A = $35"/

    \verb/let step 10 = '|- x = x'/

    \verb/let step -2 = "|- ( $7 = ph )"/

Any symbol sequence will be accepted for the \texttt{let variable}
command.  Only those symbol sequences that can be unified with the step
will be accepted for \texttt{let step}.

The \texttt{let} commands ``zap'' the proof with information that can
only be verified when the proof is built up further.  If you make an
error, the command sequence \texttt{delete
floating{\char`\_}hypotheses}, \texttt{initialize all}, and
\texttt{unify all /interactive} will undo a bad \texttt{let} assignment.

If {\em step} is zero or negative, the -{\em step}th from last unknown
step, as shown by \texttt{show new{\char`\_}proof /unknown}, will be
used.  The command \texttt{let step 0} = \verb/"/{\em
symbol-sequence}\verb/"/ will use the last unknown step, \texttt{let
step -1} = \verb/"/{\em symbol-sequence}\verb/"/ the penultimate, etc.
If {\em step} is positive, \texttt{let step} may be used to assign known
(in the sense of having previously been assigned a label with
\texttt{assign}) as well as unknown steps.

Either single or double quotes can surround the {\em symbol-sequence} as
long as they are different from any quotes inside a {\em
symbol-sequence}.  If {\em symbol-sequence} contains both kinds of
quotes, see the instructions at the end of \texttt{help let} in the
Metamath program.


\subsection{\texttt{unify} Command}\index{\texttt{unify} command}
Syntax:  \texttt{unify step} {\em step}

      and:   \texttt{unify all} [\texttt{/interactive}]

These commands, available in the Proof Assistant only, unify the source
and target of the specified step(s). If you specify a specific step, you
will be prompted to select among the unifications that are possible.  If
you specify \texttt{all}, only those steps with unique unifications will be
unified.

Optional command qualifier for \texttt{unify all}:

    \texttt{/interactive} - You will be prompted to select among the
        unifications
        that are possible for any steps that do not have unique
        unifications.  (Otherwise \texttt{unify all} will bypass these.)

See also \texttt{set unification{\char`\_}timeout}.  The default is
100000, but increasing it to 1000000 can help difficult cases.  Manually
assigning some or all of the unknown variables with the \texttt{let
variable} command also helps difficult cases.



\subsection{\texttt{initialize} Command}\index{\texttt{initialize} command}
Syntax:  \texttt{initialize step} {\em step}

    and: \texttt{initialize all}

These commands, available in the Proof Assistant\index{Proof Assistant}
only, ``de-unify'' the target and source of a step (or all steps), as
well as the hypotheses of the source, and makes all variables in the
source and the source's hypotheses unknown.  This command is useful to
help recover from incorrect unifications that resulted from an incorrect
\texttt{assign}, \texttt{let}, or unification choice.  Part or all of
the command sequence \texttt{delete floating{\char`\_}hypotheses},
\texttt{initialize all}, and \texttt{unify all /interactive} will recover
from incorrect unifications.

See also:  \texttt{unify} and \texttt{delete}



\subsection{\texttt{delete} Command}\index{\texttt{delete} command}
Syntax:  \texttt{delete step} {\em step}

   and:      \texttt{delete all} -- {\em Warning: dangerous!}

   and:      \texttt{delete floating{\char`\_}hypotheses}

These commands are available in the Proof Assistant only.  The
\texttt{delete step} command deletes the proof tree section that
branches off of the specified step and makes the step become unknown.
\texttt{delete all} is equivalent to \texttt{delete step} {\em step}
where {\em step} is the last step in the proof (i.e.\ the beginning of
the proof tree).

In most cases the \texttt{undo} command is the best way to undo
a previous step.
An alternative is to salvage your last \texttt{save
new{\char`\_}proof} by exiting and reentering the Proof Assistant.
For this to work, keep a log file open to record your work
and to do \texttt{save new{\char`\_}proof} frequently, especially before
\texttt{delete}.

\texttt{delete floating{\char`\_}hypotheses} will delete all sections of
the proof that branch off of \texttt{\$f}\index{\texttt{\$f} statement}
statements.  It is sometimes useful to do this before an
\texttt{initialize} command to recover from an error.  Note that once a
proof step with a \texttt{\$f} hypothesis as the target is completely
known, the \texttt{improve} command can usually fill in the proof for
that step.  Unlike the deletion of logical steps, \texttt{delete
floating{\char`\_}hypotheses} is a relatively safe command that is
usually easy to recover from.



\subsection{\texttt{improve} Command}\index{\texttt{improve} command}
\label{improve}
Syntax:  \texttt{improve} {\em step} [\texttt{/depth} {\em number}]
                                               [\texttt{/no{\char`\_}distinct}]

   and:   \texttt{improve first} [\texttt{/depth} {\em number}]
                                              [\texttt{/no{\char`\_}distinct}]

   and:   \texttt{improve last} [\texttt{/depth} {\em number}]
                                              [\texttt{/no{\char`\_}distinct}]

   and:   \texttt{improve all} [\texttt{/depth} {\em number}]
                                              [\texttt{/no{\char`\_}distinct}]

These commands, available in the Proof Assistant\index{Proof Assistant}
only, try to find proofs automatically for unknown steps whose symbol
sequences are completely known.  They are primarily useful for filling in
proofs of \texttt{\$f}\index{\texttt{\$f} statement} hypotheses.  The
search will be restricted to statements having no
\texttt{\$e}\index{\texttt{\$e} statement} hypotheses.

\begin{sloppypar} % narrow
Note:  If memory is limited, \texttt{improve all} on a large proof may
overflow memory.  If you use \texttt{set unification{\char`\_}timeout 1}
before \texttt{improve all}, there will usually be sufficient
improvement to easily recover and completely \texttt{improve} the proof
later on a larger computer.  Warning:  Once memory has overflowed, there
is no recovery.  If in doubt, save the intermediate proof (\texttt{save
new{\char`\_}proof} then \texttt{write source}) before \texttt{improve
all}.
\end{sloppypar}

If \texttt{last} is specified instead of {\em step} number, the last
step that is shown by \texttt{show new{\char`\_}proof /unknown} will be
used.

If \texttt{first} is specified instead of {\em step} number, the first
step that is shown by \texttt{show new{\char`\_}proof /unknown} will be
used.

If {\em step} is zero or negative, the -{\em step}th from last unknown
step, as shown by \texttt{show new{\char`\_}proof /unknown}, will be
used.  \texttt{improve -1} will use the penultimate
unknown step, \texttt{improve -2} {\em label} the antepenultimate, and
\texttt{improve 0} is the same as \texttt{improve last}.

Optional command qualifier:

    \texttt{/depth} {\em number} - This qualifier will cause the search
        to include
        statements with \texttt{\$e} hypotheses (but no new variables in
        the \texttt{\$e}
        hypotheses), provided that the backtracking has not exceeded the
        specified depth. {\em Warning:}  Try \texttt{/depth 1},
        then \texttt{2}, then \texttt{3}, etc.
        in sequence because of possible exponential blowups.  Save your
        work before trying \texttt{/depth} greater than \texttt{1}!

    \texttt{/no{\char`\_}distinct} - Skip trial statements that have
        \texttt{\$d}\index{\texttt{\$d} statement} requirements.
        This qualifier will prevent assignments that might violate \texttt{\$d}
        requirements but it also could miss possible legal assignments.

See also: \texttt{set search{\char`\_}limit}

\subsection{\texttt{save new\_proof} Command}\index{\texttt{save
new{\char`\_}proof} command}
Syntax:  \texttt{save new{\char`\_}proof} {\em label} [\texttt{/normal}]
   [\texttt{/compressed}]

The \texttt{save new{\char`\_}proof} command is available in the Proof
Assistant only.  It saves the proof in progress in the source
buffer\index{source buffer}.  \texttt{save new{\char`\_}proof} may be
used to save a completed proof, or it may be used to save a proof in
progress in order to work on it later.  If an incomplete proof is saved,
any user assignments with \texttt{let step} or \texttt{let variable}
will be lost, as will any ambiguous unifications\index{ambiguous
unification}\index{unification!ambiguous} that were resolved manually.
To help make recovery easier, it can be helpful to \texttt{improve all}
before \texttt{save new{\char`\_}proof} so that the incomplete proof
will have as much information as possible.

Note that the proof will not be permanently saved until a \texttt{write
source} command is issued.

Optional command qualifiers:

    \texttt{/normal} - The proof is saved in the normal format (i.e., as a
        sequence of labels, which is the defined format of the basic Metamath
        language).\index{basic language}  This is the default format that
        is used if a qualifier is omitted.

    \texttt{/compressed} - The proof is saved in the compressed format, which
        reduces storage requirements for a database.
        (See Appendix~\ref{compressed}.)


\section{Creating \LaTeX\ Output}\label{texout}\index{latex@{\LaTeX}}

You can generate \LaTeX\ output given the
information in a database.
The database must already include the necessary typesetting information
(see section \ref{tcomment} for how to provide this information).

The \texttt{show statement} and \texttt{show proof} commands each have a
special \texttt{/tex} command qualifier that produces \LaTeX\ output.
(The \texttt{show statement} command also has the
\texttt{/simple{\char`\_}tex} qualifier for output that is easier to
edit by hand.)  Before you can use them, you must open a \LaTeX\ file to
which to send their output.  A typical complete session will use this
sequence of Metamath commands:

\begin{verbatim}
read set.mm
open tex example.tex
show statement a1i /tex
show proof a1i /all/lemmon/renumber/tex
show statement uneq2 /tex
show proof uneq2 /all/lemmon/renumber/tex
close tex
\end{verbatim}

See Section~\ref{mathcomments} for information on comment markup and
Appendix~\ref{ASCII} for information on how math symbol translation is
specified.

To format and print the \LaTeX\ source, you will need the \LaTeX\
program, which is standard on most Linux installations and available for
Windows.  On Linux, in order to create a {\sc pdf} file, you will
typically type at the shell prompt
\begin{verbatim}
$ pdflatex example.tex
\end{verbatim}

\subsection{\texttt{open tex} Command}\index{\texttt{open tex} command}
Syntax:  \texttt{open tex} {\em file-name} [\texttt{/no{\char`\_}header}]

This command opens a file for writing \LaTeX\
source\index{latex@{\LaTeX}} and writes a \LaTeX\ header to the file.
\LaTeX\ source can be written with the \texttt{show proof}, \texttt{show
new{\char`\_}proof}, and \texttt{show statement} commands using the
\texttt{/tex} qualifier.

The mapping to \LaTeX\ symbols is defined in a special comment
containing a \texttt{\$t} token, described in Appendix~\ref{ASCII}.

There is an optional command qualifier:

    \texttt{/no{\char`\_}header} - This qualifier prevents a standard
        \LaTeX\ header and trailer
        from being included with the output \LaTeX\ code.


\subsection{\texttt{close tex} Command}\index{\texttt{close tex} command}
Syntax:  \texttt{close tex}

This command writes a trailer to any \LaTeX\ file\index{latex@{\LaTeX}}
that was opened with \texttt{open tex} (unless
\texttt{/no{\char`\_}header} was used with \texttt{open tex}) and closes
the \LaTeX\ file.


\section{Creating {\sc HTML} Output}\label{htmlout}

You can generate {\sc html} web pages given the
information in a database.
The database must already include the necessary typesetting information
(see section \ref{tcomment} for how to provide this information).
The ability to produce {\sc html} web pages was added in Metamath version
0.07.30.

To create an {\sc html} output file(s) for \texttt{\$a} or \texttt{\$p}
statement(s), use
\begin{quote}
    \texttt{show statement} {\em label-match} \texttt{/html}
\end{quote}
The output file will be named {\em label-match}\texttt{.html}
for each match.  When {\em
label-match} has wildcard (\texttt{*}) characters, all statements with
matching labels will have {\sc html} files produced for them.  Also,
when {\em label-match} has a wildcard (\texttt{*}) character, two additional
files, \texttt{mmdefinitions.html} and \texttt{mmascii.html} will be
produced.  To produce {\em only} these two additional files, you can use
\texttt{?*}, which will not match any statement label, in place of {\em
label-match}.

There are three other qualifiers for \texttt{show statement} that also
generate {\sc HTML} code.  These are \texttt{/alt{\char`\_}html},
\texttt{/brief{\char`\_}html}, and
\texttt{/brief{\char`\_}alt{\char`\_}html}, and are described in the
next section.

The command
\begin{quote}
    \texttt{show statement} {\em label-match} \texttt{/alt{\char`\_}html}
\end{quote}
does the same as \texttt{show statement} {\em label-match} \texttt{/html},
except that the {\sc html} code for the symbols is taken from
\texttt{althtmldef} statements instead of \texttt{htmldef} statements in
the \texttt{\$t} comment.

The command
\begin{verbatim}
show statement * /brief_html
\end{verbatim}
invokes a special mode that just produces definition and theorem lists
accompanied by their symbol strings, in a format suitable for copying and
pasting into another web page (such as the tutorial pages on the
Metamath web site).

Finally, the command
\begin{verbatim}
show statement * /brief_alt_html
\end{verbatim}
does the same as \texttt{show statement * / brief{\char`\_}html}
for the alternate {\sc html}
symbol representation.

A statement's comment can include a special notation that provides a
certain amount of control over the {\sc HTML} version of the comment.  See
Section~\ref{mathcomments} (p.~\pageref{mathcomments}) for the comment
markup features.

The \texttt{write theorem{\char`\_}list} and \texttt{write bibliography}
commands, which are described below, provide as a side effect complete
error checking for all of the features described in this
section.\index{error checking}

\subsection{\texttt{write theorem\_list}
Command}\index{\texttt{write theorem{\char`\_}list} command}
Syntax:  \texttt{write theorem{\char`\_}list}
[\texttt{/theorems{\char`\_}per{\char`\_}page} {\em number}]

This command writes a list of all of the \texttt{\$a} and \texttt{\$p}
statements in the database into a web page file
 called \texttt{mmtheorems.html}.
When additional files are needed, they are called
\texttt{mmtheorems2.html}, \texttt{mmtheorems3.html}, etc.

Optional command qualifier:

    \texttt{/theorems{\char`\_}per{\char`\_}page} {\em number} -
 This qualifier specifies the number of statements to
        write per web page.  The default is 100.

{\em Note:} In version 0.07.30\index{Metamath!limitations of version
0.07.30} of Metamath, the ``Nearby theorems'' links on the individual
web pages presuppose 100 theorems per page when linking to the theorem
list pages.  Therefore the \texttt{/theorems{\char`\_}per{\char`\_}page}
qualifier, if it specifies a number other than 100, will cause the
individual web pages to be out of sync and should not be used to
generate the main theorem list for the web site.  This is expected to be
fixed in a future version.


\subsection{\texttt{write bibliography}\label{wrbib}
Command}\index{\texttt{write bibliography} command}
Syntax:  \texttt{write bibliography} {\em filename}

This command reads an existing {\sc html} bibliographic cross-reference
file, normally called \texttt{mmbiblio.html}, and updates it per the
bibliographic links in the database comments.  The file is updated
between the {\sc html} comment lines \texttt{<!--
{\char`\#}START{\char`\#} -->} and \texttt{<!-- {\char`\#}END{\char`\#}
-->}.  The original input file is renamed to {\em
filename}\texttt{{\char`\~}1}.

A bibliographic reference is indicated with the reference name
in brackets, such as  \texttt{Theorem 3.1 of
[Monk] p.\ 22}.
See Section~\ref{htmlout} (p.~\pageref{htmlout}) for
syntax details.


\subsection{\texttt{write recent\_additions}
Command}\index{\texttt{write recent{\char`\_}additions} command}
Syntax:  \texttt{write recent{\char`\_}additions} {\em filename}
[\texttt{/limit} {\em number}]

This command reads an existing ``Recent Additions'' {\sc html} file,
normally called \texttt{mmrecent.html}, and updates it with the
descriptions of the most recently added theorems to the database.
 The file is updated between
the {\sc html} comment lines \texttt{<!-- {\char`\#}START{\char`\#} -->}
and \texttt{<!-- {\char`\#}END{\char`\#} -->}.  The original input file
is renamed to {\em filename}\texttt{{\char`\~}1}.

Optional command qualifier:

    \texttt{/limit} {\em number} -
 This qualifier specifies the number of most recent theorems to
   write to the output file.  The default is 100.


\section{Text File Utilities}

\subsection{\texttt{tools} Command}\index{\texttt{tools} command}
Syntax:  \texttt{tools}

This command invokes an easy-to-use, general purpose utility for
manipulating the contents of {\sc ascii} text files.  Upon typing
\texttt{tools}, the command-line prompt will change to \texttt{TOOLS>}
until you type \texttt{exit}.  The \texttt{tools} commands can be used
to perform simple, global edits on an input/output file,
such as making a character string substitution on each line, adding a
string to each line, and so on.  A typical use of this utility is
to build a \texttt{submit} input file to perform a common operation on a
list of statements obtained from \texttt{show label} or \texttt{show
usage}.

The actions of most of the \texttt{tools} commands can also be
performed with equivalent (and more powerful) Unix shell commands, and
some users may find those more efficient.  But for Windows users or
users not comfortable with Unix, \texttt{tools} provides an
easy-to-learn alternative that is adequate for most of the
script-building tasks needed to use the Metamath program effectively.

\subsection{\texttt{help} Command (in \texttt{tools})}
Syntax:  \texttt{help}

The \texttt{help} command lists the commands available in the
\texttt{tools} utility, along with a brief description.  Each command,
in turn, has its own help, such as \texttt{help add}.  As with
Metamath's \texttt{MM>} prompt, a complete command can be entered at
once, or just the command word can be typed, causing the program to
prompt for each argument.

\vskip 1ex
\noindent Line-by-line editing commands:

  \texttt{add} - Add a specified string to each line in a file.

  \texttt{clean} - Trim spaces and tabs on each line in a file; convert
         characters.

  \texttt{delete} - Delete a section of each line in a file.

  \texttt{insert} - Insert a string at a specified column in each line of
        a file.

  \texttt{substitute} - Make a simple substitution on each line of the file.

  \texttt{tag} - Like \texttt{add}, but restricted to a range of lines.

  \texttt{swap} - Swap the two halves of each line in a file.

\vskip 1ex
\noindent Other file-processing commands:

  \texttt{break} - Break up (tokenize) a file into a list of tokens (one per
        line).

  \texttt{build} - Build a file with multiple tokens per line from a list.

  \texttt{count} - Count the occurrences in a file of a specified string.

  \texttt{number} - Create a list of numbers.

  \texttt{parallel} - Put two files in parallel.

  \texttt{reverse} - Reverse the order of the lines in a file.

  \texttt{right} - Right-justify lines in a file (useful before sorting
         numbers).

%  \texttt{tag} - Tag edit updates in a program for revision control.

  \texttt{sort} - Sort the lines in a file with key starting at
         specified string.

  \texttt{match} - Extract lines containing (or not) a specified string.

  \texttt{unduplicate} - Eliminate duplicate occurrences of lines in a file.

  \texttt{duplicate} - Extract first occurrence of any line occurring
         more than

   \ \ \    once in a file, discarding lines occurring exactly once.

  \texttt{unique} - Extract lines occurring exactly once in a file.

  \texttt{type} (10 lines) - Display the first few lines in a file.
                  Similar to Unix \texttt{head}.

  \texttt{copy} - Similar to Unix \texttt{cat} but safe (same input
         and output file allowed).

  \texttt{submit} - Run a script containing \texttt{tools} commands.

\vskip 1ex

\noindent Note:
  \texttt{unduplicate}, \texttt{duplicate}, and \texttt{unique} also
 sort the lines as a side effect.


\subsection{Using \texttt{tools} to Build Metamath \texttt{submit}
Scripts}

The \texttt{break} command is typically used to break up a series of
statement labels, such as the output of Metamath's \texttt{show usage},
into one label per line.  The other \texttt{tools} commands can then be
used to add strings before and after each statement label to specify
commands to be performed on the statement.  The \texttt{parallel}
command is useful when a statement label must be mentioned more than
once on a line.

Very often a \texttt{submit} script for Metamath will require multiple
command lines for each statement being processed.  For example, you may
want to enter the Proof Assistant, \texttt{minimize{\char`\_}with} your
latest theorem, \texttt{save} the new proof, and \texttt{exit} the Proof
Assistant.  To accomplish this, you can build a file with these four
commands for each statement on a single line, separating each command
with a designated character such as \texttt{@}.  Then at the end you can
\texttt{substitute} each \texttt{@} with \texttt{{\char`\\}n} to break
up the lines into individual command lines (see \texttt{help
substitute}).


\subsection{Example of a \texttt{tools} Session}

To give you a quick feel for the \texttt{tools} utility, we show a
simple session where we create a file \texttt{n.txt} with 3 lines, add
strings before and after each line, and display the lines on the screen.
You can experiment with the various commands to gain experience with the
\texttt{tools} utility.

\begin{verbatim}
MM> tools
Entering the Text Tools utilities.
Type HELP for help, EXIT to exit.
TOOLS> number
Output file <n.tmp>? n.txt
First number <1>?
Last number <10>? 3
Increment <1>?
TOOLS> add
Input/output file? n.txt
String to add to beginning of each line <>? This is line
String to add to end of each line <>? .
The file n.txt has 3 lines; 3 were changed.
First change is on line 1:
This is line 1.
TOOLS> type n.txt
This is line 1.
This is line 2.
This is line 3.
TOOLS> exit
Exiting the Text Tools.
Type EXIT again to exit Metamath.
MM>
\end{verbatim}



\appendix
\chapter{Sample Representations}
\label{ASCII}

This Appendix provides a sample of {\sc ASCII} representations,
their corresponding traditional mathematical symbols,
and a discussion of their meanings
in the \texttt{set.mm} database.
These are provided in order of appearance.
This is only a partial list, and new definitions are routinely added.
A complete list is available at \url{http://metamath.org}.

These {\sc ASCII} representations, along
with information on how to display them,
are defined in the \texttt{set.mm} database file inside
a special comment called a \texttt{\$t} {\em
comment}\index{\texttt{\$t} comment} or {\em typesetting
comment.}\index{typesetting comment}
A typesetting comment
is indicated by the appearance of the
two-character string \texttt{\$t} at the beginning of the comment.
For more information,
see Section~\ref{tcomment}, p.~\pageref{tcomment}.

In the following table the ``{\sc ASCII}'' column shows the {\sc ASCII}
representation,
``Symbol'' shows the mathematical symbolic display
that corresponds to that {\sc ASCII} representation, ``Labels'' shows
the key label(s) that define the representation, and
``Description'' provides a description about the symbol.
As usual, ``iff'' is short for ``if and only if.''\index{iff}
In most cases the ``{\sc ASCII}'' column only shows
the key token, but it will sometimes show a sequence of tokens
if that is necessary for clarity.

{\setlength{\extrarowsep}{4pt} % Keep rows from being too close together
\begin{longtabu}   { @{} c c l X }
\textbf{ASCII} & \textbf{Symbol} & \textbf{Labels} & \textbf{Description} \\
\endhead
\texttt{|-} & $\vdash$ & &
  ``It is provable that...'' \\
\texttt{ph} & $\varphi$ & \texttt{wph} &
  The wff (boolean) variable phi,
  conventionally the first wff variable. \\
\texttt{ps} & $\psi$ & \texttt{wps} &
  The wff (boolean) variable psi,
  conventionally the second wff variable. \\
\texttt{ch} & $\chi$ & \texttt{wch} &
  The wff (boolean) variable chi,
  conventionally the third wff variable. \\
\texttt{-.} & $\lnot$ & \texttt{wn} &
  Logical not. E.g., if $\varphi$ is true, then $\lnot \varphi$ is false. \\
\texttt{->} & $\rightarrow$ & \texttt{wi} &
  Implies, also known as material implication.
  In classical logic the expression $\varphi \rightarrow \psi$ is true
  if either $\varphi$ is false or $\psi$ is true (or both), that is,
  $\varphi \rightarrow \psi$ has the same meaning as
  $\lnot \varphi \lor \psi$ (as proven in theorem \texttt{imor}). \\
\texttt{<->} & $\leftrightarrow$ &
  \hyperref[df-bi]{\texttt{df-bi}} &
  Biconditional (aka is-equals for boolean values).
  $\varphi \leftrightarrow \psi$ is true iff
  $\varphi$ and $\psi$ have the same value. \\
\texttt{\char`\\/} & $\lor$ &
  \makecell[tl]{{\hyperref[df-or]{\texttt{df-or}}}, \\
	         \hyperref[df-3or]{\texttt{df-3or}}} &
  Disjunction (logical ``or''). $\varphi \lor \psi$ is true iff
  $\varphi$, $\psi$, or both are true. \\
\texttt{/\char`\\} & $\land$ &
  \makecell[tl]{{\hyperref[df-an]{\texttt{df-an}}}, \\
                 \hyperref[df-3an]{\texttt{df-3an}}} &
  Conjunction (logical ``and''). $\varphi \land \psi$ is true iff
  both $\varphi$ and $\psi$ are true. \\
\texttt{A.} & $\forall$ &
  \texttt{wal} &
  For all; the wff $\forall x \varphi$ is true iff
  $\varphi$ is true for all values of $x$. \\
\texttt{E.} & $\exists$ &
  \hyperref[df-ex]{\texttt{df-ex}} &
  There exists; the wff
  $\exists x \varphi$ is true iff
  there is at least one $x$ where $\varphi$ is true. \\
\texttt{[ y / x ]} & $[ y / x ]$ &
  \hyperref[df-sb]{\texttt{df-sb}} &
  The wff $[ y / x ] \varphi$ produces
  the result when $y$ is properly substituted for $x$ in $\varphi$
  ($y$ replaces $x$).
  % This is elsb4
  % ( [ x / y ] z e. y <-> z e. x )
  For example,
  $[ x / y ] z \in y$ is the same as $z \in x$. \\
\texttt{E!} & $\exists !$ &
  \hyperref[df-eu]{\texttt{df-eu}} &
  There exists exactly one;
  $\exists ! x \varphi$ is true iff
  there is at least one $x$ where $\varphi$ is true. \\
\texttt{\{ y | phi \}}  & $ \{ y | \varphi \}$ &
  \hyperref[df-clab]{\texttt{df-clab}} &
  The class of all sets where $\varphi$ is true. \\
\texttt{=} & $ = $ &
  \hyperref[df-cleq]{\texttt{df-cleq}} &
  Class equality; $A = B$ iff $A$ equals $B$. \\
\texttt{e.} & $ \in $ &
  \hyperref[df-clel]{\texttt{df-clel}} &
  Class membership; $A \in B$ if $A$ is a member of $B$. \\
\texttt{{\char`\_}V} & {\rm V} &
  \hyperref[df-v]{\texttt{df-v}} &
  Class of all sets (not itself a set). \\
\texttt{C\_} & $ \subseteq $ &
  \hyperref[df-ss]{\texttt{df-ss}} &
  Subclass (subset); $A \subseteq B$ is true iff
  $A$ is a subclass of $B$. \\
\texttt{u.} & $ \cup $ &
  \hyperref[df-un]{\texttt{df-un}} &
  $A \cup B$ is the union of classes $A$ and $B$. \\
\texttt{i^i} & $ \cap $ &
  \hyperref[df-in]{\texttt{df-in}} &
  $A \cap B$ is the intersection of classes $A$ and $B$. \\
\texttt{\char`\\} & $ \setminus $ &
  \hyperref[df-dif]{\texttt{df-dif}} &
  $A \setminus B$ (set difference)
  is the class of all sets in $A$ except for those in $B$. \\
\texttt{(/)} & $ \varnothing $ &
  \hyperref[df-nul]{\texttt{df-nul}} &
  $ \varnothing $ is the empty set (aka null set). \\
\texttt{\char`\~P} & $ \cal P $ &
  \hyperref[df-pw]{\texttt{df-pw}} &
  Power class. \\
\texttt{<.\ A , B >.} & $\langle A , B \rangle$ &
  \hyperref[df-op]{\texttt{df-op}} &
  The ordered pair $\langle A , B \rangle$. \\
\texttt{( F ` A )} & $ ( F ` A ) $ &
  \hyperref[df-fv]{\texttt{df-fv}} &
  The value of function $F$ when applied to $A$. \\
\texttt{\_i} & $ i $ &
  \texttt{df-i} &
  The square root of negative one. \\
\texttt{x.} & $ \cdot $ &
  \texttt{df-mul} &
  Complex number multiplication; $2~\cdot~3~=~6$. \\
\texttt{CC} & $ \mathbb{C} $ &
  \texttt{df-c} &
  The set of complex numbers. \\
\texttt{RR} & $ \mathbb{R} $ &
  \texttt{df-r} &
  The set of real numbers. \\
\end{longtabu}
} % end of extrarowsep

\chapter{Compressed Proofs}
\label{compressed}\index{compressed proof}\index{proof!compressed}

The proofs in the \texttt{set.mm} set theory database are stored in compressed
format for efficiency.  Normally you needn't concern yourself with the
compressed format, since you can display it with the usual proof display tools
in the Metamath program (\texttt{show proof}\ldots) or convert it to the normal
RPN proof format described in Section~\ref{proof} (with \texttt{save proof}
{\em label} \texttt{/normal}).  However for sake of completeness we describe the
format here and show how it maps to the normal RPN proof format.

A compressed proof, located between \texttt{\$=} and \texttt{\$.}\ keywords, consists
of a left parenthesis, a sequence of statement labels, a right parenthesis,
and a sequence of upper-case letters \texttt{A} through \texttt{Z} (with optional
white space between them).  White space must surround the parentheses
and the labels.  The left parenthesis tells Metamath that a
compressed proof follows.  (A normal RPN proof consists of just a sequence of
labels, and a parenthesis is not a legal character in a label.)

The sequence of upper-case letters corresponds to a sequence of integers
with the following mapping.  Each integer corresponds to a proof step as
described later.
\begin{center}
  \texttt{A} = 1 \\
  \texttt{B} = 2 \\
   \ldots \\
  \texttt{T} = 20 \\
  \texttt{UA} = 21 \\
  \texttt{UB} = 22 \\
   \ldots \\
  \texttt{UT} = 40 \\
  \texttt{VA} = 41 \\
  \texttt{VB} = 42 \\
   \ldots \\
  \texttt{YT} = 120 \\
  \texttt{UUA} = 121 \\
   \ldots \\
  \texttt{YYT} = 620 \\
  \texttt{UUUA} = 621 \\
   etc.
\end{center}

In other words, \texttt{A} through \texttt{T} represent the
least-significant digit in base 20, and \texttt{U} through \texttt{Y}
represent zero or more most-significant digits in base 5, where the
digits start counting at 1 instead of the usual 0. With this scheme, we
don't need white space between these ``numbers.''

(In the design of the compressed proof format, only upper-case letters,
as opposed to say all non-whitespace printable {\sc ascii} characters other than
%\texttt{\$}, was chosen to make the compressed proof a little less
%displeasing to the eye, at the expense of a typical 20\% compression
\texttt{\$}, were chosen so as not to collide with most text editor
searches, at the expense of a typical 20\% compression
loss.  The base 5/base 20 grouping, as opposed to say base 6/base 19,
was chosen by experimentally determining the grouping that resulted in
best typical compression.)

The letter \texttt{Z} identifies (tags) a proof step that is identical to one
that occurs later on in the proof; it helps shorten the proof by not requiring
that identical proof steps be proved over and over again (which happens often
when building wff's).  The \texttt{Z} is placed immediately after the
least-significant digit (letters \texttt{A} through \texttt{T}) that ends the integer
corresponding to the step to later be referenced.

The integers that the upper-case letters correspond to are mapped to labels as
follows.  If the statement being proved has $m$ mandatory hypotheses, integers
1 through $m$ correspond to the labels of these hypotheses in the order shown
by the \texttt{show statement ... / full} command, i.e., the RPN order\index{RPN
order} of the mandatory
hypotheses.  Integers $m+1$ through $m+n$ correspond to the labels enclosed in
the parentheses of the compressed proof, in the order that they appear, where
$n$ is the number of those labels.  Integers $m+n+1$ on up don't directly
correspond to statement labels but point to proof steps identified with the
letter \texttt{Z}, so that these proof steps can be referenced later in the
proof.  Integer $m+n+1$ corresponds to the first step tagged with a \texttt{Z},
$m+n+2$ to the second step tagged with a \texttt{Z}, etc.  When the compressed
proof is converted to a normal proof, the entire subproof of a step tagged
with \texttt{Z} replaces the reference to that step.

For efficiency, Metamath works with compressed proofs directly, without
converting them internally to normal proofs.  In addition to the usual
error-checking, an error message is given if (1) a label in the label list in
parentheses does not refer to a previous \texttt{\$p} or \texttt{\$a} statement or a
non-mandatory hypothesis of the statement being proved and (2) a proof step
tagged with \texttt{Z} is referenced before the step tagged with the \texttt{Z}.

Just as in a normal proof under development (Section~\ref{unknown}), any step
or subproof that is not yet known may be represented with a single \texttt{?}.
White space does not have to appear between the \texttt{?}\ and the upper-case
letters (or other \texttt{?}'s) representing the remainder of the proof.

% April 1, 2004 Appendix C has been added back in with corrections.
%
% May 20, 2003 Appendix C was removed for now because there was a problem found
% by Bob Solovay
%
% Also, removed earlier \ref{formalspec} 's (3 cases above)
%
% Bob Solovay wrote on 30 Nov 2002:
%%%%%%%%%%%%% (start of email comment )
%      3. My next set of comments concern appendix C. I read this before I
% read Chapter 4. So I first noted that the system as presented in the
% Appendix lacked a certain formal property that I thought desirable. I
% then came up with a revised formal system that had this property. Upon
% reading Chapter 4, I noticed that the revised system was closer to the
% treatment in Chapter 4 than the system in Appendix C.
%
%         First a very minor correction:
%
%         On page 142 line 2: The condition that V(e) != V(f) should only be
% required of e, f in T such that e != f.
%
%         Here is a natural property [transitivity] that one would like
% the formal system to have:
%
%         Let Gamma be a set of statements. Suppose that the statement Phi
% is provable from Gamma and that the statement Psi is provable from Gamma
% \cup {Phi}. Then Psi is provable from Gamma.
%
%         I shall present an example to show that this property does not
% hold for the formal systems of Appendix C:
%
%         I write the example in metamath style:
%
% $c A B C D E $.
% $v x y
%
% ${
% tx $f A x $.
% ty $f B y $.
% ax1 $a C x y $.
% $}
%
% ${
% tx $f A x $.
% ty $f B y $.
% ax2-h1 $e C x y $.
% ax2 $a D y $.
% $}
%
% ${
% ty $f B y $.
% ax3-h1 $e D y $.
% ax3 $a E y $.
% $}
%
% $(These three axioms are Gamma $)
%
% ${
% tx $f A x $.
% ty $f B y $.
% Phi $p D y $=
% tx ty tx ty ax1 ax2 $.
% $}
%
% ${
% ty $f B y $.
% Psi $p E y $=
% ty ty Phi ax3 $.
% $}
%
%
% I omit the formal proofs of the following claims. [I will be glad to
% supply them upon request.]
%
% 1) Psi is not provable from Gamma;
%
% 2) Psi is provable from Gamma + Phi.
%
% Here "provable" refers to the formalism of Appendix C.
%
% The trouble of course is that Psi is lacking the variable declaration
%
% $f Ax $.
%
% In the Metamath system there is no trouble proving Psi. I attach a
% metamath file that shows this and which has been checked by the
% metamath program.
%
% I next want to indicate how I think the treatment in Appendix C should
% be revised so as to conform more closely to the metamath system of the
% main text. The revised system *does* have the transitivity property.
%
% We want to give revised definitions of "statement" and
% "provable". [cf. sections C.2.4. and C.2.5] Our new definitions will
% use the definitions given in Appendix C. So we take the following
% tack. We refer to the original notions as o-statement and o-provable. And
% we refer to the notions we are defining as n-statement and n-provable.
%
%         A n-statement is an o-statement in which the only variables
% that appear in the T component are mandatory.
%
%         To any o-statement we can associate its reduct which is a
% n-statement by dropping all the elements of T or D which contain
% non-mandatory variables.
%
%         An n-statement gamma is n-provable if there is an o-statement
% gamma' which has gamma as its reduct andf such that gamma' is
% o-provable.
%
%         It seems to me [though I am not completely sure on this point]
% that n-provability corresponds to metamath provability as discussed
% say in Chapter 4.
%
%         Attached to this letter is the metamath proof of Phi and Psi
% from Gamma discussed above.
%
%         I am still brooding over the question of whether metamath
% correctly formalizes set-theory. No doubt I will have some questions
% re this after my thoughts become clearer.
%%%%%%%%%%%%%%%% (end of email comment)

%%%%%%%%%%%%%%%% (start of 2nd email comment from Bob Solovay 1-Apr-04)
%
%         I hope that Appendix C is the one that gives a "formal" treatment
% of Metamath. At any rate, thats the appendix I want to comment on.
%
%         I'm going to suggest two changes in the definition.
%
%         First change (in the definition of statement): Require that the
% sets D, T, and E be finite.
%
%         Probably things are fine as you give them. But in the applications
% to the main metamath system they will always be finite, and its useful in
% thinking about things [at least for me] to stick to the finite case.
%
%         Second change:
%
%         First let me give an approximate description. Remove the dummy
% variables from the statement. Instead, include them in the proof.
%
%         More formally: Require that T consists of type declarations only
% for mandatory variables. Require that all the pairs in D consist of
% mandatory variables.
%
%         At the start of a proof we are allowed to declare a finite number
% of dummy variables [provided that none of them appear in any of the
% statements in E \cup {A}. We have to supply type declarations for all the
% dummy variables. We are allowed to add new $d statements referring to
% either the mandatory or dummy variables. But we require that no new $d
% statement references only mandatory variables.
%
%         I find this way of doing things more conceptual than the treatment
% in Appendix C. But the change [which I will use implicitly in later
% letters about doing Peano] is mainly aesthetic. I definitely claim that my
% results on doing Peano all apply to Metamath as it is presented in your
% book.
%
%         --Bob
%
%%%%%%%%%%%%%%%% (end of 2nd email comment)

%%
%% When uncommenting the below, also uncomment references above to {formalspec}
%%
\chapter{Metamath's Formal System}\label{formalspec}\index{Metamath!as a formal
system}

\section{Introduction}

\begin{quote}
  {\em Perfection is when there is no longer anything more to take away.}
    \flushright\sc Antoine de
     Saint-Exupery\footnote{\cite[p.~3-25]{Campbell}.}\\
\end{quote}\index{de Saint-Exupery, Antoine}

This appendix describes the theory behind the Metamath language in an abstract
way intended for mathematicians.  Specifically, we construct two
set-theo\-ret\-i\-cal objects:  a ``formal system'' (roughly, a set of syntax
rules, axioms, and logical rules) and its ``universe'' (roughly, the set of
theorems derivable in the formal system).  The Metamath computer language
provides us with a way to describe specific formal systems and, with the aid of
a proof provided by the user, to verify that given theorems
belong to their universes.

To understand this appendix, you need a basic knowledge of informal set theory.
It should be sufficient to understand, for example, Ch.\ 1 of Munkres' {\em
Topology} \cite{Munkres}\index{Munkres, James R.} or the
introductory set theory chapter
in many textbooks that introduce abstract mathematics. (Note that there are
minor notational differences among authors; e.g.\ Munkres uses $\subset$ instead
of our $\subseteq$ for ``subset.''  We use ``included in'' to mean ``a subset
of,'' and ``belongs to'' or ``is contained in'' to mean ``is an element of.'')
What we call a ``formal'' description here, unlike earlier, is actually an
informal description in the ordinary language of mathematicians.  However we
provide sufficient detail so that a mathematician could easily formalize it,
even in the language of Metamath itself if desired.  To understand the logic
examples at the end of this appendix, familiarity with an introductory book on
mathematical logic would be helpful.

\section{The Formal Description}

\subsection[Preliminaries]{Preliminaries\protect\footnotemark}%
\footnotetext{This section is taken mostly verbatim
from Tarski \cite[p.~63]{Tarski1965}\index{Tarski, Alfred}.}

By $\omega$ we denote the set of all natural numbers (non-negative integers).
Each natural number $n$ is identified with the set of all smaller numbers: $n =
\{ m | m < n \}$.  The formula $m < n$ is thus equivalent to the condition: $m
\in n$ and $m,n \in \omega$. In particular, 0 is the number zero and at the
same time the empty set $\varnothing$, $1=\{0\}$, $2=\{0,1\}$, etc. ${}^B A$
denotes the set of all functions on $B$ to $A$ (i.e.\ with domain $B$ and range
included in $A$).  The members of ${}^\omega A$ are what are called {\em simple
infinite sequences},\index{simple infinite sequence}
with all {\em terms}\index{term} in $A$.  In case $n \in \omega$, the
members of ${}^n A$ are referred to as {\em finite $n$-termed
sequences},\index{finite $n$-termed
sequence} again
with terms in $A$.  The consecutive terms (function values) of a finite or
infinite sequence $f$ are denoted by $f_0, f_1, \ldots ,f_n,\ldots$.  Every
finite sequence $f \in \bigcup _{n \in \omega} {}^n A$ uniquely determines the
number $n$ such that $f \in {}^n A$; $n$ is called the {\em
length}\index{length of a sequence ({$"|\ "|$})} of $f$ and
is denoted by $|f|$.  $\langle a \rangle$ is the sequence $f$ with $|f|=1$ and
$f_0=a$; $\langle a,b \rangle$ is the sequence $f$ with $|f|=2$, $f_0=a$,
$f_1=b$; etc.  Given two finite sequences $f$ and $g$, we denote by $f\frown g$
their {\em concatenation},\index{concatenation} i.e., the
finite sequence $h$ determined by the
conditions:
\begin{eqnarray*}
& |h| = |f|+|g|;&  \\
& h_n = f_n & \mbox{\ for\ } n < |f|;  \\
& h_{|f|+n} = g_n & \mbox{\ for\ } n < |g|.
\end{eqnarray*}

\subsection{Constants, Variables, and Expressions}

A formal system has a set of {\em symbols}\index{symbol!in
a formal system} denoted
by $\mbox{\em SM}$.  A
precise set-theo\-ret\-i\-cal definition of this set is unimportant; a symbol
could be considered a primitive or atomic element if we wish.  We assume this
set is divided into two disjoint subsets:  a set $\mbox{\em CN}$ of {\em
constants}\index{constant!in a formal system} and a set $\mbox{\em VR}$ of
{\em variables}.\index{variable!in a formal system}  $\mbox{\em CN}$ and
$\mbox{\em VR}$ are each assumed to consist of countably many symbols which
may be arranged in finite or simple infinite sequences $c_0, c_1, \ldots$ and
$v_0, v_1, \ldots$ respectively, without repeating terms.  We will represent
arbitrary symbols by metavariables $\alpha$, $\beta$, etc.

{\footnotesize\begin{quotation}
{\em Comment.} The variables $v_0, v_1, \ldots$ of our formal system
correspond to what are usually considered ``metavariables'' in
descriptions of specific formal systems in the literature.  Typically,
when describing a specific formal system a book will postulate a set of
primitive objects called variables, then proceed to describe their
properties using metavariables that range over them, never mentioning
again the actual variables themselves.  Our formal system does not
mention these primitive variable objects at all but deals directly with
metavariables, as its primitive objects, from the start.  This is a
subtle but key distinction you should keep in mind, and it makes our
definition of ``formal system'' somewhat different from that typically
found in the literature.  (So, the $\alpha$, $\beta$, etc.\ above are
actually ``metametavariables'' when used to represent $v_0, v_1,
\ldots$.)
\end{quotation}}

Finite sequences all terms of which are symbols are called {\em
expressions}.\index{expression!in a formal system}  $\mbox{\em EX}$ is
the set of all expressions; thus
\begin{displaymath}
\mbox{\em EX} = \bigcup _{n \in \omega} {}^n \mbox{\em SM}.
\end{displaymath}

A {\em constant-prefixed expression}\index{constant-prefixed expression}
is an expression of non-zero length
whose first term is a constant.  We denote the set of all constant-prefixed
expressions by $\mbox{\em EX}_C = \{ e \in \mbox{\em EX} | ( |e| > 0 \wedge
e_0 \in \mbox{\em CN} ) \}$.

A {\em constant-variable pair}\index{constant-variable pair}
is an expression of length 2 whose first term
is a constant and whose second term is a variable.  We denote the set of all
constant-variable pairs by $\mbox{\em EX}_2 = \{ e \in \mbox{\em EX}_C | ( |e|
= 2 \wedge e_1 \in \mbox{\em VR} ) \}$.


{\footnotesize\begin{quotation}
{\em Relationship to Metamath.} In general, the set $\mbox{\em SM}$
corresponds to the set of declared math symbols in a Metamath database, the
set $\mbox{\em CN}$ to those declared with \texttt{\$c} statements, and the set
$\mbox{\em VR}$ to those declared with \texttt{\$v} statements.  Of course a
Metamath database can only have a finite number of math symbols, whereas
formal systems in general can have an infinite number, although the number of
Metamath math symbols available is in principle unlimited.

The set $\mbox{\em EX}_C$ corresponds to the set of permissible expressions
for \texttt{\$e}, \texttt{\$a}, and \texttt{\$p} statements.  The set $\mbox{\em EX}_2$
corresponds to the set of permissible expressions for \texttt{\$f} statements.
\end{quotation}}

We denote by ${\cal V}(e)$ the set of all variables in an expression $e \in
\mbox{\em EX}$, i.e.\ the set of all $\alpha \in \mbox{\em VR}$ such that
$\alpha = e_n$ for some $n < |e|$.  We also denote (with abuse of notation) by
${\cal V}(E)$ the set of all variables in a collection of expressions $E
\subseteq \mbox{\em EX}$, i.e.\ $\bigcup _{e \in E} {\cal V}(e)$.


\subsection{Substitution}

Given a function $F$ from $\mbox{\em VR}$ to
$\mbox{\em EX}$, we
denote by $\sigma_{F}$ or just $\sigma$ the function from $\mbox{\em EX}$ to
$\mbox{\em EX}$ defined recursively for nonempty sequences by
\begin{eqnarray*}
& \sigma(<\alpha>) = F(\alpha) & \mbox{for\ } \alpha \in \mbox{\em VR}; \\
& \sigma(<\alpha>) = <\alpha> & \mbox{for\ } \alpha \not\in \mbox{\em VR}; \\
& \sigma(g \frown h) = \sigma(g) \frown
    \sigma(h) & \mbox{for\ } g,h \in \mbox{\em EX}.
\end{eqnarray*}
We also define $\sigma(\varnothing)=\varnothing$.  We call $\sigma$ a {\em
simultaneous substitution}\index{substitution!variable}\index{variable
substitution} (or just {\em substitution}) with {\em substitution
map}\index{substitution map} $F$.

We also denote (with abuse of notation) by $\sigma(E)$ a substitution on a
collection of expressions $E \subseteq \mbox{\em EX}$, i.e.\ the set $\{
\sigma(e) | e \in E \}$.  The collection $\sigma(E)$ may of course contain
fewer expressions than $E$ because duplicate expressions could result from the
substitution.

\subsection{Statements}

We denote by $\mbox{\em DV}$ the set of all
unordered pairs $\{\alpha, \beta \} \subseteq \mbox{\em VR}$ such that $\alpha
\neq \beta$.  $\mbox{\em DV}$ stands for ``distinct variables.''

A {\em pre-statement}\index{pre-statement!in a formal system} is a
quadruple $\langle D,T,H,A \rangle$ such that
$D\subseteq \mbox{\em DV}$, $T\subseteq \mbox{\em EX}_2$, $H\subseteq
\mbox{\em EX}_C$ and $H$ is finite,
$A\in \mbox{\em EX}_C$, ${\cal V}(H\cup\{A\}) \subseteq
{\cal V}(T)$, and $\forall e,f\in T {\ } {\cal V}(e) \neq {\cal V}(f)$ (or
equivalently, $e_1 \ne f_1$) whenever $e \neq f$. The terms of the quadruple are called {\em
distinct-variable restrictions},\index{disjoint-variable restriction!in a
formal system} {\em variable-type hypotheses},\index{variable-type
hypothesis!in a formal system} {\em logical hypotheses},\index{logical
hypothesis!in a formal system} and the {\em assertion}\index{assertion!in a
formal system} respectively.  We denote by $T_M$ ({\em mandatory variable-type
hypotheses}\index{mandatory variable-type hypothesis!in a formal system}) the
subset of $T$ such that ${\cal V}(T_M) ={\cal V}(H \cup \{A\})$.  We denote by
$D_M=\{\{\alpha,\beta\}\in D|\{\alpha,\beta\}\subseteq {\cal V}(T_M)\}$ the
{\em mandatory distinct-variable restrictions}\index{mandatory
disjoint-variable restriction!in a formal system} of the pre-statement.
The set
of {\em mandatory hypotheses}\index{mandatory hypothesis!in a formal system}
is $T_M\cup H$.  We call the quadruple $\langle D_M,T_M,H,A \rangle$
the {\em reduct}\index{reduct!in a formal system} of
the pre-statement $\langle D,T,H,A \rangle$.

A {\em statement} is the reduct of some pre-statement\index{statement!in a
formal system}.  A statement is therefore a special kind of pre-statement;
in particular, a statement is the reduct of itself.

{\footnotesize\begin{quotation}
{\em Comment.}  $T$ is a set of expressions, each of length 2, that associate
a set of constants (``variable types'') with a set of variables.  The
condition ${\cal V}(H\cup\{A\}) \subseteq {\cal V}(T) $
means that each variable occurring in a statement's logical
hypotheses or assertion must have an associated variable-type hypothesis or
``type declaration,'' in  analogy to a computer programming language, where a
variable must be declared to be say, a string or an integer.  The requirement
that $\forall e,f\in T \, e_1 \ne f_1$ for $e\neq f$
means that each variable must be
associated with a unique constant designating its variable type; e.g., a
variable might be a ``wff'' or a ``set'' but not both.

Distinct-variable restrictions are used to specify what variable substitutions
are permissible to make for the statement to remain valid.  For example, in
the theorem scheme of set theory $\lnot\forall x\,x=y$ we may not substitute
the same variable for both $x$ and $y$.  On the other hand, the theorem scheme
$x=y\to y=x$ does not require that $x$ and $y$ be distinct, so we do not
require a distinct-variable restriction, although having one
would cause no harm other than making the scheme less general.

A mandatory variable-type hypothesis is one whose variable exists in a logical
hypothesis or the assertion.  A provable pre-statement
(defined below) may require
non-mandatory variable-type hypotheses that effectively introduce ``dummy''
variables for use in its proof.  Any number of dummy variables might
be required by a specific proof; indeed, it has been shown by H.\
Andr\'{e}ka\index{Andr{\'{e}}ka, H.} \cite{Nemeti} that there is no finite
upper bound to the number of dummy variables needed to prove an arbitrary
theorem in first-order logic (with equality) having a fixed number $n>2$ of
individual variables.  (See also the Comment on p.~\pageref{nodd}.)
For this reason we do not set a finite size bound on the collections $D$ and
$T$, although in an actual application (Metamath database) these will of
course be finite, increased to whatever size is necessary as more
proofs are added.
\end{quotation}}

{\footnotesize\begin{quotation}
{\em Relationship to Metamath.} A pre-statement of a formal system
corresponds to an extended frame in a Metamath database
(Section~\ref{frames}).  The collections $D$, $T$, and $H$ correspond
respectively to the \texttt{\$d}, \texttt{\$f}, and \texttt{\$e}
statement collections in an extended frame.  The expression $A$
corresponds to the \texttt{\$a} (or \texttt{\$p}) statement in an
extended frame.

A statement of a formal system corresponds to a frame in a Metamath
database.
\end{quotation}}

\subsection{Formal Systems}

A {\em formal system}\index{formal system} is a
triple $\langle \mbox{\em CN},\mbox{\em
VR},\Gamma\rangle$ where $\Gamma$ is a set of statements.  The members of
$\Gamma$ are called {\em axiomatic statements}.\index{axiomatic
statement!in a formal system}  Sometimes we will refer to a
formal system by just $\Gamma$ when $\mbox{\em CN}$ and $\mbox{\em VR}$ are
understood.

Given a formal system $\Gamma$, the {\em closure}\index{closure}\footnote{This
definition of closure incorporates a simplification due to
Josh Purinton.\index{Purinton, Josh}.} of a
pre-statement
$\langle D,T,H,A \rangle$ is the smallest set $C$ of expressions
such that:
%\begin{enumerate}
%  \item $T\cup H\subseteq C$; and
%  \item If for some axiomatic statement
%    $\langle D_M',T_M',H',A' \rangle \in \Gamma_A$, for
%    some $E \subseteq C$, some $F \subseteq C-T$ (where ``-'' denotes
%    set difference), and some substitution
%    $\sigma$ we have
%    \begin{enumerate}
%       \item $\sigma(T_M') = E$ (where, as above, the $M$ denotes the
%           mandatory variable-type hypotheses of $T^A$);
%       \item $\sigma(H') = F$;
%       \item for all $\{\alpha,\beta\}\in D^A$ and $\subseteq
%         {\cal V}(T_M')$, for all $\gamma\in {\cal V}(\sigma(\langle \alpha
%         \rangle))$, and for all $\delta\in  {\cal V}(\sigma(\langle \beta
%         \rangle))$, we have $\{\gamma, \delta\} \in D$;
%   \end{enumerate}
%   then $\sigma(A') \in C$.
%\end{enumerate}
\begin{list}{}{\itemsep 0.0pt}
  \item[1.] $T\cup H\subseteq C$; and
  \item[2.] If for some axiomatic statement
    $\langle D_M',T_M',H',A' \rangle \in
       \Gamma$ and for some substitution
    $\sigma$ we have
    \begin{enumerate}
       \item[a.] $\sigma(T_M' \cup H') \subseteq C$; and
       \item[b.] for all $\{\alpha,\beta\}\in D_M'$, for all $\gamma\in
         {\cal V}(\sigma(\langle \alpha
         \rangle))$, and for all $\delta\in  {\cal V}(\sigma(\langle \beta
         \rangle))$, we have $\{\gamma, \delta\} \in D$;
   \end{enumerate}
   then $\sigma(A') \in C$.
\end{list}
A pre-statement $\langle D,T,H,A
\rangle$ is {\em provable}\index{provable statement!in a formal
system} if $A\in C$ i.e.\ if its assertion belongs to its
closure.  A statement is {\em provable} if it is
the reduct of a provable pre-statement.
The {\em universe}\index{universe of a formal system}
of a formal system is
the collection of all of its provable statements.  Note that the
set of axiomatic statements $\Gamma$ in a formal system is a subset of its
universe.

{\footnotesize\begin{quotation}
{\em Comment.} The first condition in the definition of closure simply says
that the hypotheses of the pre-statement are in its closure.

Condition 2(a) says that a substitution exists that makes the
mandatory hypotheses of an axiomatic statement exactly match some members of
the closure.  This is what we explicitly demonstrate in a Metamath language
proof.

%Conditions 2(a) and 2(b) say that a substitution exists that makes the
%(mandatory) hypotheses of an axiomatic statement exactly match some members of
%the closure.  This is what we explicitly demonstrate with a Metamath language
%proof.
%
%The set of expressions $F$ in condition 2(b) excludes the variable-type
%hypotheses; this is done because non-mandatory variable-type hypotheses are
%effectively ``dropped'' as irrelevant whereas logical hypotheses must be
%retained to achieve a consistent logical system.

Condition 2(b) describes how distinct-variable restrictions in the axiomatic
statement must be met.  It means that after a substitution for two variables
that must be distinct, the resulting two expressions must either contain no
variables, or if they do, they may not have variables in common, and each pair
of any variables they do have, with one variable from each expression, must be
specified as distinct in the original statement.
\end{quotation}}

{\footnotesize\begin{quotation}
{\em Relationship to Metamath.} Axiomatic statements
 and provable statements in a formal
system correspond to the frames for \texttt{\$a} and \texttt{\$p} statements
respectively in a Metamath database.  The set of axiomatic statements is a
subset of the set of provable statements in a formal system, although in a
Metamath database a \texttt{\$a} statement is distinguished by not having a
proof.  A Metamath language proof for a \texttt{\$p} statement tells the computer
how to explicitly construct a series of members of the closure ultimately
leading to a demonstration that the assertion
being proved is in the closure.  The actual closure typically contains
an infinite number of expressions.  A formal system itself does not have
an explicit object called a ``proof'' but rather the existence of a proof
is implied indirectly by membership of an assertion in a provable
statement's closure.  We do this to make the formal system easier
to describe in the language of set theory.

We also note that once established as provable, a statement may be considered
to acquire the same status as an axiomatic statement, because if the set of
axiomatic statements is extended with a provable statement, the universe of
the formal system remains unchanged (provided that $\mbox{\em VR}$ is
infinite).
In practice, this means we can build a hierarchy of provable statements to
more efficiently establish additional provable statements.  This is
what we do in Metamath when we allow proofs to reference previous
\texttt{\$p} statements as well as previous \texttt{\$a} statements.
\end{quotation}}

\section{Examples of Formal Systems}

{\footnotesize\begin{quotation}
{\em Relationship to Metamath.} The examples in this section, except Example~2,
are for the most part exact equivalents of the development in the set
theory database \texttt{set.mm}.  You may want to compare Examples~1, 3, and 5
to Section~\ref{metaaxioms}, Example 4 to Sections~\ref{metadefprop} and
\ref{metadefpred}, and Example 6 to
Section~\ref{setdefinitions}.\label{exampleref}
\end{quotation}}

\subsection{Example~1---Propositional Calculus}\index{propositional calculus}

Classical propositional calculus can be described by the following formal
system.  We assume the set of variables is infinite.  Rather than denoting the
constants and variables by $c_0, c_1, \ldots$ and $v_0, v_1, \ldots$, for
readability we will instead use more conventional symbols, with the
understanding of course that they denote distinct primitive objects.
Also for readability we may omit commas between successive terms of a
sequence; thus $\langle \mbox{wff\ } \varphi\rangle$ denotes
$\langle \mbox{wff}, \varphi\rangle$.

Let
\begin{itemize}
  \item[] $\mbox{\em CN}=\{\mbox{wff}, \vdash, \to, \lnot, (,)\}$
  \item[] $\mbox{\em VR}=\{\varphi,\psi,\chi,\ldots\}$
  \item[] $T = \{\langle \mbox{wff\ } \varphi\rangle,
             \langle \mbox{wff\ } \psi\rangle,
             \langle \mbox{wff\ } \chi\rangle,\ldots\}$, i.e.\ those
             expressions of length 2 whose first member is $\mbox{\rm wff}$
             and whose second member belongs to $\mbox{\em VR}$.\footnote{For
convenience we let $T$ be an infinite set; the definition of a statement
permits this in principle.  Since a Metamath source file has a finite size, in
practice we must of course use appropriate finite subsets of this $T$,
specifically ones containing at least the mandatory variable-type
hypotheses.  Similarly, in the source file we introduce new variables as
required, with the understanding that a potentially infinite number of
them are available.}
\noindent Then $\Gamma$ consists of the axiomatic statements that
are the reducts of the following pre-statements:
    \begin{itemize}
      \item[] $\langle\varnothing,T,\varnothing,
               \langle \mbox{wff\ }(\varphi\to\psi)\rangle\rangle$
      \item[] $\langle\varnothing,T,\varnothing,
               \langle \mbox{wff\ }\lnot\varphi\rangle\rangle$
      \item[] $\langle\varnothing,T,\varnothing,
               \langle \vdash(\varphi\to(\psi\to\varphi))
               \rangle\rangle$
      \item[] $\langle\varnothing,T,
               \varnothing,
               \langle \vdash((\varphi\to(\psi\to\chi))\to
               ((\varphi\to\psi)\to(\varphi\to\chi)))
               \rangle\rangle$
      \item[] $\langle\varnothing,T,
               \varnothing,
               \langle \vdash((\lnot\varphi\to\lnot\psi)\to
               (\psi\to\varphi))\rangle\rangle$
      \item[] $\langle\varnothing,T,
               \{\langle\vdash(\varphi\to\psi)\rangle,
                 \langle\vdash\varphi\rangle\},
               \langle\vdash\psi\rangle\rangle$
    \end{itemize}
\end{itemize}

(For example, the reduct of $\langle\varnothing,T,\varnothing,
               \langle \mbox{wff\ }(\varphi\to\psi)\rangle\rangle$
is
\begin{itemize}
\item[] $\langle\varnothing,
\{\langle \mbox{wff\ } \varphi\rangle,
             \langle \mbox{wff\ } \psi\rangle\},
             \varnothing,
               \langle \mbox{wff\ }(\varphi\to\psi)\rangle\rangle$,
\end{itemize}
which is the first axiomatic statement.)

We call the members of $\mbox{\em VR}$ {\em wff variables} or (in the context
of first-order logic which we will describe shortly) {\em wff metavariables}.
Note that the symbols $\phi$, $\psi$, etc.\ denote actual specific members of
$\mbox{\em VR}$; they are not metavariables of our expository language (which
we denote with $\alpha$, $\beta$, etc.) but are instead (meta)constant symbols
(members of $\mbox{\em SM}$) from the point of view of our expository
language.  The equivalent system of propositional calculus described in
\cite{Tarski1965} also uses the symbols $\phi$, $\psi$, etc.\ to denote wff
metavariables, but in \cite{Tarski1965} unlike here those are metavariables of
the expository language and not primitive symbols of the formal system.

The first two statements define wffs: if $\varphi$ and $\psi$ are wffs, so is
$(\varphi \to \psi)$; if $\varphi$ is a wff, so is $\lnot\varphi$. The next
three are the axioms of propositional calculus: if $\varphi$ and $\psi$ are
wffs, then $\vdash (\varphi \to (\psi \to \varphi))$ is an (axiomatic)
theorem; etc. The
last is the rule of modus ponens: if $\varphi$ and $\psi$ are wffs, and
$\vdash (\varphi\to\psi)$ and $\vdash \varphi$ are theorems, then $\vdash
\psi$ is a theorem.

The correspondence to ordinary propositional calculus is as follows.  We
consider only provable statements of the form $\langle\varnothing,
T,\varnothing,A\rangle$ with $T$ defined as above.  The first term of the
assertion $A$ of any such statement is either ``wff'' or ``$\vdash$''.  A
statement for which the first term is ``wff'' is a {\em wff} of propositional
calculus, and one where the first term is ``$\vdash$'' is a {\em
theorem (scheme)} of propositional calculus.

The universe of this formal system also contains many other provable
statements.  Those with distinct-variable restrictions are irrelevant because
propositional calculus has no constraints on substitutions.  Those that have
logical hypotheses we call {\em inferences}\index{inference} when
the logical hypotheses are of the form
$\langle\vdash\rangle\frown w$ where $w$ is a wff (with the leading constant
term ``wff'' removed).  Inferences (other than the modus ponens rule) are not a
proper part of propositional calculus but are convenient to use when building a
hierarchy of provable statements.  A provable statement with a nonsense
hypothesis such as $\langle \to,\vdash,\lnot\rangle$, and this same expression
as its assertion, we consider irrelevant; no use can be made of it in
proving theorems, since there is no way to eliminate the nonsense hypothesis.

{\footnotesize\begin{quotation}
{\em Comment.} Our use of parentheses in the definition of a wff illustrates
how axiomatic statements should be carefully stated in a way that
ties in unambiguously with the substitutions allowed by the formal system.
There are many ways we could have defined wffs---for example, Polish
prefix notation would have allowed us to omit parentheses entirely, at
the expense of readability---but we must define them in a way that is
unambiguous.  For example, if we had omitted parentheses from the
definition of $(\varphi\to \psi)$, the wff $\lnot\varphi\to \psi$ could
be interpreted as either $\lnot(\varphi\to\psi)$ or $(\lnot\varphi\to\psi)$
and would have allowed us to prove nonsense.  Note that there is no
concept of operator binding precedence built into our formal system.
\end{quotation}}

\begin{sloppy}
\subsection{Example~2---Predicate Calculus with Equality}\index{predicate
calculus}
\end{sloppy}

Here we extend Example~1 to include predicate calculus with equality,
illustrating the use of distinct-variable restrictions.  This system is the
same as Tarski's system $\mathfrak{S}_2$ in \cite{Tarski1965} (except that the
axioms of propositional calculus are different but equivalent, and a redundant
axiom is omitted).  We extend $\mbox{\em CN}$ with the constants
$\{\mbox{var},\forall,=\}$.  We extend $\mbox{\em VR}$ with an infinite set of
{\em individual metavariables}\index{individual
metavariable} $\{x,y,z,\ldots\}$ and denote this subset
$\mbox{\em Vr}$.

We also join to $\mbox{\em CN}$ a possibly infinite set $\mbox{\em Pr}$ of {\em
predicates} $\{R,S,\ldots\}$.  We associate with $\mbox{\em Pr}$ a function
$\mbox{rnk}$ from $\mbox{\em Pr}$ to $\omega$, and for $\alpha\in \mbox{\em
Pr}$ we call $\mbox{rnk}(\alpha)$ the {\em rank} of the predicate $\alpha$,
which is simply the number of ``arguments'' that the predicate has.  (Most
applications of predicate calculus will have a finite number of predicates;
for example, set theory has the single two-argument or binary predicate $\in$,
which is usually written with its arguments surrounding the predicate symbol
rather than with the prefix notation we will use for the general case.)  As a
device to facilitate our discussion, we will let $\mbox{\em Vs}$ be any fixed
one-to-one function from $\omega$ to $\mbox{\em Vr}$; thus $\mbox{\em Vs}$ is
any simple infinite sequence of individual metavariables with no repeating
terms.

In this example we will not include the function symbols that are often part of
formalizations of predicate calculus.  Using metalogical arguments that are
beyond the scope of our discussion, it can be shown that our formalization is
equivalent when functions are introduced via appropriate definitions.

We extend the set $T$ defined in Example~1 with the expressions
$\{\langle \mbox{var\ } x\rangle,$ $ \langle \mbox{var\ } y\rangle, \langle
\mbox{var\ } z\rangle,\ldots\}$.  We extend the $\Gamma$ above
with the axiomatic statements that are the reducts of the following
pre-statements:
\begin{list}{}{\itemsep 0.0pt}
      \item[] $\langle\varnothing,T,\varnothing,
               \langle \mbox{wff\ }\forall x\,\varphi\rangle\rangle$
      \item[] $\langle\varnothing,T,\varnothing,
               \langle \mbox{wff\ }x=y\rangle\rangle$
      \item[] $\langle\varnothing,T,
               \{\langle\vdash\varphi\rangle\},
               \langle\vdash\forall x\,\varphi\rangle\rangle$
      \item[] $\langle\varnothing,T,\varnothing,
               \langle \vdash((\forall x(\varphi\to\psi)
                  \to(\forall x\,\varphi\to\forall x\,\psi))
               \rangle\rangle$
      \item[] $\langle\{\{x,\varphi\}\},T,\varnothing,
               \langle \vdash(\varphi\to\forall x\,\varphi)
               \rangle\rangle$
      \item[] $\langle\{\{x,y\}\},T,\varnothing,
               \langle \vdash\lnot\forall x\lnot x=y
               \rangle\rangle$
      \item[] $\langle\varnothing,T,\varnothing,
               \langle \vdash(x=z
                  \to(x=y\to z=y))
               \rangle\rangle$
      \item[] $\langle\varnothing,T,\varnothing,
               \langle \vdash(y=z
                  \to(x=y\to x=z))
               \rangle\rangle$
\end{list}
These are the axioms not involving predicate symbols. The first two statements
extend the definition of a wff.  The third is the rule of generalization.  The
fifth states, in effect, ``For a wff $\varphi$ and variable $x$,
$\vdash(\varphi\to\forall x\,\varphi)$, provided that $x$ does not occur in
$\varphi$.''  The sixth states ``For variables $x$ and $y$,
$\vdash\lnot\forall x\lnot x = y$, provided that $x$ and $y$ are distinct.''
(This proviso is not necessary but was included by Tarski to
weaken the axiom and still show that the system is logically complete.)

Finally, for each predicate symbol $\alpha\in \mbox{\em Pr}$, we add to
$\Gamma$ an axiomatic statement, extending the definition of wff,
that is the reduct of the following pre-statement:
\begin{displaymath}
    \langle\varnothing,T,\varnothing,
            \langle \mbox{wff},\alpha\rangle\
            \frown \mbox{\em Vs}\restriction\mbox{rnk}(\alpha)\rangle
\end{displaymath}
and for each $\alpha\in \mbox{\em Pr}$ and each $n < \mbox{rnk}(\alpha)$
we add to $\Gamma$ an equality axiom that is the reduct of the
following pre-statement:
\begin{eqnarray*}
    \lefteqn{\langle\varnothing,T,\varnothing,
            \langle
      \vdash,(,\mbox{\em Vs}_n,=,\mbox{\em Vs}_{\mbox{rnk}(\alpha)},\to,
            (,\alpha\rangle\frown \mbox{\em Vs}\restriction\mbox{rnk}(\alpha)} \\
  & & \frown
            \langle\to,\alpha\rangle\frown \mbox{\em Vs}\restriction n\frown
            \langle \mbox{\em Vs}_{\mbox{rnk}(\alpha)}\rangle \\
 & & \frown
            \mbox{\em Vs}\restriction(\mbox{rnk}(\alpha)\setminus(n+1))\frown
            \langle),)\rangle\rangle
\end{eqnarray*}
where $\restriction$ denotes function domain restriction and $\setminus$
denotes set difference.  Recall that a subscript on $\mbox{\em Vs}$
denotes one of its terms.  (In the above two axiom sets commas are placed
between successive terms of sequences to prevent ambiguity, and if you examine
them with care you will be able to distinguish those parentheses that denote
constant symbols from those of our expository language that delimit function
arguments.  Although it might have been better to use boldface for our
primitive symbols, unfortunately boldface was not available for all characters
on the \LaTeX\ system used to typeset this text.)  These seemingly forbidding
axioms can be understood by analogy to concatenation of substrings in a
computer language.  They are actually relatively simple for each specific case
and will become clearer by looking at the special case of a binary predicate
$\alpha = R$ where $\mbox{rnk}(R)=2$.  Letting $\mbox{\em Vs}$ be the sequence
$\langle x,y,z,\ldots\rangle$, the axioms we would add to $\Gamma$ for this
case would be the wff extension and two equality axioms that are the
reducts of the pre-statements:
\begin{list}{}{\itemsep 0.0pt}
      \item[] $\langle\varnothing,T,\varnothing,
               \langle \mbox{wff\ }R x y\rangle\rangle$
      \item[] $\langle\varnothing,T,\varnothing,
               \langle \vdash(x=z
                  \to(R x y \to R z y))
               \rangle\rangle$
      \item[] $\langle\varnothing,T,\varnothing,
               \langle \vdash(y=z
                  \to(R x y \to R x z))
               \rangle\rangle$
\end{list}
Study these carefully to see how the general axioms above evaluate to
them.  In practice, typically only a few special cases such as this would be
needed, and in any case the Metamath language will only permit us to describe
a finite number of predicates, as opposed to the infinite number permitted by
the formal system.  (If an infinite number should be needed for some reason,
we could not define the formal system directly in the Metamath language but
could instead define it metalogically under set theory as we
do in this appendix, and only the underlying set theory, with its single
binary predicate, would be defined directly in the Metamath language.)


{\footnotesize\begin{quotation}
{\em Comment.}  As we noted earlier, the specific variables denoted by the
symbols $x,y,z,\ldots\in \mbox{\em Vr}\subseteq \mbox{\em VR}\subseteq
\mbox{\em SM}$ in Example~2 are not the actual variables of ordinary predicate
calculus but should be thought of as metavariables ranging over them.  For
example, a distinct-variable restriction would be meaningless for actual
variables of ordinary predicate calculus since two different actual variables
are by definition distinct.  And when we talk about an arbitrary
representative $\alpha\in \mbox{\em Vr}$, $\alpha$ is a metavariable (in our
expository language) that ranges over metavariables (which are primitives of
our formal system) each of which ranges over the actual individual variables
of predicate calculus (which are never mentioned in our formal system).

The constant called ``var'' above is called \texttt{setvar} in the
\texttt{set.mm} database file, but it means the same thing.  I felt
that ``var'' is a more meaningful name in the context of predicate
calculus, whose use is not limited to set theory.  For consistency we
stick with the name ``var'' throughout this Appendix, even after set
theory is introduced.
\end{quotation}}

\subsection{Free Variables and Proper Substitution}\index{free variable}
\index{proper substitution}\index{substitution!proper}

Typical representations of mathematical axioms use concepts such
as ``free variable,'' ``bound variable,'' and ``proper substitution''
as primitive notions.
A free variable is a variable that
is not a parameter of any container expression.
A bound variable is the opposite of a free variable; it is a
a variable that has been bound in a container expression.
For example, in the expression $\forall x \varphi$ (for all $x$, $\varphi$
is true), the variable $x$
is bound within the for-all ($\forall$) expression.
It is possible to change one variable to another, and that process is called
``proper substitution.''
In most books, proper substitution has a somewhat complicated recursive
definition with multiple cases based on the occurrences of free and
bound variables.
You may consult
\cite[ch.\ 3--4]{Hamilton}\index{Hamilton, Alan G.} (as well as
many other texts) for more formal details about these terms.

Using these concepts as \texttt{primitives} creates complications
for computer implementations.

In the system of Example~2, there are no primitive notions of free variable
and proper substitution.  Tarski \cite{Tarski1965} shows that this system is
logically equivalent to the more typical textbook systems that do have these
primitive notions, if we introduce these notions with appropriate definitions
and metalogic.  We could also define axioms for such systems directly,
although the recursive definitions of free variable and proper substitution
would be messy and awkward to work with.  Instead, we mention two devices that
can be used in practice to mimic these notions.  (1) Instead of introducing
special notation to express (as a logical hypothesis) ``where $x$ is not free
in $\varphi$'' we can use the logical hypothesis $\vdash(\varphi\to\forall
x\,\varphi)$.\label{effectivelybound}\index{effectively
not free}\footnote{This is a slightly weaker requirement than ``where $x$ is
not free in $\varphi$.''  If we let $\varphi$ be $x=x$, we have the theorem
$(x=x\to\forall x\,x=x)$ which satisfies the hypothesis, even though $x$ is
free in $x=x$ .  In a case like this we say that $x$ is {\em effectively not
free}\index{effectively not free} in $x=x$, since $x=x$ is logically
equivalent to $\forall x\,x=x$ in which $x$ is bound.} (2) It can be shown
that the wff $((x=y\to\varphi)\wedge\exists x(x=y\wedge\varphi))$ (with the
usual definitions of $\wedge$ and $\exists$; see Example~4 below) is logically
equivalent to ``the wff that results from proper substitution of $y$ for $x$
in $\varphi$.''  This works whether or not $x$ and $y$ are distinct.

\subsection{Metalogical Completeness}\index{metalogical completeness}

In the system of Example~2, the
following are provable pre-statements (and their reducts are
provable statements):
\begin{eqnarray*}
      & \langle\{\{x,y\}\},T,\varnothing,
               \langle \vdash\lnot\forall x\lnot x=y
               \rangle\rangle & \\
     &  \langle\varnothing,T,\varnothing,
               \langle \vdash\lnot\forall x\lnot x=x
               \rangle\rangle &
\end{eqnarray*}
whereas the following pre-statement is not to my knowledge provable (but
in any case we will pretend it's not for sake of illustration):
\begin{eqnarray*}
     &  \langle\varnothing,T,\varnothing,
               \langle \vdash\lnot\forall x\lnot x=y
               \rangle\rangle &
\end{eqnarray*}
In other words, we can prove ``$\lnot\forall x\lnot x=y$ where $x$ and $y$ are
distinct'' and separately prove ``$\lnot\forall x\lnot x=x$'', but we can't
prove the combined general case ``$\lnot\forall x\lnot x=y$'' that has no
proviso.  Now this does not compromise logical completeness, because the
variables are really metavariables and the two provable cases together cover
all possible cases.  The third case can be considered a metatheorem whose
direct proof, using the system of Example~2, lies outside the capability of the
formal system.

Also, in the system of Example~2 the following pre-statement is not to my
knowledge provable (again, a conjecture that we will pretend to be the case):
\begin{eqnarray*}
     & \langle\varnothing,T,\varnothing,
               \langle \vdash(\forall x\, \varphi\to\varphi)
               \rangle\rangle &
\end{eqnarray*}
Instead, we can only prove specific cases of $\varphi$ involving individual
metavariables, and by induction on formula length, prove as a metatheorem
outside of our formal system the general statement above.  The details of this
proof are found in \cite{Kalish}.

There does, however, exist a system of predicate calculus in which all such
``simple metatheorems'' as those above can be proved directly, and we present
it in Example~3. A {\em simple metatheorem}\index{simple metatheorem}
is any statement of the formal
system of Example~2 where all distinct variable restrictions consist of either
two individual metavariables or an individual metavariable and a wff
metavariable, and which is provable by combining cases outside the system as
above.  A system is {\em metalogically complete}\index{metalogical
completeness} if all of its simple
metatheorems are (directly) provable statements. The precise definition of
``simple metatheorem'' and the proof of the ``metalogical completeness'' of
Example~3 is found in Remark 9.6 and Theorem 9.7 of \cite{Megill}.\index{Megill,
Norman}

\begin{sloppy}
\subsection{Example~3---Metalogically Complete Predicate
Calculus with
Equality}
\end{sloppy}

For simplicity we will assume there is one binary predicate $R$;
this system suffices for set theory, where the $R$ is of course the $\in$
predicate.  We label the axioms as they appear in \cite{Megill}.  This
system is logically equivalent to that of Example~2 (when the latter is
restricted to this single binary predicate) but is also metalogically
complete.\index{metalogical completeness}

Let
\begin{itemize}
  \item[] $\mbox{\em CN}=\{\mbox{wff}, \mbox{var}, \vdash, \to, \lnot, (,),\forall,=,R\}$.
  \item[] $\mbox{\em VR}=\{\varphi,\psi,\chi,\ldots\}\cup\{x,y,z,\ldots\}$.
  \item[] $T = \{\langle \mbox{wff\ } \varphi\rangle,
             \langle \mbox{wff\ } \psi\rangle,
             \langle \mbox{wff\ } \chi\rangle,\ldots\}\cup
       \{\langle \mbox{var\ } x\rangle, \langle \mbox{var\ } y\rangle, \langle
       \mbox{var\ }z\rangle,\ldots\}$.

\noindent Then
  $\Gamma$ consists of the reducts of the following pre-statements:
    \begin{itemize}
      \item[] $\langle\varnothing,T,\varnothing,
               \langle \mbox{wff\ }(\varphi\to\psi)\rangle\rangle$
      \item[] $\langle\varnothing,T,\varnothing,
               \langle \mbox{wff\ }\lnot\varphi\rangle\rangle$
      \item[] $\langle\varnothing,T,\varnothing,
               \langle \mbox{wff\ }\forall x\,\varphi\rangle\rangle$
      \item[] $\langle\varnothing,T,\varnothing,
               \langle \mbox{wff\ }x=y\rangle\rangle$
      \item[] $\langle\varnothing,T,\varnothing,
               \langle \mbox{wff\ }Rxy\rangle\rangle$
      \item[(C1$'$)] $\langle\varnothing,T,\varnothing,
               \langle \vdash(\varphi\to(\psi\to\varphi))
               \rangle\rangle$
      \item[(C2$'$)] $\langle\varnothing,T,
               \varnothing,
               \langle \vdash((\varphi\to(\psi\to\chi))\to
               ((\varphi\to\psi)\to(\varphi\to\chi)))
               \rangle\rangle$
      \item[(C3$'$)] $\langle\varnothing,T,
               \varnothing,
               \langle \vdash((\lnot\varphi\to\lnot\psi)\to
               (\psi\to\varphi))\rangle\rangle$
      \item[(C4$'$)] $\langle\varnothing,T,
               \varnothing,
               \langle \vdash(\forall x(\forall x\,\varphi\to\psi)\to
                 (\forall x\,\varphi\to\forall x\,\psi))\rangle\rangle$
      \item[(C5$'$)] $\langle\varnothing,T,
               \varnothing,
               \langle \vdash(\forall x\,\varphi\to\varphi)\rangle\rangle$
      \item[(C6$'$)] $\langle\varnothing,T,
               \varnothing,
               \langle \vdash(\forall x\forall y\,\varphi\to
                 \forall y\forall x\,\varphi)\rangle\rangle$
      \item[(C7$'$)] $\langle\varnothing,T,
               \varnothing,
               \langle \vdash(\lnot\varphi\to\forall x\lnot\forall x\,\varphi
                 )\rangle\rangle$
      \item[(C8$'$)] $\langle\varnothing,T,
               \varnothing,
               \langle \vdash(x=y\to(x=z\to y=z))\rangle\rangle$
      \item[(C9$'$)] $\langle\varnothing,T,
               \varnothing,
               \langle \vdash(\lnot\forall x\, x=y\to(\lnot\forall x\, x=z\to
                 (y=z\to\forall x\, y=z)))\rangle\rangle$
      \item[(C10$'$)] $\langle\varnothing,T,
               \varnothing,
               \langle \vdash(\forall x(x=y\to\forall x\,\varphi)\to
                 \varphi))\rangle\rangle$
      \item[(C11$'$)] $\langle\varnothing,T,
               \varnothing,
               \langle \vdash(\forall x\, x=y\to(\forall x\,\varphi
               \to\forall y\,\varphi))\rangle\rangle$
      \item[(C12$'$)] $\langle\varnothing,T,
               \varnothing,
               \langle \vdash(x=y\to(Rxz\to Ryz))\rangle\rangle$
      \item[(C13$'$)] $\langle\varnothing,T,
               \varnothing,
               \langle \vdash(x=y\to(Rzx\to Rzy))\rangle\rangle$
      \item[(C15$'$)] $\langle\varnothing,T,
               \varnothing,
               \langle \vdash(\lnot\forall x\, x=y\to(x=y\to(\varphi
                 \to\forall x(x=y\to\varphi))))\rangle\rangle$
      \item[(C16$'$)] $\langle\{\{x,y\}\},T,
               \varnothing,
               \langle \vdash(\forall x\, x=y\to(\varphi\to\forall x\,\varphi)
                 )\rangle\rangle$
      \item[(C5)] $\langle\{\{x,\varphi\}\},T,\varnothing,
               \langle \vdash(\varphi\to\forall x\,\varphi)
               \rangle\rangle$
      \item[(MP)] $\langle\varnothing,T,
               \{\langle\vdash(\varphi\to\psi)\rangle,
                 \langle\vdash\varphi\rangle\},
               \langle\vdash\psi\rangle\rangle$
      \item[(Gen)] $\langle\varnothing,T,
               \{\langle\vdash\varphi\rangle\},
               \langle\vdash\forall x\,\varphi\rangle\rangle$
    \end{itemize}
\end{itemize}

While it is known that these axioms are ``metalogically complete,'' it is
not known whether they are independent (i.e.\ none is
redundant) in the metalogical sense; specifically, whether any axiom (possibly
with additional non-mandatory distinct-variable restrictions, for use with any
dummy variables in its proof) is provable from the others.  Note that
metalogical independence is a weaker requirement than independence in the
usual logical sense.  Not all of the above axioms are logically independent:
for example, C9$'$ can be proved as a metatheorem from the others, outside the
formal system, by combining the possible cases of distinct variables.

\subsection{Example~4---Adding Definitions}\index{definition}
There are several ways to add definitions to a formal system.  Probably the
most proper way is to consider definitions not as part of the formal system at
all but rather as abbreviations that are part of the expository metalogic
outside the formal system.  For convenience, though, we may use the formal
system itself to incorporate definitions, adding them as axiomatic extensions
to the system.  This could be done by adding a constant representing the
concept ``is defined as'' along with axioms for it. But there is a nicer way,
at least in this writer's opinion, that introduces definitions as direct
extensions to the language rather than as extralogical primitive notions.  We
introduce additional logical connectives and provide axioms for them.  For
systems of logic such as Examples 1 through 3, the additional axioms must be
conservative in the sense that no wff of the original system that was not a
theorem (when the initial term ``wff'' is replaced by ``$\vdash$'' of course)
becomes a theorem of the extended system.  In this example we extend Example~3
(or 2) with standard abbreviations of logic.

We extend $\mbox{\em CN}$ of Example~3 with new constants $\{\leftrightarrow,
\wedge,\vee,\exists\}$, corresponding to logical equivalence,\index{logical
equivalence ($\leftrightarrow$)}\index{biconditional ($\leftrightarrow$)}
conjunction,\index{conjunction ($\wedge$)} disjunction,\index{disjunction
($\vee$)} and the existential quantifier.\index{existential quantifier
($\exists$)}  We extend $\Gamma$ with the axiomatic statements that are
the reducts of the following pre-statements:
\begin{list}{}{\itemsep 0.0pt}
      \item[] $\langle\varnothing,T,\varnothing,
               \langle \mbox{wff\ }(\varphi\leftrightarrow\psi)\rangle\rangle$
      \item[] $\langle\varnothing,T,\varnothing,
               \langle \mbox{wff\ }(\varphi\vee\psi)\rangle\rangle$
      \item[] $\langle\varnothing,T,\varnothing,
               \langle \mbox{wff\ }(\varphi\wedge\psi)\rangle\rangle$
      \item[] $\langle\varnothing,T,\varnothing,
               \langle \mbox{wff\ }\exists x\, \varphi\rangle\rangle$
  \item[] $\langle\varnothing,T,\varnothing,
     \langle\vdash ( ( \varphi \leftrightarrow \psi ) \to
     ( \varphi \to \psi ) )\rangle\rangle$
  \item[] $\langle\varnothing,T,\varnothing,
     \langle\vdash ((\varphi\leftrightarrow\psi)\to
    (\psi\to\varphi))\rangle\rangle$
  \item[] $\langle\varnothing,T,\varnothing,
     \langle\vdash ((\varphi\to\psi)\to(
     (\psi\to\varphi)\to(\varphi
     \leftrightarrow\psi)))\rangle\rangle$
  \item[] $\langle\varnothing,T,\varnothing,
     \langle\vdash (( \varphi \wedge \psi ) \leftrightarrow\neg ( \varphi
     \to \neg \psi )) \rangle\rangle$
  \item[] $\langle\varnothing,T,\varnothing,
     \langle\vdash (( \varphi \vee \psi ) \leftrightarrow (\neg \varphi
     \to \psi )) \rangle\rangle$
  \item[] $\langle\varnothing,T,\varnothing,
     \langle\vdash (\exists x \,\varphi\leftrightarrow
     \lnot \forall x \lnot \varphi)\rangle\rangle$
\end{list}
The first three logical axioms (statements containing ``$\vdash$'') introduce
and effectively define logical equivalence, ``$\leftrightarrow$''.  The last
three use ``$\leftrightarrow$'' to effectively mean ``is defined as.''

\subsection{Example~5---ZFC Set Theory}\index{ZFC set theory}

Here we add to the system of Example~4 the axioms of Zermelo--Fraenkel set
theory with Choice.  For convenience we make use of the
definitions in Example~4.

In the $\mbox{\em CN}$ of Example~4 (which extends Example~3), we replace the symbol $R$
with the symbol $\in$.
More explicitly, we remove from $\Gamma$ of Example~4 the three
axiomatic statements containing $R$ and replace them with the
reducts of the following:
\begin{list}{}{\itemsep 0.0pt}
      \item[] $\langle\varnothing,T,\varnothing,
               \langle \mbox{wff\ }x\in y\rangle\rangle$
      \item[] $\langle\varnothing,T,
               \varnothing,
               \langle \vdash(x=y\to(x\in z\to y\in z))\rangle\rangle$
      \item[] $\langle\varnothing,T,
               \varnothing,
               \langle \vdash(x=y\to(z\in x\to z\in y))\rangle\rangle$
\end{list}
Letting $D=\{\{\alpha,\beta\}\in \mbox{\em DV}\,|\alpha,\beta\in \mbox{\em
Vr}\}$ (in other words all individual variables must be distinct), we extend
$\Gamma$ with the ZFC axioms, called
\index{Axiom of Extensionality}
\index{Axiom of Replacement}
\index{Axiom of Union}
\index{Axiom of Power Sets}
\index{Axiom of Regularity}
\index{Axiom of Infinity}
\index{Axiom of Choice}
Extensionality, Replacement, Union, Power
Set, Regularity, Infinity, and Choice, that are the reducts of:
\begin{list}{}{\itemsep 0.0pt}
      \item[Ext] $\langle D,T,
               \varnothing,
               \langle\vdash (\forall x(x\in y\leftrightarrow x \in z)\to y
               =z) \rangle\rangle$
      \item[Rep] $\langle D,T,
               \varnothing,
               \langle\vdash\exists x ( \exists y \forall z (\varphi \to z = y
                        ) \to
                        \forall z ( z \in x \leftrightarrow \exists x ( x \in
                        y \wedge \forall y\,\varphi ) ) )\rangle\rangle$
      \item[Un] $\langle D,T,
               \varnothing,
               \langle\vdash \exists x \forall y ( \exists x ( y \in x \wedge
               x \in z ) \to y \in x ) \rangle\rangle$
      \item[Pow] $\langle D,T,
               \varnothing,
               \langle\vdash \exists x \forall y ( \forall x ( x \in y \to x
               \in z ) \to y \in x ) \rangle\rangle$
      \item[Reg] $\langle D,T,
               \varnothing,
               \langle\vdash (  x \in y \to
                 \exists x ( x \in y \wedge \forall z ( z \in x \to \lnot z
                \in y ) ) ) \rangle\rangle$
      \item[Inf] $\langle D,T,
               \varnothing,
               \langle\vdash \exists x(y\in x\wedge\forall y(y\in
               x\to
               \exists z(y \in z\wedge z\in x))) \rangle\rangle$
      \item[AC] $\langle D,T,
               \varnothing,
               \langle\vdash \exists x \forall y \forall z ( ( y \in z
               \wedge z \in w ) \to \exists w \forall y ( \exists w
              ( ( y \in z \wedge z \in w ) \wedge ( y \in w \wedge w \in x
              ) ) \leftrightarrow y = w ) ) \rangle\rangle$
\end{list}

\subsection{Example~6---Class Notation in Set Theory}\label{class}

A powerful device that makes set theory easier (and that we have
been using all along in our informal expository language) is {\em class
abstraction notation}.\index{class abstraction}\index{abstraction class}  The
definitions we introduce are rigorously justified
as conservative by Takeuti and Zaring \cite{Takeuti}\index{Takeuti, Gaisi} or
Quine \cite{Quine}\index{Quine, Willard Van Orman}.  The key idea is to
introduce the notation $\{x|\mbox{---}\}$ which means ``the class of all $x$
such that ---'' for abstraction classes and introduce (meta)variables that
range over them.  An abstraction class may or may not be a set, depending on
whether it exists (as a set).  A class that does not exist is
called a {\em proper class}.\index{proper class}\index{class!proper}

To illustrate the use of abstraction classes we will provide some examples
of definitions that make use of them:  the empty set, class union, and
unordered pair.  Many other such definitions can be found in the
Metamath set theory database,
\texttt{set.mm}.\index{set theory database (\texttt{set.mm})}

% We intentionally break up the sequence of math symbols here
% because otherwise the overlong line goes beyond the page in narrow mode.
We extend $\mbox{\em CN}$ of Example~5 with new symbols $\{$
$\mbox{class},$ $\{,$ $|,$ $\},$ $\varnothing,$ $\cup,$ $,$ $\}$
where the inner braces and last comma are
constant symbols. (As before,
our dual use of some mathematical symbols for both our expository
language and as primitives of the formal system should be clear from context.)

We extend $\mbox{\em VR}$ of Example~5 with a set of {\em class
variables}\index{class variable}
$\{A,B,C,\ldots\}$. We extend the $T$ of Example~5 with $\{\langle
\mbox{class\ } A\rangle, \langle \mbox{class\ }B\rangle, \langle \mbox{class\ }
C\rangle,\ldots\}$.

To
introduce our definitions,
we add to $\Gamma$ of Example~5 the axiomatic statements
that are the reducts of the following pre-statements:
\begin{list}{}{\itemsep 0.0pt}
      \item[] $\langle\varnothing,T,\varnothing,
               \langle \mbox{class\ }x\rangle\rangle$
      \item[] $\langle\varnothing,T,\varnothing,
               \langle \mbox{class\ }\{x|\varphi\}\rangle\rangle$
      \item[] $\langle\varnothing,T,\varnothing,
               \langle \mbox{wff\ }A=B\rangle\rangle$
      \item[] $\langle\varnothing,T,\varnothing,
               \langle \mbox{wff\ }A\in B\rangle\rangle$
      \item[Ab] $\langle\varnothing,T,\varnothing,
               \langle \vdash ( y \in \{ x |\varphi\} \leftrightarrow
                  ( ( x = y \to\varphi) \wedge \exists x ( x = y
                  \wedge\varphi) ))
               \rangle\rangle$
      \item[Eq] $\langle\{\{x,A\},\{x,B\}\},T,\varnothing,
               \langle \vdash ( A = B \leftrightarrow
               \forall x ( x \in A \leftrightarrow x \in B ) )
               \rangle\rangle$
      \item[El] $\langle\{\{x,A\},\{x,B\}\},T,\varnothing,
               \langle \vdash ( A \in B \leftrightarrow \exists x
               ( x = A \wedge x \in B ) )
               \rangle\rangle$
\end{list}
Here we say that an individual variable is a class; $\{x|\varphi\}$ is a
class; and we extend the definition of a wff to include class equality and
membership.  Axiom Ab defines membership of a variable in a class abstraction;
the right-hand side can be read as ``the wff that results from proper
substitution of $y$ for $x$ in $\varphi$.''\footnote{Note that this definition
makes unnecessary the introduction of a separate notation similar to
$\varphi(x|y)$ for proper substitution, although we may choose to do so to be
conventional.  Incidentally, $\varphi(x|y)$ as it stands would be ambiguous in
the formal systems of our examples, since we wouldn't know whether
$\lnot\varphi(x|y)$ meant $\lnot(\varphi(x|y))$ or $(\lnot\varphi)(x|y)$.
Instead, we would have to use an unambiguous variant such as $(\varphi\,
x|y)$.}  Axioms Eq and El extend the meaning of the existing equality and
membership connectives.  This is potentially dangerous and requires careful
justification.  For example, from Eq we can derive the Axiom of Extensionality
with predicate logic alone; thus in principle we should include the Axiom of
Extensionality as a logical hypothesis.  However we do not bother to do this
since we have already presupposed that axiom earlier. The distinct variable
restrictions should be read ``where $x$ does not occur in $A$ or $B$.''  We
typically do this when the right-hand side of a definition involves an
individual variable not in the expression being defined; it is done so that
the right-hand side remains independent of the particular ``dummy'' variable
we use.

We continue to add to $\Gamma$ the following definitions
(i.e. the reducts of the following pre-statements) for empty
set,\index{empty set} class union,\index{union} and unordered
pair.\index{unordered pair}  They should be self-explanatory.  Analogous to our
use of ``$\leftrightarrow$'' to define new wffs in Example~4, we use ``$=$''
to define new abstraction terms, and both may be read informally as ``is
defined as'' in this context.
\begin{list}{}{\itemsep 0.0pt}
      \item[] $\langle\varnothing,T,\varnothing,
               \langle \mbox{class\ }\varnothing\rangle\rangle$
      \item[] $\langle\varnothing,T,\varnothing,
               \langle \vdash \varnothing = \{ x | \lnot x = x \}
               \rangle\rangle$
      \item[] $\langle\varnothing,T,\varnothing,
               \langle \mbox{class\ }(A\cup B)\rangle\rangle$
      \item[] $\langle\{\{x,A\},\{x,B\}\},T,\varnothing,
               \langle \vdash ( A \cup B ) = \{ x | ( x \in A \vee x \in B ) \}
               \rangle\rangle$
      \item[] $\langle\varnothing,T,\varnothing,
               \langle \mbox{class\ }\{A,B\}\rangle\rangle$
      \item[] $\langle\{\{x,A\},\{x,B\}\},T,\varnothing,
               \langle \vdash \{ A , B \} = \{ x | ( x = A \vee x = B ) \}
               \rangle\rangle$
\end{list}

\section{Metamath as a Formal System}\label{theorymm}

This section presupposes a familiarity with the Metamath computer language.

Our theory describes formal systems and their universes.  The Metamath
language provides a way of representing these set-theoretical objects to
a computer.  A Metamath database, being a finite set of {\sc ascii}
characters, can usually describe only a subset of a formal system and
its universe, which are typically infinite.  However the database can
contain as large a finite subset of the formal system and its universe
as we wish.  (Of course a Metamath set theory database can, in
principle, indirectly describe an entire infinite formal system by
formalizing the expository language in this Appendix.)

For purpose of our discussion, we assume the Metamath database
is in the simple form described on p.~\pageref{framelist},
consisting of all constant and variable declarations at the beginning,
followed by a sequence of extended frames each
delimited by \texttt{\$\char`\{} and \texttt{\$\char`\}}.  Any Metamath database can
be converted to this form, as described on p.~\pageref{frameconvert}.

The math symbol tokens of a Metamath source file, which are declared
with \texttt{\$c} and \texttt{\$v} statements, are names we assign to
representatives of $\mbox{\em CN}$ and $\mbox{\em VR}$.  For
definiteness we could assume that the first math symbol declared as a
variable corresponds to $v_0$, the second to $v_1$, etc., although the
exact correspondence we choose is not important.

In the Metamath language, each \texttt{\$d}, \texttt{\$f}, and
 \texttt{\$e} source
statement in an extended frame (Section~\ref{frames})
corresponds respectively to a member of the
collections $D$, $T$, and $H$ in a formal system statement $\langle
D_M,T_M,H,A\rangle$.  The math symbol strings following these Metamath keywords
correspond to a variable pair (in the case of \texttt{\$d}) or an expression (for
the other two keywords). The math symbol string following a \texttt{\$a} source
statement corresponds to expression $A$ in an axiomatic statement of the
formal system; the one following a \texttt{\$p} source statement corresponds to
$A$ in a provable statement that is not axiomatic.  In other words, each
extended frame in a Metamath database corresponds to
a pre-statement of the formal system, and a frame corresponds to
a statement of the formal system.  (Don't confuse the two meanings of
``statement'' here.  A statement of the formal system corresponds to the
several statements in a Metamath database that may constitute a
frame.)

In order for the computer to verify that a formal system statement is
provable, each \texttt{\$p} source statement is accompanied by a proof.
However, the proof does not correspond to anything in the formal system
but is simply a way of communicating to the computer the information
needed for its verification.  The proof tells the computer {\em how to
construct} specific members of closure of the formal system
pre-statement corresponding to the extended frame of the \texttt{\$p}
statement.  The final result of the construction is the member of the
closure that matches the \texttt{\$p} statement.  The abstract formal
system, on the other hand, is concerned only with the {\em existence} of
members of the closure.

As mentioned on p.~\pageref{exampleref}, Examples 1 and 3--6 in the
previous Section parallel the development of logic and set theory in the
Metamath database
\texttt{set.mm}.\index{set theory database (\texttt{set.mm})} You may
find it instructive to compare them.


\chapter{The MIU System}
\label{MIU}
\index{formal system}
\index{MIU-system}

The following is a listing of the file \texttt{miu.mm}.  It is self-explanatory.

%%%%%%%%%%%%%%%%%%%%%%%%%%%%%%%%%%%%%%%%%%%%%%%%%%%%%%%%%%%%

\begin{verbatim}
$( The MIU-system:  A simple formal system $)

$( Note:  This formal system is unusual in that it allows
empty wffs.  To work with a proof, you must type
SET EMPTY_SUBSTITUTION ON before using the PROVE command.
By default, this is OFF in order to reduce the number of
ambiguous unification possibilities that have to be selected
during the construction of a proof.  $)

$(
Hofstadter's MIU-system is a simple example of a formal
system that illustrates some concepts of Metamath.  See
Douglas R. Hofstadter, _Goedel, Escher, Bach:  An Eternal
Golden Braid_ (Vintage Books, New York, 1979), pp. 33ff. for
a description of the MIU-system.

The system has 3 constant symbols, M, I, and U.  The sole
axiom of the system is MI. There are 4 rules:
     Rule I:  If you possess a string whose last letter is I,
     you can add on a U at the end.
     Rule II:  Suppose you have Mx.  Then you may add Mxx to
     your collection.
     Rule III:  If III occurs in one of the strings in your
     collection, you may make a new string with U in place
     of III.
     Rule IV:  If UU occurs inside one of your strings, you
     can drop it.
Unfortunately, Rules III and IV do not have unique results:
strings could have more than one occurrence of III or UU.
This requires that we introduce the concept of an "MIU
well-formed formula" or wff, which allows us to construct
unique symbol sequences to which Rules III and IV can be
applied.
$)

$( First, we declare the constant symbols of the language.
Note that we need two symbols to distinguish the assertion
that a sequence is a wff from the assertion that it is a
theorem; we have arbitrarily chosen "wff" and "|-". $)
      $c M I U |- wff $. $( Declare constants $)

$( Next, we declare some variables. $)
     $v x y $.

$( Throughout our theory, we shall assume that these
variables represent wffs. $)
 wx   $f wff x $.
 wy   $f wff y $.

$( Define MIU-wffs.  We allow the empty sequence to be a
wff. $)

$( The empty sequence is a wff. $)
 we   $a wff $.
$( "M" after any wff is a wff. $)
 wM   $a wff x M $.
$( "I" after any wff is a wff. $)
 wI   $a wff x I $.
$( "U" after any wff is a wff. $)
 wU   $a wff x U $.

$( Assert the axiom. $)
 ax   $a |- M I $.

$( Assert the rules. $)
 ${
   Ia   $e |- x I $.
$( Given any theorem ending with "I", it remains a theorem
if "U" is added after it.  (We distinguish the label I_
from the math symbol I to conform to the 24-Jun-2006
Metamath spec.) $)
   I_    $a |- x I U $.
 $}
 ${
IIa  $e |- M x $.
$( Given any theorem starting with "M", it remains a theorem
if the part after the "M" is added again after it. $)
   II   $a |- M x x $.
 $}
 ${
   IIIa $e |- x I I I y $.
$( Given any theorem with "III" in the middle, it remains a
theorem if the "III" is replaced with "U". $)
   III  $a |- x U y $.
 $}
 ${
   IVa  $e |- x U U y $.
$( Given any theorem with "UU" in the middle, it remains a
theorem if the "UU" is deleted. $)
   IV   $a |- x y $.
  $}

$( Now we prove the theorem MUIIU.  You may be interested in
comparing this proof with that of Hofstadter (pp. 35 - 36).
$)
 theorem1  $p |- M U I I U $=
      we wM wU wI we wI wU we wU wI wU we wM we wI wU we wM
      wI wI wI we wI wI we wI ax II II I_ III II IV $.
\end{verbatim}\index{well-formed formula (wff)}

The \texttt{show proof /lemmon/renumber} command
yields the following display.  It is very similar
to the one in \cite[pp.~35--36]{Hofstadter}.\index{Hofstadter, Douglas R.}

\begin{verbatim}
1 ax             $a |- M I
2 1 II           $a |- M I I
3 2 II           $a |- M I I I I
4 3 I_           $a |- M I I I I U
5 4 III          $a |- M U I U
6 5 II           $a |- M U I U U I U
7 6 IV           $a |- M U I I U
\end{verbatim}

We note that Hofstadter's ``MU-puzzle,'' which asks whether
MU is a theorem of the MIU-system, cannot be answered using
the system above because the MU-puzzle is a question {\em
about} the system.  To prove the answer to the MU-puzzle,
a much more elaborate system is needed, namely one that
models the MIU-system within set theory.  (Incidentally, the
answer to the MU-puzzle is no.)

\chapter{Metamath Language EBNF}%
\label{BNF}%
\index{Metamath Language EBNF}

The following is a formal description of the basic Metamath language syntax
(with compressed proofs and support for unknown proof steps).
It is defined using the
Extended Backus--Naur Form (EBNF)\index{Extended Backus--Naur Form}\index{EBNF}
notation from W3C\index{W3C}
\textit{Extensible Markup Language (XML) 1.0 (Fifth Edition)}
(W3C Recommendation 26 November 2008) at
\url{https://www.w3.org/TR/xml/#sec-notation}.

The \texttt{database}
rule is processed until the end of the file (\texttt{EOF}).
The rules eventually require reading whitespace-separated tokens.
A token has an upper-case definition (see below)
or is a string constant in a non-token (such as \texttt{'\$a'}).
We intend for this to be correct, but if there is a conflict the
rules of section \ref{spec} govern. That section also discusses
non-syntax restrictions not shown here
(e.g., that each new label token
defined in a \texttt{hypothesis-stmt} or \texttt{assert-stmt}
must be unique).

\begin{verbatim}
database ::= outermost-scope-stmt*

outermost-scope-stmt ::=
  include-stmt | constant-stmt | stmt

/* File inclusion command; process file as a database.
   Databases should NOT have a comment in the filename. */
include-stmt ::= '$[' filename '$]'

/* Constant symbols declaration. */
constant-stmt ::= '$c' constant+ '$.'

/* A normal statement can occur in any scope. */
stmt ::= block | variable-stmt | disjoint-stmt |
  hypothesis-stmt | assert-stmt

/* A block. You can have 0 statements in a block. */
block ::= '${' stmt* '$}'

/* Variable symbols declaration. */
variable-stmt ::= '$v' variable+ '$.'

/* Disjoint variables. Simple disjoint statements have
   2 variables, i.e., "variable*" is empty for them. */
disjoint-stmt ::= '$d' variable variable variable* '$.'

hypothesis-stmt ::= floating-stmt | essential-stmt

/* Floating (variable-type) hypothesis. */
floating-stmt ::= LABEL '$f' typecode variable '$.'

/* Essential (logical) hypothesis. */
essential-stmt ::= LABEL '$e' typecode MATH-SYMBOL* '$.'

assert-stmt ::= axiom-stmt | provable-stmt

/* Axiomatic assertion. */
axiom-stmt ::= LABEL '$a' typecode MATH-SYMBOL* '$.'

/* Provable assertion. */
provable-stmt ::= LABEL '$p' typecode MATH-SYMBOL*
  '$=' proof '$.'

/* A proof. Proofs may be interspersed by comments.
   If '?' is in a proof it's an "incomplete" proof. */
proof ::= uncompressed-proof | compressed-proof
uncompressed-proof ::= (LABEL | '?')+
compressed-proof ::= '(' LABEL* ')' COMPRESSED-PROOF-BLOCK+

typecode ::= constant

filename ::= MATH-SYMBOL /* No whitespace or '$' */
constant ::= MATH-SYMBOL
variable ::= MATH-SYMBOL
\end{verbatim}

\needspace{2\baselineskip}
A \texttt{frame} is a sequence of 0 or more
\texttt{disjoint-{\allowbreak}stmt} and
\texttt{hypotheses-{\allowbreak}stmt} statements
(possibly interleaved with other non-\texttt{assert-stmt} statements)
followed by one \texttt{assert-stmt}.

\needspace{3\baselineskip}
Here are the rules for lexical processing (tokenization) beyond
the constant tokens shown above.
By convention these tokenization rules have upper-case names.
Every token is read for the longest possible length.
Whitespace-separated tokens are read sequentially;
note that the separating whitespace and \texttt{\$(} ... \texttt{\$)}
comments are skipped.

If a token definition uses another token definition, the whole thing
is considered a single token.
A pattern that is only part of a full token has a name beginning
with an underscore (``\_'').
An implementation could tokenize many tokens as a
\texttt{PRINTABLE-SEQUENCE}
and then check if it meets the more specific rule shown here.

Comments do not nest, and both \texttt{\$(} and \texttt{\$)}
have to be surrounded
by at least one whitespace character (\texttt{\_WHITECHAR}).
Technically comments end without consuming the trailing
\texttt{\_WHITECHAR}, but the trailing
\texttt{\_WHITECHAR} gets ignored anyway so we ignore that detail here.
Metamath language processors
are not required to support \texttt{\$)} followed
immediately by a bare end-of-file, because the closing
comment symbol is supposed to be followed by a
\texttt{\_WHITECHAR} such as a newline.

\begin{verbatim}
PRINTABLE-SEQUENCE ::= _PRINTABLE-CHARACTER+

MATH-SYMBOL ::= (_PRINTABLE-CHARACTER - '$')+

/* ASCII non-whitespace printable characters */
_PRINTABLE-CHARACTER ::= [#x21-#x7e]

LABEL ::= ( _LETTER-OR-DIGIT | '.' | '-' | '_' )+

_LETTER-OR-DIGIT ::= [A-Za-z0-9]

COMPRESSED-PROOF-BLOCK ::= ([A-Z] | '?')+

/* Define whitespace between tokens. The -> SKIP
   means that when whitespace is seen, it is
   skipped and we simply read again. */
WHITESPACE ::= (_WHITECHAR+ | _COMMENT) -> SKIP

/* Comments. $( ... $) and do not nest. */
_COMMENT ::= '$(' (_WHITECHAR+ (PRINTABLE-SEQUENCE - '$)')*
  _WHITECHAR+ '$)' _WHITECHAR

/* Whitespace: (' ' | '\t' | '\r' | '\n' | '\f') */
_WHITECHAR ::= [#x20#x09#x0d#x0a#x0c]
\end{verbatim}
% This EBNF was developed as a collaboration between
% David A. Wheeler\index{Wheeler, David A.},
% Mario Carneiro\index{Carneiro, Mario}, and
% Benoit Jubin\index{Jubin, Benoit}, inspired by a request
% (and a lot of initial work) by Benoit Jubin.
%
% \chapter{Disclaimer and Trademarks}
%
% Information in this document is subject to change without notice and does not
% represent a commitment on the part of Norman Megill.
% \vspace{2ex}
%
% \noindent Norman D. Megill makes no warranties, either express or implied,
% regarding the Metamath computer software package.
%
% \vspace{2ex}
%
% \noindent Any trademarks mentioned in this book are the property of
% their respective owners.  The name ``Metamath'' is a trademark of
% Norman Megill.
%
\cleardoublepage
\phantomsection  % fixes the link anchor
\addcontentsline{toc}{chapter}{\bibname}

\bibliography{metamath}
%\input{metamath.bbl}

\raggedright
\cleardoublepage
\phantomsection % fixes the link anchor
\addcontentsline{toc}{chapter}{\indexname}
%\printindex   ??
\input{metamath.ind}

\end{document}



\raggedright
\cleardoublepage
\phantomsection % fixes the link anchor
\addcontentsline{toc}{chapter}{\indexname}
%\printindex   ??
% metamath.tex - Version of 24-May-2019
% If you change the date above, also change the "Printed date" below.
% SPDX-License-Identifier: CC0-1.0
%
%                              PUBLIC DOMAIN
%
% This file (specifically, the version of this file with the above date)
% has been released into the Public Domain per the
% Creative Commons CC0 1.0 Universal (CC0 1.0) Public Domain Dedication
% https://creativecommons.org/publicdomain/zero/1.0/
%
% The public domain release applies worldwide.  In case this is not
% legally possible, the right is granted to use the work for any purpose,
% without any conditions, unless such conditions are required by law.
%
% Several short, attributed quotations from copyrighted works
% appear in this file under the ``fair use'' provision of Section 107 of
% the United States Copyright Act (Title 17 of the {\em United States
% Code}).  The public-domain status of this file is not applicable to
% those quotations.
%
% Norman Megill - email: nm(at)alum(dot)mit(dot)edu
%
% David A. Wheeler also donates his improvements to this file to the
% public domain per the CC0.  He works at the Institute for Defense Analyses
% (IDA), but IDA has agreed that this Metamath work is outside its "lane"
% and is not a work by IDA.  This was specifically confirmed by
% Margaret E. Myers (Division Director of the Information Technology
% and Systems Division) on 2019-05-24 and by Ben Lindorf (General Counsel)
% on 2019-05-22.

% This file, 'metamath.tex', is self-contained with everything needed to
% generate the the PDF file 'metamath.pdf' (the _Metamath_ book) on
% standard LaTeX 2e installations.  The auxiliary files are embedded with
% "filecontents" commands.  To generate metamath.pdf file, run these
% commands under Linux or Cygwin in the directory that contains
% 'metamath.tex':
%
%   rm -f realref.sty metamath.bib
%   touch metamath.ind
%   pdflatex metamath
%   pdflatex metamath
%   bibtex metamath
%   makeindex metamath
%   pdflatex metamath
%   pdflatex metamath
%
% The warnings that occur in the initial runs of pdflatex can be ignored.
% For the final run,
%
%   egrep -i 'error|warn' metamath.log
%
% should show exactly these 5 warnings:
%
%   LaTeX Warning: File `realref.sty' already exists on the system.
%   LaTeX Warning: File `metamath.bib' already exists on the system.
%   LaTeX Font Warning: Font shape `OMS/cmtt/m/n' undefined
%   LaTeX Font Warning: Font shape `OMS/cmtt/bx/n' undefined
%   LaTeX Font Warning: Some font shapes were not available, defaults
%       substituted.
%
% Search for "Uncomment" below if you want to suppress hyperlink boxes
% in the PDF output file
%
% TYPOGRAPHICAL NOTES:
% * It is customary to use an en dash (--) to "connect" names of different
%   people (and to denote ranges), and use a hyphen (-) for a
%   single compound name. Examples of connected multiple people are
%   Zermelo--Fraenkel, Schr\"{o}der--Bernstein, Tarski--Grothendieck,
%   Hewlett--Packard, and Backus--Naur.  Examples of a single person with
%   a compound name include Levi-Civita, Mittag-Leffler, and Burali-Forti.
% * Use non-breaking spaces after page abbreviations, e.g.,
%   p.~\pageref{note2002}.
%
% --------------------------- Start of realref.sty -----------------------------
\begin{filecontents}{realref.sty}
% Save the following as realref.sty.
% You can then use it with \usepackage{realref}
%
% This has \pageref jumping to the page on which the ref appears,
% \ref jumping to the point of the anchor, and \sectionref
% jumping to the start of section.
%
% Author:  Anthony Williams
%          Software Engineer
%          Nortel Networks Optical Components Ltd
% Date:    9 Nov 2001 (posted to comp.text.tex)
%
% The following declaration was made by Anthony Williams on
% 24 Jul 2006 (private email to Norman Megill):
%
%   ``I hereby donate the code for realref.sty posted on the
%   comp.text.tex newsgroup on 9th November 2001, accessible from
%   http://groups.google.com/group/comp.text.tex/msg/5a0e1cc13ea7fbb2
%   to the public domain.''
%
\ProvidesPackage{realref}
\RequirePackage[plainpages=false,pdfpagelabels=true]{hyperref}
\def\realref@anchorname{}
\AtBeginDocument{%
% ensure every label is a possible hyperlink target
\let\realref@oldrefstepcounter\refstepcounter%
\DeclareRobustCommand{\refstepcounter}[1]{\realref@oldrefstepcounter{#1}
\edef\realref@anchorname{\string #1.\@currentlabel}%
}%
\let\realref@oldlabel\label%
\DeclareRobustCommand{\label}[1]{\realref@oldlabel{#1}\hypertarget{#1}{}%
\@bsphack\protected@write\@auxout{}{%
    \string\expandafter\gdef\protect\csname
    page@num.#1\string\endcsname{\thepage}%
    \string\expandafter\gdef\protect\csname
    ref@num.#1\string\endcsname{\@currentlabel}%
    \string\expandafter\gdef\protect\csname
    sectionref@name.#1\string\endcsname{\realref@anchorname}%
}\@esphack}%
\DeclareRobustCommand\pageref[1]{{\edef\a{\csname
            page@num.#1\endcsname}\expandafter\hyperlink{page.\a}{\a}}}%
\DeclareRobustCommand\ref[1]{{\edef\a{\csname
            ref@num.#1\endcsname}\hyperlink{#1}{\a}}}%
\DeclareRobustCommand\sectionref[1]{{\edef\a{\csname
            ref@num.#1\endcsname}\edef\b{\csname
            sectionref@name.#1\endcsname}\hyperlink{\b}{\a}}}%
}
\end{filecontents}
% ---------------------------- End of realref.sty ------------------------------

% --------------------------- Start of metamath.bib -----------------------------
\begin{filecontents}{metamath.bib}
@book{Albers, editor = "Donald J. Albers and G. L. Alexanderson",
  title = "Mathematical People",
  publisher = "Contemporary Books, Inc.",
  address = "Chicago",
  note = "[QA28.M37]",
  year = 1985 }
@book{Anderson, author = "Alan Ross Anderson and Nuel D. Belnap",
  title = "Entailment",
  publisher = "Princeton University Press",
  address = "Princeton",
  volume = 1,
  note = "[QA9.A634 1975 v.1]",
  year = 1975}
@book{Barrow, author = "John D. Barrow",
  title = "Theories of Everything:  The Quest for Ultimate Explanation",
  publisher = "Oxford University Press",
  address = "Oxford",
  note = "[Q175.B225]",
  year = 1991 }
@book{Behnke,
  editor = "H. Behnke and F. Backmann and K. Fladt and W. S{\"{u}}ss",
  title = "Fundamentals of Mathematics",
  volume = "I",
  publisher = "The MIT Press",
  address = "Cambridge, Massachusetts",
  note = "[QA37.2.B413]",
  year = 1974 }
@book{Bell, author = "J. L. Bell and M. Machover",
  title = "A Course in Mathematical Logic",
  publisher = "North-Holland",
  address = "Amsterdam",
  note = "[QA9.B3953]",
  year = 1977 }
@inproceedings{Blass, author = "Andrea Blass",
  title = "The Interaction Between Category Theory and Set Theory",
  pages = "5--29",
  booktitle = "Mathematical Applications of Category Theory (Proceedings
     of the Special Session on Mathematical Applications
     Category Theory, 89th Annual Meeting of the American Mathematical
     Society, held in Denver, Colorado January 5--9, 1983)",
  editor = "John Walter Gray",
  year = 1983,
  note = "[QA169.A47 1983]",
  publisher = "American Mathematical Society",
  address = "Providence, Rhode Island"}
@proceedings{Bledsoe, editor = "W. W. Bledsoe and D. W. Loveland",
  title = "Automated Theorem Proving:  After 25 Years (Proceedings
     of the Special Session on Automatic Theorem Proving,
     89th Annual Meeting of the American Mathematical
     Society, held in Denver, Colorado January 5--9, 1983)",
  year = 1983,
  note = "[QA76.9.A96.S64 1983]",
  publisher = "American Mathematical Society",
  address = "Providence, Rhode Island" }
@book{Boolos, author = "George S. Boolos and Richard C. Jeffrey",
  title = "Computability and Log\-ic",
  publisher = "Cambridge University Press",
  edition = "third",
  address = "Cambridge",
  note = "[QA9.59.B66 1989]",
  year = 1989 }
@book{Campbell, author = "John Campbell",
  title = "Programmer's Progress",
  publisher = "White Star Software",
  address = "Box 51623, Palo Alto, CA 94303",
  year = 1991 }
@article{DBLP:journals/corr/Carneiro14,
  author    = {Mario Carneiro},
  title     = {Conversion of {HOL} Light proofs into Metamath},
  journal   = {CoRR},
  volume    = {abs/1412.8091},
  year      = {2014},
  url       = {http://arxiv.org/abs/1412.8091},
  archivePrefix = {arXiv},
  eprint    = {1412.8091},
  timestamp = {Mon, 13 Aug 2018 16:47:05 +0200},
  biburl    = {https://dblp.org/rec/bib/journals/corr/Carneiro14},
  bibsource = {dblp computer science bibliography, https://dblp.org}
}
@article{CarneiroND,
  author    = {Mario Carneiro},
  title     = {Natural Deductions in the Metamath Proof Language},
  url       = {http://us.metamath.org/ocat/natded.pdf},
  year      = 2014
}
@inproceedings{Chou, author = "Shang-Ching Chou",
  title = "Proving Elementary Geometry Theorems Using {W}u's Algorithm",
  pages = "243--286",
  booktitle = "Automated Theorem Proving:  After 25 Years (Proceedings
     of the Special Session on Automatic Theorem Proving,
     89th Annual Meeting of the American Mathematical
     Society, held in Denver, Colorado January 5--9, 1983)",
  editor = "W. W. Bledsoe and D. W. Loveland",
  year = 1983,
  note = "[QA76.9.A96.S64 1983]",
  publisher = "American Mathematical Society",
  address = "Providence, Rhode Island" }
@book{Clemente, author = "Daniel Clemente Laboreo",
  title = "Introduction to natural deduction",
  year = 2014,
  url = "http://www.danielclemente.com/logica/dn.en.pdf" }
@incollection{Courant, author = "Richard Courant and Herbert Robbins",
  title = "Topology",
  pages = "573--590",
  booktitle = "The World of Mathematics, Volume One",
  editor = "James R. Newman",
  publisher = "Simon and Schuster",
  address = "New York",
  note = "[QA3.W67 1988]",
  year = 1956 }
@book{Curry, author = "Haskell B. Curry",
  title = "Foundations of Mathematical Logic",
  publisher = "Dover Publications, Inc.",
  address = "New York",
  note = "[QA9.C976 1977]",
  year = 1977 }
@book{Davis, author = "Philip J. Davis and Reuben Hersh",
  title = "The Mathematical Experience",
  publisher = "Birkh{\"{a}}user Boston",
  address = "Boston",
  note = "[QA8.4.D37 1982]",
  year = 1981 }
@incollection{deMillo,
  author = "Richard de Millo and Richard Lipton and Alan Perlis",
  title = "Social Processes and Proofs of Theorems and Programs",
  pages = "267--285",
  booktitle = "New Directions in the Philosophy of Mathematics",
  editor = "Thomas Tymoczko",
  publisher = "Birkh{\"{a}}user Boston, Inc.",
  address = "Boston",
  note = "[QA8.6.N48 1986]",
  year = 1986 }
@book{Edwards, author = "Robert E. Edwards",
  title = "A Formal Background to Mathematics",
  publisher = "Springer-Verlag",
  address = "New York",
  note = "[QA37.2.E38 v.1a]",
  year = 1979 }
@book{Enderton, author = "Herbert B. Enderton",
  title = "Elements of Set Theory",
  publisher = "Academic Press, Inc.",
  address = "San Diego",
  note = "[QA248.E5]",
  year = 1977 }
@book{Goodstein, author = "R. L. Goodstein",
  title = "Development of Mathematical Logic",
  publisher = "Springer-Verlag New York Inc.",
  address = "New York",
  note = "[QA9.G6554]",
  year = 1971 }
@book{Guillen, author = "Michael Guillen",
  title = "Bridges to Infinity",
  publisher = "Jeremy P. Tarcher, Inc.",
  address = "Los Angeles",
  note = "[QA93.G8]",
  year = 1983 }
@book{Hamilton, author = "Alan G. Hamilton",
  title = "Logic for Mathematicians",
  edition = "revised",
  publisher = "Cambridge University Press",
  address = "Cambridge",
  note = "[QA9.H298]",
  year = 1988 }
@unpublished{Harrison, author = "John Robert Harrison",
  title = "Metatheory and Reflection in Theorem Proving:
    A Survey and Critique",
  note = "Technical Report
    CRC-053.
    SRI Cambridge,
    Millers Yard, Cambridge, UK,
    1995.
    Available on the Web as
{\verb+http:+}\-{\verb+//www.cl.cam.ac.uk/users/jrh/papers/reflect.html+}"}
@TECHREPORT{Harrison-thesis,
        author          = "John Robert Harrison",
        title           = "Theorem Proving with the Real Numbers",
        institution   = "University of Cambridge Computer
                         Lab\-o\-ra\-to\-ry",
        address         = "New Museums Site, Pembroke Street, Cambridge,
                           CB2 3QG, UK",
        year            = 1996,
        number          = 408,
        type            = "Technical Report",
        note            = "Author's PhD thesis,
   available on the Web at
{\verb+http:+}\-{\verb+//www.cl.cam.ac.uk+}\-{\verb+/users+}\-{\verb+/jrh+}%
\-{\verb+/papers+}\-{\verb+/thesis.html+}"}
@book{Herrlich, author = "Horst Herrlich and George E. Strecker",
  title = "Category Theory:  An Introduction",
  publisher = "Allyn and Bacon Inc.",
  address = "Boston",
  note = "[QA169.H567]",
  year = 1973 }
@article{Hindley, author = "J. Roger Hindley and David Meredith",
  title = "Principal Type-Schemes and Condensed Detachment",
  journal = "The Journal of Symbolic Logic",
  volume = 55,
  year = 1990,
  note = "[QA.J87]",
  pages = "90--105" }
@book{Hofstadter, author = "Douglas R. Hofstadter",
  title = "G{\"{o}}del, Escher, Bach",
  publisher = "Basic Books, Inc.",
  address = "New York",
  note = "[QA9.H63 1980]",
  year = 1979 }
@article{Indrzejczak, author= "Andrzej Indrzejczak",
  title = "Natural Deduction, Hybrid Systems and Modal Logic",
  journal = "Trends in Logic",
  volume = 30,
  publisher = "Springer",
  year = 2010 }
@article{Kalish, author = "D. Kalish and R. Montague",
  title = "On {T}arski's Formalization of Predicate Logic with Identity",
  journal = "Archiv f{\"{u}}r Mathematische Logik und Grundlagenfor\-schung",
  volume = 7,
  year = 1965,
  note = "[QA.A673]",
  pages = "81--101" }
@article{Kalman, author = "J. A. Kalman",
  title = "Condensed Detachment as a Rule of Inference",
  journal = "Studia Logica",
  volume = 42,
  number = 4,
  year = 1983,
  note = "[B18.P6.S933]",
  pages = "443-451" }
@book{Kline, author = "Morris Kline",
  title = "Mathematical Thought from Ancient to Modern Times",
  publisher = "Oxford University Press",
  address = "New York",
  note = "[QA21.K516 1990 v.3]",
  year = 1972 }
@book{Klinel, author = "Morris Kline",
  title = "Mathematics, The Loss of Certainty",
  publisher = "Oxford University Press",
  address = "New York",
  note = "[QA21.K525]",
  year = 1980 }
@book{Kramer, author = "Edna E. Kramer",
  title = "The Nature and Growth of Modern Mathematics",
  publisher = "Princeton University Press",
  address = "Princeton, New Jersey",
  note = "[QA93.K89 1981]",
  year = 1981 }
@article{Knill, author = "Oliver Knill",
  title = "Some Fundamental Theorems in Mathematics",
  year = "2018",
  url = "https://arxiv.org/abs/1807.08416" }
@book{Landau, author = "Edmund Landau",
  title = "Foundations of Analysis",
  publisher = "Chelsea Publishing Company",
  address = "New York",
  edition = "second",
  note = "[QA241.L2541 1960]",
  year = 1960 }
@article{Leblanc, author = "Hugues Leblanc",
  title = "On {M}eyer and {L}ambert's Quantificational Calculus {FQ}",
  journal = "The Journal of Symbolic Logic",
  volume = 33,
  year = 1968,
  note = "[QA.J87]",
  pages = "275--280" }
@article{Lejewski, author = "Czeslaw Lejewski",
  title = "On Implicational Definitions",
  journal = "Studia Logica",
  volume = 8,
  year = 1958,
  note = "[B18.P6.S933]",
  pages = "189--208" }
@book{Levy, author = "Azriel Levy",
  title = "Basic Set Theory",
  publisher = "Dover Publications",
  address = "Mineola, NY",
  year = "2002"
}
@book{Margaris, author = "Angelo Margaris",
  title = "First Order Mathematical Logic",
  publisher = "Blaisdell Publishing Company",
  address = "Waltham, Massachusetts",
  note = "[QA9.M327]",
  year = 1967}
@book{Manin, author = "Yu I. Manin",
  title = "A Course in Mathematical Logic",
  publisher = "Springer-Verlag",
  address = "New York",
  note = "[QA9.M29613]",
  year = "1977" }
@article{Mathias, author = "Adrian R. D. Mathias",
  title = "A Term of Length 4,523,659,424,929",
  journal = "Synthese",
  volume = 133,
  year = 2002,
  note = "[Q.S993]",
  pages = "75--86" }
@article{Megill, author = "Norman D. Megill",
  title = "A Finitely Axiomatized Formalization of Predicate Calculus
     with Equality",
  journal = "Notre Dame Journal of Formal Logic",
  volume = 36,
  year = 1995,
  note = "[QA.N914]",
  pages = "435--453" }
@unpublished{Megillc, author = "Norman D. Megill",
  title = "A Shorter Equivalent of the Axiom of Choice",
  month = "June",
  note = "Unpublished",
  year = 1991 }
@article{MegillBunder, author = "Norman D. Megill and Martin W.
    Bunder",
  title = "Weaker {D}-Complete Logics",
  journal = "Journal of the IGPL",
  volume = 4,
  year = 1996,
  pages = "215--225",
  note = "Available on the Web at
{\verb+http:+}\-{\verb+//www.mpi-sb.mpg.de+}\-{\verb+/igpl+}%
\-{\verb+/Journal+}\-{\verb+/V4-2+}\-{\verb+/#Megill+}"}
}
@book{Mendelson, author = "Elliott Mendelson",
  title = "Introduction to Mathematical Logic",
  edition = "second",
  publisher = "D. Van Nostrand Company, Inc.",
  address = "New York",
  note = "[QA9.M537 1979]",
  year = 1979 }
@article{Meredith, author = "David Meredith",
  title = "In Memoriam {C}arew {A}rthur {M}eredith (1904-1976)",
  journal = "Notre Dame Journal of Formal Logic",
  volume = 18,
  year = 1977,
  note = "[QA.N914]",
  pages = "513--516" }
@article{CAMeredith, author = "C. A. Meredith",
  title = "Single Axioms for the Systems ({C},{N}), ({C},{O}) and ({A},{N})
      of the Two-Valued Propositional Calculus",
  journal = "The Journal of Computing Systems",
  volume = 3,
  year = 1953,
  pages = "155--164" }
@article{Monk, author = "J. Donald Monk",
  title = "Provability With Finitely Many Variables",
  journal = "The Journal of Symbolic Logic",
  volume = 27,
  year = 1971,
  note = "[QA.J87]",
  pages = "353--358" }
@article{Monks, author = "J. Donald Monk",
  title = "Substitutionless Predicate Logic With Identity",
  journal = "Archiv f{\"{u}}r Mathematische Logik und Grundlagenfor\-schung",
  volume = 7,
  year = 1965,
  pages = "103--121" }
  %% Took out this from above to prevent LaTeX underfull warning:
  % note = "[QA.A673]",
@book{Moore, author = "A. W. Moore",
  title = "The Infinite",
  publisher = "Routledge",
  address = "New York",
  note = "[BD411.M59]",
  year = 1989}
@book{Munkres, author = "James R. Munkres",
  title = "Topology: A First Course",
  publisher = "Prentice-Hall, Inc.",
  address = "Englewood Cliffs, New Jersey",
  note = "[QA611.M82]",
  year = 1975}
@article{Nemesszeghy, author = "E. Z. Nemesszeghy and E. A. Nemesszeghy",
  title = "On Strongly Creative Definitions:  A Reply to {V}. {F}. {R}ickey",
  journal = "Logique et Analyse (N.\ S.)",
  year = 1977,
  volume = 20,
  note = "[BC.L832]",
  pages = "111--115" }
@unpublished{Nemeti, author = "N{\'{e}}meti, I.",
  title = "Algebraizations of Quantifier Logics, an Overview",
  note = "Version 11.4, preprint, Mathematical Institute, Budapest,
    1994.  A shortened version without proofs appeared in
    ``Algebraizations of quantifier logics, an introductory overview,''
   {\em Studia Logica}, 50:485--569, 1991 [B18.P6.S933]"}
@article{Pavicic, author = "M. Pavi{\v{c}}i{\'{c}}",
  title = "A New Axiomatization of Unified Quantum Logic",
  journal = "International Journal of Theoretical Physics",
  year = 1992,
  volume = 31,
  note = "[QC.I626]",
  pages = "1753 --1766" }
@book{Penrose, author = "Roger Penrose",
  title = "The Emperor's New Mind",
  publisher = "Oxford University Press",
  address = "New York",
  note = "[Q335.P415]",
  year = 1989 }
@book{PetersonI, author = "Ivars Peterson",
  title = "The Mathematical Tourist",
  publisher = "W. H. Freeman and Company",
  address = "New York",
  note = "[QA93.P475]",
  year = 1988 }
@article{Peterson, author = "Jeremy George Peterson",
  title = "An automatic theorem prover for substitution and detachment systems",
  journal = "Notre Dame Journal of Formal Logic",
  volume = 19,
  year = 1978,
  note = "[QA.N914]",
  pages = "119--122" }
@book{Quine, author = "Willard Van Orman Quine",
  title = "Set Theory and Its Logic",
  edition = "revised",
  publisher = "The Belknap Press of Harvard University Press",
  address = "Cambridge, Massachusetts",
  note = "[QA248.Q7 1969]",
  year = 1969 }
@article{Robinson, author = "J. A. Robinson",
  title = "A Machine-Oriented Logic Based on the Resolution Principle",
  journal = "Journal of the Association for Computing Machinery",
  year = 1965,
  volume = 12,
  pages = "23--41" }
@article{RobinsonT, author = "T. Thacher Robinson",
  title = "Independence of Two Nice Sets of Axioms for the Propositional
    Calculus",
  journal = "The Journal of Symbolic Logic",
  volume = 33,
  year = 1968,
  note = "[QA.J87]",
  pages = "265--270" }
@book{Rucker, author = "Rudy Rucker",
  title = "Infinity and the Mind:  The Science and Philosophy of the
    Infinite",
  publisher = "Bantam Books, Inc.",
  address = "New York",
  note = "[QA9.R79 1982]",
  year = 1982 }
@book{Russell, author = "Bertrand Russell",
  title = "Mysticism and Logic, and Other Essays",
  publisher = "Barnes \& Noble Books",
  address = "Totowa, New Jersey",
  note = "[B1649.R963.M9 1981]",
  year = 1981 }
@article{Russell2, author = "Bertrand Russell",
  title = "Recent Work on the Principles of Mathematics",
  journal = "International Monthly",
  volume = 4,
  year = 1901,
  pages = "84"}
@article{Schmidt, author = "Eric Schmidt",
  title = "Reductions in Norman Megill's axiom system for complex numbers",
  url = "http://us.metamath.org/downloads/schmidt-cnaxioms.pdf",
  year = "2012" }
@book{Shoenfield, author = "Joseph R. Shoenfield",
  title = "Mathematical Logic",
  publisher = "Addison-Wesley Publishing Company, Inc.",
  address = "Reading, Massachusetts",
  year = 1967,
  note = "[QA9.S52]" }
@book{Smullyan, author = "Raymond M. Smullyan",
  title = "Theory of Formal Systems",
  publisher = "Princeton University Press",
  address = "Princeton, New Jersey",
  year = 1961,
  note = "[QA248.5.S55]" }
@book{Solow, author = "Daniel Solow",
  title = "How to Read and Do Proofs:  An Introduction to Mathematical
    Thought Process",
  publisher = "John Wiley \& Sons",
  address = "New York",
  year = 1982,
  note = "[QA9.S577]" }
@book{Stark, author = "Harold M. Stark",
  title = "An Introduction to Number Theory",
  publisher = "Markham Publishing Company",
  address = "Chicago",
  note = "[QA241.S72 1978]",
  year = 1970 }
@article{Swart, author = "E. R. Swart",
  title = "The Philosophical Implications of the Four-Color Problem",
  journal = "American Mathematical Monthly",
  year = 1980,
  volume = 87,
  month = "November",
  note = "[QA.A5125]",
  pages = "697--707" }
@book{Szpiro, author = "George G. Szpiro",
  title = "Poincar{\'{e}}'s Prize: The Hundred-Year Quest to Solve One
    of Math's Greatest Puzzles",
  publisher = "Penguin Books Ltd",
  address = "London",
  note = "[QA43.S985 2007]",
  year = 2007}
@book{Takeuti, author = "Gaisi Takeuti and Wilson M. Zaring",
  title = "Introduction to Axiomatic Set Theory",
  edition = "second",
  publisher = "Springer-Verlag New York Inc.",
  address = "New York",
  note = "[QA248.T136 1982]",
  year = 1982}
@inproceedings{Tarski, author = "Alfred Tarski",
  title = "What is Elementary Geometry",
  pages = "16--29",
  booktitle = "The Axiomatic Method, with Special Reference to Geometry and
     Physics (Proceedings of an International Symposium held at the University
     of California, Berkeley, December 26, 1957 --- January 4, 1958)",
  editor = "Leon Henkin and Patrick Suppes and Alfred Tarski",
  year = 1959,
  publisher = "North-Holland Publishing Company",
  address = "Amsterdam"}
@article{Tarski1965, author = "Alfred Tarski",
  title = "A Simplified Formalization of Predicate Logic with Identity",
  journal = "Archiv f{\"{u}}r Mathematische Logik und Grundlagenforschung",
  volume = 7,
  year = 1965,
  note = "[QA.A673]",
  pages = "61--79" }
@book{Tymoczko,
  title = "New Directions in the Philosophy of Mathematics",
  editor = "Thomas Tymoczko",
  publisher = "Birkh{\"{a}}user Boston, Inc.",
  address = "Boston",
  note = "[QA8.6.N48 1986]",
  year = 1986 }
@incollection{Wang,
  author = "Hao Wang",
  title = "Theory and Practice in Mathematics",
  pages = "129--152",
  booktitle = "New Directions in the Philosophy of Mathematics",
  editor = "Thomas Tymoczko",
  publisher = "Birkh{\"{a}}user Boston, Inc.",
  address = "Boston",
  note = "[QA8.6.N48 1986]",
  year = 1986 }
@manual{Webster,
  title = "Webster's New Collegiate Dictionary",
  organization = "G. \& C. Merriam Co.",
  address = "Springfield, Massachusetts",
  note = "[PE1628.W4M4 1977]",
  year = 1977 }
@manual{Whitehead, author = "Alfred North Whitehead",
  title = "An Introduction to Mathematics",
  year = 1911 }
@book{PM, author = "Alfred North Whitehead and Bertrand Russell",
  title = "Principia Mathematica",
  edition = "second",
  publisher = "Cambridge University Press",
  address = "Cambridge",
  year = "1927",
  note = "(3 vols.) [QA9.W592 1927]" }
@article{DBLP:journals/corr/Whalen16,
  author    = {Daniel Whalen},
  title     = {Holophrasm: a neural Automated Theorem Prover for higher-order logic},
  journal   = {CoRR},
  volume    = {abs/1608.02644},
  year      = {2016},
  url       = {http://arxiv.org/abs/1608.02644},
  archivePrefix = {arXiv},
  eprint    = {1608.02644},
  timestamp = {Mon, 13 Aug 2018 16:46:19 +0200},
  biburl    = {https://dblp.org/rec/bib/journals/corr/Whalen16},
  bibsource = {dblp computer science bibliography, https://dblp.org} }
@article{Wiedijk-revisited,
  author = {Freek Wiedijk},
  title = {The QED Manifesto Revisited},
  year = {2007},
  url = {http://mizar.org/trybulec65/8.pdf} }
@book{Wolfram,
  author = "Stephen Wolfram",
  title = "Mathematica:  A System for Doing Mathematics by Computer",
  edition = "second",
  publisher = "Addison-Wesley Publishing Co.",
  address = "Redwood City, California",
  note = "[QA76.95.W65 1991]",
  year = 1991 }
@book{Wos, author = "Larry Wos and Ross Overbeek and Ewing Lusk and Jim Boyle",
  title = "Automated Reasoning:  Introduction and Applications",
  edition = "second",
  publisher = "McGraw-Hill, Inc.",
  address = "New York",
  note = "[QA76.9.A96.A93 1992]",
  year = 1992 }

%
%
%[1] Church, Alonzo, Introduction to Mathematical Logic,
% Volume 1, Princeton University Press, Princeton, N. J., 1956.
%
%[2] Cohen, Paul J., Set Theory and the Continuum Hypothesis,
% W. A. Benjamin, Inc., Reading, Mass., 1966.
%
%[3] Hamilton, Alan G., Logic for Mathematicians, Cambridge
% University Press,
% Cambridge, 1988.

%[6] Kleene, Stephen Cole, Introduction to Metamathematics, D.  Van
% Nostrand Company, Inc., Princeton (1952).

%[13] Tarski, Alfred, "A simplified formalization of predicate
% logic with identity," Archiv fur Mathematische Logik und
% Grundlagenforschung, vol. 7 (1965), pp. 61-79.

%[14] Tarski, Alfred and Steven Givant, A Formalization of Set
% Theory Without Variables, American Mathematical Society Colloquium
% Publications, vol. 41, American Mathematical Society,
% Providence, R. I., 1987.

%[15] Zeman, J. J., Modal Logic, Oxford University Press, Oxford, 1973.
\end{filecontents}
% --------------------------- End of metamath.bib -----------------------------


%Book: Metamath
%Author:  Norman Megill Email:  nm at alum.mit.edu
%Author:  David A. Wheeler Email:  dwheeler at dwheeler.com

% A book template example
% http://www.stsci.edu/ftp/software/tex/bookstuff/book.template

\documentclass[leqno]{book} % LaTeX 2e. 10pt. Use [leqno,12pt] for 12pt
% hyperref 2002/05/27 v6.72r  (couldn't get pagebackref to work)
\usepackage[plainpages=false,pdfpagelabels=true]{hyperref}

\usepackage{needspace}     % Enable control over page breaks
\usepackage{breqn}         % automatic equation breaking
\usepackage{microtype}     % microtypography, reduces hyphenation

% Packages for flexible tables.  We need to be able to
% wrap text within a cell (with automatically-determined widths) AND
% split a table automatically across multiple pages.
% * "tabularx" wraps text in cells but only 1 page
% * "longtable" goes across pages but by itself is incompatible with tabularx
% * "ltxtable" combines longtable and tabularx, but table contents
%    must be in a separate file.
% * "ltablex" combines tabularx and longtable - must install specially
% * "booktabs" is recommended as a way to improve the look of tables,
%   but doesn't add these capabilities.
% * "tabu" much more capable and seems to be recommended. So use that.

\usepackage{makecell}      % Enable forced line splits within a table cell
\usepackage{longtable}     % Enable multi-page tables
\usepackage{tabu}          % Multi-page tables with wrapped text in a cell

% You can find more Tex packages using commands like:
% tlmgr search --file tabu.sty
% find /usr/share/texmf-dist/ -name '*tab*'
%
%%%%%%%%%%%%%%%%%%%%%%%%%%%%%%%%%%%%%%%%%%%%%%%%%%%%%%%%%%%%%%%%%%%%%%%%%%%%
% Uncomment the next 3 lines to suppress boxes and colors on the hyperlinks
%%%%%%%%%%%%%%%%%%%%%%%%%%%%%%%%%%%%%%%%%%%%%%%%%%%%%%%%%%%%%%%%%%%%%%%%%%%%
%\hypersetup{
%colorlinks,citecolor=black,filecolor=black,linkcolor=black,urlcolor=black
%}
%
\usepackage{realref}

% Restarting page numbers: try?
%   \printglossary
%   \cleardoublepage
%   \pagenumbering{arabic}
%   \setcounter{page}{1}    ???needed
%   \include{chap1}

% not used:
% \def\R2Lurl#1#2{\mbox{\href{#1}\texttt{#2}}}

\usepackage{amssymb}

% Version 1 of book: margins: t=.4, b=.2, ll=.4, rr=.55
% \usepackage{anysize}
% % \papersize{<height>}{<width>}
% % \marginsize{<left>}{<right>}{<top>}{<bottom>}
% \papersize{9in}{6in}
% % l/r 0.6124-0.6170 works t/b 0.2418-0.3411 = 192pp. 0.2926-03118=exact
% \marginsize{0.7147in}{0.5147in}{0.4012in}{0.2012in}

\usepackage{anysize}
% \papersize{<height>}{<width>}
% \marginsize{<left>}{<right>}{<top>}{<bottom>}
\papersize{9in}{6in}
% l/r 0.85in&0.6431-0.6539 works t/b ?-?
%\marginsize{0.85in}{0.6485in}{0.55in}{0.35in}
\marginsize{0.8in}{0.65in}{0.5in}{0.3in}

% \usepackage[papersize={3.6in,4.8in},hmargin=0.1in,vmargin={0.1in,0.1in}]{geometry}  % page geometry
\usepackage{special-settings}

\raggedbottom
\makeindex

\begin{document}
% Discourage page widows and orphans:
\clubpenalty=300
\widowpenalty=300

%%%%%%% load in AMS fonts %%%%%%% % LaTeX 2.09 - obsolete in LaTeX 2e
%\input{amssym.def}
%\input{amssym.tex}
%\input{c:/texmf/tex/plain/amsfonts/amssym.def}
%\input{c:/texmf/tex/plain/amsfonts/amssym.tex}

\bibliographystyle{plain}
\pagenumbering{roman}
\pagestyle{headings}

\thispagestyle{empty}

\hfill
\vfill

\begin{center}
{\LARGE\bf Metamath} \\
\vspace{1ex}
{\large A Computer Language for Mathematical Proofs} \\
\vspace{7ex}
{\large Norman Megill} \\
\vspace{7ex}
with extensive revisions by \\
\vspace{1ex}
{\large David A. Wheeler} \\
\vspace{7ex}
% Printed date. If changing the date below, also fix the date at the beginning.
2019-05-24
\end{center}

\vfill
\hfill

\newpage
\thispagestyle{empty}

\hfill
\vfill

\begin{center}
$\sim$\ {\sc Public Domain}\ $\sim$

\vspace{2ex}
This book (including its later revisions)
has been released into the Public Domain by Norman Megill per the
Creative Commons CC0 1.0 Universal (CC0 1.0) Public Domain Dedication
(\url{https://creativecommons.org/publicdomain/zero/1.0/}).
David A. Wheeler has done the same.
The public domain release applies worldwide.  In case this is not
legally possible, the right is granted to use the work for any purpose,
without any conditions, unless such conditions are required by law.

\vspace{3ex}
Several short, attributed quotations from copyrighted works
appear in this book under the ``fair use'' provision of Section 107 of
the United States Copyright Act (Title 17 of the {\em United States
Code}).  The public-domain status of this book is not applicable to
those quotations.

\vspace{3ex}
Any trademarks used in this book are the property of their owners.

% QA76.9.L63.M??

% \vspace{1ex}
%
% \vspace{1ex}
% {\small Permission is granted to make and distribute verbatim copies of this
% book
% provided the copyright notice and this
% permission notice are preserved on all copies.}
%
% \vspace{1ex}
% {\small Permission is granted to copy and distribute modified versions of this
% book under the conditions for verbatim copying, provided that the
% entire
% resulting derived work is distributed under the terms of a permission
% notice
% identical to this one.}
%
% \vspace{1ex}
% {\small Permission is granted to copy and distribute translations of this
% book into another language, under the above conditions for modified
% versions,
% except that this permission notice may be stated in a translation
% approved by the
% author.}
%
% \vspace{1ex}
% %{\small   For a copy of the \LaTeX\ source files for this book, contact
% %the author.} \\
% \ \\
% \ \\

\vspace{7ex}
% ISBN: 1-4116-3724-0 \\
ISBN: 978-1-4116-3724-5 \\
{\ } \\
Lulu Press \\
Morrisville, North Carolina\\
USA


\hfill
\vfill

Norman Megill\\ 93 Bridge St., Lexington, MA 02421 \\
E-mail address: \texttt{nm{\char`\@}alum.mit.edu} \\
\vspace{7ex}
David A. Wheeler \\
E-mail address: \texttt{dwheeler{\char`\@}dwheeler.com} \\
% See notes added at end of Preface for revision history. \\
% For current information on the Metamath software see \\
\vspace{7ex}
\url{http://metamath.org}
\end{center}

% \newpage
% \thispagestyle{empty}
%
% \hfill
% \vfill
%
% \begin{center}
% {\it To my son Robin Dwight Megill}
% \end{center}
%
% \vfill
% \hfill
%
% \newpage

\tableofcontents
%\listoftables

\chapter*{Preface}
\markboth{PREFACE}{PREFACE}
\addcontentsline{toc}{section}{Preface}


% (For current information, see the notes added at the
% end of this preface on p.~\pageref{note2002}.)

\subsubsection{Overview}

Metamath\index{Metamath} is a computer language and an associated computer
program for archiving, verifying, and studying mathematical proofs at a very
detailed level.  The Metamath language incorporates no mathematics per se but
treats all mathematical statements as mere sequences of symbols.  You provide
Metamath with certain special sequences (axioms) that tell it what rules
of inference are allowed.  Metamath is not limited to any specific field of
mathematics.  The Metamath language is simple and robust, with an
almost total absence of hard-wired syntax, and
I\footnote{Unless otherwise noted, the words
``I,'' ``me,'' and ``my'' refer to Norman Megill\index{Megill, Norman}, while
``we,'' ``us,'' and ``our'' refer to Norman Megill and
David A. Wheeler\index{Wheeler, David A.}.}
believe that it
provides about the simplest possible framework that allows essentially all of
mathematics to be expressed with absolute rigor.

% index test
%\newcommand{\nn}[1]{#1n}
%\index{aaa@bbb}
%\index{abc!def}
%\index{abd|see{qqq}}
%\index{abe|nn}
%\index{abf|emph}
%\index{abg|(}
%\index{abg|)}

Using the Metamath language, you can build formal or mathematical
systems\index{formal system}\footnote{A formal or mathematical system consists
of a collection of symbols (such as $2$, $4$, $+$ and $=$), syntax rules that
describe how symbols may be combined to form a legal expression (called a
well-formed formula or {\em wff}, pronounced ``whiff''), some starting wffs
called axioms, and inference rules that describe how theorems may be derived
(proved) from the axioms.  A theorem is a mathematical fact such as $2+2=4$.
Strictly speaking, even an obvious fact such as this must be proved from
axioms to be formally acceptable to a mathematician.}\index{theorem}
\index{axiom}\index{rule}\index{well-formed formula (wff)} that involve
inferences from axioms.  Although a database is provided
that includes a recommended set of axioms for standard mathematics, if you
wish you can supply your own symbols, syntax, axioms, rules, and definitions.

The name ``Metamath'' was chosen to suggest that the language provides a
means for {\em describing} mathematics rather than {\em being} the
mathematics itself.  Actually in some sense any mathematical language is
metamathematical.  Symbols written on paper, or stored in a computer,
are not mathematics itself but rather a way of expressing mathematics.
For example ``7'' and ``VII'' are symbols for denoting the number seven
in Arabic and Roman numerals; neither {\em is} the number seven.

If you are able to understand and write computer programs, you should be able
to follow abstract mathematics with the aid of Metamath.  Used in conjunction
with standard textbooks, Metamath can guide you step-by-step towards an
understanding of abstract mathematics from a very rigorous viewpoint, even if
you have no formal abstract mathematics background.  By using a single,
consistent notation to express proofs, once you grasp its basic concepts
Metamath provides you with the ability to immediately follow and dissect
proofs even in totally unfamiliar areas.

Of course, just being able follow a proof will not necessarily give you an
intuitive familiarity with mathematics.  Memorizing the rules of chess does not
give you the ability to appreciate the game of a master, and knowing how the
notes on a musical score map to piano keys does not give you the ability to
hear in your head how it would sound.  But each of these can be a first step.

Metamath allows you to explore proofs in the sense that you can see the
theorem referenced at any step expanded in as much detail as you want, right
down to the underlying axioms of logic and set theory (in the case of the set
theory database provided).  While Metamath will not replace the higher-level
understanding that can only be acquired through exercises and hard work, being
able to see how gaps in a proof are filled in can give you increased
confidence that can speed up the learning process and save you time when you
get stuck.

The Metamath language breaks down a mathematical proof into its tiniest
possible parts.  These can be pieced together, like interlocking
pieces in a puzzle, only in a way that produces correct and absolutely rigorous
mathematics.

The nature of Metamath\index{Metamath} enforces very precise mathematical
thinking, similar to that involved in writing a computer program.  A crucial
difference, though, is that once a proof is verified (by the Metamath program)
to be correct, it is definitely correct; it can never have a hidden
``bug.''\index{computer program bugs}  After getting used to the kind of rigor
and accuracy provided by Metamath, you might even be tempted to
adopt the attitude that a proof should never be considered correct until it
has been verified by a computer, just as you would not completely trust a
manual calculation until you have verified it on a
calculator.

My goal
for Metamath was a system for describing and verifying
mathematics that is completely universal yet conceptually as simple as
possible.  In approaching mathematics from an axiomatic, formal viewpoint, I
wanted Metamath to be able to handle almost any mathematical system, not
necessarily with ease, but at least in principle and hopefully in practice. I
wanted it to verify proofs with absolute rigor, and for this reason Metamath
is what might be thought of as a ``compile-only'' language rather than an
algorithmic or Turing-machine language (Pascal, C, Prolog, Mathematica,
etc.).  In other words, a database written in the Metamath
language doesn't ``do'' anything; it merely exhibits mathematical knowledge
and permits this knowledge to be verified as being correct.  A program in an
algorithmic language can potentially have hidden bugs\index{computer program
bugs} as well as possibly being hard to understand.  But each token in a
Metamath database must be consistent with the database's earlier
contents according to simple, fixed rules.
If a database is verified
to be correct,\footnote{This includes
verification that a sequential list of proof steps results in the specified
theorem.} then the mathematical content is correct if the
verifier is correct and the axioms are correct.
The verification program could be incorrect, but the verification algorithm
is relatively simple (making it unlikely to be implemented incorrectly
by the Metamath program),
and there are over a dozen Metamath database verifiers
written by different people in different programming languages
(so these different verifiers can act as multiple reviewers of a database).
The most-used Metamath database, the Metamath Proof Explorer
(aka \texttt{set.mm}\index{set theory database (\texttt{set.mm})}%
\index{Metamath Proof Explorer}),
is currently verified by four different Metamath verifiers written by
four different people in four different languages, including the
original Metamath program described in this book.
The only ``bugs'' that can exist are in the statement of the axioms,
for example if the axioms are inconsistent (a famous problem shown to be
unsolvable by G\"{o}del's incompleteness theorem\index{G\"{o}del's
incompleteness theorem}).
However, real mathematical systems have very few axioms, and these can
be carefully studied.
All of this provides extraordinarily high confidence that the verified database
is in fact correct.

The Metamath program
doesn't prove theorems automatically but is designed to verify proofs
that you supply to it.
The underlying Metamath language is completely general and has no built-in,
preconceived notions about your formal system\index{formal system}, its logic
or its syntax.
For constructing proofs, the Metamath program has a Proof Assistant\index{Proof
Assistant} which helps you fill in some of a proof step's details, shows you
what choices you have at any step, and verifies the proof as you build it; but
you are still expected to provide the proof.

There are many other programs that can process or generate information
in the Metamath language, and more continue to be written.
This is in part because the Metamath language itself is very simple
and intentionally easy to automatically process.
Some programs, such as \texttt{mmj2}\index{mmj2}, include a proof assistant
that can automate some steps beyond what the Metamath program can do.
Mario Carneiro has developed an algorithm for converting proofs from
the OpenTheory interchange format, which can be translated to and from
any of the HOL family of proof languages (HOL4, HOL Light, ProofPower,
and Isabelle), into the
Metamath language.\cite{DBLP:journals/corr/Carneiro14}\index{Carneiro, Mario}.
Daniel Whalen has developed Holophrasm, which can automatically
prove many Metamath proofs using
machine learning\index{machine learning}\index{artificial intelligence}
approaches
(including multiple neural networks\index{neural networks})\cite{DBLP:journals/corr/Whalen16}\index{Whalen, Daniel}.
However,
a discussion of these other programs is beyond the scope of this book.

Like most computer languages, the Metamath\index{Metamath} language uses the
standard ({\sc ascii}) characters on a computer keyboard, so it cannot
directly represent many of the special symbols that mathematicians use.  A
useful feature of the Metamath program is its ability to convert its notation
into the \LaTeX\ typesetting language.\index{latex@{\LaTeX}}  This feature
lets you convert the {\sc ascii} tokens you've defined into standard
mathematical symbols, so you end up with symbols and formulas you are familiar
with instead of somewhat cryptic {\sc ascii} representations of them.
The Metamath program can also generate HTML\index{HTML}, making it easy
to view results on the web and to see related information by using
hypertext links.

Metamath is probably conceptually different from anything you've seen
before and some aspects may take some getting used to.  This book will
help you decide whether Metamath suits your specific needs.

\subsubsection{Setting Your Expectations}
It is important for you to understand what Metamath\index{Metamath} is and is
not.  As mentioned, the Metamath program
is {\em not} an automated theorem prover but
rather a proof verifier.  Developing a database can be tedious, hard work,
especially if you want to make the proofs as short as possible, but it becomes
easier as you build up a collection of useful theorems.  The purpose of
Metamath is simply to document existing mathematics in an absolutely rigorous,
computer-verifiable way, not to aid directly in the creation of new
mathematics.  It also is not a magic solution for learning abstract
mathematics, although it may be helpful to be able to actually see the implied
rigor behind what you are learning from textbooks, as well as providing hints
to work out proofs that you are stumped on.

As of this writing, a sizable set theory database has been developed to
provide a foundation for many fields of mathematics, but much more work would
be required to develop useful databases for specific fields.

Metamath\index{Metamath} ``knows no math;'' it just provides a framework in
which to express mathematics.  Its language is very small.  You can define two
kinds of symbols, constants\index{constant} and variables\index{variable}.
The only thing Metamath knows how to do is to substitute strings of symbols
for the variables\index{substitution!variable}\index{variable substitution} in
an expression based on instructions you provide it in a proof, subject to
certain constraints you specify for the variables.  Even the decimal
representation of a number is merely a string of certain constants (digits)
which together, in a specific context, correspond to whatever mathematical
object you choose to define for it; unlike other computer languages, there is
no actual number stored inside the computer.  In a proof, you in effect
instruct Metamath what symbol substitutions to make in previous axioms or
theorems and join a sequence of them together to result in the desired
theorem.  This kind of symbol manipulation captures the essence of mathematics
at a preaxiomatic level.

\subsubsection{Metamath and Mathematical Literature}

In advanced mathematical literature, proofs are usually presented in the form
of short outlines that often only an expert can follow.  This is partly out of
a desire for brevity, but it would also be unwise (even if it were practical)
to present proofs in complete formal detail, since the overall picture would
be lost.\index{formal proof}

A solution I envision\label{envision} that would allow mathematics to remain
acceptable to the expert, yet increase its accessibility to non-specialists,
consists of a combination of the traditional short, informal proof in print
accompanied by a complete formal proof stored in a computer database.  In an
analogy with a computer program, the informal proof is like a set of comments
that describe the overall reasoning and content of the proof, whereas the
computer database is like the actual program and provides a means for anyone,
even a non-expert, to follow the proof in as much detail as desired, exploring
it back through layers of theorems (like subroutines that call other
subroutines) all the way back to the axioms of the theory.  In addition, the
computer database would have the advantage of providing absolute assurance
that the proof is correct, since each step can be verified automatically.

There are several other approaches besides Metamath to a project such
as this.  Section~\ref{proofverifiers} discusses some of these.

To us, a noble goal would be a database with hundreds of thousands of
theorems and their computer-verifiable proofs, encompassing a significant
fraction of known mathematics and available for instant access.
These would be fully verified by multiple independently-implemented verifiers,
to provide extremely high confidence that the proofs are completely correct.
The database would allow people to investigate whatever details they were
interested in, so that they could confirm whatever portions they wished.
Whether or not Metamath is an appropriate choice remains to be seen, but in
principle we believe it is sufficient.

\subsubsection{Formalism}

Over the past fifty years, a group of French mathematicians working
collectively under the pseudonym of Bourbaki\index{Bourbaki, Nicolas} have
co-authored a series of monographs that attempt to rigorously and
consistently formalize large bodies of mathematics from foundations.  On the
one hand, certainly such an effort has its merits; on the other hand, the
Bourbaki project has been criticized for its ``scholasticism'' and
``hyperaxiomatics'' that hide the intuitive steps that lead to the results
\cite[p.~191]{Barrow}\index{Barrow, John D.}.

Metamath unabashedly carries this philosophy to its extreme and no doubt is
subject to the same kind of criticism.  Nonetheless I think that in
conjunction with conventional approaches to mathematics Metamath can serve a
useful purpose.  The Bourbaki approach is essentially pedagogic, requiring the
reader to become intimately familiar with each detail in a very large
hierarchy before he or she can proceed to the next step.  The difference with
Metamath is that the ``reader'' (user) knows that all details are contained in
its computer database, available as needed; it does not demand that the user
know everything but conveniently makes available those portions that are of
interest.  As the body of all mathematical knowledge grows larger and larger,
no one individual can have a thorough grasp of its entirety.  Metamath
can finalize and put to rest any questions about the validity of any part of it
and can make any part of it accessible, in principle, to a non-specialist.

\subsubsection{A Personal Note}
Why did I develop Metamath\index{Metamath}?  I enjoy abstract mathematics, but
I sometimes get lost in a barrage of definitions and start to lose confidence
that my proofs are correct.  Or I reach a point where I lose sight of how
anything I'm doing relates to the axioms that a theory is based on and am
sometimes suspicious that there may be some overlooked implicit axiom
accidentally introduced along the way (as happened historically with Euclidean
geometry\index{Euclidean geometry}, whose omission of Pasch's
axiom\index{Pasch's axiom} went unnoticed for 2000 years
\cite[p.~160]{Davis}!). I'm also somewhat lazy and wish to avoid the effort
involved in re-verifying the gaps in informal proofs ``left to the reader;'' I
prefer to figure them out just once and not have to go through the same
frustration a year from now when I've forgotten what I did.  Metamath provides
better recovery of my efforts than scraps of paper that I can't
decipher anymore.  But mostly I find very appealing the idea of rigorously
archiving mathematical knowledge in a computer database, providing precision,
certainty, and elimination of human error.

\subsubsection{Note on Bibliography and Index}

The Bibliography usually includes the Library of Congress classification
for a work to make it easier for you to find it in on a university
library shelf.  The Index has author references to pages where their works
are cited, even though the authors' names may not appear on those pages.

\subsubsection{Acknowledgments}

Acknowledgments are first due to my wife, Deborah (who passed away on
September 4, 1998), for critiquing the manu\-script but most of all for
her patience and support.  I also wish to thank Joe Wright, Richard
Becker, Clarke Evans, Buddha Buck, and Jeremy Henty for helpful
comments.  Any errors, omissions, and other shortcomings are of course
my responsibility.

\subsubsection{Note Added June 22, 2005}\label{note2002}

The original, unpublished version of this book was written in 1997 and
distributed via the web.  The present edition has been updated to
reflect the current Metamath program and databases, as well as more
current {\sc url}s for Internet sites.  Thanks to Josh
Purinton\index{Purinton, Josh}, One Hand
Clapping, Mel L.\ O'Cat, and Roy F. Longton for pointing out
typographical and other errors.  I have also benefitted from numerous
discussions with Raph Levien\index{Levien, Raph}, who has extended
Metamath's philosophy of rigor to result in his {\em
Ghilbert}\index{Ghilbert} proof language (\url{http://ghilbert.org}).

Robert (Bob) Solovay\index{Solovay, Robert} communicated a new result of
A.~R.~D.~Mathias on the system of Bourbaki, and the text has been
updated accordingly (p.~\pageref{bourbaki}).

Bob also pointed out a clarification of the literature regarding
category theory and inaccessible cardinals\index{category
theory}\index{cardinal, inaccessible} (p.~\pageref{categoryth}),
and a misleading statement was removed from the text.  Specifically,
contrary to a statement in previous editions, it is possible to express
``There is a proper class of inaccessible cardinals'' in the language of
ZFC.  This can be done as follows:  ``For every set $x$ there is an
inaccessible cardinal $\kappa$ such that $\kappa$ is not in $x$.''
Bob writes:\footnote{Private communication, Nov.~30, 2002.}
\begin{quotation}
     This axiom is how Grothendieck presents category theory.  To each
inaccessible cardinal $\kappa$ one associates a Grothendieck universe
\index{Grothendieck, Alexander} $U(\kappa)$.  $U(\kappa)$ consists of
those sets which lie in a transitive set of cardinality less than
$\kappa$.  Instead of the ``category of all groups,'' one works relative
to a universe [considering the category of groups of cardinality less
than $\kappa$].  Now the category whose objects are all categories
``relative to the universe $U(\kappa)$'' will be a category not
relative to this universe but to the next universe.

     All of the things category theorists like to do can be done in this
framework.  The only controversial point is whether the Grothen\-dieck
axiom is too strong for the needs of category theorists.  Mac Lane
\index{Mac Lane, Saunders} argues that ``one universe is enough'' and
Feferman\index{Feferman, Solomon} has argued that one can get by with
ordinary ZFC.  I don't find Feferman's arguments persuasive.  Mac Lane
may be right, but when I think about category theory I do it \`{a} la
Grothendieck.

        By the way Mizar\index{Mizar} adds the axiom ``there is a proper
class of inaccessibles'' precisely so as to do category theory.
\end{quotation}

The most current information on the Metamath program and databases can
always be found at \url{http://metamath.org}.


\subsubsection{Note Added June 24, 2006}\label{note2006}

The Metamath spec was restricted slightly to make parsers easier to
write.  See the footnote on p.~\pageref{namespace}.

%\subsubsection{Note Added July 24, 2006}\label{note2006b}
\subsubsection{Note Added March 10, 2007}\label{note2006b}

I am grateful to Anthony Williams\index{Williams, Anthony} for writing
the \LaTeX\ package called {\tt realref.sty} and contributing it to the
public domain.  This package allows the internal hyperlinks in a {\sc
pdf} file to anchor to specific page numbers instead of just section
titles, making the navigation of the {\sc pdf} file for this book much
more pleasant and ``logical.''

A typographical error found by Martin Kiselkov was corrected.
A confusing remark about unification was deleted per suggestion of
Mel O'Cat.

\subsubsection{Note Added May 27, 2009}\label{note2009}

Several typos found by Kim Sparre were corrected.  A note was added that
the Poincar\'{e} conjecture has been proved (p.~\pageref{poincare}).

\subsubsection{Note Added Nov. 17, 2014}\label{note2014}

The statement of the Schr\"{o}der--Bernstein theorem was corrected in
Section~\ref{trust}.  Thanks to Bob Solovay for pointing out the error.

\subsubsection{Note Added May 25, 2016}\label{note2016}

Thanks to Jerry James for correcting 16 typos.

\subsubsection{Note Added February 25, 2019}\label{note201902}

David A. Wheeler\index{Wheeler, David A.}
made a large number of improvements and updates,
in coordination with Norman Megill.
The predicate calculus axioms were renumbered, and the text makes
it clear that they are based on Tarski's system S2;
the one slight deviation in axiom ax-6 is explained and justified.
The real and complex number axioms were modified to be consistent with
\texttt{set.mm}\index{set theory database (\texttt{set.mm})}%
\index{Metamath Proof Explorer}.
Long-awaited specification changes ``1--8'' were made,
which clarified previously ambiguous points.
Some errors in the text involving \texttt{\$f} and
\texttt{\$d} statements were corrected (the spec was correct, but
the in-book explanations unintentionally contradicted the spec).
We now have a system for automatically generating narrow PDFs,
so that those with smartphones can have easy access to the current
version of this document.
A new section on deduction was added;
it discusses the standard deduction theorem,
the weak deduction theorem,
deduction style, and natural deduction.
Many minor corrections (too numerous to list here) were also made.

\subsubsection{Note Added March 7, 2019}\label{note201903}

This added a description of the Matamath language syntax in
Extended Backus--Naur Form (EBNF)\index{Extended Backus--Naur Form}\index{EBNF}
in Appendix \ref{BNF}, added a brief explanation about typecodes,
inserted more examples in the deduction section,
and added a variety of smaller improvements.

\subsubsection{Note Added April 7, 2019}\label{note201904}

This version clarified the proper substitution notation, improved the
discussion on the weak deduction theorem and natural deduction,
documented the \texttt{undo} command, updated the information on
\texttt{write source}, changed the typecode
from \texttt{set} to \texttt{setvar} to be consistent with the current
version of \texttt{set.mm}, added more documentation about comment markup
(e.g., documented how to create headings), and clarified the
differences between various assertion forms (in particular deduction form).

\subsubsection{Note Added May 20, 2019}\label{note201905}

This version fixes various small issues and added specific examples
of forms (deduction form, inference form, and closed form).

\chapter{Introduction}
\pagenumbering{arabic}

\begin{quotation}
  {\em {\em I.M.:}  No, no.  There's nothing subjective about it!  Everybody
knows what a proof is.  Just read some books, take courses from a competent
mathematician, and you'll catch on.

{\em Student:}  Are you sure?

{\em I.M.:}  Well---it is possible that you won't, if you don't have any
aptitude for it.  That can happen, too.

{\em Student:}  Then {\em you} decide what a proof is, and if I don't learn
to decide in the same way, you decide I don't have any aptitude.

{\em I.M.:}  If not me, then who?}
    \flushright\sc  ``The Ideal Mathematician''
    \index{Davis, Phillip J.}
    \footnote{\cite{Davis}, p.~40.}\\
\end{quotation}

In the past century, brilliant mathematicians have discovered almost
unimaginably profound results that rank among the crowning intellectual
achievements of mankind.  However, there is a sense in which modern abstract
mathematics is behind the times, stuck in an era before computers existed.
While no one disputes the remarkable results that have been achieved,
communicating these results in a precise way to the uninitiated is virtually
impossible.  To describe these results, a terse informal language is used which
despite its elegance is very difficult to learn.  This informal language is not
imprecise, far from it, but rather it often has omitted detail
and symbols with hidden context that are
implicitly understood by an expert but few others.  Extremely complex technical
meanings are associated with innocent-sounding English words such as
``compact'' and ``measurable'' that barely hint at what is actually being
said.  Anyone who does not keep the precise technical meaning constantly in
mind is bound to fail, and acquiring the ability to do this can be achieved
only through much practice and hard work.  Only the few who complete the
painful learning experience can join the small in-group of pure
mathematicians.  The informal language effectively cuts off the true nature of
their knowledge from most everyone else.

Metamath\index{Metamath} makes abstract mathematics more concrete.  It allows
a computer to keep track of the complexity associated with each word or symbol
with absolute rigor.  You can explore this complexity at your leisure, to
whatever degree you desire.  Whether or not you believe that concepts such as
infinity actually ``exist'' outside of the mind, Metamath lets you get to the
foundation for what's really being said.

Metamath also enables completely rigorous and thorough proof verification.
Its language is simple enough so that you
don't have to rely on the authority of experts but can verify the results
yourself, step by step.  If you want to attempt to derive your own results,
Metamath will not let you make a mistake in reasoning.
Even professional mathematicians make mistakes; Metamath makes it possible
to thoroughly verify that proofs are correct.

Metamath\index{Metamath} is a computer language and an associated computer
program for archiving, verifying, and studying mathematical proofs at a very
detailed level.
The Metamath language
describes formal\index{formal system} mathematical
systems and expresses proofs of theorems in those systems.  Such a language
is called a metalanguage\index{metalanguage} by mathematicians.
The Metamath program is a computer program that verifies
proofs expressed in the Metamath language.
The Metamath program does not have the built-in
ability to make logical inferences; it just makes a series of symbol
substitutions according to instructions given to it in a proof
and verifies that the result matches the expected theorem.  It makes logical
inferences based only on rules of logic that are contained in a set of
axioms\index{axiom}, or first principles, that you provide to it as the
starting point for proofs.

The complete specification of the Metamath language is only four pages long
(Section~\ref{spec}, p.~\pageref{spec}).  Its simplicity may at first make you
wonder how it can do much of anything at all.  But in fact the kinds of
symbol manipulations it performs are the ones that are implicitly done in all
mathematical systems at the lowest level.  You can learn it relatively quickly
and have complete confidence in any mathematical proof that it verifies.  On
the other hand, it is powerful and general enough so that virtually any
mathematical theory, from the most basic to the deeply abstract, can be
described with it.

Although in principle Metamath can be used with any
kind of mathematics, it is best suited for abstract or ``pure'' mathematics
that is mostly concerned with theorems and their proofs, as opposed to the
kind of mathematics that deals with the practical manipulation of numbers.
Examples of branches of pure mathematics are logic\index{logic},\footnote{Logic
is the study of statements that are universally true regardless of the objects
being described by the statements.  An example is the statement, ``if $P$
implies $Q$, then either $P$ is false or $Q$ is true.''} set theory\index{set
theory},\footnote{Set theory is the study of general-purpose mathematical objects called
``sets,'' and from it essentially all of mathematics can be derived.  For
example, numbers can be defined as specific sets, and their properties
can be explored using the tools of set theory.} number theory\index{number
theory},\footnote{Number theory deals with the properties of positive and
negative integers (whole numbers).} group theory\index{group
theory},\footnote{Group theory studies the properties of mathematical objects
called groups that obey a simple set of axioms and have properties of symmetry
that make them useful in many other fields.} abstract algebra\index{abstract
algebra},\footnote{Abstract algebra includes group theory and also studies
groups with additional properties that qualify them as ``rings'' and
``fields.''  The set of real numbers is a familiar example of a field.},
analysis\index{analysis} \index{real and complex numbers}\footnote{Analysis is
the study of real and complex numbers.} and
topology\index{topology}.\footnote{One area studied by topology are properties
that remain unchanged when geometrical objects undergo stretching
deformations; for example a doughnut and a coffee cup each have one hole (the
cup's hole is in its handle) and are thus considered topologically
equivalent.  In general, though, topology is the study of abstract
mathematical objects that obey a certain (surprisingly simple) set of axioms.
See, for example, Munkres \cite{Munkres}\index{Munkres, James R.}.} Even in
physics, Metamath could be applied to certain branches that make use of
abstract mathematics, such as quantum logic\index{quantum logic} (used to study
aspects of quantum mechanics\index{quantum mechanics}).

On the other hand, Metamath\index{Metamath} is less suited to applications
that deal primarily with intensive numeric computations.  Metamath does not
have any built-in representation of numbers\index{Metamath!representation of
numbers}; instead, a specific string of symbols (digits) must be syntactically
constructed as part of any proof in which an ordinary number is used.  For
this reason, numbers in Metamath are best limited to specific constants that
arise during the course of a theorem or its proof.  Numbers are only a tiny
part of the world of abstract mathematics.  The exclusion of built-in numbers
was a conscious decision to help achieve Metamath's simplicity, and there are
other software tools if you have different mathematical needs.
If you wish to quickly solve algebraic problems, the computer algebra
programs\index{computer algebra system} {\sc
macsyma}\index{macsyma@{\sc macsyma}}, Mathematica\index{Mathematica}, and
Maple\index{Maple} are specifically suited to handling numbers and
algebra efficiently.
If you wish to simply calculate numeric or matrix expressions easily,
tools such as Octave\index{Octave} may be a better choice.

After learning Metamath's basic statement types, any
tech\-ni\-cal\-ly ori\-ent\-ed person, mathematician or not, can
immediately trace
any theorem proved in the language as far back as he or she wants, all the way
to the axioms on which the theorem is based.  This ability suggests a
non-traditional way of learning about pure mathematics.  Used in conjunction
with traditional methods, Metamath could make pure mathematics accessible to
people who are not sufficiently skilled to figure out the implicit detail in
ordinary textbook proofs.  Once you learn the axioms of a theory, you can have
complete confidence that everything you need to understand a proof you are
studying is all there, at your beck and call, allowing you to focus in on any
proof step you don't understand in as much depth as you need, without worrying
about getting stuck on a step you can't figure out.\footnote{On the other
hand, writing proofs in the Metamath language is challenging, requiring
a degree of rigor far in excess of that normally taught to students.  In a
classroom setting, I doubt that writing Metamath proofs would ever replace
traditional homework exercises involving informal proofs, because the time
needed to work out the details would not allow a course to
cover much material.  For students who have trouble grasping the implied rigor
in traditional material, writing a few simple proofs in the Metamath language
might help clarify fuzzy thought processes.  Although somewhat difficult at
first, it eventually becomes fun to do, like solving a puzzle, because of the
instant feedback provided by the computer.}

Metamath\index{Metamath} is probably unlike anything you have
encountered before.  In this first chapter we will look at the philosophy and
use of computers in mathematics in order to better understand the motivation
behind Metamath.  The material in this chapter is not required in order to use
Metamath.  You may skip it if you are impatient, but I hope you will find it
educational and enjoyable.  If you want to start experimenting with the
Metamath program right away, proceed directly to Chapter~\ref{using}
(p.~\pageref{using}).  To
learn the Metamath language, skim Chapter~\ref{using} then proceed to
Chapter~\ref{languagespec} (p.~\pageref{languagespec}).

\section{Mathematics as a Computer Language}

\begin{quote}
  {\em The study of mathematics is apt to commence in
dis\-ap\-point\-ment.\ldots \\
We are told that by its aid the stars are weighted
and the billions of molecules in a drop of water are counted.  Yet, like the
ghost of Hamlet's father, this great science eludes the efforts of our mental
weapons to grasp it.}
  \flushright\sc  Alfred North Whitehead\footnote{\cite{Whitehead}, ch.\ 1.}\\
\end{quote}\index{Whitehead, Alfred North}

\subsection{Is Mathematics ``User-Friendly''?}

Suppose you have no formal training in abstract mathematics.  But popular
books you've read offer tempting glimpses of this world filled with profound
ideas that have stirred the human spirit.  You are not satisfied with the
informal, watered-down descriptions you've read but feel it is important to
grasp the underlying mathematics itself to understand its true meaning. It's
not practical to go back to school to learn it, though; you don't want to
dedicate years of your life to it.  There are many important things in life,
and you have to set priorities for what's important to you.  What would happen
if you tried to pursue it on your own, in your spare time?

After all, you were able to learn a computer programming language such as
Pascal on your own without too much difficulty, even though you had no formal
training in computers.  You don't claim to be an expert in software design,
but you can write a passable program when necessary to suit your needs.  Even
more important, you know that you can look at anyone else's Pascal program, no
matter how complex, and with enough patience figure out exactly how it works,
even though you are not a specialist.  Pascal allows you do anything that a
computer can do, at least in principle.  Thus you know you have the ability,
in principle, to follow anything that a computer program can do:  you just
have to break it down into small enough pieces.

Here's an imaginary scenario of what might happen if you na\-ive\-ly a\-dopted
this same view of abstract mathematics and tried to pick it up on your own, in
a period of time comparable to, say, learning a computer programming
language.

\subsubsection{A Non-Mathematician's Quest for Truth}

\begin{quote}
  {\em \ldots my daughters have been studying (chemistry) for several
se\-mes\-ters, think they have learned differential and integral calculus in
school, and yet even today don't know why $x\cdot y=y\cdot x$ is true.}
  \flushright\sc  Edmund Landau\footnote{\cite{Landau}, p.~vi.}\\
\end{quote}\index{Landau, Edmund}

\begin{quote}
  {\em Minus times minus is plus,\\
The reason for this we need not discuss.}
  \flushright\sc W.\ H.\ Auden\footnote{As quoted in \cite{Guillen}, p.~64.}\\
\end{quote}\index{Auden, W.\ H.}\index{Guillen, Michael}

We'll suppose you are a technically oriented professional, perhaps an engineer, a
computer programmer, or a physicist, but probably not a mathematician.  You
consider yourself reasonably intelligent.  You did well in school, learning a
variety of methods and techniques in practical mathematics such as calculus and
differential equations.  But rarely did your courses get into anything
resembling modern abstract mathematics, and proofs were something that appeared
only occasionally in your textbooks, a kind of necessary evil that was
supposed to convince you of a certain key result.  Most of your
homework consisted of exercises that gave you practice in the techniques, and
you were hardly ever asked to come up with a proof of your own.

You find yourself curious about advanced, abstract mathematics.  You are
driven by an inner conviction that it is important to understand and
appreciate some of the most profound knowledge discovered by mankind.  But it
seems very hard to learn, something that only certain gifted longhairs can
access and understand.  You are frustrated that it seems forever cut off from
you.

Eventually your curiosity drives you to do something about it.
You set for yourself a goal of ``really'' understanding mathematics:  not just
how to manipulate equations in algebra or calculus according to cookbook
rules, but rather to gain a deep understanding of where those rules come from.
In fact, you're not thinking about this kind of ordinary mathematics at all,
but about a much more abstract, ethereal realm of pure mathematics, where
famous results such as G\"{o}del's incompleteness theorem\index{G\"{o}del's
incompleteness theorem} and Cantor's different kinds of infinities
reside.

You have probably read a number of popular books, with titles like {\em
Infinity and the Mind} \cite{Rucker}\index{Rucker, Rudy}, on topics such as
these.  You found them inspiring but at the same time somewhat
unsatisfactory.  They gave you a general idea of what these results are about,
but if someone asked you to prove them, you wouldn't have the faintest idea of
where to begin.   Sure, you could give the same overall outline that you
learned from the popular books; and in a general sort of way, you do have an
understanding.  But deep down inside, you know that there is a rigor that is
missing, that probably there are many subtle steps and pitfalls along the way,
and ultimately it seems you have to place your trust in the experts in the
field.  You don't like this; you want to be able to verify these results for
yourself.

So where do you go next?  As a first step, you decide to look up some of the
original papers on the theorems you are curious about, or better, obtain some
standard textbooks in the field.  You look up a theorem you want to
understand.  Sure enough, it's there, but it's expressed with strange
terms and odd symbols that mean absolutely nothing to you.  It might as well be written in
a foreign language you've never seen before, whose symbols are totally alien.
You look at the proof, and you haven't the foggiest notion what each step
means, much less how one step follows from another.  Well, obviously you have
a lot to learn if you want to understand this stuff.

You feel that you could probably understand it by
going back to college for another three to six years and getting a math
degree.  But that does not fit in with your career and the other things in
your life and would serve no practical purpose.  You decide to seek a quicker
path.  You figure you'll just trace your way back to the beginning, step by
step, as you would do with a computer program, until you understand it.  But
you quickly find that this is not possible, since you can't even understand
enough to know what you have to trace back to.

Maybe a different approach is in order---maybe you should start at the
beginning and work your way up.  First, you read the introduction to the book
to find out what the prerequisites are.  In a similar fashion, you trace your
way back through two or three more books, finally arriving at one that seems
to start at a beginning:  it lists the axioms of arithmetic.  ``Aha!'' you
naively think, ``This must be the starting point, the source of all mathematical
knowledge.'' Or at least the starting point for mathematics dealing with
numbers; you have to start somewhere and have no idea what the starting point
for other mathematics would be.  But the word ``axioms'' looks promising.  So
you eagerly read along and work through some elementary exercises at the
beginning of the book.  You feel vaguely bothered:  these
don't seem like axioms at all, at least not in the sense that you want to
think of axioms.  Axioms imply a starting point from which everything else can
be built up, according to precise rules specified in the axiom system.  Even
though you can understand the first few proofs in an informal way,
and are able to do some of the
exercises, it's hard to pin down precisely what the
rules are.   Sure, each step seems to follow logically from the others, but
exactly what does that mean?  Is the ``logic'' just a matter of common sense,
something vague that we all understand but can never quite state precisely?

You've spent a number of years, off and on, programming computers, and you
know that in the case of computer languages there is no question of what the
rules are---they are precise and crystal clear.  If you follow them, your
program will work, and if you don't, it won't.  No matter how complex a
program, it can always be broken down into simpler and simpler pieces, until
you can ultimately identify the bits that are moved around to perform a
specific function.  Some programs might require a lot of perseverance to
accomplish this, but if you focus on a specific portion of it, you don't even
necessarily have to know how the rest of it works. Shouldn't there be an
analogy in mathematics?

You decide to apply the ultimate test:  you ask yourself how a computer could
verify or ensure that the steps in these proofs follow from one another.
Certainly mathematics must be at least as precisely defined as a computer
language, if not more so; after all, computer science itself is based on it.
If you can get a computer to verify these proofs, then you should also be
able, in principle, to understand them yourself in a very crystal clear,
precise way.

You're in for a surprise:  you can conceive of no way to convert the
proofs, which are in English, to a form that the computer can understand.
The proofs are filled with phrases such as ``assume there exists a unique
$x$\ldots'' and ``given any $y$, let $z$ be the number such that\ldots''  This
isn't the kind of logic you are used to in computer programming, where
everything, even arithmetic, reduces to Boolean ones and zeroes if you care to
break it down sufficiently.  Even though you think you understand the proofs,
there seems to be some kind of higher reasoning involved rather than precise
rules that define how you manipulate the symbols in the axioms.  Whatever it
is, it just isn't obvious how you would express it to a computer, and the more
you think about it, the more puzzled and confused you get, to the point where
you even wonder whether {\em you} really understand it.  There's a lot more to
these axioms of arithmetic than meets the eye.

Nobody ever talked about this in school in your applied math and engineering
courses.  You just learned the rules they gave you, not quite understanding
how or why they worked, sometimes vaguely suspicious or uncertain of them, and
through homework problems and osmosis learned how to present solutions that
satisfied the instructor and earned you an ``A.''  Rarely did you actually
``prove'' anything in a rigorous way, and the math majors who did do stuff
like that seemed to be in a different world.

Of course, there are computer algebra programs that can do mathematics, and
rather impressively.  They can instantly solve the integrals that you
struggled with in freshman calculus, and do much, much more.  But when you
look at these programs, what you see is a big collection of algorithms and
techniques that evolved and were added to over time, along with some basic
software that manipulates symbols.  Each algorithm that is built in is the
result of someone's theorem whose proof is omitted; you just have to trust the
person who proved it and the person who programmed it in and hope there are no
bugs.\index{computer program bugs}  Somehow this doesn't seem to be the
essence of mathematics.  Although computer algebra systems can generate
theorems with amazing speed, they can't actually prove a single one of them.

After some puzzlement, you revisit some popular books on what mathematics is
all about.  Somewhere you read that all of mathematics is actually derived
from something called ``set theory.''  This is a little confusing, because
nowhere in the book that presented the axioms of arithmetic was there any
mention of set theory, or if there was, it seemed to be just a tool that helps
you describe things better---the set of even numbers, that sort of thing.  If
set theory is the basis for all mathematics, then why are additional axioms
needed for arithmetic?

Something is wrong but you're not sure what.  One of your friends is a pure
mathematician.  He knows he is unable to communicate to you what he does for a
living and seems to have little interest in trying.  You do know that for him,
proofs are what mathematics is all about. You ask him what a proof is, and he
essentially tells you that, while of course it's based on logic, really it's
something you learn by doing it over and over until you pick it up.  He refers
you to a book, {\em How to Read and Do Proofs} \cite{Solow}.\index{Solow,
Daniel}  Although this book helps you understand traditional informal proofs,
there is still something missing you can't seem to pin down yet.

You ask your friend how you would go about having a computer verify a proof.
At first he seems puzzled by the question; why would you want to do that?
Then he says it's not something that would make any sense to do, but he's
heard that you'd have to break the proof down into thousands or even millions
of individual steps to do such a thing, because the reasoning involved is at
such a high level of abstraction.  He says that maybe it's something you could
do up to a point, but the computer would be completely impractical once you
get into any meaningful mathematics.  There, the only way you can verify a
proof is by hand, and you can only acquire the ability to do this by
specializing in the field for a couple of years in grad school.  Anyway, he
thinks it all has to do with set theory, although he has never taken a formal
course in set theory but just learned what he needed as he went along.

You are intrigued and amazed.  Apparently a mathematician can grasp as a
single concept something that would take a computer a thousand or a million
steps to verify, and have complete confidence in it.  Each one of these
thousand or million steps must be absolutely correct, or else the whole proof
is meaningless.  If you added a million numbers by hand, would you trust the
result?  How do you really know that all these steps are correct, that there
isn't some subtle pitfall in one of these million steps, like a bug in a
computer program?\index{computer program bugs}  After all, you've read that
famous mathematicians have occasionally made mistakes, and you certainly know
you've made your share on your math homework problems in school.

You recall the analogy with a computer program.  Sure, you can understand what
a large computer program such as a word processor does, as a single high-level
concept or a small set of such concepts, but your ability to understand it in
no way ensures that the program is correct and doesn't have hidden bugs.  Even
if you wrote the program yourself you can't really know this; most large
programs that you've written have had bugs that crop up at some later date, no
matter how careful you tried to be while writing them.

OK, so now it seems the reason you can't figure out how to make a
computer verify proofs is because each step really corresponds to a
million small steps.  Well, you say, a computer can do a million
calculations in a second, so maybe it's still practical to do.  Now the
puzzle becomes how to figure out what the million steps are that each
English-language step corresponds to.  Your mathematician friend hasn't
a clue, but suggests that maybe you would find the answer by studying
set theory.  Actually, your friend thinks you're a little off the wall
for even wondering such a thing.  For him, this is not what mathematics
is all about.

The subject of set theory keeps popping up, so you decide it's
time to look it up.

You decide to start off on a careful footing, so you start reading a couple of
very elementary books on set theory.  A lot of it seems pretty obvious, like
intersections, subsets, and Venn diagrams.  You thumb through one of the
books; nowhere is anything about axioms mentioned. The other book relegates to
an appendix a brief discussion that mentions a set of axioms called
``Zermelo--Fraenkel set theory''\index{Zermelo--Fraenkel set theory} and states
them in English.  You look at them and have no idea what they really mean or
what you can do with them.  The comments in this appendix say that the purpose
of mentioning them is to expose you to the idea, but imply that they are not
necessary for basic understanding and that they are really the subject matter
of advanced treatments where fine points such as a certain paradox (Russell's
paradox\index{Russell's paradox}\footnote{Russell's paradox assumes that there
exists a set $S$ that is a collection of all sets that don't contain
themselves.  Now, either $S$ contains itself or it doesn't.  If it contains
itself, it contradicts its definition.  But if it doesn't contain itself, it
also contradicts its definition.  Russell's paradox is resolved in ZF set
theory by denying that such a set $S$ exists.}) are resolved.  Wait a
minute---shouldn't the axioms be a starting point, not an ending point?  If
there are paradoxes that arise without the axioms, how do you know you won't
stumble across one accidentally when using the informal approach?

And nowhere do these books describe how ``all of mathematics can be
derived from set theory'' which by now you've heard a few times.

You find a more advanced book on set theory.  This one actually lists the
axioms of ZF set theory in plain English on page one.  {\em Now} you think
your quest has ended and you've finally found the source of all mathematical
knowledge; you just have to understand what it means.  Here, in one place, is
the basis for all of mathematics!  You stare at the axioms in awe, puzzle over
them, memorize them, hoping that if you just meditate on them long enough they
will become clear.  Of course, you haven't the slightest idea how the rest of
mathematics is ``derived'' from them; in particular, if these are the axioms
of mathematics, then why do arithmetic, group theory, and so on need their own
axioms?

You start reading this advanced book carefully, pondering the meaning of every
word, because by now you're really determined to get to the bottom of this.
The first thing the book does is explain how the axioms came about, which was
to resolve Russell's paradox.\index{Russell's paradox}  In fact that seems to
be the main purpose of their existence; that they supposedly can be used to
derive all of mathematics seems irrelevant and is not even mentioned.  Well,
you go on.  You hope the book will explain to you clearly, step by step, how
to derive things from the axioms.  After all, this is the starting point of
mathematics, like a book that explains the basics of a computer programming
language.  But something is missing.  You find you can't even understand the
first proof or do the first exercise.  Symbols such as $\exists$ and $\forall$
permeate the page without any mention of where they came from or how to
manipulate them; the author assumes you are totally familiar with them and
doesn't even tell you what they mean.  By now you know that $\exists$ means
``there exists'' and $\forall$ means ``for all,'' but shouldn't the rules for
manipulating these symbols be part of the axioms?  You still have no idea
how you could even describe the axioms to a computer.

Certainly there is something much different here from the technical
literature you're used to reading.  A computer language manual almost
always explains very clearly what all the symbols mean, precisely what
they do, and the rules used for combining them, and you work your way up
from there.

After glancing at four or five other such books, you come to the realization
that there is another whole field of study that you need just to get to the
point at which you can understand the axioms of set theory.  The field is
called ``logic.''  In fact, some of the books did recommend it as a
prerequisite, but it just didn't sink in.  You assumed logic was, well, just
logic, something that a person with common sense intuitively understood.  Why
waste your time reading boring treatises on symbolic logic, the manipulation
of 1's and 0's that computers do, when you already know that?  But this is a
different kind of logic, quite alien to you.  The subject of {\sc nand} and
{\sc nor} gates is not even touched upon or in any case has to do with only a
very small part of this field.

So your quest continues.  Skimming through the first couple of introductory
books, you get a general idea of what logic is about and what quantifiers
(``for all,'' ``there exists'') mean, but you find their examples somewhat
trivial and mildly annoying (``all dogs are animals,'' ``some animals are
dogs,'' and such).  But all you want to know is what the rules are for
manipulating the symbols so you can apply them to set theory.  Some formulas
describing the relationships among quantifiers ($\exists$ and $\forall$) are
listed in tables, along with some verbal reasoning to justify them.
Presumably, if you want to find out if a formula is correct, you go through
this same kind of mental reasoning process, possibly using images of dogs and
animals. Intuitively, the formulas seem to make sense.  But when you ask
yourself, ``What are the rules I need to get a computer to figure out whether
this formula is correct?'', you still don't know.  Certainly you don't ask the
computer to imagine dogs and animals.

You look at some more advanced logic books.  Many of them have an introductory
chapter summarizing set theory, which turns out to be a prerequisite.  You
need logic to understand set theory, but it seems you also need set theory to
understand logic!  These books jump right into proving rather advanced
theorems about logic, without offering the faintest clue about where the logic
came from that allows them to prove these theorems.

Luckily, you come across an elementary book of logic that, halfway through,
after the usual truth tables and metaphors, presents in a clear, precise way
what you've been looking for all along: the axioms!  They're divided into
propositional calculus (also called sentential logic) and predicate calculus
(also called first-order logic),\index{first-order logic} with rules so simple
and crystal clear that now you can finally program a computer to understand
them.  Indeed, they're no harder than learning how to play a game of chess.
As far as what you seem to need is concerned, the whole book could have been
written in five pages!

{\em Now} you think you've found the ultimate source of mathematical
truth.  So---the axioms of mathematics consist of these axioms of logic,
together with the axioms of ZF set theory. (By now you've also been able to
figure out how to translate the ZF axioms from English into the
actual symbols of logic which you can now manipulate according to
precise, easy-to-understand rules.)

Of course, you still don't understand how ``all of mathematics can be
derived from set theory,'' but maybe this will reveal itself in due
course.

You eagerly set out to program the axioms and rules into a computer and start
to look at the theorems you will have to prove as the logic is developed.  All
sorts of important theorems start popping up:  the deduction
theorem,\index{deduction theorem} the substitution theorem,\index{substitution
theorem} the completeness theorem of propositional calculus,\index{first-order
logic!completeness} the completeness theorem of predicate calculus.  Uh-oh,
there seems to be trouble.  They all get harder and harder, and not one of
them can be derived with the axioms and rules of logic you've just been
handed.  Instead, they all require ``metalogic'' for their proofs, a kind of
mixture of logic and set theory that allows you to prove things {\em about}
the axioms and theorems of logic rather than {\em with} them.

You plow ahead anyway.  A month later, you've spent much of your
free time getting the computer to verify proofs in propositional calculus.
You've programmed in the axioms, but you've also had to program in the
deduction theorem, the substitution theorem, and the completeness theorem of
propositional calculus, which by now you've resigned yourself to treating as
rather complex additional axioms, since they can't be proved from the axioms
you were given.  You can now get the computer to verify and even generate
complete, rigorous, formal proofs\index{formal proof}.  Never mind that they
may have 100,000 steps---at least now you can have complete, absolute
confidence in them.  Unfortunately, the only theorems you have proved are
pretty trivial and you can easily verify them in a few minutes with truth
tables, if not by inspection.

It looks like your mathematician friend was right.  Getting the computer to do
serious mathematics with this kind of rigor seems almost hopeless.  Even
worse, it seems that the further along you get, the more ``axioms'' you have
to add, as each new theorem seems to involve additional ``metamathematical''
reasoning that hasn't been formalized, and none of it can be derived from the
axioms of logic.  Not only do the proofs keep growing exponentially as you get
further along, but the program to verify them keeps getting bigger and bigger
as you program in more ``metatheorems.''\index{metatheorem}\footnote{A
metatheorem is usually a statement that is too general to be directly provable
in a theory.  For example, ``if $n_1$, $n_2$, and $n_3$ are integers, then
$n_1+n_2+n_3$ is an integer'' is a theorem of number theory.  But ``for any
integer $k > 1$, if $n_1, \ldots, n_k$ are integers, then $n_1+\ldots +n_k$ is
an integer'' is a metatheorem, in other words a family of theorems, one for
every $k$.  The reason it is not a theorem is that the general sum $n_1+\ldots
+n_k$ (as a function of $k$) is not an operation that can be defined directly
in number theory.} The bugs\index{computer program bugs} that have cropped up
so far have already made you start to lose faith in the rigor you seem to have
achieved, and you know it's just going to get worse as your program gets larger.

\subsection{Mathematics and the Non-Specialist}

\begin{quote}
  {\em A real proof is not checkable by a machine, or even by any mathematician
not privy to the gestalt, the mode of thought of the particular field of
mathematics in which the proof is located.}
  \flushright\sc  Davis and Hersh\index{Davis, Phillip J.}
  \footnote{\cite{Davis}, p.~354.}\\
\end{quote}

The bulk of abstract or theoretical mathematics is ordinarily outside
the reach of anyone but a few specialists in each field who have completed
the necessary difficult internship in order to enter its coterie.  The
typical intelligent layperson has no reasonable hope of understanding much of
it, nor even the specialist mathematician of understanding other fields.  It
is like a foreign language that has no dictionary to look up the translation;
the only way you can learn it is by living in the country for a few years.  It
is argued that the effort involved in learning a specialty is a necessary
process for acquiring a deep understanding.  Of course, this is almost certainly
true if one is to make significant contributions to a field; in particular,
``doing'' proofs is probably the most important part of a mathematician's
training.  But is it also necessary to deny outsiders access to it?  Is it
necessary that abstract mathematics be so hard for a layperson to grasp?

A computer normally is of no help whatsoever.  Most published proofs are
actually just series of hints written in an informal style that requires
considerable knowledge of the field to understand.  These are the ``real
proofs'' referred to by Davis and Hersh.\index{informal proof}  There is an
implicit understanding that, in principle, such a proof could be converted to
a complete formal proof\index{formal proof}.  However, it is said that no one
would ever attempt such a conversion, even if they could, because that would
presumably require millions of steps (Section~\ref{dream}).  Unfortunately the
informal style automatically excludes the understanding of the proof
by anyone who hasn't gone through the necessary apprenticeship. The
best that the intelligent layperson can do is to read popular books about deep
and famous results; while this can be helpful, it can also be misleading, and
the lack of detail usually leaves the reader with no ability whatsoever to
explore any aspect of the field being described.

The statements of theorems often use sophisticated notation that makes them
inaccessible to the non-specialist.  For a non-specialist who wants to achieve
a deeper understanding of a proof, the process of tracing definitions and
lemmas back through their hierarchy\index{hierarchy} quickly becomes confusing
and discouraging.  Textbooks are usually written to train mathematicians or to
communicate to people who are already mathematicians, and large gaps in proofs
are often left as exercises to the reader who is left at an impasse if he or
she becomes stuck.

I believe that eventually computers will enable non-specialists and even
intelligent laypersons to follow almost any mathematical proof in any field.
Metamath is an attempt in that direction.  If all of mathematics were as
easily accessible as a computer programming language, I could envision
computer programmers and hobbyists who otherwise lack mathematical
sophistication exploring and being amazed by the world of theorems and proofs
in obscure specialties, perhaps even coming up with results of their own.  A
tremendous advantage would be that anyone could experiment with conjectures in
any field---the computer would offer instant feedback as to whether
an inference step was correct.

Mathematicians sometimes have to put up with the annoyance of
cranks\index{cranks} who lack a fundamental understanding of mathematics but
insist that their ``proofs'' of, say, Fermat's Last Theorem\index{Fermat's
Last Theorem} be taken seriously.  I think part of the problem is that these
people are misled by informal mathematical language, treating it as if they
were reading ordinary expository English and failing to appreciate the
implicit underlying rigor.  Such cranks are rare in the field of computers,
because computer languages are much more explicit, and ultimately the proof is
in whether a computer program works or not.  With easily accessible
computer-based abstract mathematics, a mathematician could say to a crank,
``don't bother me until you've demonstrated your claim on the computer!''

% 22-May-04 nm
% Attempt to move De Millo quote so it doesn't separate from attribution
% CHANGE THIS NUMBER (AND ELIMINATE IF POSSIBLE) WHEN ABOVE TEXT CHANGES
\vspace{-0.5em}

\subsection{An Impossible Dream?}\label{dream}

\begin{quote}
  {\em Even quite basic theorems would demand almost unbelievably vast
  books to display their proofs.}
    \flushright\sc  Robert E. Edwards\footnote{\cite{Edwards}, p.~68.}\\
\end{quote}\index{Edwards, Robert E.}

\begin{quote}
  {\em Oh, of course no one ever really {\em does} it.  It would take
  forever!  You just show that you could do it, that's sufficient.}
    \flushright\sc  ``The Ideal Mathematician''
    \index{Davis, Phillip J.}\footnote{\cite{Davis},
p.~40.}\\
\end{quote}

\begin{quote}
  {\em There is a theorem in the primitive notation of set theory that
  corresponds to the arithmetic theorem `$1000+2000=3000$'.  The formula
  would be forbiddingly long\ldots even if [one] knows the definitions
  and is asked to simplify the long formula according to them, chances are
  he will make errors and arrive at some incorrect result.}
    \flushright\sc  Hao Wang\footnote{\cite{Wang}, p.~140.}\\
\end{quote}\index{Wang, Hao}

% 22-May-04 nm
% Attempt to move De Millo quote so it doesn't separate from attribution
% CHANGE THIS NUMBER (AND ELIMINATE IF POSSIBLE) WHEN ABOVE TEXT CHANGES
\vspace{-0.5em}

\begin{quote}
  {\em The {\em Principia Mathematica} was the crowning achievement of the
  formalists.  It was also the deathblow of the formalist view.\ldots
  {[Rus\-sell]} failed, in three enormous volumes, to get beyond the elementary
  facts of arithmetic.  He showed what can be done in principle and what
  cannot be done in practice.  If the mathematical process were really
  one of strict, logical progression, we would still be counting our
  fingers.\ldots
  One theoretician estimates\ldots that a demonstration of one of
  Ramanujan's conjectures assuming set theory and elementary analysis would
  take about two thousand pages; the length of a deduction from first principles
  is nearly in\-con\-ceiv\-a\-ble\ldots The probabilists argue that\ldots any
  very long proof can at best be viewed as only probably correct\ldots}
  \flushright\sc Richard de Millo et. al.\footnote{\cite{deMillo}, pp.~269,
  271.}\\
\end{quote}\index{de Millo, Richard}

A number of writers have conveyed the impression that the kind of absolute
rigor provided by Metamath\index{Metamath} is an impossible dream, suggesting
that a complete, formal verification\index{formal proof} of a typical theorem
would take millions of steps in untold volumes of books.  Even if it could be
done, the thinking sometimes goes, all meaning would be lost in such a
monstrous, tedious verification.\index{informal proof}\index{proof length}

These writers assume, however, that in order to achieve the kind of complete
formal verification they desire one must break down a proof into individual
primitive steps that make direct reference to the axioms.  This is
not necessary.  There is no reason not to make use of previously proved
theorems rather than proving them over and over.

Just as important, definitions\index{definition} can be introduced along
the way, allowing very complex formulas to be represented with few
symbols.  Not doing this can lead to absurdly long formulas.  For
example, the mere statement of
G\"{o}del's incompleteness theorem\index{G\"{o}del's
incompleteness theorem}, which can be expressed with a small number of
defined symbols, would require about 20,000 primitive symbols to express
it.\index{Boolos, George S.}\footnote{George S.\ Boolos, lecture at
Massachusetts Institute of Technology, spring 1990.} An extreme example
is Bourbaki's\label{bourbaki} language for set theory, which requires
4,523,659,424,929 symbols plus 1,179,618,517,981 disambiguatory links
(lines connecting symbol pairs, usually drawn below or above the
formula) to express the number
``one'' \cite{Mathias}.\index{Mathias, Adrian R. D.}\index{Bourbaki,
Nicolas}
% http://www.dpmms.cam.ac.uk/~ardm/

A hierarchy\index{hierarchy} of theorems and definitions permits an
exponential growth in the formula sizes and primitive proof steps to be
described with only a linear growth in the number of symbols used.  Of course,
this is how ordinary informal mathematics is normally done anyway, but with
Metamath\index{Metamath} it can be done with absolute rigor and precision.

\subsection{Beauty}


\begin{quote}
  {\em No one shall be able to drive us from the paradise that Cantor has
created for us.}
   \flushright\sc  David Hilbert\footnote{As quoted in \cite{Moore}, p.~131.}\\
\end{quote}\index{Hilbert, David}

\needspace{3\baselineskip}
\begin{quote}
  {\em Mathematics possesses not only truth, but some supreme beauty ---a
  beauty cold and austere, like that of a sculpture.}
    \flushright\sc  Bertrand
    Russell\footnote{\cite{Russell}.}\\
\end{quote}\index{Russell, Bertrand}

\begin{quote}
  {\em Euclid alone has looked on Beauty bare.}
  \flushright\sc Edna Millay\footnote{As quoted in \cite{Davis}, p.~150.}\\
\end{quote}\index{Millay, Edna}

For most people, abstract mathematics is distant, strange, and
incomprehensible.  Many popular books have tried to convey some of the sense
of beauty in famous theorems.  But even an intelligent layperson is left with
only a general idea of what a theorem is about and is hardly given the tools
needed to make use of it.  Traditionally, it is only after years of arduous
study that one can grasp the concepts needed for deep understanding.
Metamath\index{Metamath} allows you to approach the proof of the theorem from
a quite different perspective, peeling apart the formulas and definitions
layer by layer until an entirely different kind of understanding is achieved.
Every step of the proof is there, pieced together with absolute precision and
instantly available for inspection through a microscope with a magnification
as powerful as you desire.

A proof in itself can be considered an object of beauty.  Constructing an
elegant proof is an art.  Once a famous theorem has been proved, often
considerable effort is made to find simpler and more easily understood
proofs.  Creating and communicating elegant proofs is a major concern of
mathematicians.  Metamath is one way of providing a common language for
archiving and preserving this information.

The length of a proof can, to a certain extent, be considered an
objective measure of its ``beauty,'' since shorter proofs are usually
considered more elegant.  In the set theory database
\texttt{set.mm}\index{set theory database (\texttt{set.mm})}%
\index{Metamath Proof Explorer}
provided with Metamath, one goal was to make all proofs as short as possible.

\needspace{4\baselineskip}
\subsection{Simplicity}

\begin{quote}
  {\em God made man simple; man's complex problems are of his own
  devising.}
    \flushright\sc Eccles. 7:29\footnote{Jerusalem Bible.}\\
\end{quote}\index{Bible}

\needspace{3\baselineskip}
\begin{quote}
  {\em God made integers, all else is the work of man.}
    \flushright\sc Leopold Kronecker\footnote{{\em Jahresbericht
	der Deutschen Mathematiker-Vereinigung }, vol. 2, p. 19.}\\
\end{quote}\index{Kronecker, Leopold}

\needspace{3\baselineskip}
\begin{quote}
  {\em For what is clear and easily comprehended attracts; the
  complicated repels.}
    \flushright\sc David Hilbert\footnote{As quoted in \cite{deMillo},
p.~273.}\\
\end{quote}\index{Hilbert, David}

The Metamath\index{Metamath} language is simple and Spartan.  Metamath treats
all mathematical expressions as simple sequences of symbols, devoid of meaning.
The higher-level or ``metamathematical'' notions underlying Metamath are about
as simple as they could possibly be.  Each individual step in a proof involves
a single basic concept, the substitution of an expression for a variable, so
that in principle almost anyone, whether mathematician or not, can
completely understand how it was arrived at.

In one of its most basic applications, Metamath\index{Metamath} can be used to
develop the foundations of mathematics\index{foundations of mathematics} from
the very beginning.  This is done in the set theory database that is provided
with the Metamath package and is the subject matter
of Chapter~\ref{fol}. Any language (a metalanguage\index{metalanguage})
used to describe mathematics (an object language\index{object language}) must
have a mathematical content of its own, but it is desirable to keep this
content down to a bare minimum, namely that needed to make use of the
inference rules specified by the axioms.  With any metalanguage there is a
``chicken and egg'' problem somewhat like circular reasoning:  you must assume
the validity of the mathematics of the metalanguage in order to prove the
validity of the mathematics of the object language.  The mathematical content
of Metamath itself is quite limited.  Like the rules of a game of chess, the
essential concepts are simple enough so that virtually anyone should be able to
understand them (although that in itself will not let you play like
a master).  The symbols that Metamath manipulates do not in themselves
have any intrinsic meaning.  Your interpretation of the axioms that you supply
to Metamath is what gives them meaning.  Metamath is an attempt to strip down
mathematical thought to its bare essence and show you exactly how the symbols
are manipulated.

Philosophers and logicians, with various motivations, have often thought it
important to study ``weak'' fragments of logic\index{weak logic}
\cite{Anderson}\index{Anderson, Alan Ross} \cite{MegillBunder}\index{Megill,
Norman}\index{Bunder, Martin}, other unconventional systems of logic (such as
``modal'' logic\index{modal logic} \cite[ch.\ 27]{Boolos}\index{Boolos, George
S.}), and quantum logic\index{quantum logic} in physics
\cite{Pavicic}\index{Pavi{\v{c}}i{\'{c}}, M.}.  Metamath\index{Metamath}
provides a framework in which such systems can be expressed, with an absolute
precision that makes all underlying metamathematical assumptions rigorous and
crystal clear.

Some schools of philosophical thought, for example
intuitionism\index{intuitionism} and constructivism\index{constructivism},
demand that the notions underlying any mathematical system be as simple and
concrete as possible.  Metamath should meet the requirements of these
philosophies.  Metamath must be taught the symbols, axioms\index{axiom}, and
rules\index{rule} for a specific theory, from the skeptical (such as
intuitionism\index{intuitionism}\footnote{Intuitionism does not accept the law
of excluded middle (``either something is true or it is not true'').  See
\cite[p.~xi]{Tymoczko}\index{Tymoczko, Thomas} for discussion and references
on this topic.  Consider the theorem, ``There exist irrational numbers $a$ and
$b$ such that $a^b$ is rational.''  An intuitionist would reject the following
proof:  If $\sqrt{2}^{\sqrt{2}}$ is rational, we are done.  Otherwise, let
$a=\sqrt{2}^{\sqrt{2}}$ and $b=\sqrt{2}$. Then $a^b=2$, which is rational.})
to the bold (such as the axiom of choice in set theory\footnote{The axiom of
choice\index{Axiom of Choice} asserts that given any collection of pairwise
disjoint nonempty sets, there exists a set that has exactly one element in
common with each set of the collection.  It is used to prove many important
theorems in standard mathematics.  Some philosophers object to it because it
asserts the existence of a set without specifying what the set contains
\cite[p.~154]{Enderton}\index{Enderton, Herbert B.}.  In one foundation for
mathematics due to Quine\index{Quine, Willard Van Orman}, that has not been
otherwise shown to be inconsistent, the axiom of choice turns out to be false
\cite[p.~23]{Curry}\index{Curry, Haskell B.}.  The \texttt{show
trace{\char`\_}back} command of the Metamath program allows you to find out
whether the axiom of choice, or any other axiom, was assumed by a
proof.}\index{\texttt{show trace{\char`\_}back} command}).

The simplicity of the Metamath language lets the algorithm (computer program)
that verifies the validity of a Metamath proof be straightforward and
robust.  You can have confidence that the theorems it verifies really can be
derived from your axioms.

\subsection{Rigor}

\begin{quote}
  {\em Rigor became a goal with the Greeks\ldots But the efforts to
  pursue rigor to the utmost have led to an impasse in which there is
  no longer any agreement on what it really means.  Mathematics remains
  alive and vital, but only on a pragmatic basis.}
    \flushright\sc  Morris Kline\footnote{\cite{Kline}, p.~1209.}\\
\end{quote}\index{Kline, Morris}

Kline refers to a much deeper kind of rigor than that which we will discuss in
this section.  G\"{o}del's incompleteness theorem\index{G\"{o}del's
incompleteness theorem} showed that it is impossible to achieve absolute rigor
in standard mathematics because we can never prove that mathematics is
consistent (free from contradictions).\index{consistent theory}  If
mathematics is consistent, we will never know it, but must rely on faith.  If
mathematics is inconsistent, the best we can hope for is that some clever
future mathematician will discover the inconsistency.  In this case, the
axioms would probably be revised slightly to eliminate the inconsistency, as
was done in the case of Russell's paradox,\index{Russell's paradox} but the
bulk of mathematics would probably not be affected by such a discovery.
Russell's paradox, for example, did not affect most of the remarkable results
achieved by 19th-century and earlier mathematicians.  It mainly invalidated
some of Gottlob Frege's\index{Frege, Gottlob} work on the foundations of
mathematics in the late 1800's; in fact Frege's work inspired Russell's
discovery.  Despite the paradox, Frege's work contains important concepts that
have significantly influenced modern logic.  Kline's {\em Mathematics, The
Loss of Certainty} \cite{Klinel}\index{Kline, Morris} has an interesting
discussion of this topic.

What {\em can} be achieved with absolute certainty\index{certainty} is the
knowledge that if we assume the axioms are consistent and true, then the
results derived from them are true.  Part of the beauty of mathematics is that
it is the one area of human endeavor where absolute certainty can be achieved
in this sense.  A mathematical truth will remain such for eternity.  However,
our actual knowledge of whether a particular statement is a mathematical truth
is only as certain as the correctness of the proof that establishes it.  If
the proof of a statement is questionable or vague, we can't have absolute
confidence in the truth that the statement claims.

Let us look at some traditional ways of expressing proofs.

Except in the field of formal logic\index{formal logic}, almost all
traditional proofs in mathematics are really not proofs at all, but rather
proof outlines or hints as to how to go about constructing the proof.  Many
gaps\index{gaps in proofs} are left for the reader to fill in. There are
several reasons for this.  First, it is usually assumed in mathematical
literature that the person reading the proof is a mathematician familiar with
the specialty being described, and that the missing steps are obvious to such
a reader or at least that the reader is capable of filling them in.  This
attitude is fine for professional mathematicians in the specialty, but
unfortunately it often has the drawback of cutting off the rest of the world,
including mathematicians in other specialties, from understanding the proof.
We discussed one possible resolution to this on p.~\pageref{envision}.
Second, it is often assumed that a complete formal proof\index{formal proof}
would require countless millions of symbols (Section~\ref{dream}). This might
be true if the proof were to be expressed directly in terms of the axioms of
logic and set theory,\index{set theory} but it is usually not true if we allow
ourselves a hierarchy\index{hierarchy} of definitions and theorems to build
upon, using a notation that allows us to introduce new symbols, definitions,
and theorems in a precisely specified way.

Even in formal logic,\index{formal logic} formal proofs\index{formal proof}
that are considered complete still contain hidden or implicit information.
For example, a ``proof'' is usually defined as a sequence of
wffs,\index{well-formed formula (wff)}\footnote{A {\em wff} or well-formed
formula is a mathematical expression (string of symbols) constructed according
to some precise rules.  A formal mathematical system\index{formal system}
contains (1) the rules for constructing syntactically correct
wffs,\index{syntax rules} (2) a list of starting wffs called
axioms,\index{axiom} and (3) one or more rules prescribing how to derive new
wffs, called theorems\index{theorem}, from the axioms or previously derived
theorems.  An example of such a system is contained in
Metamath's\index{Metamath} set theory database, which defines a formal
system\index{formal system} from which all of standard mathematics can be
derived.  Section~\ref{startf} steps you through a complete example of a formal
system, and you may want to skim it now if you are unfamiliar with the
concept.} each of which is an axiom or follows from a rule applied to previous
wffs in the sequence.  The implicit part of the proof is the algorithm by
which a sequence of symbols is verified to be a valid wff, given the
definition of a wff.  The algorithm in this case is rather simple, but for a
computer to verify the proof,\index{automated proof verification} it must have
the algorithm built into its verification program.\footnote{It is possible, of
course, to specify wff construction syntax outside of the program itself
with a suitable input language (the Metamath language being an example), but
some proof-verification or theorem-proving programs lack the ability to extend
wff syntax in such a fashion.} If one deals exclusively with axioms and
elementary wffs, it is straightforward to implement such an algorithm.  But as
more and more definitions are added to the theory in order to make the
expression of wffs more compact, the algorithm becomes more and more
complicated.  A computer program that implements the algorithm becomes larger
and harder to understand as each definition is introduced, and thus more prone
to bugs.\index{computer program bugs}  The larger the program, the
more suspicious the mathematician may be about
the validity of its algorithms.  This is especially true because
computer programs are inherently hard to follow to begin with, and few people
enjoy verifying them manually in detail.

Metamath\index{Metamath} takes a different approach.  Metamath's ``knowledge''
is limited to the ability to substitute variables for expressions, subject to
some simple constraints.  Once the basic algorithm of Metamath is assumed to
be debugged, and perhaps independently confirmed, it
can be trusted once and for all.  The information that Metamath needs to
``understand'' mathematics is contained entirely in the body of knowledge
presented to Metamath.  Any errors in reasoning can only be errors in the
axioms or definitions contained in this body of knowledge.  As a
``constructive'' language\index{constructive language} Metamath has no
conditional branches or loops like the ones that make computer programs hard
to decipher; instead, the language can only build new sequences of symbols
from earlier sequences  of symbols.

The simplicity of the rules that underlie Metamath not only makes Metamath
easy to learn but also gives Metamath a great deal of flexibility. For
example, Metamath is not limited to describing standard first-order
logic\index{first-order logic}; higher-order logics\index{higher-order logic}
and fragments of logic\index{weak logic} can be described just as easily.
Metamath gives you the freedom to define whatever wff notation you prefer; it
has no built-in conception of the syntax of a wff.\index{well-formed formula
(wff)}  With suitable axioms and definitions, Metamath can even describe and
prove things about itself.\index{Metamath!self-description}  (John
Harrison\index{Harrison, John} discusses the ``reflection''
principle\index{reflection principle} involved in self-descriptive systems in
\cite{Harrison}.)

The flexibility of Metamath requires that its proofs specify a lot of detail,
much more than in an ordinary ``formal'' proof.\index{formal proof}  For
example, in an ordinary formal proof, a single step consists of displaying the
wff that constitutes that step.  In order for a computer program to verify
that the step is acceptable, it first must verify that the symbol sequence
being displayed is an acceptable wff.\index{automated proof verification} Most
proof verifiers have at least basic wff syntax built into their programs.
Metamath has no hard-wired knowledge of what constitutes a wff built into it;
instead every wff must be explicitly constructed based on rules defining wffs
that are present in a database.  Thus a single step in an ordinary formal
proof may be correspond to many steps in a Metamath proof. Despite the larger
number of steps, though, this does not mean that a Metamath proof must be
significantly larger than an ordinary formal proof. The reason is that since
we have constructed the wff from scratch, we know what the wff is, so there is
no reason to display it.  We only need to refer to a sequence of statements
that construct it.  In a sense, the display of the wff in an ordinary formal
proof is an implicit proof of its own validity as a wff; Metamath just makes
the proof explicit. (Section~\ref{proof} describes Metamath's proof notation.)

\section{Computers and Mathematicians}

\begin{quote}
  {\em The computer is important, but not to mathematics.}
    \flushright\sc  Paul Halmos\footnote{As quoted in \cite{Albers}, p.~121.}\\
\end{quote}\index{Halmos, Paul}

Pure mathematicians have traditionally been indifferent to computers, even to
the point of disdain.\index{computers and pure mathematics}  Computer science
itself is sometimes considered to fall in the mundane realm of ``applied''
mathematics, perhaps essential for the real world but intellectually unexciting
to those who seek the deepest truths in mathematics.  Perhaps a reason for this
attitude towards computers is that there is little or no computer software that
meets their needs, and there may be a general feeling that such software could
not even exist.  On the one hand, there are the practical computer algebra
systems, which can perform amazing symbolic manipulations in algebra and
calculus,\index{computer algebra system} yet can't prove the simplest
existence theorem, if the idea of a proof is present at all.  On the other
hand, there are specialized automated theorem provers that technically speaking
may generate correct proofs.\index{automated theorem proving}  But sometimes
their specialized input notation may be cryptic and their output perceived to
be long, inelegant, incomprehensible proofs.    The output
may be viewed with suspicion, since the program that generates it tends to be
very large, and its size increases the potential for bugs\index{computer
program bugs}.  Such a proof may be considered trustworthy only if
independently verified and ``understood'' by a human, but no one wants to
waste their time on such a boring, unrewarding chore.



\needspace{4\baselineskip}
\subsection{Trusting the Computer}

\begin{quote}
  {\em \ldots I continue to find the quasi-empirical interpretation of
  computer proofs to be the more plausible.\ldots Since not
  everything that claims to be a computer proof can be
  accepted as valid, what are the mathematical criteria for acceptable
  computer proofs?}
    \flushright\sc  Thomas Tymoczko\footnote{\cite{Tymoczko}, p.~245.}\\
\end{quote}\index{Tymoczko, Thomas}

In some cases, computers have been essential tools for proving famous
theorems.  But if a proof is so long and obscure that it can be verified in a
practical way only with a computer, it is vaguely felt to be suspicious.  For
example, proving the famous four-color theorem\index{four-color
theorem}\index{proof length} (``a map needs no more than four colors to
prevent any two adjacent countries from having the same color'') can presently
only be done with the aid of a very complex computer program which originally
required 1200 hours of computer time. There has been considerable debate about
whether such a proof can be trusted and whether such a proof is ``real''
mathematics \cite{Swart}\index{Swart, E. R.}.\index{trusting computers}

However, under normal circumstances even a skeptical mathematician would have a
great deal of confidence in the result of multiplying two numbers on a pocket
calculator, even though the precise details of what goes on are hidden from its
user.  Even the verification on a supercomputer that a huge number is prime is
trusted, especially if there is independent verification; no one bothers to
debate the philosophical significance of its ``proof,'' even though the actual
proof would be so large that it would be completely impractical to ever write
it down on paper.  It seems that if the algorithm used by the computer is
simple enough to be readily understood, then the computer can be trusted.

Metamath\index{Metamath} adopts this philosophy.  The simplicity of its
language makes it easy to learn, and because of its simplicity one can have
essentially absolute confidence that a proof is correct. All axioms, rules, and
definitions are available for inspection at any time because they are defined
by the user; there are no hidden or built-in rules that may be prone to subtle
bugs\index{computer program bugs}.  The basic algorithm at the heart of
Metamath is simple and fixed, and it can be assumed to be bug-free and robust
with a degree of confidence approaching certainty.
Independently written implementations of the Metamath verifier
can reduce any residual doubt on the part of a skeptic even further;
there are now over a dozen such implementations, written by many people.

\subsection{Trusting the Mathematician}\label{trust}

\begin{quote}
  {\em There is no Algebraist nor Mathematician so expert in his science, as
  to place entire confidence in any truth immediately upon his discovery of it,
  or regard it as any thing, but a mere probability.  Every time he runs over
  his proofs, his confidence encreases; but still more by the approbation of
  his friends; and is rais'd to its utmost perfection by the universal assent
  and applauses of the learned world.}
  \flushright\sc David Hume\footnote{{\em A Treatise of Human Nature}, as
  quoted in \cite{deMillo}, p.~267.}\\
\end{quote}\index{Hume, David}

\begin{quote}
  {\em Stanislaw Ulam estimates that mathematicians publish 200,000 theorems
  every year.  A number of these are subsequently contradicted or otherwise
  disallowed, others are thrown into doubt, and most are ignored.}
  \flushright\sc Richard de Millo et. al.\footnote{\cite{deMillo}, p.~269.}\\
\end{quote}\index{Ulam, Stanislaw}

Whether or not the computer can be trusted, humans  of course will occasionally
err. Only the most memorable proofs get independently verified, and of these
only a handful of truly great ones achieve the status of being ``known''
mathematical truths that are used without giving a second thought to their
correctness.

There are many famous examples of incorrect theorems and proofs in
mathematical literature.\index{errors in proofs}

\begin{itemize}
\item There have been thousands of purported proofs of Fermat's Last
Theorem\index{Fermat's Last Theorem} (``no integer solutions exist to $x^n +
y^n = z^n$ for $n > 2$''), by amateurs, cranks, and well-regarded
mathematicians \cite[p.~5]{Stark}\index{Stark, Harold M}.  Fermat wrote a note
in his copy of Bachet's {\em Diophantus} that he found ``a truly marvelous
proof of this theorem but this margin is too narrow to contain it''
\cite[p.~507]{Kramer}.  A recent, much publicized proof by Yoichi
Miyaoka\index{Miyaoka, Yoichi} was shown to be incorrect ({\em Science News},
April 9, 1988, p.~230).  The theorem was finally proved by Andrew
Wiles\index{Wiles, Andrew} ({\em Science News}, July 3, 1993, p.~5), but it
initially had some gaps and took over a year after its announcement to be
checked thoroughly by experts.  On Oct. 25, 1994, Wiles announced that the last
gap found in his proof had been filled in.
  \item In 1882, M. Pasch discovered that an axiom was omitted from Euclid's
formulation of geometry\index{Euclidean geometry}; without it, the proofs of
important theorems of Euclid are not valid.  Pasch's axiom\index{Pasch's
axiom} states that a line that intersects one side of a triangle must also
intersect another side, provided that it does not touch any of the triangle's
vertices.  The omission of Pasch's axiom went unnoticed for 2000
years \cite[p.~160]{Davis}, in spite of (one presumes) the thousands of
students, instructors, and mathematicians who studied Euclid.
  \item The first published proof of the famous Schr\"{o}der--Bernstein
theorem\index{Schr\"{o}der--Bernstein theorem} in set theory was incorrect
\cite[p.~148]{Enderton}\index{Enderton, Herbert B.}.  This theorem states
that if there exists a 1-to-1 function\footnote{A {\em set}\index{set} is any
collection of objects. A {\em function}\index{function} or {\em
mapping}\index{mapping} is a rule that assigns to each element of one set
(called the function's {\em domain}\index{domain}) an element from another
set.} from set $A$ into set $B$ and vice-versa, then sets $A$ and $B$ have
a 1-to-1 correspondence.  Although it sounds simple and obvious,
the standard proof is quite long and complex.
  \item In the early 1900's, Hilbert\index{Hilbert, David} published a
purported proof of the continuum hypothesis\index{continuum hypothesis}, which
was eventually established as unprovable by Cohen\index{Cohen, Paul} in 1963
\cite[p.~166]{Enderton}.  The continuum hypothesis states that no
infinity\index{infinity} (``transfinite cardinal number'')\index{cardinal,
transfinite} exists whose size (or ``cardinality''\index{cardinality}) is
between the size of the set of integers and the size of the set of real
numbers.  This hypothesis originated with German mathematician Georg
Cantor\index{Cantor, Georg} in the late 1800's, and his inability to prove it
is said to have contributed to mental illness that afflicted him in his later
years.
  \item An incorrect proof of the four-color theorem\index{four-color theorem}
was published by Kempe\index{Kempe, A. B.} in 1879
\cite[p.~582]{Courant}\index{Courant, Richard}; it stood for 11 years before
its flaw was discovered.  This theorem states that any map can be colored
using only four colors, so that no two adjacent countries have the same
color.  In 1976 the theorem was finally proved by the famous computer-assisted
proof of Haken, Appel, and Koch \cite{Swart}\index{Appel, K.}\index{Haken,
W.}\index{Koch, K.}.  Or at least it seems that way.  Mathematician
H.~S.~M.~Coxeter\index{Coxeter, H. S. M.} has doubts \cite[p.~58]{Davis}:  ``I
have a feeling that that is an untidy kind of use of the computers, and the more
you correspond with Haken and Appel, the more shaky you seem to be.''
  \item Many false ``proofs'' of the Poincar\'{e}
conjecture\index{Poincar\'{e} conjecture} have been proposed over the years.
This conjecture states that any object that mathematically behaves like a
three-dimensional sphere is a three-dimensional sphere topologically,
regardless of how it is distorted.  In March 1986, mathematicians Colin
Rourke\index{Rourke, Colin} and Eduardo R\^{e}go\index{R\^{e}go, Eduardo}
caused  a stir in the mathematical community by announcing that they had found
a proof; in November of that year the proof was found to be false \cite[p.
218]{PetersonI}.  It was finally proved in 2003 by Grigory Perelman
\label{poincare}\index{Szpiro, George}\index{Perelman, Grigory}\cite{Szpiro}.
 \end{itemize}

Many counterexamples to ``theorems'' in recent mathematical
literature related to Clifford algebras\index{Clifford algebras}
 have been found by Pertti
Lounesto (who passed away in 2002).\index{Lounesto, Pertti}
See the web page \url{http://mathforum.org/library/view/4933.html}.
% http://users.tkk.fi/~ppuska/mirror/Lounesto/counterexamples.htm

One of the purposes of Metamath\index{Metamath} is to allow proofs to be
expressed with absolute precision.  Developing a proof in the Metamath
language can be challenging, because Metamath will not permit even the
tiniest mistake.\index{errors in proofs}  But once the proof is created, its
correctness can be trusted immediately, without having to depend on the
process of peer review for confirmation.

\section{The Use of Computers in Mathematics}

\subsection{Computer Algebra Systems}

For the most part, you will find that Metamath\index{Metamath} is not a
practical tool for manipulating numbers.  (Even proving that $2 + 2 = 4$, if
you start with set theory, can be quite complex!)  Several commercial
mathematics packages are quite good at arithmetic, algebra, and calculus, and
as practical tools they are invaluable.\index{computer algebra system} But
they have no notion of proof, and cannot understand statements starting with
``there exists such and such...''.

Software packages such as Mathematica \cite{Wolfram}\index{Mathematica} do not
concern themselves with proofs but instead work directly with known results.
These packages primarily emphasize heuristic rules such as the substitution of
equals for equals to achieve simpler expressions or expressions in a different
form.  Starting with a rich collection of built-in rules and algorithms, users
can add to the collection by means of a powerful programming language.
However, results such as, say, the existence of a certain abstract object
without displaying the actual object cannot be expressed (directly) in their
languages.  The idea of a proof from a small set of axioms is absent.  Instead
this software simply assumes that each fact or rule you add to the built-in
collection of algorithms is valid.  One way to view the software is as a large
collection of axioms from which the software, with certain goals, attempts to
derive new theorems, for example equating a complex expression with a simpler
equivalent. But the terms ``theorem''\index{theorem} and
``proof,''\index{proof} for example, are not even mentioned in the index of
the user's manual for Mathematica.\index{Mathematica and proofs}  What is also
unsatisfactory from a philosophical point of view is that there is no way to
ensure the validity of the results other than by trusting the writer of each
application module or tediously checking each module by hand, similar to
checking a computer program for bugs.\index{computer program
bugs}\footnote{Two examples illustrate why the knowledge database of computer
algebra systems should sometimes be regarded with a certain caution.  If you
ask Mathematica (version 3.0) to \texttt{Solve[x\^{ }n + y\^{ }n == z\^{ }n , n]}
it will respond with \texttt{\{\{n-\char`\>-2\}, \{n-\char`\>-1\},
\{n-\char`\>1\}, \{n-\char`\>2\}\}}. In other words, Mathematica seems to
``know'' that Fermat's Last Theorem\index{Fermat's Last Theorem} is true!  (At
the time this version of Mathematica was released this fact was unknown.)  If
you ask Maple\index{Maple} to \texttt{solve(x\^{ }2 = 2\^{ }x)} then
\texttt{simplify(\{"\})}, it returns the solution set \texttt{\{2, 4\}}, apparently
unaware that 0.7666647\ldots is also a solution.} While of course extremely
valuable in applied mathematics, computer algebra systems tend to be of little
interest to the theoretical mathematician except as aids for exploring certain
specific problems.

Because of possible bugs, trusting the output of a computer algebra system for
use as theorems in a proof-verifier would defeat the latter's goal of rigor.
On the other hand, a fact such that a certain relatively large number is
prime, while easy for a computer algebra system to derive, might have a long,
tedious proof that could overwhelm a proof-verifier. One approach for linking
computer algebra systems to a proof-verifier while retaining the advantages of
both is to add a hypothesis to each such theorem indicating its source.  For
example, a constant {\sc maple} could indicate the theorem came from the Maple
package, and would mean ``assuming Maple is consistent, then\ldots''  This and
many other topics concerning the formalization of mathematics are discussed in
John Harrison's\index{Harrison, John} very interesting
PhD thesis~\cite{Harrison-thesis}.

\subsection{Automated Theorem Provers}\label{theoremprovers}

A mathematical theory is ``decidable''\index{decidable theory} if a mechanical
method or algorithm exists that is guaranteed to determine whether or not a
particular formula is a theorem.  Among the few theories that are decidable is
elementary geometry,\index{Euclidean geometry} as was shown by a classic
result of logician Alfred Tarski\index{Tarski, Alfred} in 1948
\cite{Tarski}.\footnote{Tarski's result actually applies to a subset of the
geometry discussed in elementary textbooks.  This subset includes most of what
would be considered elementary geometry but it is not powerful enough to
express, among other things, the notions of the circumference and area of a
circle.  Extending the theory in a way that includes notions such as these
makes the theory undecidable, as was also shown by Tarski.  Tarski's algorithm
is far too inefficient to implement practically on a computer.  A practical
algorithm for proving a smaller subset of geometry theorems (those not
involving concepts of ``order'' or ``continuity'') was discovered by Chinese
mathematician Wu Wen-ts\"{u}n in 1977 \cite{Chou}\index{Chou,
Shang-Ching}.}\index{Wen-ts{\"{u}}n, Wu}  But most theories, including
elementary arithmetic, are undecidable.  This fact contributes to keeping
mathematics alive and well, since many mathematicians believe
that they will never be
replaced by computers (if they believe Roger Penrose's argument that a
computer can never replace the brain \cite{Penrose}\index{Penrose, Roger}).
In fact,  elementary geometry is often considered a ``dead'' field
for the simple reason that it is decidable.

On the other hand, the undecidability of a theory does not mean that one cannot
use a computer to search for proofs, providing one is willing to give up if a
proof is not found after a reasonable amount of time.  The field of automated
theorem proving\index{automated theorem proving} specializes in pursuing such
computer searches.  Among the more successful results to date are those based
on an algorithm known as Robinson's resolution principle
\cite{Robinson}\index{Robinson's resolution principle}.

Automated theorem provers can be excellent tools for those willing to learn
how to use them.  But they are not widely used in mainstream pure
mathematics, even though they could probably be useful in many areas.  There
are several reasons for this.  Probably most important, the main goal in pure
mathematics is to arrive at results that are considered to be deep or
important; proving them is essential but secondary.  Usually, an automated
theorem prover cannot assist in this main goal, and by the time the main goal
is achieved, the mathematician may have already figured out the proof as a
by-product.  There is also a notational problem.  Mathematicians are used to
using very compact syntax where one or two symbols (heavily dependent on
context) can represent very complex concepts; this is part of the
hierarchy\index{hierarchy} they have built up to tackle difficult problems.  A
theorem prover on the other hand might require that a theorem be expressed in
``first-order logic,''\index{first-order logic} which is the logic on which
most of mathematics is ultimately based but which is not ordinarily used
directly because expressions can become very long.  Some automated theorem
provers are experimental programs, limited in their use to very specialized
areas, and the goal of many is simply research into the nature of automated
theorem proving itself.  Finally, much research remains to be done to enable
them to prove very deep theorems.  One significant result was a
computer proof by Larry Wos\index{Wos, Larry} and colleagues that every Robbins
algebra\index{Robbins algebra} is a Boolean  algebra\index{Boolean algebra}
({\em New York Times}, Dec. 10, 1996).\footnote{In 1933, E.~V.\
Huntington\index{Huntington, E. V.}
presented the following axiom system for
Boolean algebra with a unary operation $n$ and a binary operation $+$:
\begin{center}
    $x + y = y + x$ \\
    $(x + y) + z = x + (y + z)$ \\
    $n(n(x) + y) + n(n(x) + n(y)) = x$
\end{center}
Herbert Robbins\index{Robbins, Herbert}, a student of Huntington, conjectured
that the last equation can be replaced with a simpler one:
\begin{center}
    $n(n(x + y) + n(x + n(y))) = x$
\end{center}
Robbins and Huntington could not find a proof.  The problem was
later studied unsuccessfully by Tarski\index{Tarski, Alfred} and his
students, and it remained an unsolved problem until a
computer found the proof in 1996.  For more information on
the Robbins algebra problem see \cite{Wos}.}

How does Metamath\index{Metamath} relate to automated theorem provers?  A
theorem prover is primarily concerned with one theorem at a time (perhaps
tapping into a small database of known theorems) whereas Metamath is more like
a theorem archiving system, storing both the theorem and its proof in a
database for access and verification.  Metamath is one answer to ``what do you
do with the output of a theorem prover?''  and could be viewed as the
next step in the process.  Automated theorem provers could be useful tools for
helping develop its database.
Note that very long, automatically
generated proofs can make your database fat and ugly and cause Metamath's proof
verification to take a long time to run.  Unless you have a particularly good
program that generates very concise proofs, it might be best to consider the
use of automatically generated proofs as a quick-and-dirty approach, to be
manually rewritten at some later date.

The program {\sc otter}\index{otter@{\sc otter}}\footnote{\url{http://www.cs.unm.edu/\~mccune/otter/}.}, later succeeded by
prover9\index{prover9}\footnote{\url{https://www.cs.unm.edu/~mccune/mace4/}.},
have been historically influential.
The E prover\index{E prover}\footnote{\url{https://github.com/eprover/eprover}.}
is a maintained automated theorem prover
for full first-order logic with equality.
There are many other automated theorem provers as well.

If you want to combine automated theorem provers with Metamath
consider investigating
the book {\em Automated Reasoning:  Introduction and Applications}
\cite{Wos}\index{Wos, Larry}.  This book discusses
how to use {\sc otter} in a way that can
not only able to generate
relatively efficient proofs, it can even be instructed to search for
shorter proofs.  The effective use of {\sc otter} (and similar tools)
does require a certain
amount of experience, skill, and patience.  The axiom system used in the
\texttt{set.mm}\index{set theory database (\texttt{set.mm})} set theory
database can be expressed to {\sc otter} using a method described in
\cite{Megill}.\index{Megill, Norman}\footnote{To use those axioms with
{\sc otter}, they must be restated in a way that eliminates the need for
``dummy variables.''\index{dummy variable!eliminating} See the Comment
on p.~\pageref{nodd}.} When successful, this method tends to generate
short and clever proofs, but my experiments with it indicate that the
method will find proofs within a reasonable time only for relatively
easy theorems.  It is still fun to experiment with.

Reference \cite{Bledsoe}\index{Bledsoe, W. W.} surveys a number of approaches
people have explored in the field of automated theorem proving\index{automated
theorem proving}.

\subsection{Interactive Theorem Provers}\label{interactivetheoremprovers}

Finding proofs completely automatically is difficult, so there
are some interactive theorem provers that allow a human to guide the
computer to find a proof.
Examples include
HOL Light\index{HOL light}%
\footnote{\url{https://www.cl.cam.ac.uk/~jrh13/hol-light/}.},
Isabelle\index{Isabelle}%
\footnote{\url{http://www.cl.cam.ac.uk/Research/HVG/Isabelle}.},
{\sc hol}\index{hol@{\sc hol}}%
\footnote{\url{https://hol-theorem-prover.org/}.},
and
Coq\index{Coq}\footnote{\url{https://coq.inria.fr/}.}.

A major difference between most of these tools and Metamath is that the
``proofs'' are actually programs that guide the program to find a proof,
and not the proof itself.
For example, an Isabelle/HOL proof might apply a step
\texttt{apply (blast dest: rearrange reduction)}. The \texttt{blast}
instruction applies
an automatic tableux prover and returns if it found a sequence of proof
steps that work... but the sequence is not considered part of the proof.

A good overview of
higher-level proof verification languages (such as {\sc lcf}\index{lcf@{\sc
lcf}} and {\sc hol}\index{hol@{\sc hol}})
is given in \cite{Harrison}.  All of these languages are fundamentally
different from Metamath in that much of the mathematical foundational
knowledge is embedded in the underlying proof-verification program, rather
than placed directly in the database that is being verified.
These can have a steep learning curve for those without a mathematical
background.  For example, one usually must have a fair understanding of
mathematical logic in order to follow their proofs.

\subsection{Proof Verifiers}\label{proofverifiers}

A proof verifier is a program that doesn't generate proofs but instead
verifies proofs that you give it.  Many proof verifiers have limited built-in
automated proof capabilities, such as figuring out simple logical inferences
(while still being guided by a person who provides the overall proof).  Metamath
has no built-in automated proof capability other than the limited
capability of its Proof Assistant.

Proof-verification languages are not used as frequently as they might be.
Pure mathematicians are more concerned with producing new results, and such
detail and rigor would interfere with that goal.  The use of computers in pure
mathematics is primarily focused on automated theorem provers (not verifiers),
again with the ultimate goal of aiding the creation of new mathematics.
Automated theorem provers are usually concerned with attacking one theorem at
time rather than making a large, organized database easily available to the
user.  Metamath is one way to help close this gap.

By itself Metamath is a mostly a proof verifier.
This does not mean that other approaches can't be used; the difference
is that in Metamath, the results of various provers must be recorded
step-by-step so that they can be verified.

Another proof-verification language is Mizar,\index{Mizar} which can display
its proofs in the informal language that mathematicians are accustomed to.
Information on the Mizar language is available at \url{http://mizar.org}.

For the working mathematician, Mizar is an excellent tool for rigorously
documenting proofs. Mizar typesets its proofs in the informal English used by
mathematicians (and, while fine for them, are just as inscrutable by
laypersons!). A price paid for Mizar is a relatively steep learning curve of a
couple of weeks.  Several mathematicians are actively formalizing different
areas of mathematics using Mizar and publishing the proofs in a dedicated
journal. Unfortunately the task of formalizing mathematics is still looked
down upon to a certain extent since it doesn't involve the creation of ``new''
mathematics.

The closest system to Metamath is
the {\em Ghilbert}\index{Ghilbert} proof language (\url{http://ghilbert.org})
system developed by
Raph Levien\index{Levien, Raph}.
Ghilbert is a formal proof checker heavily inspired by Metamath.
Ghilbert statements are s-expressions (a la Lisp), which is easy
for computers to parse but many people find them hard to read.
There are a number of differences in their specific constructs, but
there is at least one tool to translate some Metamath materials into Ghilbert.
As of 2019 the Ghilbert community is smaller and less active than the
Metamath community.
That said, the Metamath and Ghilbert communities overlap, and fruitful
conversations between them have occurred many times over the years.

\subsection{Creating a Database of Formalized Mathematics}\label{mathdatabase}

Besides Metamath, there are several other ongoing projects with the goal of
formalizing mathematics into computer-verifiable databases.
Understanding some history will help.

The {\sc qed}\index{qed project@{\sc qed} project}%
\footnote{\url{http://www-unix.mcs.anl.gov/qed}.}
project arose in 1993 and its goals were outlined in the
{\sc qed} manifesto.
The {\sc qed} manifesto was
a proposal for a computer-based database of all mathematical knowledge,
strictly formalized and with all proofs having been checked automatically.
The project had a conference in 1994 and another in 1995;
there was also a ``twenty years of the {\sc qed} manifesto'' workshop
in 2014.
Its ideals are regularly reraised.

In a 2007 paper, Freek Wiedijk identified two reasons
for the failure of the {\sc qed} project as originally envisioned:%
\cite{Wiedijk-revisited}\index{Wiedijk, Freek}

\begin{itemize}
\item Very few people are working on formalization of mathematics. There is no compelling application for fully mechanized mathematics.
\item Formalized mathematics does not yet resemble traditional mathematics. This is partly due to the complexity of mathematical notation, and partly to the limitations of existing theorem provers and proof assistants.
\end{itemize}

But this did not end the dream of
formalizing mathematics into computer-verifiable databases.
The problems that led to the {\sc qed} manifesto are still with us,
even though the challenges were harder than originally considered.
What has happened instead is that various independent projects have
worked towards formalizing mathematics into computer-verifiable databases,
each simultaneously competing and cooperating with each other.

A concrete way to see this is
Freek Wiedijk's ``Formalizing 100 Theorems'' list%
\footnote{\url{http://www.cs.ru.nl/\%7Efreek/100/}.}
which shows the progress different systems have made on a challenge list
of 100 mathematical theorems.%
\footnote{ This is not the only list of ``interesting'' theorems.
Another interesting list was posted by Oliver Knill's list
\cite{Knill}\index{Knill, Oliver}.}
The top systems as of February 2019
(in order of the number of challenges completed) are
HOL Light, Isabelle, Metamath, Coq, and Mizar.

The Metamath 100%
\footnote{\url{http://us.metamath.org/mm\_100.html}}
page (maintained by David A. Wheeler\index{Wheeler, David A.})
shows the progress of Metamath (specifically its \texttt{set.mm} database)
against this challenge list maintained by Freek Wiedijk.
The Metamath \texttt{set.mm} database
has made a lot of progress over the years,
in part because working to prove those challenge theorems required
defining various terms and proving their properties as a prerequisite.
Here are just a few of the many statements that have been
formally proven with Metamath:

% The entries of this cause the narrow display to break poorly,
% since the short amount of text means LaTeX doesn't get a lot to work with
% and the itemize format gives it even *less* margin than usual.
% No one will mind if we make just this list flushleft, since this list
% will be internally consistent.
\begin{flushleft}
\begin{itemize}
\item 1. The Irrationality of the Square Root of 2
  (\texttt{sqr2irr}, by Norman Megill, 2001-08-20)
\item 2. The Fundamental Theorem of Algebra
  (\texttt{fta}, by Mario Carneiro, 2014-09-15)
\item 22. The Non-Denumerability of the Continuum
  (\texttt{ruc}, by Norman Megill, 2004-08-13)
\item 54. The Konigsberg Bridge Problem
  (\texttt{konigsberg}, by Mario Carneiro, 2015-04-16)
\item 83. The Friendship Theorem
  (\texttt{friendship}, by Alexander W. van der Vekens, 2018-10-09)
\end{itemize}
\end{flushleft}

We thank all of those who have developed at least one of the Metamath 100
proofs, and we particularly thank
Mario Carneiro\index{Carneiro, Mario}
who has contributed the most Metamath 100 proofs as of 2019.
The Metamath 100 page shows the list of all people who have contributed a
proof, and links to graphs and charts showing progress over time.
We encourage others to work on proving theorems not yet proven in Metamath,
since doing so improves the work as a whole.

Each of the math formalization systems (including Metamath)
has different strengths and weaknesses, depending on what you value.
Key aspects that differentiate Metamath from the other top systems are:

\begin{itemize}
\item Metamath is not tied to any particular set of axioms.
\item Metamath can show every step of every proof, no exceptions.
  Most other provers only assert that a proof can be found, and do not
  show every step. This also makes verification fast, because
  the system does not need to rediscover proof details.
\item The Metamath verifier has been re-implemented in many different
  programming languages, so verification can be done by multiple
  implementations.  In particular, the
  \texttt{set.mm}\index{set theory database (\texttt{set.mm})}%
  \index{Metamath Proof Explorer} database is verified by
  four different verifiers
  written in four different languages by four different authors.
  This greatly reduces the risk of accepting an invalid
  proof due to an error in the verifier.
\item Proofs stay proven.  In some systems, changes to the system's
  syntax or how a tactic works causes proofs to fail in later versions,
  causing older work to become essentially lost.
  Metamath's language is
  extremely small and fixed, so once a proof is added to a database,
  the database can be rechecked with later versions of the Metamath program
  and with other verifiers of Metamath databases.
  If an axiom or key definition needs to be changed, it is easy to
  manipulate the database as a whole to handle the change
  without touching the underlying verifier.
  Since re-verification of an entire database takes seconds, there
  is never a reason to delay complete verification.
  This aspect is especially compelling if your
  goal is to have a long-term database of proofs.
\item Licensing is generous.  The main Metamath databases are released to
  the public domain, and the main Metamath program is open source software
  under a standard, widely-used license.
\item Substitutions are easy to understand, even by those who are not
  professional mathematicians.
\end{itemize}

Of course, other systems may have advantages over Metamath
that are more compelling, depending on what you value.
In any case, we hope this helps you understand Metamath
within a wider context.

\subsection{In Summary}\label{computers-summary}

To summarize our discussions of computers and mathematics, computer algebra
systems can be viewed as theorem generators focusing on a narrow realm of
mathematics (numbers and their properties), automated theorem provers as proof
generators for specific theorems in a much broader realm covered by a built-in
formal system such as first-order logic, interactive theorem
provers require human guidance, proof verifiers verify proofs but
historically they have been
restricted to first-order logic.
Metamath, in contrast,
is a proof verifier and documenter whose realm is essentially unlimited.

\section{Mathematics and Metamath}

\subsection{Standard Mathematics}

There are a number of ways that Metamath\index{Metamath} can be used with
standard mathematics.  The most satisfying way philosophically is to start at
the very beginning, and develop the desired mathematics from the axioms of
logic and set theory.\index{set theory}  This is the approach taken in the
\texttt{set.mm}\index{set theory database (\texttt{set.mm})}%
\index{Metamath Proof Explorer}
database (also known as the Metamath Proof Explorer).
Among other things, this database builds up to the
axioms of real and complex numbers\index{analysis}\index{real and complex
numbers} (see Section~\ref{real}), and a standard development of analysis, for
example, could start at that point, using it as a basis.   Besides this
philosophical advantage, there are practical advantages to having all of the
tools of set theory available in the supporting infrastructure.

On the other hand, you may wish to start with the standard axioms of a
mathematical theory without going through the set theoretical proofs of those
axioms.  You will need mathematical logic to make inferences, but if you wish
you can simply introduce theorems\index{theorem} of logic as
``axioms''\index{axiom} wherever you need them, with the implicit assumption
that in principle they can be proved, if they are obvious to you.  If you
choose this approach, you will probably want to review the notation used in
\texttt{set.mm}\index{set theory database (\texttt{set.mm})} so that your own
notation will be consistent with it.

\subsection{Other Formal Systems}
\index{formal system}

Unlike some programs, Metamath\index{Metamath} is not limited to any specific
area of mathematics, nor committed to any particular mathematical philosophy
such as classical logic versus intuitionism, nor limited, say, to expressions
in first-order logic.  Although the database \texttt{set.mm}
describes standard logic and set theory, Meta\-math
is actually a general-purpose language for describing a wide variety of formal
systems.\index{formal system}  Non-standard systems such as modal
logic,\index{modal logic} intuitionist logic\index{intuitionism}, higher-order
logic\index{higher-order logic}, quantum logic\index{quantum logic}, and
category theory\index{category theory} can all be described with the Metamath
language.  You define the symbols you prefer and tell Metamath the axioms and
rules you want to start from, and Metamath will verify any inferences you make
from those axioms and rules.  A simple example of a non-standard formal system
is Hofstadter's\index{Hofstadter, Douglas R.} MIU system,\index{MIU-system}
whose Metamath description is presented in Appendix~\ref{MIU}.

This is not hypothetical.
The largest Metamath database is
\texttt{set.mm}\index{set theory database (\texttt{set.mm}}%
\index{Metamath Proof Explorer}), aka the Metamath Proof Explorer,
which uses the most common axioms for mathematical foundations
(specifically classical logic combined with Zermelo--Fraenkel
set theory\index{Zermelo--Fraenkel set theory} with the Axiom of Choice).
But other Metamath databases are available:

\begin{itemize}
\item The database
  \texttt{iset.mm}\index{intuitionistic logic database (\texttt{iset.mm})},
  aka the
  Intuitionistic Logic Explorer\index{Intuitionistic Logic Explorer},
  uses intuitionistic logic (a constructivist point of view)
  instead of classical logic.
\item The database
  \texttt{nf.mm}\index{New Foundations database (\texttt{nf.mm})},
  aka the
  New Foundations Explorer\index{New Foundations Explorer},
  constructs mathematics from scratch,
  starting from Quine's New Foundations (NF) set theory axioms.
\item The database
  \texttt{hol.mm}\index{Higher-order Logic database (\texttt{hol.mm})},
  aka the
  Higher-Order Logic (HOL) Explorer\index{Higher-Order Logic (HOL) Explorer},
  starts with HOL (also called simple type theory) and derives
  equivalents to ZFC axioms, connecting the two approaches.
\end{itemize}

Since the days of David Hilbert,\index{Hilbert, David} mathematicians have
been concerned with the fact that the metalanguage\index{metalanguage} used to
describe mathematics may be stronger than the mathematics being described.
Metamath\index{Metamath}'s underlying finitary\index{finitary proof},
constructive nature provides a good philosophical basis for studying even the
weakest logics.\index{weak logic}

The usual treatment of many non-standard formal systems\index{formal
system} uses model theory\index{model theory} or proof theory\index{proof
theory} to describe these systems; these theories, in turn, are based on
standard set theory.  In other words, a non-standard formal system is defined
as a set with certain properties, and standard set theory is used to derive
additional properties of this set.  The standard set theory database provided
with Metamath can be used for this purpose, and when used this way
the development of a special
axiom system for the non-standard formal system becomes unnecessary.  The
model- or proof-theoretic approach often allows you to prove much deeper
results with less effort.

Metamath supports both approaches.  You can define the non-standard
formal system directly, or define the non-standard formal system as
a set with certain properties, whichever you find most helpful.

%\section{Additional Remarks}

\subsection{Metamath and Its Philosophy}

Closely related to Metamath\index{Metamath} is a philosophy or way of looking
at mathematics. This philosophy is related to the formalist
philosophy\index{formalism} of Hilbert\index{Hilbert, David} and his followers
\cite[pp.~1203--1208]{Kline}\index{Kline, Morris}
\cite[p.~6]{Behnke}\index{Behnke, H.}. In this philosophy, mathematics is
viewed as nothing more than a set of rules that manipulate symbols, together
with the consequences of those rules.  While the mathematics being described
may be complex, the rules used to describe it (the
``metamathematics''\index{metamathematics}) should be as simple as possible.
In particular, proofs should be restricted to dealing with concrete objects
(the symbols we write on paper rather than the abstract concepts they
represent) in a constructive manner; these are called ``finitary''
proofs\index{finitary proof} \cite[pp.~2--3]{Shoenfield}\index{Shoenfield,
Joseph R.}.

Whether or not you find Metamath interesting or useful will in part depend on
the appeal you find in its philosophy, and this appeal will probably depend on
your particular goals with respect to mathematics.  For example, if you are a
pure mathematician at the forefront of discovering new mathematical knowledge,
you will probably find that the rigid formality of Metamath stifles your
creativity.  On the other hand, we would argue that once this knowledge is
discovered, there are advantages to documenting it in a standard format that
will make it accessible to others.  Sixty years from now, your field may be
dormant, and as Davis and Hersh put it, your ``writings would become less
translatable than those of the Maya'' \cite[p.~37]{Davis}\index{Davis, Phillip
J.}.


\subsection{A History of the Approach Behind Metamath}

Probably the one work that has had the most motivating influence on
Metamath\index{Metamath} is Whitehead and Russell's monumental {\em Principia
Mathematica} \cite{PM}\index{Whitehead, Alfred North}\index{Russell,
Bertrand}\index{principia mathematica@{\em Principia Mathematica}}, whose aim
was to deduce all of mathematics from a small number of primitive ideas, in a
very explicit way that in principle anyone could understand and follow.  While
this work was tremendously influential in its time, from a modern perspective
it suffers from several drawbacks.  Both its notation and its underlying
axioms are now considered dated and are no longer used.  From our point of
view, its development is not really as accessible as we would like to see; for
practical reasons, proofs become more and more sketchy as its mathematics
progresses, and working them out in fine detail requires a degree of
mathematical skill and patience that many people don't have.  There are
numerous small errors, which is understandable given the tedious, technical
nature of the proofs and the lack of a computer to verify the details.
However, even today {\em Principia Mathematica} stands out as the work closest
in spirit to Metamath.  It remains a mind-boggling work, and one can't help
but be amazed at seeing ``$1+1=2$'' finally appear on page 83 of Volume II
(Theorem *110.643).

The origin of the proof notation used by Metamath dates back to the 1950's,
when the logician C.~A.~Meredith expressed his proofs in a compact notation
called ``condensed detachment''\index{condensed detachment}
\cite{Hindley}\index{Hindley, J. Roger} \cite{Kalman}\index{Kalman, J. A.}
\cite{Meredith}\index{Meredith, C. A.} \cite{Peterson}\index{Peterson, Jeremy
George}.  This notation allows proofs to be communicated unambiguously by
merely referencing the axiom\index{axiom}, rule\index{rule}, or
theorem\index{theorem} used at each step, without explicitly indicating the
substitutions\index{substitution!variable}\index{variable substitution} that
have to be made to the variables in that axiom, rule, or theorem.  Ordinarily,
condensed detachment is more or less limited to propositional
calculus\index{propositional calculus}.  The concept has been extended to
first-order logic\index{first-order logic} in \cite{Megill}\index{Megill,
Norman}, making it is easy to write a small computer program to verify proofs
of simple first-order logic theorems.\index{condensed detachment!and
first-order logic}

A key concept behind the notation of condensed detachment is called
``unification,''\index{unification} which is an algorithm for determining what
substitutions\index{substitution!variable}\index{variable substitution} to
variables have to be made to make two expressions match each other.
Unification was first precisely defined by the logician J.~A.~Robinson, who
used it in the development of a powerful
theorem-proving technique called the ``resolution principle''
\cite{Robinson}\index{Robinson's resolution principle}. Metamath does not make
use of the resolution principle, which is intended for systems of first-order
logic.\index{first-order logic}  Metamath's use is not restricted to
first-order logic, and as we have mentioned it does not automatically discover
proofs.  However, unification is a key idea behind Metamath's proof
notation, and Metamath makes use of a very simple version of it
(Section~\ref{unify}).

\subsection{Metamath and First-Order Logic}

First-order logic\index{first-order logic} is the supporting structure
for standard mathematics.  On top of it is set theory, which contains
the axioms from which virtually all of mathematics can be derived---a
remarkable fact.\index{category
theory}\index{cardinal, inaccessible}\label{categoryth}\footnote{An exception seems
to be category theory.  There are several schools of thought on whether
category theory is derivable from set theory.  At a minimum, it appears
that an additional axiom is needed that asserts the existence of an
``inaccessible cardinal'' (a type of infinity so large that standard set
theory can't prove or deny that it exists).
%
%%%% (I took this out that was in previous editions:)
% But it is also argued that not just one but a ``proper class'' of them
% is needed, and the existence of proper classes is impossible in standard
% set theory.  (A proper class is a collection of sets so huge that no set
% can contain it as an element.  Proper classes can lead to
% inconsistencies such as ``Russell's paradox.''  The axioms of standard
% set theory are devised so as to deny the existence of proper classes.)
%
For more information, see
\cite[pp.~328--331]{Herrlich}\index{Herrlich, Horst} and
\cite{Blass}\index{Blass, Andrea}.}

One of the things that makes Metamath\index{Metamath} more practical for
first-order theories is a set of axioms for first-order logic designed
specifically with Metamath's approach in mind.  These are included in
the database \texttt{set.mm}\index{set theory database (\texttt{set.mm})}.
See Chapter~\ref{fol} for a detailed
description; the axioms are shown in Section~\ref{metaaxioms}.  While
logically equivalent to standard axiom systems, our axiom system breaks
up the standard axioms into smaller pieces such that from them, you can
directly derive what in other systems can only be derived as higher-level
``metatheorems.''\index{metatheorem}  In other words, it is more powerful than
the standard axioms from a metalogical point of view.  A rigorous
justification for this system and its ``metalogical
completeness''\index{metalogical completeness} is found in
\cite{Megill}\index{Megill, Norman}.  The system is closely related to a
system developed by Monk\index{Monk, J. Donald} and Tarski\index{Tarski,
Alfred} in 1965 \cite{Monks}.

For example, the formula $\exists x \, x = y $ (given $y$, there exists some
$x$ equal to it) is a theorem of logic,\footnote{Specifically, it is a theorem
of those systems of logic that assume non-empty domains.  It is not a theorem
of more general systems that include the empty domain\index{empty domain}, in
which nothing exists, period!  Such systems are called ``free
logics.''\index{free logic} For a discussion of these systems, see
\cite{Leblanc}\index{Leblanc, Hugues}.  Since our use for logic is as a basis
for set theory, which has a non-empty domain, it is more convenient (and more
traditional) to use a less general system.  An interesting curiosity is that,
using a free logic as a basis for Zermelo--Fraenkel set
theory\index{Zermelo--Fraenkel set theory} (with the redundant Axiom of the
Null Set omitted),\index{Axiom of the Null Set} we cannot even prove the
existence of a single set without assuming the axiom of infinity!\index{Axiom
of Infinity}} whether or not $x$ and $y$ are distinct variables\index{distinct
variables}.  In many systems of logic, we would have to prove two theorems to
arrive at this result.  First we would prove ``$\exists x \, x = x $,'' then
we would separately prove ``$\exists x \, x = y $, where $x$ and $y$ are
distinct variables.''  We would then combine these two special cases ``outside
of the system'' (i.e.\ in our heads) to be able to claim, ``$\exists x \, x =
y $, regardless of whether $x$ and $y$ are distinct.''  In other words, the
combination of the two special cases is a
metatheorem.  In the system of logic
used in Metamath's set theory\index{set theory database (\texttt{set.mm})}
database, the axioms of logic are broken down into small pieces that allow
them to be reassembled in such a way that theorems such as these can be proved
directly.

Breaking down the axioms in this way makes them look peculiar and not very
intuitive at first, but rest assured that they are correct and complete.  Their
correctness is ensured because they are theorem schemes of standard first-order
logic (which you can easily verify if you are a logician).  Their completeness
follows from the fact that we explicitly derive the standard axioms of
first-order logic as theorems.  Deriving the standard axioms is somewhat
tricky, but once we're there, we have at our disposal a system that is less
awkward to work with in formal proofs\index{formal proof}.  In technical terms
that logicians understand, we eliminate the cumbersome concepts of ``free
variable,''\index{free variable} ``bound variable,''\index{bound variable} and
``proper substitution''\index{proper substitution}\index{substitution!proper}
as primitive notions.  These concepts are present in our system but are
defined in terms of concepts expressed by the axioms and can be eliminated in
principle.  In standard systems, these concepts are really like additional,
implicit axioms\index{implicit axiom} that are somewhat complex and cannot be
eliminated.

The traditional approach to logic, wherein free variables and proper
substitution is defined, is also possible to do directly in the Metamath
language.  However, the notation tends to become awkward, and there are
disadvantages:  for example, extending the definition of a wff with a
definition is awkward, because the free variable and proper substitution
concepts have to have their definitions also extended.  Our choice of
axioms for \texttt{set.mm} is to a certain extent a matter of style, in
an attempt to achieve overall simplicity, but you should also be aware
that the traditional approach is possible as well if you should choose
it.

\chapter{Using the Metamath Program}
\label{using}

\section{Installation}

The way that you install Metamath\index{Metamath!installation} on your
computer system will vary for different computers.  Current
instructions are provided with the Metamath program download at
\url{http://metamath.org}.  In general, the installation is simple.
There is one file containing the Metamath program itself.  This file is
usually called \texttt{metamath} or \texttt{metamath.}{\em xxx} where
{\em xxx} is the convention (such as \texttt{exe}) for an executable
program on your operating system.  There are several additional files
containing samples of the Metamath language, all ending with
\texttt{.mm}.  The file \texttt{set.mm}\index{set theory database
(\texttt{set.mm})} contains logic and set theory and can be used as a
starting point for other areas of mathematics.

You will also need a text editor\index{text editor} capable of editing plain
{\sc ascii}\footnote{American Standard Code for Information Interchange.} text
in order to prepare your input files.\index{ascii@{\sc ascii}}  Most computers
have this capability built in.  Note that plain text is not necessarily the
default for some word processing programs\index{word processor}, especially if
they can handle different fonts; for example, with Microsoft Word\index{Word
(Microsoft)}, you must save the file in the format ``Text Only With Line
Breaks'' to get a plain text\index{plain text} file.\footnote{It is
recommended that all lines in a Metamath source file be 79 characters or less
in length for compatibility among different computer terminals.  When creating
a source file on an editor such as Word, select a monospaced
font\index{monospaced font} such as Courier\index{Courier font} or
Monaco\index{Monaco font} to make this easier to achieve.  Better yet,
just use a plain text editor such as Notepad.}

On some computer systems, Metamath does not have the capability to print
its output directly; instead, you send its output to a file (using the
\texttt{open} commands described later).  The way you print this output
file depends on your computer.\index{printers} Some computers have a
print command, whereas with others, you may have to read the file into
an editor and print it from there.

If you want to print your Metamath source files with typeset formulas
containing standard mathematical symbols, you will need the \LaTeX\
typesetting program\index{latex@{\LaTeX}}, which is widely and freely
available for most operating systems.  It runs natively on Unix and
Linux, and can be installed on Windows as part of the free Cygwin
package (\url{http://cygwin.com}).

You can also produce {\sc html}\footnote{HyperText Markup Language.}
web pages.  The {\tt help html} command in the Metamath program will
assist you with this feature.

\section{Your First Formal System}\label{start}
\subsection{From Nothing to Zero}\label{startf}

To give you a feel for what the Metamath\index{Metamath} language looks like,
we will take a look at a very simple example from formal number
theory\index{number theory}.  This example is taken from
Mendelson\index{Mendelson, Elliot} \cite[p. 123]{Mendelson}.\footnote{To keep
the example simple, we have changed the formalism slightly, and what we call
axioms\index{axiom} are strictly speaking theorems\index{theorem} in
\cite{Mendelson}.}  We will look at a small subset of this theory, namely that
part needed for the first number theory theorem proved in \cite{Mendelson}.

First we will look at a standard formal proof\index{formal proof} for the
example we have picked, then we will look at the Metamath version.  If you
have never been exposed to formal proofs, the notation may seem to be such
overkill to express such simple notions that you may wonder if you are missing
something.  You aren't.  The concepts involved are in fact very simple, and a
detailed breakdown in this fashion is necessary to express the proof in a way
that can be verified mechanically.  And as you will see, Metamath breaks the
proof down into even finer pieces so that the mechanical verification process
can be about as simple as possible.

Before we can introduce the axioms\index{axiom} of the theory, we must define
the syntax rules for forming legal expressions\index{syntax rules}
(combinations of symbols) with which those axioms can be used. The number 0 is
a {\bf term}\index{term}; and if $ t$ and $r$ are terms, so is $(t+r)$. Here,
$ t$ and $r$ are ``metavariables''\index{metavariable} ranging over terms; they
themselves do not appear as symbols in an actual term.  Some examples of
actual terms are $(0 + 0)$ and $((0+0)+0)$.  (Note that our theory describes
only the number zero and sums of zeroes.  Of course, not much can be done with
such a trivial theory, but remember that we have picked a very small subset of
complete number theory for our example.  The important thing for you to focus
on is our definitions that describe how symbols are combined to form valid
expressions, and not on the content or meaning of those expressions.) If $ t$
and $r$ are terms, an expression of the form $ t=r$ is a {\bf wff}
(well-formed formula)\index{well-formed formula (wff)}; and if $P$ and $Q$ are
wffs, so is $(P\rightarrow Q)$ (which means ``$P$ implies
$Q$''\index{implication ($\rightarrow$)} or ``if $P$ then $Q$'').
Here $P$ and $Q$ are metavariables ranging over wffs.  Examples of actual
wffs are $0=0$, $(0+0)=0$, $(0=0 \rightarrow (0+0)=0)$, and $(0=0\rightarrow
(0=0\rightarrow 0=(0+0)))$.  (Our notation makes use of more parentheses than
are customary, but the elimination of ambiguity this way simplifies our
example by avoiding the need to define operator precedence\index{operator
precedence}.)

The {\bf axioms}\index{axiom} of our theory are all wffs of the following
form, where $ t$, $r$, and $s$ are any terms:

%Latex p. 92
\renewcommand{\theequation}{A\arabic{equation}}

\begin{equation}
(t=r\rightarrow (t=s\rightarrow r=s))
\end{equation}
\begin{equation}
(t+0)=t
\end{equation}

Note that there are an infinite number of axioms since there are an infinite
number of possible terms.  A1 and A2 are properly called ``axiom
schemes,''\index{axiom scheme} but we will refer to them as ``axioms'' for
brevity.

An axiom is a {\bf theorem}; and if $P$ and $(P\rightarrow Q)$ are theorems
(where $P$ and $Q$ are wffs), then $Q$ is also a theorem.\index{theorem}  The
second part of this definition is called the modus ponens (MP) rule of
inference\index{inference rule}\index{modus ponens}.  It allows us to obtain
new theorems from old ones.

The {\bf proof}\index{proof} of a theorem is a sequence of one or more
theorems, each of which is either an axiom or the result of modus ponens
applied to two previous theorems in the sequence, and the last of which is the
theorem being proved.

The theorem we will prove for our example is very simple:  $ t=t$.  The proof of
our theorem follows.  Study it carefully until you feel sure you
understand it.\label{zeroproof}

% Use tabu so that lines will wrap automatically as needed.
\begin{tabu} { l X X }
1. & $(t+0)=t$ & (by axiom A2) \\
2. & $(t+0)=t$ & (by axiom A2) \\
3. & $((t+0)=t \rightarrow ((t+0)=t\rightarrow t=t))$ & (by axiom A1) \\
4. & $((t+0)=t\rightarrow t=t)$ & (by MP applied to steps 2 and 3) \\
5. & $t=t$ & (by MP applied to steps 1 and 4) \\
\end{tabu}

(You may wonder why step 1 is repeated twice.  This is not necessary in the
formal language we have defined, but in Metamath's ``reverse Polish
notation''\index{reverse Polish notation (RPN)} for proofs, a previous step
can be referred to only once.  The repetition of step~1 here will enable you
to see more clearly the correspondence of this proof with the
Metamath\index{Metamath} version on p.~\pageref{demoproof}.)

Our theorem is more properly called a ``theorem scheme,''\index{theorem
scheme} for it represents an infinite number of theorems, one for each
possible term $ t$.  Two examples of actual theorems would be $0=0$ and
$(0+0)=(0+0)$.  Rarely do we prove actual theorems, since by proving schemes
we can prove an infinite number of theorems in one fell swoop.  Similarly, our
proof should really be called a ``proof scheme.''\index{proof scheme}  To
obtain an actual proof, pick an actual term to use in place of $ t$, and
substitute it for $ t$ throughout the proof.

Let's discuss what we have done here.  The axioms\index{axiom} of our theory,
A1 and A2, are trivial and obvious.  Everyone knows that adding zero to
something doesn't change it, and also that if two things are equal to a third,
then they are equal to each other. In fact, stating the trivial and obvious is
a goal to strive for in any axiomatic system.  From trivial and obvious truths
that everyone agrees upon, we can prove results that are not so obvious yet
have absolute faith in them.  If we trust the axioms and the rules, we must,
by definition, trust the consequences of those axioms and rules, if logic is
to mean anything at all.

Our rule of inference\index{rule}, modus ponens\index{modus ponens}, is also
pretty obvious once you understand what it means.  If we prove a fact $P$, and
we also prove that $P$ implies $Q$, then $Q$ necessarily follows as a new
fact.  The rule provides us with a means for obtaining new facts (i.e.\
theorems\index{theorem}) from old ones.

The theorem that we have proved, $ t=t$, is so fundamental that you may wonder
why it isn't one of the axioms\index{axiom}.  In some axiom systems of
arithmetic, it {\em is} an axiom.  The choice of axioms in a theory is to some
extent arbitrary and even an art form, constrained only by the requirement
that any two equivalent axiom systems be able to derive each other as
theorems.  We could imagine that the inventor of our axiom system originally
included $ t=t$ as an axiom, then discovered that it could be derived as a
theorem from the other axioms.  Because of this, it was not necessary to
keep it as an axiom.  By eliminating it, the final set of axioms became
that much simpler.

Unless you have worked with formal proofs\index{formal proof} before, it
probably wasn't apparent to you that $ t=t$ could be derived from our two
axioms until you saw the proof. While you certainly believe that $ t=t$ is
true, you might not be able to convince an imaginary skeptic who believes only
in our two axioms until you produce the proof.  Formal proofs such as this are
hard to come up with when you first start working with them, but after you get
used to them they can become interesting and fun.  Once you understand the
idea behind formal proofs you will have grasped the fundamental principle that
underlies all of mathematics.  As the mathematics becomes more sophisticated,
its proofs become more challenging, but ultimately they all can be broken down
into individual steps as simple as the ones in our proof above.

Mendelson's\index{Mendelson, Elliot} book, from which our example was taken,
contains a number of detailed formal proofs such as these, and you may be
interested in looking it up.  The book is intended for mathematicians,
however, and most of it is rather advanced.  Popular literature describing
formal proofs\index{formal proof} include \cite[p.~296]{Rucker}\index{Rucker,
Rudy} and \cite[pp.~204--230]{Hofstadter}\index{Hofstadter, Douglas R.}.

\subsection{Converting It to Metamath}\label{convert}

Formal proofs\index{formal proof} such as the one in our example break down
logical reasoning into small, precise steps that leave little doubt that the
results follow from the axioms\index{axiom}.  You might think that this would
be the finest breakdown we can achieve in mathematics.  However, there is more
to the proof than meets the eye. Although our axioms were rather simple, a lot
of verbiage was needed before we could even state them:  we needed to define
``term,'' ``wff,'' and so on.  In addition, there are a number of implied
rules that we haven't even mentioned. For example, how do we know that step 3
of our proof follows from axiom A1? There is some hidden reasoning involved in
determining this.  Axiom A1 has two occurrences of the letter $ t$.  One of
the implied rules states that whatever we substitute for $ t$ must be a legal
term\index{term}.\footnote{Some authors make this implied rule explicit by
stating, ``only expressions of the above form are terms,'' after defining
``term.''}  The expression $ t+0$ is pretty obviously a legal term whenever $
t$ is, but suppose we wanted to substitute a huge term with thousands of
symbols?  Certainly a lot of work would be involved in determining that it
really is a term, but in ordinary formal proofs all of this work would be
considered a single ``step.''

To express our axiom system in the Metamath\index{Metamath} language, we must
describe this auxiliary information in addition to the axioms themselves.
Metamath does not know what a ``term'' or a ``wff''\index{well-formed formula
(wff)} is.  In Metamath, the specification of the ways in which we can combine
symbols to obtain terms and wffs are like little axioms in themselves.  These
auxiliary axioms are expressed in the same notation as the ``real''
axioms\index{axiom}, and Metamath does not distinguish between the two.  The
distinction is made by you, i.e.\ by the way in which you interpret the
notation you have chosen to express these two kinds of axioms.

The Metamath language breaks down mathematical proofs into tiny pieces, much
more so than in ordinary formal proofs\index{formal proof}.  If a single
step\index{proof step} involves the
substitution\index{substitution!variable}\index{variable substitution} of a
complex term for one of its variables, Metamath must see this single step
broken down into many small steps.  This fine-grained breakdown is what gives
Metamath generality and flexibility as it lets it not be limited to any
particular mathematical notation.

Metamath's proof notation is not, in itself, intended to be read by humans but
rather is in a compact format intended for a machine.  The Metamath program
will convert this notation to a form you can understand, using the \texttt{show
proof}\index{\texttt{show proof} command} command.  You can tell the program what
level of detail of the proof you want to look at.  You may want to look at
just the logical inference steps that correspond
to ordinary formal proof steps,
or you may want to see the fine-grained steps that prove that an expression is
a term.

Here, without further ado, is our example converted to the
Metamath\index{Metamath} language:\index{metavariable}\label{demo0}

\begin{verbatim}
$( Declare the constant symbols we will use $)
    $c 0 + = -> ( ) term wff |- $.
$( Declare the metavariables we will use $)
    $v t r s P Q $.
$( Specify properties of the metavariables $)
    tt $f term t $.
    tr $f term r $.
    ts $f term s $.
    wp $f wff P $.
    wq $f wff Q $.
$( Define "term" and "wff" $)
    tze $a term 0 $.
    tpl $a term ( t + r ) $.
    weq $a wff t = r $.
    wim $a wff ( P -> Q ) $.
$( State the axioms $)
    a1 $a |- ( t = r -> ( t = s -> r = s ) ) $.
    a2 $a |- ( t + 0 ) = t $.
$( Define the modus ponens inference rule $)
    ${
       min $e |- P $.
       maj $e |- ( P -> Q ) $.
       mp  $a |- Q $.
    $}
$( Prove a theorem $)
    th1 $p |- t = t $=
  $( Here is its proof: $)
       tt tze tpl tt weq tt tt weq tt a2 tt tze tpl
       tt weq tt tze tpl tt weq tt tt weq wim tt a2
       tt tze tpl tt tt a1 mp mp
     $.
\end{verbatim}\index{metavariable}

A ``database''\index{database} is a set of one or more {\sc ascii} source
files.  Here's a brief description of this Metamath\index{Metamath} database
(which consists of this single source file), so that you can understand in
general terms what is going on.  To understand the source file in detail, you
should read Chapter~\ref{languagespec}.

The database is a sequence of ``tokens,''\index{token} which are normally
separated by spaces or line breaks.  The only tokens that are built into
the Metamath language are those beginning with \texttt{\$}.  These tokens
are called ``keywords.''\index{keyword}  All other tokens are
user-defined, and their names are arbitrary.

As you might have guessed, the Metamath token \texttt{\$(}\index{\texttt{\$(} and
\texttt{\$)} auxiliary keywords} starts a comment and \texttt{\$)} ends a comment.

The Metamath tokens \texttt{\$c}\index{\texttt{\$c} statement},
\texttt{\$v}\index{\texttt{\$v} statement},
\texttt{\$e}\index{\texttt{\$e} statement},
\texttt{\$f}\index{\texttt{\$f} statement},
\texttt{\$a}\index{\texttt{\$a} statement}, and
\texttt{\$p}\index{\texttt{\$p} statement} specify ``statements'' that
end with \texttt{\$.}\,.\index{\texttt{\$.}\ keyword}

The Metamath tokens \texttt{\$c} and \texttt{\$v} each declare\index{constant
declaration}\index{variable declaration} a list of user-defined tokens, called
``math symbols,''\index{math symbol} that the database will reference later
on.  All of the math symbols they define you have seen earlier except the
turnstile symbol \texttt{|-} ($\vdash$)\index{turnstile ({$\,\vdash$})}, which is
commonly used by logicians to mean ``a proof exists for.''  For us
the turnstile is just a
convenient symbol that distinguishes expressions that are axioms\index{axiom}
or theorems\index{theorem} from expressions that are terms or wffs.

The \texttt{\$c} statement declares ``constants''\index{constant} and
the \texttt{\$v} statement declares
``variables''\index{variable}\index{constant declaration}\index{variable
declaration} (or more precisely, metavariables\index{metavariable}).  A
variable may be substituted\index{substitution!variable}\index{variable
substitution} with sequences of math symbols whereas a constant may not
be substituted with anything.

It may seem redundant to require both \texttt{\$c}\index{\texttt{\$c} statement} and
\texttt{\$v}\index{\texttt{\$v} statement} statements (since any math
symbol\index{math symbol} not specified with a \texttt{\$c} statement could be
presumed to be a variable), but this provides for better error checking and
also allows math symbols to be redeclared\index{redeclaration of symbols}
(Section~\ref{scoping}).

The token \texttt{\$f}\index{\texttt{\$f} statement} specifies a
statement called a ``variable-type hypothesis'' (also called a
``floating hypothesis'') and \texttt{\$e}\index{\texttt{\$e} statement}
specifies a ``logical hypothesis'' (also called an ``essential
hypothesis'').\index{hypothesis}\index{variable-type
hypothesis}\index{logical hypothesis}\index{floating
hypothesis}\index{essential hypothesis} The token
\texttt{\$a}\index{\texttt{\$a} statement} specifies an ``axiomatic
assertion,''\index{axiomatic assertion} and
\texttt{\$p}\index{\texttt{\$p} statement} specifies a ``provable
assertion.''\index{provable assertion} To the left of each occurrence of
these four tokens is a ``label''\index{label} that identifies the
hypothesis or assertion for later reference.  For example, the label of
the first axiomatic assertion is \texttt{tze}.  A \texttt{\$f} statement
must contain exactly two math symbols, a constant followed by a
variable.  The \texttt{\$e}, \texttt{\$a}, and \texttt{\$p} statements
each start with a constant followed by, in general, an arbitrary
sequence of math symbols.

Associated with each assertion\index{assertion} is a set of hypotheses
that must be satisfied in order for the assertion to be used in a proof.
These are called the ``mandatory hypotheses''\index{mandatory
hypothesis} of the assertion.  Among those hypotheses whose ``scope''
(described below) includes the assertion, \texttt{\$e} hypotheses are
always mandatory and \texttt{\$f}\index{\texttt{\$f} statement}
hypotheses are mandatory when they share their variable with the
assertion or its \texttt{\$e} hypotheses.  The exact rules for
determining which hypotheses are mandatory are described in detail in
Sections~\ref{frames} and \ref{scoping}.  For example, the mandatory
hypotheses of assertion \texttt{tpl} are \texttt{tt} and \texttt{tr},
whereas assertion \texttt{tze} has no mandatory hypotheses because it
contains no variables and has no \texttt{\$e}\index{\texttt{\$e}
statement} hypothesis.  Metamath's \texttt{show statement}
command\index{\texttt{show statement} command}, described in the next
section, will show you a statement's mandatory hypotheses.

Sometimes we need to make a hypothesis relevant to only certain
assertions.  The set of statements to which a hypothesis is relevant is
called its ``scope.''  The Metamath brackets,
\texttt{\$\char`\{}\index{\texttt{\$\char`\{} and \texttt{\$\char`\}}
keywords} and \texttt{\$\char`\}}, define a ``block''\index{block} that
delimits the scope of any hypothesis contained between them.  The
assertion \texttt{mp} has mandatory hypotheses \texttt{wp}, \texttt{wq},
\texttt{min}, and \texttt{maj}.  The only mandatory hypothesis of
\texttt{th1}, on the other hand, is \texttt{tt}, since \texttt{th1}
occurs outside of the block containing \texttt{min} and \texttt{maj}.

Note that \texttt{\$\char`\{} and \texttt{\$\char`\}} do not affect the
scope of assertions (\texttt{\$a} and \texttt{\$p}).  Assertions are always
available to be referenced by any later proof in the source file.

Each provable assertion (\texttt{\$p}\index{\texttt{\$p} statement}
statement) has two parts.  The first part is the
assertion\index{assertion} itself, which is a sequence of math
symbol\index{math symbol} tokens placed between the \texttt{\$p} token
and a \texttt{\$=}\index{\texttt{\$=} keyword} token.  The second part
is a ``proof,'' which is a list of label tokens placed between the
\texttt{\$=} token and the \texttt{\$.}\index{\texttt{\$.}\ keyword}\
token that ends the statement.\footnote{If you've looked at the
\texttt{set.mm} database, you may have noticed another notation used for
proofs.  The other notation is called ``compressed.''\index{compressed
proof}\index{proof!compressed} It reduces the amount of space needed to
store a proof in the database and is described in
Appendix~\ref{compressed}.  In the example above, we use
``normal''\index{normal proof}\index{proof!normal} notation.} The proof
acts as a series of instructions to the Metamath program, telling it how
to build up the sequence of math symbols contained in the assertion part of
the \texttt{\$p} statement, making use of the hypotheses of the
\texttt{\$p} statement and previous assertions.  The construction takes
place according to precise rules.  If the list of labels in the proof
causes these rules to be violated, or if the final sequence that results
does not match the assertion, the Metamath program will notify you with
an error message.

If you are familiar with reverse Polish notation (RPN), which is sometimes used
on pocket calculators, here in a nutshell is how a proof works.  Each
hypothesis label\index{hypothesis label} in the proof is pushed\index{push}
onto the RPN stack\index{stack}\index{RPN stack} as it is encountered. Each
assertion label\index{assertion label} pops\index{pop} off the stack as many
entries as the referenced assertion has mandatory hypotheses.  Variable
substitutions\index{substitution!variable}\index{variable substitution} are
computed which, when made to the referenced assertion's mandatory hypotheses,
cause these hypotheses to match the stack entries. These same substitutions
are then made to the variables in the referenced assertion itself, which is
then pushed onto the stack.  At the end of the proof, there should be one
stack entry, namely the assertion being proved.  This process is explained in
detail in Section~\ref{proof}.

Metamath's proof notation is not very readable for humans, but it allows the
proof to be stored compactly in a file.  The Metamath\index{Metamath} program
has proof display features that let you see what's going on in a more
readable way, as you will see in the next section.

The rules used in verifying a proof are not based on any built-in syntax of the
symbol sequence in an assertion\index{assertion} nor on any built-in meanings
attached to specific symbol names.  They are based strictly on symbol
matching:  constants\index{constant} must match themselves, and
variables\index{variable} may be replaced with anything that allows a match to
occur.  For example, instead of \texttt{term}, \texttt{0}, and \verb$|-$ we could
have just as well used \texttt{yellow}, \texttt{zero}, and \texttt{provable}, as long
as we did so consistently throughout the database.  Also, we could have used
\texttt{is provable} (two tokens) instead of \verb$|-$ (one token) throughout the
database.  In each of these cases, the proof would be exactly the same.  The
independence of proofs and notation means that you have a lot of flexibility to
change the notation you use without having to change any proofs.

\section{A Trial Run}\label{trialrun}

Now you are ready to try out the Metamath\index{Metamath} program.

On all computer systems, Metamath has a standard ``command line
interface'' (CLI)\index{command line interface (CLI)} that allows you to
interact with it.  You supply commands to the CLI by typing them on the
keyboard and pressing your keyboard's {\em return} key after each line
you enter.  The CLI is designed to be easy to use and has built-in help
features.

The first thing you should do is to use a text editor to create a file
called \texttt{demo0.mm} and type into it the Metamath source shown on
p.~\pageref{demo0}.  Actually, this file is included with your Metamath
software package, so check that first.  If you type it in, make sure
that you save it in the form of ``plain {\sc ascii} text with line
breaks.''  Most word processors will have this feature.

Next you must run the Metamath program.  Depending on your computer
system and how Metamath is installed, this could range from clicking the
mouse on the Metamath icon to typing \texttt{run metamath} to typing
simply \texttt{metamath}.  (Metamath's {\tt help invoke} command describes
alternate ways of invoking the Metamath program.)

When you first enter Metamath\index{Metamath}, it will be at the CLI, waiting
for your input. You will see something like the following on your screen:
\begin{verbatim}
Metamath - Version 0.172 25-Jan-2019
Type HELP for help, EXIT to exit.
MM>
\end{verbatim}
The \texttt{MM>} prompt means that Metamath is waiting for a command.
Command keywords\index{command keyword} are not case sensitive;
we will use lower-case commands in our examples.
The version number and its release date will probably be different on your
system from the one we show above.

The first thing that you need to do is to read in your
database:\index{\texttt{read} command}\footnote{If a directory path is
needed on Unix,\index{Unix file names}\index{file names!Unix} you should
enclose the path/file name in quotes to prevent Metamath from thinking
that the \texttt{/} in the path name is a command qualifier, e.g.,
\texttt{read \char`\"db/set.mm\char`\"}.  Quotes are optional when there
is no ambiguity.}
\begin{verbatim}
MM> read demo0.mm
\end{verbatim}
Remember to press the {\em return} key after entering this command.  If
you omit the file name, Metamath will prompt you for one.   The syntax for
specifying a Macintosh file name path is given in a footnote on
p.~\pageref{includef}.\index{Macintosh file names}\index{file
names!Macintosh}

If there are any syntax errors in the database, Metamath will let you know
when it reads in the file.  The one thing that Metamath does not check when
reading in a database is that all proofs are correct, because this would
slow it down too much.  It is a good idea to periodically verify the proofs in
a database you are making changes to.  To do this, use the following command
(and do it for your \texttt{demo0.mm} file now).  Note that the \texttt{*} is a
``wild card'' meaning all proofs in the file.\index{\texttt{verify proof} command}
\begin{verbatim}
MM> verify proof *
\end{verbatim}
Metamath will report any proofs that are incorrect.

It is often useful to save the information that the Metamath program displays
on the screen. You can save everything that happens on the screen by opening a
log file. You may want to do this before you read in a database so that you
can examine any errors later on.  To open a log file, type
\begin{verbatim}
MM> open log abc.log
\end{verbatim}
This will open a file called \texttt{abc.log}, and everything that appears on the
screen from this point on will be stored in this file.  The name of the log file
is arbitrary. To close the log file, type
\begin{verbatim}
MM> close log
\end{verbatim}

Several commands let you examine what's inside your database.
Section~\ref{exploring} has an overview of some useful ones.  The
\texttt{show labels} command lets you see what statement
labels\index{label} exist.  A \texttt{*} matches any combination of
characters, and \texttt{t*} refers to all labels starting with the
letter \texttt{t}.\index{\texttt{show labels} command} The \texttt{/all}
is a ``command qualifier''\index{command qualifier} that tells Metamath
to include labels of hypotheses.  (To see the syntax explained, type
\texttt{help show labels}.)  Type
\begin{verbatim}
MM> show labels t* /all
\end{verbatim}
Metamath will respond with
\begin{verbatim}
The statement number, label, and type are shown.
3 tt $f       4 tr $f       5 ts $f       8 tze $a
9 tpl $a      19 th1 $p
\end{verbatim}

You can use the \texttt{show statement} command to get information about a
particular statement.\index{\texttt{show statement} command}
For example, you can get information about the statement with label \texttt{mp}
by typing
\begin{verbatim}
MM> show statement mp /full
\end{verbatim}
Metamath will respond with
\begin{verbatim}
Statement 17 is located on line 43 of the file
"demo0.mm".
"Define the modus ponens inference rule"
17 mp $a |- Q $.
Its mandatory hypotheses in RPN order are:
  wp $f wff P $.
  wq $f wff Q $.
  min $e |- P $.
  maj $e |- ( P -> Q ) $.
The statement and its hypotheses require the
      variables:  Q P
The variables it contains are:  Q P
\end{verbatim}
The mandatory hypotheses\index{mandatory hypothesis} and their
order\index{RPN order} are
useful to know when you are trying to understand or debug a proof.

Now you are ready to look at what's really inside our proof.  First, here is
how to look at every step in the proof---not just the ones corresponding to an
ordinary formal proof\index{formal proof}, but also the ones that build up the
formulas that appear in each ordinary formal proof step.\index{\texttt{show
proof} command}
\begin{verbatim}
MM> show proof th1 /lemmon /all
\end{verbatim}

This will display the proof on the screen in the following format:
\begin{verbatim}
 1 tt            $f term t
 2 tze           $a term 0
 3 1,2 tpl       $a term ( t + 0 )
 4 tt            $f term t
 5 3,4 weq       $a wff ( t + 0 ) = t
 6 tt            $f term t
 7 tt            $f term t
 8 6,7 weq       $a wff t = t
 9 tt            $f term t
10 9 a2          $a |- ( t + 0 ) = t
11 tt            $f term t
12 tze           $a term 0
13 11,12 tpl     $a term ( t + 0 )
14 tt            $f term t
15 13,14 weq     $a wff ( t + 0 ) = t
16 tt            $f term t
17 tze           $a term 0
18 16,17 tpl     $a term ( t + 0 )
19 tt            $f term t
20 18,19 weq     $a wff ( t + 0 ) = t
21 tt            $f term t
22 tt            $f term t
23 21,22 weq     $a wff t = t
24 20,23 wim     $a wff ( ( t + 0 ) = t -> t = t )
25 tt            $f term t
26 25 a2         $a |- ( t + 0 ) = t
27 tt            $f term t
28 tze           $a term 0
29 27,28 tpl     $a term ( t + 0 )
30 tt            $f term t
31 tt            $f term t
32 29,30,31 a1   $a |- ( ( t + 0 ) = t -> ( ( t + 0 )
                                     = t -> t = t ) )
33 15,24,26,32 mp  $a |- ( ( t + 0 ) = t -> t = t )
34 5,8,10,33 mp  $a |- t = t
\end{verbatim}

The \texttt{/lemmon} command qualifier specifies what is known as a Lemmon-style
display\index{Lemmon-style proof}\index{proof!Lemmon-style}.  Omitting the
\texttt{/lemmon} qualifier results in a tree-style proof (see
p.~\pageref{treeproof} for an example) that is somewhat less explicit but
easier to follow once you get used to it.\index{tree-style
proof}\index{proof!tree-style}

The first number on each line is the step
number of the proof.  Any numbers that follow are step numbers assigned to the
hypotheses of the statement referenced by that step.  Next is the label of
the statement referenced by the step.  The statement type of the statement
referenced comes next, followed by the math symbol\index{math symbol} string
constructed by the proof up to that step.

The last step, 34, contains the statement that is being proved.

Looking at a small piece of the proof, notice that steps 3 and 4 have
established that
\texttt{( t + 0 )} and \texttt{t} are \texttt{term}\,s, and step 5 makes use of steps 3 and
4 to establish that \texttt{( t + 0 ) = t} is a \texttt{wff}.  Let Metamath
itself tell us in detail what is happening in step 5.  Note that the
``target hypothesis'' refers to where step 5 is eventually used, i.e., in step
34.
\begin{verbatim}
MM> show proof th1 /detailed_step 5
Proof step 5:  wp=weq $a wff ( t + 0 ) = t
This step assigns source "weq" ($a) to target "wp"
($f).  The source assertion requires the hypotheses
"tt" ($f, step 3) and "tr" ($f, step 4).  The parent
assertion of the target hypothesis is "mp" ($a,
step 34).
The source assertion before substitution was:
    weq $a wff t = r
The following substitutions were made to the source
assertion:
    Variable  Substituted with
     t         ( t + 0 )
     r         t
The target hypothesis before substitution was:
    wp $f wff P
The following substitution was made to the target
hypothesis:
    Variable  Substituted with
     P         ( t + 0 ) = t
\end{verbatim}

The full proof just shown is useful to understand what is going on in detail.
However, most of the time you will just be interested in
the ``essential'' or logical steps of a proof, i.e.\ those steps
that correspond to an
ordinary formal proof\index{formal proof}.  If you type
\begin{verbatim}
MM> show proof th1 /lemmon /renumber
\end{verbatim}
you will see\label{demoproof}
\begin{verbatim}
1 a2             $a |- ( t + 0 ) = t
2 a2             $a |- ( t + 0 ) = t
3 a1             $a |- ( ( t + 0 ) = t -> ( ( t + 0 )
                                     = t -> t = t ) )
4 2,3 mp         $a |- ( ( t + 0 ) = t -> t = t )
5 1,4 mp         $a |- t = t
\end{verbatim}
Compare this to the formal proof on p.~\pageref{zeroproof} and
notice the resemblance.
By default Metamath
does not show \texttt{\$f}\index{\texttt{\$f}
statement} hypotheses and everything branching off of them in the proof tree
when the proof is displayed; this makes the proof look more like an ordinary
mathematical proof, which does not normally incorporate the explicit
construction of expressions.
This is called the ``essential'' view
(at one time you had to add the
\texttt{/essential} qualifier in the \texttt{show proof}
command to get this view, but this is now the default).
You can could use the \texttt{/all} qualifier in the \texttt{show
proof} command to also show the explicit construction of expressions.
The \texttt{/renumber} qualifier means to renumber
the steps to correspond only to what is displayed.\index{\texttt{show proof}
command}

To exit Metamath, type\index{\texttt{exit} command}
\begin{verbatim}
MM> exit
\end{verbatim}

\subsection{Some Hints for Using the Command Line Interface}

We will conclude this quick introduction to Metamath\index{Metamath} with some
helpful hints on how to navigate your way through the commands.
\index{command line interface (CLI)}

When you type commands into Metamath's CLI, you only have to type as many
characters of a command keyword\index{command keyword} as are needed to make
it unambiguous.  If you type too few characters, Metamath will tell you what
the choices are.  In the case of the \texttt{read} command, only the \texttt{r} is
needed to specify it unambiguously, so you could have typed\index{\texttt{read}
command}
\begin{verbatim}
MM> r demo0.mm
\end{verbatim}
instead of
\begin{verbatim}
MM> read demo0.mm
\end{verbatim}
In our description, we always show the full command words.  When using the
Metamath CLI commands in a command file (to be read with the \texttt{submit}
command)\index{\texttt{submit} command}, it is good practice to use
the unabbreviated command to ensure your instructions will not become ambiguous
if more commands are added to the Metamath program in the future.

The command keywords\index{command
keyword} are not case sensitive; you may type either \texttt{read} or
\texttt{ReAd}.  File names may or may not be case sensitive, depending on your
computer's operating system.  Metamath label\index{label} and math
symbol\index{math symbol} tokens\index{token} are case-sensitive.

The \texttt{help} command\index{\texttt{help} command} will provide you
with a list of topics you can get help on.  You can then type
\texttt{help} {\em topic} to get help on that topic.

If you are uncertain of a command's spelling, just type as many characters
as you remember of the command.  If you have not typed enough characters to
specify it unambiguously, Metamath will tell you what choices you have.

\begin{verbatim}
MM> show s
         ^
?Ambiguous keyword - please specify SETTINGS,
STATEMENT, or SOURCE.
\end{verbatim}

If you don't know what argument to use as part of a command, type a
\texttt{?}\index{\texttt{]}@\texttt{?}\ in command lines}\ at the
argument position.  Metamath will tell you what it expected there.

\begin{verbatim}
MM> show ?
         ^
?Expected SETTINGS, LABELS, STATEMENT, SOURCE, PROOF,
MEMORY, TRACE_BACK, or USAGE.
\end{verbatim}

Finally, you may type just the first word or words of a command followed
by {\em return}.  Metamath will prompt you for the remaining part of the
command, showing you the choices at each step.  For example, instead of
typing \texttt{show statement th1 /full} you could interact in the
following manner:
\begin{verbatim}
MM> show
SETTINGS, LABELS, STATEMENT, SOURCE, PROOF,
MEMORY, TRACE_BACK, or USAGE <SETTINGS>? st
What is the statement label <th1>?
/ or nothing <nothing>? /
TEX, COMMENT_ONLY, or FULL <TEX>? f
/ or nothing <nothing>?
19 th1 $p |- t = t $= ... $.
\end{verbatim}
After each \texttt{?}\ in this mode, you must give Metamath the
information it requests.  Sometimes Metamath gives you a list of choices
with the default choice indicated by brackets \texttt{< > }. Pressing
{\em return} after the \texttt{?}\ will select the default choice.
Answering anything else will override the default.  Note that the
\texttt{/} in command qualifiers is considered a separate
token\index{token} by the parser, and this is why it is asked for
separately.

\section{Your First Proof}\label{frstprf}

Proofs are developed with the aid of the Proof Assistant\index{Proof
Assistant}.  We will now show you how the proof of theorem \texttt{th1}
was built.  So that you can repeat these steps, we will first have the
Proof Assistant erase the proof in Metamath's source buffer\index{source
buffer}, then reconstruct it.  (The source buffer is the place in memory
where Metamath stores the information in the database when it is
\texttt{read}\index{\texttt{read} command} in.  New or modified proofs
are kept in the source buffer until a \texttt{write source}
command\index{\texttt{write source} command} is issued.)  In practice, you
would place a \texttt{?}\index{\texttt{]}@\texttt{?}\ inside proofs}\
between \texttt{\$=}\index{\texttt{\$=} keyword} and
\texttt{\$.}\index{\texttt{\$.}\ keyword}\ in the database to indicate
to Metamath\index{Metamath} that the proof is unknown, and that would be
your starting point.  Whenever the \texttt{verify proof} command encounters
a proof with a \texttt{?}\ in place of a proof step, the statement is
identified as not proved.

When I first started creating Metamath proofs, I would write down
on a piece of paper the complete
formal proof\index{formal proof} as it would appear
in a \texttt{show proof} command\index{\texttt{show proof} command}; see
the display of \texttt{show proof th1 /lemmon /re\-num\-ber} above as an
example.  After you get used to using the Proof Assistant\index{Proof
Assistant} you may get to a point where you can ``see'' the proof in your mind
and let the Proof Assistant guide you in filling in the details, at least for
simpler proofs, but until you gain that experience you may find it very useful
to write down all the details in advance.
Otherwise you may waste a lot of time as you let it take you down a wrong path.
However, others do not find this approach as helpful.
For example, Thomas Brendan Leahy\index{Leahy, Thomas Brendan}
finds that it is more helpful to him to interactively
work backward from a machine-readable statement.
David A. Wheeler\index{Wheeler, David A.}
writes down a general approach, but develops the proof
interactively by switching between
working forwards (from hypotheses and facts likely to be useful) and
backwards (from the goal) until the forwards and backwards approaches meet.
In the end, use whatever approach works for you.

A proof is developed with the Proof Assistant by working backwards, starting
with the theorem\index{theorem} to be proved, and assigning each unknown step
with a theorem or hypothesis until no more unknown steps remain.  The Proof
Assistant will not let you make an assignment unless it can be ``unified''
with the unknown step.  This means that a
substitution\index{substitution!variable}\index{variable substitution} of
variables exists that will make the assignment match the unknown step.  On the
other hand, in the middle of a proof, when working backwards, often more than
one unification\index{unification} (set of substitutions) is possible, since
there is not enough information available at that point to uniquely establish
it.  In this case you can tell Metamath which unification to choose, or you
can continue to assign unknown steps until enough information is available to
make the unification unique.

We will assume you have entered Metamath and read in the database as described
above.  The following dialog shows how the proof was developed.  For more
details on what some of the commands do, refer to Section~\ref{pfcommands}.
\index{\texttt{prove} command}

\begin{verbatim}
MM> prove th1
Entering the Proof Assistant.  Type HELP for help, EXIT
to exit.  You will be working on the proof of statement th1:
  $p |- t = t
Note:  The proof you are starting with is already complete.
MM-PA>
\end{verbatim}

The \verb/MM-PA>/ prompt means we are inside the Proof
Assistant.\index{Proof Assistant} Most of the regular Metamath commands
(\texttt{show statement}, etc.) are still available if you need them.

\begin{verbatim}
MM-PA> delete all
The entire proof was deleted.
\end{verbatim}

We have deleted the whole proof so we can start from scratch.

\begin{verbatim}
MM-PA> show new_proof/lemmon/all
1 ?              $? |- t = t
\end{verbatim}

The \texttt{show new{\char`\_}proof} command\index{\texttt{show
new{\char`\_}proof} command} is like \texttt{show proof} except that we
don't specify a statement; instead, the proof we're working on is
displayed.

\begin{verbatim}
MM-PA> assign 1 mp
To undo the assignment, DELETE STEP 5 and INITIALIZE, UNIFY
if needed.
3   min=?  $? |- $2
4   maj=?  $? |- ( $2 -> t = t )
\end{verbatim}

The \texttt{assign} command\index{\texttt{assign} command} above means
``assign step 1 with the statement whose label is \texttt{mp}.''  Note
that step renumbering will constantly occur as you assign steps in the
middle of a proof; in general all steps from the step you assign until
the end of the proof will get moved up.  In this case, what used to be
step 1 is now step 5, because the (partial) proof now has five steps:
the four hypotheses of the \texttt{mp} statement and the \texttt{mp}
statement itself.  Let's look at all the steps in our partial proof:

\begin{verbatim}
MM-PA> show new_proof/lemmon/all
1 ?              $? wff $2
2 ?              $? wff t = t
3 ?              $? |- $2
4 ?              $? |- ( $2 -> t = t )
5 1,2,3,4 mp     $a |- t = t
\end{verbatim}

The symbol \texttt{\$2} is a temporary variable\index{temporary
variable} that represents a symbol sequence not yet known.  In the final
proof, all temporary variables will be eliminated.  The general format
for a temporary variable is \texttt{\$} followed by an integer.  Note
that \texttt{\$} is not a legal character in a math symbol (see
Section~\ref{dollardollar}, p.~\pageref{dollardollar}), so there will
never be a naming conflict between real symbols and temporary variables.

Unknown steps 1 and 2 are constructions of the two wffs used by the
modus ponens rule.  As you will see at the end of this section, the
Proof Assistant\index{Proof Assistant} can usually figure these steps
out by itself, and we will not have to worry about them.  Therefore from
here on we will display only the ``essential'' hypotheses, i.e.\ those
steps that correspond to traditional formal proofs\index{formal proof}.

\begin{verbatim}
MM-PA> show new_proof/lemmon
3 ?              $? |- $2
4 ?              $? |- ( $2 -> t = t )
5 3,4 mp         $a |- t = t
\end{verbatim}

Unknown steps 3 and 4 are the ones we must focus on.  They correspond to the
minor and major premises of the modus ponens rule.  We will assign them as
follows.  Notice that because of the step renumbering that takes place
after an assignment, it is advantageous to assign unknown steps in reverse
order, because earlier steps will not get renumbered.

\begin{verbatim}
MM-PA> assign 4 mp
To undo the assignment, DELETE STEP 8 and INITIALIZE, UNIFY
if needed.
3   min=?  $? |- $2
6     min=?  $? |- $4
7     maj=?  $? |- ( $4 -> ( $2 -> t = t ) )
\end{verbatim}

We are now going to describe an obscure feature that you will probably
never use but should be aware of.  The Metamath language allows empty
symbol sequences to be substituted for variables, but in most formal
systems this feature is never used.  One of the few examples where is it
used is the MIU-system\index{MIU-system} described in
Appendix~\ref{MIU}.  But such systems are rare, and by default this
feature is turned off in the Proof Assistant.  (It is always allowed for
{\tt verify proof}.)  Let us turn it on and see what
happens.\index{\texttt{set empty{\char`\_}substitution} command}

\begin{verbatim}
MM-PA> set empty_substitution on
Substitutions with empty symbol sequences is now allowed.
\end{verbatim}

With this feature enabled, more unifications will be
ambiguous\index{ambiguous unification}\index{unification!ambiguous} in
the middle of a proof, because
substitution\index{substitution!variable}\index{variable substitution}
of variables with empty symbol sequences will become an additional
possibility.  Let's see what happens when we make our next assignment.

\begin{verbatim}
MM-PA> assign 3 a2
There are 2 possible unifications.  Please select the correct
    one or Q if you want to UNIFY later.
Unify:  |- $6
 with:  |- ( $9 + 0 ) = $9
Unification #1 of 2 (weight = 7):
  Replace "$6" with "( + 0 ) ="
  Replace "$9" with ""
  Accept (A), reject (R), or quit (Q) <A>? r
\end{verbatim}

The first choice presented is the wrong one.  If we had selected it,
temporary variable \texttt{\$6} would have been assigned a truncated
wff, and temporary variable \texttt{\$9} would have been assigned an
empty sequence (which is not allowed in our system).  With this choice,
eventually we would reach a point where we would get stuck because
we would end up with steps impossible to prove.  (You may want to
try it.)  We typed \texttt{r} to reject the choice.

\begin{verbatim}
Unification #2 of 2 (weight = 21):
  Replace "$6" with "( $9 + 0 ) = $9"
  Accept (A), reject (R), or quit (Q) <A>? q
To undo the assignment, DELETE STEP 4 and INITIALIZE, UNIFY
if needed.
 7     min=?  $? |- $8
 8     maj=?  $? |- ( $8 -> ( $6 -> t = t ) )
\end{verbatim}

The second choice is correct, and normally we would type \texttt{a}
to accept it.  But instead we typed \texttt{q} to show what will happen:
it will leave the step with an unknown unification, which can be
seen as follows:

\begin{verbatim}
MM-PA> show new_proof/not_unified
 4   min    $a |- $6
        =a2  = |- ( $9 + 0 ) = $9
\end{verbatim}

Later we can unify this with the \texttt{unify}
\texttt{all/interactive} command.

The important point to remember is that occasionally you will be
presented with several unification choices while entering a proof, when
the program determines that there is not enough information yet to make
an unambiguous choice automatically (and this can happen even with
\texttt{set empty{\char`\_}substitution} turned off).  Usually it is
obvious by inspection which choice is correct, since incorrect ones will
tend to be meaningless fragments of wffs.  In addition, the correct
choice will usually be the first one presented, unlike our example
above.

Enough of this digression.  Let us go back to the default setting.

\begin{verbatim}
MM-PA> set empty_substitution off
The ability to substitute empty expressions for variables
has been turned off.  Note that this may make the Proof
Assistant too restrictive in some cases.
\end{verbatim}

If we delete the proof, start over, and get to the point where
we digressed above, there will no longer be an ambiguous unification.

\begin{verbatim}
MM-PA> assign 3 a2
To undo the assignment, DELETE STEP 4 and INITIALIZE, UNIFY
if needed.
 7     min=?  $? |- $4
 8     maj=?  $? |- ( $4 -> ( ( $5 + 0 ) = $5 -> t = t ) )
\end{verbatim}

Let us look at our proof so far, and continue.

\begin{verbatim}
MM-PA> show new_proof/lemmon
 4 a2            $a |- ( $5 + 0 ) = $5
 7 ?             $? |- $4
 8 ?             $? |- ( $4 -> ( ( $5 + 0 ) = $5 -> t = t ) )
 9 7,8 mp        $a |- ( ( $5 + 0 ) = $5 -> t = t )
10 4,9 mp        $a |- t = t
MM-PA> assign 8 a1
To undo the assignment, DELETE STEP 11 and INITIALIZE, UNIFY
if needed.
 7     min=?  $? |- ( t + 0 ) = t
MM-PA> assign 7 a2
To undo the assignment, DELETE STEP 8 and INITIALIZE, UNIFY
if needed.
MM-PA> show new_proof/lemmon
 4 a2            $a |- ( t + 0 ) = t
 8 a2            $a |- ( t + 0 ) = t
12 a1            $a |- ( ( t + 0 ) = t -> ( ( t + 0 ) = t ->
                                                    t = t ) )
13 8,12 mp       $a |- ( ( t + 0 ) = t -> t = t )
14 4,13 mp       $a |- t = t
\end{verbatim}

Now all temporary variables and unknown steps have been eliminated from the
``essential'' part of the proof.  When this is achieved, the Proof
Assistant\index{Proof Assistant} can usually figure out the rest of the proof
automatically.  (Note that the \texttt{improve} command can occasionally be
useful for filling in essential steps as well, but it only tries to make use
of statements that introduce no new variables in their hypotheses, which is
not the case for \texttt{mp}. Also it will not try to improve steps containing
temporary variables.)  Let's look at the complete proof, then run
the \texttt{improve} command, then look at it again.

\begin{verbatim}
MM-PA> show new_proof/lemmon/all
 1 ?             $? wff ( t + 0 ) = t
 2 ?             $? wff t = t
 3 ?             $? term t
 4 3 a2          $a |- ( t + 0 ) = t
 5 ?             $? wff ( t + 0 ) = t
 6 ?             $? wff ( ( t + 0 ) = t -> t = t )
 7 ?             $? term t
 8 7 a2          $a |- ( t + 0 ) = t
 9 ?             $? term ( t + 0 )
10 ?             $? term t
11 ?             $? term t
12 9,10,11 a1    $a |- ( ( t + 0 ) = t -> ( ( t + 0 ) = t ->
                                                    t = t ) )
13 5,6,8,12 mp   $a |- ( ( t + 0 ) = t -> t = t )
14 1,2,4,13 mp   $a |- t = t
\end{verbatim}

\begin{verbatim}
MM-PA> improve all
A proof of length 1 was found for step 11.
A proof of length 1 was found for step 10.
A proof of length 3 was found for step 9.
A proof of length 1 was found for step 7.
A proof of length 9 was found for step 6.
A proof of length 5 was found for step 5.
A proof of length 1 was found for step 3.
A proof of length 3 was found for step 2.
A proof of length 5 was found for step 1.
Steps 1 and above have been renumbered.
CONGRATULATIONS!  The proof is complete.  Use SAVE
NEW_PROOF to save it.  Note:  The Proof Assistant does
not detect $d violations.  After saving the proof, you
should verify it with VERIFY PROOF.
\end{verbatim}

The \texttt{save new{\char`\_}proof} command\index{\texttt{save
new{\char`\_}proof} command} will save the proof in the database.  Here
we will just display it in a form that can be clipped out of a log file
and inserted manually into the database source file with a text
editor.\index{normal proof}\index{proof!normal}

\begin{verbatim}
MM-PA> show new_proof/normal
---------Clip out the proof below this line:
      tt tze tpl tt weq tt tt weq tt a2 tt tze tpl tt weq
      tt tze tpl tt weq tt tt weq wim tt a2 tt tze tpl tt
      tt a1 mp mp $.
---------The proof of 'th1' to clip out ends above this line.
\end{verbatim}

There is another proof format called ``compressed''\index{compressed
proof}\index{proof!compressed} that you will see in databases.  It is
not important to understand how it is encoded but only to recognize it
when you see it.  Its only purpose is to reduce storage requirements for
large proofs.  A compressed proof can always be converted to a normal
one and vice-versa, and the Metamath \texttt{show proof}
commands\index{\texttt{show proof} command} work equally well with
compressed proofs.  The compressed proof format is described in
Appendix~\ref{compressed}.

\begin{verbatim}
MM-PA> show new_proof/compressed
---------Clip out the proof below this line:
      ( tze tpl weq a2 wim a1 mp ) ABCZADZAADZAEZJJKFLIA
      AGHH $.
---------The proof of 'th1' to clip out ends above this line.
\end{verbatim}

Now we will exit the Proof Assistant.  Since we made changes to the proof,
it will warn us that we have not saved it.  In this case, we don't care.

\begin{verbatim}
MM-PA> exit
Warning:  You have not saved changes to the proof.
Do you want to EXIT anyway (Y, N) <N>? y
Exiting the Proof Assistant.
Type EXIT again to exit Metamath.
\end{verbatim}

The Proof Assistant\index{Proof Assistant} has several other commands
that can help you while creating proofs.  See Section~\ref{pfcommands}
for a list of them.

A command that is often useful is \texttt{minimize{\char`\_}with
*/brief}, which tries to shorten the proof.  It can make the process
more efficient by letting you write a somewhat ``sloppy'' proof then
clean up some of the fine details of optimization for you (although it
can't perform miracles such as restructuring the overall proof).

\section{A Note About Editing a Data\-base File}

Once your source file contains proofs, there are some restrictions on
how you can edit it so that the proofs remain valid.  Pay particular
attention to these rules, since otherwise you can lose a lot of work.
It is a good idea to periodically verify all proofs with \texttt{verify
proof *} to ensure their integrity.

If your file contains only normal (as opposed to compressed) proofs, the
main rule is that you may not change the order of the mandatory
hypotheses\index{mandatory hypothesis} of any statement referenced in a
later proof.  For example, if you swap the order of the major and minor
premise in the modus ponens rule, all proofs making use of that rule
will become incorrect.  The \texttt{show statement}
command\index{\texttt{show statement} command} will show you the
mandatory hypotheses of a statement and their order.

If a statement has a compressed proof, you also must not change the
order of {\em its} mandatory hypotheses.  The compressed proof format
makes use of this information as part of the compression technique.
Note that swapping the names of two variables in a theorem will change
the order of its mandatory hypotheses.

The safest way to edit a statement, say \texttt{mytheorem}, is to
duplicate it then rename the original to \texttt{mytheoremOLD}
throughout the database.  Once the edited version is re-proved, all
statements referencing \texttt{mytheoremOLD} can be updated in the Proof
Assistant using \texttt{minimize{\char`\_}with
mytheorem
/allow{\char`\_}growth}.\index{\texttt{minimize{\char`\_}with} command}
% 3/10/07 Note: line-breaking the above results in duplicate index entries

\chapter{Abstract Mathematics Revealed}\label{fol}

\section{Logic and Set Theory}\label{logicandsettheory}

\begin{quote}
  {\em Set theory can be viewed as a form of exact theology.}
  \flushright\sc  Rudy Rucker\footnote{\cite{Barrow}, p.~31.}\\
\end{quote}\index{Rucker, Rudy}

Despite its seeming complexity, all of standard mathematics, no matter how
deep or abstract, can amazingly enough be derived from a relatively small set
of axioms\index{axiom} or first principles. The development of these axioms is
among the most impressive and important accomplishments of mathematics in the
20th century. Ultimately, these axioms can be broken down into a set of rules
for manipulating symbols that any technically oriented person can follow.

We will not spend much time trying to convey a deep, higher-level
understanding of the meaning of the axioms. This kind of understanding
requires some mathematical sophistication as well as an understanding of the
philosophy underlying the foundations of mathematics and typically develops
over time as you work with mathematics.  Our goal, instead, is to give you the
immediate ability to follow how theorems\index{theorem} are derived from the
axioms and from other theorems.  This will be similar to learning the syntax
of a computer language, which lets you follow the details in a program but
does not necessarily give you the ability to write non-trivial programs on
your own, an ability that comes with practice. For now don't be alarmed by
abstract-sounding names of the axioms; just focus on the rules for
manipulating the symbols, which follow the simple conventions of the
Metamath\index{Metamath} language.

The axioms that underlie all of standard mathematics consist of axioms of logic
and axioms of set theory. The axioms of logic are divided into two
subcategories, propositional calculus\index{propositional calculus} (sometimes
called sentential logic\index{sentential logic}) and predicate calculus
(sometimes called first-order logic\index{first-order logic}\index{quantifier
theory}\index{predicate calculus} or quantifier theory).  Propositional
calculus is a prerequisite for predicate calculus, and predicate calculus is a
prerequisite for set theory.  The version of set theory most commonly used is
Zermelo--Fraenkel set theory\index{Zermelo--Fraenkel set theory}\index{set theory}
with the axiom of choice,
often abbreviated as ZFC\index{ZFC}.

Here in a nutshell is what the axioms are all about in an informal way. The
connection between this description and symbols we will show you won't be
immediately apparent and in principle needn't ever be.  Our description just
tries to summarize what mathematicians think about when they work with the
axioms.

Logic is a set of rules that allow us determine truths given other truths.
Put another way,
logic is more or less the translation of what we would consider common sense
into a rigorous set of axioms.\index{axioms of logic}  Suppose $\varphi$,
$\psi$, and $\chi$ (the Greek letters phi, psi, and chi) represent statements
that are either true or false, and $x$ is a variable\index{variable!in predicate
calculus} ranging over some group of mathematical objects (sets, integers,
real numbers, etc.). In mathematics, a ``statement'' really means a formula,
and $\psi$ could be for example ``$x = 2$.''
Propositional calculus\index{propositional calculus}
allows us to use variables that are either true or false
and make deductions such as
``if $\varphi$ implies $\psi$ and $\psi$ implies $\chi$, then $\varphi$
implies $\chi$.''
Predicate calculus\index{predicate calculus}
extends propositional calculus by also allowing us
to discuss statements about objects (not just true and false values), including
statements about ``all'' or ``at least one'' object.
For example, predicate calculus allows to say,
``if $\varphi$ is true for all $x$, then $\varphi$ is true for some $x$.''
The logic used in \texttt{set.mm} is standard classical logic
(as opposed to other logic systems like intuitionistic logic).

Set theory\index{set theory} has to do with the manipulation of objects and
collections of objects, specifically the abstract, imaginary objects that
mathematics deals with, such as numbers. Everything that is claimed to exist
in mathematics is considered to be a set.  A set called the empty
set\index{empty set} contains nothing.  We represent the empty set by
$\varnothing$.  Many sets can be built up from the empty set.  There is a set
represented by $\{\varnothing\}$ that contains the empty set, another set
represented by $\{\varnothing,\{\varnothing\}\}$ that contains this set as
well as the empty set, another set represented by $\{\{\varnothing\}\}$ that
contains just the set that contains the empty set, and so on ad infinitum. All
mathematical objects, no matter how complex, are defined as being identical to
certain sets: the integer\index{integer} 0 is defined as the empty set, the
integer 1 is defined as $\{\varnothing\}$, the integer 2 is defined as
$\{\varnothing,\{\varnothing\}\}$.  (How these definitions were chosen doesn't
matter now, but the idea behind it is that these sets have the properties we
expect of integers once suitable operations are defined.)  Mathematical
operations, such as addition, are defined in terms of operations on
sets---their union\index{set union}, intersection\index{set intersection}, and
so on---operations you may have used in elementary school when you worked
with groups of apples and oranges.

With a leap of faith, the axioms also postulate the existence of infinite
sets\index{infinite set}, such as the set of all non-negative integers ($0, 1,
2,\ldots$, also called ``natural numbers''\index{natural number}).  This set
can't be represented with the brace notation\index{brace notation} we just
showed you, but requires a more complicated notation called ``class
abstraction.''\index{class abstraction}\index{abstraction class}  For
example, the infinite set $\{ x |
\mbox{``$x$ is a natural number''} \} $ means the ``set of all objects $x$
such that $x$ is a natural number'' i.e.\ the set of natural numbers; here,
``$x$ is a natural number'' is a rather complicated formula when broken down
into the primitive symbols.\label{expandom}\footnote{The statement ``$x$ is a
natural number'' is formally expressed as ``$x \in \omega$,'' where $\in$
(stylized epsilon) means ``is in'' or ``is an element of'' and $\omega$
(omega) means ``the set of natural numbers.''  When ``$x\in\omega$'' is
completely expanded in terms of the primitive symbols of set theory, the
result is  $\lnot$ $($ $\lnot$ $($ $\forall$ $z$ $($ $\lnot$ $\forall$ $w$ $($
$z$ $\in$ $w$ $\rightarrow$ $\lnot$ $w$ $\in$ $x$ $)$ $\rightarrow$ $z$ $\in$
$x$ $)$ $\rightarrow$ $($ $\forall$ $z$ $($ $\lnot$ $($ $\forall$ $w$ $($ $w$
$\in$ $z$ $\rightarrow$ $w$ $\in$ $x$ $)$ $\rightarrow$ $\forall$ $w$ $\lnot$
$w$ $\in$ $z$ $)$ $\rightarrow$ $\lnot$ $\forall$ $w$ $($ $w$ $\in$ $z$
$\rightarrow$ $\lnot$ $\forall$ $v$ $($ $v$ $\in$ $z$ $\rightarrow$ $\lnot$
$v$ $\in$ $w$ $)$ $)$ $)$ $\rightarrow$ $\lnot$ $\forall$ $z$ $\forall$ $w$
$($ $\lnot$ $($ $z$ $\in$ $x$ $\rightarrow$ $\lnot$ $w$ $\in$ $x$ $)$
$\rightarrow$ $($ $\lnot$ $z$ $\in$ $w$ $\rightarrow$ $($ $\lnot$ $z$ $=$ $w$
$\rightarrow$ $w$ $\in$ $z$ $)$ $)$ $)$ $)$ $)$ $\rightarrow$ $\lnot$
$\forall$ $y$ $($ $\lnot$ $($ $\lnot$ $($ $\forall$ $z$ $($ $\lnot$ $\forall$
$w$ $($ $z$ $\in$ $w$ $\rightarrow$ $\lnot$ $w$ $\in$ $y$ $)$ $\rightarrow$
$z$ $\in$ $y$ $)$ $\rightarrow$ $($ $\forall$ $z$ $($ $\lnot$ $($ $\forall$
$w$ $($ $w$ $\in$ $z$ $\rightarrow$ $w$ $\in$ $y$ $)$ $\rightarrow$ $\forall$
$w$ $\lnot$ $w$ $\in$ $z$ $)$ $\rightarrow$ $\lnot$ $\forall$ $w$ $($ $w$
$\in$ $z$ $\rightarrow$ $\lnot$ $\forall$ $v$ $($ $v$ $\in$ $z$ $\rightarrow$
$\lnot$ $v$ $\in$ $w$ $)$ $)$ $)$ $\rightarrow$ $\lnot$ $\forall$ $z$
$\forall$ $w$ $($ $\lnot$ $($ $z$ $\in$ $y$ $\rightarrow$ $\lnot$ $w$ $\in$
$y$ $)$ $\rightarrow$ $($ $\lnot$ $z$ $\in$ $w$ $\rightarrow$ $($ $\lnot$ $z$
$=$ $w$ $\rightarrow$ $w$ $\in$ $z$ $)$ $)$ $)$ $)$ $\rightarrow$ $($
$\forall$ $z$ $\lnot$ $z$ $\in$ $y$ $\rightarrow$ $\lnot$ $\forall$ $w$ $($
$\lnot$ $($ $w$ $\in$ $y$ $\rightarrow$ $\lnot$ $\forall$ $z$ $($ $w$ $\in$
$z$ $\rightarrow$ $\lnot$ $z$ $\in$ $y$ $)$ $)$ $\rightarrow$ $\lnot$ $($
$\lnot$ $\forall$ $z$ $($ $w$ $\in$ $z$ $\rightarrow$ $\lnot$ $z$ $\in$ $y$
$)$ $\rightarrow$ $w$ $\in$ $y$ $)$ $)$ $)$ $)$ $\rightarrow$ $x$ $\in$ $y$
$)$ $)$ $)$. Section~\ref{hierarchy} shows the hierarchy of definitions that
leads up to this expression.}\index{stylized epsilon ($\in$)}\index{omega
($\omega$)}  Actually, the primitive symbols don't even include the brace
notation.  The brace notation is a high-level definition, which you can find in
Section~\ref{hierarchy}.

Interestingly, the arithmetic of integers\index{integer} and
rationals\index{rational number} can be developed without appealing to the
existence of an infinite set, whereas the arithmetic of real
numbers\index{real number} requires it.

Each variable\index{variable!in set theory} in the axioms of set theory
represents an arbitrary set, and the axioms specify the legal kinds of things
you can do with these variables at a very primitive level.

Now, you may think that numbers and arithmetic are a lot more intuitive and
fundamental than sets and therefore should be the foundation of mathematics.
What is really the case is that you've dealt with numbers all your life and
are comfortable with a few rules for manipulating them such as addition and
multiplication.  Those rules only cover a small portion of what can be done
with numbers and only a very tiny fraction of the rest of mathematics.  If you
look at any elementary book on number theory, you will quickly become lost if
these are the only rules that you know.  Even though such books may present a
list of ``axioms''\index{axiom} for arithmetic, the ability to use the axioms
and to understand proofs of theorems\index{theorem} (facts) about numbers
requires an implicit mathematical talent that frustrates many people
from studying abstract mathematics.  The kind of mathematics that most people
know limits them to the practical, everyday usage of blindly manipulating
numbers and formulas, without any understanding of why those rules are correct
nor any ability to go any further.  For example, do you know why multiplying
two negative numbers yields a positive number?  Starting with set theory, you
will also start off blindly manipulating symbols according to the rules we give
you, but with the advantage that these rules will allow you, in principle, to
access {\em all} of mathematics, not just a tiny part of it.

Of course, concrete examples are often helpful in the learning process. For
example, you can verify that $2\cdot 3=3 \cdot 2$ by actually grouping
objects and can easily ``see'' how it generalizes to $x\cdot y = y\cdot x$,
even though you might not be able to rigorously prove it.  Similarly, in set
theory it can be helpful to understand how the axioms of set theory apply to
(and are correct for) small finite collections of objects.  You should be aware
that in set theory intuition can be misleading for infinite collections, and
rigorous proofs become more important.  For example, while $x\cdot y = y\cdot
x$ is correct for finite ordinals (which are the natural numbers), it is not
usually true for infinite ordinals.

\section{The Axioms for All of Mathematics}

In this section\index{axioms for mathematics}, we will show you the axioms
for all of standard mathematics (i.e.\ logic and set theory) as they are
traditionally presented.  The traditional presentation is useful for someone
with the mathematical experience needed to correctly manipulate high-level
abstract concepts.  For someone without this talent, knowing how to actually
make use of these axioms can be difficult.  The purpose of this section is to
allow you to see how the version of the axioms used in the standard
Metamath\index{Metamath} database \texttt{set.mm}\index{set
theory database (\texttt{set.mm})} relates to  the typical version
in textbooks, and also to give you an informal feel for them.

\subsection{Propositional Calculus}

Propositional calculus\index{propositional calculus} concerns itself with
statements that can be interpreted as either true or false.  Some examples of
statements (outside of mathematics) that are either true or false are ``It is
raining today'' and ``The United States has a female president.'' In
mathematics, as we mentioned, statements are really formulas.

In propositional calculus, we don't care what the statements are.  We also
treat a logical combination of statements, such as ``It is raining today and
the United States has a female president,'' no differently from a single
statement.  Statements and their combinations are called well-formed formulas
(wffs)\index{well-formed formula (wff)}.  We define wffs only in terms of
other wffs and don't define what a ``starting'' wff is.  As is common practice
in the literature, we use Greek letters to represent wffs.

Specifically, suppose $\varphi$ and $\psi$ are wffs.  Then the combinations
$\varphi\rightarrow\psi$ (``$\varphi$ implies $\psi$,'' also read ``if
$\varphi$ then $\psi$'')\index{implication ($\rightarrow$)} and $\lnot\varphi$
(``not $\varphi$'')\index{negation ($\lnot$)} are also wffs.

The three axioms of propositional calculus\index{axioms of propositional
calculus} are all wffs of the following form:\footnote{A remarkable result of
C.~A.~Meredith\index{Meredith, C. A.} squeezes these three axioms into the
single axiom $((((\varphi\rightarrow \psi)\rightarrow(\neg \chi\rightarrow\neg
\theta))\rightarrow \chi )\rightarrow \tau)\rightarrow((\tau\rightarrow
\varphi)\rightarrow(\theta\rightarrow \varphi))$ \cite{CAMeredith},
which is believed to be the shortest possible.}
\begin{center}
     $\varphi\rightarrow(\psi\rightarrow \varphi)$\\

     $(\varphi\rightarrow (\psi\rightarrow \chi))\rightarrow
((\varphi\rightarrow  \psi)\rightarrow (\varphi\rightarrow \chi))$\\

     $(\neg \varphi\rightarrow \neg\psi)\rightarrow (\psi\rightarrow
\varphi)$
\end{center}

These three axioms are widely used.
They are attributed to Jan {\L}ukasiewicz
(pronounced woo-kah-SHAY-vitch) and was popularized by Alonzo Church,
who called it system P2. (Thanks to Ted Ulrich for this information.)

There are an infinite number of axioms, one for each possible
wff\index{well-formed formula (wff)} of the above form.  (For this reason,
axioms such as the above are often called ``axiom schemes.''\index{axiom
scheme})  Each Greek letter in the axioms may be substituted with a more
complex wff to result in another axiom.  For example, substituting
$\neg(\varphi\rightarrow\chi)$ for $\varphi$ in the first axiom yields
$\neg(\varphi\rightarrow\chi)\rightarrow(\psi\rightarrow
\neg(\varphi\rightarrow\chi))$, which is still an axiom.

To deduce new true statements (theorems\index{theorem}) from the axioms, a
rule\index{rule} called ``modus ponens''\index{modus ponens} is used.  This
rule states that if the wff $\varphi$ is an axiom or a theorem, and the wff
$\varphi\rightarrow\psi$ is an axiom or a theorem, then the wff $\psi$ is also
a theorem\index{theorem}.

As a non-mathematical example of modus ponens, suppose we have proved (or
taken as an axiom) ``Bob is a man'' and separately have proved (or taken as
an axiom) ``If Bob is a man, then Bob is a human.''  Using the rule of modus
ponens, we can logically deduce, ``Bob is a human.''

From Metamath's\index{Metamath} point of view, the axioms and the rule of
modus ponens just define a mechanical means for deducing new true statements
from existing true statements, and that is the complete content of
propositional calculus as far as Metamath is concerned.  You can read a logic
textbook to gain a better understanding of their meaning, or you can just let
their meaning slowly become apparent to you after you use them for a while.

It is actually rather easy to check to see if a formula is a theorem of
propositional calculus.  Theorems of propositional calculus are also called
``tautologies.''\index{tautology}  The technique to check whether a formula is
a tautology is called the ``truth table method,''\index{truth table} and it
works like this.  A wff $\varphi\rightarrow\psi$ is false whenever $\varphi$ is true
and $\psi$ is false.  Otherwise it is true.  A wff $\lnot\varphi$ is false
whenever $\varphi$ is true and false otherwise. To verify a tautology such as
$\varphi\rightarrow(\psi\rightarrow \varphi)$, you break it down into sub-wffs and
construct a truth table that accounts for all possible combinations of true
and false assigned to the wff metavariables:
\begin{center}\begin{tabular}{|c|c|c|c|}\hline
\mbox{$\varphi$} & \mbox{$\psi$} & \mbox{$\psi\rightarrow\varphi$}
    & \mbox{$\varphi\rightarrow(\psi\rightarrow \varphi)$} \\ \hline \hline
              T   &  T    &      T       &        T    \\ \hline
              T   &  F    &      T       &        T    \\ \hline
              F   &  T    &      F       &        T    \\ \hline
              F   &  F    &      T       &        T    \\ \hline
\end{tabular}\end{center}
If all entries in the last column are true, the formula is a tautology.

Now, the truth table method doesn't tell you how to prove the tautology from
the axioms, but only that a proof exists.  Finding an actual proof (especially
one that is short and elegant) can be challenging.  Methods do exist for
automatically generating proofs in propositional calculus, but the proofs that
result can sometimes be very long.  In the Metamath \texttt{set.mm}\index{set
theory database (\texttt{set.mm})} database, most
or all proofs were created manually.

Section \ref{metadefprop} discusses various definitions
that make propositional calculus easier to use.
For example, we define:

\begin{itemize}
\item $\varphi \vee \psi$
  is true if either $\varphi$ or $\psi$ (or both) are true
  (this is disjunction\index{disjunction ($\vee$)}
  aka logical {\sc or}\index{logical {\sc or} ($\vee$)}).

\item $\varphi \wedge \psi$
  is true if both $\varphi$ and $\psi$ are true
  (this is conjunction\index{conjunction ($\wedge$)}
  aka logical {\sc and}\index{logical {\sc and} ($\wedge$)}).

\item $\varphi \leftrightarrow \psi$
  is true if $\varphi$ and $\psi$ have the same value, that is,
  they are both true or both false
  (this is the biconditional\index{biconditional ($\leftrightarrow$)}).
\end{itemize}

\subsection{Predicate Calculus}

Predicate calculus\index{predicate calculus} introduces the concept of
``individual variables,''\index{variable!in predicate calculus}\index{individual
variable} which
we will usually just call ``variables.''
These variables can represent something other than true or false (wffs),
and will always represent sets when we get to set theory.  There are also
three new symbols $\forall$\index{universal quantifier ($\forall$)},
$=$\index{equality ($=$)}, and $\in$\index{stylized epsilon ($\in$)},
read ``for all,'' ``equals,'' and ``is an element of''
respectively.  We will represent variables with the letters $x$, $y$, $z$, and
$w$, as is common practice in the literature.
For example, $\forall x \varphi$ means ``for all possible values of
$x$, $\varphi$ is true.''

In predicate calculus, we extend the definition of a wff\index{well-formed
formula (wff)}.  If $\varphi$ is a wff and $x$ and $y$ are variables, then
$\forall x \, \varphi$, $x=y$, and $x\in y$ are wffs. Note that these three new
types of wffs can be considered ``starting'' wffs from which we can build
other wffs with $\rightarrow$ and $\neg$ .  The concept of a starting wff was
absent in propositional calculus.  But starting wff or not, all we are really
concerned with is whether our wffs are correctly constructed according to
these mechanical rules.

A quick aside:
To prevent confusion, it might be best at this point to think of the variables
of Metamath\index{Metamath} as ``metavariables,''\index{metavariable} because
they are not quite the same as the variables we are introducing here.  A
(meta)variable in Metamath can be a wff or an individual variable, as well
as many other things; in general, it represents a kind of place holder for an
unspecified sequence of math symbols\index{math symbol}.

Unlike propositional calculus, no decision procedure\index{decision procedure}
analogous to the truth table method exists (nor theoretically can exist) that
will definitely determine whether a formula is a theorem of predicate
calculus.  Much of the work in the field of automated theorem
proving\index{automated theorem proving} has been dedicated to coming up with
clever heuristics for proving theorems of predicate calculus, but they can
never be guaranteed to work always.

Section \ref{metadefpred} discusses various definitions
that make predicate calculus easier to use.
For example, we define
$\exists x \varphi$ to mean
``there exists at least one possible value of $x$ where $\varphi$ is true.''

We now turn to looking at how predicate calculus can be formally
represented.

\subsubsection{Common Axioms}

There is a new rule of inference in predicate calculus:  if $\varphi$ is
an axiom or a theorem, then $\forall x \,\varphi$ is also a
theorem\index{theorem}.  This is called the rule of
``generalization.''\index{rule of generalization}
This is easily represented in Metamath.

In standard texts of logic, there are often two axioms of predicate
calculus\index{axioms of predicate calculus}:
\begin{center}
  $\forall x \,\varphi ( x ) \rightarrow \varphi ( y )$,
      where ``$y$ is properly substituted for $x$.''\\
  $\forall x ( \varphi \rightarrow \psi )\rightarrow ( \varphi \rightarrow
    \forall x\, \psi )$,
    where ``$x$ is not free in $\varphi$.''
\end{center}

Now at first glance, this seems simple:  just two axioms.  However,
conditional clauses are attached to each axiom describing requirements that
may seem puzzling to you.  In addition, the first axiom puts a variable symbol
in parentheses after each wff, seemingly violating our definition of a
wff\index{well-formed formula (wff)}; this is just an informal way of
referring to some arbitrary variable that may occur in the wff.  The
conditional clauses do, of course, have a precise meaning, but as it turns out
the precise meaning is somewhat complicated and awkward to formalize in a
way that a computer can handle easily.  Unlike propositional calculus, a
certain amount of mathematical sophistication and practice is needed to be
able to easily grasp and manipulate these concepts correctly.

Predicate calculus may be presented with or without axioms for
equality\index{axioms of equality}\index{equality ($=$)}. We will require the
axioms of equality as a prerequisite for the version of set theory we will
use.  The axioms for equality, when included, are often represented using these
two axioms:
\begin{center}
$x=x$\\ \ \\
$x=y\rightarrow (\varphi(x,x)\rightarrow\varphi(x,y))$ where ``$\varphi(x,y)$
   arises from $\varphi(x,x)$ by replacing some, but not necessarily all,
   free\index{free variable}
   occurrences of $x$ by $y$,\\ provided that $y$ is free for $x$
   in $\varphi(x,x)$.'' \end{center}
% (Mendelson p. 95)
The first equality axiom is simple, but again,
the condition on the second one is
somewhat awkward to implement on a computer.

\subsubsection{Tarski System S2}

Of course, we are not the first to notice the complications of these
predicate calculus axioms when being rigorous.

Well-known logician Alfred Tarski published in 1965
a system he called system S2\cite[p.~77]{Tarski1965}.
Tarski's system is \textit{exactly equivalent} to the traditional textbook
formalization, but (by clever use of equality axioms) it eliminates the
latter's primitive notions of ``proper substitution'' and ``free variable,''
replacing them with direct substitution and the notion of a variable
not occurring in a formula (which we express with distinct variable
constraints).

In advocating his system, Tarski wrote, ``The relatively complicated
character of [free variables and proper substitution] is a source
of certain inconveniences of both practical and theoretical nature;
this is clearly experienced both in teaching an elementary course of
mathematical logic and in formalizing the syntax of predicate logic for
some theoretical purposes''\cite[p.~61]{Tarski1965}\index{Tarski, Alfred}.

\subsubsection{Developing a Metamath Representation}

The standard textbook axioms of predicate calculus are somewhat
cumbersome to implement on a computer because of the complex notions of
``free variable''\index{free variable} and ``proper
substitution.''\index{proper substitution}\index{substitution!proper}
While it is possible to use the Metamath\index{Metamath} language to
implement these concepts, we have chosen not to implement them
as primitive constructs in the
\texttt{set.mm} set theory database.  Instead, we have eliminated them
within the axioms
by carefully crafting the axioms so as to avoid them,
building on Tarski's system S2.  This makes it
easy for a beginner to follow the steps in a proof without knowing any
advanced concepts other than the simple concept of
replacing\index{substitution!variable}\index{variable substitution}
variables with expressions.

In order to develop the concepts of free variable and proper
substitution from the axioms, we use an additional
Metamath statement type called ``disjoint variable
restriction''\index{disjoint variables} that we have not encountered
before.  In the context of the axioms, the statement \texttt{\$d} $ x\,
y$\index{\texttt{\$d} statement} simply means that $x$ and $y$ must be
distinct\index{distinct variables}, i.e.\ they may not be simultaneously
substituted\index{substitution!variable}\index{variable substitution}
with the same variable.  The statement \texttt{\$d} $ x\, \varphi$ means
variable $x$ must not occur in wff $\varphi$.  For the precise
definition of \texttt{\$d}, see Section~\ref{dollard}.

\subsubsection{Metamath representation}

The Metamath axiom system for predicate calculus
defined in set.mm uses Tarski's system S2.
As noted above, this has a different representation
than the traditional textbook formalization,
but it is \textit{exactly equivalent} to the textbook formalization,
and it is \textit{much} easier to work with.
This is reproduced as system S3 in Section 6 of
Megill's formalization \cite{Megill}\index{Megill, Norman}.

There is one exception, Tarski's axiom of existence,
which we label as axiom ax-6.
In the case of ax-6, Tarski's version is weaker because it includes a
distinct variable proviso. If we wish, we can also weaken our version
in this way and still have a metalogically complete system. Theorem
ax6 shows this by deriving, in the presence of the other axioms, our
ax-6 from Tarski's weaker version ax6v. However, we chose the stronger
version for our system because it is simpler to state and easier to use.

Tarski's system was designed for proving specific theorems rather than
more general theorem schemes. However, theorem schemes are much more
efficient than specific theorems for building a body of mathematical
knowledge, since they can be reused with different instances as
needed. While Tarski does derive some theorem schemes from his axioms,
their proofs require concepts that are ``outside'' of the system, such as
induction on formula length. The verification of such proofs is difficult
to automate in a proof verifier. (Specifically, Tarski treats the formulas
of his system as set-theoretical objects. In order to verify the proofs
of his theorem schemes, a proof verifier would need a significant amount
of set theory built into it.)

The Metamath axiom system for predicate calculus extends
Tarski's system to eliminate this difficulty. The additional
``auxilliary'' axiom
schemes (as we will call them in this section; see below) endow Tarski's
system with a nice property we call
metalogical completeness \cite[Remark 9.6]{Megill}\index{Megill, Norman}.
As a result, we can prove any theorem scheme
expressable in the ``simple metalogic'' of Tarski's system by using
only Metamath's direct substitution rule applied to the axiom system
(and no other metalogical or set-theoretical notions ``outside'' of the
system). Simple metalogic consists of schemes containing wff metavariables
(with no arguments) and/or set (also called ``individual'') metavariables,
accompanied by optional provisos each stating that two specified set
metavariables must be distinct or that a specified set metavariable may
not occur in a specified wff metavariable. Metamath's logic and set theory
axiom and rule schemes are all examples of simple metalogic. The schemes
of traditional predicate calculus with equality are examples which are
not simple metalogic, because they use wff metavariables with arguments
and have ``free for'' and ``not free in'' side conditions.

A rigorous justification for this system, using an older but
exactly equivalent set of axioms, can be
found in \cite{Megill}\index{Megill, Norman}.

This allows us to
take a different approach in the Metamath\index{Metamath} database
\texttt{set.mm}\index{set theory database (\texttt{set.mm})}.  We do not
directly use the primitive notions of ``free variable''\index{free variable}
and ``proper substitution''\index{proper
substitution}\index{substitution!proper} at all as primitive constructs.
Instead, we use a set
of axioms that are almost as simple to manipulate as those of
propositional calculus.  Our axiom system avoids complex primitive
notions by effectively embedding the complexity into the axioms
themselves.  As a result, we will end up with a larger number of axioms,
but they are ideally suited for a computer language such as Metamath.
(Section~\ref{metaaxioms} shows these axioms.)

We will not elaborate further
on the ``free variable'' and ``proper substitution''
concepts here.  You may consult
\cite[ch.\ 3--4]{Hamilton}\index{Hamilton, Alan G.} (as well as
many other books) for a precise explanation
of these concepts.  If you intend to do serious mathematical work, it is wise
to become familiar with the traditional textbook approach; even though the
concepts embedded in their axioms require a higher level of sophistication,
they can be more practical to deal with on an everyday, informal basis.  Even
if you are just developing Metamath proofs, familiarity with the traditional
approach can help you arrive at a proof outline much faster, which you can
then convert to the detail required by Metamath.

We do develop proper substitution rules later on, but in set.mm
they are defined as derived constructs; they are not primitives.

You should also note that our system of predicate calculus is specifically
tailored for set theory; thus there are only two specific predicates $=$ and
$\in$ and no functions\index{function!in predicate calculus}
or constants\index{constant!in predicate calculus} unlike more general systems.
We later add these.

\subsection{Set Theory}

Traditional Zermelo--Fraenkel set theory\index{Zermelo--Fraenkel set
theory}\index{set theory} with the Axiom of Choice
has 10 axioms, which can be expressed in the
language of predicate calculus.  In this section, we will list only the
names and brief English descriptions of these axioms, since we will give
you the precise formulas used by the Metamath\index{Metamath} set theory
database \texttt{set.mm} later on.

In the descriptions of the axioms, we assume that $x$, $y$, $z$, $w$, and $v$
represent sets.  These are the same as the variables\index{variable!in set
theory} in our predicate calculus system above, except that now we informally
think of the variables as ranging over sets.  Note that the terms
``object,''\index{object} ``set,''\index{set} ``element,''\index{element}
``collection,''\index{collection} and ``family''\index{family} are synonymous,
as are ``is an element of,'' ``is a member of,''\index{member} ``is contained
in,'' and ``belongs to.''  The different terms are used for convenience; for
example, ``a collection of sets'' is less confusing than ``a set of sets.''
A set $x$ is said to be a ``subset''\index{subset} of $y$ if every element of
$x$ is also an element of $y$; we also say $x$ is ``included in''
$y$.

The axioms are very general and apply to almost any conceivable mathematical
object, and this level of abstraction can be overwhelming at first.  To gain an
intuitive feel, it can be helpful to draw a picture illustrating the concept;
for example, a circle containing dots could represent a collection of sets,
and a smaller circle drawn inside the circle could represent a subset.
Overlapping circles can illustrate intersection and union.  Circles that
illustrate the concepts of set theory are frequently used in elementary
textbooks and are called Venn diagrams\index{Venn diagram}.\index{axioms of
set theory}

1. Axiom of Extensionality:  Two sets are identical if they contain the same
   elements.\index{Axiom of Extensionality}

2. Axiom of Pairing:  The set $\{ x , y \}$ exists.\index{Axiom of Pairing}

3. Axiom of Power Sets:  The power set of a set (the collection of all of
   its subsets) exists.  For example, the power set of $\{x,y\}$ is
   $\{\varnothing,\{x\},\{y\},\{x,y\}\}$ and it exists.\index{Axiom
of Power Sets}

4. Axiom of the Null Set:  The empty set $\varnothing$ exists.\index{Axiom of
the Null Set}

5. Axiom of Union:  The union of a set (the set containing the elements of
   its members) exists.  For example, the union of $\{\{x,y\},\{z\}\}$ is
 $\{x,y,z\}$ and
   it exists.\index{Axiom of Union}

6. Axiom of Regularity:  Roughly, no set can contain itself, nor can there
   be membership ``loops,'' such as a set being an
   element of one of its members.\index{Axiom of Regularity}

7. Axiom of Infinity:  An infinite set exists.  An example of an infinite
   set is the set of all
   integers.\index{Axiom of Infinity}

8. Axiom of Separation:  The set exists that is obtained by restricting $x$
   with some property.  For example, if the set of all integers exists,
   then the set of all even integers exists.\index{Axiom of Separation}

9. Axiom of Replacement:  The range of a function whose domain is restricted
   to the elements of a set $x$, is also a set.  For example, there
   is a function
   from integers (the function's domain) to their squares (its
   range).  If we
   restrict the domain to even integers, its range will become the set of
   squares of even integers, so this axiom asserts that the set of
    squares of even numbers exists.  Technical note:  In general, the
   ``function'' need not be a set but can be a proper class.
   \index{Axiom of Replacement}

10. Axiom of Choice:  Let $x$ be a set whose members are pairwise
  disjoint\index{disjoint sets} (i.e,
  whose members contain no elements in common).  Then there exists another
  set containing one element from each member of $x$.  For
  example, if $x$ is
  $\{\{y,z\},\{w,v\}\}$, where $y$, $z$, $w$, and $v$ are
  different sets, then a set such as $\{z,w\}$
  exists (but the axiom doesn't tell
  us which one).  (Actually the Axiom
  of Choice is redundant if the set $x$, as in this example, has a finite
  number of elements.)\index{Axiom of Choice}

The Axiom of Choice is usually considered an extension of ZF set theory rather
than a proper part of it.  It is sometimes considered philosophically
controversial because it specifies the existence of a set without specifying
what the set is. Constructive logics, including intuitionistic logic,
do not accept the axiom of choice.
Since there is some lingering controversy, we often prefer proofs that do
not use the axiom of choice (where there is a known alternative), and
in some cases we will use weaker axioms than the full axiom of choice.
That said, the axiom of choice is a powerful and widely-accepted tool,
so we do use it when needed.
ZF set theory that includes the Axiom of Choice is
called Zermelo--Fraenkel set theory with choice (ZFC\index{ZFC set theory}).

When expressed symbolically, the Axiom of Separation and the Axiom of
Replacement contain wff symbols and therefore each represent infinitely many
axioms, one for each possible wff. For this reason, they are often called
axiom schemes\index{axiom scheme}\index{well-formed formula (wff)}.

It turns out that the Axiom of the Null Set, the Axiom of Pairing, and the
Axiom of Separation can be derived from the other axioms and are therefore
unnecessary, although they tend to be included in standard texts for various
reasons (historical, philosophical, and possibly because some authors may not
know this).  In the Metamath\index{Metamath} set theory database, these
redundant axioms are derived from the other ones instead of truly
being considered axioms.
This is in keeping with our general goal of minimizing the number of
axioms we must depend on.

\subsection{Other Axioms}

Above we qualified the phrase "all of mathematics" with "essentially."
The main important missing piece is the ability to do category theory,
which requires huge sets (inaccessible cardinals) larger than those
postulated by the ZFC axioms. The Tarski--Grothendieck Axiom postulates
the existence of such sets.
Note that this is the same axiom used by Mizar for supporting
category theory.
The Tarski--Grothendieck axiom
can be viewed as a very strong replacement of the Axiom of Infinity,
the Axiom of Choice, and the Axiom of Power Sets.
The \texttt{set.mm} database includes this axiom; see the database
for details about it.
Again, we only use this axiom when we need to.
You are only likely to encounter or use this axiom if you are doing
category theory, since its use is highly specialized,
so we will not list the Tarsky-Grothendieck axiom
in the short list of axioms below.

Can there be even more axioms?
Of course.
G\"{o}del showed that no finite set of axioms or axiom schemes can completely
describe any consistent theory strong enough to include arithmetic.
But practically speaking, the ones above are the accepted foundation that
almost all mathematicians explicitly or implicitly base their work on.

\section{The Axioms in the Metamath Language}\label{metaaxioms}

Here we list the axioms as they appear in
\texttt{set.mm}\index{set theory database (\texttt{set.mm})} so you can
look them up there easily.  Incidentally, the \texttt{show statement
/tex} command\index{\texttt{show statement} command} was used to
typeset them.

%macros from show statement /tex
\newbox\mlinebox
\newbox\mtrialbox
\newbox\startprefix  % Prefix for first line of a formula
\newbox\contprefix  % Prefix for continuation line of a formula
\def\startm{  % Initialize formula line
  \setbox\mlinebox=\hbox{\unhcopy\startprefix}
}
\def\m#1{  % Add a symbol to the formula
  \setbox\mtrialbox=\hbox{\unhcopy\mlinebox $\,#1$}
  \ifdim\wd\mtrialbox>\hsize
    \box\mlinebox
    \setbox\mlinebox=\hbox{\unhcopy\contprefix $\,#1$}
  \else
    \setbox\mlinebox=\hbox{\unhbox\mtrialbox}
  \fi
}
\def\endm{  % Output the last line of a formula
  \box\mlinebox
}

% \SLASH for \ , \TOR for \/ (text OR), \TAND for /\ (text and)
% This embeds a following forced space to force the space.
\newcommand\SLASH{\char`\\~}
\newcommand\TOR{\char`\\/~}
\newcommand\TAND{/\char`\\~}
%
% Macro to output metamath raw text.
% This assumes \startprefix and \contprefix are set.
% NOTE: "\" is tricky to escape, use \SLASH, \TOR, and \TAND inside.
% Any use of "$ { ~ ^" must be escaped; ~ and ^ must be escaped specially.
% We escape { and } for consistency.
% For more about how this macro written, see:
% https://stackoverflow.com/questions/4073674/
% how-to-disable-indentation-in-particular-section-in-latex/4075706
% Use frenchspacing, or "e." will get an extra space after it.
\newlength\mystoreparindent
\newlength\mystorehangindent
\newenvironment{mmraw}{%
\setlength{\mystoreparindent}{\the\parindent}
\setlength{\mystorehangindent}{\the\hangindent}
\setlength{\parindent}{0pt} % TODO - we'll put in the \startprefix instead
\setlength{\hangindent}{\wd\the\contprefix}
\begin{flushleft}
\begin{frenchspacing}
\begin{tt}
{\unhcopy\startprefix}%
}{%
\end{tt}
\end{frenchspacing}
\end{flushleft}
\setlength{\parindent}{\mystoreparindent}
\setlength{\hangindent}{\mystorehangindent}
\vskip 1ex
}

\needspace{5\baselineskip}
\subsection{Propositional Calculus}\label{propcalc}\index{axioms of
propositional calculus}

\needspace{2\baselineskip}
Axiom of Simplification.\label{ax1}

\setbox\startprefix=\hbox{\tt \ \ ax-1\ \$a\ }
\setbox\contprefix=\hbox{\tt \ \ \ \ \ \ \ \ \ \ }
\startm
\m{\vdash}\m{(}\m{\varphi}\m{\rightarrow}\m{(}\m{\psi}\m{\rightarrow}\m{\varphi}\m{)}
\m{)}
\endm

\needspace{3\baselineskip}
\noindent Axiom of Distribution.

\setbox\startprefix=\hbox{\tt \ \ ax-2\ \$a\ }
\setbox\contprefix=\hbox{\tt \ \ \ \ \ \ \ \ \ \ }
\startm
\m{\vdash}\m{(}\m{(}\m{\varphi}\m{\rightarrow}\m{(}\m{\psi}\m{\rightarrow}\m{\chi}
\m{)}\m{)}\m{\rightarrow}\m{(}\m{(}\m{\varphi}\m{\rightarrow}\m{\psi}\m{)}\m{
\rightarrow}\m{(}\m{\varphi}\m{\rightarrow}\m{\chi}\m{)}\m{)}\m{)}
\endm

\needspace{2\baselineskip}
\noindent Axiom of Contraposition.

\setbox\startprefix=\hbox{\tt \ \ ax-3\ \$a\ }
\setbox\contprefix=\hbox{\tt \ \ \ \ \ \ \ \ \ \ }
\startm
\m{\vdash}\m{(}\m{(}\m{\lnot}\m{\varphi}\m{\rightarrow}\m{\lnot}\m{\psi}\m{)}\m{
\rightarrow}\m{(}\m{\psi}\m{\rightarrow}\m{\varphi}\m{)}\m{)}
\endm


\needspace{4\baselineskip}
\noindent Rule of Modus Ponens.\label{axmp}\index{modus ponens}

\setbox\startprefix=\hbox{\tt \ \ maj\ \$e\ }
\setbox\contprefix=\hbox{\tt \ \ \ \ \ \ \ \ \ }
\startm
\m{\vdash}\m{(}\m{\varphi}\m{\rightarrow}\m{\psi}\m{)}
\endm

\setbox\startprefix=\hbox{\tt \ \ min\ \$e\ }
\setbox\contprefix=\hbox{\tt \ \ \ \ \ \ \ \ \ }
\startm
\m{\vdash}\m{\varphi}
\endm

\setbox\startprefix=\hbox{\tt \ \ ax-mp\ \$a\ }
\setbox\contprefix=\hbox{\tt \ \ \ \ \ \ \ \ \ \ \ }
\startm
\m{\vdash}\m{\psi}
\endm


\needspace{7\baselineskip}
\subsection{Axioms of Predicate Calculus with Equality---Tarski's S2}\index{axioms of predicate calculus}

\needspace{3\baselineskip}
\noindent Rule of Generalization.\index{rule of generalization}

\setbox\startprefix=\hbox{\tt \ \ ax-g.1\ \$e\ }
\setbox\contprefix=\hbox{\tt \ \ \ \ \ \ \ \ \ \ \ \ }
\startm
\m{\vdash}\m{\varphi}
\endm

\setbox\startprefix=\hbox{\tt \ \ ax-gen\ \$a\ }
\setbox\contprefix=\hbox{\tt \ \ \ \ \ \ \ \ \ \ \ \ }
\startm
\m{\vdash}\m{\forall}\m{x}\m{\varphi}
\endm

\needspace{2\baselineskip}
\noindent Axiom of Quantified Implication.

\setbox\startprefix=\hbox{\tt \ \ ax-4\ \$a\ }
\setbox\contprefix=\hbox{\tt \ \ \ \ \ \ \ \ \ \ }
\startm
\m{\vdash}\m{(}\m{\forall}\m{x}\m{(}\m{\forall}\m{x}\m{\varphi}\m{\rightarrow}\m{
\psi}\m{)}\m{\rightarrow}\m{(}\m{\forall}\m{x}\m{\varphi}\m{\rightarrow}\m{
\forall}\m{x}\m{\psi}\m{)}\m{)}
\endm

\needspace{3\baselineskip}
\noindent Axiom of Distinctness.

% Aka: Add $d x ph $.
\setbox\startprefix=\hbox{\tt \ \ ax-5\ \$a\ }
\setbox\contprefix=\hbox{\tt \ \ \ \ \ \ \ \ \ \ }
\startm
\m{\vdash}\m{(}\m{\varphi}\m{\rightarrow}\m{\forall}\m{x}\m{\varphi}\m{)}\m{where}\m{ }\m{\$d}\m{ }\m{x}\m{ }\m{\varphi}\m{ }\m{(}\m{x}\m{ }\m{does}\m{ }\m{not}\m{ }\m{occur}\m{ }\m{in}\m{ }\m{\varphi}\m{)}
\endm

\needspace{2\baselineskip}
\noindent Axiom of Existence.

\setbox\startprefix=\hbox{\tt \ \ ax-6\ \$a\ }
\setbox\contprefix=\hbox{\tt \ \ \ \ \ \ \ \ \ \ }
\startm
\m{\vdash}\m{(}\m{\forall}\m{x}\m{(}\m{x}\m{=}\m{y}\m{\rightarrow}\m{\forall}
\m{x}\m{\varphi}\m{)}\m{\rightarrow}\m{\varphi}\m{)}
\endm

\needspace{2\baselineskip}
\noindent Axiom of Equality.

\setbox\startprefix=\hbox{\tt \ \ ax-7\ \$a\ }
\setbox\contprefix=\hbox{\tt \ \ \ \ \ \ \ \ \ \ }
\startm
\m{\vdash}\m{(}\m{x}\m{=}\m{y}\m{\rightarrow}\m{(}\m{x}\m{=}\m{z}\m{
\rightarrow}\m{y}\m{=}\m{z}\m{)}\m{)}
\endm

\needspace{2\baselineskip}
\noindent Axiom of Left Equality for Binary Predicate.

\setbox\startprefix=\hbox{\tt \ \ ax-8\ \$a\ }
\setbox\contprefix=\hbox{\tt \ \ \ \ \ \ \ \ \ \ \ }
\startm
\m{\vdash}\m{(}\m{x}\m{=}\m{y}\m{\rightarrow}\m{(}\m{x}\m{\in}\m{z}\m{
\rightarrow}\m{y}\m{\in}\m{z}\m{)}\m{)}
\endm

\needspace{2\baselineskip}
\noindent Axiom of Right Equality for Binary Predicate.

\setbox\startprefix=\hbox{\tt \ \ ax-9\ \$a\ }
\setbox\contprefix=\hbox{\tt \ \ \ \ \ \ \ \ \ \ \ }
\startm
\m{\vdash}\m{(}\m{x}\m{=}\m{y}\m{\rightarrow}\m{(}\m{z}\m{\in}\m{x}\m{
\rightarrow}\m{z}\m{\in}\m{y}\m{)}\m{)}
\endm


\needspace{4\baselineskip}
\subsection{Axioms of Predicate Calculus with Equality---Auxiliary}\index{axioms of predicate calculus - auxiliary}

\needspace{2\baselineskip}
\noindent Axiom of Quantified Negation.

\setbox\startprefix=\hbox{\tt \ \ ax-10\ \$a\ }
\setbox\contprefix=\hbox{\tt \ \ \ \ \ \ \ \ \ \ }
\startm
\m{\vdash}\m{(}\m{\lnot}\m{\forall}\m{x}\m{\lnot}\m{\forall}\m{x}\m{\varphi}\m{
\rightarrow}\m{\varphi}\m{)}
\endm

\needspace{2\baselineskip}
\noindent Axiom of Quantifier Commutation.

\setbox\startprefix=\hbox{\tt \ \ ax-11\ \$a\ }
\setbox\contprefix=\hbox{\tt \ \ \ \ \ \ \ \ \ \ }
\startm
\m{\vdash}\m{(}\m{\forall}\m{x}\m{\forall}\m{y}\m{\varphi}\m{\rightarrow}\m{
\forall}\m{y}\m{\forall}\m{x}\m{\varphi}\m{)}
\endm

\needspace{3\baselineskip}
\noindent Axiom of Substitution.

\setbox\startprefix=\hbox{\tt \ \ ax-12\ \$a\ }
\setbox\contprefix=\hbox{\tt \ \ \ \ \ \ \ \ \ \ \ }
\startm
\m{\vdash}\m{(}\m{\lnot}\m{\forall}\m{x}\m{\,x}\m{=}\m{y}\m{\rightarrow}\m{(}
\m{x}\m{=}\m{y}\m{\rightarrow}\m{(}\m{\varphi}\m{\rightarrow}\m{\forall}\m{x}\m{(}
\m{x}\m{=}\m{y}\m{\rightarrow}\m{\varphi}\m{)}\m{)}\m{)}\m{)}
\endm

\needspace{3\baselineskip}
\noindent Axiom of Quantified Equality.

\setbox\startprefix=\hbox{\tt \ \ ax-13\ \$a\ }
\setbox\contprefix=\hbox{\tt \ \ \ \ \ \ \ \ \ \ \ }
\startm
\m{\vdash}\m{(}\m{\lnot}\m{\forall}\m{z}\m{\,z}\m{=}\m{x}\m{\rightarrow}\m{(}
\m{\lnot}\m{\forall}\m{z}\m{\,z}\m{=}\m{y}\m{\rightarrow}\m{(}\m{x}\m{=}\m{y}
\m{\rightarrow}\m{\forall}\m{z}\m{\,x}\m{=}\m{y}\m{)}\m{)}\m{)}
\endm

% \noindent Axiom of Quantifier Substitution
%
% \setbox\startprefix=\hbox{\tt \ \ ax-c11n\ \$a\ }
% \setbox\contprefix=\hbox{\tt \ \ \ \ \ \ \ \ \ \ \ }
% \startm
% \m{\vdash}\m{(}\m{\forall}\m{x}\m{\,x}\m{=}\m{y}\m{\rightarrow}\m{(}\m{\forall}
% \m{x}\m{\varphi}\m{\rightarrow}\m{\forall}\m{y}\m{\varphi}\m{)}\m{)}
% \endm
%
% \noindent Axiom of Distinct Variables. (This axiom requires
% that two individual variables
% be distinct\index{\texttt{\$d} statement}\index{distinct
% variables}.)
%
% \setbox\startprefix=\hbox{\tt \ \ \ \ \ \ \ \ \$d\ }
% \setbox\contprefix=\hbox{\tt \ \ \ \ \ \ \ \ \ \ \ }
% \startm
% \m{x}\m{\,}\m{y}
% \endm
%
% \setbox\startprefix=\hbox{\tt \ \ ax-c16\ \$a\ }
% \setbox\contprefix=\hbox{\tt \ \ \ \ \ \ \ \ \ \ \ }
% \startm
% \m{\vdash}\m{(}\m{\forall}\m{x}\m{\,x}\m{=}\m{y}\m{\rightarrow}\m{(}\m{\varphi}\m{
% \rightarrow}\m{\forall}\m{x}\m{\varphi}\m{)}\m{)}
% \endm

% \noindent Axiom of Quantifier Introduction (2).  (This axiom requires
% that the individual variable not occur in the
% wff\index{\texttt{\$d} statement}\index{distinct variables}.)
%
% \setbox\startprefix=\hbox{\tt \ \ \ \ \ \ \ \ \$d\ }
% \setbox\contprefix=\hbox{\tt \ \ \ \ \ \ \ \ \ \ \ }
% \startm
% \m{x}\m{\,}\m{\varphi}
% \endm
% \setbox\startprefix=\hbox{\tt \ \ ax-5\ \$a\ }
% \setbox\contprefix=\hbox{\tt \ \ \ \ \ \ \ \ \ \ \ }
% \startm
% \m{\vdash}\m{(}\m{\varphi}\m{\rightarrow}\m{\forall}\m{x}\m{\varphi}\m{)}
% \endm

\subsection{Set Theory}\label{mmsettheoryaxioms}

In order to make the axioms of set theory\index{axioms of set theory} a little
more compact, there are several definitions from logic that we make use of
implicitly, namely, ``logical {\sc and},''\index{conjunction ($\wedge$)}
\index{logical {\sc and} ($\wedge$)} ``logical equivalence,''\index{logical
equivalence ($\leftrightarrow$)}\index{biconditional ($\leftrightarrow$)} and
``there exists.''\index{existential quantifier ($\exists$)}

\begin{center}\begin{tabular}{rcl}
  $( \varphi \wedge \psi )$ &\mbox{stands for}& $\neg ( \varphi
     \rightarrow \neg \psi )$\\
  $( \varphi \leftrightarrow \psi )$& \mbox{stands
     for}& $( ( \varphi \rightarrow \psi ) \wedge
     ( \psi \rightarrow \varphi ) )$\\
  $\exists x \,\varphi$ &\mbox{stands for}& $\neg \forall x \neg \varphi$
\end{tabular}\end{center}

In addition, the axioms of set theory require that all variables be
dis\-tinct,\index{distinct variables}\footnote{Set theory axioms can be
devised so that {\em no} variables are required to be distinct,
provided we replace \texttt{ax-c16} with an axiom stating that ``at
least two things exist,'' thus
making \texttt{ax-5} the only other axiom requiring the
\texttt{\$d} statement.  These axioms are unconventional and are not
presented here, but they can be found on the \url{http://metamath.org}
web site.  See also the Comment on
p.~\pageref{nodd}.}\index{\texttt{\$d} statement} thus we also assume:
\begin{center}
  \texttt{\$d }$x\,y\,z\,w$
\end{center}

\needspace{2\baselineskip}
\noindent Axiom of Extensionality.\index{Axiom of Extensionality}

\setbox\startprefix=\hbox{\tt \ \ ax-ext\ \$a\ }
\setbox\contprefix=\hbox{\tt \ \ \ \ \ \ \ \ \ \ \ \ }
\startm
\m{\vdash}\m{(}\m{\forall}\m{x}\m{(}\m{x}\m{\in}\m{y}\m{\leftrightarrow}\m{x}
\m{\in}\m{z}\m{)}\m{\rightarrow}\m{y}\m{=}\m{z}\m{)}
\endm

\needspace{3\baselineskip}
\noindent Axiom of Replacement.\index{Axiom of Replacement}

\setbox\startprefix=\hbox{\tt \ \ ax-rep\ \$a\ }
\setbox\contprefix=\hbox{\tt \ \ \ \ \ \ \ \ \ \ \ \ }
\startm
\m{\vdash}\m{(}\m{\forall}\m{w}\m{\exists}\m{y}\m{\forall}\m{z}\m{(}\m{%
\forall}\m{y}\m{\varphi}\m{\rightarrow}\m{z}\m{=}\m{y}\m{)}\m{\rightarrow}\m{%
\exists}\m{y}\m{\forall}\m{z}\m{(}\m{z}\m{\in}\m{y}\m{\leftrightarrow}\m{%
\exists}\m{w}\m{(}\m{w}\m{\in}\m{x}\m{\wedge}\m{\forall}\m{y}\m{\varphi}\m{)}%
\m{)}\m{)}
\endm

\needspace{2\baselineskip}
\noindent Axiom of Union.\index{Axiom of Union}

\setbox\startprefix=\hbox{\tt \ \ ax-un\ \$a\ }
\setbox\contprefix=\hbox{\tt \ \ \ \ \ \ \ \ \ \ \ }
\startm
\m{\vdash}\m{\exists}\m{x}\m{\forall}\m{y}\m{(}\m{\exists}\m{x}\m{(}\m{y}\m{
\in}\m{x}\m{\wedge}\m{x}\m{\in}\m{z}\m{)}\m{\rightarrow}\m{y}\m{\in}\m{x}\m{)}
\endm

\needspace{2\baselineskip}
\noindent Axiom of Power Sets.\index{Axiom of Power Sets}

\setbox\startprefix=\hbox{\tt \ \ ax-pow\ \$a\ }
\setbox\contprefix=\hbox{\tt \ \ \ \ \ \ \ \ \ \ \ \ }
\startm
\m{\vdash}\m{\exists}\m{x}\m{\forall}\m{y}\m{(}\m{\forall}\m{x}\m{(}\m{x}\m{
\in}\m{y}\m{\rightarrow}\m{x}\m{\in}\m{z}\m{)}\m{\rightarrow}\m{y}\m{\in}\m{x}
\m{)}
\endm

\needspace{3\baselineskip}
\noindent Axiom of Regularity.\index{Axiom of Regularity}

\setbox\startprefix=\hbox{\tt \ \ ax-reg\ \$a\ }
\setbox\contprefix=\hbox{\tt \ \ \ \ \ \ \ \ \ \ \ \ }
\startm
\m{\vdash}\m{(}\m{\exists}\m{x}\m{\,x}\m{\in}\m{y}\m{\rightarrow}\m{\exists}
\m{x}\m{(}\m{x}\m{\in}\m{y}\m{\wedge}\m{\forall}\m{z}\m{(}\m{z}\m{\in}\m{x}\m{
\rightarrow}\m{\lnot}\m{z}\m{\in}\m{y}\m{)}\m{)}\m{)}
\endm

\needspace{3\baselineskip}
\noindent Axiom of Infinity.\index{Axiom of Infinity}

\setbox\startprefix=\hbox{\tt \ \ ax-inf\ \$a\ }
\setbox\contprefix=\hbox{\tt \ \ \ \ \ \ \ \ \ \ \ \ \ \ \ }
\startm
\m{\vdash}\m{\exists}\m{x}\m{(}\m{y}\m{\in}\m{x}\m{\wedge}\m{\forall}\m{y}%
\m{(}\m{y}\m{\in}\m{x}\m{\rightarrow}\m{\exists}\m{z}\m{(}\m{y}\m{\in}\m{z}\m{%
\wedge}\m{z}\m{\in}\m{x}\m{)}\m{)}\m{)}
\endm

\needspace{4\baselineskip}
\noindent Axiom of Choice.\index{Axiom of Choice}

\setbox\startprefix=\hbox{\tt \ \ ax-ac\ \$a\ }
\setbox\contprefix=\hbox{\tt \ \ \ \ \ \ \ \ \ \ \ \ \ \ }
\startm
\m{\vdash}\m{\exists}\m{x}\m{\forall}\m{y}\m{\forall}\m{z}\m{(}\m{(}\m{y}\m{%
\in}\m{z}\m{\wedge}\m{z}\m{\in}\m{w}\m{)}\m{\rightarrow}\m{\exists}\m{w}\m{%
\forall}\m{y}\m{(}\m{\exists}\m{w}\m{(}\m{(}\m{y}\m{\in}\m{z}\m{\wedge}\m{z}%
\m{\in}\m{w}\m{)}\m{\wedge}\m{(}\m{y}\m{\in}\m{w}\m{\wedge}\m{w}\m{\in}\m{x}%
\m{)}\m{)}\m{\leftrightarrow}\m{y}\m{=}\m{w}\m{)}\m{)}
\endm

\subsection{That's It}

There you have it, the axioms for (essentially) all of mathematics!
Wonder at them and stare at them in awe.  Put a copy in your wallet, and
you will carry in your pocket the encoding for all theorems ever proved
and that ever will be proved, from the most mundane to the most
profound.

\section{A Hierarchy of Definitions}\label{hierarchy}

The axioms in the previous section in principle embody everything that can be
done within standard mathematics.  However, it is impractical to accomplish
very much by using them directly, for even simple concepts (from a human
perspective) can involve extremely long, incomprehensible formulas.
Mathematics is made practical by introducing definitions\index{definition}.
Definitions usually introduce new symbols, or at least new relationships among
existing symbols, to abbreviate more complex formulas.  An important
requirement for a definition is that there exist a straightforward
(algorithmic) method for eliminating the abbreviation by expanding it into the
more primitive symbol string that it represents.  Some
important definitions included in
the file \texttt{set.mm} are listed in this section for reference, and also to
give you a feel for why something like $\omega$\index{omega ($\omega$)} (the
set of natural numbers\index{natural number} 0, 1, 2,\ldots) becomes very
complicated when completely expanded into primitive symbols.

What is the motivation for definitions, aside from allowing complicated
expressions to be expressed more simply?  In the case of  $\omega$, one goal is
to provide a basis for the theory of natural numbers.\index{natural number}
Before set theory was invented, a set of axioms for arithmetic, called Peano's
postulates\index{Peano's postulates}, was devised and shown to have the
properties one expects for natural numbers.  Now anyone can postulate a
set of axioms, but if the axioms are inconsistent contradictions can be derived
from them.  Once a contradiction is derived, anything can be trivially
proved, including
all the facts of arithmetic and their negations.  To ensure that an
axiom system is at least as reliable as the axioms for set theory, we can
define sets and operations on those sets that satisfy the new axioms. In the
\texttt{set.mm} Metamath database, we prove that the elements of $\omega$ satisfy
Peano's postulates, and it's a long and hard journey to get there directly
from the axioms of set theory.  But the result is confidence in the
foundations of arithmetic.  And there is another advantage:  we now have all
the tools of set theory at our disposal for manipulating objects that obey the
axioms for arithmetic.

What are the criteria we use for definitions?  First, and of utmost importance,
the definition should not be {\em creative}\index{creative
definition}\index{definition!creative}, that
is it should not allow an expression that previously qualified as a wff but
was not provable, to become provable.   Second, the definition should be {\em
eliminable}\index{definition!eliminability}, that is, there should exist an
algorithmic method for converting any expression using the definition into
a logically equivalent expression that previously qualified as a wff.

In almost all cases below, definitions connect two expressions with either
$\leftrightarrow$ or $=$.  Eliminating\footnote{Here we mean the
elimination that a human might do in his or her head.  To eliminate them as
part of a Metamath proof we would invoke one of a number of
theorems that deal with transitivity of equivalence or equality; there are
many such examples in the proofs in \texttt{set.mm}.} such a definition is a
simple matter of substituting the expression on the left-hand side ({\em
definiendum}\index{definiendum} or thing being defined) with the equivalent,
more primitive expression on the right-hand side ({\em
definiens}\index{definiens} or definition).

Often a definition has variables on the right-hand side which do not appear on
the left-hand side; these are called {\em dummy variables}.\index{dummy
variable!in definitions}  In this case, any
allowable substitution (such as a new, distinct
variable) can be used when the definition is eliminated.  Dummy variables may
be used only if they are {\em effectively bound}\index{effectively bound
variable}, meaning that the definition will remain logically equivalent upon
any substitution of a dummy variable with any other {\em qualifying
expression}\index{qualifying expression}, i.e.\ any symbol string (such as
another variable) that
meets the restrictions on the dummy variable imposed by \texttt{\$d} and
\texttt{\$f} statements.  For example, we could define a constant $\perp$
(inverted tee, meaning logical ``false'') as $( \varphi \wedge \lnot \varphi
)$, i.e.\ ``phi and not phi.''  Here $\varphi$ is effectively bound because the
definition remains logically equivalent when we replace $\varphi$ with any
other wff.  (It is actually \texttt{df-fal}
in \texttt{set.mm}, which defines $\perp$.)

There are two cases where eliminating definitions is a little more
complex.  These cases are the definitions \texttt{df-bi} and
\texttt{df-cleq}.  The first stretches the concept of a definition a
little, as in effect it ``defines a definition;'' however, it meets our
requirements for a definition in that it is eliminable and does not
strengthen the language.  Theorem \texttt{bii} shows the substitution
needed to eliminate the $\leftrightarrow$\index{logical equivalence
($\leftrightarrow$)}\index{biconditional ($\leftrightarrow$)} symbol.

Definition \texttt{df-cleq}\index{equality ($=$)} extends the usage of
the equality symbol to include ``classes''\index{class} in set theory.  The
reason it is potentially problematic is that it can lead to statements which
do not follow from logic alone but presuppose the Axiom of
Extensionality\index{Axiom of Extensionality}, so we include this axiom
as a hypothesis for the definition.  We could have made \texttt{df-cleq} directly
eliminable by introducing a new equality symbol, but have chosen not to do so
in keeping with standard textbook practice.  Definitions such as \texttt{df-cleq}
that extend the meaning of existing symbols must be introduced carefully so
that they do not lead to contradictions.  Definition \texttt{df-clel} also
extends the meaning of an existing symbol ($\in$); while it doesn't strengthen
the language like \texttt{df-cleq}, this is not obvious and it must also be
subject to the same scrutiny.

Exercise:  Study how the wff $x\in\omega$, meaning ``$x$ is a natural
number,'' could be expanded in terms of primitive symbols, starting with the
definitions \texttt{df-clel} on p.~\pageref{dfclel} and \texttt{df-om} on
p.~\pageref{dfom} and working your way back.  Don't bother to work out the
details; just make sure that you understand how you could do it in principle.
The answer is shown in the footnote on p.~\pageref{expandom}.  If you
actually do work it out, you won't get exactly the same answer because we used
a few simplifications such as discarding occurrences of $\lnot\lnot$ (double
negation).

In the definitions below, we have placed the {\sc ascii} Metamath source
below each of the formulas to help you become familiar with the
notation in the database.  For simplicity, the necessary \texttt{\$f}
and \texttt{\$d} statements are not shown.  If you are in doubt, use the
\texttt{show statement}\index{\texttt{show statement} command} command
in the Metamath program to see the full statement.
A selection of this notation is summarized in Appendix~\ref{ASCII}.

To understand the motivation for these definitions, you should consult the
references indicated:  Takeuti and Zaring \cite{Takeuti}\index{Takeuti, Gaisi},
Quine \cite{Quine}\index{Quine, Willard Van Orman}, Bell and Machover
\cite{Bell}\index{Bell, J. L.}, and Enderton \cite{Enderton}\index{Enderton,
Herbert B.}.  Our list of definitions is provided more for reference than as a
learning aid.  However, by looking at a few of them you can gain a feel for
how the hierarchy is built up.  The definitions are a representative sample of
the 130 or so in \texttt{set.mm}, but they are complete with respect to the
theorem examples we will present in Section~\ref{sometheorems}.  Also, some are
slightly different from, but logically equivalent to, the ones in \texttt{set.mm}
(some of which have been revised over time to shorten them, for example).

\subsection{Definitions for Propositional Calculus}\label{metadefprop}

The symbols $\varphi$, $\psi$, and $\chi$ represent wffs.

Our first definition introduces the biconditional
connective\footnote{The term ``connective'' is informally used to mean a
symbol that is placed between two variables or adjacent to a variable,
whereas a mathematical ``constant'' usually indicates a symbol such as
the number 0 that may replace a variable or metavariable.  From
Metamath's point of view, there is no distinction between a connective
and a constant; both are constants in the Metamath
language.}\index{connective}\index{constant} (also called logical
equivalence)\index{logical equivalence
($\leftrightarrow$)}\index{biconditional ($\leftrightarrow$)}.  Unlike
most traditional developments, we have chosen not to have a separate
symbol such as ``Df.'' to mean ``is defined as.''  Instead, we will use
the biconditional connective for this purpose, as it lets us use
logic to manipulate definitions directly.  Here we state the properties
of the biconditional connective with a carefully crafted \texttt{\$a}
statement, which effectively uses the biconditional connective to define
itself.  The $\leftrightarrow$ symbol can be eliminated from a formula
using theorem \texttt{bii}, which is derived later.

\vskip 2ex
\noindent Define the biconditional connective.\label{df-bi}

\vskip 0.5ex
\setbox\startprefix=\hbox{\tt \ \ df-bi\ \$a\ }
\setbox\contprefix=\hbox{\tt \ \ \ \ \ \ \ \ \ \ \ }
\startm
\m{\vdash}\m{\lnot}\m{(}\m{(}\m{(}\m{\varphi}\m{\leftrightarrow}\m{\psi}\m{)}%
\m{\rightarrow}\m{\lnot}\m{(}\m{(}\m{\varphi}\m{\rightarrow}\m{\psi}\m{)}\m{%
\rightarrow}\m{\lnot}\m{(}\m{\psi}\m{\rightarrow}\m{\varphi}\m{)}\m{)}\m{)}\m{%
\rightarrow}\m{\lnot}\m{(}\m{\lnot}\m{(}\m{(}\m{\varphi}\m{\rightarrow}\m{%
\psi}\m{)}\m{\rightarrow}\m{\lnot}\m{(}\m{\psi}\m{\rightarrow}\m{\varphi}\m{)}%
\m{)}\m{\rightarrow}\m{(}\m{\varphi}\m{\leftrightarrow}\m{\psi}\m{)}\m{)}\m{)}
\endm
\begin{mmraw}%
|- -. ( ( ( ph <-> ps ) -> -. ( ( ph -> ps ) ->
-. ( ps -> ph ) ) ) -> -. ( -. ( ( ph -> ps ) -> -. (
ps -> ph ) ) -> ( ph <-> ps ) ) ) \$.
\end{mmraw}

\noindent This theorem relates the biconditional connective to primitive
connectives and can be used to eliminate the $\leftrightarrow$ symbol from any
wff.

\vskip 0.5ex
\setbox\startprefix=\hbox{\tt \ \ bii\ \$p\ }
\setbox\contprefix=\hbox{\tt \ \ \ \ \ \ \ \ \ }
\startm
\m{\vdash}\m{(}\m{(}\m{\varphi}\m{\leftrightarrow}\m{\psi}\m{)}\m{\leftrightarrow}
\m{\lnot}\m{(}\m{(}\m{\varphi}\m{\rightarrow}\m{\psi}\m{)}\m{\rightarrow}\m{\lnot}
\m{(}\m{\psi}\m{\rightarrow}\m{\varphi}\m{)}\m{)}\m{)}
\endm
\begin{mmraw}%
|- ( ( ph <-> ps ) <-> -. ( ( ph -> ps ) -> -. ( ps -> ph ) ) ) \$= ... \$.
\end{mmraw}

\noindent Define disjunction ({\sc or}).\index{disjunction ($\vee$)}%
\index{logical or (vee)@logical {\sc or} ($\vee$)}%
\index{df-or@\texttt{df-or}}\label{df-or}

\vskip 0.5ex
\setbox\startprefix=\hbox{\tt \ \ df-or\ \$a\ }
\setbox\contprefix=\hbox{\tt \ \ \ \ \ \ \ \ \ \ \ }
\startm
\m{\vdash}\m{(}\m{(}\m{\varphi}\m{\vee}\m{\psi}\m{)}\m{\leftrightarrow}\m{(}\m{
\lnot}\m{\varphi}\m{\rightarrow}\m{\psi}\m{)}\m{)}
\endm
\begin{mmraw}%
|- ( ( ph \TOR ps ) <-> ( -. ph -> ps ) ) \$.
\end{mmraw}

\noindent Define conjunction ({\sc and}).\index{conjunction ($\wedge$)}%
\index{logical {\sc and} ($\wedge$)}%
\index{df-an@\texttt{df-an}}\label{df-an}

\vskip 0.5ex
\setbox\startprefix=\hbox{\tt \ \ df-an\ \$a\ }
\setbox\contprefix=\hbox{\tt \ \ \ \ \ \ \ \ \ \ \ }
\startm
\m{\vdash}\m{(}\m{(}\m{\varphi}\m{\wedge}\m{\psi}\m{)}\m{\leftrightarrow}\m{\lnot}
\m{(}\m{\varphi}\m{\rightarrow}\m{\lnot}\m{\psi}\m{)}\m{)}
\endm
\begin{mmraw}%
|- ( ( ph \TAND ps ) <-> -. ( ph -> -. ps ) ) \$.
\end{mmraw}

\noindent Define disjunction ({\sc or}) of 3 wffs.%
\index{df-3or@\texttt{df-3or}}\label{df-3or}

\vskip 0.5ex
\setbox\startprefix=\hbox{\tt \ \ df-3or\ \$a\ }
\setbox\contprefix=\hbox{\tt \ \ \ \ \ \ \ \ \ \ \ \ }
\startm
\m{\vdash}\m{(}\m{(}\m{\varphi}\m{\vee}\m{\psi}\m{\vee}\m{\chi}\m{)}\m{
\leftrightarrow}\m{(}\m{(}\m{\varphi}\m{\vee}\m{\psi}\m{)}\m{\vee}\m{\chi}\m{)}
\m{)}
\endm
\begin{mmraw}%
|- ( ( ph \TOR ps \TOR ch ) <-> ( ( ph \TOR ps ) \TOR ch ) ) \$.
\end{mmraw}

\noindent Define conjunction ({\sc and}) of 3 wffs.%
\index{df-3an}\label{df-3an}

\vskip 0.5ex
\setbox\startprefix=\hbox{\tt \ \ df-3an\ \$a\ }
\setbox\contprefix=\hbox{\tt \ \ \ \ \ \ \ \ \ \ \ \ }
\startm
\m{\vdash}\m{(}\m{(}\m{\varphi}\m{\wedge}\m{\psi}\m{\wedge}\m{\chi}\m{)}\m{
\leftrightarrow}\m{(}\m{(}\m{\varphi}\m{\wedge}\m{\psi}\m{)}\m{\wedge}\m{\chi}
\m{)}\m{)}
\endm

\begin{mmraw}%
|- ( ( ph \TAND ps \TAND ch ) <-> ( ( ph \TAND ps ) \TAND ch ) ) \$.
\end{mmraw}

\subsection{Definitions for Predicate Calculus}\label{metadefpred}

The symbols $x$, $y$, and $z$ represent individual variables of predicate
calculus.  In this section, they are not necessarily distinct unless it is
explicitly
mentioned.

\vskip 2ex
\noindent Define existential quantification.
The expression $\exists x \varphi$ means
``there exists an $x$ where $\varphi$ is true.''\index{existential quantifier
($\exists$)}\label{df-ex}

\vskip 0.5ex
\setbox\startprefix=\hbox{\tt \ \ df-ex\ \$a\ }
\setbox\contprefix=\hbox{\tt \ \ \ \ \ \ \ \ \ \ \ }
\startm
\m{\vdash}\m{(}\m{\exists}\m{x}\m{\varphi}\m{\leftrightarrow}\m{\lnot}\m{\forall}
\m{x}\m{\lnot}\m{\varphi}\m{)}
\endm
\begin{mmraw}%
|- ( E. x ph <-> -. A. x -. ph ) \$.
\end{mmraw}

\noindent Define proper substitution.\index{proper
substitution}\index{substitution!proper}\label{df-sb}
In our notation, we use $[ y / x ] \varphi$ to mean ``the wff that
results when $y$ is properly substituted for $x$ in the wff
$\varphi$.''\footnote{
This can also be described
as substituting $x$ with $y$, $y$ properly replaces $x$, or
$x$ is properly replaced by $y$.}
% This is elsb4, though it currently says: ( [ x / y ] z e. y <-> z e. x )
For example,
$[ y / x ] z \in x$ is the same as $z \in y$.
One way to remember this notation is to notice that it looks like division
and recall that $( y / x ) \cdot x $ is $y$ (when $x \neq 0$).
The notation is different from the notation $\varphi ( x | y )$
that is sometimes used, because the latter notation is ambiguous for us:
for example, we don't know whether $\lnot \varphi ( x | y )$ is to be
interpreted as $\lnot ( \varphi ( x | y ) )$ or
$( \lnot \varphi ) ( x | y )$.\footnote{Because of the way
we initially defined wffs, this is the case
with any postfix connective\index{postfix connective} (one occurring after the
symbols being connected) or infix connective\index{infix connective} (one
occurring between the symbols being connected).  Metamath does not have a
built-in notion of operator binding strength that could eliminate the
ambiguity.  The initial parenthesis effectively provides a prefix
connective\index{prefix connective} to eliminate ambiguity.  Some conventions,
such as Polish notation\index{Polish notation} used in the 1930's and 1940's
by Polish logicians, use only prefix connectives and thus allow the total
elimination of parentheses, at the expense of readability.  In Metamath we
could actually redefine all notation to be Polish if we wanted to without
having to change any proofs!}  Other texts often use $\varphi(y)$ to indicate
our $[ y / x ] \varphi$, but this notation is even more ambiguous since there is
no explicit indication of what is being substituted.
Note that this
definition is valid even when
$x$ and $y$ are the same variable.  The first conjunct is a ``trick'' used to
achieve this property, making the definition look somewhat peculiar at
first.

\vskip 0.5ex
\setbox\startprefix=\hbox{\tt \ \ df-sb\ \$a\ }
\setbox\contprefix=\hbox{\tt \ \ \ \ \ \ \ \ \ \ \ }
\startm
\m{\vdash}\m{(}\m{[}\m{y}\m{/}\m{x}\m{]}\m{\varphi}\m{\leftrightarrow}\m{(}%
\m{(}\m{x}\m{=}\m{y}\m{\rightarrow}\m{\varphi}\m{)}\m{\wedge}\m{\exists}\m{x}%
\m{(}\m{x}\m{=}\m{y}\m{\wedge}\m{\varphi}\m{)}\m{)}\m{)}
\endm
\begin{mmraw}%
|- ( [ y / x ] ph <-> ( ( x = y -> ph ) \TAND E. x ( x = y \TAND ph ) ) ) \$.
\end{mmraw}


\noindent Define existential uniqueness\index{existential uniqueness
quantifier ($\exists "!$)} (``there exists exactly one'').  Note that $y$ is a
variable distinct from $x$ and not occurring in $\varphi$.\label{df-eu}

\vskip 0.5ex
\setbox\startprefix=\hbox{\tt \ \ df-eu\ \$a\ }
\setbox\contprefix=\hbox{\tt \ \ \ \ \ \ \ \ \ \ \ }
\startm
\m{\vdash}\m{(}\m{\exists}\m{{!}}\m{x}\m{\varphi}\m{\leftrightarrow}\m{\exists}
\m{y}\m{\forall}\m{x}\m{(}\m{\varphi}\m{\leftrightarrow}\m{x}\m{=}\m{y}\m{)}\m{)}
\endm

\begin{mmraw}%
|- ( E! x ph <-> E. y A. x ( ph <-> x = y ) ) \$.
\end{mmraw}

\subsection{Definitions for Set Theory}\label{setdefinitions}

The symbols $x$, $y$, $z$, and $w$ represent individual variables of
predicate calculus, which in set theory are understood to be sets.
However, using only the constructs shown so far would be very inconvenient.

To make set theory more practical, we introduce the notion of a ``class.''
A class\index{class} is either a set variable (such as $x$) or an
expression of the form $\{ x | \varphi\}$ (called an ``abstraction
class''\index{abstraction class}\index{class abstraction}).  Note that
sets (i.e.\ individual variables) always exist (this is a theorem of
logic, namely $\exists y \, y = x$ for any set $x$), whereas classes may
or may not exist (i.e.\ $\exists y \, y = A$ may or may not be true).
If a class does not exist it is called a ``proper class.''\index{proper
class}\index{class!proper} Definitions \texttt{df-clab},
\texttt{df-cleq}, and \texttt{df-clel} can be used to convert an
expression containing classes into one containing only set variables and
wff metavariables.

The symbols $A$, $B$, $C$, $D$, $ F$, $G$, and $R$ are metavariables that range
over classes.  A class metavariable $A$ may be eliminated from a wff by
replacing it with $\{ x|\varphi\}$ where neither $x$ nor $\varphi$ occur in
the wff.

The theory of classes can be shown to be an eliminable and conservative
extension of set theory. The \textbf{eliminability}
property shows that for every
formula in the extended language we can build a logically equivalent
formula in the basic language; so that even if the extended language
provides more ease to convey and formulate mathematical ideas for set
theory, its expressive power does not in fact strengthen the basic
language's expressive power.
The \textbf{conservation} property shows that for
every proof of a formula of the basic language in the extended system
we can build another proof of the same formula in the basic system;
so that, concerning theorems on sets only, the deductive powers of
the extended system and of the basic system are identical. Together,
these properties mean that the extended language can be treated as a
definitional extension that is \textbf{sound}.

A rigorous justification, which we will not give here, can be found in
Levy \cite[pp.~357-366]{Levy} supplementing his informal introduction to class
theory on pp.~7-17. Two other good treatments of class theory are provided
by Quine \cite[pp.~15-21]{Quine}\index{Quine, Willard Van Orman}
and also \cite[pp.~10-14]{Takeuti}\index{Takeuti, Gaisi}.
Quine's exposition (he calls them virtual classes)
is nicely written and very readable.

In the rest of this
section, individual variables are always assumed to be distinct from
each other unless otherwise indicated.  In addition, dummy variables on the
right-hand side of a definition do not occur in the class and wff
metavariables in the definition.

The definitions we present here are a partial but self-contained
collection selected from several hundred that appear in the current
\texttt{set.mm} database.  They are adequate for a basic development of
elementary set theory.

\vskip 2ex
\noindent Define the abstraction class.\index{abstraction class}\index{class
abstraction}\label{df-clab}  $x$ and $y$
need not be distinct.  Definition 2.1 of Quine, p.~16.  This definition may
seem puzzling since it is shorter than the expression being defined and does not
buy us anything in terms of brevity.  The reason we introduce this definition
is because it fits in neatly with the extension of the $\in$ connective
provided by \texttt{df-clel}.

\vskip 0.5ex
\setbox\startprefix=\hbox{\tt \ \ df-clab\ \$a\ }
\setbox\contprefix=\hbox{\tt \ \ \ \ \ \ \ \ \ \ \ \ \ }
\startm
\m{\vdash}\m{(}\m{x}\m{\in}\m{\{}\m{y}\m{|}\m{\varphi}\m{\}}\m{%
\leftrightarrow}\m{[}\m{x}\m{/}\m{y}\m{]}\m{\varphi}\m{)}
\endm
\begin{mmraw}%
|- ( x e. \{ y | ph \} <-> [ x / y ] ph ) \$.
\end{mmraw}

\noindent Define the equality connective between classes\index{class
equality}\label{df-cleq}.  See Quine or Chapter 4 of Takeuti and Zaring for its
justification and methods for eliminating it.  This is an example of a
somewhat ``dangerous'' definition, because it extends the use of the
existing equality symbol rather than introducing a new symbol, allowing
us to make statements in the original language that may not be true.
For example, it permits us to deduce $y = z \leftrightarrow \forall x (
x \in y \leftrightarrow x \in z )$ which is not a theorem of logic but
rather presupposes the Axiom of Extensionality,\index{Axiom of
Extensionality} which we include as a hypothesis so that we can know
when this axiom is assumed in a proof (with the \texttt{show
trace{\char`\_}back} command).  We could avoid the danger by introducing
another symbol, say $\eqcirc$, in place of $=$; this
would also have the advantage of making elimination of the definition
straightforward and would eliminate the need for Extensionality as a
hypothesis.  We would then also have the advantage of being able to
identify exactly where Extensionality truly comes into play.  One of our
theorems would be $x \eqcirc y \leftrightarrow x = y$
by invoking Extensionality.  However in keeping with standard practice
we retain the ``dangerous'' definition.

\vskip 0.5ex
\setbox\startprefix=\hbox{\tt \ \ df-cleq.1\ \$e\ }
\setbox\contprefix=\hbox{\tt \ \ \ \ \ \ \ \ \ \ \ \ \ \ \ }
\startm
\m{\vdash}\m{(}\m{\forall}\m{x}\m{(}\m{x}\m{\in}\m{y}\m{\leftrightarrow}\m{x}
\m{\in}\m{z}\m{)}\m{\rightarrow}\m{y}\m{=}\m{z}\m{)}
\endm
\setbox\startprefix=\hbox{\tt \ \ df-cleq\ \$a\ }
\setbox\contprefix=\hbox{\tt \ \ \ \ \ \ \ \ \ \ \ \ \ }
\startm
\m{\vdash}\m{(}\m{A}\m{=}\m{B}\m{\leftrightarrow}\m{\forall}\m{x}\m{(}\m{x}\m{
\in}\m{A}\m{\leftrightarrow}\m{x}\m{\in}\m{B}\m{)}\m{)}
\endm
% We need to reset the startprefix and contprefix.
\setbox\startprefix=\hbox{\tt \ \ df-cleq.1\ \$e\ }
\setbox\contprefix=\hbox{\tt \ \ \ \ \ \ \ \ \ \ \ \ \ \ \ }
\begin{mmraw}%
|- ( A. x ( x e. y <-> x e. z ) -> y = z ) \$.
\end{mmraw}
\setbox\startprefix=\hbox{\tt \ \ df-cleq\ \$a\ }
\setbox\contprefix=\hbox{\tt \ \ \ \ \ \ \ \ \ \ \ \ \ }
\begin{mmraw}%
|- ( A = B <-> A. x ( x e. A <-> x e. B ) ) \$.
\end{mmraw}

\noindent Define the membership connective between classes\index{class
membership}.  Theorem 6.3 of Quine, p.~41, which we adopt as a definition.
Note that it extends the use of the existing membership symbol, but unlike
{\tt df-cleq} it does not extend the set of valid wffs of logic when the class
metavariables are replaced with set variables.\label{dfclel}\label{df-clel}

\vskip 0.5ex
\setbox\startprefix=\hbox{\tt \ \ df-clel\ \$a\ }
\setbox\contprefix=\hbox{\tt \ \ \ \ \ \ \ \ \ \ \ \ \ }
\startm
\m{\vdash}\m{(}\m{A}\m{\in}\m{B}\m{\leftrightarrow}\m{\exists}\m{x}\m{(}\m{x}
\m{=}\m{A}\m{\wedge}\m{x}\m{\in}\m{B}\m{)}\m{)}
\endm
\begin{mmraw}%
|- ( A e. B <-> E. x ( x = A \TAND x e. B ) ) \$.?
\end{mmraw}

\noindent Define inequality.

\vskip 0.5ex
\setbox\startprefix=\hbox{\tt \ \ df-ne\ \$a\ }
\setbox\contprefix=\hbox{\tt \ \ \ \ \ \ \ \ \ \ \ }
\startm
\m{\vdash}\m{(}\m{A}\m{\ne}\m{B}\m{\leftrightarrow}\m{\lnot}\m{A}\m{=}\m{B}%
\m{)}
\endm
\begin{mmraw}%
|- ( A =/= B <-> -. A = B ) \$.
\end{mmraw}

\noindent Define restricted universal quantification.\index{universal
quantifier ($\forall$)!restricted}  Enderton, p.~22.

\vskip 0.5ex
\setbox\startprefix=\hbox{\tt \ \ df-ral\ \$a\ }
\setbox\contprefix=\hbox{\tt \ \ \ \ \ \ \ \ \ \ \ \ }
\startm
\m{\vdash}\m{(}\m{\forall}\m{x}\m{\in}\m{A}\m{\varphi}\m{\leftrightarrow}\m{%
\forall}\m{x}\m{(}\m{x}\m{\in}\m{A}\m{\rightarrow}\m{\varphi}\m{)}\m{)}
\endm
\begin{mmraw}%
|- ( A. x e. A ph <-> A. x ( x e. A -> ph ) ) \$.
\end{mmraw}

\noindent Define restricted existential quantification.\index{existential
quantifier ($\exists$)!restricted}  Enderton, p.~22.

\vskip 0.5ex
\setbox\startprefix=\hbox{\tt \ \ df-rex\ \$a\ }
\setbox\contprefix=\hbox{\tt \ \ \ \ \ \ \ \ \ \ \ \ }
\startm
\m{\vdash}\m{(}\m{\exists}\m{x}\m{\in}\m{A}\m{\varphi}\m{\leftrightarrow}\m{%
\exists}\m{x}\m{(}\m{x}\m{\in}\m{A}\m{\wedge}\m{\varphi}\m{)}\m{)}
\endm
\begin{mmraw}%
|- ( E. x e. A ph <-> E. x ( x e. A \TAND ph ) ) \$.
\end{mmraw}

\noindent Define the universal class\index{universal class ($V$)}.  Definition
5.20, p.~21, of Takeuti and Zaring.\label{df-v}

\vskip 0.5ex
\setbox\startprefix=\hbox{\tt \ \ df-v\ \$a\ }
\setbox\contprefix=\hbox{\tt \ \ \ \ \ \ \ \ \ \ }
\startm
\m{\vdash}\m{{\rm V}}\m{=}\m{\{}\m{x}\m{|}\m{x}\m{=}\m{x}\m{\}}
\endm
\begin{mmraw}%
|- {\char`\_}V = \{ x | x = x \} \$.
\end{mmraw}

\noindent Define the subclass\index{subclass}\index{subset} relationship
between two classes (called the subset relation if the classes are sets i.e.\
are not proper).  Definition 5.9 of Takeuti and Zaring, p.~17.\label{df-ss}

\vskip 0.5ex
\setbox\startprefix=\hbox{\tt \ \ df-ss\ \$a\ }
\setbox\contprefix=\hbox{\tt \ \ \ \ \ \ \ \ \ \ \ }
\startm
\m{\vdash}\m{(}\m{A}\m{\subseteq}\m{B}\m{\leftrightarrow}\m{\forall}\m{x}\m{(}
\m{x}\m{\in}\m{A}\m{\rightarrow}\m{x}\m{\in}\m{B}\m{)}\m{)}
\endm
\begin{mmraw}%
|- ( A C\_ B <-> A. x ( x e. A -> x e. B ) ) \$.
\end{mmraw}

\noindent Define the union\index{union} of two classes.  Definition 5.6 of Takeuti and Zaring,
p.~16.\label{df-un}

\vskip 0.5ex
\setbox\startprefix=\hbox{\tt \ \ df-un\ \$a\ }
\setbox\contprefix=\hbox{\tt \ \ \ \ \ \ \ \ \ \ \ }
\startm
\m{\vdash}\m{(}\m{A}\m{\cup}\m{B}\m{)}\m{=}\m{\{}\m{x}\m{|}\m{(}\m{x}\m{\in}
\m{A}\m{\vee}\m{x}\m{\in}\m{B}\m{)}\m{\}}
\endm
\begin{mmraw}%
( A u. B ) = \{ x | ( x e. A \TOR x e. B ) \} \$.
\end{mmraw}

\noindent Define the intersection\index{intersection} of two classes.  Definition 5.6 of
Takeuti and Zaring, p.~16.\label{df-in}

\vskip 0.5ex
\setbox\startprefix=\hbox{\tt \ \ df-in\ \$a\ }
\setbox\contprefix=\hbox{\tt \ \ \ \ \ \ \ \ \ \ \ }
\startm
\m{\vdash}\m{(}\m{A}\m{\cap}\m{B}\m{)}\m{=}\m{\{}\m{x}\m{|}\m{(}\m{x}\m{\in}
\m{A}\m{\wedge}\m{x}\m{\in}\m{B}\m{)}\m{\}}
\endm
% Caret ^ requires special treatment
\begin{mmraw}%
|- ( A i\^{}i B ) = \{ x | ( x e. A \TAND x e. B ) \} \$.
\end{mmraw}

\noindent Define class difference\index{class difference}\index{set difference}.
Definition 5.12 of Takeuti and Zaring, p.~20.  Several notations are used in
the literature; we chose the $\setminus$ convention instead of a minus sign to
reserve the latter for later use in, e.g., arithmetic.\label{df-dif}

\vskip 0.5ex
\setbox\startprefix=\hbox{\tt \ \ df-dif\ \$a\ }
\setbox\contprefix=\hbox{\tt \ \ \ \ \ \ \ \ \ \ \ \ }
\startm
\m{\vdash}\m{(}\m{A}\m{\setminus}\m{B}\m{)}\m{=}\m{\{}\m{x}\m{|}\m{(}\m{x}\m{
\in}\m{A}\m{\wedge}\m{\lnot}\m{x}\m{\in}\m{B}\m{)}\m{\}}
\endm
\begin{mmraw}%
( A \SLASH B ) = \{ x | ( x e. A \TAND -. x e. B ) \} \$.
\end{mmraw}

\noindent Define the empty or null set\index{empty set}\index{null set}.
Compare  Definition 5.14 of Takeuti and Zaring, p.~20.\label{df-nul}

\vskip 0.5ex
\setbox\startprefix=\hbox{\tt \ \ df-nul\ \$a\ }
\setbox\contprefix=\hbox{\tt \ \ \ \ \ \ \ \ \ \ }
\startm
\m{\vdash}\m{\varnothing}\m{=}\m{(}\m{{\rm V}}\m{\setminus}\m{{\rm V}}\m{)}
\endm
\begin{mmraw}%
|- (/) = ( {\char`\_}V \SLASH {\char`\_}V ) \$.
\end{mmraw}

\noindent Define power class\index{power set}\index{power class}.  Definition 5.10 of
Takeuti and Zaring, p.~17, but we also let it apply to proper classes.  (Note
that \verb$~P$ is the symbol for calligraphic P, the tilde
suggesting ``curly;'' see Appendix~\ref{ASCII}.)\label{df-pw}

\vskip 0.5ex
\setbox\startprefix=\hbox{\tt \ \ df-pw\ \$a\ }
\setbox\contprefix=\hbox{\tt \ \ \ \ \ \ \ \ \ \ \ }
\startm
\m{\vdash}\m{{\cal P}}\m{A}\m{=}\m{\{}\m{x}\m{|}\m{x}\m{\subseteq}\m{A}\m{\}}
\endm
% Special incantation required to put ~ into the text
\begin{mmraw}%
|- \char`\~P~A = \{ x | x C\_ A \} \$.
\end{mmraw}

\noindent Define the singleton of a class\index{singleton}.  Definition 7.1 of
Quine, p.~48.  It is well-defined for proper classes, although
it is not very meaningful in this case, where it evaluates to the empty
set.

\vskip 0.5ex
\setbox\startprefix=\hbox{\tt \ \ df-sn\ \$a\ }
\setbox\contprefix=\hbox{\tt \ \ \ \ \ \ \ \ \ \ \ }
\startm
\m{\vdash}\m{\{}\m{A}\m{\}}\m{=}\m{\{}\m{x}\m{|}\m{x}\m{=}\m{A}\m{\}}
\endm
\begin{mmraw}%
|- \{ A \} = \{ x | x = A \} \$.
\end{mmraw}%

\noindent Define an unordered pair of classes\index{unordered pair}\index{pair}.  Definition
7.1 of Quine, p.~48.

\vskip 0.5ex
\setbox\startprefix=\hbox{\tt \ \ df-pr\ \$a\ }
\setbox\contprefix=\hbox{\tt \ \ \ \ \ \ \ \ \ \ \ }
\startm
\m{\vdash}\m{\{}\m{A}\m{,}\m{B}\m{\}}\m{=}\m{(}\m{\{}\m{A}\m{\}}\m{\cup}\m{\{}
\m{B}\m{\}}\m{)}
\endm
\begin{mmraw}%
|- \{ A , B \} = ( \{ A \} u. \{ B \} ) \$.
\end{mmraw}

\noindent Define an unordered triple of classes\index{unordered triple}.  Definition of
Enderton, p.~19.

\vskip 0.5ex
\setbox\startprefix=\hbox{\tt \ \ df-tp\ \$a\ }
\setbox\contprefix=\hbox{\tt \ \ \ \ \ \ \ \ \ \ \ }
\startm
\m{\vdash}\m{\{}\m{A}\m{,}\m{B}\m{,}\m{C}\m{\}}\m{=}\m{(}\m{\{}\m{A}\m{,}\m{B}
\m{\}}\m{\cup}\m{\{}\m{C}\m{\}}\m{)}
\endm
\begin{mmraw}%
|- \{ A , B , C \} = ( \{ A , B \} u. \{ C \} ) \$.
\end{mmraw}%

\noindent Kuratowski's\index{Kuratowski, Kazimierz} ordered pair\index{ordered
pair} definition.  Definition 9.1 of Quine, p.~58. For proper classes it is
not meaningful but is well-defined for convenience.  (Note that \verb$<.$
stands for $\langle$ whereas \verb$<$ stands for $<$, and similarly for
\verb$>.$\,.)\label{df-op}

\vskip 0.5ex
\setbox\startprefix=\hbox{\tt \ \ df-op\ \$a\ }
\setbox\contprefix=\hbox{\tt \ \ \ \ \ \ \ \ \ \ \ }
\startm
\m{\vdash}\m{\langle}\m{A}\m{,}\m{B}\m{\rangle}\m{=}\m{\{}\m{\{}\m{A}\m{\}}
\m{,}\m{\{}\m{A}\m{,}\m{B}\m{\}}\m{\}}
\endm
\begin{mmraw}%
|- <. A , B >. = \{ \{ A \} , \{ A , B \} \} \$.
\end{mmraw}

\noindent Define the union of a class\index{union}.  Definition 5.5, p.~16,
of Takeuti and Zaring.

\vskip 0.5ex
\setbox\startprefix=\hbox{\tt \ \ df-uni\ \$a\ }
\setbox\contprefix=\hbox{\tt \ \ \ \ \ \ \ \ \ \ \ \ }
\startm
\m{\vdash}\m{\bigcup}\m{A}\m{=}\m{\{}\m{x}\m{|}\m{\exists}\m{y}\m{(}\m{x}\m{
\in}\m{y}\m{\wedge}\m{y}\m{\in}\m{A}\m{)}\m{\}}
\endm
\begin{mmraw}%
|- U. A = \{ x | E. y ( x e. y \TAND y e. A ) \} \$.
\end{mmraw}

\noindent Define the intersection\index{intersection} of a class.  Definition 7.35,
p.~44, of Takeuti and Zaring.

\vskip 0.5ex
\setbox\startprefix=\hbox{\tt \ \ df-int\ \$a\ }
\setbox\contprefix=\hbox{\tt \ \ \ \ \ \ \ \ \ \ \ \ }
\startm
\m{\vdash}\m{\bigcap}\m{A}\m{=}\m{\{}\m{x}\m{|}\m{\forall}\m{y}\m{(}\m{y}\m{
\in}\m{A}\m{\rightarrow}\m{x}\m{\in}\m{y}\m{)}\m{\}}
\endm
\begin{mmraw}%
|- |\^{}| A = \{ x | A. y ( y e. A -> x e. y ) \} \$.
\end{mmraw}

\noindent Define a transitive class\index{transitive class}\index{transitive
set}.  This should not be confused with a transitive relation, which is a different
concept.  Definition from p.~71 of Enderton, extended to classes.

\vskip 0.5ex
\setbox\startprefix=\hbox{\tt \ \ df-tr\ \$a\ }
\setbox\contprefix=\hbox{\tt \ \ \ \ \ \ \ \ \ \ \ }
\startm
\m{\vdash}\m{(}\m{\mbox{\rm Tr}}\m{A}\m{\leftrightarrow}\m{\bigcup}\m{A}\m{
\subseteq}\m{A}\m{)}
\endm
\begin{mmraw}%
|- ( Tr A <-> U. A C\_ A ) \$.
\end{mmraw}

\noindent Define a notation for a general binary relation\index{binary
relation}.  Definition 6.18, p.~29, of Takeuti and Zaring, generalized to
arbitrary classes.  This definition is well-defined, although not very
meaningful, when classes $A$ and/or $B$ are proper.\label{dfbr}  The lack of
parentheses (or any other connective) creates no ambiguity since we are defining
an atomic wff.

\vskip 0.5ex
\setbox\startprefix=\hbox{\tt \ \ df-br\ \$a\ }
\setbox\contprefix=\hbox{\tt \ \ \ \ \ \ \ \ \ \ \ }
\startm
\m{\vdash}\m{(}\m{A}\m{\,R}\m{\,B}\m{\leftrightarrow}\m{\langle}\m{A}\m{,}\m{B}
\m{\rangle}\m{\in}\m{R}\m{)}
\endm
\begin{mmraw}%
|- ( A R B <-> <. A , B >. e. R ) \$.
\end{mmraw}

\noindent Define an abstraction class of ordered pairs\index{abstraction
class!of ordered
pairs}.  A special case of Definition 4.16, p.~14, of Takeuti and Zaring.
Note that $ z $ must be distinct from $ x $ and $ y $,
and $ z $ must not occur in $\varphi$, but $ x $ and $ y $ may be
identical and may appear in $\varphi$.

\vskip 0.5ex
\setbox\startprefix=\hbox{\tt \ \ df-opab\ \$a\ }
\setbox\contprefix=\hbox{\tt \ \ \ \ \ \ \ \ \ \ \ \ \ }
\startm
\m{\vdash}\m{\{}\m{\langle}\m{x}\m{,}\m{y}\m{\rangle}\m{|}\m{\varphi}\m{\}}\m{=}
\m{\{}\m{z}\m{|}\m{\exists}\m{x}\m{\exists}\m{y}\m{(}\m{z}\m{=}\m{\langle}\m{x}
\m{,}\m{y}\m{\rangle}\m{\wedge}\m{\varphi}\m{)}\m{\}}
\endm

\begin{mmraw}%
|- \{ <. x , y >. | ph \} = \{ z | E. x E. y ( z =
<. x , y >. /\ ph ) \} \$.
\end{mmraw}

\noindent Define the epsilon relation\index{epsilon relation}.  Similar to Definition
6.22, p.~30, of Takeuti and Zaring.

\vskip 0.5ex
\setbox\startprefix=\hbox{\tt \ \ df-eprel\ \$a\ }
\setbox\contprefix=\hbox{\tt \ \ \ \ \ \ \ \ \ \ \ \ \ \ }
\startm
\m{\vdash}\m{{\rm E}}\m{=}\m{\{}\m{\langle}\m{x}\m{,}\m{y}\m{\rangle}\m{|}\m{x}\m{
\in}\m{y}\m{\}}
\endm
\begin{mmraw}%
|- \_E = \{ <. x , y >. | x e. y \} \$.
\end{mmraw}

\noindent Define a founded relation\index{founded relation}.  $R$ is a founded
relation on $A$ iff\index{iff} (if and only if) each nonempty subset of $A$
has an ``$R$-minimal element.''  Similar to Definition 6.21, p.~30, of
Takeuti and Zaring.

\vskip 0.5ex
\setbox\startprefix=\hbox{\tt \ \ df-fr\ \$a\ }
\setbox\contprefix=\hbox{\tt \ \ \ \ \ \ \ \ \ \ \ }
\startm
\m{\vdash}\m{(}\m{R}\m{\,\mbox{\rm Fr}}\m{\,A}\m{\leftrightarrow}\m{\forall}\m{x}
\m{(}\m{(}\m{x}\m{\subseteq}\m{A}\m{\wedge}\m{\lnot}\m{x}\m{=}\m{\varnothing}
\m{)}\m{\rightarrow}\m{\exists}\m{y}\m{(}\m{y}\m{\in}\m{x}\m{\wedge}\m{(}\m{x}
\m{\cap}\m{\{}\m{z}\m{|}\m{z}\m{\,R}\m{\,y}\m{\}}\m{)}\m{=}\m{\varnothing}\m{)}
\m{)}\m{)}
\endm
\begin{mmraw}%
|- ( R Fr A <-> A. x ( ( x C\_ A \TAND -. x = (/) ) ->
E. y ( y e. x \TAND ( x i\^{}i \{ z | z R y \} ) = (/) ) ) ) \$.
\end{mmraw}

\noindent Define a well-ordering\index{well-ordering}.  $R$ is a well-ordering of $A$ iff
it is founded on $A$ and the elements of $A$ are pairwise $R$-comparable.
Similar to Definition 6.24(2), p.~30, of Takeuti and Zaring.

\vskip 0.5ex
\setbox\startprefix=\hbox{\tt \ \ df-we\ \$a\ }
\setbox\contprefix=\hbox{\tt \ \ \ \ \ \ \ \ \ \ \ }
\startm
\m{\vdash}\m{(}\m{R}\m{\,\mbox{\rm We}}\m{\,A}\m{\leftrightarrow}\m{(}\m{R}\m{\,
\mbox{\rm Fr}}\m{\,A}\m{\wedge}\m{\forall}\m{x}\m{\forall}\m{y}\m{(}\m{(}\m{x}\m{
\in}\m{A}\m{\wedge}\m{y}\m{\in}\m{A}\m{)}\m{\rightarrow}\m{(}\m{x}\m{\,R}\m{\,y}
\m{\vee}\m{x}\m{=}\m{y}\m{\vee}\m{y}\m{\,R}\m{\,x}\m{)}\m{)}\m{)}\m{)}
\endm
\begin{mmraw}%
( R We A <-> ( R Fr A \TAND A. x A. y ( ( x e.
A \TAND y e. A ) -> ( x R y \TOR x = y \TOR y R x ) ) ) ) \$.
\end{mmraw}

\noindent Define the ordinal predicate\index{ordinal predicate}, which is true for a class
that is transitive and is well-ordered by the epsilon relation.  Similar to
definition on p.~468, Bell and Machover.

\vskip 0.5ex
\setbox\startprefix=\hbox{\tt \ \ df-ord\ \$a\ }
\setbox\contprefix=\hbox{\tt \ \ \ \ \ \ \ \ \ \ \ \ }
\startm
\m{\vdash}\m{(}\m{\mbox{\rm Ord}}\m{\,A}\m{\leftrightarrow}\m{(}
\m{\mbox{\rm Tr}}\m{\,A}\m{\wedge}\m{E}\m{\,\mbox{\rm We}}\m{\,A}\m{)}\m{)}
\endm
\begin{mmraw}%
|- ( Ord A <-> ( Tr A \TAND E We A ) ) \$.
\end{mmraw}

\noindent Define the class of all ordinal numbers\index{ordinal number}.  An ordinal number is
a set that satisfies the ordinal predicate.  Definition 7.11 of Takeuti and
Zaring, p.~38.

\vskip 0.5ex
\setbox\startprefix=\hbox{\tt \ \ df-on\ \$a\ }
\setbox\contprefix=\hbox{\tt \ \ \ \ \ \ \ \ \ \ \ }
\startm
\m{\vdash}\m{\,\mbox{\rm On}}\m{=}\m{\{}\m{x}\m{|}\m{\mbox{\rm Ord}}\m{\,x}
\m{\}}
\endm
\begin{mmraw}%
|- On = \{ x | Ord x \} \$.
\end{mmraw}

\noindent Define the limit ordinal predicate\index{limit ordinal}, which is true for a
non-empty ordinal that is not a successor (i.e.\ that is the union of itself).
Compare Bell and Machover, p.~471 and Exercise (1), p.~42 of Takeuti and
Zaring.

\vskip 0.5ex
\setbox\startprefix=\hbox{\tt \ \ df-lim\ \$a\ }
\setbox\contprefix=\hbox{\tt \ \ \ \ \ \ \ \ \ \ \ \ }
\startm
\m{\vdash}\m{(}\m{\mbox{\rm Lim}}\m{\,A}\m{\leftrightarrow}\m{(}\m{\mbox{
\rm Ord}}\m{\,A}\m{\wedge}\m{\lnot}\m{A}\m{=}\m{\varnothing}\m{\wedge}\m{A}
\m{=}\m{\bigcup}\m{A}\m{)}\m{)}
\endm
\begin{mmraw}%
|- ( Lim A <-> ( Ord A \TAND -. A = (/) \TAND A = U. A ) ) \$.
\end{mmraw}

\noindent Define the successor\index{successor} of a class.  Definition 7.22 of Takeuti
and Zaring, p.~41.  Our definition is a generalization to classes, although it
is meaningless when classes are proper.

\vskip 0.5ex
\setbox\startprefix=\hbox{\tt \ \ df-suc\ \$a\ }
\setbox\contprefix=\hbox{\tt \ \ \ \ \ \ \ \ \ \ \ \ }
\startm
\m{\vdash}\m{\,\mbox{\rm suc}}\m{\,A}\m{=}\m{(}\m{A}\m{\cup}\m{\{}\m{A}\m{\}}
\m{)}
\endm
\begin{mmraw}%
|- suc A = ( A u. \{ A \} ) \$.
\end{mmraw}

\noindent Define the class of natural numbers\index{natural number}\index{omega
($\omega$)}.  Compare Bell and Machover, p.~471.\label{dfom}

\vskip 0.5ex
\setbox\startprefix=\hbox{\tt \ \ df-om\ \$a\ }
\setbox\contprefix=\hbox{\tt \ \ \ \ \ \ \ \ \ \ \ }
\startm
\m{\vdash}\m{\omega}\m{=}\m{\{}\m{x}\m{|}\m{(}\m{\mbox{\rm Ord}}\m{\,x}\m{
\wedge}\m{\forall}\m{y}\m{(}\m{\mbox{\rm Lim}}\m{\,y}\m{\rightarrow}\m{x}\m{
\in}\m{y}\m{)}\m{)}\m{\}}
\endm
\begin{mmraw}%
|- om = \{ x | ( Ord x \TAND A. y ( Lim y -> x e. y ) ) \} \$.
\end{mmraw}

\noindent Define the Cartesian product (also called the
cross product)\index{Cartesian product}\index{cross product}
of two classes.  Definition 9.11 of Quine, p.~64.

\vskip 0.5ex
\setbox\startprefix=\hbox{\tt \ \ df-xp\ \$a\ }
\setbox\contprefix=\hbox{\tt \ \ \ \ \ \ \ \ \ \ \ }
\startm
\m{\vdash}\m{(}\m{A}\m{\times}\m{B}\m{)}\m{=}\m{\{}\m{\langle}\m{x}\m{,}\m{y}
\m{\rangle}\m{|}\m{(}\m{x}\m{\in}\m{A}\m{\wedge}\m{y}\m{\in}\m{B}\m{)}\m{\}}
\endm
\begin{mmraw}%
|- ( A X. B ) = \{ <. x , y >. | ( x e. A \TAND y e. B) \} \$.
\end{mmraw}

\noindent Define a relation\index{relation}.  Definition 6.4(1) of Takeuti and
Zaring, p.~23.

\vskip 0.5ex
\setbox\startprefix=\hbox{\tt \ \ df-rel\ \$a\ }
\setbox\contprefix=\hbox{\tt \ \ \ \ \ \ \ \ \ \ \ \ }
\startm
\m{\vdash}\m{(}\m{\mbox{\rm Rel}}\m{\,A}\m{\leftrightarrow}\m{A}\m{\subseteq}
\m{(}\m{{\rm V}}\m{\times}\m{{\rm V}}\m{)}\m{)}
\endm
\begin{mmraw}%
|- ( Rel A <-> A C\_ ( {\char`\_}V X. {\char`\_}V ) ) \$.
\end{mmraw}

\noindent Define the domain\index{domain} of a class.  Definition 6.5(1) of
Takeuti and Zaring, p.~24.

\vskip 0.5ex
\setbox\startprefix=\hbox{\tt \ \ df-dm\ \$a\ }
\setbox\contprefix=\hbox{\tt \ \ \ \ \ \ \ \ \ \ \ }
\startm
\m{\vdash}\m{\,\mbox{\rm dom}}\m{A}\m{=}\m{\{}\m{x}\m{|}\m{\exists}\m{y}\m{
\langle}\m{x}\m{,}\m{y}\m{\rangle}\m{\in}\m{A}\m{\}}
\endm
\begin{mmraw}%
|- dom A = \{ x | E. y <. x , y >. e. A \} \$.
\end{mmraw}

\noindent Define the range\index{range} of a class.  Definition 6.5(2) of
Takeuti and Zaring, p.~24.

\vskip 0.5ex
\setbox\startprefix=\hbox{\tt \ \ df-rn\ \$a\ }
\setbox\contprefix=\hbox{\tt \ \ \ \ \ \ \ \ \ \ \ }
\startm
\m{\vdash}\m{\,\mbox{\rm ran}}\m{A}\m{=}\m{\{}\m{y}\m{|}\m{\exists}\m{x}\m{
\langle}\m{x}\m{,}\m{y}\m{\rangle}\m{\in}\m{A}\m{\}}
\endm
\begin{mmraw}%
|- ran A = \{ y | E. x <. x , y >. e. A \} \$.
\end{mmraw}

\noindent Define the restriction\index{restriction} of a class.  Definition
6.6(1) of Takeuti and Zaring, p.~24.

\vskip 0.5ex
\setbox\startprefix=\hbox{\tt \ \ df-res\ \$a\ }
\setbox\contprefix=\hbox{\tt \ \ \ \ \ \ \ \ \ \ \ \ }
\startm
\m{\vdash}\m{(}\m{A}\m{\restriction}\m{B}\m{)}\m{=}\m{(}\m{A}\m{\cap}\m{(}\m{B}
\m{\times}\m{{\rm V}}\m{)}\m{)}
\endm
\begin{mmraw}%
|- ( A |` B ) = ( A i\^{}i ( B X. {\char`\_}V ) ) \$.
\end{mmraw}

\noindent Define the image\index{image} of a class.  Definition 6.6(2) of
Takeuti and Zaring, p.~24.

\vskip 0.5ex
\setbox\startprefix=\hbox{\tt \ \ df-ima\ \$a\ }
\setbox\contprefix=\hbox{\tt \ \ \ \ \ \ \ \ \ \ \ \ }
\startm
\m{\vdash}\m{(}\m{A}\m{``}\m{B}\m{)}\m{=}\m{\,\mbox{\rm ran}}\m{\,(}\m{A}\m{
\restriction}\m{B}\m{)}
\endm
\begin{mmraw}%
|- ( A " B ) = ran ( A |` B ) \$.
\end{mmraw}

\noindent Define the composition\index{composition} of two classes.  Definition 6.6(3) of
Takeuti and Zaring, p.~24.

\vskip 0.5ex
\setbox\startprefix=\hbox{\tt \ \ df-co\ \$a\ }
\setbox\contprefix=\hbox{\tt \ \ \ \ \ \ \ \ \ \ \ \ }
\startm
\m{\vdash}\m{(}\m{A}\m{\circ}\m{B}\m{)}\m{=}\m{\{}\m{\langle}\m{x}\m{,}\m{y}\m{
\rangle}\m{|}\m{\exists}\m{z}\m{(}\m{\langle}\m{x}\m{,}\m{z}\m{\rangle}\m{\in}
\m{B}\m{\wedge}\m{\langle}\m{z}\m{,}\m{y}\m{\rangle}\m{\in}\m{A}\m{)}\m{\}}
\endm
\begin{mmraw}%
|- ( A o. B ) = \{ <. x , y >. | E. z ( <. x , z
>. e. B \TAND <. z , y >. e. A ) \} \$.
\end{mmraw}

\noindent Define a function\index{function}.  Definition 6.4(4) of Takeuti and
Zaring, p.~24.

\vskip 0.5ex
\setbox\startprefix=\hbox{\tt \ \ df-fun\ \$a\ }
\setbox\contprefix=\hbox{\tt \ \ \ \ \ \ \ \ \ \ \ \ }
\startm
\m{\vdash}\m{(}\m{\mbox{\rm Fun}}\m{\,A}\m{\leftrightarrow}\m{(}
\m{\mbox{\rm Rel}}\m{\,A}\m{\wedge}
\m{\forall}\m{x}\m{\exists}\m{z}\m{\forall}\m{y}\m{(}
\m{\langle}\m{x}\m{,}\m{y}\m{\rangle}\m{\in}\m{A}\m{\rightarrow}\m{y}\m{=}\m{z}
\m{)}\m{)}\m{)}
\endm
\begin{mmraw}%
|- ( Fun A <-> ( Rel A /\ A. x E. z A. y ( <. x
   , y >. e. A -> y = z ) ) ) \$.
\end{mmraw}

\noindent Define a function with domain.  Definition 6.15(1) of Takeuti and
Zaring, p.~27.

\vskip 0.5ex
\setbox\startprefix=\hbox{\tt \ \ df-fn\ \$a\ }
\setbox\contprefix=\hbox{\tt \ \ \ \ \ \ \ \ \ \ \ }
\startm
\m{\vdash}\m{(}\m{A}\m{\,\mbox{\rm Fn}}\m{\,B}\m{\leftrightarrow}\m{(}
\m{\mbox{\rm Fun}}\m{\,A}\m{\wedge}\m{\mbox{\rm dom}}\m{\,A}\m{=}\m{B}\m{)}
\m{)}
\endm
\begin{mmraw}%
|- ( A Fn B <-> ( Fun A \TAND dom A = B ) ) \$.
\end{mmraw}

\noindent Define a function with domain and co-domain.  Definition 6.15(3)
of Takeuti and Zaring, p.~27.

\vskip 0.5ex
\setbox\startprefix=\hbox{\tt \ \ df-f\ \$a\ }
\setbox\contprefix=\hbox{\tt \ \ \ \ \ \ \ \ \ \ }
\startm
\m{\vdash}\m{(}\m{F}\m{:}\m{A}\m{\longrightarrow}\m{B}\m{
\leftrightarrow}\m{(}\m{F}\m{\,\mbox{\rm Fn}}\m{\,A}\m{\wedge}\m{
\mbox{\rm ran}}\m{\,F}\m{\subseteq}\m{B}\m{)}\m{)}
\endm
\begin{mmraw}%
|- ( F : A --> B <-> ( F Fn A \TAND ran F C\_ B ) ) \$.
\end{mmraw}

\noindent Define a one-to-one function\index{one-to-one function}.  Compare
Definition 6.15(5) of Takeuti and Zaring, p.~27.

\vskip 0.5ex
\setbox\startprefix=\hbox{\tt \ \ df-f1\ \$a\ }
\setbox\contprefix=\hbox{\tt \ \ \ \ \ \ \ \ \ \ \ }
\startm
\m{\vdash}\m{(}\m{F}\m{:}\m{A}\m{
\raisebox{.5ex}{${\textstyle{\:}_{\mbox{\footnotesize\rm
1\tt -\rm 1}}}\atop{\textstyle{
\longrightarrow}\atop{\textstyle{}^{\mbox{\footnotesize\rm {\ }}}}}$}
}\m{B}
\m{\leftrightarrow}\m{(}\m{F}\m{:}\m{A}\m{\longrightarrow}\m{B}
\m{\wedge}\m{\forall}\m{y}\m{\exists}\m{z}\m{\forall}\m{x}\m{(}\m{\langle}\m{x}
\m{,}\m{y}\m{\rangle}\m{\in}\m{F}\m{\rightarrow}\m{x}\m{=}\m{z}\m{)}\m{)}\m{)}
\endm
\begin{mmraw}%
|- ( F : A -1-1-> B <-> ( F : A --> B \TAND
   A. y E. z A. x ( <. x , y >. e. F -> x = z ) ) ) \$.
\end{mmraw}

\noindent Define an onto function\index{onto function}.  Definition 6.15(4) of Takeuti and
Zaring, p.~27.

\vskip 0.5ex
\setbox\startprefix=\hbox{\tt \ \ df-fo\ \$a\ }
\setbox\contprefix=\hbox{\tt \ \ \ \ \ \ \ \ \ \ \ }
\startm
\m{\vdash}\m{(}\m{F}\m{:}\m{A}\m{
\raisebox{.5ex}{${\textstyle{\:}_{\mbox{\footnotesize\rm
{\ }}}}\atop{\textstyle{
\longrightarrow}\atop{\textstyle{}^{\mbox{\footnotesize\rm onto}}}}$}
}\m{B}
\m{\leftrightarrow}\m{(}\m{F}\m{\,\mbox{\rm Fn}}\m{\,A}\m{\wedge}
\m{\mbox{\rm ran}}\m{\,F}\m{=}\m{B}\m{)}\m{)}
\endm
\begin{mmraw}%
|- ( F : A -onto-> B <-> ( F Fn A /\ ran F = B ) ) \$.
\end{mmraw}

\noindent Define a one-to-one, onto function.  Compare Definition 6.15(6) of
Takeuti and Zaring, p.~27.

\vskip 0.5ex
\setbox\startprefix=\hbox{\tt \ \ df-f1o\ \$a\ }
\setbox\contprefix=\hbox{\tt \ \ \ \ \ \ \ \ \ \ \ \ }
\startm
\m{\vdash}\m{(}\m{F}\m{:}\m{A}
\m{
\raisebox{.5ex}{${\textstyle{\:}_{\mbox{\footnotesize\rm
1\tt -\rm 1}}}\atop{\textstyle{
\longrightarrow}\atop{\textstyle{}^{\mbox{\footnotesize\rm onto}}}}$}
}
\m{B}
\m{\leftrightarrow}\m{(}\m{F}\m{:}\m{A}
\m{
\raisebox{.5ex}{${\textstyle{\:}_{\mbox{\footnotesize\rm
1\tt -\rm 1}}}\atop{\textstyle{
\longrightarrow}\atop{\textstyle{}^{\mbox{\footnotesize\rm {\ }}}}}$}
}
\m{B}\m{\wedge}\m{F}\m{:}\m{A}
\m{
\raisebox{.5ex}{${\textstyle{\:}_{\mbox{\footnotesize\rm
{\ }}}}\atop{\textstyle{
\longrightarrow}\atop{\textstyle{}^{\mbox{\footnotesize\rm onto}}}}$}
}
\m{B}\m{)}\m{)}
\endm
\begin{mmraw}%
|- ( F : A -1-1-onto-> B <-> ( F : A -1-1-> B? \TAND F : A -onto-> B ) ) \$.?
\end{mmraw}

\noindent Define the value of a function\index{function value}.  This
definition applies to any class and evaluates to the empty set when it is not
meaningful. Note that $ F`A$ means the same thing as the more familiar $ F(A)$
notation for a function's value at $A$.  The $ F`A$ notation is common in
formal set theory.\label{df-fv}

\vskip 0.5ex
\setbox\startprefix=\hbox{\tt \ \ df-fv\ \$a\ }
\setbox\contprefix=\hbox{\tt \ \ \ \ \ \ \ \ \ \ \ }
\startm
\m{\vdash}\m{(}\m{F}\m{`}\m{A}\m{)}\m{=}\m{\bigcup}\m{\{}\m{x}\m{|}\m{(}\m{F}%
\m{``}\m{\{}\m{A}\m{\}}\m{)}\m{=}\m{\{}\m{x}\m{\}}\m{\}}
\endm
\begin{mmraw}%
|- ( F ` A ) = U. \{ x | ( F " \{ A \} ) = \{ x \} \} \$.
\end{mmraw}

\noindent Define the result of an operation.\index{operation}  Here, $F$ is
     an operation on two
     values (such as $+$ for real numbers).   This is defined for proper
     classes $A$ and $B$ even though not meaningful in that case.  However,
     the definition can be meaningful when $F$ is a proper class.\label{dfopr}

\vskip 0.5ex
\setbox\startprefix=\hbox{\tt \ \ df-opr\ \$a\ }
\setbox\contprefix=\hbox{\tt \ \ \ \ \ \ \ \ \ \ \ \ }
\startm
\m{\vdash}\m{(}\m{A}\m{\,F}\m{\,B}\m{)}\m{=}\m{(}\m{F}\m{`}\m{\langle}\m{A}%
\m{,}\m{B}\m{\rangle}\m{)}
\endm
\begin{mmraw}%
|- ( A F B ) = ( F ` <. A , B >. ) \$.
\end{mmraw}

\section{Tricks of the Trade}\label{tricks}

In the \texttt{set.mm}\index{set theory database (\texttt{set.mm})} database our goal
was usually to conform to modern notation.  However in some cases the
relationship to standard textbook language may be obscured by several
unconventional devices we used to simplify the development and to take
advantage of the Metamath language.  In this section we will describe some
common conventions used in \texttt{set.mm}.

\begin{itemize}
\item
The turnstile\index{turnstile ({$\,\vdash$})} symbol, $\vdash$, meaning ``it
is provable that,'' is the first token of all assertions and hypotheses that
aren't syntax constructions.  This is a standard convention in logic.  (We
mentioned this earlier, but this symbol is bothersome to some people without a
logic background.  It has no deeper meaning but just provides us with a way to
distinguish syntax constructions from ordinary mathematical statements.)

\item
A hypothesis of the form

\vskip 1ex
\setbox\startprefix=\hbox{\tt \ \ \ \ \ \ \ \ \ \$e\ }
\setbox\contprefix=\hbox{\tt \ \ \ \ \ \ \ \ \ \ }
\startm
\m{\vdash}\m{(}\m{\varphi}\m{\rightarrow}\m{\forall}\m{x}\m{\varphi}\m{)}
\endm
\vskip 1ex

should be read ``assume variable $x$ is (effectively) not free in wff
$\varphi$.''\index{effectively not free}
Literally, this says ``assume it is provable that $\varphi \rightarrow \forall
x\, \varphi$.''  This device lets us avoid the complexities associated with
the standard treatment of free and bound variables.
%% Uncomment this when uncommenting section {formalspec} below
The footnote on p.~\pageref{effectivelybound} discusses this further.

\item
A statement of one of the forms

\vskip 1ex
\setbox\startprefix=\hbox{\tt \ \ \ \ \ \ \ \ \ \$a\ }
\setbox\contprefix=\hbox{\tt \ \ \ \ \ \ \ \ \ \ }
\startm
\m{\vdash}\m{(}\m{\lnot}\m{\forall}\m{x}\m{\,x}\m{=}\m{y}\m{\rightarrow}
\m{\ldots}\m{)}
\endm
\setbox\startprefix=\hbox{\tt \ \ \ \ \ \ \ \ \ \$p\ }
\setbox\contprefix=\hbox{\tt \ \ \ \ \ \ \ \ \ \ }
\startm
\m{\vdash}\m{(}\m{\lnot}\m{\forall}\m{x}\m{\,x}\m{=}\m{y}\m{\rightarrow}
\m{\ldots}\m{)}
\endm
\vskip 1ex

should be read ``if $x$ and $y$ are distinct variables, then...''  This
antecedent provides us with a technical device to avoid the need for the
\texttt{\$d} statement early in our development of predicate calculus,
permitting symbol manipulations to be as conceptually simple as those in
propositional calculus.  However, the \texttt{\$d} statement eventually
becomes a requirement, and after that this device is rarely used.

\item
The statement

\vskip 1ex
\setbox\startprefix=\hbox{\tt \ \ \ \ \ \ \ \ \ \$d\ }
\setbox\contprefix=\hbox{\tt \ \ \ \ \ \ \ \ \ \ }
\startm
\m{x}\m{\,y}
\endm
\vskip 1ex

should be read ``assume $x$ and $y$ are distinct variables.''

\item
The statement

\vskip 1ex
\setbox\startprefix=\hbox{\tt \ \ \ \ \ \ \ \ \ \$d\ }
\setbox\contprefix=\hbox{\tt \ \ \ \ \ \ \ \ \ \ }
\startm
\m{x}\m{\,\varphi}
\endm
\vskip 1ex

should be read ``assume $x$ does not occur in $\varphi$.''

\item
The statement

\vskip 1ex
\setbox\startprefix=\hbox{\tt \ \ \ \ \ \ \ \ \ \$d\ }
\setbox\contprefix=\hbox{\tt \ \ \ \ \ \ \ \ \ \ }
\startm
\m{x}\m{\,A}
\endm
\vskip 1ex

should be read ``assume variable $x$ does not occur in class $A$.''

\item
The restriction and hypothesis group

\vskip 1ex
\setbox\startprefix=\hbox{\tt \ \ \ \ \ \ \ \ \ \$d\ }
\setbox\contprefix=\hbox{\tt \ \ \ \ \ \ \ \ \ \ }
\startm
\m{x}\m{\,A}
\endm
\setbox\startprefix=\hbox{\tt \ \ \ \ \ \ \ \ \ \$d\ }
\setbox\contprefix=\hbox{\tt \ \ \ \ \ \ \ \ \ \ }
\startm
\m{x}\m{\,\psi}
\endm
\setbox\startprefix=\hbox{\tt \ \ \ \ \ \ \ \ \ \$e\ }
\setbox\contprefix=\hbox{\tt \ \ \ \ \ \ \ \ \ \ }
\startm
\m{\vdash}\m{(}\m{x}\m{=}\m{A}\m{\rightarrow}\m{(}\m{\varphi}\m{\leftrightarrow}
\m{\psi}\m{)}\m{)}
\endm
\vskip 1ex

is frequently used in place of explicit substitution, meaning ``assume
$\psi$ results from the proper substitution of $A$ for $x$ in
$\varphi$.''  Sometimes ``\texttt{\$e} $\vdash ( \psi \rightarrow
\forall x \, \psi )$'' is used instead of ``\texttt{\$d} $x\, \psi $,''
which requires only that $x$ be effectively not free in $\varphi$ but
not necessarily absent from it.  The use of implicit
substitution\index{substitution!implicit} is partly a matter of personal
style, although it may make proofs somewhat shorter than would be the
case with explicit substitution.

\item
The hypothesis


\vskip 1ex
\setbox\startprefix=\hbox{\tt \ \ \ \ \ \ \ \ \ \$e\ }
\setbox\contprefix=\hbox{\tt \ \ \ \ \ \ \ \ \ \ }
\startm
\m{\vdash}\m{A}\m{\in}\m{{\rm V}}
\endm
\vskip 1ex

should be read ``assume class $A$ is a set (i.e.\ exists).''
This is a convenient convention used by Quine.

\item
The restriction and hypothesis

\vskip 1ex
\setbox\startprefix=\hbox{\tt \ \ \ \ \ \ \ \ \ \$d\ }
\setbox\contprefix=\hbox{\tt \ \ \ \ \ \ \ \ \ \ }
\startm
\m{x}\m{\,y}
\endm
\setbox\startprefix=\hbox{\tt \ \ \ \ \ \ \ \ \ \$e\ }
\setbox\contprefix=\hbox{\tt \ \ \ \ \ \ \ \ \ \ }
\startm
\m{\vdash}\m{(}\m{y}\m{\in}\m{A}\m{\rightarrow}\m{\forall}\m{x}\m{\,y}
\m{\in}\m{A}\m{)}
\endm
\vskip 1ex

should be read ``assume variable $x$ is
(effectively) not free in class $A$.''

\end{itemize}

\section{A Theorem Sampler}\label{sometheorems}

In this section we list some of the more important theorems that are proved in
the \texttt{set.mm} database, and they illustrate the kinds of things that can be
done with Metamath.  While all of these facts are well-known results,
Metamath offers the advantage of easily allowing you to trace their
derivation back to axioms.  Our intent here is not to try to explain the
details or motivation; for this we refer you to the textbooks that are
mentioned in the descriptions.  (The \texttt{set.mm} file has bibliographic
references for the text references.)  Their proofs often embody important
concepts you may wish to explore with the Metamath program (see
Section~\ref{exploring}).  All the symbols that are used here are defined in
Section~\ref{hierarchy}.  For brevity we haven't included the \texttt{\$d}
restrictions or \texttt{\$f} hypotheses for these theorems; when you are
uncertain consult the \texttt{set.mm} database.

We start with \texttt{syl} (principle of the syllogism).
In \textit{Principia Mathematica}
Whitehead and Russell call this ``the principle of the
syllogism... because... the syllogism in Barbara is derived from them''
\cite[quote after Theorem *2.06 p.~101]{PM}.
Some authors call this law a ``hypothetical syllogism.''
At the time of this writing it is the most commonly referenced assertion in the
\texttt{set.mm} database.\footnote{
The Metamath program command \texttt{show usage}
shows the number of references.
}

\vskip 2ex
\noindent Theorem syl (principle of the syllogism)\index{Syllogism}%
\index{\texttt{syl}}\label{syl}.

\vskip 0.5ex
\setbox\startprefix=\hbox{\tt \ \ syl.1\ \$e\ }
\setbox\contprefix=\hbox{\tt \ \ \ \ \ \ \ \ \ \ \ }
\startm
\m{\vdash}\m{(}\m{\varphi}\m{ \rightarrow }\m{\psi}\m{)}
\endm
\setbox\startprefix=\hbox{\tt \ \ syl.2\ \$e\ }
\setbox\contprefix=\hbox{\tt \ \ \ \ \ \ \ \ \ \ \ }
\startm
\m{\vdash}\m{(}\m{\psi}\m{ \rightarrow }\m{\chi}\m{)}
\endm
\setbox\startprefix=\hbox{\tt \ \ syl\ \$p\ }
\setbox\contprefix=\hbox{\tt \ \ \ \ \ \ \ \ \ }
\startm
\m{\vdash}\m{(}\m{\varphi}\m{ \rightarrow }\m{\chi}\m{)}
\endm
\vskip 2ex

The following theorem is not very deep but provides us with a notational device
that is frequently used.  It allows us to use the expression ``$A \in V$'' as
a compact way of saying that class $A$ exists, i.e.\ is a set.

\vskip 2ex
\noindent Two ways to say ``$A$ is a set'':  $A$ is a member of the universe
$V$ if and only if $A$ exists (i.e.\ there exists a set equal to $A$).
Theorem 6.9 of Quine, p. 43.

\vskip 0.5ex
\setbox\startprefix=\hbox{\tt \ \ isset\ \$p\ }
\setbox\contprefix=\hbox{\tt \ \ \ \ \ \ \ \ \ \ \ }
\startm
\m{\vdash}\m{(}\m{A}\m{\in}\m{{\rm V}}\m{\leftrightarrow}\m{\exists}\m{x}\m{\,x}\m{=}
\m{A}\m{)}
\endm
\vskip 1ex

Next we prove the axioms of standard ZF set theory that were missing from our
axiom system.  From our point of view they are theorems since they
can be derived from the other axioms.

\vskip 2ex
\noindent Axiom of Separation\index{Axiom of Separation}
(Aussonderung)\index{Aussonderung} proved from the other axioms of ZF set
theory.  Compare Exercise 4 of Takeuti and Zaring, p.~22.

\vskip 0.5ex
\setbox\startprefix=\hbox{\tt \ \ inex1.1\ \$e\ }
\setbox\contprefix=\hbox{\tt \ \ \ \ \ \ \ \ \ \ \ \ \ \ \ }
\startm
\m{\vdash}\m{A}\m{\in}\m{{\rm V}}
\endm
\setbox\startprefix=\hbox{\tt \ \ inex\ \$p\ }
\setbox\contprefix=\hbox{\tt \ \ \ \ \ \ \ \ \ \ \ \ \ }
\startm
\m{\vdash}\m{(}\m{A}\m{\cap}\m{B}\m{)}\m{\in}\m{{\rm V}}
\endm
\vskip 1ex

\noindent Axiom of the Null Set\index{Axiom of the Null Set} proved from the
other axioms of ZF set theory. Corollary 5.16 of Takeuti and Zaring, p.~20.

\vskip 0.5ex
\setbox\startprefix=\hbox{\tt \ \ 0ex\ \$p\ }
\setbox\contprefix=\hbox{\tt \ \ \ \ \ \ \ \ \ \ \ \ }
\startm
\m{\vdash}\m{\varnothing}\m{\in}\m{{\rm V}}
\endm
\vskip 1ex

\noindent The Axiom of Pairing\index{Axiom of Pairing} proved from the other
axioms of ZF set theory.  Theorem 7.13 of Quine, p.~51.
\vskip 0.5ex
\setbox\startprefix=\hbox{\tt \ \ prex\ \$p\ }
\setbox\contprefix=\hbox{\tt \ \ \ \ \ \ \ \ \ \ \ \ \ \ }
\startm
\m{\vdash}\m{\{}\m{A}\m{,}\m{B}\m{\}}\m{\in}\m{{\rm V}}
\endm
\vskip 2ex

Next we will list some famous or important theorems that are proved in
the \texttt{set.mm} database.  None of them except \texttt{omex}
require the Axiom of Infinity, as you can verify with the \texttt{show
trace{\char`\_}back} Metamath command.

\vskip 2ex
\noindent The resolution of Russell's paradox\index{Russell's paradox}.  There
exists no set containing the set of all sets which are not members of
themselves.  Proposition 4.14 of Takeuti and Zaring, p.~14.

\vskip 0.5ex
\setbox\startprefix=\hbox{\tt \ \ ru\ \$p\ }
\setbox\contprefix=\hbox{\tt \ \ \ \ \ \ \ \ }
\startm
\m{\vdash}\m{\lnot}\m{\exists}\m{x}\m{\,x}\m{=}\m{\{}\m{y}\m{|}\m{\lnot}\m{y}
\m{\in}\m{y}\m{\}}
\endm
\vskip 1ex

\noindent Cantor's theorem\index{Cantor's theorem}.  No set can be mapped onto
its power set.  Compare Theorem 6B(b) of Enderton, p.~132.

\vskip 0.5ex
\setbox\startprefix=\hbox{\tt \ \ canth.1\ \$e\ }
\setbox\contprefix=\hbox{\tt \ \ \ \ \ \ \ \ \ \ \ \ \ }
\startm
\m{\vdash}\m{A}\m{\in}\m{{\rm V}}
\endm
\setbox\startprefix=\hbox{\tt \ \ canth\ \$p\ }
\setbox\contprefix=\hbox{\tt \ \ \ \ \ \ \ \ \ \ \ }
\startm
\m{\vdash}\m{\lnot}\m{F}\m{:}\m{A}\m{\raisebox{.5ex}{${\textstyle{\:}_{
\mbox{\footnotesize\rm {\ }}}}\atop{\textstyle{\longrightarrow}\atop{
\textstyle{}^{\mbox{\footnotesize\rm onto}}}}$}}\m{{\cal P}}\m{A}
\endm
\vskip 1ex

\noindent The Burali-Forti paradox\index{Burali-Forti paradox}.  No set
contains all ordinal numbers. Enderton, p.~194.  (Burali-Forti was one person,
not two.)

\vskip 0.5ex
\setbox\startprefix=\hbox{\tt \ \ onprc\ \$p\ }
\setbox\contprefix=\hbox{\tt \ \ \ \ \ \ \ \ \ \ \ \ }
\startm
\m{\vdash}\m{\lnot}\m{\mbox{\rm On}}\m{\in}\m{{\rm V}}
\endm
\vskip 1ex

\noindent Peano's postulates\index{Peano's postulates} for arithmetic.
Proposition 7.30 of Takeuti and Zaring, pp.~42--43.  The objects being
described are the members of $\omega$ i.e.\ the natural numbers 0, 1,
2,\ldots.  The successor\index{successor} operation suc means ``plus
one.''  \texttt{peano1} says that 0 (which is defined as the empty set)
is a natural number.  \texttt{peano2} says that if $A$ is a natural
number, so is $A+1$.  \texttt{peano3} says that 0 is not the successor
of any natural number.  \texttt{peano4} says that two natural numbers
are equal if and only if their successors are equal.  \texttt{peano5} is
essentially the same as mathematical induction.

\vskip 1ex
\setbox\startprefix=\hbox{\tt \ \ peano1\ \$p\ }
\setbox\contprefix=\hbox{\tt \ \ \ \ \ \ \ \ \ \ \ \ }
\startm
\m{\vdash}\m{\varnothing}\m{\in}\m{\omega}
\endm
\vskip 1.5ex

\setbox\startprefix=\hbox{\tt \ \ peano2\ \$p\ }
\setbox\contprefix=\hbox{\tt \ \ \ \ \ \ \ \ \ \ \ \ }
\startm
\m{\vdash}\m{(}\m{A}\m{\in}\m{\omega}\m{\rightarrow}\m{{\rm suc}}\m{A}\m{\in}%
\m{\omega}\m{)}
\endm
\vskip 1.5ex

\setbox\startprefix=\hbox{\tt \ \ peano3\ \$p\ }
\setbox\contprefix=\hbox{\tt \ \ \ \ \ \ \ \ \ \ \ \ }
\startm
\m{\vdash}\m{(}\m{A}\m{\in}\m{\omega}\m{\rightarrow}\m{\lnot}\m{{\rm suc}}%
\m{A}\m{=}\m{\varnothing}\m{)}
\endm
\vskip 1.5ex

\setbox\startprefix=\hbox{\tt \ \ peano4\ \$p\ }
\setbox\contprefix=\hbox{\tt \ \ \ \ \ \ \ \ \ \ \ \ }
\startm
\m{\vdash}\m{(}\m{(}\m{A}\m{\in}\m{\omega}\m{\wedge}\m{B}\m{\in}\m{\omega}%
\m{)}\m{\rightarrow}\m{(}\m{{\rm suc}}\m{A}\m{=}\m{{\rm suc}}\m{B}\m{%
\leftrightarrow}\m{A}\m{=}\m{B}\m{)}\m{)}
\endm
\vskip 1.5ex

\setbox\startprefix=\hbox{\tt \ \ peano5\ \$p\ }
\setbox\contprefix=\hbox{\tt \ \ \ \ \ \ \ \ \ \ \ \ }
\startm
\m{\vdash}\m{(}\m{(}\m{\varnothing}\m{\in}\m{A}\m{\wedge}\m{\forall}\m{x}\m{%
\in}\m{\omega}\m{(}\m{x}\m{\in}\m{A}\m{\rightarrow}\m{{\rm suc}}\m{x}\m{\in}%
\m{A}\m{)}\m{)}\m{\rightarrow}\m{\omega}\m{\subseteq}\m{A}\m{)}
\endm
\vskip 1.5ex

\noindent Finite Induction (mathematical induction).\index{finite
induction}\index{mathematical induction} The first hypothesis is the
basis and the second is the induction hypothesis.  Theorem Schema 22 of
Suppes, p.~136.

\vskip 0.5ex
\setbox\startprefix=\hbox{\tt \ \ findes.1\ \$e\ }
\setbox\contprefix=\hbox{\tt \ \ \ \ \ \ \ \ \ \ \ \ \ \ }
\startm
\m{\vdash}\m{[}\m{\varnothing}\m{/}\m{x}\m{]}\m{\varphi}
\endm
\setbox\startprefix=\hbox{\tt \ \ findes.2\ \$e\ }
\setbox\contprefix=\hbox{\tt \ \ \ \ \ \ \ \ \ \ \ \ \ \ }
\startm
\m{\vdash}\m{(}\m{x}\m{\in}\m{\omega}\m{\rightarrow}\m{(}\m{\varphi}\m{%
\rightarrow}\m{[}\m{{\rm suc}}\m{x}\m{/}\m{x}\m{]}\m{\varphi}\m{)}\m{)}
\endm
\setbox\startprefix=\hbox{\tt \ \ findes\ \$p\ }
\setbox\contprefix=\hbox{\tt \ \ \ \ \ \ \ \ \ \ \ \ }
\startm
\m{\vdash}\m{(}\m{x}\m{\in}\m{\omega}\m{\rightarrow}\m{\varphi}\m{)}
\endm
\vskip 1ex

\noindent Transfinite Induction with explicit substitution.  The first
hypothesis is the basis, the second is the induction hypothesis for
successors, and the third is the induction hypothesis for limit
ordinals.  Theorem Schema 4 of Suppes, p. 197.

\vskip 0.5ex
\setbox\startprefix=\hbox{\tt \ \ tfindes.1\ \$e\ }
\setbox\contprefix=\hbox{\tt \ \ \ \ \ \ \ \ \ \ \ \ \ \ \ }
\startm
\m{\vdash}\m{[}\m{\varnothing}\m{/}\m{x}\m{]}\m{\varphi}
\endm
\setbox\startprefix=\hbox{\tt \ \ tfindes.2\ \$e\ }
\setbox\contprefix=\hbox{\tt \ \ \ \ \ \ \ \ \ \ \ \ \ \ \ }
\startm
\m{\vdash}\m{(}\m{x}\m{\in}\m{{\rm On}}\m{\rightarrow}\m{(}\m{\varphi}\m{%
\rightarrow}\m{[}\m{{\rm suc}}\m{x}\m{/}\m{x}\m{]}\m{\varphi}\m{)}\m{)}
\endm
\setbox\startprefix=\hbox{\tt \ \ tfindes.3\ \$e\ }
\setbox\contprefix=\hbox{\tt \ \ \ \ \ \ \ \ \ \ \ \ \ \ \ }
\startm
\m{\vdash}\m{(}\m{{\rm Lim}}\m{y}\m{\rightarrow}\m{(}\m{\forall}\m{x}\m{\in}%
\m{y}\m{\varphi}\m{\rightarrow}\m{[}\m{y}\m{/}\m{x}\m{]}\m{\varphi}\m{)}\m{)}
\endm
\setbox\startprefix=\hbox{\tt \ \ tfindes\ \$p\ }
\setbox\contprefix=\hbox{\tt \ \ \ \ \ \ \ \ \ \ \ \ \ }
\startm
\m{\vdash}\m{(}\m{x}\m{\in}\m{{\rm On}}\m{\rightarrow}\m{\varphi}\m{)}
\endm
\vskip 1ex

\noindent Principle of Transfinite Recursion.\index{transfinite
recursion} Theorem 7.41 of Takeuti and Zaring, p.~47.  Transfinite
recursion is the key theorem that allows arithmetic of ordinals to be
rigorously defined, and has many other important uses as well.
Hypotheses \texttt{tfr.1} and \texttt{tfr.2} specify a certain (proper)
class $ F$.  The complicated definition of $ F$ is not important in
itself; what is important is that there be such an $ F$ with the
required properties, and we show this by displaying $ F$ explicitly.
\texttt{tfr1} states that $ F$ is a function whose domain is the set of
ordinal numbers.  \texttt{tfr2} states that any value of $ F$ is
completely determined by its previous values and the values of an
auxiliary function, $G$.  \texttt{tfr3} states that $ F$ is unique,
i.e.\ it is the only function that satisfies \texttt{tfr1} and
\texttt{tfr2}.  Note that $ f$ is an individual variable like $x$ and
$y$; it is just a mnemonic to remind us that $A$ is a collection of
functions.

\vskip 0.5ex
\setbox\startprefix=\hbox{\tt \ \ tfr.1\ \$e\ }
\setbox\contprefix=\hbox{\tt \ \ \ \ \ \ \ \ \ \ \ }
\startm
\m{\vdash}\m{A}\m{=}\m{\{}\m{f}\m{|}\m{\exists}\m{x}\m{\in}\m{{\rm On}}\m{(}%
\m{f}\m{{\rm Fn}}\m{x}\m{\wedge}\m{\forall}\m{y}\m{\in}\m{x}\m{(}\m{f}\m{`}%
\m{y}\m{)}\m{=}\m{(}\m{G}\m{`}\m{(}\m{f}\m{\restriction}\m{y}\m{)}\m{)}\m{)}%
\m{\}}
\endm
\setbox\startprefix=\hbox{\tt \ \ tfr.2\ \$e\ }
\setbox\contprefix=\hbox{\tt \ \ \ \ \ \ \ \ \ \ \ }
\startm
\m{\vdash}\m{F}\m{=}\m{\bigcup}\m{A}
\endm
\setbox\startprefix=\hbox{\tt \ \ tfr1\ \$p\ }
\setbox\contprefix=\hbox{\tt \ \ \ \ \ \ \ \ \ \ }
\startm
\m{\vdash}\m{F}\m{{\rm Fn}}\m{{\rm On}}
\endm
\setbox\startprefix=\hbox{\tt \ \ tfr2\ \$p\ }
\setbox\contprefix=\hbox{\tt \ \ \ \ \ \ \ \ \ \ }
\startm
\m{\vdash}\m{(}\m{z}\m{\in}\m{{\rm On}}\m{\rightarrow}\m{(}\m{F}\m{`}\m{z}%
\m{)}\m{=}\m{(}\m{G}\m{`}\m{(}\m{F}\m{\restriction}\m{z}\m{)}\m{)}\m{)}
\endm
\setbox\startprefix=\hbox{\tt \ \ tfr3\ \$p\ }
\setbox\contprefix=\hbox{\tt \ \ \ \ \ \ \ \ \ \ }
\startm
\m{\vdash}\m{(}\m{(}\m{B}\m{{\rm Fn}}\m{{\rm On}}\m{\wedge}\m{\forall}\m{x}\m{%
\in}\m{{\rm On}}\m{(}\m{B}\m{`}\m{x}\m{)}\m{=}\m{(}\m{G}\m{`}\m{(}\m{B}\m{%
\restriction}\m{x}\m{)}\m{)}\m{)}\m{\rightarrow}\m{B}\m{=}\m{F}\m{)}
\endm
\vskip 1ex

\noindent The existence of omega (the class of natural numbers).\index{natural
number}\index{omega ($\omega$)}\index{Axiom of Infinity}  Axiom 7 of Takeuti
and Zaring, p.~43.  (This is the only theorem in this section requiring the
Axiom of Infinity.)

\vskip 0.5ex
\setbox\startprefix=\hbox{\tt \
\ omex\ \$p\ }
\setbox\contprefix=\hbox{\tt \ \ \ \ \ \ \ \ \ \ }
\startm
\m{\vdash}\m{\omega}\m{\in}\m{{\rm V}}
\endm
%\vskip 2ex


\section{Axioms for Real and Complex Numbers}\label{real}
\index{real and complex numbers!axioms for}

This section presents the axioms for real and complex numbers.  Analysis
textbooks implicitly or explicitly use these axioms or their equivalents
as their starting point.  In the database \texttt{set.mm}, we define real
and complex numbers as (rather complicated) specific sets and derive these
axioms as {\em theorems} from the axioms of ZF set theory, using a method
called Dedekind cuts.  We omit the details of this construction, which you can
follow if you wish using the \texttt{set.mm} database in conjunction with the
textbooks referenced therein.

Once we prove those theorems, we then restate these proven theorems as axioms.
This lets us easily identify which axioms are needed for a particular complex number proof, without the obfuscation of the set theory used to derive them.
As a result,
the construction is actually unimportant other
than to show that sets exist that satisfy the axioms, and thus that the axioms
are consistent if ZF set theory is consistent.  When working with real numbers
you can think of them as being the actual sets resulting
from the construction (for definiteness), or you can
think of them as otherwise unspecified sets that happen to satisfy the axioms.
The derivation is not easy, but the fact that it works is quite remarkable
and lends support to the idea that ZFC set theory is all we need to
provide a foundation for essentially all of mathematics.

For the axioms we are given (or postulate) 8 classes:  $\mathbb{C}$ (the
set of complex numbers), $\mathbb{R}$ (the set of real numbers, a subset
of $\mathbb{C}$), $0$ (zero), $1$ (one), $i$ (square root of
$-1$), $+$ (plus), $\cdot$ (times), and
$<_{\mathbb{R}}$ (less than for just the real numbers).
Subtraction and division are defined terms and are not part of the
axioms; for their definitions see \texttt{set.mm}.

Note that the notation $(A+B)$ (and similarly $(A\cdot B)$) specifies a class
called an {\em operation},\index{operation} and is the function value of the
class $+$ at ordered pair $\langle A,B \rangle$.  An operation is defined by
statement \texttt{df-opr} on p.~\pageref{dfopr}.
The notation $A <_{\mathbb{R}} B$ specifies a
wff called a {\em binary relation}\index{binary relation} and means $\langle A,B \rangle \in \,<_{\mathbb{R}}$, as defined by statement \texttt{df-br} on p.~\pageref{dfbr}.

Our set of 8 given classes is assumed to satisfy the following 22 axioms
(in the axioms listed below, $<$ really means $<_{\mathbb{R}}$).

\vskip 2ex

\noindent 1. The real numbers are a subset of the complex numbers.

%\vskip 0.5ex
\setbox\startprefix=\hbox{\tt \ \ ax-resscn\ \$p\ }
\setbox\contprefix=\hbox{\tt \ \ \ \ \ \ \ \ \ \ \ \ \ \ }
\startm
\m{\vdash}\m{\mathbb{R}}\m{\subseteq}\m{\mathbb{C}}
\endm
%\vskip 1ex

\noindent 2. One is a complex number.

%\vskip 0.5ex
\setbox\startprefix=\hbox{\tt \ \ ax-1cn\ \$p\ }
\setbox\contprefix=\hbox{\tt \ \ \ \ \ \ \ \ \ \ \ }
\startm
\m{\vdash}\m{1}\m{\in}\m{\mathbb{C}}
\endm
%\vskip 1ex

\noindent 3. The imaginary number $i$ is a complex number.

%\vskip 0.5ex
\setbox\startprefix=\hbox{\tt \ \ ax-icn\ \$p\ }
\setbox\contprefix=\hbox{\tt \ \ \ \ \ \ \ \ \ \ \ }
\startm
\m{\vdash}\m{i}\m{\in}\m{\mathbb{C}}
\endm
%\vskip 1ex

\noindent 4. Complex numbers are closed under addition.

%\vskip 0.5ex
\setbox\startprefix=\hbox{\tt \ \ ax-addcl\ \$p\ }
\setbox\contprefix=\hbox{\tt \ \ \ \ \ \ \ \ \ \ \ \ \ }
\startm
\m{\vdash}\m{(}\m{(}\m{A}\m{\in}\m{\mathbb{C}}\m{\wedge}\m{B}\m{\in}\m{\mathbb{C}}%
\m{)}\m{\rightarrow}\m{(}\m{A}\m{+}\m{B}\m{)}\m{\in}\m{\mathbb{C}}\m{)}
\endm
%\vskip 1ex

\noindent 5. Real numbers are closed under addition.

%\vskip 0.5ex
\setbox\startprefix=\hbox{\tt \ \ ax-addrcl\ \$p\ }
\setbox\contprefix=\hbox{\tt \ \ \ \ \ \ \ \ \ \ \ \ \ \ }
\startm
\m{\vdash}\m{(}\m{(}\m{A}\m{\in}\m{\mathbb{R}}\m{\wedge}\m{B}\m{\in}\m{\mathbb{R}}%
\m{)}\m{\rightarrow}\m{(}\m{A}\m{+}\m{B}\m{)}\m{\in}\m{\mathbb{R}}\m{)}
\endm
%\vskip 1ex

\noindent 6. Complex numbers are closed under multiplication.

%\vskip 0.5ex
\setbox\startprefix=\hbox{\tt \ \ ax-mulcl\ \$p\ }
\setbox\contprefix=\hbox{\tt \ \ \ \ \ \ \ \ \ \ \ \ \ }
\startm
\m{\vdash}\m{(}\m{(}\m{A}\m{\in}\m{\mathbb{C}}\m{\wedge}\m{B}\m{\in}\m{\mathbb{C}}%
\m{)}\m{\rightarrow}\m{(}\m{A}\m{\cdot}\m{B}\m{)}\m{\in}\m{\mathbb{C}}\m{)}
\endm
%\vskip 1ex

\noindent 7. Real numbers are closed under multiplication.

%\vskip 0.5ex
\setbox\startprefix=\hbox{\tt \ \ ax-mulrcl\ \$p\ }
\setbox\contprefix=\hbox{\tt \ \ \ \ \ \ \ \ \ \ \ \ \ \ }
\startm
\m{\vdash}\m{(}\m{(}\m{A}\m{\in}\m{\mathbb{R}}\m{\wedge}\m{B}\m{\in}\m{\mathbb{R}}%
\m{)}\m{\rightarrow}\m{(}\m{A}\m{\cdot}\m{B}\m{)}\m{\in}\m{\mathbb{R}}\m{)}
\endm
%\vskip 1ex

\noindent 8. Multiplication of complex numbers is commutative.

%\vskip 0.5ex
\setbox\startprefix=\hbox{\tt \ \ ax-mulcom\ \$p\ }
\setbox\contprefix=\hbox{\tt \ \ \ \ \ \ \ \ \ \ \ \ \ \ }
\startm
\m{\vdash}\m{(}\m{(}\m{A}\m{\in}\m{\mathbb{C}}\m{\wedge}\m{B}\m{\in}\m{\mathbb{C}}%
\m{)}\m{\rightarrow}\m{(}\m{A}\m{\cdot}\m{B}\m{)}\m{=}\m{(}\m{B}\m{\cdot}\m{A}%
\m{)}\m{)}
\endm
%\vskip 1ex

\noindent 9. Addition of complex numbers is associative.

%\vskip 0.5ex
\setbox\startprefix=\hbox{\tt \ \ ax-addass\ \$p\ }
\setbox\contprefix=\hbox{\tt \ \ \ \ \ \ \ \ \ \ \ \ \ \ }
\startm
\m{\vdash}\m{(}\m{(}\m{A}\m{\in}\m{\mathbb{C}}\m{\wedge}\m{B}\m{\in}\m{\mathbb{C}}%
\m{\wedge}\m{C}\m{\in}\m{\mathbb{C}}\m{)}\m{\rightarrow}\m{(}\m{(}\m{A}\m{+}%
\m{B}\m{)}\m{+}\m{C}\m{)}\m{=}\m{(}\m{A}\m{+}\m{(}\m{B}\m{+}\m{C}\m{)}\m{)}%
\m{)}
\endm
%\vskip 1ex

\noindent 10. Multiplication of complex numbers is associative.

%\vskip 0.5ex
\setbox\startprefix=\hbox{\tt \ \ ax-mulass\ \$p\ }
\setbox\contprefix=\hbox{\tt \ \ \ \ \ \ \ \ \ \ \ \ \ \ }
\startm
\m{\vdash}\m{(}\m{(}\m{A}\m{\in}\m{\mathbb{C}}\m{\wedge}\m{B}\m{\in}\m{\mathbb{C}}%
\m{\wedge}\m{C}\m{\in}\m{\mathbb{C}}\m{)}\m{\rightarrow}\m{(}\m{(}\m{A}\m{\cdot}%
\m{B}\m{)}\m{\cdot}\m{C}\m{)}\m{=}\m{(}\m{A}\m{\cdot}\m{(}\m{B}\m{\cdot}\m{C}%
\m{)}\m{)}\m{)}
\endm
%\vskip 1ex

\noindent 11. Multiplication distributes over addition for complex numbers.

%\vskip 0.5ex
\setbox\startprefix=\hbox{\tt \ \ ax-distr\ \$p\ }
\setbox\contprefix=\hbox{\tt \ \ \ \ \ \ \ \ \ \ \ \ \ }
\startm
\m{\vdash}\m{(}\m{(}\m{A}\m{\in}\m{\mathbb{C}}\m{\wedge}\m{B}\m{\in}\m{\mathbb{C}}%
\m{\wedge}\m{C}\m{\in}\m{\mathbb{C}}\m{)}\m{\rightarrow}\m{(}\m{A}\m{\cdot}\m{(}%
\m{B}\m{+}\m{C}\m{)}\m{)}\m{=}\m{(}\m{(}\m{A}\m{\cdot}\m{B}\m{)}\m{+}\m{(}%
\m{A}\m{\cdot}\m{C}\m{)}\m{)}\m{)}
\endm
%\vskip 1ex

\noindent 12. The square of $i$ equals $-1$ (expressed as $i$-squared plus 1 is
0).

%\vskip 0.5ex
\setbox\startprefix=\hbox{\tt \ \ ax-i2m1\ \$p\ }
\setbox\contprefix=\hbox{\tt \ \ \ \ \ \ \ \ \ \ \ \ }
\startm
\m{\vdash}\m{(}\m{(}\m{i}\m{\cdot}\m{i}\m{)}\m{+}\m{1}\m{)}\m{=}\m{0}
\endm
%\vskip 1ex

\noindent 13. One and zero are distinct.

%\vskip 0.5ex
\setbox\startprefix=\hbox{\tt \ \ ax-1ne0\ \$p\ }
\setbox\contprefix=\hbox{\tt \ \ \ \ \ \ \ \ \ \ \ \ }
\startm
\m{\vdash}\m{1}\m{\ne}\m{0}
\endm
%\vskip 1ex

\noindent 14. One is an identity element for real multiplication.

%\vskip 0.5ex
\setbox\startprefix=\hbox{\tt \ \ ax-1rid\ \$p\ }
\setbox\contprefix=\hbox{\tt \ \ \ \ \ \ \ \ \ \ \ }
\startm
\m{\vdash}\m{(}\m{A}\m{\in}\m{\mathbb{R}}\m{\rightarrow}\m{(}\m{A}\m{\cdot}\m{1}%
\m{)}\m{=}\m{A}\m{)}
\endm
%\vskip 1ex

\noindent 15. Every real number has a negative.

%\vskip 0.5ex
\setbox\startprefix=\hbox{\tt \ \ ax-rnegex\ \$p\ }
\setbox\contprefix=\hbox{\tt \ \ \ \ \ \ \ \ \ \ \ \ \ \ }
\startm
\m{\vdash}\m{(}\m{A}\m{\in}\m{\mathbb{R}}\m{\rightarrow}\m{\exists}\m{x}\m{\in}%
\m{\mathbb{R}}\m{(}\m{A}\m{+}\m{x}\m{)}\m{=}\m{0}\m{)}
\endm
%\vskip 1ex

\noindent 16. Every nonzero real number has a reciprocal.

%\vskip 0.5ex
\setbox\startprefix=\hbox{\tt \ \ ax-rrecex\ \$p\ }
\setbox\contprefix=\hbox{\tt \ \ \ \ \ \ \ \ \ \ \ \ \ \ }
\startm
\m{\vdash}\m{(}\m{A}\m{\in}\m{\mathbb{R}}\m{\rightarrow}\m{(}\m{A}\m{\ne}\m{0}%
\m{\rightarrow}\m{\exists}\m{x}\m{\in}\m{\mathbb{R}}\m{(}\m{A}\m{\cdot}%
\m{x}\m{)}\m{=}\m{1}\m{)}\m{)}
\endm
%\vskip 1ex

\noindent 17. A complex number can be expressed in terms of two reals.

%\vskip 0.5ex
\setbox\startprefix=\hbox{\tt \ \ ax-cnre\ \$p\ }
\setbox\contprefix=\hbox{\tt \ \ \ \ \ \ \ \ \ \ \ \ }
\startm
\m{\vdash}\m{(}\m{A}\m{\in}\m{\mathbb{C}}\m{\rightarrow}\m{\exists}\m{x}\m{\in}%
\m{\mathbb{R}}\m{\exists}\m{y}\m{\in}\m{\mathbb{R}}\m{A}\m{=}\m{(}\m{x}\m{+}\m{(}%
\m{y}\m{\cdot}\m{i}\m{)}\m{)}\m{)}
\endm
%\vskip 1ex

\noindent 18. Ordering on reals satisfies strict trichotomy.

%\vskip 0.5ex
\setbox\startprefix=\hbox{\tt \ \ ax-pre-lttri\ \$p\ }
\setbox\contprefix=\hbox{\tt \ \ \ \ \ \ \ \ \ \ \ \ \ }
\startm
\m{\vdash}\m{(}\m{(}\m{A}\m{\in}\m{\mathbb{R}}\m{\wedge}\m{B}\m{\in}\m{\mathbb{R}}%
\m{)}\m{\rightarrow}\m{(}\m{A}\m{<}\m{B}\m{\leftrightarrow}\m{\lnot}\m{(}\m{A}%
\m{=}\m{B}\m{\vee}\m{B}\m{<}\m{A}\m{)}\m{)}\m{)}
\endm
%\vskip 1ex

\noindent 19. Ordering on reals is transitive.

%\vskip 0.5ex
\setbox\startprefix=\hbox{\tt \ \ ax-pre-lttrn\ \$p\ }
\setbox\contprefix=\hbox{\tt \ \ \ \ \ \ \ \ \ \ \ \ \ }
\startm
\m{\vdash}\m{(}\m{(}\m{A}\m{\in}\m{\mathbb{R}}\m{\wedge}\m{B}\m{\in}\m{\mathbb{R}}%
\m{\wedge}\m{C}\m{\in}\m{\mathbb{R}}\m{)}\m{\rightarrow}\m{(}\m{(}\m{A}\m{<}%
\m{B}\m{\wedge}\m{B}\m{<}\m{C}\m{)}\m{\rightarrow}\m{A}\m{<}\m{C}\m{)}\m{)}
\endm
%\vskip 1ex

\noindent 20. Ordering on reals is preserved after addition to both sides.

%\vskip 0.5ex
\setbox\startprefix=\hbox{\tt \ \ ax-pre-ltadd\ \$p\ }
\setbox\contprefix=\hbox{\tt \ \ \ \ \ \ \ \ \ \ \ \ \ }
\startm
\m{\vdash}\m{(}\m{(}\m{A}\m{\in}\m{\mathbb{R}}\m{\wedge}\m{B}\m{\in}\m{\mathbb{R}}%
\m{\wedge}\m{C}\m{\in}\m{\mathbb{R}}\m{)}\m{\rightarrow}\m{(}\m{A}\m{<}\m{B}\m{%
\rightarrow}\m{(}\m{C}\m{+}\m{A}\m{)}\m{<}\m{(}\m{C}\m{+}\m{B}\m{)}\m{)}\m{)}
\endm
%\vskip 1ex

\noindent 21. The product of two positive reals is positive.

%\vskip 0.5ex
\setbox\startprefix=\hbox{\tt \ \ ax-pre-mulgt0\ \$p\ }
\setbox\contprefix=\hbox{\tt \ \ \ \ \ \ \ \ \ \ \ \ \ \ }
\startm
\m{\vdash}\m{(}\m{(}\m{A}\m{\in}\m{\mathbb{R}}\m{\wedge}\m{B}\m{\in}\m{\mathbb{R}}%
\m{)}\m{\rightarrow}\m{(}\m{(}\m{0}\m{<}\m{A}\m{\wedge}\m{0}%
\m{<}\m{B}\m{)}\m{\rightarrow}\m{0}\m{<}\m{(}\m{A}\m{\cdot}\m{B}\m{)}%
\m{)}\m{)}
\endm
%\vskip 1ex

\noindent 22. A non-empty, bounded-above set of reals has a supremum.

%\vskip 0.5ex
\setbox\startprefix=\hbox{\tt \ \ ax-pre-sup\ \$p\ }
\setbox\contprefix=\hbox{\tt \ \ \ \ \ \ \ \ \ \ \ }
\startm
\m{\vdash}\m{(}\m{(}\m{A}\m{\subseteq}\m{\mathbb{R}}\m{\wedge}\m{A}\m{\ne}\m{%
\varnothing}\m{\wedge}\m{\exists}\m{x}\m{\in}\m{\mathbb{R}}\m{\forall}\m{y}\m{%
\in}\m{A}\m{\,y}\m{<}\m{x}\m{)}\m{\rightarrow}\m{\exists}\m{x}\m{\in}\m{%
\mathbb{R}}\m{(}\m{\forall}\m{y}\m{\in}\m{A}\m{\lnot}\m{x}\m{<}\m{y}\m{\wedge}\m{%
\forall}\m{y}\m{\in}\m{\mathbb{R}}\m{(}\m{y}\m{<}\m{x}\m{\rightarrow}\m{\exists}%
\m{z}\m{\in}\m{A}\m{\,y}\m{<}\m{z}\m{)}\m{)}\m{)}
\endm

% NOTE: The \m{...} expressions above could be represented as
% $ \vdash ( ( A \subseteq \mathbb{R} \wedge A \ne \varnothing \wedge \exists x \in \mathbb{R} \forall y \in A \,y < x ) \rightarrow \exists x \in \mathbb{R} ( \forall y \in A \lnot x < y \wedge \forall y \in \mathbb{R} ( y < x \rightarrow \exists z \in A \,y < z ) ) ) $

\vskip 2ex

This completes the set of axioms for real and complex numbers.  You may
wish to look at how subtraction, division, and decimal numbers
are defined in \texttt{set.mm}, and for fun look at the proof of $2+
2 = 4$ (theorem \texttt{2p2e4} in \texttt{set.mm})
as discussed in section \ref{2p2e4}.

In \texttt{set.mm} we define the non-negative integers $\mathbb{N}$, the integers
$\mathbb{Z}$, and the rationals $\mathbb{Q}$ as subsets of $\mathbb{R}$.  This leads
to the nice inclusion $\mathbb{N} \subseteq \mathbb{Z} \subseteq \mathbb{Q} \subseteq
\mathbb{R} \subseteq \mathbb{C}$, giving us a uniform framework in which, for
example, a property such as commutativity of complex number addition
automatically applies to integers.  The natural numbers $\mathbb{N}$
are different from the set $\omega$ we defined earlier, but both satisfy
Peano's postulates.

\subsubsection{Complex Number Axioms in Analysis Texts}

Most analysis texts construct complex numbers as ordered pairs of reals,
leading to construction-dependent properties that satisfy these axioms
but are not stated in their pure form.  (This is also done in
\texttt{set.mm} but our axioms are extracted from that construction.)
Other texts will simply state that $\mathbb{R}$ is a ``complete ordered
subfield of $\mathbb{C}$,'' leading to redundant axioms when this phrase
is completely expanded out.  In fact I have not seen a text with the
axioms in the explicit form above.
None of these axioms is unique individually, but this carefully worked out
collection of axioms is the result of years of work
by the Metamath community.

\subsubsection{Eliminating Unnecessary Complex Number Axioms}

We once had more axioms for real and complex numbers, but over years of time
we (the Metamath community)
have found ways to eliminate them (by proving them from other axioms)
or weaken them (by making weaker claims without reducing what
can be proved).
In particular, here are statements that used to be complex number
axioms but have since been formally proven (with Metamath) to be redundant:

\begin{itemize}
\item
  $\mathbb{C} \in V$.
  At one time this was listed as a ``complex number axiom''
  However, this is not properly speaking a complex number axiom,
  and in any case its proof uses axioms of set theory.
  Proven redundant by Mario Carneiro\index{Carneiro, Mario} on
  17-Nov-2014 (see \texttt{axcnex}).
\item
  $((A \in \mathbb{C} \land B \in \mathbb{C}$) $\rightarrow$
  $(A + B) = (B + A))$.
  Proved redundant by Eric Schmidt\index{Schmidt, Eric} on 19-Jun-2012,
  and formalized by Scott Fenton\index{Fenton, Scott} on 3-Jan-2013
  (see \texttt{addcom}).
\item
  $(A \in \mathbb{C} \rightarrow (A + 0) = A)$.
  Proved redundant by Eric Schmidt on 19-Jun-2012,
  and formalized by Scott Fenton on 3-Jan-2013
  (see \texttt{addid1}).
\item
  $(A \in \mathbb{C} \rightarrow \exists x \in \mathbb{C} (A + x) = 0)$.
  Proved redundant by Eric Schmidt and formalized on 21-May-2007
  (see \texttt{cnegex}).
\item
  $((A \in \mathbb{C} \land A \ne 0) \rightarrow \exists x \in \mathbb{C} (A \cdot x) = 1)$.
  Proved redundant by Eric Schmidt and formalized on 22-May-2007
  (see \texttt{recex}).
\item
  $0 \in \mathbb{R}$.
  Proved redundant by Eric Schmidt on 19-Feb-2005 and formalized 21-May-2007
  (see \texttt{0re}).
\end{itemize}

We could eliminate 0 as an axiomatic object by defining it as
$( ( i \cdot i ) + 1 )$
and replacing it with this expression throughout the axioms. If this
is done, axiom ax-i2m1 becomes redundant. However, the remaining axioms
would become longer and less intuitive.

Eric Schmidt's paper analyzing this axiom system \cite{Schmidt}
presented a proof that these remaining axioms,
with the possible exception of ax-mulcom, are independent of the others.
It is currently an open question if ax-mulcom is independent of the others.

\section{Two Plus Two Equals Four}\label{2p2e4}

Here is a proof that $2 + 2 = 4$, as proven in the theorem \texttt{2p2e4}
in the database \texttt{set.mm}.
This is a useful demonstration of what a Metamath proof can look like.
This proof may have more steps than you're used to, but each step is rigorously
proven all the way back to the axioms of logic and set theory.
This display was originally generated by the Metamath program
as an {\sc HTML} file.

In the table showing the proof ``Step'' is the sequential step number,
while its associated ``Expression'' is an expression that we have proved.
``Ref'' is the name of a theorem or axiom that justifies that expression,
and ``Hyp'' refers to previous steps (if any) that the theorem or axiom
needs so that we can use it.  Expressions are indented further than
the expressions that depend on them to show their interdependencies.

\begin{table}[!htbp]
\caption{Two plus two equals four}
\begin{tabular}{lllll}
\textbf{Step} & \textbf{Hyp} & \textbf{Ref} & \textbf{Expression} & \\
1  &       & df-2    & $ \; \; \vdash 2 = 1 + 1$  & \\
2  & 1     & oveq2i  & $ \; \vdash (2 + 2) = (2 + (1 + 1))$ & \\
3  &       & df-4    & $ \; \; \vdash 4 = (3 + 1)$ & \\
4  &       & df-3    & $ \; \; \; \vdash 3 = (2 + 1)$ & \\
5  & 4     & oveq1i  & $ \; \; \vdash (3 + 1) = ((2 + 1) + 1)$ & \\
6  &       & 2cn     & $ \; \; \; \vdash 2 \in \mathbb{C}$ & \\
7  &       & ax-1cn  & $ \; \; \; \vdash 1 \in \mathbb{C}$ & \\
8  & 6,7,7 & addassi & $ \; \; \vdash ((2 + 1) + 1) = (2 + (1 + 1))$ & \\
9  & 3,5,8 & 3eqtri  & $ \; \vdash 4 = (2 + (1 + 1))$ & \\
10 & 2,9   & eqtr4i  & $ \vdash (2 + 2) = 4$ & \\
\end{tabular}
\end{table}

Step 1 says that we can assert that $2 = 1 + 1$ because it is
justified by \texttt{df-2}.
What is \texttt{df-2}?
It is simply the definition of $2$, which in our system is defined as being
equal to $1 + 1$.  This shows how we can use definitions in proofs.

Look at Step 2 of the proof. In the Ref column, we see that it references
a previously proved theorem, \texttt{oveq2i}.
It turns out that
theorem \texttt{oveq2i} requires a
hypothesis, and in the Hyp column of Step 2 we indicate that Step 1 will
satisfy (match) this hypothesis.
If we looked at \texttt{oveq2i}
we would find that it proves that given some hypothesis
$A = B$, we can prove that $( C F A ) = ( C F B )$.
If we use \texttt{oveq2i} and apply step 1's result as the hypothesis,
that will mean that $A = 2$ and $B = ( 1 + 1 )$ within this use of
\texttt{oveq2i}.
We are free to select any value of $C$ and $F$ (subject to syntax constraints),
so we are free to select $C = 2$ and $F = +$,
producing our desired result,
$ (2 + 2) = (2 + (1 + 1))$.

Step 2 is an example of substitution.
In the end, every step in every proof uses only this one substitution rule.
All the rules of logic, and all the axioms, are expressed so that
they can be used via this one substitution rule.
So once you master substitution, you can master every Metamath proof,
no exceptions.

Each step is clear and can be immediately checked.
In the {\sc HTML} display you can even click on each reference to see why it is
justified, making it easy to see why the proof works.

\section{Deduction}\label{deduction}

Strictly speaking,
a deduction (also called an inference) is a kind of statement that needs
some hypotheses to be true in order for its conclusion to be true.
A theorem, on the other hand, has no hypotheses.
Informally we often call both of them theorems, but in this section we
will stick to the strict definitions.

It sometimes happens that we have proved a deduction of the form
$\varphi \Rightarrow \psi$\index{$\Rightarrow$}
(given hypothesis $\varphi$ we can prove $\psi$)
and we want to then prove a theorem of the form
$\varphi \rightarrow \psi$.

Converting a deduction (which uses a hypothesis) into a theorem
(which does not) is not as simple as you might think.
The deduction says, ``if we can prove $\varphi$ then we can prove $\psi$,''
which is in some sense weaker than saying
``$\varphi$ implies $\psi$.''
There is no axiom of logic that permits us to directly obtain the theorem
given the deduction.\footnote{
The conversion of a deduction to a theorem does not even hold in general
for quantum propositional calculus,
which is a weak subset of classical propositional calculus.
It has been shown that adding the Standard Deduction Theorem (discussed below)
to quantum propositional calculus turns it into classical
propositional calculus!
}

This is in contrast to going the other way.
If we have the theorem ($\varphi \rightarrow \psi$),
it is easy to recover the deduction
($\varphi \Rightarrow \psi$)
using modus ponens\index{modus ponens}
(\texttt{ax-mp}; see section \ref{axmp}).

In the following subsections we first discuss the standard deduction theorem
(the traditional but awkward way to convert deductions into theorems) and
the weak deduction theorem (a limited version of the standard deduction
theorem that is easier to use and was once widely used in
\texttt{set.mm}\index{set theory database (\texttt{set.mm})}).
In section \ref{deductionstyle} we discuss
deduction style, the newer approach we now recommend in most cases.
Deduction style uses ``deduction form,'' a form that
prefixes each hypothesis (other than definitions) and the
conclusion with a universal antecedent (``$\varphi \rightarrow$'').
Deduction style is widely used in \texttt{set.mm},
so it is useful to understand it and \textit{why} it is widely used.
Section \ref{naturaldeduction}
briefly discusses our approach for using natural deduction
within \texttt{set.mm},
as that approach is deeply related to deduction style.
We conclude with a summary of the strengths of
our approach, which we believe are compelling.

\subsection{The Standard Deduction Theorem}\label{standarddeductiontheorem}

It is possible to make use of information
contained in the deduction or its proof to assist us with the proof of
the related theorem.
In traditional logic books, there is a metatheorem called the
Deduction Theorem\index{Deduction Theorem}\index{Standard Deduction Theorem},
discovered independently by Herbrand and Tarski around 1930.
The Deduction Theorem, which we often call the Standard Deduction Theorem,
provides an algorithm for constructing a proof of a theorem from the
proof of its corresponding deduction. See, for example,
\cite[p.~56]{Margaris}\index{Margaris, Angelo}.
To construct a proof for a theorem, the
algorithm looks at each step in the proof of the original deduction and
rewrites the step with several steps wherein the hypothesis is eliminated
and becomes an antecedent.

In ordinary mathematics, no one actually carries out the algorithm,
because (in its most basic form) it involves an exponential explosion of
the number of proof steps as more hypotheses are eliminated. Instead,
the Standard Deduction Theorem is invoked simply to claim that it can
be done in principle, without actually doing it.
What's more, the algorithm is not as simple as it might first appear
when applying it rigorously.
There is a subtle restriction on the Standard Deduction Theorem
that must be taken into account involving the axiom of generalization
when working with predicate calculus (see the literature for more detail).

One of the goals of Metamath is to let you plainly see, with as few
underlying concepts as possible, how mathematics can be derived directly
from the axioms, and not indirectly according to some hidden rules
buried inside a program or understood only by logicians. If we added
the Standard Deduction Theorem to the language and proof verifier,
that would greatly complicate both and largely defeat Metamath's goal
of simplicity. In principle, we could show direct proofs by expanding
out the proof steps generated by the algorithm of the Standard Deduction
Theorem, but that is not feasible in practice because the number of proof
steps quickly becomes huge, even astronomical.
Since the algorithm of the Standard Deduction Theorem is driven by the proof,
we would have to go through that proof
all over again---starting from axioms---in order to obtain the theorem form.
In terms of proof length, there would be no savings over just
proving the theorem directly instead of first proving the deduction form.

\subsection{Weak Deduction Theorem}\label{weakdeductiontheorem}

We have developed
a more efficient method for proving a theorem from a deduction
that can be used instead of the Standard Deduction Theorem
in many (but not all) cases.
We call this more efficient method the
Weak Deduction Theorem\index{Weak Deduction Theorem}.\footnote{
There is also an unrelated ``Weak Deduction Theorem''
in the field of relevance logic, so to avoid confusion we could call
ours the ``Weak Deduction Theorem for Classical Logic.''}
Unlike the Standard Deduction Theorem, the Weak Deduction Theorem produces the
theorem directly from a special substitution instance of the deduction,
using a small, fixed number of steps roughly proportional to the length
of the final theorem.

If you come to a proof referencing the Weak Deduction Theorem
\texttt{dedth} (or one of its variants \texttt{dedthxx}),
here is how to follow the proof without getting into the details:
just click on the theorem referenced in the step
just before the reference to \texttt{dedth} and ignore everything else.
Theorem \texttt{dedth} simply turns a hypothesis into an antecedent
(i.e. the hypothesis followed by $\rightarrow$
is placed in front of the assertion, and the hypothesis
itself is eliminated) given certain conditions.

The Weak Deduction Theorem
eliminates a hypothesis $\varphi$, making it become an antecedent.
It does this by proving an expression
$ \varphi \rightarrow \psi $ given two hypotheses:
(1)
$ ( A = {\rm if} ( \varphi , A , B ) \rightarrow ( \varphi \leftrightarrow \chi ) ) $
and
(2) $\chi$.
Note that it requires that a proof exists for $\varphi$ when the class variable
$A$ is replaced with a specific class $B$. The hypothesis $\chi$
should be assigned to the inference.
You can see the details of the proof of the Weak Deduction Theorem
in theorem \texttt{dedth}.

The Weak Deduction Theorem
is probably easier to understand by studying proofs that make use of it.
For example, let's look at the proof of \texttt{renegcl}, which proves that
$ \vdash ( A \in \mathbb{R} \rightarrow - A \in \mathbb{R} )$:

\needspace{4\baselineskip}
\begin{longtabu} {l l l X}
\textbf{Step} & \textbf{Hyp} & \textbf{Ref} & \textbf{Expression} \\
  1 &  & negeq &
  $\vdash$ $($ $A$ $=$ ${\rm if}$ $($ $A$ $\in$ $\mathbb{R}$ $,$ $A$ $,$ $1$ $)$ $\rightarrow$
  $\textrm{-}$ $A$ $=$ $\textrm{-}$ ${\rm if}$ $($ $A$ $\in$ $\mathbb{R}$
  $,$ $A$ $,$ $1$ $)$ $)$ \\
 2 & 1 & eleq1d &
    $\vdash$ $($ $A$ $=$ ${\rm if}$ $($ $A$ $\in$ $\mathbb{R}$ $,$ $A$ $,$ $1$ $)$ $\rightarrow$ $($
    $\textrm{-}$ $A$ $\in$ $\mathbb{R}$ $\leftrightarrow$
    $\textrm{-}$ ${\rm if}$ $($ $A$ $\in$ $\mathbb{R}$ $,$ $A$ $,$ $1$ $)$ $\in$
    $\mathbb{R}$ $)$ $)$ \\
 3 &  & 1re & $\vdash 1 \in \mathbb{R}$ \\
 4 & 3 & elimel &
   $\vdash {\rm if} ( A \in \mathbb{R} , A , 1 ) \in \mathbb{R}$ \\
 5 & 4 & renegcli &
   $\vdash \textrm{-} {\rm if} ( A \in \mathbb{R} , A , 1 ) \in \mathbb{R}$ \\
 6 & 2,5 & dedth &
   $\vdash ( A \in \mathbb{R} \rightarrow \textrm{-} A \in \mathbb{R}$ ) \\
\end{longtabu}

The somewhat strange-looking steps in \texttt{renegcl} before step 5 are
technical stuff that makes this magic work, and they can be ignored
for a quick overview of the proof. To continue following the ``important''
part of the proof of \texttt{renegcl},
you can look at the reference to \texttt{renegcli} at step 5.

That said, let's briefly look at how
\texttt{renegcl} uses the
Weak Deduction Theorem (\texttt{dedth}) to do its job,
in case you want to do something similar or want understand it more deeply.
Let's work backwards in the proof of \texttt{renegcl}.
Step 6 applies \texttt{dedth} to produce our goal result
$ \vdash ( A \in \mathbb{R} \rightarrow\, - A \in \mathbb{R} )$.
This requires on the one hand the (substituted) deduction
\texttt{renegcli} in step 5.
By itself \texttt{renegcli} proves the deduction
$ \vdash A \in \mathbb{R} \Rightarrow\, \vdash - A \in \mathbb{R}$;
this is the deduction form we are trying to turn into theorem form,
and thus
\texttt{renegcli} has a separate hypothesis that must be fulfilled.
To fulfill the hypothesis of the invocation of
\texttt{renegcli} in step 5, it is eventually
reduced to the already proven theorem $1 \in \mathbb{R}$ in step 3.
Step 4 connects steps 3 and 5; step 4 invokes
\texttt{elimel}, a special case of \texttt{elimhyp} that eliminates
a membership hypothesis for the weak deduction theorem.
On the other hand, the equivalence of the conclusion of
\texttt{renegcl}
$( - A \in \mathbb{R} )$ and the substituted conclusion of
\texttt{renegcli} must be proven, which is done in steps 2 and 1.

The weak deduction theorem has limitations.
In particular, we must be able to prove a special case of the deduction's
hypothesis as a stand-alone theorem.
For example, we used $1 \in \mathbb{R}$ in step 3 of \texttt{renegcl}.

We used to use the weak deduction theorem
extensively within \texttt{set.mm}.
However, we now recommend applying ``deduction style''
instead in most cases, as deduction style is
often an easier and clearer approach.
Therefore, we will now describe deduction style.

\subsection{Deduction Style}\label{deductionstyle}

We now prefer to write assertions in ``deduction form''
instead of writing a proof that would require use of the standard or
weak deduction theorem.
We call this appraoch
``deduction style.''\index{deduction style}

It will be easier to explain this by first defining some terms:

\begin{itemize}
\item \textbf{closed form}\index{closed form}\index{forms!closed}:
A kind of assertion (theorem) with no hypotheses.
Typically its label has no special suffix.
An example is \texttt{unss}, which states:
$\vdash ( ( A \subseteq C \wedge B \subseteq C ) \leftrightarrow ( A \cup B )
\subseteq C )\label{eq:unss}$
\item \textbf{deduction form}\index{deduction form}\index{forms!deduction}:
A kind of assertion with one or more hypotheses
where the conclusion is an implication with
a wff variable as the antecedent (usually $\varphi$), and every hypothesis
(\$e statement)
is either (1) an implication with the same antecedent as the conclusion or
(2) a definition.
A definition
can be for a class variable (this is a class variable followed by ``='')
or a wff variable (this is a wff variable followed by $\leftrightarrow$);
class variable definitions are more common.
In practice, a proof
in deduction form will also contain many steps that are implications
where the antecedent is either that wff variable (normally $\varphi$)
or is
a conjunction (...$\land$...) including that wff variable ($\varphi$).
If an assertion is in deduction form, and other forms are also available,
then we suffix its label with ``d.''
An example is \texttt{unssd}, which states\footnote{
For brevity we show here (and in other places)
a $\&$\index{$\&$} between hypotheses\index{hypotheses}
and a $\Rightarrow$\index{$\Rightarrow$}\index{conclusion}
between the hypotheses and the conclusion.
This notation is technically not part of the Metamath language, but is
instead a convenient abbreviation to show both the hypotheses and conclusion.}:
$\vdash ( \varphi \rightarrow A \subseteq C )\quad\&\quad \vdash ( \varphi
    \rightarrow B \subseteq C )\quad\Rightarrow\quad \vdash ( \varphi
    \rightarrow ( A \cup B ) \subseteq C )\label{eq:unssd}$
\item \textbf{inference form}\index{inference form}\index{forms!inference}:
A kind of assertion with one or more hypotheses that is not in deduction form
(e.g., there is no common antecedent).
If an assertion is in inference form, and other forms are also available,
then we suffix its label with ``i.''
An example is \texttt{unssi}, which states:
$\vdash A \subseteq C\quad\&\quad \vdash B \subseteq C\quad\Rightarrow\quad
    \vdash ( A \cup B ) \subseteq C\label{eq:unssi}$
\end{itemize}

When using deduction style we express an assertion in deduction form.
This form prefixes each hypothesis (other than definitions) and the
conclusion with a universal antecedent (``$\varphi \rightarrow$'').
The antecedent (e.g., $\varphi$)
mimics the context handled in the deduction theorem, eliminating
the need to directly use the deduction theorem.

Once you have an assertion in deduction form, you can easily convert it
to inference form or closed form:

\begin{itemize}
\item To
prove some assertion Ti in inference form, given assertion Td in deduction
form, there is a simple mechanical process you can use. First take each
Ti hypothesis and insert a \texttt{T.} $\rightarrow$ prefix (``true implies'')
using \texttt{a1i}. You
can then use the existing assertion Td to prove the resulting conclusion
with a \texttt{T.} $\rightarrow$ prefix.
Finally, you can remove that prefix using \texttt{trud},
resulting in the conclusion you wanted to prove.
\item To
prove some assertion T in closed form, given assertion Td in deduction
form, there is another simple mechanical process you can use. First,
select an expression that is the conjunction (...$\land$...) of all of the
consequents of every hypothesis of Td. Next, prove that this expression
implies each of the separate hypotheses of Td in turn by eliminating
conjuncts (there are a variety of proven assertions to do this, including
\texttt{simpl},
\texttt{simpr},
\texttt{3simpa},
\texttt{3simpb},
\texttt{3simpc},
\texttt{simp1},
\texttt{simp2},
and
\texttt{simp3}).
If the
expression has nested conjunctions, inner conjuncts can be broken out by
chaining the above theorems with \texttt{syl}
(see section \ref{syl}).\footnote{
There are actually many theorems
(labeled simp* such as \texttt{simp333}) that break out inner conjuncts in one
step, but rather than learning them you can just use the chaining we
just described to prove them, and then let the Metamath program command
\texttt{minimize{\char`\_}with}\index{\texttt{minimize{\char`\_}with} command}
figure out the right ones needed to collapse them.}
As your final step, you can then apply the already-proven assertion Td
(which is in deduction form), proving assertion T in closed form.
\end{itemize}

We can also easily convert any assertion T in closed form to its related
assertion Ti in inference form by applying
modus ponens\index{modus ponens} (see section \ref{axmp}).

The deduction form antecedent can also be used to represent the context
necessary to support natural deduction systems, so we will now
discuss natural deduction.

\subsection{Natural Deduction}\label{naturaldeduction}

Natural deduction\index{natural deduction}
(ND) systems, as such, were originally introduced in
1934 by two logicians working independently: Ja\'skowski and Gentzen. ND
systems are supposed to reconstruct, in a formally proper way, traditional
ways of mathematical reasoning (such as conditional proof, indirect proof,
and proof by cases). As reconstructions they were naturally influenced
by previous work, and many specific ND systems and notations have been
developed since their original work.

There are many ND variants, but
Indrzejczak \cite[p.~31-32]{Indrzejczak}\index{Indrzejczak, Andrzej}
suggests that any natural deductive system must satisfy at
least these three criteria:

\begin{itemize}
\item ``There are some means for entering assumptions into a proof and
also for eliminating them. Usually it requires some bookkeeping devices
for indicating the scope of an assumption, and showing that a part of
a proof depending on eliminated assumption is discharged.
\item There are no (or, at least, a very limited set of) axioms, because
their role is taken over by the set of primitive rules for introduction
and elimination of logical constants which means that elementary
inferences instead of formulae are taken as primitive.
\item (A genuine) ND system admits a lot of freedom in proof construction
and possibility of applying several proof search strategies, like
conditional proof, proof by cases, proof by reductio ad absurdum etc.''
\end{itemize}

The Metamath Proof Explorer (MPE) as defined in \texttt{set.mm}
is fundamentally a Hilbert-style system.
That is, MPE is based on a larger number of axioms (compared
to natural deduction systems), a very small set of rules of inference
(modus ponens), and the context is not changed by the rules of inference
in the middle of a proof. That said, MPE proofs can be developed using
the natural deduction (ND) approach as originally developed by Ja\'skowski
and Gentzen.

The most common and recommended approach for applying ND in MPE is to use
deduction form\index{deduction form}%
\index{forms!deduction}
and apply the MPE proven assertions that are equivalent to ND rules.
For example, MPE's \texttt{jca} is equivalent to ND rule $\land$-I
(and-insertion).
We maintain a list of equivalences that you may consult.
This approach for applying an ND approach within MPE relies on Metamath's
wff metavariables in an essential way, and is described in more detail
in the presentation ``Natural Deductions in the Metamath Proof Language''
by Mario Carneiro \cite{CarneiroND}\index{Carneiro, Mario}.

In this style many steps are an implication, whose antecedent mimics
the context ($\Gamma$) of most ND systems. To add an assumption, simply add
it to the implication antecedent (typically using
\texttt{simpr}),
and use that
new antecedent for all later claims in the same scope. If you wish to
use an assertion in an ND hypothesis scope that is outside the current
ND hypothesis scope, modify the assertion so that the ND hypothesis
assumption is added to its antecedent (typically using \texttt{adantr}). Most
proof steps will be proved using rules that have hypotheses and results
of the form $\varphi \rightarrow$ ...

An example may make this clearer.
Let's show theorem 5.5 of
\cite[p.~18]{Clemente}\index{Clemente Laboreo, Daniel}
along with a line by line translation using the usual
translation of natural deduction (ND) in the Metamath Proof Explorer
(MPE) notation (this is proof \texttt{ex-natded5.5}).
The proof's original goal was to prove
$\lnot \psi$ given two hypotheses,
$( \psi \rightarrow \chi )$ and $ \lnot \chi$.
We will translate these statements into MPE deduction form
by prefixing them all with $\varphi \rightarrow$.
As a result, in MPE the goal is stated as
$( \varphi \rightarrow \lnot \psi )$, and the two hypotheses are stated as
$( \varphi \rightarrow ( \psi \rightarrow \chi ) )$ and
$( \varphi \rightarrow \lnot \chi )$.

The following table shows the proof in Fitch natural deduction style
and its MPE equivalent.
The \textit{\#} column shows the original numbering,
\textit{MPE\#} shows the number in the equivalent MPE proof
(which we will show later),
\textit{ND Expression} shows the original proof claim in ND notation,
and \textit{MPE Translation} shows its translation into MPE
as discussed in this section.
The final columns show the rationale in ND and MPE respectively.

\needspace{4\baselineskip}
{\setlength{\extrarowsep}{4pt} % Keep rows from being too close together
\begin{longtabu}   { @{} c c X X X X }
\textbf{\#} & \textbf{MPE\#} & \textbf{ND Ex\-pres\-sion} &
\textbf{MPE Trans\-lation} & \textbf{ND Ration\-ale} &
\textbf{MPE Ra\-tio\-nale} \\
\endhead

1 & 2;3 &
$( \psi \rightarrow \chi )$ &
$( \varphi \rightarrow ( \psi \rightarrow \chi ) )$ &
Given &
\$e; \texttt{adantr} to put in ND hypothesis \\

2 & 5 &
$ \lnot \chi$ &
$( \varphi \rightarrow \lnot \chi )$ &
Given &
\$e; \texttt{adantr} to put in ND hypothesis \\

3 & 1 &
... $\vert$ $\psi$ &
$( \varphi \rightarrow \psi )$ &
ND hypothesis assumption &
\texttt{simpr} \\

4 & 4 &
... $\chi$ &
$( ( \varphi \land \psi ) \rightarrow \chi )$ &
$\rightarrow$\,E 1,3 &
\texttt{mpd} 1,3 \\

5 & 6 &
... $\lnot \chi$ &
$( ( \varphi \land \psi ) \rightarrow \lnot \chi )$ &
IT 2 &
\texttt{adantr} 5 \\

6 & 7 &
$\lnot \psi$ &
$( \varphi \rightarrow \lnot \psi )$ &
$\land$\,I 3,4,5 &
\texttt{pm2.65da} 4,6 \\

\end{longtabu}
}


The original used Latin letters; we have replaced them with Greek letters
to follow Metamath naming conventions and so that it is easier to follow
the Metamath translation. The Metamath line-for-line translation of
this natural deduction approach precedes every line with an antecedent
including $\varphi$ and uses the Metamath equivalents of the natural deduction
rules. To add an assumption, the antecedent is modified to include it
(typically by using \texttt{adantr};
\texttt{simpr} is useful when you want to
depend directly on the new assumption, as is shown here).

In Metamath we can represent the two given statements as these hypotheses:

\needspace{2\baselineskip}
\begin{itemize}
\item ex-natded5.5.1 $\vdash ( \varphi \rightarrow ( \psi \rightarrow \chi ) )$
\item ex-natded5.5.2 $\vdash ( \varphi \rightarrow \lnot \chi )$
\end{itemize}

\needspace{4\baselineskip}
Here is the proof in Metamath as a line-by-line translation:

\begin{longtabu}   { l l l X }
\textbf{Step} & \textbf{Hyp} & \textbf{Ref} & \textbf{Ex\-pres\-sion} \\
\endhead
1 & & simpr & $\vdash ( ( \varphi \land \psi ) \rightarrow \psi )$ \\
2 & & ex-natded5.5.1 &
  $\vdash ( \varphi \rightarrow ( \psi \rightarrow \chi ) )$ \\
3 & 2 & adantr &
 $\vdash ( ( \varphi \land \psi ) \rightarrow ( \psi \rightarrow \chi ) )$ \\
4 & 1, 3 & mpd &
 $\vdash ( ( \varphi \land \psi ) \rightarrow \chi ) $ \\
5 & & ex-natded5.5.2 &
 $\vdash ( \varphi \rightarrow \lnot \chi )$ \\
6 & 5 & adantr &
 $\vdash ( ( \varphi \land \psi ) \rightarrow \lnot \chi )$ \\
7 & 4, 6 & pm2.65da &
 $\vdash ( \varphi \rightarrow \lnot \psi )$ \\
\end{longtabu}

Only using specific natural deduction rules directly can lead to very
long proofs, for exactly the same reason that only using axioms directly
in Hilbert-style proofs can lead to very long proofs.
If the goal is short and clear proofs,
then it is better to reuse already-proven assertions
in deduction form than to start from scratch each time
and using only basic natural deduction rules.

\subsection{Strengths of Our Approach}

As far as we know there is nothing else in the literature like either the
weak deduction theorem or Mario Carneiro\index{Carneiro, Mario}'s
natural deduction method.
In order to
transform a hypothesis into an antecedent, the literature's standard
``Deduction Theorem''\index{Deduction Theorem}\index{Standard Deduction Theorem}
requires metalogic outside of the notions provided
by the axiom system. We instead generally prefer to use Mario Carneiro's
natural deduction method, then use the weak deduction theorem in cases
where that is difficult to apply, and only then use the full standard
deduction theorem as a last resort.

The weak deduction theorem\index{Weak Deduction Theorem}
does not require any additional metalogic
but converts an inference directly into a closed form theorem, with
a rigorous proof that uses only the axiom system. Unlike the standard
Deduction Theorem, there is no implicit external justification that we
have to trust in order to use it.

Mario Carneiro's natural deduction\index{natural deduction}
method also does not require any new metalogical
notions. It avoids the Deduction Theorem's metalogic by prefixing the
hypotheses and conclusion of every would-be inference with a universal
antecedent (``$\varphi \rightarrow$'') from the very start.

We think it is impressive and satisfying that we can do so much in a
practical sense without stepping outside of our Hilbert-style axiom system.
Of course our axiomatization, which is in the form of schemes,
contains a metalogic of its own that we exploit. But this metalogic
is relatively simple, and for our Deduction Theorem alternatives,
we primarily use just the direct substitution of expressions for
metavariables.

\begin{sloppy}
\section{Exploring the Set The\-o\-ry Data\-base}\label{exploring}
\end{sloppy}
% NOTE: All examples performed in this section are
% recorded wtih "set width 61" % on set.mm as of 2019-05-28
% commit c1e7849557661260f77cfdf0f97ac4354fbb4f4d.

At this point you may wish to study the \texttt{set.mm}\index{set theory
database (\texttt{set.mm})} file in more detail.  Pay particular
attention to the assumptions needed to define wffs\index{well-formed
formula (wff)} (which are not included above), the variable types
(\texttt{\$f}\index{\texttt{\$f} statement} statements), and the
definitions that are introduced.  Start with some simple theorems in
propositional calculus, making sure you understand in detail each step
of a proof.  Once you get past the first few proofs and become familiar
with the Metamath language, any part of the \texttt{set.mm} database
will be as easy to follow, step by step, as any other part---you won't
have to undergo a ``quantum leap'' in mathematical sophistication to be
able to follow a deep proof in set theory.

Next, you may want to explore how concepts such as natural numbers are
defined and described.  This is probably best done in conjunction with
standard set theory textbooks, which can help give you a higher-level
understanding.  The \texttt{set.mm} database provides references that will get
you started.  From there, you will be on your way towards a very deep,
rigorous understanding of abstract mathematics.

The Metamath\index{Metamath} program can help you peruse a Metamath data\-base,
wheth\-er you are trying to figure out how a certain step follows in a proof or
just have a general curiosity.  We will go through some examples of the
commands, using the \texttt{set.mm}\index{set theory database (\texttt{set.mm})}
database provided with the Metamath software.  These should help get you
started.  See Chapter~\ref{commands} for a more detailed description of
the commands.  Note that we have included the full spelling of all commands to
prevent ambiguity with future commands.  In practice you may type just the
characters needed to specify each command keyword\index{command keyword}
unambiguously, often just one or two characters per keyword, and you don't
need to type them in upper case.

First run the Metamath program as described earlier.  You should see the
\verb/MM>/ prompt.  Read in the \texttt{set.mm} file:\index{\texttt{read}
command}

\begin{verbatim}
MM> read set.mm
Reading source file "set.mm"... 34554442 bytes
34554442 bytes were read into the source buffer.
The source has 155711 statements; 2254 are $a and 32250 are $p.
No errors were found.  However, proofs were not checked.
Type VERIFY PROOF * if you want to check them.
\end{verbatim}

As with most examples in this book, what you will see
will be slightly different because we are continuously
improving our databases (including \texttt{set.mm}).

Let's check the database integrity.  This may take a minute or two to run if
your computer is slow.

\begin{verbatim}
MM> verify proof *
0 10%  20%  30%  40%  50%  60%  70%  80%  90% 100%
..................................................
All proofs in the database were verified in 2.84 s.
\end{verbatim}

No errors were reported, so every proof is correct.

You need to know the names (labels) of theorems before you can look at them.
Often just examining the database file(s) with a text editor is the best
approach.  In \texttt{set.mm} there are many detailed comments, especially near
the beginning, that can help guide you. The \texttt{search} command in the
Metamath program is also handy.  The \texttt{comments} qualifier will list the
statements whose associated comment (the one immediately before it) contain a
string you give it.  For example, if you are studying Enderton's {\em Elements
of Set Theory} \cite{Enderton}\index{Enderton, Herbert B.} you may want to see
the references to it in the database.  The search string \texttt{enderton} is not
case sensitive.  (This will not show you all the database theorems that are in
Enderton's book because there is usually only one citation for a given
theorem, which may appear in several textbooks.)\index{\texttt{search}
command}

\begin{verbatim}
MM> search * "enderton" / comments
12067 unineq $p "... Exercise 20 of [Enderton] p. 32 and ..."
12459 undif2 $p "...Corollary 6K of [Enderton] p. 144. (C..."
12953 df-tp $a "...s. Definition of [Enderton] p. 19. (Co..."
13689 unissb $p ".... Exercise 5 of [Enderton] p. 26 and ..."
\end{verbatim}
\begin{center}
(etc.)
\end{center}

Or you may want to see what theorems have something to do with
conjunction (logical {\sc and}).  The quotes around the search
string are optional when there's no ambiguity.\index{\texttt{search}
command}

\begin{verbatim}
MM> search * conjunction / comments
120 a1d $p "...be replaced with a conjunction ( ~ df-an )..."
662 df-bi $a "...viated form after conjunction is introdu..."
1319 wa $a "...ff definition to include conjunction ('and')."
1321 df-an $a "Define conjunction (logical 'and'). Defini..."
1420 imnan $p "...tion in terms of conjunction. (Contribu..."
\end{verbatim}
\begin{center}
(etc.)
\end{center}


Now we will start to look at some details.  Let's look at the first
axiom of propositional calculus
(we could use \texttt{sh st} to abbreviate
\texttt{show statement}).\index{\texttt{show statement} command}

\begin{verbatim}
MM> show statement ax-1/full
Statement 19 is located on line 881 of the file "set.mm".
"Axiom _Simp_.  Axiom A1 of [Margaris] p. 49.  One of the 3
axioms of propositional calculus.  The 3 axioms are also
        ...
19 ax-1 $a |- ( ph -> ( ps -> ph ) ) $.
Its mandatory hypotheses in RPN order are:
  wph $f wff ph $.
  wps $f wff ps $.
The statement and its hypotheses require the variables:  ph
      ps
The variables it contains are:  ph ps


Statement 49 is located on line 11182 of the file "set.mm".
Its statement number for HTML pages is 6.
"Axiom _Simp_.  Axiom A1 of [Margaris] p. 49.  One of the 3
axioms of propositional calculus.  The 3 axioms are also
given as Definition 2.1 of [Hamilton] p. 28.
...
49 ax-1 $a |- ( ph -> ( ps -> ph ) ) $.
Its mandatory hypotheses in RPN order are:
  wph $f wff ph $.
  wps $f wff ps $.
The statement and its hypotheses require the variables:
  ph ps
The variables it contains are:  ph ps
\end{verbatim}

Compare this to \texttt{ax-1} on p.~\pageref{ax1}.  You can see that the
symbol \texttt{ph} is the {\sc ascii} notation for $\varphi$, etc.  To
see the mathematical symbols for any expression you may typeset it in
\LaTeX\ (type \texttt{help tex} for instructions)\index{latex@{\LaTeX}}
or, easier, just use a text editor to look at the comments where symbols
are first introduced in \texttt{set.mm}.  The hypotheses \texttt{wph}
and \texttt{wps} required by \texttt{ax-1} mean that variables
\texttt{ph} and \texttt{ps} must be wffs.

Next we'll pick a simple theorem of propositional calculus, the Principle of
Identity, which is proved directly from the axioms.  We'll look at the
statement then its proof.\index{\texttt{show statement}
command}

\begin{verbatim}
MM> show statement id1/full
Statement 116 is located on line 11371 of the file "set.mm".
Its statement number for HTML pages is 22.
"Principle of identity.  Theorem *2.08 of [WhiteheadRussell]
p. 101.  This version is proved directly from the axioms for
demonstration purposes.
...
116 id1 $p |- ( ph -> ph ) $= ... $.
Its mandatory hypotheses in RPN order are:
  wph $f wff ph $.
Its optional hypotheses are:  wps wch wth wta wet
      wze wsi wrh wmu wla wka
The statement and its hypotheses require the variables:  ph
These additional variables are allowed in its proof:
      ps ch th ta et ze si rh mu la ka
The variables it contains are:  ph
\end{verbatim}

The optional variables\index{optional variable} \texttt{ps}, \texttt{ch}, etc.\ are
available for use in a proof of this statement if we wish, and were we to do
so we would make use of optional hypotheses \texttt{wps}, \texttt{wch}, etc.  (See
Section~\ref{dollaref} for the meaning of ``optional
hypothesis.''\index{optional hypothesis}) The reason these show up in the
statement display is that statement \texttt{id1} happens to be in their scope
(see Section~\ref{scoping} for the definition of ``scope''\index{scope}), but
in fact in propositional calculus we will never make use of optional
hypotheses or variables.  This becomes important after quantifiers are
introduced, where ``dummy'' variables\index{dummy variable} are often needed
in the middle of a proof.

Let's look at the proof of statement \texttt{id1}.  We'll use the
\texttt{show proof} command, which by default suppresses the
``non-essential'' steps that construct the wffs.\index{\texttt{show proof}
command}
We will display the proof in ``lemmon' format (a non-indented format
with explicit previous step number references) and renumber the
displayed steps:

\begin{verbatim}
MM> show proof id1 /lemmon/renumber
1 ax-1           $a |- ( ph -> ( ph -> ph ) )
2 ax-1           $a |- ( ph -> ( ( ph -> ph ) -> ph ) )
3 ax-2           $a |- ( ( ph -> ( ( ph -> ph ) -> ph ) ) ->
                     ( ( ph -> ( ph -> ph ) ) -> ( ph -> ph )
                                                          ) )
4 2,3 ax-mp      $a |- ( ( ph -> ( ph -> ph ) ) -> ( ph -> ph
                                                          ) )
5 1,4 ax-mp      $a |- ( ph -> ph )
\end{verbatim}

If you have read Section~\ref{trialrun}, you'll know how to interpret this
proof.  Step~2, for example, is an application of axiom \texttt{ax-1}.  This
proof is identical to the one in Hamilton's {\em Logic for Mathematicians}
\cite[p.~32]{Hamilton}\index{Hamilton, Alan G.}.

You may want to look at what
substitutions\index{substitution!variable}\index{variable substitution} are
made into \texttt{ax-1} to arrive at step~2.  The command to do this needs to
know the ``real'' step number, so we'll display the proof again without
the \texttt{renumber} qualifier.\index{\texttt{show proof}
command}

\begin{verbatim}
MM> show proof id1 /lemmon
 9 ax-1          $a |- ( ph -> ( ph -> ph ) )
20 ax-1          $a |- ( ph -> ( ( ph -> ph ) -> ph ) )
24 ax-2          $a |- ( ( ph -> ( ( ph -> ph ) -> ph ) ) ->
                     ( ( ph -> ( ph -> ph ) ) -> ( ph -> ph )
                                                          ) )
25 20,24 ax-mp   $a |- ( ( ph -> ( ph -> ph ) ) -> ( ph -> ph
                                                          ) )
26 9,25 ax-mp    $a |- ( ph -> ph )
\end{verbatim}

The ``real'' step number is 20.  Let's look at its details.

\begin{verbatim}
MM> show proof id1 /detailed_step 20
Proof step 20:  min=ax-1 $a |- ( ph -> ( ( ph -> ph ) -> ph )
  )
This step assigns source "ax-1" ($a) to target "min" ($e).
The source assertion requires the hypotheses "wph" ($f, step
18) and "wps" ($f, step 19).  The parent assertion of the
target hypothesis is "ax-mp" ($a, step 25).
The source assertion before substitution was:
    ax-1 $a |- ( ph -> ( ps -> ph ) )
The following substitutions were made to the source
assertion:
    Variable  Substituted with
     ph        ph
     ps        ( ph -> ph )
The target hypothesis before substitution was:
    min $e |- ph
The following substitution was made to the target hypothesis:
    Variable  Substituted with
     ph        ( ph -> ( ( ph -> ph ) -> ph ) )
\end{verbatim}

This shows the substitutions\index{substitution!variable}\index{variable
substitution} made to the variables in \texttt{ax-1}.  References are made to
steps 18 and 19 which are not shown in our proof display.  To see these steps,
you can display the proof with the \texttt{all} qualifier.

Let's look at a slightly more advanced proof of propositional calculus.  Note
that \verb+/\+ is the symbol for $\wedge$ (logical {\sc and}, also
called conjunction).\index{conjunction ($\wedge$)}
\index{logical {\sc and} ($\wedge$)}

\begin{verbatim}
MM> show statement prth/full
Statement 1791 is located on line 15503 of the file "set.mm".
Its statement number for HTML pages is 559.
"Conjoin antecedents and consequents of two premises.  This
is the closed theorem form of ~ anim12d .  Theorem *3.47 of
[WhiteheadRussell] p. 113.  It was proved by Leibniz,
and it evidently pleased him enough to call it
_praeclarum theorema_ (splendid theorem).
...
1791 prth $p |- ( ( ( ph -> ps ) /\ ( ch -> th ) ) -> ( ( ph
      /\ ch ) -> ( ps /\ th ) ) ) $= ... $.
Its mandatory hypotheses in RPN order are:
  wph $f wff ph $.
  wps $f wff ps $.
  wch $f wff ch $.
  wth $f wff th $.
Its optional hypotheses are:  wta wet wze wsi wrh wmu wla wka
The statement and its hypotheses require the variables:  ph
      ps ch th
These additional variables are allowed in its proof:  ta et
      ze si rh mu la ka
The variables it contains are:  ph ps ch th


MM> show proof prth /lemmon/renumber
1 simpl          $p |- ( ( ( ph -> ps ) /\ ( ch -> th ) ) ->
                                               ( ph -> ps ) )
2 simpr          $p |- ( ( ( ph -> ps ) /\ ( ch -> th ) ) ->
                                               ( ch -> th ) )
3 1,2 anim12d    $p |- ( ( ( ph -> ps ) /\ ( ch -> th ) ) ->
                           ( ( ph /\ ch ) -> ( ps /\ th ) ) )
\end{verbatim}

There are references to a lot of unfamiliar statements.  To see what they are,
you may type the following:

\begin{verbatim}
MM> show proof prth /statement_summary
Summary of statements used in the proof of "prth":

Statement simpl is located on line 14748 of the file
"set.mm".
"Elimination of a conjunct.  Theorem *3.26 (Simp) of
[WhiteheadRussell] p. 112. ..."
  simpl $p |- ( ( ph /\ ps ) -> ph ) $= ... $.

Statement simpr is located on line 14777 of the file
"set.mm".
"Elimination of a conjunct.  Theorem *3.27 (Simp) of
[WhiteheadRussell] ..."
  simpr $p |- ( ( ph /\ ps ) -> ps ) $= ... $.

Statement anim12d is located on line 15445 of the file
"set.mm".
"Conjoin antecedents and consequents in a deduction.
..."
  anim12d.1 $e |- ( ph -> ( ps -> ch ) ) $.
  anim12d.2 $e |- ( ph -> ( th -> ta ) ) $.
  anim12d $p |- ( ph -> ( ( ps /\ th ) -> ( ch /\ ta ) ) )
      $= ... $.
\end{verbatim}
\begin{center}
(etc.)
\end{center}

Of course you can look at each of these statements and their proofs, and
so on, back to the axioms of propositional calculus if you wish.

The \texttt{search} command is useful for finding statements when you
know all or part of their contents.  The following example finds all
statements containing \verb@ph -> ps@ followed by \verb@ch -> th@.  The
\verb@$*@ is a wildcard that matches anything; the \texttt{\$} before the
\verb$*$ prevents conflicts with math symbol token names.  The \verb@*@ after
\texttt{SEARCH} is also a wildcard that in this case means ``match any label.''
\index{\texttt{search} command}

% I'm omitting this one, since readers are unlikely to see it:
% 1096 bisymOLD $p |- ( ( ( ph -> ps ) -> ( ch -> th ) ) -> ( (
%   ( ps -> ph ) -> ( th -> ch ) ) -> ( ( ph <-> ps ) -> ( ch
%    <-> th ) ) ) )
\begin{verbatim}
MM> search * "ph -> ps $* ch -> th"
1791 prth $p |- ( ( ( ph -> ps ) /\ ( ch -> th ) ) -> ( ( ph
    /\ ch ) -> ( ps /\ th ) ) )
2455 pm3.48 $p |- ( ( ( ph -> ps ) /\ ( ch -> th ) ) -> ( (
    ph \/ ch ) -> ( ps \/ th ) ) )
117859 pm11.71 $p |- ( ( E. x ph /\ E. y ch ) -> ( ( A. x (
    ph -> ps ) /\ A. y ( ch -> th ) ) <-> A. x A. y ( ( ph /\
    ch ) -> ( ps /\ th ) ) ) )
\end{verbatim}

Three statements, \texttt{prth}, \texttt{pm3.48},
 and \texttt{pm11.71}, were found to match.

To see what axioms\index{axiom} and definitions\index{definition}
\texttt{prth} ultimately depends on for its proof, you can have the
program backtrack through the hierarchy\index{hierarchy} of theorems and
definitions.\index{\texttt{show trace{\char`\_}back} command}

\begin{verbatim}
MM> show trace_back prth /essential/axioms
Statement "prth" assumes the following axioms ($a
statements):
  ax-1 ax-2 ax-3 ax-mp df-bi df-an
\end{verbatim}

Note that the 3 axioms of propositional calculus and the modus ponens rule are
needed (as expected); in addition, there are a couple of definitions that are used
along the way.  Note that Metamath makes no distinction\index{axiom vs.\
definition} between axioms\index{axiom} and definitions\index{definition}.  In
\texttt{set.mm} they have been distinguished artificially by prefixing their
labels\index{labels in \texttt{set.mm}} with \texttt{ax-} and \texttt{df-}
respectively.  For example, \texttt{df-an} defines conjunction (logical {\sc
and}), which is represented by the symbol \verb+/\+.
Section~\ref{definitions} discusses the philosophy of definitions, and the
Metamath language takes a particularly simple, conservative approach by using
the \texttt{\$a}\index{\texttt{\$a} statement} statement for both axioms and
definitions.

You can also have the program compute how many steps a proof
has\index{proof length} if we were to follow it all the way back to
\texttt{\$a} statements.

\begin{verbatim}
MM> show trace_back prth /essential/count_steps
The statement's actual proof has 3 steps.  Backtracking, a
total of 79 different subtheorems are used.  The statement
and subtheorems have a total of 274 actual steps.  If
subtheorems used only once were eliminated, there would be a
total of 38 subtheorems, and the statement and subtheorems
would have a total of 185 steps.  The proof would have 28349
steps if fully expanded back to axiom references.  The
maximum path length is 38.  A longest path is:  prth <-
anim12d <- syl2and <- sylan2d <- ancomsd <- ancom <- pm3.22
<- pm3.21 <- pm3.2 <- ex <- sylbir <- biimpri <- bicomi <-
bicom1 <- bi2 <- dfbi1 <- impbii <- bi3 <- simprim <- impi <-
con1i <- nsyl2 <- mt3d <- con1d <- notnot1 <- con2i <- nsyl3
<- mt2d <- con2d <- notnot2 <- pm2.18d <- pm2.18 <- pm2.21 <-
pm2.21d <- a1d <- syl <- mpd <- a2i <- a2i.1 .
\end{verbatim}

This tells us that we would have to inspect 274 steps if we want to
verify the proof completely starting from the axioms.  A few more
statistics are also shown.  There are one or more paths back to axioms
that are the longest; this command ferrets out one of them and shows it
to you.  There may be a sense in which the longest path length is
related to how ``deep'' the theorem is.

We might also be curious about what proofs depend on the theorem
\texttt{prth}.  If it is never used later on, we could eliminate it as
redundant if it has no intrinsic interest by itself.\index{\texttt{show
usage} command}

% I decided to show the OLD values here.
\begin{verbatim}
MM> show usage prth
Statement "prth" is directly referenced in the proofs of 18
statements:
  mo3 moOLD 2mo 2moOLD euind reuind reuss2 reusv3i opelopabt
  wemaplem2 rexanre rlimcn2 o1of2 o1rlimmul 2sqlem6 spanuni
  heicant pm11.71
\end{verbatim}

Thus \texttt{prth} is directly used by 18 proofs.
We can use the \texttt{/recursive} qualifier to include indirect use:

\begin{verbatim}
MM> show usage prth /recursive
Statement "prth" directly or indirectly affects the proofs of
24214 statements:
  mo3 mo mo3OLD eu2 moOLD eu2OLD eu3OLD mo4f mo4 eu4 mopick
...
\end{verbatim}

\subsection{A Note on the ``Compact'' Proof Format}

The present version of Metamath\index{Metamath!limitations of
version 0.07.30} will display proofs in a ``compact''\index{compact proof}
format whenever the proof is stored in compressed format in the database.  It
may be be slightly confusing unless you know how to interpret it.  (A
future version may eliminate this format from displays.)  For example,
if you display the complete proof of theorem \texttt{id1} it will start
off as follows:

\begin{verbatim}
MM> show proof id1 /lemmon/all
 1 wph           $f wff ph
 2 wph           $f wff ph
 3 wph           $f wff ph
 4 2,3 wi    @4: $a wff ( ph -> ph )
 5 1,4 wi    @5: $a wff ( ph -> ( ph -> ph ) )
 6 @4            $a wff ( ph -> ph )
\end{verbatim}

\begin{center}
{etc.}
\end{center}

Step 4 has a ``local label,''\index{local label} \texttt{@4}, assigned to it.
Later on, at step 6, the label \texttt{@1} is referenced instead of
displaying the explicit proof for that step.  This technique takes advantage
of the fact that steps in a proof often repeat, especially during the
construction of wffs.  The compact format reduces the number of steps in the
proof display and may actually be preferred by some people, which is one
reason that it is still displayed by the program.

If you want to see the normal format with the ``true'' step numbers, you can
use the following workaround:\index{\texttt{save proof} command}

\begin{verbatim}
MM> save proof id1 /normal
The proof of "id1" has been reformatted and saved internally.
Remember to use WRITE SOURCE to save it permanently.
MM> show proof id1 /lemmon/all
 1 wph           $f wff ph
 2 wph           $f wff ph
 3 wph           $f wff ph
 4 2,3 wi        $a wff ( ph -> ph )
 5 1,4 wi        $a wff ( ph -> ( ph -> ph ) )
 6 wph           $f wff ph
 7 wph           $f wff ph
 8 6,7 wi        $a wff ( ph -> ph )
\end{verbatim}

\begin{center}
{etc.}
\end{center}

Note that the original 6 steps are now 8 steps.  However, the format is
now the same as that described in Chapter~\ref{using}.

\chapter{The Metamath Language}
\label{languagespec}

\begin{quote}
  {\em Thus mathematics may be defined as the subject in which we never know
what we are talking about, nor whether what we are saying is true.}
    \flushright\sc  Bertrand Russell\footnote{\cite[p.~84]{Russell2}.}\\
\end{quote}\index{Russell, Bertrand}

Probably the most striking feature of the Metamath language is its almost
complete absence of hard-wired syntax. Metamath\index{Metamath} does not
understand any mathematics or logic other than that needed to construct finite
sequences of symbols according to a small set of simple, built-in rules.  The
only rule it uses in a proof is the substitution of an expression (symbol
sequence) for a variable, subject to a simple constraint to prevent
bound-variable clashes.  The primitive notions built into Metamath involve the
simple manipulation of finite objects (symbols) that we as humans can easily
visualize and that computers can easily deal with.  They seem to be just
about the simplest notions possible that are required to do standard
mathematics.

This chapter serves as a reference manual for the Metamath\index{Metamath}
language. It covers the tedious technical details of the language, some of
which you may wish to skim in a first reading.  On the other hand, you should
pay close attention to the defined terms in {\bf boldface}; they have precise
meanings that are important to keep in mind for later understanding.  It may
be best to first become familiar with the examples in Chapter~\ref{using} to
gain some motivation for the language.

%% Uncomment this when uncommenting section {formalspec} below
If you have some knowledge of set theory, you may wish to study this
chapter in conjunction with the formal set-theoretical description of the
Metamath language in Appendix~\ref{formalspec}.

We will use the name ``Metamath''\index{Metamath} to mean either the Metamath
computer language or the Metamath software associated with the computer
language.  We will not distinguish these two when the context is clear.

The next section contains the complete specification of the Metamath
language.
It serves as an
authoritative reference and presents the syntax in enough detail to
write a parser\index{parsing Metamath} and proof verifier.  The
specification is terse and it is probably hard to learn the language
directly from it, but we include it here for those impatient people who
prefer to see everything up front before looking at verbose expository
material.  Later sections explain this material and provide examples.
We will repeat the definitions in those sections, and you may skip the
next section at first reading and proceed to Section~\ref{tut1}
(p.~\pageref{tut1}).

\section{Specification of the Metamath Language}\label{spec}
\index{Metamath!specification}

\begin{quote}
  {\em Sometimes one has to say difficult things, but one ought to say
them as simply as one knows how.}
    \flushright\sc  G. H. Hardy\footnote{As quoted in
    \cite{deMillo}, p.~273.}\\
\end{quote}\index{Hardy, G. H.}

\subsection{Preliminaries}\label{spec1}

% Space is technically a printable character, so we'll word things
% carefully so it's unambiguous.
A Metamath {\bf database}\index{database} is built up from a top-level source
file together with any source files that are brought in through file inclusion
commands (see below).  The only characters that are allowed to appear in a
Metamath source file are the 94 non-whitespace printable {\sc
ascii}\index{ascii@{\sc ascii}} characters, which are digits, upper and lower
case letters, and the following 32 special
characters\index{special characters}:\label{spec1chars}

\begin{verbatim}
! " # $ % & ' ( ) * + , - . / :
; < = > ? @ [ \ ] ^ _ ` { | } ~
\end{verbatim}
plus the following characters which are the ``white space'' characters:
space (a printable character),
tab, carriage return, line feed, and form feed.\label{whitespace}
We will use \texttt{typewriter}
font to display the printable characters.

A Metamath database consists of a sequence of three kinds of {\bf
tokens}\index{token} separated by {\bf white space}\index{white space}
(which is any sequence of one or more white space characters).  The set
of {\bf keyword}\index{keyword} tokens is \texttt{\$\char`\{},
\texttt{\$\char`\}}, \texttt{\$c}, \texttt{\$v}, \texttt{\$f},
\texttt{\$e}, \texttt{\$d}, \texttt{\$a}, \texttt{\$p}, \texttt{\$.},
\texttt{\$=}, \texttt{\$(}, \texttt{\$)}, \texttt{\$[}, and
\texttt{\$]}.  The last four are called {\bf auxiliary}\index{auxiliary
keyword} or preprocessing keywords.  A {\bf label}\index{label} token
consists of any combination of letters, digits, and the characters
hyphen, underscore, and period.  A {\bf math symbol}\index{math symbol}
token may consist of any combination of the 92 printable standard {\sc
ascii} characters other than space or \texttt{\$}~. All tokens are
case-sensitive.

\subsection{Preprocessing}

The token \texttt{\$(} begins a {\bf comment} and
\texttt{\$)} ends a comment.\index{\texttt{\$(}
and \texttt{\$)} auxiliary keywords}\index{comment}
Comments may contain any of
the 94 non-whitespace printable characters and white space,
except they may not contain the
2-character sequences \texttt{\$(} or \texttt{\$)} (comments do not nest).
Comments are ignored (treated
like white space) for the purpose of parsing, e.g.,
\texttt{\$( \$[ \$)} is a comment.
See p.~\pageref{mathcomments} for comment typesetting conventions; these
conventions may be ignored for the purpose of parsing.

A {\bf file inclusion command} consists of \texttt{\$[} followed by a file name
followed by \texttt{\$]}.\index{\texttt{\$[} and \texttt{\$]} auxiliary
keywords}\index{included file}\index{file inclusion}
It is only allowed in the outermost scope (i.e., not between
\texttt{\$\char`\{} and \texttt{\$\char`\}})
and must not be inside a statement (e.g., it may not occur
between the label of a \texttt{\$a} statement and its \texttt{\$.}).
The file name may not
contain a \texttt{\$} or white space.  The file must exist.
The case-sensitivity
of its name follows the conventions of the operating system.  The contents of
the file replace the inclusion command.
Included files may include other files.
Only the first reference to a given file is included; any later
references to the same file (whether in the top-level file or in included
files) cause the inclusion command to be ignored (treated like white space).
A verifier may assume that file names with different strings
refer to different files for the purpose of ignoring later references.
A file self-reference is ignored, as is any reference to the top-level file
(to avoid loops).
Included files may not include a \texttt{\$(} without a matching \texttt{\$)},
may not include a \texttt{\$[} without a matching \texttt{\$]}, and may
not include incomplete statements (e.g., a \texttt{\$a} without a matching
\texttt{\$.}).
It is currently unspecified if path references are relative to the process'
current directory or the file's containing directory, so databases should
avoid using pathname separators (e.g., ``/'') in file names.

Like all tokens, the \texttt{\$(}, \texttt{\$)}, \texttt{\$[}, and \texttt{\$]} keywords
must be surrounded by white space.

\subsection{Basic Syntax}

After preprocessing, a database will consist of a sequence of {\bf
statements}.
These are the scoping statements \texttt{\$\char`\{} and
\texttt{\$\char`\}}, along with the \texttt{\$c}, \texttt{\$v},
\texttt{\$f}, \texttt{\$e}, \texttt{\$d}, \texttt{\$a}, and \texttt{\$p}
statements.

A {\bf scoping statement}\index{scoping statement} consists only of its
keyword, \texttt{\$\char`\{} or \texttt{\$\char`\}}.
A \texttt{\$\char`\{} begins a {\bf
block}\index{block} and a matching \texttt{\$\char`\}} ends the block.
Every \texttt{\$\char`\{}
must have a matching \texttt{\$\char`\}}.
Defining it recursively, we say a block
contains a sequence of zero or more tokens other
than \texttt{\$\char`\{} and \texttt{\$\char`\}} and
possibly other blocks.  There is an {\bf outermost
block}\index{block!outermost} not bracketed by \texttt{\$\char`\{} \texttt{\$\char`\}}; the end
of the outermost block is the end of the database.

% LaTeX bug? can't do \bf\tt

A {\bf \$v} or {\bf \$c statement}\index{\texttt{\$v} statement}\index{\texttt{\$c}
statement} consists of the keyword token \texttt{\$v} or \texttt{\$c} respectively,
followed by one or more math symbols,
% The word "token" is used to distinguish "$." from the sentence-ending period.
followed by the \texttt{\$.}\ token.
These
statements {\bf declare}\index{declaration} the math symbols to be {\bf
variables}\index{variable!Metamath} or {\bf constants}\index{constant}
respectively. The same math symbol may not occur twice in a given \texttt{\$v} or
\texttt{\$c} statement.

%c%A math symbol becomes an {\bf active}\index{active math symbol}
%c%when declared and stays active until the end of the block in which it is
%c%declared.  A math symbol may not be declared a second time while it is active,
%c%but it may be declared again after it becomes inactive.

A math symbol becomes {\bf active}\index{active math symbol} when declared
and stays active until the end of the block in which it is declared.  A
variable may not be declared a second time while it is active, but it
may be declared again (as a variable, but not as a constant) after it
becomes inactive.  A constant must be declared in the outermost block and may
not be declared a second time.\footnote{The rules for redeclaration may
become more general in the future;
see p.~\pageref{redeclarationf}.}\index{redeclaration of symbols}

A {\bf \$f statement}\index{\texttt{\$f} statement} consists of a label,
followed by \texttt{\$f}, followed by its typecode (an active constant),
followed by an
active variable, followed by the \texttt{\$.}\ token.  A {\bf \$e
statement}\index{\texttt{\$e} statement} consists of a label, followed
by \texttt{\$e}, followed by its typecode (an active constant),
followed by zero or more
active math symbols, followed by the \texttt{\$.}\ token.  A {\bf
hypothesis}\index{hypothesis} is a \texttt{\$f} or \texttt{\$e}
statement.
The type declared by a \texttt{\$f} statement for a given label
is global even if the variable is not
(e.g., a database may not have \texttt{wff P} in one local scope
nd \texttt{class P} in another).

A {\bf simple \$d statement}\index{\texttt{\$d} statement!simple}
consists of \texttt{\$d}, followed by two different active variables,
followed by the \texttt{\$.}\ token.  A {\bf compound \$d
statement}\index{\texttt{\$d} statement!compound} consists of
\texttt{\$d}, followed by three or more variables (all different),
followed by the \texttt{\$.}\ token.  The order of the variables in a
\texttt{\$d} statement is unimportant.  A compound \texttt{\$d}
statement is equivalent to a set of simple \texttt{\$d} statements, one
for each possible pair of variables occurring in the compound
\texttt{\$d} statement.  Henceforth in this specification we shall
assume all \texttt{\$d} statements are simple.  A \texttt{\$d} statement
is also called a {\bf disjoint} (or {\bf distinct}) {\bf variable
restriction}.\index{disjoint-variable restriction}

A {\bf \$a statement}\index{\texttt{\$a} statement} consists of a label,
followed by \texttt{\$a}, followed by its typecode (an active constant),
followed by
zero or more active math symbols, followed by the \texttt{\$.}\ token.  A {\bf
\$p statement}\index{\texttt{\$p} statement} consists of a label,
followed by \texttt{\$p}, followed by its typecode (an active constant),
followed by
zero or more active math symbols, followed by \texttt{\$=}, followed by
a sequence of labels, followed by the \texttt{\$.}\ token.  An {\bf
assertion}\index{assertion} is a \texttt{\$a} or \texttt{\$p} statement.

A \texttt{\$f}, \texttt{\$e}, or \texttt{\$d} statement is {\bf active}\index{active
statement} from the place it occurs until the end of the block it occurs in.
A \texttt{\$a} or \texttt{\$p} statement is {\bf active} from the place it occurs
through the end of the database.
There may not be two active \texttt{\$f} statements containing the same
variable.  Each variable in a \texttt{\$e}, \texttt{\$a}, or
\texttt{\$p} statement must exist in an active \texttt{\$f}
statement.\footnote{This requirement can greatly simplify the
unification algorithm (substitution calculation) required by proof
verification.}

%The label that begins each \texttt{\$f}, \texttt{\$e}, \texttt{\$a}, and
%\texttt{\$p} statement must be unique.
Each label token must be unique, and
no label token may match any math symbol
token.\label{namespace}\footnote{This
restriction was added on June 24, 2006.
It is not theoretically necessary but is imposed to make it easier to
write certain parsers.}

The set of {\bf mandatory variables}\index{mandatory variable} associated with
an assertion is the set of (zero or more) variables in the assertion and in any
active \texttt{\$e} statements.  The (possibly empty) set of {\bf mandatory
hypotheses}\index{mandatory hypothesis} is the set of all active \texttt{\$f}
statements containing mandatory variables, together with all active \texttt{\$e}
statements.
The set of {\bf mandatory {\bf \$d} statements}\index{mandatory
disjoint-variable restriction} associated with an assertion are those active
\texttt{\$d} statements whose variables are both among the assertion's
mandatory variables.

\subsection{Proof Verification}\label{spec4}

The sequence of labels between the \texttt{\$=} and \texttt{\$.}\ tokens
in a \texttt{\$p} statement is a {\bf proof}.\index{proof!Metamath} Each
label in a proof must be the label of an active statement other than the
\texttt{\$p} statement itself; thus a label must refer either to an
active hypothesis of the \texttt{\$p} statement or to an earlier
assertion.

An {\bf expression}\index{expression} is a sequence of math symbols. A {\bf
substitution map}\index{substitution map} associates a set of variables with a
set of expressions.  It is acceptable for a variable to be mapped to an
expression containing it.  A {\bf
substitution}\index{substitution!variable}\index{variable substitution} is the
simultaneous replacement of all variables in one or more expressions with the
expressions that the variables map to.

A proof is scanned in order of its label sequence.  If the label refers to an
active hypothesis, the expression in the hypothesis is pushed onto a
stack.\index{stack}\index{RPN stack}  If the label refers to an assertion, a
(unique) substitution must exist that, when made to the mandatory hypotheses
of the referenced assertion, causes them to match the topmost (i.e.\ most
recent) entries of the stack, in order of occurrence of the mandatory
hypotheses, with the topmost stack entry matching the last mandatory
hypothesis of the referenced assertion.  As many stack entries as there are
mandatory hypotheses are then popped from the stack.  The same substitution is
made to the referenced assertion, and the result is pushed onto the stack.
After the last label in the proof is processed, the stack must have a single
entry that matches the expression in the \texttt{\$p} statement containing the
proof.

%c%{\footnotesize\begin{quotation}\index{redeclaration of symbols}
%c%{{\em Comment.}\label{spec4comment} Whenever a math symbol token occurs in a
%c%{\texttt{\$c} or \texttt{\$v} statement, it is considered to designate a distinct new
%c%{symbol, even if the same token was previously declared (and is now inactive).
%c%{Thus a math token declared as a constant in two different blocks is considered
%c%{to designate two distinct constants (even though they have the same name).
%c%{The two constants will not match in a proof that references both blocks.
%c%{However, a proof referencing both blocks is acceptable as long as it doesn't
%c%{require that the constants match.  Similarly, a token declared to be a
%c%{constant for a referenced assertion will not match the same token declared to
%c%{be a variable for the \texttt{\$p} statement containing the proof.  In the case
%c%{of a token declared to be a variable for a referenced assertion, this is not
%c%{an issue since the variable can be substituted with whatever expression is
%c%{needed to achieve the required match.
%c%{\end{quotation}}
%c2%A proof may reference an assertion that contains or whose hypotheses contain a
%c2%constant that is not active for the \texttt{\$p} statement containing the proof.
%c2%However, the final result of the proof may not contain that constant. A proof
%c2%may also reference an assertion that contains or whose hypotheses contain a
%c2%variable that is not active for the \texttt{\$p} statement containing the proof.
%c2%That variable, of course, will be substituted with whatever expression is
%c2%needed to achieve the required match.

A proof may contain a \texttt{?}\ in place of a label to indicate an unknown step
(Section~\ref{unknown}).  A proof verifier may ignore any proof containing
\texttt{?}\ but should warn the user that the proof is incomplete.

A {\bf compressed proof}\index{compressed proof}\index{proof!compressed} is an
alternate proof notation described in Appen\-dix~\ref{compressed}; also see
references to ``compressed proof'' in the Index.  Compressed proofs are a
Metamath language extension which a complete proof verifier should be able to
parse and verify.

\subsubsection{Verifying Disjoint Variable Restrictions}

Each substitution made in a proof must be checked to verify that any
disjoint variable restrictions are satisfied, as follows.

If two variables replaced by a substitution exist in a mandatory \texttt{\$d}
statement\index{\texttt{\$d} statement} of the assertion referenced, the two
expressions resulting from the substitution must satisfy the following
conditions.  First, the two expressions must have no variables in common.
Second, each possible pair of variables, one from each expression, must exist
in an active \texttt{\$d} statement of the \texttt{\$p} statement containing the
proof.

\vskip 1ex

This ends the specification of the Metamath language;
see Appendix \ref{BNF} for its syntax in
Extended Backus--Naur Form (EBNF)\index{Extended Backus--Naur Form}\index{EBNF}.

\section{The Basic Keywords}\label{tut1}

Our expository material begins here.

Like most computer languages, Metamath\index{Metamath} takes its input from
one or more {\bf source files}\index{source file} which contain characters
expressed in the standard {\sc ascii} (American Standard Code for Information
Interchange)\index{ascii@{\sc ascii}} code for computers.  A source file
consists of a series of {\bf tokens}\index{token}, which are strings of
non-whitespace
printable characters (from the set of 94 shown on p.~\pageref{spec1chars})
separated by {\bf white space}\index{white space} (spaces, tabs, carriage
returns, line feeds, and form feeds). Any string consisting only of these
characters is treated the same as a single space.  The non-whitespace printable
characters\index{printable character} that Metamath recognizes are the 94
characters on standard {\sc ascii} keyboards.

Metamath has the ability to join several files together to form its
input (Section~\ref{include}).  We call the aggregate contents of all
the files after they have been joined together a {\bf
database}\index{database} to distinguish it from an individual source
file.  The tokens in a database consist of {\bf
keywords}\index{keyword}, which are built into the language, together
with two kinds of user-defined tokens called {\bf labels}\index{label}
and {\bf math symbols}\index{math symbol}.  (Often we will simply say
{\bf symbol}\index{symbol} instead of math symbol for brevity).  The set
of {\bf basic keywords}\index{basic keyword} is
\texttt{\$c}\index{\texttt{\$c} statement},
\texttt{\$v}\index{\texttt{\$v} statement},
\texttt{\$e}\index{\texttt{\$e} statement},
\texttt{\$f}\index{\texttt{\$f} statement},
\texttt{\$d}\index{\texttt{\$d} statement},
\texttt{\$a}\index{\texttt{\$a} statement},
\texttt{\$p}\index{\texttt{\$p} statement},
\texttt{\$=}\index{\texttt{\$=} keyword},
\texttt{\$.}\index{\texttt{\$.}\ keyword},
\texttt{\$\char`\{}\index{\texttt{\$\char`\{} and \texttt{\$\char`\}}
keywords}, and \texttt{\$\char`\}}.  This is the complete set of
syntactical elements of what we call the {\bf basic
language}\index{basic language} of Metamath, and with them you can
express all of the mathematics that were intended by the design of
Metamath.  You should make it a point to become very familiar with them.
Table~\ref{basickeywords} lists the basic keywords along with a brief
description of their functions.  For now, this description will give you
only a vague notion of what the keywords are for; later we will describe
the keywords in detail.


\begin{table}[htp] \caption{Summary of the basic Metamath
keywords} \label{basickeywords}
\begin{center}
\begin{tabular}{|p{4pc}|l|}
\hline
\em \centering Keyword&\em Description\\
\hline
\hline
\centering
   \texttt{\$c}&Constant symbol declaration\\
\hline
\centering
   \texttt{\$v}&Variable symbol declaration\\
\hline
\centering
   \texttt{\$d}&Disjoint variable restriction\\
\hline
\centering
   \texttt{\$f}&Variable-type (``floating'') hypothesis\\
\hline
\centering
   \texttt{\$e}&Logical (``essential'') hypothesis\\
\hline
\centering
   \texttt{\$a}&Axiomatic assertion\\
\hline
\centering
   \texttt{\$p}&Provable assertion\\
\hline
\centering
   \texttt{\$=}&Start of proof in \texttt{\$p} statement\\
\hline
\centering
   \texttt{\$.}&End of the above statement types\\
\hline
\centering
   \texttt{\$\char`\{}&Start of block\\
\hline
\centering
   \texttt{\$\char`\}}&End of block\\
\hline
\end{tabular}
\end{center}
\end{table}

%For LaTeX bug(?) where it puts tables on blank page instead of btwn text
%May have to adjust if text changes
%\newpage

There are some additional keywords, called {\bf auxiliary
keywords}\index{auxiliary keyword} that help make Metamath\index{Metamath}
more practical. These are part of the {\bf extended language}\index{extended
language}. They provide you with a means to put comments into a Metamath
source file\index{source file} and reference other source files.  We will
introduce these in later sections. Table~\ref{otherkeywords} summarizes them
so that you can recognize them now if you want to peruse some source
files while learning the basic keywords.


\begin{table}[htp] \caption{Auxiliary Metamath
keywords} \label{otherkeywords}
\begin{center}
\begin{tabular}{|p{4pc}|l|}
\hline
\em \centering Keyword&\em Description\\
\hline
\hline
\centering
   \texttt{\$(}&Start of comment\\
\hline
\centering
   \texttt{\$)}&End of comment\\
\hline
\centering
   \texttt{\$[}&Start of included source file name\\
\hline
\centering
   \texttt{\$]}&End of included source file name\\
\hline
\end{tabular}
\end{center}
\end{table}
\index{\texttt{\$(} and \texttt{\$)} auxiliary keywords}
\index{\texttt{\$[} and \texttt{\$]} auxiliary keywords}


Unlike those in some computer languages, the keywords\index{keyword} are short
two-character sequences rather than English-like words.  While this may make
them slightly more difficult to remember at first, their brevity allows
them to blend in with the mathematics being described, not
distract from it, like punctuation marks.


\subsection{User-Defined Tokens}\label{dollardollar}\index{token}

As you may have noticed, all keywords\index{keyword} begin with the \texttt{\$}
character.  This mundane monetary symbol is not ordinarily used in higher
mathematics (outside of grant proposals), so we have appropriated it to
distinguish the Metamath\index{Metamath} keywords from ordinary mathematical
symbols. The \texttt{\$} character is thus considered special and may not be
used as a character in a user-defined token.  All tokens and keywords are
case-sensitive; for example, \texttt{n} is considered to be a different character
from \texttt{N}.  Case-sensitivity makes the available {\sc ascii} character set
as rich as possible.

\subsubsection{Math Symbol Tokens}\index{token}

Math symbols\index{math symbol} are tokens used to represent the symbols
that appear in ordinary mathematical formulas.  They may consist of any
combination of the 93 non-whitespace printable {\sc ascii} characters other than
\texttt{\$}~. Some examples are \texttt{x}, \texttt{+}, \texttt{(},
\texttt{|-}, \verb$!%@?&$, and \texttt{bounded}.  For readability, it is
best to try to make these look as similar to actual mathematical symbols
as possible, within the constraints of the {\sc ascii} character set, in
order to make the resulting mathematical expressions more readable.

In the Metamath\index{Metamath} language, you express ordinary
mathematical formulas and statements as sequences of math symbols such
as \texttt{2 + 2 = 4} (five symbols, all constants).\footnote{To
eliminate ambiguity with other expressions, this is expressed in the set
theory database \texttt{set.mm} as \texttt{|- ( 2 + 2
 ) = 4 }, whose \LaTeX\ equivalent is $\vdash
(2+2)=4$.  The \,$\vdash$ means ``is a theorem'' and the
parentheses allow explicit associative grouping.}\index{turnstile
({$\,\vdash$})} They may even be English
sentences, as in \texttt{E is closed and bounded} (five symbols)---here
\texttt{E} would be a variable and the other four symbols constants.  In
principle, a Metamath database could be constructed to work with almost
any unambiguous English-language mathematical statement, but as a
practical matter the definitions needed to provide for all possible
syntax variations would be cumbersome and distracting and possibly have
subtle pitfalls accidentally built in.  We generally recommend that you
express mathematical statements with compact standard mathematical
symbols whenever possible and put their English-language descriptions in
comments.  Axioms\index{axiom} and definitions\index{definition}
(\texttt{\$a}\index{\texttt{\$a} statement} statements) are the only
places where Metamath will not detect an error, and doing this will help
reduce the number of definitions needed.

You are free to use any tokens\index{token} you like for math
symbols\index{math symbol}.  Appendix~\ref{ASCII} recommends token names to
use for symbols in set theory, and we suggest you adopt these in order to be
able to include the \texttt{set.mm} set theory database in your database.  For
printouts, you can convert the tokens in a database
to standard mathematical symbols with the \LaTeX\ typesetting program.  The
Metamath command \texttt{open tex} {\em filename}\index{\texttt{open tex} command}
produces output that can be read by \LaTeX.\index{latex@{\LaTeX}}
The correspondence
between tokens and the actual symbols is made by \texttt{latexdef}
statements inside a special database comment tagged
with \texttt{\$t}.\index{\texttt{\$t} comment}\index{typesetting comment}
  You can edit
this comment to change the definitions or add new ones.
Appendix~\ref{ASCII} describes how to do this in more detail.

% White space\index{white space} is normally used to separate math
% symbol\index{math symbol} tokens, but they may be juxtaposed without white
% space in \texttt{\$d}\index{\texttt{\$d} statement}, \texttt{\$e}\index{\texttt{\$e}
% statement}, \texttt{\$f}\index{\texttt{\$f} statement}, \texttt{\$a}\index{\texttt{\$a}
% statement}, and \texttt{\$p}\index{\texttt{\$p} statement} statements when no
% ambiguity will result.  Specifically, Metamath parses the math symbol sequence
% in one of these statements in the following manner:  when the math symbol
% sequence has been broken up into tokens\index{token} up to a given character,
% the next token is the longest string of characters that could constitute a
% math symbol that is active\index{active
% math symbol} at that point.  (See Section~\ref{scoping} for the
% definition of an active math symbol.)  For example, if \texttt{-}, \texttt{>}, and
% \texttt{->} are the only active math symbols, the juxtaposition \texttt{>-} will be
% interpreted as the two symbols \texttt{>} and \texttt{-}, whereas \texttt{->} will
% always be interpreted as that single symbol.\footnote{For better readability we
% recommend a white space between each token.  This also makes searching for a
% symbol easier to do with an editor.  Omission of optional white space is useful
% for reducing typing when assigning an expression to a temporary
% variable\index{temporary variable} with the \texttt{let variable} Metamath
% program command.}\index{\texttt{let variable} command}
%
% Keywords\index{keyword} may be placed next to math symbols without white
% space\index{white space} between them.\footnote{Again, we do not recommend
% this for readability.}
%
% The math symbols\index{math symbol} in \texttt{\$c}\index{\texttt{\$c} statement}
% and \texttt{\$v}\index{\texttt{\$v} statement} statements must always be separated
% by white space\index{white
% space}, for the obvious reason that these statements define the names
% of the symbols.
%
% Math symbols referred to in comments (see Section~\ref{comments}) must also be
% separated by white space.  This allows you to make comments about symbols that
% are not yet active\index{active
% math symbol}.  (The ``math mode'' feature of comments is also a quick and
% easy way to obtain word processing text with embedded mathematical symbols,
% independently of the main purpose of Metamath; the way to do this is described
% in Section~\ref{comments})

\subsubsection{Label Tokens}\index{token}\index{label}

Label tokens are used to identify Metamath\index{Metamath} statements for
later reference. Label tokens may contain only letters, digits, and the three
characters period, hyphen, and underscore:
\begin{verbatim}
. - _
\end{verbatim}

A label is {\bf declared}\index{label declaration} by placing it immediately
before the keyword of the statement it identifies.  For example, the label
\texttt{axiom.1} might be declared as follows:
\begin{verbatim}
axiom.1 $a |- x = x $.
\end{verbatim}

Each \texttt{\$e}\index{\texttt{\$e} statement},
\texttt{\$f}\index{\texttt{\$f} statement},
\texttt{\$a}\index{\texttt{\$a} statement}, and
\texttt{\$p}\index{\texttt{\$p} statement} statement in a database must
have a label declared for it.  No other statement types may have label
declarations.  Every label must be unique.

A label (and the statement it identifies) is {\bf referenced}\index{label
reference} by including the label between the \texttt{\$=}\index{\texttt{\$=}
keyword} and \texttt{\$.}\index{\texttt{\$.}\ keyword}\ keywords in a \texttt{\$p}
statement.  The sequence of labels\index{label sequence} between these two
keywords is called a {\bf proof}\index{proof}.  An example of a statement with
a proof that we will encounter later (Section~\ref{proof}) is
\begin{verbatim}
wnew $p wff ( s -> ( r -> p ) )
     $= ws wr wp w2 w2 $.
\end{verbatim}

You don't have to know what this means just yet, but you should know that the
label \texttt{wnew} is declared by this \texttt{\$p} statement and that the labels
\texttt{ws}, \texttt{wr}, \texttt{wp}, and \texttt{w2} are assumed to have been declared
earlier in the database and are referenced here.

\subsection{Constants and Variables}
\index{constant}
\index{variable}

An {\bf expression}\index{expression} is any sequence of math
symbols, possibly empty.

The basic Metamath\index{Metamath} language\index{basic language} has two
kinds of math symbols\index{math symbol}:  {\bf constants}\index{constant} and
{\bf variables}\index{variable}.  In a Metamath proof, a constant may not be
substituted with any expression.  A variable can be
substituted\index{substitution!variable}\index{variable substitution} with any
expression.  This sequence may include other variables and may even include
the variable being substituted.  This substitution takes place when proofs are
verified, and it will be described in Section~\ref{proof}.  The \texttt{\$f}
statement (described later in Section~\ref{dollaref}) is used to specify the
{\bf type} of a variable (i.e.\ what kind of
variable it is)\index{variable type}\index{type} and
give it a meaning typically
associated with a ``metavariable''\index{metavariable}\footnote{A metavariable
is a variable that ranges over the syntactical elements of the object language
being discussed; for example, one metavariable might represent a variable of
the object language and another metavariable might represent a formula in the
object language.} in ordinary mathematics; for example, a variable may be
specified to be a wff or well-formed formula (in logic), a set (in set
theory), or a non-negative integer (in number theory).

%\subsection{The \texttt{\$c} and \texttt{\$v} Declaration Statements}
\subsection{The \texttt{\$c} and \texttt{\$v} Declaration Statements}
\index{\texttt{\$c} statement}
\index{constant declaration}
\index{\texttt{\$v} statement}
\index{variable declaration}

Constants are introduced or {\bf declared}\index{constant declaration}
with \texttt{\$c}\index{\texttt{\$c} statement} statements, and
variables are declared\index{variable declaration} with
\texttt{\$v}\index{\texttt{\$v} statement} statements.  A {\bf simple}
declaration\index{simple declaration} statement introduces a single
constant or variable.  Its syntax is one of the following:
\begin{center}
  \texttt{\$c} {\em math-symbol} \texttt{\$.}\\
  \texttt{\$v} {\em math-symbol} \texttt{\$.}
\end{center}
The notation {\em math-symbol} means any math symbol token\index{token}.

Some examples of simple declaration statements are:
\begin{center}
  \texttt{\$c + \$.}\\
  \texttt{\$c -> \$.}\\
  \texttt{\$c ( \$.}\\
  \texttt{\$v x \$.}\\
  \texttt{\$v y2 \$.}
\end{center}

The characters in a math symbol\index{math symbol} being declared are
irrelevant to Meta\-math; for example, we could declare a right parenthesis to
be a variable,
\begin{center}
  \texttt{\$v ) \$.}\\
\end{center}
although this would be unconventional.

A {\bf compound} declaration\index{compound declaration} statement is a
shorthand for declaring several symbols at once.  Its syntax is one of the
following:
\begin{center}
  \texttt{\$c} {\em math-symbol}\ \,$\cdots$\ {\em math-symbol} \texttt{\$.}\\
  \texttt{\$v} {\em math-symbol}\ \,$\cdots$\ {\em math-symbol} \texttt{\$.}
\end{center}\index{\texttt{\$c} statement}
Here, the ellipsis (\ldots) means any number of {\em math-symbol}\,s.

An example of a compound declaration statement is:
\begin{center}
  \texttt{\$v x y mu \$.}\\
\end{center}
This is equivalent to the three simple declaration statements
\begin{center}
  \texttt{\$v x \$.}\\
  \texttt{\$v y \$.}\\
  \texttt{\$v mu \$.}\\
\end{center}
\index{\texttt{\$v} statement}

There are certain rules on where in the database math symbols may be declared,
what sections of the database are aware of them (i.e.\ where they are
``active''), and when they may be declared more than once.  These will be
discussed in Section~\ref{scoping} and specifically on
p.~\pageref{redeclaration}.

\subsection{The \texttt{\$d} Statement}\label{dollard}
\index{\texttt{\$d} statement}

The \texttt{\$d} statement is called a {\bf disjoint-variable restriction}.  The
syntax of the {\bf simple} version of this statement is
\begin{center}
  \texttt{\$d} {\em variable variable} \texttt{\$.}
\end{center}
where each {\em variable} is a previously declared variable and the two {\em
variable}\,s are different.  (More specifically, each  {\em variable} must be
an {\bf active} variable\index{active math symbol}, which means there must be
a previous \texttt{\$v} statement whose {\bf scope}\index{scope} includes the
\texttt{\$d} statement.  These terms will be defined when we discuss scoping
statements in Section~\ref{scoping}.)

In ordinary mathematics, formulas may arise that are true if the variables in
them are distinct\index{distinct variables}, but become false when those
variables are made identical. For example, the formula in logic $\exists x\,x
\neq y$, which means ``for a given $y$, there exists an $x$ that is not equal
to $y$,'' is true in most mathematical theories (namely all non-trivial
theories\index{non-trivial theory}, i.e.\ those that describe more than one
individual, such as arithmetic).  However, if we substitute $y$ with $x$, we
obtain $\exists x\,x \neq x$, which is always false, as it means ``there
exists something that is not equal to itself.''\footnote{If you are a
logician, you will recognize this as the improper substitution\index{proper
substitution}\index{substitution!proper} of a free variable\index{free
variable} with a bound variable\index{bound variable}.  Metamath makes no
inherent distinction between free and bound variables; instead, you let
Metamath know what substitutions are permissible by using \texttt{\$d} statements
in the right way in your axiom system.}\index{free vs.\ bound variable}  The
\texttt{\$d} statement allows you to specify a restriction that forbids the
substitution of one variable with another.  In
this case, we would use the statement
\begin{center}
  \texttt{\$d x y \$.}
\end{center}\index{\texttt{\$d} statement}
to specify this restriction.

The order in which the variables appear in a \texttt{\$d} statement is not
important.  We could also use
\begin{center}
  \texttt{\$d y x \$.}
\end{center}

The \texttt{\$d} statement is actually more general than this, as the
``disjoint''\index{disjoint variables} in its name suggests.  The full meaning
is that if any substitution is made to its two variables (during the
course of a proof that references a \texttt{\$a} or \texttt{\$p} statement
associated with the \texttt{\$d}), the two expressions that result from the
substitution must have no variables in common.  In addition, each possible
pair of variables, one from each expression, must be in a \texttt{\$d} statement
associated with the statement being proved.  (This requirement forces the
statement being proved to ``inherit'' the original disjoint variable
restriction.)

For example, suppose \texttt{u} is a variable.  If the restriction
\begin{center}
  \texttt{\$d A B \$.}
\end{center}
has been specified for a theorem referenced in a
proof, we may not substitute \texttt{A} with \mbox{\tt a + u} and
\texttt{B} with \mbox{\tt b + u} because these two symbol sequences have the
variable \texttt{u} in common.  Furthermore, if \texttt{a} and \texttt{b} are
variables, we may not substitute \texttt{A} with \texttt{a} and \texttt{B} with \texttt{b}
unless we have also specified \texttt{\$d a b} for the theorem being proved; in
other words, the \texttt{\$d} property associated with a pair of variables must
be effectively preserved after substitution.

The \texttt{\$d}\index{\texttt{\$d} statement} statement does {\em not} mean ``the
two variables may not be substituted with the same thing,'' as you might think
at first.  For example, substituting each of \texttt{A} and \texttt{B} in the above
example with identical symbol sequences consisting only of constants does not
cause a disjoint variable conflict, because two symbol sequences have no
variables in common (since they have no variables, period).  Similarly, a
conflict will not occur by substituting the two variables in a \texttt{\$d}
statement with the empty symbol sequence\index{empty substitution}.

The \texttt{\$d} statement does not have a direct counterpart in
ordinary mathematics, partly because the variables\index{variable} of
Metamath are not really the same as the variables\index{variable!in
ordinary mathematics} of ordinary mathematics but rather are
metavariables\index{metavariable} ranging over them (as well as over
other kinds of symbols and groups of symbols).  Depending on the
situation, we may informally interpret the \texttt{\$d} statement in
different ways.  Suppose, for example, that \texttt{x} and \texttt{y}
are variables ranging over numbers (more precisely, that \texttt{x} and
\texttt{y} are metavariables ranging over variables that range over
numbers), and that \texttt{ph} ($\varphi$) and \texttt{ps} ($\psi$) are
variables (more precisely, metavariables) ranging over formulas.  We can
make the following interpretations that correspond to the informal
language of ordinary mathematics:
\begin{quote}
\begin{tabbing}
\texttt{\$d x y \$.} means ``assume $x$ and $y$ are
distinct variables.''\\
\texttt{\$d x ph \$.} means ``assume $x$ does not
occur in $\varphi$.''\\
\texttt{\$d ph ps \$.} \=means ``assume $\varphi$ and
$\psi$ have no variables\\ \>in common.''
\end{tabbing}
\end{quote}\index{\texttt{\$d} statement}

\subsubsection{Compound \texttt{\$d} Statements}

The {\bf compound} version of the \texttt{\$d} statement is a shorthand for
specifying several variables whose substitutions must be pairwise disjoint.
Its syntax is:
\begin{center}
  \texttt{\$d} {\em variable}\ \,$\cdots$\ {\em variable} \texttt{\$.}
\end{center}\index{\texttt{\$d} statement}
Here, {\em variable} represents the token of a previously declared
variable (specifically, an active variable) and all {\em variable}\,s are
different.  The compound \texttt{\$d}
statement is internally broken up by Metamath into one simple \texttt{\$d}
statement for each possible pair of variables in the original \texttt{\$d}
statement.  For example,
\begin{center}
  \texttt{\$d w x y z \$.}
\end{center}
is equivalent to
\begin{center}
  \texttt{\$d w x \$.}\\
  \texttt{\$d w y \$.}\\
  \texttt{\$d w z \$.}\\
  \texttt{\$d x y \$.}\\
  \texttt{\$d x z \$.}\\
  \texttt{\$d y z \$.}
\end{center}

Two or more simple \texttt{\$d} statements specifying the same variable pair are
internally combined into a single \texttt{\$d} statement.  Thus the set of three
statements
\begin{center}
  \texttt{\$d x y \$.}
  \texttt{\$d x y \$.}
  \texttt{\$d y x \$.}
\end{center}
is equivalent to
\begin{center}
  \texttt{\$d x y \$.}
\end{center}

Similarly, compound \texttt{\$d} statements, after being internally broken up,
internally have their common variable pairs combined.  For example the
set of statements
\begin{center}
  \texttt{\$d x y A \$.}
  \texttt{\$d x y B \$.}
\end{center}
is equivalent to
\begin{center}
  \texttt{\$d x y \$.}
  \texttt{\$d x A \$.}
  \texttt{\$d y A \$.}
  \texttt{\$d x y \$.}
  \texttt{\$d x B \$.}
  \texttt{\$d y B \$.}
\end{center}
which is equivalent to
\begin{center}
  \texttt{\$d x y \$.}
  \texttt{\$d x A \$.}
  \texttt{\$d y A \$.}
  \texttt{\$d x B \$.}
  \texttt{\$d y B \$.}
\end{center}

Metamath\index{Metamath} automatically verifies that all \texttt{\$d}
restrictions are met whenever it verifies proofs.  \texttt{\$d} statements are
never referenced directly in proofs (this is why they do not have
labels\index{label}), but Metamath is always aware of which ones must be
satisfied (i.e.\ are active) and will notify you with an error message if any
violation occurs.

To illustrate how Metamath detects a missing \texttt{\$d}
statement, we will look at the following example from the
\texttt{set.mm} database.

\begin{verbatim}
$d x z $.  $d y z $.
$( Theorem to add distinct quantifier to atomic formula. $)
ax17eq $p |- ( x = y -> A. z x = y ) $=...
\end{verbatim}

This statement has the obvious requirement that $z$ must be
distinct\index{distinct variables} from $x$ in theorem \texttt{ax17eq} that
states $x=y \rightarrow \forall z \, x=y$ (well, obvious if you're a logician,
for otherwise we could conclude  $x=y \rightarrow \forall x \, x=y$, which is
false when the free variables $x$ and $y$ are equal.).

Let's look at what happens if we edit the database to comment out this
requirement.

\begin{verbatim}
$( $d x z $. $) $d y z $.
$( Theorem to add distinct quantifier to atomic formula. $)
ax17eq $p |- ( x = y -> A. z x = y ) $=...
\end{verbatim}

When it tries to verify the proof, Metamath will tell you that \texttt{x} and
\texttt{z} must be disjoint, because one of its steps references an axiom or
theorem that has this requirement.

\begin{verbatim}
MM> verify proof ax17eq
ax17eq ?Error at statement 1918, label "ax17eq", type "$p":
      vz wal wi vx vy vz ax-13 vx vy weq vz vx ax-c16 vx vy
                                               ^^^^^
There is a disjoint variable ($d) violation at proof step 29.
Assertion "ax-c16" requires that variables "x" and "y" be
disjoint.  But "x" was substituted with "z" and "y" was
substituted with "x".  The assertion being proved, "ax17eq",
does not require that variables "z" and "x" be disjoint.
\end{verbatim}

We can see the substitutions into \texttt{ax-c16} with the following command.

\begin{verbatim}
MM> show proof ax17eq / detailed_step 29
Proof step 29:  pm2.61dd.2=ax-c16 $a |- ( A. z z = x -> ( x =
  y -> A. z x = y ) )
This step assigns source "ax-c16" ($a) to target "pm2.61dd.2"
($e).  The source assertion requires the hypotheses "wph"
($f, step 26), "vx" ($f, step 27), and "vy" ($f, step 28).
The parent assertion of the target hypothesis is "pm2.61dd"
($p, step 36).
The source assertion before substitution was:
    ax-c16 $a |- ( A. x x = y -> ( ph -> A. x ph ) )
The following substitutions were made to the source
assertion:
    Variable  Substituted with
     x         z
     y         x
     ph        x = y
The target hypothesis before substitution was:
    pm2.61dd.2 $e |- ( ph -> ch )
The following substitutions were made to the target
hypothesis:
    Variable  Substituted with
     ph        A. z z = x
     ch        ( x = y -> A. z x = y )
\end{verbatim}

The disjoint variable restrictions of \texttt{ax-c16} can be seen from the
\texttt{show state\-ment} command.  The line that begins ``\texttt{Its mandatory
dis\-joint var\-i\-able pairs are:}\ldots'' lists any \texttt{\$d} variable
pairs in brackets.

\begin{verbatim}
MM> show statement ax-c16/full
Statement 3033 is located on line 9338 of the file "set.mm".
"Axiom of Distinct Variables. ..."
  ax-c16 $a |- ( A. x x = y -> ( ph -> A. x ph ) ) $.
Its mandatory hypotheses in RPN order are:
  wph $f wff ph $.
  vx $f set x $.
  vy $f set y $.
Its mandatory disjoint variable pairs are:  <x,y>
The statement and its hypotheses require the variables:  x y
      ph
The variables it contains are:  x y ph
\end{verbatim}

Since Metamath will always detect when \texttt{\$d}\index{\texttt{\$d} statement}
statements are needed for a proof, you don't have to worry too much about
forgetting to put one in; it can always be added if you see the error message
above.  If you put in unnecessary \texttt{\$d} statements, the worst that could
happen is that your theorem might not be as general as it could be, and this
may limit its use later on.

On the other hand, when you introduce axioms (\texttt{\$a}\index{\texttt{\$a}
statement} statements), you must be very careful to properly specify the
necessary associated \texttt{\$d} statements since Metamath has no way of knowing
whether your axioms are correct.  For example, Metamath would have no idea
that \texttt{ax-c16}, which we are telling it is an axiom of logic, would lead to
contradictions if we omitted its associated \texttt{\$d} statement.

% This was previously a comment in footnote-sized type, but it can be
% hard to read this much text in a small size.
% As a result, it's been changed to normally-sized text.
\label{nodd}
You may wonder if it is possible to develop standard
mathematics in the Metamath language without the \texttt{\$d}\index{\texttt{\$d}
statement} statement, since it seems like a nuisance that complicates proof
verification. The \texttt{\$d} statement is not needed in certain subsets of
mathematics such as propositional calculus.  However, dummy
variables\index{dummy variable!eliminating} and their associated \texttt{\$d}
statements are impossible to avoid in proofs in standard first-order logic as
well as in the variant used in \texttt{set.mm}.  In fact, there is no upper bound to
the number of dummy variables that might be needed in a proof of a theorem of
first-order logic containing 3 or more variables, as shown by H.\
Andr\'{e}ka\index{Andr{\'{e}}ka, H.} \cite{Nemeti}.  A first-order system that
avoids them entirely is given in \cite{Megill}\index{Megill, Norman}; the
trick there is simply to embed harmlessly the necessary dummy variables into a
theorem being proved so that they aren't ``dummy'' anymore, then interpret the
resulting longer theorem so as to ignore the embedded dummy variables.  If
this interests you, the system in \texttt{set.mm} obtained from \texttt{ax-1}
through \texttt{ax-c14} in \texttt{set.mm}, and deleting \texttt{ax-c16} and \texttt{ax-5},
requires no \texttt{\$d} statements but is logically complete in the sense
described in \cite{Megill}.  This means it can prove any theorem of
first-order logic as long as we add to the theorem an antecedent that embeds
dummy and any other variables that must be distinct.  In a similar fashion,
axioms for set theory can be devised that
do not require distinct variable
provisos\index{Set theory without distinct variable provisos},
as explained at
\url{http://us.metamath.org/mpeuni/mmzfcnd.html}.
Together, these in principle allow all of
mathematics to be developed under Metamath without a \texttt{\$d} statement,
although the length of the resulting theorems will grow as more and
more dummy variables become required in their proofs.

\subsection{The \texttt{\$f}
and \texttt{\$e} Statements}\label{dollaref}
\index{\texttt{\$e} statement}
\index{\texttt{\$f} statement}
\index{floating hypothesis}
\index{essential hypothesis}
\index{variable-type hypothesis}
\index{logical hypothesis}
\index{hypothesis}

Metamath has two kinds of hypo\-theses, the \texttt{\$f}\index{\texttt{\$f}
statement} or {\bf variable-type} hypothesis and the \texttt{\$e} or {\bf logical}
hypo\-the\-sis.\index{\texttt{\$d} statement}\footnote{Strictly speaking, the
\texttt{\$d} statement is also a hypothesis, but it is never directly referenced
in a proof, so we call it a restriction rather than a hypothesis to lessen
confusion.  The checking for violations of \texttt{\$d} restrictions is automatic
and built into Metamath's proof-checking algorithm.} The letters \texttt{f} and
\texttt{e} stand for ``floating''\index{floating hypothesis} (roughly meaning
used only if relevant) and ``essential''\index{essential hypothesis} (meaning
always used) respectively, for reasons that will become apparent
when we discuss frames in
Section~\ref{frames} and scoping in Section~\ref{scoping}. The syntax of these
are as follows:
\begin{center}
  {\em label} \texttt{\$f} {\em typecode} {\em variable} \texttt{\$.}\\
  {\em label} \texttt{\$e} {\em typecode}
      {\em math-symbol}\ \,$\cdots$\ {\em math-symbol} \texttt{\$.}\\
\end{center}
\index{\texttt{\$e} statement}
\index{\texttt{\$f} statement}
A hypothesis must have a {\em label}\index{label}.  The expression in a
\texttt{\$e} hypothesis consists of a typecode (an active constant math symbol)
followed by a sequence
of zero or more math symbols. Each math symbol (including {\em constant}
and {\em variable}) must be a previously declared constant or variable.  (In
addition, each math symbol must be active, which will be covered when we
discuss scoping statements in Section~\ref{scoping}.)  You use a \texttt{\$f}
hypothesis to specify the
nature or {\bf type}\index{variable type}\index{type} of a variable (such as ``let $x$ be an
integer'') and use a \texttt{\$e} hypothesis to express a logical truth (such as
``assume $x$ is prime'') that must be established in order for an assertion
requiring it to also be true.

A variable must have its type specified in a \texttt{\$f} statement before
it may be used in a \texttt{\$e}, \texttt{\$a}, or \texttt{\$p}
statement.  There may be only one (active) \texttt{\$f} statement for a
given variable.  (``Active'' is defined in Section~\ref{scoping}.)

In ordinary mathematics, theorems\index{theorem} are often expressed in the
form ``Assume $P$; then $Q$,'' where $Q$ is a statement that you can derive
if you start with statement $P$.\index{free variable}\footnote{A stronger
version of a theorem like this would be the {\em single} formula $P\rightarrow
Q$ ($P$ implies $Q$) from which the weaker version above follows by the rule
of modus ponens in logic.  We are not discussing this stronger form here.  In
the weaker form, we are saying only that if we can {\em prove} $P$, then we can
{\em prove} $Q$.  In a logician's language, if $x$ is the only free variable
in $P$ and $Q$, the stronger form is equivalent to $\forall x ( P \rightarrow
Q)$ (for all $x$, $P$ implies $Q$), whereas the weaker form is equivalent to
$\forall x P \rightarrow \forall x Q$. The stronger form implies the weaker,
but not vice-versa.  To be precise, the weaker form of the theorem is more
properly called an ``inference'' rather than a theorem.}\index{inference}
In the
Metamath\index{Metamath} language, you would express mathematical statement
$P$ as a hypothesis (a \texttt{\$e} Metamath language statement in this case) and
statement $Q$ as a provable assertion (a \texttt{\$p}\index{\texttt{\$p} statement}
statement).

Some examples of hypotheses you might encounter in logic and set theory are
\begin{center}
  \texttt{stmt1 \$f wff P \$.}\\
  \texttt{stmt2 \$f set x \$.}\\
  \texttt{stmt3 \$e |- ( P -> Q ) \$.}
\end{center}
\index{\texttt{\$e} statement}
\index{\texttt{\$f} statement}
Informally, these would be read, ``Let $P$ be a well-formed-formula,'' ``Let
$x$ be an (individual) variable,'' and ``Assume we have proved $P \rightarrow
Q$.''  The turnstile symbol \,$\vdash$\index{turnstile ({$\,\vdash$})} is
commonly used in logic texts to mean ``a proof exists for.''

To summarize:
\begin{itemize}
\item A \texttt{\$f} hypothesis tells Metamath the type or kind of its variable.
It is analogous to a variable declaration in a computer language that
tells the compiler that a variable is an integer or a floating-point
number.
\item The \texttt{\$e} hypothesis corresponds to what you would usually call a
``hypothesis'' in ordinary mathematics.
\end{itemize}

Before an assertion\index{assertion} (\texttt{\$a} or \texttt{\$p} statement) can be
referenced in a proof, all of its associated \texttt{\$f} and \texttt{\$e} hypotheses
(i.e.\ those \texttt{\$e} hypotheses that are active) must be satisfied (i.e.
established by the proof).  The meaning of ``associated'' (which we will call
{\bf mandatory} in Section~\ref{frames}) will become clear when we discuss
scoping later.

Note that after any \texttt{\$f}, \texttt{\$e},
\texttt{\$a}, or \texttt{\$p} token there is a required
\textit{typecode}\index{typecode}.
The typecode is a constant used to enforce types of expressions.
This will become clearer once we learn more about
assertions (\texttt{\$a} and \texttt{\$p} statements).
An example may also clarify their purpose.
In the
\texttt{set.mm}\index{set theory database (\texttt{set.mm})}%
\index{Metamath Proof Explorer}
database,
the following typecodes are used:

\begin{itemize}
\item \texttt{wff} :
  Well-formed formula (wff) symbol
  (read: ``the following symbol sequence is a wff'').
% The *textual* typecode for turnstile is "|-", but when read it's a little
% confusing, so I intentionally display the mathematical symbol here instead
% (I think it's clearer in this context).
\item \texttt{$\vdash$} :
  Turnstile (read: ``the following symbol sequence is provable'' or
  ``a proof exists for'').
\item \texttt{setvar} :
  Individual set variable type (read: ``the following is an
  individual set variable'').
  Note that this is \textit{not} the type of an arbitrary set expression,
  instead, it is used to ensure that there is only a single symbol used
  after quantifiers like for-all ($\forall$) and there-exists ($\exists$).
\item \texttt{class} :
  An expression that is a syntactically valid class expression.
  All valid set expressions are also valid class expression, so expressions
  of sets normally have the \texttt{class} typecode.
  Use the \texttt{class} typecode,
  \textit{not} the \texttt{setvar} typecode,
  for the type of set expressions unless you are specifically identifying
  a single set variable.
\end{itemize}

\subsection{Assertions (\texttt{\$a} and \texttt{\$p} Statements)}
\index{\texttt{\$a} statement}
\index{\texttt{\$p} statement}\index{assertion}\index{axiomatic assertion}
\index{provable assertion}

There are two types of assertions, \texttt{\$a}\index{\texttt{\$a} statement}
statements ({\bf axiomatic assertions}) and \texttt{\$p} statements ({\bf
provable assertions}).  Their syntax is as follows:
\begin{center}
  {\em label} \texttt{\$a} {\em typecode} {\em math-symbol} \ldots
         {\em math-symbol} \texttt{\$.}\\
  {\em label} \texttt{\$p} {\em typecode} {\em math-symbol} \ldots
        {\em math-symbol} \texttt{\$=} {\em proof} \texttt{\$.}
\end{center}
\index{\texttt{\$a} statement}
\index{\texttt{\$p} statement}
\index{\texttt{\$=} keyword}
An assertion always requires a {\em label}\index{label}. The expression in an
assertion consists of a typecode (an active constant)
followed by a sequence of zero
or more math symbols.  Each math symbol, including any {\em constant}, must be a
previously declared constant or variable.  (In addition, each math symbol
must be active, which will be covered when we discuss scoping statements in
Section~\ref{scoping}.)

A \texttt{\$a} statement is usually a definition of syntax (for example, if $P$
and $Q$ are wffs then so is $(P\to Q)$), an axiom\index{axiom} of ordinary
mathematics (for example, $x=x$), or a definition\index{definition} of
ordinary mathematics (for example, $x\ne y$ means $\lnot x=y$). A \texttt{\$p}
statement is a claim that a certain combination of math symbols follows from
previous assertions and is accompanied by a proof that demonstrates it.

Assertions can also be referenced in (later) proofs in order to derive new
assertions from them. The label of an assertion is used to refer to it in a
proof. Section~\ref{proof} will describe the proof in detail.

Assertions also provide the primary means for communicating the mathematical
results in the database to people.  Proofs (when conveniently displayed)
communicate to people how the results were arrived at.

\subsubsection{The \texttt{\$a} Statement}
\index{\texttt{\$a} statement}

Axiomatic assertions (\texttt{\$a} statements) represent the starting points from
which other assertions (\texttt{\$p}\index{\texttt{\$p} statement} statements) are
derived.  Their most obvious use is for specifying ordinary mathematical
axioms\index{axiom}, but they are also used for two other purposes.

First, Metamath\index{Metamath} needs to know the syntax of symbol
sequences that constitute valid mathematical statements.  A Metamath
proof must be broken down into much more detail than ordinary
mathematical proofs that you may be used to thinking of (even the
``complete'' proofs of formal logic\index{formal logic}).  This is one
of the things that makes Metamath a general-purpose language,
independent of any system of logic or even syntax.  If you want to use a
substitution instance of an assertion as a step in a proof, you must
first prove that the substitution is syntactically correct (or if you
prefer, you must ``construct'' it), showing for example that the
expression you are substituting for a wff metavariable is a valid wff.
The \texttt{\$a}\index{\texttt{\$a} statement} statement is used to
specify those combinations of symbols that are considered syntactically
valid, such as the legal forms of wffs.

Second, \texttt{\$a} statements are used to specify what are ordinarily thought of
as definitions, i.e.\ new combinations of symbols that abbreviate other
combinations of symbols.  Metamath makes no distinction\index{axiom vs.\
definition} between axioms\index{axiom} and definitions\index{definition}.
Indeed, it has been argued that such distinction should not be made even in
ordinary mathematics; see Section~\ref{definitions}, which discusses the
philosophy of definitions.  Section~\ref{hierarchy} discusses some
technical requirements for definitions.  In \texttt{set.mm} we adopt the
convention of prefixing axiom labels with \texttt{ax-} and definition labels with
\texttt{df-}\index{label}.

The results that can be derived with the Metamath language are only as good as
the \texttt{\$a}\index{\texttt{\$a} statement} statements used as their starting
point.  We cannot stress this too strongly.  For example, Metamath will
not prevent you from specifying $x\neq x$ as an axiom of logic.  It is
essential that you scrutinize all \texttt{\$a} statements with great care.
Because they are a source of potential pitfalls, it is best not to add new
ones (usually new definitions) casually; rather you should carefully evaluate
each one's necessity and advantages.

Once you have in place all of the basic axioms\index{axiom} and
rules\index{rule} of a mathematical theory, the only \texttt{\$a} statements that
you will be adding will be what are ordinarily called definitions.  In
principle, definitions should be in some sense eliminable from the language of
a theory according to some convention (usually involving logical equivalence
or equality).  The most common convention is that any formula that was
syntactically valid but not provable before the definition was introduced will
not become provable after the definition is introduced.  In an ideal world,
definitions should not be present at all if one is to have absolute confidence
in a mathematical result.  However, they are necessary to make
mathematics practical, for otherwise the resulting formulas would be
extremely long and incomprehensible.  Since the nature of definitions (in the
most general sense) does not permit them to automatically be verified as
``proper,''\index{proper definition}\index{definition!proper} the judgment of
the mathematician is required to ensure it.  (In \texttt{set.mm} effort was made
to make almost all definitions directly eliminable and thus minimize the need
for such judgment.)

If you are not a mathematician, it may be best not to add or change any
\texttt{\$a}\index{\texttt{\$a} statement} statements but instead use
the mathematical language already provided in standard databases.  This
way Metamath will not allow you to make a mistake (i.e.\ prove a false
result).


\subsection{Frames}\label{frames}

We now introduce the concept of a collection of related Metamath statements
called a frame.  Every assertion (\texttt{\$a} or \texttt{\$p} statement) in the database has
an associated frame.

A {\bf frame}\index{frame} is a sequence of \texttt{\$d}, \texttt{\$f},
and \texttt{\$e} statements (zero or more of each) followed by one
\texttt{\$a} or \texttt{\$p} statement, subject to certain conditions we
will describe.  For simplicity we will assume that all math symbol
tokens used are declared at the beginning of the database with
\texttt{\$c} and \texttt{\$v} statements (which are not properly part of
a frame).  Also for simplicity we will assume there are only simple
\texttt{\$d} statements (those with only two variables) and imagine any
compound \texttt{\$d} statements (those with more than two variables) as
broken up into simple ones.

A frame groups together those hypotheses (and \texttt{\$d} statements) relevant
to an assertion (\texttt{\$a} or \texttt{\$p} statement).  The statements in a frame
may or may not be physically adjacent in a database; we will cover
this in our discussion of scoping statements
in Section~\ref{scoping}.

A frame has the following properties:
\begin{enumerate}
 \item The set of variables contained in its \texttt{\$f} statements must
be identical to the set of variables contained in its \texttt{\$e},
\texttt{\$a}, and/or \texttt{\$p} statements.  In other words, each
variable in a \texttt{\$e}, \texttt{\$a}, or \texttt{\$p} statement must
have an associated ``variable type'' defined for it in a \texttt{\$f}
statement.
  \item No two \texttt{\$f} statements may contain the same variable.
  \item Any \texttt{\$f} statement
must occur before a \texttt{\$e} statement in which its variable occurs.
\end{enumerate}

The first property determines the set of variables occurring in a frame.
These are the {\bf mandatory
variables}\index{mandatory variable} of the frame.  The second property
tells us there must be only one type specified for a variable.
The last property is not a theoretical requirement but it
makes parsing of the database easier.

For our examples, we assume our database has the following declarations:

\begin{verbatim}
$v P Q R $.
$c -> ( ) |- wff $.
\end{verbatim}

The following sequence of statements, describing the modus ponens inference
rule, is an example of a frame:

\begin{verbatim}
wp  $f wff P $.
wq  $f wff Q $.
maj $e |- ( P -> Q ) $.
min $e |- P $.
mp  $a |- Q $.
\end{verbatim}

The following sequence of statements is not a frame because \texttt{R} does not
occur in the \texttt{\$e}'s or the \texttt{\$a}:

\begin{verbatim}
wp  $f wff P $.
wq  $f wff Q $.
wr  $f wff R $.
maj $e |- ( P -> Q ) $.
min $e |- P $.
mp  $a |- Q $.
\end{verbatim}

The following sequence of statements is not a frame because \texttt{Q} does not
occur in a \texttt{\$f}:

\begin{verbatim}
wp  $f wff P $.
maj $e |- ( P -> Q ) $.
min $e |- P $.
mp  $a |- Q $.
\end{verbatim}

The following sequence of statements is not a frame because the \texttt{\$a} statement is
not the last one:

\begin{verbatim}
wp  $f wff P $.
wq  $f wff Q $.
maj $e |- ( P -> Q ) $.
mp  $a |- Q $.
min $e |- P $.
\end{verbatim}

Associated with a frame is a sequence of {\bf mandatory
hypotheses}\index{mandatory hypothesis}.  This is simply the set of all
\texttt{\$f} and \texttt{\$e} statements in the frame, in the order they
appear.  A frame can be referenced in a later proof using the label of
the \texttt{\$a} or \texttt{\$p} assertion statement, and the proof
makes an assignment to each mandatory hypothesis in the order in which
it appears.  This means the order of the hypotheses, once chosen, must
not be changed so as not to affect later proofs referencing the frame's
assertion statement.  (The Metamath proof verifier will, of course, flag
an error if a proof becomes incorrect by doing this.)  Since proofs make
use of ``Reverse Polish notation,'' described in Section~\ref{proof}, we
call this order the {\bf RPN order}\index{RPN order} of the hypotheses.

Note that \texttt{\$d} statements are not part of the set of mandatory
hypotheses, and their order doesn't matter (as long as they satisfy the
fourth property for a frame described above).  The \texttt{\$d}
statements specify restrictions on variables that must be satisfied (and
are checked by the proof verifier) when expressions are substituted for
them in a proof, and the \texttt{\$d} statements themselves are never
referenced directly in a proof.

A frame with a \texttt{\$p} (provable) statement requires a proof as part of the
\texttt{\$p} statement.  Sometimes in a proof we want to make use of temporary or
dummy variables\index{dummy variable} that do not occur in the \texttt{\$p}
statement or its mandatory hypotheses.  To accommodate this we define an {\bf
extended frame}\index{extended frame} as a frame together with zero or more
\texttt{\$d} and \texttt{\$f} statements that reference variables not among the
mandatory variables of the frame.  Any new variables referenced are called the
{\bf optional variables}\index{optional variable} of the extended frame. If a
\texttt{\$f} statement references an optional variable it is called an {\bf
optional hypothesis}\index{optional hypothesis}, and if one or both of the
variables in a \texttt{\$d} statement are optional variables it is called an {\bf
optional disjoint-variable restriction}\index{optional disjoint-variable
restriction}.  Properties 2, 3, and 4 for a frame also apply to an extended
frame.

The concept of optional variables is not meaningful for frames with \texttt{\$a}
statements, since those statements have no proofs that might make use of them.
There is no restriction on including optional hypotheses in the extended frame
for a \texttt{\$a} statement, but they serve no purpose.

The following set of statements is an example of an extended frame, which
contains an optional variable \texttt{R} and an optional hypothesis \texttt{wr}.  In
this example, we suppose the rule of modus ponens is not an axiom but is
derived as a theorem from earlier statements (we omit its presumed proof).
Variable \texttt{R} may be used in its proof if desired (although this would
probably have no advantage in propositional calculus).  Note that the sequence
of mandatory hypotheses in RPN order is still \texttt{wp}, \texttt{wq}, \texttt{maj},
\texttt{min} (i.e.\ \texttt{wr} is omitted), and this sequence is still assumed
whenever the assertion \texttt{mp} is referenced in a subsequent proof.

\begin{verbatim}
wp  $f wff P $.
wq  $f wff Q $.
wr  $f wff R $.
maj $e |- ( P -> Q ) $.
min $e |- P $.
mp  $p |- Q $= ... $.
\end{verbatim}

Every frame is an extended frame, but not every extended frame is a frame, as
this example shows.  The underlying frame for an extended frame is
obtained by simply removing all statements containing optional variables.
Any proof referencing an assertion will ignore any extensions to its
frame, which means we may add or delete optional hypotheses at will without
affecting subsequent proofs.

The conceptually simplest way of organizing a Metamath database is as a
sequence of extended frames.  The scoping statements
\texttt{\$\char`\{}\index{\texttt{\$\char`\{} and \texttt{\$\char`\}}
keywords} and \texttt{\$\char`\}} can be used to delimit the start and
end of an extended frame, leading to the following possible structure for a
database.  \label{framelist}

\vskip 2ex
\setbox\startprefix=\hbox{\tt \ \ \ \ \ \ \ \ }
\setbox\contprefix=\hbox{}
\startm
\m{\mbox{(\texttt{\$v} {\em and} \texttt{\$c}\,{\em statements})}}
\endm
\startm
\m{\mbox{\texttt{\$\char`\{}}}
\endm
\startm
\m{\mbox{\texttt{\ \ } {\em extended frame}}}
\endm
\startm
\m{\mbox{\texttt{\$\char`\}}}}
\endm
\startm
\m{\mbox{\texttt{\$\char`\{}}}
\endm
\startm
\m{\mbox{\texttt{\ \ } {\em extended frame}}}
\endm
\startm
\m{\mbox{\texttt{\$\char`\}}}}
\endm
\startm
\m{\mbox{\texttt{\ \ \ \ \ \ \ \ \ }}\vdots}
\endm
\vskip 2ex

In practice, this structure is inconvenient because we have to repeat
any \texttt{\$f}, \texttt{\$e}, and \texttt{\$d} statements over and
over again rather than stating them once for use by several assertions.
The scoping statements, which we will discuss next, allow this to be
done.  In principle, any Metamath database can be converted to the above
format, and the above format is the most convenient to use when studying
a Metamath database as a formal system%
%% Uncomment this when uncommenting section {formalspec} below
   (Appendix \ref{formalspec})%
.
In fact, Metamath internally converts the database to the above format.
The command \texttt{show statement} in the Metamath program will show
you the contents of the frame for any \texttt{\$a} or \texttt{\$p}
statement, as well as its extension in the case of a \texttt{\$p}
statement.

%c%(provided that all ``local'' variables and constants with limited scope have
%c%unique names),

During our discussion of scoping statements, it may be helpful to
think in terms of the equivalent sequence of frames that will result when
the database is parsed.  Scoping (other than the limited
use above to delimit frames) is not a theoretical requirement for
Metamath but makes it more convenient.


\subsection{Scoping Statements (\texttt{\$\{} and \texttt{\$\}})}\label{scoping}
\index{\texttt{\$\char`\{} and \texttt{\$\char`\}} keywords}\index{scoping statement}

%c%Some Metamath statements may be needed only temporarily to
%c%serve a specific purpose, and after we're done with them we would like to
%c%disregard or ignore them.  For example, when we're finished using a variable,
%c%we might want to
%c%we might want to free up the token\index{token} used to name it so that the
%c%token can be used for other purposes later on, such as a different kind of
%c%variable or even a constant.  In the terminology of computer programming, we
%c%might want to let some symbol declarations be ``local'' rather than ``global.''
%c%\index{local symbol}\index{global symbol}

The {\bf scoping} statements, \texttt{\$\char`\{} ({\bf start of block}) and \texttt{\$\char`\}}
({\bf end of block})\index{block}, provide a means for controlling the portion
of a database over which certain statement types are recognized.  The
syntax of a scoping statement is very simple; it just consists of the
statement's keyword:
\begin{center}
\texttt{\$\char`\{}\\
\texttt{\$\char`\}}
\end{center}
\index{\texttt{\$\char`\{} and \texttt{\$\char`\}} keywords}

For example, consider the following database where we have stripped out
all tokens except the scoping statement keywords.  For the purpose of the
discussion, we have added subscripts to the scoping statements; these subscripts
do not appear in the actual database.
\[
 \mbox{\tt \ \$\char`\{}_1
 \mbox{\tt \ \$\char`\{}_2
 \mbox{\tt \ \$\char`\}}_2
 \mbox{\tt \ \$\char`\{}_3
 \mbox{\tt \ \$\char`\{}_4
 \mbox{\tt \ \$\char`\}}_4
 \mbox{\tt \ \$\char`\}}_3
 \mbox{\tt \ \$\char`\}}_1
\]
Each \texttt{\$\char`\{} statement in this example is said to be {\bf
matched} with the \texttt{\$\char`\}} statement that has the same
subscript.  Each pair of matched scoping statements defines a region of
the database called a {\bf block}.\index{block} Blocks can be {\bf
nested}\index{nested block} inside other blocks; in the example, the
block defined by $\mbox{\tt \$\char`\{}_4$ and $\mbox{\tt \$\char`\}}_4$
is nested inside the block defined by $\mbox{\tt \$\char`\{}_3$ and
$\mbox{\tt \$\char`\}}_3$ as well as inside the block defined by
$\mbox{\tt \$\char`\{}_1$ and $\mbox{\tt \$\char`\}}_1$.  In general, a
block may be empty, it may contain only non-scoping
statements,\footnote{Those statements other than \texttt{\$\char`\{} and
\texttt{\$\char`\}}.}\index{non-scoping statement} or it may contain any
mixture of other blocks and non-scoping statements.  (This is called a
``recursive'' definition\index{recursive definition} of a block.)

Associated with each block is a number called its {\bf nesting
level}\index{nesting level} that indicates how deeply the block is nested.
The nesting levels of the blocks in our example are as follows:
\[
  \underbrace{
    \mbox{\tt \ }
    \underbrace{
     \mbox{\tt \$\char`\{\ }
     \underbrace{
       \mbox{\tt \$\char`\{\ }
       \mbox{\tt \$\char`\}}
     }_{2}
     \mbox{\tt \ }
     \underbrace{
       \mbox{\tt \$\char`\{\ }
       \underbrace{
         \mbox{\tt \$\char`\{\ }
         \mbox{\tt \$\char`\}}
       }_{3}
       \mbox{\tt \ \$\char`\}}
     }_{2}
     \mbox{\tt \ \$\char`\}}
   }_{1}
   \mbox{\tt \ }
 }_{0}
\]
\index{\texttt{\$\char`\{} and \texttt{\$\char`\}} keywords}
The entire database is considered to be one big block (the {\bf outermost}
block) with a nesting level of 0.  The outermost block is {\em not} bracketed
by scoping statements.\footnote{The language was designed this way so that
several source files can be joined together more easily.}\index{outermost
block}

All non-scoping Metamath statements become recognized or {\bf
active}\index{active statement} at the place where they appear.\footnote{To
keep things slightly simpler, we do not bother to define the concept of
``active'' for the scoping statements.}  Certain of these statement types
become inactive at the end of the block in which they appear; these statement
types are:
\begin{center}
  \texttt{\$c}, \texttt{\$v}, \texttt{\$d}, \texttt{\$e}, and \texttt{\$f}.
%  \texttt{\$v}, \texttt{\$f}, \texttt{\$e}, and \texttt{\$d}.
\end{center}
\index{\texttt{\$c} statement}
\index{\texttt{\$d} statement}
\index{\texttt{\$e} statement}
\index{\texttt{\$f} statement}
\index{\texttt{\$v} statement}
The other statement types remain active forever (i.e.\ through the end of the
database); they are:
\begin{center}
  \texttt{\$a} and \texttt{\$p}.
%  \texttt{\$c}, \texttt{\$a}, and \texttt{\$p}.
\end{center}
\index{\texttt{\$a} statement}
\index{\texttt{\$p} statement}
Any statement (of these 7 types) located in the outermost
block\index{outermost block} will remain active through the end of the
database and thus are effectively ``global'' statements.\index{global
statement}

All \texttt{\$c} statements must be placed in the outermost block.  Since they are
therefore always global, they could be considered as belonging to both of the
above categories.

The {\bf scope}\index{scope} of a statement is the set of statements that
recognize it as active.

%c%The concept of ``active'' is also defined for math symbols\index{math
%c%symbol}.  Math symbols (constants\index{constant} and
%c%variables\index{variable}) become {\bf active}\index{active
%c%math symbol} in the \texttt{\$c}\index{\texttt{\$c}
%c%statement} and \texttt{\$v}\index{\texttt{\$v} statement} statements that
%c%declare them.  They become inactive when their declaration statements become
%c%inactive.

The concept of ``active'' is also defined for math symbols\index{math
symbol}.  Math symbols (constants\index{constant} and
variables\index{variable}) become {\bf active}\index{active math symbol}
in the \texttt{\$c}\index{\texttt{\$c} statement} and
\texttt{\$v}\index{\texttt{\$v} statement} statements that declare them.
A variable becomes inactive when its declaration statement becomes
inactive.  Because all \texttt{\$c} statements must be in the outermost
block, a constant will never become inactive after it is declared.

\subsubsection{Redeclaration of Math Symbols}
\index{redeclaration of symbols}\label{redeclaration}

%c%A math symbol may not be declared a second time while it is active, but it may
%c%be declared again after it becomes inactive.

A variable may not be declared a second time while it is active, but it may be
declared again after it becomes inactive.  This provides a convenient way to
introduce ``local'' variables,\index{local variable} i.e.\ temporary variables
for use in the frame of an assertion or in a proof without keeping them around
forever.  A previously declared variable may not be redeclared as a constant.

A constant may not be redeclared.  And, as mentioned above, constants must be
declared in the outermost block.

The reason variables may have limited scope but not constants is that an
assertion (\texttt{\$a} or \texttt{\$p} statement) remains available for use in
proofs through the end of the database.  Variables in an assertion's frame may
be substituted with whatever is needed in a proof step that references the
assertion, whereas constants remain fixed and may not be substituted with
anything.  The particular token used for a variable in an assertion's frame is
irrelevant when the assertion is referenced in a proof, and it doesn't matter
if that token is not available outside of the referenced assertion's frame.
Constants, however, must be globally fixed.

In the present version of the Metamath language, there is no theoretical
benefit for the feature allowing variables to be active for limited scopes
rather than global. It is just a convenience that allows them, for example, to
be locally grouped together with their corresponding \texttt{\$f} variable-type
declarations.\footnote{A future version\label{redeclarationf} of the Metamath
language may extend the language to include true local variables as well as
local constants.  I have not decided upon the best way to do this or even if
it should be done at all; the main issue is whether the benefits offset
greater confusion in learning the language.  In any case, the rules for
the present language were chosen to let existing databases be compatible with
any future language extensions.}

%c%If you declare a math symbol more than once, internally Metamath considers it a
%c%new distinct symbol, even though it has the same name.  If you are unaware of
%c%this, you may find that what you think are correct proofs are incorrectly
%c%rejected as invalid, because Metamath may tell you that a constant you
%c%previously declared does not match a newly declared math symbol with the same
%c%name.  For details on this subtle point, see the Comment on
%c%p.~\pageref{spec4comment}.  This is done purposely to allow temporary
%c%constants to be introduced while developing a subtheory, then allow their math
%c%symbol tokens to be reused later on; in general they will not refer to the
%c%same thing.  In practice, you would not ordinarily reuse the names of
%c%constants because it would tend to be confusing to the reader.  The reuse of
%c%names of variables, on the other hand, is something that is often useful to do
%c%(for example it is done frequently in \texttt{set.mm}).  Since variables in an
%c%assertion referenced in a proof can be substituted as needed to achieve a
%c%symbol match, this is not an issue.

% (This section covers a somewhat advanced topic you may want to skip
% at first reading.)
%
% Under certain circumstances, math symbol\index{math symbol}
% tokens\index{token} may be redeclared (i.e.\ the token
% may appear in more than
% one \texttt{\$c}\index{\texttt{\$c} statement} or \texttt{\$v}\index{\texttt{\$v}
% statement} statement).  You might want to do this say, to make temporary use
% of a variable name without having to worry about its affect elsewhere,
% somewhat analogous to declaring a local variable in a standard computer
% language.  Understanding what goes on when math symbol tokens are redeclared
% is a little tricky to understand at first, since it requires that we
% distinguish the token itself from the math symbol that it names.  It will help
% if we first take a peek at the internal workings of the
% Metamath\index{Metamath} program.
%
% Metamath reserves a memory location for each occurrence of a
% token\index{token} in a declaration statement (\texttt{\$c}\index{\texttt{\$c}
% statement} or \texttt{\$v}\index{\texttt{\$v} statement}).  If a given token appears
% in more than one declaration statement, it will refer to more than one memory
% locations.  A math symbol\index{math symbol} may be thought of as being one of
% these memory locations rather than as the token itself.  Only one of the
% memory locations associated with a given token may be active at any one time.
% The math symbol (memory location) that gets looked up when the token appears
% in a non-declaration statement is the one that happens to be active at that
% time.
%
% We now look at the rules for the redeclaration\index{redeclaration of symbols}
% of math symbol tokens.
% \begin{itemize}
% \item A math symbol token may not be declared twice in the
% same block.\footnote{While there is no theoretical reason for disallowing
% this, it was decided in the design of Metamath that allowing it would offer no
% advantage and might cause confusion.}
% \item An inactive math symbol may always be
% redeclared.
% \item  An active math symbol may be redeclared in a different (i.e.\
% inner) block\index{block} from the one it became active in.
% \end{itemize}
%
% When a math symbol token is redeclared, it conceptually refers to a different
% math symbol, just as it would be if it were called a different name.  In
% addition, the original math symbol that it referred to, if it was active,
% temporarily becomes inactive.  At the end of the block in which the
% redeclaration occurred, the new math symbol\index{math symbol} becomes
% inactive and the original symbol becomes active again.  This concept is
% illustrated in the following example, where the symbol \texttt{e} is
% ordinarily a constant (say Euler's constant, 2.71828...) but
% temporarily we want to use it as a ``local'' variable, say as a coefficient
% in the equation $a x^4 + b x^3 + c x^2 + d x + e$:
% \[
%   \mbox{\tt \$\char`\{\ \$c e \$.}
%   \underbrace{
%     \ \ldots\ %
%     \mbox{\tt \$\char`\{}\ \ldots\ %
%   }_{\mbox{\rm region A}}
%   \mbox{\tt \$v e \$.}
%   \underbrace{
%     \mbox{\ \ \ \ldots\ \ \ }
%   }_{\mbox{\rm region B}}
%   \mbox{\tt \$\char`\}}
%   \underbrace{
%     \mbox{\ \ \ \ldots\ \ \ }
%   }_{\mbox{\rm region C}}
%   \mbox{\tt \$\char`\}}
% \]
% \index{\texttt{\$\char`\{} and \texttt{\$\char`\}} keywords}
% In region A, the token \texttt{e} refers to a constant.  It is redeclared as a
% variable in region B, and any reference to it in this region will refer to this
% variable.  In region C, the redeclaration becomes inactive, and the original
% declaration becomes active again.  In region C, the token \texttt{x} refers to the
% original constant.
%
% As a practical matter, overuse of math symbol\index{math symbol}
% redeclarations\index{redeclaration of symbols} can be confusing (even though
% it is well-defined) and is best avoided when possible.  Here are some good
% general guidelines you can follow.  Usually, you should declare all
% constants\index{constant} in the outermost block\index{outermost block},
% especially if they are general-purpose (such as the token \verb$A.$, meaning
% $\forall$ or ``for all'').  This will make them ``globally'' active (although
% as in the example above local redeclarations will temporarily make them
% inactive.)  Most or all variables\index{variable}, on the other hand, could be
% declared in inner blocks, so that the token for them can be used later for a
% different type of variable or a constant.  (The names of the variables you
% choose are not used when you refer to an assertion\index{assertion} in a
% proof, whereas constants must match exactly.  A locally declared constant will
% not match a globally declared constant in a proof, even if they use the same
% token, because Metamath internally considers them to be different math
% symbols.)  To avoid confusion, you should generally avoid redeclaring active
% variables.  If you must redeclare them, do so at the beginning of a block.
% The temporary declaration of constants in inner blocks might be occasionally
% appropriate when you make use of a temporary definition to prove lemmas
% leading to a main result that does not make direct use of the definition.
% This way, you will not clutter up your database with a large number of
% seldom-used global constant symbols.  You might want to note that while
% inactive constants may not appear directly in an assertion (a \texttt{\$a}\index{\texttt{\$a}
% statement} or \texttt{\$p}\index{\texttt{\$p} statement}
% statement), they may be indirectly used in the proof of a \texttt{\$p} statement
% so long as they do not appear in the final math symbol sequence constructed by
% the proof.  In the end, you will have to use your best judgment, taking into
% account standard mathematical usage of the symbols as well as consideration
% for the reader of your work.
%
% \subsubsection{Reuse of Labels}\index{reuse of labels}\index{label}
%
% The \texttt{\$e}\index{\texttt{\$e} statement}, \texttt{\$f}\index{\texttt{\$f}
% statement}, \texttt{\$a}\index{\texttt{\$a} statement}, and
% \texttt{\$p}\index{\texttt{\$p}
% statement} statement types require labels, which allow them to be
% referenced later inside proofs.  A label is considered {\bf
% active}\index{active label} when the statement it is associated with is
% active.  The token\index{token} for a label may be reused
% (redeclared)\index{redeclaration of labels} provided that it is not being used
% for a currently active label.  (Unlike the tokens for math symbols, active
% label tokens may not be redeclared in an inner scope.)  Note that the labels
% of \texttt{\$a} and \texttt{\$p} statements can never be reused after these
% statements appear, because these statements remain active through the end of
% the database.
%
% You might find the reuse of labels a convenient way to have standard names for
% temporary hypotheses, such as \texttt{h1}, \texttt{h2}, etc.  This way you don't have
% to invent unique names for each of them, and in some cases it may be less
% confusing to the reader (although in other cases it might be more confusing, if
% the hypothesis is located far away from the assertion that uses
% it).\footnote{The current implementation requires that all labels, even
% inactive ones, be unique.}

\subsubsection{Frames Revisited}\index{frames and scoping statements}

Now that we have covered scoping, we will look at how an arbitrary
Metamath database can be converted to the simple sequence of extended
frames described on p.~\pageref{framelist}.  This is also how Metamath
stores the database internally when it reads in the database
source.\label{frameconvert} The method is simple.  First, we collect all
constant and variable (\texttt{\$c} and \texttt{\$v}) declarations in
the database, ignoring duplicate declarations of the same variable in
different scopes.  We then put our collected \texttt{\$c} and
\texttt{\$v} declarations at the beginning of the database, so that
their scope is the entire database.  Next, for each assertion in the
database, we determine its frame and extended frame.  The extended frame
is simply the \texttt{\$f}, \texttt{\$e}, and \texttt{\$d} statements
that are active.  The frame is the extended frame with all optional
hypotheses removed.

An equivalent way of saying this is that the extended frame of an assertion
is the collection of all \texttt{\$f}, \texttt{\$e}, and \texttt{\$d} statements
whose scope includes the assertion.
The \texttt{\$f} and \texttt{\$e} statements
occur in the order they appear
(order is irrelevant for \texttt{\$d} statements).

%c%, renaming any
%c%redeclared variables as needed so that all of them have unique names.  (The
%c%exact renaming convention is unimportant.  You might imagine renaming
%c%different declarations of math symbol \texttt{a} as \texttt{a\$1}, \texttt{a\$2}, etc.\
%c%which would prevent any conflicts since \texttt{\$} is not a legal character in a
%c%math symbol token.)

\section{The Anatomy of a Proof} \label{proof}
\index{proof!Metamath, description of}

Each provable assertion (\texttt{\$p}\index{\texttt{\$p} statement} statement) in a
database must include a {\bf proof}\index{proof}.  The proof is located
between the \texttt{\$=}\index{\texttt{\$=} keyword} and \texttt{\$.}\ keywords in the
\texttt{\$p} statement.

In the basic Metamath language\index{basic language}, a proof is a
sequence of statement labels.  This label sequence\index{label sequence}
serves as a set of instructions that the Metamath program uses to
construct a series of math symbol sequences.  The construction must
ultimately result in the math symbol sequence contained between the
\texttt{\$p}\index{\texttt{\$p} statement} and
\texttt{\$=}\index{\texttt{\$=} keyword} keywords of the \texttt{\$p}
statement.  Otherwise, the Metamath program will consider the proof
incorrect, and it will notify you with an appropriate error message when
you ask it to verify the proof.\footnote{To make the loading faster, the
Metamath program does not automatically verify proofs when you
\texttt{read} in a database unless you use the \texttt{/verify}
qualifier.  After a database has been read in, you may use the
\texttt{verify proof *} command to verify proofs.}\index{\texttt{verify
proof} command} Each label in a proof is said to {\bf
reference}\index{label reference} its corresponding statement.

Associated with any assertion\index{assertion} (\texttt{\$p} or
\texttt{\$a}\index{\texttt{\$a} statement} statement) is a set of
hypotheses (\texttt{\$f}\index{\texttt{\$f} statement} or
\texttt{\$e}\index{\texttt{\$e} statement} statements) that are active
with respect to that assertion.  Some are mandatory and the others are
optional.  You should review these concepts if necessary.

Each label\index{label} in a proof must be either the label of a
previous assertion (\texttt{\$a}\index{\texttt{\$a} statement} or
\texttt{\$p}\index{\texttt{\$p} statement} statement) or the label of an
active hypothesis (\texttt{\$e} or \texttt{\$f}\index{\texttt{\$f}
statement} statement) of the \texttt{\$p} statement containing the
proof.  Hypothesis labels may reference both the
mandatory\index{mandatory hypothesis} and the optional hypotheses of the
\texttt{\$p} statement.

The label sequence in a proof specifies a construction in {\bf reverse Polish
notation}\index{reverse Polish notation (RPN)} (RPN).  You may be familiar
with RPN if you have used older
Hewlett--Packard or similar hand-held calculators.
In the calculator analogy, a hypothesis label\index{hypothesis label} is like
a number and an assertion label\index{assertion label} is like an operation
(more precisely, an $n$-ary operation when the
assertion has $n$ \texttt{\$e}-hypotheses).
On an RPN calculator, an operation takes one or more previous numbers in an
input sequence, performs a calculation on them, and replaces those numbers and
itself with the result of the calculation.  For example, the input sequence
$2,3,+$ on an RPN calculator results in $5$, and the input sequence
$2,3,5,{\times},+$ results in $2,15,+$ which results in $17$.

Understanding how RPN is processed involves the concept of a {\bf
stack}\index{stack}\index{RPN stack}, which can be thought of as a set of
temporary memory locations that hold intermediate results.  When Metamath
encounters a hypothesis label it places or {\bf pushes}\index{push} the math
symbol sequence of the hypothesis onto the stack.  When Metamath encounters an
assertion label, it associates the most recent stack entries with the {\em
mandatory} hypotheses\index{mandatory hypothesis} of the assertion, in the
order where the most recent stack entry is associated with the last mandatory
hypothesis of the assertion.  It then determines what
substitutions\index{substitution!variable}\index{variable substitution} have
to be made into the variables of the assertion's mandatory hypotheses to make
them identical to the associated stack entries.  It then makes those same
substitutions into the assertion itself.  Finally, Metamath removes or {\bf
pops}\index{pop} the matched hypotheses from the stack and pushes the
substituted assertion onto the stack.

For the purpose of matching the mandatory hypothesis to the most recent stack
entries, whether a hypothesis is a \texttt{\$e} or \texttt{\$f} statement is
irrelevant.  The only important thing is that a set of
substitutions\footnote{In the Metamath spec (Section~\ref{spec}), we use the
singular term ``substitution'' to refer to the set of substitutions we talk
about here.} exist that allow a match (and if they don't, the proof verifier
will let you know with an error message).  The Metamath language is specified
in such a way that if a set of substitutions exists, it will be unique.
Specifically, the requirement that each variable have a type specified for it
with a \texttt{\$f} statement ensures the uniqueness.

We will illustrate this with an example.
Consider the following Metamath source file:
\begin{verbatim}
$c ( ) -> wff $.
$v p q r s $.
wp $f wff p $.
wq $f wff q $.
wr $f wff r $.
ws $f wff s $.
w2 $a wff ( p -> q ) $.
wnew $p wff ( s -> ( r -> p ) ) $= ws wr wp w2 w2 $.
\end{verbatim}
This Metamath source example shows the definition and ``proof'' (i.e.,
construction) of a well-formed formula (wff)\index{well-formed formula (wff)}
in propositional calculus.  (You may wish to type this example into a file to
experiment with the Metamath program.)  The first two statements declare
(introduce the names of) four constants and four variables.  The next four
statements specify the variable types, namely that
each variable is assumed to be a wff.  Statement \texttt{w2} defines (postulates)
a way to produce a new wff, \texttt{( p -> q )}, from two given wffs \texttt{p} and
\texttt{q}. The mandatory hypotheses of \texttt{w2} are \texttt{wp} and \texttt{wq}.
Statement \texttt{wnew} claims that \texttt{( s -> ( r -> p ) )} is a wff given
three wffs \texttt{s}, \texttt{r}, and \texttt{p}.  More precisely, \texttt{wnew} claims
that the sequence of ten symbols \texttt{wff ( s -> ( r -> p ) )} is provable from
previous assertions and the hypotheses of \texttt{wnew}.  Metamath does not know
or care what a wff is, and as far as it is concerned
the typecode \texttt{wff} is just an
arbitrary constant symbol in a math symbol sequence.  The mandatory hypotheses
of \texttt{wnew} are \texttt{wp}, \texttt{wr}, and \texttt{ws}; \texttt{wq} is an optional
hypothesis.  In our particular proof, the optional hypothesis is not
referenced, but in general, any combination of active (i.e.\ optional and
mandatory) hypotheses could be referenced.  The proof of statement \texttt{wnew}
is the sequence of five labels starting with \texttt{ws} (step~1) and ending with
\texttt{w2} (step~5).

When Metamath verifies the proof, it scans the proof from left to right.  We
will examine what happens at each step of the proof.  The stack starts off
empty.  At step 1, Metamath looks up label \texttt{ws} and determines that it is a
hypothesis, so it pushes the symbol sequence of statement \texttt{ws} onto the
stack:

\begin{center}\begin{tabular}{|l|l|}\hline
{Stack location} & {Contents} \\ \hline \hline
1 & \texttt{wff s} \\ \hline
\end{tabular}\end{center}

Metamath sees that the labels \texttt{wr} and \texttt{wp} in steps~2 and 3 are also
hypotheses, so it pushes them onto the stack.  After step~3, the stack looks
like
this:

\begin{center}\begin{tabular}{|l|l|}\hline
{Stack location} & {Contents} \\ \hline \hline
3 & \texttt{wff p} \\ \hline
2 & \texttt{wff r} \\ \hline
1 & \texttt{wff s} \\ \hline
\end{tabular}\end{center}

At step 4, Metamath sees that label \texttt{w2} is an assertion, so it must do
some processing.  First, it associates the mandatory hypotheses of \texttt{w2},
which are \texttt{wp} and \texttt{wq}, with stack locations~2 and 3, {\em in that
order}. Metamath determines that the only possible way
to make hypothesis \texttt{wp} match (become identical to) stack location~2 and
\texttt{wq} match stack location 3 is to substitute variable \texttt{p} with \texttt{r}
and \texttt{q} with \texttt{p}.  Metamath makes these substitutions into \texttt{w2} and
obtains the symbol sequence \texttt{wff ( r -> p )}.  It removes the hypotheses
from stack locations~2 and 3, then places the result into stack location~2:

\begin{center}\begin{tabular}{|l|l|}\hline
{Stack location} & {Contents} \\ \hline \hline
2 & \texttt{wff ( r -> p )} \\ \hline
1 & \texttt{wff s} \\ \hline
\end{tabular}\end{center}

At step 5, Metamath sees that label \texttt{w2} is an assertion, so it must again
do some processing.  First, it matches the mandatory hypotheses of \texttt{w2},
which are \texttt{wp} and \texttt{wq}, to stack locations 1 and 2.
Metamath determines that the only possible way to make the
hypotheses match is to substitute variable \texttt{p} with \texttt{s} and \texttt{q} with
\texttt{( r -> p )}.  Metamath makes these substitutions into \texttt{w2} and obtains
the symbol
sequence \texttt{wff ( s -> ( r -> p ) )}.  It removes stack
locations 1 and 2, then places the result into stack location~1:

\begin{center}\begin{tabular}{|l|l|}\hline
{Stack location} & {Contents} \\ \hline \hline
1 & \texttt{wff ( s -> ( r -> p ) )} \\ \hline
\end{tabular}\end{center}

After Metamath finishes processing the proof, it checks to see that the
stack contains exactly one element and that this element is
the same as the math symbol sequence in the
\texttt{\$p}\index{\texttt{\$p} statement} statement.  This is the case for our
proof of \texttt{wnew},
so we have proved \texttt{wnew} successfully.  If the result
differs, Metamath will notify you with an error message.  An error message
will also result if the stack contains more than one entry at the end of the
proof, or if the stack did not contain enough entries at any point in the
proof to match all of the mandatory hypotheses\index{mandatory hypothesis} of
an assertion.  Finally, Metamath will notify you with an error message if no
substitution is possible that will make a referenced assertion's hypothesis
match the
stack entries.  You may want to experiment with the different kinds of errors
that Metamath will detect by making some small changes in the proof of our
example.

Metamath's proof notation was designed primarily to express proofs in a
relatively compact manner, not for readability by humans.  Metamath can display
proofs in a number of different ways with the \texttt{show proof}\index{\texttt{show
proof} command} command.  The
\texttt{/lemmon} qualifier displays it in a format that is easier to read when the
proofs are short, and you saw examples of its use in Chapter~\ref{using}.  For
longer proofs, it is useful to see the tree structure of the proof.  A tree
structure is displayed when the \texttt{/lemmon} qualifier is omitted.  You will
probably find this display more convenient as you get used to it. The tree
display of the proof in our example looks like
this:\label{treeproof}\index{tree-style proof}\index{proof!tree-style}
\begin{verbatim}
1     wp=ws    $f wff s
2        wp=wr    $f wff r
3        wq=wp    $f wff p
4     wq=w2    $a wff ( r -> p )
5  wnew=w2  $a wff ( s -> ( r -> p ) )
\end{verbatim}
The number to the left of each line is the step number.  Following it is a
{\bf hypothesis association}\index{hypothesis association}, consisting of two
labels\index{label} separated by \texttt{=}.  To the left of the \texttt{=} (except
in the last step) is the label of a hypothesis of an assertion referenced
later in the proof; here, steps 1 and 4 are the hypothesis associations for
the assertion \texttt{w2} that is referenced in step 5.  A hypothesis association
is indented one level more than the assertion that uses it, so it is easy to
find the corresponding assertion by moving directly down until the indentation
level decreases to one less than where you started from.  To the right of each
\texttt{=} is the proof step label for that proof step.  The statement keyword of
the proof step label is listed next, followed by the content of the top of the
stack (the most recent stack entry) as it exists after that proof step is
processed.  With a little practice, you should have no trouble reading proofs
displayed in this format.

Metamath proofs include the syntax construction of a formula.
In standard mathematics, this kind of
construction is not considered a proper part of the proof at all, and it
certainly becomes rather boring after a while.
Therefore,
by default the \texttt{show proof}\index{\texttt{show proof}
command} command does not show the syntax construction.
Historically \texttt{show proof} command
\textit{did} show the syntax construction, and you needed to add the
\texttt{/essential} option to hide, them, but today
\texttt{/essential} is the default and you need to use
\texttt{/all} to see the syntax constructions.

When verifying a proof, Metamath will check that no mandatory
\texttt{\$d}\index{\texttt{\$d} statement}\index{mandatory \texttt{\$d}
statement} statement of an assertion referenced in a proof is violated
when substitutions\index{substitution!variable}\index{variable
substitution} are made to the variables in the assertion.  For details
see Section~\ref{spec4} or \ref{dollard}.

\subsection{The Concept of Unification} \label{unify}

During the course of verifying a proof, when Metamath\index{Metamath}
encounters an assertion label\index{assertion label}, it associates the
mandatory hypotheses\index{mandatory hypothesis} of the assertion with the top
entries of the RPN stack\index{stack}\index{RPN stack}.  Metamath then
determines what substitutions\index{substitution!variable}\index{variable
substitution} it must make to the variables in the assertion's mandatory
hypotheses in order for these hypotheses to become identical to their
corresponding stack entries.  This process is called {\bf
unification}\index{unification}.  (We also informally use the term
``unification'' to refer to a set of substitutions that results from the
process, as in ``two unifications are possible.'')  After the substitutions
are made, the hypotheses are said to be {\bf unified}.

If no such substitutions are possible, Metamath will consider the proof
incorrect and notify you with an error message.
% (deleted 3/10/07, per suggestion of Mel O'Cat:)
% The syntax of the
% Metamath language ensures that if a set of substitutions exists, it
% will be unique.

The general algorithm for unification described in the literature is
somewhat complex.
However, in the case of Metamath it is intentionally trivial.
Mandatory hypotheses must be
pushed on the proof stack in the order in which they appear.
In addition, each variable must have its type specified
with a \texttt{\$f} hypothesis before it is used
and that each \texttt{\$f} hypothesis
have the restricted syntax of a typecode (a constant) followed by a variable.
The typecode in the \texttt{\$f} hypothesis must match the first symbol of
the corresponding RPN stack entry (which will also be a constant), so
the only possible match for the variable in the \texttt{\$f} hypothesis is
the sequence of symbols in the stack entry after the initial constant.

In the Proof Assistant\index{Proof Assistant}, a more general unification
algorithm is used.  While a proof is being developed, sometimes not enough
information is available to determine a unique unification.  In this case
Metamath will ask you to pick the correct one.\index{ambiguous
unification}\index{unification!ambiguous}

\section{Extensions to the Metamath Language}\index{extended
language}

\subsection{Comments in the Metamath Language}\label{comments}
\index{markup notation}
\index{comments!markup notation}

The commenting feature allows you to annotate the contents of
a database.  Just as with most
computer languages, comments are ignored for the purpose of interpreting the
contents of the database. Comments effectively act as
additional white space\index{white
space} between tokens
when a database is parsed.

A comment may be placed at the beginning, end, or
between any two tokens\index{token} in a source file.

Comments have the following syntax:
\begin{center}
 \texttt{\$(} {\em text} \texttt{\$)}
\end{center}
Here,\index{\texttt{\$(} and \texttt{\$)} auxiliary
keywords}\index{comment} {\em text} is a string, possibly empty, of any
characters in Metamath's character set (p.~\pageref{spec1chars}), except
that the character strings \texttt{\$(} and \texttt{\$)} may not appear
in {\em text}.  Thus nested comments are not
permitted:\footnote{Computer languages have differing standards for
nested comments, and rather than picking one it was felt simplest not to
allow them at all, at least in the current version of
Metamath\index{Metamath!limitations of version 0.07.30}.} Metamath will
complain if you give it
\begin{center}
 \texttt{\$( This is a \$( nested \$) comment.\ \$)}
\end{center}
To compensate for this non-nesting behavior, I often change all \texttt{\$}'s
to \texttt{@}'s in sections of Metamath code I wish to comment out.

The Metamath program supports a number of markup mechanisms and conventions
to generate good-looking results in \LaTeX\ and {\sc html},
as discussed below.
These markup features have to do only with how the comments are typeset,
and have no effect on how Metamath verifies the proofs in the database.
The improper
use of them may result in incorrectly typeset output, but no Metamath
error messages will result during the \texttt{read} and \texttt{verify
proof} commands.  (However, the \texttt{write
theorem\texttt{\char`\_}list} command
will check for markup errors as a side-effect of its
{\sc html} generation.)
Section~\ref{texout} has instructions for creating \LaTeX\ output, and
section~\ref{htmlout} has instructions for creating
{\sc html}\index{HTML} output.

\subsubsection{Headings}\label{commentheadings}

If the \texttt{\$(} is immediately followed by a new line
starting with a heading marker, it is a header.
This can start with:

\begin{itemize}
 \item[] \texttt{\#\#\#\#} - major part header
 \item[] \texttt{\#*\#*} - section header
 \item[] \texttt{=-=-} - subsection header
 \item[] \texttt{-.-.} - subsubsection header
\end{itemize}

The line following the marker line
will be used for the table of contents entry, after trimming spaces.
The next line should be another (closing) matching marker line.
Any text after that
but before the closing \texttt{\$}, such as an extended description of the
section, will be included on the \texttt{mmtheoremsNNN.html} page.

For more information, run
\texttt{help write theorem\char`\_list}.

\subsubsection{Math mode}
\label{mathcomments}
\index{\texttt{`} inside comments}
\index{\texttt{\char`\~} inside comments}
\index{math mode}

Inside of comments, a string of tokens\index{token} enclosed in
grave accents\index{grave accent (\texttt{`})} (\texttt{`}) will be converted
to standard mathematical symbols during
{\sc HTML}\index{HTML} or \LaTeX\ output
typesetting,\index{latex@{\LaTeX}} according to the information in the
special \texttt{\$t}\index{\texttt{\$t} comment}\index{typesetting
comment} comment in the database
(see section~\ref{tcomment} for information about the typesetting
comment, and Appendix~\ref{ASCII} to see examples of its results).

The first grave accent\index{grave accent (\texttt{`})} \texttt{`}
causes the output processor to enter {\bf math mode}\index{math mode}
and the second one exits it.
In this
mode, the characters following the \texttt{`} are interpreted as a
sequence of math symbol tokens separated by white space\index{white
space}.  The tokens are looked up in the \texttt{\$t}
comment\index{\texttt{\$t} comment}\index{typesetting comment} and if
found, they will be replaced by the standard mathematical symbols that
they correspond to before being placed in the typeset output file.  If
not found, the symbol will be output as is and a warning will be issued.
The tokens do not have to be active in the database, although a warning
will be issued if they are not declared with \texttt{\$c} or
\texttt{\$v} statements.

Two consecutive
grave accents \texttt{``} are treated as a single actual grave accent
(both inside and outside of math mode) and will not cause the output
processor to enter or exit math mode.

Here is an example of its use\index{Pierce's axiom}:
\begin{center}
\texttt{\$( Pierce's axiom, ` ( ( ph -> ps ) -> ph ) -> ph ` ,\\
         is not very intuitive. \$)}
\end{center}
becomes
\begin{center}
   \texttt{\$(} Pierce's axiom, $((\varphi \rightarrow \psi)\rightarrow
\varphi)\rightarrow \varphi$, is not very intuitive. \texttt{\$)}
\end{center}

Note that the math symbol tokens\index{token} must be surrounded by white
space\index{white space}.
%, since there is no context that allows ambiguity to be
%resolved, as is the case with math symbol sequences in some of the Metamath
%statements.
White space should also surround the \texttt{`}
delimiters.

The math mode feature also gives you a quick and easy way to generate
text containing mathematical symbols, independently of the intended
purpose of Metamath.\index{Metamath!using as a math editor} To do this,
simply create your text with grave accents surrounding your formulas,
after making sure that your math symbols are mapped to \LaTeX\ symbols
as described in Appendix~\ref{ASCII}.  It is easier if you start with a
database with predefined symbols such as \texttt{set.mm}.  Use your
grave-quoted math string to replace an existing comment, then typeset
the statement corresponding to that comment following the instructions
from the \texttt{help tex} command in the Metamath program.  You will
then probably want to edit the resulting file with a text editor to fine
tune it to your exact needs.

\subsubsection{Label Mode}\index{label mode}

Outside of math mode, a tilde\index{tilde (\texttt{\char`\~})} \verb/~/
indicates to Metamath's\index{Metamath} output processor that the
token\index{token} that follows (i.e.\ the characters up to the next
white space\index{white space}) represents a statement label or URL.
This formatting mode is called {\bf label mode}\index{label mode}.
If a literal tilde
is desired (outside of math mode) instead of label mode,
use two tildes in a row to represent it.

When generating a \LaTeX\ output file,
the following token will be formatted in \texttt{typewriter}
font, and the tilde removed, to make it stand out from the rest of the text.
This formatting will be applied to all characters after the
tilde up to the first white space\index{white space}.
Whether
or not the token is an actual statement label is not checked, and the
token does not have to have the correct syntax for a label; no error
messages will be produced.  The only effect of the label mode on the
output is that typewriter font will be used for the tokens that are
placed in the \LaTeX\ output file.

When generating {\sc html},
the tokens after the tilde {\em must} be a URL (either http: or https:)
or a valid label.
Error messages will be issued during that output if they aren't.
A hyperlink will be generated to that URL or label.

\subsubsection{Link to bibliographical reference}\index{citation}%
\index{link to bibliographical reference}

Bibliographical references are handled specially when generating
{\sc html} if formatted specially.
Text in the form \texttt{[}{\em author}\texttt{]}
is considered a link to a bibliographical reference.
See \texttt{help html} and \texttt{help write
bibliography} in the Metamath program for more
information.
% \index{\texttt{\char`\[}\ldots\texttt{]} inside comments}
See also Sections~\ref{tcomment} and \ref{wrbib}.

The \texttt{[}{\em author}\texttt{]} notation will also create an entry in
the bibliography cross-reference file generated by \texttt{write
bibliography} (Section~\ref{wrbib}) for {\sc HTML}.
For this to work properly, the
surrounding comment must be formatted as follows:
\begin{quote}
    {\em keyword} {\em label} {\em noise-word}
     \texttt{[}{\em author}\texttt{] p.} {\em number}
\end{quote}
for example
\begin{verbatim}
     Theorem 5.2 of [Monk] p. 223
\end{verbatim}
The {\em keyword} is not case sensitive and must be one of the following:
\begin{verbatim}
     theorem lemma definition compare proposition corollary
     axiom rule remark exercise problem notation example
     property figure postulate equation scheme chapter
\end{verbatim}
The optional {\em label} may consist of more than one
(non-{\em keyword} and non-{\em noise-word}) word.
The optional {\em noise-word} is one of:
\begin{verbatim}
     of in from on
\end{verbatim}
and is  ignored when the cross-reference file is created.  The
\texttt{write
biblio\-graphy} command will perform error checking to verify the
above format.\index{error checking}

\subsubsection{Parentheticals}\label{parentheticals}

The end of a comment may include one or more parenthicals, that is,
statements enclosed in parentheses.
The Metamath program looks for certain parentheticals and can issue
warnings based on them.
They are:

\begin{itemize}
 \item[] \texttt{(Contributed by }
   \textit{NAME}\texttt{,} \textit{DATE}\texttt{.)} -
   document the original contributor's name and the date it was created.
 \item[] \texttt{(Revised by }
   \textit{NAME}\texttt{,} \textit{DATE}\texttt{.)} -
   document the contributor's name and creation date
   that resulted in significant revision
   (not just an automated minimization or shortening).
 \item[] \texttt{(Proof shortened by }
   \textit{NAME}\texttt{,} \textit{DATE}\texttt{.)} -
   document the contributor's name and date that developed a significant
   shortening of the proof (not just an automated minimization).
 \item[] \texttt{(Proof modification is discouraged.)} -
   Note that this proof should normally not be modified.
 \item[] \texttt{(New usage is discouraged.)} -
   Note that this assertion should normally not be used.
\end{itemize}

The \textit{DATE} must be in form YYYY-MMM-DD, where MMM is the
English abbreviation of that month.

\subsubsection{Other markup}\label{othermarkup}
\index{markup notation}

There are other markup notations for generating good-looking results
beyond math mode and label mode:

\begin{itemize}
 \item[]
         \texttt{\char`\_} (underscore)\index{\texttt{\char`\_} inside comments} -
             Italicize text starting from
              {\em space}\texttt{\char`\_}{\em non-space} (i.e.\ \texttt{\char`\_}
              with a space before it and a non-space character after it) until
             the next
             {\em non-space}\texttt{\char`\_}{\em space}.  Normal
             punctuation (e.g.\ a trailing
             comma or period) is ignored when determining {\em space}.
 \item[]
         \texttt{\char`\_} (underscore) - {\em
         non-space}\texttt{\char`\_}{\em non-space-string}, where
          {\em non-space-string} is a string of non-space characters,
         will make {\em non-space-string} become a subscript.
 \item[]
         \texttt{<HTML>}...\texttt{</HTML>} - do not convert
         ``\texttt{<}'' and ``\texttt{>}''
         in the enclosed text when generating {\sc HTML},
         otherwise process markup normally. This allows direct insertion
         of {\sc html} commands.
 \item[]
       ``\texttt{\&}ref\texttt{;}'' - insert an {\sc HTML}
         character reference.
         This is how to insert arbitrary Unicode characters
         (such as accented characters).  Currently only directly supported
         when generating {\sc HTML}.
\end{itemize}

It is recommended that spaces surround any \texttt{\char`\~} and
\texttt{`} tokens in the comment and that a space follow the {\em label}
after a \texttt{\char`\~} token.  This will make global substitutions
to change labels and symbol names much easier and also eliminate any
future chance of ambiguity.  Spaces around these tokens are automatically
removed in the final output to conform with normal rules of punctuation;
for example, a space between a trailing \texttt{`} and a left parenthesis
will be removed.

A good way to become familiar with the markup notation is to look at
the extensive examples in the \texttt{set.mm} database.

\subsection{The Typesetting Comment (\texttt{\$t})}\label{tcomment}

The typesetting comment \texttt{\$t} in the input database file
provides the information necessary to produce good-looking results.
It provides \LaTeX\ and {\sc html}
definitions for math symbols,
as well supporting as some
customization of the generated web page.
If you add a new token to a database, you should also
update the \texttt{\$t} comment information if you want to eventually
create output in \LaTeX\ or {\sc HTML}.
See the
\texttt{set.mm}\index{set theory database (\texttt{set.mm})} database
file for an extensive example of a \texttt{\$t} comment illustrating
many of the features described below.

Programs that do not need to generate good-looking presentation results,
such as programs that only verify Metamath databases,
can completely ignore typesetting comments
and just treat them as normal comments.
Even the Metamath program only consults the
\texttt{\$t} comment information when it needs to generate typeset output
in \LaTeX\ or {\sc HTML}
(e.g., when you open a \LaTeX\ output file with the \texttt{open tex} command).

We will first discuss the syntax of typesetting comments, and then
briefly discuss how this can be used within the Metamath program.

\subsubsection{Typesetting Comment Syntax Overview}

The typesetting comment is identified by the token
\texttt{\$t}\index{\texttt{\$t} comment}\index{typesetting comment} in
the comment, and the typesetting comment ends at the matching
\texttt{\$)}:
\[
  \mbox{\tt \$(\ }
  \mbox{\tt \$t\ }
  \underbrace{
    \mbox{\tt \ \ \ \ \ \ \ \ \ \ \ }
    \cdots
    \mbox{\tt \ \ \ \ \ \ \ \ \ \ \ }
  }_{\mbox{Typesetting definitions go here}}
  \mbox{\tt \ \$)}
\]

There must be one or more white space characters, and only white space
characters, between the \texttt{\$(} that starts the comment
and the \texttt{\$t} symbol,
and the \texttt{\$t} must be followed by one
or more white space characters
(see section \ref{whitespace} for the definition of white space characters).
The typesetting comment continues until the comment end token \texttt{\$)}
(which must be preceded by one or more white space characters).

In version 0.07.30\index{Metamath!limitations of version 0.07.30} of the
Metamath program, there may be only one \texttt{\$t} comment in a
database.  This restriction may be lifted in the future to allow
many \texttt{\$t} comments in a database.

Between the \texttt{\$t} symbol (and its following white space) and the
comment end token \texttt{\$)} (and its preceding white space)
is a sequence of one or more typesetting definitions, where
each definition has the form
\textit{definition-type arg arg ... ;}.
Each of the zero or more \textit{arg} values
can be either a typesetting data or a keyword
(what keywords are allowed, and where, depends on the specific
\textit{definition-type}).
The \textit{definition-type}, and each argument \textit{arg},
are separated by one or more white space characters.
Every definition ends in an unquoted semicolon;
white space is not required before the terminating semicolon of a definition.
Each definition should start on a new line.\footnote{This
restriction of the current version of Metamath
(0.07.30)\index{Metamath!limitations of version 0.07.30} may be removed
in a future version, but you should do it anyway for readability.}

For example, this typesetting definition:
\begin{center}
 \verb$latexdef "C_" as "\subseteq";$
\end{center}
defines the token \verb$C_$ as the \LaTeX\ symbol $\subseteq$ (which means
``subset'').

Typesetting data is a sequence of one or more quoted strings
(if there is more than one, they are connected by \texttt{\char`\+}).
Often a single quoted string is used to provide data for a definition, using
either double (\texttt{\char`\"}) or single (\texttt{'}) quotation marks.
However,
{\em a quoted string (enclosed in quotation marks) may not include
line breaks.}
A quoted string
may include a quotation mark that matches the enclosing quotes by repeating
the quotation mark twice.  Here are some examples:

\begin{tabu}   { l l }
\textbf{Example} & \textbf{Meaning} \\
\texttt{\char`\"a\char`\"\char`\"b\char`\"} & \texttt{a\char`\"b} \\
\texttt{'c''d'} & \texttt{c'd} \\
\texttt{\char`\"e''f\char`\"} & \texttt{e''f} \\
\texttt{'g\char`\"\char`\"h'} & \texttt{g\char`\"\char`\"h} \\
\end{tabu}

Finally, a long quoted string
may be broken up into multiple quoted strings (considered, as a whole,
a single quoted string) and joined with \texttt{\char`\+}.
You can even use multiple lines as long as a
'+' is at the end of every line except the last one.
The \texttt{\char`\+} should be preceded and followed by at least one
white space character.
Thus, for example,
\begin{center}
 \texttt{\char`\"ab\char`\"\ \char`\+\ \char`\"cd\char`\"
    \ \char`\+\ \\ 'ef'}
\end{center}
is the same as
\begin{center}
 \texttt{\char`\"abcdef\char`\"}
\end{center}

{\sc c}-style comments \texttt{/*}\ldots\texttt{*/} are also supported.

In practice, whenever you add a new math token you will often want to add
typesetting definitions using
\texttt{latexdef}, \texttt{htmldef}, and
\texttt{althtmldef}, as described below.
That way, they will all be up to date.
Of course, whether or not you want to use all three definitions will
depend on how the database is intended to be used.

Below we discuss the different possible \textit{definition-kind} options.
We will show data surrounded by double quotes (in practice they can also use
single quotes and/or be a sequence joined by \texttt{+}s).
We will use specific names for the \textit{data} to make clear what
the data is used for, such as
{\em math-token} (for a Metamath math token,
{\em latex-string} (for string to be placed in a \LaTeX\ stream),
{\em {\sc html}-code} (for {\sc html} code),
and {\em filename} (for a filename).

\subsubsection{Typesetting Comment - \LaTeX}

The syntax for a \LaTeX\ definition is:
\begin{center}
 \texttt{latexdef "}{\em math-token}\texttt{" as "}{\em latex-string}\texttt{";}
\end{center}
\index{latex definitions@\LaTeX\ definitions}%
\index{\texttt{latexdef} statement}

The {\em token-string} and {\em latex-string} are the data
(character strings) for
the token and the \LaTeX\ definition of the token, respectively,

These \LaTeX\ definitions are used by the Metamath program
when it is asked to product \LaTeX output using
the \texttt{write tex} command.

\subsubsection{Typesetting Comment - {\sc html}}

The key kinds of {\sc HTML} definitions have the following syntax:

\vskip 1ex
    \texttt{htmldef "}{\em math-token}\texttt{" as "}{\em
    {\sc html}-code}\texttt{";}\index{\texttt{htmldef} statement}
                    \ \ \ \ \ \ldots

    \texttt{althtmldef "}{\em math-token}\texttt{" as "}{\em
{\sc html}-code}\texttt{";}\index{\texttt{althtmldef} statement}

                    \ \ \ \ \ \ldots

Note that in {\sc HTML} there are two possible definitions for math tokens.
This feature is useful when
an alternate representation of symbols is desired, for example one that
uses Unicode entities and another uses {\sc gif} images.

There are many other typesetting definitions that can control {\sc HTML}.
These include:

\vskip 1ex

    \texttt{htmldef "}{\em math-token}\texttt{" as "}{\em {\sc
    html}-code}\texttt{";}

    \texttt{htmltitle "}{\em {\sc html}-code}\texttt{";}%
\index{\texttt{htmltitle} statement}

    \texttt{htmlhome "}{\em {\sc html}-code}\texttt{";}%
\index{\texttt{htmlhome} statement}

    \texttt{htmlvarcolor "}{\em {\sc html}-code}\texttt{";}%
\index{\texttt{htmlvarcolor} statement}

    \texttt{htmlbibliography "}{\em filename}\texttt{";}%
\index{\texttt{htmlbibliography} statement}

\vskip 1ex

\noindent The \texttt{htmltitle} is the {\sc html} code for a common
title, such as ``Metamath Proof Explorer.''  The \texttt{htmlhome} is
code for a link back to the home page.  The \texttt{htmlvarcolor} is
code for a color key that appears at the bottom of each proof.  The file
specified by {\em filename} is an {\sc html} file that is assumed to
have a \texttt{<A NAME=}\ldots\texttt{>} tag for each bibiographic
reference in the database comments.  For example, if
\texttt{[Monk]}\index{\texttt{\char`\[}\ldots\texttt{]} inside comments}
occurs in the comment for a theorem, then \texttt{<A NAME='Monk'>} must
be present in the file; if not, a warning message is given.

Associated with
\texttt{althtmldef}
are the statements
\vskip 1ex

    \texttt{htmldir "}{\em
      directoryname}\texttt{";}\index{\texttt{htmldir} statement}

    \texttt{althtmldir "}{\em
     directoryname}\texttt{";}\index{\texttt{althtmldir} statement}

\vskip 1ex
\noindent giving the directories of the {\sc gif} and Unicode versions
respectively; their purpose is to provide cross-linking between the
two versions in the generated web pages.

When two different types of pages need to be produced from a single
database, such as the Hilbert Space Explorer that extends the Metamath
Proof Explorer, ``extended'' variables may be declared in the
\texttt{\$t} comment:
\vskip 1ex

    \texttt{exthtmltitle "}{\em {\sc html}-code}\texttt{";}%
\index{\texttt{exthtmltitle} statement}

    \texttt{exthtmlhome "}{\em {\sc html}-code}\texttt{";}%
\index{\texttt{exthtmlhome} statement}

    \texttt{exthtmlbibliography "}{\em filename}\texttt{";}%
\index{\texttt{exthtmlbibliography} statement}

\vskip 1ex
\noindent When these are declared, you also must declare
\vskip 1ex

    \texttt{exthtmllabel "}{\em label}\texttt{";}%
\index{\texttt{exthtmllabel} statement}

\vskip 1ex \noindent that identifies the database statement where the
``extended'' section of the database starts (in our example, where the
Hilbert Space Explorer starts).  During the generation of web pages for
that starting statement and the statements after it, the {\sc html} code
assigned to \texttt{exthtmltitle} and \texttt{exthtmlhome} is used
instead of that assigned to \texttt{htmltitle} and \texttt{htmlhome},
respectively.

\begin{sloppy}
\subsection{Additional Information Com\-ment (\texttt{\$j})} \label{jcomment}
\end{sloppy}

The additional information comment, aka the
\texttt{\$j}\index{\texttt{\$j} comment}\index{additional information comment}
comment,
provides a way to add additional structured information that can
be optionally parsed by systems.

The additional information comment is parsed the same way as the
typesetting comment (\texttt{\$t}) (see section \ref{tcomment}).
That is,
the additional information comment begins with the token
\texttt{\$j} within a comment,
and continues until the comment close \texttt{\$)}.
Within an additional information comment is a sequence of one or more
commands of the form \texttt{command arg arg ... ;}
where each of the zero or more \texttt{arg} values
can be either a quoted string or a keyword.
Note that every command ends in an unquoted semicolon.
If a verifier is parsing an additional information comment, but
doesn't recognize a particular command, it must skip the command
by finding the end of the command (an unquoted semicolon).

A database may have 0 or more additional information comments.
Note, however, that a verifier may ignore these comments entirely or only
process certain commands in an additional information comment.
The \texttt{mmj2} verifier supports many commands in additional information
comments.
We encourage systems that process additional information comments
to coordinate so that they will use the same command for the same effect.

Examples of additional information comments with various commands
(from the \texttt{set.mm} database) are:

\begin{itemize}
   \item Define the syntax and logical typecodes,
     and declare that our grammar is
     unambiguous (verifiable using the KLR parser, with compositing depth 5).
\begin{verbatim}
  $( $j
    syntax 'wff';
    syntax '|-' as 'wff';
    unambiguous 'klr 5';
  $)
\end{verbatim}

   \item Register $\lnot$ and $\rightarrow$ as primitive expressions
           (lacking definitions).
\begin{verbatim}
  $( $j primitive 'wn' 'wi'; $)
\end{verbatim}

   \item There is a special justification for \texttt{df-bi}.
\begin{verbatim}
  $( $j justification 'bijust' for 'df-bi'; $)
\end{verbatim}

   \item Register $\leftrightarrow$ as an equality for its type (wff).
\begin{verbatim}
  $( $j
    equality 'wb' from 'biid' 'bicomi' 'bitri';
    definition 'dfbi1' for 'wb';
  $)
\end{verbatim}

   \item Theorem \texttt{notbii} is the congruence law for negation.
\begin{verbatim}
  $( $j congruence 'notbii'; $)
\end{verbatim}

   \item Add \texttt{setvar} as a typecode.
\begin{verbatim}
  $( $j syntax 'setvar'; $)
\end{verbatim}

   \item Register $=$ as an equality for its type (\texttt{class}).
\begin{verbatim}
  $( $j equality 'wceq' from 'eqid' 'eqcomi' 'eqtri'; $)
\end{verbatim}

\end{itemize}


\subsection{Including Other Files in a Metamath Source File} \label{include}
\index{\texttt{\$[} and \texttt{\$]} auxiliary keywords}

The keywords \texttt{\$[} and \texttt{\$]} specify a file to be
included\index{included file}\index{file inclusion} at that point in a
Metamath\index{Metamath} source file\index{source file}.  The syntax for
including a file is as follows:
\begin{center}
\texttt{\$[} {\em file-name} \texttt{\$]}
\end{center}

The {\em file-name} should be a single token\index{token} with the same syntax
as a math symbol (i.e., all 93 non-whitespace
printable characters other than \texttt{\$} are
allowed, subject to the file-naming limitations of your operating system).
Comments may appear between the \texttt{\$[} and \texttt{\$]} keywords.  Included
files may include other files, which may in turn include other files, and so
on.

For example, suppose you want to use the set theory database as the starting
point for your own theory.  The first line in your file could be
\begin{center}
\texttt{\$[ set.mm \$]}
\end{center} All of the information (axioms, theorems,
etc.) in \texttt{set.mm} and any files that {\em it} includes will become
available for you to reference in your file. This can help make your work more
modular. A drawback to including files is that if you change the name of a
symbol or the label of a statement, you must also remember to update any
references in any file that includes it.


The naming conventions for included files are the same as those of your
operating system.\footnote{On the Macintosh, prior to Mac OS X,
 a colon is used to separate disk
and folder names from your file name.  For example, {\em volume}\texttt{:}{\em
file-name} refers to the root directory, {\em volume}\texttt{:}{\em
folder-name}\texttt{:}{\em file-name} refers to a folder in root, and {\em
volume}\texttt{:}{\em folder-name}\texttt{:}\ldots\texttt{:}{\em file-name} refers to a
deeper folder.  A simple {\em file-name} refers to a file in the folder from
which you launch the Metamath application.  Under Mac OS X and later,
the Metamath program is run under the Terminal application, which
conforms to Unix naming conventions.}\index{Macintosh file
names}\index{file names!Macintosh}\label{includef} For compatibility among
operating systems, you should keep the file names as simple as possible.  A
good convention to use is {\em file}\texttt{.mm} where {\em file} is eight
characters or less, in lower case.

There is no limit to the nesting depth of included files.  One thing that you
should be aware of is that if two included files themselves include a common
third file, only the {\em first} reference to this common file will be read
in.  This allows you to include two or more files that build on a common
starting file without having to worry about label and symbol conflicts that
would occur if the common file were read in more than once.  (In fact, if a
file includes itself, the self-reference will be ignored, although of course
it would not make any sense to do that.)  This feature also means, however,
that if you try to include a common file in several inner blocks, the result
might not be what you expect, since only the first reference will be replaced
with the included file (unlike the include statement in most other computer
languages).  Thus you would normally include common files only in the
outermost block\index{outermost block}.

\subsection{Compressed Proof Format}\label{compressed1}\index{compressed
proof}\index{proof!compressed}

The proof notation presented in Section~\ref{proof} is called a
{\bf normal proof}\index{normal proof}\index{proof!normal} and in principle is
sufficient to express any proof.  However, proofs often contain steps and
subproofs that are identical.  This is particularly true in typical
Metamath\index{Metamath} applications, because Metamath requires that the math
symbol sequence (usually containing a formula) at each step be separately
constructed, that is, built up piece by piece. As a result, a lot of
repetition often results.  The {\bf compressed proof} format allows Metamath
to take advantage of this redundancy to shorten proofs.

The specification for the compressed proof format is given in
Appen\-dix~\ref{compressed}.

Normally you need not concern yourself with the details of the compressed
proof format, since the Metamath program will allow you to convert from
the normal format to the compressed format with ease, and will also
automatically convert from the compressed format when proofs are displayed.
The overall structure of the compressed format is as follows:
\begin{center}
  \texttt{\$= ( } {\em label-list} \texttt{) } {\em compressed-proof\ }\ \texttt{\$.}
\end{center}
\index{\texttt{\$=} keyword}
The first \texttt{(} serves as a flag to Metamath that a compressed proof
follows.  The {\em label-list} includes all statements referred to by the
proof except the mandatory hypotheses\index{mandatory hypothesis}.  The {\em
compressed-proof} is a compact encoding of the proof, using upper-case
letters, and can be thought of as a large integer in base 26.  White
space\index{white space} inside a {\em compressed-proof} is
optional and is ignored.

It is important to note that the order of the mandatory hypotheses of
the statement being proved must not be changed if the compressed proof
format is used, otherwise the proof will become incorrect.  The reason
for this is that the mandatory hypotheses are not mentioned explicitly
in the compressed proof in order to make the compression more efficient.
If you wish to change the order of mandatory hypotheses, you must first
convert the proof back to normal format using the \texttt{save proof
{\em statement} /normal}\index{\texttt{save proof} command} command.
Later, you can go back to compressed format with \texttt{save proof {\em
statement} /compressed}.

During error checking with the \texttt{verify proof} command, an error
found in a compressed proof may point to a character in {\em
compressed-proof}, which may not be very meaningful to you.  In this
case, try to \texttt{save proof /normal} first, then do the
\texttt{verify proof} again.  In general, it is best to make sure a
proof is correct before saving it in compressed format, because severe
errors are less likely to be recoverable than in normal format.

\subsection{Specifying Unknown Proofs or Subproofs}\label{unknown}

In a proof under development, any step or subproof that is not yet known
may be represented with a single \texttt{?}.  For the purposes of
parsing the proof, the \texttt{?}\ \index{\texttt{]}@\texttt{?}\ inside
proofs} will push a single entry onto the RPN stack just as if it were a
hypothesis.  While developing a proof with the Proof
Assistant\index{Proof Assistant}, a partially developed proof may be
saved with the \texttt{save new{\char`\_}proof}\index{\texttt{save
new{\char`\_}proof} command} command, and \texttt{?}'s will be placed at
the appropriate places.

All \texttt{\$p}\index{\texttt{\$p} statement} statements must have
proofs, even if they are entirely unknown.  Before creating a proof with
the Proof Assistant, you should specify a completely unknown proof as
follows:
\begin{center}
  {\em label} \texttt{\$p} {\em statement} \texttt{\$= ?\ \$.}
\end{center}
\index{\texttt{\$=} keyword}
\index{\texttt{]}@\texttt{?}\ inside proofs}

The \texttt{verify proof}\index{\texttt{verify proof} command} command
will check the known portions of a partial proof for errors, but will
warn you that the statement has not been proved.

Note that partially developed proofs may be saved in compressed format
if desired.  In this case, you will see one or more \texttt{?}'s in the
{\em compressed-proof} part.\index{compressed
proof}\index{proof!compressed}

\section{Axioms vs.\ Definitions}\label{definitions}

The \textit{basic}
Metamath\index{Metamath} language and program
make no distinction\index{axiom vs.\
definition} between axioms\index{axiom} and
definitions.\index{definition} The \texttt{\$a}\index{\texttt{\$a}
statement} statement is used for both.  At first, this may seem
puzzling.  In the minds of many mathematicians, the distinction is
clear, even obvious, and hardly worth discussing.  A definition is
considered to be merely an abbreviation that can be replaced by the
expression for which it stands; although unless one actually does this,
to be precise then one should say that a theorem\index{theorem} is a
consequence of the axioms {\em and} the definitions that are used in the
formulation of the theorem \cite[p.~20]{Behnke}.\index{Behnke, H.}

\subsection{What is a Definition?}

What is a definition?  In its simplest form, a definition introduces a new
symbol and provides an unambiguous rule to transform an expression containing
the new symbol to one without it.  The concept of a ``proper
definition''\index{proper definition}\index{definition!proper} (as opposed to
a creative definition)\index{creative definition}\index{definition!creative}
that is usually agreed upon is (1) the definition should not strengthen the
language and (2) any symbols introduced by the definition should be eliminable
from the language \cite{Nemesszeghy}\index{Nemesszeghy, E. Z.}.  In other
words, they are mere typographical conveniences that do not belong to the
system and are theoretically superfluous.  This may seem obvious, but in fact
the nature of definitions can be subtle, sometimes requiring difficult
metatheorems to establish that they are not creative.

A more conservative stance was taken by logician S.
Le\'{s}niewski.\index{Le\'{s}niewski, S.}
\begin{quote}
Le\'{s}niewski
regards definitions as theses of the system.  In this respect they do
not differ either from the axioms or from theorems, i.e.\ from the
theses added to the system on the basis of the rule of substitution or
the rule of detachment [modus ponens].  Once definitions have been
accepted as theses of the system, it becomes necessary to consider them
as true propositions in the same sense in which axioms are true
\cite{Lejewski}.
\end{quote}\index{Lejewski, Czeslaw}

Let us look at some simple examples of definitions in propositional
calculus.  Consider the definition of logical {\sc or}
(disjunction):\index{disjunction ($\vee$)} ``$P\vee Q$ denotes $\neg P
\rightarrow Q$ (not $P$ implies $Q$).''  It is very easy to recognize a
statement making use of this definition, because it introduces the new
symbol $\vee$ that did not previously exist in the language.  It is easy
to see that no new theorems of the original language will result from
this definition.

Next, consider a definition that eliminates parentheses:  ``$P
\rightarrow Q\rightarrow R$ denotes $P\rightarrow (Q \rightarrow R)$.''
This is more subtle, because no new symbols are introduced.  The reason
this definition is considered proper is that no new symbol sequences
that are valid wffs (well-formed formulas)\index{well-formed formula
(wff)} in the original language will result from the definition, since
``$P \rightarrow Q\rightarrow R$'' is not a wff in the original
language.  Here, we implicitly make use of the fact that there is a
decision procedure that allows us to determine whether or not a symbol
sequence is a wff, and this fact allows us to use symbol sequences that
are not wffs to represent other things (such as wffs) by means of the
definition.  However, to justify the definition as not being creative we
need to prove that ``$P \rightarrow Q\rightarrow R$'' is in fact not a
wff in the original language, and this is more difficult than in the
case where we simply introduce a new symbol.

%Now let's take this reasoning to an extreme.  Propositional calculus is a
%decidable theory,\footnote{This means that a mechanical algorithm exists to
%determine whether or not a wff is a theorem.} so in principle we could make use
%of symbol sequences that are not theorems to represent other things (say, to
%encode actual theorems in a more compact way).  For example, let us extend the
%language by defining a wff ``$P$'' in the extended language as the theorem
%``$P\rightarrow P$''\footnote{This is one of the first theorems proved in the
%Metamath database \texttt{set.mm}.}\index{set
%theory database (\texttt{set.mm})} in the original language whenever ``$P$'' is
%not a theorem in the original language.  In the extended language, any wff
%``$Q$'' thus represents a theorem; to find out what theorem (in the original
%language) ``$Q$'' represents, we determine whether ``$Q$'' is a theorem in the
%original language (before the definition was introduced).  If so, we're done; if
%not, we replace ``$Q$'' by ``$Q\rightarrow Q$'' to eliminate the definition.
%This definition is therefore eliminable, and it does not ``strengthen'' the
%language because any wff that is not a theorem is not in the set of statements
%provable in the original language and thus is available for use by definitions.
%
%Of course, a definition such as this would render practically useless the
%communication of theorems of propositional calculus; but
%this is just a human shortcoming, since we can't always easily discern what is
%and is not a theorem by inspection.  In fact, the extended theory with this
%definition has no more and no less information than the original theory; it just
%expresses certain theorems of the form ``$P\rightarrow P$''
%in a more compact way.
%
%The point here is that what constitutes a proper definition is a matter of
%judgment about whether a symbol sequence can easily be recognized by a human
%as invalid in some sense (for example, not a wff); if so, the symbol sequence
%can be appropriated for use by a definition in order to make the extended
%language more compact.  Metamath\index{Metamath} lacks the ability to make this
%judgment, since as far as Metamath is concerned the definition of a wff, for
%example, is arbitrary.  You define for Metamath how wffs\index{well-formed
%formula (wff)} are constructed according to your own preferred style.  The
%concept of a wff may not even exist in a given formal system\index{formal
%system}.  Metamath treats all definitions as if they were new axioms, and it
%is up to the human mathematician to judge whether the definition is ``proper''
%'\index{proper definition}\index{definition!proper} in some agreed-upon way.

What constitutes a definition\index{definition} versus\index{axiom vs.\
definition} an axiom\index{axiom} is sometimes arbitrary in mathematical
literature.  For example, the connectives $\vee$ ({\sc or}), $\wedge$
({\sc and}), and $\leftrightarrow$ (equivalent to) in propositional
calculus are usually considered defined symbols that can be used as
abbreviations for expressions containing the ``primitive'' connectives
$\rightarrow$ and $\neg$.  This is the way we treat them in the standard
logic and set theory database \texttt{set.mm}\index{set theory database
(\texttt{set.mm})}.  However, the first three connectives can also be
considered ``primitive,'' and axiom systems have been devised that treat
all of them as such.  For example,
\cite[p.~35]{Goodstein}\index{Goodstein, R. L.} presents one with 15
axioms, some of which in fact coincide with what we have chosen to call
definitions in \texttt{set.mm}.  In certain subsets of classical
propositional calculus, such as the intuitionist
fragment\index{intuitionism}, it can be shown that one cannot make do
with just $\rightarrow$ and $\neg$ but must treat additional connectives
as primitive in order for the system to make sense.\footnote{Two nice
systems that make the transition from intuitionistic and other weak
fragments to classical logic just by adding axioms are given in
\cite{Robinsont}\index{Robinson, T. Thacher}.}

\subsection{The Approach to Definitions in \texttt{set.mm}}

In set theory, recursive definitions define a newly introduced symbol in
terms of itself.  The justification of recursive definitions, using
several ``recursion theorems,'' is usually one of the first
sophisticated proofs a student encounters when learning set theory, and
there is a significant amount of implicit metalogic behind a recursive
definition even though the definition itself is typically simple to
state.  It is, however, possible to substitute one kind of complexity
for another.  We can eliminate the need for metalogical justification by
defining the operation directly with an explicit (but complicated)
expression, then deriving the recursive definition directly as a
theorem, using a recursion theorem ``in reverse.''  We do this in
\texttt{set.mm}, as follows.

In \texttt{set.mm} our goal was to introduce almost all definitions in
the form of two expressions connected by either $\leftrightarrow$ or
$=$, where the thing being defined does not appear on the right hand
side.  Quine calls this form ``a genuine or direct definition'' \cite[p.
174]{Quine}\index{Quine, Willard Van Orman}, which makes the definitions
very easy to eliminate and the metalogic\index{metalogic} needed to
justify them as simple as possible.

We achieved this goal in almost all cases in \texttt{set.mm}.
Sometimes this makes the definitions more complex and less
intuitive.  For example, the traditional way to define addition of
natural numbers is to define an operation called {\em
successor}\index{successor} (which means ``plus one'' and is denoted by
``${\rm suc}$''), then define addition recursively\index{recursive
definition} with the two definitions $n + 0 = n$ and $m + {\rm suc}\,n =
{\rm suc} (m + n)$.  Although this definition seems simple and obvious,
the method to eliminate the definition is not obvious:  in the second
part of the definition, addition is defined in terms of itself.  By
eliminating the definition, we don't mean repeatedly applying it to
specific $m$ and $n$ but rather showing the explicit, closed-form
set-theoretical expression that $m + n$ represents, that will work for
any $m$ and $n$ and that does not have a $+$ sign on its right-hand
side.  For a recursive definition like this not to be circular
(creative), there are some hidden, underlying assumptions we must make,
for example that the natural numbers have a certain kind of order.  In
\texttt{set.mm} we chose to start with the direct (though complex and
nonintuitive) definition then derive from it the standard recursive
definition.\footnote{The closed-form definition used in \texttt{set.mm}
for the addition operation on ordinals\index{ordinal
addition}\index{addition!of ordinals} (of which natural numbers are a
subset) is

\setbox\startprefix=\hbox{\tt \ \ df-oadd\ \$a\ }
\setbox\contprefix=\hbox{\tt \ \ \ \ \ \ \ \ \ \ \ \ \ }
\startm
\m{\vdash}\m{+_o}\m{=}\m{\{}\m{\langle}\m{\langle}\m{x}\m{,}\m{y}\m{\rangle}%
\m{,}\m{z}\m{\rangle}\m{|}\m{(}\m{(}\m{x}\m{\in}\m{{\rm On}}\m{\wedge}\m{y}\m{%
\in}\m{{\rm On}}\m{)}\m{\wedge}\m{z}\m{=}\m{(}\m{{\rm rec}}\m{(}\m{\{}\m{%
\langle}\m{w}\m{,}\m{v}\m{\rangle}\m{|}\m{v}\m{=}\m{{\rm suc}}\m{w}\m{\}}\m{,}%
\m{x}\m{)}\m{`}\m{y}\m{)}\m{)}\m{\}}
\endm

\noindent Here, the abstraction class of nested ordered pairs\index{abstraction
class!of nested ordered pairs} is defined by \texttt{df-oprab} in \texttt{set.mm},
and ${\rm rec}$ is a ``recursion operator''\index{recursion operator} with
the definition

\setbox\startprefix=\hbox{\tt \ \ df-rfg\ \$a\ }
\setbox\contprefix=\hbox{\tt \ \ \ \ \ \ \ \ \ \ \ \ }
\startm
\m{\vdash}\m{{\rm rec}}\m{(}\m{F}\m{,}\m{A}\m{)}\m{=}\m{\bigcup}\m{\{}\m{f}%
\m{|}\m{\exists}\m{x}\m{(}\m{x}\m{\in}\m{{\rm On}}\m{\wedge}\m{(}\m{f}\m{{%
\rm Fn}}\m{x}\m{\wedge}\m{\forall}\m{y}\m{(}\m{y}\m{\in}\m{x}\m{\rightarrow}%
\m{(}\m{f}\m{`}\m{y}\m{)}\m{=}\m{(}\m{\{}\m{\langle}\m{g}\m{,}\m{z}\m{\rangle}%
\m{|}\m{(}\m{(}\m{g}\m{=}\m{\varnothing}\m{\wedge}\m{z}\m{=}\m{A}\m{)}\m{\vee}%
\m{(}\m{\lnot}\m{(}\m{g}\m{=}\m{\varnothing}\m{\vee}\m{{\rm Lim}}\m{{\rm dom}}%
\m{g}\m{)}\m{\wedge}\m{z}\m{=}\m{(}\m{F}\m{`}\m{(}\m{g}\m{`}\m{\bigcup}\m{{%
\rm dom}}\m{g}\m{)}\m{)}\m{)}\m{\vee}\m{(}\m{{\rm Lim}}\m{{\rm dom}}\m{g}\m{%
\wedge}\m{z}\m{=}\m{\bigcup}\m{{\rm ran}}\m{g}\m{)}\m{)}\m{\}}\m{`}\m{(}\m{f}%
\m{\restriction}\m{y}\m{)}\m{)}\m{)}\m{)}\m{)}\m{\}}
\endm

\noindent which can be further broken down with definitions shown in
Section~\ref{setdefinitions}.  You may be surprised at the complexity of what
seems like such a simple notion.  From these definitions the simpler, more
intuitive recursive definition is derived as a set of theorems.}\index{natural
number}\index{addition}\index{recursive definition}\index{ordinal addition}
The end result is the same, but we completely eliminate the rather complex
metalogic that justifies the recursive definition.  (For a mathematician,
recursive definitions are more efficient and intuitive than
direct ones once the metalogic has been learned or possibly just accepted as
correct.  However, it was felt that direct definition in \texttt{set.mm}
maximizes rigor by minimizing metalogic.  It can be eliminated effortlessly,
something difficult to do with a recursive definition.)

\subsection{Adding Constraints on Definitions}

The basic Metamath language and the Metamath program do
not have any built-in constraints on definitions, since they are just
\$a statements.

However, nothing prevents a verification system from verifying
additional rules to impose further limitations on definitions.
For example, the \texttt{mmj2}\index{mmj2} program
supports various kinds of
additional information comments (see section \ref{jcomment}).
One of their uses is to optionally verify additional constraints,
including constraints to verify that definitions meet certain
requirements.
These additional checks are required by the
continuous integration (CI)\index{continuous integration (CI)}
checks of the
\texttt{set.mm}\index{set theory database (\texttt{set.mm})}%
\index{Metamath Proof Explorer}
database.
This approach enables us to optionally impose additional requirements
on definitions if we wish, without necessarily imposing those rules on
all databases or requiring all verification systems to implement them.
In addition, this allows us to impose specialized constraints tailored
to one database while not requiring other systems to implement
those specialized constraints.

We impose two constraints on the
\texttt{set.mm}\index{set theory database (\texttt{set.mm})}%
\index{Metamath Proof Explorer} database
via the \texttt{mmj2}\index{mmj2} program that are worth discussing here:
a parse check and a definition soundness check.

% On February 11, 2019 8:32:32 PM EST, saueran@oregonstate.edu wrote:
% The following addition to the end of set.mm is accepted by the mmj2
% parser and definition checker and the metamath verifier(at least it was
% when I checked, you should check it too), and creates a contradiction by
% proving the theorem |- ph.
% ${
% wleftp $a wff ( ( ph ) $.
% wbothp $a wff ( ph ) $.
% df-leftp $a |- ( ( ( ph ) <-> -. ph ) $.
% df-bothp $a |- ( ( ph ) <-> ph ) $.
% anything $p |- ph $=
%   ( wbothp wn wi wleftp df-leftp biimpi df-bothp mpbir mpbi simplim ax-mp)
%   ABZAMACZDZCZMOEZOCQAEZNDZRNAFGSHIOFJMNKLAHJ $.
% $}
%
% This particular problem is countered by enabling, within mmj2,
% SetParser,mmj.verify.LRParser

First,
we enable a parse check in \texttt{mmj2} (through its
\texttt{SetParser} command) that requires that all new definitions
must \textit{not} create an ambiguous parse for a KLR(5) parser.
This prevents some errors such as definitions with imbalanced parentheses.

Second, we run a definition soundness check specific to
\texttt{set.mm} or databases similar to it.
(through the \texttt{definitionCheck} macro).
Some \texttt{\$a} statements (including all ax-* statemnets)
are excluded from these checks, as they will
always fail this simple check,
but they are appropriate for most definitions.
This check imposes a set of additional rules:

\begin{enumerate}

\item New definitions must be introduced using $=$ or $\leftrightarrow$.

\item No \texttt{\$a} statement introduced before this one is allowed to use the
symbol being defined in this definition, and the definition is not
allowed to use itself (except once, in the definiendum).

\item Every variable in the definiens should not be distinct

\item Every dummy variable in the definiendum
are required to be distinct from each other and from variables in
the definiendum.
To determine this, the system will look for a "justification" theorem
in the database, and if it is not there, attempt to briefly prove
$( \varphi \rightarrow \forall x \varphi )$  for each dummy variable x.

\item Every dummy variable should be a set variable,
unless there is a justification theorem available.

\item Every dummy variable must be bound
(if the system cannot determine this a justification theorem must be
provided).

\end{enumerate}

\subsection{Summary of Approach to Definitions}

In short, when being rigorous it turns out that
definitions can be subtle, sometimes requiring difficult
metatheorems to establish that they are not creative.

Instead of building such complications into the Metamath language itself,
the basic Metmath language and program simply treat traditional
axioms and definitions as the same kind of \texttt{\$a} statement.
We have then built various tools to enable people to
verify additional conditions as their creators believe is appropriate
for those specific databases, without complicating the Metamath foundations.

\chapter{The Metamath Program}\label{commands}

This chapter provides a reference manual for the
Metamath program.\index{Metamath!commands}

Current instructions for obtaining and installing the Metamath program
can be found at the \url{http://metamath.org} web site.
For Windows, there is a pre-compiled version called
\texttt{metamath.exe}.  For Unix, Linux, and Mac OS X
(which we will refer to collectively as ``Unix''), the Metamath program
can be compiled from its source code with the command
\begin{verbatim}
gcc *.c -o metamath
\end{verbatim}
using the \texttt{gcc} {\sc c} compiler available on those systems.

In the command syntax descriptions below, fields enclosed in square
brackets [\ ] are optional.  File names may be optionally enclosed in
single or double quotes.  This is useful if the file name contains
spaces or
slashes (\texttt{/}), such as in Unix path names, \index{Unix file
names}\index{file names!Unix} that might be confused with Metamath
command qualifiers.\index{Unix file names}\index{file names!Unix}

\section{Invoking Metamath}

Unix, Linux, and Mac OS X
have a command-line interface called the {\em
bash shell}.  (In Mac OS X, select the Terminal application from
Applications/Utilities.)  To invoke Metamath from the bash shell prompt,
assuming that the Metamath program is in the current directory, type
\begin{verbatim}
bash$ ./metamath
\end{verbatim}

To invoke Metamath from a Windows DOS or Command Prompt, assuming that
the Metamath program is in the current directory (or in a directory
included in the Path system environment variable), type
\begin{verbatim}
C:\metamath>metamath
\end{verbatim}

To use command-line arguments at invocation, the command-line arguments
should be a list of Metamath commands, surrounded by quotes if they
contain spaces.  In Windows, the surrounding quotes must be double (not
single) quotes.  For example, to read the database file \texttt{set.mm},
verify all proofs, and exit the program, type (under Unix)
\begin{verbatim}
bash$ ./metamath 'read set.mm' 'verify proof *' exit
\end{verbatim}
Note that in Unix, any directory path with \texttt{/}'s must be
surrounded by quotes so Metamath will not interpret the \texttt{/} as a
command qualifier.  So if \texttt{set.mm} is in the \texttt{/tmp}
directory, use for the above example
\begin{verbatim}
bash$ ./metamath 'read "/tmp/set.mm"' 'verify proof *' exit
\end{verbatim}

For convenience, if the command-line has one argument and no spaces in
the argument, the command is implicitly assumed to be \texttt{read}.  In
this one special case, \texttt{/}'s are not interpreted as command
qualifiers, so you don't need quotes around a Unix file name.  Thus
\begin{verbatim}
bash$ ./metamath /tmp/set.mm
\end{verbatim}
and
\begin{verbatim}
bash$ ./metamath "read '/tmp/set.mm'"
\end{verbatim}
are equivalent.


\section{Controlling Metamath}

The Metamath program was first developed on a {\sc vax/vms} system, and
some aspects of its command line behavior reflect this heritage.
Hopefully you will find it reasonably user-friendly once you get used to
it.

Each command line is a sequence of English-like words separated by
spaces, as in \texttt{show settings}.  Command words are not case
sensitive, and only as many letters are needed as are necessary to
eliminate ambiguity; for example, \texttt{sh se} would work for the
command \texttt{show settings}.  In some cases arguments such as file
names, statement labels, or symbol names are required; these are
case-sensitive (although file names may not be on some operating
systems).

A command line is entered by typing it in then pressing the {\em return}
({\em enter}) key.  To find out what commands are available, type
\texttt{?} at the \texttt{MM>} prompt.  To find out the choices at any
point in a command, press {\em return} and you will be prompted for
them.  The default choice (the one selected if you just press {\em
return}) is shown in brackets (\texttt{<>}).

You may also type \texttt{?} in place of a command word to force
Metamath to tell you what the choices are.  The \texttt{?} method won't
work, though, if a non-keyword argument such as a file name is expected
at that point, because the program will think that \texttt{?} is the
value of the argument.

Some commands have one or more optional qualifiers which modify the
behavior of the command.  Qualifiers are preceded by a slash
(\texttt{/}), such as in \texttt{read set.mm / verify}.  Spaces are
optional around the \texttt{/}.  If you need to use a space or
slash in a command
argument, as in a Unix file name, put single or double quotes around the
command argument.

The \texttt{open log} command will save everything you see on the
screen and is useful to help you recover should something go wrong in a
proof, or if you want to document a bug.

If a command responds with more than a screenful, you will be
prompted to \texttt{<return> to continue, Q to quit, or S to scroll to
end}.  \texttt{Q} or \texttt{q} (not case-sensitive) will complete the
command internally but will suppress further output until the next
\texttt{MM>} prompt.  \texttt{s} will suppress further pausing until the
next \texttt{MM>} prompt.  After the first screen, you are also
presented with the choice of \texttt{b} to go back a screenful.  Note
that \texttt{b} may also be entered at the \texttt{MM>} prompt
immediately after a command to scroll back through the output of that
command.

A command line enclosed in quotes is executed by your operating system.
See Section~\ref{oscmd}.

{\em Warning:} Pressing {\sc ctrl-c} will abort the Metamath program
unconditionally.  This means any unsaved work will be lost.


\subsection{\texttt{exit} Command}\index{\texttt{exit} command}

Syntax:  \texttt{exit} [\texttt{/force}]

This command exits from Metamath.  If there have been changes to the
source with the \texttt{save proof} or \texttt{save new{\char`\_}proof}
commands, you will be given an opportunity to \texttt{write source} to
permanently save the changes.

In Proof Assistant\index{Proof Assistant} mode, the \texttt{exit} command will
return to the \verb/MM>/ prompt. If there were changes to the proof, you will
be given an opportunity to \texttt{save new{\char`\_}proof}.

The \texttt{quit} command is a synonym for \texttt{exit}.

Optional qualifier:
    \texttt{/force} - Do not prompt if changes were not saved.  This qualifier is
        useful in \texttt{submit} command files (Section~\ref{sbmt})
        to ensure predictable behavior.





\subsection{\texttt{open log} Command}\index{\texttt{open log} command}
Syntax:  \texttt{open log} {\em file-name}

This command will open a log file that will store everything you see on
the screen.  It is useful to help recovery from a mistake in a long Proof
Assistant session, or to document bugs.\index{Metamath!bugs}

The log file can be closed with \texttt{close log}.  It will automatically be
closed upon exiting Metamath.



\subsection{\texttt{close log} Command}\index{\texttt{close log} command}
Syntax:  \texttt{close log}

The \texttt{close log} command closes a log file if one is open.  See
also \texttt{open log}.




\subsection{\texttt{submit} Command}\index{\texttt{submit} command}\label{sbmt}
Syntax:  \texttt{submit} {\em filename}

This command causes further command lines to be taken from the specified
file.  Note that any line beginning with an exclamation point (\texttt{!}) is
treated as a comment (i.e.\ ignored).  Also note that the scrolling
of the screen output is continuous, so you may want to open a log file
(see \texttt{open log}) to record the results that fly by on the screen.
After the lines in the file are exhausted, Metamath returns to its
normal user interface mode.

Currently, the \texttt{submit} command is not recursive.  In other words,
\texttt{submit} commands are not allowed inside a command file.


\subsection{\texttt{erase} Command}\index{\texttt{erase} command}
Syntax:  \texttt{erase}

This command will reset Metamath to its starting state, deleting any
data\-base that was \texttt{read} in.
 If there have been changes to the
source with the \texttt{save proof} or \texttt{save new{\char`\_}proof}
commands, you will be given an opportunity to \texttt{write source} to
permanently save the changes.



\subsection{\texttt{set echo} Command}\index{\texttt{set echo} command}
Syntax:  \texttt{set echo on} or \texttt{set echo off}

The \texttt{set echo on} command will cause command lines to be echoed with any
abbreviations expanded.  While learning the Metamath commands, this
feature will show you the exact command that your abbreviated input
corresponds to.



\subsection{\texttt{set scroll} Command}\index{\texttt{set scroll} command}
Syntax:  \texttt{set scroll prompted} or \texttt{set scroll continuous}

The Metamath command line interface starts off in the \texttt{prompted} mode,
which means that you will be prompted to continue or quit after each
full screen in a long listing.  In \texttt{continuous} mode, long listings will be
scrolled without pausing.

% LaTeX bug? (1) \texttt{\_} puts out different character than
% \texttt{{\char`\_}}
%  = \verb$_$  (2) \texttt{{\char`\_}} puts out garbage in \subsection
%  argument
\subsection{\texttt{set width} Command}\index{\texttt{set
width} command}
Syntax:  \texttt{set width} {\em number}

Metamath assumes the width of your screen is 79 characters (chosen
because the Command Prompt in Windows XP has a wrapping bug at column
80).  If your screen is wider or narrower, this command allows you to
change this default screen width.  A larger width is advantageous for
logging proofs to an output file to be printed on a wide printer.  A
smaller width may be necessary on some terminals; in this case, the
wrapping of the information messages may sometimes seem somewhat
unnatural, however.  In \LaTeX\index{latex@{\LaTeX}!characters per line},
there is normally a maximum of 61 characters per line with typewriter
font.  (The examples in this book were produced with 61 characters per
line.)

\subsection{\texttt{set height} Command}\index{\texttt{set
height} command}
Syntax:  \texttt{set height} {\em number}

Metamath assumes your screen height is 24 lines of characters.  If your
screen is taller or shorter, this command lets you to change the number
of lines at which the display pauses and prompts you to continue.

\subsection{\texttt{beep} Command}\index{\texttt{beep} command}

Syntax:  \texttt{beep}

This command will produce a beep.  By typing it ahead after a
long-running command has started, it will alert you that the command is
finished.  For convenience, \texttt{b} is an abbreviation for
\texttt{beep}.

Note:  If \texttt{b} is typed at the \texttt{MM>} prompt immediately
after the end of a multiple-page display paged with ``\texttt{Press
<return> for more}...'' prompts, then the \texttt{b} will back up to the
previous page rather than perform the \texttt{beep} command.
In that case you must type the unabbreviated \texttt{beep} form
of the command.

\subsection{\texttt{more} Command}\index{\texttt{more} command}

Syntax:  \texttt{more} {\em filename}

This command will display the contents of an {\sc ascii} file on your
screen.  (This command is provided for convenience but is not very
powerful.  See Section~\ref{oscmd} to invoke your operating system's
command to do this, such as the \texttt{more} command in Unix.)

\subsection{Operating System Commands}\index{operating system
command}\label{oscmd}

A line enclosed in single or double quotes will be executed by your
computer's operating system if it has a command line interface.  For
example, on a {\sc vax/vms} system,
\verb/MM> 'dir'/
will print disk directory contents.  Note that this feature will not
work on the Macintosh prior to Mac OS X, which does not have a command
line interface.

For your convenience, the trailing quote is optional.

\subsection{Size Limitations in Metamath}

In general, there are no fixed, predefined limits\index{Metamath!memory
limits} on how many labels, tokens\index{token}, statements, etc.\ that
you may have in a database file.  The Metamath program uses 32-bit
variables (64-bit on 64-bit CPUs) as indices for almost all internal
arrays, which are allocated dynamically as needed.



\section{Reading and Writing Files}

The following commands create new files:  the \texttt{open} commands;
the \texttt{write} commands; the \texttt{/html},
\texttt{/alt{\char`\_}html}, \texttt{/brief{\char`\_}html},
\texttt{/brief{\char`\_}alt{\char`\_}html} qualifiers of \texttt{show
statement}, and \texttt{midi}.  The following commands append to files
previously opened:  the \texttt{/tex} qualifier of \texttt{show proof}
and \texttt{show new{\char`\_}proof}; the \texttt{/tex} and
\texttt{/simple{\char`\_}tex} qualifiers of \texttt{show statement}; the
\texttt{close} commands; and all screen dialog between \texttt{open log}
and \texttt{close log}.

The commands that create new files will not overwrite an existing {\em
filename} but will rename the existing one to {\em
filename}\texttt{{\char`\~}1}.  An existing {\em
filename}\texttt{{\char`\~}1} is renamed {\em
filename}\texttt{{\char`\~}2}, etc.\ up to {\em
filename}\texttt{{\char`\~}9}.  An existing {\em
filename}\texttt{{\char`\~}9} is deleted.  This makes recovery from
mistakes easier but also will clutter up your directory, so occasionally
you may want to clean up (delete) these \texttt{{\char`\~}}$n$ files.


\subsection{\texttt{read} Command}\index{\texttt{read} command}
Syntax:  \texttt{read} {\em file-name} [\texttt{/verify}]

This command will read in a Metamath language source file and any included
files.  Normally it will be the first thing you do when entering Metamath.
Statement syntax is checked, but proof syntax is not checked.
Note that the file name may be enclosed in single or double quotes;
this is useful if the file name contains slashes, as might be the case
under Unix.

If you are getting an ``\texttt{?Expected VERIFY}'' error
when trying to read a Unix file name with slashes, you probably haven't
quoted it.\index{Unix file names}\index{file names!Unix}

If you are prompted for the file name (by pressing {\em return}
 after \texttt{read})
you should {\em not} put quotes around it, even if it is a Unix file name
with slashes.

Optional command qualifier:

    \texttt{/verify} - Verify all proofs as the database is read in.  This
         qualifier will slow down reading in the file.  See \texttt{verify
         proof} for more information on file error-checking.

See also \texttt{erase}.



\subsection{\texttt{write source} Command}\index{\texttt{write source} command}
Syntax:  \texttt{write source} {\em filename}
[\texttt{/rewrap}]
[\texttt{/split}]
[\texttt{/keep\_includes}]
[\texttt{/no\_versioning}]

This command will write the contents of a Metamath\index{database}
database into a file.\index{source file}

Optional command qualifiers:

\texttt{/rewrap} -
Reformats statements and comments according to the
convention used in the set.mm database.
It unwraps the
lines in the comment before each \texttt{\$a} and \texttt{\$p} statement,
then it
rewraps the line.  You should compare the output to the original
to make sure that the desired effect results; if not, go back to
the original source.  The wrapped line length honors the
\texttt{set width}
parameter currently in effect.  Note:  Text
enclosed in \texttt{<HTML>}...\texttt{</HTML>} tags is not modified by the
\texttt{/rewrap} qualifier.
Proofs themselves are not reformatted;
use \texttt{save proof * / compressed} to do that.
An isolated tilde (\~{}) is always kept on the same line as the following
symbol, so you can find all comment references to a symbol by
searching for \~{} followed by a space and that symbol
(this is useful for finding cross-references).
Incidentally, \texttt{save proof} also honors the \texttt{set width}
parameter currently in effect.

\texttt{/split} - Files included in the source using the expression
\$[ \textit{inclfile} \$] will be
written into separate files instead of being included in a single output
file.  The name of each separately written included file will be the
\textit{inclfile} argument of its inclusion command.

\texttt{/keep\_includes} - If a source file has includes but is written as a
single file by omitting \texttt{/split}, by default the included files will
be deleted (actually just renamed with a \char`\~1 suffix unless
\texttt{/no\_versioning} is specified) to prevent the possibly confusing
source duplication in both the output file and the included file.
The \texttt{/keep\_includes} qualifier will prevent this deletion.

\texttt{/no\_versioning} - Backup files suffixed with \char`\~1 are not created.


\section{Showing Status and Statements}



\subsection{\texttt{show settings} Command}\index{\texttt{show settings} command}
Syntax:  \texttt{show settings}

This command shows the state of various parameters.

\subsection{\texttt{show memory} Command}\index{\texttt{show memory} command}
Syntax:  \texttt{show memory}

This command shows the available memory left.  It is not meaningful
on most modern operating systems,
which have virtual memory.\index{Metamath!memory usage}


\subsection{\texttt{show labels} Command}\index{\texttt{show labels} command}
Syntax:  \texttt{show labels} {\em label-match} [\texttt{/all}]
   [\texttt{/linear}]

This command shows the labels of \texttt{\$a} and \texttt{\$p}
statements that match {\em label-match}.  A \verb$*$ in {label-match}
matches zero or more characters.  For example, \verb$*abc*def$ will match all
labels containing \verb$abc$ and ending with \verb$def$.

Optional command qualifiers:

   \texttt{/all} - Include matches for \texttt{\$e} and \texttt{\$f}
   statement labels.

   \texttt{/linear} - Display only one label per line.  This can be useful for
       building scripts in conjunction with the utilities under the
       \texttt{tools}\index{\texttt{tools} command} command.



\subsection{\texttt{show statement} Command}\index{\texttt{show statement} command}
Syntax:  \texttt{show statement} {\em label-match} [{\em qualifiers} (see below)]

This command provides information about a statement.  Only statements
that have labels (\texttt{\$f}\index{\texttt{\$f} statement},
\texttt{\$e}\index{\texttt{\$e} statement},
\texttt{\$a}\index{\texttt{\$a} statement}, and
\texttt{\$p}\index{\texttt{\$p} statement}) may be specified.
If {\em label-match}
contains wildcard (\verb$*$) characters, all matching statements will be
displayed in the order they occur in the database.

Optional qualifiers (only one qualifier at a time is allowed):

    \texttt{/comment} - This qualifier includes the comment that immediately
       precedes the statement.

    \texttt{/full} - Show complete information about each statement,
       and show all
       statements matching {\em label} (including \texttt{\$e}
       and \texttt{\$f} statements).

    \texttt{/tex} - This qualifier will write the statement information to the
       \LaTeX\ file previously opened with \texttt{open tex}.  See
       Section~\ref{texout}.

    \texttt{/simple{\char`\_}tex} - The same as \texttt{/tex}, except that
       \LaTeX\ macros are not used for formatting equations, allowing easier
       manual edits of the output for slide presentations, etc.

    \texttt{/html}\index{html generation@{\sc html} generation},
       \texttt{/alt{\char`\_}html}, \texttt{/brief{\char`\_}html},
       \texttt{/brief{\char`\_}alt{\char`\_}html} -
       These qualifiers invoke a special mode of
       \texttt{show statement} that
       creates a web page for the statement.  They may not be used with
       any other qualifier.  See Section~\ref{htmlout} or
       \texttt{help html} in the program.


\subsection{\texttt{search} Command}\index{\texttt{search} command}
Syntax:  search {\em label-match}
\texttt{"}{\em symbol-match}{\tt}" [\texttt{/all}] [\texttt{/comments}]
[\texttt{/join}]

This command searches all \texttt{\$a} and \texttt{\$p} statements
matching {\em label-match} for occurrences of {\em symbol-match}.  A
\verb@*@ in {\em label-match} matches any label character.  A \verb@$*@
in {\em symbol-match} matches any sequence of symbols.  The symbols in
{\em symbol-match} must be separated by white space.  The quotes
surrounding {\em symbol-match} may be single or double quotes.  For
example, \texttt{search b}\verb@* "-> $* ch"@ will list all statements
whose labels begin with \texttt{b} and contain the symbols \verb@->@ and
\texttt{ch} surrounding any symbol sequence (including no symbol
sequence).  The wildcards \texttt{?} and \texttt{\$?} are also available
to match individual characters in labels and symbols respectively; see
\texttt{help search} in the Metamath program for details on their usage.

Optional command qualifiers:

    \texttt{/all} - Also search \texttt{\$e} and \texttt{\$f} statements.

    \texttt{/comments} - Search the comment that immediately precedes each
        label-matched statement for {\em symbol-match}.  In this case
        {\em symbol-match} is an arbitrary, non-case-sensitive character
        string.  Quotes around {\em symbol-match} are optional if there
        is no ambiguity.

    \texttt{/join} - In the case of a \texttt{\$a} or \texttt{\$p} statement,
	prepend its \texttt{\$e}
	hypotheses for searching. The
	\texttt{/join} qualifier has no effect in \texttt{/comments} mode.

\section{Displaying and Verifying Proofs}


\subsection{\texttt{show proof} Command}\index{\texttt{show proof} command}
Syntax:  \texttt{show proof} {\em label-match} [{\em qualifiers} (see below)]

This command displays the proof of the specified
\texttt{\$p}\index{\texttt{\$p} statement} statement in various formats.
The {\em label-match} may contain wildcard (\verb@$*@) characters to match
multiple statements.  Without any qualifiers, only the logical steps
will be shown (i.e.\ syntax construction steps will be omitted), in an
indented format.

Most of the time, you will use
    \texttt{show proof} {\em label}
to see just the proof steps corresponding to logical inferences.

Optional command qualifiers:

    \texttt{/essential} - The proof tree is trimmed of all
        \texttt{\$f}\index{\texttt{\$f} statement} hypotheses before
        being displayed.  (This is the default, and it is redundant to
        specify it.)

    \texttt{/all} - the proof tree is not trimmed of all \texttt{\$f} hypotheses before
        being displayed.  \texttt{/essential} and \texttt{/all} are mutually exclusive.

    \texttt{/from{\char`\_}step} {\em step} - The display starts at the specified
        step.  If
        this qualifier is omitted, the display starts at the first step.

    \texttt{/to{\char`\_}step} {\em step} - The display ends at the specified
        step.  If this
        qualifier is omitted, the display ends at the last step.

    \texttt{/tree{\char`\_}depth} {\em number} - Only
         steps at less than the specified proof
        tree depth are displayed.  Sometimes useful for obtaining an overview of
        the proof.

    \texttt{/reverse} - The steps are displayed in reverse order.

    \texttt{/renumber} - When used with \texttt{/essential}, the steps are renumbered
        to correspond only to the essential steps.

    \texttt{/tex} - The proof is converted to \LaTeX\ and\index{latex@{\LaTeX}}
        stored in the file opened
        with \texttt{open tex}.  See Section~\ref{texout} or
        \texttt{help tex} in the program.

    \texttt{/lemmon} - The proof is displayed in a non-indented format known
        as Lemmon style, with explicit previous step number references.
        If this qualifier is omitted, steps are indented in a tree format.

    \texttt{/start{\char`\_}column} {\em number} - Overrides the default column
        (16)
        at which the formula display starts in a Lemmon-style display.  May be
        used only in conjunction with \texttt{/lemmon}.

    \texttt{/normal} - The proof is displayed in normal format suitable for
        inclusion in a Metamath source file.  May not be used with any other
        qualifier.

    \texttt{/compressed} - The proof is displayed in compressed format
        suitable for inclusion in a Metamath source file.  May not be used with
        any other qualifier.

    \texttt{/statement{\char`\_}summary} - Summarizes all statements (like a
        brief \texttt{show statement})
        used by the proof.  It may not be used with any other qualifier
        except \texttt{/essential}.

    \texttt{/detailed{\char`\_}step} {\em step} - Shows the details of what is
        happening at
        a specific proof step.  May not be used with any other qualifier.
        The {\em step} is the step number shown when displaying a
        proof without the \texttt{/renumber} qualifier.


\subsection{\texttt{show usage} Command}\index{\texttt{show usage} command}
Syntax:  \texttt{show usage} {\em label-match} [\texttt{/recursive}]

This command lists the statements whose proofs make direct reference to
the statement specified.

Optional command qualifier:

    \texttt{/recursive} - Also include statements whose proofs ultimately
        depend on the statement specified.



\subsection{\texttt{show trace\_back} Command}\index{\texttt{show
       trace{\char`\_}back} command}
Syntax:  \texttt{show trace{\char`\_}back} {\em label-match} [\texttt{/essential}] [\texttt{/axioms}]
    [\texttt{/tree}] [\texttt{/depth} {\em number}]

This command lists all statements that the proof of the \texttt{\$p}
statement(s) specified by {\em label-match} depends on.

Optional command qualifiers:

    \texttt{/essential} - Restrict the trace-back to \texttt{\$e}
        \index{\texttt{\$e} statement} hypotheses of proof trees.

    \texttt{/axioms} - List only the axioms that the proof ultimately depends on.

    \texttt{/tree} - Display the trace-back in an indented tree format.

    \texttt{/depth} {\em number} - Restrict the \texttt{/tree} trace-back to the
        specified indentation depth.

    \texttt{/count{\char`\_}steps} - Count the number of steps the proof has
       all the way back to axioms.  If \texttt{/essential} is specified,
       expansions of variable-type hypotheses (syntax constructions) are not counted.

\subsection{\texttt{verify proof} Command}\index{\texttt{verify proof} command}
Syntax:  \texttt{verify proof} {\em label-match} [\texttt{/syntax{\char`\_}only}]

This command verifies the proofs of the specified statements.  {\em
label-match} may contain wild card characters (\texttt{*}) to verify
more than one proof; for example \verb/*abc*def/ will match all labels
containing \texttt{abc} and ending with \texttt{def}.
The command \texttt{verify proof *} will verify all proofs in the database.

Optional command qualifier:

    \texttt{/syntax{\char`\_}only} - This qualifier will perform a check of syntax
        and RPN
        stack violations only.  It will not verify that the proof is
        correct.  This qualifier is useful for quickly determining which
        proofs are incomplete (i.e.\ are under development and have \texttt{?}'s
        in them).

{\em Note:} \texttt{read}, followed by \texttt{verify proof *}, will ensure
 the database is free
from errors in the Metamath language but will not check the markup notation
in comments.
You can also check the markup notation by running \texttt{verify markup *}
as discussed in Section~\ref{verifymarkup}; see also the discussion
on generating {\sc HTML} in Section~\ref{htmlout}.

\subsection{\texttt{verify markup} Command}\index{\texttt{verify markup} command}\label{verifymarkup}
Syntax:  \texttt{verify markup} {\em label-match}
[\texttt{/date{\char`\_}skip}]
[\texttt{/top{\char`\_}date{\char`\_}skip}]
[\texttt{/file{\char`\_}skip}]
[\texttt{/verbose}]

This command checks comment markup and other informal conventions we have
adopted.  It error-checks the latexdef, htmldef, and althtmldef statements
in the \texttt{\$t} statement of a Metamath source file.\index{error checking}
It error-checks any \texttt{`}...\texttt{`},
\texttt{\char`\~}~\textit{label},
and bibliographic markups in statement descriptions.
It checks that
\texttt{\$p} and \texttt{\$a} statements
have the same content when their labels start with
``ax'' and ``ax-'' respectively but are otherwise identical, for example
ax4 and ax-4.
It also verifies the date consistency of ``(Contributed by...),''
``(Revised by...),'' and ``(Proof shortened by...)'' tags in the comment
above each \texttt{\$a} and \texttt{\$p} statement.

Optional command qualifiers:

    \texttt{/date{\char`\_}skip} - This qualifier will
        skip date consistency checking,
        which is usually not required for databases other than
	\texttt{set.mm}.

    \texttt{/top{\char`\_}date{\char`\_}skip} - This qualifier will check date consistency except
        that the version date at the top of the database file will not
        be checked.  Only one of
        \texttt{/date{\char`\_}skip} and
        \texttt{/top{\char`\_}date{\char`\_}skip} may be
        specified.

    \texttt{/file{\char`\_}skip} - This qualifier will skip checks that require
        external files to be present, such as checking GIF existence and
        bibliographic links to mmset.html or equivalent.  It is useful
        for doing a quick check from a directory without these files.

    \texttt{/verbose} - Provides more information.  Currently it provides a list
        of axXXX vs. ax-XXX matches.

\subsection{\texttt{save proof} Command}\index{\texttt{save proof} command}
Syntax:  \texttt{save proof} {\em label-match} [\texttt{/normal}]
   [\texttt{/compressed}]

The \texttt{save proof} command will reformat a proof in one of two formats and
replace the existing proof in the source buffer\index{source
buffer}.  It is useful for
converting between proof formats.  Note that a proof will not be
permanently saved until a \texttt{write source} command is issued.

Optional command qualifiers:

    \texttt{/normal} - The proof is saved in the normal format (i.e., as a
        sequence
        of labels, which is the defined format of the basic Metamath
        language).\index{basic language}  This is the default format that
        is used if a qualifier
        is omitted.

    \texttt{/compressed} - The proof is saved in the compressed format which
        reduces storage requirements for a database.
        See Appendix~\ref{compressed}.




\section{Creating Proofs}\label{pfcommands}\index{Proof Assistant}

Before using the Proof Assistant, you must add a \texttt{\$p} to your
source file (using a text editor) containing the statement you want to
prove.  Its proof should consist of a single \texttt{?}, meaning
``unknown step.''  Example:
\begin{verbatim}
equid $p x = x $= ? $.
\end{verbatim}

To enter the Proof assistant, type \texttt{prove} {\em label}, e.g.
\texttt{prove equid}.  Metamath will respond with the \texttt{MM-PA>}
prompt.

Proofs are created working backwards from the statement being proved,
primarily using a series of \texttt{assign} commands.  A proof is
complete when all steps are assigned to statements and all steps are
unified and completely known.  During the creation of a proof, Metamath
will allow only operations that are legal based on what is known up to
that point.  For example, it will not allow an \texttt{assign} of a
statement that cannot be unified with the unknown proof step being
assigned.

{\em Important:}
The Proof Assistant is
{\em not} a tool to help you discover proofs.  It is just a tool to help
you add them to the database.  For a tutorial read
Section~\ref{frstprf}.
To practice using the Proof Assistant, you may
want to \texttt{prove} an existing theorem, then delete all steps with
\texttt{delete all}, then re-create it with the Proof Assistant while
looking at its proof display (before deletion).
You might want to figure out your first few proofs completely
and write them down by hand, before using the Proof Assistant, though
not everyone finds that effective.

{\em Important:}
The \texttt{undo} command if very helpful when entering a proof, because
it allows you to undo a previously-entered step.
In addition, we suggest that you
keep track of your work with a log file (\texttt{open
log}) and save it frequently (\texttt{save new{\char`\_}proof},
\texttt{write source}).
You can use \texttt{delete} to reverse an \texttt{assign}.
You can also do \texttt{delete floating{\char`\_}hypotheses}, then
\texttt{initialize all}, then \texttt{unify all /interactive} to
reinitialize bad unifications made accidentally or by bad
\texttt{assign}s.  You cannot reverse a \texttt{delete} except by
a relevant \texttt{undo} or using
\texttt{exit /force} then reentering the Proof Assistant to recover from
the last \texttt{save new{\char`\_}proof}.

The following commands are available in the Proof Assistant (at the
\texttt{MM-PA>} prompt) to help you create your proof.  See the
individual commands for more detail.

\begin{itemize}
\item[]
    \texttt{show new{\char`\_}proof} [\texttt{/all},...] - Displays the
        proof in progress.  You will use this command a lot; see \texttt{help
        show new{\char`\_}proof} to become familiar with its qualifiers.  The
        qualifiers \texttt{/unknown} and \texttt{/not{\char`\_}unified} are
        useful for seeing the work remaining to be done.  The combination
        \texttt{/all/unknown} is useful for identifying dummy variables that must be
        assigned, or attempts to use illegal syntax, when \texttt{improve all}
        is unable to complete the syntax constructions.  Unknown variables are
        shown as \texttt{\$1}, \texttt{\$2},...
\item[]
    \texttt{assign} {\em step} {\em label} - Assigns an unknown {\em step}
        number with the statement
        specified by {\em label}.
\item[]
    \texttt{let variable} {\em variable}
        \texttt{= "}{\em symbol sequence}\texttt{"}
          - Forces a symbol
        sequence to replace an unknown variable (such as \texttt{\$1}) in a proof.
        It is useful
        for helping difficult unifications, and it is necessary when you have
        dummy variables that eventually must be assigned a name.
\item[]
    \texttt{let step} {\em step} \texttt{= "}{\em symbol sequence}\texttt{"} -
          Forces a symbol sequence
        to replace the contents of a proof step, provided it can be
        unified with the existing step contents.  (I rarely use this.)
\item[]
    \texttt{unify step} {\em step} (or \texttt{unify all}) - Unifies
        the source and target of
        a step.  If you specify a specific step, you will be prompted
        to select among the unifications that are possible.  If you
        specify \texttt{all}, all steps with unique unifications, but only
        those steps, will be
        unified.  \texttt{unify all /interactive} goes through all non-unified
        steps.
\item[]
    \texttt{initialize} {\em step} (or \texttt{all}) - De-unifies the target and source of
        a step (or all steps), as well as the hypotheses of the source,
        and makes all variables in the source unknown.  Useful to recover from
        an \texttt{assign} or \texttt{let} mistake that
        resulted in incorrect unifications.
\item[]
    \texttt{delete} {\em step} (or \texttt{all} or \texttt{floating{\char`\_}hypotheses}) -
        Deletes the specified
        step(s).  \texttt{delete floating{\char`\_}hypotheses}, then \texttt{initialize all}, then
        \texttt{unify all /interactive} is useful for recovering from mistakes
        where incorrect unifications assigned wrong math symbol strings to
        variables.
\item[]
    \texttt{improve} {\em step} (or \texttt{all}) -
      Automatically creates a proof for steps (with no unknown variables)
      whose proof requires no statements with \texttt{\$e} hypotheses.  Useful
      for filling in proofs of \texttt{\$f} hypotheses.  The \texttt{/depth}
      qualifier will also try statements whose \texttt{\$e} hypotheses contain
      no new variables.  {\em Warning:} Save your work (with \texttt{save
      new{\char`\_}proof} then \texttt{write source}) before using
      \texttt{/depth = 2} or greater, since the search time grows
      exponentially and may never terminate in a reasonable time, and you
      cannot interrupt the search.  I have found that it is rare for
      \texttt{/depth = 3} or greater to be useful.
 \item[]
    \texttt{save new{\char`\_}proof} - Saves the proof in progress in the program's
        internal database buffer.  To save it permanently into the database file,
        use \texttt{write source} after
        \texttt{save new{\char`\_}proof}.  To revert to the last
        \texttt{save new{\char`\_}proof},
        \texttt{exit /force} from the Proof Assistant then re-enter the Proof
        Assistant.
 \item[]
    \texttt{match step} {\em step} (or \texttt{match all}) - Shows what
        statements are
        possibilities for the \texttt{assign} statement. (This command
        is not very
        useful in its present form and hopefully will be improved
        eventually.  In the meantime, use the \texttt{search} statement for
        candidates matching specific math token combinations.)
 \item[]
 \texttt{minimize{\char`\_}with}\index{\texttt{minimize{\char`\_}with} command}
% 3/10/07 Note: line-breaking the above results in duplicate index entries
     - After a proof is complete, this command will attempt
        to match other database theorems to the proof to see if the proof
        size can be reduced as a result.  See \texttt{help
        minimize{\char`\_}with} in the
        Metamath program for its usage.
 \item[]
 \texttt{undo}\index{\texttt{undo} command}
    - Undo the effect of a proof-changing command (all but the
      \texttt{show} and \texttt{save} commands above).
 \item[]
 \texttt{redo}\index{\texttt{redo} command}
    - Reverse the previous \texttt{undo}.
\end{itemize}

The following commands set parameters that may be relevant to your proof.
Consult the individual \texttt{help set}... commands.
\begin{itemize}
   \item[] \texttt{set unification{\char`\_}timeout}
 \item[]
    \texttt{set search{\char`\_}limit}
  \item[]
    \texttt{set empty{\char`\_}substitution} - note that default is \texttt{off}
\end{itemize}

Type \texttt{exit} to exit the \texttt{MM-PA>}
 prompt and get back to the \texttt{MM>} prompt.
Another \texttt{exit} will then get you out of Metamath.



\subsection{\texttt{prove} Command}\index{\texttt{prove} command}
Syntax:  \texttt{prove} {\em label}

This command will enter the Proof Assistant\index{Proof Assistant}, which will
allow you to create or edit the proof of the specified statement.
The command-line prompt will change from \texttt{MM>} to \texttt{MM-PA>}.

Note:  In the present version of
Metamath\index{Metamath!limitations of version 0.07.30}, the Proof
Assistant does not verify that \texttt{\$d}\index{\texttt{\$d}
statement} restrictions are met as a proof is being built.  After you
have completed a proof, you should type \texttt{save new{\char`\_}proof}
followed by \texttt{verify proof} {\em label} (where {\em label} is the
statement you are proving with the \texttt{prove} command) to verify the
\texttt{\$d} restrictions.

See also: \texttt{exit}

\subsection{\texttt{set unification\_timeout} Command}\index{\texttt{set
unification{\char`\_}timeout} command}
Syntax:  \texttt{set unification{\char`\_}timeout} {\em number}

(This command is available outside the Proof Assistant but affects the
Proof Assistant\index{Proof Assistant} only.)

Sometimes the Proof Assistant will inform you that a unification
time-out occurred.  This may happen when you try to \texttt{unify}
formulas with many temporary variables\index{temporary variable}
(\texttt{\$1}, \texttt{\$2}, etc.), since the time to compute all possible
unifications may grow exponentially with the number of variables.  If
you want Metamath to try harder (and you're willing to wait longer) you
may increase this parameter.  \texttt{show settings} will show you the
current value.

\subsection{\texttt{set empty\_substitution} Command}\index{\texttt{set
empty{\char`\_}substitution} command}
% These long names can't break well in narrow mode, and even "sloppy"
% is not enough. Work around this by not demanding justification.
\begin{flushleft}
Syntax:  \texttt{set empty{\char`\_}substitution on} or \texttt{set
empty{\char`\_}substitution off}
\end{flushleft}

(This command is available outside the Proof Assistant but affects the
Proof Assistant\index{Proof Assistant} only.)

The Metamath language allows variables to be
substituted\index{substitution!variable}\index{variable substitution}
with empty symbol sequences\index{empty substitution}.  However, in many
formal systems\index{formal system} this will never happen in a valid
proof.  Allowing for this possibility increases the likelihood of
ambiguous unifications\index{ambiguous
unification}\index{unification!ambiguous} during proof creation.
The default is that
empty substitutions are not allowed; for formal systems requiring them,
you must \texttt{set empty{\char`\_}substitution on}.
(An example where this must be \texttt{on}
would be a system that implements a Deduction Rule and in
which deductions from empty assumption lists would be permissible.  The
MIU-system\index{MIU-system} described in Appendix~\ref{MIU} is another
example.)
Note that empty substitutions are
always permissible in proof verification (VERIFY PROOF...) outside the
Proof Assistant.  (See the MIU system in the Metamath book for an example
of a system needing empty substitutions; another example would be a
system that implements a Deduction Rule and in which deductions from
empty assumption lists would be permissible.)

It is better to leave this \texttt{off} when working with \texttt{set.mm}.
Note that this command does not affect the way proofs are verified with
the \texttt{verify proof} command.  Outside of the Proof Assistant,
substitution of empty sequences for math symbols is always allowed.

\subsection{\texttt{set search\_limit} Command}\index{\texttt{set
search{\char`\_}limit} command} Syntax:  \texttt{set search{\char`\_}limit} {\em
number}

(This command is available outside the Proof Assistant but affects the
Proof Assistant\index{Proof Assistant} only.)

This command sets a parameter that determines when the \texttt{improve} command
in Proof Assistant mode gives up.  If you want \texttt{improve} to search harder,
you may increase it.  The \texttt{show settings} command tells you its current
value.


\subsection{\texttt{show new\_proof} Command}\index{\texttt{show
new{\char`\_}proof} command}
Syntax:  \texttt{show new{\char`\_}proof} [{\em
qualifiers} (see below)]

This command (available only in Proof Assistant mode) displays the proof
in progress.  It is identical to the \texttt{show proof} command, except that
there is no statement argument (since it is the statement being proved) and
the following qualifiers are not available:

    \texttt{/statement{\char`\_}summary}

    \texttt{/detailed{\char`\_}step}

Also, the following additional qualifiers are available:

    \texttt{/unknown} - Shows only steps that have no statement assigned.

    \texttt{/not{\char`\_}unified} - Shows only steps that have not been unified.

Note that \texttt{/essential}, \texttt{/depth}, \texttt{/unknown}, and
\texttt{/not{\char`\_}unified} may be
used in any combination; each of them effectively filters out additional
steps from the proof display.

See also:  \texttt{show proof}






\subsection{\texttt{assign} Command}\index{\texttt{assign} command}
Syntax:   \texttt{assign} {\em step} {\em label} [\texttt{/no{\char`\_}unify}]

   and:   \texttt{assign first} {\em label}

   and:   \texttt{assign last} {\em label}


This command, available in the Proof Assistant only, assigns an unknown
step (one with \texttt{?} in the \texttt{show new{\char`\_}proof}
listing) with the statement specified by {\em label}.  The assignment
will not be allowed if the statement cannot be unified with the step.

If \texttt{last} is specified instead of {\em step} number, the last
step that is shown by \texttt{show new{\char`\_}proof /unknown} will be
used.  This can be useful for building a proof with a command file (see
\texttt{help submit}).  It also makes building proofs faster when you know
the assignment for the last step.

If \texttt{first} is specified instead of {\em step} number, the first
step that is shown by \texttt{show new{\char`\_}proof /unknown} will be
used.

If {\em step} is zero or negative, the -{\em step}th from last unknown
step, as shown by \texttt{show new{\char`\_}proof /unknown}, will be
used.  \texttt{assign -1} {\em label} will assign the penultimate
unknown step, \texttt{assign -2} {\em label} the antepenultimate, and
\texttt{assign 0} {\em label} is the same as \texttt{assign last} {\em
label}.

Optional command qualifier:

    \texttt{/no{\char`\_}unify} - do not prompt user to select a unification if there is
        more than one possibility.  This is useful for noninteractive
        command files.  Later, the user can \texttt{unify all /interactive}.
        (The assignment will still be automatically unified if there is only
        one possibility and will be refused if unification is not possible.)



\subsection{\texttt{match} Command}\index{\texttt{match} command}
Syntax:  \texttt{match step} {\em step} [\texttt{/max{\char`\_}essential{\char`\_}hyp}
{\em number}]

    and:  \texttt{match all} [\texttt{/essential}]
          [\texttt{/max{\char`\_}essential{\char`\_}hyp} {\em number}]

This command, available in the Proof Assistant only, shows what
statements can be unified with the specified step(s).  {\em Note:} In
its current form, this command is not very useful because of the large
number of matches it reports.  I rarely use it.
It may be enhanced in the future.

Optional command qualifiers:

    \texttt{/max{\char`\_}essential{\char`\_}hyp} {\em number} - filters out
        of the list any statements
        with more than the specified number of
        \texttt{\$e}\index{\texttt{\$e} statement} hypotheses

    \texttt{/essential{\char`\_}only} - in the \texttt{match all} statement, only
        the steps that
        would be listed in the \texttt{show new{\char`\_}proof /essential} display are
        matched.



\subsection{\texttt{let} Command}\index{\texttt{let} command}
Syntax: \texttt{let variable} {\em variable} = \verb/"/{\em symbol-sequence}\verb/"/

  and: \texttt{let step} {\em step} = \verb/"/{\em symbol-sequence}\verb/"/

These commands, available in the Proof Assistant\index{Proof Assistant}
only, assign a temporary variable\index{temporary variable} or unknown
step with a specific symbol sequence.  They are useful in the middle of
creating a proof, when you know what should be in the proof step but the
unification algorithm doesn't yet have enough information to completely
specify the temporary variables.  A ``temporary variable'' is one that
has the form \texttt{\$}{\em nn} in the proof display, such as
\texttt{\$1}, \texttt{\$2}, etc.  The {\em symbol-sequence} may contain
other unknown variables if desired.  Examples:

    \verb/let variable $32 = "A = B"/

    \verb/let variable $32 = "A = $35"/

    \verb/let step 10 = '|- x = x'/

    \verb/let step -2 = "|- ( $7 = ph )"/

Any symbol sequence will be accepted for the \texttt{let variable}
command.  Only those symbol sequences that can be unified with the step
will be accepted for \texttt{let step}.

The \texttt{let} commands ``zap'' the proof with information that can
only be verified when the proof is built up further.  If you make an
error, the command sequence \texttt{delete
floating{\char`\_}hypotheses}, \texttt{initialize all}, and
\texttt{unify all /interactive} will undo a bad \texttt{let} assignment.

If {\em step} is zero or negative, the -{\em step}th from last unknown
step, as shown by \texttt{show new{\char`\_}proof /unknown}, will be
used.  The command \texttt{let step 0} = \verb/"/{\em
symbol-sequence}\verb/"/ will use the last unknown step, \texttt{let
step -1} = \verb/"/{\em symbol-sequence}\verb/"/ the penultimate, etc.
If {\em step} is positive, \texttt{let step} may be used to assign known
(in the sense of having previously been assigned a label with
\texttt{assign}) as well as unknown steps.

Either single or double quotes can surround the {\em symbol-sequence} as
long as they are different from any quotes inside a {\em
symbol-sequence}.  If {\em symbol-sequence} contains both kinds of
quotes, see the instructions at the end of \texttt{help let} in the
Metamath program.


\subsection{\texttt{unify} Command}\index{\texttt{unify} command}
Syntax:  \texttt{unify step} {\em step}

      and:   \texttt{unify all} [\texttt{/interactive}]

These commands, available in the Proof Assistant only, unify the source
and target of the specified step(s). If you specify a specific step, you
will be prompted to select among the unifications that are possible.  If
you specify \texttt{all}, only those steps with unique unifications will be
unified.

Optional command qualifier for \texttt{unify all}:

    \texttt{/interactive} - You will be prompted to select among the
        unifications
        that are possible for any steps that do not have unique
        unifications.  (Otherwise \texttt{unify all} will bypass these.)

See also \texttt{set unification{\char`\_}timeout}.  The default is
100000, but increasing it to 1000000 can help difficult cases.  Manually
assigning some or all of the unknown variables with the \texttt{let
variable} command also helps difficult cases.



\subsection{\texttt{initialize} Command}\index{\texttt{initialize} command}
Syntax:  \texttt{initialize step} {\em step}

    and: \texttt{initialize all}

These commands, available in the Proof Assistant\index{Proof Assistant}
only, ``de-unify'' the target and source of a step (or all steps), as
well as the hypotheses of the source, and makes all variables in the
source and the source's hypotheses unknown.  This command is useful to
help recover from incorrect unifications that resulted from an incorrect
\texttt{assign}, \texttt{let}, or unification choice.  Part or all of
the command sequence \texttt{delete floating{\char`\_}hypotheses},
\texttt{initialize all}, and \texttt{unify all /interactive} will recover
from incorrect unifications.

See also:  \texttt{unify} and \texttt{delete}



\subsection{\texttt{delete} Command}\index{\texttt{delete} command}
Syntax:  \texttt{delete step} {\em step}

   and:      \texttt{delete all} -- {\em Warning: dangerous!}

   and:      \texttt{delete floating{\char`\_}hypotheses}

These commands are available in the Proof Assistant only.  The
\texttt{delete step} command deletes the proof tree section that
branches off of the specified step and makes the step become unknown.
\texttt{delete all} is equivalent to \texttt{delete step} {\em step}
where {\em step} is the last step in the proof (i.e.\ the beginning of
the proof tree).

In most cases the \texttt{undo} command is the best way to undo
a previous step.
An alternative is to salvage your last \texttt{save
new{\char`\_}proof} by exiting and reentering the Proof Assistant.
For this to work, keep a log file open to record your work
and to do \texttt{save new{\char`\_}proof} frequently, especially before
\texttt{delete}.

\texttt{delete floating{\char`\_}hypotheses} will delete all sections of
the proof that branch off of \texttt{\$f}\index{\texttt{\$f} statement}
statements.  It is sometimes useful to do this before an
\texttt{initialize} command to recover from an error.  Note that once a
proof step with a \texttt{\$f} hypothesis as the target is completely
known, the \texttt{improve} command can usually fill in the proof for
that step.  Unlike the deletion of logical steps, \texttt{delete
floating{\char`\_}hypotheses} is a relatively safe command that is
usually easy to recover from.



\subsection{\texttt{improve} Command}\index{\texttt{improve} command}
\label{improve}
Syntax:  \texttt{improve} {\em step} [\texttt{/depth} {\em number}]
                                               [\texttt{/no{\char`\_}distinct}]

   and:   \texttt{improve first} [\texttt{/depth} {\em number}]
                                              [\texttt{/no{\char`\_}distinct}]

   and:   \texttt{improve last} [\texttt{/depth} {\em number}]
                                              [\texttt{/no{\char`\_}distinct}]

   and:   \texttt{improve all} [\texttt{/depth} {\em number}]
                                              [\texttt{/no{\char`\_}distinct}]

These commands, available in the Proof Assistant\index{Proof Assistant}
only, try to find proofs automatically for unknown steps whose symbol
sequences are completely known.  They are primarily useful for filling in
proofs of \texttt{\$f}\index{\texttt{\$f} statement} hypotheses.  The
search will be restricted to statements having no
\texttt{\$e}\index{\texttt{\$e} statement} hypotheses.

\begin{sloppypar} % narrow
Note:  If memory is limited, \texttt{improve all} on a large proof may
overflow memory.  If you use \texttt{set unification{\char`\_}timeout 1}
before \texttt{improve all}, there will usually be sufficient
improvement to easily recover and completely \texttt{improve} the proof
later on a larger computer.  Warning:  Once memory has overflowed, there
is no recovery.  If in doubt, save the intermediate proof (\texttt{save
new{\char`\_}proof} then \texttt{write source}) before \texttt{improve
all}.
\end{sloppypar}

If \texttt{last} is specified instead of {\em step} number, the last
step that is shown by \texttt{show new{\char`\_}proof /unknown} will be
used.

If \texttt{first} is specified instead of {\em step} number, the first
step that is shown by \texttt{show new{\char`\_}proof /unknown} will be
used.

If {\em step} is zero or negative, the -{\em step}th from last unknown
step, as shown by \texttt{show new{\char`\_}proof /unknown}, will be
used.  \texttt{improve -1} will use the penultimate
unknown step, \texttt{improve -2} {\em label} the antepenultimate, and
\texttt{improve 0} is the same as \texttt{improve last}.

Optional command qualifier:

    \texttt{/depth} {\em number} - This qualifier will cause the search
        to include
        statements with \texttt{\$e} hypotheses (but no new variables in
        the \texttt{\$e}
        hypotheses), provided that the backtracking has not exceeded the
        specified depth. {\em Warning:}  Try \texttt{/depth 1},
        then \texttt{2}, then \texttt{3}, etc.
        in sequence because of possible exponential blowups.  Save your
        work before trying \texttt{/depth} greater than \texttt{1}!

    \texttt{/no{\char`\_}distinct} - Skip trial statements that have
        \texttt{\$d}\index{\texttt{\$d} statement} requirements.
        This qualifier will prevent assignments that might violate \texttt{\$d}
        requirements but it also could miss possible legal assignments.

See also: \texttt{set search{\char`\_}limit}

\subsection{\texttt{save new\_proof} Command}\index{\texttt{save
new{\char`\_}proof} command}
Syntax:  \texttt{save new{\char`\_}proof} {\em label} [\texttt{/normal}]
   [\texttt{/compressed}]

The \texttt{save new{\char`\_}proof} command is available in the Proof
Assistant only.  It saves the proof in progress in the source
buffer\index{source buffer}.  \texttt{save new{\char`\_}proof} may be
used to save a completed proof, or it may be used to save a proof in
progress in order to work on it later.  If an incomplete proof is saved,
any user assignments with \texttt{let step} or \texttt{let variable}
will be lost, as will any ambiguous unifications\index{ambiguous
unification}\index{unification!ambiguous} that were resolved manually.
To help make recovery easier, it can be helpful to \texttt{improve all}
before \texttt{save new{\char`\_}proof} so that the incomplete proof
will have as much information as possible.

Note that the proof will not be permanently saved until a \texttt{write
source} command is issued.

Optional command qualifiers:

    \texttt{/normal} - The proof is saved in the normal format (i.e., as a
        sequence of labels, which is the defined format of the basic Metamath
        language).\index{basic language}  This is the default format that
        is used if a qualifier is omitted.

    \texttt{/compressed} - The proof is saved in the compressed format, which
        reduces storage requirements for a database.
        (See Appendix~\ref{compressed}.)


\section{Creating \LaTeX\ Output}\label{texout}\index{latex@{\LaTeX}}

You can generate \LaTeX\ output given the
information in a database.
The database must already include the necessary typesetting information
(see section \ref{tcomment} for how to provide this information).

The \texttt{show statement} and \texttt{show proof} commands each have a
special \texttt{/tex} command qualifier that produces \LaTeX\ output.
(The \texttt{show statement} command also has the
\texttt{/simple{\char`\_}tex} qualifier for output that is easier to
edit by hand.)  Before you can use them, you must open a \LaTeX\ file to
which to send their output.  A typical complete session will use this
sequence of Metamath commands:

\begin{verbatim}
read set.mm
open tex example.tex
show statement a1i /tex
show proof a1i /all/lemmon/renumber/tex
show statement uneq2 /tex
show proof uneq2 /all/lemmon/renumber/tex
close tex
\end{verbatim}

See Section~\ref{mathcomments} for information on comment markup and
Appendix~\ref{ASCII} for information on how math symbol translation is
specified.

To format and print the \LaTeX\ source, you will need the \LaTeX\
program, which is standard on most Linux installations and available for
Windows.  On Linux, in order to create a {\sc pdf} file, you will
typically type at the shell prompt
\begin{verbatim}
$ pdflatex example.tex
\end{verbatim}

\subsection{\texttt{open tex} Command}\index{\texttt{open tex} command}
Syntax:  \texttt{open tex} {\em file-name} [\texttt{/no{\char`\_}header}]

This command opens a file for writing \LaTeX\
source\index{latex@{\LaTeX}} and writes a \LaTeX\ header to the file.
\LaTeX\ source can be written with the \texttt{show proof}, \texttt{show
new{\char`\_}proof}, and \texttt{show statement} commands using the
\texttt{/tex} qualifier.

The mapping to \LaTeX\ symbols is defined in a special comment
containing a \texttt{\$t} token, described in Appendix~\ref{ASCII}.

There is an optional command qualifier:

    \texttt{/no{\char`\_}header} - This qualifier prevents a standard
        \LaTeX\ header and trailer
        from being included with the output \LaTeX\ code.


\subsection{\texttt{close tex} Command}\index{\texttt{close tex} command}
Syntax:  \texttt{close tex}

This command writes a trailer to any \LaTeX\ file\index{latex@{\LaTeX}}
that was opened with \texttt{open tex} (unless
\texttt{/no{\char`\_}header} was used with \texttt{open tex}) and closes
the \LaTeX\ file.


\section{Creating {\sc HTML} Output}\label{htmlout}

You can generate {\sc html} web pages given the
information in a database.
The database must already include the necessary typesetting information
(see section \ref{tcomment} for how to provide this information).
The ability to produce {\sc html} web pages was added in Metamath version
0.07.30.

To create an {\sc html} output file(s) for \texttt{\$a} or \texttt{\$p}
statement(s), use
\begin{quote}
    \texttt{show statement} {\em label-match} \texttt{/html}
\end{quote}
The output file will be named {\em label-match}\texttt{.html}
for each match.  When {\em
label-match} has wildcard (\texttt{*}) characters, all statements with
matching labels will have {\sc html} files produced for them.  Also,
when {\em label-match} has a wildcard (\texttt{*}) character, two additional
files, \texttt{mmdefinitions.html} and \texttt{mmascii.html} will be
produced.  To produce {\em only} these two additional files, you can use
\texttt{?*}, which will not match any statement label, in place of {\em
label-match}.

There are three other qualifiers for \texttt{show statement} that also
generate {\sc HTML} code.  These are \texttt{/alt{\char`\_}html},
\texttt{/brief{\char`\_}html}, and
\texttt{/brief{\char`\_}alt{\char`\_}html}, and are described in the
next section.

The command
\begin{quote}
    \texttt{show statement} {\em label-match} \texttt{/alt{\char`\_}html}
\end{quote}
does the same as \texttt{show statement} {\em label-match} \texttt{/html},
except that the {\sc html} code for the symbols is taken from
\texttt{althtmldef} statements instead of \texttt{htmldef} statements in
the \texttt{\$t} comment.

The command
\begin{verbatim}
show statement * /brief_html
\end{verbatim}
invokes a special mode that just produces definition and theorem lists
accompanied by their symbol strings, in a format suitable for copying and
pasting into another web page (such as the tutorial pages on the
Metamath web site).

Finally, the command
\begin{verbatim}
show statement * /brief_alt_html
\end{verbatim}
does the same as \texttt{show statement * / brief{\char`\_}html}
for the alternate {\sc html}
symbol representation.

A statement's comment can include a special notation that provides a
certain amount of control over the {\sc HTML} version of the comment.  See
Section~\ref{mathcomments} (p.~\pageref{mathcomments}) for the comment
markup features.

The \texttt{write theorem{\char`\_}list} and \texttt{write bibliography}
commands, which are described below, provide as a side effect complete
error checking for all of the features described in this
section.\index{error checking}

\subsection{\texttt{write theorem\_list}
Command}\index{\texttt{write theorem{\char`\_}list} command}
Syntax:  \texttt{write theorem{\char`\_}list}
[\texttt{/theorems{\char`\_}per{\char`\_}page} {\em number}]

This command writes a list of all of the \texttt{\$a} and \texttt{\$p}
statements in the database into a web page file
 called \texttt{mmtheorems.html}.
When additional files are needed, they are called
\texttt{mmtheorems2.html}, \texttt{mmtheorems3.html}, etc.

Optional command qualifier:

    \texttt{/theorems{\char`\_}per{\char`\_}page} {\em number} -
 This qualifier specifies the number of statements to
        write per web page.  The default is 100.

{\em Note:} In version 0.07.30\index{Metamath!limitations of version
0.07.30} of Metamath, the ``Nearby theorems'' links on the individual
web pages presuppose 100 theorems per page when linking to the theorem
list pages.  Therefore the \texttt{/theorems{\char`\_}per{\char`\_}page}
qualifier, if it specifies a number other than 100, will cause the
individual web pages to be out of sync and should not be used to
generate the main theorem list for the web site.  This is expected to be
fixed in a future version.


\subsection{\texttt{write bibliography}\label{wrbib}
Command}\index{\texttt{write bibliography} command}
Syntax:  \texttt{write bibliography} {\em filename}

This command reads an existing {\sc html} bibliographic cross-reference
file, normally called \texttt{mmbiblio.html}, and updates it per the
bibliographic links in the database comments.  The file is updated
between the {\sc html} comment lines \texttt{<!--
{\char`\#}START{\char`\#} -->} and \texttt{<!-- {\char`\#}END{\char`\#}
-->}.  The original input file is renamed to {\em
filename}\texttt{{\char`\~}1}.

A bibliographic reference is indicated with the reference name
in brackets, such as  \texttt{Theorem 3.1 of
[Monk] p.\ 22}.
See Section~\ref{htmlout} (p.~\pageref{htmlout}) for
syntax details.


\subsection{\texttt{write recent\_additions}
Command}\index{\texttt{write recent{\char`\_}additions} command}
Syntax:  \texttt{write recent{\char`\_}additions} {\em filename}
[\texttt{/limit} {\em number}]

This command reads an existing ``Recent Additions'' {\sc html} file,
normally called \texttt{mmrecent.html}, and updates it with the
descriptions of the most recently added theorems to the database.
 The file is updated between
the {\sc html} comment lines \texttt{<!-- {\char`\#}START{\char`\#} -->}
and \texttt{<!-- {\char`\#}END{\char`\#} -->}.  The original input file
is renamed to {\em filename}\texttt{{\char`\~}1}.

Optional command qualifier:

    \texttt{/limit} {\em number} -
 This qualifier specifies the number of most recent theorems to
   write to the output file.  The default is 100.


\section{Text File Utilities}

\subsection{\texttt{tools} Command}\index{\texttt{tools} command}
Syntax:  \texttt{tools}

This command invokes an easy-to-use, general purpose utility for
manipulating the contents of {\sc ascii} text files.  Upon typing
\texttt{tools}, the command-line prompt will change to \texttt{TOOLS>}
until you type \texttt{exit}.  The \texttt{tools} commands can be used
to perform simple, global edits on an input/output file,
such as making a character string substitution on each line, adding a
string to each line, and so on.  A typical use of this utility is
to build a \texttt{submit} input file to perform a common operation on a
list of statements obtained from \texttt{show label} or \texttt{show
usage}.

The actions of most of the \texttt{tools} commands can also be
performed with equivalent (and more powerful) Unix shell commands, and
some users may find those more efficient.  But for Windows users or
users not comfortable with Unix, \texttt{tools} provides an
easy-to-learn alternative that is adequate for most of the
script-building tasks needed to use the Metamath program effectively.

\subsection{\texttt{help} Command (in \texttt{tools})}
Syntax:  \texttt{help}

The \texttt{help} command lists the commands available in the
\texttt{tools} utility, along with a brief description.  Each command,
in turn, has its own help, such as \texttt{help add}.  As with
Metamath's \texttt{MM>} prompt, a complete command can be entered at
once, or just the command word can be typed, causing the program to
prompt for each argument.

\vskip 1ex
\noindent Line-by-line editing commands:

  \texttt{add} - Add a specified string to each line in a file.

  \texttt{clean} - Trim spaces and tabs on each line in a file; convert
         characters.

  \texttt{delete} - Delete a section of each line in a file.

  \texttt{insert} - Insert a string at a specified column in each line of
        a file.

  \texttt{substitute} - Make a simple substitution on each line of the file.

  \texttt{tag} - Like \texttt{add}, but restricted to a range of lines.

  \texttt{swap} - Swap the two halves of each line in a file.

\vskip 1ex
\noindent Other file-processing commands:

  \texttt{break} - Break up (tokenize) a file into a list of tokens (one per
        line).

  \texttt{build} - Build a file with multiple tokens per line from a list.

  \texttt{count} - Count the occurrences in a file of a specified string.

  \texttt{number} - Create a list of numbers.

  \texttt{parallel} - Put two files in parallel.

  \texttt{reverse} - Reverse the order of the lines in a file.

  \texttt{right} - Right-justify lines in a file (useful before sorting
         numbers).

%  \texttt{tag} - Tag edit updates in a program for revision control.

  \texttt{sort} - Sort the lines in a file with key starting at
         specified string.

  \texttt{match} - Extract lines containing (or not) a specified string.

  \texttt{unduplicate} - Eliminate duplicate occurrences of lines in a file.

  \texttt{duplicate} - Extract first occurrence of any line occurring
         more than

   \ \ \    once in a file, discarding lines occurring exactly once.

  \texttt{unique} - Extract lines occurring exactly once in a file.

  \texttt{type} (10 lines) - Display the first few lines in a file.
                  Similar to Unix \texttt{head}.

  \texttt{copy} - Similar to Unix \texttt{cat} but safe (same input
         and output file allowed).

  \texttt{submit} - Run a script containing \texttt{tools} commands.

\vskip 1ex

\noindent Note:
  \texttt{unduplicate}, \texttt{duplicate}, and \texttt{unique} also
 sort the lines as a side effect.


\subsection{Using \texttt{tools} to Build Metamath \texttt{submit}
Scripts}

The \texttt{break} command is typically used to break up a series of
statement labels, such as the output of Metamath's \texttt{show usage},
into one label per line.  The other \texttt{tools} commands can then be
used to add strings before and after each statement label to specify
commands to be performed on the statement.  The \texttt{parallel}
command is useful when a statement label must be mentioned more than
once on a line.

Very often a \texttt{submit} script for Metamath will require multiple
command lines for each statement being processed.  For example, you may
want to enter the Proof Assistant, \texttt{minimize{\char`\_}with} your
latest theorem, \texttt{save} the new proof, and \texttt{exit} the Proof
Assistant.  To accomplish this, you can build a file with these four
commands for each statement on a single line, separating each command
with a designated character such as \texttt{@}.  Then at the end you can
\texttt{substitute} each \texttt{@} with \texttt{{\char`\\}n} to break
up the lines into individual command lines (see \texttt{help
substitute}).


\subsection{Example of a \texttt{tools} Session}

To give you a quick feel for the \texttt{tools} utility, we show a
simple session where we create a file \texttt{n.txt} with 3 lines, add
strings before and after each line, and display the lines on the screen.
You can experiment with the various commands to gain experience with the
\texttt{tools} utility.

\begin{verbatim}
MM> tools
Entering the Text Tools utilities.
Type HELP for help, EXIT to exit.
TOOLS> number
Output file <n.tmp>? n.txt
First number <1>?
Last number <10>? 3
Increment <1>?
TOOLS> add
Input/output file? n.txt
String to add to beginning of each line <>? This is line
String to add to end of each line <>? .
The file n.txt has 3 lines; 3 were changed.
First change is on line 1:
This is line 1.
TOOLS> type n.txt
This is line 1.
This is line 2.
This is line 3.
TOOLS> exit
Exiting the Text Tools.
Type EXIT again to exit Metamath.
MM>
\end{verbatim}



\appendix
\chapter{Sample Representations}
\label{ASCII}

This Appendix provides a sample of {\sc ASCII} representations,
their corresponding traditional mathematical symbols,
and a discussion of their meanings
in the \texttt{set.mm} database.
These are provided in order of appearance.
This is only a partial list, and new definitions are routinely added.
A complete list is available at \url{http://metamath.org}.

These {\sc ASCII} representations, along
with information on how to display them,
are defined in the \texttt{set.mm} database file inside
a special comment called a \texttt{\$t} {\em
comment}\index{\texttt{\$t} comment} or {\em typesetting
comment.}\index{typesetting comment}
A typesetting comment
is indicated by the appearance of the
two-character string \texttt{\$t} at the beginning of the comment.
For more information,
see Section~\ref{tcomment}, p.~\pageref{tcomment}.

In the following table the ``{\sc ASCII}'' column shows the {\sc ASCII}
representation,
``Symbol'' shows the mathematical symbolic display
that corresponds to that {\sc ASCII} representation, ``Labels'' shows
the key label(s) that define the representation, and
``Description'' provides a description about the symbol.
As usual, ``iff'' is short for ``if and only if.''\index{iff}
In most cases the ``{\sc ASCII}'' column only shows
the key token, but it will sometimes show a sequence of tokens
if that is necessary for clarity.

{\setlength{\extrarowsep}{4pt} % Keep rows from being too close together
\begin{longtabu}   { @{} c c l X }
\textbf{ASCII} & \textbf{Symbol} & \textbf{Labels} & \textbf{Description} \\
\endhead
\texttt{|-} & $\vdash$ & &
  ``It is provable that...'' \\
\texttt{ph} & $\varphi$ & \texttt{wph} &
  The wff (boolean) variable phi,
  conventionally the first wff variable. \\
\texttt{ps} & $\psi$ & \texttt{wps} &
  The wff (boolean) variable psi,
  conventionally the second wff variable. \\
\texttt{ch} & $\chi$ & \texttt{wch} &
  The wff (boolean) variable chi,
  conventionally the third wff variable. \\
\texttt{-.} & $\lnot$ & \texttt{wn} &
  Logical not. E.g., if $\varphi$ is true, then $\lnot \varphi$ is false. \\
\texttt{->} & $\rightarrow$ & \texttt{wi} &
  Implies, also known as material implication.
  In classical logic the expression $\varphi \rightarrow \psi$ is true
  if either $\varphi$ is false or $\psi$ is true (or both), that is,
  $\varphi \rightarrow \psi$ has the same meaning as
  $\lnot \varphi \lor \psi$ (as proven in theorem \texttt{imor}). \\
\texttt{<->} & $\leftrightarrow$ &
  \hyperref[df-bi]{\texttt{df-bi}} &
  Biconditional (aka is-equals for boolean values).
  $\varphi \leftrightarrow \psi$ is true iff
  $\varphi$ and $\psi$ have the same value. \\
\texttt{\char`\\/} & $\lor$ &
  \makecell[tl]{{\hyperref[df-or]{\texttt{df-or}}}, \\
	         \hyperref[df-3or]{\texttt{df-3or}}} &
  Disjunction (logical ``or''). $\varphi \lor \psi$ is true iff
  $\varphi$, $\psi$, or both are true. \\
\texttt{/\char`\\} & $\land$ &
  \makecell[tl]{{\hyperref[df-an]{\texttt{df-an}}}, \\
                 \hyperref[df-3an]{\texttt{df-3an}}} &
  Conjunction (logical ``and''). $\varphi \land \psi$ is true iff
  both $\varphi$ and $\psi$ are true. \\
\texttt{A.} & $\forall$ &
  \texttt{wal} &
  For all; the wff $\forall x \varphi$ is true iff
  $\varphi$ is true for all values of $x$. \\
\texttt{E.} & $\exists$ &
  \hyperref[df-ex]{\texttt{df-ex}} &
  There exists; the wff
  $\exists x \varphi$ is true iff
  there is at least one $x$ where $\varphi$ is true. \\
\texttt{[ y / x ]} & $[ y / x ]$ &
  \hyperref[df-sb]{\texttt{df-sb}} &
  The wff $[ y / x ] \varphi$ produces
  the result when $y$ is properly substituted for $x$ in $\varphi$
  ($y$ replaces $x$).
  % This is elsb4
  % ( [ x / y ] z e. y <-> z e. x )
  For example,
  $[ x / y ] z \in y$ is the same as $z \in x$. \\
\texttt{E!} & $\exists !$ &
  \hyperref[df-eu]{\texttt{df-eu}} &
  There exists exactly one;
  $\exists ! x \varphi$ is true iff
  there is at least one $x$ where $\varphi$ is true. \\
\texttt{\{ y | phi \}}  & $ \{ y | \varphi \}$ &
  \hyperref[df-clab]{\texttt{df-clab}} &
  The class of all sets where $\varphi$ is true. \\
\texttt{=} & $ = $ &
  \hyperref[df-cleq]{\texttt{df-cleq}} &
  Class equality; $A = B$ iff $A$ equals $B$. \\
\texttt{e.} & $ \in $ &
  \hyperref[df-clel]{\texttt{df-clel}} &
  Class membership; $A \in B$ if $A$ is a member of $B$. \\
\texttt{{\char`\_}V} & {\rm V} &
  \hyperref[df-v]{\texttt{df-v}} &
  Class of all sets (not itself a set). \\
\texttt{C\_} & $ \subseteq $ &
  \hyperref[df-ss]{\texttt{df-ss}} &
  Subclass (subset); $A \subseteq B$ is true iff
  $A$ is a subclass of $B$. \\
\texttt{u.} & $ \cup $ &
  \hyperref[df-un]{\texttt{df-un}} &
  $A \cup B$ is the union of classes $A$ and $B$. \\
\texttt{i^i} & $ \cap $ &
  \hyperref[df-in]{\texttt{df-in}} &
  $A \cap B$ is the intersection of classes $A$ and $B$. \\
\texttt{\char`\\} & $ \setminus $ &
  \hyperref[df-dif]{\texttt{df-dif}} &
  $A \setminus B$ (set difference)
  is the class of all sets in $A$ except for those in $B$. \\
\texttt{(/)} & $ \varnothing $ &
  \hyperref[df-nul]{\texttt{df-nul}} &
  $ \varnothing $ is the empty set (aka null set). \\
\texttt{\char`\~P} & $ \cal P $ &
  \hyperref[df-pw]{\texttt{df-pw}} &
  Power class. \\
\texttt{<.\ A , B >.} & $\langle A , B \rangle$ &
  \hyperref[df-op]{\texttt{df-op}} &
  The ordered pair $\langle A , B \rangle$. \\
\texttt{( F ` A )} & $ ( F ` A ) $ &
  \hyperref[df-fv]{\texttt{df-fv}} &
  The value of function $F$ when applied to $A$. \\
\texttt{\_i} & $ i $ &
  \texttt{df-i} &
  The square root of negative one. \\
\texttt{x.} & $ \cdot $ &
  \texttt{df-mul} &
  Complex number multiplication; $2~\cdot~3~=~6$. \\
\texttt{CC} & $ \mathbb{C} $ &
  \texttt{df-c} &
  The set of complex numbers. \\
\texttt{RR} & $ \mathbb{R} $ &
  \texttt{df-r} &
  The set of real numbers. \\
\end{longtabu}
} % end of extrarowsep

\chapter{Compressed Proofs}
\label{compressed}\index{compressed proof}\index{proof!compressed}

The proofs in the \texttt{set.mm} set theory database are stored in compressed
format for efficiency.  Normally you needn't concern yourself with the
compressed format, since you can display it with the usual proof display tools
in the Metamath program (\texttt{show proof}\ldots) or convert it to the normal
RPN proof format described in Section~\ref{proof} (with \texttt{save proof}
{\em label} \texttt{/normal}).  However for sake of completeness we describe the
format here and show how it maps to the normal RPN proof format.

A compressed proof, located between \texttt{\$=} and \texttt{\$.}\ keywords, consists
of a left parenthesis, a sequence of statement labels, a right parenthesis,
and a sequence of upper-case letters \texttt{A} through \texttt{Z} (with optional
white space between them).  White space must surround the parentheses
and the labels.  The left parenthesis tells Metamath that a
compressed proof follows.  (A normal RPN proof consists of just a sequence of
labels, and a parenthesis is not a legal character in a label.)

The sequence of upper-case letters corresponds to a sequence of integers
with the following mapping.  Each integer corresponds to a proof step as
described later.
\begin{center}
  \texttt{A} = 1 \\
  \texttt{B} = 2 \\
   \ldots \\
  \texttt{T} = 20 \\
  \texttt{UA} = 21 \\
  \texttt{UB} = 22 \\
   \ldots \\
  \texttt{UT} = 40 \\
  \texttt{VA} = 41 \\
  \texttt{VB} = 42 \\
   \ldots \\
  \texttt{YT} = 120 \\
  \texttt{UUA} = 121 \\
   \ldots \\
  \texttt{YYT} = 620 \\
  \texttt{UUUA} = 621 \\
   etc.
\end{center}

In other words, \texttt{A} through \texttt{T} represent the
least-significant digit in base 20, and \texttt{U} through \texttt{Y}
represent zero or more most-significant digits in base 5, where the
digits start counting at 1 instead of the usual 0. With this scheme, we
don't need white space between these ``numbers.''

(In the design of the compressed proof format, only upper-case letters,
as opposed to say all non-whitespace printable {\sc ascii} characters other than
%\texttt{\$}, was chosen to make the compressed proof a little less
%displeasing to the eye, at the expense of a typical 20\% compression
\texttt{\$}, were chosen so as not to collide with most text editor
searches, at the expense of a typical 20\% compression
loss.  The base 5/base 20 grouping, as opposed to say base 6/base 19,
was chosen by experimentally determining the grouping that resulted in
best typical compression.)

The letter \texttt{Z} identifies (tags) a proof step that is identical to one
that occurs later on in the proof; it helps shorten the proof by not requiring
that identical proof steps be proved over and over again (which happens often
when building wff's).  The \texttt{Z} is placed immediately after the
least-significant digit (letters \texttt{A} through \texttt{T}) that ends the integer
corresponding to the step to later be referenced.

The integers that the upper-case letters correspond to are mapped to labels as
follows.  If the statement being proved has $m$ mandatory hypotheses, integers
1 through $m$ correspond to the labels of these hypotheses in the order shown
by the \texttt{show statement ... / full} command, i.e., the RPN order\index{RPN
order} of the mandatory
hypotheses.  Integers $m+1$ through $m+n$ correspond to the labels enclosed in
the parentheses of the compressed proof, in the order that they appear, where
$n$ is the number of those labels.  Integers $m+n+1$ on up don't directly
correspond to statement labels but point to proof steps identified with the
letter \texttt{Z}, so that these proof steps can be referenced later in the
proof.  Integer $m+n+1$ corresponds to the first step tagged with a \texttt{Z},
$m+n+2$ to the second step tagged with a \texttt{Z}, etc.  When the compressed
proof is converted to a normal proof, the entire subproof of a step tagged
with \texttt{Z} replaces the reference to that step.

For efficiency, Metamath works with compressed proofs directly, without
converting them internally to normal proofs.  In addition to the usual
error-checking, an error message is given if (1) a label in the label list in
parentheses does not refer to a previous \texttt{\$p} or \texttt{\$a} statement or a
non-mandatory hypothesis of the statement being proved and (2) a proof step
tagged with \texttt{Z} is referenced before the step tagged with the \texttt{Z}.

Just as in a normal proof under development (Section~\ref{unknown}), any step
or subproof that is not yet known may be represented with a single \texttt{?}.
White space does not have to appear between the \texttt{?}\ and the upper-case
letters (or other \texttt{?}'s) representing the remainder of the proof.

% April 1, 2004 Appendix C has been added back in with corrections.
%
% May 20, 2003 Appendix C was removed for now because there was a problem found
% by Bob Solovay
%
% Also, removed earlier \ref{formalspec} 's (3 cases above)
%
% Bob Solovay wrote on 30 Nov 2002:
%%%%%%%%%%%%% (start of email comment )
%      3. My next set of comments concern appendix C. I read this before I
% read Chapter 4. So I first noted that the system as presented in the
% Appendix lacked a certain formal property that I thought desirable. I
% then came up with a revised formal system that had this property. Upon
% reading Chapter 4, I noticed that the revised system was closer to the
% treatment in Chapter 4 than the system in Appendix C.
%
%         First a very minor correction:
%
%         On page 142 line 2: The condition that V(e) != V(f) should only be
% required of e, f in T such that e != f.
%
%         Here is a natural property [transitivity] that one would like
% the formal system to have:
%
%         Let Gamma be a set of statements. Suppose that the statement Phi
% is provable from Gamma and that the statement Psi is provable from Gamma
% \cup {Phi}. Then Psi is provable from Gamma.
%
%         I shall present an example to show that this property does not
% hold for the formal systems of Appendix C:
%
%         I write the example in metamath style:
%
% $c A B C D E $.
% $v x y
%
% ${
% tx $f A x $.
% ty $f B y $.
% ax1 $a C x y $.
% $}
%
% ${
% tx $f A x $.
% ty $f B y $.
% ax2-h1 $e C x y $.
% ax2 $a D y $.
% $}
%
% ${
% ty $f B y $.
% ax3-h1 $e D y $.
% ax3 $a E y $.
% $}
%
% $(These three axioms are Gamma $)
%
% ${
% tx $f A x $.
% ty $f B y $.
% Phi $p D y $=
% tx ty tx ty ax1 ax2 $.
% $}
%
% ${
% ty $f B y $.
% Psi $p E y $=
% ty ty Phi ax3 $.
% $}
%
%
% I omit the formal proofs of the following claims. [I will be glad to
% supply them upon request.]
%
% 1) Psi is not provable from Gamma;
%
% 2) Psi is provable from Gamma + Phi.
%
% Here "provable" refers to the formalism of Appendix C.
%
% The trouble of course is that Psi is lacking the variable declaration
%
% $f Ax $.
%
% In the Metamath system there is no trouble proving Psi. I attach a
% metamath file that shows this and which has been checked by the
% metamath program.
%
% I next want to indicate how I think the treatment in Appendix C should
% be revised so as to conform more closely to the metamath system of the
% main text. The revised system *does* have the transitivity property.
%
% We want to give revised definitions of "statement" and
% "provable". [cf. sections C.2.4. and C.2.5] Our new definitions will
% use the definitions given in Appendix C. So we take the following
% tack. We refer to the original notions as o-statement and o-provable. And
% we refer to the notions we are defining as n-statement and n-provable.
%
%         A n-statement is an o-statement in which the only variables
% that appear in the T component are mandatory.
%
%         To any o-statement we can associate its reduct which is a
% n-statement by dropping all the elements of T or D which contain
% non-mandatory variables.
%
%         An n-statement gamma is n-provable if there is an o-statement
% gamma' which has gamma as its reduct andf such that gamma' is
% o-provable.
%
%         It seems to me [though I am not completely sure on this point]
% that n-provability corresponds to metamath provability as discussed
% say in Chapter 4.
%
%         Attached to this letter is the metamath proof of Phi and Psi
% from Gamma discussed above.
%
%         I am still brooding over the question of whether metamath
% correctly formalizes set-theory. No doubt I will have some questions
% re this after my thoughts become clearer.
%%%%%%%%%%%%%%%% (end of email comment)

%%%%%%%%%%%%%%%% (start of 2nd email comment from Bob Solovay 1-Apr-04)
%
%         I hope that Appendix C is the one that gives a "formal" treatment
% of Metamath. At any rate, thats the appendix I want to comment on.
%
%         I'm going to suggest two changes in the definition.
%
%         First change (in the definition of statement): Require that the
% sets D, T, and E be finite.
%
%         Probably things are fine as you give them. But in the applications
% to the main metamath system they will always be finite, and its useful in
% thinking about things [at least for me] to stick to the finite case.
%
%         Second change:
%
%         First let me give an approximate description. Remove the dummy
% variables from the statement. Instead, include them in the proof.
%
%         More formally: Require that T consists of type declarations only
% for mandatory variables. Require that all the pairs in D consist of
% mandatory variables.
%
%         At the start of a proof we are allowed to declare a finite number
% of dummy variables [provided that none of them appear in any of the
% statements in E \cup {A}. We have to supply type declarations for all the
% dummy variables. We are allowed to add new $d statements referring to
% either the mandatory or dummy variables. But we require that no new $d
% statement references only mandatory variables.
%
%         I find this way of doing things more conceptual than the treatment
% in Appendix C. But the change [which I will use implicitly in later
% letters about doing Peano] is mainly aesthetic. I definitely claim that my
% results on doing Peano all apply to Metamath as it is presented in your
% book.
%
%         --Bob
%
%%%%%%%%%%%%%%%% (end of 2nd email comment)

%%
%% When uncommenting the below, also uncomment references above to {formalspec}
%%
\chapter{Metamath's Formal System}\label{formalspec}\index{Metamath!as a formal
system}

\section{Introduction}

\begin{quote}
  {\em Perfection is when there is no longer anything more to take away.}
    \flushright\sc Antoine de
     Saint-Exupery\footnote{\cite[p.~3-25]{Campbell}.}\\
\end{quote}\index{de Saint-Exupery, Antoine}

This appendix describes the theory behind the Metamath language in an abstract
way intended for mathematicians.  Specifically, we construct two
set-theo\-ret\-i\-cal objects:  a ``formal system'' (roughly, a set of syntax
rules, axioms, and logical rules) and its ``universe'' (roughly, the set of
theorems derivable in the formal system).  The Metamath computer language
provides us with a way to describe specific formal systems and, with the aid of
a proof provided by the user, to verify that given theorems
belong to their universes.

To understand this appendix, you need a basic knowledge of informal set theory.
It should be sufficient to understand, for example, Ch.\ 1 of Munkres' {\em
Topology} \cite{Munkres}\index{Munkres, James R.} or the
introductory set theory chapter
in many textbooks that introduce abstract mathematics. (Note that there are
minor notational differences among authors; e.g.\ Munkres uses $\subset$ instead
of our $\subseteq$ for ``subset.''  We use ``included in'' to mean ``a subset
of,'' and ``belongs to'' or ``is contained in'' to mean ``is an element of.'')
What we call a ``formal'' description here, unlike earlier, is actually an
informal description in the ordinary language of mathematicians.  However we
provide sufficient detail so that a mathematician could easily formalize it,
even in the language of Metamath itself if desired.  To understand the logic
examples at the end of this appendix, familiarity with an introductory book on
mathematical logic would be helpful.

\section{The Formal Description}

\subsection[Preliminaries]{Preliminaries\protect\footnotemark}%
\footnotetext{This section is taken mostly verbatim
from Tarski \cite[p.~63]{Tarski1965}\index{Tarski, Alfred}.}

By $\omega$ we denote the set of all natural numbers (non-negative integers).
Each natural number $n$ is identified with the set of all smaller numbers: $n =
\{ m | m < n \}$.  The formula $m < n$ is thus equivalent to the condition: $m
\in n$ and $m,n \in \omega$. In particular, 0 is the number zero and at the
same time the empty set $\varnothing$, $1=\{0\}$, $2=\{0,1\}$, etc. ${}^B A$
denotes the set of all functions on $B$ to $A$ (i.e.\ with domain $B$ and range
included in $A$).  The members of ${}^\omega A$ are what are called {\em simple
infinite sequences},\index{simple infinite sequence}
with all {\em terms}\index{term} in $A$.  In case $n \in \omega$, the
members of ${}^n A$ are referred to as {\em finite $n$-termed
sequences},\index{finite $n$-termed
sequence} again
with terms in $A$.  The consecutive terms (function values) of a finite or
infinite sequence $f$ are denoted by $f_0, f_1, \ldots ,f_n,\ldots$.  Every
finite sequence $f \in \bigcup _{n \in \omega} {}^n A$ uniquely determines the
number $n$ such that $f \in {}^n A$; $n$ is called the {\em
length}\index{length of a sequence ({$"|\ "|$})} of $f$ and
is denoted by $|f|$.  $\langle a \rangle$ is the sequence $f$ with $|f|=1$ and
$f_0=a$; $\langle a,b \rangle$ is the sequence $f$ with $|f|=2$, $f_0=a$,
$f_1=b$; etc.  Given two finite sequences $f$ and $g$, we denote by $f\frown g$
their {\em concatenation},\index{concatenation} i.e., the
finite sequence $h$ determined by the
conditions:
\begin{eqnarray*}
& |h| = |f|+|g|;&  \\
& h_n = f_n & \mbox{\ for\ } n < |f|;  \\
& h_{|f|+n} = g_n & \mbox{\ for\ } n < |g|.
\end{eqnarray*}

\subsection{Constants, Variables, and Expressions}

A formal system has a set of {\em symbols}\index{symbol!in
a formal system} denoted
by $\mbox{\em SM}$.  A
precise set-theo\-ret\-i\-cal definition of this set is unimportant; a symbol
could be considered a primitive or atomic element if we wish.  We assume this
set is divided into two disjoint subsets:  a set $\mbox{\em CN}$ of {\em
constants}\index{constant!in a formal system} and a set $\mbox{\em VR}$ of
{\em variables}.\index{variable!in a formal system}  $\mbox{\em CN}$ and
$\mbox{\em VR}$ are each assumed to consist of countably many symbols which
may be arranged in finite or simple infinite sequences $c_0, c_1, \ldots$ and
$v_0, v_1, \ldots$ respectively, without repeating terms.  We will represent
arbitrary symbols by metavariables $\alpha$, $\beta$, etc.

{\footnotesize\begin{quotation}
{\em Comment.} The variables $v_0, v_1, \ldots$ of our formal system
correspond to what are usually considered ``metavariables'' in
descriptions of specific formal systems in the literature.  Typically,
when describing a specific formal system a book will postulate a set of
primitive objects called variables, then proceed to describe their
properties using metavariables that range over them, never mentioning
again the actual variables themselves.  Our formal system does not
mention these primitive variable objects at all but deals directly with
metavariables, as its primitive objects, from the start.  This is a
subtle but key distinction you should keep in mind, and it makes our
definition of ``formal system'' somewhat different from that typically
found in the literature.  (So, the $\alpha$, $\beta$, etc.\ above are
actually ``metametavariables'' when used to represent $v_0, v_1,
\ldots$.)
\end{quotation}}

Finite sequences all terms of which are symbols are called {\em
expressions}.\index{expression!in a formal system}  $\mbox{\em EX}$ is
the set of all expressions; thus
\begin{displaymath}
\mbox{\em EX} = \bigcup _{n \in \omega} {}^n \mbox{\em SM}.
\end{displaymath}

A {\em constant-prefixed expression}\index{constant-prefixed expression}
is an expression of non-zero length
whose first term is a constant.  We denote the set of all constant-prefixed
expressions by $\mbox{\em EX}_C = \{ e \in \mbox{\em EX} | ( |e| > 0 \wedge
e_0 \in \mbox{\em CN} ) \}$.

A {\em constant-variable pair}\index{constant-variable pair}
is an expression of length 2 whose first term
is a constant and whose second term is a variable.  We denote the set of all
constant-variable pairs by $\mbox{\em EX}_2 = \{ e \in \mbox{\em EX}_C | ( |e|
= 2 \wedge e_1 \in \mbox{\em VR} ) \}$.


{\footnotesize\begin{quotation}
{\em Relationship to Metamath.} In general, the set $\mbox{\em SM}$
corresponds to the set of declared math symbols in a Metamath database, the
set $\mbox{\em CN}$ to those declared with \texttt{\$c} statements, and the set
$\mbox{\em VR}$ to those declared with \texttt{\$v} statements.  Of course a
Metamath database can only have a finite number of math symbols, whereas
formal systems in general can have an infinite number, although the number of
Metamath math symbols available is in principle unlimited.

The set $\mbox{\em EX}_C$ corresponds to the set of permissible expressions
for \texttt{\$e}, \texttt{\$a}, and \texttt{\$p} statements.  The set $\mbox{\em EX}_2$
corresponds to the set of permissible expressions for \texttt{\$f} statements.
\end{quotation}}

We denote by ${\cal V}(e)$ the set of all variables in an expression $e \in
\mbox{\em EX}$, i.e.\ the set of all $\alpha \in \mbox{\em VR}$ such that
$\alpha = e_n$ for some $n < |e|$.  We also denote (with abuse of notation) by
${\cal V}(E)$ the set of all variables in a collection of expressions $E
\subseteq \mbox{\em EX}$, i.e.\ $\bigcup _{e \in E} {\cal V}(e)$.


\subsection{Substitution}

Given a function $F$ from $\mbox{\em VR}$ to
$\mbox{\em EX}$, we
denote by $\sigma_{F}$ or just $\sigma$ the function from $\mbox{\em EX}$ to
$\mbox{\em EX}$ defined recursively for nonempty sequences by
\begin{eqnarray*}
& \sigma(<\alpha>) = F(\alpha) & \mbox{for\ } \alpha \in \mbox{\em VR}; \\
& \sigma(<\alpha>) = <\alpha> & \mbox{for\ } \alpha \not\in \mbox{\em VR}; \\
& \sigma(g \frown h) = \sigma(g) \frown
    \sigma(h) & \mbox{for\ } g,h \in \mbox{\em EX}.
\end{eqnarray*}
We also define $\sigma(\varnothing)=\varnothing$.  We call $\sigma$ a {\em
simultaneous substitution}\index{substitution!variable}\index{variable
substitution} (or just {\em substitution}) with {\em substitution
map}\index{substitution map} $F$.

We also denote (with abuse of notation) by $\sigma(E)$ a substitution on a
collection of expressions $E \subseteq \mbox{\em EX}$, i.e.\ the set $\{
\sigma(e) | e \in E \}$.  The collection $\sigma(E)$ may of course contain
fewer expressions than $E$ because duplicate expressions could result from the
substitution.

\subsection{Statements}

We denote by $\mbox{\em DV}$ the set of all
unordered pairs $\{\alpha, \beta \} \subseteq \mbox{\em VR}$ such that $\alpha
\neq \beta$.  $\mbox{\em DV}$ stands for ``distinct variables.''

A {\em pre-statement}\index{pre-statement!in a formal system} is a
quadruple $\langle D,T,H,A \rangle$ such that
$D\subseteq \mbox{\em DV}$, $T\subseteq \mbox{\em EX}_2$, $H\subseteq
\mbox{\em EX}_C$ and $H$ is finite,
$A\in \mbox{\em EX}_C$, ${\cal V}(H\cup\{A\}) \subseteq
{\cal V}(T)$, and $\forall e,f\in T {\ } {\cal V}(e) \neq {\cal V}(f)$ (or
equivalently, $e_1 \ne f_1$) whenever $e \neq f$. The terms of the quadruple are called {\em
distinct-variable restrictions},\index{disjoint-variable restriction!in a
formal system} {\em variable-type hypotheses},\index{variable-type
hypothesis!in a formal system} {\em logical hypotheses},\index{logical
hypothesis!in a formal system} and the {\em assertion}\index{assertion!in a
formal system} respectively.  We denote by $T_M$ ({\em mandatory variable-type
hypotheses}\index{mandatory variable-type hypothesis!in a formal system}) the
subset of $T$ such that ${\cal V}(T_M) ={\cal V}(H \cup \{A\})$.  We denote by
$D_M=\{\{\alpha,\beta\}\in D|\{\alpha,\beta\}\subseteq {\cal V}(T_M)\}$ the
{\em mandatory distinct-variable restrictions}\index{mandatory
disjoint-variable restriction!in a formal system} of the pre-statement.
The set
of {\em mandatory hypotheses}\index{mandatory hypothesis!in a formal system}
is $T_M\cup H$.  We call the quadruple $\langle D_M,T_M,H,A \rangle$
the {\em reduct}\index{reduct!in a formal system} of
the pre-statement $\langle D,T,H,A \rangle$.

A {\em statement} is the reduct of some pre-statement\index{statement!in a
formal system}.  A statement is therefore a special kind of pre-statement;
in particular, a statement is the reduct of itself.

{\footnotesize\begin{quotation}
{\em Comment.}  $T$ is a set of expressions, each of length 2, that associate
a set of constants (``variable types'') with a set of variables.  The
condition ${\cal V}(H\cup\{A\}) \subseteq {\cal V}(T) $
means that each variable occurring in a statement's logical
hypotheses or assertion must have an associated variable-type hypothesis or
``type declaration,'' in  analogy to a computer programming language, where a
variable must be declared to be say, a string or an integer.  The requirement
that $\forall e,f\in T \, e_1 \ne f_1$ for $e\neq f$
means that each variable must be
associated with a unique constant designating its variable type; e.g., a
variable might be a ``wff'' or a ``set'' but not both.

Distinct-variable restrictions are used to specify what variable substitutions
are permissible to make for the statement to remain valid.  For example, in
the theorem scheme of set theory $\lnot\forall x\,x=y$ we may not substitute
the same variable for both $x$ and $y$.  On the other hand, the theorem scheme
$x=y\to y=x$ does not require that $x$ and $y$ be distinct, so we do not
require a distinct-variable restriction, although having one
would cause no harm other than making the scheme less general.

A mandatory variable-type hypothesis is one whose variable exists in a logical
hypothesis or the assertion.  A provable pre-statement
(defined below) may require
non-mandatory variable-type hypotheses that effectively introduce ``dummy''
variables for use in its proof.  Any number of dummy variables might
be required by a specific proof; indeed, it has been shown by H.\
Andr\'{e}ka\index{Andr{\'{e}}ka, H.} \cite{Nemeti} that there is no finite
upper bound to the number of dummy variables needed to prove an arbitrary
theorem in first-order logic (with equality) having a fixed number $n>2$ of
individual variables.  (See also the Comment on p.~\pageref{nodd}.)
For this reason we do not set a finite size bound on the collections $D$ and
$T$, although in an actual application (Metamath database) these will of
course be finite, increased to whatever size is necessary as more
proofs are added.
\end{quotation}}

{\footnotesize\begin{quotation}
{\em Relationship to Metamath.} A pre-statement of a formal system
corresponds to an extended frame in a Metamath database
(Section~\ref{frames}).  The collections $D$, $T$, and $H$ correspond
respectively to the \texttt{\$d}, \texttt{\$f}, and \texttt{\$e}
statement collections in an extended frame.  The expression $A$
corresponds to the \texttt{\$a} (or \texttt{\$p}) statement in an
extended frame.

A statement of a formal system corresponds to a frame in a Metamath
database.
\end{quotation}}

\subsection{Formal Systems}

A {\em formal system}\index{formal system} is a
triple $\langle \mbox{\em CN},\mbox{\em
VR},\Gamma\rangle$ where $\Gamma$ is a set of statements.  The members of
$\Gamma$ are called {\em axiomatic statements}.\index{axiomatic
statement!in a formal system}  Sometimes we will refer to a
formal system by just $\Gamma$ when $\mbox{\em CN}$ and $\mbox{\em VR}$ are
understood.

Given a formal system $\Gamma$, the {\em closure}\index{closure}\footnote{This
definition of closure incorporates a simplification due to
Josh Purinton.\index{Purinton, Josh}.} of a
pre-statement
$\langle D,T,H,A \rangle$ is the smallest set $C$ of expressions
such that:
%\begin{enumerate}
%  \item $T\cup H\subseteq C$; and
%  \item If for some axiomatic statement
%    $\langle D_M',T_M',H',A' \rangle \in \Gamma_A$, for
%    some $E \subseteq C$, some $F \subseteq C-T$ (where ``-'' denotes
%    set difference), and some substitution
%    $\sigma$ we have
%    \begin{enumerate}
%       \item $\sigma(T_M') = E$ (where, as above, the $M$ denotes the
%           mandatory variable-type hypotheses of $T^A$);
%       \item $\sigma(H') = F$;
%       \item for all $\{\alpha,\beta\}\in D^A$ and $\subseteq
%         {\cal V}(T_M')$, for all $\gamma\in {\cal V}(\sigma(\langle \alpha
%         \rangle))$, and for all $\delta\in  {\cal V}(\sigma(\langle \beta
%         \rangle))$, we have $\{\gamma, \delta\} \in D$;
%   \end{enumerate}
%   then $\sigma(A') \in C$.
%\end{enumerate}
\begin{list}{}{\itemsep 0.0pt}
  \item[1.] $T\cup H\subseteq C$; and
  \item[2.] If for some axiomatic statement
    $\langle D_M',T_M',H',A' \rangle \in
       \Gamma$ and for some substitution
    $\sigma$ we have
    \begin{enumerate}
       \item[a.] $\sigma(T_M' \cup H') \subseteq C$; and
       \item[b.] for all $\{\alpha,\beta\}\in D_M'$, for all $\gamma\in
         {\cal V}(\sigma(\langle \alpha
         \rangle))$, and for all $\delta\in  {\cal V}(\sigma(\langle \beta
         \rangle))$, we have $\{\gamma, \delta\} \in D$;
   \end{enumerate}
   then $\sigma(A') \in C$.
\end{list}
A pre-statement $\langle D,T,H,A
\rangle$ is {\em provable}\index{provable statement!in a formal
system} if $A\in C$ i.e.\ if its assertion belongs to its
closure.  A statement is {\em provable} if it is
the reduct of a provable pre-statement.
The {\em universe}\index{universe of a formal system}
of a formal system is
the collection of all of its provable statements.  Note that the
set of axiomatic statements $\Gamma$ in a formal system is a subset of its
universe.

{\footnotesize\begin{quotation}
{\em Comment.} The first condition in the definition of closure simply says
that the hypotheses of the pre-statement are in its closure.

Condition 2(a) says that a substitution exists that makes the
mandatory hypotheses of an axiomatic statement exactly match some members of
the closure.  This is what we explicitly demonstrate in a Metamath language
proof.

%Conditions 2(a) and 2(b) say that a substitution exists that makes the
%(mandatory) hypotheses of an axiomatic statement exactly match some members of
%the closure.  This is what we explicitly demonstrate with a Metamath language
%proof.
%
%The set of expressions $F$ in condition 2(b) excludes the variable-type
%hypotheses; this is done because non-mandatory variable-type hypotheses are
%effectively ``dropped'' as irrelevant whereas logical hypotheses must be
%retained to achieve a consistent logical system.

Condition 2(b) describes how distinct-variable restrictions in the axiomatic
statement must be met.  It means that after a substitution for two variables
that must be distinct, the resulting two expressions must either contain no
variables, or if they do, they may not have variables in common, and each pair
of any variables they do have, with one variable from each expression, must be
specified as distinct in the original statement.
\end{quotation}}

{\footnotesize\begin{quotation}
{\em Relationship to Metamath.} Axiomatic statements
 and provable statements in a formal
system correspond to the frames for \texttt{\$a} and \texttt{\$p} statements
respectively in a Metamath database.  The set of axiomatic statements is a
subset of the set of provable statements in a formal system, although in a
Metamath database a \texttt{\$a} statement is distinguished by not having a
proof.  A Metamath language proof for a \texttt{\$p} statement tells the computer
how to explicitly construct a series of members of the closure ultimately
leading to a demonstration that the assertion
being proved is in the closure.  The actual closure typically contains
an infinite number of expressions.  A formal system itself does not have
an explicit object called a ``proof'' but rather the existence of a proof
is implied indirectly by membership of an assertion in a provable
statement's closure.  We do this to make the formal system easier
to describe in the language of set theory.

We also note that once established as provable, a statement may be considered
to acquire the same status as an axiomatic statement, because if the set of
axiomatic statements is extended with a provable statement, the universe of
the formal system remains unchanged (provided that $\mbox{\em VR}$ is
infinite).
In practice, this means we can build a hierarchy of provable statements to
more efficiently establish additional provable statements.  This is
what we do in Metamath when we allow proofs to reference previous
\texttt{\$p} statements as well as previous \texttt{\$a} statements.
\end{quotation}}

\section{Examples of Formal Systems}

{\footnotesize\begin{quotation}
{\em Relationship to Metamath.} The examples in this section, except Example~2,
are for the most part exact equivalents of the development in the set
theory database \texttt{set.mm}.  You may want to compare Examples~1, 3, and 5
to Section~\ref{metaaxioms}, Example 4 to Sections~\ref{metadefprop} and
\ref{metadefpred}, and Example 6 to
Section~\ref{setdefinitions}.\label{exampleref}
\end{quotation}}

\subsection{Example~1---Propositional Calculus}\index{propositional calculus}

Classical propositional calculus can be described by the following formal
system.  We assume the set of variables is infinite.  Rather than denoting the
constants and variables by $c_0, c_1, \ldots$ and $v_0, v_1, \ldots$, for
readability we will instead use more conventional symbols, with the
understanding of course that they denote distinct primitive objects.
Also for readability we may omit commas between successive terms of a
sequence; thus $\langle \mbox{wff\ } \varphi\rangle$ denotes
$\langle \mbox{wff}, \varphi\rangle$.

Let
\begin{itemize}
  \item[] $\mbox{\em CN}=\{\mbox{wff}, \vdash, \to, \lnot, (,)\}$
  \item[] $\mbox{\em VR}=\{\varphi,\psi,\chi,\ldots\}$
  \item[] $T = \{\langle \mbox{wff\ } \varphi\rangle,
             \langle \mbox{wff\ } \psi\rangle,
             \langle \mbox{wff\ } \chi\rangle,\ldots\}$, i.e.\ those
             expressions of length 2 whose first member is $\mbox{\rm wff}$
             and whose second member belongs to $\mbox{\em VR}$.\footnote{For
convenience we let $T$ be an infinite set; the definition of a statement
permits this in principle.  Since a Metamath source file has a finite size, in
practice we must of course use appropriate finite subsets of this $T$,
specifically ones containing at least the mandatory variable-type
hypotheses.  Similarly, in the source file we introduce new variables as
required, with the understanding that a potentially infinite number of
them are available.}
\noindent Then $\Gamma$ consists of the axiomatic statements that
are the reducts of the following pre-statements:
    \begin{itemize}
      \item[] $\langle\varnothing,T,\varnothing,
               \langle \mbox{wff\ }(\varphi\to\psi)\rangle\rangle$
      \item[] $\langle\varnothing,T,\varnothing,
               \langle \mbox{wff\ }\lnot\varphi\rangle\rangle$
      \item[] $\langle\varnothing,T,\varnothing,
               \langle \vdash(\varphi\to(\psi\to\varphi))
               \rangle\rangle$
      \item[] $\langle\varnothing,T,
               \varnothing,
               \langle \vdash((\varphi\to(\psi\to\chi))\to
               ((\varphi\to\psi)\to(\varphi\to\chi)))
               \rangle\rangle$
      \item[] $\langle\varnothing,T,
               \varnothing,
               \langle \vdash((\lnot\varphi\to\lnot\psi)\to
               (\psi\to\varphi))\rangle\rangle$
      \item[] $\langle\varnothing,T,
               \{\langle\vdash(\varphi\to\psi)\rangle,
                 \langle\vdash\varphi\rangle\},
               \langle\vdash\psi\rangle\rangle$
    \end{itemize}
\end{itemize}

(For example, the reduct of $\langle\varnothing,T,\varnothing,
               \langle \mbox{wff\ }(\varphi\to\psi)\rangle\rangle$
is
\begin{itemize}
\item[] $\langle\varnothing,
\{\langle \mbox{wff\ } \varphi\rangle,
             \langle \mbox{wff\ } \psi\rangle\},
             \varnothing,
               \langle \mbox{wff\ }(\varphi\to\psi)\rangle\rangle$,
\end{itemize}
which is the first axiomatic statement.)

We call the members of $\mbox{\em VR}$ {\em wff variables} or (in the context
of first-order logic which we will describe shortly) {\em wff metavariables}.
Note that the symbols $\phi$, $\psi$, etc.\ denote actual specific members of
$\mbox{\em VR}$; they are not metavariables of our expository language (which
we denote with $\alpha$, $\beta$, etc.) but are instead (meta)constant symbols
(members of $\mbox{\em SM}$) from the point of view of our expository
language.  The equivalent system of propositional calculus described in
\cite{Tarski1965} also uses the symbols $\phi$, $\psi$, etc.\ to denote wff
metavariables, but in \cite{Tarski1965} unlike here those are metavariables of
the expository language and not primitive symbols of the formal system.

The first two statements define wffs: if $\varphi$ and $\psi$ are wffs, so is
$(\varphi \to \psi)$; if $\varphi$ is a wff, so is $\lnot\varphi$. The next
three are the axioms of propositional calculus: if $\varphi$ and $\psi$ are
wffs, then $\vdash (\varphi \to (\psi \to \varphi))$ is an (axiomatic)
theorem; etc. The
last is the rule of modus ponens: if $\varphi$ and $\psi$ are wffs, and
$\vdash (\varphi\to\psi)$ and $\vdash \varphi$ are theorems, then $\vdash
\psi$ is a theorem.

The correspondence to ordinary propositional calculus is as follows.  We
consider only provable statements of the form $\langle\varnothing,
T,\varnothing,A\rangle$ with $T$ defined as above.  The first term of the
assertion $A$ of any such statement is either ``wff'' or ``$\vdash$''.  A
statement for which the first term is ``wff'' is a {\em wff} of propositional
calculus, and one where the first term is ``$\vdash$'' is a {\em
theorem (scheme)} of propositional calculus.

The universe of this formal system also contains many other provable
statements.  Those with distinct-variable restrictions are irrelevant because
propositional calculus has no constraints on substitutions.  Those that have
logical hypotheses we call {\em inferences}\index{inference} when
the logical hypotheses are of the form
$\langle\vdash\rangle\frown w$ where $w$ is a wff (with the leading constant
term ``wff'' removed).  Inferences (other than the modus ponens rule) are not a
proper part of propositional calculus but are convenient to use when building a
hierarchy of provable statements.  A provable statement with a nonsense
hypothesis such as $\langle \to,\vdash,\lnot\rangle$, and this same expression
as its assertion, we consider irrelevant; no use can be made of it in
proving theorems, since there is no way to eliminate the nonsense hypothesis.

{\footnotesize\begin{quotation}
{\em Comment.} Our use of parentheses in the definition of a wff illustrates
how axiomatic statements should be carefully stated in a way that
ties in unambiguously with the substitutions allowed by the formal system.
There are many ways we could have defined wffs---for example, Polish
prefix notation would have allowed us to omit parentheses entirely, at
the expense of readability---but we must define them in a way that is
unambiguous.  For example, if we had omitted parentheses from the
definition of $(\varphi\to \psi)$, the wff $\lnot\varphi\to \psi$ could
be interpreted as either $\lnot(\varphi\to\psi)$ or $(\lnot\varphi\to\psi)$
and would have allowed us to prove nonsense.  Note that there is no
concept of operator binding precedence built into our formal system.
\end{quotation}}

\begin{sloppy}
\subsection{Example~2---Predicate Calculus with Equality}\index{predicate
calculus}
\end{sloppy}

Here we extend Example~1 to include predicate calculus with equality,
illustrating the use of distinct-variable restrictions.  This system is the
same as Tarski's system $\mathfrak{S}_2$ in \cite{Tarski1965} (except that the
axioms of propositional calculus are different but equivalent, and a redundant
axiom is omitted).  We extend $\mbox{\em CN}$ with the constants
$\{\mbox{var},\forall,=\}$.  We extend $\mbox{\em VR}$ with an infinite set of
{\em individual metavariables}\index{individual
metavariable} $\{x,y,z,\ldots\}$ and denote this subset
$\mbox{\em Vr}$.

We also join to $\mbox{\em CN}$ a possibly infinite set $\mbox{\em Pr}$ of {\em
predicates} $\{R,S,\ldots\}$.  We associate with $\mbox{\em Pr}$ a function
$\mbox{rnk}$ from $\mbox{\em Pr}$ to $\omega$, and for $\alpha\in \mbox{\em
Pr}$ we call $\mbox{rnk}(\alpha)$ the {\em rank} of the predicate $\alpha$,
which is simply the number of ``arguments'' that the predicate has.  (Most
applications of predicate calculus will have a finite number of predicates;
for example, set theory has the single two-argument or binary predicate $\in$,
which is usually written with its arguments surrounding the predicate symbol
rather than with the prefix notation we will use for the general case.)  As a
device to facilitate our discussion, we will let $\mbox{\em Vs}$ be any fixed
one-to-one function from $\omega$ to $\mbox{\em Vr}$; thus $\mbox{\em Vs}$ is
any simple infinite sequence of individual metavariables with no repeating
terms.

In this example we will not include the function symbols that are often part of
formalizations of predicate calculus.  Using metalogical arguments that are
beyond the scope of our discussion, it can be shown that our formalization is
equivalent when functions are introduced via appropriate definitions.

We extend the set $T$ defined in Example~1 with the expressions
$\{\langle \mbox{var\ } x\rangle,$ $ \langle \mbox{var\ } y\rangle, \langle
\mbox{var\ } z\rangle,\ldots\}$.  We extend the $\Gamma$ above
with the axiomatic statements that are the reducts of the following
pre-statements:
\begin{list}{}{\itemsep 0.0pt}
      \item[] $\langle\varnothing,T,\varnothing,
               \langle \mbox{wff\ }\forall x\,\varphi\rangle\rangle$
      \item[] $\langle\varnothing,T,\varnothing,
               \langle \mbox{wff\ }x=y\rangle\rangle$
      \item[] $\langle\varnothing,T,
               \{\langle\vdash\varphi\rangle\},
               \langle\vdash\forall x\,\varphi\rangle\rangle$
      \item[] $\langle\varnothing,T,\varnothing,
               \langle \vdash((\forall x(\varphi\to\psi)
                  \to(\forall x\,\varphi\to\forall x\,\psi))
               \rangle\rangle$
      \item[] $\langle\{\{x,\varphi\}\},T,\varnothing,
               \langle \vdash(\varphi\to\forall x\,\varphi)
               \rangle\rangle$
      \item[] $\langle\{\{x,y\}\},T,\varnothing,
               \langle \vdash\lnot\forall x\lnot x=y
               \rangle\rangle$
      \item[] $\langle\varnothing,T,\varnothing,
               \langle \vdash(x=z
                  \to(x=y\to z=y))
               \rangle\rangle$
      \item[] $\langle\varnothing,T,\varnothing,
               \langle \vdash(y=z
                  \to(x=y\to x=z))
               \rangle\rangle$
\end{list}
These are the axioms not involving predicate symbols. The first two statements
extend the definition of a wff.  The third is the rule of generalization.  The
fifth states, in effect, ``For a wff $\varphi$ and variable $x$,
$\vdash(\varphi\to\forall x\,\varphi)$, provided that $x$ does not occur in
$\varphi$.''  The sixth states ``For variables $x$ and $y$,
$\vdash\lnot\forall x\lnot x = y$, provided that $x$ and $y$ are distinct.''
(This proviso is not necessary but was included by Tarski to
weaken the axiom and still show that the system is logically complete.)

Finally, for each predicate symbol $\alpha\in \mbox{\em Pr}$, we add to
$\Gamma$ an axiomatic statement, extending the definition of wff,
that is the reduct of the following pre-statement:
\begin{displaymath}
    \langle\varnothing,T,\varnothing,
            \langle \mbox{wff},\alpha\rangle\
            \frown \mbox{\em Vs}\restriction\mbox{rnk}(\alpha)\rangle
\end{displaymath}
and for each $\alpha\in \mbox{\em Pr}$ and each $n < \mbox{rnk}(\alpha)$
we add to $\Gamma$ an equality axiom that is the reduct of the
following pre-statement:
\begin{eqnarray*}
    \lefteqn{\langle\varnothing,T,\varnothing,
            \langle
      \vdash,(,\mbox{\em Vs}_n,=,\mbox{\em Vs}_{\mbox{rnk}(\alpha)},\to,
            (,\alpha\rangle\frown \mbox{\em Vs}\restriction\mbox{rnk}(\alpha)} \\
  & & \frown
            \langle\to,\alpha\rangle\frown \mbox{\em Vs}\restriction n\frown
            \langle \mbox{\em Vs}_{\mbox{rnk}(\alpha)}\rangle \\
 & & \frown
            \mbox{\em Vs}\restriction(\mbox{rnk}(\alpha)\setminus(n+1))\frown
            \langle),)\rangle\rangle
\end{eqnarray*}
where $\restriction$ denotes function domain restriction and $\setminus$
denotes set difference.  Recall that a subscript on $\mbox{\em Vs}$
denotes one of its terms.  (In the above two axiom sets commas are placed
between successive terms of sequences to prevent ambiguity, and if you examine
them with care you will be able to distinguish those parentheses that denote
constant symbols from those of our expository language that delimit function
arguments.  Although it might have been better to use boldface for our
primitive symbols, unfortunately boldface was not available for all characters
on the \LaTeX\ system used to typeset this text.)  These seemingly forbidding
axioms can be understood by analogy to concatenation of substrings in a
computer language.  They are actually relatively simple for each specific case
and will become clearer by looking at the special case of a binary predicate
$\alpha = R$ where $\mbox{rnk}(R)=2$.  Letting $\mbox{\em Vs}$ be the sequence
$\langle x,y,z,\ldots\rangle$, the axioms we would add to $\Gamma$ for this
case would be the wff extension and two equality axioms that are the
reducts of the pre-statements:
\begin{list}{}{\itemsep 0.0pt}
      \item[] $\langle\varnothing,T,\varnothing,
               \langle \mbox{wff\ }R x y\rangle\rangle$
      \item[] $\langle\varnothing,T,\varnothing,
               \langle \vdash(x=z
                  \to(R x y \to R z y))
               \rangle\rangle$
      \item[] $\langle\varnothing,T,\varnothing,
               \langle \vdash(y=z
                  \to(R x y \to R x z))
               \rangle\rangle$
\end{list}
Study these carefully to see how the general axioms above evaluate to
them.  In practice, typically only a few special cases such as this would be
needed, and in any case the Metamath language will only permit us to describe
a finite number of predicates, as opposed to the infinite number permitted by
the formal system.  (If an infinite number should be needed for some reason,
we could not define the formal system directly in the Metamath language but
could instead define it metalogically under set theory as we
do in this appendix, and only the underlying set theory, with its single
binary predicate, would be defined directly in the Metamath language.)


{\footnotesize\begin{quotation}
{\em Comment.}  As we noted earlier, the specific variables denoted by the
symbols $x,y,z,\ldots\in \mbox{\em Vr}\subseteq \mbox{\em VR}\subseteq
\mbox{\em SM}$ in Example~2 are not the actual variables of ordinary predicate
calculus but should be thought of as metavariables ranging over them.  For
example, a distinct-variable restriction would be meaningless for actual
variables of ordinary predicate calculus since two different actual variables
are by definition distinct.  And when we talk about an arbitrary
representative $\alpha\in \mbox{\em Vr}$, $\alpha$ is a metavariable (in our
expository language) that ranges over metavariables (which are primitives of
our formal system) each of which ranges over the actual individual variables
of predicate calculus (which are never mentioned in our formal system).

The constant called ``var'' above is called \texttt{setvar} in the
\texttt{set.mm} database file, but it means the same thing.  I felt
that ``var'' is a more meaningful name in the context of predicate
calculus, whose use is not limited to set theory.  For consistency we
stick with the name ``var'' throughout this Appendix, even after set
theory is introduced.
\end{quotation}}

\subsection{Free Variables and Proper Substitution}\index{free variable}
\index{proper substitution}\index{substitution!proper}

Typical representations of mathematical axioms use concepts such
as ``free variable,'' ``bound variable,'' and ``proper substitution''
as primitive notions.
A free variable is a variable that
is not a parameter of any container expression.
A bound variable is the opposite of a free variable; it is a
a variable that has been bound in a container expression.
For example, in the expression $\forall x \varphi$ (for all $x$, $\varphi$
is true), the variable $x$
is bound within the for-all ($\forall$) expression.
It is possible to change one variable to another, and that process is called
``proper substitution.''
In most books, proper substitution has a somewhat complicated recursive
definition with multiple cases based on the occurrences of free and
bound variables.
You may consult
\cite[ch.\ 3--4]{Hamilton}\index{Hamilton, Alan G.} (as well as
many other texts) for more formal details about these terms.

Using these concepts as \texttt{primitives} creates complications
for computer implementations.

In the system of Example~2, there are no primitive notions of free variable
and proper substitution.  Tarski \cite{Tarski1965} shows that this system is
logically equivalent to the more typical textbook systems that do have these
primitive notions, if we introduce these notions with appropriate definitions
and metalogic.  We could also define axioms for such systems directly,
although the recursive definitions of free variable and proper substitution
would be messy and awkward to work with.  Instead, we mention two devices that
can be used in practice to mimic these notions.  (1) Instead of introducing
special notation to express (as a logical hypothesis) ``where $x$ is not free
in $\varphi$'' we can use the logical hypothesis $\vdash(\varphi\to\forall
x\,\varphi)$.\label{effectivelybound}\index{effectively
not free}\footnote{This is a slightly weaker requirement than ``where $x$ is
not free in $\varphi$.''  If we let $\varphi$ be $x=x$, we have the theorem
$(x=x\to\forall x\,x=x)$ which satisfies the hypothesis, even though $x$ is
free in $x=x$ .  In a case like this we say that $x$ is {\em effectively not
free}\index{effectively not free} in $x=x$, since $x=x$ is logically
equivalent to $\forall x\,x=x$ in which $x$ is bound.} (2) It can be shown
that the wff $((x=y\to\varphi)\wedge\exists x(x=y\wedge\varphi))$ (with the
usual definitions of $\wedge$ and $\exists$; see Example~4 below) is logically
equivalent to ``the wff that results from proper substitution of $y$ for $x$
in $\varphi$.''  This works whether or not $x$ and $y$ are distinct.

\subsection{Metalogical Completeness}\index{metalogical completeness}

In the system of Example~2, the
following are provable pre-statements (and their reducts are
provable statements):
\begin{eqnarray*}
      & \langle\{\{x,y\}\},T,\varnothing,
               \langle \vdash\lnot\forall x\lnot x=y
               \rangle\rangle & \\
     &  \langle\varnothing,T,\varnothing,
               \langle \vdash\lnot\forall x\lnot x=x
               \rangle\rangle &
\end{eqnarray*}
whereas the following pre-statement is not to my knowledge provable (but
in any case we will pretend it's not for sake of illustration):
\begin{eqnarray*}
     &  \langle\varnothing,T,\varnothing,
               \langle \vdash\lnot\forall x\lnot x=y
               \rangle\rangle &
\end{eqnarray*}
In other words, we can prove ``$\lnot\forall x\lnot x=y$ where $x$ and $y$ are
distinct'' and separately prove ``$\lnot\forall x\lnot x=x$'', but we can't
prove the combined general case ``$\lnot\forall x\lnot x=y$'' that has no
proviso.  Now this does not compromise logical completeness, because the
variables are really metavariables and the two provable cases together cover
all possible cases.  The third case can be considered a metatheorem whose
direct proof, using the system of Example~2, lies outside the capability of the
formal system.

Also, in the system of Example~2 the following pre-statement is not to my
knowledge provable (again, a conjecture that we will pretend to be the case):
\begin{eqnarray*}
     & \langle\varnothing,T,\varnothing,
               \langle \vdash(\forall x\, \varphi\to\varphi)
               \rangle\rangle &
\end{eqnarray*}
Instead, we can only prove specific cases of $\varphi$ involving individual
metavariables, and by induction on formula length, prove as a metatheorem
outside of our formal system the general statement above.  The details of this
proof are found in \cite{Kalish}.

There does, however, exist a system of predicate calculus in which all such
``simple metatheorems'' as those above can be proved directly, and we present
it in Example~3. A {\em simple metatheorem}\index{simple metatheorem}
is any statement of the formal
system of Example~2 where all distinct variable restrictions consist of either
two individual metavariables or an individual metavariable and a wff
metavariable, and which is provable by combining cases outside the system as
above.  A system is {\em metalogically complete}\index{metalogical
completeness} if all of its simple
metatheorems are (directly) provable statements. The precise definition of
``simple metatheorem'' and the proof of the ``metalogical completeness'' of
Example~3 is found in Remark 9.6 and Theorem 9.7 of \cite{Megill}.\index{Megill,
Norman}

\begin{sloppy}
\subsection{Example~3---Metalogically Complete Predicate
Calculus with
Equality}
\end{sloppy}

For simplicity we will assume there is one binary predicate $R$;
this system suffices for set theory, where the $R$ is of course the $\in$
predicate.  We label the axioms as they appear in \cite{Megill}.  This
system is logically equivalent to that of Example~2 (when the latter is
restricted to this single binary predicate) but is also metalogically
complete.\index{metalogical completeness}

Let
\begin{itemize}
  \item[] $\mbox{\em CN}=\{\mbox{wff}, \mbox{var}, \vdash, \to, \lnot, (,),\forall,=,R\}$.
  \item[] $\mbox{\em VR}=\{\varphi,\psi,\chi,\ldots\}\cup\{x,y,z,\ldots\}$.
  \item[] $T = \{\langle \mbox{wff\ } \varphi\rangle,
             \langle \mbox{wff\ } \psi\rangle,
             \langle \mbox{wff\ } \chi\rangle,\ldots\}\cup
       \{\langle \mbox{var\ } x\rangle, \langle \mbox{var\ } y\rangle, \langle
       \mbox{var\ }z\rangle,\ldots\}$.

\noindent Then
  $\Gamma$ consists of the reducts of the following pre-statements:
    \begin{itemize}
      \item[] $\langle\varnothing,T,\varnothing,
               \langle \mbox{wff\ }(\varphi\to\psi)\rangle\rangle$
      \item[] $\langle\varnothing,T,\varnothing,
               \langle \mbox{wff\ }\lnot\varphi\rangle\rangle$
      \item[] $\langle\varnothing,T,\varnothing,
               \langle \mbox{wff\ }\forall x\,\varphi\rangle\rangle$
      \item[] $\langle\varnothing,T,\varnothing,
               \langle \mbox{wff\ }x=y\rangle\rangle$
      \item[] $\langle\varnothing,T,\varnothing,
               \langle \mbox{wff\ }Rxy\rangle\rangle$
      \item[(C1$'$)] $\langle\varnothing,T,\varnothing,
               \langle \vdash(\varphi\to(\psi\to\varphi))
               \rangle\rangle$
      \item[(C2$'$)] $\langle\varnothing,T,
               \varnothing,
               \langle \vdash((\varphi\to(\psi\to\chi))\to
               ((\varphi\to\psi)\to(\varphi\to\chi)))
               \rangle\rangle$
      \item[(C3$'$)] $\langle\varnothing,T,
               \varnothing,
               \langle \vdash((\lnot\varphi\to\lnot\psi)\to
               (\psi\to\varphi))\rangle\rangle$
      \item[(C4$'$)] $\langle\varnothing,T,
               \varnothing,
               \langle \vdash(\forall x(\forall x\,\varphi\to\psi)\to
                 (\forall x\,\varphi\to\forall x\,\psi))\rangle\rangle$
      \item[(C5$'$)] $\langle\varnothing,T,
               \varnothing,
               \langle \vdash(\forall x\,\varphi\to\varphi)\rangle\rangle$
      \item[(C6$'$)] $\langle\varnothing,T,
               \varnothing,
               \langle \vdash(\forall x\forall y\,\varphi\to
                 \forall y\forall x\,\varphi)\rangle\rangle$
      \item[(C7$'$)] $\langle\varnothing,T,
               \varnothing,
               \langle \vdash(\lnot\varphi\to\forall x\lnot\forall x\,\varphi
                 )\rangle\rangle$
      \item[(C8$'$)] $\langle\varnothing,T,
               \varnothing,
               \langle \vdash(x=y\to(x=z\to y=z))\rangle\rangle$
      \item[(C9$'$)] $\langle\varnothing,T,
               \varnothing,
               \langle \vdash(\lnot\forall x\, x=y\to(\lnot\forall x\, x=z\to
                 (y=z\to\forall x\, y=z)))\rangle\rangle$
      \item[(C10$'$)] $\langle\varnothing,T,
               \varnothing,
               \langle \vdash(\forall x(x=y\to\forall x\,\varphi)\to
                 \varphi))\rangle\rangle$
      \item[(C11$'$)] $\langle\varnothing,T,
               \varnothing,
               \langle \vdash(\forall x\, x=y\to(\forall x\,\varphi
               \to\forall y\,\varphi))\rangle\rangle$
      \item[(C12$'$)] $\langle\varnothing,T,
               \varnothing,
               \langle \vdash(x=y\to(Rxz\to Ryz))\rangle\rangle$
      \item[(C13$'$)] $\langle\varnothing,T,
               \varnothing,
               \langle \vdash(x=y\to(Rzx\to Rzy))\rangle\rangle$
      \item[(C15$'$)] $\langle\varnothing,T,
               \varnothing,
               \langle \vdash(\lnot\forall x\, x=y\to(x=y\to(\varphi
                 \to\forall x(x=y\to\varphi))))\rangle\rangle$
      \item[(C16$'$)] $\langle\{\{x,y\}\},T,
               \varnothing,
               \langle \vdash(\forall x\, x=y\to(\varphi\to\forall x\,\varphi)
                 )\rangle\rangle$
      \item[(C5)] $\langle\{\{x,\varphi\}\},T,\varnothing,
               \langle \vdash(\varphi\to\forall x\,\varphi)
               \rangle\rangle$
      \item[(MP)] $\langle\varnothing,T,
               \{\langle\vdash(\varphi\to\psi)\rangle,
                 \langle\vdash\varphi\rangle\},
               \langle\vdash\psi\rangle\rangle$
      \item[(Gen)] $\langle\varnothing,T,
               \{\langle\vdash\varphi\rangle\},
               \langle\vdash\forall x\,\varphi\rangle\rangle$
    \end{itemize}
\end{itemize}

While it is known that these axioms are ``metalogically complete,'' it is
not known whether they are independent (i.e.\ none is
redundant) in the metalogical sense; specifically, whether any axiom (possibly
with additional non-mandatory distinct-variable restrictions, for use with any
dummy variables in its proof) is provable from the others.  Note that
metalogical independence is a weaker requirement than independence in the
usual logical sense.  Not all of the above axioms are logically independent:
for example, C9$'$ can be proved as a metatheorem from the others, outside the
formal system, by combining the possible cases of distinct variables.

\subsection{Example~4---Adding Definitions}\index{definition}
There are several ways to add definitions to a formal system.  Probably the
most proper way is to consider definitions not as part of the formal system at
all but rather as abbreviations that are part of the expository metalogic
outside the formal system.  For convenience, though, we may use the formal
system itself to incorporate definitions, adding them as axiomatic extensions
to the system.  This could be done by adding a constant representing the
concept ``is defined as'' along with axioms for it. But there is a nicer way,
at least in this writer's opinion, that introduces definitions as direct
extensions to the language rather than as extralogical primitive notions.  We
introduce additional logical connectives and provide axioms for them.  For
systems of logic such as Examples 1 through 3, the additional axioms must be
conservative in the sense that no wff of the original system that was not a
theorem (when the initial term ``wff'' is replaced by ``$\vdash$'' of course)
becomes a theorem of the extended system.  In this example we extend Example~3
(or 2) with standard abbreviations of logic.

We extend $\mbox{\em CN}$ of Example~3 with new constants $\{\leftrightarrow,
\wedge,\vee,\exists\}$, corresponding to logical equivalence,\index{logical
equivalence ($\leftrightarrow$)}\index{biconditional ($\leftrightarrow$)}
conjunction,\index{conjunction ($\wedge$)} disjunction,\index{disjunction
($\vee$)} and the existential quantifier.\index{existential quantifier
($\exists$)}  We extend $\Gamma$ with the axiomatic statements that are
the reducts of the following pre-statements:
\begin{list}{}{\itemsep 0.0pt}
      \item[] $\langle\varnothing,T,\varnothing,
               \langle \mbox{wff\ }(\varphi\leftrightarrow\psi)\rangle\rangle$
      \item[] $\langle\varnothing,T,\varnothing,
               \langle \mbox{wff\ }(\varphi\vee\psi)\rangle\rangle$
      \item[] $\langle\varnothing,T,\varnothing,
               \langle \mbox{wff\ }(\varphi\wedge\psi)\rangle\rangle$
      \item[] $\langle\varnothing,T,\varnothing,
               \langle \mbox{wff\ }\exists x\, \varphi\rangle\rangle$
  \item[] $\langle\varnothing,T,\varnothing,
     \langle\vdash ( ( \varphi \leftrightarrow \psi ) \to
     ( \varphi \to \psi ) )\rangle\rangle$
  \item[] $\langle\varnothing,T,\varnothing,
     \langle\vdash ((\varphi\leftrightarrow\psi)\to
    (\psi\to\varphi))\rangle\rangle$
  \item[] $\langle\varnothing,T,\varnothing,
     \langle\vdash ((\varphi\to\psi)\to(
     (\psi\to\varphi)\to(\varphi
     \leftrightarrow\psi)))\rangle\rangle$
  \item[] $\langle\varnothing,T,\varnothing,
     \langle\vdash (( \varphi \wedge \psi ) \leftrightarrow\neg ( \varphi
     \to \neg \psi )) \rangle\rangle$
  \item[] $\langle\varnothing,T,\varnothing,
     \langle\vdash (( \varphi \vee \psi ) \leftrightarrow (\neg \varphi
     \to \psi )) \rangle\rangle$
  \item[] $\langle\varnothing,T,\varnothing,
     \langle\vdash (\exists x \,\varphi\leftrightarrow
     \lnot \forall x \lnot \varphi)\rangle\rangle$
\end{list}
The first three logical axioms (statements containing ``$\vdash$'') introduce
and effectively define logical equivalence, ``$\leftrightarrow$''.  The last
three use ``$\leftrightarrow$'' to effectively mean ``is defined as.''

\subsection{Example~5---ZFC Set Theory}\index{ZFC set theory}

Here we add to the system of Example~4 the axioms of Zermelo--Fraenkel set
theory with Choice.  For convenience we make use of the
definitions in Example~4.

In the $\mbox{\em CN}$ of Example~4 (which extends Example~3), we replace the symbol $R$
with the symbol $\in$.
More explicitly, we remove from $\Gamma$ of Example~4 the three
axiomatic statements containing $R$ and replace them with the
reducts of the following:
\begin{list}{}{\itemsep 0.0pt}
      \item[] $\langle\varnothing,T,\varnothing,
               \langle \mbox{wff\ }x\in y\rangle\rangle$
      \item[] $\langle\varnothing,T,
               \varnothing,
               \langle \vdash(x=y\to(x\in z\to y\in z))\rangle\rangle$
      \item[] $\langle\varnothing,T,
               \varnothing,
               \langle \vdash(x=y\to(z\in x\to z\in y))\rangle\rangle$
\end{list}
Letting $D=\{\{\alpha,\beta\}\in \mbox{\em DV}\,|\alpha,\beta\in \mbox{\em
Vr}\}$ (in other words all individual variables must be distinct), we extend
$\Gamma$ with the ZFC axioms, called
\index{Axiom of Extensionality}
\index{Axiom of Replacement}
\index{Axiom of Union}
\index{Axiom of Power Sets}
\index{Axiom of Regularity}
\index{Axiom of Infinity}
\index{Axiom of Choice}
Extensionality, Replacement, Union, Power
Set, Regularity, Infinity, and Choice, that are the reducts of:
\begin{list}{}{\itemsep 0.0pt}
      \item[Ext] $\langle D,T,
               \varnothing,
               \langle\vdash (\forall x(x\in y\leftrightarrow x \in z)\to y
               =z) \rangle\rangle$
      \item[Rep] $\langle D,T,
               \varnothing,
               \langle\vdash\exists x ( \exists y \forall z (\varphi \to z = y
                        ) \to
                        \forall z ( z \in x \leftrightarrow \exists x ( x \in
                        y \wedge \forall y\,\varphi ) ) )\rangle\rangle$
      \item[Un] $\langle D,T,
               \varnothing,
               \langle\vdash \exists x \forall y ( \exists x ( y \in x \wedge
               x \in z ) \to y \in x ) \rangle\rangle$
      \item[Pow] $\langle D,T,
               \varnothing,
               \langle\vdash \exists x \forall y ( \forall x ( x \in y \to x
               \in z ) \to y \in x ) \rangle\rangle$
      \item[Reg] $\langle D,T,
               \varnothing,
               \langle\vdash (  x \in y \to
                 \exists x ( x \in y \wedge \forall z ( z \in x \to \lnot z
                \in y ) ) ) \rangle\rangle$
      \item[Inf] $\langle D,T,
               \varnothing,
               \langle\vdash \exists x(y\in x\wedge\forall y(y\in
               x\to
               \exists z(y \in z\wedge z\in x))) \rangle\rangle$
      \item[AC] $\langle D,T,
               \varnothing,
               \langle\vdash \exists x \forall y \forall z ( ( y \in z
               \wedge z \in w ) \to \exists w \forall y ( \exists w
              ( ( y \in z \wedge z \in w ) \wedge ( y \in w \wedge w \in x
              ) ) \leftrightarrow y = w ) ) \rangle\rangle$
\end{list}

\subsection{Example~6---Class Notation in Set Theory}\label{class}

A powerful device that makes set theory easier (and that we have
been using all along in our informal expository language) is {\em class
abstraction notation}.\index{class abstraction}\index{abstraction class}  The
definitions we introduce are rigorously justified
as conservative by Takeuti and Zaring \cite{Takeuti}\index{Takeuti, Gaisi} or
Quine \cite{Quine}\index{Quine, Willard Van Orman}.  The key idea is to
introduce the notation $\{x|\mbox{---}\}$ which means ``the class of all $x$
such that ---'' for abstraction classes and introduce (meta)variables that
range over them.  An abstraction class may or may not be a set, depending on
whether it exists (as a set).  A class that does not exist is
called a {\em proper class}.\index{proper class}\index{class!proper}

To illustrate the use of abstraction classes we will provide some examples
of definitions that make use of them:  the empty set, class union, and
unordered pair.  Many other such definitions can be found in the
Metamath set theory database,
\texttt{set.mm}.\index{set theory database (\texttt{set.mm})}

% We intentionally break up the sequence of math symbols here
% because otherwise the overlong line goes beyond the page in narrow mode.
We extend $\mbox{\em CN}$ of Example~5 with new symbols $\{$
$\mbox{class},$ $\{,$ $|,$ $\},$ $\varnothing,$ $\cup,$ $,$ $\}$
where the inner braces and last comma are
constant symbols. (As before,
our dual use of some mathematical symbols for both our expository
language and as primitives of the formal system should be clear from context.)

We extend $\mbox{\em VR}$ of Example~5 with a set of {\em class
variables}\index{class variable}
$\{A,B,C,\ldots\}$. We extend the $T$ of Example~5 with $\{\langle
\mbox{class\ } A\rangle, \langle \mbox{class\ }B\rangle, \langle \mbox{class\ }
C\rangle,\ldots\}$.

To
introduce our definitions,
we add to $\Gamma$ of Example~5 the axiomatic statements
that are the reducts of the following pre-statements:
\begin{list}{}{\itemsep 0.0pt}
      \item[] $\langle\varnothing,T,\varnothing,
               \langle \mbox{class\ }x\rangle\rangle$
      \item[] $\langle\varnothing,T,\varnothing,
               \langle \mbox{class\ }\{x|\varphi\}\rangle\rangle$
      \item[] $\langle\varnothing,T,\varnothing,
               \langle \mbox{wff\ }A=B\rangle\rangle$
      \item[] $\langle\varnothing,T,\varnothing,
               \langle \mbox{wff\ }A\in B\rangle\rangle$
      \item[Ab] $\langle\varnothing,T,\varnothing,
               \langle \vdash ( y \in \{ x |\varphi\} \leftrightarrow
                  ( ( x = y \to\varphi) \wedge \exists x ( x = y
                  \wedge\varphi) ))
               \rangle\rangle$
      \item[Eq] $\langle\{\{x,A\},\{x,B\}\},T,\varnothing,
               \langle \vdash ( A = B \leftrightarrow
               \forall x ( x \in A \leftrightarrow x \in B ) )
               \rangle\rangle$
      \item[El] $\langle\{\{x,A\},\{x,B\}\},T,\varnothing,
               \langle \vdash ( A \in B \leftrightarrow \exists x
               ( x = A \wedge x \in B ) )
               \rangle\rangle$
\end{list}
Here we say that an individual variable is a class; $\{x|\varphi\}$ is a
class; and we extend the definition of a wff to include class equality and
membership.  Axiom Ab defines membership of a variable in a class abstraction;
the right-hand side can be read as ``the wff that results from proper
substitution of $y$ for $x$ in $\varphi$.''\footnote{Note that this definition
makes unnecessary the introduction of a separate notation similar to
$\varphi(x|y)$ for proper substitution, although we may choose to do so to be
conventional.  Incidentally, $\varphi(x|y)$ as it stands would be ambiguous in
the formal systems of our examples, since we wouldn't know whether
$\lnot\varphi(x|y)$ meant $\lnot(\varphi(x|y))$ or $(\lnot\varphi)(x|y)$.
Instead, we would have to use an unambiguous variant such as $(\varphi\,
x|y)$.}  Axioms Eq and El extend the meaning of the existing equality and
membership connectives.  This is potentially dangerous and requires careful
justification.  For example, from Eq we can derive the Axiom of Extensionality
with predicate logic alone; thus in principle we should include the Axiom of
Extensionality as a logical hypothesis.  However we do not bother to do this
since we have already presupposed that axiom earlier. The distinct variable
restrictions should be read ``where $x$ does not occur in $A$ or $B$.''  We
typically do this when the right-hand side of a definition involves an
individual variable not in the expression being defined; it is done so that
the right-hand side remains independent of the particular ``dummy'' variable
we use.

We continue to add to $\Gamma$ the following definitions
(i.e. the reducts of the following pre-statements) for empty
set,\index{empty set} class union,\index{union} and unordered
pair.\index{unordered pair}  They should be self-explanatory.  Analogous to our
use of ``$\leftrightarrow$'' to define new wffs in Example~4, we use ``$=$''
to define new abstraction terms, and both may be read informally as ``is
defined as'' in this context.
\begin{list}{}{\itemsep 0.0pt}
      \item[] $\langle\varnothing,T,\varnothing,
               \langle \mbox{class\ }\varnothing\rangle\rangle$
      \item[] $\langle\varnothing,T,\varnothing,
               \langle \vdash \varnothing = \{ x | \lnot x = x \}
               \rangle\rangle$
      \item[] $\langle\varnothing,T,\varnothing,
               \langle \mbox{class\ }(A\cup B)\rangle\rangle$
      \item[] $\langle\{\{x,A\},\{x,B\}\},T,\varnothing,
               \langle \vdash ( A \cup B ) = \{ x | ( x \in A \vee x \in B ) \}
               \rangle\rangle$
      \item[] $\langle\varnothing,T,\varnothing,
               \langle \mbox{class\ }\{A,B\}\rangle\rangle$
      \item[] $\langle\{\{x,A\},\{x,B\}\},T,\varnothing,
               \langle \vdash \{ A , B \} = \{ x | ( x = A \vee x = B ) \}
               \rangle\rangle$
\end{list}

\section{Metamath as a Formal System}\label{theorymm}

This section presupposes a familiarity with the Metamath computer language.

Our theory describes formal systems and their universes.  The Metamath
language provides a way of representing these set-theoretical objects to
a computer.  A Metamath database, being a finite set of {\sc ascii}
characters, can usually describe only a subset of a formal system and
its universe, which are typically infinite.  However the database can
contain as large a finite subset of the formal system and its universe
as we wish.  (Of course a Metamath set theory database can, in
principle, indirectly describe an entire infinite formal system by
formalizing the expository language in this Appendix.)

For purpose of our discussion, we assume the Metamath database
is in the simple form described on p.~\pageref{framelist},
consisting of all constant and variable declarations at the beginning,
followed by a sequence of extended frames each
delimited by \texttt{\$\char`\{} and \texttt{\$\char`\}}.  Any Metamath database can
be converted to this form, as described on p.~\pageref{frameconvert}.

The math symbol tokens of a Metamath source file, which are declared
with \texttt{\$c} and \texttt{\$v} statements, are names we assign to
representatives of $\mbox{\em CN}$ and $\mbox{\em VR}$.  For
definiteness we could assume that the first math symbol declared as a
variable corresponds to $v_0$, the second to $v_1$, etc., although the
exact correspondence we choose is not important.

In the Metamath language, each \texttt{\$d}, \texttt{\$f}, and
 \texttt{\$e} source
statement in an extended frame (Section~\ref{frames})
corresponds respectively to a member of the
collections $D$, $T$, and $H$ in a formal system statement $\langle
D_M,T_M,H,A\rangle$.  The math symbol strings following these Metamath keywords
correspond to a variable pair (in the case of \texttt{\$d}) or an expression (for
the other two keywords). The math symbol string following a \texttt{\$a} source
statement corresponds to expression $A$ in an axiomatic statement of the
formal system; the one following a \texttt{\$p} source statement corresponds to
$A$ in a provable statement that is not axiomatic.  In other words, each
extended frame in a Metamath database corresponds to
a pre-statement of the formal system, and a frame corresponds to
a statement of the formal system.  (Don't confuse the two meanings of
``statement'' here.  A statement of the formal system corresponds to the
several statements in a Metamath database that may constitute a
frame.)

In order for the computer to verify that a formal system statement is
provable, each \texttt{\$p} source statement is accompanied by a proof.
However, the proof does not correspond to anything in the formal system
but is simply a way of communicating to the computer the information
needed for its verification.  The proof tells the computer {\em how to
construct} specific members of closure of the formal system
pre-statement corresponding to the extended frame of the \texttt{\$p}
statement.  The final result of the construction is the member of the
closure that matches the \texttt{\$p} statement.  The abstract formal
system, on the other hand, is concerned only with the {\em existence} of
members of the closure.

As mentioned on p.~\pageref{exampleref}, Examples 1 and 3--6 in the
previous Section parallel the development of logic and set theory in the
Metamath database
\texttt{set.mm}.\index{set theory database (\texttt{set.mm})} You may
find it instructive to compare them.


\chapter{The MIU System}
\label{MIU}
\index{formal system}
\index{MIU-system}

The following is a listing of the file \texttt{miu.mm}.  It is self-explanatory.

%%%%%%%%%%%%%%%%%%%%%%%%%%%%%%%%%%%%%%%%%%%%%%%%%%%%%%%%%%%%

\begin{verbatim}
$( The MIU-system:  A simple formal system $)

$( Note:  This formal system is unusual in that it allows
empty wffs.  To work with a proof, you must type
SET EMPTY_SUBSTITUTION ON before using the PROVE command.
By default, this is OFF in order to reduce the number of
ambiguous unification possibilities that have to be selected
during the construction of a proof.  $)

$(
Hofstadter's MIU-system is a simple example of a formal
system that illustrates some concepts of Metamath.  See
Douglas R. Hofstadter, _Goedel, Escher, Bach:  An Eternal
Golden Braid_ (Vintage Books, New York, 1979), pp. 33ff. for
a description of the MIU-system.

The system has 3 constant symbols, M, I, and U.  The sole
axiom of the system is MI. There are 4 rules:
     Rule I:  If you possess a string whose last letter is I,
     you can add on a U at the end.
     Rule II:  Suppose you have Mx.  Then you may add Mxx to
     your collection.
     Rule III:  If III occurs in one of the strings in your
     collection, you may make a new string with U in place
     of III.
     Rule IV:  If UU occurs inside one of your strings, you
     can drop it.
Unfortunately, Rules III and IV do not have unique results:
strings could have more than one occurrence of III or UU.
This requires that we introduce the concept of an "MIU
well-formed formula" or wff, which allows us to construct
unique symbol sequences to which Rules III and IV can be
applied.
$)

$( First, we declare the constant symbols of the language.
Note that we need two symbols to distinguish the assertion
that a sequence is a wff from the assertion that it is a
theorem; we have arbitrarily chosen "wff" and "|-". $)
      $c M I U |- wff $. $( Declare constants $)

$( Next, we declare some variables. $)
     $v x y $.

$( Throughout our theory, we shall assume that these
variables represent wffs. $)
 wx   $f wff x $.
 wy   $f wff y $.

$( Define MIU-wffs.  We allow the empty sequence to be a
wff. $)

$( The empty sequence is a wff. $)
 we   $a wff $.
$( "M" after any wff is a wff. $)
 wM   $a wff x M $.
$( "I" after any wff is a wff. $)
 wI   $a wff x I $.
$( "U" after any wff is a wff. $)
 wU   $a wff x U $.

$( Assert the axiom. $)
 ax   $a |- M I $.

$( Assert the rules. $)
 ${
   Ia   $e |- x I $.
$( Given any theorem ending with "I", it remains a theorem
if "U" is added after it.  (We distinguish the label I_
from the math symbol I to conform to the 24-Jun-2006
Metamath spec.) $)
   I_    $a |- x I U $.
 $}
 ${
IIa  $e |- M x $.
$( Given any theorem starting with "M", it remains a theorem
if the part after the "M" is added again after it. $)
   II   $a |- M x x $.
 $}
 ${
   IIIa $e |- x I I I y $.
$( Given any theorem with "III" in the middle, it remains a
theorem if the "III" is replaced with "U". $)
   III  $a |- x U y $.
 $}
 ${
   IVa  $e |- x U U y $.
$( Given any theorem with "UU" in the middle, it remains a
theorem if the "UU" is deleted. $)
   IV   $a |- x y $.
  $}

$( Now we prove the theorem MUIIU.  You may be interested in
comparing this proof with that of Hofstadter (pp. 35 - 36).
$)
 theorem1  $p |- M U I I U $=
      we wM wU wI we wI wU we wU wI wU we wM we wI wU we wM
      wI wI wI we wI wI we wI ax II II I_ III II IV $.
\end{verbatim}\index{well-formed formula (wff)}

The \texttt{show proof /lemmon/renumber} command
yields the following display.  It is very similar
to the one in \cite[pp.~35--36]{Hofstadter}.\index{Hofstadter, Douglas R.}

\begin{verbatim}
1 ax             $a |- M I
2 1 II           $a |- M I I
3 2 II           $a |- M I I I I
4 3 I_           $a |- M I I I I U
5 4 III          $a |- M U I U
6 5 II           $a |- M U I U U I U
7 6 IV           $a |- M U I I U
\end{verbatim}

We note that Hofstadter's ``MU-puzzle,'' which asks whether
MU is a theorem of the MIU-system, cannot be answered using
the system above because the MU-puzzle is a question {\em
about} the system.  To prove the answer to the MU-puzzle,
a much more elaborate system is needed, namely one that
models the MIU-system within set theory.  (Incidentally, the
answer to the MU-puzzle is no.)

\chapter{Metamath Language EBNF}%
\label{BNF}%
\index{Metamath Language EBNF}

The following is a formal description of the basic Metamath language syntax
(with compressed proofs and support for unknown proof steps).
It is defined using the
Extended Backus--Naur Form (EBNF)\index{Extended Backus--Naur Form}\index{EBNF}
notation from W3C\index{W3C}
\textit{Extensible Markup Language (XML) 1.0 (Fifth Edition)}
(W3C Recommendation 26 November 2008) at
\url{https://www.w3.org/TR/xml/#sec-notation}.

The \texttt{database}
rule is processed until the end of the file (\texttt{EOF}).
The rules eventually require reading whitespace-separated tokens.
A token has an upper-case definition (see below)
or is a string constant in a non-token (such as \texttt{'\$a'}).
We intend for this to be correct, but if there is a conflict the
rules of section \ref{spec} govern. That section also discusses
non-syntax restrictions not shown here
(e.g., that each new label token
defined in a \texttt{hypothesis-stmt} or \texttt{assert-stmt}
must be unique).

\begin{verbatim}
database ::= outermost-scope-stmt*

outermost-scope-stmt ::=
  include-stmt | constant-stmt | stmt

/* File inclusion command; process file as a database.
   Databases should NOT have a comment in the filename. */
include-stmt ::= '$[' filename '$]'

/* Constant symbols declaration. */
constant-stmt ::= '$c' constant+ '$.'

/* A normal statement can occur in any scope. */
stmt ::= block | variable-stmt | disjoint-stmt |
  hypothesis-stmt | assert-stmt

/* A block. You can have 0 statements in a block. */
block ::= '${' stmt* '$}'

/* Variable symbols declaration. */
variable-stmt ::= '$v' variable+ '$.'

/* Disjoint variables. Simple disjoint statements have
   2 variables, i.e., "variable*" is empty for them. */
disjoint-stmt ::= '$d' variable variable variable* '$.'

hypothesis-stmt ::= floating-stmt | essential-stmt

/* Floating (variable-type) hypothesis. */
floating-stmt ::= LABEL '$f' typecode variable '$.'

/* Essential (logical) hypothesis. */
essential-stmt ::= LABEL '$e' typecode MATH-SYMBOL* '$.'

assert-stmt ::= axiom-stmt | provable-stmt

/* Axiomatic assertion. */
axiom-stmt ::= LABEL '$a' typecode MATH-SYMBOL* '$.'

/* Provable assertion. */
provable-stmt ::= LABEL '$p' typecode MATH-SYMBOL*
  '$=' proof '$.'

/* A proof. Proofs may be interspersed by comments.
   If '?' is in a proof it's an "incomplete" proof. */
proof ::= uncompressed-proof | compressed-proof
uncompressed-proof ::= (LABEL | '?')+
compressed-proof ::= '(' LABEL* ')' COMPRESSED-PROOF-BLOCK+

typecode ::= constant

filename ::= MATH-SYMBOL /* No whitespace or '$' */
constant ::= MATH-SYMBOL
variable ::= MATH-SYMBOL
\end{verbatim}

\needspace{2\baselineskip}
A \texttt{frame} is a sequence of 0 or more
\texttt{disjoint-{\allowbreak}stmt} and
\texttt{hypotheses-{\allowbreak}stmt} statements
(possibly interleaved with other non-\texttt{assert-stmt} statements)
followed by one \texttt{assert-stmt}.

\needspace{3\baselineskip}
Here are the rules for lexical processing (tokenization) beyond
the constant tokens shown above.
By convention these tokenization rules have upper-case names.
Every token is read for the longest possible length.
Whitespace-separated tokens are read sequentially;
note that the separating whitespace and \texttt{\$(} ... \texttt{\$)}
comments are skipped.

If a token definition uses another token definition, the whole thing
is considered a single token.
A pattern that is only part of a full token has a name beginning
with an underscore (``\_'').
An implementation could tokenize many tokens as a
\texttt{PRINTABLE-SEQUENCE}
and then check if it meets the more specific rule shown here.

Comments do not nest, and both \texttt{\$(} and \texttt{\$)}
have to be surrounded
by at least one whitespace character (\texttt{\_WHITECHAR}).
Technically comments end without consuming the trailing
\texttt{\_WHITECHAR}, but the trailing
\texttt{\_WHITECHAR} gets ignored anyway so we ignore that detail here.
Metamath language processors
are not required to support \texttt{\$)} followed
immediately by a bare end-of-file, because the closing
comment symbol is supposed to be followed by a
\texttt{\_WHITECHAR} such as a newline.

\begin{verbatim}
PRINTABLE-SEQUENCE ::= _PRINTABLE-CHARACTER+

MATH-SYMBOL ::= (_PRINTABLE-CHARACTER - '$')+

/* ASCII non-whitespace printable characters */
_PRINTABLE-CHARACTER ::= [#x21-#x7e]

LABEL ::= ( _LETTER-OR-DIGIT | '.' | '-' | '_' )+

_LETTER-OR-DIGIT ::= [A-Za-z0-9]

COMPRESSED-PROOF-BLOCK ::= ([A-Z] | '?')+

/* Define whitespace between tokens. The -> SKIP
   means that when whitespace is seen, it is
   skipped and we simply read again. */
WHITESPACE ::= (_WHITECHAR+ | _COMMENT) -> SKIP

/* Comments. $( ... $) and do not nest. */
_COMMENT ::= '$(' (_WHITECHAR+ (PRINTABLE-SEQUENCE - '$)')*
  _WHITECHAR+ '$)' _WHITECHAR

/* Whitespace: (' ' | '\t' | '\r' | '\n' | '\f') */
_WHITECHAR ::= [#x20#x09#x0d#x0a#x0c]
\end{verbatim}
% This EBNF was developed as a collaboration between
% David A. Wheeler\index{Wheeler, David A.},
% Mario Carneiro\index{Carneiro, Mario}, and
% Benoit Jubin\index{Jubin, Benoit}, inspired by a request
% (and a lot of initial work) by Benoit Jubin.
%
% \chapter{Disclaimer and Trademarks}
%
% Information in this document is subject to change without notice and does not
% represent a commitment on the part of Norman Megill.
% \vspace{2ex}
%
% \noindent Norman D. Megill makes no warranties, either express or implied,
% regarding the Metamath computer software package.
%
% \vspace{2ex}
%
% \noindent Any trademarks mentioned in this book are the property of
% their respective owners.  The name ``Metamath'' is a trademark of
% Norman Megill.
%
\cleardoublepage
\phantomsection  % fixes the link anchor
\addcontentsline{toc}{chapter}{\bibname}

\bibliography{metamath}
%\input{metamath.bbl}

\raggedright
\cleardoublepage
\phantomsection % fixes the link anchor
\addcontentsline{toc}{chapter}{\indexname}
%\printindex   ??
\input{metamath.ind}

\end{document}



\end{document}



\raggedright
\cleardoublepage
\phantomsection % fixes the link anchor
\addcontentsline{toc}{chapter}{\indexname}
%\printindex   ??
% metamath.tex - Version of 24-May-2019
% If you change the date above, also change the "Printed date" below.
% SPDX-License-Identifier: CC0-1.0
%
%                              PUBLIC DOMAIN
%
% This file (specifically, the version of this file with the above date)
% has been released into the Public Domain per the
% Creative Commons CC0 1.0 Universal (CC0 1.0) Public Domain Dedication
% https://creativecommons.org/publicdomain/zero/1.0/
%
% The public domain release applies worldwide.  In case this is not
% legally possible, the right is granted to use the work for any purpose,
% without any conditions, unless such conditions are required by law.
%
% Several short, attributed quotations from copyrighted works
% appear in this file under the ``fair use'' provision of Section 107 of
% the United States Copyright Act (Title 17 of the {\em United States
% Code}).  The public-domain status of this file is not applicable to
% those quotations.
%
% Norman Megill - email: nm(at)alum(dot)mit(dot)edu
%
% David A. Wheeler also donates his improvements to this file to the
% public domain per the CC0.  He works at the Institute for Defense Analyses
% (IDA), but IDA has agreed that this Metamath work is outside its "lane"
% and is not a work by IDA.  This was specifically confirmed by
% Margaret E. Myers (Division Director of the Information Technology
% and Systems Division) on 2019-05-24 and by Ben Lindorf (General Counsel)
% on 2019-05-22.

% This file, 'metamath.tex', is self-contained with everything needed to
% generate the the PDF file 'metamath.pdf' (the _Metamath_ book) on
% standard LaTeX 2e installations.  The auxiliary files are embedded with
% "filecontents" commands.  To generate metamath.pdf file, run these
% commands under Linux or Cygwin in the directory that contains
% 'metamath.tex':
%
%   rm -f realref.sty metamath.bib
%   touch metamath.ind
%   pdflatex metamath
%   pdflatex metamath
%   bibtex metamath
%   makeindex metamath
%   pdflatex metamath
%   pdflatex metamath
%
% The warnings that occur in the initial runs of pdflatex can be ignored.
% For the final run,
%
%   egrep -i 'error|warn' metamath.log
%
% should show exactly these 5 warnings:
%
%   LaTeX Warning: File `realref.sty' already exists on the system.
%   LaTeX Warning: File `metamath.bib' already exists on the system.
%   LaTeX Font Warning: Font shape `OMS/cmtt/m/n' undefined
%   LaTeX Font Warning: Font shape `OMS/cmtt/bx/n' undefined
%   LaTeX Font Warning: Some font shapes were not available, defaults
%       substituted.
%
% Search for "Uncomment" below if you want to suppress hyperlink boxes
% in the PDF output file
%
% TYPOGRAPHICAL NOTES:
% * It is customary to use an en dash (--) to "connect" names of different
%   people (and to denote ranges), and use a hyphen (-) for a
%   single compound name. Examples of connected multiple people are
%   Zermelo--Fraenkel, Schr\"{o}der--Bernstein, Tarski--Grothendieck,
%   Hewlett--Packard, and Backus--Naur.  Examples of a single person with
%   a compound name include Levi-Civita, Mittag-Leffler, and Burali-Forti.
% * Use non-breaking spaces after page abbreviations, e.g.,
%   p.~\pageref{note2002}.
%
% --------------------------- Start of realref.sty -----------------------------
\begin{filecontents}{realref.sty}
% Save the following as realref.sty.
% You can then use it with \usepackage{realref}
%
% This has \pageref jumping to the page on which the ref appears,
% \ref jumping to the point of the anchor, and \sectionref
% jumping to the start of section.
%
% Author:  Anthony Williams
%          Software Engineer
%          Nortel Networks Optical Components Ltd
% Date:    9 Nov 2001 (posted to comp.text.tex)
%
% The following declaration was made by Anthony Williams on
% 24 Jul 2006 (private email to Norman Megill):
%
%   ``I hereby donate the code for realref.sty posted on the
%   comp.text.tex newsgroup on 9th November 2001, accessible from
%   http://groups.google.com/group/comp.text.tex/msg/5a0e1cc13ea7fbb2
%   to the public domain.''
%
\ProvidesPackage{realref}
\RequirePackage[plainpages=false,pdfpagelabels=true]{hyperref}
\def\realref@anchorname{}
\AtBeginDocument{%
% ensure every label is a possible hyperlink target
\let\realref@oldrefstepcounter\refstepcounter%
\DeclareRobustCommand{\refstepcounter}[1]{\realref@oldrefstepcounter{#1}
\edef\realref@anchorname{\string #1.\@currentlabel}%
}%
\let\realref@oldlabel\label%
\DeclareRobustCommand{\label}[1]{\realref@oldlabel{#1}\hypertarget{#1}{}%
\@bsphack\protected@write\@auxout{}{%
    \string\expandafter\gdef\protect\csname
    page@num.#1\string\endcsname{\thepage}%
    \string\expandafter\gdef\protect\csname
    ref@num.#1\string\endcsname{\@currentlabel}%
    \string\expandafter\gdef\protect\csname
    sectionref@name.#1\string\endcsname{\realref@anchorname}%
}\@esphack}%
\DeclareRobustCommand\pageref[1]{{\edef\a{\csname
            page@num.#1\endcsname}\expandafter\hyperlink{page.\a}{\a}}}%
\DeclareRobustCommand\ref[1]{{\edef\a{\csname
            ref@num.#1\endcsname}\hyperlink{#1}{\a}}}%
\DeclareRobustCommand\sectionref[1]{{\edef\a{\csname
            ref@num.#1\endcsname}\edef\b{\csname
            sectionref@name.#1\endcsname}\hyperlink{\b}{\a}}}%
}
\end{filecontents}
% ---------------------------- End of realref.sty ------------------------------

% --------------------------- Start of metamath.bib -----------------------------
\begin{filecontents}{metamath.bib}
@book{Albers, editor = "Donald J. Albers and G. L. Alexanderson",
  title = "Mathematical People",
  publisher = "Contemporary Books, Inc.",
  address = "Chicago",
  note = "[QA28.M37]",
  year = 1985 }
@book{Anderson, author = "Alan Ross Anderson and Nuel D. Belnap",
  title = "Entailment",
  publisher = "Princeton University Press",
  address = "Princeton",
  volume = 1,
  note = "[QA9.A634 1975 v.1]",
  year = 1975}
@book{Barrow, author = "John D. Barrow",
  title = "Theories of Everything:  The Quest for Ultimate Explanation",
  publisher = "Oxford University Press",
  address = "Oxford",
  note = "[Q175.B225]",
  year = 1991 }
@book{Behnke,
  editor = "H. Behnke and F. Backmann and K. Fladt and W. S{\"{u}}ss",
  title = "Fundamentals of Mathematics",
  volume = "I",
  publisher = "The MIT Press",
  address = "Cambridge, Massachusetts",
  note = "[QA37.2.B413]",
  year = 1974 }
@book{Bell, author = "J. L. Bell and M. Machover",
  title = "A Course in Mathematical Logic",
  publisher = "North-Holland",
  address = "Amsterdam",
  note = "[QA9.B3953]",
  year = 1977 }
@inproceedings{Blass, author = "Andrea Blass",
  title = "The Interaction Between Category Theory and Set Theory",
  pages = "5--29",
  booktitle = "Mathematical Applications of Category Theory (Proceedings
     of the Special Session on Mathematical Applications
     Category Theory, 89th Annual Meeting of the American Mathematical
     Society, held in Denver, Colorado January 5--9, 1983)",
  editor = "John Walter Gray",
  year = 1983,
  note = "[QA169.A47 1983]",
  publisher = "American Mathematical Society",
  address = "Providence, Rhode Island"}
@proceedings{Bledsoe, editor = "W. W. Bledsoe and D. W. Loveland",
  title = "Automated Theorem Proving:  After 25 Years (Proceedings
     of the Special Session on Automatic Theorem Proving,
     89th Annual Meeting of the American Mathematical
     Society, held in Denver, Colorado January 5--9, 1983)",
  year = 1983,
  note = "[QA76.9.A96.S64 1983]",
  publisher = "American Mathematical Society",
  address = "Providence, Rhode Island" }
@book{Boolos, author = "George S. Boolos and Richard C. Jeffrey",
  title = "Computability and Log\-ic",
  publisher = "Cambridge University Press",
  edition = "third",
  address = "Cambridge",
  note = "[QA9.59.B66 1989]",
  year = 1989 }
@book{Campbell, author = "John Campbell",
  title = "Programmer's Progress",
  publisher = "White Star Software",
  address = "Box 51623, Palo Alto, CA 94303",
  year = 1991 }
@article{DBLP:journals/corr/Carneiro14,
  author    = {Mario Carneiro},
  title     = {Conversion of {HOL} Light proofs into Metamath},
  journal   = {CoRR},
  volume    = {abs/1412.8091},
  year      = {2014},
  url       = {http://arxiv.org/abs/1412.8091},
  archivePrefix = {arXiv},
  eprint    = {1412.8091},
  timestamp = {Mon, 13 Aug 2018 16:47:05 +0200},
  biburl    = {https://dblp.org/rec/bib/journals/corr/Carneiro14},
  bibsource = {dblp computer science bibliography, https://dblp.org}
}
@article{CarneiroND,
  author    = {Mario Carneiro},
  title     = {Natural Deductions in the Metamath Proof Language},
  url       = {http://us.metamath.org/ocat/natded.pdf},
  year      = 2014
}
@inproceedings{Chou, author = "Shang-Ching Chou",
  title = "Proving Elementary Geometry Theorems Using {W}u's Algorithm",
  pages = "243--286",
  booktitle = "Automated Theorem Proving:  After 25 Years (Proceedings
     of the Special Session on Automatic Theorem Proving,
     89th Annual Meeting of the American Mathematical
     Society, held in Denver, Colorado January 5--9, 1983)",
  editor = "W. W. Bledsoe and D. W. Loveland",
  year = 1983,
  note = "[QA76.9.A96.S64 1983]",
  publisher = "American Mathematical Society",
  address = "Providence, Rhode Island" }
@book{Clemente, author = "Daniel Clemente Laboreo",
  title = "Introduction to natural deduction",
  year = 2014,
  url = "http://www.danielclemente.com/logica/dn.en.pdf" }
@incollection{Courant, author = "Richard Courant and Herbert Robbins",
  title = "Topology",
  pages = "573--590",
  booktitle = "The World of Mathematics, Volume One",
  editor = "James R. Newman",
  publisher = "Simon and Schuster",
  address = "New York",
  note = "[QA3.W67 1988]",
  year = 1956 }
@book{Curry, author = "Haskell B. Curry",
  title = "Foundations of Mathematical Logic",
  publisher = "Dover Publications, Inc.",
  address = "New York",
  note = "[QA9.C976 1977]",
  year = 1977 }
@book{Davis, author = "Philip J. Davis and Reuben Hersh",
  title = "The Mathematical Experience",
  publisher = "Birkh{\"{a}}user Boston",
  address = "Boston",
  note = "[QA8.4.D37 1982]",
  year = 1981 }
@incollection{deMillo,
  author = "Richard de Millo and Richard Lipton and Alan Perlis",
  title = "Social Processes and Proofs of Theorems and Programs",
  pages = "267--285",
  booktitle = "New Directions in the Philosophy of Mathematics",
  editor = "Thomas Tymoczko",
  publisher = "Birkh{\"{a}}user Boston, Inc.",
  address = "Boston",
  note = "[QA8.6.N48 1986]",
  year = 1986 }
@book{Edwards, author = "Robert E. Edwards",
  title = "A Formal Background to Mathematics",
  publisher = "Springer-Verlag",
  address = "New York",
  note = "[QA37.2.E38 v.1a]",
  year = 1979 }
@book{Enderton, author = "Herbert B. Enderton",
  title = "Elements of Set Theory",
  publisher = "Academic Press, Inc.",
  address = "San Diego",
  note = "[QA248.E5]",
  year = 1977 }
@book{Goodstein, author = "R. L. Goodstein",
  title = "Development of Mathematical Logic",
  publisher = "Springer-Verlag New York Inc.",
  address = "New York",
  note = "[QA9.G6554]",
  year = 1971 }
@book{Guillen, author = "Michael Guillen",
  title = "Bridges to Infinity",
  publisher = "Jeremy P. Tarcher, Inc.",
  address = "Los Angeles",
  note = "[QA93.G8]",
  year = 1983 }
@book{Hamilton, author = "Alan G. Hamilton",
  title = "Logic for Mathematicians",
  edition = "revised",
  publisher = "Cambridge University Press",
  address = "Cambridge",
  note = "[QA9.H298]",
  year = 1988 }
@unpublished{Harrison, author = "John Robert Harrison",
  title = "Metatheory and Reflection in Theorem Proving:
    A Survey and Critique",
  note = "Technical Report
    CRC-053.
    SRI Cambridge,
    Millers Yard, Cambridge, UK,
    1995.
    Available on the Web as
{\verb+http:+}\-{\verb+//www.cl.cam.ac.uk/users/jrh/papers/reflect.html+}"}
@TECHREPORT{Harrison-thesis,
        author          = "John Robert Harrison",
        title           = "Theorem Proving with the Real Numbers",
        institution   = "University of Cambridge Computer
                         Lab\-o\-ra\-to\-ry",
        address         = "New Museums Site, Pembroke Street, Cambridge,
                           CB2 3QG, UK",
        year            = 1996,
        number          = 408,
        type            = "Technical Report",
        note            = "Author's PhD thesis,
   available on the Web at
{\verb+http:+}\-{\verb+//www.cl.cam.ac.uk+}\-{\verb+/users+}\-{\verb+/jrh+}%
\-{\verb+/papers+}\-{\verb+/thesis.html+}"}
@book{Herrlich, author = "Horst Herrlich and George E. Strecker",
  title = "Category Theory:  An Introduction",
  publisher = "Allyn and Bacon Inc.",
  address = "Boston",
  note = "[QA169.H567]",
  year = 1973 }
@article{Hindley, author = "J. Roger Hindley and David Meredith",
  title = "Principal Type-Schemes and Condensed Detachment",
  journal = "The Journal of Symbolic Logic",
  volume = 55,
  year = 1990,
  note = "[QA.J87]",
  pages = "90--105" }
@book{Hofstadter, author = "Douglas R. Hofstadter",
  title = "G{\"{o}}del, Escher, Bach",
  publisher = "Basic Books, Inc.",
  address = "New York",
  note = "[QA9.H63 1980]",
  year = 1979 }
@article{Indrzejczak, author= "Andrzej Indrzejczak",
  title = "Natural Deduction, Hybrid Systems and Modal Logic",
  journal = "Trends in Logic",
  volume = 30,
  publisher = "Springer",
  year = 2010 }
@article{Kalish, author = "D. Kalish and R. Montague",
  title = "On {T}arski's Formalization of Predicate Logic with Identity",
  journal = "Archiv f{\"{u}}r Mathematische Logik und Grundlagenfor\-schung",
  volume = 7,
  year = 1965,
  note = "[QA.A673]",
  pages = "81--101" }
@article{Kalman, author = "J. A. Kalman",
  title = "Condensed Detachment as a Rule of Inference",
  journal = "Studia Logica",
  volume = 42,
  number = 4,
  year = 1983,
  note = "[B18.P6.S933]",
  pages = "443-451" }
@book{Kline, author = "Morris Kline",
  title = "Mathematical Thought from Ancient to Modern Times",
  publisher = "Oxford University Press",
  address = "New York",
  note = "[QA21.K516 1990 v.3]",
  year = 1972 }
@book{Klinel, author = "Morris Kline",
  title = "Mathematics, The Loss of Certainty",
  publisher = "Oxford University Press",
  address = "New York",
  note = "[QA21.K525]",
  year = 1980 }
@book{Kramer, author = "Edna E. Kramer",
  title = "The Nature and Growth of Modern Mathematics",
  publisher = "Princeton University Press",
  address = "Princeton, New Jersey",
  note = "[QA93.K89 1981]",
  year = 1981 }
@article{Knill, author = "Oliver Knill",
  title = "Some Fundamental Theorems in Mathematics",
  year = "2018",
  url = "https://arxiv.org/abs/1807.08416" }
@book{Landau, author = "Edmund Landau",
  title = "Foundations of Analysis",
  publisher = "Chelsea Publishing Company",
  address = "New York",
  edition = "second",
  note = "[QA241.L2541 1960]",
  year = 1960 }
@article{Leblanc, author = "Hugues Leblanc",
  title = "On {M}eyer and {L}ambert's Quantificational Calculus {FQ}",
  journal = "The Journal of Symbolic Logic",
  volume = 33,
  year = 1968,
  note = "[QA.J87]",
  pages = "275--280" }
@article{Lejewski, author = "Czeslaw Lejewski",
  title = "On Implicational Definitions",
  journal = "Studia Logica",
  volume = 8,
  year = 1958,
  note = "[B18.P6.S933]",
  pages = "189--208" }
@book{Levy, author = "Azriel Levy",
  title = "Basic Set Theory",
  publisher = "Dover Publications",
  address = "Mineola, NY",
  year = "2002"
}
@book{Margaris, author = "Angelo Margaris",
  title = "First Order Mathematical Logic",
  publisher = "Blaisdell Publishing Company",
  address = "Waltham, Massachusetts",
  note = "[QA9.M327]",
  year = 1967}
@book{Manin, author = "Yu I. Manin",
  title = "A Course in Mathematical Logic",
  publisher = "Springer-Verlag",
  address = "New York",
  note = "[QA9.M29613]",
  year = "1977" }
@article{Mathias, author = "Adrian R. D. Mathias",
  title = "A Term of Length 4,523,659,424,929",
  journal = "Synthese",
  volume = 133,
  year = 2002,
  note = "[Q.S993]",
  pages = "75--86" }
@article{Megill, author = "Norman D. Megill",
  title = "A Finitely Axiomatized Formalization of Predicate Calculus
     with Equality",
  journal = "Notre Dame Journal of Formal Logic",
  volume = 36,
  year = 1995,
  note = "[QA.N914]",
  pages = "435--453" }
@unpublished{Megillc, author = "Norman D. Megill",
  title = "A Shorter Equivalent of the Axiom of Choice",
  month = "June",
  note = "Unpublished",
  year = 1991 }
@article{MegillBunder, author = "Norman D. Megill and Martin W.
    Bunder",
  title = "Weaker {D}-Complete Logics",
  journal = "Journal of the IGPL",
  volume = 4,
  year = 1996,
  pages = "215--225",
  note = "Available on the Web at
{\verb+http:+}\-{\verb+//www.mpi-sb.mpg.de+}\-{\verb+/igpl+}%
\-{\verb+/Journal+}\-{\verb+/V4-2+}\-{\verb+/#Megill+}"}
}
@book{Mendelson, author = "Elliott Mendelson",
  title = "Introduction to Mathematical Logic",
  edition = "second",
  publisher = "D. Van Nostrand Company, Inc.",
  address = "New York",
  note = "[QA9.M537 1979]",
  year = 1979 }
@article{Meredith, author = "David Meredith",
  title = "In Memoriam {C}arew {A}rthur {M}eredith (1904-1976)",
  journal = "Notre Dame Journal of Formal Logic",
  volume = 18,
  year = 1977,
  note = "[QA.N914]",
  pages = "513--516" }
@article{CAMeredith, author = "C. A. Meredith",
  title = "Single Axioms for the Systems ({C},{N}), ({C},{O}) and ({A},{N})
      of the Two-Valued Propositional Calculus",
  journal = "The Journal of Computing Systems",
  volume = 3,
  year = 1953,
  pages = "155--164" }
@article{Monk, author = "J. Donald Monk",
  title = "Provability With Finitely Many Variables",
  journal = "The Journal of Symbolic Logic",
  volume = 27,
  year = 1971,
  note = "[QA.J87]",
  pages = "353--358" }
@article{Monks, author = "J. Donald Monk",
  title = "Substitutionless Predicate Logic With Identity",
  journal = "Archiv f{\"{u}}r Mathematische Logik und Grundlagenfor\-schung",
  volume = 7,
  year = 1965,
  pages = "103--121" }
  %% Took out this from above to prevent LaTeX underfull warning:
  % note = "[QA.A673]",
@book{Moore, author = "A. W. Moore",
  title = "The Infinite",
  publisher = "Routledge",
  address = "New York",
  note = "[BD411.M59]",
  year = 1989}
@book{Munkres, author = "James R. Munkres",
  title = "Topology: A First Course",
  publisher = "Prentice-Hall, Inc.",
  address = "Englewood Cliffs, New Jersey",
  note = "[QA611.M82]",
  year = 1975}
@article{Nemesszeghy, author = "E. Z. Nemesszeghy and E. A. Nemesszeghy",
  title = "On Strongly Creative Definitions:  A Reply to {V}. {F}. {R}ickey",
  journal = "Logique et Analyse (N.\ S.)",
  year = 1977,
  volume = 20,
  note = "[BC.L832]",
  pages = "111--115" }
@unpublished{Nemeti, author = "N{\'{e}}meti, I.",
  title = "Algebraizations of Quantifier Logics, an Overview",
  note = "Version 11.4, preprint, Mathematical Institute, Budapest,
    1994.  A shortened version without proofs appeared in
    ``Algebraizations of quantifier logics, an introductory overview,''
   {\em Studia Logica}, 50:485--569, 1991 [B18.P6.S933]"}
@article{Pavicic, author = "M. Pavi{\v{c}}i{\'{c}}",
  title = "A New Axiomatization of Unified Quantum Logic",
  journal = "International Journal of Theoretical Physics",
  year = 1992,
  volume = 31,
  note = "[QC.I626]",
  pages = "1753 --1766" }
@book{Penrose, author = "Roger Penrose",
  title = "The Emperor's New Mind",
  publisher = "Oxford University Press",
  address = "New York",
  note = "[Q335.P415]",
  year = 1989 }
@book{PetersonI, author = "Ivars Peterson",
  title = "The Mathematical Tourist",
  publisher = "W. H. Freeman and Company",
  address = "New York",
  note = "[QA93.P475]",
  year = 1988 }
@article{Peterson, author = "Jeremy George Peterson",
  title = "An automatic theorem prover for substitution and detachment systems",
  journal = "Notre Dame Journal of Formal Logic",
  volume = 19,
  year = 1978,
  note = "[QA.N914]",
  pages = "119--122" }
@book{Quine, author = "Willard Van Orman Quine",
  title = "Set Theory and Its Logic",
  edition = "revised",
  publisher = "The Belknap Press of Harvard University Press",
  address = "Cambridge, Massachusetts",
  note = "[QA248.Q7 1969]",
  year = 1969 }
@article{Robinson, author = "J. A. Robinson",
  title = "A Machine-Oriented Logic Based on the Resolution Principle",
  journal = "Journal of the Association for Computing Machinery",
  year = 1965,
  volume = 12,
  pages = "23--41" }
@article{RobinsonT, author = "T. Thacher Robinson",
  title = "Independence of Two Nice Sets of Axioms for the Propositional
    Calculus",
  journal = "The Journal of Symbolic Logic",
  volume = 33,
  year = 1968,
  note = "[QA.J87]",
  pages = "265--270" }
@book{Rucker, author = "Rudy Rucker",
  title = "Infinity and the Mind:  The Science and Philosophy of the
    Infinite",
  publisher = "Bantam Books, Inc.",
  address = "New York",
  note = "[QA9.R79 1982]",
  year = 1982 }
@book{Russell, author = "Bertrand Russell",
  title = "Mysticism and Logic, and Other Essays",
  publisher = "Barnes \& Noble Books",
  address = "Totowa, New Jersey",
  note = "[B1649.R963.M9 1981]",
  year = 1981 }
@article{Russell2, author = "Bertrand Russell",
  title = "Recent Work on the Principles of Mathematics",
  journal = "International Monthly",
  volume = 4,
  year = 1901,
  pages = "84"}
@article{Schmidt, author = "Eric Schmidt",
  title = "Reductions in Norman Megill's axiom system for complex numbers",
  url = "http://us.metamath.org/downloads/schmidt-cnaxioms.pdf",
  year = "2012" }
@book{Shoenfield, author = "Joseph R. Shoenfield",
  title = "Mathematical Logic",
  publisher = "Addison-Wesley Publishing Company, Inc.",
  address = "Reading, Massachusetts",
  year = 1967,
  note = "[QA9.S52]" }
@book{Smullyan, author = "Raymond M. Smullyan",
  title = "Theory of Formal Systems",
  publisher = "Princeton University Press",
  address = "Princeton, New Jersey",
  year = 1961,
  note = "[QA248.5.S55]" }
@book{Solow, author = "Daniel Solow",
  title = "How to Read and Do Proofs:  An Introduction to Mathematical
    Thought Process",
  publisher = "John Wiley \& Sons",
  address = "New York",
  year = 1982,
  note = "[QA9.S577]" }
@book{Stark, author = "Harold M. Stark",
  title = "An Introduction to Number Theory",
  publisher = "Markham Publishing Company",
  address = "Chicago",
  note = "[QA241.S72 1978]",
  year = 1970 }
@article{Swart, author = "E. R. Swart",
  title = "The Philosophical Implications of the Four-Color Problem",
  journal = "American Mathematical Monthly",
  year = 1980,
  volume = 87,
  month = "November",
  note = "[QA.A5125]",
  pages = "697--707" }
@book{Szpiro, author = "George G. Szpiro",
  title = "Poincar{\'{e}}'s Prize: The Hundred-Year Quest to Solve One
    of Math's Greatest Puzzles",
  publisher = "Penguin Books Ltd",
  address = "London",
  note = "[QA43.S985 2007]",
  year = 2007}
@book{Takeuti, author = "Gaisi Takeuti and Wilson M. Zaring",
  title = "Introduction to Axiomatic Set Theory",
  edition = "second",
  publisher = "Springer-Verlag New York Inc.",
  address = "New York",
  note = "[QA248.T136 1982]",
  year = 1982}
@inproceedings{Tarski, author = "Alfred Tarski",
  title = "What is Elementary Geometry",
  pages = "16--29",
  booktitle = "The Axiomatic Method, with Special Reference to Geometry and
     Physics (Proceedings of an International Symposium held at the University
     of California, Berkeley, December 26, 1957 --- January 4, 1958)",
  editor = "Leon Henkin and Patrick Suppes and Alfred Tarski",
  year = 1959,
  publisher = "North-Holland Publishing Company",
  address = "Amsterdam"}
@article{Tarski1965, author = "Alfred Tarski",
  title = "A Simplified Formalization of Predicate Logic with Identity",
  journal = "Archiv f{\"{u}}r Mathematische Logik und Grundlagenforschung",
  volume = 7,
  year = 1965,
  note = "[QA.A673]",
  pages = "61--79" }
@book{Tymoczko,
  title = "New Directions in the Philosophy of Mathematics",
  editor = "Thomas Tymoczko",
  publisher = "Birkh{\"{a}}user Boston, Inc.",
  address = "Boston",
  note = "[QA8.6.N48 1986]",
  year = 1986 }
@incollection{Wang,
  author = "Hao Wang",
  title = "Theory and Practice in Mathematics",
  pages = "129--152",
  booktitle = "New Directions in the Philosophy of Mathematics",
  editor = "Thomas Tymoczko",
  publisher = "Birkh{\"{a}}user Boston, Inc.",
  address = "Boston",
  note = "[QA8.6.N48 1986]",
  year = 1986 }
@manual{Webster,
  title = "Webster's New Collegiate Dictionary",
  organization = "G. \& C. Merriam Co.",
  address = "Springfield, Massachusetts",
  note = "[PE1628.W4M4 1977]",
  year = 1977 }
@manual{Whitehead, author = "Alfred North Whitehead",
  title = "An Introduction to Mathematics",
  year = 1911 }
@book{PM, author = "Alfred North Whitehead and Bertrand Russell",
  title = "Principia Mathematica",
  edition = "second",
  publisher = "Cambridge University Press",
  address = "Cambridge",
  year = "1927",
  note = "(3 vols.) [QA9.W592 1927]" }
@article{DBLP:journals/corr/Whalen16,
  author    = {Daniel Whalen},
  title     = {Holophrasm: a neural Automated Theorem Prover for higher-order logic},
  journal   = {CoRR},
  volume    = {abs/1608.02644},
  year      = {2016},
  url       = {http://arxiv.org/abs/1608.02644},
  archivePrefix = {arXiv},
  eprint    = {1608.02644},
  timestamp = {Mon, 13 Aug 2018 16:46:19 +0200},
  biburl    = {https://dblp.org/rec/bib/journals/corr/Whalen16},
  bibsource = {dblp computer science bibliography, https://dblp.org} }
@article{Wiedijk-revisited,
  author = {Freek Wiedijk},
  title = {The QED Manifesto Revisited},
  year = {2007},
  url = {http://mizar.org/trybulec65/8.pdf} }
@book{Wolfram,
  author = "Stephen Wolfram",
  title = "Mathematica:  A System for Doing Mathematics by Computer",
  edition = "second",
  publisher = "Addison-Wesley Publishing Co.",
  address = "Redwood City, California",
  note = "[QA76.95.W65 1991]",
  year = 1991 }
@book{Wos, author = "Larry Wos and Ross Overbeek and Ewing Lusk and Jim Boyle",
  title = "Automated Reasoning:  Introduction and Applications",
  edition = "second",
  publisher = "McGraw-Hill, Inc.",
  address = "New York",
  note = "[QA76.9.A96.A93 1992]",
  year = 1992 }

%
%
%[1] Church, Alonzo, Introduction to Mathematical Logic,
% Volume 1, Princeton University Press, Princeton, N. J., 1956.
%
%[2] Cohen, Paul J., Set Theory and the Continuum Hypothesis,
% W. A. Benjamin, Inc., Reading, Mass., 1966.
%
%[3] Hamilton, Alan G., Logic for Mathematicians, Cambridge
% University Press,
% Cambridge, 1988.

%[6] Kleene, Stephen Cole, Introduction to Metamathematics, D.  Van
% Nostrand Company, Inc., Princeton (1952).

%[13] Tarski, Alfred, "A simplified formalization of predicate
% logic with identity," Archiv fur Mathematische Logik und
% Grundlagenforschung, vol. 7 (1965), pp. 61-79.

%[14] Tarski, Alfred and Steven Givant, A Formalization of Set
% Theory Without Variables, American Mathematical Society Colloquium
% Publications, vol. 41, American Mathematical Society,
% Providence, R. I., 1987.

%[15] Zeman, J. J., Modal Logic, Oxford University Press, Oxford, 1973.
\end{filecontents}
% --------------------------- End of metamath.bib -----------------------------


%Book: Metamath
%Author:  Norman Megill Email:  nm at alum.mit.edu
%Author:  David A. Wheeler Email:  dwheeler at dwheeler.com

% A book template example
% http://www.stsci.edu/ftp/software/tex/bookstuff/book.template

\documentclass[leqno]{book} % LaTeX 2e. 10pt. Use [leqno,12pt] for 12pt
% hyperref 2002/05/27 v6.72r  (couldn't get pagebackref to work)
\usepackage[plainpages=false,pdfpagelabels=true]{hyperref}

\usepackage{needspace}     % Enable control over page breaks
\usepackage{breqn}         % automatic equation breaking
\usepackage{microtype}     % microtypography, reduces hyphenation

% Packages for flexible tables.  We need to be able to
% wrap text within a cell (with automatically-determined widths) AND
% split a table automatically across multiple pages.
% * "tabularx" wraps text in cells but only 1 page
% * "longtable" goes across pages but by itself is incompatible with tabularx
% * "ltxtable" combines longtable and tabularx, but table contents
%    must be in a separate file.
% * "ltablex" combines tabularx and longtable - must install specially
% * "booktabs" is recommended as a way to improve the look of tables,
%   but doesn't add these capabilities.
% * "tabu" much more capable and seems to be recommended. So use that.

\usepackage{makecell}      % Enable forced line splits within a table cell
\usepackage{longtable}     % Enable multi-page tables
\usepackage{tabu}          % Multi-page tables with wrapped text in a cell

% You can find more Tex packages using commands like:
% tlmgr search --file tabu.sty
% find /usr/share/texmf-dist/ -name '*tab*'
%
%%%%%%%%%%%%%%%%%%%%%%%%%%%%%%%%%%%%%%%%%%%%%%%%%%%%%%%%%%%%%%%%%%%%%%%%%%%%
% Uncomment the next 3 lines to suppress boxes and colors on the hyperlinks
%%%%%%%%%%%%%%%%%%%%%%%%%%%%%%%%%%%%%%%%%%%%%%%%%%%%%%%%%%%%%%%%%%%%%%%%%%%%
%\hypersetup{
%colorlinks,citecolor=black,filecolor=black,linkcolor=black,urlcolor=black
%}
%
\usepackage{realref}

% Restarting page numbers: try?
%   \printglossary
%   \cleardoublepage
%   \pagenumbering{arabic}
%   \setcounter{page}{1}    ???needed
%   \include{chap1}

% not used:
% \def\R2Lurl#1#2{\mbox{\href{#1}\texttt{#2}}}

\usepackage{amssymb}

% Version 1 of book: margins: t=.4, b=.2, ll=.4, rr=.55
% \usepackage{anysize}
% % \papersize{<height>}{<width>}
% % \marginsize{<left>}{<right>}{<top>}{<bottom>}
% \papersize{9in}{6in}
% % l/r 0.6124-0.6170 works t/b 0.2418-0.3411 = 192pp. 0.2926-03118=exact
% \marginsize{0.7147in}{0.5147in}{0.4012in}{0.2012in}

\usepackage{anysize}
% \papersize{<height>}{<width>}
% \marginsize{<left>}{<right>}{<top>}{<bottom>}
\papersize{9in}{6in}
% l/r 0.85in&0.6431-0.6539 works t/b ?-?
%\marginsize{0.85in}{0.6485in}{0.55in}{0.35in}
\marginsize{0.8in}{0.65in}{0.5in}{0.3in}

% \usepackage[papersize={3.6in,4.8in},hmargin=0.1in,vmargin={0.1in,0.1in}]{geometry}  % page geometry
\usepackage{special-settings}

\raggedbottom
\makeindex

\begin{document}
% Discourage page widows and orphans:
\clubpenalty=300
\widowpenalty=300

%%%%%%% load in AMS fonts %%%%%%% % LaTeX 2.09 - obsolete in LaTeX 2e
%\input{amssym.def}
%\input{amssym.tex}
%\input{c:/texmf/tex/plain/amsfonts/amssym.def}
%\input{c:/texmf/tex/plain/amsfonts/amssym.tex}

\bibliographystyle{plain}
\pagenumbering{roman}
\pagestyle{headings}

\thispagestyle{empty}

\hfill
\vfill

\begin{center}
{\LARGE\bf Metamath} \\
\vspace{1ex}
{\large A Computer Language for Mathematical Proofs} \\
\vspace{7ex}
{\large Norman Megill} \\
\vspace{7ex}
with extensive revisions by \\
\vspace{1ex}
{\large David A. Wheeler} \\
\vspace{7ex}
% Printed date. If changing the date below, also fix the date at the beginning.
2019-05-24
\end{center}

\vfill
\hfill

\newpage
\thispagestyle{empty}

\hfill
\vfill

\begin{center}
$\sim$\ {\sc Public Domain}\ $\sim$

\vspace{2ex}
This book (including its later revisions)
has been released into the Public Domain by Norman Megill per the
Creative Commons CC0 1.0 Universal (CC0 1.0) Public Domain Dedication
(\url{https://creativecommons.org/publicdomain/zero/1.0/}).
David A. Wheeler has done the same.
The public domain release applies worldwide.  In case this is not
legally possible, the right is granted to use the work for any purpose,
without any conditions, unless such conditions are required by law.

\vspace{3ex}
Several short, attributed quotations from copyrighted works
appear in this book under the ``fair use'' provision of Section 107 of
the United States Copyright Act (Title 17 of the {\em United States
Code}).  The public-domain status of this book is not applicable to
those quotations.

\vspace{3ex}
Any trademarks used in this book are the property of their owners.

% QA76.9.L63.M??

% \vspace{1ex}
%
% \vspace{1ex}
% {\small Permission is granted to make and distribute verbatim copies of this
% book
% provided the copyright notice and this
% permission notice are preserved on all copies.}
%
% \vspace{1ex}
% {\small Permission is granted to copy and distribute modified versions of this
% book under the conditions for verbatim copying, provided that the
% entire
% resulting derived work is distributed under the terms of a permission
% notice
% identical to this one.}
%
% \vspace{1ex}
% {\small Permission is granted to copy and distribute translations of this
% book into another language, under the above conditions for modified
% versions,
% except that this permission notice may be stated in a translation
% approved by the
% author.}
%
% \vspace{1ex}
% %{\small   For a copy of the \LaTeX\ source files for this book, contact
% %the author.} \\
% \ \\
% \ \\

\vspace{7ex}
% ISBN: 1-4116-3724-0 \\
ISBN: 978-1-4116-3724-5 \\
{\ } \\
Lulu Press \\
Morrisville, North Carolina\\
USA


\hfill
\vfill

Norman Megill\\ 93 Bridge St., Lexington, MA 02421 \\
E-mail address: \texttt{nm{\char`\@}alum.mit.edu} \\
\vspace{7ex}
David A. Wheeler \\
E-mail address: \texttt{dwheeler{\char`\@}dwheeler.com} \\
% See notes added at end of Preface for revision history. \\
% For current information on the Metamath software see \\
\vspace{7ex}
\url{http://metamath.org}
\end{center}

% \newpage
% \thispagestyle{empty}
%
% \hfill
% \vfill
%
% \begin{center}
% {\it To my son Robin Dwight Megill}
% \end{center}
%
% \vfill
% \hfill
%
% \newpage

\tableofcontents
%\listoftables

\chapter*{Preface}
\markboth{PREFACE}{PREFACE}
\addcontentsline{toc}{section}{Preface}


% (For current information, see the notes added at the
% end of this preface on p.~\pageref{note2002}.)

\subsubsection{Overview}

Metamath\index{Metamath} is a computer language and an associated computer
program for archiving, verifying, and studying mathematical proofs at a very
detailed level.  The Metamath language incorporates no mathematics per se but
treats all mathematical statements as mere sequences of symbols.  You provide
Metamath with certain special sequences (axioms) that tell it what rules
of inference are allowed.  Metamath is not limited to any specific field of
mathematics.  The Metamath language is simple and robust, with an
almost total absence of hard-wired syntax, and
I\footnote{Unless otherwise noted, the words
``I,'' ``me,'' and ``my'' refer to Norman Megill\index{Megill, Norman}, while
``we,'' ``us,'' and ``our'' refer to Norman Megill and
David A. Wheeler\index{Wheeler, David A.}.}
believe that it
provides about the simplest possible framework that allows essentially all of
mathematics to be expressed with absolute rigor.

% index test
%\newcommand{\nn}[1]{#1n}
%\index{aaa@bbb}
%\index{abc!def}
%\index{abd|see{qqq}}
%\index{abe|nn}
%\index{abf|emph}
%\index{abg|(}
%\index{abg|)}

Using the Metamath language, you can build formal or mathematical
systems\index{formal system}\footnote{A formal or mathematical system consists
of a collection of symbols (such as $2$, $4$, $+$ and $=$), syntax rules that
describe how symbols may be combined to form a legal expression (called a
well-formed formula or {\em wff}, pronounced ``whiff''), some starting wffs
called axioms, and inference rules that describe how theorems may be derived
(proved) from the axioms.  A theorem is a mathematical fact such as $2+2=4$.
Strictly speaking, even an obvious fact such as this must be proved from
axioms to be formally acceptable to a mathematician.}\index{theorem}
\index{axiom}\index{rule}\index{well-formed formula (wff)} that involve
inferences from axioms.  Although a database is provided
that includes a recommended set of axioms for standard mathematics, if you
wish you can supply your own symbols, syntax, axioms, rules, and definitions.

The name ``Metamath'' was chosen to suggest that the language provides a
means for {\em describing} mathematics rather than {\em being} the
mathematics itself.  Actually in some sense any mathematical language is
metamathematical.  Symbols written on paper, or stored in a computer,
are not mathematics itself but rather a way of expressing mathematics.
For example ``7'' and ``VII'' are symbols for denoting the number seven
in Arabic and Roman numerals; neither {\em is} the number seven.

If you are able to understand and write computer programs, you should be able
to follow abstract mathematics with the aid of Metamath.  Used in conjunction
with standard textbooks, Metamath can guide you step-by-step towards an
understanding of abstract mathematics from a very rigorous viewpoint, even if
you have no formal abstract mathematics background.  By using a single,
consistent notation to express proofs, once you grasp its basic concepts
Metamath provides you with the ability to immediately follow and dissect
proofs even in totally unfamiliar areas.

Of course, just being able follow a proof will not necessarily give you an
intuitive familiarity with mathematics.  Memorizing the rules of chess does not
give you the ability to appreciate the game of a master, and knowing how the
notes on a musical score map to piano keys does not give you the ability to
hear in your head how it would sound.  But each of these can be a first step.

Metamath allows you to explore proofs in the sense that you can see the
theorem referenced at any step expanded in as much detail as you want, right
down to the underlying axioms of logic and set theory (in the case of the set
theory database provided).  While Metamath will not replace the higher-level
understanding that can only be acquired through exercises and hard work, being
able to see how gaps in a proof are filled in can give you increased
confidence that can speed up the learning process and save you time when you
get stuck.

The Metamath language breaks down a mathematical proof into its tiniest
possible parts.  These can be pieced together, like interlocking
pieces in a puzzle, only in a way that produces correct and absolutely rigorous
mathematics.

The nature of Metamath\index{Metamath} enforces very precise mathematical
thinking, similar to that involved in writing a computer program.  A crucial
difference, though, is that once a proof is verified (by the Metamath program)
to be correct, it is definitely correct; it can never have a hidden
``bug.''\index{computer program bugs}  After getting used to the kind of rigor
and accuracy provided by Metamath, you might even be tempted to
adopt the attitude that a proof should never be considered correct until it
has been verified by a computer, just as you would not completely trust a
manual calculation until you have verified it on a
calculator.

My goal
for Metamath was a system for describing and verifying
mathematics that is completely universal yet conceptually as simple as
possible.  In approaching mathematics from an axiomatic, formal viewpoint, I
wanted Metamath to be able to handle almost any mathematical system, not
necessarily with ease, but at least in principle and hopefully in practice. I
wanted it to verify proofs with absolute rigor, and for this reason Metamath
is what might be thought of as a ``compile-only'' language rather than an
algorithmic or Turing-machine language (Pascal, C, Prolog, Mathematica,
etc.).  In other words, a database written in the Metamath
language doesn't ``do'' anything; it merely exhibits mathematical knowledge
and permits this knowledge to be verified as being correct.  A program in an
algorithmic language can potentially have hidden bugs\index{computer program
bugs} as well as possibly being hard to understand.  But each token in a
Metamath database must be consistent with the database's earlier
contents according to simple, fixed rules.
If a database is verified
to be correct,\footnote{This includes
verification that a sequential list of proof steps results in the specified
theorem.} then the mathematical content is correct if the
verifier is correct and the axioms are correct.
The verification program could be incorrect, but the verification algorithm
is relatively simple (making it unlikely to be implemented incorrectly
by the Metamath program),
and there are over a dozen Metamath database verifiers
written by different people in different programming languages
(so these different verifiers can act as multiple reviewers of a database).
The most-used Metamath database, the Metamath Proof Explorer
(aka \texttt{set.mm}\index{set theory database (\texttt{set.mm})}%
\index{Metamath Proof Explorer}),
is currently verified by four different Metamath verifiers written by
four different people in four different languages, including the
original Metamath program described in this book.
The only ``bugs'' that can exist are in the statement of the axioms,
for example if the axioms are inconsistent (a famous problem shown to be
unsolvable by G\"{o}del's incompleteness theorem\index{G\"{o}del's
incompleteness theorem}).
However, real mathematical systems have very few axioms, and these can
be carefully studied.
All of this provides extraordinarily high confidence that the verified database
is in fact correct.

The Metamath program
doesn't prove theorems automatically but is designed to verify proofs
that you supply to it.
The underlying Metamath language is completely general and has no built-in,
preconceived notions about your formal system\index{formal system}, its logic
or its syntax.
For constructing proofs, the Metamath program has a Proof Assistant\index{Proof
Assistant} which helps you fill in some of a proof step's details, shows you
what choices you have at any step, and verifies the proof as you build it; but
you are still expected to provide the proof.

There are many other programs that can process or generate information
in the Metamath language, and more continue to be written.
This is in part because the Metamath language itself is very simple
and intentionally easy to automatically process.
Some programs, such as \texttt{mmj2}\index{mmj2}, include a proof assistant
that can automate some steps beyond what the Metamath program can do.
Mario Carneiro has developed an algorithm for converting proofs from
the OpenTheory interchange format, which can be translated to and from
any of the HOL family of proof languages (HOL4, HOL Light, ProofPower,
and Isabelle), into the
Metamath language.\cite{DBLP:journals/corr/Carneiro14}\index{Carneiro, Mario}.
Daniel Whalen has developed Holophrasm, which can automatically
prove many Metamath proofs using
machine learning\index{machine learning}\index{artificial intelligence}
approaches
(including multiple neural networks\index{neural networks})\cite{DBLP:journals/corr/Whalen16}\index{Whalen, Daniel}.
However,
a discussion of these other programs is beyond the scope of this book.

Like most computer languages, the Metamath\index{Metamath} language uses the
standard ({\sc ascii}) characters on a computer keyboard, so it cannot
directly represent many of the special symbols that mathematicians use.  A
useful feature of the Metamath program is its ability to convert its notation
into the \LaTeX\ typesetting language.\index{latex@{\LaTeX}}  This feature
lets you convert the {\sc ascii} tokens you've defined into standard
mathematical symbols, so you end up with symbols and formulas you are familiar
with instead of somewhat cryptic {\sc ascii} representations of them.
The Metamath program can also generate HTML\index{HTML}, making it easy
to view results on the web and to see related information by using
hypertext links.

Metamath is probably conceptually different from anything you've seen
before and some aspects may take some getting used to.  This book will
help you decide whether Metamath suits your specific needs.

\subsubsection{Setting Your Expectations}
It is important for you to understand what Metamath\index{Metamath} is and is
not.  As mentioned, the Metamath program
is {\em not} an automated theorem prover but
rather a proof verifier.  Developing a database can be tedious, hard work,
especially if you want to make the proofs as short as possible, but it becomes
easier as you build up a collection of useful theorems.  The purpose of
Metamath is simply to document existing mathematics in an absolutely rigorous,
computer-verifiable way, not to aid directly in the creation of new
mathematics.  It also is not a magic solution for learning abstract
mathematics, although it may be helpful to be able to actually see the implied
rigor behind what you are learning from textbooks, as well as providing hints
to work out proofs that you are stumped on.

As of this writing, a sizable set theory database has been developed to
provide a foundation for many fields of mathematics, but much more work would
be required to develop useful databases for specific fields.

Metamath\index{Metamath} ``knows no math;'' it just provides a framework in
which to express mathematics.  Its language is very small.  You can define two
kinds of symbols, constants\index{constant} and variables\index{variable}.
The only thing Metamath knows how to do is to substitute strings of symbols
for the variables\index{substitution!variable}\index{variable substitution} in
an expression based on instructions you provide it in a proof, subject to
certain constraints you specify for the variables.  Even the decimal
representation of a number is merely a string of certain constants (digits)
which together, in a specific context, correspond to whatever mathematical
object you choose to define for it; unlike other computer languages, there is
no actual number stored inside the computer.  In a proof, you in effect
instruct Metamath what symbol substitutions to make in previous axioms or
theorems and join a sequence of them together to result in the desired
theorem.  This kind of symbol manipulation captures the essence of mathematics
at a preaxiomatic level.

\subsubsection{Metamath and Mathematical Literature}

In advanced mathematical literature, proofs are usually presented in the form
of short outlines that often only an expert can follow.  This is partly out of
a desire for brevity, but it would also be unwise (even if it were practical)
to present proofs in complete formal detail, since the overall picture would
be lost.\index{formal proof}

A solution I envision\label{envision} that would allow mathematics to remain
acceptable to the expert, yet increase its accessibility to non-specialists,
consists of a combination of the traditional short, informal proof in print
accompanied by a complete formal proof stored in a computer database.  In an
analogy with a computer program, the informal proof is like a set of comments
that describe the overall reasoning and content of the proof, whereas the
computer database is like the actual program and provides a means for anyone,
even a non-expert, to follow the proof in as much detail as desired, exploring
it back through layers of theorems (like subroutines that call other
subroutines) all the way back to the axioms of the theory.  In addition, the
computer database would have the advantage of providing absolute assurance
that the proof is correct, since each step can be verified automatically.

There are several other approaches besides Metamath to a project such
as this.  Section~\ref{proofverifiers} discusses some of these.

To us, a noble goal would be a database with hundreds of thousands of
theorems and their computer-verifiable proofs, encompassing a significant
fraction of known mathematics and available for instant access.
These would be fully verified by multiple independently-implemented verifiers,
to provide extremely high confidence that the proofs are completely correct.
The database would allow people to investigate whatever details they were
interested in, so that they could confirm whatever portions they wished.
Whether or not Metamath is an appropriate choice remains to be seen, but in
principle we believe it is sufficient.

\subsubsection{Formalism}

Over the past fifty years, a group of French mathematicians working
collectively under the pseudonym of Bourbaki\index{Bourbaki, Nicolas} have
co-authored a series of monographs that attempt to rigorously and
consistently formalize large bodies of mathematics from foundations.  On the
one hand, certainly such an effort has its merits; on the other hand, the
Bourbaki project has been criticized for its ``scholasticism'' and
``hyperaxiomatics'' that hide the intuitive steps that lead to the results
\cite[p.~191]{Barrow}\index{Barrow, John D.}.

Metamath unabashedly carries this philosophy to its extreme and no doubt is
subject to the same kind of criticism.  Nonetheless I think that in
conjunction with conventional approaches to mathematics Metamath can serve a
useful purpose.  The Bourbaki approach is essentially pedagogic, requiring the
reader to become intimately familiar with each detail in a very large
hierarchy before he or she can proceed to the next step.  The difference with
Metamath is that the ``reader'' (user) knows that all details are contained in
its computer database, available as needed; it does not demand that the user
know everything but conveniently makes available those portions that are of
interest.  As the body of all mathematical knowledge grows larger and larger,
no one individual can have a thorough grasp of its entirety.  Metamath
can finalize and put to rest any questions about the validity of any part of it
and can make any part of it accessible, in principle, to a non-specialist.

\subsubsection{A Personal Note}
Why did I develop Metamath\index{Metamath}?  I enjoy abstract mathematics, but
I sometimes get lost in a barrage of definitions and start to lose confidence
that my proofs are correct.  Or I reach a point where I lose sight of how
anything I'm doing relates to the axioms that a theory is based on and am
sometimes suspicious that there may be some overlooked implicit axiom
accidentally introduced along the way (as happened historically with Euclidean
geometry\index{Euclidean geometry}, whose omission of Pasch's
axiom\index{Pasch's axiom} went unnoticed for 2000 years
\cite[p.~160]{Davis}!). I'm also somewhat lazy and wish to avoid the effort
involved in re-verifying the gaps in informal proofs ``left to the reader;'' I
prefer to figure them out just once and not have to go through the same
frustration a year from now when I've forgotten what I did.  Metamath provides
better recovery of my efforts than scraps of paper that I can't
decipher anymore.  But mostly I find very appealing the idea of rigorously
archiving mathematical knowledge in a computer database, providing precision,
certainty, and elimination of human error.

\subsubsection{Note on Bibliography and Index}

The Bibliography usually includes the Library of Congress classification
for a work to make it easier for you to find it in on a university
library shelf.  The Index has author references to pages where their works
are cited, even though the authors' names may not appear on those pages.

\subsubsection{Acknowledgments}

Acknowledgments are first due to my wife, Deborah (who passed away on
September 4, 1998), for critiquing the manu\-script but most of all for
her patience and support.  I also wish to thank Joe Wright, Richard
Becker, Clarke Evans, Buddha Buck, and Jeremy Henty for helpful
comments.  Any errors, omissions, and other shortcomings are of course
my responsibility.

\subsubsection{Note Added June 22, 2005}\label{note2002}

The original, unpublished version of this book was written in 1997 and
distributed via the web.  The present edition has been updated to
reflect the current Metamath program and databases, as well as more
current {\sc url}s for Internet sites.  Thanks to Josh
Purinton\index{Purinton, Josh}, One Hand
Clapping, Mel L.\ O'Cat, and Roy F. Longton for pointing out
typographical and other errors.  I have also benefitted from numerous
discussions with Raph Levien\index{Levien, Raph}, who has extended
Metamath's philosophy of rigor to result in his {\em
Ghilbert}\index{Ghilbert} proof language (\url{http://ghilbert.org}).

Robert (Bob) Solovay\index{Solovay, Robert} communicated a new result of
A.~R.~D.~Mathias on the system of Bourbaki, and the text has been
updated accordingly (p.~\pageref{bourbaki}).

Bob also pointed out a clarification of the literature regarding
category theory and inaccessible cardinals\index{category
theory}\index{cardinal, inaccessible} (p.~\pageref{categoryth}),
and a misleading statement was removed from the text.  Specifically,
contrary to a statement in previous editions, it is possible to express
``There is a proper class of inaccessible cardinals'' in the language of
ZFC.  This can be done as follows:  ``For every set $x$ there is an
inaccessible cardinal $\kappa$ such that $\kappa$ is not in $x$.''
Bob writes:\footnote{Private communication, Nov.~30, 2002.}
\begin{quotation}
     This axiom is how Grothendieck presents category theory.  To each
inaccessible cardinal $\kappa$ one associates a Grothendieck universe
\index{Grothendieck, Alexander} $U(\kappa)$.  $U(\kappa)$ consists of
those sets which lie in a transitive set of cardinality less than
$\kappa$.  Instead of the ``category of all groups,'' one works relative
to a universe [considering the category of groups of cardinality less
than $\kappa$].  Now the category whose objects are all categories
``relative to the universe $U(\kappa)$'' will be a category not
relative to this universe but to the next universe.

     All of the things category theorists like to do can be done in this
framework.  The only controversial point is whether the Grothen\-dieck
axiom is too strong for the needs of category theorists.  Mac Lane
\index{Mac Lane, Saunders} argues that ``one universe is enough'' and
Feferman\index{Feferman, Solomon} has argued that one can get by with
ordinary ZFC.  I don't find Feferman's arguments persuasive.  Mac Lane
may be right, but when I think about category theory I do it \`{a} la
Grothendieck.

        By the way Mizar\index{Mizar} adds the axiom ``there is a proper
class of inaccessibles'' precisely so as to do category theory.
\end{quotation}

The most current information on the Metamath program and databases can
always be found at \url{http://metamath.org}.


\subsubsection{Note Added June 24, 2006}\label{note2006}

The Metamath spec was restricted slightly to make parsers easier to
write.  See the footnote on p.~\pageref{namespace}.

%\subsubsection{Note Added July 24, 2006}\label{note2006b}
\subsubsection{Note Added March 10, 2007}\label{note2006b}

I am grateful to Anthony Williams\index{Williams, Anthony} for writing
the \LaTeX\ package called {\tt realref.sty} and contributing it to the
public domain.  This package allows the internal hyperlinks in a {\sc
pdf} file to anchor to specific page numbers instead of just section
titles, making the navigation of the {\sc pdf} file for this book much
more pleasant and ``logical.''

A typographical error found by Martin Kiselkov was corrected.
A confusing remark about unification was deleted per suggestion of
Mel O'Cat.

\subsubsection{Note Added May 27, 2009}\label{note2009}

Several typos found by Kim Sparre were corrected.  A note was added that
the Poincar\'{e} conjecture has been proved (p.~\pageref{poincare}).

\subsubsection{Note Added Nov. 17, 2014}\label{note2014}

The statement of the Schr\"{o}der--Bernstein theorem was corrected in
Section~\ref{trust}.  Thanks to Bob Solovay for pointing out the error.

\subsubsection{Note Added May 25, 2016}\label{note2016}

Thanks to Jerry James for correcting 16 typos.

\subsubsection{Note Added February 25, 2019}\label{note201902}

David A. Wheeler\index{Wheeler, David A.}
made a large number of improvements and updates,
in coordination with Norman Megill.
The predicate calculus axioms were renumbered, and the text makes
it clear that they are based on Tarski's system S2;
the one slight deviation in axiom ax-6 is explained and justified.
The real and complex number axioms were modified to be consistent with
\texttt{set.mm}\index{set theory database (\texttt{set.mm})}%
\index{Metamath Proof Explorer}.
Long-awaited specification changes ``1--8'' were made,
which clarified previously ambiguous points.
Some errors in the text involving \texttt{\$f} and
\texttt{\$d} statements were corrected (the spec was correct, but
the in-book explanations unintentionally contradicted the spec).
We now have a system for automatically generating narrow PDFs,
so that those with smartphones can have easy access to the current
version of this document.
A new section on deduction was added;
it discusses the standard deduction theorem,
the weak deduction theorem,
deduction style, and natural deduction.
Many minor corrections (too numerous to list here) were also made.

\subsubsection{Note Added March 7, 2019}\label{note201903}

This added a description of the Matamath language syntax in
Extended Backus--Naur Form (EBNF)\index{Extended Backus--Naur Form}\index{EBNF}
in Appendix \ref{BNF}, added a brief explanation about typecodes,
inserted more examples in the deduction section,
and added a variety of smaller improvements.

\subsubsection{Note Added April 7, 2019}\label{note201904}

This version clarified the proper substitution notation, improved the
discussion on the weak deduction theorem and natural deduction,
documented the \texttt{undo} command, updated the information on
\texttt{write source}, changed the typecode
from \texttt{set} to \texttt{setvar} to be consistent with the current
version of \texttt{set.mm}, added more documentation about comment markup
(e.g., documented how to create headings), and clarified the
differences between various assertion forms (in particular deduction form).

\subsubsection{Note Added May 20, 2019}\label{note201905}

This version fixes various small issues and added specific examples
of forms (deduction form, inference form, and closed form).

\chapter{Introduction}
\pagenumbering{arabic}

\begin{quotation}
  {\em {\em I.M.:}  No, no.  There's nothing subjective about it!  Everybody
knows what a proof is.  Just read some books, take courses from a competent
mathematician, and you'll catch on.

{\em Student:}  Are you sure?

{\em I.M.:}  Well---it is possible that you won't, if you don't have any
aptitude for it.  That can happen, too.

{\em Student:}  Then {\em you} decide what a proof is, and if I don't learn
to decide in the same way, you decide I don't have any aptitude.

{\em I.M.:}  If not me, then who?}
    \flushright\sc  ``The Ideal Mathematician''
    \index{Davis, Phillip J.}
    \footnote{\cite{Davis}, p.~40.}\\
\end{quotation}

In the past century, brilliant mathematicians have discovered almost
unimaginably profound results that rank among the crowning intellectual
achievements of mankind.  However, there is a sense in which modern abstract
mathematics is behind the times, stuck in an era before computers existed.
While no one disputes the remarkable results that have been achieved,
communicating these results in a precise way to the uninitiated is virtually
impossible.  To describe these results, a terse informal language is used which
despite its elegance is very difficult to learn.  This informal language is not
imprecise, far from it, but rather it often has omitted detail
and symbols with hidden context that are
implicitly understood by an expert but few others.  Extremely complex technical
meanings are associated with innocent-sounding English words such as
``compact'' and ``measurable'' that barely hint at what is actually being
said.  Anyone who does not keep the precise technical meaning constantly in
mind is bound to fail, and acquiring the ability to do this can be achieved
only through much practice and hard work.  Only the few who complete the
painful learning experience can join the small in-group of pure
mathematicians.  The informal language effectively cuts off the true nature of
their knowledge from most everyone else.

Metamath\index{Metamath} makes abstract mathematics more concrete.  It allows
a computer to keep track of the complexity associated with each word or symbol
with absolute rigor.  You can explore this complexity at your leisure, to
whatever degree you desire.  Whether or not you believe that concepts such as
infinity actually ``exist'' outside of the mind, Metamath lets you get to the
foundation for what's really being said.

Metamath also enables completely rigorous and thorough proof verification.
Its language is simple enough so that you
don't have to rely on the authority of experts but can verify the results
yourself, step by step.  If you want to attempt to derive your own results,
Metamath will not let you make a mistake in reasoning.
Even professional mathematicians make mistakes; Metamath makes it possible
to thoroughly verify that proofs are correct.

Metamath\index{Metamath} is a computer language and an associated computer
program for archiving, verifying, and studying mathematical proofs at a very
detailed level.
The Metamath language
describes formal\index{formal system} mathematical
systems and expresses proofs of theorems in those systems.  Such a language
is called a metalanguage\index{metalanguage} by mathematicians.
The Metamath program is a computer program that verifies
proofs expressed in the Metamath language.
The Metamath program does not have the built-in
ability to make logical inferences; it just makes a series of symbol
substitutions according to instructions given to it in a proof
and verifies that the result matches the expected theorem.  It makes logical
inferences based only on rules of logic that are contained in a set of
axioms\index{axiom}, or first principles, that you provide to it as the
starting point for proofs.

The complete specification of the Metamath language is only four pages long
(Section~\ref{spec}, p.~\pageref{spec}).  Its simplicity may at first make you
wonder how it can do much of anything at all.  But in fact the kinds of
symbol manipulations it performs are the ones that are implicitly done in all
mathematical systems at the lowest level.  You can learn it relatively quickly
and have complete confidence in any mathematical proof that it verifies.  On
the other hand, it is powerful and general enough so that virtually any
mathematical theory, from the most basic to the deeply abstract, can be
described with it.

Although in principle Metamath can be used with any
kind of mathematics, it is best suited for abstract or ``pure'' mathematics
that is mostly concerned with theorems and their proofs, as opposed to the
kind of mathematics that deals with the practical manipulation of numbers.
Examples of branches of pure mathematics are logic\index{logic},\footnote{Logic
is the study of statements that are universally true regardless of the objects
being described by the statements.  An example is the statement, ``if $P$
implies $Q$, then either $P$ is false or $Q$ is true.''} set theory\index{set
theory},\footnote{Set theory is the study of general-purpose mathematical objects called
``sets,'' and from it essentially all of mathematics can be derived.  For
example, numbers can be defined as specific sets, and their properties
can be explored using the tools of set theory.} number theory\index{number
theory},\footnote{Number theory deals with the properties of positive and
negative integers (whole numbers).} group theory\index{group
theory},\footnote{Group theory studies the properties of mathematical objects
called groups that obey a simple set of axioms and have properties of symmetry
that make them useful in many other fields.} abstract algebra\index{abstract
algebra},\footnote{Abstract algebra includes group theory and also studies
groups with additional properties that qualify them as ``rings'' and
``fields.''  The set of real numbers is a familiar example of a field.},
analysis\index{analysis} \index{real and complex numbers}\footnote{Analysis is
the study of real and complex numbers.} and
topology\index{topology}.\footnote{One area studied by topology are properties
that remain unchanged when geometrical objects undergo stretching
deformations; for example a doughnut and a coffee cup each have one hole (the
cup's hole is in its handle) and are thus considered topologically
equivalent.  In general, though, topology is the study of abstract
mathematical objects that obey a certain (surprisingly simple) set of axioms.
See, for example, Munkres \cite{Munkres}\index{Munkres, James R.}.} Even in
physics, Metamath could be applied to certain branches that make use of
abstract mathematics, such as quantum logic\index{quantum logic} (used to study
aspects of quantum mechanics\index{quantum mechanics}).

On the other hand, Metamath\index{Metamath} is less suited to applications
that deal primarily with intensive numeric computations.  Metamath does not
have any built-in representation of numbers\index{Metamath!representation of
numbers}; instead, a specific string of symbols (digits) must be syntactically
constructed as part of any proof in which an ordinary number is used.  For
this reason, numbers in Metamath are best limited to specific constants that
arise during the course of a theorem or its proof.  Numbers are only a tiny
part of the world of abstract mathematics.  The exclusion of built-in numbers
was a conscious decision to help achieve Metamath's simplicity, and there are
other software tools if you have different mathematical needs.
If you wish to quickly solve algebraic problems, the computer algebra
programs\index{computer algebra system} {\sc
macsyma}\index{macsyma@{\sc macsyma}}, Mathematica\index{Mathematica}, and
Maple\index{Maple} are specifically suited to handling numbers and
algebra efficiently.
If you wish to simply calculate numeric or matrix expressions easily,
tools such as Octave\index{Octave} may be a better choice.

After learning Metamath's basic statement types, any
tech\-ni\-cal\-ly ori\-ent\-ed person, mathematician or not, can
immediately trace
any theorem proved in the language as far back as he or she wants, all the way
to the axioms on which the theorem is based.  This ability suggests a
non-traditional way of learning about pure mathematics.  Used in conjunction
with traditional methods, Metamath could make pure mathematics accessible to
people who are not sufficiently skilled to figure out the implicit detail in
ordinary textbook proofs.  Once you learn the axioms of a theory, you can have
complete confidence that everything you need to understand a proof you are
studying is all there, at your beck and call, allowing you to focus in on any
proof step you don't understand in as much depth as you need, without worrying
about getting stuck on a step you can't figure out.\footnote{On the other
hand, writing proofs in the Metamath language is challenging, requiring
a degree of rigor far in excess of that normally taught to students.  In a
classroom setting, I doubt that writing Metamath proofs would ever replace
traditional homework exercises involving informal proofs, because the time
needed to work out the details would not allow a course to
cover much material.  For students who have trouble grasping the implied rigor
in traditional material, writing a few simple proofs in the Metamath language
might help clarify fuzzy thought processes.  Although somewhat difficult at
first, it eventually becomes fun to do, like solving a puzzle, because of the
instant feedback provided by the computer.}

Metamath\index{Metamath} is probably unlike anything you have
encountered before.  In this first chapter we will look at the philosophy and
use of computers in mathematics in order to better understand the motivation
behind Metamath.  The material in this chapter is not required in order to use
Metamath.  You may skip it if you are impatient, but I hope you will find it
educational and enjoyable.  If you want to start experimenting with the
Metamath program right away, proceed directly to Chapter~\ref{using}
(p.~\pageref{using}).  To
learn the Metamath language, skim Chapter~\ref{using} then proceed to
Chapter~\ref{languagespec} (p.~\pageref{languagespec}).

\section{Mathematics as a Computer Language}

\begin{quote}
  {\em The study of mathematics is apt to commence in
dis\-ap\-point\-ment.\ldots \\
We are told that by its aid the stars are weighted
and the billions of molecules in a drop of water are counted.  Yet, like the
ghost of Hamlet's father, this great science eludes the efforts of our mental
weapons to grasp it.}
  \flushright\sc  Alfred North Whitehead\footnote{\cite{Whitehead}, ch.\ 1.}\\
\end{quote}\index{Whitehead, Alfred North}

\subsection{Is Mathematics ``User-Friendly''?}

Suppose you have no formal training in abstract mathematics.  But popular
books you've read offer tempting glimpses of this world filled with profound
ideas that have stirred the human spirit.  You are not satisfied with the
informal, watered-down descriptions you've read but feel it is important to
grasp the underlying mathematics itself to understand its true meaning. It's
not practical to go back to school to learn it, though; you don't want to
dedicate years of your life to it.  There are many important things in life,
and you have to set priorities for what's important to you.  What would happen
if you tried to pursue it on your own, in your spare time?

After all, you were able to learn a computer programming language such as
Pascal on your own without too much difficulty, even though you had no formal
training in computers.  You don't claim to be an expert in software design,
but you can write a passable program when necessary to suit your needs.  Even
more important, you know that you can look at anyone else's Pascal program, no
matter how complex, and with enough patience figure out exactly how it works,
even though you are not a specialist.  Pascal allows you do anything that a
computer can do, at least in principle.  Thus you know you have the ability,
in principle, to follow anything that a computer program can do:  you just
have to break it down into small enough pieces.

Here's an imaginary scenario of what might happen if you na\-ive\-ly a\-dopted
this same view of abstract mathematics and tried to pick it up on your own, in
a period of time comparable to, say, learning a computer programming
language.

\subsubsection{A Non-Mathematician's Quest for Truth}

\begin{quote}
  {\em \ldots my daughters have been studying (chemistry) for several
se\-mes\-ters, think they have learned differential and integral calculus in
school, and yet even today don't know why $x\cdot y=y\cdot x$ is true.}
  \flushright\sc  Edmund Landau\footnote{\cite{Landau}, p.~vi.}\\
\end{quote}\index{Landau, Edmund}

\begin{quote}
  {\em Minus times minus is plus,\\
The reason for this we need not discuss.}
  \flushright\sc W.\ H.\ Auden\footnote{As quoted in \cite{Guillen}, p.~64.}\\
\end{quote}\index{Auden, W.\ H.}\index{Guillen, Michael}

We'll suppose you are a technically oriented professional, perhaps an engineer, a
computer programmer, or a physicist, but probably not a mathematician.  You
consider yourself reasonably intelligent.  You did well in school, learning a
variety of methods and techniques in practical mathematics such as calculus and
differential equations.  But rarely did your courses get into anything
resembling modern abstract mathematics, and proofs were something that appeared
only occasionally in your textbooks, a kind of necessary evil that was
supposed to convince you of a certain key result.  Most of your
homework consisted of exercises that gave you practice in the techniques, and
you were hardly ever asked to come up with a proof of your own.

You find yourself curious about advanced, abstract mathematics.  You are
driven by an inner conviction that it is important to understand and
appreciate some of the most profound knowledge discovered by mankind.  But it
seems very hard to learn, something that only certain gifted longhairs can
access and understand.  You are frustrated that it seems forever cut off from
you.

Eventually your curiosity drives you to do something about it.
You set for yourself a goal of ``really'' understanding mathematics:  not just
how to manipulate equations in algebra or calculus according to cookbook
rules, but rather to gain a deep understanding of where those rules come from.
In fact, you're not thinking about this kind of ordinary mathematics at all,
but about a much more abstract, ethereal realm of pure mathematics, where
famous results such as G\"{o}del's incompleteness theorem\index{G\"{o}del's
incompleteness theorem} and Cantor's different kinds of infinities
reside.

You have probably read a number of popular books, with titles like {\em
Infinity and the Mind} \cite{Rucker}\index{Rucker, Rudy}, on topics such as
these.  You found them inspiring but at the same time somewhat
unsatisfactory.  They gave you a general idea of what these results are about,
but if someone asked you to prove them, you wouldn't have the faintest idea of
where to begin.   Sure, you could give the same overall outline that you
learned from the popular books; and in a general sort of way, you do have an
understanding.  But deep down inside, you know that there is a rigor that is
missing, that probably there are many subtle steps and pitfalls along the way,
and ultimately it seems you have to place your trust in the experts in the
field.  You don't like this; you want to be able to verify these results for
yourself.

So where do you go next?  As a first step, you decide to look up some of the
original papers on the theorems you are curious about, or better, obtain some
standard textbooks in the field.  You look up a theorem you want to
understand.  Sure enough, it's there, but it's expressed with strange
terms and odd symbols that mean absolutely nothing to you.  It might as well be written in
a foreign language you've never seen before, whose symbols are totally alien.
You look at the proof, and you haven't the foggiest notion what each step
means, much less how one step follows from another.  Well, obviously you have
a lot to learn if you want to understand this stuff.

You feel that you could probably understand it by
going back to college for another three to six years and getting a math
degree.  But that does not fit in with your career and the other things in
your life and would serve no practical purpose.  You decide to seek a quicker
path.  You figure you'll just trace your way back to the beginning, step by
step, as you would do with a computer program, until you understand it.  But
you quickly find that this is not possible, since you can't even understand
enough to know what you have to trace back to.

Maybe a different approach is in order---maybe you should start at the
beginning and work your way up.  First, you read the introduction to the book
to find out what the prerequisites are.  In a similar fashion, you trace your
way back through two or three more books, finally arriving at one that seems
to start at a beginning:  it lists the axioms of arithmetic.  ``Aha!'' you
naively think, ``This must be the starting point, the source of all mathematical
knowledge.'' Or at least the starting point for mathematics dealing with
numbers; you have to start somewhere and have no idea what the starting point
for other mathematics would be.  But the word ``axioms'' looks promising.  So
you eagerly read along and work through some elementary exercises at the
beginning of the book.  You feel vaguely bothered:  these
don't seem like axioms at all, at least not in the sense that you want to
think of axioms.  Axioms imply a starting point from which everything else can
be built up, according to precise rules specified in the axiom system.  Even
though you can understand the first few proofs in an informal way,
and are able to do some of the
exercises, it's hard to pin down precisely what the
rules are.   Sure, each step seems to follow logically from the others, but
exactly what does that mean?  Is the ``logic'' just a matter of common sense,
something vague that we all understand but can never quite state precisely?

You've spent a number of years, off and on, programming computers, and you
know that in the case of computer languages there is no question of what the
rules are---they are precise and crystal clear.  If you follow them, your
program will work, and if you don't, it won't.  No matter how complex a
program, it can always be broken down into simpler and simpler pieces, until
you can ultimately identify the bits that are moved around to perform a
specific function.  Some programs might require a lot of perseverance to
accomplish this, but if you focus on a specific portion of it, you don't even
necessarily have to know how the rest of it works. Shouldn't there be an
analogy in mathematics?

You decide to apply the ultimate test:  you ask yourself how a computer could
verify or ensure that the steps in these proofs follow from one another.
Certainly mathematics must be at least as precisely defined as a computer
language, if not more so; after all, computer science itself is based on it.
If you can get a computer to verify these proofs, then you should also be
able, in principle, to understand them yourself in a very crystal clear,
precise way.

You're in for a surprise:  you can conceive of no way to convert the
proofs, which are in English, to a form that the computer can understand.
The proofs are filled with phrases such as ``assume there exists a unique
$x$\ldots'' and ``given any $y$, let $z$ be the number such that\ldots''  This
isn't the kind of logic you are used to in computer programming, where
everything, even arithmetic, reduces to Boolean ones and zeroes if you care to
break it down sufficiently.  Even though you think you understand the proofs,
there seems to be some kind of higher reasoning involved rather than precise
rules that define how you manipulate the symbols in the axioms.  Whatever it
is, it just isn't obvious how you would express it to a computer, and the more
you think about it, the more puzzled and confused you get, to the point where
you even wonder whether {\em you} really understand it.  There's a lot more to
these axioms of arithmetic than meets the eye.

Nobody ever talked about this in school in your applied math and engineering
courses.  You just learned the rules they gave you, not quite understanding
how or why they worked, sometimes vaguely suspicious or uncertain of them, and
through homework problems and osmosis learned how to present solutions that
satisfied the instructor and earned you an ``A.''  Rarely did you actually
``prove'' anything in a rigorous way, and the math majors who did do stuff
like that seemed to be in a different world.

Of course, there are computer algebra programs that can do mathematics, and
rather impressively.  They can instantly solve the integrals that you
struggled with in freshman calculus, and do much, much more.  But when you
look at these programs, what you see is a big collection of algorithms and
techniques that evolved and were added to over time, along with some basic
software that manipulates symbols.  Each algorithm that is built in is the
result of someone's theorem whose proof is omitted; you just have to trust the
person who proved it and the person who programmed it in and hope there are no
bugs.\index{computer program bugs}  Somehow this doesn't seem to be the
essence of mathematics.  Although computer algebra systems can generate
theorems with amazing speed, they can't actually prove a single one of them.

After some puzzlement, you revisit some popular books on what mathematics is
all about.  Somewhere you read that all of mathematics is actually derived
from something called ``set theory.''  This is a little confusing, because
nowhere in the book that presented the axioms of arithmetic was there any
mention of set theory, or if there was, it seemed to be just a tool that helps
you describe things better---the set of even numbers, that sort of thing.  If
set theory is the basis for all mathematics, then why are additional axioms
needed for arithmetic?

Something is wrong but you're not sure what.  One of your friends is a pure
mathematician.  He knows he is unable to communicate to you what he does for a
living and seems to have little interest in trying.  You do know that for him,
proofs are what mathematics is all about. You ask him what a proof is, and he
essentially tells you that, while of course it's based on logic, really it's
something you learn by doing it over and over until you pick it up.  He refers
you to a book, {\em How to Read and Do Proofs} \cite{Solow}.\index{Solow,
Daniel}  Although this book helps you understand traditional informal proofs,
there is still something missing you can't seem to pin down yet.

You ask your friend how you would go about having a computer verify a proof.
At first he seems puzzled by the question; why would you want to do that?
Then he says it's not something that would make any sense to do, but he's
heard that you'd have to break the proof down into thousands or even millions
of individual steps to do such a thing, because the reasoning involved is at
such a high level of abstraction.  He says that maybe it's something you could
do up to a point, but the computer would be completely impractical once you
get into any meaningful mathematics.  There, the only way you can verify a
proof is by hand, and you can only acquire the ability to do this by
specializing in the field for a couple of years in grad school.  Anyway, he
thinks it all has to do with set theory, although he has never taken a formal
course in set theory but just learned what he needed as he went along.

You are intrigued and amazed.  Apparently a mathematician can grasp as a
single concept something that would take a computer a thousand or a million
steps to verify, and have complete confidence in it.  Each one of these
thousand or million steps must be absolutely correct, or else the whole proof
is meaningless.  If you added a million numbers by hand, would you trust the
result?  How do you really know that all these steps are correct, that there
isn't some subtle pitfall in one of these million steps, like a bug in a
computer program?\index{computer program bugs}  After all, you've read that
famous mathematicians have occasionally made mistakes, and you certainly know
you've made your share on your math homework problems in school.

You recall the analogy with a computer program.  Sure, you can understand what
a large computer program such as a word processor does, as a single high-level
concept or a small set of such concepts, but your ability to understand it in
no way ensures that the program is correct and doesn't have hidden bugs.  Even
if you wrote the program yourself you can't really know this; most large
programs that you've written have had bugs that crop up at some later date, no
matter how careful you tried to be while writing them.

OK, so now it seems the reason you can't figure out how to make a
computer verify proofs is because each step really corresponds to a
million small steps.  Well, you say, a computer can do a million
calculations in a second, so maybe it's still practical to do.  Now the
puzzle becomes how to figure out what the million steps are that each
English-language step corresponds to.  Your mathematician friend hasn't
a clue, but suggests that maybe you would find the answer by studying
set theory.  Actually, your friend thinks you're a little off the wall
for even wondering such a thing.  For him, this is not what mathematics
is all about.

The subject of set theory keeps popping up, so you decide it's
time to look it up.

You decide to start off on a careful footing, so you start reading a couple of
very elementary books on set theory.  A lot of it seems pretty obvious, like
intersections, subsets, and Venn diagrams.  You thumb through one of the
books; nowhere is anything about axioms mentioned. The other book relegates to
an appendix a brief discussion that mentions a set of axioms called
``Zermelo--Fraenkel set theory''\index{Zermelo--Fraenkel set theory} and states
them in English.  You look at them and have no idea what they really mean or
what you can do with them.  The comments in this appendix say that the purpose
of mentioning them is to expose you to the idea, but imply that they are not
necessary for basic understanding and that they are really the subject matter
of advanced treatments where fine points such as a certain paradox (Russell's
paradox\index{Russell's paradox}\footnote{Russell's paradox assumes that there
exists a set $S$ that is a collection of all sets that don't contain
themselves.  Now, either $S$ contains itself or it doesn't.  If it contains
itself, it contradicts its definition.  But if it doesn't contain itself, it
also contradicts its definition.  Russell's paradox is resolved in ZF set
theory by denying that such a set $S$ exists.}) are resolved.  Wait a
minute---shouldn't the axioms be a starting point, not an ending point?  If
there are paradoxes that arise without the axioms, how do you know you won't
stumble across one accidentally when using the informal approach?

And nowhere do these books describe how ``all of mathematics can be
derived from set theory'' which by now you've heard a few times.

You find a more advanced book on set theory.  This one actually lists the
axioms of ZF set theory in plain English on page one.  {\em Now} you think
your quest has ended and you've finally found the source of all mathematical
knowledge; you just have to understand what it means.  Here, in one place, is
the basis for all of mathematics!  You stare at the axioms in awe, puzzle over
them, memorize them, hoping that if you just meditate on them long enough they
will become clear.  Of course, you haven't the slightest idea how the rest of
mathematics is ``derived'' from them; in particular, if these are the axioms
of mathematics, then why do arithmetic, group theory, and so on need their own
axioms?

You start reading this advanced book carefully, pondering the meaning of every
word, because by now you're really determined to get to the bottom of this.
The first thing the book does is explain how the axioms came about, which was
to resolve Russell's paradox.\index{Russell's paradox}  In fact that seems to
be the main purpose of their existence; that they supposedly can be used to
derive all of mathematics seems irrelevant and is not even mentioned.  Well,
you go on.  You hope the book will explain to you clearly, step by step, how
to derive things from the axioms.  After all, this is the starting point of
mathematics, like a book that explains the basics of a computer programming
language.  But something is missing.  You find you can't even understand the
first proof or do the first exercise.  Symbols such as $\exists$ and $\forall$
permeate the page without any mention of where they came from or how to
manipulate them; the author assumes you are totally familiar with them and
doesn't even tell you what they mean.  By now you know that $\exists$ means
``there exists'' and $\forall$ means ``for all,'' but shouldn't the rules for
manipulating these symbols be part of the axioms?  You still have no idea
how you could even describe the axioms to a computer.

Certainly there is something much different here from the technical
literature you're used to reading.  A computer language manual almost
always explains very clearly what all the symbols mean, precisely what
they do, and the rules used for combining them, and you work your way up
from there.

After glancing at four or five other such books, you come to the realization
that there is another whole field of study that you need just to get to the
point at which you can understand the axioms of set theory.  The field is
called ``logic.''  In fact, some of the books did recommend it as a
prerequisite, but it just didn't sink in.  You assumed logic was, well, just
logic, something that a person with common sense intuitively understood.  Why
waste your time reading boring treatises on symbolic logic, the manipulation
of 1's and 0's that computers do, when you already know that?  But this is a
different kind of logic, quite alien to you.  The subject of {\sc nand} and
{\sc nor} gates is not even touched upon or in any case has to do with only a
very small part of this field.

So your quest continues.  Skimming through the first couple of introductory
books, you get a general idea of what logic is about and what quantifiers
(``for all,'' ``there exists'') mean, but you find their examples somewhat
trivial and mildly annoying (``all dogs are animals,'' ``some animals are
dogs,'' and such).  But all you want to know is what the rules are for
manipulating the symbols so you can apply them to set theory.  Some formulas
describing the relationships among quantifiers ($\exists$ and $\forall$) are
listed in tables, along with some verbal reasoning to justify them.
Presumably, if you want to find out if a formula is correct, you go through
this same kind of mental reasoning process, possibly using images of dogs and
animals. Intuitively, the formulas seem to make sense.  But when you ask
yourself, ``What are the rules I need to get a computer to figure out whether
this formula is correct?'', you still don't know.  Certainly you don't ask the
computer to imagine dogs and animals.

You look at some more advanced logic books.  Many of them have an introductory
chapter summarizing set theory, which turns out to be a prerequisite.  You
need logic to understand set theory, but it seems you also need set theory to
understand logic!  These books jump right into proving rather advanced
theorems about logic, without offering the faintest clue about where the logic
came from that allows them to prove these theorems.

Luckily, you come across an elementary book of logic that, halfway through,
after the usual truth tables and metaphors, presents in a clear, precise way
what you've been looking for all along: the axioms!  They're divided into
propositional calculus (also called sentential logic) and predicate calculus
(also called first-order logic),\index{first-order logic} with rules so simple
and crystal clear that now you can finally program a computer to understand
them.  Indeed, they're no harder than learning how to play a game of chess.
As far as what you seem to need is concerned, the whole book could have been
written in five pages!

{\em Now} you think you've found the ultimate source of mathematical
truth.  So---the axioms of mathematics consist of these axioms of logic,
together with the axioms of ZF set theory. (By now you've also been able to
figure out how to translate the ZF axioms from English into the
actual symbols of logic which you can now manipulate according to
precise, easy-to-understand rules.)

Of course, you still don't understand how ``all of mathematics can be
derived from set theory,'' but maybe this will reveal itself in due
course.

You eagerly set out to program the axioms and rules into a computer and start
to look at the theorems you will have to prove as the logic is developed.  All
sorts of important theorems start popping up:  the deduction
theorem,\index{deduction theorem} the substitution theorem,\index{substitution
theorem} the completeness theorem of propositional calculus,\index{first-order
logic!completeness} the completeness theorem of predicate calculus.  Uh-oh,
there seems to be trouble.  They all get harder and harder, and not one of
them can be derived with the axioms and rules of logic you've just been
handed.  Instead, they all require ``metalogic'' for their proofs, a kind of
mixture of logic and set theory that allows you to prove things {\em about}
the axioms and theorems of logic rather than {\em with} them.

You plow ahead anyway.  A month later, you've spent much of your
free time getting the computer to verify proofs in propositional calculus.
You've programmed in the axioms, but you've also had to program in the
deduction theorem, the substitution theorem, and the completeness theorem of
propositional calculus, which by now you've resigned yourself to treating as
rather complex additional axioms, since they can't be proved from the axioms
you were given.  You can now get the computer to verify and even generate
complete, rigorous, formal proofs\index{formal proof}.  Never mind that they
may have 100,000 steps---at least now you can have complete, absolute
confidence in them.  Unfortunately, the only theorems you have proved are
pretty trivial and you can easily verify them in a few minutes with truth
tables, if not by inspection.

It looks like your mathematician friend was right.  Getting the computer to do
serious mathematics with this kind of rigor seems almost hopeless.  Even
worse, it seems that the further along you get, the more ``axioms'' you have
to add, as each new theorem seems to involve additional ``metamathematical''
reasoning that hasn't been formalized, and none of it can be derived from the
axioms of logic.  Not only do the proofs keep growing exponentially as you get
further along, but the program to verify them keeps getting bigger and bigger
as you program in more ``metatheorems.''\index{metatheorem}\footnote{A
metatheorem is usually a statement that is too general to be directly provable
in a theory.  For example, ``if $n_1$, $n_2$, and $n_3$ are integers, then
$n_1+n_2+n_3$ is an integer'' is a theorem of number theory.  But ``for any
integer $k > 1$, if $n_1, \ldots, n_k$ are integers, then $n_1+\ldots +n_k$ is
an integer'' is a metatheorem, in other words a family of theorems, one for
every $k$.  The reason it is not a theorem is that the general sum $n_1+\ldots
+n_k$ (as a function of $k$) is not an operation that can be defined directly
in number theory.} The bugs\index{computer program bugs} that have cropped up
so far have already made you start to lose faith in the rigor you seem to have
achieved, and you know it's just going to get worse as your program gets larger.

\subsection{Mathematics and the Non-Specialist}

\begin{quote}
  {\em A real proof is not checkable by a machine, or even by any mathematician
not privy to the gestalt, the mode of thought of the particular field of
mathematics in which the proof is located.}
  \flushright\sc  Davis and Hersh\index{Davis, Phillip J.}
  \footnote{\cite{Davis}, p.~354.}\\
\end{quote}

The bulk of abstract or theoretical mathematics is ordinarily outside
the reach of anyone but a few specialists in each field who have completed
the necessary difficult internship in order to enter its coterie.  The
typical intelligent layperson has no reasonable hope of understanding much of
it, nor even the specialist mathematician of understanding other fields.  It
is like a foreign language that has no dictionary to look up the translation;
the only way you can learn it is by living in the country for a few years.  It
is argued that the effort involved in learning a specialty is a necessary
process for acquiring a deep understanding.  Of course, this is almost certainly
true if one is to make significant contributions to a field; in particular,
``doing'' proofs is probably the most important part of a mathematician's
training.  But is it also necessary to deny outsiders access to it?  Is it
necessary that abstract mathematics be so hard for a layperson to grasp?

A computer normally is of no help whatsoever.  Most published proofs are
actually just series of hints written in an informal style that requires
considerable knowledge of the field to understand.  These are the ``real
proofs'' referred to by Davis and Hersh.\index{informal proof}  There is an
implicit understanding that, in principle, such a proof could be converted to
a complete formal proof\index{formal proof}.  However, it is said that no one
would ever attempt such a conversion, even if they could, because that would
presumably require millions of steps (Section~\ref{dream}).  Unfortunately the
informal style automatically excludes the understanding of the proof
by anyone who hasn't gone through the necessary apprenticeship. The
best that the intelligent layperson can do is to read popular books about deep
and famous results; while this can be helpful, it can also be misleading, and
the lack of detail usually leaves the reader with no ability whatsoever to
explore any aspect of the field being described.

The statements of theorems often use sophisticated notation that makes them
inaccessible to the non-specialist.  For a non-specialist who wants to achieve
a deeper understanding of a proof, the process of tracing definitions and
lemmas back through their hierarchy\index{hierarchy} quickly becomes confusing
and discouraging.  Textbooks are usually written to train mathematicians or to
communicate to people who are already mathematicians, and large gaps in proofs
are often left as exercises to the reader who is left at an impasse if he or
she becomes stuck.

I believe that eventually computers will enable non-specialists and even
intelligent laypersons to follow almost any mathematical proof in any field.
Metamath is an attempt in that direction.  If all of mathematics were as
easily accessible as a computer programming language, I could envision
computer programmers and hobbyists who otherwise lack mathematical
sophistication exploring and being amazed by the world of theorems and proofs
in obscure specialties, perhaps even coming up with results of their own.  A
tremendous advantage would be that anyone could experiment with conjectures in
any field---the computer would offer instant feedback as to whether
an inference step was correct.

Mathematicians sometimes have to put up with the annoyance of
cranks\index{cranks} who lack a fundamental understanding of mathematics but
insist that their ``proofs'' of, say, Fermat's Last Theorem\index{Fermat's
Last Theorem} be taken seriously.  I think part of the problem is that these
people are misled by informal mathematical language, treating it as if they
were reading ordinary expository English and failing to appreciate the
implicit underlying rigor.  Such cranks are rare in the field of computers,
because computer languages are much more explicit, and ultimately the proof is
in whether a computer program works or not.  With easily accessible
computer-based abstract mathematics, a mathematician could say to a crank,
``don't bother me until you've demonstrated your claim on the computer!''

% 22-May-04 nm
% Attempt to move De Millo quote so it doesn't separate from attribution
% CHANGE THIS NUMBER (AND ELIMINATE IF POSSIBLE) WHEN ABOVE TEXT CHANGES
\vspace{-0.5em}

\subsection{An Impossible Dream?}\label{dream}

\begin{quote}
  {\em Even quite basic theorems would demand almost unbelievably vast
  books to display their proofs.}
    \flushright\sc  Robert E. Edwards\footnote{\cite{Edwards}, p.~68.}\\
\end{quote}\index{Edwards, Robert E.}

\begin{quote}
  {\em Oh, of course no one ever really {\em does} it.  It would take
  forever!  You just show that you could do it, that's sufficient.}
    \flushright\sc  ``The Ideal Mathematician''
    \index{Davis, Phillip J.}\footnote{\cite{Davis},
p.~40.}\\
\end{quote}

\begin{quote}
  {\em There is a theorem in the primitive notation of set theory that
  corresponds to the arithmetic theorem `$1000+2000=3000$'.  The formula
  would be forbiddingly long\ldots even if [one] knows the definitions
  and is asked to simplify the long formula according to them, chances are
  he will make errors and arrive at some incorrect result.}
    \flushright\sc  Hao Wang\footnote{\cite{Wang}, p.~140.}\\
\end{quote}\index{Wang, Hao}

% 22-May-04 nm
% Attempt to move De Millo quote so it doesn't separate from attribution
% CHANGE THIS NUMBER (AND ELIMINATE IF POSSIBLE) WHEN ABOVE TEXT CHANGES
\vspace{-0.5em}

\begin{quote}
  {\em The {\em Principia Mathematica} was the crowning achievement of the
  formalists.  It was also the deathblow of the formalist view.\ldots
  {[Rus\-sell]} failed, in three enormous volumes, to get beyond the elementary
  facts of arithmetic.  He showed what can be done in principle and what
  cannot be done in practice.  If the mathematical process were really
  one of strict, logical progression, we would still be counting our
  fingers.\ldots
  One theoretician estimates\ldots that a demonstration of one of
  Ramanujan's conjectures assuming set theory and elementary analysis would
  take about two thousand pages; the length of a deduction from first principles
  is nearly in\-con\-ceiv\-a\-ble\ldots The probabilists argue that\ldots any
  very long proof can at best be viewed as only probably correct\ldots}
  \flushright\sc Richard de Millo et. al.\footnote{\cite{deMillo}, pp.~269,
  271.}\\
\end{quote}\index{de Millo, Richard}

A number of writers have conveyed the impression that the kind of absolute
rigor provided by Metamath\index{Metamath} is an impossible dream, suggesting
that a complete, formal verification\index{formal proof} of a typical theorem
would take millions of steps in untold volumes of books.  Even if it could be
done, the thinking sometimes goes, all meaning would be lost in such a
monstrous, tedious verification.\index{informal proof}\index{proof length}

These writers assume, however, that in order to achieve the kind of complete
formal verification they desire one must break down a proof into individual
primitive steps that make direct reference to the axioms.  This is
not necessary.  There is no reason not to make use of previously proved
theorems rather than proving them over and over.

Just as important, definitions\index{definition} can be introduced along
the way, allowing very complex formulas to be represented with few
symbols.  Not doing this can lead to absurdly long formulas.  For
example, the mere statement of
G\"{o}del's incompleteness theorem\index{G\"{o}del's
incompleteness theorem}, which can be expressed with a small number of
defined symbols, would require about 20,000 primitive symbols to express
it.\index{Boolos, George S.}\footnote{George S.\ Boolos, lecture at
Massachusetts Institute of Technology, spring 1990.} An extreme example
is Bourbaki's\label{bourbaki} language for set theory, which requires
4,523,659,424,929 symbols plus 1,179,618,517,981 disambiguatory links
(lines connecting symbol pairs, usually drawn below or above the
formula) to express the number
``one'' \cite{Mathias}.\index{Mathias, Adrian R. D.}\index{Bourbaki,
Nicolas}
% http://www.dpmms.cam.ac.uk/~ardm/

A hierarchy\index{hierarchy} of theorems and definitions permits an
exponential growth in the formula sizes and primitive proof steps to be
described with only a linear growth in the number of symbols used.  Of course,
this is how ordinary informal mathematics is normally done anyway, but with
Metamath\index{Metamath} it can be done with absolute rigor and precision.

\subsection{Beauty}


\begin{quote}
  {\em No one shall be able to drive us from the paradise that Cantor has
created for us.}
   \flushright\sc  David Hilbert\footnote{As quoted in \cite{Moore}, p.~131.}\\
\end{quote}\index{Hilbert, David}

\needspace{3\baselineskip}
\begin{quote}
  {\em Mathematics possesses not only truth, but some supreme beauty ---a
  beauty cold and austere, like that of a sculpture.}
    \flushright\sc  Bertrand
    Russell\footnote{\cite{Russell}.}\\
\end{quote}\index{Russell, Bertrand}

\begin{quote}
  {\em Euclid alone has looked on Beauty bare.}
  \flushright\sc Edna Millay\footnote{As quoted in \cite{Davis}, p.~150.}\\
\end{quote}\index{Millay, Edna}

For most people, abstract mathematics is distant, strange, and
incomprehensible.  Many popular books have tried to convey some of the sense
of beauty in famous theorems.  But even an intelligent layperson is left with
only a general idea of what a theorem is about and is hardly given the tools
needed to make use of it.  Traditionally, it is only after years of arduous
study that one can grasp the concepts needed for deep understanding.
Metamath\index{Metamath} allows you to approach the proof of the theorem from
a quite different perspective, peeling apart the formulas and definitions
layer by layer until an entirely different kind of understanding is achieved.
Every step of the proof is there, pieced together with absolute precision and
instantly available for inspection through a microscope with a magnification
as powerful as you desire.

A proof in itself can be considered an object of beauty.  Constructing an
elegant proof is an art.  Once a famous theorem has been proved, often
considerable effort is made to find simpler and more easily understood
proofs.  Creating and communicating elegant proofs is a major concern of
mathematicians.  Metamath is one way of providing a common language for
archiving and preserving this information.

The length of a proof can, to a certain extent, be considered an
objective measure of its ``beauty,'' since shorter proofs are usually
considered more elegant.  In the set theory database
\texttt{set.mm}\index{set theory database (\texttt{set.mm})}%
\index{Metamath Proof Explorer}
provided with Metamath, one goal was to make all proofs as short as possible.

\needspace{4\baselineskip}
\subsection{Simplicity}

\begin{quote}
  {\em God made man simple; man's complex problems are of his own
  devising.}
    \flushright\sc Eccles. 7:29\footnote{Jerusalem Bible.}\\
\end{quote}\index{Bible}

\needspace{3\baselineskip}
\begin{quote}
  {\em God made integers, all else is the work of man.}
    \flushright\sc Leopold Kronecker\footnote{{\em Jahresbericht
	der Deutschen Mathematiker-Vereinigung }, vol. 2, p. 19.}\\
\end{quote}\index{Kronecker, Leopold}

\needspace{3\baselineskip}
\begin{quote}
  {\em For what is clear and easily comprehended attracts; the
  complicated repels.}
    \flushright\sc David Hilbert\footnote{As quoted in \cite{deMillo},
p.~273.}\\
\end{quote}\index{Hilbert, David}

The Metamath\index{Metamath} language is simple and Spartan.  Metamath treats
all mathematical expressions as simple sequences of symbols, devoid of meaning.
The higher-level or ``metamathematical'' notions underlying Metamath are about
as simple as they could possibly be.  Each individual step in a proof involves
a single basic concept, the substitution of an expression for a variable, so
that in principle almost anyone, whether mathematician or not, can
completely understand how it was arrived at.

In one of its most basic applications, Metamath\index{Metamath} can be used to
develop the foundations of mathematics\index{foundations of mathematics} from
the very beginning.  This is done in the set theory database that is provided
with the Metamath package and is the subject matter
of Chapter~\ref{fol}. Any language (a metalanguage\index{metalanguage})
used to describe mathematics (an object language\index{object language}) must
have a mathematical content of its own, but it is desirable to keep this
content down to a bare minimum, namely that needed to make use of the
inference rules specified by the axioms.  With any metalanguage there is a
``chicken and egg'' problem somewhat like circular reasoning:  you must assume
the validity of the mathematics of the metalanguage in order to prove the
validity of the mathematics of the object language.  The mathematical content
of Metamath itself is quite limited.  Like the rules of a game of chess, the
essential concepts are simple enough so that virtually anyone should be able to
understand them (although that in itself will not let you play like
a master).  The symbols that Metamath manipulates do not in themselves
have any intrinsic meaning.  Your interpretation of the axioms that you supply
to Metamath is what gives them meaning.  Metamath is an attempt to strip down
mathematical thought to its bare essence and show you exactly how the symbols
are manipulated.

Philosophers and logicians, with various motivations, have often thought it
important to study ``weak'' fragments of logic\index{weak logic}
\cite{Anderson}\index{Anderson, Alan Ross} \cite{MegillBunder}\index{Megill,
Norman}\index{Bunder, Martin}, other unconventional systems of logic (such as
``modal'' logic\index{modal logic} \cite[ch.\ 27]{Boolos}\index{Boolos, George
S.}), and quantum logic\index{quantum logic} in physics
\cite{Pavicic}\index{Pavi{\v{c}}i{\'{c}}, M.}.  Metamath\index{Metamath}
provides a framework in which such systems can be expressed, with an absolute
precision that makes all underlying metamathematical assumptions rigorous and
crystal clear.

Some schools of philosophical thought, for example
intuitionism\index{intuitionism} and constructivism\index{constructivism},
demand that the notions underlying any mathematical system be as simple and
concrete as possible.  Metamath should meet the requirements of these
philosophies.  Metamath must be taught the symbols, axioms\index{axiom}, and
rules\index{rule} for a specific theory, from the skeptical (such as
intuitionism\index{intuitionism}\footnote{Intuitionism does not accept the law
of excluded middle (``either something is true or it is not true'').  See
\cite[p.~xi]{Tymoczko}\index{Tymoczko, Thomas} for discussion and references
on this topic.  Consider the theorem, ``There exist irrational numbers $a$ and
$b$ such that $a^b$ is rational.''  An intuitionist would reject the following
proof:  If $\sqrt{2}^{\sqrt{2}}$ is rational, we are done.  Otherwise, let
$a=\sqrt{2}^{\sqrt{2}}$ and $b=\sqrt{2}$. Then $a^b=2$, which is rational.})
to the bold (such as the axiom of choice in set theory\footnote{The axiom of
choice\index{Axiom of Choice} asserts that given any collection of pairwise
disjoint nonempty sets, there exists a set that has exactly one element in
common with each set of the collection.  It is used to prove many important
theorems in standard mathematics.  Some philosophers object to it because it
asserts the existence of a set without specifying what the set contains
\cite[p.~154]{Enderton}\index{Enderton, Herbert B.}.  In one foundation for
mathematics due to Quine\index{Quine, Willard Van Orman}, that has not been
otherwise shown to be inconsistent, the axiom of choice turns out to be false
\cite[p.~23]{Curry}\index{Curry, Haskell B.}.  The \texttt{show
trace{\char`\_}back} command of the Metamath program allows you to find out
whether the axiom of choice, or any other axiom, was assumed by a
proof.}\index{\texttt{show trace{\char`\_}back} command}).

The simplicity of the Metamath language lets the algorithm (computer program)
that verifies the validity of a Metamath proof be straightforward and
robust.  You can have confidence that the theorems it verifies really can be
derived from your axioms.

\subsection{Rigor}

\begin{quote}
  {\em Rigor became a goal with the Greeks\ldots But the efforts to
  pursue rigor to the utmost have led to an impasse in which there is
  no longer any agreement on what it really means.  Mathematics remains
  alive and vital, but only on a pragmatic basis.}
    \flushright\sc  Morris Kline\footnote{\cite{Kline}, p.~1209.}\\
\end{quote}\index{Kline, Morris}

Kline refers to a much deeper kind of rigor than that which we will discuss in
this section.  G\"{o}del's incompleteness theorem\index{G\"{o}del's
incompleteness theorem} showed that it is impossible to achieve absolute rigor
in standard mathematics because we can never prove that mathematics is
consistent (free from contradictions).\index{consistent theory}  If
mathematics is consistent, we will never know it, but must rely on faith.  If
mathematics is inconsistent, the best we can hope for is that some clever
future mathematician will discover the inconsistency.  In this case, the
axioms would probably be revised slightly to eliminate the inconsistency, as
was done in the case of Russell's paradox,\index{Russell's paradox} but the
bulk of mathematics would probably not be affected by such a discovery.
Russell's paradox, for example, did not affect most of the remarkable results
achieved by 19th-century and earlier mathematicians.  It mainly invalidated
some of Gottlob Frege's\index{Frege, Gottlob} work on the foundations of
mathematics in the late 1800's; in fact Frege's work inspired Russell's
discovery.  Despite the paradox, Frege's work contains important concepts that
have significantly influenced modern logic.  Kline's {\em Mathematics, The
Loss of Certainty} \cite{Klinel}\index{Kline, Morris} has an interesting
discussion of this topic.

What {\em can} be achieved with absolute certainty\index{certainty} is the
knowledge that if we assume the axioms are consistent and true, then the
results derived from them are true.  Part of the beauty of mathematics is that
it is the one area of human endeavor where absolute certainty can be achieved
in this sense.  A mathematical truth will remain such for eternity.  However,
our actual knowledge of whether a particular statement is a mathematical truth
is only as certain as the correctness of the proof that establishes it.  If
the proof of a statement is questionable or vague, we can't have absolute
confidence in the truth that the statement claims.

Let us look at some traditional ways of expressing proofs.

Except in the field of formal logic\index{formal logic}, almost all
traditional proofs in mathematics are really not proofs at all, but rather
proof outlines or hints as to how to go about constructing the proof.  Many
gaps\index{gaps in proofs} are left for the reader to fill in. There are
several reasons for this.  First, it is usually assumed in mathematical
literature that the person reading the proof is a mathematician familiar with
the specialty being described, and that the missing steps are obvious to such
a reader or at least that the reader is capable of filling them in.  This
attitude is fine for professional mathematicians in the specialty, but
unfortunately it often has the drawback of cutting off the rest of the world,
including mathematicians in other specialties, from understanding the proof.
We discussed one possible resolution to this on p.~\pageref{envision}.
Second, it is often assumed that a complete formal proof\index{formal proof}
would require countless millions of symbols (Section~\ref{dream}). This might
be true if the proof were to be expressed directly in terms of the axioms of
logic and set theory,\index{set theory} but it is usually not true if we allow
ourselves a hierarchy\index{hierarchy} of definitions and theorems to build
upon, using a notation that allows us to introduce new symbols, definitions,
and theorems in a precisely specified way.

Even in formal logic,\index{formal logic} formal proofs\index{formal proof}
that are considered complete still contain hidden or implicit information.
For example, a ``proof'' is usually defined as a sequence of
wffs,\index{well-formed formula (wff)}\footnote{A {\em wff} or well-formed
formula is a mathematical expression (string of symbols) constructed according
to some precise rules.  A formal mathematical system\index{formal system}
contains (1) the rules for constructing syntactically correct
wffs,\index{syntax rules} (2) a list of starting wffs called
axioms,\index{axiom} and (3) one or more rules prescribing how to derive new
wffs, called theorems\index{theorem}, from the axioms or previously derived
theorems.  An example of such a system is contained in
Metamath's\index{Metamath} set theory database, which defines a formal
system\index{formal system} from which all of standard mathematics can be
derived.  Section~\ref{startf} steps you through a complete example of a formal
system, and you may want to skim it now if you are unfamiliar with the
concept.} each of which is an axiom or follows from a rule applied to previous
wffs in the sequence.  The implicit part of the proof is the algorithm by
which a sequence of symbols is verified to be a valid wff, given the
definition of a wff.  The algorithm in this case is rather simple, but for a
computer to verify the proof,\index{automated proof verification} it must have
the algorithm built into its verification program.\footnote{It is possible, of
course, to specify wff construction syntax outside of the program itself
with a suitable input language (the Metamath language being an example), but
some proof-verification or theorem-proving programs lack the ability to extend
wff syntax in such a fashion.} If one deals exclusively with axioms and
elementary wffs, it is straightforward to implement such an algorithm.  But as
more and more definitions are added to the theory in order to make the
expression of wffs more compact, the algorithm becomes more and more
complicated.  A computer program that implements the algorithm becomes larger
and harder to understand as each definition is introduced, and thus more prone
to bugs.\index{computer program bugs}  The larger the program, the
more suspicious the mathematician may be about
the validity of its algorithms.  This is especially true because
computer programs are inherently hard to follow to begin with, and few people
enjoy verifying them manually in detail.

Metamath\index{Metamath} takes a different approach.  Metamath's ``knowledge''
is limited to the ability to substitute variables for expressions, subject to
some simple constraints.  Once the basic algorithm of Metamath is assumed to
be debugged, and perhaps independently confirmed, it
can be trusted once and for all.  The information that Metamath needs to
``understand'' mathematics is contained entirely in the body of knowledge
presented to Metamath.  Any errors in reasoning can only be errors in the
axioms or definitions contained in this body of knowledge.  As a
``constructive'' language\index{constructive language} Metamath has no
conditional branches or loops like the ones that make computer programs hard
to decipher; instead, the language can only build new sequences of symbols
from earlier sequences  of symbols.

The simplicity of the rules that underlie Metamath not only makes Metamath
easy to learn but also gives Metamath a great deal of flexibility. For
example, Metamath is not limited to describing standard first-order
logic\index{first-order logic}; higher-order logics\index{higher-order logic}
and fragments of logic\index{weak logic} can be described just as easily.
Metamath gives you the freedom to define whatever wff notation you prefer; it
has no built-in conception of the syntax of a wff.\index{well-formed formula
(wff)}  With suitable axioms and definitions, Metamath can even describe and
prove things about itself.\index{Metamath!self-description}  (John
Harrison\index{Harrison, John} discusses the ``reflection''
principle\index{reflection principle} involved in self-descriptive systems in
\cite{Harrison}.)

The flexibility of Metamath requires that its proofs specify a lot of detail,
much more than in an ordinary ``formal'' proof.\index{formal proof}  For
example, in an ordinary formal proof, a single step consists of displaying the
wff that constitutes that step.  In order for a computer program to verify
that the step is acceptable, it first must verify that the symbol sequence
being displayed is an acceptable wff.\index{automated proof verification} Most
proof verifiers have at least basic wff syntax built into their programs.
Metamath has no hard-wired knowledge of what constitutes a wff built into it;
instead every wff must be explicitly constructed based on rules defining wffs
that are present in a database.  Thus a single step in an ordinary formal
proof may be correspond to many steps in a Metamath proof. Despite the larger
number of steps, though, this does not mean that a Metamath proof must be
significantly larger than an ordinary formal proof. The reason is that since
we have constructed the wff from scratch, we know what the wff is, so there is
no reason to display it.  We only need to refer to a sequence of statements
that construct it.  In a sense, the display of the wff in an ordinary formal
proof is an implicit proof of its own validity as a wff; Metamath just makes
the proof explicit. (Section~\ref{proof} describes Metamath's proof notation.)

\section{Computers and Mathematicians}

\begin{quote}
  {\em The computer is important, but not to mathematics.}
    \flushright\sc  Paul Halmos\footnote{As quoted in \cite{Albers}, p.~121.}\\
\end{quote}\index{Halmos, Paul}

Pure mathematicians have traditionally been indifferent to computers, even to
the point of disdain.\index{computers and pure mathematics}  Computer science
itself is sometimes considered to fall in the mundane realm of ``applied''
mathematics, perhaps essential for the real world but intellectually unexciting
to those who seek the deepest truths in mathematics.  Perhaps a reason for this
attitude towards computers is that there is little or no computer software that
meets their needs, and there may be a general feeling that such software could
not even exist.  On the one hand, there are the practical computer algebra
systems, which can perform amazing symbolic manipulations in algebra and
calculus,\index{computer algebra system} yet can't prove the simplest
existence theorem, if the idea of a proof is present at all.  On the other
hand, there are specialized automated theorem provers that technically speaking
may generate correct proofs.\index{automated theorem proving}  But sometimes
their specialized input notation may be cryptic and their output perceived to
be long, inelegant, incomprehensible proofs.    The output
may be viewed with suspicion, since the program that generates it tends to be
very large, and its size increases the potential for bugs\index{computer
program bugs}.  Such a proof may be considered trustworthy only if
independently verified and ``understood'' by a human, but no one wants to
waste their time on such a boring, unrewarding chore.



\needspace{4\baselineskip}
\subsection{Trusting the Computer}

\begin{quote}
  {\em \ldots I continue to find the quasi-empirical interpretation of
  computer proofs to be the more plausible.\ldots Since not
  everything that claims to be a computer proof can be
  accepted as valid, what are the mathematical criteria for acceptable
  computer proofs?}
    \flushright\sc  Thomas Tymoczko\footnote{\cite{Tymoczko}, p.~245.}\\
\end{quote}\index{Tymoczko, Thomas}

In some cases, computers have been essential tools for proving famous
theorems.  But if a proof is so long and obscure that it can be verified in a
practical way only with a computer, it is vaguely felt to be suspicious.  For
example, proving the famous four-color theorem\index{four-color
theorem}\index{proof length} (``a map needs no more than four colors to
prevent any two adjacent countries from having the same color'') can presently
only be done with the aid of a very complex computer program which originally
required 1200 hours of computer time. There has been considerable debate about
whether such a proof can be trusted and whether such a proof is ``real''
mathematics \cite{Swart}\index{Swart, E. R.}.\index{trusting computers}

However, under normal circumstances even a skeptical mathematician would have a
great deal of confidence in the result of multiplying two numbers on a pocket
calculator, even though the precise details of what goes on are hidden from its
user.  Even the verification on a supercomputer that a huge number is prime is
trusted, especially if there is independent verification; no one bothers to
debate the philosophical significance of its ``proof,'' even though the actual
proof would be so large that it would be completely impractical to ever write
it down on paper.  It seems that if the algorithm used by the computer is
simple enough to be readily understood, then the computer can be trusted.

Metamath\index{Metamath} adopts this philosophy.  The simplicity of its
language makes it easy to learn, and because of its simplicity one can have
essentially absolute confidence that a proof is correct. All axioms, rules, and
definitions are available for inspection at any time because they are defined
by the user; there are no hidden or built-in rules that may be prone to subtle
bugs\index{computer program bugs}.  The basic algorithm at the heart of
Metamath is simple and fixed, and it can be assumed to be bug-free and robust
with a degree of confidence approaching certainty.
Independently written implementations of the Metamath verifier
can reduce any residual doubt on the part of a skeptic even further;
there are now over a dozen such implementations, written by many people.

\subsection{Trusting the Mathematician}\label{trust}

\begin{quote}
  {\em There is no Algebraist nor Mathematician so expert in his science, as
  to place entire confidence in any truth immediately upon his discovery of it,
  or regard it as any thing, but a mere probability.  Every time he runs over
  his proofs, his confidence encreases; but still more by the approbation of
  his friends; and is rais'd to its utmost perfection by the universal assent
  and applauses of the learned world.}
  \flushright\sc David Hume\footnote{{\em A Treatise of Human Nature}, as
  quoted in \cite{deMillo}, p.~267.}\\
\end{quote}\index{Hume, David}

\begin{quote}
  {\em Stanislaw Ulam estimates that mathematicians publish 200,000 theorems
  every year.  A number of these are subsequently contradicted or otherwise
  disallowed, others are thrown into doubt, and most are ignored.}
  \flushright\sc Richard de Millo et. al.\footnote{\cite{deMillo}, p.~269.}\\
\end{quote}\index{Ulam, Stanislaw}

Whether or not the computer can be trusted, humans  of course will occasionally
err. Only the most memorable proofs get independently verified, and of these
only a handful of truly great ones achieve the status of being ``known''
mathematical truths that are used without giving a second thought to their
correctness.

There are many famous examples of incorrect theorems and proofs in
mathematical literature.\index{errors in proofs}

\begin{itemize}
\item There have been thousands of purported proofs of Fermat's Last
Theorem\index{Fermat's Last Theorem} (``no integer solutions exist to $x^n +
y^n = z^n$ for $n > 2$''), by amateurs, cranks, and well-regarded
mathematicians \cite[p.~5]{Stark}\index{Stark, Harold M}.  Fermat wrote a note
in his copy of Bachet's {\em Diophantus} that he found ``a truly marvelous
proof of this theorem but this margin is too narrow to contain it''
\cite[p.~507]{Kramer}.  A recent, much publicized proof by Yoichi
Miyaoka\index{Miyaoka, Yoichi} was shown to be incorrect ({\em Science News},
April 9, 1988, p.~230).  The theorem was finally proved by Andrew
Wiles\index{Wiles, Andrew} ({\em Science News}, July 3, 1993, p.~5), but it
initially had some gaps and took over a year after its announcement to be
checked thoroughly by experts.  On Oct. 25, 1994, Wiles announced that the last
gap found in his proof had been filled in.
  \item In 1882, M. Pasch discovered that an axiom was omitted from Euclid's
formulation of geometry\index{Euclidean geometry}; without it, the proofs of
important theorems of Euclid are not valid.  Pasch's axiom\index{Pasch's
axiom} states that a line that intersects one side of a triangle must also
intersect another side, provided that it does not touch any of the triangle's
vertices.  The omission of Pasch's axiom went unnoticed for 2000
years \cite[p.~160]{Davis}, in spite of (one presumes) the thousands of
students, instructors, and mathematicians who studied Euclid.
  \item The first published proof of the famous Schr\"{o}der--Bernstein
theorem\index{Schr\"{o}der--Bernstein theorem} in set theory was incorrect
\cite[p.~148]{Enderton}\index{Enderton, Herbert B.}.  This theorem states
that if there exists a 1-to-1 function\footnote{A {\em set}\index{set} is any
collection of objects. A {\em function}\index{function} or {\em
mapping}\index{mapping} is a rule that assigns to each element of one set
(called the function's {\em domain}\index{domain}) an element from another
set.} from set $A$ into set $B$ and vice-versa, then sets $A$ and $B$ have
a 1-to-1 correspondence.  Although it sounds simple and obvious,
the standard proof is quite long and complex.
  \item In the early 1900's, Hilbert\index{Hilbert, David} published a
purported proof of the continuum hypothesis\index{continuum hypothesis}, which
was eventually established as unprovable by Cohen\index{Cohen, Paul} in 1963
\cite[p.~166]{Enderton}.  The continuum hypothesis states that no
infinity\index{infinity} (``transfinite cardinal number'')\index{cardinal,
transfinite} exists whose size (or ``cardinality''\index{cardinality}) is
between the size of the set of integers and the size of the set of real
numbers.  This hypothesis originated with German mathematician Georg
Cantor\index{Cantor, Georg} in the late 1800's, and his inability to prove it
is said to have contributed to mental illness that afflicted him in his later
years.
  \item An incorrect proof of the four-color theorem\index{four-color theorem}
was published by Kempe\index{Kempe, A. B.} in 1879
\cite[p.~582]{Courant}\index{Courant, Richard}; it stood for 11 years before
its flaw was discovered.  This theorem states that any map can be colored
using only four colors, so that no two adjacent countries have the same
color.  In 1976 the theorem was finally proved by the famous computer-assisted
proof of Haken, Appel, and Koch \cite{Swart}\index{Appel, K.}\index{Haken,
W.}\index{Koch, K.}.  Or at least it seems that way.  Mathematician
H.~S.~M.~Coxeter\index{Coxeter, H. S. M.} has doubts \cite[p.~58]{Davis}:  ``I
have a feeling that that is an untidy kind of use of the computers, and the more
you correspond with Haken and Appel, the more shaky you seem to be.''
  \item Many false ``proofs'' of the Poincar\'{e}
conjecture\index{Poincar\'{e} conjecture} have been proposed over the years.
This conjecture states that any object that mathematically behaves like a
three-dimensional sphere is a three-dimensional sphere topologically,
regardless of how it is distorted.  In March 1986, mathematicians Colin
Rourke\index{Rourke, Colin} and Eduardo R\^{e}go\index{R\^{e}go, Eduardo}
caused  a stir in the mathematical community by announcing that they had found
a proof; in November of that year the proof was found to be false \cite[p.
218]{PetersonI}.  It was finally proved in 2003 by Grigory Perelman
\label{poincare}\index{Szpiro, George}\index{Perelman, Grigory}\cite{Szpiro}.
 \end{itemize}

Many counterexamples to ``theorems'' in recent mathematical
literature related to Clifford algebras\index{Clifford algebras}
 have been found by Pertti
Lounesto (who passed away in 2002).\index{Lounesto, Pertti}
See the web page \url{http://mathforum.org/library/view/4933.html}.
% http://users.tkk.fi/~ppuska/mirror/Lounesto/counterexamples.htm

One of the purposes of Metamath\index{Metamath} is to allow proofs to be
expressed with absolute precision.  Developing a proof in the Metamath
language can be challenging, because Metamath will not permit even the
tiniest mistake.\index{errors in proofs}  But once the proof is created, its
correctness can be trusted immediately, without having to depend on the
process of peer review for confirmation.

\section{The Use of Computers in Mathematics}

\subsection{Computer Algebra Systems}

For the most part, you will find that Metamath\index{Metamath} is not a
practical tool for manipulating numbers.  (Even proving that $2 + 2 = 4$, if
you start with set theory, can be quite complex!)  Several commercial
mathematics packages are quite good at arithmetic, algebra, and calculus, and
as practical tools they are invaluable.\index{computer algebra system} But
they have no notion of proof, and cannot understand statements starting with
``there exists such and such...''.

Software packages such as Mathematica \cite{Wolfram}\index{Mathematica} do not
concern themselves with proofs but instead work directly with known results.
These packages primarily emphasize heuristic rules such as the substitution of
equals for equals to achieve simpler expressions or expressions in a different
form.  Starting with a rich collection of built-in rules and algorithms, users
can add to the collection by means of a powerful programming language.
However, results such as, say, the existence of a certain abstract object
without displaying the actual object cannot be expressed (directly) in their
languages.  The idea of a proof from a small set of axioms is absent.  Instead
this software simply assumes that each fact or rule you add to the built-in
collection of algorithms is valid.  One way to view the software is as a large
collection of axioms from which the software, with certain goals, attempts to
derive new theorems, for example equating a complex expression with a simpler
equivalent. But the terms ``theorem''\index{theorem} and
``proof,''\index{proof} for example, are not even mentioned in the index of
the user's manual for Mathematica.\index{Mathematica and proofs}  What is also
unsatisfactory from a philosophical point of view is that there is no way to
ensure the validity of the results other than by trusting the writer of each
application module or tediously checking each module by hand, similar to
checking a computer program for bugs.\index{computer program
bugs}\footnote{Two examples illustrate why the knowledge database of computer
algebra systems should sometimes be regarded with a certain caution.  If you
ask Mathematica (version 3.0) to \texttt{Solve[x\^{ }n + y\^{ }n == z\^{ }n , n]}
it will respond with \texttt{\{\{n-\char`\>-2\}, \{n-\char`\>-1\},
\{n-\char`\>1\}, \{n-\char`\>2\}\}}. In other words, Mathematica seems to
``know'' that Fermat's Last Theorem\index{Fermat's Last Theorem} is true!  (At
the time this version of Mathematica was released this fact was unknown.)  If
you ask Maple\index{Maple} to \texttt{solve(x\^{ }2 = 2\^{ }x)} then
\texttt{simplify(\{"\})}, it returns the solution set \texttt{\{2, 4\}}, apparently
unaware that 0.7666647\ldots is also a solution.} While of course extremely
valuable in applied mathematics, computer algebra systems tend to be of little
interest to the theoretical mathematician except as aids for exploring certain
specific problems.

Because of possible bugs, trusting the output of a computer algebra system for
use as theorems in a proof-verifier would defeat the latter's goal of rigor.
On the other hand, a fact such that a certain relatively large number is
prime, while easy for a computer algebra system to derive, might have a long,
tedious proof that could overwhelm a proof-verifier. One approach for linking
computer algebra systems to a proof-verifier while retaining the advantages of
both is to add a hypothesis to each such theorem indicating its source.  For
example, a constant {\sc maple} could indicate the theorem came from the Maple
package, and would mean ``assuming Maple is consistent, then\ldots''  This and
many other topics concerning the formalization of mathematics are discussed in
John Harrison's\index{Harrison, John} very interesting
PhD thesis~\cite{Harrison-thesis}.

\subsection{Automated Theorem Provers}\label{theoremprovers}

A mathematical theory is ``decidable''\index{decidable theory} if a mechanical
method or algorithm exists that is guaranteed to determine whether or not a
particular formula is a theorem.  Among the few theories that are decidable is
elementary geometry,\index{Euclidean geometry} as was shown by a classic
result of logician Alfred Tarski\index{Tarski, Alfred} in 1948
\cite{Tarski}.\footnote{Tarski's result actually applies to a subset of the
geometry discussed in elementary textbooks.  This subset includes most of what
would be considered elementary geometry but it is not powerful enough to
express, among other things, the notions of the circumference and area of a
circle.  Extending the theory in a way that includes notions such as these
makes the theory undecidable, as was also shown by Tarski.  Tarski's algorithm
is far too inefficient to implement practically on a computer.  A practical
algorithm for proving a smaller subset of geometry theorems (those not
involving concepts of ``order'' or ``continuity'') was discovered by Chinese
mathematician Wu Wen-ts\"{u}n in 1977 \cite{Chou}\index{Chou,
Shang-Ching}.}\index{Wen-ts{\"{u}}n, Wu}  But most theories, including
elementary arithmetic, are undecidable.  This fact contributes to keeping
mathematics alive and well, since many mathematicians believe
that they will never be
replaced by computers (if they believe Roger Penrose's argument that a
computer can never replace the brain \cite{Penrose}\index{Penrose, Roger}).
In fact,  elementary geometry is often considered a ``dead'' field
for the simple reason that it is decidable.

On the other hand, the undecidability of a theory does not mean that one cannot
use a computer to search for proofs, providing one is willing to give up if a
proof is not found after a reasonable amount of time.  The field of automated
theorem proving\index{automated theorem proving} specializes in pursuing such
computer searches.  Among the more successful results to date are those based
on an algorithm known as Robinson's resolution principle
\cite{Robinson}\index{Robinson's resolution principle}.

Automated theorem provers can be excellent tools for those willing to learn
how to use them.  But they are not widely used in mainstream pure
mathematics, even though they could probably be useful in many areas.  There
are several reasons for this.  Probably most important, the main goal in pure
mathematics is to arrive at results that are considered to be deep or
important; proving them is essential but secondary.  Usually, an automated
theorem prover cannot assist in this main goal, and by the time the main goal
is achieved, the mathematician may have already figured out the proof as a
by-product.  There is also a notational problem.  Mathematicians are used to
using very compact syntax where one or two symbols (heavily dependent on
context) can represent very complex concepts; this is part of the
hierarchy\index{hierarchy} they have built up to tackle difficult problems.  A
theorem prover on the other hand might require that a theorem be expressed in
``first-order logic,''\index{first-order logic} which is the logic on which
most of mathematics is ultimately based but which is not ordinarily used
directly because expressions can become very long.  Some automated theorem
provers are experimental programs, limited in their use to very specialized
areas, and the goal of many is simply research into the nature of automated
theorem proving itself.  Finally, much research remains to be done to enable
them to prove very deep theorems.  One significant result was a
computer proof by Larry Wos\index{Wos, Larry} and colleagues that every Robbins
algebra\index{Robbins algebra} is a Boolean  algebra\index{Boolean algebra}
({\em New York Times}, Dec. 10, 1996).\footnote{In 1933, E.~V.\
Huntington\index{Huntington, E. V.}
presented the following axiom system for
Boolean algebra with a unary operation $n$ and a binary operation $+$:
\begin{center}
    $x + y = y + x$ \\
    $(x + y) + z = x + (y + z)$ \\
    $n(n(x) + y) + n(n(x) + n(y)) = x$
\end{center}
Herbert Robbins\index{Robbins, Herbert}, a student of Huntington, conjectured
that the last equation can be replaced with a simpler one:
\begin{center}
    $n(n(x + y) + n(x + n(y))) = x$
\end{center}
Robbins and Huntington could not find a proof.  The problem was
later studied unsuccessfully by Tarski\index{Tarski, Alfred} and his
students, and it remained an unsolved problem until a
computer found the proof in 1996.  For more information on
the Robbins algebra problem see \cite{Wos}.}

How does Metamath\index{Metamath} relate to automated theorem provers?  A
theorem prover is primarily concerned with one theorem at a time (perhaps
tapping into a small database of known theorems) whereas Metamath is more like
a theorem archiving system, storing both the theorem and its proof in a
database for access and verification.  Metamath is one answer to ``what do you
do with the output of a theorem prover?''  and could be viewed as the
next step in the process.  Automated theorem provers could be useful tools for
helping develop its database.
Note that very long, automatically
generated proofs can make your database fat and ugly and cause Metamath's proof
verification to take a long time to run.  Unless you have a particularly good
program that generates very concise proofs, it might be best to consider the
use of automatically generated proofs as a quick-and-dirty approach, to be
manually rewritten at some later date.

The program {\sc otter}\index{otter@{\sc otter}}\footnote{\url{http://www.cs.unm.edu/\~mccune/otter/}.}, later succeeded by
prover9\index{prover9}\footnote{\url{https://www.cs.unm.edu/~mccune/mace4/}.},
have been historically influential.
The E prover\index{E prover}\footnote{\url{https://github.com/eprover/eprover}.}
is a maintained automated theorem prover
for full first-order logic with equality.
There are many other automated theorem provers as well.

If you want to combine automated theorem provers with Metamath
consider investigating
the book {\em Automated Reasoning:  Introduction and Applications}
\cite{Wos}\index{Wos, Larry}.  This book discusses
how to use {\sc otter} in a way that can
not only able to generate
relatively efficient proofs, it can even be instructed to search for
shorter proofs.  The effective use of {\sc otter} (and similar tools)
does require a certain
amount of experience, skill, and patience.  The axiom system used in the
\texttt{set.mm}\index{set theory database (\texttt{set.mm})} set theory
database can be expressed to {\sc otter} using a method described in
\cite{Megill}.\index{Megill, Norman}\footnote{To use those axioms with
{\sc otter}, they must be restated in a way that eliminates the need for
``dummy variables.''\index{dummy variable!eliminating} See the Comment
on p.~\pageref{nodd}.} When successful, this method tends to generate
short and clever proofs, but my experiments with it indicate that the
method will find proofs within a reasonable time only for relatively
easy theorems.  It is still fun to experiment with.

Reference \cite{Bledsoe}\index{Bledsoe, W. W.} surveys a number of approaches
people have explored in the field of automated theorem proving\index{automated
theorem proving}.

\subsection{Interactive Theorem Provers}\label{interactivetheoremprovers}

Finding proofs completely automatically is difficult, so there
are some interactive theorem provers that allow a human to guide the
computer to find a proof.
Examples include
HOL Light\index{HOL light}%
\footnote{\url{https://www.cl.cam.ac.uk/~jrh13/hol-light/}.},
Isabelle\index{Isabelle}%
\footnote{\url{http://www.cl.cam.ac.uk/Research/HVG/Isabelle}.},
{\sc hol}\index{hol@{\sc hol}}%
\footnote{\url{https://hol-theorem-prover.org/}.},
and
Coq\index{Coq}\footnote{\url{https://coq.inria.fr/}.}.

A major difference between most of these tools and Metamath is that the
``proofs'' are actually programs that guide the program to find a proof,
and not the proof itself.
For example, an Isabelle/HOL proof might apply a step
\texttt{apply (blast dest: rearrange reduction)}. The \texttt{blast}
instruction applies
an automatic tableux prover and returns if it found a sequence of proof
steps that work... but the sequence is not considered part of the proof.

A good overview of
higher-level proof verification languages (such as {\sc lcf}\index{lcf@{\sc
lcf}} and {\sc hol}\index{hol@{\sc hol}})
is given in \cite{Harrison}.  All of these languages are fundamentally
different from Metamath in that much of the mathematical foundational
knowledge is embedded in the underlying proof-verification program, rather
than placed directly in the database that is being verified.
These can have a steep learning curve for those without a mathematical
background.  For example, one usually must have a fair understanding of
mathematical logic in order to follow their proofs.

\subsection{Proof Verifiers}\label{proofverifiers}

A proof verifier is a program that doesn't generate proofs but instead
verifies proofs that you give it.  Many proof verifiers have limited built-in
automated proof capabilities, such as figuring out simple logical inferences
(while still being guided by a person who provides the overall proof).  Metamath
has no built-in automated proof capability other than the limited
capability of its Proof Assistant.

Proof-verification languages are not used as frequently as they might be.
Pure mathematicians are more concerned with producing new results, and such
detail and rigor would interfere with that goal.  The use of computers in pure
mathematics is primarily focused on automated theorem provers (not verifiers),
again with the ultimate goal of aiding the creation of new mathematics.
Automated theorem provers are usually concerned with attacking one theorem at
time rather than making a large, organized database easily available to the
user.  Metamath is one way to help close this gap.

By itself Metamath is a mostly a proof verifier.
This does not mean that other approaches can't be used; the difference
is that in Metamath, the results of various provers must be recorded
step-by-step so that they can be verified.

Another proof-verification language is Mizar,\index{Mizar} which can display
its proofs in the informal language that mathematicians are accustomed to.
Information on the Mizar language is available at \url{http://mizar.org}.

For the working mathematician, Mizar is an excellent tool for rigorously
documenting proofs. Mizar typesets its proofs in the informal English used by
mathematicians (and, while fine for them, are just as inscrutable by
laypersons!). A price paid for Mizar is a relatively steep learning curve of a
couple of weeks.  Several mathematicians are actively formalizing different
areas of mathematics using Mizar and publishing the proofs in a dedicated
journal. Unfortunately the task of formalizing mathematics is still looked
down upon to a certain extent since it doesn't involve the creation of ``new''
mathematics.

The closest system to Metamath is
the {\em Ghilbert}\index{Ghilbert} proof language (\url{http://ghilbert.org})
system developed by
Raph Levien\index{Levien, Raph}.
Ghilbert is a formal proof checker heavily inspired by Metamath.
Ghilbert statements are s-expressions (a la Lisp), which is easy
for computers to parse but many people find them hard to read.
There are a number of differences in their specific constructs, but
there is at least one tool to translate some Metamath materials into Ghilbert.
As of 2019 the Ghilbert community is smaller and less active than the
Metamath community.
That said, the Metamath and Ghilbert communities overlap, and fruitful
conversations between them have occurred many times over the years.

\subsection{Creating a Database of Formalized Mathematics}\label{mathdatabase}

Besides Metamath, there are several other ongoing projects with the goal of
formalizing mathematics into computer-verifiable databases.
Understanding some history will help.

The {\sc qed}\index{qed project@{\sc qed} project}%
\footnote{\url{http://www-unix.mcs.anl.gov/qed}.}
project arose in 1993 and its goals were outlined in the
{\sc qed} manifesto.
The {\sc qed} manifesto was
a proposal for a computer-based database of all mathematical knowledge,
strictly formalized and with all proofs having been checked automatically.
The project had a conference in 1994 and another in 1995;
there was also a ``twenty years of the {\sc qed} manifesto'' workshop
in 2014.
Its ideals are regularly reraised.

In a 2007 paper, Freek Wiedijk identified two reasons
for the failure of the {\sc qed} project as originally envisioned:%
\cite{Wiedijk-revisited}\index{Wiedijk, Freek}

\begin{itemize}
\item Very few people are working on formalization of mathematics. There is no compelling application for fully mechanized mathematics.
\item Formalized mathematics does not yet resemble traditional mathematics. This is partly due to the complexity of mathematical notation, and partly to the limitations of existing theorem provers and proof assistants.
\end{itemize}

But this did not end the dream of
formalizing mathematics into computer-verifiable databases.
The problems that led to the {\sc qed} manifesto are still with us,
even though the challenges were harder than originally considered.
What has happened instead is that various independent projects have
worked towards formalizing mathematics into computer-verifiable databases,
each simultaneously competing and cooperating with each other.

A concrete way to see this is
Freek Wiedijk's ``Formalizing 100 Theorems'' list%
\footnote{\url{http://www.cs.ru.nl/\%7Efreek/100/}.}
which shows the progress different systems have made on a challenge list
of 100 mathematical theorems.%
\footnote{ This is not the only list of ``interesting'' theorems.
Another interesting list was posted by Oliver Knill's list
\cite{Knill}\index{Knill, Oliver}.}
The top systems as of February 2019
(in order of the number of challenges completed) are
HOL Light, Isabelle, Metamath, Coq, and Mizar.

The Metamath 100%
\footnote{\url{http://us.metamath.org/mm\_100.html}}
page (maintained by David A. Wheeler\index{Wheeler, David A.})
shows the progress of Metamath (specifically its \texttt{set.mm} database)
against this challenge list maintained by Freek Wiedijk.
The Metamath \texttt{set.mm} database
has made a lot of progress over the years,
in part because working to prove those challenge theorems required
defining various terms and proving their properties as a prerequisite.
Here are just a few of the many statements that have been
formally proven with Metamath:

% The entries of this cause the narrow display to break poorly,
% since the short amount of text means LaTeX doesn't get a lot to work with
% and the itemize format gives it even *less* margin than usual.
% No one will mind if we make just this list flushleft, since this list
% will be internally consistent.
\begin{flushleft}
\begin{itemize}
\item 1. The Irrationality of the Square Root of 2
  (\texttt{sqr2irr}, by Norman Megill, 2001-08-20)
\item 2. The Fundamental Theorem of Algebra
  (\texttt{fta}, by Mario Carneiro, 2014-09-15)
\item 22. The Non-Denumerability of the Continuum
  (\texttt{ruc}, by Norman Megill, 2004-08-13)
\item 54. The Konigsberg Bridge Problem
  (\texttt{konigsberg}, by Mario Carneiro, 2015-04-16)
\item 83. The Friendship Theorem
  (\texttt{friendship}, by Alexander W. van der Vekens, 2018-10-09)
\end{itemize}
\end{flushleft}

We thank all of those who have developed at least one of the Metamath 100
proofs, and we particularly thank
Mario Carneiro\index{Carneiro, Mario}
who has contributed the most Metamath 100 proofs as of 2019.
The Metamath 100 page shows the list of all people who have contributed a
proof, and links to graphs and charts showing progress over time.
We encourage others to work on proving theorems not yet proven in Metamath,
since doing so improves the work as a whole.

Each of the math formalization systems (including Metamath)
has different strengths and weaknesses, depending on what you value.
Key aspects that differentiate Metamath from the other top systems are:

\begin{itemize}
\item Metamath is not tied to any particular set of axioms.
\item Metamath can show every step of every proof, no exceptions.
  Most other provers only assert that a proof can be found, and do not
  show every step. This also makes verification fast, because
  the system does not need to rediscover proof details.
\item The Metamath verifier has been re-implemented in many different
  programming languages, so verification can be done by multiple
  implementations.  In particular, the
  \texttt{set.mm}\index{set theory database (\texttt{set.mm})}%
  \index{Metamath Proof Explorer} database is verified by
  four different verifiers
  written in four different languages by four different authors.
  This greatly reduces the risk of accepting an invalid
  proof due to an error in the verifier.
\item Proofs stay proven.  In some systems, changes to the system's
  syntax or how a tactic works causes proofs to fail in later versions,
  causing older work to become essentially lost.
  Metamath's language is
  extremely small and fixed, so once a proof is added to a database,
  the database can be rechecked with later versions of the Metamath program
  and with other verifiers of Metamath databases.
  If an axiom or key definition needs to be changed, it is easy to
  manipulate the database as a whole to handle the change
  without touching the underlying verifier.
  Since re-verification of an entire database takes seconds, there
  is never a reason to delay complete verification.
  This aspect is especially compelling if your
  goal is to have a long-term database of proofs.
\item Licensing is generous.  The main Metamath databases are released to
  the public domain, and the main Metamath program is open source software
  under a standard, widely-used license.
\item Substitutions are easy to understand, even by those who are not
  professional mathematicians.
\end{itemize}

Of course, other systems may have advantages over Metamath
that are more compelling, depending on what you value.
In any case, we hope this helps you understand Metamath
within a wider context.

\subsection{In Summary}\label{computers-summary}

To summarize our discussions of computers and mathematics, computer algebra
systems can be viewed as theorem generators focusing on a narrow realm of
mathematics (numbers and their properties), automated theorem provers as proof
generators for specific theorems in a much broader realm covered by a built-in
formal system such as first-order logic, interactive theorem
provers require human guidance, proof verifiers verify proofs but
historically they have been
restricted to first-order logic.
Metamath, in contrast,
is a proof verifier and documenter whose realm is essentially unlimited.

\section{Mathematics and Metamath}

\subsection{Standard Mathematics}

There are a number of ways that Metamath\index{Metamath} can be used with
standard mathematics.  The most satisfying way philosophically is to start at
the very beginning, and develop the desired mathematics from the axioms of
logic and set theory.\index{set theory}  This is the approach taken in the
\texttt{set.mm}\index{set theory database (\texttt{set.mm})}%
\index{Metamath Proof Explorer}
database (also known as the Metamath Proof Explorer).
Among other things, this database builds up to the
axioms of real and complex numbers\index{analysis}\index{real and complex
numbers} (see Section~\ref{real}), and a standard development of analysis, for
example, could start at that point, using it as a basis.   Besides this
philosophical advantage, there are practical advantages to having all of the
tools of set theory available in the supporting infrastructure.

On the other hand, you may wish to start with the standard axioms of a
mathematical theory without going through the set theoretical proofs of those
axioms.  You will need mathematical logic to make inferences, but if you wish
you can simply introduce theorems\index{theorem} of logic as
``axioms''\index{axiom} wherever you need them, with the implicit assumption
that in principle they can be proved, if they are obvious to you.  If you
choose this approach, you will probably want to review the notation used in
\texttt{set.mm}\index{set theory database (\texttt{set.mm})} so that your own
notation will be consistent with it.

\subsection{Other Formal Systems}
\index{formal system}

Unlike some programs, Metamath\index{Metamath} is not limited to any specific
area of mathematics, nor committed to any particular mathematical philosophy
such as classical logic versus intuitionism, nor limited, say, to expressions
in first-order logic.  Although the database \texttt{set.mm}
describes standard logic and set theory, Meta\-math
is actually a general-purpose language for describing a wide variety of formal
systems.\index{formal system}  Non-standard systems such as modal
logic,\index{modal logic} intuitionist logic\index{intuitionism}, higher-order
logic\index{higher-order logic}, quantum logic\index{quantum logic}, and
category theory\index{category theory} can all be described with the Metamath
language.  You define the symbols you prefer and tell Metamath the axioms and
rules you want to start from, and Metamath will verify any inferences you make
from those axioms and rules.  A simple example of a non-standard formal system
is Hofstadter's\index{Hofstadter, Douglas R.} MIU system,\index{MIU-system}
whose Metamath description is presented in Appendix~\ref{MIU}.

This is not hypothetical.
The largest Metamath database is
\texttt{set.mm}\index{set theory database (\texttt{set.mm}}%
\index{Metamath Proof Explorer}), aka the Metamath Proof Explorer,
which uses the most common axioms for mathematical foundations
(specifically classical logic combined with Zermelo--Fraenkel
set theory\index{Zermelo--Fraenkel set theory} with the Axiom of Choice).
But other Metamath databases are available:

\begin{itemize}
\item The database
  \texttt{iset.mm}\index{intuitionistic logic database (\texttt{iset.mm})},
  aka the
  Intuitionistic Logic Explorer\index{Intuitionistic Logic Explorer},
  uses intuitionistic logic (a constructivist point of view)
  instead of classical logic.
\item The database
  \texttt{nf.mm}\index{New Foundations database (\texttt{nf.mm})},
  aka the
  New Foundations Explorer\index{New Foundations Explorer},
  constructs mathematics from scratch,
  starting from Quine's New Foundations (NF) set theory axioms.
\item The database
  \texttt{hol.mm}\index{Higher-order Logic database (\texttt{hol.mm})},
  aka the
  Higher-Order Logic (HOL) Explorer\index{Higher-Order Logic (HOL) Explorer},
  starts with HOL (also called simple type theory) and derives
  equivalents to ZFC axioms, connecting the two approaches.
\end{itemize}

Since the days of David Hilbert,\index{Hilbert, David} mathematicians have
been concerned with the fact that the metalanguage\index{metalanguage} used to
describe mathematics may be stronger than the mathematics being described.
Metamath\index{Metamath}'s underlying finitary\index{finitary proof},
constructive nature provides a good philosophical basis for studying even the
weakest logics.\index{weak logic}

The usual treatment of many non-standard formal systems\index{formal
system} uses model theory\index{model theory} or proof theory\index{proof
theory} to describe these systems; these theories, in turn, are based on
standard set theory.  In other words, a non-standard formal system is defined
as a set with certain properties, and standard set theory is used to derive
additional properties of this set.  The standard set theory database provided
with Metamath can be used for this purpose, and when used this way
the development of a special
axiom system for the non-standard formal system becomes unnecessary.  The
model- or proof-theoretic approach often allows you to prove much deeper
results with less effort.

Metamath supports both approaches.  You can define the non-standard
formal system directly, or define the non-standard formal system as
a set with certain properties, whichever you find most helpful.

%\section{Additional Remarks}

\subsection{Metamath and Its Philosophy}

Closely related to Metamath\index{Metamath} is a philosophy or way of looking
at mathematics. This philosophy is related to the formalist
philosophy\index{formalism} of Hilbert\index{Hilbert, David} and his followers
\cite[pp.~1203--1208]{Kline}\index{Kline, Morris}
\cite[p.~6]{Behnke}\index{Behnke, H.}. In this philosophy, mathematics is
viewed as nothing more than a set of rules that manipulate symbols, together
with the consequences of those rules.  While the mathematics being described
may be complex, the rules used to describe it (the
``metamathematics''\index{metamathematics}) should be as simple as possible.
In particular, proofs should be restricted to dealing with concrete objects
(the symbols we write on paper rather than the abstract concepts they
represent) in a constructive manner; these are called ``finitary''
proofs\index{finitary proof} \cite[pp.~2--3]{Shoenfield}\index{Shoenfield,
Joseph R.}.

Whether or not you find Metamath interesting or useful will in part depend on
the appeal you find in its philosophy, and this appeal will probably depend on
your particular goals with respect to mathematics.  For example, if you are a
pure mathematician at the forefront of discovering new mathematical knowledge,
you will probably find that the rigid formality of Metamath stifles your
creativity.  On the other hand, we would argue that once this knowledge is
discovered, there are advantages to documenting it in a standard format that
will make it accessible to others.  Sixty years from now, your field may be
dormant, and as Davis and Hersh put it, your ``writings would become less
translatable than those of the Maya'' \cite[p.~37]{Davis}\index{Davis, Phillip
J.}.


\subsection{A History of the Approach Behind Metamath}

Probably the one work that has had the most motivating influence on
Metamath\index{Metamath} is Whitehead and Russell's monumental {\em Principia
Mathematica} \cite{PM}\index{Whitehead, Alfred North}\index{Russell,
Bertrand}\index{principia mathematica@{\em Principia Mathematica}}, whose aim
was to deduce all of mathematics from a small number of primitive ideas, in a
very explicit way that in principle anyone could understand and follow.  While
this work was tremendously influential in its time, from a modern perspective
it suffers from several drawbacks.  Both its notation and its underlying
axioms are now considered dated and are no longer used.  From our point of
view, its development is not really as accessible as we would like to see; for
practical reasons, proofs become more and more sketchy as its mathematics
progresses, and working them out in fine detail requires a degree of
mathematical skill and patience that many people don't have.  There are
numerous small errors, which is understandable given the tedious, technical
nature of the proofs and the lack of a computer to verify the details.
However, even today {\em Principia Mathematica} stands out as the work closest
in spirit to Metamath.  It remains a mind-boggling work, and one can't help
but be amazed at seeing ``$1+1=2$'' finally appear on page 83 of Volume II
(Theorem *110.643).

The origin of the proof notation used by Metamath dates back to the 1950's,
when the logician C.~A.~Meredith expressed his proofs in a compact notation
called ``condensed detachment''\index{condensed detachment}
\cite{Hindley}\index{Hindley, J. Roger} \cite{Kalman}\index{Kalman, J. A.}
\cite{Meredith}\index{Meredith, C. A.} \cite{Peterson}\index{Peterson, Jeremy
George}.  This notation allows proofs to be communicated unambiguously by
merely referencing the axiom\index{axiom}, rule\index{rule}, or
theorem\index{theorem} used at each step, without explicitly indicating the
substitutions\index{substitution!variable}\index{variable substitution} that
have to be made to the variables in that axiom, rule, or theorem.  Ordinarily,
condensed detachment is more or less limited to propositional
calculus\index{propositional calculus}.  The concept has been extended to
first-order logic\index{first-order logic} in \cite{Megill}\index{Megill,
Norman}, making it is easy to write a small computer program to verify proofs
of simple first-order logic theorems.\index{condensed detachment!and
first-order logic}

A key concept behind the notation of condensed detachment is called
``unification,''\index{unification} which is an algorithm for determining what
substitutions\index{substitution!variable}\index{variable substitution} to
variables have to be made to make two expressions match each other.
Unification was first precisely defined by the logician J.~A.~Robinson, who
used it in the development of a powerful
theorem-proving technique called the ``resolution principle''
\cite{Robinson}\index{Robinson's resolution principle}. Metamath does not make
use of the resolution principle, which is intended for systems of first-order
logic.\index{first-order logic}  Metamath's use is not restricted to
first-order logic, and as we have mentioned it does not automatically discover
proofs.  However, unification is a key idea behind Metamath's proof
notation, and Metamath makes use of a very simple version of it
(Section~\ref{unify}).

\subsection{Metamath and First-Order Logic}

First-order logic\index{first-order logic} is the supporting structure
for standard mathematics.  On top of it is set theory, which contains
the axioms from which virtually all of mathematics can be derived---a
remarkable fact.\index{category
theory}\index{cardinal, inaccessible}\label{categoryth}\footnote{An exception seems
to be category theory.  There are several schools of thought on whether
category theory is derivable from set theory.  At a minimum, it appears
that an additional axiom is needed that asserts the existence of an
``inaccessible cardinal'' (a type of infinity so large that standard set
theory can't prove or deny that it exists).
%
%%%% (I took this out that was in previous editions:)
% But it is also argued that not just one but a ``proper class'' of them
% is needed, and the existence of proper classes is impossible in standard
% set theory.  (A proper class is a collection of sets so huge that no set
% can contain it as an element.  Proper classes can lead to
% inconsistencies such as ``Russell's paradox.''  The axioms of standard
% set theory are devised so as to deny the existence of proper classes.)
%
For more information, see
\cite[pp.~328--331]{Herrlich}\index{Herrlich, Horst} and
\cite{Blass}\index{Blass, Andrea}.}

One of the things that makes Metamath\index{Metamath} more practical for
first-order theories is a set of axioms for first-order logic designed
specifically with Metamath's approach in mind.  These are included in
the database \texttt{set.mm}\index{set theory database (\texttt{set.mm})}.
See Chapter~\ref{fol} for a detailed
description; the axioms are shown in Section~\ref{metaaxioms}.  While
logically equivalent to standard axiom systems, our axiom system breaks
up the standard axioms into smaller pieces such that from them, you can
directly derive what in other systems can only be derived as higher-level
``metatheorems.''\index{metatheorem}  In other words, it is more powerful than
the standard axioms from a metalogical point of view.  A rigorous
justification for this system and its ``metalogical
completeness''\index{metalogical completeness} is found in
\cite{Megill}\index{Megill, Norman}.  The system is closely related to a
system developed by Monk\index{Monk, J. Donald} and Tarski\index{Tarski,
Alfred} in 1965 \cite{Monks}.

For example, the formula $\exists x \, x = y $ (given $y$, there exists some
$x$ equal to it) is a theorem of logic,\footnote{Specifically, it is a theorem
of those systems of logic that assume non-empty domains.  It is not a theorem
of more general systems that include the empty domain\index{empty domain}, in
which nothing exists, period!  Such systems are called ``free
logics.''\index{free logic} For a discussion of these systems, see
\cite{Leblanc}\index{Leblanc, Hugues}.  Since our use for logic is as a basis
for set theory, which has a non-empty domain, it is more convenient (and more
traditional) to use a less general system.  An interesting curiosity is that,
using a free logic as a basis for Zermelo--Fraenkel set
theory\index{Zermelo--Fraenkel set theory} (with the redundant Axiom of the
Null Set omitted),\index{Axiom of the Null Set} we cannot even prove the
existence of a single set without assuming the axiom of infinity!\index{Axiom
of Infinity}} whether or not $x$ and $y$ are distinct variables\index{distinct
variables}.  In many systems of logic, we would have to prove two theorems to
arrive at this result.  First we would prove ``$\exists x \, x = x $,'' then
we would separately prove ``$\exists x \, x = y $, where $x$ and $y$ are
distinct variables.''  We would then combine these two special cases ``outside
of the system'' (i.e.\ in our heads) to be able to claim, ``$\exists x \, x =
y $, regardless of whether $x$ and $y$ are distinct.''  In other words, the
combination of the two special cases is a
metatheorem.  In the system of logic
used in Metamath's set theory\index{set theory database (\texttt{set.mm})}
database, the axioms of logic are broken down into small pieces that allow
them to be reassembled in such a way that theorems such as these can be proved
directly.

Breaking down the axioms in this way makes them look peculiar and not very
intuitive at first, but rest assured that they are correct and complete.  Their
correctness is ensured because they are theorem schemes of standard first-order
logic (which you can easily verify if you are a logician).  Their completeness
follows from the fact that we explicitly derive the standard axioms of
first-order logic as theorems.  Deriving the standard axioms is somewhat
tricky, but once we're there, we have at our disposal a system that is less
awkward to work with in formal proofs\index{formal proof}.  In technical terms
that logicians understand, we eliminate the cumbersome concepts of ``free
variable,''\index{free variable} ``bound variable,''\index{bound variable} and
``proper substitution''\index{proper substitution}\index{substitution!proper}
as primitive notions.  These concepts are present in our system but are
defined in terms of concepts expressed by the axioms and can be eliminated in
principle.  In standard systems, these concepts are really like additional,
implicit axioms\index{implicit axiom} that are somewhat complex and cannot be
eliminated.

The traditional approach to logic, wherein free variables and proper
substitution is defined, is also possible to do directly in the Metamath
language.  However, the notation tends to become awkward, and there are
disadvantages:  for example, extending the definition of a wff with a
definition is awkward, because the free variable and proper substitution
concepts have to have their definitions also extended.  Our choice of
axioms for \texttt{set.mm} is to a certain extent a matter of style, in
an attempt to achieve overall simplicity, but you should also be aware
that the traditional approach is possible as well if you should choose
it.

\chapter{Using the Metamath Program}
\label{using}

\section{Installation}

The way that you install Metamath\index{Metamath!installation} on your
computer system will vary for different computers.  Current
instructions are provided with the Metamath program download at
\url{http://metamath.org}.  In general, the installation is simple.
There is one file containing the Metamath program itself.  This file is
usually called \texttt{metamath} or \texttt{metamath.}{\em xxx} where
{\em xxx} is the convention (such as \texttt{exe}) for an executable
program on your operating system.  There are several additional files
containing samples of the Metamath language, all ending with
\texttt{.mm}.  The file \texttt{set.mm}\index{set theory database
(\texttt{set.mm})} contains logic and set theory and can be used as a
starting point for other areas of mathematics.

You will also need a text editor\index{text editor} capable of editing plain
{\sc ascii}\footnote{American Standard Code for Information Interchange.} text
in order to prepare your input files.\index{ascii@{\sc ascii}}  Most computers
have this capability built in.  Note that plain text is not necessarily the
default for some word processing programs\index{word processor}, especially if
they can handle different fonts; for example, with Microsoft Word\index{Word
(Microsoft)}, you must save the file in the format ``Text Only With Line
Breaks'' to get a plain text\index{plain text} file.\footnote{It is
recommended that all lines in a Metamath source file be 79 characters or less
in length for compatibility among different computer terminals.  When creating
a source file on an editor such as Word, select a monospaced
font\index{monospaced font} such as Courier\index{Courier font} or
Monaco\index{Monaco font} to make this easier to achieve.  Better yet,
just use a plain text editor such as Notepad.}

On some computer systems, Metamath does not have the capability to print
its output directly; instead, you send its output to a file (using the
\texttt{open} commands described later).  The way you print this output
file depends on your computer.\index{printers} Some computers have a
print command, whereas with others, you may have to read the file into
an editor and print it from there.

If you want to print your Metamath source files with typeset formulas
containing standard mathematical symbols, you will need the \LaTeX\
typesetting program\index{latex@{\LaTeX}}, which is widely and freely
available for most operating systems.  It runs natively on Unix and
Linux, and can be installed on Windows as part of the free Cygwin
package (\url{http://cygwin.com}).

You can also produce {\sc html}\footnote{HyperText Markup Language.}
web pages.  The {\tt help html} command in the Metamath program will
assist you with this feature.

\section{Your First Formal System}\label{start}
\subsection{From Nothing to Zero}\label{startf}

To give you a feel for what the Metamath\index{Metamath} language looks like,
we will take a look at a very simple example from formal number
theory\index{number theory}.  This example is taken from
Mendelson\index{Mendelson, Elliot} \cite[p. 123]{Mendelson}.\footnote{To keep
the example simple, we have changed the formalism slightly, and what we call
axioms\index{axiom} are strictly speaking theorems\index{theorem} in
\cite{Mendelson}.}  We will look at a small subset of this theory, namely that
part needed for the first number theory theorem proved in \cite{Mendelson}.

First we will look at a standard formal proof\index{formal proof} for the
example we have picked, then we will look at the Metamath version.  If you
have never been exposed to formal proofs, the notation may seem to be such
overkill to express such simple notions that you may wonder if you are missing
something.  You aren't.  The concepts involved are in fact very simple, and a
detailed breakdown in this fashion is necessary to express the proof in a way
that can be verified mechanically.  And as you will see, Metamath breaks the
proof down into even finer pieces so that the mechanical verification process
can be about as simple as possible.

Before we can introduce the axioms\index{axiom} of the theory, we must define
the syntax rules for forming legal expressions\index{syntax rules}
(combinations of symbols) with which those axioms can be used. The number 0 is
a {\bf term}\index{term}; and if $ t$ and $r$ are terms, so is $(t+r)$. Here,
$ t$ and $r$ are ``metavariables''\index{metavariable} ranging over terms; they
themselves do not appear as symbols in an actual term.  Some examples of
actual terms are $(0 + 0)$ and $((0+0)+0)$.  (Note that our theory describes
only the number zero and sums of zeroes.  Of course, not much can be done with
such a trivial theory, but remember that we have picked a very small subset of
complete number theory for our example.  The important thing for you to focus
on is our definitions that describe how symbols are combined to form valid
expressions, and not on the content or meaning of those expressions.) If $ t$
and $r$ are terms, an expression of the form $ t=r$ is a {\bf wff}
(well-formed formula)\index{well-formed formula (wff)}; and if $P$ and $Q$ are
wffs, so is $(P\rightarrow Q)$ (which means ``$P$ implies
$Q$''\index{implication ($\rightarrow$)} or ``if $P$ then $Q$'').
Here $P$ and $Q$ are metavariables ranging over wffs.  Examples of actual
wffs are $0=0$, $(0+0)=0$, $(0=0 \rightarrow (0+0)=0)$, and $(0=0\rightarrow
(0=0\rightarrow 0=(0+0)))$.  (Our notation makes use of more parentheses than
are customary, but the elimination of ambiguity this way simplifies our
example by avoiding the need to define operator precedence\index{operator
precedence}.)

The {\bf axioms}\index{axiom} of our theory are all wffs of the following
form, where $ t$, $r$, and $s$ are any terms:

%Latex p. 92
\renewcommand{\theequation}{A\arabic{equation}}

\begin{equation}
(t=r\rightarrow (t=s\rightarrow r=s))
\end{equation}
\begin{equation}
(t+0)=t
\end{equation}

Note that there are an infinite number of axioms since there are an infinite
number of possible terms.  A1 and A2 are properly called ``axiom
schemes,''\index{axiom scheme} but we will refer to them as ``axioms'' for
brevity.

An axiom is a {\bf theorem}; and if $P$ and $(P\rightarrow Q)$ are theorems
(where $P$ and $Q$ are wffs), then $Q$ is also a theorem.\index{theorem}  The
second part of this definition is called the modus ponens (MP) rule of
inference\index{inference rule}\index{modus ponens}.  It allows us to obtain
new theorems from old ones.

The {\bf proof}\index{proof} of a theorem is a sequence of one or more
theorems, each of which is either an axiom or the result of modus ponens
applied to two previous theorems in the sequence, and the last of which is the
theorem being proved.

The theorem we will prove for our example is very simple:  $ t=t$.  The proof of
our theorem follows.  Study it carefully until you feel sure you
understand it.\label{zeroproof}

% Use tabu so that lines will wrap automatically as needed.
\begin{tabu} { l X X }
1. & $(t+0)=t$ & (by axiom A2) \\
2. & $(t+0)=t$ & (by axiom A2) \\
3. & $((t+0)=t \rightarrow ((t+0)=t\rightarrow t=t))$ & (by axiom A1) \\
4. & $((t+0)=t\rightarrow t=t)$ & (by MP applied to steps 2 and 3) \\
5. & $t=t$ & (by MP applied to steps 1 and 4) \\
\end{tabu}

(You may wonder why step 1 is repeated twice.  This is not necessary in the
formal language we have defined, but in Metamath's ``reverse Polish
notation''\index{reverse Polish notation (RPN)} for proofs, a previous step
can be referred to only once.  The repetition of step~1 here will enable you
to see more clearly the correspondence of this proof with the
Metamath\index{Metamath} version on p.~\pageref{demoproof}.)

Our theorem is more properly called a ``theorem scheme,''\index{theorem
scheme} for it represents an infinite number of theorems, one for each
possible term $ t$.  Two examples of actual theorems would be $0=0$ and
$(0+0)=(0+0)$.  Rarely do we prove actual theorems, since by proving schemes
we can prove an infinite number of theorems in one fell swoop.  Similarly, our
proof should really be called a ``proof scheme.''\index{proof scheme}  To
obtain an actual proof, pick an actual term to use in place of $ t$, and
substitute it for $ t$ throughout the proof.

Let's discuss what we have done here.  The axioms\index{axiom} of our theory,
A1 and A2, are trivial and obvious.  Everyone knows that adding zero to
something doesn't change it, and also that if two things are equal to a third,
then they are equal to each other. In fact, stating the trivial and obvious is
a goal to strive for in any axiomatic system.  From trivial and obvious truths
that everyone agrees upon, we can prove results that are not so obvious yet
have absolute faith in them.  If we trust the axioms and the rules, we must,
by definition, trust the consequences of those axioms and rules, if logic is
to mean anything at all.

Our rule of inference\index{rule}, modus ponens\index{modus ponens}, is also
pretty obvious once you understand what it means.  If we prove a fact $P$, and
we also prove that $P$ implies $Q$, then $Q$ necessarily follows as a new
fact.  The rule provides us with a means for obtaining new facts (i.e.\
theorems\index{theorem}) from old ones.

The theorem that we have proved, $ t=t$, is so fundamental that you may wonder
why it isn't one of the axioms\index{axiom}.  In some axiom systems of
arithmetic, it {\em is} an axiom.  The choice of axioms in a theory is to some
extent arbitrary and even an art form, constrained only by the requirement
that any two equivalent axiom systems be able to derive each other as
theorems.  We could imagine that the inventor of our axiom system originally
included $ t=t$ as an axiom, then discovered that it could be derived as a
theorem from the other axioms.  Because of this, it was not necessary to
keep it as an axiom.  By eliminating it, the final set of axioms became
that much simpler.

Unless you have worked with formal proofs\index{formal proof} before, it
probably wasn't apparent to you that $ t=t$ could be derived from our two
axioms until you saw the proof. While you certainly believe that $ t=t$ is
true, you might not be able to convince an imaginary skeptic who believes only
in our two axioms until you produce the proof.  Formal proofs such as this are
hard to come up with when you first start working with them, but after you get
used to them they can become interesting and fun.  Once you understand the
idea behind formal proofs you will have grasped the fundamental principle that
underlies all of mathematics.  As the mathematics becomes more sophisticated,
its proofs become more challenging, but ultimately they all can be broken down
into individual steps as simple as the ones in our proof above.

Mendelson's\index{Mendelson, Elliot} book, from which our example was taken,
contains a number of detailed formal proofs such as these, and you may be
interested in looking it up.  The book is intended for mathematicians,
however, and most of it is rather advanced.  Popular literature describing
formal proofs\index{formal proof} include \cite[p.~296]{Rucker}\index{Rucker,
Rudy} and \cite[pp.~204--230]{Hofstadter}\index{Hofstadter, Douglas R.}.

\subsection{Converting It to Metamath}\label{convert}

Formal proofs\index{formal proof} such as the one in our example break down
logical reasoning into small, precise steps that leave little doubt that the
results follow from the axioms\index{axiom}.  You might think that this would
be the finest breakdown we can achieve in mathematics.  However, there is more
to the proof than meets the eye. Although our axioms were rather simple, a lot
of verbiage was needed before we could even state them:  we needed to define
``term,'' ``wff,'' and so on.  In addition, there are a number of implied
rules that we haven't even mentioned. For example, how do we know that step 3
of our proof follows from axiom A1? There is some hidden reasoning involved in
determining this.  Axiom A1 has two occurrences of the letter $ t$.  One of
the implied rules states that whatever we substitute for $ t$ must be a legal
term\index{term}.\footnote{Some authors make this implied rule explicit by
stating, ``only expressions of the above form are terms,'' after defining
``term.''}  The expression $ t+0$ is pretty obviously a legal term whenever $
t$ is, but suppose we wanted to substitute a huge term with thousands of
symbols?  Certainly a lot of work would be involved in determining that it
really is a term, but in ordinary formal proofs all of this work would be
considered a single ``step.''

To express our axiom system in the Metamath\index{Metamath} language, we must
describe this auxiliary information in addition to the axioms themselves.
Metamath does not know what a ``term'' or a ``wff''\index{well-formed formula
(wff)} is.  In Metamath, the specification of the ways in which we can combine
symbols to obtain terms and wffs are like little axioms in themselves.  These
auxiliary axioms are expressed in the same notation as the ``real''
axioms\index{axiom}, and Metamath does not distinguish between the two.  The
distinction is made by you, i.e.\ by the way in which you interpret the
notation you have chosen to express these two kinds of axioms.

The Metamath language breaks down mathematical proofs into tiny pieces, much
more so than in ordinary formal proofs\index{formal proof}.  If a single
step\index{proof step} involves the
substitution\index{substitution!variable}\index{variable substitution} of a
complex term for one of its variables, Metamath must see this single step
broken down into many small steps.  This fine-grained breakdown is what gives
Metamath generality and flexibility as it lets it not be limited to any
particular mathematical notation.

Metamath's proof notation is not, in itself, intended to be read by humans but
rather is in a compact format intended for a machine.  The Metamath program
will convert this notation to a form you can understand, using the \texttt{show
proof}\index{\texttt{show proof} command} command.  You can tell the program what
level of detail of the proof you want to look at.  You may want to look at
just the logical inference steps that correspond
to ordinary formal proof steps,
or you may want to see the fine-grained steps that prove that an expression is
a term.

Here, without further ado, is our example converted to the
Metamath\index{Metamath} language:\index{metavariable}\label{demo0}

\begin{verbatim}
$( Declare the constant symbols we will use $)
    $c 0 + = -> ( ) term wff |- $.
$( Declare the metavariables we will use $)
    $v t r s P Q $.
$( Specify properties of the metavariables $)
    tt $f term t $.
    tr $f term r $.
    ts $f term s $.
    wp $f wff P $.
    wq $f wff Q $.
$( Define "term" and "wff" $)
    tze $a term 0 $.
    tpl $a term ( t + r ) $.
    weq $a wff t = r $.
    wim $a wff ( P -> Q ) $.
$( State the axioms $)
    a1 $a |- ( t = r -> ( t = s -> r = s ) ) $.
    a2 $a |- ( t + 0 ) = t $.
$( Define the modus ponens inference rule $)
    ${
       min $e |- P $.
       maj $e |- ( P -> Q ) $.
       mp  $a |- Q $.
    $}
$( Prove a theorem $)
    th1 $p |- t = t $=
  $( Here is its proof: $)
       tt tze tpl tt weq tt tt weq tt a2 tt tze tpl
       tt weq tt tze tpl tt weq tt tt weq wim tt a2
       tt tze tpl tt tt a1 mp mp
     $.
\end{verbatim}\index{metavariable}

A ``database''\index{database} is a set of one or more {\sc ascii} source
files.  Here's a brief description of this Metamath\index{Metamath} database
(which consists of this single source file), so that you can understand in
general terms what is going on.  To understand the source file in detail, you
should read Chapter~\ref{languagespec}.

The database is a sequence of ``tokens,''\index{token} which are normally
separated by spaces or line breaks.  The only tokens that are built into
the Metamath language are those beginning with \texttt{\$}.  These tokens
are called ``keywords.''\index{keyword}  All other tokens are
user-defined, and their names are arbitrary.

As you might have guessed, the Metamath token \texttt{\$(}\index{\texttt{\$(} and
\texttt{\$)} auxiliary keywords} starts a comment and \texttt{\$)} ends a comment.

The Metamath tokens \texttt{\$c}\index{\texttt{\$c} statement},
\texttt{\$v}\index{\texttt{\$v} statement},
\texttt{\$e}\index{\texttt{\$e} statement},
\texttt{\$f}\index{\texttt{\$f} statement},
\texttt{\$a}\index{\texttt{\$a} statement}, and
\texttt{\$p}\index{\texttt{\$p} statement} specify ``statements'' that
end with \texttt{\$.}\,.\index{\texttt{\$.}\ keyword}

The Metamath tokens \texttt{\$c} and \texttt{\$v} each declare\index{constant
declaration}\index{variable declaration} a list of user-defined tokens, called
``math symbols,''\index{math symbol} that the database will reference later
on.  All of the math symbols they define you have seen earlier except the
turnstile symbol \texttt{|-} ($\vdash$)\index{turnstile ({$\,\vdash$})}, which is
commonly used by logicians to mean ``a proof exists for.''  For us
the turnstile is just a
convenient symbol that distinguishes expressions that are axioms\index{axiom}
or theorems\index{theorem} from expressions that are terms or wffs.

The \texttt{\$c} statement declares ``constants''\index{constant} and
the \texttt{\$v} statement declares
``variables''\index{variable}\index{constant declaration}\index{variable
declaration} (or more precisely, metavariables\index{metavariable}).  A
variable may be substituted\index{substitution!variable}\index{variable
substitution} with sequences of math symbols whereas a constant may not
be substituted with anything.

It may seem redundant to require both \texttt{\$c}\index{\texttt{\$c} statement} and
\texttt{\$v}\index{\texttt{\$v} statement} statements (since any math
symbol\index{math symbol} not specified with a \texttt{\$c} statement could be
presumed to be a variable), but this provides for better error checking and
also allows math symbols to be redeclared\index{redeclaration of symbols}
(Section~\ref{scoping}).

The token \texttt{\$f}\index{\texttt{\$f} statement} specifies a
statement called a ``variable-type hypothesis'' (also called a
``floating hypothesis'') and \texttt{\$e}\index{\texttt{\$e} statement}
specifies a ``logical hypothesis'' (also called an ``essential
hypothesis'').\index{hypothesis}\index{variable-type
hypothesis}\index{logical hypothesis}\index{floating
hypothesis}\index{essential hypothesis} The token
\texttt{\$a}\index{\texttt{\$a} statement} specifies an ``axiomatic
assertion,''\index{axiomatic assertion} and
\texttt{\$p}\index{\texttt{\$p} statement} specifies a ``provable
assertion.''\index{provable assertion} To the left of each occurrence of
these four tokens is a ``label''\index{label} that identifies the
hypothesis or assertion for later reference.  For example, the label of
the first axiomatic assertion is \texttt{tze}.  A \texttt{\$f} statement
must contain exactly two math symbols, a constant followed by a
variable.  The \texttt{\$e}, \texttt{\$a}, and \texttt{\$p} statements
each start with a constant followed by, in general, an arbitrary
sequence of math symbols.

Associated with each assertion\index{assertion} is a set of hypotheses
that must be satisfied in order for the assertion to be used in a proof.
These are called the ``mandatory hypotheses''\index{mandatory
hypothesis} of the assertion.  Among those hypotheses whose ``scope''
(described below) includes the assertion, \texttt{\$e} hypotheses are
always mandatory and \texttt{\$f}\index{\texttt{\$f} statement}
hypotheses are mandatory when they share their variable with the
assertion or its \texttt{\$e} hypotheses.  The exact rules for
determining which hypotheses are mandatory are described in detail in
Sections~\ref{frames} and \ref{scoping}.  For example, the mandatory
hypotheses of assertion \texttt{tpl} are \texttt{tt} and \texttt{tr},
whereas assertion \texttt{tze} has no mandatory hypotheses because it
contains no variables and has no \texttt{\$e}\index{\texttt{\$e}
statement} hypothesis.  Metamath's \texttt{show statement}
command\index{\texttt{show statement} command}, described in the next
section, will show you a statement's mandatory hypotheses.

Sometimes we need to make a hypothesis relevant to only certain
assertions.  The set of statements to which a hypothesis is relevant is
called its ``scope.''  The Metamath brackets,
\texttt{\$\char`\{}\index{\texttt{\$\char`\{} and \texttt{\$\char`\}}
keywords} and \texttt{\$\char`\}}, define a ``block''\index{block} that
delimits the scope of any hypothesis contained between them.  The
assertion \texttt{mp} has mandatory hypotheses \texttt{wp}, \texttt{wq},
\texttt{min}, and \texttt{maj}.  The only mandatory hypothesis of
\texttt{th1}, on the other hand, is \texttt{tt}, since \texttt{th1}
occurs outside of the block containing \texttt{min} and \texttt{maj}.

Note that \texttt{\$\char`\{} and \texttt{\$\char`\}} do not affect the
scope of assertions (\texttt{\$a} and \texttt{\$p}).  Assertions are always
available to be referenced by any later proof in the source file.

Each provable assertion (\texttt{\$p}\index{\texttt{\$p} statement}
statement) has two parts.  The first part is the
assertion\index{assertion} itself, which is a sequence of math
symbol\index{math symbol} tokens placed between the \texttt{\$p} token
and a \texttt{\$=}\index{\texttt{\$=} keyword} token.  The second part
is a ``proof,'' which is a list of label tokens placed between the
\texttt{\$=} token and the \texttt{\$.}\index{\texttt{\$.}\ keyword}\
token that ends the statement.\footnote{If you've looked at the
\texttt{set.mm} database, you may have noticed another notation used for
proofs.  The other notation is called ``compressed.''\index{compressed
proof}\index{proof!compressed} It reduces the amount of space needed to
store a proof in the database and is described in
Appendix~\ref{compressed}.  In the example above, we use
``normal''\index{normal proof}\index{proof!normal} notation.} The proof
acts as a series of instructions to the Metamath program, telling it how
to build up the sequence of math symbols contained in the assertion part of
the \texttt{\$p} statement, making use of the hypotheses of the
\texttt{\$p} statement and previous assertions.  The construction takes
place according to precise rules.  If the list of labels in the proof
causes these rules to be violated, or if the final sequence that results
does not match the assertion, the Metamath program will notify you with
an error message.

If you are familiar with reverse Polish notation (RPN), which is sometimes used
on pocket calculators, here in a nutshell is how a proof works.  Each
hypothesis label\index{hypothesis label} in the proof is pushed\index{push}
onto the RPN stack\index{stack}\index{RPN stack} as it is encountered. Each
assertion label\index{assertion label} pops\index{pop} off the stack as many
entries as the referenced assertion has mandatory hypotheses.  Variable
substitutions\index{substitution!variable}\index{variable substitution} are
computed which, when made to the referenced assertion's mandatory hypotheses,
cause these hypotheses to match the stack entries. These same substitutions
are then made to the variables in the referenced assertion itself, which is
then pushed onto the stack.  At the end of the proof, there should be one
stack entry, namely the assertion being proved.  This process is explained in
detail in Section~\ref{proof}.

Metamath's proof notation is not very readable for humans, but it allows the
proof to be stored compactly in a file.  The Metamath\index{Metamath} program
has proof display features that let you see what's going on in a more
readable way, as you will see in the next section.

The rules used in verifying a proof are not based on any built-in syntax of the
symbol sequence in an assertion\index{assertion} nor on any built-in meanings
attached to specific symbol names.  They are based strictly on symbol
matching:  constants\index{constant} must match themselves, and
variables\index{variable} may be replaced with anything that allows a match to
occur.  For example, instead of \texttt{term}, \texttt{0}, and \verb$|-$ we could
have just as well used \texttt{yellow}, \texttt{zero}, and \texttt{provable}, as long
as we did so consistently throughout the database.  Also, we could have used
\texttt{is provable} (two tokens) instead of \verb$|-$ (one token) throughout the
database.  In each of these cases, the proof would be exactly the same.  The
independence of proofs and notation means that you have a lot of flexibility to
change the notation you use without having to change any proofs.

\section{A Trial Run}\label{trialrun}

Now you are ready to try out the Metamath\index{Metamath} program.

On all computer systems, Metamath has a standard ``command line
interface'' (CLI)\index{command line interface (CLI)} that allows you to
interact with it.  You supply commands to the CLI by typing them on the
keyboard and pressing your keyboard's {\em return} key after each line
you enter.  The CLI is designed to be easy to use and has built-in help
features.

The first thing you should do is to use a text editor to create a file
called \texttt{demo0.mm} and type into it the Metamath source shown on
p.~\pageref{demo0}.  Actually, this file is included with your Metamath
software package, so check that first.  If you type it in, make sure
that you save it in the form of ``plain {\sc ascii} text with line
breaks.''  Most word processors will have this feature.

Next you must run the Metamath program.  Depending on your computer
system and how Metamath is installed, this could range from clicking the
mouse on the Metamath icon to typing \texttt{run metamath} to typing
simply \texttt{metamath}.  (Metamath's {\tt help invoke} command describes
alternate ways of invoking the Metamath program.)

When you first enter Metamath\index{Metamath}, it will be at the CLI, waiting
for your input. You will see something like the following on your screen:
\begin{verbatim}
Metamath - Version 0.172 25-Jan-2019
Type HELP for help, EXIT to exit.
MM>
\end{verbatim}
The \texttt{MM>} prompt means that Metamath is waiting for a command.
Command keywords\index{command keyword} are not case sensitive;
we will use lower-case commands in our examples.
The version number and its release date will probably be different on your
system from the one we show above.

The first thing that you need to do is to read in your
database:\index{\texttt{read} command}\footnote{If a directory path is
needed on Unix,\index{Unix file names}\index{file names!Unix} you should
enclose the path/file name in quotes to prevent Metamath from thinking
that the \texttt{/} in the path name is a command qualifier, e.g.,
\texttt{read \char`\"db/set.mm\char`\"}.  Quotes are optional when there
is no ambiguity.}
\begin{verbatim}
MM> read demo0.mm
\end{verbatim}
Remember to press the {\em return} key after entering this command.  If
you omit the file name, Metamath will prompt you for one.   The syntax for
specifying a Macintosh file name path is given in a footnote on
p.~\pageref{includef}.\index{Macintosh file names}\index{file
names!Macintosh}

If there are any syntax errors in the database, Metamath will let you know
when it reads in the file.  The one thing that Metamath does not check when
reading in a database is that all proofs are correct, because this would
slow it down too much.  It is a good idea to periodically verify the proofs in
a database you are making changes to.  To do this, use the following command
(and do it for your \texttt{demo0.mm} file now).  Note that the \texttt{*} is a
``wild card'' meaning all proofs in the file.\index{\texttt{verify proof} command}
\begin{verbatim}
MM> verify proof *
\end{verbatim}
Metamath will report any proofs that are incorrect.

It is often useful to save the information that the Metamath program displays
on the screen. You can save everything that happens on the screen by opening a
log file. You may want to do this before you read in a database so that you
can examine any errors later on.  To open a log file, type
\begin{verbatim}
MM> open log abc.log
\end{verbatim}
This will open a file called \texttt{abc.log}, and everything that appears on the
screen from this point on will be stored in this file.  The name of the log file
is arbitrary. To close the log file, type
\begin{verbatim}
MM> close log
\end{verbatim}

Several commands let you examine what's inside your database.
Section~\ref{exploring} has an overview of some useful ones.  The
\texttt{show labels} command lets you see what statement
labels\index{label} exist.  A \texttt{*} matches any combination of
characters, and \texttt{t*} refers to all labels starting with the
letter \texttt{t}.\index{\texttt{show labels} command} The \texttt{/all}
is a ``command qualifier''\index{command qualifier} that tells Metamath
to include labels of hypotheses.  (To see the syntax explained, type
\texttt{help show labels}.)  Type
\begin{verbatim}
MM> show labels t* /all
\end{verbatim}
Metamath will respond with
\begin{verbatim}
The statement number, label, and type are shown.
3 tt $f       4 tr $f       5 ts $f       8 tze $a
9 tpl $a      19 th1 $p
\end{verbatim}

You can use the \texttt{show statement} command to get information about a
particular statement.\index{\texttt{show statement} command}
For example, you can get information about the statement with label \texttt{mp}
by typing
\begin{verbatim}
MM> show statement mp /full
\end{verbatim}
Metamath will respond with
\begin{verbatim}
Statement 17 is located on line 43 of the file
"demo0.mm".
"Define the modus ponens inference rule"
17 mp $a |- Q $.
Its mandatory hypotheses in RPN order are:
  wp $f wff P $.
  wq $f wff Q $.
  min $e |- P $.
  maj $e |- ( P -> Q ) $.
The statement and its hypotheses require the
      variables:  Q P
The variables it contains are:  Q P
\end{verbatim}
The mandatory hypotheses\index{mandatory hypothesis} and their
order\index{RPN order} are
useful to know when you are trying to understand or debug a proof.

Now you are ready to look at what's really inside our proof.  First, here is
how to look at every step in the proof---not just the ones corresponding to an
ordinary formal proof\index{formal proof}, but also the ones that build up the
formulas that appear in each ordinary formal proof step.\index{\texttt{show
proof} command}
\begin{verbatim}
MM> show proof th1 /lemmon /all
\end{verbatim}

This will display the proof on the screen in the following format:
\begin{verbatim}
 1 tt            $f term t
 2 tze           $a term 0
 3 1,2 tpl       $a term ( t + 0 )
 4 tt            $f term t
 5 3,4 weq       $a wff ( t + 0 ) = t
 6 tt            $f term t
 7 tt            $f term t
 8 6,7 weq       $a wff t = t
 9 tt            $f term t
10 9 a2          $a |- ( t + 0 ) = t
11 tt            $f term t
12 tze           $a term 0
13 11,12 tpl     $a term ( t + 0 )
14 tt            $f term t
15 13,14 weq     $a wff ( t + 0 ) = t
16 tt            $f term t
17 tze           $a term 0
18 16,17 tpl     $a term ( t + 0 )
19 tt            $f term t
20 18,19 weq     $a wff ( t + 0 ) = t
21 tt            $f term t
22 tt            $f term t
23 21,22 weq     $a wff t = t
24 20,23 wim     $a wff ( ( t + 0 ) = t -> t = t )
25 tt            $f term t
26 25 a2         $a |- ( t + 0 ) = t
27 tt            $f term t
28 tze           $a term 0
29 27,28 tpl     $a term ( t + 0 )
30 tt            $f term t
31 tt            $f term t
32 29,30,31 a1   $a |- ( ( t + 0 ) = t -> ( ( t + 0 )
                                     = t -> t = t ) )
33 15,24,26,32 mp  $a |- ( ( t + 0 ) = t -> t = t )
34 5,8,10,33 mp  $a |- t = t
\end{verbatim}

The \texttt{/lemmon} command qualifier specifies what is known as a Lemmon-style
display\index{Lemmon-style proof}\index{proof!Lemmon-style}.  Omitting the
\texttt{/lemmon} qualifier results in a tree-style proof (see
p.~\pageref{treeproof} for an example) that is somewhat less explicit but
easier to follow once you get used to it.\index{tree-style
proof}\index{proof!tree-style}

The first number on each line is the step
number of the proof.  Any numbers that follow are step numbers assigned to the
hypotheses of the statement referenced by that step.  Next is the label of
the statement referenced by the step.  The statement type of the statement
referenced comes next, followed by the math symbol\index{math symbol} string
constructed by the proof up to that step.

The last step, 34, contains the statement that is being proved.

Looking at a small piece of the proof, notice that steps 3 and 4 have
established that
\texttt{( t + 0 )} and \texttt{t} are \texttt{term}\,s, and step 5 makes use of steps 3 and
4 to establish that \texttt{( t + 0 ) = t} is a \texttt{wff}.  Let Metamath
itself tell us in detail what is happening in step 5.  Note that the
``target hypothesis'' refers to where step 5 is eventually used, i.e., in step
34.
\begin{verbatim}
MM> show proof th1 /detailed_step 5
Proof step 5:  wp=weq $a wff ( t + 0 ) = t
This step assigns source "weq" ($a) to target "wp"
($f).  The source assertion requires the hypotheses
"tt" ($f, step 3) and "tr" ($f, step 4).  The parent
assertion of the target hypothesis is "mp" ($a,
step 34).
The source assertion before substitution was:
    weq $a wff t = r
The following substitutions were made to the source
assertion:
    Variable  Substituted with
     t         ( t + 0 )
     r         t
The target hypothesis before substitution was:
    wp $f wff P
The following substitution was made to the target
hypothesis:
    Variable  Substituted with
     P         ( t + 0 ) = t
\end{verbatim}

The full proof just shown is useful to understand what is going on in detail.
However, most of the time you will just be interested in
the ``essential'' or logical steps of a proof, i.e.\ those steps
that correspond to an
ordinary formal proof\index{formal proof}.  If you type
\begin{verbatim}
MM> show proof th1 /lemmon /renumber
\end{verbatim}
you will see\label{demoproof}
\begin{verbatim}
1 a2             $a |- ( t + 0 ) = t
2 a2             $a |- ( t + 0 ) = t
3 a1             $a |- ( ( t + 0 ) = t -> ( ( t + 0 )
                                     = t -> t = t ) )
4 2,3 mp         $a |- ( ( t + 0 ) = t -> t = t )
5 1,4 mp         $a |- t = t
\end{verbatim}
Compare this to the formal proof on p.~\pageref{zeroproof} and
notice the resemblance.
By default Metamath
does not show \texttt{\$f}\index{\texttt{\$f}
statement} hypotheses and everything branching off of them in the proof tree
when the proof is displayed; this makes the proof look more like an ordinary
mathematical proof, which does not normally incorporate the explicit
construction of expressions.
This is called the ``essential'' view
(at one time you had to add the
\texttt{/essential} qualifier in the \texttt{show proof}
command to get this view, but this is now the default).
You can could use the \texttt{/all} qualifier in the \texttt{show
proof} command to also show the explicit construction of expressions.
The \texttt{/renumber} qualifier means to renumber
the steps to correspond only to what is displayed.\index{\texttt{show proof}
command}

To exit Metamath, type\index{\texttt{exit} command}
\begin{verbatim}
MM> exit
\end{verbatim}

\subsection{Some Hints for Using the Command Line Interface}

We will conclude this quick introduction to Metamath\index{Metamath} with some
helpful hints on how to navigate your way through the commands.
\index{command line interface (CLI)}

When you type commands into Metamath's CLI, you only have to type as many
characters of a command keyword\index{command keyword} as are needed to make
it unambiguous.  If you type too few characters, Metamath will tell you what
the choices are.  In the case of the \texttt{read} command, only the \texttt{r} is
needed to specify it unambiguously, so you could have typed\index{\texttt{read}
command}
\begin{verbatim}
MM> r demo0.mm
\end{verbatim}
instead of
\begin{verbatim}
MM> read demo0.mm
\end{verbatim}
In our description, we always show the full command words.  When using the
Metamath CLI commands in a command file (to be read with the \texttt{submit}
command)\index{\texttt{submit} command}, it is good practice to use
the unabbreviated command to ensure your instructions will not become ambiguous
if more commands are added to the Metamath program in the future.

The command keywords\index{command
keyword} are not case sensitive; you may type either \texttt{read} or
\texttt{ReAd}.  File names may or may not be case sensitive, depending on your
computer's operating system.  Metamath label\index{label} and math
symbol\index{math symbol} tokens\index{token} are case-sensitive.

The \texttt{help} command\index{\texttt{help} command} will provide you
with a list of topics you can get help on.  You can then type
\texttt{help} {\em topic} to get help on that topic.

If you are uncertain of a command's spelling, just type as many characters
as you remember of the command.  If you have not typed enough characters to
specify it unambiguously, Metamath will tell you what choices you have.

\begin{verbatim}
MM> show s
         ^
?Ambiguous keyword - please specify SETTINGS,
STATEMENT, or SOURCE.
\end{verbatim}

If you don't know what argument to use as part of a command, type a
\texttt{?}\index{\texttt{]}@\texttt{?}\ in command lines}\ at the
argument position.  Metamath will tell you what it expected there.

\begin{verbatim}
MM> show ?
         ^
?Expected SETTINGS, LABELS, STATEMENT, SOURCE, PROOF,
MEMORY, TRACE_BACK, or USAGE.
\end{verbatim}

Finally, you may type just the first word or words of a command followed
by {\em return}.  Metamath will prompt you for the remaining part of the
command, showing you the choices at each step.  For example, instead of
typing \texttt{show statement th1 /full} you could interact in the
following manner:
\begin{verbatim}
MM> show
SETTINGS, LABELS, STATEMENT, SOURCE, PROOF,
MEMORY, TRACE_BACK, or USAGE <SETTINGS>? st
What is the statement label <th1>?
/ or nothing <nothing>? /
TEX, COMMENT_ONLY, or FULL <TEX>? f
/ or nothing <nothing>?
19 th1 $p |- t = t $= ... $.
\end{verbatim}
After each \texttt{?}\ in this mode, you must give Metamath the
information it requests.  Sometimes Metamath gives you a list of choices
with the default choice indicated by brackets \texttt{< > }. Pressing
{\em return} after the \texttt{?}\ will select the default choice.
Answering anything else will override the default.  Note that the
\texttt{/} in command qualifiers is considered a separate
token\index{token} by the parser, and this is why it is asked for
separately.

\section{Your First Proof}\label{frstprf}

Proofs are developed with the aid of the Proof Assistant\index{Proof
Assistant}.  We will now show you how the proof of theorem \texttt{th1}
was built.  So that you can repeat these steps, we will first have the
Proof Assistant erase the proof in Metamath's source buffer\index{source
buffer}, then reconstruct it.  (The source buffer is the place in memory
where Metamath stores the information in the database when it is
\texttt{read}\index{\texttt{read} command} in.  New or modified proofs
are kept in the source buffer until a \texttt{write source}
command\index{\texttt{write source} command} is issued.)  In practice, you
would place a \texttt{?}\index{\texttt{]}@\texttt{?}\ inside proofs}\
between \texttt{\$=}\index{\texttt{\$=} keyword} and
\texttt{\$.}\index{\texttt{\$.}\ keyword}\ in the database to indicate
to Metamath\index{Metamath} that the proof is unknown, and that would be
your starting point.  Whenever the \texttt{verify proof} command encounters
a proof with a \texttt{?}\ in place of a proof step, the statement is
identified as not proved.

When I first started creating Metamath proofs, I would write down
on a piece of paper the complete
formal proof\index{formal proof} as it would appear
in a \texttt{show proof} command\index{\texttt{show proof} command}; see
the display of \texttt{show proof th1 /lemmon /re\-num\-ber} above as an
example.  After you get used to using the Proof Assistant\index{Proof
Assistant} you may get to a point where you can ``see'' the proof in your mind
and let the Proof Assistant guide you in filling in the details, at least for
simpler proofs, but until you gain that experience you may find it very useful
to write down all the details in advance.
Otherwise you may waste a lot of time as you let it take you down a wrong path.
However, others do not find this approach as helpful.
For example, Thomas Brendan Leahy\index{Leahy, Thomas Brendan}
finds that it is more helpful to him to interactively
work backward from a machine-readable statement.
David A. Wheeler\index{Wheeler, David A.}
writes down a general approach, but develops the proof
interactively by switching between
working forwards (from hypotheses and facts likely to be useful) and
backwards (from the goal) until the forwards and backwards approaches meet.
In the end, use whatever approach works for you.

A proof is developed with the Proof Assistant by working backwards, starting
with the theorem\index{theorem} to be proved, and assigning each unknown step
with a theorem or hypothesis until no more unknown steps remain.  The Proof
Assistant will not let you make an assignment unless it can be ``unified''
with the unknown step.  This means that a
substitution\index{substitution!variable}\index{variable substitution} of
variables exists that will make the assignment match the unknown step.  On the
other hand, in the middle of a proof, when working backwards, often more than
one unification\index{unification} (set of substitutions) is possible, since
there is not enough information available at that point to uniquely establish
it.  In this case you can tell Metamath which unification to choose, or you
can continue to assign unknown steps until enough information is available to
make the unification unique.

We will assume you have entered Metamath and read in the database as described
above.  The following dialog shows how the proof was developed.  For more
details on what some of the commands do, refer to Section~\ref{pfcommands}.
\index{\texttt{prove} command}

\begin{verbatim}
MM> prove th1
Entering the Proof Assistant.  Type HELP for help, EXIT
to exit.  You will be working on the proof of statement th1:
  $p |- t = t
Note:  The proof you are starting with is already complete.
MM-PA>
\end{verbatim}

The \verb/MM-PA>/ prompt means we are inside the Proof
Assistant.\index{Proof Assistant} Most of the regular Metamath commands
(\texttt{show statement}, etc.) are still available if you need them.

\begin{verbatim}
MM-PA> delete all
The entire proof was deleted.
\end{verbatim}

We have deleted the whole proof so we can start from scratch.

\begin{verbatim}
MM-PA> show new_proof/lemmon/all
1 ?              $? |- t = t
\end{verbatim}

The \texttt{show new{\char`\_}proof} command\index{\texttt{show
new{\char`\_}proof} command} is like \texttt{show proof} except that we
don't specify a statement; instead, the proof we're working on is
displayed.

\begin{verbatim}
MM-PA> assign 1 mp
To undo the assignment, DELETE STEP 5 and INITIALIZE, UNIFY
if needed.
3   min=?  $? |- $2
4   maj=?  $? |- ( $2 -> t = t )
\end{verbatim}

The \texttt{assign} command\index{\texttt{assign} command} above means
``assign step 1 with the statement whose label is \texttt{mp}.''  Note
that step renumbering will constantly occur as you assign steps in the
middle of a proof; in general all steps from the step you assign until
the end of the proof will get moved up.  In this case, what used to be
step 1 is now step 5, because the (partial) proof now has five steps:
the four hypotheses of the \texttt{mp} statement and the \texttt{mp}
statement itself.  Let's look at all the steps in our partial proof:

\begin{verbatim}
MM-PA> show new_proof/lemmon/all
1 ?              $? wff $2
2 ?              $? wff t = t
3 ?              $? |- $2
4 ?              $? |- ( $2 -> t = t )
5 1,2,3,4 mp     $a |- t = t
\end{verbatim}

The symbol \texttt{\$2} is a temporary variable\index{temporary
variable} that represents a symbol sequence not yet known.  In the final
proof, all temporary variables will be eliminated.  The general format
for a temporary variable is \texttt{\$} followed by an integer.  Note
that \texttt{\$} is not a legal character in a math symbol (see
Section~\ref{dollardollar}, p.~\pageref{dollardollar}), so there will
never be a naming conflict between real symbols and temporary variables.

Unknown steps 1 and 2 are constructions of the two wffs used by the
modus ponens rule.  As you will see at the end of this section, the
Proof Assistant\index{Proof Assistant} can usually figure these steps
out by itself, and we will not have to worry about them.  Therefore from
here on we will display only the ``essential'' hypotheses, i.e.\ those
steps that correspond to traditional formal proofs\index{formal proof}.

\begin{verbatim}
MM-PA> show new_proof/lemmon
3 ?              $? |- $2
4 ?              $? |- ( $2 -> t = t )
5 3,4 mp         $a |- t = t
\end{verbatim}

Unknown steps 3 and 4 are the ones we must focus on.  They correspond to the
minor and major premises of the modus ponens rule.  We will assign them as
follows.  Notice that because of the step renumbering that takes place
after an assignment, it is advantageous to assign unknown steps in reverse
order, because earlier steps will not get renumbered.

\begin{verbatim}
MM-PA> assign 4 mp
To undo the assignment, DELETE STEP 8 and INITIALIZE, UNIFY
if needed.
3   min=?  $? |- $2
6     min=?  $? |- $4
7     maj=?  $? |- ( $4 -> ( $2 -> t = t ) )
\end{verbatim}

We are now going to describe an obscure feature that you will probably
never use but should be aware of.  The Metamath language allows empty
symbol sequences to be substituted for variables, but in most formal
systems this feature is never used.  One of the few examples where is it
used is the MIU-system\index{MIU-system} described in
Appendix~\ref{MIU}.  But such systems are rare, and by default this
feature is turned off in the Proof Assistant.  (It is always allowed for
{\tt verify proof}.)  Let us turn it on and see what
happens.\index{\texttt{set empty{\char`\_}substitution} command}

\begin{verbatim}
MM-PA> set empty_substitution on
Substitutions with empty symbol sequences is now allowed.
\end{verbatim}

With this feature enabled, more unifications will be
ambiguous\index{ambiguous unification}\index{unification!ambiguous} in
the middle of a proof, because
substitution\index{substitution!variable}\index{variable substitution}
of variables with empty symbol sequences will become an additional
possibility.  Let's see what happens when we make our next assignment.

\begin{verbatim}
MM-PA> assign 3 a2
There are 2 possible unifications.  Please select the correct
    one or Q if you want to UNIFY later.
Unify:  |- $6
 with:  |- ( $9 + 0 ) = $9
Unification #1 of 2 (weight = 7):
  Replace "$6" with "( + 0 ) ="
  Replace "$9" with ""
  Accept (A), reject (R), or quit (Q) <A>? r
\end{verbatim}

The first choice presented is the wrong one.  If we had selected it,
temporary variable \texttt{\$6} would have been assigned a truncated
wff, and temporary variable \texttt{\$9} would have been assigned an
empty sequence (which is not allowed in our system).  With this choice,
eventually we would reach a point where we would get stuck because
we would end up with steps impossible to prove.  (You may want to
try it.)  We typed \texttt{r} to reject the choice.

\begin{verbatim}
Unification #2 of 2 (weight = 21):
  Replace "$6" with "( $9 + 0 ) = $9"
  Accept (A), reject (R), or quit (Q) <A>? q
To undo the assignment, DELETE STEP 4 and INITIALIZE, UNIFY
if needed.
 7     min=?  $? |- $8
 8     maj=?  $? |- ( $8 -> ( $6 -> t = t ) )
\end{verbatim}

The second choice is correct, and normally we would type \texttt{a}
to accept it.  But instead we typed \texttt{q} to show what will happen:
it will leave the step with an unknown unification, which can be
seen as follows:

\begin{verbatim}
MM-PA> show new_proof/not_unified
 4   min    $a |- $6
        =a2  = |- ( $9 + 0 ) = $9
\end{verbatim}

Later we can unify this with the \texttt{unify}
\texttt{all/interactive} command.

The important point to remember is that occasionally you will be
presented with several unification choices while entering a proof, when
the program determines that there is not enough information yet to make
an unambiguous choice automatically (and this can happen even with
\texttt{set empty{\char`\_}substitution} turned off).  Usually it is
obvious by inspection which choice is correct, since incorrect ones will
tend to be meaningless fragments of wffs.  In addition, the correct
choice will usually be the first one presented, unlike our example
above.

Enough of this digression.  Let us go back to the default setting.

\begin{verbatim}
MM-PA> set empty_substitution off
The ability to substitute empty expressions for variables
has been turned off.  Note that this may make the Proof
Assistant too restrictive in some cases.
\end{verbatim}

If we delete the proof, start over, and get to the point where
we digressed above, there will no longer be an ambiguous unification.

\begin{verbatim}
MM-PA> assign 3 a2
To undo the assignment, DELETE STEP 4 and INITIALIZE, UNIFY
if needed.
 7     min=?  $? |- $4
 8     maj=?  $? |- ( $4 -> ( ( $5 + 0 ) = $5 -> t = t ) )
\end{verbatim}

Let us look at our proof so far, and continue.

\begin{verbatim}
MM-PA> show new_proof/lemmon
 4 a2            $a |- ( $5 + 0 ) = $5
 7 ?             $? |- $4
 8 ?             $? |- ( $4 -> ( ( $5 + 0 ) = $5 -> t = t ) )
 9 7,8 mp        $a |- ( ( $5 + 0 ) = $5 -> t = t )
10 4,9 mp        $a |- t = t
MM-PA> assign 8 a1
To undo the assignment, DELETE STEP 11 and INITIALIZE, UNIFY
if needed.
 7     min=?  $? |- ( t + 0 ) = t
MM-PA> assign 7 a2
To undo the assignment, DELETE STEP 8 and INITIALIZE, UNIFY
if needed.
MM-PA> show new_proof/lemmon
 4 a2            $a |- ( t + 0 ) = t
 8 a2            $a |- ( t + 0 ) = t
12 a1            $a |- ( ( t + 0 ) = t -> ( ( t + 0 ) = t ->
                                                    t = t ) )
13 8,12 mp       $a |- ( ( t + 0 ) = t -> t = t )
14 4,13 mp       $a |- t = t
\end{verbatim}

Now all temporary variables and unknown steps have been eliminated from the
``essential'' part of the proof.  When this is achieved, the Proof
Assistant\index{Proof Assistant} can usually figure out the rest of the proof
automatically.  (Note that the \texttt{improve} command can occasionally be
useful for filling in essential steps as well, but it only tries to make use
of statements that introduce no new variables in their hypotheses, which is
not the case for \texttt{mp}. Also it will not try to improve steps containing
temporary variables.)  Let's look at the complete proof, then run
the \texttt{improve} command, then look at it again.

\begin{verbatim}
MM-PA> show new_proof/lemmon/all
 1 ?             $? wff ( t + 0 ) = t
 2 ?             $? wff t = t
 3 ?             $? term t
 4 3 a2          $a |- ( t + 0 ) = t
 5 ?             $? wff ( t + 0 ) = t
 6 ?             $? wff ( ( t + 0 ) = t -> t = t )
 7 ?             $? term t
 8 7 a2          $a |- ( t + 0 ) = t
 9 ?             $? term ( t + 0 )
10 ?             $? term t
11 ?             $? term t
12 9,10,11 a1    $a |- ( ( t + 0 ) = t -> ( ( t + 0 ) = t ->
                                                    t = t ) )
13 5,6,8,12 mp   $a |- ( ( t + 0 ) = t -> t = t )
14 1,2,4,13 mp   $a |- t = t
\end{verbatim}

\begin{verbatim}
MM-PA> improve all
A proof of length 1 was found for step 11.
A proof of length 1 was found for step 10.
A proof of length 3 was found for step 9.
A proof of length 1 was found for step 7.
A proof of length 9 was found for step 6.
A proof of length 5 was found for step 5.
A proof of length 1 was found for step 3.
A proof of length 3 was found for step 2.
A proof of length 5 was found for step 1.
Steps 1 and above have been renumbered.
CONGRATULATIONS!  The proof is complete.  Use SAVE
NEW_PROOF to save it.  Note:  The Proof Assistant does
not detect $d violations.  After saving the proof, you
should verify it with VERIFY PROOF.
\end{verbatim}

The \texttt{save new{\char`\_}proof} command\index{\texttt{save
new{\char`\_}proof} command} will save the proof in the database.  Here
we will just display it in a form that can be clipped out of a log file
and inserted manually into the database source file with a text
editor.\index{normal proof}\index{proof!normal}

\begin{verbatim}
MM-PA> show new_proof/normal
---------Clip out the proof below this line:
      tt tze tpl tt weq tt tt weq tt a2 tt tze tpl tt weq
      tt tze tpl tt weq tt tt weq wim tt a2 tt tze tpl tt
      tt a1 mp mp $.
---------The proof of 'th1' to clip out ends above this line.
\end{verbatim}

There is another proof format called ``compressed''\index{compressed
proof}\index{proof!compressed} that you will see in databases.  It is
not important to understand how it is encoded but only to recognize it
when you see it.  Its only purpose is to reduce storage requirements for
large proofs.  A compressed proof can always be converted to a normal
one and vice-versa, and the Metamath \texttt{show proof}
commands\index{\texttt{show proof} command} work equally well with
compressed proofs.  The compressed proof format is described in
Appendix~\ref{compressed}.

\begin{verbatim}
MM-PA> show new_proof/compressed
---------Clip out the proof below this line:
      ( tze tpl weq a2 wim a1 mp ) ABCZADZAADZAEZJJKFLIA
      AGHH $.
---------The proof of 'th1' to clip out ends above this line.
\end{verbatim}

Now we will exit the Proof Assistant.  Since we made changes to the proof,
it will warn us that we have not saved it.  In this case, we don't care.

\begin{verbatim}
MM-PA> exit
Warning:  You have not saved changes to the proof.
Do you want to EXIT anyway (Y, N) <N>? y
Exiting the Proof Assistant.
Type EXIT again to exit Metamath.
\end{verbatim}

The Proof Assistant\index{Proof Assistant} has several other commands
that can help you while creating proofs.  See Section~\ref{pfcommands}
for a list of them.

A command that is often useful is \texttt{minimize{\char`\_}with
*/brief}, which tries to shorten the proof.  It can make the process
more efficient by letting you write a somewhat ``sloppy'' proof then
clean up some of the fine details of optimization for you (although it
can't perform miracles such as restructuring the overall proof).

\section{A Note About Editing a Data\-base File}

Once your source file contains proofs, there are some restrictions on
how you can edit it so that the proofs remain valid.  Pay particular
attention to these rules, since otherwise you can lose a lot of work.
It is a good idea to periodically verify all proofs with \texttt{verify
proof *} to ensure their integrity.

If your file contains only normal (as opposed to compressed) proofs, the
main rule is that you may not change the order of the mandatory
hypotheses\index{mandatory hypothesis} of any statement referenced in a
later proof.  For example, if you swap the order of the major and minor
premise in the modus ponens rule, all proofs making use of that rule
will become incorrect.  The \texttt{show statement}
command\index{\texttt{show statement} command} will show you the
mandatory hypotheses of a statement and their order.

If a statement has a compressed proof, you also must not change the
order of {\em its} mandatory hypotheses.  The compressed proof format
makes use of this information as part of the compression technique.
Note that swapping the names of two variables in a theorem will change
the order of its mandatory hypotheses.

The safest way to edit a statement, say \texttt{mytheorem}, is to
duplicate it then rename the original to \texttt{mytheoremOLD}
throughout the database.  Once the edited version is re-proved, all
statements referencing \texttt{mytheoremOLD} can be updated in the Proof
Assistant using \texttt{minimize{\char`\_}with
mytheorem
/allow{\char`\_}growth}.\index{\texttt{minimize{\char`\_}with} command}
% 3/10/07 Note: line-breaking the above results in duplicate index entries

\chapter{Abstract Mathematics Revealed}\label{fol}

\section{Logic and Set Theory}\label{logicandsettheory}

\begin{quote}
  {\em Set theory can be viewed as a form of exact theology.}
  \flushright\sc  Rudy Rucker\footnote{\cite{Barrow}, p.~31.}\\
\end{quote}\index{Rucker, Rudy}

Despite its seeming complexity, all of standard mathematics, no matter how
deep or abstract, can amazingly enough be derived from a relatively small set
of axioms\index{axiom} or first principles. The development of these axioms is
among the most impressive and important accomplishments of mathematics in the
20th century. Ultimately, these axioms can be broken down into a set of rules
for manipulating symbols that any technically oriented person can follow.

We will not spend much time trying to convey a deep, higher-level
understanding of the meaning of the axioms. This kind of understanding
requires some mathematical sophistication as well as an understanding of the
philosophy underlying the foundations of mathematics and typically develops
over time as you work with mathematics.  Our goal, instead, is to give you the
immediate ability to follow how theorems\index{theorem} are derived from the
axioms and from other theorems.  This will be similar to learning the syntax
of a computer language, which lets you follow the details in a program but
does not necessarily give you the ability to write non-trivial programs on
your own, an ability that comes with practice. For now don't be alarmed by
abstract-sounding names of the axioms; just focus on the rules for
manipulating the symbols, which follow the simple conventions of the
Metamath\index{Metamath} language.

The axioms that underlie all of standard mathematics consist of axioms of logic
and axioms of set theory. The axioms of logic are divided into two
subcategories, propositional calculus\index{propositional calculus} (sometimes
called sentential logic\index{sentential logic}) and predicate calculus
(sometimes called first-order logic\index{first-order logic}\index{quantifier
theory}\index{predicate calculus} or quantifier theory).  Propositional
calculus is a prerequisite for predicate calculus, and predicate calculus is a
prerequisite for set theory.  The version of set theory most commonly used is
Zermelo--Fraenkel set theory\index{Zermelo--Fraenkel set theory}\index{set theory}
with the axiom of choice,
often abbreviated as ZFC\index{ZFC}.

Here in a nutshell is what the axioms are all about in an informal way. The
connection between this description and symbols we will show you won't be
immediately apparent and in principle needn't ever be.  Our description just
tries to summarize what mathematicians think about when they work with the
axioms.

Logic is a set of rules that allow us determine truths given other truths.
Put another way,
logic is more or less the translation of what we would consider common sense
into a rigorous set of axioms.\index{axioms of logic}  Suppose $\varphi$,
$\psi$, and $\chi$ (the Greek letters phi, psi, and chi) represent statements
that are either true or false, and $x$ is a variable\index{variable!in predicate
calculus} ranging over some group of mathematical objects (sets, integers,
real numbers, etc.). In mathematics, a ``statement'' really means a formula,
and $\psi$ could be for example ``$x = 2$.''
Propositional calculus\index{propositional calculus}
allows us to use variables that are either true or false
and make deductions such as
``if $\varphi$ implies $\psi$ and $\psi$ implies $\chi$, then $\varphi$
implies $\chi$.''
Predicate calculus\index{predicate calculus}
extends propositional calculus by also allowing us
to discuss statements about objects (not just true and false values), including
statements about ``all'' or ``at least one'' object.
For example, predicate calculus allows to say,
``if $\varphi$ is true for all $x$, then $\varphi$ is true for some $x$.''
The logic used in \texttt{set.mm} is standard classical logic
(as opposed to other logic systems like intuitionistic logic).

Set theory\index{set theory} has to do with the manipulation of objects and
collections of objects, specifically the abstract, imaginary objects that
mathematics deals with, such as numbers. Everything that is claimed to exist
in mathematics is considered to be a set.  A set called the empty
set\index{empty set} contains nothing.  We represent the empty set by
$\varnothing$.  Many sets can be built up from the empty set.  There is a set
represented by $\{\varnothing\}$ that contains the empty set, another set
represented by $\{\varnothing,\{\varnothing\}\}$ that contains this set as
well as the empty set, another set represented by $\{\{\varnothing\}\}$ that
contains just the set that contains the empty set, and so on ad infinitum. All
mathematical objects, no matter how complex, are defined as being identical to
certain sets: the integer\index{integer} 0 is defined as the empty set, the
integer 1 is defined as $\{\varnothing\}$, the integer 2 is defined as
$\{\varnothing,\{\varnothing\}\}$.  (How these definitions were chosen doesn't
matter now, but the idea behind it is that these sets have the properties we
expect of integers once suitable operations are defined.)  Mathematical
operations, such as addition, are defined in terms of operations on
sets---their union\index{set union}, intersection\index{set intersection}, and
so on---operations you may have used in elementary school when you worked
with groups of apples and oranges.

With a leap of faith, the axioms also postulate the existence of infinite
sets\index{infinite set}, such as the set of all non-negative integers ($0, 1,
2,\ldots$, also called ``natural numbers''\index{natural number}).  This set
can't be represented with the brace notation\index{brace notation} we just
showed you, but requires a more complicated notation called ``class
abstraction.''\index{class abstraction}\index{abstraction class}  For
example, the infinite set $\{ x |
\mbox{``$x$ is a natural number''} \} $ means the ``set of all objects $x$
such that $x$ is a natural number'' i.e.\ the set of natural numbers; here,
``$x$ is a natural number'' is a rather complicated formula when broken down
into the primitive symbols.\label{expandom}\footnote{The statement ``$x$ is a
natural number'' is formally expressed as ``$x \in \omega$,'' where $\in$
(stylized epsilon) means ``is in'' or ``is an element of'' and $\omega$
(omega) means ``the set of natural numbers.''  When ``$x\in\omega$'' is
completely expanded in terms of the primitive symbols of set theory, the
result is  $\lnot$ $($ $\lnot$ $($ $\forall$ $z$ $($ $\lnot$ $\forall$ $w$ $($
$z$ $\in$ $w$ $\rightarrow$ $\lnot$ $w$ $\in$ $x$ $)$ $\rightarrow$ $z$ $\in$
$x$ $)$ $\rightarrow$ $($ $\forall$ $z$ $($ $\lnot$ $($ $\forall$ $w$ $($ $w$
$\in$ $z$ $\rightarrow$ $w$ $\in$ $x$ $)$ $\rightarrow$ $\forall$ $w$ $\lnot$
$w$ $\in$ $z$ $)$ $\rightarrow$ $\lnot$ $\forall$ $w$ $($ $w$ $\in$ $z$
$\rightarrow$ $\lnot$ $\forall$ $v$ $($ $v$ $\in$ $z$ $\rightarrow$ $\lnot$
$v$ $\in$ $w$ $)$ $)$ $)$ $\rightarrow$ $\lnot$ $\forall$ $z$ $\forall$ $w$
$($ $\lnot$ $($ $z$ $\in$ $x$ $\rightarrow$ $\lnot$ $w$ $\in$ $x$ $)$
$\rightarrow$ $($ $\lnot$ $z$ $\in$ $w$ $\rightarrow$ $($ $\lnot$ $z$ $=$ $w$
$\rightarrow$ $w$ $\in$ $z$ $)$ $)$ $)$ $)$ $)$ $\rightarrow$ $\lnot$
$\forall$ $y$ $($ $\lnot$ $($ $\lnot$ $($ $\forall$ $z$ $($ $\lnot$ $\forall$
$w$ $($ $z$ $\in$ $w$ $\rightarrow$ $\lnot$ $w$ $\in$ $y$ $)$ $\rightarrow$
$z$ $\in$ $y$ $)$ $\rightarrow$ $($ $\forall$ $z$ $($ $\lnot$ $($ $\forall$
$w$ $($ $w$ $\in$ $z$ $\rightarrow$ $w$ $\in$ $y$ $)$ $\rightarrow$ $\forall$
$w$ $\lnot$ $w$ $\in$ $z$ $)$ $\rightarrow$ $\lnot$ $\forall$ $w$ $($ $w$
$\in$ $z$ $\rightarrow$ $\lnot$ $\forall$ $v$ $($ $v$ $\in$ $z$ $\rightarrow$
$\lnot$ $v$ $\in$ $w$ $)$ $)$ $)$ $\rightarrow$ $\lnot$ $\forall$ $z$
$\forall$ $w$ $($ $\lnot$ $($ $z$ $\in$ $y$ $\rightarrow$ $\lnot$ $w$ $\in$
$y$ $)$ $\rightarrow$ $($ $\lnot$ $z$ $\in$ $w$ $\rightarrow$ $($ $\lnot$ $z$
$=$ $w$ $\rightarrow$ $w$ $\in$ $z$ $)$ $)$ $)$ $)$ $\rightarrow$ $($
$\forall$ $z$ $\lnot$ $z$ $\in$ $y$ $\rightarrow$ $\lnot$ $\forall$ $w$ $($
$\lnot$ $($ $w$ $\in$ $y$ $\rightarrow$ $\lnot$ $\forall$ $z$ $($ $w$ $\in$
$z$ $\rightarrow$ $\lnot$ $z$ $\in$ $y$ $)$ $)$ $\rightarrow$ $\lnot$ $($
$\lnot$ $\forall$ $z$ $($ $w$ $\in$ $z$ $\rightarrow$ $\lnot$ $z$ $\in$ $y$
$)$ $\rightarrow$ $w$ $\in$ $y$ $)$ $)$ $)$ $)$ $\rightarrow$ $x$ $\in$ $y$
$)$ $)$ $)$. Section~\ref{hierarchy} shows the hierarchy of definitions that
leads up to this expression.}\index{stylized epsilon ($\in$)}\index{omega
($\omega$)}  Actually, the primitive symbols don't even include the brace
notation.  The brace notation is a high-level definition, which you can find in
Section~\ref{hierarchy}.

Interestingly, the arithmetic of integers\index{integer} and
rationals\index{rational number} can be developed without appealing to the
existence of an infinite set, whereas the arithmetic of real
numbers\index{real number} requires it.

Each variable\index{variable!in set theory} in the axioms of set theory
represents an arbitrary set, and the axioms specify the legal kinds of things
you can do with these variables at a very primitive level.

Now, you may think that numbers and arithmetic are a lot more intuitive and
fundamental than sets and therefore should be the foundation of mathematics.
What is really the case is that you've dealt with numbers all your life and
are comfortable with a few rules for manipulating them such as addition and
multiplication.  Those rules only cover a small portion of what can be done
with numbers and only a very tiny fraction of the rest of mathematics.  If you
look at any elementary book on number theory, you will quickly become lost if
these are the only rules that you know.  Even though such books may present a
list of ``axioms''\index{axiom} for arithmetic, the ability to use the axioms
and to understand proofs of theorems\index{theorem} (facts) about numbers
requires an implicit mathematical talent that frustrates many people
from studying abstract mathematics.  The kind of mathematics that most people
know limits them to the practical, everyday usage of blindly manipulating
numbers and formulas, without any understanding of why those rules are correct
nor any ability to go any further.  For example, do you know why multiplying
two negative numbers yields a positive number?  Starting with set theory, you
will also start off blindly manipulating symbols according to the rules we give
you, but with the advantage that these rules will allow you, in principle, to
access {\em all} of mathematics, not just a tiny part of it.

Of course, concrete examples are often helpful in the learning process. For
example, you can verify that $2\cdot 3=3 \cdot 2$ by actually grouping
objects and can easily ``see'' how it generalizes to $x\cdot y = y\cdot x$,
even though you might not be able to rigorously prove it.  Similarly, in set
theory it can be helpful to understand how the axioms of set theory apply to
(and are correct for) small finite collections of objects.  You should be aware
that in set theory intuition can be misleading for infinite collections, and
rigorous proofs become more important.  For example, while $x\cdot y = y\cdot
x$ is correct for finite ordinals (which are the natural numbers), it is not
usually true for infinite ordinals.

\section{The Axioms for All of Mathematics}

In this section\index{axioms for mathematics}, we will show you the axioms
for all of standard mathematics (i.e.\ logic and set theory) as they are
traditionally presented.  The traditional presentation is useful for someone
with the mathematical experience needed to correctly manipulate high-level
abstract concepts.  For someone without this talent, knowing how to actually
make use of these axioms can be difficult.  The purpose of this section is to
allow you to see how the version of the axioms used in the standard
Metamath\index{Metamath} database \texttt{set.mm}\index{set
theory database (\texttt{set.mm})} relates to  the typical version
in textbooks, and also to give you an informal feel for them.

\subsection{Propositional Calculus}

Propositional calculus\index{propositional calculus} concerns itself with
statements that can be interpreted as either true or false.  Some examples of
statements (outside of mathematics) that are either true or false are ``It is
raining today'' and ``The United States has a female president.'' In
mathematics, as we mentioned, statements are really formulas.

In propositional calculus, we don't care what the statements are.  We also
treat a logical combination of statements, such as ``It is raining today and
the United States has a female president,'' no differently from a single
statement.  Statements and their combinations are called well-formed formulas
(wffs)\index{well-formed formula (wff)}.  We define wffs only in terms of
other wffs and don't define what a ``starting'' wff is.  As is common practice
in the literature, we use Greek letters to represent wffs.

Specifically, suppose $\varphi$ and $\psi$ are wffs.  Then the combinations
$\varphi\rightarrow\psi$ (``$\varphi$ implies $\psi$,'' also read ``if
$\varphi$ then $\psi$'')\index{implication ($\rightarrow$)} and $\lnot\varphi$
(``not $\varphi$'')\index{negation ($\lnot$)} are also wffs.

The three axioms of propositional calculus\index{axioms of propositional
calculus} are all wffs of the following form:\footnote{A remarkable result of
C.~A.~Meredith\index{Meredith, C. A.} squeezes these three axioms into the
single axiom $((((\varphi\rightarrow \psi)\rightarrow(\neg \chi\rightarrow\neg
\theta))\rightarrow \chi )\rightarrow \tau)\rightarrow((\tau\rightarrow
\varphi)\rightarrow(\theta\rightarrow \varphi))$ \cite{CAMeredith},
which is believed to be the shortest possible.}
\begin{center}
     $\varphi\rightarrow(\psi\rightarrow \varphi)$\\

     $(\varphi\rightarrow (\psi\rightarrow \chi))\rightarrow
((\varphi\rightarrow  \psi)\rightarrow (\varphi\rightarrow \chi))$\\

     $(\neg \varphi\rightarrow \neg\psi)\rightarrow (\psi\rightarrow
\varphi)$
\end{center}

These three axioms are widely used.
They are attributed to Jan {\L}ukasiewicz
(pronounced woo-kah-SHAY-vitch) and was popularized by Alonzo Church,
who called it system P2. (Thanks to Ted Ulrich for this information.)

There are an infinite number of axioms, one for each possible
wff\index{well-formed formula (wff)} of the above form.  (For this reason,
axioms such as the above are often called ``axiom schemes.''\index{axiom
scheme})  Each Greek letter in the axioms may be substituted with a more
complex wff to result in another axiom.  For example, substituting
$\neg(\varphi\rightarrow\chi)$ for $\varphi$ in the first axiom yields
$\neg(\varphi\rightarrow\chi)\rightarrow(\psi\rightarrow
\neg(\varphi\rightarrow\chi))$, which is still an axiom.

To deduce new true statements (theorems\index{theorem}) from the axioms, a
rule\index{rule} called ``modus ponens''\index{modus ponens} is used.  This
rule states that if the wff $\varphi$ is an axiom or a theorem, and the wff
$\varphi\rightarrow\psi$ is an axiom or a theorem, then the wff $\psi$ is also
a theorem\index{theorem}.

As a non-mathematical example of modus ponens, suppose we have proved (or
taken as an axiom) ``Bob is a man'' and separately have proved (or taken as
an axiom) ``If Bob is a man, then Bob is a human.''  Using the rule of modus
ponens, we can logically deduce, ``Bob is a human.''

From Metamath's\index{Metamath} point of view, the axioms and the rule of
modus ponens just define a mechanical means for deducing new true statements
from existing true statements, and that is the complete content of
propositional calculus as far as Metamath is concerned.  You can read a logic
textbook to gain a better understanding of their meaning, or you can just let
their meaning slowly become apparent to you after you use them for a while.

It is actually rather easy to check to see if a formula is a theorem of
propositional calculus.  Theorems of propositional calculus are also called
``tautologies.''\index{tautology}  The technique to check whether a formula is
a tautology is called the ``truth table method,''\index{truth table} and it
works like this.  A wff $\varphi\rightarrow\psi$ is false whenever $\varphi$ is true
and $\psi$ is false.  Otherwise it is true.  A wff $\lnot\varphi$ is false
whenever $\varphi$ is true and false otherwise. To verify a tautology such as
$\varphi\rightarrow(\psi\rightarrow \varphi)$, you break it down into sub-wffs and
construct a truth table that accounts for all possible combinations of true
and false assigned to the wff metavariables:
\begin{center}\begin{tabular}{|c|c|c|c|}\hline
\mbox{$\varphi$} & \mbox{$\psi$} & \mbox{$\psi\rightarrow\varphi$}
    & \mbox{$\varphi\rightarrow(\psi\rightarrow \varphi)$} \\ \hline \hline
              T   &  T    &      T       &        T    \\ \hline
              T   &  F    &      T       &        T    \\ \hline
              F   &  T    &      F       &        T    \\ \hline
              F   &  F    &      T       &        T    \\ \hline
\end{tabular}\end{center}
If all entries in the last column are true, the formula is a tautology.

Now, the truth table method doesn't tell you how to prove the tautology from
the axioms, but only that a proof exists.  Finding an actual proof (especially
one that is short and elegant) can be challenging.  Methods do exist for
automatically generating proofs in propositional calculus, but the proofs that
result can sometimes be very long.  In the Metamath \texttt{set.mm}\index{set
theory database (\texttt{set.mm})} database, most
or all proofs were created manually.

Section \ref{metadefprop} discusses various definitions
that make propositional calculus easier to use.
For example, we define:

\begin{itemize}
\item $\varphi \vee \psi$
  is true if either $\varphi$ or $\psi$ (or both) are true
  (this is disjunction\index{disjunction ($\vee$)}
  aka logical {\sc or}\index{logical {\sc or} ($\vee$)}).

\item $\varphi \wedge \psi$
  is true if both $\varphi$ and $\psi$ are true
  (this is conjunction\index{conjunction ($\wedge$)}
  aka logical {\sc and}\index{logical {\sc and} ($\wedge$)}).

\item $\varphi \leftrightarrow \psi$
  is true if $\varphi$ and $\psi$ have the same value, that is,
  they are both true or both false
  (this is the biconditional\index{biconditional ($\leftrightarrow$)}).
\end{itemize}

\subsection{Predicate Calculus}

Predicate calculus\index{predicate calculus} introduces the concept of
``individual variables,''\index{variable!in predicate calculus}\index{individual
variable} which
we will usually just call ``variables.''
These variables can represent something other than true or false (wffs),
and will always represent sets when we get to set theory.  There are also
three new symbols $\forall$\index{universal quantifier ($\forall$)},
$=$\index{equality ($=$)}, and $\in$\index{stylized epsilon ($\in$)},
read ``for all,'' ``equals,'' and ``is an element of''
respectively.  We will represent variables with the letters $x$, $y$, $z$, and
$w$, as is common practice in the literature.
For example, $\forall x \varphi$ means ``for all possible values of
$x$, $\varphi$ is true.''

In predicate calculus, we extend the definition of a wff\index{well-formed
formula (wff)}.  If $\varphi$ is a wff and $x$ and $y$ are variables, then
$\forall x \, \varphi$, $x=y$, and $x\in y$ are wffs. Note that these three new
types of wffs can be considered ``starting'' wffs from which we can build
other wffs with $\rightarrow$ and $\neg$ .  The concept of a starting wff was
absent in propositional calculus.  But starting wff or not, all we are really
concerned with is whether our wffs are correctly constructed according to
these mechanical rules.

A quick aside:
To prevent confusion, it might be best at this point to think of the variables
of Metamath\index{Metamath} as ``metavariables,''\index{metavariable} because
they are not quite the same as the variables we are introducing here.  A
(meta)variable in Metamath can be a wff or an individual variable, as well
as many other things; in general, it represents a kind of place holder for an
unspecified sequence of math symbols\index{math symbol}.

Unlike propositional calculus, no decision procedure\index{decision procedure}
analogous to the truth table method exists (nor theoretically can exist) that
will definitely determine whether a formula is a theorem of predicate
calculus.  Much of the work in the field of automated theorem
proving\index{automated theorem proving} has been dedicated to coming up with
clever heuristics for proving theorems of predicate calculus, but they can
never be guaranteed to work always.

Section \ref{metadefpred} discusses various definitions
that make predicate calculus easier to use.
For example, we define
$\exists x \varphi$ to mean
``there exists at least one possible value of $x$ where $\varphi$ is true.''

We now turn to looking at how predicate calculus can be formally
represented.

\subsubsection{Common Axioms}

There is a new rule of inference in predicate calculus:  if $\varphi$ is
an axiom or a theorem, then $\forall x \,\varphi$ is also a
theorem\index{theorem}.  This is called the rule of
``generalization.''\index{rule of generalization}
This is easily represented in Metamath.

In standard texts of logic, there are often two axioms of predicate
calculus\index{axioms of predicate calculus}:
\begin{center}
  $\forall x \,\varphi ( x ) \rightarrow \varphi ( y )$,
      where ``$y$ is properly substituted for $x$.''\\
  $\forall x ( \varphi \rightarrow \psi )\rightarrow ( \varphi \rightarrow
    \forall x\, \psi )$,
    where ``$x$ is not free in $\varphi$.''
\end{center}

Now at first glance, this seems simple:  just two axioms.  However,
conditional clauses are attached to each axiom describing requirements that
may seem puzzling to you.  In addition, the first axiom puts a variable symbol
in parentheses after each wff, seemingly violating our definition of a
wff\index{well-formed formula (wff)}; this is just an informal way of
referring to some arbitrary variable that may occur in the wff.  The
conditional clauses do, of course, have a precise meaning, but as it turns out
the precise meaning is somewhat complicated and awkward to formalize in a
way that a computer can handle easily.  Unlike propositional calculus, a
certain amount of mathematical sophistication and practice is needed to be
able to easily grasp and manipulate these concepts correctly.

Predicate calculus may be presented with or without axioms for
equality\index{axioms of equality}\index{equality ($=$)}. We will require the
axioms of equality as a prerequisite for the version of set theory we will
use.  The axioms for equality, when included, are often represented using these
two axioms:
\begin{center}
$x=x$\\ \ \\
$x=y\rightarrow (\varphi(x,x)\rightarrow\varphi(x,y))$ where ``$\varphi(x,y)$
   arises from $\varphi(x,x)$ by replacing some, but not necessarily all,
   free\index{free variable}
   occurrences of $x$ by $y$,\\ provided that $y$ is free for $x$
   in $\varphi(x,x)$.'' \end{center}
% (Mendelson p. 95)
The first equality axiom is simple, but again,
the condition on the second one is
somewhat awkward to implement on a computer.

\subsubsection{Tarski System S2}

Of course, we are not the first to notice the complications of these
predicate calculus axioms when being rigorous.

Well-known logician Alfred Tarski published in 1965
a system he called system S2\cite[p.~77]{Tarski1965}.
Tarski's system is \textit{exactly equivalent} to the traditional textbook
formalization, but (by clever use of equality axioms) it eliminates the
latter's primitive notions of ``proper substitution'' and ``free variable,''
replacing them with direct substitution and the notion of a variable
not occurring in a formula (which we express with distinct variable
constraints).

In advocating his system, Tarski wrote, ``The relatively complicated
character of [free variables and proper substitution] is a source
of certain inconveniences of both practical and theoretical nature;
this is clearly experienced both in teaching an elementary course of
mathematical logic and in formalizing the syntax of predicate logic for
some theoretical purposes''\cite[p.~61]{Tarski1965}\index{Tarski, Alfred}.

\subsubsection{Developing a Metamath Representation}

The standard textbook axioms of predicate calculus are somewhat
cumbersome to implement on a computer because of the complex notions of
``free variable''\index{free variable} and ``proper
substitution.''\index{proper substitution}\index{substitution!proper}
While it is possible to use the Metamath\index{Metamath} language to
implement these concepts, we have chosen not to implement them
as primitive constructs in the
\texttt{set.mm} set theory database.  Instead, we have eliminated them
within the axioms
by carefully crafting the axioms so as to avoid them,
building on Tarski's system S2.  This makes it
easy for a beginner to follow the steps in a proof without knowing any
advanced concepts other than the simple concept of
replacing\index{substitution!variable}\index{variable substitution}
variables with expressions.

In order to develop the concepts of free variable and proper
substitution from the axioms, we use an additional
Metamath statement type called ``disjoint variable
restriction''\index{disjoint variables} that we have not encountered
before.  In the context of the axioms, the statement \texttt{\$d} $ x\,
y$\index{\texttt{\$d} statement} simply means that $x$ and $y$ must be
distinct\index{distinct variables}, i.e.\ they may not be simultaneously
substituted\index{substitution!variable}\index{variable substitution}
with the same variable.  The statement \texttt{\$d} $ x\, \varphi$ means
variable $x$ must not occur in wff $\varphi$.  For the precise
definition of \texttt{\$d}, see Section~\ref{dollard}.

\subsubsection{Metamath representation}

The Metamath axiom system for predicate calculus
defined in set.mm uses Tarski's system S2.
As noted above, this has a different representation
than the traditional textbook formalization,
but it is \textit{exactly equivalent} to the textbook formalization,
and it is \textit{much} easier to work with.
This is reproduced as system S3 in Section 6 of
Megill's formalization \cite{Megill}\index{Megill, Norman}.

There is one exception, Tarski's axiom of existence,
which we label as axiom ax-6.
In the case of ax-6, Tarski's version is weaker because it includes a
distinct variable proviso. If we wish, we can also weaken our version
in this way and still have a metalogically complete system. Theorem
ax6 shows this by deriving, in the presence of the other axioms, our
ax-6 from Tarski's weaker version ax6v. However, we chose the stronger
version for our system because it is simpler to state and easier to use.

Tarski's system was designed for proving specific theorems rather than
more general theorem schemes. However, theorem schemes are much more
efficient than specific theorems for building a body of mathematical
knowledge, since they can be reused with different instances as
needed. While Tarski does derive some theorem schemes from his axioms,
their proofs require concepts that are ``outside'' of the system, such as
induction on formula length. The verification of such proofs is difficult
to automate in a proof verifier. (Specifically, Tarski treats the formulas
of his system as set-theoretical objects. In order to verify the proofs
of his theorem schemes, a proof verifier would need a significant amount
of set theory built into it.)

The Metamath axiom system for predicate calculus extends
Tarski's system to eliminate this difficulty. The additional
``auxilliary'' axiom
schemes (as we will call them in this section; see below) endow Tarski's
system with a nice property we call
metalogical completeness \cite[Remark 9.6]{Megill}\index{Megill, Norman}.
As a result, we can prove any theorem scheme
expressable in the ``simple metalogic'' of Tarski's system by using
only Metamath's direct substitution rule applied to the axiom system
(and no other metalogical or set-theoretical notions ``outside'' of the
system). Simple metalogic consists of schemes containing wff metavariables
(with no arguments) and/or set (also called ``individual'') metavariables,
accompanied by optional provisos each stating that two specified set
metavariables must be distinct or that a specified set metavariable may
not occur in a specified wff metavariable. Metamath's logic and set theory
axiom and rule schemes are all examples of simple metalogic. The schemes
of traditional predicate calculus with equality are examples which are
not simple metalogic, because they use wff metavariables with arguments
and have ``free for'' and ``not free in'' side conditions.

A rigorous justification for this system, using an older but
exactly equivalent set of axioms, can be
found in \cite{Megill}\index{Megill, Norman}.

This allows us to
take a different approach in the Metamath\index{Metamath} database
\texttt{set.mm}\index{set theory database (\texttt{set.mm})}.  We do not
directly use the primitive notions of ``free variable''\index{free variable}
and ``proper substitution''\index{proper
substitution}\index{substitution!proper} at all as primitive constructs.
Instead, we use a set
of axioms that are almost as simple to manipulate as those of
propositional calculus.  Our axiom system avoids complex primitive
notions by effectively embedding the complexity into the axioms
themselves.  As a result, we will end up with a larger number of axioms,
but they are ideally suited for a computer language such as Metamath.
(Section~\ref{metaaxioms} shows these axioms.)

We will not elaborate further
on the ``free variable'' and ``proper substitution''
concepts here.  You may consult
\cite[ch.\ 3--4]{Hamilton}\index{Hamilton, Alan G.} (as well as
many other books) for a precise explanation
of these concepts.  If you intend to do serious mathematical work, it is wise
to become familiar with the traditional textbook approach; even though the
concepts embedded in their axioms require a higher level of sophistication,
they can be more practical to deal with on an everyday, informal basis.  Even
if you are just developing Metamath proofs, familiarity with the traditional
approach can help you arrive at a proof outline much faster, which you can
then convert to the detail required by Metamath.

We do develop proper substitution rules later on, but in set.mm
they are defined as derived constructs; they are not primitives.

You should also note that our system of predicate calculus is specifically
tailored for set theory; thus there are only two specific predicates $=$ and
$\in$ and no functions\index{function!in predicate calculus}
or constants\index{constant!in predicate calculus} unlike more general systems.
We later add these.

\subsection{Set Theory}

Traditional Zermelo--Fraenkel set theory\index{Zermelo--Fraenkel set
theory}\index{set theory} with the Axiom of Choice
has 10 axioms, which can be expressed in the
language of predicate calculus.  In this section, we will list only the
names and brief English descriptions of these axioms, since we will give
you the precise formulas used by the Metamath\index{Metamath} set theory
database \texttt{set.mm} later on.

In the descriptions of the axioms, we assume that $x$, $y$, $z$, $w$, and $v$
represent sets.  These are the same as the variables\index{variable!in set
theory} in our predicate calculus system above, except that now we informally
think of the variables as ranging over sets.  Note that the terms
``object,''\index{object} ``set,''\index{set} ``element,''\index{element}
``collection,''\index{collection} and ``family''\index{family} are synonymous,
as are ``is an element of,'' ``is a member of,''\index{member} ``is contained
in,'' and ``belongs to.''  The different terms are used for convenience; for
example, ``a collection of sets'' is less confusing than ``a set of sets.''
A set $x$ is said to be a ``subset''\index{subset} of $y$ if every element of
$x$ is also an element of $y$; we also say $x$ is ``included in''
$y$.

The axioms are very general and apply to almost any conceivable mathematical
object, and this level of abstraction can be overwhelming at first.  To gain an
intuitive feel, it can be helpful to draw a picture illustrating the concept;
for example, a circle containing dots could represent a collection of sets,
and a smaller circle drawn inside the circle could represent a subset.
Overlapping circles can illustrate intersection and union.  Circles that
illustrate the concepts of set theory are frequently used in elementary
textbooks and are called Venn diagrams\index{Venn diagram}.\index{axioms of
set theory}

1. Axiom of Extensionality:  Two sets are identical if they contain the same
   elements.\index{Axiom of Extensionality}

2. Axiom of Pairing:  The set $\{ x , y \}$ exists.\index{Axiom of Pairing}

3. Axiom of Power Sets:  The power set of a set (the collection of all of
   its subsets) exists.  For example, the power set of $\{x,y\}$ is
   $\{\varnothing,\{x\},\{y\},\{x,y\}\}$ and it exists.\index{Axiom
of Power Sets}

4. Axiom of the Null Set:  The empty set $\varnothing$ exists.\index{Axiom of
the Null Set}

5. Axiom of Union:  The union of a set (the set containing the elements of
   its members) exists.  For example, the union of $\{\{x,y\},\{z\}\}$ is
 $\{x,y,z\}$ and
   it exists.\index{Axiom of Union}

6. Axiom of Regularity:  Roughly, no set can contain itself, nor can there
   be membership ``loops,'' such as a set being an
   element of one of its members.\index{Axiom of Regularity}

7. Axiom of Infinity:  An infinite set exists.  An example of an infinite
   set is the set of all
   integers.\index{Axiom of Infinity}

8. Axiom of Separation:  The set exists that is obtained by restricting $x$
   with some property.  For example, if the set of all integers exists,
   then the set of all even integers exists.\index{Axiom of Separation}

9. Axiom of Replacement:  The range of a function whose domain is restricted
   to the elements of a set $x$, is also a set.  For example, there
   is a function
   from integers (the function's domain) to their squares (its
   range).  If we
   restrict the domain to even integers, its range will become the set of
   squares of even integers, so this axiom asserts that the set of
    squares of even numbers exists.  Technical note:  In general, the
   ``function'' need not be a set but can be a proper class.
   \index{Axiom of Replacement}

10. Axiom of Choice:  Let $x$ be a set whose members are pairwise
  disjoint\index{disjoint sets} (i.e,
  whose members contain no elements in common).  Then there exists another
  set containing one element from each member of $x$.  For
  example, if $x$ is
  $\{\{y,z\},\{w,v\}\}$, where $y$, $z$, $w$, and $v$ are
  different sets, then a set such as $\{z,w\}$
  exists (but the axiom doesn't tell
  us which one).  (Actually the Axiom
  of Choice is redundant if the set $x$, as in this example, has a finite
  number of elements.)\index{Axiom of Choice}

The Axiom of Choice is usually considered an extension of ZF set theory rather
than a proper part of it.  It is sometimes considered philosophically
controversial because it specifies the existence of a set without specifying
what the set is. Constructive logics, including intuitionistic logic,
do not accept the axiom of choice.
Since there is some lingering controversy, we often prefer proofs that do
not use the axiom of choice (where there is a known alternative), and
in some cases we will use weaker axioms than the full axiom of choice.
That said, the axiom of choice is a powerful and widely-accepted tool,
so we do use it when needed.
ZF set theory that includes the Axiom of Choice is
called Zermelo--Fraenkel set theory with choice (ZFC\index{ZFC set theory}).

When expressed symbolically, the Axiom of Separation and the Axiom of
Replacement contain wff symbols and therefore each represent infinitely many
axioms, one for each possible wff. For this reason, they are often called
axiom schemes\index{axiom scheme}\index{well-formed formula (wff)}.

It turns out that the Axiom of the Null Set, the Axiom of Pairing, and the
Axiom of Separation can be derived from the other axioms and are therefore
unnecessary, although they tend to be included in standard texts for various
reasons (historical, philosophical, and possibly because some authors may not
know this).  In the Metamath\index{Metamath} set theory database, these
redundant axioms are derived from the other ones instead of truly
being considered axioms.
This is in keeping with our general goal of minimizing the number of
axioms we must depend on.

\subsection{Other Axioms}

Above we qualified the phrase "all of mathematics" with "essentially."
The main important missing piece is the ability to do category theory,
which requires huge sets (inaccessible cardinals) larger than those
postulated by the ZFC axioms. The Tarski--Grothendieck Axiom postulates
the existence of such sets.
Note that this is the same axiom used by Mizar for supporting
category theory.
The Tarski--Grothendieck axiom
can be viewed as a very strong replacement of the Axiom of Infinity,
the Axiom of Choice, and the Axiom of Power Sets.
The \texttt{set.mm} database includes this axiom; see the database
for details about it.
Again, we only use this axiom when we need to.
You are only likely to encounter or use this axiom if you are doing
category theory, since its use is highly specialized,
so we will not list the Tarsky-Grothendieck axiom
in the short list of axioms below.

Can there be even more axioms?
Of course.
G\"{o}del showed that no finite set of axioms or axiom schemes can completely
describe any consistent theory strong enough to include arithmetic.
But practically speaking, the ones above are the accepted foundation that
almost all mathematicians explicitly or implicitly base their work on.

\section{The Axioms in the Metamath Language}\label{metaaxioms}

Here we list the axioms as they appear in
\texttt{set.mm}\index{set theory database (\texttt{set.mm})} so you can
look them up there easily.  Incidentally, the \texttt{show statement
/tex} command\index{\texttt{show statement} command} was used to
typeset them.

%macros from show statement /tex
\newbox\mlinebox
\newbox\mtrialbox
\newbox\startprefix  % Prefix for first line of a formula
\newbox\contprefix  % Prefix for continuation line of a formula
\def\startm{  % Initialize formula line
  \setbox\mlinebox=\hbox{\unhcopy\startprefix}
}
\def\m#1{  % Add a symbol to the formula
  \setbox\mtrialbox=\hbox{\unhcopy\mlinebox $\,#1$}
  \ifdim\wd\mtrialbox>\hsize
    \box\mlinebox
    \setbox\mlinebox=\hbox{\unhcopy\contprefix $\,#1$}
  \else
    \setbox\mlinebox=\hbox{\unhbox\mtrialbox}
  \fi
}
\def\endm{  % Output the last line of a formula
  \box\mlinebox
}

% \SLASH for \ , \TOR for \/ (text OR), \TAND for /\ (text and)
% This embeds a following forced space to force the space.
\newcommand\SLASH{\char`\\~}
\newcommand\TOR{\char`\\/~}
\newcommand\TAND{/\char`\\~}
%
% Macro to output metamath raw text.
% This assumes \startprefix and \contprefix are set.
% NOTE: "\" is tricky to escape, use \SLASH, \TOR, and \TAND inside.
% Any use of "$ { ~ ^" must be escaped; ~ and ^ must be escaped specially.
% We escape { and } for consistency.
% For more about how this macro written, see:
% https://stackoverflow.com/questions/4073674/
% how-to-disable-indentation-in-particular-section-in-latex/4075706
% Use frenchspacing, or "e." will get an extra space after it.
\newlength\mystoreparindent
\newlength\mystorehangindent
\newenvironment{mmraw}{%
\setlength{\mystoreparindent}{\the\parindent}
\setlength{\mystorehangindent}{\the\hangindent}
\setlength{\parindent}{0pt} % TODO - we'll put in the \startprefix instead
\setlength{\hangindent}{\wd\the\contprefix}
\begin{flushleft}
\begin{frenchspacing}
\begin{tt}
{\unhcopy\startprefix}%
}{%
\end{tt}
\end{frenchspacing}
\end{flushleft}
\setlength{\parindent}{\mystoreparindent}
\setlength{\hangindent}{\mystorehangindent}
\vskip 1ex
}

\needspace{5\baselineskip}
\subsection{Propositional Calculus}\label{propcalc}\index{axioms of
propositional calculus}

\needspace{2\baselineskip}
Axiom of Simplification.\label{ax1}

\setbox\startprefix=\hbox{\tt \ \ ax-1\ \$a\ }
\setbox\contprefix=\hbox{\tt \ \ \ \ \ \ \ \ \ \ }
\startm
\m{\vdash}\m{(}\m{\varphi}\m{\rightarrow}\m{(}\m{\psi}\m{\rightarrow}\m{\varphi}\m{)}
\m{)}
\endm

\needspace{3\baselineskip}
\noindent Axiom of Distribution.

\setbox\startprefix=\hbox{\tt \ \ ax-2\ \$a\ }
\setbox\contprefix=\hbox{\tt \ \ \ \ \ \ \ \ \ \ }
\startm
\m{\vdash}\m{(}\m{(}\m{\varphi}\m{\rightarrow}\m{(}\m{\psi}\m{\rightarrow}\m{\chi}
\m{)}\m{)}\m{\rightarrow}\m{(}\m{(}\m{\varphi}\m{\rightarrow}\m{\psi}\m{)}\m{
\rightarrow}\m{(}\m{\varphi}\m{\rightarrow}\m{\chi}\m{)}\m{)}\m{)}
\endm

\needspace{2\baselineskip}
\noindent Axiom of Contraposition.

\setbox\startprefix=\hbox{\tt \ \ ax-3\ \$a\ }
\setbox\contprefix=\hbox{\tt \ \ \ \ \ \ \ \ \ \ }
\startm
\m{\vdash}\m{(}\m{(}\m{\lnot}\m{\varphi}\m{\rightarrow}\m{\lnot}\m{\psi}\m{)}\m{
\rightarrow}\m{(}\m{\psi}\m{\rightarrow}\m{\varphi}\m{)}\m{)}
\endm


\needspace{4\baselineskip}
\noindent Rule of Modus Ponens.\label{axmp}\index{modus ponens}

\setbox\startprefix=\hbox{\tt \ \ maj\ \$e\ }
\setbox\contprefix=\hbox{\tt \ \ \ \ \ \ \ \ \ }
\startm
\m{\vdash}\m{(}\m{\varphi}\m{\rightarrow}\m{\psi}\m{)}
\endm

\setbox\startprefix=\hbox{\tt \ \ min\ \$e\ }
\setbox\contprefix=\hbox{\tt \ \ \ \ \ \ \ \ \ }
\startm
\m{\vdash}\m{\varphi}
\endm

\setbox\startprefix=\hbox{\tt \ \ ax-mp\ \$a\ }
\setbox\contprefix=\hbox{\tt \ \ \ \ \ \ \ \ \ \ \ }
\startm
\m{\vdash}\m{\psi}
\endm


\needspace{7\baselineskip}
\subsection{Axioms of Predicate Calculus with Equality---Tarski's S2}\index{axioms of predicate calculus}

\needspace{3\baselineskip}
\noindent Rule of Generalization.\index{rule of generalization}

\setbox\startprefix=\hbox{\tt \ \ ax-g.1\ \$e\ }
\setbox\contprefix=\hbox{\tt \ \ \ \ \ \ \ \ \ \ \ \ }
\startm
\m{\vdash}\m{\varphi}
\endm

\setbox\startprefix=\hbox{\tt \ \ ax-gen\ \$a\ }
\setbox\contprefix=\hbox{\tt \ \ \ \ \ \ \ \ \ \ \ \ }
\startm
\m{\vdash}\m{\forall}\m{x}\m{\varphi}
\endm

\needspace{2\baselineskip}
\noindent Axiom of Quantified Implication.

\setbox\startprefix=\hbox{\tt \ \ ax-4\ \$a\ }
\setbox\contprefix=\hbox{\tt \ \ \ \ \ \ \ \ \ \ }
\startm
\m{\vdash}\m{(}\m{\forall}\m{x}\m{(}\m{\forall}\m{x}\m{\varphi}\m{\rightarrow}\m{
\psi}\m{)}\m{\rightarrow}\m{(}\m{\forall}\m{x}\m{\varphi}\m{\rightarrow}\m{
\forall}\m{x}\m{\psi}\m{)}\m{)}
\endm

\needspace{3\baselineskip}
\noindent Axiom of Distinctness.

% Aka: Add $d x ph $.
\setbox\startprefix=\hbox{\tt \ \ ax-5\ \$a\ }
\setbox\contprefix=\hbox{\tt \ \ \ \ \ \ \ \ \ \ }
\startm
\m{\vdash}\m{(}\m{\varphi}\m{\rightarrow}\m{\forall}\m{x}\m{\varphi}\m{)}\m{where}\m{ }\m{\$d}\m{ }\m{x}\m{ }\m{\varphi}\m{ }\m{(}\m{x}\m{ }\m{does}\m{ }\m{not}\m{ }\m{occur}\m{ }\m{in}\m{ }\m{\varphi}\m{)}
\endm

\needspace{2\baselineskip}
\noindent Axiom of Existence.

\setbox\startprefix=\hbox{\tt \ \ ax-6\ \$a\ }
\setbox\contprefix=\hbox{\tt \ \ \ \ \ \ \ \ \ \ }
\startm
\m{\vdash}\m{(}\m{\forall}\m{x}\m{(}\m{x}\m{=}\m{y}\m{\rightarrow}\m{\forall}
\m{x}\m{\varphi}\m{)}\m{\rightarrow}\m{\varphi}\m{)}
\endm

\needspace{2\baselineskip}
\noindent Axiom of Equality.

\setbox\startprefix=\hbox{\tt \ \ ax-7\ \$a\ }
\setbox\contprefix=\hbox{\tt \ \ \ \ \ \ \ \ \ \ }
\startm
\m{\vdash}\m{(}\m{x}\m{=}\m{y}\m{\rightarrow}\m{(}\m{x}\m{=}\m{z}\m{
\rightarrow}\m{y}\m{=}\m{z}\m{)}\m{)}
\endm

\needspace{2\baselineskip}
\noindent Axiom of Left Equality for Binary Predicate.

\setbox\startprefix=\hbox{\tt \ \ ax-8\ \$a\ }
\setbox\contprefix=\hbox{\tt \ \ \ \ \ \ \ \ \ \ \ }
\startm
\m{\vdash}\m{(}\m{x}\m{=}\m{y}\m{\rightarrow}\m{(}\m{x}\m{\in}\m{z}\m{
\rightarrow}\m{y}\m{\in}\m{z}\m{)}\m{)}
\endm

\needspace{2\baselineskip}
\noindent Axiom of Right Equality for Binary Predicate.

\setbox\startprefix=\hbox{\tt \ \ ax-9\ \$a\ }
\setbox\contprefix=\hbox{\tt \ \ \ \ \ \ \ \ \ \ \ }
\startm
\m{\vdash}\m{(}\m{x}\m{=}\m{y}\m{\rightarrow}\m{(}\m{z}\m{\in}\m{x}\m{
\rightarrow}\m{z}\m{\in}\m{y}\m{)}\m{)}
\endm


\needspace{4\baselineskip}
\subsection{Axioms of Predicate Calculus with Equality---Auxiliary}\index{axioms of predicate calculus - auxiliary}

\needspace{2\baselineskip}
\noindent Axiom of Quantified Negation.

\setbox\startprefix=\hbox{\tt \ \ ax-10\ \$a\ }
\setbox\contprefix=\hbox{\tt \ \ \ \ \ \ \ \ \ \ }
\startm
\m{\vdash}\m{(}\m{\lnot}\m{\forall}\m{x}\m{\lnot}\m{\forall}\m{x}\m{\varphi}\m{
\rightarrow}\m{\varphi}\m{)}
\endm

\needspace{2\baselineskip}
\noindent Axiom of Quantifier Commutation.

\setbox\startprefix=\hbox{\tt \ \ ax-11\ \$a\ }
\setbox\contprefix=\hbox{\tt \ \ \ \ \ \ \ \ \ \ }
\startm
\m{\vdash}\m{(}\m{\forall}\m{x}\m{\forall}\m{y}\m{\varphi}\m{\rightarrow}\m{
\forall}\m{y}\m{\forall}\m{x}\m{\varphi}\m{)}
\endm

\needspace{3\baselineskip}
\noindent Axiom of Substitution.

\setbox\startprefix=\hbox{\tt \ \ ax-12\ \$a\ }
\setbox\contprefix=\hbox{\tt \ \ \ \ \ \ \ \ \ \ \ }
\startm
\m{\vdash}\m{(}\m{\lnot}\m{\forall}\m{x}\m{\,x}\m{=}\m{y}\m{\rightarrow}\m{(}
\m{x}\m{=}\m{y}\m{\rightarrow}\m{(}\m{\varphi}\m{\rightarrow}\m{\forall}\m{x}\m{(}
\m{x}\m{=}\m{y}\m{\rightarrow}\m{\varphi}\m{)}\m{)}\m{)}\m{)}
\endm

\needspace{3\baselineskip}
\noindent Axiom of Quantified Equality.

\setbox\startprefix=\hbox{\tt \ \ ax-13\ \$a\ }
\setbox\contprefix=\hbox{\tt \ \ \ \ \ \ \ \ \ \ \ }
\startm
\m{\vdash}\m{(}\m{\lnot}\m{\forall}\m{z}\m{\,z}\m{=}\m{x}\m{\rightarrow}\m{(}
\m{\lnot}\m{\forall}\m{z}\m{\,z}\m{=}\m{y}\m{\rightarrow}\m{(}\m{x}\m{=}\m{y}
\m{\rightarrow}\m{\forall}\m{z}\m{\,x}\m{=}\m{y}\m{)}\m{)}\m{)}
\endm

% \noindent Axiom of Quantifier Substitution
%
% \setbox\startprefix=\hbox{\tt \ \ ax-c11n\ \$a\ }
% \setbox\contprefix=\hbox{\tt \ \ \ \ \ \ \ \ \ \ \ }
% \startm
% \m{\vdash}\m{(}\m{\forall}\m{x}\m{\,x}\m{=}\m{y}\m{\rightarrow}\m{(}\m{\forall}
% \m{x}\m{\varphi}\m{\rightarrow}\m{\forall}\m{y}\m{\varphi}\m{)}\m{)}
% \endm
%
% \noindent Axiom of Distinct Variables. (This axiom requires
% that two individual variables
% be distinct\index{\texttt{\$d} statement}\index{distinct
% variables}.)
%
% \setbox\startprefix=\hbox{\tt \ \ \ \ \ \ \ \ \$d\ }
% \setbox\contprefix=\hbox{\tt \ \ \ \ \ \ \ \ \ \ \ }
% \startm
% \m{x}\m{\,}\m{y}
% \endm
%
% \setbox\startprefix=\hbox{\tt \ \ ax-c16\ \$a\ }
% \setbox\contprefix=\hbox{\tt \ \ \ \ \ \ \ \ \ \ \ }
% \startm
% \m{\vdash}\m{(}\m{\forall}\m{x}\m{\,x}\m{=}\m{y}\m{\rightarrow}\m{(}\m{\varphi}\m{
% \rightarrow}\m{\forall}\m{x}\m{\varphi}\m{)}\m{)}
% \endm

% \noindent Axiom of Quantifier Introduction (2).  (This axiom requires
% that the individual variable not occur in the
% wff\index{\texttt{\$d} statement}\index{distinct variables}.)
%
% \setbox\startprefix=\hbox{\tt \ \ \ \ \ \ \ \ \$d\ }
% \setbox\contprefix=\hbox{\tt \ \ \ \ \ \ \ \ \ \ \ }
% \startm
% \m{x}\m{\,}\m{\varphi}
% \endm
% \setbox\startprefix=\hbox{\tt \ \ ax-5\ \$a\ }
% \setbox\contprefix=\hbox{\tt \ \ \ \ \ \ \ \ \ \ \ }
% \startm
% \m{\vdash}\m{(}\m{\varphi}\m{\rightarrow}\m{\forall}\m{x}\m{\varphi}\m{)}
% \endm

\subsection{Set Theory}\label{mmsettheoryaxioms}

In order to make the axioms of set theory\index{axioms of set theory} a little
more compact, there are several definitions from logic that we make use of
implicitly, namely, ``logical {\sc and},''\index{conjunction ($\wedge$)}
\index{logical {\sc and} ($\wedge$)} ``logical equivalence,''\index{logical
equivalence ($\leftrightarrow$)}\index{biconditional ($\leftrightarrow$)} and
``there exists.''\index{existential quantifier ($\exists$)}

\begin{center}\begin{tabular}{rcl}
  $( \varphi \wedge \psi )$ &\mbox{stands for}& $\neg ( \varphi
     \rightarrow \neg \psi )$\\
  $( \varphi \leftrightarrow \psi )$& \mbox{stands
     for}& $( ( \varphi \rightarrow \psi ) \wedge
     ( \psi \rightarrow \varphi ) )$\\
  $\exists x \,\varphi$ &\mbox{stands for}& $\neg \forall x \neg \varphi$
\end{tabular}\end{center}

In addition, the axioms of set theory require that all variables be
dis\-tinct,\index{distinct variables}\footnote{Set theory axioms can be
devised so that {\em no} variables are required to be distinct,
provided we replace \texttt{ax-c16} with an axiom stating that ``at
least two things exist,'' thus
making \texttt{ax-5} the only other axiom requiring the
\texttt{\$d} statement.  These axioms are unconventional and are not
presented here, but they can be found on the \url{http://metamath.org}
web site.  See also the Comment on
p.~\pageref{nodd}.}\index{\texttt{\$d} statement} thus we also assume:
\begin{center}
  \texttt{\$d }$x\,y\,z\,w$
\end{center}

\needspace{2\baselineskip}
\noindent Axiom of Extensionality.\index{Axiom of Extensionality}

\setbox\startprefix=\hbox{\tt \ \ ax-ext\ \$a\ }
\setbox\contprefix=\hbox{\tt \ \ \ \ \ \ \ \ \ \ \ \ }
\startm
\m{\vdash}\m{(}\m{\forall}\m{x}\m{(}\m{x}\m{\in}\m{y}\m{\leftrightarrow}\m{x}
\m{\in}\m{z}\m{)}\m{\rightarrow}\m{y}\m{=}\m{z}\m{)}
\endm

\needspace{3\baselineskip}
\noindent Axiom of Replacement.\index{Axiom of Replacement}

\setbox\startprefix=\hbox{\tt \ \ ax-rep\ \$a\ }
\setbox\contprefix=\hbox{\tt \ \ \ \ \ \ \ \ \ \ \ \ }
\startm
\m{\vdash}\m{(}\m{\forall}\m{w}\m{\exists}\m{y}\m{\forall}\m{z}\m{(}\m{%
\forall}\m{y}\m{\varphi}\m{\rightarrow}\m{z}\m{=}\m{y}\m{)}\m{\rightarrow}\m{%
\exists}\m{y}\m{\forall}\m{z}\m{(}\m{z}\m{\in}\m{y}\m{\leftrightarrow}\m{%
\exists}\m{w}\m{(}\m{w}\m{\in}\m{x}\m{\wedge}\m{\forall}\m{y}\m{\varphi}\m{)}%
\m{)}\m{)}
\endm

\needspace{2\baselineskip}
\noindent Axiom of Union.\index{Axiom of Union}

\setbox\startprefix=\hbox{\tt \ \ ax-un\ \$a\ }
\setbox\contprefix=\hbox{\tt \ \ \ \ \ \ \ \ \ \ \ }
\startm
\m{\vdash}\m{\exists}\m{x}\m{\forall}\m{y}\m{(}\m{\exists}\m{x}\m{(}\m{y}\m{
\in}\m{x}\m{\wedge}\m{x}\m{\in}\m{z}\m{)}\m{\rightarrow}\m{y}\m{\in}\m{x}\m{)}
\endm

\needspace{2\baselineskip}
\noindent Axiom of Power Sets.\index{Axiom of Power Sets}

\setbox\startprefix=\hbox{\tt \ \ ax-pow\ \$a\ }
\setbox\contprefix=\hbox{\tt \ \ \ \ \ \ \ \ \ \ \ \ }
\startm
\m{\vdash}\m{\exists}\m{x}\m{\forall}\m{y}\m{(}\m{\forall}\m{x}\m{(}\m{x}\m{
\in}\m{y}\m{\rightarrow}\m{x}\m{\in}\m{z}\m{)}\m{\rightarrow}\m{y}\m{\in}\m{x}
\m{)}
\endm

\needspace{3\baselineskip}
\noindent Axiom of Regularity.\index{Axiom of Regularity}

\setbox\startprefix=\hbox{\tt \ \ ax-reg\ \$a\ }
\setbox\contprefix=\hbox{\tt \ \ \ \ \ \ \ \ \ \ \ \ }
\startm
\m{\vdash}\m{(}\m{\exists}\m{x}\m{\,x}\m{\in}\m{y}\m{\rightarrow}\m{\exists}
\m{x}\m{(}\m{x}\m{\in}\m{y}\m{\wedge}\m{\forall}\m{z}\m{(}\m{z}\m{\in}\m{x}\m{
\rightarrow}\m{\lnot}\m{z}\m{\in}\m{y}\m{)}\m{)}\m{)}
\endm

\needspace{3\baselineskip}
\noindent Axiom of Infinity.\index{Axiom of Infinity}

\setbox\startprefix=\hbox{\tt \ \ ax-inf\ \$a\ }
\setbox\contprefix=\hbox{\tt \ \ \ \ \ \ \ \ \ \ \ \ \ \ \ }
\startm
\m{\vdash}\m{\exists}\m{x}\m{(}\m{y}\m{\in}\m{x}\m{\wedge}\m{\forall}\m{y}%
\m{(}\m{y}\m{\in}\m{x}\m{\rightarrow}\m{\exists}\m{z}\m{(}\m{y}\m{\in}\m{z}\m{%
\wedge}\m{z}\m{\in}\m{x}\m{)}\m{)}\m{)}
\endm

\needspace{4\baselineskip}
\noindent Axiom of Choice.\index{Axiom of Choice}

\setbox\startprefix=\hbox{\tt \ \ ax-ac\ \$a\ }
\setbox\contprefix=\hbox{\tt \ \ \ \ \ \ \ \ \ \ \ \ \ \ }
\startm
\m{\vdash}\m{\exists}\m{x}\m{\forall}\m{y}\m{\forall}\m{z}\m{(}\m{(}\m{y}\m{%
\in}\m{z}\m{\wedge}\m{z}\m{\in}\m{w}\m{)}\m{\rightarrow}\m{\exists}\m{w}\m{%
\forall}\m{y}\m{(}\m{\exists}\m{w}\m{(}\m{(}\m{y}\m{\in}\m{z}\m{\wedge}\m{z}%
\m{\in}\m{w}\m{)}\m{\wedge}\m{(}\m{y}\m{\in}\m{w}\m{\wedge}\m{w}\m{\in}\m{x}%
\m{)}\m{)}\m{\leftrightarrow}\m{y}\m{=}\m{w}\m{)}\m{)}
\endm

\subsection{That's It}

There you have it, the axioms for (essentially) all of mathematics!
Wonder at them and stare at them in awe.  Put a copy in your wallet, and
you will carry in your pocket the encoding for all theorems ever proved
and that ever will be proved, from the most mundane to the most
profound.

\section{A Hierarchy of Definitions}\label{hierarchy}

The axioms in the previous section in principle embody everything that can be
done within standard mathematics.  However, it is impractical to accomplish
very much by using them directly, for even simple concepts (from a human
perspective) can involve extremely long, incomprehensible formulas.
Mathematics is made practical by introducing definitions\index{definition}.
Definitions usually introduce new symbols, or at least new relationships among
existing symbols, to abbreviate more complex formulas.  An important
requirement for a definition is that there exist a straightforward
(algorithmic) method for eliminating the abbreviation by expanding it into the
more primitive symbol string that it represents.  Some
important definitions included in
the file \texttt{set.mm} are listed in this section for reference, and also to
give you a feel for why something like $\omega$\index{omega ($\omega$)} (the
set of natural numbers\index{natural number} 0, 1, 2,\ldots) becomes very
complicated when completely expanded into primitive symbols.

What is the motivation for definitions, aside from allowing complicated
expressions to be expressed more simply?  In the case of  $\omega$, one goal is
to provide a basis for the theory of natural numbers.\index{natural number}
Before set theory was invented, a set of axioms for arithmetic, called Peano's
postulates\index{Peano's postulates}, was devised and shown to have the
properties one expects for natural numbers.  Now anyone can postulate a
set of axioms, but if the axioms are inconsistent contradictions can be derived
from them.  Once a contradiction is derived, anything can be trivially
proved, including
all the facts of arithmetic and their negations.  To ensure that an
axiom system is at least as reliable as the axioms for set theory, we can
define sets and operations on those sets that satisfy the new axioms. In the
\texttt{set.mm} Metamath database, we prove that the elements of $\omega$ satisfy
Peano's postulates, and it's a long and hard journey to get there directly
from the axioms of set theory.  But the result is confidence in the
foundations of arithmetic.  And there is another advantage:  we now have all
the tools of set theory at our disposal for manipulating objects that obey the
axioms for arithmetic.

What are the criteria we use for definitions?  First, and of utmost importance,
the definition should not be {\em creative}\index{creative
definition}\index{definition!creative}, that
is it should not allow an expression that previously qualified as a wff but
was not provable, to become provable.   Second, the definition should be {\em
eliminable}\index{definition!eliminability}, that is, there should exist an
algorithmic method for converting any expression using the definition into
a logically equivalent expression that previously qualified as a wff.

In almost all cases below, definitions connect two expressions with either
$\leftrightarrow$ or $=$.  Eliminating\footnote{Here we mean the
elimination that a human might do in his or her head.  To eliminate them as
part of a Metamath proof we would invoke one of a number of
theorems that deal with transitivity of equivalence or equality; there are
many such examples in the proofs in \texttt{set.mm}.} such a definition is a
simple matter of substituting the expression on the left-hand side ({\em
definiendum}\index{definiendum} or thing being defined) with the equivalent,
more primitive expression on the right-hand side ({\em
definiens}\index{definiens} or definition).

Often a definition has variables on the right-hand side which do not appear on
the left-hand side; these are called {\em dummy variables}.\index{dummy
variable!in definitions}  In this case, any
allowable substitution (such as a new, distinct
variable) can be used when the definition is eliminated.  Dummy variables may
be used only if they are {\em effectively bound}\index{effectively bound
variable}, meaning that the definition will remain logically equivalent upon
any substitution of a dummy variable with any other {\em qualifying
expression}\index{qualifying expression}, i.e.\ any symbol string (such as
another variable) that
meets the restrictions on the dummy variable imposed by \texttt{\$d} and
\texttt{\$f} statements.  For example, we could define a constant $\perp$
(inverted tee, meaning logical ``false'') as $( \varphi \wedge \lnot \varphi
)$, i.e.\ ``phi and not phi.''  Here $\varphi$ is effectively bound because the
definition remains logically equivalent when we replace $\varphi$ with any
other wff.  (It is actually \texttt{df-fal}
in \texttt{set.mm}, which defines $\perp$.)

There are two cases where eliminating definitions is a little more
complex.  These cases are the definitions \texttt{df-bi} and
\texttt{df-cleq}.  The first stretches the concept of a definition a
little, as in effect it ``defines a definition;'' however, it meets our
requirements for a definition in that it is eliminable and does not
strengthen the language.  Theorem \texttt{bii} shows the substitution
needed to eliminate the $\leftrightarrow$\index{logical equivalence
($\leftrightarrow$)}\index{biconditional ($\leftrightarrow$)} symbol.

Definition \texttt{df-cleq}\index{equality ($=$)} extends the usage of
the equality symbol to include ``classes''\index{class} in set theory.  The
reason it is potentially problematic is that it can lead to statements which
do not follow from logic alone but presuppose the Axiom of
Extensionality\index{Axiom of Extensionality}, so we include this axiom
as a hypothesis for the definition.  We could have made \texttt{df-cleq} directly
eliminable by introducing a new equality symbol, but have chosen not to do so
in keeping with standard textbook practice.  Definitions such as \texttt{df-cleq}
that extend the meaning of existing symbols must be introduced carefully so
that they do not lead to contradictions.  Definition \texttt{df-clel} also
extends the meaning of an existing symbol ($\in$); while it doesn't strengthen
the language like \texttt{df-cleq}, this is not obvious and it must also be
subject to the same scrutiny.

Exercise:  Study how the wff $x\in\omega$, meaning ``$x$ is a natural
number,'' could be expanded in terms of primitive symbols, starting with the
definitions \texttt{df-clel} on p.~\pageref{dfclel} and \texttt{df-om} on
p.~\pageref{dfom} and working your way back.  Don't bother to work out the
details; just make sure that you understand how you could do it in principle.
The answer is shown in the footnote on p.~\pageref{expandom}.  If you
actually do work it out, you won't get exactly the same answer because we used
a few simplifications such as discarding occurrences of $\lnot\lnot$ (double
negation).

In the definitions below, we have placed the {\sc ascii} Metamath source
below each of the formulas to help you become familiar with the
notation in the database.  For simplicity, the necessary \texttt{\$f}
and \texttt{\$d} statements are not shown.  If you are in doubt, use the
\texttt{show statement}\index{\texttt{show statement} command} command
in the Metamath program to see the full statement.
A selection of this notation is summarized in Appendix~\ref{ASCII}.

To understand the motivation for these definitions, you should consult the
references indicated:  Takeuti and Zaring \cite{Takeuti}\index{Takeuti, Gaisi},
Quine \cite{Quine}\index{Quine, Willard Van Orman}, Bell and Machover
\cite{Bell}\index{Bell, J. L.}, and Enderton \cite{Enderton}\index{Enderton,
Herbert B.}.  Our list of definitions is provided more for reference than as a
learning aid.  However, by looking at a few of them you can gain a feel for
how the hierarchy is built up.  The definitions are a representative sample of
the 130 or so in \texttt{set.mm}, but they are complete with respect to the
theorem examples we will present in Section~\ref{sometheorems}.  Also, some are
slightly different from, but logically equivalent to, the ones in \texttt{set.mm}
(some of which have been revised over time to shorten them, for example).

\subsection{Definitions for Propositional Calculus}\label{metadefprop}

The symbols $\varphi$, $\psi$, and $\chi$ represent wffs.

Our first definition introduces the biconditional
connective\footnote{The term ``connective'' is informally used to mean a
symbol that is placed between two variables or adjacent to a variable,
whereas a mathematical ``constant'' usually indicates a symbol such as
the number 0 that may replace a variable or metavariable.  From
Metamath's point of view, there is no distinction between a connective
and a constant; both are constants in the Metamath
language.}\index{connective}\index{constant} (also called logical
equivalence)\index{logical equivalence
($\leftrightarrow$)}\index{biconditional ($\leftrightarrow$)}.  Unlike
most traditional developments, we have chosen not to have a separate
symbol such as ``Df.'' to mean ``is defined as.''  Instead, we will use
the biconditional connective for this purpose, as it lets us use
logic to manipulate definitions directly.  Here we state the properties
of the biconditional connective with a carefully crafted \texttt{\$a}
statement, which effectively uses the biconditional connective to define
itself.  The $\leftrightarrow$ symbol can be eliminated from a formula
using theorem \texttt{bii}, which is derived later.

\vskip 2ex
\noindent Define the biconditional connective.\label{df-bi}

\vskip 0.5ex
\setbox\startprefix=\hbox{\tt \ \ df-bi\ \$a\ }
\setbox\contprefix=\hbox{\tt \ \ \ \ \ \ \ \ \ \ \ }
\startm
\m{\vdash}\m{\lnot}\m{(}\m{(}\m{(}\m{\varphi}\m{\leftrightarrow}\m{\psi}\m{)}%
\m{\rightarrow}\m{\lnot}\m{(}\m{(}\m{\varphi}\m{\rightarrow}\m{\psi}\m{)}\m{%
\rightarrow}\m{\lnot}\m{(}\m{\psi}\m{\rightarrow}\m{\varphi}\m{)}\m{)}\m{)}\m{%
\rightarrow}\m{\lnot}\m{(}\m{\lnot}\m{(}\m{(}\m{\varphi}\m{\rightarrow}\m{%
\psi}\m{)}\m{\rightarrow}\m{\lnot}\m{(}\m{\psi}\m{\rightarrow}\m{\varphi}\m{)}%
\m{)}\m{\rightarrow}\m{(}\m{\varphi}\m{\leftrightarrow}\m{\psi}\m{)}\m{)}\m{)}
\endm
\begin{mmraw}%
|- -. ( ( ( ph <-> ps ) -> -. ( ( ph -> ps ) ->
-. ( ps -> ph ) ) ) -> -. ( -. ( ( ph -> ps ) -> -. (
ps -> ph ) ) -> ( ph <-> ps ) ) ) \$.
\end{mmraw}

\noindent This theorem relates the biconditional connective to primitive
connectives and can be used to eliminate the $\leftrightarrow$ symbol from any
wff.

\vskip 0.5ex
\setbox\startprefix=\hbox{\tt \ \ bii\ \$p\ }
\setbox\contprefix=\hbox{\tt \ \ \ \ \ \ \ \ \ }
\startm
\m{\vdash}\m{(}\m{(}\m{\varphi}\m{\leftrightarrow}\m{\psi}\m{)}\m{\leftrightarrow}
\m{\lnot}\m{(}\m{(}\m{\varphi}\m{\rightarrow}\m{\psi}\m{)}\m{\rightarrow}\m{\lnot}
\m{(}\m{\psi}\m{\rightarrow}\m{\varphi}\m{)}\m{)}\m{)}
\endm
\begin{mmraw}%
|- ( ( ph <-> ps ) <-> -. ( ( ph -> ps ) -> -. ( ps -> ph ) ) ) \$= ... \$.
\end{mmraw}

\noindent Define disjunction ({\sc or}).\index{disjunction ($\vee$)}%
\index{logical or (vee)@logical {\sc or} ($\vee$)}%
\index{df-or@\texttt{df-or}}\label{df-or}

\vskip 0.5ex
\setbox\startprefix=\hbox{\tt \ \ df-or\ \$a\ }
\setbox\contprefix=\hbox{\tt \ \ \ \ \ \ \ \ \ \ \ }
\startm
\m{\vdash}\m{(}\m{(}\m{\varphi}\m{\vee}\m{\psi}\m{)}\m{\leftrightarrow}\m{(}\m{
\lnot}\m{\varphi}\m{\rightarrow}\m{\psi}\m{)}\m{)}
\endm
\begin{mmraw}%
|- ( ( ph \TOR ps ) <-> ( -. ph -> ps ) ) \$.
\end{mmraw}

\noindent Define conjunction ({\sc and}).\index{conjunction ($\wedge$)}%
\index{logical {\sc and} ($\wedge$)}%
\index{df-an@\texttt{df-an}}\label{df-an}

\vskip 0.5ex
\setbox\startprefix=\hbox{\tt \ \ df-an\ \$a\ }
\setbox\contprefix=\hbox{\tt \ \ \ \ \ \ \ \ \ \ \ }
\startm
\m{\vdash}\m{(}\m{(}\m{\varphi}\m{\wedge}\m{\psi}\m{)}\m{\leftrightarrow}\m{\lnot}
\m{(}\m{\varphi}\m{\rightarrow}\m{\lnot}\m{\psi}\m{)}\m{)}
\endm
\begin{mmraw}%
|- ( ( ph \TAND ps ) <-> -. ( ph -> -. ps ) ) \$.
\end{mmraw}

\noindent Define disjunction ({\sc or}) of 3 wffs.%
\index{df-3or@\texttt{df-3or}}\label{df-3or}

\vskip 0.5ex
\setbox\startprefix=\hbox{\tt \ \ df-3or\ \$a\ }
\setbox\contprefix=\hbox{\tt \ \ \ \ \ \ \ \ \ \ \ \ }
\startm
\m{\vdash}\m{(}\m{(}\m{\varphi}\m{\vee}\m{\psi}\m{\vee}\m{\chi}\m{)}\m{
\leftrightarrow}\m{(}\m{(}\m{\varphi}\m{\vee}\m{\psi}\m{)}\m{\vee}\m{\chi}\m{)}
\m{)}
\endm
\begin{mmraw}%
|- ( ( ph \TOR ps \TOR ch ) <-> ( ( ph \TOR ps ) \TOR ch ) ) \$.
\end{mmraw}

\noindent Define conjunction ({\sc and}) of 3 wffs.%
\index{df-3an}\label{df-3an}

\vskip 0.5ex
\setbox\startprefix=\hbox{\tt \ \ df-3an\ \$a\ }
\setbox\contprefix=\hbox{\tt \ \ \ \ \ \ \ \ \ \ \ \ }
\startm
\m{\vdash}\m{(}\m{(}\m{\varphi}\m{\wedge}\m{\psi}\m{\wedge}\m{\chi}\m{)}\m{
\leftrightarrow}\m{(}\m{(}\m{\varphi}\m{\wedge}\m{\psi}\m{)}\m{\wedge}\m{\chi}
\m{)}\m{)}
\endm

\begin{mmraw}%
|- ( ( ph \TAND ps \TAND ch ) <-> ( ( ph \TAND ps ) \TAND ch ) ) \$.
\end{mmraw}

\subsection{Definitions for Predicate Calculus}\label{metadefpred}

The symbols $x$, $y$, and $z$ represent individual variables of predicate
calculus.  In this section, they are not necessarily distinct unless it is
explicitly
mentioned.

\vskip 2ex
\noindent Define existential quantification.
The expression $\exists x \varphi$ means
``there exists an $x$ where $\varphi$ is true.''\index{existential quantifier
($\exists$)}\label{df-ex}

\vskip 0.5ex
\setbox\startprefix=\hbox{\tt \ \ df-ex\ \$a\ }
\setbox\contprefix=\hbox{\tt \ \ \ \ \ \ \ \ \ \ \ }
\startm
\m{\vdash}\m{(}\m{\exists}\m{x}\m{\varphi}\m{\leftrightarrow}\m{\lnot}\m{\forall}
\m{x}\m{\lnot}\m{\varphi}\m{)}
\endm
\begin{mmraw}%
|- ( E. x ph <-> -. A. x -. ph ) \$.
\end{mmraw}

\noindent Define proper substitution.\index{proper
substitution}\index{substitution!proper}\label{df-sb}
In our notation, we use $[ y / x ] \varphi$ to mean ``the wff that
results when $y$ is properly substituted for $x$ in the wff
$\varphi$.''\footnote{
This can also be described
as substituting $x$ with $y$, $y$ properly replaces $x$, or
$x$ is properly replaced by $y$.}
% This is elsb4, though it currently says: ( [ x / y ] z e. y <-> z e. x )
For example,
$[ y / x ] z \in x$ is the same as $z \in y$.
One way to remember this notation is to notice that it looks like division
and recall that $( y / x ) \cdot x $ is $y$ (when $x \neq 0$).
The notation is different from the notation $\varphi ( x | y )$
that is sometimes used, because the latter notation is ambiguous for us:
for example, we don't know whether $\lnot \varphi ( x | y )$ is to be
interpreted as $\lnot ( \varphi ( x | y ) )$ or
$( \lnot \varphi ) ( x | y )$.\footnote{Because of the way
we initially defined wffs, this is the case
with any postfix connective\index{postfix connective} (one occurring after the
symbols being connected) or infix connective\index{infix connective} (one
occurring between the symbols being connected).  Metamath does not have a
built-in notion of operator binding strength that could eliminate the
ambiguity.  The initial parenthesis effectively provides a prefix
connective\index{prefix connective} to eliminate ambiguity.  Some conventions,
such as Polish notation\index{Polish notation} used in the 1930's and 1940's
by Polish logicians, use only prefix connectives and thus allow the total
elimination of parentheses, at the expense of readability.  In Metamath we
could actually redefine all notation to be Polish if we wanted to without
having to change any proofs!}  Other texts often use $\varphi(y)$ to indicate
our $[ y / x ] \varphi$, but this notation is even more ambiguous since there is
no explicit indication of what is being substituted.
Note that this
definition is valid even when
$x$ and $y$ are the same variable.  The first conjunct is a ``trick'' used to
achieve this property, making the definition look somewhat peculiar at
first.

\vskip 0.5ex
\setbox\startprefix=\hbox{\tt \ \ df-sb\ \$a\ }
\setbox\contprefix=\hbox{\tt \ \ \ \ \ \ \ \ \ \ \ }
\startm
\m{\vdash}\m{(}\m{[}\m{y}\m{/}\m{x}\m{]}\m{\varphi}\m{\leftrightarrow}\m{(}%
\m{(}\m{x}\m{=}\m{y}\m{\rightarrow}\m{\varphi}\m{)}\m{\wedge}\m{\exists}\m{x}%
\m{(}\m{x}\m{=}\m{y}\m{\wedge}\m{\varphi}\m{)}\m{)}\m{)}
\endm
\begin{mmraw}%
|- ( [ y / x ] ph <-> ( ( x = y -> ph ) \TAND E. x ( x = y \TAND ph ) ) ) \$.
\end{mmraw}


\noindent Define existential uniqueness\index{existential uniqueness
quantifier ($\exists "!$)} (``there exists exactly one'').  Note that $y$ is a
variable distinct from $x$ and not occurring in $\varphi$.\label{df-eu}

\vskip 0.5ex
\setbox\startprefix=\hbox{\tt \ \ df-eu\ \$a\ }
\setbox\contprefix=\hbox{\tt \ \ \ \ \ \ \ \ \ \ \ }
\startm
\m{\vdash}\m{(}\m{\exists}\m{{!}}\m{x}\m{\varphi}\m{\leftrightarrow}\m{\exists}
\m{y}\m{\forall}\m{x}\m{(}\m{\varphi}\m{\leftrightarrow}\m{x}\m{=}\m{y}\m{)}\m{)}
\endm

\begin{mmraw}%
|- ( E! x ph <-> E. y A. x ( ph <-> x = y ) ) \$.
\end{mmraw}

\subsection{Definitions for Set Theory}\label{setdefinitions}

The symbols $x$, $y$, $z$, and $w$ represent individual variables of
predicate calculus, which in set theory are understood to be sets.
However, using only the constructs shown so far would be very inconvenient.

To make set theory more practical, we introduce the notion of a ``class.''
A class\index{class} is either a set variable (such as $x$) or an
expression of the form $\{ x | \varphi\}$ (called an ``abstraction
class''\index{abstraction class}\index{class abstraction}).  Note that
sets (i.e.\ individual variables) always exist (this is a theorem of
logic, namely $\exists y \, y = x$ for any set $x$), whereas classes may
or may not exist (i.e.\ $\exists y \, y = A$ may or may not be true).
If a class does not exist it is called a ``proper class.''\index{proper
class}\index{class!proper} Definitions \texttt{df-clab},
\texttt{df-cleq}, and \texttt{df-clel} can be used to convert an
expression containing classes into one containing only set variables and
wff metavariables.

The symbols $A$, $B$, $C$, $D$, $ F$, $G$, and $R$ are metavariables that range
over classes.  A class metavariable $A$ may be eliminated from a wff by
replacing it with $\{ x|\varphi\}$ where neither $x$ nor $\varphi$ occur in
the wff.

The theory of classes can be shown to be an eliminable and conservative
extension of set theory. The \textbf{eliminability}
property shows that for every
formula in the extended language we can build a logically equivalent
formula in the basic language; so that even if the extended language
provides more ease to convey and formulate mathematical ideas for set
theory, its expressive power does not in fact strengthen the basic
language's expressive power.
The \textbf{conservation} property shows that for
every proof of a formula of the basic language in the extended system
we can build another proof of the same formula in the basic system;
so that, concerning theorems on sets only, the deductive powers of
the extended system and of the basic system are identical. Together,
these properties mean that the extended language can be treated as a
definitional extension that is \textbf{sound}.

A rigorous justification, which we will not give here, can be found in
Levy \cite[pp.~357-366]{Levy} supplementing his informal introduction to class
theory on pp.~7-17. Two other good treatments of class theory are provided
by Quine \cite[pp.~15-21]{Quine}\index{Quine, Willard Van Orman}
and also \cite[pp.~10-14]{Takeuti}\index{Takeuti, Gaisi}.
Quine's exposition (he calls them virtual classes)
is nicely written and very readable.

In the rest of this
section, individual variables are always assumed to be distinct from
each other unless otherwise indicated.  In addition, dummy variables on the
right-hand side of a definition do not occur in the class and wff
metavariables in the definition.

The definitions we present here are a partial but self-contained
collection selected from several hundred that appear in the current
\texttt{set.mm} database.  They are adequate for a basic development of
elementary set theory.

\vskip 2ex
\noindent Define the abstraction class.\index{abstraction class}\index{class
abstraction}\label{df-clab}  $x$ and $y$
need not be distinct.  Definition 2.1 of Quine, p.~16.  This definition may
seem puzzling since it is shorter than the expression being defined and does not
buy us anything in terms of brevity.  The reason we introduce this definition
is because it fits in neatly with the extension of the $\in$ connective
provided by \texttt{df-clel}.

\vskip 0.5ex
\setbox\startprefix=\hbox{\tt \ \ df-clab\ \$a\ }
\setbox\contprefix=\hbox{\tt \ \ \ \ \ \ \ \ \ \ \ \ \ }
\startm
\m{\vdash}\m{(}\m{x}\m{\in}\m{\{}\m{y}\m{|}\m{\varphi}\m{\}}\m{%
\leftrightarrow}\m{[}\m{x}\m{/}\m{y}\m{]}\m{\varphi}\m{)}
\endm
\begin{mmraw}%
|- ( x e. \{ y | ph \} <-> [ x / y ] ph ) \$.
\end{mmraw}

\noindent Define the equality connective between classes\index{class
equality}\label{df-cleq}.  See Quine or Chapter 4 of Takeuti and Zaring for its
justification and methods for eliminating it.  This is an example of a
somewhat ``dangerous'' definition, because it extends the use of the
existing equality symbol rather than introducing a new symbol, allowing
us to make statements in the original language that may not be true.
For example, it permits us to deduce $y = z \leftrightarrow \forall x (
x \in y \leftrightarrow x \in z )$ which is not a theorem of logic but
rather presupposes the Axiom of Extensionality,\index{Axiom of
Extensionality} which we include as a hypothesis so that we can know
when this axiom is assumed in a proof (with the \texttt{show
trace{\char`\_}back} command).  We could avoid the danger by introducing
another symbol, say $\eqcirc$, in place of $=$; this
would also have the advantage of making elimination of the definition
straightforward and would eliminate the need for Extensionality as a
hypothesis.  We would then also have the advantage of being able to
identify exactly where Extensionality truly comes into play.  One of our
theorems would be $x \eqcirc y \leftrightarrow x = y$
by invoking Extensionality.  However in keeping with standard practice
we retain the ``dangerous'' definition.

\vskip 0.5ex
\setbox\startprefix=\hbox{\tt \ \ df-cleq.1\ \$e\ }
\setbox\contprefix=\hbox{\tt \ \ \ \ \ \ \ \ \ \ \ \ \ \ \ }
\startm
\m{\vdash}\m{(}\m{\forall}\m{x}\m{(}\m{x}\m{\in}\m{y}\m{\leftrightarrow}\m{x}
\m{\in}\m{z}\m{)}\m{\rightarrow}\m{y}\m{=}\m{z}\m{)}
\endm
\setbox\startprefix=\hbox{\tt \ \ df-cleq\ \$a\ }
\setbox\contprefix=\hbox{\tt \ \ \ \ \ \ \ \ \ \ \ \ \ }
\startm
\m{\vdash}\m{(}\m{A}\m{=}\m{B}\m{\leftrightarrow}\m{\forall}\m{x}\m{(}\m{x}\m{
\in}\m{A}\m{\leftrightarrow}\m{x}\m{\in}\m{B}\m{)}\m{)}
\endm
% We need to reset the startprefix and contprefix.
\setbox\startprefix=\hbox{\tt \ \ df-cleq.1\ \$e\ }
\setbox\contprefix=\hbox{\tt \ \ \ \ \ \ \ \ \ \ \ \ \ \ \ }
\begin{mmraw}%
|- ( A. x ( x e. y <-> x e. z ) -> y = z ) \$.
\end{mmraw}
\setbox\startprefix=\hbox{\tt \ \ df-cleq\ \$a\ }
\setbox\contprefix=\hbox{\tt \ \ \ \ \ \ \ \ \ \ \ \ \ }
\begin{mmraw}%
|- ( A = B <-> A. x ( x e. A <-> x e. B ) ) \$.
\end{mmraw}

\noindent Define the membership connective between classes\index{class
membership}.  Theorem 6.3 of Quine, p.~41, which we adopt as a definition.
Note that it extends the use of the existing membership symbol, but unlike
{\tt df-cleq} it does not extend the set of valid wffs of logic when the class
metavariables are replaced with set variables.\label{dfclel}\label{df-clel}

\vskip 0.5ex
\setbox\startprefix=\hbox{\tt \ \ df-clel\ \$a\ }
\setbox\contprefix=\hbox{\tt \ \ \ \ \ \ \ \ \ \ \ \ \ }
\startm
\m{\vdash}\m{(}\m{A}\m{\in}\m{B}\m{\leftrightarrow}\m{\exists}\m{x}\m{(}\m{x}
\m{=}\m{A}\m{\wedge}\m{x}\m{\in}\m{B}\m{)}\m{)}
\endm
\begin{mmraw}%
|- ( A e. B <-> E. x ( x = A \TAND x e. B ) ) \$.?
\end{mmraw}

\noindent Define inequality.

\vskip 0.5ex
\setbox\startprefix=\hbox{\tt \ \ df-ne\ \$a\ }
\setbox\contprefix=\hbox{\tt \ \ \ \ \ \ \ \ \ \ \ }
\startm
\m{\vdash}\m{(}\m{A}\m{\ne}\m{B}\m{\leftrightarrow}\m{\lnot}\m{A}\m{=}\m{B}%
\m{)}
\endm
\begin{mmraw}%
|- ( A =/= B <-> -. A = B ) \$.
\end{mmraw}

\noindent Define restricted universal quantification.\index{universal
quantifier ($\forall$)!restricted}  Enderton, p.~22.

\vskip 0.5ex
\setbox\startprefix=\hbox{\tt \ \ df-ral\ \$a\ }
\setbox\contprefix=\hbox{\tt \ \ \ \ \ \ \ \ \ \ \ \ }
\startm
\m{\vdash}\m{(}\m{\forall}\m{x}\m{\in}\m{A}\m{\varphi}\m{\leftrightarrow}\m{%
\forall}\m{x}\m{(}\m{x}\m{\in}\m{A}\m{\rightarrow}\m{\varphi}\m{)}\m{)}
\endm
\begin{mmraw}%
|- ( A. x e. A ph <-> A. x ( x e. A -> ph ) ) \$.
\end{mmraw}

\noindent Define restricted existential quantification.\index{existential
quantifier ($\exists$)!restricted}  Enderton, p.~22.

\vskip 0.5ex
\setbox\startprefix=\hbox{\tt \ \ df-rex\ \$a\ }
\setbox\contprefix=\hbox{\tt \ \ \ \ \ \ \ \ \ \ \ \ }
\startm
\m{\vdash}\m{(}\m{\exists}\m{x}\m{\in}\m{A}\m{\varphi}\m{\leftrightarrow}\m{%
\exists}\m{x}\m{(}\m{x}\m{\in}\m{A}\m{\wedge}\m{\varphi}\m{)}\m{)}
\endm
\begin{mmraw}%
|- ( E. x e. A ph <-> E. x ( x e. A \TAND ph ) ) \$.
\end{mmraw}

\noindent Define the universal class\index{universal class ($V$)}.  Definition
5.20, p.~21, of Takeuti and Zaring.\label{df-v}

\vskip 0.5ex
\setbox\startprefix=\hbox{\tt \ \ df-v\ \$a\ }
\setbox\contprefix=\hbox{\tt \ \ \ \ \ \ \ \ \ \ }
\startm
\m{\vdash}\m{{\rm V}}\m{=}\m{\{}\m{x}\m{|}\m{x}\m{=}\m{x}\m{\}}
\endm
\begin{mmraw}%
|- {\char`\_}V = \{ x | x = x \} \$.
\end{mmraw}

\noindent Define the subclass\index{subclass}\index{subset} relationship
between two classes (called the subset relation if the classes are sets i.e.\
are not proper).  Definition 5.9 of Takeuti and Zaring, p.~17.\label{df-ss}

\vskip 0.5ex
\setbox\startprefix=\hbox{\tt \ \ df-ss\ \$a\ }
\setbox\contprefix=\hbox{\tt \ \ \ \ \ \ \ \ \ \ \ }
\startm
\m{\vdash}\m{(}\m{A}\m{\subseteq}\m{B}\m{\leftrightarrow}\m{\forall}\m{x}\m{(}
\m{x}\m{\in}\m{A}\m{\rightarrow}\m{x}\m{\in}\m{B}\m{)}\m{)}
\endm
\begin{mmraw}%
|- ( A C\_ B <-> A. x ( x e. A -> x e. B ) ) \$.
\end{mmraw}

\noindent Define the union\index{union} of two classes.  Definition 5.6 of Takeuti and Zaring,
p.~16.\label{df-un}

\vskip 0.5ex
\setbox\startprefix=\hbox{\tt \ \ df-un\ \$a\ }
\setbox\contprefix=\hbox{\tt \ \ \ \ \ \ \ \ \ \ \ }
\startm
\m{\vdash}\m{(}\m{A}\m{\cup}\m{B}\m{)}\m{=}\m{\{}\m{x}\m{|}\m{(}\m{x}\m{\in}
\m{A}\m{\vee}\m{x}\m{\in}\m{B}\m{)}\m{\}}
\endm
\begin{mmraw}%
( A u. B ) = \{ x | ( x e. A \TOR x e. B ) \} \$.
\end{mmraw}

\noindent Define the intersection\index{intersection} of two classes.  Definition 5.6 of
Takeuti and Zaring, p.~16.\label{df-in}

\vskip 0.5ex
\setbox\startprefix=\hbox{\tt \ \ df-in\ \$a\ }
\setbox\contprefix=\hbox{\tt \ \ \ \ \ \ \ \ \ \ \ }
\startm
\m{\vdash}\m{(}\m{A}\m{\cap}\m{B}\m{)}\m{=}\m{\{}\m{x}\m{|}\m{(}\m{x}\m{\in}
\m{A}\m{\wedge}\m{x}\m{\in}\m{B}\m{)}\m{\}}
\endm
% Caret ^ requires special treatment
\begin{mmraw}%
|- ( A i\^{}i B ) = \{ x | ( x e. A \TAND x e. B ) \} \$.
\end{mmraw}

\noindent Define class difference\index{class difference}\index{set difference}.
Definition 5.12 of Takeuti and Zaring, p.~20.  Several notations are used in
the literature; we chose the $\setminus$ convention instead of a minus sign to
reserve the latter for later use in, e.g., arithmetic.\label{df-dif}

\vskip 0.5ex
\setbox\startprefix=\hbox{\tt \ \ df-dif\ \$a\ }
\setbox\contprefix=\hbox{\tt \ \ \ \ \ \ \ \ \ \ \ \ }
\startm
\m{\vdash}\m{(}\m{A}\m{\setminus}\m{B}\m{)}\m{=}\m{\{}\m{x}\m{|}\m{(}\m{x}\m{
\in}\m{A}\m{\wedge}\m{\lnot}\m{x}\m{\in}\m{B}\m{)}\m{\}}
\endm
\begin{mmraw}%
( A \SLASH B ) = \{ x | ( x e. A \TAND -. x e. B ) \} \$.
\end{mmraw}

\noindent Define the empty or null set\index{empty set}\index{null set}.
Compare  Definition 5.14 of Takeuti and Zaring, p.~20.\label{df-nul}

\vskip 0.5ex
\setbox\startprefix=\hbox{\tt \ \ df-nul\ \$a\ }
\setbox\contprefix=\hbox{\tt \ \ \ \ \ \ \ \ \ \ }
\startm
\m{\vdash}\m{\varnothing}\m{=}\m{(}\m{{\rm V}}\m{\setminus}\m{{\rm V}}\m{)}
\endm
\begin{mmraw}%
|- (/) = ( {\char`\_}V \SLASH {\char`\_}V ) \$.
\end{mmraw}

\noindent Define power class\index{power set}\index{power class}.  Definition 5.10 of
Takeuti and Zaring, p.~17, but we also let it apply to proper classes.  (Note
that \verb$~P$ is the symbol for calligraphic P, the tilde
suggesting ``curly;'' see Appendix~\ref{ASCII}.)\label{df-pw}

\vskip 0.5ex
\setbox\startprefix=\hbox{\tt \ \ df-pw\ \$a\ }
\setbox\contprefix=\hbox{\tt \ \ \ \ \ \ \ \ \ \ \ }
\startm
\m{\vdash}\m{{\cal P}}\m{A}\m{=}\m{\{}\m{x}\m{|}\m{x}\m{\subseteq}\m{A}\m{\}}
\endm
% Special incantation required to put ~ into the text
\begin{mmraw}%
|- \char`\~P~A = \{ x | x C\_ A \} \$.
\end{mmraw}

\noindent Define the singleton of a class\index{singleton}.  Definition 7.1 of
Quine, p.~48.  It is well-defined for proper classes, although
it is not very meaningful in this case, where it evaluates to the empty
set.

\vskip 0.5ex
\setbox\startprefix=\hbox{\tt \ \ df-sn\ \$a\ }
\setbox\contprefix=\hbox{\tt \ \ \ \ \ \ \ \ \ \ \ }
\startm
\m{\vdash}\m{\{}\m{A}\m{\}}\m{=}\m{\{}\m{x}\m{|}\m{x}\m{=}\m{A}\m{\}}
\endm
\begin{mmraw}%
|- \{ A \} = \{ x | x = A \} \$.
\end{mmraw}%

\noindent Define an unordered pair of classes\index{unordered pair}\index{pair}.  Definition
7.1 of Quine, p.~48.

\vskip 0.5ex
\setbox\startprefix=\hbox{\tt \ \ df-pr\ \$a\ }
\setbox\contprefix=\hbox{\tt \ \ \ \ \ \ \ \ \ \ \ }
\startm
\m{\vdash}\m{\{}\m{A}\m{,}\m{B}\m{\}}\m{=}\m{(}\m{\{}\m{A}\m{\}}\m{\cup}\m{\{}
\m{B}\m{\}}\m{)}
\endm
\begin{mmraw}%
|- \{ A , B \} = ( \{ A \} u. \{ B \} ) \$.
\end{mmraw}

\noindent Define an unordered triple of classes\index{unordered triple}.  Definition of
Enderton, p.~19.

\vskip 0.5ex
\setbox\startprefix=\hbox{\tt \ \ df-tp\ \$a\ }
\setbox\contprefix=\hbox{\tt \ \ \ \ \ \ \ \ \ \ \ }
\startm
\m{\vdash}\m{\{}\m{A}\m{,}\m{B}\m{,}\m{C}\m{\}}\m{=}\m{(}\m{\{}\m{A}\m{,}\m{B}
\m{\}}\m{\cup}\m{\{}\m{C}\m{\}}\m{)}
\endm
\begin{mmraw}%
|- \{ A , B , C \} = ( \{ A , B \} u. \{ C \} ) \$.
\end{mmraw}%

\noindent Kuratowski's\index{Kuratowski, Kazimierz} ordered pair\index{ordered
pair} definition.  Definition 9.1 of Quine, p.~58. For proper classes it is
not meaningful but is well-defined for convenience.  (Note that \verb$<.$
stands for $\langle$ whereas \verb$<$ stands for $<$, and similarly for
\verb$>.$\,.)\label{df-op}

\vskip 0.5ex
\setbox\startprefix=\hbox{\tt \ \ df-op\ \$a\ }
\setbox\contprefix=\hbox{\tt \ \ \ \ \ \ \ \ \ \ \ }
\startm
\m{\vdash}\m{\langle}\m{A}\m{,}\m{B}\m{\rangle}\m{=}\m{\{}\m{\{}\m{A}\m{\}}
\m{,}\m{\{}\m{A}\m{,}\m{B}\m{\}}\m{\}}
\endm
\begin{mmraw}%
|- <. A , B >. = \{ \{ A \} , \{ A , B \} \} \$.
\end{mmraw}

\noindent Define the union of a class\index{union}.  Definition 5.5, p.~16,
of Takeuti and Zaring.

\vskip 0.5ex
\setbox\startprefix=\hbox{\tt \ \ df-uni\ \$a\ }
\setbox\contprefix=\hbox{\tt \ \ \ \ \ \ \ \ \ \ \ \ }
\startm
\m{\vdash}\m{\bigcup}\m{A}\m{=}\m{\{}\m{x}\m{|}\m{\exists}\m{y}\m{(}\m{x}\m{
\in}\m{y}\m{\wedge}\m{y}\m{\in}\m{A}\m{)}\m{\}}
\endm
\begin{mmraw}%
|- U. A = \{ x | E. y ( x e. y \TAND y e. A ) \} \$.
\end{mmraw}

\noindent Define the intersection\index{intersection} of a class.  Definition 7.35,
p.~44, of Takeuti and Zaring.

\vskip 0.5ex
\setbox\startprefix=\hbox{\tt \ \ df-int\ \$a\ }
\setbox\contprefix=\hbox{\tt \ \ \ \ \ \ \ \ \ \ \ \ }
\startm
\m{\vdash}\m{\bigcap}\m{A}\m{=}\m{\{}\m{x}\m{|}\m{\forall}\m{y}\m{(}\m{y}\m{
\in}\m{A}\m{\rightarrow}\m{x}\m{\in}\m{y}\m{)}\m{\}}
\endm
\begin{mmraw}%
|- |\^{}| A = \{ x | A. y ( y e. A -> x e. y ) \} \$.
\end{mmraw}

\noindent Define a transitive class\index{transitive class}\index{transitive
set}.  This should not be confused with a transitive relation, which is a different
concept.  Definition from p.~71 of Enderton, extended to classes.

\vskip 0.5ex
\setbox\startprefix=\hbox{\tt \ \ df-tr\ \$a\ }
\setbox\contprefix=\hbox{\tt \ \ \ \ \ \ \ \ \ \ \ }
\startm
\m{\vdash}\m{(}\m{\mbox{\rm Tr}}\m{A}\m{\leftrightarrow}\m{\bigcup}\m{A}\m{
\subseteq}\m{A}\m{)}
\endm
\begin{mmraw}%
|- ( Tr A <-> U. A C\_ A ) \$.
\end{mmraw}

\noindent Define a notation for a general binary relation\index{binary
relation}.  Definition 6.18, p.~29, of Takeuti and Zaring, generalized to
arbitrary classes.  This definition is well-defined, although not very
meaningful, when classes $A$ and/or $B$ are proper.\label{dfbr}  The lack of
parentheses (or any other connective) creates no ambiguity since we are defining
an atomic wff.

\vskip 0.5ex
\setbox\startprefix=\hbox{\tt \ \ df-br\ \$a\ }
\setbox\contprefix=\hbox{\tt \ \ \ \ \ \ \ \ \ \ \ }
\startm
\m{\vdash}\m{(}\m{A}\m{\,R}\m{\,B}\m{\leftrightarrow}\m{\langle}\m{A}\m{,}\m{B}
\m{\rangle}\m{\in}\m{R}\m{)}
\endm
\begin{mmraw}%
|- ( A R B <-> <. A , B >. e. R ) \$.
\end{mmraw}

\noindent Define an abstraction class of ordered pairs\index{abstraction
class!of ordered
pairs}.  A special case of Definition 4.16, p.~14, of Takeuti and Zaring.
Note that $ z $ must be distinct from $ x $ and $ y $,
and $ z $ must not occur in $\varphi$, but $ x $ and $ y $ may be
identical and may appear in $\varphi$.

\vskip 0.5ex
\setbox\startprefix=\hbox{\tt \ \ df-opab\ \$a\ }
\setbox\contprefix=\hbox{\tt \ \ \ \ \ \ \ \ \ \ \ \ \ }
\startm
\m{\vdash}\m{\{}\m{\langle}\m{x}\m{,}\m{y}\m{\rangle}\m{|}\m{\varphi}\m{\}}\m{=}
\m{\{}\m{z}\m{|}\m{\exists}\m{x}\m{\exists}\m{y}\m{(}\m{z}\m{=}\m{\langle}\m{x}
\m{,}\m{y}\m{\rangle}\m{\wedge}\m{\varphi}\m{)}\m{\}}
\endm

\begin{mmraw}%
|- \{ <. x , y >. | ph \} = \{ z | E. x E. y ( z =
<. x , y >. /\ ph ) \} \$.
\end{mmraw}

\noindent Define the epsilon relation\index{epsilon relation}.  Similar to Definition
6.22, p.~30, of Takeuti and Zaring.

\vskip 0.5ex
\setbox\startprefix=\hbox{\tt \ \ df-eprel\ \$a\ }
\setbox\contprefix=\hbox{\tt \ \ \ \ \ \ \ \ \ \ \ \ \ \ }
\startm
\m{\vdash}\m{{\rm E}}\m{=}\m{\{}\m{\langle}\m{x}\m{,}\m{y}\m{\rangle}\m{|}\m{x}\m{
\in}\m{y}\m{\}}
\endm
\begin{mmraw}%
|- \_E = \{ <. x , y >. | x e. y \} \$.
\end{mmraw}

\noindent Define a founded relation\index{founded relation}.  $R$ is a founded
relation on $A$ iff\index{iff} (if and only if) each nonempty subset of $A$
has an ``$R$-minimal element.''  Similar to Definition 6.21, p.~30, of
Takeuti and Zaring.

\vskip 0.5ex
\setbox\startprefix=\hbox{\tt \ \ df-fr\ \$a\ }
\setbox\contprefix=\hbox{\tt \ \ \ \ \ \ \ \ \ \ \ }
\startm
\m{\vdash}\m{(}\m{R}\m{\,\mbox{\rm Fr}}\m{\,A}\m{\leftrightarrow}\m{\forall}\m{x}
\m{(}\m{(}\m{x}\m{\subseteq}\m{A}\m{\wedge}\m{\lnot}\m{x}\m{=}\m{\varnothing}
\m{)}\m{\rightarrow}\m{\exists}\m{y}\m{(}\m{y}\m{\in}\m{x}\m{\wedge}\m{(}\m{x}
\m{\cap}\m{\{}\m{z}\m{|}\m{z}\m{\,R}\m{\,y}\m{\}}\m{)}\m{=}\m{\varnothing}\m{)}
\m{)}\m{)}
\endm
\begin{mmraw}%
|- ( R Fr A <-> A. x ( ( x C\_ A \TAND -. x = (/) ) ->
E. y ( y e. x \TAND ( x i\^{}i \{ z | z R y \} ) = (/) ) ) ) \$.
\end{mmraw}

\noindent Define a well-ordering\index{well-ordering}.  $R$ is a well-ordering of $A$ iff
it is founded on $A$ and the elements of $A$ are pairwise $R$-comparable.
Similar to Definition 6.24(2), p.~30, of Takeuti and Zaring.

\vskip 0.5ex
\setbox\startprefix=\hbox{\tt \ \ df-we\ \$a\ }
\setbox\contprefix=\hbox{\tt \ \ \ \ \ \ \ \ \ \ \ }
\startm
\m{\vdash}\m{(}\m{R}\m{\,\mbox{\rm We}}\m{\,A}\m{\leftrightarrow}\m{(}\m{R}\m{\,
\mbox{\rm Fr}}\m{\,A}\m{\wedge}\m{\forall}\m{x}\m{\forall}\m{y}\m{(}\m{(}\m{x}\m{
\in}\m{A}\m{\wedge}\m{y}\m{\in}\m{A}\m{)}\m{\rightarrow}\m{(}\m{x}\m{\,R}\m{\,y}
\m{\vee}\m{x}\m{=}\m{y}\m{\vee}\m{y}\m{\,R}\m{\,x}\m{)}\m{)}\m{)}\m{)}
\endm
\begin{mmraw}%
( R We A <-> ( R Fr A \TAND A. x A. y ( ( x e.
A \TAND y e. A ) -> ( x R y \TOR x = y \TOR y R x ) ) ) ) \$.
\end{mmraw}

\noindent Define the ordinal predicate\index{ordinal predicate}, which is true for a class
that is transitive and is well-ordered by the epsilon relation.  Similar to
definition on p.~468, Bell and Machover.

\vskip 0.5ex
\setbox\startprefix=\hbox{\tt \ \ df-ord\ \$a\ }
\setbox\contprefix=\hbox{\tt \ \ \ \ \ \ \ \ \ \ \ \ }
\startm
\m{\vdash}\m{(}\m{\mbox{\rm Ord}}\m{\,A}\m{\leftrightarrow}\m{(}
\m{\mbox{\rm Tr}}\m{\,A}\m{\wedge}\m{E}\m{\,\mbox{\rm We}}\m{\,A}\m{)}\m{)}
\endm
\begin{mmraw}%
|- ( Ord A <-> ( Tr A \TAND E We A ) ) \$.
\end{mmraw}

\noindent Define the class of all ordinal numbers\index{ordinal number}.  An ordinal number is
a set that satisfies the ordinal predicate.  Definition 7.11 of Takeuti and
Zaring, p.~38.

\vskip 0.5ex
\setbox\startprefix=\hbox{\tt \ \ df-on\ \$a\ }
\setbox\contprefix=\hbox{\tt \ \ \ \ \ \ \ \ \ \ \ }
\startm
\m{\vdash}\m{\,\mbox{\rm On}}\m{=}\m{\{}\m{x}\m{|}\m{\mbox{\rm Ord}}\m{\,x}
\m{\}}
\endm
\begin{mmraw}%
|- On = \{ x | Ord x \} \$.
\end{mmraw}

\noindent Define the limit ordinal predicate\index{limit ordinal}, which is true for a
non-empty ordinal that is not a successor (i.e.\ that is the union of itself).
Compare Bell and Machover, p.~471 and Exercise (1), p.~42 of Takeuti and
Zaring.

\vskip 0.5ex
\setbox\startprefix=\hbox{\tt \ \ df-lim\ \$a\ }
\setbox\contprefix=\hbox{\tt \ \ \ \ \ \ \ \ \ \ \ \ }
\startm
\m{\vdash}\m{(}\m{\mbox{\rm Lim}}\m{\,A}\m{\leftrightarrow}\m{(}\m{\mbox{
\rm Ord}}\m{\,A}\m{\wedge}\m{\lnot}\m{A}\m{=}\m{\varnothing}\m{\wedge}\m{A}
\m{=}\m{\bigcup}\m{A}\m{)}\m{)}
\endm
\begin{mmraw}%
|- ( Lim A <-> ( Ord A \TAND -. A = (/) \TAND A = U. A ) ) \$.
\end{mmraw}

\noindent Define the successor\index{successor} of a class.  Definition 7.22 of Takeuti
and Zaring, p.~41.  Our definition is a generalization to classes, although it
is meaningless when classes are proper.

\vskip 0.5ex
\setbox\startprefix=\hbox{\tt \ \ df-suc\ \$a\ }
\setbox\contprefix=\hbox{\tt \ \ \ \ \ \ \ \ \ \ \ \ }
\startm
\m{\vdash}\m{\,\mbox{\rm suc}}\m{\,A}\m{=}\m{(}\m{A}\m{\cup}\m{\{}\m{A}\m{\}}
\m{)}
\endm
\begin{mmraw}%
|- suc A = ( A u. \{ A \} ) \$.
\end{mmraw}

\noindent Define the class of natural numbers\index{natural number}\index{omega
($\omega$)}.  Compare Bell and Machover, p.~471.\label{dfom}

\vskip 0.5ex
\setbox\startprefix=\hbox{\tt \ \ df-om\ \$a\ }
\setbox\contprefix=\hbox{\tt \ \ \ \ \ \ \ \ \ \ \ }
\startm
\m{\vdash}\m{\omega}\m{=}\m{\{}\m{x}\m{|}\m{(}\m{\mbox{\rm Ord}}\m{\,x}\m{
\wedge}\m{\forall}\m{y}\m{(}\m{\mbox{\rm Lim}}\m{\,y}\m{\rightarrow}\m{x}\m{
\in}\m{y}\m{)}\m{)}\m{\}}
\endm
\begin{mmraw}%
|- om = \{ x | ( Ord x \TAND A. y ( Lim y -> x e. y ) ) \} \$.
\end{mmraw}

\noindent Define the Cartesian product (also called the
cross product)\index{Cartesian product}\index{cross product}
of two classes.  Definition 9.11 of Quine, p.~64.

\vskip 0.5ex
\setbox\startprefix=\hbox{\tt \ \ df-xp\ \$a\ }
\setbox\contprefix=\hbox{\tt \ \ \ \ \ \ \ \ \ \ \ }
\startm
\m{\vdash}\m{(}\m{A}\m{\times}\m{B}\m{)}\m{=}\m{\{}\m{\langle}\m{x}\m{,}\m{y}
\m{\rangle}\m{|}\m{(}\m{x}\m{\in}\m{A}\m{\wedge}\m{y}\m{\in}\m{B}\m{)}\m{\}}
\endm
\begin{mmraw}%
|- ( A X. B ) = \{ <. x , y >. | ( x e. A \TAND y e. B) \} \$.
\end{mmraw}

\noindent Define a relation\index{relation}.  Definition 6.4(1) of Takeuti and
Zaring, p.~23.

\vskip 0.5ex
\setbox\startprefix=\hbox{\tt \ \ df-rel\ \$a\ }
\setbox\contprefix=\hbox{\tt \ \ \ \ \ \ \ \ \ \ \ \ }
\startm
\m{\vdash}\m{(}\m{\mbox{\rm Rel}}\m{\,A}\m{\leftrightarrow}\m{A}\m{\subseteq}
\m{(}\m{{\rm V}}\m{\times}\m{{\rm V}}\m{)}\m{)}
\endm
\begin{mmraw}%
|- ( Rel A <-> A C\_ ( {\char`\_}V X. {\char`\_}V ) ) \$.
\end{mmraw}

\noindent Define the domain\index{domain} of a class.  Definition 6.5(1) of
Takeuti and Zaring, p.~24.

\vskip 0.5ex
\setbox\startprefix=\hbox{\tt \ \ df-dm\ \$a\ }
\setbox\contprefix=\hbox{\tt \ \ \ \ \ \ \ \ \ \ \ }
\startm
\m{\vdash}\m{\,\mbox{\rm dom}}\m{A}\m{=}\m{\{}\m{x}\m{|}\m{\exists}\m{y}\m{
\langle}\m{x}\m{,}\m{y}\m{\rangle}\m{\in}\m{A}\m{\}}
\endm
\begin{mmraw}%
|- dom A = \{ x | E. y <. x , y >. e. A \} \$.
\end{mmraw}

\noindent Define the range\index{range} of a class.  Definition 6.5(2) of
Takeuti and Zaring, p.~24.

\vskip 0.5ex
\setbox\startprefix=\hbox{\tt \ \ df-rn\ \$a\ }
\setbox\contprefix=\hbox{\tt \ \ \ \ \ \ \ \ \ \ \ }
\startm
\m{\vdash}\m{\,\mbox{\rm ran}}\m{A}\m{=}\m{\{}\m{y}\m{|}\m{\exists}\m{x}\m{
\langle}\m{x}\m{,}\m{y}\m{\rangle}\m{\in}\m{A}\m{\}}
\endm
\begin{mmraw}%
|- ran A = \{ y | E. x <. x , y >. e. A \} \$.
\end{mmraw}

\noindent Define the restriction\index{restriction} of a class.  Definition
6.6(1) of Takeuti and Zaring, p.~24.

\vskip 0.5ex
\setbox\startprefix=\hbox{\tt \ \ df-res\ \$a\ }
\setbox\contprefix=\hbox{\tt \ \ \ \ \ \ \ \ \ \ \ \ }
\startm
\m{\vdash}\m{(}\m{A}\m{\restriction}\m{B}\m{)}\m{=}\m{(}\m{A}\m{\cap}\m{(}\m{B}
\m{\times}\m{{\rm V}}\m{)}\m{)}
\endm
\begin{mmraw}%
|- ( A |` B ) = ( A i\^{}i ( B X. {\char`\_}V ) ) \$.
\end{mmraw}

\noindent Define the image\index{image} of a class.  Definition 6.6(2) of
Takeuti and Zaring, p.~24.

\vskip 0.5ex
\setbox\startprefix=\hbox{\tt \ \ df-ima\ \$a\ }
\setbox\contprefix=\hbox{\tt \ \ \ \ \ \ \ \ \ \ \ \ }
\startm
\m{\vdash}\m{(}\m{A}\m{``}\m{B}\m{)}\m{=}\m{\,\mbox{\rm ran}}\m{\,(}\m{A}\m{
\restriction}\m{B}\m{)}
\endm
\begin{mmraw}%
|- ( A " B ) = ran ( A |` B ) \$.
\end{mmraw}

\noindent Define the composition\index{composition} of two classes.  Definition 6.6(3) of
Takeuti and Zaring, p.~24.

\vskip 0.5ex
\setbox\startprefix=\hbox{\tt \ \ df-co\ \$a\ }
\setbox\contprefix=\hbox{\tt \ \ \ \ \ \ \ \ \ \ \ \ }
\startm
\m{\vdash}\m{(}\m{A}\m{\circ}\m{B}\m{)}\m{=}\m{\{}\m{\langle}\m{x}\m{,}\m{y}\m{
\rangle}\m{|}\m{\exists}\m{z}\m{(}\m{\langle}\m{x}\m{,}\m{z}\m{\rangle}\m{\in}
\m{B}\m{\wedge}\m{\langle}\m{z}\m{,}\m{y}\m{\rangle}\m{\in}\m{A}\m{)}\m{\}}
\endm
\begin{mmraw}%
|- ( A o. B ) = \{ <. x , y >. | E. z ( <. x , z
>. e. B \TAND <. z , y >. e. A ) \} \$.
\end{mmraw}

\noindent Define a function\index{function}.  Definition 6.4(4) of Takeuti and
Zaring, p.~24.

\vskip 0.5ex
\setbox\startprefix=\hbox{\tt \ \ df-fun\ \$a\ }
\setbox\contprefix=\hbox{\tt \ \ \ \ \ \ \ \ \ \ \ \ }
\startm
\m{\vdash}\m{(}\m{\mbox{\rm Fun}}\m{\,A}\m{\leftrightarrow}\m{(}
\m{\mbox{\rm Rel}}\m{\,A}\m{\wedge}
\m{\forall}\m{x}\m{\exists}\m{z}\m{\forall}\m{y}\m{(}
\m{\langle}\m{x}\m{,}\m{y}\m{\rangle}\m{\in}\m{A}\m{\rightarrow}\m{y}\m{=}\m{z}
\m{)}\m{)}\m{)}
\endm
\begin{mmraw}%
|- ( Fun A <-> ( Rel A /\ A. x E. z A. y ( <. x
   , y >. e. A -> y = z ) ) ) \$.
\end{mmraw}

\noindent Define a function with domain.  Definition 6.15(1) of Takeuti and
Zaring, p.~27.

\vskip 0.5ex
\setbox\startprefix=\hbox{\tt \ \ df-fn\ \$a\ }
\setbox\contprefix=\hbox{\tt \ \ \ \ \ \ \ \ \ \ \ }
\startm
\m{\vdash}\m{(}\m{A}\m{\,\mbox{\rm Fn}}\m{\,B}\m{\leftrightarrow}\m{(}
\m{\mbox{\rm Fun}}\m{\,A}\m{\wedge}\m{\mbox{\rm dom}}\m{\,A}\m{=}\m{B}\m{)}
\m{)}
\endm
\begin{mmraw}%
|- ( A Fn B <-> ( Fun A \TAND dom A = B ) ) \$.
\end{mmraw}

\noindent Define a function with domain and co-domain.  Definition 6.15(3)
of Takeuti and Zaring, p.~27.

\vskip 0.5ex
\setbox\startprefix=\hbox{\tt \ \ df-f\ \$a\ }
\setbox\contprefix=\hbox{\tt \ \ \ \ \ \ \ \ \ \ }
\startm
\m{\vdash}\m{(}\m{F}\m{:}\m{A}\m{\longrightarrow}\m{B}\m{
\leftrightarrow}\m{(}\m{F}\m{\,\mbox{\rm Fn}}\m{\,A}\m{\wedge}\m{
\mbox{\rm ran}}\m{\,F}\m{\subseteq}\m{B}\m{)}\m{)}
\endm
\begin{mmraw}%
|- ( F : A --> B <-> ( F Fn A \TAND ran F C\_ B ) ) \$.
\end{mmraw}

\noindent Define a one-to-one function\index{one-to-one function}.  Compare
Definition 6.15(5) of Takeuti and Zaring, p.~27.

\vskip 0.5ex
\setbox\startprefix=\hbox{\tt \ \ df-f1\ \$a\ }
\setbox\contprefix=\hbox{\tt \ \ \ \ \ \ \ \ \ \ \ }
\startm
\m{\vdash}\m{(}\m{F}\m{:}\m{A}\m{
\raisebox{.5ex}{${\textstyle{\:}_{\mbox{\footnotesize\rm
1\tt -\rm 1}}}\atop{\textstyle{
\longrightarrow}\atop{\textstyle{}^{\mbox{\footnotesize\rm {\ }}}}}$}
}\m{B}
\m{\leftrightarrow}\m{(}\m{F}\m{:}\m{A}\m{\longrightarrow}\m{B}
\m{\wedge}\m{\forall}\m{y}\m{\exists}\m{z}\m{\forall}\m{x}\m{(}\m{\langle}\m{x}
\m{,}\m{y}\m{\rangle}\m{\in}\m{F}\m{\rightarrow}\m{x}\m{=}\m{z}\m{)}\m{)}\m{)}
\endm
\begin{mmraw}%
|- ( F : A -1-1-> B <-> ( F : A --> B \TAND
   A. y E. z A. x ( <. x , y >. e. F -> x = z ) ) ) \$.
\end{mmraw}

\noindent Define an onto function\index{onto function}.  Definition 6.15(4) of Takeuti and
Zaring, p.~27.

\vskip 0.5ex
\setbox\startprefix=\hbox{\tt \ \ df-fo\ \$a\ }
\setbox\contprefix=\hbox{\tt \ \ \ \ \ \ \ \ \ \ \ }
\startm
\m{\vdash}\m{(}\m{F}\m{:}\m{A}\m{
\raisebox{.5ex}{${\textstyle{\:}_{\mbox{\footnotesize\rm
{\ }}}}\atop{\textstyle{
\longrightarrow}\atop{\textstyle{}^{\mbox{\footnotesize\rm onto}}}}$}
}\m{B}
\m{\leftrightarrow}\m{(}\m{F}\m{\,\mbox{\rm Fn}}\m{\,A}\m{\wedge}
\m{\mbox{\rm ran}}\m{\,F}\m{=}\m{B}\m{)}\m{)}
\endm
\begin{mmraw}%
|- ( F : A -onto-> B <-> ( F Fn A /\ ran F = B ) ) \$.
\end{mmraw}

\noindent Define a one-to-one, onto function.  Compare Definition 6.15(6) of
Takeuti and Zaring, p.~27.

\vskip 0.5ex
\setbox\startprefix=\hbox{\tt \ \ df-f1o\ \$a\ }
\setbox\contprefix=\hbox{\tt \ \ \ \ \ \ \ \ \ \ \ \ }
\startm
\m{\vdash}\m{(}\m{F}\m{:}\m{A}
\m{
\raisebox{.5ex}{${\textstyle{\:}_{\mbox{\footnotesize\rm
1\tt -\rm 1}}}\atop{\textstyle{
\longrightarrow}\atop{\textstyle{}^{\mbox{\footnotesize\rm onto}}}}$}
}
\m{B}
\m{\leftrightarrow}\m{(}\m{F}\m{:}\m{A}
\m{
\raisebox{.5ex}{${\textstyle{\:}_{\mbox{\footnotesize\rm
1\tt -\rm 1}}}\atop{\textstyle{
\longrightarrow}\atop{\textstyle{}^{\mbox{\footnotesize\rm {\ }}}}}$}
}
\m{B}\m{\wedge}\m{F}\m{:}\m{A}
\m{
\raisebox{.5ex}{${\textstyle{\:}_{\mbox{\footnotesize\rm
{\ }}}}\atop{\textstyle{
\longrightarrow}\atop{\textstyle{}^{\mbox{\footnotesize\rm onto}}}}$}
}
\m{B}\m{)}\m{)}
\endm
\begin{mmraw}%
|- ( F : A -1-1-onto-> B <-> ( F : A -1-1-> B? \TAND F : A -onto-> B ) ) \$.?
\end{mmraw}

\noindent Define the value of a function\index{function value}.  This
definition applies to any class and evaluates to the empty set when it is not
meaningful. Note that $ F`A$ means the same thing as the more familiar $ F(A)$
notation for a function's value at $A$.  The $ F`A$ notation is common in
formal set theory.\label{df-fv}

\vskip 0.5ex
\setbox\startprefix=\hbox{\tt \ \ df-fv\ \$a\ }
\setbox\contprefix=\hbox{\tt \ \ \ \ \ \ \ \ \ \ \ }
\startm
\m{\vdash}\m{(}\m{F}\m{`}\m{A}\m{)}\m{=}\m{\bigcup}\m{\{}\m{x}\m{|}\m{(}\m{F}%
\m{``}\m{\{}\m{A}\m{\}}\m{)}\m{=}\m{\{}\m{x}\m{\}}\m{\}}
\endm
\begin{mmraw}%
|- ( F ` A ) = U. \{ x | ( F " \{ A \} ) = \{ x \} \} \$.
\end{mmraw}

\noindent Define the result of an operation.\index{operation}  Here, $F$ is
     an operation on two
     values (such as $+$ for real numbers).   This is defined for proper
     classes $A$ and $B$ even though not meaningful in that case.  However,
     the definition can be meaningful when $F$ is a proper class.\label{dfopr}

\vskip 0.5ex
\setbox\startprefix=\hbox{\tt \ \ df-opr\ \$a\ }
\setbox\contprefix=\hbox{\tt \ \ \ \ \ \ \ \ \ \ \ \ }
\startm
\m{\vdash}\m{(}\m{A}\m{\,F}\m{\,B}\m{)}\m{=}\m{(}\m{F}\m{`}\m{\langle}\m{A}%
\m{,}\m{B}\m{\rangle}\m{)}
\endm
\begin{mmraw}%
|- ( A F B ) = ( F ` <. A , B >. ) \$.
\end{mmraw}

\section{Tricks of the Trade}\label{tricks}

In the \texttt{set.mm}\index{set theory database (\texttt{set.mm})} database our goal
was usually to conform to modern notation.  However in some cases the
relationship to standard textbook language may be obscured by several
unconventional devices we used to simplify the development and to take
advantage of the Metamath language.  In this section we will describe some
common conventions used in \texttt{set.mm}.

\begin{itemize}
\item
The turnstile\index{turnstile ({$\,\vdash$})} symbol, $\vdash$, meaning ``it
is provable that,'' is the first token of all assertions and hypotheses that
aren't syntax constructions.  This is a standard convention in logic.  (We
mentioned this earlier, but this symbol is bothersome to some people without a
logic background.  It has no deeper meaning but just provides us with a way to
distinguish syntax constructions from ordinary mathematical statements.)

\item
A hypothesis of the form

\vskip 1ex
\setbox\startprefix=\hbox{\tt \ \ \ \ \ \ \ \ \ \$e\ }
\setbox\contprefix=\hbox{\tt \ \ \ \ \ \ \ \ \ \ }
\startm
\m{\vdash}\m{(}\m{\varphi}\m{\rightarrow}\m{\forall}\m{x}\m{\varphi}\m{)}
\endm
\vskip 1ex

should be read ``assume variable $x$ is (effectively) not free in wff
$\varphi$.''\index{effectively not free}
Literally, this says ``assume it is provable that $\varphi \rightarrow \forall
x\, \varphi$.''  This device lets us avoid the complexities associated with
the standard treatment of free and bound variables.
%% Uncomment this when uncommenting section {formalspec} below
The footnote on p.~\pageref{effectivelybound} discusses this further.

\item
A statement of one of the forms

\vskip 1ex
\setbox\startprefix=\hbox{\tt \ \ \ \ \ \ \ \ \ \$a\ }
\setbox\contprefix=\hbox{\tt \ \ \ \ \ \ \ \ \ \ }
\startm
\m{\vdash}\m{(}\m{\lnot}\m{\forall}\m{x}\m{\,x}\m{=}\m{y}\m{\rightarrow}
\m{\ldots}\m{)}
\endm
\setbox\startprefix=\hbox{\tt \ \ \ \ \ \ \ \ \ \$p\ }
\setbox\contprefix=\hbox{\tt \ \ \ \ \ \ \ \ \ \ }
\startm
\m{\vdash}\m{(}\m{\lnot}\m{\forall}\m{x}\m{\,x}\m{=}\m{y}\m{\rightarrow}
\m{\ldots}\m{)}
\endm
\vskip 1ex

should be read ``if $x$ and $y$ are distinct variables, then...''  This
antecedent provides us with a technical device to avoid the need for the
\texttt{\$d} statement early in our development of predicate calculus,
permitting symbol manipulations to be as conceptually simple as those in
propositional calculus.  However, the \texttt{\$d} statement eventually
becomes a requirement, and after that this device is rarely used.

\item
The statement

\vskip 1ex
\setbox\startprefix=\hbox{\tt \ \ \ \ \ \ \ \ \ \$d\ }
\setbox\contprefix=\hbox{\tt \ \ \ \ \ \ \ \ \ \ }
\startm
\m{x}\m{\,y}
\endm
\vskip 1ex

should be read ``assume $x$ and $y$ are distinct variables.''

\item
The statement

\vskip 1ex
\setbox\startprefix=\hbox{\tt \ \ \ \ \ \ \ \ \ \$d\ }
\setbox\contprefix=\hbox{\tt \ \ \ \ \ \ \ \ \ \ }
\startm
\m{x}\m{\,\varphi}
\endm
\vskip 1ex

should be read ``assume $x$ does not occur in $\varphi$.''

\item
The statement

\vskip 1ex
\setbox\startprefix=\hbox{\tt \ \ \ \ \ \ \ \ \ \$d\ }
\setbox\contprefix=\hbox{\tt \ \ \ \ \ \ \ \ \ \ }
\startm
\m{x}\m{\,A}
\endm
\vskip 1ex

should be read ``assume variable $x$ does not occur in class $A$.''

\item
The restriction and hypothesis group

\vskip 1ex
\setbox\startprefix=\hbox{\tt \ \ \ \ \ \ \ \ \ \$d\ }
\setbox\contprefix=\hbox{\tt \ \ \ \ \ \ \ \ \ \ }
\startm
\m{x}\m{\,A}
\endm
\setbox\startprefix=\hbox{\tt \ \ \ \ \ \ \ \ \ \$d\ }
\setbox\contprefix=\hbox{\tt \ \ \ \ \ \ \ \ \ \ }
\startm
\m{x}\m{\,\psi}
\endm
\setbox\startprefix=\hbox{\tt \ \ \ \ \ \ \ \ \ \$e\ }
\setbox\contprefix=\hbox{\tt \ \ \ \ \ \ \ \ \ \ }
\startm
\m{\vdash}\m{(}\m{x}\m{=}\m{A}\m{\rightarrow}\m{(}\m{\varphi}\m{\leftrightarrow}
\m{\psi}\m{)}\m{)}
\endm
\vskip 1ex

is frequently used in place of explicit substitution, meaning ``assume
$\psi$ results from the proper substitution of $A$ for $x$ in
$\varphi$.''  Sometimes ``\texttt{\$e} $\vdash ( \psi \rightarrow
\forall x \, \psi )$'' is used instead of ``\texttt{\$d} $x\, \psi $,''
which requires only that $x$ be effectively not free in $\varphi$ but
not necessarily absent from it.  The use of implicit
substitution\index{substitution!implicit} is partly a matter of personal
style, although it may make proofs somewhat shorter than would be the
case with explicit substitution.

\item
The hypothesis


\vskip 1ex
\setbox\startprefix=\hbox{\tt \ \ \ \ \ \ \ \ \ \$e\ }
\setbox\contprefix=\hbox{\tt \ \ \ \ \ \ \ \ \ \ }
\startm
\m{\vdash}\m{A}\m{\in}\m{{\rm V}}
\endm
\vskip 1ex

should be read ``assume class $A$ is a set (i.e.\ exists).''
This is a convenient convention used by Quine.

\item
The restriction and hypothesis

\vskip 1ex
\setbox\startprefix=\hbox{\tt \ \ \ \ \ \ \ \ \ \$d\ }
\setbox\contprefix=\hbox{\tt \ \ \ \ \ \ \ \ \ \ }
\startm
\m{x}\m{\,y}
\endm
\setbox\startprefix=\hbox{\tt \ \ \ \ \ \ \ \ \ \$e\ }
\setbox\contprefix=\hbox{\tt \ \ \ \ \ \ \ \ \ \ }
\startm
\m{\vdash}\m{(}\m{y}\m{\in}\m{A}\m{\rightarrow}\m{\forall}\m{x}\m{\,y}
\m{\in}\m{A}\m{)}
\endm
\vskip 1ex

should be read ``assume variable $x$ is
(effectively) not free in class $A$.''

\end{itemize}

\section{A Theorem Sampler}\label{sometheorems}

In this section we list some of the more important theorems that are proved in
the \texttt{set.mm} database, and they illustrate the kinds of things that can be
done with Metamath.  While all of these facts are well-known results,
Metamath offers the advantage of easily allowing you to trace their
derivation back to axioms.  Our intent here is not to try to explain the
details or motivation; for this we refer you to the textbooks that are
mentioned in the descriptions.  (The \texttt{set.mm} file has bibliographic
references for the text references.)  Their proofs often embody important
concepts you may wish to explore with the Metamath program (see
Section~\ref{exploring}).  All the symbols that are used here are defined in
Section~\ref{hierarchy}.  For brevity we haven't included the \texttt{\$d}
restrictions or \texttt{\$f} hypotheses for these theorems; when you are
uncertain consult the \texttt{set.mm} database.

We start with \texttt{syl} (principle of the syllogism).
In \textit{Principia Mathematica}
Whitehead and Russell call this ``the principle of the
syllogism... because... the syllogism in Barbara is derived from them''
\cite[quote after Theorem *2.06 p.~101]{PM}.
Some authors call this law a ``hypothetical syllogism.''
At the time of this writing it is the most commonly referenced assertion in the
\texttt{set.mm} database.\footnote{
The Metamath program command \texttt{show usage}
shows the number of references.
}

\vskip 2ex
\noindent Theorem syl (principle of the syllogism)\index{Syllogism}%
\index{\texttt{syl}}\label{syl}.

\vskip 0.5ex
\setbox\startprefix=\hbox{\tt \ \ syl.1\ \$e\ }
\setbox\contprefix=\hbox{\tt \ \ \ \ \ \ \ \ \ \ \ }
\startm
\m{\vdash}\m{(}\m{\varphi}\m{ \rightarrow }\m{\psi}\m{)}
\endm
\setbox\startprefix=\hbox{\tt \ \ syl.2\ \$e\ }
\setbox\contprefix=\hbox{\tt \ \ \ \ \ \ \ \ \ \ \ }
\startm
\m{\vdash}\m{(}\m{\psi}\m{ \rightarrow }\m{\chi}\m{)}
\endm
\setbox\startprefix=\hbox{\tt \ \ syl\ \$p\ }
\setbox\contprefix=\hbox{\tt \ \ \ \ \ \ \ \ \ }
\startm
\m{\vdash}\m{(}\m{\varphi}\m{ \rightarrow }\m{\chi}\m{)}
\endm
\vskip 2ex

The following theorem is not very deep but provides us with a notational device
that is frequently used.  It allows us to use the expression ``$A \in V$'' as
a compact way of saying that class $A$ exists, i.e.\ is a set.

\vskip 2ex
\noindent Two ways to say ``$A$ is a set'':  $A$ is a member of the universe
$V$ if and only if $A$ exists (i.e.\ there exists a set equal to $A$).
Theorem 6.9 of Quine, p. 43.

\vskip 0.5ex
\setbox\startprefix=\hbox{\tt \ \ isset\ \$p\ }
\setbox\contprefix=\hbox{\tt \ \ \ \ \ \ \ \ \ \ \ }
\startm
\m{\vdash}\m{(}\m{A}\m{\in}\m{{\rm V}}\m{\leftrightarrow}\m{\exists}\m{x}\m{\,x}\m{=}
\m{A}\m{)}
\endm
\vskip 1ex

Next we prove the axioms of standard ZF set theory that were missing from our
axiom system.  From our point of view they are theorems since they
can be derived from the other axioms.

\vskip 2ex
\noindent Axiom of Separation\index{Axiom of Separation}
(Aussonderung)\index{Aussonderung} proved from the other axioms of ZF set
theory.  Compare Exercise 4 of Takeuti and Zaring, p.~22.

\vskip 0.5ex
\setbox\startprefix=\hbox{\tt \ \ inex1.1\ \$e\ }
\setbox\contprefix=\hbox{\tt \ \ \ \ \ \ \ \ \ \ \ \ \ \ \ }
\startm
\m{\vdash}\m{A}\m{\in}\m{{\rm V}}
\endm
\setbox\startprefix=\hbox{\tt \ \ inex\ \$p\ }
\setbox\contprefix=\hbox{\tt \ \ \ \ \ \ \ \ \ \ \ \ \ }
\startm
\m{\vdash}\m{(}\m{A}\m{\cap}\m{B}\m{)}\m{\in}\m{{\rm V}}
\endm
\vskip 1ex

\noindent Axiom of the Null Set\index{Axiom of the Null Set} proved from the
other axioms of ZF set theory. Corollary 5.16 of Takeuti and Zaring, p.~20.

\vskip 0.5ex
\setbox\startprefix=\hbox{\tt \ \ 0ex\ \$p\ }
\setbox\contprefix=\hbox{\tt \ \ \ \ \ \ \ \ \ \ \ \ }
\startm
\m{\vdash}\m{\varnothing}\m{\in}\m{{\rm V}}
\endm
\vskip 1ex

\noindent The Axiom of Pairing\index{Axiom of Pairing} proved from the other
axioms of ZF set theory.  Theorem 7.13 of Quine, p.~51.
\vskip 0.5ex
\setbox\startprefix=\hbox{\tt \ \ prex\ \$p\ }
\setbox\contprefix=\hbox{\tt \ \ \ \ \ \ \ \ \ \ \ \ \ \ }
\startm
\m{\vdash}\m{\{}\m{A}\m{,}\m{B}\m{\}}\m{\in}\m{{\rm V}}
\endm
\vskip 2ex

Next we will list some famous or important theorems that are proved in
the \texttt{set.mm} database.  None of them except \texttt{omex}
require the Axiom of Infinity, as you can verify with the \texttt{show
trace{\char`\_}back} Metamath command.

\vskip 2ex
\noindent The resolution of Russell's paradox\index{Russell's paradox}.  There
exists no set containing the set of all sets which are not members of
themselves.  Proposition 4.14 of Takeuti and Zaring, p.~14.

\vskip 0.5ex
\setbox\startprefix=\hbox{\tt \ \ ru\ \$p\ }
\setbox\contprefix=\hbox{\tt \ \ \ \ \ \ \ \ }
\startm
\m{\vdash}\m{\lnot}\m{\exists}\m{x}\m{\,x}\m{=}\m{\{}\m{y}\m{|}\m{\lnot}\m{y}
\m{\in}\m{y}\m{\}}
\endm
\vskip 1ex

\noindent Cantor's theorem\index{Cantor's theorem}.  No set can be mapped onto
its power set.  Compare Theorem 6B(b) of Enderton, p.~132.

\vskip 0.5ex
\setbox\startprefix=\hbox{\tt \ \ canth.1\ \$e\ }
\setbox\contprefix=\hbox{\tt \ \ \ \ \ \ \ \ \ \ \ \ \ }
\startm
\m{\vdash}\m{A}\m{\in}\m{{\rm V}}
\endm
\setbox\startprefix=\hbox{\tt \ \ canth\ \$p\ }
\setbox\contprefix=\hbox{\tt \ \ \ \ \ \ \ \ \ \ \ }
\startm
\m{\vdash}\m{\lnot}\m{F}\m{:}\m{A}\m{\raisebox{.5ex}{${\textstyle{\:}_{
\mbox{\footnotesize\rm {\ }}}}\atop{\textstyle{\longrightarrow}\atop{
\textstyle{}^{\mbox{\footnotesize\rm onto}}}}$}}\m{{\cal P}}\m{A}
\endm
\vskip 1ex

\noindent The Burali-Forti paradox\index{Burali-Forti paradox}.  No set
contains all ordinal numbers. Enderton, p.~194.  (Burali-Forti was one person,
not two.)

\vskip 0.5ex
\setbox\startprefix=\hbox{\tt \ \ onprc\ \$p\ }
\setbox\contprefix=\hbox{\tt \ \ \ \ \ \ \ \ \ \ \ \ }
\startm
\m{\vdash}\m{\lnot}\m{\mbox{\rm On}}\m{\in}\m{{\rm V}}
\endm
\vskip 1ex

\noindent Peano's postulates\index{Peano's postulates} for arithmetic.
Proposition 7.30 of Takeuti and Zaring, pp.~42--43.  The objects being
described are the members of $\omega$ i.e.\ the natural numbers 0, 1,
2,\ldots.  The successor\index{successor} operation suc means ``plus
one.''  \texttt{peano1} says that 0 (which is defined as the empty set)
is a natural number.  \texttt{peano2} says that if $A$ is a natural
number, so is $A+1$.  \texttt{peano3} says that 0 is not the successor
of any natural number.  \texttt{peano4} says that two natural numbers
are equal if and only if their successors are equal.  \texttt{peano5} is
essentially the same as mathematical induction.

\vskip 1ex
\setbox\startprefix=\hbox{\tt \ \ peano1\ \$p\ }
\setbox\contprefix=\hbox{\tt \ \ \ \ \ \ \ \ \ \ \ \ }
\startm
\m{\vdash}\m{\varnothing}\m{\in}\m{\omega}
\endm
\vskip 1.5ex

\setbox\startprefix=\hbox{\tt \ \ peano2\ \$p\ }
\setbox\contprefix=\hbox{\tt \ \ \ \ \ \ \ \ \ \ \ \ }
\startm
\m{\vdash}\m{(}\m{A}\m{\in}\m{\omega}\m{\rightarrow}\m{{\rm suc}}\m{A}\m{\in}%
\m{\omega}\m{)}
\endm
\vskip 1.5ex

\setbox\startprefix=\hbox{\tt \ \ peano3\ \$p\ }
\setbox\contprefix=\hbox{\tt \ \ \ \ \ \ \ \ \ \ \ \ }
\startm
\m{\vdash}\m{(}\m{A}\m{\in}\m{\omega}\m{\rightarrow}\m{\lnot}\m{{\rm suc}}%
\m{A}\m{=}\m{\varnothing}\m{)}
\endm
\vskip 1.5ex

\setbox\startprefix=\hbox{\tt \ \ peano4\ \$p\ }
\setbox\contprefix=\hbox{\tt \ \ \ \ \ \ \ \ \ \ \ \ }
\startm
\m{\vdash}\m{(}\m{(}\m{A}\m{\in}\m{\omega}\m{\wedge}\m{B}\m{\in}\m{\omega}%
\m{)}\m{\rightarrow}\m{(}\m{{\rm suc}}\m{A}\m{=}\m{{\rm suc}}\m{B}\m{%
\leftrightarrow}\m{A}\m{=}\m{B}\m{)}\m{)}
\endm
\vskip 1.5ex

\setbox\startprefix=\hbox{\tt \ \ peano5\ \$p\ }
\setbox\contprefix=\hbox{\tt \ \ \ \ \ \ \ \ \ \ \ \ }
\startm
\m{\vdash}\m{(}\m{(}\m{\varnothing}\m{\in}\m{A}\m{\wedge}\m{\forall}\m{x}\m{%
\in}\m{\omega}\m{(}\m{x}\m{\in}\m{A}\m{\rightarrow}\m{{\rm suc}}\m{x}\m{\in}%
\m{A}\m{)}\m{)}\m{\rightarrow}\m{\omega}\m{\subseteq}\m{A}\m{)}
\endm
\vskip 1.5ex

\noindent Finite Induction (mathematical induction).\index{finite
induction}\index{mathematical induction} The first hypothesis is the
basis and the second is the induction hypothesis.  Theorem Schema 22 of
Suppes, p.~136.

\vskip 0.5ex
\setbox\startprefix=\hbox{\tt \ \ findes.1\ \$e\ }
\setbox\contprefix=\hbox{\tt \ \ \ \ \ \ \ \ \ \ \ \ \ \ }
\startm
\m{\vdash}\m{[}\m{\varnothing}\m{/}\m{x}\m{]}\m{\varphi}
\endm
\setbox\startprefix=\hbox{\tt \ \ findes.2\ \$e\ }
\setbox\contprefix=\hbox{\tt \ \ \ \ \ \ \ \ \ \ \ \ \ \ }
\startm
\m{\vdash}\m{(}\m{x}\m{\in}\m{\omega}\m{\rightarrow}\m{(}\m{\varphi}\m{%
\rightarrow}\m{[}\m{{\rm suc}}\m{x}\m{/}\m{x}\m{]}\m{\varphi}\m{)}\m{)}
\endm
\setbox\startprefix=\hbox{\tt \ \ findes\ \$p\ }
\setbox\contprefix=\hbox{\tt \ \ \ \ \ \ \ \ \ \ \ \ }
\startm
\m{\vdash}\m{(}\m{x}\m{\in}\m{\omega}\m{\rightarrow}\m{\varphi}\m{)}
\endm
\vskip 1ex

\noindent Transfinite Induction with explicit substitution.  The first
hypothesis is the basis, the second is the induction hypothesis for
successors, and the third is the induction hypothesis for limit
ordinals.  Theorem Schema 4 of Suppes, p. 197.

\vskip 0.5ex
\setbox\startprefix=\hbox{\tt \ \ tfindes.1\ \$e\ }
\setbox\contprefix=\hbox{\tt \ \ \ \ \ \ \ \ \ \ \ \ \ \ \ }
\startm
\m{\vdash}\m{[}\m{\varnothing}\m{/}\m{x}\m{]}\m{\varphi}
\endm
\setbox\startprefix=\hbox{\tt \ \ tfindes.2\ \$e\ }
\setbox\contprefix=\hbox{\tt \ \ \ \ \ \ \ \ \ \ \ \ \ \ \ }
\startm
\m{\vdash}\m{(}\m{x}\m{\in}\m{{\rm On}}\m{\rightarrow}\m{(}\m{\varphi}\m{%
\rightarrow}\m{[}\m{{\rm suc}}\m{x}\m{/}\m{x}\m{]}\m{\varphi}\m{)}\m{)}
\endm
\setbox\startprefix=\hbox{\tt \ \ tfindes.3\ \$e\ }
\setbox\contprefix=\hbox{\tt \ \ \ \ \ \ \ \ \ \ \ \ \ \ \ }
\startm
\m{\vdash}\m{(}\m{{\rm Lim}}\m{y}\m{\rightarrow}\m{(}\m{\forall}\m{x}\m{\in}%
\m{y}\m{\varphi}\m{\rightarrow}\m{[}\m{y}\m{/}\m{x}\m{]}\m{\varphi}\m{)}\m{)}
\endm
\setbox\startprefix=\hbox{\tt \ \ tfindes\ \$p\ }
\setbox\contprefix=\hbox{\tt \ \ \ \ \ \ \ \ \ \ \ \ \ }
\startm
\m{\vdash}\m{(}\m{x}\m{\in}\m{{\rm On}}\m{\rightarrow}\m{\varphi}\m{)}
\endm
\vskip 1ex

\noindent Principle of Transfinite Recursion.\index{transfinite
recursion} Theorem 7.41 of Takeuti and Zaring, p.~47.  Transfinite
recursion is the key theorem that allows arithmetic of ordinals to be
rigorously defined, and has many other important uses as well.
Hypotheses \texttt{tfr.1} and \texttt{tfr.2} specify a certain (proper)
class $ F$.  The complicated definition of $ F$ is not important in
itself; what is important is that there be such an $ F$ with the
required properties, and we show this by displaying $ F$ explicitly.
\texttt{tfr1} states that $ F$ is a function whose domain is the set of
ordinal numbers.  \texttt{tfr2} states that any value of $ F$ is
completely determined by its previous values and the values of an
auxiliary function, $G$.  \texttt{tfr3} states that $ F$ is unique,
i.e.\ it is the only function that satisfies \texttt{tfr1} and
\texttt{tfr2}.  Note that $ f$ is an individual variable like $x$ and
$y$; it is just a mnemonic to remind us that $A$ is a collection of
functions.

\vskip 0.5ex
\setbox\startprefix=\hbox{\tt \ \ tfr.1\ \$e\ }
\setbox\contprefix=\hbox{\tt \ \ \ \ \ \ \ \ \ \ \ }
\startm
\m{\vdash}\m{A}\m{=}\m{\{}\m{f}\m{|}\m{\exists}\m{x}\m{\in}\m{{\rm On}}\m{(}%
\m{f}\m{{\rm Fn}}\m{x}\m{\wedge}\m{\forall}\m{y}\m{\in}\m{x}\m{(}\m{f}\m{`}%
\m{y}\m{)}\m{=}\m{(}\m{G}\m{`}\m{(}\m{f}\m{\restriction}\m{y}\m{)}\m{)}\m{)}%
\m{\}}
\endm
\setbox\startprefix=\hbox{\tt \ \ tfr.2\ \$e\ }
\setbox\contprefix=\hbox{\tt \ \ \ \ \ \ \ \ \ \ \ }
\startm
\m{\vdash}\m{F}\m{=}\m{\bigcup}\m{A}
\endm
\setbox\startprefix=\hbox{\tt \ \ tfr1\ \$p\ }
\setbox\contprefix=\hbox{\tt \ \ \ \ \ \ \ \ \ \ }
\startm
\m{\vdash}\m{F}\m{{\rm Fn}}\m{{\rm On}}
\endm
\setbox\startprefix=\hbox{\tt \ \ tfr2\ \$p\ }
\setbox\contprefix=\hbox{\tt \ \ \ \ \ \ \ \ \ \ }
\startm
\m{\vdash}\m{(}\m{z}\m{\in}\m{{\rm On}}\m{\rightarrow}\m{(}\m{F}\m{`}\m{z}%
\m{)}\m{=}\m{(}\m{G}\m{`}\m{(}\m{F}\m{\restriction}\m{z}\m{)}\m{)}\m{)}
\endm
\setbox\startprefix=\hbox{\tt \ \ tfr3\ \$p\ }
\setbox\contprefix=\hbox{\tt \ \ \ \ \ \ \ \ \ \ }
\startm
\m{\vdash}\m{(}\m{(}\m{B}\m{{\rm Fn}}\m{{\rm On}}\m{\wedge}\m{\forall}\m{x}\m{%
\in}\m{{\rm On}}\m{(}\m{B}\m{`}\m{x}\m{)}\m{=}\m{(}\m{G}\m{`}\m{(}\m{B}\m{%
\restriction}\m{x}\m{)}\m{)}\m{)}\m{\rightarrow}\m{B}\m{=}\m{F}\m{)}
\endm
\vskip 1ex

\noindent The existence of omega (the class of natural numbers).\index{natural
number}\index{omega ($\omega$)}\index{Axiom of Infinity}  Axiom 7 of Takeuti
and Zaring, p.~43.  (This is the only theorem in this section requiring the
Axiom of Infinity.)

\vskip 0.5ex
\setbox\startprefix=\hbox{\tt \
\ omex\ \$p\ }
\setbox\contprefix=\hbox{\tt \ \ \ \ \ \ \ \ \ \ }
\startm
\m{\vdash}\m{\omega}\m{\in}\m{{\rm V}}
\endm
%\vskip 2ex


\section{Axioms for Real and Complex Numbers}\label{real}
\index{real and complex numbers!axioms for}

This section presents the axioms for real and complex numbers.  Analysis
textbooks implicitly or explicitly use these axioms or their equivalents
as their starting point.  In the database \texttt{set.mm}, we define real
and complex numbers as (rather complicated) specific sets and derive these
axioms as {\em theorems} from the axioms of ZF set theory, using a method
called Dedekind cuts.  We omit the details of this construction, which you can
follow if you wish using the \texttt{set.mm} database in conjunction with the
textbooks referenced therein.

Once we prove those theorems, we then restate these proven theorems as axioms.
This lets us easily identify which axioms are needed for a particular complex number proof, without the obfuscation of the set theory used to derive them.
As a result,
the construction is actually unimportant other
than to show that sets exist that satisfy the axioms, and thus that the axioms
are consistent if ZF set theory is consistent.  When working with real numbers
you can think of them as being the actual sets resulting
from the construction (for definiteness), or you can
think of them as otherwise unspecified sets that happen to satisfy the axioms.
The derivation is not easy, but the fact that it works is quite remarkable
and lends support to the idea that ZFC set theory is all we need to
provide a foundation for essentially all of mathematics.

For the axioms we are given (or postulate) 8 classes:  $\mathbb{C}$ (the
set of complex numbers), $\mathbb{R}$ (the set of real numbers, a subset
of $\mathbb{C}$), $0$ (zero), $1$ (one), $i$ (square root of
$-1$), $+$ (plus), $\cdot$ (times), and
$<_{\mathbb{R}}$ (less than for just the real numbers).
Subtraction and division are defined terms and are not part of the
axioms; for their definitions see \texttt{set.mm}.

Note that the notation $(A+B)$ (and similarly $(A\cdot B)$) specifies a class
called an {\em operation},\index{operation} and is the function value of the
class $+$ at ordered pair $\langle A,B \rangle$.  An operation is defined by
statement \texttt{df-opr} on p.~\pageref{dfopr}.
The notation $A <_{\mathbb{R}} B$ specifies a
wff called a {\em binary relation}\index{binary relation} and means $\langle A,B \rangle \in \,<_{\mathbb{R}}$, as defined by statement \texttt{df-br} on p.~\pageref{dfbr}.

Our set of 8 given classes is assumed to satisfy the following 22 axioms
(in the axioms listed below, $<$ really means $<_{\mathbb{R}}$).

\vskip 2ex

\noindent 1. The real numbers are a subset of the complex numbers.

%\vskip 0.5ex
\setbox\startprefix=\hbox{\tt \ \ ax-resscn\ \$p\ }
\setbox\contprefix=\hbox{\tt \ \ \ \ \ \ \ \ \ \ \ \ \ \ }
\startm
\m{\vdash}\m{\mathbb{R}}\m{\subseteq}\m{\mathbb{C}}
\endm
%\vskip 1ex

\noindent 2. One is a complex number.

%\vskip 0.5ex
\setbox\startprefix=\hbox{\tt \ \ ax-1cn\ \$p\ }
\setbox\contprefix=\hbox{\tt \ \ \ \ \ \ \ \ \ \ \ }
\startm
\m{\vdash}\m{1}\m{\in}\m{\mathbb{C}}
\endm
%\vskip 1ex

\noindent 3. The imaginary number $i$ is a complex number.

%\vskip 0.5ex
\setbox\startprefix=\hbox{\tt \ \ ax-icn\ \$p\ }
\setbox\contprefix=\hbox{\tt \ \ \ \ \ \ \ \ \ \ \ }
\startm
\m{\vdash}\m{i}\m{\in}\m{\mathbb{C}}
\endm
%\vskip 1ex

\noindent 4. Complex numbers are closed under addition.

%\vskip 0.5ex
\setbox\startprefix=\hbox{\tt \ \ ax-addcl\ \$p\ }
\setbox\contprefix=\hbox{\tt \ \ \ \ \ \ \ \ \ \ \ \ \ }
\startm
\m{\vdash}\m{(}\m{(}\m{A}\m{\in}\m{\mathbb{C}}\m{\wedge}\m{B}\m{\in}\m{\mathbb{C}}%
\m{)}\m{\rightarrow}\m{(}\m{A}\m{+}\m{B}\m{)}\m{\in}\m{\mathbb{C}}\m{)}
\endm
%\vskip 1ex

\noindent 5. Real numbers are closed under addition.

%\vskip 0.5ex
\setbox\startprefix=\hbox{\tt \ \ ax-addrcl\ \$p\ }
\setbox\contprefix=\hbox{\tt \ \ \ \ \ \ \ \ \ \ \ \ \ \ }
\startm
\m{\vdash}\m{(}\m{(}\m{A}\m{\in}\m{\mathbb{R}}\m{\wedge}\m{B}\m{\in}\m{\mathbb{R}}%
\m{)}\m{\rightarrow}\m{(}\m{A}\m{+}\m{B}\m{)}\m{\in}\m{\mathbb{R}}\m{)}
\endm
%\vskip 1ex

\noindent 6. Complex numbers are closed under multiplication.

%\vskip 0.5ex
\setbox\startprefix=\hbox{\tt \ \ ax-mulcl\ \$p\ }
\setbox\contprefix=\hbox{\tt \ \ \ \ \ \ \ \ \ \ \ \ \ }
\startm
\m{\vdash}\m{(}\m{(}\m{A}\m{\in}\m{\mathbb{C}}\m{\wedge}\m{B}\m{\in}\m{\mathbb{C}}%
\m{)}\m{\rightarrow}\m{(}\m{A}\m{\cdot}\m{B}\m{)}\m{\in}\m{\mathbb{C}}\m{)}
\endm
%\vskip 1ex

\noindent 7. Real numbers are closed under multiplication.

%\vskip 0.5ex
\setbox\startprefix=\hbox{\tt \ \ ax-mulrcl\ \$p\ }
\setbox\contprefix=\hbox{\tt \ \ \ \ \ \ \ \ \ \ \ \ \ \ }
\startm
\m{\vdash}\m{(}\m{(}\m{A}\m{\in}\m{\mathbb{R}}\m{\wedge}\m{B}\m{\in}\m{\mathbb{R}}%
\m{)}\m{\rightarrow}\m{(}\m{A}\m{\cdot}\m{B}\m{)}\m{\in}\m{\mathbb{R}}\m{)}
\endm
%\vskip 1ex

\noindent 8. Multiplication of complex numbers is commutative.

%\vskip 0.5ex
\setbox\startprefix=\hbox{\tt \ \ ax-mulcom\ \$p\ }
\setbox\contprefix=\hbox{\tt \ \ \ \ \ \ \ \ \ \ \ \ \ \ }
\startm
\m{\vdash}\m{(}\m{(}\m{A}\m{\in}\m{\mathbb{C}}\m{\wedge}\m{B}\m{\in}\m{\mathbb{C}}%
\m{)}\m{\rightarrow}\m{(}\m{A}\m{\cdot}\m{B}\m{)}\m{=}\m{(}\m{B}\m{\cdot}\m{A}%
\m{)}\m{)}
\endm
%\vskip 1ex

\noindent 9. Addition of complex numbers is associative.

%\vskip 0.5ex
\setbox\startprefix=\hbox{\tt \ \ ax-addass\ \$p\ }
\setbox\contprefix=\hbox{\tt \ \ \ \ \ \ \ \ \ \ \ \ \ \ }
\startm
\m{\vdash}\m{(}\m{(}\m{A}\m{\in}\m{\mathbb{C}}\m{\wedge}\m{B}\m{\in}\m{\mathbb{C}}%
\m{\wedge}\m{C}\m{\in}\m{\mathbb{C}}\m{)}\m{\rightarrow}\m{(}\m{(}\m{A}\m{+}%
\m{B}\m{)}\m{+}\m{C}\m{)}\m{=}\m{(}\m{A}\m{+}\m{(}\m{B}\m{+}\m{C}\m{)}\m{)}%
\m{)}
\endm
%\vskip 1ex

\noindent 10. Multiplication of complex numbers is associative.

%\vskip 0.5ex
\setbox\startprefix=\hbox{\tt \ \ ax-mulass\ \$p\ }
\setbox\contprefix=\hbox{\tt \ \ \ \ \ \ \ \ \ \ \ \ \ \ }
\startm
\m{\vdash}\m{(}\m{(}\m{A}\m{\in}\m{\mathbb{C}}\m{\wedge}\m{B}\m{\in}\m{\mathbb{C}}%
\m{\wedge}\m{C}\m{\in}\m{\mathbb{C}}\m{)}\m{\rightarrow}\m{(}\m{(}\m{A}\m{\cdot}%
\m{B}\m{)}\m{\cdot}\m{C}\m{)}\m{=}\m{(}\m{A}\m{\cdot}\m{(}\m{B}\m{\cdot}\m{C}%
\m{)}\m{)}\m{)}
\endm
%\vskip 1ex

\noindent 11. Multiplication distributes over addition for complex numbers.

%\vskip 0.5ex
\setbox\startprefix=\hbox{\tt \ \ ax-distr\ \$p\ }
\setbox\contprefix=\hbox{\tt \ \ \ \ \ \ \ \ \ \ \ \ \ }
\startm
\m{\vdash}\m{(}\m{(}\m{A}\m{\in}\m{\mathbb{C}}\m{\wedge}\m{B}\m{\in}\m{\mathbb{C}}%
\m{\wedge}\m{C}\m{\in}\m{\mathbb{C}}\m{)}\m{\rightarrow}\m{(}\m{A}\m{\cdot}\m{(}%
\m{B}\m{+}\m{C}\m{)}\m{)}\m{=}\m{(}\m{(}\m{A}\m{\cdot}\m{B}\m{)}\m{+}\m{(}%
\m{A}\m{\cdot}\m{C}\m{)}\m{)}\m{)}
\endm
%\vskip 1ex

\noindent 12. The square of $i$ equals $-1$ (expressed as $i$-squared plus 1 is
0).

%\vskip 0.5ex
\setbox\startprefix=\hbox{\tt \ \ ax-i2m1\ \$p\ }
\setbox\contprefix=\hbox{\tt \ \ \ \ \ \ \ \ \ \ \ \ }
\startm
\m{\vdash}\m{(}\m{(}\m{i}\m{\cdot}\m{i}\m{)}\m{+}\m{1}\m{)}\m{=}\m{0}
\endm
%\vskip 1ex

\noindent 13. One and zero are distinct.

%\vskip 0.5ex
\setbox\startprefix=\hbox{\tt \ \ ax-1ne0\ \$p\ }
\setbox\contprefix=\hbox{\tt \ \ \ \ \ \ \ \ \ \ \ \ }
\startm
\m{\vdash}\m{1}\m{\ne}\m{0}
\endm
%\vskip 1ex

\noindent 14. One is an identity element for real multiplication.

%\vskip 0.5ex
\setbox\startprefix=\hbox{\tt \ \ ax-1rid\ \$p\ }
\setbox\contprefix=\hbox{\tt \ \ \ \ \ \ \ \ \ \ \ }
\startm
\m{\vdash}\m{(}\m{A}\m{\in}\m{\mathbb{R}}\m{\rightarrow}\m{(}\m{A}\m{\cdot}\m{1}%
\m{)}\m{=}\m{A}\m{)}
\endm
%\vskip 1ex

\noindent 15. Every real number has a negative.

%\vskip 0.5ex
\setbox\startprefix=\hbox{\tt \ \ ax-rnegex\ \$p\ }
\setbox\contprefix=\hbox{\tt \ \ \ \ \ \ \ \ \ \ \ \ \ \ }
\startm
\m{\vdash}\m{(}\m{A}\m{\in}\m{\mathbb{R}}\m{\rightarrow}\m{\exists}\m{x}\m{\in}%
\m{\mathbb{R}}\m{(}\m{A}\m{+}\m{x}\m{)}\m{=}\m{0}\m{)}
\endm
%\vskip 1ex

\noindent 16. Every nonzero real number has a reciprocal.

%\vskip 0.5ex
\setbox\startprefix=\hbox{\tt \ \ ax-rrecex\ \$p\ }
\setbox\contprefix=\hbox{\tt \ \ \ \ \ \ \ \ \ \ \ \ \ \ }
\startm
\m{\vdash}\m{(}\m{A}\m{\in}\m{\mathbb{R}}\m{\rightarrow}\m{(}\m{A}\m{\ne}\m{0}%
\m{\rightarrow}\m{\exists}\m{x}\m{\in}\m{\mathbb{R}}\m{(}\m{A}\m{\cdot}%
\m{x}\m{)}\m{=}\m{1}\m{)}\m{)}
\endm
%\vskip 1ex

\noindent 17. A complex number can be expressed in terms of two reals.

%\vskip 0.5ex
\setbox\startprefix=\hbox{\tt \ \ ax-cnre\ \$p\ }
\setbox\contprefix=\hbox{\tt \ \ \ \ \ \ \ \ \ \ \ \ }
\startm
\m{\vdash}\m{(}\m{A}\m{\in}\m{\mathbb{C}}\m{\rightarrow}\m{\exists}\m{x}\m{\in}%
\m{\mathbb{R}}\m{\exists}\m{y}\m{\in}\m{\mathbb{R}}\m{A}\m{=}\m{(}\m{x}\m{+}\m{(}%
\m{y}\m{\cdot}\m{i}\m{)}\m{)}\m{)}
\endm
%\vskip 1ex

\noindent 18. Ordering on reals satisfies strict trichotomy.

%\vskip 0.5ex
\setbox\startprefix=\hbox{\tt \ \ ax-pre-lttri\ \$p\ }
\setbox\contprefix=\hbox{\tt \ \ \ \ \ \ \ \ \ \ \ \ \ }
\startm
\m{\vdash}\m{(}\m{(}\m{A}\m{\in}\m{\mathbb{R}}\m{\wedge}\m{B}\m{\in}\m{\mathbb{R}}%
\m{)}\m{\rightarrow}\m{(}\m{A}\m{<}\m{B}\m{\leftrightarrow}\m{\lnot}\m{(}\m{A}%
\m{=}\m{B}\m{\vee}\m{B}\m{<}\m{A}\m{)}\m{)}\m{)}
\endm
%\vskip 1ex

\noindent 19. Ordering on reals is transitive.

%\vskip 0.5ex
\setbox\startprefix=\hbox{\tt \ \ ax-pre-lttrn\ \$p\ }
\setbox\contprefix=\hbox{\tt \ \ \ \ \ \ \ \ \ \ \ \ \ }
\startm
\m{\vdash}\m{(}\m{(}\m{A}\m{\in}\m{\mathbb{R}}\m{\wedge}\m{B}\m{\in}\m{\mathbb{R}}%
\m{\wedge}\m{C}\m{\in}\m{\mathbb{R}}\m{)}\m{\rightarrow}\m{(}\m{(}\m{A}\m{<}%
\m{B}\m{\wedge}\m{B}\m{<}\m{C}\m{)}\m{\rightarrow}\m{A}\m{<}\m{C}\m{)}\m{)}
\endm
%\vskip 1ex

\noindent 20. Ordering on reals is preserved after addition to both sides.

%\vskip 0.5ex
\setbox\startprefix=\hbox{\tt \ \ ax-pre-ltadd\ \$p\ }
\setbox\contprefix=\hbox{\tt \ \ \ \ \ \ \ \ \ \ \ \ \ }
\startm
\m{\vdash}\m{(}\m{(}\m{A}\m{\in}\m{\mathbb{R}}\m{\wedge}\m{B}\m{\in}\m{\mathbb{R}}%
\m{\wedge}\m{C}\m{\in}\m{\mathbb{R}}\m{)}\m{\rightarrow}\m{(}\m{A}\m{<}\m{B}\m{%
\rightarrow}\m{(}\m{C}\m{+}\m{A}\m{)}\m{<}\m{(}\m{C}\m{+}\m{B}\m{)}\m{)}\m{)}
\endm
%\vskip 1ex

\noindent 21. The product of two positive reals is positive.

%\vskip 0.5ex
\setbox\startprefix=\hbox{\tt \ \ ax-pre-mulgt0\ \$p\ }
\setbox\contprefix=\hbox{\tt \ \ \ \ \ \ \ \ \ \ \ \ \ \ }
\startm
\m{\vdash}\m{(}\m{(}\m{A}\m{\in}\m{\mathbb{R}}\m{\wedge}\m{B}\m{\in}\m{\mathbb{R}}%
\m{)}\m{\rightarrow}\m{(}\m{(}\m{0}\m{<}\m{A}\m{\wedge}\m{0}%
\m{<}\m{B}\m{)}\m{\rightarrow}\m{0}\m{<}\m{(}\m{A}\m{\cdot}\m{B}\m{)}%
\m{)}\m{)}
\endm
%\vskip 1ex

\noindent 22. A non-empty, bounded-above set of reals has a supremum.

%\vskip 0.5ex
\setbox\startprefix=\hbox{\tt \ \ ax-pre-sup\ \$p\ }
\setbox\contprefix=\hbox{\tt \ \ \ \ \ \ \ \ \ \ \ }
\startm
\m{\vdash}\m{(}\m{(}\m{A}\m{\subseteq}\m{\mathbb{R}}\m{\wedge}\m{A}\m{\ne}\m{%
\varnothing}\m{\wedge}\m{\exists}\m{x}\m{\in}\m{\mathbb{R}}\m{\forall}\m{y}\m{%
\in}\m{A}\m{\,y}\m{<}\m{x}\m{)}\m{\rightarrow}\m{\exists}\m{x}\m{\in}\m{%
\mathbb{R}}\m{(}\m{\forall}\m{y}\m{\in}\m{A}\m{\lnot}\m{x}\m{<}\m{y}\m{\wedge}\m{%
\forall}\m{y}\m{\in}\m{\mathbb{R}}\m{(}\m{y}\m{<}\m{x}\m{\rightarrow}\m{\exists}%
\m{z}\m{\in}\m{A}\m{\,y}\m{<}\m{z}\m{)}\m{)}\m{)}
\endm

% NOTE: The \m{...} expressions above could be represented as
% $ \vdash ( ( A \subseteq \mathbb{R} \wedge A \ne \varnothing \wedge \exists x \in \mathbb{R} \forall y \in A \,y < x ) \rightarrow \exists x \in \mathbb{R} ( \forall y \in A \lnot x < y \wedge \forall y \in \mathbb{R} ( y < x \rightarrow \exists z \in A \,y < z ) ) ) $

\vskip 2ex

This completes the set of axioms for real and complex numbers.  You may
wish to look at how subtraction, division, and decimal numbers
are defined in \texttt{set.mm}, and for fun look at the proof of $2+
2 = 4$ (theorem \texttt{2p2e4} in \texttt{set.mm})
as discussed in section \ref{2p2e4}.

In \texttt{set.mm} we define the non-negative integers $\mathbb{N}$, the integers
$\mathbb{Z}$, and the rationals $\mathbb{Q}$ as subsets of $\mathbb{R}$.  This leads
to the nice inclusion $\mathbb{N} \subseteq \mathbb{Z} \subseteq \mathbb{Q} \subseteq
\mathbb{R} \subseteq \mathbb{C}$, giving us a uniform framework in which, for
example, a property such as commutativity of complex number addition
automatically applies to integers.  The natural numbers $\mathbb{N}$
are different from the set $\omega$ we defined earlier, but both satisfy
Peano's postulates.

\subsubsection{Complex Number Axioms in Analysis Texts}

Most analysis texts construct complex numbers as ordered pairs of reals,
leading to construction-dependent properties that satisfy these axioms
but are not stated in their pure form.  (This is also done in
\texttt{set.mm} but our axioms are extracted from that construction.)
Other texts will simply state that $\mathbb{R}$ is a ``complete ordered
subfield of $\mathbb{C}$,'' leading to redundant axioms when this phrase
is completely expanded out.  In fact I have not seen a text with the
axioms in the explicit form above.
None of these axioms is unique individually, but this carefully worked out
collection of axioms is the result of years of work
by the Metamath community.

\subsubsection{Eliminating Unnecessary Complex Number Axioms}

We once had more axioms for real and complex numbers, but over years of time
we (the Metamath community)
have found ways to eliminate them (by proving them from other axioms)
or weaken them (by making weaker claims without reducing what
can be proved).
In particular, here are statements that used to be complex number
axioms but have since been formally proven (with Metamath) to be redundant:

\begin{itemize}
\item
  $\mathbb{C} \in V$.
  At one time this was listed as a ``complex number axiom''
  However, this is not properly speaking a complex number axiom,
  and in any case its proof uses axioms of set theory.
  Proven redundant by Mario Carneiro\index{Carneiro, Mario} on
  17-Nov-2014 (see \texttt{axcnex}).
\item
  $((A \in \mathbb{C} \land B \in \mathbb{C}$) $\rightarrow$
  $(A + B) = (B + A))$.
  Proved redundant by Eric Schmidt\index{Schmidt, Eric} on 19-Jun-2012,
  and formalized by Scott Fenton\index{Fenton, Scott} on 3-Jan-2013
  (see \texttt{addcom}).
\item
  $(A \in \mathbb{C} \rightarrow (A + 0) = A)$.
  Proved redundant by Eric Schmidt on 19-Jun-2012,
  and formalized by Scott Fenton on 3-Jan-2013
  (see \texttt{addid1}).
\item
  $(A \in \mathbb{C} \rightarrow \exists x \in \mathbb{C} (A + x) = 0)$.
  Proved redundant by Eric Schmidt and formalized on 21-May-2007
  (see \texttt{cnegex}).
\item
  $((A \in \mathbb{C} \land A \ne 0) \rightarrow \exists x \in \mathbb{C} (A \cdot x) = 1)$.
  Proved redundant by Eric Schmidt and formalized on 22-May-2007
  (see \texttt{recex}).
\item
  $0 \in \mathbb{R}$.
  Proved redundant by Eric Schmidt on 19-Feb-2005 and formalized 21-May-2007
  (see \texttt{0re}).
\end{itemize}

We could eliminate 0 as an axiomatic object by defining it as
$( ( i \cdot i ) + 1 )$
and replacing it with this expression throughout the axioms. If this
is done, axiom ax-i2m1 becomes redundant. However, the remaining axioms
would become longer and less intuitive.

Eric Schmidt's paper analyzing this axiom system \cite{Schmidt}
presented a proof that these remaining axioms,
with the possible exception of ax-mulcom, are independent of the others.
It is currently an open question if ax-mulcom is independent of the others.

\section{Two Plus Two Equals Four}\label{2p2e4}

Here is a proof that $2 + 2 = 4$, as proven in the theorem \texttt{2p2e4}
in the database \texttt{set.mm}.
This is a useful demonstration of what a Metamath proof can look like.
This proof may have more steps than you're used to, but each step is rigorously
proven all the way back to the axioms of logic and set theory.
This display was originally generated by the Metamath program
as an {\sc HTML} file.

In the table showing the proof ``Step'' is the sequential step number,
while its associated ``Expression'' is an expression that we have proved.
``Ref'' is the name of a theorem or axiom that justifies that expression,
and ``Hyp'' refers to previous steps (if any) that the theorem or axiom
needs so that we can use it.  Expressions are indented further than
the expressions that depend on them to show their interdependencies.

\begin{table}[!htbp]
\caption{Two plus two equals four}
\begin{tabular}{lllll}
\textbf{Step} & \textbf{Hyp} & \textbf{Ref} & \textbf{Expression} & \\
1  &       & df-2    & $ \; \; \vdash 2 = 1 + 1$  & \\
2  & 1     & oveq2i  & $ \; \vdash (2 + 2) = (2 + (1 + 1))$ & \\
3  &       & df-4    & $ \; \; \vdash 4 = (3 + 1)$ & \\
4  &       & df-3    & $ \; \; \; \vdash 3 = (2 + 1)$ & \\
5  & 4     & oveq1i  & $ \; \; \vdash (3 + 1) = ((2 + 1) + 1)$ & \\
6  &       & 2cn     & $ \; \; \; \vdash 2 \in \mathbb{C}$ & \\
7  &       & ax-1cn  & $ \; \; \; \vdash 1 \in \mathbb{C}$ & \\
8  & 6,7,7 & addassi & $ \; \; \vdash ((2 + 1) + 1) = (2 + (1 + 1))$ & \\
9  & 3,5,8 & 3eqtri  & $ \; \vdash 4 = (2 + (1 + 1))$ & \\
10 & 2,9   & eqtr4i  & $ \vdash (2 + 2) = 4$ & \\
\end{tabular}
\end{table}

Step 1 says that we can assert that $2 = 1 + 1$ because it is
justified by \texttt{df-2}.
What is \texttt{df-2}?
It is simply the definition of $2$, which in our system is defined as being
equal to $1 + 1$.  This shows how we can use definitions in proofs.

Look at Step 2 of the proof. In the Ref column, we see that it references
a previously proved theorem, \texttt{oveq2i}.
It turns out that
theorem \texttt{oveq2i} requires a
hypothesis, and in the Hyp column of Step 2 we indicate that Step 1 will
satisfy (match) this hypothesis.
If we looked at \texttt{oveq2i}
we would find that it proves that given some hypothesis
$A = B$, we can prove that $( C F A ) = ( C F B )$.
If we use \texttt{oveq2i} and apply step 1's result as the hypothesis,
that will mean that $A = 2$ and $B = ( 1 + 1 )$ within this use of
\texttt{oveq2i}.
We are free to select any value of $C$ and $F$ (subject to syntax constraints),
so we are free to select $C = 2$ and $F = +$,
producing our desired result,
$ (2 + 2) = (2 + (1 + 1))$.

Step 2 is an example of substitution.
In the end, every step in every proof uses only this one substitution rule.
All the rules of logic, and all the axioms, are expressed so that
they can be used via this one substitution rule.
So once you master substitution, you can master every Metamath proof,
no exceptions.

Each step is clear and can be immediately checked.
In the {\sc HTML} display you can even click on each reference to see why it is
justified, making it easy to see why the proof works.

\section{Deduction}\label{deduction}

Strictly speaking,
a deduction (also called an inference) is a kind of statement that needs
some hypotheses to be true in order for its conclusion to be true.
A theorem, on the other hand, has no hypotheses.
Informally we often call both of them theorems, but in this section we
will stick to the strict definitions.

It sometimes happens that we have proved a deduction of the form
$\varphi \Rightarrow \psi$\index{$\Rightarrow$}
(given hypothesis $\varphi$ we can prove $\psi$)
and we want to then prove a theorem of the form
$\varphi \rightarrow \psi$.

Converting a deduction (which uses a hypothesis) into a theorem
(which does not) is not as simple as you might think.
The deduction says, ``if we can prove $\varphi$ then we can prove $\psi$,''
which is in some sense weaker than saying
``$\varphi$ implies $\psi$.''
There is no axiom of logic that permits us to directly obtain the theorem
given the deduction.\footnote{
The conversion of a deduction to a theorem does not even hold in general
for quantum propositional calculus,
which is a weak subset of classical propositional calculus.
It has been shown that adding the Standard Deduction Theorem (discussed below)
to quantum propositional calculus turns it into classical
propositional calculus!
}

This is in contrast to going the other way.
If we have the theorem ($\varphi \rightarrow \psi$),
it is easy to recover the deduction
($\varphi \Rightarrow \psi$)
using modus ponens\index{modus ponens}
(\texttt{ax-mp}; see section \ref{axmp}).

In the following subsections we first discuss the standard deduction theorem
(the traditional but awkward way to convert deductions into theorems) and
the weak deduction theorem (a limited version of the standard deduction
theorem that is easier to use and was once widely used in
\texttt{set.mm}\index{set theory database (\texttt{set.mm})}).
In section \ref{deductionstyle} we discuss
deduction style, the newer approach we now recommend in most cases.
Deduction style uses ``deduction form,'' a form that
prefixes each hypothesis (other than definitions) and the
conclusion with a universal antecedent (``$\varphi \rightarrow$'').
Deduction style is widely used in \texttt{set.mm},
so it is useful to understand it and \textit{why} it is widely used.
Section \ref{naturaldeduction}
briefly discusses our approach for using natural deduction
within \texttt{set.mm},
as that approach is deeply related to deduction style.
We conclude with a summary of the strengths of
our approach, which we believe are compelling.

\subsection{The Standard Deduction Theorem}\label{standarddeductiontheorem}

It is possible to make use of information
contained in the deduction or its proof to assist us with the proof of
the related theorem.
In traditional logic books, there is a metatheorem called the
Deduction Theorem\index{Deduction Theorem}\index{Standard Deduction Theorem},
discovered independently by Herbrand and Tarski around 1930.
The Deduction Theorem, which we often call the Standard Deduction Theorem,
provides an algorithm for constructing a proof of a theorem from the
proof of its corresponding deduction. See, for example,
\cite[p.~56]{Margaris}\index{Margaris, Angelo}.
To construct a proof for a theorem, the
algorithm looks at each step in the proof of the original deduction and
rewrites the step with several steps wherein the hypothesis is eliminated
and becomes an antecedent.

In ordinary mathematics, no one actually carries out the algorithm,
because (in its most basic form) it involves an exponential explosion of
the number of proof steps as more hypotheses are eliminated. Instead,
the Standard Deduction Theorem is invoked simply to claim that it can
be done in principle, without actually doing it.
What's more, the algorithm is not as simple as it might first appear
when applying it rigorously.
There is a subtle restriction on the Standard Deduction Theorem
that must be taken into account involving the axiom of generalization
when working with predicate calculus (see the literature for more detail).

One of the goals of Metamath is to let you plainly see, with as few
underlying concepts as possible, how mathematics can be derived directly
from the axioms, and not indirectly according to some hidden rules
buried inside a program or understood only by logicians. If we added
the Standard Deduction Theorem to the language and proof verifier,
that would greatly complicate both and largely defeat Metamath's goal
of simplicity. In principle, we could show direct proofs by expanding
out the proof steps generated by the algorithm of the Standard Deduction
Theorem, but that is not feasible in practice because the number of proof
steps quickly becomes huge, even astronomical.
Since the algorithm of the Standard Deduction Theorem is driven by the proof,
we would have to go through that proof
all over again---starting from axioms---in order to obtain the theorem form.
In terms of proof length, there would be no savings over just
proving the theorem directly instead of first proving the deduction form.

\subsection{Weak Deduction Theorem}\label{weakdeductiontheorem}

We have developed
a more efficient method for proving a theorem from a deduction
that can be used instead of the Standard Deduction Theorem
in many (but not all) cases.
We call this more efficient method the
Weak Deduction Theorem\index{Weak Deduction Theorem}.\footnote{
There is also an unrelated ``Weak Deduction Theorem''
in the field of relevance logic, so to avoid confusion we could call
ours the ``Weak Deduction Theorem for Classical Logic.''}
Unlike the Standard Deduction Theorem, the Weak Deduction Theorem produces the
theorem directly from a special substitution instance of the deduction,
using a small, fixed number of steps roughly proportional to the length
of the final theorem.

If you come to a proof referencing the Weak Deduction Theorem
\texttt{dedth} (or one of its variants \texttt{dedthxx}),
here is how to follow the proof without getting into the details:
just click on the theorem referenced in the step
just before the reference to \texttt{dedth} and ignore everything else.
Theorem \texttt{dedth} simply turns a hypothesis into an antecedent
(i.e. the hypothesis followed by $\rightarrow$
is placed in front of the assertion, and the hypothesis
itself is eliminated) given certain conditions.

The Weak Deduction Theorem
eliminates a hypothesis $\varphi$, making it become an antecedent.
It does this by proving an expression
$ \varphi \rightarrow \psi $ given two hypotheses:
(1)
$ ( A = {\rm if} ( \varphi , A , B ) \rightarrow ( \varphi \leftrightarrow \chi ) ) $
and
(2) $\chi$.
Note that it requires that a proof exists for $\varphi$ when the class variable
$A$ is replaced with a specific class $B$. The hypothesis $\chi$
should be assigned to the inference.
You can see the details of the proof of the Weak Deduction Theorem
in theorem \texttt{dedth}.

The Weak Deduction Theorem
is probably easier to understand by studying proofs that make use of it.
For example, let's look at the proof of \texttt{renegcl}, which proves that
$ \vdash ( A \in \mathbb{R} \rightarrow - A \in \mathbb{R} )$:

\needspace{4\baselineskip}
\begin{longtabu} {l l l X}
\textbf{Step} & \textbf{Hyp} & \textbf{Ref} & \textbf{Expression} \\
  1 &  & negeq &
  $\vdash$ $($ $A$ $=$ ${\rm if}$ $($ $A$ $\in$ $\mathbb{R}$ $,$ $A$ $,$ $1$ $)$ $\rightarrow$
  $\textrm{-}$ $A$ $=$ $\textrm{-}$ ${\rm if}$ $($ $A$ $\in$ $\mathbb{R}$
  $,$ $A$ $,$ $1$ $)$ $)$ \\
 2 & 1 & eleq1d &
    $\vdash$ $($ $A$ $=$ ${\rm if}$ $($ $A$ $\in$ $\mathbb{R}$ $,$ $A$ $,$ $1$ $)$ $\rightarrow$ $($
    $\textrm{-}$ $A$ $\in$ $\mathbb{R}$ $\leftrightarrow$
    $\textrm{-}$ ${\rm if}$ $($ $A$ $\in$ $\mathbb{R}$ $,$ $A$ $,$ $1$ $)$ $\in$
    $\mathbb{R}$ $)$ $)$ \\
 3 &  & 1re & $\vdash 1 \in \mathbb{R}$ \\
 4 & 3 & elimel &
   $\vdash {\rm if} ( A \in \mathbb{R} , A , 1 ) \in \mathbb{R}$ \\
 5 & 4 & renegcli &
   $\vdash \textrm{-} {\rm if} ( A \in \mathbb{R} , A , 1 ) \in \mathbb{R}$ \\
 6 & 2,5 & dedth &
   $\vdash ( A \in \mathbb{R} \rightarrow \textrm{-} A \in \mathbb{R}$ ) \\
\end{longtabu}

The somewhat strange-looking steps in \texttt{renegcl} before step 5 are
technical stuff that makes this magic work, and they can be ignored
for a quick overview of the proof. To continue following the ``important''
part of the proof of \texttt{renegcl},
you can look at the reference to \texttt{renegcli} at step 5.

That said, let's briefly look at how
\texttt{renegcl} uses the
Weak Deduction Theorem (\texttt{dedth}) to do its job,
in case you want to do something similar or want understand it more deeply.
Let's work backwards in the proof of \texttt{renegcl}.
Step 6 applies \texttt{dedth} to produce our goal result
$ \vdash ( A \in \mathbb{R} \rightarrow\, - A \in \mathbb{R} )$.
This requires on the one hand the (substituted) deduction
\texttt{renegcli} in step 5.
By itself \texttt{renegcli} proves the deduction
$ \vdash A \in \mathbb{R} \Rightarrow\, \vdash - A \in \mathbb{R}$;
this is the deduction form we are trying to turn into theorem form,
and thus
\texttt{renegcli} has a separate hypothesis that must be fulfilled.
To fulfill the hypothesis of the invocation of
\texttt{renegcli} in step 5, it is eventually
reduced to the already proven theorem $1 \in \mathbb{R}$ in step 3.
Step 4 connects steps 3 and 5; step 4 invokes
\texttt{elimel}, a special case of \texttt{elimhyp} that eliminates
a membership hypothesis for the weak deduction theorem.
On the other hand, the equivalence of the conclusion of
\texttt{renegcl}
$( - A \in \mathbb{R} )$ and the substituted conclusion of
\texttt{renegcli} must be proven, which is done in steps 2 and 1.

The weak deduction theorem has limitations.
In particular, we must be able to prove a special case of the deduction's
hypothesis as a stand-alone theorem.
For example, we used $1 \in \mathbb{R}$ in step 3 of \texttt{renegcl}.

We used to use the weak deduction theorem
extensively within \texttt{set.mm}.
However, we now recommend applying ``deduction style''
instead in most cases, as deduction style is
often an easier and clearer approach.
Therefore, we will now describe deduction style.

\subsection{Deduction Style}\label{deductionstyle}

We now prefer to write assertions in ``deduction form''
instead of writing a proof that would require use of the standard or
weak deduction theorem.
We call this appraoch
``deduction style.''\index{deduction style}

It will be easier to explain this by first defining some terms:

\begin{itemize}
\item \textbf{closed form}\index{closed form}\index{forms!closed}:
A kind of assertion (theorem) with no hypotheses.
Typically its label has no special suffix.
An example is \texttt{unss}, which states:
$\vdash ( ( A \subseteq C \wedge B \subseteq C ) \leftrightarrow ( A \cup B )
\subseteq C )\label{eq:unss}$
\item \textbf{deduction form}\index{deduction form}\index{forms!deduction}:
A kind of assertion with one or more hypotheses
where the conclusion is an implication with
a wff variable as the antecedent (usually $\varphi$), and every hypothesis
(\$e statement)
is either (1) an implication with the same antecedent as the conclusion or
(2) a definition.
A definition
can be for a class variable (this is a class variable followed by ``='')
or a wff variable (this is a wff variable followed by $\leftrightarrow$);
class variable definitions are more common.
In practice, a proof
in deduction form will also contain many steps that are implications
where the antecedent is either that wff variable (normally $\varphi$)
or is
a conjunction (...$\land$...) including that wff variable ($\varphi$).
If an assertion is in deduction form, and other forms are also available,
then we suffix its label with ``d.''
An example is \texttt{unssd}, which states\footnote{
For brevity we show here (and in other places)
a $\&$\index{$\&$} between hypotheses\index{hypotheses}
and a $\Rightarrow$\index{$\Rightarrow$}\index{conclusion}
between the hypotheses and the conclusion.
This notation is technically not part of the Metamath language, but is
instead a convenient abbreviation to show both the hypotheses and conclusion.}:
$\vdash ( \varphi \rightarrow A \subseteq C )\quad\&\quad \vdash ( \varphi
    \rightarrow B \subseteq C )\quad\Rightarrow\quad \vdash ( \varphi
    \rightarrow ( A \cup B ) \subseteq C )\label{eq:unssd}$
\item \textbf{inference form}\index{inference form}\index{forms!inference}:
A kind of assertion with one or more hypotheses that is not in deduction form
(e.g., there is no common antecedent).
If an assertion is in inference form, and other forms are also available,
then we suffix its label with ``i.''
An example is \texttt{unssi}, which states:
$\vdash A \subseteq C\quad\&\quad \vdash B \subseteq C\quad\Rightarrow\quad
    \vdash ( A \cup B ) \subseteq C\label{eq:unssi}$
\end{itemize}

When using deduction style we express an assertion in deduction form.
This form prefixes each hypothesis (other than definitions) and the
conclusion with a universal antecedent (``$\varphi \rightarrow$'').
The antecedent (e.g., $\varphi$)
mimics the context handled in the deduction theorem, eliminating
the need to directly use the deduction theorem.

Once you have an assertion in deduction form, you can easily convert it
to inference form or closed form:

\begin{itemize}
\item To
prove some assertion Ti in inference form, given assertion Td in deduction
form, there is a simple mechanical process you can use. First take each
Ti hypothesis and insert a \texttt{T.} $\rightarrow$ prefix (``true implies'')
using \texttt{a1i}. You
can then use the existing assertion Td to prove the resulting conclusion
with a \texttt{T.} $\rightarrow$ prefix.
Finally, you can remove that prefix using \texttt{trud},
resulting in the conclusion you wanted to prove.
\item To
prove some assertion T in closed form, given assertion Td in deduction
form, there is another simple mechanical process you can use. First,
select an expression that is the conjunction (...$\land$...) of all of the
consequents of every hypothesis of Td. Next, prove that this expression
implies each of the separate hypotheses of Td in turn by eliminating
conjuncts (there are a variety of proven assertions to do this, including
\texttt{simpl},
\texttt{simpr},
\texttt{3simpa},
\texttt{3simpb},
\texttt{3simpc},
\texttt{simp1},
\texttt{simp2},
and
\texttt{simp3}).
If the
expression has nested conjunctions, inner conjuncts can be broken out by
chaining the above theorems with \texttt{syl}
(see section \ref{syl}).\footnote{
There are actually many theorems
(labeled simp* such as \texttt{simp333}) that break out inner conjuncts in one
step, but rather than learning them you can just use the chaining we
just described to prove them, and then let the Metamath program command
\texttt{minimize{\char`\_}with}\index{\texttt{minimize{\char`\_}with} command}
figure out the right ones needed to collapse them.}
As your final step, you can then apply the already-proven assertion Td
(which is in deduction form), proving assertion T in closed form.
\end{itemize}

We can also easily convert any assertion T in closed form to its related
assertion Ti in inference form by applying
modus ponens\index{modus ponens} (see section \ref{axmp}).

The deduction form antecedent can also be used to represent the context
necessary to support natural deduction systems, so we will now
discuss natural deduction.

\subsection{Natural Deduction}\label{naturaldeduction}

Natural deduction\index{natural deduction}
(ND) systems, as such, were originally introduced in
1934 by two logicians working independently: Ja\'skowski and Gentzen. ND
systems are supposed to reconstruct, in a formally proper way, traditional
ways of mathematical reasoning (such as conditional proof, indirect proof,
and proof by cases). As reconstructions they were naturally influenced
by previous work, and many specific ND systems and notations have been
developed since their original work.

There are many ND variants, but
Indrzejczak \cite[p.~31-32]{Indrzejczak}\index{Indrzejczak, Andrzej}
suggests that any natural deductive system must satisfy at
least these three criteria:

\begin{itemize}
\item ``There are some means for entering assumptions into a proof and
also for eliminating them. Usually it requires some bookkeeping devices
for indicating the scope of an assumption, and showing that a part of
a proof depending on eliminated assumption is discharged.
\item There are no (or, at least, a very limited set of) axioms, because
their role is taken over by the set of primitive rules for introduction
and elimination of logical constants which means that elementary
inferences instead of formulae are taken as primitive.
\item (A genuine) ND system admits a lot of freedom in proof construction
and possibility of applying several proof search strategies, like
conditional proof, proof by cases, proof by reductio ad absurdum etc.''
\end{itemize}

The Metamath Proof Explorer (MPE) as defined in \texttt{set.mm}
is fundamentally a Hilbert-style system.
That is, MPE is based on a larger number of axioms (compared
to natural deduction systems), a very small set of rules of inference
(modus ponens), and the context is not changed by the rules of inference
in the middle of a proof. That said, MPE proofs can be developed using
the natural deduction (ND) approach as originally developed by Ja\'skowski
and Gentzen.

The most common and recommended approach for applying ND in MPE is to use
deduction form\index{deduction form}%
\index{forms!deduction}
and apply the MPE proven assertions that are equivalent to ND rules.
For example, MPE's \texttt{jca} is equivalent to ND rule $\land$-I
(and-insertion).
We maintain a list of equivalences that you may consult.
This approach for applying an ND approach within MPE relies on Metamath's
wff metavariables in an essential way, and is described in more detail
in the presentation ``Natural Deductions in the Metamath Proof Language''
by Mario Carneiro \cite{CarneiroND}\index{Carneiro, Mario}.

In this style many steps are an implication, whose antecedent mimics
the context ($\Gamma$) of most ND systems. To add an assumption, simply add
it to the implication antecedent (typically using
\texttt{simpr}),
and use that
new antecedent for all later claims in the same scope. If you wish to
use an assertion in an ND hypothesis scope that is outside the current
ND hypothesis scope, modify the assertion so that the ND hypothesis
assumption is added to its antecedent (typically using \texttt{adantr}). Most
proof steps will be proved using rules that have hypotheses and results
of the form $\varphi \rightarrow$ ...

An example may make this clearer.
Let's show theorem 5.5 of
\cite[p.~18]{Clemente}\index{Clemente Laboreo, Daniel}
along with a line by line translation using the usual
translation of natural deduction (ND) in the Metamath Proof Explorer
(MPE) notation (this is proof \texttt{ex-natded5.5}).
The proof's original goal was to prove
$\lnot \psi$ given two hypotheses,
$( \psi \rightarrow \chi )$ and $ \lnot \chi$.
We will translate these statements into MPE deduction form
by prefixing them all with $\varphi \rightarrow$.
As a result, in MPE the goal is stated as
$( \varphi \rightarrow \lnot \psi )$, and the two hypotheses are stated as
$( \varphi \rightarrow ( \psi \rightarrow \chi ) )$ and
$( \varphi \rightarrow \lnot \chi )$.

The following table shows the proof in Fitch natural deduction style
and its MPE equivalent.
The \textit{\#} column shows the original numbering,
\textit{MPE\#} shows the number in the equivalent MPE proof
(which we will show later),
\textit{ND Expression} shows the original proof claim in ND notation,
and \textit{MPE Translation} shows its translation into MPE
as discussed in this section.
The final columns show the rationale in ND and MPE respectively.

\needspace{4\baselineskip}
{\setlength{\extrarowsep}{4pt} % Keep rows from being too close together
\begin{longtabu}   { @{} c c X X X X }
\textbf{\#} & \textbf{MPE\#} & \textbf{ND Ex\-pres\-sion} &
\textbf{MPE Trans\-lation} & \textbf{ND Ration\-ale} &
\textbf{MPE Ra\-tio\-nale} \\
\endhead

1 & 2;3 &
$( \psi \rightarrow \chi )$ &
$( \varphi \rightarrow ( \psi \rightarrow \chi ) )$ &
Given &
\$e; \texttt{adantr} to put in ND hypothesis \\

2 & 5 &
$ \lnot \chi$ &
$( \varphi \rightarrow \lnot \chi )$ &
Given &
\$e; \texttt{adantr} to put in ND hypothesis \\

3 & 1 &
... $\vert$ $\psi$ &
$( \varphi \rightarrow \psi )$ &
ND hypothesis assumption &
\texttt{simpr} \\

4 & 4 &
... $\chi$ &
$( ( \varphi \land \psi ) \rightarrow \chi )$ &
$\rightarrow$\,E 1,3 &
\texttt{mpd} 1,3 \\

5 & 6 &
... $\lnot \chi$ &
$( ( \varphi \land \psi ) \rightarrow \lnot \chi )$ &
IT 2 &
\texttt{adantr} 5 \\

6 & 7 &
$\lnot \psi$ &
$( \varphi \rightarrow \lnot \psi )$ &
$\land$\,I 3,4,5 &
\texttt{pm2.65da} 4,6 \\

\end{longtabu}
}


The original used Latin letters; we have replaced them with Greek letters
to follow Metamath naming conventions and so that it is easier to follow
the Metamath translation. The Metamath line-for-line translation of
this natural deduction approach precedes every line with an antecedent
including $\varphi$ and uses the Metamath equivalents of the natural deduction
rules. To add an assumption, the antecedent is modified to include it
(typically by using \texttt{adantr};
\texttt{simpr} is useful when you want to
depend directly on the new assumption, as is shown here).

In Metamath we can represent the two given statements as these hypotheses:

\needspace{2\baselineskip}
\begin{itemize}
\item ex-natded5.5.1 $\vdash ( \varphi \rightarrow ( \psi \rightarrow \chi ) )$
\item ex-natded5.5.2 $\vdash ( \varphi \rightarrow \lnot \chi )$
\end{itemize}

\needspace{4\baselineskip}
Here is the proof in Metamath as a line-by-line translation:

\begin{longtabu}   { l l l X }
\textbf{Step} & \textbf{Hyp} & \textbf{Ref} & \textbf{Ex\-pres\-sion} \\
\endhead
1 & & simpr & $\vdash ( ( \varphi \land \psi ) \rightarrow \psi )$ \\
2 & & ex-natded5.5.1 &
  $\vdash ( \varphi \rightarrow ( \psi \rightarrow \chi ) )$ \\
3 & 2 & adantr &
 $\vdash ( ( \varphi \land \psi ) \rightarrow ( \psi \rightarrow \chi ) )$ \\
4 & 1, 3 & mpd &
 $\vdash ( ( \varphi \land \psi ) \rightarrow \chi ) $ \\
5 & & ex-natded5.5.2 &
 $\vdash ( \varphi \rightarrow \lnot \chi )$ \\
6 & 5 & adantr &
 $\vdash ( ( \varphi \land \psi ) \rightarrow \lnot \chi )$ \\
7 & 4, 6 & pm2.65da &
 $\vdash ( \varphi \rightarrow \lnot \psi )$ \\
\end{longtabu}

Only using specific natural deduction rules directly can lead to very
long proofs, for exactly the same reason that only using axioms directly
in Hilbert-style proofs can lead to very long proofs.
If the goal is short and clear proofs,
then it is better to reuse already-proven assertions
in deduction form than to start from scratch each time
and using only basic natural deduction rules.

\subsection{Strengths of Our Approach}

As far as we know there is nothing else in the literature like either the
weak deduction theorem or Mario Carneiro\index{Carneiro, Mario}'s
natural deduction method.
In order to
transform a hypothesis into an antecedent, the literature's standard
``Deduction Theorem''\index{Deduction Theorem}\index{Standard Deduction Theorem}
requires metalogic outside of the notions provided
by the axiom system. We instead generally prefer to use Mario Carneiro's
natural deduction method, then use the weak deduction theorem in cases
where that is difficult to apply, and only then use the full standard
deduction theorem as a last resort.

The weak deduction theorem\index{Weak Deduction Theorem}
does not require any additional metalogic
but converts an inference directly into a closed form theorem, with
a rigorous proof that uses only the axiom system. Unlike the standard
Deduction Theorem, there is no implicit external justification that we
have to trust in order to use it.

Mario Carneiro's natural deduction\index{natural deduction}
method also does not require any new metalogical
notions. It avoids the Deduction Theorem's metalogic by prefixing the
hypotheses and conclusion of every would-be inference with a universal
antecedent (``$\varphi \rightarrow$'') from the very start.

We think it is impressive and satisfying that we can do so much in a
practical sense without stepping outside of our Hilbert-style axiom system.
Of course our axiomatization, which is in the form of schemes,
contains a metalogic of its own that we exploit. But this metalogic
is relatively simple, and for our Deduction Theorem alternatives,
we primarily use just the direct substitution of expressions for
metavariables.

\begin{sloppy}
\section{Exploring the Set The\-o\-ry Data\-base}\label{exploring}
\end{sloppy}
% NOTE: All examples performed in this section are
% recorded wtih "set width 61" % on set.mm as of 2019-05-28
% commit c1e7849557661260f77cfdf0f97ac4354fbb4f4d.

At this point you may wish to study the \texttt{set.mm}\index{set theory
database (\texttt{set.mm})} file in more detail.  Pay particular
attention to the assumptions needed to define wffs\index{well-formed
formula (wff)} (which are not included above), the variable types
(\texttt{\$f}\index{\texttt{\$f} statement} statements), and the
definitions that are introduced.  Start with some simple theorems in
propositional calculus, making sure you understand in detail each step
of a proof.  Once you get past the first few proofs and become familiar
with the Metamath language, any part of the \texttt{set.mm} database
will be as easy to follow, step by step, as any other part---you won't
have to undergo a ``quantum leap'' in mathematical sophistication to be
able to follow a deep proof in set theory.

Next, you may want to explore how concepts such as natural numbers are
defined and described.  This is probably best done in conjunction with
standard set theory textbooks, which can help give you a higher-level
understanding.  The \texttt{set.mm} database provides references that will get
you started.  From there, you will be on your way towards a very deep,
rigorous understanding of abstract mathematics.

The Metamath\index{Metamath} program can help you peruse a Metamath data\-base,
wheth\-er you are trying to figure out how a certain step follows in a proof or
just have a general curiosity.  We will go through some examples of the
commands, using the \texttt{set.mm}\index{set theory database (\texttt{set.mm})}
database provided with the Metamath software.  These should help get you
started.  See Chapter~\ref{commands} for a more detailed description of
the commands.  Note that we have included the full spelling of all commands to
prevent ambiguity with future commands.  In practice you may type just the
characters needed to specify each command keyword\index{command keyword}
unambiguously, often just one or two characters per keyword, and you don't
need to type them in upper case.

First run the Metamath program as described earlier.  You should see the
\verb/MM>/ prompt.  Read in the \texttt{set.mm} file:\index{\texttt{read}
command}

\begin{verbatim}
MM> read set.mm
Reading source file "set.mm"... 34554442 bytes
34554442 bytes were read into the source buffer.
The source has 155711 statements; 2254 are $a and 32250 are $p.
No errors were found.  However, proofs were not checked.
Type VERIFY PROOF * if you want to check them.
\end{verbatim}

As with most examples in this book, what you will see
will be slightly different because we are continuously
improving our databases (including \texttt{set.mm}).

Let's check the database integrity.  This may take a minute or two to run if
your computer is slow.

\begin{verbatim}
MM> verify proof *
0 10%  20%  30%  40%  50%  60%  70%  80%  90% 100%
..................................................
All proofs in the database were verified in 2.84 s.
\end{verbatim}

No errors were reported, so every proof is correct.

You need to know the names (labels) of theorems before you can look at them.
Often just examining the database file(s) with a text editor is the best
approach.  In \texttt{set.mm} there are many detailed comments, especially near
the beginning, that can help guide you. The \texttt{search} command in the
Metamath program is also handy.  The \texttt{comments} qualifier will list the
statements whose associated comment (the one immediately before it) contain a
string you give it.  For example, if you are studying Enderton's {\em Elements
of Set Theory} \cite{Enderton}\index{Enderton, Herbert B.} you may want to see
the references to it in the database.  The search string \texttt{enderton} is not
case sensitive.  (This will not show you all the database theorems that are in
Enderton's book because there is usually only one citation for a given
theorem, which may appear in several textbooks.)\index{\texttt{search}
command}

\begin{verbatim}
MM> search * "enderton" / comments
12067 unineq $p "... Exercise 20 of [Enderton] p. 32 and ..."
12459 undif2 $p "...Corollary 6K of [Enderton] p. 144. (C..."
12953 df-tp $a "...s. Definition of [Enderton] p. 19. (Co..."
13689 unissb $p ".... Exercise 5 of [Enderton] p. 26 and ..."
\end{verbatim}
\begin{center}
(etc.)
\end{center}

Or you may want to see what theorems have something to do with
conjunction (logical {\sc and}).  The quotes around the search
string are optional when there's no ambiguity.\index{\texttt{search}
command}

\begin{verbatim}
MM> search * conjunction / comments
120 a1d $p "...be replaced with a conjunction ( ~ df-an )..."
662 df-bi $a "...viated form after conjunction is introdu..."
1319 wa $a "...ff definition to include conjunction ('and')."
1321 df-an $a "Define conjunction (logical 'and'). Defini..."
1420 imnan $p "...tion in terms of conjunction. (Contribu..."
\end{verbatim}
\begin{center}
(etc.)
\end{center}


Now we will start to look at some details.  Let's look at the first
axiom of propositional calculus
(we could use \texttt{sh st} to abbreviate
\texttt{show statement}).\index{\texttt{show statement} command}

\begin{verbatim}
MM> show statement ax-1/full
Statement 19 is located on line 881 of the file "set.mm".
"Axiom _Simp_.  Axiom A1 of [Margaris] p. 49.  One of the 3
axioms of propositional calculus.  The 3 axioms are also
        ...
19 ax-1 $a |- ( ph -> ( ps -> ph ) ) $.
Its mandatory hypotheses in RPN order are:
  wph $f wff ph $.
  wps $f wff ps $.
The statement and its hypotheses require the variables:  ph
      ps
The variables it contains are:  ph ps


Statement 49 is located on line 11182 of the file "set.mm".
Its statement number for HTML pages is 6.
"Axiom _Simp_.  Axiom A1 of [Margaris] p. 49.  One of the 3
axioms of propositional calculus.  The 3 axioms are also
given as Definition 2.1 of [Hamilton] p. 28.
...
49 ax-1 $a |- ( ph -> ( ps -> ph ) ) $.
Its mandatory hypotheses in RPN order are:
  wph $f wff ph $.
  wps $f wff ps $.
The statement and its hypotheses require the variables:
  ph ps
The variables it contains are:  ph ps
\end{verbatim}

Compare this to \texttt{ax-1} on p.~\pageref{ax1}.  You can see that the
symbol \texttt{ph} is the {\sc ascii} notation for $\varphi$, etc.  To
see the mathematical symbols for any expression you may typeset it in
\LaTeX\ (type \texttt{help tex} for instructions)\index{latex@{\LaTeX}}
or, easier, just use a text editor to look at the comments where symbols
are first introduced in \texttt{set.mm}.  The hypotheses \texttt{wph}
and \texttt{wps} required by \texttt{ax-1} mean that variables
\texttt{ph} and \texttt{ps} must be wffs.

Next we'll pick a simple theorem of propositional calculus, the Principle of
Identity, which is proved directly from the axioms.  We'll look at the
statement then its proof.\index{\texttt{show statement}
command}

\begin{verbatim}
MM> show statement id1/full
Statement 116 is located on line 11371 of the file "set.mm".
Its statement number for HTML pages is 22.
"Principle of identity.  Theorem *2.08 of [WhiteheadRussell]
p. 101.  This version is proved directly from the axioms for
demonstration purposes.
...
116 id1 $p |- ( ph -> ph ) $= ... $.
Its mandatory hypotheses in RPN order are:
  wph $f wff ph $.
Its optional hypotheses are:  wps wch wth wta wet
      wze wsi wrh wmu wla wka
The statement and its hypotheses require the variables:  ph
These additional variables are allowed in its proof:
      ps ch th ta et ze si rh mu la ka
The variables it contains are:  ph
\end{verbatim}

The optional variables\index{optional variable} \texttt{ps}, \texttt{ch}, etc.\ are
available for use in a proof of this statement if we wish, and were we to do
so we would make use of optional hypotheses \texttt{wps}, \texttt{wch}, etc.  (See
Section~\ref{dollaref} for the meaning of ``optional
hypothesis.''\index{optional hypothesis}) The reason these show up in the
statement display is that statement \texttt{id1} happens to be in their scope
(see Section~\ref{scoping} for the definition of ``scope''\index{scope}), but
in fact in propositional calculus we will never make use of optional
hypotheses or variables.  This becomes important after quantifiers are
introduced, where ``dummy'' variables\index{dummy variable} are often needed
in the middle of a proof.

Let's look at the proof of statement \texttt{id1}.  We'll use the
\texttt{show proof} command, which by default suppresses the
``non-essential'' steps that construct the wffs.\index{\texttt{show proof}
command}
We will display the proof in ``lemmon' format (a non-indented format
with explicit previous step number references) and renumber the
displayed steps:

\begin{verbatim}
MM> show proof id1 /lemmon/renumber
1 ax-1           $a |- ( ph -> ( ph -> ph ) )
2 ax-1           $a |- ( ph -> ( ( ph -> ph ) -> ph ) )
3 ax-2           $a |- ( ( ph -> ( ( ph -> ph ) -> ph ) ) ->
                     ( ( ph -> ( ph -> ph ) ) -> ( ph -> ph )
                                                          ) )
4 2,3 ax-mp      $a |- ( ( ph -> ( ph -> ph ) ) -> ( ph -> ph
                                                          ) )
5 1,4 ax-mp      $a |- ( ph -> ph )
\end{verbatim}

If you have read Section~\ref{trialrun}, you'll know how to interpret this
proof.  Step~2, for example, is an application of axiom \texttt{ax-1}.  This
proof is identical to the one in Hamilton's {\em Logic for Mathematicians}
\cite[p.~32]{Hamilton}\index{Hamilton, Alan G.}.

You may want to look at what
substitutions\index{substitution!variable}\index{variable substitution} are
made into \texttt{ax-1} to arrive at step~2.  The command to do this needs to
know the ``real'' step number, so we'll display the proof again without
the \texttt{renumber} qualifier.\index{\texttt{show proof}
command}

\begin{verbatim}
MM> show proof id1 /lemmon
 9 ax-1          $a |- ( ph -> ( ph -> ph ) )
20 ax-1          $a |- ( ph -> ( ( ph -> ph ) -> ph ) )
24 ax-2          $a |- ( ( ph -> ( ( ph -> ph ) -> ph ) ) ->
                     ( ( ph -> ( ph -> ph ) ) -> ( ph -> ph )
                                                          ) )
25 20,24 ax-mp   $a |- ( ( ph -> ( ph -> ph ) ) -> ( ph -> ph
                                                          ) )
26 9,25 ax-mp    $a |- ( ph -> ph )
\end{verbatim}

The ``real'' step number is 20.  Let's look at its details.

\begin{verbatim}
MM> show proof id1 /detailed_step 20
Proof step 20:  min=ax-1 $a |- ( ph -> ( ( ph -> ph ) -> ph )
  )
This step assigns source "ax-1" ($a) to target "min" ($e).
The source assertion requires the hypotheses "wph" ($f, step
18) and "wps" ($f, step 19).  The parent assertion of the
target hypothesis is "ax-mp" ($a, step 25).
The source assertion before substitution was:
    ax-1 $a |- ( ph -> ( ps -> ph ) )
The following substitutions were made to the source
assertion:
    Variable  Substituted with
     ph        ph
     ps        ( ph -> ph )
The target hypothesis before substitution was:
    min $e |- ph
The following substitution was made to the target hypothesis:
    Variable  Substituted with
     ph        ( ph -> ( ( ph -> ph ) -> ph ) )
\end{verbatim}

This shows the substitutions\index{substitution!variable}\index{variable
substitution} made to the variables in \texttt{ax-1}.  References are made to
steps 18 and 19 which are not shown in our proof display.  To see these steps,
you can display the proof with the \texttt{all} qualifier.

Let's look at a slightly more advanced proof of propositional calculus.  Note
that \verb+/\+ is the symbol for $\wedge$ (logical {\sc and}, also
called conjunction).\index{conjunction ($\wedge$)}
\index{logical {\sc and} ($\wedge$)}

\begin{verbatim}
MM> show statement prth/full
Statement 1791 is located on line 15503 of the file "set.mm".
Its statement number for HTML pages is 559.
"Conjoin antecedents and consequents of two premises.  This
is the closed theorem form of ~ anim12d .  Theorem *3.47 of
[WhiteheadRussell] p. 113.  It was proved by Leibniz,
and it evidently pleased him enough to call it
_praeclarum theorema_ (splendid theorem).
...
1791 prth $p |- ( ( ( ph -> ps ) /\ ( ch -> th ) ) -> ( ( ph
      /\ ch ) -> ( ps /\ th ) ) ) $= ... $.
Its mandatory hypotheses in RPN order are:
  wph $f wff ph $.
  wps $f wff ps $.
  wch $f wff ch $.
  wth $f wff th $.
Its optional hypotheses are:  wta wet wze wsi wrh wmu wla wka
The statement and its hypotheses require the variables:  ph
      ps ch th
These additional variables are allowed in its proof:  ta et
      ze si rh mu la ka
The variables it contains are:  ph ps ch th


MM> show proof prth /lemmon/renumber
1 simpl          $p |- ( ( ( ph -> ps ) /\ ( ch -> th ) ) ->
                                               ( ph -> ps ) )
2 simpr          $p |- ( ( ( ph -> ps ) /\ ( ch -> th ) ) ->
                                               ( ch -> th ) )
3 1,2 anim12d    $p |- ( ( ( ph -> ps ) /\ ( ch -> th ) ) ->
                           ( ( ph /\ ch ) -> ( ps /\ th ) ) )
\end{verbatim}

There are references to a lot of unfamiliar statements.  To see what they are,
you may type the following:

\begin{verbatim}
MM> show proof prth /statement_summary
Summary of statements used in the proof of "prth":

Statement simpl is located on line 14748 of the file
"set.mm".
"Elimination of a conjunct.  Theorem *3.26 (Simp) of
[WhiteheadRussell] p. 112. ..."
  simpl $p |- ( ( ph /\ ps ) -> ph ) $= ... $.

Statement simpr is located on line 14777 of the file
"set.mm".
"Elimination of a conjunct.  Theorem *3.27 (Simp) of
[WhiteheadRussell] ..."
  simpr $p |- ( ( ph /\ ps ) -> ps ) $= ... $.

Statement anim12d is located on line 15445 of the file
"set.mm".
"Conjoin antecedents and consequents in a deduction.
..."
  anim12d.1 $e |- ( ph -> ( ps -> ch ) ) $.
  anim12d.2 $e |- ( ph -> ( th -> ta ) ) $.
  anim12d $p |- ( ph -> ( ( ps /\ th ) -> ( ch /\ ta ) ) )
      $= ... $.
\end{verbatim}
\begin{center}
(etc.)
\end{center}

Of course you can look at each of these statements and their proofs, and
so on, back to the axioms of propositional calculus if you wish.

The \texttt{search} command is useful for finding statements when you
know all or part of their contents.  The following example finds all
statements containing \verb@ph -> ps@ followed by \verb@ch -> th@.  The
\verb@$*@ is a wildcard that matches anything; the \texttt{\$} before the
\verb$*$ prevents conflicts with math symbol token names.  The \verb@*@ after
\texttt{SEARCH} is also a wildcard that in this case means ``match any label.''
\index{\texttt{search} command}

% I'm omitting this one, since readers are unlikely to see it:
% 1096 bisymOLD $p |- ( ( ( ph -> ps ) -> ( ch -> th ) ) -> ( (
%   ( ps -> ph ) -> ( th -> ch ) ) -> ( ( ph <-> ps ) -> ( ch
%    <-> th ) ) ) )
\begin{verbatim}
MM> search * "ph -> ps $* ch -> th"
1791 prth $p |- ( ( ( ph -> ps ) /\ ( ch -> th ) ) -> ( ( ph
    /\ ch ) -> ( ps /\ th ) ) )
2455 pm3.48 $p |- ( ( ( ph -> ps ) /\ ( ch -> th ) ) -> ( (
    ph \/ ch ) -> ( ps \/ th ) ) )
117859 pm11.71 $p |- ( ( E. x ph /\ E. y ch ) -> ( ( A. x (
    ph -> ps ) /\ A. y ( ch -> th ) ) <-> A. x A. y ( ( ph /\
    ch ) -> ( ps /\ th ) ) ) )
\end{verbatim}

Three statements, \texttt{prth}, \texttt{pm3.48},
 and \texttt{pm11.71}, were found to match.

To see what axioms\index{axiom} and definitions\index{definition}
\texttt{prth} ultimately depends on for its proof, you can have the
program backtrack through the hierarchy\index{hierarchy} of theorems and
definitions.\index{\texttt{show trace{\char`\_}back} command}

\begin{verbatim}
MM> show trace_back prth /essential/axioms
Statement "prth" assumes the following axioms ($a
statements):
  ax-1 ax-2 ax-3 ax-mp df-bi df-an
\end{verbatim}

Note that the 3 axioms of propositional calculus and the modus ponens rule are
needed (as expected); in addition, there are a couple of definitions that are used
along the way.  Note that Metamath makes no distinction\index{axiom vs.\
definition} between axioms\index{axiom} and definitions\index{definition}.  In
\texttt{set.mm} they have been distinguished artificially by prefixing their
labels\index{labels in \texttt{set.mm}} with \texttt{ax-} and \texttt{df-}
respectively.  For example, \texttt{df-an} defines conjunction (logical {\sc
and}), which is represented by the symbol \verb+/\+.
Section~\ref{definitions} discusses the philosophy of definitions, and the
Metamath language takes a particularly simple, conservative approach by using
the \texttt{\$a}\index{\texttt{\$a} statement} statement for both axioms and
definitions.

You can also have the program compute how many steps a proof
has\index{proof length} if we were to follow it all the way back to
\texttt{\$a} statements.

\begin{verbatim}
MM> show trace_back prth /essential/count_steps
The statement's actual proof has 3 steps.  Backtracking, a
total of 79 different subtheorems are used.  The statement
and subtheorems have a total of 274 actual steps.  If
subtheorems used only once were eliminated, there would be a
total of 38 subtheorems, and the statement and subtheorems
would have a total of 185 steps.  The proof would have 28349
steps if fully expanded back to axiom references.  The
maximum path length is 38.  A longest path is:  prth <-
anim12d <- syl2and <- sylan2d <- ancomsd <- ancom <- pm3.22
<- pm3.21 <- pm3.2 <- ex <- sylbir <- biimpri <- bicomi <-
bicom1 <- bi2 <- dfbi1 <- impbii <- bi3 <- simprim <- impi <-
con1i <- nsyl2 <- mt3d <- con1d <- notnot1 <- con2i <- nsyl3
<- mt2d <- con2d <- notnot2 <- pm2.18d <- pm2.18 <- pm2.21 <-
pm2.21d <- a1d <- syl <- mpd <- a2i <- a2i.1 .
\end{verbatim}

This tells us that we would have to inspect 274 steps if we want to
verify the proof completely starting from the axioms.  A few more
statistics are also shown.  There are one or more paths back to axioms
that are the longest; this command ferrets out one of them and shows it
to you.  There may be a sense in which the longest path length is
related to how ``deep'' the theorem is.

We might also be curious about what proofs depend on the theorem
\texttt{prth}.  If it is never used later on, we could eliminate it as
redundant if it has no intrinsic interest by itself.\index{\texttt{show
usage} command}

% I decided to show the OLD values here.
\begin{verbatim}
MM> show usage prth
Statement "prth" is directly referenced in the proofs of 18
statements:
  mo3 moOLD 2mo 2moOLD euind reuind reuss2 reusv3i opelopabt
  wemaplem2 rexanre rlimcn2 o1of2 o1rlimmul 2sqlem6 spanuni
  heicant pm11.71
\end{verbatim}

Thus \texttt{prth} is directly used by 18 proofs.
We can use the \texttt{/recursive} qualifier to include indirect use:

\begin{verbatim}
MM> show usage prth /recursive
Statement "prth" directly or indirectly affects the proofs of
24214 statements:
  mo3 mo mo3OLD eu2 moOLD eu2OLD eu3OLD mo4f mo4 eu4 mopick
...
\end{verbatim}

\subsection{A Note on the ``Compact'' Proof Format}

The present version of Metamath\index{Metamath!limitations of
version 0.07.30} will display proofs in a ``compact''\index{compact proof}
format whenever the proof is stored in compressed format in the database.  It
may be be slightly confusing unless you know how to interpret it.  (A
future version may eliminate this format from displays.)  For example,
if you display the complete proof of theorem \texttt{id1} it will start
off as follows:

\begin{verbatim}
MM> show proof id1 /lemmon/all
 1 wph           $f wff ph
 2 wph           $f wff ph
 3 wph           $f wff ph
 4 2,3 wi    @4: $a wff ( ph -> ph )
 5 1,4 wi    @5: $a wff ( ph -> ( ph -> ph ) )
 6 @4            $a wff ( ph -> ph )
\end{verbatim}

\begin{center}
{etc.}
\end{center}

Step 4 has a ``local label,''\index{local label} \texttt{@4}, assigned to it.
Later on, at step 6, the label \texttt{@1} is referenced instead of
displaying the explicit proof for that step.  This technique takes advantage
of the fact that steps in a proof often repeat, especially during the
construction of wffs.  The compact format reduces the number of steps in the
proof display and may actually be preferred by some people, which is one
reason that it is still displayed by the program.

If you want to see the normal format with the ``true'' step numbers, you can
use the following workaround:\index{\texttt{save proof} command}

\begin{verbatim}
MM> save proof id1 /normal
The proof of "id1" has been reformatted and saved internally.
Remember to use WRITE SOURCE to save it permanently.
MM> show proof id1 /lemmon/all
 1 wph           $f wff ph
 2 wph           $f wff ph
 3 wph           $f wff ph
 4 2,3 wi        $a wff ( ph -> ph )
 5 1,4 wi        $a wff ( ph -> ( ph -> ph ) )
 6 wph           $f wff ph
 7 wph           $f wff ph
 8 6,7 wi        $a wff ( ph -> ph )
\end{verbatim}

\begin{center}
{etc.}
\end{center}

Note that the original 6 steps are now 8 steps.  However, the format is
now the same as that described in Chapter~\ref{using}.

\chapter{The Metamath Language}
\label{languagespec}

\begin{quote}
  {\em Thus mathematics may be defined as the subject in which we never know
what we are talking about, nor whether what we are saying is true.}
    \flushright\sc  Bertrand Russell\footnote{\cite[p.~84]{Russell2}.}\\
\end{quote}\index{Russell, Bertrand}

Probably the most striking feature of the Metamath language is its almost
complete absence of hard-wired syntax. Metamath\index{Metamath} does not
understand any mathematics or logic other than that needed to construct finite
sequences of symbols according to a small set of simple, built-in rules.  The
only rule it uses in a proof is the substitution of an expression (symbol
sequence) for a variable, subject to a simple constraint to prevent
bound-variable clashes.  The primitive notions built into Metamath involve the
simple manipulation of finite objects (symbols) that we as humans can easily
visualize and that computers can easily deal with.  They seem to be just
about the simplest notions possible that are required to do standard
mathematics.

This chapter serves as a reference manual for the Metamath\index{Metamath}
language. It covers the tedious technical details of the language, some of
which you may wish to skim in a first reading.  On the other hand, you should
pay close attention to the defined terms in {\bf boldface}; they have precise
meanings that are important to keep in mind for later understanding.  It may
be best to first become familiar with the examples in Chapter~\ref{using} to
gain some motivation for the language.

%% Uncomment this when uncommenting section {formalspec} below
If you have some knowledge of set theory, you may wish to study this
chapter in conjunction with the formal set-theoretical description of the
Metamath language in Appendix~\ref{formalspec}.

We will use the name ``Metamath''\index{Metamath} to mean either the Metamath
computer language or the Metamath software associated with the computer
language.  We will not distinguish these two when the context is clear.

The next section contains the complete specification of the Metamath
language.
It serves as an
authoritative reference and presents the syntax in enough detail to
write a parser\index{parsing Metamath} and proof verifier.  The
specification is terse and it is probably hard to learn the language
directly from it, but we include it here for those impatient people who
prefer to see everything up front before looking at verbose expository
material.  Later sections explain this material and provide examples.
We will repeat the definitions in those sections, and you may skip the
next section at first reading and proceed to Section~\ref{tut1}
(p.~\pageref{tut1}).

\section{Specification of the Metamath Language}\label{spec}
\index{Metamath!specification}

\begin{quote}
  {\em Sometimes one has to say difficult things, but one ought to say
them as simply as one knows how.}
    \flushright\sc  G. H. Hardy\footnote{As quoted in
    \cite{deMillo}, p.~273.}\\
\end{quote}\index{Hardy, G. H.}

\subsection{Preliminaries}\label{spec1}

% Space is technically a printable character, so we'll word things
% carefully so it's unambiguous.
A Metamath {\bf database}\index{database} is built up from a top-level source
file together with any source files that are brought in through file inclusion
commands (see below).  The only characters that are allowed to appear in a
Metamath source file are the 94 non-whitespace printable {\sc
ascii}\index{ascii@{\sc ascii}} characters, which are digits, upper and lower
case letters, and the following 32 special
characters\index{special characters}:\label{spec1chars}

\begin{verbatim}
! " # $ % & ' ( ) * + , - . / :
; < = > ? @ [ \ ] ^ _ ` { | } ~
\end{verbatim}
plus the following characters which are the ``white space'' characters:
space (a printable character),
tab, carriage return, line feed, and form feed.\label{whitespace}
We will use \texttt{typewriter}
font to display the printable characters.

A Metamath database consists of a sequence of three kinds of {\bf
tokens}\index{token} separated by {\bf white space}\index{white space}
(which is any sequence of one or more white space characters).  The set
of {\bf keyword}\index{keyword} tokens is \texttt{\$\char`\{},
\texttt{\$\char`\}}, \texttt{\$c}, \texttt{\$v}, \texttt{\$f},
\texttt{\$e}, \texttt{\$d}, \texttt{\$a}, \texttt{\$p}, \texttt{\$.},
\texttt{\$=}, \texttt{\$(}, \texttt{\$)}, \texttt{\$[}, and
\texttt{\$]}.  The last four are called {\bf auxiliary}\index{auxiliary
keyword} or preprocessing keywords.  A {\bf label}\index{label} token
consists of any combination of letters, digits, and the characters
hyphen, underscore, and period.  A {\bf math symbol}\index{math symbol}
token may consist of any combination of the 92 printable standard {\sc
ascii} characters other than space or \texttt{\$}~. All tokens are
case-sensitive.

\subsection{Preprocessing}

The token \texttt{\$(} begins a {\bf comment} and
\texttt{\$)} ends a comment.\index{\texttt{\$(}
and \texttt{\$)} auxiliary keywords}\index{comment}
Comments may contain any of
the 94 non-whitespace printable characters and white space,
except they may not contain the
2-character sequences \texttt{\$(} or \texttt{\$)} (comments do not nest).
Comments are ignored (treated
like white space) for the purpose of parsing, e.g.,
\texttt{\$( \$[ \$)} is a comment.
See p.~\pageref{mathcomments} for comment typesetting conventions; these
conventions may be ignored for the purpose of parsing.

A {\bf file inclusion command} consists of \texttt{\$[} followed by a file name
followed by \texttt{\$]}.\index{\texttt{\$[} and \texttt{\$]} auxiliary
keywords}\index{included file}\index{file inclusion}
It is only allowed in the outermost scope (i.e., not between
\texttt{\$\char`\{} and \texttt{\$\char`\}})
and must not be inside a statement (e.g., it may not occur
between the label of a \texttt{\$a} statement and its \texttt{\$.}).
The file name may not
contain a \texttt{\$} or white space.  The file must exist.
The case-sensitivity
of its name follows the conventions of the operating system.  The contents of
the file replace the inclusion command.
Included files may include other files.
Only the first reference to a given file is included; any later
references to the same file (whether in the top-level file or in included
files) cause the inclusion command to be ignored (treated like white space).
A verifier may assume that file names with different strings
refer to different files for the purpose of ignoring later references.
A file self-reference is ignored, as is any reference to the top-level file
(to avoid loops).
Included files may not include a \texttt{\$(} without a matching \texttt{\$)},
may not include a \texttt{\$[} without a matching \texttt{\$]}, and may
not include incomplete statements (e.g., a \texttt{\$a} without a matching
\texttt{\$.}).
It is currently unspecified if path references are relative to the process'
current directory or the file's containing directory, so databases should
avoid using pathname separators (e.g., ``/'') in file names.

Like all tokens, the \texttt{\$(}, \texttt{\$)}, \texttt{\$[}, and \texttt{\$]} keywords
must be surrounded by white space.

\subsection{Basic Syntax}

After preprocessing, a database will consist of a sequence of {\bf
statements}.
These are the scoping statements \texttt{\$\char`\{} and
\texttt{\$\char`\}}, along with the \texttt{\$c}, \texttt{\$v},
\texttt{\$f}, \texttt{\$e}, \texttt{\$d}, \texttt{\$a}, and \texttt{\$p}
statements.

A {\bf scoping statement}\index{scoping statement} consists only of its
keyword, \texttt{\$\char`\{} or \texttt{\$\char`\}}.
A \texttt{\$\char`\{} begins a {\bf
block}\index{block} and a matching \texttt{\$\char`\}} ends the block.
Every \texttt{\$\char`\{}
must have a matching \texttt{\$\char`\}}.
Defining it recursively, we say a block
contains a sequence of zero or more tokens other
than \texttt{\$\char`\{} and \texttt{\$\char`\}} and
possibly other blocks.  There is an {\bf outermost
block}\index{block!outermost} not bracketed by \texttt{\$\char`\{} \texttt{\$\char`\}}; the end
of the outermost block is the end of the database.

% LaTeX bug? can't do \bf\tt

A {\bf \$v} or {\bf \$c statement}\index{\texttt{\$v} statement}\index{\texttt{\$c}
statement} consists of the keyword token \texttt{\$v} or \texttt{\$c} respectively,
followed by one or more math symbols,
% The word "token" is used to distinguish "$." from the sentence-ending period.
followed by the \texttt{\$.}\ token.
These
statements {\bf declare}\index{declaration} the math symbols to be {\bf
variables}\index{variable!Metamath} or {\bf constants}\index{constant}
respectively. The same math symbol may not occur twice in a given \texttt{\$v} or
\texttt{\$c} statement.

%c%A math symbol becomes an {\bf active}\index{active math symbol}
%c%when declared and stays active until the end of the block in which it is
%c%declared.  A math symbol may not be declared a second time while it is active,
%c%but it may be declared again after it becomes inactive.

A math symbol becomes {\bf active}\index{active math symbol} when declared
and stays active until the end of the block in which it is declared.  A
variable may not be declared a second time while it is active, but it
may be declared again (as a variable, but not as a constant) after it
becomes inactive.  A constant must be declared in the outermost block and may
not be declared a second time.\footnote{The rules for redeclaration may
become more general in the future;
see p.~\pageref{redeclarationf}.}\index{redeclaration of symbols}

A {\bf \$f statement}\index{\texttt{\$f} statement} consists of a label,
followed by \texttt{\$f}, followed by its typecode (an active constant),
followed by an
active variable, followed by the \texttt{\$.}\ token.  A {\bf \$e
statement}\index{\texttt{\$e} statement} consists of a label, followed
by \texttt{\$e}, followed by its typecode (an active constant),
followed by zero or more
active math symbols, followed by the \texttt{\$.}\ token.  A {\bf
hypothesis}\index{hypothesis} is a \texttt{\$f} or \texttt{\$e}
statement.
The type declared by a \texttt{\$f} statement for a given label
is global even if the variable is not
(e.g., a database may not have \texttt{wff P} in one local scope
nd \texttt{class P} in another).

A {\bf simple \$d statement}\index{\texttt{\$d} statement!simple}
consists of \texttt{\$d}, followed by two different active variables,
followed by the \texttt{\$.}\ token.  A {\bf compound \$d
statement}\index{\texttt{\$d} statement!compound} consists of
\texttt{\$d}, followed by three or more variables (all different),
followed by the \texttt{\$.}\ token.  The order of the variables in a
\texttt{\$d} statement is unimportant.  A compound \texttt{\$d}
statement is equivalent to a set of simple \texttt{\$d} statements, one
for each possible pair of variables occurring in the compound
\texttt{\$d} statement.  Henceforth in this specification we shall
assume all \texttt{\$d} statements are simple.  A \texttt{\$d} statement
is also called a {\bf disjoint} (or {\bf distinct}) {\bf variable
restriction}.\index{disjoint-variable restriction}

A {\bf \$a statement}\index{\texttt{\$a} statement} consists of a label,
followed by \texttt{\$a}, followed by its typecode (an active constant),
followed by
zero or more active math symbols, followed by the \texttt{\$.}\ token.  A {\bf
\$p statement}\index{\texttt{\$p} statement} consists of a label,
followed by \texttt{\$p}, followed by its typecode (an active constant),
followed by
zero or more active math symbols, followed by \texttt{\$=}, followed by
a sequence of labels, followed by the \texttt{\$.}\ token.  An {\bf
assertion}\index{assertion} is a \texttt{\$a} or \texttt{\$p} statement.

A \texttt{\$f}, \texttt{\$e}, or \texttt{\$d} statement is {\bf active}\index{active
statement} from the place it occurs until the end of the block it occurs in.
A \texttt{\$a} or \texttt{\$p} statement is {\bf active} from the place it occurs
through the end of the database.
There may not be two active \texttt{\$f} statements containing the same
variable.  Each variable in a \texttt{\$e}, \texttt{\$a}, or
\texttt{\$p} statement must exist in an active \texttt{\$f}
statement.\footnote{This requirement can greatly simplify the
unification algorithm (substitution calculation) required by proof
verification.}

%The label that begins each \texttt{\$f}, \texttt{\$e}, \texttt{\$a}, and
%\texttt{\$p} statement must be unique.
Each label token must be unique, and
no label token may match any math symbol
token.\label{namespace}\footnote{This
restriction was added on June 24, 2006.
It is not theoretically necessary but is imposed to make it easier to
write certain parsers.}

The set of {\bf mandatory variables}\index{mandatory variable} associated with
an assertion is the set of (zero or more) variables in the assertion and in any
active \texttt{\$e} statements.  The (possibly empty) set of {\bf mandatory
hypotheses}\index{mandatory hypothesis} is the set of all active \texttt{\$f}
statements containing mandatory variables, together with all active \texttt{\$e}
statements.
The set of {\bf mandatory {\bf \$d} statements}\index{mandatory
disjoint-variable restriction} associated with an assertion are those active
\texttt{\$d} statements whose variables are both among the assertion's
mandatory variables.

\subsection{Proof Verification}\label{spec4}

The sequence of labels between the \texttt{\$=} and \texttt{\$.}\ tokens
in a \texttt{\$p} statement is a {\bf proof}.\index{proof!Metamath} Each
label in a proof must be the label of an active statement other than the
\texttt{\$p} statement itself; thus a label must refer either to an
active hypothesis of the \texttt{\$p} statement or to an earlier
assertion.

An {\bf expression}\index{expression} is a sequence of math symbols. A {\bf
substitution map}\index{substitution map} associates a set of variables with a
set of expressions.  It is acceptable for a variable to be mapped to an
expression containing it.  A {\bf
substitution}\index{substitution!variable}\index{variable substitution} is the
simultaneous replacement of all variables in one or more expressions with the
expressions that the variables map to.

A proof is scanned in order of its label sequence.  If the label refers to an
active hypothesis, the expression in the hypothesis is pushed onto a
stack.\index{stack}\index{RPN stack}  If the label refers to an assertion, a
(unique) substitution must exist that, when made to the mandatory hypotheses
of the referenced assertion, causes them to match the topmost (i.e.\ most
recent) entries of the stack, in order of occurrence of the mandatory
hypotheses, with the topmost stack entry matching the last mandatory
hypothesis of the referenced assertion.  As many stack entries as there are
mandatory hypotheses are then popped from the stack.  The same substitution is
made to the referenced assertion, and the result is pushed onto the stack.
After the last label in the proof is processed, the stack must have a single
entry that matches the expression in the \texttt{\$p} statement containing the
proof.

%c%{\footnotesize\begin{quotation}\index{redeclaration of symbols}
%c%{{\em Comment.}\label{spec4comment} Whenever a math symbol token occurs in a
%c%{\texttt{\$c} or \texttt{\$v} statement, it is considered to designate a distinct new
%c%{symbol, even if the same token was previously declared (and is now inactive).
%c%{Thus a math token declared as a constant in two different blocks is considered
%c%{to designate two distinct constants (even though they have the same name).
%c%{The two constants will not match in a proof that references both blocks.
%c%{However, a proof referencing both blocks is acceptable as long as it doesn't
%c%{require that the constants match.  Similarly, a token declared to be a
%c%{constant for a referenced assertion will not match the same token declared to
%c%{be a variable for the \texttt{\$p} statement containing the proof.  In the case
%c%{of a token declared to be a variable for a referenced assertion, this is not
%c%{an issue since the variable can be substituted with whatever expression is
%c%{needed to achieve the required match.
%c%{\end{quotation}}
%c2%A proof may reference an assertion that contains or whose hypotheses contain a
%c2%constant that is not active for the \texttt{\$p} statement containing the proof.
%c2%However, the final result of the proof may not contain that constant. A proof
%c2%may also reference an assertion that contains or whose hypotheses contain a
%c2%variable that is not active for the \texttt{\$p} statement containing the proof.
%c2%That variable, of course, will be substituted with whatever expression is
%c2%needed to achieve the required match.

A proof may contain a \texttt{?}\ in place of a label to indicate an unknown step
(Section~\ref{unknown}).  A proof verifier may ignore any proof containing
\texttt{?}\ but should warn the user that the proof is incomplete.

A {\bf compressed proof}\index{compressed proof}\index{proof!compressed} is an
alternate proof notation described in Appen\-dix~\ref{compressed}; also see
references to ``compressed proof'' in the Index.  Compressed proofs are a
Metamath language extension which a complete proof verifier should be able to
parse and verify.

\subsubsection{Verifying Disjoint Variable Restrictions}

Each substitution made in a proof must be checked to verify that any
disjoint variable restrictions are satisfied, as follows.

If two variables replaced by a substitution exist in a mandatory \texttt{\$d}
statement\index{\texttt{\$d} statement} of the assertion referenced, the two
expressions resulting from the substitution must satisfy the following
conditions.  First, the two expressions must have no variables in common.
Second, each possible pair of variables, one from each expression, must exist
in an active \texttt{\$d} statement of the \texttt{\$p} statement containing the
proof.

\vskip 1ex

This ends the specification of the Metamath language;
see Appendix \ref{BNF} for its syntax in
Extended Backus--Naur Form (EBNF)\index{Extended Backus--Naur Form}\index{EBNF}.

\section{The Basic Keywords}\label{tut1}

Our expository material begins here.

Like most computer languages, Metamath\index{Metamath} takes its input from
one or more {\bf source files}\index{source file} which contain characters
expressed in the standard {\sc ascii} (American Standard Code for Information
Interchange)\index{ascii@{\sc ascii}} code for computers.  A source file
consists of a series of {\bf tokens}\index{token}, which are strings of
non-whitespace
printable characters (from the set of 94 shown on p.~\pageref{spec1chars})
separated by {\bf white space}\index{white space} (spaces, tabs, carriage
returns, line feeds, and form feeds). Any string consisting only of these
characters is treated the same as a single space.  The non-whitespace printable
characters\index{printable character} that Metamath recognizes are the 94
characters on standard {\sc ascii} keyboards.

Metamath has the ability to join several files together to form its
input (Section~\ref{include}).  We call the aggregate contents of all
the files after they have been joined together a {\bf
database}\index{database} to distinguish it from an individual source
file.  The tokens in a database consist of {\bf
keywords}\index{keyword}, which are built into the language, together
with two kinds of user-defined tokens called {\bf labels}\index{label}
and {\bf math symbols}\index{math symbol}.  (Often we will simply say
{\bf symbol}\index{symbol} instead of math symbol for brevity).  The set
of {\bf basic keywords}\index{basic keyword} is
\texttt{\$c}\index{\texttt{\$c} statement},
\texttt{\$v}\index{\texttt{\$v} statement},
\texttt{\$e}\index{\texttt{\$e} statement},
\texttt{\$f}\index{\texttt{\$f} statement},
\texttt{\$d}\index{\texttt{\$d} statement},
\texttt{\$a}\index{\texttt{\$a} statement},
\texttt{\$p}\index{\texttt{\$p} statement},
\texttt{\$=}\index{\texttt{\$=} keyword},
\texttt{\$.}\index{\texttt{\$.}\ keyword},
\texttt{\$\char`\{}\index{\texttt{\$\char`\{} and \texttt{\$\char`\}}
keywords}, and \texttt{\$\char`\}}.  This is the complete set of
syntactical elements of what we call the {\bf basic
language}\index{basic language} of Metamath, and with them you can
express all of the mathematics that were intended by the design of
Metamath.  You should make it a point to become very familiar with them.
Table~\ref{basickeywords} lists the basic keywords along with a brief
description of their functions.  For now, this description will give you
only a vague notion of what the keywords are for; later we will describe
the keywords in detail.


\begin{table}[htp] \caption{Summary of the basic Metamath
keywords} \label{basickeywords}
\begin{center}
\begin{tabular}{|p{4pc}|l|}
\hline
\em \centering Keyword&\em Description\\
\hline
\hline
\centering
   \texttt{\$c}&Constant symbol declaration\\
\hline
\centering
   \texttt{\$v}&Variable symbol declaration\\
\hline
\centering
   \texttt{\$d}&Disjoint variable restriction\\
\hline
\centering
   \texttt{\$f}&Variable-type (``floating'') hypothesis\\
\hline
\centering
   \texttt{\$e}&Logical (``essential'') hypothesis\\
\hline
\centering
   \texttt{\$a}&Axiomatic assertion\\
\hline
\centering
   \texttt{\$p}&Provable assertion\\
\hline
\centering
   \texttt{\$=}&Start of proof in \texttt{\$p} statement\\
\hline
\centering
   \texttt{\$.}&End of the above statement types\\
\hline
\centering
   \texttt{\$\char`\{}&Start of block\\
\hline
\centering
   \texttt{\$\char`\}}&End of block\\
\hline
\end{tabular}
\end{center}
\end{table}

%For LaTeX bug(?) where it puts tables on blank page instead of btwn text
%May have to adjust if text changes
%\newpage

There are some additional keywords, called {\bf auxiliary
keywords}\index{auxiliary keyword} that help make Metamath\index{Metamath}
more practical. These are part of the {\bf extended language}\index{extended
language}. They provide you with a means to put comments into a Metamath
source file\index{source file} and reference other source files.  We will
introduce these in later sections. Table~\ref{otherkeywords} summarizes them
so that you can recognize them now if you want to peruse some source
files while learning the basic keywords.


\begin{table}[htp] \caption{Auxiliary Metamath
keywords} \label{otherkeywords}
\begin{center}
\begin{tabular}{|p{4pc}|l|}
\hline
\em \centering Keyword&\em Description\\
\hline
\hline
\centering
   \texttt{\$(}&Start of comment\\
\hline
\centering
   \texttt{\$)}&End of comment\\
\hline
\centering
   \texttt{\$[}&Start of included source file name\\
\hline
\centering
   \texttt{\$]}&End of included source file name\\
\hline
\end{tabular}
\end{center}
\end{table}
\index{\texttt{\$(} and \texttt{\$)} auxiliary keywords}
\index{\texttt{\$[} and \texttt{\$]} auxiliary keywords}


Unlike those in some computer languages, the keywords\index{keyword} are short
two-character sequences rather than English-like words.  While this may make
them slightly more difficult to remember at first, their brevity allows
them to blend in with the mathematics being described, not
distract from it, like punctuation marks.


\subsection{User-Defined Tokens}\label{dollardollar}\index{token}

As you may have noticed, all keywords\index{keyword} begin with the \texttt{\$}
character.  This mundane monetary symbol is not ordinarily used in higher
mathematics (outside of grant proposals), so we have appropriated it to
distinguish the Metamath\index{Metamath} keywords from ordinary mathematical
symbols. The \texttt{\$} character is thus considered special and may not be
used as a character in a user-defined token.  All tokens and keywords are
case-sensitive; for example, \texttt{n} is considered to be a different character
from \texttt{N}.  Case-sensitivity makes the available {\sc ascii} character set
as rich as possible.

\subsubsection{Math Symbol Tokens}\index{token}

Math symbols\index{math symbol} are tokens used to represent the symbols
that appear in ordinary mathematical formulas.  They may consist of any
combination of the 93 non-whitespace printable {\sc ascii} characters other than
\texttt{\$}~. Some examples are \texttt{x}, \texttt{+}, \texttt{(},
\texttt{|-}, \verb$!%@?&$, and \texttt{bounded}.  For readability, it is
best to try to make these look as similar to actual mathematical symbols
as possible, within the constraints of the {\sc ascii} character set, in
order to make the resulting mathematical expressions more readable.

In the Metamath\index{Metamath} language, you express ordinary
mathematical formulas and statements as sequences of math symbols such
as \texttt{2 + 2 = 4} (five symbols, all constants).\footnote{To
eliminate ambiguity with other expressions, this is expressed in the set
theory database \texttt{set.mm} as \texttt{|- ( 2 + 2
 ) = 4 }, whose \LaTeX\ equivalent is $\vdash
(2+2)=4$.  The \,$\vdash$ means ``is a theorem'' and the
parentheses allow explicit associative grouping.}\index{turnstile
({$\,\vdash$})} They may even be English
sentences, as in \texttt{E is closed and bounded} (five symbols)---here
\texttt{E} would be a variable and the other four symbols constants.  In
principle, a Metamath database could be constructed to work with almost
any unambiguous English-language mathematical statement, but as a
practical matter the definitions needed to provide for all possible
syntax variations would be cumbersome and distracting and possibly have
subtle pitfalls accidentally built in.  We generally recommend that you
express mathematical statements with compact standard mathematical
symbols whenever possible and put their English-language descriptions in
comments.  Axioms\index{axiom} and definitions\index{definition}
(\texttt{\$a}\index{\texttt{\$a} statement} statements) are the only
places where Metamath will not detect an error, and doing this will help
reduce the number of definitions needed.

You are free to use any tokens\index{token} you like for math
symbols\index{math symbol}.  Appendix~\ref{ASCII} recommends token names to
use for symbols in set theory, and we suggest you adopt these in order to be
able to include the \texttt{set.mm} set theory database in your database.  For
printouts, you can convert the tokens in a database
to standard mathematical symbols with the \LaTeX\ typesetting program.  The
Metamath command \texttt{open tex} {\em filename}\index{\texttt{open tex} command}
produces output that can be read by \LaTeX.\index{latex@{\LaTeX}}
The correspondence
between tokens and the actual symbols is made by \texttt{latexdef}
statements inside a special database comment tagged
with \texttt{\$t}.\index{\texttt{\$t} comment}\index{typesetting comment}
  You can edit
this comment to change the definitions or add new ones.
Appendix~\ref{ASCII} describes how to do this in more detail.

% White space\index{white space} is normally used to separate math
% symbol\index{math symbol} tokens, but they may be juxtaposed without white
% space in \texttt{\$d}\index{\texttt{\$d} statement}, \texttt{\$e}\index{\texttt{\$e}
% statement}, \texttt{\$f}\index{\texttt{\$f} statement}, \texttt{\$a}\index{\texttt{\$a}
% statement}, and \texttt{\$p}\index{\texttt{\$p} statement} statements when no
% ambiguity will result.  Specifically, Metamath parses the math symbol sequence
% in one of these statements in the following manner:  when the math symbol
% sequence has been broken up into tokens\index{token} up to a given character,
% the next token is the longest string of characters that could constitute a
% math symbol that is active\index{active
% math symbol} at that point.  (See Section~\ref{scoping} for the
% definition of an active math symbol.)  For example, if \texttt{-}, \texttt{>}, and
% \texttt{->} are the only active math symbols, the juxtaposition \texttt{>-} will be
% interpreted as the two symbols \texttt{>} and \texttt{-}, whereas \texttt{->} will
% always be interpreted as that single symbol.\footnote{For better readability we
% recommend a white space between each token.  This also makes searching for a
% symbol easier to do with an editor.  Omission of optional white space is useful
% for reducing typing when assigning an expression to a temporary
% variable\index{temporary variable} with the \texttt{let variable} Metamath
% program command.}\index{\texttt{let variable} command}
%
% Keywords\index{keyword} may be placed next to math symbols without white
% space\index{white space} between them.\footnote{Again, we do not recommend
% this for readability.}
%
% The math symbols\index{math symbol} in \texttt{\$c}\index{\texttt{\$c} statement}
% and \texttt{\$v}\index{\texttt{\$v} statement} statements must always be separated
% by white space\index{white
% space}, for the obvious reason that these statements define the names
% of the symbols.
%
% Math symbols referred to in comments (see Section~\ref{comments}) must also be
% separated by white space.  This allows you to make comments about symbols that
% are not yet active\index{active
% math symbol}.  (The ``math mode'' feature of comments is also a quick and
% easy way to obtain word processing text with embedded mathematical symbols,
% independently of the main purpose of Metamath; the way to do this is described
% in Section~\ref{comments})

\subsubsection{Label Tokens}\index{token}\index{label}

Label tokens are used to identify Metamath\index{Metamath} statements for
later reference. Label tokens may contain only letters, digits, and the three
characters period, hyphen, and underscore:
\begin{verbatim}
. - _
\end{verbatim}

A label is {\bf declared}\index{label declaration} by placing it immediately
before the keyword of the statement it identifies.  For example, the label
\texttt{axiom.1} might be declared as follows:
\begin{verbatim}
axiom.1 $a |- x = x $.
\end{verbatim}

Each \texttt{\$e}\index{\texttt{\$e} statement},
\texttt{\$f}\index{\texttt{\$f} statement},
\texttt{\$a}\index{\texttt{\$a} statement}, and
\texttt{\$p}\index{\texttt{\$p} statement} statement in a database must
have a label declared for it.  No other statement types may have label
declarations.  Every label must be unique.

A label (and the statement it identifies) is {\bf referenced}\index{label
reference} by including the label between the \texttt{\$=}\index{\texttt{\$=}
keyword} and \texttt{\$.}\index{\texttt{\$.}\ keyword}\ keywords in a \texttt{\$p}
statement.  The sequence of labels\index{label sequence} between these two
keywords is called a {\bf proof}\index{proof}.  An example of a statement with
a proof that we will encounter later (Section~\ref{proof}) is
\begin{verbatim}
wnew $p wff ( s -> ( r -> p ) )
     $= ws wr wp w2 w2 $.
\end{verbatim}

You don't have to know what this means just yet, but you should know that the
label \texttt{wnew} is declared by this \texttt{\$p} statement and that the labels
\texttt{ws}, \texttt{wr}, \texttt{wp}, and \texttt{w2} are assumed to have been declared
earlier in the database and are referenced here.

\subsection{Constants and Variables}
\index{constant}
\index{variable}

An {\bf expression}\index{expression} is any sequence of math
symbols, possibly empty.

The basic Metamath\index{Metamath} language\index{basic language} has two
kinds of math symbols\index{math symbol}:  {\bf constants}\index{constant} and
{\bf variables}\index{variable}.  In a Metamath proof, a constant may not be
substituted with any expression.  A variable can be
substituted\index{substitution!variable}\index{variable substitution} with any
expression.  This sequence may include other variables and may even include
the variable being substituted.  This substitution takes place when proofs are
verified, and it will be described in Section~\ref{proof}.  The \texttt{\$f}
statement (described later in Section~\ref{dollaref}) is used to specify the
{\bf type} of a variable (i.e.\ what kind of
variable it is)\index{variable type}\index{type} and
give it a meaning typically
associated with a ``metavariable''\index{metavariable}\footnote{A metavariable
is a variable that ranges over the syntactical elements of the object language
being discussed; for example, one metavariable might represent a variable of
the object language and another metavariable might represent a formula in the
object language.} in ordinary mathematics; for example, a variable may be
specified to be a wff or well-formed formula (in logic), a set (in set
theory), or a non-negative integer (in number theory).

%\subsection{The \texttt{\$c} and \texttt{\$v} Declaration Statements}
\subsection{The \texttt{\$c} and \texttt{\$v} Declaration Statements}
\index{\texttt{\$c} statement}
\index{constant declaration}
\index{\texttt{\$v} statement}
\index{variable declaration}

Constants are introduced or {\bf declared}\index{constant declaration}
with \texttt{\$c}\index{\texttt{\$c} statement} statements, and
variables are declared\index{variable declaration} with
\texttt{\$v}\index{\texttt{\$v} statement} statements.  A {\bf simple}
declaration\index{simple declaration} statement introduces a single
constant or variable.  Its syntax is one of the following:
\begin{center}
  \texttt{\$c} {\em math-symbol} \texttt{\$.}\\
  \texttt{\$v} {\em math-symbol} \texttt{\$.}
\end{center}
The notation {\em math-symbol} means any math symbol token\index{token}.

Some examples of simple declaration statements are:
\begin{center}
  \texttt{\$c + \$.}\\
  \texttt{\$c -> \$.}\\
  \texttt{\$c ( \$.}\\
  \texttt{\$v x \$.}\\
  \texttt{\$v y2 \$.}
\end{center}

The characters in a math symbol\index{math symbol} being declared are
irrelevant to Meta\-math; for example, we could declare a right parenthesis to
be a variable,
\begin{center}
  \texttt{\$v ) \$.}\\
\end{center}
although this would be unconventional.

A {\bf compound} declaration\index{compound declaration} statement is a
shorthand for declaring several symbols at once.  Its syntax is one of the
following:
\begin{center}
  \texttt{\$c} {\em math-symbol}\ \,$\cdots$\ {\em math-symbol} \texttt{\$.}\\
  \texttt{\$v} {\em math-symbol}\ \,$\cdots$\ {\em math-symbol} \texttt{\$.}
\end{center}\index{\texttt{\$c} statement}
Here, the ellipsis (\ldots) means any number of {\em math-symbol}\,s.

An example of a compound declaration statement is:
\begin{center}
  \texttt{\$v x y mu \$.}\\
\end{center}
This is equivalent to the three simple declaration statements
\begin{center}
  \texttt{\$v x \$.}\\
  \texttt{\$v y \$.}\\
  \texttt{\$v mu \$.}\\
\end{center}
\index{\texttt{\$v} statement}

There are certain rules on where in the database math symbols may be declared,
what sections of the database are aware of them (i.e.\ where they are
``active''), and when they may be declared more than once.  These will be
discussed in Section~\ref{scoping} and specifically on
p.~\pageref{redeclaration}.

\subsection{The \texttt{\$d} Statement}\label{dollard}
\index{\texttt{\$d} statement}

The \texttt{\$d} statement is called a {\bf disjoint-variable restriction}.  The
syntax of the {\bf simple} version of this statement is
\begin{center}
  \texttt{\$d} {\em variable variable} \texttt{\$.}
\end{center}
where each {\em variable} is a previously declared variable and the two {\em
variable}\,s are different.  (More specifically, each  {\em variable} must be
an {\bf active} variable\index{active math symbol}, which means there must be
a previous \texttt{\$v} statement whose {\bf scope}\index{scope} includes the
\texttt{\$d} statement.  These terms will be defined when we discuss scoping
statements in Section~\ref{scoping}.)

In ordinary mathematics, formulas may arise that are true if the variables in
them are distinct\index{distinct variables}, but become false when those
variables are made identical. For example, the formula in logic $\exists x\,x
\neq y$, which means ``for a given $y$, there exists an $x$ that is not equal
to $y$,'' is true in most mathematical theories (namely all non-trivial
theories\index{non-trivial theory}, i.e.\ those that describe more than one
individual, such as arithmetic).  However, if we substitute $y$ with $x$, we
obtain $\exists x\,x \neq x$, which is always false, as it means ``there
exists something that is not equal to itself.''\footnote{If you are a
logician, you will recognize this as the improper substitution\index{proper
substitution}\index{substitution!proper} of a free variable\index{free
variable} with a bound variable\index{bound variable}.  Metamath makes no
inherent distinction between free and bound variables; instead, you let
Metamath know what substitutions are permissible by using \texttt{\$d} statements
in the right way in your axiom system.}\index{free vs.\ bound variable}  The
\texttt{\$d} statement allows you to specify a restriction that forbids the
substitution of one variable with another.  In
this case, we would use the statement
\begin{center}
  \texttt{\$d x y \$.}
\end{center}\index{\texttt{\$d} statement}
to specify this restriction.

The order in which the variables appear in a \texttt{\$d} statement is not
important.  We could also use
\begin{center}
  \texttt{\$d y x \$.}
\end{center}

The \texttt{\$d} statement is actually more general than this, as the
``disjoint''\index{disjoint variables} in its name suggests.  The full meaning
is that if any substitution is made to its two variables (during the
course of a proof that references a \texttt{\$a} or \texttt{\$p} statement
associated with the \texttt{\$d}), the two expressions that result from the
substitution must have no variables in common.  In addition, each possible
pair of variables, one from each expression, must be in a \texttt{\$d} statement
associated with the statement being proved.  (This requirement forces the
statement being proved to ``inherit'' the original disjoint variable
restriction.)

For example, suppose \texttt{u} is a variable.  If the restriction
\begin{center}
  \texttt{\$d A B \$.}
\end{center}
has been specified for a theorem referenced in a
proof, we may not substitute \texttt{A} with \mbox{\tt a + u} and
\texttt{B} with \mbox{\tt b + u} because these two symbol sequences have the
variable \texttt{u} in common.  Furthermore, if \texttt{a} and \texttt{b} are
variables, we may not substitute \texttt{A} with \texttt{a} and \texttt{B} with \texttt{b}
unless we have also specified \texttt{\$d a b} for the theorem being proved; in
other words, the \texttt{\$d} property associated with a pair of variables must
be effectively preserved after substitution.

The \texttt{\$d}\index{\texttt{\$d} statement} statement does {\em not} mean ``the
two variables may not be substituted with the same thing,'' as you might think
at first.  For example, substituting each of \texttt{A} and \texttt{B} in the above
example with identical symbol sequences consisting only of constants does not
cause a disjoint variable conflict, because two symbol sequences have no
variables in common (since they have no variables, period).  Similarly, a
conflict will not occur by substituting the two variables in a \texttt{\$d}
statement with the empty symbol sequence\index{empty substitution}.

The \texttt{\$d} statement does not have a direct counterpart in
ordinary mathematics, partly because the variables\index{variable} of
Metamath are not really the same as the variables\index{variable!in
ordinary mathematics} of ordinary mathematics but rather are
metavariables\index{metavariable} ranging over them (as well as over
other kinds of symbols and groups of symbols).  Depending on the
situation, we may informally interpret the \texttt{\$d} statement in
different ways.  Suppose, for example, that \texttt{x} and \texttt{y}
are variables ranging over numbers (more precisely, that \texttt{x} and
\texttt{y} are metavariables ranging over variables that range over
numbers), and that \texttt{ph} ($\varphi$) and \texttt{ps} ($\psi$) are
variables (more precisely, metavariables) ranging over formulas.  We can
make the following interpretations that correspond to the informal
language of ordinary mathematics:
\begin{quote}
\begin{tabbing}
\texttt{\$d x y \$.} means ``assume $x$ and $y$ are
distinct variables.''\\
\texttt{\$d x ph \$.} means ``assume $x$ does not
occur in $\varphi$.''\\
\texttt{\$d ph ps \$.} \=means ``assume $\varphi$ and
$\psi$ have no variables\\ \>in common.''
\end{tabbing}
\end{quote}\index{\texttt{\$d} statement}

\subsubsection{Compound \texttt{\$d} Statements}

The {\bf compound} version of the \texttt{\$d} statement is a shorthand for
specifying several variables whose substitutions must be pairwise disjoint.
Its syntax is:
\begin{center}
  \texttt{\$d} {\em variable}\ \,$\cdots$\ {\em variable} \texttt{\$.}
\end{center}\index{\texttt{\$d} statement}
Here, {\em variable} represents the token of a previously declared
variable (specifically, an active variable) and all {\em variable}\,s are
different.  The compound \texttt{\$d}
statement is internally broken up by Metamath into one simple \texttt{\$d}
statement for each possible pair of variables in the original \texttt{\$d}
statement.  For example,
\begin{center}
  \texttt{\$d w x y z \$.}
\end{center}
is equivalent to
\begin{center}
  \texttt{\$d w x \$.}\\
  \texttt{\$d w y \$.}\\
  \texttt{\$d w z \$.}\\
  \texttt{\$d x y \$.}\\
  \texttt{\$d x z \$.}\\
  \texttt{\$d y z \$.}
\end{center}

Two or more simple \texttt{\$d} statements specifying the same variable pair are
internally combined into a single \texttt{\$d} statement.  Thus the set of three
statements
\begin{center}
  \texttt{\$d x y \$.}
  \texttt{\$d x y \$.}
  \texttt{\$d y x \$.}
\end{center}
is equivalent to
\begin{center}
  \texttt{\$d x y \$.}
\end{center}

Similarly, compound \texttt{\$d} statements, after being internally broken up,
internally have their common variable pairs combined.  For example the
set of statements
\begin{center}
  \texttt{\$d x y A \$.}
  \texttt{\$d x y B \$.}
\end{center}
is equivalent to
\begin{center}
  \texttt{\$d x y \$.}
  \texttt{\$d x A \$.}
  \texttt{\$d y A \$.}
  \texttt{\$d x y \$.}
  \texttt{\$d x B \$.}
  \texttt{\$d y B \$.}
\end{center}
which is equivalent to
\begin{center}
  \texttt{\$d x y \$.}
  \texttt{\$d x A \$.}
  \texttt{\$d y A \$.}
  \texttt{\$d x B \$.}
  \texttt{\$d y B \$.}
\end{center}

Metamath\index{Metamath} automatically verifies that all \texttt{\$d}
restrictions are met whenever it verifies proofs.  \texttt{\$d} statements are
never referenced directly in proofs (this is why they do not have
labels\index{label}), but Metamath is always aware of which ones must be
satisfied (i.e.\ are active) and will notify you with an error message if any
violation occurs.

To illustrate how Metamath detects a missing \texttt{\$d}
statement, we will look at the following example from the
\texttt{set.mm} database.

\begin{verbatim}
$d x z $.  $d y z $.
$( Theorem to add distinct quantifier to atomic formula. $)
ax17eq $p |- ( x = y -> A. z x = y ) $=...
\end{verbatim}

This statement has the obvious requirement that $z$ must be
distinct\index{distinct variables} from $x$ in theorem \texttt{ax17eq} that
states $x=y \rightarrow \forall z \, x=y$ (well, obvious if you're a logician,
for otherwise we could conclude  $x=y \rightarrow \forall x \, x=y$, which is
false when the free variables $x$ and $y$ are equal.).

Let's look at what happens if we edit the database to comment out this
requirement.

\begin{verbatim}
$( $d x z $. $) $d y z $.
$( Theorem to add distinct quantifier to atomic formula. $)
ax17eq $p |- ( x = y -> A. z x = y ) $=...
\end{verbatim}

When it tries to verify the proof, Metamath will tell you that \texttt{x} and
\texttt{z} must be disjoint, because one of its steps references an axiom or
theorem that has this requirement.

\begin{verbatim}
MM> verify proof ax17eq
ax17eq ?Error at statement 1918, label "ax17eq", type "$p":
      vz wal wi vx vy vz ax-13 vx vy weq vz vx ax-c16 vx vy
                                               ^^^^^
There is a disjoint variable ($d) violation at proof step 29.
Assertion "ax-c16" requires that variables "x" and "y" be
disjoint.  But "x" was substituted with "z" and "y" was
substituted with "x".  The assertion being proved, "ax17eq",
does not require that variables "z" and "x" be disjoint.
\end{verbatim}

We can see the substitutions into \texttt{ax-c16} with the following command.

\begin{verbatim}
MM> show proof ax17eq / detailed_step 29
Proof step 29:  pm2.61dd.2=ax-c16 $a |- ( A. z z = x -> ( x =
  y -> A. z x = y ) )
This step assigns source "ax-c16" ($a) to target "pm2.61dd.2"
($e).  The source assertion requires the hypotheses "wph"
($f, step 26), "vx" ($f, step 27), and "vy" ($f, step 28).
The parent assertion of the target hypothesis is "pm2.61dd"
($p, step 36).
The source assertion before substitution was:
    ax-c16 $a |- ( A. x x = y -> ( ph -> A. x ph ) )
The following substitutions were made to the source
assertion:
    Variable  Substituted with
     x         z
     y         x
     ph        x = y
The target hypothesis before substitution was:
    pm2.61dd.2 $e |- ( ph -> ch )
The following substitutions were made to the target
hypothesis:
    Variable  Substituted with
     ph        A. z z = x
     ch        ( x = y -> A. z x = y )
\end{verbatim}

The disjoint variable restrictions of \texttt{ax-c16} can be seen from the
\texttt{show state\-ment} command.  The line that begins ``\texttt{Its mandatory
dis\-joint var\-i\-able pairs are:}\ldots'' lists any \texttt{\$d} variable
pairs in brackets.

\begin{verbatim}
MM> show statement ax-c16/full
Statement 3033 is located on line 9338 of the file "set.mm".
"Axiom of Distinct Variables. ..."
  ax-c16 $a |- ( A. x x = y -> ( ph -> A. x ph ) ) $.
Its mandatory hypotheses in RPN order are:
  wph $f wff ph $.
  vx $f set x $.
  vy $f set y $.
Its mandatory disjoint variable pairs are:  <x,y>
The statement and its hypotheses require the variables:  x y
      ph
The variables it contains are:  x y ph
\end{verbatim}

Since Metamath will always detect when \texttt{\$d}\index{\texttt{\$d} statement}
statements are needed for a proof, you don't have to worry too much about
forgetting to put one in; it can always be added if you see the error message
above.  If you put in unnecessary \texttt{\$d} statements, the worst that could
happen is that your theorem might not be as general as it could be, and this
may limit its use later on.

On the other hand, when you introduce axioms (\texttt{\$a}\index{\texttt{\$a}
statement} statements), you must be very careful to properly specify the
necessary associated \texttt{\$d} statements since Metamath has no way of knowing
whether your axioms are correct.  For example, Metamath would have no idea
that \texttt{ax-c16}, which we are telling it is an axiom of logic, would lead to
contradictions if we omitted its associated \texttt{\$d} statement.

% This was previously a comment in footnote-sized type, but it can be
% hard to read this much text in a small size.
% As a result, it's been changed to normally-sized text.
\label{nodd}
You may wonder if it is possible to develop standard
mathematics in the Metamath language without the \texttt{\$d}\index{\texttt{\$d}
statement} statement, since it seems like a nuisance that complicates proof
verification. The \texttt{\$d} statement is not needed in certain subsets of
mathematics such as propositional calculus.  However, dummy
variables\index{dummy variable!eliminating} and their associated \texttt{\$d}
statements are impossible to avoid in proofs in standard first-order logic as
well as in the variant used in \texttt{set.mm}.  In fact, there is no upper bound to
the number of dummy variables that might be needed in a proof of a theorem of
first-order logic containing 3 or more variables, as shown by H.\
Andr\'{e}ka\index{Andr{\'{e}}ka, H.} \cite{Nemeti}.  A first-order system that
avoids them entirely is given in \cite{Megill}\index{Megill, Norman}; the
trick there is simply to embed harmlessly the necessary dummy variables into a
theorem being proved so that they aren't ``dummy'' anymore, then interpret the
resulting longer theorem so as to ignore the embedded dummy variables.  If
this interests you, the system in \texttt{set.mm} obtained from \texttt{ax-1}
through \texttt{ax-c14} in \texttt{set.mm}, and deleting \texttt{ax-c16} and \texttt{ax-5},
requires no \texttt{\$d} statements but is logically complete in the sense
described in \cite{Megill}.  This means it can prove any theorem of
first-order logic as long as we add to the theorem an antecedent that embeds
dummy and any other variables that must be distinct.  In a similar fashion,
axioms for set theory can be devised that
do not require distinct variable
provisos\index{Set theory without distinct variable provisos},
as explained at
\url{http://us.metamath.org/mpeuni/mmzfcnd.html}.
Together, these in principle allow all of
mathematics to be developed under Metamath without a \texttt{\$d} statement,
although the length of the resulting theorems will grow as more and
more dummy variables become required in their proofs.

\subsection{The \texttt{\$f}
and \texttt{\$e} Statements}\label{dollaref}
\index{\texttt{\$e} statement}
\index{\texttt{\$f} statement}
\index{floating hypothesis}
\index{essential hypothesis}
\index{variable-type hypothesis}
\index{logical hypothesis}
\index{hypothesis}

Metamath has two kinds of hypo\-theses, the \texttt{\$f}\index{\texttt{\$f}
statement} or {\bf variable-type} hypothesis and the \texttt{\$e} or {\bf logical}
hypo\-the\-sis.\index{\texttt{\$d} statement}\footnote{Strictly speaking, the
\texttt{\$d} statement is also a hypothesis, but it is never directly referenced
in a proof, so we call it a restriction rather than a hypothesis to lessen
confusion.  The checking for violations of \texttt{\$d} restrictions is automatic
and built into Metamath's proof-checking algorithm.} The letters \texttt{f} and
\texttt{e} stand for ``floating''\index{floating hypothesis} (roughly meaning
used only if relevant) and ``essential''\index{essential hypothesis} (meaning
always used) respectively, for reasons that will become apparent
when we discuss frames in
Section~\ref{frames} and scoping in Section~\ref{scoping}. The syntax of these
are as follows:
\begin{center}
  {\em label} \texttt{\$f} {\em typecode} {\em variable} \texttt{\$.}\\
  {\em label} \texttt{\$e} {\em typecode}
      {\em math-symbol}\ \,$\cdots$\ {\em math-symbol} \texttt{\$.}\\
\end{center}
\index{\texttt{\$e} statement}
\index{\texttt{\$f} statement}
A hypothesis must have a {\em label}\index{label}.  The expression in a
\texttt{\$e} hypothesis consists of a typecode (an active constant math symbol)
followed by a sequence
of zero or more math symbols. Each math symbol (including {\em constant}
and {\em variable}) must be a previously declared constant or variable.  (In
addition, each math symbol must be active, which will be covered when we
discuss scoping statements in Section~\ref{scoping}.)  You use a \texttt{\$f}
hypothesis to specify the
nature or {\bf type}\index{variable type}\index{type} of a variable (such as ``let $x$ be an
integer'') and use a \texttt{\$e} hypothesis to express a logical truth (such as
``assume $x$ is prime'') that must be established in order for an assertion
requiring it to also be true.

A variable must have its type specified in a \texttt{\$f} statement before
it may be used in a \texttt{\$e}, \texttt{\$a}, or \texttt{\$p}
statement.  There may be only one (active) \texttt{\$f} statement for a
given variable.  (``Active'' is defined in Section~\ref{scoping}.)

In ordinary mathematics, theorems\index{theorem} are often expressed in the
form ``Assume $P$; then $Q$,'' where $Q$ is a statement that you can derive
if you start with statement $P$.\index{free variable}\footnote{A stronger
version of a theorem like this would be the {\em single} formula $P\rightarrow
Q$ ($P$ implies $Q$) from which the weaker version above follows by the rule
of modus ponens in logic.  We are not discussing this stronger form here.  In
the weaker form, we are saying only that if we can {\em prove} $P$, then we can
{\em prove} $Q$.  In a logician's language, if $x$ is the only free variable
in $P$ and $Q$, the stronger form is equivalent to $\forall x ( P \rightarrow
Q)$ (for all $x$, $P$ implies $Q$), whereas the weaker form is equivalent to
$\forall x P \rightarrow \forall x Q$. The stronger form implies the weaker,
but not vice-versa.  To be precise, the weaker form of the theorem is more
properly called an ``inference'' rather than a theorem.}\index{inference}
In the
Metamath\index{Metamath} language, you would express mathematical statement
$P$ as a hypothesis (a \texttt{\$e} Metamath language statement in this case) and
statement $Q$ as a provable assertion (a \texttt{\$p}\index{\texttt{\$p} statement}
statement).

Some examples of hypotheses you might encounter in logic and set theory are
\begin{center}
  \texttt{stmt1 \$f wff P \$.}\\
  \texttt{stmt2 \$f set x \$.}\\
  \texttt{stmt3 \$e |- ( P -> Q ) \$.}
\end{center}
\index{\texttt{\$e} statement}
\index{\texttt{\$f} statement}
Informally, these would be read, ``Let $P$ be a well-formed-formula,'' ``Let
$x$ be an (individual) variable,'' and ``Assume we have proved $P \rightarrow
Q$.''  The turnstile symbol \,$\vdash$\index{turnstile ({$\,\vdash$})} is
commonly used in logic texts to mean ``a proof exists for.''

To summarize:
\begin{itemize}
\item A \texttt{\$f} hypothesis tells Metamath the type or kind of its variable.
It is analogous to a variable declaration in a computer language that
tells the compiler that a variable is an integer or a floating-point
number.
\item The \texttt{\$e} hypothesis corresponds to what you would usually call a
``hypothesis'' in ordinary mathematics.
\end{itemize}

Before an assertion\index{assertion} (\texttt{\$a} or \texttt{\$p} statement) can be
referenced in a proof, all of its associated \texttt{\$f} and \texttt{\$e} hypotheses
(i.e.\ those \texttt{\$e} hypotheses that are active) must be satisfied (i.e.
established by the proof).  The meaning of ``associated'' (which we will call
{\bf mandatory} in Section~\ref{frames}) will become clear when we discuss
scoping later.

Note that after any \texttt{\$f}, \texttt{\$e},
\texttt{\$a}, or \texttt{\$p} token there is a required
\textit{typecode}\index{typecode}.
The typecode is a constant used to enforce types of expressions.
This will become clearer once we learn more about
assertions (\texttt{\$a} and \texttt{\$p} statements).
An example may also clarify their purpose.
In the
\texttt{set.mm}\index{set theory database (\texttt{set.mm})}%
\index{Metamath Proof Explorer}
database,
the following typecodes are used:

\begin{itemize}
\item \texttt{wff} :
  Well-formed formula (wff) symbol
  (read: ``the following symbol sequence is a wff'').
% The *textual* typecode for turnstile is "|-", but when read it's a little
% confusing, so I intentionally display the mathematical symbol here instead
% (I think it's clearer in this context).
\item \texttt{$\vdash$} :
  Turnstile (read: ``the following symbol sequence is provable'' or
  ``a proof exists for'').
\item \texttt{setvar} :
  Individual set variable type (read: ``the following is an
  individual set variable'').
  Note that this is \textit{not} the type of an arbitrary set expression,
  instead, it is used to ensure that there is only a single symbol used
  after quantifiers like for-all ($\forall$) and there-exists ($\exists$).
\item \texttt{class} :
  An expression that is a syntactically valid class expression.
  All valid set expressions are also valid class expression, so expressions
  of sets normally have the \texttt{class} typecode.
  Use the \texttt{class} typecode,
  \textit{not} the \texttt{setvar} typecode,
  for the type of set expressions unless you are specifically identifying
  a single set variable.
\end{itemize}

\subsection{Assertions (\texttt{\$a} and \texttt{\$p} Statements)}
\index{\texttt{\$a} statement}
\index{\texttt{\$p} statement}\index{assertion}\index{axiomatic assertion}
\index{provable assertion}

There are two types of assertions, \texttt{\$a}\index{\texttt{\$a} statement}
statements ({\bf axiomatic assertions}) and \texttt{\$p} statements ({\bf
provable assertions}).  Their syntax is as follows:
\begin{center}
  {\em label} \texttt{\$a} {\em typecode} {\em math-symbol} \ldots
         {\em math-symbol} \texttt{\$.}\\
  {\em label} \texttt{\$p} {\em typecode} {\em math-symbol} \ldots
        {\em math-symbol} \texttt{\$=} {\em proof} \texttt{\$.}
\end{center}
\index{\texttt{\$a} statement}
\index{\texttt{\$p} statement}
\index{\texttt{\$=} keyword}
An assertion always requires a {\em label}\index{label}. The expression in an
assertion consists of a typecode (an active constant)
followed by a sequence of zero
or more math symbols.  Each math symbol, including any {\em constant}, must be a
previously declared constant or variable.  (In addition, each math symbol
must be active, which will be covered when we discuss scoping statements in
Section~\ref{scoping}.)

A \texttt{\$a} statement is usually a definition of syntax (for example, if $P$
and $Q$ are wffs then so is $(P\to Q)$), an axiom\index{axiom} of ordinary
mathematics (for example, $x=x$), or a definition\index{definition} of
ordinary mathematics (for example, $x\ne y$ means $\lnot x=y$). A \texttt{\$p}
statement is a claim that a certain combination of math symbols follows from
previous assertions and is accompanied by a proof that demonstrates it.

Assertions can also be referenced in (later) proofs in order to derive new
assertions from them. The label of an assertion is used to refer to it in a
proof. Section~\ref{proof} will describe the proof in detail.

Assertions also provide the primary means for communicating the mathematical
results in the database to people.  Proofs (when conveniently displayed)
communicate to people how the results were arrived at.

\subsubsection{The \texttt{\$a} Statement}
\index{\texttt{\$a} statement}

Axiomatic assertions (\texttt{\$a} statements) represent the starting points from
which other assertions (\texttt{\$p}\index{\texttt{\$p} statement} statements) are
derived.  Their most obvious use is for specifying ordinary mathematical
axioms\index{axiom}, but they are also used for two other purposes.

First, Metamath\index{Metamath} needs to know the syntax of symbol
sequences that constitute valid mathematical statements.  A Metamath
proof must be broken down into much more detail than ordinary
mathematical proofs that you may be used to thinking of (even the
``complete'' proofs of formal logic\index{formal logic}).  This is one
of the things that makes Metamath a general-purpose language,
independent of any system of logic or even syntax.  If you want to use a
substitution instance of an assertion as a step in a proof, you must
first prove that the substitution is syntactically correct (or if you
prefer, you must ``construct'' it), showing for example that the
expression you are substituting for a wff metavariable is a valid wff.
The \texttt{\$a}\index{\texttt{\$a} statement} statement is used to
specify those combinations of symbols that are considered syntactically
valid, such as the legal forms of wffs.

Second, \texttt{\$a} statements are used to specify what are ordinarily thought of
as definitions, i.e.\ new combinations of symbols that abbreviate other
combinations of symbols.  Metamath makes no distinction\index{axiom vs.\
definition} between axioms\index{axiom} and definitions\index{definition}.
Indeed, it has been argued that such distinction should not be made even in
ordinary mathematics; see Section~\ref{definitions}, which discusses the
philosophy of definitions.  Section~\ref{hierarchy} discusses some
technical requirements for definitions.  In \texttt{set.mm} we adopt the
convention of prefixing axiom labels with \texttt{ax-} and definition labels with
\texttt{df-}\index{label}.

The results that can be derived with the Metamath language are only as good as
the \texttt{\$a}\index{\texttt{\$a} statement} statements used as their starting
point.  We cannot stress this too strongly.  For example, Metamath will
not prevent you from specifying $x\neq x$ as an axiom of logic.  It is
essential that you scrutinize all \texttt{\$a} statements with great care.
Because they are a source of potential pitfalls, it is best not to add new
ones (usually new definitions) casually; rather you should carefully evaluate
each one's necessity and advantages.

Once you have in place all of the basic axioms\index{axiom} and
rules\index{rule} of a mathematical theory, the only \texttt{\$a} statements that
you will be adding will be what are ordinarily called definitions.  In
principle, definitions should be in some sense eliminable from the language of
a theory according to some convention (usually involving logical equivalence
or equality).  The most common convention is that any formula that was
syntactically valid but not provable before the definition was introduced will
not become provable after the definition is introduced.  In an ideal world,
definitions should not be present at all if one is to have absolute confidence
in a mathematical result.  However, they are necessary to make
mathematics practical, for otherwise the resulting formulas would be
extremely long and incomprehensible.  Since the nature of definitions (in the
most general sense) does not permit them to automatically be verified as
``proper,''\index{proper definition}\index{definition!proper} the judgment of
the mathematician is required to ensure it.  (In \texttt{set.mm} effort was made
to make almost all definitions directly eliminable and thus minimize the need
for such judgment.)

If you are not a mathematician, it may be best not to add or change any
\texttt{\$a}\index{\texttt{\$a} statement} statements but instead use
the mathematical language already provided in standard databases.  This
way Metamath will not allow you to make a mistake (i.e.\ prove a false
result).


\subsection{Frames}\label{frames}

We now introduce the concept of a collection of related Metamath statements
called a frame.  Every assertion (\texttt{\$a} or \texttt{\$p} statement) in the database has
an associated frame.

A {\bf frame}\index{frame} is a sequence of \texttt{\$d}, \texttt{\$f},
and \texttt{\$e} statements (zero or more of each) followed by one
\texttt{\$a} or \texttt{\$p} statement, subject to certain conditions we
will describe.  For simplicity we will assume that all math symbol
tokens used are declared at the beginning of the database with
\texttt{\$c} and \texttt{\$v} statements (which are not properly part of
a frame).  Also for simplicity we will assume there are only simple
\texttt{\$d} statements (those with only two variables) and imagine any
compound \texttt{\$d} statements (those with more than two variables) as
broken up into simple ones.

A frame groups together those hypotheses (and \texttt{\$d} statements) relevant
to an assertion (\texttt{\$a} or \texttt{\$p} statement).  The statements in a frame
may or may not be physically adjacent in a database; we will cover
this in our discussion of scoping statements
in Section~\ref{scoping}.

A frame has the following properties:
\begin{enumerate}
 \item The set of variables contained in its \texttt{\$f} statements must
be identical to the set of variables contained in its \texttt{\$e},
\texttt{\$a}, and/or \texttt{\$p} statements.  In other words, each
variable in a \texttt{\$e}, \texttt{\$a}, or \texttt{\$p} statement must
have an associated ``variable type'' defined for it in a \texttt{\$f}
statement.
  \item No two \texttt{\$f} statements may contain the same variable.
  \item Any \texttt{\$f} statement
must occur before a \texttt{\$e} statement in which its variable occurs.
\end{enumerate}

The first property determines the set of variables occurring in a frame.
These are the {\bf mandatory
variables}\index{mandatory variable} of the frame.  The second property
tells us there must be only one type specified for a variable.
The last property is not a theoretical requirement but it
makes parsing of the database easier.

For our examples, we assume our database has the following declarations:

\begin{verbatim}
$v P Q R $.
$c -> ( ) |- wff $.
\end{verbatim}

The following sequence of statements, describing the modus ponens inference
rule, is an example of a frame:

\begin{verbatim}
wp  $f wff P $.
wq  $f wff Q $.
maj $e |- ( P -> Q ) $.
min $e |- P $.
mp  $a |- Q $.
\end{verbatim}

The following sequence of statements is not a frame because \texttt{R} does not
occur in the \texttt{\$e}'s or the \texttt{\$a}:

\begin{verbatim}
wp  $f wff P $.
wq  $f wff Q $.
wr  $f wff R $.
maj $e |- ( P -> Q ) $.
min $e |- P $.
mp  $a |- Q $.
\end{verbatim}

The following sequence of statements is not a frame because \texttt{Q} does not
occur in a \texttt{\$f}:

\begin{verbatim}
wp  $f wff P $.
maj $e |- ( P -> Q ) $.
min $e |- P $.
mp  $a |- Q $.
\end{verbatim}

The following sequence of statements is not a frame because the \texttt{\$a} statement is
not the last one:

\begin{verbatim}
wp  $f wff P $.
wq  $f wff Q $.
maj $e |- ( P -> Q ) $.
mp  $a |- Q $.
min $e |- P $.
\end{verbatim}

Associated with a frame is a sequence of {\bf mandatory
hypotheses}\index{mandatory hypothesis}.  This is simply the set of all
\texttt{\$f} and \texttt{\$e} statements in the frame, in the order they
appear.  A frame can be referenced in a later proof using the label of
the \texttt{\$a} or \texttt{\$p} assertion statement, and the proof
makes an assignment to each mandatory hypothesis in the order in which
it appears.  This means the order of the hypotheses, once chosen, must
not be changed so as not to affect later proofs referencing the frame's
assertion statement.  (The Metamath proof verifier will, of course, flag
an error if a proof becomes incorrect by doing this.)  Since proofs make
use of ``Reverse Polish notation,'' described in Section~\ref{proof}, we
call this order the {\bf RPN order}\index{RPN order} of the hypotheses.

Note that \texttt{\$d} statements are not part of the set of mandatory
hypotheses, and their order doesn't matter (as long as they satisfy the
fourth property for a frame described above).  The \texttt{\$d}
statements specify restrictions on variables that must be satisfied (and
are checked by the proof verifier) when expressions are substituted for
them in a proof, and the \texttt{\$d} statements themselves are never
referenced directly in a proof.

A frame with a \texttt{\$p} (provable) statement requires a proof as part of the
\texttt{\$p} statement.  Sometimes in a proof we want to make use of temporary or
dummy variables\index{dummy variable} that do not occur in the \texttt{\$p}
statement or its mandatory hypotheses.  To accommodate this we define an {\bf
extended frame}\index{extended frame} as a frame together with zero or more
\texttt{\$d} and \texttt{\$f} statements that reference variables not among the
mandatory variables of the frame.  Any new variables referenced are called the
{\bf optional variables}\index{optional variable} of the extended frame. If a
\texttt{\$f} statement references an optional variable it is called an {\bf
optional hypothesis}\index{optional hypothesis}, and if one or both of the
variables in a \texttt{\$d} statement are optional variables it is called an {\bf
optional disjoint-variable restriction}\index{optional disjoint-variable
restriction}.  Properties 2, 3, and 4 for a frame also apply to an extended
frame.

The concept of optional variables is not meaningful for frames with \texttt{\$a}
statements, since those statements have no proofs that might make use of them.
There is no restriction on including optional hypotheses in the extended frame
for a \texttt{\$a} statement, but they serve no purpose.

The following set of statements is an example of an extended frame, which
contains an optional variable \texttt{R} and an optional hypothesis \texttt{wr}.  In
this example, we suppose the rule of modus ponens is not an axiom but is
derived as a theorem from earlier statements (we omit its presumed proof).
Variable \texttt{R} may be used in its proof if desired (although this would
probably have no advantage in propositional calculus).  Note that the sequence
of mandatory hypotheses in RPN order is still \texttt{wp}, \texttt{wq}, \texttt{maj},
\texttt{min} (i.e.\ \texttt{wr} is omitted), and this sequence is still assumed
whenever the assertion \texttt{mp} is referenced in a subsequent proof.

\begin{verbatim}
wp  $f wff P $.
wq  $f wff Q $.
wr  $f wff R $.
maj $e |- ( P -> Q ) $.
min $e |- P $.
mp  $p |- Q $= ... $.
\end{verbatim}

Every frame is an extended frame, but not every extended frame is a frame, as
this example shows.  The underlying frame for an extended frame is
obtained by simply removing all statements containing optional variables.
Any proof referencing an assertion will ignore any extensions to its
frame, which means we may add or delete optional hypotheses at will without
affecting subsequent proofs.

The conceptually simplest way of organizing a Metamath database is as a
sequence of extended frames.  The scoping statements
\texttt{\$\char`\{}\index{\texttt{\$\char`\{} and \texttt{\$\char`\}}
keywords} and \texttt{\$\char`\}} can be used to delimit the start and
end of an extended frame, leading to the following possible structure for a
database.  \label{framelist}

\vskip 2ex
\setbox\startprefix=\hbox{\tt \ \ \ \ \ \ \ \ }
\setbox\contprefix=\hbox{}
\startm
\m{\mbox{(\texttt{\$v} {\em and} \texttt{\$c}\,{\em statements})}}
\endm
\startm
\m{\mbox{\texttt{\$\char`\{}}}
\endm
\startm
\m{\mbox{\texttt{\ \ } {\em extended frame}}}
\endm
\startm
\m{\mbox{\texttt{\$\char`\}}}}
\endm
\startm
\m{\mbox{\texttt{\$\char`\{}}}
\endm
\startm
\m{\mbox{\texttt{\ \ } {\em extended frame}}}
\endm
\startm
\m{\mbox{\texttt{\$\char`\}}}}
\endm
\startm
\m{\mbox{\texttt{\ \ \ \ \ \ \ \ \ }}\vdots}
\endm
\vskip 2ex

In practice, this structure is inconvenient because we have to repeat
any \texttt{\$f}, \texttt{\$e}, and \texttt{\$d} statements over and
over again rather than stating them once for use by several assertions.
The scoping statements, which we will discuss next, allow this to be
done.  In principle, any Metamath database can be converted to the above
format, and the above format is the most convenient to use when studying
a Metamath database as a formal system%
%% Uncomment this when uncommenting section {formalspec} below
   (Appendix \ref{formalspec})%
.
In fact, Metamath internally converts the database to the above format.
The command \texttt{show statement} in the Metamath program will show
you the contents of the frame for any \texttt{\$a} or \texttt{\$p}
statement, as well as its extension in the case of a \texttt{\$p}
statement.

%c%(provided that all ``local'' variables and constants with limited scope have
%c%unique names),

During our discussion of scoping statements, it may be helpful to
think in terms of the equivalent sequence of frames that will result when
the database is parsed.  Scoping (other than the limited
use above to delimit frames) is not a theoretical requirement for
Metamath but makes it more convenient.


\subsection{Scoping Statements (\texttt{\$\{} and \texttt{\$\}})}\label{scoping}
\index{\texttt{\$\char`\{} and \texttt{\$\char`\}} keywords}\index{scoping statement}

%c%Some Metamath statements may be needed only temporarily to
%c%serve a specific purpose, and after we're done with them we would like to
%c%disregard or ignore them.  For example, when we're finished using a variable,
%c%we might want to
%c%we might want to free up the token\index{token} used to name it so that the
%c%token can be used for other purposes later on, such as a different kind of
%c%variable or even a constant.  In the terminology of computer programming, we
%c%might want to let some symbol declarations be ``local'' rather than ``global.''
%c%\index{local symbol}\index{global symbol}

The {\bf scoping} statements, \texttt{\$\char`\{} ({\bf start of block}) and \texttt{\$\char`\}}
({\bf end of block})\index{block}, provide a means for controlling the portion
of a database over which certain statement types are recognized.  The
syntax of a scoping statement is very simple; it just consists of the
statement's keyword:
\begin{center}
\texttt{\$\char`\{}\\
\texttt{\$\char`\}}
\end{center}
\index{\texttt{\$\char`\{} and \texttt{\$\char`\}} keywords}

For example, consider the following database where we have stripped out
all tokens except the scoping statement keywords.  For the purpose of the
discussion, we have added subscripts to the scoping statements; these subscripts
do not appear in the actual database.
\[
 \mbox{\tt \ \$\char`\{}_1
 \mbox{\tt \ \$\char`\{}_2
 \mbox{\tt \ \$\char`\}}_2
 \mbox{\tt \ \$\char`\{}_3
 \mbox{\tt \ \$\char`\{}_4
 \mbox{\tt \ \$\char`\}}_4
 \mbox{\tt \ \$\char`\}}_3
 \mbox{\tt \ \$\char`\}}_1
\]
Each \texttt{\$\char`\{} statement in this example is said to be {\bf
matched} with the \texttt{\$\char`\}} statement that has the same
subscript.  Each pair of matched scoping statements defines a region of
the database called a {\bf block}.\index{block} Blocks can be {\bf
nested}\index{nested block} inside other blocks; in the example, the
block defined by $\mbox{\tt \$\char`\{}_4$ and $\mbox{\tt \$\char`\}}_4$
is nested inside the block defined by $\mbox{\tt \$\char`\{}_3$ and
$\mbox{\tt \$\char`\}}_3$ as well as inside the block defined by
$\mbox{\tt \$\char`\{}_1$ and $\mbox{\tt \$\char`\}}_1$.  In general, a
block may be empty, it may contain only non-scoping
statements,\footnote{Those statements other than \texttt{\$\char`\{} and
\texttt{\$\char`\}}.}\index{non-scoping statement} or it may contain any
mixture of other blocks and non-scoping statements.  (This is called a
``recursive'' definition\index{recursive definition} of a block.)

Associated with each block is a number called its {\bf nesting
level}\index{nesting level} that indicates how deeply the block is nested.
The nesting levels of the blocks in our example are as follows:
\[
  \underbrace{
    \mbox{\tt \ }
    \underbrace{
     \mbox{\tt \$\char`\{\ }
     \underbrace{
       \mbox{\tt \$\char`\{\ }
       \mbox{\tt \$\char`\}}
     }_{2}
     \mbox{\tt \ }
     \underbrace{
       \mbox{\tt \$\char`\{\ }
       \underbrace{
         \mbox{\tt \$\char`\{\ }
         \mbox{\tt \$\char`\}}
       }_{3}
       \mbox{\tt \ \$\char`\}}
     }_{2}
     \mbox{\tt \ \$\char`\}}
   }_{1}
   \mbox{\tt \ }
 }_{0}
\]
\index{\texttt{\$\char`\{} and \texttt{\$\char`\}} keywords}
The entire database is considered to be one big block (the {\bf outermost}
block) with a nesting level of 0.  The outermost block is {\em not} bracketed
by scoping statements.\footnote{The language was designed this way so that
several source files can be joined together more easily.}\index{outermost
block}

All non-scoping Metamath statements become recognized or {\bf
active}\index{active statement} at the place where they appear.\footnote{To
keep things slightly simpler, we do not bother to define the concept of
``active'' for the scoping statements.}  Certain of these statement types
become inactive at the end of the block in which they appear; these statement
types are:
\begin{center}
  \texttt{\$c}, \texttt{\$v}, \texttt{\$d}, \texttt{\$e}, and \texttt{\$f}.
%  \texttt{\$v}, \texttt{\$f}, \texttt{\$e}, and \texttt{\$d}.
\end{center}
\index{\texttt{\$c} statement}
\index{\texttt{\$d} statement}
\index{\texttt{\$e} statement}
\index{\texttt{\$f} statement}
\index{\texttt{\$v} statement}
The other statement types remain active forever (i.e.\ through the end of the
database); they are:
\begin{center}
  \texttt{\$a} and \texttt{\$p}.
%  \texttt{\$c}, \texttt{\$a}, and \texttt{\$p}.
\end{center}
\index{\texttt{\$a} statement}
\index{\texttt{\$p} statement}
Any statement (of these 7 types) located in the outermost
block\index{outermost block} will remain active through the end of the
database and thus are effectively ``global'' statements.\index{global
statement}

All \texttt{\$c} statements must be placed in the outermost block.  Since they are
therefore always global, they could be considered as belonging to both of the
above categories.

The {\bf scope}\index{scope} of a statement is the set of statements that
recognize it as active.

%c%The concept of ``active'' is also defined for math symbols\index{math
%c%symbol}.  Math symbols (constants\index{constant} and
%c%variables\index{variable}) become {\bf active}\index{active
%c%math symbol} in the \texttt{\$c}\index{\texttt{\$c}
%c%statement} and \texttt{\$v}\index{\texttt{\$v} statement} statements that
%c%declare them.  They become inactive when their declaration statements become
%c%inactive.

The concept of ``active'' is also defined for math symbols\index{math
symbol}.  Math symbols (constants\index{constant} and
variables\index{variable}) become {\bf active}\index{active math symbol}
in the \texttt{\$c}\index{\texttt{\$c} statement} and
\texttt{\$v}\index{\texttt{\$v} statement} statements that declare them.
A variable becomes inactive when its declaration statement becomes
inactive.  Because all \texttt{\$c} statements must be in the outermost
block, a constant will never become inactive after it is declared.

\subsubsection{Redeclaration of Math Symbols}
\index{redeclaration of symbols}\label{redeclaration}

%c%A math symbol may not be declared a second time while it is active, but it may
%c%be declared again after it becomes inactive.

A variable may not be declared a second time while it is active, but it may be
declared again after it becomes inactive.  This provides a convenient way to
introduce ``local'' variables,\index{local variable} i.e.\ temporary variables
for use in the frame of an assertion or in a proof without keeping them around
forever.  A previously declared variable may not be redeclared as a constant.

A constant may not be redeclared.  And, as mentioned above, constants must be
declared in the outermost block.

The reason variables may have limited scope but not constants is that an
assertion (\texttt{\$a} or \texttt{\$p} statement) remains available for use in
proofs through the end of the database.  Variables in an assertion's frame may
be substituted with whatever is needed in a proof step that references the
assertion, whereas constants remain fixed and may not be substituted with
anything.  The particular token used for a variable in an assertion's frame is
irrelevant when the assertion is referenced in a proof, and it doesn't matter
if that token is not available outside of the referenced assertion's frame.
Constants, however, must be globally fixed.

In the present version of the Metamath language, there is no theoretical
benefit for the feature allowing variables to be active for limited scopes
rather than global. It is just a convenience that allows them, for example, to
be locally grouped together with their corresponding \texttt{\$f} variable-type
declarations.\footnote{A future version\label{redeclarationf} of the Metamath
language may extend the language to include true local variables as well as
local constants.  I have not decided upon the best way to do this or even if
it should be done at all; the main issue is whether the benefits offset
greater confusion in learning the language.  In any case, the rules for
the present language were chosen to let existing databases be compatible with
any future language extensions.}

%c%If you declare a math symbol more than once, internally Metamath considers it a
%c%new distinct symbol, even though it has the same name.  If you are unaware of
%c%this, you may find that what you think are correct proofs are incorrectly
%c%rejected as invalid, because Metamath may tell you that a constant you
%c%previously declared does not match a newly declared math symbol with the same
%c%name.  For details on this subtle point, see the Comment on
%c%p.~\pageref{spec4comment}.  This is done purposely to allow temporary
%c%constants to be introduced while developing a subtheory, then allow their math
%c%symbol tokens to be reused later on; in general they will not refer to the
%c%same thing.  In practice, you would not ordinarily reuse the names of
%c%constants because it would tend to be confusing to the reader.  The reuse of
%c%names of variables, on the other hand, is something that is often useful to do
%c%(for example it is done frequently in \texttt{set.mm}).  Since variables in an
%c%assertion referenced in a proof can be substituted as needed to achieve a
%c%symbol match, this is not an issue.

% (This section covers a somewhat advanced topic you may want to skip
% at first reading.)
%
% Under certain circumstances, math symbol\index{math symbol}
% tokens\index{token} may be redeclared (i.e.\ the token
% may appear in more than
% one \texttt{\$c}\index{\texttt{\$c} statement} or \texttt{\$v}\index{\texttt{\$v}
% statement} statement).  You might want to do this say, to make temporary use
% of a variable name without having to worry about its affect elsewhere,
% somewhat analogous to declaring a local variable in a standard computer
% language.  Understanding what goes on when math symbol tokens are redeclared
% is a little tricky to understand at first, since it requires that we
% distinguish the token itself from the math symbol that it names.  It will help
% if we first take a peek at the internal workings of the
% Metamath\index{Metamath} program.
%
% Metamath reserves a memory location for each occurrence of a
% token\index{token} in a declaration statement (\texttt{\$c}\index{\texttt{\$c}
% statement} or \texttt{\$v}\index{\texttt{\$v} statement}).  If a given token appears
% in more than one declaration statement, it will refer to more than one memory
% locations.  A math symbol\index{math symbol} may be thought of as being one of
% these memory locations rather than as the token itself.  Only one of the
% memory locations associated with a given token may be active at any one time.
% The math symbol (memory location) that gets looked up when the token appears
% in a non-declaration statement is the one that happens to be active at that
% time.
%
% We now look at the rules for the redeclaration\index{redeclaration of symbols}
% of math symbol tokens.
% \begin{itemize}
% \item A math symbol token may not be declared twice in the
% same block.\footnote{While there is no theoretical reason for disallowing
% this, it was decided in the design of Metamath that allowing it would offer no
% advantage and might cause confusion.}
% \item An inactive math symbol may always be
% redeclared.
% \item  An active math symbol may be redeclared in a different (i.e.\
% inner) block\index{block} from the one it became active in.
% \end{itemize}
%
% When a math symbol token is redeclared, it conceptually refers to a different
% math symbol, just as it would be if it were called a different name.  In
% addition, the original math symbol that it referred to, if it was active,
% temporarily becomes inactive.  At the end of the block in which the
% redeclaration occurred, the new math symbol\index{math symbol} becomes
% inactive and the original symbol becomes active again.  This concept is
% illustrated in the following example, where the symbol \texttt{e} is
% ordinarily a constant (say Euler's constant, 2.71828...) but
% temporarily we want to use it as a ``local'' variable, say as a coefficient
% in the equation $a x^4 + b x^3 + c x^2 + d x + e$:
% \[
%   \mbox{\tt \$\char`\{\ \$c e \$.}
%   \underbrace{
%     \ \ldots\ %
%     \mbox{\tt \$\char`\{}\ \ldots\ %
%   }_{\mbox{\rm region A}}
%   \mbox{\tt \$v e \$.}
%   \underbrace{
%     \mbox{\ \ \ \ldots\ \ \ }
%   }_{\mbox{\rm region B}}
%   \mbox{\tt \$\char`\}}
%   \underbrace{
%     \mbox{\ \ \ \ldots\ \ \ }
%   }_{\mbox{\rm region C}}
%   \mbox{\tt \$\char`\}}
% \]
% \index{\texttt{\$\char`\{} and \texttt{\$\char`\}} keywords}
% In region A, the token \texttt{e} refers to a constant.  It is redeclared as a
% variable in region B, and any reference to it in this region will refer to this
% variable.  In region C, the redeclaration becomes inactive, and the original
% declaration becomes active again.  In region C, the token \texttt{x} refers to the
% original constant.
%
% As a practical matter, overuse of math symbol\index{math symbol}
% redeclarations\index{redeclaration of symbols} can be confusing (even though
% it is well-defined) and is best avoided when possible.  Here are some good
% general guidelines you can follow.  Usually, you should declare all
% constants\index{constant} in the outermost block\index{outermost block},
% especially if they are general-purpose (such as the token \verb$A.$, meaning
% $\forall$ or ``for all'').  This will make them ``globally'' active (although
% as in the example above local redeclarations will temporarily make them
% inactive.)  Most or all variables\index{variable}, on the other hand, could be
% declared in inner blocks, so that the token for them can be used later for a
% different type of variable or a constant.  (The names of the variables you
% choose are not used when you refer to an assertion\index{assertion} in a
% proof, whereas constants must match exactly.  A locally declared constant will
% not match a globally declared constant in a proof, even if they use the same
% token, because Metamath internally considers them to be different math
% symbols.)  To avoid confusion, you should generally avoid redeclaring active
% variables.  If you must redeclare them, do so at the beginning of a block.
% The temporary declaration of constants in inner blocks might be occasionally
% appropriate when you make use of a temporary definition to prove lemmas
% leading to a main result that does not make direct use of the definition.
% This way, you will not clutter up your database with a large number of
% seldom-used global constant symbols.  You might want to note that while
% inactive constants may not appear directly in an assertion (a \texttt{\$a}\index{\texttt{\$a}
% statement} or \texttt{\$p}\index{\texttt{\$p} statement}
% statement), they may be indirectly used in the proof of a \texttt{\$p} statement
% so long as they do not appear in the final math symbol sequence constructed by
% the proof.  In the end, you will have to use your best judgment, taking into
% account standard mathematical usage of the symbols as well as consideration
% for the reader of your work.
%
% \subsubsection{Reuse of Labels}\index{reuse of labels}\index{label}
%
% The \texttt{\$e}\index{\texttt{\$e} statement}, \texttt{\$f}\index{\texttt{\$f}
% statement}, \texttt{\$a}\index{\texttt{\$a} statement}, and
% \texttt{\$p}\index{\texttt{\$p}
% statement} statement types require labels, which allow them to be
% referenced later inside proofs.  A label is considered {\bf
% active}\index{active label} when the statement it is associated with is
% active.  The token\index{token} for a label may be reused
% (redeclared)\index{redeclaration of labels} provided that it is not being used
% for a currently active label.  (Unlike the tokens for math symbols, active
% label tokens may not be redeclared in an inner scope.)  Note that the labels
% of \texttt{\$a} and \texttt{\$p} statements can never be reused after these
% statements appear, because these statements remain active through the end of
% the database.
%
% You might find the reuse of labels a convenient way to have standard names for
% temporary hypotheses, such as \texttt{h1}, \texttt{h2}, etc.  This way you don't have
% to invent unique names for each of them, and in some cases it may be less
% confusing to the reader (although in other cases it might be more confusing, if
% the hypothesis is located far away from the assertion that uses
% it).\footnote{The current implementation requires that all labels, even
% inactive ones, be unique.}

\subsubsection{Frames Revisited}\index{frames and scoping statements}

Now that we have covered scoping, we will look at how an arbitrary
Metamath database can be converted to the simple sequence of extended
frames described on p.~\pageref{framelist}.  This is also how Metamath
stores the database internally when it reads in the database
source.\label{frameconvert} The method is simple.  First, we collect all
constant and variable (\texttt{\$c} and \texttt{\$v}) declarations in
the database, ignoring duplicate declarations of the same variable in
different scopes.  We then put our collected \texttt{\$c} and
\texttt{\$v} declarations at the beginning of the database, so that
their scope is the entire database.  Next, for each assertion in the
database, we determine its frame and extended frame.  The extended frame
is simply the \texttt{\$f}, \texttt{\$e}, and \texttt{\$d} statements
that are active.  The frame is the extended frame with all optional
hypotheses removed.

An equivalent way of saying this is that the extended frame of an assertion
is the collection of all \texttt{\$f}, \texttt{\$e}, and \texttt{\$d} statements
whose scope includes the assertion.
The \texttt{\$f} and \texttt{\$e} statements
occur in the order they appear
(order is irrelevant for \texttt{\$d} statements).

%c%, renaming any
%c%redeclared variables as needed so that all of them have unique names.  (The
%c%exact renaming convention is unimportant.  You might imagine renaming
%c%different declarations of math symbol \texttt{a} as \texttt{a\$1}, \texttt{a\$2}, etc.\
%c%which would prevent any conflicts since \texttt{\$} is not a legal character in a
%c%math symbol token.)

\section{The Anatomy of a Proof} \label{proof}
\index{proof!Metamath, description of}

Each provable assertion (\texttt{\$p}\index{\texttt{\$p} statement} statement) in a
database must include a {\bf proof}\index{proof}.  The proof is located
between the \texttt{\$=}\index{\texttt{\$=} keyword} and \texttt{\$.}\ keywords in the
\texttt{\$p} statement.

In the basic Metamath language\index{basic language}, a proof is a
sequence of statement labels.  This label sequence\index{label sequence}
serves as a set of instructions that the Metamath program uses to
construct a series of math symbol sequences.  The construction must
ultimately result in the math symbol sequence contained between the
\texttt{\$p}\index{\texttt{\$p} statement} and
\texttt{\$=}\index{\texttt{\$=} keyword} keywords of the \texttt{\$p}
statement.  Otherwise, the Metamath program will consider the proof
incorrect, and it will notify you with an appropriate error message when
you ask it to verify the proof.\footnote{To make the loading faster, the
Metamath program does not automatically verify proofs when you
\texttt{read} in a database unless you use the \texttt{/verify}
qualifier.  After a database has been read in, you may use the
\texttt{verify proof *} command to verify proofs.}\index{\texttt{verify
proof} command} Each label in a proof is said to {\bf
reference}\index{label reference} its corresponding statement.

Associated with any assertion\index{assertion} (\texttt{\$p} or
\texttt{\$a}\index{\texttt{\$a} statement} statement) is a set of
hypotheses (\texttt{\$f}\index{\texttt{\$f} statement} or
\texttt{\$e}\index{\texttt{\$e} statement} statements) that are active
with respect to that assertion.  Some are mandatory and the others are
optional.  You should review these concepts if necessary.

Each label\index{label} in a proof must be either the label of a
previous assertion (\texttt{\$a}\index{\texttt{\$a} statement} or
\texttt{\$p}\index{\texttt{\$p} statement} statement) or the label of an
active hypothesis (\texttt{\$e} or \texttt{\$f}\index{\texttt{\$f}
statement} statement) of the \texttt{\$p} statement containing the
proof.  Hypothesis labels may reference both the
mandatory\index{mandatory hypothesis} and the optional hypotheses of the
\texttt{\$p} statement.

The label sequence in a proof specifies a construction in {\bf reverse Polish
notation}\index{reverse Polish notation (RPN)} (RPN).  You may be familiar
with RPN if you have used older
Hewlett--Packard or similar hand-held calculators.
In the calculator analogy, a hypothesis label\index{hypothesis label} is like
a number and an assertion label\index{assertion label} is like an operation
(more precisely, an $n$-ary operation when the
assertion has $n$ \texttt{\$e}-hypotheses).
On an RPN calculator, an operation takes one or more previous numbers in an
input sequence, performs a calculation on them, and replaces those numbers and
itself with the result of the calculation.  For example, the input sequence
$2,3,+$ on an RPN calculator results in $5$, and the input sequence
$2,3,5,{\times},+$ results in $2,15,+$ which results in $17$.

Understanding how RPN is processed involves the concept of a {\bf
stack}\index{stack}\index{RPN stack}, which can be thought of as a set of
temporary memory locations that hold intermediate results.  When Metamath
encounters a hypothesis label it places or {\bf pushes}\index{push} the math
symbol sequence of the hypothesis onto the stack.  When Metamath encounters an
assertion label, it associates the most recent stack entries with the {\em
mandatory} hypotheses\index{mandatory hypothesis} of the assertion, in the
order where the most recent stack entry is associated with the last mandatory
hypothesis of the assertion.  It then determines what
substitutions\index{substitution!variable}\index{variable substitution} have
to be made into the variables of the assertion's mandatory hypotheses to make
them identical to the associated stack entries.  It then makes those same
substitutions into the assertion itself.  Finally, Metamath removes or {\bf
pops}\index{pop} the matched hypotheses from the stack and pushes the
substituted assertion onto the stack.

For the purpose of matching the mandatory hypothesis to the most recent stack
entries, whether a hypothesis is a \texttt{\$e} or \texttt{\$f} statement is
irrelevant.  The only important thing is that a set of
substitutions\footnote{In the Metamath spec (Section~\ref{spec}), we use the
singular term ``substitution'' to refer to the set of substitutions we talk
about here.} exist that allow a match (and if they don't, the proof verifier
will let you know with an error message).  The Metamath language is specified
in such a way that if a set of substitutions exists, it will be unique.
Specifically, the requirement that each variable have a type specified for it
with a \texttt{\$f} statement ensures the uniqueness.

We will illustrate this with an example.
Consider the following Metamath source file:
\begin{verbatim}
$c ( ) -> wff $.
$v p q r s $.
wp $f wff p $.
wq $f wff q $.
wr $f wff r $.
ws $f wff s $.
w2 $a wff ( p -> q ) $.
wnew $p wff ( s -> ( r -> p ) ) $= ws wr wp w2 w2 $.
\end{verbatim}
This Metamath source example shows the definition and ``proof'' (i.e.,
construction) of a well-formed formula (wff)\index{well-formed formula (wff)}
in propositional calculus.  (You may wish to type this example into a file to
experiment with the Metamath program.)  The first two statements declare
(introduce the names of) four constants and four variables.  The next four
statements specify the variable types, namely that
each variable is assumed to be a wff.  Statement \texttt{w2} defines (postulates)
a way to produce a new wff, \texttt{( p -> q )}, from two given wffs \texttt{p} and
\texttt{q}. The mandatory hypotheses of \texttt{w2} are \texttt{wp} and \texttt{wq}.
Statement \texttt{wnew} claims that \texttt{( s -> ( r -> p ) )} is a wff given
three wffs \texttt{s}, \texttt{r}, and \texttt{p}.  More precisely, \texttt{wnew} claims
that the sequence of ten symbols \texttt{wff ( s -> ( r -> p ) )} is provable from
previous assertions and the hypotheses of \texttt{wnew}.  Metamath does not know
or care what a wff is, and as far as it is concerned
the typecode \texttt{wff} is just an
arbitrary constant symbol in a math symbol sequence.  The mandatory hypotheses
of \texttt{wnew} are \texttt{wp}, \texttt{wr}, and \texttt{ws}; \texttt{wq} is an optional
hypothesis.  In our particular proof, the optional hypothesis is not
referenced, but in general, any combination of active (i.e.\ optional and
mandatory) hypotheses could be referenced.  The proof of statement \texttt{wnew}
is the sequence of five labels starting with \texttt{ws} (step~1) and ending with
\texttt{w2} (step~5).

When Metamath verifies the proof, it scans the proof from left to right.  We
will examine what happens at each step of the proof.  The stack starts off
empty.  At step 1, Metamath looks up label \texttt{ws} and determines that it is a
hypothesis, so it pushes the symbol sequence of statement \texttt{ws} onto the
stack:

\begin{center}\begin{tabular}{|l|l|}\hline
{Stack location} & {Contents} \\ \hline \hline
1 & \texttt{wff s} \\ \hline
\end{tabular}\end{center}

Metamath sees that the labels \texttt{wr} and \texttt{wp} in steps~2 and 3 are also
hypotheses, so it pushes them onto the stack.  After step~3, the stack looks
like
this:

\begin{center}\begin{tabular}{|l|l|}\hline
{Stack location} & {Contents} \\ \hline \hline
3 & \texttt{wff p} \\ \hline
2 & \texttt{wff r} \\ \hline
1 & \texttt{wff s} \\ \hline
\end{tabular}\end{center}

At step 4, Metamath sees that label \texttt{w2} is an assertion, so it must do
some processing.  First, it associates the mandatory hypotheses of \texttt{w2},
which are \texttt{wp} and \texttt{wq}, with stack locations~2 and 3, {\em in that
order}. Metamath determines that the only possible way
to make hypothesis \texttt{wp} match (become identical to) stack location~2 and
\texttt{wq} match stack location 3 is to substitute variable \texttt{p} with \texttt{r}
and \texttt{q} with \texttt{p}.  Metamath makes these substitutions into \texttt{w2} and
obtains the symbol sequence \texttt{wff ( r -> p )}.  It removes the hypotheses
from stack locations~2 and 3, then places the result into stack location~2:

\begin{center}\begin{tabular}{|l|l|}\hline
{Stack location} & {Contents} \\ \hline \hline
2 & \texttt{wff ( r -> p )} \\ \hline
1 & \texttt{wff s} \\ \hline
\end{tabular}\end{center}

At step 5, Metamath sees that label \texttt{w2} is an assertion, so it must again
do some processing.  First, it matches the mandatory hypotheses of \texttt{w2},
which are \texttt{wp} and \texttt{wq}, to stack locations 1 and 2.
Metamath determines that the only possible way to make the
hypotheses match is to substitute variable \texttt{p} with \texttt{s} and \texttt{q} with
\texttt{( r -> p )}.  Metamath makes these substitutions into \texttt{w2} and obtains
the symbol
sequence \texttt{wff ( s -> ( r -> p ) )}.  It removes stack
locations 1 and 2, then places the result into stack location~1:

\begin{center}\begin{tabular}{|l|l|}\hline
{Stack location} & {Contents} \\ \hline \hline
1 & \texttt{wff ( s -> ( r -> p ) )} \\ \hline
\end{tabular}\end{center}

After Metamath finishes processing the proof, it checks to see that the
stack contains exactly one element and that this element is
the same as the math symbol sequence in the
\texttt{\$p}\index{\texttt{\$p} statement} statement.  This is the case for our
proof of \texttt{wnew},
so we have proved \texttt{wnew} successfully.  If the result
differs, Metamath will notify you with an error message.  An error message
will also result if the stack contains more than one entry at the end of the
proof, or if the stack did not contain enough entries at any point in the
proof to match all of the mandatory hypotheses\index{mandatory hypothesis} of
an assertion.  Finally, Metamath will notify you with an error message if no
substitution is possible that will make a referenced assertion's hypothesis
match the
stack entries.  You may want to experiment with the different kinds of errors
that Metamath will detect by making some small changes in the proof of our
example.

Metamath's proof notation was designed primarily to express proofs in a
relatively compact manner, not for readability by humans.  Metamath can display
proofs in a number of different ways with the \texttt{show proof}\index{\texttt{show
proof} command} command.  The
\texttt{/lemmon} qualifier displays it in a format that is easier to read when the
proofs are short, and you saw examples of its use in Chapter~\ref{using}.  For
longer proofs, it is useful to see the tree structure of the proof.  A tree
structure is displayed when the \texttt{/lemmon} qualifier is omitted.  You will
probably find this display more convenient as you get used to it. The tree
display of the proof in our example looks like
this:\label{treeproof}\index{tree-style proof}\index{proof!tree-style}
\begin{verbatim}
1     wp=ws    $f wff s
2        wp=wr    $f wff r
3        wq=wp    $f wff p
4     wq=w2    $a wff ( r -> p )
5  wnew=w2  $a wff ( s -> ( r -> p ) )
\end{verbatim}
The number to the left of each line is the step number.  Following it is a
{\bf hypothesis association}\index{hypothesis association}, consisting of two
labels\index{label} separated by \texttt{=}.  To the left of the \texttt{=} (except
in the last step) is the label of a hypothesis of an assertion referenced
later in the proof; here, steps 1 and 4 are the hypothesis associations for
the assertion \texttt{w2} that is referenced in step 5.  A hypothesis association
is indented one level more than the assertion that uses it, so it is easy to
find the corresponding assertion by moving directly down until the indentation
level decreases to one less than where you started from.  To the right of each
\texttt{=} is the proof step label for that proof step.  The statement keyword of
the proof step label is listed next, followed by the content of the top of the
stack (the most recent stack entry) as it exists after that proof step is
processed.  With a little practice, you should have no trouble reading proofs
displayed in this format.

Metamath proofs include the syntax construction of a formula.
In standard mathematics, this kind of
construction is not considered a proper part of the proof at all, and it
certainly becomes rather boring after a while.
Therefore,
by default the \texttt{show proof}\index{\texttt{show proof}
command} command does not show the syntax construction.
Historically \texttt{show proof} command
\textit{did} show the syntax construction, and you needed to add the
\texttt{/essential} option to hide, them, but today
\texttt{/essential} is the default and you need to use
\texttt{/all} to see the syntax constructions.

When verifying a proof, Metamath will check that no mandatory
\texttt{\$d}\index{\texttt{\$d} statement}\index{mandatory \texttt{\$d}
statement} statement of an assertion referenced in a proof is violated
when substitutions\index{substitution!variable}\index{variable
substitution} are made to the variables in the assertion.  For details
see Section~\ref{spec4} or \ref{dollard}.

\subsection{The Concept of Unification} \label{unify}

During the course of verifying a proof, when Metamath\index{Metamath}
encounters an assertion label\index{assertion label}, it associates the
mandatory hypotheses\index{mandatory hypothesis} of the assertion with the top
entries of the RPN stack\index{stack}\index{RPN stack}.  Metamath then
determines what substitutions\index{substitution!variable}\index{variable
substitution} it must make to the variables in the assertion's mandatory
hypotheses in order for these hypotheses to become identical to their
corresponding stack entries.  This process is called {\bf
unification}\index{unification}.  (We also informally use the term
``unification'' to refer to a set of substitutions that results from the
process, as in ``two unifications are possible.'')  After the substitutions
are made, the hypotheses are said to be {\bf unified}.

If no such substitutions are possible, Metamath will consider the proof
incorrect and notify you with an error message.
% (deleted 3/10/07, per suggestion of Mel O'Cat:)
% The syntax of the
% Metamath language ensures that if a set of substitutions exists, it
% will be unique.

The general algorithm for unification described in the literature is
somewhat complex.
However, in the case of Metamath it is intentionally trivial.
Mandatory hypotheses must be
pushed on the proof stack in the order in which they appear.
In addition, each variable must have its type specified
with a \texttt{\$f} hypothesis before it is used
and that each \texttt{\$f} hypothesis
have the restricted syntax of a typecode (a constant) followed by a variable.
The typecode in the \texttt{\$f} hypothesis must match the first symbol of
the corresponding RPN stack entry (which will also be a constant), so
the only possible match for the variable in the \texttt{\$f} hypothesis is
the sequence of symbols in the stack entry after the initial constant.

In the Proof Assistant\index{Proof Assistant}, a more general unification
algorithm is used.  While a proof is being developed, sometimes not enough
information is available to determine a unique unification.  In this case
Metamath will ask you to pick the correct one.\index{ambiguous
unification}\index{unification!ambiguous}

\section{Extensions to the Metamath Language}\index{extended
language}

\subsection{Comments in the Metamath Language}\label{comments}
\index{markup notation}
\index{comments!markup notation}

The commenting feature allows you to annotate the contents of
a database.  Just as with most
computer languages, comments are ignored for the purpose of interpreting the
contents of the database. Comments effectively act as
additional white space\index{white
space} between tokens
when a database is parsed.

A comment may be placed at the beginning, end, or
between any two tokens\index{token} in a source file.

Comments have the following syntax:
\begin{center}
 \texttt{\$(} {\em text} \texttt{\$)}
\end{center}
Here,\index{\texttt{\$(} and \texttt{\$)} auxiliary
keywords}\index{comment} {\em text} is a string, possibly empty, of any
characters in Metamath's character set (p.~\pageref{spec1chars}), except
that the character strings \texttt{\$(} and \texttt{\$)} may not appear
in {\em text}.  Thus nested comments are not
permitted:\footnote{Computer languages have differing standards for
nested comments, and rather than picking one it was felt simplest not to
allow them at all, at least in the current version of
Metamath\index{Metamath!limitations of version 0.07.30}.} Metamath will
complain if you give it
\begin{center}
 \texttt{\$( This is a \$( nested \$) comment.\ \$)}
\end{center}
To compensate for this non-nesting behavior, I often change all \texttt{\$}'s
to \texttt{@}'s in sections of Metamath code I wish to comment out.

The Metamath program supports a number of markup mechanisms and conventions
to generate good-looking results in \LaTeX\ and {\sc html},
as discussed below.
These markup features have to do only with how the comments are typeset,
and have no effect on how Metamath verifies the proofs in the database.
The improper
use of them may result in incorrectly typeset output, but no Metamath
error messages will result during the \texttt{read} and \texttt{verify
proof} commands.  (However, the \texttt{write
theorem\texttt{\char`\_}list} command
will check for markup errors as a side-effect of its
{\sc html} generation.)
Section~\ref{texout} has instructions for creating \LaTeX\ output, and
section~\ref{htmlout} has instructions for creating
{\sc html}\index{HTML} output.

\subsubsection{Headings}\label{commentheadings}

If the \texttt{\$(} is immediately followed by a new line
starting with a heading marker, it is a header.
This can start with:

\begin{itemize}
 \item[] \texttt{\#\#\#\#} - major part header
 \item[] \texttt{\#*\#*} - section header
 \item[] \texttt{=-=-} - subsection header
 \item[] \texttt{-.-.} - subsubsection header
\end{itemize}

The line following the marker line
will be used for the table of contents entry, after trimming spaces.
The next line should be another (closing) matching marker line.
Any text after that
but before the closing \texttt{\$}, such as an extended description of the
section, will be included on the \texttt{mmtheoremsNNN.html} page.

For more information, run
\texttt{help write theorem\char`\_list}.

\subsubsection{Math mode}
\label{mathcomments}
\index{\texttt{`} inside comments}
\index{\texttt{\char`\~} inside comments}
\index{math mode}

Inside of comments, a string of tokens\index{token} enclosed in
grave accents\index{grave accent (\texttt{`})} (\texttt{`}) will be converted
to standard mathematical symbols during
{\sc HTML}\index{HTML} or \LaTeX\ output
typesetting,\index{latex@{\LaTeX}} according to the information in the
special \texttt{\$t}\index{\texttt{\$t} comment}\index{typesetting
comment} comment in the database
(see section~\ref{tcomment} for information about the typesetting
comment, and Appendix~\ref{ASCII} to see examples of its results).

The first grave accent\index{grave accent (\texttt{`})} \texttt{`}
causes the output processor to enter {\bf math mode}\index{math mode}
and the second one exits it.
In this
mode, the characters following the \texttt{`} are interpreted as a
sequence of math symbol tokens separated by white space\index{white
space}.  The tokens are looked up in the \texttt{\$t}
comment\index{\texttt{\$t} comment}\index{typesetting comment} and if
found, they will be replaced by the standard mathematical symbols that
they correspond to before being placed in the typeset output file.  If
not found, the symbol will be output as is and a warning will be issued.
The tokens do not have to be active in the database, although a warning
will be issued if they are not declared with \texttt{\$c} or
\texttt{\$v} statements.

Two consecutive
grave accents \texttt{``} are treated as a single actual grave accent
(both inside and outside of math mode) and will not cause the output
processor to enter or exit math mode.

Here is an example of its use\index{Pierce's axiom}:
\begin{center}
\texttt{\$( Pierce's axiom, ` ( ( ph -> ps ) -> ph ) -> ph ` ,\\
         is not very intuitive. \$)}
\end{center}
becomes
\begin{center}
   \texttt{\$(} Pierce's axiom, $((\varphi \rightarrow \psi)\rightarrow
\varphi)\rightarrow \varphi$, is not very intuitive. \texttt{\$)}
\end{center}

Note that the math symbol tokens\index{token} must be surrounded by white
space\index{white space}.
%, since there is no context that allows ambiguity to be
%resolved, as is the case with math symbol sequences in some of the Metamath
%statements.
White space should also surround the \texttt{`}
delimiters.

The math mode feature also gives you a quick and easy way to generate
text containing mathematical symbols, independently of the intended
purpose of Metamath.\index{Metamath!using as a math editor} To do this,
simply create your text with grave accents surrounding your formulas,
after making sure that your math symbols are mapped to \LaTeX\ symbols
as described in Appendix~\ref{ASCII}.  It is easier if you start with a
database with predefined symbols such as \texttt{set.mm}.  Use your
grave-quoted math string to replace an existing comment, then typeset
the statement corresponding to that comment following the instructions
from the \texttt{help tex} command in the Metamath program.  You will
then probably want to edit the resulting file with a text editor to fine
tune it to your exact needs.

\subsubsection{Label Mode}\index{label mode}

Outside of math mode, a tilde\index{tilde (\texttt{\char`\~})} \verb/~/
indicates to Metamath's\index{Metamath} output processor that the
token\index{token} that follows (i.e.\ the characters up to the next
white space\index{white space}) represents a statement label or URL.
This formatting mode is called {\bf label mode}\index{label mode}.
If a literal tilde
is desired (outside of math mode) instead of label mode,
use two tildes in a row to represent it.

When generating a \LaTeX\ output file,
the following token will be formatted in \texttt{typewriter}
font, and the tilde removed, to make it stand out from the rest of the text.
This formatting will be applied to all characters after the
tilde up to the first white space\index{white space}.
Whether
or not the token is an actual statement label is not checked, and the
token does not have to have the correct syntax for a label; no error
messages will be produced.  The only effect of the label mode on the
output is that typewriter font will be used for the tokens that are
placed in the \LaTeX\ output file.

When generating {\sc html},
the tokens after the tilde {\em must} be a URL (either http: or https:)
or a valid label.
Error messages will be issued during that output if they aren't.
A hyperlink will be generated to that URL or label.

\subsubsection{Link to bibliographical reference}\index{citation}%
\index{link to bibliographical reference}

Bibliographical references are handled specially when generating
{\sc html} if formatted specially.
Text in the form \texttt{[}{\em author}\texttt{]}
is considered a link to a bibliographical reference.
See \texttt{help html} and \texttt{help write
bibliography} in the Metamath program for more
information.
% \index{\texttt{\char`\[}\ldots\texttt{]} inside comments}
See also Sections~\ref{tcomment} and \ref{wrbib}.

The \texttt{[}{\em author}\texttt{]} notation will also create an entry in
the bibliography cross-reference file generated by \texttt{write
bibliography} (Section~\ref{wrbib}) for {\sc HTML}.
For this to work properly, the
surrounding comment must be formatted as follows:
\begin{quote}
    {\em keyword} {\em label} {\em noise-word}
     \texttt{[}{\em author}\texttt{] p.} {\em number}
\end{quote}
for example
\begin{verbatim}
     Theorem 5.2 of [Monk] p. 223
\end{verbatim}
The {\em keyword} is not case sensitive and must be one of the following:
\begin{verbatim}
     theorem lemma definition compare proposition corollary
     axiom rule remark exercise problem notation example
     property figure postulate equation scheme chapter
\end{verbatim}
The optional {\em label} may consist of more than one
(non-{\em keyword} and non-{\em noise-word}) word.
The optional {\em noise-word} is one of:
\begin{verbatim}
     of in from on
\end{verbatim}
and is  ignored when the cross-reference file is created.  The
\texttt{write
biblio\-graphy} command will perform error checking to verify the
above format.\index{error checking}

\subsubsection{Parentheticals}\label{parentheticals}

The end of a comment may include one or more parenthicals, that is,
statements enclosed in parentheses.
The Metamath program looks for certain parentheticals and can issue
warnings based on them.
They are:

\begin{itemize}
 \item[] \texttt{(Contributed by }
   \textit{NAME}\texttt{,} \textit{DATE}\texttt{.)} -
   document the original contributor's name and the date it was created.
 \item[] \texttt{(Revised by }
   \textit{NAME}\texttt{,} \textit{DATE}\texttt{.)} -
   document the contributor's name and creation date
   that resulted in significant revision
   (not just an automated minimization or shortening).
 \item[] \texttt{(Proof shortened by }
   \textit{NAME}\texttt{,} \textit{DATE}\texttt{.)} -
   document the contributor's name and date that developed a significant
   shortening of the proof (not just an automated minimization).
 \item[] \texttt{(Proof modification is discouraged.)} -
   Note that this proof should normally not be modified.
 \item[] \texttt{(New usage is discouraged.)} -
   Note that this assertion should normally not be used.
\end{itemize}

The \textit{DATE} must be in form YYYY-MMM-DD, where MMM is the
English abbreviation of that month.

\subsubsection{Other markup}\label{othermarkup}
\index{markup notation}

There are other markup notations for generating good-looking results
beyond math mode and label mode:

\begin{itemize}
 \item[]
         \texttt{\char`\_} (underscore)\index{\texttt{\char`\_} inside comments} -
             Italicize text starting from
              {\em space}\texttt{\char`\_}{\em non-space} (i.e.\ \texttt{\char`\_}
              with a space before it and a non-space character after it) until
             the next
             {\em non-space}\texttt{\char`\_}{\em space}.  Normal
             punctuation (e.g.\ a trailing
             comma or period) is ignored when determining {\em space}.
 \item[]
         \texttt{\char`\_} (underscore) - {\em
         non-space}\texttt{\char`\_}{\em non-space-string}, where
          {\em non-space-string} is a string of non-space characters,
         will make {\em non-space-string} become a subscript.
 \item[]
         \texttt{<HTML>}...\texttt{</HTML>} - do not convert
         ``\texttt{<}'' and ``\texttt{>}''
         in the enclosed text when generating {\sc HTML},
         otherwise process markup normally. This allows direct insertion
         of {\sc html} commands.
 \item[]
       ``\texttt{\&}ref\texttt{;}'' - insert an {\sc HTML}
         character reference.
         This is how to insert arbitrary Unicode characters
         (such as accented characters).  Currently only directly supported
         when generating {\sc HTML}.
\end{itemize}

It is recommended that spaces surround any \texttt{\char`\~} and
\texttt{`} tokens in the comment and that a space follow the {\em label}
after a \texttt{\char`\~} token.  This will make global substitutions
to change labels and symbol names much easier and also eliminate any
future chance of ambiguity.  Spaces around these tokens are automatically
removed in the final output to conform with normal rules of punctuation;
for example, a space between a trailing \texttt{`} and a left parenthesis
will be removed.

A good way to become familiar with the markup notation is to look at
the extensive examples in the \texttt{set.mm} database.

\subsection{The Typesetting Comment (\texttt{\$t})}\label{tcomment}

The typesetting comment \texttt{\$t} in the input database file
provides the information necessary to produce good-looking results.
It provides \LaTeX\ and {\sc html}
definitions for math symbols,
as well supporting as some
customization of the generated web page.
If you add a new token to a database, you should also
update the \texttt{\$t} comment information if you want to eventually
create output in \LaTeX\ or {\sc HTML}.
See the
\texttt{set.mm}\index{set theory database (\texttt{set.mm})} database
file for an extensive example of a \texttt{\$t} comment illustrating
many of the features described below.

Programs that do not need to generate good-looking presentation results,
such as programs that only verify Metamath databases,
can completely ignore typesetting comments
and just treat them as normal comments.
Even the Metamath program only consults the
\texttt{\$t} comment information when it needs to generate typeset output
in \LaTeX\ or {\sc HTML}
(e.g., when you open a \LaTeX\ output file with the \texttt{open tex} command).

We will first discuss the syntax of typesetting comments, and then
briefly discuss how this can be used within the Metamath program.

\subsubsection{Typesetting Comment Syntax Overview}

The typesetting comment is identified by the token
\texttt{\$t}\index{\texttt{\$t} comment}\index{typesetting comment} in
the comment, and the typesetting comment ends at the matching
\texttt{\$)}:
\[
  \mbox{\tt \$(\ }
  \mbox{\tt \$t\ }
  \underbrace{
    \mbox{\tt \ \ \ \ \ \ \ \ \ \ \ }
    \cdots
    \mbox{\tt \ \ \ \ \ \ \ \ \ \ \ }
  }_{\mbox{Typesetting definitions go here}}
  \mbox{\tt \ \$)}
\]

There must be one or more white space characters, and only white space
characters, between the \texttt{\$(} that starts the comment
and the \texttt{\$t} symbol,
and the \texttt{\$t} must be followed by one
or more white space characters
(see section \ref{whitespace} for the definition of white space characters).
The typesetting comment continues until the comment end token \texttt{\$)}
(which must be preceded by one or more white space characters).

In version 0.07.30\index{Metamath!limitations of version 0.07.30} of the
Metamath program, there may be only one \texttt{\$t} comment in a
database.  This restriction may be lifted in the future to allow
many \texttt{\$t} comments in a database.

Between the \texttt{\$t} symbol (and its following white space) and the
comment end token \texttt{\$)} (and its preceding white space)
is a sequence of one or more typesetting definitions, where
each definition has the form
\textit{definition-type arg arg ... ;}.
Each of the zero or more \textit{arg} values
can be either a typesetting data or a keyword
(what keywords are allowed, and where, depends on the specific
\textit{definition-type}).
The \textit{definition-type}, and each argument \textit{arg},
are separated by one or more white space characters.
Every definition ends in an unquoted semicolon;
white space is not required before the terminating semicolon of a definition.
Each definition should start on a new line.\footnote{This
restriction of the current version of Metamath
(0.07.30)\index{Metamath!limitations of version 0.07.30} may be removed
in a future version, but you should do it anyway for readability.}

For example, this typesetting definition:
\begin{center}
 \verb$latexdef "C_" as "\subseteq";$
\end{center}
defines the token \verb$C_$ as the \LaTeX\ symbol $\subseteq$ (which means
``subset'').

Typesetting data is a sequence of one or more quoted strings
(if there is more than one, they are connected by \texttt{\char`\+}).
Often a single quoted string is used to provide data for a definition, using
either double (\texttt{\char`\"}) or single (\texttt{'}) quotation marks.
However,
{\em a quoted string (enclosed in quotation marks) may not include
line breaks.}
A quoted string
may include a quotation mark that matches the enclosing quotes by repeating
the quotation mark twice.  Here are some examples:

\begin{tabu}   { l l }
\textbf{Example} & \textbf{Meaning} \\
\texttt{\char`\"a\char`\"\char`\"b\char`\"} & \texttt{a\char`\"b} \\
\texttt{'c''d'} & \texttt{c'd} \\
\texttt{\char`\"e''f\char`\"} & \texttt{e''f} \\
\texttt{'g\char`\"\char`\"h'} & \texttt{g\char`\"\char`\"h} \\
\end{tabu}

Finally, a long quoted string
may be broken up into multiple quoted strings (considered, as a whole,
a single quoted string) and joined with \texttt{\char`\+}.
You can even use multiple lines as long as a
'+' is at the end of every line except the last one.
The \texttt{\char`\+} should be preceded and followed by at least one
white space character.
Thus, for example,
\begin{center}
 \texttt{\char`\"ab\char`\"\ \char`\+\ \char`\"cd\char`\"
    \ \char`\+\ \\ 'ef'}
\end{center}
is the same as
\begin{center}
 \texttt{\char`\"abcdef\char`\"}
\end{center}

{\sc c}-style comments \texttt{/*}\ldots\texttt{*/} are also supported.

In practice, whenever you add a new math token you will often want to add
typesetting definitions using
\texttt{latexdef}, \texttt{htmldef}, and
\texttt{althtmldef}, as described below.
That way, they will all be up to date.
Of course, whether or not you want to use all three definitions will
depend on how the database is intended to be used.

Below we discuss the different possible \textit{definition-kind} options.
We will show data surrounded by double quotes (in practice they can also use
single quotes and/or be a sequence joined by \texttt{+}s).
We will use specific names for the \textit{data} to make clear what
the data is used for, such as
{\em math-token} (for a Metamath math token,
{\em latex-string} (for string to be placed in a \LaTeX\ stream),
{\em {\sc html}-code} (for {\sc html} code),
and {\em filename} (for a filename).

\subsubsection{Typesetting Comment - \LaTeX}

The syntax for a \LaTeX\ definition is:
\begin{center}
 \texttt{latexdef "}{\em math-token}\texttt{" as "}{\em latex-string}\texttt{";}
\end{center}
\index{latex definitions@\LaTeX\ definitions}%
\index{\texttt{latexdef} statement}

The {\em token-string} and {\em latex-string} are the data
(character strings) for
the token and the \LaTeX\ definition of the token, respectively,

These \LaTeX\ definitions are used by the Metamath program
when it is asked to product \LaTeX output using
the \texttt{write tex} command.

\subsubsection{Typesetting Comment - {\sc html}}

The key kinds of {\sc HTML} definitions have the following syntax:

\vskip 1ex
    \texttt{htmldef "}{\em math-token}\texttt{" as "}{\em
    {\sc html}-code}\texttt{";}\index{\texttt{htmldef} statement}
                    \ \ \ \ \ \ldots

    \texttt{althtmldef "}{\em math-token}\texttt{" as "}{\em
{\sc html}-code}\texttt{";}\index{\texttt{althtmldef} statement}

                    \ \ \ \ \ \ldots

Note that in {\sc HTML} there are two possible definitions for math tokens.
This feature is useful when
an alternate representation of symbols is desired, for example one that
uses Unicode entities and another uses {\sc gif} images.

There are many other typesetting definitions that can control {\sc HTML}.
These include:

\vskip 1ex

    \texttt{htmldef "}{\em math-token}\texttt{" as "}{\em {\sc
    html}-code}\texttt{";}

    \texttt{htmltitle "}{\em {\sc html}-code}\texttt{";}%
\index{\texttt{htmltitle} statement}

    \texttt{htmlhome "}{\em {\sc html}-code}\texttt{";}%
\index{\texttt{htmlhome} statement}

    \texttt{htmlvarcolor "}{\em {\sc html}-code}\texttt{";}%
\index{\texttt{htmlvarcolor} statement}

    \texttt{htmlbibliography "}{\em filename}\texttt{";}%
\index{\texttt{htmlbibliography} statement}

\vskip 1ex

\noindent The \texttt{htmltitle} is the {\sc html} code for a common
title, such as ``Metamath Proof Explorer.''  The \texttt{htmlhome} is
code for a link back to the home page.  The \texttt{htmlvarcolor} is
code for a color key that appears at the bottom of each proof.  The file
specified by {\em filename} is an {\sc html} file that is assumed to
have a \texttt{<A NAME=}\ldots\texttt{>} tag for each bibiographic
reference in the database comments.  For example, if
\texttt{[Monk]}\index{\texttt{\char`\[}\ldots\texttt{]} inside comments}
occurs in the comment for a theorem, then \texttt{<A NAME='Monk'>} must
be present in the file; if not, a warning message is given.

Associated with
\texttt{althtmldef}
are the statements
\vskip 1ex

    \texttt{htmldir "}{\em
      directoryname}\texttt{";}\index{\texttt{htmldir} statement}

    \texttt{althtmldir "}{\em
     directoryname}\texttt{";}\index{\texttt{althtmldir} statement}

\vskip 1ex
\noindent giving the directories of the {\sc gif} and Unicode versions
respectively; their purpose is to provide cross-linking between the
two versions in the generated web pages.

When two different types of pages need to be produced from a single
database, such as the Hilbert Space Explorer that extends the Metamath
Proof Explorer, ``extended'' variables may be declared in the
\texttt{\$t} comment:
\vskip 1ex

    \texttt{exthtmltitle "}{\em {\sc html}-code}\texttt{";}%
\index{\texttt{exthtmltitle} statement}

    \texttt{exthtmlhome "}{\em {\sc html}-code}\texttt{";}%
\index{\texttt{exthtmlhome} statement}

    \texttt{exthtmlbibliography "}{\em filename}\texttt{";}%
\index{\texttt{exthtmlbibliography} statement}

\vskip 1ex
\noindent When these are declared, you also must declare
\vskip 1ex

    \texttt{exthtmllabel "}{\em label}\texttt{";}%
\index{\texttt{exthtmllabel} statement}

\vskip 1ex \noindent that identifies the database statement where the
``extended'' section of the database starts (in our example, where the
Hilbert Space Explorer starts).  During the generation of web pages for
that starting statement and the statements after it, the {\sc html} code
assigned to \texttt{exthtmltitle} and \texttt{exthtmlhome} is used
instead of that assigned to \texttt{htmltitle} and \texttt{htmlhome},
respectively.

\begin{sloppy}
\subsection{Additional Information Com\-ment (\texttt{\$j})} \label{jcomment}
\end{sloppy}

The additional information comment, aka the
\texttt{\$j}\index{\texttt{\$j} comment}\index{additional information comment}
comment,
provides a way to add additional structured information that can
be optionally parsed by systems.

The additional information comment is parsed the same way as the
typesetting comment (\texttt{\$t}) (see section \ref{tcomment}).
That is,
the additional information comment begins with the token
\texttt{\$j} within a comment,
and continues until the comment close \texttt{\$)}.
Within an additional information comment is a sequence of one or more
commands of the form \texttt{command arg arg ... ;}
where each of the zero or more \texttt{arg} values
can be either a quoted string or a keyword.
Note that every command ends in an unquoted semicolon.
If a verifier is parsing an additional information comment, but
doesn't recognize a particular command, it must skip the command
by finding the end of the command (an unquoted semicolon).

A database may have 0 or more additional information comments.
Note, however, that a verifier may ignore these comments entirely or only
process certain commands in an additional information comment.
The \texttt{mmj2} verifier supports many commands in additional information
comments.
We encourage systems that process additional information comments
to coordinate so that they will use the same command for the same effect.

Examples of additional information comments with various commands
(from the \texttt{set.mm} database) are:

\begin{itemize}
   \item Define the syntax and logical typecodes,
     and declare that our grammar is
     unambiguous (verifiable using the KLR parser, with compositing depth 5).
\begin{verbatim}
  $( $j
    syntax 'wff';
    syntax '|-' as 'wff';
    unambiguous 'klr 5';
  $)
\end{verbatim}

   \item Register $\lnot$ and $\rightarrow$ as primitive expressions
           (lacking definitions).
\begin{verbatim}
  $( $j primitive 'wn' 'wi'; $)
\end{verbatim}

   \item There is a special justification for \texttt{df-bi}.
\begin{verbatim}
  $( $j justification 'bijust' for 'df-bi'; $)
\end{verbatim}

   \item Register $\leftrightarrow$ as an equality for its type (wff).
\begin{verbatim}
  $( $j
    equality 'wb' from 'biid' 'bicomi' 'bitri';
    definition 'dfbi1' for 'wb';
  $)
\end{verbatim}

   \item Theorem \texttt{notbii} is the congruence law for negation.
\begin{verbatim}
  $( $j congruence 'notbii'; $)
\end{verbatim}

   \item Add \texttt{setvar} as a typecode.
\begin{verbatim}
  $( $j syntax 'setvar'; $)
\end{verbatim}

   \item Register $=$ as an equality for its type (\texttt{class}).
\begin{verbatim}
  $( $j equality 'wceq' from 'eqid' 'eqcomi' 'eqtri'; $)
\end{verbatim}

\end{itemize}


\subsection{Including Other Files in a Metamath Source File} \label{include}
\index{\texttt{\$[} and \texttt{\$]} auxiliary keywords}

The keywords \texttt{\$[} and \texttt{\$]} specify a file to be
included\index{included file}\index{file inclusion} at that point in a
Metamath\index{Metamath} source file\index{source file}.  The syntax for
including a file is as follows:
\begin{center}
\texttt{\$[} {\em file-name} \texttt{\$]}
\end{center}

The {\em file-name} should be a single token\index{token} with the same syntax
as a math symbol (i.e., all 93 non-whitespace
printable characters other than \texttt{\$} are
allowed, subject to the file-naming limitations of your operating system).
Comments may appear between the \texttt{\$[} and \texttt{\$]} keywords.  Included
files may include other files, which may in turn include other files, and so
on.

For example, suppose you want to use the set theory database as the starting
point for your own theory.  The first line in your file could be
\begin{center}
\texttt{\$[ set.mm \$]}
\end{center} All of the information (axioms, theorems,
etc.) in \texttt{set.mm} and any files that {\em it} includes will become
available for you to reference in your file. This can help make your work more
modular. A drawback to including files is that if you change the name of a
symbol or the label of a statement, you must also remember to update any
references in any file that includes it.


The naming conventions for included files are the same as those of your
operating system.\footnote{On the Macintosh, prior to Mac OS X,
 a colon is used to separate disk
and folder names from your file name.  For example, {\em volume}\texttt{:}{\em
file-name} refers to the root directory, {\em volume}\texttt{:}{\em
folder-name}\texttt{:}{\em file-name} refers to a folder in root, and {\em
volume}\texttt{:}{\em folder-name}\texttt{:}\ldots\texttt{:}{\em file-name} refers to a
deeper folder.  A simple {\em file-name} refers to a file in the folder from
which you launch the Metamath application.  Under Mac OS X and later,
the Metamath program is run under the Terminal application, which
conforms to Unix naming conventions.}\index{Macintosh file
names}\index{file names!Macintosh}\label{includef} For compatibility among
operating systems, you should keep the file names as simple as possible.  A
good convention to use is {\em file}\texttt{.mm} where {\em file} is eight
characters or less, in lower case.

There is no limit to the nesting depth of included files.  One thing that you
should be aware of is that if two included files themselves include a common
third file, only the {\em first} reference to this common file will be read
in.  This allows you to include two or more files that build on a common
starting file without having to worry about label and symbol conflicts that
would occur if the common file were read in more than once.  (In fact, if a
file includes itself, the self-reference will be ignored, although of course
it would not make any sense to do that.)  This feature also means, however,
that if you try to include a common file in several inner blocks, the result
might not be what you expect, since only the first reference will be replaced
with the included file (unlike the include statement in most other computer
languages).  Thus you would normally include common files only in the
outermost block\index{outermost block}.

\subsection{Compressed Proof Format}\label{compressed1}\index{compressed
proof}\index{proof!compressed}

The proof notation presented in Section~\ref{proof} is called a
{\bf normal proof}\index{normal proof}\index{proof!normal} and in principle is
sufficient to express any proof.  However, proofs often contain steps and
subproofs that are identical.  This is particularly true in typical
Metamath\index{Metamath} applications, because Metamath requires that the math
symbol sequence (usually containing a formula) at each step be separately
constructed, that is, built up piece by piece. As a result, a lot of
repetition often results.  The {\bf compressed proof} format allows Metamath
to take advantage of this redundancy to shorten proofs.

The specification for the compressed proof format is given in
Appen\-dix~\ref{compressed}.

Normally you need not concern yourself with the details of the compressed
proof format, since the Metamath program will allow you to convert from
the normal format to the compressed format with ease, and will also
automatically convert from the compressed format when proofs are displayed.
The overall structure of the compressed format is as follows:
\begin{center}
  \texttt{\$= ( } {\em label-list} \texttt{) } {\em compressed-proof\ }\ \texttt{\$.}
\end{center}
\index{\texttt{\$=} keyword}
The first \texttt{(} serves as a flag to Metamath that a compressed proof
follows.  The {\em label-list} includes all statements referred to by the
proof except the mandatory hypotheses\index{mandatory hypothesis}.  The {\em
compressed-proof} is a compact encoding of the proof, using upper-case
letters, and can be thought of as a large integer in base 26.  White
space\index{white space} inside a {\em compressed-proof} is
optional and is ignored.

It is important to note that the order of the mandatory hypotheses of
the statement being proved must not be changed if the compressed proof
format is used, otherwise the proof will become incorrect.  The reason
for this is that the mandatory hypotheses are not mentioned explicitly
in the compressed proof in order to make the compression more efficient.
If you wish to change the order of mandatory hypotheses, you must first
convert the proof back to normal format using the \texttt{save proof
{\em statement} /normal}\index{\texttt{save proof} command} command.
Later, you can go back to compressed format with \texttt{save proof {\em
statement} /compressed}.

During error checking with the \texttt{verify proof} command, an error
found in a compressed proof may point to a character in {\em
compressed-proof}, which may not be very meaningful to you.  In this
case, try to \texttt{save proof /normal} first, then do the
\texttt{verify proof} again.  In general, it is best to make sure a
proof is correct before saving it in compressed format, because severe
errors are less likely to be recoverable than in normal format.

\subsection{Specifying Unknown Proofs or Subproofs}\label{unknown}

In a proof under development, any step or subproof that is not yet known
may be represented with a single \texttt{?}.  For the purposes of
parsing the proof, the \texttt{?}\ \index{\texttt{]}@\texttt{?}\ inside
proofs} will push a single entry onto the RPN stack just as if it were a
hypothesis.  While developing a proof with the Proof
Assistant\index{Proof Assistant}, a partially developed proof may be
saved with the \texttt{save new{\char`\_}proof}\index{\texttt{save
new{\char`\_}proof} command} command, and \texttt{?}'s will be placed at
the appropriate places.

All \texttt{\$p}\index{\texttt{\$p} statement} statements must have
proofs, even if they are entirely unknown.  Before creating a proof with
the Proof Assistant, you should specify a completely unknown proof as
follows:
\begin{center}
  {\em label} \texttt{\$p} {\em statement} \texttt{\$= ?\ \$.}
\end{center}
\index{\texttt{\$=} keyword}
\index{\texttt{]}@\texttt{?}\ inside proofs}

The \texttt{verify proof}\index{\texttt{verify proof} command} command
will check the known portions of a partial proof for errors, but will
warn you that the statement has not been proved.

Note that partially developed proofs may be saved in compressed format
if desired.  In this case, you will see one or more \texttt{?}'s in the
{\em compressed-proof} part.\index{compressed
proof}\index{proof!compressed}

\section{Axioms vs.\ Definitions}\label{definitions}

The \textit{basic}
Metamath\index{Metamath} language and program
make no distinction\index{axiom vs.\
definition} between axioms\index{axiom} and
definitions.\index{definition} The \texttt{\$a}\index{\texttt{\$a}
statement} statement is used for both.  At first, this may seem
puzzling.  In the minds of many mathematicians, the distinction is
clear, even obvious, and hardly worth discussing.  A definition is
considered to be merely an abbreviation that can be replaced by the
expression for which it stands; although unless one actually does this,
to be precise then one should say that a theorem\index{theorem} is a
consequence of the axioms {\em and} the definitions that are used in the
formulation of the theorem \cite[p.~20]{Behnke}.\index{Behnke, H.}

\subsection{What is a Definition?}

What is a definition?  In its simplest form, a definition introduces a new
symbol and provides an unambiguous rule to transform an expression containing
the new symbol to one without it.  The concept of a ``proper
definition''\index{proper definition}\index{definition!proper} (as opposed to
a creative definition)\index{creative definition}\index{definition!creative}
that is usually agreed upon is (1) the definition should not strengthen the
language and (2) any symbols introduced by the definition should be eliminable
from the language \cite{Nemesszeghy}\index{Nemesszeghy, E. Z.}.  In other
words, they are mere typographical conveniences that do not belong to the
system and are theoretically superfluous.  This may seem obvious, but in fact
the nature of definitions can be subtle, sometimes requiring difficult
metatheorems to establish that they are not creative.

A more conservative stance was taken by logician S.
Le\'{s}niewski.\index{Le\'{s}niewski, S.}
\begin{quote}
Le\'{s}niewski
regards definitions as theses of the system.  In this respect they do
not differ either from the axioms or from theorems, i.e.\ from the
theses added to the system on the basis of the rule of substitution or
the rule of detachment [modus ponens].  Once definitions have been
accepted as theses of the system, it becomes necessary to consider them
as true propositions in the same sense in which axioms are true
\cite{Lejewski}.
\end{quote}\index{Lejewski, Czeslaw}

Let us look at some simple examples of definitions in propositional
calculus.  Consider the definition of logical {\sc or}
(disjunction):\index{disjunction ($\vee$)} ``$P\vee Q$ denotes $\neg P
\rightarrow Q$ (not $P$ implies $Q$).''  It is very easy to recognize a
statement making use of this definition, because it introduces the new
symbol $\vee$ that did not previously exist in the language.  It is easy
to see that no new theorems of the original language will result from
this definition.

Next, consider a definition that eliminates parentheses:  ``$P
\rightarrow Q\rightarrow R$ denotes $P\rightarrow (Q \rightarrow R)$.''
This is more subtle, because no new symbols are introduced.  The reason
this definition is considered proper is that no new symbol sequences
that are valid wffs (well-formed formulas)\index{well-formed formula
(wff)} in the original language will result from the definition, since
``$P \rightarrow Q\rightarrow R$'' is not a wff in the original
language.  Here, we implicitly make use of the fact that there is a
decision procedure that allows us to determine whether or not a symbol
sequence is a wff, and this fact allows us to use symbol sequences that
are not wffs to represent other things (such as wffs) by means of the
definition.  However, to justify the definition as not being creative we
need to prove that ``$P \rightarrow Q\rightarrow R$'' is in fact not a
wff in the original language, and this is more difficult than in the
case where we simply introduce a new symbol.

%Now let's take this reasoning to an extreme.  Propositional calculus is a
%decidable theory,\footnote{This means that a mechanical algorithm exists to
%determine whether or not a wff is a theorem.} so in principle we could make use
%of symbol sequences that are not theorems to represent other things (say, to
%encode actual theorems in a more compact way).  For example, let us extend the
%language by defining a wff ``$P$'' in the extended language as the theorem
%``$P\rightarrow P$''\footnote{This is one of the first theorems proved in the
%Metamath database \texttt{set.mm}.}\index{set
%theory database (\texttt{set.mm})} in the original language whenever ``$P$'' is
%not a theorem in the original language.  In the extended language, any wff
%``$Q$'' thus represents a theorem; to find out what theorem (in the original
%language) ``$Q$'' represents, we determine whether ``$Q$'' is a theorem in the
%original language (before the definition was introduced).  If so, we're done; if
%not, we replace ``$Q$'' by ``$Q\rightarrow Q$'' to eliminate the definition.
%This definition is therefore eliminable, and it does not ``strengthen'' the
%language because any wff that is not a theorem is not in the set of statements
%provable in the original language and thus is available for use by definitions.
%
%Of course, a definition such as this would render practically useless the
%communication of theorems of propositional calculus; but
%this is just a human shortcoming, since we can't always easily discern what is
%and is not a theorem by inspection.  In fact, the extended theory with this
%definition has no more and no less information than the original theory; it just
%expresses certain theorems of the form ``$P\rightarrow P$''
%in a more compact way.
%
%The point here is that what constitutes a proper definition is a matter of
%judgment about whether a symbol sequence can easily be recognized by a human
%as invalid in some sense (for example, not a wff); if so, the symbol sequence
%can be appropriated for use by a definition in order to make the extended
%language more compact.  Metamath\index{Metamath} lacks the ability to make this
%judgment, since as far as Metamath is concerned the definition of a wff, for
%example, is arbitrary.  You define for Metamath how wffs\index{well-formed
%formula (wff)} are constructed according to your own preferred style.  The
%concept of a wff may not even exist in a given formal system\index{formal
%system}.  Metamath treats all definitions as if they were new axioms, and it
%is up to the human mathematician to judge whether the definition is ``proper''
%'\index{proper definition}\index{definition!proper} in some agreed-upon way.

What constitutes a definition\index{definition} versus\index{axiom vs.\
definition} an axiom\index{axiom} is sometimes arbitrary in mathematical
literature.  For example, the connectives $\vee$ ({\sc or}), $\wedge$
({\sc and}), and $\leftrightarrow$ (equivalent to) in propositional
calculus are usually considered defined symbols that can be used as
abbreviations for expressions containing the ``primitive'' connectives
$\rightarrow$ and $\neg$.  This is the way we treat them in the standard
logic and set theory database \texttt{set.mm}\index{set theory database
(\texttt{set.mm})}.  However, the first three connectives can also be
considered ``primitive,'' and axiom systems have been devised that treat
all of them as such.  For example,
\cite[p.~35]{Goodstein}\index{Goodstein, R. L.} presents one with 15
axioms, some of which in fact coincide with what we have chosen to call
definitions in \texttt{set.mm}.  In certain subsets of classical
propositional calculus, such as the intuitionist
fragment\index{intuitionism}, it can be shown that one cannot make do
with just $\rightarrow$ and $\neg$ but must treat additional connectives
as primitive in order for the system to make sense.\footnote{Two nice
systems that make the transition from intuitionistic and other weak
fragments to classical logic just by adding axioms are given in
\cite{Robinsont}\index{Robinson, T. Thacher}.}

\subsection{The Approach to Definitions in \texttt{set.mm}}

In set theory, recursive definitions define a newly introduced symbol in
terms of itself.  The justification of recursive definitions, using
several ``recursion theorems,'' is usually one of the first
sophisticated proofs a student encounters when learning set theory, and
there is a significant amount of implicit metalogic behind a recursive
definition even though the definition itself is typically simple to
state.  It is, however, possible to substitute one kind of complexity
for another.  We can eliminate the need for metalogical justification by
defining the operation directly with an explicit (but complicated)
expression, then deriving the recursive definition directly as a
theorem, using a recursion theorem ``in reverse.''  We do this in
\texttt{set.mm}, as follows.

In \texttt{set.mm} our goal was to introduce almost all definitions in
the form of two expressions connected by either $\leftrightarrow$ or
$=$, where the thing being defined does not appear on the right hand
side.  Quine calls this form ``a genuine or direct definition'' \cite[p.
174]{Quine}\index{Quine, Willard Van Orman}, which makes the definitions
very easy to eliminate and the metalogic\index{metalogic} needed to
justify them as simple as possible.

We achieved this goal in almost all cases in \texttt{set.mm}.
Sometimes this makes the definitions more complex and less
intuitive.  For example, the traditional way to define addition of
natural numbers is to define an operation called {\em
successor}\index{successor} (which means ``plus one'' and is denoted by
``${\rm suc}$''), then define addition recursively\index{recursive
definition} with the two definitions $n + 0 = n$ and $m + {\rm suc}\,n =
{\rm suc} (m + n)$.  Although this definition seems simple and obvious,
the method to eliminate the definition is not obvious:  in the second
part of the definition, addition is defined in terms of itself.  By
eliminating the definition, we don't mean repeatedly applying it to
specific $m$ and $n$ but rather showing the explicit, closed-form
set-theoretical expression that $m + n$ represents, that will work for
any $m$ and $n$ and that does not have a $+$ sign on its right-hand
side.  For a recursive definition like this not to be circular
(creative), there are some hidden, underlying assumptions we must make,
for example that the natural numbers have a certain kind of order.  In
\texttt{set.mm} we chose to start with the direct (though complex and
nonintuitive) definition then derive from it the standard recursive
definition.\footnote{The closed-form definition used in \texttt{set.mm}
for the addition operation on ordinals\index{ordinal
addition}\index{addition!of ordinals} (of which natural numbers are a
subset) is

\setbox\startprefix=\hbox{\tt \ \ df-oadd\ \$a\ }
\setbox\contprefix=\hbox{\tt \ \ \ \ \ \ \ \ \ \ \ \ \ }
\startm
\m{\vdash}\m{+_o}\m{=}\m{\{}\m{\langle}\m{\langle}\m{x}\m{,}\m{y}\m{\rangle}%
\m{,}\m{z}\m{\rangle}\m{|}\m{(}\m{(}\m{x}\m{\in}\m{{\rm On}}\m{\wedge}\m{y}\m{%
\in}\m{{\rm On}}\m{)}\m{\wedge}\m{z}\m{=}\m{(}\m{{\rm rec}}\m{(}\m{\{}\m{%
\langle}\m{w}\m{,}\m{v}\m{\rangle}\m{|}\m{v}\m{=}\m{{\rm suc}}\m{w}\m{\}}\m{,}%
\m{x}\m{)}\m{`}\m{y}\m{)}\m{)}\m{\}}
\endm

\noindent Here, the abstraction class of nested ordered pairs\index{abstraction
class!of nested ordered pairs} is defined by \texttt{df-oprab} in \texttt{set.mm},
and ${\rm rec}$ is a ``recursion operator''\index{recursion operator} with
the definition

\setbox\startprefix=\hbox{\tt \ \ df-rfg\ \$a\ }
\setbox\contprefix=\hbox{\tt \ \ \ \ \ \ \ \ \ \ \ \ }
\startm
\m{\vdash}\m{{\rm rec}}\m{(}\m{F}\m{,}\m{A}\m{)}\m{=}\m{\bigcup}\m{\{}\m{f}%
\m{|}\m{\exists}\m{x}\m{(}\m{x}\m{\in}\m{{\rm On}}\m{\wedge}\m{(}\m{f}\m{{%
\rm Fn}}\m{x}\m{\wedge}\m{\forall}\m{y}\m{(}\m{y}\m{\in}\m{x}\m{\rightarrow}%
\m{(}\m{f}\m{`}\m{y}\m{)}\m{=}\m{(}\m{\{}\m{\langle}\m{g}\m{,}\m{z}\m{\rangle}%
\m{|}\m{(}\m{(}\m{g}\m{=}\m{\varnothing}\m{\wedge}\m{z}\m{=}\m{A}\m{)}\m{\vee}%
\m{(}\m{\lnot}\m{(}\m{g}\m{=}\m{\varnothing}\m{\vee}\m{{\rm Lim}}\m{{\rm dom}}%
\m{g}\m{)}\m{\wedge}\m{z}\m{=}\m{(}\m{F}\m{`}\m{(}\m{g}\m{`}\m{\bigcup}\m{{%
\rm dom}}\m{g}\m{)}\m{)}\m{)}\m{\vee}\m{(}\m{{\rm Lim}}\m{{\rm dom}}\m{g}\m{%
\wedge}\m{z}\m{=}\m{\bigcup}\m{{\rm ran}}\m{g}\m{)}\m{)}\m{\}}\m{`}\m{(}\m{f}%
\m{\restriction}\m{y}\m{)}\m{)}\m{)}\m{)}\m{)}\m{\}}
\endm

\noindent which can be further broken down with definitions shown in
Section~\ref{setdefinitions}.  You may be surprised at the complexity of what
seems like such a simple notion.  From these definitions the simpler, more
intuitive recursive definition is derived as a set of theorems.}\index{natural
number}\index{addition}\index{recursive definition}\index{ordinal addition}
The end result is the same, but we completely eliminate the rather complex
metalogic that justifies the recursive definition.  (For a mathematician,
recursive definitions are more efficient and intuitive than
direct ones once the metalogic has been learned or possibly just accepted as
correct.  However, it was felt that direct definition in \texttt{set.mm}
maximizes rigor by minimizing metalogic.  It can be eliminated effortlessly,
something difficult to do with a recursive definition.)

\subsection{Adding Constraints on Definitions}

The basic Metamath language and the Metamath program do
not have any built-in constraints on definitions, since they are just
\$a statements.

However, nothing prevents a verification system from verifying
additional rules to impose further limitations on definitions.
For example, the \texttt{mmj2}\index{mmj2} program
supports various kinds of
additional information comments (see section \ref{jcomment}).
One of their uses is to optionally verify additional constraints,
including constraints to verify that definitions meet certain
requirements.
These additional checks are required by the
continuous integration (CI)\index{continuous integration (CI)}
checks of the
\texttt{set.mm}\index{set theory database (\texttt{set.mm})}%
\index{Metamath Proof Explorer}
database.
This approach enables us to optionally impose additional requirements
on definitions if we wish, without necessarily imposing those rules on
all databases or requiring all verification systems to implement them.
In addition, this allows us to impose specialized constraints tailored
to one database while not requiring other systems to implement
those specialized constraints.

We impose two constraints on the
\texttt{set.mm}\index{set theory database (\texttt{set.mm})}%
\index{Metamath Proof Explorer} database
via the \texttt{mmj2}\index{mmj2} program that are worth discussing here:
a parse check and a definition soundness check.

% On February 11, 2019 8:32:32 PM EST, saueran@oregonstate.edu wrote:
% The following addition to the end of set.mm is accepted by the mmj2
% parser and definition checker and the metamath verifier(at least it was
% when I checked, you should check it too), and creates a contradiction by
% proving the theorem |- ph.
% ${
% wleftp $a wff ( ( ph ) $.
% wbothp $a wff ( ph ) $.
% df-leftp $a |- ( ( ( ph ) <-> -. ph ) $.
% df-bothp $a |- ( ( ph ) <-> ph ) $.
% anything $p |- ph $=
%   ( wbothp wn wi wleftp df-leftp biimpi df-bothp mpbir mpbi simplim ax-mp)
%   ABZAMACZDZCZMOEZOCQAEZNDZRNAFGSHIOFJMNKLAHJ $.
% $}
%
% This particular problem is countered by enabling, within mmj2,
% SetParser,mmj.verify.LRParser

First,
we enable a parse check in \texttt{mmj2} (through its
\texttt{SetParser} command) that requires that all new definitions
must \textit{not} create an ambiguous parse for a KLR(5) parser.
This prevents some errors such as definitions with imbalanced parentheses.

Second, we run a definition soundness check specific to
\texttt{set.mm} or databases similar to it.
(through the \texttt{definitionCheck} macro).
Some \texttt{\$a} statements (including all ax-* statemnets)
are excluded from these checks, as they will
always fail this simple check,
but they are appropriate for most definitions.
This check imposes a set of additional rules:

\begin{enumerate}

\item New definitions must be introduced using $=$ or $\leftrightarrow$.

\item No \texttt{\$a} statement introduced before this one is allowed to use the
symbol being defined in this definition, and the definition is not
allowed to use itself (except once, in the definiendum).

\item Every variable in the definiens should not be distinct

\item Every dummy variable in the definiendum
are required to be distinct from each other and from variables in
the definiendum.
To determine this, the system will look for a "justification" theorem
in the database, and if it is not there, attempt to briefly prove
$( \varphi \rightarrow \forall x \varphi )$  for each dummy variable x.

\item Every dummy variable should be a set variable,
unless there is a justification theorem available.

\item Every dummy variable must be bound
(if the system cannot determine this a justification theorem must be
provided).

\end{enumerate}

\subsection{Summary of Approach to Definitions}

In short, when being rigorous it turns out that
definitions can be subtle, sometimes requiring difficult
metatheorems to establish that they are not creative.

Instead of building such complications into the Metamath language itself,
the basic Metmath language and program simply treat traditional
axioms and definitions as the same kind of \texttt{\$a} statement.
We have then built various tools to enable people to
verify additional conditions as their creators believe is appropriate
for those specific databases, without complicating the Metamath foundations.

\chapter{The Metamath Program}\label{commands}

This chapter provides a reference manual for the
Metamath program.\index{Metamath!commands}

Current instructions for obtaining and installing the Metamath program
can be found at the \url{http://metamath.org} web site.
For Windows, there is a pre-compiled version called
\texttt{metamath.exe}.  For Unix, Linux, and Mac OS X
(which we will refer to collectively as ``Unix''), the Metamath program
can be compiled from its source code with the command
\begin{verbatim}
gcc *.c -o metamath
\end{verbatim}
using the \texttt{gcc} {\sc c} compiler available on those systems.

In the command syntax descriptions below, fields enclosed in square
brackets [\ ] are optional.  File names may be optionally enclosed in
single or double quotes.  This is useful if the file name contains
spaces or
slashes (\texttt{/}), such as in Unix path names, \index{Unix file
names}\index{file names!Unix} that might be confused with Metamath
command qualifiers.\index{Unix file names}\index{file names!Unix}

\section{Invoking Metamath}

Unix, Linux, and Mac OS X
have a command-line interface called the {\em
bash shell}.  (In Mac OS X, select the Terminal application from
Applications/Utilities.)  To invoke Metamath from the bash shell prompt,
assuming that the Metamath program is in the current directory, type
\begin{verbatim}
bash$ ./metamath
\end{verbatim}

To invoke Metamath from a Windows DOS or Command Prompt, assuming that
the Metamath program is in the current directory (or in a directory
included in the Path system environment variable), type
\begin{verbatim}
C:\metamath>metamath
\end{verbatim}

To use command-line arguments at invocation, the command-line arguments
should be a list of Metamath commands, surrounded by quotes if they
contain spaces.  In Windows, the surrounding quotes must be double (not
single) quotes.  For example, to read the database file \texttt{set.mm},
verify all proofs, and exit the program, type (under Unix)
\begin{verbatim}
bash$ ./metamath 'read set.mm' 'verify proof *' exit
\end{verbatim}
Note that in Unix, any directory path with \texttt{/}'s must be
surrounded by quotes so Metamath will not interpret the \texttt{/} as a
command qualifier.  So if \texttt{set.mm} is in the \texttt{/tmp}
directory, use for the above example
\begin{verbatim}
bash$ ./metamath 'read "/tmp/set.mm"' 'verify proof *' exit
\end{verbatim}

For convenience, if the command-line has one argument and no spaces in
the argument, the command is implicitly assumed to be \texttt{read}.  In
this one special case, \texttt{/}'s are not interpreted as command
qualifiers, so you don't need quotes around a Unix file name.  Thus
\begin{verbatim}
bash$ ./metamath /tmp/set.mm
\end{verbatim}
and
\begin{verbatim}
bash$ ./metamath "read '/tmp/set.mm'"
\end{verbatim}
are equivalent.


\section{Controlling Metamath}

The Metamath program was first developed on a {\sc vax/vms} system, and
some aspects of its command line behavior reflect this heritage.
Hopefully you will find it reasonably user-friendly once you get used to
it.

Each command line is a sequence of English-like words separated by
spaces, as in \texttt{show settings}.  Command words are not case
sensitive, and only as many letters are needed as are necessary to
eliminate ambiguity; for example, \texttt{sh se} would work for the
command \texttt{show settings}.  In some cases arguments such as file
names, statement labels, or symbol names are required; these are
case-sensitive (although file names may not be on some operating
systems).

A command line is entered by typing it in then pressing the {\em return}
({\em enter}) key.  To find out what commands are available, type
\texttt{?} at the \texttt{MM>} prompt.  To find out the choices at any
point in a command, press {\em return} and you will be prompted for
them.  The default choice (the one selected if you just press {\em
return}) is shown in brackets (\texttt{<>}).

You may also type \texttt{?} in place of a command word to force
Metamath to tell you what the choices are.  The \texttt{?} method won't
work, though, if a non-keyword argument such as a file name is expected
at that point, because the program will think that \texttt{?} is the
value of the argument.

Some commands have one or more optional qualifiers which modify the
behavior of the command.  Qualifiers are preceded by a slash
(\texttt{/}), such as in \texttt{read set.mm / verify}.  Spaces are
optional around the \texttt{/}.  If you need to use a space or
slash in a command
argument, as in a Unix file name, put single or double quotes around the
command argument.

The \texttt{open log} command will save everything you see on the
screen and is useful to help you recover should something go wrong in a
proof, or if you want to document a bug.

If a command responds with more than a screenful, you will be
prompted to \texttt{<return> to continue, Q to quit, or S to scroll to
end}.  \texttt{Q} or \texttt{q} (not case-sensitive) will complete the
command internally but will suppress further output until the next
\texttt{MM>} prompt.  \texttt{s} will suppress further pausing until the
next \texttt{MM>} prompt.  After the first screen, you are also
presented with the choice of \texttt{b} to go back a screenful.  Note
that \texttt{b} may also be entered at the \texttt{MM>} prompt
immediately after a command to scroll back through the output of that
command.

A command line enclosed in quotes is executed by your operating system.
See Section~\ref{oscmd}.

{\em Warning:} Pressing {\sc ctrl-c} will abort the Metamath program
unconditionally.  This means any unsaved work will be lost.


\subsection{\texttt{exit} Command}\index{\texttt{exit} command}

Syntax:  \texttt{exit} [\texttt{/force}]

This command exits from Metamath.  If there have been changes to the
source with the \texttt{save proof} or \texttt{save new{\char`\_}proof}
commands, you will be given an opportunity to \texttt{write source} to
permanently save the changes.

In Proof Assistant\index{Proof Assistant} mode, the \texttt{exit} command will
return to the \verb/MM>/ prompt. If there were changes to the proof, you will
be given an opportunity to \texttt{save new{\char`\_}proof}.

The \texttt{quit} command is a synonym for \texttt{exit}.

Optional qualifier:
    \texttt{/force} - Do not prompt if changes were not saved.  This qualifier is
        useful in \texttt{submit} command files (Section~\ref{sbmt})
        to ensure predictable behavior.





\subsection{\texttt{open log} Command}\index{\texttt{open log} command}
Syntax:  \texttt{open log} {\em file-name}

This command will open a log file that will store everything you see on
the screen.  It is useful to help recovery from a mistake in a long Proof
Assistant session, or to document bugs.\index{Metamath!bugs}

The log file can be closed with \texttt{close log}.  It will automatically be
closed upon exiting Metamath.



\subsection{\texttt{close log} Command}\index{\texttt{close log} command}
Syntax:  \texttt{close log}

The \texttt{close log} command closes a log file if one is open.  See
also \texttt{open log}.




\subsection{\texttt{submit} Command}\index{\texttt{submit} command}\label{sbmt}
Syntax:  \texttt{submit} {\em filename}

This command causes further command lines to be taken from the specified
file.  Note that any line beginning with an exclamation point (\texttt{!}) is
treated as a comment (i.e.\ ignored).  Also note that the scrolling
of the screen output is continuous, so you may want to open a log file
(see \texttt{open log}) to record the results that fly by on the screen.
After the lines in the file are exhausted, Metamath returns to its
normal user interface mode.

Currently, the \texttt{submit} command is not recursive.  In other words,
\texttt{submit} commands are not allowed inside a command file.


\subsection{\texttt{erase} Command}\index{\texttt{erase} command}
Syntax:  \texttt{erase}

This command will reset Metamath to its starting state, deleting any
data\-base that was \texttt{read} in.
 If there have been changes to the
source with the \texttt{save proof} or \texttt{save new{\char`\_}proof}
commands, you will be given an opportunity to \texttt{write source} to
permanently save the changes.



\subsection{\texttt{set echo} Command}\index{\texttt{set echo} command}
Syntax:  \texttt{set echo on} or \texttt{set echo off}

The \texttt{set echo on} command will cause command lines to be echoed with any
abbreviations expanded.  While learning the Metamath commands, this
feature will show you the exact command that your abbreviated input
corresponds to.



\subsection{\texttt{set scroll} Command}\index{\texttt{set scroll} command}
Syntax:  \texttt{set scroll prompted} or \texttt{set scroll continuous}

The Metamath command line interface starts off in the \texttt{prompted} mode,
which means that you will be prompted to continue or quit after each
full screen in a long listing.  In \texttt{continuous} mode, long listings will be
scrolled without pausing.

% LaTeX bug? (1) \texttt{\_} puts out different character than
% \texttt{{\char`\_}}
%  = \verb$_$  (2) \texttt{{\char`\_}} puts out garbage in \subsection
%  argument
\subsection{\texttt{set width} Command}\index{\texttt{set
width} command}
Syntax:  \texttt{set width} {\em number}

Metamath assumes the width of your screen is 79 characters (chosen
because the Command Prompt in Windows XP has a wrapping bug at column
80).  If your screen is wider or narrower, this command allows you to
change this default screen width.  A larger width is advantageous for
logging proofs to an output file to be printed on a wide printer.  A
smaller width may be necessary on some terminals; in this case, the
wrapping of the information messages may sometimes seem somewhat
unnatural, however.  In \LaTeX\index{latex@{\LaTeX}!characters per line},
there is normally a maximum of 61 characters per line with typewriter
font.  (The examples in this book were produced with 61 characters per
line.)

\subsection{\texttt{set height} Command}\index{\texttt{set
height} command}
Syntax:  \texttt{set height} {\em number}

Metamath assumes your screen height is 24 lines of characters.  If your
screen is taller or shorter, this command lets you to change the number
of lines at which the display pauses and prompts you to continue.

\subsection{\texttt{beep} Command}\index{\texttt{beep} command}

Syntax:  \texttt{beep}

This command will produce a beep.  By typing it ahead after a
long-running command has started, it will alert you that the command is
finished.  For convenience, \texttt{b} is an abbreviation for
\texttt{beep}.

Note:  If \texttt{b} is typed at the \texttt{MM>} prompt immediately
after the end of a multiple-page display paged with ``\texttt{Press
<return> for more}...'' prompts, then the \texttt{b} will back up to the
previous page rather than perform the \texttt{beep} command.
In that case you must type the unabbreviated \texttt{beep} form
of the command.

\subsection{\texttt{more} Command}\index{\texttt{more} command}

Syntax:  \texttt{more} {\em filename}

This command will display the contents of an {\sc ascii} file on your
screen.  (This command is provided for convenience but is not very
powerful.  See Section~\ref{oscmd} to invoke your operating system's
command to do this, such as the \texttt{more} command in Unix.)

\subsection{Operating System Commands}\index{operating system
command}\label{oscmd}

A line enclosed in single or double quotes will be executed by your
computer's operating system if it has a command line interface.  For
example, on a {\sc vax/vms} system,
\verb/MM> 'dir'/
will print disk directory contents.  Note that this feature will not
work on the Macintosh prior to Mac OS X, which does not have a command
line interface.

For your convenience, the trailing quote is optional.

\subsection{Size Limitations in Metamath}

In general, there are no fixed, predefined limits\index{Metamath!memory
limits} on how many labels, tokens\index{token}, statements, etc.\ that
you may have in a database file.  The Metamath program uses 32-bit
variables (64-bit on 64-bit CPUs) as indices for almost all internal
arrays, which are allocated dynamically as needed.



\section{Reading and Writing Files}

The following commands create new files:  the \texttt{open} commands;
the \texttt{write} commands; the \texttt{/html},
\texttt{/alt{\char`\_}html}, \texttt{/brief{\char`\_}html},
\texttt{/brief{\char`\_}alt{\char`\_}html} qualifiers of \texttt{show
statement}, and \texttt{midi}.  The following commands append to files
previously opened:  the \texttt{/tex} qualifier of \texttt{show proof}
and \texttt{show new{\char`\_}proof}; the \texttt{/tex} and
\texttt{/simple{\char`\_}tex} qualifiers of \texttt{show statement}; the
\texttt{close} commands; and all screen dialog between \texttt{open log}
and \texttt{close log}.

The commands that create new files will not overwrite an existing {\em
filename} but will rename the existing one to {\em
filename}\texttt{{\char`\~}1}.  An existing {\em
filename}\texttt{{\char`\~}1} is renamed {\em
filename}\texttt{{\char`\~}2}, etc.\ up to {\em
filename}\texttt{{\char`\~}9}.  An existing {\em
filename}\texttt{{\char`\~}9} is deleted.  This makes recovery from
mistakes easier but also will clutter up your directory, so occasionally
you may want to clean up (delete) these \texttt{{\char`\~}}$n$ files.


\subsection{\texttt{read} Command}\index{\texttt{read} command}
Syntax:  \texttt{read} {\em file-name} [\texttt{/verify}]

This command will read in a Metamath language source file and any included
files.  Normally it will be the first thing you do when entering Metamath.
Statement syntax is checked, but proof syntax is not checked.
Note that the file name may be enclosed in single or double quotes;
this is useful if the file name contains slashes, as might be the case
under Unix.

If you are getting an ``\texttt{?Expected VERIFY}'' error
when trying to read a Unix file name with slashes, you probably haven't
quoted it.\index{Unix file names}\index{file names!Unix}

If you are prompted for the file name (by pressing {\em return}
 after \texttt{read})
you should {\em not} put quotes around it, even if it is a Unix file name
with slashes.

Optional command qualifier:

    \texttt{/verify} - Verify all proofs as the database is read in.  This
         qualifier will slow down reading in the file.  See \texttt{verify
         proof} for more information on file error-checking.

See also \texttt{erase}.



\subsection{\texttt{write source} Command}\index{\texttt{write source} command}
Syntax:  \texttt{write source} {\em filename}
[\texttt{/rewrap}]
[\texttt{/split}]
[\texttt{/keep\_includes}]
[\texttt{/no\_versioning}]

This command will write the contents of a Metamath\index{database}
database into a file.\index{source file}

Optional command qualifiers:

\texttt{/rewrap} -
Reformats statements and comments according to the
convention used in the set.mm database.
It unwraps the
lines in the comment before each \texttt{\$a} and \texttt{\$p} statement,
then it
rewraps the line.  You should compare the output to the original
to make sure that the desired effect results; if not, go back to
the original source.  The wrapped line length honors the
\texttt{set width}
parameter currently in effect.  Note:  Text
enclosed in \texttt{<HTML>}...\texttt{</HTML>} tags is not modified by the
\texttt{/rewrap} qualifier.
Proofs themselves are not reformatted;
use \texttt{save proof * / compressed} to do that.
An isolated tilde (\~{}) is always kept on the same line as the following
symbol, so you can find all comment references to a symbol by
searching for \~{} followed by a space and that symbol
(this is useful for finding cross-references).
Incidentally, \texttt{save proof} also honors the \texttt{set width}
parameter currently in effect.

\texttt{/split} - Files included in the source using the expression
\$[ \textit{inclfile} \$] will be
written into separate files instead of being included in a single output
file.  The name of each separately written included file will be the
\textit{inclfile} argument of its inclusion command.

\texttt{/keep\_includes} - If a source file has includes but is written as a
single file by omitting \texttt{/split}, by default the included files will
be deleted (actually just renamed with a \char`\~1 suffix unless
\texttt{/no\_versioning} is specified) to prevent the possibly confusing
source duplication in both the output file and the included file.
The \texttt{/keep\_includes} qualifier will prevent this deletion.

\texttt{/no\_versioning} - Backup files suffixed with \char`\~1 are not created.


\section{Showing Status and Statements}



\subsection{\texttt{show settings} Command}\index{\texttt{show settings} command}
Syntax:  \texttt{show settings}

This command shows the state of various parameters.

\subsection{\texttt{show memory} Command}\index{\texttt{show memory} command}
Syntax:  \texttt{show memory}

This command shows the available memory left.  It is not meaningful
on most modern operating systems,
which have virtual memory.\index{Metamath!memory usage}


\subsection{\texttt{show labels} Command}\index{\texttt{show labels} command}
Syntax:  \texttt{show labels} {\em label-match} [\texttt{/all}]
   [\texttt{/linear}]

This command shows the labels of \texttt{\$a} and \texttt{\$p}
statements that match {\em label-match}.  A \verb$*$ in {label-match}
matches zero or more characters.  For example, \verb$*abc*def$ will match all
labels containing \verb$abc$ and ending with \verb$def$.

Optional command qualifiers:

   \texttt{/all} - Include matches for \texttt{\$e} and \texttt{\$f}
   statement labels.

   \texttt{/linear} - Display only one label per line.  This can be useful for
       building scripts in conjunction with the utilities under the
       \texttt{tools}\index{\texttt{tools} command} command.



\subsection{\texttt{show statement} Command}\index{\texttt{show statement} command}
Syntax:  \texttt{show statement} {\em label-match} [{\em qualifiers} (see below)]

This command provides information about a statement.  Only statements
that have labels (\texttt{\$f}\index{\texttt{\$f} statement},
\texttt{\$e}\index{\texttt{\$e} statement},
\texttt{\$a}\index{\texttt{\$a} statement}, and
\texttt{\$p}\index{\texttt{\$p} statement}) may be specified.
If {\em label-match}
contains wildcard (\verb$*$) characters, all matching statements will be
displayed in the order they occur in the database.

Optional qualifiers (only one qualifier at a time is allowed):

    \texttt{/comment} - This qualifier includes the comment that immediately
       precedes the statement.

    \texttt{/full} - Show complete information about each statement,
       and show all
       statements matching {\em label} (including \texttt{\$e}
       and \texttt{\$f} statements).

    \texttt{/tex} - This qualifier will write the statement information to the
       \LaTeX\ file previously opened with \texttt{open tex}.  See
       Section~\ref{texout}.

    \texttt{/simple{\char`\_}tex} - The same as \texttt{/tex}, except that
       \LaTeX\ macros are not used for formatting equations, allowing easier
       manual edits of the output for slide presentations, etc.

    \texttt{/html}\index{html generation@{\sc html} generation},
       \texttt{/alt{\char`\_}html}, \texttt{/brief{\char`\_}html},
       \texttt{/brief{\char`\_}alt{\char`\_}html} -
       These qualifiers invoke a special mode of
       \texttt{show statement} that
       creates a web page for the statement.  They may not be used with
       any other qualifier.  See Section~\ref{htmlout} or
       \texttt{help html} in the program.


\subsection{\texttt{search} Command}\index{\texttt{search} command}
Syntax:  search {\em label-match}
\texttt{"}{\em symbol-match}{\tt}" [\texttt{/all}] [\texttt{/comments}]
[\texttt{/join}]

This command searches all \texttt{\$a} and \texttt{\$p} statements
matching {\em label-match} for occurrences of {\em symbol-match}.  A
\verb@*@ in {\em label-match} matches any label character.  A \verb@$*@
in {\em symbol-match} matches any sequence of symbols.  The symbols in
{\em symbol-match} must be separated by white space.  The quotes
surrounding {\em symbol-match} may be single or double quotes.  For
example, \texttt{search b}\verb@* "-> $* ch"@ will list all statements
whose labels begin with \texttt{b} and contain the symbols \verb@->@ and
\texttt{ch} surrounding any symbol sequence (including no symbol
sequence).  The wildcards \texttt{?} and \texttt{\$?} are also available
to match individual characters in labels and symbols respectively; see
\texttt{help search} in the Metamath program for details on their usage.

Optional command qualifiers:

    \texttt{/all} - Also search \texttt{\$e} and \texttt{\$f} statements.

    \texttt{/comments} - Search the comment that immediately precedes each
        label-matched statement for {\em symbol-match}.  In this case
        {\em symbol-match} is an arbitrary, non-case-sensitive character
        string.  Quotes around {\em symbol-match} are optional if there
        is no ambiguity.

    \texttt{/join} - In the case of a \texttt{\$a} or \texttt{\$p} statement,
	prepend its \texttt{\$e}
	hypotheses for searching. The
	\texttt{/join} qualifier has no effect in \texttt{/comments} mode.

\section{Displaying and Verifying Proofs}


\subsection{\texttt{show proof} Command}\index{\texttt{show proof} command}
Syntax:  \texttt{show proof} {\em label-match} [{\em qualifiers} (see below)]

This command displays the proof of the specified
\texttt{\$p}\index{\texttt{\$p} statement} statement in various formats.
The {\em label-match} may contain wildcard (\verb@$*@) characters to match
multiple statements.  Without any qualifiers, only the logical steps
will be shown (i.e.\ syntax construction steps will be omitted), in an
indented format.

Most of the time, you will use
    \texttt{show proof} {\em label}
to see just the proof steps corresponding to logical inferences.

Optional command qualifiers:

    \texttt{/essential} - The proof tree is trimmed of all
        \texttt{\$f}\index{\texttt{\$f} statement} hypotheses before
        being displayed.  (This is the default, and it is redundant to
        specify it.)

    \texttt{/all} - the proof tree is not trimmed of all \texttt{\$f} hypotheses before
        being displayed.  \texttt{/essential} and \texttt{/all} are mutually exclusive.

    \texttt{/from{\char`\_}step} {\em step} - The display starts at the specified
        step.  If
        this qualifier is omitted, the display starts at the first step.

    \texttt{/to{\char`\_}step} {\em step} - The display ends at the specified
        step.  If this
        qualifier is omitted, the display ends at the last step.

    \texttt{/tree{\char`\_}depth} {\em number} - Only
         steps at less than the specified proof
        tree depth are displayed.  Sometimes useful for obtaining an overview of
        the proof.

    \texttt{/reverse} - The steps are displayed in reverse order.

    \texttt{/renumber} - When used with \texttt{/essential}, the steps are renumbered
        to correspond only to the essential steps.

    \texttt{/tex} - The proof is converted to \LaTeX\ and\index{latex@{\LaTeX}}
        stored in the file opened
        with \texttt{open tex}.  See Section~\ref{texout} or
        \texttt{help tex} in the program.

    \texttt{/lemmon} - The proof is displayed in a non-indented format known
        as Lemmon style, with explicit previous step number references.
        If this qualifier is omitted, steps are indented in a tree format.

    \texttt{/start{\char`\_}column} {\em number} - Overrides the default column
        (16)
        at which the formula display starts in a Lemmon-style display.  May be
        used only in conjunction with \texttt{/lemmon}.

    \texttt{/normal} - The proof is displayed in normal format suitable for
        inclusion in a Metamath source file.  May not be used with any other
        qualifier.

    \texttt{/compressed} - The proof is displayed in compressed format
        suitable for inclusion in a Metamath source file.  May not be used with
        any other qualifier.

    \texttt{/statement{\char`\_}summary} - Summarizes all statements (like a
        brief \texttt{show statement})
        used by the proof.  It may not be used with any other qualifier
        except \texttt{/essential}.

    \texttt{/detailed{\char`\_}step} {\em step} - Shows the details of what is
        happening at
        a specific proof step.  May not be used with any other qualifier.
        The {\em step} is the step number shown when displaying a
        proof without the \texttt{/renumber} qualifier.


\subsection{\texttt{show usage} Command}\index{\texttt{show usage} command}
Syntax:  \texttt{show usage} {\em label-match} [\texttt{/recursive}]

This command lists the statements whose proofs make direct reference to
the statement specified.

Optional command qualifier:

    \texttt{/recursive} - Also include statements whose proofs ultimately
        depend on the statement specified.



\subsection{\texttt{show trace\_back} Command}\index{\texttt{show
       trace{\char`\_}back} command}
Syntax:  \texttt{show trace{\char`\_}back} {\em label-match} [\texttt{/essential}] [\texttt{/axioms}]
    [\texttt{/tree}] [\texttt{/depth} {\em number}]

This command lists all statements that the proof of the \texttt{\$p}
statement(s) specified by {\em label-match} depends on.

Optional command qualifiers:

    \texttt{/essential} - Restrict the trace-back to \texttt{\$e}
        \index{\texttt{\$e} statement} hypotheses of proof trees.

    \texttt{/axioms} - List only the axioms that the proof ultimately depends on.

    \texttt{/tree} - Display the trace-back in an indented tree format.

    \texttt{/depth} {\em number} - Restrict the \texttt{/tree} trace-back to the
        specified indentation depth.

    \texttt{/count{\char`\_}steps} - Count the number of steps the proof has
       all the way back to axioms.  If \texttt{/essential} is specified,
       expansions of variable-type hypotheses (syntax constructions) are not counted.

\subsection{\texttt{verify proof} Command}\index{\texttt{verify proof} command}
Syntax:  \texttt{verify proof} {\em label-match} [\texttt{/syntax{\char`\_}only}]

This command verifies the proofs of the specified statements.  {\em
label-match} may contain wild card characters (\texttt{*}) to verify
more than one proof; for example \verb/*abc*def/ will match all labels
containing \texttt{abc} and ending with \texttt{def}.
The command \texttt{verify proof *} will verify all proofs in the database.

Optional command qualifier:

    \texttt{/syntax{\char`\_}only} - This qualifier will perform a check of syntax
        and RPN
        stack violations only.  It will not verify that the proof is
        correct.  This qualifier is useful for quickly determining which
        proofs are incomplete (i.e.\ are under development and have \texttt{?}'s
        in them).

{\em Note:} \texttt{read}, followed by \texttt{verify proof *}, will ensure
 the database is free
from errors in the Metamath language but will not check the markup notation
in comments.
You can also check the markup notation by running \texttt{verify markup *}
as discussed in Section~\ref{verifymarkup}; see also the discussion
on generating {\sc HTML} in Section~\ref{htmlout}.

\subsection{\texttt{verify markup} Command}\index{\texttt{verify markup} command}\label{verifymarkup}
Syntax:  \texttt{verify markup} {\em label-match}
[\texttt{/date{\char`\_}skip}]
[\texttt{/top{\char`\_}date{\char`\_}skip}]
[\texttt{/file{\char`\_}skip}]
[\texttt{/verbose}]

This command checks comment markup and other informal conventions we have
adopted.  It error-checks the latexdef, htmldef, and althtmldef statements
in the \texttt{\$t} statement of a Metamath source file.\index{error checking}
It error-checks any \texttt{`}...\texttt{`},
\texttt{\char`\~}~\textit{label},
and bibliographic markups in statement descriptions.
It checks that
\texttt{\$p} and \texttt{\$a} statements
have the same content when their labels start with
``ax'' and ``ax-'' respectively but are otherwise identical, for example
ax4 and ax-4.
It also verifies the date consistency of ``(Contributed by...),''
``(Revised by...),'' and ``(Proof shortened by...)'' tags in the comment
above each \texttt{\$a} and \texttt{\$p} statement.

Optional command qualifiers:

    \texttt{/date{\char`\_}skip} - This qualifier will
        skip date consistency checking,
        which is usually not required for databases other than
	\texttt{set.mm}.

    \texttt{/top{\char`\_}date{\char`\_}skip} - This qualifier will check date consistency except
        that the version date at the top of the database file will not
        be checked.  Only one of
        \texttt{/date{\char`\_}skip} and
        \texttt{/top{\char`\_}date{\char`\_}skip} may be
        specified.

    \texttt{/file{\char`\_}skip} - This qualifier will skip checks that require
        external files to be present, such as checking GIF existence and
        bibliographic links to mmset.html or equivalent.  It is useful
        for doing a quick check from a directory without these files.

    \texttt{/verbose} - Provides more information.  Currently it provides a list
        of axXXX vs. ax-XXX matches.

\subsection{\texttt{save proof} Command}\index{\texttt{save proof} command}
Syntax:  \texttt{save proof} {\em label-match} [\texttt{/normal}]
   [\texttt{/compressed}]

The \texttt{save proof} command will reformat a proof in one of two formats and
replace the existing proof in the source buffer\index{source
buffer}.  It is useful for
converting between proof formats.  Note that a proof will not be
permanently saved until a \texttt{write source} command is issued.

Optional command qualifiers:

    \texttt{/normal} - The proof is saved in the normal format (i.e., as a
        sequence
        of labels, which is the defined format of the basic Metamath
        language).\index{basic language}  This is the default format that
        is used if a qualifier
        is omitted.

    \texttt{/compressed} - The proof is saved in the compressed format which
        reduces storage requirements for a database.
        See Appendix~\ref{compressed}.




\section{Creating Proofs}\label{pfcommands}\index{Proof Assistant}

Before using the Proof Assistant, you must add a \texttt{\$p} to your
source file (using a text editor) containing the statement you want to
prove.  Its proof should consist of a single \texttt{?}, meaning
``unknown step.''  Example:
\begin{verbatim}
equid $p x = x $= ? $.
\end{verbatim}

To enter the Proof assistant, type \texttt{prove} {\em label}, e.g.
\texttt{prove equid}.  Metamath will respond with the \texttt{MM-PA>}
prompt.

Proofs are created working backwards from the statement being proved,
primarily using a series of \texttt{assign} commands.  A proof is
complete when all steps are assigned to statements and all steps are
unified and completely known.  During the creation of a proof, Metamath
will allow only operations that are legal based on what is known up to
that point.  For example, it will not allow an \texttt{assign} of a
statement that cannot be unified with the unknown proof step being
assigned.

{\em Important:}
The Proof Assistant is
{\em not} a tool to help you discover proofs.  It is just a tool to help
you add them to the database.  For a tutorial read
Section~\ref{frstprf}.
To practice using the Proof Assistant, you may
want to \texttt{prove} an existing theorem, then delete all steps with
\texttt{delete all}, then re-create it with the Proof Assistant while
looking at its proof display (before deletion).
You might want to figure out your first few proofs completely
and write them down by hand, before using the Proof Assistant, though
not everyone finds that effective.

{\em Important:}
The \texttt{undo} command if very helpful when entering a proof, because
it allows you to undo a previously-entered step.
In addition, we suggest that you
keep track of your work with a log file (\texttt{open
log}) and save it frequently (\texttt{save new{\char`\_}proof},
\texttt{write source}).
You can use \texttt{delete} to reverse an \texttt{assign}.
You can also do \texttt{delete floating{\char`\_}hypotheses}, then
\texttt{initialize all}, then \texttt{unify all /interactive} to
reinitialize bad unifications made accidentally or by bad
\texttt{assign}s.  You cannot reverse a \texttt{delete} except by
a relevant \texttt{undo} or using
\texttt{exit /force} then reentering the Proof Assistant to recover from
the last \texttt{save new{\char`\_}proof}.

The following commands are available in the Proof Assistant (at the
\texttt{MM-PA>} prompt) to help you create your proof.  See the
individual commands for more detail.

\begin{itemize}
\item[]
    \texttt{show new{\char`\_}proof} [\texttt{/all},...] - Displays the
        proof in progress.  You will use this command a lot; see \texttt{help
        show new{\char`\_}proof} to become familiar with its qualifiers.  The
        qualifiers \texttt{/unknown} and \texttt{/not{\char`\_}unified} are
        useful for seeing the work remaining to be done.  The combination
        \texttt{/all/unknown} is useful for identifying dummy variables that must be
        assigned, or attempts to use illegal syntax, when \texttt{improve all}
        is unable to complete the syntax constructions.  Unknown variables are
        shown as \texttt{\$1}, \texttt{\$2},...
\item[]
    \texttt{assign} {\em step} {\em label} - Assigns an unknown {\em step}
        number with the statement
        specified by {\em label}.
\item[]
    \texttt{let variable} {\em variable}
        \texttt{= "}{\em symbol sequence}\texttt{"}
          - Forces a symbol
        sequence to replace an unknown variable (such as \texttt{\$1}) in a proof.
        It is useful
        for helping difficult unifications, and it is necessary when you have
        dummy variables that eventually must be assigned a name.
\item[]
    \texttt{let step} {\em step} \texttt{= "}{\em symbol sequence}\texttt{"} -
          Forces a symbol sequence
        to replace the contents of a proof step, provided it can be
        unified with the existing step contents.  (I rarely use this.)
\item[]
    \texttt{unify step} {\em step} (or \texttt{unify all}) - Unifies
        the source and target of
        a step.  If you specify a specific step, you will be prompted
        to select among the unifications that are possible.  If you
        specify \texttt{all}, all steps with unique unifications, but only
        those steps, will be
        unified.  \texttt{unify all /interactive} goes through all non-unified
        steps.
\item[]
    \texttt{initialize} {\em step} (or \texttt{all}) - De-unifies the target and source of
        a step (or all steps), as well as the hypotheses of the source,
        and makes all variables in the source unknown.  Useful to recover from
        an \texttt{assign} or \texttt{let} mistake that
        resulted in incorrect unifications.
\item[]
    \texttt{delete} {\em step} (or \texttt{all} or \texttt{floating{\char`\_}hypotheses}) -
        Deletes the specified
        step(s).  \texttt{delete floating{\char`\_}hypotheses}, then \texttt{initialize all}, then
        \texttt{unify all /interactive} is useful for recovering from mistakes
        where incorrect unifications assigned wrong math symbol strings to
        variables.
\item[]
    \texttt{improve} {\em step} (or \texttt{all}) -
      Automatically creates a proof for steps (with no unknown variables)
      whose proof requires no statements with \texttt{\$e} hypotheses.  Useful
      for filling in proofs of \texttt{\$f} hypotheses.  The \texttt{/depth}
      qualifier will also try statements whose \texttt{\$e} hypotheses contain
      no new variables.  {\em Warning:} Save your work (with \texttt{save
      new{\char`\_}proof} then \texttt{write source}) before using
      \texttt{/depth = 2} or greater, since the search time grows
      exponentially and may never terminate in a reasonable time, and you
      cannot interrupt the search.  I have found that it is rare for
      \texttt{/depth = 3} or greater to be useful.
 \item[]
    \texttt{save new{\char`\_}proof} - Saves the proof in progress in the program's
        internal database buffer.  To save it permanently into the database file,
        use \texttt{write source} after
        \texttt{save new{\char`\_}proof}.  To revert to the last
        \texttt{save new{\char`\_}proof},
        \texttt{exit /force} from the Proof Assistant then re-enter the Proof
        Assistant.
 \item[]
    \texttt{match step} {\em step} (or \texttt{match all}) - Shows what
        statements are
        possibilities for the \texttt{assign} statement. (This command
        is not very
        useful in its present form and hopefully will be improved
        eventually.  In the meantime, use the \texttt{search} statement for
        candidates matching specific math token combinations.)
 \item[]
 \texttt{minimize{\char`\_}with}\index{\texttt{minimize{\char`\_}with} command}
% 3/10/07 Note: line-breaking the above results in duplicate index entries
     - After a proof is complete, this command will attempt
        to match other database theorems to the proof to see if the proof
        size can be reduced as a result.  See \texttt{help
        minimize{\char`\_}with} in the
        Metamath program for its usage.
 \item[]
 \texttt{undo}\index{\texttt{undo} command}
    - Undo the effect of a proof-changing command (all but the
      \texttt{show} and \texttt{save} commands above).
 \item[]
 \texttt{redo}\index{\texttt{redo} command}
    - Reverse the previous \texttt{undo}.
\end{itemize}

The following commands set parameters that may be relevant to your proof.
Consult the individual \texttt{help set}... commands.
\begin{itemize}
   \item[] \texttt{set unification{\char`\_}timeout}
 \item[]
    \texttt{set search{\char`\_}limit}
  \item[]
    \texttt{set empty{\char`\_}substitution} - note that default is \texttt{off}
\end{itemize}

Type \texttt{exit} to exit the \texttt{MM-PA>}
 prompt and get back to the \texttt{MM>} prompt.
Another \texttt{exit} will then get you out of Metamath.



\subsection{\texttt{prove} Command}\index{\texttt{prove} command}
Syntax:  \texttt{prove} {\em label}

This command will enter the Proof Assistant\index{Proof Assistant}, which will
allow you to create or edit the proof of the specified statement.
The command-line prompt will change from \texttt{MM>} to \texttt{MM-PA>}.

Note:  In the present version of
Metamath\index{Metamath!limitations of version 0.07.30}, the Proof
Assistant does not verify that \texttt{\$d}\index{\texttt{\$d}
statement} restrictions are met as a proof is being built.  After you
have completed a proof, you should type \texttt{save new{\char`\_}proof}
followed by \texttt{verify proof} {\em label} (where {\em label} is the
statement you are proving with the \texttt{prove} command) to verify the
\texttt{\$d} restrictions.

See also: \texttt{exit}

\subsection{\texttt{set unification\_timeout} Command}\index{\texttt{set
unification{\char`\_}timeout} command}
Syntax:  \texttt{set unification{\char`\_}timeout} {\em number}

(This command is available outside the Proof Assistant but affects the
Proof Assistant\index{Proof Assistant} only.)

Sometimes the Proof Assistant will inform you that a unification
time-out occurred.  This may happen when you try to \texttt{unify}
formulas with many temporary variables\index{temporary variable}
(\texttt{\$1}, \texttt{\$2}, etc.), since the time to compute all possible
unifications may grow exponentially with the number of variables.  If
you want Metamath to try harder (and you're willing to wait longer) you
may increase this parameter.  \texttt{show settings} will show you the
current value.

\subsection{\texttt{set empty\_substitution} Command}\index{\texttt{set
empty{\char`\_}substitution} command}
% These long names can't break well in narrow mode, and even "sloppy"
% is not enough. Work around this by not demanding justification.
\begin{flushleft}
Syntax:  \texttt{set empty{\char`\_}substitution on} or \texttt{set
empty{\char`\_}substitution off}
\end{flushleft}

(This command is available outside the Proof Assistant but affects the
Proof Assistant\index{Proof Assistant} only.)

The Metamath language allows variables to be
substituted\index{substitution!variable}\index{variable substitution}
with empty symbol sequences\index{empty substitution}.  However, in many
formal systems\index{formal system} this will never happen in a valid
proof.  Allowing for this possibility increases the likelihood of
ambiguous unifications\index{ambiguous
unification}\index{unification!ambiguous} during proof creation.
The default is that
empty substitutions are not allowed; for formal systems requiring them,
you must \texttt{set empty{\char`\_}substitution on}.
(An example where this must be \texttt{on}
would be a system that implements a Deduction Rule and in
which deductions from empty assumption lists would be permissible.  The
MIU-system\index{MIU-system} described in Appendix~\ref{MIU} is another
example.)
Note that empty substitutions are
always permissible in proof verification (VERIFY PROOF...) outside the
Proof Assistant.  (See the MIU system in the Metamath book for an example
of a system needing empty substitutions; another example would be a
system that implements a Deduction Rule and in which deductions from
empty assumption lists would be permissible.)

It is better to leave this \texttt{off} when working with \texttt{set.mm}.
Note that this command does not affect the way proofs are verified with
the \texttt{verify proof} command.  Outside of the Proof Assistant,
substitution of empty sequences for math symbols is always allowed.

\subsection{\texttt{set search\_limit} Command}\index{\texttt{set
search{\char`\_}limit} command} Syntax:  \texttt{set search{\char`\_}limit} {\em
number}

(This command is available outside the Proof Assistant but affects the
Proof Assistant\index{Proof Assistant} only.)

This command sets a parameter that determines when the \texttt{improve} command
in Proof Assistant mode gives up.  If you want \texttt{improve} to search harder,
you may increase it.  The \texttt{show settings} command tells you its current
value.


\subsection{\texttt{show new\_proof} Command}\index{\texttt{show
new{\char`\_}proof} command}
Syntax:  \texttt{show new{\char`\_}proof} [{\em
qualifiers} (see below)]

This command (available only in Proof Assistant mode) displays the proof
in progress.  It is identical to the \texttt{show proof} command, except that
there is no statement argument (since it is the statement being proved) and
the following qualifiers are not available:

    \texttt{/statement{\char`\_}summary}

    \texttt{/detailed{\char`\_}step}

Also, the following additional qualifiers are available:

    \texttt{/unknown} - Shows only steps that have no statement assigned.

    \texttt{/not{\char`\_}unified} - Shows only steps that have not been unified.

Note that \texttt{/essential}, \texttt{/depth}, \texttt{/unknown}, and
\texttt{/not{\char`\_}unified} may be
used in any combination; each of them effectively filters out additional
steps from the proof display.

See also:  \texttt{show proof}






\subsection{\texttt{assign} Command}\index{\texttt{assign} command}
Syntax:   \texttt{assign} {\em step} {\em label} [\texttt{/no{\char`\_}unify}]

   and:   \texttt{assign first} {\em label}

   and:   \texttt{assign last} {\em label}


This command, available in the Proof Assistant only, assigns an unknown
step (one with \texttt{?} in the \texttt{show new{\char`\_}proof}
listing) with the statement specified by {\em label}.  The assignment
will not be allowed if the statement cannot be unified with the step.

If \texttt{last} is specified instead of {\em step} number, the last
step that is shown by \texttt{show new{\char`\_}proof /unknown} will be
used.  This can be useful for building a proof with a command file (see
\texttt{help submit}).  It also makes building proofs faster when you know
the assignment for the last step.

If \texttt{first} is specified instead of {\em step} number, the first
step that is shown by \texttt{show new{\char`\_}proof /unknown} will be
used.

If {\em step} is zero or negative, the -{\em step}th from last unknown
step, as shown by \texttt{show new{\char`\_}proof /unknown}, will be
used.  \texttt{assign -1} {\em label} will assign the penultimate
unknown step, \texttt{assign -2} {\em label} the antepenultimate, and
\texttt{assign 0} {\em label} is the same as \texttt{assign last} {\em
label}.

Optional command qualifier:

    \texttt{/no{\char`\_}unify} - do not prompt user to select a unification if there is
        more than one possibility.  This is useful for noninteractive
        command files.  Later, the user can \texttt{unify all /interactive}.
        (The assignment will still be automatically unified if there is only
        one possibility and will be refused if unification is not possible.)



\subsection{\texttt{match} Command}\index{\texttt{match} command}
Syntax:  \texttt{match step} {\em step} [\texttt{/max{\char`\_}essential{\char`\_}hyp}
{\em number}]

    and:  \texttt{match all} [\texttt{/essential}]
          [\texttt{/max{\char`\_}essential{\char`\_}hyp} {\em number}]

This command, available in the Proof Assistant only, shows what
statements can be unified with the specified step(s).  {\em Note:} In
its current form, this command is not very useful because of the large
number of matches it reports.  I rarely use it.
It may be enhanced in the future.

Optional command qualifiers:

    \texttt{/max{\char`\_}essential{\char`\_}hyp} {\em number} - filters out
        of the list any statements
        with more than the specified number of
        \texttt{\$e}\index{\texttt{\$e} statement} hypotheses

    \texttt{/essential{\char`\_}only} - in the \texttt{match all} statement, only
        the steps that
        would be listed in the \texttt{show new{\char`\_}proof /essential} display are
        matched.



\subsection{\texttt{let} Command}\index{\texttt{let} command}
Syntax: \texttt{let variable} {\em variable} = \verb/"/{\em symbol-sequence}\verb/"/

  and: \texttt{let step} {\em step} = \verb/"/{\em symbol-sequence}\verb/"/

These commands, available in the Proof Assistant\index{Proof Assistant}
only, assign a temporary variable\index{temporary variable} or unknown
step with a specific symbol sequence.  They are useful in the middle of
creating a proof, when you know what should be in the proof step but the
unification algorithm doesn't yet have enough information to completely
specify the temporary variables.  A ``temporary variable'' is one that
has the form \texttt{\$}{\em nn} in the proof display, such as
\texttt{\$1}, \texttt{\$2}, etc.  The {\em symbol-sequence} may contain
other unknown variables if desired.  Examples:

    \verb/let variable $32 = "A = B"/

    \verb/let variable $32 = "A = $35"/

    \verb/let step 10 = '|- x = x'/

    \verb/let step -2 = "|- ( $7 = ph )"/

Any symbol sequence will be accepted for the \texttt{let variable}
command.  Only those symbol sequences that can be unified with the step
will be accepted for \texttt{let step}.

The \texttt{let} commands ``zap'' the proof with information that can
only be verified when the proof is built up further.  If you make an
error, the command sequence \texttt{delete
floating{\char`\_}hypotheses}, \texttt{initialize all}, and
\texttt{unify all /interactive} will undo a bad \texttt{let} assignment.

If {\em step} is zero or negative, the -{\em step}th from last unknown
step, as shown by \texttt{show new{\char`\_}proof /unknown}, will be
used.  The command \texttt{let step 0} = \verb/"/{\em
symbol-sequence}\verb/"/ will use the last unknown step, \texttt{let
step -1} = \verb/"/{\em symbol-sequence}\verb/"/ the penultimate, etc.
If {\em step} is positive, \texttt{let step} may be used to assign known
(in the sense of having previously been assigned a label with
\texttt{assign}) as well as unknown steps.

Either single or double quotes can surround the {\em symbol-sequence} as
long as they are different from any quotes inside a {\em
symbol-sequence}.  If {\em symbol-sequence} contains both kinds of
quotes, see the instructions at the end of \texttt{help let} in the
Metamath program.


\subsection{\texttt{unify} Command}\index{\texttt{unify} command}
Syntax:  \texttt{unify step} {\em step}

      and:   \texttt{unify all} [\texttt{/interactive}]

These commands, available in the Proof Assistant only, unify the source
and target of the specified step(s). If you specify a specific step, you
will be prompted to select among the unifications that are possible.  If
you specify \texttt{all}, only those steps with unique unifications will be
unified.

Optional command qualifier for \texttt{unify all}:

    \texttt{/interactive} - You will be prompted to select among the
        unifications
        that are possible for any steps that do not have unique
        unifications.  (Otherwise \texttt{unify all} will bypass these.)

See also \texttt{set unification{\char`\_}timeout}.  The default is
100000, but increasing it to 1000000 can help difficult cases.  Manually
assigning some or all of the unknown variables with the \texttt{let
variable} command also helps difficult cases.



\subsection{\texttt{initialize} Command}\index{\texttt{initialize} command}
Syntax:  \texttt{initialize step} {\em step}

    and: \texttt{initialize all}

These commands, available in the Proof Assistant\index{Proof Assistant}
only, ``de-unify'' the target and source of a step (or all steps), as
well as the hypotheses of the source, and makes all variables in the
source and the source's hypotheses unknown.  This command is useful to
help recover from incorrect unifications that resulted from an incorrect
\texttt{assign}, \texttt{let}, or unification choice.  Part or all of
the command sequence \texttt{delete floating{\char`\_}hypotheses},
\texttt{initialize all}, and \texttt{unify all /interactive} will recover
from incorrect unifications.

See also:  \texttt{unify} and \texttt{delete}



\subsection{\texttt{delete} Command}\index{\texttt{delete} command}
Syntax:  \texttt{delete step} {\em step}

   and:      \texttt{delete all} -- {\em Warning: dangerous!}

   and:      \texttt{delete floating{\char`\_}hypotheses}

These commands are available in the Proof Assistant only.  The
\texttt{delete step} command deletes the proof tree section that
branches off of the specified step and makes the step become unknown.
\texttt{delete all} is equivalent to \texttt{delete step} {\em step}
where {\em step} is the last step in the proof (i.e.\ the beginning of
the proof tree).

In most cases the \texttt{undo} command is the best way to undo
a previous step.
An alternative is to salvage your last \texttt{save
new{\char`\_}proof} by exiting and reentering the Proof Assistant.
For this to work, keep a log file open to record your work
and to do \texttt{save new{\char`\_}proof} frequently, especially before
\texttt{delete}.

\texttt{delete floating{\char`\_}hypotheses} will delete all sections of
the proof that branch off of \texttt{\$f}\index{\texttt{\$f} statement}
statements.  It is sometimes useful to do this before an
\texttt{initialize} command to recover from an error.  Note that once a
proof step with a \texttt{\$f} hypothesis as the target is completely
known, the \texttt{improve} command can usually fill in the proof for
that step.  Unlike the deletion of logical steps, \texttt{delete
floating{\char`\_}hypotheses} is a relatively safe command that is
usually easy to recover from.



\subsection{\texttt{improve} Command}\index{\texttt{improve} command}
\label{improve}
Syntax:  \texttt{improve} {\em step} [\texttt{/depth} {\em number}]
                                               [\texttt{/no{\char`\_}distinct}]

   and:   \texttt{improve first} [\texttt{/depth} {\em number}]
                                              [\texttt{/no{\char`\_}distinct}]

   and:   \texttt{improve last} [\texttt{/depth} {\em number}]
                                              [\texttt{/no{\char`\_}distinct}]

   and:   \texttt{improve all} [\texttt{/depth} {\em number}]
                                              [\texttt{/no{\char`\_}distinct}]

These commands, available in the Proof Assistant\index{Proof Assistant}
only, try to find proofs automatically for unknown steps whose symbol
sequences are completely known.  They are primarily useful for filling in
proofs of \texttt{\$f}\index{\texttt{\$f} statement} hypotheses.  The
search will be restricted to statements having no
\texttt{\$e}\index{\texttt{\$e} statement} hypotheses.

\begin{sloppypar} % narrow
Note:  If memory is limited, \texttt{improve all} on a large proof may
overflow memory.  If you use \texttt{set unification{\char`\_}timeout 1}
before \texttt{improve all}, there will usually be sufficient
improvement to easily recover and completely \texttt{improve} the proof
later on a larger computer.  Warning:  Once memory has overflowed, there
is no recovery.  If in doubt, save the intermediate proof (\texttt{save
new{\char`\_}proof} then \texttt{write source}) before \texttt{improve
all}.
\end{sloppypar}

If \texttt{last} is specified instead of {\em step} number, the last
step that is shown by \texttt{show new{\char`\_}proof /unknown} will be
used.

If \texttt{first} is specified instead of {\em step} number, the first
step that is shown by \texttt{show new{\char`\_}proof /unknown} will be
used.

If {\em step} is zero or negative, the -{\em step}th from last unknown
step, as shown by \texttt{show new{\char`\_}proof /unknown}, will be
used.  \texttt{improve -1} will use the penultimate
unknown step, \texttt{improve -2} {\em label} the antepenultimate, and
\texttt{improve 0} is the same as \texttt{improve last}.

Optional command qualifier:

    \texttt{/depth} {\em number} - This qualifier will cause the search
        to include
        statements with \texttt{\$e} hypotheses (but no new variables in
        the \texttt{\$e}
        hypotheses), provided that the backtracking has not exceeded the
        specified depth. {\em Warning:}  Try \texttt{/depth 1},
        then \texttt{2}, then \texttt{3}, etc.
        in sequence because of possible exponential blowups.  Save your
        work before trying \texttt{/depth} greater than \texttt{1}!

    \texttt{/no{\char`\_}distinct} - Skip trial statements that have
        \texttt{\$d}\index{\texttt{\$d} statement} requirements.
        This qualifier will prevent assignments that might violate \texttt{\$d}
        requirements but it also could miss possible legal assignments.

See also: \texttt{set search{\char`\_}limit}

\subsection{\texttt{save new\_proof} Command}\index{\texttt{save
new{\char`\_}proof} command}
Syntax:  \texttt{save new{\char`\_}proof} {\em label} [\texttt{/normal}]
   [\texttt{/compressed}]

The \texttt{save new{\char`\_}proof} command is available in the Proof
Assistant only.  It saves the proof in progress in the source
buffer\index{source buffer}.  \texttt{save new{\char`\_}proof} may be
used to save a completed proof, or it may be used to save a proof in
progress in order to work on it later.  If an incomplete proof is saved,
any user assignments with \texttt{let step} or \texttt{let variable}
will be lost, as will any ambiguous unifications\index{ambiguous
unification}\index{unification!ambiguous} that were resolved manually.
To help make recovery easier, it can be helpful to \texttt{improve all}
before \texttt{save new{\char`\_}proof} so that the incomplete proof
will have as much information as possible.

Note that the proof will not be permanently saved until a \texttt{write
source} command is issued.

Optional command qualifiers:

    \texttt{/normal} - The proof is saved in the normal format (i.e., as a
        sequence of labels, which is the defined format of the basic Metamath
        language).\index{basic language}  This is the default format that
        is used if a qualifier is omitted.

    \texttt{/compressed} - The proof is saved in the compressed format, which
        reduces storage requirements for a database.
        (See Appendix~\ref{compressed}.)


\section{Creating \LaTeX\ Output}\label{texout}\index{latex@{\LaTeX}}

You can generate \LaTeX\ output given the
information in a database.
The database must already include the necessary typesetting information
(see section \ref{tcomment} for how to provide this information).

The \texttt{show statement} and \texttt{show proof} commands each have a
special \texttt{/tex} command qualifier that produces \LaTeX\ output.
(The \texttt{show statement} command also has the
\texttt{/simple{\char`\_}tex} qualifier for output that is easier to
edit by hand.)  Before you can use them, you must open a \LaTeX\ file to
which to send their output.  A typical complete session will use this
sequence of Metamath commands:

\begin{verbatim}
read set.mm
open tex example.tex
show statement a1i /tex
show proof a1i /all/lemmon/renumber/tex
show statement uneq2 /tex
show proof uneq2 /all/lemmon/renumber/tex
close tex
\end{verbatim}

See Section~\ref{mathcomments} for information on comment markup and
Appendix~\ref{ASCII} for information on how math symbol translation is
specified.

To format and print the \LaTeX\ source, you will need the \LaTeX\
program, which is standard on most Linux installations and available for
Windows.  On Linux, in order to create a {\sc pdf} file, you will
typically type at the shell prompt
\begin{verbatim}
$ pdflatex example.tex
\end{verbatim}

\subsection{\texttt{open tex} Command}\index{\texttt{open tex} command}
Syntax:  \texttt{open tex} {\em file-name} [\texttt{/no{\char`\_}header}]

This command opens a file for writing \LaTeX\
source\index{latex@{\LaTeX}} and writes a \LaTeX\ header to the file.
\LaTeX\ source can be written with the \texttt{show proof}, \texttt{show
new{\char`\_}proof}, and \texttt{show statement} commands using the
\texttt{/tex} qualifier.

The mapping to \LaTeX\ symbols is defined in a special comment
containing a \texttt{\$t} token, described in Appendix~\ref{ASCII}.

There is an optional command qualifier:

    \texttt{/no{\char`\_}header} - This qualifier prevents a standard
        \LaTeX\ header and trailer
        from being included with the output \LaTeX\ code.


\subsection{\texttt{close tex} Command}\index{\texttt{close tex} command}
Syntax:  \texttt{close tex}

This command writes a trailer to any \LaTeX\ file\index{latex@{\LaTeX}}
that was opened with \texttt{open tex} (unless
\texttt{/no{\char`\_}header} was used with \texttt{open tex}) and closes
the \LaTeX\ file.


\section{Creating {\sc HTML} Output}\label{htmlout}

You can generate {\sc html} web pages given the
information in a database.
The database must already include the necessary typesetting information
(see section \ref{tcomment} for how to provide this information).
The ability to produce {\sc html} web pages was added in Metamath version
0.07.30.

To create an {\sc html} output file(s) for \texttt{\$a} or \texttt{\$p}
statement(s), use
\begin{quote}
    \texttt{show statement} {\em label-match} \texttt{/html}
\end{quote}
The output file will be named {\em label-match}\texttt{.html}
for each match.  When {\em
label-match} has wildcard (\texttt{*}) characters, all statements with
matching labels will have {\sc html} files produced for them.  Also,
when {\em label-match} has a wildcard (\texttt{*}) character, two additional
files, \texttt{mmdefinitions.html} and \texttt{mmascii.html} will be
produced.  To produce {\em only} these two additional files, you can use
\texttt{?*}, which will not match any statement label, in place of {\em
label-match}.

There are three other qualifiers for \texttt{show statement} that also
generate {\sc HTML} code.  These are \texttt{/alt{\char`\_}html},
\texttt{/brief{\char`\_}html}, and
\texttt{/brief{\char`\_}alt{\char`\_}html}, and are described in the
next section.

The command
\begin{quote}
    \texttt{show statement} {\em label-match} \texttt{/alt{\char`\_}html}
\end{quote}
does the same as \texttt{show statement} {\em label-match} \texttt{/html},
except that the {\sc html} code for the symbols is taken from
\texttt{althtmldef} statements instead of \texttt{htmldef} statements in
the \texttt{\$t} comment.

The command
\begin{verbatim}
show statement * /brief_html
\end{verbatim}
invokes a special mode that just produces definition and theorem lists
accompanied by their symbol strings, in a format suitable for copying and
pasting into another web page (such as the tutorial pages on the
Metamath web site).

Finally, the command
\begin{verbatim}
show statement * /brief_alt_html
\end{verbatim}
does the same as \texttt{show statement * / brief{\char`\_}html}
for the alternate {\sc html}
symbol representation.

A statement's comment can include a special notation that provides a
certain amount of control over the {\sc HTML} version of the comment.  See
Section~\ref{mathcomments} (p.~\pageref{mathcomments}) for the comment
markup features.

The \texttt{write theorem{\char`\_}list} and \texttt{write bibliography}
commands, which are described below, provide as a side effect complete
error checking for all of the features described in this
section.\index{error checking}

\subsection{\texttt{write theorem\_list}
Command}\index{\texttt{write theorem{\char`\_}list} command}
Syntax:  \texttt{write theorem{\char`\_}list}
[\texttt{/theorems{\char`\_}per{\char`\_}page} {\em number}]

This command writes a list of all of the \texttt{\$a} and \texttt{\$p}
statements in the database into a web page file
 called \texttt{mmtheorems.html}.
When additional files are needed, they are called
\texttt{mmtheorems2.html}, \texttt{mmtheorems3.html}, etc.

Optional command qualifier:

    \texttt{/theorems{\char`\_}per{\char`\_}page} {\em number} -
 This qualifier specifies the number of statements to
        write per web page.  The default is 100.

{\em Note:} In version 0.07.30\index{Metamath!limitations of version
0.07.30} of Metamath, the ``Nearby theorems'' links on the individual
web pages presuppose 100 theorems per page when linking to the theorem
list pages.  Therefore the \texttt{/theorems{\char`\_}per{\char`\_}page}
qualifier, if it specifies a number other than 100, will cause the
individual web pages to be out of sync and should not be used to
generate the main theorem list for the web site.  This is expected to be
fixed in a future version.


\subsection{\texttt{write bibliography}\label{wrbib}
Command}\index{\texttt{write bibliography} command}
Syntax:  \texttt{write bibliography} {\em filename}

This command reads an existing {\sc html} bibliographic cross-reference
file, normally called \texttt{mmbiblio.html}, and updates it per the
bibliographic links in the database comments.  The file is updated
between the {\sc html} comment lines \texttt{<!--
{\char`\#}START{\char`\#} -->} and \texttt{<!-- {\char`\#}END{\char`\#}
-->}.  The original input file is renamed to {\em
filename}\texttt{{\char`\~}1}.

A bibliographic reference is indicated with the reference name
in brackets, such as  \texttt{Theorem 3.1 of
[Monk] p.\ 22}.
See Section~\ref{htmlout} (p.~\pageref{htmlout}) for
syntax details.


\subsection{\texttt{write recent\_additions}
Command}\index{\texttt{write recent{\char`\_}additions} command}
Syntax:  \texttt{write recent{\char`\_}additions} {\em filename}
[\texttt{/limit} {\em number}]

This command reads an existing ``Recent Additions'' {\sc html} file,
normally called \texttt{mmrecent.html}, and updates it with the
descriptions of the most recently added theorems to the database.
 The file is updated between
the {\sc html} comment lines \texttt{<!-- {\char`\#}START{\char`\#} -->}
and \texttt{<!-- {\char`\#}END{\char`\#} -->}.  The original input file
is renamed to {\em filename}\texttt{{\char`\~}1}.

Optional command qualifier:

    \texttt{/limit} {\em number} -
 This qualifier specifies the number of most recent theorems to
   write to the output file.  The default is 100.


\section{Text File Utilities}

\subsection{\texttt{tools} Command}\index{\texttt{tools} command}
Syntax:  \texttt{tools}

This command invokes an easy-to-use, general purpose utility for
manipulating the contents of {\sc ascii} text files.  Upon typing
\texttt{tools}, the command-line prompt will change to \texttt{TOOLS>}
until you type \texttt{exit}.  The \texttt{tools} commands can be used
to perform simple, global edits on an input/output file,
such as making a character string substitution on each line, adding a
string to each line, and so on.  A typical use of this utility is
to build a \texttt{submit} input file to perform a common operation on a
list of statements obtained from \texttt{show label} or \texttt{show
usage}.

The actions of most of the \texttt{tools} commands can also be
performed with equivalent (and more powerful) Unix shell commands, and
some users may find those more efficient.  But for Windows users or
users not comfortable with Unix, \texttt{tools} provides an
easy-to-learn alternative that is adequate for most of the
script-building tasks needed to use the Metamath program effectively.

\subsection{\texttt{help} Command (in \texttt{tools})}
Syntax:  \texttt{help}

The \texttt{help} command lists the commands available in the
\texttt{tools} utility, along with a brief description.  Each command,
in turn, has its own help, such as \texttt{help add}.  As with
Metamath's \texttt{MM>} prompt, a complete command can be entered at
once, or just the command word can be typed, causing the program to
prompt for each argument.

\vskip 1ex
\noindent Line-by-line editing commands:

  \texttt{add} - Add a specified string to each line in a file.

  \texttt{clean} - Trim spaces and tabs on each line in a file; convert
         characters.

  \texttt{delete} - Delete a section of each line in a file.

  \texttt{insert} - Insert a string at a specified column in each line of
        a file.

  \texttt{substitute} - Make a simple substitution on each line of the file.

  \texttt{tag} - Like \texttt{add}, but restricted to a range of lines.

  \texttt{swap} - Swap the two halves of each line in a file.

\vskip 1ex
\noindent Other file-processing commands:

  \texttt{break} - Break up (tokenize) a file into a list of tokens (one per
        line).

  \texttt{build} - Build a file with multiple tokens per line from a list.

  \texttt{count} - Count the occurrences in a file of a specified string.

  \texttt{number} - Create a list of numbers.

  \texttt{parallel} - Put two files in parallel.

  \texttt{reverse} - Reverse the order of the lines in a file.

  \texttt{right} - Right-justify lines in a file (useful before sorting
         numbers).

%  \texttt{tag} - Tag edit updates in a program for revision control.

  \texttt{sort} - Sort the lines in a file with key starting at
         specified string.

  \texttt{match} - Extract lines containing (or not) a specified string.

  \texttt{unduplicate} - Eliminate duplicate occurrences of lines in a file.

  \texttt{duplicate} - Extract first occurrence of any line occurring
         more than

   \ \ \    once in a file, discarding lines occurring exactly once.

  \texttt{unique} - Extract lines occurring exactly once in a file.

  \texttt{type} (10 lines) - Display the first few lines in a file.
                  Similar to Unix \texttt{head}.

  \texttt{copy} - Similar to Unix \texttt{cat} but safe (same input
         and output file allowed).

  \texttt{submit} - Run a script containing \texttt{tools} commands.

\vskip 1ex

\noindent Note:
  \texttt{unduplicate}, \texttt{duplicate}, and \texttt{unique} also
 sort the lines as a side effect.


\subsection{Using \texttt{tools} to Build Metamath \texttt{submit}
Scripts}

The \texttt{break} command is typically used to break up a series of
statement labels, such as the output of Metamath's \texttt{show usage},
into one label per line.  The other \texttt{tools} commands can then be
used to add strings before and after each statement label to specify
commands to be performed on the statement.  The \texttt{parallel}
command is useful when a statement label must be mentioned more than
once on a line.

Very often a \texttt{submit} script for Metamath will require multiple
command lines for each statement being processed.  For example, you may
want to enter the Proof Assistant, \texttt{minimize{\char`\_}with} your
latest theorem, \texttt{save} the new proof, and \texttt{exit} the Proof
Assistant.  To accomplish this, you can build a file with these four
commands for each statement on a single line, separating each command
with a designated character such as \texttt{@}.  Then at the end you can
\texttt{substitute} each \texttt{@} with \texttt{{\char`\\}n} to break
up the lines into individual command lines (see \texttt{help
substitute}).


\subsection{Example of a \texttt{tools} Session}

To give you a quick feel for the \texttt{tools} utility, we show a
simple session where we create a file \texttt{n.txt} with 3 lines, add
strings before and after each line, and display the lines on the screen.
You can experiment with the various commands to gain experience with the
\texttt{tools} utility.

\begin{verbatim}
MM> tools
Entering the Text Tools utilities.
Type HELP for help, EXIT to exit.
TOOLS> number
Output file <n.tmp>? n.txt
First number <1>?
Last number <10>? 3
Increment <1>?
TOOLS> add
Input/output file? n.txt
String to add to beginning of each line <>? This is line
String to add to end of each line <>? .
The file n.txt has 3 lines; 3 were changed.
First change is on line 1:
This is line 1.
TOOLS> type n.txt
This is line 1.
This is line 2.
This is line 3.
TOOLS> exit
Exiting the Text Tools.
Type EXIT again to exit Metamath.
MM>
\end{verbatim}



\appendix
\chapter{Sample Representations}
\label{ASCII}

This Appendix provides a sample of {\sc ASCII} representations,
their corresponding traditional mathematical symbols,
and a discussion of their meanings
in the \texttt{set.mm} database.
These are provided in order of appearance.
This is only a partial list, and new definitions are routinely added.
A complete list is available at \url{http://metamath.org}.

These {\sc ASCII} representations, along
with information on how to display them,
are defined in the \texttt{set.mm} database file inside
a special comment called a \texttt{\$t} {\em
comment}\index{\texttt{\$t} comment} or {\em typesetting
comment.}\index{typesetting comment}
A typesetting comment
is indicated by the appearance of the
two-character string \texttt{\$t} at the beginning of the comment.
For more information,
see Section~\ref{tcomment}, p.~\pageref{tcomment}.

In the following table the ``{\sc ASCII}'' column shows the {\sc ASCII}
representation,
``Symbol'' shows the mathematical symbolic display
that corresponds to that {\sc ASCII} representation, ``Labels'' shows
the key label(s) that define the representation, and
``Description'' provides a description about the symbol.
As usual, ``iff'' is short for ``if and only if.''\index{iff}
In most cases the ``{\sc ASCII}'' column only shows
the key token, but it will sometimes show a sequence of tokens
if that is necessary for clarity.

{\setlength{\extrarowsep}{4pt} % Keep rows from being too close together
\begin{longtabu}   { @{} c c l X }
\textbf{ASCII} & \textbf{Symbol} & \textbf{Labels} & \textbf{Description} \\
\endhead
\texttt{|-} & $\vdash$ & &
  ``It is provable that...'' \\
\texttt{ph} & $\varphi$ & \texttt{wph} &
  The wff (boolean) variable phi,
  conventionally the first wff variable. \\
\texttt{ps} & $\psi$ & \texttt{wps} &
  The wff (boolean) variable psi,
  conventionally the second wff variable. \\
\texttt{ch} & $\chi$ & \texttt{wch} &
  The wff (boolean) variable chi,
  conventionally the third wff variable. \\
\texttt{-.} & $\lnot$ & \texttt{wn} &
  Logical not. E.g., if $\varphi$ is true, then $\lnot \varphi$ is false. \\
\texttt{->} & $\rightarrow$ & \texttt{wi} &
  Implies, also known as material implication.
  In classical logic the expression $\varphi \rightarrow \psi$ is true
  if either $\varphi$ is false or $\psi$ is true (or both), that is,
  $\varphi \rightarrow \psi$ has the same meaning as
  $\lnot \varphi \lor \psi$ (as proven in theorem \texttt{imor}). \\
\texttt{<->} & $\leftrightarrow$ &
  \hyperref[df-bi]{\texttt{df-bi}} &
  Biconditional (aka is-equals for boolean values).
  $\varphi \leftrightarrow \psi$ is true iff
  $\varphi$ and $\psi$ have the same value. \\
\texttt{\char`\\/} & $\lor$ &
  \makecell[tl]{{\hyperref[df-or]{\texttt{df-or}}}, \\
	         \hyperref[df-3or]{\texttt{df-3or}}} &
  Disjunction (logical ``or''). $\varphi \lor \psi$ is true iff
  $\varphi$, $\psi$, or both are true. \\
\texttt{/\char`\\} & $\land$ &
  \makecell[tl]{{\hyperref[df-an]{\texttt{df-an}}}, \\
                 \hyperref[df-3an]{\texttt{df-3an}}} &
  Conjunction (logical ``and''). $\varphi \land \psi$ is true iff
  both $\varphi$ and $\psi$ are true. \\
\texttt{A.} & $\forall$ &
  \texttt{wal} &
  For all; the wff $\forall x \varphi$ is true iff
  $\varphi$ is true for all values of $x$. \\
\texttt{E.} & $\exists$ &
  \hyperref[df-ex]{\texttt{df-ex}} &
  There exists; the wff
  $\exists x \varphi$ is true iff
  there is at least one $x$ where $\varphi$ is true. \\
\texttt{[ y / x ]} & $[ y / x ]$ &
  \hyperref[df-sb]{\texttt{df-sb}} &
  The wff $[ y / x ] \varphi$ produces
  the result when $y$ is properly substituted for $x$ in $\varphi$
  ($y$ replaces $x$).
  % This is elsb4
  % ( [ x / y ] z e. y <-> z e. x )
  For example,
  $[ x / y ] z \in y$ is the same as $z \in x$. \\
\texttt{E!} & $\exists !$ &
  \hyperref[df-eu]{\texttt{df-eu}} &
  There exists exactly one;
  $\exists ! x \varphi$ is true iff
  there is at least one $x$ where $\varphi$ is true. \\
\texttt{\{ y | phi \}}  & $ \{ y | \varphi \}$ &
  \hyperref[df-clab]{\texttt{df-clab}} &
  The class of all sets where $\varphi$ is true. \\
\texttt{=} & $ = $ &
  \hyperref[df-cleq]{\texttt{df-cleq}} &
  Class equality; $A = B$ iff $A$ equals $B$. \\
\texttt{e.} & $ \in $ &
  \hyperref[df-clel]{\texttt{df-clel}} &
  Class membership; $A \in B$ if $A$ is a member of $B$. \\
\texttt{{\char`\_}V} & {\rm V} &
  \hyperref[df-v]{\texttt{df-v}} &
  Class of all sets (not itself a set). \\
\texttt{C\_} & $ \subseteq $ &
  \hyperref[df-ss]{\texttt{df-ss}} &
  Subclass (subset); $A \subseteq B$ is true iff
  $A$ is a subclass of $B$. \\
\texttt{u.} & $ \cup $ &
  \hyperref[df-un]{\texttt{df-un}} &
  $A \cup B$ is the union of classes $A$ and $B$. \\
\texttt{i^i} & $ \cap $ &
  \hyperref[df-in]{\texttt{df-in}} &
  $A \cap B$ is the intersection of classes $A$ and $B$. \\
\texttt{\char`\\} & $ \setminus $ &
  \hyperref[df-dif]{\texttt{df-dif}} &
  $A \setminus B$ (set difference)
  is the class of all sets in $A$ except for those in $B$. \\
\texttt{(/)} & $ \varnothing $ &
  \hyperref[df-nul]{\texttt{df-nul}} &
  $ \varnothing $ is the empty set (aka null set). \\
\texttt{\char`\~P} & $ \cal P $ &
  \hyperref[df-pw]{\texttt{df-pw}} &
  Power class. \\
\texttt{<.\ A , B >.} & $\langle A , B \rangle$ &
  \hyperref[df-op]{\texttt{df-op}} &
  The ordered pair $\langle A , B \rangle$. \\
\texttt{( F ` A )} & $ ( F ` A ) $ &
  \hyperref[df-fv]{\texttt{df-fv}} &
  The value of function $F$ when applied to $A$. \\
\texttt{\_i} & $ i $ &
  \texttt{df-i} &
  The square root of negative one. \\
\texttt{x.} & $ \cdot $ &
  \texttt{df-mul} &
  Complex number multiplication; $2~\cdot~3~=~6$. \\
\texttt{CC} & $ \mathbb{C} $ &
  \texttt{df-c} &
  The set of complex numbers. \\
\texttt{RR} & $ \mathbb{R} $ &
  \texttt{df-r} &
  The set of real numbers. \\
\end{longtabu}
} % end of extrarowsep

\chapter{Compressed Proofs}
\label{compressed}\index{compressed proof}\index{proof!compressed}

The proofs in the \texttt{set.mm} set theory database are stored in compressed
format for efficiency.  Normally you needn't concern yourself with the
compressed format, since you can display it with the usual proof display tools
in the Metamath program (\texttt{show proof}\ldots) or convert it to the normal
RPN proof format described in Section~\ref{proof} (with \texttt{save proof}
{\em label} \texttt{/normal}).  However for sake of completeness we describe the
format here and show how it maps to the normal RPN proof format.

A compressed proof, located between \texttt{\$=} and \texttt{\$.}\ keywords, consists
of a left parenthesis, a sequence of statement labels, a right parenthesis,
and a sequence of upper-case letters \texttt{A} through \texttt{Z} (with optional
white space between them).  White space must surround the parentheses
and the labels.  The left parenthesis tells Metamath that a
compressed proof follows.  (A normal RPN proof consists of just a sequence of
labels, and a parenthesis is not a legal character in a label.)

The sequence of upper-case letters corresponds to a sequence of integers
with the following mapping.  Each integer corresponds to a proof step as
described later.
\begin{center}
  \texttt{A} = 1 \\
  \texttt{B} = 2 \\
   \ldots \\
  \texttt{T} = 20 \\
  \texttt{UA} = 21 \\
  \texttt{UB} = 22 \\
   \ldots \\
  \texttt{UT} = 40 \\
  \texttt{VA} = 41 \\
  \texttt{VB} = 42 \\
   \ldots \\
  \texttt{YT} = 120 \\
  \texttt{UUA} = 121 \\
   \ldots \\
  \texttt{YYT} = 620 \\
  \texttt{UUUA} = 621 \\
   etc.
\end{center}

In other words, \texttt{A} through \texttt{T} represent the
least-significant digit in base 20, and \texttt{U} through \texttt{Y}
represent zero or more most-significant digits in base 5, where the
digits start counting at 1 instead of the usual 0. With this scheme, we
don't need white space between these ``numbers.''

(In the design of the compressed proof format, only upper-case letters,
as opposed to say all non-whitespace printable {\sc ascii} characters other than
%\texttt{\$}, was chosen to make the compressed proof a little less
%displeasing to the eye, at the expense of a typical 20\% compression
\texttt{\$}, were chosen so as not to collide with most text editor
searches, at the expense of a typical 20\% compression
loss.  The base 5/base 20 grouping, as opposed to say base 6/base 19,
was chosen by experimentally determining the grouping that resulted in
best typical compression.)

The letter \texttt{Z} identifies (tags) a proof step that is identical to one
that occurs later on in the proof; it helps shorten the proof by not requiring
that identical proof steps be proved over and over again (which happens often
when building wff's).  The \texttt{Z} is placed immediately after the
least-significant digit (letters \texttt{A} through \texttt{T}) that ends the integer
corresponding to the step to later be referenced.

The integers that the upper-case letters correspond to are mapped to labels as
follows.  If the statement being proved has $m$ mandatory hypotheses, integers
1 through $m$ correspond to the labels of these hypotheses in the order shown
by the \texttt{show statement ... / full} command, i.e., the RPN order\index{RPN
order} of the mandatory
hypotheses.  Integers $m+1$ through $m+n$ correspond to the labels enclosed in
the parentheses of the compressed proof, in the order that they appear, where
$n$ is the number of those labels.  Integers $m+n+1$ on up don't directly
correspond to statement labels but point to proof steps identified with the
letter \texttt{Z}, so that these proof steps can be referenced later in the
proof.  Integer $m+n+1$ corresponds to the first step tagged with a \texttt{Z},
$m+n+2$ to the second step tagged with a \texttt{Z}, etc.  When the compressed
proof is converted to a normal proof, the entire subproof of a step tagged
with \texttt{Z} replaces the reference to that step.

For efficiency, Metamath works with compressed proofs directly, without
converting them internally to normal proofs.  In addition to the usual
error-checking, an error message is given if (1) a label in the label list in
parentheses does not refer to a previous \texttt{\$p} or \texttt{\$a} statement or a
non-mandatory hypothesis of the statement being proved and (2) a proof step
tagged with \texttt{Z} is referenced before the step tagged with the \texttt{Z}.

Just as in a normal proof under development (Section~\ref{unknown}), any step
or subproof that is not yet known may be represented with a single \texttt{?}.
White space does not have to appear between the \texttt{?}\ and the upper-case
letters (or other \texttt{?}'s) representing the remainder of the proof.

% April 1, 2004 Appendix C has been added back in with corrections.
%
% May 20, 2003 Appendix C was removed for now because there was a problem found
% by Bob Solovay
%
% Also, removed earlier \ref{formalspec} 's (3 cases above)
%
% Bob Solovay wrote on 30 Nov 2002:
%%%%%%%%%%%%% (start of email comment )
%      3. My next set of comments concern appendix C. I read this before I
% read Chapter 4. So I first noted that the system as presented in the
% Appendix lacked a certain formal property that I thought desirable. I
% then came up with a revised formal system that had this property. Upon
% reading Chapter 4, I noticed that the revised system was closer to the
% treatment in Chapter 4 than the system in Appendix C.
%
%         First a very minor correction:
%
%         On page 142 line 2: The condition that V(e) != V(f) should only be
% required of e, f in T such that e != f.
%
%         Here is a natural property [transitivity] that one would like
% the formal system to have:
%
%         Let Gamma be a set of statements. Suppose that the statement Phi
% is provable from Gamma and that the statement Psi is provable from Gamma
% \cup {Phi}. Then Psi is provable from Gamma.
%
%         I shall present an example to show that this property does not
% hold for the formal systems of Appendix C:
%
%         I write the example in metamath style:
%
% $c A B C D E $.
% $v x y
%
% ${
% tx $f A x $.
% ty $f B y $.
% ax1 $a C x y $.
% $}
%
% ${
% tx $f A x $.
% ty $f B y $.
% ax2-h1 $e C x y $.
% ax2 $a D y $.
% $}
%
% ${
% ty $f B y $.
% ax3-h1 $e D y $.
% ax3 $a E y $.
% $}
%
% $(These three axioms are Gamma $)
%
% ${
% tx $f A x $.
% ty $f B y $.
% Phi $p D y $=
% tx ty tx ty ax1 ax2 $.
% $}
%
% ${
% ty $f B y $.
% Psi $p E y $=
% ty ty Phi ax3 $.
% $}
%
%
% I omit the formal proofs of the following claims. [I will be glad to
% supply them upon request.]
%
% 1) Psi is not provable from Gamma;
%
% 2) Psi is provable from Gamma + Phi.
%
% Here "provable" refers to the formalism of Appendix C.
%
% The trouble of course is that Psi is lacking the variable declaration
%
% $f Ax $.
%
% In the Metamath system there is no trouble proving Psi. I attach a
% metamath file that shows this and which has been checked by the
% metamath program.
%
% I next want to indicate how I think the treatment in Appendix C should
% be revised so as to conform more closely to the metamath system of the
% main text. The revised system *does* have the transitivity property.
%
% We want to give revised definitions of "statement" and
% "provable". [cf. sections C.2.4. and C.2.5] Our new definitions will
% use the definitions given in Appendix C. So we take the following
% tack. We refer to the original notions as o-statement and o-provable. And
% we refer to the notions we are defining as n-statement and n-provable.
%
%         A n-statement is an o-statement in which the only variables
% that appear in the T component are mandatory.
%
%         To any o-statement we can associate its reduct which is a
% n-statement by dropping all the elements of T or D which contain
% non-mandatory variables.
%
%         An n-statement gamma is n-provable if there is an o-statement
% gamma' which has gamma as its reduct andf such that gamma' is
% o-provable.
%
%         It seems to me [though I am not completely sure on this point]
% that n-provability corresponds to metamath provability as discussed
% say in Chapter 4.
%
%         Attached to this letter is the metamath proof of Phi and Psi
% from Gamma discussed above.
%
%         I am still brooding over the question of whether metamath
% correctly formalizes set-theory. No doubt I will have some questions
% re this after my thoughts become clearer.
%%%%%%%%%%%%%%%% (end of email comment)

%%%%%%%%%%%%%%%% (start of 2nd email comment from Bob Solovay 1-Apr-04)
%
%         I hope that Appendix C is the one that gives a "formal" treatment
% of Metamath. At any rate, thats the appendix I want to comment on.
%
%         I'm going to suggest two changes in the definition.
%
%         First change (in the definition of statement): Require that the
% sets D, T, and E be finite.
%
%         Probably things are fine as you give them. But in the applications
% to the main metamath system they will always be finite, and its useful in
% thinking about things [at least for me] to stick to the finite case.
%
%         Second change:
%
%         First let me give an approximate description. Remove the dummy
% variables from the statement. Instead, include them in the proof.
%
%         More formally: Require that T consists of type declarations only
% for mandatory variables. Require that all the pairs in D consist of
% mandatory variables.
%
%         At the start of a proof we are allowed to declare a finite number
% of dummy variables [provided that none of them appear in any of the
% statements in E \cup {A}. We have to supply type declarations for all the
% dummy variables. We are allowed to add new $d statements referring to
% either the mandatory or dummy variables. But we require that no new $d
% statement references only mandatory variables.
%
%         I find this way of doing things more conceptual than the treatment
% in Appendix C. But the change [which I will use implicitly in later
% letters about doing Peano] is mainly aesthetic. I definitely claim that my
% results on doing Peano all apply to Metamath as it is presented in your
% book.
%
%         --Bob
%
%%%%%%%%%%%%%%%% (end of 2nd email comment)

%%
%% When uncommenting the below, also uncomment references above to {formalspec}
%%
\chapter{Metamath's Formal System}\label{formalspec}\index{Metamath!as a formal
system}

\section{Introduction}

\begin{quote}
  {\em Perfection is when there is no longer anything more to take away.}
    \flushright\sc Antoine de
     Saint-Exupery\footnote{\cite[p.~3-25]{Campbell}.}\\
\end{quote}\index{de Saint-Exupery, Antoine}

This appendix describes the theory behind the Metamath language in an abstract
way intended for mathematicians.  Specifically, we construct two
set-theo\-ret\-i\-cal objects:  a ``formal system'' (roughly, a set of syntax
rules, axioms, and logical rules) and its ``universe'' (roughly, the set of
theorems derivable in the formal system).  The Metamath computer language
provides us with a way to describe specific formal systems and, with the aid of
a proof provided by the user, to verify that given theorems
belong to their universes.

To understand this appendix, you need a basic knowledge of informal set theory.
It should be sufficient to understand, for example, Ch.\ 1 of Munkres' {\em
Topology} \cite{Munkres}\index{Munkres, James R.} or the
introductory set theory chapter
in many textbooks that introduce abstract mathematics. (Note that there are
minor notational differences among authors; e.g.\ Munkres uses $\subset$ instead
of our $\subseteq$ for ``subset.''  We use ``included in'' to mean ``a subset
of,'' and ``belongs to'' or ``is contained in'' to mean ``is an element of.'')
What we call a ``formal'' description here, unlike earlier, is actually an
informal description in the ordinary language of mathematicians.  However we
provide sufficient detail so that a mathematician could easily formalize it,
even in the language of Metamath itself if desired.  To understand the logic
examples at the end of this appendix, familiarity with an introductory book on
mathematical logic would be helpful.

\section{The Formal Description}

\subsection[Preliminaries]{Preliminaries\protect\footnotemark}%
\footnotetext{This section is taken mostly verbatim
from Tarski \cite[p.~63]{Tarski1965}\index{Tarski, Alfred}.}

By $\omega$ we denote the set of all natural numbers (non-negative integers).
Each natural number $n$ is identified with the set of all smaller numbers: $n =
\{ m | m < n \}$.  The formula $m < n$ is thus equivalent to the condition: $m
\in n$ and $m,n \in \omega$. In particular, 0 is the number zero and at the
same time the empty set $\varnothing$, $1=\{0\}$, $2=\{0,1\}$, etc. ${}^B A$
denotes the set of all functions on $B$ to $A$ (i.e.\ with domain $B$ and range
included in $A$).  The members of ${}^\omega A$ are what are called {\em simple
infinite sequences},\index{simple infinite sequence}
with all {\em terms}\index{term} in $A$.  In case $n \in \omega$, the
members of ${}^n A$ are referred to as {\em finite $n$-termed
sequences},\index{finite $n$-termed
sequence} again
with terms in $A$.  The consecutive terms (function values) of a finite or
infinite sequence $f$ are denoted by $f_0, f_1, \ldots ,f_n,\ldots$.  Every
finite sequence $f \in \bigcup _{n \in \omega} {}^n A$ uniquely determines the
number $n$ such that $f \in {}^n A$; $n$ is called the {\em
length}\index{length of a sequence ({$"|\ "|$})} of $f$ and
is denoted by $|f|$.  $\langle a \rangle$ is the sequence $f$ with $|f|=1$ and
$f_0=a$; $\langle a,b \rangle$ is the sequence $f$ with $|f|=2$, $f_0=a$,
$f_1=b$; etc.  Given two finite sequences $f$ and $g$, we denote by $f\frown g$
their {\em concatenation},\index{concatenation} i.e., the
finite sequence $h$ determined by the
conditions:
\begin{eqnarray*}
& |h| = |f|+|g|;&  \\
& h_n = f_n & \mbox{\ for\ } n < |f|;  \\
& h_{|f|+n} = g_n & \mbox{\ for\ } n < |g|.
\end{eqnarray*}

\subsection{Constants, Variables, and Expressions}

A formal system has a set of {\em symbols}\index{symbol!in
a formal system} denoted
by $\mbox{\em SM}$.  A
precise set-theo\-ret\-i\-cal definition of this set is unimportant; a symbol
could be considered a primitive or atomic element if we wish.  We assume this
set is divided into two disjoint subsets:  a set $\mbox{\em CN}$ of {\em
constants}\index{constant!in a formal system} and a set $\mbox{\em VR}$ of
{\em variables}.\index{variable!in a formal system}  $\mbox{\em CN}$ and
$\mbox{\em VR}$ are each assumed to consist of countably many symbols which
may be arranged in finite or simple infinite sequences $c_0, c_1, \ldots$ and
$v_0, v_1, \ldots$ respectively, without repeating terms.  We will represent
arbitrary symbols by metavariables $\alpha$, $\beta$, etc.

{\footnotesize\begin{quotation}
{\em Comment.} The variables $v_0, v_1, \ldots$ of our formal system
correspond to what are usually considered ``metavariables'' in
descriptions of specific formal systems in the literature.  Typically,
when describing a specific formal system a book will postulate a set of
primitive objects called variables, then proceed to describe their
properties using metavariables that range over them, never mentioning
again the actual variables themselves.  Our formal system does not
mention these primitive variable objects at all but deals directly with
metavariables, as its primitive objects, from the start.  This is a
subtle but key distinction you should keep in mind, and it makes our
definition of ``formal system'' somewhat different from that typically
found in the literature.  (So, the $\alpha$, $\beta$, etc.\ above are
actually ``metametavariables'' when used to represent $v_0, v_1,
\ldots$.)
\end{quotation}}

Finite sequences all terms of which are symbols are called {\em
expressions}.\index{expression!in a formal system}  $\mbox{\em EX}$ is
the set of all expressions; thus
\begin{displaymath}
\mbox{\em EX} = \bigcup _{n \in \omega} {}^n \mbox{\em SM}.
\end{displaymath}

A {\em constant-prefixed expression}\index{constant-prefixed expression}
is an expression of non-zero length
whose first term is a constant.  We denote the set of all constant-prefixed
expressions by $\mbox{\em EX}_C = \{ e \in \mbox{\em EX} | ( |e| > 0 \wedge
e_0 \in \mbox{\em CN} ) \}$.

A {\em constant-variable pair}\index{constant-variable pair}
is an expression of length 2 whose first term
is a constant and whose second term is a variable.  We denote the set of all
constant-variable pairs by $\mbox{\em EX}_2 = \{ e \in \mbox{\em EX}_C | ( |e|
= 2 \wedge e_1 \in \mbox{\em VR} ) \}$.


{\footnotesize\begin{quotation}
{\em Relationship to Metamath.} In general, the set $\mbox{\em SM}$
corresponds to the set of declared math symbols in a Metamath database, the
set $\mbox{\em CN}$ to those declared with \texttt{\$c} statements, and the set
$\mbox{\em VR}$ to those declared with \texttt{\$v} statements.  Of course a
Metamath database can only have a finite number of math symbols, whereas
formal systems in general can have an infinite number, although the number of
Metamath math symbols available is in principle unlimited.

The set $\mbox{\em EX}_C$ corresponds to the set of permissible expressions
for \texttt{\$e}, \texttt{\$a}, and \texttt{\$p} statements.  The set $\mbox{\em EX}_2$
corresponds to the set of permissible expressions for \texttt{\$f} statements.
\end{quotation}}

We denote by ${\cal V}(e)$ the set of all variables in an expression $e \in
\mbox{\em EX}$, i.e.\ the set of all $\alpha \in \mbox{\em VR}$ such that
$\alpha = e_n$ for some $n < |e|$.  We also denote (with abuse of notation) by
${\cal V}(E)$ the set of all variables in a collection of expressions $E
\subseteq \mbox{\em EX}$, i.e.\ $\bigcup _{e \in E} {\cal V}(e)$.


\subsection{Substitution}

Given a function $F$ from $\mbox{\em VR}$ to
$\mbox{\em EX}$, we
denote by $\sigma_{F}$ or just $\sigma$ the function from $\mbox{\em EX}$ to
$\mbox{\em EX}$ defined recursively for nonempty sequences by
\begin{eqnarray*}
& \sigma(<\alpha>) = F(\alpha) & \mbox{for\ } \alpha \in \mbox{\em VR}; \\
& \sigma(<\alpha>) = <\alpha> & \mbox{for\ } \alpha \not\in \mbox{\em VR}; \\
& \sigma(g \frown h) = \sigma(g) \frown
    \sigma(h) & \mbox{for\ } g,h \in \mbox{\em EX}.
\end{eqnarray*}
We also define $\sigma(\varnothing)=\varnothing$.  We call $\sigma$ a {\em
simultaneous substitution}\index{substitution!variable}\index{variable
substitution} (or just {\em substitution}) with {\em substitution
map}\index{substitution map} $F$.

We also denote (with abuse of notation) by $\sigma(E)$ a substitution on a
collection of expressions $E \subseteq \mbox{\em EX}$, i.e.\ the set $\{
\sigma(e) | e \in E \}$.  The collection $\sigma(E)$ may of course contain
fewer expressions than $E$ because duplicate expressions could result from the
substitution.

\subsection{Statements}

We denote by $\mbox{\em DV}$ the set of all
unordered pairs $\{\alpha, \beta \} \subseteq \mbox{\em VR}$ such that $\alpha
\neq \beta$.  $\mbox{\em DV}$ stands for ``distinct variables.''

A {\em pre-statement}\index{pre-statement!in a formal system} is a
quadruple $\langle D,T,H,A \rangle$ such that
$D\subseteq \mbox{\em DV}$, $T\subseteq \mbox{\em EX}_2$, $H\subseteq
\mbox{\em EX}_C$ and $H$ is finite,
$A\in \mbox{\em EX}_C$, ${\cal V}(H\cup\{A\}) \subseteq
{\cal V}(T)$, and $\forall e,f\in T {\ } {\cal V}(e) \neq {\cal V}(f)$ (or
equivalently, $e_1 \ne f_1$) whenever $e \neq f$. The terms of the quadruple are called {\em
distinct-variable restrictions},\index{disjoint-variable restriction!in a
formal system} {\em variable-type hypotheses},\index{variable-type
hypothesis!in a formal system} {\em logical hypotheses},\index{logical
hypothesis!in a formal system} and the {\em assertion}\index{assertion!in a
formal system} respectively.  We denote by $T_M$ ({\em mandatory variable-type
hypotheses}\index{mandatory variable-type hypothesis!in a formal system}) the
subset of $T$ such that ${\cal V}(T_M) ={\cal V}(H \cup \{A\})$.  We denote by
$D_M=\{\{\alpha,\beta\}\in D|\{\alpha,\beta\}\subseteq {\cal V}(T_M)\}$ the
{\em mandatory distinct-variable restrictions}\index{mandatory
disjoint-variable restriction!in a formal system} of the pre-statement.
The set
of {\em mandatory hypotheses}\index{mandatory hypothesis!in a formal system}
is $T_M\cup H$.  We call the quadruple $\langle D_M,T_M,H,A \rangle$
the {\em reduct}\index{reduct!in a formal system} of
the pre-statement $\langle D,T,H,A \rangle$.

A {\em statement} is the reduct of some pre-statement\index{statement!in a
formal system}.  A statement is therefore a special kind of pre-statement;
in particular, a statement is the reduct of itself.

{\footnotesize\begin{quotation}
{\em Comment.}  $T$ is a set of expressions, each of length 2, that associate
a set of constants (``variable types'') with a set of variables.  The
condition ${\cal V}(H\cup\{A\}) \subseteq {\cal V}(T) $
means that each variable occurring in a statement's logical
hypotheses or assertion must have an associated variable-type hypothesis or
``type declaration,'' in  analogy to a computer programming language, where a
variable must be declared to be say, a string or an integer.  The requirement
that $\forall e,f\in T \, e_1 \ne f_1$ for $e\neq f$
means that each variable must be
associated with a unique constant designating its variable type; e.g., a
variable might be a ``wff'' or a ``set'' but not both.

Distinct-variable restrictions are used to specify what variable substitutions
are permissible to make for the statement to remain valid.  For example, in
the theorem scheme of set theory $\lnot\forall x\,x=y$ we may not substitute
the same variable for both $x$ and $y$.  On the other hand, the theorem scheme
$x=y\to y=x$ does not require that $x$ and $y$ be distinct, so we do not
require a distinct-variable restriction, although having one
would cause no harm other than making the scheme less general.

A mandatory variable-type hypothesis is one whose variable exists in a logical
hypothesis or the assertion.  A provable pre-statement
(defined below) may require
non-mandatory variable-type hypotheses that effectively introduce ``dummy''
variables for use in its proof.  Any number of dummy variables might
be required by a specific proof; indeed, it has been shown by H.\
Andr\'{e}ka\index{Andr{\'{e}}ka, H.} \cite{Nemeti} that there is no finite
upper bound to the number of dummy variables needed to prove an arbitrary
theorem in first-order logic (with equality) having a fixed number $n>2$ of
individual variables.  (See also the Comment on p.~\pageref{nodd}.)
For this reason we do not set a finite size bound on the collections $D$ and
$T$, although in an actual application (Metamath database) these will of
course be finite, increased to whatever size is necessary as more
proofs are added.
\end{quotation}}

{\footnotesize\begin{quotation}
{\em Relationship to Metamath.} A pre-statement of a formal system
corresponds to an extended frame in a Metamath database
(Section~\ref{frames}).  The collections $D$, $T$, and $H$ correspond
respectively to the \texttt{\$d}, \texttt{\$f}, and \texttt{\$e}
statement collections in an extended frame.  The expression $A$
corresponds to the \texttt{\$a} (or \texttt{\$p}) statement in an
extended frame.

A statement of a formal system corresponds to a frame in a Metamath
database.
\end{quotation}}

\subsection{Formal Systems}

A {\em formal system}\index{formal system} is a
triple $\langle \mbox{\em CN},\mbox{\em
VR},\Gamma\rangle$ where $\Gamma$ is a set of statements.  The members of
$\Gamma$ are called {\em axiomatic statements}.\index{axiomatic
statement!in a formal system}  Sometimes we will refer to a
formal system by just $\Gamma$ when $\mbox{\em CN}$ and $\mbox{\em VR}$ are
understood.

Given a formal system $\Gamma$, the {\em closure}\index{closure}\footnote{This
definition of closure incorporates a simplification due to
Josh Purinton.\index{Purinton, Josh}.} of a
pre-statement
$\langle D,T,H,A \rangle$ is the smallest set $C$ of expressions
such that:
%\begin{enumerate}
%  \item $T\cup H\subseteq C$; and
%  \item If for some axiomatic statement
%    $\langle D_M',T_M',H',A' \rangle \in \Gamma_A$, for
%    some $E \subseteq C$, some $F \subseteq C-T$ (where ``-'' denotes
%    set difference), and some substitution
%    $\sigma$ we have
%    \begin{enumerate}
%       \item $\sigma(T_M') = E$ (where, as above, the $M$ denotes the
%           mandatory variable-type hypotheses of $T^A$);
%       \item $\sigma(H') = F$;
%       \item for all $\{\alpha,\beta\}\in D^A$ and $\subseteq
%         {\cal V}(T_M')$, for all $\gamma\in {\cal V}(\sigma(\langle \alpha
%         \rangle))$, and for all $\delta\in  {\cal V}(\sigma(\langle \beta
%         \rangle))$, we have $\{\gamma, \delta\} \in D$;
%   \end{enumerate}
%   then $\sigma(A') \in C$.
%\end{enumerate}
\begin{list}{}{\itemsep 0.0pt}
  \item[1.] $T\cup H\subseteq C$; and
  \item[2.] If for some axiomatic statement
    $\langle D_M',T_M',H',A' \rangle \in
       \Gamma$ and for some substitution
    $\sigma$ we have
    \begin{enumerate}
       \item[a.] $\sigma(T_M' \cup H') \subseteq C$; and
       \item[b.] for all $\{\alpha,\beta\}\in D_M'$, for all $\gamma\in
         {\cal V}(\sigma(\langle \alpha
         \rangle))$, and for all $\delta\in  {\cal V}(\sigma(\langle \beta
         \rangle))$, we have $\{\gamma, \delta\} \in D$;
   \end{enumerate}
   then $\sigma(A') \in C$.
\end{list}
A pre-statement $\langle D,T,H,A
\rangle$ is {\em provable}\index{provable statement!in a formal
system} if $A\in C$ i.e.\ if its assertion belongs to its
closure.  A statement is {\em provable} if it is
the reduct of a provable pre-statement.
The {\em universe}\index{universe of a formal system}
of a formal system is
the collection of all of its provable statements.  Note that the
set of axiomatic statements $\Gamma$ in a formal system is a subset of its
universe.

{\footnotesize\begin{quotation}
{\em Comment.} The first condition in the definition of closure simply says
that the hypotheses of the pre-statement are in its closure.

Condition 2(a) says that a substitution exists that makes the
mandatory hypotheses of an axiomatic statement exactly match some members of
the closure.  This is what we explicitly demonstrate in a Metamath language
proof.

%Conditions 2(a) and 2(b) say that a substitution exists that makes the
%(mandatory) hypotheses of an axiomatic statement exactly match some members of
%the closure.  This is what we explicitly demonstrate with a Metamath language
%proof.
%
%The set of expressions $F$ in condition 2(b) excludes the variable-type
%hypotheses; this is done because non-mandatory variable-type hypotheses are
%effectively ``dropped'' as irrelevant whereas logical hypotheses must be
%retained to achieve a consistent logical system.

Condition 2(b) describes how distinct-variable restrictions in the axiomatic
statement must be met.  It means that after a substitution for two variables
that must be distinct, the resulting two expressions must either contain no
variables, or if they do, they may not have variables in common, and each pair
of any variables they do have, with one variable from each expression, must be
specified as distinct in the original statement.
\end{quotation}}

{\footnotesize\begin{quotation}
{\em Relationship to Metamath.} Axiomatic statements
 and provable statements in a formal
system correspond to the frames for \texttt{\$a} and \texttt{\$p} statements
respectively in a Metamath database.  The set of axiomatic statements is a
subset of the set of provable statements in a formal system, although in a
Metamath database a \texttt{\$a} statement is distinguished by not having a
proof.  A Metamath language proof for a \texttt{\$p} statement tells the computer
how to explicitly construct a series of members of the closure ultimately
leading to a demonstration that the assertion
being proved is in the closure.  The actual closure typically contains
an infinite number of expressions.  A formal system itself does not have
an explicit object called a ``proof'' but rather the existence of a proof
is implied indirectly by membership of an assertion in a provable
statement's closure.  We do this to make the formal system easier
to describe in the language of set theory.

We also note that once established as provable, a statement may be considered
to acquire the same status as an axiomatic statement, because if the set of
axiomatic statements is extended with a provable statement, the universe of
the formal system remains unchanged (provided that $\mbox{\em VR}$ is
infinite).
In practice, this means we can build a hierarchy of provable statements to
more efficiently establish additional provable statements.  This is
what we do in Metamath when we allow proofs to reference previous
\texttt{\$p} statements as well as previous \texttt{\$a} statements.
\end{quotation}}

\section{Examples of Formal Systems}

{\footnotesize\begin{quotation}
{\em Relationship to Metamath.} The examples in this section, except Example~2,
are for the most part exact equivalents of the development in the set
theory database \texttt{set.mm}.  You may want to compare Examples~1, 3, and 5
to Section~\ref{metaaxioms}, Example 4 to Sections~\ref{metadefprop} and
\ref{metadefpred}, and Example 6 to
Section~\ref{setdefinitions}.\label{exampleref}
\end{quotation}}

\subsection{Example~1---Propositional Calculus}\index{propositional calculus}

Classical propositional calculus can be described by the following formal
system.  We assume the set of variables is infinite.  Rather than denoting the
constants and variables by $c_0, c_1, \ldots$ and $v_0, v_1, \ldots$, for
readability we will instead use more conventional symbols, with the
understanding of course that they denote distinct primitive objects.
Also for readability we may omit commas between successive terms of a
sequence; thus $\langle \mbox{wff\ } \varphi\rangle$ denotes
$\langle \mbox{wff}, \varphi\rangle$.

Let
\begin{itemize}
  \item[] $\mbox{\em CN}=\{\mbox{wff}, \vdash, \to, \lnot, (,)\}$
  \item[] $\mbox{\em VR}=\{\varphi,\psi,\chi,\ldots\}$
  \item[] $T = \{\langle \mbox{wff\ } \varphi\rangle,
             \langle \mbox{wff\ } \psi\rangle,
             \langle \mbox{wff\ } \chi\rangle,\ldots\}$, i.e.\ those
             expressions of length 2 whose first member is $\mbox{\rm wff}$
             and whose second member belongs to $\mbox{\em VR}$.\footnote{For
convenience we let $T$ be an infinite set; the definition of a statement
permits this in principle.  Since a Metamath source file has a finite size, in
practice we must of course use appropriate finite subsets of this $T$,
specifically ones containing at least the mandatory variable-type
hypotheses.  Similarly, in the source file we introduce new variables as
required, with the understanding that a potentially infinite number of
them are available.}
\noindent Then $\Gamma$ consists of the axiomatic statements that
are the reducts of the following pre-statements:
    \begin{itemize}
      \item[] $\langle\varnothing,T,\varnothing,
               \langle \mbox{wff\ }(\varphi\to\psi)\rangle\rangle$
      \item[] $\langle\varnothing,T,\varnothing,
               \langle \mbox{wff\ }\lnot\varphi\rangle\rangle$
      \item[] $\langle\varnothing,T,\varnothing,
               \langle \vdash(\varphi\to(\psi\to\varphi))
               \rangle\rangle$
      \item[] $\langle\varnothing,T,
               \varnothing,
               \langle \vdash((\varphi\to(\psi\to\chi))\to
               ((\varphi\to\psi)\to(\varphi\to\chi)))
               \rangle\rangle$
      \item[] $\langle\varnothing,T,
               \varnothing,
               \langle \vdash((\lnot\varphi\to\lnot\psi)\to
               (\psi\to\varphi))\rangle\rangle$
      \item[] $\langle\varnothing,T,
               \{\langle\vdash(\varphi\to\psi)\rangle,
                 \langle\vdash\varphi\rangle\},
               \langle\vdash\psi\rangle\rangle$
    \end{itemize}
\end{itemize}

(For example, the reduct of $\langle\varnothing,T,\varnothing,
               \langle \mbox{wff\ }(\varphi\to\psi)\rangle\rangle$
is
\begin{itemize}
\item[] $\langle\varnothing,
\{\langle \mbox{wff\ } \varphi\rangle,
             \langle \mbox{wff\ } \psi\rangle\},
             \varnothing,
               \langle \mbox{wff\ }(\varphi\to\psi)\rangle\rangle$,
\end{itemize}
which is the first axiomatic statement.)

We call the members of $\mbox{\em VR}$ {\em wff variables} or (in the context
of first-order logic which we will describe shortly) {\em wff metavariables}.
Note that the symbols $\phi$, $\psi$, etc.\ denote actual specific members of
$\mbox{\em VR}$; they are not metavariables of our expository language (which
we denote with $\alpha$, $\beta$, etc.) but are instead (meta)constant symbols
(members of $\mbox{\em SM}$) from the point of view of our expository
language.  The equivalent system of propositional calculus described in
\cite{Tarski1965} also uses the symbols $\phi$, $\psi$, etc.\ to denote wff
metavariables, but in \cite{Tarski1965} unlike here those are metavariables of
the expository language and not primitive symbols of the formal system.

The first two statements define wffs: if $\varphi$ and $\psi$ are wffs, so is
$(\varphi \to \psi)$; if $\varphi$ is a wff, so is $\lnot\varphi$. The next
three are the axioms of propositional calculus: if $\varphi$ and $\psi$ are
wffs, then $\vdash (\varphi \to (\psi \to \varphi))$ is an (axiomatic)
theorem; etc. The
last is the rule of modus ponens: if $\varphi$ and $\psi$ are wffs, and
$\vdash (\varphi\to\psi)$ and $\vdash \varphi$ are theorems, then $\vdash
\psi$ is a theorem.

The correspondence to ordinary propositional calculus is as follows.  We
consider only provable statements of the form $\langle\varnothing,
T,\varnothing,A\rangle$ with $T$ defined as above.  The first term of the
assertion $A$ of any such statement is either ``wff'' or ``$\vdash$''.  A
statement for which the first term is ``wff'' is a {\em wff} of propositional
calculus, and one where the first term is ``$\vdash$'' is a {\em
theorem (scheme)} of propositional calculus.

The universe of this formal system also contains many other provable
statements.  Those with distinct-variable restrictions are irrelevant because
propositional calculus has no constraints on substitutions.  Those that have
logical hypotheses we call {\em inferences}\index{inference} when
the logical hypotheses are of the form
$\langle\vdash\rangle\frown w$ where $w$ is a wff (with the leading constant
term ``wff'' removed).  Inferences (other than the modus ponens rule) are not a
proper part of propositional calculus but are convenient to use when building a
hierarchy of provable statements.  A provable statement with a nonsense
hypothesis such as $\langle \to,\vdash,\lnot\rangle$, and this same expression
as its assertion, we consider irrelevant; no use can be made of it in
proving theorems, since there is no way to eliminate the nonsense hypothesis.

{\footnotesize\begin{quotation}
{\em Comment.} Our use of parentheses in the definition of a wff illustrates
how axiomatic statements should be carefully stated in a way that
ties in unambiguously with the substitutions allowed by the formal system.
There are many ways we could have defined wffs---for example, Polish
prefix notation would have allowed us to omit parentheses entirely, at
the expense of readability---but we must define them in a way that is
unambiguous.  For example, if we had omitted parentheses from the
definition of $(\varphi\to \psi)$, the wff $\lnot\varphi\to \psi$ could
be interpreted as either $\lnot(\varphi\to\psi)$ or $(\lnot\varphi\to\psi)$
and would have allowed us to prove nonsense.  Note that there is no
concept of operator binding precedence built into our formal system.
\end{quotation}}

\begin{sloppy}
\subsection{Example~2---Predicate Calculus with Equality}\index{predicate
calculus}
\end{sloppy}

Here we extend Example~1 to include predicate calculus with equality,
illustrating the use of distinct-variable restrictions.  This system is the
same as Tarski's system $\mathfrak{S}_2$ in \cite{Tarski1965} (except that the
axioms of propositional calculus are different but equivalent, and a redundant
axiom is omitted).  We extend $\mbox{\em CN}$ with the constants
$\{\mbox{var},\forall,=\}$.  We extend $\mbox{\em VR}$ with an infinite set of
{\em individual metavariables}\index{individual
metavariable} $\{x,y,z,\ldots\}$ and denote this subset
$\mbox{\em Vr}$.

We also join to $\mbox{\em CN}$ a possibly infinite set $\mbox{\em Pr}$ of {\em
predicates} $\{R,S,\ldots\}$.  We associate with $\mbox{\em Pr}$ a function
$\mbox{rnk}$ from $\mbox{\em Pr}$ to $\omega$, and for $\alpha\in \mbox{\em
Pr}$ we call $\mbox{rnk}(\alpha)$ the {\em rank} of the predicate $\alpha$,
which is simply the number of ``arguments'' that the predicate has.  (Most
applications of predicate calculus will have a finite number of predicates;
for example, set theory has the single two-argument or binary predicate $\in$,
which is usually written with its arguments surrounding the predicate symbol
rather than with the prefix notation we will use for the general case.)  As a
device to facilitate our discussion, we will let $\mbox{\em Vs}$ be any fixed
one-to-one function from $\omega$ to $\mbox{\em Vr}$; thus $\mbox{\em Vs}$ is
any simple infinite sequence of individual metavariables with no repeating
terms.

In this example we will not include the function symbols that are often part of
formalizations of predicate calculus.  Using metalogical arguments that are
beyond the scope of our discussion, it can be shown that our formalization is
equivalent when functions are introduced via appropriate definitions.

We extend the set $T$ defined in Example~1 with the expressions
$\{\langle \mbox{var\ } x\rangle,$ $ \langle \mbox{var\ } y\rangle, \langle
\mbox{var\ } z\rangle,\ldots\}$.  We extend the $\Gamma$ above
with the axiomatic statements that are the reducts of the following
pre-statements:
\begin{list}{}{\itemsep 0.0pt}
      \item[] $\langle\varnothing,T,\varnothing,
               \langle \mbox{wff\ }\forall x\,\varphi\rangle\rangle$
      \item[] $\langle\varnothing,T,\varnothing,
               \langle \mbox{wff\ }x=y\rangle\rangle$
      \item[] $\langle\varnothing,T,
               \{\langle\vdash\varphi\rangle\},
               \langle\vdash\forall x\,\varphi\rangle\rangle$
      \item[] $\langle\varnothing,T,\varnothing,
               \langle \vdash((\forall x(\varphi\to\psi)
                  \to(\forall x\,\varphi\to\forall x\,\psi))
               \rangle\rangle$
      \item[] $\langle\{\{x,\varphi\}\},T,\varnothing,
               \langle \vdash(\varphi\to\forall x\,\varphi)
               \rangle\rangle$
      \item[] $\langle\{\{x,y\}\},T,\varnothing,
               \langle \vdash\lnot\forall x\lnot x=y
               \rangle\rangle$
      \item[] $\langle\varnothing,T,\varnothing,
               \langle \vdash(x=z
                  \to(x=y\to z=y))
               \rangle\rangle$
      \item[] $\langle\varnothing,T,\varnothing,
               \langle \vdash(y=z
                  \to(x=y\to x=z))
               \rangle\rangle$
\end{list}
These are the axioms not involving predicate symbols. The first two statements
extend the definition of a wff.  The third is the rule of generalization.  The
fifth states, in effect, ``For a wff $\varphi$ and variable $x$,
$\vdash(\varphi\to\forall x\,\varphi)$, provided that $x$ does not occur in
$\varphi$.''  The sixth states ``For variables $x$ and $y$,
$\vdash\lnot\forall x\lnot x = y$, provided that $x$ and $y$ are distinct.''
(This proviso is not necessary but was included by Tarski to
weaken the axiom and still show that the system is logically complete.)

Finally, for each predicate symbol $\alpha\in \mbox{\em Pr}$, we add to
$\Gamma$ an axiomatic statement, extending the definition of wff,
that is the reduct of the following pre-statement:
\begin{displaymath}
    \langle\varnothing,T,\varnothing,
            \langle \mbox{wff},\alpha\rangle\
            \frown \mbox{\em Vs}\restriction\mbox{rnk}(\alpha)\rangle
\end{displaymath}
and for each $\alpha\in \mbox{\em Pr}$ and each $n < \mbox{rnk}(\alpha)$
we add to $\Gamma$ an equality axiom that is the reduct of the
following pre-statement:
\begin{eqnarray*}
    \lefteqn{\langle\varnothing,T,\varnothing,
            \langle
      \vdash,(,\mbox{\em Vs}_n,=,\mbox{\em Vs}_{\mbox{rnk}(\alpha)},\to,
            (,\alpha\rangle\frown \mbox{\em Vs}\restriction\mbox{rnk}(\alpha)} \\
  & & \frown
            \langle\to,\alpha\rangle\frown \mbox{\em Vs}\restriction n\frown
            \langle \mbox{\em Vs}_{\mbox{rnk}(\alpha)}\rangle \\
 & & \frown
            \mbox{\em Vs}\restriction(\mbox{rnk}(\alpha)\setminus(n+1))\frown
            \langle),)\rangle\rangle
\end{eqnarray*}
where $\restriction$ denotes function domain restriction and $\setminus$
denotes set difference.  Recall that a subscript on $\mbox{\em Vs}$
denotes one of its terms.  (In the above two axiom sets commas are placed
between successive terms of sequences to prevent ambiguity, and if you examine
them with care you will be able to distinguish those parentheses that denote
constant symbols from those of our expository language that delimit function
arguments.  Although it might have been better to use boldface for our
primitive symbols, unfortunately boldface was not available for all characters
on the \LaTeX\ system used to typeset this text.)  These seemingly forbidding
axioms can be understood by analogy to concatenation of substrings in a
computer language.  They are actually relatively simple for each specific case
and will become clearer by looking at the special case of a binary predicate
$\alpha = R$ where $\mbox{rnk}(R)=2$.  Letting $\mbox{\em Vs}$ be the sequence
$\langle x,y,z,\ldots\rangle$, the axioms we would add to $\Gamma$ for this
case would be the wff extension and two equality axioms that are the
reducts of the pre-statements:
\begin{list}{}{\itemsep 0.0pt}
      \item[] $\langle\varnothing,T,\varnothing,
               \langle \mbox{wff\ }R x y\rangle\rangle$
      \item[] $\langle\varnothing,T,\varnothing,
               \langle \vdash(x=z
                  \to(R x y \to R z y))
               \rangle\rangle$
      \item[] $\langle\varnothing,T,\varnothing,
               \langle \vdash(y=z
                  \to(R x y \to R x z))
               \rangle\rangle$
\end{list}
Study these carefully to see how the general axioms above evaluate to
them.  In practice, typically only a few special cases such as this would be
needed, and in any case the Metamath language will only permit us to describe
a finite number of predicates, as opposed to the infinite number permitted by
the formal system.  (If an infinite number should be needed for some reason,
we could not define the formal system directly in the Metamath language but
could instead define it metalogically under set theory as we
do in this appendix, and only the underlying set theory, with its single
binary predicate, would be defined directly in the Metamath language.)


{\footnotesize\begin{quotation}
{\em Comment.}  As we noted earlier, the specific variables denoted by the
symbols $x,y,z,\ldots\in \mbox{\em Vr}\subseteq \mbox{\em VR}\subseteq
\mbox{\em SM}$ in Example~2 are not the actual variables of ordinary predicate
calculus but should be thought of as metavariables ranging over them.  For
example, a distinct-variable restriction would be meaningless for actual
variables of ordinary predicate calculus since two different actual variables
are by definition distinct.  And when we talk about an arbitrary
representative $\alpha\in \mbox{\em Vr}$, $\alpha$ is a metavariable (in our
expository language) that ranges over metavariables (which are primitives of
our formal system) each of which ranges over the actual individual variables
of predicate calculus (which are never mentioned in our formal system).

The constant called ``var'' above is called \texttt{setvar} in the
\texttt{set.mm} database file, but it means the same thing.  I felt
that ``var'' is a more meaningful name in the context of predicate
calculus, whose use is not limited to set theory.  For consistency we
stick with the name ``var'' throughout this Appendix, even after set
theory is introduced.
\end{quotation}}

\subsection{Free Variables and Proper Substitution}\index{free variable}
\index{proper substitution}\index{substitution!proper}

Typical representations of mathematical axioms use concepts such
as ``free variable,'' ``bound variable,'' and ``proper substitution''
as primitive notions.
A free variable is a variable that
is not a parameter of any container expression.
A bound variable is the opposite of a free variable; it is a
a variable that has been bound in a container expression.
For example, in the expression $\forall x \varphi$ (for all $x$, $\varphi$
is true), the variable $x$
is bound within the for-all ($\forall$) expression.
It is possible to change one variable to another, and that process is called
``proper substitution.''
In most books, proper substitution has a somewhat complicated recursive
definition with multiple cases based on the occurrences of free and
bound variables.
You may consult
\cite[ch.\ 3--4]{Hamilton}\index{Hamilton, Alan G.} (as well as
many other texts) for more formal details about these terms.

Using these concepts as \texttt{primitives} creates complications
for computer implementations.

In the system of Example~2, there are no primitive notions of free variable
and proper substitution.  Tarski \cite{Tarski1965} shows that this system is
logically equivalent to the more typical textbook systems that do have these
primitive notions, if we introduce these notions with appropriate definitions
and metalogic.  We could also define axioms for such systems directly,
although the recursive definitions of free variable and proper substitution
would be messy and awkward to work with.  Instead, we mention two devices that
can be used in practice to mimic these notions.  (1) Instead of introducing
special notation to express (as a logical hypothesis) ``where $x$ is not free
in $\varphi$'' we can use the logical hypothesis $\vdash(\varphi\to\forall
x\,\varphi)$.\label{effectivelybound}\index{effectively
not free}\footnote{This is a slightly weaker requirement than ``where $x$ is
not free in $\varphi$.''  If we let $\varphi$ be $x=x$, we have the theorem
$(x=x\to\forall x\,x=x)$ which satisfies the hypothesis, even though $x$ is
free in $x=x$ .  In a case like this we say that $x$ is {\em effectively not
free}\index{effectively not free} in $x=x$, since $x=x$ is logically
equivalent to $\forall x\,x=x$ in which $x$ is bound.} (2) It can be shown
that the wff $((x=y\to\varphi)\wedge\exists x(x=y\wedge\varphi))$ (with the
usual definitions of $\wedge$ and $\exists$; see Example~4 below) is logically
equivalent to ``the wff that results from proper substitution of $y$ for $x$
in $\varphi$.''  This works whether or not $x$ and $y$ are distinct.

\subsection{Metalogical Completeness}\index{metalogical completeness}

In the system of Example~2, the
following are provable pre-statements (and their reducts are
provable statements):
\begin{eqnarray*}
      & \langle\{\{x,y\}\},T,\varnothing,
               \langle \vdash\lnot\forall x\lnot x=y
               \rangle\rangle & \\
     &  \langle\varnothing,T,\varnothing,
               \langle \vdash\lnot\forall x\lnot x=x
               \rangle\rangle &
\end{eqnarray*}
whereas the following pre-statement is not to my knowledge provable (but
in any case we will pretend it's not for sake of illustration):
\begin{eqnarray*}
     &  \langle\varnothing,T,\varnothing,
               \langle \vdash\lnot\forall x\lnot x=y
               \rangle\rangle &
\end{eqnarray*}
In other words, we can prove ``$\lnot\forall x\lnot x=y$ where $x$ and $y$ are
distinct'' and separately prove ``$\lnot\forall x\lnot x=x$'', but we can't
prove the combined general case ``$\lnot\forall x\lnot x=y$'' that has no
proviso.  Now this does not compromise logical completeness, because the
variables are really metavariables and the two provable cases together cover
all possible cases.  The third case can be considered a metatheorem whose
direct proof, using the system of Example~2, lies outside the capability of the
formal system.

Also, in the system of Example~2 the following pre-statement is not to my
knowledge provable (again, a conjecture that we will pretend to be the case):
\begin{eqnarray*}
     & \langle\varnothing,T,\varnothing,
               \langle \vdash(\forall x\, \varphi\to\varphi)
               \rangle\rangle &
\end{eqnarray*}
Instead, we can only prove specific cases of $\varphi$ involving individual
metavariables, and by induction on formula length, prove as a metatheorem
outside of our formal system the general statement above.  The details of this
proof are found in \cite{Kalish}.

There does, however, exist a system of predicate calculus in which all such
``simple metatheorems'' as those above can be proved directly, and we present
it in Example~3. A {\em simple metatheorem}\index{simple metatheorem}
is any statement of the formal
system of Example~2 where all distinct variable restrictions consist of either
two individual metavariables or an individual metavariable and a wff
metavariable, and which is provable by combining cases outside the system as
above.  A system is {\em metalogically complete}\index{metalogical
completeness} if all of its simple
metatheorems are (directly) provable statements. The precise definition of
``simple metatheorem'' and the proof of the ``metalogical completeness'' of
Example~3 is found in Remark 9.6 and Theorem 9.7 of \cite{Megill}.\index{Megill,
Norman}

\begin{sloppy}
\subsection{Example~3---Metalogically Complete Predicate
Calculus with
Equality}
\end{sloppy}

For simplicity we will assume there is one binary predicate $R$;
this system suffices for set theory, where the $R$ is of course the $\in$
predicate.  We label the axioms as they appear in \cite{Megill}.  This
system is logically equivalent to that of Example~2 (when the latter is
restricted to this single binary predicate) but is also metalogically
complete.\index{metalogical completeness}

Let
\begin{itemize}
  \item[] $\mbox{\em CN}=\{\mbox{wff}, \mbox{var}, \vdash, \to, \lnot, (,),\forall,=,R\}$.
  \item[] $\mbox{\em VR}=\{\varphi,\psi,\chi,\ldots\}\cup\{x,y,z,\ldots\}$.
  \item[] $T = \{\langle \mbox{wff\ } \varphi\rangle,
             \langle \mbox{wff\ } \psi\rangle,
             \langle \mbox{wff\ } \chi\rangle,\ldots\}\cup
       \{\langle \mbox{var\ } x\rangle, \langle \mbox{var\ } y\rangle, \langle
       \mbox{var\ }z\rangle,\ldots\}$.

\noindent Then
  $\Gamma$ consists of the reducts of the following pre-statements:
    \begin{itemize}
      \item[] $\langle\varnothing,T,\varnothing,
               \langle \mbox{wff\ }(\varphi\to\psi)\rangle\rangle$
      \item[] $\langle\varnothing,T,\varnothing,
               \langle \mbox{wff\ }\lnot\varphi\rangle\rangle$
      \item[] $\langle\varnothing,T,\varnothing,
               \langle \mbox{wff\ }\forall x\,\varphi\rangle\rangle$
      \item[] $\langle\varnothing,T,\varnothing,
               \langle \mbox{wff\ }x=y\rangle\rangle$
      \item[] $\langle\varnothing,T,\varnothing,
               \langle \mbox{wff\ }Rxy\rangle\rangle$
      \item[(C1$'$)] $\langle\varnothing,T,\varnothing,
               \langle \vdash(\varphi\to(\psi\to\varphi))
               \rangle\rangle$
      \item[(C2$'$)] $\langle\varnothing,T,
               \varnothing,
               \langle \vdash((\varphi\to(\psi\to\chi))\to
               ((\varphi\to\psi)\to(\varphi\to\chi)))
               \rangle\rangle$
      \item[(C3$'$)] $\langle\varnothing,T,
               \varnothing,
               \langle \vdash((\lnot\varphi\to\lnot\psi)\to
               (\psi\to\varphi))\rangle\rangle$
      \item[(C4$'$)] $\langle\varnothing,T,
               \varnothing,
               \langle \vdash(\forall x(\forall x\,\varphi\to\psi)\to
                 (\forall x\,\varphi\to\forall x\,\psi))\rangle\rangle$
      \item[(C5$'$)] $\langle\varnothing,T,
               \varnothing,
               \langle \vdash(\forall x\,\varphi\to\varphi)\rangle\rangle$
      \item[(C6$'$)] $\langle\varnothing,T,
               \varnothing,
               \langle \vdash(\forall x\forall y\,\varphi\to
                 \forall y\forall x\,\varphi)\rangle\rangle$
      \item[(C7$'$)] $\langle\varnothing,T,
               \varnothing,
               \langle \vdash(\lnot\varphi\to\forall x\lnot\forall x\,\varphi
                 )\rangle\rangle$
      \item[(C8$'$)] $\langle\varnothing,T,
               \varnothing,
               \langle \vdash(x=y\to(x=z\to y=z))\rangle\rangle$
      \item[(C9$'$)] $\langle\varnothing,T,
               \varnothing,
               \langle \vdash(\lnot\forall x\, x=y\to(\lnot\forall x\, x=z\to
                 (y=z\to\forall x\, y=z)))\rangle\rangle$
      \item[(C10$'$)] $\langle\varnothing,T,
               \varnothing,
               \langle \vdash(\forall x(x=y\to\forall x\,\varphi)\to
                 \varphi))\rangle\rangle$
      \item[(C11$'$)] $\langle\varnothing,T,
               \varnothing,
               \langle \vdash(\forall x\, x=y\to(\forall x\,\varphi
               \to\forall y\,\varphi))\rangle\rangle$
      \item[(C12$'$)] $\langle\varnothing,T,
               \varnothing,
               \langle \vdash(x=y\to(Rxz\to Ryz))\rangle\rangle$
      \item[(C13$'$)] $\langle\varnothing,T,
               \varnothing,
               \langle \vdash(x=y\to(Rzx\to Rzy))\rangle\rangle$
      \item[(C15$'$)] $\langle\varnothing,T,
               \varnothing,
               \langle \vdash(\lnot\forall x\, x=y\to(x=y\to(\varphi
                 \to\forall x(x=y\to\varphi))))\rangle\rangle$
      \item[(C16$'$)] $\langle\{\{x,y\}\},T,
               \varnothing,
               \langle \vdash(\forall x\, x=y\to(\varphi\to\forall x\,\varphi)
                 )\rangle\rangle$
      \item[(C5)] $\langle\{\{x,\varphi\}\},T,\varnothing,
               \langle \vdash(\varphi\to\forall x\,\varphi)
               \rangle\rangle$
      \item[(MP)] $\langle\varnothing,T,
               \{\langle\vdash(\varphi\to\psi)\rangle,
                 \langle\vdash\varphi\rangle\},
               \langle\vdash\psi\rangle\rangle$
      \item[(Gen)] $\langle\varnothing,T,
               \{\langle\vdash\varphi\rangle\},
               \langle\vdash\forall x\,\varphi\rangle\rangle$
    \end{itemize}
\end{itemize}

While it is known that these axioms are ``metalogically complete,'' it is
not known whether they are independent (i.e.\ none is
redundant) in the metalogical sense; specifically, whether any axiom (possibly
with additional non-mandatory distinct-variable restrictions, for use with any
dummy variables in its proof) is provable from the others.  Note that
metalogical independence is a weaker requirement than independence in the
usual logical sense.  Not all of the above axioms are logically independent:
for example, C9$'$ can be proved as a metatheorem from the others, outside the
formal system, by combining the possible cases of distinct variables.

\subsection{Example~4---Adding Definitions}\index{definition}
There are several ways to add definitions to a formal system.  Probably the
most proper way is to consider definitions not as part of the formal system at
all but rather as abbreviations that are part of the expository metalogic
outside the formal system.  For convenience, though, we may use the formal
system itself to incorporate definitions, adding them as axiomatic extensions
to the system.  This could be done by adding a constant representing the
concept ``is defined as'' along with axioms for it. But there is a nicer way,
at least in this writer's opinion, that introduces definitions as direct
extensions to the language rather than as extralogical primitive notions.  We
introduce additional logical connectives and provide axioms for them.  For
systems of logic such as Examples 1 through 3, the additional axioms must be
conservative in the sense that no wff of the original system that was not a
theorem (when the initial term ``wff'' is replaced by ``$\vdash$'' of course)
becomes a theorem of the extended system.  In this example we extend Example~3
(or 2) with standard abbreviations of logic.

We extend $\mbox{\em CN}$ of Example~3 with new constants $\{\leftrightarrow,
\wedge,\vee,\exists\}$, corresponding to logical equivalence,\index{logical
equivalence ($\leftrightarrow$)}\index{biconditional ($\leftrightarrow$)}
conjunction,\index{conjunction ($\wedge$)} disjunction,\index{disjunction
($\vee$)} and the existential quantifier.\index{existential quantifier
($\exists$)}  We extend $\Gamma$ with the axiomatic statements that are
the reducts of the following pre-statements:
\begin{list}{}{\itemsep 0.0pt}
      \item[] $\langle\varnothing,T,\varnothing,
               \langle \mbox{wff\ }(\varphi\leftrightarrow\psi)\rangle\rangle$
      \item[] $\langle\varnothing,T,\varnothing,
               \langle \mbox{wff\ }(\varphi\vee\psi)\rangle\rangle$
      \item[] $\langle\varnothing,T,\varnothing,
               \langle \mbox{wff\ }(\varphi\wedge\psi)\rangle\rangle$
      \item[] $\langle\varnothing,T,\varnothing,
               \langle \mbox{wff\ }\exists x\, \varphi\rangle\rangle$
  \item[] $\langle\varnothing,T,\varnothing,
     \langle\vdash ( ( \varphi \leftrightarrow \psi ) \to
     ( \varphi \to \psi ) )\rangle\rangle$
  \item[] $\langle\varnothing,T,\varnothing,
     \langle\vdash ((\varphi\leftrightarrow\psi)\to
    (\psi\to\varphi))\rangle\rangle$
  \item[] $\langle\varnothing,T,\varnothing,
     \langle\vdash ((\varphi\to\psi)\to(
     (\psi\to\varphi)\to(\varphi
     \leftrightarrow\psi)))\rangle\rangle$
  \item[] $\langle\varnothing,T,\varnothing,
     \langle\vdash (( \varphi \wedge \psi ) \leftrightarrow\neg ( \varphi
     \to \neg \psi )) \rangle\rangle$
  \item[] $\langle\varnothing,T,\varnothing,
     \langle\vdash (( \varphi \vee \psi ) \leftrightarrow (\neg \varphi
     \to \psi )) \rangle\rangle$
  \item[] $\langle\varnothing,T,\varnothing,
     \langle\vdash (\exists x \,\varphi\leftrightarrow
     \lnot \forall x \lnot \varphi)\rangle\rangle$
\end{list}
The first three logical axioms (statements containing ``$\vdash$'') introduce
and effectively define logical equivalence, ``$\leftrightarrow$''.  The last
three use ``$\leftrightarrow$'' to effectively mean ``is defined as.''

\subsection{Example~5---ZFC Set Theory}\index{ZFC set theory}

Here we add to the system of Example~4 the axioms of Zermelo--Fraenkel set
theory with Choice.  For convenience we make use of the
definitions in Example~4.

In the $\mbox{\em CN}$ of Example~4 (which extends Example~3), we replace the symbol $R$
with the symbol $\in$.
More explicitly, we remove from $\Gamma$ of Example~4 the three
axiomatic statements containing $R$ and replace them with the
reducts of the following:
\begin{list}{}{\itemsep 0.0pt}
      \item[] $\langle\varnothing,T,\varnothing,
               \langle \mbox{wff\ }x\in y\rangle\rangle$
      \item[] $\langle\varnothing,T,
               \varnothing,
               \langle \vdash(x=y\to(x\in z\to y\in z))\rangle\rangle$
      \item[] $\langle\varnothing,T,
               \varnothing,
               \langle \vdash(x=y\to(z\in x\to z\in y))\rangle\rangle$
\end{list}
Letting $D=\{\{\alpha,\beta\}\in \mbox{\em DV}\,|\alpha,\beta\in \mbox{\em
Vr}\}$ (in other words all individual variables must be distinct), we extend
$\Gamma$ with the ZFC axioms, called
\index{Axiom of Extensionality}
\index{Axiom of Replacement}
\index{Axiom of Union}
\index{Axiom of Power Sets}
\index{Axiom of Regularity}
\index{Axiom of Infinity}
\index{Axiom of Choice}
Extensionality, Replacement, Union, Power
Set, Regularity, Infinity, and Choice, that are the reducts of:
\begin{list}{}{\itemsep 0.0pt}
      \item[Ext] $\langle D,T,
               \varnothing,
               \langle\vdash (\forall x(x\in y\leftrightarrow x \in z)\to y
               =z) \rangle\rangle$
      \item[Rep] $\langle D,T,
               \varnothing,
               \langle\vdash\exists x ( \exists y \forall z (\varphi \to z = y
                        ) \to
                        \forall z ( z \in x \leftrightarrow \exists x ( x \in
                        y \wedge \forall y\,\varphi ) ) )\rangle\rangle$
      \item[Un] $\langle D,T,
               \varnothing,
               \langle\vdash \exists x \forall y ( \exists x ( y \in x \wedge
               x \in z ) \to y \in x ) \rangle\rangle$
      \item[Pow] $\langle D,T,
               \varnothing,
               \langle\vdash \exists x \forall y ( \forall x ( x \in y \to x
               \in z ) \to y \in x ) \rangle\rangle$
      \item[Reg] $\langle D,T,
               \varnothing,
               \langle\vdash (  x \in y \to
                 \exists x ( x \in y \wedge \forall z ( z \in x \to \lnot z
                \in y ) ) ) \rangle\rangle$
      \item[Inf] $\langle D,T,
               \varnothing,
               \langle\vdash \exists x(y\in x\wedge\forall y(y\in
               x\to
               \exists z(y \in z\wedge z\in x))) \rangle\rangle$
      \item[AC] $\langle D,T,
               \varnothing,
               \langle\vdash \exists x \forall y \forall z ( ( y \in z
               \wedge z \in w ) \to \exists w \forall y ( \exists w
              ( ( y \in z \wedge z \in w ) \wedge ( y \in w \wedge w \in x
              ) ) \leftrightarrow y = w ) ) \rangle\rangle$
\end{list}

\subsection{Example~6---Class Notation in Set Theory}\label{class}

A powerful device that makes set theory easier (and that we have
been using all along in our informal expository language) is {\em class
abstraction notation}.\index{class abstraction}\index{abstraction class}  The
definitions we introduce are rigorously justified
as conservative by Takeuti and Zaring \cite{Takeuti}\index{Takeuti, Gaisi} or
Quine \cite{Quine}\index{Quine, Willard Van Orman}.  The key idea is to
introduce the notation $\{x|\mbox{---}\}$ which means ``the class of all $x$
such that ---'' for abstraction classes and introduce (meta)variables that
range over them.  An abstraction class may or may not be a set, depending on
whether it exists (as a set).  A class that does not exist is
called a {\em proper class}.\index{proper class}\index{class!proper}

To illustrate the use of abstraction classes we will provide some examples
of definitions that make use of them:  the empty set, class union, and
unordered pair.  Many other such definitions can be found in the
Metamath set theory database,
\texttt{set.mm}.\index{set theory database (\texttt{set.mm})}

% We intentionally break up the sequence of math symbols here
% because otherwise the overlong line goes beyond the page in narrow mode.
We extend $\mbox{\em CN}$ of Example~5 with new symbols $\{$
$\mbox{class},$ $\{,$ $|,$ $\},$ $\varnothing,$ $\cup,$ $,$ $\}$
where the inner braces and last comma are
constant symbols. (As before,
our dual use of some mathematical symbols for both our expository
language and as primitives of the formal system should be clear from context.)

We extend $\mbox{\em VR}$ of Example~5 with a set of {\em class
variables}\index{class variable}
$\{A,B,C,\ldots\}$. We extend the $T$ of Example~5 with $\{\langle
\mbox{class\ } A\rangle, \langle \mbox{class\ }B\rangle, \langle \mbox{class\ }
C\rangle,\ldots\}$.

To
introduce our definitions,
we add to $\Gamma$ of Example~5 the axiomatic statements
that are the reducts of the following pre-statements:
\begin{list}{}{\itemsep 0.0pt}
      \item[] $\langle\varnothing,T,\varnothing,
               \langle \mbox{class\ }x\rangle\rangle$
      \item[] $\langle\varnothing,T,\varnothing,
               \langle \mbox{class\ }\{x|\varphi\}\rangle\rangle$
      \item[] $\langle\varnothing,T,\varnothing,
               \langle \mbox{wff\ }A=B\rangle\rangle$
      \item[] $\langle\varnothing,T,\varnothing,
               \langle \mbox{wff\ }A\in B\rangle\rangle$
      \item[Ab] $\langle\varnothing,T,\varnothing,
               \langle \vdash ( y \in \{ x |\varphi\} \leftrightarrow
                  ( ( x = y \to\varphi) \wedge \exists x ( x = y
                  \wedge\varphi) ))
               \rangle\rangle$
      \item[Eq] $\langle\{\{x,A\},\{x,B\}\},T,\varnothing,
               \langle \vdash ( A = B \leftrightarrow
               \forall x ( x \in A \leftrightarrow x \in B ) )
               \rangle\rangle$
      \item[El] $\langle\{\{x,A\},\{x,B\}\},T,\varnothing,
               \langle \vdash ( A \in B \leftrightarrow \exists x
               ( x = A \wedge x \in B ) )
               \rangle\rangle$
\end{list}
Here we say that an individual variable is a class; $\{x|\varphi\}$ is a
class; and we extend the definition of a wff to include class equality and
membership.  Axiom Ab defines membership of a variable in a class abstraction;
the right-hand side can be read as ``the wff that results from proper
substitution of $y$ for $x$ in $\varphi$.''\footnote{Note that this definition
makes unnecessary the introduction of a separate notation similar to
$\varphi(x|y)$ for proper substitution, although we may choose to do so to be
conventional.  Incidentally, $\varphi(x|y)$ as it stands would be ambiguous in
the formal systems of our examples, since we wouldn't know whether
$\lnot\varphi(x|y)$ meant $\lnot(\varphi(x|y))$ or $(\lnot\varphi)(x|y)$.
Instead, we would have to use an unambiguous variant such as $(\varphi\,
x|y)$.}  Axioms Eq and El extend the meaning of the existing equality and
membership connectives.  This is potentially dangerous and requires careful
justification.  For example, from Eq we can derive the Axiom of Extensionality
with predicate logic alone; thus in principle we should include the Axiom of
Extensionality as a logical hypothesis.  However we do not bother to do this
since we have already presupposed that axiom earlier. The distinct variable
restrictions should be read ``where $x$ does not occur in $A$ or $B$.''  We
typically do this when the right-hand side of a definition involves an
individual variable not in the expression being defined; it is done so that
the right-hand side remains independent of the particular ``dummy'' variable
we use.

We continue to add to $\Gamma$ the following definitions
(i.e. the reducts of the following pre-statements) for empty
set,\index{empty set} class union,\index{union} and unordered
pair.\index{unordered pair}  They should be self-explanatory.  Analogous to our
use of ``$\leftrightarrow$'' to define new wffs in Example~4, we use ``$=$''
to define new abstraction terms, and both may be read informally as ``is
defined as'' in this context.
\begin{list}{}{\itemsep 0.0pt}
      \item[] $\langle\varnothing,T,\varnothing,
               \langle \mbox{class\ }\varnothing\rangle\rangle$
      \item[] $\langle\varnothing,T,\varnothing,
               \langle \vdash \varnothing = \{ x | \lnot x = x \}
               \rangle\rangle$
      \item[] $\langle\varnothing,T,\varnothing,
               \langle \mbox{class\ }(A\cup B)\rangle\rangle$
      \item[] $\langle\{\{x,A\},\{x,B\}\},T,\varnothing,
               \langle \vdash ( A \cup B ) = \{ x | ( x \in A \vee x \in B ) \}
               \rangle\rangle$
      \item[] $\langle\varnothing,T,\varnothing,
               \langle \mbox{class\ }\{A,B\}\rangle\rangle$
      \item[] $\langle\{\{x,A\},\{x,B\}\},T,\varnothing,
               \langle \vdash \{ A , B \} = \{ x | ( x = A \vee x = B ) \}
               \rangle\rangle$
\end{list}

\section{Metamath as a Formal System}\label{theorymm}

This section presupposes a familiarity with the Metamath computer language.

Our theory describes formal systems and their universes.  The Metamath
language provides a way of representing these set-theoretical objects to
a computer.  A Metamath database, being a finite set of {\sc ascii}
characters, can usually describe only a subset of a formal system and
its universe, which are typically infinite.  However the database can
contain as large a finite subset of the formal system and its universe
as we wish.  (Of course a Metamath set theory database can, in
principle, indirectly describe an entire infinite formal system by
formalizing the expository language in this Appendix.)

For purpose of our discussion, we assume the Metamath database
is in the simple form described on p.~\pageref{framelist},
consisting of all constant and variable declarations at the beginning,
followed by a sequence of extended frames each
delimited by \texttt{\$\char`\{} and \texttt{\$\char`\}}.  Any Metamath database can
be converted to this form, as described on p.~\pageref{frameconvert}.

The math symbol tokens of a Metamath source file, which are declared
with \texttt{\$c} and \texttt{\$v} statements, are names we assign to
representatives of $\mbox{\em CN}$ and $\mbox{\em VR}$.  For
definiteness we could assume that the first math symbol declared as a
variable corresponds to $v_0$, the second to $v_1$, etc., although the
exact correspondence we choose is not important.

In the Metamath language, each \texttt{\$d}, \texttt{\$f}, and
 \texttt{\$e} source
statement in an extended frame (Section~\ref{frames})
corresponds respectively to a member of the
collections $D$, $T$, and $H$ in a formal system statement $\langle
D_M,T_M,H,A\rangle$.  The math symbol strings following these Metamath keywords
correspond to a variable pair (in the case of \texttt{\$d}) or an expression (for
the other two keywords). The math symbol string following a \texttt{\$a} source
statement corresponds to expression $A$ in an axiomatic statement of the
formal system; the one following a \texttt{\$p} source statement corresponds to
$A$ in a provable statement that is not axiomatic.  In other words, each
extended frame in a Metamath database corresponds to
a pre-statement of the formal system, and a frame corresponds to
a statement of the formal system.  (Don't confuse the two meanings of
``statement'' here.  A statement of the formal system corresponds to the
several statements in a Metamath database that may constitute a
frame.)

In order for the computer to verify that a formal system statement is
provable, each \texttt{\$p} source statement is accompanied by a proof.
However, the proof does not correspond to anything in the formal system
but is simply a way of communicating to the computer the information
needed for its verification.  The proof tells the computer {\em how to
construct} specific members of closure of the formal system
pre-statement corresponding to the extended frame of the \texttt{\$p}
statement.  The final result of the construction is the member of the
closure that matches the \texttt{\$p} statement.  The abstract formal
system, on the other hand, is concerned only with the {\em existence} of
members of the closure.

As mentioned on p.~\pageref{exampleref}, Examples 1 and 3--6 in the
previous Section parallel the development of logic and set theory in the
Metamath database
\texttt{set.mm}.\index{set theory database (\texttt{set.mm})} You may
find it instructive to compare them.


\chapter{The MIU System}
\label{MIU}
\index{formal system}
\index{MIU-system}

The following is a listing of the file \texttt{miu.mm}.  It is self-explanatory.

%%%%%%%%%%%%%%%%%%%%%%%%%%%%%%%%%%%%%%%%%%%%%%%%%%%%%%%%%%%%

\begin{verbatim}
$( The MIU-system:  A simple formal system $)

$( Note:  This formal system is unusual in that it allows
empty wffs.  To work with a proof, you must type
SET EMPTY_SUBSTITUTION ON before using the PROVE command.
By default, this is OFF in order to reduce the number of
ambiguous unification possibilities that have to be selected
during the construction of a proof.  $)

$(
Hofstadter's MIU-system is a simple example of a formal
system that illustrates some concepts of Metamath.  See
Douglas R. Hofstadter, _Goedel, Escher, Bach:  An Eternal
Golden Braid_ (Vintage Books, New York, 1979), pp. 33ff. for
a description of the MIU-system.

The system has 3 constant symbols, M, I, and U.  The sole
axiom of the system is MI. There are 4 rules:
     Rule I:  If you possess a string whose last letter is I,
     you can add on a U at the end.
     Rule II:  Suppose you have Mx.  Then you may add Mxx to
     your collection.
     Rule III:  If III occurs in one of the strings in your
     collection, you may make a new string with U in place
     of III.
     Rule IV:  If UU occurs inside one of your strings, you
     can drop it.
Unfortunately, Rules III and IV do not have unique results:
strings could have more than one occurrence of III or UU.
This requires that we introduce the concept of an "MIU
well-formed formula" or wff, which allows us to construct
unique symbol sequences to which Rules III and IV can be
applied.
$)

$( First, we declare the constant symbols of the language.
Note that we need two symbols to distinguish the assertion
that a sequence is a wff from the assertion that it is a
theorem; we have arbitrarily chosen "wff" and "|-". $)
      $c M I U |- wff $. $( Declare constants $)

$( Next, we declare some variables. $)
     $v x y $.

$( Throughout our theory, we shall assume that these
variables represent wffs. $)
 wx   $f wff x $.
 wy   $f wff y $.

$( Define MIU-wffs.  We allow the empty sequence to be a
wff. $)

$( The empty sequence is a wff. $)
 we   $a wff $.
$( "M" after any wff is a wff. $)
 wM   $a wff x M $.
$( "I" after any wff is a wff. $)
 wI   $a wff x I $.
$( "U" after any wff is a wff. $)
 wU   $a wff x U $.

$( Assert the axiom. $)
 ax   $a |- M I $.

$( Assert the rules. $)
 ${
   Ia   $e |- x I $.
$( Given any theorem ending with "I", it remains a theorem
if "U" is added after it.  (We distinguish the label I_
from the math symbol I to conform to the 24-Jun-2006
Metamath spec.) $)
   I_    $a |- x I U $.
 $}
 ${
IIa  $e |- M x $.
$( Given any theorem starting with "M", it remains a theorem
if the part after the "M" is added again after it. $)
   II   $a |- M x x $.
 $}
 ${
   IIIa $e |- x I I I y $.
$( Given any theorem with "III" in the middle, it remains a
theorem if the "III" is replaced with "U". $)
   III  $a |- x U y $.
 $}
 ${
   IVa  $e |- x U U y $.
$( Given any theorem with "UU" in the middle, it remains a
theorem if the "UU" is deleted. $)
   IV   $a |- x y $.
  $}

$( Now we prove the theorem MUIIU.  You may be interested in
comparing this proof with that of Hofstadter (pp. 35 - 36).
$)
 theorem1  $p |- M U I I U $=
      we wM wU wI we wI wU we wU wI wU we wM we wI wU we wM
      wI wI wI we wI wI we wI ax II II I_ III II IV $.
\end{verbatim}\index{well-formed formula (wff)}

The \texttt{show proof /lemmon/renumber} command
yields the following display.  It is very similar
to the one in \cite[pp.~35--36]{Hofstadter}.\index{Hofstadter, Douglas R.}

\begin{verbatim}
1 ax             $a |- M I
2 1 II           $a |- M I I
3 2 II           $a |- M I I I I
4 3 I_           $a |- M I I I I U
5 4 III          $a |- M U I U
6 5 II           $a |- M U I U U I U
7 6 IV           $a |- M U I I U
\end{verbatim}

We note that Hofstadter's ``MU-puzzle,'' which asks whether
MU is a theorem of the MIU-system, cannot be answered using
the system above because the MU-puzzle is a question {\em
about} the system.  To prove the answer to the MU-puzzle,
a much more elaborate system is needed, namely one that
models the MIU-system within set theory.  (Incidentally, the
answer to the MU-puzzle is no.)

\chapter{Metamath Language EBNF}%
\label{BNF}%
\index{Metamath Language EBNF}

The following is a formal description of the basic Metamath language syntax
(with compressed proofs and support for unknown proof steps).
It is defined using the
Extended Backus--Naur Form (EBNF)\index{Extended Backus--Naur Form}\index{EBNF}
notation from W3C\index{W3C}
\textit{Extensible Markup Language (XML) 1.0 (Fifth Edition)}
(W3C Recommendation 26 November 2008) at
\url{https://www.w3.org/TR/xml/#sec-notation}.

The \texttt{database}
rule is processed until the end of the file (\texttt{EOF}).
The rules eventually require reading whitespace-separated tokens.
A token has an upper-case definition (see below)
or is a string constant in a non-token (such as \texttt{'\$a'}).
We intend for this to be correct, but if there is a conflict the
rules of section \ref{spec} govern. That section also discusses
non-syntax restrictions not shown here
(e.g., that each new label token
defined in a \texttt{hypothesis-stmt} or \texttt{assert-stmt}
must be unique).

\begin{verbatim}
database ::= outermost-scope-stmt*

outermost-scope-stmt ::=
  include-stmt | constant-stmt | stmt

/* File inclusion command; process file as a database.
   Databases should NOT have a comment in the filename. */
include-stmt ::= '$[' filename '$]'

/* Constant symbols declaration. */
constant-stmt ::= '$c' constant+ '$.'

/* A normal statement can occur in any scope. */
stmt ::= block | variable-stmt | disjoint-stmt |
  hypothesis-stmt | assert-stmt

/* A block. You can have 0 statements in a block. */
block ::= '${' stmt* '$}'

/* Variable symbols declaration. */
variable-stmt ::= '$v' variable+ '$.'

/* Disjoint variables. Simple disjoint statements have
   2 variables, i.e., "variable*" is empty for them. */
disjoint-stmt ::= '$d' variable variable variable* '$.'

hypothesis-stmt ::= floating-stmt | essential-stmt

/* Floating (variable-type) hypothesis. */
floating-stmt ::= LABEL '$f' typecode variable '$.'

/* Essential (logical) hypothesis. */
essential-stmt ::= LABEL '$e' typecode MATH-SYMBOL* '$.'

assert-stmt ::= axiom-stmt | provable-stmt

/* Axiomatic assertion. */
axiom-stmt ::= LABEL '$a' typecode MATH-SYMBOL* '$.'

/* Provable assertion. */
provable-stmt ::= LABEL '$p' typecode MATH-SYMBOL*
  '$=' proof '$.'

/* A proof. Proofs may be interspersed by comments.
   If '?' is in a proof it's an "incomplete" proof. */
proof ::= uncompressed-proof | compressed-proof
uncompressed-proof ::= (LABEL | '?')+
compressed-proof ::= '(' LABEL* ')' COMPRESSED-PROOF-BLOCK+

typecode ::= constant

filename ::= MATH-SYMBOL /* No whitespace or '$' */
constant ::= MATH-SYMBOL
variable ::= MATH-SYMBOL
\end{verbatim}

\needspace{2\baselineskip}
A \texttt{frame} is a sequence of 0 or more
\texttt{disjoint-{\allowbreak}stmt} and
\texttt{hypotheses-{\allowbreak}stmt} statements
(possibly interleaved with other non-\texttt{assert-stmt} statements)
followed by one \texttt{assert-stmt}.

\needspace{3\baselineskip}
Here are the rules for lexical processing (tokenization) beyond
the constant tokens shown above.
By convention these tokenization rules have upper-case names.
Every token is read for the longest possible length.
Whitespace-separated tokens are read sequentially;
note that the separating whitespace and \texttt{\$(} ... \texttt{\$)}
comments are skipped.

If a token definition uses another token definition, the whole thing
is considered a single token.
A pattern that is only part of a full token has a name beginning
with an underscore (``\_'').
An implementation could tokenize many tokens as a
\texttt{PRINTABLE-SEQUENCE}
and then check if it meets the more specific rule shown here.

Comments do not nest, and both \texttt{\$(} and \texttt{\$)}
have to be surrounded
by at least one whitespace character (\texttt{\_WHITECHAR}).
Technically comments end without consuming the trailing
\texttt{\_WHITECHAR}, but the trailing
\texttt{\_WHITECHAR} gets ignored anyway so we ignore that detail here.
Metamath language processors
are not required to support \texttt{\$)} followed
immediately by a bare end-of-file, because the closing
comment symbol is supposed to be followed by a
\texttt{\_WHITECHAR} such as a newline.

\begin{verbatim}
PRINTABLE-SEQUENCE ::= _PRINTABLE-CHARACTER+

MATH-SYMBOL ::= (_PRINTABLE-CHARACTER - '$')+

/* ASCII non-whitespace printable characters */
_PRINTABLE-CHARACTER ::= [#x21-#x7e]

LABEL ::= ( _LETTER-OR-DIGIT | '.' | '-' | '_' )+

_LETTER-OR-DIGIT ::= [A-Za-z0-9]

COMPRESSED-PROOF-BLOCK ::= ([A-Z] | '?')+

/* Define whitespace between tokens. The -> SKIP
   means that when whitespace is seen, it is
   skipped and we simply read again. */
WHITESPACE ::= (_WHITECHAR+ | _COMMENT) -> SKIP

/* Comments. $( ... $) and do not nest. */
_COMMENT ::= '$(' (_WHITECHAR+ (PRINTABLE-SEQUENCE - '$)')*
  _WHITECHAR+ '$)' _WHITECHAR

/* Whitespace: (' ' | '\t' | '\r' | '\n' | '\f') */
_WHITECHAR ::= [#x20#x09#x0d#x0a#x0c]
\end{verbatim}
% This EBNF was developed as a collaboration between
% David A. Wheeler\index{Wheeler, David A.},
% Mario Carneiro\index{Carneiro, Mario}, and
% Benoit Jubin\index{Jubin, Benoit}, inspired by a request
% (and a lot of initial work) by Benoit Jubin.
%
% \chapter{Disclaimer and Trademarks}
%
% Information in this document is subject to change without notice and does not
% represent a commitment on the part of Norman Megill.
% \vspace{2ex}
%
% \noindent Norman D. Megill makes no warranties, either express or implied,
% regarding the Metamath computer software package.
%
% \vspace{2ex}
%
% \noindent Any trademarks mentioned in this book are the property of
% their respective owners.  The name ``Metamath'' is a trademark of
% Norman Megill.
%
\cleardoublepage
\phantomsection  % fixes the link anchor
\addcontentsline{toc}{chapter}{\bibname}

\bibliography{metamath}
%% metamath.tex - Version of 24-May-2019
% If you change the date above, also change the "Printed date" below.
% SPDX-License-Identifier: CC0-1.0
%
%                              PUBLIC DOMAIN
%
% This file (specifically, the version of this file with the above date)
% has been released into the Public Domain per the
% Creative Commons CC0 1.0 Universal (CC0 1.0) Public Domain Dedication
% https://creativecommons.org/publicdomain/zero/1.0/
%
% The public domain release applies worldwide.  In case this is not
% legally possible, the right is granted to use the work for any purpose,
% without any conditions, unless such conditions are required by law.
%
% Several short, attributed quotations from copyrighted works
% appear in this file under the ``fair use'' provision of Section 107 of
% the United States Copyright Act (Title 17 of the {\em United States
% Code}).  The public-domain status of this file is not applicable to
% those quotations.
%
% Norman Megill - email: nm(at)alum(dot)mit(dot)edu
%
% David A. Wheeler also donates his improvements to this file to the
% public domain per the CC0.  He works at the Institute for Defense Analyses
% (IDA), but IDA has agreed that this Metamath work is outside its "lane"
% and is not a work by IDA.  This was specifically confirmed by
% Margaret E. Myers (Division Director of the Information Technology
% and Systems Division) on 2019-05-24 and by Ben Lindorf (General Counsel)
% on 2019-05-22.

% This file, 'metamath.tex', is self-contained with everything needed to
% generate the the PDF file 'metamath.pdf' (the _Metamath_ book) on
% standard LaTeX 2e installations.  The auxiliary files are embedded with
% "filecontents" commands.  To generate metamath.pdf file, run these
% commands under Linux or Cygwin in the directory that contains
% 'metamath.tex':
%
%   rm -f realref.sty metamath.bib
%   touch metamath.ind
%   pdflatex metamath
%   pdflatex metamath
%   bibtex metamath
%   makeindex metamath
%   pdflatex metamath
%   pdflatex metamath
%
% The warnings that occur in the initial runs of pdflatex can be ignored.
% For the final run,
%
%   egrep -i 'error|warn' metamath.log
%
% should show exactly these 5 warnings:
%
%   LaTeX Warning: File `realref.sty' already exists on the system.
%   LaTeX Warning: File `metamath.bib' already exists on the system.
%   LaTeX Font Warning: Font shape `OMS/cmtt/m/n' undefined
%   LaTeX Font Warning: Font shape `OMS/cmtt/bx/n' undefined
%   LaTeX Font Warning: Some font shapes were not available, defaults
%       substituted.
%
% Search for "Uncomment" below if you want to suppress hyperlink boxes
% in the PDF output file
%
% TYPOGRAPHICAL NOTES:
% * It is customary to use an en dash (--) to "connect" names of different
%   people (and to denote ranges), and use a hyphen (-) for a
%   single compound name. Examples of connected multiple people are
%   Zermelo--Fraenkel, Schr\"{o}der--Bernstein, Tarski--Grothendieck,
%   Hewlett--Packard, and Backus--Naur.  Examples of a single person with
%   a compound name include Levi-Civita, Mittag-Leffler, and Burali-Forti.
% * Use non-breaking spaces after page abbreviations, e.g.,
%   p.~\pageref{note2002}.
%
% --------------------------- Start of realref.sty -----------------------------
\begin{filecontents}{realref.sty}
% Save the following as realref.sty.
% You can then use it with \usepackage{realref}
%
% This has \pageref jumping to the page on which the ref appears,
% \ref jumping to the point of the anchor, and \sectionref
% jumping to the start of section.
%
% Author:  Anthony Williams
%          Software Engineer
%          Nortel Networks Optical Components Ltd
% Date:    9 Nov 2001 (posted to comp.text.tex)
%
% The following declaration was made by Anthony Williams on
% 24 Jul 2006 (private email to Norman Megill):
%
%   ``I hereby donate the code for realref.sty posted on the
%   comp.text.tex newsgroup on 9th November 2001, accessible from
%   http://groups.google.com/group/comp.text.tex/msg/5a0e1cc13ea7fbb2
%   to the public domain.''
%
\ProvidesPackage{realref}
\RequirePackage[plainpages=false,pdfpagelabels=true]{hyperref}
\def\realref@anchorname{}
\AtBeginDocument{%
% ensure every label is a possible hyperlink target
\let\realref@oldrefstepcounter\refstepcounter%
\DeclareRobustCommand{\refstepcounter}[1]{\realref@oldrefstepcounter{#1}
\edef\realref@anchorname{\string #1.\@currentlabel}%
}%
\let\realref@oldlabel\label%
\DeclareRobustCommand{\label}[1]{\realref@oldlabel{#1}\hypertarget{#1}{}%
\@bsphack\protected@write\@auxout{}{%
    \string\expandafter\gdef\protect\csname
    page@num.#1\string\endcsname{\thepage}%
    \string\expandafter\gdef\protect\csname
    ref@num.#1\string\endcsname{\@currentlabel}%
    \string\expandafter\gdef\protect\csname
    sectionref@name.#1\string\endcsname{\realref@anchorname}%
}\@esphack}%
\DeclareRobustCommand\pageref[1]{{\edef\a{\csname
            page@num.#1\endcsname}\expandafter\hyperlink{page.\a}{\a}}}%
\DeclareRobustCommand\ref[1]{{\edef\a{\csname
            ref@num.#1\endcsname}\hyperlink{#1}{\a}}}%
\DeclareRobustCommand\sectionref[1]{{\edef\a{\csname
            ref@num.#1\endcsname}\edef\b{\csname
            sectionref@name.#1\endcsname}\hyperlink{\b}{\a}}}%
}
\end{filecontents}
% ---------------------------- End of realref.sty ------------------------------

% --------------------------- Start of metamath.bib -----------------------------
\begin{filecontents}{metamath.bib}
@book{Albers, editor = "Donald J. Albers and G. L. Alexanderson",
  title = "Mathematical People",
  publisher = "Contemporary Books, Inc.",
  address = "Chicago",
  note = "[QA28.M37]",
  year = 1985 }
@book{Anderson, author = "Alan Ross Anderson and Nuel D. Belnap",
  title = "Entailment",
  publisher = "Princeton University Press",
  address = "Princeton",
  volume = 1,
  note = "[QA9.A634 1975 v.1]",
  year = 1975}
@book{Barrow, author = "John D. Barrow",
  title = "Theories of Everything:  The Quest for Ultimate Explanation",
  publisher = "Oxford University Press",
  address = "Oxford",
  note = "[Q175.B225]",
  year = 1991 }
@book{Behnke,
  editor = "H. Behnke and F. Backmann and K. Fladt and W. S{\"{u}}ss",
  title = "Fundamentals of Mathematics",
  volume = "I",
  publisher = "The MIT Press",
  address = "Cambridge, Massachusetts",
  note = "[QA37.2.B413]",
  year = 1974 }
@book{Bell, author = "J. L. Bell and M. Machover",
  title = "A Course in Mathematical Logic",
  publisher = "North-Holland",
  address = "Amsterdam",
  note = "[QA9.B3953]",
  year = 1977 }
@inproceedings{Blass, author = "Andrea Blass",
  title = "The Interaction Between Category Theory and Set Theory",
  pages = "5--29",
  booktitle = "Mathematical Applications of Category Theory (Proceedings
     of the Special Session on Mathematical Applications
     Category Theory, 89th Annual Meeting of the American Mathematical
     Society, held in Denver, Colorado January 5--9, 1983)",
  editor = "John Walter Gray",
  year = 1983,
  note = "[QA169.A47 1983]",
  publisher = "American Mathematical Society",
  address = "Providence, Rhode Island"}
@proceedings{Bledsoe, editor = "W. W. Bledsoe and D. W. Loveland",
  title = "Automated Theorem Proving:  After 25 Years (Proceedings
     of the Special Session on Automatic Theorem Proving,
     89th Annual Meeting of the American Mathematical
     Society, held in Denver, Colorado January 5--9, 1983)",
  year = 1983,
  note = "[QA76.9.A96.S64 1983]",
  publisher = "American Mathematical Society",
  address = "Providence, Rhode Island" }
@book{Boolos, author = "George S. Boolos and Richard C. Jeffrey",
  title = "Computability and Log\-ic",
  publisher = "Cambridge University Press",
  edition = "third",
  address = "Cambridge",
  note = "[QA9.59.B66 1989]",
  year = 1989 }
@book{Campbell, author = "John Campbell",
  title = "Programmer's Progress",
  publisher = "White Star Software",
  address = "Box 51623, Palo Alto, CA 94303",
  year = 1991 }
@article{DBLP:journals/corr/Carneiro14,
  author    = {Mario Carneiro},
  title     = {Conversion of {HOL} Light proofs into Metamath},
  journal   = {CoRR},
  volume    = {abs/1412.8091},
  year      = {2014},
  url       = {http://arxiv.org/abs/1412.8091},
  archivePrefix = {arXiv},
  eprint    = {1412.8091},
  timestamp = {Mon, 13 Aug 2018 16:47:05 +0200},
  biburl    = {https://dblp.org/rec/bib/journals/corr/Carneiro14},
  bibsource = {dblp computer science bibliography, https://dblp.org}
}
@article{CarneiroND,
  author    = {Mario Carneiro},
  title     = {Natural Deductions in the Metamath Proof Language},
  url       = {http://us.metamath.org/ocat/natded.pdf},
  year      = 2014
}
@inproceedings{Chou, author = "Shang-Ching Chou",
  title = "Proving Elementary Geometry Theorems Using {W}u's Algorithm",
  pages = "243--286",
  booktitle = "Automated Theorem Proving:  After 25 Years (Proceedings
     of the Special Session on Automatic Theorem Proving,
     89th Annual Meeting of the American Mathematical
     Society, held in Denver, Colorado January 5--9, 1983)",
  editor = "W. W. Bledsoe and D. W. Loveland",
  year = 1983,
  note = "[QA76.9.A96.S64 1983]",
  publisher = "American Mathematical Society",
  address = "Providence, Rhode Island" }
@book{Clemente, author = "Daniel Clemente Laboreo",
  title = "Introduction to natural deduction",
  year = 2014,
  url = "http://www.danielclemente.com/logica/dn.en.pdf" }
@incollection{Courant, author = "Richard Courant and Herbert Robbins",
  title = "Topology",
  pages = "573--590",
  booktitle = "The World of Mathematics, Volume One",
  editor = "James R. Newman",
  publisher = "Simon and Schuster",
  address = "New York",
  note = "[QA3.W67 1988]",
  year = 1956 }
@book{Curry, author = "Haskell B. Curry",
  title = "Foundations of Mathematical Logic",
  publisher = "Dover Publications, Inc.",
  address = "New York",
  note = "[QA9.C976 1977]",
  year = 1977 }
@book{Davis, author = "Philip J. Davis and Reuben Hersh",
  title = "The Mathematical Experience",
  publisher = "Birkh{\"{a}}user Boston",
  address = "Boston",
  note = "[QA8.4.D37 1982]",
  year = 1981 }
@incollection{deMillo,
  author = "Richard de Millo and Richard Lipton and Alan Perlis",
  title = "Social Processes and Proofs of Theorems and Programs",
  pages = "267--285",
  booktitle = "New Directions in the Philosophy of Mathematics",
  editor = "Thomas Tymoczko",
  publisher = "Birkh{\"{a}}user Boston, Inc.",
  address = "Boston",
  note = "[QA8.6.N48 1986]",
  year = 1986 }
@book{Edwards, author = "Robert E. Edwards",
  title = "A Formal Background to Mathematics",
  publisher = "Springer-Verlag",
  address = "New York",
  note = "[QA37.2.E38 v.1a]",
  year = 1979 }
@book{Enderton, author = "Herbert B. Enderton",
  title = "Elements of Set Theory",
  publisher = "Academic Press, Inc.",
  address = "San Diego",
  note = "[QA248.E5]",
  year = 1977 }
@book{Goodstein, author = "R. L. Goodstein",
  title = "Development of Mathematical Logic",
  publisher = "Springer-Verlag New York Inc.",
  address = "New York",
  note = "[QA9.G6554]",
  year = 1971 }
@book{Guillen, author = "Michael Guillen",
  title = "Bridges to Infinity",
  publisher = "Jeremy P. Tarcher, Inc.",
  address = "Los Angeles",
  note = "[QA93.G8]",
  year = 1983 }
@book{Hamilton, author = "Alan G. Hamilton",
  title = "Logic for Mathematicians",
  edition = "revised",
  publisher = "Cambridge University Press",
  address = "Cambridge",
  note = "[QA9.H298]",
  year = 1988 }
@unpublished{Harrison, author = "John Robert Harrison",
  title = "Metatheory and Reflection in Theorem Proving:
    A Survey and Critique",
  note = "Technical Report
    CRC-053.
    SRI Cambridge,
    Millers Yard, Cambridge, UK,
    1995.
    Available on the Web as
{\verb+http:+}\-{\verb+//www.cl.cam.ac.uk/users/jrh/papers/reflect.html+}"}
@TECHREPORT{Harrison-thesis,
        author          = "John Robert Harrison",
        title           = "Theorem Proving with the Real Numbers",
        institution   = "University of Cambridge Computer
                         Lab\-o\-ra\-to\-ry",
        address         = "New Museums Site, Pembroke Street, Cambridge,
                           CB2 3QG, UK",
        year            = 1996,
        number          = 408,
        type            = "Technical Report",
        note            = "Author's PhD thesis,
   available on the Web at
{\verb+http:+}\-{\verb+//www.cl.cam.ac.uk+}\-{\verb+/users+}\-{\verb+/jrh+}%
\-{\verb+/papers+}\-{\verb+/thesis.html+}"}
@book{Herrlich, author = "Horst Herrlich and George E. Strecker",
  title = "Category Theory:  An Introduction",
  publisher = "Allyn and Bacon Inc.",
  address = "Boston",
  note = "[QA169.H567]",
  year = 1973 }
@article{Hindley, author = "J. Roger Hindley and David Meredith",
  title = "Principal Type-Schemes and Condensed Detachment",
  journal = "The Journal of Symbolic Logic",
  volume = 55,
  year = 1990,
  note = "[QA.J87]",
  pages = "90--105" }
@book{Hofstadter, author = "Douglas R. Hofstadter",
  title = "G{\"{o}}del, Escher, Bach",
  publisher = "Basic Books, Inc.",
  address = "New York",
  note = "[QA9.H63 1980]",
  year = 1979 }
@article{Indrzejczak, author= "Andrzej Indrzejczak",
  title = "Natural Deduction, Hybrid Systems and Modal Logic",
  journal = "Trends in Logic",
  volume = 30,
  publisher = "Springer",
  year = 2010 }
@article{Kalish, author = "D. Kalish and R. Montague",
  title = "On {T}arski's Formalization of Predicate Logic with Identity",
  journal = "Archiv f{\"{u}}r Mathematische Logik und Grundlagenfor\-schung",
  volume = 7,
  year = 1965,
  note = "[QA.A673]",
  pages = "81--101" }
@article{Kalman, author = "J. A. Kalman",
  title = "Condensed Detachment as a Rule of Inference",
  journal = "Studia Logica",
  volume = 42,
  number = 4,
  year = 1983,
  note = "[B18.P6.S933]",
  pages = "443-451" }
@book{Kline, author = "Morris Kline",
  title = "Mathematical Thought from Ancient to Modern Times",
  publisher = "Oxford University Press",
  address = "New York",
  note = "[QA21.K516 1990 v.3]",
  year = 1972 }
@book{Klinel, author = "Morris Kline",
  title = "Mathematics, The Loss of Certainty",
  publisher = "Oxford University Press",
  address = "New York",
  note = "[QA21.K525]",
  year = 1980 }
@book{Kramer, author = "Edna E. Kramer",
  title = "The Nature and Growth of Modern Mathematics",
  publisher = "Princeton University Press",
  address = "Princeton, New Jersey",
  note = "[QA93.K89 1981]",
  year = 1981 }
@article{Knill, author = "Oliver Knill",
  title = "Some Fundamental Theorems in Mathematics",
  year = "2018",
  url = "https://arxiv.org/abs/1807.08416" }
@book{Landau, author = "Edmund Landau",
  title = "Foundations of Analysis",
  publisher = "Chelsea Publishing Company",
  address = "New York",
  edition = "second",
  note = "[QA241.L2541 1960]",
  year = 1960 }
@article{Leblanc, author = "Hugues Leblanc",
  title = "On {M}eyer and {L}ambert's Quantificational Calculus {FQ}",
  journal = "The Journal of Symbolic Logic",
  volume = 33,
  year = 1968,
  note = "[QA.J87]",
  pages = "275--280" }
@article{Lejewski, author = "Czeslaw Lejewski",
  title = "On Implicational Definitions",
  journal = "Studia Logica",
  volume = 8,
  year = 1958,
  note = "[B18.P6.S933]",
  pages = "189--208" }
@book{Levy, author = "Azriel Levy",
  title = "Basic Set Theory",
  publisher = "Dover Publications",
  address = "Mineola, NY",
  year = "2002"
}
@book{Margaris, author = "Angelo Margaris",
  title = "First Order Mathematical Logic",
  publisher = "Blaisdell Publishing Company",
  address = "Waltham, Massachusetts",
  note = "[QA9.M327]",
  year = 1967}
@book{Manin, author = "Yu I. Manin",
  title = "A Course in Mathematical Logic",
  publisher = "Springer-Verlag",
  address = "New York",
  note = "[QA9.M29613]",
  year = "1977" }
@article{Mathias, author = "Adrian R. D. Mathias",
  title = "A Term of Length 4,523,659,424,929",
  journal = "Synthese",
  volume = 133,
  year = 2002,
  note = "[Q.S993]",
  pages = "75--86" }
@article{Megill, author = "Norman D. Megill",
  title = "A Finitely Axiomatized Formalization of Predicate Calculus
     with Equality",
  journal = "Notre Dame Journal of Formal Logic",
  volume = 36,
  year = 1995,
  note = "[QA.N914]",
  pages = "435--453" }
@unpublished{Megillc, author = "Norman D. Megill",
  title = "A Shorter Equivalent of the Axiom of Choice",
  month = "June",
  note = "Unpublished",
  year = 1991 }
@article{MegillBunder, author = "Norman D. Megill and Martin W.
    Bunder",
  title = "Weaker {D}-Complete Logics",
  journal = "Journal of the IGPL",
  volume = 4,
  year = 1996,
  pages = "215--225",
  note = "Available on the Web at
{\verb+http:+}\-{\verb+//www.mpi-sb.mpg.de+}\-{\verb+/igpl+}%
\-{\verb+/Journal+}\-{\verb+/V4-2+}\-{\verb+/#Megill+}"}
}
@book{Mendelson, author = "Elliott Mendelson",
  title = "Introduction to Mathematical Logic",
  edition = "second",
  publisher = "D. Van Nostrand Company, Inc.",
  address = "New York",
  note = "[QA9.M537 1979]",
  year = 1979 }
@article{Meredith, author = "David Meredith",
  title = "In Memoriam {C}arew {A}rthur {M}eredith (1904-1976)",
  journal = "Notre Dame Journal of Formal Logic",
  volume = 18,
  year = 1977,
  note = "[QA.N914]",
  pages = "513--516" }
@article{CAMeredith, author = "C. A. Meredith",
  title = "Single Axioms for the Systems ({C},{N}), ({C},{O}) and ({A},{N})
      of the Two-Valued Propositional Calculus",
  journal = "The Journal of Computing Systems",
  volume = 3,
  year = 1953,
  pages = "155--164" }
@article{Monk, author = "J. Donald Monk",
  title = "Provability With Finitely Many Variables",
  journal = "The Journal of Symbolic Logic",
  volume = 27,
  year = 1971,
  note = "[QA.J87]",
  pages = "353--358" }
@article{Monks, author = "J. Donald Monk",
  title = "Substitutionless Predicate Logic With Identity",
  journal = "Archiv f{\"{u}}r Mathematische Logik und Grundlagenfor\-schung",
  volume = 7,
  year = 1965,
  pages = "103--121" }
  %% Took out this from above to prevent LaTeX underfull warning:
  % note = "[QA.A673]",
@book{Moore, author = "A. W. Moore",
  title = "The Infinite",
  publisher = "Routledge",
  address = "New York",
  note = "[BD411.M59]",
  year = 1989}
@book{Munkres, author = "James R. Munkres",
  title = "Topology: A First Course",
  publisher = "Prentice-Hall, Inc.",
  address = "Englewood Cliffs, New Jersey",
  note = "[QA611.M82]",
  year = 1975}
@article{Nemesszeghy, author = "E. Z. Nemesszeghy and E. A. Nemesszeghy",
  title = "On Strongly Creative Definitions:  A Reply to {V}. {F}. {R}ickey",
  journal = "Logique et Analyse (N.\ S.)",
  year = 1977,
  volume = 20,
  note = "[BC.L832]",
  pages = "111--115" }
@unpublished{Nemeti, author = "N{\'{e}}meti, I.",
  title = "Algebraizations of Quantifier Logics, an Overview",
  note = "Version 11.4, preprint, Mathematical Institute, Budapest,
    1994.  A shortened version without proofs appeared in
    ``Algebraizations of quantifier logics, an introductory overview,''
   {\em Studia Logica}, 50:485--569, 1991 [B18.P6.S933]"}
@article{Pavicic, author = "M. Pavi{\v{c}}i{\'{c}}",
  title = "A New Axiomatization of Unified Quantum Logic",
  journal = "International Journal of Theoretical Physics",
  year = 1992,
  volume = 31,
  note = "[QC.I626]",
  pages = "1753 --1766" }
@book{Penrose, author = "Roger Penrose",
  title = "The Emperor's New Mind",
  publisher = "Oxford University Press",
  address = "New York",
  note = "[Q335.P415]",
  year = 1989 }
@book{PetersonI, author = "Ivars Peterson",
  title = "The Mathematical Tourist",
  publisher = "W. H. Freeman and Company",
  address = "New York",
  note = "[QA93.P475]",
  year = 1988 }
@article{Peterson, author = "Jeremy George Peterson",
  title = "An automatic theorem prover for substitution and detachment systems",
  journal = "Notre Dame Journal of Formal Logic",
  volume = 19,
  year = 1978,
  note = "[QA.N914]",
  pages = "119--122" }
@book{Quine, author = "Willard Van Orman Quine",
  title = "Set Theory and Its Logic",
  edition = "revised",
  publisher = "The Belknap Press of Harvard University Press",
  address = "Cambridge, Massachusetts",
  note = "[QA248.Q7 1969]",
  year = 1969 }
@article{Robinson, author = "J. A. Robinson",
  title = "A Machine-Oriented Logic Based on the Resolution Principle",
  journal = "Journal of the Association for Computing Machinery",
  year = 1965,
  volume = 12,
  pages = "23--41" }
@article{RobinsonT, author = "T. Thacher Robinson",
  title = "Independence of Two Nice Sets of Axioms for the Propositional
    Calculus",
  journal = "The Journal of Symbolic Logic",
  volume = 33,
  year = 1968,
  note = "[QA.J87]",
  pages = "265--270" }
@book{Rucker, author = "Rudy Rucker",
  title = "Infinity and the Mind:  The Science and Philosophy of the
    Infinite",
  publisher = "Bantam Books, Inc.",
  address = "New York",
  note = "[QA9.R79 1982]",
  year = 1982 }
@book{Russell, author = "Bertrand Russell",
  title = "Mysticism and Logic, and Other Essays",
  publisher = "Barnes \& Noble Books",
  address = "Totowa, New Jersey",
  note = "[B1649.R963.M9 1981]",
  year = 1981 }
@article{Russell2, author = "Bertrand Russell",
  title = "Recent Work on the Principles of Mathematics",
  journal = "International Monthly",
  volume = 4,
  year = 1901,
  pages = "84"}
@article{Schmidt, author = "Eric Schmidt",
  title = "Reductions in Norman Megill's axiom system for complex numbers",
  url = "http://us.metamath.org/downloads/schmidt-cnaxioms.pdf",
  year = "2012" }
@book{Shoenfield, author = "Joseph R. Shoenfield",
  title = "Mathematical Logic",
  publisher = "Addison-Wesley Publishing Company, Inc.",
  address = "Reading, Massachusetts",
  year = 1967,
  note = "[QA9.S52]" }
@book{Smullyan, author = "Raymond M. Smullyan",
  title = "Theory of Formal Systems",
  publisher = "Princeton University Press",
  address = "Princeton, New Jersey",
  year = 1961,
  note = "[QA248.5.S55]" }
@book{Solow, author = "Daniel Solow",
  title = "How to Read and Do Proofs:  An Introduction to Mathematical
    Thought Process",
  publisher = "John Wiley \& Sons",
  address = "New York",
  year = 1982,
  note = "[QA9.S577]" }
@book{Stark, author = "Harold M. Stark",
  title = "An Introduction to Number Theory",
  publisher = "Markham Publishing Company",
  address = "Chicago",
  note = "[QA241.S72 1978]",
  year = 1970 }
@article{Swart, author = "E. R. Swart",
  title = "The Philosophical Implications of the Four-Color Problem",
  journal = "American Mathematical Monthly",
  year = 1980,
  volume = 87,
  month = "November",
  note = "[QA.A5125]",
  pages = "697--707" }
@book{Szpiro, author = "George G. Szpiro",
  title = "Poincar{\'{e}}'s Prize: The Hundred-Year Quest to Solve One
    of Math's Greatest Puzzles",
  publisher = "Penguin Books Ltd",
  address = "London",
  note = "[QA43.S985 2007]",
  year = 2007}
@book{Takeuti, author = "Gaisi Takeuti and Wilson M. Zaring",
  title = "Introduction to Axiomatic Set Theory",
  edition = "second",
  publisher = "Springer-Verlag New York Inc.",
  address = "New York",
  note = "[QA248.T136 1982]",
  year = 1982}
@inproceedings{Tarski, author = "Alfred Tarski",
  title = "What is Elementary Geometry",
  pages = "16--29",
  booktitle = "The Axiomatic Method, with Special Reference to Geometry and
     Physics (Proceedings of an International Symposium held at the University
     of California, Berkeley, December 26, 1957 --- January 4, 1958)",
  editor = "Leon Henkin and Patrick Suppes and Alfred Tarski",
  year = 1959,
  publisher = "North-Holland Publishing Company",
  address = "Amsterdam"}
@article{Tarski1965, author = "Alfred Tarski",
  title = "A Simplified Formalization of Predicate Logic with Identity",
  journal = "Archiv f{\"{u}}r Mathematische Logik und Grundlagenforschung",
  volume = 7,
  year = 1965,
  note = "[QA.A673]",
  pages = "61--79" }
@book{Tymoczko,
  title = "New Directions in the Philosophy of Mathematics",
  editor = "Thomas Tymoczko",
  publisher = "Birkh{\"{a}}user Boston, Inc.",
  address = "Boston",
  note = "[QA8.6.N48 1986]",
  year = 1986 }
@incollection{Wang,
  author = "Hao Wang",
  title = "Theory and Practice in Mathematics",
  pages = "129--152",
  booktitle = "New Directions in the Philosophy of Mathematics",
  editor = "Thomas Tymoczko",
  publisher = "Birkh{\"{a}}user Boston, Inc.",
  address = "Boston",
  note = "[QA8.6.N48 1986]",
  year = 1986 }
@manual{Webster,
  title = "Webster's New Collegiate Dictionary",
  organization = "G. \& C. Merriam Co.",
  address = "Springfield, Massachusetts",
  note = "[PE1628.W4M4 1977]",
  year = 1977 }
@manual{Whitehead, author = "Alfred North Whitehead",
  title = "An Introduction to Mathematics",
  year = 1911 }
@book{PM, author = "Alfred North Whitehead and Bertrand Russell",
  title = "Principia Mathematica",
  edition = "second",
  publisher = "Cambridge University Press",
  address = "Cambridge",
  year = "1927",
  note = "(3 vols.) [QA9.W592 1927]" }
@article{DBLP:journals/corr/Whalen16,
  author    = {Daniel Whalen},
  title     = {Holophrasm: a neural Automated Theorem Prover for higher-order logic},
  journal   = {CoRR},
  volume    = {abs/1608.02644},
  year      = {2016},
  url       = {http://arxiv.org/abs/1608.02644},
  archivePrefix = {arXiv},
  eprint    = {1608.02644},
  timestamp = {Mon, 13 Aug 2018 16:46:19 +0200},
  biburl    = {https://dblp.org/rec/bib/journals/corr/Whalen16},
  bibsource = {dblp computer science bibliography, https://dblp.org} }
@article{Wiedijk-revisited,
  author = {Freek Wiedijk},
  title = {The QED Manifesto Revisited},
  year = {2007},
  url = {http://mizar.org/trybulec65/8.pdf} }
@book{Wolfram,
  author = "Stephen Wolfram",
  title = "Mathematica:  A System for Doing Mathematics by Computer",
  edition = "second",
  publisher = "Addison-Wesley Publishing Co.",
  address = "Redwood City, California",
  note = "[QA76.95.W65 1991]",
  year = 1991 }
@book{Wos, author = "Larry Wos and Ross Overbeek and Ewing Lusk and Jim Boyle",
  title = "Automated Reasoning:  Introduction and Applications",
  edition = "second",
  publisher = "McGraw-Hill, Inc.",
  address = "New York",
  note = "[QA76.9.A96.A93 1992]",
  year = 1992 }

%
%
%[1] Church, Alonzo, Introduction to Mathematical Logic,
% Volume 1, Princeton University Press, Princeton, N. J., 1956.
%
%[2] Cohen, Paul J., Set Theory and the Continuum Hypothesis,
% W. A. Benjamin, Inc., Reading, Mass., 1966.
%
%[3] Hamilton, Alan G., Logic for Mathematicians, Cambridge
% University Press,
% Cambridge, 1988.

%[6] Kleene, Stephen Cole, Introduction to Metamathematics, D.  Van
% Nostrand Company, Inc., Princeton (1952).

%[13] Tarski, Alfred, "A simplified formalization of predicate
% logic with identity," Archiv fur Mathematische Logik und
% Grundlagenforschung, vol. 7 (1965), pp. 61-79.

%[14] Tarski, Alfred and Steven Givant, A Formalization of Set
% Theory Without Variables, American Mathematical Society Colloquium
% Publications, vol. 41, American Mathematical Society,
% Providence, R. I., 1987.

%[15] Zeman, J. J., Modal Logic, Oxford University Press, Oxford, 1973.
\end{filecontents}
% --------------------------- End of metamath.bib -----------------------------


%Book: Metamath
%Author:  Norman Megill Email:  nm at alum.mit.edu
%Author:  David A. Wheeler Email:  dwheeler at dwheeler.com

% A book template example
% http://www.stsci.edu/ftp/software/tex/bookstuff/book.template

\documentclass[leqno]{book} % LaTeX 2e. 10pt. Use [leqno,12pt] for 12pt
% hyperref 2002/05/27 v6.72r  (couldn't get pagebackref to work)
\usepackage[plainpages=false,pdfpagelabels=true]{hyperref}

\usepackage{needspace}     % Enable control over page breaks
\usepackage{breqn}         % automatic equation breaking
\usepackage{microtype}     % microtypography, reduces hyphenation

% Packages for flexible tables.  We need to be able to
% wrap text within a cell (with automatically-determined widths) AND
% split a table automatically across multiple pages.
% * "tabularx" wraps text in cells but only 1 page
% * "longtable" goes across pages but by itself is incompatible with tabularx
% * "ltxtable" combines longtable and tabularx, but table contents
%    must be in a separate file.
% * "ltablex" combines tabularx and longtable - must install specially
% * "booktabs" is recommended as a way to improve the look of tables,
%   but doesn't add these capabilities.
% * "tabu" much more capable and seems to be recommended. So use that.

\usepackage{makecell}      % Enable forced line splits within a table cell
\usepackage{longtable}     % Enable multi-page tables
\usepackage{tabu}          % Multi-page tables with wrapped text in a cell

% You can find more Tex packages using commands like:
% tlmgr search --file tabu.sty
% find /usr/share/texmf-dist/ -name '*tab*'
%
%%%%%%%%%%%%%%%%%%%%%%%%%%%%%%%%%%%%%%%%%%%%%%%%%%%%%%%%%%%%%%%%%%%%%%%%%%%%
% Uncomment the next 3 lines to suppress boxes and colors on the hyperlinks
%%%%%%%%%%%%%%%%%%%%%%%%%%%%%%%%%%%%%%%%%%%%%%%%%%%%%%%%%%%%%%%%%%%%%%%%%%%%
%\hypersetup{
%colorlinks,citecolor=black,filecolor=black,linkcolor=black,urlcolor=black
%}
%
\usepackage{realref}

% Restarting page numbers: try?
%   \printglossary
%   \cleardoublepage
%   \pagenumbering{arabic}
%   \setcounter{page}{1}    ???needed
%   \include{chap1}

% not used:
% \def\R2Lurl#1#2{\mbox{\href{#1}\texttt{#2}}}

\usepackage{amssymb}

% Version 1 of book: margins: t=.4, b=.2, ll=.4, rr=.55
% \usepackage{anysize}
% % \papersize{<height>}{<width>}
% % \marginsize{<left>}{<right>}{<top>}{<bottom>}
% \papersize{9in}{6in}
% % l/r 0.6124-0.6170 works t/b 0.2418-0.3411 = 192pp. 0.2926-03118=exact
% \marginsize{0.7147in}{0.5147in}{0.4012in}{0.2012in}

\usepackage{anysize}
% \papersize{<height>}{<width>}
% \marginsize{<left>}{<right>}{<top>}{<bottom>}
\papersize{9in}{6in}
% l/r 0.85in&0.6431-0.6539 works t/b ?-?
%\marginsize{0.85in}{0.6485in}{0.55in}{0.35in}
\marginsize{0.8in}{0.65in}{0.5in}{0.3in}

% \usepackage[papersize={3.6in,4.8in},hmargin=0.1in,vmargin={0.1in,0.1in}]{geometry}  % page geometry
\usepackage{special-settings}

\raggedbottom
\makeindex

\begin{document}
% Discourage page widows and orphans:
\clubpenalty=300
\widowpenalty=300

%%%%%%% load in AMS fonts %%%%%%% % LaTeX 2.09 - obsolete in LaTeX 2e
%\input{amssym.def}
%\input{amssym.tex}
%\input{c:/texmf/tex/plain/amsfonts/amssym.def}
%\input{c:/texmf/tex/plain/amsfonts/amssym.tex}

\bibliographystyle{plain}
\pagenumbering{roman}
\pagestyle{headings}

\thispagestyle{empty}

\hfill
\vfill

\begin{center}
{\LARGE\bf Metamath} \\
\vspace{1ex}
{\large A Computer Language for Mathematical Proofs} \\
\vspace{7ex}
{\large Norman Megill} \\
\vspace{7ex}
with extensive revisions by \\
\vspace{1ex}
{\large David A. Wheeler} \\
\vspace{7ex}
% Printed date. If changing the date below, also fix the date at the beginning.
2019-05-24
\end{center}

\vfill
\hfill

\newpage
\thispagestyle{empty}

\hfill
\vfill

\begin{center}
$\sim$\ {\sc Public Domain}\ $\sim$

\vspace{2ex}
This book (including its later revisions)
has been released into the Public Domain by Norman Megill per the
Creative Commons CC0 1.0 Universal (CC0 1.0) Public Domain Dedication
(\url{https://creativecommons.org/publicdomain/zero/1.0/}).
David A. Wheeler has done the same.
The public domain release applies worldwide.  In case this is not
legally possible, the right is granted to use the work for any purpose,
without any conditions, unless such conditions are required by law.

\vspace{3ex}
Several short, attributed quotations from copyrighted works
appear in this book under the ``fair use'' provision of Section 107 of
the United States Copyright Act (Title 17 of the {\em United States
Code}).  The public-domain status of this book is not applicable to
those quotations.

\vspace{3ex}
Any trademarks used in this book are the property of their owners.

% QA76.9.L63.M??

% \vspace{1ex}
%
% \vspace{1ex}
% {\small Permission is granted to make and distribute verbatim copies of this
% book
% provided the copyright notice and this
% permission notice are preserved on all copies.}
%
% \vspace{1ex}
% {\small Permission is granted to copy and distribute modified versions of this
% book under the conditions for verbatim copying, provided that the
% entire
% resulting derived work is distributed under the terms of a permission
% notice
% identical to this one.}
%
% \vspace{1ex}
% {\small Permission is granted to copy and distribute translations of this
% book into another language, under the above conditions for modified
% versions,
% except that this permission notice may be stated in a translation
% approved by the
% author.}
%
% \vspace{1ex}
% %{\small   For a copy of the \LaTeX\ source files for this book, contact
% %the author.} \\
% \ \\
% \ \\

\vspace{7ex}
% ISBN: 1-4116-3724-0 \\
ISBN: 978-1-4116-3724-5 \\
{\ } \\
Lulu Press \\
Morrisville, North Carolina\\
USA


\hfill
\vfill

Norman Megill\\ 93 Bridge St., Lexington, MA 02421 \\
E-mail address: \texttt{nm{\char`\@}alum.mit.edu} \\
\vspace{7ex}
David A. Wheeler \\
E-mail address: \texttt{dwheeler{\char`\@}dwheeler.com} \\
% See notes added at end of Preface for revision history. \\
% For current information on the Metamath software see \\
\vspace{7ex}
\url{http://metamath.org}
\end{center}

% \newpage
% \thispagestyle{empty}
%
% \hfill
% \vfill
%
% \begin{center}
% {\it To my son Robin Dwight Megill}
% \end{center}
%
% \vfill
% \hfill
%
% \newpage

\tableofcontents
%\listoftables

\chapter*{Preface}
\markboth{PREFACE}{PREFACE}
\addcontentsline{toc}{section}{Preface}


% (For current information, see the notes added at the
% end of this preface on p.~\pageref{note2002}.)

\subsubsection{Overview}

Metamath\index{Metamath} is a computer language and an associated computer
program for archiving, verifying, and studying mathematical proofs at a very
detailed level.  The Metamath language incorporates no mathematics per se but
treats all mathematical statements as mere sequences of symbols.  You provide
Metamath with certain special sequences (axioms) that tell it what rules
of inference are allowed.  Metamath is not limited to any specific field of
mathematics.  The Metamath language is simple and robust, with an
almost total absence of hard-wired syntax, and
I\footnote{Unless otherwise noted, the words
``I,'' ``me,'' and ``my'' refer to Norman Megill\index{Megill, Norman}, while
``we,'' ``us,'' and ``our'' refer to Norman Megill and
David A. Wheeler\index{Wheeler, David A.}.}
believe that it
provides about the simplest possible framework that allows essentially all of
mathematics to be expressed with absolute rigor.

% index test
%\newcommand{\nn}[1]{#1n}
%\index{aaa@bbb}
%\index{abc!def}
%\index{abd|see{qqq}}
%\index{abe|nn}
%\index{abf|emph}
%\index{abg|(}
%\index{abg|)}

Using the Metamath language, you can build formal or mathematical
systems\index{formal system}\footnote{A formal or mathematical system consists
of a collection of symbols (such as $2$, $4$, $+$ and $=$), syntax rules that
describe how symbols may be combined to form a legal expression (called a
well-formed formula or {\em wff}, pronounced ``whiff''), some starting wffs
called axioms, and inference rules that describe how theorems may be derived
(proved) from the axioms.  A theorem is a mathematical fact such as $2+2=4$.
Strictly speaking, even an obvious fact such as this must be proved from
axioms to be formally acceptable to a mathematician.}\index{theorem}
\index{axiom}\index{rule}\index{well-formed formula (wff)} that involve
inferences from axioms.  Although a database is provided
that includes a recommended set of axioms for standard mathematics, if you
wish you can supply your own symbols, syntax, axioms, rules, and definitions.

The name ``Metamath'' was chosen to suggest that the language provides a
means for {\em describing} mathematics rather than {\em being} the
mathematics itself.  Actually in some sense any mathematical language is
metamathematical.  Symbols written on paper, or stored in a computer,
are not mathematics itself but rather a way of expressing mathematics.
For example ``7'' and ``VII'' are symbols for denoting the number seven
in Arabic and Roman numerals; neither {\em is} the number seven.

If you are able to understand and write computer programs, you should be able
to follow abstract mathematics with the aid of Metamath.  Used in conjunction
with standard textbooks, Metamath can guide you step-by-step towards an
understanding of abstract mathematics from a very rigorous viewpoint, even if
you have no formal abstract mathematics background.  By using a single,
consistent notation to express proofs, once you grasp its basic concepts
Metamath provides you with the ability to immediately follow and dissect
proofs even in totally unfamiliar areas.

Of course, just being able follow a proof will not necessarily give you an
intuitive familiarity with mathematics.  Memorizing the rules of chess does not
give you the ability to appreciate the game of a master, and knowing how the
notes on a musical score map to piano keys does not give you the ability to
hear in your head how it would sound.  But each of these can be a first step.

Metamath allows you to explore proofs in the sense that you can see the
theorem referenced at any step expanded in as much detail as you want, right
down to the underlying axioms of logic and set theory (in the case of the set
theory database provided).  While Metamath will not replace the higher-level
understanding that can only be acquired through exercises and hard work, being
able to see how gaps in a proof are filled in can give you increased
confidence that can speed up the learning process and save you time when you
get stuck.

The Metamath language breaks down a mathematical proof into its tiniest
possible parts.  These can be pieced together, like interlocking
pieces in a puzzle, only in a way that produces correct and absolutely rigorous
mathematics.

The nature of Metamath\index{Metamath} enforces very precise mathematical
thinking, similar to that involved in writing a computer program.  A crucial
difference, though, is that once a proof is verified (by the Metamath program)
to be correct, it is definitely correct; it can never have a hidden
``bug.''\index{computer program bugs}  After getting used to the kind of rigor
and accuracy provided by Metamath, you might even be tempted to
adopt the attitude that a proof should never be considered correct until it
has been verified by a computer, just as you would not completely trust a
manual calculation until you have verified it on a
calculator.

My goal
for Metamath was a system for describing and verifying
mathematics that is completely universal yet conceptually as simple as
possible.  In approaching mathematics from an axiomatic, formal viewpoint, I
wanted Metamath to be able to handle almost any mathematical system, not
necessarily with ease, but at least in principle and hopefully in practice. I
wanted it to verify proofs with absolute rigor, and for this reason Metamath
is what might be thought of as a ``compile-only'' language rather than an
algorithmic or Turing-machine language (Pascal, C, Prolog, Mathematica,
etc.).  In other words, a database written in the Metamath
language doesn't ``do'' anything; it merely exhibits mathematical knowledge
and permits this knowledge to be verified as being correct.  A program in an
algorithmic language can potentially have hidden bugs\index{computer program
bugs} as well as possibly being hard to understand.  But each token in a
Metamath database must be consistent with the database's earlier
contents according to simple, fixed rules.
If a database is verified
to be correct,\footnote{This includes
verification that a sequential list of proof steps results in the specified
theorem.} then the mathematical content is correct if the
verifier is correct and the axioms are correct.
The verification program could be incorrect, but the verification algorithm
is relatively simple (making it unlikely to be implemented incorrectly
by the Metamath program),
and there are over a dozen Metamath database verifiers
written by different people in different programming languages
(so these different verifiers can act as multiple reviewers of a database).
The most-used Metamath database, the Metamath Proof Explorer
(aka \texttt{set.mm}\index{set theory database (\texttt{set.mm})}%
\index{Metamath Proof Explorer}),
is currently verified by four different Metamath verifiers written by
four different people in four different languages, including the
original Metamath program described in this book.
The only ``bugs'' that can exist are in the statement of the axioms,
for example if the axioms are inconsistent (a famous problem shown to be
unsolvable by G\"{o}del's incompleteness theorem\index{G\"{o}del's
incompleteness theorem}).
However, real mathematical systems have very few axioms, and these can
be carefully studied.
All of this provides extraordinarily high confidence that the verified database
is in fact correct.

The Metamath program
doesn't prove theorems automatically but is designed to verify proofs
that you supply to it.
The underlying Metamath language is completely general and has no built-in,
preconceived notions about your formal system\index{formal system}, its logic
or its syntax.
For constructing proofs, the Metamath program has a Proof Assistant\index{Proof
Assistant} which helps you fill in some of a proof step's details, shows you
what choices you have at any step, and verifies the proof as you build it; but
you are still expected to provide the proof.

There are many other programs that can process or generate information
in the Metamath language, and more continue to be written.
This is in part because the Metamath language itself is very simple
and intentionally easy to automatically process.
Some programs, such as \texttt{mmj2}\index{mmj2}, include a proof assistant
that can automate some steps beyond what the Metamath program can do.
Mario Carneiro has developed an algorithm for converting proofs from
the OpenTheory interchange format, which can be translated to and from
any of the HOL family of proof languages (HOL4, HOL Light, ProofPower,
and Isabelle), into the
Metamath language.\cite{DBLP:journals/corr/Carneiro14}\index{Carneiro, Mario}.
Daniel Whalen has developed Holophrasm, which can automatically
prove many Metamath proofs using
machine learning\index{machine learning}\index{artificial intelligence}
approaches
(including multiple neural networks\index{neural networks})\cite{DBLP:journals/corr/Whalen16}\index{Whalen, Daniel}.
However,
a discussion of these other programs is beyond the scope of this book.

Like most computer languages, the Metamath\index{Metamath} language uses the
standard ({\sc ascii}) characters on a computer keyboard, so it cannot
directly represent many of the special symbols that mathematicians use.  A
useful feature of the Metamath program is its ability to convert its notation
into the \LaTeX\ typesetting language.\index{latex@{\LaTeX}}  This feature
lets you convert the {\sc ascii} tokens you've defined into standard
mathematical symbols, so you end up with symbols and formulas you are familiar
with instead of somewhat cryptic {\sc ascii} representations of them.
The Metamath program can also generate HTML\index{HTML}, making it easy
to view results on the web and to see related information by using
hypertext links.

Metamath is probably conceptually different from anything you've seen
before and some aspects may take some getting used to.  This book will
help you decide whether Metamath suits your specific needs.

\subsubsection{Setting Your Expectations}
It is important for you to understand what Metamath\index{Metamath} is and is
not.  As mentioned, the Metamath program
is {\em not} an automated theorem prover but
rather a proof verifier.  Developing a database can be tedious, hard work,
especially if you want to make the proofs as short as possible, but it becomes
easier as you build up a collection of useful theorems.  The purpose of
Metamath is simply to document existing mathematics in an absolutely rigorous,
computer-verifiable way, not to aid directly in the creation of new
mathematics.  It also is not a magic solution for learning abstract
mathematics, although it may be helpful to be able to actually see the implied
rigor behind what you are learning from textbooks, as well as providing hints
to work out proofs that you are stumped on.

As of this writing, a sizable set theory database has been developed to
provide a foundation for many fields of mathematics, but much more work would
be required to develop useful databases for specific fields.

Metamath\index{Metamath} ``knows no math;'' it just provides a framework in
which to express mathematics.  Its language is very small.  You can define two
kinds of symbols, constants\index{constant} and variables\index{variable}.
The only thing Metamath knows how to do is to substitute strings of symbols
for the variables\index{substitution!variable}\index{variable substitution} in
an expression based on instructions you provide it in a proof, subject to
certain constraints you specify for the variables.  Even the decimal
representation of a number is merely a string of certain constants (digits)
which together, in a specific context, correspond to whatever mathematical
object you choose to define for it; unlike other computer languages, there is
no actual number stored inside the computer.  In a proof, you in effect
instruct Metamath what symbol substitutions to make in previous axioms or
theorems and join a sequence of them together to result in the desired
theorem.  This kind of symbol manipulation captures the essence of mathematics
at a preaxiomatic level.

\subsubsection{Metamath and Mathematical Literature}

In advanced mathematical literature, proofs are usually presented in the form
of short outlines that often only an expert can follow.  This is partly out of
a desire for brevity, but it would also be unwise (even if it were practical)
to present proofs in complete formal detail, since the overall picture would
be lost.\index{formal proof}

A solution I envision\label{envision} that would allow mathematics to remain
acceptable to the expert, yet increase its accessibility to non-specialists,
consists of a combination of the traditional short, informal proof in print
accompanied by a complete formal proof stored in a computer database.  In an
analogy with a computer program, the informal proof is like a set of comments
that describe the overall reasoning and content of the proof, whereas the
computer database is like the actual program and provides a means for anyone,
even a non-expert, to follow the proof in as much detail as desired, exploring
it back through layers of theorems (like subroutines that call other
subroutines) all the way back to the axioms of the theory.  In addition, the
computer database would have the advantage of providing absolute assurance
that the proof is correct, since each step can be verified automatically.

There are several other approaches besides Metamath to a project such
as this.  Section~\ref{proofverifiers} discusses some of these.

To us, a noble goal would be a database with hundreds of thousands of
theorems and their computer-verifiable proofs, encompassing a significant
fraction of known mathematics and available for instant access.
These would be fully verified by multiple independently-implemented verifiers,
to provide extremely high confidence that the proofs are completely correct.
The database would allow people to investigate whatever details they were
interested in, so that they could confirm whatever portions they wished.
Whether or not Metamath is an appropriate choice remains to be seen, but in
principle we believe it is sufficient.

\subsubsection{Formalism}

Over the past fifty years, a group of French mathematicians working
collectively under the pseudonym of Bourbaki\index{Bourbaki, Nicolas} have
co-authored a series of monographs that attempt to rigorously and
consistently formalize large bodies of mathematics from foundations.  On the
one hand, certainly such an effort has its merits; on the other hand, the
Bourbaki project has been criticized for its ``scholasticism'' and
``hyperaxiomatics'' that hide the intuitive steps that lead to the results
\cite[p.~191]{Barrow}\index{Barrow, John D.}.

Metamath unabashedly carries this philosophy to its extreme and no doubt is
subject to the same kind of criticism.  Nonetheless I think that in
conjunction with conventional approaches to mathematics Metamath can serve a
useful purpose.  The Bourbaki approach is essentially pedagogic, requiring the
reader to become intimately familiar with each detail in a very large
hierarchy before he or she can proceed to the next step.  The difference with
Metamath is that the ``reader'' (user) knows that all details are contained in
its computer database, available as needed; it does not demand that the user
know everything but conveniently makes available those portions that are of
interest.  As the body of all mathematical knowledge grows larger and larger,
no one individual can have a thorough grasp of its entirety.  Metamath
can finalize and put to rest any questions about the validity of any part of it
and can make any part of it accessible, in principle, to a non-specialist.

\subsubsection{A Personal Note}
Why did I develop Metamath\index{Metamath}?  I enjoy abstract mathematics, but
I sometimes get lost in a barrage of definitions and start to lose confidence
that my proofs are correct.  Or I reach a point where I lose sight of how
anything I'm doing relates to the axioms that a theory is based on and am
sometimes suspicious that there may be some overlooked implicit axiom
accidentally introduced along the way (as happened historically with Euclidean
geometry\index{Euclidean geometry}, whose omission of Pasch's
axiom\index{Pasch's axiom} went unnoticed for 2000 years
\cite[p.~160]{Davis}!). I'm also somewhat lazy and wish to avoid the effort
involved in re-verifying the gaps in informal proofs ``left to the reader;'' I
prefer to figure them out just once and not have to go through the same
frustration a year from now when I've forgotten what I did.  Metamath provides
better recovery of my efforts than scraps of paper that I can't
decipher anymore.  But mostly I find very appealing the idea of rigorously
archiving mathematical knowledge in a computer database, providing precision,
certainty, and elimination of human error.

\subsubsection{Note on Bibliography and Index}

The Bibliography usually includes the Library of Congress classification
for a work to make it easier for you to find it in on a university
library shelf.  The Index has author references to pages where their works
are cited, even though the authors' names may not appear on those pages.

\subsubsection{Acknowledgments}

Acknowledgments are first due to my wife, Deborah (who passed away on
September 4, 1998), for critiquing the manu\-script but most of all for
her patience and support.  I also wish to thank Joe Wright, Richard
Becker, Clarke Evans, Buddha Buck, and Jeremy Henty for helpful
comments.  Any errors, omissions, and other shortcomings are of course
my responsibility.

\subsubsection{Note Added June 22, 2005}\label{note2002}

The original, unpublished version of this book was written in 1997 and
distributed via the web.  The present edition has been updated to
reflect the current Metamath program and databases, as well as more
current {\sc url}s for Internet sites.  Thanks to Josh
Purinton\index{Purinton, Josh}, One Hand
Clapping, Mel L.\ O'Cat, and Roy F. Longton for pointing out
typographical and other errors.  I have also benefitted from numerous
discussions with Raph Levien\index{Levien, Raph}, who has extended
Metamath's philosophy of rigor to result in his {\em
Ghilbert}\index{Ghilbert} proof language (\url{http://ghilbert.org}).

Robert (Bob) Solovay\index{Solovay, Robert} communicated a new result of
A.~R.~D.~Mathias on the system of Bourbaki, and the text has been
updated accordingly (p.~\pageref{bourbaki}).

Bob also pointed out a clarification of the literature regarding
category theory and inaccessible cardinals\index{category
theory}\index{cardinal, inaccessible} (p.~\pageref{categoryth}),
and a misleading statement was removed from the text.  Specifically,
contrary to a statement in previous editions, it is possible to express
``There is a proper class of inaccessible cardinals'' in the language of
ZFC.  This can be done as follows:  ``For every set $x$ there is an
inaccessible cardinal $\kappa$ such that $\kappa$ is not in $x$.''
Bob writes:\footnote{Private communication, Nov.~30, 2002.}
\begin{quotation}
     This axiom is how Grothendieck presents category theory.  To each
inaccessible cardinal $\kappa$ one associates a Grothendieck universe
\index{Grothendieck, Alexander} $U(\kappa)$.  $U(\kappa)$ consists of
those sets which lie in a transitive set of cardinality less than
$\kappa$.  Instead of the ``category of all groups,'' one works relative
to a universe [considering the category of groups of cardinality less
than $\kappa$].  Now the category whose objects are all categories
``relative to the universe $U(\kappa)$'' will be a category not
relative to this universe but to the next universe.

     All of the things category theorists like to do can be done in this
framework.  The only controversial point is whether the Grothen\-dieck
axiom is too strong for the needs of category theorists.  Mac Lane
\index{Mac Lane, Saunders} argues that ``one universe is enough'' and
Feferman\index{Feferman, Solomon} has argued that one can get by with
ordinary ZFC.  I don't find Feferman's arguments persuasive.  Mac Lane
may be right, but when I think about category theory I do it \`{a} la
Grothendieck.

        By the way Mizar\index{Mizar} adds the axiom ``there is a proper
class of inaccessibles'' precisely so as to do category theory.
\end{quotation}

The most current information on the Metamath program and databases can
always be found at \url{http://metamath.org}.


\subsubsection{Note Added June 24, 2006}\label{note2006}

The Metamath spec was restricted slightly to make parsers easier to
write.  See the footnote on p.~\pageref{namespace}.

%\subsubsection{Note Added July 24, 2006}\label{note2006b}
\subsubsection{Note Added March 10, 2007}\label{note2006b}

I am grateful to Anthony Williams\index{Williams, Anthony} for writing
the \LaTeX\ package called {\tt realref.sty} and contributing it to the
public domain.  This package allows the internal hyperlinks in a {\sc
pdf} file to anchor to specific page numbers instead of just section
titles, making the navigation of the {\sc pdf} file for this book much
more pleasant and ``logical.''

A typographical error found by Martin Kiselkov was corrected.
A confusing remark about unification was deleted per suggestion of
Mel O'Cat.

\subsubsection{Note Added May 27, 2009}\label{note2009}

Several typos found by Kim Sparre were corrected.  A note was added that
the Poincar\'{e} conjecture has been proved (p.~\pageref{poincare}).

\subsubsection{Note Added Nov. 17, 2014}\label{note2014}

The statement of the Schr\"{o}der--Bernstein theorem was corrected in
Section~\ref{trust}.  Thanks to Bob Solovay for pointing out the error.

\subsubsection{Note Added May 25, 2016}\label{note2016}

Thanks to Jerry James for correcting 16 typos.

\subsubsection{Note Added February 25, 2019}\label{note201902}

David A. Wheeler\index{Wheeler, David A.}
made a large number of improvements and updates,
in coordination with Norman Megill.
The predicate calculus axioms were renumbered, and the text makes
it clear that they are based on Tarski's system S2;
the one slight deviation in axiom ax-6 is explained and justified.
The real and complex number axioms were modified to be consistent with
\texttt{set.mm}\index{set theory database (\texttt{set.mm})}%
\index{Metamath Proof Explorer}.
Long-awaited specification changes ``1--8'' were made,
which clarified previously ambiguous points.
Some errors in the text involving \texttt{\$f} and
\texttt{\$d} statements were corrected (the spec was correct, but
the in-book explanations unintentionally contradicted the spec).
We now have a system for automatically generating narrow PDFs,
so that those with smartphones can have easy access to the current
version of this document.
A new section on deduction was added;
it discusses the standard deduction theorem,
the weak deduction theorem,
deduction style, and natural deduction.
Many minor corrections (too numerous to list here) were also made.

\subsubsection{Note Added March 7, 2019}\label{note201903}

This added a description of the Matamath language syntax in
Extended Backus--Naur Form (EBNF)\index{Extended Backus--Naur Form}\index{EBNF}
in Appendix \ref{BNF}, added a brief explanation about typecodes,
inserted more examples in the deduction section,
and added a variety of smaller improvements.

\subsubsection{Note Added April 7, 2019}\label{note201904}

This version clarified the proper substitution notation, improved the
discussion on the weak deduction theorem and natural deduction,
documented the \texttt{undo} command, updated the information on
\texttt{write source}, changed the typecode
from \texttt{set} to \texttt{setvar} to be consistent with the current
version of \texttt{set.mm}, added more documentation about comment markup
(e.g., documented how to create headings), and clarified the
differences between various assertion forms (in particular deduction form).

\subsubsection{Note Added May 20, 2019}\label{note201905}

This version fixes various small issues and added specific examples
of forms (deduction form, inference form, and closed form).

\chapter{Introduction}
\pagenumbering{arabic}

\begin{quotation}
  {\em {\em I.M.:}  No, no.  There's nothing subjective about it!  Everybody
knows what a proof is.  Just read some books, take courses from a competent
mathematician, and you'll catch on.

{\em Student:}  Are you sure?

{\em I.M.:}  Well---it is possible that you won't, if you don't have any
aptitude for it.  That can happen, too.

{\em Student:}  Then {\em you} decide what a proof is, and if I don't learn
to decide in the same way, you decide I don't have any aptitude.

{\em I.M.:}  If not me, then who?}
    \flushright\sc  ``The Ideal Mathematician''
    \index{Davis, Phillip J.}
    \footnote{\cite{Davis}, p.~40.}\\
\end{quotation}

In the past century, brilliant mathematicians have discovered almost
unimaginably profound results that rank among the crowning intellectual
achievements of mankind.  However, there is a sense in which modern abstract
mathematics is behind the times, stuck in an era before computers existed.
While no one disputes the remarkable results that have been achieved,
communicating these results in a precise way to the uninitiated is virtually
impossible.  To describe these results, a terse informal language is used which
despite its elegance is very difficult to learn.  This informal language is not
imprecise, far from it, but rather it often has omitted detail
and symbols with hidden context that are
implicitly understood by an expert but few others.  Extremely complex technical
meanings are associated with innocent-sounding English words such as
``compact'' and ``measurable'' that barely hint at what is actually being
said.  Anyone who does not keep the precise technical meaning constantly in
mind is bound to fail, and acquiring the ability to do this can be achieved
only through much practice and hard work.  Only the few who complete the
painful learning experience can join the small in-group of pure
mathematicians.  The informal language effectively cuts off the true nature of
their knowledge from most everyone else.

Metamath\index{Metamath} makes abstract mathematics more concrete.  It allows
a computer to keep track of the complexity associated with each word or symbol
with absolute rigor.  You can explore this complexity at your leisure, to
whatever degree you desire.  Whether or not you believe that concepts such as
infinity actually ``exist'' outside of the mind, Metamath lets you get to the
foundation for what's really being said.

Metamath also enables completely rigorous and thorough proof verification.
Its language is simple enough so that you
don't have to rely on the authority of experts but can verify the results
yourself, step by step.  If you want to attempt to derive your own results,
Metamath will not let you make a mistake in reasoning.
Even professional mathematicians make mistakes; Metamath makes it possible
to thoroughly verify that proofs are correct.

Metamath\index{Metamath} is a computer language and an associated computer
program for archiving, verifying, and studying mathematical proofs at a very
detailed level.
The Metamath language
describes formal\index{formal system} mathematical
systems and expresses proofs of theorems in those systems.  Such a language
is called a metalanguage\index{metalanguage} by mathematicians.
The Metamath program is a computer program that verifies
proofs expressed in the Metamath language.
The Metamath program does not have the built-in
ability to make logical inferences; it just makes a series of symbol
substitutions according to instructions given to it in a proof
and verifies that the result matches the expected theorem.  It makes logical
inferences based only on rules of logic that are contained in a set of
axioms\index{axiom}, or first principles, that you provide to it as the
starting point for proofs.

The complete specification of the Metamath language is only four pages long
(Section~\ref{spec}, p.~\pageref{spec}).  Its simplicity may at first make you
wonder how it can do much of anything at all.  But in fact the kinds of
symbol manipulations it performs are the ones that are implicitly done in all
mathematical systems at the lowest level.  You can learn it relatively quickly
and have complete confidence in any mathematical proof that it verifies.  On
the other hand, it is powerful and general enough so that virtually any
mathematical theory, from the most basic to the deeply abstract, can be
described with it.

Although in principle Metamath can be used with any
kind of mathematics, it is best suited for abstract or ``pure'' mathematics
that is mostly concerned with theorems and their proofs, as opposed to the
kind of mathematics that deals with the practical manipulation of numbers.
Examples of branches of pure mathematics are logic\index{logic},\footnote{Logic
is the study of statements that are universally true regardless of the objects
being described by the statements.  An example is the statement, ``if $P$
implies $Q$, then either $P$ is false or $Q$ is true.''} set theory\index{set
theory},\footnote{Set theory is the study of general-purpose mathematical objects called
``sets,'' and from it essentially all of mathematics can be derived.  For
example, numbers can be defined as specific sets, and their properties
can be explored using the tools of set theory.} number theory\index{number
theory},\footnote{Number theory deals with the properties of positive and
negative integers (whole numbers).} group theory\index{group
theory},\footnote{Group theory studies the properties of mathematical objects
called groups that obey a simple set of axioms and have properties of symmetry
that make them useful in many other fields.} abstract algebra\index{abstract
algebra},\footnote{Abstract algebra includes group theory and also studies
groups with additional properties that qualify them as ``rings'' and
``fields.''  The set of real numbers is a familiar example of a field.},
analysis\index{analysis} \index{real and complex numbers}\footnote{Analysis is
the study of real and complex numbers.} and
topology\index{topology}.\footnote{One area studied by topology are properties
that remain unchanged when geometrical objects undergo stretching
deformations; for example a doughnut and a coffee cup each have one hole (the
cup's hole is in its handle) and are thus considered topologically
equivalent.  In general, though, topology is the study of abstract
mathematical objects that obey a certain (surprisingly simple) set of axioms.
See, for example, Munkres \cite{Munkres}\index{Munkres, James R.}.} Even in
physics, Metamath could be applied to certain branches that make use of
abstract mathematics, such as quantum logic\index{quantum logic} (used to study
aspects of quantum mechanics\index{quantum mechanics}).

On the other hand, Metamath\index{Metamath} is less suited to applications
that deal primarily with intensive numeric computations.  Metamath does not
have any built-in representation of numbers\index{Metamath!representation of
numbers}; instead, a specific string of symbols (digits) must be syntactically
constructed as part of any proof in which an ordinary number is used.  For
this reason, numbers in Metamath are best limited to specific constants that
arise during the course of a theorem or its proof.  Numbers are only a tiny
part of the world of abstract mathematics.  The exclusion of built-in numbers
was a conscious decision to help achieve Metamath's simplicity, and there are
other software tools if you have different mathematical needs.
If you wish to quickly solve algebraic problems, the computer algebra
programs\index{computer algebra system} {\sc
macsyma}\index{macsyma@{\sc macsyma}}, Mathematica\index{Mathematica}, and
Maple\index{Maple} are specifically suited to handling numbers and
algebra efficiently.
If you wish to simply calculate numeric or matrix expressions easily,
tools such as Octave\index{Octave} may be a better choice.

After learning Metamath's basic statement types, any
tech\-ni\-cal\-ly ori\-ent\-ed person, mathematician or not, can
immediately trace
any theorem proved in the language as far back as he or she wants, all the way
to the axioms on which the theorem is based.  This ability suggests a
non-traditional way of learning about pure mathematics.  Used in conjunction
with traditional methods, Metamath could make pure mathematics accessible to
people who are not sufficiently skilled to figure out the implicit detail in
ordinary textbook proofs.  Once you learn the axioms of a theory, you can have
complete confidence that everything you need to understand a proof you are
studying is all there, at your beck and call, allowing you to focus in on any
proof step you don't understand in as much depth as you need, without worrying
about getting stuck on a step you can't figure out.\footnote{On the other
hand, writing proofs in the Metamath language is challenging, requiring
a degree of rigor far in excess of that normally taught to students.  In a
classroom setting, I doubt that writing Metamath proofs would ever replace
traditional homework exercises involving informal proofs, because the time
needed to work out the details would not allow a course to
cover much material.  For students who have trouble grasping the implied rigor
in traditional material, writing a few simple proofs in the Metamath language
might help clarify fuzzy thought processes.  Although somewhat difficult at
first, it eventually becomes fun to do, like solving a puzzle, because of the
instant feedback provided by the computer.}

Metamath\index{Metamath} is probably unlike anything you have
encountered before.  In this first chapter we will look at the philosophy and
use of computers in mathematics in order to better understand the motivation
behind Metamath.  The material in this chapter is not required in order to use
Metamath.  You may skip it if you are impatient, but I hope you will find it
educational and enjoyable.  If you want to start experimenting with the
Metamath program right away, proceed directly to Chapter~\ref{using}
(p.~\pageref{using}).  To
learn the Metamath language, skim Chapter~\ref{using} then proceed to
Chapter~\ref{languagespec} (p.~\pageref{languagespec}).

\section{Mathematics as a Computer Language}

\begin{quote}
  {\em The study of mathematics is apt to commence in
dis\-ap\-point\-ment.\ldots \\
We are told that by its aid the stars are weighted
and the billions of molecules in a drop of water are counted.  Yet, like the
ghost of Hamlet's father, this great science eludes the efforts of our mental
weapons to grasp it.}
  \flushright\sc  Alfred North Whitehead\footnote{\cite{Whitehead}, ch.\ 1.}\\
\end{quote}\index{Whitehead, Alfred North}

\subsection{Is Mathematics ``User-Friendly''?}

Suppose you have no formal training in abstract mathematics.  But popular
books you've read offer tempting glimpses of this world filled with profound
ideas that have stirred the human spirit.  You are not satisfied with the
informal, watered-down descriptions you've read but feel it is important to
grasp the underlying mathematics itself to understand its true meaning. It's
not practical to go back to school to learn it, though; you don't want to
dedicate years of your life to it.  There are many important things in life,
and you have to set priorities for what's important to you.  What would happen
if you tried to pursue it on your own, in your spare time?

After all, you were able to learn a computer programming language such as
Pascal on your own without too much difficulty, even though you had no formal
training in computers.  You don't claim to be an expert in software design,
but you can write a passable program when necessary to suit your needs.  Even
more important, you know that you can look at anyone else's Pascal program, no
matter how complex, and with enough patience figure out exactly how it works,
even though you are not a specialist.  Pascal allows you do anything that a
computer can do, at least in principle.  Thus you know you have the ability,
in principle, to follow anything that a computer program can do:  you just
have to break it down into small enough pieces.

Here's an imaginary scenario of what might happen if you na\-ive\-ly a\-dopted
this same view of abstract mathematics and tried to pick it up on your own, in
a period of time comparable to, say, learning a computer programming
language.

\subsubsection{A Non-Mathematician's Quest for Truth}

\begin{quote}
  {\em \ldots my daughters have been studying (chemistry) for several
se\-mes\-ters, think they have learned differential and integral calculus in
school, and yet even today don't know why $x\cdot y=y\cdot x$ is true.}
  \flushright\sc  Edmund Landau\footnote{\cite{Landau}, p.~vi.}\\
\end{quote}\index{Landau, Edmund}

\begin{quote}
  {\em Minus times minus is plus,\\
The reason for this we need not discuss.}
  \flushright\sc W.\ H.\ Auden\footnote{As quoted in \cite{Guillen}, p.~64.}\\
\end{quote}\index{Auden, W.\ H.}\index{Guillen, Michael}

We'll suppose you are a technically oriented professional, perhaps an engineer, a
computer programmer, or a physicist, but probably not a mathematician.  You
consider yourself reasonably intelligent.  You did well in school, learning a
variety of methods and techniques in practical mathematics such as calculus and
differential equations.  But rarely did your courses get into anything
resembling modern abstract mathematics, and proofs were something that appeared
only occasionally in your textbooks, a kind of necessary evil that was
supposed to convince you of a certain key result.  Most of your
homework consisted of exercises that gave you practice in the techniques, and
you were hardly ever asked to come up with a proof of your own.

You find yourself curious about advanced, abstract mathematics.  You are
driven by an inner conviction that it is important to understand and
appreciate some of the most profound knowledge discovered by mankind.  But it
seems very hard to learn, something that only certain gifted longhairs can
access and understand.  You are frustrated that it seems forever cut off from
you.

Eventually your curiosity drives you to do something about it.
You set for yourself a goal of ``really'' understanding mathematics:  not just
how to manipulate equations in algebra or calculus according to cookbook
rules, but rather to gain a deep understanding of where those rules come from.
In fact, you're not thinking about this kind of ordinary mathematics at all,
but about a much more abstract, ethereal realm of pure mathematics, where
famous results such as G\"{o}del's incompleteness theorem\index{G\"{o}del's
incompleteness theorem} and Cantor's different kinds of infinities
reside.

You have probably read a number of popular books, with titles like {\em
Infinity and the Mind} \cite{Rucker}\index{Rucker, Rudy}, on topics such as
these.  You found them inspiring but at the same time somewhat
unsatisfactory.  They gave you a general idea of what these results are about,
but if someone asked you to prove them, you wouldn't have the faintest idea of
where to begin.   Sure, you could give the same overall outline that you
learned from the popular books; and in a general sort of way, you do have an
understanding.  But deep down inside, you know that there is a rigor that is
missing, that probably there are many subtle steps and pitfalls along the way,
and ultimately it seems you have to place your trust in the experts in the
field.  You don't like this; you want to be able to verify these results for
yourself.

So where do you go next?  As a first step, you decide to look up some of the
original papers on the theorems you are curious about, or better, obtain some
standard textbooks in the field.  You look up a theorem you want to
understand.  Sure enough, it's there, but it's expressed with strange
terms and odd symbols that mean absolutely nothing to you.  It might as well be written in
a foreign language you've never seen before, whose symbols are totally alien.
You look at the proof, and you haven't the foggiest notion what each step
means, much less how one step follows from another.  Well, obviously you have
a lot to learn if you want to understand this stuff.

You feel that you could probably understand it by
going back to college for another three to six years and getting a math
degree.  But that does not fit in with your career and the other things in
your life and would serve no practical purpose.  You decide to seek a quicker
path.  You figure you'll just trace your way back to the beginning, step by
step, as you would do with a computer program, until you understand it.  But
you quickly find that this is not possible, since you can't even understand
enough to know what you have to trace back to.

Maybe a different approach is in order---maybe you should start at the
beginning and work your way up.  First, you read the introduction to the book
to find out what the prerequisites are.  In a similar fashion, you trace your
way back through two or three more books, finally arriving at one that seems
to start at a beginning:  it lists the axioms of arithmetic.  ``Aha!'' you
naively think, ``This must be the starting point, the source of all mathematical
knowledge.'' Or at least the starting point for mathematics dealing with
numbers; you have to start somewhere and have no idea what the starting point
for other mathematics would be.  But the word ``axioms'' looks promising.  So
you eagerly read along and work through some elementary exercises at the
beginning of the book.  You feel vaguely bothered:  these
don't seem like axioms at all, at least not in the sense that you want to
think of axioms.  Axioms imply a starting point from which everything else can
be built up, according to precise rules specified in the axiom system.  Even
though you can understand the first few proofs in an informal way,
and are able to do some of the
exercises, it's hard to pin down precisely what the
rules are.   Sure, each step seems to follow logically from the others, but
exactly what does that mean?  Is the ``logic'' just a matter of common sense,
something vague that we all understand but can never quite state precisely?

You've spent a number of years, off and on, programming computers, and you
know that in the case of computer languages there is no question of what the
rules are---they are precise and crystal clear.  If you follow them, your
program will work, and if you don't, it won't.  No matter how complex a
program, it can always be broken down into simpler and simpler pieces, until
you can ultimately identify the bits that are moved around to perform a
specific function.  Some programs might require a lot of perseverance to
accomplish this, but if you focus on a specific portion of it, you don't even
necessarily have to know how the rest of it works. Shouldn't there be an
analogy in mathematics?

You decide to apply the ultimate test:  you ask yourself how a computer could
verify or ensure that the steps in these proofs follow from one another.
Certainly mathematics must be at least as precisely defined as a computer
language, if not more so; after all, computer science itself is based on it.
If you can get a computer to verify these proofs, then you should also be
able, in principle, to understand them yourself in a very crystal clear,
precise way.

You're in for a surprise:  you can conceive of no way to convert the
proofs, which are in English, to a form that the computer can understand.
The proofs are filled with phrases such as ``assume there exists a unique
$x$\ldots'' and ``given any $y$, let $z$ be the number such that\ldots''  This
isn't the kind of logic you are used to in computer programming, where
everything, even arithmetic, reduces to Boolean ones and zeroes if you care to
break it down sufficiently.  Even though you think you understand the proofs,
there seems to be some kind of higher reasoning involved rather than precise
rules that define how you manipulate the symbols in the axioms.  Whatever it
is, it just isn't obvious how you would express it to a computer, and the more
you think about it, the more puzzled and confused you get, to the point where
you even wonder whether {\em you} really understand it.  There's a lot more to
these axioms of arithmetic than meets the eye.

Nobody ever talked about this in school in your applied math and engineering
courses.  You just learned the rules they gave you, not quite understanding
how or why they worked, sometimes vaguely suspicious or uncertain of them, and
through homework problems and osmosis learned how to present solutions that
satisfied the instructor and earned you an ``A.''  Rarely did you actually
``prove'' anything in a rigorous way, and the math majors who did do stuff
like that seemed to be in a different world.

Of course, there are computer algebra programs that can do mathematics, and
rather impressively.  They can instantly solve the integrals that you
struggled with in freshman calculus, and do much, much more.  But when you
look at these programs, what you see is a big collection of algorithms and
techniques that evolved and were added to over time, along with some basic
software that manipulates symbols.  Each algorithm that is built in is the
result of someone's theorem whose proof is omitted; you just have to trust the
person who proved it and the person who programmed it in and hope there are no
bugs.\index{computer program bugs}  Somehow this doesn't seem to be the
essence of mathematics.  Although computer algebra systems can generate
theorems with amazing speed, they can't actually prove a single one of them.

After some puzzlement, you revisit some popular books on what mathematics is
all about.  Somewhere you read that all of mathematics is actually derived
from something called ``set theory.''  This is a little confusing, because
nowhere in the book that presented the axioms of arithmetic was there any
mention of set theory, or if there was, it seemed to be just a tool that helps
you describe things better---the set of even numbers, that sort of thing.  If
set theory is the basis for all mathematics, then why are additional axioms
needed for arithmetic?

Something is wrong but you're not sure what.  One of your friends is a pure
mathematician.  He knows he is unable to communicate to you what he does for a
living and seems to have little interest in trying.  You do know that for him,
proofs are what mathematics is all about. You ask him what a proof is, and he
essentially tells you that, while of course it's based on logic, really it's
something you learn by doing it over and over until you pick it up.  He refers
you to a book, {\em How to Read and Do Proofs} \cite{Solow}.\index{Solow,
Daniel}  Although this book helps you understand traditional informal proofs,
there is still something missing you can't seem to pin down yet.

You ask your friend how you would go about having a computer verify a proof.
At first he seems puzzled by the question; why would you want to do that?
Then he says it's not something that would make any sense to do, but he's
heard that you'd have to break the proof down into thousands or even millions
of individual steps to do such a thing, because the reasoning involved is at
such a high level of abstraction.  He says that maybe it's something you could
do up to a point, but the computer would be completely impractical once you
get into any meaningful mathematics.  There, the only way you can verify a
proof is by hand, and you can only acquire the ability to do this by
specializing in the field for a couple of years in grad school.  Anyway, he
thinks it all has to do with set theory, although he has never taken a formal
course in set theory but just learned what he needed as he went along.

You are intrigued and amazed.  Apparently a mathematician can grasp as a
single concept something that would take a computer a thousand or a million
steps to verify, and have complete confidence in it.  Each one of these
thousand or million steps must be absolutely correct, or else the whole proof
is meaningless.  If you added a million numbers by hand, would you trust the
result?  How do you really know that all these steps are correct, that there
isn't some subtle pitfall in one of these million steps, like a bug in a
computer program?\index{computer program bugs}  After all, you've read that
famous mathematicians have occasionally made mistakes, and you certainly know
you've made your share on your math homework problems in school.

You recall the analogy with a computer program.  Sure, you can understand what
a large computer program such as a word processor does, as a single high-level
concept or a small set of such concepts, but your ability to understand it in
no way ensures that the program is correct and doesn't have hidden bugs.  Even
if you wrote the program yourself you can't really know this; most large
programs that you've written have had bugs that crop up at some later date, no
matter how careful you tried to be while writing them.

OK, so now it seems the reason you can't figure out how to make a
computer verify proofs is because each step really corresponds to a
million small steps.  Well, you say, a computer can do a million
calculations in a second, so maybe it's still practical to do.  Now the
puzzle becomes how to figure out what the million steps are that each
English-language step corresponds to.  Your mathematician friend hasn't
a clue, but suggests that maybe you would find the answer by studying
set theory.  Actually, your friend thinks you're a little off the wall
for even wondering such a thing.  For him, this is not what mathematics
is all about.

The subject of set theory keeps popping up, so you decide it's
time to look it up.

You decide to start off on a careful footing, so you start reading a couple of
very elementary books on set theory.  A lot of it seems pretty obvious, like
intersections, subsets, and Venn diagrams.  You thumb through one of the
books; nowhere is anything about axioms mentioned. The other book relegates to
an appendix a brief discussion that mentions a set of axioms called
``Zermelo--Fraenkel set theory''\index{Zermelo--Fraenkel set theory} and states
them in English.  You look at them and have no idea what they really mean or
what you can do with them.  The comments in this appendix say that the purpose
of mentioning them is to expose you to the idea, but imply that they are not
necessary for basic understanding and that they are really the subject matter
of advanced treatments where fine points such as a certain paradox (Russell's
paradox\index{Russell's paradox}\footnote{Russell's paradox assumes that there
exists a set $S$ that is a collection of all sets that don't contain
themselves.  Now, either $S$ contains itself or it doesn't.  If it contains
itself, it contradicts its definition.  But if it doesn't contain itself, it
also contradicts its definition.  Russell's paradox is resolved in ZF set
theory by denying that such a set $S$ exists.}) are resolved.  Wait a
minute---shouldn't the axioms be a starting point, not an ending point?  If
there are paradoxes that arise without the axioms, how do you know you won't
stumble across one accidentally when using the informal approach?

And nowhere do these books describe how ``all of mathematics can be
derived from set theory'' which by now you've heard a few times.

You find a more advanced book on set theory.  This one actually lists the
axioms of ZF set theory in plain English on page one.  {\em Now} you think
your quest has ended and you've finally found the source of all mathematical
knowledge; you just have to understand what it means.  Here, in one place, is
the basis for all of mathematics!  You stare at the axioms in awe, puzzle over
them, memorize them, hoping that if you just meditate on them long enough they
will become clear.  Of course, you haven't the slightest idea how the rest of
mathematics is ``derived'' from them; in particular, if these are the axioms
of mathematics, then why do arithmetic, group theory, and so on need their own
axioms?

You start reading this advanced book carefully, pondering the meaning of every
word, because by now you're really determined to get to the bottom of this.
The first thing the book does is explain how the axioms came about, which was
to resolve Russell's paradox.\index{Russell's paradox}  In fact that seems to
be the main purpose of their existence; that they supposedly can be used to
derive all of mathematics seems irrelevant and is not even mentioned.  Well,
you go on.  You hope the book will explain to you clearly, step by step, how
to derive things from the axioms.  After all, this is the starting point of
mathematics, like a book that explains the basics of a computer programming
language.  But something is missing.  You find you can't even understand the
first proof or do the first exercise.  Symbols such as $\exists$ and $\forall$
permeate the page without any mention of where they came from or how to
manipulate them; the author assumes you are totally familiar with them and
doesn't even tell you what they mean.  By now you know that $\exists$ means
``there exists'' and $\forall$ means ``for all,'' but shouldn't the rules for
manipulating these symbols be part of the axioms?  You still have no idea
how you could even describe the axioms to a computer.

Certainly there is something much different here from the technical
literature you're used to reading.  A computer language manual almost
always explains very clearly what all the symbols mean, precisely what
they do, and the rules used for combining them, and you work your way up
from there.

After glancing at four or five other such books, you come to the realization
that there is another whole field of study that you need just to get to the
point at which you can understand the axioms of set theory.  The field is
called ``logic.''  In fact, some of the books did recommend it as a
prerequisite, but it just didn't sink in.  You assumed logic was, well, just
logic, something that a person with common sense intuitively understood.  Why
waste your time reading boring treatises on symbolic logic, the manipulation
of 1's and 0's that computers do, when you already know that?  But this is a
different kind of logic, quite alien to you.  The subject of {\sc nand} and
{\sc nor} gates is not even touched upon or in any case has to do with only a
very small part of this field.

So your quest continues.  Skimming through the first couple of introductory
books, you get a general idea of what logic is about and what quantifiers
(``for all,'' ``there exists'') mean, but you find their examples somewhat
trivial and mildly annoying (``all dogs are animals,'' ``some animals are
dogs,'' and such).  But all you want to know is what the rules are for
manipulating the symbols so you can apply them to set theory.  Some formulas
describing the relationships among quantifiers ($\exists$ and $\forall$) are
listed in tables, along with some verbal reasoning to justify them.
Presumably, if you want to find out if a formula is correct, you go through
this same kind of mental reasoning process, possibly using images of dogs and
animals. Intuitively, the formulas seem to make sense.  But when you ask
yourself, ``What are the rules I need to get a computer to figure out whether
this formula is correct?'', you still don't know.  Certainly you don't ask the
computer to imagine dogs and animals.

You look at some more advanced logic books.  Many of them have an introductory
chapter summarizing set theory, which turns out to be a prerequisite.  You
need logic to understand set theory, but it seems you also need set theory to
understand logic!  These books jump right into proving rather advanced
theorems about logic, without offering the faintest clue about where the logic
came from that allows them to prove these theorems.

Luckily, you come across an elementary book of logic that, halfway through,
after the usual truth tables and metaphors, presents in a clear, precise way
what you've been looking for all along: the axioms!  They're divided into
propositional calculus (also called sentential logic) and predicate calculus
(also called first-order logic),\index{first-order logic} with rules so simple
and crystal clear that now you can finally program a computer to understand
them.  Indeed, they're no harder than learning how to play a game of chess.
As far as what you seem to need is concerned, the whole book could have been
written in five pages!

{\em Now} you think you've found the ultimate source of mathematical
truth.  So---the axioms of mathematics consist of these axioms of logic,
together with the axioms of ZF set theory. (By now you've also been able to
figure out how to translate the ZF axioms from English into the
actual symbols of logic which you can now manipulate according to
precise, easy-to-understand rules.)

Of course, you still don't understand how ``all of mathematics can be
derived from set theory,'' but maybe this will reveal itself in due
course.

You eagerly set out to program the axioms and rules into a computer and start
to look at the theorems you will have to prove as the logic is developed.  All
sorts of important theorems start popping up:  the deduction
theorem,\index{deduction theorem} the substitution theorem,\index{substitution
theorem} the completeness theorem of propositional calculus,\index{first-order
logic!completeness} the completeness theorem of predicate calculus.  Uh-oh,
there seems to be trouble.  They all get harder and harder, and not one of
them can be derived with the axioms and rules of logic you've just been
handed.  Instead, they all require ``metalogic'' for their proofs, a kind of
mixture of logic and set theory that allows you to prove things {\em about}
the axioms and theorems of logic rather than {\em with} them.

You plow ahead anyway.  A month later, you've spent much of your
free time getting the computer to verify proofs in propositional calculus.
You've programmed in the axioms, but you've also had to program in the
deduction theorem, the substitution theorem, and the completeness theorem of
propositional calculus, which by now you've resigned yourself to treating as
rather complex additional axioms, since they can't be proved from the axioms
you were given.  You can now get the computer to verify and even generate
complete, rigorous, formal proofs\index{formal proof}.  Never mind that they
may have 100,000 steps---at least now you can have complete, absolute
confidence in them.  Unfortunately, the only theorems you have proved are
pretty trivial and you can easily verify them in a few minutes with truth
tables, if not by inspection.

It looks like your mathematician friend was right.  Getting the computer to do
serious mathematics with this kind of rigor seems almost hopeless.  Even
worse, it seems that the further along you get, the more ``axioms'' you have
to add, as each new theorem seems to involve additional ``metamathematical''
reasoning that hasn't been formalized, and none of it can be derived from the
axioms of logic.  Not only do the proofs keep growing exponentially as you get
further along, but the program to verify them keeps getting bigger and bigger
as you program in more ``metatheorems.''\index{metatheorem}\footnote{A
metatheorem is usually a statement that is too general to be directly provable
in a theory.  For example, ``if $n_1$, $n_2$, and $n_3$ are integers, then
$n_1+n_2+n_3$ is an integer'' is a theorem of number theory.  But ``for any
integer $k > 1$, if $n_1, \ldots, n_k$ are integers, then $n_1+\ldots +n_k$ is
an integer'' is a metatheorem, in other words a family of theorems, one for
every $k$.  The reason it is not a theorem is that the general sum $n_1+\ldots
+n_k$ (as a function of $k$) is not an operation that can be defined directly
in number theory.} The bugs\index{computer program bugs} that have cropped up
so far have already made you start to lose faith in the rigor you seem to have
achieved, and you know it's just going to get worse as your program gets larger.

\subsection{Mathematics and the Non-Specialist}

\begin{quote}
  {\em A real proof is not checkable by a machine, or even by any mathematician
not privy to the gestalt, the mode of thought of the particular field of
mathematics in which the proof is located.}
  \flushright\sc  Davis and Hersh\index{Davis, Phillip J.}
  \footnote{\cite{Davis}, p.~354.}\\
\end{quote}

The bulk of abstract or theoretical mathematics is ordinarily outside
the reach of anyone but a few specialists in each field who have completed
the necessary difficult internship in order to enter its coterie.  The
typical intelligent layperson has no reasonable hope of understanding much of
it, nor even the specialist mathematician of understanding other fields.  It
is like a foreign language that has no dictionary to look up the translation;
the only way you can learn it is by living in the country for a few years.  It
is argued that the effort involved in learning a specialty is a necessary
process for acquiring a deep understanding.  Of course, this is almost certainly
true if one is to make significant contributions to a field; in particular,
``doing'' proofs is probably the most important part of a mathematician's
training.  But is it also necessary to deny outsiders access to it?  Is it
necessary that abstract mathematics be so hard for a layperson to grasp?

A computer normally is of no help whatsoever.  Most published proofs are
actually just series of hints written in an informal style that requires
considerable knowledge of the field to understand.  These are the ``real
proofs'' referred to by Davis and Hersh.\index{informal proof}  There is an
implicit understanding that, in principle, such a proof could be converted to
a complete formal proof\index{formal proof}.  However, it is said that no one
would ever attempt such a conversion, even if they could, because that would
presumably require millions of steps (Section~\ref{dream}).  Unfortunately the
informal style automatically excludes the understanding of the proof
by anyone who hasn't gone through the necessary apprenticeship. The
best that the intelligent layperson can do is to read popular books about deep
and famous results; while this can be helpful, it can also be misleading, and
the lack of detail usually leaves the reader with no ability whatsoever to
explore any aspect of the field being described.

The statements of theorems often use sophisticated notation that makes them
inaccessible to the non-specialist.  For a non-specialist who wants to achieve
a deeper understanding of a proof, the process of tracing definitions and
lemmas back through their hierarchy\index{hierarchy} quickly becomes confusing
and discouraging.  Textbooks are usually written to train mathematicians or to
communicate to people who are already mathematicians, and large gaps in proofs
are often left as exercises to the reader who is left at an impasse if he or
she becomes stuck.

I believe that eventually computers will enable non-specialists and even
intelligent laypersons to follow almost any mathematical proof in any field.
Metamath is an attempt in that direction.  If all of mathematics were as
easily accessible as a computer programming language, I could envision
computer programmers and hobbyists who otherwise lack mathematical
sophistication exploring and being amazed by the world of theorems and proofs
in obscure specialties, perhaps even coming up with results of their own.  A
tremendous advantage would be that anyone could experiment with conjectures in
any field---the computer would offer instant feedback as to whether
an inference step was correct.

Mathematicians sometimes have to put up with the annoyance of
cranks\index{cranks} who lack a fundamental understanding of mathematics but
insist that their ``proofs'' of, say, Fermat's Last Theorem\index{Fermat's
Last Theorem} be taken seriously.  I think part of the problem is that these
people are misled by informal mathematical language, treating it as if they
were reading ordinary expository English and failing to appreciate the
implicit underlying rigor.  Such cranks are rare in the field of computers,
because computer languages are much more explicit, and ultimately the proof is
in whether a computer program works or not.  With easily accessible
computer-based abstract mathematics, a mathematician could say to a crank,
``don't bother me until you've demonstrated your claim on the computer!''

% 22-May-04 nm
% Attempt to move De Millo quote so it doesn't separate from attribution
% CHANGE THIS NUMBER (AND ELIMINATE IF POSSIBLE) WHEN ABOVE TEXT CHANGES
\vspace{-0.5em}

\subsection{An Impossible Dream?}\label{dream}

\begin{quote}
  {\em Even quite basic theorems would demand almost unbelievably vast
  books to display their proofs.}
    \flushright\sc  Robert E. Edwards\footnote{\cite{Edwards}, p.~68.}\\
\end{quote}\index{Edwards, Robert E.}

\begin{quote}
  {\em Oh, of course no one ever really {\em does} it.  It would take
  forever!  You just show that you could do it, that's sufficient.}
    \flushright\sc  ``The Ideal Mathematician''
    \index{Davis, Phillip J.}\footnote{\cite{Davis},
p.~40.}\\
\end{quote}

\begin{quote}
  {\em There is a theorem in the primitive notation of set theory that
  corresponds to the arithmetic theorem `$1000+2000=3000$'.  The formula
  would be forbiddingly long\ldots even if [one] knows the definitions
  and is asked to simplify the long formula according to them, chances are
  he will make errors and arrive at some incorrect result.}
    \flushright\sc  Hao Wang\footnote{\cite{Wang}, p.~140.}\\
\end{quote}\index{Wang, Hao}

% 22-May-04 nm
% Attempt to move De Millo quote so it doesn't separate from attribution
% CHANGE THIS NUMBER (AND ELIMINATE IF POSSIBLE) WHEN ABOVE TEXT CHANGES
\vspace{-0.5em}

\begin{quote}
  {\em The {\em Principia Mathematica} was the crowning achievement of the
  formalists.  It was also the deathblow of the formalist view.\ldots
  {[Rus\-sell]} failed, in three enormous volumes, to get beyond the elementary
  facts of arithmetic.  He showed what can be done in principle and what
  cannot be done in practice.  If the mathematical process were really
  one of strict, logical progression, we would still be counting our
  fingers.\ldots
  One theoretician estimates\ldots that a demonstration of one of
  Ramanujan's conjectures assuming set theory and elementary analysis would
  take about two thousand pages; the length of a deduction from first principles
  is nearly in\-con\-ceiv\-a\-ble\ldots The probabilists argue that\ldots any
  very long proof can at best be viewed as only probably correct\ldots}
  \flushright\sc Richard de Millo et. al.\footnote{\cite{deMillo}, pp.~269,
  271.}\\
\end{quote}\index{de Millo, Richard}

A number of writers have conveyed the impression that the kind of absolute
rigor provided by Metamath\index{Metamath} is an impossible dream, suggesting
that a complete, formal verification\index{formal proof} of a typical theorem
would take millions of steps in untold volumes of books.  Even if it could be
done, the thinking sometimes goes, all meaning would be lost in such a
monstrous, tedious verification.\index{informal proof}\index{proof length}

These writers assume, however, that in order to achieve the kind of complete
formal verification they desire one must break down a proof into individual
primitive steps that make direct reference to the axioms.  This is
not necessary.  There is no reason not to make use of previously proved
theorems rather than proving them over and over.

Just as important, definitions\index{definition} can be introduced along
the way, allowing very complex formulas to be represented with few
symbols.  Not doing this can lead to absurdly long formulas.  For
example, the mere statement of
G\"{o}del's incompleteness theorem\index{G\"{o}del's
incompleteness theorem}, which can be expressed with a small number of
defined symbols, would require about 20,000 primitive symbols to express
it.\index{Boolos, George S.}\footnote{George S.\ Boolos, lecture at
Massachusetts Institute of Technology, spring 1990.} An extreme example
is Bourbaki's\label{bourbaki} language for set theory, which requires
4,523,659,424,929 symbols plus 1,179,618,517,981 disambiguatory links
(lines connecting symbol pairs, usually drawn below or above the
formula) to express the number
``one'' \cite{Mathias}.\index{Mathias, Adrian R. D.}\index{Bourbaki,
Nicolas}
% http://www.dpmms.cam.ac.uk/~ardm/

A hierarchy\index{hierarchy} of theorems and definitions permits an
exponential growth in the formula sizes and primitive proof steps to be
described with only a linear growth in the number of symbols used.  Of course,
this is how ordinary informal mathematics is normally done anyway, but with
Metamath\index{Metamath} it can be done with absolute rigor and precision.

\subsection{Beauty}


\begin{quote}
  {\em No one shall be able to drive us from the paradise that Cantor has
created for us.}
   \flushright\sc  David Hilbert\footnote{As quoted in \cite{Moore}, p.~131.}\\
\end{quote}\index{Hilbert, David}

\needspace{3\baselineskip}
\begin{quote}
  {\em Mathematics possesses not only truth, but some supreme beauty ---a
  beauty cold and austere, like that of a sculpture.}
    \flushright\sc  Bertrand
    Russell\footnote{\cite{Russell}.}\\
\end{quote}\index{Russell, Bertrand}

\begin{quote}
  {\em Euclid alone has looked on Beauty bare.}
  \flushright\sc Edna Millay\footnote{As quoted in \cite{Davis}, p.~150.}\\
\end{quote}\index{Millay, Edna}

For most people, abstract mathematics is distant, strange, and
incomprehensible.  Many popular books have tried to convey some of the sense
of beauty in famous theorems.  But even an intelligent layperson is left with
only a general idea of what a theorem is about and is hardly given the tools
needed to make use of it.  Traditionally, it is only after years of arduous
study that one can grasp the concepts needed for deep understanding.
Metamath\index{Metamath} allows you to approach the proof of the theorem from
a quite different perspective, peeling apart the formulas and definitions
layer by layer until an entirely different kind of understanding is achieved.
Every step of the proof is there, pieced together with absolute precision and
instantly available for inspection through a microscope with a magnification
as powerful as you desire.

A proof in itself can be considered an object of beauty.  Constructing an
elegant proof is an art.  Once a famous theorem has been proved, often
considerable effort is made to find simpler and more easily understood
proofs.  Creating and communicating elegant proofs is a major concern of
mathematicians.  Metamath is one way of providing a common language for
archiving and preserving this information.

The length of a proof can, to a certain extent, be considered an
objective measure of its ``beauty,'' since shorter proofs are usually
considered more elegant.  In the set theory database
\texttt{set.mm}\index{set theory database (\texttt{set.mm})}%
\index{Metamath Proof Explorer}
provided with Metamath, one goal was to make all proofs as short as possible.

\needspace{4\baselineskip}
\subsection{Simplicity}

\begin{quote}
  {\em God made man simple; man's complex problems are of his own
  devising.}
    \flushright\sc Eccles. 7:29\footnote{Jerusalem Bible.}\\
\end{quote}\index{Bible}

\needspace{3\baselineskip}
\begin{quote}
  {\em God made integers, all else is the work of man.}
    \flushright\sc Leopold Kronecker\footnote{{\em Jahresbericht
	der Deutschen Mathematiker-Vereinigung }, vol. 2, p. 19.}\\
\end{quote}\index{Kronecker, Leopold}

\needspace{3\baselineskip}
\begin{quote}
  {\em For what is clear and easily comprehended attracts; the
  complicated repels.}
    \flushright\sc David Hilbert\footnote{As quoted in \cite{deMillo},
p.~273.}\\
\end{quote}\index{Hilbert, David}

The Metamath\index{Metamath} language is simple and Spartan.  Metamath treats
all mathematical expressions as simple sequences of symbols, devoid of meaning.
The higher-level or ``metamathematical'' notions underlying Metamath are about
as simple as they could possibly be.  Each individual step in a proof involves
a single basic concept, the substitution of an expression for a variable, so
that in principle almost anyone, whether mathematician or not, can
completely understand how it was arrived at.

In one of its most basic applications, Metamath\index{Metamath} can be used to
develop the foundations of mathematics\index{foundations of mathematics} from
the very beginning.  This is done in the set theory database that is provided
with the Metamath package and is the subject matter
of Chapter~\ref{fol}. Any language (a metalanguage\index{metalanguage})
used to describe mathematics (an object language\index{object language}) must
have a mathematical content of its own, but it is desirable to keep this
content down to a bare minimum, namely that needed to make use of the
inference rules specified by the axioms.  With any metalanguage there is a
``chicken and egg'' problem somewhat like circular reasoning:  you must assume
the validity of the mathematics of the metalanguage in order to prove the
validity of the mathematics of the object language.  The mathematical content
of Metamath itself is quite limited.  Like the rules of a game of chess, the
essential concepts are simple enough so that virtually anyone should be able to
understand them (although that in itself will not let you play like
a master).  The symbols that Metamath manipulates do not in themselves
have any intrinsic meaning.  Your interpretation of the axioms that you supply
to Metamath is what gives them meaning.  Metamath is an attempt to strip down
mathematical thought to its bare essence and show you exactly how the symbols
are manipulated.

Philosophers and logicians, with various motivations, have often thought it
important to study ``weak'' fragments of logic\index{weak logic}
\cite{Anderson}\index{Anderson, Alan Ross} \cite{MegillBunder}\index{Megill,
Norman}\index{Bunder, Martin}, other unconventional systems of logic (such as
``modal'' logic\index{modal logic} \cite[ch.\ 27]{Boolos}\index{Boolos, George
S.}), and quantum logic\index{quantum logic} in physics
\cite{Pavicic}\index{Pavi{\v{c}}i{\'{c}}, M.}.  Metamath\index{Metamath}
provides a framework in which such systems can be expressed, with an absolute
precision that makes all underlying metamathematical assumptions rigorous and
crystal clear.

Some schools of philosophical thought, for example
intuitionism\index{intuitionism} and constructivism\index{constructivism},
demand that the notions underlying any mathematical system be as simple and
concrete as possible.  Metamath should meet the requirements of these
philosophies.  Metamath must be taught the symbols, axioms\index{axiom}, and
rules\index{rule} for a specific theory, from the skeptical (such as
intuitionism\index{intuitionism}\footnote{Intuitionism does not accept the law
of excluded middle (``either something is true or it is not true'').  See
\cite[p.~xi]{Tymoczko}\index{Tymoczko, Thomas} for discussion and references
on this topic.  Consider the theorem, ``There exist irrational numbers $a$ and
$b$ such that $a^b$ is rational.''  An intuitionist would reject the following
proof:  If $\sqrt{2}^{\sqrt{2}}$ is rational, we are done.  Otherwise, let
$a=\sqrt{2}^{\sqrt{2}}$ and $b=\sqrt{2}$. Then $a^b=2$, which is rational.})
to the bold (such as the axiom of choice in set theory\footnote{The axiom of
choice\index{Axiom of Choice} asserts that given any collection of pairwise
disjoint nonempty sets, there exists a set that has exactly one element in
common with each set of the collection.  It is used to prove many important
theorems in standard mathematics.  Some philosophers object to it because it
asserts the existence of a set without specifying what the set contains
\cite[p.~154]{Enderton}\index{Enderton, Herbert B.}.  In one foundation for
mathematics due to Quine\index{Quine, Willard Van Orman}, that has not been
otherwise shown to be inconsistent, the axiom of choice turns out to be false
\cite[p.~23]{Curry}\index{Curry, Haskell B.}.  The \texttt{show
trace{\char`\_}back} command of the Metamath program allows you to find out
whether the axiom of choice, or any other axiom, was assumed by a
proof.}\index{\texttt{show trace{\char`\_}back} command}).

The simplicity of the Metamath language lets the algorithm (computer program)
that verifies the validity of a Metamath proof be straightforward and
robust.  You can have confidence that the theorems it verifies really can be
derived from your axioms.

\subsection{Rigor}

\begin{quote}
  {\em Rigor became a goal with the Greeks\ldots But the efforts to
  pursue rigor to the utmost have led to an impasse in which there is
  no longer any agreement on what it really means.  Mathematics remains
  alive and vital, but only on a pragmatic basis.}
    \flushright\sc  Morris Kline\footnote{\cite{Kline}, p.~1209.}\\
\end{quote}\index{Kline, Morris}

Kline refers to a much deeper kind of rigor than that which we will discuss in
this section.  G\"{o}del's incompleteness theorem\index{G\"{o}del's
incompleteness theorem} showed that it is impossible to achieve absolute rigor
in standard mathematics because we can never prove that mathematics is
consistent (free from contradictions).\index{consistent theory}  If
mathematics is consistent, we will never know it, but must rely on faith.  If
mathematics is inconsistent, the best we can hope for is that some clever
future mathematician will discover the inconsistency.  In this case, the
axioms would probably be revised slightly to eliminate the inconsistency, as
was done in the case of Russell's paradox,\index{Russell's paradox} but the
bulk of mathematics would probably not be affected by such a discovery.
Russell's paradox, for example, did not affect most of the remarkable results
achieved by 19th-century and earlier mathematicians.  It mainly invalidated
some of Gottlob Frege's\index{Frege, Gottlob} work on the foundations of
mathematics in the late 1800's; in fact Frege's work inspired Russell's
discovery.  Despite the paradox, Frege's work contains important concepts that
have significantly influenced modern logic.  Kline's {\em Mathematics, The
Loss of Certainty} \cite{Klinel}\index{Kline, Morris} has an interesting
discussion of this topic.

What {\em can} be achieved with absolute certainty\index{certainty} is the
knowledge that if we assume the axioms are consistent and true, then the
results derived from them are true.  Part of the beauty of mathematics is that
it is the one area of human endeavor where absolute certainty can be achieved
in this sense.  A mathematical truth will remain such for eternity.  However,
our actual knowledge of whether a particular statement is a mathematical truth
is only as certain as the correctness of the proof that establishes it.  If
the proof of a statement is questionable or vague, we can't have absolute
confidence in the truth that the statement claims.

Let us look at some traditional ways of expressing proofs.

Except in the field of formal logic\index{formal logic}, almost all
traditional proofs in mathematics are really not proofs at all, but rather
proof outlines or hints as to how to go about constructing the proof.  Many
gaps\index{gaps in proofs} are left for the reader to fill in. There are
several reasons for this.  First, it is usually assumed in mathematical
literature that the person reading the proof is a mathematician familiar with
the specialty being described, and that the missing steps are obvious to such
a reader or at least that the reader is capable of filling them in.  This
attitude is fine for professional mathematicians in the specialty, but
unfortunately it often has the drawback of cutting off the rest of the world,
including mathematicians in other specialties, from understanding the proof.
We discussed one possible resolution to this on p.~\pageref{envision}.
Second, it is often assumed that a complete formal proof\index{formal proof}
would require countless millions of symbols (Section~\ref{dream}). This might
be true if the proof were to be expressed directly in terms of the axioms of
logic and set theory,\index{set theory} but it is usually not true if we allow
ourselves a hierarchy\index{hierarchy} of definitions and theorems to build
upon, using a notation that allows us to introduce new symbols, definitions,
and theorems in a precisely specified way.

Even in formal logic,\index{formal logic} formal proofs\index{formal proof}
that are considered complete still contain hidden or implicit information.
For example, a ``proof'' is usually defined as a sequence of
wffs,\index{well-formed formula (wff)}\footnote{A {\em wff} or well-formed
formula is a mathematical expression (string of symbols) constructed according
to some precise rules.  A formal mathematical system\index{formal system}
contains (1) the rules for constructing syntactically correct
wffs,\index{syntax rules} (2) a list of starting wffs called
axioms,\index{axiom} and (3) one or more rules prescribing how to derive new
wffs, called theorems\index{theorem}, from the axioms or previously derived
theorems.  An example of such a system is contained in
Metamath's\index{Metamath} set theory database, which defines a formal
system\index{formal system} from which all of standard mathematics can be
derived.  Section~\ref{startf} steps you through a complete example of a formal
system, and you may want to skim it now if you are unfamiliar with the
concept.} each of which is an axiom or follows from a rule applied to previous
wffs in the sequence.  The implicit part of the proof is the algorithm by
which a sequence of symbols is verified to be a valid wff, given the
definition of a wff.  The algorithm in this case is rather simple, but for a
computer to verify the proof,\index{automated proof verification} it must have
the algorithm built into its verification program.\footnote{It is possible, of
course, to specify wff construction syntax outside of the program itself
with a suitable input language (the Metamath language being an example), but
some proof-verification or theorem-proving programs lack the ability to extend
wff syntax in such a fashion.} If one deals exclusively with axioms and
elementary wffs, it is straightforward to implement such an algorithm.  But as
more and more definitions are added to the theory in order to make the
expression of wffs more compact, the algorithm becomes more and more
complicated.  A computer program that implements the algorithm becomes larger
and harder to understand as each definition is introduced, and thus more prone
to bugs.\index{computer program bugs}  The larger the program, the
more suspicious the mathematician may be about
the validity of its algorithms.  This is especially true because
computer programs are inherently hard to follow to begin with, and few people
enjoy verifying them manually in detail.

Metamath\index{Metamath} takes a different approach.  Metamath's ``knowledge''
is limited to the ability to substitute variables for expressions, subject to
some simple constraints.  Once the basic algorithm of Metamath is assumed to
be debugged, and perhaps independently confirmed, it
can be trusted once and for all.  The information that Metamath needs to
``understand'' mathematics is contained entirely in the body of knowledge
presented to Metamath.  Any errors in reasoning can only be errors in the
axioms or definitions contained in this body of knowledge.  As a
``constructive'' language\index{constructive language} Metamath has no
conditional branches or loops like the ones that make computer programs hard
to decipher; instead, the language can only build new sequences of symbols
from earlier sequences  of symbols.

The simplicity of the rules that underlie Metamath not only makes Metamath
easy to learn but also gives Metamath a great deal of flexibility. For
example, Metamath is not limited to describing standard first-order
logic\index{first-order logic}; higher-order logics\index{higher-order logic}
and fragments of logic\index{weak logic} can be described just as easily.
Metamath gives you the freedom to define whatever wff notation you prefer; it
has no built-in conception of the syntax of a wff.\index{well-formed formula
(wff)}  With suitable axioms and definitions, Metamath can even describe and
prove things about itself.\index{Metamath!self-description}  (John
Harrison\index{Harrison, John} discusses the ``reflection''
principle\index{reflection principle} involved in self-descriptive systems in
\cite{Harrison}.)

The flexibility of Metamath requires that its proofs specify a lot of detail,
much more than in an ordinary ``formal'' proof.\index{formal proof}  For
example, in an ordinary formal proof, a single step consists of displaying the
wff that constitutes that step.  In order for a computer program to verify
that the step is acceptable, it first must verify that the symbol sequence
being displayed is an acceptable wff.\index{automated proof verification} Most
proof verifiers have at least basic wff syntax built into their programs.
Metamath has no hard-wired knowledge of what constitutes a wff built into it;
instead every wff must be explicitly constructed based on rules defining wffs
that are present in a database.  Thus a single step in an ordinary formal
proof may be correspond to many steps in a Metamath proof. Despite the larger
number of steps, though, this does not mean that a Metamath proof must be
significantly larger than an ordinary formal proof. The reason is that since
we have constructed the wff from scratch, we know what the wff is, so there is
no reason to display it.  We only need to refer to a sequence of statements
that construct it.  In a sense, the display of the wff in an ordinary formal
proof is an implicit proof of its own validity as a wff; Metamath just makes
the proof explicit. (Section~\ref{proof} describes Metamath's proof notation.)

\section{Computers and Mathematicians}

\begin{quote}
  {\em The computer is important, but not to mathematics.}
    \flushright\sc  Paul Halmos\footnote{As quoted in \cite{Albers}, p.~121.}\\
\end{quote}\index{Halmos, Paul}

Pure mathematicians have traditionally been indifferent to computers, even to
the point of disdain.\index{computers and pure mathematics}  Computer science
itself is sometimes considered to fall in the mundane realm of ``applied''
mathematics, perhaps essential for the real world but intellectually unexciting
to those who seek the deepest truths in mathematics.  Perhaps a reason for this
attitude towards computers is that there is little or no computer software that
meets their needs, and there may be a general feeling that such software could
not even exist.  On the one hand, there are the practical computer algebra
systems, which can perform amazing symbolic manipulations in algebra and
calculus,\index{computer algebra system} yet can't prove the simplest
existence theorem, if the idea of a proof is present at all.  On the other
hand, there are specialized automated theorem provers that technically speaking
may generate correct proofs.\index{automated theorem proving}  But sometimes
their specialized input notation may be cryptic and their output perceived to
be long, inelegant, incomprehensible proofs.    The output
may be viewed with suspicion, since the program that generates it tends to be
very large, and its size increases the potential for bugs\index{computer
program bugs}.  Such a proof may be considered trustworthy only if
independently verified and ``understood'' by a human, but no one wants to
waste their time on such a boring, unrewarding chore.



\needspace{4\baselineskip}
\subsection{Trusting the Computer}

\begin{quote}
  {\em \ldots I continue to find the quasi-empirical interpretation of
  computer proofs to be the more plausible.\ldots Since not
  everything that claims to be a computer proof can be
  accepted as valid, what are the mathematical criteria for acceptable
  computer proofs?}
    \flushright\sc  Thomas Tymoczko\footnote{\cite{Tymoczko}, p.~245.}\\
\end{quote}\index{Tymoczko, Thomas}

In some cases, computers have been essential tools for proving famous
theorems.  But if a proof is so long and obscure that it can be verified in a
practical way only with a computer, it is vaguely felt to be suspicious.  For
example, proving the famous four-color theorem\index{four-color
theorem}\index{proof length} (``a map needs no more than four colors to
prevent any two adjacent countries from having the same color'') can presently
only be done with the aid of a very complex computer program which originally
required 1200 hours of computer time. There has been considerable debate about
whether such a proof can be trusted and whether such a proof is ``real''
mathematics \cite{Swart}\index{Swart, E. R.}.\index{trusting computers}

However, under normal circumstances even a skeptical mathematician would have a
great deal of confidence in the result of multiplying two numbers on a pocket
calculator, even though the precise details of what goes on are hidden from its
user.  Even the verification on a supercomputer that a huge number is prime is
trusted, especially if there is independent verification; no one bothers to
debate the philosophical significance of its ``proof,'' even though the actual
proof would be so large that it would be completely impractical to ever write
it down on paper.  It seems that if the algorithm used by the computer is
simple enough to be readily understood, then the computer can be trusted.

Metamath\index{Metamath} adopts this philosophy.  The simplicity of its
language makes it easy to learn, and because of its simplicity one can have
essentially absolute confidence that a proof is correct. All axioms, rules, and
definitions are available for inspection at any time because they are defined
by the user; there are no hidden or built-in rules that may be prone to subtle
bugs\index{computer program bugs}.  The basic algorithm at the heart of
Metamath is simple and fixed, and it can be assumed to be bug-free and robust
with a degree of confidence approaching certainty.
Independently written implementations of the Metamath verifier
can reduce any residual doubt on the part of a skeptic even further;
there are now over a dozen such implementations, written by many people.

\subsection{Trusting the Mathematician}\label{trust}

\begin{quote}
  {\em There is no Algebraist nor Mathematician so expert in his science, as
  to place entire confidence in any truth immediately upon his discovery of it,
  or regard it as any thing, but a mere probability.  Every time he runs over
  his proofs, his confidence encreases; but still more by the approbation of
  his friends; and is rais'd to its utmost perfection by the universal assent
  and applauses of the learned world.}
  \flushright\sc David Hume\footnote{{\em A Treatise of Human Nature}, as
  quoted in \cite{deMillo}, p.~267.}\\
\end{quote}\index{Hume, David}

\begin{quote}
  {\em Stanislaw Ulam estimates that mathematicians publish 200,000 theorems
  every year.  A number of these are subsequently contradicted or otherwise
  disallowed, others are thrown into doubt, and most are ignored.}
  \flushright\sc Richard de Millo et. al.\footnote{\cite{deMillo}, p.~269.}\\
\end{quote}\index{Ulam, Stanislaw}

Whether or not the computer can be trusted, humans  of course will occasionally
err. Only the most memorable proofs get independently verified, and of these
only a handful of truly great ones achieve the status of being ``known''
mathematical truths that are used without giving a second thought to their
correctness.

There are many famous examples of incorrect theorems and proofs in
mathematical literature.\index{errors in proofs}

\begin{itemize}
\item There have been thousands of purported proofs of Fermat's Last
Theorem\index{Fermat's Last Theorem} (``no integer solutions exist to $x^n +
y^n = z^n$ for $n > 2$''), by amateurs, cranks, and well-regarded
mathematicians \cite[p.~5]{Stark}\index{Stark, Harold M}.  Fermat wrote a note
in his copy of Bachet's {\em Diophantus} that he found ``a truly marvelous
proof of this theorem but this margin is too narrow to contain it''
\cite[p.~507]{Kramer}.  A recent, much publicized proof by Yoichi
Miyaoka\index{Miyaoka, Yoichi} was shown to be incorrect ({\em Science News},
April 9, 1988, p.~230).  The theorem was finally proved by Andrew
Wiles\index{Wiles, Andrew} ({\em Science News}, July 3, 1993, p.~5), but it
initially had some gaps and took over a year after its announcement to be
checked thoroughly by experts.  On Oct. 25, 1994, Wiles announced that the last
gap found in his proof had been filled in.
  \item In 1882, M. Pasch discovered that an axiom was omitted from Euclid's
formulation of geometry\index{Euclidean geometry}; without it, the proofs of
important theorems of Euclid are not valid.  Pasch's axiom\index{Pasch's
axiom} states that a line that intersects one side of a triangle must also
intersect another side, provided that it does not touch any of the triangle's
vertices.  The omission of Pasch's axiom went unnoticed for 2000
years \cite[p.~160]{Davis}, in spite of (one presumes) the thousands of
students, instructors, and mathematicians who studied Euclid.
  \item The first published proof of the famous Schr\"{o}der--Bernstein
theorem\index{Schr\"{o}der--Bernstein theorem} in set theory was incorrect
\cite[p.~148]{Enderton}\index{Enderton, Herbert B.}.  This theorem states
that if there exists a 1-to-1 function\footnote{A {\em set}\index{set} is any
collection of objects. A {\em function}\index{function} or {\em
mapping}\index{mapping} is a rule that assigns to each element of one set
(called the function's {\em domain}\index{domain}) an element from another
set.} from set $A$ into set $B$ and vice-versa, then sets $A$ and $B$ have
a 1-to-1 correspondence.  Although it sounds simple and obvious,
the standard proof is quite long and complex.
  \item In the early 1900's, Hilbert\index{Hilbert, David} published a
purported proof of the continuum hypothesis\index{continuum hypothesis}, which
was eventually established as unprovable by Cohen\index{Cohen, Paul} in 1963
\cite[p.~166]{Enderton}.  The continuum hypothesis states that no
infinity\index{infinity} (``transfinite cardinal number'')\index{cardinal,
transfinite} exists whose size (or ``cardinality''\index{cardinality}) is
between the size of the set of integers and the size of the set of real
numbers.  This hypothesis originated with German mathematician Georg
Cantor\index{Cantor, Georg} in the late 1800's, and his inability to prove it
is said to have contributed to mental illness that afflicted him in his later
years.
  \item An incorrect proof of the four-color theorem\index{four-color theorem}
was published by Kempe\index{Kempe, A. B.} in 1879
\cite[p.~582]{Courant}\index{Courant, Richard}; it stood for 11 years before
its flaw was discovered.  This theorem states that any map can be colored
using only four colors, so that no two adjacent countries have the same
color.  In 1976 the theorem was finally proved by the famous computer-assisted
proof of Haken, Appel, and Koch \cite{Swart}\index{Appel, K.}\index{Haken,
W.}\index{Koch, K.}.  Or at least it seems that way.  Mathematician
H.~S.~M.~Coxeter\index{Coxeter, H. S. M.} has doubts \cite[p.~58]{Davis}:  ``I
have a feeling that that is an untidy kind of use of the computers, and the more
you correspond with Haken and Appel, the more shaky you seem to be.''
  \item Many false ``proofs'' of the Poincar\'{e}
conjecture\index{Poincar\'{e} conjecture} have been proposed over the years.
This conjecture states that any object that mathematically behaves like a
three-dimensional sphere is a three-dimensional sphere topologically,
regardless of how it is distorted.  In March 1986, mathematicians Colin
Rourke\index{Rourke, Colin} and Eduardo R\^{e}go\index{R\^{e}go, Eduardo}
caused  a stir in the mathematical community by announcing that they had found
a proof; in November of that year the proof was found to be false \cite[p.
218]{PetersonI}.  It was finally proved in 2003 by Grigory Perelman
\label{poincare}\index{Szpiro, George}\index{Perelman, Grigory}\cite{Szpiro}.
 \end{itemize}

Many counterexamples to ``theorems'' in recent mathematical
literature related to Clifford algebras\index{Clifford algebras}
 have been found by Pertti
Lounesto (who passed away in 2002).\index{Lounesto, Pertti}
See the web page \url{http://mathforum.org/library/view/4933.html}.
% http://users.tkk.fi/~ppuska/mirror/Lounesto/counterexamples.htm

One of the purposes of Metamath\index{Metamath} is to allow proofs to be
expressed with absolute precision.  Developing a proof in the Metamath
language can be challenging, because Metamath will not permit even the
tiniest mistake.\index{errors in proofs}  But once the proof is created, its
correctness can be trusted immediately, without having to depend on the
process of peer review for confirmation.

\section{The Use of Computers in Mathematics}

\subsection{Computer Algebra Systems}

For the most part, you will find that Metamath\index{Metamath} is not a
practical tool for manipulating numbers.  (Even proving that $2 + 2 = 4$, if
you start with set theory, can be quite complex!)  Several commercial
mathematics packages are quite good at arithmetic, algebra, and calculus, and
as practical tools they are invaluable.\index{computer algebra system} But
they have no notion of proof, and cannot understand statements starting with
``there exists such and such...''.

Software packages such as Mathematica \cite{Wolfram}\index{Mathematica} do not
concern themselves with proofs but instead work directly with known results.
These packages primarily emphasize heuristic rules such as the substitution of
equals for equals to achieve simpler expressions or expressions in a different
form.  Starting with a rich collection of built-in rules and algorithms, users
can add to the collection by means of a powerful programming language.
However, results such as, say, the existence of a certain abstract object
without displaying the actual object cannot be expressed (directly) in their
languages.  The idea of a proof from a small set of axioms is absent.  Instead
this software simply assumes that each fact or rule you add to the built-in
collection of algorithms is valid.  One way to view the software is as a large
collection of axioms from which the software, with certain goals, attempts to
derive new theorems, for example equating a complex expression with a simpler
equivalent. But the terms ``theorem''\index{theorem} and
``proof,''\index{proof} for example, are not even mentioned in the index of
the user's manual for Mathematica.\index{Mathematica and proofs}  What is also
unsatisfactory from a philosophical point of view is that there is no way to
ensure the validity of the results other than by trusting the writer of each
application module or tediously checking each module by hand, similar to
checking a computer program for bugs.\index{computer program
bugs}\footnote{Two examples illustrate why the knowledge database of computer
algebra systems should sometimes be regarded with a certain caution.  If you
ask Mathematica (version 3.0) to \texttt{Solve[x\^{ }n + y\^{ }n == z\^{ }n , n]}
it will respond with \texttt{\{\{n-\char`\>-2\}, \{n-\char`\>-1\},
\{n-\char`\>1\}, \{n-\char`\>2\}\}}. In other words, Mathematica seems to
``know'' that Fermat's Last Theorem\index{Fermat's Last Theorem} is true!  (At
the time this version of Mathematica was released this fact was unknown.)  If
you ask Maple\index{Maple} to \texttt{solve(x\^{ }2 = 2\^{ }x)} then
\texttt{simplify(\{"\})}, it returns the solution set \texttt{\{2, 4\}}, apparently
unaware that 0.7666647\ldots is also a solution.} While of course extremely
valuable in applied mathematics, computer algebra systems tend to be of little
interest to the theoretical mathematician except as aids for exploring certain
specific problems.

Because of possible bugs, trusting the output of a computer algebra system for
use as theorems in a proof-verifier would defeat the latter's goal of rigor.
On the other hand, a fact such that a certain relatively large number is
prime, while easy for a computer algebra system to derive, might have a long,
tedious proof that could overwhelm a proof-verifier. One approach for linking
computer algebra systems to a proof-verifier while retaining the advantages of
both is to add a hypothesis to each such theorem indicating its source.  For
example, a constant {\sc maple} could indicate the theorem came from the Maple
package, and would mean ``assuming Maple is consistent, then\ldots''  This and
many other topics concerning the formalization of mathematics are discussed in
John Harrison's\index{Harrison, John} very interesting
PhD thesis~\cite{Harrison-thesis}.

\subsection{Automated Theorem Provers}\label{theoremprovers}

A mathematical theory is ``decidable''\index{decidable theory} if a mechanical
method or algorithm exists that is guaranteed to determine whether or not a
particular formula is a theorem.  Among the few theories that are decidable is
elementary geometry,\index{Euclidean geometry} as was shown by a classic
result of logician Alfred Tarski\index{Tarski, Alfred} in 1948
\cite{Tarski}.\footnote{Tarski's result actually applies to a subset of the
geometry discussed in elementary textbooks.  This subset includes most of what
would be considered elementary geometry but it is not powerful enough to
express, among other things, the notions of the circumference and area of a
circle.  Extending the theory in a way that includes notions such as these
makes the theory undecidable, as was also shown by Tarski.  Tarski's algorithm
is far too inefficient to implement practically on a computer.  A practical
algorithm for proving a smaller subset of geometry theorems (those not
involving concepts of ``order'' or ``continuity'') was discovered by Chinese
mathematician Wu Wen-ts\"{u}n in 1977 \cite{Chou}\index{Chou,
Shang-Ching}.}\index{Wen-ts{\"{u}}n, Wu}  But most theories, including
elementary arithmetic, are undecidable.  This fact contributes to keeping
mathematics alive and well, since many mathematicians believe
that they will never be
replaced by computers (if they believe Roger Penrose's argument that a
computer can never replace the brain \cite{Penrose}\index{Penrose, Roger}).
In fact,  elementary geometry is often considered a ``dead'' field
for the simple reason that it is decidable.

On the other hand, the undecidability of a theory does not mean that one cannot
use a computer to search for proofs, providing one is willing to give up if a
proof is not found after a reasonable amount of time.  The field of automated
theorem proving\index{automated theorem proving} specializes in pursuing such
computer searches.  Among the more successful results to date are those based
on an algorithm known as Robinson's resolution principle
\cite{Robinson}\index{Robinson's resolution principle}.

Automated theorem provers can be excellent tools for those willing to learn
how to use them.  But they are not widely used in mainstream pure
mathematics, even though they could probably be useful in many areas.  There
are several reasons for this.  Probably most important, the main goal in pure
mathematics is to arrive at results that are considered to be deep or
important; proving them is essential but secondary.  Usually, an automated
theorem prover cannot assist in this main goal, and by the time the main goal
is achieved, the mathematician may have already figured out the proof as a
by-product.  There is also a notational problem.  Mathematicians are used to
using very compact syntax where one or two symbols (heavily dependent on
context) can represent very complex concepts; this is part of the
hierarchy\index{hierarchy} they have built up to tackle difficult problems.  A
theorem prover on the other hand might require that a theorem be expressed in
``first-order logic,''\index{first-order logic} which is the logic on which
most of mathematics is ultimately based but which is not ordinarily used
directly because expressions can become very long.  Some automated theorem
provers are experimental programs, limited in their use to very specialized
areas, and the goal of many is simply research into the nature of automated
theorem proving itself.  Finally, much research remains to be done to enable
them to prove very deep theorems.  One significant result was a
computer proof by Larry Wos\index{Wos, Larry} and colleagues that every Robbins
algebra\index{Robbins algebra} is a Boolean  algebra\index{Boolean algebra}
({\em New York Times}, Dec. 10, 1996).\footnote{In 1933, E.~V.\
Huntington\index{Huntington, E. V.}
presented the following axiom system for
Boolean algebra with a unary operation $n$ and a binary operation $+$:
\begin{center}
    $x + y = y + x$ \\
    $(x + y) + z = x + (y + z)$ \\
    $n(n(x) + y) + n(n(x) + n(y)) = x$
\end{center}
Herbert Robbins\index{Robbins, Herbert}, a student of Huntington, conjectured
that the last equation can be replaced with a simpler one:
\begin{center}
    $n(n(x + y) + n(x + n(y))) = x$
\end{center}
Robbins and Huntington could not find a proof.  The problem was
later studied unsuccessfully by Tarski\index{Tarski, Alfred} and his
students, and it remained an unsolved problem until a
computer found the proof in 1996.  For more information on
the Robbins algebra problem see \cite{Wos}.}

How does Metamath\index{Metamath} relate to automated theorem provers?  A
theorem prover is primarily concerned with one theorem at a time (perhaps
tapping into a small database of known theorems) whereas Metamath is more like
a theorem archiving system, storing both the theorem and its proof in a
database for access and verification.  Metamath is one answer to ``what do you
do with the output of a theorem prover?''  and could be viewed as the
next step in the process.  Automated theorem provers could be useful tools for
helping develop its database.
Note that very long, automatically
generated proofs can make your database fat and ugly and cause Metamath's proof
verification to take a long time to run.  Unless you have a particularly good
program that generates very concise proofs, it might be best to consider the
use of automatically generated proofs as a quick-and-dirty approach, to be
manually rewritten at some later date.

The program {\sc otter}\index{otter@{\sc otter}}\footnote{\url{http://www.cs.unm.edu/\~mccune/otter/}.}, later succeeded by
prover9\index{prover9}\footnote{\url{https://www.cs.unm.edu/~mccune/mace4/}.},
have been historically influential.
The E prover\index{E prover}\footnote{\url{https://github.com/eprover/eprover}.}
is a maintained automated theorem prover
for full first-order logic with equality.
There are many other automated theorem provers as well.

If you want to combine automated theorem provers with Metamath
consider investigating
the book {\em Automated Reasoning:  Introduction and Applications}
\cite{Wos}\index{Wos, Larry}.  This book discusses
how to use {\sc otter} in a way that can
not only able to generate
relatively efficient proofs, it can even be instructed to search for
shorter proofs.  The effective use of {\sc otter} (and similar tools)
does require a certain
amount of experience, skill, and patience.  The axiom system used in the
\texttt{set.mm}\index{set theory database (\texttt{set.mm})} set theory
database can be expressed to {\sc otter} using a method described in
\cite{Megill}.\index{Megill, Norman}\footnote{To use those axioms with
{\sc otter}, they must be restated in a way that eliminates the need for
``dummy variables.''\index{dummy variable!eliminating} See the Comment
on p.~\pageref{nodd}.} When successful, this method tends to generate
short and clever proofs, but my experiments with it indicate that the
method will find proofs within a reasonable time only for relatively
easy theorems.  It is still fun to experiment with.

Reference \cite{Bledsoe}\index{Bledsoe, W. W.} surveys a number of approaches
people have explored in the field of automated theorem proving\index{automated
theorem proving}.

\subsection{Interactive Theorem Provers}\label{interactivetheoremprovers}

Finding proofs completely automatically is difficult, so there
are some interactive theorem provers that allow a human to guide the
computer to find a proof.
Examples include
HOL Light\index{HOL light}%
\footnote{\url{https://www.cl.cam.ac.uk/~jrh13/hol-light/}.},
Isabelle\index{Isabelle}%
\footnote{\url{http://www.cl.cam.ac.uk/Research/HVG/Isabelle}.},
{\sc hol}\index{hol@{\sc hol}}%
\footnote{\url{https://hol-theorem-prover.org/}.},
and
Coq\index{Coq}\footnote{\url{https://coq.inria.fr/}.}.

A major difference between most of these tools and Metamath is that the
``proofs'' are actually programs that guide the program to find a proof,
and not the proof itself.
For example, an Isabelle/HOL proof might apply a step
\texttt{apply (blast dest: rearrange reduction)}. The \texttt{blast}
instruction applies
an automatic tableux prover and returns if it found a sequence of proof
steps that work... but the sequence is not considered part of the proof.

A good overview of
higher-level proof verification languages (such as {\sc lcf}\index{lcf@{\sc
lcf}} and {\sc hol}\index{hol@{\sc hol}})
is given in \cite{Harrison}.  All of these languages are fundamentally
different from Metamath in that much of the mathematical foundational
knowledge is embedded in the underlying proof-verification program, rather
than placed directly in the database that is being verified.
These can have a steep learning curve for those without a mathematical
background.  For example, one usually must have a fair understanding of
mathematical logic in order to follow their proofs.

\subsection{Proof Verifiers}\label{proofverifiers}

A proof verifier is a program that doesn't generate proofs but instead
verifies proofs that you give it.  Many proof verifiers have limited built-in
automated proof capabilities, such as figuring out simple logical inferences
(while still being guided by a person who provides the overall proof).  Metamath
has no built-in automated proof capability other than the limited
capability of its Proof Assistant.

Proof-verification languages are not used as frequently as they might be.
Pure mathematicians are more concerned with producing new results, and such
detail and rigor would interfere with that goal.  The use of computers in pure
mathematics is primarily focused on automated theorem provers (not verifiers),
again with the ultimate goal of aiding the creation of new mathematics.
Automated theorem provers are usually concerned with attacking one theorem at
time rather than making a large, organized database easily available to the
user.  Metamath is one way to help close this gap.

By itself Metamath is a mostly a proof verifier.
This does not mean that other approaches can't be used; the difference
is that in Metamath, the results of various provers must be recorded
step-by-step so that they can be verified.

Another proof-verification language is Mizar,\index{Mizar} which can display
its proofs in the informal language that mathematicians are accustomed to.
Information on the Mizar language is available at \url{http://mizar.org}.

For the working mathematician, Mizar is an excellent tool for rigorously
documenting proofs. Mizar typesets its proofs in the informal English used by
mathematicians (and, while fine for them, are just as inscrutable by
laypersons!). A price paid for Mizar is a relatively steep learning curve of a
couple of weeks.  Several mathematicians are actively formalizing different
areas of mathematics using Mizar and publishing the proofs in a dedicated
journal. Unfortunately the task of formalizing mathematics is still looked
down upon to a certain extent since it doesn't involve the creation of ``new''
mathematics.

The closest system to Metamath is
the {\em Ghilbert}\index{Ghilbert} proof language (\url{http://ghilbert.org})
system developed by
Raph Levien\index{Levien, Raph}.
Ghilbert is a formal proof checker heavily inspired by Metamath.
Ghilbert statements are s-expressions (a la Lisp), which is easy
for computers to parse but many people find them hard to read.
There are a number of differences in their specific constructs, but
there is at least one tool to translate some Metamath materials into Ghilbert.
As of 2019 the Ghilbert community is smaller and less active than the
Metamath community.
That said, the Metamath and Ghilbert communities overlap, and fruitful
conversations between them have occurred many times over the years.

\subsection{Creating a Database of Formalized Mathematics}\label{mathdatabase}

Besides Metamath, there are several other ongoing projects with the goal of
formalizing mathematics into computer-verifiable databases.
Understanding some history will help.

The {\sc qed}\index{qed project@{\sc qed} project}%
\footnote{\url{http://www-unix.mcs.anl.gov/qed}.}
project arose in 1993 and its goals were outlined in the
{\sc qed} manifesto.
The {\sc qed} manifesto was
a proposal for a computer-based database of all mathematical knowledge,
strictly formalized and with all proofs having been checked automatically.
The project had a conference in 1994 and another in 1995;
there was also a ``twenty years of the {\sc qed} manifesto'' workshop
in 2014.
Its ideals are regularly reraised.

In a 2007 paper, Freek Wiedijk identified two reasons
for the failure of the {\sc qed} project as originally envisioned:%
\cite{Wiedijk-revisited}\index{Wiedijk, Freek}

\begin{itemize}
\item Very few people are working on formalization of mathematics. There is no compelling application for fully mechanized mathematics.
\item Formalized mathematics does not yet resemble traditional mathematics. This is partly due to the complexity of mathematical notation, and partly to the limitations of existing theorem provers and proof assistants.
\end{itemize}

But this did not end the dream of
formalizing mathematics into computer-verifiable databases.
The problems that led to the {\sc qed} manifesto are still with us,
even though the challenges were harder than originally considered.
What has happened instead is that various independent projects have
worked towards formalizing mathematics into computer-verifiable databases,
each simultaneously competing and cooperating with each other.

A concrete way to see this is
Freek Wiedijk's ``Formalizing 100 Theorems'' list%
\footnote{\url{http://www.cs.ru.nl/\%7Efreek/100/}.}
which shows the progress different systems have made on a challenge list
of 100 mathematical theorems.%
\footnote{ This is not the only list of ``interesting'' theorems.
Another interesting list was posted by Oliver Knill's list
\cite{Knill}\index{Knill, Oliver}.}
The top systems as of February 2019
(in order of the number of challenges completed) are
HOL Light, Isabelle, Metamath, Coq, and Mizar.

The Metamath 100%
\footnote{\url{http://us.metamath.org/mm\_100.html}}
page (maintained by David A. Wheeler\index{Wheeler, David A.})
shows the progress of Metamath (specifically its \texttt{set.mm} database)
against this challenge list maintained by Freek Wiedijk.
The Metamath \texttt{set.mm} database
has made a lot of progress over the years,
in part because working to prove those challenge theorems required
defining various terms and proving their properties as a prerequisite.
Here are just a few of the many statements that have been
formally proven with Metamath:

% The entries of this cause the narrow display to break poorly,
% since the short amount of text means LaTeX doesn't get a lot to work with
% and the itemize format gives it even *less* margin than usual.
% No one will mind if we make just this list flushleft, since this list
% will be internally consistent.
\begin{flushleft}
\begin{itemize}
\item 1. The Irrationality of the Square Root of 2
  (\texttt{sqr2irr}, by Norman Megill, 2001-08-20)
\item 2. The Fundamental Theorem of Algebra
  (\texttt{fta}, by Mario Carneiro, 2014-09-15)
\item 22. The Non-Denumerability of the Continuum
  (\texttt{ruc}, by Norman Megill, 2004-08-13)
\item 54. The Konigsberg Bridge Problem
  (\texttt{konigsberg}, by Mario Carneiro, 2015-04-16)
\item 83. The Friendship Theorem
  (\texttt{friendship}, by Alexander W. van der Vekens, 2018-10-09)
\end{itemize}
\end{flushleft}

We thank all of those who have developed at least one of the Metamath 100
proofs, and we particularly thank
Mario Carneiro\index{Carneiro, Mario}
who has contributed the most Metamath 100 proofs as of 2019.
The Metamath 100 page shows the list of all people who have contributed a
proof, and links to graphs and charts showing progress over time.
We encourage others to work on proving theorems not yet proven in Metamath,
since doing so improves the work as a whole.

Each of the math formalization systems (including Metamath)
has different strengths and weaknesses, depending on what you value.
Key aspects that differentiate Metamath from the other top systems are:

\begin{itemize}
\item Metamath is not tied to any particular set of axioms.
\item Metamath can show every step of every proof, no exceptions.
  Most other provers only assert that a proof can be found, and do not
  show every step. This also makes verification fast, because
  the system does not need to rediscover proof details.
\item The Metamath verifier has been re-implemented in many different
  programming languages, so verification can be done by multiple
  implementations.  In particular, the
  \texttt{set.mm}\index{set theory database (\texttt{set.mm})}%
  \index{Metamath Proof Explorer} database is verified by
  four different verifiers
  written in four different languages by four different authors.
  This greatly reduces the risk of accepting an invalid
  proof due to an error in the verifier.
\item Proofs stay proven.  In some systems, changes to the system's
  syntax or how a tactic works causes proofs to fail in later versions,
  causing older work to become essentially lost.
  Metamath's language is
  extremely small and fixed, so once a proof is added to a database,
  the database can be rechecked with later versions of the Metamath program
  and with other verifiers of Metamath databases.
  If an axiom or key definition needs to be changed, it is easy to
  manipulate the database as a whole to handle the change
  without touching the underlying verifier.
  Since re-verification of an entire database takes seconds, there
  is never a reason to delay complete verification.
  This aspect is especially compelling if your
  goal is to have a long-term database of proofs.
\item Licensing is generous.  The main Metamath databases are released to
  the public domain, and the main Metamath program is open source software
  under a standard, widely-used license.
\item Substitutions are easy to understand, even by those who are not
  professional mathematicians.
\end{itemize}

Of course, other systems may have advantages over Metamath
that are more compelling, depending on what you value.
In any case, we hope this helps you understand Metamath
within a wider context.

\subsection{In Summary}\label{computers-summary}

To summarize our discussions of computers and mathematics, computer algebra
systems can be viewed as theorem generators focusing on a narrow realm of
mathematics (numbers and their properties), automated theorem provers as proof
generators for specific theorems in a much broader realm covered by a built-in
formal system such as first-order logic, interactive theorem
provers require human guidance, proof verifiers verify proofs but
historically they have been
restricted to first-order logic.
Metamath, in contrast,
is a proof verifier and documenter whose realm is essentially unlimited.

\section{Mathematics and Metamath}

\subsection{Standard Mathematics}

There are a number of ways that Metamath\index{Metamath} can be used with
standard mathematics.  The most satisfying way philosophically is to start at
the very beginning, and develop the desired mathematics from the axioms of
logic and set theory.\index{set theory}  This is the approach taken in the
\texttt{set.mm}\index{set theory database (\texttt{set.mm})}%
\index{Metamath Proof Explorer}
database (also known as the Metamath Proof Explorer).
Among other things, this database builds up to the
axioms of real and complex numbers\index{analysis}\index{real and complex
numbers} (see Section~\ref{real}), and a standard development of analysis, for
example, could start at that point, using it as a basis.   Besides this
philosophical advantage, there are practical advantages to having all of the
tools of set theory available in the supporting infrastructure.

On the other hand, you may wish to start with the standard axioms of a
mathematical theory without going through the set theoretical proofs of those
axioms.  You will need mathematical logic to make inferences, but if you wish
you can simply introduce theorems\index{theorem} of logic as
``axioms''\index{axiom} wherever you need them, with the implicit assumption
that in principle they can be proved, if they are obvious to you.  If you
choose this approach, you will probably want to review the notation used in
\texttt{set.mm}\index{set theory database (\texttt{set.mm})} so that your own
notation will be consistent with it.

\subsection{Other Formal Systems}
\index{formal system}

Unlike some programs, Metamath\index{Metamath} is not limited to any specific
area of mathematics, nor committed to any particular mathematical philosophy
such as classical logic versus intuitionism, nor limited, say, to expressions
in first-order logic.  Although the database \texttt{set.mm}
describes standard logic and set theory, Meta\-math
is actually a general-purpose language for describing a wide variety of formal
systems.\index{formal system}  Non-standard systems such as modal
logic,\index{modal logic} intuitionist logic\index{intuitionism}, higher-order
logic\index{higher-order logic}, quantum logic\index{quantum logic}, and
category theory\index{category theory} can all be described with the Metamath
language.  You define the symbols you prefer and tell Metamath the axioms and
rules you want to start from, and Metamath will verify any inferences you make
from those axioms and rules.  A simple example of a non-standard formal system
is Hofstadter's\index{Hofstadter, Douglas R.} MIU system,\index{MIU-system}
whose Metamath description is presented in Appendix~\ref{MIU}.

This is not hypothetical.
The largest Metamath database is
\texttt{set.mm}\index{set theory database (\texttt{set.mm}}%
\index{Metamath Proof Explorer}), aka the Metamath Proof Explorer,
which uses the most common axioms for mathematical foundations
(specifically classical logic combined with Zermelo--Fraenkel
set theory\index{Zermelo--Fraenkel set theory} with the Axiom of Choice).
But other Metamath databases are available:

\begin{itemize}
\item The database
  \texttt{iset.mm}\index{intuitionistic logic database (\texttt{iset.mm})},
  aka the
  Intuitionistic Logic Explorer\index{Intuitionistic Logic Explorer},
  uses intuitionistic logic (a constructivist point of view)
  instead of classical logic.
\item The database
  \texttt{nf.mm}\index{New Foundations database (\texttt{nf.mm})},
  aka the
  New Foundations Explorer\index{New Foundations Explorer},
  constructs mathematics from scratch,
  starting from Quine's New Foundations (NF) set theory axioms.
\item The database
  \texttt{hol.mm}\index{Higher-order Logic database (\texttt{hol.mm})},
  aka the
  Higher-Order Logic (HOL) Explorer\index{Higher-Order Logic (HOL) Explorer},
  starts with HOL (also called simple type theory) and derives
  equivalents to ZFC axioms, connecting the two approaches.
\end{itemize}

Since the days of David Hilbert,\index{Hilbert, David} mathematicians have
been concerned with the fact that the metalanguage\index{metalanguage} used to
describe mathematics may be stronger than the mathematics being described.
Metamath\index{Metamath}'s underlying finitary\index{finitary proof},
constructive nature provides a good philosophical basis for studying even the
weakest logics.\index{weak logic}

The usual treatment of many non-standard formal systems\index{formal
system} uses model theory\index{model theory} or proof theory\index{proof
theory} to describe these systems; these theories, in turn, are based on
standard set theory.  In other words, a non-standard formal system is defined
as a set with certain properties, and standard set theory is used to derive
additional properties of this set.  The standard set theory database provided
with Metamath can be used for this purpose, and when used this way
the development of a special
axiom system for the non-standard formal system becomes unnecessary.  The
model- or proof-theoretic approach often allows you to prove much deeper
results with less effort.

Metamath supports both approaches.  You can define the non-standard
formal system directly, or define the non-standard formal system as
a set with certain properties, whichever you find most helpful.

%\section{Additional Remarks}

\subsection{Metamath and Its Philosophy}

Closely related to Metamath\index{Metamath} is a philosophy or way of looking
at mathematics. This philosophy is related to the formalist
philosophy\index{formalism} of Hilbert\index{Hilbert, David} and his followers
\cite[pp.~1203--1208]{Kline}\index{Kline, Morris}
\cite[p.~6]{Behnke}\index{Behnke, H.}. In this philosophy, mathematics is
viewed as nothing more than a set of rules that manipulate symbols, together
with the consequences of those rules.  While the mathematics being described
may be complex, the rules used to describe it (the
``metamathematics''\index{metamathematics}) should be as simple as possible.
In particular, proofs should be restricted to dealing with concrete objects
(the symbols we write on paper rather than the abstract concepts they
represent) in a constructive manner; these are called ``finitary''
proofs\index{finitary proof} \cite[pp.~2--3]{Shoenfield}\index{Shoenfield,
Joseph R.}.

Whether or not you find Metamath interesting or useful will in part depend on
the appeal you find in its philosophy, and this appeal will probably depend on
your particular goals with respect to mathematics.  For example, if you are a
pure mathematician at the forefront of discovering new mathematical knowledge,
you will probably find that the rigid formality of Metamath stifles your
creativity.  On the other hand, we would argue that once this knowledge is
discovered, there are advantages to documenting it in a standard format that
will make it accessible to others.  Sixty years from now, your field may be
dormant, and as Davis and Hersh put it, your ``writings would become less
translatable than those of the Maya'' \cite[p.~37]{Davis}\index{Davis, Phillip
J.}.


\subsection{A History of the Approach Behind Metamath}

Probably the one work that has had the most motivating influence on
Metamath\index{Metamath} is Whitehead and Russell's monumental {\em Principia
Mathematica} \cite{PM}\index{Whitehead, Alfred North}\index{Russell,
Bertrand}\index{principia mathematica@{\em Principia Mathematica}}, whose aim
was to deduce all of mathematics from a small number of primitive ideas, in a
very explicit way that in principle anyone could understand and follow.  While
this work was tremendously influential in its time, from a modern perspective
it suffers from several drawbacks.  Both its notation and its underlying
axioms are now considered dated and are no longer used.  From our point of
view, its development is not really as accessible as we would like to see; for
practical reasons, proofs become more and more sketchy as its mathematics
progresses, and working them out in fine detail requires a degree of
mathematical skill and patience that many people don't have.  There are
numerous small errors, which is understandable given the tedious, technical
nature of the proofs and the lack of a computer to verify the details.
However, even today {\em Principia Mathematica} stands out as the work closest
in spirit to Metamath.  It remains a mind-boggling work, and one can't help
but be amazed at seeing ``$1+1=2$'' finally appear on page 83 of Volume II
(Theorem *110.643).

The origin of the proof notation used by Metamath dates back to the 1950's,
when the logician C.~A.~Meredith expressed his proofs in a compact notation
called ``condensed detachment''\index{condensed detachment}
\cite{Hindley}\index{Hindley, J. Roger} \cite{Kalman}\index{Kalman, J. A.}
\cite{Meredith}\index{Meredith, C. A.} \cite{Peterson}\index{Peterson, Jeremy
George}.  This notation allows proofs to be communicated unambiguously by
merely referencing the axiom\index{axiom}, rule\index{rule}, or
theorem\index{theorem} used at each step, without explicitly indicating the
substitutions\index{substitution!variable}\index{variable substitution} that
have to be made to the variables in that axiom, rule, or theorem.  Ordinarily,
condensed detachment is more or less limited to propositional
calculus\index{propositional calculus}.  The concept has been extended to
first-order logic\index{first-order logic} in \cite{Megill}\index{Megill,
Norman}, making it is easy to write a small computer program to verify proofs
of simple first-order logic theorems.\index{condensed detachment!and
first-order logic}

A key concept behind the notation of condensed detachment is called
``unification,''\index{unification} which is an algorithm for determining what
substitutions\index{substitution!variable}\index{variable substitution} to
variables have to be made to make two expressions match each other.
Unification was first precisely defined by the logician J.~A.~Robinson, who
used it in the development of a powerful
theorem-proving technique called the ``resolution principle''
\cite{Robinson}\index{Robinson's resolution principle}. Metamath does not make
use of the resolution principle, which is intended for systems of first-order
logic.\index{first-order logic}  Metamath's use is not restricted to
first-order logic, and as we have mentioned it does not automatically discover
proofs.  However, unification is a key idea behind Metamath's proof
notation, and Metamath makes use of a very simple version of it
(Section~\ref{unify}).

\subsection{Metamath and First-Order Logic}

First-order logic\index{first-order logic} is the supporting structure
for standard mathematics.  On top of it is set theory, which contains
the axioms from which virtually all of mathematics can be derived---a
remarkable fact.\index{category
theory}\index{cardinal, inaccessible}\label{categoryth}\footnote{An exception seems
to be category theory.  There are several schools of thought on whether
category theory is derivable from set theory.  At a minimum, it appears
that an additional axiom is needed that asserts the existence of an
``inaccessible cardinal'' (a type of infinity so large that standard set
theory can't prove or deny that it exists).
%
%%%% (I took this out that was in previous editions:)
% But it is also argued that not just one but a ``proper class'' of them
% is needed, and the existence of proper classes is impossible in standard
% set theory.  (A proper class is a collection of sets so huge that no set
% can contain it as an element.  Proper classes can lead to
% inconsistencies such as ``Russell's paradox.''  The axioms of standard
% set theory are devised so as to deny the existence of proper classes.)
%
For more information, see
\cite[pp.~328--331]{Herrlich}\index{Herrlich, Horst} and
\cite{Blass}\index{Blass, Andrea}.}

One of the things that makes Metamath\index{Metamath} more practical for
first-order theories is a set of axioms for first-order logic designed
specifically with Metamath's approach in mind.  These are included in
the database \texttt{set.mm}\index{set theory database (\texttt{set.mm})}.
See Chapter~\ref{fol} for a detailed
description; the axioms are shown in Section~\ref{metaaxioms}.  While
logically equivalent to standard axiom systems, our axiom system breaks
up the standard axioms into smaller pieces such that from them, you can
directly derive what in other systems can only be derived as higher-level
``metatheorems.''\index{metatheorem}  In other words, it is more powerful than
the standard axioms from a metalogical point of view.  A rigorous
justification for this system and its ``metalogical
completeness''\index{metalogical completeness} is found in
\cite{Megill}\index{Megill, Norman}.  The system is closely related to a
system developed by Monk\index{Monk, J. Donald} and Tarski\index{Tarski,
Alfred} in 1965 \cite{Monks}.

For example, the formula $\exists x \, x = y $ (given $y$, there exists some
$x$ equal to it) is a theorem of logic,\footnote{Specifically, it is a theorem
of those systems of logic that assume non-empty domains.  It is not a theorem
of more general systems that include the empty domain\index{empty domain}, in
which nothing exists, period!  Such systems are called ``free
logics.''\index{free logic} For a discussion of these systems, see
\cite{Leblanc}\index{Leblanc, Hugues}.  Since our use for logic is as a basis
for set theory, which has a non-empty domain, it is more convenient (and more
traditional) to use a less general system.  An interesting curiosity is that,
using a free logic as a basis for Zermelo--Fraenkel set
theory\index{Zermelo--Fraenkel set theory} (with the redundant Axiom of the
Null Set omitted),\index{Axiom of the Null Set} we cannot even prove the
existence of a single set without assuming the axiom of infinity!\index{Axiom
of Infinity}} whether or not $x$ and $y$ are distinct variables\index{distinct
variables}.  In many systems of logic, we would have to prove two theorems to
arrive at this result.  First we would prove ``$\exists x \, x = x $,'' then
we would separately prove ``$\exists x \, x = y $, where $x$ and $y$ are
distinct variables.''  We would then combine these two special cases ``outside
of the system'' (i.e.\ in our heads) to be able to claim, ``$\exists x \, x =
y $, regardless of whether $x$ and $y$ are distinct.''  In other words, the
combination of the two special cases is a
metatheorem.  In the system of logic
used in Metamath's set theory\index{set theory database (\texttt{set.mm})}
database, the axioms of logic are broken down into small pieces that allow
them to be reassembled in such a way that theorems such as these can be proved
directly.

Breaking down the axioms in this way makes them look peculiar and not very
intuitive at first, but rest assured that they are correct and complete.  Their
correctness is ensured because they are theorem schemes of standard first-order
logic (which you can easily verify if you are a logician).  Their completeness
follows from the fact that we explicitly derive the standard axioms of
first-order logic as theorems.  Deriving the standard axioms is somewhat
tricky, but once we're there, we have at our disposal a system that is less
awkward to work with in formal proofs\index{formal proof}.  In technical terms
that logicians understand, we eliminate the cumbersome concepts of ``free
variable,''\index{free variable} ``bound variable,''\index{bound variable} and
``proper substitution''\index{proper substitution}\index{substitution!proper}
as primitive notions.  These concepts are present in our system but are
defined in terms of concepts expressed by the axioms and can be eliminated in
principle.  In standard systems, these concepts are really like additional,
implicit axioms\index{implicit axiom} that are somewhat complex and cannot be
eliminated.

The traditional approach to logic, wherein free variables and proper
substitution is defined, is also possible to do directly in the Metamath
language.  However, the notation tends to become awkward, and there are
disadvantages:  for example, extending the definition of a wff with a
definition is awkward, because the free variable and proper substitution
concepts have to have their definitions also extended.  Our choice of
axioms for \texttt{set.mm} is to a certain extent a matter of style, in
an attempt to achieve overall simplicity, but you should also be aware
that the traditional approach is possible as well if you should choose
it.

\chapter{Using the Metamath Program}
\label{using}

\section{Installation}

The way that you install Metamath\index{Metamath!installation} on your
computer system will vary for different computers.  Current
instructions are provided with the Metamath program download at
\url{http://metamath.org}.  In general, the installation is simple.
There is one file containing the Metamath program itself.  This file is
usually called \texttt{metamath} or \texttt{metamath.}{\em xxx} where
{\em xxx} is the convention (such as \texttt{exe}) for an executable
program on your operating system.  There are several additional files
containing samples of the Metamath language, all ending with
\texttt{.mm}.  The file \texttt{set.mm}\index{set theory database
(\texttt{set.mm})} contains logic and set theory and can be used as a
starting point for other areas of mathematics.

You will also need a text editor\index{text editor} capable of editing plain
{\sc ascii}\footnote{American Standard Code for Information Interchange.} text
in order to prepare your input files.\index{ascii@{\sc ascii}}  Most computers
have this capability built in.  Note that plain text is not necessarily the
default for some word processing programs\index{word processor}, especially if
they can handle different fonts; for example, with Microsoft Word\index{Word
(Microsoft)}, you must save the file in the format ``Text Only With Line
Breaks'' to get a plain text\index{plain text} file.\footnote{It is
recommended that all lines in a Metamath source file be 79 characters or less
in length for compatibility among different computer terminals.  When creating
a source file on an editor such as Word, select a monospaced
font\index{monospaced font} such as Courier\index{Courier font} or
Monaco\index{Monaco font} to make this easier to achieve.  Better yet,
just use a plain text editor such as Notepad.}

On some computer systems, Metamath does not have the capability to print
its output directly; instead, you send its output to a file (using the
\texttt{open} commands described later).  The way you print this output
file depends on your computer.\index{printers} Some computers have a
print command, whereas with others, you may have to read the file into
an editor and print it from there.

If you want to print your Metamath source files with typeset formulas
containing standard mathematical symbols, you will need the \LaTeX\
typesetting program\index{latex@{\LaTeX}}, which is widely and freely
available for most operating systems.  It runs natively on Unix and
Linux, and can be installed on Windows as part of the free Cygwin
package (\url{http://cygwin.com}).

You can also produce {\sc html}\footnote{HyperText Markup Language.}
web pages.  The {\tt help html} command in the Metamath program will
assist you with this feature.

\section{Your First Formal System}\label{start}
\subsection{From Nothing to Zero}\label{startf}

To give you a feel for what the Metamath\index{Metamath} language looks like,
we will take a look at a very simple example from formal number
theory\index{number theory}.  This example is taken from
Mendelson\index{Mendelson, Elliot} \cite[p. 123]{Mendelson}.\footnote{To keep
the example simple, we have changed the formalism slightly, and what we call
axioms\index{axiom} are strictly speaking theorems\index{theorem} in
\cite{Mendelson}.}  We will look at a small subset of this theory, namely that
part needed for the first number theory theorem proved in \cite{Mendelson}.

First we will look at a standard formal proof\index{formal proof} for the
example we have picked, then we will look at the Metamath version.  If you
have never been exposed to formal proofs, the notation may seem to be such
overkill to express such simple notions that you may wonder if you are missing
something.  You aren't.  The concepts involved are in fact very simple, and a
detailed breakdown in this fashion is necessary to express the proof in a way
that can be verified mechanically.  And as you will see, Metamath breaks the
proof down into even finer pieces so that the mechanical verification process
can be about as simple as possible.

Before we can introduce the axioms\index{axiom} of the theory, we must define
the syntax rules for forming legal expressions\index{syntax rules}
(combinations of symbols) with which those axioms can be used. The number 0 is
a {\bf term}\index{term}; and if $ t$ and $r$ are terms, so is $(t+r)$. Here,
$ t$ and $r$ are ``metavariables''\index{metavariable} ranging over terms; they
themselves do not appear as symbols in an actual term.  Some examples of
actual terms are $(0 + 0)$ and $((0+0)+0)$.  (Note that our theory describes
only the number zero and sums of zeroes.  Of course, not much can be done with
such a trivial theory, but remember that we have picked a very small subset of
complete number theory for our example.  The important thing for you to focus
on is our definitions that describe how symbols are combined to form valid
expressions, and not on the content or meaning of those expressions.) If $ t$
and $r$ are terms, an expression of the form $ t=r$ is a {\bf wff}
(well-formed formula)\index{well-formed formula (wff)}; and if $P$ and $Q$ are
wffs, so is $(P\rightarrow Q)$ (which means ``$P$ implies
$Q$''\index{implication ($\rightarrow$)} or ``if $P$ then $Q$'').
Here $P$ and $Q$ are metavariables ranging over wffs.  Examples of actual
wffs are $0=0$, $(0+0)=0$, $(0=0 \rightarrow (0+0)=0)$, and $(0=0\rightarrow
(0=0\rightarrow 0=(0+0)))$.  (Our notation makes use of more parentheses than
are customary, but the elimination of ambiguity this way simplifies our
example by avoiding the need to define operator precedence\index{operator
precedence}.)

The {\bf axioms}\index{axiom} of our theory are all wffs of the following
form, where $ t$, $r$, and $s$ are any terms:

%Latex p. 92
\renewcommand{\theequation}{A\arabic{equation}}

\begin{equation}
(t=r\rightarrow (t=s\rightarrow r=s))
\end{equation}
\begin{equation}
(t+0)=t
\end{equation}

Note that there are an infinite number of axioms since there are an infinite
number of possible terms.  A1 and A2 are properly called ``axiom
schemes,''\index{axiom scheme} but we will refer to them as ``axioms'' for
brevity.

An axiom is a {\bf theorem}; and if $P$ and $(P\rightarrow Q)$ are theorems
(where $P$ and $Q$ are wffs), then $Q$ is also a theorem.\index{theorem}  The
second part of this definition is called the modus ponens (MP) rule of
inference\index{inference rule}\index{modus ponens}.  It allows us to obtain
new theorems from old ones.

The {\bf proof}\index{proof} of a theorem is a sequence of one or more
theorems, each of which is either an axiom or the result of modus ponens
applied to two previous theorems in the sequence, and the last of which is the
theorem being proved.

The theorem we will prove for our example is very simple:  $ t=t$.  The proof of
our theorem follows.  Study it carefully until you feel sure you
understand it.\label{zeroproof}

% Use tabu so that lines will wrap automatically as needed.
\begin{tabu} { l X X }
1. & $(t+0)=t$ & (by axiom A2) \\
2. & $(t+0)=t$ & (by axiom A2) \\
3. & $((t+0)=t \rightarrow ((t+0)=t\rightarrow t=t))$ & (by axiom A1) \\
4. & $((t+0)=t\rightarrow t=t)$ & (by MP applied to steps 2 and 3) \\
5. & $t=t$ & (by MP applied to steps 1 and 4) \\
\end{tabu}

(You may wonder why step 1 is repeated twice.  This is not necessary in the
formal language we have defined, but in Metamath's ``reverse Polish
notation''\index{reverse Polish notation (RPN)} for proofs, a previous step
can be referred to only once.  The repetition of step~1 here will enable you
to see more clearly the correspondence of this proof with the
Metamath\index{Metamath} version on p.~\pageref{demoproof}.)

Our theorem is more properly called a ``theorem scheme,''\index{theorem
scheme} for it represents an infinite number of theorems, one for each
possible term $ t$.  Two examples of actual theorems would be $0=0$ and
$(0+0)=(0+0)$.  Rarely do we prove actual theorems, since by proving schemes
we can prove an infinite number of theorems in one fell swoop.  Similarly, our
proof should really be called a ``proof scheme.''\index{proof scheme}  To
obtain an actual proof, pick an actual term to use in place of $ t$, and
substitute it for $ t$ throughout the proof.

Let's discuss what we have done here.  The axioms\index{axiom} of our theory,
A1 and A2, are trivial and obvious.  Everyone knows that adding zero to
something doesn't change it, and also that if two things are equal to a third,
then they are equal to each other. In fact, stating the trivial and obvious is
a goal to strive for in any axiomatic system.  From trivial and obvious truths
that everyone agrees upon, we can prove results that are not so obvious yet
have absolute faith in them.  If we trust the axioms and the rules, we must,
by definition, trust the consequences of those axioms and rules, if logic is
to mean anything at all.

Our rule of inference\index{rule}, modus ponens\index{modus ponens}, is also
pretty obvious once you understand what it means.  If we prove a fact $P$, and
we also prove that $P$ implies $Q$, then $Q$ necessarily follows as a new
fact.  The rule provides us with a means for obtaining new facts (i.e.\
theorems\index{theorem}) from old ones.

The theorem that we have proved, $ t=t$, is so fundamental that you may wonder
why it isn't one of the axioms\index{axiom}.  In some axiom systems of
arithmetic, it {\em is} an axiom.  The choice of axioms in a theory is to some
extent arbitrary and even an art form, constrained only by the requirement
that any two equivalent axiom systems be able to derive each other as
theorems.  We could imagine that the inventor of our axiom system originally
included $ t=t$ as an axiom, then discovered that it could be derived as a
theorem from the other axioms.  Because of this, it was not necessary to
keep it as an axiom.  By eliminating it, the final set of axioms became
that much simpler.

Unless you have worked with formal proofs\index{formal proof} before, it
probably wasn't apparent to you that $ t=t$ could be derived from our two
axioms until you saw the proof. While you certainly believe that $ t=t$ is
true, you might not be able to convince an imaginary skeptic who believes only
in our two axioms until you produce the proof.  Formal proofs such as this are
hard to come up with when you first start working with them, but after you get
used to them they can become interesting and fun.  Once you understand the
idea behind formal proofs you will have grasped the fundamental principle that
underlies all of mathematics.  As the mathematics becomes more sophisticated,
its proofs become more challenging, but ultimately they all can be broken down
into individual steps as simple as the ones in our proof above.

Mendelson's\index{Mendelson, Elliot} book, from which our example was taken,
contains a number of detailed formal proofs such as these, and you may be
interested in looking it up.  The book is intended for mathematicians,
however, and most of it is rather advanced.  Popular literature describing
formal proofs\index{formal proof} include \cite[p.~296]{Rucker}\index{Rucker,
Rudy} and \cite[pp.~204--230]{Hofstadter}\index{Hofstadter, Douglas R.}.

\subsection{Converting It to Metamath}\label{convert}

Formal proofs\index{formal proof} such as the one in our example break down
logical reasoning into small, precise steps that leave little doubt that the
results follow from the axioms\index{axiom}.  You might think that this would
be the finest breakdown we can achieve in mathematics.  However, there is more
to the proof than meets the eye. Although our axioms were rather simple, a lot
of verbiage was needed before we could even state them:  we needed to define
``term,'' ``wff,'' and so on.  In addition, there are a number of implied
rules that we haven't even mentioned. For example, how do we know that step 3
of our proof follows from axiom A1? There is some hidden reasoning involved in
determining this.  Axiom A1 has two occurrences of the letter $ t$.  One of
the implied rules states that whatever we substitute for $ t$ must be a legal
term\index{term}.\footnote{Some authors make this implied rule explicit by
stating, ``only expressions of the above form are terms,'' after defining
``term.''}  The expression $ t+0$ is pretty obviously a legal term whenever $
t$ is, but suppose we wanted to substitute a huge term with thousands of
symbols?  Certainly a lot of work would be involved in determining that it
really is a term, but in ordinary formal proofs all of this work would be
considered a single ``step.''

To express our axiom system in the Metamath\index{Metamath} language, we must
describe this auxiliary information in addition to the axioms themselves.
Metamath does not know what a ``term'' or a ``wff''\index{well-formed formula
(wff)} is.  In Metamath, the specification of the ways in which we can combine
symbols to obtain terms and wffs are like little axioms in themselves.  These
auxiliary axioms are expressed in the same notation as the ``real''
axioms\index{axiom}, and Metamath does not distinguish between the two.  The
distinction is made by you, i.e.\ by the way in which you interpret the
notation you have chosen to express these two kinds of axioms.

The Metamath language breaks down mathematical proofs into tiny pieces, much
more so than in ordinary formal proofs\index{formal proof}.  If a single
step\index{proof step} involves the
substitution\index{substitution!variable}\index{variable substitution} of a
complex term for one of its variables, Metamath must see this single step
broken down into many small steps.  This fine-grained breakdown is what gives
Metamath generality and flexibility as it lets it not be limited to any
particular mathematical notation.

Metamath's proof notation is not, in itself, intended to be read by humans but
rather is in a compact format intended for a machine.  The Metamath program
will convert this notation to a form you can understand, using the \texttt{show
proof}\index{\texttt{show proof} command} command.  You can tell the program what
level of detail of the proof you want to look at.  You may want to look at
just the logical inference steps that correspond
to ordinary formal proof steps,
or you may want to see the fine-grained steps that prove that an expression is
a term.

Here, without further ado, is our example converted to the
Metamath\index{Metamath} language:\index{metavariable}\label{demo0}

\begin{verbatim}
$( Declare the constant symbols we will use $)
    $c 0 + = -> ( ) term wff |- $.
$( Declare the metavariables we will use $)
    $v t r s P Q $.
$( Specify properties of the metavariables $)
    tt $f term t $.
    tr $f term r $.
    ts $f term s $.
    wp $f wff P $.
    wq $f wff Q $.
$( Define "term" and "wff" $)
    tze $a term 0 $.
    tpl $a term ( t + r ) $.
    weq $a wff t = r $.
    wim $a wff ( P -> Q ) $.
$( State the axioms $)
    a1 $a |- ( t = r -> ( t = s -> r = s ) ) $.
    a2 $a |- ( t + 0 ) = t $.
$( Define the modus ponens inference rule $)
    ${
       min $e |- P $.
       maj $e |- ( P -> Q ) $.
       mp  $a |- Q $.
    $}
$( Prove a theorem $)
    th1 $p |- t = t $=
  $( Here is its proof: $)
       tt tze tpl tt weq tt tt weq tt a2 tt tze tpl
       tt weq tt tze tpl tt weq tt tt weq wim tt a2
       tt tze tpl tt tt a1 mp mp
     $.
\end{verbatim}\index{metavariable}

A ``database''\index{database} is a set of one or more {\sc ascii} source
files.  Here's a brief description of this Metamath\index{Metamath} database
(which consists of this single source file), so that you can understand in
general terms what is going on.  To understand the source file in detail, you
should read Chapter~\ref{languagespec}.

The database is a sequence of ``tokens,''\index{token} which are normally
separated by spaces or line breaks.  The only tokens that are built into
the Metamath language are those beginning with \texttt{\$}.  These tokens
are called ``keywords.''\index{keyword}  All other tokens are
user-defined, and their names are arbitrary.

As you might have guessed, the Metamath token \texttt{\$(}\index{\texttt{\$(} and
\texttt{\$)} auxiliary keywords} starts a comment and \texttt{\$)} ends a comment.

The Metamath tokens \texttt{\$c}\index{\texttt{\$c} statement},
\texttt{\$v}\index{\texttt{\$v} statement},
\texttt{\$e}\index{\texttt{\$e} statement},
\texttt{\$f}\index{\texttt{\$f} statement},
\texttt{\$a}\index{\texttt{\$a} statement}, and
\texttt{\$p}\index{\texttt{\$p} statement} specify ``statements'' that
end with \texttt{\$.}\,.\index{\texttt{\$.}\ keyword}

The Metamath tokens \texttt{\$c} and \texttt{\$v} each declare\index{constant
declaration}\index{variable declaration} a list of user-defined tokens, called
``math symbols,''\index{math symbol} that the database will reference later
on.  All of the math symbols they define you have seen earlier except the
turnstile symbol \texttt{|-} ($\vdash$)\index{turnstile ({$\,\vdash$})}, which is
commonly used by logicians to mean ``a proof exists for.''  For us
the turnstile is just a
convenient symbol that distinguishes expressions that are axioms\index{axiom}
or theorems\index{theorem} from expressions that are terms or wffs.

The \texttt{\$c} statement declares ``constants''\index{constant} and
the \texttt{\$v} statement declares
``variables''\index{variable}\index{constant declaration}\index{variable
declaration} (or more precisely, metavariables\index{metavariable}).  A
variable may be substituted\index{substitution!variable}\index{variable
substitution} with sequences of math symbols whereas a constant may not
be substituted with anything.

It may seem redundant to require both \texttt{\$c}\index{\texttt{\$c} statement} and
\texttt{\$v}\index{\texttt{\$v} statement} statements (since any math
symbol\index{math symbol} not specified with a \texttt{\$c} statement could be
presumed to be a variable), but this provides for better error checking and
also allows math symbols to be redeclared\index{redeclaration of symbols}
(Section~\ref{scoping}).

The token \texttt{\$f}\index{\texttt{\$f} statement} specifies a
statement called a ``variable-type hypothesis'' (also called a
``floating hypothesis'') and \texttt{\$e}\index{\texttt{\$e} statement}
specifies a ``logical hypothesis'' (also called an ``essential
hypothesis'').\index{hypothesis}\index{variable-type
hypothesis}\index{logical hypothesis}\index{floating
hypothesis}\index{essential hypothesis} The token
\texttt{\$a}\index{\texttt{\$a} statement} specifies an ``axiomatic
assertion,''\index{axiomatic assertion} and
\texttt{\$p}\index{\texttt{\$p} statement} specifies a ``provable
assertion.''\index{provable assertion} To the left of each occurrence of
these four tokens is a ``label''\index{label} that identifies the
hypothesis or assertion for later reference.  For example, the label of
the first axiomatic assertion is \texttt{tze}.  A \texttt{\$f} statement
must contain exactly two math symbols, a constant followed by a
variable.  The \texttt{\$e}, \texttt{\$a}, and \texttt{\$p} statements
each start with a constant followed by, in general, an arbitrary
sequence of math symbols.

Associated with each assertion\index{assertion} is a set of hypotheses
that must be satisfied in order for the assertion to be used in a proof.
These are called the ``mandatory hypotheses''\index{mandatory
hypothesis} of the assertion.  Among those hypotheses whose ``scope''
(described below) includes the assertion, \texttt{\$e} hypotheses are
always mandatory and \texttt{\$f}\index{\texttt{\$f} statement}
hypotheses are mandatory when they share their variable with the
assertion or its \texttt{\$e} hypotheses.  The exact rules for
determining which hypotheses are mandatory are described in detail in
Sections~\ref{frames} and \ref{scoping}.  For example, the mandatory
hypotheses of assertion \texttt{tpl} are \texttt{tt} and \texttt{tr},
whereas assertion \texttt{tze} has no mandatory hypotheses because it
contains no variables and has no \texttt{\$e}\index{\texttt{\$e}
statement} hypothesis.  Metamath's \texttt{show statement}
command\index{\texttt{show statement} command}, described in the next
section, will show you a statement's mandatory hypotheses.

Sometimes we need to make a hypothesis relevant to only certain
assertions.  The set of statements to which a hypothesis is relevant is
called its ``scope.''  The Metamath brackets,
\texttt{\$\char`\{}\index{\texttt{\$\char`\{} and \texttt{\$\char`\}}
keywords} and \texttt{\$\char`\}}, define a ``block''\index{block} that
delimits the scope of any hypothesis contained between them.  The
assertion \texttt{mp} has mandatory hypotheses \texttt{wp}, \texttt{wq},
\texttt{min}, and \texttt{maj}.  The only mandatory hypothesis of
\texttt{th1}, on the other hand, is \texttt{tt}, since \texttt{th1}
occurs outside of the block containing \texttt{min} and \texttt{maj}.

Note that \texttt{\$\char`\{} and \texttt{\$\char`\}} do not affect the
scope of assertions (\texttt{\$a} and \texttt{\$p}).  Assertions are always
available to be referenced by any later proof in the source file.

Each provable assertion (\texttt{\$p}\index{\texttt{\$p} statement}
statement) has two parts.  The first part is the
assertion\index{assertion} itself, which is a sequence of math
symbol\index{math symbol} tokens placed between the \texttt{\$p} token
and a \texttt{\$=}\index{\texttt{\$=} keyword} token.  The second part
is a ``proof,'' which is a list of label tokens placed between the
\texttt{\$=} token and the \texttt{\$.}\index{\texttt{\$.}\ keyword}\
token that ends the statement.\footnote{If you've looked at the
\texttt{set.mm} database, you may have noticed another notation used for
proofs.  The other notation is called ``compressed.''\index{compressed
proof}\index{proof!compressed} It reduces the amount of space needed to
store a proof in the database and is described in
Appendix~\ref{compressed}.  In the example above, we use
``normal''\index{normal proof}\index{proof!normal} notation.} The proof
acts as a series of instructions to the Metamath program, telling it how
to build up the sequence of math symbols contained in the assertion part of
the \texttt{\$p} statement, making use of the hypotheses of the
\texttt{\$p} statement and previous assertions.  The construction takes
place according to precise rules.  If the list of labels in the proof
causes these rules to be violated, or if the final sequence that results
does not match the assertion, the Metamath program will notify you with
an error message.

If you are familiar with reverse Polish notation (RPN), which is sometimes used
on pocket calculators, here in a nutshell is how a proof works.  Each
hypothesis label\index{hypothesis label} in the proof is pushed\index{push}
onto the RPN stack\index{stack}\index{RPN stack} as it is encountered. Each
assertion label\index{assertion label} pops\index{pop} off the stack as many
entries as the referenced assertion has mandatory hypotheses.  Variable
substitutions\index{substitution!variable}\index{variable substitution} are
computed which, when made to the referenced assertion's mandatory hypotheses,
cause these hypotheses to match the stack entries. These same substitutions
are then made to the variables in the referenced assertion itself, which is
then pushed onto the stack.  At the end of the proof, there should be one
stack entry, namely the assertion being proved.  This process is explained in
detail in Section~\ref{proof}.

Metamath's proof notation is not very readable for humans, but it allows the
proof to be stored compactly in a file.  The Metamath\index{Metamath} program
has proof display features that let you see what's going on in a more
readable way, as you will see in the next section.

The rules used in verifying a proof are not based on any built-in syntax of the
symbol sequence in an assertion\index{assertion} nor on any built-in meanings
attached to specific symbol names.  They are based strictly on symbol
matching:  constants\index{constant} must match themselves, and
variables\index{variable} may be replaced with anything that allows a match to
occur.  For example, instead of \texttt{term}, \texttt{0}, and \verb$|-$ we could
have just as well used \texttt{yellow}, \texttt{zero}, and \texttt{provable}, as long
as we did so consistently throughout the database.  Also, we could have used
\texttt{is provable} (two tokens) instead of \verb$|-$ (one token) throughout the
database.  In each of these cases, the proof would be exactly the same.  The
independence of proofs and notation means that you have a lot of flexibility to
change the notation you use without having to change any proofs.

\section{A Trial Run}\label{trialrun}

Now you are ready to try out the Metamath\index{Metamath} program.

On all computer systems, Metamath has a standard ``command line
interface'' (CLI)\index{command line interface (CLI)} that allows you to
interact with it.  You supply commands to the CLI by typing them on the
keyboard and pressing your keyboard's {\em return} key after each line
you enter.  The CLI is designed to be easy to use and has built-in help
features.

The first thing you should do is to use a text editor to create a file
called \texttt{demo0.mm} and type into it the Metamath source shown on
p.~\pageref{demo0}.  Actually, this file is included with your Metamath
software package, so check that first.  If you type it in, make sure
that you save it in the form of ``plain {\sc ascii} text with line
breaks.''  Most word processors will have this feature.

Next you must run the Metamath program.  Depending on your computer
system and how Metamath is installed, this could range from clicking the
mouse on the Metamath icon to typing \texttt{run metamath} to typing
simply \texttt{metamath}.  (Metamath's {\tt help invoke} command describes
alternate ways of invoking the Metamath program.)

When you first enter Metamath\index{Metamath}, it will be at the CLI, waiting
for your input. You will see something like the following on your screen:
\begin{verbatim}
Metamath - Version 0.172 25-Jan-2019
Type HELP for help, EXIT to exit.
MM>
\end{verbatim}
The \texttt{MM>} prompt means that Metamath is waiting for a command.
Command keywords\index{command keyword} are not case sensitive;
we will use lower-case commands in our examples.
The version number and its release date will probably be different on your
system from the one we show above.

The first thing that you need to do is to read in your
database:\index{\texttt{read} command}\footnote{If a directory path is
needed on Unix,\index{Unix file names}\index{file names!Unix} you should
enclose the path/file name in quotes to prevent Metamath from thinking
that the \texttt{/} in the path name is a command qualifier, e.g.,
\texttt{read \char`\"db/set.mm\char`\"}.  Quotes are optional when there
is no ambiguity.}
\begin{verbatim}
MM> read demo0.mm
\end{verbatim}
Remember to press the {\em return} key after entering this command.  If
you omit the file name, Metamath will prompt you for one.   The syntax for
specifying a Macintosh file name path is given in a footnote on
p.~\pageref{includef}.\index{Macintosh file names}\index{file
names!Macintosh}

If there are any syntax errors in the database, Metamath will let you know
when it reads in the file.  The one thing that Metamath does not check when
reading in a database is that all proofs are correct, because this would
slow it down too much.  It is a good idea to periodically verify the proofs in
a database you are making changes to.  To do this, use the following command
(and do it for your \texttt{demo0.mm} file now).  Note that the \texttt{*} is a
``wild card'' meaning all proofs in the file.\index{\texttt{verify proof} command}
\begin{verbatim}
MM> verify proof *
\end{verbatim}
Metamath will report any proofs that are incorrect.

It is often useful to save the information that the Metamath program displays
on the screen. You can save everything that happens on the screen by opening a
log file. You may want to do this before you read in a database so that you
can examine any errors later on.  To open a log file, type
\begin{verbatim}
MM> open log abc.log
\end{verbatim}
This will open a file called \texttt{abc.log}, and everything that appears on the
screen from this point on will be stored in this file.  The name of the log file
is arbitrary. To close the log file, type
\begin{verbatim}
MM> close log
\end{verbatim}

Several commands let you examine what's inside your database.
Section~\ref{exploring} has an overview of some useful ones.  The
\texttt{show labels} command lets you see what statement
labels\index{label} exist.  A \texttt{*} matches any combination of
characters, and \texttt{t*} refers to all labels starting with the
letter \texttt{t}.\index{\texttt{show labels} command} The \texttt{/all}
is a ``command qualifier''\index{command qualifier} that tells Metamath
to include labels of hypotheses.  (To see the syntax explained, type
\texttt{help show labels}.)  Type
\begin{verbatim}
MM> show labels t* /all
\end{verbatim}
Metamath will respond with
\begin{verbatim}
The statement number, label, and type are shown.
3 tt $f       4 tr $f       5 ts $f       8 tze $a
9 tpl $a      19 th1 $p
\end{verbatim}

You can use the \texttt{show statement} command to get information about a
particular statement.\index{\texttt{show statement} command}
For example, you can get information about the statement with label \texttt{mp}
by typing
\begin{verbatim}
MM> show statement mp /full
\end{verbatim}
Metamath will respond with
\begin{verbatim}
Statement 17 is located on line 43 of the file
"demo0.mm".
"Define the modus ponens inference rule"
17 mp $a |- Q $.
Its mandatory hypotheses in RPN order are:
  wp $f wff P $.
  wq $f wff Q $.
  min $e |- P $.
  maj $e |- ( P -> Q ) $.
The statement and its hypotheses require the
      variables:  Q P
The variables it contains are:  Q P
\end{verbatim}
The mandatory hypotheses\index{mandatory hypothesis} and their
order\index{RPN order} are
useful to know when you are trying to understand or debug a proof.

Now you are ready to look at what's really inside our proof.  First, here is
how to look at every step in the proof---not just the ones corresponding to an
ordinary formal proof\index{formal proof}, but also the ones that build up the
formulas that appear in each ordinary formal proof step.\index{\texttt{show
proof} command}
\begin{verbatim}
MM> show proof th1 /lemmon /all
\end{verbatim}

This will display the proof on the screen in the following format:
\begin{verbatim}
 1 tt            $f term t
 2 tze           $a term 0
 3 1,2 tpl       $a term ( t + 0 )
 4 tt            $f term t
 5 3,4 weq       $a wff ( t + 0 ) = t
 6 tt            $f term t
 7 tt            $f term t
 8 6,7 weq       $a wff t = t
 9 tt            $f term t
10 9 a2          $a |- ( t + 0 ) = t
11 tt            $f term t
12 tze           $a term 0
13 11,12 tpl     $a term ( t + 0 )
14 tt            $f term t
15 13,14 weq     $a wff ( t + 0 ) = t
16 tt            $f term t
17 tze           $a term 0
18 16,17 tpl     $a term ( t + 0 )
19 tt            $f term t
20 18,19 weq     $a wff ( t + 0 ) = t
21 tt            $f term t
22 tt            $f term t
23 21,22 weq     $a wff t = t
24 20,23 wim     $a wff ( ( t + 0 ) = t -> t = t )
25 tt            $f term t
26 25 a2         $a |- ( t + 0 ) = t
27 tt            $f term t
28 tze           $a term 0
29 27,28 tpl     $a term ( t + 0 )
30 tt            $f term t
31 tt            $f term t
32 29,30,31 a1   $a |- ( ( t + 0 ) = t -> ( ( t + 0 )
                                     = t -> t = t ) )
33 15,24,26,32 mp  $a |- ( ( t + 0 ) = t -> t = t )
34 5,8,10,33 mp  $a |- t = t
\end{verbatim}

The \texttt{/lemmon} command qualifier specifies what is known as a Lemmon-style
display\index{Lemmon-style proof}\index{proof!Lemmon-style}.  Omitting the
\texttt{/lemmon} qualifier results in a tree-style proof (see
p.~\pageref{treeproof} for an example) that is somewhat less explicit but
easier to follow once you get used to it.\index{tree-style
proof}\index{proof!tree-style}

The first number on each line is the step
number of the proof.  Any numbers that follow are step numbers assigned to the
hypotheses of the statement referenced by that step.  Next is the label of
the statement referenced by the step.  The statement type of the statement
referenced comes next, followed by the math symbol\index{math symbol} string
constructed by the proof up to that step.

The last step, 34, contains the statement that is being proved.

Looking at a small piece of the proof, notice that steps 3 and 4 have
established that
\texttt{( t + 0 )} and \texttt{t} are \texttt{term}\,s, and step 5 makes use of steps 3 and
4 to establish that \texttt{( t + 0 ) = t} is a \texttt{wff}.  Let Metamath
itself tell us in detail what is happening in step 5.  Note that the
``target hypothesis'' refers to where step 5 is eventually used, i.e., in step
34.
\begin{verbatim}
MM> show proof th1 /detailed_step 5
Proof step 5:  wp=weq $a wff ( t + 0 ) = t
This step assigns source "weq" ($a) to target "wp"
($f).  The source assertion requires the hypotheses
"tt" ($f, step 3) and "tr" ($f, step 4).  The parent
assertion of the target hypothesis is "mp" ($a,
step 34).
The source assertion before substitution was:
    weq $a wff t = r
The following substitutions were made to the source
assertion:
    Variable  Substituted with
     t         ( t + 0 )
     r         t
The target hypothesis before substitution was:
    wp $f wff P
The following substitution was made to the target
hypothesis:
    Variable  Substituted with
     P         ( t + 0 ) = t
\end{verbatim}

The full proof just shown is useful to understand what is going on in detail.
However, most of the time you will just be interested in
the ``essential'' or logical steps of a proof, i.e.\ those steps
that correspond to an
ordinary formal proof\index{formal proof}.  If you type
\begin{verbatim}
MM> show proof th1 /lemmon /renumber
\end{verbatim}
you will see\label{demoproof}
\begin{verbatim}
1 a2             $a |- ( t + 0 ) = t
2 a2             $a |- ( t + 0 ) = t
3 a1             $a |- ( ( t + 0 ) = t -> ( ( t + 0 )
                                     = t -> t = t ) )
4 2,3 mp         $a |- ( ( t + 0 ) = t -> t = t )
5 1,4 mp         $a |- t = t
\end{verbatim}
Compare this to the formal proof on p.~\pageref{zeroproof} and
notice the resemblance.
By default Metamath
does not show \texttt{\$f}\index{\texttt{\$f}
statement} hypotheses and everything branching off of them in the proof tree
when the proof is displayed; this makes the proof look more like an ordinary
mathematical proof, which does not normally incorporate the explicit
construction of expressions.
This is called the ``essential'' view
(at one time you had to add the
\texttt{/essential} qualifier in the \texttt{show proof}
command to get this view, but this is now the default).
You can could use the \texttt{/all} qualifier in the \texttt{show
proof} command to also show the explicit construction of expressions.
The \texttt{/renumber} qualifier means to renumber
the steps to correspond only to what is displayed.\index{\texttt{show proof}
command}

To exit Metamath, type\index{\texttt{exit} command}
\begin{verbatim}
MM> exit
\end{verbatim}

\subsection{Some Hints for Using the Command Line Interface}

We will conclude this quick introduction to Metamath\index{Metamath} with some
helpful hints on how to navigate your way through the commands.
\index{command line interface (CLI)}

When you type commands into Metamath's CLI, you only have to type as many
characters of a command keyword\index{command keyword} as are needed to make
it unambiguous.  If you type too few characters, Metamath will tell you what
the choices are.  In the case of the \texttt{read} command, only the \texttt{r} is
needed to specify it unambiguously, so you could have typed\index{\texttt{read}
command}
\begin{verbatim}
MM> r demo0.mm
\end{verbatim}
instead of
\begin{verbatim}
MM> read demo0.mm
\end{verbatim}
In our description, we always show the full command words.  When using the
Metamath CLI commands in a command file (to be read with the \texttt{submit}
command)\index{\texttt{submit} command}, it is good practice to use
the unabbreviated command to ensure your instructions will not become ambiguous
if more commands are added to the Metamath program in the future.

The command keywords\index{command
keyword} are not case sensitive; you may type either \texttt{read} or
\texttt{ReAd}.  File names may or may not be case sensitive, depending on your
computer's operating system.  Metamath label\index{label} and math
symbol\index{math symbol} tokens\index{token} are case-sensitive.

The \texttt{help} command\index{\texttt{help} command} will provide you
with a list of topics you can get help on.  You can then type
\texttt{help} {\em topic} to get help on that topic.

If you are uncertain of a command's spelling, just type as many characters
as you remember of the command.  If you have not typed enough characters to
specify it unambiguously, Metamath will tell you what choices you have.

\begin{verbatim}
MM> show s
         ^
?Ambiguous keyword - please specify SETTINGS,
STATEMENT, or SOURCE.
\end{verbatim}

If you don't know what argument to use as part of a command, type a
\texttt{?}\index{\texttt{]}@\texttt{?}\ in command lines}\ at the
argument position.  Metamath will tell you what it expected there.

\begin{verbatim}
MM> show ?
         ^
?Expected SETTINGS, LABELS, STATEMENT, SOURCE, PROOF,
MEMORY, TRACE_BACK, or USAGE.
\end{verbatim}

Finally, you may type just the first word or words of a command followed
by {\em return}.  Metamath will prompt you for the remaining part of the
command, showing you the choices at each step.  For example, instead of
typing \texttt{show statement th1 /full} you could interact in the
following manner:
\begin{verbatim}
MM> show
SETTINGS, LABELS, STATEMENT, SOURCE, PROOF,
MEMORY, TRACE_BACK, or USAGE <SETTINGS>? st
What is the statement label <th1>?
/ or nothing <nothing>? /
TEX, COMMENT_ONLY, or FULL <TEX>? f
/ or nothing <nothing>?
19 th1 $p |- t = t $= ... $.
\end{verbatim}
After each \texttt{?}\ in this mode, you must give Metamath the
information it requests.  Sometimes Metamath gives you a list of choices
with the default choice indicated by brackets \texttt{< > }. Pressing
{\em return} after the \texttt{?}\ will select the default choice.
Answering anything else will override the default.  Note that the
\texttt{/} in command qualifiers is considered a separate
token\index{token} by the parser, and this is why it is asked for
separately.

\section{Your First Proof}\label{frstprf}

Proofs are developed with the aid of the Proof Assistant\index{Proof
Assistant}.  We will now show you how the proof of theorem \texttt{th1}
was built.  So that you can repeat these steps, we will first have the
Proof Assistant erase the proof in Metamath's source buffer\index{source
buffer}, then reconstruct it.  (The source buffer is the place in memory
where Metamath stores the information in the database when it is
\texttt{read}\index{\texttt{read} command} in.  New or modified proofs
are kept in the source buffer until a \texttt{write source}
command\index{\texttt{write source} command} is issued.)  In practice, you
would place a \texttt{?}\index{\texttt{]}@\texttt{?}\ inside proofs}\
between \texttt{\$=}\index{\texttt{\$=} keyword} and
\texttt{\$.}\index{\texttt{\$.}\ keyword}\ in the database to indicate
to Metamath\index{Metamath} that the proof is unknown, and that would be
your starting point.  Whenever the \texttt{verify proof} command encounters
a proof with a \texttt{?}\ in place of a proof step, the statement is
identified as not proved.

When I first started creating Metamath proofs, I would write down
on a piece of paper the complete
formal proof\index{formal proof} as it would appear
in a \texttt{show proof} command\index{\texttt{show proof} command}; see
the display of \texttt{show proof th1 /lemmon /re\-num\-ber} above as an
example.  After you get used to using the Proof Assistant\index{Proof
Assistant} you may get to a point where you can ``see'' the proof in your mind
and let the Proof Assistant guide you in filling in the details, at least for
simpler proofs, but until you gain that experience you may find it very useful
to write down all the details in advance.
Otherwise you may waste a lot of time as you let it take you down a wrong path.
However, others do not find this approach as helpful.
For example, Thomas Brendan Leahy\index{Leahy, Thomas Brendan}
finds that it is more helpful to him to interactively
work backward from a machine-readable statement.
David A. Wheeler\index{Wheeler, David A.}
writes down a general approach, but develops the proof
interactively by switching between
working forwards (from hypotheses and facts likely to be useful) and
backwards (from the goal) until the forwards and backwards approaches meet.
In the end, use whatever approach works for you.

A proof is developed with the Proof Assistant by working backwards, starting
with the theorem\index{theorem} to be proved, and assigning each unknown step
with a theorem or hypothesis until no more unknown steps remain.  The Proof
Assistant will not let you make an assignment unless it can be ``unified''
with the unknown step.  This means that a
substitution\index{substitution!variable}\index{variable substitution} of
variables exists that will make the assignment match the unknown step.  On the
other hand, in the middle of a proof, when working backwards, often more than
one unification\index{unification} (set of substitutions) is possible, since
there is not enough information available at that point to uniquely establish
it.  In this case you can tell Metamath which unification to choose, or you
can continue to assign unknown steps until enough information is available to
make the unification unique.

We will assume you have entered Metamath and read in the database as described
above.  The following dialog shows how the proof was developed.  For more
details on what some of the commands do, refer to Section~\ref{pfcommands}.
\index{\texttt{prove} command}

\begin{verbatim}
MM> prove th1
Entering the Proof Assistant.  Type HELP for help, EXIT
to exit.  You will be working on the proof of statement th1:
  $p |- t = t
Note:  The proof you are starting with is already complete.
MM-PA>
\end{verbatim}

The \verb/MM-PA>/ prompt means we are inside the Proof
Assistant.\index{Proof Assistant} Most of the regular Metamath commands
(\texttt{show statement}, etc.) are still available if you need them.

\begin{verbatim}
MM-PA> delete all
The entire proof was deleted.
\end{verbatim}

We have deleted the whole proof so we can start from scratch.

\begin{verbatim}
MM-PA> show new_proof/lemmon/all
1 ?              $? |- t = t
\end{verbatim}

The \texttt{show new{\char`\_}proof} command\index{\texttt{show
new{\char`\_}proof} command} is like \texttt{show proof} except that we
don't specify a statement; instead, the proof we're working on is
displayed.

\begin{verbatim}
MM-PA> assign 1 mp
To undo the assignment, DELETE STEP 5 and INITIALIZE, UNIFY
if needed.
3   min=?  $? |- $2
4   maj=?  $? |- ( $2 -> t = t )
\end{verbatim}

The \texttt{assign} command\index{\texttt{assign} command} above means
``assign step 1 with the statement whose label is \texttt{mp}.''  Note
that step renumbering will constantly occur as you assign steps in the
middle of a proof; in general all steps from the step you assign until
the end of the proof will get moved up.  In this case, what used to be
step 1 is now step 5, because the (partial) proof now has five steps:
the four hypotheses of the \texttt{mp} statement and the \texttt{mp}
statement itself.  Let's look at all the steps in our partial proof:

\begin{verbatim}
MM-PA> show new_proof/lemmon/all
1 ?              $? wff $2
2 ?              $? wff t = t
3 ?              $? |- $2
4 ?              $? |- ( $2 -> t = t )
5 1,2,3,4 mp     $a |- t = t
\end{verbatim}

The symbol \texttt{\$2} is a temporary variable\index{temporary
variable} that represents a symbol sequence not yet known.  In the final
proof, all temporary variables will be eliminated.  The general format
for a temporary variable is \texttt{\$} followed by an integer.  Note
that \texttt{\$} is not a legal character in a math symbol (see
Section~\ref{dollardollar}, p.~\pageref{dollardollar}), so there will
never be a naming conflict between real symbols and temporary variables.

Unknown steps 1 and 2 are constructions of the two wffs used by the
modus ponens rule.  As you will see at the end of this section, the
Proof Assistant\index{Proof Assistant} can usually figure these steps
out by itself, and we will not have to worry about them.  Therefore from
here on we will display only the ``essential'' hypotheses, i.e.\ those
steps that correspond to traditional formal proofs\index{formal proof}.

\begin{verbatim}
MM-PA> show new_proof/lemmon
3 ?              $? |- $2
4 ?              $? |- ( $2 -> t = t )
5 3,4 mp         $a |- t = t
\end{verbatim}

Unknown steps 3 and 4 are the ones we must focus on.  They correspond to the
minor and major premises of the modus ponens rule.  We will assign them as
follows.  Notice that because of the step renumbering that takes place
after an assignment, it is advantageous to assign unknown steps in reverse
order, because earlier steps will not get renumbered.

\begin{verbatim}
MM-PA> assign 4 mp
To undo the assignment, DELETE STEP 8 and INITIALIZE, UNIFY
if needed.
3   min=?  $? |- $2
6     min=?  $? |- $4
7     maj=?  $? |- ( $4 -> ( $2 -> t = t ) )
\end{verbatim}

We are now going to describe an obscure feature that you will probably
never use but should be aware of.  The Metamath language allows empty
symbol sequences to be substituted for variables, but in most formal
systems this feature is never used.  One of the few examples where is it
used is the MIU-system\index{MIU-system} described in
Appendix~\ref{MIU}.  But such systems are rare, and by default this
feature is turned off in the Proof Assistant.  (It is always allowed for
{\tt verify proof}.)  Let us turn it on and see what
happens.\index{\texttt{set empty{\char`\_}substitution} command}

\begin{verbatim}
MM-PA> set empty_substitution on
Substitutions with empty symbol sequences is now allowed.
\end{verbatim}

With this feature enabled, more unifications will be
ambiguous\index{ambiguous unification}\index{unification!ambiguous} in
the middle of a proof, because
substitution\index{substitution!variable}\index{variable substitution}
of variables with empty symbol sequences will become an additional
possibility.  Let's see what happens when we make our next assignment.

\begin{verbatim}
MM-PA> assign 3 a2
There are 2 possible unifications.  Please select the correct
    one or Q if you want to UNIFY later.
Unify:  |- $6
 with:  |- ( $9 + 0 ) = $9
Unification #1 of 2 (weight = 7):
  Replace "$6" with "( + 0 ) ="
  Replace "$9" with ""
  Accept (A), reject (R), or quit (Q) <A>? r
\end{verbatim}

The first choice presented is the wrong one.  If we had selected it,
temporary variable \texttt{\$6} would have been assigned a truncated
wff, and temporary variable \texttt{\$9} would have been assigned an
empty sequence (which is not allowed in our system).  With this choice,
eventually we would reach a point where we would get stuck because
we would end up with steps impossible to prove.  (You may want to
try it.)  We typed \texttt{r} to reject the choice.

\begin{verbatim}
Unification #2 of 2 (weight = 21):
  Replace "$6" with "( $9 + 0 ) = $9"
  Accept (A), reject (R), or quit (Q) <A>? q
To undo the assignment, DELETE STEP 4 and INITIALIZE, UNIFY
if needed.
 7     min=?  $? |- $8
 8     maj=?  $? |- ( $8 -> ( $6 -> t = t ) )
\end{verbatim}

The second choice is correct, and normally we would type \texttt{a}
to accept it.  But instead we typed \texttt{q} to show what will happen:
it will leave the step with an unknown unification, which can be
seen as follows:

\begin{verbatim}
MM-PA> show new_proof/not_unified
 4   min    $a |- $6
        =a2  = |- ( $9 + 0 ) = $9
\end{verbatim}

Later we can unify this with the \texttt{unify}
\texttt{all/interactive} command.

The important point to remember is that occasionally you will be
presented with several unification choices while entering a proof, when
the program determines that there is not enough information yet to make
an unambiguous choice automatically (and this can happen even with
\texttt{set empty{\char`\_}substitution} turned off).  Usually it is
obvious by inspection which choice is correct, since incorrect ones will
tend to be meaningless fragments of wffs.  In addition, the correct
choice will usually be the first one presented, unlike our example
above.

Enough of this digression.  Let us go back to the default setting.

\begin{verbatim}
MM-PA> set empty_substitution off
The ability to substitute empty expressions for variables
has been turned off.  Note that this may make the Proof
Assistant too restrictive in some cases.
\end{verbatim}

If we delete the proof, start over, and get to the point where
we digressed above, there will no longer be an ambiguous unification.

\begin{verbatim}
MM-PA> assign 3 a2
To undo the assignment, DELETE STEP 4 and INITIALIZE, UNIFY
if needed.
 7     min=?  $? |- $4
 8     maj=?  $? |- ( $4 -> ( ( $5 + 0 ) = $5 -> t = t ) )
\end{verbatim}

Let us look at our proof so far, and continue.

\begin{verbatim}
MM-PA> show new_proof/lemmon
 4 a2            $a |- ( $5 + 0 ) = $5
 7 ?             $? |- $4
 8 ?             $? |- ( $4 -> ( ( $5 + 0 ) = $5 -> t = t ) )
 9 7,8 mp        $a |- ( ( $5 + 0 ) = $5 -> t = t )
10 4,9 mp        $a |- t = t
MM-PA> assign 8 a1
To undo the assignment, DELETE STEP 11 and INITIALIZE, UNIFY
if needed.
 7     min=?  $? |- ( t + 0 ) = t
MM-PA> assign 7 a2
To undo the assignment, DELETE STEP 8 and INITIALIZE, UNIFY
if needed.
MM-PA> show new_proof/lemmon
 4 a2            $a |- ( t + 0 ) = t
 8 a2            $a |- ( t + 0 ) = t
12 a1            $a |- ( ( t + 0 ) = t -> ( ( t + 0 ) = t ->
                                                    t = t ) )
13 8,12 mp       $a |- ( ( t + 0 ) = t -> t = t )
14 4,13 mp       $a |- t = t
\end{verbatim}

Now all temporary variables and unknown steps have been eliminated from the
``essential'' part of the proof.  When this is achieved, the Proof
Assistant\index{Proof Assistant} can usually figure out the rest of the proof
automatically.  (Note that the \texttt{improve} command can occasionally be
useful for filling in essential steps as well, but it only tries to make use
of statements that introduce no new variables in their hypotheses, which is
not the case for \texttt{mp}. Also it will not try to improve steps containing
temporary variables.)  Let's look at the complete proof, then run
the \texttt{improve} command, then look at it again.

\begin{verbatim}
MM-PA> show new_proof/lemmon/all
 1 ?             $? wff ( t + 0 ) = t
 2 ?             $? wff t = t
 3 ?             $? term t
 4 3 a2          $a |- ( t + 0 ) = t
 5 ?             $? wff ( t + 0 ) = t
 6 ?             $? wff ( ( t + 0 ) = t -> t = t )
 7 ?             $? term t
 8 7 a2          $a |- ( t + 0 ) = t
 9 ?             $? term ( t + 0 )
10 ?             $? term t
11 ?             $? term t
12 9,10,11 a1    $a |- ( ( t + 0 ) = t -> ( ( t + 0 ) = t ->
                                                    t = t ) )
13 5,6,8,12 mp   $a |- ( ( t + 0 ) = t -> t = t )
14 1,2,4,13 mp   $a |- t = t
\end{verbatim}

\begin{verbatim}
MM-PA> improve all
A proof of length 1 was found for step 11.
A proof of length 1 was found for step 10.
A proof of length 3 was found for step 9.
A proof of length 1 was found for step 7.
A proof of length 9 was found for step 6.
A proof of length 5 was found for step 5.
A proof of length 1 was found for step 3.
A proof of length 3 was found for step 2.
A proof of length 5 was found for step 1.
Steps 1 and above have been renumbered.
CONGRATULATIONS!  The proof is complete.  Use SAVE
NEW_PROOF to save it.  Note:  The Proof Assistant does
not detect $d violations.  After saving the proof, you
should verify it with VERIFY PROOF.
\end{verbatim}

The \texttt{save new{\char`\_}proof} command\index{\texttt{save
new{\char`\_}proof} command} will save the proof in the database.  Here
we will just display it in a form that can be clipped out of a log file
and inserted manually into the database source file with a text
editor.\index{normal proof}\index{proof!normal}

\begin{verbatim}
MM-PA> show new_proof/normal
---------Clip out the proof below this line:
      tt tze tpl tt weq tt tt weq tt a2 tt tze tpl tt weq
      tt tze tpl tt weq tt tt weq wim tt a2 tt tze tpl tt
      tt a1 mp mp $.
---------The proof of 'th1' to clip out ends above this line.
\end{verbatim}

There is another proof format called ``compressed''\index{compressed
proof}\index{proof!compressed} that you will see in databases.  It is
not important to understand how it is encoded but only to recognize it
when you see it.  Its only purpose is to reduce storage requirements for
large proofs.  A compressed proof can always be converted to a normal
one and vice-versa, and the Metamath \texttt{show proof}
commands\index{\texttt{show proof} command} work equally well with
compressed proofs.  The compressed proof format is described in
Appendix~\ref{compressed}.

\begin{verbatim}
MM-PA> show new_proof/compressed
---------Clip out the proof below this line:
      ( tze tpl weq a2 wim a1 mp ) ABCZADZAADZAEZJJKFLIA
      AGHH $.
---------The proof of 'th1' to clip out ends above this line.
\end{verbatim}

Now we will exit the Proof Assistant.  Since we made changes to the proof,
it will warn us that we have not saved it.  In this case, we don't care.

\begin{verbatim}
MM-PA> exit
Warning:  You have not saved changes to the proof.
Do you want to EXIT anyway (Y, N) <N>? y
Exiting the Proof Assistant.
Type EXIT again to exit Metamath.
\end{verbatim}

The Proof Assistant\index{Proof Assistant} has several other commands
that can help you while creating proofs.  See Section~\ref{pfcommands}
for a list of them.

A command that is often useful is \texttt{minimize{\char`\_}with
*/brief}, which tries to shorten the proof.  It can make the process
more efficient by letting you write a somewhat ``sloppy'' proof then
clean up some of the fine details of optimization for you (although it
can't perform miracles such as restructuring the overall proof).

\section{A Note About Editing a Data\-base File}

Once your source file contains proofs, there are some restrictions on
how you can edit it so that the proofs remain valid.  Pay particular
attention to these rules, since otherwise you can lose a lot of work.
It is a good idea to periodically verify all proofs with \texttt{verify
proof *} to ensure their integrity.

If your file contains only normal (as opposed to compressed) proofs, the
main rule is that you may not change the order of the mandatory
hypotheses\index{mandatory hypothesis} of any statement referenced in a
later proof.  For example, if you swap the order of the major and minor
premise in the modus ponens rule, all proofs making use of that rule
will become incorrect.  The \texttt{show statement}
command\index{\texttt{show statement} command} will show you the
mandatory hypotheses of a statement and their order.

If a statement has a compressed proof, you also must not change the
order of {\em its} mandatory hypotheses.  The compressed proof format
makes use of this information as part of the compression technique.
Note that swapping the names of two variables in a theorem will change
the order of its mandatory hypotheses.

The safest way to edit a statement, say \texttt{mytheorem}, is to
duplicate it then rename the original to \texttt{mytheoremOLD}
throughout the database.  Once the edited version is re-proved, all
statements referencing \texttt{mytheoremOLD} can be updated in the Proof
Assistant using \texttt{minimize{\char`\_}with
mytheorem
/allow{\char`\_}growth}.\index{\texttt{minimize{\char`\_}with} command}
% 3/10/07 Note: line-breaking the above results in duplicate index entries

\chapter{Abstract Mathematics Revealed}\label{fol}

\section{Logic and Set Theory}\label{logicandsettheory}

\begin{quote}
  {\em Set theory can be viewed as a form of exact theology.}
  \flushright\sc  Rudy Rucker\footnote{\cite{Barrow}, p.~31.}\\
\end{quote}\index{Rucker, Rudy}

Despite its seeming complexity, all of standard mathematics, no matter how
deep or abstract, can amazingly enough be derived from a relatively small set
of axioms\index{axiom} or first principles. The development of these axioms is
among the most impressive and important accomplishments of mathematics in the
20th century. Ultimately, these axioms can be broken down into a set of rules
for manipulating symbols that any technically oriented person can follow.

We will not spend much time trying to convey a deep, higher-level
understanding of the meaning of the axioms. This kind of understanding
requires some mathematical sophistication as well as an understanding of the
philosophy underlying the foundations of mathematics and typically develops
over time as you work with mathematics.  Our goal, instead, is to give you the
immediate ability to follow how theorems\index{theorem} are derived from the
axioms and from other theorems.  This will be similar to learning the syntax
of a computer language, which lets you follow the details in a program but
does not necessarily give you the ability to write non-trivial programs on
your own, an ability that comes with practice. For now don't be alarmed by
abstract-sounding names of the axioms; just focus on the rules for
manipulating the symbols, which follow the simple conventions of the
Metamath\index{Metamath} language.

The axioms that underlie all of standard mathematics consist of axioms of logic
and axioms of set theory. The axioms of logic are divided into two
subcategories, propositional calculus\index{propositional calculus} (sometimes
called sentential logic\index{sentential logic}) and predicate calculus
(sometimes called first-order logic\index{first-order logic}\index{quantifier
theory}\index{predicate calculus} or quantifier theory).  Propositional
calculus is a prerequisite for predicate calculus, and predicate calculus is a
prerequisite for set theory.  The version of set theory most commonly used is
Zermelo--Fraenkel set theory\index{Zermelo--Fraenkel set theory}\index{set theory}
with the axiom of choice,
often abbreviated as ZFC\index{ZFC}.

Here in a nutshell is what the axioms are all about in an informal way. The
connection between this description and symbols we will show you won't be
immediately apparent and in principle needn't ever be.  Our description just
tries to summarize what mathematicians think about when they work with the
axioms.

Logic is a set of rules that allow us determine truths given other truths.
Put another way,
logic is more or less the translation of what we would consider common sense
into a rigorous set of axioms.\index{axioms of logic}  Suppose $\varphi$,
$\psi$, and $\chi$ (the Greek letters phi, psi, and chi) represent statements
that are either true or false, and $x$ is a variable\index{variable!in predicate
calculus} ranging over some group of mathematical objects (sets, integers,
real numbers, etc.). In mathematics, a ``statement'' really means a formula,
and $\psi$ could be for example ``$x = 2$.''
Propositional calculus\index{propositional calculus}
allows us to use variables that are either true or false
and make deductions such as
``if $\varphi$ implies $\psi$ and $\psi$ implies $\chi$, then $\varphi$
implies $\chi$.''
Predicate calculus\index{predicate calculus}
extends propositional calculus by also allowing us
to discuss statements about objects (not just true and false values), including
statements about ``all'' or ``at least one'' object.
For example, predicate calculus allows to say,
``if $\varphi$ is true for all $x$, then $\varphi$ is true for some $x$.''
The logic used in \texttt{set.mm} is standard classical logic
(as opposed to other logic systems like intuitionistic logic).

Set theory\index{set theory} has to do with the manipulation of objects and
collections of objects, specifically the abstract, imaginary objects that
mathematics deals with, such as numbers. Everything that is claimed to exist
in mathematics is considered to be a set.  A set called the empty
set\index{empty set} contains nothing.  We represent the empty set by
$\varnothing$.  Many sets can be built up from the empty set.  There is a set
represented by $\{\varnothing\}$ that contains the empty set, another set
represented by $\{\varnothing,\{\varnothing\}\}$ that contains this set as
well as the empty set, another set represented by $\{\{\varnothing\}\}$ that
contains just the set that contains the empty set, and so on ad infinitum. All
mathematical objects, no matter how complex, are defined as being identical to
certain sets: the integer\index{integer} 0 is defined as the empty set, the
integer 1 is defined as $\{\varnothing\}$, the integer 2 is defined as
$\{\varnothing,\{\varnothing\}\}$.  (How these definitions were chosen doesn't
matter now, but the idea behind it is that these sets have the properties we
expect of integers once suitable operations are defined.)  Mathematical
operations, such as addition, are defined in terms of operations on
sets---their union\index{set union}, intersection\index{set intersection}, and
so on---operations you may have used in elementary school when you worked
with groups of apples and oranges.

With a leap of faith, the axioms also postulate the existence of infinite
sets\index{infinite set}, such as the set of all non-negative integers ($0, 1,
2,\ldots$, also called ``natural numbers''\index{natural number}).  This set
can't be represented with the brace notation\index{brace notation} we just
showed you, but requires a more complicated notation called ``class
abstraction.''\index{class abstraction}\index{abstraction class}  For
example, the infinite set $\{ x |
\mbox{``$x$ is a natural number''} \} $ means the ``set of all objects $x$
such that $x$ is a natural number'' i.e.\ the set of natural numbers; here,
``$x$ is a natural number'' is a rather complicated formula when broken down
into the primitive symbols.\label{expandom}\footnote{The statement ``$x$ is a
natural number'' is formally expressed as ``$x \in \omega$,'' where $\in$
(stylized epsilon) means ``is in'' or ``is an element of'' and $\omega$
(omega) means ``the set of natural numbers.''  When ``$x\in\omega$'' is
completely expanded in terms of the primitive symbols of set theory, the
result is  $\lnot$ $($ $\lnot$ $($ $\forall$ $z$ $($ $\lnot$ $\forall$ $w$ $($
$z$ $\in$ $w$ $\rightarrow$ $\lnot$ $w$ $\in$ $x$ $)$ $\rightarrow$ $z$ $\in$
$x$ $)$ $\rightarrow$ $($ $\forall$ $z$ $($ $\lnot$ $($ $\forall$ $w$ $($ $w$
$\in$ $z$ $\rightarrow$ $w$ $\in$ $x$ $)$ $\rightarrow$ $\forall$ $w$ $\lnot$
$w$ $\in$ $z$ $)$ $\rightarrow$ $\lnot$ $\forall$ $w$ $($ $w$ $\in$ $z$
$\rightarrow$ $\lnot$ $\forall$ $v$ $($ $v$ $\in$ $z$ $\rightarrow$ $\lnot$
$v$ $\in$ $w$ $)$ $)$ $)$ $\rightarrow$ $\lnot$ $\forall$ $z$ $\forall$ $w$
$($ $\lnot$ $($ $z$ $\in$ $x$ $\rightarrow$ $\lnot$ $w$ $\in$ $x$ $)$
$\rightarrow$ $($ $\lnot$ $z$ $\in$ $w$ $\rightarrow$ $($ $\lnot$ $z$ $=$ $w$
$\rightarrow$ $w$ $\in$ $z$ $)$ $)$ $)$ $)$ $)$ $\rightarrow$ $\lnot$
$\forall$ $y$ $($ $\lnot$ $($ $\lnot$ $($ $\forall$ $z$ $($ $\lnot$ $\forall$
$w$ $($ $z$ $\in$ $w$ $\rightarrow$ $\lnot$ $w$ $\in$ $y$ $)$ $\rightarrow$
$z$ $\in$ $y$ $)$ $\rightarrow$ $($ $\forall$ $z$ $($ $\lnot$ $($ $\forall$
$w$ $($ $w$ $\in$ $z$ $\rightarrow$ $w$ $\in$ $y$ $)$ $\rightarrow$ $\forall$
$w$ $\lnot$ $w$ $\in$ $z$ $)$ $\rightarrow$ $\lnot$ $\forall$ $w$ $($ $w$
$\in$ $z$ $\rightarrow$ $\lnot$ $\forall$ $v$ $($ $v$ $\in$ $z$ $\rightarrow$
$\lnot$ $v$ $\in$ $w$ $)$ $)$ $)$ $\rightarrow$ $\lnot$ $\forall$ $z$
$\forall$ $w$ $($ $\lnot$ $($ $z$ $\in$ $y$ $\rightarrow$ $\lnot$ $w$ $\in$
$y$ $)$ $\rightarrow$ $($ $\lnot$ $z$ $\in$ $w$ $\rightarrow$ $($ $\lnot$ $z$
$=$ $w$ $\rightarrow$ $w$ $\in$ $z$ $)$ $)$ $)$ $)$ $\rightarrow$ $($
$\forall$ $z$ $\lnot$ $z$ $\in$ $y$ $\rightarrow$ $\lnot$ $\forall$ $w$ $($
$\lnot$ $($ $w$ $\in$ $y$ $\rightarrow$ $\lnot$ $\forall$ $z$ $($ $w$ $\in$
$z$ $\rightarrow$ $\lnot$ $z$ $\in$ $y$ $)$ $)$ $\rightarrow$ $\lnot$ $($
$\lnot$ $\forall$ $z$ $($ $w$ $\in$ $z$ $\rightarrow$ $\lnot$ $z$ $\in$ $y$
$)$ $\rightarrow$ $w$ $\in$ $y$ $)$ $)$ $)$ $)$ $\rightarrow$ $x$ $\in$ $y$
$)$ $)$ $)$. Section~\ref{hierarchy} shows the hierarchy of definitions that
leads up to this expression.}\index{stylized epsilon ($\in$)}\index{omega
($\omega$)}  Actually, the primitive symbols don't even include the brace
notation.  The brace notation is a high-level definition, which you can find in
Section~\ref{hierarchy}.

Interestingly, the arithmetic of integers\index{integer} and
rationals\index{rational number} can be developed without appealing to the
existence of an infinite set, whereas the arithmetic of real
numbers\index{real number} requires it.

Each variable\index{variable!in set theory} in the axioms of set theory
represents an arbitrary set, and the axioms specify the legal kinds of things
you can do with these variables at a very primitive level.

Now, you may think that numbers and arithmetic are a lot more intuitive and
fundamental than sets and therefore should be the foundation of mathematics.
What is really the case is that you've dealt with numbers all your life and
are comfortable with a few rules for manipulating them such as addition and
multiplication.  Those rules only cover a small portion of what can be done
with numbers and only a very tiny fraction of the rest of mathematics.  If you
look at any elementary book on number theory, you will quickly become lost if
these are the only rules that you know.  Even though such books may present a
list of ``axioms''\index{axiom} for arithmetic, the ability to use the axioms
and to understand proofs of theorems\index{theorem} (facts) about numbers
requires an implicit mathematical talent that frustrates many people
from studying abstract mathematics.  The kind of mathematics that most people
know limits them to the practical, everyday usage of blindly manipulating
numbers and formulas, without any understanding of why those rules are correct
nor any ability to go any further.  For example, do you know why multiplying
two negative numbers yields a positive number?  Starting with set theory, you
will also start off blindly manipulating symbols according to the rules we give
you, but with the advantage that these rules will allow you, in principle, to
access {\em all} of mathematics, not just a tiny part of it.

Of course, concrete examples are often helpful in the learning process. For
example, you can verify that $2\cdot 3=3 \cdot 2$ by actually grouping
objects and can easily ``see'' how it generalizes to $x\cdot y = y\cdot x$,
even though you might not be able to rigorously prove it.  Similarly, in set
theory it can be helpful to understand how the axioms of set theory apply to
(and are correct for) small finite collections of objects.  You should be aware
that in set theory intuition can be misleading for infinite collections, and
rigorous proofs become more important.  For example, while $x\cdot y = y\cdot
x$ is correct for finite ordinals (which are the natural numbers), it is not
usually true for infinite ordinals.

\section{The Axioms for All of Mathematics}

In this section\index{axioms for mathematics}, we will show you the axioms
for all of standard mathematics (i.e.\ logic and set theory) as they are
traditionally presented.  The traditional presentation is useful for someone
with the mathematical experience needed to correctly manipulate high-level
abstract concepts.  For someone without this talent, knowing how to actually
make use of these axioms can be difficult.  The purpose of this section is to
allow you to see how the version of the axioms used in the standard
Metamath\index{Metamath} database \texttt{set.mm}\index{set
theory database (\texttt{set.mm})} relates to  the typical version
in textbooks, and also to give you an informal feel for them.

\subsection{Propositional Calculus}

Propositional calculus\index{propositional calculus} concerns itself with
statements that can be interpreted as either true or false.  Some examples of
statements (outside of mathematics) that are either true or false are ``It is
raining today'' and ``The United States has a female president.'' In
mathematics, as we mentioned, statements are really formulas.

In propositional calculus, we don't care what the statements are.  We also
treat a logical combination of statements, such as ``It is raining today and
the United States has a female president,'' no differently from a single
statement.  Statements and their combinations are called well-formed formulas
(wffs)\index{well-formed formula (wff)}.  We define wffs only in terms of
other wffs and don't define what a ``starting'' wff is.  As is common practice
in the literature, we use Greek letters to represent wffs.

Specifically, suppose $\varphi$ and $\psi$ are wffs.  Then the combinations
$\varphi\rightarrow\psi$ (``$\varphi$ implies $\psi$,'' also read ``if
$\varphi$ then $\psi$'')\index{implication ($\rightarrow$)} and $\lnot\varphi$
(``not $\varphi$'')\index{negation ($\lnot$)} are also wffs.

The three axioms of propositional calculus\index{axioms of propositional
calculus} are all wffs of the following form:\footnote{A remarkable result of
C.~A.~Meredith\index{Meredith, C. A.} squeezes these three axioms into the
single axiom $((((\varphi\rightarrow \psi)\rightarrow(\neg \chi\rightarrow\neg
\theta))\rightarrow \chi )\rightarrow \tau)\rightarrow((\tau\rightarrow
\varphi)\rightarrow(\theta\rightarrow \varphi))$ \cite{CAMeredith},
which is believed to be the shortest possible.}
\begin{center}
     $\varphi\rightarrow(\psi\rightarrow \varphi)$\\

     $(\varphi\rightarrow (\psi\rightarrow \chi))\rightarrow
((\varphi\rightarrow  \psi)\rightarrow (\varphi\rightarrow \chi))$\\

     $(\neg \varphi\rightarrow \neg\psi)\rightarrow (\psi\rightarrow
\varphi)$
\end{center}

These three axioms are widely used.
They are attributed to Jan {\L}ukasiewicz
(pronounced woo-kah-SHAY-vitch) and was popularized by Alonzo Church,
who called it system P2. (Thanks to Ted Ulrich for this information.)

There are an infinite number of axioms, one for each possible
wff\index{well-formed formula (wff)} of the above form.  (For this reason,
axioms such as the above are often called ``axiom schemes.''\index{axiom
scheme})  Each Greek letter in the axioms may be substituted with a more
complex wff to result in another axiom.  For example, substituting
$\neg(\varphi\rightarrow\chi)$ for $\varphi$ in the first axiom yields
$\neg(\varphi\rightarrow\chi)\rightarrow(\psi\rightarrow
\neg(\varphi\rightarrow\chi))$, which is still an axiom.

To deduce new true statements (theorems\index{theorem}) from the axioms, a
rule\index{rule} called ``modus ponens''\index{modus ponens} is used.  This
rule states that if the wff $\varphi$ is an axiom or a theorem, and the wff
$\varphi\rightarrow\psi$ is an axiom or a theorem, then the wff $\psi$ is also
a theorem\index{theorem}.

As a non-mathematical example of modus ponens, suppose we have proved (or
taken as an axiom) ``Bob is a man'' and separately have proved (or taken as
an axiom) ``If Bob is a man, then Bob is a human.''  Using the rule of modus
ponens, we can logically deduce, ``Bob is a human.''

From Metamath's\index{Metamath} point of view, the axioms and the rule of
modus ponens just define a mechanical means for deducing new true statements
from existing true statements, and that is the complete content of
propositional calculus as far as Metamath is concerned.  You can read a logic
textbook to gain a better understanding of their meaning, or you can just let
their meaning slowly become apparent to you after you use them for a while.

It is actually rather easy to check to see if a formula is a theorem of
propositional calculus.  Theorems of propositional calculus are also called
``tautologies.''\index{tautology}  The technique to check whether a formula is
a tautology is called the ``truth table method,''\index{truth table} and it
works like this.  A wff $\varphi\rightarrow\psi$ is false whenever $\varphi$ is true
and $\psi$ is false.  Otherwise it is true.  A wff $\lnot\varphi$ is false
whenever $\varphi$ is true and false otherwise. To verify a tautology such as
$\varphi\rightarrow(\psi\rightarrow \varphi)$, you break it down into sub-wffs and
construct a truth table that accounts for all possible combinations of true
and false assigned to the wff metavariables:
\begin{center}\begin{tabular}{|c|c|c|c|}\hline
\mbox{$\varphi$} & \mbox{$\psi$} & \mbox{$\psi\rightarrow\varphi$}
    & \mbox{$\varphi\rightarrow(\psi\rightarrow \varphi)$} \\ \hline \hline
              T   &  T    &      T       &        T    \\ \hline
              T   &  F    &      T       &        T    \\ \hline
              F   &  T    &      F       &        T    \\ \hline
              F   &  F    &      T       &        T    \\ \hline
\end{tabular}\end{center}
If all entries in the last column are true, the formula is a tautology.

Now, the truth table method doesn't tell you how to prove the tautology from
the axioms, but only that a proof exists.  Finding an actual proof (especially
one that is short and elegant) can be challenging.  Methods do exist for
automatically generating proofs in propositional calculus, but the proofs that
result can sometimes be very long.  In the Metamath \texttt{set.mm}\index{set
theory database (\texttt{set.mm})} database, most
or all proofs were created manually.

Section \ref{metadefprop} discusses various definitions
that make propositional calculus easier to use.
For example, we define:

\begin{itemize}
\item $\varphi \vee \psi$
  is true if either $\varphi$ or $\psi$ (or both) are true
  (this is disjunction\index{disjunction ($\vee$)}
  aka logical {\sc or}\index{logical {\sc or} ($\vee$)}).

\item $\varphi \wedge \psi$
  is true if both $\varphi$ and $\psi$ are true
  (this is conjunction\index{conjunction ($\wedge$)}
  aka logical {\sc and}\index{logical {\sc and} ($\wedge$)}).

\item $\varphi \leftrightarrow \psi$
  is true if $\varphi$ and $\psi$ have the same value, that is,
  they are both true or both false
  (this is the biconditional\index{biconditional ($\leftrightarrow$)}).
\end{itemize}

\subsection{Predicate Calculus}

Predicate calculus\index{predicate calculus} introduces the concept of
``individual variables,''\index{variable!in predicate calculus}\index{individual
variable} which
we will usually just call ``variables.''
These variables can represent something other than true or false (wffs),
and will always represent sets when we get to set theory.  There are also
three new symbols $\forall$\index{universal quantifier ($\forall$)},
$=$\index{equality ($=$)}, and $\in$\index{stylized epsilon ($\in$)},
read ``for all,'' ``equals,'' and ``is an element of''
respectively.  We will represent variables with the letters $x$, $y$, $z$, and
$w$, as is common practice in the literature.
For example, $\forall x \varphi$ means ``for all possible values of
$x$, $\varphi$ is true.''

In predicate calculus, we extend the definition of a wff\index{well-formed
formula (wff)}.  If $\varphi$ is a wff and $x$ and $y$ are variables, then
$\forall x \, \varphi$, $x=y$, and $x\in y$ are wffs. Note that these three new
types of wffs can be considered ``starting'' wffs from which we can build
other wffs with $\rightarrow$ and $\neg$ .  The concept of a starting wff was
absent in propositional calculus.  But starting wff or not, all we are really
concerned with is whether our wffs are correctly constructed according to
these mechanical rules.

A quick aside:
To prevent confusion, it might be best at this point to think of the variables
of Metamath\index{Metamath} as ``metavariables,''\index{metavariable} because
they are not quite the same as the variables we are introducing here.  A
(meta)variable in Metamath can be a wff or an individual variable, as well
as many other things; in general, it represents a kind of place holder for an
unspecified sequence of math symbols\index{math symbol}.

Unlike propositional calculus, no decision procedure\index{decision procedure}
analogous to the truth table method exists (nor theoretically can exist) that
will definitely determine whether a formula is a theorem of predicate
calculus.  Much of the work in the field of automated theorem
proving\index{automated theorem proving} has been dedicated to coming up with
clever heuristics for proving theorems of predicate calculus, but they can
never be guaranteed to work always.

Section \ref{metadefpred} discusses various definitions
that make predicate calculus easier to use.
For example, we define
$\exists x \varphi$ to mean
``there exists at least one possible value of $x$ where $\varphi$ is true.''

We now turn to looking at how predicate calculus can be formally
represented.

\subsubsection{Common Axioms}

There is a new rule of inference in predicate calculus:  if $\varphi$ is
an axiom or a theorem, then $\forall x \,\varphi$ is also a
theorem\index{theorem}.  This is called the rule of
``generalization.''\index{rule of generalization}
This is easily represented in Metamath.

In standard texts of logic, there are often two axioms of predicate
calculus\index{axioms of predicate calculus}:
\begin{center}
  $\forall x \,\varphi ( x ) \rightarrow \varphi ( y )$,
      where ``$y$ is properly substituted for $x$.''\\
  $\forall x ( \varphi \rightarrow \psi )\rightarrow ( \varphi \rightarrow
    \forall x\, \psi )$,
    where ``$x$ is not free in $\varphi$.''
\end{center}

Now at first glance, this seems simple:  just two axioms.  However,
conditional clauses are attached to each axiom describing requirements that
may seem puzzling to you.  In addition, the first axiom puts a variable symbol
in parentheses after each wff, seemingly violating our definition of a
wff\index{well-formed formula (wff)}; this is just an informal way of
referring to some arbitrary variable that may occur in the wff.  The
conditional clauses do, of course, have a precise meaning, but as it turns out
the precise meaning is somewhat complicated and awkward to formalize in a
way that a computer can handle easily.  Unlike propositional calculus, a
certain amount of mathematical sophistication and practice is needed to be
able to easily grasp and manipulate these concepts correctly.

Predicate calculus may be presented with or without axioms for
equality\index{axioms of equality}\index{equality ($=$)}. We will require the
axioms of equality as a prerequisite for the version of set theory we will
use.  The axioms for equality, when included, are often represented using these
two axioms:
\begin{center}
$x=x$\\ \ \\
$x=y\rightarrow (\varphi(x,x)\rightarrow\varphi(x,y))$ where ``$\varphi(x,y)$
   arises from $\varphi(x,x)$ by replacing some, but not necessarily all,
   free\index{free variable}
   occurrences of $x$ by $y$,\\ provided that $y$ is free for $x$
   in $\varphi(x,x)$.'' \end{center}
% (Mendelson p. 95)
The first equality axiom is simple, but again,
the condition on the second one is
somewhat awkward to implement on a computer.

\subsubsection{Tarski System S2}

Of course, we are not the first to notice the complications of these
predicate calculus axioms when being rigorous.

Well-known logician Alfred Tarski published in 1965
a system he called system S2\cite[p.~77]{Tarski1965}.
Tarski's system is \textit{exactly equivalent} to the traditional textbook
formalization, but (by clever use of equality axioms) it eliminates the
latter's primitive notions of ``proper substitution'' and ``free variable,''
replacing them with direct substitution and the notion of a variable
not occurring in a formula (which we express with distinct variable
constraints).

In advocating his system, Tarski wrote, ``The relatively complicated
character of [free variables and proper substitution] is a source
of certain inconveniences of both practical and theoretical nature;
this is clearly experienced both in teaching an elementary course of
mathematical logic and in formalizing the syntax of predicate logic for
some theoretical purposes''\cite[p.~61]{Tarski1965}\index{Tarski, Alfred}.

\subsubsection{Developing a Metamath Representation}

The standard textbook axioms of predicate calculus are somewhat
cumbersome to implement on a computer because of the complex notions of
``free variable''\index{free variable} and ``proper
substitution.''\index{proper substitution}\index{substitution!proper}
While it is possible to use the Metamath\index{Metamath} language to
implement these concepts, we have chosen not to implement them
as primitive constructs in the
\texttt{set.mm} set theory database.  Instead, we have eliminated them
within the axioms
by carefully crafting the axioms so as to avoid them,
building on Tarski's system S2.  This makes it
easy for a beginner to follow the steps in a proof without knowing any
advanced concepts other than the simple concept of
replacing\index{substitution!variable}\index{variable substitution}
variables with expressions.

In order to develop the concepts of free variable and proper
substitution from the axioms, we use an additional
Metamath statement type called ``disjoint variable
restriction''\index{disjoint variables} that we have not encountered
before.  In the context of the axioms, the statement \texttt{\$d} $ x\,
y$\index{\texttt{\$d} statement} simply means that $x$ and $y$ must be
distinct\index{distinct variables}, i.e.\ they may not be simultaneously
substituted\index{substitution!variable}\index{variable substitution}
with the same variable.  The statement \texttt{\$d} $ x\, \varphi$ means
variable $x$ must not occur in wff $\varphi$.  For the precise
definition of \texttt{\$d}, see Section~\ref{dollard}.

\subsubsection{Metamath representation}

The Metamath axiom system for predicate calculus
defined in set.mm uses Tarski's system S2.
As noted above, this has a different representation
than the traditional textbook formalization,
but it is \textit{exactly equivalent} to the textbook formalization,
and it is \textit{much} easier to work with.
This is reproduced as system S3 in Section 6 of
Megill's formalization \cite{Megill}\index{Megill, Norman}.

There is one exception, Tarski's axiom of existence,
which we label as axiom ax-6.
In the case of ax-6, Tarski's version is weaker because it includes a
distinct variable proviso. If we wish, we can also weaken our version
in this way and still have a metalogically complete system. Theorem
ax6 shows this by deriving, in the presence of the other axioms, our
ax-6 from Tarski's weaker version ax6v. However, we chose the stronger
version for our system because it is simpler to state and easier to use.

Tarski's system was designed for proving specific theorems rather than
more general theorem schemes. However, theorem schemes are much more
efficient than specific theorems for building a body of mathematical
knowledge, since they can be reused with different instances as
needed. While Tarski does derive some theorem schemes from his axioms,
their proofs require concepts that are ``outside'' of the system, such as
induction on formula length. The verification of such proofs is difficult
to automate in a proof verifier. (Specifically, Tarski treats the formulas
of his system as set-theoretical objects. In order to verify the proofs
of his theorem schemes, a proof verifier would need a significant amount
of set theory built into it.)

The Metamath axiom system for predicate calculus extends
Tarski's system to eliminate this difficulty. The additional
``auxilliary'' axiom
schemes (as we will call them in this section; see below) endow Tarski's
system with a nice property we call
metalogical completeness \cite[Remark 9.6]{Megill}\index{Megill, Norman}.
As a result, we can prove any theorem scheme
expressable in the ``simple metalogic'' of Tarski's system by using
only Metamath's direct substitution rule applied to the axiom system
(and no other metalogical or set-theoretical notions ``outside'' of the
system). Simple metalogic consists of schemes containing wff metavariables
(with no arguments) and/or set (also called ``individual'') metavariables,
accompanied by optional provisos each stating that two specified set
metavariables must be distinct or that a specified set metavariable may
not occur in a specified wff metavariable. Metamath's logic and set theory
axiom and rule schemes are all examples of simple metalogic. The schemes
of traditional predicate calculus with equality are examples which are
not simple metalogic, because they use wff metavariables with arguments
and have ``free for'' and ``not free in'' side conditions.

A rigorous justification for this system, using an older but
exactly equivalent set of axioms, can be
found in \cite{Megill}\index{Megill, Norman}.

This allows us to
take a different approach in the Metamath\index{Metamath} database
\texttt{set.mm}\index{set theory database (\texttt{set.mm})}.  We do not
directly use the primitive notions of ``free variable''\index{free variable}
and ``proper substitution''\index{proper
substitution}\index{substitution!proper} at all as primitive constructs.
Instead, we use a set
of axioms that are almost as simple to manipulate as those of
propositional calculus.  Our axiom system avoids complex primitive
notions by effectively embedding the complexity into the axioms
themselves.  As a result, we will end up with a larger number of axioms,
but they are ideally suited for a computer language such as Metamath.
(Section~\ref{metaaxioms} shows these axioms.)

We will not elaborate further
on the ``free variable'' and ``proper substitution''
concepts here.  You may consult
\cite[ch.\ 3--4]{Hamilton}\index{Hamilton, Alan G.} (as well as
many other books) for a precise explanation
of these concepts.  If you intend to do serious mathematical work, it is wise
to become familiar with the traditional textbook approach; even though the
concepts embedded in their axioms require a higher level of sophistication,
they can be more practical to deal with on an everyday, informal basis.  Even
if you are just developing Metamath proofs, familiarity with the traditional
approach can help you arrive at a proof outline much faster, which you can
then convert to the detail required by Metamath.

We do develop proper substitution rules later on, but in set.mm
they are defined as derived constructs; they are not primitives.

You should also note that our system of predicate calculus is specifically
tailored for set theory; thus there are only two specific predicates $=$ and
$\in$ and no functions\index{function!in predicate calculus}
or constants\index{constant!in predicate calculus} unlike more general systems.
We later add these.

\subsection{Set Theory}

Traditional Zermelo--Fraenkel set theory\index{Zermelo--Fraenkel set
theory}\index{set theory} with the Axiom of Choice
has 10 axioms, which can be expressed in the
language of predicate calculus.  In this section, we will list only the
names and brief English descriptions of these axioms, since we will give
you the precise formulas used by the Metamath\index{Metamath} set theory
database \texttt{set.mm} later on.

In the descriptions of the axioms, we assume that $x$, $y$, $z$, $w$, and $v$
represent sets.  These are the same as the variables\index{variable!in set
theory} in our predicate calculus system above, except that now we informally
think of the variables as ranging over sets.  Note that the terms
``object,''\index{object} ``set,''\index{set} ``element,''\index{element}
``collection,''\index{collection} and ``family''\index{family} are synonymous,
as are ``is an element of,'' ``is a member of,''\index{member} ``is contained
in,'' and ``belongs to.''  The different terms are used for convenience; for
example, ``a collection of sets'' is less confusing than ``a set of sets.''
A set $x$ is said to be a ``subset''\index{subset} of $y$ if every element of
$x$ is also an element of $y$; we also say $x$ is ``included in''
$y$.

The axioms are very general and apply to almost any conceivable mathematical
object, and this level of abstraction can be overwhelming at first.  To gain an
intuitive feel, it can be helpful to draw a picture illustrating the concept;
for example, a circle containing dots could represent a collection of sets,
and a smaller circle drawn inside the circle could represent a subset.
Overlapping circles can illustrate intersection and union.  Circles that
illustrate the concepts of set theory are frequently used in elementary
textbooks and are called Venn diagrams\index{Venn diagram}.\index{axioms of
set theory}

1. Axiom of Extensionality:  Two sets are identical if they contain the same
   elements.\index{Axiom of Extensionality}

2. Axiom of Pairing:  The set $\{ x , y \}$ exists.\index{Axiom of Pairing}

3. Axiom of Power Sets:  The power set of a set (the collection of all of
   its subsets) exists.  For example, the power set of $\{x,y\}$ is
   $\{\varnothing,\{x\},\{y\},\{x,y\}\}$ and it exists.\index{Axiom
of Power Sets}

4. Axiom of the Null Set:  The empty set $\varnothing$ exists.\index{Axiom of
the Null Set}

5. Axiom of Union:  The union of a set (the set containing the elements of
   its members) exists.  For example, the union of $\{\{x,y\},\{z\}\}$ is
 $\{x,y,z\}$ and
   it exists.\index{Axiom of Union}

6. Axiom of Regularity:  Roughly, no set can contain itself, nor can there
   be membership ``loops,'' such as a set being an
   element of one of its members.\index{Axiom of Regularity}

7. Axiom of Infinity:  An infinite set exists.  An example of an infinite
   set is the set of all
   integers.\index{Axiom of Infinity}

8. Axiom of Separation:  The set exists that is obtained by restricting $x$
   with some property.  For example, if the set of all integers exists,
   then the set of all even integers exists.\index{Axiom of Separation}

9. Axiom of Replacement:  The range of a function whose domain is restricted
   to the elements of a set $x$, is also a set.  For example, there
   is a function
   from integers (the function's domain) to their squares (its
   range).  If we
   restrict the domain to even integers, its range will become the set of
   squares of even integers, so this axiom asserts that the set of
    squares of even numbers exists.  Technical note:  In general, the
   ``function'' need not be a set but can be a proper class.
   \index{Axiom of Replacement}

10. Axiom of Choice:  Let $x$ be a set whose members are pairwise
  disjoint\index{disjoint sets} (i.e,
  whose members contain no elements in common).  Then there exists another
  set containing one element from each member of $x$.  For
  example, if $x$ is
  $\{\{y,z\},\{w,v\}\}$, where $y$, $z$, $w$, and $v$ are
  different sets, then a set such as $\{z,w\}$
  exists (but the axiom doesn't tell
  us which one).  (Actually the Axiom
  of Choice is redundant if the set $x$, as in this example, has a finite
  number of elements.)\index{Axiom of Choice}

The Axiom of Choice is usually considered an extension of ZF set theory rather
than a proper part of it.  It is sometimes considered philosophically
controversial because it specifies the existence of a set without specifying
what the set is. Constructive logics, including intuitionistic logic,
do not accept the axiom of choice.
Since there is some lingering controversy, we often prefer proofs that do
not use the axiom of choice (where there is a known alternative), and
in some cases we will use weaker axioms than the full axiom of choice.
That said, the axiom of choice is a powerful and widely-accepted tool,
so we do use it when needed.
ZF set theory that includes the Axiom of Choice is
called Zermelo--Fraenkel set theory with choice (ZFC\index{ZFC set theory}).

When expressed symbolically, the Axiom of Separation and the Axiom of
Replacement contain wff symbols and therefore each represent infinitely many
axioms, one for each possible wff. For this reason, they are often called
axiom schemes\index{axiom scheme}\index{well-formed formula (wff)}.

It turns out that the Axiom of the Null Set, the Axiom of Pairing, and the
Axiom of Separation can be derived from the other axioms and are therefore
unnecessary, although they tend to be included in standard texts for various
reasons (historical, philosophical, and possibly because some authors may not
know this).  In the Metamath\index{Metamath} set theory database, these
redundant axioms are derived from the other ones instead of truly
being considered axioms.
This is in keeping with our general goal of minimizing the number of
axioms we must depend on.

\subsection{Other Axioms}

Above we qualified the phrase "all of mathematics" with "essentially."
The main important missing piece is the ability to do category theory,
which requires huge sets (inaccessible cardinals) larger than those
postulated by the ZFC axioms. The Tarski--Grothendieck Axiom postulates
the existence of such sets.
Note that this is the same axiom used by Mizar for supporting
category theory.
The Tarski--Grothendieck axiom
can be viewed as a very strong replacement of the Axiom of Infinity,
the Axiom of Choice, and the Axiom of Power Sets.
The \texttt{set.mm} database includes this axiom; see the database
for details about it.
Again, we only use this axiom when we need to.
You are only likely to encounter or use this axiom if you are doing
category theory, since its use is highly specialized,
so we will not list the Tarsky-Grothendieck axiom
in the short list of axioms below.

Can there be even more axioms?
Of course.
G\"{o}del showed that no finite set of axioms or axiom schemes can completely
describe any consistent theory strong enough to include arithmetic.
But practically speaking, the ones above are the accepted foundation that
almost all mathematicians explicitly or implicitly base their work on.

\section{The Axioms in the Metamath Language}\label{metaaxioms}

Here we list the axioms as they appear in
\texttt{set.mm}\index{set theory database (\texttt{set.mm})} so you can
look them up there easily.  Incidentally, the \texttt{show statement
/tex} command\index{\texttt{show statement} command} was used to
typeset them.

%macros from show statement /tex
\newbox\mlinebox
\newbox\mtrialbox
\newbox\startprefix  % Prefix for first line of a formula
\newbox\contprefix  % Prefix for continuation line of a formula
\def\startm{  % Initialize formula line
  \setbox\mlinebox=\hbox{\unhcopy\startprefix}
}
\def\m#1{  % Add a symbol to the formula
  \setbox\mtrialbox=\hbox{\unhcopy\mlinebox $\,#1$}
  \ifdim\wd\mtrialbox>\hsize
    \box\mlinebox
    \setbox\mlinebox=\hbox{\unhcopy\contprefix $\,#1$}
  \else
    \setbox\mlinebox=\hbox{\unhbox\mtrialbox}
  \fi
}
\def\endm{  % Output the last line of a formula
  \box\mlinebox
}

% \SLASH for \ , \TOR for \/ (text OR), \TAND for /\ (text and)
% This embeds a following forced space to force the space.
\newcommand\SLASH{\char`\\~}
\newcommand\TOR{\char`\\/~}
\newcommand\TAND{/\char`\\~}
%
% Macro to output metamath raw text.
% This assumes \startprefix and \contprefix are set.
% NOTE: "\" is tricky to escape, use \SLASH, \TOR, and \TAND inside.
% Any use of "$ { ~ ^" must be escaped; ~ and ^ must be escaped specially.
% We escape { and } for consistency.
% For more about how this macro written, see:
% https://stackoverflow.com/questions/4073674/
% how-to-disable-indentation-in-particular-section-in-latex/4075706
% Use frenchspacing, or "e." will get an extra space after it.
\newlength\mystoreparindent
\newlength\mystorehangindent
\newenvironment{mmraw}{%
\setlength{\mystoreparindent}{\the\parindent}
\setlength{\mystorehangindent}{\the\hangindent}
\setlength{\parindent}{0pt} % TODO - we'll put in the \startprefix instead
\setlength{\hangindent}{\wd\the\contprefix}
\begin{flushleft}
\begin{frenchspacing}
\begin{tt}
{\unhcopy\startprefix}%
}{%
\end{tt}
\end{frenchspacing}
\end{flushleft}
\setlength{\parindent}{\mystoreparindent}
\setlength{\hangindent}{\mystorehangindent}
\vskip 1ex
}

\needspace{5\baselineskip}
\subsection{Propositional Calculus}\label{propcalc}\index{axioms of
propositional calculus}

\needspace{2\baselineskip}
Axiom of Simplification.\label{ax1}

\setbox\startprefix=\hbox{\tt \ \ ax-1\ \$a\ }
\setbox\contprefix=\hbox{\tt \ \ \ \ \ \ \ \ \ \ }
\startm
\m{\vdash}\m{(}\m{\varphi}\m{\rightarrow}\m{(}\m{\psi}\m{\rightarrow}\m{\varphi}\m{)}
\m{)}
\endm

\needspace{3\baselineskip}
\noindent Axiom of Distribution.

\setbox\startprefix=\hbox{\tt \ \ ax-2\ \$a\ }
\setbox\contprefix=\hbox{\tt \ \ \ \ \ \ \ \ \ \ }
\startm
\m{\vdash}\m{(}\m{(}\m{\varphi}\m{\rightarrow}\m{(}\m{\psi}\m{\rightarrow}\m{\chi}
\m{)}\m{)}\m{\rightarrow}\m{(}\m{(}\m{\varphi}\m{\rightarrow}\m{\psi}\m{)}\m{
\rightarrow}\m{(}\m{\varphi}\m{\rightarrow}\m{\chi}\m{)}\m{)}\m{)}
\endm

\needspace{2\baselineskip}
\noindent Axiom of Contraposition.

\setbox\startprefix=\hbox{\tt \ \ ax-3\ \$a\ }
\setbox\contprefix=\hbox{\tt \ \ \ \ \ \ \ \ \ \ }
\startm
\m{\vdash}\m{(}\m{(}\m{\lnot}\m{\varphi}\m{\rightarrow}\m{\lnot}\m{\psi}\m{)}\m{
\rightarrow}\m{(}\m{\psi}\m{\rightarrow}\m{\varphi}\m{)}\m{)}
\endm


\needspace{4\baselineskip}
\noindent Rule of Modus Ponens.\label{axmp}\index{modus ponens}

\setbox\startprefix=\hbox{\tt \ \ maj\ \$e\ }
\setbox\contprefix=\hbox{\tt \ \ \ \ \ \ \ \ \ }
\startm
\m{\vdash}\m{(}\m{\varphi}\m{\rightarrow}\m{\psi}\m{)}
\endm

\setbox\startprefix=\hbox{\tt \ \ min\ \$e\ }
\setbox\contprefix=\hbox{\tt \ \ \ \ \ \ \ \ \ }
\startm
\m{\vdash}\m{\varphi}
\endm

\setbox\startprefix=\hbox{\tt \ \ ax-mp\ \$a\ }
\setbox\contprefix=\hbox{\tt \ \ \ \ \ \ \ \ \ \ \ }
\startm
\m{\vdash}\m{\psi}
\endm


\needspace{7\baselineskip}
\subsection{Axioms of Predicate Calculus with Equality---Tarski's S2}\index{axioms of predicate calculus}

\needspace{3\baselineskip}
\noindent Rule of Generalization.\index{rule of generalization}

\setbox\startprefix=\hbox{\tt \ \ ax-g.1\ \$e\ }
\setbox\contprefix=\hbox{\tt \ \ \ \ \ \ \ \ \ \ \ \ }
\startm
\m{\vdash}\m{\varphi}
\endm

\setbox\startprefix=\hbox{\tt \ \ ax-gen\ \$a\ }
\setbox\contprefix=\hbox{\tt \ \ \ \ \ \ \ \ \ \ \ \ }
\startm
\m{\vdash}\m{\forall}\m{x}\m{\varphi}
\endm

\needspace{2\baselineskip}
\noindent Axiom of Quantified Implication.

\setbox\startprefix=\hbox{\tt \ \ ax-4\ \$a\ }
\setbox\contprefix=\hbox{\tt \ \ \ \ \ \ \ \ \ \ }
\startm
\m{\vdash}\m{(}\m{\forall}\m{x}\m{(}\m{\forall}\m{x}\m{\varphi}\m{\rightarrow}\m{
\psi}\m{)}\m{\rightarrow}\m{(}\m{\forall}\m{x}\m{\varphi}\m{\rightarrow}\m{
\forall}\m{x}\m{\psi}\m{)}\m{)}
\endm

\needspace{3\baselineskip}
\noindent Axiom of Distinctness.

% Aka: Add $d x ph $.
\setbox\startprefix=\hbox{\tt \ \ ax-5\ \$a\ }
\setbox\contprefix=\hbox{\tt \ \ \ \ \ \ \ \ \ \ }
\startm
\m{\vdash}\m{(}\m{\varphi}\m{\rightarrow}\m{\forall}\m{x}\m{\varphi}\m{)}\m{where}\m{ }\m{\$d}\m{ }\m{x}\m{ }\m{\varphi}\m{ }\m{(}\m{x}\m{ }\m{does}\m{ }\m{not}\m{ }\m{occur}\m{ }\m{in}\m{ }\m{\varphi}\m{)}
\endm

\needspace{2\baselineskip}
\noindent Axiom of Existence.

\setbox\startprefix=\hbox{\tt \ \ ax-6\ \$a\ }
\setbox\contprefix=\hbox{\tt \ \ \ \ \ \ \ \ \ \ }
\startm
\m{\vdash}\m{(}\m{\forall}\m{x}\m{(}\m{x}\m{=}\m{y}\m{\rightarrow}\m{\forall}
\m{x}\m{\varphi}\m{)}\m{\rightarrow}\m{\varphi}\m{)}
\endm

\needspace{2\baselineskip}
\noindent Axiom of Equality.

\setbox\startprefix=\hbox{\tt \ \ ax-7\ \$a\ }
\setbox\contprefix=\hbox{\tt \ \ \ \ \ \ \ \ \ \ }
\startm
\m{\vdash}\m{(}\m{x}\m{=}\m{y}\m{\rightarrow}\m{(}\m{x}\m{=}\m{z}\m{
\rightarrow}\m{y}\m{=}\m{z}\m{)}\m{)}
\endm

\needspace{2\baselineskip}
\noindent Axiom of Left Equality for Binary Predicate.

\setbox\startprefix=\hbox{\tt \ \ ax-8\ \$a\ }
\setbox\contprefix=\hbox{\tt \ \ \ \ \ \ \ \ \ \ \ }
\startm
\m{\vdash}\m{(}\m{x}\m{=}\m{y}\m{\rightarrow}\m{(}\m{x}\m{\in}\m{z}\m{
\rightarrow}\m{y}\m{\in}\m{z}\m{)}\m{)}
\endm

\needspace{2\baselineskip}
\noindent Axiom of Right Equality for Binary Predicate.

\setbox\startprefix=\hbox{\tt \ \ ax-9\ \$a\ }
\setbox\contprefix=\hbox{\tt \ \ \ \ \ \ \ \ \ \ \ }
\startm
\m{\vdash}\m{(}\m{x}\m{=}\m{y}\m{\rightarrow}\m{(}\m{z}\m{\in}\m{x}\m{
\rightarrow}\m{z}\m{\in}\m{y}\m{)}\m{)}
\endm


\needspace{4\baselineskip}
\subsection{Axioms of Predicate Calculus with Equality---Auxiliary}\index{axioms of predicate calculus - auxiliary}

\needspace{2\baselineskip}
\noindent Axiom of Quantified Negation.

\setbox\startprefix=\hbox{\tt \ \ ax-10\ \$a\ }
\setbox\contprefix=\hbox{\tt \ \ \ \ \ \ \ \ \ \ }
\startm
\m{\vdash}\m{(}\m{\lnot}\m{\forall}\m{x}\m{\lnot}\m{\forall}\m{x}\m{\varphi}\m{
\rightarrow}\m{\varphi}\m{)}
\endm

\needspace{2\baselineskip}
\noindent Axiom of Quantifier Commutation.

\setbox\startprefix=\hbox{\tt \ \ ax-11\ \$a\ }
\setbox\contprefix=\hbox{\tt \ \ \ \ \ \ \ \ \ \ }
\startm
\m{\vdash}\m{(}\m{\forall}\m{x}\m{\forall}\m{y}\m{\varphi}\m{\rightarrow}\m{
\forall}\m{y}\m{\forall}\m{x}\m{\varphi}\m{)}
\endm

\needspace{3\baselineskip}
\noindent Axiom of Substitution.

\setbox\startprefix=\hbox{\tt \ \ ax-12\ \$a\ }
\setbox\contprefix=\hbox{\tt \ \ \ \ \ \ \ \ \ \ \ }
\startm
\m{\vdash}\m{(}\m{\lnot}\m{\forall}\m{x}\m{\,x}\m{=}\m{y}\m{\rightarrow}\m{(}
\m{x}\m{=}\m{y}\m{\rightarrow}\m{(}\m{\varphi}\m{\rightarrow}\m{\forall}\m{x}\m{(}
\m{x}\m{=}\m{y}\m{\rightarrow}\m{\varphi}\m{)}\m{)}\m{)}\m{)}
\endm

\needspace{3\baselineskip}
\noindent Axiom of Quantified Equality.

\setbox\startprefix=\hbox{\tt \ \ ax-13\ \$a\ }
\setbox\contprefix=\hbox{\tt \ \ \ \ \ \ \ \ \ \ \ }
\startm
\m{\vdash}\m{(}\m{\lnot}\m{\forall}\m{z}\m{\,z}\m{=}\m{x}\m{\rightarrow}\m{(}
\m{\lnot}\m{\forall}\m{z}\m{\,z}\m{=}\m{y}\m{\rightarrow}\m{(}\m{x}\m{=}\m{y}
\m{\rightarrow}\m{\forall}\m{z}\m{\,x}\m{=}\m{y}\m{)}\m{)}\m{)}
\endm

% \noindent Axiom of Quantifier Substitution
%
% \setbox\startprefix=\hbox{\tt \ \ ax-c11n\ \$a\ }
% \setbox\contprefix=\hbox{\tt \ \ \ \ \ \ \ \ \ \ \ }
% \startm
% \m{\vdash}\m{(}\m{\forall}\m{x}\m{\,x}\m{=}\m{y}\m{\rightarrow}\m{(}\m{\forall}
% \m{x}\m{\varphi}\m{\rightarrow}\m{\forall}\m{y}\m{\varphi}\m{)}\m{)}
% \endm
%
% \noindent Axiom of Distinct Variables. (This axiom requires
% that two individual variables
% be distinct\index{\texttt{\$d} statement}\index{distinct
% variables}.)
%
% \setbox\startprefix=\hbox{\tt \ \ \ \ \ \ \ \ \$d\ }
% \setbox\contprefix=\hbox{\tt \ \ \ \ \ \ \ \ \ \ \ }
% \startm
% \m{x}\m{\,}\m{y}
% \endm
%
% \setbox\startprefix=\hbox{\tt \ \ ax-c16\ \$a\ }
% \setbox\contprefix=\hbox{\tt \ \ \ \ \ \ \ \ \ \ \ }
% \startm
% \m{\vdash}\m{(}\m{\forall}\m{x}\m{\,x}\m{=}\m{y}\m{\rightarrow}\m{(}\m{\varphi}\m{
% \rightarrow}\m{\forall}\m{x}\m{\varphi}\m{)}\m{)}
% \endm

% \noindent Axiom of Quantifier Introduction (2).  (This axiom requires
% that the individual variable not occur in the
% wff\index{\texttt{\$d} statement}\index{distinct variables}.)
%
% \setbox\startprefix=\hbox{\tt \ \ \ \ \ \ \ \ \$d\ }
% \setbox\contprefix=\hbox{\tt \ \ \ \ \ \ \ \ \ \ \ }
% \startm
% \m{x}\m{\,}\m{\varphi}
% \endm
% \setbox\startprefix=\hbox{\tt \ \ ax-5\ \$a\ }
% \setbox\contprefix=\hbox{\tt \ \ \ \ \ \ \ \ \ \ \ }
% \startm
% \m{\vdash}\m{(}\m{\varphi}\m{\rightarrow}\m{\forall}\m{x}\m{\varphi}\m{)}
% \endm

\subsection{Set Theory}\label{mmsettheoryaxioms}

In order to make the axioms of set theory\index{axioms of set theory} a little
more compact, there are several definitions from logic that we make use of
implicitly, namely, ``logical {\sc and},''\index{conjunction ($\wedge$)}
\index{logical {\sc and} ($\wedge$)} ``logical equivalence,''\index{logical
equivalence ($\leftrightarrow$)}\index{biconditional ($\leftrightarrow$)} and
``there exists.''\index{existential quantifier ($\exists$)}

\begin{center}\begin{tabular}{rcl}
  $( \varphi \wedge \psi )$ &\mbox{stands for}& $\neg ( \varphi
     \rightarrow \neg \psi )$\\
  $( \varphi \leftrightarrow \psi )$& \mbox{stands
     for}& $( ( \varphi \rightarrow \psi ) \wedge
     ( \psi \rightarrow \varphi ) )$\\
  $\exists x \,\varphi$ &\mbox{stands for}& $\neg \forall x \neg \varphi$
\end{tabular}\end{center}

In addition, the axioms of set theory require that all variables be
dis\-tinct,\index{distinct variables}\footnote{Set theory axioms can be
devised so that {\em no} variables are required to be distinct,
provided we replace \texttt{ax-c16} with an axiom stating that ``at
least two things exist,'' thus
making \texttt{ax-5} the only other axiom requiring the
\texttt{\$d} statement.  These axioms are unconventional and are not
presented here, but they can be found on the \url{http://metamath.org}
web site.  See also the Comment on
p.~\pageref{nodd}.}\index{\texttt{\$d} statement} thus we also assume:
\begin{center}
  \texttt{\$d }$x\,y\,z\,w$
\end{center}

\needspace{2\baselineskip}
\noindent Axiom of Extensionality.\index{Axiom of Extensionality}

\setbox\startprefix=\hbox{\tt \ \ ax-ext\ \$a\ }
\setbox\contprefix=\hbox{\tt \ \ \ \ \ \ \ \ \ \ \ \ }
\startm
\m{\vdash}\m{(}\m{\forall}\m{x}\m{(}\m{x}\m{\in}\m{y}\m{\leftrightarrow}\m{x}
\m{\in}\m{z}\m{)}\m{\rightarrow}\m{y}\m{=}\m{z}\m{)}
\endm

\needspace{3\baselineskip}
\noindent Axiom of Replacement.\index{Axiom of Replacement}

\setbox\startprefix=\hbox{\tt \ \ ax-rep\ \$a\ }
\setbox\contprefix=\hbox{\tt \ \ \ \ \ \ \ \ \ \ \ \ }
\startm
\m{\vdash}\m{(}\m{\forall}\m{w}\m{\exists}\m{y}\m{\forall}\m{z}\m{(}\m{%
\forall}\m{y}\m{\varphi}\m{\rightarrow}\m{z}\m{=}\m{y}\m{)}\m{\rightarrow}\m{%
\exists}\m{y}\m{\forall}\m{z}\m{(}\m{z}\m{\in}\m{y}\m{\leftrightarrow}\m{%
\exists}\m{w}\m{(}\m{w}\m{\in}\m{x}\m{\wedge}\m{\forall}\m{y}\m{\varphi}\m{)}%
\m{)}\m{)}
\endm

\needspace{2\baselineskip}
\noindent Axiom of Union.\index{Axiom of Union}

\setbox\startprefix=\hbox{\tt \ \ ax-un\ \$a\ }
\setbox\contprefix=\hbox{\tt \ \ \ \ \ \ \ \ \ \ \ }
\startm
\m{\vdash}\m{\exists}\m{x}\m{\forall}\m{y}\m{(}\m{\exists}\m{x}\m{(}\m{y}\m{
\in}\m{x}\m{\wedge}\m{x}\m{\in}\m{z}\m{)}\m{\rightarrow}\m{y}\m{\in}\m{x}\m{)}
\endm

\needspace{2\baselineskip}
\noindent Axiom of Power Sets.\index{Axiom of Power Sets}

\setbox\startprefix=\hbox{\tt \ \ ax-pow\ \$a\ }
\setbox\contprefix=\hbox{\tt \ \ \ \ \ \ \ \ \ \ \ \ }
\startm
\m{\vdash}\m{\exists}\m{x}\m{\forall}\m{y}\m{(}\m{\forall}\m{x}\m{(}\m{x}\m{
\in}\m{y}\m{\rightarrow}\m{x}\m{\in}\m{z}\m{)}\m{\rightarrow}\m{y}\m{\in}\m{x}
\m{)}
\endm

\needspace{3\baselineskip}
\noindent Axiom of Regularity.\index{Axiom of Regularity}

\setbox\startprefix=\hbox{\tt \ \ ax-reg\ \$a\ }
\setbox\contprefix=\hbox{\tt \ \ \ \ \ \ \ \ \ \ \ \ }
\startm
\m{\vdash}\m{(}\m{\exists}\m{x}\m{\,x}\m{\in}\m{y}\m{\rightarrow}\m{\exists}
\m{x}\m{(}\m{x}\m{\in}\m{y}\m{\wedge}\m{\forall}\m{z}\m{(}\m{z}\m{\in}\m{x}\m{
\rightarrow}\m{\lnot}\m{z}\m{\in}\m{y}\m{)}\m{)}\m{)}
\endm

\needspace{3\baselineskip}
\noindent Axiom of Infinity.\index{Axiom of Infinity}

\setbox\startprefix=\hbox{\tt \ \ ax-inf\ \$a\ }
\setbox\contprefix=\hbox{\tt \ \ \ \ \ \ \ \ \ \ \ \ \ \ \ }
\startm
\m{\vdash}\m{\exists}\m{x}\m{(}\m{y}\m{\in}\m{x}\m{\wedge}\m{\forall}\m{y}%
\m{(}\m{y}\m{\in}\m{x}\m{\rightarrow}\m{\exists}\m{z}\m{(}\m{y}\m{\in}\m{z}\m{%
\wedge}\m{z}\m{\in}\m{x}\m{)}\m{)}\m{)}
\endm

\needspace{4\baselineskip}
\noindent Axiom of Choice.\index{Axiom of Choice}

\setbox\startprefix=\hbox{\tt \ \ ax-ac\ \$a\ }
\setbox\contprefix=\hbox{\tt \ \ \ \ \ \ \ \ \ \ \ \ \ \ }
\startm
\m{\vdash}\m{\exists}\m{x}\m{\forall}\m{y}\m{\forall}\m{z}\m{(}\m{(}\m{y}\m{%
\in}\m{z}\m{\wedge}\m{z}\m{\in}\m{w}\m{)}\m{\rightarrow}\m{\exists}\m{w}\m{%
\forall}\m{y}\m{(}\m{\exists}\m{w}\m{(}\m{(}\m{y}\m{\in}\m{z}\m{\wedge}\m{z}%
\m{\in}\m{w}\m{)}\m{\wedge}\m{(}\m{y}\m{\in}\m{w}\m{\wedge}\m{w}\m{\in}\m{x}%
\m{)}\m{)}\m{\leftrightarrow}\m{y}\m{=}\m{w}\m{)}\m{)}
\endm

\subsection{That's It}

There you have it, the axioms for (essentially) all of mathematics!
Wonder at them and stare at them in awe.  Put a copy in your wallet, and
you will carry in your pocket the encoding for all theorems ever proved
and that ever will be proved, from the most mundane to the most
profound.

\section{A Hierarchy of Definitions}\label{hierarchy}

The axioms in the previous section in principle embody everything that can be
done within standard mathematics.  However, it is impractical to accomplish
very much by using them directly, for even simple concepts (from a human
perspective) can involve extremely long, incomprehensible formulas.
Mathematics is made practical by introducing definitions\index{definition}.
Definitions usually introduce new symbols, or at least new relationships among
existing symbols, to abbreviate more complex formulas.  An important
requirement for a definition is that there exist a straightforward
(algorithmic) method for eliminating the abbreviation by expanding it into the
more primitive symbol string that it represents.  Some
important definitions included in
the file \texttt{set.mm} are listed in this section for reference, and also to
give you a feel for why something like $\omega$\index{omega ($\omega$)} (the
set of natural numbers\index{natural number} 0, 1, 2,\ldots) becomes very
complicated when completely expanded into primitive symbols.

What is the motivation for definitions, aside from allowing complicated
expressions to be expressed more simply?  In the case of  $\omega$, one goal is
to provide a basis for the theory of natural numbers.\index{natural number}
Before set theory was invented, a set of axioms for arithmetic, called Peano's
postulates\index{Peano's postulates}, was devised and shown to have the
properties one expects for natural numbers.  Now anyone can postulate a
set of axioms, but if the axioms are inconsistent contradictions can be derived
from them.  Once a contradiction is derived, anything can be trivially
proved, including
all the facts of arithmetic and their negations.  To ensure that an
axiom system is at least as reliable as the axioms for set theory, we can
define sets and operations on those sets that satisfy the new axioms. In the
\texttt{set.mm} Metamath database, we prove that the elements of $\omega$ satisfy
Peano's postulates, and it's a long and hard journey to get there directly
from the axioms of set theory.  But the result is confidence in the
foundations of arithmetic.  And there is another advantage:  we now have all
the tools of set theory at our disposal for manipulating objects that obey the
axioms for arithmetic.

What are the criteria we use for definitions?  First, and of utmost importance,
the definition should not be {\em creative}\index{creative
definition}\index{definition!creative}, that
is it should not allow an expression that previously qualified as a wff but
was not provable, to become provable.   Second, the definition should be {\em
eliminable}\index{definition!eliminability}, that is, there should exist an
algorithmic method for converting any expression using the definition into
a logically equivalent expression that previously qualified as a wff.

In almost all cases below, definitions connect two expressions with either
$\leftrightarrow$ or $=$.  Eliminating\footnote{Here we mean the
elimination that a human might do in his or her head.  To eliminate them as
part of a Metamath proof we would invoke one of a number of
theorems that deal with transitivity of equivalence or equality; there are
many such examples in the proofs in \texttt{set.mm}.} such a definition is a
simple matter of substituting the expression on the left-hand side ({\em
definiendum}\index{definiendum} or thing being defined) with the equivalent,
more primitive expression on the right-hand side ({\em
definiens}\index{definiens} or definition).

Often a definition has variables on the right-hand side which do not appear on
the left-hand side; these are called {\em dummy variables}.\index{dummy
variable!in definitions}  In this case, any
allowable substitution (such as a new, distinct
variable) can be used when the definition is eliminated.  Dummy variables may
be used only if they are {\em effectively bound}\index{effectively bound
variable}, meaning that the definition will remain logically equivalent upon
any substitution of a dummy variable with any other {\em qualifying
expression}\index{qualifying expression}, i.e.\ any symbol string (such as
another variable) that
meets the restrictions on the dummy variable imposed by \texttt{\$d} and
\texttt{\$f} statements.  For example, we could define a constant $\perp$
(inverted tee, meaning logical ``false'') as $( \varphi \wedge \lnot \varphi
)$, i.e.\ ``phi and not phi.''  Here $\varphi$ is effectively bound because the
definition remains logically equivalent when we replace $\varphi$ with any
other wff.  (It is actually \texttt{df-fal}
in \texttt{set.mm}, which defines $\perp$.)

There are two cases where eliminating definitions is a little more
complex.  These cases are the definitions \texttt{df-bi} and
\texttt{df-cleq}.  The first stretches the concept of a definition a
little, as in effect it ``defines a definition;'' however, it meets our
requirements for a definition in that it is eliminable and does not
strengthen the language.  Theorem \texttt{bii} shows the substitution
needed to eliminate the $\leftrightarrow$\index{logical equivalence
($\leftrightarrow$)}\index{biconditional ($\leftrightarrow$)} symbol.

Definition \texttt{df-cleq}\index{equality ($=$)} extends the usage of
the equality symbol to include ``classes''\index{class} in set theory.  The
reason it is potentially problematic is that it can lead to statements which
do not follow from logic alone but presuppose the Axiom of
Extensionality\index{Axiom of Extensionality}, so we include this axiom
as a hypothesis for the definition.  We could have made \texttt{df-cleq} directly
eliminable by introducing a new equality symbol, but have chosen not to do so
in keeping with standard textbook practice.  Definitions such as \texttt{df-cleq}
that extend the meaning of existing symbols must be introduced carefully so
that they do not lead to contradictions.  Definition \texttt{df-clel} also
extends the meaning of an existing symbol ($\in$); while it doesn't strengthen
the language like \texttt{df-cleq}, this is not obvious and it must also be
subject to the same scrutiny.

Exercise:  Study how the wff $x\in\omega$, meaning ``$x$ is a natural
number,'' could be expanded in terms of primitive symbols, starting with the
definitions \texttt{df-clel} on p.~\pageref{dfclel} and \texttt{df-om} on
p.~\pageref{dfom} and working your way back.  Don't bother to work out the
details; just make sure that you understand how you could do it in principle.
The answer is shown in the footnote on p.~\pageref{expandom}.  If you
actually do work it out, you won't get exactly the same answer because we used
a few simplifications such as discarding occurrences of $\lnot\lnot$ (double
negation).

In the definitions below, we have placed the {\sc ascii} Metamath source
below each of the formulas to help you become familiar with the
notation in the database.  For simplicity, the necessary \texttt{\$f}
and \texttt{\$d} statements are not shown.  If you are in doubt, use the
\texttt{show statement}\index{\texttt{show statement} command} command
in the Metamath program to see the full statement.
A selection of this notation is summarized in Appendix~\ref{ASCII}.

To understand the motivation for these definitions, you should consult the
references indicated:  Takeuti and Zaring \cite{Takeuti}\index{Takeuti, Gaisi},
Quine \cite{Quine}\index{Quine, Willard Van Orman}, Bell and Machover
\cite{Bell}\index{Bell, J. L.}, and Enderton \cite{Enderton}\index{Enderton,
Herbert B.}.  Our list of definitions is provided more for reference than as a
learning aid.  However, by looking at a few of them you can gain a feel for
how the hierarchy is built up.  The definitions are a representative sample of
the 130 or so in \texttt{set.mm}, but they are complete with respect to the
theorem examples we will present in Section~\ref{sometheorems}.  Also, some are
slightly different from, but logically equivalent to, the ones in \texttt{set.mm}
(some of which have been revised over time to shorten them, for example).

\subsection{Definitions for Propositional Calculus}\label{metadefprop}

The symbols $\varphi$, $\psi$, and $\chi$ represent wffs.

Our first definition introduces the biconditional
connective\footnote{The term ``connective'' is informally used to mean a
symbol that is placed between two variables or adjacent to a variable,
whereas a mathematical ``constant'' usually indicates a symbol such as
the number 0 that may replace a variable or metavariable.  From
Metamath's point of view, there is no distinction between a connective
and a constant; both are constants in the Metamath
language.}\index{connective}\index{constant} (also called logical
equivalence)\index{logical equivalence
($\leftrightarrow$)}\index{biconditional ($\leftrightarrow$)}.  Unlike
most traditional developments, we have chosen not to have a separate
symbol such as ``Df.'' to mean ``is defined as.''  Instead, we will use
the biconditional connective for this purpose, as it lets us use
logic to manipulate definitions directly.  Here we state the properties
of the biconditional connective with a carefully crafted \texttt{\$a}
statement, which effectively uses the biconditional connective to define
itself.  The $\leftrightarrow$ symbol can be eliminated from a formula
using theorem \texttt{bii}, which is derived later.

\vskip 2ex
\noindent Define the biconditional connective.\label{df-bi}

\vskip 0.5ex
\setbox\startprefix=\hbox{\tt \ \ df-bi\ \$a\ }
\setbox\contprefix=\hbox{\tt \ \ \ \ \ \ \ \ \ \ \ }
\startm
\m{\vdash}\m{\lnot}\m{(}\m{(}\m{(}\m{\varphi}\m{\leftrightarrow}\m{\psi}\m{)}%
\m{\rightarrow}\m{\lnot}\m{(}\m{(}\m{\varphi}\m{\rightarrow}\m{\psi}\m{)}\m{%
\rightarrow}\m{\lnot}\m{(}\m{\psi}\m{\rightarrow}\m{\varphi}\m{)}\m{)}\m{)}\m{%
\rightarrow}\m{\lnot}\m{(}\m{\lnot}\m{(}\m{(}\m{\varphi}\m{\rightarrow}\m{%
\psi}\m{)}\m{\rightarrow}\m{\lnot}\m{(}\m{\psi}\m{\rightarrow}\m{\varphi}\m{)}%
\m{)}\m{\rightarrow}\m{(}\m{\varphi}\m{\leftrightarrow}\m{\psi}\m{)}\m{)}\m{)}
\endm
\begin{mmraw}%
|- -. ( ( ( ph <-> ps ) -> -. ( ( ph -> ps ) ->
-. ( ps -> ph ) ) ) -> -. ( -. ( ( ph -> ps ) -> -. (
ps -> ph ) ) -> ( ph <-> ps ) ) ) \$.
\end{mmraw}

\noindent This theorem relates the biconditional connective to primitive
connectives and can be used to eliminate the $\leftrightarrow$ symbol from any
wff.

\vskip 0.5ex
\setbox\startprefix=\hbox{\tt \ \ bii\ \$p\ }
\setbox\contprefix=\hbox{\tt \ \ \ \ \ \ \ \ \ }
\startm
\m{\vdash}\m{(}\m{(}\m{\varphi}\m{\leftrightarrow}\m{\psi}\m{)}\m{\leftrightarrow}
\m{\lnot}\m{(}\m{(}\m{\varphi}\m{\rightarrow}\m{\psi}\m{)}\m{\rightarrow}\m{\lnot}
\m{(}\m{\psi}\m{\rightarrow}\m{\varphi}\m{)}\m{)}\m{)}
\endm
\begin{mmraw}%
|- ( ( ph <-> ps ) <-> -. ( ( ph -> ps ) -> -. ( ps -> ph ) ) ) \$= ... \$.
\end{mmraw}

\noindent Define disjunction ({\sc or}).\index{disjunction ($\vee$)}%
\index{logical or (vee)@logical {\sc or} ($\vee$)}%
\index{df-or@\texttt{df-or}}\label{df-or}

\vskip 0.5ex
\setbox\startprefix=\hbox{\tt \ \ df-or\ \$a\ }
\setbox\contprefix=\hbox{\tt \ \ \ \ \ \ \ \ \ \ \ }
\startm
\m{\vdash}\m{(}\m{(}\m{\varphi}\m{\vee}\m{\psi}\m{)}\m{\leftrightarrow}\m{(}\m{
\lnot}\m{\varphi}\m{\rightarrow}\m{\psi}\m{)}\m{)}
\endm
\begin{mmraw}%
|- ( ( ph \TOR ps ) <-> ( -. ph -> ps ) ) \$.
\end{mmraw}

\noindent Define conjunction ({\sc and}).\index{conjunction ($\wedge$)}%
\index{logical {\sc and} ($\wedge$)}%
\index{df-an@\texttt{df-an}}\label{df-an}

\vskip 0.5ex
\setbox\startprefix=\hbox{\tt \ \ df-an\ \$a\ }
\setbox\contprefix=\hbox{\tt \ \ \ \ \ \ \ \ \ \ \ }
\startm
\m{\vdash}\m{(}\m{(}\m{\varphi}\m{\wedge}\m{\psi}\m{)}\m{\leftrightarrow}\m{\lnot}
\m{(}\m{\varphi}\m{\rightarrow}\m{\lnot}\m{\psi}\m{)}\m{)}
\endm
\begin{mmraw}%
|- ( ( ph \TAND ps ) <-> -. ( ph -> -. ps ) ) \$.
\end{mmraw}

\noindent Define disjunction ({\sc or}) of 3 wffs.%
\index{df-3or@\texttt{df-3or}}\label{df-3or}

\vskip 0.5ex
\setbox\startprefix=\hbox{\tt \ \ df-3or\ \$a\ }
\setbox\contprefix=\hbox{\tt \ \ \ \ \ \ \ \ \ \ \ \ }
\startm
\m{\vdash}\m{(}\m{(}\m{\varphi}\m{\vee}\m{\psi}\m{\vee}\m{\chi}\m{)}\m{
\leftrightarrow}\m{(}\m{(}\m{\varphi}\m{\vee}\m{\psi}\m{)}\m{\vee}\m{\chi}\m{)}
\m{)}
\endm
\begin{mmraw}%
|- ( ( ph \TOR ps \TOR ch ) <-> ( ( ph \TOR ps ) \TOR ch ) ) \$.
\end{mmraw}

\noindent Define conjunction ({\sc and}) of 3 wffs.%
\index{df-3an}\label{df-3an}

\vskip 0.5ex
\setbox\startprefix=\hbox{\tt \ \ df-3an\ \$a\ }
\setbox\contprefix=\hbox{\tt \ \ \ \ \ \ \ \ \ \ \ \ }
\startm
\m{\vdash}\m{(}\m{(}\m{\varphi}\m{\wedge}\m{\psi}\m{\wedge}\m{\chi}\m{)}\m{
\leftrightarrow}\m{(}\m{(}\m{\varphi}\m{\wedge}\m{\psi}\m{)}\m{\wedge}\m{\chi}
\m{)}\m{)}
\endm

\begin{mmraw}%
|- ( ( ph \TAND ps \TAND ch ) <-> ( ( ph \TAND ps ) \TAND ch ) ) \$.
\end{mmraw}

\subsection{Definitions for Predicate Calculus}\label{metadefpred}

The symbols $x$, $y$, and $z$ represent individual variables of predicate
calculus.  In this section, they are not necessarily distinct unless it is
explicitly
mentioned.

\vskip 2ex
\noindent Define existential quantification.
The expression $\exists x \varphi$ means
``there exists an $x$ where $\varphi$ is true.''\index{existential quantifier
($\exists$)}\label{df-ex}

\vskip 0.5ex
\setbox\startprefix=\hbox{\tt \ \ df-ex\ \$a\ }
\setbox\contprefix=\hbox{\tt \ \ \ \ \ \ \ \ \ \ \ }
\startm
\m{\vdash}\m{(}\m{\exists}\m{x}\m{\varphi}\m{\leftrightarrow}\m{\lnot}\m{\forall}
\m{x}\m{\lnot}\m{\varphi}\m{)}
\endm
\begin{mmraw}%
|- ( E. x ph <-> -. A. x -. ph ) \$.
\end{mmraw}

\noindent Define proper substitution.\index{proper
substitution}\index{substitution!proper}\label{df-sb}
In our notation, we use $[ y / x ] \varphi$ to mean ``the wff that
results when $y$ is properly substituted for $x$ in the wff
$\varphi$.''\footnote{
This can also be described
as substituting $x$ with $y$, $y$ properly replaces $x$, or
$x$ is properly replaced by $y$.}
% This is elsb4, though it currently says: ( [ x / y ] z e. y <-> z e. x )
For example,
$[ y / x ] z \in x$ is the same as $z \in y$.
One way to remember this notation is to notice that it looks like division
and recall that $( y / x ) \cdot x $ is $y$ (when $x \neq 0$).
The notation is different from the notation $\varphi ( x | y )$
that is sometimes used, because the latter notation is ambiguous for us:
for example, we don't know whether $\lnot \varphi ( x | y )$ is to be
interpreted as $\lnot ( \varphi ( x | y ) )$ or
$( \lnot \varphi ) ( x | y )$.\footnote{Because of the way
we initially defined wffs, this is the case
with any postfix connective\index{postfix connective} (one occurring after the
symbols being connected) or infix connective\index{infix connective} (one
occurring between the symbols being connected).  Metamath does not have a
built-in notion of operator binding strength that could eliminate the
ambiguity.  The initial parenthesis effectively provides a prefix
connective\index{prefix connective} to eliminate ambiguity.  Some conventions,
such as Polish notation\index{Polish notation} used in the 1930's and 1940's
by Polish logicians, use only prefix connectives and thus allow the total
elimination of parentheses, at the expense of readability.  In Metamath we
could actually redefine all notation to be Polish if we wanted to without
having to change any proofs!}  Other texts often use $\varphi(y)$ to indicate
our $[ y / x ] \varphi$, but this notation is even more ambiguous since there is
no explicit indication of what is being substituted.
Note that this
definition is valid even when
$x$ and $y$ are the same variable.  The first conjunct is a ``trick'' used to
achieve this property, making the definition look somewhat peculiar at
first.

\vskip 0.5ex
\setbox\startprefix=\hbox{\tt \ \ df-sb\ \$a\ }
\setbox\contprefix=\hbox{\tt \ \ \ \ \ \ \ \ \ \ \ }
\startm
\m{\vdash}\m{(}\m{[}\m{y}\m{/}\m{x}\m{]}\m{\varphi}\m{\leftrightarrow}\m{(}%
\m{(}\m{x}\m{=}\m{y}\m{\rightarrow}\m{\varphi}\m{)}\m{\wedge}\m{\exists}\m{x}%
\m{(}\m{x}\m{=}\m{y}\m{\wedge}\m{\varphi}\m{)}\m{)}\m{)}
\endm
\begin{mmraw}%
|- ( [ y / x ] ph <-> ( ( x = y -> ph ) \TAND E. x ( x = y \TAND ph ) ) ) \$.
\end{mmraw}


\noindent Define existential uniqueness\index{existential uniqueness
quantifier ($\exists "!$)} (``there exists exactly one'').  Note that $y$ is a
variable distinct from $x$ and not occurring in $\varphi$.\label{df-eu}

\vskip 0.5ex
\setbox\startprefix=\hbox{\tt \ \ df-eu\ \$a\ }
\setbox\contprefix=\hbox{\tt \ \ \ \ \ \ \ \ \ \ \ }
\startm
\m{\vdash}\m{(}\m{\exists}\m{{!}}\m{x}\m{\varphi}\m{\leftrightarrow}\m{\exists}
\m{y}\m{\forall}\m{x}\m{(}\m{\varphi}\m{\leftrightarrow}\m{x}\m{=}\m{y}\m{)}\m{)}
\endm

\begin{mmraw}%
|- ( E! x ph <-> E. y A. x ( ph <-> x = y ) ) \$.
\end{mmraw}

\subsection{Definitions for Set Theory}\label{setdefinitions}

The symbols $x$, $y$, $z$, and $w$ represent individual variables of
predicate calculus, which in set theory are understood to be sets.
However, using only the constructs shown so far would be very inconvenient.

To make set theory more practical, we introduce the notion of a ``class.''
A class\index{class} is either a set variable (such as $x$) or an
expression of the form $\{ x | \varphi\}$ (called an ``abstraction
class''\index{abstraction class}\index{class abstraction}).  Note that
sets (i.e.\ individual variables) always exist (this is a theorem of
logic, namely $\exists y \, y = x$ for any set $x$), whereas classes may
or may not exist (i.e.\ $\exists y \, y = A$ may or may not be true).
If a class does not exist it is called a ``proper class.''\index{proper
class}\index{class!proper} Definitions \texttt{df-clab},
\texttt{df-cleq}, and \texttt{df-clel} can be used to convert an
expression containing classes into one containing only set variables and
wff metavariables.

The symbols $A$, $B$, $C$, $D$, $ F$, $G$, and $R$ are metavariables that range
over classes.  A class metavariable $A$ may be eliminated from a wff by
replacing it with $\{ x|\varphi\}$ where neither $x$ nor $\varphi$ occur in
the wff.

The theory of classes can be shown to be an eliminable and conservative
extension of set theory. The \textbf{eliminability}
property shows that for every
formula in the extended language we can build a logically equivalent
formula in the basic language; so that even if the extended language
provides more ease to convey and formulate mathematical ideas for set
theory, its expressive power does not in fact strengthen the basic
language's expressive power.
The \textbf{conservation} property shows that for
every proof of a formula of the basic language in the extended system
we can build another proof of the same formula in the basic system;
so that, concerning theorems on sets only, the deductive powers of
the extended system and of the basic system are identical. Together,
these properties mean that the extended language can be treated as a
definitional extension that is \textbf{sound}.

A rigorous justification, which we will not give here, can be found in
Levy \cite[pp.~357-366]{Levy} supplementing his informal introduction to class
theory on pp.~7-17. Two other good treatments of class theory are provided
by Quine \cite[pp.~15-21]{Quine}\index{Quine, Willard Van Orman}
and also \cite[pp.~10-14]{Takeuti}\index{Takeuti, Gaisi}.
Quine's exposition (he calls them virtual classes)
is nicely written and very readable.

In the rest of this
section, individual variables are always assumed to be distinct from
each other unless otherwise indicated.  In addition, dummy variables on the
right-hand side of a definition do not occur in the class and wff
metavariables in the definition.

The definitions we present here are a partial but self-contained
collection selected from several hundred that appear in the current
\texttt{set.mm} database.  They are adequate for a basic development of
elementary set theory.

\vskip 2ex
\noindent Define the abstraction class.\index{abstraction class}\index{class
abstraction}\label{df-clab}  $x$ and $y$
need not be distinct.  Definition 2.1 of Quine, p.~16.  This definition may
seem puzzling since it is shorter than the expression being defined and does not
buy us anything in terms of brevity.  The reason we introduce this definition
is because it fits in neatly with the extension of the $\in$ connective
provided by \texttt{df-clel}.

\vskip 0.5ex
\setbox\startprefix=\hbox{\tt \ \ df-clab\ \$a\ }
\setbox\contprefix=\hbox{\tt \ \ \ \ \ \ \ \ \ \ \ \ \ }
\startm
\m{\vdash}\m{(}\m{x}\m{\in}\m{\{}\m{y}\m{|}\m{\varphi}\m{\}}\m{%
\leftrightarrow}\m{[}\m{x}\m{/}\m{y}\m{]}\m{\varphi}\m{)}
\endm
\begin{mmraw}%
|- ( x e. \{ y | ph \} <-> [ x / y ] ph ) \$.
\end{mmraw}

\noindent Define the equality connective between classes\index{class
equality}\label{df-cleq}.  See Quine or Chapter 4 of Takeuti and Zaring for its
justification and methods for eliminating it.  This is an example of a
somewhat ``dangerous'' definition, because it extends the use of the
existing equality symbol rather than introducing a new symbol, allowing
us to make statements in the original language that may not be true.
For example, it permits us to deduce $y = z \leftrightarrow \forall x (
x \in y \leftrightarrow x \in z )$ which is not a theorem of logic but
rather presupposes the Axiom of Extensionality,\index{Axiom of
Extensionality} which we include as a hypothesis so that we can know
when this axiom is assumed in a proof (with the \texttt{show
trace{\char`\_}back} command).  We could avoid the danger by introducing
another symbol, say $\eqcirc$, in place of $=$; this
would also have the advantage of making elimination of the definition
straightforward and would eliminate the need for Extensionality as a
hypothesis.  We would then also have the advantage of being able to
identify exactly where Extensionality truly comes into play.  One of our
theorems would be $x \eqcirc y \leftrightarrow x = y$
by invoking Extensionality.  However in keeping with standard practice
we retain the ``dangerous'' definition.

\vskip 0.5ex
\setbox\startprefix=\hbox{\tt \ \ df-cleq.1\ \$e\ }
\setbox\contprefix=\hbox{\tt \ \ \ \ \ \ \ \ \ \ \ \ \ \ \ }
\startm
\m{\vdash}\m{(}\m{\forall}\m{x}\m{(}\m{x}\m{\in}\m{y}\m{\leftrightarrow}\m{x}
\m{\in}\m{z}\m{)}\m{\rightarrow}\m{y}\m{=}\m{z}\m{)}
\endm
\setbox\startprefix=\hbox{\tt \ \ df-cleq\ \$a\ }
\setbox\contprefix=\hbox{\tt \ \ \ \ \ \ \ \ \ \ \ \ \ }
\startm
\m{\vdash}\m{(}\m{A}\m{=}\m{B}\m{\leftrightarrow}\m{\forall}\m{x}\m{(}\m{x}\m{
\in}\m{A}\m{\leftrightarrow}\m{x}\m{\in}\m{B}\m{)}\m{)}
\endm
% We need to reset the startprefix and contprefix.
\setbox\startprefix=\hbox{\tt \ \ df-cleq.1\ \$e\ }
\setbox\contprefix=\hbox{\tt \ \ \ \ \ \ \ \ \ \ \ \ \ \ \ }
\begin{mmraw}%
|- ( A. x ( x e. y <-> x e. z ) -> y = z ) \$.
\end{mmraw}
\setbox\startprefix=\hbox{\tt \ \ df-cleq\ \$a\ }
\setbox\contprefix=\hbox{\tt \ \ \ \ \ \ \ \ \ \ \ \ \ }
\begin{mmraw}%
|- ( A = B <-> A. x ( x e. A <-> x e. B ) ) \$.
\end{mmraw}

\noindent Define the membership connective between classes\index{class
membership}.  Theorem 6.3 of Quine, p.~41, which we adopt as a definition.
Note that it extends the use of the existing membership symbol, but unlike
{\tt df-cleq} it does not extend the set of valid wffs of logic when the class
metavariables are replaced with set variables.\label{dfclel}\label{df-clel}

\vskip 0.5ex
\setbox\startprefix=\hbox{\tt \ \ df-clel\ \$a\ }
\setbox\contprefix=\hbox{\tt \ \ \ \ \ \ \ \ \ \ \ \ \ }
\startm
\m{\vdash}\m{(}\m{A}\m{\in}\m{B}\m{\leftrightarrow}\m{\exists}\m{x}\m{(}\m{x}
\m{=}\m{A}\m{\wedge}\m{x}\m{\in}\m{B}\m{)}\m{)}
\endm
\begin{mmraw}%
|- ( A e. B <-> E. x ( x = A \TAND x e. B ) ) \$.?
\end{mmraw}

\noindent Define inequality.

\vskip 0.5ex
\setbox\startprefix=\hbox{\tt \ \ df-ne\ \$a\ }
\setbox\contprefix=\hbox{\tt \ \ \ \ \ \ \ \ \ \ \ }
\startm
\m{\vdash}\m{(}\m{A}\m{\ne}\m{B}\m{\leftrightarrow}\m{\lnot}\m{A}\m{=}\m{B}%
\m{)}
\endm
\begin{mmraw}%
|- ( A =/= B <-> -. A = B ) \$.
\end{mmraw}

\noindent Define restricted universal quantification.\index{universal
quantifier ($\forall$)!restricted}  Enderton, p.~22.

\vskip 0.5ex
\setbox\startprefix=\hbox{\tt \ \ df-ral\ \$a\ }
\setbox\contprefix=\hbox{\tt \ \ \ \ \ \ \ \ \ \ \ \ }
\startm
\m{\vdash}\m{(}\m{\forall}\m{x}\m{\in}\m{A}\m{\varphi}\m{\leftrightarrow}\m{%
\forall}\m{x}\m{(}\m{x}\m{\in}\m{A}\m{\rightarrow}\m{\varphi}\m{)}\m{)}
\endm
\begin{mmraw}%
|- ( A. x e. A ph <-> A. x ( x e. A -> ph ) ) \$.
\end{mmraw}

\noindent Define restricted existential quantification.\index{existential
quantifier ($\exists$)!restricted}  Enderton, p.~22.

\vskip 0.5ex
\setbox\startprefix=\hbox{\tt \ \ df-rex\ \$a\ }
\setbox\contprefix=\hbox{\tt \ \ \ \ \ \ \ \ \ \ \ \ }
\startm
\m{\vdash}\m{(}\m{\exists}\m{x}\m{\in}\m{A}\m{\varphi}\m{\leftrightarrow}\m{%
\exists}\m{x}\m{(}\m{x}\m{\in}\m{A}\m{\wedge}\m{\varphi}\m{)}\m{)}
\endm
\begin{mmraw}%
|- ( E. x e. A ph <-> E. x ( x e. A \TAND ph ) ) \$.
\end{mmraw}

\noindent Define the universal class\index{universal class ($V$)}.  Definition
5.20, p.~21, of Takeuti and Zaring.\label{df-v}

\vskip 0.5ex
\setbox\startprefix=\hbox{\tt \ \ df-v\ \$a\ }
\setbox\contprefix=\hbox{\tt \ \ \ \ \ \ \ \ \ \ }
\startm
\m{\vdash}\m{{\rm V}}\m{=}\m{\{}\m{x}\m{|}\m{x}\m{=}\m{x}\m{\}}
\endm
\begin{mmraw}%
|- {\char`\_}V = \{ x | x = x \} \$.
\end{mmraw}

\noindent Define the subclass\index{subclass}\index{subset} relationship
between two classes (called the subset relation if the classes are sets i.e.\
are not proper).  Definition 5.9 of Takeuti and Zaring, p.~17.\label{df-ss}

\vskip 0.5ex
\setbox\startprefix=\hbox{\tt \ \ df-ss\ \$a\ }
\setbox\contprefix=\hbox{\tt \ \ \ \ \ \ \ \ \ \ \ }
\startm
\m{\vdash}\m{(}\m{A}\m{\subseteq}\m{B}\m{\leftrightarrow}\m{\forall}\m{x}\m{(}
\m{x}\m{\in}\m{A}\m{\rightarrow}\m{x}\m{\in}\m{B}\m{)}\m{)}
\endm
\begin{mmraw}%
|- ( A C\_ B <-> A. x ( x e. A -> x e. B ) ) \$.
\end{mmraw}

\noindent Define the union\index{union} of two classes.  Definition 5.6 of Takeuti and Zaring,
p.~16.\label{df-un}

\vskip 0.5ex
\setbox\startprefix=\hbox{\tt \ \ df-un\ \$a\ }
\setbox\contprefix=\hbox{\tt \ \ \ \ \ \ \ \ \ \ \ }
\startm
\m{\vdash}\m{(}\m{A}\m{\cup}\m{B}\m{)}\m{=}\m{\{}\m{x}\m{|}\m{(}\m{x}\m{\in}
\m{A}\m{\vee}\m{x}\m{\in}\m{B}\m{)}\m{\}}
\endm
\begin{mmraw}%
( A u. B ) = \{ x | ( x e. A \TOR x e. B ) \} \$.
\end{mmraw}

\noindent Define the intersection\index{intersection} of two classes.  Definition 5.6 of
Takeuti and Zaring, p.~16.\label{df-in}

\vskip 0.5ex
\setbox\startprefix=\hbox{\tt \ \ df-in\ \$a\ }
\setbox\contprefix=\hbox{\tt \ \ \ \ \ \ \ \ \ \ \ }
\startm
\m{\vdash}\m{(}\m{A}\m{\cap}\m{B}\m{)}\m{=}\m{\{}\m{x}\m{|}\m{(}\m{x}\m{\in}
\m{A}\m{\wedge}\m{x}\m{\in}\m{B}\m{)}\m{\}}
\endm
% Caret ^ requires special treatment
\begin{mmraw}%
|- ( A i\^{}i B ) = \{ x | ( x e. A \TAND x e. B ) \} \$.
\end{mmraw}

\noindent Define class difference\index{class difference}\index{set difference}.
Definition 5.12 of Takeuti and Zaring, p.~20.  Several notations are used in
the literature; we chose the $\setminus$ convention instead of a minus sign to
reserve the latter for later use in, e.g., arithmetic.\label{df-dif}

\vskip 0.5ex
\setbox\startprefix=\hbox{\tt \ \ df-dif\ \$a\ }
\setbox\contprefix=\hbox{\tt \ \ \ \ \ \ \ \ \ \ \ \ }
\startm
\m{\vdash}\m{(}\m{A}\m{\setminus}\m{B}\m{)}\m{=}\m{\{}\m{x}\m{|}\m{(}\m{x}\m{
\in}\m{A}\m{\wedge}\m{\lnot}\m{x}\m{\in}\m{B}\m{)}\m{\}}
\endm
\begin{mmraw}%
( A \SLASH B ) = \{ x | ( x e. A \TAND -. x e. B ) \} \$.
\end{mmraw}

\noindent Define the empty or null set\index{empty set}\index{null set}.
Compare  Definition 5.14 of Takeuti and Zaring, p.~20.\label{df-nul}

\vskip 0.5ex
\setbox\startprefix=\hbox{\tt \ \ df-nul\ \$a\ }
\setbox\contprefix=\hbox{\tt \ \ \ \ \ \ \ \ \ \ }
\startm
\m{\vdash}\m{\varnothing}\m{=}\m{(}\m{{\rm V}}\m{\setminus}\m{{\rm V}}\m{)}
\endm
\begin{mmraw}%
|- (/) = ( {\char`\_}V \SLASH {\char`\_}V ) \$.
\end{mmraw}

\noindent Define power class\index{power set}\index{power class}.  Definition 5.10 of
Takeuti and Zaring, p.~17, but we also let it apply to proper classes.  (Note
that \verb$~P$ is the symbol for calligraphic P, the tilde
suggesting ``curly;'' see Appendix~\ref{ASCII}.)\label{df-pw}

\vskip 0.5ex
\setbox\startprefix=\hbox{\tt \ \ df-pw\ \$a\ }
\setbox\contprefix=\hbox{\tt \ \ \ \ \ \ \ \ \ \ \ }
\startm
\m{\vdash}\m{{\cal P}}\m{A}\m{=}\m{\{}\m{x}\m{|}\m{x}\m{\subseteq}\m{A}\m{\}}
\endm
% Special incantation required to put ~ into the text
\begin{mmraw}%
|- \char`\~P~A = \{ x | x C\_ A \} \$.
\end{mmraw}

\noindent Define the singleton of a class\index{singleton}.  Definition 7.1 of
Quine, p.~48.  It is well-defined for proper classes, although
it is not very meaningful in this case, where it evaluates to the empty
set.

\vskip 0.5ex
\setbox\startprefix=\hbox{\tt \ \ df-sn\ \$a\ }
\setbox\contprefix=\hbox{\tt \ \ \ \ \ \ \ \ \ \ \ }
\startm
\m{\vdash}\m{\{}\m{A}\m{\}}\m{=}\m{\{}\m{x}\m{|}\m{x}\m{=}\m{A}\m{\}}
\endm
\begin{mmraw}%
|- \{ A \} = \{ x | x = A \} \$.
\end{mmraw}%

\noindent Define an unordered pair of classes\index{unordered pair}\index{pair}.  Definition
7.1 of Quine, p.~48.

\vskip 0.5ex
\setbox\startprefix=\hbox{\tt \ \ df-pr\ \$a\ }
\setbox\contprefix=\hbox{\tt \ \ \ \ \ \ \ \ \ \ \ }
\startm
\m{\vdash}\m{\{}\m{A}\m{,}\m{B}\m{\}}\m{=}\m{(}\m{\{}\m{A}\m{\}}\m{\cup}\m{\{}
\m{B}\m{\}}\m{)}
\endm
\begin{mmraw}%
|- \{ A , B \} = ( \{ A \} u. \{ B \} ) \$.
\end{mmraw}

\noindent Define an unordered triple of classes\index{unordered triple}.  Definition of
Enderton, p.~19.

\vskip 0.5ex
\setbox\startprefix=\hbox{\tt \ \ df-tp\ \$a\ }
\setbox\contprefix=\hbox{\tt \ \ \ \ \ \ \ \ \ \ \ }
\startm
\m{\vdash}\m{\{}\m{A}\m{,}\m{B}\m{,}\m{C}\m{\}}\m{=}\m{(}\m{\{}\m{A}\m{,}\m{B}
\m{\}}\m{\cup}\m{\{}\m{C}\m{\}}\m{)}
\endm
\begin{mmraw}%
|- \{ A , B , C \} = ( \{ A , B \} u. \{ C \} ) \$.
\end{mmraw}%

\noindent Kuratowski's\index{Kuratowski, Kazimierz} ordered pair\index{ordered
pair} definition.  Definition 9.1 of Quine, p.~58. For proper classes it is
not meaningful but is well-defined for convenience.  (Note that \verb$<.$
stands for $\langle$ whereas \verb$<$ stands for $<$, and similarly for
\verb$>.$\,.)\label{df-op}

\vskip 0.5ex
\setbox\startprefix=\hbox{\tt \ \ df-op\ \$a\ }
\setbox\contprefix=\hbox{\tt \ \ \ \ \ \ \ \ \ \ \ }
\startm
\m{\vdash}\m{\langle}\m{A}\m{,}\m{B}\m{\rangle}\m{=}\m{\{}\m{\{}\m{A}\m{\}}
\m{,}\m{\{}\m{A}\m{,}\m{B}\m{\}}\m{\}}
\endm
\begin{mmraw}%
|- <. A , B >. = \{ \{ A \} , \{ A , B \} \} \$.
\end{mmraw}

\noindent Define the union of a class\index{union}.  Definition 5.5, p.~16,
of Takeuti and Zaring.

\vskip 0.5ex
\setbox\startprefix=\hbox{\tt \ \ df-uni\ \$a\ }
\setbox\contprefix=\hbox{\tt \ \ \ \ \ \ \ \ \ \ \ \ }
\startm
\m{\vdash}\m{\bigcup}\m{A}\m{=}\m{\{}\m{x}\m{|}\m{\exists}\m{y}\m{(}\m{x}\m{
\in}\m{y}\m{\wedge}\m{y}\m{\in}\m{A}\m{)}\m{\}}
\endm
\begin{mmraw}%
|- U. A = \{ x | E. y ( x e. y \TAND y e. A ) \} \$.
\end{mmraw}

\noindent Define the intersection\index{intersection} of a class.  Definition 7.35,
p.~44, of Takeuti and Zaring.

\vskip 0.5ex
\setbox\startprefix=\hbox{\tt \ \ df-int\ \$a\ }
\setbox\contprefix=\hbox{\tt \ \ \ \ \ \ \ \ \ \ \ \ }
\startm
\m{\vdash}\m{\bigcap}\m{A}\m{=}\m{\{}\m{x}\m{|}\m{\forall}\m{y}\m{(}\m{y}\m{
\in}\m{A}\m{\rightarrow}\m{x}\m{\in}\m{y}\m{)}\m{\}}
\endm
\begin{mmraw}%
|- |\^{}| A = \{ x | A. y ( y e. A -> x e. y ) \} \$.
\end{mmraw}

\noindent Define a transitive class\index{transitive class}\index{transitive
set}.  This should not be confused with a transitive relation, which is a different
concept.  Definition from p.~71 of Enderton, extended to classes.

\vskip 0.5ex
\setbox\startprefix=\hbox{\tt \ \ df-tr\ \$a\ }
\setbox\contprefix=\hbox{\tt \ \ \ \ \ \ \ \ \ \ \ }
\startm
\m{\vdash}\m{(}\m{\mbox{\rm Tr}}\m{A}\m{\leftrightarrow}\m{\bigcup}\m{A}\m{
\subseteq}\m{A}\m{)}
\endm
\begin{mmraw}%
|- ( Tr A <-> U. A C\_ A ) \$.
\end{mmraw}

\noindent Define a notation for a general binary relation\index{binary
relation}.  Definition 6.18, p.~29, of Takeuti and Zaring, generalized to
arbitrary classes.  This definition is well-defined, although not very
meaningful, when classes $A$ and/or $B$ are proper.\label{dfbr}  The lack of
parentheses (or any other connective) creates no ambiguity since we are defining
an atomic wff.

\vskip 0.5ex
\setbox\startprefix=\hbox{\tt \ \ df-br\ \$a\ }
\setbox\contprefix=\hbox{\tt \ \ \ \ \ \ \ \ \ \ \ }
\startm
\m{\vdash}\m{(}\m{A}\m{\,R}\m{\,B}\m{\leftrightarrow}\m{\langle}\m{A}\m{,}\m{B}
\m{\rangle}\m{\in}\m{R}\m{)}
\endm
\begin{mmraw}%
|- ( A R B <-> <. A , B >. e. R ) \$.
\end{mmraw}

\noindent Define an abstraction class of ordered pairs\index{abstraction
class!of ordered
pairs}.  A special case of Definition 4.16, p.~14, of Takeuti and Zaring.
Note that $ z $ must be distinct from $ x $ and $ y $,
and $ z $ must not occur in $\varphi$, but $ x $ and $ y $ may be
identical and may appear in $\varphi$.

\vskip 0.5ex
\setbox\startprefix=\hbox{\tt \ \ df-opab\ \$a\ }
\setbox\contprefix=\hbox{\tt \ \ \ \ \ \ \ \ \ \ \ \ \ }
\startm
\m{\vdash}\m{\{}\m{\langle}\m{x}\m{,}\m{y}\m{\rangle}\m{|}\m{\varphi}\m{\}}\m{=}
\m{\{}\m{z}\m{|}\m{\exists}\m{x}\m{\exists}\m{y}\m{(}\m{z}\m{=}\m{\langle}\m{x}
\m{,}\m{y}\m{\rangle}\m{\wedge}\m{\varphi}\m{)}\m{\}}
\endm

\begin{mmraw}%
|- \{ <. x , y >. | ph \} = \{ z | E. x E. y ( z =
<. x , y >. /\ ph ) \} \$.
\end{mmraw}

\noindent Define the epsilon relation\index{epsilon relation}.  Similar to Definition
6.22, p.~30, of Takeuti and Zaring.

\vskip 0.5ex
\setbox\startprefix=\hbox{\tt \ \ df-eprel\ \$a\ }
\setbox\contprefix=\hbox{\tt \ \ \ \ \ \ \ \ \ \ \ \ \ \ }
\startm
\m{\vdash}\m{{\rm E}}\m{=}\m{\{}\m{\langle}\m{x}\m{,}\m{y}\m{\rangle}\m{|}\m{x}\m{
\in}\m{y}\m{\}}
\endm
\begin{mmraw}%
|- \_E = \{ <. x , y >. | x e. y \} \$.
\end{mmraw}

\noindent Define a founded relation\index{founded relation}.  $R$ is a founded
relation on $A$ iff\index{iff} (if and only if) each nonempty subset of $A$
has an ``$R$-minimal element.''  Similar to Definition 6.21, p.~30, of
Takeuti and Zaring.

\vskip 0.5ex
\setbox\startprefix=\hbox{\tt \ \ df-fr\ \$a\ }
\setbox\contprefix=\hbox{\tt \ \ \ \ \ \ \ \ \ \ \ }
\startm
\m{\vdash}\m{(}\m{R}\m{\,\mbox{\rm Fr}}\m{\,A}\m{\leftrightarrow}\m{\forall}\m{x}
\m{(}\m{(}\m{x}\m{\subseteq}\m{A}\m{\wedge}\m{\lnot}\m{x}\m{=}\m{\varnothing}
\m{)}\m{\rightarrow}\m{\exists}\m{y}\m{(}\m{y}\m{\in}\m{x}\m{\wedge}\m{(}\m{x}
\m{\cap}\m{\{}\m{z}\m{|}\m{z}\m{\,R}\m{\,y}\m{\}}\m{)}\m{=}\m{\varnothing}\m{)}
\m{)}\m{)}
\endm
\begin{mmraw}%
|- ( R Fr A <-> A. x ( ( x C\_ A \TAND -. x = (/) ) ->
E. y ( y e. x \TAND ( x i\^{}i \{ z | z R y \} ) = (/) ) ) ) \$.
\end{mmraw}

\noindent Define a well-ordering\index{well-ordering}.  $R$ is a well-ordering of $A$ iff
it is founded on $A$ and the elements of $A$ are pairwise $R$-comparable.
Similar to Definition 6.24(2), p.~30, of Takeuti and Zaring.

\vskip 0.5ex
\setbox\startprefix=\hbox{\tt \ \ df-we\ \$a\ }
\setbox\contprefix=\hbox{\tt \ \ \ \ \ \ \ \ \ \ \ }
\startm
\m{\vdash}\m{(}\m{R}\m{\,\mbox{\rm We}}\m{\,A}\m{\leftrightarrow}\m{(}\m{R}\m{\,
\mbox{\rm Fr}}\m{\,A}\m{\wedge}\m{\forall}\m{x}\m{\forall}\m{y}\m{(}\m{(}\m{x}\m{
\in}\m{A}\m{\wedge}\m{y}\m{\in}\m{A}\m{)}\m{\rightarrow}\m{(}\m{x}\m{\,R}\m{\,y}
\m{\vee}\m{x}\m{=}\m{y}\m{\vee}\m{y}\m{\,R}\m{\,x}\m{)}\m{)}\m{)}\m{)}
\endm
\begin{mmraw}%
( R We A <-> ( R Fr A \TAND A. x A. y ( ( x e.
A \TAND y e. A ) -> ( x R y \TOR x = y \TOR y R x ) ) ) ) \$.
\end{mmraw}

\noindent Define the ordinal predicate\index{ordinal predicate}, which is true for a class
that is transitive and is well-ordered by the epsilon relation.  Similar to
definition on p.~468, Bell and Machover.

\vskip 0.5ex
\setbox\startprefix=\hbox{\tt \ \ df-ord\ \$a\ }
\setbox\contprefix=\hbox{\tt \ \ \ \ \ \ \ \ \ \ \ \ }
\startm
\m{\vdash}\m{(}\m{\mbox{\rm Ord}}\m{\,A}\m{\leftrightarrow}\m{(}
\m{\mbox{\rm Tr}}\m{\,A}\m{\wedge}\m{E}\m{\,\mbox{\rm We}}\m{\,A}\m{)}\m{)}
\endm
\begin{mmraw}%
|- ( Ord A <-> ( Tr A \TAND E We A ) ) \$.
\end{mmraw}

\noindent Define the class of all ordinal numbers\index{ordinal number}.  An ordinal number is
a set that satisfies the ordinal predicate.  Definition 7.11 of Takeuti and
Zaring, p.~38.

\vskip 0.5ex
\setbox\startprefix=\hbox{\tt \ \ df-on\ \$a\ }
\setbox\contprefix=\hbox{\tt \ \ \ \ \ \ \ \ \ \ \ }
\startm
\m{\vdash}\m{\,\mbox{\rm On}}\m{=}\m{\{}\m{x}\m{|}\m{\mbox{\rm Ord}}\m{\,x}
\m{\}}
\endm
\begin{mmraw}%
|- On = \{ x | Ord x \} \$.
\end{mmraw}

\noindent Define the limit ordinal predicate\index{limit ordinal}, which is true for a
non-empty ordinal that is not a successor (i.e.\ that is the union of itself).
Compare Bell and Machover, p.~471 and Exercise (1), p.~42 of Takeuti and
Zaring.

\vskip 0.5ex
\setbox\startprefix=\hbox{\tt \ \ df-lim\ \$a\ }
\setbox\contprefix=\hbox{\tt \ \ \ \ \ \ \ \ \ \ \ \ }
\startm
\m{\vdash}\m{(}\m{\mbox{\rm Lim}}\m{\,A}\m{\leftrightarrow}\m{(}\m{\mbox{
\rm Ord}}\m{\,A}\m{\wedge}\m{\lnot}\m{A}\m{=}\m{\varnothing}\m{\wedge}\m{A}
\m{=}\m{\bigcup}\m{A}\m{)}\m{)}
\endm
\begin{mmraw}%
|- ( Lim A <-> ( Ord A \TAND -. A = (/) \TAND A = U. A ) ) \$.
\end{mmraw}

\noindent Define the successor\index{successor} of a class.  Definition 7.22 of Takeuti
and Zaring, p.~41.  Our definition is a generalization to classes, although it
is meaningless when classes are proper.

\vskip 0.5ex
\setbox\startprefix=\hbox{\tt \ \ df-suc\ \$a\ }
\setbox\contprefix=\hbox{\tt \ \ \ \ \ \ \ \ \ \ \ \ }
\startm
\m{\vdash}\m{\,\mbox{\rm suc}}\m{\,A}\m{=}\m{(}\m{A}\m{\cup}\m{\{}\m{A}\m{\}}
\m{)}
\endm
\begin{mmraw}%
|- suc A = ( A u. \{ A \} ) \$.
\end{mmraw}

\noindent Define the class of natural numbers\index{natural number}\index{omega
($\omega$)}.  Compare Bell and Machover, p.~471.\label{dfom}

\vskip 0.5ex
\setbox\startprefix=\hbox{\tt \ \ df-om\ \$a\ }
\setbox\contprefix=\hbox{\tt \ \ \ \ \ \ \ \ \ \ \ }
\startm
\m{\vdash}\m{\omega}\m{=}\m{\{}\m{x}\m{|}\m{(}\m{\mbox{\rm Ord}}\m{\,x}\m{
\wedge}\m{\forall}\m{y}\m{(}\m{\mbox{\rm Lim}}\m{\,y}\m{\rightarrow}\m{x}\m{
\in}\m{y}\m{)}\m{)}\m{\}}
\endm
\begin{mmraw}%
|- om = \{ x | ( Ord x \TAND A. y ( Lim y -> x e. y ) ) \} \$.
\end{mmraw}

\noindent Define the Cartesian product (also called the
cross product)\index{Cartesian product}\index{cross product}
of two classes.  Definition 9.11 of Quine, p.~64.

\vskip 0.5ex
\setbox\startprefix=\hbox{\tt \ \ df-xp\ \$a\ }
\setbox\contprefix=\hbox{\tt \ \ \ \ \ \ \ \ \ \ \ }
\startm
\m{\vdash}\m{(}\m{A}\m{\times}\m{B}\m{)}\m{=}\m{\{}\m{\langle}\m{x}\m{,}\m{y}
\m{\rangle}\m{|}\m{(}\m{x}\m{\in}\m{A}\m{\wedge}\m{y}\m{\in}\m{B}\m{)}\m{\}}
\endm
\begin{mmraw}%
|- ( A X. B ) = \{ <. x , y >. | ( x e. A \TAND y e. B) \} \$.
\end{mmraw}

\noindent Define a relation\index{relation}.  Definition 6.4(1) of Takeuti and
Zaring, p.~23.

\vskip 0.5ex
\setbox\startprefix=\hbox{\tt \ \ df-rel\ \$a\ }
\setbox\contprefix=\hbox{\tt \ \ \ \ \ \ \ \ \ \ \ \ }
\startm
\m{\vdash}\m{(}\m{\mbox{\rm Rel}}\m{\,A}\m{\leftrightarrow}\m{A}\m{\subseteq}
\m{(}\m{{\rm V}}\m{\times}\m{{\rm V}}\m{)}\m{)}
\endm
\begin{mmraw}%
|- ( Rel A <-> A C\_ ( {\char`\_}V X. {\char`\_}V ) ) \$.
\end{mmraw}

\noindent Define the domain\index{domain} of a class.  Definition 6.5(1) of
Takeuti and Zaring, p.~24.

\vskip 0.5ex
\setbox\startprefix=\hbox{\tt \ \ df-dm\ \$a\ }
\setbox\contprefix=\hbox{\tt \ \ \ \ \ \ \ \ \ \ \ }
\startm
\m{\vdash}\m{\,\mbox{\rm dom}}\m{A}\m{=}\m{\{}\m{x}\m{|}\m{\exists}\m{y}\m{
\langle}\m{x}\m{,}\m{y}\m{\rangle}\m{\in}\m{A}\m{\}}
\endm
\begin{mmraw}%
|- dom A = \{ x | E. y <. x , y >. e. A \} \$.
\end{mmraw}

\noindent Define the range\index{range} of a class.  Definition 6.5(2) of
Takeuti and Zaring, p.~24.

\vskip 0.5ex
\setbox\startprefix=\hbox{\tt \ \ df-rn\ \$a\ }
\setbox\contprefix=\hbox{\tt \ \ \ \ \ \ \ \ \ \ \ }
\startm
\m{\vdash}\m{\,\mbox{\rm ran}}\m{A}\m{=}\m{\{}\m{y}\m{|}\m{\exists}\m{x}\m{
\langle}\m{x}\m{,}\m{y}\m{\rangle}\m{\in}\m{A}\m{\}}
\endm
\begin{mmraw}%
|- ran A = \{ y | E. x <. x , y >. e. A \} \$.
\end{mmraw}

\noindent Define the restriction\index{restriction} of a class.  Definition
6.6(1) of Takeuti and Zaring, p.~24.

\vskip 0.5ex
\setbox\startprefix=\hbox{\tt \ \ df-res\ \$a\ }
\setbox\contprefix=\hbox{\tt \ \ \ \ \ \ \ \ \ \ \ \ }
\startm
\m{\vdash}\m{(}\m{A}\m{\restriction}\m{B}\m{)}\m{=}\m{(}\m{A}\m{\cap}\m{(}\m{B}
\m{\times}\m{{\rm V}}\m{)}\m{)}
\endm
\begin{mmraw}%
|- ( A |` B ) = ( A i\^{}i ( B X. {\char`\_}V ) ) \$.
\end{mmraw}

\noindent Define the image\index{image} of a class.  Definition 6.6(2) of
Takeuti and Zaring, p.~24.

\vskip 0.5ex
\setbox\startprefix=\hbox{\tt \ \ df-ima\ \$a\ }
\setbox\contprefix=\hbox{\tt \ \ \ \ \ \ \ \ \ \ \ \ }
\startm
\m{\vdash}\m{(}\m{A}\m{``}\m{B}\m{)}\m{=}\m{\,\mbox{\rm ran}}\m{\,(}\m{A}\m{
\restriction}\m{B}\m{)}
\endm
\begin{mmraw}%
|- ( A " B ) = ran ( A |` B ) \$.
\end{mmraw}

\noindent Define the composition\index{composition} of two classes.  Definition 6.6(3) of
Takeuti and Zaring, p.~24.

\vskip 0.5ex
\setbox\startprefix=\hbox{\tt \ \ df-co\ \$a\ }
\setbox\contprefix=\hbox{\tt \ \ \ \ \ \ \ \ \ \ \ \ }
\startm
\m{\vdash}\m{(}\m{A}\m{\circ}\m{B}\m{)}\m{=}\m{\{}\m{\langle}\m{x}\m{,}\m{y}\m{
\rangle}\m{|}\m{\exists}\m{z}\m{(}\m{\langle}\m{x}\m{,}\m{z}\m{\rangle}\m{\in}
\m{B}\m{\wedge}\m{\langle}\m{z}\m{,}\m{y}\m{\rangle}\m{\in}\m{A}\m{)}\m{\}}
\endm
\begin{mmraw}%
|- ( A o. B ) = \{ <. x , y >. | E. z ( <. x , z
>. e. B \TAND <. z , y >. e. A ) \} \$.
\end{mmraw}

\noindent Define a function\index{function}.  Definition 6.4(4) of Takeuti and
Zaring, p.~24.

\vskip 0.5ex
\setbox\startprefix=\hbox{\tt \ \ df-fun\ \$a\ }
\setbox\contprefix=\hbox{\tt \ \ \ \ \ \ \ \ \ \ \ \ }
\startm
\m{\vdash}\m{(}\m{\mbox{\rm Fun}}\m{\,A}\m{\leftrightarrow}\m{(}
\m{\mbox{\rm Rel}}\m{\,A}\m{\wedge}
\m{\forall}\m{x}\m{\exists}\m{z}\m{\forall}\m{y}\m{(}
\m{\langle}\m{x}\m{,}\m{y}\m{\rangle}\m{\in}\m{A}\m{\rightarrow}\m{y}\m{=}\m{z}
\m{)}\m{)}\m{)}
\endm
\begin{mmraw}%
|- ( Fun A <-> ( Rel A /\ A. x E. z A. y ( <. x
   , y >. e. A -> y = z ) ) ) \$.
\end{mmraw}

\noindent Define a function with domain.  Definition 6.15(1) of Takeuti and
Zaring, p.~27.

\vskip 0.5ex
\setbox\startprefix=\hbox{\tt \ \ df-fn\ \$a\ }
\setbox\contprefix=\hbox{\tt \ \ \ \ \ \ \ \ \ \ \ }
\startm
\m{\vdash}\m{(}\m{A}\m{\,\mbox{\rm Fn}}\m{\,B}\m{\leftrightarrow}\m{(}
\m{\mbox{\rm Fun}}\m{\,A}\m{\wedge}\m{\mbox{\rm dom}}\m{\,A}\m{=}\m{B}\m{)}
\m{)}
\endm
\begin{mmraw}%
|- ( A Fn B <-> ( Fun A \TAND dom A = B ) ) \$.
\end{mmraw}

\noindent Define a function with domain and co-domain.  Definition 6.15(3)
of Takeuti and Zaring, p.~27.

\vskip 0.5ex
\setbox\startprefix=\hbox{\tt \ \ df-f\ \$a\ }
\setbox\contprefix=\hbox{\tt \ \ \ \ \ \ \ \ \ \ }
\startm
\m{\vdash}\m{(}\m{F}\m{:}\m{A}\m{\longrightarrow}\m{B}\m{
\leftrightarrow}\m{(}\m{F}\m{\,\mbox{\rm Fn}}\m{\,A}\m{\wedge}\m{
\mbox{\rm ran}}\m{\,F}\m{\subseteq}\m{B}\m{)}\m{)}
\endm
\begin{mmraw}%
|- ( F : A --> B <-> ( F Fn A \TAND ran F C\_ B ) ) \$.
\end{mmraw}

\noindent Define a one-to-one function\index{one-to-one function}.  Compare
Definition 6.15(5) of Takeuti and Zaring, p.~27.

\vskip 0.5ex
\setbox\startprefix=\hbox{\tt \ \ df-f1\ \$a\ }
\setbox\contprefix=\hbox{\tt \ \ \ \ \ \ \ \ \ \ \ }
\startm
\m{\vdash}\m{(}\m{F}\m{:}\m{A}\m{
\raisebox{.5ex}{${\textstyle{\:}_{\mbox{\footnotesize\rm
1\tt -\rm 1}}}\atop{\textstyle{
\longrightarrow}\atop{\textstyle{}^{\mbox{\footnotesize\rm {\ }}}}}$}
}\m{B}
\m{\leftrightarrow}\m{(}\m{F}\m{:}\m{A}\m{\longrightarrow}\m{B}
\m{\wedge}\m{\forall}\m{y}\m{\exists}\m{z}\m{\forall}\m{x}\m{(}\m{\langle}\m{x}
\m{,}\m{y}\m{\rangle}\m{\in}\m{F}\m{\rightarrow}\m{x}\m{=}\m{z}\m{)}\m{)}\m{)}
\endm
\begin{mmraw}%
|- ( F : A -1-1-> B <-> ( F : A --> B \TAND
   A. y E. z A. x ( <. x , y >. e. F -> x = z ) ) ) \$.
\end{mmraw}

\noindent Define an onto function\index{onto function}.  Definition 6.15(4) of Takeuti and
Zaring, p.~27.

\vskip 0.5ex
\setbox\startprefix=\hbox{\tt \ \ df-fo\ \$a\ }
\setbox\contprefix=\hbox{\tt \ \ \ \ \ \ \ \ \ \ \ }
\startm
\m{\vdash}\m{(}\m{F}\m{:}\m{A}\m{
\raisebox{.5ex}{${\textstyle{\:}_{\mbox{\footnotesize\rm
{\ }}}}\atop{\textstyle{
\longrightarrow}\atop{\textstyle{}^{\mbox{\footnotesize\rm onto}}}}$}
}\m{B}
\m{\leftrightarrow}\m{(}\m{F}\m{\,\mbox{\rm Fn}}\m{\,A}\m{\wedge}
\m{\mbox{\rm ran}}\m{\,F}\m{=}\m{B}\m{)}\m{)}
\endm
\begin{mmraw}%
|- ( F : A -onto-> B <-> ( F Fn A /\ ran F = B ) ) \$.
\end{mmraw}

\noindent Define a one-to-one, onto function.  Compare Definition 6.15(6) of
Takeuti and Zaring, p.~27.

\vskip 0.5ex
\setbox\startprefix=\hbox{\tt \ \ df-f1o\ \$a\ }
\setbox\contprefix=\hbox{\tt \ \ \ \ \ \ \ \ \ \ \ \ }
\startm
\m{\vdash}\m{(}\m{F}\m{:}\m{A}
\m{
\raisebox{.5ex}{${\textstyle{\:}_{\mbox{\footnotesize\rm
1\tt -\rm 1}}}\atop{\textstyle{
\longrightarrow}\atop{\textstyle{}^{\mbox{\footnotesize\rm onto}}}}$}
}
\m{B}
\m{\leftrightarrow}\m{(}\m{F}\m{:}\m{A}
\m{
\raisebox{.5ex}{${\textstyle{\:}_{\mbox{\footnotesize\rm
1\tt -\rm 1}}}\atop{\textstyle{
\longrightarrow}\atop{\textstyle{}^{\mbox{\footnotesize\rm {\ }}}}}$}
}
\m{B}\m{\wedge}\m{F}\m{:}\m{A}
\m{
\raisebox{.5ex}{${\textstyle{\:}_{\mbox{\footnotesize\rm
{\ }}}}\atop{\textstyle{
\longrightarrow}\atop{\textstyle{}^{\mbox{\footnotesize\rm onto}}}}$}
}
\m{B}\m{)}\m{)}
\endm
\begin{mmraw}%
|- ( F : A -1-1-onto-> B <-> ( F : A -1-1-> B? \TAND F : A -onto-> B ) ) \$.?
\end{mmraw}

\noindent Define the value of a function\index{function value}.  This
definition applies to any class and evaluates to the empty set when it is not
meaningful. Note that $ F`A$ means the same thing as the more familiar $ F(A)$
notation for a function's value at $A$.  The $ F`A$ notation is common in
formal set theory.\label{df-fv}

\vskip 0.5ex
\setbox\startprefix=\hbox{\tt \ \ df-fv\ \$a\ }
\setbox\contprefix=\hbox{\tt \ \ \ \ \ \ \ \ \ \ \ }
\startm
\m{\vdash}\m{(}\m{F}\m{`}\m{A}\m{)}\m{=}\m{\bigcup}\m{\{}\m{x}\m{|}\m{(}\m{F}%
\m{``}\m{\{}\m{A}\m{\}}\m{)}\m{=}\m{\{}\m{x}\m{\}}\m{\}}
\endm
\begin{mmraw}%
|- ( F ` A ) = U. \{ x | ( F " \{ A \} ) = \{ x \} \} \$.
\end{mmraw}

\noindent Define the result of an operation.\index{operation}  Here, $F$ is
     an operation on two
     values (such as $+$ for real numbers).   This is defined for proper
     classes $A$ and $B$ even though not meaningful in that case.  However,
     the definition can be meaningful when $F$ is a proper class.\label{dfopr}

\vskip 0.5ex
\setbox\startprefix=\hbox{\tt \ \ df-opr\ \$a\ }
\setbox\contprefix=\hbox{\tt \ \ \ \ \ \ \ \ \ \ \ \ }
\startm
\m{\vdash}\m{(}\m{A}\m{\,F}\m{\,B}\m{)}\m{=}\m{(}\m{F}\m{`}\m{\langle}\m{A}%
\m{,}\m{B}\m{\rangle}\m{)}
\endm
\begin{mmraw}%
|- ( A F B ) = ( F ` <. A , B >. ) \$.
\end{mmraw}

\section{Tricks of the Trade}\label{tricks}

In the \texttt{set.mm}\index{set theory database (\texttt{set.mm})} database our goal
was usually to conform to modern notation.  However in some cases the
relationship to standard textbook language may be obscured by several
unconventional devices we used to simplify the development and to take
advantage of the Metamath language.  In this section we will describe some
common conventions used in \texttt{set.mm}.

\begin{itemize}
\item
The turnstile\index{turnstile ({$\,\vdash$})} symbol, $\vdash$, meaning ``it
is provable that,'' is the first token of all assertions and hypotheses that
aren't syntax constructions.  This is a standard convention in logic.  (We
mentioned this earlier, but this symbol is bothersome to some people without a
logic background.  It has no deeper meaning but just provides us with a way to
distinguish syntax constructions from ordinary mathematical statements.)

\item
A hypothesis of the form

\vskip 1ex
\setbox\startprefix=\hbox{\tt \ \ \ \ \ \ \ \ \ \$e\ }
\setbox\contprefix=\hbox{\tt \ \ \ \ \ \ \ \ \ \ }
\startm
\m{\vdash}\m{(}\m{\varphi}\m{\rightarrow}\m{\forall}\m{x}\m{\varphi}\m{)}
\endm
\vskip 1ex

should be read ``assume variable $x$ is (effectively) not free in wff
$\varphi$.''\index{effectively not free}
Literally, this says ``assume it is provable that $\varphi \rightarrow \forall
x\, \varphi$.''  This device lets us avoid the complexities associated with
the standard treatment of free and bound variables.
%% Uncomment this when uncommenting section {formalspec} below
The footnote on p.~\pageref{effectivelybound} discusses this further.

\item
A statement of one of the forms

\vskip 1ex
\setbox\startprefix=\hbox{\tt \ \ \ \ \ \ \ \ \ \$a\ }
\setbox\contprefix=\hbox{\tt \ \ \ \ \ \ \ \ \ \ }
\startm
\m{\vdash}\m{(}\m{\lnot}\m{\forall}\m{x}\m{\,x}\m{=}\m{y}\m{\rightarrow}
\m{\ldots}\m{)}
\endm
\setbox\startprefix=\hbox{\tt \ \ \ \ \ \ \ \ \ \$p\ }
\setbox\contprefix=\hbox{\tt \ \ \ \ \ \ \ \ \ \ }
\startm
\m{\vdash}\m{(}\m{\lnot}\m{\forall}\m{x}\m{\,x}\m{=}\m{y}\m{\rightarrow}
\m{\ldots}\m{)}
\endm
\vskip 1ex

should be read ``if $x$ and $y$ are distinct variables, then...''  This
antecedent provides us with a technical device to avoid the need for the
\texttt{\$d} statement early in our development of predicate calculus,
permitting symbol manipulations to be as conceptually simple as those in
propositional calculus.  However, the \texttt{\$d} statement eventually
becomes a requirement, and after that this device is rarely used.

\item
The statement

\vskip 1ex
\setbox\startprefix=\hbox{\tt \ \ \ \ \ \ \ \ \ \$d\ }
\setbox\contprefix=\hbox{\tt \ \ \ \ \ \ \ \ \ \ }
\startm
\m{x}\m{\,y}
\endm
\vskip 1ex

should be read ``assume $x$ and $y$ are distinct variables.''

\item
The statement

\vskip 1ex
\setbox\startprefix=\hbox{\tt \ \ \ \ \ \ \ \ \ \$d\ }
\setbox\contprefix=\hbox{\tt \ \ \ \ \ \ \ \ \ \ }
\startm
\m{x}\m{\,\varphi}
\endm
\vskip 1ex

should be read ``assume $x$ does not occur in $\varphi$.''

\item
The statement

\vskip 1ex
\setbox\startprefix=\hbox{\tt \ \ \ \ \ \ \ \ \ \$d\ }
\setbox\contprefix=\hbox{\tt \ \ \ \ \ \ \ \ \ \ }
\startm
\m{x}\m{\,A}
\endm
\vskip 1ex

should be read ``assume variable $x$ does not occur in class $A$.''

\item
The restriction and hypothesis group

\vskip 1ex
\setbox\startprefix=\hbox{\tt \ \ \ \ \ \ \ \ \ \$d\ }
\setbox\contprefix=\hbox{\tt \ \ \ \ \ \ \ \ \ \ }
\startm
\m{x}\m{\,A}
\endm
\setbox\startprefix=\hbox{\tt \ \ \ \ \ \ \ \ \ \$d\ }
\setbox\contprefix=\hbox{\tt \ \ \ \ \ \ \ \ \ \ }
\startm
\m{x}\m{\,\psi}
\endm
\setbox\startprefix=\hbox{\tt \ \ \ \ \ \ \ \ \ \$e\ }
\setbox\contprefix=\hbox{\tt \ \ \ \ \ \ \ \ \ \ }
\startm
\m{\vdash}\m{(}\m{x}\m{=}\m{A}\m{\rightarrow}\m{(}\m{\varphi}\m{\leftrightarrow}
\m{\psi}\m{)}\m{)}
\endm
\vskip 1ex

is frequently used in place of explicit substitution, meaning ``assume
$\psi$ results from the proper substitution of $A$ for $x$ in
$\varphi$.''  Sometimes ``\texttt{\$e} $\vdash ( \psi \rightarrow
\forall x \, \psi )$'' is used instead of ``\texttt{\$d} $x\, \psi $,''
which requires only that $x$ be effectively not free in $\varphi$ but
not necessarily absent from it.  The use of implicit
substitution\index{substitution!implicit} is partly a matter of personal
style, although it may make proofs somewhat shorter than would be the
case with explicit substitution.

\item
The hypothesis


\vskip 1ex
\setbox\startprefix=\hbox{\tt \ \ \ \ \ \ \ \ \ \$e\ }
\setbox\contprefix=\hbox{\tt \ \ \ \ \ \ \ \ \ \ }
\startm
\m{\vdash}\m{A}\m{\in}\m{{\rm V}}
\endm
\vskip 1ex

should be read ``assume class $A$ is a set (i.e.\ exists).''
This is a convenient convention used by Quine.

\item
The restriction and hypothesis

\vskip 1ex
\setbox\startprefix=\hbox{\tt \ \ \ \ \ \ \ \ \ \$d\ }
\setbox\contprefix=\hbox{\tt \ \ \ \ \ \ \ \ \ \ }
\startm
\m{x}\m{\,y}
\endm
\setbox\startprefix=\hbox{\tt \ \ \ \ \ \ \ \ \ \$e\ }
\setbox\contprefix=\hbox{\tt \ \ \ \ \ \ \ \ \ \ }
\startm
\m{\vdash}\m{(}\m{y}\m{\in}\m{A}\m{\rightarrow}\m{\forall}\m{x}\m{\,y}
\m{\in}\m{A}\m{)}
\endm
\vskip 1ex

should be read ``assume variable $x$ is
(effectively) not free in class $A$.''

\end{itemize}

\section{A Theorem Sampler}\label{sometheorems}

In this section we list some of the more important theorems that are proved in
the \texttt{set.mm} database, and they illustrate the kinds of things that can be
done with Metamath.  While all of these facts are well-known results,
Metamath offers the advantage of easily allowing you to trace their
derivation back to axioms.  Our intent here is not to try to explain the
details or motivation; for this we refer you to the textbooks that are
mentioned in the descriptions.  (The \texttt{set.mm} file has bibliographic
references for the text references.)  Their proofs often embody important
concepts you may wish to explore with the Metamath program (see
Section~\ref{exploring}).  All the symbols that are used here are defined in
Section~\ref{hierarchy}.  For brevity we haven't included the \texttt{\$d}
restrictions or \texttt{\$f} hypotheses for these theorems; when you are
uncertain consult the \texttt{set.mm} database.

We start with \texttt{syl} (principle of the syllogism).
In \textit{Principia Mathematica}
Whitehead and Russell call this ``the principle of the
syllogism... because... the syllogism in Barbara is derived from them''
\cite[quote after Theorem *2.06 p.~101]{PM}.
Some authors call this law a ``hypothetical syllogism.''
At the time of this writing it is the most commonly referenced assertion in the
\texttt{set.mm} database.\footnote{
The Metamath program command \texttt{show usage}
shows the number of references.
}

\vskip 2ex
\noindent Theorem syl (principle of the syllogism)\index{Syllogism}%
\index{\texttt{syl}}\label{syl}.

\vskip 0.5ex
\setbox\startprefix=\hbox{\tt \ \ syl.1\ \$e\ }
\setbox\contprefix=\hbox{\tt \ \ \ \ \ \ \ \ \ \ \ }
\startm
\m{\vdash}\m{(}\m{\varphi}\m{ \rightarrow }\m{\psi}\m{)}
\endm
\setbox\startprefix=\hbox{\tt \ \ syl.2\ \$e\ }
\setbox\contprefix=\hbox{\tt \ \ \ \ \ \ \ \ \ \ \ }
\startm
\m{\vdash}\m{(}\m{\psi}\m{ \rightarrow }\m{\chi}\m{)}
\endm
\setbox\startprefix=\hbox{\tt \ \ syl\ \$p\ }
\setbox\contprefix=\hbox{\tt \ \ \ \ \ \ \ \ \ }
\startm
\m{\vdash}\m{(}\m{\varphi}\m{ \rightarrow }\m{\chi}\m{)}
\endm
\vskip 2ex

The following theorem is not very deep but provides us with a notational device
that is frequently used.  It allows us to use the expression ``$A \in V$'' as
a compact way of saying that class $A$ exists, i.e.\ is a set.

\vskip 2ex
\noindent Two ways to say ``$A$ is a set'':  $A$ is a member of the universe
$V$ if and only if $A$ exists (i.e.\ there exists a set equal to $A$).
Theorem 6.9 of Quine, p. 43.

\vskip 0.5ex
\setbox\startprefix=\hbox{\tt \ \ isset\ \$p\ }
\setbox\contprefix=\hbox{\tt \ \ \ \ \ \ \ \ \ \ \ }
\startm
\m{\vdash}\m{(}\m{A}\m{\in}\m{{\rm V}}\m{\leftrightarrow}\m{\exists}\m{x}\m{\,x}\m{=}
\m{A}\m{)}
\endm
\vskip 1ex

Next we prove the axioms of standard ZF set theory that were missing from our
axiom system.  From our point of view they are theorems since they
can be derived from the other axioms.

\vskip 2ex
\noindent Axiom of Separation\index{Axiom of Separation}
(Aussonderung)\index{Aussonderung} proved from the other axioms of ZF set
theory.  Compare Exercise 4 of Takeuti and Zaring, p.~22.

\vskip 0.5ex
\setbox\startprefix=\hbox{\tt \ \ inex1.1\ \$e\ }
\setbox\contprefix=\hbox{\tt \ \ \ \ \ \ \ \ \ \ \ \ \ \ \ }
\startm
\m{\vdash}\m{A}\m{\in}\m{{\rm V}}
\endm
\setbox\startprefix=\hbox{\tt \ \ inex\ \$p\ }
\setbox\contprefix=\hbox{\tt \ \ \ \ \ \ \ \ \ \ \ \ \ }
\startm
\m{\vdash}\m{(}\m{A}\m{\cap}\m{B}\m{)}\m{\in}\m{{\rm V}}
\endm
\vskip 1ex

\noindent Axiom of the Null Set\index{Axiom of the Null Set} proved from the
other axioms of ZF set theory. Corollary 5.16 of Takeuti and Zaring, p.~20.

\vskip 0.5ex
\setbox\startprefix=\hbox{\tt \ \ 0ex\ \$p\ }
\setbox\contprefix=\hbox{\tt \ \ \ \ \ \ \ \ \ \ \ \ }
\startm
\m{\vdash}\m{\varnothing}\m{\in}\m{{\rm V}}
\endm
\vskip 1ex

\noindent The Axiom of Pairing\index{Axiom of Pairing} proved from the other
axioms of ZF set theory.  Theorem 7.13 of Quine, p.~51.
\vskip 0.5ex
\setbox\startprefix=\hbox{\tt \ \ prex\ \$p\ }
\setbox\contprefix=\hbox{\tt \ \ \ \ \ \ \ \ \ \ \ \ \ \ }
\startm
\m{\vdash}\m{\{}\m{A}\m{,}\m{B}\m{\}}\m{\in}\m{{\rm V}}
\endm
\vskip 2ex

Next we will list some famous or important theorems that are proved in
the \texttt{set.mm} database.  None of them except \texttt{omex}
require the Axiom of Infinity, as you can verify with the \texttt{show
trace{\char`\_}back} Metamath command.

\vskip 2ex
\noindent The resolution of Russell's paradox\index{Russell's paradox}.  There
exists no set containing the set of all sets which are not members of
themselves.  Proposition 4.14 of Takeuti and Zaring, p.~14.

\vskip 0.5ex
\setbox\startprefix=\hbox{\tt \ \ ru\ \$p\ }
\setbox\contprefix=\hbox{\tt \ \ \ \ \ \ \ \ }
\startm
\m{\vdash}\m{\lnot}\m{\exists}\m{x}\m{\,x}\m{=}\m{\{}\m{y}\m{|}\m{\lnot}\m{y}
\m{\in}\m{y}\m{\}}
\endm
\vskip 1ex

\noindent Cantor's theorem\index{Cantor's theorem}.  No set can be mapped onto
its power set.  Compare Theorem 6B(b) of Enderton, p.~132.

\vskip 0.5ex
\setbox\startprefix=\hbox{\tt \ \ canth.1\ \$e\ }
\setbox\contprefix=\hbox{\tt \ \ \ \ \ \ \ \ \ \ \ \ \ }
\startm
\m{\vdash}\m{A}\m{\in}\m{{\rm V}}
\endm
\setbox\startprefix=\hbox{\tt \ \ canth\ \$p\ }
\setbox\contprefix=\hbox{\tt \ \ \ \ \ \ \ \ \ \ \ }
\startm
\m{\vdash}\m{\lnot}\m{F}\m{:}\m{A}\m{\raisebox{.5ex}{${\textstyle{\:}_{
\mbox{\footnotesize\rm {\ }}}}\atop{\textstyle{\longrightarrow}\atop{
\textstyle{}^{\mbox{\footnotesize\rm onto}}}}$}}\m{{\cal P}}\m{A}
\endm
\vskip 1ex

\noindent The Burali-Forti paradox\index{Burali-Forti paradox}.  No set
contains all ordinal numbers. Enderton, p.~194.  (Burali-Forti was one person,
not two.)

\vskip 0.5ex
\setbox\startprefix=\hbox{\tt \ \ onprc\ \$p\ }
\setbox\contprefix=\hbox{\tt \ \ \ \ \ \ \ \ \ \ \ \ }
\startm
\m{\vdash}\m{\lnot}\m{\mbox{\rm On}}\m{\in}\m{{\rm V}}
\endm
\vskip 1ex

\noindent Peano's postulates\index{Peano's postulates} for arithmetic.
Proposition 7.30 of Takeuti and Zaring, pp.~42--43.  The objects being
described are the members of $\omega$ i.e.\ the natural numbers 0, 1,
2,\ldots.  The successor\index{successor} operation suc means ``plus
one.''  \texttt{peano1} says that 0 (which is defined as the empty set)
is a natural number.  \texttt{peano2} says that if $A$ is a natural
number, so is $A+1$.  \texttt{peano3} says that 0 is not the successor
of any natural number.  \texttt{peano4} says that two natural numbers
are equal if and only if their successors are equal.  \texttt{peano5} is
essentially the same as mathematical induction.

\vskip 1ex
\setbox\startprefix=\hbox{\tt \ \ peano1\ \$p\ }
\setbox\contprefix=\hbox{\tt \ \ \ \ \ \ \ \ \ \ \ \ }
\startm
\m{\vdash}\m{\varnothing}\m{\in}\m{\omega}
\endm
\vskip 1.5ex

\setbox\startprefix=\hbox{\tt \ \ peano2\ \$p\ }
\setbox\contprefix=\hbox{\tt \ \ \ \ \ \ \ \ \ \ \ \ }
\startm
\m{\vdash}\m{(}\m{A}\m{\in}\m{\omega}\m{\rightarrow}\m{{\rm suc}}\m{A}\m{\in}%
\m{\omega}\m{)}
\endm
\vskip 1.5ex

\setbox\startprefix=\hbox{\tt \ \ peano3\ \$p\ }
\setbox\contprefix=\hbox{\tt \ \ \ \ \ \ \ \ \ \ \ \ }
\startm
\m{\vdash}\m{(}\m{A}\m{\in}\m{\omega}\m{\rightarrow}\m{\lnot}\m{{\rm suc}}%
\m{A}\m{=}\m{\varnothing}\m{)}
\endm
\vskip 1.5ex

\setbox\startprefix=\hbox{\tt \ \ peano4\ \$p\ }
\setbox\contprefix=\hbox{\tt \ \ \ \ \ \ \ \ \ \ \ \ }
\startm
\m{\vdash}\m{(}\m{(}\m{A}\m{\in}\m{\omega}\m{\wedge}\m{B}\m{\in}\m{\omega}%
\m{)}\m{\rightarrow}\m{(}\m{{\rm suc}}\m{A}\m{=}\m{{\rm suc}}\m{B}\m{%
\leftrightarrow}\m{A}\m{=}\m{B}\m{)}\m{)}
\endm
\vskip 1.5ex

\setbox\startprefix=\hbox{\tt \ \ peano5\ \$p\ }
\setbox\contprefix=\hbox{\tt \ \ \ \ \ \ \ \ \ \ \ \ }
\startm
\m{\vdash}\m{(}\m{(}\m{\varnothing}\m{\in}\m{A}\m{\wedge}\m{\forall}\m{x}\m{%
\in}\m{\omega}\m{(}\m{x}\m{\in}\m{A}\m{\rightarrow}\m{{\rm suc}}\m{x}\m{\in}%
\m{A}\m{)}\m{)}\m{\rightarrow}\m{\omega}\m{\subseteq}\m{A}\m{)}
\endm
\vskip 1.5ex

\noindent Finite Induction (mathematical induction).\index{finite
induction}\index{mathematical induction} The first hypothesis is the
basis and the second is the induction hypothesis.  Theorem Schema 22 of
Suppes, p.~136.

\vskip 0.5ex
\setbox\startprefix=\hbox{\tt \ \ findes.1\ \$e\ }
\setbox\contprefix=\hbox{\tt \ \ \ \ \ \ \ \ \ \ \ \ \ \ }
\startm
\m{\vdash}\m{[}\m{\varnothing}\m{/}\m{x}\m{]}\m{\varphi}
\endm
\setbox\startprefix=\hbox{\tt \ \ findes.2\ \$e\ }
\setbox\contprefix=\hbox{\tt \ \ \ \ \ \ \ \ \ \ \ \ \ \ }
\startm
\m{\vdash}\m{(}\m{x}\m{\in}\m{\omega}\m{\rightarrow}\m{(}\m{\varphi}\m{%
\rightarrow}\m{[}\m{{\rm suc}}\m{x}\m{/}\m{x}\m{]}\m{\varphi}\m{)}\m{)}
\endm
\setbox\startprefix=\hbox{\tt \ \ findes\ \$p\ }
\setbox\contprefix=\hbox{\tt \ \ \ \ \ \ \ \ \ \ \ \ }
\startm
\m{\vdash}\m{(}\m{x}\m{\in}\m{\omega}\m{\rightarrow}\m{\varphi}\m{)}
\endm
\vskip 1ex

\noindent Transfinite Induction with explicit substitution.  The first
hypothesis is the basis, the second is the induction hypothesis for
successors, and the third is the induction hypothesis for limit
ordinals.  Theorem Schema 4 of Suppes, p. 197.

\vskip 0.5ex
\setbox\startprefix=\hbox{\tt \ \ tfindes.1\ \$e\ }
\setbox\contprefix=\hbox{\tt \ \ \ \ \ \ \ \ \ \ \ \ \ \ \ }
\startm
\m{\vdash}\m{[}\m{\varnothing}\m{/}\m{x}\m{]}\m{\varphi}
\endm
\setbox\startprefix=\hbox{\tt \ \ tfindes.2\ \$e\ }
\setbox\contprefix=\hbox{\tt \ \ \ \ \ \ \ \ \ \ \ \ \ \ \ }
\startm
\m{\vdash}\m{(}\m{x}\m{\in}\m{{\rm On}}\m{\rightarrow}\m{(}\m{\varphi}\m{%
\rightarrow}\m{[}\m{{\rm suc}}\m{x}\m{/}\m{x}\m{]}\m{\varphi}\m{)}\m{)}
\endm
\setbox\startprefix=\hbox{\tt \ \ tfindes.3\ \$e\ }
\setbox\contprefix=\hbox{\tt \ \ \ \ \ \ \ \ \ \ \ \ \ \ \ }
\startm
\m{\vdash}\m{(}\m{{\rm Lim}}\m{y}\m{\rightarrow}\m{(}\m{\forall}\m{x}\m{\in}%
\m{y}\m{\varphi}\m{\rightarrow}\m{[}\m{y}\m{/}\m{x}\m{]}\m{\varphi}\m{)}\m{)}
\endm
\setbox\startprefix=\hbox{\tt \ \ tfindes\ \$p\ }
\setbox\contprefix=\hbox{\tt \ \ \ \ \ \ \ \ \ \ \ \ \ }
\startm
\m{\vdash}\m{(}\m{x}\m{\in}\m{{\rm On}}\m{\rightarrow}\m{\varphi}\m{)}
\endm
\vskip 1ex

\noindent Principle of Transfinite Recursion.\index{transfinite
recursion} Theorem 7.41 of Takeuti and Zaring, p.~47.  Transfinite
recursion is the key theorem that allows arithmetic of ordinals to be
rigorously defined, and has many other important uses as well.
Hypotheses \texttt{tfr.1} and \texttt{tfr.2} specify a certain (proper)
class $ F$.  The complicated definition of $ F$ is not important in
itself; what is important is that there be such an $ F$ with the
required properties, and we show this by displaying $ F$ explicitly.
\texttt{tfr1} states that $ F$ is a function whose domain is the set of
ordinal numbers.  \texttt{tfr2} states that any value of $ F$ is
completely determined by its previous values and the values of an
auxiliary function, $G$.  \texttt{tfr3} states that $ F$ is unique,
i.e.\ it is the only function that satisfies \texttt{tfr1} and
\texttt{tfr2}.  Note that $ f$ is an individual variable like $x$ and
$y$; it is just a mnemonic to remind us that $A$ is a collection of
functions.

\vskip 0.5ex
\setbox\startprefix=\hbox{\tt \ \ tfr.1\ \$e\ }
\setbox\contprefix=\hbox{\tt \ \ \ \ \ \ \ \ \ \ \ }
\startm
\m{\vdash}\m{A}\m{=}\m{\{}\m{f}\m{|}\m{\exists}\m{x}\m{\in}\m{{\rm On}}\m{(}%
\m{f}\m{{\rm Fn}}\m{x}\m{\wedge}\m{\forall}\m{y}\m{\in}\m{x}\m{(}\m{f}\m{`}%
\m{y}\m{)}\m{=}\m{(}\m{G}\m{`}\m{(}\m{f}\m{\restriction}\m{y}\m{)}\m{)}\m{)}%
\m{\}}
\endm
\setbox\startprefix=\hbox{\tt \ \ tfr.2\ \$e\ }
\setbox\contprefix=\hbox{\tt \ \ \ \ \ \ \ \ \ \ \ }
\startm
\m{\vdash}\m{F}\m{=}\m{\bigcup}\m{A}
\endm
\setbox\startprefix=\hbox{\tt \ \ tfr1\ \$p\ }
\setbox\contprefix=\hbox{\tt \ \ \ \ \ \ \ \ \ \ }
\startm
\m{\vdash}\m{F}\m{{\rm Fn}}\m{{\rm On}}
\endm
\setbox\startprefix=\hbox{\tt \ \ tfr2\ \$p\ }
\setbox\contprefix=\hbox{\tt \ \ \ \ \ \ \ \ \ \ }
\startm
\m{\vdash}\m{(}\m{z}\m{\in}\m{{\rm On}}\m{\rightarrow}\m{(}\m{F}\m{`}\m{z}%
\m{)}\m{=}\m{(}\m{G}\m{`}\m{(}\m{F}\m{\restriction}\m{z}\m{)}\m{)}\m{)}
\endm
\setbox\startprefix=\hbox{\tt \ \ tfr3\ \$p\ }
\setbox\contprefix=\hbox{\tt \ \ \ \ \ \ \ \ \ \ }
\startm
\m{\vdash}\m{(}\m{(}\m{B}\m{{\rm Fn}}\m{{\rm On}}\m{\wedge}\m{\forall}\m{x}\m{%
\in}\m{{\rm On}}\m{(}\m{B}\m{`}\m{x}\m{)}\m{=}\m{(}\m{G}\m{`}\m{(}\m{B}\m{%
\restriction}\m{x}\m{)}\m{)}\m{)}\m{\rightarrow}\m{B}\m{=}\m{F}\m{)}
\endm
\vskip 1ex

\noindent The existence of omega (the class of natural numbers).\index{natural
number}\index{omega ($\omega$)}\index{Axiom of Infinity}  Axiom 7 of Takeuti
and Zaring, p.~43.  (This is the only theorem in this section requiring the
Axiom of Infinity.)

\vskip 0.5ex
\setbox\startprefix=\hbox{\tt \
\ omex\ \$p\ }
\setbox\contprefix=\hbox{\tt \ \ \ \ \ \ \ \ \ \ }
\startm
\m{\vdash}\m{\omega}\m{\in}\m{{\rm V}}
\endm
%\vskip 2ex


\section{Axioms for Real and Complex Numbers}\label{real}
\index{real and complex numbers!axioms for}

This section presents the axioms for real and complex numbers.  Analysis
textbooks implicitly or explicitly use these axioms or their equivalents
as their starting point.  In the database \texttt{set.mm}, we define real
and complex numbers as (rather complicated) specific sets and derive these
axioms as {\em theorems} from the axioms of ZF set theory, using a method
called Dedekind cuts.  We omit the details of this construction, which you can
follow if you wish using the \texttt{set.mm} database in conjunction with the
textbooks referenced therein.

Once we prove those theorems, we then restate these proven theorems as axioms.
This lets us easily identify which axioms are needed for a particular complex number proof, without the obfuscation of the set theory used to derive them.
As a result,
the construction is actually unimportant other
than to show that sets exist that satisfy the axioms, and thus that the axioms
are consistent if ZF set theory is consistent.  When working with real numbers
you can think of them as being the actual sets resulting
from the construction (for definiteness), or you can
think of them as otherwise unspecified sets that happen to satisfy the axioms.
The derivation is not easy, but the fact that it works is quite remarkable
and lends support to the idea that ZFC set theory is all we need to
provide a foundation for essentially all of mathematics.

For the axioms we are given (or postulate) 8 classes:  $\mathbb{C}$ (the
set of complex numbers), $\mathbb{R}$ (the set of real numbers, a subset
of $\mathbb{C}$), $0$ (zero), $1$ (one), $i$ (square root of
$-1$), $+$ (plus), $\cdot$ (times), and
$<_{\mathbb{R}}$ (less than for just the real numbers).
Subtraction and division are defined terms and are not part of the
axioms; for their definitions see \texttt{set.mm}.

Note that the notation $(A+B)$ (and similarly $(A\cdot B)$) specifies a class
called an {\em operation},\index{operation} and is the function value of the
class $+$ at ordered pair $\langle A,B \rangle$.  An operation is defined by
statement \texttt{df-opr} on p.~\pageref{dfopr}.
The notation $A <_{\mathbb{R}} B$ specifies a
wff called a {\em binary relation}\index{binary relation} and means $\langle A,B \rangle \in \,<_{\mathbb{R}}$, as defined by statement \texttt{df-br} on p.~\pageref{dfbr}.

Our set of 8 given classes is assumed to satisfy the following 22 axioms
(in the axioms listed below, $<$ really means $<_{\mathbb{R}}$).

\vskip 2ex

\noindent 1. The real numbers are a subset of the complex numbers.

%\vskip 0.5ex
\setbox\startprefix=\hbox{\tt \ \ ax-resscn\ \$p\ }
\setbox\contprefix=\hbox{\tt \ \ \ \ \ \ \ \ \ \ \ \ \ \ }
\startm
\m{\vdash}\m{\mathbb{R}}\m{\subseteq}\m{\mathbb{C}}
\endm
%\vskip 1ex

\noindent 2. One is a complex number.

%\vskip 0.5ex
\setbox\startprefix=\hbox{\tt \ \ ax-1cn\ \$p\ }
\setbox\contprefix=\hbox{\tt \ \ \ \ \ \ \ \ \ \ \ }
\startm
\m{\vdash}\m{1}\m{\in}\m{\mathbb{C}}
\endm
%\vskip 1ex

\noindent 3. The imaginary number $i$ is a complex number.

%\vskip 0.5ex
\setbox\startprefix=\hbox{\tt \ \ ax-icn\ \$p\ }
\setbox\contprefix=\hbox{\tt \ \ \ \ \ \ \ \ \ \ \ }
\startm
\m{\vdash}\m{i}\m{\in}\m{\mathbb{C}}
\endm
%\vskip 1ex

\noindent 4. Complex numbers are closed under addition.

%\vskip 0.5ex
\setbox\startprefix=\hbox{\tt \ \ ax-addcl\ \$p\ }
\setbox\contprefix=\hbox{\tt \ \ \ \ \ \ \ \ \ \ \ \ \ }
\startm
\m{\vdash}\m{(}\m{(}\m{A}\m{\in}\m{\mathbb{C}}\m{\wedge}\m{B}\m{\in}\m{\mathbb{C}}%
\m{)}\m{\rightarrow}\m{(}\m{A}\m{+}\m{B}\m{)}\m{\in}\m{\mathbb{C}}\m{)}
\endm
%\vskip 1ex

\noindent 5. Real numbers are closed under addition.

%\vskip 0.5ex
\setbox\startprefix=\hbox{\tt \ \ ax-addrcl\ \$p\ }
\setbox\contprefix=\hbox{\tt \ \ \ \ \ \ \ \ \ \ \ \ \ \ }
\startm
\m{\vdash}\m{(}\m{(}\m{A}\m{\in}\m{\mathbb{R}}\m{\wedge}\m{B}\m{\in}\m{\mathbb{R}}%
\m{)}\m{\rightarrow}\m{(}\m{A}\m{+}\m{B}\m{)}\m{\in}\m{\mathbb{R}}\m{)}
\endm
%\vskip 1ex

\noindent 6. Complex numbers are closed under multiplication.

%\vskip 0.5ex
\setbox\startprefix=\hbox{\tt \ \ ax-mulcl\ \$p\ }
\setbox\contprefix=\hbox{\tt \ \ \ \ \ \ \ \ \ \ \ \ \ }
\startm
\m{\vdash}\m{(}\m{(}\m{A}\m{\in}\m{\mathbb{C}}\m{\wedge}\m{B}\m{\in}\m{\mathbb{C}}%
\m{)}\m{\rightarrow}\m{(}\m{A}\m{\cdot}\m{B}\m{)}\m{\in}\m{\mathbb{C}}\m{)}
\endm
%\vskip 1ex

\noindent 7. Real numbers are closed under multiplication.

%\vskip 0.5ex
\setbox\startprefix=\hbox{\tt \ \ ax-mulrcl\ \$p\ }
\setbox\contprefix=\hbox{\tt \ \ \ \ \ \ \ \ \ \ \ \ \ \ }
\startm
\m{\vdash}\m{(}\m{(}\m{A}\m{\in}\m{\mathbb{R}}\m{\wedge}\m{B}\m{\in}\m{\mathbb{R}}%
\m{)}\m{\rightarrow}\m{(}\m{A}\m{\cdot}\m{B}\m{)}\m{\in}\m{\mathbb{R}}\m{)}
\endm
%\vskip 1ex

\noindent 8. Multiplication of complex numbers is commutative.

%\vskip 0.5ex
\setbox\startprefix=\hbox{\tt \ \ ax-mulcom\ \$p\ }
\setbox\contprefix=\hbox{\tt \ \ \ \ \ \ \ \ \ \ \ \ \ \ }
\startm
\m{\vdash}\m{(}\m{(}\m{A}\m{\in}\m{\mathbb{C}}\m{\wedge}\m{B}\m{\in}\m{\mathbb{C}}%
\m{)}\m{\rightarrow}\m{(}\m{A}\m{\cdot}\m{B}\m{)}\m{=}\m{(}\m{B}\m{\cdot}\m{A}%
\m{)}\m{)}
\endm
%\vskip 1ex

\noindent 9. Addition of complex numbers is associative.

%\vskip 0.5ex
\setbox\startprefix=\hbox{\tt \ \ ax-addass\ \$p\ }
\setbox\contprefix=\hbox{\tt \ \ \ \ \ \ \ \ \ \ \ \ \ \ }
\startm
\m{\vdash}\m{(}\m{(}\m{A}\m{\in}\m{\mathbb{C}}\m{\wedge}\m{B}\m{\in}\m{\mathbb{C}}%
\m{\wedge}\m{C}\m{\in}\m{\mathbb{C}}\m{)}\m{\rightarrow}\m{(}\m{(}\m{A}\m{+}%
\m{B}\m{)}\m{+}\m{C}\m{)}\m{=}\m{(}\m{A}\m{+}\m{(}\m{B}\m{+}\m{C}\m{)}\m{)}%
\m{)}
\endm
%\vskip 1ex

\noindent 10. Multiplication of complex numbers is associative.

%\vskip 0.5ex
\setbox\startprefix=\hbox{\tt \ \ ax-mulass\ \$p\ }
\setbox\contprefix=\hbox{\tt \ \ \ \ \ \ \ \ \ \ \ \ \ \ }
\startm
\m{\vdash}\m{(}\m{(}\m{A}\m{\in}\m{\mathbb{C}}\m{\wedge}\m{B}\m{\in}\m{\mathbb{C}}%
\m{\wedge}\m{C}\m{\in}\m{\mathbb{C}}\m{)}\m{\rightarrow}\m{(}\m{(}\m{A}\m{\cdot}%
\m{B}\m{)}\m{\cdot}\m{C}\m{)}\m{=}\m{(}\m{A}\m{\cdot}\m{(}\m{B}\m{\cdot}\m{C}%
\m{)}\m{)}\m{)}
\endm
%\vskip 1ex

\noindent 11. Multiplication distributes over addition for complex numbers.

%\vskip 0.5ex
\setbox\startprefix=\hbox{\tt \ \ ax-distr\ \$p\ }
\setbox\contprefix=\hbox{\tt \ \ \ \ \ \ \ \ \ \ \ \ \ }
\startm
\m{\vdash}\m{(}\m{(}\m{A}\m{\in}\m{\mathbb{C}}\m{\wedge}\m{B}\m{\in}\m{\mathbb{C}}%
\m{\wedge}\m{C}\m{\in}\m{\mathbb{C}}\m{)}\m{\rightarrow}\m{(}\m{A}\m{\cdot}\m{(}%
\m{B}\m{+}\m{C}\m{)}\m{)}\m{=}\m{(}\m{(}\m{A}\m{\cdot}\m{B}\m{)}\m{+}\m{(}%
\m{A}\m{\cdot}\m{C}\m{)}\m{)}\m{)}
\endm
%\vskip 1ex

\noindent 12. The square of $i$ equals $-1$ (expressed as $i$-squared plus 1 is
0).

%\vskip 0.5ex
\setbox\startprefix=\hbox{\tt \ \ ax-i2m1\ \$p\ }
\setbox\contprefix=\hbox{\tt \ \ \ \ \ \ \ \ \ \ \ \ }
\startm
\m{\vdash}\m{(}\m{(}\m{i}\m{\cdot}\m{i}\m{)}\m{+}\m{1}\m{)}\m{=}\m{0}
\endm
%\vskip 1ex

\noindent 13. One and zero are distinct.

%\vskip 0.5ex
\setbox\startprefix=\hbox{\tt \ \ ax-1ne0\ \$p\ }
\setbox\contprefix=\hbox{\tt \ \ \ \ \ \ \ \ \ \ \ \ }
\startm
\m{\vdash}\m{1}\m{\ne}\m{0}
\endm
%\vskip 1ex

\noindent 14. One is an identity element for real multiplication.

%\vskip 0.5ex
\setbox\startprefix=\hbox{\tt \ \ ax-1rid\ \$p\ }
\setbox\contprefix=\hbox{\tt \ \ \ \ \ \ \ \ \ \ \ }
\startm
\m{\vdash}\m{(}\m{A}\m{\in}\m{\mathbb{R}}\m{\rightarrow}\m{(}\m{A}\m{\cdot}\m{1}%
\m{)}\m{=}\m{A}\m{)}
\endm
%\vskip 1ex

\noindent 15. Every real number has a negative.

%\vskip 0.5ex
\setbox\startprefix=\hbox{\tt \ \ ax-rnegex\ \$p\ }
\setbox\contprefix=\hbox{\tt \ \ \ \ \ \ \ \ \ \ \ \ \ \ }
\startm
\m{\vdash}\m{(}\m{A}\m{\in}\m{\mathbb{R}}\m{\rightarrow}\m{\exists}\m{x}\m{\in}%
\m{\mathbb{R}}\m{(}\m{A}\m{+}\m{x}\m{)}\m{=}\m{0}\m{)}
\endm
%\vskip 1ex

\noindent 16. Every nonzero real number has a reciprocal.

%\vskip 0.5ex
\setbox\startprefix=\hbox{\tt \ \ ax-rrecex\ \$p\ }
\setbox\contprefix=\hbox{\tt \ \ \ \ \ \ \ \ \ \ \ \ \ \ }
\startm
\m{\vdash}\m{(}\m{A}\m{\in}\m{\mathbb{R}}\m{\rightarrow}\m{(}\m{A}\m{\ne}\m{0}%
\m{\rightarrow}\m{\exists}\m{x}\m{\in}\m{\mathbb{R}}\m{(}\m{A}\m{\cdot}%
\m{x}\m{)}\m{=}\m{1}\m{)}\m{)}
\endm
%\vskip 1ex

\noindent 17. A complex number can be expressed in terms of two reals.

%\vskip 0.5ex
\setbox\startprefix=\hbox{\tt \ \ ax-cnre\ \$p\ }
\setbox\contprefix=\hbox{\tt \ \ \ \ \ \ \ \ \ \ \ \ }
\startm
\m{\vdash}\m{(}\m{A}\m{\in}\m{\mathbb{C}}\m{\rightarrow}\m{\exists}\m{x}\m{\in}%
\m{\mathbb{R}}\m{\exists}\m{y}\m{\in}\m{\mathbb{R}}\m{A}\m{=}\m{(}\m{x}\m{+}\m{(}%
\m{y}\m{\cdot}\m{i}\m{)}\m{)}\m{)}
\endm
%\vskip 1ex

\noindent 18. Ordering on reals satisfies strict trichotomy.

%\vskip 0.5ex
\setbox\startprefix=\hbox{\tt \ \ ax-pre-lttri\ \$p\ }
\setbox\contprefix=\hbox{\tt \ \ \ \ \ \ \ \ \ \ \ \ \ }
\startm
\m{\vdash}\m{(}\m{(}\m{A}\m{\in}\m{\mathbb{R}}\m{\wedge}\m{B}\m{\in}\m{\mathbb{R}}%
\m{)}\m{\rightarrow}\m{(}\m{A}\m{<}\m{B}\m{\leftrightarrow}\m{\lnot}\m{(}\m{A}%
\m{=}\m{B}\m{\vee}\m{B}\m{<}\m{A}\m{)}\m{)}\m{)}
\endm
%\vskip 1ex

\noindent 19. Ordering on reals is transitive.

%\vskip 0.5ex
\setbox\startprefix=\hbox{\tt \ \ ax-pre-lttrn\ \$p\ }
\setbox\contprefix=\hbox{\tt \ \ \ \ \ \ \ \ \ \ \ \ \ }
\startm
\m{\vdash}\m{(}\m{(}\m{A}\m{\in}\m{\mathbb{R}}\m{\wedge}\m{B}\m{\in}\m{\mathbb{R}}%
\m{\wedge}\m{C}\m{\in}\m{\mathbb{R}}\m{)}\m{\rightarrow}\m{(}\m{(}\m{A}\m{<}%
\m{B}\m{\wedge}\m{B}\m{<}\m{C}\m{)}\m{\rightarrow}\m{A}\m{<}\m{C}\m{)}\m{)}
\endm
%\vskip 1ex

\noindent 20. Ordering on reals is preserved after addition to both sides.

%\vskip 0.5ex
\setbox\startprefix=\hbox{\tt \ \ ax-pre-ltadd\ \$p\ }
\setbox\contprefix=\hbox{\tt \ \ \ \ \ \ \ \ \ \ \ \ \ }
\startm
\m{\vdash}\m{(}\m{(}\m{A}\m{\in}\m{\mathbb{R}}\m{\wedge}\m{B}\m{\in}\m{\mathbb{R}}%
\m{\wedge}\m{C}\m{\in}\m{\mathbb{R}}\m{)}\m{\rightarrow}\m{(}\m{A}\m{<}\m{B}\m{%
\rightarrow}\m{(}\m{C}\m{+}\m{A}\m{)}\m{<}\m{(}\m{C}\m{+}\m{B}\m{)}\m{)}\m{)}
\endm
%\vskip 1ex

\noindent 21. The product of two positive reals is positive.

%\vskip 0.5ex
\setbox\startprefix=\hbox{\tt \ \ ax-pre-mulgt0\ \$p\ }
\setbox\contprefix=\hbox{\tt \ \ \ \ \ \ \ \ \ \ \ \ \ \ }
\startm
\m{\vdash}\m{(}\m{(}\m{A}\m{\in}\m{\mathbb{R}}\m{\wedge}\m{B}\m{\in}\m{\mathbb{R}}%
\m{)}\m{\rightarrow}\m{(}\m{(}\m{0}\m{<}\m{A}\m{\wedge}\m{0}%
\m{<}\m{B}\m{)}\m{\rightarrow}\m{0}\m{<}\m{(}\m{A}\m{\cdot}\m{B}\m{)}%
\m{)}\m{)}
\endm
%\vskip 1ex

\noindent 22. A non-empty, bounded-above set of reals has a supremum.

%\vskip 0.5ex
\setbox\startprefix=\hbox{\tt \ \ ax-pre-sup\ \$p\ }
\setbox\contprefix=\hbox{\tt \ \ \ \ \ \ \ \ \ \ \ }
\startm
\m{\vdash}\m{(}\m{(}\m{A}\m{\subseteq}\m{\mathbb{R}}\m{\wedge}\m{A}\m{\ne}\m{%
\varnothing}\m{\wedge}\m{\exists}\m{x}\m{\in}\m{\mathbb{R}}\m{\forall}\m{y}\m{%
\in}\m{A}\m{\,y}\m{<}\m{x}\m{)}\m{\rightarrow}\m{\exists}\m{x}\m{\in}\m{%
\mathbb{R}}\m{(}\m{\forall}\m{y}\m{\in}\m{A}\m{\lnot}\m{x}\m{<}\m{y}\m{\wedge}\m{%
\forall}\m{y}\m{\in}\m{\mathbb{R}}\m{(}\m{y}\m{<}\m{x}\m{\rightarrow}\m{\exists}%
\m{z}\m{\in}\m{A}\m{\,y}\m{<}\m{z}\m{)}\m{)}\m{)}
\endm

% NOTE: The \m{...} expressions above could be represented as
% $ \vdash ( ( A \subseteq \mathbb{R} \wedge A \ne \varnothing \wedge \exists x \in \mathbb{R} \forall y \in A \,y < x ) \rightarrow \exists x \in \mathbb{R} ( \forall y \in A \lnot x < y \wedge \forall y \in \mathbb{R} ( y < x \rightarrow \exists z \in A \,y < z ) ) ) $

\vskip 2ex

This completes the set of axioms for real and complex numbers.  You may
wish to look at how subtraction, division, and decimal numbers
are defined in \texttt{set.mm}, and for fun look at the proof of $2+
2 = 4$ (theorem \texttt{2p2e4} in \texttt{set.mm})
as discussed in section \ref{2p2e4}.

In \texttt{set.mm} we define the non-negative integers $\mathbb{N}$, the integers
$\mathbb{Z}$, and the rationals $\mathbb{Q}$ as subsets of $\mathbb{R}$.  This leads
to the nice inclusion $\mathbb{N} \subseteq \mathbb{Z} \subseteq \mathbb{Q} \subseteq
\mathbb{R} \subseteq \mathbb{C}$, giving us a uniform framework in which, for
example, a property such as commutativity of complex number addition
automatically applies to integers.  The natural numbers $\mathbb{N}$
are different from the set $\omega$ we defined earlier, but both satisfy
Peano's postulates.

\subsubsection{Complex Number Axioms in Analysis Texts}

Most analysis texts construct complex numbers as ordered pairs of reals,
leading to construction-dependent properties that satisfy these axioms
but are not stated in their pure form.  (This is also done in
\texttt{set.mm} but our axioms are extracted from that construction.)
Other texts will simply state that $\mathbb{R}$ is a ``complete ordered
subfield of $\mathbb{C}$,'' leading to redundant axioms when this phrase
is completely expanded out.  In fact I have not seen a text with the
axioms in the explicit form above.
None of these axioms is unique individually, but this carefully worked out
collection of axioms is the result of years of work
by the Metamath community.

\subsubsection{Eliminating Unnecessary Complex Number Axioms}

We once had more axioms for real and complex numbers, but over years of time
we (the Metamath community)
have found ways to eliminate them (by proving them from other axioms)
or weaken them (by making weaker claims without reducing what
can be proved).
In particular, here are statements that used to be complex number
axioms but have since been formally proven (with Metamath) to be redundant:

\begin{itemize}
\item
  $\mathbb{C} \in V$.
  At one time this was listed as a ``complex number axiom''
  However, this is not properly speaking a complex number axiom,
  and in any case its proof uses axioms of set theory.
  Proven redundant by Mario Carneiro\index{Carneiro, Mario} on
  17-Nov-2014 (see \texttt{axcnex}).
\item
  $((A \in \mathbb{C} \land B \in \mathbb{C}$) $\rightarrow$
  $(A + B) = (B + A))$.
  Proved redundant by Eric Schmidt\index{Schmidt, Eric} on 19-Jun-2012,
  and formalized by Scott Fenton\index{Fenton, Scott} on 3-Jan-2013
  (see \texttt{addcom}).
\item
  $(A \in \mathbb{C} \rightarrow (A + 0) = A)$.
  Proved redundant by Eric Schmidt on 19-Jun-2012,
  and formalized by Scott Fenton on 3-Jan-2013
  (see \texttt{addid1}).
\item
  $(A \in \mathbb{C} \rightarrow \exists x \in \mathbb{C} (A + x) = 0)$.
  Proved redundant by Eric Schmidt and formalized on 21-May-2007
  (see \texttt{cnegex}).
\item
  $((A \in \mathbb{C} \land A \ne 0) \rightarrow \exists x \in \mathbb{C} (A \cdot x) = 1)$.
  Proved redundant by Eric Schmidt and formalized on 22-May-2007
  (see \texttt{recex}).
\item
  $0 \in \mathbb{R}$.
  Proved redundant by Eric Schmidt on 19-Feb-2005 and formalized 21-May-2007
  (see \texttt{0re}).
\end{itemize}

We could eliminate 0 as an axiomatic object by defining it as
$( ( i \cdot i ) + 1 )$
and replacing it with this expression throughout the axioms. If this
is done, axiom ax-i2m1 becomes redundant. However, the remaining axioms
would become longer and less intuitive.

Eric Schmidt's paper analyzing this axiom system \cite{Schmidt}
presented a proof that these remaining axioms,
with the possible exception of ax-mulcom, are independent of the others.
It is currently an open question if ax-mulcom is independent of the others.

\section{Two Plus Two Equals Four}\label{2p2e4}

Here is a proof that $2 + 2 = 4$, as proven in the theorem \texttt{2p2e4}
in the database \texttt{set.mm}.
This is a useful demonstration of what a Metamath proof can look like.
This proof may have more steps than you're used to, but each step is rigorously
proven all the way back to the axioms of logic and set theory.
This display was originally generated by the Metamath program
as an {\sc HTML} file.

In the table showing the proof ``Step'' is the sequential step number,
while its associated ``Expression'' is an expression that we have proved.
``Ref'' is the name of a theorem or axiom that justifies that expression,
and ``Hyp'' refers to previous steps (if any) that the theorem or axiom
needs so that we can use it.  Expressions are indented further than
the expressions that depend on them to show their interdependencies.

\begin{table}[!htbp]
\caption{Two plus two equals four}
\begin{tabular}{lllll}
\textbf{Step} & \textbf{Hyp} & \textbf{Ref} & \textbf{Expression} & \\
1  &       & df-2    & $ \; \; \vdash 2 = 1 + 1$  & \\
2  & 1     & oveq2i  & $ \; \vdash (2 + 2) = (2 + (1 + 1))$ & \\
3  &       & df-4    & $ \; \; \vdash 4 = (3 + 1)$ & \\
4  &       & df-3    & $ \; \; \; \vdash 3 = (2 + 1)$ & \\
5  & 4     & oveq1i  & $ \; \; \vdash (3 + 1) = ((2 + 1) + 1)$ & \\
6  &       & 2cn     & $ \; \; \; \vdash 2 \in \mathbb{C}$ & \\
7  &       & ax-1cn  & $ \; \; \; \vdash 1 \in \mathbb{C}$ & \\
8  & 6,7,7 & addassi & $ \; \; \vdash ((2 + 1) + 1) = (2 + (1 + 1))$ & \\
9  & 3,5,8 & 3eqtri  & $ \; \vdash 4 = (2 + (1 + 1))$ & \\
10 & 2,9   & eqtr4i  & $ \vdash (2 + 2) = 4$ & \\
\end{tabular}
\end{table}

Step 1 says that we can assert that $2 = 1 + 1$ because it is
justified by \texttt{df-2}.
What is \texttt{df-2}?
It is simply the definition of $2$, which in our system is defined as being
equal to $1 + 1$.  This shows how we can use definitions in proofs.

Look at Step 2 of the proof. In the Ref column, we see that it references
a previously proved theorem, \texttt{oveq2i}.
It turns out that
theorem \texttt{oveq2i} requires a
hypothesis, and in the Hyp column of Step 2 we indicate that Step 1 will
satisfy (match) this hypothesis.
If we looked at \texttt{oveq2i}
we would find that it proves that given some hypothesis
$A = B$, we can prove that $( C F A ) = ( C F B )$.
If we use \texttt{oveq2i} and apply step 1's result as the hypothesis,
that will mean that $A = 2$ and $B = ( 1 + 1 )$ within this use of
\texttt{oveq2i}.
We are free to select any value of $C$ and $F$ (subject to syntax constraints),
so we are free to select $C = 2$ and $F = +$,
producing our desired result,
$ (2 + 2) = (2 + (1 + 1))$.

Step 2 is an example of substitution.
In the end, every step in every proof uses only this one substitution rule.
All the rules of logic, and all the axioms, are expressed so that
they can be used via this one substitution rule.
So once you master substitution, you can master every Metamath proof,
no exceptions.

Each step is clear and can be immediately checked.
In the {\sc HTML} display you can even click on each reference to see why it is
justified, making it easy to see why the proof works.

\section{Deduction}\label{deduction}

Strictly speaking,
a deduction (also called an inference) is a kind of statement that needs
some hypotheses to be true in order for its conclusion to be true.
A theorem, on the other hand, has no hypotheses.
Informally we often call both of them theorems, but in this section we
will stick to the strict definitions.

It sometimes happens that we have proved a deduction of the form
$\varphi \Rightarrow \psi$\index{$\Rightarrow$}
(given hypothesis $\varphi$ we can prove $\psi$)
and we want to then prove a theorem of the form
$\varphi \rightarrow \psi$.

Converting a deduction (which uses a hypothesis) into a theorem
(which does not) is not as simple as you might think.
The deduction says, ``if we can prove $\varphi$ then we can prove $\psi$,''
which is in some sense weaker than saying
``$\varphi$ implies $\psi$.''
There is no axiom of logic that permits us to directly obtain the theorem
given the deduction.\footnote{
The conversion of a deduction to a theorem does not even hold in general
for quantum propositional calculus,
which is a weak subset of classical propositional calculus.
It has been shown that adding the Standard Deduction Theorem (discussed below)
to quantum propositional calculus turns it into classical
propositional calculus!
}

This is in contrast to going the other way.
If we have the theorem ($\varphi \rightarrow \psi$),
it is easy to recover the deduction
($\varphi \Rightarrow \psi$)
using modus ponens\index{modus ponens}
(\texttt{ax-mp}; see section \ref{axmp}).

In the following subsections we first discuss the standard deduction theorem
(the traditional but awkward way to convert deductions into theorems) and
the weak deduction theorem (a limited version of the standard deduction
theorem that is easier to use and was once widely used in
\texttt{set.mm}\index{set theory database (\texttt{set.mm})}).
In section \ref{deductionstyle} we discuss
deduction style, the newer approach we now recommend in most cases.
Deduction style uses ``deduction form,'' a form that
prefixes each hypothesis (other than definitions) and the
conclusion with a universal antecedent (``$\varphi \rightarrow$'').
Deduction style is widely used in \texttt{set.mm},
so it is useful to understand it and \textit{why} it is widely used.
Section \ref{naturaldeduction}
briefly discusses our approach for using natural deduction
within \texttt{set.mm},
as that approach is deeply related to deduction style.
We conclude with a summary of the strengths of
our approach, which we believe are compelling.

\subsection{The Standard Deduction Theorem}\label{standarddeductiontheorem}

It is possible to make use of information
contained in the deduction or its proof to assist us with the proof of
the related theorem.
In traditional logic books, there is a metatheorem called the
Deduction Theorem\index{Deduction Theorem}\index{Standard Deduction Theorem},
discovered independently by Herbrand and Tarski around 1930.
The Deduction Theorem, which we often call the Standard Deduction Theorem,
provides an algorithm for constructing a proof of a theorem from the
proof of its corresponding deduction. See, for example,
\cite[p.~56]{Margaris}\index{Margaris, Angelo}.
To construct a proof for a theorem, the
algorithm looks at each step in the proof of the original deduction and
rewrites the step with several steps wherein the hypothesis is eliminated
and becomes an antecedent.

In ordinary mathematics, no one actually carries out the algorithm,
because (in its most basic form) it involves an exponential explosion of
the number of proof steps as more hypotheses are eliminated. Instead,
the Standard Deduction Theorem is invoked simply to claim that it can
be done in principle, without actually doing it.
What's more, the algorithm is not as simple as it might first appear
when applying it rigorously.
There is a subtle restriction on the Standard Deduction Theorem
that must be taken into account involving the axiom of generalization
when working with predicate calculus (see the literature for more detail).

One of the goals of Metamath is to let you plainly see, with as few
underlying concepts as possible, how mathematics can be derived directly
from the axioms, and not indirectly according to some hidden rules
buried inside a program or understood only by logicians. If we added
the Standard Deduction Theorem to the language and proof verifier,
that would greatly complicate both and largely defeat Metamath's goal
of simplicity. In principle, we could show direct proofs by expanding
out the proof steps generated by the algorithm of the Standard Deduction
Theorem, but that is not feasible in practice because the number of proof
steps quickly becomes huge, even astronomical.
Since the algorithm of the Standard Deduction Theorem is driven by the proof,
we would have to go through that proof
all over again---starting from axioms---in order to obtain the theorem form.
In terms of proof length, there would be no savings over just
proving the theorem directly instead of first proving the deduction form.

\subsection{Weak Deduction Theorem}\label{weakdeductiontheorem}

We have developed
a more efficient method for proving a theorem from a deduction
that can be used instead of the Standard Deduction Theorem
in many (but not all) cases.
We call this more efficient method the
Weak Deduction Theorem\index{Weak Deduction Theorem}.\footnote{
There is also an unrelated ``Weak Deduction Theorem''
in the field of relevance logic, so to avoid confusion we could call
ours the ``Weak Deduction Theorem for Classical Logic.''}
Unlike the Standard Deduction Theorem, the Weak Deduction Theorem produces the
theorem directly from a special substitution instance of the deduction,
using a small, fixed number of steps roughly proportional to the length
of the final theorem.

If you come to a proof referencing the Weak Deduction Theorem
\texttt{dedth} (or one of its variants \texttt{dedthxx}),
here is how to follow the proof without getting into the details:
just click on the theorem referenced in the step
just before the reference to \texttt{dedth} and ignore everything else.
Theorem \texttt{dedth} simply turns a hypothesis into an antecedent
(i.e. the hypothesis followed by $\rightarrow$
is placed in front of the assertion, and the hypothesis
itself is eliminated) given certain conditions.

The Weak Deduction Theorem
eliminates a hypothesis $\varphi$, making it become an antecedent.
It does this by proving an expression
$ \varphi \rightarrow \psi $ given two hypotheses:
(1)
$ ( A = {\rm if} ( \varphi , A , B ) \rightarrow ( \varphi \leftrightarrow \chi ) ) $
and
(2) $\chi$.
Note that it requires that a proof exists for $\varphi$ when the class variable
$A$ is replaced with a specific class $B$. The hypothesis $\chi$
should be assigned to the inference.
You can see the details of the proof of the Weak Deduction Theorem
in theorem \texttt{dedth}.

The Weak Deduction Theorem
is probably easier to understand by studying proofs that make use of it.
For example, let's look at the proof of \texttt{renegcl}, which proves that
$ \vdash ( A \in \mathbb{R} \rightarrow - A \in \mathbb{R} )$:

\needspace{4\baselineskip}
\begin{longtabu} {l l l X}
\textbf{Step} & \textbf{Hyp} & \textbf{Ref} & \textbf{Expression} \\
  1 &  & negeq &
  $\vdash$ $($ $A$ $=$ ${\rm if}$ $($ $A$ $\in$ $\mathbb{R}$ $,$ $A$ $,$ $1$ $)$ $\rightarrow$
  $\textrm{-}$ $A$ $=$ $\textrm{-}$ ${\rm if}$ $($ $A$ $\in$ $\mathbb{R}$
  $,$ $A$ $,$ $1$ $)$ $)$ \\
 2 & 1 & eleq1d &
    $\vdash$ $($ $A$ $=$ ${\rm if}$ $($ $A$ $\in$ $\mathbb{R}$ $,$ $A$ $,$ $1$ $)$ $\rightarrow$ $($
    $\textrm{-}$ $A$ $\in$ $\mathbb{R}$ $\leftrightarrow$
    $\textrm{-}$ ${\rm if}$ $($ $A$ $\in$ $\mathbb{R}$ $,$ $A$ $,$ $1$ $)$ $\in$
    $\mathbb{R}$ $)$ $)$ \\
 3 &  & 1re & $\vdash 1 \in \mathbb{R}$ \\
 4 & 3 & elimel &
   $\vdash {\rm if} ( A \in \mathbb{R} , A , 1 ) \in \mathbb{R}$ \\
 5 & 4 & renegcli &
   $\vdash \textrm{-} {\rm if} ( A \in \mathbb{R} , A , 1 ) \in \mathbb{R}$ \\
 6 & 2,5 & dedth &
   $\vdash ( A \in \mathbb{R} \rightarrow \textrm{-} A \in \mathbb{R}$ ) \\
\end{longtabu}

The somewhat strange-looking steps in \texttt{renegcl} before step 5 are
technical stuff that makes this magic work, and they can be ignored
for a quick overview of the proof. To continue following the ``important''
part of the proof of \texttt{renegcl},
you can look at the reference to \texttt{renegcli} at step 5.

That said, let's briefly look at how
\texttt{renegcl} uses the
Weak Deduction Theorem (\texttt{dedth}) to do its job,
in case you want to do something similar or want understand it more deeply.
Let's work backwards in the proof of \texttt{renegcl}.
Step 6 applies \texttt{dedth} to produce our goal result
$ \vdash ( A \in \mathbb{R} \rightarrow\, - A \in \mathbb{R} )$.
This requires on the one hand the (substituted) deduction
\texttt{renegcli} in step 5.
By itself \texttt{renegcli} proves the deduction
$ \vdash A \in \mathbb{R} \Rightarrow\, \vdash - A \in \mathbb{R}$;
this is the deduction form we are trying to turn into theorem form,
and thus
\texttt{renegcli} has a separate hypothesis that must be fulfilled.
To fulfill the hypothesis of the invocation of
\texttt{renegcli} in step 5, it is eventually
reduced to the already proven theorem $1 \in \mathbb{R}$ in step 3.
Step 4 connects steps 3 and 5; step 4 invokes
\texttt{elimel}, a special case of \texttt{elimhyp} that eliminates
a membership hypothesis for the weak deduction theorem.
On the other hand, the equivalence of the conclusion of
\texttt{renegcl}
$( - A \in \mathbb{R} )$ and the substituted conclusion of
\texttt{renegcli} must be proven, which is done in steps 2 and 1.

The weak deduction theorem has limitations.
In particular, we must be able to prove a special case of the deduction's
hypothesis as a stand-alone theorem.
For example, we used $1 \in \mathbb{R}$ in step 3 of \texttt{renegcl}.

We used to use the weak deduction theorem
extensively within \texttt{set.mm}.
However, we now recommend applying ``deduction style''
instead in most cases, as deduction style is
often an easier and clearer approach.
Therefore, we will now describe deduction style.

\subsection{Deduction Style}\label{deductionstyle}

We now prefer to write assertions in ``deduction form''
instead of writing a proof that would require use of the standard or
weak deduction theorem.
We call this appraoch
``deduction style.''\index{deduction style}

It will be easier to explain this by first defining some terms:

\begin{itemize}
\item \textbf{closed form}\index{closed form}\index{forms!closed}:
A kind of assertion (theorem) with no hypotheses.
Typically its label has no special suffix.
An example is \texttt{unss}, which states:
$\vdash ( ( A \subseteq C \wedge B \subseteq C ) \leftrightarrow ( A \cup B )
\subseteq C )\label{eq:unss}$
\item \textbf{deduction form}\index{deduction form}\index{forms!deduction}:
A kind of assertion with one or more hypotheses
where the conclusion is an implication with
a wff variable as the antecedent (usually $\varphi$), and every hypothesis
(\$e statement)
is either (1) an implication with the same antecedent as the conclusion or
(2) a definition.
A definition
can be for a class variable (this is a class variable followed by ``='')
or a wff variable (this is a wff variable followed by $\leftrightarrow$);
class variable definitions are more common.
In practice, a proof
in deduction form will also contain many steps that are implications
where the antecedent is either that wff variable (normally $\varphi$)
or is
a conjunction (...$\land$...) including that wff variable ($\varphi$).
If an assertion is in deduction form, and other forms are also available,
then we suffix its label with ``d.''
An example is \texttt{unssd}, which states\footnote{
For brevity we show here (and in other places)
a $\&$\index{$\&$} between hypotheses\index{hypotheses}
and a $\Rightarrow$\index{$\Rightarrow$}\index{conclusion}
between the hypotheses and the conclusion.
This notation is technically not part of the Metamath language, but is
instead a convenient abbreviation to show both the hypotheses and conclusion.}:
$\vdash ( \varphi \rightarrow A \subseteq C )\quad\&\quad \vdash ( \varphi
    \rightarrow B \subseteq C )\quad\Rightarrow\quad \vdash ( \varphi
    \rightarrow ( A \cup B ) \subseteq C )\label{eq:unssd}$
\item \textbf{inference form}\index{inference form}\index{forms!inference}:
A kind of assertion with one or more hypotheses that is not in deduction form
(e.g., there is no common antecedent).
If an assertion is in inference form, and other forms are also available,
then we suffix its label with ``i.''
An example is \texttt{unssi}, which states:
$\vdash A \subseteq C\quad\&\quad \vdash B \subseteq C\quad\Rightarrow\quad
    \vdash ( A \cup B ) \subseteq C\label{eq:unssi}$
\end{itemize}

When using deduction style we express an assertion in deduction form.
This form prefixes each hypothesis (other than definitions) and the
conclusion with a universal antecedent (``$\varphi \rightarrow$'').
The antecedent (e.g., $\varphi$)
mimics the context handled in the deduction theorem, eliminating
the need to directly use the deduction theorem.

Once you have an assertion in deduction form, you can easily convert it
to inference form or closed form:

\begin{itemize}
\item To
prove some assertion Ti in inference form, given assertion Td in deduction
form, there is a simple mechanical process you can use. First take each
Ti hypothesis and insert a \texttt{T.} $\rightarrow$ prefix (``true implies'')
using \texttt{a1i}. You
can then use the existing assertion Td to prove the resulting conclusion
with a \texttt{T.} $\rightarrow$ prefix.
Finally, you can remove that prefix using \texttt{trud},
resulting in the conclusion you wanted to prove.
\item To
prove some assertion T in closed form, given assertion Td in deduction
form, there is another simple mechanical process you can use. First,
select an expression that is the conjunction (...$\land$...) of all of the
consequents of every hypothesis of Td. Next, prove that this expression
implies each of the separate hypotheses of Td in turn by eliminating
conjuncts (there are a variety of proven assertions to do this, including
\texttt{simpl},
\texttt{simpr},
\texttt{3simpa},
\texttt{3simpb},
\texttt{3simpc},
\texttt{simp1},
\texttt{simp2},
and
\texttt{simp3}).
If the
expression has nested conjunctions, inner conjuncts can be broken out by
chaining the above theorems with \texttt{syl}
(see section \ref{syl}).\footnote{
There are actually many theorems
(labeled simp* such as \texttt{simp333}) that break out inner conjuncts in one
step, but rather than learning them you can just use the chaining we
just described to prove them, and then let the Metamath program command
\texttt{minimize{\char`\_}with}\index{\texttt{minimize{\char`\_}with} command}
figure out the right ones needed to collapse them.}
As your final step, you can then apply the already-proven assertion Td
(which is in deduction form), proving assertion T in closed form.
\end{itemize}

We can also easily convert any assertion T in closed form to its related
assertion Ti in inference form by applying
modus ponens\index{modus ponens} (see section \ref{axmp}).

The deduction form antecedent can also be used to represent the context
necessary to support natural deduction systems, so we will now
discuss natural deduction.

\subsection{Natural Deduction}\label{naturaldeduction}

Natural deduction\index{natural deduction}
(ND) systems, as such, were originally introduced in
1934 by two logicians working independently: Ja\'skowski and Gentzen. ND
systems are supposed to reconstruct, in a formally proper way, traditional
ways of mathematical reasoning (such as conditional proof, indirect proof,
and proof by cases). As reconstructions they were naturally influenced
by previous work, and many specific ND systems and notations have been
developed since their original work.

There are many ND variants, but
Indrzejczak \cite[p.~31-32]{Indrzejczak}\index{Indrzejczak, Andrzej}
suggests that any natural deductive system must satisfy at
least these three criteria:

\begin{itemize}
\item ``There are some means for entering assumptions into a proof and
also for eliminating them. Usually it requires some bookkeeping devices
for indicating the scope of an assumption, and showing that a part of
a proof depending on eliminated assumption is discharged.
\item There are no (or, at least, a very limited set of) axioms, because
their role is taken over by the set of primitive rules for introduction
and elimination of logical constants which means that elementary
inferences instead of formulae are taken as primitive.
\item (A genuine) ND system admits a lot of freedom in proof construction
and possibility of applying several proof search strategies, like
conditional proof, proof by cases, proof by reductio ad absurdum etc.''
\end{itemize}

The Metamath Proof Explorer (MPE) as defined in \texttt{set.mm}
is fundamentally a Hilbert-style system.
That is, MPE is based on a larger number of axioms (compared
to natural deduction systems), a very small set of rules of inference
(modus ponens), and the context is not changed by the rules of inference
in the middle of a proof. That said, MPE proofs can be developed using
the natural deduction (ND) approach as originally developed by Ja\'skowski
and Gentzen.

The most common and recommended approach for applying ND in MPE is to use
deduction form\index{deduction form}%
\index{forms!deduction}
and apply the MPE proven assertions that are equivalent to ND rules.
For example, MPE's \texttt{jca} is equivalent to ND rule $\land$-I
(and-insertion).
We maintain a list of equivalences that you may consult.
This approach for applying an ND approach within MPE relies on Metamath's
wff metavariables in an essential way, and is described in more detail
in the presentation ``Natural Deductions in the Metamath Proof Language''
by Mario Carneiro \cite{CarneiroND}\index{Carneiro, Mario}.

In this style many steps are an implication, whose antecedent mimics
the context ($\Gamma$) of most ND systems. To add an assumption, simply add
it to the implication antecedent (typically using
\texttt{simpr}),
and use that
new antecedent for all later claims in the same scope. If you wish to
use an assertion in an ND hypothesis scope that is outside the current
ND hypothesis scope, modify the assertion so that the ND hypothesis
assumption is added to its antecedent (typically using \texttt{adantr}). Most
proof steps will be proved using rules that have hypotheses and results
of the form $\varphi \rightarrow$ ...

An example may make this clearer.
Let's show theorem 5.5 of
\cite[p.~18]{Clemente}\index{Clemente Laboreo, Daniel}
along with a line by line translation using the usual
translation of natural deduction (ND) in the Metamath Proof Explorer
(MPE) notation (this is proof \texttt{ex-natded5.5}).
The proof's original goal was to prove
$\lnot \psi$ given two hypotheses,
$( \psi \rightarrow \chi )$ and $ \lnot \chi$.
We will translate these statements into MPE deduction form
by prefixing them all with $\varphi \rightarrow$.
As a result, in MPE the goal is stated as
$( \varphi \rightarrow \lnot \psi )$, and the two hypotheses are stated as
$( \varphi \rightarrow ( \psi \rightarrow \chi ) )$ and
$( \varphi \rightarrow \lnot \chi )$.

The following table shows the proof in Fitch natural deduction style
and its MPE equivalent.
The \textit{\#} column shows the original numbering,
\textit{MPE\#} shows the number in the equivalent MPE proof
(which we will show later),
\textit{ND Expression} shows the original proof claim in ND notation,
and \textit{MPE Translation} shows its translation into MPE
as discussed in this section.
The final columns show the rationale in ND and MPE respectively.

\needspace{4\baselineskip}
{\setlength{\extrarowsep}{4pt} % Keep rows from being too close together
\begin{longtabu}   { @{} c c X X X X }
\textbf{\#} & \textbf{MPE\#} & \textbf{ND Ex\-pres\-sion} &
\textbf{MPE Trans\-lation} & \textbf{ND Ration\-ale} &
\textbf{MPE Ra\-tio\-nale} \\
\endhead

1 & 2;3 &
$( \psi \rightarrow \chi )$ &
$( \varphi \rightarrow ( \psi \rightarrow \chi ) )$ &
Given &
\$e; \texttt{adantr} to put in ND hypothesis \\

2 & 5 &
$ \lnot \chi$ &
$( \varphi \rightarrow \lnot \chi )$ &
Given &
\$e; \texttt{adantr} to put in ND hypothesis \\

3 & 1 &
... $\vert$ $\psi$ &
$( \varphi \rightarrow \psi )$ &
ND hypothesis assumption &
\texttt{simpr} \\

4 & 4 &
... $\chi$ &
$( ( \varphi \land \psi ) \rightarrow \chi )$ &
$\rightarrow$\,E 1,3 &
\texttt{mpd} 1,3 \\

5 & 6 &
... $\lnot \chi$ &
$( ( \varphi \land \psi ) \rightarrow \lnot \chi )$ &
IT 2 &
\texttt{adantr} 5 \\

6 & 7 &
$\lnot \psi$ &
$( \varphi \rightarrow \lnot \psi )$ &
$\land$\,I 3,4,5 &
\texttt{pm2.65da} 4,6 \\

\end{longtabu}
}


The original used Latin letters; we have replaced them with Greek letters
to follow Metamath naming conventions and so that it is easier to follow
the Metamath translation. The Metamath line-for-line translation of
this natural deduction approach precedes every line with an antecedent
including $\varphi$ and uses the Metamath equivalents of the natural deduction
rules. To add an assumption, the antecedent is modified to include it
(typically by using \texttt{adantr};
\texttt{simpr} is useful when you want to
depend directly on the new assumption, as is shown here).

In Metamath we can represent the two given statements as these hypotheses:

\needspace{2\baselineskip}
\begin{itemize}
\item ex-natded5.5.1 $\vdash ( \varphi \rightarrow ( \psi \rightarrow \chi ) )$
\item ex-natded5.5.2 $\vdash ( \varphi \rightarrow \lnot \chi )$
\end{itemize}

\needspace{4\baselineskip}
Here is the proof in Metamath as a line-by-line translation:

\begin{longtabu}   { l l l X }
\textbf{Step} & \textbf{Hyp} & \textbf{Ref} & \textbf{Ex\-pres\-sion} \\
\endhead
1 & & simpr & $\vdash ( ( \varphi \land \psi ) \rightarrow \psi )$ \\
2 & & ex-natded5.5.1 &
  $\vdash ( \varphi \rightarrow ( \psi \rightarrow \chi ) )$ \\
3 & 2 & adantr &
 $\vdash ( ( \varphi \land \psi ) \rightarrow ( \psi \rightarrow \chi ) )$ \\
4 & 1, 3 & mpd &
 $\vdash ( ( \varphi \land \psi ) \rightarrow \chi ) $ \\
5 & & ex-natded5.5.2 &
 $\vdash ( \varphi \rightarrow \lnot \chi )$ \\
6 & 5 & adantr &
 $\vdash ( ( \varphi \land \psi ) \rightarrow \lnot \chi )$ \\
7 & 4, 6 & pm2.65da &
 $\vdash ( \varphi \rightarrow \lnot \psi )$ \\
\end{longtabu}

Only using specific natural deduction rules directly can lead to very
long proofs, for exactly the same reason that only using axioms directly
in Hilbert-style proofs can lead to very long proofs.
If the goal is short and clear proofs,
then it is better to reuse already-proven assertions
in deduction form than to start from scratch each time
and using only basic natural deduction rules.

\subsection{Strengths of Our Approach}

As far as we know there is nothing else in the literature like either the
weak deduction theorem or Mario Carneiro\index{Carneiro, Mario}'s
natural deduction method.
In order to
transform a hypothesis into an antecedent, the literature's standard
``Deduction Theorem''\index{Deduction Theorem}\index{Standard Deduction Theorem}
requires metalogic outside of the notions provided
by the axiom system. We instead generally prefer to use Mario Carneiro's
natural deduction method, then use the weak deduction theorem in cases
where that is difficult to apply, and only then use the full standard
deduction theorem as a last resort.

The weak deduction theorem\index{Weak Deduction Theorem}
does not require any additional metalogic
but converts an inference directly into a closed form theorem, with
a rigorous proof that uses only the axiom system. Unlike the standard
Deduction Theorem, there is no implicit external justification that we
have to trust in order to use it.

Mario Carneiro's natural deduction\index{natural deduction}
method also does not require any new metalogical
notions. It avoids the Deduction Theorem's metalogic by prefixing the
hypotheses and conclusion of every would-be inference with a universal
antecedent (``$\varphi \rightarrow$'') from the very start.

We think it is impressive and satisfying that we can do so much in a
practical sense without stepping outside of our Hilbert-style axiom system.
Of course our axiomatization, which is in the form of schemes,
contains a metalogic of its own that we exploit. But this metalogic
is relatively simple, and for our Deduction Theorem alternatives,
we primarily use just the direct substitution of expressions for
metavariables.

\begin{sloppy}
\section{Exploring the Set The\-o\-ry Data\-base}\label{exploring}
\end{sloppy}
% NOTE: All examples performed in this section are
% recorded wtih "set width 61" % on set.mm as of 2019-05-28
% commit c1e7849557661260f77cfdf0f97ac4354fbb4f4d.

At this point you may wish to study the \texttt{set.mm}\index{set theory
database (\texttt{set.mm})} file in more detail.  Pay particular
attention to the assumptions needed to define wffs\index{well-formed
formula (wff)} (which are not included above), the variable types
(\texttt{\$f}\index{\texttt{\$f} statement} statements), and the
definitions that are introduced.  Start with some simple theorems in
propositional calculus, making sure you understand in detail each step
of a proof.  Once you get past the first few proofs and become familiar
with the Metamath language, any part of the \texttt{set.mm} database
will be as easy to follow, step by step, as any other part---you won't
have to undergo a ``quantum leap'' in mathematical sophistication to be
able to follow a deep proof in set theory.

Next, you may want to explore how concepts such as natural numbers are
defined and described.  This is probably best done in conjunction with
standard set theory textbooks, which can help give you a higher-level
understanding.  The \texttt{set.mm} database provides references that will get
you started.  From there, you will be on your way towards a very deep,
rigorous understanding of abstract mathematics.

The Metamath\index{Metamath} program can help you peruse a Metamath data\-base,
wheth\-er you are trying to figure out how a certain step follows in a proof or
just have a general curiosity.  We will go through some examples of the
commands, using the \texttt{set.mm}\index{set theory database (\texttt{set.mm})}
database provided with the Metamath software.  These should help get you
started.  See Chapter~\ref{commands} for a more detailed description of
the commands.  Note that we have included the full spelling of all commands to
prevent ambiguity with future commands.  In practice you may type just the
characters needed to specify each command keyword\index{command keyword}
unambiguously, often just one or two characters per keyword, and you don't
need to type them in upper case.

First run the Metamath program as described earlier.  You should see the
\verb/MM>/ prompt.  Read in the \texttt{set.mm} file:\index{\texttt{read}
command}

\begin{verbatim}
MM> read set.mm
Reading source file "set.mm"... 34554442 bytes
34554442 bytes were read into the source buffer.
The source has 155711 statements; 2254 are $a and 32250 are $p.
No errors were found.  However, proofs were not checked.
Type VERIFY PROOF * if you want to check them.
\end{verbatim}

As with most examples in this book, what you will see
will be slightly different because we are continuously
improving our databases (including \texttt{set.mm}).

Let's check the database integrity.  This may take a minute or two to run if
your computer is slow.

\begin{verbatim}
MM> verify proof *
0 10%  20%  30%  40%  50%  60%  70%  80%  90% 100%
..................................................
All proofs in the database were verified in 2.84 s.
\end{verbatim}

No errors were reported, so every proof is correct.

You need to know the names (labels) of theorems before you can look at them.
Often just examining the database file(s) with a text editor is the best
approach.  In \texttt{set.mm} there are many detailed comments, especially near
the beginning, that can help guide you. The \texttt{search} command in the
Metamath program is also handy.  The \texttt{comments} qualifier will list the
statements whose associated comment (the one immediately before it) contain a
string you give it.  For example, if you are studying Enderton's {\em Elements
of Set Theory} \cite{Enderton}\index{Enderton, Herbert B.} you may want to see
the references to it in the database.  The search string \texttt{enderton} is not
case sensitive.  (This will not show you all the database theorems that are in
Enderton's book because there is usually only one citation for a given
theorem, which may appear in several textbooks.)\index{\texttt{search}
command}

\begin{verbatim}
MM> search * "enderton" / comments
12067 unineq $p "... Exercise 20 of [Enderton] p. 32 and ..."
12459 undif2 $p "...Corollary 6K of [Enderton] p. 144. (C..."
12953 df-tp $a "...s. Definition of [Enderton] p. 19. (Co..."
13689 unissb $p ".... Exercise 5 of [Enderton] p. 26 and ..."
\end{verbatim}
\begin{center}
(etc.)
\end{center}

Or you may want to see what theorems have something to do with
conjunction (logical {\sc and}).  The quotes around the search
string are optional when there's no ambiguity.\index{\texttt{search}
command}

\begin{verbatim}
MM> search * conjunction / comments
120 a1d $p "...be replaced with a conjunction ( ~ df-an )..."
662 df-bi $a "...viated form after conjunction is introdu..."
1319 wa $a "...ff definition to include conjunction ('and')."
1321 df-an $a "Define conjunction (logical 'and'). Defini..."
1420 imnan $p "...tion in terms of conjunction. (Contribu..."
\end{verbatim}
\begin{center}
(etc.)
\end{center}


Now we will start to look at some details.  Let's look at the first
axiom of propositional calculus
(we could use \texttt{sh st} to abbreviate
\texttt{show statement}).\index{\texttt{show statement} command}

\begin{verbatim}
MM> show statement ax-1/full
Statement 19 is located on line 881 of the file "set.mm".
"Axiom _Simp_.  Axiom A1 of [Margaris] p. 49.  One of the 3
axioms of propositional calculus.  The 3 axioms are also
        ...
19 ax-1 $a |- ( ph -> ( ps -> ph ) ) $.
Its mandatory hypotheses in RPN order are:
  wph $f wff ph $.
  wps $f wff ps $.
The statement and its hypotheses require the variables:  ph
      ps
The variables it contains are:  ph ps


Statement 49 is located on line 11182 of the file "set.mm".
Its statement number for HTML pages is 6.
"Axiom _Simp_.  Axiom A1 of [Margaris] p. 49.  One of the 3
axioms of propositional calculus.  The 3 axioms are also
given as Definition 2.1 of [Hamilton] p. 28.
...
49 ax-1 $a |- ( ph -> ( ps -> ph ) ) $.
Its mandatory hypotheses in RPN order are:
  wph $f wff ph $.
  wps $f wff ps $.
The statement and its hypotheses require the variables:
  ph ps
The variables it contains are:  ph ps
\end{verbatim}

Compare this to \texttt{ax-1} on p.~\pageref{ax1}.  You can see that the
symbol \texttt{ph} is the {\sc ascii} notation for $\varphi$, etc.  To
see the mathematical symbols for any expression you may typeset it in
\LaTeX\ (type \texttt{help tex} for instructions)\index{latex@{\LaTeX}}
or, easier, just use a text editor to look at the comments where symbols
are first introduced in \texttt{set.mm}.  The hypotheses \texttt{wph}
and \texttt{wps} required by \texttt{ax-1} mean that variables
\texttt{ph} and \texttt{ps} must be wffs.

Next we'll pick a simple theorem of propositional calculus, the Principle of
Identity, which is proved directly from the axioms.  We'll look at the
statement then its proof.\index{\texttt{show statement}
command}

\begin{verbatim}
MM> show statement id1/full
Statement 116 is located on line 11371 of the file "set.mm".
Its statement number for HTML pages is 22.
"Principle of identity.  Theorem *2.08 of [WhiteheadRussell]
p. 101.  This version is proved directly from the axioms for
demonstration purposes.
...
116 id1 $p |- ( ph -> ph ) $= ... $.
Its mandatory hypotheses in RPN order are:
  wph $f wff ph $.
Its optional hypotheses are:  wps wch wth wta wet
      wze wsi wrh wmu wla wka
The statement and its hypotheses require the variables:  ph
These additional variables are allowed in its proof:
      ps ch th ta et ze si rh mu la ka
The variables it contains are:  ph
\end{verbatim}

The optional variables\index{optional variable} \texttt{ps}, \texttt{ch}, etc.\ are
available for use in a proof of this statement if we wish, and were we to do
so we would make use of optional hypotheses \texttt{wps}, \texttt{wch}, etc.  (See
Section~\ref{dollaref} for the meaning of ``optional
hypothesis.''\index{optional hypothesis}) The reason these show up in the
statement display is that statement \texttt{id1} happens to be in their scope
(see Section~\ref{scoping} for the definition of ``scope''\index{scope}), but
in fact in propositional calculus we will never make use of optional
hypotheses or variables.  This becomes important after quantifiers are
introduced, where ``dummy'' variables\index{dummy variable} are often needed
in the middle of a proof.

Let's look at the proof of statement \texttt{id1}.  We'll use the
\texttt{show proof} command, which by default suppresses the
``non-essential'' steps that construct the wffs.\index{\texttt{show proof}
command}
We will display the proof in ``lemmon' format (a non-indented format
with explicit previous step number references) and renumber the
displayed steps:

\begin{verbatim}
MM> show proof id1 /lemmon/renumber
1 ax-1           $a |- ( ph -> ( ph -> ph ) )
2 ax-1           $a |- ( ph -> ( ( ph -> ph ) -> ph ) )
3 ax-2           $a |- ( ( ph -> ( ( ph -> ph ) -> ph ) ) ->
                     ( ( ph -> ( ph -> ph ) ) -> ( ph -> ph )
                                                          ) )
4 2,3 ax-mp      $a |- ( ( ph -> ( ph -> ph ) ) -> ( ph -> ph
                                                          ) )
5 1,4 ax-mp      $a |- ( ph -> ph )
\end{verbatim}

If you have read Section~\ref{trialrun}, you'll know how to interpret this
proof.  Step~2, for example, is an application of axiom \texttt{ax-1}.  This
proof is identical to the one in Hamilton's {\em Logic for Mathematicians}
\cite[p.~32]{Hamilton}\index{Hamilton, Alan G.}.

You may want to look at what
substitutions\index{substitution!variable}\index{variable substitution} are
made into \texttt{ax-1} to arrive at step~2.  The command to do this needs to
know the ``real'' step number, so we'll display the proof again without
the \texttt{renumber} qualifier.\index{\texttt{show proof}
command}

\begin{verbatim}
MM> show proof id1 /lemmon
 9 ax-1          $a |- ( ph -> ( ph -> ph ) )
20 ax-1          $a |- ( ph -> ( ( ph -> ph ) -> ph ) )
24 ax-2          $a |- ( ( ph -> ( ( ph -> ph ) -> ph ) ) ->
                     ( ( ph -> ( ph -> ph ) ) -> ( ph -> ph )
                                                          ) )
25 20,24 ax-mp   $a |- ( ( ph -> ( ph -> ph ) ) -> ( ph -> ph
                                                          ) )
26 9,25 ax-mp    $a |- ( ph -> ph )
\end{verbatim}

The ``real'' step number is 20.  Let's look at its details.

\begin{verbatim}
MM> show proof id1 /detailed_step 20
Proof step 20:  min=ax-1 $a |- ( ph -> ( ( ph -> ph ) -> ph )
  )
This step assigns source "ax-1" ($a) to target "min" ($e).
The source assertion requires the hypotheses "wph" ($f, step
18) and "wps" ($f, step 19).  The parent assertion of the
target hypothesis is "ax-mp" ($a, step 25).
The source assertion before substitution was:
    ax-1 $a |- ( ph -> ( ps -> ph ) )
The following substitutions were made to the source
assertion:
    Variable  Substituted with
     ph        ph
     ps        ( ph -> ph )
The target hypothesis before substitution was:
    min $e |- ph
The following substitution was made to the target hypothesis:
    Variable  Substituted with
     ph        ( ph -> ( ( ph -> ph ) -> ph ) )
\end{verbatim}

This shows the substitutions\index{substitution!variable}\index{variable
substitution} made to the variables in \texttt{ax-1}.  References are made to
steps 18 and 19 which are not shown in our proof display.  To see these steps,
you can display the proof with the \texttt{all} qualifier.

Let's look at a slightly more advanced proof of propositional calculus.  Note
that \verb+/\+ is the symbol for $\wedge$ (logical {\sc and}, also
called conjunction).\index{conjunction ($\wedge$)}
\index{logical {\sc and} ($\wedge$)}

\begin{verbatim}
MM> show statement prth/full
Statement 1791 is located on line 15503 of the file "set.mm".
Its statement number for HTML pages is 559.
"Conjoin antecedents and consequents of two premises.  This
is the closed theorem form of ~ anim12d .  Theorem *3.47 of
[WhiteheadRussell] p. 113.  It was proved by Leibniz,
and it evidently pleased him enough to call it
_praeclarum theorema_ (splendid theorem).
...
1791 prth $p |- ( ( ( ph -> ps ) /\ ( ch -> th ) ) -> ( ( ph
      /\ ch ) -> ( ps /\ th ) ) ) $= ... $.
Its mandatory hypotheses in RPN order are:
  wph $f wff ph $.
  wps $f wff ps $.
  wch $f wff ch $.
  wth $f wff th $.
Its optional hypotheses are:  wta wet wze wsi wrh wmu wla wka
The statement and its hypotheses require the variables:  ph
      ps ch th
These additional variables are allowed in its proof:  ta et
      ze si rh mu la ka
The variables it contains are:  ph ps ch th


MM> show proof prth /lemmon/renumber
1 simpl          $p |- ( ( ( ph -> ps ) /\ ( ch -> th ) ) ->
                                               ( ph -> ps ) )
2 simpr          $p |- ( ( ( ph -> ps ) /\ ( ch -> th ) ) ->
                                               ( ch -> th ) )
3 1,2 anim12d    $p |- ( ( ( ph -> ps ) /\ ( ch -> th ) ) ->
                           ( ( ph /\ ch ) -> ( ps /\ th ) ) )
\end{verbatim}

There are references to a lot of unfamiliar statements.  To see what they are,
you may type the following:

\begin{verbatim}
MM> show proof prth /statement_summary
Summary of statements used in the proof of "prth":

Statement simpl is located on line 14748 of the file
"set.mm".
"Elimination of a conjunct.  Theorem *3.26 (Simp) of
[WhiteheadRussell] p. 112. ..."
  simpl $p |- ( ( ph /\ ps ) -> ph ) $= ... $.

Statement simpr is located on line 14777 of the file
"set.mm".
"Elimination of a conjunct.  Theorem *3.27 (Simp) of
[WhiteheadRussell] ..."
  simpr $p |- ( ( ph /\ ps ) -> ps ) $= ... $.

Statement anim12d is located on line 15445 of the file
"set.mm".
"Conjoin antecedents and consequents in a deduction.
..."
  anim12d.1 $e |- ( ph -> ( ps -> ch ) ) $.
  anim12d.2 $e |- ( ph -> ( th -> ta ) ) $.
  anim12d $p |- ( ph -> ( ( ps /\ th ) -> ( ch /\ ta ) ) )
      $= ... $.
\end{verbatim}
\begin{center}
(etc.)
\end{center}

Of course you can look at each of these statements and their proofs, and
so on, back to the axioms of propositional calculus if you wish.

The \texttt{search} command is useful for finding statements when you
know all or part of their contents.  The following example finds all
statements containing \verb@ph -> ps@ followed by \verb@ch -> th@.  The
\verb@$*@ is a wildcard that matches anything; the \texttt{\$} before the
\verb$*$ prevents conflicts with math symbol token names.  The \verb@*@ after
\texttt{SEARCH} is also a wildcard that in this case means ``match any label.''
\index{\texttt{search} command}

% I'm omitting this one, since readers are unlikely to see it:
% 1096 bisymOLD $p |- ( ( ( ph -> ps ) -> ( ch -> th ) ) -> ( (
%   ( ps -> ph ) -> ( th -> ch ) ) -> ( ( ph <-> ps ) -> ( ch
%    <-> th ) ) ) )
\begin{verbatim}
MM> search * "ph -> ps $* ch -> th"
1791 prth $p |- ( ( ( ph -> ps ) /\ ( ch -> th ) ) -> ( ( ph
    /\ ch ) -> ( ps /\ th ) ) )
2455 pm3.48 $p |- ( ( ( ph -> ps ) /\ ( ch -> th ) ) -> ( (
    ph \/ ch ) -> ( ps \/ th ) ) )
117859 pm11.71 $p |- ( ( E. x ph /\ E. y ch ) -> ( ( A. x (
    ph -> ps ) /\ A. y ( ch -> th ) ) <-> A. x A. y ( ( ph /\
    ch ) -> ( ps /\ th ) ) ) )
\end{verbatim}

Three statements, \texttt{prth}, \texttt{pm3.48},
 and \texttt{pm11.71}, were found to match.

To see what axioms\index{axiom} and definitions\index{definition}
\texttt{prth} ultimately depends on for its proof, you can have the
program backtrack through the hierarchy\index{hierarchy} of theorems and
definitions.\index{\texttt{show trace{\char`\_}back} command}

\begin{verbatim}
MM> show trace_back prth /essential/axioms
Statement "prth" assumes the following axioms ($a
statements):
  ax-1 ax-2 ax-3 ax-mp df-bi df-an
\end{verbatim}

Note that the 3 axioms of propositional calculus and the modus ponens rule are
needed (as expected); in addition, there are a couple of definitions that are used
along the way.  Note that Metamath makes no distinction\index{axiom vs.\
definition} between axioms\index{axiom} and definitions\index{definition}.  In
\texttt{set.mm} they have been distinguished artificially by prefixing their
labels\index{labels in \texttt{set.mm}} with \texttt{ax-} and \texttt{df-}
respectively.  For example, \texttt{df-an} defines conjunction (logical {\sc
and}), which is represented by the symbol \verb+/\+.
Section~\ref{definitions} discusses the philosophy of definitions, and the
Metamath language takes a particularly simple, conservative approach by using
the \texttt{\$a}\index{\texttt{\$a} statement} statement for both axioms and
definitions.

You can also have the program compute how many steps a proof
has\index{proof length} if we were to follow it all the way back to
\texttt{\$a} statements.

\begin{verbatim}
MM> show trace_back prth /essential/count_steps
The statement's actual proof has 3 steps.  Backtracking, a
total of 79 different subtheorems are used.  The statement
and subtheorems have a total of 274 actual steps.  If
subtheorems used only once were eliminated, there would be a
total of 38 subtheorems, and the statement and subtheorems
would have a total of 185 steps.  The proof would have 28349
steps if fully expanded back to axiom references.  The
maximum path length is 38.  A longest path is:  prth <-
anim12d <- syl2and <- sylan2d <- ancomsd <- ancom <- pm3.22
<- pm3.21 <- pm3.2 <- ex <- sylbir <- biimpri <- bicomi <-
bicom1 <- bi2 <- dfbi1 <- impbii <- bi3 <- simprim <- impi <-
con1i <- nsyl2 <- mt3d <- con1d <- notnot1 <- con2i <- nsyl3
<- mt2d <- con2d <- notnot2 <- pm2.18d <- pm2.18 <- pm2.21 <-
pm2.21d <- a1d <- syl <- mpd <- a2i <- a2i.1 .
\end{verbatim}

This tells us that we would have to inspect 274 steps if we want to
verify the proof completely starting from the axioms.  A few more
statistics are also shown.  There are one or more paths back to axioms
that are the longest; this command ferrets out one of them and shows it
to you.  There may be a sense in which the longest path length is
related to how ``deep'' the theorem is.

We might also be curious about what proofs depend on the theorem
\texttt{prth}.  If it is never used later on, we could eliminate it as
redundant if it has no intrinsic interest by itself.\index{\texttt{show
usage} command}

% I decided to show the OLD values here.
\begin{verbatim}
MM> show usage prth
Statement "prth" is directly referenced in the proofs of 18
statements:
  mo3 moOLD 2mo 2moOLD euind reuind reuss2 reusv3i opelopabt
  wemaplem2 rexanre rlimcn2 o1of2 o1rlimmul 2sqlem6 spanuni
  heicant pm11.71
\end{verbatim}

Thus \texttt{prth} is directly used by 18 proofs.
We can use the \texttt{/recursive} qualifier to include indirect use:

\begin{verbatim}
MM> show usage prth /recursive
Statement "prth" directly or indirectly affects the proofs of
24214 statements:
  mo3 mo mo3OLD eu2 moOLD eu2OLD eu3OLD mo4f mo4 eu4 mopick
...
\end{verbatim}

\subsection{A Note on the ``Compact'' Proof Format}

The present version of Metamath\index{Metamath!limitations of
version 0.07.30} will display proofs in a ``compact''\index{compact proof}
format whenever the proof is stored in compressed format in the database.  It
may be be slightly confusing unless you know how to interpret it.  (A
future version may eliminate this format from displays.)  For example,
if you display the complete proof of theorem \texttt{id1} it will start
off as follows:

\begin{verbatim}
MM> show proof id1 /lemmon/all
 1 wph           $f wff ph
 2 wph           $f wff ph
 3 wph           $f wff ph
 4 2,3 wi    @4: $a wff ( ph -> ph )
 5 1,4 wi    @5: $a wff ( ph -> ( ph -> ph ) )
 6 @4            $a wff ( ph -> ph )
\end{verbatim}

\begin{center}
{etc.}
\end{center}

Step 4 has a ``local label,''\index{local label} \texttt{@4}, assigned to it.
Later on, at step 6, the label \texttt{@1} is referenced instead of
displaying the explicit proof for that step.  This technique takes advantage
of the fact that steps in a proof often repeat, especially during the
construction of wffs.  The compact format reduces the number of steps in the
proof display and may actually be preferred by some people, which is one
reason that it is still displayed by the program.

If you want to see the normal format with the ``true'' step numbers, you can
use the following workaround:\index{\texttt{save proof} command}

\begin{verbatim}
MM> save proof id1 /normal
The proof of "id1" has been reformatted and saved internally.
Remember to use WRITE SOURCE to save it permanently.
MM> show proof id1 /lemmon/all
 1 wph           $f wff ph
 2 wph           $f wff ph
 3 wph           $f wff ph
 4 2,3 wi        $a wff ( ph -> ph )
 5 1,4 wi        $a wff ( ph -> ( ph -> ph ) )
 6 wph           $f wff ph
 7 wph           $f wff ph
 8 6,7 wi        $a wff ( ph -> ph )
\end{verbatim}

\begin{center}
{etc.}
\end{center}

Note that the original 6 steps are now 8 steps.  However, the format is
now the same as that described in Chapter~\ref{using}.

\chapter{The Metamath Language}
\label{languagespec}

\begin{quote}
  {\em Thus mathematics may be defined as the subject in which we never know
what we are talking about, nor whether what we are saying is true.}
    \flushright\sc  Bertrand Russell\footnote{\cite[p.~84]{Russell2}.}\\
\end{quote}\index{Russell, Bertrand}

Probably the most striking feature of the Metamath language is its almost
complete absence of hard-wired syntax. Metamath\index{Metamath} does not
understand any mathematics or logic other than that needed to construct finite
sequences of symbols according to a small set of simple, built-in rules.  The
only rule it uses in a proof is the substitution of an expression (symbol
sequence) for a variable, subject to a simple constraint to prevent
bound-variable clashes.  The primitive notions built into Metamath involve the
simple manipulation of finite objects (symbols) that we as humans can easily
visualize and that computers can easily deal with.  They seem to be just
about the simplest notions possible that are required to do standard
mathematics.

This chapter serves as a reference manual for the Metamath\index{Metamath}
language. It covers the tedious technical details of the language, some of
which you may wish to skim in a first reading.  On the other hand, you should
pay close attention to the defined terms in {\bf boldface}; they have precise
meanings that are important to keep in mind for later understanding.  It may
be best to first become familiar with the examples in Chapter~\ref{using} to
gain some motivation for the language.

%% Uncomment this when uncommenting section {formalspec} below
If you have some knowledge of set theory, you may wish to study this
chapter in conjunction with the formal set-theoretical description of the
Metamath language in Appendix~\ref{formalspec}.

We will use the name ``Metamath''\index{Metamath} to mean either the Metamath
computer language or the Metamath software associated with the computer
language.  We will not distinguish these two when the context is clear.

The next section contains the complete specification of the Metamath
language.
It serves as an
authoritative reference and presents the syntax in enough detail to
write a parser\index{parsing Metamath} and proof verifier.  The
specification is terse and it is probably hard to learn the language
directly from it, but we include it here for those impatient people who
prefer to see everything up front before looking at verbose expository
material.  Later sections explain this material and provide examples.
We will repeat the definitions in those sections, and you may skip the
next section at first reading and proceed to Section~\ref{tut1}
(p.~\pageref{tut1}).

\section{Specification of the Metamath Language}\label{spec}
\index{Metamath!specification}

\begin{quote}
  {\em Sometimes one has to say difficult things, but one ought to say
them as simply as one knows how.}
    \flushright\sc  G. H. Hardy\footnote{As quoted in
    \cite{deMillo}, p.~273.}\\
\end{quote}\index{Hardy, G. H.}

\subsection{Preliminaries}\label{spec1}

% Space is technically a printable character, so we'll word things
% carefully so it's unambiguous.
A Metamath {\bf database}\index{database} is built up from a top-level source
file together with any source files that are brought in through file inclusion
commands (see below).  The only characters that are allowed to appear in a
Metamath source file are the 94 non-whitespace printable {\sc
ascii}\index{ascii@{\sc ascii}} characters, which are digits, upper and lower
case letters, and the following 32 special
characters\index{special characters}:\label{spec1chars}

\begin{verbatim}
! " # $ % & ' ( ) * + , - . / :
; < = > ? @ [ \ ] ^ _ ` { | } ~
\end{verbatim}
plus the following characters which are the ``white space'' characters:
space (a printable character),
tab, carriage return, line feed, and form feed.\label{whitespace}
We will use \texttt{typewriter}
font to display the printable characters.

A Metamath database consists of a sequence of three kinds of {\bf
tokens}\index{token} separated by {\bf white space}\index{white space}
(which is any sequence of one or more white space characters).  The set
of {\bf keyword}\index{keyword} tokens is \texttt{\$\char`\{},
\texttt{\$\char`\}}, \texttt{\$c}, \texttt{\$v}, \texttt{\$f},
\texttt{\$e}, \texttt{\$d}, \texttt{\$a}, \texttt{\$p}, \texttt{\$.},
\texttt{\$=}, \texttt{\$(}, \texttt{\$)}, \texttt{\$[}, and
\texttt{\$]}.  The last four are called {\bf auxiliary}\index{auxiliary
keyword} or preprocessing keywords.  A {\bf label}\index{label} token
consists of any combination of letters, digits, and the characters
hyphen, underscore, and period.  A {\bf math symbol}\index{math symbol}
token may consist of any combination of the 92 printable standard {\sc
ascii} characters other than space or \texttt{\$}~. All tokens are
case-sensitive.

\subsection{Preprocessing}

The token \texttt{\$(} begins a {\bf comment} and
\texttt{\$)} ends a comment.\index{\texttt{\$(}
and \texttt{\$)} auxiliary keywords}\index{comment}
Comments may contain any of
the 94 non-whitespace printable characters and white space,
except they may not contain the
2-character sequences \texttt{\$(} or \texttt{\$)} (comments do not nest).
Comments are ignored (treated
like white space) for the purpose of parsing, e.g.,
\texttt{\$( \$[ \$)} is a comment.
See p.~\pageref{mathcomments} for comment typesetting conventions; these
conventions may be ignored for the purpose of parsing.

A {\bf file inclusion command} consists of \texttt{\$[} followed by a file name
followed by \texttt{\$]}.\index{\texttt{\$[} and \texttt{\$]} auxiliary
keywords}\index{included file}\index{file inclusion}
It is only allowed in the outermost scope (i.e., not between
\texttt{\$\char`\{} and \texttt{\$\char`\}})
and must not be inside a statement (e.g., it may not occur
between the label of a \texttt{\$a} statement and its \texttt{\$.}).
The file name may not
contain a \texttt{\$} or white space.  The file must exist.
The case-sensitivity
of its name follows the conventions of the operating system.  The contents of
the file replace the inclusion command.
Included files may include other files.
Only the first reference to a given file is included; any later
references to the same file (whether in the top-level file or in included
files) cause the inclusion command to be ignored (treated like white space).
A verifier may assume that file names with different strings
refer to different files for the purpose of ignoring later references.
A file self-reference is ignored, as is any reference to the top-level file
(to avoid loops).
Included files may not include a \texttt{\$(} without a matching \texttt{\$)},
may not include a \texttt{\$[} without a matching \texttt{\$]}, and may
not include incomplete statements (e.g., a \texttt{\$a} without a matching
\texttt{\$.}).
It is currently unspecified if path references are relative to the process'
current directory or the file's containing directory, so databases should
avoid using pathname separators (e.g., ``/'') in file names.

Like all tokens, the \texttt{\$(}, \texttt{\$)}, \texttt{\$[}, and \texttt{\$]} keywords
must be surrounded by white space.

\subsection{Basic Syntax}

After preprocessing, a database will consist of a sequence of {\bf
statements}.
These are the scoping statements \texttt{\$\char`\{} and
\texttt{\$\char`\}}, along with the \texttt{\$c}, \texttt{\$v},
\texttt{\$f}, \texttt{\$e}, \texttt{\$d}, \texttt{\$a}, and \texttt{\$p}
statements.

A {\bf scoping statement}\index{scoping statement} consists only of its
keyword, \texttt{\$\char`\{} or \texttt{\$\char`\}}.
A \texttt{\$\char`\{} begins a {\bf
block}\index{block} and a matching \texttt{\$\char`\}} ends the block.
Every \texttt{\$\char`\{}
must have a matching \texttt{\$\char`\}}.
Defining it recursively, we say a block
contains a sequence of zero or more tokens other
than \texttt{\$\char`\{} and \texttt{\$\char`\}} and
possibly other blocks.  There is an {\bf outermost
block}\index{block!outermost} not bracketed by \texttt{\$\char`\{} \texttt{\$\char`\}}; the end
of the outermost block is the end of the database.

% LaTeX bug? can't do \bf\tt

A {\bf \$v} or {\bf \$c statement}\index{\texttt{\$v} statement}\index{\texttt{\$c}
statement} consists of the keyword token \texttt{\$v} or \texttt{\$c} respectively,
followed by one or more math symbols,
% The word "token" is used to distinguish "$." from the sentence-ending period.
followed by the \texttt{\$.}\ token.
These
statements {\bf declare}\index{declaration} the math symbols to be {\bf
variables}\index{variable!Metamath} or {\bf constants}\index{constant}
respectively. The same math symbol may not occur twice in a given \texttt{\$v} or
\texttt{\$c} statement.

%c%A math symbol becomes an {\bf active}\index{active math symbol}
%c%when declared and stays active until the end of the block in which it is
%c%declared.  A math symbol may not be declared a second time while it is active,
%c%but it may be declared again after it becomes inactive.

A math symbol becomes {\bf active}\index{active math symbol} when declared
and stays active until the end of the block in which it is declared.  A
variable may not be declared a second time while it is active, but it
may be declared again (as a variable, but not as a constant) after it
becomes inactive.  A constant must be declared in the outermost block and may
not be declared a second time.\footnote{The rules for redeclaration may
become more general in the future;
see p.~\pageref{redeclarationf}.}\index{redeclaration of symbols}

A {\bf \$f statement}\index{\texttt{\$f} statement} consists of a label,
followed by \texttt{\$f}, followed by its typecode (an active constant),
followed by an
active variable, followed by the \texttt{\$.}\ token.  A {\bf \$e
statement}\index{\texttt{\$e} statement} consists of a label, followed
by \texttt{\$e}, followed by its typecode (an active constant),
followed by zero or more
active math symbols, followed by the \texttt{\$.}\ token.  A {\bf
hypothesis}\index{hypothesis} is a \texttt{\$f} or \texttt{\$e}
statement.
The type declared by a \texttt{\$f} statement for a given label
is global even if the variable is not
(e.g., a database may not have \texttt{wff P} in one local scope
nd \texttt{class P} in another).

A {\bf simple \$d statement}\index{\texttt{\$d} statement!simple}
consists of \texttt{\$d}, followed by two different active variables,
followed by the \texttt{\$.}\ token.  A {\bf compound \$d
statement}\index{\texttt{\$d} statement!compound} consists of
\texttt{\$d}, followed by three or more variables (all different),
followed by the \texttt{\$.}\ token.  The order of the variables in a
\texttt{\$d} statement is unimportant.  A compound \texttt{\$d}
statement is equivalent to a set of simple \texttt{\$d} statements, one
for each possible pair of variables occurring in the compound
\texttt{\$d} statement.  Henceforth in this specification we shall
assume all \texttt{\$d} statements are simple.  A \texttt{\$d} statement
is also called a {\bf disjoint} (or {\bf distinct}) {\bf variable
restriction}.\index{disjoint-variable restriction}

A {\bf \$a statement}\index{\texttt{\$a} statement} consists of a label,
followed by \texttt{\$a}, followed by its typecode (an active constant),
followed by
zero or more active math symbols, followed by the \texttt{\$.}\ token.  A {\bf
\$p statement}\index{\texttt{\$p} statement} consists of a label,
followed by \texttt{\$p}, followed by its typecode (an active constant),
followed by
zero or more active math symbols, followed by \texttt{\$=}, followed by
a sequence of labels, followed by the \texttt{\$.}\ token.  An {\bf
assertion}\index{assertion} is a \texttt{\$a} or \texttt{\$p} statement.

A \texttt{\$f}, \texttt{\$e}, or \texttt{\$d} statement is {\bf active}\index{active
statement} from the place it occurs until the end of the block it occurs in.
A \texttt{\$a} or \texttt{\$p} statement is {\bf active} from the place it occurs
through the end of the database.
There may not be two active \texttt{\$f} statements containing the same
variable.  Each variable in a \texttt{\$e}, \texttt{\$a}, or
\texttt{\$p} statement must exist in an active \texttt{\$f}
statement.\footnote{This requirement can greatly simplify the
unification algorithm (substitution calculation) required by proof
verification.}

%The label that begins each \texttt{\$f}, \texttt{\$e}, \texttt{\$a}, and
%\texttt{\$p} statement must be unique.
Each label token must be unique, and
no label token may match any math symbol
token.\label{namespace}\footnote{This
restriction was added on June 24, 2006.
It is not theoretically necessary but is imposed to make it easier to
write certain parsers.}

The set of {\bf mandatory variables}\index{mandatory variable} associated with
an assertion is the set of (zero or more) variables in the assertion and in any
active \texttt{\$e} statements.  The (possibly empty) set of {\bf mandatory
hypotheses}\index{mandatory hypothesis} is the set of all active \texttt{\$f}
statements containing mandatory variables, together with all active \texttt{\$e}
statements.
The set of {\bf mandatory {\bf \$d} statements}\index{mandatory
disjoint-variable restriction} associated with an assertion are those active
\texttt{\$d} statements whose variables are both among the assertion's
mandatory variables.

\subsection{Proof Verification}\label{spec4}

The sequence of labels between the \texttt{\$=} and \texttt{\$.}\ tokens
in a \texttt{\$p} statement is a {\bf proof}.\index{proof!Metamath} Each
label in a proof must be the label of an active statement other than the
\texttt{\$p} statement itself; thus a label must refer either to an
active hypothesis of the \texttt{\$p} statement or to an earlier
assertion.

An {\bf expression}\index{expression} is a sequence of math symbols. A {\bf
substitution map}\index{substitution map} associates a set of variables with a
set of expressions.  It is acceptable for a variable to be mapped to an
expression containing it.  A {\bf
substitution}\index{substitution!variable}\index{variable substitution} is the
simultaneous replacement of all variables in one or more expressions with the
expressions that the variables map to.

A proof is scanned in order of its label sequence.  If the label refers to an
active hypothesis, the expression in the hypothesis is pushed onto a
stack.\index{stack}\index{RPN stack}  If the label refers to an assertion, a
(unique) substitution must exist that, when made to the mandatory hypotheses
of the referenced assertion, causes them to match the topmost (i.e.\ most
recent) entries of the stack, in order of occurrence of the mandatory
hypotheses, with the topmost stack entry matching the last mandatory
hypothesis of the referenced assertion.  As many stack entries as there are
mandatory hypotheses are then popped from the stack.  The same substitution is
made to the referenced assertion, and the result is pushed onto the stack.
After the last label in the proof is processed, the stack must have a single
entry that matches the expression in the \texttt{\$p} statement containing the
proof.

%c%{\footnotesize\begin{quotation}\index{redeclaration of symbols}
%c%{{\em Comment.}\label{spec4comment} Whenever a math symbol token occurs in a
%c%{\texttt{\$c} or \texttt{\$v} statement, it is considered to designate a distinct new
%c%{symbol, even if the same token was previously declared (and is now inactive).
%c%{Thus a math token declared as a constant in two different blocks is considered
%c%{to designate two distinct constants (even though they have the same name).
%c%{The two constants will not match in a proof that references both blocks.
%c%{However, a proof referencing both blocks is acceptable as long as it doesn't
%c%{require that the constants match.  Similarly, a token declared to be a
%c%{constant for a referenced assertion will not match the same token declared to
%c%{be a variable for the \texttt{\$p} statement containing the proof.  In the case
%c%{of a token declared to be a variable for a referenced assertion, this is not
%c%{an issue since the variable can be substituted with whatever expression is
%c%{needed to achieve the required match.
%c%{\end{quotation}}
%c2%A proof may reference an assertion that contains or whose hypotheses contain a
%c2%constant that is not active for the \texttt{\$p} statement containing the proof.
%c2%However, the final result of the proof may not contain that constant. A proof
%c2%may also reference an assertion that contains or whose hypotheses contain a
%c2%variable that is not active for the \texttt{\$p} statement containing the proof.
%c2%That variable, of course, will be substituted with whatever expression is
%c2%needed to achieve the required match.

A proof may contain a \texttt{?}\ in place of a label to indicate an unknown step
(Section~\ref{unknown}).  A proof verifier may ignore any proof containing
\texttt{?}\ but should warn the user that the proof is incomplete.

A {\bf compressed proof}\index{compressed proof}\index{proof!compressed} is an
alternate proof notation described in Appen\-dix~\ref{compressed}; also see
references to ``compressed proof'' in the Index.  Compressed proofs are a
Metamath language extension which a complete proof verifier should be able to
parse and verify.

\subsubsection{Verifying Disjoint Variable Restrictions}

Each substitution made in a proof must be checked to verify that any
disjoint variable restrictions are satisfied, as follows.

If two variables replaced by a substitution exist in a mandatory \texttt{\$d}
statement\index{\texttt{\$d} statement} of the assertion referenced, the two
expressions resulting from the substitution must satisfy the following
conditions.  First, the two expressions must have no variables in common.
Second, each possible pair of variables, one from each expression, must exist
in an active \texttt{\$d} statement of the \texttt{\$p} statement containing the
proof.

\vskip 1ex

This ends the specification of the Metamath language;
see Appendix \ref{BNF} for its syntax in
Extended Backus--Naur Form (EBNF)\index{Extended Backus--Naur Form}\index{EBNF}.

\section{The Basic Keywords}\label{tut1}

Our expository material begins here.

Like most computer languages, Metamath\index{Metamath} takes its input from
one or more {\bf source files}\index{source file} which contain characters
expressed in the standard {\sc ascii} (American Standard Code for Information
Interchange)\index{ascii@{\sc ascii}} code for computers.  A source file
consists of a series of {\bf tokens}\index{token}, which are strings of
non-whitespace
printable characters (from the set of 94 shown on p.~\pageref{spec1chars})
separated by {\bf white space}\index{white space} (spaces, tabs, carriage
returns, line feeds, and form feeds). Any string consisting only of these
characters is treated the same as a single space.  The non-whitespace printable
characters\index{printable character} that Metamath recognizes are the 94
characters on standard {\sc ascii} keyboards.

Metamath has the ability to join several files together to form its
input (Section~\ref{include}).  We call the aggregate contents of all
the files after they have been joined together a {\bf
database}\index{database} to distinguish it from an individual source
file.  The tokens in a database consist of {\bf
keywords}\index{keyword}, which are built into the language, together
with two kinds of user-defined tokens called {\bf labels}\index{label}
and {\bf math symbols}\index{math symbol}.  (Often we will simply say
{\bf symbol}\index{symbol} instead of math symbol for brevity).  The set
of {\bf basic keywords}\index{basic keyword} is
\texttt{\$c}\index{\texttt{\$c} statement},
\texttt{\$v}\index{\texttt{\$v} statement},
\texttt{\$e}\index{\texttt{\$e} statement},
\texttt{\$f}\index{\texttt{\$f} statement},
\texttt{\$d}\index{\texttt{\$d} statement},
\texttt{\$a}\index{\texttt{\$a} statement},
\texttt{\$p}\index{\texttt{\$p} statement},
\texttt{\$=}\index{\texttt{\$=} keyword},
\texttt{\$.}\index{\texttt{\$.}\ keyword},
\texttt{\$\char`\{}\index{\texttt{\$\char`\{} and \texttt{\$\char`\}}
keywords}, and \texttt{\$\char`\}}.  This is the complete set of
syntactical elements of what we call the {\bf basic
language}\index{basic language} of Metamath, and with them you can
express all of the mathematics that were intended by the design of
Metamath.  You should make it a point to become very familiar with them.
Table~\ref{basickeywords} lists the basic keywords along with a brief
description of their functions.  For now, this description will give you
only a vague notion of what the keywords are for; later we will describe
the keywords in detail.


\begin{table}[htp] \caption{Summary of the basic Metamath
keywords} \label{basickeywords}
\begin{center}
\begin{tabular}{|p{4pc}|l|}
\hline
\em \centering Keyword&\em Description\\
\hline
\hline
\centering
   \texttt{\$c}&Constant symbol declaration\\
\hline
\centering
   \texttt{\$v}&Variable symbol declaration\\
\hline
\centering
   \texttt{\$d}&Disjoint variable restriction\\
\hline
\centering
   \texttt{\$f}&Variable-type (``floating'') hypothesis\\
\hline
\centering
   \texttt{\$e}&Logical (``essential'') hypothesis\\
\hline
\centering
   \texttt{\$a}&Axiomatic assertion\\
\hline
\centering
   \texttt{\$p}&Provable assertion\\
\hline
\centering
   \texttt{\$=}&Start of proof in \texttt{\$p} statement\\
\hline
\centering
   \texttt{\$.}&End of the above statement types\\
\hline
\centering
   \texttt{\$\char`\{}&Start of block\\
\hline
\centering
   \texttt{\$\char`\}}&End of block\\
\hline
\end{tabular}
\end{center}
\end{table}

%For LaTeX bug(?) where it puts tables on blank page instead of btwn text
%May have to adjust if text changes
%\newpage

There are some additional keywords, called {\bf auxiliary
keywords}\index{auxiliary keyword} that help make Metamath\index{Metamath}
more practical. These are part of the {\bf extended language}\index{extended
language}. They provide you with a means to put comments into a Metamath
source file\index{source file} and reference other source files.  We will
introduce these in later sections. Table~\ref{otherkeywords} summarizes them
so that you can recognize them now if you want to peruse some source
files while learning the basic keywords.


\begin{table}[htp] \caption{Auxiliary Metamath
keywords} \label{otherkeywords}
\begin{center}
\begin{tabular}{|p{4pc}|l|}
\hline
\em \centering Keyword&\em Description\\
\hline
\hline
\centering
   \texttt{\$(}&Start of comment\\
\hline
\centering
   \texttt{\$)}&End of comment\\
\hline
\centering
   \texttt{\$[}&Start of included source file name\\
\hline
\centering
   \texttt{\$]}&End of included source file name\\
\hline
\end{tabular}
\end{center}
\end{table}
\index{\texttt{\$(} and \texttt{\$)} auxiliary keywords}
\index{\texttt{\$[} and \texttt{\$]} auxiliary keywords}


Unlike those in some computer languages, the keywords\index{keyword} are short
two-character sequences rather than English-like words.  While this may make
them slightly more difficult to remember at first, their brevity allows
them to blend in with the mathematics being described, not
distract from it, like punctuation marks.


\subsection{User-Defined Tokens}\label{dollardollar}\index{token}

As you may have noticed, all keywords\index{keyword} begin with the \texttt{\$}
character.  This mundane monetary symbol is not ordinarily used in higher
mathematics (outside of grant proposals), so we have appropriated it to
distinguish the Metamath\index{Metamath} keywords from ordinary mathematical
symbols. The \texttt{\$} character is thus considered special and may not be
used as a character in a user-defined token.  All tokens and keywords are
case-sensitive; for example, \texttt{n} is considered to be a different character
from \texttt{N}.  Case-sensitivity makes the available {\sc ascii} character set
as rich as possible.

\subsubsection{Math Symbol Tokens}\index{token}

Math symbols\index{math symbol} are tokens used to represent the symbols
that appear in ordinary mathematical formulas.  They may consist of any
combination of the 93 non-whitespace printable {\sc ascii} characters other than
\texttt{\$}~. Some examples are \texttt{x}, \texttt{+}, \texttt{(},
\texttt{|-}, \verb$!%@?&$, and \texttt{bounded}.  For readability, it is
best to try to make these look as similar to actual mathematical symbols
as possible, within the constraints of the {\sc ascii} character set, in
order to make the resulting mathematical expressions more readable.

In the Metamath\index{Metamath} language, you express ordinary
mathematical formulas and statements as sequences of math symbols such
as \texttt{2 + 2 = 4} (five symbols, all constants).\footnote{To
eliminate ambiguity with other expressions, this is expressed in the set
theory database \texttt{set.mm} as \texttt{|- ( 2 + 2
 ) = 4 }, whose \LaTeX\ equivalent is $\vdash
(2+2)=4$.  The \,$\vdash$ means ``is a theorem'' and the
parentheses allow explicit associative grouping.}\index{turnstile
({$\,\vdash$})} They may even be English
sentences, as in \texttt{E is closed and bounded} (five symbols)---here
\texttt{E} would be a variable and the other four symbols constants.  In
principle, a Metamath database could be constructed to work with almost
any unambiguous English-language mathematical statement, but as a
practical matter the definitions needed to provide for all possible
syntax variations would be cumbersome and distracting and possibly have
subtle pitfalls accidentally built in.  We generally recommend that you
express mathematical statements with compact standard mathematical
symbols whenever possible and put their English-language descriptions in
comments.  Axioms\index{axiom} and definitions\index{definition}
(\texttt{\$a}\index{\texttt{\$a} statement} statements) are the only
places where Metamath will not detect an error, and doing this will help
reduce the number of definitions needed.

You are free to use any tokens\index{token} you like for math
symbols\index{math symbol}.  Appendix~\ref{ASCII} recommends token names to
use for symbols in set theory, and we suggest you adopt these in order to be
able to include the \texttt{set.mm} set theory database in your database.  For
printouts, you can convert the tokens in a database
to standard mathematical symbols with the \LaTeX\ typesetting program.  The
Metamath command \texttt{open tex} {\em filename}\index{\texttt{open tex} command}
produces output that can be read by \LaTeX.\index{latex@{\LaTeX}}
The correspondence
between tokens and the actual symbols is made by \texttt{latexdef}
statements inside a special database comment tagged
with \texttt{\$t}.\index{\texttt{\$t} comment}\index{typesetting comment}
  You can edit
this comment to change the definitions or add new ones.
Appendix~\ref{ASCII} describes how to do this in more detail.

% White space\index{white space} is normally used to separate math
% symbol\index{math symbol} tokens, but they may be juxtaposed without white
% space in \texttt{\$d}\index{\texttt{\$d} statement}, \texttt{\$e}\index{\texttt{\$e}
% statement}, \texttt{\$f}\index{\texttt{\$f} statement}, \texttt{\$a}\index{\texttt{\$a}
% statement}, and \texttt{\$p}\index{\texttt{\$p} statement} statements when no
% ambiguity will result.  Specifically, Metamath parses the math symbol sequence
% in one of these statements in the following manner:  when the math symbol
% sequence has been broken up into tokens\index{token} up to a given character,
% the next token is the longest string of characters that could constitute a
% math symbol that is active\index{active
% math symbol} at that point.  (See Section~\ref{scoping} for the
% definition of an active math symbol.)  For example, if \texttt{-}, \texttt{>}, and
% \texttt{->} are the only active math symbols, the juxtaposition \texttt{>-} will be
% interpreted as the two symbols \texttt{>} and \texttt{-}, whereas \texttt{->} will
% always be interpreted as that single symbol.\footnote{For better readability we
% recommend a white space between each token.  This also makes searching for a
% symbol easier to do with an editor.  Omission of optional white space is useful
% for reducing typing when assigning an expression to a temporary
% variable\index{temporary variable} with the \texttt{let variable} Metamath
% program command.}\index{\texttt{let variable} command}
%
% Keywords\index{keyword} may be placed next to math symbols without white
% space\index{white space} between them.\footnote{Again, we do not recommend
% this for readability.}
%
% The math symbols\index{math symbol} in \texttt{\$c}\index{\texttt{\$c} statement}
% and \texttt{\$v}\index{\texttt{\$v} statement} statements must always be separated
% by white space\index{white
% space}, for the obvious reason that these statements define the names
% of the symbols.
%
% Math symbols referred to in comments (see Section~\ref{comments}) must also be
% separated by white space.  This allows you to make comments about symbols that
% are not yet active\index{active
% math symbol}.  (The ``math mode'' feature of comments is also a quick and
% easy way to obtain word processing text with embedded mathematical symbols,
% independently of the main purpose of Metamath; the way to do this is described
% in Section~\ref{comments})

\subsubsection{Label Tokens}\index{token}\index{label}

Label tokens are used to identify Metamath\index{Metamath} statements for
later reference. Label tokens may contain only letters, digits, and the three
characters period, hyphen, and underscore:
\begin{verbatim}
. - _
\end{verbatim}

A label is {\bf declared}\index{label declaration} by placing it immediately
before the keyword of the statement it identifies.  For example, the label
\texttt{axiom.1} might be declared as follows:
\begin{verbatim}
axiom.1 $a |- x = x $.
\end{verbatim}

Each \texttt{\$e}\index{\texttt{\$e} statement},
\texttt{\$f}\index{\texttt{\$f} statement},
\texttt{\$a}\index{\texttt{\$a} statement}, and
\texttt{\$p}\index{\texttt{\$p} statement} statement in a database must
have a label declared for it.  No other statement types may have label
declarations.  Every label must be unique.

A label (and the statement it identifies) is {\bf referenced}\index{label
reference} by including the label between the \texttt{\$=}\index{\texttt{\$=}
keyword} and \texttt{\$.}\index{\texttt{\$.}\ keyword}\ keywords in a \texttt{\$p}
statement.  The sequence of labels\index{label sequence} between these two
keywords is called a {\bf proof}\index{proof}.  An example of a statement with
a proof that we will encounter later (Section~\ref{proof}) is
\begin{verbatim}
wnew $p wff ( s -> ( r -> p ) )
     $= ws wr wp w2 w2 $.
\end{verbatim}

You don't have to know what this means just yet, but you should know that the
label \texttt{wnew} is declared by this \texttt{\$p} statement and that the labels
\texttt{ws}, \texttt{wr}, \texttt{wp}, and \texttt{w2} are assumed to have been declared
earlier in the database and are referenced here.

\subsection{Constants and Variables}
\index{constant}
\index{variable}

An {\bf expression}\index{expression} is any sequence of math
symbols, possibly empty.

The basic Metamath\index{Metamath} language\index{basic language} has two
kinds of math symbols\index{math symbol}:  {\bf constants}\index{constant} and
{\bf variables}\index{variable}.  In a Metamath proof, a constant may not be
substituted with any expression.  A variable can be
substituted\index{substitution!variable}\index{variable substitution} with any
expression.  This sequence may include other variables and may even include
the variable being substituted.  This substitution takes place when proofs are
verified, and it will be described in Section~\ref{proof}.  The \texttt{\$f}
statement (described later in Section~\ref{dollaref}) is used to specify the
{\bf type} of a variable (i.e.\ what kind of
variable it is)\index{variable type}\index{type} and
give it a meaning typically
associated with a ``metavariable''\index{metavariable}\footnote{A metavariable
is a variable that ranges over the syntactical elements of the object language
being discussed; for example, one metavariable might represent a variable of
the object language and another metavariable might represent a formula in the
object language.} in ordinary mathematics; for example, a variable may be
specified to be a wff or well-formed formula (in logic), a set (in set
theory), or a non-negative integer (in number theory).

%\subsection{The \texttt{\$c} and \texttt{\$v} Declaration Statements}
\subsection{The \texttt{\$c} and \texttt{\$v} Declaration Statements}
\index{\texttt{\$c} statement}
\index{constant declaration}
\index{\texttt{\$v} statement}
\index{variable declaration}

Constants are introduced or {\bf declared}\index{constant declaration}
with \texttt{\$c}\index{\texttt{\$c} statement} statements, and
variables are declared\index{variable declaration} with
\texttt{\$v}\index{\texttt{\$v} statement} statements.  A {\bf simple}
declaration\index{simple declaration} statement introduces a single
constant or variable.  Its syntax is one of the following:
\begin{center}
  \texttt{\$c} {\em math-symbol} \texttt{\$.}\\
  \texttt{\$v} {\em math-symbol} \texttt{\$.}
\end{center}
The notation {\em math-symbol} means any math symbol token\index{token}.

Some examples of simple declaration statements are:
\begin{center}
  \texttt{\$c + \$.}\\
  \texttt{\$c -> \$.}\\
  \texttt{\$c ( \$.}\\
  \texttt{\$v x \$.}\\
  \texttt{\$v y2 \$.}
\end{center}

The characters in a math symbol\index{math symbol} being declared are
irrelevant to Meta\-math; for example, we could declare a right parenthesis to
be a variable,
\begin{center}
  \texttt{\$v ) \$.}\\
\end{center}
although this would be unconventional.

A {\bf compound} declaration\index{compound declaration} statement is a
shorthand for declaring several symbols at once.  Its syntax is one of the
following:
\begin{center}
  \texttt{\$c} {\em math-symbol}\ \,$\cdots$\ {\em math-symbol} \texttt{\$.}\\
  \texttt{\$v} {\em math-symbol}\ \,$\cdots$\ {\em math-symbol} \texttt{\$.}
\end{center}\index{\texttt{\$c} statement}
Here, the ellipsis (\ldots) means any number of {\em math-symbol}\,s.

An example of a compound declaration statement is:
\begin{center}
  \texttt{\$v x y mu \$.}\\
\end{center}
This is equivalent to the three simple declaration statements
\begin{center}
  \texttt{\$v x \$.}\\
  \texttt{\$v y \$.}\\
  \texttt{\$v mu \$.}\\
\end{center}
\index{\texttt{\$v} statement}

There are certain rules on where in the database math symbols may be declared,
what sections of the database are aware of them (i.e.\ where they are
``active''), and when they may be declared more than once.  These will be
discussed in Section~\ref{scoping} and specifically on
p.~\pageref{redeclaration}.

\subsection{The \texttt{\$d} Statement}\label{dollard}
\index{\texttt{\$d} statement}

The \texttt{\$d} statement is called a {\bf disjoint-variable restriction}.  The
syntax of the {\bf simple} version of this statement is
\begin{center}
  \texttt{\$d} {\em variable variable} \texttt{\$.}
\end{center}
where each {\em variable} is a previously declared variable and the two {\em
variable}\,s are different.  (More specifically, each  {\em variable} must be
an {\bf active} variable\index{active math symbol}, which means there must be
a previous \texttt{\$v} statement whose {\bf scope}\index{scope} includes the
\texttt{\$d} statement.  These terms will be defined when we discuss scoping
statements in Section~\ref{scoping}.)

In ordinary mathematics, formulas may arise that are true if the variables in
them are distinct\index{distinct variables}, but become false when those
variables are made identical. For example, the formula in logic $\exists x\,x
\neq y$, which means ``for a given $y$, there exists an $x$ that is not equal
to $y$,'' is true in most mathematical theories (namely all non-trivial
theories\index{non-trivial theory}, i.e.\ those that describe more than one
individual, such as arithmetic).  However, if we substitute $y$ with $x$, we
obtain $\exists x\,x \neq x$, which is always false, as it means ``there
exists something that is not equal to itself.''\footnote{If you are a
logician, you will recognize this as the improper substitution\index{proper
substitution}\index{substitution!proper} of a free variable\index{free
variable} with a bound variable\index{bound variable}.  Metamath makes no
inherent distinction between free and bound variables; instead, you let
Metamath know what substitutions are permissible by using \texttt{\$d} statements
in the right way in your axiom system.}\index{free vs.\ bound variable}  The
\texttt{\$d} statement allows you to specify a restriction that forbids the
substitution of one variable with another.  In
this case, we would use the statement
\begin{center}
  \texttt{\$d x y \$.}
\end{center}\index{\texttt{\$d} statement}
to specify this restriction.

The order in which the variables appear in a \texttt{\$d} statement is not
important.  We could also use
\begin{center}
  \texttt{\$d y x \$.}
\end{center}

The \texttt{\$d} statement is actually more general than this, as the
``disjoint''\index{disjoint variables} in its name suggests.  The full meaning
is that if any substitution is made to its two variables (during the
course of a proof that references a \texttt{\$a} or \texttt{\$p} statement
associated with the \texttt{\$d}), the two expressions that result from the
substitution must have no variables in common.  In addition, each possible
pair of variables, one from each expression, must be in a \texttt{\$d} statement
associated with the statement being proved.  (This requirement forces the
statement being proved to ``inherit'' the original disjoint variable
restriction.)

For example, suppose \texttt{u} is a variable.  If the restriction
\begin{center}
  \texttt{\$d A B \$.}
\end{center}
has been specified for a theorem referenced in a
proof, we may not substitute \texttt{A} with \mbox{\tt a + u} and
\texttt{B} with \mbox{\tt b + u} because these two symbol sequences have the
variable \texttt{u} in common.  Furthermore, if \texttt{a} and \texttt{b} are
variables, we may not substitute \texttt{A} with \texttt{a} and \texttt{B} with \texttt{b}
unless we have also specified \texttt{\$d a b} for the theorem being proved; in
other words, the \texttt{\$d} property associated with a pair of variables must
be effectively preserved after substitution.

The \texttt{\$d}\index{\texttt{\$d} statement} statement does {\em not} mean ``the
two variables may not be substituted with the same thing,'' as you might think
at first.  For example, substituting each of \texttt{A} and \texttt{B} in the above
example with identical symbol sequences consisting only of constants does not
cause a disjoint variable conflict, because two symbol sequences have no
variables in common (since they have no variables, period).  Similarly, a
conflict will not occur by substituting the two variables in a \texttt{\$d}
statement with the empty symbol sequence\index{empty substitution}.

The \texttt{\$d} statement does not have a direct counterpart in
ordinary mathematics, partly because the variables\index{variable} of
Metamath are not really the same as the variables\index{variable!in
ordinary mathematics} of ordinary mathematics but rather are
metavariables\index{metavariable} ranging over them (as well as over
other kinds of symbols and groups of symbols).  Depending on the
situation, we may informally interpret the \texttt{\$d} statement in
different ways.  Suppose, for example, that \texttt{x} and \texttt{y}
are variables ranging over numbers (more precisely, that \texttt{x} and
\texttt{y} are metavariables ranging over variables that range over
numbers), and that \texttt{ph} ($\varphi$) and \texttt{ps} ($\psi$) are
variables (more precisely, metavariables) ranging over formulas.  We can
make the following interpretations that correspond to the informal
language of ordinary mathematics:
\begin{quote}
\begin{tabbing}
\texttt{\$d x y \$.} means ``assume $x$ and $y$ are
distinct variables.''\\
\texttt{\$d x ph \$.} means ``assume $x$ does not
occur in $\varphi$.''\\
\texttt{\$d ph ps \$.} \=means ``assume $\varphi$ and
$\psi$ have no variables\\ \>in common.''
\end{tabbing}
\end{quote}\index{\texttt{\$d} statement}

\subsubsection{Compound \texttt{\$d} Statements}

The {\bf compound} version of the \texttt{\$d} statement is a shorthand for
specifying several variables whose substitutions must be pairwise disjoint.
Its syntax is:
\begin{center}
  \texttt{\$d} {\em variable}\ \,$\cdots$\ {\em variable} \texttt{\$.}
\end{center}\index{\texttt{\$d} statement}
Here, {\em variable} represents the token of a previously declared
variable (specifically, an active variable) and all {\em variable}\,s are
different.  The compound \texttt{\$d}
statement is internally broken up by Metamath into one simple \texttt{\$d}
statement for each possible pair of variables in the original \texttt{\$d}
statement.  For example,
\begin{center}
  \texttt{\$d w x y z \$.}
\end{center}
is equivalent to
\begin{center}
  \texttt{\$d w x \$.}\\
  \texttt{\$d w y \$.}\\
  \texttt{\$d w z \$.}\\
  \texttt{\$d x y \$.}\\
  \texttt{\$d x z \$.}\\
  \texttt{\$d y z \$.}
\end{center}

Two or more simple \texttt{\$d} statements specifying the same variable pair are
internally combined into a single \texttt{\$d} statement.  Thus the set of three
statements
\begin{center}
  \texttt{\$d x y \$.}
  \texttt{\$d x y \$.}
  \texttt{\$d y x \$.}
\end{center}
is equivalent to
\begin{center}
  \texttt{\$d x y \$.}
\end{center}

Similarly, compound \texttt{\$d} statements, after being internally broken up,
internally have their common variable pairs combined.  For example the
set of statements
\begin{center}
  \texttt{\$d x y A \$.}
  \texttt{\$d x y B \$.}
\end{center}
is equivalent to
\begin{center}
  \texttt{\$d x y \$.}
  \texttt{\$d x A \$.}
  \texttt{\$d y A \$.}
  \texttt{\$d x y \$.}
  \texttt{\$d x B \$.}
  \texttt{\$d y B \$.}
\end{center}
which is equivalent to
\begin{center}
  \texttt{\$d x y \$.}
  \texttt{\$d x A \$.}
  \texttt{\$d y A \$.}
  \texttt{\$d x B \$.}
  \texttt{\$d y B \$.}
\end{center}

Metamath\index{Metamath} automatically verifies that all \texttt{\$d}
restrictions are met whenever it verifies proofs.  \texttt{\$d} statements are
never referenced directly in proofs (this is why they do not have
labels\index{label}), but Metamath is always aware of which ones must be
satisfied (i.e.\ are active) and will notify you with an error message if any
violation occurs.

To illustrate how Metamath detects a missing \texttt{\$d}
statement, we will look at the following example from the
\texttt{set.mm} database.

\begin{verbatim}
$d x z $.  $d y z $.
$( Theorem to add distinct quantifier to atomic formula. $)
ax17eq $p |- ( x = y -> A. z x = y ) $=...
\end{verbatim}

This statement has the obvious requirement that $z$ must be
distinct\index{distinct variables} from $x$ in theorem \texttt{ax17eq} that
states $x=y \rightarrow \forall z \, x=y$ (well, obvious if you're a logician,
for otherwise we could conclude  $x=y \rightarrow \forall x \, x=y$, which is
false when the free variables $x$ and $y$ are equal.).

Let's look at what happens if we edit the database to comment out this
requirement.

\begin{verbatim}
$( $d x z $. $) $d y z $.
$( Theorem to add distinct quantifier to atomic formula. $)
ax17eq $p |- ( x = y -> A. z x = y ) $=...
\end{verbatim}

When it tries to verify the proof, Metamath will tell you that \texttt{x} and
\texttt{z} must be disjoint, because one of its steps references an axiom or
theorem that has this requirement.

\begin{verbatim}
MM> verify proof ax17eq
ax17eq ?Error at statement 1918, label "ax17eq", type "$p":
      vz wal wi vx vy vz ax-13 vx vy weq vz vx ax-c16 vx vy
                                               ^^^^^
There is a disjoint variable ($d) violation at proof step 29.
Assertion "ax-c16" requires that variables "x" and "y" be
disjoint.  But "x" was substituted with "z" and "y" was
substituted with "x".  The assertion being proved, "ax17eq",
does not require that variables "z" and "x" be disjoint.
\end{verbatim}

We can see the substitutions into \texttt{ax-c16} with the following command.

\begin{verbatim}
MM> show proof ax17eq / detailed_step 29
Proof step 29:  pm2.61dd.2=ax-c16 $a |- ( A. z z = x -> ( x =
  y -> A. z x = y ) )
This step assigns source "ax-c16" ($a) to target "pm2.61dd.2"
($e).  The source assertion requires the hypotheses "wph"
($f, step 26), "vx" ($f, step 27), and "vy" ($f, step 28).
The parent assertion of the target hypothesis is "pm2.61dd"
($p, step 36).
The source assertion before substitution was:
    ax-c16 $a |- ( A. x x = y -> ( ph -> A. x ph ) )
The following substitutions were made to the source
assertion:
    Variable  Substituted with
     x         z
     y         x
     ph        x = y
The target hypothesis before substitution was:
    pm2.61dd.2 $e |- ( ph -> ch )
The following substitutions were made to the target
hypothesis:
    Variable  Substituted with
     ph        A. z z = x
     ch        ( x = y -> A. z x = y )
\end{verbatim}

The disjoint variable restrictions of \texttt{ax-c16} can be seen from the
\texttt{show state\-ment} command.  The line that begins ``\texttt{Its mandatory
dis\-joint var\-i\-able pairs are:}\ldots'' lists any \texttt{\$d} variable
pairs in brackets.

\begin{verbatim}
MM> show statement ax-c16/full
Statement 3033 is located on line 9338 of the file "set.mm".
"Axiom of Distinct Variables. ..."
  ax-c16 $a |- ( A. x x = y -> ( ph -> A. x ph ) ) $.
Its mandatory hypotheses in RPN order are:
  wph $f wff ph $.
  vx $f set x $.
  vy $f set y $.
Its mandatory disjoint variable pairs are:  <x,y>
The statement and its hypotheses require the variables:  x y
      ph
The variables it contains are:  x y ph
\end{verbatim}

Since Metamath will always detect when \texttt{\$d}\index{\texttt{\$d} statement}
statements are needed for a proof, you don't have to worry too much about
forgetting to put one in; it can always be added if you see the error message
above.  If you put in unnecessary \texttt{\$d} statements, the worst that could
happen is that your theorem might not be as general as it could be, and this
may limit its use later on.

On the other hand, when you introduce axioms (\texttt{\$a}\index{\texttt{\$a}
statement} statements), you must be very careful to properly specify the
necessary associated \texttt{\$d} statements since Metamath has no way of knowing
whether your axioms are correct.  For example, Metamath would have no idea
that \texttt{ax-c16}, which we are telling it is an axiom of logic, would lead to
contradictions if we omitted its associated \texttt{\$d} statement.

% This was previously a comment in footnote-sized type, but it can be
% hard to read this much text in a small size.
% As a result, it's been changed to normally-sized text.
\label{nodd}
You may wonder if it is possible to develop standard
mathematics in the Metamath language without the \texttt{\$d}\index{\texttt{\$d}
statement} statement, since it seems like a nuisance that complicates proof
verification. The \texttt{\$d} statement is not needed in certain subsets of
mathematics such as propositional calculus.  However, dummy
variables\index{dummy variable!eliminating} and their associated \texttt{\$d}
statements are impossible to avoid in proofs in standard first-order logic as
well as in the variant used in \texttt{set.mm}.  In fact, there is no upper bound to
the number of dummy variables that might be needed in a proof of a theorem of
first-order logic containing 3 or more variables, as shown by H.\
Andr\'{e}ka\index{Andr{\'{e}}ka, H.} \cite{Nemeti}.  A first-order system that
avoids them entirely is given in \cite{Megill}\index{Megill, Norman}; the
trick there is simply to embed harmlessly the necessary dummy variables into a
theorem being proved so that they aren't ``dummy'' anymore, then interpret the
resulting longer theorem so as to ignore the embedded dummy variables.  If
this interests you, the system in \texttt{set.mm} obtained from \texttt{ax-1}
through \texttt{ax-c14} in \texttt{set.mm}, and deleting \texttt{ax-c16} and \texttt{ax-5},
requires no \texttt{\$d} statements but is logically complete in the sense
described in \cite{Megill}.  This means it can prove any theorem of
first-order logic as long as we add to the theorem an antecedent that embeds
dummy and any other variables that must be distinct.  In a similar fashion,
axioms for set theory can be devised that
do not require distinct variable
provisos\index{Set theory without distinct variable provisos},
as explained at
\url{http://us.metamath.org/mpeuni/mmzfcnd.html}.
Together, these in principle allow all of
mathematics to be developed under Metamath without a \texttt{\$d} statement,
although the length of the resulting theorems will grow as more and
more dummy variables become required in their proofs.

\subsection{The \texttt{\$f}
and \texttt{\$e} Statements}\label{dollaref}
\index{\texttt{\$e} statement}
\index{\texttt{\$f} statement}
\index{floating hypothesis}
\index{essential hypothesis}
\index{variable-type hypothesis}
\index{logical hypothesis}
\index{hypothesis}

Metamath has two kinds of hypo\-theses, the \texttt{\$f}\index{\texttt{\$f}
statement} or {\bf variable-type} hypothesis and the \texttt{\$e} or {\bf logical}
hypo\-the\-sis.\index{\texttt{\$d} statement}\footnote{Strictly speaking, the
\texttt{\$d} statement is also a hypothesis, but it is never directly referenced
in a proof, so we call it a restriction rather than a hypothesis to lessen
confusion.  The checking for violations of \texttt{\$d} restrictions is automatic
and built into Metamath's proof-checking algorithm.} The letters \texttt{f} and
\texttt{e} stand for ``floating''\index{floating hypothesis} (roughly meaning
used only if relevant) and ``essential''\index{essential hypothesis} (meaning
always used) respectively, for reasons that will become apparent
when we discuss frames in
Section~\ref{frames} and scoping in Section~\ref{scoping}. The syntax of these
are as follows:
\begin{center}
  {\em label} \texttt{\$f} {\em typecode} {\em variable} \texttt{\$.}\\
  {\em label} \texttt{\$e} {\em typecode}
      {\em math-symbol}\ \,$\cdots$\ {\em math-symbol} \texttt{\$.}\\
\end{center}
\index{\texttt{\$e} statement}
\index{\texttt{\$f} statement}
A hypothesis must have a {\em label}\index{label}.  The expression in a
\texttt{\$e} hypothesis consists of a typecode (an active constant math symbol)
followed by a sequence
of zero or more math symbols. Each math symbol (including {\em constant}
and {\em variable}) must be a previously declared constant or variable.  (In
addition, each math symbol must be active, which will be covered when we
discuss scoping statements in Section~\ref{scoping}.)  You use a \texttt{\$f}
hypothesis to specify the
nature or {\bf type}\index{variable type}\index{type} of a variable (such as ``let $x$ be an
integer'') and use a \texttt{\$e} hypothesis to express a logical truth (such as
``assume $x$ is prime'') that must be established in order for an assertion
requiring it to also be true.

A variable must have its type specified in a \texttt{\$f} statement before
it may be used in a \texttt{\$e}, \texttt{\$a}, or \texttt{\$p}
statement.  There may be only one (active) \texttt{\$f} statement for a
given variable.  (``Active'' is defined in Section~\ref{scoping}.)

In ordinary mathematics, theorems\index{theorem} are often expressed in the
form ``Assume $P$; then $Q$,'' where $Q$ is a statement that you can derive
if you start with statement $P$.\index{free variable}\footnote{A stronger
version of a theorem like this would be the {\em single} formula $P\rightarrow
Q$ ($P$ implies $Q$) from which the weaker version above follows by the rule
of modus ponens in logic.  We are not discussing this stronger form here.  In
the weaker form, we are saying only that if we can {\em prove} $P$, then we can
{\em prove} $Q$.  In a logician's language, if $x$ is the only free variable
in $P$ and $Q$, the stronger form is equivalent to $\forall x ( P \rightarrow
Q)$ (for all $x$, $P$ implies $Q$), whereas the weaker form is equivalent to
$\forall x P \rightarrow \forall x Q$. The stronger form implies the weaker,
but not vice-versa.  To be precise, the weaker form of the theorem is more
properly called an ``inference'' rather than a theorem.}\index{inference}
In the
Metamath\index{Metamath} language, you would express mathematical statement
$P$ as a hypothesis (a \texttt{\$e} Metamath language statement in this case) and
statement $Q$ as a provable assertion (a \texttt{\$p}\index{\texttt{\$p} statement}
statement).

Some examples of hypotheses you might encounter in logic and set theory are
\begin{center}
  \texttt{stmt1 \$f wff P \$.}\\
  \texttt{stmt2 \$f set x \$.}\\
  \texttt{stmt3 \$e |- ( P -> Q ) \$.}
\end{center}
\index{\texttt{\$e} statement}
\index{\texttt{\$f} statement}
Informally, these would be read, ``Let $P$ be a well-formed-formula,'' ``Let
$x$ be an (individual) variable,'' and ``Assume we have proved $P \rightarrow
Q$.''  The turnstile symbol \,$\vdash$\index{turnstile ({$\,\vdash$})} is
commonly used in logic texts to mean ``a proof exists for.''

To summarize:
\begin{itemize}
\item A \texttt{\$f} hypothesis tells Metamath the type or kind of its variable.
It is analogous to a variable declaration in a computer language that
tells the compiler that a variable is an integer or a floating-point
number.
\item The \texttt{\$e} hypothesis corresponds to what you would usually call a
``hypothesis'' in ordinary mathematics.
\end{itemize}

Before an assertion\index{assertion} (\texttt{\$a} or \texttt{\$p} statement) can be
referenced in a proof, all of its associated \texttt{\$f} and \texttt{\$e} hypotheses
(i.e.\ those \texttt{\$e} hypotheses that are active) must be satisfied (i.e.
established by the proof).  The meaning of ``associated'' (which we will call
{\bf mandatory} in Section~\ref{frames}) will become clear when we discuss
scoping later.

Note that after any \texttt{\$f}, \texttt{\$e},
\texttt{\$a}, or \texttt{\$p} token there is a required
\textit{typecode}\index{typecode}.
The typecode is a constant used to enforce types of expressions.
This will become clearer once we learn more about
assertions (\texttt{\$a} and \texttt{\$p} statements).
An example may also clarify their purpose.
In the
\texttt{set.mm}\index{set theory database (\texttt{set.mm})}%
\index{Metamath Proof Explorer}
database,
the following typecodes are used:

\begin{itemize}
\item \texttt{wff} :
  Well-formed formula (wff) symbol
  (read: ``the following symbol sequence is a wff'').
% The *textual* typecode for turnstile is "|-", but when read it's a little
% confusing, so I intentionally display the mathematical symbol here instead
% (I think it's clearer in this context).
\item \texttt{$\vdash$} :
  Turnstile (read: ``the following symbol sequence is provable'' or
  ``a proof exists for'').
\item \texttt{setvar} :
  Individual set variable type (read: ``the following is an
  individual set variable'').
  Note that this is \textit{not} the type of an arbitrary set expression,
  instead, it is used to ensure that there is only a single symbol used
  after quantifiers like for-all ($\forall$) and there-exists ($\exists$).
\item \texttt{class} :
  An expression that is a syntactically valid class expression.
  All valid set expressions are also valid class expression, so expressions
  of sets normally have the \texttt{class} typecode.
  Use the \texttt{class} typecode,
  \textit{not} the \texttt{setvar} typecode,
  for the type of set expressions unless you are specifically identifying
  a single set variable.
\end{itemize}

\subsection{Assertions (\texttt{\$a} and \texttt{\$p} Statements)}
\index{\texttt{\$a} statement}
\index{\texttt{\$p} statement}\index{assertion}\index{axiomatic assertion}
\index{provable assertion}

There are two types of assertions, \texttt{\$a}\index{\texttt{\$a} statement}
statements ({\bf axiomatic assertions}) and \texttt{\$p} statements ({\bf
provable assertions}).  Their syntax is as follows:
\begin{center}
  {\em label} \texttt{\$a} {\em typecode} {\em math-symbol} \ldots
         {\em math-symbol} \texttt{\$.}\\
  {\em label} \texttt{\$p} {\em typecode} {\em math-symbol} \ldots
        {\em math-symbol} \texttt{\$=} {\em proof} \texttt{\$.}
\end{center}
\index{\texttt{\$a} statement}
\index{\texttt{\$p} statement}
\index{\texttt{\$=} keyword}
An assertion always requires a {\em label}\index{label}. The expression in an
assertion consists of a typecode (an active constant)
followed by a sequence of zero
or more math symbols.  Each math symbol, including any {\em constant}, must be a
previously declared constant or variable.  (In addition, each math symbol
must be active, which will be covered when we discuss scoping statements in
Section~\ref{scoping}.)

A \texttt{\$a} statement is usually a definition of syntax (for example, if $P$
and $Q$ are wffs then so is $(P\to Q)$), an axiom\index{axiom} of ordinary
mathematics (for example, $x=x$), or a definition\index{definition} of
ordinary mathematics (for example, $x\ne y$ means $\lnot x=y$). A \texttt{\$p}
statement is a claim that a certain combination of math symbols follows from
previous assertions and is accompanied by a proof that demonstrates it.

Assertions can also be referenced in (later) proofs in order to derive new
assertions from them. The label of an assertion is used to refer to it in a
proof. Section~\ref{proof} will describe the proof in detail.

Assertions also provide the primary means for communicating the mathematical
results in the database to people.  Proofs (when conveniently displayed)
communicate to people how the results were arrived at.

\subsubsection{The \texttt{\$a} Statement}
\index{\texttt{\$a} statement}

Axiomatic assertions (\texttt{\$a} statements) represent the starting points from
which other assertions (\texttt{\$p}\index{\texttt{\$p} statement} statements) are
derived.  Their most obvious use is for specifying ordinary mathematical
axioms\index{axiom}, but they are also used for two other purposes.

First, Metamath\index{Metamath} needs to know the syntax of symbol
sequences that constitute valid mathematical statements.  A Metamath
proof must be broken down into much more detail than ordinary
mathematical proofs that you may be used to thinking of (even the
``complete'' proofs of formal logic\index{formal logic}).  This is one
of the things that makes Metamath a general-purpose language,
independent of any system of logic or even syntax.  If you want to use a
substitution instance of an assertion as a step in a proof, you must
first prove that the substitution is syntactically correct (or if you
prefer, you must ``construct'' it), showing for example that the
expression you are substituting for a wff metavariable is a valid wff.
The \texttt{\$a}\index{\texttt{\$a} statement} statement is used to
specify those combinations of symbols that are considered syntactically
valid, such as the legal forms of wffs.

Second, \texttt{\$a} statements are used to specify what are ordinarily thought of
as definitions, i.e.\ new combinations of symbols that abbreviate other
combinations of symbols.  Metamath makes no distinction\index{axiom vs.\
definition} between axioms\index{axiom} and definitions\index{definition}.
Indeed, it has been argued that such distinction should not be made even in
ordinary mathematics; see Section~\ref{definitions}, which discusses the
philosophy of definitions.  Section~\ref{hierarchy} discusses some
technical requirements for definitions.  In \texttt{set.mm} we adopt the
convention of prefixing axiom labels with \texttt{ax-} and definition labels with
\texttt{df-}\index{label}.

The results that can be derived with the Metamath language are only as good as
the \texttt{\$a}\index{\texttt{\$a} statement} statements used as their starting
point.  We cannot stress this too strongly.  For example, Metamath will
not prevent you from specifying $x\neq x$ as an axiom of logic.  It is
essential that you scrutinize all \texttt{\$a} statements with great care.
Because they are a source of potential pitfalls, it is best not to add new
ones (usually new definitions) casually; rather you should carefully evaluate
each one's necessity and advantages.

Once you have in place all of the basic axioms\index{axiom} and
rules\index{rule} of a mathematical theory, the only \texttt{\$a} statements that
you will be adding will be what are ordinarily called definitions.  In
principle, definitions should be in some sense eliminable from the language of
a theory according to some convention (usually involving logical equivalence
or equality).  The most common convention is that any formula that was
syntactically valid but not provable before the definition was introduced will
not become provable after the definition is introduced.  In an ideal world,
definitions should not be present at all if one is to have absolute confidence
in a mathematical result.  However, they are necessary to make
mathematics practical, for otherwise the resulting formulas would be
extremely long and incomprehensible.  Since the nature of definitions (in the
most general sense) does not permit them to automatically be verified as
``proper,''\index{proper definition}\index{definition!proper} the judgment of
the mathematician is required to ensure it.  (In \texttt{set.mm} effort was made
to make almost all definitions directly eliminable and thus minimize the need
for such judgment.)

If you are not a mathematician, it may be best not to add or change any
\texttt{\$a}\index{\texttt{\$a} statement} statements but instead use
the mathematical language already provided in standard databases.  This
way Metamath will not allow you to make a mistake (i.e.\ prove a false
result).


\subsection{Frames}\label{frames}

We now introduce the concept of a collection of related Metamath statements
called a frame.  Every assertion (\texttt{\$a} or \texttt{\$p} statement) in the database has
an associated frame.

A {\bf frame}\index{frame} is a sequence of \texttt{\$d}, \texttt{\$f},
and \texttt{\$e} statements (zero or more of each) followed by one
\texttt{\$a} or \texttt{\$p} statement, subject to certain conditions we
will describe.  For simplicity we will assume that all math symbol
tokens used are declared at the beginning of the database with
\texttt{\$c} and \texttt{\$v} statements (which are not properly part of
a frame).  Also for simplicity we will assume there are only simple
\texttt{\$d} statements (those with only two variables) and imagine any
compound \texttt{\$d} statements (those with more than two variables) as
broken up into simple ones.

A frame groups together those hypotheses (and \texttt{\$d} statements) relevant
to an assertion (\texttt{\$a} or \texttt{\$p} statement).  The statements in a frame
may or may not be physically adjacent in a database; we will cover
this in our discussion of scoping statements
in Section~\ref{scoping}.

A frame has the following properties:
\begin{enumerate}
 \item The set of variables contained in its \texttt{\$f} statements must
be identical to the set of variables contained in its \texttt{\$e},
\texttt{\$a}, and/or \texttt{\$p} statements.  In other words, each
variable in a \texttt{\$e}, \texttt{\$a}, or \texttt{\$p} statement must
have an associated ``variable type'' defined for it in a \texttt{\$f}
statement.
  \item No two \texttt{\$f} statements may contain the same variable.
  \item Any \texttt{\$f} statement
must occur before a \texttt{\$e} statement in which its variable occurs.
\end{enumerate}

The first property determines the set of variables occurring in a frame.
These are the {\bf mandatory
variables}\index{mandatory variable} of the frame.  The second property
tells us there must be only one type specified for a variable.
The last property is not a theoretical requirement but it
makes parsing of the database easier.

For our examples, we assume our database has the following declarations:

\begin{verbatim}
$v P Q R $.
$c -> ( ) |- wff $.
\end{verbatim}

The following sequence of statements, describing the modus ponens inference
rule, is an example of a frame:

\begin{verbatim}
wp  $f wff P $.
wq  $f wff Q $.
maj $e |- ( P -> Q ) $.
min $e |- P $.
mp  $a |- Q $.
\end{verbatim}

The following sequence of statements is not a frame because \texttt{R} does not
occur in the \texttt{\$e}'s or the \texttt{\$a}:

\begin{verbatim}
wp  $f wff P $.
wq  $f wff Q $.
wr  $f wff R $.
maj $e |- ( P -> Q ) $.
min $e |- P $.
mp  $a |- Q $.
\end{verbatim}

The following sequence of statements is not a frame because \texttt{Q} does not
occur in a \texttt{\$f}:

\begin{verbatim}
wp  $f wff P $.
maj $e |- ( P -> Q ) $.
min $e |- P $.
mp  $a |- Q $.
\end{verbatim}

The following sequence of statements is not a frame because the \texttt{\$a} statement is
not the last one:

\begin{verbatim}
wp  $f wff P $.
wq  $f wff Q $.
maj $e |- ( P -> Q ) $.
mp  $a |- Q $.
min $e |- P $.
\end{verbatim}

Associated with a frame is a sequence of {\bf mandatory
hypotheses}\index{mandatory hypothesis}.  This is simply the set of all
\texttt{\$f} and \texttt{\$e} statements in the frame, in the order they
appear.  A frame can be referenced in a later proof using the label of
the \texttt{\$a} or \texttt{\$p} assertion statement, and the proof
makes an assignment to each mandatory hypothesis in the order in which
it appears.  This means the order of the hypotheses, once chosen, must
not be changed so as not to affect later proofs referencing the frame's
assertion statement.  (The Metamath proof verifier will, of course, flag
an error if a proof becomes incorrect by doing this.)  Since proofs make
use of ``Reverse Polish notation,'' described in Section~\ref{proof}, we
call this order the {\bf RPN order}\index{RPN order} of the hypotheses.

Note that \texttt{\$d} statements are not part of the set of mandatory
hypotheses, and their order doesn't matter (as long as they satisfy the
fourth property for a frame described above).  The \texttt{\$d}
statements specify restrictions on variables that must be satisfied (and
are checked by the proof verifier) when expressions are substituted for
them in a proof, and the \texttt{\$d} statements themselves are never
referenced directly in a proof.

A frame with a \texttt{\$p} (provable) statement requires a proof as part of the
\texttt{\$p} statement.  Sometimes in a proof we want to make use of temporary or
dummy variables\index{dummy variable} that do not occur in the \texttt{\$p}
statement or its mandatory hypotheses.  To accommodate this we define an {\bf
extended frame}\index{extended frame} as a frame together with zero or more
\texttt{\$d} and \texttt{\$f} statements that reference variables not among the
mandatory variables of the frame.  Any new variables referenced are called the
{\bf optional variables}\index{optional variable} of the extended frame. If a
\texttt{\$f} statement references an optional variable it is called an {\bf
optional hypothesis}\index{optional hypothesis}, and if one or both of the
variables in a \texttt{\$d} statement are optional variables it is called an {\bf
optional disjoint-variable restriction}\index{optional disjoint-variable
restriction}.  Properties 2, 3, and 4 for a frame also apply to an extended
frame.

The concept of optional variables is not meaningful for frames with \texttt{\$a}
statements, since those statements have no proofs that might make use of them.
There is no restriction on including optional hypotheses in the extended frame
for a \texttt{\$a} statement, but they serve no purpose.

The following set of statements is an example of an extended frame, which
contains an optional variable \texttt{R} and an optional hypothesis \texttt{wr}.  In
this example, we suppose the rule of modus ponens is not an axiom but is
derived as a theorem from earlier statements (we omit its presumed proof).
Variable \texttt{R} may be used in its proof if desired (although this would
probably have no advantage in propositional calculus).  Note that the sequence
of mandatory hypotheses in RPN order is still \texttt{wp}, \texttt{wq}, \texttt{maj},
\texttt{min} (i.e.\ \texttt{wr} is omitted), and this sequence is still assumed
whenever the assertion \texttt{mp} is referenced in a subsequent proof.

\begin{verbatim}
wp  $f wff P $.
wq  $f wff Q $.
wr  $f wff R $.
maj $e |- ( P -> Q ) $.
min $e |- P $.
mp  $p |- Q $= ... $.
\end{verbatim}

Every frame is an extended frame, but not every extended frame is a frame, as
this example shows.  The underlying frame for an extended frame is
obtained by simply removing all statements containing optional variables.
Any proof referencing an assertion will ignore any extensions to its
frame, which means we may add or delete optional hypotheses at will without
affecting subsequent proofs.

The conceptually simplest way of organizing a Metamath database is as a
sequence of extended frames.  The scoping statements
\texttt{\$\char`\{}\index{\texttt{\$\char`\{} and \texttt{\$\char`\}}
keywords} and \texttt{\$\char`\}} can be used to delimit the start and
end of an extended frame, leading to the following possible structure for a
database.  \label{framelist}

\vskip 2ex
\setbox\startprefix=\hbox{\tt \ \ \ \ \ \ \ \ }
\setbox\contprefix=\hbox{}
\startm
\m{\mbox{(\texttt{\$v} {\em and} \texttt{\$c}\,{\em statements})}}
\endm
\startm
\m{\mbox{\texttt{\$\char`\{}}}
\endm
\startm
\m{\mbox{\texttt{\ \ } {\em extended frame}}}
\endm
\startm
\m{\mbox{\texttt{\$\char`\}}}}
\endm
\startm
\m{\mbox{\texttt{\$\char`\{}}}
\endm
\startm
\m{\mbox{\texttt{\ \ } {\em extended frame}}}
\endm
\startm
\m{\mbox{\texttt{\$\char`\}}}}
\endm
\startm
\m{\mbox{\texttt{\ \ \ \ \ \ \ \ \ }}\vdots}
\endm
\vskip 2ex

In practice, this structure is inconvenient because we have to repeat
any \texttt{\$f}, \texttt{\$e}, and \texttt{\$d} statements over and
over again rather than stating them once for use by several assertions.
The scoping statements, which we will discuss next, allow this to be
done.  In principle, any Metamath database can be converted to the above
format, and the above format is the most convenient to use when studying
a Metamath database as a formal system%
%% Uncomment this when uncommenting section {formalspec} below
   (Appendix \ref{formalspec})%
.
In fact, Metamath internally converts the database to the above format.
The command \texttt{show statement} in the Metamath program will show
you the contents of the frame for any \texttt{\$a} or \texttt{\$p}
statement, as well as its extension in the case of a \texttt{\$p}
statement.

%c%(provided that all ``local'' variables and constants with limited scope have
%c%unique names),

During our discussion of scoping statements, it may be helpful to
think in terms of the equivalent sequence of frames that will result when
the database is parsed.  Scoping (other than the limited
use above to delimit frames) is not a theoretical requirement for
Metamath but makes it more convenient.


\subsection{Scoping Statements (\texttt{\$\{} and \texttt{\$\}})}\label{scoping}
\index{\texttt{\$\char`\{} and \texttt{\$\char`\}} keywords}\index{scoping statement}

%c%Some Metamath statements may be needed only temporarily to
%c%serve a specific purpose, and after we're done with them we would like to
%c%disregard or ignore them.  For example, when we're finished using a variable,
%c%we might want to
%c%we might want to free up the token\index{token} used to name it so that the
%c%token can be used for other purposes later on, such as a different kind of
%c%variable or even a constant.  In the terminology of computer programming, we
%c%might want to let some symbol declarations be ``local'' rather than ``global.''
%c%\index{local symbol}\index{global symbol}

The {\bf scoping} statements, \texttt{\$\char`\{} ({\bf start of block}) and \texttt{\$\char`\}}
({\bf end of block})\index{block}, provide a means for controlling the portion
of a database over which certain statement types are recognized.  The
syntax of a scoping statement is very simple; it just consists of the
statement's keyword:
\begin{center}
\texttt{\$\char`\{}\\
\texttt{\$\char`\}}
\end{center}
\index{\texttt{\$\char`\{} and \texttt{\$\char`\}} keywords}

For example, consider the following database where we have stripped out
all tokens except the scoping statement keywords.  For the purpose of the
discussion, we have added subscripts to the scoping statements; these subscripts
do not appear in the actual database.
\[
 \mbox{\tt \ \$\char`\{}_1
 \mbox{\tt \ \$\char`\{}_2
 \mbox{\tt \ \$\char`\}}_2
 \mbox{\tt \ \$\char`\{}_3
 \mbox{\tt \ \$\char`\{}_4
 \mbox{\tt \ \$\char`\}}_4
 \mbox{\tt \ \$\char`\}}_3
 \mbox{\tt \ \$\char`\}}_1
\]
Each \texttt{\$\char`\{} statement in this example is said to be {\bf
matched} with the \texttt{\$\char`\}} statement that has the same
subscript.  Each pair of matched scoping statements defines a region of
the database called a {\bf block}.\index{block} Blocks can be {\bf
nested}\index{nested block} inside other blocks; in the example, the
block defined by $\mbox{\tt \$\char`\{}_4$ and $\mbox{\tt \$\char`\}}_4$
is nested inside the block defined by $\mbox{\tt \$\char`\{}_3$ and
$\mbox{\tt \$\char`\}}_3$ as well as inside the block defined by
$\mbox{\tt \$\char`\{}_1$ and $\mbox{\tt \$\char`\}}_1$.  In general, a
block may be empty, it may contain only non-scoping
statements,\footnote{Those statements other than \texttt{\$\char`\{} and
\texttt{\$\char`\}}.}\index{non-scoping statement} or it may contain any
mixture of other blocks and non-scoping statements.  (This is called a
``recursive'' definition\index{recursive definition} of a block.)

Associated with each block is a number called its {\bf nesting
level}\index{nesting level} that indicates how deeply the block is nested.
The nesting levels of the blocks in our example are as follows:
\[
  \underbrace{
    \mbox{\tt \ }
    \underbrace{
     \mbox{\tt \$\char`\{\ }
     \underbrace{
       \mbox{\tt \$\char`\{\ }
       \mbox{\tt \$\char`\}}
     }_{2}
     \mbox{\tt \ }
     \underbrace{
       \mbox{\tt \$\char`\{\ }
       \underbrace{
         \mbox{\tt \$\char`\{\ }
         \mbox{\tt \$\char`\}}
       }_{3}
       \mbox{\tt \ \$\char`\}}
     }_{2}
     \mbox{\tt \ \$\char`\}}
   }_{1}
   \mbox{\tt \ }
 }_{0}
\]
\index{\texttt{\$\char`\{} and \texttt{\$\char`\}} keywords}
The entire database is considered to be one big block (the {\bf outermost}
block) with a nesting level of 0.  The outermost block is {\em not} bracketed
by scoping statements.\footnote{The language was designed this way so that
several source files can be joined together more easily.}\index{outermost
block}

All non-scoping Metamath statements become recognized or {\bf
active}\index{active statement} at the place where they appear.\footnote{To
keep things slightly simpler, we do not bother to define the concept of
``active'' for the scoping statements.}  Certain of these statement types
become inactive at the end of the block in which they appear; these statement
types are:
\begin{center}
  \texttt{\$c}, \texttt{\$v}, \texttt{\$d}, \texttt{\$e}, and \texttt{\$f}.
%  \texttt{\$v}, \texttt{\$f}, \texttt{\$e}, and \texttt{\$d}.
\end{center}
\index{\texttt{\$c} statement}
\index{\texttt{\$d} statement}
\index{\texttt{\$e} statement}
\index{\texttt{\$f} statement}
\index{\texttt{\$v} statement}
The other statement types remain active forever (i.e.\ through the end of the
database); they are:
\begin{center}
  \texttt{\$a} and \texttt{\$p}.
%  \texttt{\$c}, \texttt{\$a}, and \texttt{\$p}.
\end{center}
\index{\texttt{\$a} statement}
\index{\texttt{\$p} statement}
Any statement (of these 7 types) located in the outermost
block\index{outermost block} will remain active through the end of the
database and thus are effectively ``global'' statements.\index{global
statement}

All \texttt{\$c} statements must be placed in the outermost block.  Since they are
therefore always global, they could be considered as belonging to both of the
above categories.

The {\bf scope}\index{scope} of a statement is the set of statements that
recognize it as active.

%c%The concept of ``active'' is also defined for math symbols\index{math
%c%symbol}.  Math symbols (constants\index{constant} and
%c%variables\index{variable}) become {\bf active}\index{active
%c%math symbol} in the \texttt{\$c}\index{\texttt{\$c}
%c%statement} and \texttt{\$v}\index{\texttt{\$v} statement} statements that
%c%declare them.  They become inactive when their declaration statements become
%c%inactive.

The concept of ``active'' is also defined for math symbols\index{math
symbol}.  Math symbols (constants\index{constant} and
variables\index{variable}) become {\bf active}\index{active math symbol}
in the \texttt{\$c}\index{\texttt{\$c} statement} and
\texttt{\$v}\index{\texttt{\$v} statement} statements that declare them.
A variable becomes inactive when its declaration statement becomes
inactive.  Because all \texttt{\$c} statements must be in the outermost
block, a constant will never become inactive after it is declared.

\subsubsection{Redeclaration of Math Symbols}
\index{redeclaration of symbols}\label{redeclaration}

%c%A math symbol may not be declared a second time while it is active, but it may
%c%be declared again after it becomes inactive.

A variable may not be declared a second time while it is active, but it may be
declared again after it becomes inactive.  This provides a convenient way to
introduce ``local'' variables,\index{local variable} i.e.\ temporary variables
for use in the frame of an assertion or in a proof without keeping them around
forever.  A previously declared variable may not be redeclared as a constant.

A constant may not be redeclared.  And, as mentioned above, constants must be
declared in the outermost block.

The reason variables may have limited scope but not constants is that an
assertion (\texttt{\$a} or \texttt{\$p} statement) remains available for use in
proofs through the end of the database.  Variables in an assertion's frame may
be substituted with whatever is needed in a proof step that references the
assertion, whereas constants remain fixed and may not be substituted with
anything.  The particular token used for a variable in an assertion's frame is
irrelevant when the assertion is referenced in a proof, and it doesn't matter
if that token is not available outside of the referenced assertion's frame.
Constants, however, must be globally fixed.

In the present version of the Metamath language, there is no theoretical
benefit for the feature allowing variables to be active for limited scopes
rather than global. It is just a convenience that allows them, for example, to
be locally grouped together with their corresponding \texttt{\$f} variable-type
declarations.\footnote{A future version\label{redeclarationf} of the Metamath
language may extend the language to include true local variables as well as
local constants.  I have not decided upon the best way to do this or even if
it should be done at all; the main issue is whether the benefits offset
greater confusion in learning the language.  In any case, the rules for
the present language were chosen to let existing databases be compatible with
any future language extensions.}

%c%If you declare a math symbol more than once, internally Metamath considers it a
%c%new distinct symbol, even though it has the same name.  If you are unaware of
%c%this, you may find that what you think are correct proofs are incorrectly
%c%rejected as invalid, because Metamath may tell you that a constant you
%c%previously declared does not match a newly declared math symbol with the same
%c%name.  For details on this subtle point, see the Comment on
%c%p.~\pageref{spec4comment}.  This is done purposely to allow temporary
%c%constants to be introduced while developing a subtheory, then allow their math
%c%symbol tokens to be reused later on; in general they will not refer to the
%c%same thing.  In practice, you would not ordinarily reuse the names of
%c%constants because it would tend to be confusing to the reader.  The reuse of
%c%names of variables, on the other hand, is something that is often useful to do
%c%(for example it is done frequently in \texttt{set.mm}).  Since variables in an
%c%assertion referenced in a proof can be substituted as needed to achieve a
%c%symbol match, this is not an issue.

% (This section covers a somewhat advanced topic you may want to skip
% at first reading.)
%
% Under certain circumstances, math symbol\index{math symbol}
% tokens\index{token} may be redeclared (i.e.\ the token
% may appear in more than
% one \texttt{\$c}\index{\texttt{\$c} statement} or \texttt{\$v}\index{\texttt{\$v}
% statement} statement).  You might want to do this say, to make temporary use
% of a variable name without having to worry about its affect elsewhere,
% somewhat analogous to declaring a local variable in a standard computer
% language.  Understanding what goes on when math symbol tokens are redeclared
% is a little tricky to understand at first, since it requires that we
% distinguish the token itself from the math symbol that it names.  It will help
% if we first take a peek at the internal workings of the
% Metamath\index{Metamath} program.
%
% Metamath reserves a memory location for each occurrence of a
% token\index{token} in a declaration statement (\texttt{\$c}\index{\texttt{\$c}
% statement} or \texttt{\$v}\index{\texttt{\$v} statement}).  If a given token appears
% in more than one declaration statement, it will refer to more than one memory
% locations.  A math symbol\index{math symbol} may be thought of as being one of
% these memory locations rather than as the token itself.  Only one of the
% memory locations associated with a given token may be active at any one time.
% The math symbol (memory location) that gets looked up when the token appears
% in a non-declaration statement is the one that happens to be active at that
% time.
%
% We now look at the rules for the redeclaration\index{redeclaration of symbols}
% of math symbol tokens.
% \begin{itemize}
% \item A math symbol token may not be declared twice in the
% same block.\footnote{While there is no theoretical reason for disallowing
% this, it was decided in the design of Metamath that allowing it would offer no
% advantage and might cause confusion.}
% \item An inactive math symbol may always be
% redeclared.
% \item  An active math symbol may be redeclared in a different (i.e.\
% inner) block\index{block} from the one it became active in.
% \end{itemize}
%
% When a math symbol token is redeclared, it conceptually refers to a different
% math symbol, just as it would be if it were called a different name.  In
% addition, the original math symbol that it referred to, if it was active,
% temporarily becomes inactive.  At the end of the block in which the
% redeclaration occurred, the new math symbol\index{math symbol} becomes
% inactive and the original symbol becomes active again.  This concept is
% illustrated in the following example, where the symbol \texttt{e} is
% ordinarily a constant (say Euler's constant, 2.71828...) but
% temporarily we want to use it as a ``local'' variable, say as a coefficient
% in the equation $a x^4 + b x^3 + c x^2 + d x + e$:
% \[
%   \mbox{\tt \$\char`\{\ \$c e \$.}
%   \underbrace{
%     \ \ldots\ %
%     \mbox{\tt \$\char`\{}\ \ldots\ %
%   }_{\mbox{\rm region A}}
%   \mbox{\tt \$v e \$.}
%   \underbrace{
%     \mbox{\ \ \ \ldots\ \ \ }
%   }_{\mbox{\rm region B}}
%   \mbox{\tt \$\char`\}}
%   \underbrace{
%     \mbox{\ \ \ \ldots\ \ \ }
%   }_{\mbox{\rm region C}}
%   \mbox{\tt \$\char`\}}
% \]
% \index{\texttt{\$\char`\{} and \texttt{\$\char`\}} keywords}
% In region A, the token \texttt{e} refers to a constant.  It is redeclared as a
% variable in region B, and any reference to it in this region will refer to this
% variable.  In region C, the redeclaration becomes inactive, and the original
% declaration becomes active again.  In region C, the token \texttt{x} refers to the
% original constant.
%
% As a practical matter, overuse of math symbol\index{math symbol}
% redeclarations\index{redeclaration of symbols} can be confusing (even though
% it is well-defined) and is best avoided when possible.  Here are some good
% general guidelines you can follow.  Usually, you should declare all
% constants\index{constant} in the outermost block\index{outermost block},
% especially if they are general-purpose (such as the token \verb$A.$, meaning
% $\forall$ or ``for all'').  This will make them ``globally'' active (although
% as in the example above local redeclarations will temporarily make them
% inactive.)  Most or all variables\index{variable}, on the other hand, could be
% declared in inner blocks, so that the token for them can be used later for a
% different type of variable or a constant.  (The names of the variables you
% choose are not used when you refer to an assertion\index{assertion} in a
% proof, whereas constants must match exactly.  A locally declared constant will
% not match a globally declared constant in a proof, even if they use the same
% token, because Metamath internally considers them to be different math
% symbols.)  To avoid confusion, you should generally avoid redeclaring active
% variables.  If you must redeclare them, do so at the beginning of a block.
% The temporary declaration of constants in inner blocks might be occasionally
% appropriate when you make use of a temporary definition to prove lemmas
% leading to a main result that does not make direct use of the definition.
% This way, you will not clutter up your database with a large number of
% seldom-used global constant symbols.  You might want to note that while
% inactive constants may not appear directly in an assertion (a \texttt{\$a}\index{\texttt{\$a}
% statement} or \texttt{\$p}\index{\texttt{\$p} statement}
% statement), they may be indirectly used in the proof of a \texttt{\$p} statement
% so long as they do not appear in the final math symbol sequence constructed by
% the proof.  In the end, you will have to use your best judgment, taking into
% account standard mathematical usage of the symbols as well as consideration
% for the reader of your work.
%
% \subsubsection{Reuse of Labels}\index{reuse of labels}\index{label}
%
% The \texttt{\$e}\index{\texttt{\$e} statement}, \texttt{\$f}\index{\texttt{\$f}
% statement}, \texttt{\$a}\index{\texttt{\$a} statement}, and
% \texttt{\$p}\index{\texttt{\$p}
% statement} statement types require labels, which allow them to be
% referenced later inside proofs.  A label is considered {\bf
% active}\index{active label} when the statement it is associated with is
% active.  The token\index{token} for a label may be reused
% (redeclared)\index{redeclaration of labels} provided that it is not being used
% for a currently active label.  (Unlike the tokens for math symbols, active
% label tokens may not be redeclared in an inner scope.)  Note that the labels
% of \texttt{\$a} and \texttt{\$p} statements can never be reused after these
% statements appear, because these statements remain active through the end of
% the database.
%
% You might find the reuse of labels a convenient way to have standard names for
% temporary hypotheses, such as \texttt{h1}, \texttt{h2}, etc.  This way you don't have
% to invent unique names for each of them, and in some cases it may be less
% confusing to the reader (although in other cases it might be more confusing, if
% the hypothesis is located far away from the assertion that uses
% it).\footnote{The current implementation requires that all labels, even
% inactive ones, be unique.}

\subsubsection{Frames Revisited}\index{frames and scoping statements}

Now that we have covered scoping, we will look at how an arbitrary
Metamath database can be converted to the simple sequence of extended
frames described on p.~\pageref{framelist}.  This is also how Metamath
stores the database internally when it reads in the database
source.\label{frameconvert} The method is simple.  First, we collect all
constant and variable (\texttt{\$c} and \texttt{\$v}) declarations in
the database, ignoring duplicate declarations of the same variable in
different scopes.  We then put our collected \texttt{\$c} and
\texttt{\$v} declarations at the beginning of the database, so that
their scope is the entire database.  Next, for each assertion in the
database, we determine its frame and extended frame.  The extended frame
is simply the \texttt{\$f}, \texttt{\$e}, and \texttt{\$d} statements
that are active.  The frame is the extended frame with all optional
hypotheses removed.

An equivalent way of saying this is that the extended frame of an assertion
is the collection of all \texttt{\$f}, \texttt{\$e}, and \texttt{\$d} statements
whose scope includes the assertion.
The \texttt{\$f} and \texttt{\$e} statements
occur in the order they appear
(order is irrelevant for \texttt{\$d} statements).

%c%, renaming any
%c%redeclared variables as needed so that all of them have unique names.  (The
%c%exact renaming convention is unimportant.  You might imagine renaming
%c%different declarations of math symbol \texttt{a} as \texttt{a\$1}, \texttt{a\$2}, etc.\
%c%which would prevent any conflicts since \texttt{\$} is not a legal character in a
%c%math symbol token.)

\section{The Anatomy of a Proof} \label{proof}
\index{proof!Metamath, description of}

Each provable assertion (\texttt{\$p}\index{\texttt{\$p} statement} statement) in a
database must include a {\bf proof}\index{proof}.  The proof is located
between the \texttt{\$=}\index{\texttt{\$=} keyword} and \texttt{\$.}\ keywords in the
\texttt{\$p} statement.

In the basic Metamath language\index{basic language}, a proof is a
sequence of statement labels.  This label sequence\index{label sequence}
serves as a set of instructions that the Metamath program uses to
construct a series of math symbol sequences.  The construction must
ultimately result in the math symbol sequence contained between the
\texttt{\$p}\index{\texttt{\$p} statement} and
\texttt{\$=}\index{\texttt{\$=} keyword} keywords of the \texttt{\$p}
statement.  Otherwise, the Metamath program will consider the proof
incorrect, and it will notify you with an appropriate error message when
you ask it to verify the proof.\footnote{To make the loading faster, the
Metamath program does not automatically verify proofs when you
\texttt{read} in a database unless you use the \texttt{/verify}
qualifier.  After a database has been read in, you may use the
\texttt{verify proof *} command to verify proofs.}\index{\texttt{verify
proof} command} Each label in a proof is said to {\bf
reference}\index{label reference} its corresponding statement.

Associated with any assertion\index{assertion} (\texttt{\$p} or
\texttt{\$a}\index{\texttt{\$a} statement} statement) is a set of
hypotheses (\texttt{\$f}\index{\texttt{\$f} statement} or
\texttt{\$e}\index{\texttt{\$e} statement} statements) that are active
with respect to that assertion.  Some are mandatory and the others are
optional.  You should review these concepts if necessary.

Each label\index{label} in a proof must be either the label of a
previous assertion (\texttt{\$a}\index{\texttt{\$a} statement} or
\texttt{\$p}\index{\texttt{\$p} statement} statement) or the label of an
active hypothesis (\texttt{\$e} or \texttt{\$f}\index{\texttt{\$f}
statement} statement) of the \texttt{\$p} statement containing the
proof.  Hypothesis labels may reference both the
mandatory\index{mandatory hypothesis} and the optional hypotheses of the
\texttt{\$p} statement.

The label sequence in a proof specifies a construction in {\bf reverse Polish
notation}\index{reverse Polish notation (RPN)} (RPN).  You may be familiar
with RPN if you have used older
Hewlett--Packard or similar hand-held calculators.
In the calculator analogy, a hypothesis label\index{hypothesis label} is like
a number and an assertion label\index{assertion label} is like an operation
(more precisely, an $n$-ary operation when the
assertion has $n$ \texttt{\$e}-hypotheses).
On an RPN calculator, an operation takes one or more previous numbers in an
input sequence, performs a calculation on them, and replaces those numbers and
itself with the result of the calculation.  For example, the input sequence
$2,3,+$ on an RPN calculator results in $5$, and the input sequence
$2,3,5,{\times},+$ results in $2,15,+$ which results in $17$.

Understanding how RPN is processed involves the concept of a {\bf
stack}\index{stack}\index{RPN stack}, which can be thought of as a set of
temporary memory locations that hold intermediate results.  When Metamath
encounters a hypothesis label it places or {\bf pushes}\index{push} the math
symbol sequence of the hypothesis onto the stack.  When Metamath encounters an
assertion label, it associates the most recent stack entries with the {\em
mandatory} hypotheses\index{mandatory hypothesis} of the assertion, in the
order where the most recent stack entry is associated with the last mandatory
hypothesis of the assertion.  It then determines what
substitutions\index{substitution!variable}\index{variable substitution} have
to be made into the variables of the assertion's mandatory hypotheses to make
them identical to the associated stack entries.  It then makes those same
substitutions into the assertion itself.  Finally, Metamath removes or {\bf
pops}\index{pop} the matched hypotheses from the stack and pushes the
substituted assertion onto the stack.

For the purpose of matching the mandatory hypothesis to the most recent stack
entries, whether a hypothesis is a \texttt{\$e} or \texttt{\$f} statement is
irrelevant.  The only important thing is that a set of
substitutions\footnote{In the Metamath spec (Section~\ref{spec}), we use the
singular term ``substitution'' to refer to the set of substitutions we talk
about here.} exist that allow a match (and if they don't, the proof verifier
will let you know with an error message).  The Metamath language is specified
in such a way that if a set of substitutions exists, it will be unique.
Specifically, the requirement that each variable have a type specified for it
with a \texttt{\$f} statement ensures the uniqueness.

We will illustrate this with an example.
Consider the following Metamath source file:
\begin{verbatim}
$c ( ) -> wff $.
$v p q r s $.
wp $f wff p $.
wq $f wff q $.
wr $f wff r $.
ws $f wff s $.
w2 $a wff ( p -> q ) $.
wnew $p wff ( s -> ( r -> p ) ) $= ws wr wp w2 w2 $.
\end{verbatim}
This Metamath source example shows the definition and ``proof'' (i.e.,
construction) of a well-formed formula (wff)\index{well-formed formula (wff)}
in propositional calculus.  (You may wish to type this example into a file to
experiment with the Metamath program.)  The first two statements declare
(introduce the names of) four constants and four variables.  The next four
statements specify the variable types, namely that
each variable is assumed to be a wff.  Statement \texttt{w2} defines (postulates)
a way to produce a new wff, \texttt{( p -> q )}, from two given wffs \texttt{p} and
\texttt{q}. The mandatory hypotheses of \texttt{w2} are \texttt{wp} and \texttt{wq}.
Statement \texttt{wnew} claims that \texttt{( s -> ( r -> p ) )} is a wff given
three wffs \texttt{s}, \texttt{r}, and \texttt{p}.  More precisely, \texttt{wnew} claims
that the sequence of ten symbols \texttt{wff ( s -> ( r -> p ) )} is provable from
previous assertions and the hypotheses of \texttt{wnew}.  Metamath does not know
or care what a wff is, and as far as it is concerned
the typecode \texttt{wff} is just an
arbitrary constant symbol in a math symbol sequence.  The mandatory hypotheses
of \texttt{wnew} are \texttt{wp}, \texttt{wr}, and \texttt{ws}; \texttt{wq} is an optional
hypothesis.  In our particular proof, the optional hypothesis is not
referenced, but in general, any combination of active (i.e.\ optional and
mandatory) hypotheses could be referenced.  The proof of statement \texttt{wnew}
is the sequence of five labels starting with \texttt{ws} (step~1) and ending with
\texttt{w2} (step~5).

When Metamath verifies the proof, it scans the proof from left to right.  We
will examine what happens at each step of the proof.  The stack starts off
empty.  At step 1, Metamath looks up label \texttt{ws} and determines that it is a
hypothesis, so it pushes the symbol sequence of statement \texttt{ws} onto the
stack:

\begin{center}\begin{tabular}{|l|l|}\hline
{Stack location} & {Contents} \\ \hline \hline
1 & \texttt{wff s} \\ \hline
\end{tabular}\end{center}

Metamath sees that the labels \texttt{wr} and \texttt{wp} in steps~2 and 3 are also
hypotheses, so it pushes them onto the stack.  After step~3, the stack looks
like
this:

\begin{center}\begin{tabular}{|l|l|}\hline
{Stack location} & {Contents} \\ \hline \hline
3 & \texttt{wff p} \\ \hline
2 & \texttt{wff r} \\ \hline
1 & \texttt{wff s} \\ \hline
\end{tabular}\end{center}

At step 4, Metamath sees that label \texttt{w2} is an assertion, so it must do
some processing.  First, it associates the mandatory hypotheses of \texttt{w2},
which are \texttt{wp} and \texttt{wq}, with stack locations~2 and 3, {\em in that
order}. Metamath determines that the only possible way
to make hypothesis \texttt{wp} match (become identical to) stack location~2 and
\texttt{wq} match stack location 3 is to substitute variable \texttt{p} with \texttt{r}
and \texttt{q} with \texttt{p}.  Metamath makes these substitutions into \texttt{w2} and
obtains the symbol sequence \texttt{wff ( r -> p )}.  It removes the hypotheses
from stack locations~2 and 3, then places the result into stack location~2:

\begin{center}\begin{tabular}{|l|l|}\hline
{Stack location} & {Contents} \\ \hline \hline
2 & \texttt{wff ( r -> p )} \\ \hline
1 & \texttt{wff s} \\ \hline
\end{tabular}\end{center}

At step 5, Metamath sees that label \texttt{w2} is an assertion, so it must again
do some processing.  First, it matches the mandatory hypotheses of \texttt{w2},
which are \texttt{wp} and \texttt{wq}, to stack locations 1 and 2.
Metamath determines that the only possible way to make the
hypotheses match is to substitute variable \texttt{p} with \texttt{s} and \texttt{q} with
\texttt{( r -> p )}.  Metamath makes these substitutions into \texttt{w2} and obtains
the symbol
sequence \texttt{wff ( s -> ( r -> p ) )}.  It removes stack
locations 1 and 2, then places the result into stack location~1:

\begin{center}\begin{tabular}{|l|l|}\hline
{Stack location} & {Contents} \\ \hline \hline
1 & \texttt{wff ( s -> ( r -> p ) )} \\ \hline
\end{tabular}\end{center}

After Metamath finishes processing the proof, it checks to see that the
stack contains exactly one element and that this element is
the same as the math symbol sequence in the
\texttt{\$p}\index{\texttt{\$p} statement} statement.  This is the case for our
proof of \texttt{wnew},
so we have proved \texttt{wnew} successfully.  If the result
differs, Metamath will notify you with an error message.  An error message
will also result if the stack contains more than one entry at the end of the
proof, or if the stack did not contain enough entries at any point in the
proof to match all of the mandatory hypotheses\index{mandatory hypothesis} of
an assertion.  Finally, Metamath will notify you with an error message if no
substitution is possible that will make a referenced assertion's hypothesis
match the
stack entries.  You may want to experiment with the different kinds of errors
that Metamath will detect by making some small changes in the proof of our
example.

Metamath's proof notation was designed primarily to express proofs in a
relatively compact manner, not for readability by humans.  Metamath can display
proofs in a number of different ways with the \texttt{show proof}\index{\texttt{show
proof} command} command.  The
\texttt{/lemmon} qualifier displays it in a format that is easier to read when the
proofs are short, and you saw examples of its use in Chapter~\ref{using}.  For
longer proofs, it is useful to see the tree structure of the proof.  A tree
structure is displayed when the \texttt{/lemmon} qualifier is omitted.  You will
probably find this display more convenient as you get used to it. The tree
display of the proof in our example looks like
this:\label{treeproof}\index{tree-style proof}\index{proof!tree-style}
\begin{verbatim}
1     wp=ws    $f wff s
2        wp=wr    $f wff r
3        wq=wp    $f wff p
4     wq=w2    $a wff ( r -> p )
5  wnew=w2  $a wff ( s -> ( r -> p ) )
\end{verbatim}
The number to the left of each line is the step number.  Following it is a
{\bf hypothesis association}\index{hypothesis association}, consisting of two
labels\index{label} separated by \texttt{=}.  To the left of the \texttt{=} (except
in the last step) is the label of a hypothesis of an assertion referenced
later in the proof; here, steps 1 and 4 are the hypothesis associations for
the assertion \texttt{w2} that is referenced in step 5.  A hypothesis association
is indented one level more than the assertion that uses it, so it is easy to
find the corresponding assertion by moving directly down until the indentation
level decreases to one less than where you started from.  To the right of each
\texttt{=} is the proof step label for that proof step.  The statement keyword of
the proof step label is listed next, followed by the content of the top of the
stack (the most recent stack entry) as it exists after that proof step is
processed.  With a little practice, you should have no trouble reading proofs
displayed in this format.

Metamath proofs include the syntax construction of a formula.
In standard mathematics, this kind of
construction is not considered a proper part of the proof at all, and it
certainly becomes rather boring after a while.
Therefore,
by default the \texttt{show proof}\index{\texttt{show proof}
command} command does not show the syntax construction.
Historically \texttt{show proof} command
\textit{did} show the syntax construction, and you needed to add the
\texttt{/essential} option to hide, them, but today
\texttt{/essential} is the default and you need to use
\texttt{/all} to see the syntax constructions.

When verifying a proof, Metamath will check that no mandatory
\texttt{\$d}\index{\texttt{\$d} statement}\index{mandatory \texttt{\$d}
statement} statement of an assertion referenced in a proof is violated
when substitutions\index{substitution!variable}\index{variable
substitution} are made to the variables in the assertion.  For details
see Section~\ref{spec4} or \ref{dollard}.

\subsection{The Concept of Unification} \label{unify}

During the course of verifying a proof, when Metamath\index{Metamath}
encounters an assertion label\index{assertion label}, it associates the
mandatory hypotheses\index{mandatory hypothesis} of the assertion with the top
entries of the RPN stack\index{stack}\index{RPN stack}.  Metamath then
determines what substitutions\index{substitution!variable}\index{variable
substitution} it must make to the variables in the assertion's mandatory
hypotheses in order for these hypotheses to become identical to their
corresponding stack entries.  This process is called {\bf
unification}\index{unification}.  (We also informally use the term
``unification'' to refer to a set of substitutions that results from the
process, as in ``two unifications are possible.'')  After the substitutions
are made, the hypotheses are said to be {\bf unified}.

If no such substitutions are possible, Metamath will consider the proof
incorrect and notify you with an error message.
% (deleted 3/10/07, per suggestion of Mel O'Cat:)
% The syntax of the
% Metamath language ensures that if a set of substitutions exists, it
% will be unique.

The general algorithm for unification described in the literature is
somewhat complex.
However, in the case of Metamath it is intentionally trivial.
Mandatory hypotheses must be
pushed on the proof stack in the order in which they appear.
In addition, each variable must have its type specified
with a \texttt{\$f} hypothesis before it is used
and that each \texttt{\$f} hypothesis
have the restricted syntax of a typecode (a constant) followed by a variable.
The typecode in the \texttt{\$f} hypothesis must match the first symbol of
the corresponding RPN stack entry (which will also be a constant), so
the only possible match for the variable in the \texttt{\$f} hypothesis is
the sequence of symbols in the stack entry after the initial constant.

In the Proof Assistant\index{Proof Assistant}, a more general unification
algorithm is used.  While a proof is being developed, sometimes not enough
information is available to determine a unique unification.  In this case
Metamath will ask you to pick the correct one.\index{ambiguous
unification}\index{unification!ambiguous}

\section{Extensions to the Metamath Language}\index{extended
language}

\subsection{Comments in the Metamath Language}\label{comments}
\index{markup notation}
\index{comments!markup notation}

The commenting feature allows you to annotate the contents of
a database.  Just as with most
computer languages, comments are ignored for the purpose of interpreting the
contents of the database. Comments effectively act as
additional white space\index{white
space} between tokens
when a database is parsed.

A comment may be placed at the beginning, end, or
between any two tokens\index{token} in a source file.

Comments have the following syntax:
\begin{center}
 \texttt{\$(} {\em text} \texttt{\$)}
\end{center}
Here,\index{\texttt{\$(} and \texttt{\$)} auxiliary
keywords}\index{comment} {\em text} is a string, possibly empty, of any
characters in Metamath's character set (p.~\pageref{spec1chars}), except
that the character strings \texttt{\$(} and \texttt{\$)} may not appear
in {\em text}.  Thus nested comments are not
permitted:\footnote{Computer languages have differing standards for
nested comments, and rather than picking one it was felt simplest not to
allow them at all, at least in the current version of
Metamath\index{Metamath!limitations of version 0.07.30}.} Metamath will
complain if you give it
\begin{center}
 \texttt{\$( This is a \$( nested \$) comment.\ \$)}
\end{center}
To compensate for this non-nesting behavior, I often change all \texttt{\$}'s
to \texttt{@}'s in sections of Metamath code I wish to comment out.

The Metamath program supports a number of markup mechanisms and conventions
to generate good-looking results in \LaTeX\ and {\sc html},
as discussed below.
These markup features have to do only with how the comments are typeset,
and have no effect on how Metamath verifies the proofs in the database.
The improper
use of them may result in incorrectly typeset output, but no Metamath
error messages will result during the \texttt{read} and \texttt{verify
proof} commands.  (However, the \texttt{write
theorem\texttt{\char`\_}list} command
will check for markup errors as a side-effect of its
{\sc html} generation.)
Section~\ref{texout} has instructions for creating \LaTeX\ output, and
section~\ref{htmlout} has instructions for creating
{\sc html}\index{HTML} output.

\subsubsection{Headings}\label{commentheadings}

If the \texttt{\$(} is immediately followed by a new line
starting with a heading marker, it is a header.
This can start with:

\begin{itemize}
 \item[] \texttt{\#\#\#\#} - major part header
 \item[] \texttt{\#*\#*} - section header
 \item[] \texttt{=-=-} - subsection header
 \item[] \texttt{-.-.} - subsubsection header
\end{itemize}

The line following the marker line
will be used for the table of contents entry, after trimming spaces.
The next line should be another (closing) matching marker line.
Any text after that
but before the closing \texttt{\$}, such as an extended description of the
section, will be included on the \texttt{mmtheoremsNNN.html} page.

For more information, run
\texttt{help write theorem\char`\_list}.

\subsubsection{Math mode}
\label{mathcomments}
\index{\texttt{`} inside comments}
\index{\texttt{\char`\~} inside comments}
\index{math mode}

Inside of comments, a string of tokens\index{token} enclosed in
grave accents\index{grave accent (\texttt{`})} (\texttt{`}) will be converted
to standard mathematical symbols during
{\sc HTML}\index{HTML} or \LaTeX\ output
typesetting,\index{latex@{\LaTeX}} according to the information in the
special \texttt{\$t}\index{\texttt{\$t} comment}\index{typesetting
comment} comment in the database
(see section~\ref{tcomment} for information about the typesetting
comment, and Appendix~\ref{ASCII} to see examples of its results).

The first grave accent\index{grave accent (\texttt{`})} \texttt{`}
causes the output processor to enter {\bf math mode}\index{math mode}
and the second one exits it.
In this
mode, the characters following the \texttt{`} are interpreted as a
sequence of math symbol tokens separated by white space\index{white
space}.  The tokens are looked up in the \texttt{\$t}
comment\index{\texttt{\$t} comment}\index{typesetting comment} and if
found, they will be replaced by the standard mathematical symbols that
they correspond to before being placed in the typeset output file.  If
not found, the symbol will be output as is and a warning will be issued.
The tokens do not have to be active in the database, although a warning
will be issued if they are not declared with \texttt{\$c} or
\texttt{\$v} statements.

Two consecutive
grave accents \texttt{``} are treated as a single actual grave accent
(both inside and outside of math mode) and will not cause the output
processor to enter or exit math mode.

Here is an example of its use\index{Pierce's axiom}:
\begin{center}
\texttt{\$( Pierce's axiom, ` ( ( ph -> ps ) -> ph ) -> ph ` ,\\
         is not very intuitive. \$)}
\end{center}
becomes
\begin{center}
   \texttt{\$(} Pierce's axiom, $((\varphi \rightarrow \psi)\rightarrow
\varphi)\rightarrow \varphi$, is not very intuitive. \texttt{\$)}
\end{center}

Note that the math symbol tokens\index{token} must be surrounded by white
space\index{white space}.
%, since there is no context that allows ambiguity to be
%resolved, as is the case with math symbol sequences in some of the Metamath
%statements.
White space should also surround the \texttt{`}
delimiters.

The math mode feature also gives you a quick and easy way to generate
text containing mathematical symbols, independently of the intended
purpose of Metamath.\index{Metamath!using as a math editor} To do this,
simply create your text with grave accents surrounding your formulas,
after making sure that your math symbols are mapped to \LaTeX\ symbols
as described in Appendix~\ref{ASCII}.  It is easier if you start with a
database with predefined symbols such as \texttt{set.mm}.  Use your
grave-quoted math string to replace an existing comment, then typeset
the statement corresponding to that comment following the instructions
from the \texttt{help tex} command in the Metamath program.  You will
then probably want to edit the resulting file with a text editor to fine
tune it to your exact needs.

\subsubsection{Label Mode}\index{label mode}

Outside of math mode, a tilde\index{tilde (\texttt{\char`\~})} \verb/~/
indicates to Metamath's\index{Metamath} output processor that the
token\index{token} that follows (i.e.\ the characters up to the next
white space\index{white space}) represents a statement label or URL.
This formatting mode is called {\bf label mode}\index{label mode}.
If a literal tilde
is desired (outside of math mode) instead of label mode,
use two tildes in a row to represent it.

When generating a \LaTeX\ output file,
the following token will be formatted in \texttt{typewriter}
font, and the tilde removed, to make it stand out from the rest of the text.
This formatting will be applied to all characters after the
tilde up to the first white space\index{white space}.
Whether
or not the token is an actual statement label is not checked, and the
token does not have to have the correct syntax for a label; no error
messages will be produced.  The only effect of the label mode on the
output is that typewriter font will be used for the tokens that are
placed in the \LaTeX\ output file.

When generating {\sc html},
the tokens after the tilde {\em must} be a URL (either http: or https:)
or a valid label.
Error messages will be issued during that output if they aren't.
A hyperlink will be generated to that URL or label.

\subsubsection{Link to bibliographical reference}\index{citation}%
\index{link to bibliographical reference}

Bibliographical references are handled specially when generating
{\sc html} if formatted specially.
Text in the form \texttt{[}{\em author}\texttt{]}
is considered a link to a bibliographical reference.
See \texttt{help html} and \texttt{help write
bibliography} in the Metamath program for more
information.
% \index{\texttt{\char`\[}\ldots\texttt{]} inside comments}
See also Sections~\ref{tcomment} and \ref{wrbib}.

The \texttt{[}{\em author}\texttt{]} notation will also create an entry in
the bibliography cross-reference file generated by \texttt{write
bibliography} (Section~\ref{wrbib}) for {\sc HTML}.
For this to work properly, the
surrounding comment must be formatted as follows:
\begin{quote}
    {\em keyword} {\em label} {\em noise-word}
     \texttt{[}{\em author}\texttt{] p.} {\em number}
\end{quote}
for example
\begin{verbatim}
     Theorem 5.2 of [Monk] p. 223
\end{verbatim}
The {\em keyword} is not case sensitive and must be one of the following:
\begin{verbatim}
     theorem lemma definition compare proposition corollary
     axiom rule remark exercise problem notation example
     property figure postulate equation scheme chapter
\end{verbatim}
The optional {\em label} may consist of more than one
(non-{\em keyword} and non-{\em noise-word}) word.
The optional {\em noise-word} is one of:
\begin{verbatim}
     of in from on
\end{verbatim}
and is  ignored when the cross-reference file is created.  The
\texttt{write
biblio\-graphy} command will perform error checking to verify the
above format.\index{error checking}

\subsubsection{Parentheticals}\label{parentheticals}

The end of a comment may include one or more parenthicals, that is,
statements enclosed in parentheses.
The Metamath program looks for certain parentheticals and can issue
warnings based on them.
They are:

\begin{itemize}
 \item[] \texttt{(Contributed by }
   \textit{NAME}\texttt{,} \textit{DATE}\texttt{.)} -
   document the original contributor's name and the date it was created.
 \item[] \texttt{(Revised by }
   \textit{NAME}\texttt{,} \textit{DATE}\texttt{.)} -
   document the contributor's name and creation date
   that resulted in significant revision
   (not just an automated minimization or shortening).
 \item[] \texttt{(Proof shortened by }
   \textit{NAME}\texttt{,} \textit{DATE}\texttt{.)} -
   document the contributor's name and date that developed a significant
   shortening of the proof (not just an automated minimization).
 \item[] \texttt{(Proof modification is discouraged.)} -
   Note that this proof should normally not be modified.
 \item[] \texttt{(New usage is discouraged.)} -
   Note that this assertion should normally not be used.
\end{itemize}

The \textit{DATE} must be in form YYYY-MMM-DD, where MMM is the
English abbreviation of that month.

\subsubsection{Other markup}\label{othermarkup}
\index{markup notation}

There are other markup notations for generating good-looking results
beyond math mode and label mode:

\begin{itemize}
 \item[]
         \texttt{\char`\_} (underscore)\index{\texttt{\char`\_} inside comments} -
             Italicize text starting from
              {\em space}\texttt{\char`\_}{\em non-space} (i.e.\ \texttt{\char`\_}
              with a space before it and a non-space character after it) until
             the next
             {\em non-space}\texttt{\char`\_}{\em space}.  Normal
             punctuation (e.g.\ a trailing
             comma or period) is ignored when determining {\em space}.
 \item[]
         \texttt{\char`\_} (underscore) - {\em
         non-space}\texttt{\char`\_}{\em non-space-string}, where
          {\em non-space-string} is a string of non-space characters,
         will make {\em non-space-string} become a subscript.
 \item[]
         \texttt{<HTML>}...\texttt{</HTML>} - do not convert
         ``\texttt{<}'' and ``\texttt{>}''
         in the enclosed text when generating {\sc HTML},
         otherwise process markup normally. This allows direct insertion
         of {\sc html} commands.
 \item[]
       ``\texttt{\&}ref\texttt{;}'' - insert an {\sc HTML}
         character reference.
         This is how to insert arbitrary Unicode characters
         (such as accented characters).  Currently only directly supported
         when generating {\sc HTML}.
\end{itemize}

It is recommended that spaces surround any \texttt{\char`\~} and
\texttt{`} tokens in the comment and that a space follow the {\em label}
after a \texttt{\char`\~} token.  This will make global substitutions
to change labels and symbol names much easier and also eliminate any
future chance of ambiguity.  Spaces around these tokens are automatically
removed in the final output to conform with normal rules of punctuation;
for example, a space between a trailing \texttt{`} and a left parenthesis
will be removed.

A good way to become familiar with the markup notation is to look at
the extensive examples in the \texttt{set.mm} database.

\subsection{The Typesetting Comment (\texttt{\$t})}\label{tcomment}

The typesetting comment \texttt{\$t} in the input database file
provides the information necessary to produce good-looking results.
It provides \LaTeX\ and {\sc html}
definitions for math symbols,
as well supporting as some
customization of the generated web page.
If you add a new token to a database, you should also
update the \texttt{\$t} comment information if you want to eventually
create output in \LaTeX\ or {\sc HTML}.
See the
\texttt{set.mm}\index{set theory database (\texttt{set.mm})} database
file for an extensive example of a \texttt{\$t} comment illustrating
many of the features described below.

Programs that do not need to generate good-looking presentation results,
such as programs that only verify Metamath databases,
can completely ignore typesetting comments
and just treat them as normal comments.
Even the Metamath program only consults the
\texttt{\$t} comment information when it needs to generate typeset output
in \LaTeX\ or {\sc HTML}
(e.g., when you open a \LaTeX\ output file with the \texttt{open tex} command).

We will first discuss the syntax of typesetting comments, and then
briefly discuss how this can be used within the Metamath program.

\subsubsection{Typesetting Comment Syntax Overview}

The typesetting comment is identified by the token
\texttt{\$t}\index{\texttt{\$t} comment}\index{typesetting comment} in
the comment, and the typesetting comment ends at the matching
\texttt{\$)}:
\[
  \mbox{\tt \$(\ }
  \mbox{\tt \$t\ }
  \underbrace{
    \mbox{\tt \ \ \ \ \ \ \ \ \ \ \ }
    \cdots
    \mbox{\tt \ \ \ \ \ \ \ \ \ \ \ }
  }_{\mbox{Typesetting definitions go here}}
  \mbox{\tt \ \$)}
\]

There must be one or more white space characters, and only white space
characters, between the \texttt{\$(} that starts the comment
and the \texttt{\$t} symbol,
and the \texttt{\$t} must be followed by one
or more white space characters
(see section \ref{whitespace} for the definition of white space characters).
The typesetting comment continues until the comment end token \texttt{\$)}
(which must be preceded by one or more white space characters).

In version 0.07.30\index{Metamath!limitations of version 0.07.30} of the
Metamath program, there may be only one \texttt{\$t} comment in a
database.  This restriction may be lifted in the future to allow
many \texttt{\$t} comments in a database.

Between the \texttt{\$t} symbol (and its following white space) and the
comment end token \texttt{\$)} (and its preceding white space)
is a sequence of one or more typesetting definitions, where
each definition has the form
\textit{definition-type arg arg ... ;}.
Each of the zero or more \textit{arg} values
can be either a typesetting data or a keyword
(what keywords are allowed, and where, depends on the specific
\textit{definition-type}).
The \textit{definition-type}, and each argument \textit{arg},
are separated by one or more white space characters.
Every definition ends in an unquoted semicolon;
white space is not required before the terminating semicolon of a definition.
Each definition should start on a new line.\footnote{This
restriction of the current version of Metamath
(0.07.30)\index{Metamath!limitations of version 0.07.30} may be removed
in a future version, but you should do it anyway for readability.}

For example, this typesetting definition:
\begin{center}
 \verb$latexdef "C_" as "\subseteq";$
\end{center}
defines the token \verb$C_$ as the \LaTeX\ symbol $\subseteq$ (which means
``subset'').

Typesetting data is a sequence of one or more quoted strings
(if there is more than one, they are connected by \texttt{\char`\+}).
Often a single quoted string is used to provide data for a definition, using
either double (\texttt{\char`\"}) or single (\texttt{'}) quotation marks.
However,
{\em a quoted string (enclosed in quotation marks) may not include
line breaks.}
A quoted string
may include a quotation mark that matches the enclosing quotes by repeating
the quotation mark twice.  Here are some examples:

\begin{tabu}   { l l }
\textbf{Example} & \textbf{Meaning} \\
\texttt{\char`\"a\char`\"\char`\"b\char`\"} & \texttt{a\char`\"b} \\
\texttt{'c''d'} & \texttt{c'd} \\
\texttt{\char`\"e''f\char`\"} & \texttt{e''f} \\
\texttt{'g\char`\"\char`\"h'} & \texttt{g\char`\"\char`\"h} \\
\end{tabu}

Finally, a long quoted string
may be broken up into multiple quoted strings (considered, as a whole,
a single quoted string) and joined with \texttt{\char`\+}.
You can even use multiple lines as long as a
'+' is at the end of every line except the last one.
The \texttt{\char`\+} should be preceded and followed by at least one
white space character.
Thus, for example,
\begin{center}
 \texttt{\char`\"ab\char`\"\ \char`\+\ \char`\"cd\char`\"
    \ \char`\+\ \\ 'ef'}
\end{center}
is the same as
\begin{center}
 \texttt{\char`\"abcdef\char`\"}
\end{center}

{\sc c}-style comments \texttt{/*}\ldots\texttt{*/} are also supported.

In practice, whenever you add a new math token you will often want to add
typesetting definitions using
\texttt{latexdef}, \texttt{htmldef}, and
\texttt{althtmldef}, as described below.
That way, they will all be up to date.
Of course, whether or not you want to use all three definitions will
depend on how the database is intended to be used.

Below we discuss the different possible \textit{definition-kind} options.
We will show data surrounded by double quotes (in practice they can also use
single quotes and/or be a sequence joined by \texttt{+}s).
We will use specific names for the \textit{data} to make clear what
the data is used for, such as
{\em math-token} (for a Metamath math token,
{\em latex-string} (for string to be placed in a \LaTeX\ stream),
{\em {\sc html}-code} (for {\sc html} code),
and {\em filename} (for a filename).

\subsubsection{Typesetting Comment - \LaTeX}

The syntax for a \LaTeX\ definition is:
\begin{center}
 \texttt{latexdef "}{\em math-token}\texttt{" as "}{\em latex-string}\texttt{";}
\end{center}
\index{latex definitions@\LaTeX\ definitions}%
\index{\texttt{latexdef} statement}

The {\em token-string} and {\em latex-string} are the data
(character strings) for
the token and the \LaTeX\ definition of the token, respectively,

These \LaTeX\ definitions are used by the Metamath program
when it is asked to product \LaTeX output using
the \texttt{write tex} command.

\subsubsection{Typesetting Comment - {\sc html}}

The key kinds of {\sc HTML} definitions have the following syntax:

\vskip 1ex
    \texttt{htmldef "}{\em math-token}\texttt{" as "}{\em
    {\sc html}-code}\texttt{";}\index{\texttt{htmldef} statement}
                    \ \ \ \ \ \ldots

    \texttt{althtmldef "}{\em math-token}\texttt{" as "}{\em
{\sc html}-code}\texttt{";}\index{\texttt{althtmldef} statement}

                    \ \ \ \ \ \ldots

Note that in {\sc HTML} there are two possible definitions for math tokens.
This feature is useful when
an alternate representation of symbols is desired, for example one that
uses Unicode entities and another uses {\sc gif} images.

There are many other typesetting definitions that can control {\sc HTML}.
These include:

\vskip 1ex

    \texttt{htmldef "}{\em math-token}\texttt{" as "}{\em {\sc
    html}-code}\texttt{";}

    \texttt{htmltitle "}{\em {\sc html}-code}\texttt{";}%
\index{\texttt{htmltitle} statement}

    \texttt{htmlhome "}{\em {\sc html}-code}\texttt{";}%
\index{\texttt{htmlhome} statement}

    \texttt{htmlvarcolor "}{\em {\sc html}-code}\texttt{";}%
\index{\texttt{htmlvarcolor} statement}

    \texttt{htmlbibliography "}{\em filename}\texttt{";}%
\index{\texttt{htmlbibliography} statement}

\vskip 1ex

\noindent The \texttt{htmltitle} is the {\sc html} code for a common
title, such as ``Metamath Proof Explorer.''  The \texttt{htmlhome} is
code for a link back to the home page.  The \texttt{htmlvarcolor} is
code for a color key that appears at the bottom of each proof.  The file
specified by {\em filename} is an {\sc html} file that is assumed to
have a \texttt{<A NAME=}\ldots\texttt{>} tag for each bibiographic
reference in the database comments.  For example, if
\texttt{[Monk]}\index{\texttt{\char`\[}\ldots\texttt{]} inside comments}
occurs in the comment for a theorem, then \texttt{<A NAME='Monk'>} must
be present in the file; if not, a warning message is given.

Associated with
\texttt{althtmldef}
are the statements
\vskip 1ex

    \texttt{htmldir "}{\em
      directoryname}\texttt{";}\index{\texttt{htmldir} statement}

    \texttt{althtmldir "}{\em
     directoryname}\texttt{";}\index{\texttt{althtmldir} statement}

\vskip 1ex
\noindent giving the directories of the {\sc gif} and Unicode versions
respectively; their purpose is to provide cross-linking between the
two versions in the generated web pages.

When two different types of pages need to be produced from a single
database, such as the Hilbert Space Explorer that extends the Metamath
Proof Explorer, ``extended'' variables may be declared in the
\texttt{\$t} comment:
\vskip 1ex

    \texttt{exthtmltitle "}{\em {\sc html}-code}\texttt{";}%
\index{\texttt{exthtmltitle} statement}

    \texttt{exthtmlhome "}{\em {\sc html}-code}\texttt{";}%
\index{\texttt{exthtmlhome} statement}

    \texttt{exthtmlbibliography "}{\em filename}\texttt{";}%
\index{\texttt{exthtmlbibliography} statement}

\vskip 1ex
\noindent When these are declared, you also must declare
\vskip 1ex

    \texttt{exthtmllabel "}{\em label}\texttt{";}%
\index{\texttt{exthtmllabel} statement}

\vskip 1ex \noindent that identifies the database statement where the
``extended'' section of the database starts (in our example, where the
Hilbert Space Explorer starts).  During the generation of web pages for
that starting statement and the statements after it, the {\sc html} code
assigned to \texttt{exthtmltitle} and \texttt{exthtmlhome} is used
instead of that assigned to \texttt{htmltitle} and \texttt{htmlhome},
respectively.

\begin{sloppy}
\subsection{Additional Information Com\-ment (\texttt{\$j})} \label{jcomment}
\end{sloppy}

The additional information comment, aka the
\texttt{\$j}\index{\texttt{\$j} comment}\index{additional information comment}
comment,
provides a way to add additional structured information that can
be optionally parsed by systems.

The additional information comment is parsed the same way as the
typesetting comment (\texttt{\$t}) (see section \ref{tcomment}).
That is,
the additional information comment begins with the token
\texttt{\$j} within a comment,
and continues until the comment close \texttt{\$)}.
Within an additional information comment is a sequence of one or more
commands of the form \texttt{command arg arg ... ;}
where each of the zero or more \texttt{arg} values
can be either a quoted string or a keyword.
Note that every command ends in an unquoted semicolon.
If a verifier is parsing an additional information comment, but
doesn't recognize a particular command, it must skip the command
by finding the end of the command (an unquoted semicolon).

A database may have 0 or more additional information comments.
Note, however, that a verifier may ignore these comments entirely or only
process certain commands in an additional information comment.
The \texttt{mmj2} verifier supports many commands in additional information
comments.
We encourage systems that process additional information comments
to coordinate so that they will use the same command for the same effect.

Examples of additional information comments with various commands
(from the \texttt{set.mm} database) are:

\begin{itemize}
   \item Define the syntax and logical typecodes,
     and declare that our grammar is
     unambiguous (verifiable using the KLR parser, with compositing depth 5).
\begin{verbatim}
  $( $j
    syntax 'wff';
    syntax '|-' as 'wff';
    unambiguous 'klr 5';
  $)
\end{verbatim}

   \item Register $\lnot$ and $\rightarrow$ as primitive expressions
           (lacking definitions).
\begin{verbatim}
  $( $j primitive 'wn' 'wi'; $)
\end{verbatim}

   \item There is a special justification for \texttt{df-bi}.
\begin{verbatim}
  $( $j justification 'bijust' for 'df-bi'; $)
\end{verbatim}

   \item Register $\leftrightarrow$ as an equality for its type (wff).
\begin{verbatim}
  $( $j
    equality 'wb' from 'biid' 'bicomi' 'bitri';
    definition 'dfbi1' for 'wb';
  $)
\end{verbatim}

   \item Theorem \texttt{notbii} is the congruence law for negation.
\begin{verbatim}
  $( $j congruence 'notbii'; $)
\end{verbatim}

   \item Add \texttt{setvar} as a typecode.
\begin{verbatim}
  $( $j syntax 'setvar'; $)
\end{verbatim}

   \item Register $=$ as an equality for its type (\texttt{class}).
\begin{verbatim}
  $( $j equality 'wceq' from 'eqid' 'eqcomi' 'eqtri'; $)
\end{verbatim}

\end{itemize}


\subsection{Including Other Files in a Metamath Source File} \label{include}
\index{\texttt{\$[} and \texttt{\$]} auxiliary keywords}

The keywords \texttt{\$[} and \texttt{\$]} specify a file to be
included\index{included file}\index{file inclusion} at that point in a
Metamath\index{Metamath} source file\index{source file}.  The syntax for
including a file is as follows:
\begin{center}
\texttt{\$[} {\em file-name} \texttt{\$]}
\end{center}

The {\em file-name} should be a single token\index{token} with the same syntax
as a math symbol (i.e., all 93 non-whitespace
printable characters other than \texttt{\$} are
allowed, subject to the file-naming limitations of your operating system).
Comments may appear between the \texttt{\$[} and \texttt{\$]} keywords.  Included
files may include other files, which may in turn include other files, and so
on.

For example, suppose you want to use the set theory database as the starting
point for your own theory.  The first line in your file could be
\begin{center}
\texttt{\$[ set.mm \$]}
\end{center} All of the information (axioms, theorems,
etc.) in \texttt{set.mm} and any files that {\em it} includes will become
available for you to reference in your file. This can help make your work more
modular. A drawback to including files is that if you change the name of a
symbol or the label of a statement, you must also remember to update any
references in any file that includes it.


The naming conventions for included files are the same as those of your
operating system.\footnote{On the Macintosh, prior to Mac OS X,
 a colon is used to separate disk
and folder names from your file name.  For example, {\em volume}\texttt{:}{\em
file-name} refers to the root directory, {\em volume}\texttt{:}{\em
folder-name}\texttt{:}{\em file-name} refers to a folder in root, and {\em
volume}\texttt{:}{\em folder-name}\texttt{:}\ldots\texttt{:}{\em file-name} refers to a
deeper folder.  A simple {\em file-name} refers to a file in the folder from
which you launch the Metamath application.  Under Mac OS X and later,
the Metamath program is run under the Terminal application, which
conforms to Unix naming conventions.}\index{Macintosh file
names}\index{file names!Macintosh}\label{includef} For compatibility among
operating systems, you should keep the file names as simple as possible.  A
good convention to use is {\em file}\texttt{.mm} where {\em file} is eight
characters or less, in lower case.

There is no limit to the nesting depth of included files.  One thing that you
should be aware of is that if two included files themselves include a common
third file, only the {\em first} reference to this common file will be read
in.  This allows you to include two or more files that build on a common
starting file without having to worry about label and symbol conflicts that
would occur if the common file were read in more than once.  (In fact, if a
file includes itself, the self-reference will be ignored, although of course
it would not make any sense to do that.)  This feature also means, however,
that if you try to include a common file in several inner blocks, the result
might not be what you expect, since only the first reference will be replaced
with the included file (unlike the include statement in most other computer
languages).  Thus you would normally include common files only in the
outermost block\index{outermost block}.

\subsection{Compressed Proof Format}\label{compressed1}\index{compressed
proof}\index{proof!compressed}

The proof notation presented in Section~\ref{proof} is called a
{\bf normal proof}\index{normal proof}\index{proof!normal} and in principle is
sufficient to express any proof.  However, proofs often contain steps and
subproofs that are identical.  This is particularly true in typical
Metamath\index{Metamath} applications, because Metamath requires that the math
symbol sequence (usually containing a formula) at each step be separately
constructed, that is, built up piece by piece. As a result, a lot of
repetition often results.  The {\bf compressed proof} format allows Metamath
to take advantage of this redundancy to shorten proofs.

The specification for the compressed proof format is given in
Appen\-dix~\ref{compressed}.

Normally you need not concern yourself with the details of the compressed
proof format, since the Metamath program will allow you to convert from
the normal format to the compressed format with ease, and will also
automatically convert from the compressed format when proofs are displayed.
The overall structure of the compressed format is as follows:
\begin{center}
  \texttt{\$= ( } {\em label-list} \texttt{) } {\em compressed-proof\ }\ \texttt{\$.}
\end{center}
\index{\texttt{\$=} keyword}
The first \texttt{(} serves as a flag to Metamath that a compressed proof
follows.  The {\em label-list} includes all statements referred to by the
proof except the mandatory hypotheses\index{mandatory hypothesis}.  The {\em
compressed-proof} is a compact encoding of the proof, using upper-case
letters, and can be thought of as a large integer in base 26.  White
space\index{white space} inside a {\em compressed-proof} is
optional and is ignored.

It is important to note that the order of the mandatory hypotheses of
the statement being proved must not be changed if the compressed proof
format is used, otherwise the proof will become incorrect.  The reason
for this is that the mandatory hypotheses are not mentioned explicitly
in the compressed proof in order to make the compression more efficient.
If you wish to change the order of mandatory hypotheses, you must first
convert the proof back to normal format using the \texttt{save proof
{\em statement} /normal}\index{\texttt{save proof} command} command.
Later, you can go back to compressed format with \texttt{save proof {\em
statement} /compressed}.

During error checking with the \texttt{verify proof} command, an error
found in a compressed proof may point to a character in {\em
compressed-proof}, which may not be very meaningful to you.  In this
case, try to \texttt{save proof /normal} first, then do the
\texttt{verify proof} again.  In general, it is best to make sure a
proof is correct before saving it in compressed format, because severe
errors are less likely to be recoverable than in normal format.

\subsection{Specifying Unknown Proofs or Subproofs}\label{unknown}

In a proof under development, any step or subproof that is not yet known
may be represented with a single \texttt{?}.  For the purposes of
parsing the proof, the \texttt{?}\ \index{\texttt{]}@\texttt{?}\ inside
proofs} will push a single entry onto the RPN stack just as if it were a
hypothesis.  While developing a proof with the Proof
Assistant\index{Proof Assistant}, a partially developed proof may be
saved with the \texttt{save new{\char`\_}proof}\index{\texttt{save
new{\char`\_}proof} command} command, and \texttt{?}'s will be placed at
the appropriate places.

All \texttt{\$p}\index{\texttt{\$p} statement} statements must have
proofs, even if they are entirely unknown.  Before creating a proof with
the Proof Assistant, you should specify a completely unknown proof as
follows:
\begin{center}
  {\em label} \texttt{\$p} {\em statement} \texttt{\$= ?\ \$.}
\end{center}
\index{\texttt{\$=} keyword}
\index{\texttt{]}@\texttt{?}\ inside proofs}

The \texttt{verify proof}\index{\texttt{verify proof} command} command
will check the known portions of a partial proof for errors, but will
warn you that the statement has not been proved.

Note that partially developed proofs may be saved in compressed format
if desired.  In this case, you will see one or more \texttt{?}'s in the
{\em compressed-proof} part.\index{compressed
proof}\index{proof!compressed}

\section{Axioms vs.\ Definitions}\label{definitions}

The \textit{basic}
Metamath\index{Metamath} language and program
make no distinction\index{axiom vs.\
definition} between axioms\index{axiom} and
definitions.\index{definition} The \texttt{\$a}\index{\texttt{\$a}
statement} statement is used for both.  At first, this may seem
puzzling.  In the minds of many mathematicians, the distinction is
clear, even obvious, and hardly worth discussing.  A definition is
considered to be merely an abbreviation that can be replaced by the
expression for which it stands; although unless one actually does this,
to be precise then one should say that a theorem\index{theorem} is a
consequence of the axioms {\em and} the definitions that are used in the
formulation of the theorem \cite[p.~20]{Behnke}.\index{Behnke, H.}

\subsection{What is a Definition?}

What is a definition?  In its simplest form, a definition introduces a new
symbol and provides an unambiguous rule to transform an expression containing
the new symbol to one without it.  The concept of a ``proper
definition''\index{proper definition}\index{definition!proper} (as opposed to
a creative definition)\index{creative definition}\index{definition!creative}
that is usually agreed upon is (1) the definition should not strengthen the
language and (2) any symbols introduced by the definition should be eliminable
from the language \cite{Nemesszeghy}\index{Nemesszeghy, E. Z.}.  In other
words, they are mere typographical conveniences that do not belong to the
system and are theoretically superfluous.  This may seem obvious, but in fact
the nature of definitions can be subtle, sometimes requiring difficult
metatheorems to establish that they are not creative.

A more conservative stance was taken by logician S.
Le\'{s}niewski.\index{Le\'{s}niewski, S.}
\begin{quote}
Le\'{s}niewski
regards definitions as theses of the system.  In this respect they do
not differ either from the axioms or from theorems, i.e.\ from the
theses added to the system on the basis of the rule of substitution or
the rule of detachment [modus ponens].  Once definitions have been
accepted as theses of the system, it becomes necessary to consider them
as true propositions in the same sense in which axioms are true
\cite{Lejewski}.
\end{quote}\index{Lejewski, Czeslaw}

Let us look at some simple examples of definitions in propositional
calculus.  Consider the definition of logical {\sc or}
(disjunction):\index{disjunction ($\vee$)} ``$P\vee Q$ denotes $\neg P
\rightarrow Q$ (not $P$ implies $Q$).''  It is very easy to recognize a
statement making use of this definition, because it introduces the new
symbol $\vee$ that did not previously exist in the language.  It is easy
to see that no new theorems of the original language will result from
this definition.

Next, consider a definition that eliminates parentheses:  ``$P
\rightarrow Q\rightarrow R$ denotes $P\rightarrow (Q \rightarrow R)$.''
This is more subtle, because no new symbols are introduced.  The reason
this definition is considered proper is that no new symbol sequences
that are valid wffs (well-formed formulas)\index{well-formed formula
(wff)} in the original language will result from the definition, since
``$P \rightarrow Q\rightarrow R$'' is not a wff in the original
language.  Here, we implicitly make use of the fact that there is a
decision procedure that allows us to determine whether or not a symbol
sequence is a wff, and this fact allows us to use symbol sequences that
are not wffs to represent other things (such as wffs) by means of the
definition.  However, to justify the definition as not being creative we
need to prove that ``$P \rightarrow Q\rightarrow R$'' is in fact not a
wff in the original language, and this is more difficult than in the
case where we simply introduce a new symbol.

%Now let's take this reasoning to an extreme.  Propositional calculus is a
%decidable theory,\footnote{This means that a mechanical algorithm exists to
%determine whether or not a wff is a theorem.} so in principle we could make use
%of symbol sequences that are not theorems to represent other things (say, to
%encode actual theorems in a more compact way).  For example, let us extend the
%language by defining a wff ``$P$'' in the extended language as the theorem
%``$P\rightarrow P$''\footnote{This is one of the first theorems proved in the
%Metamath database \texttt{set.mm}.}\index{set
%theory database (\texttt{set.mm})} in the original language whenever ``$P$'' is
%not a theorem in the original language.  In the extended language, any wff
%``$Q$'' thus represents a theorem; to find out what theorem (in the original
%language) ``$Q$'' represents, we determine whether ``$Q$'' is a theorem in the
%original language (before the definition was introduced).  If so, we're done; if
%not, we replace ``$Q$'' by ``$Q\rightarrow Q$'' to eliminate the definition.
%This definition is therefore eliminable, and it does not ``strengthen'' the
%language because any wff that is not a theorem is not in the set of statements
%provable in the original language and thus is available for use by definitions.
%
%Of course, a definition such as this would render practically useless the
%communication of theorems of propositional calculus; but
%this is just a human shortcoming, since we can't always easily discern what is
%and is not a theorem by inspection.  In fact, the extended theory with this
%definition has no more and no less information than the original theory; it just
%expresses certain theorems of the form ``$P\rightarrow P$''
%in a more compact way.
%
%The point here is that what constitutes a proper definition is a matter of
%judgment about whether a symbol sequence can easily be recognized by a human
%as invalid in some sense (for example, not a wff); if so, the symbol sequence
%can be appropriated for use by a definition in order to make the extended
%language more compact.  Metamath\index{Metamath} lacks the ability to make this
%judgment, since as far as Metamath is concerned the definition of a wff, for
%example, is arbitrary.  You define for Metamath how wffs\index{well-formed
%formula (wff)} are constructed according to your own preferred style.  The
%concept of a wff may not even exist in a given formal system\index{formal
%system}.  Metamath treats all definitions as if they were new axioms, and it
%is up to the human mathematician to judge whether the definition is ``proper''
%'\index{proper definition}\index{definition!proper} in some agreed-upon way.

What constitutes a definition\index{definition} versus\index{axiom vs.\
definition} an axiom\index{axiom} is sometimes arbitrary in mathematical
literature.  For example, the connectives $\vee$ ({\sc or}), $\wedge$
({\sc and}), and $\leftrightarrow$ (equivalent to) in propositional
calculus are usually considered defined symbols that can be used as
abbreviations for expressions containing the ``primitive'' connectives
$\rightarrow$ and $\neg$.  This is the way we treat them in the standard
logic and set theory database \texttt{set.mm}\index{set theory database
(\texttt{set.mm})}.  However, the first three connectives can also be
considered ``primitive,'' and axiom systems have been devised that treat
all of them as such.  For example,
\cite[p.~35]{Goodstein}\index{Goodstein, R. L.} presents one with 15
axioms, some of which in fact coincide with what we have chosen to call
definitions in \texttt{set.mm}.  In certain subsets of classical
propositional calculus, such as the intuitionist
fragment\index{intuitionism}, it can be shown that one cannot make do
with just $\rightarrow$ and $\neg$ but must treat additional connectives
as primitive in order for the system to make sense.\footnote{Two nice
systems that make the transition from intuitionistic and other weak
fragments to classical logic just by adding axioms are given in
\cite{Robinsont}\index{Robinson, T. Thacher}.}

\subsection{The Approach to Definitions in \texttt{set.mm}}

In set theory, recursive definitions define a newly introduced symbol in
terms of itself.  The justification of recursive definitions, using
several ``recursion theorems,'' is usually one of the first
sophisticated proofs a student encounters when learning set theory, and
there is a significant amount of implicit metalogic behind a recursive
definition even though the definition itself is typically simple to
state.  It is, however, possible to substitute one kind of complexity
for another.  We can eliminate the need for metalogical justification by
defining the operation directly with an explicit (but complicated)
expression, then deriving the recursive definition directly as a
theorem, using a recursion theorem ``in reverse.''  We do this in
\texttt{set.mm}, as follows.

In \texttt{set.mm} our goal was to introduce almost all definitions in
the form of two expressions connected by either $\leftrightarrow$ or
$=$, where the thing being defined does not appear on the right hand
side.  Quine calls this form ``a genuine or direct definition'' \cite[p.
174]{Quine}\index{Quine, Willard Van Orman}, which makes the definitions
very easy to eliminate and the metalogic\index{metalogic} needed to
justify them as simple as possible.

We achieved this goal in almost all cases in \texttt{set.mm}.
Sometimes this makes the definitions more complex and less
intuitive.  For example, the traditional way to define addition of
natural numbers is to define an operation called {\em
successor}\index{successor} (which means ``plus one'' and is denoted by
``${\rm suc}$''), then define addition recursively\index{recursive
definition} with the two definitions $n + 0 = n$ and $m + {\rm suc}\,n =
{\rm suc} (m + n)$.  Although this definition seems simple and obvious,
the method to eliminate the definition is not obvious:  in the second
part of the definition, addition is defined in terms of itself.  By
eliminating the definition, we don't mean repeatedly applying it to
specific $m$ and $n$ but rather showing the explicit, closed-form
set-theoretical expression that $m + n$ represents, that will work for
any $m$ and $n$ and that does not have a $+$ sign on its right-hand
side.  For a recursive definition like this not to be circular
(creative), there are some hidden, underlying assumptions we must make,
for example that the natural numbers have a certain kind of order.  In
\texttt{set.mm} we chose to start with the direct (though complex and
nonintuitive) definition then derive from it the standard recursive
definition.\footnote{The closed-form definition used in \texttt{set.mm}
for the addition operation on ordinals\index{ordinal
addition}\index{addition!of ordinals} (of which natural numbers are a
subset) is

\setbox\startprefix=\hbox{\tt \ \ df-oadd\ \$a\ }
\setbox\contprefix=\hbox{\tt \ \ \ \ \ \ \ \ \ \ \ \ \ }
\startm
\m{\vdash}\m{+_o}\m{=}\m{\{}\m{\langle}\m{\langle}\m{x}\m{,}\m{y}\m{\rangle}%
\m{,}\m{z}\m{\rangle}\m{|}\m{(}\m{(}\m{x}\m{\in}\m{{\rm On}}\m{\wedge}\m{y}\m{%
\in}\m{{\rm On}}\m{)}\m{\wedge}\m{z}\m{=}\m{(}\m{{\rm rec}}\m{(}\m{\{}\m{%
\langle}\m{w}\m{,}\m{v}\m{\rangle}\m{|}\m{v}\m{=}\m{{\rm suc}}\m{w}\m{\}}\m{,}%
\m{x}\m{)}\m{`}\m{y}\m{)}\m{)}\m{\}}
\endm

\noindent Here, the abstraction class of nested ordered pairs\index{abstraction
class!of nested ordered pairs} is defined by \texttt{df-oprab} in \texttt{set.mm},
and ${\rm rec}$ is a ``recursion operator''\index{recursion operator} with
the definition

\setbox\startprefix=\hbox{\tt \ \ df-rfg\ \$a\ }
\setbox\contprefix=\hbox{\tt \ \ \ \ \ \ \ \ \ \ \ \ }
\startm
\m{\vdash}\m{{\rm rec}}\m{(}\m{F}\m{,}\m{A}\m{)}\m{=}\m{\bigcup}\m{\{}\m{f}%
\m{|}\m{\exists}\m{x}\m{(}\m{x}\m{\in}\m{{\rm On}}\m{\wedge}\m{(}\m{f}\m{{%
\rm Fn}}\m{x}\m{\wedge}\m{\forall}\m{y}\m{(}\m{y}\m{\in}\m{x}\m{\rightarrow}%
\m{(}\m{f}\m{`}\m{y}\m{)}\m{=}\m{(}\m{\{}\m{\langle}\m{g}\m{,}\m{z}\m{\rangle}%
\m{|}\m{(}\m{(}\m{g}\m{=}\m{\varnothing}\m{\wedge}\m{z}\m{=}\m{A}\m{)}\m{\vee}%
\m{(}\m{\lnot}\m{(}\m{g}\m{=}\m{\varnothing}\m{\vee}\m{{\rm Lim}}\m{{\rm dom}}%
\m{g}\m{)}\m{\wedge}\m{z}\m{=}\m{(}\m{F}\m{`}\m{(}\m{g}\m{`}\m{\bigcup}\m{{%
\rm dom}}\m{g}\m{)}\m{)}\m{)}\m{\vee}\m{(}\m{{\rm Lim}}\m{{\rm dom}}\m{g}\m{%
\wedge}\m{z}\m{=}\m{\bigcup}\m{{\rm ran}}\m{g}\m{)}\m{)}\m{\}}\m{`}\m{(}\m{f}%
\m{\restriction}\m{y}\m{)}\m{)}\m{)}\m{)}\m{)}\m{\}}
\endm

\noindent which can be further broken down with definitions shown in
Section~\ref{setdefinitions}.  You may be surprised at the complexity of what
seems like such a simple notion.  From these definitions the simpler, more
intuitive recursive definition is derived as a set of theorems.}\index{natural
number}\index{addition}\index{recursive definition}\index{ordinal addition}
The end result is the same, but we completely eliminate the rather complex
metalogic that justifies the recursive definition.  (For a mathematician,
recursive definitions are more efficient and intuitive than
direct ones once the metalogic has been learned or possibly just accepted as
correct.  However, it was felt that direct definition in \texttt{set.mm}
maximizes rigor by minimizing metalogic.  It can be eliminated effortlessly,
something difficult to do with a recursive definition.)

\subsection{Adding Constraints on Definitions}

The basic Metamath language and the Metamath program do
not have any built-in constraints on definitions, since they are just
\$a statements.

However, nothing prevents a verification system from verifying
additional rules to impose further limitations on definitions.
For example, the \texttt{mmj2}\index{mmj2} program
supports various kinds of
additional information comments (see section \ref{jcomment}).
One of their uses is to optionally verify additional constraints,
including constraints to verify that definitions meet certain
requirements.
These additional checks are required by the
continuous integration (CI)\index{continuous integration (CI)}
checks of the
\texttt{set.mm}\index{set theory database (\texttt{set.mm})}%
\index{Metamath Proof Explorer}
database.
This approach enables us to optionally impose additional requirements
on definitions if we wish, without necessarily imposing those rules on
all databases or requiring all verification systems to implement them.
In addition, this allows us to impose specialized constraints tailored
to one database while not requiring other systems to implement
those specialized constraints.

We impose two constraints on the
\texttt{set.mm}\index{set theory database (\texttt{set.mm})}%
\index{Metamath Proof Explorer} database
via the \texttt{mmj2}\index{mmj2} program that are worth discussing here:
a parse check and a definition soundness check.

% On February 11, 2019 8:32:32 PM EST, saueran@oregonstate.edu wrote:
% The following addition to the end of set.mm is accepted by the mmj2
% parser and definition checker and the metamath verifier(at least it was
% when I checked, you should check it too), and creates a contradiction by
% proving the theorem |- ph.
% ${
% wleftp $a wff ( ( ph ) $.
% wbothp $a wff ( ph ) $.
% df-leftp $a |- ( ( ( ph ) <-> -. ph ) $.
% df-bothp $a |- ( ( ph ) <-> ph ) $.
% anything $p |- ph $=
%   ( wbothp wn wi wleftp df-leftp biimpi df-bothp mpbir mpbi simplim ax-mp)
%   ABZAMACZDZCZMOEZOCQAEZNDZRNAFGSHIOFJMNKLAHJ $.
% $}
%
% This particular problem is countered by enabling, within mmj2,
% SetParser,mmj.verify.LRParser

First,
we enable a parse check in \texttt{mmj2} (through its
\texttt{SetParser} command) that requires that all new definitions
must \textit{not} create an ambiguous parse for a KLR(5) parser.
This prevents some errors such as definitions with imbalanced parentheses.

Second, we run a definition soundness check specific to
\texttt{set.mm} or databases similar to it.
(through the \texttt{definitionCheck} macro).
Some \texttt{\$a} statements (including all ax-* statemnets)
are excluded from these checks, as they will
always fail this simple check,
but they are appropriate for most definitions.
This check imposes a set of additional rules:

\begin{enumerate}

\item New definitions must be introduced using $=$ or $\leftrightarrow$.

\item No \texttt{\$a} statement introduced before this one is allowed to use the
symbol being defined in this definition, and the definition is not
allowed to use itself (except once, in the definiendum).

\item Every variable in the definiens should not be distinct

\item Every dummy variable in the definiendum
are required to be distinct from each other and from variables in
the definiendum.
To determine this, the system will look for a "justification" theorem
in the database, and if it is not there, attempt to briefly prove
$( \varphi \rightarrow \forall x \varphi )$  for each dummy variable x.

\item Every dummy variable should be a set variable,
unless there is a justification theorem available.

\item Every dummy variable must be bound
(if the system cannot determine this a justification theorem must be
provided).

\end{enumerate}

\subsection{Summary of Approach to Definitions}

In short, when being rigorous it turns out that
definitions can be subtle, sometimes requiring difficult
metatheorems to establish that they are not creative.

Instead of building such complications into the Metamath language itself,
the basic Metmath language and program simply treat traditional
axioms and definitions as the same kind of \texttt{\$a} statement.
We have then built various tools to enable people to
verify additional conditions as their creators believe is appropriate
for those specific databases, without complicating the Metamath foundations.

\chapter{The Metamath Program}\label{commands}

This chapter provides a reference manual for the
Metamath program.\index{Metamath!commands}

Current instructions for obtaining and installing the Metamath program
can be found at the \url{http://metamath.org} web site.
For Windows, there is a pre-compiled version called
\texttt{metamath.exe}.  For Unix, Linux, and Mac OS X
(which we will refer to collectively as ``Unix''), the Metamath program
can be compiled from its source code with the command
\begin{verbatim}
gcc *.c -o metamath
\end{verbatim}
using the \texttt{gcc} {\sc c} compiler available on those systems.

In the command syntax descriptions below, fields enclosed in square
brackets [\ ] are optional.  File names may be optionally enclosed in
single or double quotes.  This is useful if the file name contains
spaces or
slashes (\texttt{/}), such as in Unix path names, \index{Unix file
names}\index{file names!Unix} that might be confused with Metamath
command qualifiers.\index{Unix file names}\index{file names!Unix}

\section{Invoking Metamath}

Unix, Linux, and Mac OS X
have a command-line interface called the {\em
bash shell}.  (In Mac OS X, select the Terminal application from
Applications/Utilities.)  To invoke Metamath from the bash shell prompt,
assuming that the Metamath program is in the current directory, type
\begin{verbatim}
bash$ ./metamath
\end{verbatim}

To invoke Metamath from a Windows DOS or Command Prompt, assuming that
the Metamath program is in the current directory (or in a directory
included in the Path system environment variable), type
\begin{verbatim}
C:\metamath>metamath
\end{verbatim}

To use command-line arguments at invocation, the command-line arguments
should be a list of Metamath commands, surrounded by quotes if they
contain spaces.  In Windows, the surrounding quotes must be double (not
single) quotes.  For example, to read the database file \texttt{set.mm},
verify all proofs, and exit the program, type (under Unix)
\begin{verbatim}
bash$ ./metamath 'read set.mm' 'verify proof *' exit
\end{verbatim}
Note that in Unix, any directory path with \texttt{/}'s must be
surrounded by quotes so Metamath will not interpret the \texttt{/} as a
command qualifier.  So if \texttt{set.mm} is in the \texttt{/tmp}
directory, use for the above example
\begin{verbatim}
bash$ ./metamath 'read "/tmp/set.mm"' 'verify proof *' exit
\end{verbatim}

For convenience, if the command-line has one argument and no spaces in
the argument, the command is implicitly assumed to be \texttt{read}.  In
this one special case, \texttt{/}'s are not interpreted as command
qualifiers, so you don't need quotes around a Unix file name.  Thus
\begin{verbatim}
bash$ ./metamath /tmp/set.mm
\end{verbatim}
and
\begin{verbatim}
bash$ ./metamath "read '/tmp/set.mm'"
\end{verbatim}
are equivalent.


\section{Controlling Metamath}

The Metamath program was first developed on a {\sc vax/vms} system, and
some aspects of its command line behavior reflect this heritage.
Hopefully you will find it reasonably user-friendly once you get used to
it.

Each command line is a sequence of English-like words separated by
spaces, as in \texttt{show settings}.  Command words are not case
sensitive, and only as many letters are needed as are necessary to
eliminate ambiguity; for example, \texttt{sh se} would work for the
command \texttt{show settings}.  In some cases arguments such as file
names, statement labels, or symbol names are required; these are
case-sensitive (although file names may not be on some operating
systems).

A command line is entered by typing it in then pressing the {\em return}
({\em enter}) key.  To find out what commands are available, type
\texttt{?} at the \texttt{MM>} prompt.  To find out the choices at any
point in a command, press {\em return} and you will be prompted for
them.  The default choice (the one selected if you just press {\em
return}) is shown in brackets (\texttt{<>}).

You may also type \texttt{?} in place of a command word to force
Metamath to tell you what the choices are.  The \texttt{?} method won't
work, though, if a non-keyword argument such as a file name is expected
at that point, because the program will think that \texttt{?} is the
value of the argument.

Some commands have one or more optional qualifiers which modify the
behavior of the command.  Qualifiers are preceded by a slash
(\texttt{/}), such as in \texttt{read set.mm / verify}.  Spaces are
optional around the \texttt{/}.  If you need to use a space or
slash in a command
argument, as in a Unix file name, put single or double quotes around the
command argument.

The \texttt{open log} command will save everything you see on the
screen and is useful to help you recover should something go wrong in a
proof, or if you want to document a bug.

If a command responds with more than a screenful, you will be
prompted to \texttt{<return> to continue, Q to quit, or S to scroll to
end}.  \texttt{Q} or \texttt{q} (not case-sensitive) will complete the
command internally but will suppress further output until the next
\texttt{MM>} prompt.  \texttt{s} will suppress further pausing until the
next \texttt{MM>} prompt.  After the first screen, you are also
presented with the choice of \texttt{b} to go back a screenful.  Note
that \texttt{b} may also be entered at the \texttt{MM>} prompt
immediately after a command to scroll back through the output of that
command.

A command line enclosed in quotes is executed by your operating system.
See Section~\ref{oscmd}.

{\em Warning:} Pressing {\sc ctrl-c} will abort the Metamath program
unconditionally.  This means any unsaved work will be lost.


\subsection{\texttt{exit} Command}\index{\texttt{exit} command}

Syntax:  \texttt{exit} [\texttt{/force}]

This command exits from Metamath.  If there have been changes to the
source with the \texttt{save proof} or \texttt{save new{\char`\_}proof}
commands, you will be given an opportunity to \texttt{write source} to
permanently save the changes.

In Proof Assistant\index{Proof Assistant} mode, the \texttt{exit} command will
return to the \verb/MM>/ prompt. If there were changes to the proof, you will
be given an opportunity to \texttt{save new{\char`\_}proof}.

The \texttt{quit} command is a synonym for \texttt{exit}.

Optional qualifier:
    \texttt{/force} - Do not prompt if changes were not saved.  This qualifier is
        useful in \texttt{submit} command files (Section~\ref{sbmt})
        to ensure predictable behavior.





\subsection{\texttt{open log} Command}\index{\texttt{open log} command}
Syntax:  \texttt{open log} {\em file-name}

This command will open a log file that will store everything you see on
the screen.  It is useful to help recovery from a mistake in a long Proof
Assistant session, or to document bugs.\index{Metamath!bugs}

The log file can be closed with \texttt{close log}.  It will automatically be
closed upon exiting Metamath.



\subsection{\texttt{close log} Command}\index{\texttt{close log} command}
Syntax:  \texttt{close log}

The \texttt{close log} command closes a log file if one is open.  See
also \texttt{open log}.




\subsection{\texttt{submit} Command}\index{\texttt{submit} command}\label{sbmt}
Syntax:  \texttt{submit} {\em filename}

This command causes further command lines to be taken from the specified
file.  Note that any line beginning with an exclamation point (\texttt{!}) is
treated as a comment (i.e.\ ignored).  Also note that the scrolling
of the screen output is continuous, so you may want to open a log file
(see \texttt{open log}) to record the results that fly by on the screen.
After the lines in the file are exhausted, Metamath returns to its
normal user interface mode.

Currently, the \texttt{submit} command is not recursive.  In other words,
\texttt{submit} commands are not allowed inside a command file.


\subsection{\texttt{erase} Command}\index{\texttt{erase} command}
Syntax:  \texttt{erase}

This command will reset Metamath to its starting state, deleting any
data\-base that was \texttt{read} in.
 If there have been changes to the
source with the \texttt{save proof} or \texttt{save new{\char`\_}proof}
commands, you will be given an opportunity to \texttt{write source} to
permanently save the changes.



\subsection{\texttt{set echo} Command}\index{\texttt{set echo} command}
Syntax:  \texttt{set echo on} or \texttt{set echo off}

The \texttt{set echo on} command will cause command lines to be echoed with any
abbreviations expanded.  While learning the Metamath commands, this
feature will show you the exact command that your abbreviated input
corresponds to.



\subsection{\texttt{set scroll} Command}\index{\texttt{set scroll} command}
Syntax:  \texttt{set scroll prompted} or \texttt{set scroll continuous}

The Metamath command line interface starts off in the \texttt{prompted} mode,
which means that you will be prompted to continue or quit after each
full screen in a long listing.  In \texttt{continuous} mode, long listings will be
scrolled without pausing.

% LaTeX bug? (1) \texttt{\_} puts out different character than
% \texttt{{\char`\_}}
%  = \verb$_$  (2) \texttt{{\char`\_}} puts out garbage in \subsection
%  argument
\subsection{\texttt{set width} Command}\index{\texttt{set
width} command}
Syntax:  \texttt{set width} {\em number}

Metamath assumes the width of your screen is 79 characters (chosen
because the Command Prompt in Windows XP has a wrapping bug at column
80).  If your screen is wider or narrower, this command allows you to
change this default screen width.  A larger width is advantageous for
logging proofs to an output file to be printed on a wide printer.  A
smaller width may be necessary on some terminals; in this case, the
wrapping of the information messages may sometimes seem somewhat
unnatural, however.  In \LaTeX\index{latex@{\LaTeX}!characters per line},
there is normally a maximum of 61 characters per line with typewriter
font.  (The examples in this book were produced with 61 characters per
line.)

\subsection{\texttt{set height} Command}\index{\texttt{set
height} command}
Syntax:  \texttt{set height} {\em number}

Metamath assumes your screen height is 24 lines of characters.  If your
screen is taller or shorter, this command lets you to change the number
of lines at which the display pauses and prompts you to continue.

\subsection{\texttt{beep} Command}\index{\texttt{beep} command}

Syntax:  \texttt{beep}

This command will produce a beep.  By typing it ahead after a
long-running command has started, it will alert you that the command is
finished.  For convenience, \texttt{b} is an abbreviation for
\texttt{beep}.

Note:  If \texttt{b} is typed at the \texttt{MM>} prompt immediately
after the end of a multiple-page display paged with ``\texttt{Press
<return> for more}...'' prompts, then the \texttt{b} will back up to the
previous page rather than perform the \texttt{beep} command.
In that case you must type the unabbreviated \texttt{beep} form
of the command.

\subsection{\texttt{more} Command}\index{\texttt{more} command}

Syntax:  \texttt{more} {\em filename}

This command will display the contents of an {\sc ascii} file on your
screen.  (This command is provided for convenience but is not very
powerful.  See Section~\ref{oscmd} to invoke your operating system's
command to do this, such as the \texttt{more} command in Unix.)

\subsection{Operating System Commands}\index{operating system
command}\label{oscmd}

A line enclosed in single or double quotes will be executed by your
computer's operating system if it has a command line interface.  For
example, on a {\sc vax/vms} system,
\verb/MM> 'dir'/
will print disk directory contents.  Note that this feature will not
work on the Macintosh prior to Mac OS X, which does not have a command
line interface.

For your convenience, the trailing quote is optional.

\subsection{Size Limitations in Metamath}

In general, there are no fixed, predefined limits\index{Metamath!memory
limits} on how many labels, tokens\index{token}, statements, etc.\ that
you may have in a database file.  The Metamath program uses 32-bit
variables (64-bit on 64-bit CPUs) as indices for almost all internal
arrays, which are allocated dynamically as needed.



\section{Reading and Writing Files}

The following commands create new files:  the \texttt{open} commands;
the \texttt{write} commands; the \texttt{/html},
\texttt{/alt{\char`\_}html}, \texttt{/brief{\char`\_}html},
\texttt{/brief{\char`\_}alt{\char`\_}html} qualifiers of \texttt{show
statement}, and \texttt{midi}.  The following commands append to files
previously opened:  the \texttt{/tex} qualifier of \texttt{show proof}
and \texttt{show new{\char`\_}proof}; the \texttt{/tex} and
\texttt{/simple{\char`\_}tex} qualifiers of \texttt{show statement}; the
\texttt{close} commands; and all screen dialog between \texttt{open log}
and \texttt{close log}.

The commands that create new files will not overwrite an existing {\em
filename} but will rename the existing one to {\em
filename}\texttt{{\char`\~}1}.  An existing {\em
filename}\texttt{{\char`\~}1} is renamed {\em
filename}\texttt{{\char`\~}2}, etc.\ up to {\em
filename}\texttt{{\char`\~}9}.  An existing {\em
filename}\texttt{{\char`\~}9} is deleted.  This makes recovery from
mistakes easier but also will clutter up your directory, so occasionally
you may want to clean up (delete) these \texttt{{\char`\~}}$n$ files.


\subsection{\texttt{read} Command}\index{\texttt{read} command}
Syntax:  \texttt{read} {\em file-name} [\texttt{/verify}]

This command will read in a Metamath language source file and any included
files.  Normally it will be the first thing you do when entering Metamath.
Statement syntax is checked, but proof syntax is not checked.
Note that the file name may be enclosed in single or double quotes;
this is useful if the file name contains slashes, as might be the case
under Unix.

If you are getting an ``\texttt{?Expected VERIFY}'' error
when trying to read a Unix file name with slashes, you probably haven't
quoted it.\index{Unix file names}\index{file names!Unix}

If you are prompted for the file name (by pressing {\em return}
 after \texttt{read})
you should {\em not} put quotes around it, even if it is a Unix file name
with slashes.

Optional command qualifier:

    \texttt{/verify} - Verify all proofs as the database is read in.  This
         qualifier will slow down reading in the file.  See \texttt{verify
         proof} for more information on file error-checking.

See also \texttt{erase}.



\subsection{\texttt{write source} Command}\index{\texttt{write source} command}
Syntax:  \texttt{write source} {\em filename}
[\texttt{/rewrap}]
[\texttt{/split}]
[\texttt{/keep\_includes}]
[\texttt{/no\_versioning}]

This command will write the contents of a Metamath\index{database}
database into a file.\index{source file}

Optional command qualifiers:

\texttt{/rewrap} -
Reformats statements and comments according to the
convention used in the set.mm database.
It unwraps the
lines in the comment before each \texttt{\$a} and \texttt{\$p} statement,
then it
rewraps the line.  You should compare the output to the original
to make sure that the desired effect results; if not, go back to
the original source.  The wrapped line length honors the
\texttt{set width}
parameter currently in effect.  Note:  Text
enclosed in \texttt{<HTML>}...\texttt{</HTML>} tags is not modified by the
\texttt{/rewrap} qualifier.
Proofs themselves are not reformatted;
use \texttt{save proof * / compressed} to do that.
An isolated tilde (\~{}) is always kept on the same line as the following
symbol, so you can find all comment references to a symbol by
searching for \~{} followed by a space and that symbol
(this is useful for finding cross-references).
Incidentally, \texttt{save proof} also honors the \texttt{set width}
parameter currently in effect.

\texttt{/split} - Files included in the source using the expression
\$[ \textit{inclfile} \$] will be
written into separate files instead of being included in a single output
file.  The name of each separately written included file will be the
\textit{inclfile} argument of its inclusion command.

\texttt{/keep\_includes} - If a source file has includes but is written as a
single file by omitting \texttt{/split}, by default the included files will
be deleted (actually just renamed with a \char`\~1 suffix unless
\texttt{/no\_versioning} is specified) to prevent the possibly confusing
source duplication in both the output file and the included file.
The \texttt{/keep\_includes} qualifier will prevent this deletion.

\texttt{/no\_versioning} - Backup files suffixed with \char`\~1 are not created.


\section{Showing Status and Statements}



\subsection{\texttt{show settings} Command}\index{\texttt{show settings} command}
Syntax:  \texttt{show settings}

This command shows the state of various parameters.

\subsection{\texttt{show memory} Command}\index{\texttt{show memory} command}
Syntax:  \texttt{show memory}

This command shows the available memory left.  It is not meaningful
on most modern operating systems,
which have virtual memory.\index{Metamath!memory usage}


\subsection{\texttt{show labels} Command}\index{\texttt{show labels} command}
Syntax:  \texttt{show labels} {\em label-match} [\texttt{/all}]
   [\texttt{/linear}]

This command shows the labels of \texttt{\$a} and \texttt{\$p}
statements that match {\em label-match}.  A \verb$*$ in {label-match}
matches zero or more characters.  For example, \verb$*abc*def$ will match all
labels containing \verb$abc$ and ending with \verb$def$.

Optional command qualifiers:

   \texttt{/all} - Include matches for \texttt{\$e} and \texttt{\$f}
   statement labels.

   \texttt{/linear} - Display only one label per line.  This can be useful for
       building scripts in conjunction with the utilities under the
       \texttt{tools}\index{\texttt{tools} command} command.



\subsection{\texttt{show statement} Command}\index{\texttt{show statement} command}
Syntax:  \texttt{show statement} {\em label-match} [{\em qualifiers} (see below)]

This command provides information about a statement.  Only statements
that have labels (\texttt{\$f}\index{\texttt{\$f} statement},
\texttt{\$e}\index{\texttt{\$e} statement},
\texttt{\$a}\index{\texttt{\$a} statement}, and
\texttt{\$p}\index{\texttt{\$p} statement}) may be specified.
If {\em label-match}
contains wildcard (\verb$*$) characters, all matching statements will be
displayed in the order they occur in the database.

Optional qualifiers (only one qualifier at a time is allowed):

    \texttt{/comment} - This qualifier includes the comment that immediately
       precedes the statement.

    \texttt{/full} - Show complete information about each statement,
       and show all
       statements matching {\em label} (including \texttt{\$e}
       and \texttt{\$f} statements).

    \texttt{/tex} - This qualifier will write the statement information to the
       \LaTeX\ file previously opened with \texttt{open tex}.  See
       Section~\ref{texout}.

    \texttt{/simple{\char`\_}tex} - The same as \texttt{/tex}, except that
       \LaTeX\ macros are not used for formatting equations, allowing easier
       manual edits of the output for slide presentations, etc.

    \texttt{/html}\index{html generation@{\sc html} generation},
       \texttt{/alt{\char`\_}html}, \texttt{/brief{\char`\_}html},
       \texttt{/brief{\char`\_}alt{\char`\_}html} -
       These qualifiers invoke a special mode of
       \texttt{show statement} that
       creates a web page for the statement.  They may not be used with
       any other qualifier.  See Section~\ref{htmlout} or
       \texttt{help html} in the program.


\subsection{\texttt{search} Command}\index{\texttt{search} command}
Syntax:  search {\em label-match}
\texttt{"}{\em symbol-match}{\tt}" [\texttt{/all}] [\texttt{/comments}]
[\texttt{/join}]

This command searches all \texttt{\$a} and \texttt{\$p} statements
matching {\em label-match} for occurrences of {\em symbol-match}.  A
\verb@*@ in {\em label-match} matches any label character.  A \verb@$*@
in {\em symbol-match} matches any sequence of symbols.  The symbols in
{\em symbol-match} must be separated by white space.  The quotes
surrounding {\em symbol-match} may be single or double quotes.  For
example, \texttt{search b}\verb@* "-> $* ch"@ will list all statements
whose labels begin with \texttt{b} and contain the symbols \verb@->@ and
\texttt{ch} surrounding any symbol sequence (including no symbol
sequence).  The wildcards \texttt{?} and \texttt{\$?} are also available
to match individual characters in labels and symbols respectively; see
\texttt{help search} in the Metamath program for details on their usage.

Optional command qualifiers:

    \texttt{/all} - Also search \texttt{\$e} and \texttt{\$f} statements.

    \texttt{/comments} - Search the comment that immediately precedes each
        label-matched statement for {\em symbol-match}.  In this case
        {\em symbol-match} is an arbitrary, non-case-sensitive character
        string.  Quotes around {\em symbol-match} are optional if there
        is no ambiguity.

    \texttt{/join} - In the case of a \texttt{\$a} or \texttt{\$p} statement,
	prepend its \texttt{\$e}
	hypotheses for searching. The
	\texttt{/join} qualifier has no effect in \texttt{/comments} mode.

\section{Displaying and Verifying Proofs}


\subsection{\texttt{show proof} Command}\index{\texttt{show proof} command}
Syntax:  \texttt{show proof} {\em label-match} [{\em qualifiers} (see below)]

This command displays the proof of the specified
\texttt{\$p}\index{\texttt{\$p} statement} statement in various formats.
The {\em label-match} may contain wildcard (\verb@$*@) characters to match
multiple statements.  Without any qualifiers, only the logical steps
will be shown (i.e.\ syntax construction steps will be omitted), in an
indented format.

Most of the time, you will use
    \texttt{show proof} {\em label}
to see just the proof steps corresponding to logical inferences.

Optional command qualifiers:

    \texttt{/essential} - The proof tree is trimmed of all
        \texttt{\$f}\index{\texttt{\$f} statement} hypotheses before
        being displayed.  (This is the default, and it is redundant to
        specify it.)

    \texttt{/all} - the proof tree is not trimmed of all \texttt{\$f} hypotheses before
        being displayed.  \texttt{/essential} and \texttt{/all} are mutually exclusive.

    \texttt{/from{\char`\_}step} {\em step} - The display starts at the specified
        step.  If
        this qualifier is omitted, the display starts at the first step.

    \texttt{/to{\char`\_}step} {\em step} - The display ends at the specified
        step.  If this
        qualifier is omitted, the display ends at the last step.

    \texttt{/tree{\char`\_}depth} {\em number} - Only
         steps at less than the specified proof
        tree depth are displayed.  Sometimes useful for obtaining an overview of
        the proof.

    \texttt{/reverse} - The steps are displayed in reverse order.

    \texttt{/renumber} - When used with \texttt{/essential}, the steps are renumbered
        to correspond only to the essential steps.

    \texttt{/tex} - The proof is converted to \LaTeX\ and\index{latex@{\LaTeX}}
        stored in the file opened
        with \texttt{open tex}.  See Section~\ref{texout} or
        \texttt{help tex} in the program.

    \texttt{/lemmon} - The proof is displayed in a non-indented format known
        as Lemmon style, with explicit previous step number references.
        If this qualifier is omitted, steps are indented in a tree format.

    \texttt{/start{\char`\_}column} {\em number} - Overrides the default column
        (16)
        at which the formula display starts in a Lemmon-style display.  May be
        used only in conjunction with \texttt{/lemmon}.

    \texttt{/normal} - The proof is displayed in normal format suitable for
        inclusion in a Metamath source file.  May not be used with any other
        qualifier.

    \texttt{/compressed} - The proof is displayed in compressed format
        suitable for inclusion in a Metamath source file.  May not be used with
        any other qualifier.

    \texttt{/statement{\char`\_}summary} - Summarizes all statements (like a
        brief \texttt{show statement})
        used by the proof.  It may not be used with any other qualifier
        except \texttt{/essential}.

    \texttt{/detailed{\char`\_}step} {\em step} - Shows the details of what is
        happening at
        a specific proof step.  May not be used with any other qualifier.
        The {\em step} is the step number shown when displaying a
        proof without the \texttt{/renumber} qualifier.


\subsection{\texttt{show usage} Command}\index{\texttt{show usage} command}
Syntax:  \texttt{show usage} {\em label-match} [\texttt{/recursive}]

This command lists the statements whose proofs make direct reference to
the statement specified.

Optional command qualifier:

    \texttt{/recursive} - Also include statements whose proofs ultimately
        depend on the statement specified.



\subsection{\texttt{show trace\_back} Command}\index{\texttt{show
       trace{\char`\_}back} command}
Syntax:  \texttt{show trace{\char`\_}back} {\em label-match} [\texttt{/essential}] [\texttt{/axioms}]
    [\texttt{/tree}] [\texttt{/depth} {\em number}]

This command lists all statements that the proof of the \texttt{\$p}
statement(s) specified by {\em label-match} depends on.

Optional command qualifiers:

    \texttt{/essential} - Restrict the trace-back to \texttt{\$e}
        \index{\texttt{\$e} statement} hypotheses of proof trees.

    \texttt{/axioms} - List only the axioms that the proof ultimately depends on.

    \texttt{/tree} - Display the trace-back in an indented tree format.

    \texttt{/depth} {\em number} - Restrict the \texttt{/tree} trace-back to the
        specified indentation depth.

    \texttt{/count{\char`\_}steps} - Count the number of steps the proof has
       all the way back to axioms.  If \texttt{/essential} is specified,
       expansions of variable-type hypotheses (syntax constructions) are not counted.

\subsection{\texttt{verify proof} Command}\index{\texttt{verify proof} command}
Syntax:  \texttt{verify proof} {\em label-match} [\texttt{/syntax{\char`\_}only}]

This command verifies the proofs of the specified statements.  {\em
label-match} may contain wild card characters (\texttt{*}) to verify
more than one proof; for example \verb/*abc*def/ will match all labels
containing \texttt{abc} and ending with \texttt{def}.
The command \texttt{verify proof *} will verify all proofs in the database.

Optional command qualifier:

    \texttt{/syntax{\char`\_}only} - This qualifier will perform a check of syntax
        and RPN
        stack violations only.  It will not verify that the proof is
        correct.  This qualifier is useful for quickly determining which
        proofs are incomplete (i.e.\ are under development and have \texttt{?}'s
        in them).

{\em Note:} \texttt{read}, followed by \texttt{verify proof *}, will ensure
 the database is free
from errors in the Metamath language but will not check the markup notation
in comments.
You can also check the markup notation by running \texttt{verify markup *}
as discussed in Section~\ref{verifymarkup}; see also the discussion
on generating {\sc HTML} in Section~\ref{htmlout}.

\subsection{\texttt{verify markup} Command}\index{\texttt{verify markup} command}\label{verifymarkup}
Syntax:  \texttt{verify markup} {\em label-match}
[\texttt{/date{\char`\_}skip}]
[\texttt{/top{\char`\_}date{\char`\_}skip}]
[\texttt{/file{\char`\_}skip}]
[\texttt{/verbose}]

This command checks comment markup and other informal conventions we have
adopted.  It error-checks the latexdef, htmldef, and althtmldef statements
in the \texttt{\$t} statement of a Metamath source file.\index{error checking}
It error-checks any \texttt{`}...\texttt{`},
\texttt{\char`\~}~\textit{label},
and bibliographic markups in statement descriptions.
It checks that
\texttt{\$p} and \texttt{\$a} statements
have the same content when their labels start with
``ax'' and ``ax-'' respectively but are otherwise identical, for example
ax4 and ax-4.
It also verifies the date consistency of ``(Contributed by...),''
``(Revised by...),'' and ``(Proof shortened by...)'' tags in the comment
above each \texttt{\$a} and \texttt{\$p} statement.

Optional command qualifiers:

    \texttt{/date{\char`\_}skip} - This qualifier will
        skip date consistency checking,
        which is usually not required for databases other than
	\texttt{set.mm}.

    \texttt{/top{\char`\_}date{\char`\_}skip} - This qualifier will check date consistency except
        that the version date at the top of the database file will not
        be checked.  Only one of
        \texttt{/date{\char`\_}skip} and
        \texttt{/top{\char`\_}date{\char`\_}skip} may be
        specified.

    \texttt{/file{\char`\_}skip} - This qualifier will skip checks that require
        external files to be present, such as checking GIF existence and
        bibliographic links to mmset.html or equivalent.  It is useful
        for doing a quick check from a directory without these files.

    \texttt{/verbose} - Provides more information.  Currently it provides a list
        of axXXX vs. ax-XXX matches.

\subsection{\texttt{save proof} Command}\index{\texttt{save proof} command}
Syntax:  \texttt{save proof} {\em label-match} [\texttt{/normal}]
   [\texttt{/compressed}]

The \texttt{save proof} command will reformat a proof in one of two formats and
replace the existing proof in the source buffer\index{source
buffer}.  It is useful for
converting between proof formats.  Note that a proof will not be
permanently saved until a \texttt{write source} command is issued.

Optional command qualifiers:

    \texttt{/normal} - The proof is saved in the normal format (i.e., as a
        sequence
        of labels, which is the defined format of the basic Metamath
        language).\index{basic language}  This is the default format that
        is used if a qualifier
        is omitted.

    \texttt{/compressed} - The proof is saved in the compressed format which
        reduces storage requirements for a database.
        See Appendix~\ref{compressed}.




\section{Creating Proofs}\label{pfcommands}\index{Proof Assistant}

Before using the Proof Assistant, you must add a \texttt{\$p} to your
source file (using a text editor) containing the statement you want to
prove.  Its proof should consist of a single \texttt{?}, meaning
``unknown step.''  Example:
\begin{verbatim}
equid $p x = x $= ? $.
\end{verbatim}

To enter the Proof assistant, type \texttt{prove} {\em label}, e.g.
\texttt{prove equid}.  Metamath will respond with the \texttt{MM-PA>}
prompt.

Proofs are created working backwards from the statement being proved,
primarily using a series of \texttt{assign} commands.  A proof is
complete when all steps are assigned to statements and all steps are
unified and completely known.  During the creation of a proof, Metamath
will allow only operations that are legal based on what is known up to
that point.  For example, it will not allow an \texttt{assign} of a
statement that cannot be unified with the unknown proof step being
assigned.

{\em Important:}
The Proof Assistant is
{\em not} a tool to help you discover proofs.  It is just a tool to help
you add them to the database.  For a tutorial read
Section~\ref{frstprf}.
To practice using the Proof Assistant, you may
want to \texttt{prove} an existing theorem, then delete all steps with
\texttt{delete all}, then re-create it with the Proof Assistant while
looking at its proof display (before deletion).
You might want to figure out your first few proofs completely
and write them down by hand, before using the Proof Assistant, though
not everyone finds that effective.

{\em Important:}
The \texttt{undo} command if very helpful when entering a proof, because
it allows you to undo a previously-entered step.
In addition, we suggest that you
keep track of your work with a log file (\texttt{open
log}) and save it frequently (\texttt{save new{\char`\_}proof},
\texttt{write source}).
You can use \texttt{delete} to reverse an \texttt{assign}.
You can also do \texttt{delete floating{\char`\_}hypotheses}, then
\texttt{initialize all}, then \texttt{unify all /interactive} to
reinitialize bad unifications made accidentally or by bad
\texttt{assign}s.  You cannot reverse a \texttt{delete} except by
a relevant \texttt{undo} or using
\texttt{exit /force} then reentering the Proof Assistant to recover from
the last \texttt{save new{\char`\_}proof}.

The following commands are available in the Proof Assistant (at the
\texttt{MM-PA>} prompt) to help you create your proof.  See the
individual commands for more detail.

\begin{itemize}
\item[]
    \texttt{show new{\char`\_}proof} [\texttt{/all},...] - Displays the
        proof in progress.  You will use this command a lot; see \texttt{help
        show new{\char`\_}proof} to become familiar with its qualifiers.  The
        qualifiers \texttt{/unknown} and \texttt{/not{\char`\_}unified} are
        useful for seeing the work remaining to be done.  The combination
        \texttt{/all/unknown} is useful for identifying dummy variables that must be
        assigned, or attempts to use illegal syntax, when \texttt{improve all}
        is unable to complete the syntax constructions.  Unknown variables are
        shown as \texttt{\$1}, \texttt{\$2},...
\item[]
    \texttt{assign} {\em step} {\em label} - Assigns an unknown {\em step}
        number with the statement
        specified by {\em label}.
\item[]
    \texttt{let variable} {\em variable}
        \texttt{= "}{\em symbol sequence}\texttt{"}
          - Forces a symbol
        sequence to replace an unknown variable (such as \texttt{\$1}) in a proof.
        It is useful
        for helping difficult unifications, and it is necessary when you have
        dummy variables that eventually must be assigned a name.
\item[]
    \texttt{let step} {\em step} \texttt{= "}{\em symbol sequence}\texttt{"} -
          Forces a symbol sequence
        to replace the contents of a proof step, provided it can be
        unified with the existing step contents.  (I rarely use this.)
\item[]
    \texttt{unify step} {\em step} (or \texttt{unify all}) - Unifies
        the source and target of
        a step.  If you specify a specific step, you will be prompted
        to select among the unifications that are possible.  If you
        specify \texttt{all}, all steps with unique unifications, but only
        those steps, will be
        unified.  \texttt{unify all /interactive} goes through all non-unified
        steps.
\item[]
    \texttt{initialize} {\em step} (or \texttt{all}) - De-unifies the target and source of
        a step (or all steps), as well as the hypotheses of the source,
        and makes all variables in the source unknown.  Useful to recover from
        an \texttt{assign} or \texttt{let} mistake that
        resulted in incorrect unifications.
\item[]
    \texttt{delete} {\em step} (or \texttt{all} or \texttt{floating{\char`\_}hypotheses}) -
        Deletes the specified
        step(s).  \texttt{delete floating{\char`\_}hypotheses}, then \texttt{initialize all}, then
        \texttt{unify all /interactive} is useful for recovering from mistakes
        where incorrect unifications assigned wrong math symbol strings to
        variables.
\item[]
    \texttt{improve} {\em step} (or \texttt{all}) -
      Automatically creates a proof for steps (with no unknown variables)
      whose proof requires no statements with \texttt{\$e} hypotheses.  Useful
      for filling in proofs of \texttt{\$f} hypotheses.  The \texttt{/depth}
      qualifier will also try statements whose \texttt{\$e} hypotheses contain
      no new variables.  {\em Warning:} Save your work (with \texttt{save
      new{\char`\_}proof} then \texttt{write source}) before using
      \texttt{/depth = 2} or greater, since the search time grows
      exponentially and may never terminate in a reasonable time, and you
      cannot interrupt the search.  I have found that it is rare for
      \texttt{/depth = 3} or greater to be useful.
 \item[]
    \texttt{save new{\char`\_}proof} - Saves the proof in progress in the program's
        internal database buffer.  To save it permanently into the database file,
        use \texttt{write source} after
        \texttt{save new{\char`\_}proof}.  To revert to the last
        \texttt{save new{\char`\_}proof},
        \texttt{exit /force} from the Proof Assistant then re-enter the Proof
        Assistant.
 \item[]
    \texttt{match step} {\em step} (or \texttt{match all}) - Shows what
        statements are
        possibilities for the \texttt{assign} statement. (This command
        is not very
        useful in its present form and hopefully will be improved
        eventually.  In the meantime, use the \texttt{search} statement for
        candidates matching specific math token combinations.)
 \item[]
 \texttt{minimize{\char`\_}with}\index{\texttt{minimize{\char`\_}with} command}
% 3/10/07 Note: line-breaking the above results in duplicate index entries
     - After a proof is complete, this command will attempt
        to match other database theorems to the proof to see if the proof
        size can be reduced as a result.  See \texttt{help
        minimize{\char`\_}with} in the
        Metamath program for its usage.
 \item[]
 \texttt{undo}\index{\texttt{undo} command}
    - Undo the effect of a proof-changing command (all but the
      \texttt{show} and \texttt{save} commands above).
 \item[]
 \texttt{redo}\index{\texttt{redo} command}
    - Reverse the previous \texttt{undo}.
\end{itemize}

The following commands set parameters that may be relevant to your proof.
Consult the individual \texttt{help set}... commands.
\begin{itemize}
   \item[] \texttt{set unification{\char`\_}timeout}
 \item[]
    \texttt{set search{\char`\_}limit}
  \item[]
    \texttt{set empty{\char`\_}substitution} - note that default is \texttt{off}
\end{itemize}

Type \texttt{exit} to exit the \texttt{MM-PA>}
 prompt and get back to the \texttt{MM>} prompt.
Another \texttt{exit} will then get you out of Metamath.



\subsection{\texttt{prove} Command}\index{\texttt{prove} command}
Syntax:  \texttt{prove} {\em label}

This command will enter the Proof Assistant\index{Proof Assistant}, which will
allow you to create or edit the proof of the specified statement.
The command-line prompt will change from \texttt{MM>} to \texttt{MM-PA>}.

Note:  In the present version of
Metamath\index{Metamath!limitations of version 0.07.30}, the Proof
Assistant does not verify that \texttt{\$d}\index{\texttt{\$d}
statement} restrictions are met as a proof is being built.  After you
have completed a proof, you should type \texttt{save new{\char`\_}proof}
followed by \texttt{verify proof} {\em label} (where {\em label} is the
statement you are proving with the \texttt{prove} command) to verify the
\texttt{\$d} restrictions.

See also: \texttt{exit}

\subsection{\texttt{set unification\_timeout} Command}\index{\texttt{set
unification{\char`\_}timeout} command}
Syntax:  \texttt{set unification{\char`\_}timeout} {\em number}

(This command is available outside the Proof Assistant but affects the
Proof Assistant\index{Proof Assistant} only.)

Sometimes the Proof Assistant will inform you that a unification
time-out occurred.  This may happen when you try to \texttt{unify}
formulas with many temporary variables\index{temporary variable}
(\texttt{\$1}, \texttt{\$2}, etc.), since the time to compute all possible
unifications may grow exponentially with the number of variables.  If
you want Metamath to try harder (and you're willing to wait longer) you
may increase this parameter.  \texttt{show settings} will show you the
current value.

\subsection{\texttt{set empty\_substitution} Command}\index{\texttt{set
empty{\char`\_}substitution} command}
% These long names can't break well in narrow mode, and even "sloppy"
% is not enough. Work around this by not demanding justification.
\begin{flushleft}
Syntax:  \texttt{set empty{\char`\_}substitution on} or \texttt{set
empty{\char`\_}substitution off}
\end{flushleft}

(This command is available outside the Proof Assistant but affects the
Proof Assistant\index{Proof Assistant} only.)

The Metamath language allows variables to be
substituted\index{substitution!variable}\index{variable substitution}
with empty symbol sequences\index{empty substitution}.  However, in many
formal systems\index{formal system} this will never happen in a valid
proof.  Allowing for this possibility increases the likelihood of
ambiguous unifications\index{ambiguous
unification}\index{unification!ambiguous} during proof creation.
The default is that
empty substitutions are not allowed; for formal systems requiring them,
you must \texttt{set empty{\char`\_}substitution on}.
(An example where this must be \texttt{on}
would be a system that implements a Deduction Rule and in
which deductions from empty assumption lists would be permissible.  The
MIU-system\index{MIU-system} described in Appendix~\ref{MIU} is another
example.)
Note that empty substitutions are
always permissible in proof verification (VERIFY PROOF...) outside the
Proof Assistant.  (See the MIU system in the Metamath book for an example
of a system needing empty substitutions; another example would be a
system that implements a Deduction Rule and in which deductions from
empty assumption lists would be permissible.)

It is better to leave this \texttt{off} when working with \texttt{set.mm}.
Note that this command does not affect the way proofs are verified with
the \texttt{verify proof} command.  Outside of the Proof Assistant,
substitution of empty sequences for math symbols is always allowed.

\subsection{\texttt{set search\_limit} Command}\index{\texttt{set
search{\char`\_}limit} command} Syntax:  \texttt{set search{\char`\_}limit} {\em
number}

(This command is available outside the Proof Assistant but affects the
Proof Assistant\index{Proof Assistant} only.)

This command sets a parameter that determines when the \texttt{improve} command
in Proof Assistant mode gives up.  If you want \texttt{improve} to search harder,
you may increase it.  The \texttt{show settings} command tells you its current
value.


\subsection{\texttt{show new\_proof} Command}\index{\texttt{show
new{\char`\_}proof} command}
Syntax:  \texttt{show new{\char`\_}proof} [{\em
qualifiers} (see below)]

This command (available only in Proof Assistant mode) displays the proof
in progress.  It is identical to the \texttt{show proof} command, except that
there is no statement argument (since it is the statement being proved) and
the following qualifiers are not available:

    \texttt{/statement{\char`\_}summary}

    \texttt{/detailed{\char`\_}step}

Also, the following additional qualifiers are available:

    \texttt{/unknown} - Shows only steps that have no statement assigned.

    \texttt{/not{\char`\_}unified} - Shows only steps that have not been unified.

Note that \texttt{/essential}, \texttt{/depth}, \texttt{/unknown}, and
\texttt{/not{\char`\_}unified} may be
used in any combination; each of them effectively filters out additional
steps from the proof display.

See also:  \texttt{show proof}






\subsection{\texttt{assign} Command}\index{\texttt{assign} command}
Syntax:   \texttt{assign} {\em step} {\em label} [\texttt{/no{\char`\_}unify}]

   and:   \texttt{assign first} {\em label}

   and:   \texttt{assign last} {\em label}


This command, available in the Proof Assistant only, assigns an unknown
step (one with \texttt{?} in the \texttt{show new{\char`\_}proof}
listing) with the statement specified by {\em label}.  The assignment
will not be allowed if the statement cannot be unified with the step.

If \texttt{last} is specified instead of {\em step} number, the last
step that is shown by \texttt{show new{\char`\_}proof /unknown} will be
used.  This can be useful for building a proof with a command file (see
\texttt{help submit}).  It also makes building proofs faster when you know
the assignment for the last step.

If \texttt{first} is specified instead of {\em step} number, the first
step that is shown by \texttt{show new{\char`\_}proof /unknown} will be
used.

If {\em step} is zero or negative, the -{\em step}th from last unknown
step, as shown by \texttt{show new{\char`\_}proof /unknown}, will be
used.  \texttt{assign -1} {\em label} will assign the penultimate
unknown step, \texttt{assign -2} {\em label} the antepenultimate, and
\texttt{assign 0} {\em label} is the same as \texttt{assign last} {\em
label}.

Optional command qualifier:

    \texttt{/no{\char`\_}unify} - do not prompt user to select a unification if there is
        more than one possibility.  This is useful for noninteractive
        command files.  Later, the user can \texttt{unify all /interactive}.
        (The assignment will still be automatically unified if there is only
        one possibility and will be refused if unification is not possible.)



\subsection{\texttt{match} Command}\index{\texttt{match} command}
Syntax:  \texttt{match step} {\em step} [\texttt{/max{\char`\_}essential{\char`\_}hyp}
{\em number}]

    and:  \texttt{match all} [\texttt{/essential}]
          [\texttt{/max{\char`\_}essential{\char`\_}hyp} {\em number}]

This command, available in the Proof Assistant only, shows what
statements can be unified with the specified step(s).  {\em Note:} In
its current form, this command is not very useful because of the large
number of matches it reports.  I rarely use it.
It may be enhanced in the future.

Optional command qualifiers:

    \texttt{/max{\char`\_}essential{\char`\_}hyp} {\em number} - filters out
        of the list any statements
        with more than the specified number of
        \texttt{\$e}\index{\texttt{\$e} statement} hypotheses

    \texttt{/essential{\char`\_}only} - in the \texttt{match all} statement, only
        the steps that
        would be listed in the \texttt{show new{\char`\_}proof /essential} display are
        matched.



\subsection{\texttt{let} Command}\index{\texttt{let} command}
Syntax: \texttt{let variable} {\em variable} = \verb/"/{\em symbol-sequence}\verb/"/

  and: \texttt{let step} {\em step} = \verb/"/{\em symbol-sequence}\verb/"/

These commands, available in the Proof Assistant\index{Proof Assistant}
only, assign a temporary variable\index{temporary variable} or unknown
step with a specific symbol sequence.  They are useful in the middle of
creating a proof, when you know what should be in the proof step but the
unification algorithm doesn't yet have enough information to completely
specify the temporary variables.  A ``temporary variable'' is one that
has the form \texttt{\$}{\em nn} in the proof display, such as
\texttt{\$1}, \texttt{\$2}, etc.  The {\em symbol-sequence} may contain
other unknown variables if desired.  Examples:

    \verb/let variable $32 = "A = B"/

    \verb/let variable $32 = "A = $35"/

    \verb/let step 10 = '|- x = x'/

    \verb/let step -2 = "|- ( $7 = ph )"/

Any symbol sequence will be accepted for the \texttt{let variable}
command.  Only those symbol sequences that can be unified with the step
will be accepted for \texttt{let step}.

The \texttt{let} commands ``zap'' the proof with information that can
only be verified when the proof is built up further.  If you make an
error, the command sequence \texttt{delete
floating{\char`\_}hypotheses}, \texttt{initialize all}, and
\texttt{unify all /interactive} will undo a bad \texttt{let} assignment.

If {\em step} is zero or negative, the -{\em step}th from last unknown
step, as shown by \texttt{show new{\char`\_}proof /unknown}, will be
used.  The command \texttt{let step 0} = \verb/"/{\em
symbol-sequence}\verb/"/ will use the last unknown step, \texttt{let
step -1} = \verb/"/{\em symbol-sequence}\verb/"/ the penultimate, etc.
If {\em step} is positive, \texttt{let step} may be used to assign known
(in the sense of having previously been assigned a label with
\texttt{assign}) as well as unknown steps.

Either single or double quotes can surround the {\em symbol-sequence} as
long as they are different from any quotes inside a {\em
symbol-sequence}.  If {\em symbol-sequence} contains both kinds of
quotes, see the instructions at the end of \texttt{help let} in the
Metamath program.


\subsection{\texttt{unify} Command}\index{\texttt{unify} command}
Syntax:  \texttt{unify step} {\em step}

      and:   \texttt{unify all} [\texttt{/interactive}]

These commands, available in the Proof Assistant only, unify the source
and target of the specified step(s). If you specify a specific step, you
will be prompted to select among the unifications that are possible.  If
you specify \texttt{all}, only those steps with unique unifications will be
unified.

Optional command qualifier for \texttt{unify all}:

    \texttt{/interactive} - You will be prompted to select among the
        unifications
        that are possible for any steps that do not have unique
        unifications.  (Otherwise \texttt{unify all} will bypass these.)

See also \texttt{set unification{\char`\_}timeout}.  The default is
100000, but increasing it to 1000000 can help difficult cases.  Manually
assigning some or all of the unknown variables with the \texttt{let
variable} command also helps difficult cases.



\subsection{\texttt{initialize} Command}\index{\texttt{initialize} command}
Syntax:  \texttt{initialize step} {\em step}

    and: \texttt{initialize all}

These commands, available in the Proof Assistant\index{Proof Assistant}
only, ``de-unify'' the target and source of a step (or all steps), as
well as the hypotheses of the source, and makes all variables in the
source and the source's hypotheses unknown.  This command is useful to
help recover from incorrect unifications that resulted from an incorrect
\texttt{assign}, \texttt{let}, or unification choice.  Part or all of
the command sequence \texttt{delete floating{\char`\_}hypotheses},
\texttt{initialize all}, and \texttt{unify all /interactive} will recover
from incorrect unifications.

See also:  \texttt{unify} and \texttt{delete}



\subsection{\texttt{delete} Command}\index{\texttt{delete} command}
Syntax:  \texttt{delete step} {\em step}

   and:      \texttt{delete all} -- {\em Warning: dangerous!}

   and:      \texttt{delete floating{\char`\_}hypotheses}

These commands are available in the Proof Assistant only.  The
\texttt{delete step} command deletes the proof tree section that
branches off of the specified step and makes the step become unknown.
\texttt{delete all} is equivalent to \texttt{delete step} {\em step}
where {\em step} is the last step in the proof (i.e.\ the beginning of
the proof tree).

In most cases the \texttt{undo} command is the best way to undo
a previous step.
An alternative is to salvage your last \texttt{save
new{\char`\_}proof} by exiting and reentering the Proof Assistant.
For this to work, keep a log file open to record your work
and to do \texttt{save new{\char`\_}proof} frequently, especially before
\texttt{delete}.

\texttt{delete floating{\char`\_}hypotheses} will delete all sections of
the proof that branch off of \texttt{\$f}\index{\texttt{\$f} statement}
statements.  It is sometimes useful to do this before an
\texttt{initialize} command to recover from an error.  Note that once a
proof step with a \texttt{\$f} hypothesis as the target is completely
known, the \texttt{improve} command can usually fill in the proof for
that step.  Unlike the deletion of logical steps, \texttt{delete
floating{\char`\_}hypotheses} is a relatively safe command that is
usually easy to recover from.



\subsection{\texttt{improve} Command}\index{\texttt{improve} command}
\label{improve}
Syntax:  \texttt{improve} {\em step} [\texttt{/depth} {\em number}]
                                               [\texttt{/no{\char`\_}distinct}]

   and:   \texttt{improve first} [\texttt{/depth} {\em number}]
                                              [\texttt{/no{\char`\_}distinct}]

   and:   \texttt{improve last} [\texttt{/depth} {\em number}]
                                              [\texttt{/no{\char`\_}distinct}]

   and:   \texttt{improve all} [\texttt{/depth} {\em number}]
                                              [\texttt{/no{\char`\_}distinct}]

These commands, available in the Proof Assistant\index{Proof Assistant}
only, try to find proofs automatically for unknown steps whose symbol
sequences are completely known.  They are primarily useful for filling in
proofs of \texttt{\$f}\index{\texttt{\$f} statement} hypotheses.  The
search will be restricted to statements having no
\texttt{\$e}\index{\texttt{\$e} statement} hypotheses.

\begin{sloppypar} % narrow
Note:  If memory is limited, \texttt{improve all} on a large proof may
overflow memory.  If you use \texttt{set unification{\char`\_}timeout 1}
before \texttt{improve all}, there will usually be sufficient
improvement to easily recover and completely \texttt{improve} the proof
later on a larger computer.  Warning:  Once memory has overflowed, there
is no recovery.  If in doubt, save the intermediate proof (\texttt{save
new{\char`\_}proof} then \texttt{write source}) before \texttt{improve
all}.
\end{sloppypar}

If \texttt{last} is specified instead of {\em step} number, the last
step that is shown by \texttt{show new{\char`\_}proof /unknown} will be
used.

If \texttt{first} is specified instead of {\em step} number, the first
step that is shown by \texttt{show new{\char`\_}proof /unknown} will be
used.

If {\em step} is zero or negative, the -{\em step}th from last unknown
step, as shown by \texttt{show new{\char`\_}proof /unknown}, will be
used.  \texttt{improve -1} will use the penultimate
unknown step, \texttt{improve -2} {\em label} the antepenultimate, and
\texttt{improve 0} is the same as \texttt{improve last}.

Optional command qualifier:

    \texttt{/depth} {\em number} - This qualifier will cause the search
        to include
        statements with \texttt{\$e} hypotheses (but no new variables in
        the \texttt{\$e}
        hypotheses), provided that the backtracking has not exceeded the
        specified depth. {\em Warning:}  Try \texttt{/depth 1},
        then \texttt{2}, then \texttt{3}, etc.
        in sequence because of possible exponential blowups.  Save your
        work before trying \texttt{/depth} greater than \texttt{1}!

    \texttt{/no{\char`\_}distinct} - Skip trial statements that have
        \texttt{\$d}\index{\texttt{\$d} statement} requirements.
        This qualifier will prevent assignments that might violate \texttt{\$d}
        requirements but it also could miss possible legal assignments.

See also: \texttt{set search{\char`\_}limit}

\subsection{\texttt{save new\_proof} Command}\index{\texttt{save
new{\char`\_}proof} command}
Syntax:  \texttt{save new{\char`\_}proof} {\em label} [\texttt{/normal}]
   [\texttt{/compressed}]

The \texttt{save new{\char`\_}proof} command is available in the Proof
Assistant only.  It saves the proof in progress in the source
buffer\index{source buffer}.  \texttt{save new{\char`\_}proof} may be
used to save a completed proof, or it may be used to save a proof in
progress in order to work on it later.  If an incomplete proof is saved,
any user assignments with \texttt{let step} or \texttt{let variable}
will be lost, as will any ambiguous unifications\index{ambiguous
unification}\index{unification!ambiguous} that were resolved manually.
To help make recovery easier, it can be helpful to \texttt{improve all}
before \texttt{save new{\char`\_}proof} so that the incomplete proof
will have as much information as possible.

Note that the proof will not be permanently saved until a \texttt{write
source} command is issued.

Optional command qualifiers:

    \texttt{/normal} - The proof is saved in the normal format (i.e., as a
        sequence of labels, which is the defined format of the basic Metamath
        language).\index{basic language}  This is the default format that
        is used if a qualifier is omitted.

    \texttt{/compressed} - The proof is saved in the compressed format, which
        reduces storage requirements for a database.
        (See Appendix~\ref{compressed}.)


\section{Creating \LaTeX\ Output}\label{texout}\index{latex@{\LaTeX}}

You can generate \LaTeX\ output given the
information in a database.
The database must already include the necessary typesetting information
(see section \ref{tcomment} for how to provide this information).

The \texttt{show statement} and \texttt{show proof} commands each have a
special \texttt{/tex} command qualifier that produces \LaTeX\ output.
(The \texttt{show statement} command also has the
\texttt{/simple{\char`\_}tex} qualifier for output that is easier to
edit by hand.)  Before you can use them, you must open a \LaTeX\ file to
which to send their output.  A typical complete session will use this
sequence of Metamath commands:

\begin{verbatim}
read set.mm
open tex example.tex
show statement a1i /tex
show proof a1i /all/lemmon/renumber/tex
show statement uneq2 /tex
show proof uneq2 /all/lemmon/renumber/tex
close tex
\end{verbatim}

See Section~\ref{mathcomments} for information on comment markup and
Appendix~\ref{ASCII} for information on how math symbol translation is
specified.

To format and print the \LaTeX\ source, you will need the \LaTeX\
program, which is standard on most Linux installations and available for
Windows.  On Linux, in order to create a {\sc pdf} file, you will
typically type at the shell prompt
\begin{verbatim}
$ pdflatex example.tex
\end{verbatim}

\subsection{\texttt{open tex} Command}\index{\texttt{open tex} command}
Syntax:  \texttt{open tex} {\em file-name} [\texttt{/no{\char`\_}header}]

This command opens a file for writing \LaTeX\
source\index{latex@{\LaTeX}} and writes a \LaTeX\ header to the file.
\LaTeX\ source can be written with the \texttt{show proof}, \texttt{show
new{\char`\_}proof}, and \texttt{show statement} commands using the
\texttt{/tex} qualifier.

The mapping to \LaTeX\ symbols is defined in a special comment
containing a \texttt{\$t} token, described in Appendix~\ref{ASCII}.

There is an optional command qualifier:

    \texttt{/no{\char`\_}header} - This qualifier prevents a standard
        \LaTeX\ header and trailer
        from being included with the output \LaTeX\ code.


\subsection{\texttt{close tex} Command}\index{\texttt{close tex} command}
Syntax:  \texttt{close tex}

This command writes a trailer to any \LaTeX\ file\index{latex@{\LaTeX}}
that was opened with \texttt{open tex} (unless
\texttt{/no{\char`\_}header} was used with \texttt{open tex}) and closes
the \LaTeX\ file.


\section{Creating {\sc HTML} Output}\label{htmlout}

You can generate {\sc html} web pages given the
information in a database.
The database must already include the necessary typesetting information
(see section \ref{tcomment} for how to provide this information).
The ability to produce {\sc html} web pages was added in Metamath version
0.07.30.

To create an {\sc html} output file(s) for \texttt{\$a} or \texttt{\$p}
statement(s), use
\begin{quote}
    \texttt{show statement} {\em label-match} \texttt{/html}
\end{quote}
The output file will be named {\em label-match}\texttt{.html}
for each match.  When {\em
label-match} has wildcard (\texttt{*}) characters, all statements with
matching labels will have {\sc html} files produced for them.  Also,
when {\em label-match} has a wildcard (\texttt{*}) character, two additional
files, \texttt{mmdefinitions.html} and \texttt{mmascii.html} will be
produced.  To produce {\em only} these two additional files, you can use
\texttt{?*}, which will not match any statement label, in place of {\em
label-match}.

There are three other qualifiers for \texttt{show statement} that also
generate {\sc HTML} code.  These are \texttt{/alt{\char`\_}html},
\texttt{/brief{\char`\_}html}, and
\texttt{/brief{\char`\_}alt{\char`\_}html}, and are described in the
next section.

The command
\begin{quote}
    \texttt{show statement} {\em label-match} \texttt{/alt{\char`\_}html}
\end{quote}
does the same as \texttt{show statement} {\em label-match} \texttt{/html},
except that the {\sc html} code for the symbols is taken from
\texttt{althtmldef} statements instead of \texttt{htmldef} statements in
the \texttt{\$t} comment.

The command
\begin{verbatim}
show statement * /brief_html
\end{verbatim}
invokes a special mode that just produces definition and theorem lists
accompanied by their symbol strings, in a format suitable for copying and
pasting into another web page (such as the tutorial pages on the
Metamath web site).

Finally, the command
\begin{verbatim}
show statement * /brief_alt_html
\end{verbatim}
does the same as \texttt{show statement * / brief{\char`\_}html}
for the alternate {\sc html}
symbol representation.

A statement's comment can include a special notation that provides a
certain amount of control over the {\sc HTML} version of the comment.  See
Section~\ref{mathcomments} (p.~\pageref{mathcomments}) for the comment
markup features.

The \texttt{write theorem{\char`\_}list} and \texttt{write bibliography}
commands, which are described below, provide as a side effect complete
error checking for all of the features described in this
section.\index{error checking}

\subsection{\texttt{write theorem\_list}
Command}\index{\texttt{write theorem{\char`\_}list} command}
Syntax:  \texttt{write theorem{\char`\_}list}
[\texttt{/theorems{\char`\_}per{\char`\_}page} {\em number}]

This command writes a list of all of the \texttt{\$a} and \texttt{\$p}
statements in the database into a web page file
 called \texttt{mmtheorems.html}.
When additional files are needed, they are called
\texttt{mmtheorems2.html}, \texttt{mmtheorems3.html}, etc.

Optional command qualifier:

    \texttt{/theorems{\char`\_}per{\char`\_}page} {\em number} -
 This qualifier specifies the number of statements to
        write per web page.  The default is 100.

{\em Note:} In version 0.07.30\index{Metamath!limitations of version
0.07.30} of Metamath, the ``Nearby theorems'' links on the individual
web pages presuppose 100 theorems per page when linking to the theorem
list pages.  Therefore the \texttt{/theorems{\char`\_}per{\char`\_}page}
qualifier, if it specifies a number other than 100, will cause the
individual web pages to be out of sync and should not be used to
generate the main theorem list for the web site.  This is expected to be
fixed in a future version.


\subsection{\texttt{write bibliography}\label{wrbib}
Command}\index{\texttt{write bibliography} command}
Syntax:  \texttt{write bibliography} {\em filename}

This command reads an existing {\sc html} bibliographic cross-reference
file, normally called \texttt{mmbiblio.html}, and updates it per the
bibliographic links in the database comments.  The file is updated
between the {\sc html} comment lines \texttt{<!--
{\char`\#}START{\char`\#} -->} and \texttt{<!-- {\char`\#}END{\char`\#}
-->}.  The original input file is renamed to {\em
filename}\texttt{{\char`\~}1}.

A bibliographic reference is indicated with the reference name
in brackets, such as  \texttt{Theorem 3.1 of
[Monk] p.\ 22}.
See Section~\ref{htmlout} (p.~\pageref{htmlout}) for
syntax details.


\subsection{\texttt{write recent\_additions}
Command}\index{\texttt{write recent{\char`\_}additions} command}
Syntax:  \texttt{write recent{\char`\_}additions} {\em filename}
[\texttt{/limit} {\em number}]

This command reads an existing ``Recent Additions'' {\sc html} file,
normally called \texttt{mmrecent.html}, and updates it with the
descriptions of the most recently added theorems to the database.
 The file is updated between
the {\sc html} comment lines \texttt{<!-- {\char`\#}START{\char`\#} -->}
and \texttt{<!-- {\char`\#}END{\char`\#} -->}.  The original input file
is renamed to {\em filename}\texttt{{\char`\~}1}.

Optional command qualifier:

    \texttt{/limit} {\em number} -
 This qualifier specifies the number of most recent theorems to
   write to the output file.  The default is 100.


\section{Text File Utilities}

\subsection{\texttt{tools} Command}\index{\texttt{tools} command}
Syntax:  \texttt{tools}

This command invokes an easy-to-use, general purpose utility for
manipulating the contents of {\sc ascii} text files.  Upon typing
\texttt{tools}, the command-line prompt will change to \texttt{TOOLS>}
until you type \texttt{exit}.  The \texttt{tools} commands can be used
to perform simple, global edits on an input/output file,
such as making a character string substitution on each line, adding a
string to each line, and so on.  A typical use of this utility is
to build a \texttt{submit} input file to perform a common operation on a
list of statements obtained from \texttt{show label} or \texttt{show
usage}.

The actions of most of the \texttt{tools} commands can also be
performed with equivalent (and more powerful) Unix shell commands, and
some users may find those more efficient.  But for Windows users or
users not comfortable with Unix, \texttt{tools} provides an
easy-to-learn alternative that is adequate for most of the
script-building tasks needed to use the Metamath program effectively.

\subsection{\texttt{help} Command (in \texttt{tools})}
Syntax:  \texttt{help}

The \texttt{help} command lists the commands available in the
\texttt{tools} utility, along with a brief description.  Each command,
in turn, has its own help, such as \texttt{help add}.  As with
Metamath's \texttt{MM>} prompt, a complete command can be entered at
once, or just the command word can be typed, causing the program to
prompt for each argument.

\vskip 1ex
\noindent Line-by-line editing commands:

  \texttt{add} - Add a specified string to each line in a file.

  \texttt{clean} - Trim spaces and tabs on each line in a file; convert
         characters.

  \texttt{delete} - Delete a section of each line in a file.

  \texttt{insert} - Insert a string at a specified column in each line of
        a file.

  \texttt{substitute} - Make a simple substitution on each line of the file.

  \texttt{tag} - Like \texttt{add}, but restricted to a range of lines.

  \texttt{swap} - Swap the two halves of each line in a file.

\vskip 1ex
\noindent Other file-processing commands:

  \texttt{break} - Break up (tokenize) a file into a list of tokens (one per
        line).

  \texttt{build} - Build a file with multiple tokens per line from a list.

  \texttt{count} - Count the occurrences in a file of a specified string.

  \texttt{number} - Create a list of numbers.

  \texttt{parallel} - Put two files in parallel.

  \texttt{reverse} - Reverse the order of the lines in a file.

  \texttt{right} - Right-justify lines in a file (useful before sorting
         numbers).

%  \texttt{tag} - Tag edit updates in a program for revision control.

  \texttt{sort} - Sort the lines in a file with key starting at
         specified string.

  \texttt{match} - Extract lines containing (or not) a specified string.

  \texttt{unduplicate} - Eliminate duplicate occurrences of lines in a file.

  \texttt{duplicate} - Extract first occurrence of any line occurring
         more than

   \ \ \    once in a file, discarding lines occurring exactly once.

  \texttt{unique} - Extract lines occurring exactly once in a file.

  \texttt{type} (10 lines) - Display the first few lines in a file.
                  Similar to Unix \texttt{head}.

  \texttt{copy} - Similar to Unix \texttt{cat} but safe (same input
         and output file allowed).

  \texttt{submit} - Run a script containing \texttt{tools} commands.

\vskip 1ex

\noindent Note:
  \texttt{unduplicate}, \texttt{duplicate}, and \texttt{unique} also
 sort the lines as a side effect.


\subsection{Using \texttt{tools} to Build Metamath \texttt{submit}
Scripts}

The \texttt{break} command is typically used to break up a series of
statement labels, such as the output of Metamath's \texttt{show usage},
into one label per line.  The other \texttt{tools} commands can then be
used to add strings before and after each statement label to specify
commands to be performed on the statement.  The \texttt{parallel}
command is useful when a statement label must be mentioned more than
once on a line.

Very often a \texttt{submit} script for Metamath will require multiple
command lines for each statement being processed.  For example, you may
want to enter the Proof Assistant, \texttt{minimize{\char`\_}with} your
latest theorem, \texttt{save} the new proof, and \texttt{exit} the Proof
Assistant.  To accomplish this, you can build a file with these four
commands for each statement on a single line, separating each command
with a designated character such as \texttt{@}.  Then at the end you can
\texttt{substitute} each \texttt{@} with \texttt{{\char`\\}n} to break
up the lines into individual command lines (see \texttt{help
substitute}).


\subsection{Example of a \texttt{tools} Session}

To give you a quick feel for the \texttt{tools} utility, we show a
simple session where we create a file \texttt{n.txt} with 3 lines, add
strings before and after each line, and display the lines on the screen.
You can experiment with the various commands to gain experience with the
\texttt{tools} utility.

\begin{verbatim}
MM> tools
Entering the Text Tools utilities.
Type HELP for help, EXIT to exit.
TOOLS> number
Output file <n.tmp>? n.txt
First number <1>?
Last number <10>? 3
Increment <1>?
TOOLS> add
Input/output file? n.txt
String to add to beginning of each line <>? This is line
String to add to end of each line <>? .
The file n.txt has 3 lines; 3 were changed.
First change is on line 1:
This is line 1.
TOOLS> type n.txt
This is line 1.
This is line 2.
This is line 3.
TOOLS> exit
Exiting the Text Tools.
Type EXIT again to exit Metamath.
MM>
\end{verbatim}



\appendix
\chapter{Sample Representations}
\label{ASCII}

This Appendix provides a sample of {\sc ASCII} representations,
their corresponding traditional mathematical symbols,
and a discussion of their meanings
in the \texttt{set.mm} database.
These are provided in order of appearance.
This is only a partial list, and new definitions are routinely added.
A complete list is available at \url{http://metamath.org}.

These {\sc ASCII} representations, along
with information on how to display them,
are defined in the \texttt{set.mm} database file inside
a special comment called a \texttt{\$t} {\em
comment}\index{\texttt{\$t} comment} or {\em typesetting
comment.}\index{typesetting comment}
A typesetting comment
is indicated by the appearance of the
two-character string \texttt{\$t} at the beginning of the comment.
For more information,
see Section~\ref{tcomment}, p.~\pageref{tcomment}.

In the following table the ``{\sc ASCII}'' column shows the {\sc ASCII}
representation,
``Symbol'' shows the mathematical symbolic display
that corresponds to that {\sc ASCII} representation, ``Labels'' shows
the key label(s) that define the representation, and
``Description'' provides a description about the symbol.
As usual, ``iff'' is short for ``if and only if.''\index{iff}
In most cases the ``{\sc ASCII}'' column only shows
the key token, but it will sometimes show a sequence of tokens
if that is necessary for clarity.

{\setlength{\extrarowsep}{4pt} % Keep rows from being too close together
\begin{longtabu}   { @{} c c l X }
\textbf{ASCII} & \textbf{Symbol} & \textbf{Labels} & \textbf{Description} \\
\endhead
\texttt{|-} & $\vdash$ & &
  ``It is provable that...'' \\
\texttt{ph} & $\varphi$ & \texttt{wph} &
  The wff (boolean) variable phi,
  conventionally the first wff variable. \\
\texttt{ps} & $\psi$ & \texttt{wps} &
  The wff (boolean) variable psi,
  conventionally the second wff variable. \\
\texttt{ch} & $\chi$ & \texttt{wch} &
  The wff (boolean) variable chi,
  conventionally the third wff variable. \\
\texttt{-.} & $\lnot$ & \texttt{wn} &
  Logical not. E.g., if $\varphi$ is true, then $\lnot \varphi$ is false. \\
\texttt{->} & $\rightarrow$ & \texttt{wi} &
  Implies, also known as material implication.
  In classical logic the expression $\varphi \rightarrow \psi$ is true
  if either $\varphi$ is false or $\psi$ is true (or both), that is,
  $\varphi \rightarrow \psi$ has the same meaning as
  $\lnot \varphi \lor \psi$ (as proven in theorem \texttt{imor}). \\
\texttt{<->} & $\leftrightarrow$ &
  \hyperref[df-bi]{\texttt{df-bi}} &
  Biconditional (aka is-equals for boolean values).
  $\varphi \leftrightarrow \psi$ is true iff
  $\varphi$ and $\psi$ have the same value. \\
\texttt{\char`\\/} & $\lor$ &
  \makecell[tl]{{\hyperref[df-or]{\texttt{df-or}}}, \\
	         \hyperref[df-3or]{\texttt{df-3or}}} &
  Disjunction (logical ``or''). $\varphi \lor \psi$ is true iff
  $\varphi$, $\psi$, or both are true. \\
\texttt{/\char`\\} & $\land$ &
  \makecell[tl]{{\hyperref[df-an]{\texttt{df-an}}}, \\
                 \hyperref[df-3an]{\texttt{df-3an}}} &
  Conjunction (logical ``and''). $\varphi \land \psi$ is true iff
  both $\varphi$ and $\psi$ are true. \\
\texttt{A.} & $\forall$ &
  \texttt{wal} &
  For all; the wff $\forall x \varphi$ is true iff
  $\varphi$ is true for all values of $x$. \\
\texttt{E.} & $\exists$ &
  \hyperref[df-ex]{\texttt{df-ex}} &
  There exists; the wff
  $\exists x \varphi$ is true iff
  there is at least one $x$ where $\varphi$ is true. \\
\texttt{[ y / x ]} & $[ y / x ]$ &
  \hyperref[df-sb]{\texttt{df-sb}} &
  The wff $[ y / x ] \varphi$ produces
  the result when $y$ is properly substituted for $x$ in $\varphi$
  ($y$ replaces $x$).
  % This is elsb4
  % ( [ x / y ] z e. y <-> z e. x )
  For example,
  $[ x / y ] z \in y$ is the same as $z \in x$. \\
\texttt{E!} & $\exists !$ &
  \hyperref[df-eu]{\texttt{df-eu}} &
  There exists exactly one;
  $\exists ! x \varphi$ is true iff
  there is at least one $x$ where $\varphi$ is true. \\
\texttt{\{ y | phi \}}  & $ \{ y | \varphi \}$ &
  \hyperref[df-clab]{\texttt{df-clab}} &
  The class of all sets where $\varphi$ is true. \\
\texttt{=} & $ = $ &
  \hyperref[df-cleq]{\texttt{df-cleq}} &
  Class equality; $A = B$ iff $A$ equals $B$. \\
\texttt{e.} & $ \in $ &
  \hyperref[df-clel]{\texttt{df-clel}} &
  Class membership; $A \in B$ if $A$ is a member of $B$. \\
\texttt{{\char`\_}V} & {\rm V} &
  \hyperref[df-v]{\texttt{df-v}} &
  Class of all sets (not itself a set). \\
\texttt{C\_} & $ \subseteq $ &
  \hyperref[df-ss]{\texttt{df-ss}} &
  Subclass (subset); $A \subseteq B$ is true iff
  $A$ is a subclass of $B$. \\
\texttt{u.} & $ \cup $ &
  \hyperref[df-un]{\texttt{df-un}} &
  $A \cup B$ is the union of classes $A$ and $B$. \\
\texttt{i^i} & $ \cap $ &
  \hyperref[df-in]{\texttt{df-in}} &
  $A \cap B$ is the intersection of classes $A$ and $B$. \\
\texttt{\char`\\} & $ \setminus $ &
  \hyperref[df-dif]{\texttt{df-dif}} &
  $A \setminus B$ (set difference)
  is the class of all sets in $A$ except for those in $B$. \\
\texttt{(/)} & $ \varnothing $ &
  \hyperref[df-nul]{\texttt{df-nul}} &
  $ \varnothing $ is the empty set (aka null set). \\
\texttt{\char`\~P} & $ \cal P $ &
  \hyperref[df-pw]{\texttt{df-pw}} &
  Power class. \\
\texttt{<.\ A , B >.} & $\langle A , B \rangle$ &
  \hyperref[df-op]{\texttt{df-op}} &
  The ordered pair $\langle A , B \rangle$. \\
\texttt{( F ` A )} & $ ( F ` A ) $ &
  \hyperref[df-fv]{\texttt{df-fv}} &
  The value of function $F$ when applied to $A$. \\
\texttt{\_i} & $ i $ &
  \texttt{df-i} &
  The square root of negative one. \\
\texttt{x.} & $ \cdot $ &
  \texttt{df-mul} &
  Complex number multiplication; $2~\cdot~3~=~6$. \\
\texttt{CC} & $ \mathbb{C} $ &
  \texttt{df-c} &
  The set of complex numbers. \\
\texttt{RR} & $ \mathbb{R} $ &
  \texttt{df-r} &
  The set of real numbers. \\
\end{longtabu}
} % end of extrarowsep

\chapter{Compressed Proofs}
\label{compressed}\index{compressed proof}\index{proof!compressed}

The proofs in the \texttt{set.mm} set theory database are stored in compressed
format for efficiency.  Normally you needn't concern yourself with the
compressed format, since you can display it with the usual proof display tools
in the Metamath program (\texttt{show proof}\ldots) or convert it to the normal
RPN proof format described in Section~\ref{proof} (with \texttt{save proof}
{\em label} \texttt{/normal}).  However for sake of completeness we describe the
format here and show how it maps to the normal RPN proof format.

A compressed proof, located between \texttt{\$=} and \texttt{\$.}\ keywords, consists
of a left parenthesis, a sequence of statement labels, a right parenthesis,
and a sequence of upper-case letters \texttt{A} through \texttt{Z} (with optional
white space between them).  White space must surround the parentheses
and the labels.  The left parenthesis tells Metamath that a
compressed proof follows.  (A normal RPN proof consists of just a sequence of
labels, and a parenthesis is not a legal character in a label.)

The sequence of upper-case letters corresponds to a sequence of integers
with the following mapping.  Each integer corresponds to a proof step as
described later.
\begin{center}
  \texttt{A} = 1 \\
  \texttt{B} = 2 \\
   \ldots \\
  \texttt{T} = 20 \\
  \texttt{UA} = 21 \\
  \texttt{UB} = 22 \\
   \ldots \\
  \texttt{UT} = 40 \\
  \texttt{VA} = 41 \\
  \texttt{VB} = 42 \\
   \ldots \\
  \texttt{YT} = 120 \\
  \texttt{UUA} = 121 \\
   \ldots \\
  \texttt{YYT} = 620 \\
  \texttt{UUUA} = 621 \\
   etc.
\end{center}

In other words, \texttt{A} through \texttt{T} represent the
least-significant digit in base 20, and \texttt{U} through \texttt{Y}
represent zero or more most-significant digits in base 5, where the
digits start counting at 1 instead of the usual 0. With this scheme, we
don't need white space between these ``numbers.''

(In the design of the compressed proof format, only upper-case letters,
as opposed to say all non-whitespace printable {\sc ascii} characters other than
%\texttt{\$}, was chosen to make the compressed proof a little less
%displeasing to the eye, at the expense of a typical 20\% compression
\texttt{\$}, were chosen so as not to collide with most text editor
searches, at the expense of a typical 20\% compression
loss.  The base 5/base 20 grouping, as opposed to say base 6/base 19,
was chosen by experimentally determining the grouping that resulted in
best typical compression.)

The letter \texttt{Z} identifies (tags) a proof step that is identical to one
that occurs later on in the proof; it helps shorten the proof by not requiring
that identical proof steps be proved over and over again (which happens often
when building wff's).  The \texttt{Z} is placed immediately after the
least-significant digit (letters \texttt{A} through \texttt{T}) that ends the integer
corresponding to the step to later be referenced.

The integers that the upper-case letters correspond to are mapped to labels as
follows.  If the statement being proved has $m$ mandatory hypotheses, integers
1 through $m$ correspond to the labels of these hypotheses in the order shown
by the \texttt{show statement ... / full} command, i.e., the RPN order\index{RPN
order} of the mandatory
hypotheses.  Integers $m+1$ through $m+n$ correspond to the labels enclosed in
the parentheses of the compressed proof, in the order that they appear, where
$n$ is the number of those labels.  Integers $m+n+1$ on up don't directly
correspond to statement labels but point to proof steps identified with the
letter \texttt{Z}, so that these proof steps can be referenced later in the
proof.  Integer $m+n+1$ corresponds to the first step tagged with a \texttt{Z},
$m+n+2$ to the second step tagged with a \texttt{Z}, etc.  When the compressed
proof is converted to a normal proof, the entire subproof of a step tagged
with \texttt{Z} replaces the reference to that step.

For efficiency, Metamath works with compressed proofs directly, without
converting them internally to normal proofs.  In addition to the usual
error-checking, an error message is given if (1) a label in the label list in
parentheses does not refer to a previous \texttt{\$p} or \texttt{\$a} statement or a
non-mandatory hypothesis of the statement being proved and (2) a proof step
tagged with \texttt{Z} is referenced before the step tagged with the \texttt{Z}.

Just as in a normal proof under development (Section~\ref{unknown}), any step
or subproof that is not yet known may be represented with a single \texttt{?}.
White space does not have to appear between the \texttt{?}\ and the upper-case
letters (or other \texttt{?}'s) representing the remainder of the proof.

% April 1, 2004 Appendix C has been added back in with corrections.
%
% May 20, 2003 Appendix C was removed for now because there was a problem found
% by Bob Solovay
%
% Also, removed earlier \ref{formalspec} 's (3 cases above)
%
% Bob Solovay wrote on 30 Nov 2002:
%%%%%%%%%%%%% (start of email comment )
%      3. My next set of comments concern appendix C. I read this before I
% read Chapter 4. So I first noted that the system as presented in the
% Appendix lacked a certain formal property that I thought desirable. I
% then came up with a revised formal system that had this property. Upon
% reading Chapter 4, I noticed that the revised system was closer to the
% treatment in Chapter 4 than the system in Appendix C.
%
%         First a very minor correction:
%
%         On page 142 line 2: The condition that V(e) != V(f) should only be
% required of e, f in T such that e != f.
%
%         Here is a natural property [transitivity] that one would like
% the formal system to have:
%
%         Let Gamma be a set of statements. Suppose that the statement Phi
% is provable from Gamma and that the statement Psi is provable from Gamma
% \cup {Phi}. Then Psi is provable from Gamma.
%
%         I shall present an example to show that this property does not
% hold for the formal systems of Appendix C:
%
%         I write the example in metamath style:
%
% $c A B C D E $.
% $v x y
%
% ${
% tx $f A x $.
% ty $f B y $.
% ax1 $a C x y $.
% $}
%
% ${
% tx $f A x $.
% ty $f B y $.
% ax2-h1 $e C x y $.
% ax2 $a D y $.
% $}
%
% ${
% ty $f B y $.
% ax3-h1 $e D y $.
% ax3 $a E y $.
% $}
%
% $(These three axioms are Gamma $)
%
% ${
% tx $f A x $.
% ty $f B y $.
% Phi $p D y $=
% tx ty tx ty ax1 ax2 $.
% $}
%
% ${
% ty $f B y $.
% Psi $p E y $=
% ty ty Phi ax3 $.
% $}
%
%
% I omit the formal proofs of the following claims. [I will be glad to
% supply them upon request.]
%
% 1) Psi is not provable from Gamma;
%
% 2) Psi is provable from Gamma + Phi.
%
% Here "provable" refers to the formalism of Appendix C.
%
% The trouble of course is that Psi is lacking the variable declaration
%
% $f Ax $.
%
% In the Metamath system there is no trouble proving Psi. I attach a
% metamath file that shows this and which has been checked by the
% metamath program.
%
% I next want to indicate how I think the treatment in Appendix C should
% be revised so as to conform more closely to the metamath system of the
% main text. The revised system *does* have the transitivity property.
%
% We want to give revised definitions of "statement" and
% "provable". [cf. sections C.2.4. and C.2.5] Our new definitions will
% use the definitions given in Appendix C. So we take the following
% tack. We refer to the original notions as o-statement and o-provable. And
% we refer to the notions we are defining as n-statement and n-provable.
%
%         A n-statement is an o-statement in which the only variables
% that appear in the T component are mandatory.
%
%         To any o-statement we can associate its reduct which is a
% n-statement by dropping all the elements of T or D which contain
% non-mandatory variables.
%
%         An n-statement gamma is n-provable if there is an o-statement
% gamma' which has gamma as its reduct andf such that gamma' is
% o-provable.
%
%         It seems to me [though I am not completely sure on this point]
% that n-provability corresponds to metamath provability as discussed
% say in Chapter 4.
%
%         Attached to this letter is the metamath proof of Phi and Psi
% from Gamma discussed above.
%
%         I am still brooding over the question of whether metamath
% correctly formalizes set-theory. No doubt I will have some questions
% re this after my thoughts become clearer.
%%%%%%%%%%%%%%%% (end of email comment)

%%%%%%%%%%%%%%%% (start of 2nd email comment from Bob Solovay 1-Apr-04)
%
%         I hope that Appendix C is the one that gives a "formal" treatment
% of Metamath. At any rate, thats the appendix I want to comment on.
%
%         I'm going to suggest two changes in the definition.
%
%         First change (in the definition of statement): Require that the
% sets D, T, and E be finite.
%
%         Probably things are fine as you give them. But in the applications
% to the main metamath system they will always be finite, and its useful in
% thinking about things [at least for me] to stick to the finite case.
%
%         Second change:
%
%         First let me give an approximate description. Remove the dummy
% variables from the statement. Instead, include them in the proof.
%
%         More formally: Require that T consists of type declarations only
% for mandatory variables. Require that all the pairs in D consist of
% mandatory variables.
%
%         At the start of a proof we are allowed to declare a finite number
% of dummy variables [provided that none of them appear in any of the
% statements in E \cup {A}. We have to supply type declarations for all the
% dummy variables. We are allowed to add new $d statements referring to
% either the mandatory or dummy variables. But we require that no new $d
% statement references only mandatory variables.
%
%         I find this way of doing things more conceptual than the treatment
% in Appendix C. But the change [which I will use implicitly in later
% letters about doing Peano] is mainly aesthetic. I definitely claim that my
% results on doing Peano all apply to Metamath as it is presented in your
% book.
%
%         --Bob
%
%%%%%%%%%%%%%%%% (end of 2nd email comment)

%%
%% When uncommenting the below, also uncomment references above to {formalspec}
%%
\chapter{Metamath's Formal System}\label{formalspec}\index{Metamath!as a formal
system}

\section{Introduction}

\begin{quote}
  {\em Perfection is when there is no longer anything more to take away.}
    \flushright\sc Antoine de
     Saint-Exupery\footnote{\cite[p.~3-25]{Campbell}.}\\
\end{quote}\index{de Saint-Exupery, Antoine}

This appendix describes the theory behind the Metamath language in an abstract
way intended for mathematicians.  Specifically, we construct two
set-theo\-ret\-i\-cal objects:  a ``formal system'' (roughly, a set of syntax
rules, axioms, and logical rules) and its ``universe'' (roughly, the set of
theorems derivable in the formal system).  The Metamath computer language
provides us with a way to describe specific formal systems and, with the aid of
a proof provided by the user, to verify that given theorems
belong to their universes.

To understand this appendix, you need a basic knowledge of informal set theory.
It should be sufficient to understand, for example, Ch.\ 1 of Munkres' {\em
Topology} \cite{Munkres}\index{Munkres, James R.} or the
introductory set theory chapter
in many textbooks that introduce abstract mathematics. (Note that there are
minor notational differences among authors; e.g.\ Munkres uses $\subset$ instead
of our $\subseteq$ for ``subset.''  We use ``included in'' to mean ``a subset
of,'' and ``belongs to'' or ``is contained in'' to mean ``is an element of.'')
What we call a ``formal'' description here, unlike earlier, is actually an
informal description in the ordinary language of mathematicians.  However we
provide sufficient detail so that a mathematician could easily formalize it,
even in the language of Metamath itself if desired.  To understand the logic
examples at the end of this appendix, familiarity with an introductory book on
mathematical logic would be helpful.

\section{The Formal Description}

\subsection[Preliminaries]{Preliminaries\protect\footnotemark}%
\footnotetext{This section is taken mostly verbatim
from Tarski \cite[p.~63]{Tarski1965}\index{Tarski, Alfred}.}

By $\omega$ we denote the set of all natural numbers (non-negative integers).
Each natural number $n$ is identified with the set of all smaller numbers: $n =
\{ m | m < n \}$.  The formula $m < n$ is thus equivalent to the condition: $m
\in n$ and $m,n \in \omega$. In particular, 0 is the number zero and at the
same time the empty set $\varnothing$, $1=\{0\}$, $2=\{0,1\}$, etc. ${}^B A$
denotes the set of all functions on $B$ to $A$ (i.e.\ with domain $B$ and range
included in $A$).  The members of ${}^\omega A$ are what are called {\em simple
infinite sequences},\index{simple infinite sequence}
with all {\em terms}\index{term} in $A$.  In case $n \in \omega$, the
members of ${}^n A$ are referred to as {\em finite $n$-termed
sequences},\index{finite $n$-termed
sequence} again
with terms in $A$.  The consecutive terms (function values) of a finite or
infinite sequence $f$ are denoted by $f_0, f_1, \ldots ,f_n,\ldots$.  Every
finite sequence $f \in \bigcup _{n \in \omega} {}^n A$ uniquely determines the
number $n$ such that $f \in {}^n A$; $n$ is called the {\em
length}\index{length of a sequence ({$"|\ "|$})} of $f$ and
is denoted by $|f|$.  $\langle a \rangle$ is the sequence $f$ with $|f|=1$ and
$f_0=a$; $\langle a,b \rangle$ is the sequence $f$ with $|f|=2$, $f_0=a$,
$f_1=b$; etc.  Given two finite sequences $f$ and $g$, we denote by $f\frown g$
their {\em concatenation},\index{concatenation} i.e., the
finite sequence $h$ determined by the
conditions:
\begin{eqnarray*}
& |h| = |f|+|g|;&  \\
& h_n = f_n & \mbox{\ for\ } n < |f|;  \\
& h_{|f|+n} = g_n & \mbox{\ for\ } n < |g|.
\end{eqnarray*}

\subsection{Constants, Variables, and Expressions}

A formal system has a set of {\em symbols}\index{symbol!in
a formal system} denoted
by $\mbox{\em SM}$.  A
precise set-theo\-ret\-i\-cal definition of this set is unimportant; a symbol
could be considered a primitive or atomic element if we wish.  We assume this
set is divided into two disjoint subsets:  a set $\mbox{\em CN}$ of {\em
constants}\index{constant!in a formal system} and a set $\mbox{\em VR}$ of
{\em variables}.\index{variable!in a formal system}  $\mbox{\em CN}$ and
$\mbox{\em VR}$ are each assumed to consist of countably many symbols which
may be arranged in finite or simple infinite sequences $c_0, c_1, \ldots$ and
$v_0, v_1, \ldots$ respectively, without repeating terms.  We will represent
arbitrary symbols by metavariables $\alpha$, $\beta$, etc.

{\footnotesize\begin{quotation}
{\em Comment.} The variables $v_0, v_1, \ldots$ of our formal system
correspond to what are usually considered ``metavariables'' in
descriptions of specific formal systems in the literature.  Typically,
when describing a specific formal system a book will postulate a set of
primitive objects called variables, then proceed to describe their
properties using metavariables that range over them, never mentioning
again the actual variables themselves.  Our formal system does not
mention these primitive variable objects at all but deals directly with
metavariables, as its primitive objects, from the start.  This is a
subtle but key distinction you should keep in mind, and it makes our
definition of ``formal system'' somewhat different from that typically
found in the literature.  (So, the $\alpha$, $\beta$, etc.\ above are
actually ``metametavariables'' when used to represent $v_0, v_1,
\ldots$.)
\end{quotation}}

Finite sequences all terms of which are symbols are called {\em
expressions}.\index{expression!in a formal system}  $\mbox{\em EX}$ is
the set of all expressions; thus
\begin{displaymath}
\mbox{\em EX} = \bigcup _{n \in \omega} {}^n \mbox{\em SM}.
\end{displaymath}

A {\em constant-prefixed expression}\index{constant-prefixed expression}
is an expression of non-zero length
whose first term is a constant.  We denote the set of all constant-prefixed
expressions by $\mbox{\em EX}_C = \{ e \in \mbox{\em EX} | ( |e| > 0 \wedge
e_0 \in \mbox{\em CN} ) \}$.

A {\em constant-variable pair}\index{constant-variable pair}
is an expression of length 2 whose first term
is a constant and whose second term is a variable.  We denote the set of all
constant-variable pairs by $\mbox{\em EX}_2 = \{ e \in \mbox{\em EX}_C | ( |e|
= 2 \wedge e_1 \in \mbox{\em VR} ) \}$.


{\footnotesize\begin{quotation}
{\em Relationship to Metamath.} In general, the set $\mbox{\em SM}$
corresponds to the set of declared math symbols in a Metamath database, the
set $\mbox{\em CN}$ to those declared with \texttt{\$c} statements, and the set
$\mbox{\em VR}$ to those declared with \texttt{\$v} statements.  Of course a
Metamath database can only have a finite number of math symbols, whereas
formal systems in general can have an infinite number, although the number of
Metamath math symbols available is in principle unlimited.

The set $\mbox{\em EX}_C$ corresponds to the set of permissible expressions
for \texttt{\$e}, \texttt{\$a}, and \texttt{\$p} statements.  The set $\mbox{\em EX}_2$
corresponds to the set of permissible expressions for \texttt{\$f} statements.
\end{quotation}}

We denote by ${\cal V}(e)$ the set of all variables in an expression $e \in
\mbox{\em EX}$, i.e.\ the set of all $\alpha \in \mbox{\em VR}$ such that
$\alpha = e_n$ for some $n < |e|$.  We also denote (with abuse of notation) by
${\cal V}(E)$ the set of all variables in a collection of expressions $E
\subseteq \mbox{\em EX}$, i.e.\ $\bigcup _{e \in E} {\cal V}(e)$.


\subsection{Substitution}

Given a function $F$ from $\mbox{\em VR}$ to
$\mbox{\em EX}$, we
denote by $\sigma_{F}$ or just $\sigma$ the function from $\mbox{\em EX}$ to
$\mbox{\em EX}$ defined recursively for nonempty sequences by
\begin{eqnarray*}
& \sigma(<\alpha>) = F(\alpha) & \mbox{for\ } \alpha \in \mbox{\em VR}; \\
& \sigma(<\alpha>) = <\alpha> & \mbox{for\ } \alpha \not\in \mbox{\em VR}; \\
& \sigma(g \frown h) = \sigma(g) \frown
    \sigma(h) & \mbox{for\ } g,h \in \mbox{\em EX}.
\end{eqnarray*}
We also define $\sigma(\varnothing)=\varnothing$.  We call $\sigma$ a {\em
simultaneous substitution}\index{substitution!variable}\index{variable
substitution} (or just {\em substitution}) with {\em substitution
map}\index{substitution map} $F$.

We also denote (with abuse of notation) by $\sigma(E)$ a substitution on a
collection of expressions $E \subseteq \mbox{\em EX}$, i.e.\ the set $\{
\sigma(e) | e \in E \}$.  The collection $\sigma(E)$ may of course contain
fewer expressions than $E$ because duplicate expressions could result from the
substitution.

\subsection{Statements}

We denote by $\mbox{\em DV}$ the set of all
unordered pairs $\{\alpha, \beta \} \subseteq \mbox{\em VR}$ such that $\alpha
\neq \beta$.  $\mbox{\em DV}$ stands for ``distinct variables.''

A {\em pre-statement}\index{pre-statement!in a formal system} is a
quadruple $\langle D,T,H,A \rangle$ such that
$D\subseteq \mbox{\em DV}$, $T\subseteq \mbox{\em EX}_2$, $H\subseteq
\mbox{\em EX}_C$ and $H$ is finite,
$A\in \mbox{\em EX}_C$, ${\cal V}(H\cup\{A\}) \subseteq
{\cal V}(T)$, and $\forall e,f\in T {\ } {\cal V}(e) \neq {\cal V}(f)$ (or
equivalently, $e_1 \ne f_1$) whenever $e \neq f$. The terms of the quadruple are called {\em
distinct-variable restrictions},\index{disjoint-variable restriction!in a
formal system} {\em variable-type hypotheses},\index{variable-type
hypothesis!in a formal system} {\em logical hypotheses},\index{logical
hypothesis!in a formal system} and the {\em assertion}\index{assertion!in a
formal system} respectively.  We denote by $T_M$ ({\em mandatory variable-type
hypotheses}\index{mandatory variable-type hypothesis!in a formal system}) the
subset of $T$ such that ${\cal V}(T_M) ={\cal V}(H \cup \{A\})$.  We denote by
$D_M=\{\{\alpha,\beta\}\in D|\{\alpha,\beta\}\subseteq {\cal V}(T_M)\}$ the
{\em mandatory distinct-variable restrictions}\index{mandatory
disjoint-variable restriction!in a formal system} of the pre-statement.
The set
of {\em mandatory hypotheses}\index{mandatory hypothesis!in a formal system}
is $T_M\cup H$.  We call the quadruple $\langle D_M,T_M,H,A \rangle$
the {\em reduct}\index{reduct!in a formal system} of
the pre-statement $\langle D,T,H,A \rangle$.

A {\em statement} is the reduct of some pre-statement\index{statement!in a
formal system}.  A statement is therefore a special kind of pre-statement;
in particular, a statement is the reduct of itself.

{\footnotesize\begin{quotation}
{\em Comment.}  $T$ is a set of expressions, each of length 2, that associate
a set of constants (``variable types'') with a set of variables.  The
condition ${\cal V}(H\cup\{A\}) \subseteq {\cal V}(T) $
means that each variable occurring in a statement's logical
hypotheses or assertion must have an associated variable-type hypothesis or
``type declaration,'' in  analogy to a computer programming language, where a
variable must be declared to be say, a string or an integer.  The requirement
that $\forall e,f\in T \, e_1 \ne f_1$ for $e\neq f$
means that each variable must be
associated with a unique constant designating its variable type; e.g., a
variable might be a ``wff'' or a ``set'' but not both.

Distinct-variable restrictions are used to specify what variable substitutions
are permissible to make for the statement to remain valid.  For example, in
the theorem scheme of set theory $\lnot\forall x\,x=y$ we may not substitute
the same variable for both $x$ and $y$.  On the other hand, the theorem scheme
$x=y\to y=x$ does not require that $x$ and $y$ be distinct, so we do not
require a distinct-variable restriction, although having one
would cause no harm other than making the scheme less general.

A mandatory variable-type hypothesis is one whose variable exists in a logical
hypothesis or the assertion.  A provable pre-statement
(defined below) may require
non-mandatory variable-type hypotheses that effectively introduce ``dummy''
variables for use in its proof.  Any number of dummy variables might
be required by a specific proof; indeed, it has been shown by H.\
Andr\'{e}ka\index{Andr{\'{e}}ka, H.} \cite{Nemeti} that there is no finite
upper bound to the number of dummy variables needed to prove an arbitrary
theorem in first-order logic (with equality) having a fixed number $n>2$ of
individual variables.  (See also the Comment on p.~\pageref{nodd}.)
For this reason we do not set a finite size bound on the collections $D$ and
$T$, although in an actual application (Metamath database) these will of
course be finite, increased to whatever size is necessary as more
proofs are added.
\end{quotation}}

{\footnotesize\begin{quotation}
{\em Relationship to Metamath.} A pre-statement of a formal system
corresponds to an extended frame in a Metamath database
(Section~\ref{frames}).  The collections $D$, $T$, and $H$ correspond
respectively to the \texttt{\$d}, \texttt{\$f}, and \texttt{\$e}
statement collections in an extended frame.  The expression $A$
corresponds to the \texttt{\$a} (or \texttt{\$p}) statement in an
extended frame.

A statement of a formal system corresponds to a frame in a Metamath
database.
\end{quotation}}

\subsection{Formal Systems}

A {\em formal system}\index{formal system} is a
triple $\langle \mbox{\em CN},\mbox{\em
VR},\Gamma\rangle$ where $\Gamma$ is a set of statements.  The members of
$\Gamma$ are called {\em axiomatic statements}.\index{axiomatic
statement!in a formal system}  Sometimes we will refer to a
formal system by just $\Gamma$ when $\mbox{\em CN}$ and $\mbox{\em VR}$ are
understood.

Given a formal system $\Gamma$, the {\em closure}\index{closure}\footnote{This
definition of closure incorporates a simplification due to
Josh Purinton.\index{Purinton, Josh}.} of a
pre-statement
$\langle D,T,H,A \rangle$ is the smallest set $C$ of expressions
such that:
%\begin{enumerate}
%  \item $T\cup H\subseteq C$; and
%  \item If for some axiomatic statement
%    $\langle D_M',T_M',H',A' \rangle \in \Gamma_A$, for
%    some $E \subseteq C$, some $F \subseteq C-T$ (where ``-'' denotes
%    set difference), and some substitution
%    $\sigma$ we have
%    \begin{enumerate}
%       \item $\sigma(T_M') = E$ (where, as above, the $M$ denotes the
%           mandatory variable-type hypotheses of $T^A$);
%       \item $\sigma(H') = F$;
%       \item for all $\{\alpha,\beta\}\in D^A$ and $\subseteq
%         {\cal V}(T_M')$, for all $\gamma\in {\cal V}(\sigma(\langle \alpha
%         \rangle))$, and for all $\delta\in  {\cal V}(\sigma(\langle \beta
%         \rangle))$, we have $\{\gamma, \delta\} \in D$;
%   \end{enumerate}
%   then $\sigma(A') \in C$.
%\end{enumerate}
\begin{list}{}{\itemsep 0.0pt}
  \item[1.] $T\cup H\subseteq C$; and
  \item[2.] If for some axiomatic statement
    $\langle D_M',T_M',H',A' \rangle \in
       \Gamma$ and for some substitution
    $\sigma$ we have
    \begin{enumerate}
       \item[a.] $\sigma(T_M' \cup H') \subseteq C$; and
       \item[b.] for all $\{\alpha,\beta\}\in D_M'$, for all $\gamma\in
         {\cal V}(\sigma(\langle \alpha
         \rangle))$, and for all $\delta\in  {\cal V}(\sigma(\langle \beta
         \rangle))$, we have $\{\gamma, \delta\} \in D$;
   \end{enumerate}
   then $\sigma(A') \in C$.
\end{list}
A pre-statement $\langle D,T,H,A
\rangle$ is {\em provable}\index{provable statement!in a formal
system} if $A\in C$ i.e.\ if its assertion belongs to its
closure.  A statement is {\em provable} if it is
the reduct of a provable pre-statement.
The {\em universe}\index{universe of a formal system}
of a formal system is
the collection of all of its provable statements.  Note that the
set of axiomatic statements $\Gamma$ in a formal system is a subset of its
universe.

{\footnotesize\begin{quotation}
{\em Comment.} The first condition in the definition of closure simply says
that the hypotheses of the pre-statement are in its closure.

Condition 2(a) says that a substitution exists that makes the
mandatory hypotheses of an axiomatic statement exactly match some members of
the closure.  This is what we explicitly demonstrate in a Metamath language
proof.

%Conditions 2(a) and 2(b) say that a substitution exists that makes the
%(mandatory) hypotheses of an axiomatic statement exactly match some members of
%the closure.  This is what we explicitly demonstrate with a Metamath language
%proof.
%
%The set of expressions $F$ in condition 2(b) excludes the variable-type
%hypotheses; this is done because non-mandatory variable-type hypotheses are
%effectively ``dropped'' as irrelevant whereas logical hypotheses must be
%retained to achieve a consistent logical system.

Condition 2(b) describes how distinct-variable restrictions in the axiomatic
statement must be met.  It means that after a substitution for two variables
that must be distinct, the resulting two expressions must either contain no
variables, or if they do, they may not have variables in common, and each pair
of any variables they do have, with one variable from each expression, must be
specified as distinct in the original statement.
\end{quotation}}

{\footnotesize\begin{quotation}
{\em Relationship to Metamath.} Axiomatic statements
 and provable statements in a formal
system correspond to the frames for \texttt{\$a} and \texttt{\$p} statements
respectively in a Metamath database.  The set of axiomatic statements is a
subset of the set of provable statements in a formal system, although in a
Metamath database a \texttt{\$a} statement is distinguished by not having a
proof.  A Metamath language proof for a \texttt{\$p} statement tells the computer
how to explicitly construct a series of members of the closure ultimately
leading to a demonstration that the assertion
being proved is in the closure.  The actual closure typically contains
an infinite number of expressions.  A formal system itself does not have
an explicit object called a ``proof'' but rather the existence of a proof
is implied indirectly by membership of an assertion in a provable
statement's closure.  We do this to make the formal system easier
to describe in the language of set theory.

We also note that once established as provable, a statement may be considered
to acquire the same status as an axiomatic statement, because if the set of
axiomatic statements is extended with a provable statement, the universe of
the formal system remains unchanged (provided that $\mbox{\em VR}$ is
infinite).
In practice, this means we can build a hierarchy of provable statements to
more efficiently establish additional provable statements.  This is
what we do in Metamath when we allow proofs to reference previous
\texttt{\$p} statements as well as previous \texttt{\$a} statements.
\end{quotation}}

\section{Examples of Formal Systems}

{\footnotesize\begin{quotation}
{\em Relationship to Metamath.} The examples in this section, except Example~2,
are for the most part exact equivalents of the development in the set
theory database \texttt{set.mm}.  You may want to compare Examples~1, 3, and 5
to Section~\ref{metaaxioms}, Example 4 to Sections~\ref{metadefprop} and
\ref{metadefpred}, and Example 6 to
Section~\ref{setdefinitions}.\label{exampleref}
\end{quotation}}

\subsection{Example~1---Propositional Calculus}\index{propositional calculus}

Classical propositional calculus can be described by the following formal
system.  We assume the set of variables is infinite.  Rather than denoting the
constants and variables by $c_0, c_1, \ldots$ and $v_0, v_1, \ldots$, for
readability we will instead use more conventional symbols, with the
understanding of course that they denote distinct primitive objects.
Also for readability we may omit commas between successive terms of a
sequence; thus $\langle \mbox{wff\ } \varphi\rangle$ denotes
$\langle \mbox{wff}, \varphi\rangle$.

Let
\begin{itemize}
  \item[] $\mbox{\em CN}=\{\mbox{wff}, \vdash, \to, \lnot, (,)\}$
  \item[] $\mbox{\em VR}=\{\varphi,\psi,\chi,\ldots\}$
  \item[] $T = \{\langle \mbox{wff\ } \varphi\rangle,
             \langle \mbox{wff\ } \psi\rangle,
             \langle \mbox{wff\ } \chi\rangle,\ldots\}$, i.e.\ those
             expressions of length 2 whose first member is $\mbox{\rm wff}$
             and whose second member belongs to $\mbox{\em VR}$.\footnote{For
convenience we let $T$ be an infinite set; the definition of a statement
permits this in principle.  Since a Metamath source file has a finite size, in
practice we must of course use appropriate finite subsets of this $T$,
specifically ones containing at least the mandatory variable-type
hypotheses.  Similarly, in the source file we introduce new variables as
required, with the understanding that a potentially infinite number of
them are available.}
\noindent Then $\Gamma$ consists of the axiomatic statements that
are the reducts of the following pre-statements:
    \begin{itemize}
      \item[] $\langle\varnothing,T,\varnothing,
               \langle \mbox{wff\ }(\varphi\to\psi)\rangle\rangle$
      \item[] $\langle\varnothing,T,\varnothing,
               \langle \mbox{wff\ }\lnot\varphi\rangle\rangle$
      \item[] $\langle\varnothing,T,\varnothing,
               \langle \vdash(\varphi\to(\psi\to\varphi))
               \rangle\rangle$
      \item[] $\langle\varnothing,T,
               \varnothing,
               \langle \vdash((\varphi\to(\psi\to\chi))\to
               ((\varphi\to\psi)\to(\varphi\to\chi)))
               \rangle\rangle$
      \item[] $\langle\varnothing,T,
               \varnothing,
               \langle \vdash((\lnot\varphi\to\lnot\psi)\to
               (\psi\to\varphi))\rangle\rangle$
      \item[] $\langle\varnothing,T,
               \{\langle\vdash(\varphi\to\psi)\rangle,
                 \langle\vdash\varphi\rangle\},
               \langle\vdash\psi\rangle\rangle$
    \end{itemize}
\end{itemize}

(For example, the reduct of $\langle\varnothing,T,\varnothing,
               \langle \mbox{wff\ }(\varphi\to\psi)\rangle\rangle$
is
\begin{itemize}
\item[] $\langle\varnothing,
\{\langle \mbox{wff\ } \varphi\rangle,
             \langle \mbox{wff\ } \psi\rangle\},
             \varnothing,
               \langle \mbox{wff\ }(\varphi\to\psi)\rangle\rangle$,
\end{itemize}
which is the first axiomatic statement.)

We call the members of $\mbox{\em VR}$ {\em wff variables} or (in the context
of first-order logic which we will describe shortly) {\em wff metavariables}.
Note that the symbols $\phi$, $\psi$, etc.\ denote actual specific members of
$\mbox{\em VR}$; they are not metavariables of our expository language (which
we denote with $\alpha$, $\beta$, etc.) but are instead (meta)constant symbols
(members of $\mbox{\em SM}$) from the point of view of our expository
language.  The equivalent system of propositional calculus described in
\cite{Tarski1965} also uses the symbols $\phi$, $\psi$, etc.\ to denote wff
metavariables, but in \cite{Tarski1965} unlike here those are metavariables of
the expository language and not primitive symbols of the formal system.

The first two statements define wffs: if $\varphi$ and $\psi$ are wffs, so is
$(\varphi \to \psi)$; if $\varphi$ is a wff, so is $\lnot\varphi$. The next
three are the axioms of propositional calculus: if $\varphi$ and $\psi$ are
wffs, then $\vdash (\varphi \to (\psi \to \varphi))$ is an (axiomatic)
theorem; etc. The
last is the rule of modus ponens: if $\varphi$ and $\psi$ are wffs, and
$\vdash (\varphi\to\psi)$ and $\vdash \varphi$ are theorems, then $\vdash
\psi$ is a theorem.

The correspondence to ordinary propositional calculus is as follows.  We
consider only provable statements of the form $\langle\varnothing,
T,\varnothing,A\rangle$ with $T$ defined as above.  The first term of the
assertion $A$ of any such statement is either ``wff'' or ``$\vdash$''.  A
statement for which the first term is ``wff'' is a {\em wff} of propositional
calculus, and one where the first term is ``$\vdash$'' is a {\em
theorem (scheme)} of propositional calculus.

The universe of this formal system also contains many other provable
statements.  Those with distinct-variable restrictions are irrelevant because
propositional calculus has no constraints on substitutions.  Those that have
logical hypotheses we call {\em inferences}\index{inference} when
the logical hypotheses are of the form
$\langle\vdash\rangle\frown w$ where $w$ is a wff (with the leading constant
term ``wff'' removed).  Inferences (other than the modus ponens rule) are not a
proper part of propositional calculus but are convenient to use when building a
hierarchy of provable statements.  A provable statement with a nonsense
hypothesis such as $\langle \to,\vdash,\lnot\rangle$, and this same expression
as its assertion, we consider irrelevant; no use can be made of it in
proving theorems, since there is no way to eliminate the nonsense hypothesis.

{\footnotesize\begin{quotation}
{\em Comment.} Our use of parentheses in the definition of a wff illustrates
how axiomatic statements should be carefully stated in a way that
ties in unambiguously with the substitutions allowed by the formal system.
There are many ways we could have defined wffs---for example, Polish
prefix notation would have allowed us to omit parentheses entirely, at
the expense of readability---but we must define them in a way that is
unambiguous.  For example, if we had omitted parentheses from the
definition of $(\varphi\to \psi)$, the wff $\lnot\varphi\to \psi$ could
be interpreted as either $\lnot(\varphi\to\psi)$ or $(\lnot\varphi\to\psi)$
and would have allowed us to prove nonsense.  Note that there is no
concept of operator binding precedence built into our formal system.
\end{quotation}}

\begin{sloppy}
\subsection{Example~2---Predicate Calculus with Equality}\index{predicate
calculus}
\end{sloppy}

Here we extend Example~1 to include predicate calculus with equality,
illustrating the use of distinct-variable restrictions.  This system is the
same as Tarski's system $\mathfrak{S}_2$ in \cite{Tarski1965} (except that the
axioms of propositional calculus are different but equivalent, and a redundant
axiom is omitted).  We extend $\mbox{\em CN}$ with the constants
$\{\mbox{var},\forall,=\}$.  We extend $\mbox{\em VR}$ with an infinite set of
{\em individual metavariables}\index{individual
metavariable} $\{x,y,z,\ldots\}$ and denote this subset
$\mbox{\em Vr}$.

We also join to $\mbox{\em CN}$ a possibly infinite set $\mbox{\em Pr}$ of {\em
predicates} $\{R,S,\ldots\}$.  We associate with $\mbox{\em Pr}$ a function
$\mbox{rnk}$ from $\mbox{\em Pr}$ to $\omega$, and for $\alpha\in \mbox{\em
Pr}$ we call $\mbox{rnk}(\alpha)$ the {\em rank} of the predicate $\alpha$,
which is simply the number of ``arguments'' that the predicate has.  (Most
applications of predicate calculus will have a finite number of predicates;
for example, set theory has the single two-argument or binary predicate $\in$,
which is usually written with its arguments surrounding the predicate symbol
rather than with the prefix notation we will use for the general case.)  As a
device to facilitate our discussion, we will let $\mbox{\em Vs}$ be any fixed
one-to-one function from $\omega$ to $\mbox{\em Vr}$; thus $\mbox{\em Vs}$ is
any simple infinite sequence of individual metavariables with no repeating
terms.

In this example we will not include the function symbols that are often part of
formalizations of predicate calculus.  Using metalogical arguments that are
beyond the scope of our discussion, it can be shown that our formalization is
equivalent when functions are introduced via appropriate definitions.

We extend the set $T$ defined in Example~1 with the expressions
$\{\langle \mbox{var\ } x\rangle,$ $ \langle \mbox{var\ } y\rangle, \langle
\mbox{var\ } z\rangle,\ldots\}$.  We extend the $\Gamma$ above
with the axiomatic statements that are the reducts of the following
pre-statements:
\begin{list}{}{\itemsep 0.0pt}
      \item[] $\langle\varnothing,T,\varnothing,
               \langle \mbox{wff\ }\forall x\,\varphi\rangle\rangle$
      \item[] $\langle\varnothing,T,\varnothing,
               \langle \mbox{wff\ }x=y\rangle\rangle$
      \item[] $\langle\varnothing,T,
               \{\langle\vdash\varphi\rangle\},
               \langle\vdash\forall x\,\varphi\rangle\rangle$
      \item[] $\langle\varnothing,T,\varnothing,
               \langle \vdash((\forall x(\varphi\to\psi)
                  \to(\forall x\,\varphi\to\forall x\,\psi))
               \rangle\rangle$
      \item[] $\langle\{\{x,\varphi\}\},T,\varnothing,
               \langle \vdash(\varphi\to\forall x\,\varphi)
               \rangle\rangle$
      \item[] $\langle\{\{x,y\}\},T,\varnothing,
               \langle \vdash\lnot\forall x\lnot x=y
               \rangle\rangle$
      \item[] $\langle\varnothing,T,\varnothing,
               \langle \vdash(x=z
                  \to(x=y\to z=y))
               \rangle\rangle$
      \item[] $\langle\varnothing,T,\varnothing,
               \langle \vdash(y=z
                  \to(x=y\to x=z))
               \rangle\rangle$
\end{list}
These are the axioms not involving predicate symbols. The first two statements
extend the definition of a wff.  The third is the rule of generalization.  The
fifth states, in effect, ``For a wff $\varphi$ and variable $x$,
$\vdash(\varphi\to\forall x\,\varphi)$, provided that $x$ does not occur in
$\varphi$.''  The sixth states ``For variables $x$ and $y$,
$\vdash\lnot\forall x\lnot x = y$, provided that $x$ and $y$ are distinct.''
(This proviso is not necessary but was included by Tarski to
weaken the axiom and still show that the system is logically complete.)

Finally, for each predicate symbol $\alpha\in \mbox{\em Pr}$, we add to
$\Gamma$ an axiomatic statement, extending the definition of wff,
that is the reduct of the following pre-statement:
\begin{displaymath}
    \langle\varnothing,T,\varnothing,
            \langle \mbox{wff},\alpha\rangle\
            \frown \mbox{\em Vs}\restriction\mbox{rnk}(\alpha)\rangle
\end{displaymath}
and for each $\alpha\in \mbox{\em Pr}$ and each $n < \mbox{rnk}(\alpha)$
we add to $\Gamma$ an equality axiom that is the reduct of the
following pre-statement:
\begin{eqnarray*}
    \lefteqn{\langle\varnothing,T,\varnothing,
            \langle
      \vdash,(,\mbox{\em Vs}_n,=,\mbox{\em Vs}_{\mbox{rnk}(\alpha)},\to,
            (,\alpha\rangle\frown \mbox{\em Vs}\restriction\mbox{rnk}(\alpha)} \\
  & & \frown
            \langle\to,\alpha\rangle\frown \mbox{\em Vs}\restriction n\frown
            \langle \mbox{\em Vs}_{\mbox{rnk}(\alpha)}\rangle \\
 & & \frown
            \mbox{\em Vs}\restriction(\mbox{rnk}(\alpha)\setminus(n+1))\frown
            \langle),)\rangle\rangle
\end{eqnarray*}
where $\restriction$ denotes function domain restriction and $\setminus$
denotes set difference.  Recall that a subscript on $\mbox{\em Vs}$
denotes one of its terms.  (In the above two axiom sets commas are placed
between successive terms of sequences to prevent ambiguity, and if you examine
them with care you will be able to distinguish those parentheses that denote
constant symbols from those of our expository language that delimit function
arguments.  Although it might have been better to use boldface for our
primitive symbols, unfortunately boldface was not available for all characters
on the \LaTeX\ system used to typeset this text.)  These seemingly forbidding
axioms can be understood by analogy to concatenation of substrings in a
computer language.  They are actually relatively simple for each specific case
and will become clearer by looking at the special case of a binary predicate
$\alpha = R$ where $\mbox{rnk}(R)=2$.  Letting $\mbox{\em Vs}$ be the sequence
$\langle x,y,z,\ldots\rangle$, the axioms we would add to $\Gamma$ for this
case would be the wff extension and two equality axioms that are the
reducts of the pre-statements:
\begin{list}{}{\itemsep 0.0pt}
      \item[] $\langle\varnothing,T,\varnothing,
               \langle \mbox{wff\ }R x y\rangle\rangle$
      \item[] $\langle\varnothing,T,\varnothing,
               \langle \vdash(x=z
                  \to(R x y \to R z y))
               \rangle\rangle$
      \item[] $\langle\varnothing,T,\varnothing,
               \langle \vdash(y=z
                  \to(R x y \to R x z))
               \rangle\rangle$
\end{list}
Study these carefully to see how the general axioms above evaluate to
them.  In practice, typically only a few special cases such as this would be
needed, and in any case the Metamath language will only permit us to describe
a finite number of predicates, as opposed to the infinite number permitted by
the formal system.  (If an infinite number should be needed for some reason,
we could not define the formal system directly in the Metamath language but
could instead define it metalogically under set theory as we
do in this appendix, and only the underlying set theory, with its single
binary predicate, would be defined directly in the Metamath language.)


{\footnotesize\begin{quotation}
{\em Comment.}  As we noted earlier, the specific variables denoted by the
symbols $x,y,z,\ldots\in \mbox{\em Vr}\subseteq \mbox{\em VR}\subseteq
\mbox{\em SM}$ in Example~2 are not the actual variables of ordinary predicate
calculus but should be thought of as metavariables ranging over them.  For
example, a distinct-variable restriction would be meaningless for actual
variables of ordinary predicate calculus since two different actual variables
are by definition distinct.  And when we talk about an arbitrary
representative $\alpha\in \mbox{\em Vr}$, $\alpha$ is a metavariable (in our
expository language) that ranges over metavariables (which are primitives of
our formal system) each of which ranges over the actual individual variables
of predicate calculus (which are never mentioned in our formal system).

The constant called ``var'' above is called \texttt{setvar} in the
\texttt{set.mm} database file, but it means the same thing.  I felt
that ``var'' is a more meaningful name in the context of predicate
calculus, whose use is not limited to set theory.  For consistency we
stick with the name ``var'' throughout this Appendix, even after set
theory is introduced.
\end{quotation}}

\subsection{Free Variables and Proper Substitution}\index{free variable}
\index{proper substitution}\index{substitution!proper}

Typical representations of mathematical axioms use concepts such
as ``free variable,'' ``bound variable,'' and ``proper substitution''
as primitive notions.
A free variable is a variable that
is not a parameter of any container expression.
A bound variable is the opposite of a free variable; it is a
a variable that has been bound in a container expression.
For example, in the expression $\forall x \varphi$ (for all $x$, $\varphi$
is true), the variable $x$
is bound within the for-all ($\forall$) expression.
It is possible to change one variable to another, and that process is called
``proper substitution.''
In most books, proper substitution has a somewhat complicated recursive
definition with multiple cases based on the occurrences of free and
bound variables.
You may consult
\cite[ch.\ 3--4]{Hamilton}\index{Hamilton, Alan G.} (as well as
many other texts) for more formal details about these terms.

Using these concepts as \texttt{primitives} creates complications
for computer implementations.

In the system of Example~2, there are no primitive notions of free variable
and proper substitution.  Tarski \cite{Tarski1965} shows that this system is
logically equivalent to the more typical textbook systems that do have these
primitive notions, if we introduce these notions with appropriate definitions
and metalogic.  We could also define axioms for such systems directly,
although the recursive definitions of free variable and proper substitution
would be messy and awkward to work with.  Instead, we mention two devices that
can be used in practice to mimic these notions.  (1) Instead of introducing
special notation to express (as a logical hypothesis) ``where $x$ is not free
in $\varphi$'' we can use the logical hypothesis $\vdash(\varphi\to\forall
x\,\varphi)$.\label{effectivelybound}\index{effectively
not free}\footnote{This is a slightly weaker requirement than ``where $x$ is
not free in $\varphi$.''  If we let $\varphi$ be $x=x$, we have the theorem
$(x=x\to\forall x\,x=x)$ which satisfies the hypothesis, even though $x$ is
free in $x=x$ .  In a case like this we say that $x$ is {\em effectively not
free}\index{effectively not free} in $x=x$, since $x=x$ is logically
equivalent to $\forall x\,x=x$ in which $x$ is bound.} (2) It can be shown
that the wff $((x=y\to\varphi)\wedge\exists x(x=y\wedge\varphi))$ (with the
usual definitions of $\wedge$ and $\exists$; see Example~4 below) is logically
equivalent to ``the wff that results from proper substitution of $y$ for $x$
in $\varphi$.''  This works whether or not $x$ and $y$ are distinct.

\subsection{Metalogical Completeness}\index{metalogical completeness}

In the system of Example~2, the
following are provable pre-statements (and their reducts are
provable statements):
\begin{eqnarray*}
      & \langle\{\{x,y\}\},T,\varnothing,
               \langle \vdash\lnot\forall x\lnot x=y
               \rangle\rangle & \\
     &  \langle\varnothing,T,\varnothing,
               \langle \vdash\lnot\forall x\lnot x=x
               \rangle\rangle &
\end{eqnarray*}
whereas the following pre-statement is not to my knowledge provable (but
in any case we will pretend it's not for sake of illustration):
\begin{eqnarray*}
     &  \langle\varnothing,T,\varnothing,
               \langle \vdash\lnot\forall x\lnot x=y
               \rangle\rangle &
\end{eqnarray*}
In other words, we can prove ``$\lnot\forall x\lnot x=y$ where $x$ and $y$ are
distinct'' and separately prove ``$\lnot\forall x\lnot x=x$'', but we can't
prove the combined general case ``$\lnot\forall x\lnot x=y$'' that has no
proviso.  Now this does not compromise logical completeness, because the
variables are really metavariables and the two provable cases together cover
all possible cases.  The third case can be considered a metatheorem whose
direct proof, using the system of Example~2, lies outside the capability of the
formal system.

Also, in the system of Example~2 the following pre-statement is not to my
knowledge provable (again, a conjecture that we will pretend to be the case):
\begin{eqnarray*}
     & \langle\varnothing,T,\varnothing,
               \langle \vdash(\forall x\, \varphi\to\varphi)
               \rangle\rangle &
\end{eqnarray*}
Instead, we can only prove specific cases of $\varphi$ involving individual
metavariables, and by induction on formula length, prove as a metatheorem
outside of our formal system the general statement above.  The details of this
proof are found in \cite{Kalish}.

There does, however, exist a system of predicate calculus in which all such
``simple metatheorems'' as those above can be proved directly, and we present
it in Example~3. A {\em simple metatheorem}\index{simple metatheorem}
is any statement of the formal
system of Example~2 where all distinct variable restrictions consist of either
two individual metavariables or an individual metavariable and a wff
metavariable, and which is provable by combining cases outside the system as
above.  A system is {\em metalogically complete}\index{metalogical
completeness} if all of its simple
metatheorems are (directly) provable statements. The precise definition of
``simple metatheorem'' and the proof of the ``metalogical completeness'' of
Example~3 is found in Remark 9.6 and Theorem 9.7 of \cite{Megill}.\index{Megill,
Norman}

\begin{sloppy}
\subsection{Example~3---Metalogically Complete Predicate
Calculus with
Equality}
\end{sloppy}

For simplicity we will assume there is one binary predicate $R$;
this system suffices for set theory, where the $R$ is of course the $\in$
predicate.  We label the axioms as they appear in \cite{Megill}.  This
system is logically equivalent to that of Example~2 (when the latter is
restricted to this single binary predicate) but is also metalogically
complete.\index{metalogical completeness}

Let
\begin{itemize}
  \item[] $\mbox{\em CN}=\{\mbox{wff}, \mbox{var}, \vdash, \to, \lnot, (,),\forall,=,R\}$.
  \item[] $\mbox{\em VR}=\{\varphi,\psi,\chi,\ldots\}\cup\{x,y,z,\ldots\}$.
  \item[] $T = \{\langle \mbox{wff\ } \varphi\rangle,
             \langle \mbox{wff\ } \psi\rangle,
             \langle \mbox{wff\ } \chi\rangle,\ldots\}\cup
       \{\langle \mbox{var\ } x\rangle, \langle \mbox{var\ } y\rangle, \langle
       \mbox{var\ }z\rangle,\ldots\}$.

\noindent Then
  $\Gamma$ consists of the reducts of the following pre-statements:
    \begin{itemize}
      \item[] $\langle\varnothing,T,\varnothing,
               \langle \mbox{wff\ }(\varphi\to\psi)\rangle\rangle$
      \item[] $\langle\varnothing,T,\varnothing,
               \langle \mbox{wff\ }\lnot\varphi\rangle\rangle$
      \item[] $\langle\varnothing,T,\varnothing,
               \langle \mbox{wff\ }\forall x\,\varphi\rangle\rangle$
      \item[] $\langle\varnothing,T,\varnothing,
               \langle \mbox{wff\ }x=y\rangle\rangle$
      \item[] $\langle\varnothing,T,\varnothing,
               \langle \mbox{wff\ }Rxy\rangle\rangle$
      \item[(C1$'$)] $\langle\varnothing,T,\varnothing,
               \langle \vdash(\varphi\to(\psi\to\varphi))
               \rangle\rangle$
      \item[(C2$'$)] $\langle\varnothing,T,
               \varnothing,
               \langle \vdash((\varphi\to(\psi\to\chi))\to
               ((\varphi\to\psi)\to(\varphi\to\chi)))
               \rangle\rangle$
      \item[(C3$'$)] $\langle\varnothing,T,
               \varnothing,
               \langle \vdash((\lnot\varphi\to\lnot\psi)\to
               (\psi\to\varphi))\rangle\rangle$
      \item[(C4$'$)] $\langle\varnothing,T,
               \varnothing,
               \langle \vdash(\forall x(\forall x\,\varphi\to\psi)\to
                 (\forall x\,\varphi\to\forall x\,\psi))\rangle\rangle$
      \item[(C5$'$)] $\langle\varnothing,T,
               \varnothing,
               \langle \vdash(\forall x\,\varphi\to\varphi)\rangle\rangle$
      \item[(C6$'$)] $\langle\varnothing,T,
               \varnothing,
               \langle \vdash(\forall x\forall y\,\varphi\to
                 \forall y\forall x\,\varphi)\rangle\rangle$
      \item[(C7$'$)] $\langle\varnothing,T,
               \varnothing,
               \langle \vdash(\lnot\varphi\to\forall x\lnot\forall x\,\varphi
                 )\rangle\rangle$
      \item[(C8$'$)] $\langle\varnothing,T,
               \varnothing,
               \langle \vdash(x=y\to(x=z\to y=z))\rangle\rangle$
      \item[(C9$'$)] $\langle\varnothing,T,
               \varnothing,
               \langle \vdash(\lnot\forall x\, x=y\to(\lnot\forall x\, x=z\to
                 (y=z\to\forall x\, y=z)))\rangle\rangle$
      \item[(C10$'$)] $\langle\varnothing,T,
               \varnothing,
               \langle \vdash(\forall x(x=y\to\forall x\,\varphi)\to
                 \varphi))\rangle\rangle$
      \item[(C11$'$)] $\langle\varnothing,T,
               \varnothing,
               \langle \vdash(\forall x\, x=y\to(\forall x\,\varphi
               \to\forall y\,\varphi))\rangle\rangle$
      \item[(C12$'$)] $\langle\varnothing,T,
               \varnothing,
               \langle \vdash(x=y\to(Rxz\to Ryz))\rangle\rangle$
      \item[(C13$'$)] $\langle\varnothing,T,
               \varnothing,
               \langle \vdash(x=y\to(Rzx\to Rzy))\rangle\rangle$
      \item[(C15$'$)] $\langle\varnothing,T,
               \varnothing,
               \langle \vdash(\lnot\forall x\, x=y\to(x=y\to(\varphi
                 \to\forall x(x=y\to\varphi))))\rangle\rangle$
      \item[(C16$'$)] $\langle\{\{x,y\}\},T,
               \varnothing,
               \langle \vdash(\forall x\, x=y\to(\varphi\to\forall x\,\varphi)
                 )\rangle\rangle$
      \item[(C5)] $\langle\{\{x,\varphi\}\},T,\varnothing,
               \langle \vdash(\varphi\to\forall x\,\varphi)
               \rangle\rangle$
      \item[(MP)] $\langle\varnothing,T,
               \{\langle\vdash(\varphi\to\psi)\rangle,
                 \langle\vdash\varphi\rangle\},
               \langle\vdash\psi\rangle\rangle$
      \item[(Gen)] $\langle\varnothing,T,
               \{\langle\vdash\varphi\rangle\},
               \langle\vdash\forall x\,\varphi\rangle\rangle$
    \end{itemize}
\end{itemize}

While it is known that these axioms are ``metalogically complete,'' it is
not known whether they are independent (i.e.\ none is
redundant) in the metalogical sense; specifically, whether any axiom (possibly
with additional non-mandatory distinct-variable restrictions, for use with any
dummy variables in its proof) is provable from the others.  Note that
metalogical independence is a weaker requirement than independence in the
usual logical sense.  Not all of the above axioms are logically independent:
for example, C9$'$ can be proved as a metatheorem from the others, outside the
formal system, by combining the possible cases of distinct variables.

\subsection{Example~4---Adding Definitions}\index{definition}
There are several ways to add definitions to a formal system.  Probably the
most proper way is to consider definitions not as part of the formal system at
all but rather as abbreviations that are part of the expository metalogic
outside the formal system.  For convenience, though, we may use the formal
system itself to incorporate definitions, adding them as axiomatic extensions
to the system.  This could be done by adding a constant representing the
concept ``is defined as'' along with axioms for it. But there is a nicer way,
at least in this writer's opinion, that introduces definitions as direct
extensions to the language rather than as extralogical primitive notions.  We
introduce additional logical connectives and provide axioms for them.  For
systems of logic such as Examples 1 through 3, the additional axioms must be
conservative in the sense that no wff of the original system that was not a
theorem (when the initial term ``wff'' is replaced by ``$\vdash$'' of course)
becomes a theorem of the extended system.  In this example we extend Example~3
(or 2) with standard abbreviations of logic.

We extend $\mbox{\em CN}$ of Example~3 with new constants $\{\leftrightarrow,
\wedge,\vee,\exists\}$, corresponding to logical equivalence,\index{logical
equivalence ($\leftrightarrow$)}\index{biconditional ($\leftrightarrow$)}
conjunction,\index{conjunction ($\wedge$)} disjunction,\index{disjunction
($\vee$)} and the existential quantifier.\index{existential quantifier
($\exists$)}  We extend $\Gamma$ with the axiomatic statements that are
the reducts of the following pre-statements:
\begin{list}{}{\itemsep 0.0pt}
      \item[] $\langle\varnothing,T,\varnothing,
               \langle \mbox{wff\ }(\varphi\leftrightarrow\psi)\rangle\rangle$
      \item[] $\langle\varnothing,T,\varnothing,
               \langle \mbox{wff\ }(\varphi\vee\psi)\rangle\rangle$
      \item[] $\langle\varnothing,T,\varnothing,
               \langle \mbox{wff\ }(\varphi\wedge\psi)\rangle\rangle$
      \item[] $\langle\varnothing,T,\varnothing,
               \langle \mbox{wff\ }\exists x\, \varphi\rangle\rangle$
  \item[] $\langle\varnothing,T,\varnothing,
     \langle\vdash ( ( \varphi \leftrightarrow \psi ) \to
     ( \varphi \to \psi ) )\rangle\rangle$
  \item[] $\langle\varnothing,T,\varnothing,
     \langle\vdash ((\varphi\leftrightarrow\psi)\to
    (\psi\to\varphi))\rangle\rangle$
  \item[] $\langle\varnothing,T,\varnothing,
     \langle\vdash ((\varphi\to\psi)\to(
     (\psi\to\varphi)\to(\varphi
     \leftrightarrow\psi)))\rangle\rangle$
  \item[] $\langle\varnothing,T,\varnothing,
     \langle\vdash (( \varphi \wedge \psi ) \leftrightarrow\neg ( \varphi
     \to \neg \psi )) \rangle\rangle$
  \item[] $\langle\varnothing,T,\varnothing,
     \langle\vdash (( \varphi \vee \psi ) \leftrightarrow (\neg \varphi
     \to \psi )) \rangle\rangle$
  \item[] $\langle\varnothing,T,\varnothing,
     \langle\vdash (\exists x \,\varphi\leftrightarrow
     \lnot \forall x \lnot \varphi)\rangle\rangle$
\end{list}
The first three logical axioms (statements containing ``$\vdash$'') introduce
and effectively define logical equivalence, ``$\leftrightarrow$''.  The last
three use ``$\leftrightarrow$'' to effectively mean ``is defined as.''

\subsection{Example~5---ZFC Set Theory}\index{ZFC set theory}

Here we add to the system of Example~4 the axioms of Zermelo--Fraenkel set
theory with Choice.  For convenience we make use of the
definitions in Example~4.

In the $\mbox{\em CN}$ of Example~4 (which extends Example~3), we replace the symbol $R$
with the symbol $\in$.
More explicitly, we remove from $\Gamma$ of Example~4 the three
axiomatic statements containing $R$ and replace them with the
reducts of the following:
\begin{list}{}{\itemsep 0.0pt}
      \item[] $\langle\varnothing,T,\varnothing,
               \langle \mbox{wff\ }x\in y\rangle\rangle$
      \item[] $\langle\varnothing,T,
               \varnothing,
               \langle \vdash(x=y\to(x\in z\to y\in z))\rangle\rangle$
      \item[] $\langle\varnothing,T,
               \varnothing,
               \langle \vdash(x=y\to(z\in x\to z\in y))\rangle\rangle$
\end{list}
Letting $D=\{\{\alpha,\beta\}\in \mbox{\em DV}\,|\alpha,\beta\in \mbox{\em
Vr}\}$ (in other words all individual variables must be distinct), we extend
$\Gamma$ with the ZFC axioms, called
\index{Axiom of Extensionality}
\index{Axiom of Replacement}
\index{Axiom of Union}
\index{Axiom of Power Sets}
\index{Axiom of Regularity}
\index{Axiom of Infinity}
\index{Axiom of Choice}
Extensionality, Replacement, Union, Power
Set, Regularity, Infinity, and Choice, that are the reducts of:
\begin{list}{}{\itemsep 0.0pt}
      \item[Ext] $\langle D,T,
               \varnothing,
               \langle\vdash (\forall x(x\in y\leftrightarrow x \in z)\to y
               =z) \rangle\rangle$
      \item[Rep] $\langle D,T,
               \varnothing,
               \langle\vdash\exists x ( \exists y \forall z (\varphi \to z = y
                        ) \to
                        \forall z ( z \in x \leftrightarrow \exists x ( x \in
                        y \wedge \forall y\,\varphi ) ) )\rangle\rangle$
      \item[Un] $\langle D,T,
               \varnothing,
               \langle\vdash \exists x \forall y ( \exists x ( y \in x \wedge
               x \in z ) \to y \in x ) \rangle\rangle$
      \item[Pow] $\langle D,T,
               \varnothing,
               \langle\vdash \exists x \forall y ( \forall x ( x \in y \to x
               \in z ) \to y \in x ) \rangle\rangle$
      \item[Reg] $\langle D,T,
               \varnothing,
               \langle\vdash (  x \in y \to
                 \exists x ( x \in y \wedge \forall z ( z \in x \to \lnot z
                \in y ) ) ) \rangle\rangle$
      \item[Inf] $\langle D,T,
               \varnothing,
               \langle\vdash \exists x(y\in x\wedge\forall y(y\in
               x\to
               \exists z(y \in z\wedge z\in x))) \rangle\rangle$
      \item[AC] $\langle D,T,
               \varnothing,
               \langle\vdash \exists x \forall y \forall z ( ( y \in z
               \wedge z \in w ) \to \exists w \forall y ( \exists w
              ( ( y \in z \wedge z \in w ) \wedge ( y \in w \wedge w \in x
              ) ) \leftrightarrow y = w ) ) \rangle\rangle$
\end{list}

\subsection{Example~6---Class Notation in Set Theory}\label{class}

A powerful device that makes set theory easier (and that we have
been using all along in our informal expository language) is {\em class
abstraction notation}.\index{class abstraction}\index{abstraction class}  The
definitions we introduce are rigorously justified
as conservative by Takeuti and Zaring \cite{Takeuti}\index{Takeuti, Gaisi} or
Quine \cite{Quine}\index{Quine, Willard Van Orman}.  The key idea is to
introduce the notation $\{x|\mbox{---}\}$ which means ``the class of all $x$
such that ---'' for abstraction classes and introduce (meta)variables that
range over them.  An abstraction class may or may not be a set, depending on
whether it exists (as a set).  A class that does not exist is
called a {\em proper class}.\index{proper class}\index{class!proper}

To illustrate the use of abstraction classes we will provide some examples
of definitions that make use of them:  the empty set, class union, and
unordered pair.  Many other such definitions can be found in the
Metamath set theory database,
\texttt{set.mm}.\index{set theory database (\texttt{set.mm})}

% We intentionally break up the sequence of math symbols here
% because otherwise the overlong line goes beyond the page in narrow mode.
We extend $\mbox{\em CN}$ of Example~5 with new symbols $\{$
$\mbox{class},$ $\{,$ $|,$ $\},$ $\varnothing,$ $\cup,$ $,$ $\}$
where the inner braces and last comma are
constant symbols. (As before,
our dual use of some mathematical symbols for both our expository
language and as primitives of the formal system should be clear from context.)

We extend $\mbox{\em VR}$ of Example~5 with a set of {\em class
variables}\index{class variable}
$\{A,B,C,\ldots\}$. We extend the $T$ of Example~5 with $\{\langle
\mbox{class\ } A\rangle, \langle \mbox{class\ }B\rangle, \langle \mbox{class\ }
C\rangle,\ldots\}$.

To
introduce our definitions,
we add to $\Gamma$ of Example~5 the axiomatic statements
that are the reducts of the following pre-statements:
\begin{list}{}{\itemsep 0.0pt}
      \item[] $\langle\varnothing,T,\varnothing,
               \langle \mbox{class\ }x\rangle\rangle$
      \item[] $\langle\varnothing,T,\varnothing,
               \langle \mbox{class\ }\{x|\varphi\}\rangle\rangle$
      \item[] $\langle\varnothing,T,\varnothing,
               \langle \mbox{wff\ }A=B\rangle\rangle$
      \item[] $\langle\varnothing,T,\varnothing,
               \langle \mbox{wff\ }A\in B\rangle\rangle$
      \item[Ab] $\langle\varnothing,T,\varnothing,
               \langle \vdash ( y \in \{ x |\varphi\} \leftrightarrow
                  ( ( x = y \to\varphi) \wedge \exists x ( x = y
                  \wedge\varphi) ))
               \rangle\rangle$
      \item[Eq] $\langle\{\{x,A\},\{x,B\}\},T,\varnothing,
               \langle \vdash ( A = B \leftrightarrow
               \forall x ( x \in A \leftrightarrow x \in B ) )
               \rangle\rangle$
      \item[El] $\langle\{\{x,A\},\{x,B\}\},T,\varnothing,
               \langle \vdash ( A \in B \leftrightarrow \exists x
               ( x = A \wedge x \in B ) )
               \rangle\rangle$
\end{list}
Here we say that an individual variable is a class; $\{x|\varphi\}$ is a
class; and we extend the definition of a wff to include class equality and
membership.  Axiom Ab defines membership of a variable in a class abstraction;
the right-hand side can be read as ``the wff that results from proper
substitution of $y$ for $x$ in $\varphi$.''\footnote{Note that this definition
makes unnecessary the introduction of a separate notation similar to
$\varphi(x|y)$ for proper substitution, although we may choose to do so to be
conventional.  Incidentally, $\varphi(x|y)$ as it stands would be ambiguous in
the formal systems of our examples, since we wouldn't know whether
$\lnot\varphi(x|y)$ meant $\lnot(\varphi(x|y))$ or $(\lnot\varphi)(x|y)$.
Instead, we would have to use an unambiguous variant such as $(\varphi\,
x|y)$.}  Axioms Eq and El extend the meaning of the existing equality and
membership connectives.  This is potentially dangerous and requires careful
justification.  For example, from Eq we can derive the Axiom of Extensionality
with predicate logic alone; thus in principle we should include the Axiom of
Extensionality as a logical hypothesis.  However we do not bother to do this
since we have already presupposed that axiom earlier. The distinct variable
restrictions should be read ``where $x$ does not occur in $A$ or $B$.''  We
typically do this when the right-hand side of a definition involves an
individual variable not in the expression being defined; it is done so that
the right-hand side remains independent of the particular ``dummy'' variable
we use.

We continue to add to $\Gamma$ the following definitions
(i.e. the reducts of the following pre-statements) for empty
set,\index{empty set} class union,\index{union} and unordered
pair.\index{unordered pair}  They should be self-explanatory.  Analogous to our
use of ``$\leftrightarrow$'' to define new wffs in Example~4, we use ``$=$''
to define new abstraction terms, and both may be read informally as ``is
defined as'' in this context.
\begin{list}{}{\itemsep 0.0pt}
      \item[] $\langle\varnothing,T,\varnothing,
               \langle \mbox{class\ }\varnothing\rangle\rangle$
      \item[] $\langle\varnothing,T,\varnothing,
               \langle \vdash \varnothing = \{ x | \lnot x = x \}
               \rangle\rangle$
      \item[] $\langle\varnothing,T,\varnothing,
               \langle \mbox{class\ }(A\cup B)\rangle\rangle$
      \item[] $\langle\{\{x,A\},\{x,B\}\},T,\varnothing,
               \langle \vdash ( A \cup B ) = \{ x | ( x \in A \vee x \in B ) \}
               \rangle\rangle$
      \item[] $\langle\varnothing,T,\varnothing,
               \langle \mbox{class\ }\{A,B\}\rangle\rangle$
      \item[] $\langle\{\{x,A\},\{x,B\}\},T,\varnothing,
               \langle \vdash \{ A , B \} = \{ x | ( x = A \vee x = B ) \}
               \rangle\rangle$
\end{list}

\section{Metamath as a Formal System}\label{theorymm}

This section presupposes a familiarity with the Metamath computer language.

Our theory describes formal systems and their universes.  The Metamath
language provides a way of representing these set-theoretical objects to
a computer.  A Metamath database, being a finite set of {\sc ascii}
characters, can usually describe only a subset of a formal system and
its universe, which are typically infinite.  However the database can
contain as large a finite subset of the formal system and its universe
as we wish.  (Of course a Metamath set theory database can, in
principle, indirectly describe an entire infinite formal system by
formalizing the expository language in this Appendix.)

For purpose of our discussion, we assume the Metamath database
is in the simple form described on p.~\pageref{framelist},
consisting of all constant and variable declarations at the beginning,
followed by a sequence of extended frames each
delimited by \texttt{\$\char`\{} and \texttt{\$\char`\}}.  Any Metamath database can
be converted to this form, as described on p.~\pageref{frameconvert}.

The math symbol tokens of a Metamath source file, which are declared
with \texttt{\$c} and \texttt{\$v} statements, are names we assign to
representatives of $\mbox{\em CN}$ and $\mbox{\em VR}$.  For
definiteness we could assume that the first math symbol declared as a
variable corresponds to $v_0$, the second to $v_1$, etc., although the
exact correspondence we choose is not important.

In the Metamath language, each \texttt{\$d}, \texttt{\$f}, and
 \texttt{\$e} source
statement in an extended frame (Section~\ref{frames})
corresponds respectively to a member of the
collections $D$, $T$, and $H$ in a formal system statement $\langle
D_M,T_M,H,A\rangle$.  The math symbol strings following these Metamath keywords
correspond to a variable pair (in the case of \texttt{\$d}) or an expression (for
the other two keywords). The math symbol string following a \texttt{\$a} source
statement corresponds to expression $A$ in an axiomatic statement of the
formal system; the one following a \texttt{\$p} source statement corresponds to
$A$ in a provable statement that is not axiomatic.  In other words, each
extended frame in a Metamath database corresponds to
a pre-statement of the formal system, and a frame corresponds to
a statement of the formal system.  (Don't confuse the two meanings of
``statement'' here.  A statement of the formal system corresponds to the
several statements in a Metamath database that may constitute a
frame.)

In order for the computer to verify that a formal system statement is
provable, each \texttt{\$p} source statement is accompanied by a proof.
However, the proof does not correspond to anything in the formal system
but is simply a way of communicating to the computer the information
needed for its verification.  The proof tells the computer {\em how to
construct} specific members of closure of the formal system
pre-statement corresponding to the extended frame of the \texttt{\$p}
statement.  The final result of the construction is the member of the
closure that matches the \texttt{\$p} statement.  The abstract formal
system, on the other hand, is concerned only with the {\em existence} of
members of the closure.

As mentioned on p.~\pageref{exampleref}, Examples 1 and 3--6 in the
previous Section parallel the development of logic and set theory in the
Metamath database
\texttt{set.mm}.\index{set theory database (\texttt{set.mm})} You may
find it instructive to compare them.


\chapter{The MIU System}
\label{MIU}
\index{formal system}
\index{MIU-system}

The following is a listing of the file \texttt{miu.mm}.  It is self-explanatory.

%%%%%%%%%%%%%%%%%%%%%%%%%%%%%%%%%%%%%%%%%%%%%%%%%%%%%%%%%%%%

\begin{verbatim}
$( The MIU-system:  A simple formal system $)

$( Note:  This formal system is unusual in that it allows
empty wffs.  To work with a proof, you must type
SET EMPTY_SUBSTITUTION ON before using the PROVE command.
By default, this is OFF in order to reduce the number of
ambiguous unification possibilities that have to be selected
during the construction of a proof.  $)

$(
Hofstadter's MIU-system is a simple example of a formal
system that illustrates some concepts of Metamath.  See
Douglas R. Hofstadter, _Goedel, Escher, Bach:  An Eternal
Golden Braid_ (Vintage Books, New York, 1979), pp. 33ff. for
a description of the MIU-system.

The system has 3 constant symbols, M, I, and U.  The sole
axiom of the system is MI. There are 4 rules:
     Rule I:  If you possess a string whose last letter is I,
     you can add on a U at the end.
     Rule II:  Suppose you have Mx.  Then you may add Mxx to
     your collection.
     Rule III:  If III occurs in one of the strings in your
     collection, you may make a new string with U in place
     of III.
     Rule IV:  If UU occurs inside one of your strings, you
     can drop it.
Unfortunately, Rules III and IV do not have unique results:
strings could have more than one occurrence of III or UU.
This requires that we introduce the concept of an "MIU
well-formed formula" or wff, which allows us to construct
unique symbol sequences to which Rules III and IV can be
applied.
$)

$( First, we declare the constant symbols of the language.
Note that we need two symbols to distinguish the assertion
that a sequence is a wff from the assertion that it is a
theorem; we have arbitrarily chosen "wff" and "|-". $)
      $c M I U |- wff $. $( Declare constants $)

$( Next, we declare some variables. $)
     $v x y $.

$( Throughout our theory, we shall assume that these
variables represent wffs. $)
 wx   $f wff x $.
 wy   $f wff y $.

$( Define MIU-wffs.  We allow the empty sequence to be a
wff. $)

$( The empty sequence is a wff. $)
 we   $a wff $.
$( "M" after any wff is a wff. $)
 wM   $a wff x M $.
$( "I" after any wff is a wff. $)
 wI   $a wff x I $.
$( "U" after any wff is a wff. $)
 wU   $a wff x U $.

$( Assert the axiom. $)
 ax   $a |- M I $.

$( Assert the rules. $)
 ${
   Ia   $e |- x I $.
$( Given any theorem ending with "I", it remains a theorem
if "U" is added after it.  (We distinguish the label I_
from the math symbol I to conform to the 24-Jun-2006
Metamath spec.) $)
   I_    $a |- x I U $.
 $}
 ${
IIa  $e |- M x $.
$( Given any theorem starting with "M", it remains a theorem
if the part after the "M" is added again after it. $)
   II   $a |- M x x $.
 $}
 ${
   IIIa $e |- x I I I y $.
$( Given any theorem with "III" in the middle, it remains a
theorem if the "III" is replaced with "U". $)
   III  $a |- x U y $.
 $}
 ${
   IVa  $e |- x U U y $.
$( Given any theorem with "UU" in the middle, it remains a
theorem if the "UU" is deleted. $)
   IV   $a |- x y $.
  $}

$( Now we prove the theorem MUIIU.  You may be interested in
comparing this proof with that of Hofstadter (pp. 35 - 36).
$)
 theorem1  $p |- M U I I U $=
      we wM wU wI we wI wU we wU wI wU we wM we wI wU we wM
      wI wI wI we wI wI we wI ax II II I_ III II IV $.
\end{verbatim}\index{well-formed formula (wff)}

The \texttt{show proof /lemmon/renumber} command
yields the following display.  It is very similar
to the one in \cite[pp.~35--36]{Hofstadter}.\index{Hofstadter, Douglas R.}

\begin{verbatim}
1 ax             $a |- M I
2 1 II           $a |- M I I
3 2 II           $a |- M I I I I
4 3 I_           $a |- M I I I I U
5 4 III          $a |- M U I U
6 5 II           $a |- M U I U U I U
7 6 IV           $a |- M U I I U
\end{verbatim}

We note that Hofstadter's ``MU-puzzle,'' which asks whether
MU is a theorem of the MIU-system, cannot be answered using
the system above because the MU-puzzle is a question {\em
about} the system.  To prove the answer to the MU-puzzle,
a much more elaborate system is needed, namely one that
models the MIU-system within set theory.  (Incidentally, the
answer to the MU-puzzle is no.)

\chapter{Metamath Language EBNF}%
\label{BNF}%
\index{Metamath Language EBNF}

The following is a formal description of the basic Metamath language syntax
(with compressed proofs and support for unknown proof steps).
It is defined using the
Extended Backus--Naur Form (EBNF)\index{Extended Backus--Naur Form}\index{EBNF}
notation from W3C\index{W3C}
\textit{Extensible Markup Language (XML) 1.0 (Fifth Edition)}
(W3C Recommendation 26 November 2008) at
\url{https://www.w3.org/TR/xml/#sec-notation}.

The \texttt{database}
rule is processed until the end of the file (\texttt{EOF}).
The rules eventually require reading whitespace-separated tokens.
A token has an upper-case definition (see below)
or is a string constant in a non-token (such as \texttt{'\$a'}).
We intend for this to be correct, but if there is a conflict the
rules of section \ref{spec} govern. That section also discusses
non-syntax restrictions not shown here
(e.g., that each new label token
defined in a \texttt{hypothesis-stmt} or \texttt{assert-stmt}
must be unique).

\begin{verbatim}
database ::= outermost-scope-stmt*

outermost-scope-stmt ::=
  include-stmt | constant-stmt | stmt

/* File inclusion command; process file as a database.
   Databases should NOT have a comment in the filename. */
include-stmt ::= '$[' filename '$]'

/* Constant symbols declaration. */
constant-stmt ::= '$c' constant+ '$.'

/* A normal statement can occur in any scope. */
stmt ::= block | variable-stmt | disjoint-stmt |
  hypothesis-stmt | assert-stmt

/* A block. You can have 0 statements in a block. */
block ::= '${' stmt* '$}'

/* Variable symbols declaration. */
variable-stmt ::= '$v' variable+ '$.'

/* Disjoint variables. Simple disjoint statements have
   2 variables, i.e., "variable*" is empty for them. */
disjoint-stmt ::= '$d' variable variable variable* '$.'

hypothesis-stmt ::= floating-stmt | essential-stmt

/* Floating (variable-type) hypothesis. */
floating-stmt ::= LABEL '$f' typecode variable '$.'

/* Essential (logical) hypothesis. */
essential-stmt ::= LABEL '$e' typecode MATH-SYMBOL* '$.'

assert-stmt ::= axiom-stmt | provable-stmt

/* Axiomatic assertion. */
axiom-stmt ::= LABEL '$a' typecode MATH-SYMBOL* '$.'

/* Provable assertion. */
provable-stmt ::= LABEL '$p' typecode MATH-SYMBOL*
  '$=' proof '$.'

/* A proof. Proofs may be interspersed by comments.
   If '?' is in a proof it's an "incomplete" proof. */
proof ::= uncompressed-proof | compressed-proof
uncompressed-proof ::= (LABEL | '?')+
compressed-proof ::= '(' LABEL* ')' COMPRESSED-PROOF-BLOCK+

typecode ::= constant

filename ::= MATH-SYMBOL /* No whitespace or '$' */
constant ::= MATH-SYMBOL
variable ::= MATH-SYMBOL
\end{verbatim}

\needspace{2\baselineskip}
A \texttt{frame} is a sequence of 0 or more
\texttt{disjoint-{\allowbreak}stmt} and
\texttt{hypotheses-{\allowbreak}stmt} statements
(possibly interleaved with other non-\texttt{assert-stmt} statements)
followed by one \texttt{assert-stmt}.

\needspace{3\baselineskip}
Here are the rules for lexical processing (tokenization) beyond
the constant tokens shown above.
By convention these tokenization rules have upper-case names.
Every token is read for the longest possible length.
Whitespace-separated tokens are read sequentially;
note that the separating whitespace and \texttt{\$(} ... \texttt{\$)}
comments are skipped.

If a token definition uses another token definition, the whole thing
is considered a single token.
A pattern that is only part of a full token has a name beginning
with an underscore (``\_'').
An implementation could tokenize many tokens as a
\texttt{PRINTABLE-SEQUENCE}
and then check if it meets the more specific rule shown here.

Comments do not nest, and both \texttt{\$(} and \texttt{\$)}
have to be surrounded
by at least one whitespace character (\texttt{\_WHITECHAR}).
Technically comments end without consuming the trailing
\texttt{\_WHITECHAR}, but the trailing
\texttt{\_WHITECHAR} gets ignored anyway so we ignore that detail here.
Metamath language processors
are not required to support \texttt{\$)} followed
immediately by a bare end-of-file, because the closing
comment symbol is supposed to be followed by a
\texttt{\_WHITECHAR} such as a newline.

\begin{verbatim}
PRINTABLE-SEQUENCE ::= _PRINTABLE-CHARACTER+

MATH-SYMBOL ::= (_PRINTABLE-CHARACTER - '$')+

/* ASCII non-whitespace printable characters */
_PRINTABLE-CHARACTER ::= [#x21-#x7e]

LABEL ::= ( _LETTER-OR-DIGIT | '.' | '-' | '_' )+

_LETTER-OR-DIGIT ::= [A-Za-z0-9]

COMPRESSED-PROOF-BLOCK ::= ([A-Z] | '?')+

/* Define whitespace between tokens. The -> SKIP
   means that when whitespace is seen, it is
   skipped and we simply read again. */
WHITESPACE ::= (_WHITECHAR+ | _COMMENT) -> SKIP

/* Comments. $( ... $) and do not nest. */
_COMMENT ::= '$(' (_WHITECHAR+ (PRINTABLE-SEQUENCE - '$)')*
  _WHITECHAR+ '$)' _WHITECHAR

/* Whitespace: (' ' | '\t' | '\r' | '\n' | '\f') */
_WHITECHAR ::= [#x20#x09#x0d#x0a#x0c]
\end{verbatim}
% This EBNF was developed as a collaboration between
% David A. Wheeler\index{Wheeler, David A.},
% Mario Carneiro\index{Carneiro, Mario}, and
% Benoit Jubin\index{Jubin, Benoit}, inspired by a request
% (and a lot of initial work) by Benoit Jubin.
%
% \chapter{Disclaimer and Trademarks}
%
% Information in this document is subject to change without notice and does not
% represent a commitment on the part of Norman Megill.
% \vspace{2ex}
%
% \noindent Norman D. Megill makes no warranties, either express or implied,
% regarding the Metamath computer software package.
%
% \vspace{2ex}
%
% \noindent Any trademarks mentioned in this book are the property of
% their respective owners.  The name ``Metamath'' is a trademark of
% Norman Megill.
%
\cleardoublepage
\phantomsection  % fixes the link anchor
\addcontentsline{toc}{chapter}{\bibname}

\bibliography{metamath}
%\input{metamath.bbl}

\raggedright
\cleardoublepage
\phantomsection % fixes the link anchor
\addcontentsline{toc}{chapter}{\indexname}
%\printindex   ??
\input{metamath.ind}

\end{document}



\raggedright
\cleardoublepage
\phantomsection % fixes the link anchor
\addcontentsline{toc}{chapter}{\indexname}
%\printindex   ??
% metamath.tex - Version of 24-May-2019
% If you change the date above, also change the "Printed date" below.
% SPDX-License-Identifier: CC0-1.0
%
%                              PUBLIC DOMAIN
%
% This file (specifically, the version of this file with the above date)
% has been released into the Public Domain per the
% Creative Commons CC0 1.0 Universal (CC0 1.0) Public Domain Dedication
% https://creativecommons.org/publicdomain/zero/1.0/
%
% The public domain release applies worldwide.  In case this is not
% legally possible, the right is granted to use the work for any purpose,
% without any conditions, unless such conditions are required by law.
%
% Several short, attributed quotations from copyrighted works
% appear in this file under the ``fair use'' provision of Section 107 of
% the United States Copyright Act (Title 17 of the {\em United States
% Code}).  The public-domain status of this file is not applicable to
% those quotations.
%
% Norman Megill - email: nm(at)alum(dot)mit(dot)edu
%
% David A. Wheeler also donates his improvements to this file to the
% public domain per the CC0.  He works at the Institute for Defense Analyses
% (IDA), but IDA has agreed that this Metamath work is outside its "lane"
% and is not a work by IDA.  This was specifically confirmed by
% Margaret E. Myers (Division Director of the Information Technology
% and Systems Division) on 2019-05-24 and by Ben Lindorf (General Counsel)
% on 2019-05-22.

% This file, 'metamath.tex', is self-contained with everything needed to
% generate the the PDF file 'metamath.pdf' (the _Metamath_ book) on
% standard LaTeX 2e installations.  The auxiliary files are embedded with
% "filecontents" commands.  To generate metamath.pdf file, run these
% commands under Linux or Cygwin in the directory that contains
% 'metamath.tex':
%
%   rm -f realref.sty metamath.bib
%   touch metamath.ind
%   pdflatex metamath
%   pdflatex metamath
%   bibtex metamath
%   makeindex metamath
%   pdflatex metamath
%   pdflatex metamath
%
% The warnings that occur in the initial runs of pdflatex can be ignored.
% For the final run,
%
%   egrep -i 'error|warn' metamath.log
%
% should show exactly these 5 warnings:
%
%   LaTeX Warning: File `realref.sty' already exists on the system.
%   LaTeX Warning: File `metamath.bib' already exists on the system.
%   LaTeX Font Warning: Font shape `OMS/cmtt/m/n' undefined
%   LaTeX Font Warning: Font shape `OMS/cmtt/bx/n' undefined
%   LaTeX Font Warning: Some font shapes were not available, defaults
%       substituted.
%
% Search for "Uncomment" below if you want to suppress hyperlink boxes
% in the PDF output file
%
% TYPOGRAPHICAL NOTES:
% * It is customary to use an en dash (--) to "connect" names of different
%   people (and to denote ranges), and use a hyphen (-) for a
%   single compound name. Examples of connected multiple people are
%   Zermelo--Fraenkel, Schr\"{o}der--Bernstein, Tarski--Grothendieck,
%   Hewlett--Packard, and Backus--Naur.  Examples of a single person with
%   a compound name include Levi-Civita, Mittag-Leffler, and Burali-Forti.
% * Use non-breaking spaces after page abbreviations, e.g.,
%   p.~\pageref{note2002}.
%
% --------------------------- Start of realref.sty -----------------------------
\begin{filecontents}{realref.sty}
% Save the following as realref.sty.
% You can then use it with \usepackage{realref}
%
% This has \pageref jumping to the page on which the ref appears,
% \ref jumping to the point of the anchor, and \sectionref
% jumping to the start of section.
%
% Author:  Anthony Williams
%          Software Engineer
%          Nortel Networks Optical Components Ltd
% Date:    9 Nov 2001 (posted to comp.text.tex)
%
% The following declaration was made by Anthony Williams on
% 24 Jul 2006 (private email to Norman Megill):
%
%   ``I hereby donate the code for realref.sty posted on the
%   comp.text.tex newsgroup on 9th November 2001, accessible from
%   http://groups.google.com/group/comp.text.tex/msg/5a0e1cc13ea7fbb2
%   to the public domain.''
%
\ProvidesPackage{realref}
\RequirePackage[plainpages=false,pdfpagelabels=true]{hyperref}
\def\realref@anchorname{}
\AtBeginDocument{%
% ensure every label is a possible hyperlink target
\let\realref@oldrefstepcounter\refstepcounter%
\DeclareRobustCommand{\refstepcounter}[1]{\realref@oldrefstepcounter{#1}
\edef\realref@anchorname{\string #1.\@currentlabel}%
}%
\let\realref@oldlabel\label%
\DeclareRobustCommand{\label}[1]{\realref@oldlabel{#1}\hypertarget{#1}{}%
\@bsphack\protected@write\@auxout{}{%
    \string\expandafter\gdef\protect\csname
    page@num.#1\string\endcsname{\thepage}%
    \string\expandafter\gdef\protect\csname
    ref@num.#1\string\endcsname{\@currentlabel}%
    \string\expandafter\gdef\protect\csname
    sectionref@name.#1\string\endcsname{\realref@anchorname}%
}\@esphack}%
\DeclareRobustCommand\pageref[1]{{\edef\a{\csname
            page@num.#1\endcsname}\expandafter\hyperlink{page.\a}{\a}}}%
\DeclareRobustCommand\ref[1]{{\edef\a{\csname
            ref@num.#1\endcsname}\hyperlink{#1}{\a}}}%
\DeclareRobustCommand\sectionref[1]{{\edef\a{\csname
            ref@num.#1\endcsname}\edef\b{\csname
            sectionref@name.#1\endcsname}\hyperlink{\b}{\a}}}%
}
\end{filecontents}
% ---------------------------- End of realref.sty ------------------------------

% --------------------------- Start of metamath.bib -----------------------------
\begin{filecontents}{metamath.bib}
@book{Albers, editor = "Donald J. Albers and G. L. Alexanderson",
  title = "Mathematical People",
  publisher = "Contemporary Books, Inc.",
  address = "Chicago",
  note = "[QA28.M37]",
  year = 1985 }
@book{Anderson, author = "Alan Ross Anderson and Nuel D. Belnap",
  title = "Entailment",
  publisher = "Princeton University Press",
  address = "Princeton",
  volume = 1,
  note = "[QA9.A634 1975 v.1]",
  year = 1975}
@book{Barrow, author = "John D. Barrow",
  title = "Theories of Everything:  The Quest for Ultimate Explanation",
  publisher = "Oxford University Press",
  address = "Oxford",
  note = "[Q175.B225]",
  year = 1991 }
@book{Behnke,
  editor = "H. Behnke and F. Backmann and K. Fladt and W. S{\"{u}}ss",
  title = "Fundamentals of Mathematics",
  volume = "I",
  publisher = "The MIT Press",
  address = "Cambridge, Massachusetts",
  note = "[QA37.2.B413]",
  year = 1974 }
@book{Bell, author = "J. L. Bell and M. Machover",
  title = "A Course in Mathematical Logic",
  publisher = "North-Holland",
  address = "Amsterdam",
  note = "[QA9.B3953]",
  year = 1977 }
@inproceedings{Blass, author = "Andrea Blass",
  title = "The Interaction Between Category Theory and Set Theory",
  pages = "5--29",
  booktitle = "Mathematical Applications of Category Theory (Proceedings
     of the Special Session on Mathematical Applications
     Category Theory, 89th Annual Meeting of the American Mathematical
     Society, held in Denver, Colorado January 5--9, 1983)",
  editor = "John Walter Gray",
  year = 1983,
  note = "[QA169.A47 1983]",
  publisher = "American Mathematical Society",
  address = "Providence, Rhode Island"}
@proceedings{Bledsoe, editor = "W. W. Bledsoe and D. W. Loveland",
  title = "Automated Theorem Proving:  After 25 Years (Proceedings
     of the Special Session on Automatic Theorem Proving,
     89th Annual Meeting of the American Mathematical
     Society, held in Denver, Colorado January 5--9, 1983)",
  year = 1983,
  note = "[QA76.9.A96.S64 1983]",
  publisher = "American Mathematical Society",
  address = "Providence, Rhode Island" }
@book{Boolos, author = "George S. Boolos and Richard C. Jeffrey",
  title = "Computability and Log\-ic",
  publisher = "Cambridge University Press",
  edition = "third",
  address = "Cambridge",
  note = "[QA9.59.B66 1989]",
  year = 1989 }
@book{Campbell, author = "John Campbell",
  title = "Programmer's Progress",
  publisher = "White Star Software",
  address = "Box 51623, Palo Alto, CA 94303",
  year = 1991 }
@article{DBLP:journals/corr/Carneiro14,
  author    = {Mario Carneiro},
  title     = {Conversion of {HOL} Light proofs into Metamath},
  journal   = {CoRR},
  volume    = {abs/1412.8091},
  year      = {2014},
  url       = {http://arxiv.org/abs/1412.8091},
  archivePrefix = {arXiv},
  eprint    = {1412.8091},
  timestamp = {Mon, 13 Aug 2018 16:47:05 +0200},
  biburl    = {https://dblp.org/rec/bib/journals/corr/Carneiro14},
  bibsource = {dblp computer science bibliography, https://dblp.org}
}
@article{CarneiroND,
  author    = {Mario Carneiro},
  title     = {Natural Deductions in the Metamath Proof Language},
  url       = {http://us.metamath.org/ocat/natded.pdf},
  year      = 2014
}
@inproceedings{Chou, author = "Shang-Ching Chou",
  title = "Proving Elementary Geometry Theorems Using {W}u's Algorithm",
  pages = "243--286",
  booktitle = "Automated Theorem Proving:  After 25 Years (Proceedings
     of the Special Session on Automatic Theorem Proving,
     89th Annual Meeting of the American Mathematical
     Society, held in Denver, Colorado January 5--9, 1983)",
  editor = "W. W. Bledsoe and D. W. Loveland",
  year = 1983,
  note = "[QA76.9.A96.S64 1983]",
  publisher = "American Mathematical Society",
  address = "Providence, Rhode Island" }
@book{Clemente, author = "Daniel Clemente Laboreo",
  title = "Introduction to natural deduction",
  year = 2014,
  url = "http://www.danielclemente.com/logica/dn.en.pdf" }
@incollection{Courant, author = "Richard Courant and Herbert Robbins",
  title = "Topology",
  pages = "573--590",
  booktitle = "The World of Mathematics, Volume One",
  editor = "James R. Newman",
  publisher = "Simon and Schuster",
  address = "New York",
  note = "[QA3.W67 1988]",
  year = 1956 }
@book{Curry, author = "Haskell B. Curry",
  title = "Foundations of Mathematical Logic",
  publisher = "Dover Publications, Inc.",
  address = "New York",
  note = "[QA9.C976 1977]",
  year = 1977 }
@book{Davis, author = "Philip J. Davis and Reuben Hersh",
  title = "The Mathematical Experience",
  publisher = "Birkh{\"{a}}user Boston",
  address = "Boston",
  note = "[QA8.4.D37 1982]",
  year = 1981 }
@incollection{deMillo,
  author = "Richard de Millo and Richard Lipton and Alan Perlis",
  title = "Social Processes and Proofs of Theorems and Programs",
  pages = "267--285",
  booktitle = "New Directions in the Philosophy of Mathematics",
  editor = "Thomas Tymoczko",
  publisher = "Birkh{\"{a}}user Boston, Inc.",
  address = "Boston",
  note = "[QA8.6.N48 1986]",
  year = 1986 }
@book{Edwards, author = "Robert E. Edwards",
  title = "A Formal Background to Mathematics",
  publisher = "Springer-Verlag",
  address = "New York",
  note = "[QA37.2.E38 v.1a]",
  year = 1979 }
@book{Enderton, author = "Herbert B. Enderton",
  title = "Elements of Set Theory",
  publisher = "Academic Press, Inc.",
  address = "San Diego",
  note = "[QA248.E5]",
  year = 1977 }
@book{Goodstein, author = "R. L. Goodstein",
  title = "Development of Mathematical Logic",
  publisher = "Springer-Verlag New York Inc.",
  address = "New York",
  note = "[QA9.G6554]",
  year = 1971 }
@book{Guillen, author = "Michael Guillen",
  title = "Bridges to Infinity",
  publisher = "Jeremy P. Tarcher, Inc.",
  address = "Los Angeles",
  note = "[QA93.G8]",
  year = 1983 }
@book{Hamilton, author = "Alan G. Hamilton",
  title = "Logic for Mathematicians",
  edition = "revised",
  publisher = "Cambridge University Press",
  address = "Cambridge",
  note = "[QA9.H298]",
  year = 1988 }
@unpublished{Harrison, author = "John Robert Harrison",
  title = "Metatheory and Reflection in Theorem Proving:
    A Survey and Critique",
  note = "Technical Report
    CRC-053.
    SRI Cambridge,
    Millers Yard, Cambridge, UK,
    1995.
    Available on the Web as
{\verb+http:+}\-{\verb+//www.cl.cam.ac.uk/users/jrh/papers/reflect.html+}"}
@TECHREPORT{Harrison-thesis,
        author          = "John Robert Harrison",
        title           = "Theorem Proving with the Real Numbers",
        institution   = "University of Cambridge Computer
                         Lab\-o\-ra\-to\-ry",
        address         = "New Museums Site, Pembroke Street, Cambridge,
                           CB2 3QG, UK",
        year            = 1996,
        number          = 408,
        type            = "Technical Report",
        note            = "Author's PhD thesis,
   available on the Web at
{\verb+http:+}\-{\verb+//www.cl.cam.ac.uk+}\-{\verb+/users+}\-{\verb+/jrh+}%
\-{\verb+/papers+}\-{\verb+/thesis.html+}"}
@book{Herrlich, author = "Horst Herrlich and George E. Strecker",
  title = "Category Theory:  An Introduction",
  publisher = "Allyn and Bacon Inc.",
  address = "Boston",
  note = "[QA169.H567]",
  year = 1973 }
@article{Hindley, author = "J. Roger Hindley and David Meredith",
  title = "Principal Type-Schemes and Condensed Detachment",
  journal = "The Journal of Symbolic Logic",
  volume = 55,
  year = 1990,
  note = "[QA.J87]",
  pages = "90--105" }
@book{Hofstadter, author = "Douglas R. Hofstadter",
  title = "G{\"{o}}del, Escher, Bach",
  publisher = "Basic Books, Inc.",
  address = "New York",
  note = "[QA9.H63 1980]",
  year = 1979 }
@article{Indrzejczak, author= "Andrzej Indrzejczak",
  title = "Natural Deduction, Hybrid Systems and Modal Logic",
  journal = "Trends in Logic",
  volume = 30,
  publisher = "Springer",
  year = 2010 }
@article{Kalish, author = "D. Kalish and R. Montague",
  title = "On {T}arski's Formalization of Predicate Logic with Identity",
  journal = "Archiv f{\"{u}}r Mathematische Logik und Grundlagenfor\-schung",
  volume = 7,
  year = 1965,
  note = "[QA.A673]",
  pages = "81--101" }
@article{Kalman, author = "J. A. Kalman",
  title = "Condensed Detachment as a Rule of Inference",
  journal = "Studia Logica",
  volume = 42,
  number = 4,
  year = 1983,
  note = "[B18.P6.S933]",
  pages = "443-451" }
@book{Kline, author = "Morris Kline",
  title = "Mathematical Thought from Ancient to Modern Times",
  publisher = "Oxford University Press",
  address = "New York",
  note = "[QA21.K516 1990 v.3]",
  year = 1972 }
@book{Klinel, author = "Morris Kline",
  title = "Mathematics, The Loss of Certainty",
  publisher = "Oxford University Press",
  address = "New York",
  note = "[QA21.K525]",
  year = 1980 }
@book{Kramer, author = "Edna E. Kramer",
  title = "The Nature and Growth of Modern Mathematics",
  publisher = "Princeton University Press",
  address = "Princeton, New Jersey",
  note = "[QA93.K89 1981]",
  year = 1981 }
@article{Knill, author = "Oliver Knill",
  title = "Some Fundamental Theorems in Mathematics",
  year = "2018",
  url = "https://arxiv.org/abs/1807.08416" }
@book{Landau, author = "Edmund Landau",
  title = "Foundations of Analysis",
  publisher = "Chelsea Publishing Company",
  address = "New York",
  edition = "second",
  note = "[QA241.L2541 1960]",
  year = 1960 }
@article{Leblanc, author = "Hugues Leblanc",
  title = "On {M}eyer and {L}ambert's Quantificational Calculus {FQ}",
  journal = "The Journal of Symbolic Logic",
  volume = 33,
  year = 1968,
  note = "[QA.J87]",
  pages = "275--280" }
@article{Lejewski, author = "Czeslaw Lejewski",
  title = "On Implicational Definitions",
  journal = "Studia Logica",
  volume = 8,
  year = 1958,
  note = "[B18.P6.S933]",
  pages = "189--208" }
@book{Levy, author = "Azriel Levy",
  title = "Basic Set Theory",
  publisher = "Dover Publications",
  address = "Mineola, NY",
  year = "2002"
}
@book{Margaris, author = "Angelo Margaris",
  title = "First Order Mathematical Logic",
  publisher = "Blaisdell Publishing Company",
  address = "Waltham, Massachusetts",
  note = "[QA9.M327]",
  year = 1967}
@book{Manin, author = "Yu I. Manin",
  title = "A Course in Mathematical Logic",
  publisher = "Springer-Verlag",
  address = "New York",
  note = "[QA9.M29613]",
  year = "1977" }
@article{Mathias, author = "Adrian R. D. Mathias",
  title = "A Term of Length 4,523,659,424,929",
  journal = "Synthese",
  volume = 133,
  year = 2002,
  note = "[Q.S993]",
  pages = "75--86" }
@article{Megill, author = "Norman D. Megill",
  title = "A Finitely Axiomatized Formalization of Predicate Calculus
     with Equality",
  journal = "Notre Dame Journal of Formal Logic",
  volume = 36,
  year = 1995,
  note = "[QA.N914]",
  pages = "435--453" }
@unpublished{Megillc, author = "Norman D. Megill",
  title = "A Shorter Equivalent of the Axiom of Choice",
  month = "June",
  note = "Unpublished",
  year = 1991 }
@article{MegillBunder, author = "Norman D. Megill and Martin W.
    Bunder",
  title = "Weaker {D}-Complete Logics",
  journal = "Journal of the IGPL",
  volume = 4,
  year = 1996,
  pages = "215--225",
  note = "Available on the Web at
{\verb+http:+}\-{\verb+//www.mpi-sb.mpg.de+}\-{\verb+/igpl+}%
\-{\verb+/Journal+}\-{\verb+/V4-2+}\-{\verb+/#Megill+}"}
}
@book{Mendelson, author = "Elliott Mendelson",
  title = "Introduction to Mathematical Logic",
  edition = "second",
  publisher = "D. Van Nostrand Company, Inc.",
  address = "New York",
  note = "[QA9.M537 1979]",
  year = 1979 }
@article{Meredith, author = "David Meredith",
  title = "In Memoriam {C}arew {A}rthur {M}eredith (1904-1976)",
  journal = "Notre Dame Journal of Formal Logic",
  volume = 18,
  year = 1977,
  note = "[QA.N914]",
  pages = "513--516" }
@article{CAMeredith, author = "C. A. Meredith",
  title = "Single Axioms for the Systems ({C},{N}), ({C},{O}) and ({A},{N})
      of the Two-Valued Propositional Calculus",
  journal = "The Journal of Computing Systems",
  volume = 3,
  year = 1953,
  pages = "155--164" }
@article{Monk, author = "J. Donald Monk",
  title = "Provability With Finitely Many Variables",
  journal = "The Journal of Symbolic Logic",
  volume = 27,
  year = 1971,
  note = "[QA.J87]",
  pages = "353--358" }
@article{Monks, author = "J. Donald Monk",
  title = "Substitutionless Predicate Logic With Identity",
  journal = "Archiv f{\"{u}}r Mathematische Logik und Grundlagenfor\-schung",
  volume = 7,
  year = 1965,
  pages = "103--121" }
  %% Took out this from above to prevent LaTeX underfull warning:
  % note = "[QA.A673]",
@book{Moore, author = "A. W. Moore",
  title = "The Infinite",
  publisher = "Routledge",
  address = "New York",
  note = "[BD411.M59]",
  year = 1989}
@book{Munkres, author = "James R. Munkres",
  title = "Topology: A First Course",
  publisher = "Prentice-Hall, Inc.",
  address = "Englewood Cliffs, New Jersey",
  note = "[QA611.M82]",
  year = 1975}
@article{Nemesszeghy, author = "E. Z. Nemesszeghy and E. A. Nemesszeghy",
  title = "On Strongly Creative Definitions:  A Reply to {V}. {F}. {R}ickey",
  journal = "Logique et Analyse (N.\ S.)",
  year = 1977,
  volume = 20,
  note = "[BC.L832]",
  pages = "111--115" }
@unpublished{Nemeti, author = "N{\'{e}}meti, I.",
  title = "Algebraizations of Quantifier Logics, an Overview",
  note = "Version 11.4, preprint, Mathematical Institute, Budapest,
    1994.  A shortened version without proofs appeared in
    ``Algebraizations of quantifier logics, an introductory overview,''
   {\em Studia Logica}, 50:485--569, 1991 [B18.P6.S933]"}
@article{Pavicic, author = "M. Pavi{\v{c}}i{\'{c}}",
  title = "A New Axiomatization of Unified Quantum Logic",
  journal = "International Journal of Theoretical Physics",
  year = 1992,
  volume = 31,
  note = "[QC.I626]",
  pages = "1753 --1766" }
@book{Penrose, author = "Roger Penrose",
  title = "The Emperor's New Mind",
  publisher = "Oxford University Press",
  address = "New York",
  note = "[Q335.P415]",
  year = 1989 }
@book{PetersonI, author = "Ivars Peterson",
  title = "The Mathematical Tourist",
  publisher = "W. H. Freeman and Company",
  address = "New York",
  note = "[QA93.P475]",
  year = 1988 }
@article{Peterson, author = "Jeremy George Peterson",
  title = "An automatic theorem prover for substitution and detachment systems",
  journal = "Notre Dame Journal of Formal Logic",
  volume = 19,
  year = 1978,
  note = "[QA.N914]",
  pages = "119--122" }
@book{Quine, author = "Willard Van Orman Quine",
  title = "Set Theory and Its Logic",
  edition = "revised",
  publisher = "The Belknap Press of Harvard University Press",
  address = "Cambridge, Massachusetts",
  note = "[QA248.Q7 1969]",
  year = 1969 }
@article{Robinson, author = "J. A. Robinson",
  title = "A Machine-Oriented Logic Based on the Resolution Principle",
  journal = "Journal of the Association for Computing Machinery",
  year = 1965,
  volume = 12,
  pages = "23--41" }
@article{RobinsonT, author = "T. Thacher Robinson",
  title = "Independence of Two Nice Sets of Axioms for the Propositional
    Calculus",
  journal = "The Journal of Symbolic Logic",
  volume = 33,
  year = 1968,
  note = "[QA.J87]",
  pages = "265--270" }
@book{Rucker, author = "Rudy Rucker",
  title = "Infinity and the Mind:  The Science and Philosophy of the
    Infinite",
  publisher = "Bantam Books, Inc.",
  address = "New York",
  note = "[QA9.R79 1982]",
  year = 1982 }
@book{Russell, author = "Bertrand Russell",
  title = "Mysticism and Logic, and Other Essays",
  publisher = "Barnes \& Noble Books",
  address = "Totowa, New Jersey",
  note = "[B1649.R963.M9 1981]",
  year = 1981 }
@article{Russell2, author = "Bertrand Russell",
  title = "Recent Work on the Principles of Mathematics",
  journal = "International Monthly",
  volume = 4,
  year = 1901,
  pages = "84"}
@article{Schmidt, author = "Eric Schmidt",
  title = "Reductions in Norman Megill's axiom system for complex numbers",
  url = "http://us.metamath.org/downloads/schmidt-cnaxioms.pdf",
  year = "2012" }
@book{Shoenfield, author = "Joseph R. Shoenfield",
  title = "Mathematical Logic",
  publisher = "Addison-Wesley Publishing Company, Inc.",
  address = "Reading, Massachusetts",
  year = 1967,
  note = "[QA9.S52]" }
@book{Smullyan, author = "Raymond M. Smullyan",
  title = "Theory of Formal Systems",
  publisher = "Princeton University Press",
  address = "Princeton, New Jersey",
  year = 1961,
  note = "[QA248.5.S55]" }
@book{Solow, author = "Daniel Solow",
  title = "How to Read and Do Proofs:  An Introduction to Mathematical
    Thought Process",
  publisher = "John Wiley \& Sons",
  address = "New York",
  year = 1982,
  note = "[QA9.S577]" }
@book{Stark, author = "Harold M. Stark",
  title = "An Introduction to Number Theory",
  publisher = "Markham Publishing Company",
  address = "Chicago",
  note = "[QA241.S72 1978]",
  year = 1970 }
@article{Swart, author = "E. R. Swart",
  title = "The Philosophical Implications of the Four-Color Problem",
  journal = "American Mathematical Monthly",
  year = 1980,
  volume = 87,
  month = "November",
  note = "[QA.A5125]",
  pages = "697--707" }
@book{Szpiro, author = "George G. Szpiro",
  title = "Poincar{\'{e}}'s Prize: The Hundred-Year Quest to Solve One
    of Math's Greatest Puzzles",
  publisher = "Penguin Books Ltd",
  address = "London",
  note = "[QA43.S985 2007]",
  year = 2007}
@book{Takeuti, author = "Gaisi Takeuti and Wilson M. Zaring",
  title = "Introduction to Axiomatic Set Theory",
  edition = "second",
  publisher = "Springer-Verlag New York Inc.",
  address = "New York",
  note = "[QA248.T136 1982]",
  year = 1982}
@inproceedings{Tarski, author = "Alfred Tarski",
  title = "What is Elementary Geometry",
  pages = "16--29",
  booktitle = "The Axiomatic Method, with Special Reference to Geometry and
     Physics (Proceedings of an International Symposium held at the University
     of California, Berkeley, December 26, 1957 --- January 4, 1958)",
  editor = "Leon Henkin and Patrick Suppes and Alfred Tarski",
  year = 1959,
  publisher = "North-Holland Publishing Company",
  address = "Amsterdam"}
@article{Tarski1965, author = "Alfred Tarski",
  title = "A Simplified Formalization of Predicate Logic with Identity",
  journal = "Archiv f{\"{u}}r Mathematische Logik und Grundlagenforschung",
  volume = 7,
  year = 1965,
  note = "[QA.A673]",
  pages = "61--79" }
@book{Tymoczko,
  title = "New Directions in the Philosophy of Mathematics",
  editor = "Thomas Tymoczko",
  publisher = "Birkh{\"{a}}user Boston, Inc.",
  address = "Boston",
  note = "[QA8.6.N48 1986]",
  year = 1986 }
@incollection{Wang,
  author = "Hao Wang",
  title = "Theory and Practice in Mathematics",
  pages = "129--152",
  booktitle = "New Directions in the Philosophy of Mathematics",
  editor = "Thomas Tymoczko",
  publisher = "Birkh{\"{a}}user Boston, Inc.",
  address = "Boston",
  note = "[QA8.6.N48 1986]",
  year = 1986 }
@manual{Webster,
  title = "Webster's New Collegiate Dictionary",
  organization = "G. \& C. Merriam Co.",
  address = "Springfield, Massachusetts",
  note = "[PE1628.W4M4 1977]",
  year = 1977 }
@manual{Whitehead, author = "Alfred North Whitehead",
  title = "An Introduction to Mathematics",
  year = 1911 }
@book{PM, author = "Alfred North Whitehead and Bertrand Russell",
  title = "Principia Mathematica",
  edition = "second",
  publisher = "Cambridge University Press",
  address = "Cambridge",
  year = "1927",
  note = "(3 vols.) [QA9.W592 1927]" }
@article{DBLP:journals/corr/Whalen16,
  author    = {Daniel Whalen},
  title     = {Holophrasm: a neural Automated Theorem Prover for higher-order logic},
  journal   = {CoRR},
  volume    = {abs/1608.02644},
  year      = {2016},
  url       = {http://arxiv.org/abs/1608.02644},
  archivePrefix = {arXiv},
  eprint    = {1608.02644},
  timestamp = {Mon, 13 Aug 2018 16:46:19 +0200},
  biburl    = {https://dblp.org/rec/bib/journals/corr/Whalen16},
  bibsource = {dblp computer science bibliography, https://dblp.org} }
@article{Wiedijk-revisited,
  author = {Freek Wiedijk},
  title = {The QED Manifesto Revisited},
  year = {2007},
  url = {http://mizar.org/trybulec65/8.pdf} }
@book{Wolfram,
  author = "Stephen Wolfram",
  title = "Mathematica:  A System for Doing Mathematics by Computer",
  edition = "second",
  publisher = "Addison-Wesley Publishing Co.",
  address = "Redwood City, California",
  note = "[QA76.95.W65 1991]",
  year = 1991 }
@book{Wos, author = "Larry Wos and Ross Overbeek and Ewing Lusk and Jim Boyle",
  title = "Automated Reasoning:  Introduction and Applications",
  edition = "second",
  publisher = "McGraw-Hill, Inc.",
  address = "New York",
  note = "[QA76.9.A96.A93 1992]",
  year = 1992 }

%
%
%[1] Church, Alonzo, Introduction to Mathematical Logic,
% Volume 1, Princeton University Press, Princeton, N. J., 1956.
%
%[2] Cohen, Paul J., Set Theory and the Continuum Hypothesis,
% W. A. Benjamin, Inc., Reading, Mass., 1966.
%
%[3] Hamilton, Alan G., Logic for Mathematicians, Cambridge
% University Press,
% Cambridge, 1988.

%[6] Kleene, Stephen Cole, Introduction to Metamathematics, D.  Van
% Nostrand Company, Inc., Princeton (1952).

%[13] Tarski, Alfred, "A simplified formalization of predicate
% logic with identity," Archiv fur Mathematische Logik und
% Grundlagenforschung, vol. 7 (1965), pp. 61-79.

%[14] Tarski, Alfred and Steven Givant, A Formalization of Set
% Theory Without Variables, American Mathematical Society Colloquium
% Publications, vol. 41, American Mathematical Society,
% Providence, R. I., 1987.

%[15] Zeman, J. J., Modal Logic, Oxford University Press, Oxford, 1973.
\end{filecontents}
% --------------------------- End of metamath.bib -----------------------------


%Book: Metamath
%Author:  Norman Megill Email:  nm at alum.mit.edu
%Author:  David A. Wheeler Email:  dwheeler at dwheeler.com

% A book template example
% http://www.stsci.edu/ftp/software/tex/bookstuff/book.template

\documentclass[leqno]{book} % LaTeX 2e. 10pt. Use [leqno,12pt] for 12pt
% hyperref 2002/05/27 v6.72r  (couldn't get pagebackref to work)
\usepackage[plainpages=false,pdfpagelabels=true]{hyperref}

\usepackage{needspace}     % Enable control over page breaks
\usepackage{breqn}         % automatic equation breaking
\usepackage{microtype}     % microtypography, reduces hyphenation

% Packages for flexible tables.  We need to be able to
% wrap text within a cell (with automatically-determined widths) AND
% split a table automatically across multiple pages.
% * "tabularx" wraps text in cells but only 1 page
% * "longtable" goes across pages but by itself is incompatible with tabularx
% * "ltxtable" combines longtable and tabularx, but table contents
%    must be in a separate file.
% * "ltablex" combines tabularx and longtable - must install specially
% * "booktabs" is recommended as a way to improve the look of tables,
%   but doesn't add these capabilities.
% * "tabu" much more capable and seems to be recommended. So use that.

\usepackage{makecell}      % Enable forced line splits within a table cell
\usepackage{longtable}     % Enable multi-page tables
\usepackage{tabu}          % Multi-page tables with wrapped text in a cell

% You can find more Tex packages using commands like:
% tlmgr search --file tabu.sty
% find /usr/share/texmf-dist/ -name '*tab*'
%
%%%%%%%%%%%%%%%%%%%%%%%%%%%%%%%%%%%%%%%%%%%%%%%%%%%%%%%%%%%%%%%%%%%%%%%%%%%%
% Uncomment the next 3 lines to suppress boxes and colors on the hyperlinks
%%%%%%%%%%%%%%%%%%%%%%%%%%%%%%%%%%%%%%%%%%%%%%%%%%%%%%%%%%%%%%%%%%%%%%%%%%%%
%\hypersetup{
%colorlinks,citecolor=black,filecolor=black,linkcolor=black,urlcolor=black
%}
%
\usepackage{realref}

% Restarting page numbers: try?
%   \printglossary
%   \cleardoublepage
%   \pagenumbering{arabic}
%   \setcounter{page}{1}    ???needed
%   \include{chap1}

% not used:
% \def\R2Lurl#1#2{\mbox{\href{#1}\texttt{#2}}}

\usepackage{amssymb}

% Version 1 of book: margins: t=.4, b=.2, ll=.4, rr=.55
% \usepackage{anysize}
% % \papersize{<height>}{<width>}
% % \marginsize{<left>}{<right>}{<top>}{<bottom>}
% \papersize{9in}{6in}
% % l/r 0.6124-0.6170 works t/b 0.2418-0.3411 = 192pp. 0.2926-03118=exact
% \marginsize{0.7147in}{0.5147in}{0.4012in}{0.2012in}

\usepackage{anysize}
% \papersize{<height>}{<width>}
% \marginsize{<left>}{<right>}{<top>}{<bottom>}
\papersize{9in}{6in}
% l/r 0.85in&0.6431-0.6539 works t/b ?-?
%\marginsize{0.85in}{0.6485in}{0.55in}{0.35in}
\marginsize{0.8in}{0.65in}{0.5in}{0.3in}

% \usepackage[papersize={3.6in,4.8in},hmargin=0.1in,vmargin={0.1in,0.1in}]{geometry}  % page geometry
\usepackage{special-settings}

\raggedbottom
\makeindex

\begin{document}
% Discourage page widows and orphans:
\clubpenalty=300
\widowpenalty=300

%%%%%%% load in AMS fonts %%%%%%% % LaTeX 2.09 - obsolete in LaTeX 2e
%\input{amssym.def}
%\input{amssym.tex}
%\input{c:/texmf/tex/plain/amsfonts/amssym.def}
%\input{c:/texmf/tex/plain/amsfonts/amssym.tex}

\bibliographystyle{plain}
\pagenumbering{roman}
\pagestyle{headings}

\thispagestyle{empty}

\hfill
\vfill

\begin{center}
{\LARGE\bf Metamath} \\
\vspace{1ex}
{\large A Computer Language for Mathematical Proofs} \\
\vspace{7ex}
{\large Norman Megill} \\
\vspace{7ex}
with extensive revisions by \\
\vspace{1ex}
{\large David A. Wheeler} \\
\vspace{7ex}
% Printed date. If changing the date below, also fix the date at the beginning.
2019-05-24
\end{center}

\vfill
\hfill

\newpage
\thispagestyle{empty}

\hfill
\vfill

\begin{center}
$\sim$\ {\sc Public Domain}\ $\sim$

\vspace{2ex}
This book (including its later revisions)
has been released into the Public Domain by Norman Megill per the
Creative Commons CC0 1.0 Universal (CC0 1.0) Public Domain Dedication
(\url{https://creativecommons.org/publicdomain/zero/1.0/}).
David A. Wheeler has done the same.
The public domain release applies worldwide.  In case this is not
legally possible, the right is granted to use the work for any purpose,
without any conditions, unless such conditions are required by law.

\vspace{3ex}
Several short, attributed quotations from copyrighted works
appear in this book under the ``fair use'' provision of Section 107 of
the United States Copyright Act (Title 17 of the {\em United States
Code}).  The public-domain status of this book is not applicable to
those quotations.

\vspace{3ex}
Any trademarks used in this book are the property of their owners.

% QA76.9.L63.M??

% \vspace{1ex}
%
% \vspace{1ex}
% {\small Permission is granted to make and distribute verbatim copies of this
% book
% provided the copyright notice and this
% permission notice are preserved on all copies.}
%
% \vspace{1ex}
% {\small Permission is granted to copy and distribute modified versions of this
% book under the conditions for verbatim copying, provided that the
% entire
% resulting derived work is distributed under the terms of a permission
% notice
% identical to this one.}
%
% \vspace{1ex}
% {\small Permission is granted to copy and distribute translations of this
% book into another language, under the above conditions for modified
% versions,
% except that this permission notice may be stated in a translation
% approved by the
% author.}
%
% \vspace{1ex}
% %{\small   For a copy of the \LaTeX\ source files for this book, contact
% %the author.} \\
% \ \\
% \ \\

\vspace{7ex}
% ISBN: 1-4116-3724-0 \\
ISBN: 978-1-4116-3724-5 \\
{\ } \\
Lulu Press \\
Morrisville, North Carolina\\
USA


\hfill
\vfill

Norman Megill\\ 93 Bridge St., Lexington, MA 02421 \\
E-mail address: \texttt{nm{\char`\@}alum.mit.edu} \\
\vspace{7ex}
David A. Wheeler \\
E-mail address: \texttt{dwheeler{\char`\@}dwheeler.com} \\
% See notes added at end of Preface for revision history. \\
% For current information on the Metamath software see \\
\vspace{7ex}
\url{http://metamath.org}
\end{center}

% \newpage
% \thispagestyle{empty}
%
% \hfill
% \vfill
%
% \begin{center}
% {\it To my son Robin Dwight Megill}
% \end{center}
%
% \vfill
% \hfill
%
% \newpage

\tableofcontents
%\listoftables

\chapter*{Preface}
\markboth{PREFACE}{PREFACE}
\addcontentsline{toc}{section}{Preface}


% (For current information, see the notes added at the
% end of this preface on p.~\pageref{note2002}.)

\subsubsection{Overview}

Metamath\index{Metamath} is a computer language and an associated computer
program for archiving, verifying, and studying mathematical proofs at a very
detailed level.  The Metamath language incorporates no mathematics per se but
treats all mathematical statements as mere sequences of symbols.  You provide
Metamath with certain special sequences (axioms) that tell it what rules
of inference are allowed.  Metamath is not limited to any specific field of
mathematics.  The Metamath language is simple and robust, with an
almost total absence of hard-wired syntax, and
I\footnote{Unless otherwise noted, the words
``I,'' ``me,'' and ``my'' refer to Norman Megill\index{Megill, Norman}, while
``we,'' ``us,'' and ``our'' refer to Norman Megill and
David A. Wheeler\index{Wheeler, David A.}.}
believe that it
provides about the simplest possible framework that allows essentially all of
mathematics to be expressed with absolute rigor.

% index test
%\newcommand{\nn}[1]{#1n}
%\index{aaa@bbb}
%\index{abc!def}
%\index{abd|see{qqq}}
%\index{abe|nn}
%\index{abf|emph}
%\index{abg|(}
%\index{abg|)}

Using the Metamath language, you can build formal or mathematical
systems\index{formal system}\footnote{A formal or mathematical system consists
of a collection of symbols (such as $2$, $4$, $+$ and $=$), syntax rules that
describe how symbols may be combined to form a legal expression (called a
well-formed formula or {\em wff}, pronounced ``whiff''), some starting wffs
called axioms, and inference rules that describe how theorems may be derived
(proved) from the axioms.  A theorem is a mathematical fact such as $2+2=4$.
Strictly speaking, even an obvious fact such as this must be proved from
axioms to be formally acceptable to a mathematician.}\index{theorem}
\index{axiom}\index{rule}\index{well-formed formula (wff)} that involve
inferences from axioms.  Although a database is provided
that includes a recommended set of axioms for standard mathematics, if you
wish you can supply your own symbols, syntax, axioms, rules, and definitions.

The name ``Metamath'' was chosen to suggest that the language provides a
means for {\em describing} mathematics rather than {\em being} the
mathematics itself.  Actually in some sense any mathematical language is
metamathematical.  Symbols written on paper, or stored in a computer,
are not mathematics itself but rather a way of expressing mathematics.
For example ``7'' and ``VII'' are symbols for denoting the number seven
in Arabic and Roman numerals; neither {\em is} the number seven.

If you are able to understand and write computer programs, you should be able
to follow abstract mathematics with the aid of Metamath.  Used in conjunction
with standard textbooks, Metamath can guide you step-by-step towards an
understanding of abstract mathematics from a very rigorous viewpoint, even if
you have no formal abstract mathematics background.  By using a single,
consistent notation to express proofs, once you grasp its basic concepts
Metamath provides you with the ability to immediately follow and dissect
proofs even in totally unfamiliar areas.

Of course, just being able follow a proof will not necessarily give you an
intuitive familiarity with mathematics.  Memorizing the rules of chess does not
give you the ability to appreciate the game of a master, and knowing how the
notes on a musical score map to piano keys does not give you the ability to
hear in your head how it would sound.  But each of these can be a first step.

Metamath allows you to explore proofs in the sense that you can see the
theorem referenced at any step expanded in as much detail as you want, right
down to the underlying axioms of logic and set theory (in the case of the set
theory database provided).  While Metamath will not replace the higher-level
understanding that can only be acquired through exercises and hard work, being
able to see how gaps in a proof are filled in can give you increased
confidence that can speed up the learning process and save you time when you
get stuck.

The Metamath language breaks down a mathematical proof into its tiniest
possible parts.  These can be pieced together, like interlocking
pieces in a puzzle, only in a way that produces correct and absolutely rigorous
mathematics.

The nature of Metamath\index{Metamath} enforces very precise mathematical
thinking, similar to that involved in writing a computer program.  A crucial
difference, though, is that once a proof is verified (by the Metamath program)
to be correct, it is definitely correct; it can never have a hidden
``bug.''\index{computer program bugs}  After getting used to the kind of rigor
and accuracy provided by Metamath, you might even be tempted to
adopt the attitude that a proof should never be considered correct until it
has been verified by a computer, just as you would not completely trust a
manual calculation until you have verified it on a
calculator.

My goal
for Metamath was a system for describing and verifying
mathematics that is completely universal yet conceptually as simple as
possible.  In approaching mathematics from an axiomatic, formal viewpoint, I
wanted Metamath to be able to handle almost any mathematical system, not
necessarily with ease, but at least in principle and hopefully in practice. I
wanted it to verify proofs with absolute rigor, and for this reason Metamath
is what might be thought of as a ``compile-only'' language rather than an
algorithmic or Turing-machine language (Pascal, C, Prolog, Mathematica,
etc.).  In other words, a database written in the Metamath
language doesn't ``do'' anything; it merely exhibits mathematical knowledge
and permits this knowledge to be verified as being correct.  A program in an
algorithmic language can potentially have hidden bugs\index{computer program
bugs} as well as possibly being hard to understand.  But each token in a
Metamath database must be consistent with the database's earlier
contents according to simple, fixed rules.
If a database is verified
to be correct,\footnote{This includes
verification that a sequential list of proof steps results in the specified
theorem.} then the mathematical content is correct if the
verifier is correct and the axioms are correct.
The verification program could be incorrect, but the verification algorithm
is relatively simple (making it unlikely to be implemented incorrectly
by the Metamath program),
and there are over a dozen Metamath database verifiers
written by different people in different programming languages
(so these different verifiers can act as multiple reviewers of a database).
The most-used Metamath database, the Metamath Proof Explorer
(aka \texttt{set.mm}\index{set theory database (\texttt{set.mm})}%
\index{Metamath Proof Explorer}),
is currently verified by four different Metamath verifiers written by
four different people in four different languages, including the
original Metamath program described in this book.
The only ``bugs'' that can exist are in the statement of the axioms,
for example if the axioms are inconsistent (a famous problem shown to be
unsolvable by G\"{o}del's incompleteness theorem\index{G\"{o}del's
incompleteness theorem}).
However, real mathematical systems have very few axioms, and these can
be carefully studied.
All of this provides extraordinarily high confidence that the verified database
is in fact correct.

The Metamath program
doesn't prove theorems automatically but is designed to verify proofs
that you supply to it.
The underlying Metamath language is completely general and has no built-in,
preconceived notions about your formal system\index{formal system}, its logic
or its syntax.
For constructing proofs, the Metamath program has a Proof Assistant\index{Proof
Assistant} which helps you fill in some of a proof step's details, shows you
what choices you have at any step, and verifies the proof as you build it; but
you are still expected to provide the proof.

There are many other programs that can process or generate information
in the Metamath language, and more continue to be written.
This is in part because the Metamath language itself is very simple
and intentionally easy to automatically process.
Some programs, such as \texttt{mmj2}\index{mmj2}, include a proof assistant
that can automate some steps beyond what the Metamath program can do.
Mario Carneiro has developed an algorithm for converting proofs from
the OpenTheory interchange format, which can be translated to and from
any of the HOL family of proof languages (HOL4, HOL Light, ProofPower,
and Isabelle), into the
Metamath language.\cite{DBLP:journals/corr/Carneiro14}\index{Carneiro, Mario}.
Daniel Whalen has developed Holophrasm, which can automatically
prove many Metamath proofs using
machine learning\index{machine learning}\index{artificial intelligence}
approaches
(including multiple neural networks\index{neural networks})\cite{DBLP:journals/corr/Whalen16}\index{Whalen, Daniel}.
However,
a discussion of these other programs is beyond the scope of this book.

Like most computer languages, the Metamath\index{Metamath} language uses the
standard ({\sc ascii}) characters on a computer keyboard, so it cannot
directly represent many of the special symbols that mathematicians use.  A
useful feature of the Metamath program is its ability to convert its notation
into the \LaTeX\ typesetting language.\index{latex@{\LaTeX}}  This feature
lets you convert the {\sc ascii} tokens you've defined into standard
mathematical symbols, so you end up with symbols and formulas you are familiar
with instead of somewhat cryptic {\sc ascii} representations of them.
The Metamath program can also generate HTML\index{HTML}, making it easy
to view results on the web and to see related information by using
hypertext links.

Metamath is probably conceptually different from anything you've seen
before and some aspects may take some getting used to.  This book will
help you decide whether Metamath suits your specific needs.

\subsubsection{Setting Your Expectations}
It is important for you to understand what Metamath\index{Metamath} is and is
not.  As mentioned, the Metamath program
is {\em not} an automated theorem prover but
rather a proof verifier.  Developing a database can be tedious, hard work,
especially if you want to make the proofs as short as possible, but it becomes
easier as you build up a collection of useful theorems.  The purpose of
Metamath is simply to document existing mathematics in an absolutely rigorous,
computer-verifiable way, not to aid directly in the creation of new
mathematics.  It also is not a magic solution for learning abstract
mathematics, although it may be helpful to be able to actually see the implied
rigor behind what you are learning from textbooks, as well as providing hints
to work out proofs that you are stumped on.

As of this writing, a sizable set theory database has been developed to
provide a foundation for many fields of mathematics, but much more work would
be required to develop useful databases for specific fields.

Metamath\index{Metamath} ``knows no math;'' it just provides a framework in
which to express mathematics.  Its language is very small.  You can define two
kinds of symbols, constants\index{constant} and variables\index{variable}.
The only thing Metamath knows how to do is to substitute strings of symbols
for the variables\index{substitution!variable}\index{variable substitution} in
an expression based on instructions you provide it in a proof, subject to
certain constraints you specify for the variables.  Even the decimal
representation of a number is merely a string of certain constants (digits)
which together, in a specific context, correspond to whatever mathematical
object you choose to define for it; unlike other computer languages, there is
no actual number stored inside the computer.  In a proof, you in effect
instruct Metamath what symbol substitutions to make in previous axioms or
theorems and join a sequence of them together to result in the desired
theorem.  This kind of symbol manipulation captures the essence of mathematics
at a preaxiomatic level.

\subsubsection{Metamath and Mathematical Literature}

In advanced mathematical literature, proofs are usually presented in the form
of short outlines that often only an expert can follow.  This is partly out of
a desire for brevity, but it would also be unwise (even if it were practical)
to present proofs in complete formal detail, since the overall picture would
be lost.\index{formal proof}

A solution I envision\label{envision} that would allow mathematics to remain
acceptable to the expert, yet increase its accessibility to non-specialists,
consists of a combination of the traditional short, informal proof in print
accompanied by a complete formal proof stored in a computer database.  In an
analogy with a computer program, the informal proof is like a set of comments
that describe the overall reasoning and content of the proof, whereas the
computer database is like the actual program and provides a means for anyone,
even a non-expert, to follow the proof in as much detail as desired, exploring
it back through layers of theorems (like subroutines that call other
subroutines) all the way back to the axioms of the theory.  In addition, the
computer database would have the advantage of providing absolute assurance
that the proof is correct, since each step can be verified automatically.

There are several other approaches besides Metamath to a project such
as this.  Section~\ref{proofverifiers} discusses some of these.

To us, a noble goal would be a database with hundreds of thousands of
theorems and their computer-verifiable proofs, encompassing a significant
fraction of known mathematics and available for instant access.
These would be fully verified by multiple independently-implemented verifiers,
to provide extremely high confidence that the proofs are completely correct.
The database would allow people to investigate whatever details they were
interested in, so that they could confirm whatever portions they wished.
Whether or not Metamath is an appropriate choice remains to be seen, but in
principle we believe it is sufficient.

\subsubsection{Formalism}

Over the past fifty years, a group of French mathematicians working
collectively under the pseudonym of Bourbaki\index{Bourbaki, Nicolas} have
co-authored a series of monographs that attempt to rigorously and
consistently formalize large bodies of mathematics from foundations.  On the
one hand, certainly such an effort has its merits; on the other hand, the
Bourbaki project has been criticized for its ``scholasticism'' and
``hyperaxiomatics'' that hide the intuitive steps that lead to the results
\cite[p.~191]{Barrow}\index{Barrow, John D.}.

Metamath unabashedly carries this philosophy to its extreme and no doubt is
subject to the same kind of criticism.  Nonetheless I think that in
conjunction with conventional approaches to mathematics Metamath can serve a
useful purpose.  The Bourbaki approach is essentially pedagogic, requiring the
reader to become intimately familiar with each detail in a very large
hierarchy before he or she can proceed to the next step.  The difference with
Metamath is that the ``reader'' (user) knows that all details are contained in
its computer database, available as needed; it does not demand that the user
know everything but conveniently makes available those portions that are of
interest.  As the body of all mathematical knowledge grows larger and larger,
no one individual can have a thorough grasp of its entirety.  Metamath
can finalize and put to rest any questions about the validity of any part of it
and can make any part of it accessible, in principle, to a non-specialist.

\subsubsection{A Personal Note}
Why did I develop Metamath\index{Metamath}?  I enjoy abstract mathematics, but
I sometimes get lost in a barrage of definitions and start to lose confidence
that my proofs are correct.  Or I reach a point where I lose sight of how
anything I'm doing relates to the axioms that a theory is based on and am
sometimes suspicious that there may be some overlooked implicit axiom
accidentally introduced along the way (as happened historically with Euclidean
geometry\index{Euclidean geometry}, whose omission of Pasch's
axiom\index{Pasch's axiom} went unnoticed for 2000 years
\cite[p.~160]{Davis}!). I'm also somewhat lazy and wish to avoid the effort
involved in re-verifying the gaps in informal proofs ``left to the reader;'' I
prefer to figure them out just once and not have to go through the same
frustration a year from now when I've forgotten what I did.  Metamath provides
better recovery of my efforts than scraps of paper that I can't
decipher anymore.  But mostly I find very appealing the idea of rigorously
archiving mathematical knowledge in a computer database, providing precision,
certainty, and elimination of human error.

\subsubsection{Note on Bibliography and Index}

The Bibliography usually includes the Library of Congress classification
for a work to make it easier for you to find it in on a university
library shelf.  The Index has author references to pages where their works
are cited, even though the authors' names may not appear on those pages.

\subsubsection{Acknowledgments}

Acknowledgments are first due to my wife, Deborah (who passed away on
September 4, 1998), for critiquing the manu\-script but most of all for
her patience and support.  I also wish to thank Joe Wright, Richard
Becker, Clarke Evans, Buddha Buck, and Jeremy Henty for helpful
comments.  Any errors, omissions, and other shortcomings are of course
my responsibility.

\subsubsection{Note Added June 22, 2005}\label{note2002}

The original, unpublished version of this book was written in 1997 and
distributed via the web.  The present edition has been updated to
reflect the current Metamath program and databases, as well as more
current {\sc url}s for Internet sites.  Thanks to Josh
Purinton\index{Purinton, Josh}, One Hand
Clapping, Mel L.\ O'Cat, and Roy F. Longton for pointing out
typographical and other errors.  I have also benefitted from numerous
discussions with Raph Levien\index{Levien, Raph}, who has extended
Metamath's philosophy of rigor to result in his {\em
Ghilbert}\index{Ghilbert} proof language (\url{http://ghilbert.org}).

Robert (Bob) Solovay\index{Solovay, Robert} communicated a new result of
A.~R.~D.~Mathias on the system of Bourbaki, and the text has been
updated accordingly (p.~\pageref{bourbaki}).

Bob also pointed out a clarification of the literature regarding
category theory and inaccessible cardinals\index{category
theory}\index{cardinal, inaccessible} (p.~\pageref{categoryth}),
and a misleading statement was removed from the text.  Specifically,
contrary to a statement in previous editions, it is possible to express
``There is a proper class of inaccessible cardinals'' in the language of
ZFC.  This can be done as follows:  ``For every set $x$ there is an
inaccessible cardinal $\kappa$ such that $\kappa$ is not in $x$.''
Bob writes:\footnote{Private communication, Nov.~30, 2002.}
\begin{quotation}
     This axiom is how Grothendieck presents category theory.  To each
inaccessible cardinal $\kappa$ one associates a Grothendieck universe
\index{Grothendieck, Alexander} $U(\kappa)$.  $U(\kappa)$ consists of
those sets which lie in a transitive set of cardinality less than
$\kappa$.  Instead of the ``category of all groups,'' one works relative
to a universe [considering the category of groups of cardinality less
than $\kappa$].  Now the category whose objects are all categories
``relative to the universe $U(\kappa)$'' will be a category not
relative to this universe but to the next universe.

     All of the things category theorists like to do can be done in this
framework.  The only controversial point is whether the Grothen\-dieck
axiom is too strong for the needs of category theorists.  Mac Lane
\index{Mac Lane, Saunders} argues that ``one universe is enough'' and
Feferman\index{Feferman, Solomon} has argued that one can get by with
ordinary ZFC.  I don't find Feferman's arguments persuasive.  Mac Lane
may be right, but when I think about category theory I do it \`{a} la
Grothendieck.

        By the way Mizar\index{Mizar} adds the axiom ``there is a proper
class of inaccessibles'' precisely so as to do category theory.
\end{quotation}

The most current information on the Metamath program and databases can
always be found at \url{http://metamath.org}.


\subsubsection{Note Added June 24, 2006}\label{note2006}

The Metamath spec was restricted slightly to make parsers easier to
write.  See the footnote on p.~\pageref{namespace}.

%\subsubsection{Note Added July 24, 2006}\label{note2006b}
\subsubsection{Note Added March 10, 2007}\label{note2006b}

I am grateful to Anthony Williams\index{Williams, Anthony} for writing
the \LaTeX\ package called {\tt realref.sty} and contributing it to the
public domain.  This package allows the internal hyperlinks in a {\sc
pdf} file to anchor to specific page numbers instead of just section
titles, making the navigation of the {\sc pdf} file for this book much
more pleasant and ``logical.''

A typographical error found by Martin Kiselkov was corrected.
A confusing remark about unification was deleted per suggestion of
Mel O'Cat.

\subsubsection{Note Added May 27, 2009}\label{note2009}

Several typos found by Kim Sparre were corrected.  A note was added that
the Poincar\'{e} conjecture has been proved (p.~\pageref{poincare}).

\subsubsection{Note Added Nov. 17, 2014}\label{note2014}

The statement of the Schr\"{o}der--Bernstein theorem was corrected in
Section~\ref{trust}.  Thanks to Bob Solovay for pointing out the error.

\subsubsection{Note Added May 25, 2016}\label{note2016}

Thanks to Jerry James for correcting 16 typos.

\subsubsection{Note Added February 25, 2019}\label{note201902}

David A. Wheeler\index{Wheeler, David A.}
made a large number of improvements and updates,
in coordination with Norman Megill.
The predicate calculus axioms were renumbered, and the text makes
it clear that they are based on Tarski's system S2;
the one slight deviation in axiom ax-6 is explained and justified.
The real and complex number axioms were modified to be consistent with
\texttt{set.mm}\index{set theory database (\texttt{set.mm})}%
\index{Metamath Proof Explorer}.
Long-awaited specification changes ``1--8'' were made,
which clarified previously ambiguous points.
Some errors in the text involving \texttt{\$f} and
\texttt{\$d} statements were corrected (the spec was correct, but
the in-book explanations unintentionally contradicted the spec).
We now have a system for automatically generating narrow PDFs,
so that those with smartphones can have easy access to the current
version of this document.
A new section on deduction was added;
it discusses the standard deduction theorem,
the weak deduction theorem,
deduction style, and natural deduction.
Many minor corrections (too numerous to list here) were also made.

\subsubsection{Note Added March 7, 2019}\label{note201903}

This added a description of the Matamath language syntax in
Extended Backus--Naur Form (EBNF)\index{Extended Backus--Naur Form}\index{EBNF}
in Appendix \ref{BNF}, added a brief explanation about typecodes,
inserted more examples in the deduction section,
and added a variety of smaller improvements.

\subsubsection{Note Added April 7, 2019}\label{note201904}

This version clarified the proper substitution notation, improved the
discussion on the weak deduction theorem and natural deduction,
documented the \texttt{undo} command, updated the information on
\texttt{write source}, changed the typecode
from \texttt{set} to \texttt{setvar} to be consistent with the current
version of \texttt{set.mm}, added more documentation about comment markup
(e.g., documented how to create headings), and clarified the
differences between various assertion forms (in particular deduction form).

\subsubsection{Note Added May 20, 2019}\label{note201905}

This version fixes various small issues and added specific examples
of forms (deduction form, inference form, and closed form).

\chapter{Introduction}
\pagenumbering{arabic}

\begin{quotation}
  {\em {\em I.M.:}  No, no.  There's nothing subjective about it!  Everybody
knows what a proof is.  Just read some books, take courses from a competent
mathematician, and you'll catch on.

{\em Student:}  Are you sure?

{\em I.M.:}  Well---it is possible that you won't, if you don't have any
aptitude for it.  That can happen, too.

{\em Student:}  Then {\em you} decide what a proof is, and if I don't learn
to decide in the same way, you decide I don't have any aptitude.

{\em I.M.:}  If not me, then who?}
    \flushright\sc  ``The Ideal Mathematician''
    \index{Davis, Phillip J.}
    \footnote{\cite{Davis}, p.~40.}\\
\end{quotation}

In the past century, brilliant mathematicians have discovered almost
unimaginably profound results that rank among the crowning intellectual
achievements of mankind.  However, there is a sense in which modern abstract
mathematics is behind the times, stuck in an era before computers existed.
While no one disputes the remarkable results that have been achieved,
communicating these results in a precise way to the uninitiated is virtually
impossible.  To describe these results, a terse informal language is used which
despite its elegance is very difficult to learn.  This informal language is not
imprecise, far from it, but rather it often has omitted detail
and symbols with hidden context that are
implicitly understood by an expert but few others.  Extremely complex technical
meanings are associated with innocent-sounding English words such as
``compact'' and ``measurable'' that barely hint at what is actually being
said.  Anyone who does not keep the precise technical meaning constantly in
mind is bound to fail, and acquiring the ability to do this can be achieved
only through much practice and hard work.  Only the few who complete the
painful learning experience can join the small in-group of pure
mathematicians.  The informal language effectively cuts off the true nature of
their knowledge from most everyone else.

Metamath\index{Metamath} makes abstract mathematics more concrete.  It allows
a computer to keep track of the complexity associated with each word or symbol
with absolute rigor.  You can explore this complexity at your leisure, to
whatever degree you desire.  Whether or not you believe that concepts such as
infinity actually ``exist'' outside of the mind, Metamath lets you get to the
foundation for what's really being said.

Metamath also enables completely rigorous and thorough proof verification.
Its language is simple enough so that you
don't have to rely on the authority of experts but can verify the results
yourself, step by step.  If you want to attempt to derive your own results,
Metamath will not let you make a mistake in reasoning.
Even professional mathematicians make mistakes; Metamath makes it possible
to thoroughly verify that proofs are correct.

Metamath\index{Metamath} is a computer language and an associated computer
program for archiving, verifying, and studying mathematical proofs at a very
detailed level.
The Metamath language
describes formal\index{formal system} mathematical
systems and expresses proofs of theorems in those systems.  Such a language
is called a metalanguage\index{metalanguage} by mathematicians.
The Metamath program is a computer program that verifies
proofs expressed in the Metamath language.
The Metamath program does not have the built-in
ability to make logical inferences; it just makes a series of symbol
substitutions according to instructions given to it in a proof
and verifies that the result matches the expected theorem.  It makes logical
inferences based only on rules of logic that are contained in a set of
axioms\index{axiom}, or first principles, that you provide to it as the
starting point for proofs.

The complete specification of the Metamath language is only four pages long
(Section~\ref{spec}, p.~\pageref{spec}).  Its simplicity may at first make you
wonder how it can do much of anything at all.  But in fact the kinds of
symbol manipulations it performs are the ones that are implicitly done in all
mathematical systems at the lowest level.  You can learn it relatively quickly
and have complete confidence in any mathematical proof that it verifies.  On
the other hand, it is powerful and general enough so that virtually any
mathematical theory, from the most basic to the deeply abstract, can be
described with it.

Although in principle Metamath can be used with any
kind of mathematics, it is best suited for abstract or ``pure'' mathematics
that is mostly concerned with theorems and their proofs, as opposed to the
kind of mathematics that deals with the practical manipulation of numbers.
Examples of branches of pure mathematics are logic\index{logic},\footnote{Logic
is the study of statements that are universally true regardless of the objects
being described by the statements.  An example is the statement, ``if $P$
implies $Q$, then either $P$ is false or $Q$ is true.''} set theory\index{set
theory},\footnote{Set theory is the study of general-purpose mathematical objects called
``sets,'' and from it essentially all of mathematics can be derived.  For
example, numbers can be defined as specific sets, and their properties
can be explored using the tools of set theory.} number theory\index{number
theory},\footnote{Number theory deals with the properties of positive and
negative integers (whole numbers).} group theory\index{group
theory},\footnote{Group theory studies the properties of mathematical objects
called groups that obey a simple set of axioms and have properties of symmetry
that make them useful in many other fields.} abstract algebra\index{abstract
algebra},\footnote{Abstract algebra includes group theory and also studies
groups with additional properties that qualify them as ``rings'' and
``fields.''  The set of real numbers is a familiar example of a field.},
analysis\index{analysis} \index{real and complex numbers}\footnote{Analysis is
the study of real and complex numbers.} and
topology\index{topology}.\footnote{One area studied by topology are properties
that remain unchanged when geometrical objects undergo stretching
deformations; for example a doughnut and a coffee cup each have one hole (the
cup's hole is in its handle) and are thus considered topologically
equivalent.  In general, though, topology is the study of abstract
mathematical objects that obey a certain (surprisingly simple) set of axioms.
See, for example, Munkres \cite{Munkres}\index{Munkres, James R.}.} Even in
physics, Metamath could be applied to certain branches that make use of
abstract mathematics, such as quantum logic\index{quantum logic} (used to study
aspects of quantum mechanics\index{quantum mechanics}).

On the other hand, Metamath\index{Metamath} is less suited to applications
that deal primarily with intensive numeric computations.  Metamath does not
have any built-in representation of numbers\index{Metamath!representation of
numbers}; instead, a specific string of symbols (digits) must be syntactically
constructed as part of any proof in which an ordinary number is used.  For
this reason, numbers in Metamath are best limited to specific constants that
arise during the course of a theorem or its proof.  Numbers are only a tiny
part of the world of abstract mathematics.  The exclusion of built-in numbers
was a conscious decision to help achieve Metamath's simplicity, and there are
other software tools if you have different mathematical needs.
If you wish to quickly solve algebraic problems, the computer algebra
programs\index{computer algebra system} {\sc
macsyma}\index{macsyma@{\sc macsyma}}, Mathematica\index{Mathematica}, and
Maple\index{Maple} are specifically suited to handling numbers and
algebra efficiently.
If you wish to simply calculate numeric or matrix expressions easily,
tools such as Octave\index{Octave} may be a better choice.

After learning Metamath's basic statement types, any
tech\-ni\-cal\-ly ori\-ent\-ed person, mathematician or not, can
immediately trace
any theorem proved in the language as far back as he or she wants, all the way
to the axioms on which the theorem is based.  This ability suggests a
non-traditional way of learning about pure mathematics.  Used in conjunction
with traditional methods, Metamath could make pure mathematics accessible to
people who are not sufficiently skilled to figure out the implicit detail in
ordinary textbook proofs.  Once you learn the axioms of a theory, you can have
complete confidence that everything you need to understand a proof you are
studying is all there, at your beck and call, allowing you to focus in on any
proof step you don't understand in as much depth as you need, without worrying
about getting stuck on a step you can't figure out.\footnote{On the other
hand, writing proofs in the Metamath language is challenging, requiring
a degree of rigor far in excess of that normally taught to students.  In a
classroom setting, I doubt that writing Metamath proofs would ever replace
traditional homework exercises involving informal proofs, because the time
needed to work out the details would not allow a course to
cover much material.  For students who have trouble grasping the implied rigor
in traditional material, writing a few simple proofs in the Metamath language
might help clarify fuzzy thought processes.  Although somewhat difficult at
first, it eventually becomes fun to do, like solving a puzzle, because of the
instant feedback provided by the computer.}

Metamath\index{Metamath} is probably unlike anything you have
encountered before.  In this first chapter we will look at the philosophy and
use of computers in mathematics in order to better understand the motivation
behind Metamath.  The material in this chapter is not required in order to use
Metamath.  You may skip it if you are impatient, but I hope you will find it
educational and enjoyable.  If you want to start experimenting with the
Metamath program right away, proceed directly to Chapter~\ref{using}
(p.~\pageref{using}).  To
learn the Metamath language, skim Chapter~\ref{using} then proceed to
Chapter~\ref{languagespec} (p.~\pageref{languagespec}).

\section{Mathematics as a Computer Language}

\begin{quote}
  {\em The study of mathematics is apt to commence in
dis\-ap\-point\-ment.\ldots \\
We are told that by its aid the stars are weighted
and the billions of molecules in a drop of water are counted.  Yet, like the
ghost of Hamlet's father, this great science eludes the efforts of our mental
weapons to grasp it.}
  \flushright\sc  Alfred North Whitehead\footnote{\cite{Whitehead}, ch.\ 1.}\\
\end{quote}\index{Whitehead, Alfred North}

\subsection{Is Mathematics ``User-Friendly''?}

Suppose you have no formal training in abstract mathematics.  But popular
books you've read offer tempting glimpses of this world filled with profound
ideas that have stirred the human spirit.  You are not satisfied with the
informal, watered-down descriptions you've read but feel it is important to
grasp the underlying mathematics itself to understand its true meaning. It's
not practical to go back to school to learn it, though; you don't want to
dedicate years of your life to it.  There are many important things in life,
and you have to set priorities for what's important to you.  What would happen
if you tried to pursue it on your own, in your spare time?

After all, you were able to learn a computer programming language such as
Pascal on your own without too much difficulty, even though you had no formal
training in computers.  You don't claim to be an expert in software design,
but you can write a passable program when necessary to suit your needs.  Even
more important, you know that you can look at anyone else's Pascal program, no
matter how complex, and with enough patience figure out exactly how it works,
even though you are not a specialist.  Pascal allows you do anything that a
computer can do, at least in principle.  Thus you know you have the ability,
in principle, to follow anything that a computer program can do:  you just
have to break it down into small enough pieces.

Here's an imaginary scenario of what might happen if you na\-ive\-ly a\-dopted
this same view of abstract mathematics and tried to pick it up on your own, in
a period of time comparable to, say, learning a computer programming
language.

\subsubsection{A Non-Mathematician's Quest for Truth}

\begin{quote}
  {\em \ldots my daughters have been studying (chemistry) for several
se\-mes\-ters, think they have learned differential and integral calculus in
school, and yet even today don't know why $x\cdot y=y\cdot x$ is true.}
  \flushright\sc  Edmund Landau\footnote{\cite{Landau}, p.~vi.}\\
\end{quote}\index{Landau, Edmund}

\begin{quote}
  {\em Minus times minus is plus,\\
The reason for this we need not discuss.}
  \flushright\sc W.\ H.\ Auden\footnote{As quoted in \cite{Guillen}, p.~64.}\\
\end{quote}\index{Auden, W.\ H.}\index{Guillen, Michael}

We'll suppose you are a technically oriented professional, perhaps an engineer, a
computer programmer, or a physicist, but probably not a mathematician.  You
consider yourself reasonably intelligent.  You did well in school, learning a
variety of methods and techniques in practical mathematics such as calculus and
differential equations.  But rarely did your courses get into anything
resembling modern abstract mathematics, and proofs were something that appeared
only occasionally in your textbooks, a kind of necessary evil that was
supposed to convince you of a certain key result.  Most of your
homework consisted of exercises that gave you practice in the techniques, and
you were hardly ever asked to come up with a proof of your own.

You find yourself curious about advanced, abstract mathematics.  You are
driven by an inner conviction that it is important to understand and
appreciate some of the most profound knowledge discovered by mankind.  But it
seems very hard to learn, something that only certain gifted longhairs can
access and understand.  You are frustrated that it seems forever cut off from
you.

Eventually your curiosity drives you to do something about it.
You set for yourself a goal of ``really'' understanding mathematics:  not just
how to manipulate equations in algebra or calculus according to cookbook
rules, but rather to gain a deep understanding of where those rules come from.
In fact, you're not thinking about this kind of ordinary mathematics at all,
but about a much more abstract, ethereal realm of pure mathematics, where
famous results such as G\"{o}del's incompleteness theorem\index{G\"{o}del's
incompleteness theorem} and Cantor's different kinds of infinities
reside.

You have probably read a number of popular books, with titles like {\em
Infinity and the Mind} \cite{Rucker}\index{Rucker, Rudy}, on topics such as
these.  You found them inspiring but at the same time somewhat
unsatisfactory.  They gave you a general idea of what these results are about,
but if someone asked you to prove them, you wouldn't have the faintest idea of
where to begin.   Sure, you could give the same overall outline that you
learned from the popular books; and in a general sort of way, you do have an
understanding.  But deep down inside, you know that there is a rigor that is
missing, that probably there are many subtle steps and pitfalls along the way,
and ultimately it seems you have to place your trust in the experts in the
field.  You don't like this; you want to be able to verify these results for
yourself.

So where do you go next?  As a first step, you decide to look up some of the
original papers on the theorems you are curious about, or better, obtain some
standard textbooks in the field.  You look up a theorem you want to
understand.  Sure enough, it's there, but it's expressed with strange
terms and odd symbols that mean absolutely nothing to you.  It might as well be written in
a foreign language you've never seen before, whose symbols are totally alien.
You look at the proof, and you haven't the foggiest notion what each step
means, much less how one step follows from another.  Well, obviously you have
a lot to learn if you want to understand this stuff.

You feel that you could probably understand it by
going back to college for another three to six years and getting a math
degree.  But that does not fit in with your career and the other things in
your life and would serve no practical purpose.  You decide to seek a quicker
path.  You figure you'll just trace your way back to the beginning, step by
step, as you would do with a computer program, until you understand it.  But
you quickly find that this is not possible, since you can't even understand
enough to know what you have to trace back to.

Maybe a different approach is in order---maybe you should start at the
beginning and work your way up.  First, you read the introduction to the book
to find out what the prerequisites are.  In a similar fashion, you trace your
way back through two or three more books, finally arriving at one that seems
to start at a beginning:  it lists the axioms of arithmetic.  ``Aha!'' you
naively think, ``This must be the starting point, the source of all mathematical
knowledge.'' Or at least the starting point for mathematics dealing with
numbers; you have to start somewhere and have no idea what the starting point
for other mathematics would be.  But the word ``axioms'' looks promising.  So
you eagerly read along and work through some elementary exercises at the
beginning of the book.  You feel vaguely bothered:  these
don't seem like axioms at all, at least not in the sense that you want to
think of axioms.  Axioms imply a starting point from which everything else can
be built up, according to precise rules specified in the axiom system.  Even
though you can understand the first few proofs in an informal way,
and are able to do some of the
exercises, it's hard to pin down precisely what the
rules are.   Sure, each step seems to follow logically from the others, but
exactly what does that mean?  Is the ``logic'' just a matter of common sense,
something vague that we all understand but can never quite state precisely?

You've spent a number of years, off and on, programming computers, and you
know that in the case of computer languages there is no question of what the
rules are---they are precise and crystal clear.  If you follow them, your
program will work, and if you don't, it won't.  No matter how complex a
program, it can always be broken down into simpler and simpler pieces, until
you can ultimately identify the bits that are moved around to perform a
specific function.  Some programs might require a lot of perseverance to
accomplish this, but if you focus on a specific portion of it, you don't even
necessarily have to know how the rest of it works. Shouldn't there be an
analogy in mathematics?

You decide to apply the ultimate test:  you ask yourself how a computer could
verify or ensure that the steps in these proofs follow from one another.
Certainly mathematics must be at least as precisely defined as a computer
language, if not more so; after all, computer science itself is based on it.
If you can get a computer to verify these proofs, then you should also be
able, in principle, to understand them yourself in a very crystal clear,
precise way.

You're in for a surprise:  you can conceive of no way to convert the
proofs, which are in English, to a form that the computer can understand.
The proofs are filled with phrases such as ``assume there exists a unique
$x$\ldots'' and ``given any $y$, let $z$ be the number such that\ldots''  This
isn't the kind of logic you are used to in computer programming, where
everything, even arithmetic, reduces to Boolean ones and zeroes if you care to
break it down sufficiently.  Even though you think you understand the proofs,
there seems to be some kind of higher reasoning involved rather than precise
rules that define how you manipulate the symbols in the axioms.  Whatever it
is, it just isn't obvious how you would express it to a computer, and the more
you think about it, the more puzzled and confused you get, to the point where
you even wonder whether {\em you} really understand it.  There's a lot more to
these axioms of arithmetic than meets the eye.

Nobody ever talked about this in school in your applied math and engineering
courses.  You just learned the rules they gave you, not quite understanding
how or why they worked, sometimes vaguely suspicious or uncertain of them, and
through homework problems and osmosis learned how to present solutions that
satisfied the instructor and earned you an ``A.''  Rarely did you actually
``prove'' anything in a rigorous way, and the math majors who did do stuff
like that seemed to be in a different world.

Of course, there are computer algebra programs that can do mathematics, and
rather impressively.  They can instantly solve the integrals that you
struggled with in freshman calculus, and do much, much more.  But when you
look at these programs, what you see is a big collection of algorithms and
techniques that evolved and were added to over time, along with some basic
software that manipulates symbols.  Each algorithm that is built in is the
result of someone's theorem whose proof is omitted; you just have to trust the
person who proved it and the person who programmed it in and hope there are no
bugs.\index{computer program bugs}  Somehow this doesn't seem to be the
essence of mathematics.  Although computer algebra systems can generate
theorems with amazing speed, they can't actually prove a single one of them.

After some puzzlement, you revisit some popular books on what mathematics is
all about.  Somewhere you read that all of mathematics is actually derived
from something called ``set theory.''  This is a little confusing, because
nowhere in the book that presented the axioms of arithmetic was there any
mention of set theory, or if there was, it seemed to be just a tool that helps
you describe things better---the set of even numbers, that sort of thing.  If
set theory is the basis for all mathematics, then why are additional axioms
needed for arithmetic?

Something is wrong but you're not sure what.  One of your friends is a pure
mathematician.  He knows he is unable to communicate to you what he does for a
living and seems to have little interest in trying.  You do know that for him,
proofs are what mathematics is all about. You ask him what a proof is, and he
essentially tells you that, while of course it's based on logic, really it's
something you learn by doing it over and over until you pick it up.  He refers
you to a book, {\em How to Read and Do Proofs} \cite{Solow}.\index{Solow,
Daniel}  Although this book helps you understand traditional informal proofs,
there is still something missing you can't seem to pin down yet.

You ask your friend how you would go about having a computer verify a proof.
At first he seems puzzled by the question; why would you want to do that?
Then he says it's not something that would make any sense to do, but he's
heard that you'd have to break the proof down into thousands or even millions
of individual steps to do such a thing, because the reasoning involved is at
such a high level of abstraction.  He says that maybe it's something you could
do up to a point, but the computer would be completely impractical once you
get into any meaningful mathematics.  There, the only way you can verify a
proof is by hand, and you can only acquire the ability to do this by
specializing in the field for a couple of years in grad school.  Anyway, he
thinks it all has to do with set theory, although he has never taken a formal
course in set theory but just learned what he needed as he went along.

You are intrigued and amazed.  Apparently a mathematician can grasp as a
single concept something that would take a computer a thousand or a million
steps to verify, and have complete confidence in it.  Each one of these
thousand or million steps must be absolutely correct, or else the whole proof
is meaningless.  If you added a million numbers by hand, would you trust the
result?  How do you really know that all these steps are correct, that there
isn't some subtle pitfall in one of these million steps, like a bug in a
computer program?\index{computer program bugs}  After all, you've read that
famous mathematicians have occasionally made mistakes, and you certainly know
you've made your share on your math homework problems in school.

You recall the analogy with a computer program.  Sure, you can understand what
a large computer program such as a word processor does, as a single high-level
concept or a small set of such concepts, but your ability to understand it in
no way ensures that the program is correct and doesn't have hidden bugs.  Even
if you wrote the program yourself you can't really know this; most large
programs that you've written have had bugs that crop up at some later date, no
matter how careful you tried to be while writing them.

OK, so now it seems the reason you can't figure out how to make a
computer verify proofs is because each step really corresponds to a
million small steps.  Well, you say, a computer can do a million
calculations in a second, so maybe it's still practical to do.  Now the
puzzle becomes how to figure out what the million steps are that each
English-language step corresponds to.  Your mathematician friend hasn't
a clue, but suggests that maybe you would find the answer by studying
set theory.  Actually, your friend thinks you're a little off the wall
for even wondering such a thing.  For him, this is not what mathematics
is all about.

The subject of set theory keeps popping up, so you decide it's
time to look it up.

You decide to start off on a careful footing, so you start reading a couple of
very elementary books on set theory.  A lot of it seems pretty obvious, like
intersections, subsets, and Venn diagrams.  You thumb through one of the
books; nowhere is anything about axioms mentioned. The other book relegates to
an appendix a brief discussion that mentions a set of axioms called
``Zermelo--Fraenkel set theory''\index{Zermelo--Fraenkel set theory} and states
them in English.  You look at them and have no idea what they really mean or
what you can do with them.  The comments in this appendix say that the purpose
of mentioning them is to expose you to the idea, but imply that they are not
necessary for basic understanding and that they are really the subject matter
of advanced treatments where fine points such as a certain paradox (Russell's
paradox\index{Russell's paradox}\footnote{Russell's paradox assumes that there
exists a set $S$ that is a collection of all sets that don't contain
themselves.  Now, either $S$ contains itself or it doesn't.  If it contains
itself, it contradicts its definition.  But if it doesn't contain itself, it
also contradicts its definition.  Russell's paradox is resolved in ZF set
theory by denying that such a set $S$ exists.}) are resolved.  Wait a
minute---shouldn't the axioms be a starting point, not an ending point?  If
there are paradoxes that arise without the axioms, how do you know you won't
stumble across one accidentally when using the informal approach?

And nowhere do these books describe how ``all of mathematics can be
derived from set theory'' which by now you've heard a few times.

You find a more advanced book on set theory.  This one actually lists the
axioms of ZF set theory in plain English on page one.  {\em Now} you think
your quest has ended and you've finally found the source of all mathematical
knowledge; you just have to understand what it means.  Here, in one place, is
the basis for all of mathematics!  You stare at the axioms in awe, puzzle over
them, memorize them, hoping that if you just meditate on them long enough they
will become clear.  Of course, you haven't the slightest idea how the rest of
mathematics is ``derived'' from them; in particular, if these are the axioms
of mathematics, then why do arithmetic, group theory, and so on need their own
axioms?

You start reading this advanced book carefully, pondering the meaning of every
word, because by now you're really determined to get to the bottom of this.
The first thing the book does is explain how the axioms came about, which was
to resolve Russell's paradox.\index{Russell's paradox}  In fact that seems to
be the main purpose of their existence; that they supposedly can be used to
derive all of mathematics seems irrelevant and is not even mentioned.  Well,
you go on.  You hope the book will explain to you clearly, step by step, how
to derive things from the axioms.  After all, this is the starting point of
mathematics, like a book that explains the basics of a computer programming
language.  But something is missing.  You find you can't even understand the
first proof or do the first exercise.  Symbols such as $\exists$ and $\forall$
permeate the page without any mention of where they came from or how to
manipulate them; the author assumes you are totally familiar with them and
doesn't even tell you what they mean.  By now you know that $\exists$ means
``there exists'' and $\forall$ means ``for all,'' but shouldn't the rules for
manipulating these symbols be part of the axioms?  You still have no idea
how you could even describe the axioms to a computer.

Certainly there is something much different here from the technical
literature you're used to reading.  A computer language manual almost
always explains very clearly what all the symbols mean, precisely what
they do, and the rules used for combining them, and you work your way up
from there.

After glancing at four or five other such books, you come to the realization
that there is another whole field of study that you need just to get to the
point at which you can understand the axioms of set theory.  The field is
called ``logic.''  In fact, some of the books did recommend it as a
prerequisite, but it just didn't sink in.  You assumed logic was, well, just
logic, something that a person with common sense intuitively understood.  Why
waste your time reading boring treatises on symbolic logic, the manipulation
of 1's and 0's that computers do, when you already know that?  But this is a
different kind of logic, quite alien to you.  The subject of {\sc nand} and
{\sc nor} gates is not even touched upon or in any case has to do with only a
very small part of this field.

So your quest continues.  Skimming through the first couple of introductory
books, you get a general idea of what logic is about and what quantifiers
(``for all,'' ``there exists'') mean, but you find their examples somewhat
trivial and mildly annoying (``all dogs are animals,'' ``some animals are
dogs,'' and such).  But all you want to know is what the rules are for
manipulating the symbols so you can apply them to set theory.  Some formulas
describing the relationships among quantifiers ($\exists$ and $\forall$) are
listed in tables, along with some verbal reasoning to justify them.
Presumably, if you want to find out if a formula is correct, you go through
this same kind of mental reasoning process, possibly using images of dogs and
animals. Intuitively, the formulas seem to make sense.  But when you ask
yourself, ``What are the rules I need to get a computer to figure out whether
this formula is correct?'', you still don't know.  Certainly you don't ask the
computer to imagine dogs and animals.

You look at some more advanced logic books.  Many of them have an introductory
chapter summarizing set theory, which turns out to be a prerequisite.  You
need logic to understand set theory, but it seems you also need set theory to
understand logic!  These books jump right into proving rather advanced
theorems about logic, without offering the faintest clue about where the logic
came from that allows them to prove these theorems.

Luckily, you come across an elementary book of logic that, halfway through,
after the usual truth tables and metaphors, presents in a clear, precise way
what you've been looking for all along: the axioms!  They're divided into
propositional calculus (also called sentential logic) and predicate calculus
(also called first-order logic),\index{first-order logic} with rules so simple
and crystal clear that now you can finally program a computer to understand
them.  Indeed, they're no harder than learning how to play a game of chess.
As far as what you seem to need is concerned, the whole book could have been
written in five pages!

{\em Now} you think you've found the ultimate source of mathematical
truth.  So---the axioms of mathematics consist of these axioms of logic,
together with the axioms of ZF set theory. (By now you've also been able to
figure out how to translate the ZF axioms from English into the
actual symbols of logic which you can now manipulate according to
precise, easy-to-understand rules.)

Of course, you still don't understand how ``all of mathematics can be
derived from set theory,'' but maybe this will reveal itself in due
course.

You eagerly set out to program the axioms and rules into a computer and start
to look at the theorems you will have to prove as the logic is developed.  All
sorts of important theorems start popping up:  the deduction
theorem,\index{deduction theorem} the substitution theorem,\index{substitution
theorem} the completeness theorem of propositional calculus,\index{first-order
logic!completeness} the completeness theorem of predicate calculus.  Uh-oh,
there seems to be trouble.  They all get harder and harder, and not one of
them can be derived with the axioms and rules of logic you've just been
handed.  Instead, they all require ``metalogic'' for their proofs, a kind of
mixture of logic and set theory that allows you to prove things {\em about}
the axioms and theorems of logic rather than {\em with} them.

You plow ahead anyway.  A month later, you've spent much of your
free time getting the computer to verify proofs in propositional calculus.
You've programmed in the axioms, but you've also had to program in the
deduction theorem, the substitution theorem, and the completeness theorem of
propositional calculus, which by now you've resigned yourself to treating as
rather complex additional axioms, since they can't be proved from the axioms
you were given.  You can now get the computer to verify and even generate
complete, rigorous, formal proofs\index{formal proof}.  Never mind that they
may have 100,000 steps---at least now you can have complete, absolute
confidence in them.  Unfortunately, the only theorems you have proved are
pretty trivial and you can easily verify them in a few minutes with truth
tables, if not by inspection.

It looks like your mathematician friend was right.  Getting the computer to do
serious mathematics with this kind of rigor seems almost hopeless.  Even
worse, it seems that the further along you get, the more ``axioms'' you have
to add, as each new theorem seems to involve additional ``metamathematical''
reasoning that hasn't been formalized, and none of it can be derived from the
axioms of logic.  Not only do the proofs keep growing exponentially as you get
further along, but the program to verify them keeps getting bigger and bigger
as you program in more ``metatheorems.''\index{metatheorem}\footnote{A
metatheorem is usually a statement that is too general to be directly provable
in a theory.  For example, ``if $n_1$, $n_2$, and $n_3$ are integers, then
$n_1+n_2+n_3$ is an integer'' is a theorem of number theory.  But ``for any
integer $k > 1$, if $n_1, \ldots, n_k$ are integers, then $n_1+\ldots +n_k$ is
an integer'' is a metatheorem, in other words a family of theorems, one for
every $k$.  The reason it is not a theorem is that the general sum $n_1+\ldots
+n_k$ (as a function of $k$) is not an operation that can be defined directly
in number theory.} The bugs\index{computer program bugs} that have cropped up
so far have already made you start to lose faith in the rigor you seem to have
achieved, and you know it's just going to get worse as your program gets larger.

\subsection{Mathematics and the Non-Specialist}

\begin{quote}
  {\em A real proof is not checkable by a machine, or even by any mathematician
not privy to the gestalt, the mode of thought of the particular field of
mathematics in which the proof is located.}
  \flushright\sc  Davis and Hersh\index{Davis, Phillip J.}
  \footnote{\cite{Davis}, p.~354.}\\
\end{quote}

The bulk of abstract or theoretical mathematics is ordinarily outside
the reach of anyone but a few specialists in each field who have completed
the necessary difficult internship in order to enter its coterie.  The
typical intelligent layperson has no reasonable hope of understanding much of
it, nor even the specialist mathematician of understanding other fields.  It
is like a foreign language that has no dictionary to look up the translation;
the only way you can learn it is by living in the country for a few years.  It
is argued that the effort involved in learning a specialty is a necessary
process for acquiring a deep understanding.  Of course, this is almost certainly
true if one is to make significant contributions to a field; in particular,
``doing'' proofs is probably the most important part of a mathematician's
training.  But is it also necessary to deny outsiders access to it?  Is it
necessary that abstract mathematics be so hard for a layperson to grasp?

A computer normally is of no help whatsoever.  Most published proofs are
actually just series of hints written in an informal style that requires
considerable knowledge of the field to understand.  These are the ``real
proofs'' referred to by Davis and Hersh.\index{informal proof}  There is an
implicit understanding that, in principle, such a proof could be converted to
a complete formal proof\index{formal proof}.  However, it is said that no one
would ever attempt such a conversion, even if they could, because that would
presumably require millions of steps (Section~\ref{dream}).  Unfortunately the
informal style automatically excludes the understanding of the proof
by anyone who hasn't gone through the necessary apprenticeship. The
best that the intelligent layperson can do is to read popular books about deep
and famous results; while this can be helpful, it can also be misleading, and
the lack of detail usually leaves the reader with no ability whatsoever to
explore any aspect of the field being described.

The statements of theorems often use sophisticated notation that makes them
inaccessible to the non-specialist.  For a non-specialist who wants to achieve
a deeper understanding of a proof, the process of tracing definitions and
lemmas back through their hierarchy\index{hierarchy} quickly becomes confusing
and discouraging.  Textbooks are usually written to train mathematicians or to
communicate to people who are already mathematicians, and large gaps in proofs
are often left as exercises to the reader who is left at an impasse if he or
she becomes stuck.

I believe that eventually computers will enable non-specialists and even
intelligent laypersons to follow almost any mathematical proof in any field.
Metamath is an attempt in that direction.  If all of mathematics were as
easily accessible as a computer programming language, I could envision
computer programmers and hobbyists who otherwise lack mathematical
sophistication exploring and being amazed by the world of theorems and proofs
in obscure specialties, perhaps even coming up with results of their own.  A
tremendous advantage would be that anyone could experiment with conjectures in
any field---the computer would offer instant feedback as to whether
an inference step was correct.

Mathematicians sometimes have to put up with the annoyance of
cranks\index{cranks} who lack a fundamental understanding of mathematics but
insist that their ``proofs'' of, say, Fermat's Last Theorem\index{Fermat's
Last Theorem} be taken seriously.  I think part of the problem is that these
people are misled by informal mathematical language, treating it as if they
were reading ordinary expository English and failing to appreciate the
implicit underlying rigor.  Such cranks are rare in the field of computers,
because computer languages are much more explicit, and ultimately the proof is
in whether a computer program works or not.  With easily accessible
computer-based abstract mathematics, a mathematician could say to a crank,
``don't bother me until you've demonstrated your claim on the computer!''

% 22-May-04 nm
% Attempt to move De Millo quote so it doesn't separate from attribution
% CHANGE THIS NUMBER (AND ELIMINATE IF POSSIBLE) WHEN ABOVE TEXT CHANGES
\vspace{-0.5em}

\subsection{An Impossible Dream?}\label{dream}

\begin{quote}
  {\em Even quite basic theorems would demand almost unbelievably vast
  books to display their proofs.}
    \flushright\sc  Robert E. Edwards\footnote{\cite{Edwards}, p.~68.}\\
\end{quote}\index{Edwards, Robert E.}

\begin{quote}
  {\em Oh, of course no one ever really {\em does} it.  It would take
  forever!  You just show that you could do it, that's sufficient.}
    \flushright\sc  ``The Ideal Mathematician''
    \index{Davis, Phillip J.}\footnote{\cite{Davis},
p.~40.}\\
\end{quote}

\begin{quote}
  {\em There is a theorem in the primitive notation of set theory that
  corresponds to the arithmetic theorem `$1000+2000=3000$'.  The formula
  would be forbiddingly long\ldots even if [one] knows the definitions
  and is asked to simplify the long formula according to them, chances are
  he will make errors and arrive at some incorrect result.}
    \flushright\sc  Hao Wang\footnote{\cite{Wang}, p.~140.}\\
\end{quote}\index{Wang, Hao}

% 22-May-04 nm
% Attempt to move De Millo quote so it doesn't separate from attribution
% CHANGE THIS NUMBER (AND ELIMINATE IF POSSIBLE) WHEN ABOVE TEXT CHANGES
\vspace{-0.5em}

\begin{quote}
  {\em The {\em Principia Mathematica} was the crowning achievement of the
  formalists.  It was also the deathblow of the formalist view.\ldots
  {[Rus\-sell]} failed, in three enormous volumes, to get beyond the elementary
  facts of arithmetic.  He showed what can be done in principle and what
  cannot be done in practice.  If the mathematical process were really
  one of strict, logical progression, we would still be counting our
  fingers.\ldots
  One theoretician estimates\ldots that a demonstration of one of
  Ramanujan's conjectures assuming set theory and elementary analysis would
  take about two thousand pages; the length of a deduction from first principles
  is nearly in\-con\-ceiv\-a\-ble\ldots The probabilists argue that\ldots any
  very long proof can at best be viewed as only probably correct\ldots}
  \flushright\sc Richard de Millo et. al.\footnote{\cite{deMillo}, pp.~269,
  271.}\\
\end{quote}\index{de Millo, Richard}

A number of writers have conveyed the impression that the kind of absolute
rigor provided by Metamath\index{Metamath} is an impossible dream, suggesting
that a complete, formal verification\index{formal proof} of a typical theorem
would take millions of steps in untold volumes of books.  Even if it could be
done, the thinking sometimes goes, all meaning would be lost in such a
monstrous, tedious verification.\index{informal proof}\index{proof length}

These writers assume, however, that in order to achieve the kind of complete
formal verification they desire one must break down a proof into individual
primitive steps that make direct reference to the axioms.  This is
not necessary.  There is no reason not to make use of previously proved
theorems rather than proving them over and over.

Just as important, definitions\index{definition} can be introduced along
the way, allowing very complex formulas to be represented with few
symbols.  Not doing this can lead to absurdly long formulas.  For
example, the mere statement of
G\"{o}del's incompleteness theorem\index{G\"{o}del's
incompleteness theorem}, which can be expressed with a small number of
defined symbols, would require about 20,000 primitive symbols to express
it.\index{Boolos, George S.}\footnote{George S.\ Boolos, lecture at
Massachusetts Institute of Technology, spring 1990.} An extreme example
is Bourbaki's\label{bourbaki} language for set theory, which requires
4,523,659,424,929 symbols plus 1,179,618,517,981 disambiguatory links
(lines connecting symbol pairs, usually drawn below or above the
formula) to express the number
``one'' \cite{Mathias}.\index{Mathias, Adrian R. D.}\index{Bourbaki,
Nicolas}
% http://www.dpmms.cam.ac.uk/~ardm/

A hierarchy\index{hierarchy} of theorems and definitions permits an
exponential growth in the formula sizes and primitive proof steps to be
described with only a linear growth in the number of symbols used.  Of course,
this is how ordinary informal mathematics is normally done anyway, but with
Metamath\index{Metamath} it can be done with absolute rigor and precision.

\subsection{Beauty}


\begin{quote}
  {\em No one shall be able to drive us from the paradise that Cantor has
created for us.}
   \flushright\sc  David Hilbert\footnote{As quoted in \cite{Moore}, p.~131.}\\
\end{quote}\index{Hilbert, David}

\needspace{3\baselineskip}
\begin{quote}
  {\em Mathematics possesses not only truth, but some supreme beauty ---a
  beauty cold and austere, like that of a sculpture.}
    \flushright\sc  Bertrand
    Russell\footnote{\cite{Russell}.}\\
\end{quote}\index{Russell, Bertrand}

\begin{quote}
  {\em Euclid alone has looked on Beauty bare.}
  \flushright\sc Edna Millay\footnote{As quoted in \cite{Davis}, p.~150.}\\
\end{quote}\index{Millay, Edna}

For most people, abstract mathematics is distant, strange, and
incomprehensible.  Many popular books have tried to convey some of the sense
of beauty in famous theorems.  But even an intelligent layperson is left with
only a general idea of what a theorem is about and is hardly given the tools
needed to make use of it.  Traditionally, it is only after years of arduous
study that one can grasp the concepts needed for deep understanding.
Metamath\index{Metamath} allows you to approach the proof of the theorem from
a quite different perspective, peeling apart the formulas and definitions
layer by layer until an entirely different kind of understanding is achieved.
Every step of the proof is there, pieced together with absolute precision and
instantly available for inspection through a microscope with a magnification
as powerful as you desire.

A proof in itself can be considered an object of beauty.  Constructing an
elegant proof is an art.  Once a famous theorem has been proved, often
considerable effort is made to find simpler and more easily understood
proofs.  Creating and communicating elegant proofs is a major concern of
mathematicians.  Metamath is one way of providing a common language for
archiving and preserving this information.

The length of a proof can, to a certain extent, be considered an
objective measure of its ``beauty,'' since shorter proofs are usually
considered more elegant.  In the set theory database
\texttt{set.mm}\index{set theory database (\texttt{set.mm})}%
\index{Metamath Proof Explorer}
provided with Metamath, one goal was to make all proofs as short as possible.

\needspace{4\baselineskip}
\subsection{Simplicity}

\begin{quote}
  {\em God made man simple; man's complex problems are of his own
  devising.}
    \flushright\sc Eccles. 7:29\footnote{Jerusalem Bible.}\\
\end{quote}\index{Bible}

\needspace{3\baselineskip}
\begin{quote}
  {\em God made integers, all else is the work of man.}
    \flushright\sc Leopold Kronecker\footnote{{\em Jahresbericht
	der Deutschen Mathematiker-Vereinigung }, vol. 2, p. 19.}\\
\end{quote}\index{Kronecker, Leopold}

\needspace{3\baselineskip}
\begin{quote}
  {\em For what is clear and easily comprehended attracts; the
  complicated repels.}
    \flushright\sc David Hilbert\footnote{As quoted in \cite{deMillo},
p.~273.}\\
\end{quote}\index{Hilbert, David}

The Metamath\index{Metamath} language is simple and Spartan.  Metamath treats
all mathematical expressions as simple sequences of symbols, devoid of meaning.
The higher-level or ``metamathematical'' notions underlying Metamath are about
as simple as they could possibly be.  Each individual step in a proof involves
a single basic concept, the substitution of an expression for a variable, so
that in principle almost anyone, whether mathematician or not, can
completely understand how it was arrived at.

In one of its most basic applications, Metamath\index{Metamath} can be used to
develop the foundations of mathematics\index{foundations of mathematics} from
the very beginning.  This is done in the set theory database that is provided
with the Metamath package and is the subject matter
of Chapter~\ref{fol}. Any language (a metalanguage\index{metalanguage})
used to describe mathematics (an object language\index{object language}) must
have a mathematical content of its own, but it is desirable to keep this
content down to a bare minimum, namely that needed to make use of the
inference rules specified by the axioms.  With any metalanguage there is a
``chicken and egg'' problem somewhat like circular reasoning:  you must assume
the validity of the mathematics of the metalanguage in order to prove the
validity of the mathematics of the object language.  The mathematical content
of Metamath itself is quite limited.  Like the rules of a game of chess, the
essential concepts are simple enough so that virtually anyone should be able to
understand them (although that in itself will not let you play like
a master).  The symbols that Metamath manipulates do not in themselves
have any intrinsic meaning.  Your interpretation of the axioms that you supply
to Metamath is what gives them meaning.  Metamath is an attempt to strip down
mathematical thought to its bare essence and show you exactly how the symbols
are manipulated.

Philosophers and logicians, with various motivations, have often thought it
important to study ``weak'' fragments of logic\index{weak logic}
\cite{Anderson}\index{Anderson, Alan Ross} \cite{MegillBunder}\index{Megill,
Norman}\index{Bunder, Martin}, other unconventional systems of logic (such as
``modal'' logic\index{modal logic} \cite[ch.\ 27]{Boolos}\index{Boolos, George
S.}), and quantum logic\index{quantum logic} in physics
\cite{Pavicic}\index{Pavi{\v{c}}i{\'{c}}, M.}.  Metamath\index{Metamath}
provides a framework in which such systems can be expressed, with an absolute
precision that makes all underlying metamathematical assumptions rigorous and
crystal clear.

Some schools of philosophical thought, for example
intuitionism\index{intuitionism} and constructivism\index{constructivism},
demand that the notions underlying any mathematical system be as simple and
concrete as possible.  Metamath should meet the requirements of these
philosophies.  Metamath must be taught the symbols, axioms\index{axiom}, and
rules\index{rule} for a specific theory, from the skeptical (such as
intuitionism\index{intuitionism}\footnote{Intuitionism does not accept the law
of excluded middle (``either something is true or it is not true'').  See
\cite[p.~xi]{Tymoczko}\index{Tymoczko, Thomas} for discussion and references
on this topic.  Consider the theorem, ``There exist irrational numbers $a$ and
$b$ such that $a^b$ is rational.''  An intuitionist would reject the following
proof:  If $\sqrt{2}^{\sqrt{2}}$ is rational, we are done.  Otherwise, let
$a=\sqrt{2}^{\sqrt{2}}$ and $b=\sqrt{2}$. Then $a^b=2$, which is rational.})
to the bold (such as the axiom of choice in set theory\footnote{The axiom of
choice\index{Axiom of Choice} asserts that given any collection of pairwise
disjoint nonempty sets, there exists a set that has exactly one element in
common with each set of the collection.  It is used to prove many important
theorems in standard mathematics.  Some philosophers object to it because it
asserts the existence of a set without specifying what the set contains
\cite[p.~154]{Enderton}\index{Enderton, Herbert B.}.  In one foundation for
mathematics due to Quine\index{Quine, Willard Van Orman}, that has not been
otherwise shown to be inconsistent, the axiom of choice turns out to be false
\cite[p.~23]{Curry}\index{Curry, Haskell B.}.  The \texttt{show
trace{\char`\_}back} command of the Metamath program allows you to find out
whether the axiom of choice, or any other axiom, was assumed by a
proof.}\index{\texttt{show trace{\char`\_}back} command}).

The simplicity of the Metamath language lets the algorithm (computer program)
that verifies the validity of a Metamath proof be straightforward and
robust.  You can have confidence that the theorems it verifies really can be
derived from your axioms.

\subsection{Rigor}

\begin{quote}
  {\em Rigor became a goal with the Greeks\ldots But the efforts to
  pursue rigor to the utmost have led to an impasse in which there is
  no longer any agreement on what it really means.  Mathematics remains
  alive and vital, but only on a pragmatic basis.}
    \flushright\sc  Morris Kline\footnote{\cite{Kline}, p.~1209.}\\
\end{quote}\index{Kline, Morris}

Kline refers to a much deeper kind of rigor than that which we will discuss in
this section.  G\"{o}del's incompleteness theorem\index{G\"{o}del's
incompleteness theorem} showed that it is impossible to achieve absolute rigor
in standard mathematics because we can never prove that mathematics is
consistent (free from contradictions).\index{consistent theory}  If
mathematics is consistent, we will never know it, but must rely on faith.  If
mathematics is inconsistent, the best we can hope for is that some clever
future mathematician will discover the inconsistency.  In this case, the
axioms would probably be revised slightly to eliminate the inconsistency, as
was done in the case of Russell's paradox,\index{Russell's paradox} but the
bulk of mathematics would probably not be affected by such a discovery.
Russell's paradox, for example, did not affect most of the remarkable results
achieved by 19th-century and earlier mathematicians.  It mainly invalidated
some of Gottlob Frege's\index{Frege, Gottlob} work on the foundations of
mathematics in the late 1800's; in fact Frege's work inspired Russell's
discovery.  Despite the paradox, Frege's work contains important concepts that
have significantly influenced modern logic.  Kline's {\em Mathematics, The
Loss of Certainty} \cite{Klinel}\index{Kline, Morris} has an interesting
discussion of this topic.

What {\em can} be achieved with absolute certainty\index{certainty} is the
knowledge that if we assume the axioms are consistent and true, then the
results derived from them are true.  Part of the beauty of mathematics is that
it is the one area of human endeavor where absolute certainty can be achieved
in this sense.  A mathematical truth will remain such for eternity.  However,
our actual knowledge of whether a particular statement is a mathematical truth
is only as certain as the correctness of the proof that establishes it.  If
the proof of a statement is questionable or vague, we can't have absolute
confidence in the truth that the statement claims.

Let us look at some traditional ways of expressing proofs.

Except in the field of formal logic\index{formal logic}, almost all
traditional proofs in mathematics are really not proofs at all, but rather
proof outlines or hints as to how to go about constructing the proof.  Many
gaps\index{gaps in proofs} are left for the reader to fill in. There are
several reasons for this.  First, it is usually assumed in mathematical
literature that the person reading the proof is a mathematician familiar with
the specialty being described, and that the missing steps are obvious to such
a reader or at least that the reader is capable of filling them in.  This
attitude is fine for professional mathematicians in the specialty, but
unfortunately it often has the drawback of cutting off the rest of the world,
including mathematicians in other specialties, from understanding the proof.
We discussed one possible resolution to this on p.~\pageref{envision}.
Second, it is often assumed that a complete formal proof\index{formal proof}
would require countless millions of symbols (Section~\ref{dream}). This might
be true if the proof were to be expressed directly in terms of the axioms of
logic and set theory,\index{set theory} but it is usually not true if we allow
ourselves a hierarchy\index{hierarchy} of definitions and theorems to build
upon, using a notation that allows us to introduce new symbols, definitions,
and theorems in a precisely specified way.

Even in formal logic,\index{formal logic} formal proofs\index{formal proof}
that are considered complete still contain hidden or implicit information.
For example, a ``proof'' is usually defined as a sequence of
wffs,\index{well-formed formula (wff)}\footnote{A {\em wff} or well-formed
formula is a mathematical expression (string of symbols) constructed according
to some precise rules.  A formal mathematical system\index{formal system}
contains (1) the rules for constructing syntactically correct
wffs,\index{syntax rules} (2) a list of starting wffs called
axioms,\index{axiom} and (3) one or more rules prescribing how to derive new
wffs, called theorems\index{theorem}, from the axioms or previously derived
theorems.  An example of such a system is contained in
Metamath's\index{Metamath} set theory database, which defines a formal
system\index{formal system} from which all of standard mathematics can be
derived.  Section~\ref{startf} steps you through a complete example of a formal
system, and you may want to skim it now if you are unfamiliar with the
concept.} each of which is an axiom or follows from a rule applied to previous
wffs in the sequence.  The implicit part of the proof is the algorithm by
which a sequence of symbols is verified to be a valid wff, given the
definition of a wff.  The algorithm in this case is rather simple, but for a
computer to verify the proof,\index{automated proof verification} it must have
the algorithm built into its verification program.\footnote{It is possible, of
course, to specify wff construction syntax outside of the program itself
with a suitable input language (the Metamath language being an example), but
some proof-verification or theorem-proving programs lack the ability to extend
wff syntax in such a fashion.} If one deals exclusively with axioms and
elementary wffs, it is straightforward to implement such an algorithm.  But as
more and more definitions are added to the theory in order to make the
expression of wffs more compact, the algorithm becomes more and more
complicated.  A computer program that implements the algorithm becomes larger
and harder to understand as each definition is introduced, and thus more prone
to bugs.\index{computer program bugs}  The larger the program, the
more suspicious the mathematician may be about
the validity of its algorithms.  This is especially true because
computer programs are inherently hard to follow to begin with, and few people
enjoy verifying them manually in detail.

Metamath\index{Metamath} takes a different approach.  Metamath's ``knowledge''
is limited to the ability to substitute variables for expressions, subject to
some simple constraints.  Once the basic algorithm of Metamath is assumed to
be debugged, and perhaps independently confirmed, it
can be trusted once and for all.  The information that Metamath needs to
``understand'' mathematics is contained entirely in the body of knowledge
presented to Metamath.  Any errors in reasoning can only be errors in the
axioms or definitions contained in this body of knowledge.  As a
``constructive'' language\index{constructive language} Metamath has no
conditional branches or loops like the ones that make computer programs hard
to decipher; instead, the language can only build new sequences of symbols
from earlier sequences  of symbols.

The simplicity of the rules that underlie Metamath not only makes Metamath
easy to learn but also gives Metamath a great deal of flexibility. For
example, Metamath is not limited to describing standard first-order
logic\index{first-order logic}; higher-order logics\index{higher-order logic}
and fragments of logic\index{weak logic} can be described just as easily.
Metamath gives you the freedom to define whatever wff notation you prefer; it
has no built-in conception of the syntax of a wff.\index{well-formed formula
(wff)}  With suitable axioms and definitions, Metamath can even describe and
prove things about itself.\index{Metamath!self-description}  (John
Harrison\index{Harrison, John} discusses the ``reflection''
principle\index{reflection principle} involved in self-descriptive systems in
\cite{Harrison}.)

The flexibility of Metamath requires that its proofs specify a lot of detail,
much more than in an ordinary ``formal'' proof.\index{formal proof}  For
example, in an ordinary formal proof, a single step consists of displaying the
wff that constitutes that step.  In order for a computer program to verify
that the step is acceptable, it first must verify that the symbol sequence
being displayed is an acceptable wff.\index{automated proof verification} Most
proof verifiers have at least basic wff syntax built into their programs.
Metamath has no hard-wired knowledge of what constitutes a wff built into it;
instead every wff must be explicitly constructed based on rules defining wffs
that are present in a database.  Thus a single step in an ordinary formal
proof may be correspond to many steps in a Metamath proof. Despite the larger
number of steps, though, this does not mean that a Metamath proof must be
significantly larger than an ordinary formal proof. The reason is that since
we have constructed the wff from scratch, we know what the wff is, so there is
no reason to display it.  We only need to refer to a sequence of statements
that construct it.  In a sense, the display of the wff in an ordinary formal
proof is an implicit proof of its own validity as a wff; Metamath just makes
the proof explicit. (Section~\ref{proof} describes Metamath's proof notation.)

\section{Computers and Mathematicians}

\begin{quote}
  {\em The computer is important, but not to mathematics.}
    \flushright\sc  Paul Halmos\footnote{As quoted in \cite{Albers}, p.~121.}\\
\end{quote}\index{Halmos, Paul}

Pure mathematicians have traditionally been indifferent to computers, even to
the point of disdain.\index{computers and pure mathematics}  Computer science
itself is sometimes considered to fall in the mundane realm of ``applied''
mathematics, perhaps essential for the real world but intellectually unexciting
to those who seek the deepest truths in mathematics.  Perhaps a reason for this
attitude towards computers is that there is little or no computer software that
meets their needs, and there may be a general feeling that such software could
not even exist.  On the one hand, there are the practical computer algebra
systems, which can perform amazing symbolic manipulations in algebra and
calculus,\index{computer algebra system} yet can't prove the simplest
existence theorem, if the idea of a proof is present at all.  On the other
hand, there are specialized automated theorem provers that technically speaking
may generate correct proofs.\index{automated theorem proving}  But sometimes
their specialized input notation may be cryptic and their output perceived to
be long, inelegant, incomprehensible proofs.    The output
may be viewed with suspicion, since the program that generates it tends to be
very large, and its size increases the potential for bugs\index{computer
program bugs}.  Such a proof may be considered trustworthy only if
independently verified and ``understood'' by a human, but no one wants to
waste their time on such a boring, unrewarding chore.



\needspace{4\baselineskip}
\subsection{Trusting the Computer}

\begin{quote}
  {\em \ldots I continue to find the quasi-empirical interpretation of
  computer proofs to be the more plausible.\ldots Since not
  everything that claims to be a computer proof can be
  accepted as valid, what are the mathematical criteria for acceptable
  computer proofs?}
    \flushright\sc  Thomas Tymoczko\footnote{\cite{Tymoczko}, p.~245.}\\
\end{quote}\index{Tymoczko, Thomas}

In some cases, computers have been essential tools for proving famous
theorems.  But if a proof is so long and obscure that it can be verified in a
practical way only with a computer, it is vaguely felt to be suspicious.  For
example, proving the famous four-color theorem\index{four-color
theorem}\index{proof length} (``a map needs no more than four colors to
prevent any two adjacent countries from having the same color'') can presently
only be done with the aid of a very complex computer program which originally
required 1200 hours of computer time. There has been considerable debate about
whether such a proof can be trusted and whether such a proof is ``real''
mathematics \cite{Swart}\index{Swart, E. R.}.\index{trusting computers}

However, under normal circumstances even a skeptical mathematician would have a
great deal of confidence in the result of multiplying two numbers on a pocket
calculator, even though the precise details of what goes on are hidden from its
user.  Even the verification on a supercomputer that a huge number is prime is
trusted, especially if there is independent verification; no one bothers to
debate the philosophical significance of its ``proof,'' even though the actual
proof would be so large that it would be completely impractical to ever write
it down on paper.  It seems that if the algorithm used by the computer is
simple enough to be readily understood, then the computer can be trusted.

Metamath\index{Metamath} adopts this philosophy.  The simplicity of its
language makes it easy to learn, and because of its simplicity one can have
essentially absolute confidence that a proof is correct. All axioms, rules, and
definitions are available for inspection at any time because they are defined
by the user; there are no hidden or built-in rules that may be prone to subtle
bugs\index{computer program bugs}.  The basic algorithm at the heart of
Metamath is simple and fixed, and it can be assumed to be bug-free and robust
with a degree of confidence approaching certainty.
Independently written implementations of the Metamath verifier
can reduce any residual doubt on the part of a skeptic even further;
there are now over a dozen such implementations, written by many people.

\subsection{Trusting the Mathematician}\label{trust}

\begin{quote}
  {\em There is no Algebraist nor Mathematician so expert in his science, as
  to place entire confidence in any truth immediately upon his discovery of it,
  or regard it as any thing, but a mere probability.  Every time he runs over
  his proofs, his confidence encreases; but still more by the approbation of
  his friends; and is rais'd to its utmost perfection by the universal assent
  and applauses of the learned world.}
  \flushright\sc David Hume\footnote{{\em A Treatise of Human Nature}, as
  quoted in \cite{deMillo}, p.~267.}\\
\end{quote}\index{Hume, David}

\begin{quote}
  {\em Stanislaw Ulam estimates that mathematicians publish 200,000 theorems
  every year.  A number of these are subsequently contradicted or otherwise
  disallowed, others are thrown into doubt, and most are ignored.}
  \flushright\sc Richard de Millo et. al.\footnote{\cite{deMillo}, p.~269.}\\
\end{quote}\index{Ulam, Stanislaw}

Whether or not the computer can be trusted, humans  of course will occasionally
err. Only the most memorable proofs get independently verified, and of these
only a handful of truly great ones achieve the status of being ``known''
mathematical truths that are used without giving a second thought to their
correctness.

There are many famous examples of incorrect theorems and proofs in
mathematical literature.\index{errors in proofs}

\begin{itemize}
\item There have been thousands of purported proofs of Fermat's Last
Theorem\index{Fermat's Last Theorem} (``no integer solutions exist to $x^n +
y^n = z^n$ for $n > 2$''), by amateurs, cranks, and well-regarded
mathematicians \cite[p.~5]{Stark}\index{Stark, Harold M}.  Fermat wrote a note
in his copy of Bachet's {\em Diophantus} that he found ``a truly marvelous
proof of this theorem but this margin is too narrow to contain it''
\cite[p.~507]{Kramer}.  A recent, much publicized proof by Yoichi
Miyaoka\index{Miyaoka, Yoichi} was shown to be incorrect ({\em Science News},
April 9, 1988, p.~230).  The theorem was finally proved by Andrew
Wiles\index{Wiles, Andrew} ({\em Science News}, July 3, 1993, p.~5), but it
initially had some gaps and took over a year after its announcement to be
checked thoroughly by experts.  On Oct. 25, 1994, Wiles announced that the last
gap found in his proof had been filled in.
  \item In 1882, M. Pasch discovered that an axiom was omitted from Euclid's
formulation of geometry\index{Euclidean geometry}; without it, the proofs of
important theorems of Euclid are not valid.  Pasch's axiom\index{Pasch's
axiom} states that a line that intersects one side of a triangle must also
intersect another side, provided that it does not touch any of the triangle's
vertices.  The omission of Pasch's axiom went unnoticed for 2000
years \cite[p.~160]{Davis}, in spite of (one presumes) the thousands of
students, instructors, and mathematicians who studied Euclid.
  \item The first published proof of the famous Schr\"{o}der--Bernstein
theorem\index{Schr\"{o}der--Bernstein theorem} in set theory was incorrect
\cite[p.~148]{Enderton}\index{Enderton, Herbert B.}.  This theorem states
that if there exists a 1-to-1 function\footnote{A {\em set}\index{set} is any
collection of objects. A {\em function}\index{function} or {\em
mapping}\index{mapping} is a rule that assigns to each element of one set
(called the function's {\em domain}\index{domain}) an element from another
set.} from set $A$ into set $B$ and vice-versa, then sets $A$ and $B$ have
a 1-to-1 correspondence.  Although it sounds simple and obvious,
the standard proof is quite long and complex.
  \item In the early 1900's, Hilbert\index{Hilbert, David} published a
purported proof of the continuum hypothesis\index{continuum hypothesis}, which
was eventually established as unprovable by Cohen\index{Cohen, Paul} in 1963
\cite[p.~166]{Enderton}.  The continuum hypothesis states that no
infinity\index{infinity} (``transfinite cardinal number'')\index{cardinal,
transfinite} exists whose size (or ``cardinality''\index{cardinality}) is
between the size of the set of integers and the size of the set of real
numbers.  This hypothesis originated with German mathematician Georg
Cantor\index{Cantor, Georg} in the late 1800's, and his inability to prove it
is said to have contributed to mental illness that afflicted him in his later
years.
  \item An incorrect proof of the four-color theorem\index{four-color theorem}
was published by Kempe\index{Kempe, A. B.} in 1879
\cite[p.~582]{Courant}\index{Courant, Richard}; it stood for 11 years before
its flaw was discovered.  This theorem states that any map can be colored
using only four colors, so that no two adjacent countries have the same
color.  In 1976 the theorem was finally proved by the famous computer-assisted
proof of Haken, Appel, and Koch \cite{Swart}\index{Appel, K.}\index{Haken,
W.}\index{Koch, K.}.  Or at least it seems that way.  Mathematician
H.~S.~M.~Coxeter\index{Coxeter, H. S. M.} has doubts \cite[p.~58]{Davis}:  ``I
have a feeling that that is an untidy kind of use of the computers, and the more
you correspond with Haken and Appel, the more shaky you seem to be.''
  \item Many false ``proofs'' of the Poincar\'{e}
conjecture\index{Poincar\'{e} conjecture} have been proposed over the years.
This conjecture states that any object that mathematically behaves like a
three-dimensional sphere is a three-dimensional sphere topologically,
regardless of how it is distorted.  In March 1986, mathematicians Colin
Rourke\index{Rourke, Colin} and Eduardo R\^{e}go\index{R\^{e}go, Eduardo}
caused  a stir in the mathematical community by announcing that they had found
a proof; in November of that year the proof was found to be false \cite[p.
218]{PetersonI}.  It was finally proved in 2003 by Grigory Perelman
\label{poincare}\index{Szpiro, George}\index{Perelman, Grigory}\cite{Szpiro}.
 \end{itemize}

Many counterexamples to ``theorems'' in recent mathematical
literature related to Clifford algebras\index{Clifford algebras}
 have been found by Pertti
Lounesto (who passed away in 2002).\index{Lounesto, Pertti}
See the web page \url{http://mathforum.org/library/view/4933.html}.
% http://users.tkk.fi/~ppuska/mirror/Lounesto/counterexamples.htm

One of the purposes of Metamath\index{Metamath} is to allow proofs to be
expressed with absolute precision.  Developing a proof in the Metamath
language can be challenging, because Metamath will not permit even the
tiniest mistake.\index{errors in proofs}  But once the proof is created, its
correctness can be trusted immediately, without having to depend on the
process of peer review for confirmation.

\section{The Use of Computers in Mathematics}

\subsection{Computer Algebra Systems}

For the most part, you will find that Metamath\index{Metamath} is not a
practical tool for manipulating numbers.  (Even proving that $2 + 2 = 4$, if
you start with set theory, can be quite complex!)  Several commercial
mathematics packages are quite good at arithmetic, algebra, and calculus, and
as practical tools they are invaluable.\index{computer algebra system} But
they have no notion of proof, and cannot understand statements starting with
``there exists such and such...''.

Software packages such as Mathematica \cite{Wolfram}\index{Mathematica} do not
concern themselves with proofs but instead work directly with known results.
These packages primarily emphasize heuristic rules such as the substitution of
equals for equals to achieve simpler expressions or expressions in a different
form.  Starting with a rich collection of built-in rules and algorithms, users
can add to the collection by means of a powerful programming language.
However, results such as, say, the existence of a certain abstract object
without displaying the actual object cannot be expressed (directly) in their
languages.  The idea of a proof from a small set of axioms is absent.  Instead
this software simply assumes that each fact or rule you add to the built-in
collection of algorithms is valid.  One way to view the software is as a large
collection of axioms from which the software, with certain goals, attempts to
derive new theorems, for example equating a complex expression with a simpler
equivalent. But the terms ``theorem''\index{theorem} and
``proof,''\index{proof} for example, are not even mentioned in the index of
the user's manual for Mathematica.\index{Mathematica and proofs}  What is also
unsatisfactory from a philosophical point of view is that there is no way to
ensure the validity of the results other than by trusting the writer of each
application module or tediously checking each module by hand, similar to
checking a computer program for bugs.\index{computer program
bugs}\footnote{Two examples illustrate why the knowledge database of computer
algebra systems should sometimes be regarded with a certain caution.  If you
ask Mathematica (version 3.0) to \texttt{Solve[x\^{ }n + y\^{ }n == z\^{ }n , n]}
it will respond with \texttt{\{\{n-\char`\>-2\}, \{n-\char`\>-1\},
\{n-\char`\>1\}, \{n-\char`\>2\}\}}. In other words, Mathematica seems to
``know'' that Fermat's Last Theorem\index{Fermat's Last Theorem} is true!  (At
the time this version of Mathematica was released this fact was unknown.)  If
you ask Maple\index{Maple} to \texttt{solve(x\^{ }2 = 2\^{ }x)} then
\texttt{simplify(\{"\})}, it returns the solution set \texttt{\{2, 4\}}, apparently
unaware that 0.7666647\ldots is also a solution.} While of course extremely
valuable in applied mathematics, computer algebra systems tend to be of little
interest to the theoretical mathematician except as aids for exploring certain
specific problems.

Because of possible bugs, trusting the output of a computer algebra system for
use as theorems in a proof-verifier would defeat the latter's goal of rigor.
On the other hand, a fact such that a certain relatively large number is
prime, while easy for a computer algebra system to derive, might have a long,
tedious proof that could overwhelm a proof-verifier. One approach for linking
computer algebra systems to a proof-verifier while retaining the advantages of
both is to add a hypothesis to each such theorem indicating its source.  For
example, a constant {\sc maple} could indicate the theorem came from the Maple
package, and would mean ``assuming Maple is consistent, then\ldots''  This and
many other topics concerning the formalization of mathematics are discussed in
John Harrison's\index{Harrison, John} very interesting
PhD thesis~\cite{Harrison-thesis}.

\subsection{Automated Theorem Provers}\label{theoremprovers}

A mathematical theory is ``decidable''\index{decidable theory} if a mechanical
method or algorithm exists that is guaranteed to determine whether or not a
particular formula is a theorem.  Among the few theories that are decidable is
elementary geometry,\index{Euclidean geometry} as was shown by a classic
result of logician Alfred Tarski\index{Tarski, Alfred} in 1948
\cite{Tarski}.\footnote{Tarski's result actually applies to a subset of the
geometry discussed in elementary textbooks.  This subset includes most of what
would be considered elementary geometry but it is not powerful enough to
express, among other things, the notions of the circumference and area of a
circle.  Extending the theory in a way that includes notions such as these
makes the theory undecidable, as was also shown by Tarski.  Tarski's algorithm
is far too inefficient to implement practically on a computer.  A practical
algorithm for proving a smaller subset of geometry theorems (those not
involving concepts of ``order'' or ``continuity'') was discovered by Chinese
mathematician Wu Wen-ts\"{u}n in 1977 \cite{Chou}\index{Chou,
Shang-Ching}.}\index{Wen-ts{\"{u}}n, Wu}  But most theories, including
elementary arithmetic, are undecidable.  This fact contributes to keeping
mathematics alive and well, since many mathematicians believe
that they will never be
replaced by computers (if they believe Roger Penrose's argument that a
computer can never replace the brain \cite{Penrose}\index{Penrose, Roger}).
In fact,  elementary geometry is often considered a ``dead'' field
for the simple reason that it is decidable.

On the other hand, the undecidability of a theory does not mean that one cannot
use a computer to search for proofs, providing one is willing to give up if a
proof is not found after a reasonable amount of time.  The field of automated
theorem proving\index{automated theorem proving} specializes in pursuing such
computer searches.  Among the more successful results to date are those based
on an algorithm known as Robinson's resolution principle
\cite{Robinson}\index{Robinson's resolution principle}.

Automated theorem provers can be excellent tools for those willing to learn
how to use them.  But they are not widely used in mainstream pure
mathematics, even though they could probably be useful in many areas.  There
are several reasons for this.  Probably most important, the main goal in pure
mathematics is to arrive at results that are considered to be deep or
important; proving them is essential but secondary.  Usually, an automated
theorem prover cannot assist in this main goal, and by the time the main goal
is achieved, the mathematician may have already figured out the proof as a
by-product.  There is also a notational problem.  Mathematicians are used to
using very compact syntax where one or two symbols (heavily dependent on
context) can represent very complex concepts; this is part of the
hierarchy\index{hierarchy} they have built up to tackle difficult problems.  A
theorem prover on the other hand might require that a theorem be expressed in
``first-order logic,''\index{first-order logic} which is the logic on which
most of mathematics is ultimately based but which is not ordinarily used
directly because expressions can become very long.  Some automated theorem
provers are experimental programs, limited in their use to very specialized
areas, and the goal of many is simply research into the nature of automated
theorem proving itself.  Finally, much research remains to be done to enable
them to prove very deep theorems.  One significant result was a
computer proof by Larry Wos\index{Wos, Larry} and colleagues that every Robbins
algebra\index{Robbins algebra} is a Boolean  algebra\index{Boolean algebra}
({\em New York Times}, Dec. 10, 1996).\footnote{In 1933, E.~V.\
Huntington\index{Huntington, E. V.}
presented the following axiom system for
Boolean algebra with a unary operation $n$ and a binary operation $+$:
\begin{center}
    $x + y = y + x$ \\
    $(x + y) + z = x + (y + z)$ \\
    $n(n(x) + y) + n(n(x) + n(y)) = x$
\end{center}
Herbert Robbins\index{Robbins, Herbert}, a student of Huntington, conjectured
that the last equation can be replaced with a simpler one:
\begin{center}
    $n(n(x + y) + n(x + n(y))) = x$
\end{center}
Robbins and Huntington could not find a proof.  The problem was
later studied unsuccessfully by Tarski\index{Tarski, Alfred} and his
students, and it remained an unsolved problem until a
computer found the proof in 1996.  For more information on
the Robbins algebra problem see \cite{Wos}.}

How does Metamath\index{Metamath} relate to automated theorem provers?  A
theorem prover is primarily concerned with one theorem at a time (perhaps
tapping into a small database of known theorems) whereas Metamath is more like
a theorem archiving system, storing both the theorem and its proof in a
database for access and verification.  Metamath is one answer to ``what do you
do with the output of a theorem prover?''  and could be viewed as the
next step in the process.  Automated theorem provers could be useful tools for
helping develop its database.
Note that very long, automatically
generated proofs can make your database fat and ugly and cause Metamath's proof
verification to take a long time to run.  Unless you have a particularly good
program that generates very concise proofs, it might be best to consider the
use of automatically generated proofs as a quick-and-dirty approach, to be
manually rewritten at some later date.

The program {\sc otter}\index{otter@{\sc otter}}\footnote{\url{http://www.cs.unm.edu/\~mccune/otter/}.}, later succeeded by
prover9\index{prover9}\footnote{\url{https://www.cs.unm.edu/~mccune/mace4/}.},
have been historically influential.
The E prover\index{E prover}\footnote{\url{https://github.com/eprover/eprover}.}
is a maintained automated theorem prover
for full first-order logic with equality.
There are many other automated theorem provers as well.

If you want to combine automated theorem provers with Metamath
consider investigating
the book {\em Automated Reasoning:  Introduction and Applications}
\cite{Wos}\index{Wos, Larry}.  This book discusses
how to use {\sc otter} in a way that can
not only able to generate
relatively efficient proofs, it can even be instructed to search for
shorter proofs.  The effective use of {\sc otter} (and similar tools)
does require a certain
amount of experience, skill, and patience.  The axiom system used in the
\texttt{set.mm}\index{set theory database (\texttt{set.mm})} set theory
database can be expressed to {\sc otter} using a method described in
\cite{Megill}.\index{Megill, Norman}\footnote{To use those axioms with
{\sc otter}, they must be restated in a way that eliminates the need for
``dummy variables.''\index{dummy variable!eliminating} See the Comment
on p.~\pageref{nodd}.} When successful, this method tends to generate
short and clever proofs, but my experiments with it indicate that the
method will find proofs within a reasonable time only for relatively
easy theorems.  It is still fun to experiment with.

Reference \cite{Bledsoe}\index{Bledsoe, W. W.} surveys a number of approaches
people have explored in the field of automated theorem proving\index{automated
theorem proving}.

\subsection{Interactive Theorem Provers}\label{interactivetheoremprovers}

Finding proofs completely automatically is difficult, so there
are some interactive theorem provers that allow a human to guide the
computer to find a proof.
Examples include
HOL Light\index{HOL light}%
\footnote{\url{https://www.cl.cam.ac.uk/~jrh13/hol-light/}.},
Isabelle\index{Isabelle}%
\footnote{\url{http://www.cl.cam.ac.uk/Research/HVG/Isabelle}.},
{\sc hol}\index{hol@{\sc hol}}%
\footnote{\url{https://hol-theorem-prover.org/}.},
and
Coq\index{Coq}\footnote{\url{https://coq.inria.fr/}.}.

A major difference between most of these tools and Metamath is that the
``proofs'' are actually programs that guide the program to find a proof,
and not the proof itself.
For example, an Isabelle/HOL proof might apply a step
\texttt{apply (blast dest: rearrange reduction)}. The \texttt{blast}
instruction applies
an automatic tableux prover and returns if it found a sequence of proof
steps that work... but the sequence is not considered part of the proof.

A good overview of
higher-level proof verification languages (such as {\sc lcf}\index{lcf@{\sc
lcf}} and {\sc hol}\index{hol@{\sc hol}})
is given in \cite{Harrison}.  All of these languages are fundamentally
different from Metamath in that much of the mathematical foundational
knowledge is embedded in the underlying proof-verification program, rather
than placed directly in the database that is being verified.
These can have a steep learning curve for those without a mathematical
background.  For example, one usually must have a fair understanding of
mathematical logic in order to follow their proofs.

\subsection{Proof Verifiers}\label{proofverifiers}

A proof verifier is a program that doesn't generate proofs but instead
verifies proofs that you give it.  Many proof verifiers have limited built-in
automated proof capabilities, such as figuring out simple logical inferences
(while still being guided by a person who provides the overall proof).  Metamath
has no built-in automated proof capability other than the limited
capability of its Proof Assistant.

Proof-verification languages are not used as frequently as they might be.
Pure mathematicians are more concerned with producing new results, and such
detail and rigor would interfere with that goal.  The use of computers in pure
mathematics is primarily focused on automated theorem provers (not verifiers),
again with the ultimate goal of aiding the creation of new mathematics.
Automated theorem provers are usually concerned with attacking one theorem at
time rather than making a large, organized database easily available to the
user.  Metamath is one way to help close this gap.

By itself Metamath is a mostly a proof verifier.
This does not mean that other approaches can't be used; the difference
is that in Metamath, the results of various provers must be recorded
step-by-step so that they can be verified.

Another proof-verification language is Mizar,\index{Mizar} which can display
its proofs in the informal language that mathematicians are accustomed to.
Information on the Mizar language is available at \url{http://mizar.org}.

For the working mathematician, Mizar is an excellent tool for rigorously
documenting proofs. Mizar typesets its proofs in the informal English used by
mathematicians (and, while fine for them, are just as inscrutable by
laypersons!). A price paid for Mizar is a relatively steep learning curve of a
couple of weeks.  Several mathematicians are actively formalizing different
areas of mathematics using Mizar and publishing the proofs in a dedicated
journal. Unfortunately the task of formalizing mathematics is still looked
down upon to a certain extent since it doesn't involve the creation of ``new''
mathematics.

The closest system to Metamath is
the {\em Ghilbert}\index{Ghilbert} proof language (\url{http://ghilbert.org})
system developed by
Raph Levien\index{Levien, Raph}.
Ghilbert is a formal proof checker heavily inspired by Metamath.
Ghilbert statements are s-expressions (a la Lisp), which is easy
for computers to parse but many people find them hard to read.
There are a number of differences in their specific constructs, but
there is at least one tool to translate some Metamath materials into Ghilbert.
As of 2019 the Ghilbert community is smaller and less active than the
Metamath community.
That said, the Metamath and Ghilbert communities overlap, and fruitful
conversations between them have occurred many times over the years.

\subsection{Creating a Database of Formalized Mathematics}\label{mathdatabase}

Besides Metamath, there are several other ongoing projects with the goal of
formalizing mathematics into computer-verifiable databases.
Understanding some history will help.

The {\sc qed}\index{qed project@{\sc qed} project}%
\footnote{\url{http://www-unix.mcs.anl.gov/qed}.}
project arose in 1993 and its goals were outlined in the
{\sc qed} manifesto.
The {\sc qed} manifesto was
a proposal for a computer-based database of all mathematical knowledge,
strictly formalized and with all proofs having been checked automatically.
The project had a conference in 1994 and another in 1995;
there was also a ``twenty years of the {\sc qed} manifesto'' workshop
in 2014.
Its ideals are regularly reraised.

In a 2007 paper, Freek Wiedijk identified two reasons
for the failure of the {\sc qed} project as originally envisioned:%
\cite{Wiedijk-revisited}\index{Wiedijk, Freek}

\begin{itemize}
\item Very few people are working on formalization of mathematics. There is no compelling application for fully mechanized mathematics.
\item Formalized mathematics does not yet resemble traditional mathematics. This is partly due to the complexity of mathematical notation, and partly to the limitations of existing theorem provers and proof assistants.
\end{itemize}

But this did not end the dream of
formalizing mathematics into computer-verifiable databases.
The problems that led to the {\sc qed} manifesto are still with us,
even though the challenges were harder than originally considered.
What has happened instead is that various independent projects have
worked towards formalizing mathematics into computer-verifiable databases,
each simultaneously competing and cooperating with each other.

A concrete way to see this is
Freek Wiedijk's ``Formalizing 100 Theorems'' list%
\footnote{\url{http://www.cs.ru.nl/\%7Efreek/100/}.}
which shows the progress different systems have made on a challenge list
of 100 mathematical theorems.%
\footnote{ This is not the only list of ``interesting'' theorems.
Another interesting list was posted by Oliver Knill's list
\cite{Knill}\index{Knill, Oliver}.}
The top systems as of February 2019
(in order of the number of challenges completed) are
HOL Light, Isabelle, Metamath, Coq, and Mizar.

The Metamath 100%
\footnote{\url{http://us.metamath.org/mm\_100.html}}
page (maintained by David A. Wheeler\index{Wheeler, David A.})
shows the progress of Metamath (specifically its \texttt{set.mm} database)
against this challenge list maintained by Freek Wiedijk.
The Metamath \texttt{set.mm} database
has made a lot of progress over the years,
in part because working to prove those challenge theorems required
defining various terms and proving their properties as a prerequisite.
Here are just a few of the many statements that have been
formally proven with Metamath:

% The entries of this cause the narrow display to break poorly,
% since the short amount of text means LaTeX doesn't get a lot to work with
% and the itemize format gives it even *less* margin than usual.
% No one will mind if we make just this list flushleft, since this list
% will be internally consistent.
\begin{flushleft}
\begin{itemize}
\item 1. The Irrationality of the Square Root of 2
  (\texttt{sqr2irr}, by Norman Megill, 2001-08-20)
\item 2. The Fundamental Theorem of Algebra
  (\texttt{fta}, by Mario Carneiro, 2014-09-15)
\item 22. The Non-Denumerability of the Continuum
  (\texttt{ruc}, by Norman Megill, 2004-08-13)
\item 54. The Konigsberg Bridge Problem
  (\texttt{konigsberg}, by Mario Carneiro, 2015-04-16)
\item 83. The Friendship Theorem
  (\texttt{friendship}, by Alexander W. van der Vekens, 2018-10-09)
\end{itemize}
\end{flushleft}

We thank all of those who have developed at least one of the Metamath 100
proofs, and we particularly thank
Mario Carneiro\index{Carneiro, Mario}
who has contributed the most Metamath 100 proofs as of 2019.
The Metamath 100 page shows the list of all people who have contributed a
proof, and links to graphs and charts showing progress over time.
We encourage others to work on proving theorems not yet proven in Metamath,
since doing so improves the work as a whole.

Each of the math formalization systems (including Metamath)
has different strengths and weaknesses, depending on what you value.
Key aspects that differentiate Metamath from the other top systems are:

\begin{itemize}
\item Metamath is not tied to any particular set of axioms.
\item Metamath can show every step of every proof, no exceptions.
  Most other provers only assert that a proof can be found, and do not
  show every step. This also makes verification fast, because
  the system does not need to rediscover proof details.
\item The Metamath verifier has been re-implemented in many different
  programming languages, so verification can be done by multiple
  implementations.  In particular, the
  \texttt{set.mm}\index{set theory database (\texttt{set.mm})}%
  \index{Metamath Proof Explorer} database is verified by
  four different verifiers
  written in four different languages by four different authors.
  This greatly reduces the risk of accepting an invalid
  proof due to an error in the verifier.
\item Proofs stay proven.  In some systems, changes to the system's
  syntax or how a tactic works causes proofs to fail in later versions,
  causing older work to become essentially lost.
  Metamath's language is
  extremely small and fixed, so once a proof is added to a database,
  the database can be rechecked with later versions of the Metamath program
  and with other verifiers of Metamath databases.
  If an axiom or key definition needs to be changed, it is easy to
  manipulate the database as a whole to handle the change
  without touching the underlying verifier.
  Since re-verification of an entire database takes seconds, there
  is never a reason to delay complete verification.
  This aspect is especially compelling if your
  goal is to have a long-term database of proofs.
\item Licensing is generous.  The main Metamath databases are released to
  the public domain, and the main Metamath program is open source software
  under a standard, widely-used license.
\item Substitutions are easy to understand, even by those who are not
  professional mathematicians.
\end{itemize}

Of course, other systems may have advantages over Metamath
that are more compelling, depending on what you value.
In any case, we hope this helps you understand Metamath
within a wider context.

\subsection{In Summary}\label{computers-summary}

To summarize our discussions of computers and mathematics, computer algebra
systems can be viewed as theorem generators focusing on a narrow realm of
mathematics (numbers and their properties), automated theorem provers as proof
generators for specific theorems in a much broader realm covered by a built-in
formal system such as first-order logic, interactive theorem
provers require human guidance, proof verifiers verify proofs but
historically they have been
restricted to first-order logic.
Metamath, in contrast,
is a proof verifier and documenter whose realm is essentially unlimited.

\section{Mathematics and Metamath}

\subsection{Standard Mathematics}

There are a number of ways that Metamath\index{Metamath} can be used with
standard mathematics.  The most satisfying way philosophically is to start at
the very beginning, and develop the desired mathematics from the axioms of
logic and set theory.\index{set theory}  This is the approach taken in the
\texttt{set.mm}\index{set theory database (\texttt{set.mm})}%
\index{Metamath Proof Explorer}
database (also known as the Metamath Proof Explorer).
Among other things, this database builds up to the
axioms of real and complex numbers\index{analysis}\index{real and complex
numbers} (see Section~\ref{real}), and a standard development of analysis, for
example, could start at that point, using it as a basis.   Besides this
philosophical advantage, there are practical advantages to having all of the
tools of set theory available in the supporting infrastructure.

On the other hand, you may wish to start with the standard axioms of a
mathematical theory without going through the set theoretical proofs of those
axioms.  You will need mathematical logic to make inferences, but if you wish
you can simply introduce theorems\index{theorem} of logic as
``axioms''\index{axiom} wherever you need them, with the implicit assumption
that in principle they can be proved, if they are obvious to you.  If you
choose this approach, you will probably want to review the notation used in
\texttt{set.mm}\index{set theory database (\texttt{set.mm})} so that your own
notation will be consistent with it.

\subsection{Other Formal Systems}
\index{formal system}

Unlike some programs, Metamath\index{Metamath} is not limited to any specific
area of mathematics, nor committed to any particular mathematical philosophy
such as classical logic versus intuitionism, nor limited, say, to expressions
in first-order logic.  Although the database \texttt{set.mm}
describes standard logic and set theory, Meta\-math
is actually a general-purpose language for describing a wide variety of formal
systems.\index{formal system}  Non-standard systems such as modal
logic,\index{modal logic} intuitionist logic\index{intuitionism}, higher-order
logic\index{higher-order logic}, quantum logic\index{quantum logic}, and
category theory\index{category theory} can all be described with the Metamath
language.  You define the symbols you prefer and tell Metamath the axioms and
rules you want to start from, and Metamath will verify any inferences you make
from those axioms and rules.  A simple example of a non-standard formal system
is Hofstadter's\index{Hofstadter, Douglas R.} MIU system,\index{MIU-system}
whose Metamath description is presented in Appendix~\ref{MIU}.

This is not hypothetical.
The largest Metamath database is
\texttt{set.mm}\index{set theory database (\texttt{set.mm}}%
\index{Metamath Proof Explorer}), aka the Metamath Proof Explorer,
which uses the most common axioms for mathematical foundations
(specifically classical logic combined with Zermelo--Fraenkel
set theory\index{Zermelo--Fraenkel set theory} with the Axiom of Choice).
But other Metamath databases are available:

\begin{itemize}
\item The database
  \texttt{iset.mm}\index{intuitionistic logic database (\texttt{iset.mm})},
  aka the
  Intuitionistic Logic Explorer\index{Intuitionistic Logic Explorer},
  uses intuitionistic logic (a constructivist point of view)
  instead of classical logic.
\item The database
  \texttt{nf.mm}\index{New Foundations database (\texttt{nf.mm})},
  aka the
  New Foundations Explorer\index{New Foundations Explorer},
  constructs mathematics from scratch,
  starting from Quine's New Foundations (NF) set theory axioms.
\item The database
  \texttt{hol.mm}\index{Higher-order Logic database (\texttt{hol.mm})},
  aka the
  Higher-Order Logic (HOL) Explorer\index{Higher-Order Logic (HOL) Explorer},
  starts with HOL (also called simple type theory) and derives
  equivalents to ZFC axioms, connecting the two approaches.
\end{itemize}

Since the days of David Hilbert,\index{Hilbert, David} mathematicians have
been concerned with the fact that the metalanguage\index{metalanguage} used to
describe mathematics may be stronger than the mathematics being described.
Metamath\index{Metamath}'s underlying finitary\index{finitary proof},
constructive nature provides a good philosophical basis for studying even the
weakest logics.\index{weak logic}

The usual treatment of many non-standard formal systems\index{formal
system} uses model theory\index{model theory} or proof theory\index{proof
theory} to describe these systems; these theories, in turn, are based on
standard set theory.  In other words, a non-standard formal system is defined
as a set with certain properties, and standard set theory is used to derive
additional properties of this set.  The standard set theory database provided
with Metamath can be used for this purpose, and when used this way
the development of a special
axiom system for the non-standard formal system becomes unnecessary.  The
model- or proof-theoretic approach often allows you to prove much deeper
results with less effort.

Metamath supports both approaches.  You can define the non-standard
formal system directly, or define the non-standard formal system as
a set with certain properties, whichever you find most helpful.

%\section{Additional Remarks}

\subsection{Metamath and Its Philosophy}

Closely related to Metamath\index{Metamath} is a philosophy or way of looking
at mathematics. This philosophy is related to the formalist
philosophy\index{formalism} of Hilbert\index{Hilbert, David} and his followers
\cite[pp.~1203--1208]{Kline}\index{Kline, Morris}
\cite[p.~6]{Behnke}\index{Behnke, H.}. In this philosophy, mathematics is
viewed as nothing more than a set of rules that manipulate symbols, together
with the consequences of those rules.  While the mathematics being described
may be complex, the rules used to describe it (the
``metamathematics''\index{metamathematics}) should be as simple as possible.
In particular, proofs should be restricted to dealing with concrete objects
(the symbols we write on paper rather than the abstract concepts they
represent) in a constructive manner; these are called ``finitary''
proofs\index{finitary proof} \cite[pp.~2--3]{Shoenfield}\index{Shoenfield,
Joseph R.}.

Whether or not you find Metamath interesting or useful will in part depend on
the appeal you find in its philosophy, and this appeal will probably depend on
your particular goals with respect to mathematics.  For example, if you are a
pure mathematician at the forefront of discovering new mathematical knowledge,
you will probably find that the rigid formality of Metamath stifles your
creativity.  On the other hand, we would argue that once this knowledge is
discovered, there are advantages to documenting it in a standard format that
will make it accessible to others.  Sixty years from now, your field may be
dormant, and as Davis and Hersh put it, your ``writings would become less
translatable than those of the Maya'' \cite[p.~37]{Davis}\index{Davis, Phillip
J.}.


\subsection{A History of the Approach Behind Metamath}

Probably the one work that has had the most motivating influence on
Metamath\index{Metamath} is Whitehead and Russell's monumental {\em Principia
Mathematica} \cite{PM}\index{Whitehead, Alfred North}\index{Russell,
Bertrand}\index{principia mathematica@{\em Principia Mathematica}}, whose aim
was to deduce all of mathematics from a small number of primitive ideas, in a
very explicit way that in principle anyone could understand and follow.  While
this work was tremendously influential in its time, from a modern perspective
it suffers from several drawbacks.  Both its notation and its underlying
axioms are now considered dated and are no longer used.  From our point of
view, its development is not really as accessible as we would like to see; for
practical reasons, proofs become more and more sketchy as its mathematics
progresses, and working them out in fine detail requires a degree of
mathematical skill and patience that many people don't have.  There are
numerous small errors, which is understandable given the tedious, technical
nature of the proofs and the lack of a computer to verify the details.
However, even today {\em Principia Mathematica} stands out as the work closest
in spirit to Metamath.  It remains a mind-boggling work, and one can't help
but be amazed at seeing ``$1+1=2$'' finally appear on page 83 of Volume II
(Theorem *110.643).

The origin of the proof notation used by Metamath dates back to the 1950's,
when the logician C.~A.~Meredith expressed his proofs in a compact notation
called ``condensed detachment''\index{condensed detachment}
\cite{Hindley}\index{Hindley, J. Roger} \cite{Kalman}\index{Kalman, J. A.}
\cite{Meredith}\index{Meredith, C. A.} \cite{Peterson}\index{Peterson, Jeremy
George}.  This notation allows proofs to be communicated unambiguously by
merely referencing the axiom\index{axiom}, rule\index{rule}, or
theorem\index{theorem} used at each step, without explicitly indicating the
substitutions\index{substitution!variable}\index{variable substitution} that
have to be made to the variables in that axiom, rule, or theorem.  Ordinarily,
condensed detachment is more or less limited to propositional
calculus\index{propositional calculus}.  The concept has been extended to
first-order logic\index{first-order logic} in \cite{Megill}\index{Megill,
Norman}, making it is easy to write a small computer program to verify proofs
of simple first-order logic theorems.\index{condensed detachment!and
first-order logic}

A key concept behind the notation of condensed detachment is called
``unification,''\index{unification} which is an algorithm for determining what
substitutions\index{substitution!variable}\index{variable substitution} to
variables have to be made to make two expressions match each other.
Unification was first precisely defined by the logician J.~A.~Robinson, who
used it in the development of a powerful
theorem-proving technique called the ``resolution principle''
\cite{Robinson}\index{Robinson's resolution principle}. Metamath does not make
use of the resolution principle, which is intended for systems of first-order
logic.\index{first-order logic}  Metamath's use is not restricted to
first-order logic, and as we have mentioned it does not automatically discover
proofs.  However, unification is a key idea behind Metamath's proof
notation, and Metamath makes use of a very simple version of it
(Section~\ref{unify}).

\subsection{Metamath and First-Order Logic}

First-order logic\index{first-order logic} is the supporting structure
for standard mathematics.  On top of it is set theory, which contains
the axioms from which virtually all of mathematics can be derived---a
remarkable fact.\index{category
theory}\index{cardinal, inaccessible}\label{categoryth}\footnote{An exception seems
to be category theory.  There are several schools of thought on whether
category theory is derivable from set theory.  At a minimum, it appears
that an additional axiom is needed that asserts the existence of an
``inaccessible cardinal'' (a type of infinity so large that standard set
theory can't prove or deny that it exists).
%
%%%% (I took this out that was in previous editions:)
% But it is also argued that not just one but a ``proper class'' of them
% is needed, and the existence of proper classes is impossible in standard
% set theory.  (A proper class is a collection of sets so huge that no set
% can contain it as an element.  Proper classes can lead to
% inconsistencies such as ``Russell's paradox.''  The axioms of standard
% set theory are devised so as to deny the existence of proper classes.)
%
For more information, see
\cite[pp.~328--331]{Herrlich}\index{Herrlich, Horst} and
\cite{Blass}\index{Blass, Andrea}.}

One of the things that makes Metamath\index{Metamath} more practical for
first-order theories is a set of axioms for first-order logic designed
specifically with Metamath's approach in mind.  These are included in
the database \texttt{set.mm}\index{set theory database (\texttt{set.mm})}.
See Chapter~\ref{fol} for a detailed
description; the axioms are shown in Section~\ref{metaaxioms}.  While
logically equivalent to standard axiom systems, our axiom system breaks
up the standard axioms into smaller pieces such that from them, you can
directly derive what in other systems can only be derived as higher-level
``metatheorems.''\index{metatheorem}  In other words, it is more powerful than
the standard axioms from a metalogical point of view.  A rigorous
justification for this system and its ``metalogical
completeness''\index{metalogical completeness} is found in
\cite{Megill}\index{Megill, Norman}.  The system is closely related to a
system developed by Monk\index{Monk, J. Donald} and Tarski\index{Tarski,
Alfred} in 1965 \cite{Monks}.

For example, the formula $\exists x \, x = y $ (given $y$, there exists some
$x$ equal to it) is a theorem of logic,\footnote{Specifically, it is a theorem
of those systems of logic that assume non-empty domains.  It is not a theorem
of more general systems that include the empty domain\index{empty domain}, in
which nothing exists, period!  Such systems are called ``free
logics.''\index{free logic} For a discussion of these systems, see
\cite{Leblanc}\index{Leblanc, Hugues}.  Since our use for logic is as a basis
for set theory, which has a non-empty domain, it is more convenient (and more
traditional) to use a less general system.  An interesting curiosity is that,
using a free logic as a basis for Zermelo--Fraenkel set
theory\index{Zermelo--Fraenkel set theory} (with the redundant Axiom of the
Null Set omitted),\index{Axiom of the Null Set} we cannot even prove the
existence of a single set without assuming the axiom of infinity!\index{Axiom
of Infinity}} whether or not $x$ and $y$ are distinct variables\index{distinct
variables}.  In many systems of logic, we would have to prove two theorems to
arrive at this result.  First we would prove ``$\exists x \, x = x $,'' then
we would separately prove ``$\exists x \, x = y $, where $x$ and $y$ are
distinct variables.''  We would then combine these two special cases ``outside
of the system'' (i.e.\ in our heads) to be able to claim, ``$\exists x \, x =
y $, regardless of whether $x$ and $y$ are distinct.''  In other words, the
combination of the two special cases is a
metatheorem.  In the system of logic
used in Metamath's set theory\index{set theory database (\texttt{set.mm})}
database, the axioms of logic are broken down into small pieces that allow
them to be reassembled in such a way that theorems such as these can be proved
directly.

Breaking down the axioms in this way makes them look peculiar and not very
intuitive at first, but rest assured that they are correct and complete.  Their
correctness is ensured because they are theorem schemes of standard first-order
logic (which you can easily verify if you are a logician).  Their completeness
follows from the fact that we explicitly derive the standard axioms of
first-order logic as theorems.  Deriving the standard axioms is somewhat
tricky, but once we're there, we have at our disposal a system that is less
awkward to work with in formal proofs\index{formal proof}.  In technical terms
that logicians understand, we eliminate the cumbersome concepts of ``free
variable,''\index{free variable} ``bound variable,''\index{bound variable} and
``proper substitution''\index{proper substitution}\index{substitution!proper}
as primitive notions.  These concepts are present in our system but are
defined in terms of concepts expressed by the axioms and can be eliminated in
principle.  In standard systems, these concepts are really like additional,
implicit axioms\index{implicit axiom} that are somewhat complex and cannot be
eliminated.

The traditional approach to logic, wherein free variables and proper
substitution is defined, is also possible to do directly in the Metamath
language.  However, the notation tends to become awkward, and there are
disadvantages:  for example, extending the definition of a wff with a
definition is awkward, because the free variable and proper substitution
concepts have to have their definitions also extended.  Our choice of
axioms for \texttt{set.mm} is to a certain extent a matter of style, in
an attempt to achieve overall simplicity, but you should also be aware
that the traditional approach is possible as well if you should choose
it.

\chapter{Using the Metamath Program}
\label{using}

\section{Installation}

The way that you install Metamath\index{Metamath!installation} on your
computer system will vary for different computers.  Current
instructions are provided with the Metamath program download at
\url{http://metamath.org}.  In general, the installation is simple.
There is one file containing the Metamath program itself.  This file is
usually called \texttt{metamath} or \texttt{metamath.}{\em xxx} where
{\em xxx} is the convention (such as \texttt{exe}) for an executable
program on your operating system.  There are several additional files
containing samples of the Metamath language, all ending with
\texttt{.mm}.  The file \texttt{set.mm}\index{set theory database
(\texttt{set.mm})} contains logic and set theory and can be used as a
starting point for other areas of mathematics.

You will also need a text editor\index{text editor} capable of editing plain
{\sc ascii}\footnote{American Standard Code for Information Interchange.} text
in order to prepare your input files.\index{ascii@{\sc ascii}}  Most computers
have this capability built in.  Note that plain text is not necessarily the
default for some word processing programs\index{word processor}, especially if
they can handle different fonts; for example, with Microsoft Word\index{Word
(Microsoft)}, you must save the file in the format ``Text Only With Line
Breaks'' to get a plain text\index{plain text} file.\footnote{It is
recommended that all lines in a Metamath source file be 79 characters or less
in length for compatibility among different computer terminals.  When creating
a source file on an editor such as Word, select a monospaced
font\index{monospaced font} such as Courier\index{Courier font} or
Monaco\index{Monaco font} to make this easier to achieve.  Better yet,
just use a plain text editor such as Notepad.}

On some computer systems, Metamath does not have the capability to print
its output directly; instead, you send its output to a file (using the
\texttt{open} commands described later).  The way you print this output
file depends on your computer.\index{printers} Some computers have a
print command, whereas with others, you may have to read the file into
an editor and print it from there.

If you want to print your Metamath source files with typeset formulas
containing standard mathematical symbols, you will need the \LaTeX\
typesetting program\index{latex@{\LaTeX}}, which is widely and freely
available for most operating systems.  It runs natively on Unix and
Linux, and can be installed on Windows as part of the free Cygwin
package (\url{http://cygwin.com}).

You can also produce {\sc html}\footnote{HyperText Markup Language.}
web pages.  The {\tt help html} command in the Metamath program will
assist you with this feature.

\section{Your First Formal System}\label{start}
\subsection{From Nothing to Zero}\label{startf}

To give you a feel for what the Metamath\index{Metamath} language looks like,
we will take a look at a very simple example from formal number
theory\index{number theory}.  This example is taken from
Mendelson\index{Mendelson, Elliot} \cite[p. 123]{Mendelson}.\footnote{To keep
the example simple, we have changed the formalism slightly, and what we call
axioms\index{axiom} are strictly speaking theorems\index{theorem} in
\cite{Mendelson}.}  We will look at a small subset of this theory, namely that
part needed for the first number theory theorem proved in \cite{Mendelson}.

First we will look at a standard formal proof\index{formal proof} for the
example we have picked, then we will look at the Metamath version.  If you
have never been exposed to formal proofs, the notation may seem to be such
overkill to express such simple notions that you may wonder if you are missing
something.  You aren't.  The concepts involved are in fact very simple, and a
detailed breakdown in this fashion is necessary to express the proof in a way
that can be verified mechanically.  And as you will see, Metamath breaks the
proof down into even finer pieces so that the mechanical verification process
can be about as simple as possible.

Before we can introduce the axioms\index{axiom} of the theory, we must define
the syntax rules for forming legal expressions\index{syntax rules}
(combinations of symbols) with which those axioms can be used. The number 0 is
a {\bf term}\index{term}; and if $ t$ and $r$ are terms, so is $(t+r)$. Here,
$ t$ and $r$ are ``metavariables''\index{metavariable} ranging over terms; they
themselves do not appear as symbols in an actual term.  Some examples of
actual terms are $(0 + 0)$ and $((0+0)+0)$.  (Note that our theory describes
only the number zero and sums of zeroes.  Of course, not much can be done with
such a trivial theory, but remember that we have picked a very small subset of
complete number theory for our example.  The important thing for you to focus
on is our definitions that describe how symbols are combined to form valid
expressions, and not on the content or meaning of those expressions.) If $ t$
and $r$ are terms, an expression of the form $ t=r$ is a {\bf wff}
(well-formed formula)\index{well-formed formula (wff)}; and if $P$ and $Q$ are
wffs, so is $(P\rightarrow Q)$ (which means ``$P$ implies
$Q$''\index{implication ($\rightarrow$)} or ``if $P$ then $Q$'').
Here $P$ and $Q$ are metavariables ranging over wffs.  Examples of actual
wffs are $0=0$, $(0+0)=0$, $(0=0 \rightarrow (0+0)=0)$, and $(0=0\rightarrow
(0=0\rightarrow 0=(0+0)))$.  (Our notation makes use of more parentheses than
are customary, but the elimination of ambiguity this way simplifies our
example by avoiding the need to define operator precedence\index{operator
precedence}.)

The {\bf axioms}\index{axiom} of our theory are all wffs of the following
form, where $ t$, $r$, and $s$ are any terms:

%Latex p. 92
\renewcommand{\theequation}{A\arabic{equation}}

\begin{equation}
(t=r\rightarrow (t=s\rightarrow r=s))
\end{equation}
\begin{equation}
(t+0)=t
\end{equation}

Note that there are an infinite number of axioms since there are an infinite
number of possible terms.  A1 and A2 are properly called ``axiom
schemes,''\index{axiom scheme} but we will refer to them as ``axioms'' for
brevity.

An axiom is a {\bf theorem}; and if $P$ and $(P\rightarrow Q)$ are theorems
(where $P$ and $Q$ are wffs), then $Q$ is also a theorem.\index{theorem}  The
second part of this definition is called the modus ponens (MP) rule of
inference\index{inference rule}\index{modus ponens}.  It allows us to obtain
new theorems from old ones.

The {\bf proof}\index{proof} of a theorem is a sequence of one or more
theorems, each of which is either an axiom or the result of modus ponens
applied to two previous theorems in the sequence, and the last of which is the
theorem being proved.

The theorem we will prove for our example is very simple:  $ t=t$.  The proof of
our theorem follows.  Study it carefully until you feel sure you
understand it.\label{zeroproof}

% Use tabu so that lines will wrap automatically as needed.
\begin{tabu} { l X X }
1. & $(t+0)=t$ & (by axiom A2) \\
2. & $(t+0)=t$ & (by axiom A2) \\
3. & $((t+0)=t \rightarrow ((t+0)=t\rightarrow t=t))$ & (by axiom A1) \\
4. & $((t+0)=t\rightarrow t=t)$ & (by MP applied to steps 2 and 3) \\
5. & $t=t$ & (by MP applied to steps 1 and 4) \\
\end{tabu}

(You may wonder why step 1 is repeated twice.  This is not necessary in the
formal language we have defined, but in Metamath's ``reverse Polish
notation''\index{reverse Polish notation (RPN)} for proofs, a previous step
can be referred to only once.  The repetition of step~1 here will enable you
to see more clearly the correspondence of this proof with the
Metamath\index{Metamath} version on p.~\pageref{demoproof}.)

Our theorem is more properly called a ``theorem scheme,''\index{theorem
scheme} for it represents an infinite number of theorems, one for each
possible term $ t$.  Two examples of actual theorems would be $0=0$ and
$(0+0)=(0+0)$.  Rarely do we prove actual theorems, since by proving schemes
we can prove an infinite number of theorems in one fell swoop.  Similarly, our
proof should really be called a ``proof scheme.''\index{proof scheme}  To
obtain an actual proof, pick an actual term to use in place of $ t$, and
substitute it for $ t$ throughout the proof.

Let's discuss what we have done here.  The axioms\index{axiom} of our theory,
A1 and A2, are trivial and obvious.  Everyone knows that adding zero to
something doesn't change it, and also that if two things are equal to a third,
then they are equal to each other. In fact, stating the trivial and obvious is
a goal to strive for in any axiomatic system.  From trivial and obvious truths
that everyone agrees upon, we can prove results that are not so obvious yet
have absolute faith in them.  If we trust the axioms and the rules, we must,
by definition, trust the consequences of those axioms and rules, if logic is
to mean anything at all.

Our rule of inference\index{rule}, modus ponens\index{modus ponens}, is also
pretty obvious once you understand what it means.  If we prove a fact $P$, and
we also prove that $P$ implies $Q$, then $Q$ necessarily follows as a new
fact.  The rule provides us with a means for obtaining new facts (i.e.\
theorems\index{theorem}) from old ones.

The theorem that we have proved, $ t=t$, is so fundamental that you may wonder
why it isn't one of the axioms\index{axiom}.  In some axiom systems of
arithmetic, it {\em is} an axiom.  The choice of axioms in a theory is to some
extent arbitrary and even an art form, constrained only by the requirement
that any two equivalent axiom systems be able to derive each other as
theorems.  We could imagine that the inventor of our axiom system originally
included $ t=t$ as an axiom, then discovered that it could be derived as a
theorem from the other axioms.  Because of this, it was not necessary to
keep it as an axiom.  By eliminating it, the final set of axioms became
that much simpler.

Unless you have worked with formal proofs\index{formal proof} before, it
probably wasn't apparent to you that $ t=t$ could be derived from our two
axioms until you saw the proof. While you certainly believe that $ t=t$ is
true, you might not be able to convince an imaginary skeptic who believes only
in our two axioms until you produce the proof.  Formal proofs such as this are
hard to come up with when you first start working with them, but after you get
used to them they can become interesting and fun.  Once you understand the
idea behind formal proofs you will have grasped the fundamental principle that
underlies all of mathematics.  As the mathematics becomes more sophisticated,
its proofs become more challenging, but ultimately they all can be broken down
into individual steps as simple as the ones in our proof above.

Mendelson's\index{Mendelson, Elliot} book, from which our example was taken,
contains a number of detailed formal proofs such as these, and you may be
interested in looking it up.  The book is intended for mathematicians,
however, and most of it is rather advanced.  Popular literature describing
formal proofs\index{formal proof} include \cite[p.~296]{Rucker}\index{Rucker,
Rudy} and \cite[pp.~204--230]{Hofstadter}\index{Hofstadter, Douglas R.}.

\subsection{Converting It to Metamath}\label{convert}

Formal proofs\index{formal proof} such as the one in our example break down
logical reasoning into small, precise steps that leave little doubt that the
results follow from the axioms\index{axiom}.  You might think that this would
be the finest breakdown we can achieve in mathematics.  However, there is more
to the proof than meets the eye. Although our axioms were rather simple, a lot
of verbiage was needed before we could even state them:  we needed to define
``term,'' ``wff,'' and so on.  In addition, there are a number of implied
rules that we haven't even mentioned. For example, how do we know that step 3
of our proof follows from axiom A1? There is some hidden reasoning involved in
determining this.  Axiom A1 has two occurrences of the letter $ t$.  One of
the implied rules states that whatever we substitute for $ t$ must be a legal
term\index{term}.\footnote{Some authors make this implied rule explicit by
stating, ``only expressions of the above form are terms,'' after defining
``term.''}  The expression $ t+0$ is pretty obviously a legal term whenever $
t$ is, but suppose we wanted to substitute a huge term with thousands of
symbols?  Certainly a lot of work would be involved in determining that it
really is a term, but in ordinary formal proofs all of this work would be
considered a single ``step.''

To express our axiom system in the Metamath\index{Metamath} language, we must
describe this auxiliary information in addition to the axioms themselves.
Metamath does not know what a ``term'' or a ``wff''\index{well-formed formula
(wff)} is.  In Metamath, the specification of the ways in which we can combine
symbols to obtain terms and wffs are like little axioms in themselves.  These
auxiliary axioms are expressed in the same notation as the ``real''
axioms\index{axiom}, and Metamath does not distinguish between the two.  The
distinction is made by you, i.e.\ by the way in which you interpret the
notation you have chosen to express these two kinds of axioms.

The Metamath language breaks down mathematical proofs into tiny pieces, much
more so than in ordinary formal proofs\index{formal proof}.  If a single
step\index{proof step} involves the
substitution\index{substitution!variable}\index{variable substitution} of a
complex term for one of its variables, Metamath must see this single step
broken down into many small steps.  This fine-grained breakdown is what gives
Metamath generality and flexibility as it lets it not be limited to any
particular mathematical notation.

Metamath's proof notation is not, in itself, intended to be read by humans but
rather is in a compact format intended for a machine.  The Metamath program
will convert this notation to a form you can understand, using the \texttt{show
proof}\index{\texttt{show proof} command} command.  You can tell the program what
level of detail of the proof you want to look at.  You may want to look at
just the logical inference steps that correspond
to ordinary formal proof steps,
or you may want to see the fine-grained steps that prove that an expression is
a term.

Here, without further ado, is our example converted to the
Metamath\index{Metamath} language:\index{metavariable}\label{demo0}

\begin{verbatim}
$( Declare the constant symbols we will use $)
    $c 0 + = -> ( ) term wff |- $.
$( Declare the metavariables we will use $)
    $v t r s P Q $.
$( Specify properties of the metavariables $)
    tt $f term t $.
    tr $f term r $.
    ts $f term s $.
    wp $f wff P $.
    wq $f wff Q $.
$( Define "term" and "wff" $)
    tze $a term 0 $.
    tpl $a term ( t + r ) $.
    weq $a wff t = r $.
    wim $a wff ( P -> Q ) $.
$( State the axioms $)
    a1 $a |- ( t = r -> ( t = s -> r = s ) ) $.
    a2 $a |- ( t + 0 ) = t $.
$( Define the modus ponens inference rule $)
    ${
       min $e |- P $.
       maj $e |- ( P -> Q ) $.
       mp  $a |- Q $.
    $}
$( Prove a theorem $)
    th1 $p |- t = t $=
  $( Here is its proof: $)
       tt tze tpl tt weq tt tt weq tt a2 tt tze tpl
       tt weq tt tze tpl tt weq tt tt weq wim tt a2
       tt tze tpl tt tt a1 mp mp
     $.
\end{verbatim}\index{metavariable}

A ``database''\index{database} is a set of one or more {\sc ascii} source
files.  Here's a brief description of this Metamath\index{Metamath} database
(which consists of this single source file), so that you can understand in
general terms what is going on.  To understand the source file in detail, you
should read Chapter~\ref{languagespec}.

The database is a sequence of ``tokens,''\index{token} which are normally
separated by spaces or line breaks.  The only tokens that are built into
the Metamath language are those beginning with \texttt{\$}.  These tokens
are called ``keywords.''\index{keyword}  All other tokens are
user-defined, and their names are arbitrary.

As you might have guessed, the Metamath token \texttt{\$(}\index{\texttt{\$(} and
\texttt{\$)} auxiliary keywords} starts a comment and \texttt{\$)} ends a comment.

The Metamath tokens \texttt{\$c}\index{\texttt{\$c} statement},
\texttt{\$v}\index{\texttt{\$v} statement},
\texttt{\$e}\index{\texttt{\$e} statement},
\texttt{\$f}\index{\texttt{\$f} statement},
\texttt{\$a}\index{\texttt{\$a} statement}, and
\texttt{\$p}\index{\texttt{\$p} statement} specify ``statements'' that
end with \texttt{\$.}\,.\index{\texttt{\$.}\ keyword}

The Metamath tokens \texttt{\$c} and \texttt{\$v} each declare\index{constant
declaration}\index{variable declaration} a list of user-defined tokens, called
``math symbols,''\index{math symbol} that the database will reference later
on.  All of the math symbols they define you have seen earlier except the
turnstile symbol \texttt{|-} ($\vdash$)\index{turnstile ({$\,\vdash$})}, which is
commonly used by logicians to mean ``a proof exists for.''  For us
the turnstile is just a
convenient symbol that distinguishes expressions that are axioms\index{axiom}
or theorems\index{theorem} from expressions that are terms or wffs.

The \texttt{\$c} statement declares ``constants''\index{constant} and
the \texttt{\$v} statement declares
``variables''\index{variable}\index{constant declaration}\index{variable
declaration} (or more precisely, metavariables\index{metavariable}).  A
variable may be substituted\index{substitution!variable}\index{variable
substitution} with sequences of math symbols whereas a constant may not
be substituted with anything.

It may seem redundant to require both \texttt{\$c}\index{\texttt{\$c} statement} and
\texttt{\$v}\index{\texttt{\$v} statement} statements (since any math
symbol\index{math symbol} not specified with a \texttt{\$c} statement could be
presumed to be a variable), but this provides for better error checking and
also allows math symbols to be redeclared\index{redeclaration of symbols}
(Section~\ref{scoping}).

The token \texttt{\$f}\index{\texttt{\$f} statement} specifies a
statement called a ``variable-type hypothesis'' (also called a
``floating hypothesis'') and \texttt{\$e}\index{\texttt{\$e} statement}
specifies a ``logical hypothesis'' (also called an ``essential
hypothesis'').\index{hypothesis}\index{variable-type
hypothesis}\index{logical hypothesis}\index{floating
hypothesis}\index{essential hypothesis} The token
\texttt{\$a}\index{\texttt{\$a} statement} specifies an ``axiomatic
assertion,''\index{axiomatic assertion} and
\texttt{\$p}\index{\texttt{\$p} statement} specifies a ``provable
assertion.''\index{provable assertion} To the left of each occurrence of
these four tokens is a ``label''\index{label} that identifies the
hypothesis or assertion for later reference.  For example, the label of
the first axiomatic assertion is \texttt{tze}.  A \texttt{\$f} statement
must contain exactly two math symbols, a constant followed by a
variable.  The \texttt{\$e}, \texttt{\$a}, and \texttt{\$p} statements
each start with a constant followed by, in general, an arbitrary
sequence of math symbols.

Associated with each assertion\index{assertion} is a set of hypotheses
that must be satisfied in order for the assertion to be used in a proof.
These are called the ``mandatory hypotheses''\index{mandatory
hypothesis} of the assertion.  Among those hypotheses whose ``scope''
(described below) includes the assertion, \texttt{\$e} hypotheses are
always mandatory and \texttt{\$f}\index{\texttt{\$f} statement}
hypotheses are mandatory when they share their variable with the
assertion or its \texttt{\$e} hypotheses.  The exact rules for
determining which hypotheses are mandatory are described in detail in
Sections~\ref{frames} and \ref{scoping}.  For example, the mandatory
hypotheses of assertion \texttt{tpl} are \texttt{tt} and \texttt{tr},
whereas assertion \texttt{tze} has no mandatory hypotheses because it
contains no variables and has no \texttt{\$e}\index{\texttt{\$e}
statement} hypothesis.  Metamath's \texttt{show statement}
command\index{\texttt{show statement} command}, described in the next
section, will show you a statement's mandatory hypotheses.

Sometimes we need to make a hypothesis relevant to only certain
assertions.  The set of statements to which a hypothesis is relevant is
called its ``scope.''  The Metamath brackets,
\texttt{\$\char`\{}\index{\texttt{\$\char`\{} and \texttt{\$\char`\}}
keywords} and \texttt{\$\char`\}}, define a ``block''\index{block} that
delimits the scope of any hypothesis contained between them.  The
assertion \texttt{mp} has mandatory hypotheses \texttt{wp}, \texttt{wq},
\texttt{min}, and \texttt{maj}.  The only mandatory hypothesis of
\texttt{th1}, on the other hand, is \texttt{tt}, since \texttt{th1}
occurs outside of the block containing \texttt{min} and \texttt{maj}.

Note that \texttt{\$\char`\{} and \texttt{\$\char`\}} do not affect the
scope of assertions (\texttt{\$a} and \texttt{\$p}).  Assertions are always
available to be referenced by any later proof in the source file.

Each provable assertion (\texttt{\$p}\index{\texttt{\$p} statement}
statement) has two parts.  The first part is the
assertion\index{assertion} itself, which is a sequence of math
symbol\index{math symbol} tokens placed between the \texttt{\$p} token
and a \texttt{\$=}\index{\texttt{\$=} keyword} token.  The second part
is a ``proof,'' which is a list of label tokens placed between the
\texttt{\$=} token and the \texttt{\$.}\index{\texttt{\$.}\ keyword}\
token that ends the statement.\footnote{If you've looked at the
\texttt{set.mm} database, you may have noticed another notation used for
proofs.  The other notation is called ``compressed.''\index{compressed
proof}\index{proof!compressed} It reduces the amount of space needed to
store a proof in the database and is described in
Appendix~\ref{compressed}.  In the example above, we use
``normal''\index{normal proof}\index{proof!normal} notation.} The proof
acts as a series of instructions to the Metamath program, telling it how
to build up the sequence of math symbols contained in the assertion part of
the \texttt{\$p} statement, making use of the hypotheses of the
\texttt{\$p} statement and previous assertions.  The construction takes
place according to precise rules.  If the list of labels in the proof
causes these rules to be violated, or if the final sequence that results
does not match the assertion, the Metamath program will notify you with
an error message.

If you are familiar with reverse Polish notation (RPN), which is sometimes used
on pocket calculators, here in a nutshell is how a proof works.  Each
hypothesis label\index{hypothesis label} in the proof is pushed\index{push}
onto the RPN stack\index{stack}\index{RPN stack} as it is encountered. Each
assertion label\index{assertion label} pops\index{pop} off the stack as many
entries as the referenced assertion has mandatory hypotheses.  Variable
substitutions\index{substitution!variable}\index{variable substitution} are
computed which, when made to the referenced assertion's mandatory hypotheses,
cause these hypotheses to match the stack entries. These same substitutions
are then made to the variables in the referenced assertion itself, which is
then pushed onto the stack.  At the end of the proof, there should be one
stack entry, namely the assertion being proved.  This process is explained in
detail in Section~\ref{proof}.

Metamath's proof notation is not very readable for humans, but it allows the
proof to be stored compactly in a file.  The Metamath\index{Metamath} program
has proof display features that let you see what's going on in a more
readable way, as you will see in the next section.

The rules used in verifying a proof are not based on any built-in syntax of the
symbol sequence in an assertion\index{assertion} nor on any built-in meanings
attached to specific symbol names.  They are based strictly on symbol
matching:  constants\index{constant} must match themselves, and
variables\index{variable} may be replaced with anything that allows a match to
occur.  For example, instead of \texttt{term}, \texttt{0}, and \verb$|-$ we could
have just as well used \texttt{yellow}, \texttt{zero}, and \texttt{provable}, as long
as we did so consistently throughout the database.  Also, we could have used
\texttt{is provable} (two tokens) instead of \verb$|-$ (one token) throughout the
database.  In each of these cases, the proof would be exactly the same.  The
independence of proofs and notation means that you have a lot of flexibility to
change the notation you use without having to change any proofs.

\section{A Trial Run}\label{trialrun}

Now you are ready to try out the Metamath\index{Metamath} program.

On all computer systems, Metamath has a standard ``command line
interface'' (CLI)\index{command line interface (CLI)} that allows you to
interact with it.  You supply commands to the CLI by typing them on the
keyboard and pressing your keyboard's {\em return} key after each line
you enter.  The CLI is designed to be easy to use and has built-in help
features.

The first thing you should do is to use a text editor to create a file
called \texttt{demo0.mm} and type into it the Metamath source shown on
p.~\pageref{demo0}.  Actually, this file is included with your Metamath
software package, so check that first.  If you type it in, make sure
that you save it in the form of ``plain {\sc ascii} text with line
breaks.''  Most word processors will have this feature.

Next you must run the Metamath program.  Depending on your computer
system and how Metamath is installed, this could range from clicking the
mouse on the Metamath icon to typing \texttt{run metamath} to typing
simply \texttt{metamath}.  (Metamath's {\tt help invoke} command describes
alternate ways of invoking the Metamath program.)

When you first enter Metamath\index{Metamath}, it will be at the CLI, waiting
for your input. You will see something like the following on your screen:
\begin{verbatim}
Metamath - Version 0.172 25-Jan-2019
Type HELP for help, EXIT to exit.
MM>
\end{verbatim}
The \texttt{MM>} prompt means that Metamath is waiting for a command.
Command keywords\index{command keyword} are not case sensitive;
we will use lower-case commands in our examples.
The version number and its release date will probably be different on your
system from the one we show above.

The first thing that you need to do is to read in your
database:\index{\texttt{read} command}\footnote{If a directory path is
needed on Unix,\index{Unix file names}\index{file names!Unix} you should
enclose the path/file name in quotes to prevent Metamath from thinking
that the \texttt{/} in the path name is a command qualifier, e.g.,
\texttt{read \char`\"db/set.mm\char`\"}.  Quotes are optional when there
is no ambiguity.}
\begin{verbatim}
MM> read demo0.mm
\end{verbatim}
Remember to press the {\em return} key after entering this command.  If
you omit the file name, Metamath will prompt you for one.   The syntax for
specifying a Macintosh file name path is given in a footnote on
p.~\pageref{includef}.\index{Macintosh file names}\index{file
names!Macintosh}

If there are any syntax errors in the database, Metamath will let you know
when it reads in the file.  The one thing that Metamath does not check when
reading in a database is that all proofs are correct, because this would
slow it down too much.  It is a good idea to periodically verify the proofs in
a database you are making changes to.  To do this, use the following command
(and do it for your \texttt{demo0.mm} file now).  Note that the \texttt{*} is a
``wild card'' meaning all proofs in the file.\index{\texttt{verify proof} command}
\begin{verbatim}
MM> verify proof *
\end{verbatim}
Metamath will report any proofs that are incorrect.

It is often useful to save the information that the Metamath program displays
on the screen. You can save everything that happens on the screen by opening a
log file. You may want to do this before you read in a database so that you
can examine any errors later on.  To open a log file, type
\begin{verbatim}
MM> open log abc.log
\end{verbatim}
This will open a file called \texttt{abc.log}, and everything that appears on the
screen from this point on will be stored in this file.  The name of the log file
is arbitrary. To close the log file, type
\begin{verbatim}
MM> close log
\end{verbatim}

Several commands let you examine what's inside your database.
Section~\ref{exploring} has an overview of some useful ones.  The
\texttt{show labels} command lets you see what statement
labels\index{label} exist.  A \texttt{*} matches any combination of
characters, and \texttt{t*} refers to all labels starting with the
letter \texttt{t}.\index{\texttt{show labels} command} The \texttt{/all}
is a ``command qualifier''\index{command qualifier} that tells Metamath
to include labels of hypotheses.  (To see the syntax explained, type
\texttt{help show labels}.)  Type
\begin{verbatim}
MM> show labels t* /all
\end{verbatim}
Metamath will respond with
\begin{verbatim}
The statement number, label, and type are shown.
3 tt $f       4 tr $f       5 ts $f       8 tze $a
9 tpl $a      19 th1 $p
\end{verbatim}

You can use the \texttt{show statement} command to get information about a
particular statement.\index{\texttt{show statement} command}
For example, you can get information about the statement with label \texttt{mp}
by typing
\begin{verbatim}
MM> show statement mp /full
\end{verbatim}
Metamath will respond with
\begin{verbatim}
Statement 17 is located on line 43 of the file
"demo0.mm".
"Define the modus ponens inference rule"
17 mp $a |- Q $.
Its mandatory hypotheses in RPN order are:
  wp $f wff P $.
  wq $f wff Q $.
  min $e |- P $.
  maj $e |- ( P -> Q ) $.
The statement and its hypotheses require the
      variables:  Q P
The variables it contains are:  Q P
\end{verbatim}
The mandatory hypotheses\index{mandatory hypothesis} and their
order\index{RPN order} are
useful to know when you are trying to understand or debug a proof.

Now you are ready to look at what's really inside our proof.  First, here is
how to look at every step in the proof---not just the ones corresponding to an
ordinary formal proof\index{formal proof}, but also the ones that build up the
formulas that appear in each ordinary formal proof step.\index{\texttt{show
proof} command}
\begin{verbatim}
MM> show proof th1 /lemmon /all
\end{verbatim}

This will display the proof on the screen in the following format:
\begin{verbatim}
 1 tt            $f term t
 2 tze           $a term 0
 3 1,2 tpl       $a term ( t + 0 )
 4 tt            $f term t
 5 3,4 weq       $a wff ( t + 0 ) = t
 6 tt            $f term t
 7 tt            $f term t
 8 6,7 weq       $a wff t = t
 9 tt            $f term t
10 9 a2          $a |- ( t + 0 ) = t
11 tt            $f term t
12 tze           $a term 0
13 11,12 tpl     $a term ( t + 0 )
14 tt            $f term t
15 13,14 weq     $a wff ( t + 0 ) = t
16 tt            $f term t
17 tze           $a term 0
18 16,17 tpl     $a term ( t + 0 )
19 tt            $f term t
20 18,19 weq     $a wff ( t + 0 ) = t
21 tt            $f term t
22 tt            $f term t
23 21,22 weq     $a wff t = t
24 20,23 wim     $a wff ( ( t + 0 ) = t -> t = t )
25 tt            $f term t
26 25 a2         $a |- ( t + 0 ) = t
27 tt            $f term t
28 tze           $a term 0
29 27,28 tpl     $a term ( t + 0 )
30 tt            $f term t
31 tt            $f term t
32 29,30,31 a1   $a |- ( ( t + 0 ) = t -> ( ( t + 0 )
                                     = t -> t = t ) )
33 15,24,26,32 mp  $a |- ( ( t + 0 ) = t -> t = t )
34 5,8,10,33 mp  $a |- t = t
\end{verbatim}

The \texttt{/lemmon} command qualifier specifies what is known as a Lemmon-style
display\index{Lemmon-style proof}\index{proof!Lemmon-style}.  Omitting the
\texttt{/lemmon} qualifier results in a tree-style proof (see
p.~\pageref{treeproof} for an example) that is somewhat less explicit but
easier to follow once you get used to it.\index{tree-style
proof}\index{proof!tree-style}

The first number on each line is the step
number of the proof.  Any numbers that follow are step numbers assigned to the
hypotheses of the statement referenced by that step.  Next is the label of
the statement referenced by the step.  The statement type of the statement
referenced comes next, followed by the math symbol\index{math symbol} string
constructed by the proof up to that step.

The last step, 34, contains the statement that is being proved.

Looking at a small piece of the proof, notice that steps 3 and 4 have
established that
\texttt{( t + 0 )} and \texttt{t} are \texttt{term}\,s, and step 5 makes use of steps 3 and
4 to establish that \texttt{( t + 0 ) = t} is a \texttt{wff}.  Let Metamath
itself tell us in detail what is happening in step 5.  Note that the
``target hypothesis'' refers to where step 5 is eventually used, i.e., in step
34.
\begin{verbatim}
MM> show proof th1 /detailed_step 5
Proof step 5:  wp=weq $a wff ( t + 0 ) = t
This step assigns source "weq" ($a) to target "wp"
($f).  The source assertion requires the hypotheses
"tt" ($f, step 3) and "tr" ($f, step 4).  The parent
assertion of the target hypothesis is "mp" ($a,
step 34).
The source assertion before substitution was:
    weq $a wff t = r
The following substitutions were made to the source
assertion:
    Variable  Substituted with
     t         ( t + 0 )
     r         t
The target hypothesis before substitution was:
    wp $f wff P
The following substitution was made to the target
hypothesis:
    Variable  Substituted with
     P         ( t + 0 ) = t
\end{verbatim}

The full proof just shown is useful to understand what is going on in detail.
However, most of the time you will just be interested in
the ``essential'' or logical steps of a proof, i.e.\ those steps
that correspond to an
ordinary formal proof\index{formal proof}.  If you type
\begin{verbatim}
MM> show proof th1 /lemmon /renumber
\end{verbatim}
you will see\label{demoproof}
\begin{verbatim}
1 a2             $a |- ( t + 0 ) = t
2 a2             $a |- ( t + 0 ) = t
3 a1             $a |- ( ( t + 0 ) = t -> ( ( t + 0 )
                                     = t -> t = t ) )
4 2,3 mp         $a |- ( ( t + 0 ) = t -> t = t )
5 1,4 mp         $a |- t = t
\end{verbatim}
Compare this to the formal proof on p.~\pageref{zeroproof} and
notice the resemblance.
By default Metamath
does not show \texttt{\$f}\index{\texttt{\$f}
statement} hypotheses and everything branching off of them in the proof tree
when the proof is displayed; this makes the proof look more like an ordinary
mathematical proof, which does not normally incorporate the explicit
construction of expressions.
This is called the ``essential'' view
(at one time you had to add the
\texttt{/essential} qualifier in the \texttt{show proof}
command to get this view, but this is now the default).
You can could use the \texttt{/all} qualifier in the \texttt{show
proof} command to also show the explicit construction of expressions.
The \texttt{/renumber} qualifier means to renumber
the steps to correspond only to what is displayed.\index{\texttt{show proof}
command}

To exit Metamath, type\index{\texttt{exit} command}
\begin{verbatim}
MM> exit
\end{verbatim}

\subsection{Some Hints for Using the Command Line Interface}

We will conclude this quick introduction to Metamath\index{Metamath} with some
helpful hints on how to navigate your way through the commands.
\index{command line interface (CLI)}

When you type commands into Metamath's CLI, you only have to type as many
characters of a command keyword\index{command keyword} as are needed to make
it unambiguous.  If you type too few characters, Metamath will tell you what
the choices are.  In the case of the \texttt{read} command, only the \texttt{r} is
needed to specify it unambiguously, so you could have typed\index{\texttt{read}
command}
\begin{verbatim}
MM> r demo0.mm
\end{verbatim}
instead of
\begin{verbatim}
MM> read demo0.mm
\end{verbatim}
In our description, we always show the full command words.  When using the
Metamath CLI commands in a command file (to be read with the \texttt{submit}
command)\index{\texttt{submit} command}, it is good practice to use
the unabbreviated command to ensure your instructions will not become ambiguous
if more commands are added to the Metamath program in the future.

The command keywords\index{command
keyword} are not case sensitive; you may type either \texttt{read} or
\texttt{ReAd}.  File names may or may not be case sensitive, depending on your
computer's operating system.  Metamath label\index{label} and math
symbol\index{math symbol} tokens\index{token} are case-sensitive.

The \texttt{help} command\index{\texttt{help} command} will provide you
with a list of topics you can get help on.  You can then type
\texttt{help} {\em topic} to get help on that topic.

If you are uncertain of a command's spelling, just type as many characters
as you remember of the command.  If you have not typed enough characters to
specify it unambiguously, Metamath will tell you what choices you have.

\begin{verbatim}
MM> show s
         ^
?Ambiguous keyword - please specify SETTINGS,
STATEMENT, or SOURCE.
\end{verbatim}

If you don't know what argument to use as part of a command, type a
\texttt{?}\index{\texttt{]}@\texttt{?}\ in command lines}\ at the
argument position.  Metamath will tell you what it expected there.

\begin{verbatim}
MM> show ?
         ^
?Expected SETTINGS, LABELS, STATEMENT, SOURCE, PROOF,
MEMORY, TRACE_BACK, or USAGE.
\end{verbatim}

Finally, you may type just the first word or words of a command followed
by {\em return}.  Metamath will prompt you for the remaining part of the
command, showing you the choices at each step.  For example, instead of
typing \texttt{show statement th1 /full} you could interact in the
following manner:
\begin{verbatim}
MM> show
SETTINGS, LABELS, STATEMENT, SOURCE, PROOF,
MEMORY, TRACE_BACK, or USAGE <SETTINGS>? st
What is the statement label <th1>?
/ or nothing <nothing>? /
TEX, COMMENT_ONLY, or FULL <TEX>? f
/ or nothing <nothing>?
19 th1 $p |- t = t $= ... $.
\end{verbatim}
After each \texttt{?}\ in this mode, you must give Metamath the
information it requests.  Sometimes Metamath gives you a list of choices
with the default choice indicated by brackets \texttt{< > }. Pressing
{\em return} after the \texttt{?}\ will select the default choice.
Answering anything else will override the default.  Note that the
\texttt{/} in command qualifiers is considered a separate
token\index{token} by the parser, and this is why it is asked for
separately.

\section{Your First Proof}\label{frstprf}

Proofs are developed with the aid of the Proof Assistant\index{Proof
Assistant}.  We will now show you how the proof of theorem \texttt{th1}
was built.  So that you can repeat these steps, we will first have the
Proof Assistant erase the proof in Metamath's source buffer\index{source
buffer}, then reconstruct it.  (The source buffer is the place in memory
where Metamath stores the information in the database when it is
\texttt{read}\index{\texttt{read} command} in.  New or modified proofs
are kept in the source buffer until a \texttt{write source}
command\index{\texttt{write source} command} is issued.)  In practice, you
would place a \texttt{?}\index{\texttt{]}@\texttt{?}\ inside proofs}\
between \texttt{\$=}\index{\texttt{\$=} keyword} and
\texttt{\$.}\index{\texttt{\$.}\ keyword}\ in the database to indicate
to Metamath\index{Metamath} that the proof is unknown, and that would be
your starting point.  Whenever the \texttt{verify proof} command encounters
a proof with a \texttt{?}\ in place of a proof step, the statement is
identified as not proved.

When I first started creating Metamath proofs, I would write down
on a piece of paper the complete
formal proof\index{formal proof} as it would appear
in a \texttt{show proof} command\index{\texttt{show proof} command}; see
the display of \texttt{show proof th1 /lemmon /re\-num\-ber} above as an
example.  After you get used to using the Proof Assistant\index{Proof
Assistant} you may get to a point where you can ``see'' the proof in your mind
and let the Proof Assistant guide you in filling in the details, at least for
simpler proofs, but until you gain that experience you may find it very useful
to write down all the details in advance.
Otherwise you may waste a lot of time as you let it take you down a wrong path.
However, others do not find this approach as helpful.
For example, Thomas Brendan Leahy\index{Leahy, Thomas Brendan}
finds that it is more helpful to him to interactively
work backward from a machine-readable statement.
David A. Wheeler\index{Wheeler, David A.}
writes down a general approach, but develops the proof
interactively by switching between
working forwards (from hypotheses and facts likely to be useful) and
backwards (from the goal) until the forwards and backwards approaches meet.
In the end, use whatever approach works for you.

A proof is developed with the Proof Assistant by working backwards, starting
with the theorem\index{theorem} to be proved, and assigning each unknown step
with a theorem or hypothesis until no more unknown steps remain.  The Proof
Assistant will not let you make an assignment unless it can be ``unified''
with the unknown step.  This means that a
substitution\index{substitution!variable}\index{variable substitution} of
variables exists that will make the assignment match the unknown step.  On the
other hand, in the middle of a proof, when working backwards, often more than
one unification\index{unification} (set of substitutions) is possible, since
there is not enough information available at that point to uniquely establish
it.  In this case you can tell Metamath which unification to choose, or you
can continue to assign unknown steps until enough information is available to
make the unification unique.

We will assume you have entered Metamath and read in the database as described
above.  The following dialog shows how the proof was developed.  For more
details on what some of the commands do, refer to Section~\ref{pfcommands}.
\index{\texttt{prove} command}

\begin{verbatim}
MM> prove th1
Entering the Proof Assistant.  Type HELP for help, EXIT
to exit.  You will be working on the proof of statement th1:
  $p |- t = t
Note:  The proof you are starting with is already complete.
MM-PA>
\end{verbatim}

The \verb/MM-PA>/ prompt means we are inside the Proof
Assistant.\index{Proof Assistant} Most of the regular Metamath commands
(\texttt{show statement}, etc.) are still available if you need them.

\begin{verbatim}
MM-PA> delete all
The entire proof was deleted.
\end{verbatim}

We have deleted the whole proof so we can start from scratch.

\begin{verbatim}
MM-PA> show new_proof/lemmon/all
1 ?              $? |- t = t
\end{verbatim}

The \texttt{show new{\char`\_}proof} command\index{\texttt{show
new{\char`\_}proof} command} is like \texttt{show proof} except that we
don't specify a statement; instead, the proof we're working on is
displayed.

\begin{verbatim}
MM-PA> assign 1 mp
To undo the assignment, DELETE STEP 5 and INITIALIZE, UNIFY
if needed.
3   min=?  $? |- $2
4   maj=?  $? |- ( $2 -> t = t )
\end{verbatim}

The \texttt{assign} command\index{\texttt{assign} command} above means
``assign step 1 with the statement whose label is \texttt{mp}.''  Note
that step renumbering will constantly occur as you assign steps in the
middle of a proof; in general all steps from the step you assign until
the end of the proof will get moved up.  In this case, what used to be
step 1 is now step 5, because the (partial) proof now has five steps:
the four hypotheses of the \texttt{mp} statement and the \texttt{mp}
statement itself.  Let's look at all the steps in our partial proof:

\begin{verbatim}
MM-PA> show new_proof/lemmon/all
1 ?              $? wff $2
2 ?              $? wff t = t
3 ?              $? |- $2
4 ?              $? |- ( $2 -> t = t )
5 1,2,3,4 mp     $a |- t = t
\end{verbatim}

The symbol \texttt{\$2} is a temporary variable\index{temporary
variable} that represents a symbol sequence not yet known.  In the final
proof, all temporary variables will be eliminated.  The general format
for a temporary variable is \texttt{\$} followed by an integer.  Note
that \texttt{\$} is not a legal character in a math symbol (see
Section~\ref{dollardollar}, p.~\pageref{dollardollar}), so there will
never be a naming conflict between real symbols and temporary variables.

Unknown steps 1 and 2 are constructions of the two wffs used by the
modus ponens rule.  As you will see at the end of this section, the
Proof Assistant\index{Proof Assistant} can usually figure these steps
out by itself, and we will not have to worry about them.  Therefore from
here on we will display only the ``essential'' hypotheses, i.e.\ those
steps that correspond to traditional formal proofs\index{formal proof}.

\begin{verbatim}
MM-PA> show new_proof/lemmon
3 ?              $? |- $2
4 ?              $? |- ( $2 -> t = t )
5 3,4 mp         $a |- t = t
\end{verbatim}

Unknown steps 3 and 4 are the ones we must focus on.  They correspond to the
minor and major premises of the modus ponens rule.  We will assign them as
follows.  Notice that because of the step renumbering that takes place
after an assignment, it is advantageous to assign unknown steps in reverse
order, because earlier steps will not get renumbered.

\begin{verbatim}
MM-PA> assign 4 mp
To undo the assignment, DELETE STEP 8 and INITIALIZE, UNIFY
if needed.
3   min=?  $? |- $2
6     min=?  $? |- $4
7     maj=?  $? |- ( $4 -> ( $2 -> t = t ) )
\end{verbatim}

We are now going to describe an obscure feature that you will probably
never use but should be aware of.  The Metamath language allows empty
symbol sequences to be substituted for variables, but in most formal
systems this feature is never used.  One of the few examples where is it
used is the MIU-system\index{MIU-system} described in
Appendix~\ref{MIU}.  But such systems are rare, and by default this
feature is turned off in the Proof Assistant.  (It is always allowed for
{\tt verify proof}.)  Let us turn it on and see what
happens.\index{\texttt{set empty{\char`\_}substitution} command}

\begin{verbatim}
MM-PA> set empty_substitution on
Substitutions with empty symbol sequences is now allowed.
\end{verbatim}

With this feature enabled, more unifications will be
ambiguous\index{ambiguous unification}\index{unification!ambiguous} in
the middle of a proof, because
substitution\index{substitution!variable}\index{variable substitution}
of variables with empty symbol sequences will become an additional
possibility.  Let's see what happens when we make our next assignment.

\begin{verbatim}
MM-PA> assign 3 a2
There are 2 possible unifications.  Please select the correct
    one or Q if you want to UNIFY later.
Unify:  |- $6
 with:  |- ( $9 + 0 ) = $9
Unification #1 of 2 (weight = 7):
  Replace "$6" with "( + 0 ) ="
  Replace "$9" with ""
  Accept (A), reject (R), or quit (Q) <A>? r
\end{verbatim}

The first choice presented is the wrong one.  If we had selected it,
temporary variable \texttt{\$6} would have been assigned a truncated
wff, and temporary variable \texttt{\$9} would have been assigned an
empty sequence (which is not allowed in our system).  With this choice,
eventually we would reach a point where we would get stuck because
we would end up with steps impossible to prove.  (You may want to
try it.)  We typed \texttt{r} to reject the choice.

\begin{verbatim}
Unification #2 of 2 (weight = 21):
  Replace "$6" with "( $9 + 0 ) = $9"
  Accept (A), reject (R), or quit (Q) <A>? q
To undo the assignment, DELETE STEP 4 and INITIALIZE, UNIFY
if needed.
 7     min=?  $? |- $8
 8     maj=?  $? |- ( $8 -> ( $6 -> t = t ) )
\end{verbatim}

The second choice is correct, and normally we would type \texttt{a}
to accept it.  But instead we typed \texttt{q} to show what will happen:
it will leave the step with an unknown unification, which can be
seen as follows:

\begin{verbatim}
MM-PA> show new_proof/not_unified
 4   min    $a |- $6
        =a2  = |- ( $9 + 0 ) = $9
\end{verbatim}

Later we can unify this with the \texttt{unify}
\texttt{all/interactive} command.

The important point to remember is that occasionally you will be
presented with several unification choices while entering a proof, when
the program determines that there is not enough information yet to make
an unambiguous choice automatically (and this can happen even with
\texttt{set empty{\char`\_}substitution} turned off).  Usually it is
obvious by inspection which choice is correct, since incorrect ones will
tend to be meaningless fragments of wffs.  In addition, the correct
choice will usually be the first one presented, unlike our example
above.

Enough of this digression.  Let us go back to the default setting.

\begin{verbatim}
MM-PA> set empty_substitution off
The ability to substitute empty expressions for variables
has been turned off.  Note that this may make the Proof
Assistant too restrictive in some cases.
\end{verbatim}

If we delete the proof, start over, and get to the point where
we digressed above, there will no longer be an ambiguous unification.

\begin{verbatim}
MM-PA> assign 3 a2
To undo the assignment, DELETE STEP 4 and INITIALIZE, UNIFY
if needed.
 7     min=?  $? |- $4
 8     maj=?  $? |- ( $4 -> ( ( $5 + 0 ) = $5 -> t = t ) )
\end{verbatim}

Let us look at our proof so far, and continue.

\begin{verbatim}
MM-PA> show new_proof/lemmon
 4 a2            $a |- ( $5 + 0 ) = $5
 7 ?             $? |- $4
 8 ?             $? |- ( $4 -> ( ( $5 + 0 ) = $5 -> t = t ) )
 9 7,8 mp        $a |- ( ( $5 + 0 ) = $5 -> t = t )
10 4,9 mp        $a |- t = t
MM-PA> assign 8 a1
To undo the assignment, DELETE STEP 11 and INITIALIZE, UNIFY
if needed.
 7     min=?  $? |- ( t + 0 ) = t
MM-PA> assign 7 a2
To undo the assignment, DELETE STEP 8 and INITIALIZE, UNIFY
if needed.
MM-PA> show new_proof/lemmon
 4 a2            $a |- ( t + 0 ) = t
 8 a2            $a |- ( t + 0 ) = t
12 a1            $a |- ( ( t + 0 ) = t -> ( ( t + 0 ) = t ->
                                                    t = t ) )
13 8,12 mp       $a |- ( ( t + 0 ) = t -> t = t )
14 4,13 mp       $a |- t = t
\end{verbatim}

Now all temporary variables and unknown steps have been eliminated from the
``essential'' part of the proof.  When this is achieved, the Proof
Assistant\index{Proof Assistant} can usually figure out the rest of the proof
automatically.  (Note that the \texttt{improve} command can occasionally be
useful for filling in essential steps as well, but it only tries to make use
of statements that introduce no new variables in their hypotheses, which is
not the case for \texttt{mp}. Also it will not try to improve steps containing
temporary variables.)  Let's look at the complete proof, then run
the \texttt{improve} command, then look at it again.

\begin{verbatim}
MM-PA> show new_proof/lemmon/all
 1 ?             $? wff ( t + 0 ) = t
 2 ?             $? wff t = t
 3 ?             $? term t
 4 3 a2          $a |- ( t + 0 ) = t
 5 ?             $? wff ( t + 0 ) = t
 6 ?             $? wff ( ( t + 0 ) = t -> t = t )
 7 ?             $? term t
 8 7 a2          $a |- ( t + 0 ) = t
 9 ?             $? term ( t + 0 )
10 ?             $? term t
11 ?             $? term t
12 9,10,11 a1    $a |- ( ( t + 0 ) = t -> ( ( t + 0 ) = t ->
                                                    t = t ) )
13 5,6,8,12 mp   $a |- ( ( t + 0 ) = t -> t = t )
14 1,2,4,13 mp   $a |- t = t
\end{verbatim}

\begin{verbatim}
MM-PA> improve all
A proof of length 1 was found for step 11.
A proof of length 1 was found for step 10.
A proof of length 3 was found for step 9.
A proof of length 1 was found for step 7.
A proof of length 9 was found for step 6.
A proof of length 5 was found for step 5.
A proof of length 1 was found for step 3.
A proof of length 3 was found for step 2.
A proof of length 5 was found for step 1.
Steps 1 and above have been renumbered.
CONGRATULATIONS!  The proof is complete.  Use SAVE
NEW_PROOF to save it.  Note:  The Proof Assistant does
not detect $d violations.  After saving the proof, you
should verify it with VERIFY PROOF.
\end{verbatim}

The \texttt{save new{\char`\_}proof} command\index{\texttt{save
new{\char`\_}proof} command} will save the proof in the database.  Here
we will just display it in a form that can be clipped out of a log file
and inserted manually into the database source file with a text
editor.\index{normal proof}\index{proof!normal}

\begin{verbatim}
MM-PA> show new_proof/normal
---------Clip out the proof below this line:
      tt tze tpl tt weq tt tt weq tt a2 tt tze tpl tt weq
      tt tze tpl tt weq tt tt weq wim tt a2 tt tze tpl tt
      tt a1 mp mp $.
---------The proof of 'th1' to clip out ends above this line.
\end{verbatim}

There is another proof format called ``compressed''\index{compressed
proof}\index{proof!compressed} that you will see in databases.  It is
not important to understand how it is encoded but only to recognize it
when you see it.  Its only purpose is to reduce storage requirements for
large proofs.  A compressed proof can always be converted to a normal
one and vice-versa, and the Metamath \texttt{show proof}
commands\index{\texttt{show proof} command} work equally well with
compressed proofs.  The compressed proof format is described in
Appendix~\ref{compressed}.

\begin{verbatim}
MM-PA> show new_proof/compressed
---------Clip out the proof below this line:
      ( tze tpl weq a2 wim a1 mp ) ABCZADZAADZAEZJJKFLIA
      AGHH $.
---------The proof of 'th1' to clip out ends above this line.
\end{verbatim}

Now we will exit the Proof Assistant.  Since we made changes to the proof,
it will warn us that we have not saved it.  In this case, we don't care.

\begin{verbatim}
MM-PA> exit
Warning:  You have not saved changes to the proof.
Do you want to EXIT anyway (Y, N) <N>? y
Exiting the Proof Assistant.
Type EXIT again to exit Metamath.
\end{verbatim}

The Proof Assistant\index{Proof Assistant} has several other commands
that can help you while creating proofs.  See Section~\ref{pfcommands}
for a list of them.

A command that is often useful is \texttt{minimize{\char`\_}with
*/brief}, which tries to shorten the proof.  It can make the process
more efficient by letting you write a somewhat ``sloppy'' proof then
clean up some of the fine details of optimization for you (although it
can't perform miracles such as restructuring the overall proof).

\section{A Note About Editing a Data\-base File}

Once your source file contains proofs, there are some restrictions on
how you can edit it so that the proofs remain valid.  Pay particular
attention to these rules, since otherwise you can lose a lot of work.
It is a good idea to periodically verify all proofs with \texttt{verify
proof *} to ensure their integrity.

If your file contains only normal (as opposed to compressed) proofs, the
main rule is that you may not change the order of the mandatory
hypotheses\index{mandatory hypothesis} of any statement referenced in a
later proof.  For example, if you swap the order of the major and minor
premise in the modus ponens rule, all proofs making use of that rule
will become incorrect.  The \texttt{show statement}
command\index{\texttt{show statement} command} will show you the
mandatory hypotheses of a statement and their order.

If a statement has a compressed proof, you also must not change the
order of {\em its} mandatory hypotheses.  The compressed proof format
makes use of this information as part of the compression technique.
Note that swapping the names of two variables in a theorem will change
the order of its mandatory hypotheses.

The safest way to edit a statement, say \texttt{mytheorem}, is to
duplicate it then rename the original to \texttt{mytheoremOLD}
throughout the database.  Once the edited version is re-proved, all
statements referencing \texttt{mytheoremOLD} can be updated in the Proof
Assistant using \texttt{minimize{\char`\_}with
mytheorem
/allow{\char`\_}growth}.\index{\texttt{minimize{\char`\_}with} command}
% 3/10/07 Note: line-breaking the above results in duplicate index entries

\chapter{Abstract Mathematics Revealed}\label{fol}

\section{Logic and Set Theory}\label{logicandsettheory}

\begin{quote}
  {\em Set theory can be viewed as a form of exact theology.}
  \flushright\sc  Rudy Rucker\footnote{\cite{Barrow}, p.~31.}\\
\end{quote}\index{Rucker, Rudy}

Despite its seeming complexity, all of standard mathematics, no matter how
deep or abstract, can amazingly enough be derived from a relatively small set
of axioms\index{axiom} or first principles. The development of these axioms is
among the most impressive and important accomplishments of mathematics in the
20th century. Ultimately, these axioms can be broken down into a set of rules
for manipulating symbols that any technically oriented person can follow.

We will not spend much time trying to convey a deep, higher-level
understanding of the meaning of the axioms. This kind of understanding
requires some mathematical sophistication as well as an understanding of the
philosophy underlying the foundations of mathematics and typically develops
over time as you work with mathematics.  Our goal, instead, is to give you the
immediate ability to follow how theorems\index{theorem} are derived from the
axioms and from other theorems.  This will be similar to learning the syntax
of a computer language, which lets you follow the details in a program but
does not necessarily give you the ability to write non-trivial programs on
your own, an ability that comes with practice. For now don't be alarmed by
abstract-sounding names of the axioms; just focus on the rules for
manipulating the symbols, which follow the simple conventions of the
Metamath\index{Metamath} language.

The axioms that underlie all of standard mathematics consist of axioms of logic
and axioms of set theory. The axioms of logic are divided into two
subcategories, propositional calculus\index{propositional calculus} (sometimes
called sentential logic\index{sentential logic}) and predicate calculus
(sometimes called first-order logic\index{first-order logic}\index{quantifier
theory}\index{predicate calculus} or quantifier theory).  Propositional
calculus is a prerequisite for predicate calculus, and predicate calculus is a
prerequisite for set theory.  The version of set theory most commonly used is
Zermelo--Fraenkel set theory\index{Zermelo--Fraenkel set theory}\index{set theory}
with the axiom of choice,
often abbreviated as ZFC\index{ZFC}.

Here in a nutshell is what the axioms are all about in an informal way. The
connection between this description and symbols we will show you won't be
immediately apparent and in principle needn't ever be.  Our description just
tries to summarize what mathematicians think about when they work with the
axioms.

Logic is a set of rules that allow us determine truths given other truths.
Put another way,
logic is more or less the translation of what we would consider common sense
into a rigorous set of axioms.\index{axioms of logic}  Suppose $\varphi$,
$\psi$, and $\chi$ (the Greek letters phi, psi, and chi) represent statements
that are either true or false, and $x$ is a variable\index{variable!in predicate
calculus} ranging over some group of mathematical objects (sets, integers,
real numbers, etc.). In mathematics, a ``statement'' really means a formula,
and $\psi$ could be for example ``$x = 2$.''
Propositional calculus\index{propositional calculus}
allows us to use variables that are either true or false
and make deductions such as
``if $\varphi$ implies $\psi$ and $\psi$ implies $\chi$, then $\varphi$
implies $\chi$.''
Predicate calculus\index{predicate calculus}
extends propositional calculus by also allowing us
to discuss statements about objects (not just true and false values), including
statements about ``all'' or ``at least one'' object.
For example, predicate calculus allows to say,
``if $\varphi$ is true for all $x$, then $\varphi$ is true for some $x$.''
The logic used in \texttt{set.mm} is standard classical logic
(as opposed to other logic systems like intuitionistic logic).

Set theory\index{set theory} has to do with the manipulation of objects and
collections of objects, specifically the abstract, imaginary objects that
mathematics deals with, such as numbers. Everything that is claimed to exist
in mathematics is considered to be a set.  A set called the empty
set\index{empty set} contains nothing.  We represent the empty set by
$\varnothing$.  Many sets can be built up from the empty set.  There is a set
represented by $\{\varnothing\}$ that contains the empty set, another set
represented by $\{\varnothing,\{\varnothing\}\}$ that contains this set as
well as the empty set, another set represented by $\{\{\varnothing\}\}$ that
contains just the set that contains the empty set, and so on ad infinitum. All
mathematical objects, no matter how complex, are defined as being identical to
certain sets: the integer\index{integer} 0 is defined as the empty set, the
integer 1 is defined as $\{\varnothing\}$, the integer 2 is defined as
$\{\varnothing,\{\varnothing\}\}$.  (How these definitions were chosen doesn't
matter now, but the idea behind it is that these sets have the properties we
expect of integers once suitable operations are defined.)  Mathematical
operations, such as addition, are defined in terms of operations on
sets---their union\index{set union}, intersection\index{set intersection}, and
so on---operations you may have used in elementary school when you worked
with groups of apples and oranges.

With a leap of faith, the axioms also postulate the existence of infinite
sets\index{infinite set}, such as the set of all non-negative integers ($0, 1,
2,\ldots$, also called ``natural numbers''\index{natural number}).  This set
can't be represented with the brace notation\index{brace notation} we just
showed you, but requires a more complicated notation called ``class
abstraction.''\index{class abstraction}\index{abstraction class}  For
example, the infinite set $\{ x |
\mbox{``$x$ is a natural number''} \} $ means the ``set of all objects $x$
such that $x$ is a natural number'' i.e.\ the set of natural numbers; here,
``$x$ is a natural number'' is a rather complicated formula when broken down
into the primitive symbols.\label{expandom}\footnote{The statement ``$x$ is a
natural number'' is formally expressed as ``$x \in \omega$,'' where $\in$
(stylized epsilon) means ``is in'' or ``is an element of'' and $\omega$
(omega) means ``the set of natural numbers.''  When ``$x\in\omega$'' is
completely expanded in terms of the primitive symbols of set theory, the
result is  $\lnot$ $($ $\lnot$ $($ $\forall$ $z$ $($ $\lnot$ $\forall$ $w$ $($
$z$ $\in$ $w$ $\rightarrow$ $\lnot$ $w$ $\in$ $x$ $)$ $\rightarrow$ $z$ $\in$
$x$ $)$ $\rightarrow$ $($ $\forall$ $z$ $($ $\lnot$ $($ $\forall$ $w$ $($ $w$
$\in$ $z$ $\rightarrow$ $w$ $\in$ $x$ $)$ $\rightarrow$ $\forall$ $w$ $\lnot$
$w$ $\in$ $z$ $)$ $\rightarrow$ $\lnot$ $\forall$ $w$ $($ $w$ $\in$ $z$
$\rightarrow$ $\lnot$ $\forall$ $v$ $($ $v$ $\in$ $z$ $\rightarrow$ $\lnot$
$v$ $\in$ $w$ $)$ $)$ $)$ $\rightarrow$ $\lnot$ $\forall$ $z$ $\forall$ $w$
$($ $\lnot$ $($ $z$ $\in$ $x$ $\rightarrow$ $\lnot$ $w$ $\in$ $x$ $)$
$\rightarrow$ $($ $\lnot$ $z$ $\in$ $w$ $\rightarrow$ $($ $\lnot$ $z$ $=$ $w$
$\rightarrow$ $w$ $\in$ $z$ $)$ $)$ $)$ $)$ $)$ $\rightarrow$ $\lnot$
$\forall$ $y$ $($ $\lnot$ $($ $\lnot$ $($ $\forall$ $z$ $($ $\lnot$ $\forall$
$w$ $($ $z$ $\in$ $w$ $\rightarrow$ $\lnot$ $w$ $\in$ $y$ $)$ $\rightarrow$
$z$ $\in$ $y$ $)$ $\rightarrow$ $($ $\forall$ $z$ $($ $\lnot$ $($ $\forall$
$w$ $($ $w$ $\in$ $z$ $\rightarrow$ $w$ $\in$ $y$ $)$ $\rightarrow$ $\forall$
$w$ $\lnot$ $w$ $\in$ $z$ $)$ $\rightarrow$ $\lnot$ $\forall$ $w$ $($ $w$
$\in$ $z$ $\rightarrow$ $\lnot$ $\forall$ $v$ $($ $v$ $\in$ $z$ $\rightarrow$
$\lnot$ $v$ $\in$ $w$ $)$ $)$ $)$ $\rightarrow$ $\lnot$ $\forall$ $z$
$\forall$ $w$ $($ $\lnot$ $($ $z$ $\in$ $y$ $\rightarrow$ $\lnot$ $w$ $\in$
$y$ $)$ $\rightarrow$ $($ $\lnot$ $z$ $\in$ $w$ $\rightarrow$ $($ $\lnot$ $z$
$=$ $w$ $\rightarrow$ $w$ $\in$ $z$ $)$ $)$ $)$ $)$ $\rightarrow$ $($
$\forall$ $z$ $\lnot$ $z$ $\in$ $y$ $\rightarrow$ $\lnot$ $\forall$ $w$ $($
$\lnot$ $($ $w$ $\in$ $y$ $\rightarrow$ $\lnot$ $\forall$ $z$ $($ $w$ $\in$
$z$ $\rightarrow$ $\lnot$ $z$ $\in$ $y$ $)$ $)$ $\rightarrow$ $\lnot$ $($
$\lnot$ $\forall$ $z$ $($ $w$ $\in$ $z$ $\rightarrow$ $\lnot$ $z$ $\in$ $y$
$)$ $\rightarrow$ $w$ $\in$ $y$ $)$ $)$ $)$ $)$ $\rightarrow$ $x$ $\in$ $y$
$)$ $)$ $)$. Section~\ref{hierarchy} shows the hierarchy of definitions that
leads up to this expression.}\index{stylized epsilon ($\in$)}\index{omega
($\omega$)}  Actually, the primitive symbols don't even include the brace
notation.  The brace notation is a high-level definition, which you can find in
Section~\ref{hierarchy}.

Interestingly, the arithmetic of integers\index{integer} and
rationals\index{rational number} can be developed without appealing to the
existence of an infinite set, whereas the arithmetic of real
numbers\index{real number} requires it.

Each variable\index{variable!in set theory} in the axioms of set theory
represents an arbitrary set, and the axioms specify the legal kinds of things
you can do with these variables at a very primitive level.

Now, you may think that numbers and arithmetic are a lot more intuitive and
fundamental than sets and therefore should be the foundation of mathematics.
What is really the case is that you've dealt with numbers all your life and
are comfortable with a few rules for manipulating them such as addition and
multiplication.  Those rules only cover a small portion of what can be done
with numbers and only a very tiny fraction of the rest of mathematics.  If you
look at any elementary book on number theory, you will quickly become lost if
these are the only rules that you know.  Even though such books may present a
list of ``axioms''\index{axiom} for arithmetic, the ability to use the axioms
and to understand proofs of theorems\index{theorem} (facts) about numbers
requires an implicit mathematical talent that frustrates many people
from studying abstract mathematics.  The kind of mathematics that most people
know limits them to the practical, everyday usage of blindly manipulating
numbers and formulas, without any understanding of why those rules are correct
nor any ability to go any further.  For example, do you know why multiplying
two negative numbers yields a positive number?  Starting with set theory, you
will also start off blindly manipulating symbols according to the rules we give
you, but with the advantage that these rules will allow you, in principle, to
access {\em all} of mathematics, not just a tiny part of it.

Of course, concrete examples are often helpful in the learning process. For
example, you can verify that $2\cdot 3=3 \cdot 2$ by actually grouping
objects and can easily ``see'' how it generalizes to $x\cdot y = y\cdot x$,
even though you might not be able to rigorously prove it.  Similarly, in set
theory it can be helpful to understand how the axioms of set theory apply to
(and are correct for) small finite collections of objects.  You should be aware
that in set theory intuition can be misleading for infinite collections, and
rigorous proofs become more important.  For example, while $x\cdot y = y\cdot
x$ is correct for finite ordinals (which are the natural numbers), it is not
usually true for infinite ordinals.

\section{The Axioms for All of Mathematics}

In this section\index{axioms for mathematics}, we will show you the axioms
for all of standard mathematics (i.e.\ logic and set theory) as they are
traditionally presented.  The traditional presentation is useful for someone
with the mathematical experience needed to correctly manipulate high-level
abstract concepts.  For someone without this talent, knowing how to actually
make use of these axioms can be difficult.  The purpose of this section is to
allow you to see how the version of the axioms used in the standard
Metamath\index{Metamath} database \texttt{set.mm}\index{set
theory database (\texttt{set.mm})} relates to  the typical version
in textbooks, and also to give you an informal feel for them.

\subsection{Propositional Calculus}

Propositional calculus\index{propositional calculus} concerns itself with
statements that can be interpreted as either true or false.  Some examples of
statements (outside of mathematics) that are either true or false are ``It is
raining today'' and ``The United States has a female president.'' In
mathematics, as we mentioned, statements are really formulas.

In propositional calculus, we don't care what the statements are.  We also
treat a logical combination of statements, such as ``It is raining today and
the United States has a female president,'' no differently from a single
statement.  Statements and their combinations are called well-formed formulas
(wffs)\index{well-formed formula (wff)}.  We define wffs only in terms of
other wffs and don't define what a ``starting'' wff is.  As is common practice
in the literature, we use Greek letters to represent wffs.

Specifically, suppose $\varphi$ and $\psi$ are wffs.  Then the combinations
$\varphi\rightarrow\psi$ (``$\varphi$ implies $\psi$,'' also read ``if
$\varphi$ then $\psi$'')\index{implication ($\rightarrow$)} and $\lnot\varphi$
(``not $\varphi$'')\index{negation ($\lnot$)} are also wffs.

The three axioms of propositional calculus\index{axioms of propositional
calculus} are all wffs of the following form:\footnote{A remarkable result of
C.~A.~Meredith\index{Meredith, C. A.} squeezes these three axioms into the
single axiom $((((\varphi\rightarrow \psi)\rightarrow(\neg \chi\rightarrow\neg
\theta))\rightarrow \chi )\rightarrow \tau)\rightarrow((\tau\rightarrow
\varphi)\rightarrow(\theta\rightarrow \varphi))$ \cite{CAMeredith},
which is believed to be the shortest possible.}
\begin{center}
     $\varphi\rightarrow(\psi\rightarrow \varphi)$\\

     $(\varphi\rightarrow (\psi\rightarrow \chi))\rightarrow
((\varphi\rightarrow  \psi)\rightarrow (\varphi\rightarrow \chi))$\\

     $(\neg \varphi\rightarrow \neg\psi)\rightarrow (\psi\rightarrow
\varphi)$
\end{center}

These three axioms are widely used.
They are attributed to Jan {\L}ukasiewicz
(pronounced woo-kah-SHAY-vitch) and was popularized by Alonzo Church,
who called it system P2. (Thanks to Ted Ulrich for this information.)

There are an infinite number of axioms, one for each possible
wff\index{well-formed formula (wff)} of the above form.  (For this reason,
axioms such as the above are often called ``axiom schemes.''\index{axiom
scheme})  Each Greek letter in the axioms may be substituted with a more
complex wff to result in another axiom.  For example, substituting
$\neg(\varphi\rightarrow\chi)$ for $\varphi$ in the first axiom yields
$\neg(\varphi\rightarrow\chi)\rightarrow(\psi\rightarrow
\neg(\varphi\rightarrow\chi))$, which is still an axiom.

To deduce new true statements (theorems\index{theorem}) from the axioms, a
rule\index{rule} called ``modus ponens''\index{modus ponens} is used.  This
rule states that if the wff $\varphi$ is an axiom or a theorem, and the wff
$\varphi\rightarrow\psi$ is an axiom or a theorem, then the wff $\psi$ is also
a theorem\index{theorem}.

As a non-mathematical example of modus ponens, suppose we have proved (or
taken as an axiom) ``Bob is a man'' and separately have proved (or taken as
an axiom) ``If Bob is a man, then Bob is a human.''  Using the rule of modus
ponens, we can logically deduce, ``Bob is a human.''

From Metamath's\index{Metamath} point of view, the axioms and the rule of
modus ponens just define a mechanical means for deducing new true statements
from existing true statements, and that is the complete content of
propositional calculus as far as Metamath is concerned.  You can read a logic
textbook to gain a better understanding of their meaning, or you can just let
their meaning slowly become apparent to you after you use them for a while.

It is actually rather easy to check to see if a formula is a theorem of
propositional calculus.  Theorems of propositional calculus are also called
``tautologies.''\index{tautology}  The technique to check whether a formula is
a tautology is called the ``truth table method,''\index{truth table} and it
works like this.  A wff $\varphi\rightarrow\psi$ is false whenever $\varphi$ is true
and $\psi$ is false.  Otherwise it is true.  A wff $\lnot\varphi$ is false
whenever $\varphi$ is true and false otherwise. To verify a tautology such as
$\varphi\rightarrow(\psi\rightarrow \varphi)$, you break it down into sub-wffs and
construct a truth table that accounts for all possible combinations of true
and false assigned to the wff metavariables:
\begin{center}\begin{tabular}{|c|c|c|c|}\hline
\mbox{$\varphi$} & \mbox{$\psi$} & \mbox{$\psi\rightarrow\varphi$}
    & \mbox{$\varphi\rightarrow(\psi\rightarrow \varphi)$} \\ \hline \hline
              T   &  T    &      T       &        T    \\ \hline
              T   &  F    &      T       &        T    \\ \hline
              F   &  T    &      F       &        T    \\ \hline
              F   &  F    &      T       &        T    \\ \hline
\end{tabular}\end{center}
If all entries in the last column are true, the formula is a tautology.

Now, the truth table method doesn't tell you how to prove the tautology from
the axioms, but only that a proof exists.  Finding an actual proof (especially
one that is short and elegant) can be challenging.  Methods do exist for
automatically generating proofs in propositional calculus, but the proofs that
result can sometimes be very long.  In the Metamath \texttt{set.mm}\index{set
theory database (\texttt{set.mm})} database, most
or all proofs were created manually.

Section \ref{metadefprop} discusses various definitions
that make propositional calculus easier to use.
For example, we define:

\begin{itemize}
\item $\varphi \vee \psi$
  is true if either $\varphi$ or $\psi$ (or both) are true
  (this is disjunction\index{disjunction ($\vee$)}
  aka logical {\sc or}\index{logical {\sc or} ($\vee$)}).

\item $\varphi \wedge \psi$
  is true if both $\varphi$ and $\psi$ are true
  (this is conjunction\index{conjunction ($\wedge$)}
  aka logical {\sc and}\index{logical {\sc and} ($\wedge$)}).

\item $\varphi \leftrightarrow \psi$
  is true if $\varphi$ and $\psi$ have the same value, that is,
  they are both true or both false
  (this is the biconditional\index{biconditional ($\leftrightarrow$)}).
\end{itemize}

\subsection{Predicate Calculus}

Predicate calculus\index{predicate calculus} introduces the concept of
``individual variables,''\index{variable!in predicate calculus}\index{individual
variable} which
we will usually just call ``variables.''
These variables can represent something other than true or false (wffs),
and will always represent sets when we get to set theory.  There are also
three new symbols $\forall$\index{universal quantifier ($\forall$)},
$=$\index{equality ($=$)}, and $\in$\index{stylized epsilon ($\in$)},
read ``for all,'' ``equals,'' and ``is an element of''
respectively.  We will represent variables with the letters $x$, $y$, $z$, and
$w$, as is common practice in the literature.
For example, $\forall x \varphi$ means ``for all possible values of
$x$, $\varphi$ is true.''

In predicate calculus, we extend the definition of a wff\index{well-formed
formula (wff)}.  If $\varphi$ is a wff and $x$ and $y$ are variables, then
$\forall x \, \varphi$, $x=y$, and $x\in y$ are wffs. Note that these three new
types of wffs can be considered ``starting'' wffs from which we can build
other wffs with $\rightarrow$ and $\neg$ .  The concept of a starting wff was
absent in propositional calculus.  But starting wff or not, all we are really
concerned with is whether our wffs are correctly constructed according to
these mechanical rules.

A quick aside:
To prevent confusion, it might be best at this point to think of the variables
of Metamath\index{Metamath} as ``metavariables,''\index{metavariable} because
they are not quite the same as the variables we are introducing here.  A
(meta)variable in Metamath can be a wff or an individual variable, as well
as many other things; in general, it represents a kind of place holder for an
unspecified sequence of math symbols\index{math symbol}.

Unlike propositional calculus, no decision procedure\index{decision procedure}
analogous to the truth table method exists (nor theoretically can exist) that
will definitely determine whether a formula is a theorem of predicate
calculus.  Much of the work in the field of automated theorem
proving\index{automated theorem proving} has been dedicated to coming up with
clever heuristics for proving theorems of predicate calculus, but they can
never be guaranteed to work always.

Section \ref{metadefpred} discusses various definitions
that make predicate calculus easier to use.
For example, we define
$\exists x \varphi$ to mean
``there exists at least one possible value of $x$ where $\varphi$ is true.''

We now turn to looking at how predicate calculus can be formally
represented.

\subsubsection{Common Axioms}

There is a new rule of inference in predicate calculus:  if $\varphi$ is
an axiom or a theorem, then $\forall x \,\varphi$ is also a
theorem\index{theorem}.  This is called the rule of
``generalization.''\index{rule of generalization}
This is easily represented in Metamath.

In standard texts of logic, there are often two axioms of predicate
calculus\index{axioms of predicate calculus}:
\begin{center}
  $\forall x \,\varphi ( x ) \rightarrow \varphi ( y )$,
      where ``$y$ is properly substituted for $x$.''\\
  $\forall x ( \varphi \rightarrow \psi )\rightarrow ( \varphi \rightarrow
    \forall x\, \psi )$,
    where ``$x$ is not free in $\varphi$.''
\end{center}

Now at first glance, this seems simple:  just two axioms.  However,
conditional clauses are attached to each axiom describing requirements that
may seem puzzling to you.  In addition, the first axiom puts a variable symbol
in parentheses after each wff, seemingly violating our definition of a
wff\index{well-formed formula (wff)}; this is just an informal way of
referring to some arbitrary variable that may occur in the wff.  The
conditional clauses do, of course, have a precise meaning, but as it turns out
the precise meaning is somewhat complicated and awkward to formalize in a
way that a computer can handle easily.  Unlike propositional calculus, a
certain amount of mathematical sophistication and practice is needed to be
able to easily grasp and manipulate these concepts correctly.

Predicate calculus may be presented with or without axioms for
equality\index{axioms of equality}\index{equality ($=$)}. We will require the
axioms of equality as a prerequisite for the version of set theory we will
use.  The axioms for equality, when included, are often represented using these
two axioms:
\begin{center}
$x=x$\\ \ \\
$x=y\rightarrow (\varphi(x,x)\rightarrow\varphi(x,y))$ where ``$\varphi(x,y)$
   arises from $\varphi(x,x)$ by replacing some, but not necessarily all,
   free\index{free variable}
   occurrences of $x$ by $y$,\\ provided that $y$ is free for $x$
   in $\varphi(x,x)$.'' \end{center}
% (Mendelson p. 95)
The first equality axiom is simple, but again,
the condition on the second one is
somewhat awkward to implement on a computer.

\subsubsection{Tarski System S2}

Of course, we are not the first to notice the complications of these
predicate calculus axioms when being rigorous.

Well-known logician Alfred Tarski published in 1965
a system he called system S2\cite[p.~77]{Tarski1965}.
Tarski's system is \textit{exactly equivalent} to the traditional textbook
formalization, but (by clever use of equality axioms) it eliminates the
latter's primitive notions of ``proper substitution'' and ``free variable,''
replacing them with direct substitution and the notion of a variable
not occurring in a formula (which we express with distinct variable
constraints).

In advocating his system, Tarski wrote, ``The relatively complicated
character of [free variables and proper substitution] is a source
of certain inconveniences of both practical and theoretical nature;
this is clearly experienced both in teaching an elementary course of
mathematical logic and in formalizing the syntax of predicate logic for
some theoretical purposes''\cite[p.~61]{Tarski1965}\index{Tarski, Alfred}.

\subsubsection{Developing a Metamath Representation}

The standard textbook axioms of predicate calculus are somewhat
cumbersome to implement on a computer because of the complex notions of
``free variable''\index{free variable} and ``proper
substitution.''\index{proper substitution}\index{substitution!proper}
While it is possible to use the Metamath\index{Metamath} language to
implement these concepts, we have chosen not to implement them
as primitive constructs in the
\texttt{set.mm} set theory database.  Instead, we have eliminated them
within the axioms
by carefully crafting the axioms so as to avoid them,
building on Tarski's system S2.  This makes it
easy for a beginner to follow the steps in a proof without knowing any
advanced concepts other than the simple concept of
replacing\index{substitution!variable}\index{variable substitution}
variables with expressions.

In order to develop the concepts of free variable and proper
substitution from the axioms, we use an additional
Metamath statement type called ``disjoint variable
restriction''\index{disjoint variables} that we have not encountered
before.  In the context of the axioms, the statement \texttt{\$d} $ x\,
y$\index{\texttt{\$d} statement} simply means that $x$ and $y$ must be
distinct\index{distinct variables}, i.e.\ they may not be simultaneously
substituted\index{substitution!variable}\index{variable substitution}
with the same variable.  The statement \texttt{\$d} $ x\, \varphi$ means
variable $x$ must not occur in wff $\varphi$.  For the precise
definition of \texttt{\$d}, see Section~\ref{dollard}.

\subsubsection{Metamath representation}

The Metamath axiom system for predicate calculus
defined in set.mm uses Tarski's system S2.
As noted above, this has a different representation
than the traditional textbook formalization,
but it is \textit{exactly equivalent} to the textbook formalization,
and it is \textit{much} easier to work with.
This is reproduced as system S3 in Section 6 of
Megill's formalization \cite{Megill}\index{Megill, Norman}.

There is one exception, Tarski's axiom of existence,
which we label as axiom ax-6.
In the case of ax-6, Tarski's version is weaker because it includes a
distinct variable proviso. If we wish, we can also weaken our version
in this way and still have a metalogically complete system. Theorem
ax6 shows this by deriving, in the presence of the other axioms, our
ax-6 from Tarski's weaker version ax6v. However, we chose the stronger
version for our system because it is simpler to state and easier to use.

Tarski's system was designed for proving specific theorems rather than
more general theorem schemes. However, theorem schemes are much more
efficient than specific theorems for building a body of mathematical
knowledge, since they can be reused with different instances as
needed. While Tarski does derive some theorem schemes from his axioms,
their proofs require concepts that are ``outside'' of the system, such as
induction on formula length. The verification of such proofs is difficult
to automate in a proof verifier. (Specifically, Tarski treats the formulas
of his system as set-theoretical objects. In order to verify the proofs
of his theorem schemes, a proof verifier would need a significant amount
of set theory built into it.)

The Metamath axiom system for predicate calculus extends
Tarski's system to eliminate this difficulty. The additional
``auxilliary'' axiom
schemes (as we will call them in this section; see below) endow Tarski's
system with a nice property we call
metalogical completeness \cite[Remark 9.6]{Megill}\index{Megill, Norman}.
As a result, we can prove any theorem scheme
expressable in the ``simple metalogic'' of Tarski's system by using
only Metamath's direct substitution rule applied to the axiom system
(and no other metalogical or set-theoretical notions ``outside'' of the
system). Simple metalogic consists of schemes containing wff metavariables
(with no arguments) and/or set (also called ``individual'') metavariables,
accompanied by optional provisos each stating that two specified set
metavariables must be distinct or that a specified set metavariable may
not occur in a specified wff metavariable. Metamath's logic and set theory
axiom and rule schemes are all examples of simple metalogic. The schemes
of traditional predicate calculus with equality are examples which are
not simple metalogic, because they use wff metavariables with arguments
and have ``free for'' and ``not free in'' side conditions.

A rigorous justification for this system, using an older but
exactly equivalent set of axioms, can be
found in \cite{Megill}\index{Megill, Norman}.

This allows us to
take a different approach in the Metamath\index{Metamath} database
\texttt{set.mm}\index{set theory database (\texttt{set.mm})}.  We do not
directly use the primitive notions of ``free variable''\index{free variable}
and ``proper substitution''\index{proper
substitution}\index{substitution!proper} at all as primitive constructs.
Instead, we use a set
of axioms that are almost as simple to manipulate as those of
propositional calculus.  Our axiom system avoids complex primitive
notions by effectively embedding the complexity into the axioms
themselves.  As a result, we will end up with a larger number of axioms,
but they are ideally suited for a computer language such as Metamath.
(Section~\ref{metaaxioms} shows these axioms.)

We will not elaborate further
on the ``free variable'' and ``proper substitution''
concepts here.  You may consult
\cite[ch.\ 3--4]{Hamilton}\index{Hamilton, Alan G.} (as well as
many other books) for a precise explanation
of these concepts.  If you intend to do serious mathematical work, it is wise
to become familiar with the traditional textbook approach; even though the
concepts embedded in their axioms require a higher level of sophistication,
they can be more practical to deal with on an everyday, informal basis.  Even
if you are just developing Metamath proofs, familiarity with the traditional
approach can help you arrive at a proof outline much faster, which you can
then convert to the detail required by Metamath.

We do develop proper substitution rules later on, but in set.mm
they are defined as derived constructs; they are not primitives.

You should also note that our system of predicate calculus is specifically
tailored for set theory; thus there are only two specific predicates $=$ and
$\in$ and no functions\index{function!in predicate calculus}
or constants\index{constant!in predicate calculus} unlike more general systems.
We later add these.

\subsection{Set Theory}

Traditional Zermelo--Fraenkel set theory\index{Zermelo--Fraenkel set
theory}\index{set theory} with the Axiom of Choice
has 10 axioms, which can be expressed in the
language of predicate calculus.  In this section, we will list only the
names and brief English descriptions of these axioms, since we will give
you the precise formulas used by the Metamath\index{Metamath} set theory
database \texttt{set.mm} later on.

In the descriptions of the axioms, we assume that $x$, $y$, $z$, $w$, and $v$
represent sets.  These are the same as the variables\index{variable!in set
theory} in our predicate calculus system above, except that now we informally
think of the variables as ranging over sets.  Note that the terms
``object,''\index{object} ``set,''\index{set} ``element,''\index{element}
``collection,''\index{collection} and ``family''\index{family} are synonymous,
as are ``is an element of,'' ``is a member of,''\index{member} ``is contained
in,'' and ``belongs to.''  The different terms are used for convenience; for
example, ``a collection of sets'' is less confusing than ``a set of sets.''
A set $x$ is said to be a ``subset''\index{subset} of $y$ if every element of
$x$ is also an element of $y$; we also say $x$ is ``included in''
$y$.

The axioms are very general and apply to almost any conceivable mathematical
object, and this level of abstraction can be overwhelming at first.  To gain an
intuitive feel, it can be helpful to draw a picture illustrating the concept;
for example, a circle containing dots could represent a collection of sets,
and a smaller circle drawn inside the circle could represent a subset.
Overlapping circles can illustrate intersection and union.  Circles that
illustrate the concepts of set theory are frequently used in elementary
textbooks and are called Venn diagrams\index{Venn diagram}.\index{axioms of
set theory}

1. Axiom of Extensionality:  Two sets are identical if they contain the same
   elements.\index{Axiom of Extensionality}

2. Axiom of Pairing:  The set $\{ x , y \}$ exists.\index{Axiom of Pairing}

3. Axiom of Power Sets:  The power set of a set (the collection of all of
   its subsets) exists.  For example, the power set of $\{x,y\}$ is
   $\{\varnothing,\{x\},\{y\},\{x,y\}\}$ and it exists.\index{Axiom
of Power Sets}

4. Axiom of the Null Set:  The empty set $\varnothing$ exists.\index{Axiom of
the Null Set}

5. Axiom of Union:  The union of a set (the set containing the elements of
   its members) exists.  For example, the union of $\{\{x,y\},\{z\}\}$ is
 $\{x,y,z\}$ and
   it exists.\index{Axiom of Union}

6. Axiom of Regularity:  Roughly, no set can contain itself, nor can there
   be membership ``loops,'' such as a set being an
   element of one of its members.\index{Axiom of Regularity}

7. Axiom of Infinity:  An infinite set exists.  An example of an infinite
   set is the set of all
   integers.\index{Axiom of Infinity}

8. Axiom of Separation:  The set exists that is obtained by restricting $x$
   with some property.  For example, if the set of all integers exists,
   then the set of all even integers exists.\index{Axiom of Separation}

9. Axiom of Replacement:  The range of a function whose domain is restricted
   to the elements of a set $x$, is also a set.  For example, there
   is a function
   from integers (the function's domain) to their squares (its
   range).  If we
   restrict the domain to even integers, its range will become the set of
   squares of even integers, so this axiom asserts that the set of
    squares of even numbers exists.  Technical note:  In general, the
   ``function'' need not be a set but can be a proper class.
   \index{Axiom of Replacement}

10. Axiom of Choice:  Let $x$ be a set whose members are pairwise
  disjoint\index{disjoint sets} (i.e,
  whose members contain no elements in common).  Then there exists another
  set containing one element from each member of $x$.  For
  example, if $x$ is
  $\{\{y,z\},\{w,v\}\}$, where $y$, $z$, $w$, and $v$ are
  different sets, then a set such as $\{z,w\}$
  exists (but the axiom doesn't tell
  us which one).  (Actually the Axiom
  of Choice is redundant if the set $x$, as in this example, has a finite
  number of elements.)\index{Axiom of Choice}

The Axiom of Choice is usually considered an extension of ZF set theory rather
than a proper part of it.  It is sometimes considered philosophically
controversial because it specifies the existence of a set without specifying
what the set is. Constructive logics, including intuitionistic logic,
do not accept the axiom of choice.
Since there is some lingering controversy, we often prefer proofs that do
not use the axiom of choice (where there is a known alternative), and
in some cases we will use weaker axioms than the full axiom of choice.
That said, the axiom of choice is a powerful and widely-accepted tool,
so we do use it when needed.
ZF set theory that includes the Axiom of Choice is
called Zermelo--Fraenkel set theory with choice (ZFC\index{ZFC set theory}).

When expressed symbolically, the Axiom of Separation and the Axiom of
Replacement contain wff symbols and therefore each represent infinitely many
axioms, one for each possible wff. For this reason, they are often called
axiom schemes\index{axiom scheme}\index{well-formed formula (wff)}.

It turns out that the Axiom of the Null Set, the Axiom of Pairing, and the
Axiom of Separation can be derived from the other axioms and are therefore
unnecessary, although they tend to be included in standard texts for various
reasons (historical, philosophical, and possibly because some authors may not
know this).  In the Metamath\index{Metamath} set theory database, these
redundant axioms are derived from the other ones instead of truly
being considered axioms.
This is in keeping with our general goal of minimizing the number of
axioms we must depend on.

\subsection{Other Axioms}

Above we qualified the phrase "all of mathematics" with "essentially."
The main important missing piece is the ability to do category theory,
which requires huge sets (inaccessible cardinals) larger than those
postulated by the ZFC axioms. The Tarski--Grothendieck Axiom postulates
the existence of such sets.
Note that this is the same axiom used by Mizar for supporting
category theory.
The Tarski--Grothendieck axiom
can be viewed as a very strong replacement of the Axiom of Infinity,
the Axiom of Choice, and the Axiom of Power Sets.
The \texttt{set.mm} database includes this axiom; see the database
for details about it.
Again, we only use this axiom when we need to.
You are only likely to encounter or use this axiom if you are doing
category theory, since its use is highly specialized,
so we will not list the Tarsky-Grothendieck axiom
in the short list of axioms below.

Can there be even more axioms?
Of course.
G\"{o}del showed that no finite set of axioms or axiom schemes can completely
describe any consistent theory strong enough to include arithmetic.
But practically speaking, the ones above are the accepted foundation that
almost all mathematicians explicitly or implicitly base their work on.

\section{The Axioms in the Metamath Language}\label{metaaxioms}

Here we list the axioms as they appear in
\texttt{set.mm}\index{set theory database (\texttt{set.mm})} so you can
look them up there easily.  Incidentally, the \texttt{show statement
/tex} command\index{\texttt{show statement} command} was used to
typeset them.

%macros from show statement /tex
\newbox\mlinebox
\newbox\mtrialbox
\newbox\startprefix  % Prefix for first line of a formula
\newbox\contprefix  % Prefix for continuation line of a formula
\def\startm{  % Initialize formula line
  \setbox\mlinebox=\hbox{\unhcopy\startprefix}
}
\def\m#1{  % Add a symbol to the formula
  \setbox\mtrialbox=\hbox{\unhcopy\mlinebox $\,#1$}
  \ifdim\wd\mtrialbox>\hsize
    \box\mlinebox
    \setbox\mlinebox=\hbox{\unhcopy\contprefix $\,#1$}
  \else
    \setbox\mlinebox=\hbox{\unhbox\mtrialbox}
  \fi
}
\def\endm{  % Output the last line of a formula
  \box\mlinebox
}

% \SLASH for \ , \TOR for \/ (text OR), \TAND for /\ (text and)
% This embeds a following forced space to force the space.
\newcommand\SLASH{\char`\\~}
\newcommand\TOR{\char`\\/~}
\newcommand\TAND{/\char`\\~}
%
% Macro to output metamath raw text.
% This assumes \startprefix and \contprefix are set.
% NOTE: "\" is tricky to escape, use \SLASH, \TOR, and \TAND inside.
% Any use of "$ { ~ ^" must be escaped; ~ and ^ must be escaped specially.
% We escape { and } for consistency.
% For more about how this macro written, see:
% https://stackoverflow.com/questions/4073674/
% how-to-disable-indentation-in-particular-section-in-latex/4075706
% Use frenchspacing, or "e." will get an extra space after it.
\newlength\mystoreparindent
\newlength\mystorehangindent
\newenvironment{mmraw}{%
\setlength{\mystoreparindent}{\the\parindent}
\setlength{\mystorehangindent}{\the\hangindent}
\setlength{\parindent}{0pt} % TODO - we'll put in the \startprefix instead
\setlength{\hangindent}{\wd\the\contprefix}
\begin{flushleft}
\begin{frenchspacing}
\begin{tt}
{\unhcopy\startprefix}%
}{%
\end{tt}
\end{frenchspacing}
\end{flushleft}
\setlength{\parindent}{\mystoreparindent}
\setlength{\hangindent}{\mystorehangindent}
\vskip 1ex
}

\needspace{5\baselineskip}
\subsection{Propositional Calculus}\label{propcalc}\index{axioms of
propositional calculus}

\needspace{2\baselineskip}
Axiom of Simplification.\label{ax1}

\setbox\startprefix=\hbox{\tt \ \ ax-1\ \$a\ }
\setbox\contprefix=\hbox{\tt \ \ \ \ \ \ \ \ \ \ }
\startm
\m{\vdash}\m{(}\m{\varphi}\m{\rightarrow}\m{(}\m{\psi}\m{\rightarrow}\m{\varphi}\m{)}
\m{)}
\endm

\needspace{3\baselineskip}
\noindent Axiom of Distribution.

\setbox\startprefix=\hbox{\tt \ \ ax-2\ \$a\ }
\setbox\contprefix=\hbox{\tt \ \ \ \ \ \ \ \ \ \ }
\startm
\m{\vdash}\m{(}\m{(}\m{\varphi}\m{\rightarrow}\m{(}\m{\psi}\m{\rightarrow}\m{\chi}
\m{)}\m{)}\m{\rightarrow}\m{(}\m{(}\m{\varphi}\m{\rightarrow}\m{\psi}\m{)}\m{
\rightarrow}\m{(}\m{\varphi}\m{\rightarrow}\m{\chi}\m{)}\m{)}\m{)}
\endm

\needspace{2\baselineskip}
\noindent Axiom of Contraposition.

\setbox\startprefix=\hbox{\tt \ \ ax-3\ \$a\ }
\setbox\contprefix=\hbox{\tt \ \ \ \ \ \ \ \ \ \ }
\startm
\m{\vdash}\m{(}\m{(}\m{\lnot}\m{\varphi}\m{\rightarrow}\m{\lnot}\m{\psi}\m{)}\m{
\rightarrow}\m{(}\m{\psi}\m{\rightarrow}\m{\varphi}\m{)}\m{)}
\endm


\needspace{4\baselineskip}
\noindent Rule of Modus Ponens.\label{axmp}\index{modus ponens}

\setbox\startprefix=\hbox{\tt \ \ maj\ \$e\ }
\setbox\contprefix=\hbox{\tt \ \ \ \ \ \ \ \ \ }
\startm
\m{\vdash}\m{(}\m{\varphi}\m{\rightarrow}\m{\psi}\m{)}
\endm

\setbox\startprefix=\hbox{\tt \ \ min\ \$e\ }
\setbox\contprefix=\hbox{\tt \ \ \ \ \ \ \ \ \ }
\startm
\m{\vdash}\m{\varphi}
\endm

\setbox\startprefix=\hbox{\tt \ \ ax-mp\ \$a\ }
\setbox\contprefix=\hbox{\tt \ \ \ \ \ \ \ \ \ \ \ }
\startm
\m{\vdash}\m{\psi}
\endm


\needspace{7\baselineskip}
\subsection{Axioms of Predicate Calculus with Equality---Tarski's S2}\index{axioms of predicate calculus}

\needspace{3\baselineskip}
\noindent Rule of Generalization.\index{rule of generalization}

\setbox\startprefix=\hbox{\tt \ \ ax-g.1\ \$e\ }
\setbox\contprefix=\hbox{\tt \ \ \ \ \ \ \ \ \ \ \ \ }
\startm
\m{\vdash}\m{\varphi}
\endm

\setbox\startprefix=\hbox{\tt \ \ ax-gen\ \$a\ }
\setbox\contprefix=\hbox{\tt \ \ \ \ \ \ \ \ \ \ \ \ }
\startm
\m{\vdash}\m{\forall}\m{x}\m{\varphi}
\endm

\needspace{2\baselineskip}
\noindent Axiom of Quantified Implication.

\setbox\startprefix=\hbox{\tt \ \ ax-4\ \$a\ }
\setbox\contprefix=\hbox{\tt \ \ \ \ \ \ \ \ \ \ }
\startm
\m{\vdash}\m{(}\m{\forall}\m{x}\m{(}\m{\forall}\m{x}\m{\varphi}\m{\rightarrow}\m{
\psi}\m{)}\m{\rightarrow}\m{(}\m{\forall}\m{x}\m{\varphi}\m{\rightarrow}\m{
\forall}\m{x}\m{\psi}\m{)}\m{)}
\endm

\needspace{3\baselineskip}
\noindent Axiom of Distinctness.

% Aka: Add $d x ph $.
\setbox\startprefix=\hbox{\tt \ \ ax-5\ \$a\ }
\setbox\contprefix=\hbox{\tt \ \ \ \ \ \ \ \ \ \ }
\startm
\m{\vdash}\m{(}\m{\varphi}\m{\rightarrow}\m{\forall}\m{x}\m{\varphi}\m{)}\m{where}\m{ }\m{\$d}\m{ }\m{x}\m{ }\m{\varphi}\m{ }\m{(}\m{x}\m{ }\m{does}\m{ }\m{not}\m{ }\m{occur}\m{ }\m{in}\m{ }\m{\varphi}\m{)}
\endm

\needspace{2\baselineskip}
\noindent Axiom of Existence.

\setbox\startprefix=\hbox{\tt \ \ ax-6\ \$a\ }
\setbox\contprefix=\hbox{\tt \ \ \ \ \ \ \ \ \ \ }
\startm
\m{\vdash}\m{(}\m{\forall}\m{x}\m{(}\m{x}\m{=}\m{y}\m{\rightarrow}\m{\forall}
\m{x}\m{\varphi}\m{)}\m{\rightarrow}\m{\varphi}\m{)}
\endm

\needspace{2\baselineskip}
\noindent Axiom of Equality.

\setbox\startprefix=\hbox{\tt \ \ ax-7\ \$a\ }
\setbox\contprefix=\hbox{\tt \ \ \ \ \ \ \ \ \ \ }
\startm
\m{\vdash}\m{(}\m{x}\m{=}\m{y}\m{\rightarrow}\m{(}\m{x}\m{=}\m{z}\m{
\rightarrow}\m{y}\m{=}\m{z}\m{)}\m{)}
\endm

\needspace{2\baselineskip}
\noindent Axiom of Left Equality for Binary Predicate.

\setbox\startprefix=\hbox{\tt \ \ ax-8\ \$a\ }
\setbox\contprefix=\hbox{\tt \ \ \ \ \ \ \ \ \ \ \ }
\startm
\m{\vdash}\m{(}\m{x}\m{=}\m{y}\m{\rightarrow}\m{(}\m{x}\m{\in}\m{z}\m{
\rightarrow}\m{y}\m{\in}\m{z}\m{)}\m{)}
\endm

\needspace{2\baselineskip}
\noindent Axiom of Right Equality for Binary Predicate.

\setbox\startprefix=\hbox{\tt \ \ ax-9\ \$a\ }
\setbox\contprefix=\hbox{\tt \ \ \ \ \ \ \ \ \ \ \ }
\startm
\m{\vdash}\m{(}\m{x}\m{=}\m{y}\m{\rightarrow}\m{(}\m{z}\m{\in}\m{x}\m{
\rightarrow}\m{z}\m{\in}\m{y}\m{)}\m{)}
\endm


\needspace{4\baselineskip}
\subsection{Axioms of Predicate Calculus with Equality---Auxiliary}\index{axioms of predicate calculus - auxiliary}

\needspace{2\baselineskip}
\noindent Axiom of Quantified Negation.

\setbox\startprefix=\hbox{\tt \ \ ax-10\ \$a\ }
\setbox\contprefix=\hbox{\tt \ \ \ \ \ \ \ \ \ \ }
\startm
\m{\vdash}\m{(}\m{\lnot}\m{\forall}\m{x}\m{\lnot}\m{\forall}\m{x}\m{\varphi}\m{
\rightarrow}\m{\varphi}\m{)}
\endm

\needspace{2\baselineskip}
\noindent Axiom of Quantifier Commutation.

\setbox\startprefix=\hbox{\tt \ \ ax-11\ \$a\ }
\setbox\contprefix=\hbox{\tt \ \ \ \ \ \ \ \ \ \ }
\startm
\m{\vdash}\m{(}\m{\forall}\m{x}\m{\forall}\m{y}\m{\varphi}\m{\rightarrow}\m{
\forall}\m{y}\m{\forall}\m{x}\m{\varphi}\m{)}
\endm

\needspace{3\baselineskip}
\noindent Axiom of Substitution.

\setbox\startprefix=\hbox{\tt \ \ ax-12\ \$a\ }
\setbox\contprefix=\hbox{\tt \ \ \ \ \ \ \ \ \ \ \ }
\startm
\m{\vdash}\m{(}\m{\lnot}\m{\forall}\m{x}\m{\,x}\m{=}\m{y}\m{\rightarrow}\m{(}
\m{x}\m{=}\m{y}\m{\rightarrow}\m{(}\m{\varphi}\m{\rightarrow}\m{\forall}\m{x}\m{(}
\m{x}\m{=}\m{y}\m{\rightarrow}\m{\varphi}\m{)}\m{)}\m{)}\m{)}
\endm

\needspace{3\baselineskip}
\noindent Axiom of Quantified Equality.

\setbox\startprefix=\hbox{\tt \ \ ax-13\ \$a\ }
\setbox\contprefix=\hbox{\tt \ \ \ \ \ \ \ \ \ \ \ }
\startm
\m{\vdash}\m{(}\m{\lnot}\m{\forall}\m{z}\m{\,z}\m{=}\m{x}\m{\rightarrow}\m{(}
\m{\lnot}\m{\forall}\m{z}\m{\,z}\m{=}\m{y}\m{\rightarrow}\m{(}\m{x}\m{=}\m{y}
\m{\rightarrow}\m{\forall}\m{z}\m{\,x}\m{=}\m{y}\m{)}\m{)}\m{)}
\endm

% \noindent Axiom of Quantifier Substitution
%
% \setbox\startprefix=\hbox{\tt \ \ ax-c11n\ \$a\ }
% \setbox\contprefix=\hbox{\tt \ \ \ \ \ \ \ \ \ \ \ }
% \startm
% \m{\vdash}\m{(}\m{\forall}\m{x}\m{\,x}\m{=}\m{y}\m{\rightarrow}\m{(}\m{\forall}
% \m{x}\m{\varphi}\m{\rightarrow}\m{\forall}\m{y}\m{\varphi}\m{)}\m{)}
% \endm
%
% \noindent Axiom of Distinct Variables. (This axiom requires
% that two individual variables
% be distinct\index{\texttt{\$d} statement}\index{distinct
% variables}.)
%
% \setbox\startprefix=\hbox{\tt \ \ \ \ \ \ \ \ \$d\ }
% \setbox\contprefix=\hbox{\tt \ \ \ \ \ \ \ \ \ \ \ }
% \startm
% \m{x}\m{\,}\m{y}
% \endm
%
% \setbox\startprefix=\hbox{\tt \ \ ax-c16\ \$a\ }
% \setbox\contprefix=\hbox{\tt \ \ \ \ \ \ \ \ \ \ \ }
% \startm
% \m{\vdash}\m{(}\m{\forall}\m{x}\m{\,x}\m{=}\m{y}\m{\rightarrow}\m{(}\m{\varphi}\m{
% \rightarrow}\m{\forall}\m{x}\m{\varphi}\m{)}\m{)}
% \endm

% \noindent Axiom of Quantifier Introduction (2).  (This axiom requires
% that the individual variable not occur in the
% wff\index{\texttt{\$d} statement}\index{distinct variables}.)
%
% \setbox\startprefix=\hbox{\tt \ \ \ \ \ \ \ \ \$d\ }
% \setbox\contprefix=\hbox{\tt \ \ \ \ \ \ \ \ \ \ \ }
% \startm
% \m{x}\m{\,}\m{\varphi}
% \endm
% \setbox\startprefix=\hbox{\tt \ \ ax-5\ \$a\ }
% \setbox\contprefix=\hbox{\tt \ \ \ \ \ \ \ \ \ \ \ }
% \startm
% \m{\vdash}\m{(}\m{\varphi}\m{\rightarrow}\m{\forall}\m{x}\m{\varphi}\m{)}
% \endm

\subsection{Set Theory}\label{mmsettheoryaxioms}

In order to make the axioms of set theory\index{axioms of set theory} a little
more compact, there are several definitions from logic that we make use of
implicitly, namely, ``logical {\sc and},''\index{conjunction ($\wedge$)}
\index{logical {\sc and} ($\wedge$)} ``logical equivalence,''\index{logical
equivalence ($\leftrightarrow$)}\index{biconditional ($\leftrightarrow$)} and
``there exists.''\index{existential quantifier ($\exists$)}

\begin{center}\begin{tabular}{rcl}
  $( \varphi \wedge \psi )$ &\mbox{stands for}& $\neg ( \varphi
     \rightarrow \neg \psi )$\\
  $( \varphi \leftrightarrow \psi )$& \mbox{stands
     for}& $( ( \varphi \rightarrow \psi ) \wedge
     ( \psi \rightarrow \varphi ) )$\\
  $\exists x \,\varphi$ &\mbox{stands for}& $\neg \forall x \neg \varphi$
\end{tabular}\end{center}

In addition, the axioms of set theory require that all variables be
dis\-tinct,\index{distinct variables}\footnote{Set theory axioms can be
devised so that {\em no} variables are required to be distinct,
provided we replace \texttt{ax-c16} with an axiom stating that ``at
least two things exist,'' thus
making \texttt{ax-5} the only other axiom requiring the
\texttt{\$d} statement.  These axioms are unconventional and are not
presented here, but they can be found on the \url{http://metamath.org}
web site.  See also the Comment on
p.~\pageref{nodd}.}\index{\texttt{\$d} statement} thus we also assume:
\begin{center}
  \texttt{\$d }$x\,y\,z\,w$
\end{center}

\needspace{2\baselineskip}
\noindent Axiom of Extensionality.\index{Axiom of Extensionality}

\setbox\startprefix=\hbox{\tt \ \ ax-ext\ \$a\ }
\setbox\contprefix=\hbox{\tt \ \ \ \ \ \ \ \ \ \ \ \ }
\startm
\m{\vdash}\m{(}\m{\forall}\m{x}\m{(}\m{x}\m{\in}\m{y}\m{\leftrightarrow}\m{x}
\m{\in}\m{z}\m{)}\m{\rightarrow}\m{y}\m{=}\m{z}\m{)}
\endm

\needspace{3\baselineskip}
\noindent Axiom of Replacement.\index{Axiom of Replacement}

\setbox\startprefix=\hbox{\tt \ \ ax-rep\ \$a\ }
\setbox\contprefix=\hbox{\tt \ \ \ \ \ \ \ \ \ \ \ \ }
\startm
\m{\vdash}\m{(}\m{\forall}\m{w}\m{\exists}\m{y}\m{\forall}\m{z}\m{(}\m{%
\forall}\m{y}\m{\varphi}\m{\rightarrow}\m{z}\m{=}\m{y}\m{)}\m{\rightarrow}\m{%
\exists}\m{y}\m{\forall}\m{z}\m{(}\m{z}\m{\in}\m{y}\m{\leftrightarrow}\m{%
\exists}\m{w}\m{(}\m{w}\m{\in}\m{x}\m{\wedge}\m{\forall}\m{y}\m{\varphi}\m{)}%
\m{)}\m{)}
\endm

\needspace{2\baselineskip}
\noindent Axiom of Union.\index{Axiom of Union}

\setbox\startprefix=\hbox{\tt \ \ ax-un\ \$a\ }
\setbox\contprefix=\hbox{\tt \ \ \ \ \ \ \ \ \ \ \ }
\startm
\m{\vdash}\m{\exists}\m{x}\m{\forall}\m{y}\m{(}\m{\exists}\m{x}\m{(}\m{y}\m{
\in}\m{x}\m{\wedge}\m{x}\m{\in}\m{z}\m{)}\m{\rightarrow}\m{y}\m{\in}\m{x}\m{)}
\endm

\needspace{2\baselineskip}
\noindent Axiom of Power Sets.\index{Axiom of Power Sets}

\setbox\startprefix=\hbox{\tt \ \ ax-pow\ \$a\ }
\setbox\contprefix=\hbox{\tt \ \ \ \ \ \ \ \ \ \ \ \ }
\startm
\m{\vdash}\m{\exists}\m{x}\m{\forall}\m{y}\m{(}\m{\forall}\m{x}\m{(}\m{x}\m{
\in}\m{y}\m{\rightarrow}\m{x}\m{\in}\m{z}\m{)}\m{\rightarrow}\m{y}\m{\in}\m{x}
\m{)}
\endm

\needspace{3\baselineskip}
\noindent Axiom of Regularity.\index{Axiom of Regularity}

\setbox\startprefix=\hbox{\tt \ \ ax-reg\ \$a\ }
\setbox\contprefix=\hbox{\tt \ \ \ \ \ \ \ \ \ \ \ \ }
\startm
\m{\vdash}\m{(}\m{\exists}\m{x}\m{\,x}\m{\in}\m{y}\m{\rightarrow}\m{\exists}
\m{x}\m{(}\m{x}\m{\in}\m{y}\m{\wedge}\m{\forall}\m{z}\m{(}\m{z}\m{\in}\m{x}\m{
\rightarrow}\m{\lnot}\m{z}\m{\in}\m{y}\m{)}\m{)}\m{)}
\endm

\needspace{3\baselineskip}
\noindent Axiom of Infinity.\index{Axiom of Infinity}

\setbox\startprefix=\hbox{\tt \ \ ax-inf\ \$a\ }
\setbox\contprefix=\hbox{\tt \ \ \ \ \ \ \ \ \ \ \ \ \ \ \ }
\startm
\m{\vdash}\m{\exists}\m{x}\m{(}\m{y}\m{\in}\m{x}\m{\wedge}\m{\forall}\m{y}%
\m{(}\m{y}\m{\in}\m{x}\m{\rightarrow}\m{\exists}\m{z}\m{(}\m{y}\m{\in}\m{z}\m{%
\wedge}\m{z}\m{\in}\m{x}\m{)}\m{)}\m{)}
\endm

\needspace{4\baselineskip}
\noindent Axiom of Choice.\index{Axiom of Choice}

\setbox\startprefix=\hbox{\tt \ \ ax-ac\ \$a\ }
\setbox\contprefix=\hbox{\tt \ \ \ \ \ \ \ \ \ \ \ \ \ \ }
\startm
\m{\vdash}\m{\exists}\m{x}\m{\forall}\m{y}\m{\forall}\m{z}\m{(}\m{(}\m{y}\m{%
\in}\m{z}\m{\wedge}\m{z}\m{\in}\m{w}\m{)}\m{\rightarrow}\m{\exists}\m{w}\m{%
\forall}\m{y}\m{(}\m{\exists}\m{w}\m{(}\m{(}\m{y}\m{\in}\m{z}\m{\wedge}\m{z}%
\m{\in}\m{w}\m{)}\m{\wedge}\m{(}\m{y}\m{\in}\m{w}\m{\wedge}\m{w}\m{\in}\m{x}%
\m{)}\m{)}\m{\leftrightarrow}\m{y}\m{=}\m{w}\m{)}\m{)}
\endm

\subsection{That's It}

There you have it, the axioms for (essentially) all of mathematics!
Wonder at them and stare at them in awe.  Put a copy in your wallet, and
you will carry in your pocket the encoding for all theorems ever proved
and that ever will be proved, from the most mundane to the most
profound.

\section{A Hierarchy of Definitions}\label{hierarchy}

The axioms in the previous section in principle embody everything that can be
done within standard mathematics.  However, it is impractical to accomplish
very much by using them directly, for even simple concepts (from a human
perspective) can involve extremely long, incomprehensible formulas.
Mathematics is made practical by introducing definitions\index{definition}.
Definitions usually introduce new symbols, or at least new relationships among
existing symbols, to abbreviate more complex formulas.  An important
requirement for a definition is that there exist a straightforward
(algorithmic) method for eliminating the abbreviation by expanding it into the
more primitive symbol string that it represents.  Some
important definitions included in
the file \texttt{set.mm} are listed in this section for reference, and also to
give you a feel for why something like $\omega$\index{omega ($\omega$)} (the
set of natural numbers\index{natural number} 0, 1, 2,\ldots) becomes very
complicated when completely expanded into primitive symbols.

What is the motivation for definitions, aside from allowing complicated
expressions to be expressed more simply?  In the case of  $\omega$, one goal is
to provide a basis for the theory of natural numbers.\index{natural number}
Before set theory was invented, a set of axioms for arithmetic, called Peano's
postulates\index{Peano's postulates}, was devised and shown to have the
properties one expects for natural numbers.  Now anyone can postulate a
set of axioms, but if the axioms are inconsistent contradictions can be derived
from them.  Once a contradiction is derived, anything can be trivially
proved, including
all the facts of arithmetic and their negations.  To ensure that an
axiom system is at least as reliable as the axioms for set theory, we can
define sets and operations on those sets that satisfy the new axioms. In the
\texttt{set.mm} Metamath database, we prove that the elements of $\omega$ satisfy
Peano's postulates, and it's a long and hard journey to get there directly
from the axioms of set theory.  But the result is confidence in the
foundations of arithmetic.  And there is another advantage:  we now have all
the tools of set theory at our disposal for manipulating objects that obey the
axioms for arithmetic.

What are the criteria we use for definitions?  First, and of utmost importance,
the definition should not be {\em creative}\index{creative
definition}\index{definition!creative}, that
is it should not allow an expression that previously qualified as a wff but
was not provable, to become provable.   Second, the definition should be {\em
eliminable}\index{definition!eliminability}, that is, there should exist an
algorithmic method for converting any expression using the definition into
a logically equivalent expression that previously qualified as a wff.

In almost all cases below, definitions connect two expressions with either
$\leftrightarrow$ or $=$.  Eliminating\footnote{Here we mean the
elimination that a human might do in his or her head.  To eliminate them as
part of a Metamath proof we would invoke one of a number of
theorems that deal with transitivity of equivalence or equality; there are
many such examples in the proofs in \texttt{set.mm}.} such a definition is a
simple matter of substituting the expression on the left-hand side ({\em
definiendum}\index{definiendum} or thing being defined) with the equivalent,
more primitive expression on the right-hand side ({\em
definiens}\index{definiens} or definition).

Often a definition has variables on the right-hand side which do not appear on
the left-hand side; these are called {\em dummy variables}.\index{dummy
variable!in definitions}  In this case, any
allowable substitution (such as a new, distinct
variable) can be used when the definition is eliminated.  Dummy variables may
be used only if they are {\em effectively bound}\index{effectively bound
variable}, meaning that the definition will remain logically equivalent upon
any substitution of a dummy variable with any other {\em qualifying
expression}\index{qualifying expression}, i.e.\ any symbol string (such as
another variable) that
meets the restrictions on the dummy variable imposed by \texttt{\$d} and
\texttt{\$f} statements.  For example, we could define a constant $\perp$
(inverted tee, meaning logical ``false'') as $( \varphi \wedge \lnot \varphi
)$, i.e.\ ``phi and not phi.''  Here $\varphi$ is effectively bound because the
definition remains logically equivalent when we replace $\varphi$ with any
other wff.  (It is actually \texttt{df-fal}
in \texttt{set.mm}, which defines $\perp$.)

There are two cases where eliminating definitions is a little more
complex.  These cases are the definitions \texttt{df-bi} and
\texttt{df-cleq}.  The first stretches the concept of a definition a
little, as in effect it ``defines a definition;'' however, it meets our
requirements for a definition in that it is eliminable and does not
strengthen the language.  Theorem \texttt{bii} shows the substitution
needed to eliminate the $\leftrightarrow$\index{logical equivalence
($\leftrightarrow$)}\index{biconditional ($\leftrightarrow$)} symbol.

Definition \texttt{df-cleq}\index{equality ($=$)} extends the usage of
the equality symbol to include ``classes''\index{class} in set theory.  The
reason it is potentially problematic is that it can lead to statements which
do not follow from logic alone but presuppose the Axiom of
Extensionality\index{Axiom of Extensionality}, so we include this axiom
as a hypothesis for the definition.  We could have made \texttt{df-cleq} directly
eliminable by introducing a new equality symbol, but have chosen not to do so
in keeping with standard textbook practice.  Definitions such as \texttt{df-cleq}
that extend the meaning of existing symbols must be introduced carefully so
that they do not lead to contradictions.  Definition \texttt{df-clel} also
extends the meaning of an existing symbol ($\in$); while it doesn't strengthen
the language like \texttt{df-cleq}, this is not obvious and it must also be
subject to the same scrutiny.

Exercise:  Study how the wff $x\in\omega$, meaning ``$x$ is a natural
number,'' could be expanded in terms of primitive symbols, starting with the
definitions \texttt{df-clel} on p.~\pageref{dfclel} and \texttt{df-om} on
p.~\pageref{dfom} and working your way back.  Don't bother to work out the
details; just make sure that you understand how you could do it in principle.
The answer is shown in the footnote on p.~\pageref{expandom}.  If you
actually do work it out, you won't get exactly the same answer because we used
a few simplifications such as discarding occurrences of $\lnot\lnot$ (double
negation).

In the definitions below, we have placed the {\sc ascii} Metamath source
below each of the formulas to help you become familiar with the
notation in the database.  For simplicity, the necessary \texttt{\$f}
and \texttt{\$d} statements are not shown.  If you are in doubt, use the
\texttt{show statement}\index{\texttt{show statement} command} command
in the Metamath program to see the full statement.
A selection of this notation is summarized in Appendix~\ref{ASCII}.

To understand the motivation for these definitions, you should consult the
references indicated:  Takeuti and Zaring \cite{Takeuti}\index{Takeuti, Gaisi},
Quine \cite{Quine}\index{Quine, Willard Van Orman}, Bell and Machover
\cite{Bell}\index{Bell, J. L.}, and Enderton \cite{Enderton}\index{Enderton,
Herbert B.}.  Our list of definitions is provided more for reference than as a
learning aid.  However, by looking at a few of them you can gain a feel for
how the hierarchy is built up.  The definitions are a representative sample of
the 130 or so in \texttt{set.mm}, but they are complete with respect to the
theorem examples we will present in Section~\ref{sometheorems}.  Also, some are
slightly different from, but logically equivalent to, the ones in \texttt{set.mm}
(some of which have been revised over time to shorten them, for example).

\subsection{Definitions for Propositional Calculus}\label{metadefprop}

The symbols $\varphi$, $\psi$, and $\chi$ represent wffs.

Our first definition introduces the biconditional
connective\footnote{The term ``connective'' is informally used to mean a
symbol that is placed between two variables or adjacent to a variable,
whereas a mathematical ``constant'' usually indicates a symbol such as
the number 0 that may replace a variable or metavariable.  From
Metamath's point of view, there is no distinction between a connective
and a constant; both are constants in the Metamath
language.}\index{connective}\index{constant} (also called logical
equivalence)\index{logical equivalence
($\leftrightarrow$)}\index{biconditional ($\leftrightarrow$)}.  Unlike
most traditional developments, we have chosen not to have a separate
symbol such as ``Df.'' to mean ``is defined as.''  Instead, we will use
the biconditional connective for this purpose, as it lets us use
logic to manipulate definitions directly.  Here we state the properties
of the biconditional connective with a carefully crafted \texttt{\$a}
statement, which effectively uses the biconditional connective to define
itself.  The $\leftrightarrow$ symbol can be eliminated from a formula
using theorem \texttt{bii}, which is derived later.

\vskip 2ex
\noindent Define the biconditional connective.\label{df-bi}

\vskip 0.5ex
\setbox\startprefix=\hbox{\tt \ \ df-bi\ \$a\ }
\setbox\contprefix=\hbox{\tt \ \ \ \ \ \ \ \ \ \ \ }
\startm
\m{\vdash}\m{\lnot}\m{(}\m{(}\m{(}\m{\varphi}\m{\leftrightarrow}\m{\psi}\m{)}%
\m{\rightarrow}\m{\lnot}\m{(}\m{(}\m{\varphi}\m{\rightarrow}\m{\psi}\m{)}\m{%
\rightarrow}\m{\lnot}\m{(}\m{\psi}\m{\rightarrow}\m{\varphi}\m{)}\m{)}\m{)}\m{%
\rightarrow}\m{\lnot}\m{(}\m{\lnot}\m{(}\m{(}\m{\varphi}\m{\rightarrow}\m{%
\psi}\m{)}\m{\rightarrow}\m{\lnot}\m{(}\m{\psi}\m{\rightarrow}\m{\varphi}\m{)}%
\m{)}\m{\rightarrow}\m{(}\m{\varphi}\m{\leftrightarrow}\m{\psi}\m{)}\m{)}\m{)}
\endm
\begin{mmraw}%
|- -. ( ( ( ph <-> ps ) -> -. ( ( ph -> ps ) ->
-. ( ps -> ph ) ) ) -> -. ( -. ( ( ph -> ps ) -> -. (
ps -> ph ) ) -> ( ph <-> ps ) ) ) \$.
\end{mmraw}

\noindent This theorem relates the biconditional connective to primitive
connectives and can be used to eliminate the $\leftrightarrow$ symbol from any
wff.

\vskip 0.5ex
\setbox\startprefix=\hbox{\tt \ \ bii\ \$p\ }
\setbox\contprefix=\hbox{\tt \ \ \ \ \ \ \ \ \ }
\startm
\m{\vdash}\m{(}\m{(}\m{\varphi}\m{\leftrightarrow}\m{\psi}\m{)}\m{\leftrightarrow}
\m{\lnot}\m{(}\m{(}\m{\varphi}\m{\rightarrow}\m{\psi}\m{)}\m{\rightarrow}\m{\lnot}
\m{(}\m{\psi}\m{\rightarrow}\m{\varphi}\m{)}\m{)}\m{)}
\endm
\begin{mmraw}%
|- ( ( ph <-> ps ) <-> -. ( ( ph -> ps ) -> -. ( ps -> ph ) ) ) \$= ... \$.
\end{mmraw}

\noindent Define disjunction ({\sc or}).\index{disjunction ($\vee$)}%
\index{logical or (vee)@logical {\sc or} ($\vee$)}%
\index{df-or@\texttt{df-or}}\label{df-or}

\vskip 0.5ex
\setbox\startprefix=\hbox{\tt \ \ df-or\ \$a\ }
\setbox\contprefix=\hbox{\tt \ \ \ \ \ \ \ \ \ \ \ }
\startm
\m{\vdash}\m{(}\m{(}\m{\varphi}\m{\vee}\m{\psi}\m{)}\m{\leftrightarrow}\m{(}\m{
\lnot}\m{\varphi}\m{\rightarrow}\m{\psi}\m{)}\m{)}
\endm
\begin{mmraw}%
|- ( ( ph \TOR ps ) <-> ( -. ph -> ps ) ) \$.
\end{mmraw}

\noindent Define conjunction ({\sc and}).\index{conjunction ($\wedge$)}%
\index{logical {\sc and} ($\wedge$)}%
\index{df-an@\texttt{df-an}}\label{df-an}

\vskip 0.5ex
\setbox\startprefix=\hbox{\tt \ \ df-an\ \$a\ }
\setbox\contprefix=\hbox{\tt \ \ \ \ \ \ \ \ \ \ \ }
\startm
\m{\vdash}\m{(}\m{(}\m{\varphi}\m{\wedge}\m{\psi}\m{)}\m{\leftrightarrow}\m{\lnot}
\m{(}\m{\varphi}\m{\rightarrow}\m{\lnot}\m{\psi}\m{)}\m{)}
\endm
\begin{mmraw}%
|- ( ( ph \TAND ps ) <-> -. ( ph -> -. ps ) ) \$.
\end{mmraw}

\noindent Define disjunction ({\sc or}) of 3 wffs.%
\index{df-3or@\texttt{df-3or}}\label{df-3or}

\vskip 0.5ex
\setbox\startprefix=\hbox{\tt \ \ df-3or\ \$a\ }
\setbox\contprefix=\hbox{\tt \ \ \ \ \ \ \ \ \ \ \ \ }
\startm
\m{\vdash}\m{(}\m{(}\m{\varphi}\m{\vee}\m{\psi}\m{\vee}\m{\chi}\m{)}\m{
\leftrightarrow}\m{(}\m{(}\m{\varphi}\m{\vee}\m{\psi}\m{)}\m{\vee}\m{\chi}\m{)}
\m{)}
\endm
\begin{mmraw}%
|- ( ( ph \TOR ps \TOR ch ) <-> ( ( ph \TOR ps ) \TOR ch ) ) \$.
\end{mmraw}

\noindent Define conjunction ({\sc and}) of 3 wffs.%
\index{df-3an}\label{df-3an}

\vskip 0.5ex
\setbox\startprefix=\hbox{\tt \ \ df-3an\ \$a\ }
\setbox\contprefix=\hbox{\tt \ \ \ \ \ \ \ \ \ \ \ \ }
\startm
\m{\vdash}\m{(}\m{(}\m{\varphi}\m{\wedge}\m{\psi}\m{\wedge}\m{\chi}\m{)}\m{
\leftrightarrow}\m{(}\m{(}\m{\varphi}\m{\wedge}\m{\psi}\m{)}\m{\wedge}\m{\chi}
\m{)}\m{)}
\endm

\begin{mmraw}%
|- ( ( ph \TAND ps \TAND ch ) <-> ( ( ph \TAND ps ) \TAND ch ) ) \$.
\end{mmraw}

\subsection{Definitions for Predicate Calculus}\label{metadefpred}

The symbols $x$, $y$, and $z$ represent individual variables of predicate
calculus.  In this section, they are not necessarily distinct unless it is
explicitly
mentioned.

\vskip 2ex
\noindent Define existential quantification.
The expression $\exists x \varphi$ means
``there exists an $x$ where $\varphi$ is true.''\index{existential quantifier
($\exists$)}\label{df-ex}

\vskip 0.5ex
\setbox\startprefix=\hbox{\tt \ \ df-ex\ \$a\ }
\setbox\contprefix=\hbox{\tt \ \ \ \ \ \ \ \ \ \ \ }
\startm
\m{\vdash}\m{(}\m{\exists}\m{x}\m{\varphi}\m{\leftrightarrow}\m{\lnot}\m{\forall}
\m{x}\m{\lnot}\m{\varphi}\m{)}
\endm
\begin{mmraw}%
|- ( E. x ph <-> -. A. x -. ph ) \$.
\end{mmraw}

\noindent Define proper substitution.\index{proper
substitution}\index{substitution!proper}\label{df-sb}
In our notation, we use $[ y / x ] \varphi$ to mean ``the wff that
results when $y$ is properly substituted for $x$ in the wff
$\varphi$.''\footnote{
This can also be described
as substituting $x$ with $y$, $y$ properly replaces $x$, or
$x$ is properly replaced by $y$.}
% This is elsb4, though it currently says: ( [ x / y ] z e. y <-> z e. x )
For example,
$[ y / x ] z \in x$ is the same as $z \in y$.
One way to remember this notation is to notice that it looks like division
and recall that $( y / x ) \cdot x $ is $y$ (when $x \neq 0$).
The notation is different from the notation $\varphi ( x | y )$
that is sometimes used, because the latter notation is ambiguous for us:
for example, we don't know whether $\lnot \varphi ( x | y )$ is to be
interpreted as $\lnot ( \varphi ( x | y ) )$ or
$( \lnot \varphi ) ( x | y )$.\footnote{Because of the way
we initially defined wffs, this is the case
with any postfix connective\index{postfix connective} (one occurring after the
symbols being connected) or infix connective\index{infix connective} (one
occurring between the symbols being connected).  Metamath does not have a
built-in notion of operator binding strength that could eliminate the
ambiguity.  The initial parenthesis effectively provides a prefix
connective\index{prefix connective} to eliminate ambiguity.  Some conventions,
such as Polish notation\index{Polish notation} used in the 1930's and 1940's
by Polish logicians, use only prefix connectives and thus allow the total
elimination of parentheses, at the expense of readability.  In Metamath we
could actually redefine all notation to be Polish if we wanted to without
having to change any proofs!}  Other texts often use $\varphi(y)$ to indicate
our $[ y / x ] \varphi$, but this notation is even more ambiguous since there is
no explicit indication of what is being substituted.
Note that this
definition is valid even when
$x$ and $y$ are the same variable.  The first conjunct is a ``trick'' used to
achieve this property, making the definition look somewhat peculiar at
first.

\vskip 0.5ex
\setbox\startprefix=\hbox{\tt \ \ df-sb\ \$a\ }
\setbox\contprefix=\hbox{\tt \ \ \ \ \ \ \ \ \ \ \ }
\startm
\m{\vdash}\m{(}\m{[}\m{y}\m{/}\m{x}\m{]}\m{\varphi}\m{\leftrightarrow}\m{(}%
\m{(}\m{x}\m{=}\m{y}\m{\rightarrow}\m{\varphi}\m{)}\m{\wedge}\m{\exists}\m{x}%
\m{(}\m{x}\m{=}\m{y}\m{\wedge}\m{\varphi}\m{)}\m{)}\m{)}
\endm
\begin{mmraw}%
|- ( [ y / x ] ph <-> ( ( x = y -> ph ) \TAND E. x ( x = y \TAND ph ) ) ) \$.
\end{mmraw}


\noindent Define existential uniqueness\index{existential uniqueness
quantifier ($\exists "!$)} (``there exists exactly one'').  Note that $y$ is a
variable distinct from $x$ and not occurring in $\varphi$.\label{df-eu}

\vskip 0.5ex
\setbox\startprefix=\hbox{\tt \ \ df-eu\ \$a\ }
\setbox\contprefix=\hbox{\tt \ \ \ \ \ \ \ \ \ \ \ }
\startm
\m{\vdash}\m{(}\m{\exists}\m{{!}}\m{x}\m{\varphi}\m{\leftrightarrow}\m{\exists}
\m{y}\m{\forall}\m{x}\m{(}\m{\varphi}\m{\leftrightarrow}\m{x}\m{=}\m{y}\m{)}\m{)}
\endm

\begin{mmraw}%
|- ( E! x ph <-> E. y A. x ( ph <-> x = y ) ) \$.
\end{mmraw}

\subsection{Definitions for Set Theory}\label{setdefinitions}

The symbols $x$, $y$, $z$, and $w$ represent individual variables of
predicate calculus, which in set theory are understood to be sets.
However, using only the constructs shown so far would be very inconvenient.

To make set theory more practical, we introduce the notion of a ``class.''
A class\index{class} is either a set variable (such as $x$) or an
expression of the form $\{ x | \varphi\}$ (called an ``abstraction
class''\index{abstraction class}\index{class abstraction}).  Note that
sets (i.e.\ individual variables) always exist (this is a theorem of
logic, namely $\exists y \, y = x$ for any set $x$), whereas classes may
or may not exist (i.e.\ $\exists y \, y = A$ may or may not be true).
If a class does not exist it is called a ``proper class.''\index{proper
class}\index{class!proper} Definitions \texttt{df-clab},
\texttt{df-cleq}, and \texttt{df-clel} can be used to convert an
expression containing classes into one containing only set variables and
wff metavariables.

The symbols $A$, $B$, $C$, $D$, $ F$, $G$, and $R$ are metavariables that range
over classes.  A class metavariable $A$ may be eliminated from a wff by
replacing it with $\{ x|\varphi\}$ where neither $x$ nor $\varphi$ occur in
the wff.

The theory of classes can be shown to be an eliminable and conservative
extension of set theory. The \textbf{eliminability}
property shows that for every
formula in the extended language we can build a logically equivalent
formula in the basic language; so that even if the extended language
provides more ease to convey and formulate mathematical ideas for set
theory, its expressive power does not in fact strengthen the basic
language's expressive power.
The \textbf{conservation} property shows that for
every proof of a formula of the basic language in the extended system
we can build another proof of the same formula in the basic system;
so that, concerning theorems on sets only, the deductive powers of
the extended system and of the basic system are identical. Together,
these properties mean that the extended language can be treated as a
definitional extension that is \textbf{sound}.

A rigorous justification, which we will not give here, can be found in
Levy \cite[pp.~357-366]{Levy} supplementing his informal introduction to class
theory on pp.~7-17. Two other good treatments of class theory are provided
by Quine \cite[pp.~15-21]{Quine}\index{Quine, Willard Van Orman}
and also \cite[pp.~10-14]{Takeuti}\index{Takeuti, Gaisi}.
Quine's exposition (he calls them virtual classes)
is nicely written and very readable.

In the rest of this
section, individual variables are always assumed to be distinct from
each other unless otherwise indicated.  In addition, dummy variables on the
right-hand side of a definition do not occur in the class and wff
metavariables in the definition.

The definitions we present here are a partial but self-contained
collection selected from several hundred that appear in the current
\texttt{set.mm} database.  They are adequate for a basic development of
elementary set theory.

\vskip 2ex
\noindent Define the abstraction class.\index{abstraction class}\index{class
abstraction}\label{df-clab}  $x$ and $y$
need not be distinct.  Definition 2.1 of Quine, p.~16.  This definition may
seem puzzling since it is shorter than the expression being defined and does not
buy us anything in terms of brevity.  The reason we introduce this definition
is because it fits in neatly with the extension of the $\in$ connective
provided by \texttt{df-clel}.

\vskip 0.5ex
\setbox\startprefix=\hbox{\tt \ \ df-clab\ \$a\ }
\setbox\contprefix=\hbox{\tt \ \ \ \ \ \ \ \ \ \ \ \ \ }
\startm
\m{\vdash}\m{(}\m{x}\m{\in}\m{\{}\m{y}\m{|}\m{\varphi}\m{\}}\m{%
\leftrightarrow}\m{[}\m{x}\m{/}\m{y}\m{]}\m{\varphi}\m{)}
\endm
\begin{mmraw}%
|- ( x e. \{ y | ph \} <-> [ x / y ] ph ) \$.
\end{mmraw}

\noindent Define the equality connective between classes\index{class
equality}\label{df-cleq}.  See Quine or Chapter 4 of Takeuti and Zaring for its
justification and methods for eliminating it.  This is an example of a
somewhat ``dangerous'' definition, because it extends the use of the
existing equality symbol rather than introducing a new symbol, allowing
us to make statements in the original language that may not be true.
For example, it permits us to deduce $y = z \leftrightarrow \forall x (
x \in y \leftrightarrow x \in z )$ which is not a theorem of logic but
rather presupposes the Axiom of Extensionality,\index{Axiom of
Extensionality} which we include as a hypothesis so that we can know
when this axiom is assumed in a proof (with the \texttt{show
trace{\char`\_}back} command).  We could avoid the danger by introducing
another symbol, say $\eqcirc$, in place of $=$; this
would also have the advantage of making elimination of the definition
straightforward and would eliminate the need for Extensionality as a
hypothesis.  We would then also have the advantage of being able to
identify exactly where Extensionality truly comes into play.  One of our
theorems would be $x \eqcirc y \leftrightarrow x = y$
by invoking Extensionality.  However in keeping with standard practice
we retain the ``dangerous'' definition.

\vskip 0.5ex
\setbox\startprefix=\hbox{\tt \ \ df-cleq.1\ \$e\ }
\setbox\contprefix=\hbox{\tt \ \ \ \ \ \ \ \ \ \ \ \ \ \ \ }
\startm
\m{\vdash}\m{(}\m{\forall}\m{x}\m{(}\m{x}\m{\in}\m{y}\m{\leftrightarrow}\m{x}
\m{\in}\m{z}\m{)}\m{\rightarrow}\m{y}\m{=}\m{z}\m{)}
\endm
\setbox\startprefix=\hbox{\tt \ \ df-cleq\ \$a\ }
\setbox\contprefix=\hbox{\tt \ \ \ \ \ \ \ \ \ \ \ \ \ }
\startm
\m{\vdash}\m{(}\m{A}\m{=}\m{B}\m{\leftrightarrow}\m{\forall}\m{x}\m{(}\m{x}\m{
\in}\m{A}\m{\leftrightarrow}\m{x}\m{\in}\m{B}\m{)}\m{)}
\endm
% We need to reset the startprefix and contprefix.
\setbox\startprefix=\hbox{\tt \ \ df-cleq.1\ \$e\ }
\setbox\contprefix=\hbox{\tt \ \ \ \ \ \ \ \ \ \ \ \ \ \ \ }
\begin{mmraw}%
|- ( A. x ( x e. y <-> x e. z ) -> y = z ) \$.
\end{mmraw}
\setbox\startprefix=\hbox{\tt \ \ df-cleq\ \$a\ }
\setbox\contprefix=\hbox{\tt \ \ \ \ \ \ \ \ \ \ \ \ \ }
\begin{mmraw}%
|- ( A = B <-> A. x ( x e. A <-> x e. B ) ) \$.
\end{mmraw}

\noindent Define the membership connective between classes\index{class
membership}.  Theorem 6.3 of Quine, p.~41, which we adopt as a definition.
Note that it extends the use of the existing membership symbol, but unlike
{\tt df-cleq} it does not extend the set of valid wffs of logic when the class
metavariables are replaced with set variables.\label{dfclel}\label{df-clel}

\vskip 0.5ex
\setbox\startprefix=\hbox{\tt \ \ df-clel\ \$a\ }
\setbox\contprefix=\hbox{\tt \ \ \ \ \ \ \ \ \ \ \ \ \ }
\startm
\m{\vdash}\m{(}\m{A}\m{\in}\m{B}\m{\leftrightarrow}\m{\exists}\m{x}\m{(}\m{x}
\m{=}\m{A}\m{\wedge}\m{x}\m{\in}\m{B}\m{)}\m{)}
\endm
\begin{mmraw}%
|- ( A e. B <-> E. x ( x = A \TAND x e. B ) ) \$.?
\end{mmraw}

\noindent Define inequality.

\vskip 0.5ex
\setbox\startprefix=\hbox{\tt \ \ df-ne\ \$a\ }
\setbox\contprefix=\hbox{\tt \ \ \ \ \ \ \ \ \ \ \ }
\startm
\m{\vdash}\m{(}\m{A}\m{\ne}\m{B}\m{\leftrightarrow}\m{\lnot}\m{A}\m{=}\m{B}%
\m{)}
\endm
\begin{mmraw}%
|- ( A =/= B <-> -. A = B ) \$.
\end{mmraw}

\noindent Define restricted universal quantification.\index{universal
quantifier ($\forall$)!restricted}  Enderton, p.~22.

\vskip 0.5ex
\setbox\startprefix=\hbox{\tt \ \ df-ral\ \$a\ }
\setbox\contprefix=\hbox{\tt \ \ \ \ \ \ \ \ \ \ \ \ }
\startm
\m{\vdash}\m{(}\m{\forall}\m{x}\m{\in}\m{A}\m{\varphi}\m{\leftrightarrow}\m{%
\forall}\m{x}\m{(}\m{x}\m{\in}\m{A}\m{\rightarrow}\m{\varphi}\m{)}\m{)}
\endm
\begin{mmraw}%
|- ( A. x e. A ph <-> A. x ( x e. A -> ph ) ) \$.
\end{mmraw}

\noindent Define restricted existential quantification.\index{existential
quantifier ($\exists$)!restricted}  Enderton, p.~22.

\vskip 0.5ex
\setbox\startprefix=\hbox{\tt \ \ df-rex\ \$a\ }
\setbox\contprefix=\hbox{\tt \ \ \ \ \ \ \ \ \ \ \ \ }
\startm
\m{\vdash}\m{(}\m{\exists}\m{x}\m{\in}\m{A}\m{\varphi}\m{\leftrightarrow}\m{%
\exists}\m{x}\m{(}\m{x}\m{\in}\m{A}\m{\wedge}\m{\varphi}\m{)}\m{)}
\endm
\begin{mmraw}%
|- ( E. x e. A ph <-> E. x ( x e. A \TAND ph ) ) \$.
\end{mmraw}

\noindent Define the universal class\index{universal class ($V$)}.  Definition
5.20, p.~21, of Takeuti and Zaring.\label{df-v}

\vskip 0.5ex
\setbox\startprefix=\hbox{\tt \ \ df-v\ \$a\ }
\setbox\contprefix=\hbox{\tt \ \ \ \ \ \ \ \ \ \ }
\startm
\m{\vdash}\m{{\rm V}}\m{=}\m{\{}\m{x}\m{|}\m{x}\m{=}\m{x}\m{\}}
\endm
\begin{mmraw}%
|- {\char`\_}V = \{ x | x = x \} \$.
\end{mmraw}

\noindent Define the subclass\index{subclass}\index{subset} relationship
between two classes (called the subset relation if the classes are sets i.e.\
are not proper).  Definition 5.9 of Takeuti and Zaring, p.~17.\label{df-ss}

\vskip 0.5ex
\setbox\startprefix=\hbox{\tt \ \ df-ss\ \$a\ }
\setbox\contprefix=\hbox{\tt \ \ \ \ \ \ \ \ \ \ \ }
\startm
\m{\vdash}\m{(}\m{A}\m{\subseteq}\m{B}\m{\leftrightarrow}\m{\forall}\m{x}\m{(}
\m{x}\m{\in}\m{A}\m{\rightarrow}\m{x}\m{\in}\m{B}\m{)}\m{)}
\endm
\begin{mmraw}%
|- ( A C\_ B <-> A. x ( x e. A -> x e. B ) ) \$.
\end{mmraw}

\noindent Define the union\index{union} of two classes.  Definition 5.6 of Takeuti and Zaring,
p.~16.\label{df-un}

\vskip 0.5ex
\setbox\startprefix=\hbox{\tt \ \ df-un\ \$a\ }
\setbox\contprefix=\hbox{\tt \ \ \ \ \ \ \ \ \ \ \ }
\startm
\m{\vdash}\m{(}\m{A}\m{\cup}\m{B}\m{)}\m{=}\m{\{}\m{x}\m{|}\m{(}\m{x}\m{\in}
\m{A}\m{\vee}\m{x}\m{\in}\m{B}\m{)}\m{\}}
\endm
\begin{mmraw}%
( A u. B ) = \{ x | ( x e. A \TOR x e. B ) \} \$.
\end{mmraw}

\noindent Define the intersection\index{intersection} of two classes.  Definition 5.6 of
Takeuti and Zaring, p.~16.\label{df-in}

\vskip 0.5ex
\setbox\startprefix=\hbox{\tt \ \ df-in\ \$a\ }
\setbox\contprefix=\hbox{\tt \ \ \ \ \ \ \ \ \ \ \ }
\startm
\m{\vdash}\m{(}\m{A}\m{\cap}\m{B}\m{)}\m{=}\m{\{}\m{x}\m{|}\m{(}\m{x}\m{\in}
\m{A}\m{\wedge}\m{x}\m{\in}\m{B}\m{)}\m{\}}
\endm
% Caret ^ requires special treatment
\begin{mmraw}%
|- ( A i\^{}i B ) = \{ x | ( x e. A \TAND x e. B ) \} \$.
\end{mmraw}

\noindent Define class difference\index{class difference}\index{set difference}.
Definition 5.12 of Takeuti and Zaring, p.~20.  Several notations are used in
the literature; we chose the $\setminus$ convention instead of a minus sign to
reserve the latter for later use in, e.g., arithmetic.\label{df-dif}

\vskip 0.5ex
\setbox\startprefix=\hbox{\tt \ \ df-dif\ \$a\ }
\setbox\contprefix=\hbox{\tt \ \ \ \ \ \ \ \ \ \ \ \ }
\startm
\m{\vdash}\m{(}\m{A}\m{\setminus}\m{B}\m{)}\m{=}\m{\{}\m{x}\m{|}\m{(}\m{x}\m{
\in}\m{A}\m{\wedge}\m{\lnot}\m{x}\m{\in}\m{B}\m{)}\m{\}}
\endm
\begin{mmraw}%
( A \SLASH B ) = \{ x | ( x e. A \TAND -. x e. B ) \} \$.
\end{mmraw}

\noindent Define the empty or null set\index{empty set}\index{null set}.
Compare  Definition 5.14 of Takeuti and Zaring, p.~20.\label{df-nul}

\vskip 0.5ex
\setbox\startprefix=\hbox{\tt \ \ df-nul\ \$a\ }
\setbox\contprefix=\hbox{\tt \ \ \ \ \ \ \ \ \ \ }
\startm
\m{\vdash}\m{\varnothing}\m{=}\m{(}\m{{\rm V}}\m{\setminus}\m{{\rm V}}\m{)}
\endm
\begin{mmraw}%
|- (/) = ( {\char`\_}V \SLASH {\char`\_}V ) \$.
\end{mmraw}

\noindent Define power class\index{power set}\index{power class}.  Definition 5.10 of
Takeuti and Zaring, p.~17, but we also let it apply to proper classes.  (Note
that \verb$~P$ is the symbol for calligraphic P, the tilde
suggesting ``curly;'' see Appendix~\ref{ASCII}.)\label{df-pw}

\vskip 0.5ex
\setbox\startprefix=\hbox{\tt \ \ df-pw\ \$a\ }
\setbox\contprefix=\hbox{\tt \ \ \ \ \ \ \ \ \ \ \ }
\startm
\m{\vdash}\m{{\cal P}}\m{A}\m{=}\m{\{}\m{x}\m{|}\m{x}\m{\subseteq}\m{A}\m{\}}
\endm
% Special incantation required to put ~ into the text
\begin{mmraw}%
|- \char`\~P~A = \{ x | x C\_ A \} \$.
\end{mmraw}

\noindent Define the singleton of a class\index{singleton}.  Definition 7.1 of
Quine, p.~48.  It is well-defined for proper classes, although
it is not very meaningful in this case, where it evaluates to the empty
set.

\vskip 0.5ex
\setbox\startprefix=\hbox{\tt \ \ df-sn\ \$a\ }
\setbox\contprefix=\hbox{\tt \ \ \ \ \ \ \ \ \ \ \ }
\startm
\m{\vdash}\m{\{}\m{A}\m{\}}\m{=}\m{\{}\m{x}\m{|}\m{x}\m{=}\m{A}\m{\}}
\endm
\begin{mmraw}%
|- \{ A \} = \{ x | x = A \} \$.
\end{mmraw}%

\noindent Define an unordered pair of classes\index{unordered pair}\index{pair}.  Definition
7.1 of Quine, p.~48.

\vskip 0.5ex
\setbox\startprefix=\hbox{\tt \ \ df-pr\ \$a\ }
\setbox\contprefix=\hbox{\tt \ \ \ \ \ \ \ \ \ \ \ }
\startm
\m{\vdash}\m{\{}\m{A}\m{,}\m{B}\m{\}}\m{=}\m{(}\m{\{}\m{A}\m{\}}\m{\cup}\m{\{}
\m{B}\m{\}}\m{)}
\endm
\begin{mmraw}%
|- \{ A , B \} = ( \{ A \} u. \{ B \} ) \$.
\end{mmraw}

\noindent Define an unordered triple of classes\index{unordered triple}.  Definition of
Enderton, p.~19.

\vskip 0.5ex
\setbox\startprefix=\hbox{\tt \ \ df-tp\ \$a\ }
\setbox\contprefix=\hbox{\tt \ \ \ \ \ \ \ \ \ \ \ }
\startm
\m{\vdash}\m{\{}\m{A}\m{,}\m{B}\m{,}\m{C}\m{\}}\m{=}\m{(}\m{\{}\m{A}\m{,}\m{B}
\m{\}}\m{\cup}\m{\{}\m{C}\m{\}}\m{)}
\endm
\begin{mmraw}%
|- \{ A , B , C \} = ( \{ A , B \} u. \{ C \} ) \$.
\end{mmraw}%

\noindent Kuratowski's\index{Kuratowski, Kazimierz} ordered pair\index{ordered
pair} definition.  Definition 9.1 of Quine, p.~58. For proper classes it is
not meaningful but is well-defined for convenience.  (Note that \verb$<.$
stands for $\langle$ whereas \verb$<$ stands for $<$, and similarly for
\verb$>.$\,.)\label{df-op}

\vskip 0.5ex
\setbox\startprefix=\hbox{\tt \ \ df-op\ \$a\ }
\setbox\contprefix=\hbox{\tt \ \ \ \ \ \ \ \ \ \ \ }
\startm
\m{\vdash}\m{\langle}\m{A}\m{,}\m{B}\m{\rangle}\m{=}\m{\{}\m{\{}\m{A}\m{\}}
\m{,}\m{\{}\m{A}\m{,}\m{B}\m{\}}\m{\}}
\endm
\begin{mmraw}%
|- <. A , B >. = \{ \{ A \} , \{ A , B \} \} \$.
\end{mmraw}

\noindent Define the union of a class\index{union}.  Definition 5.5, p.~16,
of Takeuti and Zaring.

\vskip 0.5ex
\setbox\startprefix=\hbox{\tt \ \ df-uni\ \$a\ }
\setbox\contprefix=\hbox{\tt \ \ \ \ \ \ \ \ \ \ \ \ }
\startm
\m{\vdash}\m{\bigcup}\m{A}\m{=}\m{\{}\m{x}\m{|}\m{\exists}\m{y}\m{(}\m{x}\m{
\in}\m{y}\m{\wedge}\m{y}\m{\in}\m{A}\m{)}\m{\}}
\endm
\begin{mmraw}%
|- U. A = \{ x | E. y ( x e. y \TAND y e. A ) \} \$.
\end{mmraw}

\noindent Define the intersection\index{intersection} of a class.  Definition 7.35,
p.~44, of Takeuti and Zaring.

\vskip 0.5ex
\setbox\startprefix=\hbox{\tt \ \ df-int\ \$a\ }
\setbox\contprefix=\hbox{\tt \ \ \ \ \ \ \ \ \ \ \ \ }
\startm
\m{\vdash}\m{\bigcap}\m{A}\m{=}\m{\{}\m{x}\m{|}\m{\forall}\m{y}\m{(}\m{y}\m{
\in}\m{A}\m{\rightarrow}\m{x}\m{\in}\m{y}\m{)}\m{\}}
\endm
\begin{mmraw}%
|- |\^{}| A = \{ x | A. y ( y e. A -> x e. y ) \} \$.
\end{mmraw}

\noindent Define a transitive class\index{transitive class}\index{transitive
set}.  This should not be confused with a transitive relation, which is a different
concept.  Definition from p.~71 of Enderton, extended to classes.

\vskip 0.5ex
\setbox\startprefix=\hbox{\tt \ \ df-tr\ \$a\ }
\setbox\contprefix=\hbox{\tt \ \ \ \ \ \ \ \ \ \ \ }
\startm
\m{\vdash}\m{(}\m{\mbox{\rm Tr}}\m{A}\m{\leftrightarrow}\m{\bigcup}\m{A}\m{
\subseteq}\m{A}\m{)}
\endm
\begin{mmraw}%
|- ( Tr A <-> U. A C\_ A ) \$.
\end{mmraw}

\noindent Define a notation for a general binary relation\index{binary
relation}.  Definition 6.18, p.~29, of Takeuti and Zaring, generalized to
arbitrary classes.  This definition is well-defined, although not very
meaningful, when classes $A$ and/or $B$ are proper.\label{dfbr}  The lack of
parentheses (or any other connective) creates no ambiguity since we are defining
an atomic wff.

\vskip 0.5ex
\setbox\startprefix=\hbox{\tt \ \ df-br\ \$a\ }
\setbox\contprefix=\hbox{\tt \ \ \ \ \ \ \ \ \ \ \ }
\startm
\m{\vdash}\m{(}\m{A}\m{\,R}\m{\,B}\m{\leftrightarrow}\m{\langle}\m{A}\m{,}\m{B}
\m{\rangle}\m{\in}\m{R}\m{)}
\endm
\begin{mmraw}%
|- ( A R B <-> <. A , B >. e. R ) \$.
\end{mmraw}

\noindent Define an abstraction class of ordered pairs\index{abstraction
class!of ordered
pairs}.  A special case of Definition 4.16, p.~14, of Takeuti and Zaring.
Note that $ z $ must be distinct from $ x $ and $ y $,
and $ z $ must not occur in $\varphi$, but $ x $ and $ y $ may be
identical and may appear in $\varphi$.

\vskip 0.5ex
\setbox\startprefix=\hbox{\tt \ \ df-opab\ \$a\ }
\setbox\contprefix=\hbox{\tt \ \ \ \ \ \ \ \ \ \ \ \ \ }
\startm
\m{\vdash}\m{\{}\m{\langle}\m{x}\m{,}\m{y}\m{\rangle}\m{|}\m{\varphi}\m{\}}\m{=}
\m{\{}\m{z}\m{|}\m{\exists}\m{x}\m{\exists}\m{y}\m{(}\m{z}\m{=}\m{\langle}\m{x}
\m{,}\m{y}\m{\rangle}\m{\wedge}\m{\varphi}\m{)}\m{\}}
\endm

\begin{mmraw}%
|- \{ <. x , y >. | ph \} = \{ z | E. x E. y ( z =
<. x , y >. /\ ph ) \} \$.
\end{mmraw}

\noindent Define the epsilon relation\index{epsilon relation}.  Similar to Definition
6.22, p.~30, of Takeuti and Zaring.

\vskip 0.5ex
\setbox\startprefix=\hbox{\tt \ \ df-eprel\ \$a\ }
\setbox\contprefix=\hbox{\tt \ \ \ \ \ \ \ \ \ \ \ \ \ \ }
\startm
\m{\vdash}\m{{\rm E}}\m{=}\m{\{}\m{\langle}\m{x}\m{,}\m{y}\m{\rangle}\m{|}\m{x}\m{
\in}\m{y}\m{\}}
\endm
\begin{mmraw}%
|- \_E = \{ <. x , y >. | x e. y \} \$.
\end{mmraw}

\noindent Define a founded relation\index{founded relation}.  $R$ is a founded
relation on $A$ iff\index{iff} (if and only if) each nonempty subset of $A$
has an ``$R$-minimal element.''  Similar to Definition 6.21, p.~30, of
Takeuti and Zaring.

\vskip 0.5ex
\setbox\startprefix=\hbox{\tt \ \ df-fr\ \$a\ }
\setbox\contprefix=\hbox{\tt \ \ \ \ \ \ \ \ \ \ \ }
\startm
\m{\vdash}\m{(}\m{R}\m{\,\mbox{\rm Fr}}\m{\,A}\m{\leftrightarrow}\m{\forall}\m{x}
\m{(}\m{(}\m{x}\m{\subseteq}\m{A}\m{\wedge}\m{\lnot}\m{x}\m{=}\m{\varnothing}
\m{)}\m{\rightarrow}\m{\exists}\m{y}\m{(}\m{y}\m{\in}\m{x}\m{\wedge}\m{(}\m{x}
\m{\cap}\m{\{}\m{z}\m{|}\m{z}\m{\,R}\m{\,y}\m{\}}\m{)}\m{=}\m{\varnothing}\m{)}
\m{)}\m{)}
\endm
\begin{mmraw}%
|- ( R Fr A <-> A. x ( ( x C\_ A \TAND -. x = (/) ) ->
E. y ( y e. x \TAND ( x i\^{}i \{ z | z R y \} ) = (/) ) ) ) \$.
\end{mmraw}

\noindent Define a well-ordering\index{well-ordering}.  $R$ is a well-ordering of $A$ iff
it is founded on $A$ and the elements of $A$ are pairwise $R$-comparable.
Similar to Definition 6.24(2), p.~30, of Takeuti and Zaring.

\vskip 0.5ex
\setbox\startprefix=\hbox{\tt \ \ df-we\ \$a\ }
\setbox\contprefix=\hbox{\tt \ \ \ \ \ \ \ \ \ \ \ }
\startm
\m{\vdash}\m{(}\m{R}\m{\,\mbox{\rm We}}\m{\,A}\m{\leftrightarrow}\m{(}\m{R}\m{\,
\mbox{\rm Fr}}\m{\,A}\m{\wedge}\m{\forall}\m{x}\m{\forall}\m{y}\m{(}\m{(}\m{x}\m{
\in}\m{A}\m{\wedge}\m{y}\m{\in}\m{A}\m{)}\m{\rightarrow}\m{(}\m{x}\m{\,R}\m{\,y}
\m{\vee}\m{x}\m{=}\m{y}\m{\vee}\m{y}\m{\,R}\m{\,x}\m{)}\m{)}\m{)}\m{)}
\endm
\begin{mmraw}%
( R We A <-> ( R Fr A \TAND A. x A. y ( ( x e.
A \TAND y e. A ) -> ( x R y \TOR x = y \TOR y R x ) ) ) ) \$.
\end{mmraw}

\noindent Define the ordinal predicate\index{ordinal predicate}, which is true for a class
that is transitive and is well-ordered by the epsilon relation.  Similar to
definition on p.~468, Bell and Machover.

\vskip 0.5ex
\setbox\startprefix=\hbox{\tt \ \ df-ord\ \$a\ }
\setbox\contprefix=\hbox{\tt \ \ \ \ \ \ \ \ \ \ \ \ }
\startm
\m{\vdash}\m{(}\m{\mbox{\rm Ord}}\m{\,A}\m{\leftrightarrow}\m{(}
\m{\mbox{\rm Tr}}\m{\,A}\m{\wedge}\m{E}\m{\,\mbox{\rm We}}\m{\,A}\m{)}\m{)}
\endm
\begin{mmraw}%
|- ( Ord A <-> ( Tr A \TAND E We A ) ) \$.
\end{mmraw}

\noindent Define the class of all ordinal numbers\index{ordinal number}.  An ordinal number is
a set that satisfies the ordinal predicate.  Definition 7.11 of Takeuti and
Zaring, p.~38.

\vskip 0.5ex
\setbox\startprefix=\hbox{\tt \ \ df-on\ \$a\ }
\setbox\contprefix=\hbox{\tt \ \ \ \ \ \ \ \ \ \ \ }
\startm
\m{\vdash}\m{\,\mbox{\rm On}}\m{=}\m{\{}\m{x}\m{|}\m{\mbox{\rm Ord}}\m{\,x}
\m{\}}
\endm
\begin{mmraw}%
|- On = \{ x | Ord x \} \$.
\end{mmraw}

\noindent Define the limit ordinal predicate\index{limit ordinal}, which is true for a
non-empty ordinal that is not a successor (i.e.\ that is the union of itself).
Compare Bell and Machover, p.~471 and Exercise (1), p.~42 of Takeuti and
Zaring.

\vskip 0.5ex
\setbox\startprefix=\hbox{\tt \ \ df-lim\ \$a\ }
\setbox\contprefix=\hbox{\tt \ \ \ \ \ \ \ \ \ \ \ \ }
\startm
\m{\vdash}\m{(}\m{\mbox{\rm Lim}}\m{\,A}\m{\leftrightarrow}\m{(}\m{\mbox{
\rm Ord}}\m{\,A}\m{\wedge}\m{\lnot}\m{A}\m{=}\m{\varnothing}\m{\wedge}\m{A}
\m{=}\m{\bigcup}\m{A}\m{)}\m{)}
\endm
\begin{mmraw}%
|- ( Lim A <-> ( Ord A \TAND -. A = (/) \TAND A = U. A ) ) \$.
\end{mmraw}

\noindent Define the successor\index{successor} of a class.  Definition 7.22 of Takeuti
and Zaring, p.~41.  Our definition is a generalization to classes, although it
is meaningless when classes are proper.

\vskip 0.5ex
\setbox\startprefix=\hbox{\tt \ \ df-suc\ \$a\ }
\setbox\contprefix=\hbox{\tt \ \ \ \ \ \ \ \ \ \ \ \ }
\startm
\m{\vdash}\m{\,\mbox{\rm suc}}\m{\,A}\m{=}\m{(}\m{A}\m{\cup}\m{\{}\m{A}\m{\}}
\m{)}
\endm
\begin{mmraw}%
|- suc A = ( A u. \{ A \} ) \$.
\end{mmraw}

\noindent Define the class of natural numbers\index{natural number}\index{omega
($\omega$)}.  Compare Bell and Machover, p.~471.\label{dfom}

\vskip 0.5ex
\setbox\startprefix=\hbox{\tt \ \ df-om\ \$a\ }
\setbox\contprefix=\hbox{\tt \ \ \ \ \ \ \ \ \ \ \ }
\startm
\m{\vdash}\m{\omega}\m{=}\m{\{}\m{x}\m{|}\m{(}\m{\mbox{\rm Ord}}\m{\,x}\m{
\wedge}\m{\forall}\m{y}\m{(}\m{\mbox{\rm Lim}}\m{\,y}\m{\rightarrow}\m{x}\m{
\in}\m{y}\m{)}\m{)}\m{\}}
\endm
\begin{mmraw}%
|- om = \{ x | ( Ord x \TAND A. y ( Lim y -> x e. y ) ) \} \$.
\end{mmraw}

\noindent Define the Cartesian product (also called the
cross product)\index{Cartesian product}\index{cross product}
of two classes.  Definition 9.11 of Quine, p.~64.

\vskip 0.5ex
\setbox\startprefix=\hbox{\tt \ \ df-xp\ \$a\ }
\setbox\contprefix=\hbox{\tt \ \ \ \ \ \ \ \ \ \ \ }
\startm
\m{\vdash}\m{(}\m{A}\m{\times}\m{B}\m{)}\m{=}\m{\{}\m{\langle}\m{x}\m{,}\m{y}
\m{\rangle}\m{|}\m{(}\m{x}\m{\in}\m{A}\m{\wedge}\m{y}\m{\in}\m{B}\m{)}\m{\}}
\endm
\begin{mmraw}%
|- ( A X. B ) = \{ <. x , y >. | ( x e. A \TAND y e. B) \} \$.
\end{mmraw}

\noindent Define a relation\index{relation}.  Definition 6.4(1) of Takeuti and
Zaring, p.~23.

\vskip 0.5ex
\setbox\startprefix=\hbox{\tt \ \ df-rel\ \$a\ }
\setbox\contprefix=\hbox{\tt \ \ \ \ \ \ \ \ \ \ \ \ }
\startm
\m{\vdash}\m{(}\m{\mbox{\rm Rel}}\m{\,A}\m{\leftrightarrow}\m{A}\m{\subseteq}
\m{(}\m{{\rm V}}\m{\times}\m{{\rm V}}\m{)}\m{)}
\endm
\begin{mmraw}%
|- ( Rel A <-> A C\_ ( {\char`\_}V X. {\char`\_}V ) ) \$.
\end{mmraw}

\noindent Define the domain\index{domain} of a class.  Definition 6.5(1) of
Takeuti and Zaring, p.~24.

\vskip 0.5ex
\setbox\startprefix=\hbox{\tt \ \ df-dm\ \$a\ }
\setbox\contprefix=\hbox{\tt \ \ \ \ \ \ \ \ \ \ \ }
\startm
\m{\vdash}\m{\,\mbox{\rm dom}}\m{A}\m{=}\m{\{}\m{x}\m{|}\m{\exists}\m{y}\m{
\langle}\m{x}\m{,}\m{y}\m{\rangle}\m{\in}\m{A}\m{\}}
\endm
\begin{mmraw}%
|- dom A = \{ x | E. y <. x , y >. e. A \} \$.
\end{mmraw}

\noindent Define the range\index{range} of a class.  Definition 6.5(2) of
Takeuti and Zaring, p.~24.

\vskip 0.5ex
\setbox\startprefix=\hbox{\tt \ \ df-rn\ \$a\ }
\setbox\contprefix=\hbox{\tt \ \ \ \ \ \ \ \ \ \ \ }
\startm
\m{\vdash}\m{\,\mbox{\rm ran}}\m{A}\m{=}\m{\{}\m{y}\m{|}\m{\exists}\m{x}\m{
\langle}\m{x}\m{,}\m{y}\m{\rangle}\m{\in}\m{A}\m{\}}
\endm
\begin{mmraw}%
|- ran A = \{ y | E. x <. x , y >. e. A \} \$.
\end{mmraw}

\noindent Define the restriction\index{restriction} of a class.  Definition
6.6(1) of Takeuti and Zaring, p.~24.

\vskip 0.5ex
\setbox\startprefix=\hbox{\tt \ \ df-res\ \$a\ }
\setbox\contprefix=\hbox{\tt \ \ \ \ \ \ \ \ \ \ \ \ }
\startm
\m{\vdash}\m{(}\m{A}\m{\restriction}\m{B}\m{)}\m{=}\m{(}\m{A}\m{\cap}\m{(}\m{B}
\m{\times}\m{{\rm V}}\m{)}\m{)}
\endm
\begin{mmraw}%
|- ( A |` B ) = ( A i\^{}i ( B X. {\char`\_}V ) ) \$.
\end{mmraw}

\noindent Define the image\index{image} of a class.  Definition 6.6(2) of
Takeuti and Zaring, p.~24.

\vskip 0.5ex
\setbox\startprefix=\hbox{\tt \ \ df-ima\ \$a\ }
\setbox\contprefix=\hbox{\tt \ \ \ \ \ \ \ \ \ \ \ \ }
\startm
\m{\vdash}\m{(}\m{A}\m{``}\m{B}\m{)}\m{=}\m{\,\mbox{\rm ran}}\m{\,(}\m{A}\m{
\restriction}\m{B}\m{)}
\endm
\begin{mmraw}%
|- ( A " B ) = ran ( A |` B ) \$.
\end{mmraw}

\noindent Define the composition\index{composition} of two classes.  Definition 6.6(3) of
Takeuti and Zaring, p.~24.

\vskip 0.5ex
\setbox\startprefix=\hbox{\tt \ \ df-co\ \$a\ }
\setbox\contprefix=\hbox{\tt \ \ \ \ \ \ \ \ \ \ \ \ }
\startm
\m{\vdash}\m{(}\m{A}\m{\circ}\m{B}\m{)}\m{=}\m{\{}\m{\langle}\m{x}\m{,}\m{y}\m{
\rangle}\m{|}\m{\exists}\m{z}\m{(}\m{\langle}\m{x}\m{,}\m{z}\m{\rangle}\m{\in}
\m{B}\m{\wedge}\m{\langle}\m{z}\m{,}\m{y}\m{\rangle}\m{\in}\m{A}\m{)}\m{\}}
\endm
\begin{mmraw}%
|- ( A o. B ) = \{ <. x , y >. | E. z ( <. x , z
>. e. B \TAND <. z , y >. e. A ) \} \$.
\end{mmraw}

\noindent Define a function\index{function}.  Definition 6.4(4) of Takeuti and
Zaring, p.~24.

\vskip 0.5ex
\setbox\startprefix=\hbox{\tt \ \ df-fun\ \$a\ }
\setbox\contprefix=\hbox{\tt \ \ \ \ \ \ \ \ \ \ \ \ }
\startm
\m{\vdash}\m{(}\m{\mbox{\rm Fun}}\m{\,A}\m{\leftrightarrow}\m{(}
\m{\mbox{\rm Rel}}\m{\,A}\m{\wedge}
\m{\forall}\m{x}\m{\exists}\m{z}\m{\forall}\m{y}\m{(}
\m{\langle}\m{x}\m{,}\m{y}\m{\rangle}\m{\in}\m{A}\m{\rightarrow}\m{y}\m{=}\m{z}
\m{)}\m{)}\m{)}
\endm
\begin{mmraw}%
|- ( Fun A <-> ( Rel A /\ A. x E. z A. y ( <. x
   , y >. e. A -> y = z ) ) ) \$.
\end{mmraw}

\noindent Define a function with domain.  Definition 6.15(1) of Takeuti and
Zaring, p.~27.

\vskip 0.5ex
\setbox\startprefix=\hbox{\tt \ \ df-fn\ \$a\ }
\setbox\contprefix=\hbox{\tt \ \ \ \ \ \ \ \ \ \ \ }
\startm
\m{\vdash}\m{(}\m{A}\m{\,\mbox{\rm Fn}}\m{\,B}\m{\leftrightarrow}\m{(}
\m{\mbox{\rm Fun}}\m{\,A}\m{\wedge}\m{\mbox{\rm dom}}\m{\,A}\m{=}\m{B}\m{)}
\m{)}
\endm
\begin{mmraw}%
|- ( A Fn B <-> ( Fun A \TAND dom A = B ) ) \$.
\end{mmraw}

\noindent Define a function with domain and co-domain.  Definition 6.15(3)
of Takeuti and Zaring, p.~27.

\vskip 0.5ex
\setbox\startprefix=\hbox{\tt \ \ df-f\ \$a\ }
\setbox\contprefix=\hbox{\tt \ \ \ \ \ \ \ \ \ \ }
\startm
\m{\vdash}\m{(}\m{F}\m{:}\m{A}\m{\longrightarrow}\m{B}\m{
\leftrightarrow}\m{(}\m{F}\m{\,\mbox{\rm Fn}}\m{\,A}\m{\wedge}\m{
\mbox{\rm ran}}\m{\,F}\m{\subseteq}\m{B}\m{)}\m{)}
\endm
\begin{mmraw}%
|- ( F : A --> B <-> ( F Fn A \TAND ran F C\_ B ) ) \$.
\end{mmraw}

\noindent Define a one-to-one function\index{one-to-one function}.  Compare
Definition 6.15(5) of Takeuti and Zaring, p.~27.

\vskip 0.5ex
\setbox\startprefix=\hbox{\tt \ \ df-f1\ \$a\ }
\setbox\contprefix=\hbox{\tt \ \ \ \ \ \ \ \ \ \ \ }
\startm
\m{\vdash}\m{(}\m{F}\m{:}\m{A}\m{
\raisebox{.5ex}{${\textstyle{\:}_{\mbox{\footnotesize\rm
1\tt -\rm 1}}}\atop{\textstyle{
\longrightarrow}\atop{\textstyle{}^{\mbox{\footnotesize\rm {\ }}}}}$}
}\m{B}
\m{\leftrightarrow}\m{(}\m{F}\m{:}\m{A}\m{\longrightarrow}\m{B}
\m{\wedge}\m{\forall}\m{y}\m{\exists}\m{z}\m{\forall}\m{x}\m{(}\m{\langle}\m{x}
\m{,}\m{y}\m{\rangle}\m{\in}\m{F}\m{\rightarrow}\m{x}\m{=}\m{z}\m{)}\m{)}\m{)}
\endm
\begin{mmraw}%
|- ( F : A -1-1-> B <-> ( F : A --> B \TAND
   A. y E. z A. x ( <. x , y >. e. F -> x = z ) ) ) \$.
\end{mmraw}

\noindent Define an onto function\index{onto function}.  Definition 6.15(4) of Takeuti and
Zaring, p.~27.

\vskip 0.5ex
\setbox\startprefix=\hbox{\tt \ \ df-fo\ \$a\ }
\setbox\contprefix=\hbox{\tt \ \ \ \ \ \ \ \ \ \ \ }
\startm
\m{\vdash}\m{(}\m{F}\m{:}\m{A}\m{
\raisebox{.5ex}{${\textstyle{\:}_{\mbox{\footnotesize\rm
{\ }}}}\atop{\textstyle{
\longrightarrow}\atop{\textstyle{}^{\mbox{\footnotesize\rm onto}}}}$}
}\m{B}
\m{\leftrightarrow}\m{(}\m{F}\m{\,\mbox{\rm Fn}}\m{\,A}\m{\wedge}
\m{\mbox{\rm ran}}\m{\,F}\m{=}\m{B}\m{)}\m{)}
\endm
\begin{mmraw}%
|- ( F : A -onto-> B <-> ( F Fn A /\ ran F = B ) ) \$.
\end{mmraw}

\noindent Define a one-to-one, onto function.  Compare Definition 6.15(6) of
Takeuti and Zaring, p.~27.

\vskip 0.5ex
\setbox\startprefix=\hbox{\tt \ \ df-f1o\ \$a\ }
\setbox\contprefix=\hbox{\tt \ \ \ \ \ \ \ \ \ \ \ \ }
\startm
\m{\vdash}\m{(}\m{F}\m{:}\m{A}
\m{
\raisebox{.5ex}{${\textstyle{\:}_{\mbox{\footnotesize\rm
1\tt -\rm 1}}}\atop{\textstyle{
\longrightarrow}\atop{\textstyle{}^{\mbox{\footnotesize\rm onto}}}}$}
}
\m{B}
\m{\leftrightarrow}\m{(}\m{F}\m{:}\m{A}
\m{
\raisebox{.5ex}{${\textstyle{\:}_{\mbox{\footnotesize\rm
1\tt -\rm 1}}}\atop{\textstyle{
\longrightarrow}\atop{\textstyle{}^{\mbox{\footnotesize\rm {\ }}}}}$}
}
\m{B}\m{\wedge}\m{F}\m{:}\m{A}
\m{
\raisebox{.5ex}{${\textstyle{\:}_{\mbox{\footnotesize\rm
{\ }}}}\atop{\textstyle{
\longrightarrow}\atop{\textstyle{}^{\mbox{\footnotesize\rm onto}}}}$}
}
\m{B}\m{)}\m{)}
\endm
\begin{mmraw}%
|- ( F : A -1-1-onto-> B <-> ( F : A -1-1-> B? \TAND F : A -onto-> B ) ) \$.?
\end{mmraw}

\noindent Define the value of a function\index{function value}.  This
definition applies to any class and evaluates to the empty set when it is not
meaningful. Note that $ F`A$ means the same thing as the more familiar $ F(A)$
notation for a function's value at $A$.  The $ F`A$ notation is common in
formal set theory.\label{df-fv}

\vskip 0.5ex
\setbox\startprefix=\hbox{\tt \ \ df-fv\ \$a\ }
\setbox\contprefix=\hbox{\tt \ \ \ \ \ \ \ \ \ \ \ }
\startm
\m{\vdash}\m{(}\m{F}\m{`}\m{A}\m{)}\m{=}\m{\bigcup}\m{\{}\m{x}\m{|}\m{(}\m{F}%
\m{``}\m{\{}\m{A}\m{\}}\m{)}\m{=}\m{\{}\m{x}\m{\}}\m{\}}
\endm
\begin{mmraw}%
|- ( F ` A ) = U. \{ x | ( F " \{ A \} ) = \{ x \} \} \$.
\end{mmraw}

\noindent Define the result of an operation.\index{operation}  Here, $F$ is
     an operation on two
     values (such as $+$ for real numbers).   This is defined for proper
     classes $A$ and $B$ even though not meaningful in that case.  However,
     the definition can be meaningful when $F$ is a proper class.\label{dfopr}

\vskip 0.5ex
\setbox\startprefix=\hbox{\tt \ \ df-opr\ \$a\ }
\setbox\contprefix=\hbox{\tt \ \ \ \ \ \ \ \ \ \ \ \ }
\startm
\m{\vdash}\m{(}\m{A}\m{\,F}\m{\,B}\m{)}\m{=}\m{(}\m{F}\m{`}\m{\langle}\m{A}%
\m{,}\m{B}\m{\rangle}\m{)}
\endm
\begin{mmraw}%
|- ( A F B ) = ( F ` <. A , B >. ) \$.
\end{mmraw}

\section{Tricks of the Trade}\label{tricks}

In the \texttt{set.mm}\index{set theory database (\texttt{set.mm})} database our goal
was usually to conform to modern notation.  However in some cases the
relationship to standard textbook language may be obscured by several
unconventional devices we used to simplify the development and to take
advantage of the Metamath language.  In this section we will describe some
common conventions used in \texttt{set.mm}.

\begin{itemize}
\item
The turnstile\index{turnstile ({$\,\vdash$})} symbol, $\vdash$, meaning ``it
is provable that,'' is the first token of all assertions and hypotheses that
aren't syntax constructions.  This is a standard convention in logic.  (We
mentioned this earlier, but this symbol is bothersome to some people without a
logic background.  It has no deeper meaning but just provides us with a way to
distinguish syntax constructions from ordinary mathematical statements.)

\item
A hypothesis of the form

\vskip 1ex
\setbox\startprefix=\hbox{\tt \ \ \ \ \ \ \ \ \ \$e\ }
\setbox\contprefix=\hbox{\tt \ \ \ \ \ \ \ \ \ \ }
\startm
\m{\vdash}\m{(}\m{\varphi}\m{\rightarrow}\m{\forall}\m{x}\m{\varphi}\m{)}
\endm
\vskip 1ex

should be read ``assume variable $x$ is (effectively) not free in wff
$\varphi$.''\index{effectively not free}
Literally, this says ``assume it is provable that $\varphi \rightarrow \forall
x\, \varphi$.''  This device lets us avoid the complexities associated with
the standard treatment of free and bound variables.
%% Uncomment this when uncommenting section {formalspec} below
The footnote on p.~\pageref{effectivelybound} discusses this further.

\item
A statement of one of the forms

\vskip 1ex
\setbox\startprefix=\hbox{\tt \ \ \ \ \ \ \ \ \ \$a\ }
\setbox\contprefix=\hbox{\tt \ \ \ \ \ \ \ \ \ \ }
\startm
\m{\vdash}\m{(}\m{\lnot}\m{\forall}\m{x}\m{\,x}\m{=}\m{y}\m{\rightarrow}
\m{\ldots}\m{)}
\endm
\setbox\startprefix=\hbox{\tt \ \ \ \ \ \ \ \ \ \$p\ }
\setbox\contprefix=\hbox{\tt \ \ \ \ \ \ \ \ \ \ }
\startm
\m{\vdash}\m{(}\m{\lnot}\m{\forall}\m{x}\m{\,x}\m{=}\m{y}\m{\rightarrow}
\m{\ldots}\m{)}
\endm
\vskip 1ex

should be read ``if $x$ and $y$ are distinct variables, then...''  This
antecedent provides us with a technical device to avoid the need for the
\texttt{\$d} statement early in our development of predicate calculus,
permitting symbol manipulations to be as conceptually simple as those in
propositional calculus.  However, the \texttt{\$d} statement eventually
becomes a requirement, and after that this device is rarely used.

\item
The statement

\vskip 1ex
\setbox\startprefix=\hbox{\tt \ \ \ \ \ \ \ \ \ \$d\ }
\setbox\contprefix=\hbox{\tt \ \ \ \ \ \ \ \ \ \ }
\startm
\m{x}\m{\,y}
\endm
\vskip 1ex

should be read ``assume $x$ and $y$ are distinct variables.''

\item
The statement

\vskip 1ex
\setbox\startprefix=\hbox{\tt \ \ \ \ \ \ \ \ \ \$d\ }
\setbox\contprefix=\hbox{\tt \ \ \ \ \ \ \ \ \ \ }
\startm
\m{x}\m{\,\varphi}
\endm
\vskip 1ex

should be read ``assume $x$ does not occur in $\varphi$.''

\item
The statement

\vskip 1ex
\setbox\startprefix=\hbox{\tt \ \ \ \ \ \ \ \ \ \$d\ }
\setbox\contprefix=\hbox{\tt \ \ \ \ \ \ \ \ \ \ }
\startm
\m{x}\m{\,A}
\endm
\vskip 1ex

should be read ``assume variable $x$ does not occur in class $A$.''

\item
The restriction and hypothesis group

\vskip 1ex
\setbox\startprefix=\hbox{\tt \ \ \ \ \ \ \ \ \ \$d\ }
\setbox\contprefix=\hbox{\tt \ \ \ \ \ \ \ \ \ \ }
\startm
\m{x}\m{\,A}
\endm
\setbox\startprefix=\hbox{\tt \ \ \ \ \ \ \ \ \ \$d\ }
\setbox\contprefix=\hbox{\tt \ \ \ \ \ \ \ \ \ \ }
\startm
\m{x}\m{\,\psi}
\endm
\setbox\startprefix=\hbox{\tt \ \ \ \ \ \ \ \ \ \$e\ }
\setbox\contprefix=\hbox{\tt \ \ \ \ \ \ \ \ \ \ }
\startm
\m{\vdash}\m{(}\m{x}\m{=}\m{A}\m{\rightarrow}\m{(}\m{\varphi}\m{\leftrightarrow}
\m{\psi}\m{)}\m{)}
\endm
\vskip 1ex

is frequently used in place of explicit substitution, meaning ``assume
$\psi$ results from the proper substitution of $A$ for $x$ in
$\varphi$.''  Sometimes ``\texttt{\$e} $\vdash ( \psi \rightarrow
\forall x \, \psi )$'' is used instead of ``\texttt{\$d} $x\, \psi $,''
which requires only that $x$ be effectively not free in $\varphi$ but
not necessarily absent from it.  The use of implicit
substitution\index{substitution!implicit} is partly a matter of personal
style, although it may make proofs somewhat shorter than would be the
case with explicit substitution.

\item
The hypothesis


\vskip 1ex
\setbox\startprefix=\hbox{\tt \ \ \ \ \ \ \ \ \ \$e\ }
\setbox\contprefix=\hbox{\tt \ \ \ \ \ \ \ \ \ \ }
\startm
\m{\vdash}\m{A}\m{\in}\m{{\rm V}}
\endm
\vskip 1ex

should be read ``assume class $A$ is a set (i.e.\ exists).''
This is a convenient convention used by Quine.

\item
The restriction and hypothesis

\vskip 1ex
\setbox\startprefix=\hbox{\tt \ \ \ \ \ \ \ \ \ \$d\ }
\setbox\contprefix=\hbox{\tt \ \ \ \ \ \ \ \ \ \ }
\startm
\m{x}\m{\,y}
\endm
\setbox\startprefix=\hbox{\tt \ \ \ \ \ \ \ \ \ \$e\ }
\setbox\contprefix=\hbox{\tt \ \ \ \ \ \ \ \ \ \ }
\startm
\m{\vdash}\m{(}\m{y}\m{\in}\m{A}\m{\rightarrow}\m{\forall}\m{x}\m{\,y}
\m{\in}\m{A}\m{)}
\endm
\vskip 1ex

should be read ``assume variable $x$ is
(effectively) not free in class $A$.''

\end{itemize}

\section{A Theorem Sampler}\label{sometheorems}

In this section we list some of the more important theorems that are proved in
the \texttt{set.mm} database, and they illustrate the kinds of things that can be
done with Metamath.  While all of these facts are well-known results,
Metamath offers the advantage of easily allowing you to trace their
derivation back to axioms.  Our intent here is not to try to explain the
details or motivation; for this we refer you to the textbooks that are
mentioned in the descriptions.  (The \texttt{set.mm} file has bibliographic
references for the text references.)  Their proofs often embody important
concepts you may wish to explore with the Metamath program (see
Section~\ref{exploring}).  All the symbols that are used here are defined in
Section~\ref{hierarchy}.  For brevity we haven't included the \texttt{\$d}
restrictions or \texttt{\$f} hypotheses for these theorems; when you are
uncertain consult the \texttt{set.mm} database.

We start with \texttt{syl} (principle of the syllogism).
In \textit{Principia Mathematica}
Whitehead and Russell call this ``the principle of the
syllogism... because... the syllogism in Barbara is derived from them''
\cite[quote after Theorem *2.06 p.~101]{PM}.
Some authors call this law a ``hypothetical syllogism.''
At the time of this writing it is the most commonly referenced assertion in the
\texttt{set.mm} database.\footnote{
The Metamath program command \texttt{show usage}
shows the number of references.
}

\vskip 2ex
\noindent Theorem syl (principle of the syllogism)\index{Syllogism}%
\index{\texttt{syl}}\label{syl}.

\vskip 0.5ex
\setbox\startprefix=\hbox{\tt \ \ syl.1\ \$e\ }
\setbox\contprefix=\hbox{\tt \ \ \ \ \ \ \ \ \ \ \ }
\startm
\m{\vdash}\m{(}\m{\varphi}\m{ \rightarrow }\m{\psi}\m{)}
\endm
\setbox\startprefix=\hbox{\tt \ \ syl.2\ \$e\ }
\setbox\contprefix=\hbox{\tt \ \ \ \ \ \ \ \ \ \ \ }
\startm
\m{\vdash}\m{(}\m{\psi}\m{ \rightarrow }\m{\chi}\m{)}
\endm
\setbox\startprefix=\hbox{\tt \ \ syl\ \$p\ }
\setbox\contprefix=\hbox{\tt \ \ \ \ \ \ \ \ \ }
\startm
\m{\vdash}\m{(}\m{\varphi}\m{ \rightarrow }\m{\chi}\m{)}
\endm
\vskip 2ex

The following theorem is not very deep but provides us with a notational device
that is frequently used.  It allows us to use the expression ``$A \in V$'' as
a compact way of saying that class $A$ exists, i.e.\ is a set.

\vskip 2ex
\noindent Two ways to say ``$A$ is a set'':  $A$ is a member of the universe
$V$ if and only if $A$ exists (i.e.\ there exists a set equal to $A$).
Theorem 6.9 of Quine, p. 43.

\vskip 0.5ex
\setbox\startprefix=\hbox{\tt \ \ isset\ \$p\ }
\setbox\contprefix=\hbox{\tt \ \ \ \ \ \ \ \ \ \ \ }
\startm
\m{\vdash}\m{(}\m{A}\m{\in}\m{{\rm V}}\m{\leftrightarrow}\m{\exists}\m{x}\m{\,x}\m{=}
\m{A}\m{)}
\endm
\vskip 1ex

Next we prove the axioms of standard ZF set theory that were missing from our
axiom system.  From our point of view they are theorems since they
can be derived from the other axioms.

\vskip 2ex
\noindent Axiom of Separation\index{Axiom of Separation}
(Aussonderung)\index{Aussonderung} proved from the other axioms of ZF set
theory.  Compare Exercise 4 of Takeuti and Zaring, p.~22.

\vskip 0.5ex
\setbox\startprefix=\hbox{\tt \ \ inex1.1\ \$e\ }
\setbox\contprefix=\hbox{\tt \ \ \ \ \ \ \ \ \ \ \ \ \ \ \ }
\startm
\m{\vdash}\m{A}\m{\in}\m{{\rm V}}
\endm
\setbox\startprefix=\hbox{\tt \ \ inex\ \$p\ }
\setbox\contprefix=\hbox{\tt \ \ \ \ \ \ \ \ \ \ \ \ \ }
\startm
\m{\vdash}\m{(}\m{A}\m{\cap}\m{B}\m{)}\m{\in}\m{{\rm V}}
\endm
\vskip 1ex

\noindent Axiom of the Null Set\index{Axiom of the Null Set} proved from the
other axioms of ZF set theory. Corollary 5.16 of Takeuti and Zaring, p.~20.

\vskip 0.5ex
\setbox\startprefix=\hbox{\tt \ \ 0ex\ \$p\ }
\setbox\contprefix=\hbox{\tt \ \ \ \ \ \ \ \ \ \ \ \ }
\startm
\m{\vdash}\m{\varnothing}\m{\in}\m{{\rm V}}
\endm
\vskip 1ex

\noindent The Axiom of Pairing\index{Axiom of Pairing} proved from the other
axioms of ZF set theory.  Theorem 7.13 of Quine, p.~51.
\vskip 0.5ex
\setbox\startprefix=\hbox{\tt \ \ prex\ \$p\ }
\setbox\contprefix=\hbox{\tt \ \ \ \ \ \ \ \ \ \ \ \ \ \ }
\startm
\m{\vdash}\m{\{}\m{A}\m{,}\m{B}\m{\}}\m{\in}\m{{\rm V}}
\endm
\vskip 2ex

Next we will list some famous or important theorems that are proved in
the \texttt{set.mm} database.  None of them except \texttt{omex}
require the Axiom of Infinity, as you can verify with the \texttt{show
trace{\char`\_}back} Metamath command.

\vskip 2ex
\noindent The resolution of Russell's paradox\index{Russell's paradox}.  There
exists no set containing the set of all sets which are not members of
themselves.  Proposition 4.14 of Takeuti and Zaring, p.~14.

\vskip 0.5ex
\setbox\startprefix=\hbox{\tt \ \ ru\ \$p\ }
\setbox\contprefix=\hbox{\tt \ \ \ \ \ \ \ \ }
\startm
\m{\vdash}\m{\lnot}\m{\exists}\m{x}\m{\,x}\m{=}\m{\{}\m{y}\m{|}\m{\lnot}\m{y}
\m{\in}\m{y}\m{\}}
\endm
\vskip 1ex

\noindent Cantor's theorem\index{Cantor's theorem}.  No set can be mapped onto
its power set.  Compare Theorem 6B(b) of Enderton, p.~132.

\vskip 0.5ex
\setbox\startprefix=\hbox{\tt \ \ canth.1\ \$e\ }
\setbox\contprefix=\hbox{\tt \ \ \ \ \ \ \ \ \ \ \ \ \ }
\startm
\m{\vdash}\m{A}\m{\in}\m{{\rm V}}
\endm
\setbox\startprefix=\hbox{\tt \ \ canth\ \$p\ }
\setbox\contprefix=\hbox{\tt \ \ \ \ \ \ \ \ \ \ \ }
\startm
\m{\vdash}\m{\lnot}\m{F}\m{:}\m{A}\m{\raisebox{.5ex}{${\textstyle{\:}_{
\mbox{\footnotesize\rm {\ }}}}\atop{\textstyle{\longrightarrow}\atop{
\textstyle{}^{\mbox{\footnotesize\rm onto}}}}$}}\m{{\cal P}}\m{A}
\endm
\vskip 1ex

\noindent The Burali-Forti paradox\index{Burali-Forti paradox}.  No set
contains all ordinal numbers. Enderton, p.~194.  (Burali-Forti was one person,
not two.)

\vskip 0.5ex
\setbox\startprefix=\hbox{\tt \ \ onprc\ \$p\ }
\setbox\contprefix=\hbox{\tt \ \ \ \ \ \ \ \ \ \ \ \ }
\startm
\m{\vdash}\m{\lnot}\m{\mbox{\rm On}}\m{\in}\m{{\rm V}}
\endm
\vskip 1ex

\noindent Peano's postulates\index{Peano's postulates} for arithmetic.
Proposition 7.30 of Takeuti and Zaring, pp.~42--43.  The objects being
described are the members of $\omega$ i.e.\ the natural numbers 0, 1,
2,\ldots.  The successor\index{successor} operation suc means ``plus
one.''  \texttt{peano1} says that 0 (which is defined as the empty set)
is a natural number.  \texttt{peano2} says that if $A$ is a natural
number, so is $A+1$.  \texttt{peano3} says that 0 is not the successor
of any natural number.  \texttt{peano4} says that two natural numbers
are equal if and only if their successors are equal.  \texttt{peano5} is
essentially the same as mathematical induction.

\vskip 1ex
\setbox\startprefix=\hbox{\tt \ \ peano1\ \$p\ }
\setbox\contprefix=\hbox{\tt \ \ \ \ \ \ \ \ \ \ \ \ }
\startm
\m{\vdash}\m{\varnothing}\m{\in}\m{\omega}
\endm
\vskip 1.5ex

\setbox\startprefix=\hbox{\tt \ \ peano2\ \$p\ }
\setbox\contprefix=\hbox{\tt \ \ \ \ \ \ \ \ \ \ \ \ }
\startm
\m{\vdash}\m{(}\m{A}\m{\in}\m{\omega}\m{\rightarrow}\m{{\rm suc}}\m{A}\m{\in}%
\m{\omega}\m{)}
\endm
\vskip 1.5ex

\setbox\startprefix=\hbox{\tt \ \ peano3\ \$p\ }
\setbox\contprefix=\hbox{\tt \ \ \ \ \ \ \ \ \ \ \ \ }
\startm
\m{\vdash}\m{(}\m{A}\m{\in}\m{\omega}\m{\rightarrow}\m{\lnot}\m{{\rm suc}}%
\m{A}\m{=}\m{\varnothing}\m{)}
\endm
\vskip 1.5ex

\setbox\startprefix=\hbox{\tt \ \ peano4\ \$p\ }
\setbox\contprefix=\hbox{\tt \ \ \ \ \ \ \ \ \ \ \ \ }
\startm
\m{\vdash}\m{(}\m{(}\m{A}\m{\in}\m{\omega}\m{\wedge}\m{B}\m{\in}\m{\omega}%
\m{)}\m{\rightarrow}\m{(}\m{{\rm suc}}\m{A}\m{=}\m{{\rm suc}}\m{B}\m{%
\leftrightarrow}\m{A}\m{=}\m{B}\m{)}\m{)}
\endm
\vskip 1.5ex

\setbox\startprefix=\hbox{\tt \ \ peano5\ \$p\ }
\setbox\contprefix=\hbox{\tt \ \ \ \ \ \ \ \ \ \ \ \ }
\startm
\m{\vdash}\m{(}\m{(}\m{\varnothing}\m{\in}\m{A}\m{\wedge}\m{\forall}\m{x}\m{%
\in}\m{\omega}\m{(}\m{x}\m{\in}\m{A}\m{\rightarrow}\m{{\rm suc}}\m{x}\m{\in}%
\m{A}\m{)}\m{)}\m{\rightarrow}\m{\omega}\m{\subseteq}\m{A}\m{)}
\endm
\vskip 1.5ex

\noindent Finite Induction (mathematical induction).\index{finite
induction}\index{mathematical induction} The first hypothesis is the
basis and the second is the induction hypothesis.  Theorem Schema 22 of
Suppes, p.~136.

\vskip 0.5ex
\setbox\startprefix=\hbox{\tt \ \ findes.1\ \$e\ }
\setbox\contprefix=\hbox{\tt \ \ \ \ \ \ \ \ \ \ \ \ \ \ }
\startm
\m{\vdash}\m{[}\m{\varnothing}\m{/}\m{x}\m{]}\m{\varphi}
\endm
\setbox\startprefix=\hbox{\tt \ \ findes.2\ \$e\ }
\setbox\contprefix=\hbox{\tt \ \ \ \ \ \ \ \ \ \ \ \ \ \ }
\startm
\m{\vdash}\m{(}\m{x}\m{\in}\m{\omega}\m{\rightarrow}\m{(}\m{\varphi}\m{%
\rightarrow}\m{[}\m{{\rm suc}}\m{x}\m{/}\m{x}\m{]}\m{\varphi}\m{)}\m{)}
\endm
\setbox\startprefix=\hbox{\tt \ \ findes\ \$p\ }
\setbox\contprefix=\hbox{\tt \ \ \ \ \ \ \ \ \ \ \ \ }
\startm
\m{\vdash}\m{(}\m{x}\m{\in}\m{\omega}\m{\rightarrow}\m{\varphi}\m{)}
\endm
\vskip 1ex

\noindent Transfinite Induction with explicit substitution.  The first
hypothesis is the basis, the second is the induction hypothesis for
successors, and the third is the induction hypothesis for limit
ordinals.  Theorem Schema 4 of Suppes, p. 197.

\vskip 0.5ex
\setbox\startprefix=\hbox{\tt \ \ tfindes.1\ \$e\ }
\setbox\contprefix=\hbox{\tt \ \ \ \ \ \ \ \ \ \ \ \ \ \ \ }
\startm
\m{\vdash}\m{[}\m{\varnothing}\m{/}\m{x}\m{]}\m{\varphi}
\endm
\setbox\startprefix=\hbox{\tt \ \ tfindes.2\ \$e\ }
\setbox\contprefix=\hbox{\tt \ \ \ \ \ \ \ \ \ \ \ \ \ \ \ }
\startm
\m{\vdash}\m{(}\m{x}\m{\in}\m{{\rm On}}\m{\rightarrow}\m{(}\m{\varphi}\m{%
\rightarrow}\m{[}\m{{\rm suc}}\m{x}\m{/}\m{x}\m{]}\m{\varphi}\m{)}\m{)}
\endm
\setbox\startprefix=\hbox{\tt \ \ tfindes.3\ \$e\ }
\setbox\contprefix=\hbox{\tt \ \ \ \ \ \ \ \ \ \ \ \ \ \ \ }
\startm
\m{\vdash}\m{(}\m{{\rm Lim}}\m{y}\m{\rightarrow}\m{(}\m{\forall}\m{x}\m{\in}%
\m{y}\m{\varphi}\m{\rightarrow}\m{[}\m{y}\m{/}\m{x}\m{]}\m{\varphi}\m{)}\m{)}
\endm
\setbox\startprefix=\hbox{\tt \ \ tfindes\ \$p\ }
\setbox\contprefix=\hbox{\tt \ \ \ \ \ \ \ \ \ \ \ \ \ }
\startm
\m{\vdash}\m{(}\m{x}\m{\in}\m{{\rm On}}\m{\rightarrow}\m{\varphi}\m{)}
\endm
\vskip 1ex

\noindent Principle of Transfinite Recursion.\index{transfinite
recursion} Theorem 7.41 of Takeuti and Zaring, p.~47.  Transfinite
recursion is the key theorem that allows arithmetic of ordinals to be
rigorously defined, and has many other important uses as well.
Hypotheses \texttt{tfr.1} and \texttt{tfr.2} specify a certain (proper)
class $ F$.  The complicated definition of $ F$ is not important in
itself; what is important is that there be such an $ F$ with the
required properties, and we show this by displaying $ F$ explicitly.
\texttt{tfr1} states that $ F$ is a function whose domain is the set of
ordinal numbers.  \texttt{tfr2} states that any value of $ F$ is
completely determined by its previous values and the values of an
auxiliary function, $G$.  \texttt{tfr3} states that $ F$ is unique,
i.e.\ it is the only function that satisfies \texttt{tfr1} and
\texttt{tfr2}.  Note that $ f$ is an individual variable like $x$ and
$y$; it is just a mnemonic to remind us that $A$ is a collection of
functions.

\vskip 0.5ex
\setbox\startprefix=\hbox{\tt \ \ tfr.1\ \$e\ }
\setbox\contprefix=\hbox{\tt \ \ \ \ \ \ \ \ \ \ \ }
\startm
\m{\vdash}\m{A}\m{=}\m{\{}\m{f}\m{|}\m{\exists}\m{x}\m{\in}\m{{\rm On}}\m{(}%
\m{f}\m{{\rm Fn}}\m{x}\m{\wedge}\m{\forall}\m{y}\m{\in}\m{x}\m{(}\m{f}\m{`}%
\m{y}\m{)}\m{=}\m{(}\m{G}\m{`}\m{(}\m{f}\m{\restriction}\m{y}\m{)}\m{)}\m{)}%
\m{\}}
\endm
\setbox\startprefix=\hbox{\tt \ \ tfr.2\ \$e\ }
\setbox\contprefix=\hbox{\tt \ \ \ \ \ \ \ \ \ \ \ }
\startm
\m{\vdash}\m{F}\m{=}\m{\bigcup}\m{A}
\endm
\setbox\startprefix=\hbox{\tt \ \ tfr1\ \$p\ }
\setbox\contprefix=\hbox{\tt \ \ \ \ \ \ \ \ \ \ }
\startm
\m{\vdash}\m{F}\m{{\rm Fn}}\m{{\rm On}}
\endm
\setbox\startprefix=\hbox{\tt \ \ tfr2\ \$p\ }
\setbox\contprefix=\hbox{\tt \ \ \ \ \ \ \ \ \ \ }
\startm
\m{\vdash}\m{(}\m{z}\m{\in}\m{{\rm On}}\m{\rightarrow}\m{(}\m{F}\m{`}\m{z}%
\m{)}\m{=}\m{(}\m{G}\m{`}\m{(}\m{F}\m{\restriction}\m{z}\m{)}\m{)}\m{)}
\endm
\setbox\startprefix=\hbox{\tt \ \ tfr3\ \$p\ }
\setbox\contprefix=\hbox{\tt \ \ \ \ \ \ \ \ \ \ }
\startm
\m{\vdash}\m{(}\m{(}\m{B}\m{{\rm Fn}}\m{{\rm On}}\m{\wedge}\m{\forall}\m{x}\m{%
\in}\m{{\rm On}}\m{(}\m{B}\m{`}\m{x}\m{)}\m{=}\m{(}\m{G}\m{`}\m{(}\m{B}\m{%
\restriction}\m{x}\m{)}\m{)}\m{)}\m{\rightarrow}\m{B}\m{=}\m{F}\m{)}
\endm
\vskip 1ex

\noindent The existence of omega (the class of natural numbers).\index{natural
number}\index{omega ($\omega$)}\index{Axiom of Infinity}  Axiom 7 of Takeuti
and Zaring, p.~43.  (This is the only theorem in this section requiring the
Axiom of Infinity.)

\vskip 0.5ex
\setbox\startprefix=\hbox{\tt \
\ omex\ \$p\ }
\setbox\contprefix=\hbox{\tt \ \ \ \ \ \ \ \ \ \ }
\startm
\m{\vdash}\m{\omega}\m{\in}\m{{\rm V}}
\endm
%\vskip 2ex


\section{Axioms for Real and Complex Numbers}\label{real}
\index{real and complex numbers!axioms for}

This section presents the axioms for real and complex numbers.  Analysis
textbooks implicitly or explicitly use these axioms or their equivalents
as their starting point.  In the database \texttt{set.mm}, we define real
and complex numbers as (rather complicated) specific sets and derive these
axioms as {\em theorems} from the axioms of ZF set theory, using a method
called Dedekind cuts.  We omit the details of this construction, which you can
follow if you wish using the \texttt{set.mm} database in conjunction with the
textbooks referenced therein.

Once we prove those theorems, we then restate these proven theorems as axioms.
This lets us easily identify which axioms are needed for a particular complex number proof, without the obfuscation of the set theory used to derive them.
As a result,
the construction is actually unimportant other
than to show that sets exist that satisfy the axioms, and thus that the axioms
are consistent if ZF set theory is consistent.  When working with real numbers
you can think of them as being the actual sets resulting
from the construction (for definiteness), or you can
think of them as otherwise unspecified sets that happen to satisfy the axioms.
The derivation is not easy, but the fact that it works is quite remarkable
and lends support to the idea that ZFC set theory is all we need to
provide a foundation for essentially all of mathematics.

For the axioms we are given (or postulate) 8 classes:  $\mathbb{C}$ (the
set of complex numbers), $\mathbb{R}$ (the set of real numbers, a subset
of $\mathbb{C}$), $0$ (zero), $1$ (one), $i$ (square root of
$-1$), $+$ (plus), $\cdot$ (times), and
$<_{\mathbb{R}}$ (less than for just the real numbers).
Subtraction and division are defined terms and are not part of the
axioms; for their definitions see \texttt{set.mm}.

Note that the notation $(A+B)$ (and similarly $(A\cdot B)$) specifies a class
called an {\em operation},\index{operation} and is the function value of the
class $+$ at ordered pair $\langle A,B \rangle$.  An operation is defined by
statement \texttt{df-opr} on p.~\pageref{dfopr}.
The notation $A <_{\mathbb{R}} B$ specifies a
wff called a {\em binary relation}\index{binary relation} and means $\langle A,B \rangle \in \,<_{\mathbb{R}}$, as defined by statement \texttt{df-br} on p.~\pageref{dfbr}.

Our set of 8 given classes is assumed to satisfy the following 22 axioms
(in the axioms listed below, $<$ really means $<_{\mathbb{R}}$).

\vskip 2ex

\noindent 1. The real numbers are a subset of the complex numbers.

%\vskip 0.5ex
\setbox\startprefix=\hbox{\tt \ \ ax-resscn\ \$p\ }
\setbox\contprefix=\hbox{\tt \ \ \ \ \ \ \ \ \ \ \ \ \ \ }
\startm
\m{\vdash}\m{\mathbb{R}}\m{\subseteq}\m{\mathbb{C}}
\endm
%\vskip 1ex

\noindent 2. One is a complex number.

%\vskip 0.5ex
\setbox\startprefix=\hbox{\tt \ \ ax-1cn\ \$p\ }
\setbox\contprefix=\hbox{\tt \ \ \ \ \ \ \ \ \ \ \ }
\startm
\m{\vdash}\m{1}\m{\in}\m{\mathbb{C}}
\endm
%\vskip 1ex

\noindent 3. The imaginary number $i$ is a complex number.

%\vskip 0.5ex
\setbox\startprefix=\hbox{\tt \ \ ax-icn\ \$p\ }
\setbox\contprefix=\hbox{\tt \ \ \ \ \ \ \ \ \ \ \ }
\startm
\m{\vdash}\m{i}\m{\in}\m{\mathbb{C}}
\endm
%\vskip 1ex

\noindent 4. Complex numbers are closed under addition.

%\vskip 0.5ex
\setbox\startprefix=\hbox{\tt \ \ ax-addcl\ \$p\ }
\setbox\contprefix=\hbox{\tt \ \ \ \ \ \ \ \ \ \ \ \ \ }
\startm
\m{\vdash}\m{(}\m{(}\m{A}\m{\in}\m{\mathbb{C}}\m{\wedge}\m{B}\m{\in}\m{\mathbb{C}}%
\m{)}\m{\rightarrow}\m{(}\m{A}\m{+}\m{B}\m{)}\m{\in}\m{\mathbb{C}}\m{)}
\endm
%\vskip 1ex

\noindent 5. Real numbers are closed under addition.

%\vskip 0.5ex
\setbox\startprefix=\hbox{\tt \ \ ax-addrcl\ \$p\ }
\setbox\contprefix=\hbox{\tt \ \ \ \ \ \ \ \ \ \ \ \ \ \ }
\startm
\m{\vdash}\m{(}\m{(}\m{A}\m{\in}\m{\mathbb{R}}\m{\wedge}\m{B}\m{\in}\m{\mathbb{R}}%
\m{)}\m{\rightarrow}\m{(}\m{A}\m{+}\m{B}\m{)}\m{\in}\m{\mathbb{R}}\m{)}
\endm
%\vskip 1ex

\noindent 6. Complex numbers are closed under multiplication.

%\vskip 0.5ex
\setbox\startprefix=\hbox{\tt \ \ ax-mulcl\ \$p\ }
\setbox\contprefix=\hbox{\tt \ \ \ \ \ \ \ \ \ \ \ \ \ }
\startm
\m{\vdash}\m{(}\m{(}\m{A}\m{\in}\m{\mathbb{C}}\m{\wedge}\m{B}\m{\in}\m{\mathbb{C}}%
\m{)}\m{\rightarrow}\m{(}\m{A}\m{\cdot}\m{B}\m{)}\m{\in}\m{\mathbb{C}}\m{)}
\endm
%\vskip 1ex

\noindent 7. Real numbers are closed under multiplication.

%\vskip 0.5ex
\setbox\startprefix=\hbox{\tt \ \ ax-mulrcl\ \$p\ }
\setbox\contprefix=\hbox{\tt \ \ \ \ \ \ \ \ \ \ \ \ \ \ }
\startm
\m{\vdash}\m{(}\m{(}\m{A}\m{\in}\m{\mathbb{R}}\m{\wedge}\m{B}\m{\in}\m{\mathbb{R}}%
\m{)}\m{\rightarrow}\m{(}\m{A}\m{\cdot}\m{B}\m{)}\m{\in}\m{\mathbb{R}}\m{)}
\endm
%\vskip 1ex

\noindent 8. Multiplication of complex numbers is commutative.

%\vskip 0.5ex
\setbox\startprefix=\hbox{\tt \ \ ax-mulcom\ \$p\ }
\setbox\contprefix=\hbox{\tt \ \ \ \ \ \ \ \ \ \ \ \ \ \ }
\startm
\m{\vdash}\m{(}\m{(}\m{A}\m{\in}\m{\mathbb{C}}\m{\wedge}\m{B}\m{\in}\m{\mathbb{C}}%
\m{)}\m{\rightarrow}\m{(}\m{A}\m{\cdot}\m{B}\m{)}\m{=}\m{(}\m{B}\m{\cdot}\m{A}%
\m{)}\m{)}
\endm
%\vskip 1ex

\noindent 9. Addition of complex numbers is associative.

%\vskip 0.5ex
\setbox\startprefix=\hbox{\tt \ \ ax-addass\ \$p\ }
\setbox\contprefix=\hbox{\tt \ \ \ \ \ \ \ \ \ \ \ \ \ \ }
\startm
\m{\vdash}\m{(}\m{(}\m{A}\m{\in}\m{\mathbb{C}}\m{\wedge}\m{B}\m{\in}\m{\mathbb{C}}%
\m{\wedge}\m{C}\m{\in}\m{\mathbb{C}}\m{)}\m{\rightarrow}\m{(}\m{(}\m{A}\m{+}%
\m{B}\m{)}\m{+}\m{C}\m{)}\m{=}\m{(}\m{A}\m{+}\m{(}\m{B}\m{+}\m{C}\m{)}\m{)}%
\m{)}
\endm
%\vskip 1ex

\noindent 10. Multiplication of complex numbers is associative.

%\vskip 0.5ex
\setbox\startprefix=\hbox{\tt \ \ ax-mulass\ \$p\ }
\setbox\contprefix=\hbox{\tt \ \ \ \ \ \ \ \ \ \ \ \ \ \ }
\startm
\m{\vdash}\m{(}\m{(}\m{A}\m{\in}\m{\mathbb{C}}\m{\wedge}\m{B}\m{\in}\m{\mathbb{C}}%
\m{\wedge}\m{C}\m{\in}\m{\mathbb{C}}\m{)}\m{\rightarrow}\m{(}\m{(}\m{A}\m{\cdot}%
\m{B}\m{)}\m{\cdot}\m{C}\m{)}\m{=}\m{(}\m{A}\m{\cdot}\m{(}\m{B}\m{\cdot}\m{C}%
\m{)}\m{)}\m{)}
\endm
%\vskip 1ex

\noindent 11. Multiplication distributes over addition for complex numbers.

%\vskip 0.5ex
\setbox\startprefix=\hbox{\tt \ \ ax-distr\ \$p\ }
\setbox\contprefix=\hbox{\tt \ \ \ \ \ \ \ \ \ \ \ \ \ }
\startm
\m{\vdash}\m{(}\m{(}\m{A}\m{\in}\m{\mathbb{C}}\m{\wedge}\m{B}\m{\in}\m{\mathbb{C}}%
\m{\wedge}\m{C}\m{\in}\m{\mathbb{C}}\m{)}\m{\rightarrow}\m{(}\m{A}\m{\cdot}\m{(}%
\m{B}\m{+}\m{C}\m{)}\m{)}\m{=}\m{(}\m{(}\m{A}\m{\cdot}\m{B}\m{)}\m{+}\m{(}%
\m{A}\m{\cdot}\m{C}\m{)}\m{)}\m{)}
\endm
%\vskip 1ex

\noindent 12. The square of $i$ equals $-1$ (expressed as $i$-squared plus 1 is
0).

%\vskip 0.5ex
\setbox\startprefix=\hbox{\tt \ \ ax-i2m1\ \$p\ }
\setbox\contprefix=\hbox{\tt \ \ \ \ \ \ \ \ \ \ \ \ }
\startm
\m{\vdash}\m{(}\m{(}\m{i}\m{\cdot}\m{i}\m{)}\m{+}\m{1}\m{)}\m{=}\m{0}
\endm
%\vskip 1ex

\noindent 13. One and zero are distinct.

%\vskip 0.5ex
\setbox\startprefix=\hbox{\tt \ \ ax-1ne0\ \$p\ }
\setbox\contprefix=\hbox{\tt \ \ \ \ \ \ \ \ \ \ \ \ }
\startm
\m{\vdash}\m{1}\m{\ne}\m{0}
\endm
%\vskip 1ex

\noindent 14. One is an identity element for real multiplication.

%\vskip 0.5ex
\setbox\startprefix=\hbox{\tt \ \ ax-1rid\ \$p\ }
\setbox\contprefix=\hbox{\tt \ \ \ \ \ \ \ \ \ \ \ }
\startm
\m{\vdash}\m{(}\m{A}\m{\in}\m{\mathbb{R}}\m{\rightarrow}\m{(}\m{A}\m{\cdot}\m{1}%
\m{)}\m{=}\m{A}\m{)}
\endm
%\vskip 1ex

\noindent 15. Every real number has a negative.

%\vskip 0.5ex
\setbox\startprefix=\hbox{\tt \ \ ax-rnegex\ \$p\ }
\setbox\contprefix=\hbox{\tt \ \ \ \ \ \ \ \ \ \ \ \ \ \ }
\startm
\m{\vdash}\m{(}\m{A}\m{\in}\m{\mathbb{R}}\m{\rightarrow}\m{\exists}\m{x}\m{\in}%
\m{\mathbb{R}}\m{(}\m{A}\m{+}\m{x}\m{)}\m{=}\m{0}\m{)}
\endm
%\vskip 1ex

\noindent 16. Every nonzero real number has a reciprocal.

%\vskip 0.5ex
\setbox\startprefix=\hbox{\tt \ \ ax-rrecex\ \$p\ }
\setbox\contprefix=\hbox{\tt \ \ \ \ \ \ \ \ \ \ \ \ \ \ }
\startm
\m{\vdash}\m{(}\m{A}\m{\in}\m{\mathbb{R}}\m{\rightarrow}\m{(}\m{A}\m{\ne}\m{0}%
\m{\rightarrow}\m{\exists}\m{x}\m{\in}\m{\mathbb{R}}\m{(}\m{A}\m{\cdot}%
\m{x}\m{)}\m{=}\m{1}\m{)}\m{)}
\endm
%\vskip 1ex

\noindent 17. A complex number can be expressed in terms of two reals.

%\vskip 0.5ex
\setbox\startprefix=\hbox{\tt \ \ ax-cnre\ \$p\ }
\setbox\contprefix=\hbox{\tt \ \ \ \ \ \ \ \ \ \ \ \ }
\startm
\m{\vdash}\m{(}\m{A}\m{\in}\m{\mathbb{C}}\m{\rightarrow}\m{\exists}\m{x}\m{\in}%
\m{\mathbb{R}}\m{\exists}\m{y}\m{\in}\m{\mathbb{R}}\m{A}\m{=}\m{(}\m{x}\m{+}\m{(}%
\m{y}\m{\cdot}\m{i}\m{)}\m{)}\m{)}
\endm
%\vskip 1ex

\noindent 18. Ordering on reals satisfies strict trichotomy.

%\vskip 0.5ex
\setbox\startprefix=\hbox{\tt \ \ ax-pre-lttri\ \$p\ }
\setbox\contprefix=\hbox{\tt \ \ \ \ \ \ \ \ \ \ \ \ \ }
\startm
\m{\vdash}\m{(}\m{(}\m{A}\m{\in}\m{\mathbb{R}}\m{\wedge}\m{B}\m{\in}\m{\mathbb{R}}%
\m{)}\m{\rightarrow}\m{(}\m{A}\m{<}\m{B}\m{\leftrightarrow}\m{\lnot}\m{(}\m{A}%
\m{=}\m{B}\m{\vee}\m{B}\m{<}\m{A}\m{)}\m{)}\m{)}
\endm
%\vskip 1ex

\noindent 19. Ordering on reals is transitive.

%\vskip 0.5ex
\setbox\startprefix=\hbox{\tt \ \ ax-pre-lttrn\ \$p\ }
\setbox\contprefix=\hbox{\tt \ \ \ \ \ \ \ \ \ \ \ \ \ }
\startm
\m{\vdash}\m{(}\m{(}\m{A}\m{\in}\m{\mathbb{R}}\m{\wedge}\m{B}\m{\in}\m{\mathbb{R}}%
\m{\wedge}\m{C}\m{\in}\m{\mathbb{R}}\m{)}\m{\rightarrow}\m{(}\m{(}\m{A}\m{<}%
\m{B}\m{\wedge}\m{B}\m{<}\m{C}\m{)}\m{\rightarrow}\m{A}\m{<}\m{C}\m{)}\m{)}
\endm
%\vskip 1ex

\noindent 20. Ordering on reals is preserved after addition to both sides.

%\vskip 0.5ex
\setbox\startprefix=\hbox{\tt \ \ ax-pre-ltadd\ \$p\ }
\setbox\contprefix=\hbox{\tt \ \ \ \ \ \ \ \ \ \ \ \ \ }
\startm
\m{\vdash}\m{(}\m{(}\m{A}\m{\in}\m{\mathbb{R}}\m{\wedge}\m{B}\m{\in}\m{\mathbb{R}}%
\m{\wedge}\m{C}\m{\in}\m{\mathbb{R}}\m{)}\m{\rightarrow}\m{(}\m{A}\m{<}\m{B}\m{%
\rightarrow}\m{(}\m{C}\m{+}\m{A}\m{)}\m{<}\m{(}\m{C}\m{+}\m{B}\m{)}\m{)}\m{)}
\endm
%\vskip 1ex

\noindent 21. The product of two positive reals is positive.

%\vskip 0.5ex
\setbox\startprefix=\hbox{\tt \ \ ax-pre-mulgt0\ \$p\ }
\setbox\contprefix=\hbox{\tt \ \ \ \ \ \ \ \ \ \ \ \ \ \ }
\startm
\m{\vdash}\m{(}\m{(}\m{A}\m{\in}\m{\mathbb{R}}\m{\wedge}\m{B}\m{\in}\m{\mathbb{R}}%
\m{)}\m{\rightarrow}\m{(}\m{(}\m{0}\m{<}\m{A}\m{\wedge}\m{0}%
\m{<}\m{B}\m{)}\m{\rightarrow}\m{0}\m{<}\m{(}\m{A}\m{\cdot}\m{B}\m{)}%
\m{)}\m{)}
\endm
%\vskip 1ex

\noindent 22. A non-empty, bounded-above set of reals has a supremum.

%\vskip 0.5ex
\setbox\startprefix=\hbox{\tt \ \ ax-pre-sup\ \$p\ }
\setbox\contprefix=\hbox{\tt \ \ \ \ \ \ \ \ \ \ \ }
\startm
\m{\vdash}\m{(}\m{(}\m{A}\m{\subseteq}\m{\mathbb{R}}\m{\wedge}\m{A}\m{\ne}\m{%
\varnothing}\m{\wedge}\m{\exists}\m{x}\m{\in}\m{\mathbb{R}}\m{\forall}\m{y}\m{%
\in}\m{A}\m{\,y}\m{<}\m{x}\m{)}\m{\rightarrow}\m{\exists}\m{x}\m{\in}\m{%
\mathbb{R}}\m{(}\m{\forall}\m{y}\m{\in}\m{A}\m{\lnot}\m{x}\m{<}\m{y}\m{\wedge}\m{%
\forall}\m{y}\m{\in}\m{\mathbb{R}}\m{(}\m{y}\m{<}\m{x}\m{\rightarrow}\m{\exists}%
\m{z}\m{\in}\m{A}\m{\,y}\m{<}\m{z}\m{)}\m{)}\m{)}
\endm

% NOTE: The \m{...} expressions above could be represented as
% $ \vdash ( ( A \subseteq \mathbb{R} \wedge A \ne \varnothing \wedge \exists x \in \mathbb{R} \forall y \in A \,y < x ) \rightarrow \exists x \in \mathbb{R} ( \forall y \in A \lnot x < y \wedge \forall y \in \mathbb{R} ( y < x \rightarrow \exists z \in A \,y < z ) ) ) $

\vskip 2ex

This completes the set of axioms for real and complex numbers.  You may
wish to look at how subtraction, division, and decimal numbers
are defined in \texttt{set.mm}, and for fun look at the proof of $2+
2 = 4$ (theorem \texttt{2p2e4} in \texttt{set.mm})
as discussed in section \ref{2p2e4}.

In \texttt{set.mm} we define the non-negative integers $\mathbb{N}$, the integers
$\mathbb{Z}$, and the rationals $\mathbb{Q}$ as subsets of $\mathbb{R}$.  This leads
to the nice inclusion $\mathbb{N} \subseteq \mathbb{Z} \subseteq \mathbb{Q} \subseteq
\mathbb{R} \subseteq \mathbb{C}$, giving us a uniform framework in which, for
example, a property such as commutativity of complex number addition
automatically applies to integers.  The natural numbers $\mathbb{N}$
are different from the set $\omega$ we defined earlier, but both satisfy
Peano's postulates.

\subsubsection{Complex Number Axioms in Analysis Texts}

Most analysis texts construct complex numbers as ordered pairs of reals,
leading to construction-dependent properties that satisfy these axioms
but are not stated in their pure form.  (This is also done in
\texttt{set.mm} but our axioms are extracted from that construction.)
Other texts will simply state that $\mathbb{R}$ is a ``complete ordered
subfield of $\mathbb{C}$,'' leading to redundant axioms when this phrase
is completely expanded out.  In fact I have not seen a text with the
axioms in the explicit form above.
None of these axioms is unique individually, but this carefully worked out
collection of axioms is the result of years of work
by the Metamath community.

\subsubsection{Eliminating Unnecessary Complex Number Axioms}

We once had more axioms for real and complex numbers, but over years of time
we (the Metamath community)
have found ways to eliminate them (by proving them from other axioms)
or weaken them (by making weaker claims without reducing what
can be proved).
In particular, here are statements that used to be complex number
axioms but have since been formally proven (with Metamath) to be redundant:

\begin{itemize}
\item
  $\mathbb{C} \in V$.
  At one time this was listed as a ``complex number axiom''
  However, this is not properly speaking a complex number axiom,
  and in any case its proof uses axioms of set theory.
  Proven redundant by Mario Carneiro\index{Carneiro, Mario} on
  17-Nov-2014 (see \texttt{axcnex}).
\item
  $((A \in \mathbb{C} \land B \in \mathbb{C}$) $\rightarrow$
  $(A + B) = (B + A))$.
  Proved redundant by Eric Schmidt\index{Schmidt, Eric} on 19-Jun-2012,
  and formalized by Scott Fenton\index{Fenton, Scott} on 3-Jan-2013
  (see \texttt{addcom}).
\item
  $(A \in \mathbb{C} \rightarrow (A + 0) = A)$.
  Proved redundant by Eric Schmidt on 19-Jun-2012,
  and formalized by Scott Fenton on 3-Jan-2013
  (see \texttt{addid1}).
\item
  $(A \in \mathbb{C} \rightarrow \exists x \in \mathbb{C} (A + x) = 0)$.
  Proved redundant by Eric Schmidt and formalized on 21-May-2007
  (see \texttt{cnegex}).
\item
  $((A \in \mathbb{C} \land A \ne 0) \rightarrow \exists x \in \mathbb{C} (A \cdot x) = 1)$.
  Proved redundant by Eric Schmidt and formalized on 22-May-2007
  (see \texttt{recex}).
\item
  $0 \in \mathbb{R}$.
  Proved redundant by Eric Schmidt on 19-Feb-2005 and formalized 21-May-2007
  (see \texttt{0re}).
\end{itemize}

We could eliminate 0 as an axiomatic object by defining it as
$( ( i \cdot i ) + 1 )$
and replacing it with this expression throughout the axioms. If this
is done, axiom ax-i2m1 becomes redundant. However, the remaining axioms
would become longer and less intuitive.

Eric Schmidt's paper analyzing this axiom system \cite{Schmidt}
presented a proof that these remaining axioms,
with the possible exception of ax-mulcom, are independent of the others.
It is currently an open question if ax-mulcom is independent of the others.

\section{Two Plus Two Equals Four}\label{2p2e4}

Here is a proof that $2 + 2 = 4$, as proven in the theorem \texttt{2p2e4}
in the database \texttt{set.mm}.
This is a useful demonstration of what a Metamath proof can look like.
This proof may have more steps than you're used to, but each step is rigorously
proven all the way back to the axioms of logic and set theory.
This display was originally generated by the Metamath program
as an {\sc HTML} file.

In the table showing the proof ``Step'' is the sequential step number,
while its associated ``Expression'' is an expression that we have proved.
``Ref'' is the name of a theorem or axiom that justifies that expression,
and ``Hyp'' refers to previous steps (if any) that the theorem or axiom
needs so that we can use it.  Expressions are indented further than
the expressions that depend on them to show their interdependencies.

\begin{table}[!htbp]
\caption{Two plus two equals four}
\begin{tabular}{lllll}
\textbf{Step} & \textbf{Hyp} & \textbf{Ref} & \textbf{Expression} & \\
1  &       & df-2    & $ \; \; \vdash 2 = 1 + 1$  & \\
2  & 1     & oveq2i  & $ \; \vdash (2 + 2) = (2 + (1 + 1))$ & \\
3  &       & df-4    & $ \; \; \vdash 4 = (3 + 1)$ & \\
4  &       & df-3    & $ \; \; \; \vdash 3 = (2 + 1)$ & \\
5  & 4     & oveq1i  & $ \; \; \vdash (3 + 1) = ((2 + 1) + 1)$ & \\
6  &       & 2cn     & $ \; \; \; \vdash 2 \in \mathbb{C}$ & \\
7  &       & ax-1cn  & $ \; \; \; \vdash 1 \in \mathbb{C}$ & \\
8  & 6,7,7 & addassi & $ \; \; \vdash ((2 + 1) + 1) = (2 + (1 + 1))$ & \\
9  & 3,5,8 & 3eqtri  & $ \; \vdash 4 = (2 + (1 + 1))$ & \\
10 & 2,9   & eqtr4i  & $ \vdash (2 + 2) = 4$ & \\
\end{tabular}
\end{table}

Step 1 says that we can assert that $2 = 1 + 1$ because it is
justified by \texttt{df-2}.
What is \texttt{df-2}?
It is simply the definition of $2$, which in our system is defined as being
equal to $1 + 1$.  This shows how we can use definitions in proofs.

Look at Step 2 of the proof. In the Ref column, we see that it references
a previously proved theorem, \texttt{oveq2i}.
It turns out that
theorem \texttt{oveq2i} requires a
hypothesis, and in the Hyp column of Step 2 we indicate that Step 1 will
satisfy (match) this hypothesis.
If we looked at \texttt{oveq2i}
we would find that it proves that given some hypothesis
$A = B$, we can prove that $( C F A ) = ( C F B )$.
If we use \texttt{oveq2i} and apply step 1's result as the hypothesis,
that will mean that $A = 2$ and $B = ( 1 + 1 )$ within this use of
\texttt{oveq2i}.
We are free to select any value of $C$ and $F$ (subject to syntax constraints),
so we are free to select $C = 2$ and $F = +$,
producing our desired result,
$ (2 + 2) = (2 + (1 + 1))$.

Step 2 is an example of substitution.
In the end, every step in every proof uses only this one substitution rule.
All the rules of logic, and all the axioms, are expressed so that
they can be used via this one substitution rule.
So once you master substitution, you can master every Metamath proof,
no exceptions.

Each step is clear and can be immediately checked.
In the {\sc HTML} display you can even click on each reference to see why it is
justified, making it easy to see why the proof works.

\section{Deduction}\label{deduction}

Strictly speaking,
a deduction (also called an inference) is a kind of statement that needs
some hypotheses to be true in order for its conclusion to be true.
A theorem, on the other hand, has no hypotheses.
Informally we often call both of them theorems, but in this section we
will stick to the strict definitions.

It sometimes happens that we have proved a deduction of the form
$\varphi \Rightarrow \psi$\index{$\Rightarrow$}
(given hypothesis $\varphi$ we can prove $\psi$)
and we want to then prove a theorem of the form
$\varphi \rightarrow \psi$.

Converting a deduction (which uses a hypothesis) into a theorem
(which does not) is not as simple as you might think.
The deduction says, ``if we can prove $\varphi$ then we can prove $\psi$,''
which is in some sense weaker than saying
``$\varphi$ implies $\psi$.''
There is no axiom of logic that permits us to directly obtain the theorem
given the deduction.\footnote{
The conversion of a deduction to a theorem does not even hold in general
for quantum propositional calculus,
which is a weak subset of classical propositional calculus.
It has been shown that adding the Standard Deduction Theorem (discussed below)
to quantum propositional calculus turns it into classical
propositional calculus!
}

This is in contrast to going the other way.
If we have the theorem ($\varphi \rightarrow \psi$),
it is easy to recover the deduction
($\varphi \Rightarrow \psi$)
using modus ponens\index{modus ponens}
(\texttt{ax-mp}; see section \ref{axmp}).

In the following subsections we first discuss the standard deduction theorem
(the traditional but awkward way to convert deductions into theorems) and
the weak deduction theorem (a limited version of the standard deduction
theorem that is easier to use and was once widely used in
\texttt{set.mm}\index{set theory database (\texttt{set.mm})}).
In section \ref{deductionstyle} we discuss
deduction style, the newer approach we now recommend in most cases.
Deduction style uses ``deduction form,'' a form that
prefixes each hypothesis (other than definitions) and the
conclusion with a universal antecedent (``$\varphi \rightarrow$'').
Deduction style is widely used in \texttt{set.mm},
so it is useful to understand it and \textit{why} it is widely used.
Section \ref{naturaldeduction}
briefly discusses our approach for using natural deduction
within \texttt{set.mm},
as that approach is deeply related to deduction style.
We conclude with a summary of the strengths of
our approach, which we believe are compelling.

\subsection{The Standard Deduction Theorem}\label{standarddeductiontheorem}

It is possible to make use of information
contained in the deduction or its proof to assist us with the proof of
the related theorem.
In traditional logic books, there is a metatheorem called the
Deduction Theorem\index{Deduction Theorem}\index{Standard Deduction Theorem},
discovered independently by Herbrand and Tarski around 1930.
The Deduction Theorem, which we often call the Standard Deduction Theorem,
provides an algorithm for constructing a proof of a theorem from the
proof of its corresponding deduction. See, for example,
\cite[p.~56]{Margaris}\index{Margaris, Angelo}.
To construct a proof for a theorem, the
algorithm looks at each step in the proof of the original deduction and
rewrites the step with several steps wherein the hypothesis is eliminated
and becomes an antecedent.

In ordinary mathematics, no one actually carries out the algorithm,
because (in its most basic form) it involves an exponential explosion of
the number of proof steps as more hypotheses are eliminated. Instead,
the Standard Deduction Theorem is invoked simply to claim that it can
be done in principle, without actually doing it.
What's more, the algorithm is not as simple as it might first appear
when applying it rigorously.
There is a subtle restriction on the Standard Deduction Theorem
that must be taken into account involving the axiom of generalization
when working with predicate calculus (see the literature for more detail).

One of the goals of Metamath is to let you plainly see, with as few
underlying concepts as possible, how mathematics can be derived directly
from the axioms, and not indirectly according to some hidden rules
buried inside a program or understood only by logicians. If we added
the Standard Deduction Theorem to the language and proof verifier,
that would greatly complicate both and largely defeat Metamath's goal
of simplicity. In principle, we could show direct proofs by expanding
out the proof steps generated by the algorithm of the Standard Deduction
Theorem, but that is not feasible in practice because the number of proof
steps quickly becomes huge, even astronomical.
Since the algorithm of the Standard Deduction Theorem is driven by the proof,
we would have to go through that proof
all over again---starting from axioms---in order to obtain the theorem form.
In terms of proof length, there would be no savings over just
proving the theorem directly instead of first proving the deduction form.

\subsection{Weak Deduction Theorem}\label{weakdeductiontheorem}

We have developed
a more efficient method for proving a theorem from a deduction
that can be used instead of the Standard Deduction Theorem
in many (but not all) cases.
We call this more efficient method the
Weak Deduction Theorem\index{Weak Deduction Theorem}.\footnote{
There is also an unrelated ``Weak Deduction Theorem''
in the field of relevance logic, so to avoid confusion we could call
ours the ``Weak Deduction Theorem for Classical Logic.''}
Unlike the Standard Deduction Theorem, the Weak Deduction Theorem produces the
theorem directly from a special substitution instance of the deduction,
using a small, fixed number of steps roughly proportional to the length
of the final theorem.

If you come to a proof referencing the Weak Deduction Theorem
\texttt{dedth} (or one of its variants \texttt{dedthxx}),
here is how to follow the proof without getting into the details:
just click on the theorem referenced in the step
just before the reference to \texttt{dedth} and ignore everything else.
Theorem \texttt{dedth} simply turns a hypothesis into an antecedent
(i.e. the hypothesis followed by $\rightarrow$
is placed in front of the assertion, and the hypothesis
itself is eliminated) given certain conditions.

The Weak Deduction Theorem
eliminates a hypothesis $\varphi$, making it become an antecedent.
It does this by proving an expression
$ \varphi \rightarrow \psi $ given two hypotheses:
(1)
$ ( A = {\rm if} ( \varphi , A , B ) \rightarrow ( \varphi \leftrightarrow \chi ) ) $
and
(2) $\chi$.
Note that it requires that a proof exists for $\varphi$ when the class variable
$A$ is replaced with a specific class $B$. The hypothesis $\chi$
should be assigned to the inference.
You can see the details of the proof of the Weak Deduction Theorem
in theorem \texttt{dedth}.

The Weak Deduction Theorem
is probably easier to understand by studying proofs that make use of it.
For example, let's look at the proof of \texttt{renegcl}, which proves that
$ \vdash ( A \in \mathbb{R} \rightarrow - A \in \mathbb{R} )$:

\needspace{4\baselineskip}
\begin{longtabu} {l l l X}
\textbf{Step} & \textbf{Hyp} & \textbf{Ref} & \textbf{Expression} \\
  1 &  & negeq &
  $\vdash$ $($ $A$ $=$ ${\rm if}$ $($ $A$ $\in$ $\mathbb{R}$ $,$ $A$ $,$ $1$ $)$ $\rightarrow$
  $\textrm{-}$ $A$ $=$ $\textrm{-}$ ${\rm if}$ $($ $A$ $\in$ $\mathbb{R}$
  $,$ $A$ $,$ $1$ $)$ $)$ \\
 2 & 1 & eleq1d &
    $\vdash$ $($ $A$ $=$ ${\rm if}$ $($ $A$ $\in$ $\mathbb{R}$ $,$ $A$ $,$ $1$ $)$ $\rightarrow$ $($
    $\textrm{-}$ $A$ $\in$ $\mathbb{R}$ $\leftrightarrow$
    $\textrm{-}$ ${\rm if}$ $($ $A$ $\in$ $\mathbb{R}$ $,$ $A$ $,$ $1$ $)$ $\in$
    $\mathbb{R}$ $)$ $)$ \\
 3 &  & 1re & $\vdash 1 \in \mathbb{R}$ \\
 4 & 3 & elimel &
   $\vdash {\rm if} ( A \in \mathbb{R} , A , 1 ) \in \mathbb{R}$ \\
 5 & 4 & renegcli &
   $\vdash \textrm{-} {\rm if} ( A \in \mathbb{R} , A , 1 ) \in \mathbb{R}$ \\
 6 & 2,5 & dedth &
   $\vdash ( A \in \mathbb{R} \rightarrow \textrm{-} A \in \mathbb{R}$ ) \\
\end{longtabu}

The somewhat strange-looking steps in \texttt{renegcl} before step 5 are
technical stuff that makes this magic work, and they can be ignored
for a quick overview of the proof. To continue following the ``important''
part of the proof of \texttt{renegcl},
you can look at the reference to \texttt{renegcli} at step 5.

That said, let's briefly look at how
\texttt{renegcl} uses the
Weak Deduction Theorem (\texttt{dedth}) to do its job,
in case you want to do something similar or want understand it more deeply.
Let's work backwards in the proof of \texttt{renegcl}.
Step 6 applies \texttt{dedth} to produce our goal result
$ \vdash ( A \in \mathbb{R} \rightarrow\, - A \in \mathbb{R} )$.
This requires on the one hand the (substituted) deduction
\texttt{renegcli} in step 5.
By itself \texttt{renegcli} proves the deduction
$ \vdash A \in \mathbb{R} \Rightarrow\, \vdash - A \in \mathbb{R}$;
this is the deduction form we are trying to turn into theorem form,
and thus
\texttt{renegcli} has a separate hypothesis that must be fulfilled.
To fulfill the hypothesis of the invocation of
\texttt{renegcli} in step 5, it is eventually
reduced to the already proven theorem $1 \in \mathbb{R}$ in step 3.
Step 4 connects steps 3 and 5; step 4 invokes
\texttt{elimel}, a special case of \texttt{elimhyp} that eliminates
a membership hypothesis for the weak deduction theorem.
On the other hand, the equivalence of the conclusion of
\texttt{renegcl}
$( - A \in \mathbb{R} )$ and the substituted conclusion of
\texttt{renegcli} must be proven, which is done in steps 2 and 1.

The weak deduction theorem has limitations.
In particular, we must be able to prove a special case of the deduction's
hypothesis as a stand-alone theorem.
For example, we used $1 \in \mathbb{R}$ in step 3 of \texttt{renegcl}.

We used to use the weak deduction theorem
extensively within \texttt{set.mm}.
However, we now recommend applying ``deduction style''
instead in most cases, as deduction style is
often an easier and clearer approach.
Therefore, we will now describe deduction style.

\subsection{Deduction Style}\label{deductionstyle}

We now prefer to write assertions in ``deduction form''
instead of writing a proof that would require use of the standard or
weak deduction theorem.
We call this appraoch
``deduction style.''\index{deduction style}

It will be easier to explain this by first defining some terms:

\begin{itemize}
\item \textbf{closed form}\index{closed form}\index{forms!closed}:
A kind of assertion (theorem) with no hypotheses.
Typically its label has no special suffix.
An example is \texttt{unss}, which states:
$\vdash ( ( A \subseteq C \wedge B \subseteq C ) \leftrightarrow ( A \cup B )
\subseteq C )\label{eq:unss}$
\item \textbf{deduction form}\index{deduction form}\index{forms!deduction}:
A kind of assertion with one or more hypotheses
where the conclusion is an implication with
a wff variable as the antecedent (usually $\varphi$), and every hypothesis
(\$e statement)
is either (1) an implication with the same antecedent as the conclusion or
(2) a definition.
A definition
can be for a class variable (this is a class variable followed by ``='')
or a wff variable (this is a wff variable followed by $\leftrightarrow$);
class variable definitions are more common.
In practice, a proof
in deduction form will also contain many steps that are implications
where the antecedent is either that wff variable (normally $\varphi$)
or is
a conjunction (...$\land$...) including that wff variable ($\varphi$).
If an assertion is in deduction form, and other forms are also available,
then we suffix its label with ``d.''
An example is \texttt{unssd}, which states\footnote{
For brevity we show here (and in other places)
a $\&$\index{$\&$} between hypotheses\index{hypotheses}
and a $\Rightarrow$\index{$\Rightarrow$}\index{conclusion}
between the hypotheses and the conclusion.
This notation is technically not part of the Metamath language, but is
instead a convenient abbreviation to show both the hypotheses and conclusion.}:
$\vdash ( \varphi \rightarrow A \subseteq C )\quad\&\quad \vdash ( \varphi
    \rightarrow B \subseteq C )\quad\Rightarrow\quad \vdash ( \varphi
    \rightarrow ( A \cup B ) \subseteq C )\label{eq:unssd}$
\item \textbf{inference form}\index{inference form}\index{forms!inference}:
A kind of assertion with one or more hypotheses that is not in deduction form
(e.g., there is no common antecedent).
If an assertion is in inference form, and other forms are also available,
then we suffix its label with ``i.''
An example is \texttt{unssi}, which states:
$\vdash A \subseteq C\quad\&\quad \vdash B \subseteq C\quad\Rightarrow\quad
    \vdash ( A \cup B ) \subseteq C\label{eq:unssi}$
\end{itemize}

When using deduction style we express an assertion in deduction form.
This form prefixes each hypothesis (other than definitions) and the
conclusion with a universal antecedent (``$\varphi \rightarrow$'').
The antecedent (e.g., $\varphi$)
mimics the context handled in the deduction theorem, eliminating
the need to directly use the deduction theorem.

Once you have an assertion in deduction form, you can easily convert it
to inference form or closed form:

\begin{itemize}
\item To
prove some assertion Ti in inference form, given assertion Td in deduction
form, there is a simple mechanical process you can use. First take each
Ti hypothesis and insert a \texttt{T.} $\rightarrow$ prefix (``true implies'')
using \texttt{a1i}. You
can then use the existing assertion Td to prove the resulting conclusion
with a \texttt{T.} $\rightarrow$ prefix.
Finally, you can remove that prefix using \texttt{trud},
resulting in the conclusion you wanted to prove.
\item To
prove some assertion T in closed form, given assertion Td in deduction
form, there is another simple mechanical process you can use. First,
select an expression that is the conjunction (...$\land$...) of all of the
consequents of every hypothesis of Td. Next, prove that this expression
implies each of the separate hypotheses of Td in turn by eliminating
conjuncts (there are a variety of proven assertions to do this, including
\texttt{simpl},
\texttt{simpr},
\texttt{3simpa},
\texttt{3simpb},
\texttt{3simpc},
\texttt{simp1},
\texttt{simp2},
and
\texttt{simp3}).
If the
expression has nested conjunctions, inner conjuncts can be broken out by
chaining the above theorems with \texttt{syl}
(see section \ref{syl}).\footnote{
There are actually many theorems
(labeled simp* such as \texttt{simp333}) that break out inner conjuncts in one
step, but rather than learning them you can just use the chaining we
just described to prove them, and then let the Metamath program command
\texttt{minimize{\char`\_}with}\index{\texttt{minimize{\char`\_}with} command}
figure out the right ones needed to collapse them.}
As your final step, you can then apply the already-proven assertion Td
(which is in deduction form), proving assertion T in closed form.
\end{itemize}

We can also easily convert any assertion T in closed form to its related
assertion Ti in inference form by applying
modus ponens\index{modus ponens} (see section \ref{axmp}).

The deduction form antecedent can also be used to represent the context
necessary to support natural deduction systems, so we will now
discuss natural deduction.

\subsection{Natural Deduction}\label{naturaldeduction}

Natural deduction\index{natural deduction}
(ND) systems, as such, were originally introduced in
1934 by two logicians working independently: Ja\'skowski and Gentzen. ND
systems are supposed to reconstruct, in a formally proper way, traditional
ways of mathematical reasoning (such as conditional proof, indirect proof,
and proof by cases). As reconstructions they were naturally influenced
by previous work, and many specific ND systems and notations have been
developed since their original work.

There are many ND variants, but
Indrzejczak \cite[p.~31-32]{Indrzejczak}\index{Indrzejczak, Andrzej}
suggests that any natural deductive system must satisfy at
least these three criteria:

\begin{itemize}
\item ``There are some means for entering assumptions into a proof and
also for eliminating them. Usually it requires some bookkeeping devices
for indicating the scope of an assumption, and showing that a part of
a proof depending on eliminated assumption is discharged.
\item There are no (or, at least, a very limited set of) axioms, because
their role is taken over by the set of primitive rules for introduction
and elimination of logical constants which means that elementary
inferences instead of formulae are taken as primitive.
\item (A genuine) ND system admits a lot of freedom in proof construction
and possibility of applying several proof search strategies, like
conditional proof, proof by cases, proof by reductio ad absurdum etc.''
\end{itemize}

The Metamath Proof Explorer (MPE) as defined in \texttt{set.mm}
is fundamentally a Hilbert-style system.
That is, MPE is based on a larger number of axioms (compared
to natural deduction systems), a very small set of rules of inference
(modus ponens), and the context is not changed by the rules of inference
in the middle of a proof. That said, MPE proofs can be developed using
the natural deduction (ND) approach as originally developed by Ja\'skowski
and Gentzen.

The most common and recommended approach for applying ND in MPE is to use
deduction form\index{deduction form}%
\index{forms!deduction}
and apply the MPE proven assertions that are equivalent to ND rules.
For example, MPE's \texttt{jca} is equivalent to ND rule $\land$-I
(and-insertion).
We maintain a list of equivalences that you may consult.
This approach for applying an ND approach within MPE relies on Metamath's
wff metavariables in an essential way, and is described in more detail
in the presentation ``Natural Deductions in the Metamath Proof Language''
by Mario Carneiro \cite{CarneiroND}\index{Carneiro, Mario}.

In this style many steps are an implication, whose antecedent mimics
the context ($\Gamma$) of most ND systems. To add an assumption, simply add
it to the implication antecedent (typically using
\texttt{simpr}),
and use that
new antecedent for all later claims in the same scope. If you wish to
use an assertion in an ND hypothesis scope that is outside the current
ND hypothesis scope, modify the assertion so that the ND hypothesis
assumption is added to its antecedent (typically using \texttt{adantr}). Most
proof steps will be proved using rules that have hypotheses and results
of the form $\varphi \rightarrow$ ...

An example may make this clearer.
Let's show theorem 5.5 of
\cite[p.~18]{Clemente}\index{Clemente Laboreo, Daniel}
along with a line by line translation using the usual
translation of natural deduction (ND) in the Metamath Proof Explorer
(MPE) notation (this is proof \texttt{ex-natded5.5}).
The proof's original goal was to prove
$\lnot \psi$ given two hypotheses,
$( \psi \rightarrow \chi )$ and $ \lnot \chi$.
We will translate these statements into MPE deduction form
by prefixing them all with $\varphi \rightarrow$.
As a result, in MPE the goal is stated as
$( \varphi \rightarrow \lnot \psi )$, and the two hypotheses are stated as
$( \varphi \rightarrow ( \psi \rightarrow \chi ) )$ and
$( \varphi \rightarrow \lnot \chi )$.

The following table shows the proof in Fitch natural deduction style
and its MPE equivalent.
The \textit{\#} column shows the original numbering,
\textit{MPE\#} shows the number in the equivalent MPE proof
(which we will show later),
\textit{ND Expression} shows the original proof claim in ND notation,
and \textit{MPE Translation} shows its translation into MPE
as discussed in this section.
The final columns show the rationale in ND and MPE respectively.

\needspace{4\baselineskip}
{\setlength{\extrarowsep}{4pt} % Keep rows from being too close together
\begin{longtabu}   { @{} c c X X X X }
\textbf{\#} & \textbf{MPE\#} & \textbf{ND Ex\-pres\-sion} &
\textbf{MPE Trans\-lation} & \textbf{ND Ration\-ale} &
\textbf{MPE Ra\-tio\-nale} \\
\endhead

1 & 2;3 &
$( \psi \rightarrow \chi )$ &
$( \varphi \rightarrow ( \psi \rightarrow \chi ) )$ &
Given &
\$e; \texttt{adantr} to put in ND hypothesis \\

2 & 5 &
$ \lnot \chi$ &
$( \varphi \rightarrow \lnot \chi )$ &
Given &
\$e; \texttt{adantr} to put in ND hypothesis \\

3 & 1 &
... $\vert$ $\psi$ &
$( \varphi \rightarrow \psi )$ &
ND hypothesis assumption &
\texttt{simpr} \\

4 & 4 &
... $\chi$ &
$( ( \varphi \land \psi ) \rightarrow \chi )$ &
$\rightarrow$\,E 1,3 &
\texttt{mpd} 1,3 \\

5 & 6 &
... $\lnot \chi$ &
$( ( \varphi \land \psi ) \rightarrow \lnot \chi )$ &
IT 2 &
\texttt{adantr} 5 \\

6 & 7 &
$\lnot \psi$ &
$( \varphi \rightarrow \lnot \psi )$ &
$\land$\,I 3,4,5 &
\texttt{pm2.65da} 4,6 \\

\end{longtabu}
}


The original used Latin letters; we have replaced them with Greek letters
to follow Metamath naming conventions and so that it is easier to follow
the Metamath translation. The Metamath line-for-line translation of
this natural deduction approach precedes every line with an antecedent
including $\varphi$ and uses the Metamath equivalents of the natural deduction
rules. To add an assumption, the antecedent is modified to include it
(typically by using \texttt{adantr};
\texttt{simpr} is useful when you want to
depend directly on the new assumption, as is shown here).

In Metamath we can represent the two given statements as these hypotheses:

\needspace{2\baselineskip}
\begin{itemize}
\item ex-natded5.5.1 $\vdash ( \varphi \rightarrow ( \psi \rightarrow \chi ) )$
\item ex-natded5.5.2 $\vdash ( \varphi \rightarrow \lnot \chi )$
\end{itemize}

\needspace{4\baselineskip}
Here is the proof in Metamath as a line-by-line translation:

\begin{longtabu}   { l l l X }
\textbf{Step} & \textbf{Hyp} & \textbf{Ref} & \textbf{Ex\-pres\-sion} \\
\endhead
1 & & simpr & $\vdash ( ( \varphi \land \psi ) \rightarrow \psi )$ \\
2 & & ex-natded5.5.1 &
  $\vdash ( \varphi \rightarrow ( \psi \rightarrow \chi ) )$ \\
3 & 2 & adantr &
 $\vdash ( ( \varphi \land \psi ) \rightarrow ( \psi \rightarrow \chi ) )$ \\
4 & 1, 3 & mpd &
 $\vdash ( ( \varphi \land \psi ) \rightarrow \chi ) $ \\
5 & & ex-natded5.5.2 &
 $\vdash ( \varphi \rightarrow \lnot \chi )$ \\
6 & 5 & adantr &
 $\vdash ( ( \varphi \land \psi ) \rightarrow \lnot \chi )$ \\
7 & 4, 6 & pm2.65da &
 $\vdash ( \varphi \rightarrow \lnot \psi )$ \\
\end{longtabu}

Only using specific natural deduction rules directly can lead to very
long proofs, for exactly the same reason that only using axioms directly
in Hilbert-style proofs can lead to very long proofs.
If the goal is short and clear proofs,
then it is better to reuse already-proven assertions
in deduction form than to start from scratch each time
and using only basic natural deduction rules.

\subsection{Strengths of Our Approach}

As far as we know there is nothing else in the literature like either the
weak deduction theorem or Mario Carneiro\index{Carneiro, Mario}'s
natural deduction method.
In order to
transform a hypothesis into an antecedent, the literature's standard
``Deduction Theorem''\index{Deduction Theorem}\index{Standard Deduction Theorem}
requires metalogic outside of the notions provided
by the axiom system. We instead generally prefer to use Mario Carneiro's
natural deduction method, then use the weak deduction theorem in cases
where that is difficult to apply, and only then use the full standard
deduction theorem as a last resort.

The weak deduction theorem\index{Weak Deduction Theorem}
does not require any additional metalogic
but converts an inference directly into a closed form theorem, with
a rigorous proof that uses only the axiom system. Unlike the standard
Deduction Theorem, there is no implicit external justification that we
have to trust in order to use it.

Mario Carneiro's natural deduction\index{natural deduction}
method also does not require any new metalogical
notions. It avoids the Deduction Theorem's metalogic by prefixing the
hypotheses and conclusion of every would-be inference with a universal
antecedent (``$\varphi \rightarrow$'') from the very start.

We think it is impressive and satisfying that we can do so much in a
practical sense without stepping outside of our Hilbert-style axiom system.
Of course our axiomatization, which is in the form of schemes,
contains a metalogic of its own that we exploit. But this metalogic
is relatively simple, and for our Deduction Theorem alternatives,
we primarily use just the direct substitution of expressions for
metavariables.

\begin{sloppy}
\section{Exploring the Set The\-o\-ry Data\-base}\label{exploring}
\end{sloppy}
% NOTE: All examples performed in this section are
% recorded wtih "set width 61" % on set.mm as of 2019-05-28
% commit c1e7849557661260f77cfdf0f97ac4354fbb4f4d.

At this point you may wish to study the \texttt{set.mm}\index{set theory
database (\texttt{set.mm})} file in more detail.  Pay particular
attention to the assumptions needed to define wffs\index{well-formed
formula (wff)} (which are not included above), the variable types
(\texttt{\$f}\index{\texttt{\$f} statement} statements), and the
definitions that are introduced.  Start with some simple theorems in
propositional calculus, making sure you understand in detail each step
of a proof.  Once you get past the first few proofs and become familiar
with the Metamath language, any part of the \texttt{set.mm} database
will be as easy to follow, step by step, as any other part---you won't
have to undergo a ``quantum leap'' in mathematical sophistication to be
able to follow a deep proof in set theory.

Next, you may want to explore how concepts such as natural numbers are
defined and described.  This is probably best done in conjunction with
standard set theory textbooks, which can help give you a higher-level
understanding.  The \texttt{set.mm} database provides references that will get
you started.  From there, you will be on your way towards a very deep,
rigorous understanding of abstract mathematics.

The Metamath\index{Metamath} program can help you peruse a Metamath data\-base,
wheth\-er you are trying to figure out how a certain step follows in a proof or
just have a general curiosity.  We will go through some examples of the
commands, using the \texttt{set.mm}\index{set theory database (\texttt{set.mm})}
database provided with the Metamath software.  These should help get you
started.  See Chapter~\ref{commands} for a more detailed description of
the commands.  Note that we have included the full spelling of all commands to
prevent ambiguity with future commands.  In practice you may type just the
characters needed to specify each command keyword\index{command keyword}
unambiguously, often just one or two characters per keyword, and you don't
need to type them in upper case.

First run the Metamath program as described earlier.  You should see the
\verb/MM>/ prompt.  Read in the \texttt{set.mm} file:\index{\texttt{read}
command}

\begin{verbatim}
MM> read set.mm
Reading source file "set.mm"... 34554442 bytes
34554442 bytes were read into the source buffer.
The source has 155711 statements; 2254 are $a and 32250 are $p.
No errors were found.  However, proofs were not checked.
Type VERIFY PROOF * if you want to check them.
\end{verbatim}

As with most examples in this book, what you will see
will be slightly different because we are continuously
improving our databases (including \texttt{set.mm}).

Let's check the database integrity.  This may take a minute or two to run if
your computer is slow.

\begin{verbatim}
MM> verify proof *
0 10%  20%  30%  40%  50%  60%  70%  80%  90% 100%
..................................................
All proofs in the database were verified in 2.84 s.
\end{verbatim}

No errors were reported, so every proof is correct.

You need to know the names (labels) of theorems before you can look at them.
Often just examining the database file(s) with a text editor is the best
approach.  In \texttt{set.mm} there are many detailed comments, especially near
the beginning, that can help guide you. The \texttt{search} command in the
Metamath program is also handy.  The \texttt{comments} qualifier will list the
statements whose associated comment (the one immediately before it) contain a
string you give it.  For example, if you are studying Enderton's {\em Elements
of Set Theory} \cite{Enderton}\index{Enderton, Herbert B.} you may want to see
the references to it in the database.  The search string \texttt{enderton} is not
case sensitive.  (This will not show you all the database theorems that are in
Enderton's book because there is usually only one citation for a given
theorem, which may appear in several textbooks.)\index{\texttt{search}
command}

\begin{verbatim}
MM> search * "enderton" / comments
12067 unineq $p "... Exercise 20 of [Enderton] p. 32 and ..."
12459 undif2 $p "...Corollary 6K of [Enderton] p. 144. (C..."
12953 df-tp $a "...s. Definition of [Enderton] p. 19. (Co..."
13689 unissb $p ".... Exercise 5 of [Enderton] p. 26 and ..."
\end{verbatim}
\begin{center}
(etc.)
\end{center}

Or you may want to see what theorems have something to do with
conjunction (logical {\sc and}).  The quotes around the search
string are optional when there's no ambiguity.\index{\texttt{search}
command}

\begin{verbatim}
MM> search * conjunction / comments
120 a1d $p "...be replaced with a conjunction ( ~ df-an )..."
662 df-bi $a "...viated form after conjunction is introdu..."
1319 wa $a "...ff definition to include conjunction ('and')."
1321 df-an $a "Define conjunction (logical 'and'). Defini..."
1420 imnan $p "...tion in terms of conjunction. (Contribu..."
\end{verbatim}
\begin{center}
(etc.)
\end{center}


Now we will start to look at some details.  Let's look at the first
axiom of propositional calculus
(we could use \texttt{sh st} to abbreviate
\texttt{show statement}).\index{\texttt{show statement} command}

\begin{verbatim}
MM> show statement ax-1/full
Statement 19 is located on line 881 of the file "set.mm".
"Axiom _Simp_.  Axiom A1 of [Margaris] p. 49.  One of the 3
axioms of propositional calculus.  The 3 axioms are also
        ...
19 ax-1 $a |- ( ph -> ( ps -> ph ) ) $.
Its mandatory hypotheses in RPN order are:
  wph $f wff ph $.
  wps $f wff ps $.
The statement and its hypotheses require the variables:  ph
      ps
The variables it contains are:  ph ps


Statement 49 is located on line 11182 of the file "set.mm".
Its statement number for HTML pages is 6.
"Axiom _Simp_.  Axiom A1 of [Margaris] p. 49.  One of the 3
axioms of propositional calculus.  The 3 axioms are also
given as Definition 2.1 of [Hamilton] p. 28.
...
49 ax-1 $a |- ( ph -> ( ps -> ph ) ) $.
Its mandatory hypotheses in RPN order are:
  wph $f wff ph $.
  wps $f wff ps $.
The statement and its hypotheses require the variables:
  ph ps
The variables it contains are:  ph ps
\end{verbatim}

Compare this to \texttt{ax-1} on p.~\pageref{ax1}.  You can see that the
symbol \texttt{ph} is the {\sc ascii} notation for $\varphi$, etc.  To
see the mathematical symbols for any expression you may typeset it in
\LaTeX\ (type \texttt{help tex} for instructions)\index{latex@{\LaTeX}}
or, easier, just use a text editor to look at the comments where symbols
are first introduced in \texttt{set.mm}.  The hypotheses \texttt{wph}
and \texttt{wps} required by \texttt{ax-1} mean that variables
\texttt{ph} and \texttt{ps} must be wffs.

Next we'll pick a simple theorem of propositional calculus, the Principle of
Identity, which is proved directly from the axioms.  We'll look at the
statement then its proof.\index{\texttt{show statement}
command}

\begin{verbatim}
MM> show statement id1/full
Statement 116 is located on line 11371 of the file "set.mm".
Its statement number for HTML pages is 22.
"Principle of identity.  Theorem *2.08 of [WhiteheadRussell]
p. 101.  This version is proved directly from the axioms for
demonstration purposes.
...
116 id1 $p |- ( ph -> ph ) $= ... $.
Its mandatory hypotheses in RPN order are:
  wph $f wff ph $.
Its optional hypotheses are:  wps wch wth wta wet
      wze wsi wrh wmu wla wka
The statement and its hypotheses require the variables:  ph
These additional variables are allowed in its proof:
      ps ch th ta et ze si rh mu la ka
The variables it contains are:  ph
\end{verbatim}

The optional variables\index{optional variable} \texttt{ps}, \texttt{ch}, etc.\ are
available for use in a proof of this statement if we wish, and were we to do
so we would make use of optional hypotheses \texttt{wps}, \texttt{wch}, etc.  (See
Section~\ref{dollaref} for the meaning of ``optional
hypothesis.''\index{optional hypothesis}) The reason these show up in the
statement display is that statement \texttt{id1} happens to be in their scope
(see Section~\ref{scoping} for the definition of ``scope''\index{scope}), but
in fact in propositional calculus we will never make use of optional
hypotheses or variables.  This becomes important after quantifiers are
introduced, where ``dummy'' variables\index{dummy variable} are often needed
in the middle of a proof.

Let's look at the proof of statement \texttt{id1}.  We'll use the
\texttt{show proof} command, which by default suppresses the
``non-essential'' steps that construct the wffs.\index{\texttt{show proof}
command}
We will display the proof in ``lemmon' format (a non-indented format
with explicit previous step number references) and renumber the
displayed steps:

\begin{verbatim}
MM> show proof id1 /lemmon/renumber
1 ax-1           $a |- ( ph -> ( ph -> ph ) )
2 ax-1           $a |- ( ph -> ( ( ph -> ph ) -> ph ) )
3 ax-2           $a |- ( ( ph -> ( ( ph -> ph ) -> ph ) ) ->
                     ( ( ph -> ( ph -> ph ) ) -> ( ph -> ph )
                                                          ) )
4 2,3 ax-mp      $a |- ( ( ph -> ( ph -> ph ) ) -> ( ph -> ph
                                                          ) )
5 1,4 ax-mp      $a |- ( ph -> ph )
\end{verbatim}

If you have read Section~\ref{trialrun}, you'll know how to interpret this
proof.  Step~2, for example, is an application of axiom \texttt{ax-1}.  This
proof is identical to the one in Hamilton's {\em Logic for Mathematicians}
\cite[p.~32]{Hamilton}\index{Hamilton, Alan G.}.

You may want to look at what
substitutions\index{substitution!variable}\index{variable substitution} are
made into \texttt{ax-1} to arrive at step~2.  The command to do this needs to
know the ``real'' step number, so we'll display the proof again without
the \texttt{renumber} qualifier.\index{\texttt{show proof}
command}

\begin{verbatim}
MM> show proof id1 /lemmon
 9 ax-1          $a |- ( ph -> ( ph -> ph ) )
20 ax-1          $a |- ( ph -> ( ( ph -> ph ) -> ph ) )
24 ax-2          $a |- ( ( ph -> ( ( ph -> ph ) -> ph ) ) ->
                     ( ( ph -> ( ph -> ph ) ) -> ( ph -> ph )
                                                          ) )
25 20,24 ax-mp   $a |- ( ( ph -> ( ph -> ph ) ) -> ( ph -> ph
                                                          ) )
26 9,25 ax-mp    $a |- ( ph -> ph )
\end{verbatim}

The ``real'' step number is 20.  Let's look at its details.

\begin{verbatim}
MM> show proof id1 /detailed_step 20
Proof step 20:  min=ax-1 $a |- ( ph -> ( ( ph -> ph ) -> ph )
  )
This step assigns source "ax-1" ($a) to target "min" ($e).
The source assertion requires the hypotheses "wph" ($f, step
18) and "wps" ($f, step 19).  The parent assertion of the
target hypothesis is "ax-mp" ($a, step 25).
The source assertion before substitution was:
    ax-1 $a |- ( ph -> ( ps -> ph ) )
The following substitutions were made to the source
assertion:
    Variable  Substituted with
     ph        ph
     ps        ( ph -> ph )
The target hypothesis before substitution was:
    min $e |- ph
The following substitution was made to the target hypothesis:
    Variable  Substituted with
     ph        ( ph -> ( ( ph -> ph ) -> ph ) )
\end{verbatim}

This shows the substitutions\index{substitution!variable}\index{variable
substitution} made to the variables in \texttt{ax-1}.  References are made to
steps 18 and 19 which are not shown in our proof display.  To see these steps,
you can display the proof with the \texttt{all} qualifier.

Let's look at a slightly more advanced proof of propositional calculus.  Note
that \verb+/\+ is the symbol for $\wedge$ (logical {\sc and}, also
called conjunction).\index{conjunction ($\wedge$)}
\index{logical {\sc and} ($\wedge$)}

\begin{verbatim}
MM> show statement prth/full
Statement 1791 is located on line 15503 of the file "set.mm".
Its statement number for HTML pages is 559.
"Conjoin antecedents and consequents of two premises.  This
is the closed theorem form of ~ anim12d .  Theorem *3.47 of
[WhiteheadRussell] p. 113.  It was proved by Leibniz,
and it evidently pleased him enough to call it
_praeclarum theorema_ (splendid theorem).
...
1791 prth $p |- ( ( ( ph -> ps ) /\ ( ch -> th ) ) -> ( ( ph
      /\ ch ) -> ( ps /\ th ) ) ) $= ... $.
Its mandatory hypotheses in RPN order are:
  wph $f wff ph $.
  wps $f wff ps $.
  wch $f wff ch $.
  wth $f wff th $.
Its optional hypotheses are:  wta wet wze wsi wrh wmu wla wka
The statement and its hypotheses require the variables:  ph
      ps ch th
These additional variables are allowed in its proof:  ta et
      ze si rh mu la ka
The variables it contains are:  ph ps ch th


MM> show proof prth /lemmon/renumber
1 simpl          $p |- ( ( ( ph -> ps ) /\ ( ch -> th ) ) ->
                                               ( ph -> ps ) )
2 simpr          $p |- ( ( ( ph -> ps ) /\ ( ch -> th ) ) ->
                                               ( ch -> th ) )
3 1,2 anim12d    $p |- ( ( ( ph -> ps ) /\ ( ch -> th ) ) ->
                           ( ( ph /\ ch ) -> ( ps /\ th ) ) )
\end{verbatim}

There are references to a lot of unfamiliar statements.  To see what they are,
you may type the following:

\begin{verbatim}
MM> show proof prth /statement_summary
Summary of statements used in the proof of "prth":

Statement simpl is located on line 14748 of the file
"set.mm".
"Elimination of a conjunct.  Theorem *3.26 (Simp) of
[WhiteheadRussell] p. 112. ..."
  simpl $p |- ( ( ph /\ ps ) -> ph ) $= ... $.

Statement simpr is located on line 14777 of the file
"set.mm".
"Elimination of a conjunct.  Theorem *3.27 (Simp) of
[WhiteheadRussell] ..."
  simpr $p |- ( ( ph /\ ps ) -> ps ) $= ... $.

Statement anim12d is located on line 15445 of the file
"set.mm".
"Conjoin antecedents and consequents in a deduction.
..."
  anim12d.1 $e |- ( ph -> ( ps -> ch ) ) $.
  anim12d.2 $e |- ( ph -> ( th -> ta ) ) $.
  anim12d $p |- ( ph -> ( ( ps /\ th ) -> ( ch /\ ta ) ) )
      $= ... $.
\end{verbatim}
\begin{center}
(etc.)
\end{center}

Of course you can look at each of these statements and their proofs, and
so on, back to the axioms of propositional calculus if you wish.

The \texttt{search} command is useful for finding statements when you
know all or part of their contents.  The following example finds all
statements containing \verb@ph -> ps@ followed by \verb@ch -> th@.  The
\verb@$*@ is a wildcard that matches anything; the \texttt{\$} before the
\verb$*$ prevents conflicts with math symbol token names.  The \verb@*@ after
\texttt{SEARCH} is also a wildcard that in this case means ``match any label.''
\index{\texttt{search} command}

% I'm omitting this one, since readers are unlikely to see it:
% 1096 bisymOLD $p |- ( ( ( ph -> ps ) -> ( ch -> th ) ) -> ( (
%   ( ps -> ph ) -> ( th -> ch ) ) -> ( ( ph <-> ps ) -> ( ch
%    <-> th ) ) ) )
\begin{verbatim}
MM> search * "ph -> ps $* ch -> th"
1791 prth $p |- ( ( ( ph -> ps ) /\ ( ch -> th ) ) -> ( ( ph
    /\ ch ) -> ( ps /\ th ) ) )
2455 pm3.48 $p |- ( ( ( ph -> ps ) /\ ( ch -> th ) ) -> ( (
    ph \/ ch ) -> ( ps \/ th ) ) )
117859 pm11.71 $p |- ( ( E. x ph /\ E. y ch ) -> ( ( A. x (
    ph -> ps ) /\ A. y ( ch -> th ) ) <-> A. x A. y ( ( ph /\
    ch ) -> ( ps /\ th ) ) ) )
\end{verbatim}

Three statements, \texttt{prth}, \texttt{pm3.48},
 and \texttt{pm11.71}, were found to match.

To see what axioms\index{axiom} and definitions\index{definition}
\texttt{prth} ultimately depends on for its proof, you can have the
program backtrack through the hierarchy\index{hierarchy} of theorems and
definitions.\index{\texttt{show trace{\char`\_}back} command}

\begin{verbatim}
MM> show trace_back prth /essential/axioms
Statement "prth" assumes the following axioms ($a
statements):
  ax-1 ax-2 ax-3 ax-mp df-bi df-an
\end{verbatim}

Note that the 3 axioms of propositional calculus and the modus ponens rule are
needed (as expected); in addition, there are a couple of definitions that are used
along the way.  Note that Metamath makes no distinction\index{axiom vs.\
definition} between axioms\index{axiom} and definitions\index{definition}.  In
\texttt{set.mm} they have been distinguished artificially by prefixing their
labels\index{labels in \texttt{set.mm}} with \texttt{ax-} and \texttt{df-}
respectively.  For example, \texttt{df-an} defines conjunction (logical {\sc
and}), which is represented by the symbol \verb+/\+.
Section~\ref{definitions} discusses the philosophy of definitions, and the
Metamath language takes a particularly simple, conservative approach by using
the \texttt{\$a}\index{\texttt{\$a} statement} statement for both axioms and
definitions.

You can also have the program compute how many steps a proof
has\index{proof length} if we were to follow it all the way back to
\texttt{\$a} statements.

\begin{verbatim}
MM> show trace_back prth /essential/count_steps
The statement's actual proof has 3 steps.  Backtracking, a
total of 79 different subtheorems are used.  The statement
and subtheorems have a total of 274 actual steps.  If
subtheorems used only once were eliminated, there would be a
total of 38 subtheorems, and the statement and subtheorems
would have a total of 185 steps.  The proof would have 28349
steps if fully expanded back to axiom references.  The
maximum path length is 38.  A longest path is:  prth <-
anim12d <- syl2and <- sylan2d <- ancomsd <- ancom <- pm3.22
<- pm3.21 <- pm3.2 <- ex <- sylbir <- biimpri <- bicomi <-
bicom1 <- bi2 <- dfbi1 <- impbii <- bi3 <- simprim <- impi <-
con1i <- nsyl2 <- mt3d <- con1d <- notnot1 <- con2i <- nsyl3
<- mt2d <- con2d <- notnot2 <- pm2.18d <- pm2.18 <- pm2.21 <-
pm2.21d <- a1d <- syl <- mpd <- a2i <- a2i.1 .
\end{verbatim}

This tells us that we would have to inspect 274 steps if we want to
verify the proof completely starting from the axioms.  A few more
statistics are also shown.  There are one or more paths back to axioms
that are the longest; this command ferrets out one of them and shows it
to you.  There may be a sense in which the longest path length is
related to how ``deep'' the theorem is.

We might also be curious about what proofs depend on the theorem
\texttt{prth}.  If it is never used later on, we could eliminate it as
redundant if it has no intrinsic interest by itself.\index{\texttt{show
usage} command}

% I decided to show the OLD values here.
\begin{verbatim}
MM> show usage prth
Statement "prth" is directly referenced in the proofs of 18
statements:
  mo3 moOLD 2mo 2moOLD euind reuind reuss2 reusv3i opelopabt
  wemaplem2 rexanre rlimcn2 o1of2 o1rlimmul 2sqlem6 spanuni
  heicant pm11.71
\end{verbatim}

Thus \texttt{prth} is directly used by 18 proofs.
We can use the \texttt{/recursive} qualifier to include indirect use:

\begin{verbatim}
MM> show usage prth /recursive
Statement "prth" directly or indirectly affects the proofs of
24214 statements:
  mo3 mo mo3OLD eu2 moOLD eu2OLD eu3OLD mo4f mo4 eu4 mopick
...
\end{verbatim}

\subsection{A Note on the ``Compact'' Proof Format}

The present version of Metamath\index{Metamath!limitations of
version 0.07.30} will display proofs in a ``compact''\index{compact proof}
format whenever the proof is stored in compressed format in the database.  It
may be be slightly confusing unless you know how to interpret it.  (A
future version may eliminate this format from displays.)  For example,
if you display the complete proof of theorem \texttt{id1} it will start
off as follows:

\begin{verbatim}
MM> show proof id1 /lemmon/all
 1 wph           $f wff ph
 2 wph           $f wff ph
 3 wph           $f wff ph
 4 2,3 wi    @4: $a wff ( ph -> ph )
 5 1,4 wi    @5: $a wff ( ph -> ( ph -> ph ) )
 6 @4            $a wff ( ph -> ph )
\end{verbatim}

\begin{center}
{etc.}
\end{center}

Step 4 has a ``local label,''\index{local label} \texttt{@4}, assigned to it.
Later on, at step 6, the label \texttt{@1} is referenced instead of
displaying the explicit proof for that step.  This technique takes advantage
of the fact that steps in a proof often repeat, especially during the
construction of wffs.  The compact format reduces the number of steps in the
proof display and may actually be preferred by some people, which is one
reason that it is still displayed by the program.

If you want to see the normal format with the ``true'' step numbers, you can
use the following workaround:\index{\texttt{save proof} command}

\begin{verbatim}
MM> save proof id1 /normal
The proof of "id1" has been reformatted and saved internally.
Remember to use WRITE SOURCE to save it permanently.
MM> show proof id1 /lemmon/all
 1 wph           $f wff ph
 2 wph           $f wff ph
 3 wph           $f wff ph
 4 2,3 wi        $a wff ( ph -> ph )
 5 1,4 wi        $a wff ( ph -> ( ph -> ph ) )
 6 wph           $f wff ph
 7 wph           $f wff ph
 8 6,7 wi        $a wff ( ph -> ph )
\end{verbatim}

\begin{center}
{etc.}
\end{center}

Note that the original 6 steps are now 8 steps.  However, the format is
now the same as that described in Chapter~\ref{using}.

\chapter{The Metamath Language}
\label{languagespec}

\begin{quote}
  {\em Thus mathematics may be defined as the subject in which we never know
what we are talking about, nor whether what we are saying is true.}
    \flushright\sc  Bertrand Russell\footnote{\cite[p.~84]{Russell2}.}\\
\end{quote}\index{Russell, Bertrand}

Probably the most striking feature of the Metamath language is its almost
complete absence of hard-wired syntax. Metamath\index{Metamath} does not
understand any mathematics or logic other than that needed to construct finite
sequences of symbols according to a small set of simple, built-in rules.  The
only rule it uses in a proof is the substitution of an expression (symbol
sequence) for a variable, subject to a simple constraint to prevent
bound-variable clashes.  The primitive notions built into Metamath involve the
simple manipulation of finite objects (symbols) that we as humans can easily
visualize and that computers can easily deal with.  They seem to be just
about the simplest notions possible that are required to do standard
mathematics.

This chapter serves as a reference manual for the Metamath\index{Metamath}
language. It covers the tedious technical details of the language, some of
which you may wish to skim in a first reading.  On the other hand, you should
pay close attention to the defined terms in {\bf boldface}; they have precise
meanings that are important to keep in mind for later understanding.  It may
be best to first become familiar with the examples in Chapter~\ref{using} to
gain some motivation for the language.

%% Uncomment this when uncommenting section {formalspec} below
If you have some knowledge of set theory, you may wish to study this
chapter in conjunction with the formal set-theoretical description of the
Metamath language in Appendix~\ref{formalspec}.

We will use the name ``Metamath''\index{Metamath} to mean either the Metamath
computer language or the Metamath software associated with the computer
language.  We will not distinguish these two when the context is clear.

The next section contains the complete specification of the Metamath
language.
It serves as an
authoritative reference and presents the syntax in enough detail to
write a parser\index{parsing Metamath} and proof verifier.  The
specification is terse and it is probably hard to learn the language
directly from it, but we include it here for those impatient people who
prefer to see everything up front before looking at verbose expository
material.  Later sections explain this material and provide examples.
We will repeat the definitions in those sections, and you may skip the
next section at first reading and proceed to Section~\ref{tut1}
(p.~\pageref{tut1}).

\section{Specification of the Metamath Language}\label{spec}
\index{Metamath!specification}

\begin{quote}
  {\em Sometimes one has to say difficult things, but one ought to say
them as simply as one knows how.}
    \flushright\sc  G. H. Hardy\footnote{As quoted in
    \cite{deMillo}, p.~273.}\\
\end{quote}\index{Hardy, G. H.}

\subsection{Preliminaries}\label{spec1}

% Space is technically a printable character, so we'll word things
% carefully so it's unambiguous.
A Metamath {\bf database}\index{database} is built up from a top-level source
file together with any source files that are brought in through file inclusion
commands (see below).  The only characters that are allowed to appear in a
Metamath source file are the 94 non-whitespace printable {\sc
ascii}\index{ascii@{\sc ascii}} characters, which are digits, upper and lower
case letters, and the following 32 special
characters\index{special characters}:\label{spec1chars}

\begin{verbatim}
! " # $ % & ' ( ) * + , - . / :
; < = > ? @ [ \ ] ^ _ ` { | } ~
\end{verbatim}
plus the following characters which are the ``white space'' characters:
space (a printable character),
tab, carriage return, line feed, and form feed.\label{whitespace}
We will use \texttt{typewriter}
font to display the printable characters.

A Metamath database consists of a sequence of three kinds of {\bf
tokens}\index{token} separated by {\bf white space}\index{white space}
(which is any sequence of one or more white space characters).  The set
of {\bf keyword}\index{keyword} tokens is \texttt{\$\char`\{},
\texttt{\$\char`\}}, \texttt{\$c}, \texttt{\$v}, \texttt{\$f},
\texttt{\$e}, \texttt{\$d}, \texttt{\$a}, \texttt{\$p}, \texttt{\$.},
\texttt{\$=}, \texttt{\$(}, \texttt{\$)}, \texttt{\$[}, and
\texttt{\$]}.  The last four are called {\bf auxiliary}\index{auxiliary
keyword} or preprocessing keywords.  A {\bf label}\index{label} token
consists of any combination of letters, digits, and the characters
hyphen, underscore, and period.  A {\bf math symbol}\index{math symbol}
token may consist of any combination of the 92 printable standard {\sc
ascii} characters other than space or \texttt{\$}~. All tokens are
case-sensitive.

\subsection{Preprocessing}

The token \texttt{\$(} begins a {\bf comment} and
\texttt{\$)} ends a comment.\index{\texttt{\$(}
and \texttt{\$)} auxiliary keywords}\index{comment}
Comments may contain any of
the 94 non-whitespace printable characters and white space,
except they may not contain the
2-character sequences \texttt{\$(} or \texttt{\$)} (comments do not nest).
Comments are ignored (treated
like white space) for the purpose of parsing, e.g.,
\texttt{\$( \$[ \$)} is a comment.
See p.~\pageref{mathcomments} for comment typesetting conventions; these
conventions may be ignored for the purpose of parsing.

A {\bf file inclusion command} consists of \texttt{\$[} followed by a file name
followed by \texttt{\$]}.\index{\texttt{\$[} and \texttt{\$]} auxiliary
keywords}\index{included file}\index{file inclusion}
It is only allowed in the outermost scope (i.e., not between
\texttt{\$\char`\{} and \texttt{\$\char`\}})
and must not be inside a statement (e.g., it may not occur
between the label of a \texttt{\$a} statement and its \texttt{\$.}).
The file name may not
contain a \texttt{\$} or white space.  The file must exist.
The case-sensitivity
of its name follows the conventions of the operating system.  The contents of
the file replace the inclusion command.
Included files may include other files.
Only the first reference to a given file is included; any later
references to the same file (whether in the top-level file or in included
files) cause the inclusion command to be ignored (treated like white space).
A verifier may assume that file names with different strings
refer to different files for the purpose of ignoring later references.
A file self-reference is ignored, as is any reference to the top-level file
(to avoid loops).
Included files may not include a \texttt{\$(} without a matching \texttt{\$)},
may not include a \texttt{\$[} without a matching \texttt{\$]}, and may
not include incomplete statements (e.g., a \texttt{\$a} without a matching
\texttt{\$.}).
It is currently unspecified if path references are relative to the process'
current directory or the file's containing directory, so databases should
avoid using pathname separators (e.g., ``/'') in file names.

Like all tokens, the \texttt{\$(}, \texttt{\$)}, \texttt{\$[}, and \texttt{\$]} keywords
must be surrounded by white space.

\subsection{Basic Syntax}

After preprocessing, a database will consist of a sequence of {\bf
statements}.
These are the scoping statements \texttt{\$\char`\{} and
\texttt{\$\char`\}}, along with the \texttt{\$c}, \texttt{\$v},
\texttt{\$f}, \texttt{\$e}, \texttt{\$d}, \texttt{\$a}, and \texttt{\$p}
statements.

A {\bf scoping statement}\index{scoping statement} consists only of its
keyword, \texttt{\$\char`\{} or \texttt{\$\char`\}}.
A \texttt{\$\char`\{} begins a {\bf
block}\index{block} and a matching \texttt{\$\char`\}} ends the block.
Every \texttt{\$\char`\{}
must have a matching \texttt{\$\char`\}}.
Defining it recursively, we say a block
contains a sequence of zero or more tokens other
than \texttt{\$\char`\{} and \texttt{\$\char`\}} and
possibly other blocks.  There is an {\bf outermost
block}\index{block!outermost} not bracketed by \texttt{\$\char`\{} \texttt{\$\char`\}}; the end
of the outermost block is the end of the database.

% LaTeX bug? can't do \bf\tt

A {\bf \$v} or {\bf \$c statement}\index{\texttt{\$v} statement}\index{\texttt{\$c}
statement} consists of the keyword token \texttt{\$v} or \texttt{\$c} respectively,
followed by one or more math symbols,
% The word "token" is used to distinguish "$." from the sentence-ending period.
followed by the \texttt{\$.}\ token.
These
statements {\bf declare}\index{declaration} the math symbols to be {\bf
variables}\index{variable!Metamath} or {\bf constants}\index{constant}
respectively. The same math symbol may not occur twice in a given \texttt{\$v} or
\texttt{\$c} statement.

%c%A math symbol becomes an {\bf active}\index{active math symbol}
%c%when declared and stays active until the end of the block in which it is
%c%declared.  A math symbol may not be declared a second time while it is active,
%c%but it may be declared again after it becomes inactive.

A math symbol becomes {\bf active}\index{active math symbol} when declared
and stays active until the end of the block in which it is declared.  A
variable may not be declared a second time while it is active, but it
may be declared again (as a variable, but not as a constant) after it
becomes inactive.  A constant must be declared in the outermost block and may
not be declared a second time.\footnote{The rules for redeclaration may
become more general in the future;
see p.~\pageref{redeclarationf}.}\index{redeclaration of symbols}

A {\bf \$f statement}\index{\texttt{\$f} statement} consists of a label,
followed by \texttt{\$f}, followed by its typecode (an active constant),
followed by an
active variable, followed by the \texttt{\$.}\ token.  A {\bf \$e
statement}\index{\texttt{\$e} statement} consists of a label, followed
by \texttt{\$e}, followed by its typecode (an active constant),
followed by zero or more
active math symbols, followed by the \texttt{\$.}\ token.  A {\bf
hypothesis}\index{hypothesis} is a \texttt{\$f} or \texttt{\$e}
statement.
The type declared by a \texttt{\$f} statement for a given label
is global even if the variable is not
(e.g., a database may not have \texttt{wff P} in one local scope
nd \texttt{class P} in another).

A {\bf simple \$d statement}\index{\texttt{\$d} statement!simple}
consists of \texttt{\$d}, followed by two different active variables,
followed by the \texttt{\$.}\ token.  A {\bf compound \$d
statement}\index{\texttt{\$d} statement!compound} consists of
\texttt{\$d}, followed by three or more variables (all different),
followed by the \texttt{\$.}\ token.  The order of the variables in a
\texttt{\$d} statement is unimportant.  A compound \texttt{\$d}
statement is equivalent to a set of simple \texttt{\$d} statements, one
for each possible pair of variables occurring in the compound
\texttt{\$d} statement.  Henceforth in this specification we shall
assume all \texttt{\$d} statements are simple.  A \texttt{\$d} statement
is also called a {\bf disjoint} (or {\bf distinct}) {\bf variable
restriction}.\index{disjoint-variable restriction}

A {\bf \$a statement}\index{\texttt{\$a} statement} consists of a label,
followed by \texttt{\$a}, followed by its typecode (an active constant),
followed by
zero or more active math symbols, followed by the \texttt{\$.}\ token.  A {\bf
\$p statement}\index{\texttt{\$p} statement} consists of a label,
followed by \texttt{\$p}, followed by its typecode (an active constant),
followed by
zero or more active math symbols, followed by \texttt{\$=}, followed by
a sequence of labels, followed by the \texttt{\$.}\ token.  An {\bf
assertion}\index{assertion} is a \texttt{\$a} or \texttt{\$p} statement.

A \texttt{\$f}, \texttt{\$e}, or \texttt{\$d} statement is {\bf active}\index{active
statement} from the place it occurs until the end of the block it occurs in.
A \texttt{\$a} or \texttt{\$p} statement is {\bf active} from the place it occurs
through the end of the database.
There may not be two active \texttt{\$f} statements containing the same
variable.  Each variable in a \texttt{\$e}, \texttt{\$a}, or
\texttt{\$p} statement must exist in an active \texttt{\$f}
statement.\footnote{This requirement can greatly simplify the
unification algorithm (substitution calculation) required by proof
verification.}

%The label that begins each \texttt{\$f}, \texttt{\$e}, \texttt{\$a}, and
%\texttt{\$p} statement must be unique.
Each label token must be unique, and
no label token may match any math symbol
token.\label{namespace}\footnote{This
restriction was added on June 24, 2006.
It is not theoretically necessary but is imposed to make it easier to
write certain parsers.}

The set of {\bf mandatory variables}\index{mandatory variable} associated with
an assertion is the set of (zero or more) variables in the assertion and in any
active \texttt{\$e} statements.  The (possibly empty) set of {\bf mandatory
hypotheses}\index{mandatory hypothesis} is the set of all active \texttt{\$f}
statements containing mandatory variables, together with all active \texttt{\$e}
statements.
The set of {\bf mandatory {\bf \$d} statements}\index{mandatory
disjoint-variable restriction} associated with an assertion are those active
\texttt{\$d} statements whose variables are both among the assertion's
mandatory variables.

\subsection{Proof Verification}\label{spec4}

The sequence of labels between the \texttt{\$=} and \texttt{\$.}\ tokens
in a \texttt{\$p} statement is a {\bf proof}.\index{proof!Metamath} Each
label in a proof must be the label of an active statement other than the
\texttt{\$p} statement itself; thus a label must refer either to an
active hypothesis of the \texttt{\$p} statement or to an earlier
assertion.

An {\bf expression}\index{expression} is a sequence of math symbols. A {\bf
substitution map}\index{substitution map} associates a set of variables with a
set of expressions.  It is acceptable for a variable to be mapped to an
expression containing it.  A {\bf
substitution}\index{substitution!variable}\index{variable substitution} is the
simultaneous replacement of all variables in one or more expressions with the
expressions that the variables map to.

A proof is scanned in order of its label sequence.  If the label refers to an
active hypothesis, the expression in the hypothesis is pushed onto a
stack.\index{stack}\index{RPN stack}  If the label refers to an assertion, a
(unique) substitution must exist that, when made to the mandatory hypotheses
of the referenced assertion, causes them to match the topmost (i.e.\ most
recent) entries of the stack, in order of occurrence of the mandatory
hypotheses, with the topmost stack entry matching the last mandatory
hypothesis of the referenced assertion.  As many stack entries as there are
mandatory hypotheses are then popped from the stack.  The same substitution is
made to the referenced assertion, and the result is pushed onto the stack.
After the last label in the proof is processed, the stack must have a single
entry that matches the expression in the \texttt{\$p} statement containing the
proof.

%c%{\footnotesize\begin{quotation}\index{redeclaration of symbols}
%c%{{\em Comment.}\label{spec4comment} Whenever a math symbol token occurs in a
%c%{\texttt{\$c} or \texttt{\$v} statement, it is considered to designate a distinct new
%c%{symbol, even if the same token was previously declared (and is now inactive).
%c%{Thus a math token declared as a constant in two different blocks is considered
%c%{to designate two distinct constants (even though they have the same name).
%c%{The two constants will not match in a proof that references both blocks.
%c%{However, a proof referencing both blocks is acceptable as long as it doesn't
%c%{require that the constants match.  Similarly, a token declared to be a
%c%{constant for a referenced assertion will not match the same token declared to
%c%{be a variable for the \texttt{\$p} statement containing the proof.  In the case
%c%{of a token declared to be a variable for a referenced assertion, this is not
%c%{an issue since the variable can be substituted with whatever expression is
%c%{needed to achieve the required match.
%c%{\end{quotation}}
%c2%A proof may reference an assertion that contains or whose hypotheses contain a
%c2%constant that is not active for the \texttt{\$p} statement containing the proof.
%c2%However, the final result of the proof may not contain that constant. A proof
%c2%may also reference an assertion that contains or whose hypotheses contain a
%c2%variable that is not active for the \texttt{\$p} statement containing the proof.
%c2%That variable, of course, will be substituted with whatever expression is
%c2%needed to achieve the required match.

A proof may contain a \texttt{?}\ in place of a label to indicate an unknown step
(Section~\ref{unknown}).  A proof verifier may ignore any proof containing
\texttt{?}\ but should warn the user that the proof is incomplete.

A {\bf compressed proof}\index{compressed proof}\index{proof!compressed} is an
alternate proof notation described in Appen\-dix~\ref{compressed}; also see
references to ``compressed proof'' in the Index.  Compressed proofs are a
Metamath language extension which a complete proof verifier should be able to
parse and verify.

\subsubsection{Verifying Disjoint Variable Restrictions}

Each substitution made in a proof must be checked to verify that any
disjoint variable restrictions are satisfied, as follows.

If two variables replaced by a substitution exist in a mandatory \texttt{\$d}
statement\index{\texttt{\$d} statement} of the assertion referenced, the two
expressions resulting from the substitution must satisfy the following
conditions.  First, the two expressions must have no variables in common.
Second, each possible pair of variables, one from each expression, must exist
in an active \texttt{\$d} statement of the \texttt{\$p} statement containing the
proof.

\vskip 1ex

This ends the specification of the Metamath language;
see Appendix \ref{BNF} for its syntax in
Extended Backus--Naur Form (EBNF)\index{Extended Backus--Naur Form}\index{EBNF}.

\section{The Basic Keywords}\label{tut1}

Our expository material begins here.

Like most computer languages, Metamath\index{Metamath} takes its input from
one or more {\bf source files}\index{source file} which contain characters
expressed in the standard {\sc ascii} (American Standard Code for Information
Interchange)\index{ascii@{\sc ascii}} code for computers.  A source file
consists of a series of {\bf tokens}\index{token}, which are strings of
non-whitespace
printable characters (from the set of 94 shown on p.~\pageref{spec1chars})
separated by {\bf white space}\index{white space} (spaces, tabs, carriage
returns, line feeds, and form feeds). Any string consisting only of these
characters is treated the same as a single space.  The non-whitespace printable
characters\index{printable character} that Metamath recognizes are the 94
characters on standard {\sc ascii} keyboards.

Metamath has the ability to join several files together to form its
input (Section~\ref{include}).  We call the aggregate contents of all
the files after they have been joined together a {\bf
database}\index{database} to distinguish it from an individual source
file.  The tokens in a database consist of {\bf
keywords}\index{keyword}, which are built into the language, together
with two kinds of user-defined tokens called {\bf labels}\index{label}
and {\bf math symbols}\index{math symbol}.  (Often we will simply say
{\bf symbol}\index{symbol} instead of math symbol for brevity).  The set
of {\bf basic keywords}\index{basic keyword} is
\texttt{\$c}\index{\texttt{\$c} statement},
\texttt{\$v}\index{\texttt{\$v} statement},
\texttt{\$e}\index{\texttt{\$e} statement},
\texttt{\$f}\index{\texttt{\$f} statement},
\texttt{\$d}\index{\texttt{\$d} statement},
\texttt{\$a}\index{\texttt{\$a} statement},
\texttt{\$p}\index{\texttt{\$p} statement},
\texttt{\$=}\index{\texttt{\$=} keyword},
\texttt{\$.}\index{\texttt{\$.}\ keyword},
\texttt{\$\char`\{}\index{\texttt{\$\char`\{} and \texttt{\$\char`\}}
keywords}, and \texttt{\$\char`\}}.  This is the complete set of
syntactical elements of what we call the {\bf basic
language}\index{basic language} of Metamath, and with them you can
express all of the mathematics that were intended by the design of
Metamath.  You should make it a point to become very familiar with them.
Table~\ref{basickeywords} lists the basic keywords along with a brief
description of their functions.  For now, this description will give you
only a vague notion of what the keywords are for; later we will describe
the keywords in detail.


\begin{table}[htp] \caption{Summary of the basic Metamath
keywords} \label{basickeywords}
\begin{center}
\begin{tabular}{|p{4pc}|l|}
\hline
\em \centering Keyword&\em Description\\
\hline
\hline
\centering
   \texttt{\$c}&Constant symbol declaration\\
\hline
\centering
   \texttt{\$v}&Variable symbol declaration\\
\hline
\centering
   \texttt{\$d}&Disjoint variable restriction\\
\hline
\centering
   \texttt{\$f}&Variable-type (``floating'') hypothesis\\
\hline
\centering
   \texttt{\$e}&Logical (``essential'') hypothesis\\
\hline
\centering
   \texttt{\$a}&Axiomatic assertion\\
\hline
\centering
   \texttt{\$p}&Provable assertion\\
\hline
\centering
   \texttt{\$=}&Start of proof in \texttt{\$p} statement\\
\hline
\centering
   \texttt{\$.}&End of the above statement types\\
\hline
\centering
   \texttt{\$\char`\{}&Start of block\\
\hline
\centering
   \texttt{\$\char`\}}&End of block\\
\hline
\end{tabular}
\end{center}
\end{table}

%For LaTeX bug(?) where it puts tables on blank page instead of btwn text
%May have to adjust if text changes
%\newpage

There are some additional keywords, called {\bf auxiliary
keywords}\index{auxiliary keyword} that help make Metamath\index{Metamath}
more practical. These are part of the {\bf extended language}\index{extended
language}. They provide you with a means to put comments into a Metamath
source file\index{source file} and reference other source files.  We will
introduce these in later sections. Table~\ref{otherkeywords} summarizes them
so that you can recognize them now if you want to peruse some source
files while learning the basic keywords.


\begin{table}[htp] \caption{Auxiliary Metamath
keywords} \label{otherkeywords}
\begin{center}
\begin{tabular}{|p{4pc}|l|}
\hline
\em \centering Keyword&\em Description\\
\hline
\hline
\centering
   \texttt{\$(}&Start of comment\\
\hline
\centering
   \texttt{\$)}&End of comment\\
\hline
\centering
   \texttt{\$[}&Start of included source file name\\
\hline
\centering
   \texttt{\$]}&End of included source file name\\
\hline
\end{tabular}
\end{center}
\end{table}
\index{\texttt{\$(} and \texttt{\$)} auxiliary keywords}
\index{\texttt{\$[} and \texttt{\$]} auxiliary keywords}


Unlike those in some computer languages, the keywords\index{keyword} are short
two-character sequences rather than English-like words.  While this may make
them slightly more difficult to remember at first, their brevity allows
them to blend in with the mathematics being described, not
distract from it, like punctuation marks.


\subsection{User-Defined Tokens}\label{dollardollar}\index{token}

As you may have noticed, all keywords\index{keyword} begin with the \texttt{\$}
character.  This mundane monetary symbol is not ordinarily used in higher
mathematics (outside of grant proposals), so we have appropriated it to
distinguish the Metamath\index{Metamath} keywords from ordinary mathematical
symbols. The \texttt{\$} character is thus considered special and may not be
used as a character in a user-defined token.  All tokens and keywords are
case-sensitive; for example, \texttt{n} is considered to be a different character
from \texttt{N}.  Case-sensitivity makes the available {\sc ascii} character set
as rich as possible.

\subsubsection{Math Symbol Tokens}\index{token}

Math symbols\index{math symbol} are tokens used to represent the symbols
that appear in ordinary mathematical formulas.  They may consist of any
combination of the 93 non-whitespace printable {\sc ascii} characters other than
\texttt{\$}~. Some examples are \texttt{x}, \texttt{+}, \texttt{(},
\texttt{|-}, \verb$!%@?&$, and \texttt{bounded}.  For readability, it is
best to try to make these look as similar to actual mathematical symbols
as possible, within the constraints of the {\sc ascii} character set, in
order to make the resulting mathematical expressions more readable.

In the Metamath\index{Metamath} language, you express ordinary
mathematical formulas and statements as sequences of math symbols such
as \texttt{2 + 2 = 4} (five symbols, all constants).\footnote{To
eliminate ambiguity with other expressions, this is expressed in the set
theory database \texttt{set.mm} as \texttt{|- ( 2 + 2
 ) = 4 }, whose \LaTeX\ equivalent is $\vdash
(2+2)=4$.  The \,$\vdash$ means ``is a theorem'' and the
parentheses allow explicit associative grouping.}\index{turnstile
({$\,\vdash$})} They may even be English
sentences, as in \texttt{E is closed and bounded} (five symbols)---here
\texttt{E} would be a variable and the other four symbols constants.  In
principle, a Metamath database could be constructed to work with almost
any unambiguous English-language mathematical statement, but as a
practical matter the definitions needed to provide for all possible
syntax variations would be cumbersome and distracting and possibly have
subtle pitfalls accidentally built in.  We generally recommend that you
express mathematical statements with compact standard mathematical
symbols whenever possible and put their English-language descriptions in
comments.  Axioms\index{axiom} and definitions\index{definition}
(\texttt{\$a}\index{\texttt{\$a} statement} statements) are the only
places where Metamath will not detect an error, and doing this will help
reduce the number of definitions needed.

You are free to use any tokens\index{token} you like for math
symbols\index{math symbol}.  Appendix~\ref{ASCII} recommends token names to
use for symbols in set theory, and we suggest you adopt these in order to be
able to include the \texttt{set.mm} set theory database in your database.  For
printouts, you can convert the tokens in a database
to standard mathematical symbols with the \LaTeX\ typesetting program.  The
Metamath command \texttt{open tex} {\em filename}\index{\texttt{open tex} command}
produces output that can be read by \LaTeX.\index{latex@{\LaTeX}}
The correspondence
between tokens and the actual symbols is made by \texttt{latexdef}
statements inside a special database comment tagged
with \texttt{\$t}.\index{\texttt{\$t} comment}\index{typesetting comment}
  You can edit
this comment to change the definitions or add new ones.
Appendix~\ref{ASCII} describes how to do this in more detail.

% White space\index{white space} is normally used to separate math
% symbol\index{math symbol} tokens, but they may be juxtaposed without white
% space in \texttt{\$d}\index{\texttt{\$d} statement}, \texttt{\$e}\index{\texttt{\$e}
% statement}, \texttt{\$f}\index{\texttt{\$f} statement}, \texttt{\$a}\index{\texttt{\$a}
% statement}, and \texttt{\$p}\index{\texttt{\$p} statement} statements when no
% ambiguity will result.  Specifically, Metamath parses the math symbol sequence
% in one of these statements in the following manner:  when the math symbol
% sequence has been broken up into tokens\index{token} up to a given character,
% the next token is the longest string of characters that could constitute a
% math symbol that is active\index{active
% math symbol} at that point.  (See Section~\ref{scoping} for the
% definition of an active math symbol.)  For example, if \texttt{-}, \texttt{>}, and
% \texttt{->} are the only active math symbols, the juxtaposition \texttt{>-} will be
% interpreted as the two symbols \texttt{>} and \texttt{-}, whereas \texttt{->} will
% always be interpreted as that single symbol.\footnote{For better readability we
% recommend a white space between each token.  This also makes searching for a
% symbol easier to do with an editor.  Omission of optional white space is useful
% for reducing typing when assigning an expression to a temporary
% variable\index{temporary variable} with the \texttt{let variable} Metamath
% program command.}\index{\texttt{let variable} command}
%
% Keywords\index{keyword} may be placed next to math symbols without white
% space\index{white space} between them.\footnote{Again, we do not recommend
% this for readability.}
%
% The math symbols\index{math symbol} in \texttt{\$c}\index{\texttt{\$c} statement}
% and \texttt{\$v}\index{\texttt{\$v} statement} statements must always be separated
% by white space\index{white
% space}, for the obvious reason that these statements define the names
% of the symbols.
%
% Math symbols referred to in comments (see Section~\ref{comments}) must also be
% separated by white space.  This allows you to make comments about symbols that
% are not yet active\index{active
% math symbol}.  (The ``math mode'' feature of comments is also a quick and
% easy way to obtain word processing text with embedded mathematical symbols,
% independently of the main purpose of Metamath; the way to do this is described
% in Section~\ref{comments})

\subsubsection{Label Tokens}\index{token}\index{label}

Label tokens are used to identify Metamath\index{Metamath} statements for
later reference. Label tokens may contain only letters, digits, and the three
characters period, hyphen, and underscore:
\begin{verbatim}
. - _
\end{verbatim}

A label is {\bf declared}\index{label declaration} by placing it immediately
before the keyword of the statement it identifies.  For example, the label
\texttt{axiom.1} might be declared as follows:
\begin{verbatim}
axiom.1 $a |- x = x $.
\end{verbatim}

Each \texttt{\$e}\index{\texttt{\$e} statement},
\texttt{\$f}\index{\texttt{\$f} statement},
\texttt{\$a}\index{\texttt{\$a} statement}, and
\texttt{\$p}\index{\texttt{\$p} statement} statement in a database must
have a label declared for it.  No other statement types may have label
declarations.  Every label must be unique.

A label (and the statement it identifies) is {\bf referenced}\index{label
reference} by including the label between the \texttt{\$=}\index{\texttt{\$=}
keyword} and \texttt{\$.}\index{\texttt{\$.}\ keyword}\ keywords in a \texttt{\$p}
statement.  The sequence of labels\index{label sequence} between these two
keywords is called a {\bf proof}\index{proof}.  An example of a statement with
a proof that we will encounter later (Section~\ref{proof}) is
\begin{verbatim}
wnew $p wff ( s -> ( r -> p ) )
     $= ws wr wp w2 w2 $.
\end{verbatim}

You don't have to know what this means just yet, but you should know that the
label \texttt{wnew} is declared by this \texttt{\$p} statement and that the labels
\texttt{ws}, \texttt{wr}, \texttt{wp}, and \texttt{w2} are assumed to have been declared
earlier in the database and are referenced here.

\subsection{Constants and Variables}
\index{constant}
\index{variable}

An {\bf expression}\index{expression} is any sequence of math
symbols, possibly empty.

The basic Metamath\index{Metamath} language\index{basic language} has two
kinds of math symbols\index{math symbol}:  {\bf constants}\index{constant} and
{\bf variables}\index{variable}.  In a Metamath proof, a constant may not be
substituted with any expression.  A variable can be
substituted\index{substitution!variable}\index{variable substitution} with any
expression.  This sequence may include other variables and may even include
the variable being substituted.  This substitution takes place when proofs are
verified, and it will be described in Section~\ref{proof}.  The \texttt{\$f}
statement (described later in Section~\ref{dollaref}) is used to specify the
{\bf type} of a variable (i.e.\ what kind of
variable it is)\index{variable type}\index{type} and
give it a meaning typically
associated with a ``metavariable''\index{metavariable}\footnote{A metavariable
is a variable that ranges over the syntactical elements of the object language
being discussed; for example, one metavariable might represent a variable of
the object language and another metavariable might represent a formula in the
object language.} in ordinary mathematics; for example, a variable may be
specified to be a wff or well-formed formula (in logic), a set (in set
theory), or a non-negative integer (in number theory).

%\subsection{The \texttt{\$c} and \texttt{\$v} Declaration Statements}
\subsection{The \texttt{\$c} and \texttt{\$v} Declaration Statements}
\index{\texttt{\$c} statement}
\index{constant declaration}
\index{\texttt{\$v} statement}
\index{variable declaration}

Constants are introduced or {\bf declared}\index{constant declaration}
with \texttt{\$c}\index{\texttt{\$c} statement} statements, and
variables are declared\index{variable declaration} with
\texttt{\$v}\index{\texttt{\$v} statement} statements.  A {\bf simple}
declaration\index{simple declaration} statement introduces a single
constant or variable.  Its syntax is one of the following:
\begin{center}
  \texttt{\$c} {\em math-symbol} \texttt{\$.}\\
  \texttt{\$v} {\em math-symbol} \texttt{\$.}
\end{center}
The notation {\em math-symbol} means any math symbol token\index{token}.

Some examples of simple declaration statements are:
\begin{center}
  \texttt{\$c + \$.}\\
  \texttt{\$c -> \$.}\\
  \texttt{\$c ( \$.}\\
  \texttt{\$v x \$.}\\
  \texttt{\$v y2 \$.}
\end{center}

The characters in a math symbol\index{math symbol} being declared are
irrelevant to Meta\-math; for example, we could declare a right parenthesis to
be a variable,
\begin{center}
  \texttt{\$v ) \$.}\\
\end{center}
although this would be unconventional.

A {\bf compound} declaration\index{compound declaration} statement is a
shorthand for declaring several symbols at once.  Its syntax is one of the
following:
\begin{center}
  \texttt{\$c} {\em math-symbol}\ \,$\cdots$\ {\em math-symbol} \texttt{\$.}\\
  \texttt{\$v} {\em math-symbol}\ \,$\cdots$\ {\em math-symbol} \texttt{\$.}
\end{center}\index{\texttt{\$c} statement}
Here, the ellipsis (\ldots) means any number of {\em math-symbol}\,s.

An example of a compound declaration statement is:
\begin{center}
  \texttt{\$v x y mu \$.}\\
\end{center}
This is equivalent to the three simple declaration statements
\begin{center}
  \texttt{\$v x \$.}\\
  \texttt{\$v y \$.}\\
  \texttt{\$v mu \$.}\\
\end{center}
\index{\texttt{\$v} statement}

There are certain rules on where in the database math symbols may be declared,
what sections of the database are aware of them (i.e.\ where they are
``active''), and when they may be declared more than once.  These will be
discussed in Section~\ref{scoping} and specifically on
p.~\pageref{redeclaration}.

\subsection{The \texttt{\$d} Statement}\label{dollard}
\index{\texttt{\$d} statement}

The \texttt{\$d} statement is called a {\bf disjoint-variable restriction}.  The
syntax of the {\bf simple} version of this statement is
\begin{center}
  \texttt{\$d} {\em variable variable} \texttt{\$.}
\end{center}
where each {\em variable} is a previously declared variable and the two {\em
variable}\,s are different.  (More specifically, each  {\em variable} must be
an {\bf active} variable\index{active math symbol}, which means there must be
a previous \texttt{\$v} statement whose {\bf scope}\index{scope} includes the
\texttt{\$d} statement.  These terms will be defined when we discuss scoping
statements in Section~\ref{scoping}.)

In ordinary mathematics, formulas may arise that are true if the variables in
them are distinct\index{distinct variables}, but become false when those
variables are made identical. For example, the formula in logic $\exists x\,x
\neq y$, which means ``for a given $y$, there exists an $x$ that is not equal
to $y$,'' is true in most mathematical theories (namely all non-trivial
theories\index{non-trivial theory}, i.e.\ those that describe more than one
individual, such as arithmetic).  However, if we substitute $y$ with $x$, we
obtain $\exists x\,x \neq x$, which is always false, as it means ``there
exists something that is not equal to itself.''\footnote{If you are a
logician, you will recognize this as the improper substitution\index{proper
substitution}\index{substitution!proper} of a free variable\index{free
variable} with a bound variable\index{bound variable}.  Metamath makes no
inherent distinction between free and bound variables; instead, you let
Metamath know what substitutions are permissible by using \texttt{\$d} statements
in the right way in your axiom system.}\index{free vs.\ bound variable}  The
\texttt{\$d} statement allows you to specify a restriction that forbids the
substitution of one variable with another.  In
this case, we would use the statement
\begin{center}
  \texttt{\$d x y \$.}
\end{center}\index{\texttt{\$d} statement}
to specify this restriction.

The order in which the variables appear in a \texttt{\$d} statement is not
important.  We could also use
\begin{center}
  \texttt{\$d y x \$.}
\end{center}

The \texttt{\$d} statement is actually more general than this, as the
``disjoint''\index{disjoint variables} in its name suggests.  The full meaning
is that if any substitution is made to its two variables (during the
course of a proof that references a \texttt{\$a} or \texttt{\$p} statement
associated with the \texttt{\$d}), the two expressions that result from the
substitution must have no variables in common.  In addition, each possible
pair of variables, one from each expression, must be in a \texttt{\$d} statement
associated with the statement being proved.  (This requirement forces the
statement being proved to ``inherit'' the original disjoint variable
restriction.)

For example, suppose \texttt{u} is a variable.  If the restriction
\begin{center}
  \texttt{\$d A B \$.}
\end{center}
has been specified for a theorem referenced in a
proof, we may not substitute \texttt{A} with \mbox{\tt a + u} and
\texttt{B} with \mbox{\tt b + u} because these two symbol sequences have the
variable \texttt{u} in common.  Furthermore, if \texttt{a} and \texttt{b} are
variables, we may not substitute \texttt{A} with \texttt{a} and \texttt{B} with \texttt{b}
unless we have also specified \texttt{\$d a b} for the theorem being proved; in
other words, the \texttt{\$d} property associated with a pair of variables must
be effectively preserved after substitution.

The \texttt{\$d}\index{\texttt{\$d} statement} statement does {\em not} mean ``the
two variables may not be substituted with the same thing,'' as you might think
at first.  For example, substituting each of \texttt{A} and \texttt{B} in the above
example with identical symbol sequences consisting only of constants does not
cause a disjoint variable conflict, because two symbol sequences have no
variables in common (since they have no variables, period).  Similarly, a
conflict will not occur by substituting the two variables in a \texttt{\$d}
statement with the empty symbol sequence\index{empty substitution}.

The \texttt{\$d} statement does not have a direct counterpart in
ordinary mathematics, partly because the variables\index{variable} of
Metamath are not really the same as the variables\index{variable!in
ordinary mathematics} of ordinary mathematics but rather are
metavariables\index{metavariable} ranging over them (as well as over
other kinds of symbols and groups of symbols).  Depending on the
situation, we may informally interpret the \texttt{\$d} statement in
different ways.  Suppose, for example, that \texttt{x} and \texttt{y}
are variables ranging over numbers (more precisely, that \texttt{x} and
\texttt{y} are metavariables ranging over variables that range over
numbers), and that \texttt{ph} ($\varphi$) and \texttt{ps} ($\psi$) are
variables (more precisely, metavariables) ranging over formulas.  We can
make the following interpretations that correspond to the informal
language of ordinary mathematics:
\begin{quote}
\begin{tabbing}
\texttt{\$d x y \$.} means ``assume $x$ and $y$ are
distinct variables.''\\
\texttt{\$d x ph \$.} means ``assume $x$ does not
occur in $\varphi$.''\\
\texttt{\$d ph ps \$.} \=means ``assume $\varphi$ and
$\psi$ have no variables\\ \>in common.''
\end{tabbing}
\end{quote}\index{\texttt{\$d} statement}

\subsubsection{Compound \texttt{\$d} Statements}

The {\bf compound} version of the \texttt{\$d} statement is a shorthand for
specifying several variables whose substitutions must be pairwise disjoint.
Its syntax is:
\begin{center}
  \texttt{\$d} {\em variable}\ \,$\cdots$\ {\em variable} \texttt{\$.}
\end{center}\index{\texttt{\$d} statement}
Here, {\em variable} represents the token of a previously declared
variable (specifically, an active variable) and all {\em variable}\,s are
different.  The compound \texttt{\$d}
statement is internally broken up by Metamath into one simple \texttt{\$d}
statement for each possible pair of variables in the original \texttt{\$d}
statement.  For example,
\begin{center}
  \texttt{\$d w x y z \$.}
\end{center}
is equivalent to
\begin{center}
  \texttt{\$d w x \$.}\\
  \texttt{\$d w y \$.}\\
  \texttt{\$d w z \$.}\\
  \texttt{\$d x y \$.}\\
  \texttt{\$d x z \$.}\\
  \texttt{\$d y z \$.}
\end{center}

Two or more simple \texttt{\$d} statements specifying the same variable pair are
internally combined into a single \texttt{\$d} statement.  Thus the set of three
statements
\begin{center}
  \texttt{\$d x y \$.}
  \texttt{\$d x y \$.}
  \texttt{\$d y x \$.}
\end{center}
is equivalent to
\begin{center}
  \texttt{\$d x y \$.}
\end{center}

Similarly, compound \texttt{\$d} statements, after being internally broken up,
internally have their common variable pairs combined.  For example the
set of statements
\begin{center}
  \texttt{\$d x y A \$.}
  \texttt{\$d x y B \$.}
\end{center}
is equivalent to
\begin{center}
  \texttt{\$d x y \$.}
  \texttt{\$d x A \$.}
  \texttt{\$d y A \$.}
  \texttt{\$d x y \$.}
  \texttt{\$d x B \$.}
  \texttt{\$d y B \$.}
\end{center}
which is equivalent to
\begin{center}
  \texttt{\$d x y \$.}
  \texttt{\$d x A \$.}
  \texttt{\$d y A \$.}
  \texttt{\$d x B \$.}
  \texttt{\$d y B \$.}
\end{center}

Metamath\index{Metamath} automatically verifies that all \texttt{\$d}
restrictions are met whenever it verifies proofs.  \texttt{\$d} statements are
never referenced directly in proofs (this is why they do not have
labels\index{label}), but Metamath is always aware of which ones must be
satisfied (i.e.\ are active) and will notify you with an error message if any
violation occurs.

To illustrate how Metamath detects a missing \texttt{\$d}
statement, we will look at the following example from the
\texttt{set.mm} database.

\begin{verbatim}
$d x z $.  $d y z $.
$( Theorem to add distinct quantifier to atomic formula. $)
ax17eq $p |- ( x = y -> A. z x = y ) $=...
\end{verbatim}

This statement has the obvious requirement that $z$ must be
distinct\index{distinct variables} from $x$ in theorem \texttt{ax17eq} that
states $x=y \rightarrow \forall z \, x=y$ (well, obvious if you're a logician,
for otherwise we could conclude  $x=y \rightarrow \forall x \, x=y$, which is
false when the free variables $x$ and $y$ are equal.).

Let's look at what happens if we edit the database to comment out this
requirement.

\begin{verbatim}
$( $d x z $. $) $d y z $.
$( Theorem to add distinct quantifier to atomic formula. $)
ax17eq $p |- ( x = y -> A. z x = y ) $=...
\end{verbatim}

When it tries to verify the proof, Metamath will tell you that \texttt{x} and
\texttt{z} must be disjoint, because one of its steps references an axiom or
theorem that has this requirement.

\begin{verbatim}
MM> verify proof ax17eq
ax17eq ?Error at statement 1918, label "ax17eq", type "$p":
      vz wal wi vx vy vz ax-13 vx vy weq vz vx ax-c16 vx vy
                                               ^^^^^
There is a disjoint variable ($d) violation at proof step 29.
Assertion "ax-c16" requires that variables "x" and "y" be
disjoint.  But "x" was substituted with "z" and "y" was
substituted with "x".  The assertion being proved, "ax17eq",
does not require that variables "z" and "x" be disjoint.
\end{verbatim}

We can see the substitutions into \texttt{ax-c16} with the following command.

\begin{verbatim}
MM> show proof ax17eq / detailed_step 29
Proof step 29:  pm2.61dd.2=ax-c16 $a |- ( A. z z = x -> ( x =
  y -> A. z x = y ) )
This step assigns source "ax-c16" ($a) to target "pm2.61dd.2"
($e).  The source assertion requires the hypotheses "wph"
($f, step 26), "vx" ($f, step 27), and "vy" ($f, step 28).
The parent assertion of the target hypothesis is "pm2.61dd"
($p, step 36).
The source assertion before substitution was:
    ax-c16 $a |- ( A. x x = y -> ( ph -> A. x ph ) )
The following substitutions were made to the source
assertion:
    Variable  Substituted with
     x         z
     y         x
     ph        x = y
The target hypothesis before substitution was:
    pm2.61dd.2 $e |- ( ph -> ch )
The following substitutions were made to the target
hypothesis:
    Variable  Substituted with
     ph        A. z z = x
     ch        ( x = y -> A. z x = y )
\end{verbatim}

The disjoint variable restrictions of \texttt{ax-c16} can be seen from the
\texttt{show state\-ment} command.  The line that begins ``\texttt{Its mandatory
dis\-joint var\-i\-able pairs are:}\ldots'' lists any \texttt{\$d} variable
pairs in brackets.

\begin{verbatim}
MM> show statement ax-c16/full
Statement 3033 is located on line 9338 of the file "set.mm".
"Axiom of Distinct Variables. ..."
  ax-c16 $a |- ( A. x x = y -> ( ph -> A. x ph ) ) $.
Its mandatory hypotheses in RPN order are:
  wph $f wff ph $.
  vx $f set x $.
  vy $f set y $.
Its mandatory disjoint variable pairs are:  <x,y>
The statement and its hypotheses require the variables:  x y
      ph
The variables it contains are:  x y ph
\end{verbatim}

Since Metamath will always detect when \texttt{\$d}\index{\texttt{\$d} statement}
statements are needed for a proof, you don't have to worry too much about
forgetting to put one in; it can always be added if you see the error message
above.  If you put in unnecessary \texttt{\$d} statements, the worst that could
happen is that your theorem might not be as general as it could be, and this
may limit its use later on.

On the other hand, when you introduce axioms (\texttt{\$a}\index{\texttt{\$a}
statement} statements), you must be very careful to properly specify the
necessary associated \texttt{\$d} statements since Metamath has no way of knowing
whether your axioms are correct.  For example, Metamath would have no idea
that \texttt{ax-c16}, which we are telling it is an axiom of logic, would lead to
contradictions if we omitted its associated \texttt{\$d} statement.

% This was previously a comment in footnote-sized type, but it can be
% hard to read this much text in a small size.
% As a result, it's been changed to normally-sized text.
\label{nodd}
You may wonder if it is possible to develop standard
mathematics in the Metamath language without the \texttt{\$d}\index{\texttt{\$d}
statement} statement, since it seems like a nuisance that complicates proof
verification. The \texttt{\$d} statement is not needed in certain subsets of
mathematics such as propositional calculus.  However, dummy
variables\index{dummy variable!eliminating} and their associated \texttt{\$d}
statements are impossible to avoid in proofs in standard first-order logic as
well as in the variant used in \texttt{set.mm}.  In fact, there is no upper bound to
the number of dummy variables that might be needed in a proof of a theorem of
first-order logic containing 3 or more variables, as shown by H.\
Andr\'{e}ka\index{Andr{\'{e}}ka, H.} \cite{Nemeti}.  A first-order system that
avoids them entirely is given in \cite{Megill}\index{Megill, Norman}; the
trick there is simply to embed harmlessly the necessary dummy variables into a
theorem being proved so that they aren't ``dummy'' anymore, then interpret the
resulting longer theorem so as to ignore the embedded dummy variables.  If
this interests you, the system in \texttt{set.mm} obtained from \texttt{ax-1}
through \texttt{ax-c14} in \texttt{set.mm}, and deleting \texttt{ax-c16} and \texttt{ax-5},
requires no \texttt{\$d} statements but is logically complete in the sense
described in \cite{Megill}.  This means it can prove any theorem of
first-order logic as long as we add to the theorem an antecedent that embeds
dummy and any other variables that must be distinct.  In a similar fashion,
axioms for set theory can be devised that
do not require distinct variable
provisos\index{Set theory without distinct variable provisos},
as explained at
\url{http://us.metamath.org/mpeuni/mmzfcnd.html}.
Together, these in principle allow all of
mathematics to be developed under Metamath without a \texttt{\$d} statement,
although the length of the resulting theorems will grow as more and
more dummy variables become required in their proofs.

\subsection{The \texttt{\$f}
and \texttt{\$e} Statements}\label{dollaref}
\index{\texttt{\$e} statement}
\index{\texttt{\$f} statement}
\index{floating hypothesis}
\index{essential hypothesis}
\index{variable-type hypothesis}
\index{logical hypothesis}
\index{hypothesis}

Metamath has two kinds of hypo\-theses, the \texttt{\$f}\index{\texttt{\$f}
statement} or {\bf variable-type} hypothesis and the \texttt{\$e} or {\bf logical}
hypo\-the\-sis.\index{\texttt{\$d} statement}\footnote{Strictly speaking, the
\texttt{\$d} statement is also a hypothesis, but it is never directly referenced
in a proof, so we call it a restriction rather than a hypothesis to lessen
confusion.  The checking for violations of \texttt{\$d} restrictions is automatic
and built into Metamath's proof-checking algorithm.} The letters \texttt{f} and
\texttt{e} stand for ``floating''\index{floating hypothesis} (roughly meaning
used only if relevant) and ``essential''\index{essential hypothesis} (meaning
always used) respectively, for reasons that will become apparent
when we discuss frames in
Section~\ref{frames} and scoping in Section~\ref{scoping}. The syntax of these
are as follows:
\begin{center}
  {\em label} \texttt{\$f} {\em typecode} {\em variable} \texttt{\$.}\\
  {\em label} \texttt{\$e} {\em typecode}
      {\em math-symbol}\ \,$\cdots$\ {\em math-symbol} \texttt{\$.}\\
\end{center}
\index{\texttt{\$e} statement}
\index{\texttt{\$f} statement}
A hypothesis must have a {\em label}\index{label}.  The expression in a
\texttt{\$e} hypothesis consists of a typecode (an active constant math symbol)
followed by a sequence
of zero or more math symbols. Each math symbol (including {\em constant}
and {\em variable}) must be a previously declared constant or variable.  (In
addition, each math symbol must be active, which will be covered when we
discuss scoping statements in Section~\ref{scoping}.)  You use a \texttt{\$f}
hypothesis to specify the
nature or {\bf type}\index{variable type}\index{type} of a variable (such as ``let $x$ be an
integer'') and use a \texttt{\$e} hypothesis to express a logical truth (such as
``assume $x$ is prime'') that must be established in order for an assertion
requiring it to also be true.

A variable must have its type specified in a \texttt{\$f} statement before
it may be used in a \texttt{\$e}, \texttt{\$a}, or \texttt{\$p}
statement.  There may be only one (active) \texttt{\$f} statement for a
given variable.  (``Active'' is defined in Section~\ref{scoping}.)

In ordinary mathematics, theorems\index{theorem} are often expressed in the
form ``Assume $P$; then $Q$,'' where $Q$ is a statement that you can derive
if you start with statement $P$.\index{free variable}\footnote{A stronger
version of a theorem like this would be the {\em single} formula $P\rightarrow
Q$ ($P$ implies $Q$) from which the weaker version above follows by the rule
of modus ponens in logic.  We are not discussing this stronger form here.  In
the weaker form, we are saying only that if we can {\em prove} $P$, then we can
{\em prove} $Q$.  In a logician's language, if $x$ is the only free variable
in $P$ and $Q$, the stronger form is equivalent to $\forall x ( P \rightarrow
Q)$ (for all $x$, $P$ implies $Q$), whereas the weaker form is equivalent to
$\forall x P \rightarrow \forall x Q$. The stronger form implies the weaker,
but not vice-versa.  To be precise, the weaker form of the theorem is more
properly called an ``inference'' rather than a theorem.}\index{inference}
In the
Metamath\index{Metamath} language, you would express mathematical statement
$P$ as a hypothesis (a \texttt{\$e} Metamath language statement in this case) and
statement $Q$ as a provable assertion (a \texttt{\$p}\index{\texttt{\$p} statement}
statement).

Some examples of hypotheses you might encounter in logic and set theory are
\begin{center}
  \texttt{stmt1 \$f wff P \$.}\\
  \texttt{stmt2 \$f set x \$.}\\
  \texttt{stmt3 \$e |- ( P -> Q ) \$.}
\end{center}
\index{\texttt{\$e} statement}
\index{\texttt{\$f} statement}
Informally, these would be read, ``Let $P$ be a well-formed-formula,'' ``Let
$x$ be an (individual) variable,'' and ``Assume we have proved $P \rightarrow
Q$.''  The turnstile symbol \,$\vdash$\index{turnstile ({$\,\vdash$})} is
commonly used in logic texts to mean ``a proof exists for.''

To summarize:
\begin{itemize}
\item A \texttt{\$f} hypothesis tells Metamath the type or kind of its variable.
It is analogous to a variable declaration in a computer language that
tells the compiler that a variable is an integer or a floating-point
number.
\item The \texttt{\$e} hypothesis corresponds to what you would usually call a
``hypothesis'' in ordinary mathematics.
\end{itemize}

Before an assertion\index{assertion} (\texttt{\$a} or \texttt{\$p} statement) can be
referenced in a proof, all of its associated \texttt{\$f} and \texttt{\$e} hypotheses
(i.e.\ those \texttt{\$e} hypotheses that are active) must be satisfied (i.e.
established by the proof).  The meaning of ``associated'' (which we will call
{\bf mandatory} in Section~\ref{frames}) will become clear when we discuss
scoping later.

Note that after any \texttt{\$f}, \texttt{\$e},
\texttt{\$a}, or \texttt{\$p} token there is a required
\textit{typecode}\index{typecode}.
The typecode is a constant used to enforce types of expressions.
This will become clearer once we learn more about
assertions (\texttt{\$a} and \texttt{\$p} statements).
An example may also clarify their purpose.
In the
\texttt{set.mm}\index{set theory database (\texttt{set.mm})}%
\index{Metamath Proof Explorer}
database,
the following typecodes are used:

\begin{itemize}
\item \texttt{wff} :
  Well-formed formula (wff) symbol
  (read: ``the following symbol sequence is a wff'').
% The *textual* typecode for turnstile is "|-", but when read it's a little
% confusing, so I intentionally display the mathematical symbol here instead
% (I think it's clearer in this context).
\item \texttt{$\vdash$} :
  Turnstile (read: ``the following symbol sequence is provable'' or
  ``a proof exists for'').
\item \texttt{setvar} :
  Individual set variable type (read: ``the following is an
  individual set variable'').
  Note that this is \textit{not} the type of an arbitrary set expression,
  instead, it is used to ensure that there is only a single symbol used
  after quantifiers like for-all ($\forall$) and there-exists ($\exists$).
\item \texttt{class} :
  An expression that is a syntactically valid class expression.
  All valid set expressions are also valid class expression, so expressions
  of sets normally have the \texttt{class} typecode.
  Use the \texttt{class} typecode,
  \textit{not} the \texttt{setvar} typecode,
  for the type of set expressions unless you are specifically identifying
  a single set variable.
\end{itemize}

\subsection{Assertions (\texttt{\$a} and \texttt{\$p} Statements)}
\index{\texttt{\$a} statement}
\index{\texttt{\$p} statement}\index{assertion}\index{axiomatic assertion}
\index{provable assertion}

There are two types of assertions, \texttt{\$a}\index{\texttt{\$a} statement}
statements ({\bf axiomatic assertions}) and \texttt{\$p} statements ({\bf
provable assertions}).  Their syntax is as follows:
\begin{center}
  {\em label} \texttt{\$a} {\em typecode} {\em math-symbol} \ldots
         {\em math-symbol} \texttt{\$.}\\
  {\em label} \texttt{\$p} {\em typecode} {\em math-symbol} \ldots
        {\em math-symbol} \texttt{\$=} {\em proof} \texttt{\$.}
\end{center}
\index{\texttt{\$a} statement}
\index{\texttt{\$p} statement}
\index{\texttt{\$=} keyword}
An assertion always requires a {\em label}\index{label}. The expression in an
assertion consists of a typecode (an active constant)
followed by a sequence of zero
or more math symbols.  Each math symbol, including any {\em constant}, must be a
previously declared constant or variable.  (In addition, each math symbol
must be active, which will be covered when we discuss scoping statements in
Section~\ref{scoping}.)

A \texttt{\$a} statement is usually a definition of syntax (for example, if $P$
and $Q$ are wffs then so is $(P\to Q)$), an axiom\index{axiom} of ordinary
mathematics (for example, $x=x$), or a definition\index{definition} of
ordinary mathematics (for example, $x\ne y$ means $\lnot x=y$). A \texttt{\$p}
statement is a claim that a certain combination of math symbols follows from
previous assertions and is accompanied by a proof that demonstrates it.

Assertions can also be referenced in (later) proofs in order to derive new
assertions from them. The label of an assertion is used to refer to it in a
proof. Section~\ref{proof} will describe the proof in detail.

Assertions also provide the primary means for communicating the mathematical
results in the database to people.  Proofs (when conveniently displayed)
communicate to people how the results were arrived at.

\subsubsection{The \texttt{\$a} Statement}
\index{\texttt{\$a} statement}

Axiomatic assertions (\texttt{\$a} statements) represent the starting points from
which other assertions (\texttt{\$p}\index{\texttt{\$p} statement} statements) are
derived.  Their most obvious use is for specifying ordinary mathematical
axioms\index{axiom}, but they are also used for two other purposes.

First, Metamath\index{Metamath} needs to know the syntax of symbol
sequences that constitute valid mathematical statements.  A Metamath
proof must be broken down into much more detail than ordinary
mathematical proofs that you may be used to thinking of (even the
``complete'' proofs of formal logic\index{formal logic}).  This is one
of the things that makes Metamath a general-purpose language,
independent of any system of logic or even syntax.  If you want to use a
substitution instance of an assertion as a step in a proof, you must
first prove that the substitution is syntactically correct (or if you
prefer, you must ``construct'' it), showing for example that the
expression you are substituting for a wff metavariable is a valid wff.
The \texttt{\$a}\index{\texttt{\$a} statement} statement is used to
specify those combinations of symbols that are considered syntactically
valid, such as the legal forms of wffs.

Second, \texttt{\$a} statements are used to specify what are ordinarily thought of
as definitions, i.e.\ new combinations of symbols that abbreviate other
combinations of symbols.  Metamath makes no distinction\index{axiom vs.\
definition} between axioms\index{axiom} and definitions\index{definition}.
Indeed, it has been argued that such distinction should not be made even in
ordinary mathematics; see Section~\ref{definitions}, which discusses the
philosophy of definitions.  Section~\ref{hierarchy} discusses some
technical requirements for definitions.  In \texttt{set.mm} we adopt the
convention of prefixing axiom labels with \texttt{ax-} and definition labels with
\texttt{df-}\index{label}.

The results that can be derived with the Metamath language are only as good as
the \texttt{\$a}\index{\texttt{\$a} statement} statements used as their starting
point.  We cannot stress this too strongly.  For example, Metamath will
not prevent you from specifying $x\neq x$ as an axiom of logic.  It is
essential that you scrutinize all \texttt{\$a} statements with great care.
Because they are a source of potential pitfalls, it is best not to add new
ones (usually new definitions) casually; rather you should carefully evaluate
each one's necessity and advantages.

Once you have in place all of the basic axioms\index{axiom} and
rules\index{rule} of a mathematical theory, the only \texttt{\$a} statements that
you will be adding will be what are ordinarily called definitions.  In
principle, definitions should be in some sense eliminable from the language of
a theory according to some convention (usually involving logical equivalence
or equality).  The most common convention is that any formula that was
syntactically valid but not provable before the definition was introduced will
not become provable after the definition is introduced.  In an ideal world,
definitions should not be present at all if one is to have absolute confidence
in a mathematical result.  However, they are necessary to make
mathematics practical, for otherwise the resulting formulas would be
extremely long and incomprehensible.  Since the nature of definitions (in the
most general sense) does not permit them to automatically be verified as
``proper,''\index{proper definition}\index{definition!proper} the judgment of
the mathematician is required to ensure it.  (In \texttt{set.mm} effort was made
to make almost all definitions directly eliminable and thus minimize the need
for such judgment.)

If you are not a mathematician, it may be best not to add or change any
\texttt{\$a}\index{\texttt{\$a} statement} statements but instead use
the mathematical language already provided in standard databases.  This
way Metamath will not allow you to make a mistake (i.e.\ prove a false
result).


\subsection{Frames}\label{frames}

We now introduce the concept of a collection of related Metamath statements
called a frame.  Every assertion (\texttt{\$a} or \texttt{\$p} statement) in the database has
an associated frame.

A {\bf frame}\index{frame} is a sequence of \texttt{\$d}, \texttt{\$f},
and \texttt{\$e} statements (zero or more of each) followed by one
\texttt{\$a} or \texttt{\$p} statement, subject to certain conditions we
will describe.  For simplicity we will assume that all math symbol
tokens used are declared at the beginning of the database with
\texttt{\$c} and \texttt{\$v} statements (which are not properly part of
a frame).  Also for simplicity we will assume there are only simple
\texttt{\$d} statements (those with only two variables) and imagine any
compound \texttt{\$d} statements (those with more than two variables) as
broken up into simple ones.

A frame groups together those hypotheses (and \texttt{\$d} statements) relevant
to an assertion (\texttt{\$a} or \texttt{\$p} statement).  The statements in a frame
may or may not be physically adjacent in a database; we will cover
this in our discussion of scoping statements
in Section~\ref{scoping}.

A frame has the following properties:
\begin{enumerate}
 \item The set of variables contained in its \texttt{\$f} statements must
be identical to the set of variables contained in its \texttt{\$e},
\texttt{\$a}, and/or \texttt{\$p} statements.  In other words, each
variable in a \texttt{\$e}, \texttt{\$a}, or \texttt{\$p} statement must
have an associated ``variable type'' defined for it in a \texttt{\$f}
statement.
  \item No two \texttt{\$f} statements may contain the same variable.
  \item Any \texttt{\$f} statement
must occur before a \texttt{\$e} statement in which its variable occurs.
\end{enumerate}

The first property determines the set of variables occurring in a frame.
These are the {\bf mandatory
variables}\index{mandatory variable} of the frame.  The second property
tells us there must be only one type specified for a variable.
The last property is not a theoretical requirement but it
makes parsing of the database easier.

For our examples, we assume our database has the following declarations:

\begin{verbatim}
$v P Q R $.
$c -> ( ) |- wff $.
\end{verbatim}

The following sequence of statements, describing the modus ponens inference
rule, is an example of a frame:

\begin{verbatim}
wp  $f wff P $.
wq  $f wff Q $.
maj $e |- ( P -> Q ) $.
min $e |- P $.
mp  $a |- Q $.
\end{verbatim}

The following sequence of statements is not a frame because \texttt{R} does not
occur in the \texttt{\$e}'s or the \texttt{\$a}:

\begin{verbatim}
wp  $f wff P $.
wq  $f wff Q $.
wr  $f wff R $.
maj $e |- ( P -> Q ) $.
min $e |- P $.
mp  $a |- Q $.
\end{verbatim}

The following sequence of statements is not a frame because \texttt{Q} does not
occur in a \texttt{\$f}:

\begin{verbatim}
wp  $f wff P $.
maj $e |- ( P -> Q ) $.
min $e |- P $.
mp  $a |- Q $.
\end{verbatim}

The following sequence of statements is not a frame because the \texttt{\$a} statement is
not the last one:

\begin{verbatim}
wp  $f wff P $.
wq  $f wff Q $.
maj $e |- ( P -> Q ) $.
mp  $a |- Q $.
min $e |- P $.
\end{verbatim}

Associated with a frame is a sequence of {\bf mandatory
hypotheses}\index{mandatory hypothesis}.  This is simply the set of all
\texttt{\$f} and \texttt{\$e} statements in the frame, in the order they
appear.  A frame can be referenced in a later proof using the label of
the \texttt{\$a} or \texttt{\$p} assertion statement, and the proof
makes an assignment to each mandatory hypothesis in the order in which
it appears.  This means the order of the hypotheses, once chosen, must
not be changed so as not to affect later proofs referencing the frame's
assertion statement.  (The Metamath proof verifier will, of course, flag
an error if a proof becomes incorrect by doing this.)  Since proofs make
use of ``Reverse Polish notation,'' described in Section~\ref{proof}, we
call this order the {\bf RPN order}\index{RPN order} of the hypotheses.

Note that \texttt{\$d} statements are not part of the set of mandatory
hypotheses, and their order doesn't matter (as long as they satisfy the
fourth property for a frame described above).  The \texttt{\$d}
statements specify restrictions on variables that must be satisfied (and
are checked by the proof verifier) when expressions are substituted for
them in a proof, and the \texttt{\$d} statements themselves are never
referenced directly in a proof.

A frame with a \texttt{\$p} (provable) statement requires a proof as part of the
\texttt{\$p} statement.  Sometimes in a proof we want to make use of temporary or
dummy variables\index{dummy variable} that do not occur in the \texttt{\$p}
statement or its mandatory hypotheses.  To accommodate this we define an {\bf
extended frame}\index{extended frame} as a frame together with zero or more
\texttt{\$d} and \texttt{\$f} statements that reference variables not among the
mandatory variables of the frame.  Any new variables referenced are called the
{\bf optional variables}\index{optional variable} of the extended frame. If a
\texttt{\$f} statement references an optional variable it is called an {\bf
optional hypothesis}\index{optional hypothesis}, and if one or both of the
variables in a \texttt{\$d} statement are optional variables it is called an {\bf
optional disjoint-variable restriction}\index{optional disjoint-variable
restriction}.  Properties 2, 3, and 4 for a frame also apply to an extended
frame.

The concept of optional variables is not meaningful for frames with \texttt{\$a}
statements, since those statements have no proofs that might make use of them.
There is no restriction on including optional hypotheses in the extended frame
for a \texttt{\$a} statement, but they serve no purpose.

The following set of statements is an example of an extended frame, which
contains an optional variable \texttt{R} and an optional hypothesis \texttt{wr}.  In
this example, we suppose the rule of modus ponens is not an axiom but is
derived as a theorem from earlier statements (we omit its presumed proof).
Variable \texttt{R} may be used in its proof if desired (although this would
probably have no advantage in propositional calculus).  Note that the sequence
of mandatory hypotheses in RPN order is still \texttt{wp}, \texttt{wq}, \texttt{maj},
\texttt{min} (i.e.\ \texttt{wr} is omitted), and this sequence is still assumed
whenever the assertion \texttt{mp} is referenced in a subsequent proof.

\begin{verbatim}
wp  $f wff P $.
wq  $f wff Q $.
wr  $f wff R $.
maj $e |- ( P -> Q ) $.
min $e |- P $.
mp  $p |- Q $= ... $.
\end{verbatim}

Every frame is an extended frame, but not every extended frame is a frame, as
this example shows.  The underlying frame for an extended frame is
obtained by simply removing all statements containing optional variables.
Any proof referencing an assertion will ignore any extensions to its
frame, which means we may add or delete optional hypotheses at will without
affecting subsequent proofs.

The conceptually simplest way of organizing a Metamath database is as a
sequence of extended frames.  The scoping statements
\texttt{\$\char`\{}\index{\texttt{\$\char`\{} and \texttt{\$\char`\}}
keywords} and \texttt{\$\char`\}} can be used to delimit the start and
end of an extended frame, leading to the following possible structure for a
database.  \label{framelist}

\vskip 2ex
\setbox\startprefix=\hbox{\tt \ \ \ \ \ \ \ \ }
\setbox\contprefix=\hbox{}
\startm
\m{\mbox{(\texttt{\$v} {\em and} \texttt{\$c}\,{\em statements})}}
\endm
\startm
\m{\mbox{\texttt{\$\char`\{}}}
\endm
\startm
\m{\mbox{\texttt{\ \ } {\em extended frame}}}
\endm
\startm
\m{\mbox{\texttt{\$\char`\}}}}
\endm
\startm
\m{\mbox{\texttt{\$\char`\{}}}
\endm
\startm
\m{\mbox{\texttt{\ \ } {\em extended frame}}}
\endm
\startm
\m{\mbox{\texttt{\$\char`\}}}}
\endm
\startm
\m{\mbox{\texttt{\ \ \ \ \ \ \ \ \ }}\vdots}
\endm
\vskip 2ex

In practice, this structure is inconvenient because we have to repeat
any \texttt{\$f}, \texttt{\$e}, and \texttt{\$d} statements over and
over again rather than stating them once for use by several assertions.
The scoping statements, which we will discuss next, allow this to be
done.  In principle, any Metamath database can be converted to the above
format, and the above format is the most convenient to use when studying
a Metamath database as a formal system%
%% Uncomment this when uncommenting section {formalspec} below
   (Appendix \ref{formalspec})%
.
In fact, Metamath internally converts the database to the above format.
The command \texttt{show statement} in the Metamath program will show
you the contents of the frame for any \texttt{\$a} or \texttt{\$p}
statement, as well as its extension in the case of a \texttt{\$p}
statement.

%c%(provided that all ``local'' variables and constants with limited scope have
%c%unique names),

During our discussion of scoping statements, it may be helpful to
think in terms of the equivalent sequence of frames that will result when
the database is parsed.  Scoping (other than the limited
use above to delimit frames) is not a theoretical requirement for
Metamath but makes it more convenient.


\subsection{Scoping Statements (\texttt{\$\{} and \texttt{\$\}})}\label{scoping}
\index{\texttt{\$\char`\{} and \texttt{\$\char`\}} keywords}\index{scoping statement}

%c%Some Metamath statements may be needed only temporarily to
%c%serve a specific purpose, and after we're done with them we would like to
%c%disregard or ignore them.  For example, when we're finished using a variable,
%c%we might want to
%c%we might want to free up the token\index{token} used to name it so that the
%c%token can be used for other purposes later on, such as a different kind of
%c%variable or even a constant.  In the terminology of computer programming, we
%c%might want to let some symbol declarations be ``local'' rather than ``global.''
%c%\index{local symbol}\index{global symbol}

The {\bf scoping} statements, \texttt{\$\char`\{} ({\bf start of block}) and \texttt{\$\char`\}}
({\bf end of block})\index{block}, provide a means for controlling the portion
of a database over which certain statement types are recognized.  The
syntax of a scoping statement is very simple; it just consists of the
statement's keyword:
\begin{center}
\texttt{\$\char`\{}\\
\texttt{\$\char`\}}
\end{center}
\index{\texttt{\$\char`\{} and \texttt{\$\char`\}} keywords}

For example, consider the following database where we have stripped out
all tokens except the scoping statement keywords.  For the purpose of the
discussion, we have added subscripts to the scoping statements; these subscripts
do not appear in the actual database.
\[
 \mbox{\tt \ \$\char`\{}_1
 \mbox{\tt \ \$\char`\{}_2
 \mbox{\tt \ \$\char`\}}_2
 \mbox{\tt \ \$\char`\{}_3
 \mbox{\tt \ \$\char`\{}_4
 \mbox{\tt \ \$\char`\}}_4
 \mbox{\tt \ \$\char`\}}_3
 \mbox{\tt \ \$\char`\}}_1
\]
Each \texttt{\$\char`\{} statement in this example is said to be {\bf
matched} with the \texttt{\$\char`\}} statement that has the same
subscript.  Each pair of matched scoping statements defines a region of
the database called a {\bf block}.\index{block} Blocks can be {\bf
nested}\index{nested block} inside other blocks; in the example, the
block defined by $\mbox{\tt \$\char`\{}_4$ and $\mbox{\tt \$\char`\}}_4$
is nested inside the block defined by $\mbox{\tt \$\char`\{}_3$ and
$\mbox{\tt \$\char`\}}_3$ as well as inside the block defined by
$\mbox{\tt \$\char`\{}_1$ and $\mbox{\tt \$\char`\}}_1$.  In general, a
block may be empty, it may contain only non-scoping
statements,\footnote{Those statements other than \texttt{\$\char`\{} and
\texttt{\$\char`\}}.}\index{non-scoping statement} or it may contain any
mixture of other blocks and non-scoping statements.  (This is called a
``recursive'' definition\index{recursive definition} of a block.)

Associated with each block is a number called its {\bf nesting
level}\index{nesting level} that indicates how deeply the block is nested.
The nesting levels of the blocks in our example are as follows:
\[
  \underbrace{
    \mbox{\tt \ }
    \underbrace{
     \mbox{\tt \$\char`\{\ }
     \underbrace{
       \mbox{\tt \$\char`\{\ }
       \mbox{\tt \$\char`\}}
     }_{2}
     \mbox{\tt \ }
     \underbrace{
       \mbox{\tt \$\char`\{\ }
       \underbrace{
         \mbox{\tt \$\char`\{\ }
         \mbox{\tt \$\char`\}}
       }_{3}
       \mbox{\tt \ \$\char`\}}
     }_{2}
     \mbox{\tt \ \$\char`\}}
   }_{1}
   \mbox{\tt \ }
 }_{0}
\]
\index{\texttt{\$\char`\{} and \texttt{\$\char`\}} keywords}
The entire database is considered to be one big block (the {\bf outermost}
block) with a nesting level of 0.  The outermost block is {\em not} bracketed
by scoping statements.\footnote{The language was designed this way so that
several source files can be joined together more easily.}\index{outermost
block}

All non-scoping Metamath statements become recognized or {\bf
active}\index{active statement} at the place where they appear.\footnote{To
keep things slightly simpler, we do not bother to define the concept of
``active'' for the scoping statements.}  Certain of these statement types
become inactive at the end of the block in which they appear; these statement
types are:
\begin{center}
  \texttt{\$c}, \texttt{\$v}, \texttt{\$d}, \texttt{\$e}, and \texttt{\$f}.
%  \texttt{\$v}, \texttt{\$f}, \texttt{\$e}, and \texttt{\$d}.
\end{center}
\index{\texttt{\$c} statement}
\index{\texttt{\$d} statement}
\index{\texttt{\$e} statement}
\index{\texttt{\$f} statement}
\index{\texttt{\$v} statement}
The other statement types remain active forever (i.e.\ through the end of the
database); they are:
\begin{center}
  \texttt{\$a} and \texttt{\$p}.
%  \texttt{\$c}, \texttt{\$a}, and \texttt{\$p}.
\end{center}
\index{\texttt{\$a} statement}
\index{\texttt{\$p} statement}
Any statement (of these 7 types) located in the outermost
block\index{outermost block} will remain active through the end of the
database and thus are effectively ``global'' statements.\index{global
statement}

All \texttt{\$c} statements must be placed in the outermost block.  Since they are
therefore always global, they could be considered as belonging to both of the
above categories.

The {\bf scope}\index{scope} of a statement is the set of statements that
recognize it as active.

%c%The concept of ``active'' is also defined for math symbols\index{math
%c%symbol}.  Math symbols (constants\index{constant} and
%c%variables\index{variable}) become {\bf active}\index{active
%c%math symbol} in the \texttt{\$c}\index{\texttt{\$c}
%c%statement} and \texttt{\$v}\index{\texttt{\$v} statement} statements that
%c%declare them.  They become inactive when their declaration statements become
%c%inactive.

The concept of ``active'' is also defined for math symbols\index{math
symbol}.  Math symbols (constants\index{constant} and
variables\index{variable}) become {\bf active}\index{active math symbol}
in the \texttt{\$c}\index{\texttt{\$c} statement} and
\texttt{\$v}\index{\texttt{\$v} statement} statements that declare them.
A variable becomes inactive when its declaration statement becomes
inactive.  Because all \texttt{\$c} statements must be in the outermost
block, a constant will never become inactive after it is declared.

\subsubsection{Redeclaration of Math Symbols}
\index{redeclaration of symbols}\label{redeclaration}

%c%A math symbol may not be declared a second time while it is active, but it may
%c%be declared again after it becomes inactive.

A variable may not be declared a second time while it is active, but it may be
declared again after it becomes inactive.  This provides a convenient way to
introduce ``local'' variables,\index{local variable} i.e.\ temporary variables
for use in the frame of an assertion or in a proof without keeping them around
forever.  A previously declared variable may not be redeclared as a constant.

A constant may not be redeclared.  And, as mentioned above, constants must be
declared in the outermost block.

The reason variables may have limited scope but not constants is that an
assertion (\texttt{\$a} or \texttt{\$p} statement) remains available for use in
proofs through the end of the database.  Variables in an assertion's frame may
be substituted with whatever is needed in a proof step that references the
assertion, whereas constants remain fixed and may not be substituted with
anything.  The particular token used for a variable in an assertion's frame is
irrelevant when the assertion is referenced in a proof, and it doesn't matter
if that token is not available outside of the referenced assertion's frame.
Constants, however, must be globally fixed.

In the present version of the Metamath language, there is no theoretical
benefit for the feature allowing variables to be active for limited scopes
rather than global. It is just a convenience that allows them, for example, to
be locally grouped together with their corresponding \texttt{\$f} variable-type
declarations.\footnote{A future version\label{redeclarationf} of the Metamath
language may extend the language to include true local variables as well as
local constants.  I have not decided upon the best way to do this or even if
it should be done at all; the main issue is whether the benefits offset
greater confusion in learning the language.  In any case, the rules for
the present language were chosen to let existing databases be compatible with
any future language extensions.}

%c%If you declare a math symbol more than once, internally Metamath considers it a
%c%new distinct symbol, even though it has the same name.  If you are unaware of
%c%this, you may find that what you think are correct proofs are incorrectly
%c%rejected as invalid, because Metamath may tell you that a constant you
%c%previously declared does not match a newly declared math symbol with the same
%c%name.  For details on this subtle point, see the Comment on
%c%p.~\pageref{spec4comment}.  This is done purposely to allow temporary
%c%constants to be introduced while developing a subtheory, then allow their math
%c%symbol tokens to be reused later on; in general they will not refer to the
%c%same thing.  In practice, you would not ordinarily reuse the names of
%c%constants because it would tend to be confusing to the reader.  The reuse of
%c%names of variables, on the other hand, is something that is often useful to do
%c%(for example it is done frequently in \texttt{set.mm}).  Since variables in an
%c%assertion referenced in a proof can be substituted as needed to achieve a
%c%symbol match, this is not an issue.

% (This section covers a somewhat advanced topic you may want to skip
% at first reading.)
%
% Under certain circumstances, math symbol\index{math symbol}
% tokens\index{token} may be redeclared (i.e.\ the token
% may appear in more than
% one \texttt{\$c}\index{\texttt{\$c} statement} or \texttt{\$v}\index{\texttt{\$v}
% statement} statement).  You might want to do this say, to make temporary use
% of a variable name without having to worry about its affect elsewhere,
% somewhat analogous to declaring a local variable in a standard computer
% language.  Understanding what goes on when math symbol tokens are redeclared
% is a little tricky to understand at first, since it requires that we
% distinguish the token itself from the math symbol that it names.  It will help
% if we first take a peek at the internal workings of the
% Metamath\index{Metamath} program.
%
% Metamath reserves a memory location for each occurrence of a
% token\index{token} in a declaration statement (\texttt{\$c}\index{\texttt{\$c}
% statement} or \texttt{\$v}\index{\texttt{\$v} statement}).  If a given token appears
% in more than one declaration statement, it will refer to more than one memory
% locations.  A math symbol\index{math symbol} may be thought of as being one of
% these memory locations rather than as the token itself.  Only one of the
% memory locations associated with a given token may be active at any one time.
% The math symbol (memory location) that gets looked up when the token appears
% in a non-declaration statement is the one that happens to be active at that
% time.
%
% We now look at the rules for the redeclaration\index{redeclaration of symbols}
% of math symbol tokens.
% \begin{itemize}
% \item A math symbol token may not be declared twice in the
% same block.\footnote{While there is no theoretical reason for disallowing
% this, it was decided in the design of Metamath that allowing it would offer no
% advantage and might cause confusion.}
% \item An inactive math symbol may always be
% redeclared.
% \item  An active math symbol may be redeclared in a different (i.e.\
% inner) block\index{block} from the one it became active in.
% \end{itemize}
%
% When a math symbol token is redeclared, it conceptually refers to a different
% math symbol, just as it would be if it were called a different name.  In
% addition, the original math symbol that it referred to, if it was active,
% temporarily becomes inactive.  At the end of the block in which the
% redeclaration occurred, the new math symbol\index{math symbol} becomes
% inactive and the original symbol becomes active again.  This concept is
% illustrated in the following example, where the symbol \texttt{e} is
% ordinarily a constant (say Euler's constant, 2.71828...) but
% temporarily we want to use it as a ``local'' variable, say as a coefficient
% in the equation $a x^4 + b x^3 + c x^2 + d x + e$:
% \[
%   \mbox{\tt \$\char`\{\ \$c e \$.}
%   \underbrace{
%     \ \ldots\ %
%     \mbox{\tt \$\char`\{}\ \ldots\ %
%   }_{\mbox{\rm region A}}
%   \mbox{\tt \$v e \$.}
%   \underbrace{
%     \mbox{\ \ \ \ldots\ \ \ }
%   }_{\mbox{\rm region B}}
%   \mbox{\tt \$\char`\}}
%   \underbrace{
%     \mbox{\ \ \ \ldots\ \ \ }
%   }_{\mbox{\rm region C}}
%   \mbox{\tt \$\char`\}}
% \]
% \index{\texttt{\$\char`\{} and \texttt{\$\char`\}} keywords}
% In region A, the token \texttt{e} refers to a constant.  It is redeclared as a
% variable in region B, and any reference to it in this region will refer to this
% variable.  In region C, the redeclaration becomes inactive, and the original
% declaration becomes active again.  In region C, the token \texttt{x} refers to the
% original constant.
%
% As a practical matter, overuse of math symbol\index{math symbol}
% redeclarations\index{redeclaration of symbols} can be confusing (even though
% it is well-defined) and is best avoided when possible.  Here are some good
% general guidelines you can follow.  Usually, you should declare all
% constants\index{constant} in the outermost block\index{outermost block},
% especially if they are general-purpose (such as the token \verb$A.$, meaning
% $\forall$ or ``for all'').  This will make them ``globally'' active (although
% as in the example above local redeclarations will temporarily make them
% inactive.)  Most or all variables\index{variable}, on the other hand, could be
% declared in inner blocks, so that the token for them can be used later for a
% different type of variable or a constant.  (The names of the variables you
% choose are not used when you refer to an assertion\index{assertion} in a
% proof, whereas constants must match exactly.  A locally declared constant will
% not match a globally declared constant in a proof, even if they use the same
% token, because Metamath internally considers them to be different math
% symbols.)  To avoid confusion, you should generally avoid redeclaring active
% variables.  If you must redeclare them, do so at the beginning of a block.
% The temporary declaration of constants in inner blocks might be occasionally
% appropriate when you make use of a temporary definition to prove lemmas
% leading to a main result that does not make direct use of the definition.
% This way, you will not clutter up your database with a large number of
% seldom-used global constant symbols.  You might want to note that while
% inactive constants may not appear directly in an assertion (a \texttt{\$a}\index{\texttt{\$a}
% statement} or \texttt{\$p}\index{\texttt{\$p} statement}
% statement), they may be indirectly used in the proof of a \texttt{\$p} statement
% so long as they do not appear in the final math symbol sequence constructed by
% the proof.  In the end, you will have to use your best judgment, taking into
% account standard mathematical usage of the symbols as well as consideration
% for the reader of your work.
%
% \subsubsection{Reuse of Labels}\index{reuse of labels}\index{label}
%
% The \texttt{\$e}\index{\texttt{\$e} statement}, \texttt{\$f}\index{\texttt{\$f}
% statement}, \texttt{\$a}\index{\texttt{\$a} statement}, and
% \texttt{\$p}\index{\texttt{\$p}
% statement} statement types require labels, which allow them to be
% referenced later inside proofs.  A label is considered {\bf
% active}\index{active label} when the statement it is associated with is
% active.  The token\index{token} for a label may be reused
% (redeclared)\index{redeclaration of labels} provided that it is not being used
% for a currently active label.  (Unlike the tokens for math symbols, active
% label tokens may not be redeclared in an inner scope.)  Note that the labels
% of \texttt{\$a} and \texttt{\$p} statements can never be reused after these
% statements appear, because these statements remain active through the end of
% the database.
%
% You might find the reuse of labels a convenient way to have standard names for
% temporary hypotheses, such as \texttt{h1}, \texttt{h2}, etc.  This way you don't have
% to invent unique names for each of them, and in some cases it may be less
% confusing to the reader (although in other cases it might be more confusing, if
% the hypothesis is located far away from the assertion that uses
% it).\footnote{The current implementation requires that all labels, even
% inactive ones, be unique.}

\subsubsection{Frames Revisited}\index{frames and scoping statements}

Now that we have covered scoping, we will look at how an arbitrary
Metamath database can be converted to the simple sequence of extended
frames described on p.~\pageref{framelist}.  This is also how Metamath
stores the database internally when it reads in the database
source.\label{frameconvert} The method is simple.  First, we collect all
constant and variable (\texttt{\$c} and \texttt{\$v}) declarations in
the database, ignoring duplicate declarations of the same variable in
different scopes.  We then put our collected \texttt{\$c} and
\texttt{\$v} declarations at the beginning of the database, so that
their scope is the entire database.  Next, for each assertion in the
database, we determine its frame and extended frame.  The extended frame
is simply the \texttt{\$f}, \texttt{\$e}, and \texttt{\$d} statements
that are active.  The frame is the extended frame with all optional
hypotheses removed.

An equivalent way of saying this is that the extended frame of an assertion
is the collection of all \texttt{\$f}, \texttt{\$e}, and \texttt{\$d} statements
whose scope includes the assertion.
The \texttt{\$f} and \texttt{\$e} statements
occur in the order they appear
(order is irrelevant for \texttt{\$d} statements).

%c%, renaming any
%c%redeclared variables as needed so that all of them have unique names.  (The
%c%exact renaming convention is unimportant.  You might imagine renaming
%c%different declarations of math symbol \texttt{a} as \texttt{a\$1}, \texttt{a\$2}, etc.\
%c%which would prevent any conflicts since \texttt{\$} is not a legal character in a
%c%math symbol token.)

\section{The Anatomy of a Proof} \label{proof}
\index{proof!Metamath, description of}

Each provable assertion (\texttt{\$p}\index{\texttt{\$p} statement} statement) in a
database must include a {\bf proof}\index{proof}.  The proof is located
between the \texttt{\$=}\index{\texttt{\$=} keyword} and \texttt{\$.}\ keywords in the
\texttt{\$p} statement.

In the basic Metamath language\index{basic language}, a proof is a
sequence of statement labels.  This label sequence\index{label sequence}
serves as a set of instructions that the Metamath program uses to
construct a series of math symbol sequences.  The construction must
ultimately result in the math symbol sequence contained between the
\texttt{\$p}\index{\texttt{\$p} statement} and
\texttt{\$=}\index{\texttt{\$=} keyword} keywords of the \texttt{\$p}
statement.  Otherwise, the Metamath program will consider the proof
incorrect, and it will notify you with an appropriate error message when
you ask it to verify the proof.\footnote{To make the loading faster, the
Metamath program does not automatically verify proofs when you
\texttt{read} in a database unless you use the \texttt{/verify}
qualifier.  After a database has been read in, you may use the
\texttt{verify proof *} command to verify proofs.}\index{\texttt{verify
proof} command} Each label in a proof is said to {\bf
reference}\index{label reference} its corresponding statement.

Associated with any assertion\index{assertion} (\texttt{\$p} or
\texttt{\$a}\index{\texttt{\$a} statement} statement) is a set of
hypotheses (\texttt{\$f}\index{\texttt{\$f} statement} or
\texttt{\$e}\index{\texttt{\$e} statement} statements) that are active
with respect to that assertion.  Some are mandatory and the others are
optional.  You should review these concepts if necessary.

Each label\index{label} in a proof must be either the label of a
previous assertion (\texttt{\$a}\index{\texttt{\$a} statement} or
\texttt{\$p}\index{\texttt{\$p} statement} statement) or the label of an
active hypothesis (\texttt{\$e} or \texttt{\$f}\index{\texttt{\$f}
statement} statement) of the \texttt{\$p} statement containing the
proof.  Hypothesis labels may reference both the
mandatory\index{mandatory hypothesis} and the optional hypotheses of the
\texttt{\$p} statement.

The label sequence in a proof specifies a construction in {\bf reverse Polish
notation}\index{reverse Polish notation (RPN)} (RPN).  You may be familiar
with RPN if you have used older
Hewlett--Packard or similar hand-held calculators.
In the calculator analogy, a hypothesis label\index{hypothesis label} is like
a number and an assertion label\index{assertion label} is like an operation
(more precisely, an $n$-ary operation when the
assertion has $n$ \texttt{\$e}-hypotheses).
On an RPN calculator, an operation takes one or more previous numbers in an
input sequence, performs a calculation on them, and replaces those numbers and
itself with the result of the calculation.  For example, the input sequence
$2,3,+$ on an RPN calculator results in $5$, and the input sequence
$2,3,5,{\times},+$ results in $2,15,+$ which results in $17$.

Understanding how RPN is processed involves the concept of a {\bf
stack}\index{stack}\index{RPN stack}, which can be thought of as a set of
temporary memory locations that hold intermediate results.  When Metamath
encounters a hypothesis label it places or {\bf pushes}\index{push} the math
symbol sequence of the hypothesis onto the stack.  When Metamath encounters an
assertion label, it associates the most recent stack entries with the {\em
mandatory} hypotheses\index{mandatory hypothesis} of the assertion, in the
order where the most recent stack entry is associated with the last mandatory
hypothesis of the assertion.  It then determines what
substitutions\index{substitution!variable}\index{variable substitution} have
to be made into the variables of the assertion's mandatory hypotheses to make
them identical to the associated stack entries.  It then makes those same
substitutions into the assertion itself.  Finally, Metamath removes or {\bf
pops}\index{pop} the matched hypotheses from the stack and pushes the
substituted assertion onto the stack.

For the purpose of matching the mandatory hypothesis to the most recent stack
entries, whether a hypothesis is a \texttt{\$e} or \texttt{\$f} statement is
irrelevant.  The only important thing is that a set of
substitutions\footnote{In the Metamath spec (Section~\ref{spec}), we use the
singular term ``substitution'' to refer to the set of substitutions we talk
about here.} exist that allow a match (and if they don't, the proof verifier
will let you know with an error message).  The Metamath language is specified
in such a way that if a set of substitutions exists, it will be unique.
Specifically, the requirement that each variable have a type specified for it
with a \texttt{\$f} statement ensures the uniqueness.

We will illustrate this with an example.
Consider the following Metamath source file:
\begin{verbatim}
$c ( ) -> wff $.
$v p q r s $.
wp $f wff p $.
wq $f wff q $.
wr $f wff r $.
ws $f wff s $.
w2 $a wff ( p -> q ) $.
wnew $p wff ( s -> ( r -> p ) ) $= ws wr wp w2 w2 $.
\end{verbatim}
This Metamath source example shows the definition and ``proof'' (i.e.,
construction) of a well-formed formula (wff)\index{well-formed formula (wff)}
in propositional calculus.  (You may wish to type this example into a file to
experiment with the Metamath program.)  The first two statements declare
(introduce the names of) four constants and four variables.  The next four
statements specify the variable types, namely that
each variable is assumed to be a wff.  Statement \texttt{w2} defines (postulates)
a way to produce a new wff, \texttt{( p -> q )}, from two given wffs \texttt{p} and
\texttt{q}. The mandatory hypotheses of \texttt{w2} are \texttt{wp} and \texttt{wq}.
Statement \texttt{wnew} claims that \texttt{( s -> ( r -> p ) )} is a wff given
three wffs \texttt{s}, \texttt{r}, and \texttt{p}.  More precisely, \texttt{wnew} claims
that the sequence of ten symbols \texttt{wff ( s -> ( r -> p ) )} is provable from
previous assertions and the hypotheses of \texttt{wnew}.  Metamath does not know
or care what a wff is, and as far as it is concerned
the typecode \texttt{wff} is just an
arbitrary constant symbol in a math symbol sequence.  The mandatory hypotheses
of \texttt{wnew} are \texttt{wp}, \texttt{wr}, and \texttt{ws}; \texttt{wq} is an optional
hypothesis.  In our particular proof, the optional hypothesis is not
referenced, but in general, any combination of active (i.e.\ optional and
mandatory) hypotheses could be referenced.  The proof of statement \texttt{wnew}
is the sequence of five labels starting with \texttt{ws} (step~1) and ending with
\texttt{w2} (step~5).

When Metamath verifies the proof, it scans the proof from left to right.  We
will examine what happens at each step of the proof.  The stack starts off
empty.  At step 1, Metamath looks up label \texttt{ws} and determines that it is a
hypothesis, so it pushes the symbol sequence of statement \texttt{ws} onto the
stack:

\begin{center}\begin{tabular}{|l|l|}\hline
{Stack location} & {Contents} \\ \hline \hline
1 & \texttt{wff s} \\ \hline
\end{tabular}\end{center}

Metamath sees that the labels \texttt{wr} and \texttt{wp} in steps~2 and 3 are also
hypotheses, so it pushes them onto the stack.  After step~3, the stack looks
like
this:

\begin{center}\begin{tabular}{|l|l|}\hline
{Stack location} & {Contents} \\ \hline \hline
3 & \texttt{wff p} \\ \hline
2 & \texttt{wff r} \\ \hline
1 & \texttt{wff s} \\ \hline
\end{tabular}\end{center}

At step 4, Metamath sees that label \texttt{w2} is an assertion, so it must do
some processing.  First, it associates the mandatory hypotheses of \texttt{w2},
which are \texttt{wp} and \texttt{wq}, with stack locations~2 and 3, {\em in that
order}. Metamath determines that the only possible way
to make hypothesis \texttt{wp} match (become identical to) stack location~2 and
\texttt{wq} match stack location 3 is to substitute variable \texttt{p} with \texttt{r}
and \texttt{q} with \texttt{p}.  Metamath makes these substitutions into \texttt{w2} and
obtains the symbol sequence \texttt{wff ( r -> p )}.  It removes the hypotheses
from stack locations~2 and 3, then places the result into stack location~2:

\begin{center}\begin{tabular}{|l|l|}\hline
{Stack location} & {Contents} \\ \hline \hline
2 & \texttt{wff ( r -> p )} \\ \hline
1 & \texttt{wff s} \\ \hline
\end{tabular}\end{center}

At step 5, Metamath sees that label \texttt{w2} is an assertion, so it must again
do some processing.  First, it matches the mandatory hypotheses of \texttt{w2},
which are \texttt{wp} and \texttt{wq}, to stack locations 1 and 2.
Metamath determines that the only possible way to make the
hypotheses match is to substitute variable \texttt{p} with \texttt{s} and \texttt{q} with
\texttt{( r -> p )}.  Metamath makes these substitutions into \texttt{w2} and obtains
the symbol
sequence \texttt{wff ( s -> ( r -> p ) )}.  It removes stack
locations 1 and 2, then places the result into stack location~1:

\begin{center}\begin{tabular}{|l|l|}\hline
{Stack location} & {Contents} \\ \hline \hline
1 & \texttt{wff ( s -> ( r -> p ) )} \\ \hline
\end{tabular}\end{center}

After Metamath finishes processing the proof, it checks to see that the
stack contains exactly one element and that this element is
the same as the math symbol sequence in the
\texttt{\$p}\index{\texttt{\$p} statement} statement.  This is the case for our
proof of \texttt{wnew},
so we have proved \texttt{wnew} successfully.  If the result
differs, Metamath will notify you with an error message.  An error message
will also result if the stack contains more than one entry at the end of the
proof, or if the stack did not contain enough entries at any point in the
proof to match all of the mandatory hypotheses\index{mandatory hypothesis} of
an assertion.  Finally, Metamath will notify you with an error message if no
substitution is possible that will make a referenced assertion's hypothesis
match the
stack entries.  You may want to experiment with the different kinds of errors
that Metamath will detect by making some small changes in the proof of our
example.

Metamath's proof notation was designed primarily to express proofs in a
relatively compact manner, not for readability by humans.  Metamath can display
proofs in a number of different ways with the \texttt{show proof}\index{\texttt{show
proof} command} command.  The
\texttt{/lemmon} qualifier displays it in a format that is easier to read when the
proofs are short, and you saw examples of its use in Chapter~\ref{using}.  For
longer proofs, it is useful to see the tree structure of the proof.  A tree
structure is displayed when the \texttt{/lemmon} qualifier is omitted.  You will
probably find this display more convenient as you get used to it. The tree
display of the proof in our example looks like
this:\label{treeproof}\index{tree-style proof}\index{proof!tree-style}
\begin{verbatim}
1     wp=ws    $f wff s
2        wp=wr    $f wff r
3        wq=wp    $f wff p
4     wq=w2    $a wff ( r -> p )
5  wnew=w2  $a wff ( s -> ( r -> p ) )
\end{verbatim}
The number to the left of each line is the step number.  Following it is a
{\bf hypothesis association}\index{hypothesis association}, consisting of two
labels\index{label} separated by \texttt{=}.  To the left of the \texttt{=} (except
in the last step) is the label of a hypothesis of an assertion referenced
later in the proof; here, steps 1 and 4 are the hypothesis associations for
the assertion \texttt{w2} that is referenced in step 5.  A hypothesis association
is indented one level more than the assertion that uses it, so it is easy to
find the corresponding assertion by moving directly down until the indentation
level decreases to one less than where you started from.  To the right of each
\texttt{=} is the proof step label for that proof step.  The statement keyword of
the proof step label is listed next, followed by the content of the top of the
stack (the most recent stack entry) as it exists after that proof step is
processed.  With a little practice, you should have no trouble reading proofs
displayed in this format.

Metamath proofs include the syntax construction of a formula.
In standard mathematics, this kind of
construction is not considered a proper part of the proof at all, and it
certainly becomes rather boring after a while.
Therefore,
by default the \texttt{show proof}\index{\texttt{show proof}
command} command does not show the syntax construction.
Historically \texttt{show proof} command
\textit{did} show the syntax construction, and you needed to add the
\texttt{/essential} option to hide, them, but today
\texttt{/essential} is the default and you need to use
\texttt{/all} to see the syntax constructions.

When verifying a proof, Metamath will check that no mandatory
\texttt{\$d}\index{\texttt{\$d} statement}\index{mandatory \texttt{\$d}
statement} statement of an assertion referenced in a proof is violated
when substitutions\index{substitution!variable}\index{variable
substitution} are made to the variables in the assertion.  For details
see Section~\ref{spec4} or \ref{dollard}.

\subsection{The Concept of Unification} \label{unify}

During the course of verifying a proof, when Metamath\index{Metamath}
encounters an assertion label\index{assertion label}, it associates the
mandatory hypotheses\index{mandatory hypothesis} of the assertion with the top
entries of the RPN stack\index{stack}\index{RPN stack}.  Metamath then
determines what substitutions\index{substitution!variable}\index{variable
substitution} it must make to the variables in the assertion's mandatory
hypotheses in order for these hypotheses to become identical to their
corresponding stack entries.  This process is called {\bf
unification}\index{unification}.  (We also informally use the term
``unification'' to refer to a set of substitutions that results from the
process, as in ``two unifications are possible.'')  After the substitutions
are made, the hypotheses are said to be {\bf unified}.

If no such substitutions are possible, Metamath will consider the proof
incorrect and notify you with an error message.
% (deleted 3/10/07, per suggestion of Mel O'Cat:)
% The syntax of the
% Metamath language ensures that if a set of substitutions exists, it
% will be unique.

The general algorithm for unification described in the literature is
somewhat complex.
However, in the case of Metamath it is intentionally trivial.
Mandatory hypotheses must be
pushed on the proof stack in the order in which they appear.
In addition, each variable must have its type specified
with a \texttt{\$f} hypothesis before it is used
and that each \texttt{\$f} hypothesis
have the restricted syntax of a typecode (a constant) followed by a variable.
The typecode in the \texttt{\$f} hypothesis must match the first symbol of
the corresponding RPN stack entry (which will also be a constant), so
the only possible match for the variable in the \texttt{\$f} hypothesis is
the sequence of symbols in the stack entry after the initial constant.

In the Proof Assistant\index{Proof Assistant}, a more general unification
algorithm is used.  While a proof is being developed, sometimes not enough
information is available to determine a unique unification.  In this case
Metamath will ask you to pick the correct one.\index{ambiguous
unification}\index{unification!ambiguous}

\section{Extensions to the Metamath Language}\index{extended
language}

\subsection{Comments in the Metamath Language}\label{comments}
\index{markup notation}
\index{comments!markup notation}

The commenting feature allows you to annotate the contents of
a database.  Just as with most
computer languages, comments are ignored for the purpose of interpreting the
contents of the database. Comments effectively act as
additional white space\index{white
space} between tokens
when a database is parsed.

A comment may be placed at the beginning, end, or
between any two tokens\index{token} in a source file.

Comments have the following syntax:
\begin{center}
 \texttt{\$(} {\em text} \texttt{\$)}
\end{center}
Here,\index{\texttt{\$(} and \texttt{\$)} auxiliary
keywords}\index{comment} {\em text} is a string, possibly empty, of any
characters in Metamath's character set (p.~\pageref{spec1chars}), except
that the character strings \texttt{\$(} and \texttt{\$)} may not appear
in {\em text}.  Thus nested comments are not
permitted:\footnote{Computer languages have differing standards for
nested comments, and rather than picking one it was felt simplest not to
allow them at all, at least in the current version of
Metamath\index{Metamath!limitations of version 0.07.30}.} Metamath will
complain if you give it
\begin{center}
 \texttt{\$( This is a \$( nested \$) comment.\ \$)}
\end{center}
To compensate for this non-nesting behavior, I often change all \texttt{\$}'s
to \texttt{@}'s in sections of Metamath code I wish to comment out.

The Metamath program supports a number of markup mechanisms and conventions
to generate good-looking results in \LaTeX\ and {\sc html},
as discussed below.
These markup features have to do only with how the comments are typeset,
and have no effect on how Metamath verifies the proofs in the database.
The improper
use of them may result in incorrectly typeset output, but no Metamath
error messages will result during the \texttt{read} and \texttt{verify
proof} commands.  (However, the \texttt{write
theorem\texttt{\char`\_}list} command
will check for markup errors as a side-effect of its
{\sc html} generation.)
Section~\ref{texout} has instructions for creating \LaTeX\ output, and
section~\ref{htmlout} has instructions for creating
{\sc html}\index{HTML} output.

\subsubsection{Headings}\label{commentheadings}

If the \texttt{\$(} is immediately followed by a new line
starting with a heading marker, it is a header.
This can start with:

\begin{itemize}
 \item[] \texttt{\#\#\#\#} - major part header
 \item[] \texttt{\#*\#*} - section header
 \item[] \texttt{=-=-} - subsection header
 \item[] \texttt{-.-.} - subsubsection header
\end{itemize}

The line following the marker line
will be used for the table of contents entry, after trimming spaces.
The next line should be another (closing) matching marker line.
Any text after that
but before the closing \texttt{\$}, such as an extended description of the
section, will be included on the \texttt{mmtheoremsNNN.html} page.

For more information, run
\texttt{help write theorem\char`\_list}.

\subsubsection{Math mode}
\label{mathcomments}
\index{\texttt{`} inside comments}
\index{\texttt{\char`\~} inside comments}
\index{math mode}

Inside of comments, a string of tokens\index{token} enclosed in
grave accents\index{grave accent (\texttt{`})} (\texttt{`}) will be converted
to standard mathematical symbols during
{\sc HTML}\index{HTML} or \LaTeX\ output
typesetting,\index{latex@{\LaTeX}} according to the information in the
special \texttt{\$t}\index{\texttt{\$t} comment}\index{typesetting
comment} comment in the database
(see section~\ref{tcomment} for information about the typesetting
comment, and Appendix~\ref{ASCII} to see examples of its results).

The first grave accent\index{grave accent (\texttt{`})} \texttt{`}
causes the output processor to enter {\bf math mode}\index{math mode}
and the second one exits it.
In this
mode, the characters following the \texttt{`} are interpreted as a
sequence of math symbol tokens separated by white space\index{white
space}.  The tokens are looked up in the \texttt{\$t}
comment\index{\texttt{\$t} comment}\index{typesetting comment} and if
found, they will be replaced by the standard mathematical symbols that
they correspond to before being placed in the typeset output file.  If
not found, the symbol will be output as is and a warning will be issued.
The tokens do not have to be active in the database, although a warning
will be issued if they are not declared with \texttt{\$c} or
\texttt{\$v} statements.

Two consecutive
grave accents \texttt{``} are treated as a single actual grave accent
(both inside and outside of math mode) and will not cause the output
processor to enter or exit math mode.

Here is an example of its use\index{Pierce's axiom}:
\begin{center}
\texttt{\$( Pierce's axiom, ` ( ( ph -> ps ) -> ph ) -> ph ` ,\\
         is not very intuitive. \$)}
\end{center}
becomes
\begin{center}
   \texttt{\$(} Pierce's axiom, $((\varphi \rightarrow \psi)\rightarrow
\varphi)\rightarrow \varphi$, is not very intuitive. \texttt{\$)}
\end{center}

Note that the math symbol tokens\index{token} must be surrounded by white
space\index{white space}.
%, since there is no context that allows ambiguity to be
%resolved, as is the case with math symbol sequences in some of the Metamath
%statements.
White space should also surround the \texttt{`}
delimiters.

The math mode feature also gives you a quick and easy way to generate
text containing mathematical symbols, independently of the intended
purpose of Metamath.\index{Metamath!using as a math editor} To do this,
simply create your text with grave accents surrounding your formulas,
after making sure that your math symbols are mapped to \LaTeX\ symbols
as described in Appendix~\ref{ASCII}.  It is easier if you start with a
database with predefined symbols such as \texttt{set.mm}.  Use your
grave-quoted math string to replace an existing comment, then typeset
the statement corresponding to that comment following the instructions
from the \texttt{help tex} command in the Metamath program.  You will
then probably want to edit the resulting file with a text editor to fine
tune it to your exact needs.

\subsubsection{Label Mode}\index{label mode}

Outside of math mode, a tilde\index{tilde (\texttt{\char`\~})} \verb/~/
indicates to Metamath's\index{Metamath} output processor that the
token\index{token} that follows (i.e.\ the characters up to the next
white space\index{white space}) represents a statement label or URL.
This formatting mode is called {\bf label mode}\index{label mode}.
If a literal tilde
is desired (outside of math mode) instead of label mode,
use two tildes in a row to represent it.

When generating a \LaTeX\ output file,
the following token will be formatted in \texttt{typewriter}
font, and the tilde removed, to make it stand out from the rest of the text.
This formatting will be applied to all characters after the
tilde up to the first white space\index{white space}.
Whether
or not the token is an actual statement label is not checked, and the
token does not have to have the correct syntax for a label; no error
messages will be produced.  The only effect of the label mode on the
output is that typewriter font will be used for the tokens that are
placed in the \LaTeX\ output file.

When generating {\sc html},
the tokens after the tilde {\em must} be a URL (either http: or https:)
or a valid label.
Error messages will be issued during that output if they aren't.
A hyperlink will be generated to that URL or label.

\subsubsection{Link to bibliographical reference}\index{citation}%
\index{link to bibliographical reference}

Bibliographical references are handled specially when generating
{\sc html} if formatted specially.
Text in the form \texttt{[}{\em author}\texttt{]}
is considered a link to a bibliographical reference.
See \texttt{help html} and \texttt{help write
bibliography} in the Metamath program for more
information.
% \index{\texttt{\char`\[}\ldots\texttt{]} inside comments}
See also Sections~\ref{tcomment} and \ref{wrbib}.

The \texttt{[}{\em author}\texttt{]} notation will also create an entry in
the bibliography cross-reference file generated by \texttt{write
bibliography} (Section~\ref{wrbib}) for {\sc HTML}.
For this to work properly, the
surrounding comment must be formatted as follows:
\begin{quote}
    {\em keyword} {\em label} {\em noise-word}
     \texttt{[}{\em author}\texttt{] p.} {\em number}
\end{quote}
for example
\begin{verbatim}
     Theorem 5.2 of [Monk] p. 223
\end{verbatim}
The {\em keyword} is not case sensitive and must be one of the following:
\begin{verbatim}
     theorem lemma definition compare proposition corollary
     axiom rule remark exercise problem notation example
     property figure postulate equation scheme chapter
\end{verbatim}
The optional {\em label} may consist of more than one
(non-{\em keyword} and non-{\em noise-word}) word.
The optional {\em noise-word} is one of:
\begin{verbatim}
     of in from on
\end{verbatim}
and is  ignored when the cross-reference file is created.  The
\texttt{write
biblio\-graphy} command will perform error checking to verify the
above format.\index{error checking}

\subsubsection{Parentheticals}\label{parentheticals}

The end of a comment may include one or more parenthicals, that is,
statements enclosed in parentheses.
The Metamath program looks for certain parentheticals and can issue
warnings based on them.
They are:

\begin{itemize}
 \item[] \texttt{(Contributed by }
   \textit{NAME}\texttt{,} \textit{DATE}\texttt{.)} -
   document the original contributor's name and the date it was created.
 \item[] \texttt{(Revised by }
   \textit{NAME}\texttt{,} \textit{DATE}\texttt{.)} -
   document the contributor's name and creation date
   that resulted in significant revision
   (not just an automated minimization or shortening).
 \item[] \texttt{(Proof shortened by }
   \textit{NAME}\texttt{,} \textit{DATE}\texttt{.)} -
   document the contributor's name and date that developed a significant
   shortening of the proof (not just an automated minimization).
 \item[] \texttt{(Proof modification is discouraged.)} -
   Note that this proof should normally not be modified.
 \item[] \texttt{(New usage is discouraged.)} -
   Note that this assertion should normally not be used.
\end{itemize}

The \textit{DATE} must be in form YYYY-MMM-DD, where MMM is the
English abbreviation of that month.

\subsubsection{Other markup}\label{othermarkup}
\index{markup notation}

There are other markup notations for generating good-looking results
beyond math mode and label mode:

\begin{itemize}
 \item[]
         \texttt{\char`\_} (underscore)\index{\texttt{\char`\_} inside comments} -
             Italicize text starting from
              {\em space}\texttt{\char`\_}{\em non-space} (i.e.\ \texttt{\char`\_}
              with a space before it and a non-space character after it) until
             the next
             {\em non-space}\texttt{\char`\_}{\em space}.  Normal
             punctuation (e.g.\ a trailing
             comma or period) is ignored when determining {\em space}.
 \item[]
         \texttt{\char`\_} (underscore) - {\em
         non-space}\texttt{\char`\_}{\em non-space-string}, where
          {\em non-space-string} is a string of non-space characters,
         will make {\em non-space-string} become a subscript.
 \item[]
         \texttt{<HTML>}...\texttt{</HTML>} - do not convert
         ``\texttt{<}'' and ``\texttt{>}''
         in the enclosed text when generating {\sc HTML},
         otherwise process markup normally. This allows direct insertion
         of {\sc html} commands.
 \item[]
       ``\texttt{\&}ref\texttt{;}'' - insert an {\sc HTML}
         character reference.
         This is how to insert arbitrary Unicode characters
         (such as accented characters).  Currently only directly supported
         when generating {\sc HTML}.
\end{itemize}

It is recommended that spaces surround any \texttt{\char`\~} and
\texttt{`} tokens in the comment and that a space follow the {\em label}
after a \texttt{\char`\~} token.  This will make global substitutions
to change labels and symbol names much easier and also eliminate any
future chance of ambiguity.  Spaces around these tokens are automatically
removed in the final output to conform with normal rules of punctuation;
for example, a space between a trailing \texttt{`} and a left parenthesis
will be removed.

A good way to become familiar with the markup notation is to look at
the extensive examples in the \texttt{set.mm} database.

\subsection{The Typesetting Comment (\texttt{\$t})}\label{tcomment}

The typesetting comment \texttt{\$t} in the input database file
provides the information necessary to produce good-looking results.
It provides \LaTeX\ and {\sc html}
definitions for math symbols,
as well supporting as some
customization of the generated web page.
If you add a new token to a database, you should also
update the \texttt{\$t} comment information if you want to eventually
create output in \LaTeX\ or {\sc HTML}.
See the
\texttt{set.mm}\index{set theory database (\texttt{set.mm})} database
file for an extensive example of a \texttt{\$t} comment illustrating
many of the features described below.

Programs that do not need to generate good-looking presentation results,
such as programs that only verify Metamath databases,
can completely ignore typesetting comments
and just treat them as normal comments.
Even the Metamath program only consults the
\texttt{\$t} comment information when it needs to generate typeset output
in \LaTeX\ or {\sc HTML}
(e.g., when you open a \LaTeX\ output file with the \texttt{open tex} command).

We will first discuss the syntax of typesetting comments, and then
briefly discuss how this can be used within the Metamath program.

\subsubsection{Typesetting Comment Syntax Overview}

The typesetting comment is identified by the token
\texttt{\$t}\index{\texttt{\$t} comment}\index{typesetting comment} in
the comment, and the typesetting comment ends at the matching
\texttt{\$)}:
\[
  \mbox{\tt \$(\ }
  \mbox{\tt \$t\ }
  \underbrace{
    \mbox{\tt \ \ \ \ \ \ \ \ \ \ \ }
    \cdots
    \mbox{\tt \ \ \ \ \ \ \ \ \ \ \ }
  }_{\mbox{Typesetting definitions go here}}
  \mbox{\tt \ \$)}
\]

There must be one or more white space characters, and only white space
characters, between the \texttt{\$(} that starts the comment
and the \texttt{\$t} symbol,
and the \texttt{\$t} must be followed by one
or more white space characters
(see section \ref{whitespace} for the definition of white space characters).
The typesetting comment continues until the comment end token \texttt{\$)}
(which must be preceded by one or more white space characters).

In version 0.07.30\index{Metamath!limitations of version 0.07.30} of the
Metamath program, there may be only one \texttt{\$t} comment in a
database.  This restriction may be lifted in the future to allow
many \texttt{\$t} comments in a database.

Between the \texttt{\$t} symbol (and its following white space) and the
comment end token \texttt{\$)} (and its preceding white space)
is a sequence of one or more typesetting definitions, where
each definition has the form
\textit{definition-type arg arg ... ;}.
Each of the zero or more \textit{arg} values
can be either a typesetting data or a keyword
(what keywords are allowed, and where, depends on the specific
\textit{definition-type}).
The \textit{definition-type}, and each argument \textit{arg},
are separated by one or more white space characters.
Every definition ends in an unquoted semicolon;
white space is not required before the terminating semicolon of a definition.
Each definition should start on a new line.\footnote{This
restriction of the current version of Metamath
(0.07.30)\index{Metamath!limitations of version 0.07.30} may be removed
in a future version, but you should do it anyway for readability.}

For example, this typesetting definition:
\begin{center}
 \verb$latexdef "C_" as "\subseteq";$
\end{center}
defines the token \verb$C_$ as the \LaTeX\ symbol $\subseteq$ (which means
``subset'').

Typesetting data is a sequence of one or more quoted strings
(if there is more than one, they are connected by \texttt{\char`\+}).
Often a single quoted string is used to provide data for a definition, using
either double (\texttt{\char`\"}) or single (\texttt{'}) quotation marks.
However,
{\em a quoted string (enclosed in quotation marks) may not include
line breaks.}
A quoted string
may include a quotation mark that matches the enclosing quotes by repeating
the quotation mark twice.  Here are some examples:

\begin{tabu}   { l l }
\textbf{Example} & \textbf{Meaning} \\
\texttt{\char`\"a\char`\"\char`\"b\char`\"} & \texttt{a\char`\"b} \\
\texttt{'c''d'} & \texttt{c'd} \\
\texttt{\char`\"e''f\char`\"} & \texttt{e''f} \\
\texttt{'g\char`\"\char`\"h'} & \texttt{g\char`\"\char`\"h} \\
\end{tabu}

Finally, a long quoted string
may be broken up into multiple quoted strings (considered, as a whole,
a single quoted string) and joined with \texttt{\char`\+}.
You can even use multiple lines as long as a
'+' is at the end of every line except the last one.
The \texttt{\char`\+} should be preceded and followed by at least one
white space character.
Thus, for example,
\begin{center}
 \texttt{\char`\"ab\char`\"\ \char`\+\ \char`\"cd\char`\"
    \ \char`\+\ \\ 'ef'}
\end{center}
is the same as
\begin{center}
 \texttt{\char`\"abcdef\char`\"}
\end{center}

{\sc c}-style comments \texttt{/*}\ldots\texttt{*/} are also supported.

In practice, whenever you add a new math token you will often want to add
typesetting definitions using
\texttt{latexdef}, \texttt{htmldef}, and
\texttt{althtmldef}, as described below.
That way, they will all be up to date.
Of course, whether or not you want to use all three definitions will
depend on how the database is intended to be used.

Below we discuss the different possible \textit{definition-kind} options.
We will show data surrounded by double quotes (in practice they can also use
single quotes and/or be a sequence joined by \texttt{+}s).
We will use specific names for the \textit{data} to make clear what
the data is used for, such as
{\em math-token} (for a Metamath math token,
{\em latex-string} (for string to be placed in a \LaTeX\ stream),
{\em {\sc html}-code} (for {\sc html} code),
and {\em filename} (for a filename).

\subsubsection{Typesetting Comment - \LaTeX}

The syntax for a \LaTeX\ definition is:
\begin{center}
 \texttt{latexdef "}{\em math-token}\texttt{" as "}{\em latex-string}\texttt{";}
\end{center}
\index{latex definitions@\LaTeX\ definitions}%
\index{\texttt{latexdef} statement}

The {\em token-string} and {\em latex-string} are the data
(character strings) for
the token and the \LaTeX\ definition of the token, respectively,

These \LaTeX\ definitions are used by the Metamath program
when it is asked to product \LaTeX output using
the \texttt{write tex} command.

\subsubsection{Typesetting Comment - {\sc html}}

The key kinds of {\sc HTML} definitions have the following syntax:

\vskip 1ex
    \texttt{htmldef "}{\em math-token}\texttt{" as "}{\em
    {\sc html}-code}\texttt{";}\index{\texttt{htmldef} statement}
                    \ \ \ \ \ \ldots

    \texttt{althtmldef "}{\em math-token}\texttt{" as "}{\em
{\sc html}-code}\texttt{";}\index{\texttt{althtmldef} statement}

                    \ \ \ \ \ \ldots

Note that in {\sc HTML} there are two possible definitions for math tokens.
This feature is useful when
an alternate representation of symbols is desired, for example one that
uses Unicode entities and another uses {\sc gif} images.

There are many other typesetting definitions that can control {\sc HTML}.
These include:

\vskip 1ex

    \texttt{htmldef "}{\em math-token}\texttt{" as "}{\em {\sc
    html}-code}\texttt{";}

    \texttt{htmltitle "}{\em {\sc html}-code}\texttt{";}%
\index{\texttt{htmltitle} statement}

    \texttt{htmlhome "}{\em {\sc html}-code}\texttt{";}%
\index{\texttt{htmlhome} statement}

    \texttt{htmlvarcolor "}{\em {\sc html}-code}\texttt{";}%
\index{\texttt{htmlvarcolor} statement}

    \texttt{htmlbibliography "}{\em filename}\texttt{";}%
\index{\texttt{htmlbibliography} statement}

\vskip 1ex

\noindent The \texttt{htmltitle} is the {\sc html} code for a common
title, such as ``Metamath Proof Explorer.''  The \texttt{htmlhome} is
code for a link back to the home page.  The \texttt{htmlvarcolor} is
code for a color key that appears at the bottom of each proof.  The file
specified by {\em filename} is an {\sc html} file that is assumed to
have a \texttt{<A NAME=}\ldots\texttt{>} tag for each bibiographic
reference in the database comments.  For example, if
\texttt{[Monk]}\index{\texttt{\char`\[}\ldots\texttt{]} inside comments}
occurs in the comment for a theorem, then \texttt{<A NAME='Monk'>} must
be present in the file; if not, a warning message is given.

Associated with
\texttt{althtmldef}
are the statements
\vskip 1ex

    \texttt{htmldir "}{\em
      directoryname}\texttt{";}\index{\texttt{htmldir} statement}

    \texttt{althtmldir "}{\em
     directoryname}\texttt{";}\index{\texttt{althtmldir} statement}

\vskip 1ex
\noindent giving the directories of the {\sc gif} and Unicode versions
respectively; their purpose is to provide cross-linking between the
two versions in the generated web pages.

When two different types of pages need to be produced from a single
database, such as the Hilbert Space Explorer that extends the Metamath
Proof Explorer, ``extended'' variables may be declared in the
\texttt{\$t} comment:
\vskip 1ex

    \texttt{exthtmltitle "}{\em {\sc html}-code}\texttt{";}%
\index{\texttt{exthtmltitle} statement}

    \texttt{exthtmlhome "}{\em {\sc html}-code}\texttt{";}%
\index{\texttt{exthtmlhome} statement}

    \texttt{exthtmlbibliography "}{\em filename}\texttt{";}%
\index{\texttt{exthtmlbibliography} statement}

\vskip 1ex
\noindent When these are declared, you also must declare
\vskip 1ex

    \texttt{exthtmllabel "}{\em label}\texttt{";}%
\index{\texttt{exthtmllabel} statement}

\vskip 1ex \noindent that identifies the database statement where the
``extended'' section of the database starts (in our example, where the
Hilbert Space Explorer starts).  During the generation of web pages for
that starting statement and the statements after it, the {\sc html} code
assigned to \texttt{exthtmltitle} and \texttt{exthtmlhome} is used
instead of that assigned to \texttt{htmltitle} and \texttt{htmlhome},
respectively.

\begin{sloppy}
\subsection{Additional Information Com\-ment (\texttt{\$j})} \label{jcomment}
\end{sloppy}

The additional information comment, aka the
\texttt{\$j}\index{\texttt{\$j} comment}\index{additional information comment}
comment,
provides a way to add additional structured information that can
be optionally parsed by systems.

The additional information comment is parsed the same way as the
typesetting comment (\texttt{\$t}) (see section \ref{tcomment}).
That is,
the additional information comment begins with the token
\texttt{\$j} within a comment,
and continues until the comment close \texttt{\$)}.
Within an additional information comment is a sequence of one or more
commands of the form \texttt{command arg arg ... ;}
where each of the zero or more \texttt{arg} values
can be either a quoted string or a keyword.
Note that every command ends in an unquoted semicolon.
If a verifier is parsing an additional information comment, but
doesn't recognize a particular command, it must skip the command
by finding the end of the command (an unquoted semicolon).

A database may have 0 or more additional information comments.
Note, however, that a verifier may ignore these comments entirely or only
process certain commands in an additional information comment.
The \texttt{mmj2} verifier supports many commands in additional information
comments.
We encourage systems that process additional information comments
to coordinate so that they will use the same command for the same effect.

Examples of additional information comments with various commands
(from the \texttt{set.mm} database) are:

\begin{itemize}
   \item Define the syntax and logical typecodes,
     and declare that our grammar is
     unambiguous (verifiable using the KLR parser, with compositing depth 5).
\begin{verbatim}
  $( $j
    syntax 'wff';
    syntax '|-' as 'wff';
    unambiguous 'klr 5';
  $)
\end{verbatim}

   \item Register $\lnot$ and $\rightarrow$ as primitive expressions
           (lacking definitions).
\begin{verbatim}
  $( $j primitive 'wn' 'wi'; $)
\end{verbatim}

   \item There is a special justification for \texttt{df-bi}.
\begin{verbatim}
  $( $j justification 'bijust' for 'df-bi'; $)
\end{verbatim}

   \item Register $\leftrightarrow$ as an equality for its type (wff).
\begin{verbatim}
  $( $j
    equality 'wb' from 'biid' 'bicomi' 'bitri';
    definition 'dfbi1' for 'wb';
  $)
\end{verbatim}

   \item Theorem \texttt{notbii} is the congruence law for negation.
\begin{verbatim}
  $( $j congruence 'notbii'; $)
\end{verbatim}

   \item Add \texttt{setvar} as a typecode.
\begin{verbatim}
  $( $j syntax 'setvar'; $)
\end{verbatim}

   \item Register $=$ as an equality for its type (\texttt{class}).
\begin{verbatim}
  $( $j equality 'wceq' from 'eqid' 'eqcomi' 'eqtri'; $)
\end{verbatim}

\end{itemize}


\subsection{Including Other Files in a Metamath Source File} \label{include}
\index{\texttt{\$[} and \texttt{\$]} auxiliary keywords}

The keywords \texttt{\$[} and \texttt{\$]} specify a file to be
included\index{included file}\index{file inclusion} at that point in a
Metamath\index{Metamath} source file\index{source file}.  The syntax for
including a file is as follows:
\begin{center}
\texttt{\$[} {\em file-name} \texttt{\$]}
\end{center}

The {\em file-name} should be a single token\index{token} with the same syntax
as a math symbol (i.e., all 93 non-whitespace
printable characters other than \texttt{\$} are
allowed, subject to the file-naming limitations of your operating system).
Comments may appear between the \texttt{\$[} and \texttt{\$]} keywords.  Included
files may include other files, which may in turn include other files, and so
on.

For example, suppose you want to use the set theory database as the starting
point for your own theory.  The first line in your file could be
\begin{center}
\texttt{\$[ set.mm \$]}
\end{center} All of the information (axioms, theorems,
etc.) in \texttt{set.mm} and any files that {\em it} includes will become
available for you to reference in your file. This can help make your work more
modular. A drawback to including files is that if you change the name of a
symbol or the label of a statement, you must also remember to update any
references in any file that includes it.


The naming conventions for included files are the same as those of your
operating system.\footnote{On the Macintosh, prior to Mac OS X,
 a colon is used to separate disk
and folder names from your file name.  For example, {\em volume}\texttt{:}{\em
file-name} refers to the root directory, {\em volume}\texttt{:}{\em
folder-name}\texttt{:}{\em file-name} refers to a folder in root, and {\em
volume}\texttt{:}{\em folder-name}\texttt{:}\ldots\texttt{:}{\em file-name} refers to a
deeper folder.  A simple {\em file-name} refers to a file in the folder from
which you launch the Metamath application.  Under Mac OS X and later,
the Metamath program is run under the Terminal application, which
conforms to Unix naming conventions.}\index{Macintosh file
names}\index{file names!Macintosh}\label{includef} For compatibility among
operating systems, you should keep the file names as simple as possible.  A
good convention to use is {\em file}\texttt{.mm} where {\em file} is eight
characters or less, in lower case.

There is no limit to the nesting depth of included files.  One thing that you
should be aware of is that if two included files themselves include a common
third file, only the {\em first} reference to this common file will be read
in.  This allows you to include two or more files that build on a common
starting file without having to worry about label and symbol conflicts that
would occur if the common file were read in more than once.  (In fact, if a
file includes itself, the self-reference will be ignored, although of course
it would not make any sense to do that.)  This feature also means, however,
that if you try to include a common file in several inner blocks, the result
might not be what you expect, since only the first reference will be replaced
with the included file (unlike the include statement in most other computer
languages).  Thus you would normally include common files only in the
outermost block\index{outermost block}.

\subsection{Compressed Proof Format}\label{compressed1}\index{compressed
proof}\index{proof!compressed}

The proof notation presented in Section~\ref{proof} is called a
{\bf normal proof}\index{normal proof}\index{proof!normal} and in principle is
sufficient to express any proof.  However, proofs often contain steps and
subproofs that are identical.  This is particularly true in typical
Metamath\index{Metamath} applications, because Metamath requires that the math
symbol sequence (usually containing a formula) at each step be separately
constructed, that is, built up piece by piece. As a result, a lot of
repetition often results.  The {\bf compressed proof} format allows Metamath
to take advantage of this redundancy to shorten proofs.

The specification for the compressed proof format is given in
Appen\-dix~\ref{compressed}.

Normally you need not concern yourself with the details of the compressed
proof format, since the Metamath program will allow you to convert from
the normal format to the compressed format with ease, and will also
automatically convert from the compressed format when proofs are displayed.
The overall structure of the compressed format is as follows:
\begin{center}
  \texttt{\$= ( } {\em label-list} \texttt{) } {\em compressed-proof\ }\ \texttt{\$.}
\end{center}
\index{\texttt{\$=} keyword}
The first \texttt{(} serves as a flag to Metamath that a compressed proof
follows.  The {\em label-list} includes all statements referred to by the
proof except the mandatory hypotheses\index{mandatory hypothesis}.  The {\em
compressed-proof} is a compact encoding of the proof, using upper-case
letters, and can be thought of as a large integer in base 26.  White
space\index{white space} inside a {\em compressed-proof} is
optional and is ignored.

It is important to note that the order of the mandatory hypotheses of
the statement being proved must not be changed if the compressed proof
format is used, otherwise the proof will become incorrect.  The reason
for this is that the mandatory hypotheses are not mentioned explicitly
in the compressed proof in order to make the compression more efficient.
If you wish to change the order of mandatory hypotheses, you must first
convert the proof back to normal format using the \texttt{save proof
{\em statement} /normal}\index{\texttt{save proof} command} command.
Later, you can go back to compressed format with \texttt{save proof {\em
statement} /compressed}.

During error checking with the \texttt{verify proof} command, an error
found in a compressed proof may point to a character in {\em
compressed-proof}, which may not be very meaningful to you.  In this
case, try to \texttt{save proof /normal} first, then do the
\texttt{verify proof} again.  In general, it is best to make sure a
proof is correct before saving it in compressed format, because severe
errors are less likely to be recoverable than in normal format.

\subsection{Specifying Unknown Proofs or Subproofs}\label{unknown}

In a proof under development, any step or subproof that is not yet known
may be represented with a single \texttt{?}.  For the purposes of
parsing the proof, the \texttt{?}\ \index{\texttt{]}@\texttt{?}\ inside
proofs} will push a single entry onto the RPN stack just as if it were a
hypothesis.  While developing a proof with the Proof
Assistant\index{Proof Assistant}, a partially developed proof may be
saved with the \texttt{save new{\char`\_}proof}\index{\texttt{save
new{\char`\_}proof} command} command, and \texttt{?}'s will be placed at
the appropriate places.

All \texttt{\$p}\index{\texttt{\$p} statement} statements must have
proofs, even if they are entirely unknown.  Before creating a proof with
the Proof Assistant, you should specify a completely unknown proof as
follows:
\begin{center}
  {\em label} \texttt{\$p} {\em statement} \texttt{\$= ?\ \$.}
\end{center}
\index{\texttt{\$=} keyword}
\index{\texttt{]}@\texttt{?}\ inside proofs}

The \texttt{verify proof}\index{\texttt{verify proof} command} command
will check the known portions of a partial proof for errors, but will
warn you that the statement has not been proved.

Note that partially developed proofs may be saved in compressed format
if desired.  In this case, you will see one or more \texttt{?}'s in the
{\em compressed-proof} part.\index{compressed
proof}\index{proof!compressed}

\section{Axioms vs.\ Definitions}\label{definitions}

The \textit{basic}
Metamath\index{Metamath} language and program
make no distinction\index{axiom vs.\
definition} between axioms\index{axiom} and
definitions.\index{definition} The \texttt{\$a}\index{\texttt{\$a}
statement} statement is used for both.  At first, this may seem
puzzling.  In the minds of many mathematicians, the distinction is
clear, even obvious, and hardly worth discussing.  A definition is
considered to be merely an abbreviation that can be replaced by the
expression for which it stands; although unless one actually does this,
to be precise then one should say that a theorem\index{theorem} is a
consequence of the axioms {\em and} the definitions that are used in the
formulation of the theorem \cite[p.~20]{Behnke}.\index{Behnke, H.}

\subsection{What is a Definition?}

What is a definition?  In its simplest form, a definition introduces a new
symbol and provides an unambiguous rule to transform an expression containing
the new symbol to one without it.  The concept of a ``proper
definition''\index{proper definition}\index{definition!proper} (as opposed to
a creative definition)\index{creative definition}\index{definition!creative}
that is usually agreed upon is (1) the definition should not strengthen the
language and (2) any symbols introduced by the definition should be eliminable
from the language \cite{Nemesszeghy}\index{Nemesszeghy, E. Z.}.  In other
words, they are mere typographical conveniences that do not belong to the
system and are theoretically superfluous.  This may seem obvious, but in fact
the nature of definitions can be subtle, sometimes requiring difficult
metatheorems to establish that they are not creative.

A more conservative stance was taken by logician S.
Le\'{s}niewski.\index{Le\'{s}niewski, S.}
\begin{quote}
Le\'{s}niewski
regards definitions as theses of the system.  In this respect they do
not differ either from the axioms or from theorems, i.e.\ from the
theses added to the system on the basis of the rule of substitution or
the rule of detachment [modus ponens].  Once definitions have been
accepted as theses of the system, it becomes necessary to consider them
as true propositions in the same sense in which axioms are true
\cite{Lejewski}.
\end{quote}\index{Lejewski, Czeslaw}

Let us look at some simple examples of definitions in propositional
calculus.  Consider the definition of logical {\sc or}
(disjunction):\index{disjunction ($\vee$)} ``$P\vee Q$ denotes $\neg P
\rightarrow Q$ (not $P$ implies $Q$).''  It is very easy to recognize a
statement making use of this definition, because it introduces the new
symbol $\vee$ that did not previously exist in the language.  It is easy
to see that no new theorems of the original language will result from
this definition.

Next, consider a definition that eliminates parentheses:  ``$P
\rightarrow Q\rightarrow R$ denotes $P\rightarrow (Q \rightarrow R)$.''
This is more subtle, because no new symbols are introduced.  The reason
this definition is considered proper is that no new symbol sequences
that are valid wffs (well-formed formulas)\index{well-formed formula
(wff)} in the original language will result from the definition, since
``$P \rightarrow Q\rightarrow R$'' is not a wff in the original
language.  Here, we implicitly make use of the fact that there is a
decision procedure that allows us to determine whether or not a symbol
sequence is a wff, and this fact allows us to use symbol sequences that
are not wffs to represent other things (such as wffs) by means of the
definition.  However, to justify the definition as not being creative we
need to prove that ``$P \rightarrow Q\rightarrow R$'' is in fact not a
wff in the original language, and this is more difficult than in the
case where we simply introduce a new symbol.

%Now let's take this reasoning to an extreme.  Propositional calculus is a
%decidable theory,\footnote{This means that a mechanical algorithm exists to
%determine whether or not a wff is a theorem.} so in principle we could make use
%of symbol sequences that are not theorems to represent other things (say, to
%encode actual theorems in a more compact way).  For example, let us extend the
%language by defining a wff ``$P$'' in the extended language as the theorem
%``$P\rightarrow P$''\footnote{This is one of the first theorems proved in the
%Metamath database \texttt{set.mm}.}\index{set
%theory database (\texttt{set.mm})} in the original language whenever ``$P$'' is
%not a theorem in the original language.  In the extended language, any wff
%``$Q$'' thus represents a theorem; to find out what theorem (in the original
%language) ``$Q$'' represents, we determine whether ``$Q$'' is a theorem in the
%original language (before the definition was introduced).  If so, we're done; if
%not, we replace ``$Q$'' by ``$Q\rightarrow Q$'' to eliminate the definition.
%This definition is therefore eliminable, and it does not ``strengthen'' the
%language because any wff that is not a theorem is not in the set of statements
%provable in the original language and thus is available for use by definitions.
%
%Of course, a definition such as this would render practically useless the
%communication of theorems of propositional calculus; but
%this is just a human shortcoming, since we can't always easily discern what is
%and is not a theorem by inspection.  In fact, the extended theory with this
%definition has no more and no less information than the original theory; it just
%expresses certain theorems of the form ``$P\rightarrow P$''
%in a more compact way.
%
%The point here is that what constitutes a proper definition is a matter of
%judgment about whether a symbol sequence can easily be recognized by a human
%as invalid in some sense (for example, not a wff); if so, the symbol sequence
%can be appropriated for use by a definition in order to make the extended
%language more compact.  Metamath\index{Metamath} lacks the ability to make this
%judgment, since as far as Metamath is concerned the definition of a wff, for
%example, is arbitrary.  You define for Metamath how wffs\index{well-formed
%formula (wff)} are constructed according to your own preferred style.  The
%concept of a wff may not even exist in a given formal system\index{formal
%system}.  Metamath treats all definitions as if they were new axioms, and it
%is up to the human mathematician to judge whether the definition is ``proper''
%'\index{proper definition}\index{definition!proper} in some agreed-upon way.

What constitutes a definition\index{definition} versus\index{axiom vs.\
definition} an axiom\index{axiom} is sometimes arbitrary in mathematical
literature.  For example, the connectives $\vee$ ({\sc or}), $\wedge$
({\sc and}), and $\leftrightarrow$ (equivalent to) in propositional
calculus are usually considered defined symbols that can be used as
abbreviations for expressions containing the ``primitive'' connectives
$\rightarrow$ and $\neg$.  This is the way we treat them in the standard
logic and set theory database \texttt{set.mm}\index{set theory database
(\texttt{set.mm})}.  However, the first three connectives can also be
considered ``primitive,'' and axiom systems have been devised that treat
all of them as such.  For example,
\cite[p.~35]{Goodstein}\index{Goodstein, R. L.} presents one with 15
axioms, some of which in fact coincide with what we have chosen to call
definitions in \texttt{set.mm}.  In certain subsets of classical
propositional calculus, such as the intuitionist
fragment\index{intuitionism}, it can be shown that one cannot make do
with just $\rightarrow$ and $\neg$ but must treat additional connectives
as primitive in order for the system to make sense.\footnote{Two nice
systems that make the transition from intuitionistic and other weak
fragments to classical logic just by adding axioms are given in
\cite{Robinsont}\index{Robinson, T. Thacher}.}

\subsection{The Approach to Definitions in \texttt{set.mm}}

In set theory, recursive definitions define a newly introduced symbol in
terms of itself.  The justification of recursive definitions, using
several ``recursion theorems,'' is usually one of the first
sophisticated proofs a student encounters when learning set theory, and
there is a significant amount of implicit metalogic behind a recursive
definition even though the definition itself is typically simple to
state.  It is, however, possible to substitute one kind of complexity
for another.  We can eliminate the need for metalogical justification by
defining the operation directly with an explicit (but complicated)
expression, then deriving the recursive definition directly as a
theorem, using a recursion theorem ``in reverse.''  We do this in
\texttt{set.mm}, as follows.

In \texttt{set.mm} our goal was to introduce almost all definitions in
the form of two expressions connected by either $\leftrightarrow$ or
$=$, where the thing being defined does not appear on the right hand
side.  Quine calls this form ``a genuine or direct definition'' \cite[p.
174]{Quine}\index{Quine, Willard Van Orman}, which makes the definitions
very easy to eliminate and the metalogic\index{metalogic} needed to
justify them as simple as possible.

We achieved this goal in almost all cases in \texttt{set.mm}.
Sometimes this makes the definitions more complex and less
intuitive.  For example, the traditional way to define addition of
natural numbers is to define an operation called {\em
successor}\index{successor} (which means ``plus one'' and is denoted by
``${\rm suc}$''), then define addition recursively\index{recursive
definition} with the two definitions $n + 0 = n$ and $m + {\rm suc}\,n =
{\rm suc} (m + n)$.  Although this definition seems simple and obvious,
the method to eliminate the definition is not obvious:  in the second
part of the definition, addition is defined in terms of itself.  By
eliminating the definition, we don't mean repeatedly applying it to
specific $m$ and $n$ but rather showing the explicit, closed-form
set-theoretical expression that $m + n$ represents, that will work for
any $m$ and $n$ and that does not have a $+$ sign on its right-hand
side.  For a recursive definition like this not to be circular
(creative), there are some hidden, underlying assumptions we must make,
for example that the natural numbers have a certain kind of order.  In
\texttt{set.mm} we chose to start with the direct (though complex and
nonintuitive) definition then derive from it the standard recursive
definition.\footnote{The closed-form definition used in \texttt{set.mm}
for the addition operation on ordinals\index{ordinal
addition}\index{addition!of ordinals} (of which natural numbers are a
subset) is

\setbox\startprefix=\hbox{\tt \ \ df-oadd\ \$a\ }
\setbox\contprefix=\hbox{\tt \ \ \ \ \ \ \ \ \ \ \ \ \ }
\startm
\m{\vdash}\m{+_o}\m{=}\m{\{}\m{\langle}\m{\langle}\m{x}\m{,}\m{y}\m{\rangle}%
\m{,}\m{z}\m{\rangle}\m{|}\m{(}\m{(}\m{x}\m{\in}\m{{\rm On}}\m{\wedge}\m{y}\m{%
\in}\m{{\rm On}}\m{)}\m{\wedge}\m{z}\m{=}\m{(}\m{{\rm rec}}\m{(}\m{\{}\m{%
\langle}\m{w}\m{,}\m{v}\m{\rangle}\m{|}\m{v}\m{=}\m{{\rm suc}}\m{w}\m{\}}\m{,}%
\m{x}\m{)}\m{`}\m{y}\m{)}\m{)}\m{\}}
\endm

\noindent Here, the abstraction class of nested ordered pairs\index{abstraction
class!of nested ordered pairs} is defined by \texttt{df-oprab} in \texttt{set.mm},
and ${\rm rec}$ is a ``recursion operator''\index{recursion operator} with
the definition

\setbox\startprefix=\hbox{\tt \ \ df-rfg\ \$a\ }
\setbox\contprefix=\hbox{\tt \ \ \ \ \ \ \ \ \ \ \ \ }
\startm
\m{\vdash}\m{{\rm rec}}\m{(}\m{F}\m{,}\m{A}\m{)}\m{=}\m{\bigcup}\m{\{}\m{f}%
\m{|}\m{\exists}\m{x}\m{(}\m{x}\m{\in}\m{{\rm On}}\m{\wedge}\m{(}\m{f}\m{{%
\rm Fn}}\m{x}\m{\wedge}\m{\forall}\m{y}\m{(}\m{y}\m{\in}\m{x}\m{\rightarrow}%
\m{(}\m{f}\m{`}\m{y}\m{)}\m{=}\m{(}\m{\{}\m{\langle}\m{g}\m{,}\m{z}\m{\rangle}%
\m{|}\m{(}\m{(}\m{g}\m{=}\m{\varnothing}\m{\wedge}\m{z}\m{=}\m{A}\m{)}\m{\vee}%
\m{(}\m{\lnot}\m{(}\m{g}\m{=}\m{\varnothing}\m{\vee}\m{{\rm Lim}}\m{{\rm dom}}%
\m{g}\m{)}\m{\wedge}\m{z}\m{=}\m{(}\m{F}\m{`}\m{(}\m{g}\m{`}\m{\bigcup}\m{{%
\rm dom}}\m{g}\m{)}\m{)}\m{)}\m{\vee}\m{(}\m{{\rm Lim}}\m{{\rm dom}}\m{g}\m{%
\wedge}\m{z}\m{=}\m{\bigcup}\m{{\rm ran}}\m{g}\m{)}\m{)}\m{\}}\m{`}\m{(}\m{f}%
\m{\restriction}\m{y}\m{)}\m{)}\m{)}\m{)}\m{)}\m{\}}
\endm

\noindent which can be further broken down with definitions shown in
Section~\ref{setdefinitions}.  You may be surprised at the complexity of what
seems like such a simple notion.  From these definitions the simpler, more
intuitive recursive definition is derived as a set of theorems.}\index{natural
number}\index{addition}\index{recursive definition}\index{ordinal addition}
The end result is the same, but we completely eliminate the rather complex
metalogic that justifies the recursive definition.  (For a mathematician,
recursive definitions are more efficient and intuitive than
direct ones once the metalogic has been learned or possibly just accepted as
correct.  However, it was felt that direct definition in \texttt{set.mm}
maximizes rigor by minimizing metalogic.  It can be eliminated effortlessly,
something difficult to do with a recursive definition.)

\subsection{Adding Constraints on Definitions}

The basic Metamath language and the Metamath program do
not have any built-in constraints on definitions, since they are just
\$a statements.

However, nothing prevents a verification system from verifying
additional rules to impose further limitations on definitions.
For example, the \texttt{mmj2}\index{mmj2} program
supports various kinds of
additional information comments (see section \ref{jcomment}).
One of their uses is to optionally verify additional constraints,
including constraints to verify that definitions meet certain
requirements.
These additional checks are required by the
continuous integration (CI)\index{continuous integration (CI)}
checks of the
\texttt{set.mm}\index{set theory database (\texttt{set.mm})}%
\index{Metamath Proof Explorer}
database.
This approach enables us to optionally impose additional requirements
on definitions if we wish, without necessarily imposing those rules on
all databases or requiring all verification systems to implement them.
In addition, this allows us to impose specialized constraints tailored
to one database while not requiring other systems to implement
those specialized constraints.

We impose two constraints on the
\texttt{set.mm}\index{set theory database (\texttt{set.mm})}%
\index{Metamath Proof Explorer} database
via the \texttt{mmj2}\index{mmj2} program that are worth discussing here:
a parse check and a definition soundness check.

% On February 11, 2019 8:32:32 PM EST, saueran@oregonstate.edu wrote:
% The following addition to the end of set.mm is accepted by the mmj2
% parser and definition checker and the metamath verifier(at least it was
% when I checked, you should check it too), and creates a contradiction by
% proving the theorem |- ph.
% ${
% wleftp $a wff ( ( ph ) $.
% wbothp $a wff ( ph ) $.
% df-leftp $a |- ( ( ( ph ) <-> -. ph ) $.
% df-bothp $a |- ( ( ph ) <-> ph ) $.
% anything $p |- ph $=
%   ( wbothp wn wi wleftp df-leftp biimpi df-bothp mpbir mpbi simplim ax-mp)
%   ABZAMACZDZCZMOEZOCQAEZNDZRNAFGSHIOFJMNKLAHJ $.
% $}
%
% This particular problem is countered by enabling, within mmj2,
% SetParser,mmj.verify.LRParser

First,
we enable a parse check in \texttt{mmj2} (through its
\texttt{SetParser} command) that requires that all new definitions
must \textit{not} create an ambiguous parse for a KLR(5) parser.
This prevents some errors such as definitions with imbalanced parentheses.

Second, we run a definition soundness check specific to
\texttt{set.mm} or databases similar to it.
(through the \texttt{definitionCheck} macro).
Some \texttt{\$a} statements (including all ax-* statemnets)
are excluded from these checks, as they will
always fail this simple check,
but they are appropriate for most definitions.
This check imposes a set of additional rules:

\begin{enumerate}

\item New definitions must be introduced using $=$ or $\leftrightarrow$.

\item No \texttt{\$a} statement introduced before this one is allowed to use the
symbol being defined in this definition, and the definition is not
allowed to use itself (except once, in the definiendum).

\item Every variable in the definiens should not be distinct

\item Every dummy variable in the definiendum
are required to be distinct from each other and from variables in
the definiendum.
To determine this, the system will look for a "justification" theorem
in the database, and if it is not there, attempt to briefly prove
$( \varphi \rightarrow \forall x \varphi )$  for each dummy variable x.

\item Every dummy variable should be a set variable,
unless there is a justification theorem available.

\item Every dummy variable must be bound
(if the system cannot determine this a justification theorem must be
provided).

\end{enumerate}

\subsection{Summary of Approach to Definitions}

In short, when being rigorous it turns out that
definitions can be subtle, sometimes requiring difficult
metatheorems to establish that they are not creative.

Instead of building such complications into the Metamath language itself,
the basic Metmath language and program simply treat traditional
axioms and definitions as the same kind of \texttt{\$a} statement.
We have then built various tools to enable people to
verify additional conditions as their creators believe is appropriate
for those specific databases, without complicating the Metamath foundations.

\chapter{The Metamath Program}\label{commands}

This chapter provides a reference manual for the
Metamath program.\index{Metamath!commands}

Current instructions for obtaining and installing the Metamath program
can be found at the \url{http://metamath.org} web site.
For Windows, there is a pre-compiled version called
\texttt{metamath.exe}.  For Unix, Linux, and Mac OS X
(which we will refer to collectively as ``Unix''), the Metamath program
can be compiled from its source code with the command
\begin{verbatim}
gcc *.c -o metamath
\end{verbatim}
using the \texttt{gcc} {\sc c} compiler available on those systems.

In the command syntax descriptions below, fields enclosed in square
brackets [\ ] are optional.  File names may be optionally enclosed in
single or double quotes.  This is useful if the file name contains
spaces or
slashes (\texttt{/}), such as in Unix path names, \index{Unix file
names}\index{file names!Unix} that might be confused with Metamath
command qualifiers.\index{Unix file names}\index{file names!Unix}

\section{Invoking Metamath}

Unix, Linux, and Mac OS X
have a command-line interface called the {\em
bash shell}.  (In Mac OS X, select the Terminal application from
Applications/Utilities.)  To invoke Metamath from the bash shell prompt,
assuming that the Metamath program is in the current directory, type
\begin{verbatim}
bash$ ./metamath
\end{verbatim}

To invoke Metamath from a Windows DOS or Command Prompt, assuming that
the Metamath program is in the current directory (or in a directory
included in the Path system environment variable), type
\begin{verbatim}
C:\metamath>metamath
\end{verbatim}

To use command-line arguments at invocation, the command-line arguments
should be a list of Metamath commands, surrounded by quotes if they
contain spaces.  In Windows, the surrounding quotes must be double (not
single) quotes.  For example, to read the database file \texttt{set.mm},
verify all proofs, and exit the program, type (under Unix)
\begin{verbatim}
bash$ ./metamath 'read set.mm' 'verify proof *' exit
\end{verbatim}
Note that in Unix, any directory path with \texttt{/}'s must be
surrounded by quotes so Metamath will not interpret the \texttt{/} as a
command qualifier.  So if \texttt{set.mm} is in the \texttt{/tmp}
directory, use for the above example
\begin{verbatim}
bash$ ./metamath 'read "/tmp/set.mm"' 'verify proof *' exit
\end{verbatim}

For convenience, if the command-line has one argument and no spaces in
the argument, the command is implicitly assumed to be \texttt{read}.  In
this one special case, \texttt{/}'s are not interpreted as command
qualifiers, so you don't need quotes around a Unix file name.  Thus
\begin{verbatim}
bash$ ./metamath /tmp/set.mm
\end{verbatim}
and
\begin{verbatim}
bash$ ./metamath "read '/tmp/set.mm'"
\end{verbatim}
are equivalent.


\section{Controlling Metamath}

The Metamath program was first developed on a {\sc vax/vms} system, and
some aspects of its command line behavior reflect this heritage.
Hopefully you will find it reasonably user-friendly once you get used to
it.

Each command line is a sequence of English-like words separated by
spaces, as in \texttt{show settings}.  Command words are not case
sensitive, and only as many letters are needed as are necessary to
eliminate ambiguity; for example, \texttt{sh se} would work for the
command \texttt{show settings}.  In some cases arguments such as file
names, statement labels, or symbol names are required; these are
case-sensitive (although file names may not be on some operating
systems).

A command line is entered by typing it in then pressing the {\em return}
({\em enter}) key.  To find out what commands are available, type
\texttt{?} at the \texttt{MM>} prompt.  To find out the choices at any
point in a command, press {\em return} and you will be prompted for
them.  The default choice (the one selected if you just press {\em
return}) is shown in brackets (\texttt{<>}).

You may also type \texttt{?} in place of a command word to force
Metamath to tell you what the choices are.  The \texttt{?} method won't
work, though, if a non-keyword argument such as a file name is expected
at that point, because the program will think that \texttt{?} is the
value of the argument.

Some commands have one or more optional qualifiers which modify the
behavior of the command.  Qualifiers are preceded by a slash
(\texttt{/}), such as in \texttt{read set.mm / verify}.  Spaces are
optional around the \texttt{/}.  If you need to use a space or
slash in a command
argument, as in a Unix file name, put single or double quotes around the
command argument.

The \texttt{open log} command will save everything you see on the
screen and is useful to help you recover should something go wrong in a
proof, or if you want to document a bug.

If a command responds with more than a screenful, you will be
prompted to \texttt{<return> to continue, Q to quit, or S to scroll to
end}.  \texttt{Q} or \texttt{q} (not case-sensitive) will complete the
command internally but will suppress further output until the next
\texttt{MM>} prompt.  \texttt{s} will suppress further pausing until the
next \texttt{MM>} prompt.  After the first screen, you are also
presented with the choice of \texttt{b} to go back a screenful.  Note
that \texttt{b} may also be entered at the \texttt{MM>} prompt
immediately after a command to scroll back through the output of that
command.

A command line enclosed in quotes is executed by your operating system.
See Section~\ref{oscmd}.

{\em Warning:} Pressing {\sc ctrl-c} will abort the Metamath program
unconditionally.  This means any unsaved work will be lost.


\subsection{\texttt{exit} Command}\index{\texttt{exit} command}

Syntax:  \texttt{exit} [\texttt{/force}]

This command exits from Metamath.  If there have been changes to the
source with the \texttt{save proof} or \texttt{save new{\char`\_}proof}
commands, you will be given an opportunity to \texttt{write source} to
permanently save the changes.

In Proof Assistant\index{Proof Assistant} mode, the \texttt{exit} command will
return to the \verb/MM>/ prompt. If there were changes to the proof, you will
be given an opportunity to \texttt{save new{\char`\_}proof}.

The \texttt{quit} command is a synonym for \texttt{exit}.

Optional qualifier:
    \texttt{/force} - Do not prompt if changes were not saved.  This qualifier is
        useful in \texttt{submit} command files (Section~\ref{sbmt})
        to ensure predictable behavior.





\subsection{\texttt{open log} Command}\index{\texttt{open log} command}
Syntax:  \texttt{open log} {\em file-name}

This command will open a log file that will store everything you see on
the screen.  It is useful to help recovery from a mistake in a long Proof
Assistant session, or to document bugs.\index{Metamath!bugs}

The log file can be closed with \texttt{close log}.  It will automatically be
closed upon exiting Metamath.



\subsection{\texttt{close log} Command}\index{\texttt{close log} command}
Syntax:  \texttt{close log}

The \texttt{close log} command closes a log file if one is open.  See
also \texttt{open log}.




\subsection{\texttt{submit} Command}\index{\texttt{submit} command}\label{sbmt}
Syntax:  \texttt{submit} {\em filename}

This command causes further command lines to be taken from the specified
file.  Note that any line beginning with an exclamation point (\texttt{!}) is
treated as a comment (i.e.\ ignored).  Also note that the scrolling
of the screen output is continuous, so you may want to open a log file
(see \texttt{open log}) to record the results that fly by on the screen.
After the lines in the file are exhausted, Metamath returns to its
normal user interface mode.

Currently, the \texttt{submit} command is not recursive.  In other words,
\texttt{submit} commands are not allowed inside a command file.


\subsection{\texttt{erase} Command}\index{\texttt{erase} command}
Syntax:  \texttt{erase}

This command will reset Metamath to its starting state, deleting any
data\-base that was \texttt{read} in.
 If there have been changes to the
source with the \texttt{save proof} or \texttt{save new{\char`\_}proof}
commands, you will be given an opportunity to \texttt{write source} to
permanently save the changes.



\subsection{\texttt{set echo} Command}\index{\texttt{set echo} command}
Syntax:  \texttt{set echo on} or \texttt{set echo off}

The \texttt{set echo on} command will cause command lines to be echoed with any
abbreviations expanded.  While learning the Metamath commands, this
feature will show you the exact command that your abbreviated input
corresponds to.



\subsection{\texttt{set scroll} Command}\index{\texttt{set scroll} command}
Syntax:  \texttt{set scroll prompted} or \texttt{set scroll continuous}

The Metamath command line interface starts off in the \texttt{prompted} mode,
which means that you will be prompted to continue or quit after each
full screen in a long listing.  In \texttt{continuous} mode, long listings will be
scrolled without pausing.

% LaTeX bug? (1) \texttt{\_} puts out different character than
% \texttt{{\char`\_}}
%  = \verb$_$  (2) \texttt{{\char`\_}} puts out garbage in \subsection
%  argument
\subsection{\texttt{set width} Command}\index{\texttt{set
width} command}
Syntax:  \texttt{set width} {\em number}

Metamath assumes the width of your screen is 79 characters (chosen
because the Command Prompt in Windows XP has a wrapping bug at column
80).  If your screen is wider or narrower, this command allows you to
change this default screen width.  A larger width is advantageous for
logging proofs to an output file to be printed on a wide printer.  A
smaller width may be necessary on some terminals; in this case, the
wrapping of the information messages may sometimes seem somewhat
unnatural, however.  In \LaTeX\index{latex@{\LaTeX}!characters per line},
there is normally a maximum of 61 characters per line with typewriter
font.  (The examples in this book were produced with 61 characters per
line.)

\subsection{\texttt{set height} Command}\index{\texttt{set
height} command}
Syntax:  \texttt{set height} {\em number}

Metamath assumes your screen height is 24 lines of characters.  If your
screen is taller or shorter, this command lets you to change the number
of lines at which the display pauses and prompts you to continue.

\subsection{\texttt{beep} Command}\index{\texttt{beep} command}

Syntax:  \texttt{beep}

This command will produce a beep.  By typing it ahead after a
long-running command has started, it will alert you that the command is
finished.  For convenience, \texttt{b} is an abbreviation for
\texttt{beep}.

Note:  If \texttt{b} is typed at the \texttt{MM>} prompt immediately
after the end of a multiple-page display paged with ``\texttt{Press
<return> for more}...'' prompts, then the \texttt{b} will back up to the
previous page rather than perform the \texttt{beep} command.
In that case you must type the unabbreviated \texttt{beep} form
of the command.

\subsection{\texttt{more} Command}\index{\texttt{more} command}

Syntax:  \texttt{more} {\em filename}

This command will display the contents of an {\sc ascii} file on your
screen.  (This command is provided for convenience but is not very
powerful.  See Section~\ref{oscmd} to invoke your operating system's
command to do this, such as the \texttt{more} command in Unix.)

\subsection{Operating System Commands}\index{operating system
command}\label{oscmd}

A line enclosed in single or double quotes will be executed by your
computer's operating system if it has a command line interface.  For
example, on a {\sc vax/vms} system,
\verb/MM> 'dir'/
will print disk directory contents.  Note that this feature will not
work on the Macintosh prior to Mac OS X, which does not have a command
line interface.

For your convenience, the trailing quote is optional.

\subsection{Size Limitations in Metamath}

In general, there are no fixed, predefined limits\index{Metamath!memory
limits} on how many labels, tokens\index{token}, statements, etc.\ that
you may have in a database file.  The Metamath program uses 32-bit
variables (64-bit on 64-bit CPUs) as indices for almost all internal
arrays, which are allocated dynamically as needed.



\section{Reading and Writing Files}

The following commands create new files:  the \texttt{open} commands;
the \texttt{write} commands; the \texttt{/html},
\texttt{/alt{\char`\_}html}, \texttt{/brief{\char`\_}html},
\texttt{/brief{\char`\_}alt{\char`\_}html} qualifiers of \texttt{show
statement}, and \texttt{midi}.  The following commands append to files
previously opened:  the \texttt{/tex} qualifier of \texttt{show proof}
and \texttt{show new{\char`\_}proof}; the \texttt{/tex} and
\texttt{/simple{\char`\_}tex} qualifiers of \texttt{show statement}; the
\texttt{close} commands; and all screen dialog between \texttt{open log}
and \texttt{close log}.

The commands that create new files will not overwrite an existing {\em
filename} but will rename the existing one to {\em
filename}\texttt{{\char`\~}1}.  An existing {\em
filename}\texttt{{\char`\~}1} is renamed {\em
filename}\texttt{{\char`\~}2}, etc.\ up to {\em
filename}\texttt{{\char`\~}9}.  An existing {\em
filename}\texttt{{\char`\~}9} is deleted.  This makes recovery from
mistakes easier but also will clutter up your directory, so occasionally
you may want to clean up (delete) these \texttt{{\char`\~}}$n$ files.


\subsection{\texttt{read} Command}\index{\texttt{read} command}
Syntax:  \texttt{read} {\em file-name} [\texttt{/verify}]

This command will read in a Metamath language source file and any included
files.  Normally it will be the first thing you do when entering Metamath.
Statement syntax is checked, but proof syntax is not checked.
Note that the file name may be enclosed in single or double quotes;
this is useful if the file name contains slashes, as might be the case
under Unix.

If you are getting an ``\texttt{?Expected VERIFY}'' error
when trying to read a Unix file name with slashes, you probably haven't
quoted it.\index{Unix file names}\index{file names!Unix}

If you are prompted for the file name (by pressing {\em return}
 after \texttt{read})
you should {\em not} put quotes around it, even if it is a Unix file name
with slashes.

Optional command qualifier:

    \texttt{/verify} - Verify all proofs as the database is read in.  This
         qualifier will slow down reading in the file.  See \texttt{verify
         proof} for more information on file error-checking.

See also \texttt{erase}.



\subsection{\texttt{write source} Command}\index{\texttt{write source} command}
Syntax:  \texttt{write source} {\em filename}
[\texttt{/rewrap}]
[\texttt{/split}]
[\texttt{/keep\_includes}]
[\texttt{/no\_versioning}]

This command will write the contents of a Metamath\index{database}
database into a file.\index{source file}

Optional command qualifiers:

\texttt{/rewrap} -
Reformats statements and comments according to the
convention used in the set.mm database.
It unwraps the
lines in the comment before each \texttt{\$a} and \texttt{\$p} statement,
then it
rewraps the line.  You should compare the output to the original
to make sure that the desired effect results; if not, go back to
the original source.  The wrapped line length honors the
\texttt{set width}
parameter currently in effect.  Note:  Text
enclosed in \texttt{<HTML>}...\texttt{</HTML>} tags is not modified by the
\texttt{/rewrap} qualifier.
Proofs themselves are not reformatted;
use \texttt{save proof * / compressed} to do that.
An isolated tilde (\~{}) is always kept on the same line as the following
symbol, so you can find all comment references to a symbol by
searching for \~{} followed by a space and that symbol
(this is useful for finding cross-references).
Incidentally, \texttt{save proof} also honors the \texttt{set width}
parameter currently in effect.

\texttt{/split} - Files included in the source using the expression
\$[ \textit{inclfile} \$] will be
written into separate files instead of being included in a single output
file.  The name of each separately written included file will be the
\textit{inclfile} argument of its inclusion command.

\texttt{/keep\_includes} - If a source file has includes but is written as a
single file by omitting \texttt{/split}, by default the included files will
be deleted (actually just renamed with a \char`\~1 suffix unless
\texttt{/no\_versioning} is specified) to prevent the possibly confusing
source duplication in both the output file and the included file.
The \texttt{/keep\_includes} qualifier will prevent this deletion.

\texttt{/no\_versioning} - Backup files suffixed with \char`\~1 are not created.


\section{Showing Status and Statements}



\subsection{\texttt{show settings} Command}\index{\texttt{show settings} command}
Syntax:  \texttt{show settings}

This command shows the state of various parameters.

\subsection{\texttt{show memory} Command}\index{\texttt{show memory} command}
Syntax:  \texttt{show memory}

This command shows the available memory left.  It is not meaningful
on most modern operating systems,
which have virtual memory.\index{Metamath!memory usage}


\subsection{\texttt{show labels} Command}\index{\texttt{show labels} command}
Syntax:  \texttt{show labels} {\em label-match} [\texttt{/all}]
   [\texttt{/linear}]

This command shows the labels of \texttt{\$a} and \texttt{\$p}
statements that match {\em label-match}.  A \verb$*$ in {label-match}
matches zero or more characters.  For example, \verb$*abc*def$ will match all
labels containing \verb$abc$ and ending with \verb$def$.

Optional command qualifiers:

   \texttt{/all} - Include matches for \texttt{\$e} and \texttt{\$f}
   statement labels.

   \texttt{/linear} - Display only one label per line.  This can be useful for
       building scripts in conjunction with the utilities under the
       \texttt{tools}\index{\texttt{tools} command} command.



\subsection{\texttt{show statement} Command}\index{\texttt{show statement} command}
Syntax:  \texttt{show statement} {\em label-match} [{\em qualifiers} (see below)]

This command provides information about a statement.  Only statements
that have labels (\texttt{\$f}\index{\texttt{\$f} statement},
\texttt{\$e}\index{\texttt{\$e} statement},
\texttt{\$a}\index{\texttt{\$a} statement}, and
\texttt{\$p}\index{\texttt{\$p} statement}) may be specified.
If {\em label-match}
contains wildcard (\verb$*$) characters, all matching statements will be
displayed in the order they occur in the database.

Optional qualifiers (only one qualifier at a time is allowed):

    \texttt{/comment} - This qualifier includes the comment that immediately
       precedes the statement.

    \texttt{/full} - Show complete information about each statement,
       and show all
       statements matching {\em label} (including \texttt{\$e}
       and \texttt{\$f} statements).

    \texttt{/tex} - This qualifier will write the statement information to the
       \LaTeX\ file previously opened with \texttt{open tex}.  See
       Section~\ref{texout}.

    \texttt{/simple{\char`\_}tex} - The same as \texttt{/tex}, except that
       \LaTeX\ macros are not used for formatting equations, allowing easier
       manual edits of the output for slide presentations, etc.

    \texttt{/html}\index{html generation@{\sc html} generation},
       \texttt{/alt{\char`\_}html}, \texttt{/brief{\char`\_}html},
       \texttt{/brief{\char`\_}alt{\char`\_}html} -
       These qualifiers invoke a special mode of
       \texttt{show statement} that
       creates a web page for the statement.  They may not be used with
       any other qualifier.  See Section~\ref{htmlout} or
       \texttt{help html} in the program.


\subsection{\texttt{search} Command}\index{\texttt{search} command}
Syntax:  search {\em label-match}
\texttt{"}{\em symbol-match}{\tt}" [\texttt{/all}] [\texttt{/comments}]
[\texttt{/join}]

This command searches all \texttt{\$a} and \texttt{\$p} statements
matching {\em label-match} for occurrences of {\em symbol-match}.  A
\verb@*@ in {\em label-match} matches any label character.  A \verb@$*@
in {\em symbol-match} matches any sequence of symbols.  The symbols in
{\em symbol-match} must be separated by white space.  The quotes
surrounding {\em symbol-match} may be single or double quotes.  For
example, \texttt{search b}\verb@* "-> $* ch"@ will list all statements
whose labels begin with \texttt{b} and contain the symbols \verb@->@ and
\texttt{ch} surrounding any symbol sequence (including no symbol
sequence).  The wildcards \texttt{?} and \texttt{\$?} are also available
to match individual characters in labels and symbols respectively; see
\texttt{help search} in the Metamath program for details on their usage.

Optional command qualifiers:

    \texttt{/all} - Also search \texttt{\$e} and \texttt{\$f} statements.

    \texttt{/comments} - Search the comment that immediately precedes each
        label-matched statement for {\em symbol-match}.  In this case
        {\em symbol-match} is an arbitrary, non-case-sensitive character
        string.  Quotes around {\em symbol-match} are optional if there
        is no ambiguity.

    \texttt{/join} - In the case of a \texttt{\$a} or \texttt{\$p} statement,
	prepend its \texttt{\$e}
	hypotheses for searching. The
	\texttt{/join} qualifier has no effect in \texttt{/comments} mode.

\section{Displaying and Verifying Proofs}


\subsection{\texttt{show proof} Command}\index{\texttt{show proof} command}
Syntax:  \texttt{show proof} {\em label-match} [{\em qualifiers} (see below)]

This command displays the proof of the specified
\texttt{\$p}\index{\texttt{\$p} statement} statement in various formats.
The {\em label-match} may contain wildcard (\verb@$*@) characters to match
multiple statements.  Without any qualifiers, only the logical steps
will be shown (i.e.\ syntax construction steps will be omitted), in an
indented format.

Most of the time, you will use
    \texttt{show proof} {\em label}
to see just the proof steps corresponding to logical inferences.

Optional command qualifiers:

    \texttt{/essential} - The proof tree is trimmed of all
        \texttt{\$f}\index{\texttt{\$f} statement} hypotheses before
        being displayed.  (This is the default, and it is redundant to
        specify it.)

    \texttt{/all} - the proof tree is not trimmed of all \texttt{\$f} hypotheses before
        being displayed.  \texttt{/essential} and \texttt{/all} are mutually exclusive.

    \texttt{/from{\char`\_}step} {\em step} - The display starts at the specified
        step.  If
        this qualifier is omitted, the display starts at the first step.

    \texttt{/to{\char`\_}step} {\em step} - The display ends at the specified
        step.  If this
        qualifier is omitted, the display ends at the last step.

    \texttt{/tree{\char`\_}depth} {\em number} - Only
         steps at less than the specified proof
        tree depth are displayed.  Sometimes useful for obtaining an overview of
        the proof.

    \texttt{/reverse} - The steps are displayed in reverse order.

    \texttt{/renumber} - When used with \texttt{/essential}, the steps are renumbered
        to correspond only to the essential steps.

    \texttt{/tex} - The proof is converted to \LaTeX\ and\index{latex@{\LaTeX}}
        stored in the file opened
        with \texttt{open tex}.  See Section~\ref{texout} or
        \texttt{help tex} in the program.

    \texttt{/lemmon} - The proof is displayed in a non-indented format known
        as Lemmon style, with explicit previous step number references.
        If this qualifier is omitted, steps are indented in a tree format.

    \texttt{/start{\char`\_}column} {\em number} - Overrides the default column
        (16)
        at which the formula display starts in a Lemmon-style display.  May be
        used only in conjunction with \texttt{/lemmon}.

    \texttt{/normal} - The proof is displayed in normal format suitable for
        inclusion in a Metamath source file.  May not be used with any other
        qualifier.

    \texttt{/compressed} - The proof is displayed in compressed format
        suitable for inclusion in a Metamath source file.  May not be used with
        any other qualifier.

    \texttt{/statement{\char`\_}summary} - Summarizes all statements (like a
        brief \texttt{show statement})
        used by the proof.  It may not be used with any other qualifier
        except \texttt{/essential}.

    \texttt{/detailed{\char`\_}step} {\em step} - Shows the details of what is
        happening at
        a specific proof step.  May not be used with any other qualifier.
        The {\em step} is the step number shown when displaying a
        proof without the \texttt{/renumber} qualifier.


\subsection{\texttt{show usage} Command}\index{\texttt{show usage} command}
Syntax:  \texttt{show usage} {\em label-match} [\texttt{/recursive}]

This command lists the statements whose proofs make direct reference to
the statement specified.

Optional command qualifier:

    \texttt{/recursive} - Also include statements whose proofs ultimately
        depend on the statement specified.



\subsection{\texttt{show trace\_back} Command}\index{\texttt{show
       trace{\char`\_}back} command}
Syntax:  \texttt{show trace{\char`\_}back} {\em label-match} [\texttt{/essential}] [\texttt{/axioms}]
    [\texttt{/tree}] [\texttt{/depth} {\em number}]

This command lists all statements that the proof of the \texttt{\$p}
statement(s) specified by {\em label-match} depends on.

Optional command qualifiers:

    \texttt{/essential} - Restrict the trace-back to \texttt{\$e}
        \index{\texttt{\$e} statement} hypotheses of proof trees.

    \texttt{/axioms} - List only the axioms that the proof ultimately depends on.

    \texttt{/tree} - Display the trace-back in an indented tree format.

    \texttt{/depth} {\em number} - Restrict the \texttt{/tree} trace-back to the
        specified indentation depth.

    \texttt{/count{\char`\_}steps} - Count the number of steps the proof has
       all the way back to axioms.  If \texttt{/essential} is specified,
       expansions of variable-type hypotheses (syntax constructions) are not counted.

\subsection{\texttt{verify proof} Command}\index{\texttt{verify proof} command}
Syntax:  \texttt{verify proof} {\em label-match} [\texttt{/syntax{\char`\_}only}]

This command verifies the proofs of the specified statements.  {\em
label-match} may contain wild card characters (\texttt{*}) to verify
more than one proof; for example \verb/*abc*def/ will match all labels
containing \texttt{abc} and ending with \texttt{def}.
The command \texttt{verify proof *} will verify all proofs in the database.

Optional command qualifier:

    \texttt{/syntax{\char`\_}only} - This qualifier will perform a check of syntax
        and RPN
        stack violations only.  It will not verify that the proof is
        correct.  This qualifier is useful for quickly determining which
        proofs are incomplete (i.e.\ are under development and have \texttt{?}'s
        in them).

{\em Note:} \texttt{read}, followed by \texttt{verify proof *}, will ensure
 the database is free
from errors in the Metamath language but will not check the markup notation
in comments.
You can also check the markup notation by running \texttt{verify markup *}
as discussed in Section~\ref{verifymarkup}; see also the discussion
on generating {\sc HTML} in Section~\ref{htmlout}.

\subsection{\texttt{verify markup} Command}\index{\texttt{verify markup} command}\label{verifymarkup}
Syntax:  \texttt{verify markup} {\em label-match}
[\texttt{/date{\char`\_}skip}]
[\texttt{/top{\char`\_}date{\char`\_}skip}]
[\texttt{/file{\char`\_}skip}]
[\texttt{/verbose}]

This command checks comment markup and other informal conventions we have
adopted.  It error-checks the latexdef, htmldef, and althtmldef statements
in the \texttt{\$t} statement of a Metamath source file.\index{error checking}
It error-checks any \texttt{`}...\texttt{`},
\texttt{\char`\~}~\textit{label},
and bibliographic markups in statement descriptions.
It checks that
\texttt{\$p} and \texttt{\$a} statements
have the same content when their labels start with
``ax'' and ``ax-'' respectively but are otherwise identical, for example
ax4 and ax-4.
It also verifies the date consistency of ``(Contributed by...),''
``(Revised by...),'' and ``(Proof shortened by...)'' tags in the comment
above each \texttt{\$a} and \texttt{\$p} statement.

Optional command qualifiers:

    \texttt{/date{\char`\_}skip} - This qualifier will
        skip date consistency checking,
        which is usually not required for databases other than
	\texttt{set.mm}.

    \texttt{/top{\char`\_}date{\char`\_}skip} - This qualifier will check date consistency except
        that the version date at the top of the database file will not
        be checked.  Only one of
        \texttt{/date{\char`\_}skip} and
        \texttt{/top{\char`\_}date{\char`\_}skip} may be
        specified.

    \texttt{/file{\char`\_}skip} - This qualifier will skip checks that require
        external files to be present, such as checking GIF existence and
        bibliographic links to mmset.html or equivalent.  It is useful
        for doing a quick check from a directory without these files.

    \texttt{/verbose} - Provides more information.  Currently it provides a list
        of axXXX vs. ax-XXX matches.

\subsection{\texttt{save proof} Command}\index{\texttt{save proof} command}
Syntax:  \texttt{save proof} {\em label-match} [\texttt{/normal}]
   [\texttt{/compressed}]

The \texttt{save proof} command will reformat a proof in one of two formats and
replace the existing proof in the source buffer\index{source
buffer}.  It is useful for
converting between proof formats.  Note that a proof will not be
permanently saved until a \texttt{write source} command is issued.

Optional command qualifiers:

    \texttt{/normal} - The proof is saved in the normal format (i.e., as a
        sequence
        of labels, which is the defined format of the basic Metamath
        language).\index{basic language}  This is the default format that
        is used if a qualifier
        is omitted.

    \texttt{/compressed} - The proof is saved in the compressed format which
        reduces storage requirements for a database.
        See Appendix~\ref{compressed}.




\section{Creating Proofs}\label{pfcommands}\index{Proof Assistant}

Before using the Proof Assistant, you must add a \texttt{\$p} to your
source file (using a text editor) containing the statement you want to
prove.  Its proof should consist of a single \texttt{?}, meaning
``unknown step.''  Example:
\begin{verbatim}
equid $p x = x $= ? $.
\end{verbatim}

To enter the Proof assistant, type \texttt{prove} {\em label}, e.g.
\texttt{prove equid}.  Metamath will respond with the \texttt{MM-PA>}
prompt.

Proofs are created working backwards from the statement being proved,
primarily using a series of \texttt{assign} commands.  A proof is
complete when all steps are assigned to statements and all steps are
unified and completely known.  During the creation of a proof, Metamath
will allow only operations that are legal based on what is known up to
that point.  For example, it will not allow an \texttt{assign} of a
statement that cannot be unified with the unknown proof step being
assigned.

{\em Important:}
The Proof Assistant is
{\em not} a tool to help you discover proofs.  It is just a tool to help
you add them to the database.  For a tutorial read
Section~\ref{frstprf}.
To practice using the Proof Assistant, you may
want to \texttt{prove} an existing theorem, then delete all steps with
\texttt{delete all}, then re-create it with the Proof Assistant while
looking at its proof display (before deletion).
You might want to figure out your first few proofs completely
and write them down by hand, before using the Proof Assistant, though
not everyone finds that effective.

{\em Important:}
The \texttt{undo} command if very helpful when entering a proof, because
it allows you to undo a previously-entered step.
In addition, we suggest that you
keep track of your work with a log file (\texttt{open
log}) and save it frequently (\texttt{save new{\char`\_}proof},
\texttt{write source}).
You can use \texttt{delete} to reverse an \texttt{assign}.
You can also do \texttt{delete floating{\char`\_}hypotheses}, then
\texttt{initialize all}, then \texttt{unify all /interactive} to
reinitialize bad unifications made accidentally or by bad
\texttt{assign}s.  You cannot reverse a \texttt{delete} except by
a relevant \texttt{undo} or using
\texttt{exit /force} then reentering the Proof Assistant to recover from
the last \texttt{save new{\char`\_}proof}.

The following commands are available in the Proof Assistant (at the
\texttt{MM-PA>} prompt) to help you create your proof.  See the
individual commands for more detail.

\begin{itemize}
\item[]
    \texttt{show new{\char`\_}proof} [\texttt{/all},...] - Displays the
        proof in progress.  You will use this command a lot; see \texttt{help
        show new{\char`\_}proof} to become familiar with its qualifiers.  The
        qualifiers \texttt{/unknown} and \texttt{/not{\char`\_}unified} are
        useful for seeing the work remaining to be done.  The combination
        \texttt{/all/unknown} is useful for identifying dummy variables that must be
        assigned, or attempts to use illegal syntax, when \texttt{improve all}
        is unable to complete the syntax constructions.  Unknown variables are
        shown as \texttt{\$1}, \texttt{\$2},...
\item[]
    \texttt{assign} {\em step} {\em label} - Assigns an unknown {\em step}
        number with the statement
        specified by {\em label}.
\item[]
    \texttt{let variable} {\em variable}
        \texttt{= "}{\em symbol sequence}\texttt{"}
          - Forces a symbol
        sequence to replace an unknown variable (such as \texttt{\$1}) in a proof.
        It is useful
        for helping difficult unifications, and it is necessary when you have
        dummy variables that eventually must be assigned a name.
\item[]
    \texttt{let step} {\em step} \texttt{= "}{\em symbol sequence}\texttt{"} -
          Forces a symbol sequence
        to replace the contents of a proof step, provided it can be
        unified with the existing step contents.  (I rarely use this.)
\item[]
    \texttt{unify step} {\em step} (or \texttt{unify all}) - Unifies
        the source and target of
        a step.  If you specify a specific step, you will be prompted
        to select among the unifications that are possible.  If you
        specify \texttt{all}, all steps with unique unifications, but only
        those steps, will be
        unified.  \texttt{unify all /interactive} goes through all non-unified
        steps.
\item[]
    \texttt{initialize} {\em step} (or \texttt{all}) - De-unifies the target and source of
        a step (or all steps), as well as the hypotheses of the source,
        and makes all variables in the source unknown.  Useful to recover from
        an \texttt{assign} or \texttt{let} mistake that
        resulted in incorrect unifications.
\item[]
    \texttt{delete} {\em step} (or \texttt{all} or \texttt{floating{\char`\_}hypotheses}) -
        Deletes the specified
        step(s).  \texttt{delete floating{\char`\_}hypotheses}, then \texttt{initialize all}, then
        \texttt{unify all /interactive} is useful for recovering from mistakes
        where incorrect unifications assigned wrong math symbol strings to
        variables.
\item[]
    \texttt{improve} {\em step} (or \texttt{all}) -
      Automatically creates a proof for steps (with no unknown variables)
      whose proof requires no statements with \texttt{\$e} hypotheses.  Useful
      for filling in proofs of \texttt{\$f} hypotheses.  The \texttt{/depth}
      qualifier will also try statements whose \texttt{\$e} hypotheses contain
      no new variables.  {\em Warning:} Save your work (with \texttt{save
      new{\char`\_}proof} then \texttt{write source}) before using
      \texttt{/depth = 2} or greater, since the search time grows
      exponentially and may never terminate in a reasonable time, and you
      cannot interrupt the search.  I have found that it is rare for
      \texttt{/depth = 3} or greater to be useful.
 \item[]
    \texttt{save new{\char`\_}proof} - Saves the proof in progress in the program's
        internal database buffer.  To save it permanently into the database file,
        use \texttt{write source} after
        \texttt{save new{\char`\_}proof}.  To revert to the last
        \texttt{save new{\char`\_}proof},
        \texttt{exit /force} from the Proof Assistant then re-enter the Proof
        Assistant.
 \item[]
    \texttt{match step} {\em step} (or \texttt{match all}) - Shows what
        statements are
        possibilities for the \texttt{assign} statement. (This command
        is not very
        useful in its present form and hopefully will be improved
        eventually.  In the meantime, use the \texttt{search} statement for
        candidates matching specific math token combinations.)
 \item[]
 \texttt{minimize{\char`\_}with}\index{\texttt{minimize{\char`\_}with} command}
% 3/10/07 Note: line-breaking the above results in duplicate index entries
     - After a proof is complete, this command will attempt
        to match other database theorems to the proof to see if the proof
        size can be reduced as a result.  See \texttt{help
        minimize{\char`\_}with} in the
        Metamath program for its usage.
 \item[]
 \texttt{undo}\index{\texttt{undo} command}
    - Undo the effect of a proof-changing command (all but the
      \texttt{show} and \texttt{save} commands above).
 \item[]
 \texttt{redo}\index{\texttt{redo} command}
    - Reverse the previous \texttt{undo}.
\end{itemize}

The following commands set parameters that may be relevant to your proof.
Consult the individual \texttt{help set}... commands.
\begin{itemize}
   \item[] \texttt{set unification{\char`\_}timeout}
 \item[]
    \texttt{set search{\char`\_}limit}
  \item[]
    \texttt{set empty{\char`\_}substitution} - note that default is \texttt{off}
\end{itemize}

Type \texttt{exit} to exit the \texttt{MM-PA>}
 prompt and get back to the \texttt{MM>} prompt.
Another \texttt{exit} will then get you out of Metamath.



\subsection{\texttt{prove} Command}\index{\texttt{prove} command}
Syntax:  \texttt{prove} {\em label}

This command will enter the Proof Assistant\index{Proof Assistant}, which will
allow you to create or edit the proof of the specified statement.
The command-line prompt will change from \texttt{MM>} to \texttt{MM-PA>}.

Note:  In the present version of
Metamath\index{Metamath!limitations of version 0.07.30}, the Proof
Assistant does not verify that \texttt{\$d}\index{\texttt{\$d}
statement} restrictions are met as a proof is being built.  After you
have completed a proof, you should type \texttt{save new{\char`\_}proof}
followed by \texttt{verify proof} {\em label} (where {\em label} is the
statement you are proving with the \texttt{prove} command) to verify the
\texttt{\$d} restrictions.

See also: \texttt{exit}

\subsection{\texttt{set unification\_timeout} Command}\index{\texttt{set
unification{\char`\_}timeout} command}
Syntax:  \texttt{set unification{\char`\_}timeout} {\em number}

(This command is available outside the Proof Assistant but affects the
Proof Assistant\index{Proof Assistant} only.)

Sometimes the Proof Assistant will inform you that a unification
time-out occurred.  This may happen when you try to \texttt{unify}
formulas with many temporary variables\index{temporary variable}
(\texttt{\$1}, \texttt{\$2}, etc.), since the time to compute all possible
unifications may grow exponentially with the number of variables.  If
you want Metamath to try harder (and you're willing to wait longer) you
may increase this parameter.  \texttt{show settings} will show you the
current value.

\subsection{\texttt{set empty\_substitution} Command}\index{\texttt{set
empty{\char`\_}substitution} command}
% These long names can't break well in narrow mode, and even "sloppy"
% is not enough. Work around this by not demanding justification.
\begin{flushleft}
Syntax:  \texttt{set empty{\char`\_}substitution on} or \texttt{set
empty{\char`\_}substitution off}
\end{flushleft}

(This command is available outside the Proof Assistant but affects the
Proof Assistant\index{Proof Assistant} only.)

The Metamath language allows variables to be
substituted\index{substitution!variable}\index{variable substitution}
with empty symbol sequences\index{empty substitution}.  However, in many
formal systems\index{formal system} this will never happen in a valid
proof.  Allowing for this possibility increases the likelihood of
ambiguous unifications\index{ambiguous
unification}\index{unification!ambiguous} during proof creation.
The default is that
empty substitutions are not allowed; for formal systems requiring them,
you must \texttt{set empty{\char`\_}substitution on}.
(An example where this must be \texttt{on}
would be a system that implements a Deduction Rule and in
which deductions from empty assumption lists would be permissible.  The
MIU-system\index{MIU-system} described in Appendix~\ref{MIU} is another
example.)
Note that empty substitutions are
always permissible in proof verification (VERIFY PROOF...) outside the
Proof Assistant.  (See the MIU system in the Metamath book for an example
of a system needing empty substitutions; another example would be a
system that implements a Deduction Rule and in which deductions from
empty assumption lists would be permissible.)

It is better to leave this \texttt{off} when working with \texttt{set.mm}.
Note that this command does not affect the way proofs are verified with
the \texttt{verify proof} command.  Outside of the Proof Assistant,
substitution of empty sequences for math symbols is always allowed.

\subsection{\texttt{set search\_limit} Command}\index{\texttt{set
search{\char`\_}limit} command} Syntax:  \texttt{set search{\char`\_}limit} {\em
number}

(This command is available outside the Proof Assistant but affects the
Proof Assistant\index{Proof Assistant} only.)

This command sets a parameter that determines when the \texttt{improve} command
in Proof Assistant mode gives up.  If you want \texttt{improve} to search harder,
you may increase it.  The \texttt{show settings} command tells you its current
value.


\subsection{\texttt{show new\_proof} Command}\index{\texttt{show
new{\char`\_}proof} command}
Syntax:  \texttt{show new{\char`\_}proof} [{\em
qualifiers} (see below)]

This command (available only in Proof Assistant mode) displays the proof
in progress.  It is identical to the \texttt{show proof} command, except that
there is no statement argument (since it is the statement being proved) and
the following qualifiers are not available:

    \texttt{/statement{\char`\_}summary}

    \texttt{/detailed{\char`\_}step}

Also, the following additional qualifiers are available:

    \texttt{/unknown} - Shows only steps that have no statement assigned.

    \texttt{/not{\char`\_}unified} - Shows only steps that have not been unified.

Note that \texttt{/essential}, \texttt{/depth}, \texttt{/unknown}, and
\texttt{/not{\char`\_}unified} may be
used in any combination; each of them effectively filters out additional
steps from the proof display.

See also:  \texttt{show proof}






\subsection{\texttt{assign} Command}\index{\texttt{assign} command}
Syntax:   \texttt{assign} {\em step} {\em label} [\texttt{/no{\char`\_}unify}]

   and:   \texttt{assign first} {\em label}

   and:   \texttt{assign last} {\em label}


This command, available in the Proof Assistant only, assigns an unknown
step (one with \texttt{?} in the \texttt{show new{\char`\_}proof}
listing) with the statement specified by {\em label}.  The assignment
will not be allowed if the statement cannot be unified with the step.

If \texttt{last} is specified instead of {\em step} number, the last
step that is shown by \texttt{show new{\char`\_}proof /unknown} will be
used.  This can be useful for building a proof with a command file (see
\texttt{help submit}).  It also makes building proofs faster when you know
the assignment for the last step.

If \texttt{first} is specified instead of {\em step} number, the first
step that is shown by \texttt{show new{\char`\_}proof /unknown} will be
used.

If {\em step} is zero or negative, the -{\em step}th from last unknown
step, as shown by \texttt{show new{\char`\_}proof /unknown}, will be
used.  \texttt{assign -1} {\em label} will assign the penultimate
unknown step, \texttt{assign -2} {\em label} the antepenultimate, and
\texttt{assign 0} {\em label} is the same as \texttt{assign last} {\em
label}.

Optional command qualifier:

    \texttt{/no{\char`\_}unify} - do not prompt user to select a unification if there is
        more than one possibility.  This is useful for noninteractive
        command files.  Later, the user can \texttt{unify all /interactive}.
        (The assignment will still be automatically unified if there is only
        one possibility and will be refused if unification is not possible.)



\subsection{\texttt{match} Command}\index{\texttt{match} command}
Syntax:  \texttt{match step} {\em step} [\texttt{/max{\char`\_}essential{\char`\_}hyp}
{\em number}]

    and:  \texttt{match all} [\texttt{/essential}]
          [\texttt{/max{\char`\_}essential{\char`\_}hyp} {\em number}]

This command, available in the Proof Assistant only, shows what
statements can be unified with the specified step(s).  {\em Note:} In
its current form, this command is not very useful because of the large
number of matches it reports.  I rarely use it.
It may be enhanced in the future.

Optional command qualifiers:

    \texttt{/max{\char`\_}essential{\char`\_}hyp} {\em number} - filters out
        of the list any statements
        with more than the specified number of
        \texttt{\$e}\index{\texttt{\$e} statement} hypotheses

    \texttt{/essential{\char`\_}only} - in the \texttt{match all} statement, only
        the steps that
        would be listed in the \texttt{show new{\char`\_}proof /essential} display are
        matched.



\subsection{\texttt{let} Command}\index{\texttt{let} command}
Syntax: \texttt{let variable} {\em variable} = \verb/"/{\em symbol-sequence}\verb/"/

  and: \texttt{let step} {\em step} = \verb/"/{\em symbol-sequence}\verb/"/

These commands, available in the Proof Assistant\index{Proof Assistant}
only, assign a temporary variable\index{temporary variable} or unknown
step with a specific symbol sequence.  They are useful in the middle of
creating a proof, when you know what should be in the proof step but the
unification algorithm doesn't yet have enough information to completely
specify the temporary variables.  A ``temporary variable'' is one that
has the form \texttt{\$}{\em nn} in the proof display, such as
\texttt{\$1}, \texttt{\$2}, etc.  The {\em symbol-sequence} may contain
other unknown variables if desired.  Examples:

    \verb/let variable $32 = "A = B"/

    \verb/let variable $32 = "A = $35"/

    \verb/let step 10 = '|- x = x'/

    \verb/let step -2 = "|- ( $7 = ph )"/

Any symbol sequence will be accepted for the \texttt{let variable}
command.  Only those symbol sequences that can be unified with the step
will be accepted for \texttt{let step}.

The \texttt{let} commands ``zap'' the proof with information that can
only be verified when the proof is built up further.  If you make an
error, the command sequence \texttt{delete
floating{\char`\_}hypotheses}, \texttt{initialize all}, and
\texttt{unify all /interactive} will undo a bad \texttt{let} assignment.

If {\em step} is zero or negative, the -{\em step}th from last unknown
step, as shown by \texttt{show new{\char`\_}proof /unknown}, will be
used.  The command \texttt{let step 0} = \verb/"/{\em
symbol-sequence}\verb/"/ will use the last unknown step, \texttt{let
step -1} = \verb/"/{\em symbol-sequence}\verb/"/ the penultimate, etc.
If {\em step} is positive, \texttt{let step} may be used to assign known
(in the sense of having previously been assigned a label with
\texttt{assign}) as well as unknown steps.

Either single or double quotes can surround the {\em symbol-sequence} as
long as they are different from any quotes inside a {\em
symbol-sequence}.  If {\em symbol-sequence} contains both kinds of
quotes, see the instructions at the end of \texttt{help let} in the
Metamath program.


\subsection{\texttt{unify} Command}\index{\texttt{unify} command}
Syntax:  \texttt{unify step} {\em step}

      and:   \texttt{unify all} [\texttt{/interactive}]

These commands, available in the Proof Assistant only, unify the source
and target of the specified step(s). If you specify a specific step, you
will be prompted to select among the unifications that are possible.  If
you specify \texttt{all}, only those steps with unique unifications will be
unified.

Optional command qualifier for \texttt{unify all}:

    \texttt{/interactive} - You will be prompted to select among the
        unifications
        that are possible for any steps that do not have unique
        unifications.  (Otherwise \texttt{unify all} will bypass these.)

See also \texttt{set unification{\char`\_}timeout}.  The default is
100000, but increasing it to 1000000 can help difficult cases.  Manually
assigning some or all of the unknown variables with the \texttt{let
variable} command also helps difficult cases.



\subsection{\texttt{initialize} Command}\index{\texttt{initialize} command}
Syntax:  \texttt{initialize step} {\em step}

    and: \texttt{initialize all}

These commands, available in the Proof Assistant\index{Proof Assistant}
only, ``de-unify'' the target and source of a step (or all steps), as
well as the hypotheses of the source, and makes all variables in the
source and the source's hypotheses unknown.  This command is useful to
help recover from incorrect unifications that resulted from an incorrect
\texttt{assign}, \texttt{let}, or unification choice.  Part or all of
the command sequence \texttt{delete floating{\char`\_}hypotheses},
\texttt{initialize all}, and \texttt{unify all /interactive} will recover
from incorrect unifications.

See also:  \texttt{unify} and \texttt{delete}



\subsection{\texttt{delete} Command}\index{\texttt{delete} command}
Syntax:  \texttt{delete step} {\em step}

   and:      \texttt{delete all} -- {\em Warning: dangerous!}

   and:      \texttt{delete floating{\char`\_}hypotheses}

These commands are available in the Proof Assistant only.  The
\texttt{delete step} command deletes the proof tree section that
branches off of the specified step and makes the step become unknown.
\texttt{delete all} is equivalent to \texttt{delete step} {\em step}
where {\em step} is the last step in the proof (i.e.\ the beginning of
the proof tree).

In most cases the \texttt{undo} command is the best way to undo
a previous step.
An alternative is to salvage your last \texttt{save
new{\char`\_}proof} by exiting and reentering the Proof Assistant.
For this to work, keep a log file open to record your work
and to do \texttt{save new{\char`\_}proof} frequently, especially before
\texttt{delete}.

\texttt{delete floating{\char`\_}hypotheses} will delete all sections of
the proof that branch off of \texttt{\$f}\index{\texttt{\$f} statement}
statements.  It is sometimes useful to do this before an
\texttt{initialize} command to recover from an error.  Note that once a
proof step with a \texttt{\$f} hypothesis as the target is completely
known, the \texttt{improve} command can usually fill in the proof for
that step.  Unlike the deletion of logical steps, \texttt{delete
floating{\char`\_}hypotheses} is a relatively safe command that is
usually easy to recover from.



\subsection{\texttt{improve} Command}\index{\texttt{improve} command}
\label{improve}
Syntax:  \texttt{improve} {\em step} [\texttt{/depth} {\em number}]
                                               [\texttt{/no{\char`\_}distinct}]

   and:   \texttt{improve first} [\texttt{/depth} {\em number}]
                                              [\texttt{/no{\char`\_}distinct}]

   and:   \texttt{improve last} [\texttt{/depth} {\em number}]
                                              [\texttt{/no{\char`\_}distinct}]

   and:   \texttt{improve all} [\texttt{/depth} {\em number}]
                                              [\texttt{/no{\char`\_}distinct}]

These commands, available in the Proof Assistant\index{Proof Assistant}
only, try to find proofs automatically for unknown steps whose symbol
sequences are completely known.  They are primarily useful for filling in
proofs of \texttt{\$f}\index{\texttt{\$f} statement} hypotheses.  The
search will be restricted to statements having no
\texttt{\$e}\index{\texttt{\$e} statement} hypotheses.

\begin{sloppypar} % narrow
Note:  If memory is limited, \texttt{improve all} on a large proof may
overflow memory.  If you use \texttt{set unification{\char`\_}timeout 1}
before \texttt{improve all}, there will usually be sufficient
improvement to easily recover and completely \texttt{improve} the proof
later on a larger computer.  Warning:  Once memory has overflowed, there
is no recovery.  If in doubt, save the intermediate proof (\texttt{save
new{\char`\_}proof} then \texttt{write source}) before \texttt{improve
all}.
\end{sloppypar}

If \texttt{last} is specified instead of {\em step} number, the last
step that is shown by \texttt{show new{\char`\_}proof /unknown} will be
used.

If \texttt{first} is specified instead of {\em step} number, the first
step that is shown by \texttt{show new{\char`\_}proof /unknown} will be
used.

If {\em step} is zero or negative, the -{\em step}th from last unknown
step, as shown by \texttt{show new{\char`\_}proof /unknown}, will be
used.  \texttt{improve -1} will use the penultimate
unknown step, \texttt{improve -2} {\em label} the antepenultimate, and
\texttt{improve 0} is the same as \texttt{improve last}.

Optional command qualifier:

    \texttt{/depth} {\em number} - This qualifier will cause the search
        to include
        statements with \texttt{\$e} hypotheses (but no new variables in
        the \texttt{\$e}
        hypotheses), provided that the backtracking has not exceeded the
        specified depth. {\em Warning:}  Try \texttt{/depth 1},
        then \texttt{2}, then \texttt{3}, etc.
        in sequence because of possible exponential blowups.  Save your
        work before trying \texttt{/depth} greater than \texttt{1}!

    \texttt{/no{\char`\_}distinct} - Skip trial statements that have
        \texttt{\$d}\index{\texttt{\$d} statement} requirements.
        This qualifier will prevent assignments that might violate \texttt{\$d}
        requirements but it also could miss possible legal assignments.

See also: \texttt{set search{\char`\_}limit}

\subsection{\texttt{save new\_proof} Command}\index{\texttt{save
new{\char`\_}proof} command}
Syntax:  \texttt{save new{\char`\_}proof} {\em label} [\texttt{/normal}]
   [\texttt{/compressed}]

The \texttt{save new{\char`\_}proof} command is available in the Proof
Assistant only.  It saves the proof in progress in the source
buffer\index{source buffer}.  \texttt{save new{\char`\_}proof} may be
used to save a completed proof, or it may be used to save a proof in
progress in order to work on it later.  If an incomplete proof is saved,
any user assignments with \texttt{let step} or \texttt{let variable}
will be lost, as will any ambiguous unifications\index{ambiguous
unification}\index{unification!ambiguous} that were resolved manually.
To help make recovery easier, it can be helpful to \texttt{improve all}
before \texttt{save new{\char`\_}proof} so that the incomplete proof
will have as much information as possible.

Note that the proof will not be permanently saved until a \texttt{write
source} command is issued.

Optional command qualifiers:

    \texttt{/normal} - The proof is saved in the normal format (i.e., as a
        sequence of labels, which is the defined format of the basic Metamath
        language).\index{basic language}  This is the default format that
        is used if a qualifier is omitted.

    \texttt{/compressed} - The proof is saved in the compressed format, which
        reduces storage requirements for a database.
        (See Appendix~\ref{compressed}.)


\section{Creating \LaTeX\ Output}\label{texout}\index{latex@{\LaTeX}}

You can generate \LaTeX\ output given the
information in a database.
The database must already include the necessary typesetting information
(see section \ref{tcomment} for how to provide this information).

The \texttt{show statement} and \texttt{show proof} commands each have a
special \texttt{/tex} command qualifier that produces \LaTeX\ output.
(The \texttt{show statement} command also has the
\texttt{/simple{\char`\_}tex} qualifier for output that is easier to
edit by hand.)  Before you can use them, you must open a \LaTeX\ file to
which to send their output.  A typical complete session will use this
sequence of Metamath commands:

\begin{verbatim}
read set.mm
open tex example.tex
show statement a1i /tex
show proof a1i /all/lemmon/renumber/tex
show statement uneq2 /tex
show proof uneq2 /all/lemmon/renumber/tex
close tex
\end{verbatim}

See Section~\ref{mathcomments} for information on comment markup and
Appendix~\ref{ASCII} for information on how math symbol translation is
specified.

To format and print the \LaTeX\ source, you will need the \LaTeX\
program, which is standard on most Linux installations and available for
Windows.  On Linux, in order to create a {\sc pdf} file, you will
typically type at the shell prompt
\begin{verbatim}
$ pdflatex example.tex
\end{verbatim}

\subsection{\texttt{open tex} Command}\index{\texttt{open tex} command}
Syntax:  \texttt{open tex} {\em file-name} [\texttt{/no{\char`\_}header}]

This command opens a file for writing \LaTeX\
source\index{latex@{\LaTeX}} and writes a \LaTeX\ header to the file.
\LaTeX\ source can be written with the \texttt{show proof}, \texttt{show
new{\char`\_}proof}, and \texttt{show statement} commands using the
\texttt{/tex} qualifier.

The mapping to \LaTeX\ symbols is defined in a special comment
containing a \texttt{\$t} token, described in Appendix~\ref{ASCII}.

There is an optional command qualifier:

    \texttt{/no{\char`\_}header} - This qualifier prevents a standard
        \LaTeX\ header and trailer
        from being included with the output \LaTeX\ code.


\subsection{\texttt{close tex} Command}\index{\texttt{close tex} command}
Syntax:  \texttt{close tex}

This command writes a trailer to any \LaTeX\ file\index{latex@{\LaTeX}}
that was opened with \texttt{open tex} (unless
\texttt{/no{\char`\_}header} was used with \texttt{open tex}) and closes
the \LaTeX\ file.


\section{Creating {\sc HTML} Output}\label{htmlout}

You can generate {\sc html} web pages given the
information in a database.
The database must already include the necessary typesetting information
(see section \ref{tcomment} for how to provide this information).
The ability to produce {\sc html} web pages was added in Metamath version
0.07.30.

To create an {\sc html} output file(s) for \texttt{\$a} or \texttt{\$p}
statement(s), use
\begin{quote}
    \texttt{show statement} {\em label-match} \texttt{/html}
\end{quote}
The output file will be named {\em label-match}\texttt{.html}
for each match.  When {\em
label-match} has wildcard (\texttt{*}) characters, all statements with
matching labels will have {\sc html} files produced for them.  Also,
when {\em label-match} has a wildcard (\texttt{*}) character, two additional
files, \texttt{mmdefinitions.html} and \texttt{mmascii.html} will be
produced.  To produce {\em only} these two additional files, you can use
\texttt{?*}, which will not match any statement label, in place of {\em
label-match}.

There are three other qualifiers for \texttt{show statement} that also
generate {\sc HTML} code.  These are \texttt{/alt{\char`\_}html},
\texttt{/brief{\char`\_}html}, and
\texttt{/brief{\char`\_}alt{\char`\_}html}, and are described in the
next section.

The command
\begin{quote}
    \texttt{show statement} {\em label-match} \texttt{/alt{\char`\_}html}
\end{quote}
does the same as \texttt{show statement} {\em label-match} \texttt{/html},
except that the {\sc html} code for the symbols is taken from
\texttt{althtmldef} statements instead of \texttt{htmldef} statements in
the \texttt{\$t} comment.

The command
\begin{verbatim}
show statement * /brief_html
\end{verbatim}
invokes a special mode that just produces definition and theorem lists
accompanied by their symbol strings, in a format suitable for copying and
pasting into another web page (such as the tutorial pages on the
Metamath web site).

Finally, the command
\begin{verbatim}
show statement * /brief_alt_html
\end{verbatim}
does the same as \texttt{show statement * / brief{\char`\_}html}
for the alternate {\sc html}
symbol representation.

A statement's comment can include a special notation that provides a
certain amount of control over the {\sc HTML} version of the comment.  See
Section~\ref{mathcomments} (p.~\pageref{mathcomments}) for the comment
markup features.

The \texttt{write theorem{\char`\_}list} and \texttt{write bibliography}
commands, which are described below, provide as a side effect complete
error checking for all of the features described in this
section.\index{error checking}

\subsection{\texttt{write theorem\_list}
Command}\index{\texttt{write theorem{\char`\_}list} command}
Syntax:  \texttt{write theorem{\char`\_}list}
[\texttt{/theorems{\char`\_}per{\char`\_}page} {\em number}]

This command writes a list of all of the \texttt{\$a} and \texttt{\$p}
statements in the database into a web page file
 called \texttt{mmtheorems.html}.
When additional files are needed, they are called
\texttt{mmtheorems2.html}, \texttt{mmtheorems3.html}, etc.

Optional command qualifier:

    \texttt{/theorems{\char`\_}per{\char`\_}page} {\em number} -
 This qualifier specifies the number of statements to
        write per web page.  The default is 100.

{\em Note:} In version 0.07.30\index{Metamath!limitations of version
0.07.30} of Metamath, the ``Nearby theorems'' links on the individual
web pages presuppose 100 theorems per page when linking to the theorem
list pages.  Therefore the \texttt{/theorems{\char`\_}per{\char`\_}page}
qualifier, if it specifies a number other than 100, will cause the
individual web pages to be out of sync and should not be used to
generate the main theorem list for the web site.  This is expected to be
fixed in a future version.


\subsection{\texttt{write bibliography}\label{wrbib}
Command}\index{\texttt{write bibliography} command}
Syntax:  \texttt{write bibliography} {\em filename}

This command reads an existing {\sc html} bibliographic cross-reference
file, normally called \texttt{mmbiblio.html}, and updates it per the
bibliographic links in the database comments.  The file is updated
between the {\sc html} comment lines \texttt{<!--
{\char`\#}START{\char`\#} -->} and \texttt{<!-- {\char`\#}END{\char`\#}
-->}.  The original input file is renamed to {\em
filename}\texttt{{\char`\~}1}.

A bibliographic reference is indicated with the reference name
in brackets, such as  \texttt{Theorem 3.1 of
[Monk] p.\ 22}.
See Section~\ref{htmlout} (p.~\pageref{htmlout}) for
syntax details.


\subsection{\texttt{write recent\_additions}
Command}\index{\texttt{write recent{\char`\_}additions} command}
Syntax:  \texttt{write recent{\char`\_}additions} {\em filename}
[\texttt{/limit} {\em number}]

This command reads an existing ``Recent Additions'' {\sc html} file,
normally called \texttt{mmrecent.html}, and updates it with the
descriptions of the most recently added theorems to the database.
 The file is updated between
the {\sc html} comment lines \texttt{<!-- {\char`\#}START{\char`\#} -->}
and \texttt{<!-- {\char`\#}END{\char`\#} -->}.  The original input file
is renamed to {\em filename}\texttt{{\char`\~}1}.

Optional command qualifier:

    \texttt{/limit} {\em number} -
 This qualifier specifies the number of most recent theorems to
   write to the output file.  The default is 100.


\section{Text File Utilities}

\subsection{\texttt{tools} Command}\index{\texttt{tools} command}
Syntax:  \texttt{tools}

This command invokes an easy-to-use, general purpose utility for
manipulating the contents of {\sc ascii} text files.  Upon typing
\texttt{tools}, the command-line prompt will change to \texttt{TOOLS>}
until you type \texttt{exit}.  The \texttt{tools} commands can be used
to perform simple, global edits on an input/output file,
such as making a character string substitution on each line, adding a
string to each line, and so on.  A typical use of this utility is
to build a \texttt{submit} input file to perform a common operation on a
list of statements obtained from \texttt{show label} or \texttt{show
usage}.

The actions of most of the \texttt{tools} commands can also be
performed with equivalent (and more powerful) Unix shell commands, and
some users may find those more efficient.  But for Windows users or
users not comfortable with Unix, \texttt{tools} provides an
easy-to-learn alternative that is adequate for most of the
script-building tasks needed to use the Metamath program effectively.

\subsection{\texttt{help} Command (in \texttt{tools})}
Syntax:  \texttt{help}

The \texttt{help} command lists the commands available in the
\texttt{tools} utility, along with a brief description.  Each command,
in turn, has its own help, such as \texttt{help add}.  As with
Metamath's \texttt{MM>} prompt, a complete command can be entered at
once, or just the command word can be typed, causing the program to
prompt for each argument.

\vskip 1ex
\noindent Line-by-line editing commands:

  \texttt{add} - Add a specified string to each line in a file.

  \texttt{clean} - Trim spaces and tabs on each line in a file; convert
         characters.

  \texttt{delete} - Delete a section of each line in a file.

  \texttt{insert} - Insert a string at a specified column in each line of
        a file.

  \texttt{substitute} - Make a simple substitution on each line of the file.

  \texttt{tag} - Like \texttt{add}, but restricted to a range of lines.

  \texttt{swap} - Swap the two halves of each line in a file.

\vskip 1ex
\noindent Other file-processing commands:

  \texttt{break} - Break up (tokenize) a file into a list of tokens (one per
        line).

  \texttt{build} - Build a file with multiple tokens per line from a list.

  \texttt{count} - Count the occurrences in a file of a specified string.

  \texttt{number} - Create a list of numbers.

  \texttt{parallel} - Put two files in parallel.

  \texttt{reverse} - Reverse the order of the lines in a file.

  \texttt{right} - Right-justify lines in a file (useful before sorting
         numbers).

%  \texttt{tag} - Tag edit updates in a program for revision control.

  \texttt{sort} - Sort the lines in a file with key starting at
         specified string.

  \texttt{match} - Extract lines containing (or not) a specified string.

  \texttt{unduplicate} - Eliminate duplicate occurrences of lines in a file.

  \texttt{duplicate} - Extract first occurrence of any line occurring
         more than

   \ \ \    once in a file, discarding lines occurring exactly once.

  \texttt{unique} - Extract lines occurring exactly once in a file.

  \texttt{type} (10 lines) - Display the first few lines in a file.
                  Similar to Unix \texttt{head}.

  \texttt{copy} - Similar to Unix \texttt{cat} but safe (same input
         and output file allowed).

  \texttt{submit} - Run a script containing \texttt{tools} commands.

\vskip 1ex

\noindent Note:
  \texttt{unduplicate}, \texttt{duplicate}, and \texttt{unique} also
 sort the lines as a side effect.


\subsection{Using \texttt{tools} to Build Metamath \texttt{submit}
Scripts}

The \texttt{break} command is typically used to break up a series of
statement labels, such as the output of Metamath's \texttt{show usage},
into one label per line.  The other \texttt{tools} commands can then be
used to add strings before and after each statement label to specify
commands to be performed on the statement.  The \texttt{parallel}
command is useful when a statement label must be mentioned more than
once on a line.

Very often a \texttt{submit} script for Metamath will require multiple
command lines for each statement being processed.  For example, you may
want to enter the Proof Assistant, \texttt{minimize{\char`\_}with} your
latest theorem, \texttt{save} the new proof, and \texttt{exit} the Proof
Assistant.  To accomplish this, you can build a file with these four
commands for each statement on a single line, separating each command
with a designated character such as \texttt{@}.  Then at the end you can
\texttt{substitute} each \texttt{@} with \texttt{{\char`\\}n} to break
up the lines into individual command lines (see \texttt{help
substitute}).


\subsection{Example of a \texttt{tools} Session}

To give you a quick feel for the \texttt{tools} utility, we show a
simple session where we create a file \texttt{n.txt} with 3 lines, add
strings before and after each line, and display the lines on the screen.
You can experiment with the various commands to gain experience with the
\texttt{tools} utility.

\begin{verbatim}
MM> tools
Entering the Text Tools utilities.
Type HELP for help, EXIT to exit.
TOOLS> number
Output file <n.tmp>? n.txt
First number <1>?
Last number <10>? 3
Increment <1>?
TOOLS> add
Input/output file? n.txt
String to add to beginning of each line <>? This is line
String to add to end of each line <>? .
The file n.txt has 3 lines; 3 were changed.
First change is on line 1:
This is line 1.
TOOLS> type n.txt
This is line 1.
This is line 2.
This is line 3.
TOOLS> exit
Exiting the Text Tools.
Type EXIT again to exit Metamath.
MM>
\end{verbatim}



\appendix
\chapter{Sample Representations}
\label{ASCII}

This Appendix provides a sample of {\sc ASCII} representations,
their corresponding traditional mathematical symbols,
and a discussion of their meanings
in the \texttt{set.mm} database.
These are provided in order of appearance.
This is only a partial list, and new definitions are routinely added.
A complete list is available at \url{http://metamath.org}.

These {\sc ASCII} representations, along
with information on how to display them,
are defined in the \texttt{set.mm} database file inside
a special comment called a \texttt{\$t} {\em
comment}\index{\texttt{\$t} comment} or {\em typesetting
comment.}\index{typesetting comment}
A typesetting comment
is indicated by the appearance of the
two-character string \texttt{\$t} at the beginning of the comment.
For more information,
see Section~\ref{tcomment}, p.~\pageref{tcomment}.

In the following table the ``{\sc ASCII}'' column shows the {\sc ASCII}
representation,
``Symbol'' shows the mathematical symbolic display
that corresponds to that {\sc ASCII} representation, ``Labels'' shows
the key label(s) that define the representation, and
``Description'' provides a description about the symbol.
As usual, ``iff'' is short for ``if and only if.''\index{iff}
In most cases the ``{\sc ASCII}'' column only shows
the key token, but it will sometimes show a sequence of tokens
if that is necessary for clarity.

{\setlength{\extrarowsep}{4pt} % Keep rows from being too close together
\begin{longtabu}   { @{} c c l X }
\textbf{ASCII} & \textbf{Symbol} & \textbf{Labels} & \textbf{Description} \\
\endhead
\texttt{|-} & $\vdash$ & &
  ``It is provable that...'' \\
\texttt{ph} & $\varphi$ & \texttt{wph} &
  The wff (boolean) variable phi,
  conventionally the first wff variable. \\
\texttt{ps} & $\psi$ & \texttt{wps} &
  The wff (boolean) variable psi,
  conventionally the second wff variable. \\
\texttt{ch} & $\chi$ & \texttt{wch} &
  The wff (boolean) variable chi,
  conventionally the third wff variable. \\
\texttt{-.} & $\lnot$ & \texttt{wn} &
  Logical not. E.g., if $\varphi$ is true, then $\lnot \varphi$ is false. \\
\texttt{->} & $\rightarrow$ & \texttt{wi} &
  Implies, also known as material implication.
  In classical logic the expression $\varphi \rightarrow \psi$ is true
  if either $\varphi$ is false or $\psi$ is true (or both), that is,
  $\varphi \rightarrow \psi$ has the same meaning as
  $\lnot \varphi \lor \psi$ (as proven in theorem \texttt{imor}). \\
\texttt{<->} & $\leftrightarrow$ &
  \hyperref[df-bi]{\texttt{df-bi}} &
  Biconditional (aka is-equals for boolean values).
  $\varphi \leftrightarrow \psi$ is true iff
  $\varphi$ and $\psi$ have the same value. \\
\texttt{\char`\\/} & $\lor$ &
  \makecell[tl]{{\hyperref[df-or]{\texttt{df-or}}}, \\
	         \hyperref[df-3or]{\texttt{df-3or}}} &
  Disjunction (logical ``or''). $\varphi \lor \psi$ is true iff
  $\varphi$, $\psi$, or both are true. \\
\texttt{/\char`\\} & $\land$ &
  \makecell[tl]{{\hyperref[df-an]{\texttt{df-an}}}, \\
                 \hyperref[df-3an]{\texttt{df-3an}}} &
  Conjunction (logical ``and''). $\varphi \land \psi$ is true iff
  both $\varphi$ and $\psi$ are true. \\
\texttt{A.} & $\forall$ &
  \texttt{wal} &
  For all; the wff $\forall x \varphi$ is true iff
  $\varphi$ is true for all values of $x$. \\
\texttt{E.} & $\exists$ &
  \hyperref[df-ex]{\texttt{df-ex}} &
  There exists; the wff
  $\exists x \varphi$ is true iff
  there is at least one $x$ where $\varphi$ is true. \\
\texttt{[ y / x ]} & $[ y / x ]$ &
  \hyperref[df-sb]{\texttt{df-sb}} &
  The wff $[ y / x ] \varphi$ produces
  the result when $y$ is properly substituted for $x$ in $\varphi$
  ($y$ replaces $x$).
  % This is elsb4
  % ( [ x / y ] z e. y <-> z e. x )
  For example,
  $[ x / y ] z \in y$ is the same as $z \in x$. \\
\texttt{E!} & $\exists !$ &
  \hyperref[df-eu]{\texttt{df-eu}} &
  There exists exactly one;
  $\exists ! x \varphi$ is true iff
  there is at least one $x$ where $\varphi$ is true. \\
\texttt{\{ y | phi \}}  & $ \{ y | \varphi \}$ &
  \hyperref[df-clab]{\texttt{df-clab}} &
  The class of all sets where $\varphi$ is true. \\
\texttt{=} & $ = $ &
  \hyperref[df-cleq]{\texttt{df-cleq}} &
  Class equality; $A = B$ iff $A$ equals $B$. \\
\texttt{e.} & $ \in $ &
  \hyperref[df-clel]{\texttt{df-clel}} &
  Class membership; $A \in B$ if $A$ is a member of $B$. \\
\texttt{{\char`\_}V} & {\rm V} &
  \hyperref[df-v]{\texttt{df-v}} &
  Class of all sets (not itself a set). \\
\texttt{C\_} & $ \subseteq $ &
  \hyperref[df-ss]{\texttt{df-ss}} &
  Subclass (subset); $A \subseteq B$ is true iff
  $A$ is a subclass of $B$. \\
\texttt{u.} & $ \cup $ &
  \hyperref[df-un]{\texttt{df-un}} &
  $A \cup B$ is the union of classes $A$ and $B$. \\
\texttt{i^i} & $ \cap $ &
  \hyperref[df-in]{\texttt{df-in}} &
  $A \cap B$ is the intersection of classes $A$ and $B$. \\
\texttt{\char`\\} & $ \setminus $ &
  \hyperref[df-dif]{\texttt{df-dif}} &
  $A \setminus B$ (set difference)
  is the class of all sets in $A$ except for those in $B$. \\
\texttt{(/)} & $ \varnothing $ &
  \hyperref[df-nul]{\texttt{df-nul}} &
  $ \varnothing $ is the empty set (aka null set). \\
\texttt{\char`\~P} & $ \cal P $ &
  \hyperref[df-pw]{\texttt{df-pw}} &
  Power class. \\
\texttt{<.\ A , B >.} & $\langle A , B \rangle$ &
  \hyperref[df-op]{\texttt{df-op}} &
  The ordered pair $\langle A , B \rangle$. \\
\texttt{( F ` A )} & $ ( F ` A ) $ &
  \hyperref[df-fv]{\texttt{df-fv}} &
  The value of function $F$ when applied to $A$. \\
\texttt{\_i} & $ i $ &
  \texttt{df-i} &
  The square root of negative one. \\
\texttt{x.} & $ \cdot $ &
  \texttt{df-mul} &
  Complex number multiplication; $2~\cdot~3~=~6$. \\
\texttt{CC} & $ \mathbb{C} $ &
  \texttt{df-c} &
  The set of complex numbers. \\
\texttt{RR} & $ \mathbb{R} $ &
  \texttt{df-r} &
  The set of real numbers. \\
\end{longtabu}
} % end of extrarowsep

\chapter{Compressed Proofs}
\label{compressed}\index{compressed proof}\index{proof!compressed}

The proofs in the \texttt{set.mm} set theory database are stored in compressed
format for efficiency.  Normally you needn't concern yourself with the
compressed format, since you can display it with the usual proof display tools
in the Metamath program (\texttt{show proof}\ldots) or convert it to the normal
RPN proof format described in Section~\ref{proof} (with \texttt{save proof}
{\em label} \texttt{/normal}).  However for sake of completeness we describe the
format here and show how it maps to the normal RPN proof format.

A compressed proof, located between \texttt{\$=} and \texttt{\$.}\ keywords, consists
of a left parenthesis, a sequence of statement labels, a right parenthesis,
and a sequence of upper-case letters \texttt{A} through \texttt{Z} (with optional
white space between them).  White space must surround the parentheses
and the labels.  The left parenthesis tells Metamath that a
compressed proof follows.  (A normal RPN proof consists of just a sequence of
labels, and a parenthesis is not a legal character in a label.)

The sequence of upper-case letters corresponds to a sequence of integers
with the following mapping.  Each integer corresponds to a proof step as
described later.
\begin{center}
  \texttt{A} = 1 \\
  \texttt{B} = 2 \\
   \ldots \\
  \texttt{T} = 20 \\
  \texttt{UA} = 21 \\
  \texttt{UB} = 22 \\
   \ldots \\
  \texttt{UT} = 40 \\
  \texttt{VA} = 41 \\
  \texttt{VB} = 42 \\
   \ldots \\
  \texttt{YT} = 120 \\
  \texttt{UUA} = 121 \\
   \ldots \\
  \texttt{YYT} = 620 \\
  \texttt{UUUA} = 621 \\
   etc.
\end{center}

In other words, \texttt{A} through \texttt{T} represent the
least-significant digit in base 20, and \texttt{U} through \texttt{Y}
represent zero or more most-significant digits in base 5, where the
digits start counting at 1 instead of the usual 0. With this scheme, we
don't need white space between these ``numbers.''

(In the design of the compressed proof format, only upper-case letters,
as opposed to say all non-whitespace printable {\sc ascii} characters other than
%\texttt{\$}, was chosen to make the compressed proof a little less
%displeasing to the eye, at the expense of a typical 20\% compression
\texttt{\$}, were chosen so as not to collide with most text editor
searches, at the expense of a typical 20\% compression
loss.  The base 5/base 20 grouping, as opposed to say base 6/base 19,
was chosen by experimentally determining the grouping that resulted in
best typical compression.)

The letter \texttt{Z} identifies (tags) a proof step that is identical to one
that occurs later on in the proof; it helps shorten the proof by not requiring
that identical proof steps be proved over and over again (which happens often
when building wff's).  The \texttt{Z} is placed immediately after the
least-significant digit (letters \texttt{A} through \texttt{T}) that ends the integer
corresponding to the step to later be referenced.

The integers that the upper-case letters correspond to are mapped to labels as
follows.  If the statement being proved has $m$ mandatory hypotheses, integers
1 through $m$ correspond to the labels of these hypotheses in the order shown
by the \texttt{show statement ... / full} command, i.e., the RPN order\index{RPN
order} of the mandatory
hypotheses.  Integers $m+1$ through $m+n$ correspond to the labels enclosed in
the parentheses of the compressed proof, in the order that they appear, where
$n$ is the number of those labels.  Integers $m+n+1$ on up don't directly
correspond to statement labels but point to proof steps identified with the
letter \texttt{Z}, so that these proof steps can be referenced later in the
proof.  Integer $m+n+1$ corresponds to the first step tagged with a \texttt{Z},
$m+n+2$ to the second step tagged with a \texttt{Z}, etc.  When the compressed
proof is converted to a normal proof, the entire subproof of a step tagged
with \texttt{Z} replaces the reference to that step.

For efficiency, Metamath works with compressed proofs directly, without
converting them internally to normal proofs.  In addition to the usual
error-checking, an error message is given if (1) a label in the label list in
parentheses does not refer to a previous \texttt{\$p} or \texttt{\$a} statement or a
non-mandatory hypothesis of the statement being proved and (2) a proof step
tagged with \texttt{Z} is referenced before the step tagged with the \texttt{Z}.

Just as in a normal proof under development (Section~\ref{unknown}), any step
or subproof that is not yet known may be represented with a single \texttt{?}.
White space does not have to appear between the \texttt{?}\ and the upper-case
letters (or other \texttt{?}'s) representing the remainder of the proof.

% April 1, 2004 Appendix C has been added back in with corrections.
%
% May 20, 2003 Appendix C was removed for now because there was a problem found
% by Bob Solovay
%
% Also, removed earlier \ref{formalspec} 's (3 cases above)
%
% Bob Solovay wrote on 30 Nov 2002:
%%%%%%%%%%%%% (start of email comment )
%      3. My next set of comments concern appendix C. I read this before I
% read Chapter 4. So I first noted that the system as presented in the
% Appendix lacked a certain formal property that I thought desirable. I
% then came up with a revised formal system that had this property. Upon
% reading Chapter 4, I noticed that the revised system was closer to the
% treatment in Chapter 4 than the system in Appendix C.
%
%         First a very minor correction:
%
%         On page 142 line 2: The condition that V(e) != V(f) should only be
% required of e, f in T such that e != f.
%
%         Here is a natural property [transitivity] that one would like
% the formal system to have:
%
%         Let Gamma be a set of statements. Suppose that the statement Phi
% is provable from Gamma and that the statement Psi is provable from Gamma
% \cup {Phi}. Then Psi is provable from Gamma.
%
%         I shall present an example to show that this property does not
% hold for the formal systems of Appendix C:
%
%         I write the example in metamath style:
%
% $c A B C D E $.
% $v x y
%
% ${
% tx $f A x $.
% ty $f B y $.
% ax1 $a C x y $.
% $}
%
% ${
% tx $f A x $.
% ty $f B y $.
% ax2-h1 $e C x y $.
% ax2 $a D y $.
% $}
%
% ${
% ty $f B y $.
% ax3-h1 $e D y $.
% ax3 $a E y $.
% $}
%
% $(These three axioms are Gamma $)
%
% ${
% tx $f A x $.
% ty $f B y $.
% Phi $p D y $=
% tx ty tx ty ax1 ax2 $.
% $}
%
% ${
% ty $f B y $.
% Psi $p E y $=
% ty ty Phi ax3 $.
% $}
%
%
% I omit the formal proofs of the following claims. [I will be glad to
% supply them upon request.]
%
% 1) Psi is not provable from Gamma;
%
% 2) Psi is provable from Gamma + Phi.
%
% Here "provable" refers to the formalism of Appendix C.
%
% The trouble of course is that Psi is lacking the variable declaration
%
% $f Ax $.
%
% In the Metamath system there is no trouble proving Psi. I attach a
% metamath file that shows this and which has been checked by the
% metamath program.
%
% I next want to indicate how I think the treatment in Appendix C should
% be revised so as to conform more closely to the metamath system of the
% main text. The revised system *does* have the transitivity property.
%
% We want to give revised definitions of "statement" and
% "provable". [cf. sections C.2.4. and C.2.5] Our new definitions will
% use the definitions given in Appendix C. So we take the following
% tack. We refer to the original notions as o-statement and o-provable. And
% we refer to the notions we are defining as n-statement and n-provable.
%
%         A n-statement is an o-statement in which the only variables
% that appear in the T component are mandatory.
%
%         To any o-statement we can associate its reduct which is a
% n-statement by dropping all the elements of T or D which contain
% non-mandatory variables.
%
%         An n-statement gamma is n-provable if there is an o-statement
% gamma' which has gamma as its reduct andf such that gamma' is
% o-provable.
%
%         It seems to me [though I am not completely sure on this point]
% that n-provability corresponds to metamath provability as discussed
% say in Chapter 4.
%
%         Attached to this letter is the metamath proof of Phi and Psi
% from Gamma discussed above.
%
%         I am still brooding over the question of whether metamath
% correctly formalizes set-theory. No doubt I will have some questions
% re this after my thoughts become clearer.
%%%%%%%%%%%%%%%% (end of email comment)

%%%%%%%%%%%%%%%% (start of 2nd email comment from Bob Solovay 1-Apr-04)
%
%         I hope that Appendix C is the one that gives a "formal" treatment
% of Metamath. At any rate, thats the appendix I want to comment on.
%
%         I'm going to suggest two changes in the definition.
%
%         First change (in the definition of statement): Require that the
% sets D, T, and E be finite.
%
%         Probably things are fine as you give them. But in the applications
% to the main metamath system they will always be finite, and its useful in
% thinking about things [at least for me] to stick to the finite case.
%
%         Second change:
%
%         First let me give an approximate description. Remove the dummy
% variables from the statement. Instead, include them in the proof.
%
%         More formally: Require that T consists of type declarations only
% for mandatory variables. Require that all the pairs in D consist of
% mandatory variables.
%
%         At the start of a proof we are allowed to declare a finite number
% of dummy variables [provided that none of them appear in any of the
% statements in E \cup {A}. We have to supply type declarations for all the
% dummy variables. We are allowed to add new $d statements referring to
% either the mandatory or dummy variables. But we require that no new $d
% statement references only mandatory variables.
%
%         I find this way of doing things more conceptual than the treatment
% in Appendix C. But the change [which I will use implicitly in later
% letters about doing Peano] is mainly aesthetic. I definitely claim that my
% results on doing Peano all apply to Metamath as it is presented in your
% book.
%
%         --Bob
%
%%%%%%%%%%%%%%%% (end of 2nd email comment)

%%
%% When uncommenting the below, also uncomment references above to {formalspec}
%%
\chapter{Metamath's Formal System}\label{formalspec}\index{Metamath!as a formal
system}

\section{Introduction}

\begin{quote}
  {\em Perfection is when there is no longer anything more to take away.}
    \flushright\sc Antoine de
     Saint-Exupery\footnote{\cite[p.~3-25]{Campbell}.}\\
\end{quote}\index{de Saint-Exupery, Antoine}

This appendix describes the theory behind the Metamath language in an abstract
way intended for mathematicians.  Specifically, we construct two
set-theo\-ret\-i\-cal objects:  a ``formal system'' (roughly, a set of syntax
rules, axioms, and logical rules) and its ``universe'' (roughly, the set of
theorems derivable in the formal system).  The Metamath computer language
provides us with a way to describe specific formal systems and, with the aid of
a proof provided by the user, to verify that given theorems
belong to their universes.

To understand this appendix, you need a basic knowledge of informal set theory.
It should be sufficient to understand, for example, Ch.\ 1 of Munkres' {\em
Topology} \cite{Munkres}\index{Munkres, James R.} or the
introductory set theory chapter
in many textbooks that introduce abstract mathematics. (Note that there are
minor notational differences among authors; e.g.\ Munkres uses $\subset$ instead
of our $\subseteq$ for ``subset.''  We use ``included in'' to mean ``a subset
of,'' and ``belongs to'' or ``is contained in'' to mean ``is an element of.'')
What we call a ``formal'' description here, unlike earlier, is actually an
informal description in the ordinary language of mathematicians.  However we
provide sufficient detail so that a mathematician could easily formalize it,
even in the language of Metamath itself if desired.  To understand the logic
examples at the end of this appendix, familiarity with an introductory book on
mathematical logic would be helpful.

\section{The Formal Description}

\subsection[Preliminaries]{Preliminaries\protect\footnotemark}%
\footnotetext{This section is taken mostly verbatim
from Tarski \cite[p.~63]{Tarski1965}\index{Tarski, Alfred}.}

By $\omega$ we denote the set of all natural numbers (non-negative integers).
Each natural number $n$ is identified with the set of all smaller numbers: $n =
\{ m | m < n \}$.  The formula $m < n$ is thus equivalent to the condition: $m
\in n$ and $m,n \in \omega$. In particular, 0 is the number zero and at the
same time the empty set $\varnothing$, $1=\{0\}$, $2=\{0,1\}$, etc. ${}^B A$
denotes the set of all functions on $B$ to $A$ (i.e.\ with domain $B$ and range
included in $A$).  The members of ${}^\omega A$ are what are called {\em simple
infinite sequences},\index{simple infinite sequence}
with all {\em terms}\index{term} in $A$.  In case $n \in \omega$, the
members of ${}^n A$ are referred to as {\em finite $n$-termed
sequences},\index{finite $n$-termed
sequence} again
with terms in $A$.  The consecutive terms (function values) of a finite or
infinite sequence $f$ are denoted by $f_0, f_1, \ldots ,f_n,\ldots$.  Every
finite sequence $f \in \bigcup _{n \in \omega} {}^n A$ uniquely determines the
number $n$ such that $f \in {}^n A$; $n$ is called the {\em
length}\index{length of a sequence ({$"|\ "|$})} of $f$ and
is denoted by $|f|$.  $\langle a \rangle$ is the sequence $f$ with $|f|=1$ and
$f_0=a$; $\langle a,b \rangle$ is the sequence $f$ with $|f|=2$, $f_0=a$,
$f_1=b$; etc.  Given two finite sequences $f$ and $g$, we denote by $f\frown g$
their {\em concatenation},\index{concatenation} i.e., the
finite sequence $h$ determined by the
conditions:
\begin{eqnarray*}
& |h| = |f|+|g|;&  \\
& h_n = f_n & \mbox{\ for\ } n < |f|;  \\
& h_{|f|+n} = g_n & \mbox{\ for\ } n < |g|.
\end{eqnarray*}

\subsection{Constants, Variables, and Expressions}

A formal system has a set of {\em symbols}\index{symbol!in
a formal system} denoted
by $\mbox{\em SM}$.  A
precise set-theo\-ret\-i\-cal definition of this set is unimportant; a symbol
could be considered a primitive or atomic element if we wish.  We assume this
set is divided into two disjoint subsets:  a set $\mbox{\em CN}$ of {\em
constants}\index{constant!in a formal system} and a set $\mbox{\em VR}$ of
{\em variables}.\index{variable!in a formal system}  $\mbox{\em CN}$ and
$\mbox{\em VR}$ are each assumed to consist of countably many symbols which
may be arranged in finite or simple infinite sequences $c_0, c_1, \ldots$ and
$v_0, v_1, \ldots$ respectively, without repeating terms.  We will represent
arbitrary symbols by metavariables $\alpha$, $\beta$, etc.

{\footnotesize\begin{quotation}
{\em Comment.} The variables $v_0, v_1, \ldots$ of our formal system
correspond to what are usually considered ``metavariables'' in
descriptions of specific formal systems in the literature.  Typically,
when describing a specific formal system a book will postulate a set of
primitive objects called variables, then proceed to describe their
properties using metavariables that range over them, never mentioning
again the actual variables themselves.  Our formal system does not
mention these primitive variable objects at all but deals directly with
metavariables, as its primitive objects, from the start.  This is a
subtle but key distinction you should keep in mind, and it makes our
definition of ``formal system'' somewhat different from that typically
found in the literature.  (So, the $\alpha$, $\beta$, etc.\ above are
actually ``metametavariables'' when used to represent $v_0, v_1,
\ldots$.)
\end{quotation}}

Finite sequences all terms of which are symbols are called {\em
expressions}.\index{expression!in a formal system}  $\mbox{\em EX}$ is
the set of all expressions; thus
\begin{displaymath}
\mbox{\em EX} = \bigcup _{n \in \omega} {}^n \mbox{\em SM}.
\end{displaymath}

A {\em constant-prefixed expression}\index{constant-prefixed expression}
is an expression of non-zero length
whose first term is a constant.  We denote the set of all constant-prefixed
expressions by $\mbox{\em EX}_C = \{ e \in \mbox{\em EX} | ( |e| > 0 \wedge
e_0 \in \mbox{\em CN} ) \}$.

A {\em constant-variable pair}\index{constant-variable pair}
is an expression of length 2 whose first term
is a constant and whose second term is a variable.  We denote the set of all
constant-variable pairs by $\mbox{\em EX}_2 = \{ e \in \mbox{\em EX}_C | ( |e|
= 2 \wedge e_1 \in \mbox{\em VR} ) \}$.


{\footnotesize\begin{quotation}
{\em Relationship to Metamath.} In general, the set $\mbox{\em SM}$
corresponds to the set of declared math symbols in a Metamath database, the
set $\mbox{\em CN}$ to those declared with \texttt{\$c} statements, and the set
$\mbox{\em VR}$ to those declared with \texttt{\$v} statements.  Of course a
Metamath database can only have a finite number of math symbols, whereas
formal systems in general can have an infinite number, although the number of
Metamath math symbols available is in principle unlimited.

The set $\mbox{\em EX}_C$ corresponds to the set of permissible expressions
for \texttt{\$e}, \texttt{\$a}, and \texttt{\$p} statements.  The set $\mbox{\em EX}_2$
corresponds to the set of permissible expressions for \texttt{\$f} statements.
\end{quotation}}

We denote by ${\cal V}(e)$ the set of all variables in an expression $e \in
\mbox{\em EX}$, i.e.\ the set of all $\alpha \in \mbox{\em VR}$ such that
$\alpha = e_n$ for some $n < |e|$.  We also denote (with abuse of notation) by
${\cal V}(E)$ the set of all variables in a collection of expressions $E
\subseteq \mbox{\em EX}$, i.e.\ $\bigcup _{e \in E} {\cal V}(e)$.


\subsection{Substitution}

Given a function $F$ from $\mbox{\em VR}$ to
$\mbox{\em EX}$, we
denote by $\sigma_{F}$ or just $\sigma$ the function from $\mbox{\em EX}$ to
$\mbox{\em EX}$ defined recursively for nonempty sequences by
\begin{eqnarray*}
& \sigma(<\alpha>) = F(\alpha) & \mbox{for\ } \alpha \in \mbox{\em VR}; \\
& \sigma(<\alpha>) = <\alpha> & \mbox{for\ } \alpha \not\in \mbox{\em VR}; \\
& \sigma(g \frown h) = \sigma(g) \frown
    \sigma(h) & \mbox{for\ } g,h \in \mbox{\em EX}.
\end{eqnarray*}
We also define $\sigma(\varnothing)=\varnothing$.  We call $\sigma$ a {\em
simultaneous substitution}\index{substitution!variable}\index{variable
substitution} (or just {\em substitution}) with {\em substitution
map}\index{substitution map} $F$.

We also denote (with abuse of notation) by $\sigma(E)$ a substitution on a
collection of expressions $E \subseteq \mbox{\em EX}$, i.e.\ the set $\{
\sigma(e) | e \in E \}$.  The collection $\sigma(E)$ may of course contain
fewer expressions than $E$ because duplicate expressions could result from the
substitution.

\subsection{Statements}

We denote by $\mbox{\em DV}$ the set of all
unordered pairs $\{\alpha, \beta \} \subseteq \mbox{\em VR}$ such that $\alpha
\neq \beta$.  $\mbox{\em DV}$ stands for ``distinct variables.''

A {\em pre-statement}\index{pre-statement!in a formal system} is a
quadruple $\langle D,T,H,A \rangle$ such that
$D\subseteq \mbox{\em DV}$, $T\subseteq \mbox{\em EX}_2$, $H\subseteq
\mbox{\em EX}_C$ and $H$ is finite,
$A\in \mbox{\em EX}_C$, ${\cal V}(H\cup\{A\}) \subseteq
{\cal V}(T)$, and $\forall e,f\in T {\ } {\cal V}(e) \neq {\cal V}(f)$ (or
equivalently, $e_1 \ne f_1$) whenever $e \neq f$. The terms of the quadruple are called {\em
distinct-variable restrictions},\index{disjoint-variable restriction!in a
formal system} {\em variable-type hypotheses},\index{variable-type
hypothesis!in a formal system} {\em logical hypotheses},\index{logical
hypothesis!in a formal system} and the {\em assertion}\index{assertion!in a
formal system} respectively.  We denote by $T_M$ ({\em mandatory variable-type
hypotheses}\index{mandatory variable-type hypothesis!in a formal system}) the
subset of $T$ such that ${\cal V}(T_M) ={\cal V}(H \cup \{A\})$.  We denote by
$D_M=\{\{\alpha,\beta\}\in D|\{\alpha,\beta\}\subseteq {\cal V}(T_M)\}$ the
{\em mandatory distinct-variable restrictions}\index{mandatory
disjoint-variable restriction!in a formal system} of the pre-statement.
The set
of {\em mandatory hypotheses}\index{mandatory hypothesis!in a formal system}
is $T_M\cup H$.  We call the quadruple $\langle D_M,T_M,H,A \rangle$
the {\em reduct}\index{reduct!in a formal system} of
the pre-statement $\langle D,T,H,A \rangle$.

A {\em statement} is the reduct of some pre-statement\index{statement!in a
formal system}.  A statement is therefore a special kind of pre-statement;
in particular, a statement is the reduct of itself.

{\footnotesize\begin{quotation}
{\em Comment.}  $T$ is a set of expressions, each of length 2, that associate
a set of constants (``variable types'') with a set of variables.  The
condition ${\cal V}(H\cup\{A\}) \subseteq {\cal V}(T) $
means that each variable occurring in a statement's logical
hypotheses or assertion must have an associated variable-type hypothesis or
``type declaration,'' in  analogy to a computer programming language, where a
variable must be declared to be say, a string or an integer.  The requirement
that $\forall e,f\in T \, e_1 \ne f_1$ for $e\neq f$
means that each variable must be
associated with a unique constant designating its variable type; e.g., a
variable might be a ``wff'' or a ``set'' but not both.

Distinct-variable restrictions are used to specify what variable substitutions
are permissible to make for the statement to remain valid.  For example, in
the theorem scheme of set theory $\lnot\forall x\,x=y$ we may not substitute
the same variable for both $x$ and $y$.  On the other hand, the theorem scheme
$x=y\to y=x$ does not require that $x$ and $y$ be distinct, so we do not
require a distinct-variable restriction, although having one
would cause no harm other than making the scheme less general.

A mandatory variable-type hypothesis is one whose variable exists in a logical
hypothesis or the assertion.  A provable pre-statement
(defined below) may require
non-mandatory variable-type hypotheses that effectively introduce ``dummy''
variables for use in its proof.  Any number of dummy variables might
be required by a specific proof; indeed, it has been shown by H.\
Andr\'{e}ka\index{Andr{\'{e}}ka, H.} \cite{Nemeti} that there is no finite
upper bound to the number of dummy variables needed to prove an arbitrary
theorem in first-order logic (with equality) having a fixed number $n>2$ of
individual variables.  (See also the Comment on p.~\pageref{nodd}.)
For this reason we do not set a finite size bound on the collections $D$ and
$T$, although in an actual application (Metamath database) these will of
course be finite, increased to whatever size is necessary as more
proofs are added.
\end{quotation}}

{\footnotesize\begin{quotation}
{\em Relationship to Metamath.} A pre-statement of a formal system
corresponds to an extended frame in a Metamath database
(Section~\ref{frames}).  The collections $D$, $T$, and $H$ correspond
respectively to the \texttt{\$d}, \texttt{\$f}, and \texttt{\$e}
statement collections in an extended frame.  The expression $A$
corresponds to the \texttt{\$a} (or \texttt{\$p}) statement in an
extended frame.

A statement of a formal system corresponds to a frame in a Metamath
database.
\end{quotation}}

\subsection{Formal Systems}

A {\em formal system}\index{formal system} is a
triple $\langle \mbox{\em CN},\mbox{\em
VR},\Gamma\rangle$ where $\Gamma$ is a set of statements.  The members of
$\Gamma$ are called {\em axiomatic statements}.\index{axiomatic
statement!in a formal system}  Sometimes we will refer to a
formal system by just $\Gamma$ when $\mbox{\em CN}$ and $\mbox{\em VR}$ are
understood.

Given a formal system $\Gamma$, the {\em closure}\index{closure}\footnote{This
definition of closure incorporates a simplification due to
Josh Purinton.\index{Purinton, Josh}.} of a
pre-statement
$\langle D,T,H,A \rangle$ is the smallest set $C$ of expressions
such that:
%\begin{enumerate}
%  \item $T\cup H\subseteq C$; and
%  \item If for some axiomatic statement
%    $\langle D_M',T_M',H',A' \rangle \in \Gamma_A$, for
%    some $E \subseteq C$, some $F \subseteq C-T$ (where ``-'' denotes
%    set difference), and some substitution
%    $\sigma$ we have
%    \begin{enumerate}
%       \item $\sigma(T_M') = E$ (where, as above, the $M$ denotes the
%           mandatory variable-type hypotheses of $T^A$);
%       \item $\sigma(H') = F$;
%       \item for all $\{\alpha,\beta\}\in D^A$ and $\subseteq
%         {\cal V}(T_M')$, for all $\gamma\in {\cal V}(\sigma(\langle \alpha
%         \rangle))$, and for all $\delta\in  {\cal V}(\sigma(\langle \beta
%         \rangle))$, we have $\{\gamma, \delta\} \in D$;
%   \end{enumerate}
%   then $\sigma(A') \in C$.
%\end{enumerate}
\begin{list}{}{\itemsep 0.0pt}
  \item[1.] $T\cup H\subseteq C$; and
  \item[2.] If for some axiomatic statement
    $\langle D_M',T_M',H',A' \rangle \in
       \Gamma$ and for some substitution
    $\sigma$ we have
    \begin{enumerate}
       \item[a.] $\sigma(T_M' \cup H') \subseteq C$; and
       \item[b.] for all $\{\alpha,\beta\}\in D_M'$, for all $\gamma\in
         {\cal V}(\sigma(\langle \alpha
         \rangle))$, and for all $\delta\in  {\cal V}(\sigma(\langle \beta
         \rangle))$, we have $\{\gamma, \delta\} \in D$;
   \end{enumerate}
   then $\sigma(A') \in C$.
\end{list}
A pre-statement $\langle D,T,H,A
\rangle$ is {\em provable}\index{provable statement!in a formal
system} if $A\in C$ i.e.\ if its assertion belongs to its
closure.  A statement is {\em provable} if it is
the reduct of a provable pre-statement.
The {\em universe}\index{universe of a formal system}
of a formal system is
the collection of all of its provable statements.  Note that the
set of axiomatic statements $\Gamma$ in a formal system is a subset of its
universe.

{\footnotesize\begin{quotation}
{\em Comment.} The first condition in the definition of closure simply says
that the hypotheses of the pre-statement are in its closure.

Condition 2(a) says that a substitution exists that makes the
mandatory hypotheses of an axiomatic statement exactly match some members of
the closure.  This is what we explicitly demonstrate in a Metamath language
proof.

%Conditions 2(a) and 2(b) say that a substitution exists that makes the
%(mandatory) hypotheses of an axiomatic statement exactly match some members of
%the closure.  This is what we explicitly demonstrate with a Metamath language
%proof.
%
%The set of expressions $F$ in condition 2(b) excludes the variable-type
%hypotheses; this is done because non-mandatory variable-type hypotheses are
%effectively ``dropped'' as irrelevant whereas logical hypotheses must be
%retained to achieve a consistent logical system.

Condition 2(b) describes how distinct-variable restrictions in the axiomatic
statement must be met.  It means that after a substitution for two variables
that must be distinct, the resulting two expressions must either contain no
variables, or if they do, they may not have variables in common, and each pair
of any variables they do have, with one variable from each expression, must be
specified as distinct in the original statement.
\end{quotation}}

{\footnotesize\begin{quotation}
{\em Relationship to Metamath.} Axiomatic statements
 and provable statements in a formal
system correspond to the frames for \texttt{\$a} and \texttt{\$p} statements
respectively in a Metamath database.  The set of axiomatic statements is a
subset of the set of provable statements in a formal system, although in a
Metamath database a \texttt{\$a} statement is distinguished by not having a
proof.  A Metamath language proof for a \texttt{\$p} statement tells the computer
how to explicitly construct a series of members of the closure ultimately
leading to a demonstration that the assertion
being proved is in the closure.  The actual closure typically contains
an infinite number of expressions.  A formal system itself does not have
an explicit object called a ``proof'' but rather the existence of a proof
is implied indirectly by membership of an assertion in a provable
statement's closure.  We do this to make the formal system easier
to describe in the language of set theory.

We also note that once established as provable, a statement may be considered
to acquire the same status as an axiomatic statement, because if the set of
axiomatic statements is extended with a provable statement, the universe of
the formal system remains unchanged (provided that $\mbox{\em VR}$ is
infinite).
In practice, this means we can build a hierarchy of provable statements to
more efficiently establish additional provable statements.  This is
what we do in Metamath when we allow proofs to reference previous
\texttt{\$p} statements as well as previous \texttt{\$a} statements.
\end{quotation}}

\section{Examples of Formal Systems}

{\footnotesize\begin{quotation}
{\em Relationship to Metamath.} The examples in this section, except Example~2,
are for the most part exact equivalents of the development in the set
theory database \texttt{set.mm}.  You may want to compare Examples~1, 3, and 5
to Section~\ref{metaaxioms}, Example 4 to Sections~\ref{metadefprop} and
\ref{metadefpred}, and Example 6 to
Section~\ref{setdefinitions}.\label{exampleref}
\end{quotation}}

\subsection{Example~1---Propositional Calculus}\index{propositional calculus}

Classical propositional calculus can be described by the following formal
system.  We assume the set of variables is infinite.  Rather than denoting the
constants and variables by $c_0, c_1, \ldots$ and $v_0, v_1, \ldots$, for
readability we will instead use more conventional symbols, with the
understanding of course that they denote distinct primitive objects.
Also for readability we may omit commas between successive terms of a
sequence; thus $\langle \mbox{wff\ } \varphi\rangle$ denotes
$\langle \mbox{wff}, \varphi\rangle$.

Let
\begin{itemize}
  \item[] $\mbox{\em CN}=\{\mbox{wff}, \vdash, \to, \lnot, (,)\}$
  \item[] $\mbox{\em VR}=\{\varphi,\psi,\chi,\ldots\}$
  \item[] $T = \{\langle \mbox{wff\ } \varphi\rangle,
             \langle \mbox{wff\ } \psi\rangle,
             \langle \mbox{wff\ } \chi\rangle,\ldots\}$, i.e.\ those
             expressions of length 2 whose first member is $\mbox{\rm wff}$
             and whose second member belongs to $\mbox{\em VR}$.\footnote{For
convenience we let $T$ be an infinite set; the definition of a statement
permits this in principle.  Since a Metamath source file has a finite size, in
practice we must of course use appropriate finite subsets of this $T$,
specifically ones containing at least the mandatory variable-type
hypotheses.  Similarly, in the source file we introduce new variables as
required, with the understanding that a potentially infinite number of
them are available.}
\noindent Then $\Gamma$ consists of the axiomatic statements that
are the reducts of the following pre-statements:
    \begin{itemize}
      \item[] $\langle\varnothing,T,\varnothing,
               \langle \mbox{wff\ }(\varphi\to\psi)\rangle\rangle$
      \item[] $\langle\varnothing,T,\varnothing,
               \langle \mbox{wff\ }\lnot\varphi\rangle\rangle$
      \item[] $\langle\varnothing,T,\varnothing,
               \langle \vdash(\varphi\to(\psi\to\varphi))
               \rangle\rangle$
      \item[] $\langle\varnothing,T,
               \varnothing,
               \langle \vdash((\varphi\to(\psi\to\chi))\to
               ((\varphi\to\psi)\to(\varphi\to\chi)))
               \rangle\rangle$
      \item[] $\langle\varnothing,T,
               \varnothing,
               \langle \vdash((\lnot\varphi\to\lnot\psi)\to
               (\psi\to\varphi))\rangle\rangle$
      \item[] $\langle\varnothing,T,
               \{\langle\vdash(\varphi\to\psi)\rangle,
                 \langle\vdash\varphi\rangle\},
               \langle\vdash\psi\rangle\rangle$
    \end{itemize}
\end{itemize}

(For example, the reduct of $\langle\varnothing,T,\varnothing,
               \langle \mbox{wff\ }(\varphi\to\psi)\rangle\rangle$
is
\begin{itemize}
\item[] $\langle\varnothing,
\{\langle \mbox{wff\ } \varphi\rangle,
             \langle \mbox{wff\ } \psi\rangle\},
             \varnothing,
               \langle \mbox{wff\ }(\varphi\to\psi)\rangle\rangle$,
\end{itemize}
which is the first axiomatic statement.)

We call the members of $\mbox{\em VR}$ {\em wff variables} or (in the context
of first-order logic which we will describe shortly) {\em wff metavariables}.
Note that the symbols $\phi$, $\psi$, etc.\ denote actual specific members of
$\mbox{\em VR}$; they are not metavariables of our expository language (which
we denote with $\alpha$, $\beta$, etc.) but are instead (meta)constant symbols
(members of $\mbox{\em SM}$) from the point of view of our expository
language.  The equivalent system of propositional calculus described in
\cite{Tarski1965} also uses the symbols $\phi$, $\psi$, etc.\ to denote wff
metavariables, but in \cite{Tarski1965} unlike here those are metavariables of
the expository language and not primitive symbols of the formal system.

The first two statements define wffs: if $\varphi$ and $\psi$ are wffs, so is
$(\varphi \to \psi)$; if $\varphi$ is a wff, so is $\lnot\varphi$. The next
three are the axioms of propositional calculus: if $\varphi$ and $\psi$ are
wffs, then $\vdash (\varphi \to (\psi \to \varphi))$ is an (axiomatic)
theorem; etc. The
last is the rule of modus ponens: if $\varphi$ and $\psi$ are wffs, and
$\vdash (\varphi\to\psi)$ and $\vdash \varphi$ are theorems, then $\vdash
\psi$ is a theorem.

The correspondence to ordinary propositional calculus is as follows.  We
consider only provable statements of the form $\langle\varnothing,
T,\varnothing,A\rangle$ with $T$ defined as above.  The first term of the
assertion $A$ of any such statement is either ``wff'' or ``$\vdash$''.  A
statement for which the first term is ``wff'' is a {\em wff} of propositional
calculus, and one where the first term is ``$\vdash$'' is a {\em
theorem (scheme)} of propositional calculus.

The universe of this formal system also contains many other provable
statements.  Those with distinct-variable restrictions are irrelevant because
propositional calculus has no constraints on substitutions.  Those that have
logical hypotheses we call {\em inferences}\index{inference} when
the logical hypotheses are of the form
$\langle\vdash\rangle\frown w$ where $w$ is a wff (with the leading constant
term ``wff'' removed).  Inferences (other than the modus ponens rule) are not a
proper part of propositional calculus but are convenient to use when building a
hierarchy of provable statements.  A provable statement with a nonsense
hypothesis such as $\langle \to,\vdash,\lnot\rangle$, and this same expression
as its assertion, we consider irrelevant; no use can be made of it in
proving theorems, since there is no way to eliminate the nonsense hypothesis.

{\footnotesize\begin{quotation}
{\em Comment.} Our use of parentheses in the definition of a wff illustrates
how axiomatic statements should be carefully stated in a way that
ties in unambiguously with the substitutions allowed by the formal system.
There are many ways we could have defined wffs---for example, Polish
prefix notation would have allowed us to omit parentheses entirely, at
the expense of readability---but we must define them in a way that is
unambiguous.  For example, if we had omitted parentheses from the
definition of $(\varphi\to \psi)$, the wff $\lnot\varphi\to \psi$ could
be interpreted as either $\lnot(\varphi\to\psi)$ or $(\lnot\varphi\to\psi)$
and would have allowed us to prove nonsense.  Note that there is no
concept of operator binding precedence built into our formal system.
\end{quotation}}

\begin{sloppy}
\subsection{Example~2---Predicate Calculus with Equality}\index{predicate
calculus}
\end{sloppy}

Here we extend Example~1 to include predicate calculus with equality,
illustrating the use of distinct-variable restrictions.  This system is the
same as Tarski's system $\mathfrak{S}_2$ in \cite{Tarski1965} (except that the
axioms of propositional calculus are different but equivalent, and a redundant
axiom is omitted).  We extend $\mbox{\em CN}$ with the constants
$\{\mbox{var},\forall,=\}$.  We extend $\mbox{\em VR}$ with an infinite set of
{\em individual metavariables}\index{individual
metavariable} $\{x,y,z,\ldots\}$ and denote this subset
$\mbox{\em Vr}$.

We also join to $\mbox{\em CN}$ a possibly infinite set $\mbox{\em Pr}$ of {\em
predicates} $\{R,S,\ldots\}$.  We associate with $\mbox{\em Pr}$ a function
$\mbox{rnk}$ from $\mbox{\em Pr}$ to $\omega$, and for $\alpha\in \mbox{\em
Pr}$ we call $\mbox{rnk}(\alpha)$ the {\em rank} of the predicate $\alpha$,
which is simply the number of ``arguments'' that the predicate has.  (Most
applications of predicate calculus will have a finite number of predicates;
for example, set theory has the single two-argument or binary predicate $\in$,
which is usually written with its arguments surrounding the predicate symbol
rather than with the prefix notation we will use for the general case.)  As a
device to facilitate our discussion, we will let $\mbox{\em Vs}$ be any fixed
one-to-one function from $\omega$ to $\mbox{\em Vr}$; thus $\mbox{\em Vs}$ is
any simple infinite sequence of individual metavariables with no repeating
terms.

In this example we will not include the function symbols that are often part of
formalizations of predicate calculus.  Using metalogical arguments that are
beyond the scope of our discussion, it can be shown that our formalization is
equivalent when functions are introduced via appropriate definitions.

We extend the set $T$ defined in Example~1 with the expressions
$\{\langle \mbox{var\ } x\rangle,$ $ \langle \mbox{var\ } y\rangle, \langle
\mbox{var\ } z\rangle,\ldots\}$.  We extend the $\Gamma$ above
with the axiomatic statements that are the reducts of the following
pre-statements:
\begin{list}{}{\itemsep 0.0pt}
      \item[] $\langle\varnothing,T,\varnothing,
               \langle \mbox{wff\ }\forall x\,\varphi\rangle\rangle$
      \item[] $\langle\varnothing,T,\varnothing,
               \langle \mbox{wff\ }x=y\rangle\rangle$
      \item[] $\langle\varnothing,T,
               \{\langle\vdash\varphi\rangle\},
               \langle\vdash\forall x\,\varphi\rangle\rangle$
      \item[] $\langle\varnothing,T,\varnothing,
               \langle \vdash((\forall x(\varphi\to\psi)
                  \to(\forall x\,\varphi\to\forall x\,\psi))
               \rangle\rangle$
      \item[] $\langle\{\{x,\varphi\}\},T,\varnothing,
               \langle \vdash(\varphi\to\forall x\,\varphi)
               \rangle\rangle$
      \item[] $\langle\{\{x,y\}\},T,\varnothing,
               \langle \vdash\lnot\forall x\lnot x=y
               \rangle\rangle$
      \item[] $\langle\varnothing,T,\varnothing,
               \langle \vdash(x=z
                  \to(x=y\to z=y))
               \rangle\rangle$
      \item[] $\langle\varnothing,T,\varnothing,
               \langle \vdash(y=z
                  \to(x=y\to x=z))
               \rangle\rangle$
\end{list}
These are the axioms not involving predicate symbols. The first two statements
extend the definition of a wff.  The third is the rule of generalization.  The
fifth states, in effect, ``For a wff $\varphi$ and variable $x$,
$\vdash(\varphi\to\forall x\,\varphi)$, provided that $x$ does not occur in
$\varphi$.''  The sixth states ``For variables $x$ and $y$,
$\vdash\lnot\forall x\lnot x = y$, provided that $x$ and $y$ are distinct.''
(This proviso is not necessary but was included by Tarski to
weaken the axiom and still show that the system is logically complete.)

Finally, for each predicate symbol $\alpha\in \mbox{\em Pr}$, we add to
$\Gamma$ an axiomatic statement, extending the definition of wff,
that is the reduct of the following pre-statement:
\begin{displaymath}
    \langle\varnothing,T,\varnothing,
            \langle \mbox{wff},\alpha\rangle\
            \frown \mbox{\em Vs}\restriction\mbox{rnk}(\alpha)\rangle
\end{displaymath}
and for each $\alpha\in \mbox{\em Pr}$ and each $n < \mbox{rnk}(\alpha)$
we add to $\Gamma$ an equality axiom that is the reduct of the
following pre-statement:
\begin{eqnarray*}
    \lefteqn{\langle\varnothing,T,\varnothing,
            \langle
      \vdash,(,\mbox{\em Vs}_n,=,\mbox{\em Vs}_{\mbox{rnk}(\alpha)},\to,
            (,\alpha\rangle\frown \mbox{\em Vs}\restriction\mbox{rnk}(\alpha)} \\
  & & \frown
            \langle\to,\alpha\rangle\frown \mbox{\em Vs}\restriction n\frown
            \langle \mbox{\em Vs}_{\mbox{rnk}(\alpha)}\rangle \\
 & & \frown
            \mbox{\em Vs}\restriction(\mbox{rnk}(\alpha)\setminus(n+1))\frown
            \langle),)\rangle\rangle
\end{eqnarray*}
where $\restriction$ denotes function domain restriction and $\setminus$
denotes set difference.  Recall that a subscript on $\mbox{\em Vs}$
denotes one of its terms.  (In the above two axiom sets commas are placed
between successive terms of sequences to prevent ambiguity, and if you examine
them with care you will be able to distinguish those parentheses that denote
constant symbols from those of our expository language that delimit function
arguments.  Although it might have been better to use boldface for our
primitive symbols, unfortunately boldface was not available for all characters
on the \LaTeX\ system used to typeset this text.)  These seemingly forbidding
axioms can be understood by analogy to concatenation of substrings in a
computer language.  They are actually relatively simple for each specific case
and will become clearer by looking at the special case of a binary predicate
$\alpha = R$ where $\mbox{rnk}(R)=2$.  Letting $\mbox{\em Vs}$ be the sequence
$\langle x,y,z,\ldots\rangle$, the axioms we would add to $\Gamma$ for this
case would be the wff extension and two equality axioms that are the
reducts of the pre-statements:
\begin{list}{}{\itemsep 0.0pt}
      \item[] $\langle\varnothing,T,\varnothing,
               \langle \mbox{wff\ }R x y\rangle\rangle$
      \item[] $\langle\varnothing,T,\varnothing,
               \langle \vdash(x=z
                  \to(R x y \to R z y))
               \rangle\rangle$
      \item[] $\langle\varnothing,T,\varnothing,
               \langle \vdash(y=z
                  \to(R x y \to R x z))
               \rangle\rangle$
\end{list}
Study these carefully to see how the general axioms above evaluate to
them.  In practice, typically only a few special cases such as this would be
needed, and in any case the Metamath language will only permit us to describe
a finite number of predicates, as opposed to the infinite number permitted by
the formal system.  (If an infinite number should be needed for some reason,
we could not define the formal system directly in the Metamath language but
could instead define it metalogically under set theory as we
do in this appendix, and only the underlying set theory, with its single
binary predicate, would be defined directly in the Metamath language.)


{\footnotesize\begin{quotation}
{\em Comment.}  As we noted earlier, the specific variables denoted by the
symbols $x,y,z,\ldots\in \mbox{\em Vr}\subseteq \mbox{\em VR}\subseteq
\mbox{\em SM}$ in Example~2 are not the actual variables of ordinary predicate
calculus but should be thought of as metavariables ranging over them.  For
example, a distinct-variable restriction would be meaningless for actual
variables of ordinary predicate calculus since two different actual variables
are by definition distinct.  And when we talk about an arbitrary
representative $\alpha\in \mbox{\em Vr}$, $\alpha$ is a metavariable (in our
expository language) that ranges over metavariables (which are primitives of
our formal system) each of which ranges over the actual individual variables
of predicate calculus (which are never mentioned in our formal system).

The constant called ``var'' above is called \texttt{setvar} in the
\texttt{set.mm} database file, but it means the same thing.  I felt
that ``var'' is a more meaningful name in the context of predicate
calculus, whose use is not limited to set theory.  For consistency we
stick with the name ``var'' throughout this Appendix, even after set
theory is introduced.
\end{quotation}}

\subsection{Free Variables and Proper Substitution}\index{free variable}
\index{proper substitution}\index{substitution!proper}

Typical representations of mathematical axioms use concepts such
as ``free variable,'' ``bound variable,'' and ``proper substitution''
as primitive notions.
A free variable is a variable that
is not a parameter of any container expression.
A bound variable is the opposite of a free variable; it is a
a variable that has been bound in a container expression.
For example, in the expression $\forall x \varphi$ (for all $x$, $\varphi$
is true), the variable $x$
is bound within the for-all ($\forall$) expression.
It is possible to change one variable to another, and that process is called
``proper substitution.''
In most books, proper substitution has a somewhat complicated recursive
definition with multiple cases based on the occurrences of free and
bound variables.
You may consult
\cite[ch.\ 3--4]{Hamilton}\index{Hamilton, Alan G.} (as well as
many other texts) for more formal details about these terms.

Using these concepts as \texttt{primitives} creates complications
for computer implementations.

In the system of Example~2, there are no primitive notions of free variable
and proper substitution.  Tarski \cite{Tarski1965} shows that this system is
logically equivalent to the more typical textbook systems that do have these
primitive notions, if we introduce these notions with appropriate definitions
and metalogic.  We could also define axioms for such systems directly,
although the recursive definitions of free variable and proper substitution
would be messy and awkward to work with.  Instead, we mention two devices that
can be used in practice to mimic these notions.  (1) Instead of introducing
special notation to express (as a logical hypothesis) ``where $x$ is not free
in $\varphi$'' we can use the logical hypothesis $\vdash(\varphi\to\forall
x\,\varphi)$.\label{effectivelybound}\index{effectively
not free}\footnote{This is a slightly weaker requirement than ``where $x$ is
not free in $\varphi$.''  If we let $\varphi$ be $x=x$, we have the theorem
$(x=x\to\forall x\,x=x)$ which satisfies the hypothesis, even though $x$ is
free in $x=x$ .  In a case like this we say that $x$ is {\em effectively not
free}\index{effectively not free} in $x=x$, since $x=x$ is logically
equivalent to $\forall x\,x=x$ in which $x$ is bound.} (2) It can be shown
that the wff $((x=y\to\varphi)\wedge\exists x(x=y\wedge\varphi))$ (with the
usual definitions of $\wedge$ and $\exists$; see Example~4 below) is logically
equivalent to ``the wff that results from proper substitution of $y$ for $x$
in $\varphi$.''  This works whether or not $x$ and $y$ are distinct.

\subsection{Metalogical Completeness}\index{metalogical completeness}

In the system of Example~2, the
following are provable pre-statements (and their reducts are
provable statements):
\begin{eqnarray*}
      & \langle\{\{x,y\}\},T,\varnothing,
               \langle \vdash\lnot\forall x\lnot x=y
               \rangle\rangle & \\
     &  \langle\varnothing,T,\varnothing,
               \langle \vdash\lnot\forall x\lnot x=x
               \rangle\rangle &
\end{eqnarray*}
whereas the following pre-statement is not to my knowledge provable (but
in any case we will pretend it's not for sake of illustration):
\begin{eqnarray*}
     &  \langle\varnothing,T,\varnothing,
               \langle \vdash\lnot\forall x\lnot x=y
               \rangle\rangle &
\end{eqnarray*}
In other words, we can prove ``$\lnot\forall x\lnot x=y$ where $x$ and $y$ are
distinct'' and separately prove ``$\lnot\forall x\lnot x=x$'', but we can't
prove the combined general case ``$\lnot\forall x\lnot x=y$'' that has no
proviso.  Now this does not compromise logical completeness, because the
variables are really metavariables and the two provable cases together cover
all possible cases.  The third case can be considered a metatheorem whose
direct proof, using the system of Example~2, lies outside the capability of the
formal system.

Also, in the system of Example~2 the following pre-statement is not to my
knowledge provable (again, a conjecture that we will pretend to be the case):
\begin{eqnarray*}
     & \langle\varnothing,T,\varnothing,
               \langle \vdash(\forall x\, \varphi\to\varphi)
               \rangle\rangle &
\end{eqnarray*}
Instead, we can only prove specific cases of $\varphi$ involving individual
metavariables, and by induction on formula length, prove as a metatheorem
outside of our formal system the general statement above.  The details of this
proof are found in \cite{Kalish}.

There does, however, exist a system of predicate calculus in which all such
``simple metatheorems'' as those above can be proved directly, and we present
it in Example~3. A {\em simple metatheorem}\index{simple metatheorem}
is any statement of the formal
system of Example~2 where all distinct variable restrictions consist of either
two individual metavariables or an individual metavariable and a wff
metavariable, and which is provable by combining cases outside the system as
above.  A system is {\em metalogically complete}\index{metalogical
completeness} if all of its simple
metatheorems are (directly) provable statements. The precise definition of
``simple metatheorem'' and the proof of the ``metalogical completeness'' of
Example~3 is found in Remark 9.6 and Theorem 9.7 of \cite{Megill}.\index{Megill,
Norman}

\begin{sloppy}
\subsection{Example~3---Metalogically Complete Predicate
Calculus with
Equality}
\end{sloppy}

For simplicity we will assume there is one binary predicate $R$;
this system suffices for set theory, where the $R$ is of course the $\in$
predicate.  We label the axioms as they appear in \cite{Megill}.  This
system is logically equivalent to that of Example~2 (when the latter is
restricted to this single binary predicate) but is also metalogically
complete.\index{metalogical completeness}

Let
\begin{itemize}
  \item[] $\mbox{\em CN}=\{\mbox{wff}, \mbox{var}, \vdash, \to, \lnot, (,),\forall,=,R\}$.
  \item[] $\mbox{\em VR}=\{\varphi,\psi,\chi,\ldots\}\cup\{x,y,z,\ldots\}$.
  \item[] $T = \{\langle \mbox{wff\ } \varphi\rangle,
             \langle \mbox{wff\ } \psi\rangle,
             \langle \mbox{wff\ } \chi\rangle,\ldots\}\cup
       \{\langle \mbox{var\ } x\rangle, \langle \mbox{var\ } y\rangle, \langle
       \mbox{var\ }z\rangle,\ldots\}$.

\noindent Then
  $\Gamma$ consists of the reducts of the following pre-statements:
    \begin{itemize}
      \item[] $\langle\varnothing,T,\varnothing,
               \langle \mbox{wff\ }(\varphi\to\psi)\rangle\rangle$
      \item[] $\langle\varnothing,T,\varnothing,
               \langle \mbox{wff\ }\lnot\varphi\rangle\rangle$
      \item[] $\langle\varnothing,T,\varnothing,
               \langle \mbox{wff\ }\forall x\,\varphi\rangle\rangle$
      \item[] $\langle\varnothing,T,\varnothing,
               \langle \mbox{wff\ }x=y\rangle\rangle$
      \item[] $\langle\varnothing,T,\varnothing,
               \langle \mbox{wff\ }Rxy\rangle\rangle$
      \item[(C1$'$)] $\langle\varnothing,T,\varnothing,
               \langle \vdash(\varphi\to(\psi\to\varphi))
               \rangle\rangle$
      \item[(C2$'$)] $\langle\varnothing,T,
               \varnothing,
               \langle \vdash((\varphi\to(\psi\to\chi))\to
               ((\varphi\to\psi)\to(\varphi\to\chi)))
               \rangle\rangle$
      \item[(C3$'$)] $\langle\varnothing,T,
               \varnothing,
               \langle \vdash((\lnot\varphi\to\lnot\psi)\to
               (\psi\to\varphi))\rangle\rangle$
      \item[(C4$'$)] $\langle\varnothing,T,
               \varnothing,
               \langle \vdash(\forall x(\forall x\,\varphi\to\psi)\to
                 (\forall x\,\varphi\to\forall x\,\psi))\rangle\rangle$
      \item[(C5$'$)] $\langle\varnothing,T,
               \varnothing,
               \langle \vdash(\forall x\,\varphi\to\varphi)\rangle\rangle$
      \item[(C6$'$)] $\langle\varnothing,T,
               \varnothing,
               \langle \vdash(\forall x\forall y\,\varphi\to
                 \forall y\forall x\,\varphi)\rangle\rangle$
      \item[(C7$'$)] $\langle\varnothing,T,
               \varnothing,
               \langle \vdash(\lnot\varphi\to\forall x\lnot\forall x\,\varphi
                 )\rangle\rangle$
      \item[(C8$'$)] $\langle\varnothing,T,
               \varnothing,
               \langle \vdash(x=y\to(x=z\to y=z))\rangle\rangle$
      \item[(C9$'$)] $\langle\varnothing,T,
               \varnothing,
               \langle \vdash(\lnot\forall x\, x=y\to(\lnot\forall x\, x=z\to
                 (y=z\to\forall x\, y=z)))\rangle\rangle$
      \item[(C10$'$)] $\langle\varnothing,T,
               \varnothing,
               \langle \vdash(\forall x(x=y\to\forall x\,\varphi)\to
                 \varphi))\rangle\rangle$
      \item[(C11$'$)] $\langle\varnothing,T,
               \varnothing,
               \langle \vdash(\forall x\, x=y\to(\forall x\,\varphi
               \to\forall y\,\varphi))\rangle\rangle$
      \item[(C12$'$)] $\langle\varnothing,T,
               \varnothing,
               \langle \vdash(x=y\to(Rxz\to Ryz))\rangle\rangle$
      \item[(C13$'$)] $\langle\varnothing,T,
               \varnothing,
               \langle \vdash(x=y\to(Rzx\to Rzy))\rangle\rangle$
      \item[(C15$'$)] $\langle\varnothing,T,
               \varnothing,
               \langle \vdash(\lnot\forall x\, x=y\to(x=y\to(\varphi
                 \to\forall x(x=y\to\varphi))))\rangle\rangle$
      \item[(C16$'$)] $\langle\{\{x,y\}\},T,
               \varnothing,
               \langle \vdash(\forall x\, x=y\to(\varphi\to\forall x\,\varphi)
                 )\rangle\rangle$
      \item[(C5)] $\langle\{\{x,\varphi\}\},T,\varnothing,
               \langle \vdash(\varphi\to\forall x\,\varphi)
               \rangle\rangle$
      \item[(MP)] $\langle\varnothing,T,
               \{\langle\vdash(\varphi\to\psi)\rangle,
                 \langle\vdash\varphi\rangle\},
               \langle\vdash\psi\rangle\rangle$
      \item[(Gen)] $\langle\varnothing,T,
               \{\langle\vdash\varphi\rangle\},
               \langle\vdash\forall x\,\varphi\rangle\rangle$
    \end{itemize}
\end{itemize}

While it is known that these axioms are ``metalogically complete,'' it is
not known whether they are independent (i.e.\ none is
redundant) in the metalogical sense; specifically, whether any axiom (possibly
with additional non-mandatory distinct-variable restrictions, for use with any
dummy variables in its proof) is provable from the others.  Note that
metalogical independence is a weaker requirement than independence in the
usual logical sense.  Not all of the above axioms are logically independent:
for example, C9$'$ can be proved as a metatheorem from the others, outside the
formal system, by combining the possible cases of distinct variables.

\subsection{Example~4---Adding Definitions}\index{definition}
There are several ways to add definitions to a formal system.  Probably the
most proper way is to consider definitions not as part of the formal system at
all but rather as abbreviations that are part of the expository metalogic
outside the formal system.  For convenience, though, we may use the formal
system itself to incorporate definitions, adding them as axiomatic extensions
to the system.  This could be done by adding a constant representing the
concept ``is defined as'' along with axioms for it. But there is a nicer way,
at least in this writer's opinion, that introduces definitions as direct
extensions to the language rather than as extralogical primitive notions.  We
introduce additional logical connectives and provide axioms for them.  For
systems of logic such as Examples 1 through 3, the additional axioms must be
conservative in the sense that no wff of the original system that was not a
theorem (when the initial term ``wff'' is replaced by ``$\vdash$'' of course)
becomes a theorem of the extended system.  In this example we extend Example~3
(or 2) with standard abbreviations of logic.

We extend $\mbox{\em CN}$ of Example~3 with new constants $\{\leftrightarrow,
\wedge,\vee,\exists\}$, corresponding to logical equivalence,\index{logical
equivalence ($\leftrightarrow$)}\index{biconditional ($\leftrightarrow$)}
conjunction,\index{conjunction ($\wedge$)} disjunction,\index{disjunction
($\vee$)} and the existential quantifier.\index{existential quantifier
($\exists$)}  We extend $\Gamma$ with the axiomatic statements that are
the reducts of the following pre-statements:
\begin{list}{}{\itemsep 0.0pt}
      \item[] $\langle\varnothing,T,\varnothing,
               \langle \mbox{wff\ }(\varphi\leftrightarrow\psi)\rangle\rangle$
      \item[] $\langle\varnothing,T,\varnothing,
               \langle \mbox{wff\ }(\varphi\vee\psi)\rangle\rangle$
      \item[] $\langle\varnothing,T,\varnothing,
               \langle \mbox{wff\ }(\varphi\wedge\psi)\rangle\rangle$
      \item[] $\langle\varnothing,T,\varnothing,
               \langle \mbox{wff\ }\exists x\, \varphi\rangle\rangle$
  \item[] $\langle\varnothing,T,\varnothing,
     \langle\vdash ( ( \varphi \leftrightarrow \psi ) \to
     ( \varphi \to \psi ) )\rangle\rangle$
  \item[] $\langle\varnothing,T,\varnothing,
     \langle\vdash ((\varphi\leftrightarrow\psi)\to
    (\psi\to\varphi))\rangle\rangle$
  \item[] $\langle\varnothing,T,\varnothing,
     \langle\vdash ((\varphi\to\psi)\to(
     (\psi\to\varphi)\to(\varphi
     \leftrightarrow\psi)))\rangle\rangle$
  \item[] $\langle\varnothing,T,\varnothing,
     \langle\vdash (( \varphi \wedge \psi ) \leftrightarrow\neg ( \varphi
     \to \neg \psi )) \rangle\rangle$
  \item[] $\langle\varnothing,T,\varnothing,
     \langle\vdash (( \varphi \vee \psi ) \leftrightarrow (\neg \varphi
     \to \psi )) \rangle\rangle$
  \item[] $\langle\varnothing,T,\varnothing,
     \langle\vdash (\exists x \,\varphi\leftrightarrow
     \lnot \forall x \lnot \varphi)\rangle\rangle$
\end{list}
The first three logical axioms (statements containing ``$\vdash$'') introduce
and effectively define logical equivalence, ``$\leftrightarrow$''.  The last
three use ``$\leftrightarrow$'' to effectively mean ``is defined as.''

\subsection{Example~5---ZFC Set Theory}\index{ZFC set theory}

Here we add to the system of Example~4 the axioms of Zermelo--Fraenkel set
theory with Choice.  For convenience we make use of the
definitions in Example~4.

In the $\mbox{\em CN}$ of Example~4 (which extends Example~3), we replace the symbol $R$
with the symbol $\in$.
More explicitly, we remove from $\Gamma$ of Example~4 the three
axiomatic statements containing $R$ and replace them with the
reducts of the following:
\begin{list}{}{\itemsep 0.0pt}
      \item[] $\langle\varnothing,T,\varnothing,
               \langle \mbox{wff\ }x\in y\rangle\rangle$
      \item[] $\langle\varnothing,T,
               \varnothing,
               \langle \vdash(x=y\to(x\in z\to y\in z))\rangle\rangle$
      \item[] $\langle\varnothing,T,
               \varnothing,
               \langle \vdash(x=y\to(z\in x\to z\in y))\rangle\rangle$
\end{list}
Letting $D=\{\{\alpha,\beta\}\in \mbox{\em DV}\,|\alpha,\beta\in \mbox{\em
Vr}\}$ (in other words all individual variables must be distinct), we extend
$\Gamma$ with the ZFC axioms, called
\index{Axiom of Extensionality}
\index{Axiom of Replacement}
\index{Axiom of Union}
\index{Axiom of Power Sets}
\index{Axiom of Regularity}
\index{Axiom of Infinity}
\index{Axiom of Choice}
Extensionality, Replacement, Union, Power
Set, Regularity, Infinity, and Choice, that are the reducts of:
\begin{list}{}{\itemsep 0.0pt}
      \item[Ext] $\langle D,T,
               \varnothing,
               \langle\vdash (\forall x(x\in y\leftrightarrow x \in z)\to y
               =z) \rangle\rangle$
      \item[Rep] $\langle D,T,
               \varnothing,
               \langle\vdash\exists x ( \exists y \forall z (\varphi \to z = y
                        ) \to
                        \forall z ( z \in x \leftrightarrow \exists x ( x \in
                        y \wedge \forall y\,\varphi ) ) )\rangle\rangle$
      \item[Un] $\langle D,T,
               \varnothing,
               \langle\vdash \exists x \forall y ( \exists x ( y \in x \wedge
               x \in z ) \to y \in x ) \rangle\rangle$
      \item[Pow] $\langle D,T,
               \varnothing,
               \langle\vdash \exists x \forall y ( \forall x ( x \in y \to x
               \in z ) \to y \in x ) \rangle\rangle$
      \item[Reg] $\langle D,T,
               \varnothing,
               \langle\vdash (  x \in y \to
                 \exists x ( x \in y \wedge \forall z ( z \in x \to \lnot z
                \in y ) ) ) \rangle\rangle$
      \item[Inf] $\langle D,T,
               \varnothing,
               \langle\vdash \exists x(y\in x\wedge\forall y(y\in
               x\to
               \exists z(y \in z\wedge z\in x))) \rangle\rangle$
      \item[AC] $\langle D,T,
               \varnothing,
               \langle\vdash \exists x \forall y \forall z ( ( y \in z
               \wedge z \in w ) \to \exists w \forall y ( \exists w
              ( ( y \in z \wedge z \in w ) \wedge ( y \in w \wedge w \in x
              ) ) \leftrightarrow y = w ) ) \rangle\rangle$
\end{list}

\subsection{Example~6---Class Notation in Set Theory}\label{class}

A powerful device that makes set theory easier (and that we have
been using all along in our informal expository language) is {\em class
abstraction notation}.\index{class abstraction}\index{abstraction class}  The
definitions we introduce are rigorously justified
as conservative by Takeuti and Zaring \cite{Takeuti}\index{Takeuti, Gaisi} or
Quine \cite{Quine}\index{Quine, Willard Van Orman}.  The key idea is to
introduce the notation $\{x|\mbox{---}\}$ which means ``the class of all $x$
such that ---'' for abstraction classes and introduce (meta)variables that
range over them.  An abstraction class may or may not be a set, depending on
whether it exists (as a set).  A class that does not exist is
called a {\em proper class}.\index{proper class}\index{class!proper}

To illustrate the use of abstraction classes we will provide some examples
of definitions that make use of them:  the empty set, class union, and
unordered pair.  Many other such definitions can be found in the
Metamath set theory database,
\texttt{set.mm}.\index{set theory database (\texttt{set.mm})}

% We intentionally break up the sequence of math symbols here
% because otherwise the overlong line goes beyond the page in narrow mode.
We extend $\mbox{\em CN}$ of Example~5 with new symbols $\{$
$\mbox{class},$ $\{,$ $|,$ $\},$ $\varnothing,$ $\cup,$ $,$ $\}$
where the inner braces and last comma are
constant symbols. (As before,
our dual use of some mathematical symbols for both our expository
language and as primitives of the formal system should be clear from context.)

We extend $\mbox{\em VR}$ of Example~5 with a set of {\em class
variables}\index{class variable}
$\{A,B,C,\ldots\}$. We extend the $T$ of Example~5 with $\{\langle
\mbox{class\ } A\rangle, \langle \mbox{class\ }B\rangle, \langle \mbox{class\ }
C\rangle,\ldots\}$.

To
introduce our definitions,
we add to $\Gamma$ of Example~5 the axiomatic statements
that are the reducts of the following pre-statements:
\begin{list}{}{\itemsep 0.0pt}
      \item[] $\langle\varnothing,T,\varnothing,
               \langle \mbox{class\ }x\rangle\rangle$
      \item[] $\langle\varnothing,T,\varnothing,
               \langle \mbox{class\ }\{x|\varphi\}\rangle\rangle$
      \item[] $\langle\varnothing,T,\varnothing,
               \langle \mbox{wff\ }A=B\rangle\rangle$
      \item[] $\langle\varnothing,T,\varnothing,
               \langle \mbox{wff\ }A\in B\rangle\rangle$
      \item[Ab] $\langle\varnothing,T,\varnothing,
               \langle \vdash ( y \in \{ x |\varphi\} \leftrightarrow
                  ( ( x = y \to\varphi) \wedge \exists x ( x = y
                  \wedge\varphi) ))
               \rangle\rangle$
      \item[Eq] $\langle\{\{x,A\},\{x,B\}\},T,\varnothing,
               \langle \vdash ( A = B \leftrightarrow
               \forall x ( x \in A \leftrightarrow x \in B ) )
               \rangle\rangle$
      \item[El] $\langle\{\{x,A\},\{x,B\}\},T,\varnothing,
               \langle \vdash ( A \in B \leftrightarrow \exists x
               ( x = A \wedge x \in B ) )
               \rangle\rangle$
\end{list}
Here we say that an individual variable is a class; $\{x|\varphi\}$ is a
class; and we extend the definition of a wff to include class equality and
membership.  Axiom Ab defines membership of a variable in a class abstraction;
the right-hand side can be read as ``the wff that results from proper
substitution of $y$ for $x$ in $\varphi$.''\footnote{Note that this definition
makes unnecessary the introduction of a separate notation similar to
$\varphi(x|y)$ for proper substitution, although we may choose to do so to be
conventional.  Incidentally, $\varphi(x|y)$ as it stands would be ambiguous in
the formal systems of our examples, since we wouldn't know whether
$\lnot\varphi(x|y)$ meant $\lnot(\varphi(x|y))$ or $(\lnot\varphi)(x|y)$.
Instead, we would have to use an unambiguous variant such as $(\varphi\,
x|y)$.}  Axioms Eq and El extend the meaning of the existing equality and
membership connectives.  This is potentially dangerous and requires careful
justification.  For example, from Eq we can derive the Axiom of Extensionality
with predicate logic alone; thus in principle we should include the Axiom of
Extensionality as a logical hypothesis.  However we do not bother to do this
since we have already presupposed that axiom earlier. The distinct variable
restrictions should be read ``where $x$ does not occur in $A$ or $B$.''  We
typically do this when the right-hand side of a definition involves an
individual variable not in the expression being defined; it is done so that
the right-hand side remains independent of the particular ``dummy'' variable
we use.

We continue to add to $\Gamma$ the following definitions
(i.e. the reducts of the following pre-statements) for empty
set,\index{empty set} class union,\index{union} and unordered
pair.\index{unordered pair}  They should be self-explanatory.  Analogous to our
use of ``$\leftrightarrow$'' to define new wffs in Example~4, we use ``$=$''
to define new abstraction terms, and both may be read informally as ``is
defined as'' in this context.
\begin{list}{}{\itemsep 0.0pt}
      \item[] $\langle\varnothing,T,\varnothing,
               \langle \mbox{class\ }\varnothing\rangle\rangle$
      \item[] $\langle\varnothing,T,\varnothing,
               \langle \vdash \varnothing = \{ x | \lnot x = x \}
               \rangle\rangle$
      \item[] $\langle\varnothing,T,\varnothing,
               \langle \mbox{class\ }(A\cup B)\rangle\rangle$
      \item[] $\langle\{\{x,A\},\{x,B\}\},T,\varnothing,
               \langle \vdash ( A \cup B ) = \{ x | ( x \in A \vee x \in B ) \}
               \rangle\rangle$
      \item[] $\langle\varnothing,T,\varnothing,
               \langle \mbox{class\ }\{A,B\}\rangle\rangle$
      \item[] $\langle\{\{x,A\},\{x,B\}\},T,\varnothing,
               \langle \vdash \{ A , B \} = \{ x | ( x = A \vee x = B ) \}
               \rangle\rangle$
\end{list}

\section{Metamath as a Formal System}\label{theorymm}

This section presupposes a familiarity with the Metamath computer language.

Our theory describes formal systems and their universes.  The Metamath
language provides a way of representing these set-theoretical objects to
a computer.  A Metamath database, being a finite set of {\sc ascii}
characters, can usually describe only a subset of a formal system and
its universe, which are typically infinite.  However the database can
contain as large a finite subset of the formal system and its universe
as we wish.  (Of course a Metamath set theory database can, in
principle, indirectly describe an entire infinite formal system by
formalizing the expository language in this Appendix.)

For purpose of our discussion, we assume the Metamath database
is in the simple form described on p.~\pageref{framelist},
consisting of all constant and variable declarations at the beginning,
followed by a sequence of extended frames each
delimited by \texttt{\$\char`\{} and \texttt{\$\char`\}}.  Any Metamath database can
be converted to this form, as described on p.~\pageref{frameconvert}.

The math symbol tokens of a Metamath source file, which are declared
with \texttt{\$c} and \texttt{\$v} statements, are names we assign to
representatives of $\mbox{\em CN}$ and $\mbox{\em VR}$.  For
definiteness we could assume that the first math symbol declared as a
variable corresponds to $v_0$, the second to $v_1$, etc., although the
exact correspondence we choose is not important.

In the Metamath language, each \texttt{\$d}, \texttt{\$f}, and
 \texttt{\$e} source
statement in an extended frame (Section~\ref{frames})
corresponds respectively to a member of the
collections $D$, $T$, and $H$ in a formal system statement $\langle
D_M,T_M,H,A\rangle$.  The math symbol strings following these Metamath keywords
correspond to a variable pair (in the case of \texttt{\$d}) or an expression (for
the other two keywords). The math symbol string following a \texttt{\$a} source
statement corresponds to expression $A$ in an axiomatic statement of the
formal system; the one following a \texttt{\$p} source statement corresponds to
$A$ in a provable statement that is not axiomatic.  In other words, each
extended frame in a Metamath database corresponds to
a pre-statement of the formal system, and a frame corresponds to
a statement of the formal system.  (Don't confuse the two meanings of
``statement'' here.  A statement of the formal system corresponds to the
several statements in a Metamath database that may constitute a
frame.)

In order for the computer to verify that a formal system statement is
provable, each \texttt{\$p} source statement is accompanied by a proof.
However, the proof does not correspond to anything in the formal system
but is simply a way of communicating to the computer the information
needed for its verification.  The proof tells the computer {\em how to
construct} specific members of closure of the formal system
pre-statement corresponding to the extended frame of the \texttt{\$p}
statement.  The final result of the construction is the member of the
closure that matches the \texttt{\$p} statement.  The abstract formal
system, on the other hand, is concerned only with the {\em existence} of
members of the closure.

As mentioned on p.~\pageref{exampleref}, Examples 1 and 3--6 in the
previous Section parallel the development of logic and set theory in the
Metamath database
\texttt{set.mm}.\index{set theory database (\texttt{set.mm})} You may
find it instructive to compare them.


\chapter{The MIU System}
\label{MIU}
\index{formal system}
\index{MIU-system}

The following is a listing of the file \texttt{miu.mm}.  It is self-explanatory.

%%%%%%%%%%%%%%%%%%%%%%%%%%%%%%%%%%%%%%%%%%%%%%%%%%%%%%%%%%%%

\begin{verbatim}
$( The MIU-system:  A simple formal system $)

$( Note:  This formal system is unusual in that it allows
empty wffs.  To work with a proof, you must type
SET EMPTY_SUBSTITUTION ON before using the PROVE command.
By default, this is OFF in order to reduce the number of
ambiguous unification possibilities that have to be selected
during the construction of a proof.  $)

$(
Hofstadter's MIU-system is a simple example of a formal
system that illustrates some concepts of Metamath.  See
Douglas R. Hofstadter, _Goedel, Escher, Bach:  An Eternal
Golden Braid_ (Vintage Books, New York, 1979), pp. 33ff. for
a description of the MIU-system.

The system has 3 constant symbols, M, I, and U.  The sole
axiom of the system is MI. There are 4 rules:
     Rule I:  If you possess a string whose last letter is I,
     you can add on a U at the end.
     Rule II:  Suppose you have Mx.  Then you may add Mxx to
     your collection.
     Rule III:  If III occurs in one of the strings in your
     collection, you may make a new string with U in place
     of III.
     Rule IV:  If UU occurs inside one of your strings, you
     can drop it.
Unfortunately, Rules III and IV do not have unique results:
strings could have more than one occurrence of III or UU.
This requires that we introduce the concept of an "MIU
well-formed formula" or wff, which allows us to construct
unique symbol sequences to which Rules III and IV can be
applied.
$)

$( First, we declare the constant symbols of the language.
Note that we need two symbols to distinguish the assertion
that a sequence is a wff from the assertion that it is a
theorem; we have arbitrarily chosen "wff" and "|-". $)
      $c M I U |- wff $. $( Declare constants $)

$( Next, we declare some variables. $)
     $v x y $.

$( Throughout our theory, we shall assume that these
variables represent wffs. $)
 wx   $f wff x $.
 wy   $f wff y $.

$( Define MIU-wffs.  We allow the empty sequence to be a
wff. $)

$( The empty sequence is a wff. $)
 we   $a wff $.
$( "M" after any wff is a wff. $)
 wM   $a wff x M $.
$( "I" after any wff is a wff. $)
 wI   $a wff x I $.
$( "U" after any wff is a wff. $)
 wU   $a wff x U $.

$( Assert the axiom. $)
 ax   $a |- M I $.

$( Assert the rules. $)
 ${
   Ia   $e |- x I $.
$( Given any theorem ending with "I", it remains a theorem
if "U" is added after it.  (We distinguish the label I_
from the math symbol I to conform to the 24-Jun-2006
Metamath spec.) $)
   I_    $a |- x I U $.
 $}
 ${
IIa  $e |- M x $.
$( Given any theorem starting with "M", it remains a theorem
if the part after the "M" is added again after it. $)
   II   $a |- M x x $.
 $}
 ${
   IIIa $e |- x I I I y $.
$( Given any theorem with "III" in the middle, it remains a
theorem if the "III" is replaced with "U". $)
   III  $a |- x U y $.
 $}
 ${
   IVa  $e |- x U U y $.
$( Given any theorem with "UU" in the middle, it remains a
theorem if the "UU" is deleted. $)
   IV   $a |- x y $.
  $}

$( Now we prove the theorem MUIIU.  You may be interested in
comparing this proof with that of Hofstadter (pp. 35 - 36).
$)
 theorem1  $p |- M U I I U $=
      we wM wU wI we wI wU we wU wI wU we wM we wI wU we wM
      wI wI wI we wI wI we wI ax II II I_ III II IV $.
\end{verbatim}\index{well-formed formula (wff)}

The \texttt{show proof /lemmon/renumber} command
yields the following display.  It is very similar
to the one in \cite[pp.~35--36]{Hofstadter}.\index{Hofstadter, Douglas R.}

\begin{verbatim}
1 ax             $a |- M I
2 1 II           $a |- M I I
3 2 II           $a |- M I I I I
4 3 I_           $a |- M I I I I U
5 4 III          $a |- M U I U
6 5 II           $a |- M U I U U I U
7 6 IV           $a |- M U I I U
\end{verbatim}

We note that Hofstadter's ``MU-puzzle,'' which asks whether
MU is a theorem of the MIU-system, cannot be answered using
the system above because the MU-puzzle is a question {\em
about} the system.  To prove the answer to the MU-puzzle,
a much more elaborate system is needed, namely one that
models the MIU-system within set theory.  (Incidentally, the
answer to the MU-puzzle is no.)

\chapter{Metamath Language EBNF}%
\label{BNF}%
\index{Metamath Language EBNF}

The following is a formal description of the basic Metamath language syntax
(with compressed proofs and support for unknown proof steps).
It is defined using the
Extended Backus--Naur Form (EBNF)\index{Extended Backus--Naur Form}\index{EBNF}
notation from W3C\index{W3C}
\textit{Extensible Markup Language (XML) 1.0 (Fifth Edition)}
(W3C Recommendation 26 November 2008) at
\url{https://www.w3.org/TR/xml/#sec-notation}.

The \texttt{database}
rule is processed until the end of the file (\texttt{EOF}).
The rules eventually require reading whitespace-separated tokens.
A token has an upper-case definition (see below)
or is a string constant in a non-token (such as \texttt{'\$a'}).
We intend for this to be correct, but if there is a conflict the
rules of section \ref{spec} govern. That section also discusses
non-syntax restrictions not shown here
(e.g., that each new label token
defined in a \texttt{hypothesis-stmt} or \texttt{assert-stmt}
must be unique).

\begin{verbatim}
database ::= outermost-scope-stmt*

outermost-scope-stmt ::=
  include-stmt | constant-stmt | stmt

/* File inclusion command; process file as a database.
   Databases should NOT have a comment in the filename. */
include-stmt ::= '$[' filename '$]'

/* Constant symbols declaration. */
constant-stmt ::= '$c' constant+ '$.'

/* A normal statement can occur in any scope. */
stmt ::= block | variable-stmt | disjoint-stmt |
  hypothesis-stmt | assert-stmt

/* A block. You can have 0 statements in a block. */
block ::= '${' stmt* '$}'

/* Variable symbols declaration. */
variable-stmt ::= '$v' variable+ '$.'

/* Disjoint variables. Simple disjoint statements have
   2 variables, i.e., "variable*" is empty for them. */
disjoint-stmt ::= '$d' variable variable variable* '$.'

hypothesis-stmt ::= floating-stmt | essential-stmt

/* Floating (variable-type) hypothesis. */
floating-stmt ::= LABEL '$f' typecode variable '$.'

/* Essential (logical) hypothesis. */
essential-stmt ::= LABEL '$e' typecode MATH-SYMBOL* '$.'

assert-stmt ::= axiom-stmt | provable-stmt

/* Axiomatic assertion. */
axiom-stmt ::= LABEL '$a' typecode MATH-SYMBOL* '$.'

/* Provable assertion. */
provable-stmt ::= LABEL '$p' typecode MATH-SYMBOL*
  '$=' proof '$.'

/* A proof. Proofs may be interspersed by comments.
   If '?' is in a proof it's an "incomplete" proof. */
proof ::= uncompressed-proof | compressed-proof
uncompressed-proof ::= (LABEL | '?')+
compressed-proof ::= '(' LABEL* ')' COMPRESSED-PROOF-BLOCK+

typecode ::= constant

filename ::= MATH-SYMBOL /* No whitespace or '$' */
constant ::= MATH-SYMBOL
variable ::= MATH-SYMBOL
\end{verbatim}

\needspace{2\baselineskip}
A \texttt{frame} is a sequence of 0 or more
\texttt{disjoint-{\allowbreak}stmt} and
\texttt{hypotheses-{\allowbreak}stmt} statements
(possibly interleaved with other non-\texttt{assert-stmt} statements)
followed by one \texttt{assert-stmt}.

\needspace{3\baselineskip}
Here are the rules for lexical processing (tokenization) beyond
the constant tokens shown above.
By convention these tokenization rules have upper-case names.
Every token is read for the longest possible length.
Whitespace-separated tokens are read sequentially;
note that the separating whitespace and \texttt{\$(} ... \texttt{\$)}
comments are skipped.

If a token definition uses another token definition, the whole thing
is considered a single token.
A pattern that is only part of a full token has a name beginning
with an underscore (``\_'').
An implementation could tokenize many tokens as a
\texttt{PRINTABLE-SEQUENCE}
and then check if it meets the more specific rule shown here.

Comments do not nest, and both \texttt{\$(} and \texttt{\$)}
have to be surrounded
by at least one whitespace character (\texttt{\_WHITECHAR}).
Technically comments end without consuming the trailing
\texttt{\_WHITECHAR}, but the trailing
\texttt{\_WHITECHAR} gets ignored anyway so we ignore that detail here.
Metamath language processors
are not required to support \texttt{\$)} followed
immediately by a bare end-of-file, because the closing
comment symbol is supposed to be followed by a
\texttt{\_WHITECHAR} such as a newline.

\begin{verbatim}
PRINTABLE-SEQUENCE ::= _PRINTABLE-CHARACTER+

MATH-SYMBOL ::= (_PRINTABLE-CHARACTER - '$')+

/* ASCII non-whitespace printable characters */
_PRINTABLE-CHARACTER ::= [#x21-#x7e]

LABEL ::= ( _LETTER-OR-DIGIT | '.' | '-' | '_' )+

_LETTER-OR-DIGIT ::= [A-Za-z0-9]

COMPRESSED-PROOF-BLOCK ::= ([A-Z] | '?')+

/* Define whitespace between tokens. The -> SKIP
   means that when whitespace is seen, it is
   skipped and we simply read again. */
WHITESPACE ::= (_WHITECHAR+ | _COMMENT) -> SKIP

/* Comments. $( ... $) and do not nest. */
_COMMENT ::= '$(' (_WHITECHAR+ (PRINTABLE-SEQUENCE - '$)')*
  _WHITECHAR+ '$)' _WHITECHAR

/* Whitespace: (' ' | '\t' | '\r' | '\n' | '\f') */
_WHITECHAR ::= [#x20#x09#x0d#x0a#x0c]
\end{verbatim}
% This EBNF was developed as a collaboration between
% David A. Wheeler\index{Wheeler, David A.},
% Mario Carneiro\index{Carneiro, Mario}, and
% Benoit Jubin\index{Jubin, Benoit}, inspired by a request
% (and a lot of initial work) by Benoit Jubin.
%
% \chapter{Disclaimer and Trademarks}
%
% Information in this document is subject to change without notice and does not
% represent a commitment on the part of Norman Megill.
% \vspace{2ex}
%
% \noindent Norman D. Megill makes no warranties, either express or implied,
% regarding the Metamath computer software package.
%
% \vspace{2ex}
%
% \noindent Any trademarks mentioned in this book are the property of
% their respective owners.  The name ``Metamath'' is a trademark of
% Norman Megill.
%
\cleardoublepage
\phantomsection  % fixes the link anchor
\addcontentsline{toc}{chapter}{\bibname}

\bibliography{metamath}
%\input{metamath.bbl}

\raggedright
\cleardoublepage
\phantomsection % fixes the link anchor
\addcontentsline{toc}{chapter}{\indexname}
%\printindex   ??
\input{metamath.ind}

\end{document}



\end{document}



\end{document}



\end{document}

